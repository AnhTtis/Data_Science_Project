\documentclass[letterpaper, 10 pt, conference]{ieeeconf}  
\IEEEoverridecommandlockouts % Command needed if using \thanks command
\overrideIEEEmargins         % Needed to meet printer requirements.

%In case you encounter the following error:
%Error 1010 The PDF file may be corrupt (unable to open PDF file) OR
%Error 1000 An error occurred while parsing a contents stream. Unable to analyze the PDF file.
%This is a known problem with pdfLaTeX conversion filter. The file cannot be opened with acrobat reader
%Please use one of the alternatives below to circumvent this error by uncommenting one or the other
%\pdfobjcompresslevel=0
%\pdfminorversion=4

% See the \addtolength command later in the file to balance the column lengths
% on the last page of the document

\pdfminorversion=4

% Math symbols
\usepackage{amsmath,amsfonts}
\let\proof\relax
\let\endproof\relax
\let\labelindent\relax
\usepackage{amssymb, amsthm}
\newtheorem{thm}{Theorem}
\newtheorem{cor}{Corollary}
\newtheorem{defn}{Definition}
\usepackage{mathtools}

\DeclareMathOperator{\dist}{d}

\usepackage{mydefs,mymath,kernMach,diffge}
\newcommand{\domain}{\Real^{n_d}}
\newcommand{\subD}{\mathcal{D}}
\newcommand{\epsD}{{\mathcal{D}}_\epsilon}
\newcommand{\HsubD}{{\mathcal{H}}_{\mathcal{D}}}
\newcommand{\HepsD}{{\mathcal{H}}_{\epsilon}}
\newcommand{\HspaceN}[1]{\mathcal{H}^{#1}}
\newcommand{\safeS}{\mathcal{S}}
\newcommand{\HSel}{\xi}
\newcommand{\objF}{f}
\renewcommand{\kernMap}{{\vec{k}}}
\newcommand{\RBF}{\varphi}
\newcommand{\kpFun}{{\vec{p}}_2}
\newcommand{\slkmFunc}{f_{\text{KM}}}

\newcommand{\sampSetSafe}{\sampSet^{\text{s}}}
\newcommand{\sampSetUnsafe}{\sampSet^{\text{u}}}
\newcommand{\setSafe}{\mathcal{S}^*}
\newcommand{\setUnsafe}{\mathcal{U}^*}
\newcommand{\tsafe}{\textit{safe}}
\newcommand{\tunsafe}{\textit{unsafe}}

\newcommand{\coverSet}{{\rm{C}}}
\newcommand{\coverSetClosed}{{\bar {\rm{C}}}}

\newcommand{\numPoly}{n_p}
\newcommand{\numKernPoly}{N_p}      % Max polynomial order.
\newcommand{\numKernRBF}{N_b}       % Number of kernel function types for RBF.

\newcommand{\SKMBFlong}{shallow kernel machine-zeroing barrier function}
\newcommand{\SKMBF}{SKM-ZBF}
\newcommand{\ZBFlong}{zeroing barrier function}
\newcommand{\ZBF}{ZBF}



% Formatting / Typesetting
\usepackage{calc}
\usepackage{cases}
\usepackage{url}
%\usepackage{array}
%\usepackage{tabularx}
%\usepackage{diagbox}
%\usepackage{subcaption}
%\usepackage[ruled, vlined, linesnumbered]{algorithm2e}
%%\usepackage{algpseudocode}
%%\usepackage{algorithm}
%%\usepackage{algorithmic}

% Graphics / visuals
\usepackage{float,color,graphicx}
\usepackage{tikz}
\usepackage{verbatim}
\usetikzlibrary{calc}
\usetikzlibrary{patterns}
\usepackage{pgfplots}
\pgfplotsset{compat=1.16}

\usepackage{enumitem}
\usepackage[font=footnotesize,skip=4pt]{caption}
\usepackage{cite}
\usepackage{balance}

\usepackage{sgap}

\graphicspath{{nnGeom/}}

%Comment-related macros:
\newcommand{\todo}[1]{\textcolor{blue}{\textit{(#1)}}}
\newcommand{\TODO}[1]{\textcolor{red}{[TODO: #1]}}
\newcommand{\notsure}[1]{\textcolor{olive}{#1}}
\newcommand{\toedit}[1]{\textcolor{red}{#1}}
\newcommand{\huh}[1]{\textcolor{cyan}{#1}}

\newcommand{\pGap}{\em Potential Gap}
\newcommand{\saferGap}{\em Safer Gap}
\newcommand{\keyhole}{\em Keyhole ZBF}

\title{\LARGE \bf Safer Gap: A Gap-based Local Planner for Safe Navigation with Nonholonomic Mobile Robots}

\author{{Shiyu Feng$^{1,\dagger}$, Ahmad Abuaish$^{2,\dagger}$, Patricio A. Vela$^{2}$}
\thanks{*This work was supported in part by NSF Award \#1849333, by DARPA PAI, and by KACST Fellowship.}%
\thanks{$\dagger$ Equal contribution}
\thanks{$^{1}$S. Feng is with the School of Mechanical Engineering and the School of Electrical and Computer Engineering, Georgia Institute of Technology, Atlanta, GA 30308, USA.
{\tt\small shiyufeng@gatech.edu}}
\thanks{$^{2}$ A. Abuaish , and P.A. Vela are with the School of Electrical and Computer
Engineering and the Institute for Robotics and Intelligent Machines, Georgia Institute of Technology, Atlanta, GA 30308, USA.
{\tt\small \{aabuaish, pvela\}@gatech.edu}}%
}


%\renewcommand{\qedsymbol}{$\blacksquare$}
%\theoremstyle{plain}
%\newtheorem{theorem}{Theorem}
%\newtheorem{prop}{Proposition}
%\newtheorem{probstat}{Problem Statement}
%\newtheorem{lemma}{Lemma}
%
%\theoremstyle{definition}
%\newtheorem{rmk}{Remark}
%\newtheorem{claim}{Claim}
%\newtheorem{assum}{Assumption}
%\newtheorem{definition}{Definition}
%
%\newcommand{\R}{\mathbb{R}}
%\DeclareMathAlphabet{\mathcal}{OMS}{cmsy}{m}{n}
%\newcommand*{\QEDB}{\hfill\ensuremath{\square}}
%\newcommand*{\QEDA}{\hfill\ensuremath{\blacksquare}}
%\newcommand{\kfun}{k_{\phi}}

\begin{document}

\maketitle
\thispagestyle{empty}
\pagestyle{empty}

%%%%%%%%%%%%%%%%%%%%%%%%%%%%%%%%%%%%%%%%%%%%%%%%%%%%%%%%%%%%%%%%%%%%%%%%%%%%%%%%
\begin{abstract}
This paper extends the gap-based navigation technique in {\pGap} by guaranteeing safety for nonholonomic robots for all tiers of the local planner hierarchy, so called {\saferGap}. The first tier generates a B\'{e}zier-based collision-free path through gaps. 
A subset of navigable free-space from the robot through a gap, called the keyhole, is defined to be the 
union of the largest collision-free disc centered on the robot and a trapezoidal region directed through the gap. It is encoded by a shallow neural network zeroing barrier function (ZBF). 
Nonlinear model predictive control (NMPC), with {\keyhole} constraints and output tracking of the B\'{e}zier path, synthesizes a safe kinematically-feasible trajectory. Low-level use of the 
 {\keyhole}  within a point-wise optimization-based safe control synthesis module
 serves as a final safety layer.
 Simulation and experimental validation of {\saferGap} confirm its collision-free navigation properties.
%. Finally, a control BF point-wise optimization safe control synthesis is employed to act as a last safety measure to cope with a fast-changing environment to which the NMPC loop cannot react within the required time frame. 
\end{abstract}

%%%%%%%%%%%%%%%%%%%%%%%%%%%%%%%%%%%%%%%%%%%%%%%%%%%%%%%%%%%%%%%%%%%%%%%%%%%%%%%%
\section{Introduction \label{sec:intro}}
\section{Introduction}

The increasing complexity of source code poses a key challenge to the reliability of large-scale software systems. Software bugs in these systems can lead to safety issues~\cite{bug_safety} for users around the world as well as cause non-negligible financial losses~\cite{bug_loss}. As such, developers have to spend a large amount of time and effort on bug fixing. Consequently, \aprfull (\apr), designed to automatically generate patches to fix software bugs, has attracted wide attention from both academia and industry~\cite{long2016prophet, legoues2012genprog, long2015spr, lou2020can, tufano2018empstudy}. 


To achieve \apr, one popular approach is known as Generate-and-Validate (G\&V)~\cite{qi2015gv, ghanbari2019prapr, lou2020can, le2016hdrepair, legoues2012genprog, wen2018capgen, hua2018sketchfix, martinez2016astor, koyuncu2020fixminder, liu2019tbar, liu2019avatar}, which is typically based on the following pipeline: First, fault localization techniques~\cite{wong2016fl, abreu2007ochiai, zhang2013injecting, papadakis2015metallaxis, li2019deepfl, li2017transforming} are applied to determine the suspicious locations in programs where bugs are likely to exist. Then, the buggy locations are used by the \apr tools to generate a list of patches that replace buggy lines with correct lines. Afterward, each patch is validated against the original test suite to identify any \emph{plausible patches} (i.e., passing all tests in the test suite). Finally, to determine the \emph{correct patches}, developers examine the list of plausible patches to see if any of them can correctly fix the bug. 

Traditional \apr tools can mainly be categorized into heuristic-based~\cite{legoues2012genprog, le2016hdrepair, wen2018capgen}, constraint-based~\cite{mechtaev2016angelix, le2017s3, demacro2014nopol, long2015spr} and \template~\cite{ghanbari2019prapr, hua2018sketchfix, martinez2016astor, liu2019tbar, liu2019avatar}. Among these traditional tools, \template \apr tools~\cite{ghanbari2019prapr, liu2019tbar, benton2020effectiveness} have been able to achieve state-of-the-art results. \Template \apr tools typically leverage pre-defined templates (e.g., adding a nullness check) for bug fixing. However, since these fix templates are typically handcrafted, the number and types of bugs they are able to fix can be limited. 



To address the limitations of traditional \apr, researchers have proposed various \learning \apr tools~\cite{li2020dlfix, chen2018sequencer, jiang2021cure, lutellier2020coconut, zhu2021recoder, ye2022rewardrepair} based on the \nmtfull (\nmt) architecture~\cite{sutskever2014mt} where the input is the buggy code snippets and the goal is to translate the buggy code snippets into a fixed version. To accomplish this, \learning \apr tools require supervised training datasets with pairs of both buggy and fixed code snippets in order to learn how to perform this translation step. These training data are usually obtained by mining historical bug fixes using heuristics/keywords~\cite{dallmeier2007benchmark}, which can be imprecise for identifying bug-fixing commits; even the actual bug-fixing commits can include irrelevant code changes, leading to further pollution in the dataset~\cite{xia2022alpharepair}.
% 
Moreover, it can be hard for such \apr tools to generalize and fix bug types unseen during training. 



To better leverage recent advances in \plmfull{s} (\plm{s}), researchers~\cite{xia2022alpharepair, xia2023repairstudy, kolak2022patch, prenner2021codexws} have directly applied \plm{s} to generate patches without bug-fixing datasets. These \llm-based \apr tools work by either directly generating a complete code function~\cite{prenner2021codexws, xia2023repairstudy} or predict/infill the correct code snippet given its surrounding context~\cite{xia2022alpharepair, xia2023repairstudy}. By directly using \llm{s} that are pre-trained on billions of open-source code snippets, \llm-based \apr tools can achieve state-of-the-art performance on many repair datasets~\cite{xia2022alpharepair}. 


% 
%
%

Traditional \apr tools have long used the insight of the \emph{plastic surgery hypothesis}~\cite{barr2014plastic} where it states that the code ingredients to fix a bug already exist within the same project. Traditional \apr tools have manually designed pattern-~\cite{ghanbari2019prapr, saha2017elixir} or heuristic-based~\cite{jiang2018simfix, legoues2012genprog} approaches to finding and using such relevant code ingredients to generate fixes for bugs. However, the plastic surgery hypothesis has been largely ignored in \llm-based \apr. In fact, \llm provides a unique opportunity to fully automate the plastic surgery hypothesis idea via fine-tuning (learning project-specific information via model updates from the buggy project) and prompting (directly providing relevant code ingredients to the model), and make it directly applicable to different languages (since the \llm{s} are typically multi-lingual).%
Moreover, despite the intensive manual efforts involved, traditional \apr tools still cannot fully leverage project-specific information due to large search space for leveraging/composing existing code ingredients. In contrast, the project-specific information can effectively leveraged by \llm{s} due to their power in code understanding/vectorization, e.g., even partial/imprecise information may still guide \llm{s} in correct patch generation!
 To this end, we ask the question: \emph{How useful is the plastic surgery hypothesis in the era of \plm{s}}?








\mypara{Our Work.} To answer the question, we present \ourtech{\xspace} -- a \llm-based approach that automatically utilizes the plastic surgery hypothesis by systematically combining multiple fine-tuning and prompting strategies for \apr. \ourtech fine-tunes \plm{s} using two novel domain-specific training strategies: \textbf{\epfinetune} -- we fine-tune using the original buggy project by aggressively masking out a high percentage of tokens, which allows \plm to learn project-specific code tokens and programming styles; and \textbf{\rofinetune} -- which only masks out a single continuous code sequence per training sample, allowing the model to get used to the final \csapr task of predicting a single continuous code sequence. Furthermore, we directly leverage the ability for \plm{s} to understand natural language instructions and introduce a novel prompting strategy, \textbf{\idprompting}, which uses information retrieval and static analysis to obtain a list of relevant identifiers for the buggy lines. While such relevant identifiers are critical for fixing some difficult bugs, they may not be seen by the \llm during inference due to limited context window size. Through the use of prompting, we directly tell the model to use these extracted identifiers (relevant code ingredients) to generate the correct code. Finally, to perform repair, we combine all four model variants (including the base model, both fine-tuned models and the base model with prompting) for the final repair.





While our insight of leveraging the plastic surgery hypothesis for \llm-based \apr is generalizable across different types of \plm{s}, to implement \ourtech, we choose a recent \plm{\xspace}, \ctfive~\cite{wang2021codet5}, which is pre-trained on millions of open-source code snippets. \ctfive is an encoder-decoder model trained using \mspfull (\msp) objective where a percentage of tokens are masked out and each continuous masked token sequence is referred to as a masked span. Also, although we only extract relevant identifiers from the current buggy project (since this paper focuses on the plastic surgery hypothesis), our work can be easily extended to obtain other code information (such as relevant statements or functions) from other sources, such as  the massive pre-training corpora~\cite{husain2020codesearchnet} or historical bug-fixing datasets~\cite{jiang2019infer}, which can provide more coding knowledge for \llm{s}. Besides, although we mainly focus on using traditional string comparison algorithms for information retrieval in this paper, these techniques can be easily replaced by other frequency-based retrieval~\cite{robertson2009probabilistic} and neural search (or embedding-based search)~\cite{reimers2019sentence}.
  In summary, this paper makes the following contributions:


%


\begin{itemize}[noitemsep, leftmargin=*, topsep=0pt]
    \item \textbf{Dimension.} This paper is the first to revisit the important plastic surgery hypothesis in the era of \llm{s}. It opens up a new dimension for \llm-based \apr to incorporate previously neglected information from the buggy project itself to boost \apr performance. Furthermore, it demonstrates the promising future of retrieval-based prompting for modern \llm-based \apr.
    \item \textbf{Implementation.} We implement \ourtech based on the recent \ctfive model. We augment the model using two novel fine-tuning strategies: \epfinetune and \rofinetune, along with a novel prompting strategy based on information retrieval and static analysis: \idprompting. We combine the patches generated by all four models together and perform patch ranking to speed up \apr.% 
    \item \textbf{Evaluation Study.} We conduct an extensive evaluation against state-of-the-art \apr tools. On the widely studied \dfj 1.2 and 2.0 datasets~\cite{just2014dfj}, \ourtech is able to achieve the new state-of-the-art results of 89 and 44 correct bug fixes (15 and 8 more than best baseline) respectively.  Furthermore, we perform a broad ablation study to justify our design. \ourtech demonstrates for the first time that the plastic surgery hypothesis can substantially boost \llm-based \apr and advance state-of-the-art \apr, while being fully automated and general. Moreover, even partial/imprecise code ingredients may still effectively guide \llm{s} for \apr!
\end{itemize}



%%%%%%%%%%%%%%%%%%%%%%%%%%%%%%%%%%%%%%%%%%%%%%%%%%%%%%%%%%%%%%%%%%%%%%%%%%%%%%%%
\section{Related Work \label{sec:back}}
\section{Background on Network Calculus}
\label{sec: background}


\begin{figure*}[tbh]
\centering
\begin{subfigure}[b]{0.3\textwidth}
    \centering
    \includegraphics[width=\linewidth]{images/in-out.png}
    \caption{Arrival and departure data and their relation with delay $d(t)$ and backlog $b(t)$. For a FIFO system, the delay is the horizontal distance between $R(t)$ and $R^*(t)$ but some other multiplexing techniques may shift the data to a later priority, causing a longer delay.}
    \label{fig: data in-out}
\end{subfigure}
\hfill
\begin{subfigure}[b]{0.35\textwidth}
    \centering
    \includegraphics[width=\linewidth]{images/arrival-service.png}
    \caption{Characteristics of an arrival curve and a service curve. From any point of observation, the arriving data never exceeds its arrival curve; the departure data is also never less than the service curve with respect to the data arrival.}
    \label{fig: arrival-service curves}
\end{subfigure}
\hfill
\begin{subfigure}[b]{0.33\textwidth}
    \centering
    \includegraphics[width=\linewidth]{images/bound.png}
    \caption{Delay and backlog bounds of a system. Backlog is the maximum vertical distance between $\alpha(t)$ and $\beta(t)$; FIFO delay is their maximum horizontal distance; but for arbitrary multiplexing, the delay guarantee is when the system clears its buffer, thus it's the intersection of $\alpha(t)$ and $\beta(t)$.}
    \label{fig: system bounds}
\end{subfigure}
\caption{Network calculus framework. We let $R(t)$ and $R^*(t)$ be the arrival and departure data flow of a system; $\alpha(t)$ be the piecewise linear concave arrival curve and $\beta(t)$ be the piecewise linear convex service curve of a system.}
% \hossein{Better to show piece-wise linear concave arrival curve and piece-wise linear convex service curve instead of token-bucket and rate-latency.}}
\end{figure*}

We recall some of the network calculus essentials for a better understanding of the framework used in Saihu. In the following context, we use the following notation: $\mbb{R}^+$ is the set of non-negative real numbers; $[x]_+$ denotes $\max(0, x)$

The data flow is by convention modeled as a left-continuous wide-sense increasing function $R(t): \mbb{R}^+ \mapsto \mbb{R}^+$ with respect to time $t$~\cite{ncbook2001leboudec}. 

A system $\mcal{S}$ receives arrival data described as a cumulative function $R(t)$ and delivers departure data as another cumulative function $R^*(t)$. Figure~\ref{fig: data in-out} illustrates such a system $\mcal{S}$. The benefit of representing a system like this is that we can observe system backlog and delay with such a model. 

\begin{definition}[Backlog and Delay~\cite{ncbook2001leboudec}]
    The backlog of a system at time~$t$ is
    \begin{equation}
        b(t) = R(t) - R^*(t)
    \end{equation}
    
    The virtual delay of a FIFO system at time $t$ is
    \begin{equation}
        d_{FIFO}(t) = \inf \lbp \tau \geq 0 : R(t) \leq R^*(t+\tau) \rbp
    \end{equation}
\end{definition}



The backlog of a system can be viewed as the vertical distance between $R$ and $R^*$. The FIFO (\textit{First-in First-out}) delay is the horizontal distance between $R$ and $R^*$. One may obtain other delay values if the multiplexing technique is not FIFO.

% \begin{figure}
%     \centering
%     \includegraphics[width=0.9\linewidth]{images/in-out.png}
%     \caption{In/out data flow; delay and backlog}
%     \label{fig: data in-out}
% \end{figure}

Since we are interested in the system guarantee instead of a single instance of data flow, we would like to have general bounds to the arrival and departure data flows. Therefore, we define \textit{arrival curve} and \textit{service curve} as the bounds of arrival and departure data flows.

\begin{definition}[Arrival Curve~\cite{ncbook2001leboudec}]
    Given a wide-sense increasing function $\alpha: \mbb{R}^+ \mapsto \mbb{R}^+$, we say that a flow $R(t)$ is $\alpha$-constrained if and only if for all $s \leq t$:
    \begin{equation}
        R(t) - R(s) \leq \alpha(t-s)
    \end{equation}
    We say $R(t)$ has $\alpha$ as an arrival curve.
\end{definition}

\begin{definition}[Service Curve~\cite{ncbook2001leboudec}]
    Given a wide-sense increasing function $\beta: \mbb{R}^+ \mapsto \mbb{R}^+$ and $\beta(0) = 0$. A system $\mcal{S}$ having $R(t)$ and $R^*(t)$ as its arrival and departure flows. We say $\mcal{S}$ offers a service curve $\beta$ if and only if
    \begin{equation}
        R^*(t) \geq (R \otimes \beta)(t) =: \inf_{s \leq t} \lbp R(s) + \beta(t-s) \rbp
    \end{equation}
    where $\otimes$ denotes the min-plus convolution
\end{definition}

Figure~\ref{fig: arrival-service curves} illustrates the arrival and service curves. Any segment of arrival flow $R(t)$ is constrained by arrival curve $\alpha$ and the output curve $R^*(t)$ is always no less than the curve $R\otimes\beta$. As a result, an arrival curve upper bounds the incoming traffic, and a service curve lower bounds the outgoing traffic.

% \begin{figure}
%     \centering
%     \includegraphics[width=\linewidth]{images/arrival-service.png}
%     \caption{Arrival/Service curve}
%     \label{fig: arrival-service curves}
% \end{figure}

We consider 2 special types of curves throughout this paper, \textit{token-bucket} (or sometimes called \textit{leaky-bucket}) curve and \textit{rate-Latency} curve.

\begin{definition}[Token-bucket and Rate-latency~\cite{ncbook2001leboudec}]
    A token-bucket curve $\gamma_{r,b}$ with arrival rate $r$ and burst $b$ is defined as
    \begin{equation}
        \gamma_{r,b}(t) = b + rt
    \end{equation}

    A rate-latency curve $\beta_{R,T}$ with service rate $R$ and latency $T$ is defined as
    \begin{equation}
        \beta_{R,T}(t) = R \lb t - T \rb_+
    \end{equation}
\end{definition}

A token-bucket curve is determined by a burst $b$ and an arrival rate~$r$. Burst represents the maximum possible data volume that can arrive simultaneously, and arrival rate represents the maximum long-term data rate~\cite{bouillard2022tradeoff}.
A rate-latency curve is determined by a latency~$T$ and a service rate~$R$. Latency represents the time a server needs before starting to process the incoming data, and service rate represents the minimum rate to process data after the initial latency.

With the help of arrival and service curves, we can derive delay and backlog bounds for a system $\mcal{S}$ illustrated in Figure~\ref{fig: system bounds}. Suppose a system $\mcal{S}$ has arrival curve $\alpha$ and service curve~$\beta$, its worst-case backlog $b^*$ is the maximum vertical distance between~$\alpha$ and~$\beta$. Similarly, depending on the multiplexing technique applied to the system, its worst-case delay bound $d^*$ is the maximum horizontal distance between $\alpha$ and $\beta$ if $\mcal{S}$ is a FIFO system. If we don't have any information about its multiplexing technique, referred to as arbitrary multiplexing, the best we can say is that when $\alpha$ and $\beta$ intersect each other, where all data has been delivered out of the system. Consequently, the worst-case delay bound for arbitrary multiplexing is the time required for $\mcal{S}$ to clear its buffer.

% \begin{figure}
%     \centering
%     \includegraphics[width=\linewidth]{images/bound.png}
%     \caption{System delay/backlog bounds}
%     \label{fig: system bounds}
% \end{figure}

While a service curve captures the slowest possible output speed of a system, a link's transmission capacity limits the speed as well. Hence, we model this phenomenon using a \textit{greedy shaper} with a sub-additive function $\sigma: \mbb{R}^+ \mapsto \mbb{R}^+$ concatenated with a server. We consider a concatenation as shown in Figure \ref{fig: system}. By convention we assume $\sigma(0) = 0$ and $\beta(t) \leq \sigma(t), \forall t \in \mbb{R}^+$, meaning that the buffer is cleared at the beginning and the service never exceed its physical limitation. With the above definition, such greedy shaper conserves the service provided by the system due to theorem \ref{thm: shaping}.

\begin{figure}[thb]
    \centering
    \includegraphics[width=0.7\linewidth]{images/system.png}
    \caption{Shaping of departure data. A flow that has an arrival curve $\alpha$ feeds into a server with an arrival data flow $R(t)$. The server having service curve $\beta$ takes $R(t)$ and gives a departure data flow $R^*(t)$ to a shaper with shaping function $\sigma$. The shaper takes $R^*(t)$ and shape the data flow as another departure $D(t)$.}
    \label{fig: system}
\end{figure}


\begin{theorem}[Shaping conserves service \cite{ncbook2001leboudec}]
\label{thm: shaping}
Following the system shown in Figure \ref{fig: system}, we have
\begin{equation}
     D = R^* \otimes \sigma \geq \lp R \otimes \beta \rp \otimes \sigma = R \otimes \lp \beta \otimes \sigma \rp = R \otimes \beta
\end{equation}
\end{theorem}

In the following context, we model the shaping function $\sigma$ as a token-bucket curve $\gamma_{C,L}$ with transmission capacity $C$ and the packet size $L$ to capture the link capacity and packetization~\cite{bouillard2022tradeoff}.


%%%%%%%%%%%%%%%%%%%%%%%%%%%%%%%%%%%%%%%%%%%%%%%%%%%%%%%%%%%%%%%%%%%%%%%%%%%%%%%%
\section{{\saferGap} Local Planner \label{sec:safe}}

\begin{figure}[t]
  \vspace*{0.065in}
  \centering
  \scalebox{0.75}{
  \begin{tikzpicture}[inner sep=0pt,outer sep=0pt]
    \node[anchor=south west] at ($(0, 0)$)
    	{{\tikzstyle{block} = [draw, rectangle, fill=black!5, minimum height=1.em, minimum width=1em, inner sep=4pt]
\tikzstyle{newblock} = [draw, rectangle, fill=black!15, minimum height=1.em, minimum width=1em, inner sep=4pt]

\begin{tikzpicture}

% \fill[black!5] (-2.9,5.3) -- (0,0) -- (2.8,6.3) -- cycle;
\draw[fill=black!5] (0,0) circle (2.5);
\fill[black!5] (-2.5,4.5) -- (-2.5,0) -- (1.85,1.65) -- (2.5,5.5) -- cycle;
\fill[black!15] (0,0) circle (1.8);
\fill[black!15] (-1.8,4.6) -- (-1.8,0) -- (1.1,1.35) -- (1.85,5.35) -- cycle;
\draw[black!45] (0,0) circle (1.8);

\node[anchor=center,rotate=45] at (0,0) {
\begin{tikzpicture}[auto]
\node[anchor=center] (a1) at ($(0, 0)$)
    	{{\includegraphics[trim={1cm 1cm 1cm 1cm},clip,height=0.3in]{figs/turtlebot_trans.png}}};
\end{tikzpicture}
};
\draw[blue,very thick] (1.85,1.65) -- (2.5,5.5);
\draw[blue,very thick] (-2.5,0) -- (-2.5,4.5);

\draw (2.5,5.5) --++(-60:1cm);
\draw (2.5,5.5) --++(30:1cm);
\fill[pattern=north west lines,rotate around={30:(2.5,5.5)}] (2.5,5.5) rectangle ++(1,-1); 
\draw (-2.5,4.5) --++(-120:1cm);
\draw (-2.5,4.5) --++(-210:1cm);
\fill[pattern=north west lines,rotate around={-30:(-2.5,4.5)}] (-2.5,4.5) rectangle ++(-1,-1); 

\draw (2.,2.5) --++(-90:1cm);
\draw (2.,2.5) --++(0:1cm);
\draw (2.,1.5) --++(0:1cm);
\fill[pattern=north west lines,rotate around={0:(2.,2.5)}] (2.,2.5) rectangle ++(1,-1); 

\draw[color=red, very thick](0,0) circle (2.5);

\draw[->, black, ultra thick] (0,0) -- (0.5,0.5);
% \draw[black, thick, rotate around={45:(0,0)}] (-0.5,0.2) rectangle (0.5, -0.2);



\filldraw[black] (0.3,5.2) circle (3pt) node[anchor=north, yshift=1.5em] {\large $\goal$};

\filldraw[red] (0.7,1.64) circle (3pt) node[anchor=north,xshift=-2em, yshift=-0.5em,text=black] {\large $\circPt$};

\draw[black, dashed, ultra thick] (0,0) -- (0.7,1.64);
\draw[black, dashed, ultra thick] (0,0) -- (0.5,0.5);
\draw[black, dashed, ultra thick] (0.5,0.5) -- (0.85,1.1);
\draw[black, dashed, ultra thick] (0.85,1.1) -- (0.7,1.64);
\draw[brown, ultra thick] (0,0) .. controls (0.5,0.5) and (0.85,1.1) .. (0.7,1.64);
\draw[black, dashed, ultra thick] (0.7,1.64) -- (0.5,2.36);
\draw[black, dashed, ultra thick] (0.5,2.36) -- (0.3,5.2);
\draw[black, dashed, ultra thick] (0.7,1.64) -- (0.3,5.2);
\draw[cyan, ultra thick] (0.7,1.64) .. controls (0.5,2.36) .. (0.3,5.2);

\node[anchor=west, xshift=-4em] at (0,0) {\large $\pose(0)$};
\node[anchor=west, xshift=0.2em, yshift=-1.2em] at (0.5,0.5) {\large $\eleInd{\cpf}{1}$};
\node[anchor=west, xshift=0.3em, yshift=-1em] at (0.85,1.1) {\large $\eleInd{\cpf}{2}$};
\node[anchor=north, xshift=-2em, yshift=2.2em] at (0.7,1.64) {\large $\eleInd{\cps}{1}$};

\node[anchor=east, xshift=-0.5em, yshift=0.5em] at (2.5,5.5) {\large $\lgap$};
\node[anchor=west, xshift=0.2em, yshift=1em] at (-2.5,4.5) {\large $\rgap$};
\node[anchor=east, xshift=0em, yshift=2em] at (1.85,1.65) {\large $\point_l$};
\node[anchor=west, xshift=0.3em, yshift=0em] at (-2.5,0) {\large $\point_r$};
\node[anchor=east, xshift=-0.9em, yshift=0em] at (-1.6,3.3) {\large $\gapSide_r$};
\node[anchor=west, xshift=0.8em, yshift=0em] at (1.4,4.2) {\large $\gapSide_l$};

\node[anchor=south, xshift=0em, yshift=0.5em] at (0,-2.5) {\large $\gapCirc$};
\node[anchor=east, xshift=0em, yshift=0em] at (-0.3, 4.4) {\large $\gapPoly$};

% Legends
\node[block, anchor=west, xshift=0em, yshift=0em] (tri) at (-16em,3em) {};
\node[anchor=west, text centered, text width=7em] (tri_name) at ($(tri.east) + (0.5em, 0)$) {Free Space $\freeSpace$};

\node[block, anchor=west, xshift=0em, yshift=0em,pattern=north west lines] (obs) at (-16em,9.5em) {};
\node[anchor=west, text centered, text width=7em] (obs_name) at ($(obs.east) + (0.5em, 0)$) {Obstacles};

\draw[blue, very thick] (-16em, 7.5em) -- (-15em, 7.5em);
\node[anchor=south, text centered, text width=5em] (side_name) at ($(tri_name.north) + (0em, 3.5em)$) {Gap Sides};

\draw[red, very thick] (-16em, 5.5em) -- (-15em, 5.5em);
\node[anchor=south, text centered, text width=8em] (circ_name) at ($(tri_name.north) + (0em, 1.5em)$) {Gap Circle $\gapCirc$};

\node[newblock, anchor=north, xshift=0em, yshift=0em] (nf) at ($(tri.south)+(0,-1.4em)$) {};
\node[anchor=west, text centered, text width=7em] (nf_name) at ($(nf.east) + (0.5em, 0)$) {Inflated Free \\ Space $\infFreeSpace$};

\draw[black, dashed, ultra thick] (-16em, -2em) -- (-15em, -2em);
\node[anchor=north, text centered, text width=6em] (b_name) at ($(nf_name.south) + (0em, -0.8em)$) {B\'{e}zier Polygons};

\draw[brown, ultra thick] (-16em, -4.5em) -- (-15em, -4.5em);
\draw[cyan, ultra thick] (-16em, -5.5em) -- (-15em, -5.5em);
\node[anchor=north, text centered, text width=6em] (c_name) at ($(b_name.south) + (0em, -0.8em)$) {B\'{e}zier Trajectories};

\end{tikzpicture}}};
  \end{tikzpicture}%
  }
  \caption{B\'{e}zier trajectory synthesis. $\pose(0)$ is the robot origin. $\eleInd{\cpf}{1}$ and $\eleInd{\cpf}{2}$ are the second and third control points for the first cubic B\'{e}zier curve. Blue lines $\gapSide_l$ and $\gapSide_r$ are left and right gap sides. Red circle is the largest circular free space in egocircle. $\point_l$ and $\point_r$ are the left and right intersection points of $\gapCirc$ and gap sides. Local waypoint $\goal$ is inside the inflated safe region $\infFreeSpace$ to guarantee safety. $\circPt$ is the goal biased point on $\infGapCirc$. Dashed lines show the B\'{e}zier polygons. The combination of brown and cyan paths is the final synthesized path. \label{fig:bezier_safe}}
  % \vspace*{-1.5em}
\end{figure}


This section introduces a gap-based local planning policy to guarantee safe navigation for nonholonomic mobile robots, so called {\saferGap}. It incorporates line-of-sight visibility from gap detection to construct collision free space. Then safety and passibility are maintained during the design of path planning and motion control. 

\subsection{Joined B\'{e}zier Path Planning \label{sec:bp}}

In our previous work \cite{pgap,bgap}, gap-based perception space and B\'{e}zier-based trajectory synthesis are demonstrated to have good navigation performance. However, safety is only guaranteed for the holonomic robot model. We propose to synthesize smooth and safe paths from gaps based on joined B\'{e}zier curves. 
Gap depicts the open region between two obstacles by considering robot's line-of-sight visibility, as shown in Fig.~\ref{fig:bezier_safe}. It is generated from egocircle and follows the same procedure in \cite{pgap}. The egocircle \cite{Smith2020} is an ego-centric 1D array that contains spatial and temporal information of the environment. 

\subsubsection{Collision-free space generation}
Similar to \cite{bgap}, a collision-free space $\freeSpace$ is geometrically constructed for each gap. However, triangle regions constructed in \cite{bgap} are too compact when one side of the robot is close to an obstacle; thus, a richer polygonal space should be created. The largest circular collision-free space $\gapCirc$ within egocircle $\egoCirc$ is found, e.g., red circle in Fig.~\ref{fig:bezier_safe}. Two gap points, $\lgap$ and $\rgap$, are initially connected to the tangent points of $\gapCirc$. The raw gap sides are formulated. To be noticed, the tangent point corresponding to $\lgap$ always has smaller polar angle than $\rgap$ in the robot local frame. If raw gap sides are obstructed by other obstacles, inward rotations are applied about $\lgap$ (clockwise) and $\rgap$ (counter clockwise) until there is no obstruction. The maximum rotation can push the tangent point to the center of $\gapCirc$, which constructs a minimum collision free space $\freeSpace$, which is the same as \cite{bgap}. After rotation, the gap sides, $\gapSide_l$ and $\gapSide_r$ are finalized. We define the intersections between $\gapSide$ and $\gapCirc$ are $\point_l$ and $\point_r$. Two gap points and two intersected points formulate a collision free polygon $\gapPoly$. The full collision free space $\freeSpace$ is 
\begin{equation}
    \freeSpace = \gapPoly \cup \gapCirc
\end{equation}
Considering robot geometry, $\freeSpace$ is inflated as $\infFreeSpace$ in Fig.~\ref{fig:bezier_safe}. The inflation size is a function of robot radius. Intersected points after inflation are denoted as $\inflate{\point_{l}}$ and $\inflate{\point_{r}}$. Inflated gap circle is $\infGapCirc$. Any path $\bPath$ within the inflated collision free space $\bPath \in \infFreeSpace$ can guarantee safety for the full robot geometry.

% In order to generate smooth and safe paths for nonholonomic robots, joined B\'{e}zier curves are described. The first segment is a cubic B\'{e}zier curve parameterized by $u$ inside $\infGapCirc$,
\subsubsection{Joined B\'{e}zier curves}
The first segment is a cubic B\'{e}zier curve parameterized by $u$ inside $\infGapCirc$,
\begin{equation} \label{eq:cb}
\begin{aligned}
  \bezierCurve_1(u) &= \sum_{i=0}^{n=3} \binom{n}{i} (1-u)^{n-i} u^i \eleInd{\cpf}{i} \\
  \binom{n}{i} &= \frac{n!}{i!(n-i)!}, \; 0 \leq u \leq 1
\end{aligned}
\end{equation}
where $\eleInd{\cpf}{i}$ is the $i$th control point of $\bezierCurve_1$.

Since the gap is detected in robot local frame, robot center is used as the first control point $\eleInd{\cpf}{0} = \pose(0)$. An intermediate point $\circPt$ is defined on the arc between $\inflate{\point_l}$ and $\inflate{\point_r}$, and served as the last control point $\eleInd{\cpf}{3}$. The other two control points are designed from initial orientation $\theta(0)$, linear velocity $\nu(0)$, and acceleration $\acc(0)$ of the nonholonomic robot. $\eleInd{\cpf}{1}-\eleInd{\cpf}{0}$ is co-linear with the unit orientation vector $\orient(0)=[cos(\theta(0)),sin(\theta(0))]$. Curve velocities and accelerations are obtained from the first and second derivatives of cubic B\'{e}zier curve, 
\begin{align}
 \dot{\bezierCurve_1}(u) &= 3 \sum_{i=0}^{2} \binom{2}{i} (1-u)^{2-i} u^i (\eleInd{\cpf}{i+1}-\eleInd{\cpf}{i})  \\
  \dot{\bezierCurve_1}(0)&=3(\eleInd{\cpf}{1}-\eleInd{\cpf}{0}) \\
  % \ddot{\bezierCurve_1}(u) &= 6 (1-u) (\eleInd{\cpf}{2}-2\eleInd{\cpf}{1}+\eleInd{\cpf}{0}) + 6u (\eleInd{\cpf}{3}-2\eleInd{\cpf}{2}+\eleInd{\cpf}{1}) \nonumber \\
  \ddot{\bezierCurve_1}(u) &= 6 \sum_{i=0}^{1} \binom{1}{i} (1-u)^{1-i} u^i (\eleInd{\cpf}{i+2}-2\eleInd{\cpf}{i+1}+\eleInd{\cpf}{i})  \\
  \ddot{\bezierCurve_1}(0)&=6(\eleInd{\cpf}{2}-2\eleInd{\cpf}{1}+\eleInd{\cpf}{0}).
\end{align}

The curve parameter $u \in [0,1]$ should be scaled to time $t \in [0,\timeScaleF]$ and $t = \timeScaleF u$. The final time $\timeScaleF$ is estimated by $||\circPt-\pose(0)||/\nu_d$, where $\nu_d$ is the robot desired linear velocity. Then the scaled B\'{e}zier path $\scale{\bezierCurve}_1(t)=\bezierCurve_1(t/\timeScaleF)$, and
\begin{align}
    \dot{\scale{\bezierCurve}_1}(t) &= \frac{1}{\timeScaleF} \dot{\bezierCurve_1}(\frac{t}{\timeScaleF}) \\
    \ddot{\scale{\bezierCurve}_1}(t) &= \frac{1}{\timeScaleF^2} \ddot{\bezierCurve_1}(\frac{t}{\timeScaleF}).
\end{align}

Set $||\dot{\scale{\bezierCurve}_1}(0)||=\nu(0)$, which needs $||\eleInd{\cpf}{1}-\eleInd{\cpf}{0}||=\timeScaleF \nu(0)/3$. Then set $\ddot{\scale{\bezierCurve}_1}(0)=\acc(0)$, all control points for the first B\'{e}zier path segment $\bezierCurve_1(u), u \in [0,1]$ are uniquely defined
\begin{equation}
\begin{aligned}
    \eleInd{\cpf}{0} &= \pose(0) \\
    \eleInd{\cpf}{1} &= \pose(0) + \frac{\timeScaleF \nu(0)}{3}\orient(0) \\
    \eleInd{\cpf}{2} &= \frac{\timeScaleF^2}{6} \acc(0) - \eleInd{\cpf}{0} + 2 \eleInd{\cpf}{1} \\
    \eleInd{\cpf}{3} &= \circPt
\end{aligned}
\end{equation}

The second path segment is generated from a quadratic B\'{e}zier curve
\begin{equation} 
  \bezierCurve_2(u) = (1 - u)^2 \eleInd{\cps}{0} + 2(1 - u)u \eleInd{\cps}{1} + u^2 \eleInd{\cps}{2}.
\end{equation}
where $\eleInd{\cps}{0}=\circPt$. 

G1 continuity maintains a smooth connection between B\'{e}zier curves. Therefore, the direction vector $\directVec$ should satisfy the equality:
\begin{equation} 
  \directVec = \frac{\eleInd{\cps}{1}-\eleInd{\cps}{0}}{||\eleInd{\cps}{1}-\eleInd{\cps}{0}||} = \frac{\eleInd{\cpf}{3}-\eleInd{\cpf}{2}}{||\eleInd{\cpf}{3}-\eleInd{\cpf}{2}||}
\end{equation}
The magnitude of $\eleInd{\cps}{1}-\eleInd{\cps}{0}$ is calculated by desired linear velocity $\nu_d$. With quadratic B\'{e}zier curve and similar scale mechanism,
\begin{align}
  \dot{\bezierCurve_2}(u) &= 2(1-u)(\eleInd{\cps}{1} - \eleInd{\cps}{0}) + 2u(\eleInd{\cps}{2} - \eleInd{\cps}{1}) \\
  \dot{\bezierCurve_2}(0) &= 2(\eleInd{\cps}{1} - \eleInd{\cps}{0}) \\
  \dot{\scale{\bezierCurve}_2}(t) &= \frac{1}{\timeScaleS} \dot{\bezierCurve_2}(\frac{t}{\timeScaleS})
\end{align}
where $\timeScaleS=||\goal-\circPt||/\nu_d$. 

Similarly, set $||\dot{\scale{\bezierCurve}_2}(0)||=\nu_d$, which requires $||\eleInd{\cps}{1}-\eleInd{\cps}{0}||=\timeScaleS \nu_d / 2$. When $\circPt$ is close to $\inflate{\gapSide_l}$ or $\inflate{\gapSide_r}$, $\eleInd{\cps}{1}$ is possible to be outside of the inflated gap sides after scaling. A length scale number $\lambda \in (0,1]$ is calculated to bound $\eleInd{\cps}{1}$ inside $\infFreeSpace$. All control points for the second B\'{e}zier path segment $\bezierCurve_2(u)$ are constrained
\begin{equation}
\begin{aligned}
    \eleInd{\cps}{0} &= \circPt \\
    \eleInd{\cps}{1} &= \circPt + \lambda \frac{\timeScaleS \nu_d}{2}\directVec \\
    \eleInd{\cps}{2} &= \goal \\
\end{aligned}
\end{equation}

Local waypoint $\goal$ candidates are initially found based on global plans and then bounded by $\infFreeSpace$. The intermediate point $\circPt$ starts with the middle point of the arc, then is biased by the relative position between $\pose(0)$ and $\goal$ to synthesize smoother paths. If $\goal$ is within $\infGapCirc$, only first B\'{e}zier segment is computed. The final B\'{e}zier-based path is 
\begin{equation}
    \bPath(u) = \begin{cases}
        \bezierCurve_1(u), & \goal \in \gapCirc \\
        \bezierCurve_1(u) \cup \bezierCurve_2(u), & \text{otherwise}
    \end{cases}
\end{equation}

From the above design, the first B\'{e}zier polygon for $\bezierCurve_1(u)$ is always within $\infGapCirc$. The second B\'{e}zier polygon is within the convex region $\inflate{\gapPoly}$.
Therefore, the joined B\'{e}zier path is inside the inflated collision free space, $\bPath(u) \subseteq \infFreeSpace$. Safety and passibility are achieved for nonholonomic robots.
It only takes $\leq 2ms$ to generate path for each gap (on Intel i7-8700). The full path planning time depends on the number of detected gaps. A set of new paths are synthesized in every planning loop.


\subsubsection{Path scoring}
A scoring function is computed for each joined B\'{e}zier path to choose the best executed one $\bPath^*$. This function is an improved version from \cite{pgap} by adding an orientation cost. The path has lower deviation from robot's orientation is preferable, since nonholonomic robots cannot suddenly turn backwards. It is also helpful to pick the correct path when the final goal point is on the other side of walls.
\begin{multline*}
    \nonumber 
    J(\bPath) = \sum_{\tPose \in \bPath} 
        C(\dist(\tPose,\egoCirc)) + w_1||\tPose_\text{end} - \pose^*|| 
       + w_2 |\theta_\text{end} - \theta(0)| 
\end{multline*}
\vspace{-1.em}
\begin{eqnarray}
  \nonumber 
  \small{\text{where} \quad
    C(d) = \begin{cases}
      c_{\text{obs}} e^{-w_2 (d - \rIns)}, & d > \rIns \\
      0, & d > r_{\text{max}}\\
      \infty, \text{otherwise}
    \end{cases}}
\end{eqnarray}
$\dist(\tPose,\egoCirc)$ is the distance from path pose $\tPose=[\text{x}_1, \text{x}_2, \theta]^T$ to the nearest point on egocircle $\egoCirc$. $||\tPose_\text{end} - \pose^*||$ measures the distance between the end pose of $\bPath$ and the local goal $\pose^*$ from a global plan. $|\theta_\text{end} - \theta(0)|$ is the angle difference between end pose and initial pose. $\rIns$ and $r_{\text{max}}$ are proportional to the robot radius to control the safe distance. $w_1$, $w_2$ and $c_{\text{obs}}$ are tunable weights. Each time, every best path $\bPath^*_i$ compares with the previous executed path $\bPath^*_{i-1}$ to decide whether switching to the new path.
One example is shown in Fig.~\ref{fig:bezier_path}. The best path (red) is selected from a set of B\'{e}zier path candidates.

\begin{figure}[t]
 \vspace*{0.065in}
  \centering
  \scalebox{0.75}{
  \begin{tikzpicture}[inner sep=0pt,outer sep=0pt]
    \node[anchor=south west] at ($(0, 0)$)
    	{{\includegraphics[height=2.8in]{figs/b_path2_comp.png}}};
  \end{tikzpicture}%
  }
  \caption{Joined B\'{e}zier paths for all gaps. Blue is egocircle $\egoCirc$. Yellow are 5 detected gaps. Green points are local waypoints $\goal$. Black paths are the synthesized B\'{e}zier paths $\bPath$. Red path is the selected $\bPath^*$ based on the scoring equation. \label{fig:bezier_path}}
  \vspace*{-1.5em}
\end{figure}
\newcommand{\trajRef}{\tPose_{\text{ref}}}
\newcommand{\safeFunc}{\mathcal{S}}

\subsection{NMPC Trajectory Tracking \label{sec:nmpc}}

Safe joined B\'{e}zier path $\bPath^*$ is generated in \S \ref{sec:bp}. In order to safely track the path for nonholonomic model, NMPC is applied. Assume the unicycle nonholonomic model  with state $\tPose=[\text{x}_1, \text{x}_2, \theta]^T$ and control $\tControl=[\nu, \omega]^T$. 
\begin{equation}
\begin{aligned}
    \dot{\text{x}}_1 &= \nu cos(\theta) \\
    \dot{\text{x}}_2 &= \nu sin(\theta) \\
    \dot{\theta} &= \omega
\end{aligned}
\end{equation}
In order to assign time stamps to path $\bPath^*$ based on nonholonomic dynamics, near-identity trajectory $\trajRef(t)$ \cite{1025398} is synthesized given the path and desired linear velocity $\nu_d$. Time stamps and the velocity profile $\tControl_{\text{ref}}$ are assigned to the dynamically feasible trajectory reference. However, it is possible to slightly deviate from the original B\'{e}zier path. NMPC with the safety \emph{Keyhole} ZBF constraint can guarantee safety during tracking. The scheme is formulated with initial state $\tPose(t)$ and control $\tControl(t)$ at current time $t$:
\begin{equation}
\begin{aligned}
\min_{\tControl(t+k)} \quad J(t) &= \sum_{k=0}^{N-1} ||\tPose(t+k) - \trajRef(t+k)||_Q \\ 
& \qquad + ||\tControl(t+k) - \tControl_{\text{ref}}(t+k)||_R \\
\textrm{s.t.} \quad & \tPose(t+k+1) = f(\tPose(t+k), \tControl(t+k)) \\
& \tControl_{lb} \leq \tControl(t+k) \leq \tControl_{ub} \\
& \boldsymbol{a}_{lb} \leq |\tControl(t+k+1)-\tControl(t+k)| \leq \boldsymbol{a}_{ub} \\
& h(\tPose(t+k)) \geq 0
\end{aligned}
\end{equation}
where $\norm{z}_Q=z^TQz$, and $\tControl_{lb}$, $\boldsymbol{a}_{lb}$, $\tControl_{ub}$, and $\boldsymbol{a}_{ub}$ are the lower and upper bounds of velocities and accelerations to maintain smooth motions. $N$ is the number of time step in the prediction horizon. $Q$ and $R$ are the state and control weights. $h(\tPose)$ is the {\keyhole}, which represents the inflated collision-free space $\infFreeSpace$.
% \subsection{\toedit{Keyhole Barrier Function Synthsis} \label{sec:keyhole}}
% \TODO{Ahmad}
\subsection{{\keyhole} Synthesis \label{sec:keyhole}}

% This section describes the approach to model the safety function $\safeFunc$ in NMPC.
The inflated collision-free space $\infFreeSpace$ is captured by the zero level-set of the {\keyhole}. We will use a shallow, two-layer neural network with rectified linear units (ReLU) to model the barrier function. In order to keep the network shallow and minimal, we need to leverage the geometry of keyhole shape, i.e., the straight lines and the circle. The complete expression of the {\keyhole} is 
\begin{equation} \eqlabel{nn}
 \begin{split}
%\begin{align}
    h(x) =\  & \alpha_1 R_1 + \alpha_2 R_2 + \alpha_3 R_3  + \alpha_4 R_c  + \alpha_5 R_1R_2 \\
         & + \alpha_6 R_cR_1 + \alpha_7 R_cR_2  + \alpha_8 R_cR_3  \\
         & + \alpha_9 R_1R_2R_3 + \alpha_{10} R_1R_4R_5  + \alpha_{11} R_2R_4R_5  \\
         & + \alpha_{12} R_cR_1R_4 + \alpha_{13} R_cR_2R_4  + \alpha_{14} R_cR_1R_2 \\
         & + \alpha_{15} R_cR_1R_2R_3  + b
%\end{align}
 \end{split}
\end{equation}
Effectively, all points in the domain are mapped onto the level-sets of the line and circle equations (layer 1) and their polynomial combinations (layer 2). Any point that maps onto a negative level-set is set to zero by the ReLU. As shown in \eqref{nn} by the subscripts of $R$, three additional straight lines are added, line 3, 4, and 5. Fig.~\figref{keyhole-lines}-a shows an illustrative example of the keyhole shape with line 3, which connects points $\inflate{\point_l}$ and $\inflate{\point_r}$. Lines 4 and 5 were added to cope with a special keyhole configuration shown in Fig.~\figref{keyhole-lines}-b.
% \begin{align}
%     h(x)=&\alpha_1 ReLU_1 + \alpha_2 ReLU_2 + \alpha_3 ReLU_3 +\nonumber \\
%          &\alpha_4 ReLU_1ReLU_2 + \alpha_5 ReLU_1ReLU_2ReLU_3 + \nonumber \\
%          &\alpha_6 ReLU_1ReLU_4ReLU_5 + \alpha_7 ReLU_2ReLU_4ReLU_5 + \nonumber \\
%          &\alpha_8 ReLU_cReLU_1ReLU_4 + \alpha_9 ReLU_cReLU_2ReLU_4 + \nonumber \\
%          &\alpha_{10} ReLU_cReLU_1 + \alpha_{11} ReLU_cReLU_2 + \alpha_{12} ReLU_cReLU_3 + \nonumber \\
%          &\alpha_{13} ReLU_cReLU_1ReLU_2 + \alpha_{14} ReLU_cReLU_1ReLU_2ReLU_3 + \nonumber \\
%          &\alpha_{15} ReLU_c + \alpha_{16}
% \end{align}
where, $x=[\text{x}_1,\text{x}_2]^T$, $R_i=ReLU(c_i^Tx+d_i)$, $R_c=ReLU(r^2-(x-x_c)^T(x-x_c))$, $ReLU(z)=max(0,z)$, $c_i$ and $d_i$ are the coefficients for the straight lines, and $x_c$ and $r$ are the center and radius of $\infGapCirc$, respectively.

\begin{figure}[t]
  \vspace*{0.06in}
  \centering
  \scalebox{0.75}{
  \begin{tikzpicture}[inner sep=0pt,outer sep=0pt]
    \node[anchor=south west] at ($(0, 0)$)
    	{{% \tikzstyle{block} = [draw, rectangle, fill=black!5, minimum height=1.em, minimum width=1em, inner sep=4pt]
% \tikzstyle{newblock} = [draw, rectangle, fill=black!15, minimum height=1.em, minimum width=1em, inner sep=4pt]


\begin{tikzpicture}
% \fill[black!5] (-2.5,4.5) -- (-2.5,0) -- (1.85,1.65) -- (2.5,5.5) -- cycle;
% \draw[fill=black!5] (0,0) circle (2.5);

% \draw[blue,very thick] (1.85,1.65) -- (2.5,5.5);
% \draw[blue,very thick] (-2.5,0) -- (-2.5,4.5);
% \draw[blue,very thick] (-2.5,0) -- (1.85,1.65);

% \node[anchor=west, xshift=-70pt, yshift=-5pt] at (1.85,1.65) {\large line 3};
% \node[anchor=west, xshift=-20pt, yshift=60pt] at (1.85,1.65) {\large line 1};
% \node[anchor=west, xshift=5pt, yshift=70pt] at (-2.5,0) {\large line 2};

\fill[black!5] (-8.5,4.5) -- (-8.5,0) -- (-4.15,1.65) -- (-3.5,5.5) -- cycle;
\draw[fill=black!5] (-6,0) circle (2.5);

\draw[blue,very thick] (-4.15,1.65) -- (-3.5,5.5);
\draw[blue,very thick] (-8.5,0) -- (-8.5,4.5);
\draw[black,very thick, dashed] (-8.5,0) -- (-4.15,1.65);

\node[anchor=west, xshift=-70pt, yshift=-5pt] at (-4.15,1.65) {\large line 3};
\node[anchor=west, xshift=-20pt, yshift=60pt] at (-4.15,1.65) {\large line 1};
\node[anchor=west, xshift=5pt, yshift=70pt] at (-8.5,0) {\large line 2};

\node[anchor=west, xshift=0pt, yshift=-83pt] at (-6,0) {\large (a)};

\fill[black!5] (-1.3,3.8) -- (-2.25,1.09) -- (1.4,2.0712) -- (0,2.8) -- (0.3,3.8) -- cycle;
\draw[fill=black!5] (0,0) circle (2.5);

% \draw[blue,very thick] (1.4,2.0712) -- (0,3);
% \draw[blue,very thick] (-2.45,0.5) -- (-1.6,4.8);
\draw[blue,very thick] (1.4,2.0712) -- (0,2.8);
\draw[blue,very thick] (-2.25,1.09) -- (-1.3,3.8);
\draw[black,very thick, dashed] (1.4,2.0712) -- (-2.25,1.09);

\draw[black,very thick, dashed] (0,2.8) -- (0.3,3.8);
\draw[black,very thick, dashed] (0,2.8) -- (-1.5,3.3);


\node[anchor=west, xshift=40pt, yshift=40pt] at (-2.45,0.5) {\large line 3};
\node[anchor=west, xshift=-15pt, yshift=15pt] at (1,2.3) {\large line 1};
\node[anchor=west, xshift=-18pt, yshift=50pt] at (-2.45,0.5) {\large line 2};
\node[anchor=west, xshift=10pt, yshift=-12pt] at (-0.2,4.5) {\large line 4};
\node[anchor=west, xshift=-15pt, yshift=8pt] at (-0.5,3.2) {\large line 5};

\node[anchor=west, xshift=0pt, yshift=-83pt] at (0,0) {\large (b)};


\end{tikzpicture}}};
  \end{tikzpicture}%
  }
  \caption{Keyhole diagram with additional virtual lines.\figlabel{keyhole-lines}}
  \vspace*{-1.5em}
\end{figure}

The synthesis process for the ZBF (i.e., training of the neural network) follows the technique presented in \cite{AA22PV}, which is a linear program (LP). The LP needs sampled sets, $\sampSet^u$ and $\sampSet^s$,
from the unsafe and safe regions, respectively. The unsafe points are sampled along the gap lines and circle edge, excluding the arc between the gap lines. The safe points are generated from the unsafe point by pushing them along the gradient inwards an $\epsilon$ distance. $\epsilon$ then should be set to a small value, e.g., 3$\%$ of the circle radius. 
The linear program for learning $\alpha_i$ and $b$ coefficients is 
\begin{equation}  \eqlabel{lp}
\begin{aligned}
    \min_{u} \quad & \vec{1}^T\alpha\\
    \text{s.t.} \quad & h(x_i)\leq -1,\, \forall i\in\set{l:x_l\in\sampSet^u}\\
                \quad & h(x_j)\geq +1,\, \forall j\in\set{l:x_l\in\sampSet^s}\\
                \quad & b\leq0,\ \alpha_k\geq 0,\, \forall k=1,\cdots,15
\end{aligned}
\end{equation}
where $\vec{1}=[1,\cdots,1]^T$ and $\alpha=[\alpha_1,\cdots,\alpha_{15}]$.
The coefficients $\alpha_i$ have a positivity constraint while the $b$ has a negativity constraint. Those constraints are needed so that the cost function acts as $L_1$ regulation, which promotes sparsity in the solution. Also, the value constant $\pm 1$ affects the scaling of the ZBF, much like for support vector machines. The synthesized {\keyhole} for the examples given in Fig.~\figref{keyhole-lines} are shown in Fig.~\figref{keyhole-levelset}.

The linear program was solved using Google OR-Tools \cite{ortools} in C++. For 2000 runs on the development  machine (Ubuntu 20.04, Intel i7-8750H CPU), the maximum, minimum, and average execution times were 1.6 ms, 0.71 ms, and 0.75 ms, respectively.

\subsubsection{{\keyhole} suitability as a CBF}
The {\keyhole} meets the requirements to be used as a CBF. It is monotonic across the boundary and differentiable everywhere in the positive region. The neural network in \eqref{nn} has no dead gradient, given that the terms are multiplicative combinations of the line and circle equations. Although due to the ReLU, the gradient may be non-smooth, it is not a problem for optimization, as subgradients can be used.

\begin{figure}[t]
    \vspace*{0.1in}
    \centering
    \begin{tikzpicture}[inner sep=0pt, outer sep=0pt]
        \node (fig_a) at (0in,0in)
        {\includegraphics[width=0.5\linewidth]{figs/keyhole_levelset2.jpg}};
        \node[anchor=west, xshift=70pt] (fig_b) at (fig_a)
        {\includegraphics[width=0.5\linewidth]{figs/keyhole_levelset5.jpg}};

        \node[anchor=north, xshift=-15pt, yshift=-80pt] at (fig_a){(a)};
        \node[anchor=north, xshift=-15pt, yshift=-80pt] at (fig_b){(b)};
    \end{tikzpicture}
    
    \caption{{\keyhole} for configurations in Fig.~\figref{keyhole-lines}. The value of the ZBF is represented by the color map, and the zero level-set of the ZBF is depicted by the yellow dashed line. The unsafe region outside the ZBF boundary has a negative value.}
    \figlabel{keyhole-levelset}
    \vspace*{-1.5em}
\end{figure}


\subsection{Keyhole Control Barrier Function \label{sec:cbf}}
% \TODO{Ahmad}

Since the domain of the {\keyhole} will be the position of the robot (excluding orientation), a single integrator model is assumed for the robot in CBF-QP. The reference control command is the instantaneous translational velocity of the robot, i.e., $u_r=[\nu_r\cos\theta,\nu_r\sin\theta]^T$. Again, the subscript $r$ denotes the outputs from the reference controller, which NMPC in this case. The calculated safe velocity commands by CBF-QP \eqref{cbf-qp}, $u=[\dot{x}_s,\dot{y}_s]^T$, are mapped to the robot commands using \eqref{w-qp} and \eqref{v-qp}. $\Delta\theta$ is the angle difference between the vectors $u_r$ and $u$ and is added to the current rotation rate to correct the angle difference. $k_{\omega}$ is a positive tunable parameter. The translational velocity is damped down proportional to the ratio of $\Delta\theta$ to a maximum angle $\theta_{max}$. If $\abs{\Delta\theta}\geq\theta_{max}$, the robot will only rotate. 
\begin{align}
    \omega &= \omega_r + k_{\omega}\Delta\theta \eqlabel{w-qp} \\ 
    \nu      &= \max\left(0,1-\frac{\abs{\Delta\theta}}{\theta_{max}}\right)\norm{u} \eqlabel{v-qp}
\end{align}

Overall, the {\saferGap} local planner is designed to maintain safety and passibility for nonholonomic mobile robots. From joined B\'{e}zier path planning, NMPC trajectory tracking with {\keyhole}, and control barrier function, safety guarantee is proved.


%%%%%%%%%%%%%%%%%%%%%%%%%%%%%%%%%%%%%%%%%%%%%%%%%%%%%%%%%%%%%%%%%%%%%%%%%%%%%%%%
\section{Experiments \label{sec:exp}}
\section{Experiments}
\label{sec:exp}

In this section, we demonstrate the wide range of applications and the high capabilities of Uni-Fusion. 
First, we evaluate Uni-Fusion in application 1) Incremental surface and color reconstruction, comparing its performance with SOTAs.
%
For applications 2) and 5), which are new topics, no specific benchmarks are available. 
Therefore, we showcase the performance on existing results.
%
Next, we implement application 3) and compare it with SOTA zero-shot semantic segmentation models.
%
Finally, for application 4), since infrared data is not commonly used, we collect our own dataset containing infrared values and show all applications on this data.

\subsection{Implementation Details}
\label{sec:exp:details}

In the experiments, we use our sample-based GPIS for local geometry encoding.
For each point, two additional points are sampled along normal direction, one positive and one negative, with distance $d_s=0.1$ in the local voxel's normalized space. 
Compared to derivative-based GPIS, our sample-based GPIS is more efficient in both space and time. 
For the encoder, we randomly sample $256$ anchor points from the range $[-0.5,0.5]^3$.
We utilize the first $20$ eigenpairs, resulting in a feature dimension of $20$.
The model selection process is discussed in the ablation study.

Different latent maps use different granularities.
For the surface LIM, we use a voxel size of $5\si{\centi\meter}$. 
For color which requires later comparison to NeRF, we use a voxel size of $2\si{\centi\meter}$.
For other property LIM and feature LIM, we use a voxel size of $10\si{\centi\meter}$.

For smooth reconstruction, the encoded voxel is designed overlapped following~\cite{huang2021di}.
The encoded voxel uses twice the voxel size, resulting in a half-space overlap with each neighboring voxel.
During meshing, SDFs are retrieved and interpolated from the overlapped voxels~\cite{huang2021di}.
While for the remaining properties, we sample only from its own voxel part.

The implementation runs on a PC with AMD Ryzen 9 5950X 16-core CPU and an Nvidia Geforce RTX 3090 GPU (24 GB).

\subsection{Datasets}

We evaluate incremental reconstruction on the ScanNet dataset~\cite{dai2017scannet}, TUM RGB-D dataset~\cite{sturm2012benchmark}, and Replica dataset~\cite{sucar2021imap}.
Using MSG-Net~\cite{zhang2018multi}'s material set, we transfer styles to the 3D canvas.
For open-vocabulary scene understanding, we evaluate on ScanNet segmentation data~\cite{qi2017pointnet++} and S3DIS dataset~\cite{armeni20163d}.

\subsubsection{ScanNet~\cite{dai2017scannet}}

ScanNet is a densely annotated RGB-D video dataset.
It is captured with the structure sensor~\cite{occipital} and contains 1513 scenes for training and validation.
For each scene, both images and a 3D mesh is provided, along with their 2D and 3D semantic annotations. 

ScanNet provides 312 scenes for validation, which contains a wide range of different room structures.
It has now been widely used in the thorough evaluation of the performance of reconstruction and semantic segmentation.

\subsubsection{TUM RGB-D~\cite{sturm2012benchmark}}

TUM RGB-D is a benchmark to mainly evaluate the tracking performance.
It is captured with Microsoft Kinect sensor together with ground-truth trajectory from the sensor.

\subsubsection{Replica~\cite{sucar2021imap}}

The Replica dataset refers to iMAP's pre-processed dataset~\cite{sucar2021imap}.
It is a synthetic rendered RGB-D dataset from given 3D models.
The advantage of including this dataset is that Replica does not have motion blur. 
This is better to evaluate the capability of the algorithms on reconstructing surface color.

\subsubsection{MSG-Net Style~\cite{zhang2018multi}}

MSG-Net provides material images for transfering the styles.
We select 21style fold for demonstration.
These images are given in \cref{fig:style} together with our result.

\subsubsection{ScanNet Point Cloud Segmentation Data~\cite{qi2017pointnet++}}

For point cloud semantic segmentation benchmarking, PointNet++~\cite{qi2017pointnet++} preprocesses the original ScanNet~\cite{dai2017scannet} and generates subsampled point clouds and corresponding annotations for each scene.

\subsubsection{S3DIS~\cite{armeni20163d} and 2D-3D-S~\cite{armeni2017joint}}

S3DIS is a semantic segmentation dataset for 3D point clouds.
Which is also a subset of the 2D-3D-S dataset.
The 2D-3D-S dataset is a multi-modality dataset containing 2D, 2.5D and 3D domains. 
This dataset is densely annotated with semantic labels.

Note that 2D-3D-S's 2D captures is not a RGB-D video as ScanNet.
2D-3D-S's images only have small overlap. 
Therefore, it is only suitable for semantic segmentation and not for incremental reconstruction.

\subsection{Baselines}

For online surface mapping evaluation, we select TSDF-Fusion~\cite{curless1996volumetric}, iMAP~\cite{sucar2021imap}, SOTA DI-Fusion~\cite{huang2021di} and BNV-Fusion~\cite{li2022bnv} as four baseline methods.

For the color field, we choose TSDF-Fusion~\cite{curless1996volumetric}, $\sigma$-Fusion~\cite{rosinol2023probabilistic}, iMAP~\cite{sucar2021imap}, NICE-SLAM~\cite{zhu2022nice} and even the recent hot Neural Radiance Fields model NeRF-SLAM~\cite{rosinol2022nerf} as five baselines.
While including NeRF in the comparison may not be entirely fair, we want to show how Uni-Fusion narrows the performance gap.

For the scene understanding application, we evaluate generalized zero-shot point cloud semantic segmentation with ZSLPC~\cite{cheraghian2019zero}, DeViSe~\cite{frome2013devise} and SOTA 3DGenZ~\cite{michele2021generative} for comparison.

\subsection{Metrics}

For incremental reconstruction, we evaluate the geometric reconstruction using \textbf{Accuracy}, \textbf{Completeness}, and \textbf{F1 score} according to SOTA BNV-Fusion. It firstly uniformly samples $100,000$ points from the reconstruction and ground truth meshes respectively.
Then \textbf{Accuracy} (\textbf{Completeness}) measures the percentage of reconstruction-to-groundtruth (groundtruth-to-reconstruction) distances that are lower than $2.5\si{\centi\meter}$ threshold. \textbf{F1 score} is the harmonic mean of accuracy and completeness.
For tracking performance, we use \textbf{ATE RMSE}.

To evaluate color reconstruction, we follow SOTA on this topic, NeRF to render both depth and RGB images to compare the image level \textbf{Depth L1} and \textbf{RGB PSNR}.

To compare scene understanding, we follow zero-shot point cloud semantic segmentation SOTA 3DGenZ to evaluate the \textbf{Intersection-of-Union (IoU)} and \textbf{Accuracy}.


\subsection{Reconstruction Results}

For evaluation, we first use the ScanNet validation set with 312 sequences to thoroughly test the geometric reconstruction on a large variant of scenes.
%
Then, we use TUM RGB-D to compare our modified tracking model with related works.
Because this part is not the main contribution, we give a rough overview of the tracking results.
%
To fairly evaluate the color reconstruction, we use the high quality rendered Replica dataset to compare with related works, including NeRF.

%\subsubsection{Object}
% on instance-gp
% Objective data usually has more fine detail
% 1. for detail precision
% A: no, object reconstruction is not as good as instance-ngp, so cancelled.
\begin{table*}[!]
	\centering
	\caption{Comparison to ScanNet~\cite{dai2017scannet}.
       Our method generalizes better to various scenes.
       $^*$ indicates the result from our runs of the official BNV-Fusion code.}
	\small
	%\setlength{\tabcolsep}{5mm}
	\setlength{\tabcolsep}{0.9em}
		%\resizebox{\textwidth}{!}{
		\begin{tabular}{l  c c c| c c c }
			\toprule
			Method & \begin{tabular}{@{}c@{}}Pre-Train\\ with extra dataset\end{tabular} & \begin{tabular}{@{}c@{}}Train \\ with sequences\end{tabular} & Real-time & Accuracy (\%) $\uparrow$ & Completeness (\%) $\uparrow$ & F1 score $\uparrow$ \\
			\midrule
			TSDF Fusion~\cite{zhou2018open3d} & None & None & $\checkmark$ &73.83 & 85.85 & 78.84 \\
			iMAP~\cite{sucar2021imap} & None & Online train& &68.96 & 82.12 & 74.96 \\
			DI-Fusion~\cite{huang2021di} &Object Pretrain & None & $\checkmark$&66.34 & 79.65 & 72.97 \\
			BNV-Fusion~\cite{li2022bnv} &Object Pretrain &  Post Optimization& &{74.90} & \textbf{88.12} & {80.56} \\
			BNV-Fusion$^{*}$~\cite{li2022bnv} &Object Pretrain & Post Optimization &&{73.42} & {81.75} & {77.18} \\
			\textbf{Uni-Fusion (Ours)} &None &None &$\checkmark$&\textbf{80.43} & {84.91} & \textbf{82.44} \\
			\bottomrule
		\end{tabular}
	  %}
	\label{tab:scannet}
	\vspace{-.6cm}
\end{table*}
\begin{figure*}[t]
	\subfloat[width=.33\textwidth][Accuracy]{
		\centering
		\includegraphics[width=.22\linewidth]{im/exp/recons/scannet/scannet_acc.png}
		\includegraphics[width=.1\linewidth]{im/exp/recons/scannet/scannet_acc_box.png}
	}
	\subfloat[width=.33\textwidth][Completeness]{
		\centering
		\includegraphics[width=.22\linewidth]{im/exp/recons/scannet/scannet_comp.png}
		\includegraphics[width=.1\linewidth]{im/exp/recons/scannet/scannet_comp_box.png}
	}
	\subfloat[width=.33\textwidth][F1 score]{
		\centering
		\includegraphics[width=.22\linewidth]{im/exp/recons/scannet/scannet_F1.png}
		\includegraphics[width=.1\linewidth]{im/exp/recons/scannet/scannet_F1_box.png}
	}
	\label{fig:recon:scannet:elementwise}
	\caption{Quantitative comparison on 312 scenes of the ScanNet validation set.
       We demonstrate the performance of SOTA BNV-Fusion and our Uni-Fusion.
       We sort our evaluation value and reordered all of the scores.
       The zigzag pink is the BNV-Fusion result;
       we also plot a deep-pink smoothed curve for better visualization.}
\end{figure*}

\subsubsection{Evaluation on ScanNet Dataset~\cite{dai2017scannet}}
\label{sec:exp:scannet}

We use the 312 diversified scenes from the ScanNet validation set to evaluate the performance of surface reconstruction. 
We follow the pure mapping SOTA BNV-Fusion to take every 10th posed frame as input. 
%
Without using any learning (in contrast iMAP, DI-Fusion, and BNV-Fusion do) or any post optimization (as BNV-Fusion does), our Uni-Fusion is capable to achieve precise continuous mapping performance. 

As shown in~\cref{tab:scannet}, our Uni-Fusion achieves \textbf{$+6$ higher accuracy} than the incremental surface reconstruction SOTA BNV-Fusion.
Our model does not exceed on completeness comparing to BNV-Fusion that support completion in post-optimization.
Though, Uni-Fusion's completion is still much higher than one other optimization based model iMAP.
%We consider it because our model does not support hole-completion as the optimization based models iMap and BNV-Fusion.
Overall, our Uni-Fusion model achieves higher F1-scores that quantifies the overall quality.

Please note that, SOTA BNV-Fusion is not real-time capable, since it requires post optimization of all fed frames.
Without the post-optimization, the real-time model Di-Fusion shows much worse results.
However, our \textbf{real-time} model \textbf{Uni-Fusion} is able to achieves \textbf{much better} reconstruction quality than these approaches even without post-optimization. 

\newcommand{\scannetImSize}{.16}
\begin{figure*}[t!]
	\centering
	\setlength{\tabcolsep}{0.1em}
	\renewcommand{\arraystretch}{.1}
	\begin{tabular}{|c | c |c |||c |c | c|}
		\hline
		{\Large{BNV-Fusion}} & {\Large{Uni-Fusion}} &{\Large{Ground Truth}} & {\Large{BNV-Fusion}} &{\Large{Uni-Fusion}} & {\Large{Ground Truth}} \\ \hline \hline
		
\includegraphics[width=\scannetImSize\linewidth]{im/exp/recons/scannet_qualifi/scene0568_00_bnv.png}
		&\includegraphics[width=\scannetImSize\linewidth]{im/exp/recons/scannet_qualifi/scene0568_00_mine.png}
		&\includegraphics[width=\scannetImSize\linewidth]{im/exp/recons/scannet_qualifi/scene0568_00_gt.png}
		&		\includegraphics[width=\scannetImSize\linewidth]{im/exp/recons/scannet_qualifi/scene0164_00_bnv.png}
		&\includegraphics[width=\scannetImSize\linewidth]{im/exp/recons/scannet_qualifi/scene0164_00_mine.png}
		&\includegraphics[width=\scannetImSize\linewidth]{im/exp/recons/scannet_qualifi/scene0164_00_gt.png}\\
		
\includegraphics[width=\scannetImSize\linewidth]{im/exp/recons/scannet_qualifi/scene0249_00_bnv.png}
		&\includegraphics[width=\scannetImSize\linewidth]{im/exp/recons/scannet_qualifi/scene0249_00_mine.png}
		&\includegraphics[width=\scannetImSize\linewidth]{im/exp/recons/scannet_qualifi/scene0249_00_gt.png}
		&		\includegraphics[width=\scannetImSize\linewidth]{im/exp/recons/scannet_qualifi/scene0435_00_bnv.png}
		&\includegraphics[width=\scannetImSize\linewidth]{im/exp/recons/scannet_qualifi/scene0435_00_mine.png}
		&\includegraphics[width=\scannetImSize\linewidth]{im/exp/recons/scannet_qualifi/scene0435_00_gt.png}\\
		
\includegraphics[width=\scannetImSize\linewidth]{im/exp/recons/scannet_qualifi/scene0046_00_bnv.png}
		&\includegraphics[width=\scannetImSize\linewidth]{im/exp/recons/scannet_qualifi/scene0046_00_mine.png}
		&\includegraphics[width=\scannetImSize\linewidth]{im/exp/recons/scannet_qualifi/scene0046_00_gt.png}
		&		\includegraphics[width=\scannetImSize\linewidth]{im/exp/recons/scannet_qualifi/scene0050_00_bnv.png}
		&\includegraphics[width=\scannetImSize\linewidth]{im/exp/recons/scannet_qualifi/scene0050_00_mine.png}
		&\includegraphics[width=\scannetImSize\linewidth]{im/exp/recons/scannet_qualifi/scene0050_00_gt.png}\\
		\hline
	\end{tabular}
	%\captionof{figure}
	\caption{Surface reconstruction on ScanNet dataset.}
	\label{fig:recons:scannet_demo}
	\vspace{-.5cm}
\end{figure*}

We additionally run BNV-Fusion's official implementation (emphasized with $^*$) on the 312 videos of ScanNet and do a post element-wise comparison in \cref{fig:recon:scannet:elementwise}. 
Our result is the {\color{Cyan}light blue} curve, BNV-Fusion's result is colored with {\color{Lavender}pink}.
Scene index is sorted corresponding to the score value of Uni-Fusion.
For better visualization, we smooth BNV-Fusion's curve and plot it with dark pink.
It is obvious that the score of Uni-Fusion is overall higher than BNV-Fusion's. 
Moreover, we use box-plot to conclude the statistics besides the curve plot. Uni-Fusion's scores are distributed in a higher region. For completeness which is less obvious better, Uni-Fusion's box is smaller while in a relative higher position. This means that Uni-Fusion has more stable completeness result while BNV-Fusion is more likely to get low completeness in some cases.

To summarize, our model is almost better on all 312 scenes on all accuracy, completeness and F1-score.
Which is also revealed in \cref{tab:scannet} with BNV-Fusion$^*$, that the BNV-Fusion's official implementation does not exceed Uni-Fusion on all metrics.

We plot reconstruction on selected scenes from ScanNet in~\cref{fig:recons:scannet_demo}. 
Both BNV-Fusion and our Uni-Fusion are able to produce high quality reconstruction.
We see that BNV-Fusion gives lots of small meshes on walls, which are shown as small particles in the reconstruction. 
We consider it is because BNV-Fusion use very small voxel size ($0.02\si{\meter}$) to get a high score.
This is also revealed by their \textbf{\SI{247}{MB}} mesh in average, while ours is only \textbf{\SI{54}{Mb}} in average.
Furthermore, our Uni-Fusion's mesh is more smooth and
%Both BNV-Fusion and Uni-Fusion demonstrate high quality result.
also provides high-precise color to the mesh which is not available for the Surface SOTA.

%In this test, we purely evaluate the surface reconstruction capacity with SOTAs. 
%While reconstruction is not merely surface.  
%Thus in the following, we find benchmarks for both surface and color.


\subsubsection{Tracking Evaluation on TUM RGB-D Dataset~\cite{sturm2012benchmark}}
% follow nice-slam

In the above test, we compare the performance of pure mapping.
While tracking is not the contribution focus in our paper, it is part of the reconstruction model. We follow the novel reconstruction model NICE-SLAM~\cite{zhu2022nice} to evaluate the camera tracking on the small-scale TUM RGB-D dataset.
Our Uni-Fusion uses a coarse-to-fine strategy for 3D reconstruction tracking.
From~\cref{tab:tum_rmse}, it demonstrates overall better ATE RMSE than other implicit representation models.

\begin{table}[]
		\caption{Tracking on TUM RGB-D~\cite{sturm2012benchmark}.
		ATE RMSE [$\si{\centi\meter}$] ($\downarrow$) is used as the evaluation metric.
	}
	\centering
	\footnotesize
	\setlength{\tabcolsep}{0.7em}
	\resizebox{\linewidth}{!}{
		\begin{tabular}{l|ccc}
			\hline
			& \tt{fr1/desk} &  \tt{fr2/xyz} &  \tt{fr3/office} \\
			
			\hline
			{iMAP}~\cite{sucar2021imap}      & 4.9 & 2.0 & 5.8  \\
			{iMAP$^*$}~\cite{sucar2021imap} & 7.2 & 2.1  & 9.0 \\
			{DI-Fusion~\cite{huang2021di}} & 4.4 & 2.3 & 15.6 \\
			NICE-SLAM~\cite{zhu2022nice}           & 2.7 & 1.8 & 3.0 \\
			Ours& 1.8& 0.5& 2.1 \\
			\hline
			{BAD-SLAM}\cite{schops2019bad} & 1.7  & 1.1  & 1.7 \\
			{Kintinuous}\cite{whelan2012kintinuous} & 3.7  &  2.9  & 3.0 \\
			{ORB-SLAM2}\cite{mur2017orb} & \bf 1.6  & \bf 0.4  & \bf 1.0 \\
			\hline
	\end{tabular}}
	\vspace{2pt}

	\label{tab:tum_rmse}
\end{table}

On the other hand, there also exist high accuracy algorithms from SLAM. 
By additional using Bundle Adjustment and Loop-closing techniques, their tracking quality is much better than all of the implicit based models.

%But it is dangerous to directly apply SLAM result on reconstruction. Please find our demonstration in Fig [?]. Which explains the more widely used frame-to-model strategy in 3D reconstruction.
Even though, our coarse-to-fine strategy firstly ensure not easy to lose track. Secondly, it is more suitable for surface fitting.

Which further support our test in Replica dataset.



\begin{table*}[t!]
	\centering
	\caption{Geometric (L1) and Photometric (PSNR) evaluation on the Replica dataset~\cite{sucar2021imap}.}
	\footnotesize
	\setlength{\tabcolsep}{0.36em}
	\renewcommand{\arraystretch}{1.2}
	\begin{tabular}{clcccccccccccccccccc}
		\toprule
		& & \multicolumn{1}{c}{\makecell{\tt{office-0}}} & \multicolumn{1}{c}{\makecell{\tt{office-1}}} & \multicolumn{1}{c}{\makecell{\tt{office-2}}}& \multicolumn{1}{c}{\makecell{\tt{office-3}}} & \multicolumn{1}{c}{\makecell{\tt{office-4}}} & \multicolumn{1}{c}{\makecell{\tt{room-0}}} & \multicolumn{1}{c}{\makecell{\tt{room-1}}} &  \multicolumn{1}{c}{\makecell{\tt{room-2}}} & Avg. \\
		\midrule
		\multicolumn{5}{l}{\textit{Non-continuous mapping method}}\\
		\multirow{2}{*}{\makecell{\textbf{TSDF-Fusion}~\cite{curless1996volumetric}}}
		& {\bf Depth L1} [$\si{\centi\meter}$] $\downarrow$
	 & 14.11 & 10.50 & 30.89 & 28.92 & 22.83	& 23.51 & 20.94 & 23.34 & 21.88 \\
		& {\bf PSNR } [$\si{\dB}$] $\uparrow$
		& 11.16 & 15.92 & 4.86 & 5.68 & 5.46 & 3.43 & 4.51 & 5.57 & 7.07 \\
		
		\midrule
		\multirow{2}{*}{\makecell{\textbf{$\sigma$-Fusion}\cite{rosinol2023probabilistic} }}
		& {\bf Depth L1} [$\si{\centi\meter}$] $\downarrow$
		 & 13.80 & 10.21 & 22.27 & 28.70 & 22.21& 21.92 & 19.28 & 22.40 & 20.10 \\
		& {\bf PSNR } [$\si{\dB}$] $\uparrow$
		 & 11.16 & 15.92 & 4.86 & 5.69 & 5.46& 3.45  & 4.51 & 5.57 & 7.08 \\
		
		
		
		
		
		\midrule
		\midrule
		\multicolumn{5}{l}{\textit{Continuous mapping method}}\\
		\multirow{2}{*}{\makecell{\textbf{iMAP$^*$}~\cite{sucar2021imap}}}
		& {\bf Depth L1} [$\si{\centi\meter}$] $\downarrow$
		 & 6.43 & 7.41 & 14.23 & 8.68 & 6.80& 5.70 & 4.93 & 6.94 & 7.64\\
		& {\bf PSNR } [$\si{\dB}$] $\uparrow$
		& 7.39 & 11.89 & 8.12 & 5.62 & 5.98& 5.66 & 5.31 & 5.64  & 6.95\\
		\midrule
		\multirow{2}{*}{{\makecell{\textbf{Nice-SLAM}~\cite{zhu2022nice} }}}
		& {\bf Depth L1} [$\si{\centi\meter}$] $\downarrow$
		& { 1.51 } & { 0.93 } & { 8.41 } & { 10.48 } & {2.43} & { 2.53 } & { 3.45 } & { 2.93 }  & { 4.08 } \\
		& {\bf PSNR } [$\si{\dB}$] $\uparrow$
		 & { 22.44 } & { 25.22 } & { 22.79 } & { 22.94 } & { 24.72 } & \textbf{ 29.90 } & \textbf{ 29.12 } & { 19.80 }& { 24.61 } \\
		
		
		
		
		\midrule	
		\multirow{2}{*}{{\makecell{\textbf{Uni-Fusion} (Ours) }}}
		% using abs(diff)
		%	& {\bf Depth L1} [$\si{\centi\meter}$] $\downarrow$ &\textbf{1.98}&\textbf{1.18}&\textbf{1.64}&\textbf{1.23}&\textbf{0.84}&\textbf{1.61}&\textbf{3.01}&\textbf{1.60} &\textbf{1.64}
		% follow nerf-slam to remove outlier gt first
		& {\bf Depth L1} [$\si{\centi\meter}$] $\downarrow$
		& \textbf{0.79}&\textbf{0.56}&\textbf{1.59}&\textbf{2.71}&\textbf{1.66}&\textbf{1.94}&\textbf{0.69}&\textbf{1.80}& \textbf{1.47}
		\\
		& {\bf PSNR } [$\si{\dB}$] $\uparrow$ &\textbf{33.88}&\textbf{33.31}&\textbf{25.84}&\textbf{26.01}&\textbf{28.14}&24.02&26.20&\textbf{27.17} &\textbf{28.07}
		\\
		
		\midrule
		\midrule
		\multicolumn{5}{l}{\textit{Neural radiance field method}}\\
		\multirow{2}{*}{{\makecell{\textbf{NeRF-SLAM}~\cite{rosinol2022nerf} }}}
		& {\bf Depth L1} [$\si{\centi\meter}$] $\downarrow$
	 & {2.49}   & {1.98}  & {9.13}  & {10.58} & {3.59}	& {2.97}  & {2.63}  & {2.58}  & {4.49} \\
		& {\bf PSNR } [$\si{\dB}$] $\uparrow$
	 & \textbf{48.07}  & \textbf{53.44} & \textbf{39.30} & \textbf{38.63} & \textbf{39.21} 	& \textbf{34.90} & \textbf{36.95} & \textbf{40.75}& \textbf{41.40} \\
		
		\bottomrule
	\end{tabular}%
	
	\label{tab:replica_per_scene}
\end{table*}


\begin{table*}[t!]
	\centering
	\caption{Differences among different Surface \& Color reconstruction models.}
	\small
	\setlength{\tabcolsep}{.6em}
	%{
		%\resizebox{\textwidth}{!}{
			\begin{tabular}{l | c c c c c c }
				\toprule
				Method & 
				\begin{tabular}{@{}c@{}}Pre-Train\\ with extra dataset\end{tabular}
				& \begin{tabular}{@{}c@{}}Train\\ with sequences\end{tabular}
				& Real-time	
				& Direct Output &  \begin{tabular}{@{}c@{}}Light\\ direction\end{tabular} 
				&Render\\
				\hline 
				TSDF-Fusion & None & None & $\checkmark$& Discrete TSDF &  &Ray Rasterization\\\hline
				$\sigma$-Fusion & None & None &$\checkmark$&Discrete TSDF  && Ray Rasterization\\\hline
				iMAP & None & Online Train && MLPs  & &Volumetric Rendering\\\hline
				NICE-SLAM & \begin{tabular}{@{}c@{}}Pretrain\\ with indoor dataset\end{tabular} & Online Train&& Neural Implicit Grid&  & Volumetric Rendering\\\hline
				
				NeRF-SLAM & None & Train hundred epochs &-&NeRF & $\checkmark$ &Volumetric Rendering \\\hline
				
				\textbf{Uni-Fusion} & None & None&$\checkmark$& Latent Implicit Map && Ray Rasterization\\				
				\hline
			\end{tabular}
		%}
	%}
	\label{tab:replica_diff}
\end{table*}
\subsubsection{Evaluation on Replica RGB-D Dataset~\cite{sucar2021imap}}
In this evaluation, we compare with implicit reconstruction (TSDF-Fusion, $\sigma$-Fusion) and latent implicit reconstruction models (iMAP, NICE-SLAM) that support colors. 
We also add a large-scale NeRF model, NeRF-SLAM in to the table.  
NeRF is SOTA in view-synthesis task, which is unfair to direct compare with the rest. As the rest model does not even model light directions.
We add NeRF in this part to demonstrate that Uni-Fusion strongly reduce the gap.
Note that, NeRF-SLAM embeds external tracking model ~\cite{teed2021droid} to provide poses while using SOTA NeRF implementation Instance-ngp~\cite{muller2022instant} for NeRF construction.
%Therefore it is considered the SOTA to model the colors.

Uni-Fusion track and follow our previous setting in ScanNet test to take every 10 frames for mapping.
NICE-SLAM and NeRF-SLAM produce depth and color by rendering,
To obtain result from Uni-Fusion, we cast rays from virtual camera to our result surface mesh for depth image. 
Then Uni-Fusion infer the cast points in Uni-Fusion's color LIM for color result.

From~\cref{tab:replica_per_scene}, Uni-Fusion demonstrate
best Depth L1 on all scenes with an average of \textbf{$\pmb{1.47}$$\si{\centi\meter}$ depth L1}. Which is \textbf{$\pmb{177\%}$ boost} comparing to the second best.

Moreover, excluding NeRF, our Uni-Fusion also shows the best performance to model the colors with an average of $28.07$$\si{\dB}$ PSNR.

However, it is strange that NICE-SLAM lost details while in two cases, it shows better PSNR than Uni-Fusion. 
To highlight the true result,
we plot the rendered image in \cref{fig:replica_render}.
It is obvious that our Uni-Fusion models the details of painting, carpet and quilt well. 
While NICE-SLAM just roughly models the average color.

Moreover, from the  \cref{fig:replica_render}, our Uni-Fusion's rendering quality is as precise as NeRF. 
Please also find the painting, carpet and quilt, Uni-Fusion recovered the original appearance well.
Please find the {\color{green} green window} for the emphasized region.
Uni-Fusion reproduce the high-quality appearance which is very close to NeRF on qualitative evaluation.
%It can hardly find difference between the results from NeRF-SLAM, Uni-Fusion and Ground Truth.
%
But, Uni-Fusion still has a quantitative score gap to the color rending of NeRF ($41.4$$\si{\dB}$).
Though the Uni-Fusion's rendering result is highly close to NeRF and ground truth.
%
We consider the main reasons are that \textbf{1.} Uni-Fusion does not model the light directions to points, which is essential to NeRF.
\textbf{2.} NeRF optimizes on the rendering image quality by focussing mainly on color while less on depth.
It can be revealed by the higher color rendering score with much worse depth rendering than our Uni-Fusion.
\textbf{3.} our Uni-Fusion does not support hole filling.
This directly leads to black holes in our rendered images.

We summarize the differences in \cref{tab:replica_diff}.
Similar to TSDF-Fusion and $\sigma$-Fusion, our Uni-Fusion is a forward method which, does not need any training, i.e., pre- or online training. 
Uni-Fusion also produces similar to NICE-SLAM and NeRF-SLAM an implicit map with set of latent that outputs results at arbitrary resolution.
However, we differ on the extracting of the signed distance field.
%FIXME: I do not understand the next sentence.
Uni-Fusion's latent feature rule its own region independently.
Each query value is directly inferred with the corresponding ruling latent.
While NICE-SLAM and NeRF-SLAM use a much denser grid to interpolate query features. This requires volumetric rendering for inference.

Similar to TSDF-Fusion, $\sigma$-Fusion, our Uni-Fusion is also a real-time algorithm.
iMAP, NICE-SLAM and NeRF-SLAM run hardly in real-time.
NeRF-SLAM is claiming to be real-time, which is questionable as they still need hundreds of epochs training after feeding the data.

Nevertheless, optimization with backpropagation learns pixel-to-pixel well.
It is theoretically advanced for a regression-and-fusion strategy. 
Though Uni-Fusion demonstrates its high capability to model the color, NeRF-like post-optimization would still be a good direction for further improvements of Uni-Fusion.

\newcommand{\replicaImSize}{.24}
\begin{figure*}[t]
	\centering
	\setlength{\tabcolsep}{0.1em}
	\renewcommand{\arraystretch}{.1}
	\begin{tabular}{|c | c |c |c| }
		 \hline
		{\Large{NICE-SLAM}} &{\Large{NeRF-SLAM}}&\textbf{\Large{Uni-Fusion}}&\Large{Ground Truth}\\
		%		\hline
		%		\includegraphics[width=\replicaImSize\linewidth]{im/exp/recons/replica/nice-slam/of2_1286.png} &
		%		\includegraphics[width=\replicaImSize\linewidth]{im/exp/recons/replica/mine/of2_1286.jpg} &
		%		\includegraphics[width=\replicaImSize\linewidth]{im/exp/recons/replica/mine/of2_1286.jpg} &
		%		\includegraphics[width=\replicaImSize\linewidth]{im/exp/recons/replica/gt/of2_1286.jpg} \\
		
		\hline
		\includegraphics[width=\replicaImSize\linewidth]{im/exp/recons/replica/nice-slam/rm0_769_window.png} &
		\includegraphics[width=\replicaImSize\linewidth]{im/exp/recons/replica/nerf-slam/rm0_769_window.jpg} &
		\includegraphics[width=\replicaImSize\linewidth]{im/exp/recons/replica/mine/rm0_769_window.jpg} &
		\includegraphics[width=\replicaImSize\linewidth]{im/exp/recons/replica/gt/rm0_769_window.jpg} \\
		\hline
		\includegraphics[width=\replicaImSize\linewidth]{im/exp/recons/replica/nice-slam/of3_575_window.png} &
		\includegraphics[width=\replicaImSize\linewidth]{im/exp/recons/replica/nerf-slam/of3_575_window.jpg} &
		\includegraphics[width=\replicaImSize\linewidth]{im/exp/recons/replica/mine/of3_575_window.jpg} &
		\includegraphics[width=\replicaImSize\linewidth]{im/exp/recons/replica/gt/of3_575_window.jpg} \\
		\hline
		\includegraphics[width=\replicaImSize\linewidth]{im/exp/recons/replica/nice-slam/rm1_425_window.png} &
		\includegraphics[width=\replicaImSize\linewidth]{im/exp/recons/replica/nerf-slam/rm1_425_window.jpg} &
		\includegraphics[width=\replicaImSize\linewidth]{im/exp/recons/replica/mine/rm1_425_window.jpg} &
		\includegraphics[width=\replicaImSize\linewidth]{im/exp/recons/replica/gt/rm1_425_window.jpg} \\
		\hline
		\includegraphics[width=\replicaImSize\linewidth]{im/exp/recons/replica/nice-slam/rm2_1085_window.png} &
		\includegraphics[width=\replicaImSize\linewidth]{im/exp/recons/replica/nerf-slam/rm2_1085_window.jpg} &
		\includegraphics[width=\replicaImSize\linewidth]{im/exp/recons/replica/mine/rm2_1085_window.jpg} &
		\includegraphics[width=\replicaImSize\linewidth]{im/exp/recons/replica/gt/rm2_1085_window.jpg} \\		
		
	\end{tabular}
	%\captionof{figure}
	\caption{Demonstration of color rendering on the Replica dataset. Fine appearances are highlighted in {\color{green}green window}. Small flaws are in a {\color{red}red} box.}
	\label{fig:replica_render}
	\vspace{-.5cm}
\end{figure*}

%(2) NeRF model learning radiance field that model the light on different direction on surface. While Uni-Fusion naturally treat different directional light the same color.

%\begin{table*}[t!]
%	\centering
%	\setlength{\tabcolsep}{0.1em}
%	\renewcommand{\arraystretch}{.1}
%	\begin{tabular}{c | c |c |c |c }
%		\hline 
%		\rotatebox{90}{\large{NICE-SLAM}} &
%		\includegraphics[width=\replicaImSize\linewidth]{im/exp/recons/replica/nice-slam/of3_575.png} &
%		\includegraphics[width=\replicaImSize\linewidth]{im/exp/recons/replica/nice-slam/rm0_769.png} &
%		\includegraphics[width=\replicaImSize\linewidth]{im/exp/recons/replica/nice-slam/rm1_425.png} &
%		\includegraphics[width=\replicaImSize\linewidth]{im/exp/recons/replica/nice-slam/rm2_1085.png} \\
%		\hline
%		\rotatebox{90}{\large{NeRF-SLAM}} &
%		\includegraphics[width=\replicaImSize\linewidth]{im/exp/recons/replica/mine/of3_575.jpg} &
%		\includegraphics[width=\replicaImSize\linewidth]{im/exp/recons/replica/mine/rm0_769.jpg} &
%		\includegraphics[width=\replicaImSize\linewidth]{im/exp/recons/replica/mine/rm1_425.jpg} &
%		\includegraphics[width=\replicaImSize\linewidth]{im/exp/recons/replica/mine/rm2_1085.jpg} \\	
%		\hline
%		\rotatebox{90}{\textbf{\Large{Uni-Fusion}}} &
%		\includegraphics[width=\replicaImSize\linewidth]{im/exp/recons/replica/mine/of3_575.jpg} &
%		\includegraphics[width=\replicaImSize\linewidth]{im/exp/recons/replica/mine/rm0_769.jpg} &
%		\includegraphics[width=\replicaImSize\linewidth]{im/exp/recons/replica/mine/rm1_425.jpg} &
%		\includegraphics[width=\replicaImSize\linewidth]{im/exp/recons/replica/mine/rm2_1085.jpg} \\
%		\hline
%		\rotatebox{90}{\large{Ground Truth}} &
%		\includegraphics[width=\replicaImSize\linewidth]{im/exp/recons/replica/gt/of3_575.jpg} &
%		\includegraphics[width=\replicaImSize\linewidth]{im/exp/recons/replica/gt/rm0_769.jpg} &
%		\includegraphics[width=\replicaImSize\linewidth]{im/exp/recons/replica/gt/rm1_425.jpg} &
%		\includegraphics[width=\replicaImSize\linewidth]{im/exp/recons/replica/gt/rm2_1085.jpg} \\
%		\hline		
%		
%	\end{tabular}
%	\captionof{figure}{Demonstration of color rendering on Replica dataset.}
%\end{table*}

\subsection{Ablation study}
\label{exp:surface:ablation}

\begin{figure}[]
	\centering
%		\subfloat[width=\textwidth][Sample based]{
%		\centering
%		\includegraphics[width=.7\linewidth]{im/exp/ablation/GPIS/seq3_sample_color.png}
%	}\\
%	\subfloat[width=\textwidth][Derivative based]{
%		\centering
%		\includegraphics[width=.7\linewidth]{im/exp/ablation/GPIS/seq3_derivative_color.png}
%	}
		\includegraphics[width=.7\linewidth]{im/exp/ablation/GPIS/seq3_sample_color_a.png}
		\includegraphics[width=.7\linewidth]{im/exp/ablation/GPIS/seq3_derivative_color_b.png}
	\caption{Ablation study on surface construction basis. (a) Sample based. (b) Derivative based.}
	\label{fig:ablation:GPIS}
\end{figure}


\begin{table}[]
	\caption{Ablation study on tracking.
	}
	\centering
	\footnotesize
	\setlength{\tabcolsep}{0.7em}
	\resizebox{\linewidth}{!}{
		\begin{tabular}{l|ccc}
			\hline
			& \tt{fr1/desk} &  \tt{fr2/xyz} &  \tt{fr3/office} \\
			\hline
			External& 2.1& 0.5& 2.5 \\
			External+Internal&1.8& 0.5& 2.1 \\
			\hline
	\end{tabular}}
	\vspace{-2pt}
	%\vspace{-1cm}
	\label{tab:tum_rmse2}
\end{table}

\begin{figure}
	\centering
	\includegraphics[width=.7\linewidth]{im/exp/ablation/voxel_size/seq_voxel_size.png}
	%	\subfloat[width=.33\textwidth][0.1]{
		%		\centering
		%		\includegraphics[width=.33\linewidth]{im/exp/ablation/voxel_size/seq3_0_1_color.png}
		%	}
	%	\subfloat[width=.3\textwidth][0.05]{
		%		\centering
		%		\includegraphics[width=.33\linewidth]{im/exp/ablation/GPIS/seq3_sample_color.png}
		%	}
	%	\subfloat[width=.33\textwidth][0.02]{
		%	\centering
		%	\includegraphics[width=.33\linewidth]{im/exp/ablation/voxel_size/seq3_0_02_color.png}
		%	}
	\caption{Ablation study on voxel size.}
	\label{fig:ablation:voxel_size}
\end{figure}




 \newcommand{\styleImSize}{.2}
\begin{figure*}[b!]
	\vspace{-.5cm}
	\centering
	\setlength{\tabcolsep}{0.1em}
	\renewcommand{\arraystretch}{.1}
	\resizebox{\textwidth}{!}{\begin{tabular}{ccccc}
			%		\includegraphics[width=\styleImSize\line]{im/exp/style/style/0} &
			%		\includegraphics[width=\styleImSize\linewidth]{im/exp/style/style/1} &
			%		\includegraphics[width=\styleImSize\linewidth]{im/exp/style/style/2} &
			%		\includegraphics[width=\styleImSize\linewidth]{im/exp/style/style/3} &
			%		\includegraphics[width=\styleImSize\linewidth]{im/exp/style/style/4} &
			%		\includegraphics[width=\styleImSize\linewidth]{im/exp/style/style/5} &
			%		\includegraphics[width=\styleImSize\linewidth]{im/exp/style/style/6} \\
			\hline\hline
			\includegraphics[width=\styleImSize\linewidth]{im/exp/style/processed/office0_0.png} &
			\includegraphics[width=\styleImSize\linewidth]{im/exp/style/processed/office0_1.png} &
			\includegraphics[width=\styleImSize\linewidth]{im/exp/style/processed/office0_2.png} &
			\includegraphics[width=\styleImSize\linewidth]{im/exp/style/processed/office0_3.png} &
			\includegraphics[width=\styleImSize\linewidth]{im/exp/style/processed/office0_4.png} \\
			\includegraphics[width=\styleImSize\linewidth]{im/exp/style/processed/office0_5.png} &
			\includegraphics[width=\styleImSize\linewidth]{im/exp/style/processed/office0_6.png} &
			\includegraphics[width=\styleImSize\linewidth]{im/exp/style/processed/office0_7.png} &
			\includegraphics[width=\styleImSize\linewidth]{im/exp/style/processed/office0_8.png} &
			\includegraphics[width=\styleImSize\linewidth]{im/exp/style/processed/office0_9.png} \\
			\includegraphics[width=\styleImSize\linewidth]{im/exp/style/processed/office0_10.png} &
			\includegraphics[width=\styleImSize\linewidth]{im/exp/style/processed/office0_11.png} &
			\includegraphics[width=\styleImSize\linewidth]{im/exp/style/processed/office0_12.png} &
			\includegraphics[width=\styleImSize\linewidth]{im/exp/style/processed/office0_13.png} &
			\includegraphics[width=\styleImSize\linewidth]{im/exp/style/processed/office0_14.png} \\
			\includegraphics[width=\styleImSize\linewidth]{im/exp/style/processed/office0_15.png} &
			%\includegraphics[width=\styleImSize\linewidth]{im/exp/style/processed/office0_16.png} &
			\includegraphics[width=\styleImSize\linewidth]{im/exp/style/processed/office0_17.png} &
			\includegraphics[width=\styleImSize\linewidth]{im/exp/style/processed/office0_18.png} &
			\includegraphics[width=\styleImSize\linewidth]{im/exp/style/processed/office0_19.png} &
			\includegraphics[width=\styleImSize\linewidth]{im/exp/style/processed/office0_20.png} 
			\\	\hline
		\end{tabular}
	}
	%\captionof{figure}
	\caption{Style transfer on 3D canvas.}
	\label{fig:style}
\end{figure*}


\subsubsection{ Sample-based or Derivative-based}

We select the surface model with our own captured sequences. 
All settings are detailed in \cref{sec:exp:details}.
As shown in~\cref{fig:ablation:GPIS}, reconstruction of Yijun's office is demonstrated. 
Both models are able to construct, but the derivative-based model produces a lot of noise on the surface.
This is because for smoothness purpose, we build voxels that are overlapped to its neighbor, which causes redundant voxels near the surface.
For those redundant voxels, no center sample is provided and thus the derivative based surface construction builds bad SDFs on unknow region of the voxels.


Instead, sample-based surface construction does not have this problem as it adds more points in voxels and is able to construct highly-smooth surfaces.
From which, we find well constructed and colored white board, chair, school bag and even the oranges.

\subsubsection{Tracking}


Our Uni-Fusion use a coarse-to-fine strategy for tracking. 
An external tracking model is running in one thread aside from the mapping thread.
In the mapping thread, it takes pose result from the external tracking and applies the internal tracking for colored point cloud.

The result is demonstrated in~\cref{tab:tum_rmse2}. 
The coarse-to-fine is relatively better on trajectory estimation.

\subsubsection{Voxel size}

Testing the office scene, we vary the voxel size from low to high. 
From~\cref{fig:ablation:voxel_size}, when low voxel size $0.02$m is used, the surface is rough.
Then when voxel size goes larger, the smoothness is improved.
However, when we use $0.1$m voxel size, the surface color is blur. 
Considering Uni-Fusion produces a surface color field, the quality of surface directly affect the coloring.
Thus, continuing enlarging the voxel size also results in worse surface results.

Therefore, in the above experiments, $0.05$m voxel size is utilized for surface construction.
In addition, each voxel for encoding are actually with size $0.1$m, since we use overlapped voxel.

%\subsubsection{Anchor number and feature dimension}


\subsection{Application: 2D-to-3D Transfer}
\label{sec:fabircated_prop}

Applications such as 2) and 4) can be easily integrated with application 1) incremental reconstruction (\cref{sec:incremental_reconstruction}) by incorporating the fabricated result together with the point cloud.
%
For instance, given RGB-D frames, we detect saliency or transfer image styles to generate a fabricated $X$ image. Here, $X$ represents saliency, style, or other properties. 
By combining $X$ with depth information through unprojection,
we assign
the fabricated values to corresponding points, resulting in point pairs ($\V X$, $\V Q_{X}$).

Similar to the reconstruction pipeline in~\cref{fig:recons_and_scene_understanding}, we employ encoding (\cref{sec:encoder}) and fusion (\cref{eq:fuse}) to construct a global LIM for the fabricated properties $X$.
This global LIM represents a surface $X$ fields that is utilized for subsequent inference.

While it is possible to similarly transfer a 2D semantic image to 3D,
it may not be feasible in practice due to the need for multiple passes of different categories of semantic information 
 on the same dataset (such as object, usability, etc.).
Therefore, in the following section, we demonstrate the construction of a surface feature field for scene understanding application that satisfies various 
requirements through a single mapping pass.

\begin{table*}[b!]
	%\renewcommand{\arraystretch}{0.9}
	%\setlength{\tabcolsep}{3pt}
	\caption{GZSL semantic segmentation results. Scores are in \%.
	  $^\dagger$ indicate 3DGenZ's adaption of the method.
       Note that, Uni-Fusion-SU does not even train with the seen classes.}
	\centering
	\begin{tabular}{l|c|c |c ||ccc|ccc}
		\toprule
		\multicolumn{1}{c}{}& \multicolumn{2}{c|}{Training set} & Inference input &\multicolumn{3}{c|}{ScanNet } & \multicolumn{3}{c}{S3DIS}\\
		& Backbone & Classifier & &$Seen$& $Unseen$ & $All$&$Seen$& $Unseen$ & $All$
%		\multicolumn{3}{c|}{mIoU} & 
%		\multicolumn{3}{c|}{mIoU} \\ 
%		&&&& $Seen$& $Unseen$ & $All$&$Seen$& $Unseen$ & $All$
		%\cellcolor{white}{}  
		%\cellcolor{white}{}
		\\
		\midrule
		
		\multicolumn{5}{l}{\textit{Supervised methods with different levels of supervision}}\\
		
		Full supervision & $seen \cup unseen$ & $seen \cup unseen$ & Point Cloud &43.3&51.9 &45.1&74.0&50.0&66.6 \\
		
		ZSL backbone & $seen$ & $seen \cup unseen$  &Point Cloud&41.5&39.2 & 40.3&60.9& 21.5&  48.7 \\
		
		ZSL-trivial & $seen$ & $seen$ &Point Cloud&39.2&0.0&31.3&70.2 &0.0&48.6  \\
		\midrule
		\multicolumn{5}{l}{\textit{Generalized zero-shot-learning methods}}\\
		
		ZSLPC-Seg~\cite{cheraghian2019zero}$^\dagger$ & $seen$ & $unseen$  &Point Cloud&28.2&0.0& 22.6&65.6 &0.0& 45.3\\
		
		DeViSe-3DSeg~\cite{frome2013devise}$^\dagger$ & $seen$ & $unseen$   &Point Cloud &20.0&0.0&16.0&70.2&0.0& 48.6\\ 
		%ZSLPC-Seg~\cite{cheraghian2019zero}$^\dagger$ & $seen$ & $unseen$  &  4.0&13.9\\
		%DeViSe-3DSeg~\cite{frome2013devise}$^\dagger$ & $seen$ & $unseen$   &  3.0&10.9\\
		3DGenZ~\cite{michele2021generative} & $seen$ & $seen \cup \hat{unseen}$  &Point Cloud &32.8&7.7& {27.8}&53.1&7.3&   \textbf{39.0} \\
		\midrule
		\multicolumn{5}{l}{\textit{Zero-shot learning + map fusion}}\\
		Uni-Fusion-SU (Ours) &None&None&Sparse Frames&31.0&\textbf{41.9}&\textbf{32.9} &31.3&\textbf{24.0}&29.0\\
		\bottomrule
		\multicolumn{1}{l}{}\\[-7pt]
	\end{tabular}

	\label{tab:sem_seg_overview}
\end{table*}

\begin{figure*}[t!]
	\centering
	\setlength{\tabcolsep}{0.1em}
	\renewcommand{\arraystretch}{.1}
	\begin{tabular}{|c | c |c |||c |c | c|}
		\toprule
		{\Large{3DGenZ}} & {\Large{Uni-Fusion}} &{\Large{Ground Truth}} & {\Large{3DGenZ}} &{\Large{Uni-Fusion-SU}} & {\Large{Ground Truth}} \\ \midrule
		
		\includegraphics[width=\scannetImSize\linewidth]{im/exp//ss/gen3dz_0568.png}
		&\includegraphics[width=\scannetImSize\linewidth]{im/exp//ss/mine_0568.png}
		&\includegraphics[width=\scannetImSize\linewidth]{im/exp//ss/gt_0568.png}
		&		\includegraphics[width=\scannetImSize\linewidth]{im/exp//ss/gen3dz_0164.png}
		&\includegraphics[width=\scannetImSize\linewidth]{im/exp//ss/mine_0164.png}
		&\includegraphics[width=\scannetImSize\linewidth]{im/exp//ss/gt_0164.png}\\
		
		
		\includegraphics[width=\scannetImSize\linewidth]{im/exp//ss/gen3dz_0249.png}
		&\includegraphics[width=\scannetImSize\linewidth]{im/exp//ss/mine_0249.png}
		&\includegraphics[width=\scannetImSize\linewidth]{im/exp//ss/gt_0249.png}
		&		\includegraphics[width=\scannetImSize\linewidth]{im/exp//ss/gen3dz_0435.png}
		&\includegraphics[width=\scannetImSize\linewidth]{im/exp//ss/mine_0435.png}
		&\includegraphics[width=\scannetImSize\linewidth]{im/exp//ss/gt_0435.png}\\
		
		
		\includegraphics[width=\scannetImSize\linewidth]{im/exp//ss/gen3dz_0046.png}
		&\includegraphics[width=\scannetImSize\linewidth]{im/exp//ss/mine_0046.png}
		&\includegraphics[width=\scannetImSize\linewidth]{im/exp//ss/gt_0046.png}
		&		\includegraphics[width=\scannetImSize\linewidth]{im/exp//ss/gen3dz_0050.png}
		&\includegraphics[width=\scannetImSize\linewidth]{im/exp//ss/mine_0050.png}
		&\includegraphics[width=\scannetImSize\linewidth]{im/exp//ss/gt_0050.png}\\
		\bottomrule
		
	\end{tabular}
	\includegraphics[width=\linewidth]{im/ss_colorbar}
	%\captionof{figure}
	\caption{Demonstration of semantic segmentation on the ScanNet dataset.
       Selected scenes are consistent with~\cref{fig:recons:scannet_demo}}
	\label{fig:segmentation_demo}
	
\end{figure*}

\subsection{Scene Understanding Results}

Saliency detection effectively highlights the objects of interest.
This is also considered part of 3D semantic understanding.
However, as the semantics categories vary, fusing different categories of semantics into multiple LIMs can be inefficient.
%
Therefore, in this section, we utilize Uni-Fusion to fuse and construct a surface field for high-dimensional CLIP embeddings.
With a single LIM, we can generate different semantic results based on corresponding commands.
%
Since now our Uni-Fusion works with OpenSeg for scene understanding purposes, we call it Uni-Fusion-SU.

\subsubsection{Semantic Segmentation}
\label{sec:exp:semantic}

We first evaluate our model on generalized zero-shot point cloud semantic segmentation application.
Generalized Zero-Shot Learning (GZSL) differs from Zero-Shot Learning (ZSL) in that ZSL only predicts classes unseen during training, while GZSL predicts both unseen and seen classes~\cite{michele2021generative}.
Therefore, comparing our results with GZSL SOTAs provides a better understanding of the potential of Uni-Fusion-SU, as it does not train on both seen and unseen. 

This test uses ScanNet and S3DIS datasets for benchmarking. 
It is important to note that the \textbf{compared baselines are trained on the corresponding datasets}.
Our Uni-Fusion-SU uses OpenSeg to provide the 2D image level feature ebmedding.
Although \textbf{Uni-Fusion-SU} is also zero-shot, \textbf{it does not touch any ScanNet or S3DIS annotations}.

We demonstrate the mIoU scores in~\cref{tab:sem_seg_overview}.
In particular, our model achieves best results among the zero-shot learning methods on the ScanNet dataset and remains competitive with fully supervised methods.

Furthermore, we provide results specifically for the unseen classes in~\cref{sup:tab:sn_acc_miou}.
Although not as good as the fully supervised approach, Uni-Fusion-SU performs much better than 3DGenZ.
In addition, our Uni-Fusion-SU demonstrates high precision in classes such as sofa and Toilet, even when compared to the fully supervised model.

\begin{table}[htbp]
		\caption{Classwise GZSL semantic segmentation performance (\%) on the ScanNet unseen split.}
	\centering
	\newcommand*\rotext{\multicolumn{1}{R{45}{1em}}}
	\setlength{\tabcolsep}{1.7pt}
	\begin{tabular}{@{}l@{~}c|rrrr|r@{}}
		\toprule		
		& &
		{\textbf{Bookshelf}} & {\textbf{Desk}} & {\textbf{Sofa}} & {\textbf{Toilet}} & \stackbox{mean} \\
		
		\midrule
		FSL (Fully supervise) & IoU & 	56.9&	30.0&	57.4&	63.4 & 51.9
		\\ 
		3DGenZ (Zero-shot) & IoU & 	6.3&	3.3&	13.1&	8.1 & 7.7
		\\
		Uni-Fusion-SU (Ours) & IoU &38.3&16.8&51.7&60.9&41.9
	\\ \midrule 
	3DGenZ (Zero-shot)& Acc. & 	13.4&	5.9&	49.6&	26.3 &23.8
	\\
	Uni-Fusion-SU (Ours) & Acc. &61.9&29.6&67.4&91.6& 62.6
		\\
		\bottomrule
	\end{tabular}

	\label{sup:tab:sn_acc_miou}
\end{table}

However, in the S3DIS dataset, our model does not outperform 3DGenZ and other methods as shown in~\cref{tab:sem_seg_overview}.

Even in the result of unsceened data, as presented in \cref{sup:tab:s3dis_acc_miou}, we observe that Uni-Fusion-SU hardly finds some classed, e.g. Beam and Column, which are not commonly annotated objects. 
However, for common objects like sofa and window, our model performs much better.

\begin{table}[htbp]
		\caption{Classwise GZSL semantic segmentation performance (\%) on the S3DIS unseen split.}
	\centering
	\newcommand*\rotext{\multicolumn{1}{R{45}{1em}}}
	\setlength{\tabcolsep}{1.7pt}
	\begin{tabular}{@{}l@{~}c|rrrr|r@{}}
		\toprule		
		& &
		{\textbf{Beam}} & {\textbf{Column}} & {\textbf{Sofa}} & {\textbf{Window}} & \stackbox{mean} \\
		
		\midrule
		FSL (Fully supervise) & IoU & 	63.1&	10.2&	54.1&	72.4 & 50.0
		\\ 
		3DGenZ (Zero-shot) & IoU & 	13.9&	2.4&4.9&	8.1 &7.3
		\\
		Uni-Fusion-SU (Ours) & IoU &5.5&0.02&57.4&32.9&	24.0
		\\ \midrule 
		3DGenZ (Zero-shot) & Acc. & 	20.0&	9.1&	62.4&	23.7 &28.8
		\\
		Uni-Fusion-SU (Ours) & Acc. &41.5&0.02&78.3&42.1& 40.5
		\\	
		\bottomrule
	\end{tabular}

	\label{sup:tab:s3dis_acc_miou}
\end{table}

We present the results of the semantic segmentation in~\cref{fig:segmentation_demo}. 
It is evident that, 3DGenZ's result contains more noise, as seen in the spotted sofa, bed and other objects.
Conversely, Uni-Fusion-SU's results are generally smoother and more precise.

%
%\begin{figure*}[htbp]
%	\centering
%	\includegraphics[width=.3\linewidth]{example-image-golden}
%	\includegraphics[width=.3\linewidth]{example-image-golden}
%	\includegraphics[width=.3\linewidth]{example-image-golden}
%	\\
%	\includegraphics[width=.3\linewidth]{example-image-golden}
%	\includegraphics[width=.3\linewidth]{example-image-golden}
%	\includegraphics[width=.3\linewidth]{example-image-golden}
%	
%	\caption{Semantic segmentation result on ScanNet.}
%\end{figure*}
%
%\begin{figure*}[htbp]
%	\centering
%	\includegraphics[width=.3\linewidth]{example-image-golden}
%	\includegraphics[width=.3\linewidth]{example-image-golden}
%	\includegraphics[width=.3\linewidth]{example-image-golden}
%	\\
%	\includegraphics[width=.3\linewidth]{example-image-golden}
%	\includegraphics[width=.3\linewidth]{example-image-golden}
%	\includegraphics[width=.3\linewidth]{example-image-golden}
%	
%	\caption{Semantic segmentation result on S3DIS.}
%\end{figure*}

\subsubsection{Scene Understanding with Different Properties}

\begin{figure*}[t!]
	\centering
	\setlength{\tabcolsep}{0.1em}
	\renewcommand{\arraystretch}{.1}
	\resizebox{\textwidth}{!}{\begin{tabular}{|c | c | c | c | c | c|}
			\toprule 
			& \textbf{scene0568\_00} & \textbf{scene0249\_00} & \textbf{scene0435\_00} & \textbf{office3} & \textbf{room0}\\
			\midrule
			{} &
			\raisebox{-.5\height}{\includegraphics[width=\fabImSize\linewidth]{im/exp/fab/scannet/0568_color.png}} & %\raisebox{-.5\height}{\includegraphics[width=\fabImSize\linewidth]{im/exp/fab/scannet/0164_color.png}} &
			\raisebox{-.5\height}{\includegraphics[width=\fabImSize\linewidth]{im/exp/fab/scannet/0249_color.png}} & \raisebox{-.5\height}{\includegraphics[width=\fabImSize\linewidth]{im/exp/fab/scannet/0435_color.png}}
			&
			\raisebox{-.5\height}{\includegraphics[width=\fabImSize\linewidth]{im/exp/fab/replica/office3_color.png}}
			&
			\raisebox{-.5\height}{\includegraphics[width=\fabImSize\linewidth]{im/exp/fab/replica/room0_color.png}}\\ %\raisebox{-.5\height}{\includegraphics[width=\fabImSize\linewidth]{im/exp/fab/scannet/0050_color.png}} %\includegraphics[width=\fabImSize\linewidth]{im/exp/fab/replica/office3_color.png}
			\\
			\textbf{Desk}  &
			\raisebox{-.5\height}{\includegraphics[width=\fabImSize\linewidth]{im/exp/fab/scannet/0568_lt_desk.png}}&
			%\raisebox{-.5\height}{\includegraphics[width=\fabImSize\linewidth]{im/exp/fab/scannet/0164_lt_desk.png}}&
			\raisebox{-.5\height}{\includegraphics[width=\fabImSize\linewidth]{im/exp/fab/scannet/0249_lt_desk.png}}&
			\raisebox{-.5\height}{\includegraphics[width=\fabImSize\linewidth]{im/exp/fab/scannet/0435_lt_desk.png}}
			&
			\raisebox{-.5\height}{\includegraphics[width=\fabImSize\linewidth]{im/exp/fab/replica/office3_lt_desk.png}}
			&
			\raisebox{-.5\height}{\includegraphics[width=\fabImSize\linewidth]{im/exp/fab/replica/room0_lt_desk.png}}\\
			%\raisebox{-.5\height}{\includegraphics[width=\fabImSize\linewidth]{im/exp/fab/scannet/0050_lt_desk.png}}
			%\includegraphics[width=\fabImSize\linewidth]{im/exp/fab/replica/office3_saliency.png}
			\\
			
			\textbf{Sofa} &
			\raisebox{-.5\height}{\includegraphics[width=\fabImSize\linewidth]{im/exp/fab/scannet/0568_lt_sofa.png}} &
			%\raisebox{-.5\height}{\includegraphics[width=\fabImSize\linewidth]{im/exp/fab/scannet/0164_lt_sofa.png}} &
			\raisebox{-.5\height}{\includegraphics[width=\fabImSize\linewidth]{im/exp/fab/scannet/0249_lt_sofa.png}} &
			\raisebox{-.5\height}{\includegraphics[width=\fabImSize\linewidth]{im/exp/fab/scannet/0435_lt_sofa.png}}&
			\raisebox{-.5\height}{\includegraphics[width=\fabImSize\linewidth]{im/exp/fab/replica/office3_lt_sofa.png}}
			&
			\raisebox{-.5\height}{\includegraphics[width=\fabImSize\linewidth]{im/exp/fab/replica/room0_lt_sofa.png}}\\
			%\raisebox{-.5\height}{\includegraphics[width=\fabImSize\linewidth]{im/exp/fab/scannet/0050_lt_sofa.png}}
			%\includegraphics[width=\fabImSize\linewidth]{im/exp/fab/replica/office3_style.png}
			\\
			\textbf{Work} &
			\raisebox{-.5\height}{\includegraphics[width=\fabImSize\linewidth]{im/exp/fab/scannet/0568_lt_work.png}} &
			%\raisebox{-.5\height}{\includegraphics[width=\fabImSize\linewidth]{im/exp/fab/scannet/0164_lt_work.png}} &
			\raisebox{-.5\height}{\includegraphics[width=\fabImSize\linewidth]{im/exp/fab/scannet/0249_lt_work.png}} &
			\raisebox{-.5\height}{\includegraphics[width=\fabImSize\linewidth]{im/exp/fab/scannet/0435_lt_work.png}}&
			\raisebox{-.5\height}{\includegraphics[width=\fabImSize\linewidth]{im/exp/fab/replica/office3_lt_work.png}}
			&
			\raisebox{-.5\height}{\includegraphics[width=\fabImSize\linewidth]{im/exp/fab/replica/room0_lt_work.png}}\\
			%\raisebox{-.5\height}{\includegraphics[width=\fabImSize\linewidth]{im/exp/fab/scannet/0050_lt_work.png}}
			%\includegraphics[width=\fabImSize\linewidth]{im/exp/fab/replica/office3_style.png}
			\\
			\textbf{Sittable} &
			\raisebox{-.5\height}{\includegraphics[width=\fabImSize\linewidth]{im/exp/fab/scannet/0568_lt_sit.png}} &
			%\raisebox{-.5\height}{\includegraphics[width=\fabImSize\linewidth]{im/exp/fab/scannet/0164_lt_sit.png}} &
			\raisebox{-.5\height}{\includegraphics[width=\fabImSize\linewidth]{im/exp/fab/scannet/0249_lt_sit.png}} &
			\raisebox{-.5\height}{\includegraphics[width=\fabImSize\linewidth]{im/exp/fab/scannet/0435_lt_sit.png}}&
			\raisebox{-.5\height}{\includegraphics[width=\fabImSize\linewidth]{im/exp/fab/replica/office3_lt_sit.png}}
			&
			\raisebox{-.5\height}{\includegraphics[width=\fabImSize\linewidth]{im/exp/fab/replica/room0_lt_sit.png}}\\
			%\raisebox{-.5\height}{\includegraphics[width=\fabImSize\linewidth]{im/exp/fab/scannet/0050_lt_sit.png}}
			%\includegraphics[width=\fabImSize\linewidth]{im/exp/fab/replica/office3_style.png}
			\\
			\textbf{Wood} &
			\raisebox{-.5\height}{\includegraphics[width=\fabImSize\linewidth]{im/exp/fab/scannet/0568_lt_wood.png}} &
			%\raisebox{-.5\height}{\includegraphics[width=\fabImSize\linewidth]{im/exp/fab/scannet/0164_lt_wood.png}} &
			\raisebox{-.5\height}{\includegraphics[width=\fabImSize\linewidth]{im/exp/fab/scannet/0249_lt_wood.png}} &
			\raisebox{-.5\height}{\includegraphics[width=\fabImSize\linewidth]{im/exp/fab/scannet/0435_lt_wood.png}}&
			\raisebox{-.5\height}{\includegraphics[width=\fabImSize\linewidth]{im/exp/fab/replica/office3_lt_wood.png}}
			&
			\raisebox{-.5\height}{\includegraphics[width=\fabImSize\linewidth]{im/exp/fab/replica/room0_lt_wood.png}}\\
			%\raisebox{-.5\height}{\includegraphics[width=\fabImSize\linewidth]{im/exp/fab/scannet/0050_lt_wood.png}}
			%\includegraphics[width=\fabImSize\linewidth]{im/exp/fab/replica/office3_style.png}
			\\
			
			
			\bottomrule
		\end{tabular}
	}
	%\captionof{figure}
	\caption{Demonstration of the original mesh, highlighted semantic mesh given various queries.}
	\label{fig:fab_lt}
	\vspace{-.5cm}
\end{figure*}

The main contribution of this application is that, Uni-Fusion is the first model to construct a continuous mapping of high-dimensional embeddings onto the surface without the need for any training of the map representation.
%
In the previous experiment (\cref{sec:exp:semantic}), we evaluate the performance of generalized zero-shot semantic segmentation.
However, the potential of Uni-Fusion goes beyond semantic segmentation.
%
By constructing a LIM, we obtain a surface CLIP feature field.
This enables us to query various semantic categories such as 
%without the need of multiple LIMs or rerun for other properties, we query 
\textbf{Object, Room Type, Material, Affordance and Activity} without requiring multiple LIMs or re-running the model.

We present the results in \cref{fig:fab_lt}, 
where we query object (desk, sofa), activity (work), affordance (sittable), and material (wood).
Uni-Fusion-SU accurately identifies and highlights the object and material regions.
However, for less specific commands such as work or sittable, the model provides a wider range of results with less confidence (indicated by dull yellow).
Nevertheless, the suggested options are also roughly correct.









\subsection{Time}

We run all of the applications in a single pass using our captured office sequences and evaluate the time cost of construction and fusion of each LIM. 
The average time cost across frames is shown in~\cref{tab:time}.

\begin{table}[htbp]
	\caption{Time required for each frame.
	}
	\centering
	\footnotesize
	\setlength{\tabcolsep}{0.7em}
	\resizebox{\linewidth}{!}{
		\begin{tabular}{l|ccccccc}
			\toprule
			&Surface & Color & Infrared & Style & Saliency & Latent&Internal Track \\ \midrule
			Time ($\si{\second}$)&0.100 & 0.038 & 0.045 & 0.048 & 0.045 &0.011 &0.225 \\ \bottomrule
	\end{tabular}}
	
	\label{tab:time}
\end{table}

\newcommand{\mineImSize}{.32}
%\begin{table*}[t!]
%	\centering
%	\setlength{\tabcolsep}{0.1em}
%	\renewcommand{\arraystretch}{.1}
%	\begin{tabular}{|c | c |c |}
%		\hline 
%		{Color} &{Infrared} & {Saliency} \\
%		\includegraphics[width=\mineImSize\linewidth]{im/exp/fab/mine/office/seq3_color.png} &
%		\includegraphics[width=\mineImSize\linewidth]{im/exp/fab/mine/office/seq3_color.png} &
%		\includegraphics[width=\mineImSize\linewidth]{im/exp/fab/mine/office/seq3_saliency.png} \\
%		{Style 1}&{Style 1}&{Style 1}\\
%		\includegraphics[width=\mineImSize\linewidth]{im/exp/fab/mine/office/seq3_style.png}&
%		\includegraphics[width=\mineImSize\linewidth]{im/exp/fab/mine/office/seq3_style.png}&
%		\includegraphics[width=\mineImSize\linewidth]{im/exp/fab/mine/office/seq3_style.png}\\
%		{Sofa}&{Desk}&{Soft}\\
%		\includegraphics[width=\mineImSize\linewidth]{im/exp/fab/mine/office/seq3_lt_sofa.png}&
%		\includegraphics[width=\mineImSize\linewidth]{im/exp/fab/mine/office/seq3_lt_desk.png}&
%		\includegraphics[width=\mineImSize\linewidth]{im/exp/fab/mine/office/seq3_lt_soft.png}\\		
%		\hline
%	\end{tabular}
%	\captionof{figure}{Demonstration on captured office data.}
%	\label{fig:mine_demo}
%\end{table*}
%\begin{figure*}[t!]
%	\centering
%	\setlength{\tabcolsep}{0.1em}
%	\renewcommand{\arraystretch}{.1}
%	\begin{tabular}{|c | c |c |}
%		\hline \hline
%		\includegraphics[width=\mineImSize\linewidth]{im/exp/fab/mine/office/seq3_w_slam_color.png}&	\includegraphics[width=\mineImSize\linewidth]{im/exp/fab/mine/office/seq3_w_slam_ir.png}&	\includegraphics[width=\mineImSize\linewidth]{im/exp/fab/mine/office/seq3_w_slam_saliency.png}\\
%		{Color} &{Infrared} & {Saliency}\\
%<<<<<<< HEAD


%=======
%		
%		\includegraphics[width=\mineImSize\linewidth]{im/exp/fab/mine/office/seq3_w_slam_style.png}
%		&\includegraphics[width=\mineImSize\linewidth]{im/exp/fab/mine/office/seq3_w_slam_lt_desk.png}
%		&\includegraphics[width=\mineImSize\linewidth]{im/exp/fab/mine/office/seq3_w_slam_lt_wood.png}\\
%		{Style} & {Object-desk} & {Material-wood} \\\hline
%>>>>>>> e014bc950c14dec9ffa1d2d7a6de9b7abfefabdd
%	\end{tabular}
%	%\captionof{figure}
%	\caption{Demonstration on captured Office data.}
%	\label{fig:office}
%\end{figure*}

\begin{figure*}[]
	\centering
	\setlength{\tabcolsep}{0.1em}
	\renewcommand{\arraystretch}{.1}
	\begin{tabular}{|c | c |c |}
	\hline \hline
	\includegraphics[width=\mineImSize\linewidth]{im/exp/fab/mine/appartment2/appartment2_color.png}&	\includegraphics[width=\mineImSize\linewidth]{im/exp/fab/mine/appartment2/appartment2_ir.png}&	\includegraphics[width=\mineImSize\linewidth]{im/exp/fab/mine/appartment2/appartment2_saliency.png}\\
		{Color} &{Infrared} & {Saliency}\\
	\includegraphics[width=\mineImSize\linewidth]{im/exp/fab/mine/appartment2/appartment2_style.png}
&\includegraphics[width=\mineImSize\linewidth]{im/exp/fab/mine/appartment2/appartment2_lt_sofa.png}
&\includegraphics[width=\mineImSize\linewidth]{im/exp/fab/mine/appartment2/appartment2_lt_desk.png}
\\
{Style} & {Object-sofa} & {Object-desk}\\
\includegraphics[width=\mineImSize\linewidth]{im/exp/fab/mine/appartment2/appartment2_lt_coat.png}
&\includegraphics[width=\mineImSize\linewidth]{im/exp/fab/mine/appartment2/appartment2_lt_sit.png}
&\includegraphics[width=\mineImSize\linewidth]{im/exp/fab/mine/appartment2/appartment2_lt_wood.png}\\
{Object-coat} & {Affordance-sit} & {Material-wood} \\\hline
	\end{tabular}
%\captionof{figure}
\caption{Demonstration on the captured apartment data.}
\label{fig:appartment}
%\vspace{-.5cm}
\end{figure*}


Using depth and property images of size $720\times1280$ as input, it is evident from the table, that our model operates at a frequency of $\sim10\si{\hertz}$ for  surface (sample mode) LIM construction and integration. 
It alse achieves a frequency of over $20\si{\hertz}$ for color, infrared, style, and saliency.
These results demonstrate the suitability of Uni-Fusion for real-time applications.

However, our internal tracking process takes around $0.225\si{\second}$ per frame, which is relatively slower compared to the mapping module. 
Nevertheless, Uni-Fusion uses external tracking to prevent tracking loss, enabling our internal tracking and mapping to operate at a lower frequency.
As a result, the entire model can be effectively applied in real-time in various scenarios.

\section{Extensive experiment on our own dataset}

In previous experiments, we evaluate the capabilities of Uni-Fusion in different applications. 
To further demonstrate its effectiveness in robotic environmental understanding, we capture our own dataset to show all applications together.

We capture two scenes: The office and apartment of the first author using a Microsoft Kinect Azure. 
%
RGB-D and infrared video are captured. After calibration, RGB, depth, infrared inputs have resolution of $720\times1280$.
Uni-Fusion tracks and reconstructs all applications in one pass.
%
While office data has been involved in ablation study (\cref{exp:surface:ablation}), we showcase all applications using the apartment dataset, as depicted in~\cref{fig:appartment}.

For better visualization, the ceiling of reconstruction is removed.
The top row of images presents the colored mesh with room details, the infrared mesh revealing the lighting effect, and the saliency reconstruction highlighting objects crucial for navigation.
Additionally, we select the second style from~\cref{fig:style} for style transfer to the apartment canvas.
%
As a result, the wooden floor in the room is colored with dark green.
The whole apartment is in a warm style.

The remaining results are generated from the surface field of the CLIP embeddings. 
We issue commands to locate objects, e.g., where is the sofa, desk and coat.
In addition, it easily identifies affordances such as being sittable.
For material, it successfully detects the wooden floor in each room.



%%%%%%%%%%%%%%%%%%%%%%%%%%%%%%%%%%%%%%%%%%%%%%%%%%%%%%%%%%%%%%%%%%%%%%%%%%%%%%%%
\section{Conclusion \label{sec:conclusion}}
\section{Conclusion}\label{sec:conclusion}
In this work, we focus on addressing the fundamental challenge of OOD detection tasks, which is how to fully understand the semantic discrepancy between the ID/OOD samples. We reveal that the key to success in the realistic SCOOD task is to allocate as many ID samples in the unlabeled set correctly as possible. To this end, we propose a novel uncertainty-aware optimal transport scheme that introduces class-specific energy scores as guidance for effective label assignment. Experimental results show that our method achieves better performance than previous state-of-the-art methods on SCOOD benchmarks.

\textbf{Limitations.} In addition to temperature scaling, other techniques such as feature clipping applied in ReAct~\cite{sun2021react} also enhance the performance of energy score, so how to obtain an OOD score that best fits the SCOOD task can be further explored. Moreover, a setting highly related to SCOOD has been proposed in \cite{katz2022training} and formulated as a constrained optimization problem. We will also theoretically analyze these practical OOD settings in our feature work.

% \section*{Acknowledgments}
\textbf{Acknowledgments.} 
This work is supported by National Key R\&D Program of China under Grant 2020AAA0105701, National Natural Science Foundation of China (NSFC) under Grants 61872327, Major Special Science and Technology Project of Anhui, National Natural Science Foundation of China (62033012) and Ant Group through Ant Research Intern Program.


\balance

\bibliographystyle{IEEEtran}
\bibliography{ref}

\end{document}
