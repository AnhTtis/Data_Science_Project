\newcommand{\trajRef}{\tPose_{\text{ref}}}
\newcommand{\safeFunc}{\mathcal{S}}

\subsection{NMPC Trajectory Tracking \label{sec:nmpc}}

Safe joined B\'{e}zier path $\bPath^*$ is generated in \S \ref{sec:bp}. In order to safely track the path for nonholonomic model, NMPC is applied. Assume the unicycle nonholonomic model  with state $\tPose=[\text{x}_1, \text{x}_2, \theta]^T$ and control $\tControl=[\nu, \omega]^T$. 
\begin{equation}
\begin{aligned}
    \dot{\text{x}}_1 &= \nu cos(\theta) \\
    \dot{\text{x}}_2 &= \nu sin(\theta) \\
    \dot{\theta} &= \omega
\end{aligned}
\end{equation}
In order to assign time stamps to path $\bPath^*$ based on nonholonomic dynamics, near-identity trajectory $\trajRef(t)$ \cite{1025398} is synthesized given the path and desired linear velocity $\nu_d$. Time stamps and the velocity profile $\tControl_{\text{ref}}$ are assigned to the dynamically feasible trajectory reference. However, it is possible to slightly deviate from the original B\'{e}zier path. NMPC with the safety \emph{Keyhole} ZBF constraint can guarantee safety during tracking. The scheme is formulated with initial state $\tPose(t)$ and control $\tControl(t)$ at current time $t$:
\begin{equation}
\begin{aligned}
\min_{\tControl(t+k)} \quad J(t) &= \sum_{k=0}^{N-1} ||\tPose(t+k) - \trajRef(t+k)||_Q \\ 
& \qquad + ||\tControl(t+k) - \tControl_{\text{ref}}(t+k)||_R \\
\textrm{s.t.} \quad & \tPose(t+k+1) = f(\tPose(t+k), \tControl(t+k)) \\
& \tControl_{lb} \leq \tControl(t+k) \leq \tControl_{ub} \\
& \boldsymbol{a}_{lb} \leq |\tControl(t+k+1)-\tControl(t+k)| \leq \boldsymbol{a}_{ub} \\
& h(\tPose(t+k)) \geq 0
\end{aligned}
\end{equation}
where $\norm{z}_Q=z^TQz$, and $\tControl_{lb}$, $\boldsymbol{a}_{lb}$, $\tControl_{ub}$, and $\boldsymbol{a}_{ub}$ are the lower and upper bounds of velocities and accelerations to maintain smooth motions. $N$ is the number of time step in the prediction horizon. $Q$ and $R$ are the state and control weights. $h(\tPose)$ is the {\keyhole}, which represents the inflated collision-free space $\infFreeSpace$.