% CVPR 2023 Paper Template
% based on the CVPR template provided by Ming-Ming Cheng (https://github.com/MCG-NKU/CVPR_Template)
% modified and extended by Stefan Roth (stefan.roth@NOSPAMtu-darmstadt.de)

\documentclass[10pt,twocolumn,letterpaper]{article}

%%%%%%%%% PAPER TYPE  - PLEASE UPDATE FOR FINAL VERSION
% \usepackage[review]{cvpr}      % To produce the REVIEW version
% \usepackage{cvpr}              % To produce the CAMERA-READY version
\usepackage[pagenumbers]{cvpr} % To force page numbers, e.g. for an arXiv version

% Include other packages here, before hyperref.
\usepackage{graphicx}
\usepackage{amsmath}
\usepackage{amssymb}
\usepackage{booktabs}


% It is strongly recommended to use hyperref, especially for the review version.
% hyperref with option pagebackref eases the reviewers' job.
% Please disable hyperref *only* if you encounter grave issues, e.g. with the
% file validation for the camera-ready version.
%
% If you comment hyperref and then uncomment it, you should delete
% ReviewTempalte.aux before re-running LaTeX.
% (Or just hit 'q' on the first LaTeX run, let it finish, and you
%  should be clear).
\usepackage[pagebackref,breaklinks,colorlinks]{hyperref}
% \usepackage[accsupp]{axessibility}

\usepackage[utf8]{inputenc} % allow utf-8 input
\usepackage[T1]{fontenc}    % use 8-bit T1 fonts
\usepackage{hyperref}       % hyperlinks
\usepackage{url}            % simple URL typesetting
\usepackage{booktabs}       % professional-quality tables
\usepackage{amsfonts}       % blackboard math symbols
\usepackage{nicefrac}       % compact symbols for 1/2, etc.
\usepackage{microtype}      % microtypography
\usepackage{xcolor,colortbl}          % colors
\usepackage{multirow}
\usepackage{graphicx}
% \usepackage{subfigure}
\usepackage{algorithm}
\usepackage{algorithmic}
\usepackage{wrapfig}
\usepackage{diagbox}
\usepackage{color}
\usepackage{amsmath}
\usepackage{amssymb}

\usepackage{subcaption}
\usepackage{amsthm}
\newtheorem{condition}{Condition}
\newtheorem{observation}{Observation}
\newtheorem{case}{Case}

\usepackage{xcolor}
\newcommand\blfootnote[1]{%
  \begingroup
  \renewcommand\thefootnote{}\footnote{#1}%
  \addtocounter{footnote}{-1}%
  \endgroup
}
\newcommand{\lv}[1]{{\color{black}{#1}}}
\newcommand{\lvv}[1]{{\color{black}{#1}}}

\newcommand{\tabincell}[2]{\begin{tabular}{@{}#1@{}}#2\end{tabular}}
\newcommand{\x}{\boldsymbol{x}}
\newcommand{\y}{\boldsymbol{y}}
\newcommand{\p}{\boldsymbol{p}}
\newcommand{\X}{\boldsymbol{X}}
\renewcommand{\algorithmicrequire}{\textbf{Input:}}
\renewcommand{\algorithmicensure}{\textbf{Output:}}


% Support for easy cross-referencing
\usepackage[capitalize]{cleveref}
\crefname{section}{Sec.}{Secs.}
\Crefname{section}{Section}{Sections}
\Crefname{table}{Table}{Tables}
\crefname{table}{Tab.}{Tabs.}


%%%%%%%%% PAPER ID  - PLEASE UPDATE
\def\cvprPaperID{12384} % *** Enter the CVPR Paper ID here
\def\confName{CVPR}
\def\confYear{2023}


\begin{document}

%%%%%%%%% TITLE - PLEASE UPDATE
\title{Improving Generalization with Domain Convex Game}
% \title{Supplementary Material for: Improving Generalization with Domain Convex Game}

% \author{First Author\\
% Institution1\\
% Institution1 address\\
% {\tt\small firstauthor@i1.org}
% % For a paper whose authors are all at the same institution,
% % omit the following lines up until the closing ``}''.
% % Additional authors and addresses can be added with ``\and'',
% % just like the second author.
% % To save space, use either the email address or home page, not both
% \and
% Second Author\\
% Institution2\\
% First line of institution2 address\\
% {\tt\small secondauthor@i2.org}
% }
\author{
\textbf{Fangrui Lv}\textsuperscript{\rm 1}
~
%\And
\textbf{Jian Liang}
~
%\And
\textbf{Shuang Li}\textsuperscript{\rm 1,$*$}
~
%\And
\textbf{Jinming Zhang}\textsuperscript{\rm 1}
% \\[0.1cm]
%\And
~
% \textbf{Chi Harold Liu}\textsuperscript{\rm 1}
% ~
%\And
\textbf{Di Liu}
% ~
% %\And
% \textbf{Fei Wang}\textsuperscript{\rm 3}
%\And
\\ [0.25cm]
\textsuperscript{\rm 1} Beijing Institute of Technology, China
% \ \
% \textsuperscript{\rm 2}Alibaba Group, China
% \ \
% \textsuperscript{\rm 3}\un{Cornell University, America}
\\[0.1cm]
{$^1$ \tt\small \{fangruilv,shuangli,jinming-zhang\}@bit.edu.cn}
% \\
% {\tt\small liangjianzb12@gmail.com}
% ~~
% {\tt\small liudi010@gmail.com}
% }
~~
{\tt\small \{liangjianzb12,liudi010\}@gmail.com}
}


\maketitle

%%%%%%%%% ABSTRACT
\begin{abstract}
Domain generalization (DG) tends to alleviate the poor generalization capability of deep neural networks by learning model with multiple source domains. 
A classical solution to DG is domain augmentation, the common belief of which is that diversifying source domains will be conducive to the out-of-distribution generalization. 
However, these claims are understood intuitively, rather than mathematically. 
Our explorations empirically reveal that the correlation between model generalization and the \lvv{diversity of domains} \lvv{may be} not strictly positive, which limits the effectiveness of domain augmentation. This work therefore aim to guarantee and further enhance the validity of this strand.
To this end, we propose a new perspective on DG that recasts it as a convex game between domains. We first encourage each diversified domain to enhance model generalization by elaborately designing a regularization term based on supermodularity. Meanwhile, a sample filter is constructed to eliminate low-quality samples, \lvv{thereby avoiding the impact of potentially harmful information}.
Our framework presents a new avenue for the formal analysis of DG, heuristic analysis and extensive experiments demonstrate the rationality and effectiveness.
\blfootnote{$*$ Corresponding author.}
\footnote{\quad Code is available at "https://github.com/BIT-DA/DCG".}

\vspace{-4mm}

\end{abstract}


%%%%%%%% BODY TEXT
%\vspace{-3mm}
\section{Introduction}
\label{sec:introduction}


\begin{comment}
    Points to be made in the intro
        - High value GEMM workloads have complex dependencies []
        - Traditional pipelining does not respond well to these dependencies
        -Keeping operands on-chip is key to get a good performance.
                - Caches - Implicit and Workload agnostic + hardware overheads
                - Spads - Explicitly managed (large map-space and expensive search and buffer allocation)
                - CHORD: A hybrid implicit and explicit management, which is cycle-level implicit and coarse-grained explicit, so the cost lowers enough to manage the SRAM online..
        - SCORE: Fast scheduler for low intensity ops that deals with more complex dependencies, and also provides metadata to CHORD..
        - Skewed GEMMs can have surprisingly low arithmetic intensity??
        
    


    
\end{comment}





As Deep Learning (DL) emerged as a high-value workload, the computer architecture community responded by proposing custom accelerators for DNNs~\cite{eyeriss2016isca,kwon2018maeri,tpu-isca,nvdla}. 
The most common fundamental operation for these DNNs was matrix multiplication, often expanded from convolutional layers that were common in DNNs at that time. These GEMMs offered good reuse opportunities because of their large dimensions and relatively cubic aspect ratios, allowing early DNN accelerators to successfully schedule each layer independently and achieve maximum utilization for the networks they targeted at design-time. Simultaneously, these schedules could be efficiently implemented using \emph{scratchpad} buffers that were explicitly controlled by the schedule to the stage intermediate \emph{tiles} of data according to the traversal order within each GEMM, without concern for inter-layer efficiency.

%Furthermore, prior works have also looked at the accelerators for sparse tensor workloads~\cite{sigma,eie,extensor,eyeriss2} which eliminate irrelevant multiplications and memory accesses.}% Unfortunately, this approach lowers intensity even further, and straightforwardly applying these to CG on a GEMM-by-GEMM operation basis results in compute under-utilization due to limited memory bandwidth.


Recently, work like the Tensor Algebra COmplier (TACO) \cite{XXXTaco} has spurred research interest in generalized tensor applications, including sparse tensor algebra. %The computer architecture community has responded with recent proposals that generalize custom accelerators to efficiently target a superset of DNNs \cite{extensor, tensaurus, dave2020hardware, tensorlib, flextensor} namely, applications represented as arbitrary DAGs of \emph{tensor operations} and non-linearities for DL activation functions.
However, this generalization brings new challenges: tensor-algebra applications have diverse shapes, sparsity ratios, and dependency graphs. %Worse, this is coupled with a trend in DNNs away from large, monolithic GEMMs and towards numerous small independent matrix multiplications\footnote{Often called \emph{batched} GEMMs, which should not be confused with DNN batch size.} with less cubic aspect ratios.
This is significant because of an underappreciated fundamental property: namely, \textit{not all dense GEMMs with large number of multiplications are compute-bound even in the best case}. As shown in \autoref{fig:ai1}, a \emph{skewed} aspect ratio inherently decreases \emph{arithmetic intensity} (AI), thus making the individual GEMM memory-bound and leaving datapath resources idle. For example, to saturate its datapaths the TPU v3 and v4 architectures require an AI of approximately 150 and 250 respectively~\cite{tpuv4}.  %Sparse tensor algebra introduces similar challenges, as removing multiplication-by-zero decreases the AI numerator, while transferring \emph{metadata} for compression formats increases the denominator relatively.}

\insertFigureScaled{ai1}{Degradation of arithmetic intensity on two GEMMs with the same number of multiplications due to aspect ratio skew.\vspace{-3mm}}{.75}



%\insertFigure{no-fusion}{Pipelining cannot simply be applied to complex DAGs due to - 1) Delayed downstream dependency, 2) varying shapes, 3) consumers at multiple reuse distances 4) need to preserve layout across conusmers.\vspace{-3mm}}


%The reason for this is low arithmetic intensity for a skewed GEMM and it can be as low below 1 op/byte. Currently, most scheduling strategies optimize matrix multiplications independently and compose them to execute in an op-by-op manner, which leads to low performance in applications with skewed GEMMs.}

\begin{comment}
A real-world quantification of this phenomenon is captured by HPCG benchmark~\cite{dongarra2015hpcg,hpcg2021}, which runs Conjugate Gradient (CG)---a widely used HPC solver application represented as a DAG of tensor operations.


\begin{scriptsize}
    
\begin{table}[t!]
\begin{scriptsize}

\begin{center}

    
    %\vspace{-2mm}
  \center
  \caption{Performance of CG compared to Linpack (HPL) on Top5 supercomputers. Adapted from HPCG~\cite{hpcg2021}.
} 
    \label{tables:hpcg}
  %\Rav{There is space for remarks too if needed}}
  %\TK{@Raveesh - by x do you mean you can do either s or t for that dimension? So does that mean each row is actually multiple datapoints? Thats confusing.}\Rav{It means that the datapoint that we choose for evaluation of that dataflow can have varaible tile sizes for dimension marked by X. S and T on the other hand mean that its NECESSARILY spatial or temporal}}
  \begin{tabular}{|l|l|l|l|l|}
    \hline
    \textbf{Supercomputer} & \textbf{HPL} &\textbf{HPCG} & \textbf{HPCG flops} &\textbf{HPCG:}  \\
     & \textbf{Pflops/s} &\textbf{Pflops/s} & \textbf{as \% of HPL } &\textbf{\% of peak}  \\
    \hline
1. Frontier & 1206 & 14.05 & 1.16\% & 0.8\% \\ \hline
2. Aurora & 1012 & 5.61 & 0.55 & 0.3\\\hline
3. Eagle & 561.2 & \multicolumn{3}{l|}{\revision{Not available}} \\\hline
4. Fugaku & 442.01 & 16 & 3.62\% & 3\% \\ \hline
5. Lumi & 379.7 & 4.587 & 1.2\% & 0.87\% \\ \hline
%6. Leornardo & 238.7 & 3.114 & 1.3\% & 1\% \\ \hline
%7. Summit & 148.6 & 2.93 & 2\% & 1.5\% \\ \hline
  \end{tabular}
%\vspace{-5mm}
\end{center}
\end{scriptsize}

\end{table}
\end{scriptsize}



%\footnote{HPCG is a benchmark for supercomputers that runs CG. \MP{Footnote 1 adds no real information. Cut. }}
As \autoref{tables:hpcg} shows, CG achieves only 1-3\% of peak performance on top 7 supercomputers. 
Therefore, intra-operation reuse alone is not sufficient.
The overall throughput can be increased by improving the arithmetic intensity (i.e., on-chip data reuse), which can be done by seeking inter-operation reuse.
\end{comment}

Thus to achieve full utilization we must consider inter-GEMM operation scheduling. Prior works have also shown that simple pipelining of adjacent operation~\cite{tileflow,isca-pip,yan2020hygcn} can improve things, but these solutions exclude significant potential sources of reuse, including delayed downstream consumers, and multiple consumers with varying reuse distances or traversal orders as~\autoref{fig:no-fusion} shows. To make matters worse, these extra options explode the scheduling space for finding good scratchpad configurations, as the total combinations and proportions of allocations explodes with operation DAG depth and the number of tensors involved. This means that exhaustive schedule-space exploration techniques become unacceptably slow, and also that heuristic solutions are more likely to miss the true optimal.

%It is also challenging to simply apply traditional inter-operation pipelining in cases of these dependencies, because of the delayed downstream dependencies, varying shapes of skewed GEMMs and multiple of these downstream consumers as~\autoref{fig:no-fusion} shows. %\TK{Fig 4 getting referenced before Fig 3, i suggest bring it before Fig 4 then} 
%~\autoref{fig:dfg} shows an actual complex cascade of tensor operations found in CG, a high-value HPC workload rich in these complexities.
%Thus, we need mechanisms to cache these intermediate tensors within SRAMs, since there are situations where simple pipelining cannot be applied.

%Prior works like buffets~\cite{buffets} make use of \emph{explicit decoupled data orchectration} to supplement scratchpad RAM with simple pointer and credit management scoreboarding. This is called explicit because data placement and RAM replacement are schedule-controlled, and decoupled because it operates on bulk-synchronous fills. %These work well for executing one layer at a time, when all dimensions of matrices have sufficient reuse.
%While explicit orchestration works well for scheduling single matrix multiplication at a time, considering data placement, for inter-operation reuse statically in general tensor algebra is a diabolically hard problem. 
% In order to reuse such operands, specially where intra-operation reuse is simply not enough, storing these operands in the on-chip buffer is essential, and multiple of these operands contend for space inside the buffer. We show in~\autoref{sec:arguments} that the design-space to accommodate multiple tensors inside a buffer space explodes in complexity. ~\autoref{sec:arguments} also discusses why running statically known DAGs does not necessarily imply that scratchpads are not burdensome and don't have design-time overheads.
%Overall, the co-dependence on schedule, available scratchpad capacity and different downstream consumers make the explicit scheduling problem too challenging.

\insertFigure{no-fusion}{Pipelining cannot simply be applied to complex DAGs due to - 1) Delayed downstream dependency, 2) varying shapes, 3) consumers at multiple reuse distances 4) need to preserve layout across conusmers. We discuss this in detail in~\autoref{sec:nofusion}\vspace{-3mm}}

Of course, scratchpads can be supplemented with pointer and credit management logic to become queues or buffets~\cite{buffets}, but these do not solve the schedule explosion problem as their staging decisions are still \emph{explicitly} controlled. Alternatively, caches are widespread on-chip storage structures that use \emph{implicit} data orchestration. Ideally, the presence of the cache is not architecturally exposed and the hardware itself does the best job possible of reusing the data. (In practice, best optimization is often made by cache-aware scheduling policies, blurring the line between implicit and explicit.) However, the area and energy overhead for tag matching make it less appealing for custom accelerators, as well as the possibility of increased misses due to conflicts. Overheads aside, cache policies typically operate at line granularity of unified address streams, rather than having higher-level knowledge of tensors, blocks, or intended reuse distance. This results in cache policies often rejecting the data that may have high reuse frequency (by virtue of reuse of the whole block) but might not be the immediate vicinity (i.e., high reuse distance).

In order to effectively exploit all available sources of reuse, we propose a unique approach: co-designing the buffer storage idiom with the scheduling algorithm. Our goal is that it becomes tractable to find schedules that obtain a sufficient level of inter-operation reuse to reach the compute bound, even in the face of complex DAGs of dependencies. To achieve this, we propose ~\SpadNamenospace\footnote{\SpadNameexp}, a novel \emph{explicit+implicit hybrid} buffering scheme that aims to combine the ``best of both worlds'' by placing coarse-grained decisions under the explicit control of the scheduler for efficiency, while making low-level fine-grained decisions implicitly like a cache, thus significantly reducing the size of the schedule space.
%~\SpadName is workload-aware, as it uses high-level metadata like starting and ending global address of a tensor, tensor-level reuse frequency and distance from the workload (the explicit component), but implicitly controls placement of elements of tensors without being burdensome to the programmer. ~%\TK{it might be worth adding a footnote saying you use scratchpad and buffer interchangeably in this paper} 
%\TK{based on what the footnote says its unclear whether chord is a scratchpad or cache }
Notably, this approach also significantly reduces the area and energy overheads of traditional line-level caches. %Specifically,~\SpadName uses per-tensor replacement policies that can be configured by the schedule at coarse granularity.
%for operands with downstream consumers that uses implicit replacement at a tensor granularity rather than a line granularity. 
%In this work we propose two such policies- \\(1) \PolicyA - \SpadName is filled in the queue order and once the buffer is full, the spilling data is sent straight to the DRAM.\\
%(2) \PolicyBnospace\footnote{\PolicyBexp} - If the current tensor has a higher reuse frequency and lower reuse distance (in case of same frequency) than the previously written tensor, the current tensor starts replacing the previous tensor by the tail.
%Because of its extremely coarsened granularity, ~\SpadName retains the benefit of a scratchpad over a cache (i.e., with minimal (<1\% of that of cache) tag matching overhead) 
%Hence it gets rid of caches' tag matching overhead and 
%while
%our proposed implicit tensor replacement policies~\PolicyA and \PolicyB remove the challenge of statically coming up with the data placement and replacement strategies statically for operands in multiple operations.

To schedule accelerators that use this structure, we propose ~\DataflowNamenospace\footnote{\DataflowNameexp} a downstream dependency aware novel scheduling strategy. 
\DataflowName identifies the delayed downstream dependencies that require writeback from those where pipelining would work, and steers the tensors with delayed writeback dependencies to~\SpadNamenospace. In order to exploit the reuse on delayed downstream consumers, it is important to make sure that the order in which the elements are produced as same as the order in which they are consumed, otherwise, layout transformation is required. Unlike prior mappers which search for tile-sizes for fine-grained buffer allocation, \DataflowNamenospace's involvement in buffer allocation is coarse-grained at an operand granularity rather than element-wise granularity, and low-level fine-grained decisions are made implicitly by~\SpadNamenospace's policies. 

All in all, we show that \SpadName and \DataflowName are applicable to diverse tensor applications and achieve 2.51x geomean speedup and 4x improvement in energy efficiency across a broad range of workloads (\autoref{sec:eval}).

%\TK{minimizing layout transformation is an important feature. Lets highlight it more explcitly. As in mention that the dataflow of the current op and downstream op may be different requiring expensive layout transformation, and SCORE tries to minimize that.}%It consists of the following steps - \\ 
%\TK{there seems to be some missing text here?}
%(1) Marking all the edges in the DAG of tensor operations with pipelining opportunities. This is useful in situations where tensors have delayed downstream consumers which can be visualized as a long edge in the DAG.\\
%(2) Assigning loop orders and tiling strategies to ensure that pipelining is actually exploited and layout transformation overhead of a tensor is minimized.\\
%\DataflowName reduces the contending tensors and also ensures minimum layout transformation (aka swizzling) which is also the motivation behind~\PolicyA policy.
%The main reason is the low arithmetic intensity achieved by the execution of Conjugate Gradient on CPUs/GPUs. 
%The main reason for this that the tensor multiplications---which we term \emph{operation} for this paper---%\MP{This definition is also too important to be in a footnote
 %used in CG have extremely unbalanced aspect ratios, which significantly lowers arithmetic intensity and data reuse.} %\MP{I feel like this point should come earlier, perhaps before the previous paragraph.}
%We use the term \emph{skewed} GEMMs to refer to such SpMMs/GEMMs---though notably, in the limit they can devolve to matrix-vector multiplication, i.e. if the shape of the matrix is 100,000:8.}%The main reason for this is costly data movement and communication between compute clusters/pods\MP{This explanation doesn't make sense either. Needs a reuse/intensity component. Even if this made sense, does GOGETA actually fix this problem?}.




%\TK{@Raveesh - I think we can move this para to para 3. Basically start with current para 2 on GEMMs, reuse and spatial accelerators. Then introduce CG. So intersperse this para with current para 3}


%making inter-operation reuse essential.

%This work explores inter-operation reuse opportunities for kernels like CG to enhance its arithmetic intensity.

%However, the~\GEMM in CG have orders of magnitude lower arithmetic intensity compared to SpMMs/GEMMs in DNNs and offer less reuse opportunities within one operation as~\autoref{fig:ai}~(\autoref{sec:ai}) discusses later in the paper. 
%Thus, executing CG at the granularity of a GEMM, reading tensors from DRAM and writing the output of each GEMM to DRAM, makes it highly memory bound leading to low compute utilization as~\autoref{\:hpcg} shows. 
\begin{comment}
Fusing \textit{adjacent} low AI matrix multiplication operators has recently been leveraged for applications like Graph Convolutional Networks (GCN)~\cite{garg2021understanding,yan2020hygcn,liang2020engn} and Transformers~\cite{flat,flashattention} 
wherein tiles of the intermediate tensor are manifested and consumed within the 
on-chip memory hierarchy in a \textit{pipelined} manner, reducing  intermediate output data movement to and from DRAM. We call this \emph{adjacent-op pipelining} in this work.
Other recent works have looked at enumerating the design-space for such adjacent-op pipelining~\cite{garg2021understanding} and 
%\TK{is this what the isca paper does?}\RG{Yep}
identifying optimal pipelined dataflow choices~\cite{isca-pip}, thereby enhancing better compute utilization when running memory-bound tensors.

While traditional adjacent-op pipelining is promising for DAGs with linear chains of operators (i.e., most DNNs today and GCNs), we show in this work that it is insufficient to capture nuances in more complex DAGs, such as those used in HPC kernels like CG.
Specifically, delayed downstream consumption of fanout of tensors, implies that traditional pipelining which overwrites the previously consumed tile cannot be applied directly due to a later dependency.%~\autoref{fig:no-fusion} shows an example, where tensor S has a delayed consumer.
 Moreover nuances such as data layouts of one tensor consumed in various operations is not captured by adjacent pipelining.
%that prior works cannot capture, as we discuss later. \autoref{table:related} enumerates this.

In this work, we coin the term \emph{generalized inter-operation reuse} to widen the scope of inter-operation reuse beyond adjacent operators to include additional reuse opportunities (\autoref{table:related}).
%as a wider generalization of traditional \emph{adjacent-operator pipelining}, introducing three additional reuse opportunities (\autoref{table:related}). 
We also propose \DataflowName (\TitleExpansion), which is a systematic strategy for mapping DAG of tensor operations exploiting the generalized inter-operation reuse opportunities, with the goal of enhancing the AI of the overall application.
%\DataflowName is applicable to any DAG of operations. 
We demonstrate that while \DataflowName is essential for extracting reuse in complex DAGs like CG, it can also be applied to simpler DAGs like GCNs and DNNs, thus preserving generalizability.

%\insertFigurePartnnnn{no-fusion}{A part of the CG tensor dependency graph where a node represents the equation in~\autoref{alg:cg_einsum} and edge represents the output of the source node equation. Please refer to ~\autoref{fig:dfg} for complete graph of CG.\vspace{-3mm}}%\RG{contraction heavy}}
We summarize our contributions below:
%\TK{I think the contribitions can be listed more succinctly. I would suggest each contribution bullet pointing to a specific section of paper. }

\squishlist
%\item We characterize the challenge of skewed GEMMs/SpMMs in tensor-algebra applications and the challenges of generalizing traditional inter-operation pipelining~\footnote{Not to be confused with inter-operation "reuse" since pipelining is a narrow aspect of inter-operation reuse}. (\autoref{sec:ai}).
%We also observe that these patterns are frequent across other HPC workloads which we discuss in ~\autoref{sec:background}.
%\item We propose a systematic methodology to formulate the data reuse opportunities in an arbitrary DAG of tensor operations and based on the reuse opportunities, we derive the loop orders and tile sizes.(\autoref{sec:dataflows}).
%\item We co-optimize the loop orders and tiling strategies for the GEMM operations in order to leverage reuse from both traditional inter-operation pipelining and distance based inter-operation reuse.

\item We propose~\DataflowName (\autoref{sec:score}), a scheduler which identifies the operands with delayed dependencies that require writeback (and hence \SpadNamenospace), and proposes a schedule that maximizes inter-operation reuse and minimizes the layout transformation of the operands across different consumers.
\item We propose~\SpadName(\autoref{sec:chorus}), a buffer structure for operands with downstream consumers that uses tensor-operand level replacement, that reduces tag match overhead and considers a more wholistic view of the object rather than a cache line. Cycle-level implicit tensor level replacement also eases the burden of solving tensor allocation involving multiple tensors statically which is a hard problem. 
%\item %Prior works on pipelining~\cite{flat,tangram,garg2021understanding,yan2020hygcn} divide the whole compute region into contiguous chunks and an operation is mapped on to a chunk as~\autoref{fig:spacetime} (top sub-figure) shows.\TK{seems odd to cite Fig 12 in intro}
%However, there is not much reuse within a single operation and the whole tensor needs to be communicated between the chunks. 
%\TK{this seems like a very specific optimization - and confusing here about what is lower BW vs higher BW NoC .. in fact i dont recall our arch section / Table IV talking about two NoCs with diff BWs? cant you state this contribution more generally about a communication BW optimized scalable inter-operation tiling strategy?}
%We propose a scalable inter-operation tiling strategy that reduces inter-cluster communication(~\autoref{sec:tiling}).  %a mapping that reduces memory accesses and communication at the tensor dependency graph level.
%\end{itemize}
%\item We propose buffer management schemes that reuse initial tiles of the output and can also replace the tensor based on future reuse distance and frequency~(\autoref{sec:tornado}). %the notion of Tensor-operand level reuse distance for such patterns and a tensor organization strategy inside the buffer hierarchy (\autoref{sec:tornado}).

\item We demonstrate the limitations of caches, namely, area overhead and line-level policies and the limitations of scratchpad, namely complexities in static buffer allocation of multiple delayed downstream consumers (\autoref{sec:arguments}).

\item We show that \SpadName and \DataflowName are applicable to diverse tensor applications and achieve 2.51x geomean speedup and 4x improvement in energy efficiency across a broad range of workloads (\autoref{sec:eval}).

%\item \reviewme{ } 
%\item We propose a novel data orchestration technique \DataflowName which allows for efficient replacement, placement and prefetching of data for applications with variable reuse distance patterns. Since reuse distance is a generalization, \DataflowName can be used for applications with only intra-operation reuse and inter-operation pipelining opportunities as well.
%\item We propose a novel buffer idiom \TODO{name} which improves performance and energy over caches and scratchpads by \TODO{xx} and \TODO{xx}
\squishend
\end{comment}


\begin{comment}
\emph{Inter-operation pipelining (aka fusion)} has been shown to be beneficial for accelerators for applications like Graph Neural Networks~\cite{garg2021understanding,yan2020hygcn,liang2020engn} where the intermediate tensor is reused between an SpMM and a GEMM,  reducing  intermediate output data movement to and from DRAM.
It has been shown that this approach is more challenging~\cite{dnnfusion,flat} and has a larger design-space~\cite{garg2021understanding} than straightforward element-wise fusion done by ML compilers today, for example, matrix-multiplication and ReLU fusion.
%Unfortunately, \textit{traditional inter-operation pipelining} often consumes the intermediate data and does not keep it in the memory. 
%Traditionally, \textit{inter-operation pipelining} has been exploited in prior works~\cite{tangram,garg2021understanding,yan2020hygcn,flat} %across various application domains 
%to reduce the data movement to DRAM, by consuming the portion of the tensor as it is produced 
\TK{I think we need to transition to saying that inter-operation pipelining does not capture all inter-operation reuse opportunities as this work identifies.}
Unfortunately, the additional complexity of inter-operation dependency graphs in certain applications introduces additional challenges which confounds the attempts at \emph{traditional inter-operation pipelining.} Furthermore, just capturing reuse in adjacent operations misses the overall opportunity to consider future instances of reuse of tensors in the entire program. \reviewme{We use CG as an example to discuss these challenges and missed opportunities:}

(1) Operations can have a delayed downstream consumer. Therefore, the data must remain resident in the memory hierarchy. However, traditional pipelining overwrites tiles that are consumed by the adjacent operation.
%~\autoref{fig:dfg} shows CG's dependency graph. Output of operation 1 is required in a delayed downstream consumer (op4) in addition to the adjacent consumer (op2). %The intermediate matrix cannot be overwritten or discarded, since its required in a future computation. 


%This complex dependency graph can often complicate the optimization of loop orders and tiling strategies to determine efficient intra-operation and inter-operation dataflows.


(2) CG has some tensor operations with contracted rank being much larger than the other ranks. This diminishes the benefit of pipelining because significant computation is required to produce any given final sum; therefore, making it a rate limiting step. 
%This prevents pipelining the entire graph.
%\TK{the following line seems out of place - since its one of many techniques we propose. Should maybe remove it}
%We propose \textit{pipelining with writeback}, which involves traditional pipelining and writing back the intermediate data to the buffer, hence incurring only write accesses and avoiding the read accesses for the intermediate matrix.

\insertFigure{no-fusion}{A part of the CG tensor dependency graph where a node represents the equation in~\autoref{alg:cg_einsum} and edge represents the output of the source node equation. The data cannot be consumed and be shielded from memory hierarchy since its reused again in another tensor. Please refer to ~\autoref{fig:dfg} for complete graph of CG.\vspace{-3mm}}%\RG{contraction heavy}}

(3) As the DAG becomes more complex, it is important to make sure that the loop order choice minimizes the data layout transformation (we use the term swizzling for it) of these tensors across various consumer operations.

(4) The downstream consumers also result in multiple \textit{tensor operand reuse distances}.

In this work, we identify various \textit{inter-operation} reuse opportunities to enhance the arithmetic intensity of such applications.
Our proposed \emph{generalized inter-operation reuse} is a wider generalization of \emph{traditional inter-operation pipelining}. We propose \DataflowName (\TitleExpansion), a strategy for mapping DAG of tensor operations which exploits reuse opportunities between operations.

\DataflowName is applicable to any DAG of operations. So while it helps extract reuse in the complex DAGs like CG, these patterns can also apply to simpler DAGs like GCNs, thus preserving generalizability.

%\reviewme{\DataflowName consists of the following contributions.

 %First, we identify the inter-operation reuse patterns in an arbitrary DAG of tensor operations. We propose a methodology to mark the edge of the DAG with the appropriate reuse pattern. This addresses the DAG complexity problems.

% Second, we propose loop orders that try to take the maximum advantage of the reuse patterns, given that the ability to extract reuse also depends upon the order of loops in these operations. The loop order also minimize swizzling

 %Third, prior works on pipelining~\cite{flat,tangram,garg2021understanding,yan2020hygcn}, divide the whole compute region into contiguous chunks and an operation is mapped on to a chunk as in~\autoref{fig:spacetime} (top sub-figure) shows. However, there is not much reuse within a single operation and the whole tensor needs to be communicated between the chunks. Instead, we pipeline within the cluster and split dominant ranks across cluster (\autoref{fig:spacetime} bottom).


% Fourth, given that we minimize swizzling, we propose a buffer management scheme that writes the data into the SRAM in the same order and skips the SRAM once its full. Then it reads the portion that's already in the SRAM first. We show that this can improve the op-by-op baseline considerably. We also propose buffer management strategies to prioritize operations with low reuse distances and high reuse frequencies.

% }

%In this work, we identify various \textit{inter-operation} reuse opportunities which are not limited to \textit{inter-operation pipelining}.
%Please note that \textit{inter-operation reuse} is a wider generalization of \textit{traditional inter-operation pipelining} here, since inter-operation pipelining only exploits reuse between consecutive operations. \autoref{fig:venn} shows the scope of our work on inter-operation dataflows compared to prior works on acceleration.
%One example is \textit{pipelining with writeback}

%Moreover, it is important to focus on leveraging the knowledge of future occurrence of the tensor beyond the immediate tensor. In CG, these tensor operands have variable \textit{tensor operand reuse distances}. This pattern is frequent in various scientific applications as~\autoref{sec:background} shows. This complex dependency graph can often complicate the determination of loop orders and tiling strategies to determine efficient intra-operation and inter-operation dataflows.

%Also, given that the graph of tensors becomes more complex, it is important to make sure that the loop order choice minimzes the transformation of layouts of these tensors across various operations where that tensor is used. We use the term \textit{swizzling} for the layout transformation. Thus we need to minimize swizzling.

%\RG{Can we cut this para and make contributions slightly longer ?}
%Based on the insights from the tensor dependency graphs of various applications, we propose a systematic methodology to identify, classify and exploit reuse opportunities in an arbitrary graph of tensor operations targeting spatial accelerators. This also involves deriving the loop orders and tile sizes for individual operations since the ability to exploit the inter-operation reuse can also depend on the individual operation's dataflow.
%Some other applications which have individual \GEMM of low arithmetic intensity include Graph Neural Network, Transformers etc. where our methodology is applicable, however, we often refer to Conjugate Gradient for demonstration purposes in this paper since its tensor dependency graph already has the characteristics of low intensity Graph and ML workloads but also has additional unique characteristics providing opportunity to demonstrate different kinds of inter-operation reuse.
%We also propose a scalable tiling strategy that involves splitting a memory bound GEMM by the dominating rank into sub-tensors and reusing the data in a fine-grained manner between the sub-tensors from different operations within a compute node as the bottom half of~\autoref{fig:spacetime} shows.
% (shown later in~\autoref{fig:spacetime})\TK{not sure if we need to cite a figure that'll come this late. Might be better to point to the section that will discuss this}. 
%
% We propose \DataflowName (\TitleExpansion), a strategy for mapping DAG of tensor operations which exploits reuse opportunities between operations. It comprises of identifying inter-operation reuse patterns from the DAG structure, loop-reordering and tiling based on those patterns and custom scratchpad management strategies. 
Our key contributions are as follows:
%\vspace{-5mm}

%\begin{itemize}
\squishlist
\item We characterize the challenge of skewed GEMMs/SpMMs in tensor-algebra applications and systematically formulate \emph{generalized inter-operation reuse opportunities} (beyond pipelining) in a DAG of operations (\autoref{sec:dataflows}).
%We also observe that these patterns are frequent across other HPC workloads which we discuss in ~\autoref{sec:background}.
%\item We propose a systematic methodology to formulate the data reuse opportunities in an arbitrary DAG of tensor operations and based on the reuse opportunities, we derive the loop orders and tile sizes.(\autoref{sec:dataflows}).
%\item We co-optimize the loop orders and tiling strategies for the GEMM operations in order to leverage reuse from both traditional inter-operation pipelining and distance based inter-operation reuse.
\item Based on the reuse opportunities, we derive loop orders for the operations that allow consumers to maximally reuse the data, and require minimum changes to memory layout~(\autoref{sec:loop}).
\item Prior works on pipelining~\cite{flat,tangram,garg2021understanding,yan2020hygcn} divide the whole compute region into contiguous chunks and an operation is mapped on to a chunk as~\autoref{fig:spacetime} (top sub-figure) shows. However, there is not much reuse within a single operation and the whole tensor needs to be communicated between the chunks. We propose a scalable inter-operation tiling strategy which splits the dominant rank across compute clusters and pipelines the operations within a cluster as~\autoref{fig:spacetime} (bottom) shows~(\autoref{sec:tiling}).  %a mapping that reduces memory accesses and communication at the tensor dependency graph level.
%\end{itemize}
\item We propose a buffer management scheme that writes the data into the SRAM in the same order and skips the SRAM once its full. Then it reads the portion that's already in the SRAM taking advantage of swizzle minimzation. We also propose buffer management strategies that prioritize operations with low reuse distances and high reuse frequencies~(\autoref{sec:tornado}). %the notion of Tensor-operand level reuse distance for such patterns and a tensor organization strategy inside the buffer hierarchy (\autoref{sec:tornado}).
\item \DataflowName achieves \reviewme {geomean 5.97x (ranging from 1.34x to 23x)} improvement in the arithmetic intensity over operation by operation execution~(\autoref{sec:eval}).

\item \reviewme{\DataflowName is generally applicable to diverse tensor applications as it applies to all DAG structures. We also evaluate it on GCNs and ResNets and obtain 2.71x and xx geomean improvement in arithemetic intensity over op-by-op baseline~(\autoref{sec:eval}).} 
%\item We propose a novel data orchestration technique \DataflowName which allows for efficient replacement, placement and prefetching of data for applications with variable reuse distance patterns. Since reuse distance is a generalization, \DataflowName can be used for applications with only intra-operation reuse and inter-operation pipelining opportunities as well.
%\item We propose a novel buffer idiom \TODO{name} which improves performance and energy over caches and scratchpads by \TODO{xx} and \TODO{xx}
\squishend
\end{comment}

\begin{comment}

\begin{enumerate}
    \item Identifying inter-operation reuse patterns in an arbitrary DAG of operations
    \item Deriving loop orders to make sure that they are amenable to inter-operation reuse and minimize swizzling
    \item Proposing tiling strategy that does pipelining within the compute node and splits the large rank across nodes (bottom half of~\autoref{fig:spacetime}).
    \item Proposing scratchpad management strategies like reuse distance and frequency based tensor replacement and proposing strategies to extract reuse from the portion of the tensor that does fit in SRAM.
    
\end{enumerate}
\vspace{-1mm}
Our methodology is also applicable to %(and evaluated on) 
other applications with low intensity GEMMs like GNNs and transformers. %where our methodology is applicable,
However, we often refer to Conjugate Gradient as the application for demonstration purposes since its tensor dependency graph has the characteristics of low intensity Graph and ML workloads and also has additional unique characteristics providing opportunity to demonstrate different kinds of inter-operation reuse.
\autoref{fig:venn} shows the scope of our work on inter-operation dataflows compared to prior works on accelerator dataflow.
%Our proposed \textit{inter-operation reuse} is a wider generalization of \textit{traditional inter-operation pipelining} here, since inter-operation pipelining only exploits reuse between consecutive operations.
%\autoref{fig:venn} shows the scope of our work on inter-operation dataflows compared to prior works on acceleration.

%Therefore data orchestration which accounts for such reuse is critical for HPC workloads. Cache replacement policies often have a global view of an individual line rather than a tensor as a whole, which limits the applicability for such algorithms. Moreover, LRU replacement does not see the future reuse of the tensor and replaces it if it has not been used recently. Belady's optimal replacement policy requires knowledge of future accesses for each line and often requires additional structures and meta data accesses to replace one line which is costly to implement~\cite{popt-hpca21}. Scratchpads, on the other hand provide ability to the programmer to have control over data orchestration. However, these scratchpads are often over-provisioned for the worst case applications.

%To this end, we propose \DataflowName data orchestration that analyzes the tensor operand-level reuse patterns in the algorithm and the dataflows of the individual GEMMs to manage the data in the memory hierarchy. \DataflowName also enables efficient prefetching for these HPC algorithms due to the knowledge of the future tensor reuse pattern and dataflows.

 %Some other applications with low arithmetic intensity include Graph Neural Network, Transformers etc., however, we often refer to Conjugate Gradient for demonstration purposes in this paper since its DAG already has the characteristics of Graph and ML workloads but also has additional unique characteristics providing opportunity to demonstrate different kinds of inter-operation reuse.


\noindent  \textbf{\textit{The key contributions of this paper are:}}
\end{comment}

%Our methodology is also applicable to %(and evaluated on) 
%other applications with low intensity GEMMs like GNNs and transformers. %where our methodology is applicable,
%However, we often refer to Conjugate Gradient as the application for demonstration purposes since its tensor dependency graph has the characteristics of low intensity Graph and ML workloads and also has additional unique characteristics providing opportunity to demonstrate different kinds of inter-operation reuse.
%\autoref{fig:venn} shows the scope of our work on inter-operation dataflows compared to prior works on accelerator dataflow.
%Our proposed \textit{inter-operation reuse} is a wider generalization of \textit{traditional inter-operation pipelining} here, since inter-operation pipelining only exploits reuse between consecutive operations.
%\autoref{fig:venn} shows the scope of our work on inter-operation dataflows compared to prior works on acceleration.


































\begin{comment}
\section{Problem and Motivation}
\label{sec:introduction}

Sparse and Dense matrix multiplications are prime operations for a variety of applications spanning Graph Analytics~\cite{kipf2017semisupervised,hamilton2017inductive}, High-Performance Computing~\cite{cools2017communication,cerebras} and Artificial Intelligence~\cite{resnet,nlp}. The GEMM operations used in DNNs offer vast reuse opportunities~\cite{kwon2019understanding,interstellar,timeloop,eyeriss2016isca} owing to their high arithmetic intensity. This has led to a plethora of spatial accelerators for DNN applications~\cite{eyeriss2016isca,kwon2018maeri,tpu-isca,nvdla,shi}. Prior works have also looked at the acceleration for Sparse workloads~\cite{sigma,eie,extensor,eyeriss2} by eliminating redundant matrix multiplications and memory accesses.

However, certain PDE solver algorithms used scientific applications like the Conjugate Gradient~\cite{hestenes1952methods} have SpMM/SpMV and GEMM/GEMV operations with low arithmetic intensity~\cite{cerebras}. The best supercomputers ranked on High-Performance Linpack benchmark are able to achieve only upto 3\% of the performance on the HPCG benchmark~\cite{cerebras,osti_1089988}.
~\autoref{alg:cg_einsum} shows the Block Conjugate Gradient Algorithm. ~\autoref{fig:results}a) and b) show maximum achievable arithmetic intensity of individual operations. Please note that instead of counting multiplications and additions separately, we count MAC as a unit of computation. We also consider matrix addition of, for example, $X^{k-1}$ with $P^{k-1}.\alpha^{k-1}$ in the equation $X^{k}=X^{k-1}+P^{k-1}\alpha^{k-1}$ to be done immediately after the MAC and we absorb the addition in same operation. Therefore, effectively we count number of multiplications per memory access of an element (independent of the bit precision). Please note, here number of RHS is a parameter with values 1,8,16,32 and 64. 

\insertFigure{cg}{Block CG Algorithm.}


~\autoref{fig:results}a) shows the maximum achievable arithmetic intensity for individual SpMM operations for the suitesparse CG matrices. We note that the SpMMs, especially the highly sparse ones like barth4 have extremely low arithmetic intensity even for RHS=64.
~\autoref{fig:results}b) shows the maximum achievable arithmetic intensity of dense GEMMs in an individual operation in terms of number of multiplications per memory access. We notice that the arithmetic intensity of the DenseGEMMs is severely limited by the number of RHS (ie. the number of problems solved simultaneously.) Therefore traditional DNN accelerators~\cite{eyeriss2016isca,kwon2018maeri,tpu-isca} do not help accelerate these workloads due to fundamentally low reuse.

~\autoref{fig:results}c) shows the arithmetic intensity for one iteration of the CG loop. Please note that for matrix inverse, we consider LU factorization followed by Triangular Solve. Also, we conservatively count accesses to $P$ and $P^T$ separately owing to different layout in the memory. We note that the amount of reuse that can be extracted between the operations is high and this motivates us to design a new dataflow for HPCG workloads which tries to maximize inter-operation rather than optimizing for inter-operation reuse.

\end{comment}

\section{Related Work}

\textbf{Domain Generalization} researches out-of-distribution generalization with knowledge only extracted from multiple source domains.
A promising direction is to diversify training domains so as to improve generalization, referring as to domain augmentation~\cite{CrossGrad,AdvAug,MixStyle,FACT,L2A-OT}. L2A-OT~\cite{L2A-OT} creates pseudo-novel domains from source data by maximizing an optimal transport-based divergence measure. CrossGrad~\cite{CrossGrad} generates samples from fictitious domains via gradient-based domain perturbation while AdvAug~\cite{AdvAug} achieves so via adversarially perturbing images. 
MixStyle~\cite{MixStyle} and FACT~\cite{FACT} mix style information of different instances to synthetic novel domains.
Instead of enriching domain diversity, another popular solution that learning domain-invariant representations by distribution alignment via kernel-based optimization \cite{DICA,SCA}, adversarial learning \cite{MMD-AAE,CCSA}, \lv{or using uncertainty modeling~\cite{DSU}}
 demonstrate effectiveness for model generalization.
Other recent DG works also explore low-rank decomposition \cite{CSD}, self-supervised signals~\cite{Jigen}, gradient-guided dropout \cite{RSC}, etc. 
Though our proposed framework builds on the domain augmentation group, we aim to guarantee and further enhance their efficacy beyond via a convex game perspective.

\textbf{Convex Game} is a highly interesting class of cooperative games introduced by~\cite{ConvexGame}. A game is called convex when it satisfies the condition that the profit obtained by the cooperation of two coalitions plus the profit obtained by their intersection will not be less than the sum of profit obtained by the two respectively (a.k.a. supermodularity)~\cite{ConvexGame,supermodularity,convexfuzzygame}.
Co-Mixup\cite{Co-Mixup} 
formulates the optimal construction of mixup augmentation data while encouraging diversity among them by introducing supermodularity. Nevertheless, it is applied to supervised learning which aims to construct salience mixed samples.
Recently, ~\cite{onlineDG} rethinks the single-round minmax setting of DG and recasts it as a repeated online game between a player minimizing risk and an adversary presenting test distributions in light of online convex optimization~\cite{onlineconvexopti}. We note that the definition of convex game exploited in our work follows~\cite{ConvexGame}, distinct from that in~\cite{onlineDG, onlineconvexopti}.
To the best of our knowledge, this work is the first to introduce convex game into DG to enhance generalization capability. 


\textbf{Meta Learning}~\cite{learning} is a long-term research exploring to learn how to train a particular model through the training of a meta-model~\cite{learntooptimize,MAML,fewshot}, and has drawn increasing attention from DG community~\cite{MetaReg,MASF,FCN,MLDG} recently. The main idea is to simulate domain shift during training by drawing virtual-train/test domains from the original source domains. 
MLDG~\cite{MLDG} originates the episode training paradigm from \cite{MAML}, back-propagating the second-order gradients from an ordinary task loss on random meta-test domains split from the source domains. 
Subsequent meta learning-based DG methods utilize a similar strategy to meta-learn a regularizer~\cite{MetaReg}, feature-critic network~\cite{FCN}, or semantic relationships~\cite{MASF}.
Different from the former paradigm that purely leverages the gradient of task objective, which may cause sub-optimal,
we utilize the ordinary task losses to construct a supermodularity regularization with more stable optimization, aiming to encourage each training domain to contribute to model generalization.

\section{Domain Convex Game}
\label{sec:method}
Motivated by such an observation in Section~\ref{sec:intro}, we propose Domain Convex Game (DCG) framework to train models that can best utilize domain diversity, as illustrated in Fig.~\ref{fig:framework}.
First, we cast DG as a convex game between domains and design a novel regularization term employing the supermodularity, which encourages each domain to benefit model generalization. Further, we construct a sample filter based on the regularization to exclude bad samples that may cause negative effect on generalization. 
In this section, we define the problem setup and present the general form of DCG.


\subsection{Preliminary}
Assuming that there are $P$ source domains of data $\mathcal{D}_s = \cup_{k=1}^PD_k$ with $n_k$ labelled samples $\{(\boldsymbol{x}^k_i, y^k_i)\}_{i=1}^{n_k}$ in the $k$-th domain $D_k$, where $\boldsymbol{x}^k_i$ and $y^k_i \in \{1,2,\cdots,C\}$ denote the samples and corresponding labels. DG aims to train a domain-agnostic model $f(\cdot,\boldsymbol{\theta})$ parametrized with $\boldsymbol{\theta}$ on source domains that can generalize well on unseen target domain(s) $\mathcal{D}_t$.
As an effective solution for DG, domain augmentation aims to enrich the diversity of source domains generally by synthesizing novel domains via mixing domain-related information, hence boosting model generalization~\cite{L2A-OT,MixStyle,FACT}.
\lvv{Our work builds on this strand, and the key insight is to ensure and further improve its efficacy by better leveraging the domain diversity. For concision, in this paper, we adopt a simple Fourier-based augmentation technique~\cite{FACT,FDA} to prepare our diversified source domains. Note that the augmentation strategy is substitutable.}

% In this paper, we exploit the Fourier-based augmentation technique [48,49] to prepare our diversified source domains for its simplicity . Also, other alternatives can be used.

\lvv{Technically}, owing to the property that the phase component of Fourier spectrum preserves high-level semantics of the original signal, while the amplitude component contains low-level statistics~\cite{1981fourier,1982fourier}, we augment the source data by distorting the amplitude information while keeping the phase information unchanged. Specifically, we mix the amplitude spectrum of an instance with that of another arbitrary instance by a linear interpolation strategy to synthesize augmented instances from novel domains. We refer readers to~\cite{FDA,FACT} for implementation details.
Since each augmented sample is generated by mixing domain information of sample pairs from random source domains in a random proportion, \lv{it has statistics distinct from the others so that can be regarded as drawn from a novel augmented domain}. Thus, we possess another $Q$ augmented source domains 
of data $\mathcal{D}_s^{aug} = \cup_{k=1}^QD_{P+k}$
with only one sample $\{(\boldsymbol{x}^{P+k}_i, y^{P+k}_i)\}_{i=1}^{1}$ in the $(P+k)$-th domain $D_{P+k}$, where $\boldsymbol{x}^{P+k}_i$ and $y^{P+k}_i$ denote the augmented samples and corresponding labels.
\lv{Note that the number of augmented domains generated this way is equivalent to the total number of all the original samples since each original sample pair will generate a pair of augmented samples.}
The goal of DCG is to train a generalizable model $f(\cdot,\boldsymbol{\theta})$ for unseen target domain(s) $\mathcal{D}_t$ with the aid of 
all $P+Q$ diversified source domains $\mathcal{D}_s\cup \mathcal{D}_s^{aug}$.

\begin{figure}[t]
  \centering
  \includegraphics[width=1.0\linewidth]{Figures/Fig_framework.pdf}
  \caption{The pipeline of DCG. We first randomly split the diversified training domains into meta-train and meta-test domains, and generate four coalitions from the former according to the definition of convex game. Then we conduct meta learning on the four coalitions respectively and construct our regularization loss utilizing the meta-test losses of them based on the supermodularity. Meanwhile, we eliminate the low-quality samples by a sample filter and calculate supervision loss on the retained samples. }
  \label{fig:framework}
  \vspace{-4mm}
\end{figure}

\subsection{Supermodularity Regularization Term}
\label{sec:sm_reg}
Let $M = \{1,2,\cdots,m\}$ be a finite set of players and $2^M$ is the family of $2^{|M|}$ subsets of $M$. A cooperative game with player set $M$ is a map $v:2^M \xrightarrow{} \mathbb{R}$. For coalition $S\in2^M$, $v(S)$ is called the worth of $S$, and is interpreted as the total profit that $S$ can obtain when the players in $S$ cooperate. 
A game is called convex if it satisfies the
\textit{supermodularity property}~\cite{convexfuzzygame,ConvexGame,supermodularity}, i.e., for each $S,T\in2^M$:
\begin{equation}
  v(S \cup T) + v(S \cap T) \ge v(S) + v(T).
  \label{eq:convex_sm}
\end{equation}
According to this definition, we can obtain: 
\begin{equation}
  v(S \cup \{i\} \cup \{j\}) - v(S \cup \{i\}) \ge v(S \cup \{j\}) - v(S),
\end{equation}
where $S\in2^M\verb|\|\{\varnothing\}$ and $i,j$ are two players not in $S$. 
We can see that convex game requires each player to contribute to the coalition, which is consistent with our key insight, that is, each training domain is expected to benefit model generalization. More than this, convex game also possesses \textit{increasing marginal contribution property} for players, which may not hold in DG. However, this property does not hinder our goal, but can further alleviate the \textit{decreasing marginal contribution} for domains, as discussed in Section~\ref{sec:intro}.

Thus, we first cast DG as a convex game between domains.
To achieve this, at each training iteration, we randomly split the original source data $\mathcal{D}_s$ into $P-V$ meta-train domains of data $\Tilde{\mathcal{D}_s}$ and $V$ meta-test domains of data $\tilde{\mathcal{D}}_t$, where $\Tilde{\mathcal{D}_s}$ and $\tilde{\mathcal{D}}_t$ share no domain. Then we pick out the augmented domains generated by data in $\Tilde{\mathcal{D}_s}$, denoted as $\tilde{\mathcal{D}}_s^{aug}$, and incorporate them into the meta-train domains. 
This strategy to conduct meta-train/test domains is to mimic the real train-test domain shift in domain augmentation strand, which is discussed in Section~\ref{sec:discussion_main}. 
Then, since one domain may contain multiple samples, we specifically consider involving a specific convex game: \emph{convex fuzzy game}~\cite{convexfuzzygame} where each player (i.e., each domain) can be partitioned into multiple parts (each part represents a sample in DG).
Now we have a finite set of partitioned players 
$\tilde{M} = \tilde{\mathcal{D}_s} \cup\tilde{\mathcal{D}}_s^{aug}$.
We can obtain coalitions $S,T \in 2^{\tilde{M}}$ by randomly sampling two sets of data from meta-train data $\tilde{\mathcal{D}_s} \cup\tilde{\mathcal{D}}_s^{aug}$, respectively. And $S\cup T, S\cap T$ can be naturally constructed by the union and intersection of $S$ and $T$.
As for the profit $v(O), O\in\{S,T,S\cup T, S\cap T\}$, we take the generalization performance evaluated on virtual-test domains $\tilde{\mathcal{D}_t}$ after the meta-training on each coalition $O$ as the value of profit $v(O)$.

Specifically, assuming a loss function $\ell(f(\boldsymbol{x},\boldsymbol{\theta}),y)$ for a sample between its output and label, e.g., cross-entropy loss for classification task, we first conduct virtual training on the four coalitions $\{S,T,S\cup T, S\cap T\}$, respectively, with the optimization objective:
\begin{equation}
  \small
  \mathcal{F}(O) := \sum_{x\in O}\ell(f(\boldsymbol{x},\boldsymbol{\theta}),y)), O \in \{S,T,S\cup T, S\cap T\}.
\end{equation}
Then the updated virtual parameters $\boldsymbol{\theta}'$ can be computed using one step of gradient descent:
\begin{equation}
  \boldsymbol{\theta}' = \boldsymbol{\theta} - \alpha \nabla_{\boldsymbol{\theta}}\mathcal{F}(O),
\end{equation}
where $\alpha$ is the virtual step size and is empirically set to be the same as the learning rate in our experiments. Thus, we can have the corresponding meta-test loss evaluated on the virtual-test domains $\tilde{\mathcal{D}_t}$ as below:
\begin{equation}
  \mathcal{G}(\boldsymbol{\theta}') := \mathbb{E}_{\boldsymbol{x} \in \tilde{\mathcal{D}_t}} \ell(f(\boldsymbol{x},\boldsymbol{\theta}'),y).
\end{equation}
This objective simulates test on unseen domains, thus can measure the model generalization obtained by training with one coalition, i.e., $v(O)=-\mathcal{G}(\boldsymbol{\theta}')$.
Hence, the supermodularity regularization can be constructed naturally utilizing the meta-test losses of the four coalitions based on Eq.~\eqref{eq:convex_sm}:
\begin{equation}
\small
\begin{split}
  \mathcal{L}_{sm} = \max\{0,&\mathcal{G}(\boldsymbol{\theta} - \alpha\nabla_{\boldsymbol{\theta}}\mathcal{F}(S\cup T)) + \mathcal{G}(\boldsymbol{\theta} - \alpha\nabla_{\boldsymbol{\theta}}\mathcal{F}(S\cap T))\\ - &\mathcal{G}(\boldsymbol{\theta} - \alpha\nabla_{\boldsymbol{\theta}}\mathcal{F}(S)) - \mathcal{G}(\boldsymbol{\theta} - \alpha\nabla_{\boldsymbol{\theta}}\mathcal{F}(T))\}.
\end{split}
\label{eq:l_reg}
\end{equation}
\lv{Here we exploit a $max(0,\cdot)$ function combined with the pure supermodularity to construct our regularization. In this way, $\mathcal{L}_{sm}>0$ only when the inequality in Eq.~\eqref{eq:convex_sm} is violated, i.e., the domain marginal contribution is decreasing. Thus, the limit of our regularization optimization corresponds to constant marginal contribution, not the inappropriate increasing marginal contribution.} Therefore, this regularization term can not only encourage each training domain to contribute to model generalization, but also alleviate the decrease of marginal contributions to some extent, enabling the model to fully leverage the rich information in diversified domains.


\subsection{Sample Filter}
Through the optimization of the regularization term, the model will be trained to better utilize the rich information of diversified source domains. However, what we cannot avoid is that there may exist some low-quality samples with harmful information to model generalization.
For instance, noisy samples will disturb model to learn generalizable knowledge; while redundant samples 
may lead to overfitting that hinder the model from learning more diverse patterns.

In this view, we further conduct a sample filter  to avoid the negative impact of low-quality samples.
Considering that the proposed regularization aims to penalize the decreasing marginal contribution of domains and then better utilize the diverse information, the samples that contribute more to the regularization loss (i.e., cause larger increase) are more unfavorable to our goal, hindering the improvement of model generalization.
Thus, we try to measure the contribution of each input to our regularization loss and define the contribution as its score.
Inspired by \cite{LRP} which defines the contribution of each input to the prediction by introducing layer-wise relevance propagation, we formulate the score of each input as the elementwise product between the input and its gradient to regularization loss, i.e., Input $\times$ Gradient:

\begin{equation}
  score = \boldsymbol{x}^T\nabla_{\boldsymbol{x}}\mathcal{L}_{sm},x\in \tilde{\mathcal{D}_s}\cup\tilde{\mathcal{D}}_s^{aug}.
  \label{eq:score}
\end{equation}
The higher the score of the sample, the greater the regularization loss will be increased caused by it, and the more it will hinder model from benefiting from diversified domains.
Therefore, we pick out the samples with the top-$k$ score, denoted as $\mathcal{D}_{del}$, and cast them away when calculating the supervision loss for diversified source domains \lv{to eliminate the negative effect of low quality samples}:
\begin{equation}
  \mathcal{L}_{sup} = \mathbb{E}_{\boldsymbol{x}\in\mathcal{D}_s\cup\mathcal{D}_s^{aug}\verb|\| \mathcal{D}_{del}} \ell(f(\boldsymbol{x},\boldsymbol{\theta}),y).
  \label{eq:l_cls}
\end{equation}
Thus, we optimize the regularization loss to enable model to better utilize the rich information within diversified domains. In the meanwhile, we eliminate the low-quality samples \lv{(e.g, noisy samples, redundant samples, etc)} by the sample filter to avoid their negative effects. \lv{Moreover, it is found that different types of low-quality samples are more likely to be discarded in different training stages, as discussed in Section~\ref{sec:theory_proof}. And we have explored out that low quality sample filtering is necessary for both original and augmented samples in Section~\ref{sec:discussion_main}.}

The overall optimization objective is:
\begin{equation}
  \arg \min_{\boldsymbol{\theta}} \mathcal{L}_{sup} + \omega \mathcal{L}_{sm},
  \label{eq:overall}
\end{equation}
where $\omega$ weights the supervision loss and the regularization term.
The overall methodological flow is illustrated schematically in Fig.~\ref{fig:framework} and summarized in Appendix~\ref{sec:alg}.
% \vspace{-1mm}
%   \begin{algorithm}[H]
%     \vspace{-1mm}
%     \caption{The Algorithm of Domain Convex Game.}
%     \label{alg:DCG}
%     \begin{algorithmic} [1]
%     \REQUIRE $P+Q$ diversified source domains $\mathcal{D}_s \cup \mathcal{D}_s^{aug}$; Hyper-parameters: $\omega, k$.
%      \STATE randomly initialize model parameters $\boldsymbol{\theta}$.
%     \FOR{iter in iterations}
%     \STATE Randomly sample a mini-batch of $\mathcal{D}_s$ as $B$ and a mini-batch of $\mathcal{D}_s^{aug}$ as $B^{aug}$. 
%     \STATE Split: $\tilde{\mathcal{D}_s}$ and $\tilde{\mathcal{D}_t}$ $\xleftarrow{} B$, Pick out: $\tilde{\mathcal{D}}_s^{aug}$ from $B^{aug}$.
%     \STATE Construct coalitions $S,T$ by randomly sampling from $\tilde{\mathcal{D}_s} \cup \tilde{\mathcal{D}}_s^{aug}$; construct coalitions $S\cup T, S\cap T$.
%     \STATE Calculate supermodularity regularization loss $\mathcal{L}_{sm}$ as Eq.~\eqref{eq:l_reg}.
%     \STATE Pick out low-quality samples $\mathcal{D}_{del}$ with the top-$k$ score calculated by Eq.~\eqref{eq:score}.
%     \STATE Calculate supervision loss $\mathcal{L}_{sup}$ as Eq.~\eqref{eq:l_cls}
%     \STATE Update $\boldsymbol{\theta} = \arg \min_{\boldsymbol{\theta}} \mathcal{L}_{sup} + \omega\mathcal{L}_{sm}$.
%     \ENDFOR
%     \end{algorithmic}
%   \end{algorithm}
%   \vspace{-4mm}
\section{Heuristic Analysis}
\label{sec:theory_proof}
In section~\ref{sec:method} we cast DG as a domain convex game and present the detailed formulation of our framework which revolves around the proposed regularization term.
Though this term is designed directly according to the supermodularity and has clear objective to achieve our goals, someone may still be curious about the mechanisms behind its effectiveness.
So in this section, we provide some heuristic analyses and intuitive explanations to further validate the rationality.

For brevity, we take $S=\{(\boldsymbol{x}_i, y_i)\}$ and $T=\{(\boldsymbol{x}_j,y_j)\}$ as an example, where $\boldsymbol{x}_i$ and $\boldsymbol{x}_j$ are from different domains.
According to Eq.~\eqref{eq:l_reg}, the optimization goal of our proposed regularization is to make the following inequality hold:
\begin{equation}\small
\begin{split}
    &\mathcal{G}(\boldsymbol{\theta} - \nabla_{\boldsymbol{\theta}}\ell(f(\boldsymbol{x}_i,\boldsymbol{\theta}),y_i) -\nabla_{\boldsymbol{\theta}}\ell(f(\boldsymbol{x}_j,\boldsymbol{\theta}),y_j))  + \mathcal{G}(\boldsymbol{\theta})\\
    & - \mathcal{G}(\boldsymbol{\theta} - \nabla_{\boldsymbol{\theta}} \ell(f(\boldsymbol{x}_i,\boldsymbol{\theta}),y_i))
     - \mathcal{G}(\boldsymbol{\theta} - \nabla_{\boldsymbol{\theta}} \ell(f(\boldsymbol{x}_j,\boldsymbol{\theta}),y_j)) \le 0.
\end{split}
\label{eq:example_obj}
\end{equation}
We then carry out the second-order Taylor expansion on the terms in Eq.~\eqref{eq:example_obj} and obtain:
\begin{equation}
\small
\begin{split}
    &(\nabla_i + \nabla_j)^T H (\nabla_i + \nabla_j) -  \nabla_i^T H \nabla_i - \nabla_j^T H \nabla_j \\
   & =\nabla_i^T H \nabla_j + \nabla_j^T H \nabla_i    \le 0   ,
\end{split}
\label{eq:taylor_expansion}
\end{equation}
$\nabla_i, \nabla_j$ denote $\nabla_{\boldsymbol{\theta}} \ell(f(\boldsymbol{x}_i,\boldsymbol{\theta}),y_i)$, $\nabla_{\boldsymbol{\theta}} \ell(f(\boldsymbol{x}_j,\boldsymbol{\theta}),y_j)$ respectively, $H = \frac{ \partial^{2} \mathcal{G}(\boldsymbol{\theta})}{\partial \boldsymbol{\theta} \partial \boldsymbol{\theta}^T}$ is the Hessian matrix of $\mathcal{G}(\boldsymbol{\theta})$.
We can see that all the zero- and first-order terms of the Taylor-expansion have been dissolved and only the second-order terms are left, which makes the optimization more stable.


Since Hessian matrix $H$ is a real symmetric matrix, for the case where $H$ is positive (negative) definite, we can perform Cholesky decomposition on $H (-H)$ as $L^T L$, where $L$ is an upper triangular matrix with real and positive diagonal elements. Thus, Eq.~\eqref{eq:taylor_expansion} can be further deduced as follows:
\begin{equation}
\small
\begin{split}
&\nabla_i^T H \nabla_j + \nabla_j^T H \nabla_i\\
& =
\begin{cases}
    (L \nabla_i)^T (L \nabla_j) + (L \nabla_j)^T (L \nabla_i) \le 0,& \text{$H \succ 0$,}\\
    -((L \nabla_i)^T (L \nabla_j) + (L \nabla_j)^T (L \nabla_i)) \le 0,& \text{$H \prec 0$.}
    \end{cases}
\end{split}
\label{eq:h_decomposition}
\end{equation}
Denote $L \nabla_i, L \nabla_j$ as $\tilde{\nabla_i}, \tilde{\nabla_j}$ respectively, which can be regarded as a mapping transformation of the original gradients. Specifically, $\nabla_i, \nabla_j$ are sample gradients generated in the original "training space" during the meta-training process, while $\tilde{\nabla_i}, \tilde{\nabla_j}$ are sample gradients transformed by matrix $L$. Since $L$ is derivated from the regularization term calculated on meta-test data that can indicate the model generalization, we can intuitively regard the transformed $\tilde{\nabla_i}, \tilde{\nabla_j}$ as sample gradients mapped to a "generalization space". Therefore, constraining sample gradients in this mapped "generalization space" may generalize better on the real test set compared to constraining the gradients in the original "training space".

Then two main cases can be analysed respectively.
\begin{case}
For Hessian matrix $H \prec 0$ (a.k.a. negative definite), Eq.~\eqref{eq:taylor_expansion} holds when \lv{${\tilde{\nabla_i}}^T\tilde{\nabla_j} \ge 0$}.
\label{cond:non-positive}
\end{case}

\textbf{\textit{mechanism.}}
When $H \prec 0$, i.e., achieving local maxima, which suggests inferior model generalization, the proposed regularization would help the model improve by enforcing domain consistency on discriminability, that is, pulling the samples from different classes apart and bringing the ones from the same class closer \lv{in the "generalization space"}. 
As for sample filtering, samples that possess inconsistent gradients, e.g., noisy samples, are more prone to be discarded.


\textbf{\textit{analysis.}}
\lv{As Eq.~\eqref{eq:h_decomposition} shows, our regularization aims to make the inner product of transformed sample gradients positive when $H \prec 0$, i.e., make the sample gradients consistent in the "generalization space".}
Assuming samples $\boldsymbol{x}_i$, $\boldsymbol{x}_j$ belong to the same class, then their transformed gradients will be inconsistent when they are apart in the "generalization space", and be consistent when they are close.  
In contrast, if $\boldsymbol{x}_i$, $\boldsymbol{x}_j$ are from different classes, their transformed gradients would certainly be inconsistent if the samples are close, since they share the same model while possessing different labels. Thus, the optimization of our regularization will draw the samples from the same class closer while pulling the ones from different classes apart to make the gradients consistent, which enforces domain consistency on discriminability. 
\lv{As for sample filtering, the samples that possess very inconsistent gradients are contrary to our goal most and are more likely to obtain larger scores, which are generally noise samples since they are often located at outliers. Therefore, the noise samples are more prone to be discarded in this case.}

\begin{case}
For Hessian matrix $H \succ 0$ (a.k.a. positive definite), Eq.~\eqref{eq:taylor_expansion} holds when \lv{${\tilde{\nabla_i}}^T\tilde{\nabla_j} \le 0$}.
\label{cond:positive}
\end{case}

\textbf{\textit{mechanism.}}
When $H \succ 0$, i.e., achieving local optima, the proposed regularization would help the model jump out by further squeezing out the information within hard samples, that is, detecting the hard samples and then assigning them larger weights implicitly. As for sample filtering, samples that possess very consistent gradients, e.g., redundant samples, are more prone to be discarded.

\textbf{\textit{analysis.}}
As Eq.~\eqref{eq:h_decomposition} shows, our regularization aims to make the inner product of the transformed sample gradients negative  when $H \succ 0$, i.e., make the gradients inconsistent in the "generalization space". This objective is contrary to our main supervision loss that aims to make all the samples clustered, so it can be regarded as an adversarial optimization. Concretely, the regularization enables the model to generate and detect samples with inconsistent gradients which generally be hard samples since they are often far away from the class center. Then these hard samples would contribute more to the main supervision loss and thus can be considered as being assigned larger weights implicitly during the optimization, just like the mechanism of focal loss~\cite{focal_loss}. Thus, our regularization can help model jump out of the local optima by squeezing out more information within hard samples, avoiding the model depending on easy patterns or even overfitting on redundant ones. 
For sample filtering, the samples that produce very consistent gradients, which also means they are redundant ones to a certain, are more likely to be detrimental to our regularization loss and be filtered. 


For the general case that $H$ is not fully positive or negative definite, we can take SVD decomposition and regard the model as combined by positive or negative definite sub-matrices. Then our conclusion holds for each subspace represented by each submatrix.

    
\section{Experiments}
\label{sec:exp}

\subsection{Dataset and Implementation Details}
\label{sec:dataset}
To evaluate our method, we conduct extensive experiments on three popular benchmarks for DG:
\noindent\textbf{PACS}\cite{pacs} is an object recognition benchmark that covers 9991 images of 7 categories from four different domains, i.e., Art, Cartoon, Photo and Sketch, which with large discrepancy in image styles.
\noindent\textbf{Office-Home}\cite{home} is a commonly-used benchmark including four domains (Art, Clipart, Product, RealWorld). It contains
15,500 images of 65 classes in total.
\noindent\textbf{mini-DomainNet}\cite{dael} is a very large-scale domain generalization benchmark consists of about 140k images with 126 classes from four different domains (Clipart, Painting, Real, Sketch).
For all benchmarks, we conduct the commonly used leave-one-domain-out experiments~\cite{DBA} and adopt ResNet-18/50 pre-trained on ImageNet~\cite{resnet} as backbone.
We train the network using mini-batch SGD with batch size 16, momentum 0.9 and weight decay 5e-4.
The initial learning rate is 0.001 and decayed by 0.1 at 80\% of the total epochs. 
For hyper-parameters, we set $\omega = 0.1$ and $k=5$ for all experiments, which are selected on validation set following standard protocol.
All results are reported based on the average accuracy over three independent runs. More details and results with error bars are provided in Appendix. 



\subsection{Experimental Results}


\begin{table}
%   \setlength{\tabcolsep}{3.0pt}
  \centering
  \small
%   \vspace{-5mm}
    % \vspace{-2mm}
    %   \setlength{\tabcolsep}{0.4mm}{
      \resizebox{\columnwidth}{!}{
      \begin{tabular}{l|cccc|c}
      \toprule
      Methods & Art & Cartoon & Photo & Sketch & Avg. \\
      \midrule
      \multicolumn{6}{c}{\textit{ResNet18}} \\
      \midrule
      DeepAll\cite{FACT} & 77.63 & 76.77 & 95.85 & 69.50 & 79.94 \\
    %   MetaReg~\cite{MetaReg} & 83.70& 77.20 & 95.50 & 70.30 & 81.70 \\
    %   JiGen~\cite{Jigen} & 79.42 & 75.25 & 96.03 & 71.35 & 80.51 \\
      MLDG~\cite{MLDG} & 78.70 & 73.30 & 94.00 & 65.10 & 80.70 \\
      MASF~\cite{MASF} & 80.29 & 77.17 & 94.99 & 71.69 & 81.04 \\
      L2A-OT~\cite{L2A-OT} & 83.30 & 78.20 & \underline{96.20} & 73.60 & 82.80 \\
      DDAIG~\cite{DEEPALL} & 84.20 & 78.10 & 95.30 & 74.70 & 83.10 \\
      RSC~\cite{RSC} & 83.43 & \underline{80.31} & 95.99 & 80.85 & 85.15 \\
      MixStyle ~\cite{MixStyle} & 84.10 &  78.80 & 96.10 &  75.90 & 83.70 \\
      FACT \cite{FACT}& \underline{85.37} &78.38& 95.15& 79.15 &84.51 \\
      % \lv{ITL-Net\cite{ITL-Net} } & 83.90 & 78.90 & 94.80 & 80.10 & 84.40 \\
      \lv{DSU~\cite{DSU} } & 83.60 &79.60& 95.80& 77.60& 84.10\\
      \lvv{STNP}~\cite{STNP} & 84.41 & 79.25 & 94.93 & \textbf{83.27} & \underline{85.47}\\
      \midrule
      DCG (\textit{ours})  & \textbf{85.94} &	\textbf{80.76}	& \textbf{96.41} &	\underline{82.08}	& \textbf{86.30} \\
      \midrule
      \multicolumn{6}{c}{\lvv{\textit{ResNet50}}} \\
      \midrule
      DeepAll~\cite{FACT} & 84.94 & 76.98 & 97.64 & 76.75 & 84.08 \\
      % MetaReg~\cite{MetaReg} & 87.20 & 79.20 & 97.60 & 70.30 & 83.60 \\
      % MASF~\cite{MASF}  & 82.89 & 80.49 & 95.01 & 72.29 & 82.67 \\
      % EISNet~\cite{EISNet} & 86.64 & 81.53 & 97.11 & 78.07 & 85.84 \\
      % MatchDG \cite{MatchDG} & 85.61 & 82.12 & \textbf{97.94} & 78.76 & 86.11 \\
      % \lv{Contrastive-ACE~\cite{contrastive-ACE}} & \lv{88.8} &\lv{ 81.9} & \lv{97.7} & \lv{80.6} & \lv{87.3}\\
      RSC~\cite{RSC} & 87.89 & 82.16 & \underline{97.92} & 83.35 & 87.83 \\
      % FACT \cite{FACT} & 89.63 & 81.77 & 96.75 & 84.46 & 88.15 \\
      FACT \cite{FACT} & 89.63 & 81.77 & 96.75 & 84.46 & 88.15 \\
      DDG~\cite{DDG}  &88.90& \underline{85.00}& 97.20& 84.30& 88.90\\
      PCL~\cite{PCL} &90.20& 83.90& \textbf{98.10}& 82.60& 88.70\\
      STNP~\cite{STNP} &\textbf{90.35}& 84.20& 96.73& \underline{85.18}& \underline{89.11}\\
      \midrule
      DCG (\textit{ours}) & \underline{90.24} & \textbf{85.12} & 97.76 & \textbf{86.31} & \textbf{89.84} \\
      \bottomrule
      \end{tabular}}
      \vspace{-2mm}
      \caption{Leave-one-domain-out results on PACS.}
    \label{tab:pacs_res18}
  \vspace{-3mm}
\end{table}

\noindent\textbf{Results on PACS} \lvv{based on ReNet-18 and ResNet-50 are summarized in Table~\ref{tab:pacs_res18}. 
It is clear that DCG achieves the best performance among all the competitors \lvv{on both backbones}.
% Our method is the first to reach $86\%$ average accuracy on PACS dataset, which exceeds the DeepAll baseline by $6.
% 4\%$.
We notice that DCG surpasses the Fourier based augmentation method FACT by a large margin of $1.8\%$ and $1.7\%$ on ResNet-18 and ResNet-50, respectively, which indicate the importance of encouraging each domain to contribute to model generalization.
Especially, on the harder target domains Cartoon and Sketch, our method still outperforms the SOTA. 
There also exist cases where DCG performs relatively poorly, this may due to the task is relatively simple (e.g. \textit{photo}).
In general, the comparisons reveal the effectiveness of DCG and further demonstrate that the convex game between domains improves model generalization.}

\begin{table}
  %   \setlength{\tabcolsep}{3.0pt}
    \centering
    \small
        %\setlength{\tabcolsep}{0.4mm}{
        \resizebox{\columnwidth}{!}{
        \begin{tabular}{l|cccc|c}
        \toprule
        Methods & Art & Clipart & Product & Real & Avg. \\
        \midrule
        DeepAll  & 57.88 & 52.72 & 73.50 & 74.80 & 64.72 \\
      %   CCSA~\cite{CCSA}  & 59.90 & 49.90 & 74.10 & 75.70 & 64.90 \\
        MLDG~\cite{MLDG} &52.88 & 45.72 & 69.90 & 72.68 & 60.30 \\
        SagNet~\cite{SagNet}  & 60.20  &45.38& 70.42& 73.38& 62.34\\
      %   MMD-AAE~\cite{MMD-AAE} & 56.50 & 47.30 & 72.10 & 74.80 & 62.70 \\
      %   CrossGrad~\cite{CrossGrad} & 58.40 & 49.40 & 73.90 & 75.80 & 64.40 \\
      %   Jigen~\cite{Jigen} & 53.04 & 47.51 & 71.47 & 72.79 & 61.20 \\
        RSC~\cite{RSC}   & 58.42 & 47.90 & 71.63 & 74.54 & 63.12 \\
        DDAIG~\cite{DEEPALL} & 59.20 & 52.30 & 74.60 & 76.00 & 65.50 \\
        L2A-OT~\cite{L2A-OT} & \underline{60.60} & 50.10 & \underline{74.80} & \textbf{77.00} & 65.60 \\
        MixStyle~\cite{MixStyle} & 58.70 & 53.40 & 74.20 & 75.90 & 65.50\\
        FACT \cite{FACT}& 60.34 & 54.85 & 74.48 & 76.55 & \underline{66.56} \\
        \lv{DSU~\cite{DSU}} & 60.20 & 54.80 & 74.10 & 75.10 & 66.10 \\
        \lvv{STNP~\cite{STNP}} & 59.55 & \underline{55.01} & 73.57 & 75.52 & 65.89\\
        \midrule
        DCG (\textit{ours}) & \textbf{60.67} &	\textbf{55.46} &	\textbf{75.26}	& \underline{76.82} &	\textbf{67.05}  \\
        \bottomrule
        \end{tabular}}
        \vspace{-2mm}
        \caption{Leave-one-domain-out results on Office-Home.
        %   with ResNet-18. The best and second-best results are bold and underlined.
        }
      \label{tab:officehome}
      \vspace{-5mm}
  \end{table}
  
  
  
  \noindent\textbf{Results on Office-Home} \lvv{based on ReNet-18} are presented in Table~\ref{tab:officehome}, where we beat all the compared baselines in terms of the average accuracy. 
  Due to the similarity to the pre-trained dataset ImageNet,
  DeepAll acts as a strong baseline on Office-Home.
  Many previous DG methods, such as MLDG, SagNet, and RSC, can not improve over the simple DeepAll baseline.
  Nevertheless, our DCG achieves a consistent improvement over DeepAll on all the held-out domains. 
  Moreover, DCG surpasses the latest domain augmentation methods L2A-OT and FACT. The incremental advantages may be due to the relatively smaller domain shift, where the decreasing marginal contribution of domains is more severe. 
  % However, this still justifies the superiority of DCG.
  
  
  


\begin{table}
%   \setlength{\tabcolsep}{0.5mm}
  \centering
  \small
%   \vspace{-0.5mm}
    %\setlength{\tabcolsep}{0.35mm}{
    \resizebox{\columnwidth}{!}{
    \begin{tabular}{l|cccc|c}
    \toprule
    Methods & Clipart & Painting & Real & Sketch & Avg. \\
    \midrule
    DeepAll & 65.30 &	58.40	&64.70&	59.00&	61.86  \\
    ERM~\cite{ERM}  & 65.50 & 57.10 & 62.30 & 57.10 & 60.50 \\
     MLDG~\cite{MLDG} & 65.70 & 57.00 & 63.70 & 58.10 & 61.12 \\
     Mixup~\cite{mixup} & \underline{67.10} & 59.10 & 64.30 & 59.20 & 62.42 \\
    MMD~\cite{MMD} & 65.00 & 58.00 & 63.80 & 58.40& 61.30 \\
    SagNet~\cite{SagNet} & 65.00 & 58.10 & 64.20 & 58.10 & 61.35 \\
     CORAL~\cite{CORAL} & 66.50 & \underline{59.50} & \underline{66.00} & \underline{59.50} & \underline{62.87} \\
     MTL~\cite{MTL} & 65.30 & 59.00 & 65.60 & 58.50 & 62.10 \\
    \midrule
     DCG (\textit{ours}) & \textbf{69.38} &	\textbf{61.79}&	\textbf{66.34}&	\textbf{63.21}&	\textbf{65.18}  \\
    \bottomrule
    \end{tabular}}
    \vspace{-2mm}
    \caption{Leave-one-domain-out results on mini-DomainNet. 
    %   with ResNet-18. The best and second-best results are bold and underlined.
    }
  \label{tab:domainnet}
  \vspace{-5mm}
\end{table}





\begin{figure*}[t]
  \centering
  \vspace{-1mm}
   \includegraphics[width=1.0\linewidth]{Figures/Fig_visualization.pdf}
  \vspace{-7mm}
   \caption{The visualization of samples with top-$k$ and bottom-$k$ score respectively with Cartoon as the unseen target domain. }
   \label{fig:visualization}
   \vspace{-3mm}
\end{figure*}


\noindent\textbf{Results on Mini-DomainNet} \lvv{based on ReNet-18} are shown in Table~\ref{tab:domainnet}. The much larger number of categories and images makes DomainNet a much more challenging benchmark. DCG still achieves the state-of-the-art performance of $65.18\%$, surpassing the SOTA by a large margin of $2.31\%$. \lvv{It indicates that the waste of diversified information in large datasets is more serious, further validating our efficacy.}









\subsection{Analysis}


\noindent\textbf{Ablation Study.}
In Table~\ref{tab:ablation}, we investigate the role of each component in DCG, including Fourier augmentation (Aug.), supermodularity regularization (Reg. ($\mathcal{L}_{sm}$)) and sample filter (Filter. ($\mathcal{F}_{sm}$)).
The Baseline is trained only with the supervision loss of all the original source data. We incorporating our supermodularity regularization $\mathcal{L}_{sm}$ with the Fourier augmentation to obtain Model 3, which greatly surpasses Model 1, demonstrating the significance of encouraging each diversified domain to contribute to generalization. 
Besides, we aslo apply a regularization $\mathcal{L}_{maml}$ which sums the meta-test losses of all the tasks as MAML~\cite{MAML} to conduct Model 2, its inferiority to Model 3 indicates conducting 
\begin{table}
  \setlength{\tabcolsep}{0.4mm}
  \centering
  \small
  \centering
    % \setlength{\tabcolsep}{0.1mm}{
    \resizebox{\columnwidth}{!}{
    \begin{tabular}{c|ccc|cccc|c}
    \toprule
    Method & Aug. & Reg. & Filter. & Art & Cartoon & Photo & Sketch & Avg.\\
    \midrule
    Baseline & - & - & - & 77.6 & 76.8 & 95.9 & 69.5 & \lv{79.9}\\
    \midrule
    Model 1 & $\checkmark$ & - & - & 83.9& 77.0 & 95.6 & 77.4 & \lv{83.4}\\
    Model 2 & $\checkmark$ & $\mathcal{L}_{maml}$ & - & 84.7 &	79.0 &	95.7 &	80.1 &	\lv{84.9}	\\
    Model 3 & $\checkmark$ & $\mathcal{L}_{sm}$ & - &85.1 &	80.1 &	95.9 &	81.4 & \lv{85.6} \\
    Model 4 & $\checkmark$ & - & $\mathcal{F}_{maml}$ & 84.1 &	77.7 &	95.5 &	78.2 &	\lv{83.9} \\
    Model 5 & $\checkmark$ & - & $\mathcal{F}_{sm}$ & 84.4 &	78.2 &	95.8 &	79.3 &	\lv{84.4} \\
    Model 6 & $\checkmark$ & $\mathcal{L}_{maml}$ & $\mathcal{F}_{maml}$ & 85.3 & 79.9 & 96.0 &	81.5 &	\lv{85.7} \\
    \midrule
    DCG & $\checkmark$ & $\mathcal{L}_{sm}$ & $\mathcal{F}_{sm}$ & \textbf{85.9} &	\textbf{80.8}&	\textbf{96.4} &	\textbf{82.1} &	\textbf{86.3}\\
    \bottomrule
    \end{tabular}}
    \vspace{-2mm}
    \caption{Ablation study of DCG on PACS dataset.}
  \label{tab:ablation}
 \vspace{-5mm}
\end{table}
convex game between domains is more helpful to generalization than simply applying the meta loss. 
Comparing Model 5 with Model 1, we can observe that the proposed sample filter is also conducive to generalization, suggesting the importance of eliminating nonprofitable information. Finally, DCG performs best in all variants, indicating that the two proposed components complement and benefit each other.


\begin{figure*}
  \centering
  \begin{subfigure}{0.23\linewidth}
  \includegraphics[width=1.0\linewidth]{Figures/Fig_increase_cartoon.pdf}
    \caption{Cartoon.}
    \label{fig:inc_cartoon}
  \end{subfigure}
  \hfill
  \begin{subfigure}{0.23\linewidth}
  \includegraphics[width=1.0\linewidth]{Figures/Fig_increase_sketch.pdf}
    \caption{Sketch.}
    \label{fig:inc_sketch}
  \end{subfigure}
  \hfill
  \begin{subfigure}{0.23\linewidth}
    \includegraphics[width=1.0\linewidth]{Figures/Fig_sensitivity_cartoon.pdf}
      \caption{Cartoon.}
      \label{fig:sens_cartoon}
    \end{subfigure}
    \hfill
    \begin{subfigure}{0.23\linewidth}
    \includegraphics[width=1.0\linewidth]{Figures/Fig_sensitivity_sketch.pdf}
      \caption{Sketch.}
      \label{fig:sens_sketch}
    \end{subfigure}
    \vspace{-2mm}
  \caption{(a)(b): relation between model generalization 
and domain diversity; (c)(d): sensitivity to hyper-parameters $\omega$ and $k$; with Cartoon and Sketch on PACS
dataset as the unseen target domain.}
  \vspace{-5mm}
\end{figure*}


\noindent\textbf{Generalization with Domain Diversity.}
Figure~\ref{fig:inc_cartoon},~\ref{fig:inc_sketch} show the model generalization with the increase of domain diversity. We use the classification accuracy on the held-out target domain as the metric of model generalization across domains, and the number of augmented domains to measure the domain diversity.
It is clear that on both Cartoon and Sketch tasks, the model generalization capability of the baseline methods do not necessarily improve with the increase of domain diversity, but sometimes decrease instead. While in our DCG, the model generalization increases monotonically with the domain diversity on the whole and the decrease of marginal contribution of domains is alleviated. 
Meanwhile, in a few cases, the generalization of DCG drops a little when domain diversity increases. This is reasonable since the additional augmented domains may be low-quality or harmful to generalization. 
The results demonstrate that our framework indeed encourages each diversified domain to contribute to model generalization, hence guarantee and further improve the performance of domain augmentation methods.




\noindent\textbf{Visualization of Filtered Samples.}
To visually verify that our sample filter can effectively eliminate low-quality samples, we provide the samples that obtain the top-$k$ / bottom-$k$ score the most times in the whole training process in Figure~\ref{fig:visualization}.
We can see that the discarded original samples with top-$k$ score in the first row either be noisy images that have messy background and fuzzy objects, or be images containing naive or classical information which may be redundant. While the high-quality original images in the bottom row are all vivid and rich in information. As for the augmented samples, the discarded ones are almost distinguishable while the retained high-quality ones are limpid.
These comparisons demonstrate the effectiveness of our sample filter.



\noindent\textbf{Sensitivity of Hyper-parameters.}
Figure~\ref{fig:sens_cartoon},~\ref{fig:sens_sketch} show the sensitivity of DCG to hyper-parameters $\omega$ and $k$. Specifically, the value of $\omega$ varies from $\{0.01, 0.05, 0.1, 0.5, 1,0\}$, while
$k$ changes from $\{1, 3, 5, 7, 9\}$.  
It can be observed
that DCG achieves competitive performances robustly under a wide range of hyper-parameter values, i.e., $0.05 \le \omega \le 0.3$ and $3 \le k \le 7$, in either task Cartoon or Sketch, which further verifies the stability of our method.





\subsection{Discussion}
\label{sec:discussion_main}

\begin{table}
%  \setlength{\tabcolsep}{0.8mm}
  \centering
  \small
 \resizebox{\columnwidth}{!}{
 \begin{tabular}{c|cccc|c}
    \toprule
     Methods & Art & Cartoon & Photo & Sketch & Avg.\\
    \midrule
     Random\_meta\_split &85.6	&80.2&	96.0&	81.8&	85.9\\
    \midrule
     Filter\_only\_on\_aug & 85.4 & 80.6 & 96.7 &	81.8 &	86.1 \\
     Filter\_only\_on\_ori & 85.2 &80.0 & 96.5 &	82.3 &	86.0\\
    \midrule
    DCG & \textbf{85.9} &	\textbf{80.8}&	\textbf{96.4} &	\textbf{82.1} &	\textbf{86.3}\\
    \bottomrule
    \end{tabular}}
    \vspace{-2mm}
    \caption{Leave-one-domain-out results on PACS.}
    \label{tab:discussion}
    \vspace{-5mm}
\end{table}
\noindent\textbf{How to conduct the meta-train and meta-test domains?}
In DCG, we considers all the diversified domains $\mathcal{D}_s \cup \mathcal{D}_s^{aug}$ into training. We first randomly split the original source domains $\mathcal{D}_s$ into meta-train and meta-test domains, next pick out the domains in $\mathcal{D}_s^{aug}$ that are augmented by the current meta-train domains and then merge them into together. Thus, there is no domain augmented by the meta-test domains in the meta-train domains, and vice versa. However, why don't we also randomly split $\mathcal{D}_s^{aug}$ into two parts, since each diversified domain can be regarded as a novel domain?
We conduct experiments of this variant and the results in Table~\ref{tab:discussion} shows inferior performance to DCG. 
This may be because the synthetic novel domains still contain part of the domain-related information of the original ones. In this view, the strategy to conduct meta-train/test domains in Section~\ref{sec:sm_reg} can guarantee the meta-test domains are completely unseen, which better simulates the domain shift between diversified source domains and the held-out unseen target domain.

\noindent\textbf{Is low-quality sample filtering necessary for both original and augmented samples?}
We conduct experiments that apply the proposed sample filter only on the original samples or augmented samples and the results are shown in Table~\ref{tab:discussion}. 
It can be seen that both variants suffer from a performance drop, which indicates that there exist low-quality samples among both original and augmented samples.
Limiting the filtering range will make some low-quality samples be retained to participate in the training process, which may damage the model generalization.
Besides, the performance of only filtering the original samples is slightly lower than that of only filtering the augmented ones, which should be due to the augmented samples being less natural.



\vspace{2mm}
\section{Conclusion \lv{\& Limitation}}
\label{sec:conclusion}
\vspace{1mm}
\lvv{This work explores the relation of model generalization and domain diversity, aiming to guarantee and further enhance the efficacy of domain augmentation strand. We then propose a framework to enable each diversified domain contribute to generalization by casting DG as 
a convex game between domains.
Heuristic analysis and comprehensive experiments demonstrate our rationality and effectiveness.}
Note that we mainly focus on the mixup-based domain augmentation techniques for clarity, while the extension of DCG to other GAN-based techniques needs to be further explored. Besides, it also remains an open problem to design a more efficient strategy to avoid the decrease in training efficiency caused by meta-learning. 
Nevertheless, we believe our work can inspire the future work of enriching domain diversity 
with improved generalization capability. 


\lvv{\paragraph{Acknowledgements.} This paper was supported by National Key R$\&$D Program of China (No. 2021YFB3301503), and also supported by the National Natural Science Foundation of China under Grant No. 61902028.}


%%%%%%%%% REFERENCES
\clearpage
{\small
\bibliographystyle{ieee_fullname}
\bibliography{egbib}
}

%%%%%%%%% APPENDIX
\clearpage
\appendix


  \section{Social Impact}
  \label{sec:impact}
  Our work focuses on domain generalization and attempts to make each training domain contribute to model generalization, which validates and further enhances the effectiveness of domain augmentation strand.
  This method produces a positive impact on the society and  community, saves the cost and time of data annotation, boosts the reusability of knowledge across domains, and greatly improves the efficiency. Nevertheless, this work suffers from some negative influences, which is worthy of further research and exploration. Specifically, more jobs of classification or target detection for rare or variable conditions may be cancelled. Moerover, we should be cautious about the result of the failure of the system, which could render people believe that classification was unbiased. Still, it might be not, which might be misleading, e.g., when using the system in a highly variable unseen target domain.
  
  \section{Algorithm of DCG}
  \label{sec:alg}
  
  \lvv{In this work, we propose a Domain Convex Game (DCG) framework to guarantee and further
  enhance the validity of domain augmentation approaches by casting DG as a convex game between domains. Here, we summarize the training process of DCG based on the discussions in main body as Algorithm~\ref{alg:DCG}.}

  \begin{algorithm}[H]
    \vspace{-1mm}
    \caption{The Algorithm of Domain Convex Game.}
    \label{alg:DCG}
    \begin{algorithmic} [1]
    \REQUIRE $P+Q$ diversified source domains $\mathcal{D}_s \cup \mathcal{D}_s^{aug}$; Hyper-parameters: $\omega, k$.
     \STATE randomly initialize model parameters $\boldsymbol{\theta}$.
    \FOR{iter in iterations}
    \STATE Randomly sample a mini-batch of $\mathcal{D}_s$ as $B$ and a mini-batch of $\mathcal{D}_s^{aug}$ as $B^{aug}$. 
    \STATE Split: $\tilde{\mathcal{D}_s}$ and $\tilde{\mathcal{D}_t}$ $\xleftarrow{} B$, Pick out: $\tilde{\mathcal{D}}_s^{aug}$ from $B^{aug}$.
    \STATE Construct coalitions $S,T$ by randomly sampling from $\tilde{\mathcal{D}_s} \cup \tilde{\mathcal{D}}_s^{aug}$; construct coalitions $S\cup T, S\cap T$.
    \STATE Calculate supermodularity regularization loss $\mathcal{L}_{sm}$ as Eq.~\eqref{eq:l_reg}.
    \STATE Pick out low-quality samples $\mathcal{D}_{del}$ with the top-$k$ score calculated by Eq.~\eqref{eq:score}.
    \STATE Calculate supervision loss $\mathcal{L}_{sup}$ as Eq.~\eqref{eq:l_cls}.
    \STATE Update $\boldsymbol{\theta} = \arg \min_{\boldsymbol{\theta}} \mathcal{L}_{sup} + \omega\mathcal{L}_{sm}$.
    \ENDFOR
    \end{algorithmic}
  \end{algorithm}
%   \vspace{-4mm}



  \section{Experimental Details}
  \label{sec:implementation}
  For all benchmarks, we conduct the commonly used leave-one-domain-out experiments~\cite{DBA}, where we choose one domain as the unseen target domain for evaluation, and train the model on all remaining domains. 
  We adopt the standard augmentation protocol as in~\cite{Jigen}, all images are resized to 224 × 224, following with random  resized cropping, horizontal flipping and color jittering. And the Fourier domain augmentation strategy utilized to diversify source domains closely follows the implementations in~\cite{FACT}. 
  \lvv{The network backbone is set to ResNet-18 or ResNet-50 pre-trained on ImageNet~\cite{resnet} following other related works.} 
  We train the network using mini-batch SGD with batch size 16, momentum 0.9 and weight decay 5e-4 for 50 epochs. The initial learning rate is 0.001 and decayed by 0.1 at 80\% of the total epochs. The meta step size $\alpha$ is set to be the same as the learning rate.
  For the hyper-parameters, i.e., the weight of regularization loss $\omega$ and the number of discarded bad samples in each iteration $k$, their values are selected on validation data following standard practice, where we use 90\% of available data as training data and 10\% as validation data. Specifically, we set $\omega = 0.1$ and $k=5$ for all experiments.
  Our framework is implemented with PyTorch on NVIDIA GeForce RTX 3090 GPUs. All results are reported based on the average accuracy over three independent runs for a fair comparison.

  

  \section{Additional Results}
  \lvv{\subsection{Time cost analysis}}
  \lvv{We conduct experiments to study the efficiency of our method in the training and inference stages respectively, and the results are shown in Table~\ref{tab:time_cost}. For the training stage, the time cost of DCG is indeed relatively high, which is due to
 the use of meta learning when constructing the regularization term and the backpropagation when calculating the score for sample filter. For substitute, we may edit the backpropagation path that computes gradients of inputs only on a smaller subnetwork to reduce time cost. Besides, we can see that for the inference stage, our DCG method is as efficient as other methods and does not incur additional time costs.  Note that this work is an innovative effort to study the relation between model generalization and domain diversity, which is in a preliminary stage. And we will further explore more efficient techniques in future research.}
  
 \begin{table}[htbp]
  \centering
    \setlength{\tabcolsep}{1.0mm}{
    % \resizebox{\columnwidth}{!}{
    \begin{tabular}{l|cc}
    \toprule
    Methods & Training & Inference\\
    \midrule
    DEEPALL~\cite{FACT} & 168 s & 5 s\\
    FACT~\cite{FACT} & 186 s & 5 s\\
    MLDG~\cite{MLDG} & 275 s & 5 s\\
    \midrule
    DCG w$/$o Filter. & 349 s & 5 s\\
    DCG & 467 s & 5 s\\
    \bottomrule
    \end{tabular}}
    \caption{Running Time per Epoch.}
  \label{tab:time_cost}
\end{table} 


  

% \begin{table*}[htp]
%   \centering
%     \setlength{\tabcolsep}{1.0mm}{
%     % \resizebox{1.0\columnwidth}{!}{
%     \begin{tabular}{l|cccc|c}
%     \toprule
%     Methods & Art & Cartoon & Photo & Sketch & Avg. \\
%     %\midrule
%     %\multicolumn{6}{c}{\textit{ResNet18}} \\
%     \midrule
%     RSC~\cite{RSC} & 87.89 & 82.16 & 97.92 & 83.35 & 87.83 \\
%     PCL \cite{PCL}& 90.20 &83.90& 98.10& 82.60 &	88.70 \\
%     \midrule
%     DeepAll\cite{DEEPALL} & 84.94$\pm$0.66 & 76.98$\pm$1.13 & 97.64$\pm$0.10 & 76.75$\pm$0.41 & 84.08 \\
%     FACT \cite{FACT}& 89.63$\pm$0.51 &81.77$\pm$0.19& 96.75$\pm$0.10& 84.46$\pm$0.78 &	88.15 \\
%     DDG \cite{DDG}& 88.90$\pm$0.60 &85.00$\pm$1.90& 97.20$\pm$1.20& 84.30$\pm$0.70 &	88.90 \\
%     STNP \cite{STNP} & 90.35$\pm$0.62 & 84.20$\pm$1.43 & 96.73$\pm$0.46 & 85.18$\pm$0.46 & 89.11\\
%     \midrule
%     DCG (\textit{ours})  & \textbf{85.94$\pm$0.21} &	\textbf{80.76$\pm$0.36}	& \textbf{96.41$\pm$0.17} &	\textbf{82.08$\pm$0.44}	& \textbf{86.30} \\
   
%     \bottomrule
%     \end{tabular}}
%     \caption{Leave-one-domain-out results on PACS (ResNet-50).}
%   \label{tab:pacs_res50}
% \end{table*}


% \begin{table*}[htp]
%   \centering
%     \setlength{\tabcolsep}{1.0mm}{
%     % \resizebox{\columnwidth}{!}{
%     \begin{tabular}{l|ccc|cc}
%     \toprule
%     Stage & DEEPALL~\cite{FACT} & FACT~\cite{FACT} & MLDG~\cite{MLDG} & DCG w$/$o Filter. & DCG\\
%     \midrule
%     Training &  168 s &	186 s &	275 s &349 s& 467 s\\
%     Inference & 5 s &5 s&5 s& 5 s&5 s\\
%     \bottomrule
%     \end{tabular}}
%     \caption{Running Time per Epoch.}
%   \label{tab:time_cost}
% \end{table*}

  
\begin{table*}[htpb]
  \centering
    \setlength{\tabcolsep}{4.0mm}{
    % \resizebox{1.0\columnwidth}{!}{
    \begin{tabular}{l|cccc|c}
    \toprule
    Methods & Art & Cartoon & Photo & Sketch & Avg. \\
    \midrule
    \multicolumn{6}{c}{\textit{ResNet18}} \\
    \midrule\
    MLDG~\cite{MLDG} & 78.70 & 73.30 & 94.00 & 65.10 & 80.70 \\
    L2A-OT~\cite{L2A-OT} & 83.30 & 78.20 & 96.20 & 73.60 & 82.80 \\
    RSC~\cite{RSC} & 83.43 & 80.31 & 95.99 & 80.85 & 85.15 \\
      DSU \cite{DSU} & 83.60 & 79.60 & 95.80 & 77.60 & 84.10\\
    \midrule
    DeepAll\cite{DEEPALL} & 77.63$\pm$0.84 & 76.77$\pm$0.33 & 95.85$\pm$0.20 & 69.50$\pm$1.26 & 79.94 \\
     MASF~\cite{MASF} & 80.29$\pm$0.18& 77.17$\pm$0.08& 94.99$\pm$0.09& 71.69$\pm$0.22& 81.04 \\
    DDAIG~\cite{DEEPALL} & 84.20$\pm$0.30& 78.10$\pm$0.60& 95.30$\pm$0.40& 74.70$\pm$0.80 & 83.10 \\
    MixStyle ~\cite{MixStyle} & 84.10$\pm$0.40 &  78.80$\pm$0.40 & 96.10$\pm$0.30 &  75.90$\pm$0.90 & 83.70 \\
    FACT \cite{FACT}& 85.37$\pm$0.29 &78.38$\pm$0.29& 95.15$\pm$0.26& 79.15$\pm$0.69 &	84.51 \\
    STNP \cite{STNP} & 84.41$\pm$0.62 & 79.25$\pm$0.98 & 94.93$\pm$0.07 & \textbf{83.27$\pm$2.03} & 85.47\\
    \midrule
    DCG (\textit{ours})  & \textbf{85.94$\pm$0.21} &	\textbf{80.76$\pm$0.36}	& \textbf{96.41$\pm$0.17} &	82.08$\pm$0.44	& \textbf{86.30} \\
  \midrule
  \multicolumn{6}{c}{\textit{ResNet50}} \\
  \midrule
  RSC~\cite{RSC} & 87.89 & 82.16 & 97.92 & 83.35 & 87.83 \\
  PCL \cite{PCL}& 90.20 &83.90& \textbf{98.10}& 82.60 &	88.70 \\
  \midrule
  DeepAll\cite{DEEPALL} & 84.94$\pm$0.66 & 76.98$\pm$1.13 & 97.64$\pm$0.10 & 76.75$\pm$0.41 & 84.08 \\
  FACT \cite{FACT}& 89.63$\pm$0.51 &81.77$\pm$0.19& 96.75$\pm$0.10& 84.46$\pm$0.78 &	88.15 \\
  DDG \cite{DDG}& 88.90$\pm$0.60 &85.00$\pm$1.90& 97.20$\pm$1.20& 84.30$\pm$0.70 &	88.90 \\
  STNP \cite{STNP} & \textbf{90.35$\pm$0.62} & 84.20$\pm$1.43 & 96.73$\pm$0.46 & 85.18$\pm$0.46 & 89.11\\
  \midrule
  DCG (\textit{ours})  & 90.24$\pm$0.48 &	\textbf{85.12$\pm$0.79}	& 97.76$\pm$0.13 &	\textbf{86.31$\pm$0.64}	& \textbf{89.84} \\
 
    \bottomrule
    \end{tabular}}
    \caption{Leave-one-domain-out results on PACS.}
  \label{tab:pacs_all}
\end{table*}


\begin{table*}[htpb]
  % \setlength{\abovecaptionskip}{0.cm}
  % \setlength{\belowcaptionskip}{0.cm}
  \centering
   \setlength{\tabcolsep}{4.0mm}{
    % \resizebox{0.95\columnwidth}{!}{
    \begin{tabular}{l|cccc|c}
    \toprule
    Methods & Art & Clipart & Product & Real & Avg. \\
    \midrule
    MLDG~\cite{MLDG} &52.88 & 45.72 & 69.90 & 72.68 & 60.30 \\
    SagNet \cite{SagNet} & 60.20 & 45.38 & 70.42 & 73.38 & 62.34\\
  %   Jigen~\cite{Jigen} & 53.04 & 47.51 & 71.47 & 72.79 & 61.20 \\
    RSC~\cite{RSC}   & 58.42 & 47.90 & 71.63 & 74.54 & 63.12 \\
    L2A-OT~\cite{L2A-OT} & 60.60 & 50.10 & 74.80 & \textbf{77.00} & 65.60 \\
    DSU \cite{DSU} & 60.20 & 54.80 & 74.10 & 75.10 & 66.10\\
    \midrule
    DeepAll \cite{DEEPALL} & 57.88$\pm$0.20 & 52.72$\pm$0.50& 73.50$\pm$0.30& 74.80$\pm$0.10 & 64.72 \\
  %   CCSA~\cite{CCSA}  & 59.90$\pm$0.30 &49.90$\pm$0.40& 74.10$\pm$0.20 &75.70$\pm$0.20 & 64.90 \\
  %   MMD-AAE~\cite{MMD-AAE} & 56.50$\pm$0.40& 47.30$\pm$0.30& 72.10$\pm$0.30& 74.80$\pm$0.20& 62.70 \\
  %   CrossGrad~\cite{CrossGrad} & 58.40$\pm$0.70& 49.40$\pm$0.40 &73.90$\pm$0.20 &75.80$\pm$0.10 & 64.40 \\
    DDAIG~\cite{DEEPALL} & 59.20$\pm$0.10& 52.30$\pm$0.30& 74.60$\pm$0.30& 76.00$\pm$0.10 & 65.50 \\
    MixStyle \cite{MixStyle} & 58.70$\pm$0.30 & 53.40$\pm$0.20 & 74.20$\pm$0.10 & 75.90$\pm$0.10 & 65.50\\
    FACT \cite{FACT}& 60.34$\pm$0.11& 54.85$\pm$0.37& 74.48$\pm$0.13& 76.55$\pm$0.10 & 66.56 \\
    STNP \cite{STNP} & 59.55$\pm$0.21 & 55.01$\pm$0.29 & 73.57$\pm$0.28 & 75.52$\pm$0.21 & 65.89\\
    \midrule
     DCG (\textit{ours}) & \textbf{60.67$\pm$0.14} &	\textbf{55.46$\pm$0.32} &	\textbf{75.26$\pm$0.18}	& 76.82$\pm$0.09 &	\textbf{67.05}  \\
    \bottomrule
    \end{tabular}}
    \caption{Leave-one-domain-out results on Office-Home.
    %   with ResNet-18. The best and second-best results are bold and underlined.
    }
  \label{tab:home_all}
\end{table*}




\subsection{Experimental Results with Error Bars}
\label{sec:additional_res} 
For the sake of objective, we run all the experiments multiple times with random seed. We report the average results in the main body of paper for elegant, and show the complete results with error bars in the form of mean$\pm$std below (Table.~\ref{tab:pacs_all},~\ref{tab:home_all},~\ref{tab:domainnet_all}).



\begin{table*}[b]
  % \setlength{\abovecaptionskip}{0.cm}
  % \setlength{\belowcaptionskip}{0.cm}
  \centering
   \setlength{\tabcolsep}{4.0mm}{
    % \resizebox{0.95\columnwidth}{!}{
    \begin{tabular}{l|cccc|c}
    \toprule
    Methods & Clipart & Painting & Real & Sketch & Avg. \\
    \midrule
    DeepAll \cite{DEEPALL} & 65.30 &	58.40	&64.70&	59.00&	61.86  \\
    \midrule
    ERM~\cite{ERM}  & 65.50 $\pm$ 0.3& 57.10 $\pm$ 0.5& 62.30 $\pm$ 0.2& 57.10 $\pm$ 0.1 & 60.50 \\
     MLDG~\cite{MLDG} & 65.70 $\pm$ 0.2& 57.00 $\pm$ 0.2& 63.70 $\pm$ 0.3& 58.10 $\pm$ 0.1 & 61.12 \\
     Mixup~\cite{mixup} & 67.10 $\pm$ 0.2& 59.10 $\pm$ 0.5& 64.30 $\pm$ 0.3& 59.20 $\pm$ 0.3 & 62.42 \\
    MMD~\cite{MMD} & 65.00 $\pm$ 0.5& 58.00 $\pm$ 0.2& 63.80 $\pm$ 0.2& 58.40 $\pm$ 0.7& 61.30 \\
    SagNet~\cite{SagNet} & 65.00 $\pm$ 0.4& 58.10 $\pm$ 0.2 &64.20 $\pm$ 0.3& 58.10 $\pm$ 0.4 & 61.35 \\
     CORAL~\cite{CORAL} & 66.50 $\pm$ 0.2& 59.50 $\pm$ 0.4 &66.00 $\pm$ 0.6& 59.50 $\pm$ 0.1 & 62.87 \\
     MTL~\cite{MTL} & 65.30 $\pm$ 0.5 &59.00 $\pm$ 0.4& 65.60 $\pm$ 0.4& 58.50 $\pm$ 0.2 & 62.10 \\
    \midrule
     DCG (\textit{ours}) & \textbf{69.38$\pm$0.19} &	\textbf{61.79$\pm$0.22}&	\textbf{66.34$\pm$0.27}&	\textbf{63.21$\pm$0.09}&	\textbf{65.18}  \\
    \bottomrule
    \end{tabular}}
    \caption{Leave-one-domain-out results on Mini-DomainNet.
%   with ResNet-18. The best and second-best results are bold and underlined.
}
  \label{tab:domainnet_all}
\end{table*}
%  \clearpage
%  \clearpage
% {\small
% \bibliographystyle{ieee_fullname}
% \bibliography{egbib}
% }


\end{document}
