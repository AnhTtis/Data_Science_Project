\section{INTRODUCTION}
The field of orbital robotics is gaining traction, as humans' presence in space expands and the sustainability of the orbital environment as well as its infrastructure are becoming increasingly important. 
Especially for highly complex operations such as on-orbit servicing, active debris removal, and on-orbit assembly, accurate and high-performance determination of the target's pose is vital. 
Due to their lower mass, power consumption, and system complexity, pose estimation systems based solely on monocular cameras have recently emerged as an appealing alternative to systems based on active sensors. 
However, the orbit is an extremely challenging domain for camera-based algorithms due to hostile illumination, aggressive reflections, high contrast, and background noise. 
These conditions drastically deteriorate the performance of classical, shape-based algorithms due to obscured geometric features. %because they can obscure geometric features of the target.

Recently, pose estimation methods based on \gls{DL} have shown great capabilities %to deal with such visual effects 
in terrestrial scenarios~\cite{sundermeyer2023bop}.
However, the space domain remains hesitant to widely adopt such methods, due to high computational cost, availability of space-qualified \gls{GPUs}, software verification, and validation concerns. 
Moreover, learning-based methods often require large amounts of annotated training data.% to achieve good performance.
Therefore, the pose estimation community is drifting towards the use of synthetic training data which requires drastically less human effort~\cite{denninger_blenderproc_nodate,muller_photorealistic_2021}.

\begin{figure}
    \centering
    %\def\svgwidth{\columnwidth}
    \import{images/}{introduction.pdf_tex}
    \caption{We predict 2D-3D correspondences and a pixel-wise coordinate estimation error using a single \gls{CNN}. Then, we leverage the error prediction with different thresholds $\epsilon \in [\epsilon_1,\dots,\epsilon_n]$ to formulate multiple pose hypotheses using \gls{PnP}.}
    \label{fig:introduction}
\vspace{-3mm}
\end{figure}
To this end, the \gls{ESA} and Stanford University published the SPEED+ dataset~\cite{Park2021-jr} -- and hosted an accompanying competition -- designed to investigate the performance gap between training on synthetic and testing on real data in the space domain. 
The competition has seen avid participation from the community and a host of solutions have been submitted. 
\begin{figure*}
    \centering
    \def\svgwidth{\textwidth}
    \import{images/}{method.pdf_tex}
    \caption{Overview of our proposed framework using \method. Given an RGB image, an object detector estimates a \gls{RoI} which is input to \method. Then, it estimates several pixel-wise features: (a) normalized 3D model coordinates, (b) coordinate error (c) foreground confidence, and (d) surface region class labels. The predictions (a), (b), and (c) are used to formulate pose hypotheses using \gls{PnP}. At last, each hypothesis is optimized and a final probability per pose is calculated.}
    \label{fig:overview}
\vspace{-3mm}
\end{figure*}
Many of the best-performing methods~\cite{Perez-Villar_undated-ly, Wang2022-rs} rely on regressing sparse 2D-3D correspondences, called keypoints, from the input image to solve the 6D pose through a variant of the \gls{PnP} algorithm. 
While these methods generally perform well, they have some downsides. 
Namely, they require manual selection of keypoints and can struggle whenever a large number of keypoints is occluded or obscured due to other visual effects. 
Moreover, as shown in recent terrestrial benchmarks~\cite{sundermeyer2023bop} they fall behind dense correspondence methods in terms of accuracy.
In contrast, methods that densely establish 2D-3D correspondence can struggle with inaccurate models and symmetries.


In this work, we propose to overcome these limitations by establishing dense 2D-3D correspondences while simultaneously training the model to predict errors in its correspondence regression. 
In essence, this allows the pipeline to programmatically discard unreliable correspondences which can be a result of 3D model inaccuracies, occlusions, or harsh visual effects. 
Given the predicted error map, we generate multiple pose hypotheses (see Fig.~\ref{fig:introduction}) which are further refined. 
At last, we select the most probable pose hypothesis, given color statistics and learned features.  
These additions, significantly boost the performance of our dense correspondence-based approach outperforming all prior satellite pose estimation methods and achieving state-of-the-art on the competitive SPEED+ benchmark. 
To summarize, the contributions of this paper are
\begin{enumerate*}[label=\textbf{\arabic*)}]
    \item  the \underline{E}rror-\underline{a}ware, \underline{g}eometrically-guid\underline{e}d, co\underline{r}respondence network (EagerNet), a 6D pose estimation algorithm that is robust to 3D modelling errors and harsh visual conditions,
     \item a hybrid learning- and region-based 6D pose refinement, 
     \item error weighed multi-hypothesis generation and probabilistic pose selection, 
     \item state-of-the-art results on SPEED+~\cite{Park2021-jr} with ablations on the method and data, and 
     \item a study on the effects of 3D model quality on the TUD-L~\cite{hodan_bop_2018} dataset.
\end{enumerate*}






% Sim2Real
% Bad Models



% \gls{DL} no factor in space although terrestrial applications are dominated by this tool: transfer/adapt these ideas to the context of space that also comes with various challenges

% Inaccurate models

% contributions
% \begin{itemize}
%     \item error-aware 6D pose estimation algorithm that is robust to 3D modelling errors and harsh visual conditions
%     \item learning incorporated in region-based 6D pose refinement 
%     \item probabilistic multi-hypothesis testing to unlock potential of error predictions
% \end{itemize}



% \begin{itemize}
% 	\item Motivation: As humans' presence in the universe/space expands, the relevance of space/orbit manipulation increases (maintenance, refueling, ``orbit cleaning")
% 	\item Space perception challenging: lightning conditions etc. \comMD{might be good to add figure with challenging examples and reference here}
% 	\item currently  \gls{DL} no factor in space although terrestrial applications are dominated by this tool: transfer/adapt these ideas to the context of space that also comes with various challenges %metion them
% 	\item inaccurate model \comMD{why relevant for space task?} \todo{ask Martin regarding works (in BOP?!) that have some findings on that}  \comMS{I am not aware of methods explicitly accounting for inaccurate models. But direct methods (correspondence-free) might be less susceptible. e.g. see Fig. 11 in "Correspondence-free pose estimation for 3D objects from noisy
% depth data"}
% 	\item Speed+ challenge
% 	\item ablation on TUD-L \comMD{mention why this choice}
% \end{itemize}


% %brainstorm method name
% Error-Aware, Geometry-guidEd, corRespondence network
% EagerNet?