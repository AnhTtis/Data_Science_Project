\begin{abstract}
We study the massless Vlasov equation on the exterior of the subextremal and extremal Reissner--Nordstr\"om spacetimes. We prove that moments decay at an exponential rate in the subextremal case and at a polynomial rate in the extremal case. This polynomial rate is shown to be sharp along the event horizon. In the extremal case we show that transversal derivatives of certain components of the energy momentum tensor do not decay along the event horizon if the solution and its first time derivative are initially supported on a neighbourhood of the event horizon. The non-decay of transversal derivatives in the extremal case is compared to the work of Aretakis on instability for the wave equation. Unlike Aretakis' results for the wave equation, which exploit a hierarchy of conservation laws, our proof is based entirely on a quantitative analysis of the geodesic flow and conservation laws do not feature in the present work.
\end{abstract}

%We show decay for moments of solutions and in the extremal case non-decay for derivatives under certain conditions.

%The non-decay of transversal derivatives in the extremal case is compared to the work of Aretakis on instability for the wave equation. In contrast to Aretakis' results however, conservation laws do not feature in the present work.