\section{Decay for massless Vlasov on extremal Reissner--Nordstr\"om} \label{section_ERN}
This section is dedicated to the proofs of Theorems~\ref{maintheoremERNprecise},~\ref{ERN_slowdecayprop} and~\ref{ERN_nondecaytransversal}. Recall that Theorem~\ref{maintheoremERNprecise} asserts that moments of solutions to the massless Vlasov equation decay at least at a polynomial rate, Theorem~\ref{ERN_slowdecayprop} establishes the sharpness of that rate of decay along the event horizon under certain conditions, while Theorem~\ref{ERN_nondecaytransversal} addresses the non-decay of transversal derivatives along the event horizon. Similarly to the case of the Schwarzschild solution, the main step in the proof is obtaining control over the momentum support of a solution. This will be the content of the two main Propositions~\ref{psupport_propERN} and~\ref{sharpness_prop_rough} of this chapter. \\

Proposition~\ref{psupport_propERN} is the extremal analogue of Proposition~\ref{psupport_prop} in the subextremal case and its statement analogously involves two subsets $\ERNtrappedsupportset$ and $\ERNsmallsupportset$ of the mass-shell, which were defined in Section~\ref{sec_subsets}. The set $\ERNtrappedsupportset \subset \mathcal{P}$ contains the geodesics which are almost trapped at the photon sphere and its definition is virtually identical to the Schwarzschild case. In particular, the set $\ERNtrappedsupportsetx$ approximates a $2$-cone in $\mathcal{P}_x$ at an exponential rate as $\tau(x) \rightarrow \infty$. The set $\ERNsmallsupportset \subset \mathcal{P}$ contains those geodesics which are almost trapped at the event horizon. As in the subextremal case it is made up of geodesics which originate close to the event horizon and are slowly outgoing. In addition, $\ERNsmallsupportset$ contains geodesics which are slowly infalling towards the event horizon. This latter type of geodesic is unique to the extremal case and does not exist in the subextremal case. Geometrically, the set $\ERNsmallsupportsetx$ may again be described as a small cylinder. In contrast to the subextremal case however, the height and radius of this cylinder are no longer exponentially small in $\tau(x)$ but inverse polynomial. See Figure~\ref{figure_fibres}. \\

Recall from Section~\ref{sec_subsets} that the definitions of the sets $\ERNtrappedsupportset$ and $\ERNsmallsupportset$ involve a choice of certain constants $\bigc, \decayrateERN > 0, \tauzero > m$, and the family $\badsetaplarge_\delta$ requires a choice of constant $\constbsl$, all of which will be chosen appropriately in the proof. We also recall the constant $\rsuppconst$ from Assumption~\ref{assumption_support}.

\begin{prop} \label{psupport_propERN}
Let $f_0 \in L^\infty(\mathcal{P}_0)$ satisfy Assumption~\ref{assumption_support} and let $f$ be the unique solution to the massless Vlasov equation on the on extremal Reissner--Nordstr\"om exterior with initial distribution $f_0$. Then there exist dimensionless constants $\bigc, \decayrate, C_0 > 0$ such that if we choose $\tauzero \geq C_0 \frac{\rsuppconst^2}{m}$,
\begin{equation}
	\supp(f) \cap \left\{ (x,p) \in \mathcal{P} \; | \; \tau(x) \geq \tauzero \right\} \subset \trappedsupportset_{\bigc, \decayrate, \tauzero} \cup \smallsupportset_{\bigc, \decayrate, \tauzero},
\end{equation}
In addition, there exists a dimensionless constant $\constbsl > 0$ such that for every $\bar{\tau} > \tauzero$ there exists $\delta > 0$ satisfying $\delta \sim \bar{\tau}^{-1}$ so that the following inclusion holds
\begin{equation} \label{pi0inclusion}
	\pi_0 \Big( (\smallsupportset \setminus \trappedsupportset) \cap \supp(f) \cap \left\{ (x,p) \in \mathcal{P} \; | \; \tau(x) \geq \bar{\tau} \right\} \Big) \subset \badsetaplarge_{\constbsl,\delta}.
\end{equation}
\end{prop}

%\begin{rem}
%Like in the Schwarzschild case discussed above, in case that $(x,p) \in \ERNtrappedsupportset$ the rate stated above at which $p^{t^*}$ and $p^r$ decay to their limit values as $\tau \rightarrow \infty$ may be improved in certain cases. Since the components of the momentum for a point $(x,p) \in \ERNsmallsupportset$ only decay at an inverse polynomial rate however, this fact will not be of use here.
%\end{rem}

While Proposition~\ref{psupport_propERN} is the central component for proving upper bounds for moments of a solution to the massless Vlasov equation, Proposition~\ref{sharpness_prop_rough} is the key ingredient in proving lower bounds, as well as non-decay of transversal derivatives. As explained above, one way in which the set $\ERNsmallsupportset$ differs from its subextremal counterpart is the rate at which the height and radius of the cylinder $\ERNsmallsupportsetx$ shrink as $\tau(x) \rightarrow \infty$. Another crucial difference is that even though both the radial and angular momentum components are small, the $p^{t^*}$-component is not. In fact we will show the existence of a family of geodesics which cross the event horizon at arbitrarily late times while satisfying $p^{t^*} \sim \psuppconst$ on the event horizon. The initial data of these geodesics are (essentially) contained in the family of subsets $\badsetapprox_\delta$ defined in Definition~\ref{def_badsetapprox}. Our second main Proposition~\ref{sharpness_prop_rough} now makes this precise.

%More precisely, if $(x,p) \in \ERNsmallsupportset$ is expressed in $(t^*,r)$-coordinates as $(x,p) = (t^*,r,\omega,p^{t^*},p^r,\pslash) \in \ERNsmallsupportset$ then
%\begin{equation} \label{ineq_ptstar_intro_secextremal}
%	p^{t^*} \lesssim \min \left( \frac{1}{\Osqern(r)} \frac{m^2}{(\tau(x)-\tauzero)^2}, 1 \right) \psuppconst.
%\end{equation}
%Inequality~\eqref{ineq_ptstar_intro_secextremal} is indicative of the fact that the $p^{t^*}$-component does not decay along the event horizon. Let us elaborate on this. At a positive distance from the event horizon, inequality~\eqref{ineq_ptstar_intro_secextremal} implies decay of the $p^{t^*}$-component: if $0 < \delta < 1$ and $r \geq (1+\delta)m$ then
%\begin{equation}
%	p^{t^*} \lesssim \frac{1}{\delta^2} \frac{m^2}{(\tau(x)-\tauzero)^2} \psuppconst.
%\end{equation}
%Close to or on the event horizon, inequality~\eqref{ineq_ptstar_intro_secextremal} only implies boundedness of the $p^{t^*}$-component.

\begin{prop} \label{sharpness_prop_rough}
Let $\slowsupportset \subset  \mathcal{P}|_{\mathcal{H}^+}$ as in Definition~\ref{defi_slowsupportset}. Then for every point on the event horizon $x \in \mathcal{H}^+$ with $\tau(x)$ large enough the fibre $\slowsupportsetx$ satisfies
\begin{itemize}
    \item Every $p \in \slowsupportsetx$ expressed in $(t^*,r)$-coordinates as $p = (p^{t^*}, p^r, \pslash)$ satisfies $p^{t^*} \sim \psuppconst$.
    \item The subset $\slowsupportsetx \subset \mathcal{P}_x$ has volume $\vol \slowsupportsetx \sim \tau(x)^{-2}$.
\end{itemize}
Furthermore for all $0 < \delta < \frac{1}{2}$ there exists a $\tauslow \sim \frac{m}{\delta}$ such that for any $x \in \mathcal{H}^+$ with $\tau(x) > \tauslow$
\begin{equation}
	\slowsupportsetx \subset \mathcal{P}_x \cap \bigcup_{s \geq 0} S_s(\badsetapprox_\delta),
\end{equation}
that is to say the geodesics with initial data in $\badsetapprox_\delta$ which cross the point $x$ populate the subset $\slowsupportsetx$. Every geodesic with initial data in $\badsetapprox_\delta$ will eventually cross the event horizon $\mathcal{H}^+$ and every point in $\badsetapprox_\delta$ satisfies Assumption~\ref{assumption_support}.
\end{prop}

%Older way to state the same Proposition
%\begin{prop} \label{sharpness_prop_rough}
%For all $0 < \delta < \frac{1}{2}$ there exists a time $\tauslow \sim \frac{m}{\delta}$ such that for any point $x \in \mathcal{H}^+$ with $\tau(x) \in (\tauslow, \infty)$ the following hold:
%\begin{itemize}
%    \item There exists a family of geodesics with initial data in $\badsetapprox_\delta$ which eventually cross the point $x$.
 %   \item The momenta of these geodesics populate an open subset $\slowsupportsetx \subset \mathcal{P}_x$ such that every $p \in \slowsupportsetx$ expressed in $(t^*,r)$-coordinates as $p = (p^{t^*}, p^r, \pslash)$ satisfies $p^{t^*} \sim \psuppconst$.
 %   \item The subset $\slowsupportsetx \subset \mathcal{P}_x$ has volume $\vol \slowsupportsetx \sim \tau(x)^{-2}$.
%\end{itemize}
%Furthermore, every point in $\badsetapprox_\delta$ satisfies Assumption~\ref{assumption_support}.
%\end{prop}

Throughout the discussion we will be careful to point out similarities to and differences from the corresponding estimates and results on the Schwarzschild exterior. Often times we will refrain from giving proofs when it is clear that they are analogous to the Schwarzschild case. In Section~\ref{prelimestimates_ERN} we establish preliminary estimates satisfied by all points on the mass-shell. We then bound the time required by a given geodesic to cross a certain region of spacetime in Section~\ref{timeestimate_section_ERN}, entirely analogous to the corresponding estimate on the Schwarzschild exterior. This is the analogue of the \emph{almost-trapping estimate} and the key to proving both Proposition~\ref{psupport_propERN} and Proposition~\ref{sharpness_prop_rough}, see Section~\ref{sec_radialgeod}. Using these estimates we are then able to prove Proposition~\ref{psupport_propERN} in Section~\ref{psupport_section_ERN}. In the following Section~\ref{psupporteventhorizon} we then prove a precise version of Proposition~\ref{sharpness_prop_rough}, the main ingredient necessary to allow us to show a lower bound for the rate of decay. Using the knowledge gained about the momentum support of a solution, we then apply Proposition~\ref{psupport_propERN} and estimate the volume of the sets $\ERNtrappedsupportset$ and $\ERNsmallsupportset$ to prove Theorem~\ref{maintheoremERNprecise} on upper bounds in Section~\ref{estimating_moments_section_ERN}. In Section~\ref{sec_proof_lowerboundERN} we apply Proposition~\ref{sharpness_prop_rough} to prove Theorem~\ref{ERN_slowdecayprop} on lower bounds. Finally we study transversal derivatives of the energy momentum tensor in Section~\ref{section_derivativesERN} and provide the proof of Theorem~\ref{ERN_nondecaytransversal}, which will again make substantial use of Proposition~\ref{sharpness_prop_rough}.


\subsection{Preliminary estimates} \label{prelimestimates_ERN}
We begin the discussion for the extremal case by obtaining preliminary estimates for momenta that are contained in the mass-shell. In a completely analogous fashion to Lemma~\ref{ptstar_bound}, we prove a bound for the $p^{t^*}$-component considered as a function on $\mathcal{P}$. Since the method is proof is virtually identical, we will state the result without proof.

\begin{lem} \label{ptstar_bounds_ern}
Let  $(x,p) = (t^*,r,\omega, p^{t^*},p^r,\pslash) \in \mathcal{P}$. Then if $p^r > 0$ and $r > m$ we have the bound
\begin{equation} \label{ptstar_bound_eq1_ERN}
	\frac{m^2}{r^2} E \left( 1 + \left| \trapern \right| \right) \lesssim p^{t^*} \leq \frac{2 E}{\Osqern}.
\end{equation}
If $p^r \leq 0$ and $r \geq m$ then
\begin{equation} \label{ptstar_bound_eq3_ERN}
	\frac{m^2}{r^2} E \left( 1 - \trapern \right) \lesssim p^{t^*} \lesssim \left( 1 + \frac{m^2}{r^2} \left| \trapern \right| \right) E \lesssim \frac{E}{\Osqern}.
\end{equation}
In particular we may conclude that if $\delta > 0$ and $(1+\delta)m \leq r$ we have
\begin{equation}
	p^{t^*} \lesssim_\delta E.
\end{equation}
\end{lem}

Secondly, we obtain bounds for the quotient $p^{t^*}/p^r$ in analogy to Lemma~\ref{lem_quotientbound}. We point out here that at a positive distance from the event horizon, the geodesic flow on extremal Reissner--Nordstr\"om is entirely comparable to that on Schwarzschild. 

\begin{lem} \label{prptstar_ern}
For $(x,p) = (t^*,r,\omega,p^{t^*},p^r,\pslash) \in \mathcal{P}$ and $\mathfrak{s} = \sgn(p^r) \in \{-1,+1\}$ we have
\begin{equation} \label{estimate_quotient_ERN}
	\frac{p^r}{p^{t^*}} \sim \mathfrak{s} \sqrt{(r-2m)^2+ \trapschw \mathfrak{a}} \, \frac{1}{r} \mathfrak{A}_s,
\end{equation}
where $\mathfrak{a} \sim m^2 \Osqern$ and for $0 < \delta < 1$ fixed we have
\begin{equation}
	\mathfrak{A}_s \sim_\delta \begin{cases}
		1 & r \geq (1+\delta)m \\
		\begin{cases}
			\frac{1}{2- \trapschw} & \mathfrak{s}=-1 \\
			\Osqern & \mathfrak{s}=1
		\end{cases} & m \leq r \leq (1+ \delta)m
	\end{cases}.
\end{equation}
In particular, away from the horizon the quotient takes an identical form for $\mathfrak{s} \in \{-1,+1\}$ apart from the obvious difference in sign.
\end{lem}

\begin{rem}
Apart from the obvious difference in the position of the event horizon and the photon sphere and replacing the subextremal with the extremal lapse function $\OsqS$, the form of estimate~\eqref{estimate_quotient_ERN} is identical to that for the Schwarzschild exterior given in Lemma~\ref{lem_quotientbound}. In addition, the same comments about continuity of the right hand side of equation~\eqref{estimate_quotient_ERN} apply, see Remark~\ref{rem_prelimestimates_continuity}.
\end{rem}

\begin{proof}
We begin by pointing out that most of the computations and bounds will be similar in form to the case of Schwarzschild. Expressing both sides of the expression~\eqref{def_eps} above in $(t^*,r)$-coordinates and collecting terms results in a polynomial expression similar to the Schwarzschild case
\begin{equation}
	\left( b - \trapschw (1-\Osqern)^2 \right) (p^r)^2 +2(1-\Osqern) \left( a + \trapschw \Osqern  \right) p^{t^*} p^r - \Osqern  \left( a + \trapschw \Osqern  \right) (p^{t^*})^2 = 0,
\end{equation}
where we have abbreviated
\begin{equation} \label{def_ab_ern}
	a = \frac{r^2}{16m^2} - \Osqern = \left(1-\frac{2m}{r}\right)^2 \left(1+\frac{4m}{r}-\frac{4m^2}{r^2}\right) \frac{r^2}{16m^2}, \quad b = (1-\Osqern)^2 + \frac{r^2}{16m^2} (2-\Osqern).
\end{equation}
As in the Schwarzschild case, we may interpret this as a quadratic equation in the variable $p^r$ and solve for $p^r$ as a function of $p^{t^*},r,\trapschw$. The equation then has discriminant
\begin{equation}
	\Delta = 4 (a+\trapschw \Osqern) (a (1-\Osqern)^2+ b \Osqern) (p^{t^*})^2,
\end{equation}
so that $\sgn{(\Delta)} = \sgn{(a+\trapschw \Osqern)}$. Note carefully that from inequality~\eqref{epsbound} or equivalently from conservation of energy~\eqref{Schw::energy_conservation} it immediately follows that
\begin{equation}
	a + \trapern \Osqern = \frac{r^2}{16 m^2} - (1 - \trapern) \Osqern = \frac{1}{16m^2} \left( r^2 - \frac{L^2}{E^2} \Osqern \right) \geq 0,
\end{equation}
so that we may conclude $\Delta \geq 0$ and the quadratic equation either has two distinct real roots or one real-valued double root. Lemma~\ref{express_ptstar} allows us to conclude that $p^{t^*} > 0$ unless $p^\mu = 0$ for all $\mu \in \{t^*,r,\omega\}$. Keeping this in mind we find
\begin{equation}
	p^r = \frac{-(1-\Osqern)(a + \trapschw \Osqern) + \mathfrak{s} \sqrt{(a+\trapschw \Osqern) \left( a (1-\Osqern)^2 + b \Osqern \right)}}{ \left( b - \trapschw (1-\Osqern)^2 \right)} p^{t^*},
\end{equation}
where $\mathfrak{s} \in \{-1,+1 \}$ is the usual sign in the solution formula for quadratic equations. We begin by showing that $\sgn(p^r) = \mathfrak{s}$. First note that the bound~\eqref{epsbound} implies $b-\trapschw (1-\Osqern)^2 \geq (2-\Osqern) \frac{r^2}{16m^2} > 0$ by definition of $b$. Thus we see immediately that if $\mathfrak{s}=-1$ then $p^r < 0$. Let us then assume that $\mathfrak{s}=1$. Then the following rearrangement shows that
\begin{align}
	p^r|_{\mathfrak{s}=1} > 0 & \iff - \left(1-\Osqern \right) \sqrt{a+\trapschw \Osqern} + \sqrt{a \left(1-\Osqern \right)^2 + b \Osqern} > 0 \\
	& \iff a\left(1-\Osqern \right)^2 + b \Osqern > \left(1-\Osqern \right)^2 \left(a + \trapschw \Osqern \right) \\
	& \iff b > \trapschw \left(1-\Osqern \right)^2,
\end{align}
which we know to be true by the above computation. Therefore we have shown that $\sgn(p^r) = \mathfrak{s}$. \\

We next bound the various expressions occuring in the quotient. We begin by looking at the expression occuring in the denominator $b- \trapschw \left(1-\Osqern \right)^2$. From the definition of $b$ it follows that $b- \trapschw \left(1-\Osqern \right)^2 = (1- \trapschw) \left(1-\Osqern \right)^2 + \frac{r^2}{16m^2} \left( 2- \Osqern \right)$. By inserting the bound~\eqref{epsbound} on $\trapschw$ we find
\begin{equation}
	\frac{r^2}{m^2} \lesssim b- \trapschw \left(1-\Osqern \right)^2  \lesssim \frac{r^2}{m^2} \frac{1}{\Osqern}.
\end{equation}
In fact we can easily obtain a more precise estimate of the expression $b- \trapschw \left(1-\Osqern \right)^2$ by considering the region close to and far from the event horizon separately. Let $0 < \delta < 1$ be as in the statement of the result, then we find
\begin{equation}
	b- \trapschw \left(1-\Osqern \right)^2 \sim_\delta \begin{cases}
		1  + \left| \trapschw \right| & m \leq r \leq (1+\delta) m \\
		 \frac{r^2}{m^2} & (1+\delta)m \leq r
	\end{cases}.
\end{equation}
Let us now turn to the numerator of the quotient. We immediately note that by the definition of $a$ and $b$ we have
\begin{equation}
	a \left( 1 - \Osqern \right)^2 + b \Osqern = \frac{r^2}{16m^2}.
\end{equation}
We next consider the expression $a+ \trapschw \Osqern$. We note immediately that
\begin{equation}
	0 \leq a + \trapschw \Osqern = \frac{r^2}{16m^2} - (1- \trapern) \Osqern \leq \frac{r^2}{16m^2}.
\end{equation}
By introducing the function
\begin{equation}
	0 \leq \mathfrak{a} = 16 m^2 \frac{(r-m)^2}{r^2+4mr-4m^2} \leq 16 m^2,
\end{equation}
we may rewrite
\begin{equation}
	a + \trapschw \Osqern \sim \frac{1}{m^2} \left( (r-2m)^2 + \mathfrak{a} \trapschw \right).
\end{equation}
Let us now bound the quotient in case $\mathfrak{s}=-1$. Begin by considering the numerator
\begin{align}
	&-(1-\Osqern)(a + \trapschw \Osqern) - \sqrt{(a+\trapschw \Osqern) \left( a (1-\Osqern)^2 + b \Osqern \right)} \\
	=&-\sqrt{a+\trapschw \Osqern} \left[ \underbrace{\left( 1 - \Osqern \right)\sqrt{a+\trapschw \Osqern}}_{\lesssim 1} + \underbrace{\sqrt{a (1-\Osqern)^2 + b \Osqern}}_{\sim \frac{r}{m}} \right] \\
	\sim&- \sqrt{a+\trapschw \Osqern} \frac{r}{m},
\end{align}
where we have made use of the fact that in the whole exterior $1-\Osqern \sim \frac{m}{r}$. Combining this with the bound on $b-\trapschw \Osqern$ we find
\begin{equation}
	\frac{p^r}{p^{t^*}} \sim - \sqrt{(r-2m)^2+ \trapschw \mathfrak{a}} \frac{r}{r^2 - \trapschw \frac{m^4}{2r^2}}.
\end{equation}
For convenience, let us note that the second factor looks like
\begin{equation}
	\frac{r}{r^2 - \trapschw \frac{m^4}{2r^2}} \sim_\delta \begin{cases}
		\frac{1}{m(2- \trapschw)} & m \leq r \leq (1+\delta)m \\
		\frac{1}{r} & (1+\delta)m \leq r
	\end{cases}.
\end{equation}
Note that the only difference to the Schwarzschild case is hidden inside the function $\mathfrak{a}$, which in the extremal case vanishes quadratically at the event horizon, as opposed to vanishing linearly at the event horizon in the subextremal case. \\

Let us now consider the case $\mathfrak{s}=1$. Entirely similar to the Schwarzschild case we use the fact that for $0 \leq z \leq 1$ the relation $1- z \sim 1 - \sqrt{z}$ holds and the fact that
\begin{equation}
	0 \leq \frac{(1-\Osqern)^2 (a + \trapern \Osqern)}{a (1-\Osqern)^2 + b \Osqern} = \frac{16m^2}{r^2}  (1-\Osqern)^2 (a + \trapern \Osqern) \leq (1 - \Osqern)^2 \leq  1,
\end{equation}
to conclude
\begin{align}
	&-(1-\Osqern)(a + \trapschw \Osqern) + \sqrt{(a+\trapschw \Osqern) \left( a (1-\Osqern)^2 + b \Osqern \right)} \\
	=& \sqrt{a+\trapschw \Osqern} \left[ \sqrt{a (1-\Osqern)^2 + b \Osqern} - \sqrt{\left( 1-\Osqern \right)^2 \left( a+ \trapschw \Osqern \right)} \right] \\
	=& \sqrt{a+\trapschw \Osqern} \frac{r}{\sqrt{16} m} \left[ 1 - \sqrt{\frac{(1-\Osqern)^2 (a + \trapern \Osqern)}{a (1-\Osqern)^2 + b \Osqern}} \right] \\
	\sim& \sqrt{a+\trapschw \Osqern} \frac{r}{m} \left[ 1-\frac{(1-\Osqern)^2 (a + \trapern \Osqern)}{a (1-\Osqern)^2 + b \Osqern} \right] \\
	\sim & \sqrt{a+\trapschw \Osqern} \frac{m}{r} \Osqern \left( b - \trapschw \left(1 - \Osqern \right) \right).
\end{align}
Therefore there is a cancellation with the denominator and we find
\begin{equation}
	\frac{p^r}{p^{t^*}} \sim \frac{1}{r} \Osqern \sqrt{(r-2m)^2+ \trapschw \mathfrak{a}},
\end{equation}
where we point out that again there is a difference to the Schwarzschild case hidden in the factor $\mathfrak{a}$. In summary we find
\begin{equation}
	\frac{p^r}{p^{t^*}} \sim \mathfrak{s} \sqrt{(r-2m)^2+ \trapschw \mathfrak{a}} \, \frac{1}{r} \mathfrak{A}_s,
\end{equation}
where the coefficient $\mathfrak{A}_s \sim_\delta 1$ away from the horizon regardless of the sign $\mathfrak{s}$ and close to the horizon
\begin{equation}
	\mathfrak{A}_s \sim_\delta \begin{cases}
		\frac{1}{2- \trapschw} & \mathfrak{s}=-1 \\
		\Osqern & \mathfrak{s}=1
	\end{cases}.
\end{equation}
We again point out that the factor $\mathfrak{a}$ takes a different form in the extremal case.
\end{proof}


\subsection{The almost-trapping estimate} \label{timeestimate_section_ERN}
In this section we will prove a bound for the time a given geodesic requires to cross a certain region of spacetime in terms of the radius the geodesic occupies at the beginning and at the end of the segment and the value of its trapping parameter. The main result of this section is Lemma~\ref{tauestimate_ERN}.

\begin{lem}[The almost-trapping estimate] \label{tauestimate_ERN}
Let $\gamma: [s_0,s_1] \rightarrow \Mrn$ be  an affinely parameterised future-oriented null geodesic. Let us express $\gamma$ in $(t^*,r)$-coordinates as $\gamma(s) = (t^*(s),r(s),\omega(s))$ with momentum $\dot{\gamma}(s) = (p^{t^*}(s),p^r(s),\pslash(s))$. Assume that $\gamma$ intersects $\Sigma_0$ at radius $r_0 = r(s_0) \leq \rsuppconst$. Let $0 < \delta < 1$ and denote $\mathfrak{s} = \chi_{(0,\infty)}\left( p^r(s_0) \right)$, where $\chi_{(0,\infty)}$ denotes the characteristic function of the set $(0,\infty) \subset \R$. There exists a constant $C_0 > 0$ such that if we assume $\tauzero \geq C_0 \frac{\rsuppconst^2}{m}$ then for all $s \in [s_0,s_1]$ with $m \leq r(s) \leq (1+\delta)m$ we have the bound
\begin{equation}
	\frac{1}{m} \left( \tau(\gamma(s)) - \tauzero \right) \lesssim_{\delta} \begin{cases}
		\mathfrak{s} \frac{1}{\sqrt{\Osqern(r_0)}} & \text{if } p^r(s) > 0 \\
		\left( 1 + \left| \trapschw \right| \right) \sqrt{\Osqern(r_0)} + \mathfrak{s} \frac{1}{\sqrt{\Osqern(r_0)}} + \left( \log \left| \trapern \right| \right)_-  & \text{if } p^r(s) \leq 0
		\end{cases}.
\end{equation}
If $r(s) \geq (1+\delta)m$ we have
\begin{equation}
	\frac{1}{m} \left( \tau(\gamma(s)) - \tauzero \right) \lesssim_{\delta} \left( \log \left| \trapern \right| \right)_- + \mathfrak{s} \frac{1}{\sqrt{\Osqern(r_0)}},
\end{equation}
for both signs of $p^r$. We have abbreviated $\trapern = \trapern(\gamma,\dot{\gamma})$.
\end{lem}

The strategy of proof is roughly the same as in Section~\ref{section_tstar}. We first obtain a bound for the time a geodesic requires to cross a certain region of spacetime as measured in the $t^*$-coordinate in Lemma~\ref{tstar_estimate_ern}. This is the extremal analogue of Lemma~\ref{tstar_lem_schw}. Since the geodesic flow at a positive distance from the event horizon is entirely comparable to the geodesic flow on Schwarzschild at a positive distance from the event horizon, we do not prove bounds for the region away from the event horizon. In fact, it is straightforward to verify that both Lemma~\ref{pr_like_E} and Lemma~\ref{tauestimate_far} hold verbatim in the extremal Reissner--Nordstr\"om case. Having established these results, the argument to prove Lemma~\ref{tauestimate_ERN} proceeds along identical lines as for the corresponding Lemma~\ref{taubound} in the Schwarzschild case and we do not reproduce it here. It therefore only remains to prove Lemma~\ref{tstar_estimate_ern}. See Section~\ref{sec_radialgeod} for an informal discussion of the proof in the special case of radial geodesics.

\subsubsection{The $t^*$-time estimate}
As in Section~\ref{sec_tstarestimate} we split the black hole exterior in several regions to simplify our statement of the $t^*$-time estimate. Fix three constants $0< \delta_1,\delta_2 < 1$ and $\delta_3 > 0$ such that $0 < \delta_1 + \delta_2 < 1$. Consider the following intervals
\begin{align}
	\mathfrak{I}_{\mathcal{H}^+} &= [m, (1+\delta_1)m], \\
	\mathfrak{I}_{\text{int}} &= [(1+\delta_1)m, (2-\delta_2)m], \\
	\mathfrak{I}_{\text{ps}} &= [(2-\delta_2)m, (2+\delta_3)m], \\
	\mathfrak{I}_{\text{flat}} &= [(2+\delta_3)m, \infty),
\end{align}
where we have suppressed the dependence on $\delta_1,\delta_2,\delta_3$ in the notation.

\begin{lem} \label{tstar_estimate_ern}
Denote by $\gamma$ an affinely parameterised future-oriented null geodesic expressed in $(t^*,r)$-coordinates as $\gamma(s) = (t^*(s),r(s),\omega(s))$ with momentum $\dot{\gamma}(s) = (p^{t^*}(s),p^r(s),\pslash(s))$. Assume that $p^r \neq 0$ in the interval of affine parameter time $[s_1,s_2]$, so that the radius $r$ is a strictly monotonical function of $s$. Let us assume that $r(s_i) = r_i$ for $i=1,2$ and let $0 < \delta_1,\delta_2,\delta_3$ be as above. Then
\begin{equation} \label{t*bound_ERN}
		\frac{t^*(s_2) - t^*(s_1)}{m} \lesssim_{\delta_1,\delta_2,\delta_3 } \begin{cases}
			\begin{cases}
				\left( 1 + \left| \trapschw \right| \right) \sqrt{\Osqern(r_1)} & p^r \leq 0 \text{ on } [r_2,r_1] \\
				\frac{1}{\sqrt{\Osqern(r_1)}} & p^r > 0 \text{ on } [r_1,r_2]
			\end{cases} & r_1,r_2 \in \mathfrak{I}_{\mathcal{H}^+} \\
			1 & r_1,r_2 \in \mathfrak{I}_{\text{int}} \\
			1 + \left( \log \left| \trapschw \right| \right)_- & r_1,r_2 \in \mathfrak{I}_{\text{ps}} \\
			1 + \frac{r_2}{m} & r_1,r_2 \in \mathfrak{I}_{\text{flat}}
		\end{cases}
\end{equation}
where we have used the shorthand $\trapschw = \trapschw(\gamma(s),\dot{\gamma(s)})$. In addition if we assume that $p^r \leq 0$ for $s \in [s_1,s_2]$ and that $r_2 = m, r_1 \in \mathfrak{I}_{\mathcal{H}^+}$ we in fact also have the lower bound
\begin{equation} \label{tstar_lowerbound_ern}
	\frac{t^*(s_2) - t^*(s_1)}{m} \gtrsim_{\delta_1} \left( 1 + \left| \trapschw \right| \right) \sqrt{\Osqern(r_1)}.
\end{equation}
If we assume instead that $r_2 > m, r_1 \in \mathfrak{I}_{\mathcal{H}^+}$ then the above bound readily implies the upper bound
\begin{equation} \label{tstar_upperbound_triv}
	\frac{t^*(s_2) - t^*(s_1)}{m} \lesssim_{\delta_1} \frac{1}{\sqrt{\Osqern(r_2)}}.
\end{equation}
Furthermore if we assume that $p^r > 0$ for $s \in [s_1,s_2]$ and $\trapschw \ll -1$ then we may assume that $r_{\text{min}}^-(\trapern) \in \mathcal{I}_{\mathcal{H}^+}$. If we assume $r_2 =  r_{\text{min}}^-(\trapern)$ and $r_1 \leq \frac{1}{2} (r_2 + m)$ then we in fact also have the lower bound
\begin{equation}
	\frac{t^*(s_2) - t^*(s_1)}{m} \gtrsim_{\delta_1} \frac{1}{\sqrt{\Osqern(r_1)}}.
\end{equation}
\end{lem}

\begin{rem}
Let us compare Lemma~\ref{tstar_estimate_ern} to the corresponding Lemma~\ref{tstar_lem_schw} in the subextremal case. The behaviour close to the event horizon is different. If we let $\gamma$ be an infalling geodesic close to the event horizon with $\trapern(\gamma) \ll -1$, a qualitatively different behaviour is exhibited in the extremal as compared to the subextremal case. To be more precise, if we assume $\gamma: [s_1,s_2] \rightarrow \Mern$ is an infalling null geodesic on ERN and is such that $r(s_1) = r_{\text{min}}^-(\trapern), r(s_2) = m$, then Lemma~\ref{tstar_estimate_ern} above shows that $t^*(s_2) - t^*(s_1) \sim m \sqrt{ 1 + \left| \trapern(\gamma) \right| }$ if $\trapern(\gamma) \ll -1$. The corresponding bound in the subextremal case for the time a geodesic requires to fall into the black hole however remains constant as $\trapschw(\gamma) \rightarrow - \infty$. More precisely, if we analogously let $\gamma: [s_1,s_2] \rightarrow \Mschw$ be an infalling geodesic on Schwarzschild with $r(s_1) = r_{\text{min}}^-(\trapern), r(s_2) = 2m$, then Lemma~\ref{tstar_lem_schw} implies $t^*(s_2) - t^*(s_1) \lesssim m$. The heuristic reason for this is the degeneracy of the red-shift effect. If instead we let $\gamma$ be an outgoing geodesic originating close to the event horizon, the behaviour is qualitatively comparable to the subextremal case. However, the time required to leave the region close to the event horizon is inverse linear in the initial distance from $\mathcal{H}^+$ in the extremal case and logarithmic in the subextremal case.
\end{rem}

\begin{proof}
The proof will proceed along the same lines as the proof of the corresponding Lemma~\ref{tstar_lem_schw} for the Schwarzschild background. We use $(t^*,r)$-coordinates as defined in Section~\ref{rngeometry}. In the entire argument all momenta and radii are considered along the geodesic $\gamma$ and we will frequently omit the dependence on the affine parameter $s$. Since for $s \in [s_1,s_2]$ we have $p^r \neq 0$ we may conclude in an identical fashion to the Schwarzschild case that
\begin{equation}
t^*(s_2) - t^*(s_1) = \int_{r_1}^{r_2} \frac{p^{t^*}}{p^r} \, d r.
\end{equation}
where we carefully note that if $p^r < 0$, the integration boundaries will be such that $r_2 < r_1$ so that the orientation of the integral ensures the correct sign. Note also that we are slightly abusing notation and using the radius $r$ as an integration variable. We will now use Lemma~\ref{prptstar_ern} to bound the integrand in different regions of spacetime. \\

It follows immediately from Lemma~\ref{prptstar_ern} that away from the event horizon the quotient $\frac{p^{t^*}}{p^r}$ is comparable to the quotient in the Schwarzschild case apart from the apparent difference in location of the event horizon and photon sphere. Therefore if $r_1,r_2 \geq (1+\delta)m$ for $0 < \delta < 1$, the computations take on an almost identical form to those in the proof of Lemma~\ref{tstar_lem_schw}. We therefore focus our attention on the region close to the event horizon and assume that $m \leq r_1,r_2 \leq (1+\delta) m$ for some fixed $0 < \delta = \delta_1 < 1$. We will omit the index of $\delta_1$ for simplicity of notation. \\

Let us begin by considering the case that $p^r \leq 0$ so that $r_1 \geq r_2$ and
\begin{equation}
\int_{r_1}^{r_2} \frac{p^{t^*}}{p^r} \, d r = \int_{r_2}^{r_1} \frac{p^{t^*}}{\left| p^r \right|} \, d r.
\end{equation}
We further distinguish the two cases that $\trapern < 0$ and $\trapern \geq 0$. If $\trapschw \geq 0$ let us apply Lemma~\ref{prptstar_ern} and note that $(r-2m)^2 + \trapschw \mathfrak{a} \sim_\delta m^2$ and $\frac{1}{(2- \trapschw)} \sim 1$, so that $\frac{p^{t^*}}{|p^r|} \sim_\delta 1$. Hence
\begin{equation}
	\int_{r_2}^{r_1} \frac{p^{t^*}}{\left| p^r \right|} \, d r \sim_\delta r_1 - r_2 \leq r_1 - m.
\end{equation}
In particular it is evident that if $r_2 = m$ and $\trapern \geq 0$ then the bound~\eqref{tstar_lowerbound_ern} holds. If $\trapschw < 0$, let us again apply Lemma~\ref{prptstar_ern} and write
\begin{equation}
	(r-2m)^2 + \trapschw \mathfrak{a} = (r-2m)^2 \left( 1 + \trapschw (r-m)^2 \mathfrak{b} \right) \sim_\delta m^2 \left( 1 + \trapschw (r-m)^2 \mathfrak{b} \right) = m^2 \left( 1 + \trapschw x^2 \mathfrak{b} \right),
\end{equation}
where we have introduced the change of coordinates $x=r-m$ and defined the coefficient
$\mathfrak{b}(x) = 16m^2 (x-m)^{-2} (x^2+6mx+m^2)^{-1}$. Note that $m^2 \mathfrak{b} \sim_\delta 1$ in the region of integration. We find
\begin{equation}
	\int_{r_2}^{r_1} \frac{p^{t^*}}{|p^r|} \, dr \sim_\delta \int_{r_2}^{r_1} \frac{2-\trapschw}{\sqrt{1+\trapschw (r-m)^2 \mathfrak{b}}} \, dr = \int_{r_2-m}^{r_1-m} \frac{2-\trapschw}{\sqrt{1+\trapschw x^2 \mathfrak{b}}} \, dx \sim_\delta m \int_{y_2}^{y_1} \frac{2-\trapschw}{\sqrt{1+\trapschw y^2}} \, dy,
\end{equation}
where we have made the two changes of coordinate $x = r-m$ and $y = x \sqrt{\mathfrak{b}(x)}$ and have furthermore correspondingly set $y_i =  (r_i-m) \sqrt{\mathfrak{b}(r_i-m)}$ for $i=1,2$. In the last step we made use of the fact that the Jacobian
\begin{equation}
    \left| \sqrt{\mathfrak{b}} \left( 1 + \frac{x}{2} \frac{\mathfrak{b}'}{\mathfrak{b}} \right) \right|^{-1} = \mathfrak{b}^{-\frac{1}{2}} \frac{(x-1) (x^2+6 x+1)}{(x+1)^3} \sim_{\delta} m.
\end{equation}
We note that by definition $1+ \trapschw y^2 = 1 - \left| \trapschw \right| y^2 \geq 0$ so that $0 \leq \sqrt{\left| \trapschw \right|} y_i \leq 1$ for $i=1,2$. We then compute
\begin{align}
	&\int_{y_2}^{y_1} \frac{2-\trapern}{\sqrt{1+\trapern y^2}} \, dy = \frac{2+ \left| \trapern \right|}{\sqrt{\left| \trapern \right|}} \int_{\sqrt{\left| \trapern \right|} y_2}^{\sqrt{\left| \trapern \right|} y_1} \frac{1}{\sqrt{1-z^2}} \, dz \sim \frac{2+ \left| \trapern \right|}{\sqrt{\left| \trapern \right|}} \int_{\sqrt{\left| \trapern \right|} y_2}^{\sqrt{\left| \trapern \right|} y_1} \frac{1}{\sqrt{1-z}} \, dz \\
	&= 2 \frac{2+ \left| \trapern \right|}{\sqrt{\left| \trapern \right|}} \left( \sqrt{1-\sqrt{\left| \trapern \right|} y_2}-\sqrt{1-\sqrt{\left| \trapern \right|} y_1} \right) \leq 2 \frac{2+ \left| \trapern \right|}{\sqrt{\left| \trapern \right|}} \left( 1-\sqrt{1-\sqrt{\left| \trapern \right|} y_1} \right) \label{inequ_proof_tstart_ern} \\
	&\sim \frac{2+ \left| \trapern \right|}{\sqrt{\left| \trapern \right|}} \sqrt{\left| \trapern \right|} y_1 = \left( 2+ \left| \trapern \right| \right) y_1 \sim_\delta \left( 1 + \left| \trapern \right| \right) \sqrt{\Osqern(r_1)}.
\end{align}
Note carefully that in line~\eqref{inequ_proof_tstart_ern} we are being somewhat wasteful. In fact, whenever we assume that $(\sqrt{\left| \trapern \right|} y_1,\sqrt{\left| \trapern \right|} y_2) \notin [1-\kappa,1]^2$ for some fixed $0 < \kappa < 1$ we have the estimate
\begin{equation}
	\frac{2+ \left| \trapern \right|}{\sqrt{\left| \trapern \right|}} \left( \sqrt{1-\sqrt{\left| \trapern \right|} y_2}-\sqrt{1-\sqrt{\left| \trapern \right|} y_1} \right) \sim_\kappa \left( 1+ \left| \trapern \right| \right) \left( y_1 - y_2 \right),
\end{equation}
so that it follows in particular when we consider the special case that the geodesic arrives at the event horizon at affine parameter time $s_2$ so that $y_2 = 0 \iff r_2 = m$ we have the estimate
\begin{equation}
	\int_{m}^{r_1} \frac{p^{t^*}}{|p^r|} \, dr \sim_\delta \left( 1+ \left| \trapern \right| \right) (r_1-m),
\end{equation}
which readily allows us to conclude the validity of bound~\eqref{tstar_lowerbound_ern}. Next notice that if we assume $r_1 \geq r_2 > m$ we may further bound
\begin{equation}
	 \left( 1 + \left| \trapern \right| \right) \sqrt{\Osqern(r_1)} \lesssim_\delta \sqrt{1 + \left| \trapern \right|} \lesssim_\delta \frac{1}{\sqrt{\Osqern(r_2)}},
\end{equation}
as claimed in equation~\eqref{tstar_upperbound_triv}. Alternatively, one may also readily deduce the bound claimed in equation~\eqref{tstar_upperbound_triv} from the computation above. \\

Let us now consider the case that $p^r \geq 0$ and let $0 < \delta = \delta_2 < 1$ now. We will again omit the index of $\delta_2$ for simplicity of notation. Then depending on the sign of $\trapschw$, the geodesic will eventually either cross the photon sphere or scatter off it, however we are only interested in the region close to the horizon here. Note that since $p^r \geq 0$ we have
\begin{equation}
	\int_{r_1}^{r_2} \frac{p^{t^*}}{p^r} \, d r = \int_{r_1}^{r_2} \frac{p^{t^*}}{\left| p^r \right|} \, d r.
\end{equation}
If $\trapschw \geq 0$ we begin by noting that Lemma~\ref{prptstar_ern} implies that $\frac{p^{t^*}}{|p^r|} \sim_\delta \frac{1}{\Osqern}$ in a similar fashion to the case that $p^r < 0$. We therefore find
\begin{equation}
	\int_{r_1}^{r_2} \frac{p^{t^*}}{|p^r|} \, dr \sim_\delta \int_{r_1}^{r_2} \frac{1}{\Osqern} \, dr \sim \left[ - \frac{m^2}{r-m} \right]_{r_1}^{r_2} = \frac{m^2}{r_1-m} - \frac{m^2}{r_2-m} \leq \frac{m}{\sqrt{\Osqern(r_1)}}.
\end{equation}
When $\trapschw < 0$ the geodesic must scatter off the photon sphere. We apply Lemma~\ref{prptstar_ern} as above and again rewrite $(r-2m)^2 + \trapschw \mathfrak{a} = (r-2m)^2 \left( 1 + \trapschw (r-m)^2 \mathfrak{b} \right) \sim_\delta m^2 \left( 1 + \trapschw (r-m)^2 \mathfrak{b} \right) = m^2 \left( 1 + \trapschw x^2 \mathfrak{b} \right)$ where we have introduced the change of coordinates $x=r-m$ and defined $\mathfrak{b}$ as above. We compute
\begin{equation}
	\int_{r_1}^{r_2} \frac{p^{t^*}}{|p^r|} \, dr \sim_\delta \int_{r_1-m}^{r_2-m} \frac{m^2}{x^2 \sqrt{1+\trapschw x^2 \mathfrak{b}}} \, dx \sim_\delta m \int_{y_1}^{y_2} \frac{1}{y^2 \sqrt{1+\trapschw y^2}} \, dy,
\end{equation}
where we have made the change of variables $y = x \sqrt{\mathfrak{b}(x)}$ and defined $y_i = (r_i-m) \sqrt{\mathfrak{b}(r_i-m)}$ for $i=1,2$. In the last step we have also made use of the fact that the Jacobian
\begin{equation}
    \left| \sqrt{\mathfrak{b}} \left( 1 + \frac{x}{2} \frac{\mathfrak{b}'}{\mathfrak{b}} \right) \right|^{-1} \sim_{\delta} m,
\end{equation}
as well as $m^2 \mathfrak{b} \sim_{\delta} 1$ as already shown above. We further compute
\begin{align}
	\int_{y_1}^{y_2} \frac{1}{y^2 \sqrt{1+\trapschw y^2}} \, dy &= \sqrt{\left| \trapschw \right|} \int_{\sqrt{\left| \trapschw \right|} y_1}^{\sqrt{\left| \trapschw \right|} y_2} \frac{1}{z^2 \sqrt{1-z^2}} \, dz = \sqrt{\left| \trapschw \right|} \left[ - \frac{\sqrt{1-z^2}}{z} \right]_{\sqrt{\left| \trapschw \right|}y_1}^{\sqrt{\left| \trapschw \right|}y_2} \\
	&\leq \sqrt{\left| \trapschw \right|} \frac{\sqrt{1-\left| \trapschw \right| y_1^2}}{\sqrt{\left| \trapschw \right|}y_1} \leq \frac{\sqrt{1-\left| \trapschw \right| y_1^2}}{y_1} \lesssim_{\delta} \frac{1}{\sqrt{\Osqern(r_1)}}.
\end{align}
Next note that since $\left| \trapern \right| \Osqern(r_{\text{min}}^-(\trapern)) \sim_\delta 1$ we find that if $\trapschw < 0$ is large enough in absolute value, we have $r_{\text{min}}^-(\trapern) \leq \min((1+\delta) m, \frac{3}{2}m)$. By definition, if we set $r_2 = r_{\text{min}}^-(\trapern)$ then $1 - \left| \trapern \right| y_2^2 = 0$. Therefore the integral in this case evaluates to
\begin{equation}
	\int_{y_1}^{y_2} \frac{1}{y^2 \sqrt{1+\trapschw y^2}} \, dy = \frac{\sqrt{1-\left| \trapschw \right| y_1^2}}{y_1}.
\end{equation}
Recall that $m^2 \mathfrak{b} \sim_\delta 1$. In fact one may easily see that the explicit bound $11 \leq m^2 \mathfrak{b} \leq 16$ holds for $m \leq r \leq \frac{3}{2} m$. If we now addition assume that $r_1 \leq \frac{1}{2}(r_2+m)$ then it follows that $r_1 -m \leq \frac{1}{2}(r_2-m)$ and furthermore that
\begin{equation}
	y_1 = (r_1-m) \sqrt{\mathfrak{b}(r_1-m)} \leq \frac{1}{2} \frac{\sqrt{16}}{\sqrt{11}} (r_2-m) \sqrt{\mathfrak{b}(r_2-m)} < \frac{7}{10} y_2 = \frac{7}{10} \frac{1}{\sqrt{\left| \trapern \right|}}.
\end{equation}
Therefore we find that $\sqrt{1-\left| \trapschw \right| y_1^2} \geq \frac{1}{2}$ and conclude that
\begin{equation}
	\int_{y_1}^{y_2} \frac{1}{y^2 \sqrt{1+\trapschw y^2}} \, dy \gtrsim \frac{1}{y_1}.
\end{equation}
This concludes the proof.
\end{proof}

\subsection{Upper bounds for the momentum support} \label{psupport_section_ERN}
In this section we give the proof of Proposition~\ref{psupport_propERN}, which is the extremal analogue of Proposition~\ref{psupport_prop} and establishes an upper bound on the size of the momentum support of a solution $f$ to the massless Vlasov equation as $\tau \rightarrow \infty$. Analogously to Section~\ref{section_psupport} above, before we prove Proposition~\ref{psupport_propERN}, we first establish a bound concerning energy and the trapping parameter as well as the simpler boundedness of the momentum support.

\subsubsection{Compactness of momentum support}
In this subsection we prove two results concerning the boundedness of the momentum support of a solution to the massless Vlasov equation with initially compact support, analogous to Section~\ref{sec_compmomsupport} above.

\begin{lem} \label{energy_trapparam_bounded_ern}
Assume that $f$ is a solution to the massless Vlasov equation with initial distribution $f_0 \in L^\infty(\mathcal{P}_0)$ that satisfies Assumption~\ref{assumption_support}. Then there exists a constant $C>0$ such that for all $(x,p) = (t^*,r,\omega, p^{t^*},p^r,\pslash) \in \supp(f) \subset \mathcal{P}$ with $\tau(x) \geq 0$ we have $E \left( 1 + \left| \trapern \right| \right) \leq C \frac{\rsuppconst^2}{m^2} \psuppconst$.
\end{lem}
\begin{proof}
We only provide an outline of the proof, since it proceeds entirely along the lines of the proof of the corresponding Lemma~\ref{E_eps_bound_schw} in the subextremal case. Since both $E$ and $\trapern$ are conserved quantities, it suffices to consider a point $(x,p) = (t^*,r,\omega,p^{t^*},p^r,\pslash) \in \supp(f_0)$. Inequality~\eqref{epsbound} implies that
\begin{equation}
	\left( 1 + \left| \trapern \right| \right) \lesssim \frac{r^2}{m^2} \frac{1}{\Osqern(r)} \leq \frac{\rsuppconst^2}{m^2} \frac{1}{\Osqern(r)}.
\end{equation}
We distinguish the two cases that $p^r > 0$ and $p^r \leq 0$. In case $p^r > 0$ we find from the explicit coordinate expression of $E$ that $E \leq \Osqern(r) \psuppconst$ so that we conclude the claimed bound. If $p^r \leq 0$ we use Lemma~\ref{express_ptstar} to find
\begin{equation}
    \frac{L^2}{E} \leq 2 r^2 p^{t^*}, \quad E \leq \psuppconst,
\end{equation}
which allows us to conclude
\begin{equation}
    E \left( 1 + \left| \trapern \right| \right) \lesssim E + \frac{1}{m^2} \frac{L^2}{E} \lesssim E + \frac{r^2}{m^2} p^{t^*}.
\end{equation}
By making use of Assumption~\ref{assumption_support} again we conclude the bound as claimed.
\end{proof}

\begin{lem} \label{psupport_boundedness_ERN}
Let $f$ be the solution to the massless Vlasov equation with initial data $f_0 \in L^\infty(\mathcal{P}_0)$ that satisfy Assumption~\ref{assumption_support}. Then there exists a dimensionless constant $C>0$ such that for all $(x,p) = (t^*,r,\omega, p^{t^*},p^r,\pslash) \in \supp(f) \subset \mathcal{P}$ with $\tau(x) \geq 0$, we have
\begin{equation}
	p^{t^*} \leq C \psuppconst, \quad \left| p^r \right| \leq \psuppconst, \quad \left| \pslash \right|_{\gslash} \leq \frac{\rsuppconst}{r} \psuppconst.
\end{equation}
\end{lem}
\begin{proof}
The structure of the argument is again identical to that of the proof of Lemma~\ref{psupport_bounded}, so that we will allow ourselves to omit certain details. Consider a future-directed maximally defined null geodesic $\gamma: [0,s_1) \rightarrow \Mschw$ with affine parameter $s$. We express $\gamma(s) = x(s) = (t^*(s),r(s),\omega(s))$ and $\dot{\gamma}(s) = p(s) = (p^{t^*}(s),p^r(s),\pslash(s))$ in $(t^*,r)$-coordinates and assume that $(x(0),p(0)) \in \supp(f_0)$. From conservation of energy~\eqref{Schw::energy_conservation} and the explicit coordinate expression of $E$ and $L$ we readily conclude that for all $s \geq 0$ the following bounds are satisfied
\begin{equation}
	\left| p^r(s) \right| \leq E \leq \psuppconst, \quad \left| \pslash(s) \right|_{\gslash} = \frac{L}{r(s)} \leq \frac{\rsuppconst}{r(s)} \psuppconst.
\end{equation}
To obtain a bound for the $p^{t^*}$-component, first observe that for $0 < \delta < 1$ and $r(s) \geq (1+\delta)m$ Lemma~\ref{ptstar_bounds_ern} implies the bound
\begin{equation}
    p^{t^*}(s) \lesssim_\delta E \leq \psuppconst.
\end{equation}
We may therefore assume that $m \leq r(s) \leq (1+\delta)m$. Consider first the case that $p^r(s) > 0$. Then for all $s' \in [0,s]$ we have $p^r(s') > 0$, which together with Lemma~\ref{express_ptstar} and the geodesic equations allows us to conclude $\frac{d}{ds} p^{t^*}(s') < 0$ for $s' \in (0,s)$. Therefore we find
\begin{equation}
    p^{t^*}(s) \leq p^{t^*}(0) \leq \psuppconst.
\end{equation}
Consider now the case that $p^r(s) \leq 0$. Let us denote by $\trapern_\delta$ the value of $\trapern$ for which $r_{\text{min}}^-(\trapern) = (1+\delta)m$. If we first assume that $\left| \trapern \right| \leq \left| \trapern_\delta \right|$ then Lemma~\ref{ptstar_bounds_ern} implies
\begin{equation}
    p^{t^*}(s) \leq (1 + \left| \trapern \right|) E \lesssim_\delta E \leq \psuppconst.
\end{equation}
Let us therefore assume that $\trapern < \trapern_\delta$, so that $m \leq r(s') \leq (1+\delta)m$ for all $s' \in [0,s]$. Let $s^* \in [0,s]$ be the smallest affine parameter such that $p^r(s^*) \leq 0$, so that $p^r(s') \leq 0$ for all $s' \in [s^*,s]$. Using equality~\eqref{express_ptstar_neg} evaluted at parameter times $s$ and $s^*$ we then conclude that
\begin{equation}
    p^{t^*}(s) \lesssim E + p^{t^*}(s^*).
\end{equation}
If $s^* = 0$, there is nothing left to show. Otherwise, we must have $p^r(s^*) = 0$, $p^r(s') > 0$ for $s' \in [0,s^*)$ and $r(s^*) = r_{\text{min}}^-(\trapern)$. We have shown above that in this case
\begin{equation}
    p^{t^*}(s^*) \leq p^{t^*}(0),
\end{equation}
which allows us to obtain the desired estimate.
\end{proof}

\subsubsection{Proof of Proposition~\ref{psupport_propERN}}
We now prove Proposition~\ref{psupport_propERN}. We remind the reader that the definitions of the sets $\trappedsupportset, \smallsupportset \subset \mathcal{P}$ involve a choice of constants $\bigc, \decayrate$ and $\tauzero$, while the definition of $\badsetaplarge$ involves the choice of a constant $\constbsl$, see Section~\ref{sec_subsets}.

%\begin{prop}
%Assume that $f_0 \in L^\infty(\mathcal{P}_0)$ satisfies Assumption~\ref{assumption_support} and let $f$ be the unique solution to the massless Vlasov equation such that $f|_{\Sigma_0} = f_0$. There exist dimensionless constants $C_0 > 0, \bigc, \decayrate > 0$ so that for any $(x,p) = (t^*,r,\omega,p^{t^*},p^r,\pslash) \in \supp(f)$ with $\tau(x) \geq \tauzero = C_0 \frac{\rsuppconst^2}{m}$ the following holds:
%\begin{equation} \label{eqn_prop51_refined}
%	(x,p) \in \trappedsupportset_{\bigc,\decayrate,\tauzero} \cup \smallsupportset_{\bigc,\decayrate,\tauzero}.
%\end{equation}
%
%Furthermore, for all $\bar{\tau} > \tauzero$  there exists $\bar{\delta} > 0$ with $\bar{\delta} \sim \bar{\tau}^{-1}$ such that
%\begin{equation} \label{pi0inclusion_proof}
%	\pi_0 \Big( (\smallsupportset \setminus \trappedsupportset) \cap \supp(f) \cap \left\{ (x,p) \in \mathcal{P} \; | \; \tau(x) \geq \bar{\tau} \right\} \Big) \subset \badsetaplarge_{\constbsl,\bar{\delta}},
%\end{equation}
%where we may choose the constant $\constbsl = 1$. Every point $(x,p) \in \badsetaplarge_{\constbsl,\bar{\delta}}$ satisfies Assumption~\ref{assumption_support}.
%\end{prop}
\begin{proof}[Proof of Proposition~\ref{psupport_propERN}]
Let $\gamma: [0,s] \rightarrow \Mern$ be an affinely parametrised future-directed null geodesic segment expressed in $(t^*,r)$-coordinates. Assume $(\gamma(0),\dot{\gamma}(0)) \in \supp(f_0)$ and let us for simplicity of notation suppress the dependence on $s$, so we write $(\gamma(s),\dot{\gamma}(s)) = (t^*,r,\omega,p^{t^*},p^r,\pslash)$ and $\tau(x(s)) = \tau$. Let us choose $\tauzero$ like in Lemma~\ref{tauestimate_ERN}. We will appeal repeatedly to Lemma~\ref{tauestimate_ERN}. Let $0 < \delta < 1$ and split the exterior in the region near the black hole, where $m \leq r \leq (1+\delta)m$, and the region far from the black hole, where $r \geq (1+\delta)m$. We distinguish between the two cases that $p^r \leq 0$ and $p^r > 0$ at time $\tau$. Each case will be broken down into further cases based on whether the geodesic is almost trapped at the event horizon or at the photon sphere.

%In each of these cases, we will begin by showing a bound for the $p^r$-component. Although the way in which we show the bound for $p^r$-component itself does not depend on its sign, the distinction based on the sign of $p^r$ becomes necessary when we next show a bound for the $p^{t^*}$-component. Finally we will show bounds for the angular momentum components, whose proof also does not depend on the sign of $p^r$.

\paragraph{Case 1: $p^r \leq 0$ at time $\tau$.}
In this case Lemma~\ref{tauestimate_ERN} allows us to conclude the bounds
\begin{equation} \label{Case1ERN}
	\frac{1}{m} \left( \tau - \tauzero \right) \lesssim \begin{cases}
		 \left( 1 + \left| \trapschw \right| \right) \sqrt{\Osqern(r(0))} + \mathfrak{s} \frac{1}{\sqrt{\Osqern(r(0))}} + \left( \log \left| \trapschw \right| \right)_-  & m \leq r \leq (1+\delta)m \\
		\mathfrak{s} \frac{1}{\sqrt{\Osqern(r(0))}} + \left( \log \left| \trapschw \right| \right)_-  & r \geq (1+\delta)m
	\end{cases},
\end{equation}
where we have defined $\mathfrak{s} = \chi_{(0,\infty)}\left( p^r(0) \right)$ as in the statement of the lemma. Similarly to the proof of Proposition~\ref{psupport_prop} above, this bound implies that we must be in one of the following three cases.

\paragraph{Case 1.1: $\gamma$ is almost trapped at the photon sphere.}
In this case we may assume $\left| \trapern \right| < 1$ and
\begin{equation}
	\frac{1}{m} \left( \tau - \tauzero \right) \lesssim_\delta \log \left| \left| \trapern \right| \right| \implies \left| \trapern \right| \leq e^{- \frac{\decayrateERN}{m} (\tau - \tauzero)},
\end{equation}
for an appropriate constant $c = c(\delta) > 0$ arising from the constant implicit in inequality~\eqref{Case1ERN}. In fact, if we assume that $\tau \geq \tauzero + C' m$ for a constant $C' > 0$ we find $\frac{1}{m} \left( \tau - \tauzero \right) \geq C'$ so that $\left| \trapern \right| \leq e^{- C' c}$. Choosing $C'$ suitably large we find that we may in fact assume $\left| \trapern \right| < \frac{1}{2}$ and consequently $\sqrt{\frac{1}{24}} \leq m \frac{E}{L} \leq \sqrt{\frac{1}{8}} < 1$. The argument in this case now proceeds virtually identically to the corresponding Case~1.1 in the proof of Proposition~\ref{psupport_prop} above. Using energy conservation we find
\begin{equation}
	(p^r)^2 = E^2 \trapern + \left( \frac{1}{16m^2} - \frac{\Osqern}{r^2} \right) L^2.
\end{equation}
Dividing both sides by $\frac{L^2}{m^2}$ and using the fact that for any $a,b > 0, \eta \geq 0$ the bound $\left| a^2 - b^2 \right| \leq \eta$  implies the bound $\left| a - b \right| \leq \sqrt{\eta}$ we find
\begin{equation} \label{prbound_ERN_proof}
	\left| m \frac{\left| p^r \right| }{L}  - \sqrt{ \frac{1}{16} - \frac{m^2}{r^2} \Osqern } \right| \leq \sqrt{\left| \trapern \right|} m \frac{E}{L} \leq \sqrt{\left| \trapern \right|} \leq e^{-\frac{\decayrateERN}{2m} (\tau - \tauzero)}.
\end{equation}
Just like in the proof of Proposition~\ref{psupport_prop} above we see that this rate of decay cannot be improved with our methods at the photon sphere. In any region where $\left| p^r \right| \sim E$ we can in fact improve the rate of decay to
\begin{equation}
	\left| m \frac{\left| p^r \right| }{L}  - \sqrt{ \frac{1}{16} - \frac{m^2}{r^2} \Osqern } \right| \leq e^{- \frac{\decayrateERN}{m} (\tau - \tauzero)}.
\end{equation}
In particular we can again conclude in an identical manner that if we choose $\delta' > 0$ and assume $r \notin [(2-\delta')m, (2+\delta')m]$ and $\tau \gtrsim \tauzero + \left| \ln \delta' \right|$ we have $\left| p^r \right| \sim E$ and can therefore obtain the better rate of decay in this region. \\

To obtain a bound on the $p^{t^*}$-component, we proceed in an entirely similar way to the subextremal case. Using Lemma~\ref{express_ptstar} we may express
\begin{equation}
	p^{t^*} =  \frac{- \left( 1 - \Osqern \right) \left| p^r \right| + \sqrt{(p^r)^2 + \frac{\Osqern}{r^2} L^2}}{\Osqern} =  \frac{- \left( 1 - \Osqern \right) \frac{\left| p^r \right|}{L} + \sqrt{\frac{(p^r)^2}{L^2} + \frac{\Osqern}{r^2}}}{\Osqern} L.
\end{equation}
If we first assume that $r \geq (1+\delta)m$ so that $\Osqern \sim_\delta 1$ and we may make use of the elementary bound $\left| \sqrt{a^2+x} - \sqrt{b^2+x} \right| \leq \left| a-b \right|$ for all $x \geq 0$ we find that
\begin{equation}
	\left| m \frac{p^{t^*}}{L} - \left( \frac{ \sqrt{\frac{1}{16}} - \left( 1 - \Osqern \right) \sqrt{ \frac{1}{16} - \frac{m^2}{r^2} \Osqern }}{\Osqern} \right) \right| \lesssim_\delta \left| m \frac{\left| p^r \right| }{L}  - \sqrt{ \frac{1}{16} - \frac{m^2}{r^2} \Osqern } \right|.
\end{equation}
Let us therefore consider the remaining case that $m \leq r \leq (1+ \delta)m$. We argue in a virtually identical manner to the Schwarzschild case and make use of Claim~\ref{lemma::F} shown in the proof of Lemma~\ref{psupport_prop} above. In this case we make the change of variables $x = \frac{r}{m}$ and let $\Psi = \Osqern$, noting again that $\Osqern$ may readily be expressed as a function of $x$. Then by construction
\begin{equation}
    F \left( m \frac{\left| p^r \right|}{L}, r \right) \frac{L}{m} = p^{t^*},
\end{equation}
and we conclude that
\begin{equation}
		\left| m \frac{p^{t^*}}{L} - \left( \frac{ \sqrt{\frac{1}{16}} - \left( 1 - \Osqern \right) \sqrt{ \frac{1}{16} - \frac{m^2}{r^2} \Osqern }}{\Osqern} \right) \right| = \left| F \left( m \frac{\left| p^r \right|}{L}, r \right) - F \left(\sqrt{ \frac{1}{16} - \frac{m^2}{r^2} \Osqern } , r \right) \right|.
\end{equation}
Now in order to apply the claim above, note that since $m \leq r \leq (1+\delta)m$,
\begin{equation}
	\sqrt{ \frac{1}{16} - \frac{m^2}{r^2} \Osqern } \geq \frac{1-\delta}{\sqrt{16}}.
\end{equation}
Recall that we assume $\tau \geq \tauzero + C' m$ with some suitably large constant $C' > 0$. Repeating the argument from above, we see that we may in fact assume that $\left| \trapern \right| < \left( \frac{1- \delta}{2 \sqrt{16}} \right)^2$ by choosing $C' = C'(\delta)$ larger if necessary. Together with inequality~\eqref{prbound_ERN_proof} this then implies
\begin{equation}
	m \frac{\left| p^r \right|}{L} \geq  \frac{1-\delta}{2 \sqrt{16}} > 0.
\end{equation}
Using the results of Claim~\ref{lemma::F}  we may conclude
\begin{equation}
	\left| F \left( m \frac{\left| p^r \right|}{L}, r \right) - F \left(\sqrt{ \frac{1}{16} - \frac{m^2}{r^2} \Osqern } , r \right) \right| \lesssim_\delta \left| m \frac{\left| p^r \right| }{L}  - \sqrt{ \frac{1}{16} - \frac{m^2}{r^2} \Osqern } \right|.
\end{equation}
Therefore we have shown that this bound in fact holds for all $r \geq m$ and we conclude the desired bound
\begin{equation}
		\left| m \frac{p^{t^*}}{L} - \left( \frac{ \sqrt{\frac{1}{16}} - \left( 1 - \Osqern \right) \sqrt{ \frac{1}{16} - \frac{m^2}{r^2} \Osqern }}{\Osqern} \right) \right| \lesssim_\delta e^{-\frac{\decayrateERN}{2m} (\tau-\tauzero)},
\end{equation}
where we note that the same remarks about the rate of decay apply here as for the $p^r$-component. It only remains to obtain a bound on the angular momentum component. By Lemma~\ref{psupport_boundedness_ERN} we know that $L \leq \rsuppconst \psuppconst$ so that
\begin{equation}
	\left| \pslash \right|_{\gslash}  = \frac{L}{r} \leq \frac{\rsuppconst}{r} \psuppconst.
\end{equation}


\paragraph{Case 1.2: $\gamma$ is initially outgoing and starts close to $\mathcal{H}^+$.}
In this case we may assume that $p^r(0) > 0$ and
\begin{equation} \label{eqn_case12_ern}
	\frac{1}{m} (\tau - \tauzero) \lesssim_\delta \frac{1}{\sqrt{\Osqern(r(0))}} \implies \Osqern(r(0)) \lesssim_\delta \frac{m^2}{(\tau-\tauzero)^2}.
\end{equation}
If we assume that $\tau \geq \tauzero + C' m$ for a constant $C' > 0$ then $\Osqern(r(0)) \leq \frac{1}{(C')^2}$, in particular we may assume $C' > 2$ so that $r(0) < 2m$. From the definition of energy we find
\begin{equation}
	E = \Osqern(r(0)) p^{t^*}(0) - \left(1 - \Osqern(r(0)) \right) p^r(0) \leq \Osqern(r(0)) \psuppconst \lesssim_\delta \frac{m^2}{(\tau-\tauzero)^2} \psuppconst,
\end{equation}
where we have used the fact that $p^r(0) > 0$. Conservation of energy~\eqref{Schw::energy_conservation} immediately implies
\begin{equation}
	\left| p^r \right| \leq E  \lesssim_\delta \frac{m^2}{(\tau-\tauzero)^2} \psuppconst.
\end{equation}
To obtain a bound on the $p^{t^*}$ component apply Lemma~\ref{ptstar_bounds_ern} noting that $p^r \leq 0$ to find
\begin{equation}
	p^{t^*} \lesssim \left(1 + \frac{m^2}{r^2} \left| \trapern \right| \right) E \leq  \left(1 + \left| \trapern \right| \right) \Osqern(r(0)) \psuppconst \leq \psuppconst.
\end{equation}
Note carefully that the bound we have shown for the energy $E$ only depends on our choice of $\delta$. By making use of Lemma~\ref{ptstar_bounds_ern} again we can conclude the bound
\begin{equation}
	p^{t^*} \lesssim \frac{E}{\Osqern}  \lesssim_\delta  \frac{1}{\Osqern} \frac{m^2}{(\tau-\tauzero)^2} \psuppconst.
\end{equation}
We see that one can obtain a decay estimate (with a constant that degenerates as $r \rightarrow m$) at an arbitrarily small distance to the event horizon. We claim however that the bound $p^{t^*} \lesssim \psuppconst$ is generically sharp up to a constant multiple along the event horizon, see Proposition~\ref{sharpness_lemma} below for a discussion. Finally to obtain a bound on the angular momentum components, we find from conservation of energy that
\begin{equation}
	\frac{\Osqern(r(0))}{r(0)^2} L^2 \leq E^2 \leq \Osqern(r(0))^2 \psuppconst^2,
\end{equation}
so that we find
\begin{equation} \label{eqn_case12_L_ern}
	 \left| \pslash \right|_{\gslash} = \frac{L}{r} \leq \sqrt{\Osqern(r(0))} \frac{r(0)}{r} \psuppconst \lesssim_\delta \frac{m}{\tau - \tauzero} \frac{\rsuppconst}{r} \psuppconst.
\end{equation}
Now assume that $\tau(x) > \bar{\tau} > \tauzero$. To show that $(x(0),p(0)) \in \badsetaplarge_{\bar{\delta}}$ for $\bar{\delta} = \bar{\tau}^{-1}$, first note that from~\eqref{eqn_case12_ern} it follows that $r(0) \leq (1+\bar{\delta})m$ if we assume $\bar{\tau}$ is large enough. Next note that since $p^r(0) > 0$, it directly follows that
\begin{equation}
	p^r(0) \leq E \leq \Osqern(r(0)) \psuppconst.
\end{equation}
To bound the angular component, we argue precisely as in~\eqref{eqn_case12_L_ern} to find that
\begin{equation}
	\left| \pslash(0) \right|_{\gslash} \leq  \sqrt{\Osqern(r(0))} \psuppconst.
\end{equation}

\paragraph{Case 1.3: $\gamma$ is slowly infalling.}
Note carefully that in the subextremal case discussed above, there were only the analogues of Cases~1.1 and 1.2 just discussed. This third case is unique to extremal black holes, although the proof proceeds in an almost identical manner to Case~1.2 above. Here we may assume that $m \leq r \leq (1+\delta)m$ and we have the bound
\begin{equation} \label{eqn_case13_timebound}
	\frac{1}{m}(\tau - \tauzero) \lesssim_\delta \left( 1 + \left| \trapern \right| \right) \sqrt{\Osqern(r(0))}.
\end{equation}
Using the bound $\left( 1 + \left| \trapern \right| \right) \Osqern(r(0)) \lesssim 1$ we conclude
\begin{equation} \label{eqn_case13_ern}
	\frac{1}{m}(\tau - \tauzero) \lesssim \sqrt{ 1 + \left| \trapern \right|} \implies \Osqern(r(0)) \lesssim \frac{1}{1 + \left| \trapern \right|} \lesssim \frac{m^2}{(\tau - \tauzero)^2}.
\end{equation}
Using Lemma~\ref{energy_trapparam_bounded_ern} we find for the energy
\begin{equation}
	E \lesssim \frac{\psuppconst}{1 + \left| \trapern \right|} \lesssim \psuppconst \frac{m^2}{(\tau - \tauzero)^2}.
\end{equation}
Using conservation of energy~\eqref{Schw::energy_conservation} again we readily conclude
\begin{equation}
	\left| p^r \right| \leq E \lesssim \frac{m^2}{(\tau - \tauzero)^2} \psuppconst.
\end{equation}
To bound the $p^{t^*}$-component, we proceed precisely as in Case~1.2 above and apply Lemma~\ref{express_ptstar} together with Lemma~\ref{energy_trapparam_bounded_ern} to find that the bound
\begin{equation}
	p^{t^*} \lesssim \psuppconst
\end{equation}
holds for all $r \geq m$, while for $r > m$ we may also conclude the alternative bound
\begin{equation}
	p^{t^*} \lesssim_\delta  \frac{1}{\Osqern} \frac{m^2}{(\tau - \tauzero)^2} \psuppconst.
\end{equation}
Like in Case~1.2 we claim that the bound $p^{t^*} \lesssim \psuppconst$ is sharp up to a constant for generic initial data along the event horizon and refer to Section~\ref{psupporteventhorizon} for a discussion. To obtain a bound on the angular momentum components, we apply Lemma~\ref{energy_trapparam_bounded_ern} together with the definition of the trapping parameter~\ref{def_eps_def} to find
\begin{equation} \label{eqn_case13_L_ern}
	\frac{L^2}{m^2} \lesssim \left( 1 + \left| \trapern \right| \right) E^2 \lesssim E \psuppconst \lesssim \frac{m^2}{(\tau - \tauzero)^2} \psuppconst^2.
\end{equation}
From the definition of angular momentum we therefore find
\begin{equation}
	\left| \pslash \right|_{\gslash} = \frac{L}{r} \lesssim_\delta \frac{m}{\tau-\tauzero} \frac{m}{r} \psuppconst.
\end{equation}
Now assume that $\tau(x) > \bar{\tau} > \tauzero$. In order to prove that $(x(0),p(0)) \in \badsetaplarge_{\bar{\delta}}$ with $\bar{\delta} = \bar{\tau}^{-1}$, note again that from~\eqref{eqn_case13_ern} it follows $r(0) \leq (1+\bar{\delta})m$ if $\bar{\tau}$ is large enough. In order to bound the $p^r(0)$-component, apply Lemma~\ref{energy_trapparam_bounded_ern} combined with~\eqref{eqn_case13_timebound} to find
\begin{equation}
	\left| p^r(0) \right| \leq E \lesssim \frac{1}{1+\left| \trapschw \right|} \psuppconst \lesssim \sqrt{\Osqern(r(0))} \frac{m}{\tau-\tauzero} \psuppconst.
\end{equation}
If we assume that $\bar{\tau}$ is large enough, we may conclude
\begin{equation}
	\left| p^r(0) \right| \leq  \sqrt{\Osqern(r(0))} \psuppconst.
\end{equation}
To bound the angular component over $\Sigma_0$, we make use of~\eqref{eqn_case13_L_ern} to find
\begin{equation}
	\left| \pslash(0) \right|_{\gslash}^2 \lesssim E \psuppconst \lesssim \sqrt{\Osqern(r(0))} \frac{m}{\tau-\tauzero} \psuppconst^2.
\end{equation}
If we again assume $\bar{\tau}$ to be large enough, we can conclude
\begin{equation}
	\left| \pslash(0) \right|_{\gslash} \leq (\Osqern(r(0)))^{\frac{1}{4}} \psuppconst.
\end{equation}

\paragraph{Case 2: $p^r > 0$ at time $\tau$.}
In this case we assume that $p^r > 0$ at time $\tau$ and from Lemma~\ref{tauestimate_ERN} we conclude the bound
\begin{equation} \label{Case2ERN}
	\frac{1}{m}(\tau-\tauzero) \lesssim_\delta \begin{cases}
		\mathfrak{s} \frac{1}{\sqrt{\Osqern(r(0))}} & m \leq r \leq (1+\delta)m \\
		\mathfrak{s} \frac{1}{\sqrt{\Osqern(r(0))}} + \left( \log \left| \trapern \right| \right)_- & r \geq (1+\delta)m
	\end{cases},
\end{equation}
where $\mathfrak{s} = \chi_{(0,\infty)}(p^r(0))$. Like in the Schwarzschild case, we therefore distinguish two cases.

\paragraph{Case 2.1: $\gamma$ is almost trapped at the photon sphere.}
In this case we may assume that $r \geq (1+\delta)m$ and $\left| \trapern \right| < 1$ and we have the bound
\begin{equation}
	\frac{1}{m} \left( \tau - \tauzero \right) \lesssim_\delta \log \left| \left| \trapern \right| \right| \implies \left| \trapern \right| \leq e^{-\frac{\decayrateERN}{m} (\tau - \tauzero)},
\end{equation}
for an appropriate constant $c = c(\delta) > 0$ arising from the constant implicit in inequality~\eqref{Case2ERN}. Proceeding exactly as above in Case~1.1, we see that if we assume $\tau \gtrsim \tauzero + 1$ with a large enough constant, we may assume $m \frac{E}{L} \leq \frac{1}{2}$. In an identical way to Case~1.1 above we conclude
\begin{equation}
	\left| m \frac{\left| p^r \right| }{L}  - \sqrt{ \frac{1}{16} - \frac{m^2}{r^2} \Osqern } \right| \leq e^{-\frac{\decayrateERN}{2m} (\tau - \tauzero)},
\end{equation}
where again the same remarks as in Case~1.1 apply about improving the rate of decay away from the photon sphere. To obtain a bound on the $p^{t^*}$-component we apply Lemma~\ref{express_ptstar}, noting that $p^r > 0$ and find
\begin{equation}
	p^{t^*} = \frac{\left( 1 - \Osqern \right) \frac{\left| p^r \right|}{L} + \sqrt{\frac{(p^r)^2}{L^2} + \frac{\Osqern}{r^2}}}{\Osqern} L.
\end{equation}
Note that we may assume $r \geq (1+\delta)m$ so that $\Osqern \sim_\delta 1$. Dividing both sides by $\frac{m}{L}$ and applying the bound $\left| \sqrt{a^2+x} - \sqrt{b^2+x} \right| \leq \left| a-b \right|$, which holds for all $x \geq 0$, we readily find
\begin{equation}
	\left| m \frac{p^{t^*}}{L} - \left( \frac{\left( 1 - \Osqern \right) \sqrt{ \frac{1}{16} - \frac{m^2}{r^2} \Osqern } + \sqrt{\frac{1}{16}}}{\Osqern} \right) \right| \lesssim_\delta \left| m \frac{\left| p^r \right| }{L}  - \sqrt{ \frac{1}{16} - \frac{m^2}{r^2} \Osqern } \right|.
\end{equation}
In particular, the $p^{t^*}$-component decays at the same rate as the $p^r$-component. Finally we obtain a bound on the angular momentum component in an identical manner to Case~1.1 above.


\paragraph{Case 2.2: $\gamma$ is initially outgoing and starts close to $\mathcal{H}^+$.}
In this case we may assume that $p^r(0) > 0$ and we have
\begin{equation}
	\frac{1}{m}(\tau-\tauzero) \lesssim_\delta \frac{1}{\sqrt{\Osqern(r(0))}} \implies \Osqern(r(0)) \lesssim_\delta \frac{m^2}{(\tau-\tauzero)^2}.
\end{equation}
We initially proceed in an identical manner to Case~1.2 above and note that by assuming $\tau \gtrsim \tauzero + m$ with a large enough constant, we may assume $r(0) < 2m$. As above we conclude
\begin{equation}
	\left| p^r \right| \leq E \leq \Osqern(r(0)) p^{t^*}(0) \lesssim_\delta \frac{m^2}{(\tau-\tauzero)^2} \psuppconst.
\end{equation}
To obtain a bound on the $p^{t^*}$-component let us first assume $r \geq m$ and apply Lemma~\ref{ptstar_bounds_ern} using the fact that $p^r > 0$ at time $\tau$ to find
\begin{equation}
	p^{t^*} \lesssim \frac{E}{\Osqern(r)} \leq \frac{\Osqern(r(0))}{\Osqern(r)} \psuppconst \leq \psuppconst,
\end{equation}
where in the last step we have used that $r \geq r(0)$, since $p^r = p^r(s) > 0$ implies that $p^r(s') > 0$ for all parameter times $s' \in [0, s]$. Of course we have already obtained this bound in Lemma~\ref{psupport_boundedness_ERN} above. If we assume $r > m$ by Lemma~\ref{ptstar_bounds_ern} we may conclude the better bound
\begin{equation}
	p^{t^*} \lesssim \frac{E}{\Osqern} \lesssim_\delta  \frac{1}{\Osqern} \frac{m^2}{(\tau-\tauzero)^2}.
\end{equation}
As in Case~1.2 above we claim that the bound $p^{t^*} \lesssim \psuppconst$ is sharp up to a constant along the event horizon and defer to Section~\ref{psupporteventhorizon} for a discussion. We obtain a bound on the angular momentum components in an identical manner to Case~1.2. Finally, the proof that $(x(0),p(0)) \in \badsetaplarge_{\bar{\delta}}$ with $\bar{\delta} = \bar{\tau}^{-1}$ if $\tau(x) > \bar{\tau} > \tauzero$ proceeds in an identical fashion to Case~1.2 above.
\end{proof}


\subsection{Lower bounds for the momentum support} \label{psupporteventhorizon}
In this section we will study the momentum support of a solution along the event horizon and provide the proof of Proposition~\ref{sharpness_prop_rough}. In Proposition~\ref{psupport_propERN} above we have shown that for any $(x,p) \in \ERNsmallsupportset$ the $p^{t^*}$-component satisfies the bound
\begin{equation}
	p^{t^*} \leq \bigc \psuppconst \min \left( \frac{1}{\Osqern} \frac{m^2}{(\tau(x)-\tauzero)^2}, 1 \right).
\end{equation}
Therefore, we have shown that at any positive distance from the event horizon, i.e. $r \geq (1+\delta)m$ for some $\delta > 0$, the $p^{t^*}$-component decays as $\tau(x) \rightarrow \infty$, with a constant that degenerates like $\delta^{-2}$ as $\delta \rightarrow 0$. We will now establish the existence of a family of geodesics with initial data in the sets $\badsetapprox_\delta$ and which eventually cross the event horizon at arbitrarily late times $\tau$, while their momentum satisfies $p^{t^*} \sim \psuppconst$ on the event horizon. \\

We break up Proposition~\ref{sharpness_prop_rough} into three separate results: Propositions~\ref{properties_of_badsetapprox}-\ref{ptstar_size_slowsupport}. Proposition~\ref{properties_of_badsetapprox} establishes some basic properties of the family $\badsetapprox_{c_1,c_2,\delta}$. Then we prove in Proposition~\ref{sharpness_lemma} that the sets $\badsetapprox_\delta$ and $\badsettau$ are included in one another if we choose $\tauslow \sim m \delta^{-1}$ and make an appropriate choice for all other constants involved in the definition. Roughly speaking, this implies that geodesics with initial data in $\badsetapprox_\delta$ populate the set $\slowsupportset$ along the event horizon. Proposition~\ref{sharpness_lemma} also shows that all defining constants may be chosen in such a way that every point $(x,p) \in \badsetapprox_\delta$ satisfies Assumption~\ref{assumption_support}. Finally, we estimate the volume of the set $\slowsupportsetx$ along the event horizon in Proposition~\ref{ptstar_size_slowsupport}. Taken together, these propositions constitute the proof of Proposition~\ref{sharpness_prop_rough}.

\subsubsection{Properties of $\badsetapprox$}
In this subsection we prove a quantitative estimate for the volume of the set $\badsetapprox_{\delta}$ and establish that every $p = (p^{t^*},p^r,\pslash) \in \badsetapprox_{\delta}$ satisfies $p^{t^*} \sim \psuppconst$.

\begin{prop}[Properties of $\badsetapprox$] \label{properties_of_badsetapprox}
Let $0 < c_1<c_2$ and $0< \delta < \frac{1}{2}$. Then every $(x,p) \in \badsetapprox_{c_1,c_2,\delta}$ expressed in $(t^*,r)$-coordinates as $(x,p) = (t^*,r,\omega,p^{t^*},p^r,\pslash)$ satisfies $p^{t^*} \sim_{c_1,c_2} \psuppconst$. Furthermore $\vol \badsetapprox_{c_1,c_2,\delta} \sim_{c_1,c_2} \delta^3 m^3 \psuppconst^2$.
\end{prop}
\begin{proof}
Let $(x,p) = (t^*,r,\omega,p^{t^*},p^r,\pslash) \in \badsetapprox$. We first obtain a bound on energy and angular momentum by using conservation of energy to find
\begin{gather}
    c_1 \Osqern \psuppconst \leq E = \sqrt{(p^r)^2 + \Osqern \left| \pslash \right|_{\gslash}^2} \leq \sqrt{2} c_2 \Osqern \psuppconst, \label{proof_badsetapprox_1} \\
    c_1 \sqrt{\Osqern} \psuppconst \leq \left| \pslash \right|_{\gslash} \leq \frac{L}{m} \leq 2 \left| \pslash \right|_{\gslash} \leq 2 c_2 \sqrt{\Osqern} \psuppconst. \label{proof_badsetapprox_2}
\end{gather}
In order to obtain a bound on $p^{t^*}$ we need to distinguish the two cases that $p^r > 0$ and $p^r \leq 0$. Let us turn to the case that $p^r \leq 0$ first. We apply Lemma~\ref{express_ptstar} to find
\begin{equation} \label{ptstar_bound_equn_proof_badset}
    \frac{1}{4} \frac{c_1^2}{c_2} \psuppconst \leq p^{t^*} = \left| p^r \right| + \frac{\left| \pslash \right|_{\gslash}^2}{E + \left| p^r \right|} \leq \left( c_2 \delta^2 + \frac{c_2^2}{c_1} \right) \psuppconst,
\end{equation}
where we made use of the fact that $D(r) \leq \delta^2$. Let us now turn to the remaining case that $p^r > 0$. In this case Lemma~\ref{express_ptstar} allows us to conclude
\begin{equation}
    c_1 \psuppconst \leq \frac{E}{\Osqern} \leq p^{t^*} = \frac{E + \left[ 1-\Osqern \right] p^r}{\Osqern} \leq \frac{2 E}{\Osqern} \leq 4 c_2 \psuppconst.
\end{equation}
Next fix a point $x \in \Sigma_0$ with $m \leq r \leq (1+\delta)m$ and let us compute $\vol \badsetapprox_x = \badsetapprox \cap \mathcal{P}_x$. We parameterise the null-cone by using the coordinates induced by $(t^*,r)$-coordinates and eliminating the $p^{t^*}$-component of the momentum, as discussed in Section~\ref{sec_parametrising_massshell}. As in the proof of Lemma~\ref{lem_boundedness_moments_schw} we introduce a change of variables $\pslash \mapsto (p^1,p^2)$ with the property that $L^2 = r^2 \left| \pslash \right|_{\gslash}^2 = (p^1)^2 + (p^2)^2$. We may explicitly realise these variables by considering spherical coordinates on the sphere $\sphere$ in an identical manner to the proof of Lemma~\eqref{lem_boundedness_moments_schw} above. From equation~\eqref{dmu_first_comp_schw} after applying our change of variable we find
\begin{equation}
	d \mu_x = \frac{r^2 \sin \theta}{\Osqern p^{t^*} - \left( 1 - \Osqern \right) p^r} \, d p^r d p^\theta d p^\phi = \frac{1}{r^2} \frac{1}{E} \, d p^r d p^1 d p^2 .
\end{equation}
Next we introduce a further change of coordinates $(p^1,p^2) \mapsto (L,\angle (p^1,p^2) ) =: (L,\pslashangle)$, or in other words we use radial coordinates in the angular variables $(p^1,p^2)$. Note that $\pslashangle \in [0,2 \pi)$ so that we find



\begin{equation}
    \vol \badsetapprox_x = \int_{\badsetapprox_x} 1 \, d \mu_x \sim \int_{c_1 m \psuppconst \sqrt{\Osqern(r)}}^{c_2 m \psuppconst \sqrt{\Osqern(r)}} \int_{0}^{c_2 \psuppconst \Osqern(r)} \frac{1}{m^2} \frac{L}{E} \, d p^r d L.
\end{equation}
According to~\eqref{proof_badsetapprox_1} and~\eqref{proof_badsetapprox_2} above we have that in the set $\badsetapprox_x$ the bound
\begin{equation}
    \frac{1}{m} \frac{L}{E} \sim_{c_1,c_2} \frac{1}{\sqrt{\Osqern}}
\end{equation}
holds, so that we may readily compute
\begin{equation}
    \int_{\badsetapprox_x} 1 \, d \mu_x \sim_{c_1,c_2} \Osqern \psuppconst^2.
\end{equation}
Recalling that the hypersurface $\Sigma_0$ coincides with the set $\{ t^* = 0 \}$ when $r \leq (1+\delta)m$, we find that
\begin{equation}
    \vol \badsetapprox = \int_{m}^{(1+\delta)m} \int_{\sphere} \int_{\badsetapprox_x} r^2 \, d \mu_x d \omega dr \sim_{c_1,c_2} m^2 \psuppconst^2 \int_{m}^{(1+\delta)m} \Osqern \, dr \sim_{c_1,c_2} \delta^3 m^3 \psuppconst^2.
\end{equation}
This concludes the proof.
\end{proof}

\subsubsection{$\badset$ and $\badsetapprox$ are comparable}

We next show that $\badsettau \approx \badsetapprox_{\delta}$ for any $0 < \delta < \frac{1}{2}$, if we choose $\tauslow \sim \delta^{-1}$ and all remaining constants in the definition are chosen appropriately.

\begin{prop}[$\badsettau$ and $\badsetapprox_\delta$ are comparable] \label{sharpness_lemma}
There exist constants $C_1,C_2, \ubar{c}_1, \ubar{c}_2, \bar{c}_1, \bar{c}_2 > 0$ independent of $\psuppconst,\rsuppconst$ such that for any $0 < \bar{\delta} < \frac{1}{2}$, there exist constants $\tauslow > m$ and $0< \ubar{\delta} < \bar{\delta}$ such that $\tauslow \sim \bar{\delta}^{-1}$ and the following chain of inclusions holds
\begin{equation}
    \badsetapprox_{\ubar{c}_1, \ubar{c}_2, \ubar{\delta}} \subset \badsetallconst \subset \badsetapprox_{\bar{c}_1, \bar{c}_2, \bar{\delta}}.
\end{equation}
Furthermore each $(x,p) \in \badsetapprox_{\bar{c}_1, \bar{c}_2, \bar{\delta}}$ satisfies Assumption~\ref{assumption_support} and $\badsetapprox_{\bar{c}_1, \bar{c}_2, \bar{\delta}} \subset \badsetaplarge_{\bar{\delta}}$ for every $0 < \bar{\delta} < \frac{1}{2}$.
\end{prop}
\begin{proof}
We first show the existence of constants such that $\badsetallconst \subset \badsetapprox_{c_1,c_2,\delta}$, and then we show that the reverse inequality holds for for an appropriate and compatible choice of constants. For simplicity we will often write $\badsettau = \badsetallconst$ and $\badsetapprox_{\delta} = \badsetapprox_{c_1,c_2,\delta}$ when no confusion can arise.

\paragraph{Step 1: The inclusion $\badset \subset \badsetapprox$.}
We begin by showing the existence of constants such that $\badsetallconst \subset \badsetapprox_{c_1,c_2,\delta}$. Let $0 < \delta < \frac{1}{2}$ and $\gamma: [0,s] \rightarrow \Mern$ be an affinely parametrised future-directed null geodesic segment expressed in $(t^*,r)$-coordinates. We will denote $\gamma(s') = x(s')$ and $\dot{\gamma}(s') = p(s')$ for $s' \in [0,s]$. Assume $(x(0),p(0)) \in \badsettau$ and that we have chosen the parameter time $s$ such that $x(s) \in \mathcal{H^+}$. We now aim to derive bounds on $r(0)$ and the components of $p(0)$. Let us begin by deriving estimates on the energy, angular momentum and trapping parameter of the geodesic. As usual we abbreviate $E = E(\gamma)$ and $L = L(\gamma)$. \\


Recall that the relation $E = - p^r(s) = \left| p^r(s) \right|$ is implied by conservation of energy along the event horizon $\mathcal{H}^+$. Using the definition of $\slowsupportset$ and this relation we therefore immediately obtain the bounds
\begin{equation}
\begin{gathered}
    C_1 \frac{\psuppconst}{\tau(x(s))^2} \leq E \leq C_2 \frac{\psuppconst}{\tau(x(s))^2}, \\
    C_1 \frac{\psuppconst}{\tau(x(s))} \leq \frac{L}{m} \leq C_2 \frac{\psuppconst}{\tau(x(s))},
\end{gathered}
\end{equation}
We conclude further that
\begin{equation} \label{proof_slow_epsestimate}
    \left( \frac{C_1}{C_2} \right)^2 \tau(x(s))^2 \leq \frac{1}{m^2} \frac{L^2}{E^2} \leq \left( \frac{C_2}{C_1} \right)^2 \tau(x(s))^2.
\end{equation}
Note that by definition of $\slowsupportset$, we must have $\tau(x(s)) \geq \tauslow$. If we assume that $\tauslow^2 > 16 \left( \frac{C_2}{C_1} \right)^2 (1+\delta^{-2})$ then inequality~\eqref{proof_slow_epsestimate} allows us to conclude $\trapern < - \delta^{-2}$. Let us henceforth assume that $\tauslow$ satisfies this bound. We will later choose the constants $C_1,C_2$ independently of $\delta$. \\

We next want to obtain a relationship between the time of arrival at the event horizon $\tau(x(s))$ and the values $r(0)$ and $p(0)$. Combined with our bound on the trapping parameter $\trapern$ this will allow us to conclude a bound on $\Osqern(r(0))$. First note that since $\trapern < 0$ the geodesic must certainly remain inside the photon sphere for all times, so $m \leq r(s') < 2m$ for $s' \in [0,s]$. In fact, from inequality~\eqref{epsbound} it immediately follows that for all times
\begin{equation}
	\Osqern(r(s')) \leq \frac{1}{1 + \left| \trapern \right|} \frac{r(s')^2}{16m^2} \leq  \frac{1}{1 + \left| \trapern \right|} \frac{1}{4} < \delta^{-2},
\end{equation}
where we have used that $r(s') < 2m$ for all $s' \in [0,s]$ and our assumption on $\tauslow$. Therefore
\begin{equation}
    m < r(s') < (1+\delta)m
\end{equation}
for all affine parameter times $s' \in [0,s]$. In particular it follows that $m < r(0) < (1+\delta)m < \frac{3}{2}m$. Next note that since $r(s') < \frac{3}{2}m$ for all times, we have $\tau(x(s')) = t^*(s')$ and we may apply the results of Lemma~\ref{tstar_estimate_ern} directly. We need to distinguish between the two cases that $p^r(0) > 0$ and $p^r(0) \leq 0$. Let us first assume that $p^r(0) \leq 0$, so that $p^r(s') < 0$ for all $s' \in [0,s]$. Therefore we may apply Lemma~\ref{tstar_estimate_ern} with $\delta_1 = \frac{1}{2}$ and find
\begin{equation}
    \tau(x(s)) = \tau(x(s)) - \tau(x(0)) \sim \left( 1 + \left| \trapern \right| \right) \sqrt{\Osqern(r(0))}.
\end{equation}
Using inequality~\eqref{proof_slow_epsestimate} combined with the fact that $1 + \left| \trapern \right| = \frac{1}{16m^2} \frac{L^2}{E^2}$ we may now conclude the existence of constants $\tilde{C}_4 > \tilde{C}_3 > 0$ such that
\begin{equation}
    \tilde{C}_3 \left( \frac{C_1}{C_2} \right)^4 \frac{1}{\tau(x(s))^2} \leq \Osqern(r(0)) \leq \tilde{C}_4 \left( \frac{C_2}{C_1} \right)^4 \frac{1}{\tau(x(s))^2}.
\end{equation}
Let us now turn to the remaining case that $p^r(0) > 0$. The geodesic must necessarily be initially outgoing, then scatter off the photon sphere and eventually fall into the black hole. Let us denote by $s^* < 0$ the affine time parameter for which $r(s^*) = r_{\text{min}}^-(\trapern)$ or equivalently $p^r(s^*) = 0$. Note that $0 < s^*$ and $r(0) < r_{\text{min}}^-(\trapern)$. In order to obtain a bound $\Osqern(r(0))$ let us now distinguish further between the two cases that $r(0) < \frac{1}{2} (r_{\text{min}}^-(\trapern) + m)$ and $r(0) \geq \frac{1}{2} (r_{\text{min}}^-(\trapern) + m)$. In the latter case, we may immediately conclude
\begin{equation}
    \Osqern(r(0)) \sim \Osqern(r_{\text{min}}^-(\trapern)) \sim \frac{1}{1+\left| \trapern \right|} \sim \frac{1}{\tau(x(s))^2},
\end{equation}
where we have made use of inequality~\eqref{proof_slow_epsestimate}. In the former case that $r(0) < \frac{1}{2} (r_{\text{min}}^-(\trapern) + m)$ an application of Lemma~\ref{tstar_estimate_ern} with $\delta_1 = \frac{1}{2}$ yields
\begin{equation}
\begin{gathered}
	\tau(x(s^*)) = \tau(x(s^*)) - \tau(x(0)) \sim \frac{1}{\sqrt{\Osqern(r(0))}}, \\
	\tau(x(s)) - \tau(x(s^*)) \sim \left( 1 + \left| \trapern \right| \right) \sqrt{\Osqern(r(0))}.
\end{gathered}
\end{equation}
Recall that $\left( 1 + \left| \trapern \right| \right) \Osqern(r(0)) \lesssim 1$ so that
\begin{equation}
    \tau(x(s)) \sim \frac{1}{\sqrt{\Osqern(r(0))}} + \left( 1 + \left| \trapern \right| \right) \sqrt{\Osqern(r(0))} \sim \frac{1}{\sqrt{\Osqern(r(0))}}.
\end{equation}
This allows us to conclude the existence of constants $\bar{C}_3,\bar{C}_4$ such that the bound
\begin{equation}
    \bar{C}_3 \frac{1}{\tau(x(s))^2} \leq \Osqern(r(0)) \leq \bar{C}_4  \frac{1}{\tau(x(s))^2}
\end{equation}
holds. By choosing $C_3 = \min(\bar{C}_3, \tilde{C}_3)$ and $C_4 = \max(\bar{C}_4, \tilde{C}_4)$ and noting that $C_1 < C_2$ by assumption we find that the inequality
\begin{equation}
    C_3 \left( \frac{C_1}{C_2} \right)^4 \frac{1}{\tau(x(s))^2} \leq \Osqern(r(0)) \leq C_4 \left( \frac{C_2}{C_1} \right)^4 \frac{1}{\tau(x(s))^2}
\end{equation}
holds for both signs of $p^r(0)$. We may without loss of generality assume that $C_3 < \frac{1}{8}$. \\

Finally we turn to deriving bounds on $p(0)$. We may immediately derive bounds on the radial momentum component by noting
\begin{equation}
    \left| p^r(0) \right| \leq E \leq C_2 \frac{\psuppconst}{\tau(x(s))^2} \leq \frac{(C_2)^5}{C_3 (C_1)^4} \Osqern(r(0)) \psuppconst.
\end{equation}
For the angular momentum component we note $\left| \pslash(0) \right|_{\gslash} = \frac{L}{r(0)}$ so that
\begin{equation}
    \frac{(C_1)^3}{2 C_4 (C_2)^2} \sqrt{\Osqern(r(0))} \psuppconst \leq \frac{C_1}{2} \frac{\psuppconst}{\tau(x(s))} \leq \left| \pslash(0) \right|_{\gslash} \leq C_2 \frac{\psuppconst}{\tau(x(s))} \leq \frac{(C_2)^3}{C_3 (C_1)^2} \sqrt{\Osqern(r(0))} \psuppconst.
\end{equation}
Therefore for any choice of $C_1<C_2$ we may conclude that $(x(0),p(0)) \in \badsetapprox_{c_1,c_2,\delta}$ if we choose
\begin{equation}
    (\tauslow)^2 = 32 \left( \frac{C_2}{C_1} \right)^2 (1+\delta^{-2}), \quad c_1 = \frac{(C_1)^3}{2 C_4 (C_2)^2}, \quad c_2 = \frac{(C_2)^5}{C_3 (C_1)^4}.
\end{equation}
Let us now see that we may choose the constants $C_1,C_2$ such that every $(x(0),p(0)) \in \badsetapprox_{c_1,c_2,\delta}$ satisfies Assumption~\ref{assumption_support}. For the radial and angular momentum components it will suffice to show that we may choose $c_2 \leq 1$. For the $p^{t^*}$-component recall from the proof of Proposition~\ref{properties_of_badsetapprox} the bound
\begin{equation}
    p^{t^*}(0) \leq \max \left( 4c_2, c_2 \delta^2 + \frac{c_2^2}{c_1}  \right) \psuppconst .
\end{equation}
We now claim that it suffices to assume that $C_2 < \frac{(C_3)^2}{8 C_4}$ and choose $0 < C_1 < C_2$ such that $(C_1)^{11} > \frac{8 C_4}{(C_3)^2} (C_2)^{12}$. Let us first show that such a constant $C_1$ exists. By our assumption on $C_2$ we conclude that $\frac{8 C_4}{(C_3)^2} (C_2)^{12} < (C_2)^{11}$, so that we may choose any $C \in \left( \frac{8 C_4}{(C_3)^2} (C_2)^{12} , (C_2)^{11} \right)$ and set $C_1 = C^{\frac{1}{11}}$. Now note that our assumptions on $C_1,C_2$ imply that
\begin{equation}
    c_2 = \frac{(C_2)^5}{C_3 (C_1)^4} \leq \frac{2 C_4}{(C_3)^2} \frac{(C_2)^{12}}{(C_1)^{11}} < \frac{1}{4}, \quad \frac{c_2^2}{c_1} = \frac{2 C_4}{(C_3)^2} \frac{(C_2)^{12}}{(C_1)^{11}} < \frac{1}{4},
\end{equation}
where we recall that we may assume $C_3 < 1 < C_4$ without loss of generality. From these bounds and the fact that $C_1 < C_2$ and $\Osqern(r(0)) \leq 1$ it easily follows that 
\begin{equation}
    p^{t^*}(0), \left| p^r(0) \right|, \left| \pslash(0) \right|_{\gslash} \leq \psuppconst.
\end{equation}
We remark that the particular choice of exponents of the constants is a matter of choice, and many other choices are possible. This concludes the proof that $\badsetallconst \subset \badsetapprox_{c_1,c_2,\delta}$ and that points in $\badsetapprox_{c_1,c_2,\delta}$ satisfy Assumption~\ref{assumption_support} for our choice of constants. To see that $\badsetapprox_{c_1,c_2,\delta} \subset \badsetaplarge_{\constbsl,\delta}$, recall that in the proof of Proposition~\ref{psupport_propERN} we showed that we may choose $\constbsl = 1$. Thus the statement follows from the definition since we may assume $c_2 < \frac{1}{4}$. 

\paragraph{Step 2: The reverse inclusion $\badsetapprox \subset \badset$.}
Let us now show that there exists a compatible choice of $C_1,C_2$, as well as $c_1,c_2$ such that for any $\tauslow > m$ suitably large there exists a $\delta > 0$ such that the reverse inclusion $\badsetapprox_{c_1,c_2,\delta} \subset \badsetallconst$ holds. Suppose that $\gamma: [0,s] \rightarrow \Mern$ is a maximally defined geodesic segment as above with $(x(0),p(0)) \in \badsetapprox_\delta$. We begin by bounding the energy and angular momentum of $\gamma$. The fact that $(x(0),p(0)) \in \badsetapprox_\delta$ and conservation of energy imply
\begin{equation} \label{proof_approx_badset_below_E}
    c_1 \Osqern(r(0)) \psuppconst \leq \sqrt{\Osqern(r(0))} \left| \pslash(0) \right|_{\gslash} \leq E = \sqrt{(p^r(0))^2 + \Osqern(r(0)) \left| \pslash(0) \right|_{\gslash}^2} \leq \sqrt{2} c_2 \Osqern(r(0)) \psuppconst.
\end{equation}
Similarly we immediately obtain the bound
\begin{equation} \label{proof_approx_badset_below_L}
    c_1 \sqrt{\Osqern(r(0))} \psuppconst \leq \frac{L}{m} \leq 2 c_2 \sqrt{\Osqern(r(0))} \psuppconst.
\end{equation}
It follows readily that
\begin{equation}
    \frac{1}{32 \delta^2} \frac{(c_1)^2}{(c_2)^2} \leq \frac{1}{32} \frac{(c_1)^2}{(c_2)^2} \frac{1}{\Osqern(r(0))} \leq \frac{1}{16 m^2} \frac{L^2}{E^2} \leq \frac{1}{4} \frac{(c_2)^2}{(c_1)^2} \frac{1}{\Osqern(r(0))}.
\end{equation}
Note that for $0 < c_1 < c_2$ chosen arbitrarily, we may always choose $0 < \delta < 1$ so that $\frac{1}{16 m^2} \frac{L^2}{E^2} > 3$. Using the definition of the trapping parameter we may therefore assume that $\trapern = \trapern(\gamma) < -2$ and
\begin{equation} \label{proof_approx_badset_below_eps}
    \frac{1}{64} \frac{(c_1)^2}{(c_2)^2} \frac{1}{\Osqern(r(0))} < \left| \trapern \right| < \frac{1}{4} \frac{(c_2)^2}{(c_1)^2} \frac{1}{\Osqern(r(0))}.
\end{equation}
Next we want to show that $\gamma$ indeed crosses the event horizon after a finite time and obtain a relationship between the time of crossing and $r(0)$. Let us distinguish the two cases that $p^r(0) > 0$ and $p^r(0) \leq 0$. In the latter case, we note that necessarily $p^r(s') < 0$ for all $s' \in (0,s]$ so that the geodesic must fall into the black hole. Since we have assumed $\gamma$ to be maximally defined, we find $r(s) = m$. We argue in an identical manner to above to find
\begin{equation}
    \tau(x(s)) \sim m (1 + \left| \trapern \right|) \sqrt{\Osqern(r(0))}.
\end{equation}
Using equation~\eqref{proof_approx_badset_below_eps} above we conclude the existence of constants $\tilde{c}_3 < \tilde{c}_4$ such that
\begin{equation}
    \tilde{c}_3 \left( \frac{c_1}{c_2} \right)^4 \frac{m^2}{\tau(x(s))^2}  \leq D(r(0)) \leq \tilde{c}_4 \left( \frac{c_2}{c_1} \right)^4 \frac{m^2}{\tau(x(s))^2}.
\end{equation}
In the case where $p^r(0) > 0$ note that $\trapern < -2$ so that $\gamma$ is initially outgoing and must eventually scatter off the photon sphere and fall into the black hole. Again we argue as above that
\begin{equation}
    \tau(x(s)) \sim \frac{m}{\sqrt{\Osqern(r(0))}},
\end{equation}
which again allows us to conclude the existence of constants $\bar{c}_3 < \bar{c}_4$ such that
\begin{equation}
    \bar{c}_3 \frac{m^2}{\tau(x(s))^2} \leq \Osqern(r(0)) \leq \bar{c}_4 \frac{m^2}{\tau(x(s))^2}.
\end{equation}
We have therefore shown the existence of constants $c_3 < c_4$ such that for both signs of $p^r(0)$ we have
\begin{equation} \label{proof_badset_approx_below_time}
    c_3 \left( \frac{c_1}{c_2} \right)^4 \frac{m^2}{\tau(x(s))^2}  \leq D(r(0)) \leq c_4 \left( \frac{c_2}{c_1} \right)^4 \frac{m^2}{\tau(x(s))^2}.
\end{equation}
In order to show that $(x(0),p(0)) \in \badsettau$ we need to consider the values of $\tau(x(s))$ and of $p^r(s), \pslash(s)$. Recall that we consider $\tauslow > m$ to be given and $C_1 < C_2$ to satisfy the assumptions placed on these constants in the argument above. First note that by inequality~\eqref{proof_badset_approx_below_time} we find
\begin{equation}
    \tau(x(s))^2 \geq c_3 \left( \frac{c_1}{c_2} \right)^4 \frac{m^2}{\delta^2}.
\end{equation}
Assuming we choose $c_1,c_2$ as a function of $C_1,C_2$ only, we may choose $0 < \delta < 1$ small enough as a function of $\tauslow$ so that $\tau(x(s)) > \tauslow$. Next note that since $r(s) = m$ we have $\left| \pslash(s) \right|_{\gslash} = \frac{L}{m}$ and $\left| p^r(s) \right| = E$. Combining inequalities \eqref{proof_approx_badset_below_E} and \eqref{proof_approx_badset_below_L} with inequality~\eqref{proof_badset_approx_below_time} we see that $p^r(s)$ and $\pslash(s)$ satisfy the bounds
\begin{equation}
    C_1 \psuppconst \frac{m}{\tau(x(s))} \leq \left| \pslash(s) \right|_{\gslash} \leq C_2 \psuppconst \frac{m}{\tau(x(s))}, \quad C_1 \psuppconst \frac{m^2}{\tau(x(s))^2} \leq \left| p^r(s) \right| \leq C_2 \psuppconst \frac{m^2}{\tau(x(s))^2}
\end{equation}
if we choose the constants $c_1 < c_2$ to satisfy
\begin{gather}
    \frac{(c_2)^5}{(c_1)^4} \leq \frac{C_2}{2 c_4}, \quad \frac{(c_1)^5}{(c_2)^4} \geq \frac{C_1}{c_3}, \label{c1c2ineq1} \\
    \frac{(c_2)^3}{(c_1)^2} \leq \frac{C_2}{2 \sqrt{c_4}}, \quad  \frac{(c_1)^3}{(c_2)^2} \geq \frac{C_1}{\sqrt{c_3}} \label{c1c2ineq2}.
\end{gather}
We may assume without loss of generality that $c_3 < 1 < c_4$. Since we also assume $c_1 < c_2$ it follows readily that inequalities~\eqref{c1c2ineq1} imply inequalities~\eqref{c1c2ineq2}. It is now easy to verify that there exist $c_1 < c_2$ satisfying inequalities~\eqref{c1c2ineq1} if and only if
\begin{equation}
    \frac{C_1}{c_3} < \frac{C_2}{2 c_4} \iff C_1 < \frac{c_3}{2 c_4} C_2.
\end{equation}
Recall that we assumed in the above argument $C_1 < C_2$ and $(C_1)^{11} > \frac{8 C_4}{(C_3)^2} (C_2)^{12}$. We would now like to strengthen the upper bound to $C_1 < \frac{c_3}{2 c_4} C_2$ while leaving the lower bound the same. To show the existence of such constants $C_1,C_2$, first choose $0 < C_2 < \frac{c_3^{11}}{2^{11}c_4^{11}} \frac{(C_3)^2}{8 C_4}$ and then note that this bound implies $\frac{8 C_4}{(C_3)^2} (C_2)^{12} < \frac{c_3^{11}}{2^{11}c_4^{11}} (C_2)^{11}$. Choose $C \in \left( \frac{8 C_4}{(C_3)^2} (C_2)^{12} , \frac{c_3^{11}}{2^{11}c_4^{11}} (C_2)^{11} \right)$ and define $C_1 = C^{\frac{1}{11}}$. Then $C_1$ satisfies both bounds required. We again note that the choice of exponents of the constants involved in proving our estimates are a matter of choice and many other choices are possible. This proves the existence of $C_1,C_2$ compatible with the assumptions above, as well as $c_1,c_2$ chosen as a function of $C_1,C_2$ and $\delta$ chosen as a function of $\tauslow,c_1,c_2$ such that $\badsetapprox_{c_1,c_2,\delta} \subset \badsetallconst$. Finally note that we only require an upper bound on $C_2$, so that we may choose it sufficiently small to ensure that $\slowsupportset \subset \smallsupportset$.
\end{proof}

\subsubsection{Properties of $\slowsupportset$}
Finally we establish a bound on the phase-space volume of each fibre of the set $\slowsupportset$ and prove that every momentum $p = (p^{t^*},p^r,\pslash) \in \slowsupportsetx$ satisfies $p^{t^*} \sim \psuppconst$.

\begin{prop}[Properties of $\slowsupportset$] \label{ptstar_size_slowsupport}
In the notation of Proposition~\ref{sharpness_lemma}, the constants $C_1, C_2$ may be chosen such that the conclusions of Proposition~\ref{sharpness_lemma} hold true and in addition for every $(x,p) \in \slowsupportset$ expressed in $(t^*,r)$-coordinates as $(x,p) = (t^*,r,\omega,p^{t^*},p^r,\pslash)$ the following bound holds
\begin{equation}
    \frac{(C_1)^2}{2 C_2} \psuppconst \leq p^{t^*} \leq \psuppconst.
\end{equation}
Furthermore we have
\begin{equation}
    \vol \slowsupportsetx = \int_{\slowsupportsetx} 1 \, d \mu_x \sim_{C_1,C_2} \psuppconst^2 \frac{m^2}{(\tau(x))^2}.
\end{equation}
\end{prop}
\begin{proof}
Recall from the proof of Proposition~\ref{sharpness_lemma} that we chose the constants $C_1,C_2$ by first choosing $C_2$ small enough, concretely $C_2 < \bar{C}$ for some constant $\bar{C} < \frac{1}{8}$ defined in the proof and then choosing $C_1$ so that in particular the bounds
\begin{equation} \label{C1C2bound_slowset_proof}
    2C_1 < C_2, \quad \frac{(C_2)^2}{C_1} \leq \frac{(C_2)^{12}}{(C_2)^{11}} \leq \frac{1}{8},
\end{equation}
hold. We remark that the choice of exponents of $C_1,C_2$ was made for convenience and many other choices are possible. Recall also that in the notation of Proposition~\ref{sharpness_lemma}, the constants $C_1,C_2$ were chosen independently of $\bar{\delta}$. Let us now in addition assume that $C_2$ also satisfies the lower bound $C_2 > \frac{1}{2} \bar{C}$. Recall from Lemma~\ref{express_ptstar} that along the event horizon $\mathcal{H}^+$ we have $- p^r = E$ and
\begin{equation}
    p^{t^*} = E + \frac{\left| \pslash \right|_{\gslash}^2}{2E}.
\end{equation}
This immediately allows us to conclude the bounds
\begin{equation}
    \left( \frac{C_1}{\tau(x)^2} + \frac{(C_1)^2}{2 C_2} \right) \psuppconst \leq p^{t^*} \leq \left(\frac{C_2}{\tau(x)^2} + \frac{(C_2)^2}{2 C_1} \right) \psuppconst.
\end{equation}
If we assume $\tau(x) > \tauslow > 1$ then inequality~\eqref{C1C2bound_slowset_proof} immediately implies that $p^{t^*} \leq \psuppconst$. Let us now turn to estimating the volume of $\slowsupportsetx$. Note that the bound $C_2 > \frac{1}{2} \bar{C}$ and inequality~\eqref{C1C2bound_slowset_proof} imply
\begin{equation} \label{proof_ern_slow_volume}
    \bar{C}^2 < 4 (C_2)^2 < 2 C_1 < C_2 < \bar{C}. 
\end{equation}
Let us note at this point that for $C_1,C_2$ satisfying inequality~\eqref{proof_ern_slow_volume} the following relations may be verified to hold
\begin{gather}
    \frac{\bar{C}}{2} \left( \sqrt{C_2} - \sqrt{C_1} \right) < C_2 - C_1 < \sqrt{C_2} - \sqrt{C_1}, \label{proof_slow_CC_1}\\
    \frac{\bar{C}}{4} < \frac{1}{2} C_2 < C_2 - C_1 < C_2 < \bar{C}, \label{proof_slow_CC_2}\\
    \frac{C_2}{C_1} < \frac{1}{2 C_2} < \frac{1}{\bar{C}} .\label{proof_slow_CC_3}
\end{gather}
As in the proof of Proposition~\ref{properties_of_badsetapprox} above, we parametrise the null-cone using $(t^*,r)$-coordinates by eliminating the $p^{t^*}$-coordinate using the mass-shell relation. We again introduce a change of coordinates $\pslash \mapsto (p^1,p^2)$ with the property that $L^2 = r^2 \left| \pslash \right|_{\gslash}^2 = (p^1)^2 + (p^2)^2$. Equation~\eqref{dmu_first_comp_schw} allows us to express the volume form $d \mu_x$ as
\begin{equation}
	d \mu = \frac{1}{r^2} \frac{1}{E} \, d p^r d p^1 d p^2.
\end{equation}
Like in the proof of Proposition~\ref{properties_of_badsetapprox} we introduce a further change of coordinates. We use radial coordinates in the variables $(p^1,p^2)$ so that $(p^1,p^2) \mapsto (L,\angle (p^1,p^2) ) =: (L,\pslashangle)$. Therefore an element $p \in \slowsupportsetx$ is now parametrised by $(p^r,L,\pslashangle)$. For any $x \in \mathcal{H}^+$ we may therefore parametrise the set $\slowsupportsetx$ in the above coordinates as the set of points $p$ which satisfy
\begin{equation}
\begin{gathered}
    C_1 \frac{\psuppconst}{\tau(x)^2} \leq \left| p^r \right| \leq C_2 \frac{\psuppconst}{\tau(x)^2}, \\
    m C_1 \frac{\psuppconst}{\tau(x)} \leq L \leq m C_2 \frac{\psuppconst}{\tau(x)}, \\
    0 \leq \pslashangle \leq 2\pi.
\end{gathered}
\end{equation}
Recalling that $r=m$ and $\left| p^r \right| = E$ in our case, we find
\begin{equation}
    \int_{\slowsupportsetx} 1 \, d \mu_x = \int_{ \frac{C_1 \psuppconst}{\tau(x)^2}}^{ \frac{C_2 \psuppconst}{\tau(x)^2}} \int_{ \frac{m C_1 \psuppconst}{\tau(x)}}^{ \frac{m C_2 \psuppconst}{\tau(x)}} \int_0^{2 \pi} \frac{1}{m^2} \frac{L}{E} \, d \pslashangle d L d E.
\end{equation}
The definition of $\slowsupportset$ and relation~\eqref{proof_slow_CC_3} derived above readily implies the bound
\begin{equation}
    \psuppconst \lesssim \frac{C_1}{C_2} \psuppconst \leq \frac{1}{m^2} \frac{L^2}{E} \leq \frac{C_2}{C_1} \psuppconst \leq 8 \psuppconst.
\end{equation}
Therefore we conclude the relation
\begin{equation}
    \frac{1}{m} \frac{L}{E} = \sqrt{\frac{1}{m^2} \frac{L^2}{E}} \frac{1}{\sqrt{E}} \sim \frac{\sqrt{\psuppconst}}{\sqrt{E}}.
\end{equation}
Finally this allows us to conclude
\begin{align}
    \int_{\slowsupportsetx} 1 \, d \mu_x &\sim \sqrt{\psuppconst} \left( \int_{ \frac{C_1 \psuppconst}{\tau(x)^2}}^{ \frac{C_2 \psuppconst}{\tau(x)^2}} \frac{1}{\sqrt{E}} \, dE \right) 
    \left( \int_{ \frac{C_1 \psuppconst}{\tau(x)}}^{ \frac{C_2 \psuppconst}{\tau(x)}} 1 \, d L \right) \\
    &\sim \frac{\psuppconst^2}{\tau(x)^2} \left( \sqrt{C_2} - \sqrt{C_1} \right) \left( C_2 - C_1 \right) \sim \frac{\psuppconst^2}{\tau(x)^2},
\end{align}
where in the last step we have made use of the relations~\eqref{proof_slow_CC_1} and~\eqref{proof_slow_CC_2} described above. This concludes the proof.
\end{proof}


\subsection{Polynomial decay of moments} \label{estimating_moments_section_ERN}
We are now well equipped to prove Theorem~\ref{maintheoremERNprecise}, showing decay for moments of a solution $f$ to the massless Vlasov equation. We will begin the discussion by showing a simpler boundedness statement for the volume of the support in Lemma~\ref{lem_boundedness_moments_ERN}, before providing the proof of Theorem~\ref{maintheoremERNprecise}.

\subsubsection{Finiteness of volume of the momentum support}
We show a quantitative bound for the volume of the momentum support of a solution to the massless Vlasov equation with compactly supported initial data. Lemma~\ref{lem_boundedness_moments_ERN} is the extremal analogue to Lemma~\ref{lem_boundedness_moments_schw}.


\begin{lem} \label{lem_boundedness_moments_ERN}
Assume that $f$ solves the massless Vlasov equation on extremal Reissner--Nordstr\"om and its initial distribution $f_0: \mathcal{P}_0 \rightarrow \R$ satisfies Assumption~\ref{assumption_support}. Then for all $x \in \Mern$ with $\tau(x) \geq 0$ we have
\begin{equation}
	\int_{\supp(f(x,\cdot))} 1 \, d \mu_x  \lesssim \frac{\rsuppconst^2}{r^2} \psuppconst^2.
\end{equation}
\end{lem}
\begin{proof}
The strategy of proof is identical to that of Lemma~\ref{lem_boundedness_moments_schw}. We parameterise the null-cone by using the coordinates induced by $(t^*,r)$-coordinates and eliminating the $p^{t^*}$-component of the momentum, as discussed in Section~\ref{sec_parametrising_massshell}. As in the proof of Proposition~\ref{properties_of_badsetapprox} we introduce a change of variables $\pslash \mapsto (p^1,p^2)$ with the property that $L^2 = r^2 \left| \pslash \right|_{\gslash}^2 = (p^1)^2 + (p^2)^2$ and the further change of coordinates $(p^1,p^2) \mapsto (L,\angle (p^1,p^2) ) =: (L,\pslashangle)$, or in other words we use radial coordinates in the angular variables $(p^1,p^2)$. Note that $\pslashangle \in [0,2 \pi)$ and recall that by virtue of Lemma~\ref{psupport_boundedness_ERN} any $(x,p) \in \supp(f)$ satisfies the bounds $p^{t^*} \lesssim \psuppconst, \left| p^r \right| \leq \psuppconst$ and $\left| p^i \right| \leq L \leq \rsuppconst \psuppconst$. We therefore find
\begin{equation}
	\int_{\supp(f(x,\cdot))} 1 \, d \mu_x \leq \int_{-\psuppconst}^{\psuppconst} \int_{0}^{\rsuppconst \psuppconst} \int_0^{2 \pi} \frac{1}{r^2} \frac{L}{E} \, d \pslashangle d L d p^r \lesssim \int_{-\psuppconst}^{\psuppconst} \int_{0}^{\rsuppconst \psuppconst} \frac{1}{r^2} \frac{L}{E} \, d L d p^r.
\end{equation}
By Lemma~\ref{energy_trapparam_bounded_ern} the bound $E (1 + \left| \trapschw \right|) \lesssim \frac{\rsuppconst^2}{m^2} \psuppconst$ holds in the support of $f$. Therefore
\begin{equation}
	\frac{1}{r^2} \frac{L}{E} \lesssim \frac{m}{r^2} \sqrt{1 + \left| \trapschw \right|} \lesssim \frac{\rsuppconst}{r^2} \psuppconst^{\frac{1}{2}} \frac{1}{\sqrt{\left| p^r \right|}}.
\end{equation}
We conclude the bound
\begin{equation}
	\int_{\supp(f(x,\cdot))} 1 \, d \mu_x \lesssim \frac{\rsuppconst}{r^2} \psuppconst^{\frac{1}{2}} \left( \int_{-\psuppconst}^{\psuppconst} \frac{1}{\sqrt{\left| p^r \right|}} \, d p^r \right) \left( \int_{0}^{\rsuppconst \psuppconst} 1 \, d L \right) \lesssim \frac{\rsuppconst^2}{r^2} \psuppconst^2,
\end{equation}
as claimed.
\end{proof}

\subsubsection{Proof of Theorem~\ref{maintheoremERNprecise}}
We can now prove Theorem~\ref{maintheoremERNprecise} by taking full advantage of the estimates established in the sections above. The proof will proceed in an entirely similar fashion to that of Theorem~\ref{maintheorem_precise} for the subextremal case. 

\begin{proof}[Proof of Theorem~\ref{maintheoremERNprecise}]
Like in the proof of Theorem~\ref{maintheorem_precise} we parametrise the null-cone $\mathcal{P}$ using $(t^*,r)$-coordinates by expressing the $p^{t^*}$-coordinate as a function of the remaining ones. If we introduce spherical coordinates on the sphere $\sphere$ the null-cone is then explicitly parametrised by $(t^*,r,\theta,\phi,p^r,p^\theta,p^\phi)$. According to equation~\eqref{dmu_first_comp_schw} the induced volume form on each fibre $\mathcal{P}_x$ then takes the form
\begin{equation}
	d \mu_x =  \frac{r^2 \sin \theta}{\Osqern p^{t^*} - \left( 1 - \Osqern \right) p^r} \, d p^r d p^\theta d p^\phi =  \frac{r^2 \sin \theta}{E} \, d p^r d p^\theta d p^\phi.
\end{equation}
By Proposition~\ref{psupport_propERN} we know that analogously to the Schwarzschild case
\begin{equation}
\supp(f) \cap \left\{ (x,p) \in \mathcal{P} \; | \; \tau(x) \gtrsim \tauzero + 1 \right\} \subset \ERNtrappedsupportset \cup \ERNsmallsupportset.
\end{equation}
Let us assume in this proof that $w \geq 0$, since otherwise we may replace $w$ by $\left| w \right|$. Recall our assumption that $w$ is bounded in $x$ according to Definition~\ref{boundedness_in_x} and apply Lemma~\ref{psupport_boundedness_ERN} to conclude that any $p \in \supp(f(x,\cdot))$ satisfies $\left| p \right| \lesssim \frac{\rsuppconst \psuppconst}{m}$, so that
\begin{equation}
	W := \max_{(x,p) \in \supp(f)} \left| w(x,p) \right| < \infty.
\end{equation}
We immediately conclude the bound
\begin{equation} \label{proof_ERN_moments_1}
	\int_{\mathcal{P}_x} w f \, d \mu_x \leq \| f_0 \|_{L^\infty} W \left( \int_{\ERNtrappedsupportsetx} 1 \, d \mu_x + \int_{\ERNsmallsupportsetx} 1 \, d \mu_x \right) .
\end{equation}
Let us now estimate the two integrals on the right hand side of equation~\eqref{proof_ERN_moments_1}. Like in the proof of Theorem~\ref{maintheorem_precise} above, we introduce a change of coordinates on each fibe $\mathcal{P}_x$. Assume that $r \geq m$ and $p \in \mathcal{P}_x$ and consider the transformation
\begin{equation}
	(p^r,p^\theta,p^\phi) \mapsto \left( p^r,\sqrt{(p^\theta)^2 + \sin^2 \theta (p^\phi)^2}, \angle(p^\theta,p^\phi) \right) =: (p^r, l, \pslashangle),
\end{equation}
where we note that $\pslashangle = \angle(p^\theta,p^\phi) \in [0,2 \pi)$. We first estimate $\int_{\ERNtrappedsupportsetx} 1 \, d \mu_x = \vol \ERNtrappedsupportsetx$. Note that if we assume $p \in \ERNtrappedsupportsetx$ then the length $l$ satisfies $l = \frac{1}{r} \left| \pslash \right|_{\gslash} \leq \frac{\rsuppconst \psuppconst}{r^2}$ and $p^r \in [p^r_-, p^r_+]$, if we define
\begin{equation}
	p^r_{\pm} = \frac{L}{m} \left( \sqrt{ \frac{1}{16} - \frac{m^2}{r^2} \Osqern } \pm \bigc e^{-\frac{c}{2m}(\tau(x) - \tauzero)} \right).
\end{equation}
Hence we obtain the bound
\begin{align}
	\int_{\ERNtrappedsupportsetx} 1 \, d \mu_x &\lesssim \int_0^{2 \pi} \int_0^{\frac{\rsuppconst \psuppconst}{r^2}} \int_{p^r_{-}}^{p^r_{+}} \frac{r^2 l}{E} \, d \bar{p}^r d l d \pslashangle \\
	&= \int_0^{2 \pi} \int_0^{\rsuppconst \psuppconst} \int_{p^r_{-}}^{p^r_{+}} \frac{1}{r^2} \frac{L}{E} \, d \bar{p}^r d L d \pslashangle.
\end{align}
Now note that for any $(x,p) \in \ERNtrappedsupportset$ with $m \leq r < \infty$, we have $\left| \trapern \right| \leq C_1 e^{-\frac{c}{2m}(\tau(x) - \tauzero)} \leq C_1$ so that
\begin{equation}
	\frac{1}{r^2} \frac{L}{E} \lesssim \frac{m}{r^2} \sqrt{1 + \left| \trapschw \right|} \leq \frac{m}{r^2} \sqrt{1+ C_1}.
\end{equation}
To compute the integral, we introduce another change of coordinates
	\begin{equation} \label{prtildeERN}
	(p^r, L) \mapsto \left( m \frac{p^r}{L} - \sqrt{ \frac{1}{16} - \frac{m^2}{r^2} \Osqern }, L \right) = (\tilde{p}^r, \tilde{L}).
\end{equation}
Therefore
\begin{equation}
	\int_0^{\rsuppconst \psuppconst} \int_{p^r_{-}}^{p^r_{+}} 1 \, d p^r d L = \left( \int_0^{\rsuppconst \psuppconst} \frac{\tilde{L}}{m} \, d \tilde{L} \right) \left( \int_{-C_1 e^{-\frac{c}{2m}(\tau(x) - \tauzero)}}^{C_1 e^{-\frac{c}{2m}(\tau(x) - \tauzero)}} 1 \, d \tilde{p}^r \right) = \frac{C_1 \rsuppconst^2 \psuppconst^2}{m} e^{-\frac{c}{2m}(\tau(x) - \tauzero)}.
\end{equation}
In summary we have therefore obtained the bound
\begin{equation}
	\int_{\trappedsupportsetx} 1 \, d \mu \lesssim (1+C_1)^{\frac{3}{2}} \rsuppconst^{2} \psuppconst^{2} \frac{1}{r^2} e^{-\frac{c}{2m}(\tau(x) - \tauzero)}.
\end{equation}
Like in the proof of Theorem~\ref{maintheorem_precise} we obtain a better decay rate for the moment whenever we can show a better decay rate for the $p^r$ component.  Let us now turn to the task of estimating the integral $\int_{\ERNsmallsupportsetx} w \, d \mu_x = \vol \ERNsmallsupportsetx$. If we assume that $p \in \ERNsmallsupportsetx$ the length $l$ satisfies the bound $l = \frac{1}{r} \left| \pslash \right|_{\gslash} \leq C_2 \frac{\rsuppconst \psuppconst}{r^2} \frac{m}{\tau(x) - \tauzero}$ and $\left| p^r \right| \leq C_2 \psuppconst \frac{m^2}{(\tau(x) - \tauzero)^2}$. Therefore
\begin{equation}
	\int_{\ERNsmallsupportsetx} 1 \, d \mu_x \lesssim \int_0^{2 \pi} \int_{0}^{C_2 \rsuppconst \psuppconst \frac{m}{\tau(x) - \tauzero}} \int_0^{C_2 \psuppconst \frac{m^2}{(\tau(x) - \tauzero)^2} } \frac{1}{r^2} \frac{L}{E} \, d p^r d L d \pslashangle.
\end{equation}
Lemma~\ref{energy_trapparam_bounded_ern} implies that $\left( 1 + \left| \trapschw \right| \right) \left| p^r \right| \leq \left( 1 + \left| \trapschw \right| \right) E \lesssim \frac{\rsuppconst^2}{m^2} \psuppconst$ for any $(x,p) \in \supp(f)$. Therefore
\begin{equation}
	\frac{1}{r^2} \frac{L}{E} \lesssim \frac{m}{r^2} \sqrt{1 + \left| \trapschw \right|} \lesssim \frac{\rsuppconst}{r^2} \sqrt{\psuppconst} \frac{1}{\sqrt{\left| p^r \right|}}.
\end{equation}
We conclude the bound
\begin{align}
	\int_{\ERNsmallsupportsetx} 1 \, d \mu_x &\lesssim \frac{\rsuppconst}{r^2} \sqrt{\psuppconst} \left( \int_{0}^{C_2 \rsuppconst \psuppconst \frac{m}{\tau(x) - \tauzero}} 1 \, d L \right) \left( \int_0^{C_2 \psuppconst \frac{m^2}{(\tau(x) - \tauzero)^2} } \frac{1}{\sqrt{\left| p^r \right|}} \, d p^r \right) \\
	&\sim (C_2)^{\frac{3}{2}} \frac{\rsuppconst^2 \psuppconst^2}{r^2} \frac{m^2}{(\tau(x) - \tauzero)^2}.
\end{align}
As opposed to the Schwarzschild case, the rate of decay for $\vol \ERNsmallsupportsetx$ is only inverse quadratic and therefore strictly slower than the exponential decay rate of $\vol \ERNtrappedsupportsetx$. We have therefore shown the bound
\begin{equation}
    \int_{\supp(f(x,\cdot))} 1 \, d \mu_x \leq C \frac{\rsuppconst^2 \psuppconst^2}{r^2} \frac{m^2}{(\tau(x) - \tauzero)^2},
\end{equation}
for an appropriate constant $C > 0$. Let us now assume that $f_0$ is initially supported away from the event horizon. Consider a geodesic $\gamma$ such that $(\gamma(0), \dot{\gamma}(0)) \in \supp(f_0)$ like in the proof of Proposition~\ref{psupport_propERN} and assume that $\tau(x) \gtrsim 1 + \tauzero$ with a sufficiently large constant. It follows readily from the proof of Proposition~\ref{psupport_propERN} that there exists a constant $C'$ such that either $\gamma$ populates the set $\ERNtrappedsupportset$ or its initial data must satisfy $\Osqern(r(0)) \lesssim \frac{m^2}{(\tau(x) - \tauzero)^2}$ for $\tau(x) \geq C' (1 + \tauzero)$. For $\tau(x) > \tauzero + \frac{m}{\delta}$ however, this implies $r(0) < (1+\delta)m$, which contradicts our assumption on the support of $f_0$. Therefore we conclude that
\begin{equation}
    \supp(f) \cap \left\{ (x,p) \in \mathcal{P} : \tau(x) \gtrsim 1 + \tauzero + \frac{m}{\delta} \right\} \subset \ERNtrappedsupportset.
\end{equation}
This immediately implies that for $x \in \Mern$ with $\tau(x) \gtrsim 1 + \tauzero + \frac{m}{\delta}$ the better decay rate
\begin{equation}
    \int_{\supp(f(x,\cdot))} 1 \, d \mu_x \leq C \frac{\rsuppconst^2 \psuppconst^2}{r^2} e^{-\frac{c}{2m} \left( \tau(x) - \tauzero \right)}
\end{equation}
holds. Like in the case of Schwarzschild, if we assume that the initial support of $f$ is away from the event horizon, then this exponential decay rate may be improved away from the photon sphere. If we let $0 < \delta' < 1$ and assume that $r \notin [(2-\delta')m,(2+\delta')m]$ and $\tau \gtrsim \tauzero - m \ln \delta' + \frac{m}{\delta}$ then the rate of decay may be improved to
\begin{equation}
	\int_{\supp(f(x,\cdot))} 1 \, d \mu \lesssim C \frac{\rsuppconst^2 \psuppconst^2}{r^2} e^{-\frac{c}{m}(\tau(x) - \tauzero)}.
\end{equation}
For the remaining part of the proof, we will no longer assume that $f$ is supported away from the event horizon. \\

We now show that we can improve the rate of decay for weights of the form $w = \left| p^r \right|^a (p^{t^*})^b \left| \pslash \right|_{\gslash}^c$ for $a,b,c \geq 0$. First let us note that by Lemma~\ref{lem_boundedness_moments_ERN} for all $(x,p) \in \supp(f)$ we have
\begin{equation}
    \left| p^r \right|^a (p^{t^*})^b \left| \pslash \right|_{\gslash}^c \lesssim \frac{\psuppconst^{a+b+c} \rsuppconst^c}{r^c}.
\end{equation}
From our above computations we conclude
\begin{equation}
    \int_{\ERNtrappedsupportsetx} \left| p^r \right|^a (p^{t^*})^b \left| \pslash \right|_{\gslash}^c \, d \mu_x \lesssim \frac{\psuppconst^{2+a+b+c} \rsuppconst^{2+c}}{r^{2+c}} e^{- \frac{c}{2m} (\tau(x)-\tauzero)}.
\end{equation}
Therefore the contribution from the almost trapped set still decays at an exponential rate. From the definition of the set $\ERNsmallsupportset$ it follows that
\begin{equation}
    \int_{\ERNsmallsupportsetx} \left| p^r \right|^a (p^{t^*})^b \left| \pslash \right|_{\gslash}^c \, d \mu_x \lesssim \frac{\psuppconst^{2+a+b+c} \rsuppconst^{2+c}}{r^{2+c}} \left( \frac{m}{\tau(x) - \tauzero} \right)^{2+2a+c} \left[ \min \left( \frac{1}{\Osqern} \frac{m^2}{(\tau(x)-\tauzero)^2}, 1 \right) \right]^b.
\end{equation}
If we assume that $r \geq (1 + \delta) m $ for some $0 < \delta < 1$, then we may conclude
\begin{equation} \label{inequ_proof_momentsERN}
    \int_{\supp(f(x,\cdot))} \left| p^r \right|^a (p^{t^*})^b \left| \pslash \right|_{\gslash}^c \, d \mu_x \lesssim \frac{1}{\delta^2} \frac{\psuppconst^{2+a+b+c} \rsuppconst^{2+c}}{r^{2+c}} \left( \frac{m}{\tau(x) - \tauzero} \right)^{2(1+a+b)+c},
\end{equation}
whereas if we assume $m \leq r \leq (1+\delta)m$ then the factor of $p^{t^*}$ does not improve the rate of decay and we find
\begin{equation}
    \int_{\supp(f(x,\cdot))} \left| p^r \right|^a (p^{t^*})^b \left| \pslash \right|_{\gslash}^c \, d \mu_x \lesssim \frac{\psuppconst^{2+a+b+c} \rsuppconst^{2+c}}{r^{2+c}} \left( \frac{m}{\tau(x) - \tauzero} \right)^{2(1+a)+c}.
\end{equation}
Next let us turn to weights of the form $w = \left| p^v \right|^a \left| p^u \right|^b \left| \pslash \right|_{\gslash}^c$ for $a,b,c > 0$, where we made use of double null coordinates and the induced coordinates on $\mathcal{P}$ to express a point $(x,p) \in \mathcal{P}$ as $(x,p) = (u,v,\omega,p^u,p^v,\pslash)$. Making the change of coordinates we compute $p^v = p^{t^*} + p^r$, so that $p^v \leq C p^{t^*} \lesssim \psuppconst$ for all $(x,p) \in \supp(f)$ by Lemma~\ref{psupport_boundedness_ERN}. Since the asymptotically flat regions of extremal Reissner--Nordstr\"om and Schwarzschild are comparable, an appropriate analogue of Lemma~\ref{tauestimate_far} holds here. We may therefore conclude that for all $(x,p) \in \supp(f)$ with $\tau(x) \geq \tauzero$ and $r \geq R$
\begin{equation}
    p^u \leq C \frac{\rsuppconst^2}{r^2} p^v \leq C \frac{\rsuppconst^2}{r^2} p^{t^*}.
\end{equation}
This allows us to conclude that
\begin{equation}
    \int_{\supp(f(x,\cdot))} \left| p^v \right|^a \left| p^u \right|^b \left| \pslash \right|_{\gslash}^c \, d \mu_x \leq C^{b} \frac{\rsuppconst^{2 b}}{r^{2 b}} \int_{\supp(f(x,\cdot))} (p^{t^*})^{a+b} \left| \pslash \right|_{\gslash}^c \, d \mu_x,
\end{equation}
which in combination with inequality~\eqref{inequ_proof_momentsERN} shown above immediately allows us to conclude the desired bound. The theorem follows after rescaling the constants $C$ and $c$ appropriately.
\end{proof}

\subsection{Sharpness of polynomial decay along the event horizon} \label{sec_proof_lowerboundERN}
In this section we provide the proof of Theorem~\ref{ERN_slowdecayprop}, establishing a lower bound on the rate of decay.

\begin{proof}[Proof of Theorem~\ref{ERN_slowdecayprop}]
We begin by recalling that we have shown $\badsetapprox_\delta \subset \badsetaplarge_\delta$ for all $0 < \delta < \frac{1}{2}$ in Proposition~\ref{sharpness_lemma}. Therefore, we may assume that $\inf\nolimits_{\badsetapprox_\delta} f_0 > 0$, for otherwise there is nothing to prove. Recall from Proposition~\ref{sharpness_lemma} that by choosing $\tauslow \sim \delta^{-1}$ and all remaining constants appropriately, we may also assume that $\badsettau \subset \badsetapprox_\delta$. This implies $\badsettau \subset \supp(f_0)$ and therefore by definition that $\slowsupportsettau \subset \supp(f)|_{\mathcal{H}^+}$. Note that by definition of $\slowsupportsettau$ and Proposition~\ref{ptstar_size_slowsupport} we have that for any $(x,p) \in \slowsupportsettau$ the bound
\begin{equation}
    w_{a,b,c}(x,p) = \left| p^r \right|^a \left| \pslash \right|_{\gslash}^b (p^{t^*})^c \geq \frac{(C_1)^{a+b+2c}}{(2C_2)^c} \psuppconst^{a+b+c} \frac{m^{2a + b}}{\tau(x)^{2a+b}}
\end{equation}
holds. Now let $x \in \mathcal{H}^+$ with $\tau(x) > \tauslow$. Then since $f \geq 0$ we find that
\begin{align}
    \int_{\mathcal{P}_x} w_{a,b,c} f \, d \mu_x &\geq \int_{\slowsupportsettaux} w_{a,b,c} f \, d \mu_x \\
    &\geq \left( \min_{\slowsupportsettaux} w_{a,b,c} \right) \left( \min_{\slowsupportsettaux} f \right) \vol(\slowsupportsettaux) \\
    &\gtrsim \psuppconst^{2+a+b+c} \frac{m^{2(a+1) + b}}{\tau(x)^{2(a+1)+b}} \inf\nolimits_{\badsettau} f_0 \\
    &\gtrsim \psuppconst^{2+a+b+c} \frac{m^{2(a+1) + b}}{\tau(x)^{2(a+1)+b}} \inf\nolimits_{\badsetaplarge_\delta} f_0,
\end{align}
where we made use of the chain of inclusions $\badsettau \subset \badsetapprox_\delta \subset \badsetaplarge_\delta$ in the last step and in the penultimate estimate we used  Proposition~\ref{ptstar_size_slowsupport} to estimate the volume of $\slowsupportsettaux$ and the size of momentum components in the set $\slowsupportsettaux$.
\end{proof}


\subsection{Non-decay of transversal derivatives along the event horizon} \label{section_derivativesERN}
In this section we provide the proof of Theorem~\ref{ERN_nondecaytransversal}. Note carefully that our proof is not based on the existence of a conservation law. Instead we will make careful use of the precise characterisation of the sets $\badsettau$ obtained above.

\begin{proof}[Proof of Theorem~\ref{ERN_nondecaytransversal}]

The argument proceeds along the lines explained in Section~\ref{subsec_growth_overview}. We will assume without loss of generality that $\partial_{t^*} f |_{\Sigma_0} \geq 0$ on the set $\badsetaplarge_\delta$, otherwise replace $f$ with $-f$. Throughout the proof we employ the convention that Latin letters refer to spherical components.


\paragraph{Step 1: Commuting the derivative inside.}
We begin by showing that for all points along the extremal event horizon $x \in \mathcal{H}^+$, the following identity holds:
\begin{equation} \label{proof_nondecay_decomposition}
\left( \partial_{r} \int_{\sphere} T^{t^* t^*}[f] \, d \omega \right) \bigg|_{r=m} = \int_{\sphere} T^{t^* t^*}\left[ \frac{p^{t^*}}{\left| p^r \right|} \partial_{t^*} f \right] \, d \omega \bigg|_{r=m} + \mathcal{E},
\end{equation}
where for all $x \in \mathcal{H}^+$ with $\tau(x) > \tauzero$ the error term $\mathcal{E}$ satisfies
\begin{equation} \label{proof_nondecay_errorbound}
    \left| \mathcal{E} \right| \lesssim \psuppconst^4 \rsuppconst^2 \left( \| f_0 \|_{L^{\infty}} + \| \partial_{t^*} f_0 \|_{L^{\infty}} \right) \frac{1}{(\tau(x) - \tauzero)^2}.
\end{equation}
Let us define $\tilde{f} = \frac{p^{t^*}}{p^r} f$ and note that $\tilde{f}$ is well-defined on $\{ (x,p) \in \mathcal{P} : p^r \neq 0 \}$. We remind the reader of the discussion in Section~\ref{rngeometry} and recall that the coordinate system given by $(p^r,\pslash)$ on a fibre $\mathcal{P}_x$ degenerates at points where $r=m$ and $p^r=0$. We will therefore assume in our computations that $p^r \neq 0$. We find
\begin{align}
    X \left( \frac{p^{t^*}}{p^r} \right) = - \left( \Gamma^{t^*}_{\alpha \beta} p^\alpha p^\beta \right) \frac{1}{p^r} + \left( \Gamma^{r}_{\alpha \beta} p^\alpha p^\beta \right) \frac{p^{t^*}}{(p^r)^2},
\end{align}
where $\Gamma^{\mu}_{\alpha \beta}$ denote the Christoffel symbols of the extremal Reissner--Nordstr\"om metric in $(t^*,r)$-coordinates. An explicit computation shows that along the event horizon
\begin{equation}
    X \left( \frac{p^{t^*}}{p^r} \right) \Bigg|_{r=m} = - \frac{2}{m} \left( p^r + p^{t^*} \right).
\end{equation}
We next recall the standard identity $\nabla_\mu T^{\mu \nu} [\tilde{f}] = \int_{\mathcal{P}_x} p^{\nu} X(\tilde{f}) \, d \mu_x$, see~\cite{martin,riosecosarbach,sarbachzannias} for an in-depth discussion on how to relate derivatives of the energy momentum tensor of a solution to derivatives of the solution itself. Applying this identity, we find that for all $x \in \mathcal{H}^+$
\begin{equation}
    \nabla_\mu T^{t^* \mu} [\tilde{f}] = \int_{\mathcal{P}_x} p^{t^*} X(\tilde{f}) \, d \mu_x = \int_{\mathcal{P}_x} p^{t^*} X \left( \frac{p^{t^*}}{p^r} \right) f \, d \mu_x = - \frac{2}{m} \left( T^{t^* t^*}[f] + T^{t^* r}[f] \right).
\end{equation}
Let us for the moment omit the dependence on $\tilde{f}$ and write $T^{\mu \nu} = T^{\mu \nu}[\tilde{f}]$. Then it follows that
\begin{equation}
    \nabla_A \left( T^{A \mu} (\partial_{t^*})_\mu \right) = \left( \nabla_A T^{A \mu} \right) (\partial_{t^*})_\mu + g_{\mu \nu} \Gamma^{\nu}_{A t^*} T^{A \mu} = \nabla_A T^{A t^*},
\end{equation}
where we used that in the whole exterior $\Gamma^{\nu}_{A t^*} = 0$ for all $\nu$. Note carefully that by choosing a time $t^* >0$ and $r=m$ we obtain an embedding of $\iota: \sphere \hookrightarrow \Mern$ such that the Lorentzian metric $\gERN$ restricts to the Riemannian metric $\iota^* \gERN = m^2 d \omega$ where $d \omega$ denotes the standard round metric on $\sphere$. This induces the natural restriction map $\Gamma(T \Mern) \rightarrow \Gamma(T\sphere)$ of vector fields on $\Mern$ to vector fields on $\sphere$, by virtue of the orthogonal decomposition $\iota^* T \Mern = 
T \Mern|_{\sphere} = T \sphere \oplus T^{\perp} \sphere$. Expressed in coordinates, this means that for any choice of coordinates $\omega$ on $\sphere$ and fixed $t^* > 0$ and $r=m$ the restriction of the vector field with components $(T^{t^* \mu})_{\mu = t^*,r,A,B}$ has the components $(T^{\mu t^*})_{\mu = A,B}$, where as usual we denote the spherical indices by capital letters $A,B$. Therefore the spherical part of the divergence may be rewritten as the divergence of a vector field on $\sphere$ so that upon integration
\begin{equation} \label{proof_nondecay_equ}
    \int_{\sphere} \nabla_\mu T^{t^* \mu} [\tilde{f}] \, d \omega \bigg|_{r=m} = \int_{\sphere} \nabla_{t^*} T^{t^* t^*} [\tilde{f}] + \nabla_{r} T^{t^* r} [\tilde{f}] \, d \omega \bigg|_{r=m} .
\end{equation}
Let us now study the two remaining terms and notice that by the definition of the covariant derivative we find that along the event horizon
\begin{equation}
    \nabla_{t^*} T^{t^* t^*} \Big|_{r=m} = \partial_{t^*} T^{t^* t^*} + 2 \Gamma^{t^*}_{\alpha t^*} T^{\alpha t^*} = \partial_{t^*} T^{t^* t^*}.
\end{equation}
Recall that $\partial_{t^*} \in \Gamma(T \Mern)$ is the timelike Killing vector field associated to the stationarity of the Reissner--Nordstr\"om solution. This implies the identity $\partial_{t^*} T^{t^* t^*}[f] = T^{t^* t^*}[ \partial_{t^*}  f]$, where by a slight abuse of notation we use the same symbol $\partial_{t^*}$ to denote the coordinate vector field of the coordinates $(t^*,r,\omega)$ on $\Mern$ on the left hand side, and the coordinate vector field of $(t^*,r,\omega,p^r,\pslash)$ on $\mathcal{P}$ on the right hand side. More formally, we consider the complete lift of the Killing vector field $\partial_{t^*}$ to the mass-shell here, see again~\cite{martin,riosecosarbach,sarbachzannias} for a discussion in greater generality. It also readily follows that $[X,\partial_{t^*}] = 0$ so that $\partial_{t^*} f$ is again a solution to the massless Vlasov equation, where we consider $\partial_{t^*}$ as a vector field on $\mathcal{P}$. Therefore
\begin{equation}
    \int_{\sphere} \nabla_{t^*} T^{t^* t^*}[\tilde{f}] \, d \omega \bigg|_{r=m} = \int_{\sphere} T^{t^* t^*}\left[ \frac{p^{t^*}}{p^r} \partial_{t^*} f \right] \, d \omega \bigg|_{r=m}.
\end{equation}
For the second term we note that $T^{t^* r}[\tilde{f}] = T^{t^* t^*}[f]$ by definition of $\tilde{f}$. By the definition of the covariant derivative
\begin{equation}
    \nabla_{r} T^{t^* r} \Big|_{r=m} = \partial_{r} T^{t^* r} + \Gamma^{t^*}_{\alpha r} T^{\alpha r} + \Gamma^{r}_{\alpha r} T^{\alpha t^*} = \partial_{r} T^{t^* r},
\end{equation}
which allows us to conclude 
\begin{equation}
    \nabla_{r} T^{t^* r}[\tilde{f}] \Big|_{r=m} = \partial_{r} T^{t^* t^*}[f] \Big|_{r=m} .
\end{equation}
Therefore we find
\begin{equation}
    \int_{\sphere} \nabla_{r} T^{t^* r}[\tilde{f}] \, d \omega \bigg|_{r=m} = \left( \partial_{r} \int_{\sphere} T^{t^* t^*}[f] \, d \omega \right) \bigg|_{r=m}.
\end{equation}
Finally we use equation~\eqref{proof_nondecay_equ} above to see that
\begin{equation}
    \left( \partial_{r} \int_{\sphere} T^{t^* t^*}[f] \, d \omega \right) \bigg|_{r=m} = \int_{\sphere} T^{t^* t^*}\left[ - \frac{p^{t^*}}{p^r} \partial_{t^*} f \right] \, d \omega \bigg|_{r=m} + \mathcal{E},
\end{equation}
where we have abbreviated
\begin{equation}
    \mathcal{E} := - \frac{2}{m} \int_{\sphere} \left( T^{t^* t^*}[f] + T^{t^* r}[f] \right) \, d \omega \bigg|_{r=m}.
\end{equation}
Note that for $(x,p) \in \supp(f)$ with $x \in \mathcal{H}^+$ we have $p^r \leq 0$, so that $- p^r = \left| p^r \right|$. By Theorem~\ref{maintheoremERNprecise} above, for any $x \in \mathcal{H}^+$ with $\tau(x) > \tauzero$ we have the bounds
\begin{equation}
    \left| T^{t^* r}[f] \right|, T^{t^* t^*}[f] \lesssim \psuppconst^4 \rsuppconst^2 \| f_0 \|_{L^{\infty}} \frac{1}{(\tau(x) - \tauzero)^2},
\end{equation}
so that $\mathcal{E}$ satisfies inequality~\eqref{proof_nondecay_errorbound} as claimed above.

\paragraph{Step 2: Separating off the decaying part.}

We now have an expression for the transversal derivative of $\int_{\sphere} T^{t^* t^*}[f] \, d \omega$ evaluated along the event horizon. The next step is to identify the part of the expression that is decaying. In what follows, we will then show that the remainder is indeed non-decaying. Recall that Proposition~\ref{psupport_propERN} implies that all $(x,p) \in \supp(f)$ with $\tau(x) \geq \tauzero$ must be either almost trapped at the photon sphere or at the event horizon, in other words $(x,p) \in \ERNtrappedsupportset \cup \ERNsmallsupportset$. For $x \in \mathcal{H}^+$ with $\tau(x) > \tauzero$ we therefore conclude
\begin{equation}
    T^{t^* t^*}\left[ \frac{p^{t^*}}{\left|p^r \right|} \partial_{t^*} f \right] = \int_{\ERNtrappedsupportsetx} \frac{(p^{t^*})^3}{\left| p^r \right|} \partial_{t^*} f \, d \mu_x + \int_{ \ERNsmallsupportsetx \setminus \ERNtrappedsupportsetx} \frac{(p^{t^*})^3}{\left| p^r \right|} \partial_{t^*} f \, d \mu_x.
\end{equation}
By definition for all $(x,p) \in \ERNtrappedsupportset$ with $\tau(x) \geq \tauzero$ we have $m \frac{\left| p^r \right|}{L} \geq \frac{1}{8}$ and $m \frac{p^{t^*}}{L} \leq \frac{9}{2}$. Therefore
\begin{equation}
    \frac{p^{t^*}}{\left| p^r \right|} = \frac{m \frac{p^{t^*}}{L}}{m \frac{\left| p^r \right|}{L}} \lesssim 1.
\end{equation}
This implies that
\begin{equation}
    \left| \int_{\ERNtrappedsupportsetx} \frac{(p^{t^*})^3}{\left| p^r \right|} \partial_{t^*} f \, d \mu_x \right| \lesssim \psuppconst^4 \frac{\rsuppconst^2}{m^2} \| (\partial_{t^*}f)_0 \|_{L^\infty} e^{-\decayrate (\tau(x)-\tauzero)},
\end{equation}
so that we may absorb the spherical average of this term into the error term $\mathcal{E}$.

\paragraph{Step 3: Proving the lower bound.}
It now remains to estimate the term
\begin{equation}
	\int_{ \ERNsmallsupportsetx \setminus \ERNtrappedsupportsetx} \frac{(p^{t^*})^3}{\left| p^r \right|} \partial_{t^*} f \, d \mu_x
\end{equation}
from below. To accomplish this, we first show that $\partial_{t^*} f$ has a sign on the set $\ERNsmallsupportsetx \setminus \ERNtrappedsupportsetx$ for all points $x \in \mathcal{H}^+$ with $\tau(x) > \tauslow \sim \delta^{-1}$. Then we show that for such late times, the support of $f$ contains the set $\slowsupportsetx$ as a subset. This is then used by noting that on $\slowsupportsetx$ we can estimate $\frac{p^{t^*}}{\left| p^r \right|}$ from below. \\

To see that $\partial_{t^*} f \geq 0$ on the set $\ERNsmallsupportsetx \setminus \ERNtrappedsupportsetx$ for every $x \in \mathcal{H}^+$ with $\tau(x) > \tauslow \sim \delta^{-1}$, recall that $\partial_{t^*} f$ solves the massless Vlasov equation. Therefore, we may equivalently show that the statement holds on the set
\begin{equation}
	\mathcal{D}_0 := \pi_0 \big( (\smallsupportset \setminus \trappedsupportset) \cap \supp(f) \cap \left\{ \tau \geq \tauslow \right\} \big),
\end{equation}
where we recall that $\pi_0 : \mathcal{P} \rightarrow \mathcal{P}_0$ transports points backwards in time along the geodesic flow until it intersects the initial hypersurface $\Sigma_0$, see Definition~\ref{defn_pi0}. Proposition~\ref{psupport_propERN} implies that $\mathcal{D}_0 \subset \badsetaplarge_{\delta}$, so that the claim follows from our assumption that $\partial_{t^*} f \geq 0$ on $\badsetaplarge_{\delta}$. \\

Recall that we assume $\badsetaplarge_\delta \subset \supp(f_0)$. Proposition~\ref{sharpness_lemma} implies that $\badsetapprox_\delta \subset \badsetaplarge_\delta$, so that by Proposition~\ref{sharpness_lemma} again we find $\slowsupportset_{\tauslow} \subset \supp(f)|_{\mathcal{H}^+}$, where the time parameter $\tauslow$ satisfies $\tauslow \sim \delta^{-1}$. By definition of $\slowsupportset_\tauslow$ and Proposition~\ref{ptstar_size_slowsupport} we see that for $(x,p) \in \slowsupportset_\tauslow$ the bound
\begin{equation}
	\frac{p^{t^*}}{\left| p^r \right|} \geq \frac{(C_1)^2}{2 (C_2)^2} \tau(x)^2
\end{equation}
holds, where $C_1,C_2$ denote the constants used in the definition of $\slowsupportset_\tauslow$. We therefore find
\begin{equation}
    \int_{ \ERNsmallsupportsetx \setminus \ERNtrappedsupportsetx} \frac{(p^{t^*})^3}{\left| p^r \right|} \partial_{t^*} f \, d \mu_x \geq \int_{ \slowsupportset_{\tauslow,x}} \frac{(p^{t^*})^3}{\left| p^r \right|} \partial_{t^*} f \, d \mu_x \gtrsim \psuppconst^2 \tau(x)^2 \int_{ \slowsupportset_{\tauslow,x}} \left| \partial_{t^*} f \right| \, d \mu_x.
\end{equation}
By definition of $\badsettau$ and since $\partial_{t^*} f$ solves the massless Vlasov equation, for all $(x,p) \in \badsettau$
\begin{equation}
	\left| \partial_{t^*} f (x,p) \right| \geq \inf_{\badsettau} \left| \partial_{t^*} f_0 \right|.
\end{equation}
We conclude
\begin{equation}
    \int_{ \ERNsmallsupportsetx} \frac{(p^{t^*})^3}{\left| p^r \right|} \partial_{t^*} f \, d \mu_x \gtrsim \psuppconst^2 \tau(x)^2 \vol \slowsupportsettaux \left( \inf_{\badsettau} \left| \partial_{t^*} f_0 \right| \right) \sim \psuppconst^4 \left( \inf_{\badsettau} \left| \partial_{t^*} f_0 \right| \right),
\end{equation}
where we made use of Proposition~\ref{ptstar_size_slowsupport} to estimate the volume of $\slowsupportsettaux$ in the last step. Using that for any $\delta > 0$, all remaining constants involved in the definition of the sets may be chosen such that $\badset \subset \badsetapprox_\delta \subset \badsetaplarge_{\delta}$ as proven in Proposition~\ref{sharpness_lemma}, we finally conclude
\begin{equation}
    \left( \partial_{r} \int_{\sphere} T^{t^* t^*}[f] \, d \omega \right) \bigg|_{r=m} \gtrsim \psuppconst^4 \left( \inf_{\badsettau} \left| \partial_{t^*} f_0 \right| \right) + \mathcal{E} \geq \psuppconst^4
    \left( \inf_{\badsetaplarge_\delta} \left| \partial_{t^*} f_0 \right| \right) + \mathcal{E}.
\end{equation}
Given the error bound~\eqref{proof_nondecay_errorbound}, this allows us to conclude the desired lower bound if we assume that $\tau(x) - \tauzero \geq C m \left( \inf\nolimits_{\badsetaplarge_\delta} \left| \partial_{t^*} f_0 \right| \right)^{-\frac{1}{2}}$ with a large enough constant $C > 1$.
\end{proof}
