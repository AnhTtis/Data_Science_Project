\section{Introduction}
Understanding the late time dynamics of the Einstein equations in the vicinity of black hole solutions is a highly active area of research. The Einstein equations are at the heart of the theory of general relativity and may be expressed as
\begin{equation}
    \text{Ric}(g)_{\mu \nu} - \frac{1}{2} g_{\mu \nu} R(g) = T_{\mu \nu},
\end{equation}
where $T_{\mu \nu}$ is the energy-momentum tensor of an appropriate matter model and the equations close by specifying appropriate evolution equations for the matter model. See~\cite{general} for an introduction to general relativity. Despite tremendous progress on subextremal black holes, many open problems remain in the extremal limit. For instance, for the extremal Kerr spacetime, two key geometric phenomena complicate its study:
\begin{itemize}
    \item the fact that the red-shift effect degenerates and
    \item the coupling of trapping and superradiance.
\end{itemize}
The first effect is already present in the simpler extremal Reissner--Nordstr\"om spacetime and is known to lead to an instability for the scalar wave equation, see Section~\ref{section_aretakis}. By contrast, the effect of the lack of decoupling of trapping and superradiance is not understood even at the level of massless linear fields.

%Alternative text here: The degeneracy of the red-shift effect is known to lead to an instability on the linear level. This so-called Aretakis instability is well understood for the scalar wave equation. The Aretakis instability is also present for the extremal Reissner--Nordstr\"om black hole, which does not exhibit the second geometric phenomenon outlined above.

\subsection{Main results}
In this work we consider the massless Vlasov equation on the exterior of the subextremal and extremal Reissner--Nordstr\"om solution $(\Mrn,\gRN)$. In its most familiar form the metric takes the form
\begin{equation} \label{RN_metric_intro}
    g = - \Osqrn dt^2 + \OsqrnNEG dr^2 + r^2 d \omega^2, \quad \Osqrn = 1 - \frac{2m}{r} + \frac{q^2}{r^2},
\end{equation}
where $d \omega^2$ is the usual round metric on $\sphere$ and we assume $m>0$ and $\left| q \right| \leq m$. For this range of parameters~\eqref{RN_metric_intro} is known to describe a black hole. The parameter $m$ represents the mass of the black hole and $q$ its electrical charge. The Reissner--Nordstr\"om family is the unique (up to diffeomorphism) stationary spherically symmetric $2$-parameter family of solutions to the Einstein--Maxwell equations. We say the solution is subextremal when $\left| q \right| < m$ and extremal when $\left| q \right| = m$. \\

The massless Vlasov equation or collisionless Boltzmann equation is a kinetic particle model describing a distribution of collisionless particles moving at the speed of light in a given spacetime. Phenomena in general relativity are often understood by first studying massless linear fields without back-reaction, such as the massless Vlasov equation. Examples of nonlinear results that have their origins in this line of reasoning include the work of Poisson--Israel~\cite{poissonisreal1,poissonisreal2} on instability of black hole interiors and Moschidis'~\cite{moschidismasslessvlasov,moschidisnulldust} proof of the AdS instability conjecture~\cite{adsinstabilityconj}.\\

Let $(\Mrn,\gRN)$ denote the Reissner--Nordstr\"om exterior. A coordinate system $(x^\mu)$ on $\Mrn$ induces the conjugate coordinate system $(x^\mu, p^\mu)$ on the tangent bundle $T \Mrn$ by representing each $p \in T_x \Mrn$ as $p = p^\mu \partial_\mu|_x$. We define the mass-shell $\mathcal{P} \subset T \Mrn$ as the set
\begin{equation}
\mathcal{P} = \Big\{ (x,p) \in T \Mrn : \, \gRN (x)(p,p)=0, \, p \text{ is future-directed} \Big\} .
\end{equation}
A function $f: \mathcal{P} \rightarrow \R_{\geq 0}$ solves the massless Vlasov equation if $f$ is conserved along the geodesic flow or equivalently if
\begin{equation}
    X(f) = \left( p^\mu \partial_\mu - \Gamma^\mu_{\alpha \beta} p^\alpha p^\beta \partial_{p^\mu} \right) f = 0,
\end{equation}
where $X$ denotes the geodesic spray, $\Gamma^\mu_{\alpha \beta}$ denote the Christoffel symbols in the coordinates $(x^\mu)$ and we have given the explicit expression of the geodesic spray in conjugate coordinates $(x^\mu, p^\mu)$. Let $\tau: \Mrn \rightarrow \R$ such that $\Sigma_0 = \{ \tau = 0 \}$ is a spherically symmetric Cauchy hypersurface which connects the event horizon and future null infinity and such that the hypersurfaces $\Sigma_\tau$ of constant $\tau > 0$ are obtained from $\Sigma_0$ by the flow of the timelike Killing field. See Figure~\ref{penrose_tau} for an illustration and Section~\ref{timecoordinate} for a precise definition. If we denote the mass-shell over the initial Cauchy hypersurface $\Sigma_0$ by $\mathcal{P}_0 = \mathcal{P}|_{\Sigma_0}$ then we may treat the massless Vlasov equation as the initial value problem
\begin{equation}
	\begin{cases}
		X(f) = 0 \\
		f|_{\mathcal{P}_0} = f_0
	\end{cases} .
\end{equation}
For every sufficiently regular initial distribution $f_0: \mathcal{P}_0 \rightarrow \R_{\geq 0}$ there exists a unique solution $f$. In Section~\ref{sec_vlasov} we define the natural measure $d \mu_x$ on the fibre $\mathcal{P}_x$ induced by the metric. Given a continuous weight $w: \mathcal{P} \rightarrow \R$, we define moments of a solution by $\int_{\mathcal{P}_x} w f \, d \mu_x$. We will be particularly interested in weights of the form $w = p^\alpha p^\beta$ for arbitrary spacetime indices $\alpha,\beta$. Their associated moments define the components of the energy momentum tensor $T^{\alpha \beta}[f] = \int_{\mathcal{P}_x} p^\alpha p^\beta f \, d \mu_x$ of the solution $f$. The energy momentum tensor is divergence free and is a natural object of interest since it carries physical information and features in the coupled massless Einstein--Vlasov system. \\


%We define a time function with respect to which we measure decay.

%We refer the reader to Section~\ref{prelim} for a more detailed discussion of the geometry of the subextremal and extremal Reissner--Nordstr\"om solutions and the massless Vlasov equation.

\begin{figure}
    \centering
    \includegraphics[scale=0.4]{figures/penrose_diagram.png}
    \caption{Penrose diagram of the Reissner--Nordstr\"{o}m black hole exterior. We denote the event horizon by $\mathcal{H}^+$ and future null infinity by $\mathcal{I}^+$. The hypersurfaces $\Sigma_0$ and $\Sigma_\tau$ for some $\tau > 0$ are being displayed. Refer to Section~\ref{timecoordinate} for the definition of the time function $\tau$ and the associated surfaces $\Sigma_\tau$.}
    \label{penrose_tau}
\end{figure}

\subsubsection{Exponential decay on subextremal Reissner--Nordstr\"om} \label{subsec_expdecay}

We may now state our main theorem on the subextremal Reissner--Nordstr\"om background. In Section~\ref{section_mainthms} below we will state a more precise version of Theorem~\ref{maintheorem} in the form of Theorem~\ref{maintheorem_precise}.

\begin{thm}[Exponential decay on subextremal Reissner--Nordstr\"om] \label{maintheorem}
Let $f_0: \mathcal{P}_0 \rightarrow [0,\infty)$ be smooth and compactly supported and let $f$ be the unique solution to the massless Vlasov equation on subextremal Reissner--Nordstr\"om with initial data $f_0$. Let $w: \mathcal{P} \rightarrow [0,\infty)$ be smooth and bounded in $x$ (see Definition~\ref{boundedness_in_x}). Then there exist constants $C = C(w, \supp(f_0), m,q)$ and $c = c(m,q)$ such that for all $x \in \Mrn$ with $\tau = \tau(x) \geq 0$, the following decay estimate holds:
\begin{equation}
\int_{\mathcal{P}_x} w f \, d \mu_x \leq C \| f_0 \|_{L^{\infty}} \frac{1}{r^2} e^{-c \tau}.
\end{equation}
\end{thm}

\begin{rem}[Decay of the energy-momentum tensor] \label{rem_expdecay_tensor}
A direct consequence of Theorem~\ref{maintheorem} is that all components of the energy-momentum tensor decay at an exponential rate. We will compare this with the wave equation later in Section~\ref{section_aretakis}. By making use of the $(t^*,r)$-coordinate system defined in Section~\ref{rngeometry} and applying Theorem~\ref{maintheorem} with the weights $w = (p^{t^*})^2$ respectively $w = (p^r)^2$ we find that in particular
\begin{equation}
\begin{gathered}
    \sup_{\Sigma_\tau} T^{t^* t^*}[f] \leq C \| f_0 \|_{L^\infty} e^{-c \tau}, \\
    \sup_{\Sigma_\tau} T^{rr}[f] \leq C \| f_0 \|_{L^\infty} e^{-c \tau},
\end{gathered}
\end{equation}
where $c = c(m,q)$ is as in Theorem~\ref{maintheorem} and $C = C(\supp(f_0),m,q)$ is the constant $C$ from above evaluated at $w = (p^r)^2$ or $w = (p^{t^*})^2$ respectively.
\end{rem}

\begin{rem}
For certain weights $w$, the rate of decay can be improved. Furthermore as $r \rightarrow \infty$ one recovers the decay rates~in~$r$ from Minkowski space. We remind the reader at this point that the Reissner--Nordstr\"om solution is asymptotically flat. In Theorem~\ref{maintheorem_precise} we provide a complete characterisation of decay rates in $\tau$ and $r$ for polynomial weights. 
\end{rem}

\begin{rem}
The constant $C$ degenerates as $\left| q \right| \rightarrow m$. Its precise dependence on the support of the initial data $f_0$ and the weight $w$ is given in Theorem~\ref{maintheorem_precise} below. For reasons of clarity we have assumed $f_0$ and $w$ to be smooth, however it will become clear from the proof that far less regularity (in fact, little more than measurability) is sufficient.
\end{rem}

We note the recent work by Bigorgne~\cite{leo} who proves faster than polynomial decay for moments of solutions to the massless Vlasov equation on Schwarzschild. The proof adapts the $r^p$-method of Dafermos--Rodnianski~\cite{rp} originally devised for the wave equation. We remark that the result generalises to the full subextremal Reissner--Nordstr\"om range. Velozo~\cite{renato} has shown nonlinear stability of Schwarzschild as a solution to the coupled spherically symmetric massless Einstein--Vlasov system. In~\cite{renato} exponential decay is independently shown for solutions to the massless Vlasov equation on spacetimes close to Schwarzschild by exploiting the hyperbolic nature of the geodesic flow around the photon sphere. Andersson--Blue--Joudioux~\cite{anderssonblue} have shown an integrated energy decay estimate for the massless Vlasov equation on very slowly rotating Kerr backgrounds.

\subsubsection{Polynomial decay on extremal Reissner--Nordstr\"om}

We now state our main results on the extremal Reissner--Nordstr\"om spacetime. In contrast to the subextremal case, moments only decay at an inverse polynomial rate in general. Responsible for this slower rate of decay is the existence of certain null geodesics which are ``almost generators'' of the extremal event horizon $\mathcal{H}^+ = \{ r = m \}$. Over $\Sigma_0$ these ``almost generators'' lie in a one-parameter family $\badsetaplarge_\delta \subset \mathcal{P}_0$ defined explicitly in Definition~\ref{defi_slowsupportset} below. We give a brief explanation of these sets here and refer the reader to Section~\ref{section_ERN} for more details. The parameter $\delta > 0$ measures ``closeness'' to generators of the event horizon. The sets $\badsetaplarge_\delta$ form a decreasing family as $\delta \searrow 0$ and $\bigcap_{\delta > 0} \badsetaplarge_\delta$ is supported over the event horizon $\mathcal{H}^+$ and only contains null generators of $\mathcal{H}^+$. Furthermore, each set $\badsetaplarge_\delta$ is compact with non-empty interior in $\mathcal{P}$. We will state a more precise version of the following result in the form of Theorems~\ref{maintheoremERNprecise} and~\ref{ERN_slowdecayprop} in Section~\ref{section_mainthms}.

\begin{thm}[Polynomial decay on extremal Reissner--Nordstr\"om] \label{maintheoremERN}
Let $f_0: \mathcal{P}_0 \rightarrow [0,\infty)$ be smooth and compactly supported and let $f$ be the unique solution to the massless Vlasov equation on subextremal Reissner--Nordstr\"om with initial data $f_0$. Let $w: \mathcal{P} \rightarrow [0,\infty)$ be smooth and bounded in $x$ as in Definition~\ref{boundedness_in_x}. There exists a constant $C = C(w, \supp(f_0), m)$ such that for all $x \in \Mern$ with $\tau(x) \geq 0$ the following decay estimate holds
\begin{equation} \label{decay_eqn_mainthmERN}
\int_{\mathcal{P}_x} w f \, d \mu_x \leq C \| f_0 \|_{L^{\infty}} \frac{1}{r^2} \frac{1}{\tau^2}.
\end{equation}
Moreover the polynomial rate of decay is sharp along the event horizon, in the following sense: Let $\delta > 0$ and assume that $\badsetaplarge_\delta \subset \supp(f_0)$. Then for all $x \in \mathcal{H}^+$ with $\tau = \tau(x) \gtrsim \delta^{-1}$ we have
\begin{equation}
    \int_{\mathcal{P}_x} f \, d \mu_x \geq C \left( \inf_{(x,p)\in \badsetaplarge_\delta} {f_0(x,p)} \right) \frac{1}{\tau^{2}},
\end{equation}
for an appropriate constant $C = C(\supp(f_0),m)$.

Finally, if $f_0$ is supported away from the event horizon $\mathcal{H}^+ = \{ r = m \}$, then $\int_{\mathcal{P}_x} f \, d \mu_x$ decays at an exponential rate in $\tau = \tau(x)$ globally, similar to the subextremal case.
\end{thm}

\begin{rem}[Decay of the energy-momentum tensor] \label{rem_poldecay_tensor}
The rate of decay stated in equation~\eqref{decay_eqn_mainthmERN} holds for general weights and may be improved depending on the weight $w$ and the distance from the event horizon, for example. Making use of $(t^*,r)$-coordinates defined in Section~\ref{rngeometry}, consider the weight $w = (p^{t^*})^2$, whose associated moment defines a component of the energy-momentum tensor $T^{t^* t^*}[f]$. Then along the event horizon
\begin{equation} \label{remark_tstar_rate_horizon}
    C \left( \inf_{\badsetaplarge_\delta} {f_0} \right) \tau^{-2} \leq \sup_{\Sigma_\tau \cap \mathcal{H}^+} T^{t^* t^*}[f] \leq C \| f_0 \|_{L^\infty} \tau^{-2},
\end{equation}
where the constant $C = C(\supp(f_0), m)$ is as in Theorem~\ref{maintheoremERN} above. At a positive distance from the event horizon we have for $\delta > 0$
\begin{equation}
    \sup_{\Sigma_\tau \cap \{ r \geq (1+\delta)m \} } T^{t^* t^*}[f] \leq C \| f_0 \|_{L^\infty} \tau^{-4},
\end{equation}
where $C = C(\supp(f_0), m, \delta)$ is as in Theorem~\ref{maintheoremERN}. On the other hand, for the weight $w = (p^r)^2$ and its associated moment $T^{rr}[f]$ we find
\begin{align}
    \sup_{\Sigma_\tau} T^{rr}[f] &\leq C \| f_0 \|_{L^\infty} \tau^{-6}, \\
    \sup_{\Sigma_\tau \cap \mathcal{H}^+} T^{rr}[f] &\geq C \left( \inf_{\badsetaplarge_\delta} {f_0} \right) \tau^{-6},
\end{align}
for an appropriate constant $C = C(\supp(f_0), m)$. We point out that in contrast to $T^{t^* t^*}[f]$, the upper bound holds uniformly in the whole exterior up to and including the event horizon.
\end{rem}

\begin{rem}
As in the subextremal case, we recover the rates of $r$-decay from flat Minkowski spacetime as $r \rightarrow \infty$. We will provide a full characterisation of decay rates in both $\tau$ and $r$ for polynomial weights in Theorems~\ref{maintheoremERNprecise} and~\ref{ERN_slowdecayprop} below. Similar remarks about the dependence of the constant $C$ on the size of the support of $f_0$, as well as the smoothness of $f_0$ and the weight $w$ apply here as in the subextremal case.
\end{rem}

\begin{rem} \label{rem_intro_smoothness}
As shown in Theorem~\ref{maintheoremERN}, solutions with initial data supported at a positive distance from $\mathcal{H}^+$ decay at an exponential rate, whereas solutions supported on a neighbourhood $\badsetaplarge_\delta$ of the generators of $\mathcal{H}^+$ only decay at a polynomial rate. If the initial data $f_0$ are smooth and supported on the generators of the event horizon, then there exists $0 < \delta < \frac{1}{2}$ such that $\badsetaplarge_\delta \subset \supp(f_0)$. One expects that for non-smooth data, any decay rate between a polynomial and exponential rate can be achieved.

%This leaves a gap of initial data which may satisfy $\supp(f_0) \cap \badsetaplarge_\delta = \emptyset$ for all $\delta > 0$ yet $\{0\} \times [m, (1+\eta)m] \times \sphere \subset \pi (\supp(f_0))$ for some $\eta > 0$, where $\pi: \mathcal{P} \rightarrow M$ denotes the standard projection from the tangent bundle and where we used $(t^*,r)$-coordinates. We conjecture that for such initial data, any rate interpolating between an inverse polynomial and exponential rate can be achieved. \textcolor{orange}{However such data will most likely not be smooth.}
\end{rem}

\subsubsection{Non-decay for transversal derivatives on extremal Reissner--Nordstr\"om}

Finally, we prove the following result which holds along the event horizon of extremal Reissner--Nordstr\"om, see Theorem~\ref{ERN_nondecaytransversal} below for a more precise version of the result. It is most easily stated in the $(t^*,r)$-coordinates defined in Section~\ref{rngeometry} below. In these coordinates, the timelike Killing derivative may be expressed as $\partial_{t^*}$ and we note that $\partial_r$ is transversal to the event horizon.

\begin{thm}[Non-decay for transversal derivatives on extremal Reissner--Nordstr\"om] \label{maintheorem_ern_nondecay_rough}
Assume $f_0: \mathcal{P}_0 \rightarrow [0,\infty)$ is smooth and compactly supported and let $f$ be the unique solution to the massless Vlasov equation on extremal Reissner--Nordstr\"om with initial data $f_0$. If we assume in addition that there exists $0 < \delta$ such that $\badsetaplarge_\delta \subset \supp(f_0)$ and $\partial_{t^*} f(x,p) \neq 0$ for all $(x,p) \in \badsetaplarge_\delta$, then
\begin{equation} \label{eqn_lowerbound_der_ern}
    \left| \partial_r \int_{\sphere} T^{t^* t^*}[f] \, d \omega \right| \geq C  \inf_{(x,p) \in \badsetaplarge_\delta} \left| \partial_{t^*} f(x,p) \right|,
\end{equation}
for all $x \in \mathcal{H}^+$ with $\tau(x)$ sufficiently large and a suitable constant $C = C(\supp(f_0),m)$. Therefore transversal derivatives of the energy momentum tensor do not decay along the event horizon in general.
\end{thm}

\begin{rem}
We may interpret the timelike Killing derivative $\partial_{t^*} f |_{\Sigma_0}$ as an operator acting on initial data by using the geodesic spray $X$ to express $\partial_{t^*}$ as a function of derivatives which are tangential to $\mathcal{P}_0$.
\end{rem}

\begin{rem}
One expects higher-order derivatives to grow along the event horizon:
\begin{equation} \label{expected_growth}
	\left| \partial_r^k \int_{\sphere} T^{t^* t^*}[f] \, d \omega \right| \geq \tau^{2k-2} C[f_0], \quad k \in \mathbb{Z}_{\geq 0},
\end{equation}
where $C[f_0]$ is a suitable higher-order term analogous to the term on the right hand side of equation~\eqref{eqn_lowerbound_der_ern} involving $k$ derivatives of $f$. The case $k \geq 2$ is not considered in this work.
\end{rem}

\begin{rem} \label{rem_intro_nondecaythm}
In the same spirit as Remark~\ref{rem_intro_smoothness} above, we note that if we assume $f_0 \geq 0$ to be smooth, and that $f_0$ and $\partial_{t^*} f |_{\Sigma_0}$ are nowhere vanishing on the generators of the event horizon, it follows that there exists a $\delta > 0$ such that $\badsetaplarge_\delta \subset \supp(f_0)$ and $\partial_{t^*} f(x,p) \neq 0$ for all $(x,p) \in \badsetaplarge_\delta$.
\end{rem}

\subsection{Comparison with the wave equation and the Aretakis instability} \label{section_aretakis}
Classically, the most studied linear field model on black hole backgrounds is the wave equation
\begin{equation} \label{waveequn}
    \Box_g \psi = g^{\mu \nu} \nabla_\mu \nabla_\nu \psi = 0,
\end{equation}
going back to the result by Wald~\cite{wald} and Kay--Wald~\cite{kaywald} on boundedness of solutions to the wave equation on the Schwarzschild spacetime. Solutions to the scalar wave equation have since been shown to be bounded and decaying on subextremal Kerr and subextremal Reissner--Nordstr\"om~\cite{kerrsubextremalfull,dafermosrodnianskishlap} and precise late-time asymptotics (Price's law) have been established in~\cite{priceslawRN,priceslawKerr,priceslaw2018}. Works on the wave equation have recently been upgraded to nonlinear results by Dafermos--Holzegel--Rodnianski--Taylor~\cite{schwarzschildnonlinearstable}, who show nonlinear stability of Schwarzschild as a solution to the Einstein vacuum equations, see also Giorgi--Klainerman--Szeftel~\cite{klainerman3}. In contrast to the subextremal case, the extremal Reissner--Nordstr\"om and extremal Kerr solutions admit a linear instability mechanism along their event horizons, leading to the so-called Aretakis instability~\cite{aretakis1,aretakis2, aretakis3,subextkerr}. \\

Let us consider the wave equation in some more detail for sake of comparison with the results obtained in this work. Assume that $\psi$ is a smooth solution to the wave equation~\eqref{waveequn} on the subextremal or extremal Reissner--Nordstr\"om exterior with regular compactly supported initial data $\psi_0 = \psi |_{\Sigma_0},  \psi_1 = \partial_{t^*} \psi |_{\Sigma_0}$. The energy-momentum tensor for the wave equation is defined in coordinates by
\begin{equation}
T_W^{\mu \nu}[\psi] = \nabla^\mu \psi \nabla^\nu \psi - \frac{1}{2} g^{\mu \nu} (g^{\alpha \beta} \nabla_\alpha \psi \nabla_\beta \psi).
\end{equation}
We express the energy momentum tensor in $(t^*,r)$-coordinates here, see Section~\ref{rngeometry}. For the purpose of comparing the wave equation to the massless Vlasov equation, let us define $k$-th order transversal moments for the wave equation and the massless Vlasov equation as in Table~\ref{table_moments}. Note that along the event horizon, $T_W^{t^* t^*}[\psi] \big|_{r=r_+} \approx \left| \partial \psi \right|^2$. We conclude $T_W^{t^* t^*}[\partial_r^{k-1} \psi] \big|_{r=r_+} = (\partial_r^{k} \psi)^2 + \text{l.o.t.}$ for all $k \geq 1$, where the lower order terms contain at most $k-1$ transversal derivatives. Therefore, for both the wave equation and the massless Vlasov equation, the order of the transversal moment corresponds to the number of transversal derivatives of the solution along the event horizon. \\

For simplicity, we will only compare decay rates in a neighbourhood of the black hole, where $r \leq R$ for a suitably large radius $R$. On the subextremal Reissner--Nordstr\"om exterior the following result holds.

\setlength{\arrayrulewidth}{0.2mm}
\setlength{\tabcolsep}{18pt}
\renewcommand{\arraystretch}{1.8}
\begin{table}
	\centering
	\begin{tabular}{ p{4.5cm}|c|c  }
		%\hline
		%\multicolumn{3}{|c|}{$k$-th order transversal moments} \\
		 & $k=0$ & $k \geq 1$ \\
		\hline
		Wave equation& $\left| \psi \right|^2$ & $T_W^{t^* t^*}[\partial_r^{k-1} \psi]$  \\
		\hline
		Massless Vlasov equation & $T^{t^* t^*}[f]$ & $T^{t^* t^*}[\partial_r^k f]$ \\
		%\hline
	\end{tabular}
\caption{Definition of $k$-th order transversal moments for the wave equation and the massless Vlasov equation, for the sake of comparing the two models.}
\label{table_moments}
\end{table}
%We refer to $\left| \psi \right|^2$ as the zeroth-order moment for the wave equation, and to $T_W^{t^* t^*}[\partial_r^{k-1} \psi]$ for $k \geq 1$ as a $k$-th order transversal moment. Along the event horizon, the $k$-th order transversal moment reduces to $\big| \partial_r^k \psi \big|^2$ for all $k \geq 0$. For the massless Vlasov equation let us refer to $T^{t^* t^*}[\partial_r^k f] = \int_{\mathcal{P}_x} (p^{t^*})^2 \,  \partial_r^k f \, d \mu_x $ for $k \geq 0$ as a $k$-th order transversal moment.

\begin{thm}[Angelopoulos--Aretakis--Gajic~\cite{priceslawKerr,priceslaw2018}] \label{aretakis_thm_1}
Let $\psi$ be a solution to the wave equation on the subextremal Reissner--Nordstr\"om exterior as above. Then
\begin{equation}
    \sup_{\Sigma_\tau \cap \{ r \leq R \} } \left| \psi \right|^2 \leq C \frac{E[\psi_0,\psi_1]}{\tau^6}, \quad \sup_{\Sigma_\tau \cap \{ r \leq R \} } T_W^{t^* t^*}[\psi] \leq C \frac{E[\psi_0,\psi_1]}{\tau^8},
\end{equation}
for an appropriate constant $C > 0$ and where $E[\psi_0,\psi_1]$ denotes a suitable weighted and higher order initial data norm. For generic initial data these rates are sharp along the event horizon and along constant area radius hypersurfaces on the exterior.
\end{thm}

On the extremal Reissner--Nordstr\"om spacetime we need to distinguish between the event horizon and the region at a positive distance from the black hole. In the following Theorem we consider the region away from the black hole. We remind the reader at this point that the event horizon of the extremal Reissner--Nordstr\"om black hole is located at the radius $r= m$.

\begin{thm}[Aretakis~\cite{aretakis1,aretakis2, aretakis3}] \label{aretakis_thm_2}
Let $\psi$ be a solution to the wave equation on the extremal Reissner--Nordstr\"om exterior as above. Then for $\delta > 0$
\begin{equation}
    \sup_{\Sigma_\tau \cap \{ (1+\delta)m \leq r \leq R \} } \left| \psi \right|^2 \leq C(\delta) \frac{E[\psi_0,\psi_1]}{\tau^4} ,\quad \sup_{\Sigma_\tau \cap \{ (1+\delta)m \leq r \leq R \} } T_W^{t^* t^*}[\psi] \leq C(\delta) \frac{E[\psi_0,\psi_1]}{\tau^{4}},
\end{equation}
for an appropriate constant $C = C(\delta) > 0$ and where $E[\psi_0,\psi_1]$ denotes a suitable weighted and higher order initial data energy. For generic initial data these rates are sharp along constant area radius hypersurfaces at a positive distance from the event horizon.
\end{thm}

We remark that there is an explicit characterisation for the class of data for which the rates in Theorems~\ref{aretakis_thm_1} and~\ref{aretakis_thm_2} are sharp in the cited works. Finally, along the extremal event horizon, for generic solutions higher-order transversal moments and higher do not decay:

\begin{thm}[Aretakis~\cite{aretakis1,aretakis2, aretakis3}] \label{aretakis_thm3}
Let $\psi$ be a solution to the wave equation on the extremal Reissner--Nordstr\"om exterior as above. Along the event horizon, the solution itself decays, while the first order transversal moment does not decay:
\begin{equation}
	\left| \psi \right|^2 \Big|_{r=m} \leq C_0 \frac{H_0[\psi]}{\tau^2}, \quad \int_{\sphere} T_W^{t^* t^*}[\psi] \, d \omega \Big|_{r=m} \geq C_1 (H_0[\psi])^2,
\end{equation}
for suitable constants $C_0,C_1 > 0$ and where the so-called horizon charge $H_0[\psi]$ is conserved along the event horizon and may be expressed in $(t^*,r)$-coordinates as
\begin{equation}
	H_0[\psi] = \frac{m^2}{4 \pi} \int_{\sphere} (\partial_{t^*} - \partial_r) (r \psi)|_{r=m} \, d \omega.
\end{equation}
In fact, higher order moments grow polynomially:
\begin{equation} \label{intro_compwave_growth}
\int_{\sphere} T_W^{t^* t^*}[\partial_r^{k-1} \psi] \, d \omega \Big|_{r=m} \geq C_k \tau^{2k-2} (H_0[\psi])^2, \quad k \geq 1,
\end{equation}
for suitable constants $C_k >0$. The growth~\eqref{intro_compwave_growth} is known as the Aretakis instability.
\end{thm}

Let us compare Theorems~\ref{aretakis_thm_1}-\ref{aretakis_thm3} with Remark~\ref{rem_expdecay_tensor}, Remark~\ref{rem_poldecay_tensor} and Theorem~\ref{maintheorem_ern_nondecay_rough}. We point out two key similarities between the wave equation and the massless Vlasov equation:
\begin{itemize}
	\item The rate of decay of transversal moments of a fixed order at a positive distance from the black hole is slower in the extremal case as compared to the subextremal case. Consider the case of zeroth-order moments as an example. For the massless Vlasov equation, $T^{t^* t^*}[f]$ decays at an exponential rate in the subextremal case, but only at a polynomial rate in the extremal case. Likewise, for the wave equation, $\left| \psi \right|^2$ decays at a slower polynomial rate in the extremal case as compared to the subextremal case.
	\item Along the event horizon of extremal Reissner--Nordstr\"om, zeroth-order moments decay, while first-order transversal moments (which involve one transversal derivative of the solution) are non-decaying. In fact, if we compare Theorem~\ref{aretakis_thm3} on the wave equation with the conjectured growth for higher order transversal moments for the massless Vlasov equation as stated in Remark~\ref{expected_growth}, we find that the rate of growth for higher-order transversal moments is the same for the wave equation and massless Vlasov equation.
\end{itemize}

We take note of a key difference between the proofs of Theorem~\ref{aretakis_thm3} and our Theorem~\ref{maintheorem_ern_nondecay_rough}. The proof of Theorem~\ref{aretakis_thm3} is based on an infinite hierarchy of conservation laws for solutions of the wave equation~\eqref{waveequn} on extremal Reissner--Nordstr\"om. Conservation laws do not feature in the proof of Theorem~\ref{maintheorem_ern_nondecay_rough}, however.


\subsection{Overview of the proofs}
In this subsection we explain the key ideas of the proofs of our main Theorems~\ref{maintheorem},~\ref{maintheoremERN} and~\ref{maintheorem_ern_nondecay_rough}. The argument rests on a precise understanding of the momentum support of a solution to the massless Vlasov equation. The core ideas are much the same for both the subextremal and extremal Reissner--Nordstr\"om spacetimes. We will therefore set out by explaining the main ideas for the subextremal case and point out the key differences in the extremal case later.

\subsubsection{Decay of momentum support implies decay of moments}
Consider a solution $f : \mathcal{P} \rightarrow \R$ to the massless Vlasov equation on the subextremal Reissner--Nordstr\"om exterior with compactly supported and bounded initial data $f_0$. Since $f$ satisfies a transport equation along future-directed null geodesics, the solution itself does not decay pointwise. Instead, as a first step, we estimate moments of the solution $f$ as follows
\begin{equation}
\left| \int_{\mathcal{P}_x} w f \, d \mu_x \right| \leq C(w) \| f_0 \|_{L^\infty} \int_{\supp{f(x, \cdot)}} 1 \, d \mu_x,
\end{equation}
where we made use of the fact that $\left| f \right| \leq \| f_0 \|_{L^\infty}$ since $f$ is transported along null geodesics and we have abbreviated $C(w) = \sup_{(x,p) \in \supp(f)} \left| w(x,p) \right|$. We will primarily be interested in the case of weights $w$ which are polynomial functions of the momentum. For such weights, if we assume that $f_0$ is compactly supported, we can show a quantitative upper bound on $C(w) < \infty$ in terms of the size of the initial support by making use of the fact that the momentum support of $f$ remains compact for all times. To prove decay of moments, it therefore suffices to show that the the volume of the momentum support of $f$ decays. We therefore aim to show that
\begin{equation}
\vol \, \supp{f(x, \cdot)} = \int_{\supp{f(x, \cdot)}} 1 \, d \mu_x \leq C e^{-c \tau(x)}, \quad \tau(x) \rightarrow \infty,
\end{equation}
for appropriate constants $C,c > 0$. The key to the proof of Theorem~\ref{maintheorem} is therefore to understand the decay of the phase-space volume of the momentum support $\supp{f(x, \cdot)} \subset \mathcal{P}_x$ of a solution $f$ as time tends to infinity.

\subsubsection{Geodesics in the momentum support are almost trapped}
By a careful study of the geodesic flow we will show that for points $x \in \Mrn$ with $\tau(x) \gg 1$, there are two types of momenta in the support of $f(x,\cdot)$. More formally, we prove that
\begin{equation} \label{supp_intro_ideaproof}
    \supp{f(x, \cdot)} \subset \trappedsupportsetx \cup \smallsupportsetx \subset \mathcal{P}_x,
\end{equation}
where the set $\trappedsupportsetx$ contains momenta of geodesics which are almost trapped at the photon sphere and $\smallsupportsetx$ contains momenta of geodesics which are almost trapped at the event horizon. We say that an affinely parametrised null geodesic is (exactly) trapped at a hypersurface if it is future complete and approaches the hypersurface as its affine parameter tends to infinity. We note that the set of geodesics which are exactly trapped at the photon sphere forms a co-dimension one submanifold of the mass-shell $\mathcal{P}$.\footnote{In other words, for each fixed point $x$ in the exterior, consider the set of null momenta $p \in \mathcal{P}_x$ with the property that the unique future-directed null geodesic with initial data $(x,p)$ is trapped at the photon sphere. Then this set of trapped momenta forms a two-dimensional subset of the three-dimensional cone $\mathcal{P}_x$.} A geodesic can only be trapped at the event horizon if it is a generator of the event horizon. The sets $\trappedsupportset$ and $\smallsupportset$ capturing the effect of trapping then turn out to be exponentially small neighbourhoods of the respective sets of exactly trapped geodesics. As a consequence, their phase-space volume is exponentially small in $\tau(x)$:
\begin{equation} \label{eqn_vol_trappedsets_intro}
\begin{gathered}
    \vol \trappedsupportsetx \sim e^{-c \tau(x)}, \\
    \vol \smallsupportsetx \sim e^{-c \tau(x)}.
\end{gathered}
\end{equation}
In order to prove~\eqref{supp_intro_ideaproof}, we require a quantitative estimate of how close to being trapped at the photon sphere or the event horizon a given geodesic which has not scattered to infinity or fallen into the black hole after a given time is.

\subsubsection{The almost-trapping estimate} \label{subsubsec_geodflow}
Consider a point $(x,p) \in \supp(f)$ with $\tau(x) \gg 1$ and let $\gamma$ denote the unique null geodesic determined by it. Since $f$ solves the massless Vlasov equation, $\gamma$ must intersect $\supp(f_0)$. Since the time function $\tau$ connects the event horizon and future null infinity, $\gamma$ has neither scattered to infinity nor fallen into the black hole before time $\tau(x)$, thus $\gamma$ can be thought of as \emph{almost trapped}. Understanding the momentum support $\supp{f(x,\cdot)}$ now amounts to estimating the components of $p = \dot{\gamma}$ in terms of initial data and $\tau(x)$. \\

The key to accomplishing this is the \emph{almost-trapping estimate}, which bounds the time that the geodesic $\gamma$ requires to cross a certain region of spacetime in terms of how close the geodesic is to being trapped. To make this precise, we introduce the trapping parameter $\trapschw : \mathcal{P} \rightarrow [-\infty,1]$ in Definition~\ref{def_eps_def}. The trapping parameter is conserved along the geodesic flow and measures how far a geodesic is from being trapped at the photon sphere. In particular a null geodesic $\gamma$ is trapped at the photon sphere if and only if $\trapschw(\gamma,\dot{\gamma}) = 0$. The \emph{almost-trapping estimate} may then be stated as
\begin{equation} \label{eqn_almosttrapping_intro}
	\tau(x) \leq C \left( 1 + ( \log \left| \trapschw \right| )_- + ( \log \, [(1+\left| \trapschw \right|) \OsqS(r_0) ] )_- \right),
\end{equation}
where $\trapschw = \trapschw(\gamma,\dot{\gamma})$ and $\gamma$ intersects $\Sigma_0$ at radius $r_0$.  We provide a sketch of the proof in Section~\ref{sec_radialgeod}. \\

The term $( \log \left| \trapschw \right| )_-$ arises from trapping at the photon sphere, while $( \log \, [(1+\left| \trapschw \right|) \OsqS(r_0) ] )_-$ arises from trapping at the event horizon. Indeed, assume $\tau(x) \geq 2C$. The inequality~\eqref{eqn_almosttrapping_intro} implies $( \log \left| \trapschw \right| )_- \geq c \tau(x)$ or $ ( \log \, [(1+\left| \trapschw \right|) \OsqS(r_0) ] )_- \geq c \tau(x)$, where $c =( 4C)^{-1}$. Let us consider the case where $( \log \left| \trapschw \right| )_- \geq c \tau(x)$ in some more detail. We find $ \left| \trapschw \right| \leq e^{- c\tau(x)}$ and thus $\gamma$ is exponentially close to being trapped at the photon sphere. By definition of the set $\trappedsupportsetx$, this implies $p = \dot{\gamma} \in \trappedsupportsetx$. We use the mass-shell relation in combination with the assumption that $\supp(f_0)$ is compact to bound the size of the components of $p$. This allows us to estimate the volume of the set $\trappedsupportsetx$. In the latter case, where $( \log \, [(1+\left| \trapschw \right|) \OsqS(r_0) ] )_- \geq c \tau(x)$, we readily conclude $(1+\left| \trapschw \right|) \OsqS(r(0)) \leq e^{-c\tau(x)}$, which may be shown to correspond to $\gamma$ being exponentially close to being trapped at the event horizon and $p = \dot{\gamma} \in \smallsupportsetx$. We argue analogously to the first case and thereby prove estimate~\eqref{eqn_vol_trappedsets_intro}.

%We next explain the interpretation of the almost trapping estimate~\eqref{eqn_almosttrapping_intro} and use it to prove~\eqref{eqn_vol_trappedsets_intro}.

%In a neighbourhood of the black hole, we use $(t^*,r)$-coordinates (see Section~\ref{rngeometry}) to define the time function as $\tau = t^*$. Assume that $\gamma$ is affinely parametrised in such a way that $(\gamma(0),\dot{\gamma}(0)) \in \supp(f_0)$ and $(\gamma(s),\dot{\gamma}(s)) = (x,p)$. If we use $(t^*,r)$-coordinates to express $\gamma = (t^*,r,\omega)$ and $\dot{\gamma} = (p^{t^*},p^r,\pslash)$ then
%\begin{equation}
%    t^*(s) - t^*(0) = \int_{0}^{s} \frac{dt^*}{d s} \, d s = \int_{0}^{s} p^{t^*} \, d s = \int_{r(0)}^{r(s)} \frac{p^{t^*}}{p^r} \, d r.
%\end{equation}
%Here we made use of the fact that the geodesic equations readily imply that there is at most one point where $p^r = 0$ and we split the integral along this point if necessary.

\subsubsection{Key differences between the extremal and subextremal case}
We next point out the similarities and differences between the null geodesic flow on the subextremal and extremal Reissner--Nordstr\"om exterior. While the structure of trapping at the photon sphere remains virtually identical, there are two key differences concerning trapping at the event horizon:
\begin{itemize}
	\item Outgoing null geodesics, which are almost trapped at the event horizon, leave the region close to the black hole at a slower rate in the extremal case as compared to the subextremal case, due to the subextremal geodesics being \emph{red-shifted}. This phenomenon is the reason that moments of solutions to the massless Vlasov equation decay at a slower rate in the extremal as compared to the subextremal case. We illustrate this for the special case of radial geodesics in Section~\ref{sec_radialgeod} below.
	\item There exist ingoing null geodesics in the extremal Reissner--Nordstr\"om spacetime which are almost trapped at the event horizon but fall into the black hole after a finite time. These geodesics are unique to the extremal case and are key to proving non-decay of transversal derivatives of moments along the event horizon. Such geodesics are not radial, so the discussion is deferred until Section~\ref{psupporteventhorizon}.
\end{itemize}

\subsubsection{Proof of  the almost-trapping estimate} \label{sec_radialgeod}
In this subsection we obtain the almost-trapping estimate for the special case of radial geodesics. We discuss both the subextremal estimate~\eqref{eqn_almosttrapping_intro} and its extremal analogue~\eqref{eqn_almosttrapping_intro_ERN}, which is stated in Section~\ref{into_sec_extremal_upperbounds} below. This will serve to clarify the strategy of proof and to point out a key difference between the subextremal and extremal case. We invite the reader to compare the form of estimates~\eqref{eqn_almosttrapping_intro} and~\eqref{eqn_almosttrapping_intro_ERN}. Note that radial geodesics cannot be trapped at the photon sphere. Therefore, only the rightmost term on the right hand side of~\eqref{eqn_almosttrapping_intro} respectively~\eqref{eqn_almosttrapping_intro_ERN} is relevant for radial geodesics. However, radial geodesics exhibit the phenomenon of trapping at the event horizon in a simple way. See Figure~\ref{penrose_radial} for a visual guide to the behaviour of outgoing radial null geodesics. \\

\begin{figure}
	\centering
	\includegraphics[scale=0.4]{figures/penrose_diagram_radial_geodesics.png}
	\caption{Propagation of radial null geodesics in the Penrose diagram of the Reissner--Nordstr\"{o}m exterior, see also Figure~\ref{penrose_tau} above. Several outgoing radial null geodesics $\gamma$ are depicted (dashed lines), together with the $\Sigma_\tau$-hypersurfaces which they coincide with in the region $r \geq R$. The initial distance from the event horizon is $\delta$. The time parameter $\tau$, for which $\gamma$ intersects $\Sigma_\tau$ at the area radius $r = R$ depends only on $\delta$ and satisfies $\tau \rightarrow \infty$ as $\delta \rightarrow 0$. The quantitative relation between $\delta$ and $\tau$ depends on whether the black hole is extremal or subextremal, see equation~\eqref{ineq_radial_time}.}
	\label{penrose_radial}
\end{figure}

%We remind the reader of the mixed spacelike-null nature of the time function $\tau$, see Figure~\ref{penrose_tau}. We use the mass-shell relation in combination with considerations in the asymptotically flat region to establish the \emph{almost-trapping estimate} \\

%We begin by recalling that the symmetries of the Reissner--Nordstr\"om metric are associated with conserved quantities along the geodesic flow: the energy $E$ (time invariance) and total angular momentum $L$ (spherical symmetry).  \\

Consider an affinely parametrised future-directed radial null geodesic $\gamma$ on the subextremal or extremal Reissner--Nordstr\"om exterior. We express $\gamma$ in $(t^*,r)$-coordinates as $\gamma(s) = (t^*(s),r(s),\omega(s))$ and its momentum as $\dot{\gamma}(s) = (p^{t^*}(s),p^r(s),\pslash(s))$. Since $\gamma$ is radial, we have $\left| \pslash(s) \right|_\gslash = 0$ for all $s$. We assume in addition that $\gamma$ is outgoing, so that $p^r(0) > 0$ and that it intersects $\Sigma_0$ at a distance $\delta$ from the event horizon, so $r(0) - r_+ = \delta$. If $\delta = 0$, then $\gamma$ is trapped at the event horizon, which means that $\gamma(s) \in \mathcal{H}^+$ for all $s \geq 0$. We may think of $\delta$ as measuring the distance to being trapped at the event horizon. Let us therefore assume that $\delta > 0$ is small. Using that $\gamma$ is radial, the geodesic equations reduce to
\begin{equation}
	\dot{p}^{t^*} = - \frac{1}{2} \frac{d \Osqrn}{d r} \left( p^{t^*} + p^r \right)^2, \quad \dot{p}^r = 0.
\end{equation}
Making use of the fact that $p^r(0) > 0$, we conclude that for all affine parameters $s \geq 0$,
\begin{equation} \label{eqn_radial_explicit_soln}
	p^r(s) = p^r(0), \quad p^{t^*}(s) = \frac{2-\Osqrn(r(s))}{\Osqrn(r(s))} p^r(s)	.
\end{equation}
By solving the geodesic equations in double null coordinates (introduced in Section~\ref{altcoordsystems}) or alternatively by visual inspection of Figure~\ref{penrose_radial}, we conclude that radial null geodesics propagate along the hypersurfaces $\Sigma_\tau \cap \{ r \geq R \}$ in the region where $r \geq R$. Therefore, we need only estimate the time $\gamma$ requires to travel outward from its initial radius $r(0)$ to the radius $R > r_+$. We follow the method outlined in Section~\ref{subsubsec_geodflow} above combined with equation~\eqref{eqn_radial_explicit_soln} to find
\begin{align} \label{eqn_overview_time}
	t^*(s) - t^*(0) &= \int_{r(0)}^{r(s)} \frac{p^{t^*}}{p^r} \, dr \sim  \int_{r(0)}^{R} \frac{1}{\Osqrn} \, dr \sim \int_{r(0)-r_+}^{R-r_+} \frac{1}{x(2\kappa+x)} \, dx \\
	&= \left[ \frac{\log x - \log(x+2\kappa)}{\kappa} \right]_{\delta}^{R-r_+}  \sim \frac{\log (\delta + 2\kappa) - \log(\delta)}{\kappa},
\end{align}
where we say $a \sim b$ if there exist constants $C_2 > C_1 > 0$ such that $C_1 a \leq b \leq C_2 a$ and where we have also introduced the \emph{surface gravity}
\begin{equation}
\kappa = \frac{1}{2} \frac{d}{dr} \Osqrn \bigg|_{r=r_+}.
\end{equation}
The surface gravity allows us to approximate the lapse $\Osqrn \approx (r-r_+)(r-r_+ + 2\kappa) = x (x+2 \kappa)$, where we introduced the change of coordinates $x = r-r_+$. Crucially, the surface gravity vanishes in the extremal case and is positive in the subextremal case, see Section~\ref{rnmetricmanifold} below. We conclude
\begin{equation} \label{ineq_radial_time}
	\tau(\gamma(s)) \sim \begin{cases}
		C(\kappa) \left|\log \delta \right| & \kappa > 0 \\
		C \delta^{-1} & \kappa = 0
	\end{cases}.
\end{equation}
This establishes the almost-trapping estimates~\eqref{eqn_almosttrapping_intro} respectively~\eqref{eqn_almosttrapping_intro_ERN} for radial geodesics. For non-radial geodesics, we follow a similar approach and consider the first equality in~\eqref{eqn_overview_time}. In lieu of solving the geodesic equations explicitly, we use the mass-shell relation to estimate the quotient $p^{t^*}/p^r$ as a function of the trapping parameter $\trapschw$ and radius $r$. Estimating the resulting integral leads to the non-radial analogue of~\eqref{eqn_overview_time}. In the asymptotically flat region, we show similarly that null geodesics propagate approximately along hypersurfaces $\Sigma_\tau \cap \{ r \geq R \}$. Finally we combine the non-radial analogue of~\eqref{eqn_overview_time} with the considerations in the asymptotically flat region to obtain the almost-trapping estimate. We treat the general case in Lemma~\ref{taubound} for the subextremal case and Lemma~\ref{tauestimate_ERN} for the extremal case. \\

Finally we discuss how to bound the size of the momentum components, which is the key step in establishing~\eqref{eqn_vol_trappedsets_intro} respectively its extremal analogue~\eqref{eqn_vol_trappedsets_intro_ERN}. We make use of the fact that $\gamma$ is assumed to intersect the compact support of the initial distribution, so that $0 \leq p^r(0), p^{t^*}(0) \leq \psuppconst$ for some constant $\psuppconst > 0$. Combined with equation~\eqref{eqn_radial_explicit_soln} this implies
\begin{equation}
	\left| p^r(s) \right| = \left| p^r(0) \right| \leq \Osqrn(r(0)) p^{t^*}(0) \leq (r(0)-r_+)(r(0)-r_+ + 2 \kappa) p^{t^*}(0) \leq \delta (2 \kappa + \delta) \psuppconst.
\end{equation}
We now combine this bound with inequality~\eqref{ineq_radial_time}, which relates $\delta$ and $\tau(\gamma(s))$ to find
\begin{equation} \label{eqn_prbound_intro}
	\left| p^r(s) \right| \leq \begin{cases}
		C(\kappa) \psuppconst \delta \; \leq \; C(\kappa) \psuppconst e^{-c \tau(\gamma(s))} & \kappa > 0 \\
		C \psuppconst \delta^2 \, \quad \leq \; C \psuppconst \tau(\gamma(s))^{-1} & \kappa = 0
	\end{cases}.
\end{equation}
Inequality~\eqref{eqn_prbound_intro} is the key ingredient for proving estimate~\eqref{eqn_vol_trappedsets_intro} in the subextremal case, respectively its extremal analogue~\eqref{eqn_vol_trappedsets_intro_ERN}.

%\begin{equation}
%	p^v(s) = \frac{2}{\Osqrn} p^r(s), \quad p^u(s) = 0.
%\end{equation}

\subsubsection{The extremal case: Upper bounds} \label{into_sec_extremal_upperbounds}
The strategy of proof remains much the same in the subextremal case. At the heart of the argument lies again the almost-trapping estimate, which in the extremal case takes the form
\begin{equation} \label{eqn_almosttrapping_intro_ERN}
	\tau(x) \leq C \left( 1 + ( \log \left| \trapschw \right| )_- + \frac{1}{\sqrt{\OsqS(r_0)}} \right).
\end{equation}
We argue in a similar fashion to the subextremal case to obtain an analogue of inclusion~\eqref{supp_intro_ideaproof}, with two analogous sets $\ERNtrappedsupportsetx, \ERNsmallsupportsetx \subset \mathcal{P}_x$. The structure and volume of the set $\ERNtrappedsupportsetx$ of geodesics which are almost trapped at the photon sphere remains virtually identical to the subextremal case. Crucially however, the volume of the set $\ERNsmallsupportsetx$ of geodesics almost trapped at the event horizon only decays polynomially in $\tau(x)$:
\begin{equation} \label{eqn_vol_trappedsets_intro_ERN}
\begin{gathered}
    \vol \trappedsupportsetx \sim e^{-c \tau(x)}, \\
    \vol \smallsupportsetx \sim \frac{1}{\tau(x)^2}.
\end{gathered}
\end{equation}
This can already be appreciated at the level of radial geodesics, as shown in Section~\ref{sec_radialgeod}. The slower decay of the volume of $\smallsupportsetx$ directly translates into a slower rate of decay for moments of solutions. An additional key difference is that while in the subextremal case all momentum components decay, in the extremal case the $p^{t^*}$-component can only be shown to be bounded along the event horizon.

\subsubsection{The extremal case: Lower bounds}
In order to show lower bounds for moments of a solution $f$, we utilise the existence of slowly infalling geodesics to construct a family of geodesics crossing the event horizon at arbitrarily late times. The initial data of this family of geodesics is captured in the family of subsets $\badsetaplarge_\delta \subset \mathcal{P}|_{\Sigma_0}$. Consider a point along the event horizon $x \in \mathcal{H}^+$ with $\tau(x) \gg 1$. Then the set of all geodesics intersecting $\badsetaplarge_\delta$ and the point $x$ populates a subset $\slowsupportsetx \subset \smallsupportsetx$. This subset satisfies $\vol \slowsupportsetx \sim \vol \smallsupportsetx$ and any momentum $p \in \slowsupportsetx$ satisfies $p^{t^*} \sim 1$. This demonstrates that the upper bounds obtained before are in fact sharp, as long as $\badsetaplarge_\delta \subset \supp {f(x,\cdot)}$ for some $\delta > 0$.

\subsubsection{The extremal case: Growth for transversal derivatives} \label{subsec_growth_overview}

Finally, the proof of Theorem~\ref{maintheorem_ern_nondecay_rough} proceeds by first commuting the transversal $\partial_r$-derivative inside the energy momentum tensor in equation~\eqref{eqn_lowerbound_der_ern}. This incurs only lower order error terms which can be controlled. We then use the massless Vlasov equation to argue that, up to lower order error terms
\begin{equation}
\int_{\sphere} \int_{\mathcal{P}_x} \partial_r f \, d \mu_x d \omega = \int_{\sphere} \int_{\mathcal{P}_x} \frac{p^{t^*}}{p^r} \partial_{t^*} f \, d \mu_x d \omega + \dots,
\end{equation}
where $\partial_{t^*}$ is the stationary Killing derivative expressed in $(t^*,r)$-coordinates. Notice that $\partial_{t^*} f$ is again a solution to the massless Vlasov equation if $f$ is a solution. Now Theorem~\ref{maintheorem_ern_nondecay_rough} follows essentially by noticing that any $p = (p^{t^*},p^r,\pslash) \in \slowsupportsetx$ expressed in $(t^*,r)$-coordinates satisfies $p^{t^*} \sim 1$ and $\left| p^r \right| \leq C \tau(x)^{-2}$. Therefore the growth of the fraction $\frac{p^{t^*}}{\left| p^r \right|}$ exactly cancels the decay of the volume of the momentum support.

\subsubsection{Outlook: The Kerr spacetime}
We have already alluded to the Kerr spacetime in the introduction. The geodesic flow on the subextremal and extremal Kerr exterior is similarly determined in full by conserved quantities. In Kerr with $a \neq 0$ one encounters the difficulty that the energy associated to the stationarity is non-positive in the ergoregion, a phenomenon known as \emph{superradiance}. We nonetheless expect that the same principles of proof will apply to the subextremal Kerr solution. On the extremal Kerr spacetime one encounters further the fundamental phenomenon that trapping and superradiance do not decouple. This is the main challenge in understanding the behaviour of linear fields outside the axisymmetric class on extremal Kerr. We hope that the massless Vlasov equation, along with the techniques introduced in the present work, will provide insight into this phenomenon.

\subsection{Related work}

\subsubsection{Related results on the Vlasov equation}
We have already mentioned the work of Bigorgne~\cite{leo}, Velozo~\cite{renato} and Andersson--Blue--Joudioux~\cite{anderssonblue} on the massless Vlasov equation in Section~\ref{subsec_expdecay}. In the asymptotically flat case, outside the realm of black holes, the Minkowski spacetime has been shown to be stable as a solution to the coupled massless Einstein--Vlasov system. This was first accomplished by Taylor~\cite{martin} and subsequently by Joudioux--Thaller--Kroon~\cite{minkstab2} and Bigorgne--Fajman--Joudioux--Smulevici--Thaller~\cite{fajman}. These works were preceded by the result of Dafermos~\cite{mihalisSpherSymmVlasov}, who demonstrated stability of Minkowski as a solution to massless Einstein--Vlasov under the assumption of spherical symmetry. In the massive case, the first result is due to Rein--Rendall~\cite{reinrendall}, who showed stability of Minkowski space as a solution to the spherically symmetric massive Einstein--Vlasov system. Lindblad--Taylor~\cite{lindbladtaylor} and independently Fajman--Joudioux--Smulevici~\cite{fajman2} established stability of Minkowski space as a solution to the \emph{massive} Einstein--Vlasov system without assumptions of symmetry. \\

For the connection to experimental results, we point out the work of Bieri--Garfinkle~\cite{bierigarfinkle}, who model neutrino radiation in general relativity as a coupled Einstein--null-fluid system and show that neutrino radiation enlarges the Christodoulou memory effect~\cite{christodoulougravwaves} of gravitational waves. The null-fluid matter model can be regarded as a limiting case of the massless Vlasov system.


\subsubsection{Related results on instability of extremal black holes}
We have discussed the work of Aretakis~\cite{aretakis1,aretakis2,aretakis3} on the horizon instability of extremal Reissner--Nordstr\"om for the wave equation in Section~\ref{section_aretakis} above. Since then, there has been a tremendous amount of progress expanding these results. Angelopoulos--Aretakis--Gajic~\cite{nonlinearwavesERN} have since shown that the Aretakis instability holds for a class of nonlinear wave equations satisfying the null condition on the extremal Reissner--Nordstr\"om spacetime. Recently, Apetroaie~\cite{apetroaie} has shown that instability results along the event horizon of extremal Reissner--Nordstr\"om persist for the linearised Einstein equations. Aretakis~\cite{aretakisgeneralhorizons} has generalised the instability result for the wave equation to a more general class of axisymmetric extremal horizons, which includes as special cases the event horizons of the extremal Reissner--Nordstr\"om and extremal Kerr spacetimes. Teixeira~da~Costa~\cite{rita} has shown that despite the presence of a linear instability, there are no exponentially growing modes for the Teukolsky equation on extremal Kerr black holes. Lucietti--Reall~\cite{reall} have extended the conservation law that lies at the heart of the horizon instability to any extremal black hole and a larger class of theories in several dimensions. \\

For the extremal Kerr spacetime, Aretakis~\cite{subextkerr} has obtained boundedness and decay of solutions to the wave equation under the assumption of axisymmetry. A new instability result for higher azimuthal mode solutions to the wave equation on extremal Kerr has recently been obtained by Gajic~\cite{dejanforthcoming}, who proved the existence of stronger asymptotic instabilities for non-axisymmetric solutions to the wave equation. Understanding linear fields in generality, in particular including the regime in which trapping meets superradiance, is one of the biggest open problems in extremal black hole dynamics. \\

Despite the fact that massless linear fields on extremal Reissner--Nordstr\"om are well understood, an understanding of the behaviour of the solution on the nonlinear level is still lacking. Dafermos--Holzegel--Rodnianski--Taylor~\cite{schwarzschildnonlinearstable} conjecture a weak form of nonlinear asymptotic stability of the extremal Reissner--Nordstr\"om spacetime as a solution to the Einstein--Maxwell equations. More precisely, they conjecture that on a suitable finite codimension `submanifold' of the moduli space of initial data the maximal Cauchy development asymptotes an extremal Reissner--Nordstr\"om solution, while satisfying only weaker decay along the event horizon and suitable higher order quantities blow up polynomially along the event horizon (growing `horizon hair').

\subsection{Outline}
In Section~\ref{prelim} we discuss some preliminaries on the massless Vlasov equation and the geometry of the subextremal and extremal Reissner--Nordstr\"om solution. In Section~\ref{section_mainthms} we state our main Theorems~\ref{maintheorem},~\ref{maintheoremERN},~\ref{maintheorem_ern_nondecay_rough} in a more precise form. Section~\ref{section_proof} is devoted to the subextremal case and the proof of Theorem~\ref{maintheorem}. In Section~\ref{section_ERN}, we turn towards the extremal case and provide the proofs of Theorems~\ref{maintheoremERN} and~\ref{maintheorem_ern_nondecay_rough}.