\section{Geometry of the Reissner--Nordstr\"om solution} \label{prelim}
In this section we discuss the necessary preliminaries and definitions to state and prove our main theorems in a more precise way. We introduce the Reissner--Nordstr\"om family of solutions and express the metric in several coordinate systems which we shall use throughout the work in Section~\ref{rngeometry}. We then define the time function used to measure decay in Section~\ref{timecoordinate} and introduce the massless Vlasov equation on the Reissner--Nordstr\"om exterior in Section~\ref{sec_vlasov}. In the final subsection~\ref{sec_subsets} we introduce a family of subsets of the mass-shell which will play a crucial role in characterising the momentum support of a solution to the massless Vlasov equation in the proofs.

\subsection{The Reissner--Nordstr\"om metric} \label{rnmetricmanifold} \label{rngeometry}
The Reissner--Nordstr\"om metric is a stationary spherically symmetric solution to the Einstein--Maxwell equations and models the exterior of a non-rotating charged black hole. See~\cite{oneill, waldbook, dafermosETHnotes, aretakisbook} for a thorough discussion of the geometric concepts used here. \\

Let $\sphere$ denote the standard $2$-sphere equipped with the standard round metric $d \omega^2$. For the most part we will not require an explicit choice of coordinates on the sphere $\sphere$, although we will occasionally make use of the standard spherical coordinates $(\theta,\phi)$. If we do not require explicit coordinates on the $2$-sphere we will denote points on the sphere by $\omega \in \sphere$. \\

We introduce the exterior of the Reissner--Nordstr\"om black hole in $(t^*,r)$-coordinates, which shall be the coordinate system we will mainly make use of in this discussion. In these coordinates the metric takes the form
\begin{equation}
\gRN = - \Osqrn (dt^*)^2 + 2 (1-\Osqrn) dt^* dr + (2-\Osqrn) dr^2 + r^2 d \omega^2, \quad \Osqrn = 1 - \frac{2m}{r} + \frac{q^2}{r^2},
\end{equation}
where the parameter $m > 0$ represents the black hole's mass and $\left| q \right| \leq m$ represents the black hole's electrical charge. We denote the spherical part of the metric by $\gslash = r^2 d \omega^2$. Note that $\Osqrn$ has two roots which we denote by $0 < r_- \leq r_+$. In this study we are concerned with the exterior of the black hole, which is the manifold with boundary
\begin{equation}
(t^*,r,\omega) \in \Mrn = \R \times [r_+,\infty) \times \sphere.
\end{equation}
The stationarity of the metric gives rise to the timelike Killing vector field $T =  \partial_{t^*}$, while the spherical symmetry induces the usual rotational Killing vector fields $\{ \Omega_i \}_{i=1,2,3}$ generating a representation of the Lie algebra $\text{so}(3)$. We call the null hypersurface
\begin{equation}
\mathcal{H}^+ = \Big\{ r = r_+, t^* > 0 \Big\} \subset \Mrn
\end{equation}
the future event horizon of the black hole. The normal to $\mathcal{H}^+$ is easily computed to be the stationary timelike Killing vector field $T$. A computation reveals that along $\mathcal{H}^+$
\begin{equation}
    \nabla_T T = \kappa T, \quad \kappa = \frac{1}{2} \frac{d \Osqrn}{dr}\bigg|_{r=r_+} .
\end{equation}
We say that the solution is subextremal when $\kappa > 0$ or equivalently $r_- < r_+ \iff \left| q \right| < m$ and it is extremal when $\kappa = 0$ or equivalently $r_- = r_+ \iff \left| q \right| = m$. See also~\cite[Section 1.5]{aretakisbook} for an in-depth discussion of the notions of extremal and subextremal black holes. The timelike hypersurface $\left\{ r = r_{ph} , t^* > 0 \right\} \subset \Mrn$ with 
\begin{equation}
    r_{ph} = \frac{3m}{2}\left( 1 + \sqrt{1- \frac{8 q^2}{9 m^2}} \right)
\end{equation}
is called the photon sphere of the black hole. Both the event horizon and the photon sphere permit trapped null geodesics, see Section~\ref{prelim_massshell_estimates} below. This trapping effect is key to understanding the decay of moments of solutions to the massless Vlasov equation. \\

From Section~\ref{section_mainthms} onward we will be mostly concerned with two special cases of the Reissner--Nordstr\"om family: the Schwarzschild solution, obtained by setting $q=0$ and the extremal Reissner--Nordstr\"om (ERN) solution, obtained by setting $\left| q \right| = m$. For the reader's convenience, we summarise the relevant metric and geometric quantities here:
\begin{align}
    \text{Schwarzschild: }\quad &\Osqrn = 1 - \frac{2m}{r}, & &r_+ = 2m, & &r_{ph} = 3m, & & \kappa = \frac{1}{4m}, \label{table_geom_values_schw}\\
    \text{ERN: }\quad &\Osqrn = \left( 1 - \frac{m}{r} \right)^2, & &r_+ = m, & &r_{ph} = 2m, & & \kappa = 0 . \label{table_geom_values_ERN}
\end{align}

\begin{rem}
By a slight abuse of notation, we will use the same symbols to denote any geometric quantities related to either the Schwarzschild or the extremal Reissner--Nordstr\"om solution. It will be apparent from the context which member of the Reissner--Nordstr\"om family we are referring to. Thus, if we are discussing the extremal Reissner--Nordstr\"om solution, then $\Osqrn, r_+, r_{ph}$ etc. should all be set to their corresponding values given in~\eqref{table_geom_values_ERN} above. Likewise for the Schwarzschild solution with~\eqref{table_geom_values_schw}.
\end{rem}

\subsubsection{Alternative coordinate systems} \label{altcoordsystems}
Let us restrict to the region $\{r > r_+ \} \subset \Mrn$ and apply the coordinate transformation
\begin{equation}
t^* = t + F(r), \quad \frac{dF}{dr} = \frac{1-\Osqrn}{\Osqrn}, \quad \lim_{r \rightarrow \infty} F(r) = 0.
\end{equation}
In this way we obtain the local Boyer--Lindquist coordinates $(t,r,\omega)$, which are defined only in the range $t \in \R,r>r_+, \omega \in \sphere$. In these coordinates the metric may be expressed as
\begin{equation}
\gRN = - \Osqrn dt^2 + \OsqrnNEG dr^2 + r^2 d \omega^2.
\end{equation}
We may solve for $F$ explicitly. We provide the closed form of $F$ for the Schwarzschild exterior and the extremal Reissner--Nordstr\"om exterior:
\begin{equation}
    F(r) = \begin{cases}
        2m \log(r-2m) & q = 0 \\
        2m \log(r-m) - \frac{m^2}{r-m} & \left| q \right| = m
    \end{cases}.
\end{equation}
We note that the metric may be smoothly extended to the region where $0 < r < \infty$, so that it is the Boyer--Lindquist coordinate system that becomes singular as $r \rightarrow r_+$, not the spacetime itself. \\

Let us introduce two further useful coordinate systems, Regge--Wheeler coordinates and double null coordinates. If $(t,r,\omega)$ denote Boyer--Lindquist coordinates as above, we define
\begin{equation} \label{eqn_rstar_defn}
    \frac{d r^*}{dr} = \frac{1}{\Osqrn}, \quad r^*(r_{ph}) = 0.
\end{equation}
to obtain Regge--Wheeler coordinates $(t,r^*,\omega)$. The exterior excluding the event horizon then corresponds to the region $(t,r^*,\omega) \in \R \times \R \times \sphere$ and the metric takes the form
\begin{equation}
\gRN = \Osqrn \left( - dt^2 + (d r^*)^2 \right) + r^2 d \omega^2.
\end{equation}
We may solve equation~\eqref{eqn_rstar_defn} explicitly and find that in the special cases of the Schwarzschild and extremal solutions
\begin{equation}
r^* = \begin{cases}
        r + 2m \log(r-2m) -3m - 2m \log m & q = 0 \\
        r + 2m \log(r-m) - \frac{m^2}{r-m} - 2m \log(m) - m & \left| q \right| = m
    \end{cases}.
\end{equation}
For the double null coordinate system, we use Regge--Wheeler coordinates and define
\begin{equation} \label{regwheelertodoublenull}
u = t-r^*, \quad v = t+r^*,
\end{equation}
to obtain the double null coordinate system $(u,v,\omega)$. The coordinates are defined over the same region and the metric takes the form
\begin{equation}
\gRN = -\Osqrn du dv + r^2 d \omega^2.
\end{equation}
We refer to the idealised null hypersurface $\mathcal{I}^+ = \{ v = \infty \}$ as future null infinity. Let us emphasise again at this point that all the coordinate systems discussed in this section break down at the event horizon.

\subsection{The time function $\tau$} \label{timecoordinate}
Let us define the specific time function with respect to which we will state the decay result. We choose a function $\tau: \Mrn \rightarrow \R$ such that $\nabla \tau$ is future-directed causal and such that the hypersurfaces $\Sigma_{\tau'} = \{ x \in \Mrn \, | \, \tau(x) = \tau' \}$ are spherically symmetric Cauchy hypersurfaces and connect the event horizon $\mathcal{H}^+$ and future null infinity $\mathcal{I}^+$. For concreteness, consider the following
\begin{defi} \label{defn_timecoord_Schw}
Let $R > r_{ph}$ and define the time function $\tau: \Mrn \rightarrow \R$ by
\begin{equation}
\tau(x) = \begin{cases}
	t^*(x) & r_+ \leq r(x) \leq R \\
	u(x) + c(R) & R \leq r(x)
\end{cases},
\end{equation}
where we use $(t^*,r)$-coordinates in the region $r_+ \leq r \leq R$ and double null coordinates $(u,v)$ in the region $r \geq R$. The constant $c(R)$ is chosen to ensure continuity of $\tau$. Explicitly in the Schwarzschild respectively extremal Reissner--Nordstr\"om spacetime
\begin{equation}
    c(R) = \begin{cases}
        R + 4m \log(R-2m) - 3m - 2m \log(m) & q = 0 \\
        R + 4m \log(R-m) - \frac{2m^2}{R-m} - 2m \log(m) - m & \left| q \right| = m
    \end{cases} .
\end{equation}
We denote hypersurfaces of constant $\tau$ by $\Sigma_\tau$, see Figure~\ref{penrose_tau}.
\end{defi}

Note carefully that we have only chosen $\tau$ to be continuous. The derivative $\nabla \tau(x)$ is not defined at points $x \in \Mrn$ where $r(x) = R$. However both a left-sided and a right-sided derivative of $\tau$ may be defined and they are both causal. It is with respect to $\tau$ that we will state the decay of the momentum support of $f$ below. We will subsequently refer to the region $\mathcal{A} := \{ r \geq R \}$ as the asymptotically flat region, or in other words the region which is in some sense far away from the black hole. We will frequently suppress the argument of $\tau$ when it is clear from the context that we are referring to $\tau(x)$ for a given $x \in \Mrn$.

\subsection{The mass-shell and the massless Vlasov equation} \label{sec_vlasov}
In this subsection we introduce the massless Vlasov equation on the Reissner--Nordstr\"om exterior $\Mrn$ and discuss some important associated definitions.

\subsubsection{The geodesic flow on Reissner--Nordstr\"om}
Let us denote by $\{ S_s \}_{s \in \R}$ the geodesic flow on $\Mrn$, so that for any point $(x,p) \in T \Mrn$
\begin{equation} \label{geodesicflow}
S_s(x,p) = (\gamma_{(x,p)}(s),\dot{\gamma}_{(x,p)}(s)),
\end{equation}
where $\gamma_{(x,p)}$ is the unique affinely parametrised geodesic with $\gamma_{(x,p)}(0) = (x,p)$ and we only define $S_s(x,p)$ if $\gamma_{(x,p)}(s) \in \Mrn$. The generator of the geodesic flow is a vector field on $TM$ called the geodesic spray $X$. Let $(x^\mu)$ denote one of the coordinate systems on the exterior $\Mrn$ defined above. Then we can obtain an associated coordinate system $(x^\mu, p^\mu)$ on the tangent bundle $T \Mrn$ by representing any tangent vector $p \in T_x \Mrn$ as $p = p^\mu \partial_\mu$, so that $(x,p) \in T \Mrn$ has the coordinates $( x^\mu, p^\mu )$. In the coordinates on $T \Mrn$ associated to $(x^\mu)$ the geodesic spray takes the form
\begin{equation}
	X = p^\mu \partial_\mu - \Gamma^\mu_{\alpha \beta} p^\alpha p^\beta \partial_{p^\mu},
\end{equation}
where $\Gamma^\mu_{\alpha \beta}$ denote the Christoffel symbols of $g$ in $(x^\mu)$-coordinates.

\subsubsection{The mass-shell and massless Vlasov equation}
The bundle of future light-cones or mass-shell on $(\Mrn,\gRN)$ is defined as
\begin{equation}
\mathcal{P} = \Big\{ (x,p) \in T \Mrn : \, \gRN(x)(p,p)=0, \, p \text{ is future-directed} \Big\} .
\end{equation}
The condition $\gRN(x)(p,p)=0$ is referred to as the mass-shell relation. If we use $(t^*,r)$-coordinates $(x^\mu) = (t^*,r,\theta_1,\theta_2)$ with an explicit choice of coordinates $(\theta_1,\theta_2)$ on $\sphere$, we denote the conjugate coordinates on the tangent bundle by $(x^\mu,p^\mu) = (t^*,r,\theta_1,\theta_2,p^{t^*},p^r,p^{\theta_1},p^{\theta_2})$. Often times, we do not require an explicit choice of coordinates on $\sphere$. In this case, we simply denote points on $\sphere$ by $\omega = (\theta_1,\theta_2)$ and write $\pslash = (p^{\theta_1},p^{\theta_2})$ for the angular momentum components, so that the conjugate coordinates on the tangent bundle become $(x^\mu,p^\mu) = (t^*,r,\omega,p^{t^*},p^r,\pslash)$. We also introduce the notation $\left| \pslash \right|_{\gslash}^2 = \gslash_{AB} p^A p^B$, where $A,B \in \{ \theta_1, \theta_2 \}$. We may then express the mass-shell relation in $(t^*,r)$-coordinates as
\begin{equation}
\gRN(x)(p,p)= - \Osqrn (p^{t^*} )^2 + 2(1-\Osqrn) p^{t^*} p^r + (2-\Osqrn) (p^r)^2 + \left| \pslash \right|_{\gslash}^2 = 0.
\end{equation}
Note that the geodesic flow preserves lengths, so that the geodesic spray restricts to a vector field on the mass-shell $\mathcal{P}$. We can now state the following

\begin{defi}[Massless Vlasov equation]
Let $f: \mathcal{P} \rightarrow \R_{\geq 0}$ be smooth. We say that $f$ solves the massless Vlasov equation if $X(f) = 0$ or equivalently if $f$ is conserved  along the geodesic flow.
\end{defi}

\begin{rem}
If $\Sigma_0 \subset M$ denotes a Cauchy hypersurface we define $\mathcal{P}_0 = \mathcal{P}|_{\Sigma_0}$. We may then specify initial data $f_0: \mathcal{P}_0 \rightarrow \R_{\geq 0}$ and obtain a unique solution $f: \mathcal{P} \rightarrow \R_{\geq 0}$ attaining these initial data by transporting along future-directed null geodesics.
\end{rem}

\begin{assumption}[Compactness of support of initial distribution] \label{assumption_support}
	Let $f_0: \mathcal{P}_0 \rightarrow \R$ be a continuous bounded initial distribution on the (subextremal or extremal) Reissner--Nordstr\"om background. We assume the existence of constants $\rsuppconst > 0$ and $\psuppconst > 0$ such that
	\begin{equation} \label{eqn_assumption_support}
		\supp(f_0) \subset \Big\{ (x,p) \in \mathcal{P} : x \in \Sigma_0, \; r \leq \rsuppconst, \; p^{t^*}, \left| p^r \right|, \left| \pslash \right|_{\gslash} \leq \psuppconst \Big\},
	\end{equation}
	where we have used $(t^*,r)$-coordinates and their associated coordinates on the tangent bundle.
\end{assumption}

\begin{rem}
The constant $\rsuppconst$ bounds the spatial support of $f_0$, while $\psuppconst$ provides a bound on the size of the momentum support of $f_0$.
\end{rem}

\begin{rem}
We will often say that a null geodesic segment $\gamma: [s_1,s_2] \rightarrow \Mrn$ satisfies Assumption~\ref{assumption_support}, by which we mean that there exists $s \in [s_1,s_2]$ for which $\gamma(s) \in \Sigma_0$ and $(\gamma(s),\dot{\gamma}(s))$ is contained in the set on the right hand side of~\eqref{eqn_assumption_support}. The constants $\rsuppconst$ and $\psuppconst$ defined above will appear throughout the proofs.
\end{rem}


\subsubsection{Conserved quantities of the geodesic flow}
The symmetries of the Reissner--Nordstr\"om metric imply the existence of quantities which are conserved along the geodesic flow. The stationarity respectively spherical symmetry of the Reissner--Nordstr\"om metric imply that the energy $E: \mathcal{P} \rightarrow \R_{\geq 0}$ respectively the total angular momentum $L^2: \mathcal{P} \rightarrow \R_{\geq 0}$ are conserved, or in other words $X(E) = X(L) = 0$. Let us use $(t^*,r)$-coordinates and denote the conjugate coordinates on the tangent bundle by $(t^*,r,\omega,p^{t^*},p^r,\pslash)$. Then we may express the energy and angular momentum explicitly as
\begin{equation} \label{def_energy}
	E = \Osqrn p^{t^*} - (1-\Osqrn) p^r, \quad L^2 = r^2 \left| \pslash \right|^2_{\gslash}.
\end{equation}
Combining these explicit expressions with the mass-shell relation shows that for all points on the mass-shell $(x,p) = (t^*,r,\omega,p^{t^*},p^r,\pslash) \in \mathcal{P}$ expressed in $(t^*,r)$-coordinates the conservation of energy identity holds,
\begin{equation} \label{Schw::energy_conservation}
	E^2 = (p^r)^2 + \frac{\Osqrn}{r^2} L^2.
\end{equation}

\begin{defi}[Trapping parameter] \label{def_eps_def}
We define the trapping parameter $\trapschw: \mathcal{P} \rightarrow [-\infty,1]$ to be the conserved quantity satisfying the following equality
\begin{equation} \label{def_eps}
	E^2 - \rnconstant \frac{L^2}{m^2} = \trapschw E^2,
\end{equation}
where $\rnconstant$ depends only on the member of the Reissner--Nordstr\"om family under consideration and is explicitly defined as
\begin{equation}
	\rnconstant = \rnconstant \left( \frac{\left| q \right|}{m} \right) = \frac{m^2}{r^2} \Osqrn \Bigg|_{r = r_{ph}} = \begin{cases}
		\frac{1}{27} & q = 0 \\
		\frac{1}{16} & \left| q \right| = m
	\end{cases}.
\end{equation}
\end{defi}

\begin{rem}
The trapping parameter measures the failure of a null geodesic $\gamma$ to be trapped at the photon sphere. Consider a future-directed null geodesic $\gamma(s) = (t^*(s),r(s),\omega(s))$ with affine parameter $s$ expressed in $(t^*,r)$-coordinates. Then $\gamma$ is trapped at the photon sphere if $\gamma$ is future complete and $r(s) \rightarrow r_{ph}$ as $s \rightarrow \infty$. It is a classic fact that a future-directed null geodesic $\gamma$ is trapped at the photon sphere if and only if $\trapschw(\gamma,\dot{\gamma}) = 0$. It follows that at any point $x \in \Mrn \setminus \mathcal{H}^+$, the subset of all $p \in \mathcal{P}_x$ with the property that the geodesic with initial data $(x,p)$ is trapped at the photon sphere, is identical with the codimension one subset $\{ p \in \mathcal{P}_x : \trapern(x,p) = 0 \} \subset \mathcal{P}_x$.
\end{rem}

\begin{rem}
Conservation of energy~\eqref{Schw::energy_conservation} immediately implies $\frac{L^2}{E^2} \leq \frac{r^2}{\Omega^2}$ so that
\begin{equation}  \label{epsbound}
- \infty \leq - \frac{1}{\Osqrn} \frac{r^2}{m^2} \left( \rnconstant - \frac{m^2}{r^2} \Osqrn \right) \leq \trapschw \leq 1,
\end{equation}
where we note that by definition
\begin{equation}
\rnconstant - \frac{m^2}{r^2} \Osqrn \sim \left( 1 - \frac{r_{ph}}{r} \right)^2.
\end{equation}
This bound may be interpreted as saying that for a given radius $r$, the trapping parameter $\trapschw$ is restricted to a finite range depending on the value of $r$. Alternatively, we may interpret this inequality as saying that for fixed $\trapschw < 0$, the radius $r$ is restricted. 
\end{rem}

\begin{defi}
Assume that $-\infty < \trapschw < 0$, then the equation
\begin{equation}
	\frac{1}{\Osqrn} \frac{r^2}{m^2} \left( \rnconstant - \frac{m^2}{r^2} \Osqrn \right) = \left| \trapschw \right|
\end{equation}
has exactly two real solutions $r_{\text{min}}^-(\trapschw)$ and $r_{\text{min}}^+(\trapschw)$, labelled such that
\begin{equation}
	r_+ < r_{\text{min}}^-(\trapschw) < r_{ph} < r_{\text{min}}^+(\trapschw) < \infty.
\end{equation}
We call $r_{\text{min}}^-(\trapschw), r_{\text{min}}^+(\trapschw)$ the radii of closest approach to the photon sphere for a given negative trapping parameter.
\end{defi}

\subsubsection{Parametrising the mass-shell} \label{sec_parametrising_massshell}
To obtain a parametrisation of the mass-shell $\mathcal{P}$ we consider $(t^*,r)$-coordinates $(x^\mu) = (t^*,r,\omega)$ and their associated coordinates $(x^\mu, p^\mu) = (t^*,r,\omega,p^{t^*},p^r,\pslash)$ on the tangent bundle. We then eliminate the variable $p^{t^*}$ by using the mass-shell relation and express it as a function of $r,p^r$ and $\pslash$. We make this explicit in the following
\begin{lem} \label{express_ptstar}
For $(x,p) = (t^*,r,\omega, p^{t^*},p^r,\pslash) \in \mathcal{P}$ with $r > r_+$ we have
\begin{equation} \label{pt_expressed_via_pr}
	p^{t^*} = \frac{(1-\Osqrn) p^r + \sqrt{(p^r)^2 + \Osqrn \left| \pslash \right|_{\gslash}^2 }}{\Osqrn} = \frac{E + (1-\Osqrn) p^r}{\Osqrn}.
\end{equation}
Furthermore when we assume that $p^r \leq 0$ and $r \geq r_+$ we find the alternative expression
\begin{equation} \label{express_ptstar_neg}
	p^{t^*} = \left| p^r \right| + \frac{\left| \pslash \right|_{\gslash}^2}{E + \left| p^r \right|}.
\end{equation}
Along the event horizon $r = r_+$ and we necessarily have $p^r \leq 0$ and
\begin{equation} \label{pt_expressed_via_pr_horizon}
	p^{t^*} = \left| p^r \right| +\frac{\left| \pslash \right|_{\gslash}^2}{2 E}.
\end{equation}
In particular $p^{t^*} \geq 0$ for all $r \geq 2m$ and $p^{t^*} = 0$ if and only if $p = 0$.
\end{lem}
\begin{proof}
Let us begin by noting that the definition of energy $E$ implies
\begin{equation}
	E = \OsqS p^{t^*} - (1-\Osqrn) p^r,
\end{equation}
which we may rearrange in order to immediately conclude the relation
\begin{equation}
	p^{t^*} = \frac{E + (1-\Osqrn) p^r}{\OsqS},
\end{equation}
whenever $r > r_+$. Conservation of energy~\eqref{Schw::energy_conservation} now implies
\begin{equation}
	E = \sqrt{ (p^r)^2 + \frac{\OsqS}{r^2} L^2 } =\sqrt{ (p^r)^2 + \OsqS \left| \pslash \right|_{\gslash}^2 } ,
\end{equation}
which inserted into the relation above immediately allows us to conclude~\eqref{pt_expressed_via_pr}. If we now assume that $p^r \leq 0$ and $r > r_+$, note that again using conservation of energy we may rewrite
\begin{equation}
	p^{t^*} = \frac{E - (1-\Osqrn) \left| p^r \right|}{\OsqS} = \left| p^r \right| + \frac{E - \left| p^r \right|}{\OsqS} =\left| p^r \right| + \frac{1}{\OsqS} \frac{E^2 - (p^r)^2}{E + \left| p^r \right|} = \left| p^r \right| + \frac{\left| \pslash \right|_{\gslash}^2}{E + \left| p^r \right|}.
\end{equation}
In particular we note that at the event horizon, future-orientedness of $p$ implies that $- p^r = E \geq 0$. Therefore this relation holds for all $r \geq r_+$ and evaluating it at $r = r_+$ recovers equation~\eqref{pt_expressed_via_pr_horizon}. Alternatively this may be seen by an application of L'Hôpital's rule. Note that as a consequence of the mass-shell relation $p^{t^*} = 0$ implies $L=p^r=0$ and therefore $p^\mu = 0$ for all $\mu$. 
\end{proof}

\begin{rem}
We denote the resulting parametrisation of $\mathcal{P}$ by $(t^*,r,\omega,p^r,\pslash)$. Note carefully that if $r= r_+$ and $p^r = 0$ the mass-shell relation implies $\pslash = 0$ and the coordinate system $(p^r,\pslash)$ on $\mathcal{P}_{(t^*,r_+,\omega)}$ over a point on the event horizon degenerates. This will however not cause any issues in practice, as the reader may readily verify that the set $\{ (p^{t^*},p^r,\pslash) \in \mathcal{P}_{(t^*,r_+,\omega)} : p^r = 0 \}$ has vanishing measure, see below.
\end{rem}

\subsubsection{Moments of solutions to the massless Vlasov equation}
The metric induces a natural volume form on each fibre $\mathcal{P}_x$ for $x \in M$. In order to define the volume form, we choose explicit coordinates on the sphere. Let $(\theta,\phi)$ denote the usual spherical coordinates on $\sphere$\footnote{Note that since spherical coordinates do not cover the whole $2$-sphere, we need to repeat the argument with rotated versions of the coordinate system twice. The argument remains however identical so that we will only give it once.} so that the $(t^*,r)$-coordinates now become $(t^*,r,\theta,\phi)$. The conjugate parametrisation of the mass-shell is now denoted by $(t^*,r,\theta,\phi,p^r,p^\theta,p^\phi)$. The integration measure $d \mu_x$ on $\mathcal{P}_x$ is then readily computed to be
\begin{equation} \label{dmu_first_comp_schw}
	d \mu_x = \frac{r^2 \sin \theta}{\Osqrn p^{t^*} - \left( 1 - \Osqrn \right) p^r} \, d p^r d p^\theta d p^\phi,
\end{equation}
where $p^{t^*}$ is understood to be expressed as a function of the remaining variables as in Lemma~\ref{express_ptstar}. We immediately note that future directedness of $p$ implies that $\Osqrn p^{t^*} - \left( 1 - \Osqrn \right) p^r = - \gRN(\partial_{t^*},p) \geq 0$. Using this measure on $\mathcal{P}_x$ we may now define moments of the distribution $f$.

\begin{defi}[Moments of $f$]
Let $w,f: \mathcal{P} \rightarrow \R$ be smooth. Then the moment of $f$ associated to the weight $w$ is defined as
\begin{equation}
\int_{\mathcal{P}_x} w f \, d \mu_x \in [-\infty,+\infty].
\end{equation}
\end{defi}

\begin{rem}
Note that a priori the moment associated to a weight $w$ might not be finite. However, it will follow from Lemma~\ref{lem_boundedness_moments_schw} respectively Lemma~\ref{lem_boundedness_moments_ERN} that $\int_{\mathcal{P}_x} w f \, d \mu_x \in \R$ if $f$ is a solution to the massless Vlasov equation with compactly supported initial data.
\end{rem}

\begin{defi}[Boundedness in $x$] \label{boundedness_in_x}
Let $C > 0$ be a constant and consider the set
\begin{equation}
B_C = \Big\{ (x,p) \in \mathcal{P} : p^{t^*}, \left| p^r \right|, \left| \pslash \right|_{\gslash} \leq C \Big\} \subset \mathcal{P}.
\end{equation}
Let $w: \mathcal{P} \rightarrow \R$ be a smooth weight. Then we say that $w$ is bounded in $x$ if for all $C>0$
\begin{equation}
    \sup_{(x,p) \in B_C} \left| w(x,p) \right| < \infty.
\end{equation}
\end{defi}

\begin{defi}[Energy momentum tensor]
Let $(x^\mu)$ denote a coordinate system on $\Mrn$ and $(x^\mu,p^\mu)$ the associated coordinate system on $T \Mrn$. Then
\begin{equation} \label{energymomentum_defn}
T^{\alpha \beta}[f] = \int_{\mathcal{P}_x} p^\alpha p^\beta f(x,p) \, d \mu_x
\end{equation}
defines the \emph{energy momentum tensor} associated to the solution $f$ of the massless Vlasov equation in the coordinates $( x^\mu)$.
\end{defi}

\begin{rem}
For smooth $f$ we have the divergence identity
\begin{equation}
\nabla_\mu T^{\mu \nu}[f] =  \int_{\mathcal{P}_x} X(f) p^\nu \, d \mu_x,
\end{equation}
so that if $X(f) = 0$, the energy momentum tensor is conserved. See~\cite{martin,riosecosarbach,sarbachzannias} for a general discussion on how to relate derivatives of the energy momentum tensor to derivatives of $f$.
%In addition, the energy momentum tensor has the following positivity property: If $X,Y$ are both future-directed and causal, then $T_{\mu \nu} X^{\mu} Y^{\nu} \geq 0$ (\emph{weak energy condition}). In addition we have $T_{\mu \nu} X^{\mu} X^{\nu} \geq 0$ for any vector field $X$ (\emph{non-negative pressure condition}). Furthermore the mass-shell relation implies that $T$ is \emph{trace-free}, in other words $T^{\mu}_{\mu} = 0$.
\end{rem}

\subsection{The set of almost-trapped geodesics} \label{sec_subsets}
In this section we define various subsets of the mass-shell $\mathcal{P}$ that will play a central role later in the proofs of the main theorems.

\begin{defi}[Almost trapping at the photon sphere] \label{defn_trapping_ph}
Let $\mathcal{P}$ denote the mass-shell on the subextremal or extremal Reissner--Nordstr\"om exterior. Fix constants $\bigc,\decayrate > 0, \tauzero > m$ and recall the constants $\rsuppconst, \psuppconst$ introduced in Assumption~\ref{assumption_support}. We define
\begin{align} \label{defn_trappedsupportset}
	\trappedsupportset_{\bigc,\decayrate,\tauzero} = \left\{ (x,p) \in \mathcal{P} \; \Big| \; \begin{matrix}
		\left| \trapschw \right| \leq \bigc e^{-\frac{\decayrate}{2m}(\tau(x)-\tauzero)}, \quad
		\left| \pslash \right|_{\gslash} \leq \frac{\rsuppconst \psuppconst}{r} \\
		\left| m \frac{\left| p^r \right|}{L} - \sqrt{\rnconstant - \frac{m^2}{r^2} \Osqrn} \right| \leq \bigc e^{-\frac{\decayrate}{2m}(\tau(x)-\tauzero)} \\
		\left| m \frac{ p^{t^*}}{L} - \left( \frac{\sgn(p^r) (1-\Osqrn) \sqrt{\rnconstant - \frac{m^2}{r^2} \Osqrn} + \sqrt{\rnconstant}}{\Osqrn} \right) \right| \leq \bigc e^{-\frac{\decayrate}{2m}(\tau(x)-\tauzero)}
	\end{matrix} \right\},
\end{align}
where we used $(t^*,r)$-coordinates and their induced coordinates on the tangent bundle to represent each point as $(x,p) = (t^*,r,\omega,p^{t^*},p^r,\pslash)$ and the constant $\rnconstant$ is defined as in Definition~\ref{def_eps_def}. For a point $x \in \Mrn$ we denote the fibre by $\trappedsupportsetx = \trappedsupportset \cap \mathcal{P}_x$.
\end{defi}

\begin{rem}
We remark that the expression
\begin{equation} \label{exprn_defn_trappedset}
	\frac{1}{\Osqrn} \left( \sgn(p^r) (1-\Osqrn) \sqrt{\rnconstant- \frac{m^2}{r^2} \Osqrn} + \sqrt{\rnconstant} \right),
\end{equation}
used in the definition of the set $\trappedsupportset$, has a finite limit as $r \rightarrow r_+$ if and only if $p^r \leq 0$. Note that  if $(x,p) \in \mathcal{P}$ and $x \in \mathcal{H}^+$, then by future-directedness of $p$ we must have $p^r \leq 0$. Therefore, the set $\trappedsupportset$ is well-defined on the whole exterior, but it is not continuously defined up to and including the event horizon $\mathcal{H}^+$. However, we will only need to evaluate the expression~\eqref{exprn_defn_trappedset} along null geodesics, and it follows from the properties of the geodesic flow that the limit $r \rightarrow r_+$ can only occur if $p^r \leq 0$.
\end{rem}

\begin{defi}[Almost trapping at the event horizon] \label{defn_trapping_hor}
Let $\mathcal{P}$ denote the mass-shell on the subextremal or extremal Reissner--Nordstr\"om exterior. Fix constants $\bigc,\decayrate > 0, \tauzero > m$ and recall the constants $\rsuppconst, \psuppconst$ introduced in Assumption~\ref{assumption_support}. In the subextremal case we define
\begin{align} \label{defn_smallsupportset}
	\smallsupportset_{\bigc,\decayrate,\tauzero} = \left\{ (x,p) \in \mathcal{P} \; \Bigg| \; \begin{matrix}
		\sqrt{1+ \left| \trapschw \right|} \left| \pslash \right|_{\gslash} \leq \bigc \rsuppconst \psuppconst \frac{1}{r} e^{- \frac{\decayrate}{2m}(\tau(x)-\tauzero)} \\
		\left( 1+ \left| \trapschw \right| \right) \left| p^r \right| \leq \bigc \psuppconst e^{-\frac{\decayrate}{m}(\tau(x)-\tauzero)} \\
		p^{t^*} \leq \bigc \psuppconst e^{-\frac{\decayrate}{m}(\tau(x)-\tauzero)}
	\end{matrix} \right\},
\end{align}
whereas in the extremal case we define
\begin{align} \label{defn_smallsupportsetERN}
	\ERNsmallsupportset_{\bigc,\decayrate,\tauzero} = \left\{ (x,p) \in \mathcal{P} \; \Bigg| \; \begin{matrix}
		\left| \pslash \right|_{\gslash} \leq \bigc \psuppconst \frac{\rsuppconst}{r} \frac{m}{\left| \tau(x)-\tauzero \right|} \\
		\left| p^r \right| \leq \bigc \psuppconst \frac{m^2}{(\tau(x)-\tauzero)^2} \\
		p^{t^*} \leq \bigc \psuppconst \min \left( \frac{1}{\Osqern} \frac{m^2}{(\tau(x)-\tauzero)^2}, 1 \right)
	\end{matrix} \right\},
\end{align}
where in both cases we used $(t^*,r)$-coordinates as in the preceding definition. For a point $x \in \Mschw$ we denote the fibre by $\smallsupportsetx = \smallsupportset \cap \mathcal{P}_x$.
\end{defi}

\begin{rem}
We will from now on omit the dependence of the sets $\trappedsupportset_{\bigc,\decayrate,\tauzero}$ and $\ERNsmallsupportset_{\bigc,\decayrate,\tauzero}$ on the constants $\bigc,\decayrate,\tauzero$ and simply write $\trappedsupportset_{\bigc,\decayrate,\tauzero} = \trappedsupportset$ and $\ERNsmallsupportset_{\bigc,\decayrate,\tauzero} = \ERNsmallsupportset$. The proofs of the main theorems will reveal how the constants $\bigc,\decayrate,\tauzero$ must be chosen. In particular, $\bigc,\decayrate$ are universal constants depending only on the geometry of Reissner--Nordstr\"om through the quotient $\frac{\left| q \right|}{m}$, while $\tauzero$ will depend only on the size of the initial support of the solution to the massless Vlasov equation under consideration.
\end{rem}

\begin{rem}
The set $\trappedsupportset \subset \mathcal{P}$ contains geodesics which are almost trapped at the photon sphere, while $\smallsupportset \subset \mathcal{P}$ contains those geodesics which are almost trapped at the event horizon. Geometrically, each fibre $\trappedsupportsetx$ is an approximate $2$-cone in $\mathcal{P}_x$, while $\smallsupportsetx$ is a small cylinder around the origin, see Figure~\ref{figure_fibres}. We will later show that if a geodesic has not scattered to infinity or fallen into the black hole after a finite (fixed) time $\tauzero$, then it must be contained in either $\trappedsupportset$ or $\smallsupportset$ after this time. This is accomplished in Proposition~\ref{psupport_prop} for the subextremal case and in Proposition~\ref{psupport_propERN} for the extremal case and forms the core of the proof of decay of moments of solutions to the massless Vlasov equation. See Sections~\ref{section_proof} and~\ref{section_ERN}.
\end{rem}

\begin{rem}
Both sets involve a choice of certain constants $\bigc, \decayrateERN > 0, \tauzero > m$, which will be chosen appropriately in the proofs of Proposition~\ref{psupport_prop} for the subextremal case respectively Proposition~\ref{psupport_propERN} for the extremal case.
\end{rem}

The following definitions only apply to the \emph{extremal} Reissner--Nordstr\"om spacetime.

\begin{defi} \label{def_badsetapprox}
Let $\mathcal{P}$ denote the mass-shell on the extremal Reissner--Nordstr\"om exterior. Let $0 < \delta < \frac{1}{2}$ and $c_2 > c_1 > 0$ and let $\psuppconst$ be as in Assumption~\ref{assumption_support}. Define the following subset of the mass-shell over the initial hypersurface $\Sigma_0$
\begin{equation}
		\badsetapprox_{c_1,c_2,\delta} = \left\{ (x,p) \in \mathcal{P}_0 \; \Bigg| \; m \leq r \leq (1 + \delta) m, \; \begin{matrix}
		\left| p^r \right| \leq c_2 \psuppconst \Osqern(r) \\
		c_1 \psuppconst \sqrt{\Osqern(r)} \leq \left| \pslash \right|_{\gslash} \leq c_2 \psuppconst \sqrt{\Osqern(r)}
	\end{matrix} \right\} \subset \mathcal{P}_0.
 \end{equation}
Let furthermore $\constbsl > 0$ and define the following subset of $\mathcal{P}_0$
 \begin{equation}
		\badsetaplarge_{\constbsl,\delta} = \left\{ (x,p) \in \mathcal{P}_0 \; \Bigg| \; m \leq r \leq (1 + \delta) m, \; \begin{matrix}
	- \constbsl \psuppconst \sqrt{\Osqern(r)} \leq p^r \leq \psuppconst \Osqern(r) \\
	\left| \pslash \right|_{\gslash} \leq \constbsl \psuppconst (\Osqern(r))^{\frac{1}{4}}
\end{matrix} \right\} \subset \mathcal{P}_0.
\end{equation}
\end{defi}

\begin{rem}
	We will at times omit the dependence on the constants $c_1,c_2$ and $\constbsl$ and prefer to think of the sets $\badsetapprox_{c_1,c_2,\delta}, \badsetaplarge_{\constbsl,\delta}$ as one-parameter families parametrised by $0 < \delta < \frac{1}{2}$ with an appropriate choice of constants $c_1, c_2$ and $\constbsl$, respectively. For $\badsetapprox_{c_1,c_2,\delta}$, this choice will be specified in the proof of Lemma~\ref{sharpness_lemma}, while for $\badsetaplarge_{\constbsl,\delta}$ the choice of $\constbsl$ will be made in the proof of Proposition~\ref{psupport_propERN}. We will then simply write $\badsetapprox_{\delta} = \badsetapprox_{c_1,c_2,\delta}$ and $ \badsetaplarge_{\delta} =  \badsetaplarge_{\constbsl,\delta}$. When no confusion can arise, we will at times omit all indices.
\end{rem}

\begin{rem}
We will later choose the constants $c_1,c_2,\constbsl$ from Definition~\ref{def_badsetapprox} such that $\constbsl > c_2$, so that $\badsetapprox_{c_1,c_2,\delta} \subset \badsetaplarge_{\constbsl,\delta}$ for all $0 < \delta < \frac{1}{2}$. Moreover, the fibres of $\badsetaplarge_\delta$ are uniformly bounded. More precisely, every point $(x,p) \in \badsetaplarge_\delta$ satisfies Assumption~\ref{assumption_support}. Therefore, the assumption that $f_0$ is compactly supported, as defined in Assumption~\ref{assumption_support}, is compatible with the assumption that $\badsetaplarge_\delta \subset \supp(f_0)$.
\end{rem}

\begin{defi} \label{defi_slowsupportset}
Let $\mathcal{P}$ denote the mass-shell on the extremal Reissner--Nordstr\"om exterior. Let $C_2 > C_1 > 0$ and $\tauslow > m$ be constants and recall the constant $\psuppconst$ from Assumption~\ref{assumption_support}. We define the following subset of the mass-shell over the event horizon
	\begin{equation}
		\slowsupportset_{C_1,C_2,\tauslow} = \left\{ (x,p) \in \mathcal{P} \; \Bigg| \; \tau(x) > \tauslow, r=m, \; \begin{matrix}
			C_1 \psuppconst \frac{m}{\tau(x)} \leq \left| \pslash \right|_{\gslash} \leq C_2 \psuppconst \frac{m}{\tau(x)} \\
			C_1 \psuppconst \frac{m^2}{\tau(x)^2} \leq \left| p^r \right| \leq C_2 \psuppconst \frac{m^2}{\tau(x)^2}
		\end{matrix} \right\} \subset \mathcal{P} |_{\mathcal{H}^+}.
	\end{equation}
\end{defi}

\begin{rem}
	We will often omit the dependence on $C_1,C_2$ and consider $\slowsupportset_\tauslow = \slowsupportset_{C_1,C_2,\tauslow}$ as a one-parameter family parametrised by $\tauslow$ for an appropriate choice of constants $C_1,C_2$. At times we will omit all constants and merely write $\slowsupportset = \slowsupportset_{C_1,C_2,\tauslow}$.
\end{rem}

\begin{defi} \label{defn_pi0}
Consider the mass-shell $\mathcal{P}$ on the extremal Reissner--Nordstr\"om exterior and the mass-shell $\mathcal{P}_0 = \mathcal{P} |_{\Sigma_0}$ over the initial hypersurface $\Sigma_0$. We define the following map
	\begin{equation}
		\pi_0 : \mathcal{P} \rightarrow \mathcal{P}_0, \quad (x,p) \mapsto S_{s_0}(x,p),
	\end{equation}
	where $S_s$ denotes the geodesic flow on the mass-shell $\mathcal{P}$ as defined in~\eqref{geodesicflow} above and $s_0$ is the unique affine parameter time such that $S_{s_0}(x,p) \in \mathcal{P}_0$.
\end{defi}

\begin{rem}
The set $\badsettau$ is the set of initial data of the geodesics which generate the set $\slowsupportset_{\tauslow}$ over the event horizon. We will prove in Lemma~\ref{sharpness_lemma} that the sets $\badsettau$ and $\badsetapprox_{\delta}$ are comparable if we assume $\tauslow \sim \delta^{-1}$. To make this a bit more precise, Lemma~\ref{sharpness_lemma} establishes the existence of a suitable choice of constants $c_1, c_2, C_1,C_2$ such that for any of $0 < \delta < \frac{1}{2}$, there exists $\tauslow \sim \delta^{-1}$ such that $\badsetallconst \approx \badsetapprox_{c_1,c_2,\delta}$. Likewise, in Proposition~\ref{psupport_propERN} we show that (roughly speaking) $\pi_0(\smallsupportset_{\tauzero} \cap \left\{ \tau \geq \bar{\tau} \right\}) \subset \badsetaplarge_\delta$ for any $ \bar{\tau} > \tauzero$ if we assume $\delta \sim \bar{\tau}^{-1}$ and choose the remaining constants appropriately.
\end{rem}

\begin{figure} \label{figure_fibres}
\includegraphics[width = \textwidth]{figures/drawing_trappedset.png}
\centering
\caption{A plot of the fibres $\trappedsupportsetx$ in Figure~(a) and $\smallsupportsetx$ in Figure~(b) for a point $x \in \Mschw$. We parametrise the fibre $\mathcal{P}_x$ by $(p^r,p^1,p^2) \in \R^3$. This coordinate system is obtained from the coordinates on the tangent bundle associated to $(t^*,r)$-coordinates by using the mass-shell relation to eliminate the $p^{t^*}$-variable. Then we choose $(p^1,p^2)$ such that $L^2 = r^2 \left| \pslash \right|_{\gslash}^2 = (p^1)^2 + (p^2)^2$. The set $\trappedsupportsetx$ is the region between the inner truncated cone (with apex angle $\alpha - \beta$) and the outer truncated cone (with apex angle $\alpha + \beta$). The centre truncated cone (with apex angle $\alpha$) is the subset of momenta which generate exactly trapped geodesics. We explain how $\alpha,\beta$ and $\rho$ relate to the quantities in Definition~\ref{defn_trapping_ph}. The apex angle $\alpha$ of the centre truncated cone may be seen to scale like $\tan \alpha \sim \sqrt{\rnconstant - \frac{m^2}{r^2} \Osqrn} \sim \left( 1 - \frac{r_{ph}}{r} \right)^{-1}$. Moreover, the centre cone has a maximal radius of $\rho \leq \rsuppconst \psuppconst$. In particular over the photon sphere, where $r = r_{ph}$, the apex angle is $\alpha = \frac{\pi}{2}$ and the centre cone flattens to the disk $\{ p^r = 0, L \leq \rsuppconst \psuppconst \}$. The angle $\beta$ is exponentially small in $\tau(x)$, so that $\trappedsupportsetx$ approximates the centre cone as $\tau(x) \rightarrow \infty$. The set $\smallsupportsetx$ is a small cylinder around the origin with radius $\rho$ and height $h$. Height and radius relate to each other like $h \sim \rho^2$. The height $h$ is proportional to the right hand side of the upper bound for the $p^r$-component in Definition~\ref{defn_trapping_hor}. Therefore, $h$ is exponentially small in $\tau(x)$ in the subextremal case and inverse quadratic in $\tau(x)$ in the extremal case.}
\end{figure}

\subsection{A note on constants}
Unless otherwise stated, the letter $C>0$ will denote a constant which depends only on the member of the Reissner--Nordstr\"om family under consideration through the ratio $\left| q \right| / m$. The symbol $\lesssim$ is used to express inequalities which hold up to multiplicative constants. We say $a \sim b$ for $a,b > 0$ if and only if there exist constants $0 < c < C < \infty$ such that $c \leq \frac{a}{b} \leq C$. If the constants $c,C$ in this inequality depend on some parameter $\lambda$, we reflect this in the notation by $a \sim_\lambda b$.

In Assumption~\ref{assumption_support} we introduced the constants $\rsuppconst,\psuppconst$, which bound the size of the spatial support respectively the size of the momentum support of an initial distribution. These constants appear throughout the proof and we always make dependence on them explicit in our estimates. Several definitions in Section~\ref{sec_subsets} require a choice of constants $\bigc, \decayrate, C_1, C_2, c_1, c_2, \constbsl, \tauslow, \tauzero$. These will be chosen as appropriate functions of the Reissner--Nordstr\"om parameters $(m,q)$ and $\rsuppconst, \psuppconst$ in the course of the argument. We will make the dependence on these constants explicit in estimates where necessary and suppress it when doing so would not yield any more insight.

%\label{defi_trappedsets}
%\label{def_ERN_subsets}

