\section{Decay for massless Vlasov on Schwarzschild} \label{section_proof}
In this section we provide the proof of Theorem~\ref{maintheorem_precise}. We will only show the result for the Schwarzschild spacetime, which is the special case $q=0$. The proof generalises almost verbatim to the full subextremal range $\left| q \right| < m$, the main difference being that all dimensionless constants then depend on the quotient $\left| q \right| / m$. The main step in the proof consists in obtaining control over the momentum support of a solution $f$ to the massless Vlasov equation. In other words, we aim to understand the volume of the set $\supp(f(x,\cdot)) \subset \mathcal{P}_x$ at any given point $x \in \Mschw$ with $\tau(x) \gg 1$. This is the content of the main Proposition~\ref{psupport_prop} of this section. \\

Before stating Proposition~\ref{psupport_prop}, we remind the reader of the definition of the sets $\trappedsupportset$ and $\smallsupportset$ in Section~\ref{sec_subsets}. The set $\trappedsupportset \subset \mathcal{P}$ contains those geodesics which are almost trapped at the photon sphere. Such geodesics spend a lot of time in a neighbourhood of the photon sphere. Let $(x,p) \in \trappedsupportset$ such that $\tau(x)$ is large. Then $p$ approximately lies on a truncated $2$-cone in $\mathcal{P}_x$. The distance of $p$ to this $2$-cone is exponentially small in the time $\tau(x)$. The set $\smallsupportset \subset \mathcal{P}$ contains those geodesics which are almost trapped on the event horizon. The geodesics populating this set originate close to the event horizon, are initially outgoing and require a lot of time to leave the vicinity of the black hole. If $(x,p) \in \smallsupportset$ then $p$ lies in a cylinder around the origin with exponentially small height and radius. See Figure~\ref{figure_fibres}. \\

Both $\trappedsupportset$ and $\smallsupportset$ depend on a choice of constants $\bigc, \decayrate,\tauzero$. These constants are chosen appropriately in the proof of Proposition~\ref{psupport_prop}. Recall also the constant $\rsuppconst$ defined in Assumption~\ref{assumption_support}.

\begin{prop} \label{psupport_prop}
Let $f_0 \in L^\infty(\mathcal{P}_0)$ satisfy Assumption~\ref{assumption_support} and let $f$ be the unique solution to the massless Vlasov equation on the Schwarzschild exterior with initial distribution $f_0$. Then there exist dimensionless constants $\bigc, \decayrate, C_0 > 0$ such that if we choose $\tauzero \geq C_0  \frac{\rsuppconst^2}{m}$, then
\begin{equation}
	\supp(f) \cap \left\{ (x,p) \in \mathcal{P} \; | \; \tau(x) \geq \tauzero \right\} \subset \trappedsupportset \cup \smallsupportset,
\end{equation}
where we have abbreviated $\trappedsupportset = \trappedsupportset_{\bigc,\decayrate,\tauzero}$ and $\smallsupportset = \smallsupportset_{\bigc,\decayrate,\tauzero}$.
\end{prop}

%\begin{rem} \label{remark_explain_psupport}
%In the case that $(x,p) \in \trappedsupportset$ the rate stated above at which $p^{t^*}$ and $p^r$ decay to their limit values as $\tau(x) \rightarrow \infty$ may be improved in certain cases. Specifically, let $0 < \delta' < 1$ and assume that  $r \notin [(3-\delta')m,(3+\delta')m]$. If we in addition assume that $\tau(x) \gtrsim \tauzero - \ln \delta'$ then we may in fact improve the decay rate from $\frac{c}{2}$ to $c$. However it will be demonstrated in the proof below that the decay rate as stated may not be improved with our methods at the photon sphere.
%\end{rem}

Having stated the main proposition, let us outline the strategy of proof. We establish preliminary estimates for momenta that are contained in the mass-shell in Section~\ref{prelim_massshell_estimates}. In Section~\ref{section_tstar} we then consider a fixed null geodesic $\gamma: [s_0,s] \rightarrow \Mschw$ such that $\gamma(s_0) \in \Sigma_0$ and use the estimates from Section~\ref{prelim_massshell_estimates} to obtain a bound on the time required by the geodesic to cross a certain region of spacetime. Specifically, we bound $\tau(\gamma(s))$ in terms of the radii $r(s_0),r(s)$ and the trapping parameter of the geodesic. This is the \emph{almost-trapping estimate} introduced in Section~\ref{subsubsec_geodflow} and is the key to proving Proposition~\ref{psupport_prop}. In Section~\ref{section_psupport} we then use the almost-trapping estimate to prove Proposition~\ref{psupport_prop}. We recall the outline of the argument from Section~\ref{sec_radialgeod}: Given a point $x \in \Mschw$ with $\tau(x)$ large, suppose a geodesic populates the momentum support at $x \in \Mschw$. The bounds derived in Section~\ref{section_tstar} then allow us to relate the time $\tau(x)$ with the trapping parameter and initial data of the geodesic. Combined with our assumption of compactly supported initial data, this allows us to derive bounds on the momentum of the geodesic at the point $x \in \Mschw$. Finally, we prove Theorem~\ref{maintheorem_precise} in Section~\ref{section_integralestimate}. Given Proposition~\ref{psupport_prop}, it only remains to show that the volume of the sets $\trappedsupportset$ and $\smallsupportset$ is exponentially small in $\tau$, which amounts to a simple computation involving the estimation of some integrals.

\subsection{Preliminary estimates} \label{prelim_massshell_estimates}
In this section we prove estimates for momenta that are contained in the mass-shell. We express our results in $(t^*,r)$-coordinates as introduced in Section~\ref{rnmetricmanifold} and consider the conjugate coordinates $(x,p) = (t^*,r,\omega,p^{t^*},p^r,\pslash)$ as functions on the mass-shell $\mathcal{P}$. Lemma~\ref{ptstar_bound} establishes a bound for the $p^{t^*}$-component and Lemma~\ref{lem_quotientbound} provides a bound for the quotient ${p^{t^*}}/{p^r}$.

\begin{lem} \label{ptstar_bound}
Let  $(x,p) = (t^*,r,\omega, p^{t^*},p^r,\pslash) \in \mathcal{P}$. Then if $p^r > 0$ and $r > 2m$ we have the bound
\begin{equation} \label{ptstar_bound_eq1}
	\frac{m^2}{r^2} E \left( 1 + \left| \trapschw \right| \right) \lesssim p^{t^*} \leq \frac{2 E}{\OsqS},
\end{equation}
where $\trapschw$ denotes the trapping parameter introduced in Definition~\ref{def_eps_def}. If $p^r \leq 0$ and $r \geq 2m$ then
\begin{equation} \label{ptstar_bound_eq3}
	\frac{m^2}{r^2} E \left( 1 - \trapschw \right) \lesssim p^{t^*} \lesssim \left( 1 + \frac{m^2}{r^2} \left| \trapschw \right| \right) E \lesssim \frac{E}{\OsqS}.
\end{equation}
In particular, we may conclude that if $\delta > 0$ and $(2+\delta)m \leq r$ we have
\begin{equation}
	p^{t^*} \lesssim_\delta E.
\end{equation}
\end{lem}
\begin{proof}
We use Lemma~\ref{express_ptstar} to express $p^{t^*}$ as
\begin{equation}
p^{t^*} = \frac{E + \frac{2m}{r} p^r}{\OsqS}.
\end{equation}
Let us first assume that $p^r > 0$ and $r > 2m$. Using the fact that conservation of energy~\eqref{Schw::energy_conservation} implies $\left| p^r \right| \leq E$ we readily conclude
\begin{equation}
	\frac{E}{\OsqS} \leq p^{t^*} \leq 2 \frac{E}{\OsqS}.
\end{equation}
Now recall inequality~\eqref{epsbound} which immediately implies $\OsqS \frac{m^2}{r^2} \left( 1 + \left| \trapschw \right| \right) \lesssim 1$. Therefore we can conclude the bound
\begin{equation}
	\frac{m^2}{r^2} \left( 1 + \left| \trapschw \right| \right) E \lesssim \frac{E}{\OsqS} \leq p^{t^*}.
\end{equation}
Let us turn to the case that $p^r \leq 0$ and $r \geq 2m$. In this case, we again apply Lemma~\ref{express_ptstar} to represent $p^{t^*}$ as in equality~\eqref{express_ptstar_neg} in order to deduce
\begin{equation}
	p^{t^*} = \left| p^r \right| + \frac{\left| \pslash \right|_{\gslash}^2}{E + \left| p^r \right|} \leq E +  \frac{\left| \pslash \right|_{\gslash}^2}{E} = E \left( 1 + \frac{1}{r^2} \frac{L^2}{E^2} \right) \lesssim E \left( 1 + \frac{m^2}{r^2} \left| \trapschw \right| \right),
\end{equation}
where we have used the inequality $\frac{27}{m^2} \frac{L^2}{E^2} = 1 - \trapschw \leq 1 + \left| \trapschw \right|$ by definition of the trapping parameter. In the other direction we find
\begin{equation}
	p^{t^*} \geq \frac{\left| \pslash \right|_{\gslash}^2}{E} = \frac{1}{r^2} \frac{L^2}{E^2} E \gtrsim \frac{m^2}{r^2} \left( 1 - \trapschw \right) E.
\end{equation}
This concludes the proof.
\end{proof}

\begin{lem} \label{lem_quotientbound}
For $(x,p) = (t^*,r,\omega, p^{t^*},p^r,\pslash) \in \mathcal{P}$ and $\signpr = \sgn(p^r) \in \{ -1, + 1\}$ we have
\begin{equation} \label{lem_quotientbound_eqn}
	\frac{p^r}{p^{t^*}} \sim \signpr \sqrt{(r-3m)^2+ \trapschw \mathfrak{a}} \, \frac{1}{r} \mathfrak{A}_s,
\end{equation}
where $\mathfrak{a} \sim m^2 \OsqS$, $\trapschw$ denotes the trapping parameter from Definition~\ref{def_eps_def} and for any $0 < \delta < 1$
\begin{equation}
	\mathfrak{A}_s \sim_\delta \begin{cases}
		1 & r \geq (2+\delta)m \\
		\begin{cases}
			\frac{1}{2- \trapschw} & \signpr=-1 \\
			\OsqS & \signpr=1
		\end{cases} & 2m \leq r \leq (2+ \delta)m
	\end{cases}.
\end{equation}
In particular, away from the horizon the quotient takes an identical form for $\signpr \in \{-1,+1\}$ apart from the obvious difference in sign. As an immediate consequence, we conclude that $\left| p^r \right| \lesssim p^{t^*}$.
\end{lem}

\begin{rem} \label{rem_prelimestimates_continuity}
	The quotient $\frac{p^r}{p^{t^*}}$ is continuous as a function on the mass-shell $\mathcal{P}$ up to and including the event horizon. It may however be easily verified that the function on the right hand side of equation~\eqref{lem_quotientbound_eqn} is discontinuous at any point $(x,p) = (t^*,r,\omega,p^{t^*},p^r,\pslash) \in \mathcal{P}$ that satisfies $r=r_+$ and $p^r = 0$. We will however only be interested in evaluating equation~\eqref{lem_quotientbound_eqn} along null geodesics, and the reader may readily verify that the right hand side is continuous along every future-directed null geodesic.
\end{rem}

\begin{proof}
	We begin by noting that the left hand side of~\eqref{def_eps} expressed in $(t^*,r)$-coordinates reads
	\begin{equation}
		E^2 - \frac{L^2}{27 m^2} = - \OsqS a (p^{t^*})^2 + \frac{4m}{r} a \, p^{t^*} p^r + b (p^r)^2,
	\end{equation}
	where we have abbreviated
	\begin{equation} \label{def_ab}
		a = \left(1-\frac{3m}{r}\right)^2 \left(1+\frac{6m}{r}\right) \frac{r^2}{27m^2}, \quad b = \frac{r^2}{27 m^2} \left(1+\frac{2m}{r}\right)+\frac{4m^2}{r^2}.
	\end{equation}
	The right hand side becomes
	\begin{equation}
		\trapschw E^2 = \trapschw \left( \OsqS p^{t^*} - \frac{2m}{r} p^r \right)^2 = \trapschw \left( (\OsqS)^2 (p^{t^*})^2 - \frac{4m}{r} \OsqS p^{t^*} p^r + \frac{4m^2}{r^2} (p^r)^2 \right).
	\end{equation}
	Collecting terms on one side results in the following quadratic equation:
	\begin{equation}
		\left( b - \trapschw \frac{4m^2}{r^2} \right) (p^r)^2 + \frac{4m}{r} \left( a + \trapschw \OsqS  \right) p^{t^*} p^r - \OsqS  \left( a + \trapschw \OsqS  \right) (p^{t^*})^2 = 0.
	\end{equation}
	We may solve this to express $p^r$ as a function of $r,p^{t^*}$ and $\trapschw$. Before we do so let us consider the discriminant of the quadratic equation which is given by
	\begin{equation}
		\Delta = (a + \trapschw \OsqS) \left( \frac{16m^2}{r^2}a + 4 \OsqS b \right) (p^{t^*})^2,
	\end{equation}
	so that $\sgn(\Delta) = \sgn(a + \trapschw \OsqS)$. Therefore the equation has no (real) solution when $a+ \trapschw \OsqS < 0$ which happens exactly when $\trapschw < 0$ and the radius is in the neighbourhood of the photon sphere specified by $\{ r \in [2m, \infty) \, | \, (r-3m)^2(r+6m) < |\trapschw| 27 m^2 (r-2m) \}$. This follows naturally from energy conservation since $\trapschw < 0$ implies that the energy is insufficient to reach a certain neighbourhood of the photon sphere, where the potential energy reaches its maximum. Recall from above that the condition $E > 0$ (which dictates that $p$ be future-directed) implies $p^{t^*} > 0$, and if $r=2m$ and $E=0$, then $p^{t^*} \geq 0$. We therefore obtain
	\begin{equation} \label{quotient}
		p^r = \frac{-\frac{4m}{r} (a + \trapschw \OsqS) + \signpr \sqrt{(a + \trapschw \OsqS) \left( \frac{16m^2}{r^2}a + 4 \OsqS b \right)}}{2\left( b - \trapschw \frac{4m^2}{r^2} \right)} p^{t^*},
	\end{equation}
	where $\signpr \in \{ +1,-1\}$ is the usual sign in the formula for quadratic equations. Let us now study the quotient $\frac{p^r}{p^{t^*}}$ in a bit more detail. \\
	
	We begin by showing that $\sgn(p^r) = \signpr$. Note that from $\trapschw \in (-\infty,1]$ and the definition of $b$ in~\eqref{def_ab} we have that $b - \trapschw \frac{4m^2}{r^2} \geq \frac{r^2}{27 m^2} \left(1+\frac{2m}{r}\right) > 0$. Recall also that $\sgn(\Delta) = \sgn(a+\trapschw \OsqS)$ so that we may assume $a+\trapschw \OsqS > 0$. Therefore it is immediately apparent that if $\signpr = -1$ then $p^r < 0$. Conversely if $\signpr=+1$ we make some simple rearrangements
	\begin{align}
		p^r|_{\signpr=1} > 0 &\iff - \frac{4m}{r} \sqrt{a + \trapschw \OsqS} + \sqrt{\frac{16m^2}{r^2}a + 4 \OsqS b} > 0 \\
		&\iff \frac{16m^2}{r^2}a + 4 \OsqS b > \frac{16m^2}{r^2} (a + \trapschw \OsqS) \\
		&\iff b > \frac{4m^2}{r^2} \trapschw,
	\end{align}
	which we know to be true from the bound on $b$ mentioned already above. Thus we conclude that $\sgn(p^r) = \signpr$ as claimed. \\
	
	We proceed by bounding various expressions occuring in the quotient. Let us begin by looking at the denominator $b-\trapschw \frac{4m^2}{r^2}$. First note that on the whole exterior, $b \sim \frac{r^2}{m^2}$. For given $r$, we know that $1 - \frac{1}{27m^2} \frac{r^2}{\OsqS} \leq \trapschw \leq 1$. Inserting this we find
	\begin{equation} \label{ERN:quotient_denominator}
		\frac{r^2}{27m^2} \left( 1+ \frac{2m}{r} \right) \leq b-\trapschw \frac{4m^2}{r^2} \leq \frac{r^2}{27m^2} \frac{1}{\OsqS}.
	\end{equation}
	In particular we immediately see that if we divide the exterior into two regions $r-2m \geq \delta m > 0$ (far from the horizon) and $0 \leq r-2m \leq \delta m$ (close to the horizon), then far from $\mathcal{H}^+$ we have $b- \trapschw \frac{4m^2}{r^2} \sim_\delta \frac{r^2}{m^2}$ and close to $\mathcal{H}^+$ one finds $b- \trapschw \frac{4m^2}{r^2} \sim_\delta 2 -  \trapschw$. In fact, in view of the bounds for $\trapschw$ this means that in the whole exterior
	\begin{equation}
		b- \trapschw \frac{4m^2}{r^2} \sim \frac{r^2}{m^2} - \trapschw \frac{m^2}{2r^2}.
	\end{equation}
	For the expression $a + \trapschw \OsqS$ we first note that
	\begin{equation}
		0 \leq a + \trapschw \OsqS \lesssim \frac{r^2}{m^2}.
	\end{equation}
	By introducing the shorthand $\mathfrak{a} = 27 m^2 \frac{r-2m}{r+6m}$ we may rewrite
	\begin{equation}
		a + \trapschw \OsqS = \left(1+\frac{6m}{r}\right) \frac{1}{27m^2} \left( (r-3m)^2 + \trapschw \mathfrak{a} \right) \sim \frac{1}{m^2} \left( (r-3m)^2 + \trapschw \mathfrak{a} \right),
	\end{equation}
	where we note that $\OsqS \lesssim \frac{\mathfrak{a}}{m^2} \lesssim 1$ in the exterior. \\
	
	Let us now simplify the quotient~\eqref{quotient} in the case that $\signpr=-1$. We begin by looking at the numerator and note that in the whole exterior
	\begin{align*}
		&-\frac{4m}{r} (a + \trapschw \OsqS) - \sqrt{(a + \trapschw \OsqS) \left( \frac{16m^2}{r^2}a + 4 \OsqS b \right)} \\
		= &- \sqrt{a + \trapschw \OsqS} \left[ \underbrace{\frac{4m}{r} \sqrt{a + \trapschw \OsqS}}_{\lesssim 1} + \underbrace{\sqrt{\frac{16m^2}{r^2}a + 4 \OsqS b }}_{\sim \frac{r}{m}} \right] \\
		\sim &-\sqrt{a + \trapschw \OsqS} \frac{r}{m}.
	\end{align*}
	Together with the bounds~\eqref{ERN:quotient_denominator} for the denominator we find that in the whole exterior if $\signpr=-1$,
	\begin{equation}
		\frac{p^r}{p^{t^*}} \sim - \sqrt{(r-3m)^2+ \trapschw \mathfrak{a}} \frac{r}{r^2 - \trapschw \frac{m^4}{2r^2}}.
	\end{equation}
	For convenience, let us note that the second factor looks like
	\begin{equation}
		\frac{r}{r^2 - \trapschw \frac{m^4}{2r^2}} \sim_\delta \begin{cases}
			\frac{1}{m(2- \trapschw)} & 2m \leq r \leq (2+\delta)m \\
			\frac{1}{r} & (2+\delta)m \leq r
		\end{cases}.
	\end{equation}
	
	Let us now consider the case that $p^r \geq 0$. We again begin by looking at the numerator, this time with $\signpr=1$:
	\begin{align*}
		&-\frac{4m}{r} (a + \trapschw \OsqS) + \sqrt{(a + \trapschw \OsqS) \left( \frac{16m^2}{r^2}a + 4 \OsqS b \right)} \\
		= & \sqrt{a + \trapschw \OsqS} \left[ \sqrt{\frac{16m^2}{r^2}a + 4 \OsqS b } - \sqrt{\frac{16m^2}{r^2} \left(a + \trapschw \OsqS \right)} \right].
	\end{align*}
	Let us study the expression in square brackets. For brevity let us define $x = \frac{16m^2}{r^2}a + 4 \OsqS b$ and $y = \frac{16m^2}{r^2} \left(a + \trapschw \OsqS \right)$. From above we know that $y \geq 0, x \sim \frac{r^2}{m^2}$ and further $x-y = 4 \OsqS (b- \trapschw \frac{4m^2}{r^2}) \geq 0$. We now make use of the following simple relation\footnote{Since $0 \leq z \leq 1$ the inequality $1-z \geq 1-\sqrt{z}$ is trivial. For the other direction note simply that $0 \leq (1-\sqrt{z})^2 = 1 - 2 \sqrt{z} + z$ which is equivalent to $1- z \leq 2 (1 - \sqrt{z})$.}: For $0 \leq z \leq 1$ we have $1-\sqrt{z} \sim 1-z$. Let us apply this with $z = \frac{y}{x}$, so that $\sqrt{x} - \sqrt{y} = \sqrt{x} \left( 1 - \sqrt{\frac{y}{x}} \right) \sim \frac{1}{\sqrt{x}} (x-y) \sim \frac{m}{r} \OsqS (b- \trapschw \frac{4m^2}{r^2})$. It thus follows that
	\begin{equation}
		\frac{p^r}{p^{t^*}} \sim \frac{1}{r} \OsqS \sqrt{(r-3m)^2+ \trapschw \mathfrak{a}}.
	\end{equation}
	This concludes the proof.
\end{proof}



\subsection{The almost-trapping estimate}  \label{section_tstar}
In this section we estimate the time a null geodesic requires to cross a certain region of spacetime in terms of the radius travelled by the geodesic and the value of its trapping parameter. The main result of this section is

\begin{lem}[The almost-trapping estimate] \label{taubound}
Let $\gamma: [s_0,s_1] \rightarrow \Mrn$ be  an affinely parameterised future-oriented null geodesic. Let us express $\gamma$ in $(t^*,r)$-coordinates as $\gamma(s) = (t^*(s),r(s),\omega(s))$ with momentum $\dot{\gamma}(s) = (p^{t^*}(s),p^r(s),\pslash(s))$. Assume that $\gamma$ intersects $\Sigma_0$ at radius $r_0 = r(s_0) \leq \rsuppconst$. Let $0 < \delta < 1$ and denote $\mathfrak{s} = \chi_{(0,\infty)}\left( p^r(s_0) \right)$, where $\chi_{(0,\infty)}$ denotes the characteristic function of the set $(0,\infty) \subset \R$. There exists a constant $C > 0$ such that if we assume $\tauzero \geq C(m + \frac{\rsuppconst^2}{m})$ then for all $s \in [s_0,s_1]$ with $2m \leq r(s) \leq (2+\delta)m$ we have the bound
\begin{equation}
	\frac{1}{m} \left( \tau(\gamma(s)) - \tauzero \right) \lesssim_{\delta}
	\begin{cases}
		\left| \log \left( \frac{\OsqS(r_0)}{\OsqS(r(s))} \right) \right| & \text{if } p^r(s) > 0 \\
		\mathfrak{s} \left( \log \, (1+\left| \trapschw \right|) \OsqS(r_0) \right)_- + \left( \log \left| \trapschw \right| \right)_-  & \text{if } p^r(s) \leq 0
	\end{cases},
\end{equation}
and if $r(s) \geq (2+\delta)m$ we have
\begin{equation}
	\frac{1}{m} \left( \tau(\gamma(s)) - \tauzero \right) \lesssim_{\delta} \left( \log \left| \trapschw \right| \right)_- + \mathfrak{s} \left( \log \, (1+\left| \trapschw \right|) \OsqS(r_0) \right)_-,
\end{equation}
independent of the sign of $p^r(s)$, where we have abbreviated $\trapschw = \trapschw(\gamma,\dot{\gamma})$.
\end{lem}

The proof will proceed by first establishing a bound for the time measured in the $t^*$-coordinate introduced in Section~\ref{rnmetricmanifold}. Specifically, let $\gamma: [s_1,s_2] \rightarrow \Mschw$ be an affinely parameterised future-directed null geodesic segment with affine parameter $s$ and let us express $\gamma(s) = (t^*(s),r(s),\omega(s))$ in $(t^*,r)$-coordinates. We derive an estimate for the time $t^*(s_2) - t^*(s_1)$ in terms of the radii $r(s_1),r(s_2)$ and the trapping parameter $\trapschw(\gamma)$ in Lemma~\ref{tstar_lem_schw}. Because of the mixed spacelike-null nature of the time function $\tau$, we consider the asymptotically flat region separately in Lemma~\ref{tauestimate_far}.
We prove that in this region, the behaviour of geodesics is comparable to the flat (Minkowski) case. Finally we combine the $t^*$-estimates with the estimates in the asymptotically flat region to obtain a bound on the time $\tau(\gamma(s_2)) - \tau(\gamma(s_1))$ in terms of $r(s_1),r(s_2)$ and $\trapschw(\gamma)$ and prove Lemma~\ref{taubound}. See Section~\ref{sec_radialgeod} for an informal discussion of the proof in the special case of radial geodesics.

\subsubsection{The $t^*$-time estimate} \label{sec_tstarestimate}
In this section, we derive a bound for the time a null geodesic requires to cross a certain region of spacetime, where we measure time using the $t^*$-coordinate. Let us split the black hole exterior in several regions to simplify our statement of the time estimate. Fix three constants $0< \delta_1,\delta_2 < 1$ and $\delta_3 > 0$ such that $0 < \delta_1 + \delta_2 < 1$. Consider the following intervals of Schwarzschild radius
\begin{align}
	\mathfrak{I}_{\mathcal{H}^+} &= [2m, (2+\delta_1)m], \\
	\mathfrak{I}_{\text{int}} &= [(2+\delta_1)m, (3-\delta_2)m], \\
	\mathfrak{I}_{\text{ps}} &= [(3-\delta_2)m, (3+\delta_3)m], \\
	\mathfrak{I}_{\text{flat}} &= [(3+\delta_3)m, \infty),
\end{align}
where we have suppressed the dependence on $\delta_1,\delta_2,\delta_3$ in the notation. Therefore $r \in \mathfrak{I}_{\mathcal{H}^+}$ signifies closeness to the event horizon and $r \in \mathfrak{I}_{\text{ps}}$ closeness to the photon sphere, whereas the interval $\mathfrak{I}_{\text{int}}$ denotes the region between the horizon and photon sphere (at positive distance from both) and $\mathfrak{I}_{\text{flat}}$ denotes the asymptotically flat region.

\begin{lem} \label{tstar_lem_schw}
Denote by $\gamma$ an affinely parameterised future-oriented null geodesic expressed in $(t^*,r)$-coordinates as $\gamma(s) = (t^*(s),r(s),\omega(s))$ with momentum $\dot{\gamma}(s) = (p^{t^*}(s),p^r(s),\pslash(s))$. Assume that $p^r \neq 0$ in the interval of affine parameter time $[s_1,s_2]$, so that the radius $r$ is a strictly monotone function of $s$. Let us assume that $r(s_i) = r_i$ for $i=1,2$ and let $0 < \delta_1,\delta_2,\delta_3$ be as above. Then
\begin{equation} \label{t*bound}
	\frac{t^*(s_2) - t^*(s_1)}{m} \lesssim_{\delta_1,\delta_2,\delta_3 } \begin{cases}
		\begin{cases}
			1 & p^r \leq 0 \text{ on } [r_2,r_1] \\
			1 + \left| \log \left( \frac{\OsqS(r_1)}{\OsqS(r_2)} \right) \right| & p^r > 0 \text{ on } [r_1,r_2]
		\end{cases} & r_1,r_2 \in \mathfrak{I}_{\mathcal{H}^+} \\
		1 & r_1,r_2 \in \mathfrak{I}_{\text{int}} \\
		1 + \left( \log \left| \trapschw \right| \right)_- & r_1,r_2 \in \mathfrak{I}_{\text{ps}} \\
		1 + \frac{\max(r_1,r_2)}{m} & r_1,r_2 \in \mathfrak{I}_{\text{flat}}
	\end{cases}
\end{equation}
where we have used the shorthand $\trapschw = \trapschw(\gamma(s),\dot{\gamma(s)})$.
\end{lem}
\begin{proof}
For the entirety of the proof, all momenta and radii are considered along the geodesic $\gamma$. We will for the most part omit the dependence on the affine parameter $s$ and write for instance $p^r = p^r(s)$ to simplify notation. Since in the interval of affine parameter time $[s_1,s_2]$ we have $p^r \neq 0$ the radius $r$ is a strictly monotone function of $s$ and we have
\begin{equation}
t^*(s_2) - t^*(s_1) = \int_{s_1}^{s_2} \frac{dt^*}{d s} \, d s = \int_{s_1}^{s_2} p^{t^*} \, d s = \int_{r_1}^{r_2} \frac{p^{t^*}}{p^r} \, d r,
\end{equation}
where we carefully note that if $p^r < 0$, the integration boundaries will be such that $r_2 < r_1$ so that the orientation of the integral ensures the correct (positive) sign. We will now use Lemma~\ref{lem_quotientbound} to bound the integrand in different regions of spacetime.

\paragraph{Region close to horizon.}
Let us begin by assuming that $r_1,r_2 \in \mathcal{I}_{\mathcal{H}^+}$ and simply write $\delta = \delta_1$. Let us first assume that $p^r \leq 0$. Note that this means that the geodesic will eventually fall into the black hole. We may therefore estimate
\begin{equation}
    \int_{r_1}^{r_2} \frac{p^{t^*}}{p^r} \, d r \leq \int_{2m}^{r_1} \frac{p^{t^*}}{\left| p^r \right|} \, d r.
\end{equation}
If $\trapschw \geq 0$, clearly $(r-3m)^2 + \trapschw \mathfrak{a} \sim_\delta m^2$ and $\frac{r}{r^2 - \trapschw m^2} \sim_\delta \frac{1}{m(2- \trapschw)} \sim \frac{1}{m}$, so that $\frac{p^{t^*}}{|p^r|} \sim_\delta 1$. Hence
\begin{equation}
\int_{2m}^{r_1} \frac{p^{t^*}}{|p^r|} \, dr \sim_\delta r_1-2m \lesssim_\delta m.
\end{equation}
If $\trapschw < 0$, let us rewrite 
\begin{equation}
(r-3m)^2 + \trapschw \mathfrak{a} = (r-3m)^2 \left( 1 + \trapschw (r-2m) \mathfrak{b} \right) \sim_\delta m^2 \left( 1 + \trapschw (r-2m) \mathfrak{b} \right) = m^2 \left( 1 + \trapschw x \mathfrak{b} \right),
\end{equation}
where we have introduced the change of coordinates $x=r-2m$ and defined the coefficient $\mathfrak{b}(x) = 27m^2 (x+8m)^{-1}(x-m)^{-2}$. Note that $m \mathfrak{b} \sim_\delta 1$ in the region of integration. We find
\begin{equation}
\int_{2m}^{r_1} \frac{p^{t^*}}{|p^r|} \, dr \sim_\delta \int_{2m}^{r_1} \frac{2-\trapschw}{\sqrt{1+\trapschw (r-2m) \mathfrak{b}}} \, dr = \int_{0}^{r_1-2m} \frac{2-\trapschw}{\sqrt{1+\trapschw x \mathfrak{b}}} \, dx \sim_\delta \int_0^{y_1} \frac{2-\trapschw}{\sqrt{1+\trapschw y}} \, dy,
\end{equation}
where we have made the change of coordinates $y = x \mathfrak{b}(x)$ and $y_1 = (r-1-2m) \mathfrak{b}(r_1-2m)$. In the last step we made use of the fact that the Jacobian
\begin{equation}
    \left|\mathfrak{b} + x \mathfrak{b}' \right|^{-1} = \frac{(x-m)^3 (x+8m)^2}{54 m^2 (x+2m)^2} \sim_{\delta} m
\end{equation}
in the region of concern. Finally we compute
\begin{align} \label{corrboundSS}
\int_0^{y_1} \frac{2-\trapschw}{\sqrt{1+\trapschw y}} \, dy &= \frac{2 + |\trapschw|}{|\trapschw|} \int_0^{|\trapschw| y_1} \frac{1}{\sqrt{1-z}} \, dz = 2 \frac{2 + |\trapschw|}{|\trapschw|} \left( 1-\sqrt{1-|\trapschw| y_1} \right) \\
 &\sim \frac{2 + |\trapschw|}{|\trapschw|} |\trapschw| y_1 = (2 + |\trapschw|) y_1 \lesssim_\delta 1,
\end{align}
where in the last step we used that it follows from the bounds on $\trapschw$ that $0 \leq y_1 \leq \frac{1}{|\trapschw|}$. In total it follows that regardless of the sign and size of $\trapschw$ if $p^r \leq 0$ we have
\begin{equation} \label{time::infalling_horizon}
\int_{2m}^{r_1} \frac{p^{t^*}}{|p^r|} \, dr \lesssim_\delta m.
\end{equation}

Let us now consider the case that $p^r \geq 0$. If $\trapschw \geq 0$, we note $(r-3m)^2 + \trapschw \mathfrak{a} \sim_\delta m^2$ like in the case where $p^r \leq 0$. In this case however it follows that $\frac{p^{t^*}}{|p^r|} \sim_\delta \frac{1}{\OsqS}$. Therefore
\begin{equation}
\int_{r_1}^{r_2} \frac{p^{t^*}}{|p^r|} \, dr \sim_\delta \int_{r_1}^{r_2} \frac{1}{\OsqS} \, dr \sim_\delta m \int_{\OsqS(r_1)}^{\OsqS(r_2)} \frac{1}{z} \, dz = m \left| \log \left( \frac{\OsqS(r_1)}{\OsqS(r_2)} \right) \right|,
\end{equation}
where we introduced the change of coordinates $z = \OsqS(r)$ and made use of the fact that the Jacobian $\frac{r^2}{2m} \sim_\delta m$. If $\trapschw < 0$, we rewrite as above
\begin{equation}
    (r-3m)^2 + \trapschw \mathfrak{a} = (r-3m)^2 \left( 1 + \trapschw (r-2m) \mathfrak{b} \right) \sim_\delta m^2 \left( 1 + \trapschw (r-2m) \mathfrak{b} \right) = m^2 \left( 1 + \trapschw x \mathfrak{b} \right),
\end{equation}
where we use the change of coordinates $x=r-2m$ and define $\mathfrak{b}(x) = 27m^2 (x+8m)^{-1}(x-m)^{-2}$. Note that $m \mathfrak{b} \sim_\delta 1$ in the region of integration. We compute
\begin{equation}
\int_{r_1}^{r_2} \frac{p^{t^*}}{|p^r|} \, dr \sim_\delta \int_{r_1-2m}^{r_2-2m} \frac{1}{x \sqrt{1+\trapschw x \mathfrak{b}(x)}} \, dx \sim_\delta m \int_{y_1}^{y_2} \frac{1}{y \sqrt{1+\trapschw y}} \, dy,
\end{equation}
where we have made the change of variables $y = x \mathfrak{b}(x)$, set $y_i = (r_i-2m) \mathfrak{b}(r_i-2m)$ for $i=1,2$ and made use of the fact that the Jacobian $\left|\mathfrak{b} + x \mathfrak{b}' \right|^{-1} \sim_{\delta} m$ in the region of integration. Observing that $0 \leq 1 + \trapschw y \leq 1$ we then find
\begin{equation}
    \int_{y_1}^{y_2} \frac{1}{y \sqrt{1+\trapschw y}} \, dy = -2 \left[ \tanh ^{-1}\left(\sqrt{1+\trapschw y}\right) \right]_{y_1}^{y_2} = \left[ \log \left( \frac{1-\sqrt{1+\trapschw y}}{1+\sqrt{1+\trapschw y}} \right) \right]_{y_1}^{y_2}.
\end{equation}
Note carefully that since $y_1 < y_2$ we find that $1 \leq (1+\sqrt{1+\trapschw y_1})(1+\sqrt{1+\trapschw y_2})^{-1} \leq 2$ and
\begin{equation}
    \left[ \log \left( \frac{1-\sqrt{1+\trapschw y}}{1+\sqrt{1+\trapschw y}} \right) \right]_{y_1}^{y_2} \leq \log 2 + \log \left( \frac{1-\sqrt{1-\left|\trapschw \right| y_2}}{1-\sqrt{1- \left| \trapschw \right| y_1}} \right) \leq 2 \log 2 + \log \left( \frac{y_2}{y_1} \right),
\end{equation}
where we used that $\left| \trapschw \right| y_i \leq 1$. Plugging in the definition of $y$ we finally find
\begin{equation}
    \log \left( \frac{y_2}{y_1} \right) \lesssim 1 + \log \left( \frac{\OsqS(r_2)}{\OsqS(r_1)} \right),
\end{equation}
which allows us to conclude the desired bound.

\paragraph{Region far from horizon}
Let us now assume that $r_1,r_2 \in \mathfrak{I}_{\text{int}} \cup \mathfrak{I}_{\text{ps}} \cup \mathfrak{I}_{\text{flat}}$ and let us again for simplicity write $\delta = \delta_1$. Since the quotient $\frac{p^{t^*}}{|p^r|}$ is comparable in case $p^r \leq 0$ and $p^r \geq 0$, we do not need to distinguish different cases based on the sign of $p^r$. Notice that $\mathfrak{a} \sim_\delta m^2$ in this region. We introduce the change of variables $x=r-3m$ so that $\sqrt{(r-3m)^2 + \trapschw \mathfrak{a}} = \sqrt{x^2 + \trapschw \mathfrak{a}}$. We then introduce a further change of variable $y = \mathfrak{a}(x)^{-\frac{1}{2}} x$ and compute
\begin{equation} \label{prooftstar_schw_eq}
\int_{r_1}^{r_2} \frac{p^{t^*}}{p^r} \, dr \sim \left| \int_{r_1-3m}^{r_2-3m} \frac{x+3m}{\sqrt{x^2 + \trapschw \mathfrak{a}}} \, dx \right| = \left| \int_{y_1}^{y_2} \frac{y + \frac{3m}{\sqrt{\mathfrak{a}}}}{\sqrt{y^2 + \trapschw}} \, \mathfrak{J}(y) \, dy \right| \sim_\delta m \left| \int_{y_1}^{y_2} \frac{|y| + 1}{\sqrt{y^2 + \trapschw}} \, dy \right| ,
\end{equation}
where we made use of the fact that the Jacobian satisfies
\begin{equation}
    \mathfrak{J} = \left| \mathfrak{a}^{-\frac{1}{2}}  \left( 1- \frac{1}{2} \frac{\mathfrak{a}' }{\mathfrak{a}} x \right) \right|^{-1} = \mathfrak{a}^{\frac{1}{2}} \frac{(x+m)(x+9m)}{(x+3m)^2} \sim_{\delta} m.    
\end{equation}
Note that $\sgn(y) = \sgn(r-3m)$, so that $r_i < 3m $ if and only if $y_i < 0$. It follows from equation~\eqref{prooftstar_schw_eq} that we may assume $y_1 < y_2$ or equivalently $p^r > 0$ without loss of generality. \\

Let us begin by remarking that if $\max( 2 \sqrt{|\trapschw|}, y_1, -y_2) \geq \eta > 0$ then
\begin{equation}
    \int_{r_1}^{r_2} \frac{p^{t^*}}{|p^r|} \, dr \lesssim_{\eta,\delta} 1 + r_2 .
\end{equation}
In other words, if either $2 \sqrt{|\trapschw|} \geq \eta > 0$ or if $|\trapschw|$ is small but the integration boundaries satisfy either $y_1 \leq y_2 \leq - \eta < 0$ or $y_2 \geq y_1 \geq \eta >0$, then the above bound holds. To see this, first assume that $4 |\trapschw| \, \geq \eta^2 >0$. Introduce the rescaled variable $y = \sqrt{|\trapschw|} z$ so that
\begin{equation}
\int_{y_1}^{y_2} \frac{|y| + 1}{\sqrt{y^2 + \trapschw}} \, dy \leq  \sqrt{|\trapschw|} \int_{z_1}^{z_2} \frac{|z| + \frac{2}{\eta}}{\sqrt{z^2 + \sgn(\trapschw)}} \, dz,
\end{equation}
where $ \sqrt{|\trapschw|} z_i = y_i$ for $i=1,2$. We notice that in case $\trapschw > 0$, the integrand is bounded by a constant for the argument $z \in \R$. Therefore we find the bound
\begin{equation}
\int_{r_1}^{r_2} \frac{p^{t^*}}{p^r} \, dr \lesssim_{\eta,\delta} \sqrt{|\trapschw|} (z_2-z_1) = y_2 - y_1 \sim_\delta \left| r_2 - r_1 \right|.
\end{equation}
In case $\trapschw < 0$, let us first consider the case that $z_1 \leq z_2 \leq -1$ and note immediately that since we assume $r_1 \geq (2+\delta)m$, it follows that $\sqrt{|\trapschw|} \leq \left|y_1 \right| = - y_1 \lesssim_\delta 1$. Therefore in this region we have $1 \lesssim_\eta \left|\trapschw \right| \lesssim_\delta 1$ and therefore also $\left|z_1 \right| \sim_{\delta,\eta} \left| y_1 \right|$ so that
\begin{equation}
\int_{z_1}^{z_2} \frac{|z| + \frac{2}{\eta}}{\sqrt{z^2 -1}} \, dz \leq \int_{z_1}^{-1} \frac{|z| + \frac{2}{\eta}}{\sqrt{z^2 -1}} \, dz \lesssim_{\delta,\eta} 1,
\end{equation}
where we merely exploited the local integrability of the integrand. Let us now turn to the case that $\trapschw < 0$ and $1 \leq z_1 \leq z_2$. We notice that
\begin{equation}
\mathfrak{I} := \frac{|z| + \frac{2}{\eta}}{\sqrt{z^2 -1}} \sim_\eta \begin{cases}
\frac{1}{\sqrt{z-1}} & z \in [1,2] \\
1 & z \in [2, \infty]
\end{cases}.
\end{equation}
Hence upon integration, we find that if $z_2 \leq 2$, then
\begin{equation}
    \int_{z_1}^{z_2} \mathfrak{I} \, dz \sim_\eta \int_{z_1}^{z_2} \frac{1}{\sqrt{z-1}} \, dz \leq \sqrt{z_2-z_1} ,   
\end{equation}
making use of the inequality $\sqrt{x}-\sqrt{y} \leq \sqrt{x-y}$ for $x \geq y \geq 0$. When $z_1 \geq 2$ then
\begin{equation}
    \int_{z_1}^{z_2} \mathfrak{I} \, dz \sim_\eta \int_{z_1}^{z_2} 1 \, dz = z_2-z_1.
\end{equation}
The case that $z_1 \leq 2$ and $z_2 \geq 2$ is then covered by summing the above two bounds. In summary we obtain
\begin{equation}
\int_{y_1}^{y_2} \frac{|y| + 1}{\sqrt{y^2 + \trapschw}} \, dy \leq \begin{cases}
|\trapschw|^{\frac{1}{4}} \sqrt{y_2-y_1} & \text{if } y_2,y_1 \sim \sqrt{|\trapschw|} \\
y_2-y_1 & \text{if } y_2,y_1 \geq 2 \sqrt{|\trapschw|}
\end{cases} \; \leq 1 + y_2 \lesssim_\delta 1 + \frac{r_2}{m}.
\end{equation}
We conclude the bound as claimed above. The above computation also reveals the reason why we only show the upper bound $1+y_2$ instead of $y_2-y_1$ for the integral: Suppose $d > 0$, then the time it takes to move from $y_2 = (1+d) \sqrt{|\trapschw|}$ to $y_1 = \sqrt{|\trapschw|}$ is like $\sqrt{|\trapschw|} \sqrt{d} \gg \sqrt{|\trapschw|} d = y_2-y_1$ when $d$ is small. If we now finally instead assume that $\min (|y_1|, |y_2|) \geq \eta > 0$ then by the above argument we may in addition assume $|\trapschw| \, \leq \frac{\eta^2}{4}$. Then it is clear that we have $y^2 + \trapschw \geq \frac{y^2}{2} \geq \frac{\eta^2}{2}$. It follows that
\begin{equation}
    \frac{|y| + 1}{\sqrt{y^2 + \trapschw}} \lesssim_{\eta} 1,
\end{equation}
for $y \notin [-\eta,\eta]$. Therefore we again conclude the better bound 
\begin{equation} \label{timebound_away_from_photonsphere}
\int_{r_1}^{r_2} \frac{p^{t^*}}{p^r} \, dr \lesssim_{\eta,\delta} \left| r_2 - r_1 \right|.
\end{equation}

Therefore we may from here on restrict to the case where both $\left| \trapschw \right| \leq \frac{\eta^2}{4}$ and the region of integration is close to the photon sphere, i.e. $[y_1,y_2] \subset [-\eta,\eta]$. Consider the case that $\trapschw \leq 0$ first. It follows readily from the geodesic equations that the geodesic cannot cross the photon sphere, so that we must either have $- \eta \leq y_1 \leq y_2 \leq -\sqrt{|\trapschw|}$ or $\sqrt{|\trapschw|} \leq y_1 \leq y_2 \leq \eta$. By symmetry of the integral we may assume the latter without loss of generality. We bound
\begin{equation}
\int_{y_1}^{y_2} \frac{y + 1}{\sqrt{y^2 - |\trapschw|}} \, dy \leq \int_{y_1}^{\eta} \frac{y + 1}{\sqrt{y^2 - |\trapschw|}} \, dy = \left[ \sqrt{y^2 - |\trapschw|} + \log \left( y + \sqrt{y^2 - |\trapschw|} \right) \right]_{y_1}^{\eta}.
\end{equation}
We may choose $\eta$ sufficiently small and assume $\left| \trapschw \right| \leq \frac{\eta^2}{4}$ so that the relation
\begin{gather}
    \left( 1 + \sqrt{\frac{3}{4}} \right) \eta \leq \eta + \sqrt{\eta^2 - |\trapschw|} \leq 2 \eta < 1,
\end{gather}
holds. This immediately implies that $\left| \log \left( \eta + \sqrt{\eta^2 - |\trapschw|} \right) \right| \geq \left| \log ( 2 \eta ) \right| > \eta \geq \sqrt{\eta^2 - |\trapschw|}$ and
\begin{equation}
    \sqrt{\eta^2 - |\trapschw|} + \log \left( \eta + \sqrt{\eta^2 - |\trapschw|} \right) < 0.
\end{equation}
Using this information we find
\begin{align}
\left[ \sqrt{y^2 - |\trapschw|} + \log \left( y + \sqrt{y^2 - |\trapschw|} \right) \right]_{y_1}^{\eta} \leq \left| \log \left( y_1 + \sqrt{y_1^2 - |\trapschw|} \right) \right| \leq \left| \log(y_1) \right| \leq \frac{1}{2} \left| \log( \trapschw ) \right|,
\end{align}
where we used that $y_1 \geq \sqrt{|\trapschw|}$ in the last step. Let us now turn to the case where $\trapschw \geq 0$. In this case the geodesic may cross the photon sphere. By symmetry of the integrand around the origin we may therefore assume without loss of generality that either $0 \leq y_1 \leq y_2 \leq \eta$ or $-\eta \leq y_1 < 0 \leq y_2 \leq \eta$. In the former case, we find similar to above
\begin{equation}
\int_{y_1}^{y_2} \frac{y + 1}{\sqrt{y^2 + \trapschw}} \, dy \leq \int_{y_1}^{\eta} \frac{y + 1}{\sqrt{y^2 + \trapschw}} \, dy = \left[ \sqrt{y^2 + \trapschw} + \log \left( y + \sqrt{y^2 + \trapschw} \right) \right]_{y_1}^{\eta}.
\end{equation}
Again we assume  $\left| \trapschw \right| \leq \frac{\eta^2}{4}$ and choose $\eta$ sufficiently small so that similar to above
\begin{equation} \label{yintegralbound}
\left[ \sqrt{y^2 + \trapschw} + \log \left( y + \sqrt{y^2 + \trapschw} \right) \right]_{y_1}^{\eta} \leq \left| \log \left( y_1 + \sqrt{y_1^2 + \trapschw} \right) \right| \leq \frac{1}{2} \left| \log ( \trapschw ) \right|.
\end{equation}
Assume now that $y_1<0$, this means that the geodesic crosses the photon sphere in the affine parameter time interval under consideration. We split the range of integration in two parts $[y_1,y_2] = [y_1,0) \cup [0,y_2]$. For the first part $[y_1,0)$ we apply the transformation $y \mapsto -y$ and therefore find for both $i=1$ and $i=2$ that
\begin{equation}
\int_{0}^{|y_i|} \frac{y + 1}{\sqrt{y^2 + \trapschw}} \, dy \leq \int_{0}^{\eta} \frac{y + 1}{\sqrt{y^2 + \trapschw}} \, dy \leq \frac{1}{2} \left| \log ( \trapschw ) \right|,
\end{equation}
where this bound is simply obtained from~\eqref{yintegralbound} by setting $y_i=0$. This concludes the proof.
\end{proof}

\subsubsection{The asymptotically flat region}

We now turn our attention to the asymptotically flat region $\mathcal{A} = \{ r \geq R \} \cap \Mschw$. Let $\tauzero, R_0 \geq m$ be constants to be chosen later and define
\begin{equation}
    \mathcal{A}_{0} = \left( \left\{ \tau \geq \tauzero \right\} \cup \left\{ r \geq R_0 \right\} \right) \cap \mathcal{A} \subset \mathcal{A}.
\end{equation}

\begin{lem} \label{pr_like_E}
Let us denote by $\gamma: [s_0,s_1) \rightarrow \Mschw$ an affinely parametrised future-oriented null geodesic segment expressed in $(t^*,r)$-coordinates as in Lemma~\ref{tstar_lem_schw}.  Assume that $\gamma(s_0) \in \Sigma_0$ and $r_0 = r(s_0) \leq \rsuppconst$. Let us assume in addition that $\gamma$ is maximally defined, so that either $s_1 < \infty$ and $\lim_{s \rightarrow s_1} r(s) = 2m$ or $s_1 = \infty$. Let $\tauzero \geq C \rsuppconst$ with a suitable constant $C > 1$ and $R_0 = \max(2 \rsuppconst, R)$. Then for all $s \in [s_0,s_1)$ with $\gamma(s) \in \mathcal{A}_0$ we have
\begin{equation}
E \geq p^r(s) \geq \frac{1}{2} E > 0,
\end{equation}
where $E = E(\gamma,\dot{\gamma})$. In particular we may assume all geodesics in the region $\mathcal{A}_0$ to be outgoing.
\end{lem}
\begin{proof}
We will begin by considering the time $\gamma$ requires to cross a certain region measured in $t^*,u$ and $\tau$-time and establishing a relationship between these times. Let $[s_3,s_4] \subset [s_0,s_1)$ and for any function $\mathfrak{t}: \Mschw \rightarrow \R$ let us introduce the shorthand $\Delta \mathfrak{t}(s_3,s_4) = \left| \mathfrak{t}(\gamma(s_4)) - \mathfrak{t}(\gamma(s_3)) \right|$. Assume that $[s_3,s_4] \subset [s_0,s_1)$ is such that $p^r(s) \neq 0$ for $s \in [s_3,s_4]$. Recalling the definition of double null coordinates on $\Mschw$ we then have
\begin{align} \label{delta_u_estimate}
\Delta u(s_3,s_4) &= \int_{s_3}^{s_4} p^u(s) \, ds = \int_{r(s_3)}^{r(s_4)} \frac{p^u}{p^r} \, dr \sim \int_{r(s_3)}^{r(s_4)} \frac{p^{t^*}}{p^r} - \frac{r+2m}{r-2m} \, dr \\
&\sim \Delta t^*(s_3,s_4) - \int_{r(s_3)}^{r(s_4)}  \frac{r+2m}{r-2m} \, dr,
\end{align}
where we carefully note that if $p^r < 0$, the integration boundaries will be such that $r(s_3) < r(s_4)$ so that the orientation of the integral ensures the correct (positive) sign. If $r(s_3),r(s_4) \geq R > 3m$ then by definition of the time function $\tau$ we have $\Delta \tau(s_3,s_4) = \Delta u(s_3,s_4)$. We conclude that
\begin{align}
\Delta \tau(s_3,s_4) + \Delta r(s_3,s_4) \sim \Delta t^*(s_3,s_4) \quad &\text{if } p^r(s) > 0 \; \forall s \in [s_3,s_4], \label{timerelations_outgoing} \\
\Delta \tau(s_3,s_4) \sim \Delta t^*(s_3,s_4) + \Delta r(s_3,s_4) \quad &\text{if } p^r(s) < 0 \; \forall s \in [s_3,s_4]. \label{timerelations_incoming}
\end{align}
Next we show that if we choose $\tauzero$ large enough, then for all $s \in [s_0,s_1)$ such that $\gamma(s) \in \mathcal{A}$ and $\tau(\gamma(s)) \geq \tauzero$ we have $p^r(s) \geq 0$. Carefully note that the geodesic equations imply
\begin{equation} \label{geodequnsign}
    \sgn \left( \frac{d}{ds} p^r(s) \right) = 1 \quad \forall \, s \in [s_0,s_1) \text{ with } r(s) > 3m.
\end{equation}
Therefore if $s^* \in [s_0,s_1)$ is such that $r(s^*) \geq R$ and $p^r(s^*) \geq 0$ then $p^r(s) \geq 0$ for all $s \geq s^*$. Suppose there exists $s^* \in [s_0,s_1)$ such that $p^r(s^*) < 0$. Then it must be true that $p^r(s) < 0$ for all $s \in [s_0,s^*]$. Furthermore it follows that there must exist a time $s' \geq s^*$ such that either $p^r(s') = 0$ or $r(s') = R$. Equation~\eqref{geodequnsign} then implies that for all $s > s'$ we either have $r(s) < R$ or $r(s) \geq R$ and $p^r(s) \geq 0$, so that we need only estimate $\Delta \tau(s_0,s')$. Using equation~\eqref{timerelations_incoming} and Lemma~\ref{tstar_lem_schw} we find that
\begin{equation}
    \Delta \tau(s_0,s') \lesssim \Delta t^*(s_0,s') + \Delta r(s_0,s') \lesssim 1 + 2 r(s_0) \lesssim \rsuppconst,
\end{equation}
where we used that by assumption $\gamma$ intersects $\Sigma_0$ at radius $r(s_0) = r_0 \leq \rsuppconst$. Therefore we have shown the existence of a constant $C > 0$ such that for any parameter time $s$ with $r(s) \geq R$ and $\tau(\gamma(s)) \geq C \rsuppconst$, we must have $p^r(s) \geq 0$. Let us denote the smallest parameter time $s \in [s_0,s_2)$ such that $r(s) \geq R$ and $p^r(s) \geq 0$ by $s'$ (assuming it exists). Then we have just shown that $\tau(\gamma(s')) \leq C \rsuppconst$ and further that $R \leq r(s') \leq \rsuppconst$. \\

Next we show that we only need to wait a finite further time until $p^r \geq \frac{1}{2} E$. Let us recall the energy conservation law along the geodesic stated in $(t^*,r)$-coordinates
\begin{equation}
E^2 = (p^r)^2 + \frac{\OsqS}{r^2} L^2 \iff \left( \frac{p^r}{E} \right)^2 = 1 - \frac{\OsqS}{r^2} \left( \frac{L}{E} \right)^2.
\end{equation}
First note that this immediately implies the bound $\frac{r^2}{\OsqS} \geq \left( \frac{L}{E} \right)^2$ for every point on the mass-shell. For any point $(x,p) \in \mathcal{P}$ the inequality $p^r \geq \frac{1}{2} E$ is then equivalent to the improved bound
\begin{equation} \label{improvedbound}
\frac{3}{4} \frac{r^2}{\OsqS} \geq \left( \frac{L}{E} \right)^2.
\end{equation}
Suppose this bound holds when evaluated at the point $(\gamma(\tilde{s}),\dot{\gamma}(\tilde{s}))$ for some $\tilde{s} \geq s'$, then it will hold for all future times $s \geq \tilde{s}$ since $p^r \geq 0$ and the function $\frac{r^2}{\OsqS}$ is increasing in $r$ when $r \geq 3m$. Let therefore $\tilde{s}$ be the parameter time such that the radius $\tilde{r} = r(\tilde{s})$ satisfies the equality
\begin{equation}
\frac{3}{4} \frac{r^2}{\OsqS}\Bigg|_{r=\tilde{r}} = \frac{r^2}{\OsqS} \Bigg|_{r=r(\tau')}.
\end{equation}
Then the inequality $\frac{r^2}{\OsqS} \geq \left( \frac{L}{E} \right)^2$ which holds for every point on the mass-shell implies that the improved bound~\eqref{improvedbound} holds for every time $s \geq \tilde{s}$. Our choice of $\tilde{r}$ implies that $\tilde{r} \sim r(\tau') \lesssim \rsuppconst$. From~\eqref{timerelations_outgoing} it then follows that $\Delta \tau(s',\tilde{s}) \lesssim \Delta t^*(s',\tilde{s}) \lesssim 1 + \tilde{r} \lesssim \rsuppconst$ as claimed. Therefore there exists a constant $C > 0$ such that for any parameter time $s$ with $r(s) \geq R$ and $\tau(\gamma(s)) \geq C \rsuppconst$, we must have $p^r(s) \geq \frac{E}{2}$. Let us therefore define $\tauzero = C \rsuppconst$. \\

Finally let us show that if we assume $r$ is sufficiently large but do not place a restriction on the time $\tau$ we may still conclude $p^r \geq \frac{1}{2}E$. To see this, recall that this inequality is equivalent to the improved bound
\begin{equation}
\frac{3}{4} \frac{r^2}{\OsqS} \geq \left( \frac{L}{E} \right)^2.
\end{equation}
Therefore it is enough to choose $R_0$ so that
\begin{equation}
\frac{3}{4} \frac{r^2}{\OsqS}\Bigg|_{r=R_0} \geq \frac{r^2}{\OsqS} \Bigg|_{r=
\rsuppconst},
\end{equation}
since this implies for all $r \geq R_0$
\begin{equation}
\frac{3}{4} \frac{r^2}{\OsqS}\Bigg|_{r} \geq \frac{3}{4} \frac{r^2}{\OsqS}\Bigg|_{r=R_0} \geq \frac{r^2}{\OsqS} \Bigg|_{r=
\rsuppconst} \geq \frac{r^2}{\OsqS} \Bigg|_{r=r(0)} \geq \left( \frac{L}{E} \right)^2.
\end{equation}
Therefore we may choose $R_0 = 2\rsuppconst$ to satisfy the above bound.
\end{proof}

We will now derive an estimate in the subset $\mathcal{A}_0$. Note that the complement of $\mathcal{A}_0$ in the asymptotically flat region $\mathcal{A}$ is the compact set
\begin{equation}
\mathcal{A} \setminus \mathcal{A}_0 = \left\{ 0 \leq \tau \leq \tauzero \wedge R \leq r \leq R_0 \right\}.
\end{equation}
Therefore we only leave a compact set where we do not prove estimates. We show in Lemma~\ref{psupport_bounded} below that solutions with compactly supported initial data have compact momentum support for all times. This immediately implies that all moments of solutions to the massless Vlasov equation remain bounded in any compact region of spacetime.

\begin{lem} \label{tauestimate_far}
Denote by $\gamma: [s_0,s_1] \rightarrow \Mschw$ an affinely parametrised future-oriented null geodesic segment. Assume that $\gamma(s_0) \in \Sigma_0$ and $r_0 = r(s_0) \leq \rsuppconst$. Let $s^* \in [s_0,s_1]$ be such that $\gamma([s^*,s_1]) \subset \mathcal{A}_0$. If we express $\gamma$ in double null coordinates, then the $p^v$-component remains approximately constant along $\gamma$ in the region $\mathcal{A}_0$. In fact for all $s \in [s^*,s_1]$:
\begin{equation}
    p^v(s) \sim E(\gamma,\dot{\gamma}).
\end{equation}
Furthermore for all $s \in [s^*,s_1]$ we have the bound
\begin{equation}
    r(s)^2 p^u(s) \lesssim \rsuppconst^2 p^v(s).
\end{equation}
In addition for all $[s_3,s_4] \subset [s^*,s_1]$ we have the bound
\begin{equation} \label{taubound_far}
\tau(\gamma(s_4)) - \tau(\gamma(s_3)) \lesssim m \left( 1 + \frac{\rsuppconst^2}{m^2} \right),
\end{equation}
or in other words outgoing null geodesics in this region approximately move along $\Sigma_\tau \cap \{ r \geq R \}$.
\end{lem}

\begin{rem}
We conclude that, measured in double null coordinates, outgoing null geodesics in the asymptotically flat region behave essentially like in flat Minkowski space, see~\cite[Section 7]{martin}. Recall however that there is a logarithmic divergence between double null coordinates in Schwarzschild and Minkowski space.
\end{rem}

\begin{proof}
 Recall that by Lemma~\ref{pr_like_E} we may in particular assume all geodesics in the region $\mathcal{A}_0$ to be outgoing. We begin by showing that along $\gamma$ the $p^v$ component remains approximately constant while the $p^u$ component decays in $r$ as $r \rightarrow \infty$. By Lemma~\ref{pr_like_E} we have the bound $\frac{1}{2} \leq \frac{p^r(s)}{E} \leq 1$ for $s \in [s^*,s_1]$ or equivalently when $\gamma(s) \in \mathcal{A}_0$. Let us express the fraction $\frac{p^r}{E}$ first in $(t,r^*)$-coordinates and then in double null coordinates:
\begin{equation}
\frac{p^r}{E} = \frac{\OsqS p^{r^*}}{\OsqS p^t} = \frac{p^v-p^u}{p^v+p^u} = \frac{1 - \frac{p^u}{p^v}}{1 + \frac{p^u}{p^v}},
\end{equation}
which upon noticing that the function $x \mapsto \frac{1-x}{1+x}$ is self-inverse gives for $s \in [s^*,s_1]$
\begin{equation} \label{p3p4}
\frac{p^u(s)}{p^v(s)} = \frac{1 - \frac{p^r(s)}{E}}{1 + \frac{p^r(s)}{E}} \leq \frac{1}{2}.
\end{equation}
Let us furthermore express the energy in double null coordinates as $E = \OsqS \left( p^u + p^v \right)$. Recall that future-directedness of the geodesic and the mass-shell relation imply $E \geq 0$ and $p^u p^v \geq 0$, which combined allows us to conclude that $p^u(s) \geq 0$ and $p^v(s) \geq 0$. Therefore we immediately have the bound $\OsqS(r(s)) p^v(s) \leq E$. Now using that we have shown above $2 p^u(s) \leq p^v(s)$ we find $E \leq \frac{3}{2} \OsqS(r(s)) p^v(s)$ from which we find
\begin{equation}
\frac{2}{3} E \leq \OsqS(r(s)) p^v(s) \leq E.
\end{equation}
Since we assumed $R > 3m$ we have that $\frac{1}{3} \leq \OsqS \leq 1$ in the asymptotically flat region. Therefore
\begin{equation}
\frac{2}{3} E \leq p^v(s) \leq 3 E.
\end{equation}
In particular it follows that $p^v$ remains approximately constant along a geodesic segment contained entirely in the region $\mathcal{A}_0$. \\

Next we show that $r^2 p^u \lesssim p^v$ in the region $\mathcal{A}_0$. To prove this, let us again consider energy conservation in $(t^*,r)$-coordinates
\begin{equation} \label{energyconservation2}
E^2 = (p^r)^2 + \frac{\OsqS}{r^2} L^2 \iff \left( \frac{p^r}{E} \right)^2 = 1 - \frac{\OsqS}{r^2} \left( \frac{L}{E} \right)^2.
\end{equation}
Let us denote the smallest parameter time $s \in [s_0,s_1]$ for which $r(s) \geq R$ by $s'$. Then by our assumption on $\gamma$ it follows that $R \leq r(s') \leq \rsuppconst$ and therefore
\begin{equation}
    \left( \frac{L}{E} \right)^2 \leq \frac{r(s')^2}{\OsqS(r(s'))} \leq 3 \rsuppconst^2.
\end{equation}
Therefore we conclude that
\begin{equation}
0 \leq 1 - \left| \frac{p^r(s)}{E} \right| \leq 1 - \left( \frac{p^r(s)}{E} \right)^2 \leq \frac{3 \rsuppconst^2}{r(s)^2}.
\end{equation}
Using equation~\eqref{p3p4} we find
\begin{equation} \label{p3p42}
\frac{p^u(s)}{p^v(s)} = \frac{1 - \frac{p^r(s)}{E}}{1 + \frac{p^r(s)}{E}} \leq \min \left( \frac{1}{2},\frac{3 \rsuppconst^2}{r(s)^2} \right).
\end{equation}
Finally let us show the bound for the $\tau$-time. Let $[s_3,s_4] \subset [s^*,s_1]$ and notice that by definition $\tau(\gamma(s_4)) - \tau(\gamma(s_3)) = u(s_4) - u(s_3)$, so that
\begin{equation}
u(s_4) - u(s_3) = \int_{s_3}^{s_4} p^u(s) \, ds = \int_{r(s_3)}^{r(s_4)} \frac{p^u}{p^r } \, dr \sim \int_{r(s_3)}^{r(s_4)} \frac{p^u}{p^v} \, dr \lesssim \rsuppconst^2 \int_{r(s_3)}^{r(s_4)} \frac{1}{r^2} \, dr \lesssim \frac{\rsuppconst^2}{R},
\end{equation}
where we have used that in the region $\mathcal{A}_0$ the relation $p^r \sim E \sim p^v$ holds. Finally noting that we chose $R > 3m$ the claimed bound readily follows.
\end{proof}

\subsubsection{Proof of Lemma~\ref{taubound}}

We now combine the estimates established in the previous sections in order to prove Lemma~\ref{taubound}. The proof will also make reference to certain basic facts about the geodesic flow on the Schwarzschild exterior, see~\cite[Chapter 6.3]{waldbook} for a comprehensive discussion.

\begin{proof}[Proof of Lemma~\ref{taubound}]
To simplify notation we will adopt the convention that all of the bounds given in this proof will depend on $\delta$, although it will be obvious which bounds exactly will depend on $\delta$. Like in the proof of Lemma~\ref{tstar_lem_schw} we will consider the regions close to and far from the event horizon separately.

\paragraph{Region close to horizon.}
Let us first treat the case that $2m \leq r(s) \leq (2+\delta)m$. Then if $p^r(s) > 0$ it follows readily from the form of the geodesic equations that $p^r(s') > 0$ and $r_0 \leq r(s') \leq (2+\delta)m$ for all $s' \in [s_0,s]$. Therefore by definition
\begin{equation}
    \tau(\gamma(s)) - \tauzero \leq t^*(s) = t^*(s) - t^*(s_0),
\end{equation}
which we may directly bound using Lemma~\ref{tstar_lem_schw} if we set $\delta_1 = \delta$ there to find
\begin{equation}
\frac{1}{m} \left( \tau(\gamma(s)) - \tauzero \right) \lesssim 1 + \left| \log \left( \frac{\OsqS(r_0)}{\OsqS(r(s))} \right) \right|.
\end{equation}
In the other case that $p^r(s) \leq 0$ we need to distinguish by the sign of $\trapschw$. Let us recall that if $\trapschw \geq 0$, the sign of $p^r$ will not change along $\gamma$, whereas in the case that $\trapschw < 0$, $p^r$ must in fact change sign along $\gamma$, see~\cite[Chapter 6.3]{waldbook}. Thus if $\trapschw \geq 0$, we must have $p^r(s') \leq 0$ for all $s' \in [s_0,s]$ and $r_0 \geq r(s)$. If we choose $\tauzero$ as in Lemma~\ref{pr_like_E}, the lemma implies that for $s' \in [s_0,s_1]$ with $\gamma(s') \in \mathcal{A}_0$ necessarily $p^r(s') > 0$. In other words, for times $\tau \geq \tauzero$ geodesics may cross from the region $\{ 2m \leq r \leq R \}$ into the asymptotically flat region but not the other way around. Therefore for all $s' \leq s$ such that $\tau(\gamma(s')) \geq \tauzero$ we have $2m \leq r(s') \leq R$ for $s' \in [\tilde{s},s]$. Therefore we may again apply Lemma~\ref{tstar_lem_schw} directly and find
\begin{equation}
\frac{1}{m} \left( \tau(\gamma(s)) - \tauzero \right) \lesssim 1 + \left( \log \left| \trapschw \right| \right)_-.
\end{equation}
If $\trapschw < 0$ then it must be the case that $2m \leq r(s') \leq 3m$ for all $s' \in [s_0,s]$ and $p^r(s_0) > 0$. Thus there must exist $\tilde{s} \in (s_0,s)$ with $p^r(\tilde{s}) = 0$, separating the evolution into an initial period where $p^r > 0$ and the geodesic travels outward from $r_0$ to the radius of closest approach to the photon sphere $r_{\text{min}}^-(\trapschw)$, followed by a period where $p^r \leq 0$ during which the geodesic travels inward from $r_{\text{min}}^-(\trapschw)$ to $r$. Noting that the geodesic remains inside the photon sphere, we need to apply Lemma~\ref{tstar_lem_schw} to both those periods and add the resulting bounds. For the period $[\tilde{s},s]$ where $p^r \leq 0$ Lemma~\ref{tstar_lem_schw} immediately implies that
\begin{equation}
\frac{1}{m} \left( \tau(\gamma(s)) - \tau(\gamma(\tilde{s})) \right) \lesssim 1 + \left( \log \left| \trapschw \right| \right)_-.
\end{equation}
For the initial period $[s_0,\tilde{s})$ where $p^r > 0$ let us distinguish between the two cases that $r_0 \leq (2+\delta)m$ and $r_0 > (2+\delta)m$. In the latter case, we may apply Lemma~\ref{tstar_lem_schw} with $\delta_1=\delta$ to find
\begin{equation}
\frac{1}{m} \left( \tau(\gamma(\tilde{s})) - \tau(\gamma(s_0)) \right) = \frac{\tau(\gamma(\tilde{s}))}{m} \lesssim 1 + \left( \log \left| \trapschw \right| \right)_-.
\end{equation}
Assume therefore that $r_0 \leq (2+\delta)m$. In this case let us further distinguish the two cases that $r_{\text{min}}^-(\trapschw) > (2+\delta)m$ and $r_{\text{min}}^-(\trapschw) \leq (2+\delta)m$. In the first case, we apply Lemma~\ref{tstar_lem_schw} with $\delta_1=\delta$ to find
\begin{align}
\frac{\tau(\gamma(\tilde{s}))}{m} &\lesssim 1 + \left| \log \left( \frac{\OsqS(r_0)}{\OsqS((2+\delta)m)} \right) \right| + \left( \log \left| \trapschw \right| \right)_- \\
&\lesssim 1 + \left| \log \left( \frac{\OsqS(r_0)}{\OsqS(r_{\text{min}}^-(\trapschw))} \right) \right| + \left( \log \left| \trapschw \right| \right)_-,
\end{align}
where we used that $\OsqS(r_{\text{min}}^-(\trapschw)) > \OsqS((2+\delta)m)$ by assumption. If $r_{\text{min}}^-(\trapschw) \leq (2+\delta)m$, Lemma~\ref{tstar_lem_schw} with $\delta_1=\delta$ gives 
\begin{equation}
\frac{\tau(\gamma(\tilde{s}))}{m} \lesssim 1 + \left| \log \left( \frac{\OsqS(r_0)}{\OsqS(r_{\text{min}}^-(\trapschw))} \right) \right|.
\end{equation}
Therefore in total we find, whenever $r_0 \leq (2+\delta)m$ we have the bound
\begin{equation}
\frac{1}{m} \left( \tau(\gamma(s)) - \tauzero \right) \lesssim 1 + \left| \log \left( \frac{\OsqS(r_0)}{\OsqS(r_{\text{min}}^-(\trapschw))} \right) \right| + \left( \log \left| \trapschw \right| \right)_-.
\end{equation}
Recall that energy conservation~\eqref{Schw::energy_conservation} implies that the bound $\frac{L^2}{E^2} \leq \frac{r^2}{\OsqS}$ holds for any point on the mass-shell. The definition of $\trapschw$ in equation~\eqref{def_eps} directly implies  $\frac{1}{27m^2} \frac{L^2}{E^2} = 1 - \trapschw$. Therefore for any $2m \leq r \leq 3m$ and $\trapschw < 0$ we find
\begin{equation}
\frac{1}{\OsqS} \geq \frac{1}{r^2} \frac{L^2}{E^2} \geq \frac{1}{9m^2} \frac{L^2}{E^2} \geq 3 \left( 1 + \left| \trapschw \right| \right) \geq 1 + \left| \trapschw \right|.
\end{equation}
Therefore certainly we have the bound
\begin{equation}
\frac{1}{\OsqS(r_{\text{min}}^-(\trapschw))} \geq 1 + \left| \trapschw \right|,
\end{equation}
where we note that in fact $\OsqS(r_{\text{min}}^-(\trapschw)) \rightarrow \frac{1}{3}$ as $\trapschw \rightarrow 0$ and $\OsqS(r_{\text{min}}^-(\trapschw)) \rightarrow 0$ as $\trapschw \rightarrow - \infty$ so that this bound is in fact sharp. Note also that from~\eqref{epsbound} it readily follows that for $2m \leq r \leq 3m$ we have $(1 + \left| \trapschw \right|) \OsqS(r) \leq \frac{2}{3}$. We find
\begin{equation}
\frac{1}{m} \left( \tau(\gamma(s)) - \tauzero \right) \lesssim 1 + \left| \log \left( (1 + \left| \trapschw \right|) \OsqS(r_0) \right) \right| + \left( \log \left| \trapschw \right| \right)_-.
\end{equation}
Note finally that we may multiply the second summand with $\mathfrak{s} = \chi_{(0,\infty)}\left( p^r(0) \right)$ since we concluded $p^r(0) > 0$ for this case.

\paragraph{Region far from the horizon.}
Let us now treat the case that $(2+\delta)m \leq r(s) < \infty$. In fact, when $(2+\delta)m \leq r(s) \leq 3m$ the structure of the argument is almost identical to the case where $2m 
\leq r(s) \leq (2+\delta)m$. The case that $p^r(s) \leq 0$ is indeed treated identically to above. If $p^r(s) > 0$ the argument needs to be modified slightly. It follows as above that $p^r(s') > 0$ for all $s' \in [s_0,s]$. Therefore the geodesic is steadily outgoing from its initial radius $r_0$ to $r(s)$. Let us distinguish the two cases that $r_0 > (2+\delta)m$ and $r_0 \leq (2+\delta)m$. In the first case Lemma~\ref{tstar_lem_schw} implies
\begin{equation}
\frac{1}{m} \left( \tau(\gamma(s)) - \tauzero \right) \lesssim 1 + \left( \log \left| \trapschw \right| \right)_-.
\end{equation}
In the second case the radius of the geodesic first increases from $r_0$ to $(2+\delta)m$ and then from $(2+\delta)m$ to $r(s)$, where we now note that $r(s)$ may be close to $3m$. Therefore Lemma~\ref{tstar_lem_schw} implies 
\begin{align}
\frac{1}{m} \left( \tau(\gamma(s)) - \tauzero \right) &\lesssim 1 + \left| \log \left( \frac{\OsqS(r_0)}{\OsqS((2+\delta)m)} \right) \right| + \left( \log \left| \trapschw \right| \right)_- \\
&\leq 1 + \left| \log \left( \frac{\OsqS(r_0)}{\OsqS(r(s))} \right) \right| + \left( \log \left| \trapschw \right| \right)_-,
\end{align}
where we used that $\OsqS(r(s)) \geq \OsqS((2+\delta)m)$. Let us now see how we can simplify the second summand further. We begin by noting that since $r(s) \geq (2+\delta)m$ certainly $\frac{\delta}{3} \leq \OsqS(r(s)) \leq 1$. Inequality~\eqref{epsbound} allows us to conclude that $-\frac{4}{\delta} \leq \trapschw \leq 1$. Therefore
\begin{equation}
\frac{\OsqS(r_0)}{\OsqS(r(s))} \sim_\delta \OsqS(r_0) \sim_\delta \left( 1 + \left| \trapschw \right| \right) \OsqS(r_0).
\end{equation}
As above we note that~\eqref{epsbound} readily implies that for $2m \leq r \leq 3m$ we have $(1 + \left| \trapschw \right|) \OsqS(r) \leq \frac{2}{3}$. We have therefore simplified our estimate to
\begin{equation}
\frac{1}{m} \left( \tau(\gamma(s)) - \tauzero \right) \lesssim 1 + \left| \log \left( (1 + \left| \trapschw \right|) \OsqS(r_0) \right) \right| + \left( \log \left| \trapschw \right| \right)_-.
\end{equation}
As above we note that since we assume $p^r > 0$ in this case we may multiply the second summand with the factor $\mathfrak{s}$. \\

Let us therefore assume that $3m < r(s) < \infty$. Depending on the values of $\trapschw, r_0$ and $p^r(s_0)$, the geodesic may cross any region of the exterior before arriving at the radius $r(s)$. Let us distinguish the two cases that $2m \leq r_0 \leq 3m$ and $r_0 > 3m$. In the first case, it must be the case that $p^r(s_0) > 0$ and $\trapschw > 0$, since the geodesic must cross the photon sphere eventually. It follows that $p^r(s') > 0$ for all $s' \in [s_0,s_1]$. Let us denote by $s_0 \leq \tilde{s} < s$ the parameter time such that $r(\tilde{s}) = 3m$. We have shown above that
\begin{equation}
\frac{1}{m} \left( \tau(\gamma(\tilde{s})) - \tau(\gamma(s_0)) \right) = \frac{\tau(\gamma(\tilde{s}))}{m} \lesssim 1 + \left| \log \left( (1 + \left| \trapschw \right|) \OsqS(r_0) \right) \right| + \left( \log \left| \trapschw \right| \right)_-.
\end{equation}
It only remains to consider the period of parameter time $[\tilde{s},s]$. We again distinguish between the two further cases that $r(s) \leq R$ and $r(s) > R$. If $r(s) \leq R$ we invoke Lemma~\ref{tstar_lem_schw} with $\delta_3 = \frac{R}{m} -3$ so that $(3+\delta_3)m=R$. We obtain the bound
\begin{equation}
\frac{1}{m} \left( \tau(\gamma(s)) - \tau(\gamma(\tilde{s})) \right) \lesssim 1 + \left( \log \left| \trapschw \right| \right)_-.
\end{equation}
In case that $r(s) > R$ the geodesic will require additional time to cross the region between $R$ and $r(s)$, which may be bounded as in Lemma~\ref{tauestimate_far}. We obtain
\begin{equation}
\frac{1}{m} \left( \tau(\gamma(s)) - \tau(\gamma(\tilde{s})) \right) \lesssim 1 + \left( \log \left| \trapschw \right| \right)_- + \frac{\rsuppconst^2}{m^2}.
\end{equation}
Adding up the bounds for the two periods we considered, we find
\begin{equation}
\frac{\tau(\gamma(s))}{m} \lesssim 1 + \frac{\rsuppconst^2}{m^2} + \left| \log \left( (1 + \left| \trapschw \right|) \OsqS(r_0) \right) \right| + \left( \log \left| \trapschw \right| \right)_-.
\end{equation}
Finally we note that if we assume $\tauzero \gtrsim \frac{1}{m} \rsuppconst^2$ with a sufficiently large constant, we may absorb the term $\frac{1}{m^2} \rsuppconst^2$ to the left hand side and obtain
\begin{equation}
\frac{1}{m} \left( \tau(\gamma(s)) - \tauzero \right) \lesssim 1 + \left| \log \left( (1 + \left| \trapschw \right|) \OsqS(r_0) \right) \right| + \left( \log \left| \trapschw \right| \right)_-.
\end{equation}
Let us now consider the remaining case that $r_0 > 3m$. We can now distinguish the two cases that $p^r(s_0) \geq 0$ or $p^r(s_0) < 0$. In the first case it follows that $p^r(s') \geq 0$ for all $s' \in [s_0, s_1]$ and therefore we may easily conclude
\begin{equation}
\frac{1}{m} \left( \tau(\gamma(s)) - \tauzero \right) \lesssim 1 + \left( \log \left| \trapschw \right| \right)_-,
\end{equation}
from Lemma~\ref{tstar_lem_schw} with $\delta_3$ as above, Lemma~\ref{tauestimate_far} and where we assume $\tauzero$ as above to absorb the term proportional to $\frac{1}{m^2} \rsuppconst^2$. In the case that $p^r(s_0) < 0$ let us distinguish between the two cases that $\trapschw \geq 0$ and $\trapschw < 0$. If $\trapschw \geq 0$, we must have $p^r(s') < 0$ for all $s' \in [s_0,s]$ and therefore $r(s') \leq r_0$. If $R \leq r(s) \leq r_0$ we note that we have chosen $\tauzero$ large enough so that
\begin{equation}
\frac{1}{m} \left( \tau(\gamma(s)) - \tauzero \right) \lesssim 1.
\end{equation}
If $3m < r(s) \leq R$, the geodesic requires time at most $\tauzero$ to reach the radius $R$, and then we may apply Lemma~\ref{tstar_lem_schw} with $\delta_3$ as above to find
\begin{equation}
\frac{1}{m} \left( \tau(\gamma(s)) - \tauzero \right) \lesssim 1 + \left( \log \left| \trapschw \right| \right)_-.
\end{equation}
If $\trapschw < 0$, the geodesic will initially be ingoing until it reaches the radius $r^+_{\text{min}}(\trapschw)$ at parameter time $\tilde{s}$, after which time it will scatter off the photon sphere to infinity and cross the region between $r^+_{\text{min}}(\trapschw)$ and $r(s)$. We argue in a similar fashion to above to conclude that
\begin{align}
\frac{1}{m} \left( \tau(\gamma(\tilde{s})) - \tau(\gamma(s_0)) \right) &\lesssim 1 + \tauzero + \left( \log \left| \trapschw \right| \right)_- ,\\
\frac{1}{m} \left( \tau(\gamma(s)) - \tau(\gamma(\tilde{s})) \right) &\lesssim 1 + \left( \log \left| \trapschw \right| \right)_- + \frac{\rsuppconst^2}{m^2}.
\end{align}
Again by assuming $\tauzero$ to be large enough as above and adding the two estimates, we conclude
\begin{equation}
\frac{1}{m} \left( \tau(\gamma(s)) - \tauzero \right) \lesssim 1 + \left( \log \left| \trapschw \right| \right)_-.
\end{equation}
Finally we note that we may absorb any constant terms on the right hand side of the estimates to the left hand side by choosing $\tauzero$ suitably large. This concludes the proof of Lemma~\ref{taubound}.
\end{proof}


\subsection{Estimating the momentum support} \label{section_psupport}
This section is dedicated to the proof of Proposition~\ref{psupport_prop}. We make full use of the estimates established in Sections~\ref{prelim_massshell_estimates} and~\ref{section_tstar}. Before we prove Proposition~\ref{psupport_prop}, we will show two simpler statements about the boundedness of the momentum-support of $f$ for all times.

\subsubsection{Compactness of momentum support} \label{sec_compmomsupport}
In this subsection, we first establish a useful bound concerning the energy and trapping parameter on the mass-shell $\mathcal{P}$ in Lemma~\ref{E_eps_bound_schw}. Then we show that the momentum support of a solution to the massless Vlasov equation with compactly supported initial data remains compact for all times in Lemma~\ref{psupport_bounded}.

\begin{lem} \label{E_eps_bound_schw}
Let $f$ be the solution to the massless Vlasov equation with initial data $f_0 \in L^\infty(\mathcal{P}_0)$ that satisfy Assumption~\ref{assumption_support}. Then there exists a dimensionless constant $C>0$ such that for all $(x,p) = (t^*,r,\omega, p^{t^*},p^r,\pslash) \in \supp(f) \subset \mathcal{P}$ with $\tau(x) \geq 0$ we have $E \left( 1 + \left| \trapschw \right| \right) \leq C \frac{\rsuppconst^2}{m^2} \psuppconst$.
\end{lem}
\begin{proof}
Since both the energy $E$ and the trapping parameter $\trapschw$ are conserved along the geodesic flow, it suffices to consider the quantity $E \left( 1 + \left| \trapschw \right| \right)$ at time $\tau = 0$. Let us therefore assume that $(x,p) = (t^*,r,\omega, p^{t^*},p^r,\pslash) \in \supp(f_0)$. Recall that inequality~\eqref{epsbound} implies the bound
\begin{equation}
	\left( 1 + \left| \trapschw \right| \right) \lesssim \frac{r^2}{m^2} \frac{1}{\OsqS(r)} \leq \frac{\rsuppconst^2}{m^2} \frac{1}{\OsqS(r)}.
\end{equation}
The definition of energy immediately implies
\begin{equation}
	E = \OsqS(r) p^{t^*} - \frac{2m}{r} p^r \leq \left( \OsqS(r) + \frac{2m}{r} \right) \psuppconst = \psuppconst,
\end{equation}
for both signs of $p^r$. Let us now distinguish the two cases that $p^r > 0$ and $p^r \leq 0$. If we first assume that $p^r > 0$, we find from the definition of energy that in fact the better bound
\begin{equation}
	E \leq \OsqS(r) \psuppconst
\end{equation}
holds, so that we immediately conclude
\begin{equation}
	E \left( 1 + \left| \trapschw \right| \right) \lesssim \frac{\rsuppconst^2}{m^2} \psuppconst.
\end{equation}
If we assume that $p^r \leq 0$, we use Lemma~\ref{express_ptstar} in order to conclude the bound
\begin{equation}
	p^{t^*} = \left| p^r \right| + \frac{1}{r^2} \frac{L^2}{E + \left| p^r \right|} \geq \frac{1}{2 r^2} \frac{L^2}{E}.
\end{equation}
This immediately allows us to conclude that
\begin{equation}
	E \left( 1 + \left| \trapschw \right| \right) \lesssim E + \frac{1}{m^2} \frac{L^2}{E} \lesssim E + \frac{r^2}{m^2} p^{t^*} \lesssim \frac{\rsuppconst^2}{m^2} \psuppconst,
\end{equation}
which concludes the proof.
\end{proof}

\begin{lem} \label{psupport_bounded}
Let $f$ be the solution to the massless Vlasov equation with initial data $f_0 \in L^\infty(\mathcal{P}_0)$ that satisfy Assumption~\ref{assumption_support}. Then there exists a dimensionless constant $C>0$ such that for all $(x,p) = (t^*,r,\omega, p^{t^*},p^r,\pslash) \in \supp(f) \subset \mathcal{P}$ with $\tau(x) \geq 0$, we have
\begin{equation}
		p^{t^*} \leq C \psuppconst, \quad \left| p^r \right| \leq \psuppconst, \quad \left| \pslash \right|_{\gslash} \leq \frac{\rsuppconst}{r} \psuppconst.
\end{equation}
\end{lem}
\begin{proof}
Consider a future-directed maximally defined null geodesic $\gamma: [0,s_1) \rightarrow \Mschw$ with affine parameter $s$. Let us express $\gamma(s) = x(s) = (t^*(s),r(s),\omega(s))$ and $\dot{\gamma}(s) = p(s) = (p^{t^*}(s),p^r(s),\pslash(s))$ in $(t^*,r)$-coordinates and assume that $(x(0),p(0)) \in \supp(f_0)$. We begin by noting that Assumption~\ref{assumption_support} on the initial support allows us to conclude that
\begin{equation}
	E = \OsqS(r(0)) p^{t^*}(0) - \frac{2m}{r(0)} p^r(0) \leq \left( \OsqS(r(0)) + \frac{2m}{r(0)} \right) \psuppconst = \psuppconst
\end{equation}
and likewise $L^2 = r(0)^2 \left| \pslash(0) \right|_{\gslash}^2 \leq \rsuppconst^2 \psuppconst^2$. From conservation of energy~\eqref{Schw::energy_conservation} and the definition of $L^2$ in equation~\eqref{def_energy} it immediately follows that for all $s \geq 0$ the following bounds are satisfied
\begin{equation}
	\left| p^r(s) \right| \leq E \leq \psuppconst, \quad \left| \pslash(s) \right|_{\gslash} = \frac{L}{r(s)} \leq \frac{\rsuppconst}{r(s)} \psuppconst.
\end{equation}
In order to bound the $p^{t^*}$-component let us choose $0 < \delta < 1$ and distinguish between the regions $2m \leq r \leq (2+\delta)m$ and $r \geq (2+\delta)m$. For any affine parameter time $s$ such that $r(s) \geq (2+\delta)m$ then Lemma~\ref{express_ptstar} implies
\begin{equation}
	p^{t^*}(s) \lesssim_\delta E \leq \psuppconst.
\end{equation}
Let us therefore consider the case that $2m \leq r(s) \leq (2+ \delta)m$. We distinguish the two cases that $p^r(s) > 0$ and $p^r(s) \leq 0$. If we assume that $p^r(s) > 0$, then necessarily $p^r(s') > 0$ for all $0 \leq s' \leq s$. In this case we use Lemma~\ref{express_ptstar} to represent
\begin{equation}
	p^{t^*}(s) = \frac{E + \frac{2m}{r(s)} p^r(s)}{\OsqS(r(s))}.
\end{equation}
The geodesic equations readily imply that $\partial_s p^r(s') < 0$ for $s' \in [0, s)$ and since $p^r(s') > 0$ we also conclude $\partial_s \OsqS \geq 0$ for $s' \in [0, s)$. Therefore we conclude that $\partial_s p^{t^*}(s') \leq 0$. We deduce that
\begin{equation}
	p^{t^*}(s) \leq p^{t^*}(0) \leq \psuppconst.
\end{equation}
Let us now consider the case that $p^r(s) \leq 0$. In this case we first apply Lemma~\ref{ptstar_bound} to find
\begin{equation}
	p^{t^*}(s) \lesssim \left( 1 + \left| \trapschw \right| \right) E.
\end{equation}
Let us define
\begin{equation}
	\trapschw_\delta = -\frac{1}{\OsqS} \left( 1-\frac{3m}{r} \right)^2 \left( 1+\frac{6m}{r} \right) \frac{r^2}{27m^2} \Bigg|_{r = (2+\delta)m}.
\end{equation}
If we assume that $\left| \trapschw \right| \leq 1 + \left| \trapschw_\delta \right|$ then we immediately conclude that
\begin{equation}
	p^{t^*}(s) \lesssim \left( 1 + \left| \trapschw \right| \right) E \lesssim_\delta \psuppconst.
\end{equation}
Let us therefore consider the remaining case that $p^r(s) \leq 0$ and $\trapschw < \trapschw_\delta < 0$. In this case inequality~\eqref{epsbound} implies that it must be true that $2m \leq r(s') \leq (2+\delta)m$ for $s' \geq 0$. We now distinguish further between the two cases that $p^r(0) > 0$ and $p^r(0) \leq 0$. If we first assume that $p^r(0) \leq 0$, it follows that $p^r(s') \leq 0$ for $s' \in [0,s]$. We may therefore use equality~\eqref{express_ptstar_neg} from Lemma~\ref{express_ptstar} to express $p^{t^*}$ at parameter time $s$ and at parameter time $0$ to find
\begin{align}
	p^{t^*}(0) &= \left|  p^r(0) \right| + \frac{\left| \pslash(0) \right|_{\gslash}^2}{E + \left| p^r(0) \right|} \geq \frac{1}{2 r(0)^2} \frac{L^2}{E}, \\
	p^{t^*}(s) &= \left| p^r(s) \right| + \frac{\left| \pslash(s) \right|_{\gslash}^2}{E + \left| p^r(s) \right|} \leq E + \frac{1}{r(s)^2} \frac{L^2}{E},
\end{align}
so that we may conclude
\begin{equation}
	p^{t^*}(s) \leq E + 2 \frac{r(0)^2}{r(s)^2} p^{t^*}(0) \lesssim \psuppconst.
\end{equation}
Finally let us turn to the remaining case that $p^r(0) > 0$. Since $p^r(s) \leq 0$ we conclude the existence of $s^* \in (0,s]$ such that $p^r(s^*) = 0$, $p^r(s') > 0$ for $s' \in [0,s^*)$ and $p^r(s') \leq 0$ for $s' \in [s^*,s]$. At affine parameter time $s^*$ we find furthermore that $r(s^*) = r_{\text{min}}^-(\trapschw)$ so that $\OsqS(r(s^*))  \left( 1 + \left| \trapschw \right| \right) \sim 1$. We may therefore conclude
\begin{equation}
	p^{t^*}(s) \lesssim E \left( 1 + \left| \trapschw \right| \right) \sim \frac{E}{\OsqS(r(s^*))} = p^{t^*}(s^*).
\end{equation}
Now we note that for $s' \in [0,s^*)$ we may argue as above to conclude that
\begin{equation}
	p^{t^*}(s^*) \leq p^{t^*}(0) \leq \psuppconst,
\end{equation}
which finally allows us to conclude
\begin{equation}
	p^{t^*}(s) \lesssim \psuppconst,
\end{equation}
as claimed.
\end{proof}

\subsubsection{Proof of Proposition~\ref{psupport_prop}}

We can now finally provide the proof of Proposition~\ref{psupport_prop}. We remind the reader that the definitions of the sets $\trappedsupportset, \smallsupportset \subset \mathcal{P}$ involve a choice of constants $\bigc, \decayrate$ and $\tauzero$, see Section~\ref{sec_subsets}.

%\begin{prop}
%Assume that $f_0 \in L^\infty(\mathcal{P}_0)$ satisfies Assumption~\ref{assumption_support} and let $f$ be the unique solution to the massless Vlasov equation with $f|_{\Sigma_0} = f_0$. Then there exist dimensionless constants $\bigc > 0, \decayrate > 0, C_0 > 0$ so that if $\tauzero = C_0 \frac{\rsuppconst^2}{m}$, then all $(x,p) = (t^*,r,\omega,p^{t^*},p^r,\pslash) \in \supp(f)$ such that $\tau(x) \geq \tauzero$ satisfy
%\begin{equation} \label{eqn_prop41_refined}
%	(x,p) \in \trappedsupportset \cup \smallsupportset.
%\end{equation}
%Furthermore we can refine this estimate as follows: Let $0 < \delta < 1$. Then we can choose $\decayrate = \decayrate(\delta) > 0$ such that inclusion~\eqref{eqn_prop41_refined} still holds for all $(x,p) \in \supp(f)$ and in addition if $2m \leq r \leq (2+\delta)m$ and $p^r > 0$, then we in fact know that $(x,p) \in \smallsupportset$.
%\end{prop}

\begin{proof}[Proof of Proposition~\ref{psupport_prop}]
Consider an affinely parametrised  future-directed null geodesic segment $\gamma: [0,s] \rightarrow \Mschw$ expressed in $(t^*,r)$-coordinates as before. Assume that $(x(0),p(0)) \in \supp(f_0)$. For the sake of simplicity, we suppress the dependence on $s$. Let us write $(x(s),p(s)) = (x,p) = (t^*,r,\omega,p^{t^*},p^r,\pslash)$ and $\tau(x(s)) = \tau$. Therefore the geodesic $\gamma$ meets the hypersurface $\Sigma_\tau$ at the point $x \in \Mschw$ and therefore populates the $p$-support of $f$ at this point. We will repeatedly apply Lemma~\ref{taubound} in the course of the proof. For simplicity of notation we will omit the dependence of the bounds on $\delta$ in this proof. We will distinguish between the two cases that $p^r > 0$ and $p^r \leq 0$ at time $\tau$. Let us consider first

\paragraph{Case 1: $p^r \leq 0$ at time $\tau$.}
In this case we may apply Lemma~\ref{taubound} to conclude the bound
\begin{equation} \label{tauestimate_in_proof}
\frac{1}{m} \left( \tau - \tauzero \right) \lesssim \left( \log \left| \trapschw \right| \right)_- + \mathfrak{s} \left( \log \, (1+\left| \trapschw \right|) \OsqS(r(0)) \right)_-,
\end{equation}
which holds regardless of the value of $2m \leq r < \infty$ and where we define $\mathfrak{s} = \chi_{(0,\infty)}\left( p^r(0) \right)$ as in the lemma to ensure that the second term only appears when $p^r(0) > 0$ and $2m < r(0) < 3m$. Therefore at least one of the two terms must be larger than a multiple of $\frac{1}{m} \left( \tau - \tauzero \right)$ and we will further distinguish two cases.

\paragraph{Case 1.1: $\gamma$ is almost trapped.}
In this case we may assume that the first term is large so that $\left| \trapschw \right| < 1$ and
\begin{equation}
\frac{1}{m} \left( \tau - \tauzero \right) \lesssim \left| \log \left| \trapschw \right| \right| \iff \left| \trapschw \right| \leq e^{- \frac{\decayrate}{m} (\tau-\tauzero)},
\end{equation}
for an appropriate constant $\decayrate$ arising from the constant implicit in inequality~\eqref{tauestimate_in_proof}. From the definition~\eqref{def_eps} of $\trapschw$ it immediately follows that
\begin{align}
\left| 1 - \frac{1}{\sqrt{27}m} \frac{L}{E} \right| &\leq \left| \trapschw \right| \leq e^{- \frac{\decayrate}{m} (\tau-\tauzero)}, \label{ELquotient_bound} \\
\left| E - \frac{L}{\sqrt{27}m} \right| &= E \left| 1 - \frac{1}{\sqrt{27}m} \frac{L}{E} \right| \leq \psuppconst  \left| \trapschw \right| \leq  \psuppconst e^{- \frac{\decayrate}{m} (\tau-\tauzero)} \label{ELquotient_bound2}, \\
\left| E^2 - \frac{L^2}{27 m^2} \right| &= \left| \trapschw \right| E^2 \leq \psuppconst^2 \left| \trapschw \right| \leq \psuppconst^2 e^{- \frac{\decayrate}{m} (\tau-\tauzero)} \label{ELquotient_bound3},
\end{align}
where we used the bound $E \leq \psuppconst$ which we claim follows from our assumptions on the initial support of $f$ as given in Assumption~\ref{assumption_support}. To see this, recall the definition of $E$ in $(t^*,r)$-coordinates as $E = \OsqS p^{t^*} - \frac{2m}{r} p^r$. We readily conclude that $E \leq \psuppconst$ regardless of the sign of $p^r$. Similarly it follows that $L \leq 2 \rsuppconst^2 \psuppconst$. Since we have in fact assumed $\tau \geq m + \tauzero$ so that $e^{- \frac{\decayrate}{m} (\tau-\tauzero)} \leq e^{- \decayrate} < 1$ and we may conclude from~\eqref{ELquotient_bound} that $\frac{E}{L} \lesssim \frac{1}{m}$. \\

Let us now compute the values of $p^{t^*}$ and $p^r$ in terms of $E$ and $L$. From energy conservation~\eqref{Schw::energy_conservation} we immediately find that
\begin{equation}
(p^r)^2 = E^2 \trapschw + \left( \frac{1}{27m^2} - \frac{\OsqS}{r^2} \right) L^2.
\end{equation}
Dividing this relation by $\frac{L^2}{m^2}$ and noting that $\frac{E}{L} \lesssim \frac{1}{m}$ it follows that
\begin{equation}
\left| m^2 \frac{\left| p^r \right|^2}{L^2} - \left(\frac{1}{27} - \frac{m^2}{r^2} \OsqS \right) \right| \lesssim m^2 \frac{E^2}{L^2} e^{-\frac{\decayrate}{m}(\tau-\tauzero)} \lesssim e^{-\frac{\decayrate}{m}(\tau-\tauzero)}.
\end{equation}
Using the fact that for any $a,b \in \R_{>0}$ such that $\left| a^2-b^2 \right| \leq \eta$ then $\left| a-b \right| \leq \sqrt{\eta}$ we find
\begin{equation}
\left| m \frac{\left| p^r \right|}{L} - \sqrt{\frac{1}{27} - \frac{m^2}{r^2} \OsqS} \right| \lesssim m \frac{E}{L} e^{-\frac{\decayrate}{2m}(\tau-\tauzero)} \lesssim e^{-\frac{\decayrate}{2m}(\tau-\tauzero)}.
\end{equation}
It is easy to see that (given a decay rate for $\trapschw$) this is the optimal decay rate at the photon sphere since there the relation $(p^r)^2 = \trapschw E^2$ or equivalently $\left| p^r \right| = \sqrt{\left| \trapschw \right|} E$ holds. Therefore at the photon sphere, $p^r$ must decay at half the rate at which $\trapschw$ decays. However in any region where $E \sim \left| p^r \right|$ holds we may conclude the better bound
\begin{equation}
\left| m \frac{\left| p^r \right|}{L} - \sqrt{\frac{1}{27} - \frac{m^2}{r^2} \OsqS} \right| \lesssim e^{-\frac{\decayrate}{m}(\tau-\tauzero)},
\end{equation}
by making use of the fact that for any $a,b \in \R_{>0}$ such that $\left| a^2 - b^2 \right| \leq a^2 \eta$ we have $\left| a - b \right| \leq a \eta$. To this end let us note that energy conservation may be reformulated as 
\begin{equation} \label{Elikepr}
\left[ \left( 1 - \frac{3m}{r} \right)^2 \left( 1 + \frac{6m}{r} \right) + 27 m^2 \frac{\OsqS} {r^2} \trapschw \right] E^2 = (p^r)^2.
\end{equation}
Therefore we immediately see that when $r=2m$ and in the limit as $r \rightarrow \infty$ we have $E \sim \left| p^r \right|$ and therefore the better decay rate. In fact, let $0 < \delta' < 1$ be fixed and consider the region where $r \notin [(3-\delta')m,(3+\delta')m]$. Then clearly for $\tau \gtrsim \tauzero - \ln \delta'$ we have $E^2 \sim (p^r)^2$ from~\eqref{Elikepr} and therefore again the better decay rate. Recall also that we have shown above that $\frac{1}{2}E \leq p^r \leq E$ in the region $\mathcal{A}_0$. \\

To obtain a result on $p^{t^*}$ component, recall that we have expressed $p^{t^*}$ using the mass-shell relation in Lemma~\ref{express_ptstar}. Since we assume $p^r \leq 0$ we have
\begin{equation}
p^{t^*} = \frac{-\frac{2m}{r} \frac{\left| p^r \right|}{L} + \sqrt{\frac{\left| p^r \right|^2}{L^2} + \frac{\OsqS}{r^2}}}{\OsqS} L.
\end{equation}
We note that by an application of L'Hôpital's rule the quotient in the expression has a regular limit as $r \rightarrow 2m$ for any values of $\left| p^r \right|$ and $L$. Therefore this expression is regular for all $2m \leq r < \infty$. However, in order to show the required bounds, we find it convenient to consider the region close to and far from the event horizon separately. \\

Let $0 < \delta < 1$ as in the statement of the proposition and assume first that $r \geq (2+ \delta)m$. Here $\OsqS \sim_\delta 1$ and using the fact that $\left| \sqrt{a^2+x} - \sqrt{b^2+x} \right| \leq \left| a-b \right|$ for all $x \geq 0$ we immediately deduce the bound
\begin{equation}
\left| m \frac{ p^{t^*}}{L} - \left( \frac{- \frac{2m}{r} \sqrt{\frac{1}{27} - \frac{m^2}{r^2} \OsqS} + \sqrt{\frac{1}{27}}}{\OsqS} \right) \right| \lesssim \left| m \frac{\left| p^r \right|}{L} - \sqrt{\frac{1}{27} - \frac{m^2}{r^2} \OsqS} \right|.
\end{equation}
We conclude that in the region $r \geq (2+ \delta)m$, $p^{t^*}$ decays to its limit at the same rate as $p^r$, so that in particular the same remarks on the rate of decay apply as above. Let us now assume that $2m \leq r \leq (2+ \delta)m$.

\begin{claim} \label{lemma::F}
Let $x_0 \geq \frac{1}{\sqrt{2}}$ and assume that $\Psi: [x_0,\infty) \rightarrow \R$ is a smooth function satisfying the conditions $\Psi(x_0) = 0$, $\Psi(x) > 0$ for all $x > x_0$ and $\Psi'(x) > 0$ for all $x \geq x_0$. Let $0 < c < 1$ and consider the function $F: [c,\infty) \times [x_0,\infty) \rightarrow \R$ defined by
\begin{equation}
	F(a,x) = \frac{-\left( 1- \Psi \right) a + \sqrt{a^2 + \frac{1}{x^2} \Psi}}{\Psi}.
\end{equation}
Then $F$ is continuous and Lipschitz in the first parameter, more precisely for $a \geq b \geq c$ we have
\begin{equation}
	- \frac{1}{c^2} (a-b) \leq F(a,r) - F(b,r) \leq a-b.
\end{equation}
\end{claim}
\begin{subproof}
We begin by noting it suffices to show continuity of $F$ when $x = x_0$, since $\Psi(x) > 0$ for $x > x_0$. An application of L'Hôpital's rule together with the assumption $\Psi'(x_0) > 0$ yields that $F$ is continuous up to and including $x=x_0$. Let $a \geq b \geq c > 0$, then the inequality $F(a,r) - F(b,r) \leq a-b$ is equivalent to
\begin{equation}
    \sqrt{a^2 + \frac{1}{x^2} \Psi} - \sqrt{b^2 + \frac{1}{x^2} \Psi} \leq a-b.
\end{equation}
This follows immediately from the inequality $\sqrt{a^2 + y} - \sqrt{b^2 + y} \leq (a-b)$ which holds for all $a \geq b \geq 0$ and $y \geq 0$. The lower bound $F(a,r) - F(b,r) \geq - \frac{1}{c^2}(a-b)$ is equivalent to
\begin{equation}
    \sqrt{a^2 + \frac{1}{x^2} \Psi} - \sqrt{b^2 + \frac{1}{x^2} \Psi} \geq \left[ 1 - \Psi \left( 1 + \frac{1}{c^2} \right) \right] (a-b).
\end{equation}
Evaluated at $x=x_0$ both sides agree. To see why the inequality holds for all $x \geq x_0$, note that both sides are decreasing in $x$. It suffices to show that the left side decreases at a slower rate than the right side. Taking derivatives on both sides we see it suffices to show
\begin{equation}
    \left[ \frac{\sqrt{a^2 + \frac{m^2}{r^2} \OsqS} - \sqrt{b^2 + \frac{m^2}{r^2} \OsqS}}{\sqrt{ \left( a^2 + \frac{m^2}{r^2} \OsqS \right) \left( b^2 + \frac{m^2}{r^2} \OsqS \right)} } \right] \frac{1}{2} \frac{d}{dx} \left( \frac{1}{x^2} \Psi \right) \leq \left( 1+ \frac{1}{c^2} \right) (a-b) \frac{d}{dx} \Psi.
\end{equation}
Again using the inequality $\sqrt{a^2 + y} - \sqrt{b^2 + y} \leq (a-b)$ for all $a \geq b \geq 0$ and $y \geq 0$ we see that
\begin{equation}
    \frac{\sqrt{a^2 + \frac{1}{x^2} \Psi} - \sqrt{b^2 + \frac{1}{x^2} \Psi}}{\sqrt{ \left( a^2 + \frac{1}{x^2} \Psi \right) \left( b^2 + \frac{1}{x^2} \Psi \right)} } \leq \frac{a-b}{b^2},
\end{equation}
so that it suffices to show
\begin{equation}
    \frac{1}{2 b^2} \frac{d}{dx} \left( \frac{1}{x^2} \Psi \right) \leq \left( 1+ \frac{1}{c^2} \right) \frac{d}{dx} \Psi.
\end{equation}
Using that $b \geq c$ it is in fact enough to show
\begin{equation}
    \frac{d}{dx} \left( \frac{1}{x^2} \Psi \right) \leq 2 \frac{d}{dx} \Psi.
\end{equation}
By simply expanding the left hand side we find
\begin{equation}
    \frac{d}{dx} \left( \frac{1}{x^2} \Psi \right) = \frac{1}{x^2} \frac{d}{dx} \Psi - \frac{2}{x^3} \Psi \leq \frac{1}{x_0^2} \frac{d}{dx} \Psi \leq 2 \frac{d}{dx} \Psi,
\end{equation}
since we assumed that $x \geq x_0 \geq \frac{1}{\sqrt{2}}$. This concludes the proof of the claim.
\end{subproof} \\

We now apply this claim with the substitution $x = \frac{r}{m}$ and choose $x_0 = 2$ and $\Psi = \OsqS$, noting that $\OsqS$ can indeed readily be expressed as a function of $x$. With these substitutions and choices note that we then have
\begin{equation}
F \left( r , m \frac{\left| p^r \right|}{L} \right) \frac{L}{m} = p^{t^*}.
\end{equation}
Therefore we need only insert the limit value of $m \frac{\left| p^r \right|}{L}$ as $\tau \rightarrow \infty$ to find the limit of $p^{t^*}$. Let us remark that
\begin{equation}
F \left( r , \sqrt{\frac{1}{27} - \frac{m^2}{r^2} \OsqS} \right) = \frac{- \frac{2m}{r} \sqrt{\frac{1}{27} - \frac{m^2}{r^2} \OsqS} + \sqrt{\frac{1}{27}}}{\OsqS} \rightarrow \frac{35}{24 \sqrt{3}} ,
\end{equation}
as $r \rightarrow 2m$. Therefore we find
\begin{equation}
\left| m \frac{p^{t^*}}{L} - \left( \frac{- \frac{2m}{r} \sqrt{\frac{1}{27} - \frac{m^2}{r^2} \OsqS} + \sqrt{\frac{1}{27}}}{\OsqS} \right) \right| \leq \left| F \left( r , m \frac{\left| p^r \right|}{L} \right) - F \left( r , \sqrt{\frac{1}{27} - \frac{m^2}{r^2} \OsqS} \right) \right|.
\end{equation}
Note that since we assume $2m \leq r \leq (2+\delta)m$ we have that $\sqrt{\frac{1}{27} - \frac{m^2}{r^2} \OsqS} \geq \frac{1-\delta}{10} > 0$ and by assuming that $\tauzero$ is sufficiently large we may assume that also $m \frac{\left| p^r \right|}{L} \geq \frac{1-\delta}{20}$ for $\tau \geq \tauzero$ and $2m \leq r \leq (2+\delta)m$. Therefore we may apply Claim~\ref{lemma::F} above to see
\begin{equation}
\left| F \left( r , m \frac{\left| p^r \right|}{L} \right) - F \left( r , \sqrt{\frac{1}{27} - \frac{m^2}{r^2} \OsqS} \right) \right| \lesssim_\delta \left| m \frac{\left| p^r \right|}{L} - \sqrt{\frac{1}{27} - \frac{m^2}{r^2} \OsqS} \right|,
\end{equation}
so that we again see that the $p^{t^*}$ component decays to its limit value at the same rate as the $p^r$ component. In particular, the same remarks on the decay rate apply as above. Finally note that to bound the angular momentum component we need merely note that by Assumption~\ref{assumption_support} we have $L \leq \rsuppconst \psuppconst$ and therefore
\begin{equation}
\left| \pslash \right|_{\gslash} = \frac{L}{r} \leq \rsuppconst \psuppconst \frac{1}{r}.
\end{equation}

\paragraph{Case 1.2: $\gamma$ is initially outgoing and starts close to $\mathcal{H}^+$.}
In this case we may assume that the second summand is large, so that in particular we may assume that $p^r(0) > 0$ and we have the bound
\begin{equation} \label{case2_bound_prneg}
\frac{1}{m} \left(\tau - \tauzero \right) \lesssim \left( \log \, (1+\left| \trapschw \right|) \OsqS(r(0)) \right)_- \implies \left( 1 + \left|\trapschw\right| \right) \OsqS(r(0)) \lesssim e^{- \frac{\decayrate}{m} (\tau-\tauzero)},
\end{equation}
for an appropriate constant $\decayrate$ arising from the constant implicit in inequality~\eqref{tauestimate_in_proof}. Note that we do not obtain a bound on the size of $\left| \trapschw \right|$ in this case. \\

Let us recall the definition of $E$ in $(t^*,r)$-coordinates as $E = \OsqS p^{t^*} - \frac{2m}{r} p^r$. Above we obtained bounds for all signs of $p^r(0)$. Combined with future-directedness which implies $E \geq 0$ we in fact find the better bounds $E \leq \OsqS(r(0)) \psuppconst$ and $\frac{2m}{r(0)} p^r(0) \leq \OsqS(r(0)) \psuppconst$. Using energy conservation and the bound on $E$ we also find $L \leq \rsuppconst \psuppconst \sqrt{\OsqS(0)}$ regardless of the size and sign of $\trapschw$. \\

To bound the $p^r$ component, note that energy conservation immediately implies $\left| p^r \right| \leq E$. Therefore an application of inequality~\eqref{case2_bound_prneg} gives
\begin{equation}
\left( 1+ \left| \trapschw \right| \right) \left| p^r \right| \leq \left( 1+ \left| \trapschw \right| \right) E \leq \psuppconst \left( 1+ \left| \trapschw \right| \right) \OsqS(r(0)) \lesssim \psuppconst e^{-\frac{\decayrate}{m} (\tau-\tauzero)}.
\end{equation}
For the $p^{t^*}$ component we again need to distinguish between the region close to and far from the event horizon. Let $0 < \delta < 1$ be as in the statement of the proposition and let us consider the case $r \geq (2+\delta)m$ first. By Lemma~\ref{ptstar_bound} we know that $p^{t^*} \lesssim_\delta E$ so that by the same argument we find
\begin{equation}
p^{t^*} \leq \left( 1+ \left| \trapschw \right| \right) p^{t^*} \lesssim_\delta  \psuppconst e^{-\frac{\decayrate}{m} (\tau-\tauzero)}.
\end{equation}
In particular the decay rate in $\tau$ is the same as for the $p^r$ component. Let us now assume that $2m \leq r \leq (2+\delta)m$. We again apply Lemma~\ref{ptstar_bound} to deduce the bound
\begin{equation}
p^{t^*} \lesssim (1 + \left| \trapschw \right|) E \leq \psuppconst \left( 1 + \left| \trapschw \right| \right) \OsqS(r(0)) \lesssim \psuppconst e^{-\frac{\decayrate}{m} (\tau-\tauzero)},
\end{equation}
so that we obtain again the same decay rate as for the $p^r$ component. Finally we find for the angular momentum
\begin{equation}
\sqrt{1 + \left| \trapschw \right|} L \leq \rsuppconst \psuppconst \sqrt{\left( 1 + \left| \trapschw \right| \right) \OsqS(r(0))} \lesssim \rsuppconst \psuppconst e^{-\frac{\decayrate}{2m}(\tau - \tauzero)}
\end{equation}
which immediately gives the desired bound for $\pslash$ when we make use of the relation $\left| \pslash \right|_{\gslash} = \frac{L}{r}$. Therefore the angular component of the momentum decays at half the rate of the $p^r$ and $p^{t^*}$ components in this case.


\paragraph{Case 2: $p^r > 0$ at time $\tau$.}
In this case we will find it necessary to consider the region close to the event horizon and far from it separately. Let $0 < \delta < 1$ as in the statement of the proposition and let us assume first that $r \geq (2+\delta)m$. Here the situation is very similar to the case already considered above where $p^r \leq 0$. By Lemma~\ref{taubound} we have
\begin{equation}
\frac{1}{m} \left(\tau - \tauzero \right) \lesssim \left( \log \left| \trapschw \right| \right)_- + \mathfrak{s} \left( \log \, (1+\left| \trapschw \right|) \OsqS(r(0)) \right)_-.
\end{equation}
Again we consider two cases as above, depending on which of the two summands is large.

\paragraph{Case 2.1: $\gamma$ is almost trapped and $r \geq (2+\delta)m$.} In this case we may assume that
\begin{equation}
\frac{1}{m} \left(\tau - \tauzero \right) \lesssim \left| \log \left| \trapschw \right| \right| \iff \left| \trapschw \right| \leq e^{- \frac{\decayrate}{m} (\tau-\tauzero)}.
\end{equation}
The argument in this case will proceed along parallel lines to that of Case~1.1. As above, we may conclude the bounds~\eqref{ELquotient_bound},~\eqref{ELquotient_bound2},~\eqref{ELquotient_bound3}. To obtain bounds for the $p^r$ component, we note that the analogous argument used in Case~1.1 did not depend on $\sgn(p^r)$. Therefore we readily conclude as above the bound
\begin{equation}
\left| m \frac{\left| p^r \right|}{L} - \sqrt{\frac{1}{27} - \frac{m^2}{r^2} \OsqS} \right|  \lesssim e^{-\frac{\decayrate}{2m}(\tau-\tauzero)},
\end{equation}
and note that the same remarks on the rate of decay apply as above. In order to obtain a bound for the $p^{t^*}$ component, let us express $p^{t^*}$ using the mass-shell relation in equation~\eqref{pt_expressed_via_pr}. Since we are in the case that $p^r > 0$ we find
\begin{equation} \label{ptstar_massshell_case21}
p^{t^*} = \frac{\frac{2m}{r} \frac{p^r}{L} + \sqrt{\frac{\left( p^r \right)^2}{L^2} + \frac{\OsqS}{r^2}}}{\OsqS} L.
\end{equation}
As above using the fact that $\left| \sqrt{a^2+x} - \sqrt{b^2+x} \right| \leq \left| a-b \right|$ for all $x \geq 0$ we immediately deduce the bound
\begin{equation}
\left| m \frac{ p^{t^*}}{L} - \left( \frac{ \frac{2m}{r} \sqrt{\frac{1}{27} - \frac{m^2}{r^2} \OsqS} + \sqrt{\frac{1}{27}}}{\OsqS} \right) \right| \lesssim \left| m \frac{\left| p^r \right|}{L} - \sqrt{\frac{1}{27} - \frac{m^2}{r^2} \OsqS} \right|.
\end{equation}
We again conclude that in the region $r \geq (2+ \delta)m$, $p^{t^*}$ decays to its limit at the same rate as the $p^r$ component. Carefully note that in the case where $p^r > 0$, the expression~\eqref{ptstar_massshell_case21} does not have a finite limit as $r \rightarrow 2m$. The bound on the angular component of the momentum is obtained in an identical manner to Case~1.1 above.

\paragraph{Case 2.2: $\gamma$ is initially outgoing, starts close to $\mathcal{H}^+$ and $r \geq (2+\delta)m$.}
In this case we may assume that $p^r(0) > 0$ and we have the bound
\begin{equation}
\frac{1}{m} \left(\tau - \tauzero \right) \lesssim \left( \log \, (1+\left| \trapschw \right|) \OsqS(r(0)) \right)_- \implies \left( 1 + \left|\trapschw\right| \right) \OsqS(0) \lesssim e^{-\frac{\decayrate}{m} (\tau-\tauzero)},
\end{equation}
for an appropriate constant $\decayrate$ again arising from the constant implicit in inequality~\eqref{tauestimate_in_proof}. Identically to Case~1.2 above, from the fact that $p^r(0) > 0$ we obtain the bounds $E \leq \OsqS(r(0)) \psuppconst$ and $L \leq \rsuppconst \psuppconst \sqrt{\OsqS(r(0))}$. The bounds for the $p^r$, $p^{t^*}$ and angular momentum components are shown virtually identically to Case~1.2 above. \\

To bound the $p^r$ component, note that energy conservation implies $\left| p^r \right| \leq E$. Therefore we have
\begin{equation}
	\left( 1+ \left| \trapschw \right| \right) \left| p^r \right| \leq \left( 1+ \left| \trapschw \right| \right) E \leq \psuppconst \left( 1+ \left| \trapschw \right| \right) \OsqS(r(0)) \lesssim \psuppconst e^{-\frac{\decayrate}{m} (\tau-\tauzero)}.
\end{equation}
For the $p^{t^*}$ component we apply Lemma~\ref{ptstar_bound} as above to deduce
\begin{equation}
p^{t^*} \lesssim_\delta E \lesssim \psuppconst e^{-\frac{\decayrate}{m} (\tau-\tauzero)}.
\end{equation}
For the angular momentum $L$ we argue identically to Case~1.2 above. \\

Let us now turn to the remaining case that $2m \leq r \leq (2+\delta)m$. Here there is only one

\paragraph{Case 2.3: $\gamma$ is initially outgoing, starts close to $\mathcal{H}^+$ and $2m \leq r \leq (2+\delta)m$.}
In this case we may assume that $p^r(0) > 0$ and $2m \leq r(0) \leq r \leq (2+\delta)m$ and therefore in fact $p^r > 0$ for all times up to and including $\tau$. We have the bound
\begin{equation}
\frac{1}{m} \left( \tau - \tauzero \right) \lesssim \left| \log \left( \frac{\OsqS(r(0))}{\OsqS(r)} \right) \right| \implies \frac{\OsqS(r(0))}{\OsqS(r)} \lesssim e^{-\frac{\decayrate}{m} (\tau-\tauzero)},
\end{equation}
where again an appropriate constant $\decayrate$ is implied by the constant implicit in inequality~\eqref{tauestimate_in_proof}. As in the preceding section we may conclude the bounds $E \leq \OsqS(r(0)) \psuppconst$ and $L \leq \rsuppconst \psuppconst \sqrt{\OsqS(r(0))}$ from $p^r(0) > 0$ and future-directedness of the geodesic. \\

To bound the $p^r$ component, we again make use of conservation of energy~\eqref{Schw::energy_conservation} together with inequality~\eqref{epsbound} to find
\begin{equation}
\left( 1+ \left| \trapschw \right| \right) p^r \lesssim \frac{p^r}{\OsqS(r)} \leq \frac{E}{\OsqS(r)} \leq \psuppconst \frac{\OsqS(r(0))}{\OsqS(r)} \lesssim \psuppconst e^{-\frac{\decayrate}{m} (\tau-\tauzero)}.
\end{equation}
In order to obtain a bound for the $p^{t^*}$ component we make use of Lemma~\ref{ptstar_bound} again to find
\begin{equation}
p^{t^*} \lesssim \frac{E}{\OsqS(r)} \leq \psuppconst \frac{\OsqS(r(0))}{\OsqS(r)} \lesssim \psuppconst e^{-\frac{\decayrate}{m} (\tau-\tauzero)}.
\end{equation}
To bound the angular momentum we note as before
\begin{equation}
\sqrt{1 + \left| \trapschw \right|} L \lesssim \frac{L}{ \sqrt{\OsqS(r)}} \leq \rsuppconst \psuppconst \sqrt{\frac{\OsqS(0)}{\OsqS(r)}} \lesssim \rsuppconst \psuppconst e^{- \frac{\decayrate}{2m} (\tau-\tauzero)}.
\end{equation}
We obtain the desired bound by finally again making use of the relation $\left| \pslash \right|_{\gslash} = \frac{L}{r}$. This concludes the proof.
\end{proof}

\subsection{Exponential decay of moments} \label{section_integralestimate}
In this section we provide the proof of our main Theorem~\ref{maintheorem_precise}. The proof fundamentally makes use of Proposition~\ref{psupport_prop} to estimate moments of a solution $f$ to the massless Vlasov equation. Before establishing the proof of Theorem~\ref{maintheorem_precise}, we first show the simpler statement that the volume of $\supp(f(x,\cdot)) \subset \mathcal{P}_x$ remains finite for all times.

\subsubsection{Finiteness of volume of the momentum support}

\begin{lem} \label{lem_boundedness_moments_schw}
Assume that $f$ solves the massless Vlasov equation on Schwarzschild and its initial distribution $f_0: \mathcal{P}_0 \rightarrow \R$ satisfies Assumption~\ref{assumption_support}. Then for all $x \in \Mschw$ with $\tau(x) \geq 0$ we have
\begin{equation}
	\int_{\mathcal{P}_x} \chi_{\supp(f(x,\cdot))} \, d \mu_x  \lesssim \frac{\rsuppconst^2}{r^2} \psuppconst^2.
\end{equation}
\end{lem}
\begin{proof}
We remind the reader of the parametrisation of the mass-shell in $(t^*,r)$-coordinates obtained by using the mass-shell relation to eliminate the $p^{t^*}$-coordinate, see Section~\ref{sec_parametrising_massshell}. We begin by introducing a change of variables $\pslash \mapsto (p^1,p^2)$ with the property that $L^2 = r^2 \left| \pslash \right|_{\gslash}^2 = (p^1)^2 + (p^2)^2$. We may explicitly realise these variables by considering spherical coordinates $(\theta,\phi)$  on the sphere $\sphere$ and the associated variables $(p^\theta, p^\phi)$ parametrising the tangent space. Then $(p^1,p^2)$ and $(p^\theta, p^\phi)$ are related via $p^1 = r^2 p^\theta, p^2 = r^2 \sin \theta p^\phi$. Let us now recall equation~\eqref{dmu_first_comp_schw}, which allows us to express the volume form $d \mu_x$ explicitly as
\begin{equation}
	d \mu =  \frac{r^2 \sin \theta}{\OsqS p^{t^*} - \frac{2m}{r} p^r} \, d p^r d p^\theta d p^\phi =  \frac{1}{r^2} \frac{1}{E} \, d p^r d p^1 d p^2 .
\end{equation}
Next we introduce a further change of coordinates $(p^1,p^2) \mapsto (L,\angle (p^1,p^2) ) =: (L,\pslashangle)$, or in other words we use radial coordinates in the angular variables $(p^1,p^2)$. Note that $\pslashangle \in [0,2 \pi)$ and recall that by virtue of Lemma~\ref{psupport_bounded} any $(x,p) \in \supp(f)$ must satisfy the bounds $p^{t^*} \lesssim \psuppconst, \left| p^r \right| \leq \psuppconst$ and $\left| p^i \right| \leq L \leq \rsuppconst \psuppconst$. We therefore find
\begin{equation}
	\int_{\supp(f(x,\cdot))} 1 \, d \mu_x \leq \int_{-\psuppconst}^{\psuppconst} \int_{0}^{\rsuppconst \psuppconst} \int_0^{2 \pi} \frac{1}{r^2} \frac{L}{E} \, d \pslashangle d L d p^r \lesssim \int_{-\psuppconst}^{\psuppconst} \int_{0}^{\rsuppconst \psuppconst} \frac{1}{r^2} \frac{L}{E} \, d L d p^r.
\end{equation}
Recall further from Lemma~\ref{E_eps_bound_schw} that the bound $E (1 + \left| \trapschw \right|) \lesssim \frac{\rsuppconst^2}{m^2} \psuppconst$ must hold for all $(x,p) \in \supp(f)$. Therefore we find that
\begin{equation}
	\frac{1}{r^2} \frac{L}{E} \lesssim \frac{m}{r^2} \sqrt{1 + \left| \trapschw \right|} \lesssim \frac{\rsuppconst}{r^2} \psuppconst^{\frac{1}{2}} \frac{1}{\sqrt{\left| p^r \right|}}.
\end{equation}
We conclude the bound
\begin{equation}
	\int_{\supp(f(x,\cdot))} 1 \, d \mu_x \lesssim \frac{\rsuppconst}{r^2} \psuppconst^{\frac{1}{2}} \int_{-\psuppconst}^{\psuppconst} \int_{0}^{\rsuppconst \psuppconst} \frac{1}{\sqrt{\left| p^r \right|}} \, d L d p^r \lesssim \frac{\rsuppconst^2}{r^2} \psuppconst^2,
\end{equation}
as claimed.
\end{proof}

\subsubsection{Proof of Theorem~\ref{maintheorem_precise}}

\begin{proof}[Proof of Theorem~\ref{maintheorem_precise}]
We again parametrise the mass-shell $\mathcal{P}$ in $(t^*,r)$-coordinates as discussed in Section~\ref{sec_parametrising_massshell}. If we introduce spherical coordinates on the sphere $\sphere$ the mass-shell is then explicitly parametrised by $(t^*,r,\theta,\phi,p^r,p^\theta,p^\phi)$. Recall that we may then express the induced volume form on each fibre $\mathcal{P}_x$ in these coordinates as
\begin{equation}
	d \mu_x =  \frac{r^2 \sin \theta}{\OsqS p^{t^*} - \frac{2m}{r} p^r} \, d p^r d p^\theta d p^\phi =  \frac{r^2 \sin \theta}{E} \, d p^r d p^\theta d p^\phi,
\end{equation}
according to equation~\eqref{dmu_first_comp_schw}. To improve upon the bounds we obtained in the above Lemma~\ref{lem_boundedness_moments_schw} we need to make use of our knowledge of the momentum support of the solution $f$. Recall that Proposition~\ref{psupport_prop} implies
\begin{equation}
\supp(f) \subset \trappedsupportset \cup \smallsupportset
\end{equation}
and let us denote the fibres over a point $x \in M$ by $\trappedsupportsetx = \trappedsupportset \cap \mathcal{P}_x$  and $\smallsupportsetx = \smallsupportset \cap \mathcal{P}_x$. Let us assume in this proof that $w \geq 0$, since otherwise we may replace $w$ by $\left| w \right|$. We will slightly abuse notation and write $\tau = \tau(x)$ for simplicity. Since certainly $\left| f(x,p) \right| \leq \| f_0 \|_{L^\infty}$ we immediately conclude the bound
\begin{equation}
	\int_{\mathcal{P}_x} w f \, d \mu_x \leq \| f_0 \|_{L^\infty} \left( \int_{\trappedsupportsetx} w \, d \mu_x + \int_{\smallsupportsetx} w \, d \mu_x \right) .
\end{equation}
Recall that heuristically, for large times $\tau$ the set $\trappedsupportsetx \subset \mathcal{P}_x$ asymptotically approaches a cone, whereas the set $\smallsupportsetx \subset \mathcal{P}_x$ is contained in an exponentially small neighbourhood of the origin $p=0$. Let us begin as in the proof of Lemma~\ref{lem_boundedness_moments_schw} by introducing the change of coordinates we will use to estimate both integrals. Assume that $r \geq 2m$ and $p \in \mathcal{P}_x$ and consider the transformation
\begin{equation}
	(p^r,p^\theta,p^\phi) \mapsto \left( p^r,\sqrt{(p^\theta)^2 + \sin^2 \theta (p^\phi)^2}, \angle(p^\theta,p^\phi) \right) =: (p^r, l, \pslashangle),
\end{equation}
where we note that $\pslashangle = \angle(p^\theta,p^\phi) \in [0,2 \pi)$. If we assume $p \in \trappedsupportsetx$ then the length $l$ satisfies $l = \frac{1}{r^2} L \leq \frac{\rsuppconst \psuppconst}{r^2}$ and $p^r \in [p^r_-, p^r_+]$ where we have abbreviated
\begin{equation}
	p^r_{\pm} = \frac{L}{m} \left( \sqrt{\frac{1}{27} - \frac{m^2}{r^2} \OsqS} \pm C e^{-\frac{\decayrate}{2m}(\tau - \tauzero)} \right).
\end{equation}
If we assume that $p \in \smallsupportsetx$ the length $l$ satisfies the bound $l = \frac{1}{r^2} L \leq C \frac{\rsuppconst \psuppconst}{r^2} e^{- \frac{\decayrate}{2m} (\tau - \tauzero)}$ and we know $\left( 1 + \left| \trapschw \right| \right) \left| p^r \right| \leq C \psuppconst e^{-\frac{\decayrate}{m} (\tau - \tauzero)}$. Let us first consider the task of estimating the integral
\begin{equation}
	\int_{\trappedsupportsetx} w \, d \mu_x.
\end{equation}
Let us recall our assumption on $w$ and apply Lemma~\ref{psupport_bounded} to conclude that any $p \in \supp(f(x,\cdot))$ satisfies $\left| p \right| \lesssim \frac{\rsuppconst \psuppconst}{m}$, so that
\begin{equation}
	W := \max_{(x,p) \in \supp(f)} \left| w(x,p) \right| < \infty.
\end{equation}
We therefore conclude the estimate
\begin{align}
	\int_{\trappedsupportsetx} w \, d \mu_x &\lesssim W \int_0^{2 \pi} \int_0^{\frac{\rsuppconst \psuppconst}{r^2}} \int_{p^r_{-}}^{p^r_{+}} \frac{r^2 l}{E} \, d p^r d l d \pslashangle \\
	&= W \int_0^{2 \pi} \int_0^{\rsuppconst \psuppconst} \int_{p^r_{-}}^{p^r_{+}} \frac{1}{r^2} \frac{L}{E} \, d p^r d L d \pslashangle,
\end{align}
Now note that for any $(x,p) \in \trappedsupportset$ with $2m \leq r < \infty$, we have $\left| \trapschw \right| \leq C e^{-\frac{\decayrate}{m}(\tau - \tauzero)} \leq C$ so that
\begin{equation}
	\frac{1}{r^2} \frac{L}{E} \lesssim \frac{m}{r^2} \sqrt{1 + \left| \trapschw \right|} \leq \frac{m}{r^2} \sqrt{1+ C}.
\end{equation}
We therefore use the change of coordinates
\begin{equation} \label{prtilde}
	(p^r, L) \mapsto \left( m \frac{p^r}{L} - \sqrt{\frac{1}{27} - \frac{m^2}{r^2} \OsqS}, L \right) = (\tilde{p}^r, \tilde{L})
\end{equation}
to find
\begin{equation}
	\int_0^{\rsuppconst \psuppconst} \int_{p^r_{-}}^{p^r_{+}} 1 \, d p^r d L = \left( \int_0^{\rsuppconst \psuppconst} \frac{\tilde{L}}{m} \, d \tilde{L} \right) \left( \int_{-C e^{-\frac{\decayrate}{2m}(\tau - \tauzero)}}^{C e^{-\frac{\decayrate}{2m}(\tau - \tauzero)}} 1 \, d \tilde{p}^r \right) = \frac{C \rsuppconst^2 \psuppconst^2}{m} e^{-\frac{\decayrate}{2m}(\tau - \tauzero)}.
\end{equation}
In summary we have therefore obtained the bound
\begin{equation}
	\int_{\trappedsupportsetx} w \, d \mu \lesssim W (1+C)^{\frac{3}{2}} \rsuppconst^{2} \psuppconst^{2} \frac{1}{r^2} e^{-\frac{\decayrate}{2m}(\tau - \tauzero)}.
\end{equation}
We note that whenever we can show a better decay rate for the $p^r$ component, this will directly translate into a better decay rate for the moment here.  Let us now turn to the task of estimating the integral
\begin{equation}
	\int_{\smallsupportsetx} w \, d \mu_x.
\end{equation}
As above we simply use the fact that we may bound $w$ in the support of $f$ by a constant in the first instance in order to find the bound
\begin{equation}
	\int_{\smallsupportsetx} w \, d \mu_x \lesssim W \int_0^{2 \pi} \int_{0}^{C \rsuppconst \psuppconst e^{-\frac{\decayrate}{2m}(\tau - \tauzero)}} \int_0^{C \psuppconst e^{-\frac{\decayrate}{m}(\tau - \tauzero)}} \frac{1}{r^2} \frac{L}{E} \, d p^r d L d \pslashangle.
\end{equation}
As mentioned above, for $p \in \smallsupportsetx$ we know that $\left( 1 + \left| \trapschw \right| \right) \left| p^r \right| \leq C \psuppconst e^{-\frac{\decayrate}{m}(\tau - \tauzero)}$ from which we deduce
\begin{equation}
	\frac{1}{r^2} \frac{L}{E} \lesssim \frac{m}{r^2} \sqrt{1 + \left| \trapschw \right|} \leq \sqrt{C \psuppconst} e^{-\frac{\decayrate}{2m}(\tau - \tauzero)} \frac{m}{r^2} \frac{1}{\sqrt{\left| p^r \right|}}.
\end{equation}
Therefore we conclude the bound
\begin{align}
	\int_{\smallsupportsetx} w \, d \mu_x &\lesssim W \sqrt{C \psuppconst} \frac{m}{r^2} e^{-\frac{\decayrate}{2m}(\tau - \tauzero)} \left( \int_{0}^{C \rsuppconst \psuppconst e^{-\frac{\decayrate}{2m}(\tau - \tauzero)}} 1 \, d L \right) \left( \int_0^{C \psuppconst e^{-\frac{\decayrate}{m}(\tau - \tauzero)}} \frac{1}{\sqrt{\left| p^r \right|}} \, d p^r \right) \\
	&\lesssim W C^2 \psuppconst^2 \rsuppconst \frac{m}{r^2} e^{-\frac{3}{2} \frac{\decayrate}{m} (\tau - \tauzero)}.
\end{align}
Note carefully that we have shown that the decay rate for the integral $\int_{\smallsupportsetx} w \, d \mu_x$ is always strictly better than the rate for $\int_{\trappedsupportsetx} w \, d \mu_x$. Making use of the fact that certainly $\rsuppconst \geq 2m$ we therefore obtain the estimate
\begin{equation}
	\int_{\supp(f(x,\cdot))} w \, d \mu \lesssim W (1+C)^2 \frac{\psuppconst^2 \rsuppconst^2}{r^2} e^{-\frac{\decayrate}{2m}(\tau - \tauzero)}.
\end{equation}
The decay rate in $\tau$ here is the decay rate we can show for the $p^r$ component in the set $\trappedsupportset$. Therefore if we let $0 < \delta' < 1$ and assume that $r \notin [(3-\delta')m,(3+\delta')m]$ and $\tau \gtrsim \tauzero - \ln \delta'$ then the rate of decay may be improved to
\begin{equation}
	\int_{\supp(f(x,\cdot))} w \, d \mu \lesssim W (1+C)^2 \frac{\psuppconst^2 \rsuppconst^2}{r^2} e^{-\frac{\decayrate}{m}(\tau - \tauzero)}.
\end{equation}
Let us now turn to the question of which weights $w$ improve the decay rate further. Recall the definition of the weight $w_{\alpha,\beta}$ as well as the constant $c_{\alpha,\beta}$ from the statement of the theorem. Note carefully that for all $r \geq 2m$,
\begin{equation}
	\sqrt{\frac{1}{27} - \frac{m^2}{r^2} \OsqS} \sim \left| 1 - \frac{3m}{r} \right|.
\end{equation}
In particular, when evaluated at $r = 3m$, the weight $w_{\alpha,\beta}$ reduces to $w_{\alpha,\beta} = \left| p^r \right|^{\alpha + \beta}$. We will now show that the moment associated to the weight $w_{\alpha,\beta}$ decays at an improved exponential rate. Proceeding as above, we bound the two integrals $\int_{\trappedsupportsetx} w \, d \mu_x$ and $\int_{\smallsupportsetx} w \, d \mu_x$ separately. Let us begin by noting that for $(x,p) \in \trappedsupportset$,
\begin{equation}
	w_{\alpha,\beta} \leq \psuppconst^\beta \left| \left| p^r \right| - \sqrt{\frac{1}{27} - \frac{m^2}{r^2} \OsqS} \frac{L}{m} \right|^\alpha \leq \left( \frac{\rsuppconst}{m} \right)^\alpha \psuppconst^{\alpha + \beta} \left| \tilde{p}^r \right|^\alpha,
\end{equation}
where we have used the variable $\tilde{p}^r$ defined above in equation~\eqref{prtilde} as well as Lemma~\ref{psupport_bounded}. Proceeding identically to the way we estimated $\int_{\trappedsupportsetx} w \, d \mu_x$ above, we therefore find
\begin{align}
	\int_{\trappedsupportsetx} w_{\alpha,\beta} \, d \mu_x &\lesssim \frac{m}{r^2} \sqrt{1+C} \left( \frac{\rsuppconst}{m} \right)^\alpha \psuppconst^{\alpha + \beta} \left( \int_0^{\rsuppconst \psuppconst} \frac{\tilde{L}}{m} \, d \tilde{L} \right) \left( \int_{-C e^{-\frac{\decayrate}{2m}(\tau - \tauzero)}}^{C e^{-\frac{\decayrate}{2m}(\tau - \tauzero)}} \left| \tilde{p}^r \right|^\alpha \, d \tilde{p}^r \right) \\
	&\lesssim \frac{1}{r^2} C^{\frac{3}{2}+\alpha} \frac{\rsuppconst^{2+\alpha} \psuppconst^{2+\alpha+\beta}}{m^\alpha} e^{- \frac{\decayrate}{2m}(1+\alpha)(\tau-\tauzero)}.
\end{align}
Similarly we see that for any $(x,p) \in \smallsupportset$ the following bound holds
\begin{equation}
	w_{\alpha,\beta} \lesssim \left( \frac{\rsuppconst}{m} \right)^\alpha \psuppconst^\alpha \left| p^r \right|^\beta.
\end{equation}
Again following the same steps of computation as above we find
\begin{align}
	\int_{\smallsupportsetx} w_{\alpha,\beta} \, d \mu_x &\lesssim C^{\frac{3}{2}} \frac{m}{r^2} \frac{\rsuppconst^{1+\alpha} \psuppconst^{\frac{3}{2} + \alpha}}{m^\alpha} e^{-\frac{\decayrate}{m} (\tau - \tauzero)} \left( \int_0^{C \psuppconst e^{-\frac{\decayrate}{m}(\tau - \tauzero)}} \left| p^r \right|^{\beta-\frac{1}{2}} \, d p^r \right) \\
	&\lesssim C^{2 + \beta} \frac{m}{r^2} \frac{\rsuppconst^{1+\alpha} \psuppconst^{2 + \alpha + \beta}}{m^\alpha} e^{-\frac{\decayrate}{m} (\frac{3}{2} + \beta)(\tau - \tauzero)}.
\end{align}
Carefully note that these bounds hold uniformly in $r \geq 2m$. We may summarise the two bounds by concluding
\begin{equation}
	\int_{\supp(f(x,\cdot))} w_{\alpha,\beta} \, d \mu_x \lesssim C^{2 + \alpha + \beta} \frac{1}{r^2} \frac{\rsuppconst^{2+\alpha} \psuppconst^{2 + \alpha + \beta}}{m^\alpha} e^{-\frac{\decayrate_{\alpha,\beta}}{2m} (\tau-\tauzero)}.
\end{equation}
Next let us note that in the special case where we consider the decay of the moment along the photon sphere, i.e. we fix $r= 3m$ as $\tau \rightarrow \infty$ we note that the weight simplifies to
\begin{equation}
	w_{\alpha,\beta} = \left| p^r \right|^{\alpha + \beta} = \left| p^r \right|^{\kappa}.
\end{equation}
Noting that at $r=3m$ we have $\tilde{p}^r \frac{L}{m} = p^r$ and we find the improved bounds
\begin{align}
	\int_{\trappedsupportsetx} \left| p^r \right|^{\kappa} \, d \mu_x \Big|_{r=3m} &\lesssim C^{\frac{3}{2} + \kappa} \frac{1}{r^2} \frac{\rsuppconst^{2+\kappa} \psuppconst^{2+\kappa}}{m^{\kappa}} e^{-\frac{\decayrate}{2m}(1+\kappa)(\tau-\tauzero)}, \\
	\int_{\smallsupportsetx} \left| p^r \right|^{\kappa} \, d \mu_x \Big|_{r=3m} &\lesssim C^{2+\kappa} \frac{1}{r^2} \rsuppconst^2 \psuppconst^{2+\kappa} e^{-\frac{\decayrate}{m}(\frac{3}{2} + \kappa) (\tau-\tauzero)},
\end{align}
by a computation completely analogous to above. Therefore we conclude for $\kappa > 0$ the bound
\begin{equation}
	\int_{\supp(f(x,\cdot))} \left| p^r \right|^\kappa \, d \mu_x \Big|_{r=3m} \lesssim C^{2+\kappa} \frac{1}{r^2} \frac{\rsuppconst^{2+\kappa} \psuppconst^{2+\kappa}}{m^{\kappa}} e^{-\frac{\decayrate}{2m}(1+\kappa)(\tau-\tauzero)}.
\end{equation}
Finally let us turn to weights of the form $w = \left| p^v \right|^\alpha \left| p^u \right|^\beta \left| \pslash \right|_{\gslash}^\gamma$ for $\alpha,\beta,\gamma \geq 0$, where we made use of double null coordinates and the induced coordinates on $\mathcal{P}$ to express a point $(x,p) \in \mathcal{P}$ as $(x,p) = (u,v,\omega,p^u,p^v,\pslash)$. We compute the change of coordinates and find that $p^v = p^{t^*} + p^r$. Therefore Lemma~\ref{tauestimate_far}, Lemma~\ref{psupport_bounded} and our choice of $\tauzero$ together imply the bounds
\begin{equation}
    p^u \leq C \frac{\rsuppconst^2}{r^2} p^v \leq C \frac{\rsuppconst^2}{r^2} \psuppconst, \quad p^v \leq C \psuppconst, \quad \left| \pslash \right|_{\gslash} \leq \frac{\rsuppconst \psuppconst}{r},
\end{equation}
for all $(x,p) \in \supp(f)$ such that $\tau(x) \geq \tauzero$ and $r \geq R$, where $r$ denotes the Schwarzschild radius of the point $x \in \Mschw$ and $C> 0$ is a suitable constant. Inserting this bound in the integral and then bounding the remaining integral as above allows us to conclude the better bound
\begin{equation}
    \int_{\supp(f(x,\cdot))} \left| p^v \right|^\alpha \left| p^u \right|^\beta \left| \pslash \right|_{\gslash}^\gamma \, d \mu_x \lesssim C^{2 + \alpha + \beta + \gamma} \psuppconst^{2 + \alpha + \beta + \gamma}
    \frac{\rsuppconst^{2(1+\beta) + \gamma}}{r^{2(1+ \beta) + \gamma}} e^{-\frac{\decayrate}{m} (\tau(x) - \tauzero)}.
\end{equation}
The theorem as stated now follows after rescaling the constants $C$ and $\decayrate$ appropriately.
\end{proof}