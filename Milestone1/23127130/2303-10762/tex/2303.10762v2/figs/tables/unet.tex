
\begin{table}
    \centering
    \small
    \begin{tabular}{ p{4em}ccc }
    \toprule 
    \textbf{Model}&  $\boldsymbol{C_{in}}$  & $\boldsymbol{C_{out}}$ & $\boldsymbol{f_{act}}$ \\
    
    \midrule
    
    \multirow{4}{2.em}{\textbf{Encoder}} &  16  & 32 & Leaky-ReLU \\
      & 32  & 64 & Leaky-ReLU  \\
      & 64  & 128 & Leaky-ReLU  \\
      & 128  & 256 & Leaky-ReLU  \\
    
    \midrule
    
    \multirow{5}{2.em}{\textbf{Decoder}} &  256+256  & 128 & Leaky-ReLU  \\
      & 128+128  & 64 & Leaky-ReLU  \\
      & 64+64  & 32 & Leaky-ReLU  \\
      & 32+32  & 32 & Leaky-ReLU  \\
      & 32  & 3 & TanH  \\
    
     \bottomrule 
     
    \end{tabular}
    \caption{U-Net model architecture. Each row represents a convolutional block, with $C_{in}$ and $C_{out}$ indicating the input and output channels, respectively. $f_{act}$ is the activation function used in the block. The symbol ``+'' in $C_{in}$ indicates that the layer input includes skip connections. 
    \label{tab:unet}}
\end{table}
