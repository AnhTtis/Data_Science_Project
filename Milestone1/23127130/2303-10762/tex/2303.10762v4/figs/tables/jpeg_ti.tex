
% Comparison of proposed method to others

\begin{table*}[ht]
    \centering
    \footnotesize
    \begin{tabular}{ cclccccccc }
    \toprule 
    $\boldsymbol{N_P}$ & $\boldsymbol{N_S}$ & \textbf{Method} & \textbf{SD 1.4} & \textbf{SD 2.1} & \textbf{MJ} &\textbf{\dalM}& \textbf{GLIDE} & \textbf{\dalT} & \textbf{Mean}\\
    \midrule
    \multirow{2}{*}{\textbf{0}}  &  \multirow{2}{*}{1024} 
    & \RESM & 72.2 & \textbf{93.9} & 81.1 & 87.9 & \textbf{84.4} & \textbf{73.3} & \textbf{82.1} \\
    & & DIF & \textbf{76.9} & 62.4  & \textbf{83.1} & \textbf{89.8} & 82.7 & 69.7 & 77.4 \\
    
    \midrule

    \multirow{2}{*}{\textbf{720k}} & 1024 
    & \RESML & 99.0 & 98.1 & 98.2 & 93.5 & 94.3 & 79.7 & 93.8 \\
    \cmidrule{2-10}
    & 0 & \RESML & 99.3 & 98.6 & 98.5 & 75.0 & 56.0 & 52.0 & 79.9\\
    \bottomrule
    \end{tabular}
    \caption{Classification accuracy (\%). $N_S$ and $N_P$ are the amount of train samples and pre-train dataset size. Real images and fake images are compressed. DIF achieves slightly worse accuracy relative to \RESM. Pre-trained model \RESML{} shows similar results to uncompressed setting, presumably due to augmentations during training. 
    }
    \label{tab:jpegTI}
\end{table*}
