\section{Conclusions and Future Work}\label{sec:conclusion}

This study provides a twofold contribution: the development of new methods for synthetic image detection and the establishment of a methodology for model lineage analysis. We have shown that CNNs naturally exhibit image artifacts, which we leverage in our method called DIF. DIF extracts fingerprints from generated images, allowing us to detect images that come from the same model or its fine-tuned versions. Our method achieves high detection accuracy, surpassing methods trained under the same conditions and performing similarly to pre-trained state-of-the-art detectors for generated images from popular models. Remarkably, we achieve these results using a small number of generated images (up to 512), while other detectors require significantly more samples for training.

In terms of model lineage analysis, we employ cross-detection as a means to trace fine-tuned generative models. Notably, our analysis reveals that MidJourney is indeed a fine-tuned variant of the Stable Diffusion 1.x model.

However, we identified several drawbacks that require further investigation. Some image generators produce weak fingerprints, which are challenging for DIF. Furthermore, the method performs poorly on certain compression methods and blurred images. We believe that the effect of compression on fingerprint extraction and detection requires additional attention and is a topic for future research. 
