

\section{Datasets}\label{apx:data}

\subsection{GAN Datasets}

The GAN datasets that we use in this work are from the supplementary materials of Wang~\etal~\cite{art:easycnn}. Train set consists of 360k real images corresponding to 20 LSUN classes~\cite{art:lsun} and 360k images generated by 20 ProGAN~\cite{art:progan} models. Models are trained per LSUN class. For the ProGAN$_t$ dataset we randomly select 2,000 images per real and fake class, for each LSUN image class, resulting in 4,000 images per LSUN class overall. Test set of~\cite{art:easycnn} contains images generated by a number of GANs, specifically by StyleGAN~\cite{art:stylegan}, StyleGAN2~\cite{art:stylegan2}, BigGAN~\cite{art:biggan}, StarGAN~\cite{art:stargan}, GauGAN~\cite{art:gaugan} and ProGAN~\cite{art:progan}. To the latter we refer as ProGAN$_e$, which is different from ProGAN$_t$ only in the amount of images per class.

\subsection{Fined-Tuned Stable Diffusion Models}

We will outline the process of constructing datasets for the fine-tuned Stable Diffusion models discussed in \cref{sec:source}.
In all cases, fine-tuning involves the DreamBooth method~\cite{art:dreambooth}, specifically \TI{} is trained to reproduce a target object or style from a pre-defined keyword. There are three models: SD 1.4S, SD 1.5A, and SD 2.0R. The authors customly fine-tuned SD 1.4S with 10 photos. SD 1.5A\footnote{\url{https://huggingface.co/DGSpitzer/Cyberpunk-Anime-Diffusion}}
is a publicly available model that was fine-tuned in two stages: first, SD 1.4 was fine-tuned with 680k anime images, and then, after replacing the image decoder with one from SD 1.5, it was additionally fine-tuned with~\cite{art:dreambooth} on a different set of anime images. The last model, SD 2.0R\footnote{\url{https://huggingface.co/nousr/robo-diffusion-2-base}}%
, is fine-tuned with an unspecified amount of robot images. Following above, keywords are known for each case, as SD 1.4S is trained by us, and for others, keyword is specified in the instructions for each model. To produce images with a new models a keyword is added to captions from the \laion{} dataset~\cite{misc:laion5b} in the following format: ``keyword caption''. This forces fine-tuned model to generate images with new information previously unknown to the source models. \cref{fig:sdExamples} shows an example of outputs resulting from the same caption.

\section{Additional Results}\label{apx:adresults}


\subsection{Results of Gray Image Reconstruction}

For completeness, we present additional examples of the gray image reconstruction experiment (\cref{sec:induce}).
in Figure \ref{fig:monochromeExpand}. In each case, the model produces grid-like artifacts that are densely distributed throughout the image or concentrated within a small region. For each output, we also show the Fourier transform, which has been adjusted for better visualization. We have also included the gray image and its \fft{} for reference, which have undergone the same adjustments.

\begin{figure*}[t]
\captionsetup[subfigure]{labelformat=empty}
\begin{subfigure}{0.16\linewidth}
    \centering
    \begin{minipage}{0.95\textwidth}
        \begin{tikzpicture}[spy using outlines={rectangle,red,magnification=3.5,size=1.5cm, connect spies}]
        \node {\includegraphics[width=1\linewidth, frame]{figs/imgs/blank/target.png}};
        \end{tikzpicture}
        \begin{tikzpicture}[auto]
        \node {\includegraphics[width=1\linewidth]{figs/imgs/blank/targetFFT.png}};
        \end{tikzpicture}
    \end{minipage}
    \caption{Target}
    \end{subfigure}%
     \hspace{30pt}
    \begin{subfigure}{0.16\linewidth}
    \centering
    \begin{minipage}{0.95\textwidth}
        \begin{tikzpicture}[spy using outlines={rectangle,red,magnification=3.5,size=1.5cm, connect spies}]
        \node {\includegraphics[width=1\linewidth,frame]{figs/imgs/blank_extended/blank10_2000.png}};
        \tikzset{spy on other={
        \node {\pgfimage[width=1\linewidth]{figs/imgs/blank_extended/blank10_2000.png}};}}
        \spy on (0.75,-0.5) in node [below] at (0.8,1.5);
        \end{tikzpicture}
        \begin{tikzpicture}[auto]
        \node {\includegraphics[width=1\linewidth,frame]{figs/imgs/blank_extended/blank10_2000FFT.png}};
        \end{tikzpicture}
    \end{minipage}
    \caption{$\hat{Y}_1$}
    \end{subfigure}%
    \hfill
    \begin{subfigure}{0.16\linewidth}
    \centering
    \begin{minipage}{0.95\textwidth}
        \begin{tikzpicture}[spy using outlines={rectangle,red,magnification=3.5,size=1.5cm, connect spies}]
        \node {\includegraphics[width=1\linewidth,frame]{figs/imgs/blank_extended/blank19_2000.png}};
        \tikzset{spy on other={
        \node {\pgfimage[width=1\linewidth]{figs/imgs/blank_extended/blank19_2000.png}};}}
        \spy on (0.75,-0.5) in node [below] at (0.8,1.5);
        \end{tikzpicture}
        \begin{tikzpicture}[auto]
        \node {\includegraphics[width=1\linewidth,frame]{figs/imgs/blank_extended/blank19_2000FFT.png}};
        \end{tikzpicture}
    \end{minipage}
    \caption{$\hat{Y}_2$}
    \end{subfigure}%
    \hfill
    \begin{subfigure}{0.16\linewidth}
    \centering
    \begin{minipage}{0.95\textwidth}
        \begin{tikzpicture}[spy using outlines={rectangle,red,magnification=3.5,size=1.5cm, connect spies}]
        \node {\includegraphics[width=1\linewidth, frame]{figs/imgs/blank_extended/blank33_2000.png}};
        \tikzset{spy on other={
        \node {\pgfimage[width=1\linewidth]{figs/imgs/blank_extended/blank33_2000.png}};}}
        \spy on (0.75,-0.5) in node [below] at (0.8,1.5);
        \end{tikzpicture}
        \begin{tikzpicture}[auto]
        \node {\includegraphics[width=1\linewidth, frame]{figs/imgs/blank_extended/blank33_2000FFT.png}};
        \end{tikzpicture}
    \end{minipage}
    \caption{$\hat{Y}_3$}
    \end{subfigure}%
    \hfill
    \begin{subfigure}{0.16\linewidth}
    \centering
    \begin{minipage}{0.95\textwidth}
        \begin{tikzpicture}[spy using outlines={rectangle,red,magnification=3.5,size=1.5cm, connect spies}]
        \node {\includegraphics[width=1\textwidth, frame]{figs/imgs/blank_extended/blank4_2000.png}};
        \tikzset{spy on other={
        \node {\pgfimage[width=1\textwidth]{figs/imgs/blank_extended/blank4_2000.png}};}}
        \spy on (0.75,-0.5) in node [below] at (0.8,1.5);
        \end{tikzpicture}
        \begin{tikzpicture}[auto]
        \node {\includegraphics[width=1\textwidth, frame]{figs/imgs/blank_extended/blank4_2000FFT.png}};
        \end{tikzpicture}
    \end{minipage}
    \caption{$\hat{Y}_4$}
    \end{subfigure}%
    
     \caption{Results of gray image reconstruction. Left column is the target gray image, other columns are reconstruction results. Bottom row contains \fft{s} of images from the top row. 
     In all cases, the model produces grid artifacts.
     }
     \label{fig:monochromeExpand}
\end{figure*}%


\subsection{The Effect of Train Set Size}

In \cref{sec:detection}
we present an evaluation of the proposed method as a detector of generated images. To evaluate the method under various amounts of training samples, we measure performance on images produced by both GAN and \TI{} models, as shown in \cref{fig:trainGAN,fig:trainT2I}. We observe that the accuracy remains stable across most of the models and starts to degrade drastically at $N_S = 128$, where we observe a 5\% drop for the majority of the models.

\begin{figure}[h]
    \centering
      \includegraphics[width=\linewidth]{figs/imgs/charts/gan_N.pdf}
      \caption{Accuracy (\%) as a function of train samples for GAN models.}
      \label{fig:trainGAN}
\end{figure}%
\begin{figure}[h]
    \centering
      \includegraphics[width=\linewidth]{figs/imgs/charts/t2i_N.pdf}
      \caption{Accuracy (\%) as a function of train samples for \TI{} models.}
      \label{fig:trainT2I}
\end{figure}%


\subsection{The Effect of Booster Loss}

We compare the detection accuracy with and without the use of booster loss and augmentation (\cref{sec:method}). The results are presented in \cref{tab:accuracyDIFs,tab:gans} for GAN models and \TI{s}, respectively. We observe that, for most models, there is little to no variation in accuracy. However, we do observe an increase in accuracy for ProGAN$_e$ and \dalT{} models by 9.5\% and 1.9\%, respectively, which are challenging datasets for DIF. 


% \begin{table}[ht]
%     \centering
%     \small
%     \begin{tabular}{ lcccc }
%     \toprule 
%     \textbf{Image Class} & w/o $\mathcal{L_{B}}$, $\alpha = 1.0$ & w $\mathcal{L_{B}}, \alpha = 0.5$ & DnCNN, $\mathcal{L_{B}}, \alpha = 0.5$ & w/o $\mathcal{L_{B}}$, DnCNN \\
%     \midrule
%     BigGAN & \textbf{98.2}  & 97.3 & 96.8 & 96.9 \\
%     CycleGAN & \textbf{96.4}  & 95.4 & 92.4 & 94.4 \\
%     GauGAN & 88.1  & 92.2 & \textbf{92.6} & 91.8  \\
%     StarGAN & 99.5 & \textbf{99.9} &  99.8 & \textbf{99.9}  \\
%     StyleGAN & \textbf{99.7} & 93.0 & 96.1 & 96.6 \\
%     StyleGAN2 &  \textbf{93.2} & 85.4 & 92.0 & 91.5  \\
%     ProGAN &  56.6 & 61.6 & \textbf{67.1} &  57.6  \\
%     \midrule
%     Mean &  91.2 & 75.0 & \textbf{91.3} & 89.8  \\
%     \bottomrule
%     \end{tabular}
%     \caption{Accuracy of a DIF by GAN models image classes in percents. \\
%     \label{tab:gans}}
% \end{table}

\begin{table}[ht]
    \centering
    \small
    \begin{tabular}{ lcc }
    \toprule 
    \textbf{GAN Model} & $\mathcal{L_{B}}$ & w/o $\mathcal{L_{B}}$\\
    \midrule
    BigGAN  & 96.8 & \textbf{96.9} \\
    CycleGAN  & 92.4 & \textbf{94.4} \\
    GauGAN  & \textbf{92.6} & 91.8  \\
    StarGAN &  99.8 & \textbf{99.9}  \\
    StyleGAN & 96.1 & \textbf{96.6} \\
    StyleGAN2 & \textbf{92.0} & 91.5  \\
    ProGAN$_e$ & \textbf{67.1} &  57.6  \\
    \bottomrule
    \end{tabular}
    \caption{Accuracy (\%) for images generated by GAN models. Usage of $\mathcal{L_{B}}$ implies augmentations during the training.
    \label{tab:gans}}
\end{table}

\begin{table}[]
\centering
\resizebox{1.0\linewidth}{!}{
\begin{tabular}{l|cccc}
\toprule
&  FID $\downarrow$ & Aesthetic Score~\cite{aesthetic_classifier} $\uparrow$  & CLIP Score~\cite{clip} $\uparrow$ & HPS $\uparrow$  \\
\midrule
SD 1.4     & 19.72 & 5.90 & 0.2816 & 0.1898 \\
Adapted model & 19.35 & 6.06 & 0.2831 & 0.1916 \\
\bottomrule
\end{tabular}
}
\vspace{0.3cm}
\caption{Comparison between the original SD v1.4 and the adapted model.}
\label{tab:comparison}
\end{table}
\begin{figure*}[t]
\setlength{\tabcolsep}{2pt}
\begin{tabular}{ccccc}
    \centering
    \includegraphics[width=0.19\textwidth,frame]{figs/imgs/sd_examples/sd14_img_742180.png}% 
    & 
    \includegraphics[width=0.19\textwidth,frame]{figs/imgs/sd_examples/sd14s_img_742180.png}% 
    & 
    \includegraphics[width=0.19\textwidth,frame]{figs/imgs/sd_examples/sd15r_img_742180.png}% 
    & 
    \includegraphics[width=0.19\textwidth,frame]{figs/imgs/sd_examples/sd21_img_742180.png}%
    & 
    \includegraphics[width=0.19\textwidth,frame]{figs/imgs/sd_examples/sd20r_img_742180.png} 
    \\%
    SD 1.4 & SD 1.4S & SD 1.5A & SD 2.1 & SD 2.0R
    \end{tabular}
    \caption{Examples of 
    images produced by Stable Diffusion models and their variants. Images correspond to the caption ``Best Public Arts Installations park 2015'' and addition of model specific keywords.}
    \label{fig:sdExamples}
\end{figure*}


\subsection{Cross-Detection for GAN Models}

We report the complete map of cross-detection for datasets corresponding to GAN and ProGAN$_t$ models in \cref{fig:crossevalGAN,fig:crossevalPROGANA} respectively. As shown in the figures, the cross-correlation is low for all the models, thus models were trained separately.
\begin{figure*}
    \centering
    \includegraphics[width=0.7\linewidth]{figs/imgs/charts/gan_cross_eval.pdf}%
    \vspace*{0em}
    \caption{Cross-detection accuracy (\%) for GAN models.}
    \label{fig:crossevalGAN}
\end{figure*}


\begin{figure*}[t]
    \centering
    \includegraphics[width=1.\textwidth]{figs/imgs/charts/progan_cross_eval.pdf}%
    \caption{Accuracy cross-detection for ProGAN models.}
    \label{fig:crossevalPROGANA}
\end{figure*}



\subsection{Cross-Detection for Custom Trained ProGANs}

Models $P_A$, $P_B$, $\hat{P}_A$ and $\hat{P}_B$ were trained on centrally cropped images ($128 \times 128$ px) from AFHQ~\cite{art:stargan2}. We report the full cross-detection matrix for the four models in \cref{fig:crossevalCUSTOMF}. The cross-detection across all the models is similar, namely no symmetry between different models and high-symmetry within converged models. In addition we present the examples of images generated by $P_A$ ordered according to check point epochs at \cref{fig:catExamples}. Image quality corresponds to the convergence state of the model. Starting from epoch 40 the model produces visually appealing results, that correspond to symmetric and high cross-detection accuracy observed in \cref{fig:crossevalCUSTOMF}. Cross-detection for $\hat{P}_A$ is symmetric, but with lower cross-detection accuracy values. We suspect that in this specific case, the model did not converge properly.


   

\begin{figure*}[!htb]
    \setlength{\tabcolsep}{2pt}
    \centering
    \begin{tabular}{p{1em}ccccc}
    \centering
    \rotatebox{90}{\;\;\; Image Example} &
    \includegraphics[width=0.185\textwidth, frame]{figs/imgs/cat_examples/cat_ep_20.png}%
    & 
    \includegraphics[width=0.185\textwidth, frame]{figs/imgs/cat_examples/cat_ep_32.png}%
    & 
    \includegraphics[width=0.185\textwidth, frame]{figs/imgs/cat_examples/cat_ep_40.png}%
    & 
    \includegraphics[width=0.185\textwidth, frame]{figs/imgs/cat_examples/cat_ep_52.png}%
    & 
    \includegraphics[width=0.185\textwidth, frame]{figs/imgs/cat_examples/cat_ep_70.png} 
    \\%
    & Epoch 20 & Epoch 32 & Epoch 40 & Epoch 52 & Epoch 70
    \end{tabular}
    \caption{Examples of 
    images produced by $P_A$ ordered according to check point epochs. Starting from epoch 40 images retain high-quality.}
    \label{fig:catExamples}
    \centering
    \includegraphics[width=0.9\textwidth]{figs/imgs/charts/progans_cross_eval_full.pdf}%
    \caption{Cross-detection accuracy (\%) for ProGAN models $P_A, P_B, \hat{P}_A$ and $\hat{P}_B$. Clusters of high cross-detection accuracy re-appear for each model at epoch 40,52 and 70.}
    \label{fig:crossevalCUSTOMF}
\end{figure*} 


