%%
%% This is file `sample-authordraft.tex',
%% generated with the docstrip utility.
%%
%% The original source files were:
%%
%% samples.dtx  (with options: `authordraft')
%% 
%% IMPORTANT NOTICE:
%% 
%% For the copyright see the source file.
%% 
%% Any modified versions of this file must be renamed
%% with new filenames distinct from sample-authordraft.tex.
%% 
%% For distribution of the original source see the terms
%% for copying and modification in the file samples.dtx.
%% 
%% This generated file may be distributed as long as the
%% original source files, as listed above, are part of the
%% same distribution. (The sources need not necessarily be
%% in the same archive or directory.)
%% 
\documentclass{article} % For LaTeX2e
\usepackage{iclr2023_conference,times}

% Optional math commands from https://github.com/goodfeli/dlbook_notation.
\newcommand{\bbox}{\text{bbox}}
\newcommand{\alphapck}{\alpha_\bbox}
\newcommand{\kcycle}{\text{k-CyPCK}}
\newcommand{\cycle}{\text{-CyPCK}}

\newcommand{\I}{\mathbf{I}}
\newcommand{\Ia}{\I^\text{a}}
\newcommand{\Ib}{\I^\text{b}}
\newcommand{\Iatob}{\I^\text{a $\rightarrow$ b}}
\newcommand{\F}{\mathbf{F}}
\newcommand{\Fa}{\F^\text{a}}
\newcommand{\Fb}{\F^\text{b}}
\newcommand{\f}{\mathbf{f}}
\newcommand{\fa}{\f^\text{a}}
\newcommand{\fb}{\f^\text{b}}
\newcommand{\p}{\mathbf{p}}
\newcommand{\pa}{\p^\text{a}}
\newcommand{\pb}{\p^\text{b}}
\newcommand{\A}{\boldsymbol{\Phi}_\text{align}}
\newcommand{\G}{\mathbf{G}}
\newcommand{\C}{\mathbf{C}}
\newcommand{\Ca}{\C^\text{a}}
\newcommand{\Cb}{\C^\text{b}}
\newcommand{\cc}{\mathbf{c}}
\newcommand{\cca}{\cc^\text{a}}
\newcommand{\ccb}{\cc^\text{b}}
\newcommand{\Irec}{\I_\text{Recon}}
\newcommand{\M}{\mathbf{M}}
\newcommand{\Mrec}{\M_\text{Recon}}
\newcommand{\loss}{\mathcal{L}}
\newcommand{\T}{\mathcal{T}}
\newcommand{\W}{\mathcal{W}}
\newcommand{\Id}{\mathcal{I}}

%% 
%% To ensure 100% compatibility, please check the white list of
%% approved LaTeX packages to be used with the Master Article Template at
%% https://www.acm.org/publications/taps/whitelist-of-latex-packages 
%% before creating your document. The white list page provides 
%% information on how to submit additional LaTeX packages for 
%% review and adoption.
%% Fonts used in the template cannot be substituted; margin 
%% adjustments are not allowed.

%%
%% \BibTeX command to typeset BibTeX logo in the docs

%% Rights management information.  This information is sent to you
%% when you complete the rights form.  These commands have SAMPLE
%% values in them; it is your responsibility as an author to replace
%% the commands and values with those provided to you when you
%% complete the rights form.
%\renewcommand\footnotetextcopyrightpermission[1]{}
%\copyrightyear{2023}
%\acmYear{2023}
%\setcopyright{acmlicensed}\acmConference[TBD Conference]{TBD'23}{TBD TBD--TBD, 2023}{TBD}
%\acmBooktitle{}
%\acmPrice{}
%\acmDOI{}
%\acmISBN{}



%\documentclass{article}

\usepackage[utf8]{inputenc}
\usepackage{graphicx}
\usepackage[labelfont=bf, font=footnotesize]{caption}
\usepackage{subcaption}

\usepackage{amsthm}
\usepackage{amsmath}
%\usepackage{amssymb}
\usepackage{float}
\usepackage{xcolor}
\usepackage{nameref}
\usepackage{appendix}
\usepackage{algorithm}
\usepackage{algpseudocode}
    
\usepackage{etoolbox,lipsum}
\usepackage{multirow}


\theoremstyle{definition}
\newtheorem{condition}{Bias Condition}
\newtheorem{extension}{Bias Extension}
\setcounter{condition}{-1}
%\newcommand{\R}{\mathbb{R}}

\author{José Pombal\textsuperscript{1,2,3}, Pedro Saleiro\textsuperscript{1}, Mário A. T. Figueiredo\textsuperscript{2,3} \& Pedro Bizarro\textsuperscript{1} \\
\textsuperscript{1}Feedzai \quad \textsuperscript{2}Instituto Superior Técnico, Universidade de Lisboa \quad \textsuperscript{3}Instituto de Telecomunicações \\
%\texttt{jose.pombal@unbabel.com} \quad 
\texttt{pedro.saleiro@feedzai.com}
}



\title{Fairness-Aware Data Valuation \\
for Supervised Learning}

\newcommand{\fix}{\marginpar{FIX}}
\newcommand{\new}{\marginpar{NEW}}

\iclrfinalcopy % Uncomment for camera-ready version, but NOT for submission.


\begin{document}

\maketitle


\begin{abstract}
    \noindent
    %Currently, no data valuation work --- the field that studies the value of training instances towards a certain predictive task --- considers how instances may influence both performance and fairness of machine learning (ML) models.
    %
    %If they are not performance \textit{and} fairness-aware, data valuation techniques, e.g., used for data pre-processing or in active learning, may run the risk of propagating and exacerbating bias embedded within the data.
    %
    %Such a risk has materialized in recent years, with the rapid adoption of ML methods in high-stakes decision-making settings leading to numerous works showing how they can discriminate against certain groups --- giving rise to the field of fair ML.
    %
    Data valuation is a ML field that studies the value of training instances towards a given predictive task. Although data bias is one of the main sources of downstream model unfairness, previous work in data valuation does not consider how training instances may influence both performance and fairness of ML models.
    %
    Thus, we propose \textbf{F}airness-\textbf{A}ware \textbf{D}ata Valuati\textbf{O}n (FADO), a framework
    %that allows measuring which training instances contribute the most to both performance and fairness of models, and 
    that can be used to incorporate fairness concerns into a series of ML-related tasks (e.g., data pre-processing, exploratory data analysis, active learning).
    %
    %We begin by laying out the principles of the framework, and showing how it can be adapted to a myriad of definitions of fairness, including inter-sectional subgroup fairness.
    %
    We propose an entropy-based data valuation metric suited to address our two-pronged goal of maximizing both performance and fairness, which is more computationally efficient than existing metrics.
    %
    We then show how FADO can be applied as the basis for unfairness mitigation pre-processing techniques.
    %
    Our methods achieve promising results --- up to a 40 p.p. improvement in fairness at a less than 1 p.p. loss in performance compared to a baseline --- and promote fairness in a data-centric way, where a deeper understanding of data quality takes center stage.
\end{abstract}

\section{Introduction}\label{sec:intro}

A key ingredient in machine learning (ML) is the data used to train models \citep{hestnessDeepLearningScaling2017,najafabadiDeepLearningApplications2015}. 
%
Even specific instances have been shown to play a crucial part in the performance of ML models \citep{tonevaEmpiricalStudyExample2019,ngiamDomainAdaptiveTransfer2018,zhuLearningTransferLearn2020}: there can be more valuable and less valuable training data instances, with respect to a certain metric of interest.
%
The field of data valuation has arisen in this context, with the goal of associating each training datum with its corresponding value.
%
Several approaches for data valuation have been proposed \citep{ghorbaniDataShapleyEquitable2019,yoonDataValuationUsing2020,ghorbaniDistributionalFrameworkData2020a,kwonBetaShapleyUnified2022,simDataValuationMachine2022} and have been applied in a series of applications, such as noisy labels detection, and active learning, with varying degrees of success \citep{ghorbaniDataShapleyEquitable2019,yoonDataValuationUsing2020,kwonBetaShapleyUnified2022,ghorbaniDataShapleyValuation2021}.

However, along with their exceptional performance in some tasks, ML algorithms have been shown to have the potential to exacerbate and propagate biases embedded within data \citep{exacerbate-inequities-kirchner2016machine,exacerbate-inequities-howard2018ugly,exacerbate-inequities-o2016weapons}.
%
This is especially true if such models are trained with regards only to some performance metric, and if they are treated as unamenable black boxes.
%
Such a stance is particularly problematic in high-stakes decision-making domains, where biased policies may result in certain social groups being systematically discriminated, and locked out of important goods and services (e.g., by race, sex).
%
This phenomenon has sparked the creation of the field of fair ML, which has grown rapidly in recent years \citep{catonFairnessMachineLearning2020,mehrabiSurveyBiasFairness2021,pessachReviewFairnessMachine2022}.

Just as data plays a part in the performance of models, so does it play a part in their potential for discrimination~\citep{saleiro2020dealing,rodolfa2020bias,chakrabortyBiasMachineLearning2021,chenWhyMyClassifier2018}.
%
Indeed, data bias can shape the landscape of fairness and performance~\citep{pombalUnderstandingUnfairnessFraud2022,pombalPrisonersTheirOwn2022}, but little work has been made to understand the sources of these biases.
%
Thus, we propose fairness-aware data valuation (FADO), a data valuation framework that allows measuring which training instances contribute the most to both performance and fairness of models. 
%
We also propose an entropy-based data valuation metric that is specifically tailored to relate to model performance and data bias concerns.
%
%Data valuation has been used with respect to fairness metrics \citep{karlasDataDebuggingShapley2022}, but never with a two-pronged goal of performance and fairness.
%
We argue that the proposed framework embodies the ultimate goal of fair ML, which is to incorporate fairness concerns into ML systems, while keeping their performance high.
%
FADO can be used in the same ways as data valuation, and, as such, can be a vehicle to introduce fairness concerns into a series of upstream and downstream tasks, such as data pre-processing, active learning, data generation, and exploratory data analysis (EDA).
%
\iffalse
The latter is seen by practitioners as a fundamental tool for understanding unfairness \citep{dengExploringHowMachine2022a}, but has been largely disregarded in the fair ML literature.
%
In this work, we show the efficacy of FADO as a means to conduct unfairness mitigation pre-processing interventions.
\fi


In summary, we make the following contributions:
\textbf{1)} we propose fairness-aware data valuation framework based on a notion of utility that incorporates the value of specific instances towards both performance and fairness (lack of data bias). It can be used with continuous and categorical targets and protected attributes, as well as to promote subgroup fairness (see Section~\ref{subsec:framework}).
%
\textbf{2)} we introduce an entropy-based notion of data value, which can be used to calculate an instance's value towards accurate prediction of a target $Y$, and data bias with respect to one or more protected attributes (see Section~\ref{sec:entropy}) in any predictive task.
%
\textbf{3)} we conduct an evaluation of how the framework may be leveraged for fairness-aware training data sampling and re-weighing, achieving promising results on a real world use-case.
%We present empirical results that show that our general-purpose framework exhibits promising fairness-accuracy trade-offs, compared to existing methods in the literature designed specifically for the task at hand --- a 25 p.p. improvement in fairness at 1 p.p. loss in performance.




\section{Background \& Related Work}\label{sec:bg}

%\subsection{Data Valuation}

This work focuses on supervised learning, where data valuation is usually framed in the following way~\citep{ghorbaniDataShapleyEquitable2019,ghorbaniDataShapleyValuation2021,ghorbaniDistributionalFrameworkData2020a,yoonDataValuationUsing2020,karlasDataDebuggingShapley2022}:
%
given a set of training data with \textit{n} data instances, each characterized by a feature vector \(x_i\) (in which one, or more, protected attribute(s) \(z_i\) may be included), and a target variable \(y_i\), the goal is to find the value \(v_i\) of each training instance, with respect to some value metric of interest \(V\), obtained by a learning algorithm \(f\) trained to predict \(y_i\), given \(x_i\).
%
How \(v_i\) is calculated varies from work to work.
%
\citet{ghorbaniDataShapleyEquitable2019} take a game-theoretical approach and calculate the Data Shapley value of each data instance.
%
%For some instance \(i\), this is done through training a model on all possible coalitions of training instances with and without instance \(i\), computing \(V\) for each, and returning the difference between the average \(V\) in coalitions where \(i\) is present, and the average \(V\) in the rest of the coalitions.
%
Although this method enjoys a series of desirable properties, it is intractable to compute, and approximations have been proposed by the original authors and in other works \citep{kwonBetaShapleyUnified2022,ghorbaniDistributionalFrameworkData2020a, karlasDataDebuggingShapley2022}.
%
Conversely, \citet{yoonDataValuationUsing2020} take a reinforcement learning approach (DVRL), where a data valuator model is jointly trained with the predictor of the task from which \(V\) is calculated.

As for the metric of interest, \citet{ghorbaniDataShapleyEquitable2019} and \citet{yoonDataValuationUsing2020} set \textit{V} as the accuracy of a predictor \(f\) on unseen data (test set), with \(v_i\) being the importance of data instance \(i\) towards achieving the performance that predictor \(f\) obtains when trained on the whole dataset.
%
However, \(V\) can be any metric of interest; \citet{karlasDataDebuggingShapley2022}, for example, perform data valuation with equalized odds difference \citep{hardtEqualityOpportunitySupervised2016} as \(V\).
%
That said, a glaring gap in the literature is that no work defines \(V\) as a combination of performance and fairness.
These components, if fairness is mentioned, are always shown separately.
%
However, the main goal of fair ML is to maximize both predictive performance and fairness.
%
Furthermore, besides some applications in active learning and noisy instance detection, data valuation has never been used for fairness-promoting interventions.% --- except indirectly by \citet{liAchievingFairnessNo2022}. 
%
To bridge this gap, we lay out a framework that defines \(V\) as a function of both performance and data bias (fairness), and show how it can be used to promote observational fairness.
%
Being an inherently data-centric task, we also believe that fairness-aware data valuation addresses the lack of exploratory data analysis tools in fair ML.

\iffalse
\subsection{Bias Taxonomy}~\label{subsec:bias-taxonomy}

To be able to reason about the types of bias present in the data, and understand how fairness-aware data valuation works under distinct biases, we leverage the data bias taxonomy proposed by \citet{pombalUnderstandingUnfairnessFraud2022}.
%
Like in that work, we refer to the features of a dataset are $X$, the class label as $Y$, and the protected attribute as $Z$. 
%
The following definitions use the inequality sign ($\neq$) to mean a statistically significant difference, not a \textit{strict} inequality.

\textbf{Base Bias Condition}
    \begin{equation}\label{eq:cond-broad-bias}
        \mathbb{P}[X, Y] \neq \mathbb{P}[X, Y | Z].
    \end{equation}
To satisfy this, $Z$ must be statistically related to either $X$, $Y$, or both. 
%
The following biases imply this condition.

\textbf{Prevalence Disparities}
    \begin{equation}\label{eq:cond-prev}
        \mathbb{P}\left[Y\right] \neq \mathbb{P}\left[Y | Z\right],
    \end{equation}
\textit{i.e.}, the class probability depends on the protected group.

\textbf{Group-wise Class-conditional Distribution Bias}
\begin{equation}\label{eq:cond-condist}
    \mathbb{P}[X|Y] \neq \mathbb{P}[X | Y, Z].
\end{equation}
This condition reads `the distribution of the features conditioned on the target depends on the protected attribute'.

The last condition does not imply the Base bias condition, but can still be relevant for fairness:

\textbf{Group size disparities.}
%
Let $z$ be a particular group from a given protected attribute $Z$, and $N$ the number of possible groups. Under group size disparities, we have
%
\begin{equation}\label{eq:ext-group-size}
    \mathbb{P}\left[Z=z\right] \neq \frac{1}{N} .
\end{equation}
\fi


\section{Fairness-Aware Data Valuation}\label{sec:fado}

\subsection{Framework}\label{subsec:framework}

As mentioned before, the key component of our framework is the definition of \(V\) as a function of both performance and fairness, rather than having them be separate components.
%
Our task is not only to accurately predict \textit{Y}, but also to do it fairly.
%
We shall refer to our metric of interest as utility (\(U\)) from here on in.
%
This nomenclature is inspired by literature on the field of Economics, where the well-being (utility) of an individual is defined as a function of the utility derived from a weighted combination of factors (e.g., economic goods), rather than a single one.
%
In our case, utility is a function of performance and fairness; how it is maximized depends on the definition of the function and on the weights attributed to performance and fairness.
%
Thus, the utility of a single data instance can be written as \(U_i(v_{y_i}, v_{z_i})\), where \(v_{y_i}\) is the performance-related valuation of instance \(i\), and \(v_{z_i}\) is the fairness-related one.

%The choice of \(v\)'s and of how they interact is up to the practitioner, as these decisions are heavily task-dependent --- it can be a Shapley value or any other measurement that is adequate (\citet{simDataValuationMachine2022} have put forth a set of desiderata for data valuation metrics).
%
%In the next section, we introduce our own \(V\) metric, which is tied to the data bias taxonomy proposed by \citet{pombalUnderstandingUnfairnessFraud2022}.
%
%This metric should not, however, be seen as the only metric that can be used within our framework.
%
We propose two types of utility functions; a linear scalarization one:

\begin{equation}
    U_i = \alpha v_{y_i} + (1 - \alpha) v_{z_i},
\end{equation}
where \(\alpha \in [0, 1]\) governs the relative weight placed on performance and fairness, and a multiplicative scalarization one, inspired by the Cobb-Douglas formulation~\citep{douglasCobbDouglasProductionFunction1976} (widely used in Economics):

\begin{equation}
    U_i = v_{y_i}^{\alpha} \cdot v_{z_i}^{1 - \alpha},
\end{equation}
where \(\alpha\) plays the same role as above. 
%
Each of these utilities abide by the principles of fair ML and, thus, the FADO framework, as they strike a balance between performance and fairness: both unfair, high-performing systems, and fair, low-performing systems, are inferior in terms of utility to systems that achieve acceptable values for both metrics.
%
However, the two functions encode distinct preferences as far as this balance goes.
%
For example, the multiplicative utility is zero if either valuation is zero, whereas the linear utility is not.
%
Thus, if a practitioner is not interested in gains in performance if they do not lead to gains in fairness, they should opt for the multiplicative utility.

This general framework allows reasoning about and obtaining fairness-aware valuations for data instances, which can then be used for downstream tasks.
%
Our framework can also be extended to applications concerned with subgroup fairness (fairness across a series of protected attributes and their combinations)%\footnote{For example, for protected attributes race and sex, one would be concerned with the fairness among \(race_i / race_j\) and \(sex_n / sex_m\), instead of just one attribute.})
; one needs only to include several \(Z\) terms in the utility function.
%
For example, in a case where there are \textit{k} protected attributes \(Z_k\), a linear utility function can be defined as:

\iffalse
\begin{equation}
U_i = \alpha v_{y_i} + \beta_a v_{z_{a_i}} + \beta_b v_{z_{b_i}}, 
\end{equation}
where the scalar terms govern the utility placed on performance, and the fairness of each group.
%
Alternatively, if both attributes are categorical, one could include a single \(Z\) term, where \(Z\) would be the label-set defined by the union between all the possible categories in each attribute (e.g., \(Z\) would have 4 labels, if each attribute had 2 groups). 
%
In the general case, for \textit{k} protected attributes:
\fi

\begin{equation}
    U_i = \alpha v_{y_i} + \sum_{j=1}^{k} \beta_j v_{z_{j_i}}, 
\end{equation} 

%In Section~\ref{subsec:proc} we will show how fairness-aware data valuation can be used in an unfairness mitigation pre-processing intervention.
%
In the next Section, we propose an entropy-based metric for FADO, which is versatile for both performance- and fairness-related data valuation.




\subsection{Entropy Metric}\label{sec:entropy}

%\subsection{Introducing the Metric}\label{sec:entropy-intro}

The main challenge posed by the framework is to define and calculate a \(v_y\) for the predictive task at hand, and \(v_z\) for the fairness component of the utility.
%
These could be, for example, the Data Shapley value with respect to a validation set accuracy in the predictive task (\(v_y\)) and the Shapley value with respect to, for instance, a validation set equality of opportunity difference among protected groups in the same predictive task (\(v_z\)).
%
Such a choice would be appropriate within the framework, but entails two difficulties: first, it limits the user to a specific fairness metric, which as we have seen, often implies sacrificing others; second, data Shapley values are intractably hard to compute, thus they must be approximated.
%
Instead, we propose a more metric-agnostic valuation, which is inspired by the literature on uncertainty estimation and active learning (for \(v_y\)), and on the data bias taxonomy proposed by \citet{pombalUnderstandingUnfairnessFraud2022}.

For some data instance \(i\), we propose to define \(v_{y_i}\) as the corresponding prediction Shannon entropy, averaged over the output of one or more trained models\footnote{In the binary classification case; for multi-class, the metric could be entropy for non-binary distributions; for regression, the metric would have to be the prediction's difference from the sample mean, for example.}.
%
In binary classification, the Shannon entropy (\textit{E}) for uncertainty sampling for an instance \(i\) and prediction \(\hat{y}_i\) is:

\begin{equation}
    V_{y_i} = E_{y_i} = \hat{y}_i \cdot \log_2 \hat{y}_i + (1 - \hat{y}_i) \cdot \log_2 (1 - \hat{y}_i)
\end{equation}

The rationale behind choosing the entropy without the target label is inspired by active learning, where there is abundant data outside the training set, but a small budget to label it, and thus it is vital to choose which instances to label.
%
A popular method in active learning is \textit{uncertainty sampling} \citep{lewis1994sequential,settlesActiveLearningLiterature2009}, where an ML model trained on data subsequently chooses to obtain labels for the observations on which it is most unsure (with the highest entropy).
%
The idea is that the model will learn more from adding these observations to the training set, than from adding observations on which the model is sure, as the training data already contains that information.
%
This metric is not perfect, since being confident does not necessarily mean being right, and high entropy observations may simply be noisy and not necessarily useful.
%
However, it is a useful valuation heuristic, which has been shown to work well in many settings, including fraud detection~\citep{lewis1994sequential,yangBenchmarkComparisonActive2018,barataActiveLearningImbalanced2021}.
%
We extend this rationale to data valuation and attribute the highest values to the instances with the highest entropy. %(the calculation of entropy in our case is more involved than in active learning, as we will describe in Section~\ref{subsec:entropy-algorithms}).
%
In other words, we attribute less value to instances where the model is very sure, since there is reason to believe that these encode redundant information to the task at hand. 

As for \(v_{z_i}\), the metric is the same, but using a predictor of \textit{Z}, rather than of \textit{Y}.
%
For an instance \(i\), a binary protected attribute \(z_i\), and predicted group \(\hat{z_i}\), we have:

\begin{equation}
    V_{z_i} = E_{z_i} = \hat{z}_i \cdot \log_2 \hat{z}_i + (1 - \hat{z})_i \cdot \log_2 (1 - \hat{z}_i)
\end{equation}

\iffalse
%
A predictor of Z can be used to diagnose data bias %\textcolor{red}{(SHOW THIS IN THIS PAPER)}
, which is strongly related to downstream unfairness, yet not tied to a specific metric.
%
Indeed, if that predictor were to pass an independence test among Z, X and Y --- i.e., if it were no better than random at predicting Z using the rest of the data --- then its predictions are expected to all have the same utility (or lack thereof) with respect to data bias, as they will all have the same entropy\footnote{Or cross-entropy, if there are more than 2 groups, or, if the attribute is continuous, the MSE would be the same if the mean of Z were predicted for every instance. If there are two groups, with different sizes, entropy would not be maximal, but would be the same across all observations in a dataset.}.
%
It would either be equal to 1, if group sizes are equal --- for N protected groups, the model is expected to always predict \(\frac{1}{N}\) --- or to 0, if there is a majority group, as the model will predict that all observations belong to that group.

Conversely, if the model does not pass the independence test, then instances with lower entropy have features and a target variable that are more associated to Z, and thus contribute towards data bias if the model has access to them in some other predictive task (earning a lower valuation in this respect).
%
Following this rationale, we would prefer instances with higher entropy in the task of estimating \(\mathbb{P}[Z | X, Y]\) in order to minimize data bias.
\fi

%
Notice that the metric is essentially the same for \(v_y\) and \(v_z\), but for different reasons.
%
In the first case, we want to prioritize observations on which the model had more difficulty during training, since these might contain less redundant information.
%
In the second case, where the variable in question (\textit{Z}) is not the target for the task at hand, and is seen by the model even at inference time, we prioritize observations where the model had more difficulty in establishing a relationship among \textit{X}, \textit{Y}, and \textit{Z}, leveraging the fact the model has no explicit incentive to draw such relationships.
%
This is directly related to mitigating the base bias condition of the taxonomy of \citet{pombalUnderstandingUnfairnessFraud2022} (\(\mathbb{P}[X, Y] \neq \mathbb{P}[X, Y | Z]\)), and so related to promoting fairness.
%
This metric can also be seen as a means to promote ``procedural'' fairness ~\citep{greenbergTaxonomyOrganizationalJustice1987} in the learning process of the model: we are ``telling'' the model to pay less attention during training to the protected attribute, and whatever variables may be related with it.

The resulting combination of \(v_y\) and \(v_z\) is a utility measure that balances both performance and fairness by assigning lower value to redundant information about \textit{Y} and excessive information about \textit{Z} in the data.
%
Using a linear scalarization utility function, we would get:

\begin{equation}
    U_i = \alpha E_{y_i} + (1 - \alpha) E_{z_{i}}
\end{equation}
%
\iffalse
Again, this entropy-based metric can easily be extended to the case of subgroup fairness by building a predictor for all the protected attributes (categorical or continuous) of concern:

\begin{equation}
    U_i = \alpha E_{y_i} + \sum_{j=1}^{k} \beta_j E_{z_{j_i}}, 
\end{equation}
\fi

Recently, \citet{xuOnlineDataValuation2022} proposed using entropy as a measure of data value for online ML tasks (with respect only to performance).
%
Other metrics, such as data shapley, are not suitable for these settings, as they require a fixed dataset.
%
Our rationale for using entropy is similar to that of Xu et al., but we extend it seamlessly to the context of data bias and fairness, and to offline ML tasks.
%
In terms of computation, our method requires only the additional training of a model for \textit{Z}.
%
This corresponds --- in our case of binary classification and a binary protected attribute --- to a roughly two-fold increase in training time, which is much faster than Data Shapley or DVRL alternatives.




\section{Unfairness Mitigation Pre-processing with FADO}\label{sec:applications}

\subsection{Dataset}\label{subsec:datasets}

Based on the proposed framework, we apply unfairness mitigation pre-processing interventions on AOF, a real-world large-scale bank account-opening fraud dataset. %(used previously by \citet{f.cruzPromotingFairnessHyperparameter2021,pombalPrisonersTheirOwn2022,pombalUnderstandingUnfairnessFraud2022}).
%
Fraud detection is an extremely pertinent field for fair ML, since holding a bank account is seen as a basic right in the European Union~\citep{basic_account_EU}.
%
In this use case of fraud, fraudsters attempt to impersonate someone to open an account, and quickly exhaust its line of credit.
%
Each row in the dataset corresponds to an application for opening a bank account, submitted via the online portal of a large retail bank.
%
Data was collected over an 8-month period and contains over 500K rows.
%
The first 6 months are used for training and the last 2 months are used for testing, mimicking the procedure of a real-world production environment\footnote{Because of client-related privacy issues, no further details on the data can be provided; however, the dataset introduced by \citet{jesus2022turning} was generated to be faithful to the one used here. Please see {https://www.kaggle.com/datasets/sgpjesus/bank-account-fraud-dataset-neurips-2022} for more information.}.
%
%As a dynamic real-world environment, some distribution drift is expected across time, both from naturally-occurring changes in the behavior of legitimate customers, as well as shifts in the illicit behavior of fraudsters as they learn to better fool the production model.
%
%In terms of data biases, this Base dataset suffers from group size disparities, from prevalence disparities, and from distinct group-wise class-conditional distribution bias.

%Finally, the three variants consist of the AOF dataset with synthetically injected data biases, following the taxonomy presented in Section~\ref{subsec:bias-taxonomy}:
\iffalse
\begin{itemize}
    \item \textbf{Variant I} -- synthetic binary protected attribute (Z) generated with a coinflip; this dataset is \textbf{unbiased}.
    \item \textbf{Variant II} -- synthetic binary protected attribute (Z), generated such that the data only has \textbf{prevalence disparities} (Z is correlated with Y only).
    \item \textbf{Variant III} -- synthetic binary protected attribute (Z), and two synthetic features, \(x_1\) and \(x_2\), which are correlated with Z and Y, such that the data has \textbf{group-wise conditional class distribution bias}. Importantly, only these features and Y are related to Z; the rest of the data is not.
\end{itemize}
\fi

\subsection{Evaluation Framework}
Fraud rate (positive label prevalence) is about $ 1\% $ in both sets.
This means that a naïve classifier that labels all observations as \textit{not fraud} achieves a test set accuracy of almost $(99\%)$.
%
Such large class imbalance entails certain additional challenges for learning~\citep{heLearningImbalancedData2009}, and calls for a specific evaluation framework.
%
In particular, bank account providers are not willing to go above a certain level of FPR, because each false positive may lead to customer attrition (unhappy clients who may wish to leave the bank).
%
At an enterprise-wide level, this may represent losses that outweigh the gains of detecting fraud.
The goal is then to maximize the detection of fraudulent applicants (high global true positive rate, TPR), while maintaining low customer attrition (low global false positive rate).
%
As such, we evaluate the model's TPR at a fixed FPR, imposed as a business requirement in our case-study; we assess the FPR ceiling of $5\%$.
%
%A more typical metric such as accuracy would not be informative, since it is trivial to obtain ~99\% accuracy by classifying all observations as not fraud (recall that fraud rate is around 1\%).
%
As for fairness, we consider three popular metrics for classification settings: the ratio of group-wise FPRs, \textit{predictive equality} \citep{corbett-daviesAlgorithmicDecisionMaking2017}, the ratio of group-wise FNRs, a form of \textit{equality of opportunity} \citep{hardtEqualityOpportunitySupervised2016}, and the ratio of group-wise predicted positive rate, \textit{demographic parity} \citep{dworkFairnessAwareness2012}.
%
The protected attribute used will be the client's age (to counteract \textit{ageism}).


%As for fairness, we must ensure that automated customer screening systems do not disproportionately affect certain protected sub-groups of the population, directly or indirectly.
%
%Fairness with respect to the positive labels is measured as the ratio between group-wise false negative rates (FNR).
%Equalizing FNR is equivalent to the well-known \textit{equality of opportunity} metric~\citep{hardtEqualityOpportunitySupervised2016}, which dictates equal true positive rates (TPR), $TPR = 1 - FNR$.
%In our setting, this ensures that equal proportions of fraud are being caught for each sub-group.
%
%On the other hand, fairness with respect to the label negatives is measured as the ratio between group-wise false positive rates (FPR).
%Within our case-study, equalizing FPR (also known as \textit{predictive equality}~\citep{corbett-daviesAlgorithmicDecisionMaking2017}) ensures no sub-group is being disproportionately denied access to banking services\footnote{Though ommitted for conciseness and lack of practical motivation, the conclusions for demographic parity are roughly the same as for predictive equality across experiments.}.
%
%We also measure \textit{demographic parity}~\citep{dworkFairnessAwareness2012}, which assess whether the model is flagging the same percentage of individuals in each group as fraudulent.

\subsection{Experimental Setup}\label{subsec:experimental-setup}

\iffalse
Data valuation techniques have previously been used for tasks such as active learning \citep{ghorbaniDataShapleyValuation2021} and the detection of noisy/corrupted observations \citep{ghorbaniDataShapleyEquitable2019, yoonDataValuationUsing2020}.
%
In this section, we use our proposed framework and data valuation metric for exploratory data analysis and a fairness intervention.
%
In the latter, we leverage the knowledge of the utility of specific observations in terms of both performance and fairness in training, to achieve better fairness-accuracy trade-offs in testing on a typical fraud detection task.
%
To this end, we perform sampling or reweighting procedures to the training set, based on the computed valuations.
%
We sample observations, prioritising the most valuable ones, or assign distinct weights to each observation during the training of ML models (i.e., assign larger weights to the most valuable observations).
%
As shown in Section~\ref{subsec:proc}, such pre-processing strategies lead to favourable fairness-accuracy trade-offs that are comparable to, or beat, popular methods from the literature, such as prevalence sampling and reweighing~\cite{kamiranDataPreprocessingTechniques2012}.
\fi

We wish to apply some pre-processing method on our training data in order to maximize performance and fairness on unseen data.
%For the following experiments, we will use the AOF dataset with client age as the protected attribute.
%
%Both the target and the protected attribute are binary, but we stress that our data valuation framework can deal with categorical or continuous variables, as well as with more than one protected attribute.
%
%The idea is to perform a sort of utility-aware sampling, or reweighting of observations in the training set, such that fairness-unaware models trained on it obtain better fairness-accuracy trade-offs on unseen data (test set).
%
To this end, our pre-processing method consists of three steps: first, computing each training instance's value with respect to \textit{Y} and \textit{Z}. 
%
Second, computing the utility of each instance, given a utility function and respective parameters. 
%
Third, sampling or reweighing the training set before training models.
%
When sampling is done, the instances with lower utility that belong to the protected group with lowest fraud prevalence are discarded first.
%
In the case of reweighting, the instances with higher utility are assigned larger weights during training.
%
We consider two strategies: utility-aware prevalence sampling (UASP), and utility-aware reweighting (UAR).
%
UASP consists of undersampling the data, such that protected groups end up with the same prevalence (fraud rate); in our dataset, this means discarding label negatives of the group with the lowest fraud rate, starting with the instances with the least utility, until fraud rates are balanced.
%
UAR implies assigning weights to each observation in the training set prior to training, where the weight corresponds to the instance's utility --- the full training set is used.
%
We experiment with several hyperparameter configurations of the above approaches (see Appendix \ref{app-fadv-hp}).

After each preprocessing intervention, the resulting dataset (and instance weights, if it is the case) is used to train 25 LighGBM models (with hyperparameters sampled from a grid), which are then used to make predictions on a test set (the same 25 models are used throughout).
%
LightGBM was chosen for being a state-of-the-art algorithm for tabular data, and for easily allowing weights to be assigned to instances during training.
%
%The performance of each model is assessed with TPR at a binarization threshold such that 5\%FPR is achieved, and several fairness measurements are taken: \textit{predictive equality}, \textit{demographic parity}, and \textit{equality of opportunity}\footnote{In the form of min/max ratios; e.g., for predictive equality \(\frac{min(FPR_A, FPR_B)}{max(FPR_A, FPR_B)}\), such that the resulting figure lies between 0 and 1 (and since the focal point of this particular analysis is not which group is being discriminated, but rather the extent of the discrimination).}.
%
To highlight the models that achieve the best performance-fairness trade-offs, we also analyze the Pareto frontier~\citep{pareto1919manuale} of the landscape.
%
We also outline the 80\% rule~\citep{80percentrule} threshold as a guideline for how fair models should be: the best models are those that achieve the highest performance above this threshold of fairness.
%
We compare our fairness-aware data valuation approaches to the case of no intervention (the full training set), random prevalence sampling (RPS, same as UASP, but observations are randomly discarded, instead of using a utility-based order) and reweighing (RW) \citep{kamiranDataPreprocessingTechniques2012}.
%
%Going back to the \textit{predictive equality} (ratio of group-wise FPR) formula analyzed by \citet{chouldechovaFairPredictionDisparate2017}, for example, we can see that the RPR and RW methods are targeting solely the prevalence component of the equation, while our UASP tackles all terms, and UAS and UAR tackle the precision and TPR terms.
%
%We thus expect our methods to yield more favourable fairness-accuracy trade-offs across a range of metrics.

%The only existing work that performs a similar experiment is by \citet{liAchievingFairnessNo2022}, who leverage influence functions to weigh instances according to their importance towards achieving high performance, and satisfying a specific fairness constraint.
%
%This can be seen as a particular instantiation of our general fairness-aware data valuation framework.
%
%Unfortunately, we could not include their method in our comparison, since the existing implementation requires using Gurobi~\citep{gurobi}, a paid software for which we had no license.
%Either way, it cannot be used with tree models (the state-of-the-art ML models for tabular data).
%
%Comparing their method with our metric is a definite priority for future work.



\iffalse
\subsection{Exploratory Data Analysis}\label{subsec:eda}


Having computed the valuations of each instance (using one of the proposed entropy algorithms), we can use them to perform some exploratory data analysis.
%
Going back to the biased AOF dataset variants used by \citet{pombalUnderstandingUnfairnessFraud2022}, we use data valuation to gain further insight on data bias.
%
To this end, we plot the entropy of both classifiers used (one to obtain \(v_y\) and the other to obtain \(v_z\)) to understand how the two valuations relate under each bias condition, and gauge which instances contribute the most to bias.

Looking at Figures \ref{fig:dv-exploration-base} to \ref{fig:dv-exploration-cond}, we can see that the landscape of valuations, like the one of fairness, changes with data bias.
%
In particular, in the cases where datasets suffer from more ``aggressive'' biases (the Base dataset and Variant III), there are correlations between \(v_y\) and \(v_z\), which also change with the protected group under analysis.
%
This means that there is more room to prioritize those instances that have large contributions to the predictive task (high \(v_y\)), but low contributions to data bias (low \(v_z\)).
%
How this prioritization is made depends on the utility function employed.
%
In the next section, we show how leveraging training instance utility for training data sampling and reweighing can have positive consequences on both downstream performance and fairness in the Base dataset.


\begin{figure}[H]
    \centering
    \includegraphics[width=\linewidth]{Figures/aof_base_plots/FADO_matrix_aof_base_full.png}
    \caption{
    Fairness-aware data valuation matrix split by protected group for the Base dataset (which features group size disparity, prevalence disparity, and group-wise distinct class-conditional distribution biases).
    %
    Each point is a training instance. 
    %
    There is a negative correlation between Z and Y entropies, with the latter having a tendency to be higher for lower values of Z.
    %
    While observations do not pass 0.4 in Z entropy, unprivileged group seems to have ``floor'' on this metric (Z entropy is never lower than 0.1), indicating that models are not as sure when classyfying the protected attribute in these cases (some observations of the privileged group show entropies lower than 0.05).
    %
    Thus, there is room to prioritize certain observations that contribute towards the performance valuation and not to the bias valuation.
    }
    \label{fig:dv-exploration-base}
\end{figure}

\begin{figure}[H]
    \centering
    \includegraphics[width=\linewidth]{Figures/aof_base_plots/FADO_matrix_aof_50_50_full.png}
    \caption{
    Fairness-aware data valuation matrix split by protected group for the Variant I dataset (unbiased). 
    %
    Each point is a training instance.
    %
    Z entropy is high for all values of Y entropy, indicating, correctly, that the data is not biased.
    %
    The distribution of entropies is very similar across protected groups, which also makes sense given the unbiased nature of this dataset.
    %
    There is no benefit in utility-aware sampling or reweighing in this case.
    }
    \label{fig:dv-exploration-gs}
\end{figure}

\begin{figure}[H]
    \centering
    \includegraphics[width=\linewidth]{Figures/aof_base_plots/FADO_matrix_aof_double_50_50_full.png}
    \caption{
    Fairness-aware data valuation matrix split by protected group for the Variant II dataset (prevalence disparity bias). 
    %
    Each point is a training instance.
    %
    Tough Z entropy is generally high for both groups, the unprivileged one (higher fraud rate) has more observations of higher Y entropy and lower Z entropy.
    %
    This is indicates that some performance gains can be achieved by leveraging the data's biases, which is true, but detrimental to fairness. 
    }
    \label{fig:dv-exploration-prev}
\end{figure}

\begin{figure}[H]
    \centering
    \includegraphics[width=\linewidth]{Figures/aof_base_plots/FADO_matrix_aof_conditional_50_50_full.png}
    \caption{
    Fairness-aware data valuation matrix split by protected group for the Variant II dataset (group-wise distinct class-conditional distribution bias).
    %
    Each point is a training instance.
    %
    It is clear that Z and Y valuations correlate for the privileged group (especially for fraudulent instances).
    %
    This makes sense, since the data was engineered such that fraud became easier to detect on the observations of this group.
    %
    There are valuable observations for Y that are not as valuable for Z (bias), meaning there can be gains from sampling and reweighing according to utility.
    }
    \label{fig:dv-exploration-cond}
\end{figure}
\fi


\subsection{Results \& Discussion}\label{subsec:proc}

Figures \ref{fig:dv-joint-zoom} summarizes the performance-fairness trade-off landscape achieved by the models in the test set, trained on data that underwent the pre-processing techniques mentioned in the previous section.
%
It plots performance on the x-axis, and fairness on the y-axis; circles represent the no intervention and literature baselines, while triangles and squares represent our prevalence sampling and reweighing methods, respectively.
%
On our methods, the color varies between green and red.
%
Greener points were trained on data that was pre-processed with a higher weight on fairness in the FADO utility function; redder points had a higher weight placed on performance.
%
The underlying goal of this task is to maximize both performance and fairness, so the best points are the ones closest to the top right corner of the plots.
%
In particular, we care about the points with the highest performance above the 0.8 line, since this represents the aforementioned 80\% rule.

\begin{figure*}[htpb]
    \centering
    \includegraphics[width=\linewidth]{Figures/aof_base_plots/joint_plot_zoom_colorblind_noUAS.pdf}
    \caption{
Each point corresponds to one of 25 LightGBM models trained on the corresponding pre-processesed training set (see legend). %the same 25 models were trained for many configurations of our methods (for example, UAS has more than 10 configurations, yielding more than 250 points).
    The axes represent performance and fairness on the test set in the task of predicting fraud (Y).
    Points that lie on the Pareto frontier  are represented by the larger opaque markers; small opaque markers are present to aid the visibility of baselines. 
    %A given point lies on the Pareto frontier if there is no other point that, at that level of performance/fairness, achieves better fairness/performance.
    The \(\alpha\) colorbar represents the weight attributed to performance (\(\alpha\)) and fairness (1 - \(\alpha\)) in the utility function of our methods.
    %
    %The promising fairness-accuracy trade-off mentioned in the abstract can be found by comparing the rightmost dark blue circle on the leftmost plot, with the rightmost point above the 0.8 line.
    %The Pareto frontiers are largely populated with our sampling and reweighing techniques, with many non-Pareto points achieving favourable trade-offs too --- beating, sometimes in performance and fairness, the models trained on the dataset without intervention.
    }
    \label{fig:dv-joint-zoom}
\end{figure*}


First, as expected, there seems to be a relationship between alpha (the parameter that governs the weight placed on performance and data bias in the utility functions) and the fairness-accuracy landscape --- the top performing models tend to be related to a higher \(\alpha\), while the fairest models feature lower \(\alpha\).
%
Furthermore, the best trade-offs on the Pareto frontier (high performance and high fairness) seem to be achieved by mid-range values (\(\alpha\) around 0.5), which suggests that a balanced valuation of points allows models to be much fairer at little loss in performance.
%
Notice, for example, on the leftmost plot of Figure~\ref{fig:dv-joint-zoom}, the opaque squares with TPR around 0.77 and fairness above 0.85.
%
It also seems that, for all fairness metrics, steep increases in fairness can be achieved with only slight drops in performance.
%
The relative abundance of points coloured between blue and red can be explained by the fact that for each of our proposed methods (UAPS, and UAR) there are more hyperparameters to choose from, and with which we experimented.

Comparing our methods with those from the literature, ours present the best performance above the 0.8 fairness line for all three fairness metrics tested.
%
Figure~\ref{fig:dv-joint-zoom} shows that even outside the Pareto frontier, our methods seem to occupy spaces of higher fairness, for a given level of performance, than RPS and RW --- especially our reweighing approach (UAR).
%
This meets our expectations, since our methods attempt to tackle all sources of data bias, instead of just prevalence disparity which is the case with RPS and RW (though the latter achieves this in an indirect way, without discarding examples).
%
Comparing among our own methods, reweighing approaches seem to obtain the best trade-offs. 
%
Not only do they feature several points on the Pareto frontier of all fairness metrics, but they also seem to perform better in general.
%
This was in part to be expected, as these methods do not require examples to be discarded, thus, in principle, leaving more information in the data for models to learn from.
%
Disregarding fairness for a moment, we can see that our approaches, for high values of \(\alpha\), are useful to increase the performance of ML models.
%
They can reach up to about 1 extra TPR point, when compared to the best performing models trained on the training set without intervention.

%All in all, our findings are promising, and motivate us to experiment with more datasets in future work.
%
%Moreover, we would like to build a more automatic system, where the ideal configuration of hyperparameters can be found by receiving signal from the performance and fairness obtained on a validation set, rather than brute-force experimentation.
%
%Finally, despite this being a particular application of our data valuation framework, we stress that the value vector obtained as a product of the valuation procedure can be used for exploratory data analysis in order to find examples that contribute to data bias, and to promote fairness in other applications like data collection and data generation.



%\subsection{Active Learning}\label{subsec:al}

%Work in progress.



\section{Conclusions and Future Work}\label{sec:conclusions}

We introduced a framework and metric to relate performance and data bias to specific data instances.
%
Our framework can be applied with any type of classification or regression task, as well with any number of categorical and continuous protected attributes.
%
We also introduced an entropy-based notion of value and utility, which incorporates both performance and fairness concerns.
%
To validate the framework and the proposed metric, we showed how they could be applied in the context of an unfairness mitigation pre-processing intervention in a binary classification setting (training set sampling and reweighing).
%
A benchmark on our real-world fraud detection dataset showed that our methods often outperform existing ones in the literature, in terms of fairness-accuracy trade-offs.
%
In the future, we would like to extend this benchmark to inlcude other datasets and baselines.
%
We believe that tracing data bias to specific training instances is a promising practice to further the understanding of algorithmic unfairness and to add another layer of insight to data exploration and other tasks along the ML pipeline, like active learning --- a priority for future work as well.

% BIBLIOGRAPHY
%\clearpage
\bibliography{refs}
\bibliographystyle{iclr2023_conference}

\vfill
\pagebreak

% APPENDIX (MAX 2 PAGE SUPPLEMENT)
\appendix

\section{Algorithms to Compute Entropy}\label{subsec:entropy-algorithms}

Computing the required entropy-based metric for \(v\) is not as straightforward as in active learning, since our goal is to obtain valuations for a fully-labeled training set, and for two variables (Y and Z).
%
With this in mind, we propose two algorithms.
%
The first relies on creating bags from the training data of seen and unseen data, and training several models on each bag, from which entropy is obtained --- akin to k-fold cross-validation.
%
This is more closely related to active learning, since it involves a component of unseen data for which the model makes predictions.
%
The second algorithm is arguably more straightforward as it does not involve sampling the training set, but is more a measure of epistemic uncertainty rather than a combination of epistemic and aleatoric uncertainty~\citep{horaAleatoryEpistemicUncertainty1996} (as is the case with the active learning case)\footnote{Loosely speaking, the latter pertains to uncertainty incorporated by the model, while the former pertains to uncertainty inherent to the data.}.

\begin{algorithm}
    \caption{FADO Out-of-bag Entropy Algorithm}\label{alg:dv-bagging-based}
    \textbf{Input:} training set \textit{D}; variable of interest \textit{var}; number of bags \textit{n\_bags}; \% of train in each unseen set in each bag \textit{pct\_unseen}; models to train (pre-defined by user) \textit{M}\\
    \textbf{Output:} valuation vector with respect to \textit{var} \textit{V}
    \begin{algorithmic}[1]
        \State \textit{in\_bag\_sets}, \textit{out\_of\_bag\_sets} $\gets$ Split \textit{D} into \textit{n\_bags}, each with \textit{pct\_unseen} of \textit{D} into an out-of-bag set, and the rest into an in-bag set 
        \Comment{Data is sampled without replacement within each bag, but with replacement across bags; each observation must appear in at least one set of out-of-bag data}
        \State \textit{v\_list} \(\gets\) Empty list for intermediate target variable valuations
        \For{\texttt{in\_bag, out\_of\_bag \(\in\) [in\_bag\_sets, out\_of\_bag\_sets]}}
            \For{\texttt{m \(\in\) M}}
                \State \textit{fitted\_m} \(\gets\) Train \textit{m} on \textit{in\_bag} to predict \textit{var}
                \State \textit{v} \(\gets\) Vector of entropies of all predictions of \textit{fitted\_m} on \textit{out\_of\_bag}
                \State \textit{v\_list} \(\gets\) Append \textit{v}
            \EndFor
        \EndFor
        \State \textit{V} \(\gets\) Average entropies of each observation over all intermediate valuations in \textit{v\_list} 
    \end{algorithmic}
\end{algorithm}

\begin{algorithm}
    \caption{FADO In-bag Entropy Algorithm}\label{alg:dv-epistemic}
    \textbf{Input:} training set \textit{D}; variable of interest \textit{var}; models to train (pre-defined by user) \textit{M}\\
    \textbf{Output:} valuation vector with respect to \textit{var} \textit{V}
    \begin{algorithmic}[1]
        \For{\texttt{m \(\in\) M}}
            \State \textit{fitted\_m} \(\gets\) Train \textit{m} on \textit{D} to predict \textit{var}
            \State \textit{v} \(\gets\) Vector of entropies of all predictions of \textit{fitted\_m} on \textit{D}
            \State \textit{v\_list} \(\gets\) Append \textit{v}
        \EndFor
        \State \textit{V} \(\gets\) Average entropies of each observation over all intermediate valuations in \textit{v\_list} 
    \end{algorithmic}
\end{algorithm}

\begin{figure*}[h]
    \includegraphics[width=\linewidth]{Figures/DV-Bagging-Iteration-.pdf}
    \caption{One iteration of the bagging-based algorithm, with one bag, one model for Y, and one model for Z.
    The final utility vector is obtained by averaging out all entropies obtained over each iteration, for each observation, and combining them into utilities.}
    \label{fig:dv-bagging-iteration}
\end{figure*}

To obtain a measure of utility, one must run either algorithm with Y as the target (to calculate \(v_y\)) and with Z as the target (to obtain \(v_z\)), for the same training set, models, and bagging splits (if algorithm 1 is chosen).
%
The final utility measure is then dependent on the chosen utility function, and its parameters.
%
As we argued in the previous section, we believe this entropy-based valuation is best suited for a utility function to which both \(v_y\) and \(v_z\) contribute positively (e.g., the linear and multiplicative utilities shown in Section~\ref{subsec:framework}).
%
In terms of the desiderata for data valuation strategies put forth recently by \citet{simDataValuationMachine2022}, our entropy-based metric satisfies D1-D3, and D7 and D9 if the sum of the entropies of each observation is seen to be the value of the whole dataset.
%

\section{Additional Results Plot}

\begin{figure*}[htpb]
    \centering
    \includegraphics[width=\linewidth]{Figures/aof_base_plots/joint_plot_full.pdf}
    \caption{
    Zoomed out version of Figure~\ref{fig:dv-joint-zoom}.
    }
    \label{fig:dv-joint-full}
\end{figure*}


\section{Hyperparameters}
\label{app-fadv-hp}

\begin{table}[H]
    \centering
    \begin{tabular}{@{}llll@{}}
    \textit{\textbf{Hyperparameter\hspace{20pt}}} & \textit{\textbf{Distribution\hspace{20pt}}} & \textit{\textbf{Values\hspace{20pt}}} & \textit{\textbf{Description}} \\
    \textit{boosting\_type}      & -           & ``goss''        & Boosting type                \\
    \textit{n\_estimators}       & Log-uniform & \{20, 10000\} & Number of base tree learners \\
    \textit{num\_leaves}         & Log-uniform & \{10, 1000\}  & Maximum leaves of a tree     \\
    \textit{min\_child\_samples} & Log-uniform & \{5, 300\}    & Min. num. samples needed to create a leaf node                          \\
    \textit{max\_depth}          & Log-uniform & \{2, 20\}     & Maximum depth of a tree      \\
    \textit{learning\_rate}      & Log-uniform & {[}0.02, 0.5{]} & Boosting learning rate       \\
    \end{tabular}%
    \caption{Model hyperparameter grid for LightGBM.}
    \label{tab:app-hp-lgbm}
\end{table}


To compute data valuations, all permutations of the hyperparameters below were tried (\(v_y\) is the entropy of the Y classifier, \(v_z\) is the entropy of the Z classifier):

\textbf{Utility functions}: \(U = \alpha v_y + (1 - \alpha) v_z\), \(U = \alpha v_y - (1 - \alpha) v_z\),% \(U = -\alpha v_y + (1 - \alpha) v_z\), \(U = -\alpha v_y - (1 - \alpha) v_z\), \(U = \frac{v_y^\alpha}{v_z^{1 - \alpha}}\), \(U = \frac{v_z^{1 - \alpha}}{v_y^{\alpha}}\), 
\(U = v_y^{\alpha} v_z^{1 - \alpha}\);

\textbf{alpha} (\(\alpha\)): [0, 0.1, 0.2, 0.3, 0.4, 0.5, 0.6, 0.7, 0.8, 0.9, 1];

%\textbf{Training set fraction}: [0, 0.1, 0.2, 0.3, 0.4, 0.5, 0.6, 0.7, 0.8, 0.9, 1] (only used for UAS);

\textbf{Weights scaling}: ``min-max scaling'', ``no scaling'' (only used for UAR);

\textbf{Entropy algorithms}: ``bagging'', ``epistemic'' (the ones proposed in Section~\ref{subsec:entropy-algorithms});




\end{document}