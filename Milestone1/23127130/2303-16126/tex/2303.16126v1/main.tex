% Template for the submission to:
%   Econometrica   [ecta]
%
%%%%%%%%%%%%%%%%%%%%%%%%%%%%%%%%%%%%%%%%%%%%%%
%% In this template, the places where you   %%
%% need to fill in your information are     %%
%% indicated by '???'.                      %%
%%                                          %%
%% Please do not use \input{...} to include %%
%% other tex files. Submit your LaTeX       %%
%% manuscript as one .tex document.         %%
%%%%%%%%%%%%%%%%%%%%%%%%%%%%%%%%%%%%%%%%%%%%%%

% use option [draft] for initial submission
%            [final] for the prepublication
\documentclass[ecta,nameyear,final]{econsocart}
%
\makeatletter
\def\@journal{arXiv}
\def\journal@url{https://arxiv.org}
\makeatother
%\usepackage{}
\RequirePackage[colorlinks,citecolor=blue,linkcolor=blue,urlcolor=blue,pagebackref]{hyperref}

\startlocaldefs

%%%%%%%%%%%%%%%%%%%%%%%%%%%%%%%%%%%%%%%%%%%%%%
%%                                          %%
%% Uncomment next line to change            %%
%% the type of equation numbering           %%
%%                                          %%
%%%%%%%%%%%%%%%%%%%%%%%%%%%%%%%%%%%%%%%%%%%%%%
%%\numberwithin{equation}{section}
%%%%%%%%%%%%%%%%%%%%%%%%%%%%%%%%%%%%%%%%%%%%%%
%%                                          %%
%% For Assumption, Axiom, Claim, Corollary, %%
%% Lemma, Theorem, Proposition, Hypothezis, %%
%% Fact                                     %%
%\theoremstyle{plain}                 %%
%%                                          %%
%%%%%%%%%%%%%%%%%%%%%%%%%%%%%%%%%%%%%%%%%%%%%%
\theoremstyle{plain}
\newtheorem{theorem}{Theorem}[section]
\newtheorem{lemma}[theorem]{Lemma}
%\newtheorem{???}{???}[???]
%\newtheorem{???}[???]{???}
%%%%%%%%%%%%%%%%%%%%%%%%%%%%%%%%%%%%%%%%%%%%%%
%%                                          %%
%% For Definition, Example, Remark,         %%
%% Notation, Property                       %%
%% use \theoremstyle{remark}                %%
%%                                          %%
%%%%%%%%%%%%%%%%%%%%%%%%%%%%%%%%%%%%%%%%%%%%%%
%\newenvironment{proof}[1][Proof]{\noindent \textbf{#1.} }{\  \rule{0.5em}{0.5em}}
%\theoremstyle{remark}
%\newtheorem{???}{???}
%\newtheorem*{???}{???}
%\newtheorem{???}{???}[???]
%\newtheorem{???}[???]{???}

%%%%%%%%%%%%%%%%%%%%%%%%%%%%%%%%%%%%%%%%%%%%%%
%% Please put your definitions here:        %%
%%%%%%%%%%%%%%%%%%%%%%%%%%%%%%%%%%%%%%%%%%%%%%


\endlocaldefs

\begin{document}

\begin{frontmatter}

\title{ The Value of Information and Circular Settings}
\runtitle{Value of Information}

\begin{aug}
% use \particle for den|der|de|van|von (only lc!)
% [id=?,addressref=?,corref]{\fnms{}~\snm{}\ead[label=e?]{}\thanksref{}}
%
%% e-mail is mandatory for each author
%
%%% initials in fnms (if any) with spaces
%
% Author Orchid ID: enter ID or remove command
\newcommand{\orcidauthorA}{0000-0002-7026-5149} 
\newcommand{\orcidauthorB}{0000-0002-2356-1447} 
% Add \orcidA{} behind the author's name
%\newcommand{\orcidauthorB}{0000-0000-000-000X} % Add \orcidB{} behind the author's name

% Current address and/or shared authorship

\author[id=au1,addressref={add1,add11}]{\fnms{Stefan}~\snm{Behringer}\ead[label=e1]{stefan@stefanbehringer.com}}
\author[id=au2,addressref={add2}]{\fnms{Roman V.}~\snm{Belavkin}\ead[label=e2]{R.Belavkin@mdx.ac.uk}}

%%%%%%%%%%%%%%%%%%%%%%%%%%%%%%%%%%%%%%%%%%%%%%
%% Addresses                                %%
%%%%%%%%%%%%%%%%%%%%%%%%%%%%%%%%%%%%%%%%%%%%%%
\address[id=add1]{%
\orgdiv{Department of Economics},
\orgname{Universität Bielefeld}}
\address[id=add11]{%
\orgdiv{Sciences Po}
\orgname{}}

\address[id=add2]{%
\orgdiv{School of Sciences and Technology},
\orgname{Middlesex University London}}
\end{aug}

%% Put support info here.  Reminder: do not thank the handling coeditor anonymously or by name
\support{We thank the participants at MaxEnt2022 at IHP Paris, Martino Trassinelli, Tilman Börgers, Pierpaolo Battigalli, Roberto Serrano, and anonymous referees for comments.
.}
%
\begin{abstract}
We present a universal concept for the \emph{Value of Information (VoI)} based on Claude Shannon's information and work of Ruslan Stratonovich that has desirable properties for Bayesian decision theory and demand analysis. The Shannon/Stratonovich \emph{VoI} concept is compared to the concept of Hartley \emph{VoI} and applied to an epitome economic application of a circular setting generalizing an example of Ruslan Stratonovich and allowing for a network structure and an investigation of various economic \emph{transport costs}.
\end{abstract}

\begin{keyword}
\kwd{Value of Information}
\kwd{Nonconcavity}
\kwd{Shannon's Information}
\kwd{Bayesian decision theory}
\end{keyword}

\end{frontmatter}
%%%%%%%%%%%%%%%%%%%%%%%%%%%%%%%%%%%%%%%%%%%%%%%%%%%%%%%%%%%%%%%%%%%%%%%%%
%%%% Main text entry area:
%%%%%%%%%%%%%%%%%%%%%%%%%%%%%%%%%%%%%%%%%%%%%%%%%%%%%%%%%%%%%%%%%%%%%%%%%


\section{Introduction}

The problem of how to value and price information in Bayesian decision theory is fundamental but unresolved see e.g. \cite{Moscarini2002}. 
Decision makers with different priors of the state of nature will invalidate a comparison of statistical experiments (signals) using Blackwell's (1951) theorem. Hence a universal \emph{VoI} cannot be found.\footnote{Prior restrictions and restrictions on problems are investigated in e.g. \cite{Lehmann1988}, \cite{DeLara2020}, and \cite{Cabrales2013}. Large numbers are investigated in \cite{Mu2021}.} Furthermore the decision makers' preferences as well as the nature of the decision problem for which information has to be obtained and measured will render the agreement on a universal \emph{VoI} concept even more difficult, see e.g. \cite{Cabrales2013}, or \cite{Frankel2019}. 


Despite its obvious attractiveness for decision theoretic problems, transfers of a \emph{VoI} concept into economic theory involving rational agents has long been sparse. Most of the attention has focused on the demand side properties of such a concept, e.g. \cite{Arrow1971}, \cite{Behringer2021a}, \cite{Behringer2021b}, \cite{Marshak1972}, \cite{Chade2002} and recently also its cost side, see \cite{Pomatto2022}. 

A major concern of \cite{Radner1984}, who have looked at a \emph{VoI} concept from the demand side, have been its non-concavity, i.e. the possibility of increasing marginal returns over some range. If concavity is violated this poses problems for existence and uniqueness of Nash equilibria and its demand for information in partial- and general-equilibrium settings.

In this paper we present a universally applicable \emph{VoI} concept that can accommodate the shortcomings of alternative existing concepts. In particular it can be applied to any decision problem under uncertainty under any experimental signal structure, any nature of the decision problem and is independent of priors. The concept is founded in the works of Claude Shannon and Ruslan Stratonovich and the resulting subject of \emph{Information Theory} , \cite{Shannon1948}, \cite{Stratonovich2020}, \cite{Stratonovich1965}. The Shannon/Stratonovich approach leads to a \emph{VoI} that is always non-negative and always retains its concavity properties. 


 Shannon entropy considerations have also been employed within the literature on rational inattention (see \cite{Sims2003}\ and surveys in \cite{Gabaix2019} or \cite{Mackowiak2023}). In order to show the applicability of the concept to alternative economic settings we extend an example in Stratonovich (1965) and show how to calculate the Shannon/Stratonovich \emph{VoI} in a generic circular setting that has many uses in economic theory. 

The circle is a convenient and general vehicle to introduce real or perceived space into economic modelling. This is due to two important geometrical properties, its compactness and its absence of boundary. The circle may be subdivided into different segments with the resulting spacial network structure allowing for the occurrence of a \emph{transport cost}, i.e. in the simplest case of \emph{physical space}, the \emph{taxi cost} that occurs when moving a vehicle between segments. Varying these \emph{transport cost} allows for the adjustment of the real or perceived spatial impact on consumer welfare for particular economic settings. 

Eventually the \emph{VoI} provides an answer to the question how much a decision maker (the taxi drivers) would be willing to invest to get better information about the statistical properties of the random variable (the location of the passenger) rather than experiment (locate her taxi) at random.

One well established model where the circle is employed the model \emph{perceived space} is the \cite{Salop1979} circle in price theory that is used to model imperfect and monopolistic competition.

Another application of \emph{physical space} relates to dynamic renewable resource economics. In \cite{Behringer2014} resources have a spatial dimension so that harvesting requires the agent to decide about the speed with which to move along a circle so that physical transport costs from movement of the commodity occur.\footnote{These results are generalized in \cite{Zelikin2017}.}  Recent extensions of macro-economic growth-theory to spatial dimensions also employ the circle in e.g. \cite{Brock2008}.

Within the generic circular economic application, the Shannon/Stratonovich \emph{VoI} is compared to a simpler discrete information setting (the \emph{Hartley information}) that often appears in game and decision theoretic applications \cite{Hirshleifer1979}, and various alternative specifications of transport costs and the stochastic environment are investigated. 


\section{The Value of Information}

Following the approach of \cite{Stratonovich1965}, and \cite{Stratonovich2020}, we let $\Omega=X\times U$ denote sets of random variables. We optimize a utility function $U(x,u)$ over the joint probability measure $p(u,x)$ where $x=x_{i},i=1....n$ is a random variable and $u=u_{j},j=1....n$ some action. This maximization is subject to an information constraint where Shannon information is found as the difference of entropies, see \cite{Shannon1948}, the Kullback-Leibler divergence. The total amount of Shannon information is found by integrating over the joint measure. 

We thus have the program:
\begin{eqnarray*}
U(I) &\equiv &\sup \left[ E_{p(u,x)}\left \{ U(x,u)\right \} \text{ s.t. }%
I_{xy}\leq I\right] \\
&=&-\inf_{p(u,x)}\left[ \int_{X,U}c(x,u)p(x,u)dxdu\right] \text{ s.t. }%
\int_{X,U}\ln \left[ \frac{p(x,u)}{\text{ }p(x)p(u)}\right] p(x,u)dxdu\leq I
\end{eqnarray*}%
where we assume the product measure and  $p(x)p(u)$ as the reference measure in Shannon information where $p(x)$ and $p(u)$ are the marginals. 

We now \emph{marginalize}, the joint measure assuming that the $p(x)$ is
its marginal when integrating over the action set $U$, i.e. 
\begin{equation}
\int_{U}p(x,u)du\text{ }  \tag{C}  \label{C}=p(x)
\end{equation}%
so that the \emph{a-priori} distribution is unchanged. Note that $p(x)$
need not be normalized which then implies that the optimal joint measure $%
p(x,u)$ is not normalized either. Normalization satisfies the following fundamental Lemma: 

\begin{lemma} \label{Lemma A0}
Let $X,Y$ and $\left \langle \cdot ,\cdot \right \rangle :X\times Y\rightarrow 
\mathbb{R}$ be the dual pair of spaces with non-degenerate bilinear form. Consider the
normalization map $N:Y\backslash \left \{ 1\right \} _{\bot }\rightarrow Y:$%
\[
N[y]=\frac{y}{\left \langle 1,y\right \rangle }.
\]%
The domain of $N$ is the complement $Y\backslash \left \{ 1\right \} _{\bot }$
of annihilator $\left \{ 1\right \} _{\bot }=\{y:\left \langle 1,y\right \rangle
=0\}$ of the element $1\in X,$ (i.e. $\{y:\left \langle 1,y\right \rangle \neq
0\}.$ The image of $N$ is an affine space $\{y:\left \langle 1,y\right \rangle
=1\}.$ Normalization then has the following properties:%
\[
\left \langle 1,N[y]\right \rangle =1
\]%
\[N\circ N =N \\
\]%
\[
N[c\cdot y] =N[y],\text{ }\forall c\in \mathbb{R}\backslash \left \{ 0\right \}.
\]
\end{lemma} 



We employ a Lagrangian dual approach (c.f. \cite{Rockafellar1989} p.18ff) with
\emph{marginalization constraint} (\ref{C}):%
\[
\mathcal{K=}\int_{X,U}\ln \left[ \frac{p_{ij}(x,u)}{\text{ }p(x)p(u)}\right]
p_{ij}(x,u)dxdu+
\]
\[
\beta \left( \int_{X,U}c(x,u)p_{ij}(x,u)dxdu-C\right)
+\sum_{x,u}\gamma _{ij}\left( \int_{U}p_{ij}(x,u)du-p(x)\right) 
\]%
with a set of partial derivatives that give a set of necessary conditions
(i-iii) for a zero-gradient $\nabla \mathcal{K}(x,u,\beta ,\gamma _{ij})$.
Maximizing over the joint measure $p_{ij}(x,u)$ we find (i)

\begin{eqnarray*}
\frac{\partial \mathcal{K}(p(x,u),\beta )}{\partial p_{ij}(x,u)} 
&=&\ln \left[ \frac{p_{ij}(x,u)}{\text{ }p(x)p(u)}\right] +1_{ij}+\beta
c(x,u)+\gamma _{i}(x)\\
&=&\beta c(x,u)-\ln p(x)-\ln p(u)+\ln p(x,u)+1_{ij}+\gamma _{i}(x)=0_{ij}
\end{eqnarray*}% 
where $1_{ij}$ and $0_{ij}$ are planes with normal vectors $(0,0,1)$ in $%
(x,u,z)$-space at distances $1$ and $0$ from the origin. Also:
\[
\frac{\partial \sum_{x,u}\gamma _{ij}\left( \int_{U}p_{ij}(x,u)du\right) }{%
\partial p_{ij}}=\frac{\partial \sum_{x,u}\gamma _{ij}\left( p_{i}(x)\right) 
}{\partial p_{i}}=\gamma _{i}(x) 
\]%
Thus (i) may be written as 
\[
p(x,u)=e^{-\gamma _{i}(x)-\beta c(x,u)-1_{ij}}p(x)p(u) 
\]%
and necessary conditions (ii-iii)\ are then 
\[
\frac{\partial \mathcal{K}(p(x,u),\beta )}{\partial \beta }%
=\int_{X,U}c(x,u)p(x,u)dxdu-C=0 \tag{ii}  \label{ii}
\]%
and
\[
\frac{\partial \mathcal{K}(p(x,u),\beta )}{\partial \gamma _{i}(x)}%
=\int_{U}p(x,u)du-p(x)=0_{i}. \tag{iii}  \label{iii}
\]

\begin{lemma}\label{Lemma A1}
For the optimal partially normalized measure according to (\ref{C}), integration over U
\[
\int_{U}p^{p-norm}(x,u)du=p(x) 
\]%
implies (and is implied by)%
\[
\int_{U}e^{-\gamma _{i}(x)-\beta c(x,u)}p(u)du=1. \tag{1}  \label{1}
\]%
Integration over X%
\[
\int_{X}p^{p-norm}(x,u)dx=p(u) 
\]%
and implies (and is implied by) 
\[
\int_{X}e^{-\gamma _{i}(x)-\beta c(x,u)}p(x)dx=1. \tag{2}  \label{2}
\]
\end{lemma} 


\begin{proof}
See Appendix.
\end{proof}


We also have:

\begin{lemma}\label{Lemma A3}
Given two sets X and Y with $\Omega=X\times Y$. Partial normalization of some function f(x,y) w.r.t. y satisfies 
\[
N_y\left[ k(x)f(x,y\right] =N_y\left[ f(x,y\right] 
\]
and is thus invariant to multiplication by some function c(x).
\end{lemma} 

\begin{proof}
See that partial normalization w.r.t. y implies:
\[
N_y\left[ k(x)f(x,y\right] =\frac{k(x)f(x,y)}{\int_{Y}^{} k(x)f(x,y)dx}=\frac{f(x,y)}{\int_{Y}^{} f(x,y)dx}=N_y\left[ f(x,y\right].
\]
\end{proof}

We further find:

\begin{lemma}\label{Lemma A2}
The optimal measure, with marginalization is%
\[
p^{p-norm}(x,u)=e^{-\gamma _{i}(x)-\beta c(x,u)}p(u) 
\]%
where 
\[
\gamma _{i}(x)=\frac{\partial \sum_{x,u}\gamma _{ij}\left(
\int_{U}p_{ij}(x,u)du\right) }{\partial p_{ij}}=-\ln \left[ \int_{U}e^{-\beta
c(x,u)}p(u)du\right] 
\]%
is the Lagrange multiplier (a function in $z-x$ space) of the
marginalization constraint (\ref{C}).
\end{lemma} 

\begin{proof}
See Appendix.
\end{proof}

\begin{theorem}\label{Theorem 2.1}
The condition for the optimal joint measure $p^{p-norm}(x,u)$ can be derived as%
\[
\Gamma (\beta )-\beta \Gamma ^{\prime }(\beta )=-I_{x,u} 
\]%
when marginalized according to (\ref{C}) where $\Gamma (\beta )=\ln{Z(\beta )}$ is the cumulant generating function, $Z(\beta )$ is the partition function, and $\beta $ the inverse Lagrange multiplier of the information constraint. Also
\[
\Gamma ^{\prime }(\beta )=E_{x,u}\left \{ c(x,u)\right \} . 
\]%
\end{theorem} 

\begin{proof}
In Lemma \ref{Lemma A2} the optimal marginalized measure was found as
\[
p^{p-norm}(x,u)=e^{-\underbrace{\ln \left[ \int_{U}e^{-\beta c(x,u)}p(u)du%
\right] }_{\gamma _{i}(x,\beta )}-\beta c(x,u)}p(u) 
\]%
for which
\begin{eqnarray*}
\frac{\partial \gamma _{i}(x,\beta )}{\partial \beta } &=&\frac{%
\int_{U}e^{-\beta c(x,u)}p(u)(-c(x,u))du}{\int_{U}e^{-\beta c(x,u)}p(u)du%
} \\
&=&\frac{%
\int_{U}e^{-\gamma _{i}(x)-\beta c(x,u)}p(u)p(x)(-c(x,u))du}{p(x)} \\&=&-\int_{U}\frac{p^{p-norm}(x,u)%
}{p(x)}c(x,u)du.
\end{eqnarray*}%
from Lemma \ref{Lemma A1} and Lemma \ref{Lemma A3}. 

\bigskip
Taking logs of $p^{p-norm}(x,u)$ we have%
\[
\gamma _{i}(x,\beta )+\beta c(x,u)=-\ln \left[ \frac{p(x,u)}{p(x)p(u)}\right]
\]%
Multiplying by $p^{p-norm}(x,u)$ and integrating over $X$ and $U$ we have

\[
\int_{X \times U}p^{p-norm}(x,u)\gamma _{i}(x,\beta )dxdu+\beta
\int_{X \times U}p^{p-norm}(x,u)c(x,u)dxdu=
\]
\[
-\int_{X \times U}p^{p-norm}(x,u)\ln 
\left[ \frac{p(x,u)}{p(x)p(u)}\right]dxdu 
\]%

so using Fubini's theorem and (\ref{C}) again we find 
\[
\int_{X}\gamma _{i}(x,\beta )p(x)dx-\beta \int_{X}\frac{\partial \gamma
_{i}(x,\beta )}{\partial \beta }p(x)dx=-\int_{X \times U}p^{p-norm}(x,u)\ln \left[ 
\frac{p(x,u)}{p(x)p(u)}\right]dxdu . 
\]%
Denoting the cumulant generating function as $\Gamma (\beta )\equiv \int_{X}\gamma _{i}(x,\beta )p(x)dx$ then 
\[
\Gamma (\beta )-\beta \frac{\partial \Gamma (\beta )}{\partial \beta }%
=-E_{x,u}\left \{ \ln \left[ \frac{p(x,u)}{p(x)p(u)}\right] \right \} 
\]%
or 
\[
\Gamma (\beta )-\beta \frac{\partial \Gamma (\beta )}{\partial \beta }%
=E_{x,u}\left \{ I(x,u)\right \} =-I 
\]%
as $\beta $ is the multiplier in the dual, $\beta ^{-1}$ is the multiplier of
the information constraint and 
\[
\Gamma ^{\prime }(\beta )=E_{x,u}\left \{ c(x,u)\right \} =C.
\]
\end{proof}


Note that  $p(x)$ need not be normalized. The analysis may similarly be performed if instead of joint measures conditional measures are employed that are normalized by definition (see \cite{Stratonovich1965}. Because the Hamiltonian is the Legendre-Fenchel transform (see \cite{Tikhomirov2011} §2), of the Lagrangian one may alternatively show from the equation of motion that the $\beta ^{-1}$ is the multiplier of the information constraint.

The optimal measure can be written as%
\[
p^{p-norm}(x,u)=\frac{e^{-\beta c(x,u)}}{\int_{U}e^{-\beta c(x,u)}p(u)du}%
p(x)p(u)=\frac{e^{-\beta c(x,u)}}{e^{\gamma (\beta ,x)}}p(x)p(u)
\]%
and if the linear transformation $G[\cdot ]=\int_{X}e^{-\beta c(x,u)}(\cdot
)dx$ has an inverse, then from (\ref{2}) in Lemma \ref{Lemma A1} 
we have that%
\begin{eqnarray*}
G\left[ \frac{p(x)}{\int_{U}e^{-\beta c(x,u)}p(u)du} \right] &=&G\left[ \frac{p(x)}{e^{\gamma
(\beta ,x)}}\right]=1 
\\&\Leftrightarrow & \\
G^{-1}[1] &=&\int_{U}g^{-1}(x,u)du=\frac{p(x)}{e^{\gamma (\beta ,x)}}.
\end{eqnarray*}%
This leads to the final lemma for an \emph{independence condition} of $p(x)$ and the normalizer $e^{\gamma (\beta ,x)}$ of the optimal measure:
\begin{lemma} \label{Lemma 2.6}
\[
\frac{p(x)}{e^{\gamma (\beta ,x)}}=Z^{-1}\in
\mathbb{R} \backslash\{0\} \Leftrightarrow G[1]=[G^{-1}[1]]^{-1}=Z=\frac{e^{\gamma (\beta ,x)}}{%
p(x)} \in
\mathbb{R} \backslash\{0\}.
\]
so that \emph{the two} marginalization conditions follow: 
\[
\int_{X}p^{p-norm}(x,u)dx=p(u)\text{ and }\int_{U}p^{p-norm}(x,u)du=p(x).
\]
Additionally%
\[
G[1]=\int_{X}e^{-\beta c(x,u)}dx=Z\text{ and }G^{-1}[1]=\int_{U}e^{\beta
c(x,u)}du=Z^{-1}.
\]
\end{lemma}

\begin{proof}
See Appendix.
\end{proof}




\section{The Stratonovich (1965) Example}

\cite{Stratonovich1965} presents a \emph{discrete} canonical example where there are 8 possible realizations of a random variable $x$ on a circle and costs, given some choice $u$ of the agent, are given as \emph{transport costs} of the form
\[
c(x,u)=\min \left \{ \left \vert x-u\right \vert ,8-\left \vert
x-u\right
\vert \right \} 
\]%

As the simplest economic application think of the case of a \emph{taxi} driver who has to pick up a passenger in an idealized city or route, i.e. on the periphery of a circle. She is not paid for the pickup and wants to minimize the transport costs (e.g. those that are given above) that she has for this pickup. She can locate herself optimally and passengers appears at random at equally spaced points with equal probability. She may also acquire binary information about on which half, quarter, eighth,... of the circle a passenger will appear. What would be her willingness to pay for such information? The answer to this question is the \emph{Hartley VoI} that for 8 and 16 possible passenger locations you can see as dots in Figure \ref{fig:1}.

Ruslan Stratonovich's general \emph{VoI} analysis is outlined above and in Theorem \ref{Theorem 2.1} can be applied to this problem to generate the \emph{Shannon/Stratonovich VoI}. 


We extend the stochastic environment of the main motivating example in \cite{Stratonovich1965} to an even number $n$ of uniformly distributed points on the circle which we normalize to the unit circle with circumference 2$\pi$.

Stratonovich employs a transformed transport cost function where $\beta$ is the inverse Lagrange multiplier for the information constraint.
\begin{eqnarray*}
Z_{n}(\beta ) &=&\sum_{x}e^{-\beta c(x,u)}=\left(
1+2\sum_{x=1}^{\frac{n}{2}-1}e^{-\beta x}+e^{-\beta \frac{n}{2}}\right)  \\
&=&\left( 1-e^{-\beta \frac{n}{2}}\right) \coth \left(\frac{\beta }{2}\right)
. \\
&
\end{eqnarray*}

We next provide a definition: 
Let $X$ and $Y$ be linear spaces and let some $g:X\times Y\rightarrow 
%TCIMACRO{\U{211d} }%
%BeginExpansion
\mathbb{R},
%EndExpansion
$ then we call $g$ \emph{translation invariant} if%
\[
\forall w\in Y~\exists \text{ }v\in X\text{ such that }g(x+v,y+w)=g(x,y).
\]
As the information constraint is an inequality, $\beta ^{-1}$$\geq 0$. The  \emph{cumulant generating function} can be derived as
\[
\ln Z_{n}(\beta )=\Gamma_{n} (\beta ). 
\]%
The fact that $\Gamma_{n} ^{\prime }(\beta )=E_{x,u}\left \{ c(x,u)\right \}$ can be seen by expanding expected costs as
\begin{eqnarray*}
\Gamma_{n} ^{\prime }(\beta ) &=&\frac{1}{Z_{n}(\beta )}Z_{n}^{\prime }(\beta )
\\
&=&-\frac{1}{n}\left( 
\begin{array}{c}
\frac{e^{-0\beta }}{\frac{1}{n}\left( 1+2\sum_{j=1}^{\frac{n}{2}-1}e^{-\beta
j}+e^{-\beta \frac{n}{2}}\right) }(0)+ \\ 
\frac{\sum_{x=1}^{\frac{n}{2}-1}e^{-\beta x}}{\frac{1}{n}\left(
1+2\sum_{j=1}^{\frac{n}{2}-1}e^{-\beta j}+e^{-\beta \frac{n}{2}}\right) }%
(2\cdot x)+\frac{e^{-\frac{n}{2}\beta }}{\frac{1}{n}\left( \frac{e^{-\beta }+1%
}{e^{-\beta }-1}(e^{-4\beta }-1)\right) \allowbreak }(\frac{n}{2})%
\end{array}%
\right)  \\
&=&E_{x,u}\left \{ c(x,u)\right \}  \\
&=&\frac{n}{2(e^{\beta \frac{n}{2}}-1)}-\cosh (\beta )\coth (\beta )+\sinh
(\beta )
\end{eqnarray*}%
with%
\[
\lim_{n\rightarrow \infty }\Gamma_{n} ^{\prime }(\beta )=-\cosh (\beta )\coth
(\beta )+\sinh (\beta )=-csch(\beta )\text{ if }\beta >0.
\]%
where $csch$($\beta )$ is the hyperbolic
cosecant of $\beta .$
Alternatively if follows from the Legendre-Fenchel transform.
Information is then calculated from 
\[
I=\beta \Gamma_{n} ^{\prime }(\beta )-\Gamma_{n} (\beta ). 
\]
Following \cite{Stratonovich1965} we define the Shannon/Stratonovich \emph{VoI} function as 
\[
V(I)=U(I)-U(I_{min}) 
\]
where we may set the constant $I_{min}$ to $I(0)$ to satisfy a participation constraint. 


\begin{lemma}\label{Lemma T1}
The cost function of the Stratonovich (1965) Example is translation invariant.
\end{lemma}

\begin{proof}
See Appendix.
\end{proof}


Plotting $V(I)$ parametrically for $\beta$ yields a \emph{VoI} for the $n$ values of 8 (blue) and 16 (yellow) and the corresponding \emph{Hartley VoI} (black dots). For practical purposes it is important to note that the difference between the two \emph{VoI} will be vanishing asymptotically, as shown in \cite{Stratonovich2020}.

\begin{figure}%[htbp]
\begin{center}
\includegraphics[width=13cm]{images/images/Testcc.pdf}
\end{center}
\caption{VoI for circle}
\label{fig:1}
\end{figure}

With a circle that grows with $n$ the \emph{VoI} is found to be \emph{increasing in $n$}, i.e. generates a higher \emph{VoI} for any given level of information $I$. Despite the fact that with more possible realizations of the random variable the value of a unit of information will decrease, the large increase in total transport costs is overriding this \emph{information effect}.


\section{The extended Stratonovich (1965) Example on a unit circle}

Next we look at what happens to the extended Stratonovich example on a
\emph{unit} circle, i.e. one where the circumference remains $2\pi $ independently of the number of uniformly distributed points $n$. 


For the unit circle the transformed costs can be derived as 
\begin{eqnarray*}
\sum_{x}e^{-\beta (c(x,u))}=
1+2\sum_{x=1}^{\frac{n}{2}-1}e^{-\beta x\left( \frac{2\pi}{n} \right)}+e^{-\beta \pi }&=& e^{-\beta \pi }(e^{\beta \pi }-1)\coth\left( \frac{\beta
\pi }{n}\right)
\end{eqnarray*}
which is unbounded in $n$. For a large number of points on the unit circle the relevant partition function however converges.



\begin{lemma}\label{Lemma T2}
With linear costs there exists a reference measure p(u) such that the relevant partition function of the Stratonovich
(1965)\ example on the unit circle is finite, i.e. $\exists \beta ^{-1}>0$ such
that%
\[
Z(\beta )=\sum_{x}e^{-\beta c(x-u,0)}p(x)<\infty \forall n\in \lbrack 1,\infty ).
\]

\end{lemma}

\begin{proof}
See Appendix.
\end{proof}


So the relevant partition function $Z_{n}(\beta)$ can be used to normalize an otherwise unbounded measure. General existence conditions are investigated in \cite{Belavkin2012}. Depending on the choice of the \emph{transport cost} function, this may impose some extra bounds $\beta$. Intuitively we can only consider $\beta$ for which the partition sum is finite. Otherwise there is no probability and the satisfaction of this condition will be checked for the examples that follow.

\begin{lemma}
The relevant partition function for linear cost has a closed form solution%
\[
\lim_{n\rightarrow \infty }Z_{uc}(\beta )=\frac{1-e^{-\beta \pi }}{\beta \pi }. 
\]
\end{lemma}

Compared to the results for the non-normalized circle here the average transport costs remain constant at $\pi /2$. Hence there is no effect on the VoI from the cost side. On the other hand the total \emph{VoI} is now decreasing in $n$. This result is driven purely by the information effect, but there is a limit value, i.e. a  \emph{lower bound}
for the Shannon/Stratonovich \emph{VoI}. 

\bigskip
Figure~\ref{fig:2} shows the \emph{VoI} for the $n$ values of 8 (blue) and 16 (yellow), and the limit value (green).

\begin{figure}[htbp]
\begin{center}
\includegraphics[width=13cm]{images/images/Testaa.pdf}
\end{center}
\caption{VoI for unit circle}
\label{fig:2}
\end{figure}

% \pagebreak

\subsection{The extended Stratonovich (1965) Example on the unit circle with non-linear costs}

In contrast to the above we now assume non-linear transport costs. Note that the cost functions remain \emph{translation invariant}, independently of the exact \emph{transport cost function} employed. Non-linearity would be an appropriate choice to model for example \emph{economies of scale} in perceived or actual distances on the circle. We will first investigate the case where these costs are the log of distances. We then have %
\[
c_{ln}(x,u)=\min \left \{ \ln (\left \vert x-u\right \vert ),\ln (n-\left \vert
x-u\right \vert )\right \}. 
\]%
The relevant partition function becomes:
\begin{eqnarray*}
Z_{\ln}(\beta ) &=&\frac{1}{n}+\frac{2}{n}\sum_{i=1}^{%
\frac{n}{2}-1}e^{-\beta \ln \left(i\frac{2\pi }{n}\right)}+\frac{1}{n}e^{-\beta \ln
(\pi )} \\
&=&\frac{1}{n}+\frac{2}{n}\sum_{i=1}^{\frac{n}{2}-1}\left( i\frac{2\pi }{n}%
\right) ^{-\beta }+\frac{1}{n}\pi ^{-\beta }
\end{eqnarray*}
or 
\begin{eqnarray*}
Z_{\ln}(\beta ) &=&\left(\frac{2}{n}\right)^{1-\beta }\pi ^{-\beta }H\left( \frac{n}{2}-1,\beta \right)
+\frac{1+\pi ^{-\beta }}{n}
\end{eqnarray*}
with $H(n/2-1,\beta )=$ $\sum_{i=1}^{\frac{n}{2}-1}\frac{1}{i^{\beta
}}$ being the Harmonic number of order $\beta $ which again allows for a closed-form solution. 
The \emph{VoI} is now increasing in $n$. Expected transport costs of having $n$ points on a unit circle 
and logarithmic costs between any point and some chosen origin are%
\[
E\{c_{ln}(x,u),n\}=0\cdot \frac{1}{n}+2\cdot \frac{1}{n} \sum_{i=1}^{\frac{n}{2}%
-1}\ln \left(i\frac{2\pi }{n}\right)+\frac{1}{n}\ln \pi,
\]
so that they decrease in $n$. However, they also converge as for a large number of points a limit can be found as%
\[
\lim_{n\rightarrow \infty }E\{c_{ln}(x,u),n\}=\ln \pi -1. 
\]
 Also again a limit of the relevant partition function can be found

\begin{lemma}
If $\beta <1$ then the relevant partition funciton is
\[
\lim_{n\rightarrow \infty }Z_{\ln}(\beta )=\lim_{n\rightarrow \infty }\left \{\left(\frac{2}{n}%
\right)^{1-\beta }\pi ^{-\beta }H\left(\frac{n}{2}-1,\beta\right)+\frac{1+\pi ^{-\beta
}}{n}\right \}<\infty \text{. } 
\]%
\end{lemma}

This limit can again be found in closed form solution as 
\[
\lim_{n\rightarrow \infty }Z_{\ln}(\beta )=\frac{\pi ^{-\beta }}{1-\beta }. 
\]

%\pagebreak
Figure~\ref{fig:3} shows the \emph{VoI} for the $n$ values of 8 (blue) and 16 (yellow), and the limit value (green).
\begin{figure}[htbp]
\begin{center}
\includegraphics[width=13cm]{images/images/Testbb.pdf}
\end{center}
\caption{VoI for unit circle with log costs}
\label{fig:3}
\end{figure}

Similar concave transformations such as root transport costs of the form
\[
c_{root}(x,u)=\min \left \{ \sqrt{\left \vert x-u\right \vert },\sqrt{n-\left \vert
x-u\right \vert }\right \} 
\]%
with expected costs 
\[
E\{c_{root}(x,u),n\}=0\cdot \frac{1}{n}+2\cdot \frac{1}{n} \sum_{i=1}^{\frac{n}{2}%
-1}\sqrt{i\left(\frac{2\pi }{n}\right)}+\frac{1}{n}\sqrt{\pi } 
\]%
that imply increasing average transport costs but do not yield closed form solutions for $Z(\beta )$. Still the procedure of parametric plots yield information about this setting which can be seen in Figure 
Figure 4 which shows the \emph{VoI} for the $n$ values of 8 (blue) and 16 (yellow), and the limit value (green) with root costs. Here it is interesting to note that despite the fact that cost is increasing in n, and hence makes information more valuable to avoid such larger costs the effect is overcompensated by the information effect that dilutes information when more possible realizations of the random variable are to be considered.
\begin{figure}[htbp]
\begin{center}
\includegraphics[width=13cm]{images/images/Testroot.pdf}
\end{center}
\caption{VoI for unit circle with root costs}
\label{fig:4}
\end{figure}

\begin{figure}[htbp]
\begin{center}
\includegraphics[width=13cm]{images/images/Testquad.pdf}
\end{center}
\caption{VoI for unit circle with quadratic costs}
\label{fig:5}
\end{figure}

As mentioned above, finding closed form solutions for $Z(\beta )$ is not generic and depends mutually on the choice of the cost function and the stochastic and spatial environment. Quadratic costs on the circle, as a natural choice for a non-concave transformation, are another case where $Z(\beta )$ does not have a closed form. A parametric plot reveals that again we find that the \emph{VoI} is decreasing in n, see Figure 5. \cite{Belavkin2022} has shown that in a non-circular setting on the real line such quadratic costs with a normal reference measure do lead to closed form solutions for $Z(\beta )$ and optimal conditional probabilities that are Gaussian.

\section{Conclusion}
We have presented the Shannon/Stratonovich \emph{VoI} concept that overcomes many of the shortcomings of alternative approaches. For example the concept avoids issues of non-concavities that are central to economic concerns as put forward prominently in the works of \cite{Radner1984}.

We have then applied the concept to generic economic situations that are set on the periphery of a circle as used for example in industrial organization, spatial macroeconomics, or renewable resource economics. 

We observe that when extending the original example of \cite{Stratonovich1965} to a \emph{unit} circle, i.e. one that is bounded by the same interval of values, having more options $n$ (e.g. more firm locations, more product varieties, more resource patches) leads to an \emph{information effect} that \emph{lowers} the \emph{VoI} for any given level of information even if these costs are increasing in $n$ and therefore increasing the \emph{VoI} ceteris paribus.

This finding holds independently of whether costs are better approximated by a concave or a convex function of the distance on the unit circle, e.g. by a quadratic or a square root transformation. In all cases the \emph{VoI} actually \emph{decreases} with $n$ as the information effect (the dilution of present information) is relatively stronger than the cost effect, with the exception of logarithmic costs.

The existence of a limit for the \emph{VoI} in $n$ implies that even if locations, varieties, or patches on the circle are best modelled with a density, the \emph{VoI} graph retains a positive slope despite the vanishing information effect.


%%%%%%%%%%%%%%%%%%%%%%%%%%%%%%%%%%%%%%%%%%%%%%
%% Single Appendix:            %%
%%%%%%%%%%%%%%%%%%%%%%%%%%%%%%%%%%%%%%%%%%%%%%
%\begin{appendix}
%\section*{???} %% if no title is needed, leave empty \section*{}.
%\end{appendix}
%%%%%%%%%%%%%%%%%%%%%%%%%%%%%%%%%%%%%%%%%%%%%%
%% Multiple Appendixes:        %%
%%%%%%%%%%%%%%%%%%%%%%%%%%%%%%%%%%%%%%%%%%%%%%



\begin{appendix}
\section*{}

\begin{proof}
OF LEMMA 2.2: Given 
\[
p(x,u)=e^{-\gamma _{i}(x)-\beta c(x,u)-1_{ij}}p(x)p(u) 
\]%
then even if only partially normalized the optimal measure satisfies%
\begin{eqnarray*}
p^{p-norm}(x,u) &=&\frac{p(x,u)}{\int_{U}p(x,u)du}=\frac{e^{-\beta c(x,u)-1_{ij}}}{\int_{U}e^{-\beta
c(x,u)-1_{ij}}p(x)p(u)}p(x)p(u)du= \\
&&\frac{e^{-\beta c(x,u)}}{\int_{U}e^{-\beta
c(x,u)}p(u)du}p(u)
\end{eqnarray*}%
by Lemma \ref{Lemma A0} so we may disregard the $(x,u,1)$ plane as normalization pertains in dual vector space. Using the form%
\[
p^{p-norm}(x,u)=e^{-\gamma _{i}(x)-\beta c(x,u)}p(x)p(u) 
\]%
and integrating the optimal partially normalized measure over $U$ given the marginalization condition (\ref{C}) yields 
\[
\int_{U}p^{p-norm}(x,u)du=p(x)\int_{U}e^{-\gamma _{i}(x)-\beta
c(x,u)}p(u)du=p(x) 
\]%
iff%
\[
\int_{U}e^{-\gamma _{i}(x)-\beta c(x,u)}p(u)du=1 
\]%
Integrating the optimal partially normalized measure over $X$ yields 
\[
\int_{X}p^{p-norm}(x,u)dx=p(u)\int_{X}e^{-\gamma _{i}(x)-\beta
c(x,u)}p(x)dx=p(u) 
\]%
iff%
\[
\int_{X}e^{-\gamma _{i}(x)-\beta c(x,u)}p(x)dx=1. 
\]
\end{proof}

\begin{proof}
OF LEMMA 2.4: By condition i) and ii) of the Lagrangian we can write the optimal measure as%
\[
p^{p-norm}(x,u)=\frac{e^{-\beta c(x,u)}p(x)p(u)}{\int_{U}e^{-\beta
c(x,u)}p(x)p(u)du}=\frac{e^{-\beta c(x,u)}p(u)}{\int_{U}e^{-\beta
c(x,u)}p(u)du} 
\]%
following Lemma \ref{Lemma A3} and thus%
\[
p^{p-norm}(x,u)=e^{-\ln \left[ \int_{U}e^{-\beta c(x,u)}p(u)du\right]
-\beta c(x,u)}p(u) 
\]%
so that by the Lagrange conditions for the marginalized optimal solution it
follows that%
\[
\gamma _{i}(x)=-\ln \left[ \int_{U}e^{-\beta c(x,u)}p(u)du\right]
.
\]
\end{proof}

\begin{proof}
OF LEMMA 2.6:
If $\frac{p(x)}{e^{\gamma (\beta ,x)}}=Z^{-1}\in
\mathbb{R} \backslash\{0\},$ then  
\[
G\left[ \frac{p(x)}{e^{\gamma (\beta ,x)}} \right]=G[Z^{-1}]=\int_{X}e^{-\beta
c(x,u)}(Z^{-1})dx=Z^{-1}\int_{X}e^{-\beta c(x,u)}dx=Z^{-1}G[1]=1
\]%
where the last equality follows from by (\ref{2}) in Lemma \ref{Lemma A1}. Note that%
\[
Z=\int_{X}e^{-\beta c(x,u)}dx.\tag{I}  \label{I}
\]%\tag{I}  \label{I}
This gives
\[
G[1]=Z=\frac{e^{\gamma (\beta ,x)}}{p(x)} \in
\mathbb{R} \backslash\{0\}
\]%
If $G[1]=Z\in 
\mathbb{R}
\backslash \{0\},$ then $G$ is invertible, and using $G^{-1}[1]=\frac{p(x)}{%
e^{\gamma (\beta ,x)}}$ we have%
\[
1=G^{-1}[Z]=\int_{U}g^{-1}(x,u)Zdu=Z\int_{U}g^{-1}(x,u)du=ZG^{-1}[1]=Z\frac{%
p(x)}{e^{\gamma (\beta ,x)}}
\]%
from which it follows that $\frac{p(x)}{e^{\gamma (\beta ,x)}}=Z^{-1}\in
\mathbb{R} \backslash\{0\}.$

For $\frac{p(x)}{e^{\gamma (\beta ,x)}}=Z^{-1}\in
\mathbb{R} \backslash\{0\}$ we therefore have%
\[
p^{p-norm}(x,u)=\frac{e^{-\beta c(x,u)}}{\int_{U}e^{-\beta c(x,u)}p(u)du}%
p(x)p(u)=\frac{e^{-\beta c(x,u)}}{e^{\gamma (\beta ,x)}}p(x)p(u)=\frac{%
e^{-\beta c(x,u)}}{Z}p(u)
\]%
with%
\[
Z=\int_{X}e^{-\beta c(x,u)}dx=\frac{\int_{U}e^{-\beta c(x,u)}p(u)du}{p(x)}\tag{II}  \label{II}
\]%
so that \emph{both} marginalization conditions now follow: 
\[
\int_{X}p^{p-norm}(x,u)dx=\int_{X}\frac{e^{-\beta c(x,u)}}{Z}p(u)dx=\frac{Z}{%
Z}p(u)=p(u)
\]
\[
\int_{U}p^{p-norm}(x,u)du=\int_{U}\frac{e^{-\beta c(x,u)}}{Z}p(u)du=\frac{%
e^{-\beta c(x,u)}}{Z}=\frac{e^{-\beta c(x,u)}}{e^{-\beta c(x,u)}}p(x)=p(x).
\]%
Additionally%
\[
G[1]=\int_{X}e^{-\beta c(x,u)}dx=Z\text{ and }G^{-1}[1]=\int_{U}e^{\beta
c(x,u)}du=Z^{-1}.
\]


\end{proof}


\begin{proof}
OF LEMMA 3.1:
Translation invariance for costs $c:X\times U\rightarrow 
%TCIMACRO{\U{211d} }%
%BeginExpansion
\mathbb{R}
%EndExpansion
$ is defined as%
\[
\forall w\in U~\exists \text{ }v\in X\text{ such that }c(x+v,y+w)=c(x,y)
\]%
In the circle example with uniform density the impact of action is nullified by defining costs symmetrically relative to any position, e.g. $%
w=-y$ for any $u\in U.$ Then there is a $v\in X$ such that $c(x,u)=c(x+v,0)$%
. This implies that also the transformation $e^{-\beta c(x,u)}$ is translation invariant and
\[
\sum_{x}e^{-\beta c(x,u)}=\sum_{x}e^{-\beta c(x-u,0)}.
\]%
With costs defined in absolute value, the residual costs also satisfy
\[
\sum_{x}e^{-\beta c(\left \vert x-u\right \vert ,0)}=\sum_{u}e^{-\beta c(\left \vert x-u\right \vert ,0)}.
\]
\end{proof}


\begin{proof}
OF LEMMA 3.2:

With translation invariance let $C(x)=\sum_{u}e^{-\beta c(x-u,0)}p(u).$ This
depends on the reference prior $p(u)$ that may be any (normalized) measure.
So we may choose $p(u)$ such that $C(x)=\sum_{u}e^{-\beta
c(x-u,0)}p(u)=\sum_{u}g(x-u,0)p(u)=Z$ for $Z\in 
%TCIMACRO{\U{211d} }%
%BeginExpansion
\mathbb{R}
%EndExpansion
\backslash \{0\}$. For example on the
unit-circle with linear residual cost, total costs of all actions are 
\[
\sum_{u}c(\left \vert x-u\right \vert ,0)=\sum_{x}c(\left \vert x-u\right \vert ,0)=0+2\sum_{x=1}^{\frac{n}{2}-1}x(\frac{2\pi }{n})+\pi
=\allowbreak \frac{1}{2}\pi \left( n-2\right) +\pi =\allowbreak \frac{1}{2}%
\pi n
\]%
so that average costs are 
\[
\sum_{x}c(x-u,0)p(x)=\pi /2\text{ if }p(x)=1/n.
\]%
independent of $x.$Then 
\[
\lim_{n\rightarrow \infty }\sum_{x}g(x-u,0)p(x) =\frac{1-e^{-\beta \pi }}{\beta \pi}<\infty 
\text{ for some }\beta^{-1} >0\text{ and }p(x)=1/n
\]
so also 
\[
\lim_{n\rightarrow \infty }\sum_{u}g(x-u,0)p(u)=Z_{\infty }
\]%
is bounded for reference measure $p(u)=1/n$ and some $\beta ^{-1}>0$. The optimal
joint measure was found as%
\[
p(x,u)=\frac{p(x)g(x,u)p(u)}{C(x)}=\frac{p(x)g(x,u)p(u)}{Z}
\]%
summing over all $x$ yields 
\[
\sum_{x}\frac{p(x)g(x,u)}{Z}=1
\]%
due to partial marginalization w.r.t. $x$. Then assuming the transformation $%
L[.]=\sum_{x}g(x,u)[.]$ has an inverse $L^{-1}[.]$ we have%
\[
L\left[ \frac{p(x)}{Z}\right] =Z^{-1}\sum_{x}g(x-u,0)p(x)=1\text{ iff }%
L^{-1}[1]=\frac{p(x)}{Z}
\]%
and thus with linear residual costs the relevant partition function is:   
\[
Z(\beta )=\sum_{x}g(x-u,0)p(x)
\]%
which is bounded for all $n$ for some $\beta^{-1} >0$. 
Also%
\[
\sum_{x}p(x,u)=\sum_{x}\frac{p(x)g(x,u)p(u)}{Z}=\frac{Zp(u)}{Z}=p(u)
\]%
and%
\[
\sum_{u}p(x,u)=\sum_{u}\frac{p(x)g(x,u)p(u)}{Z}=\frac{p(x)C(x)}{Z}%
=p(x).
\]

\end{proof}



\end{appendix}
%% if your bibliography is in bibtex format, uncomment commands:
%\bibliographystyle{ecta-fullname}  Style BST file
%\bibliography{sample.bib}   Bibliography file (usually '*.bib')

%% or include bibliography directly:

\newpage

\begin{thebibliography}{}

\bibitem[Arrow(1971)]{Arrow1971}
ARROW, KENNETH (1971): \textquotedblleft The Value of and Demand for Information\textquotedblright, in \emph{Decision and Organization}, ed. by C.B. McGuire and R. Radner. Amsterdam: North-Holland.

\bibitem[Behringer and Upmann(2014)]{Behringer2014}
BEHRINGER, STEFAN AND  THORSTEN UPMANN (2014): \textquotedblleft Optimal harvesting of a spatial renewable resource\textquotedblright, \emph{Journal of Economic Dynamics and Control}, 42, 105-120.

\bibitem[Behringer(2021a)]{Behringer2021a}
BEHRINGER, STEFAN (2021a): \textquotedblleft Multiplicative Normal Noise and Nonconcavity in the Value of Information\textquotedblright, \emph{Theoretical Economics Letters}, 11, 116-124.

\bibitem[Behringer(2021b)]{Behringer2021b}
BEHRINGER, STEFAN (2021b): \textquotedblleft Expanding Multi-Market Monopoly and Nonconcavity in the Value of Information \textquotedblright, arXiv. https://arxiv.org/abs/2111.00839

\bibitem[Belavkin(2012)]{Belavkin2012}
BELAVKIN, ROMAN (2012): \textquotedblleft Optimal measures and Markov transition kernels\textquotedblright, in \emph{Journal of Global Optimizaiton}, 55, 387-416. 

\bibitem[Belavkin(2022)]{Belavkin2022}
BELAVKIN, ROMAN (2022): \textquotedblleft Value of Shannon's Information Examples and On Normalization of Measures \textquotedblright, Unpublished Manuscript, University of Nottingham.

\bibitem[Brock and Xepapadeas(2008)]{Brock2008}
BROCK, WILLIAM AND ANASTASIOS XEPAPADEAS, (2008): \textquotedblleft Diffusion-induced instability and pattern formation in infinite horizon recursive optimal control\textquotedblright, \emph{Journal of Economic Dynamics and Control}, 32 (9), 2745-2787.

\bibitem[Cabrales(2013)]{Cabrales2013}
CABRALES, ANTONIO AND OLIVIER GOSSNER, AND ROBERTO SERRANO (2013): \textquotedblleft Entropy and the Value of Information for Investors\textquotedblright, \emph{American Economic Review}, 102, 1, 360-377.

\bibitem[Chade and Schlee(2002)]{Chade2002}
CHADE, HECTOR AND EDWARD SCHLEE (2002): \textquotedblleft Another look at the Radner-Stiglitz nonconcavity in the value of information\textquotedblright, \emph{Journal of Economic Theory}, 107, 421-452.

\bibitem[DeLara and Gossner(2020)]{DeLara2020}
DE LARA, MICHELE AND OLIVIER GOSSNER (2020): \textquotedblleft Payoff-Beliefs Duality and the Value of Information\textquotedblright, \emph{SIAM Journal on Optimization}, Vol. 30, 1, 464-489.

\bibitem[Frankel and Kamencia(2019)]{Frankel2019}
FRANKEL, ALEXANDER AND EMIR KAMENICA (2019): \textquotedblleft Quantifying information and uncertainty\textquotedblright, \emph{American Economic Review}, 109(10), 3650-3680.

\bibitem[Gabaix(2019)]{Gabaix2019}
GABAIX, XAVIER (2019): \textquotedblleft Behavioral Inattention\textquotedblright, in \emph{Handbook of Behavioural Economics: Applications and Foundations}, ed. by. Bernheim B.D., DellaVigna, S., and Laibson, D., Elsevier, 2019, Volume 2, Chapter 4, 261-343.

\bibitem[Hirshleifer and Riley(1979)]{Hirshleifer1979}
HIRSHLEIFER, JACK AND JOHN RILEY (1979): \textquotedblleft The Analytics of Uncertainty and Information - An Expository Survey\textquotedblright, \emph{Journal of Economic Literature}, XVII, 1375-1421.

\bibitem[Lehmann(1988)]{Lehmann1988}
LEHMANN, ERICH (1988): \textquotedblleft Comparing location experiments\textquotedblright, \emph{Annals of Statistics}, 16(2), 521-533.

\bibitem[Mackowiak et al.(2023)]{Mackowiak2023}
MACKOWIAK, BARTOSZ, FILIP MATEJA, AND MIRKO WIEDERHOLT (2023): \textquotedblleft Rational Inattention: A Review\textquotedblright, \emph{Journal of Economic Literature},  61 (1): 226-73.
  
\bibitem[Marshak and Radner(1972)]{Marshak1972}
MARSHAK, JACOB AND ROY RADNER (1972): \emph{Economic Theory of Teams}. Yale University Press.

\bibitem[Moscarini and Smith(2002)]{Moscarini2002}
MOSCARINI, GIUSEPPE AND LONES SMITH (2002): \textquotedblleft The law of large demand for information\textquotedblright, \emph{Econometrica}, 70. No. 6. 2351-2366. 

\bibitem[Mu et al.(2021)]{Mu2021}
MU, XIAOSHENG, LUCIANO POMATTO, AND PHILIPP STRACK, AND OMER TAMUZ (2021): \textquotedblleft From Blackwell Dominance in Large Samples to Rényi Divergences and Back Again\textquotedblright, \emph{Econometrica}, 89. No. 1. 475-506. 

\bibitem[Pomatto et al.(2022)]{Pomatto2022}
POMATTO, LUCIANO, PHILIPP STRACK, AND OMER TAMUZ (2020): \textquotedblleft The cost of information\textquotedblright, in \emph{arXiv}, https://arxiv.org/abs/1812.04211

\bibitem[Radner and Stiglitz(1984)]{Radner1984}
RADNER, ROY AND JOSEPH STIGLITZ (1984): \textquotedblleft A nonconcavity in the value of information \textquotedblright, in \emph{Bayesian Models of Economic Theory}, ed. by M. Boyer and R. Kilstrom. Elsevier, 1984, 33-52.

\bibitem[Rockafellar(1989)]{Rockafellar1989}
ROCKAFELLAR, TERRY (1989): \emph{Conjugate Duality and Optimization, Regional conference series in applied mathematics}, SIAM. 

\bibitem[Salop(1979)]{Salop1979}
SALOP, STEPHEN (1979): \textquotedblleft Monopolistic competition with outside goods\textquotedblright,\emph{The Bell Journal of Economics}, 10(1), 141-156.

\bibitem[Shannon(1948)]{Shannon1948}
SHANNON, CLAUDE (1948): \textquotedblleft A Mathematical Theory of Communication \textquotedblright, \emph{Bell System Technical Journal}, 27 (3), 379-423.

\bibitem[Sims(2003)]{Sims2003}
SIMS, CHRISTOPHER (2003): \textquotedblleft Implications of rational inattention \textquotedblright, \emph{Journal of monetary Economics}, 50 (2), 665-690.

\bibitem[Stratonovich(1965)]{Stratonovich1965}
STRATONOVICH, RUSLAN (1965): \textquotedblleft On the Value of Information \textquotedblright, \emph{Izv. USSR Acad. Sci. Tech. Cybern.}, 5, 2-12.

\bibitem[Stratonovich(2020)]{Stratonovich2020}
STRATONOVICH, RUSLAN (2020): \emph{Theory of Information and its Value} ed. by R.V. Belavkin, P.M. Pardalos, J.C. Principe, original 1975. 

\bibitem[Tikhomirov(2011)]{Tikhomirov2011}
TIKHOMIROV, VLADIMIR (2011): \textquotedblleft Analysis II\textquotedblright, \emph{Convex Analysis and Approximation Theory, Encyclopaedia of Mathematical Sciences}, 14, Theory. ed. by R.V. Gamkrelidze. Springer, original 1990. 

\bibitem[Zelikin et al.(2017))]{Zelikin2017}
ZELIKIN, MICHAIL, LEV LOKUTSIEVSKIY, AND SERGEY SKOPINCEV (2017): \textquotedblleft On optimal harvesting of a resource on a circle\textquotedblright, \emph{Matematicheskie Zametki}, Vol. 102, 4, 565-578.


% \bibitem{b1}
\end{thebibliography}

\end{document}
