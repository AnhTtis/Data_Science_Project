%%%%%%%%%%%%%%%%%%%%%%%%%%%%%%%%%%%%%%%%%%%%%%%%%%%%%%%%%%%%%%%%%%%%%%%%%%%%%%%%
%2345678901234567890123456789012345678901234567890123456789012345678901234567890
%        1         2         3         4         5         6         7         8

\documentclass[letterpaper, 10 pt, conference]{ieeeconf}  % Comment this line out if you need a4paper
% \documentclass[letterpaper, 10 pt, conference]{IEEEtran}

% \documentclass[a4paper, 10pt, conference]{ieeeconf}      % Use this line for a4 paper

\IEEEoverridecommandlockouts                              % This command is only needed if 
                                                          % you want to use the \thanks command

% \overrideIEEEmargins   
% Needed to meet printer requirements.

%\usepackage[numbers]{natbib}
% Allows citations like [1]--[3] instead of [1],[2],[3]
\usepackage[noadjust]{cite}
\def\citepunct{,}
\def\citedash{--}




% The following packages can be found on http:\\www.ctan.org
\usepackage{graphicx} % for pdf, bitmapped graphics files
%\usepackage{epsfig} % for postscript graphics files
%\usepackage{mathptmx} % assumes new font selection scheme installed
%\usepackage{times} % assumes new font selection scheme installed
\usepackage{amsmath} % assumes amsmath package installed
\usepackage{amssymb}  % assumes amsmath package installed
\usepackage{float}
\usepackage{hyperref}
\usepackage{color}
\usepackage{algorithm}
\usepackage{algpseudocode}
\usepackage{subcaption}
\usepackage{xfrac}


\usepackage{booktabs, multirow} % for borders and merged ranges
\usepackage{soul}% for underlines
\let\labelindent\relax
\usepackage{enumitem}

\usepackage[table]{xcolor} % for cell colors
\usepackage{changepage,threeparttable} % for wide tables
%If the table is too wide, replace \begin{table}[!htp]...\end{table} with
%\begin{adjustwidth}{-2.5 cm}{-2.5 cm}\centering\begin{threeparttable}[!htb]...\end{threeparttable}\end{adjustwidth}
% \pagenumbering{arabic}
\definecolor{international_orange}{RGB}{240, 74, 0}
\newcommand{\IH}[1]{
 {\textcolor{magenta}{#1}}}

\newcommand{\YN}[1]{
 {\textcolor{blue}{\textbf{[#1 --YN]}}}}
 
\newcommand{\CE}[1]{
 {\textcolor{YellowOrange}{\textbf{[#1 --CE]}}}}
 
\newcommand{\BS}[1]{
 {\textcolor{green}{\textbf{[#1 --BS]}}}}

\newcommand{\tucker}[1]{
 {\textcolor{international_orange}{\textbf{[#1 --TRH]}}}}

\newcommand{\NB}[1]{
 {\textcolor{magenta}{#1}}}
 
 \newcommand{\MM}[1]{
 {\textcolor{BlueViolet}{\textbf{[#1 --MM]}}}}
 
  \newcommand{\DF}[1]{
 {\textcolor{red}{\textbf{[#1 --DF]}}}}
 
  \newcommand{\FR}[1]{
 {\textcolor{cyan}{\textbf{[#1 --FR]}}}}
 
 
\newcommand{\units}[1]{\text{#1}}
 
\title{\LARGE \bf
DefGraspNets: Grasp Planning on 3D Fields with Graph Neural Nets 
}


\author{Isabella Huang$^{1}$, Yashraj Narang$^{2}$, Ruzena Bajcsy$^{1}$, Fabio Ramos$^{2,3}$, Tucker Hermans$^{2,4}$, Dieter Fox$^{2,5}$% <-this % stops a space
\thanks{$^{1}$Department of Electrical Engineering and Computer Sciences, University of California, Berkeley, USA ;$^{2}$NVIDIA Corporation, Seattle, USA;$^{3}$School of Computer Science, University of Sydney, Sydney, Australia;$^{4}$School of Computing, University of Utah, Salt Lake City, USA;$^{5}$Paul G. Allen School of Computer Science \& Engineering, University of Washington, Seattle, USA}}
\frenchspacing
\usepackage[skip=1pt,font=small,labelfont=bf]{caption}
\begin{document}



\maketitle
\thispagestyle{empty} 
\pagestyle{empty} 



%%%%%%%%%%%%%%%%%%%%%%%%%%%%%%%%%%%%%%%%%%%%%%%%%%%%%%%%%%%%%%%%%%%%%%%%%%%%%%%%
\begin{abstract}
  Robotic grasping of 3D deformable objects is critical for real-world applications such as food handling and robotic surgery. Unlike rigid and articulated objects, 3D deformable objects have infinite degrees of freedom. Fully defining their state requires 3D  deformation and stress fields, which are exceptionally difficult to analytically compute or experimentally measure. Thus, evaluating grasp candidates for grasp planning typically requires accurate, but slow 3D finite element method (FEM) simulation. Sampling-based grasp planning is often impractical, as it requires evaluation of a large number of grasp candidates. Gradient-based grasp planning can be more efficient, but requires a differentiable model to synthesize optimal grasps from initial candidates.
  Differentiable FEM simulators may fill this role, but are typically no faster than standard FEM. In this work, we propose learning a predictive graph neural network (GNN), DefGraspNets, to act as our differentiable model.
  We train DefGraspNets to predict 3D stress and deformation fields based on FEM-based grasp simulations.
  DefGraspNets not only runs up to $1500$x faster than the FEM simulator, 
  but also enables fast gradient-based grasp optimization over 3D stress and deformation metrics. We design DefGraspNets to align with real-world grasp planning practices and demonstrate generalization across multiple test sets, including real-world experiments.
\end{abstract}


%%%%%%%%%%%%%%%%%%%%%%%%%%%%%%%%%%%%%%%%%%%%%%%%%%%%%%%%%%%%%%%%%%%%%%%%%%%%%%%%
% \begin{figure}[t]
%     % \begin{subfigure}{1\linewidth}
%     %   \centering
%     % %   \includegraphics[width=1\linewidth]{figs/fig_1_moti_textattn.pdf}  
%     % %   \includegraphics[width=1\linewidth]{figs/fig_1_moti_textattn_v2.pdf}  
%     %   \includegraphics[width=1\linewidth]{figs/fig_1_moti_textattn_v5.pdf}  
%     %   \vspace{-0.5cm}
%     %     \caption{Amount of attention added to each video clip from the source video and query text in the self-attention layers of Moment-DETR encoder.}
%     %     % \caption{Distribution of attention for source and query in Moment-DETR encoder}
%     %     % Visualization of video clip's self-attention score in Moment-DETR encoder.
%     %   \label{fig:fig1_text_attn_ex}
%     % \end{subfigure}%\hfill% or  or \hspace{0.3\textwidth}
%     \vspace{0.2cm}
%     % \begin{subfigure}{1\linewidth}
%       \centering
%     %   \includegraphics[width=1\linewidth]{figs/fig1_moti_negattn.pdf}  
%       \includegraphics[width=1\linewidth]{figs/fig1_moti_negattn_v3.pdf}  
%       \vspace{-0.4cm}
%     %   \caption{Correspondence of saliency scores on the relevance between video clips and the text query.}
%     % \caption{Predicted saliency scores against the video relevant positive query and video irrelevant negative query}
%       \label{fig:fig1_neg_attn_ex}
%     % \end{subfigure}%\hfill% or  or \hspace{0.3\textwidth}
%     \caption{
%     % 원준 원본
%     % (a) Comparison between attention scores of source and query for each video clip~(We sum the attention scores from video and text). 
%     % We observe that the attention scores are dominated by other clips in the source video. 
%     % Text queries do not account for much attention regardless of the relevance to the video clips.
%     % \textbf{(a)} Inspection of the query dependency in Moment-DETR encoder.
%     % % We visualize the attention score of video tokens in the transformer encoder and observe that text query accounts for only a low portion of attention.
%     % % This tendency occurs regardless of the relevance between the text query and video clips. 
%     % We visualize the attention score of video tokens in the transformer encoder and observe 1) text query only accounts for a low portion of attention, and 2) relevance between video-query pair does not affect the attention scores ratio of text.
%     \textbf{(b)} Comparison of highlight-ness when relevant and non-relevant queries are input.
%     As observed in , existing work only uses queries to play an insignificant role, thereby may not be capable of detecting false queries and considering the video-query relevance even when the problem in (a) is resolved. 
%     % \SE{} % 이 부분이 "not capable of" 란 용어가 세다는 피드백이 있는 듯 합니다. 이러한 능력이 없다는 것은 굉장히 강한 어조인거 같기는 하고, 이러한 경우들이 종종 있다거나 좀 약화시킬 필요가 있어보이긴 하네요.
%     On the other hand, our QD-DETR yields a query-dependent representation that the relevance between the source video and query text is updated in the saliency scores.
%     There is a large gap between positive and negative saliency scores, and scores are consistent since the clips are all highly correlated to others.
%     }
%     \label{fig:motivation_ex}
%     % \captionsetup{belowskip=13pt}
%     % \setlength{\belowcaptionskip}{-10pt}
% \end{figure}
\begin{figure}
    \centering
    \includegraphics[width=1\linewidth]{figs/fig1_moti_negattn_1111.pdf}
    % \includegraphics[width=1\linewidth]{figs/fig1_moti_negattn_1109.pdf}
    % \includegraphics[width=1\linewidth]{figs/fig1_moti_negattn_stat.pdf}
    \vspace{-0.6cm}
    \caption{
        % \SE{} % 수정 필요
        Comparison of highlight-ness~(saliency score) when relevant and non-relevant queries are given.
        We found that the existing work only uses queries to play an insignificant role, thereby may not be capable of detecting negative queries and video-query relevance; saliency scores for clips in ground-truth~(GT) moments are low and equivalent for positive and negative queries.
        % This also results in mispredicted moments when ground-truth~(GT) moment is dominated by clips unrelated to GT since their prediction is highly focused on the video.
        % \SE{} % 여기 한번 더 보면 좋을 듯 합니다. GT moment에 unrelated한 clip이 많으면? label이 틀렷을 경우를 말씀하시는건지?
        % As observed in saliency graph, existing work only uses queries to play an insignificant role, thereby may not be capable of detecting false queries and considering the video-query relevance.
        On the other hand, query-dependent representations of QD-DETR result in corresponding saliency scores to the video-query relevance and precisely localized moments.
        % On the other hand, our QD-DETR yields a query-dependent representation that the
        % saliency scores are in accordance with the relevance between the video and query.
        % text is in accordance with the saliency scores.
        % There is a large gap between positive and negative saliency scores, and scores are consistent since the clips are all highly correlated to others.
}
    \label{fig:motivation_ex}
\end{figure}


\section{Introduction}
% 원준 원본
% Along with the advance of digital devices and platforms, video is now one of the most desired data type for consumers. However, although the large information capacity of videos may be beneficial in many aspects, e.g., informative and entertaining, on the contrary perspective, videos are time-consuming, and hard to search for desirable moments. 
% This has led many creators to use extra manpower to crop and edit the video to generate highlight clips to gain the consumer’s attention.
Along with the advance of digital devices and platforms, video is now one of the most desired data types for consumers~\cite{apostolidis2021video,wu2017deep}.
% SE: Video aware deep learning application & survey papers?
Although the large information capacity of videos might be beneficial in many aspects, e.g., informative and entertaining, inspecting the videos is time-consuming, so that it is hard to capture the desired moments~\cite{anne2017localizing,apostolidis2021video}. 
% This has led many creators to use extra manpower to crop and edit the video to generate highlight clips to gain the consumer’s attention.


% On the other side, 
Indeed, the need to retrieve user-requested or highlight moments within videos is greatly raised.
Numerous research efforts were put into the search for the requested moments in the video~\cite{anne2017localizing, gao2017tall, liu2015multi, escorcia2019temporal} and summarizing the video highlights~\cite{zhang2016video, mahasseni2017unsupervised, badamdorj2022contrastive, wei2022learning}.
% Numerous research efforts were put into the search for the requested moments in the video~\cite{anne2017localizing, gao2017tall, liu2015multi, escorcia2019temporal}, summarizing the video to generate highlights was another popular topic~\cite{zhang2016video, mahasseni2017unsupervised, badamdorj2022contrastive, wei2022learning}.
Recently, Moment-DETR~\cite{momentdetr} further spotlighted the topic by proposing a QVHighlights dataset that enables the model to perform both tasks, retrieving the moments with their highlight-ness, simultaneously.

% 원준 원본
% To detect the desired moments, previous works employed transformer encoder-decoder architectural designs to fuse the text query into the video representations. Moment-DETR~\cite{mDETR} modified detection transformer to process capture the moment as a set, and UMT~\cite{umt} implemented transformer decoder as to output clip-wise saliency. 
% Yet to their outstanding breakthroughs in the literature of moment retrieval with the seminal architectures, their limitation is that the role of the given text query is insignificant in representing the query-conditioned video representation; the attention mechanism of moment DETR is not explicitly conditioned on the text query, and the text query is conditioned on multi-modal clips where the differences between the clips are smoothed after encoding process in UMT.



% \begin{figure}[t]
% \centering
%     \begin{subfigure}[l]{0.37\linewidth}
%       \centering
%       \vspace{0.20cm}
%     %   \includegraphics[width=1\linewidth]{figs/fig_1_moti_textattn.pdf}  
%     %   \includegraphics[width=1\linewidth]{figs/fig_1_moti_textattn_v2.pdf}  
%       \includegraphics[width=1\linewidth]{figs/fig1_moti_violin_a.pdf}  
%       \vspace{-0.60cm}
%     %   \caption{text attention}
%         \caption{Importance of queries in video representation}
%       \label{fig:fig1_text_attn}
%     \end{subfigure}%\hfill% or  or \hspace{0.3\textwidth}
%     \vspace{0.2cm}
%     \begin{subfigure}[r]{0.61\linewidth}
%       \centering
%     %   \includegraphics[width=1\linewidth]{figs/fig1_moti_negattn.pdf}  
%       \includegraphics[width=1\linewidth]{figs/fig1_moti_violin_b.pdf}  
%     %   \caption{neg attention}
%         % \caption{Relation between the highlight-ness and the relevance between videos and query texts.}
%         \caption{Highlight-ness~(saliency) histogram of positive and negative video-query pairs\SE{}}
%       \label{fig:fig1_neg_attn}
%     \end{subfigure}%\hfill% or  or \hspace{0.3\textwidth}
%     % \vspace{-0.2cm}
%     \caption{Overall statistics for attention scores in Fig.~\ref{fig:motivation_ex} in QVHighlights dataset. 
%     (a) For the attention scores that measure how much the text query is generally involved in video representation, we use violin plots to show the probability density. We plot the score for each layer in the encoder.
%     % (b) Using the histogram, we compare how the baseline and QD-DETR yield different salient scores given the positive and negative video-text pairs.
%     (b) Saliency histogram shows the distributional gap between positive and negative video-text query pairs of baseline~(Moment-DETR) and proposed QD-DETR.\SE{}
%     }
%     \label{fig:motivation}
%     % \captionsetup{belowskip=13pt}
%     % \setlength{\belowcaptionskip}{-10pt}
% \end{figure}

% \begin{figure}[t]
% \centering

%     \begin{subfigure}[r]{1\linewidth}
%       \centering
%       \hspace{-0.2cm}
%     %   \includegraphics[width=1\linewidth]{figs/fig1_moti_negattn.pdf}  
%       \includegraphics[width=1.1\linewidth]{figs/fig1_moti_violin_a_v2.pdf}  
%     %   \caption{neg attention}
%         % \caption{Relation between the highlight-ness and the relevance between videos and query texts.}
%         \vspace{-0.5cm}
%         % \caption{Saliency histogram of positive and negative video-query pairs}
%         \caption{We plot the histograms and its average value~(dotted line) to compare saliency scores when true and false text queries are given for each method. (left) Since the video representations do not include much textual information, both the true and false queries yield similar saliency scores. (Middle) Even when the video representation is enforced to be updated with the textual information, the issue is not much resolved. (Right) By extracting discriminative features in the text query, distributions are differentiated.
%         % \SE{} % R1@0.5 설명
%         Also, R1@0.5 indicates evaluation metric, Recall at 1 with IoU 0.5 threshold on QVhighlight \textit{val} set.
%         }
%       \label{fig:fig1_neg_attn}
%     \end{subfigure}%\hfill% or  or \hspace{0.3\textwidth}
%     \\
%     \begin{tabular}{cc}
%     \hspace{-0.2cm}
%         \begin{minipage}{.4\linewidth}
%             \begin{subfigure}[l]{1\linewidth}
%               \centering
%             %   \vspace{0.20cm}
%             %   \includegraphics[width=1\linewidth]{figs/fig_1_moti_textattn.pdf}  
%             %   \includegraphics[width=1\linewidth]{figs/fig_1_moti_textattn_v2.pdf}  
%               \includegraphics[width=1\linewidth]{figs/fig1_moti_violin_a.pdf}  
%               \vspace{-0.60cm}
%             %   \caption{text attention}
%                 \caption{Importance of queries in video representation}
%               \label{fig:fig1_text_attn}
%             \end{subfigure}%\hfill% or  or \hspace{0.3\textwidth}
%         \end{minipage}
        
%         \begin{minipage}{.6\linewidth}
%             \vspace{-0.2cm}
%             \caption{Overall statistics of Fig.~\ref{fig:motivation_ex} in QVHighlights dataset. 
%             (a) Saliency histogram shows the distributional gap between positive and negative video-text query pairs.
%             % (a) For the attention scores that measure how much the text query is generally involved in video representation, we use violin plots to show the probability density. We plot the score for each layer in the encoder.
%             % (b) Using the histogram, we compare how the baseline and QD-DETR yield different salient scores given the positive and negative video-text pairs.
%             % (b) Text ratio in self-attention layer to  of Moment-DETR
%             % (b) Ratio of text when representing video tokens in self-attention of Moment-DETR.
%             % (b) Magnitude of attention text query involved.
%             % (b) Attention score of video tokens
%             % (b) Magnitude of text query to refine the video tokens in self-attention layer of Moment-DETR.
%             (b) Probability density depicting the weight of the text query in attention score for video clips. Scores are from the self-attention layers in Moment-DETR encoder.
%             % (b) The text query ratio in attention score of video clips (Self-attention layer in Moment-DETR encoder). We use violin plots to show probability density.
%             % 텍스트 쿼리가, 비디오 피쳐에 얼만큼 attend 하는지
%             }
%         \end{minipage}
    
%     \end{tabular}
%     \vspace{-0.5cm}
%     \label{fig:moti}
%     % \captionsetup{belowskip=13pt}
%     % \setlength{\belowcaptionskip}{-10pt}
% \end{figure}


% \begin{figure}
%     \centering
%     % \includegraphics[width=1\linewidth]{figs/fig1_moti_negattn_1109.pdf}
%     \includegraphics[width=1\linewidth]{figs/fig1_moti_negattn_stat_v2.pdf}
%     \vspace{-0.8cm}
%     \caption{
%         Histogram of saliency when the positive and negative queries are given. We plot the histograms and its average value~(dotted line) to compare saliency scores when relevant~(positive) and irrelevant~(negative) text queries are given for each method. (Left) Since the video representations do not properly reflect textual information, both the positive and negative queries yield similar saliency scores. 
%         % (Middle) Even when the video representation is enforced to be updated with the textual information, the issue is not much resolved. 
%         (Right) By representing video clips in query-dependent manner, distributions are differentiated.
%     }
%     \vspace{-0.6cm}
%     \label{fig:motivation}
% \end{figure}


% One of the demanding task is moment retrieval task, which is detecting the desired moments from the given query, typically the text query.
When describing the moment, one of the most favored types of query is the natural language sentence~(text)\cite{anne2017localizing}. 
While early methods utilized convolution networks~\cite{zhang2020learning, gao2021fast, wang2020temporally}, recent approaches have shown that deploying the attention mechanism of transformer architecture is more effective to fuse the text query into the video representation.
% To handle these modalities, previous works simply employed the attention mechanism of transformer architecture to fuse the text query into the video representation.
For example, Moment-DETR~\cite{momentdetr} introduced the transformer architecture which processes both text and video tokens as input by modifying the detection transformer~(DETR), and UMT~\cite{umt} proposed transformer architectures to take multi-modal sources, e.g., video and audio. 
Also, they utilized the text queries in the transformer decoder.
Although they brought breakthroughs in the field of MR/HD with seminal architectures, they overlooked the role of the text query.
To validate our claim, we investigate the Moment-DETR~\cite{momentdetr} in terms of the impact of text query in MR/HD~(Fig.\ref{fig:motivation_ex}).
Given the video clips with a relevant positive query and an irrelevant negative query, we observe that the baseline often neglects the given text query when estimating the query-relevance scores, i.e., saliency scores, for each video clip.
% the output saliency score, i.e. query-relevance scores.
% Based on the observation, we traced the actual saliency prediction of the model against both the video-relevant query and the irrelevant dummy one where we find that the baseline often neglects the given text query when estimating the query-relevance scores of video clips.
% For example, in Fig.~\ref{fig:motivation_ex}, saliency scores are not affected even when the query is substituted with the dummy.
% % General statistics for Fig.~\ref{fig:motivation_ex} is shown in Fig.~\ref{fig:motivation}. 
% General statistics corresponding to Fig.~\ref{fig:motivation_ex} are also shown in Fig.~\ref{fig:motivation}.



% The limitation of the concrete baseline~\cite{momentdetr} is inspected in two different aspects; 1) Utilization of text-query in the encoding process and 2) the output saliency score, i.e. query-relevance scores.
% Firstly, we visualize the attention score when video clips are given as a query in self-attention. 
% We observe that the text queries have relatively small impacts compared to other video features, as shown in Fig.~\ref{fig:fig1_text_attn_ex}.
% That is, the text does not account for much in representing every video clip, although the goal of MR/HD is to detect query-relevant moments.
% Based on the observation, we traced the actual saliency prediction of the model against both the video-relevant query and the irrelevant dummy one where we find that the baseline often neglects the given text query when estimating the query-relevance scores of video clips.
% For example, in Fig.~\ref{fig:motivation_ex}, saliency scores are not affected even when the query is substituted with the dummy.
% % General statistics for Fig.~\ref{fig:motivation_ex} is shown in Fig.~\ref{fig:motivation}. 
% General statistics are also shown in Fig.~\ref{fig:motivation}.

% Consequently, in Fig.~\ref{fig:fig1_neg_attn_ex}~(b), we found that the baseline often neglects the given text query when estimating the query-relevance scores of video clips; 
% For example, 


% We validate the previous work sometimes neglects the given query when estimating the saliency of video clips.
% For example, there is an example that the saliency scores from positive and negative queries cannot be distinguishable, as shown in Fig.~\ref{fig:fig1_neg_attn_ex}.
% % 우리는 추가로 text attention을 추가도 해봤지만, 효과가 있긴 했으나, still 이슈가 있는 것을 확인하였다?
% % Still, we observe that assuring the high attendance of text queries does not resolve the overlap which motivates us to question the quality of the naive use of task-agnostic text representation~\cite{momentdetr, umt}.
% We found that introducing the text-attention for ensuring the high attendance of text queries relieve the overlap, but there still be a severe overlap.


% To validate their limitations, we inspect the impacts of text queries in the concrete baseline~\cite{momentdetr} with the two different aspects, 1) tendency of attention in self-attention layer and 2) saliency score, i.e. query-relevance scores. \SE{} % attention 이 갑자기 등장하는가?
% Firstly, we visualize the attention score when video clips are given as a query in self-attention. We observe the text queries have relatively low attention scores compared to the video features, as shown in Fig.~\ref{fig:fig1_text_attn_ex}.
% That is, the text does not account for much in representing every video clip, although the goal of MR/HD is to detect query-relevant moments.
% Based on this observation, we trace the actual saliency prediction of the model against both positive and negative text queries.
% We validate the previous work sometimes neglects the given query when estimating the saliency of video clips.
% For example, there is an example that the saliency scores from positive and negative queries cannot be distinguishable, as shown in Fig.~\ref{fig:fig1_neg_attn_ex}.
% % 우리는 추가로 text attention을 추가도 해봤지만, 효과가 있긴 했으나, still 이슈가 있는 것을 확인하였다?
% % Still, we observe that assuring the high attendance of text queries does not resolve the overlap which motivates us to question the quality of the naive use of task-agnostic text representation~\cite{momentdetr, umt}.
% We found that introducing the text-attention for ensuring the high attendance of text queries relieve the overlap, but there still be a severe overlap.



% Thus, we 
% query dependency를 높이기 위해 
% Cross-attention? text-attention? detailed explanation on text-attention should be needed?
% By handling these two issues, we find that more precise retrieval can be achieved.
% 
% 
%
% By projecting video-discriminative text features with high text attendance to source video, we f 
% We also find the need to improve the quality of query features since assuring high text attendance also results in...
% pairs are not finetuned to be discriminative that even the similarity within the pairs does not reflect the relevance between the query and the video clips.
% General statistics for Fig.~\ref{fig:motivation_ex} is shown in Fig.~\ref{fig:motivation}. 
% \SE{} % 이거 ??로 뜨는데, 위처럼 figure 그리면 label이 안되는걸까요
% \SE{}
% 형님 아래 사항 생각 좀 해보는게 좋을 거 같아요.
% fig 1. (a) 그림만 봤을 때 모든 clip에 대해 text attention이 일정이상 존재하긴 하니까, 뭔가 not assured to be conditioned가 와닿지 않는거 같아요.
% + 왜 text가 항상 attend 해야하나?
% not assured to be conditioned --> text shows relatively low affects compared to video 같이 실제 나타난 현상까지 같이 적으면 어떨까 싶어요.
% fig 1. (b) 덜 반영한다?

% \SU{}
% 일단 text가 attend 잘 되어야 한다는 것에 좀 궁금점이 생깁니다. 결국에는 text와 관련있는 frame들을 attend해서 higlight를 찾아야 하는게 아닐까요? 그리고, 현제 저희의 모델 구조상 text query가 Key와 Value로 거의 활용되고 있는데 그렇다면 결국에는 해당 모델은 text에 대한 attention이 전혀 없다고 봐도 무방하지 않을까요? 그런 면에서 text attention을 강조하는게 좀 걸리긴 합니다.

% Specifically, the text query is not assured to be explicitly conditioned on every clip of the video, and as the query texts are evenly treated, discriminative keywords may not be spotlighted.
% attention mechanism of Moment-DETR is not explicitly conditioned on the text query as shown in Fig~\ref{}(d), and in UMT, the text are only used for conditioning the queries while the video representation are refined itself by self-attention.

% \begin{figure}[t]
%     \begin{subfigure}{1\linewidth}
%       \centering
%     %   \includegraphics[width=1\linewidth]{figs/fig_1_moti_textattn.pdf}  
%     %   \includegraphics[width=1\linewidth]{figs/fig_1_moti_textattn_v2.pdf}  
%       \includegraphics[width=1\linewidth]{figs/fig_1_moti_textattn_v4.pdf}  
%       \vspace{-0.5cm}
%     %   \caption{text attention}
%         \caption{Distribution of attention scores in Moment-DETR encoder}
%       \label{fig:fig1_text_attn}
%     \end{subfigure}%\hfill% or  or \hspace{0.3\textwidth}
%     \vspace{0.2cm}
%     \begin{subfigure}{1\linewidth}
%       \centering
%     %   \includegraphics[width=1\linewidth]{figs/fig1_moti_negattn.pdf}  
%       \includegraphics[width=1\linewidth]{figs/fig1_moti_negattn_v2.pdf}  
%       \vspace{-0.5cm}
%     %   \caption{neg attention}
%         \caption{Saliency score against positive and negative text queries}
%       \label{fig:fig1_neg_attn}
%     \end{subfigure}%\hfill% or  or \hspace{0.3\textwidth}
%     \vspace{0.2cm}
%     \begin{subfigure}{1\linewidth}
%       \centering
%     %   \includegraphics[width=1\linewidth]{figs/fig1_moti_violin.pdf}  
%       \includegraphics[width=1\linewidth]{figs/fig1_moti_violin_v2.pdf}  
%       \vspace{-0.5cm}
%       \caption{violin}
%       \label{fig:fig1_violin}
%     \end{subfigure}%\hfill% or  or \hspace{0.3\textwidth}
%     \vspace{-0.2cm}
%     \caption{(a) 1. portion of text attention vs. video attention 2. relation with text query and content (e.g. fg, bg) of clip seems not to affect the attention score
%     (b) 1. high variability even though entire clips are highly correlated with the given text query 2. positive and negative query makes overlaps on saliency score distribution
%     (3) actual distribution on validation dataset.}
%     \label{fig:motivation}
%     % \captionsetup{belowskip=13pt}
%     % \setlength{\belowcaptionskip}{-10pt}
% \end{figure}

To this end, we propose Query-Dependent DETR~(QD-DETR) that produces query-dependent video representation.
% Our key focus is to ensure each clip in predicted moments is explicitly conditioned by the query, particularly on the video-descriptive portion of the text query.
% Our key focus is to ensure that query-relevant clips are predicted by enforcing each clip to be explicitly conditioned by the query.
%Our key focus is to ensure that the model prediction for each clip is highly relevant to the query.
Our key focus is to ensure that the model's prediction for each clip is highly dependent on the query.
% by enforcing each clip to be explicitly conditioned by the query. :)
% hmm...
% \SE {} % "query-relevant clips are predicted" 이 문장이 좀 애매한거 같습니다. relevant 클립을 놓지지 않고 찾는 것을 보장한다? 이런 느낌인지 아니면 높은 saliency 를 주는게 목적이다? model prediction이 query-relevance를 반영하는 것을 보장한다?
% Our key focus is to ensure that the model prediction reflects query-relevance of clips by enforcing each clip to be explicitly conditioned by the query.
First, to fully utilize the contextual information in the query, we revise the transformer encoder to be equipped with cross-attention layers at the very first layers.
% 상익's thought :  single video - query간의 관계만 고려 - 같은 word가 더 많이 쓰이는 것을 보고 
% 교수님's thought : neg pair 를 쓰면 쿼리를 보지 않고서는 video clip간만 고려하는 것이 사라짐. 왜냐면 0으로 내보내야 하기 때문. --> SE: relative difference 만 고려하다가, 
By inserting a video as the query and a text as the key and value of the cross-attention layers, our encoder enforces the engagement of the text query in extracting video representation.
% 원준 교수님 코멘트 반영해서 다시
Then, in order to not only inject a lot of textual information into the video feature but also make it fully exploited, we leverage the negative video-query pairs generated by mixing the original pairs.
Specifically, the model is learned to suppress the saliency scores of such  negative~(irrelevant) pairs.
Our expectation is the increased contribution of the text query in prediction since the videos will be sometimes required to yield high saliency scores and sometimes low ones depending on whether the text query is relevant or not.
% \SE{}
% learns to?
% By suppressing the saliency scores of the irrelevant video-query pairs, the model learns to spotlight only the video-specific discriminative words in the query.
% % \SE{} % ====================== 상익 수정 ========================
% However, this architectural design still lacks the capability of identifying the video-descriptive keywords in the query.
% % However, this architectural design still lacks in identifying proper query relevance.
% This is because the current training scheme only focuses on the interactions of video and clips within a single video while neglecting information shared throughout the entire video.
% % We argue the problem of the current training scheme that only focuses on distinguishing the clips in a single video while neglecting information shared throughout the entire video.
% Therefore, we leverage the negative video-query relationships to enhance the capability of identifying the contextual similarity of query and video clips.
% 
% 원준 원본 
% However, this architectural design heavily relies on the quality of the text query.
% Therefore, we leverage the negative video-query relationships to enable the model to emphasize key corresponding query features.
% By suppressing the saliency scores of the irrelevant video-query pairs, the model learns to spotlight only the video-specific discriminative words in the query.
% =========================================================
Lastly, to apply the dynamic criterion to mark highlights for each instance, we deploy a saliency token to represent the entire video and utilize it as an input-adaptive saliency criterion. 
With all components combined, our QD-DETR produces query-dependent video representation by integrating source and query modalities.
This further allows the use of positional queries~\cite{dabdetr} in the transformer decoder.
% Furthermore, we can exploit the advanced DETR decoder architectures using the positional information, e.g., DAB-DETR, since our encoded tokens consist of identical position representations from a single modality.
% \SE{} % ====================== 상익 수정 ========================
% Furthermore, we can exploit the advanced DETR decoder architectures using the positional information, e.g., DAB-DETR, since our video clip tokens consist of identical position representations from a single modality.
% 원준 원본
% It also enables the use of advanced DETR decoder architectures, e.g., DAB-DETR, for the first time, as these works exploit the position information within a single modality.
% =========================================================
Overall, our superior performances over the existing approaches validate the significance of the role of text query for MR/HD.
% Our extensive experiments on QVHighlights, TVSum, and Charades-STA datasets validate the significance of considering the role and the quality of text query.

% All components combined with dynamic anchor moments for the query of decoder, our FOQUE fosters the query-dependent video representation, thereby making the 
% All components combined, our modified transformer encoding process fosters the query-dependent video representation thereby achieving the state-of-the-art results on various benchmarks of moment-retrieval and highlight detection.
	
% -	Video Platform & Streamer & Consumer의 증가. 
% Video는 다른 데이터 타입보다 정보가 많아 유용하지만, 이는 다른 말로 해석하면 video를 보는 것은 time-consuming 하고, 원하는 것을 찾아보기에는 힘들 수 있음.
% 따라서, 많은 매체에서는 사람들의 더 많은 이목을 끌기 위해 highlight 비디오라는 것을 편집하여 공유도 함.
% 하지만, highlight video를 만들기 위해 사람의 노력이 필요한 현 시점에서, This spotlights the need to retrieve the user-requested / Highlight moments in the video.

% -	이전에도 이러한 문제를 해결하기 위해 (asdfasdf) for moment retrieval, (asdfasdf) for highlight detection 등이 제안 되었지만, 이들은 비디오의 특정 영역을 찾는다는 공통된 목적을 가지고 있으면서도, 데이터 셋의 한계로 인해 따로 연구되었음. 이를 문제 삼으며, 최근에는 두 task를 동시에 학습할 수 있는 dataset이 소개 되었는데, 컴퓨터비전에서 최근 각광을 받고 있는 Transformer 모델 도입과 함께 큰 발전을 거듭하고 있음.

% -	구체적으로, 이 두가지 task를 수행하기 위해서는 transformer를 두가지 방법으로 이용할 수 있는데, moment-DETR 처럼 moment 를 clip의 set 단위로 예측할 수 있고, UMT 처럼 clip-wise prediction을 할 수 있음. 하지만, 이들은 query를 condition이 아닌 video와 동등한 레벨로 취급하거나 [mDETR], 매 클립이 self-attention으로 mixing 된 후에 condition을 걸어주어 clip간의 차이를 확실하지 이용하지 못하였고, 또한, 확실하게 condition으로 주지 못하였고, video와 query 사이의 관계를 한정적으로만 이용하였다.

% -	따라서, we explore three different ways to fully exploit query information. First, we design one-way cross-attention layer to condition every clip with the query features. Then, we utilized the negative video-text pairs to better model the relationships between the video and the text embeddings. Lastly, we define the saliency token to be the video-query dependent saliency estimator.


















% ===================== neg pair 부분 ===========================
% Nevertheless, the current training scheme, only considering the given video-query pair, still disturbs the model from identifying proper query-relevance prediction.
% In detail, the model focus on learning the fine-grained discrepancy between video clips, while neglecting the information they share, which contains significant clues to understand the context of video.
% Therefore, we leverage the negative video-query relationships to enhance the capability of identifying the contextual similarity of query and video clips.
% Therefore, we leverage the negative video-query relationships by suppressing those pairs, so that enhance the capability of identifying the contextual similarity of query and video clips.
% We hypothsize the diversity in query-video pairs are insufficient to learn the general relationship between text query and video.
% Therefore, we leverage the negative video-query relationships by suppressing the saliency scores of the irrelevant video-query pairs.
% However, this architectural design still lacks in identifying proper query relevance.
% We argue that the current training scheme only focuses on learning the fine-grained discrepancy between clips in a single video, while neglecting the information they share, which contains significant clues to understand the context of the video.
% Therefore, we leverage the negative video-query relationships to enhance the capability of identifying the contextual similarity of query and video clips.
% However, this architectural design still lacks in identifying proper query relevance.
% We argue the problem of the current training scheme that only focuses on learning the fine-grained discrepancy between clips in a single video.
% That is, the current design neglects the information shared throughout the video, although it contains significant clues to understand the context of the video.
\section{Related Work}
\label{sec:related_work}
\subsection{Co-Speech Gesture Synthesis}
The early approaches for generating co-speech gestures often involve creating linguistic rules to translate speech input into a sequence of pre-collected gesture segments, which are typically referred to as rule-based methods \cite{cassell1994rulefullbody,cassell2001beat,kipp2004gesture,kopp2006bml}. \citet{wagner2014rulereview} provide a comprehensive review of these methods. Rule-based methods produce interpretable and controllable results, but creating gesture datasets and rules requires significant effort. To alleviate the manual effort of designing rules in rule-based methods, data-driven approaches have gradually become predominant in this field. \citet{nyatsanga2023data_driven_gesture_survey} offer a thorough survey of these methods. Early data-driven approaches aim to directly learn mapping rules from data through statistical models \cite{neff2008videogesture,levine2009prosodygesture,levine2010gesturecontroller} and combine them with predefined gesture units for gesture generation. Later, the powerful modeling capability of deep neural networks makes it possible to train complex end-to-end models using raw speech-gesture data directly. One option is deterministic models, such as MLP \cite{kucherenko2020gesticulator}, CNN \cite{habibie2021videogesture}, RNN \cite{yoon2019robot,yoon2020trimodalgesture,bhattacharya2021affectivegesture,liu2022hierarchicalgesture}, and Transformer \cite{bhattacharya2021text2gestures}. Another choice is generative models, including flow-based models \cite{alexanderson2020stylegesture,ye2022styleflowgesture}, VAEs \cite{li2021audio2gesture,ghorbani2022zeroeggs}, and VQ-VAE \cite{yi2022talkshow,yazdian2022gesture2vec,liu2022vqgesturevideo}. Due to the inherent many-to-many relationship between speech and gesture, end-to-end models can generate natural-looking gestures but face challenges in ensuring content matching between speech and generated gestures \cite{yoon2022genea}. To address this issue, some neural systems aim to explicitly model both rhythm and semantics from the perspective of model structure \cite{kucherenko2021speech2properties2gestures,ao2022rhythmicgesticulator,liu2022disco} or training supervision strategy \cite{liang2022seeg}. Furthermore, hybrid systems, such as the combination of deep features and motion graphs \cite{zhou2022gesturemaster}, have been proposed to harness the advantages of different approaches. Recently, diffusion models \cite{sohldickstein2015diffusion,song2020improvedscore,ho2020ddpm} have demonstrated impressive results in image synthesis \cite{ramesh2022dalle2} and human motion generation \cite{tevet2022humanmotiondiffusion, zhang2022motiondiffuse}. Inspired by these works, our system adapts the latent diffusion model \cite{rombach2022latentdiffusion} for the co-speech gesture generation task and achieves appealing results.

\subsection{Style Control for Human Motion}
A typical approach to style control for human motion involves specifying a motion clip as a reference and transferring the reference clip's style to the source motion. This task is also known as \emph{style transfer}. Early works in motion style transfer integrate traditional machine learning techniques with manually defined features to infer motion styles \cite{hsu2005motion_style_translation,ma2010motion_style_transfer,xia2015realtime_motion_style_transfer,yumer2016spectral_motion_style_transfer}. Recently, deep learning-based methods have significantly enhanced motion quality. \citet{holden2016deepmotion} first propose a learning framework enabling motion style control through optimization in the motion manifold space. \citet{du2019stylemotioncvae} improve transfer efficiency by training a conditional VAE. \citet{mason2018few-shot_motion_style_transfer} use few-shot learning to generate stylized locomotion. \citet{aberman2020adain} employ a temporally invariant adaptive instance normalization (AdaIN) layer for target style injection, eliminating the need for paired data during training. \citet{wen2021stylemotionflow} achieve unsupervised style transfer using a flow model. \citet{jang2022motionpuzzle} introduce a method capable of controlling styles for individual body parts.

Previous co-speech gesture synthesis systems with style control can be categorized based on whether or not they require style labels. For methods needing labeled data, early works can only learn an individual style for one generator \cite{levine2010gesturecontroller,neff2008videogesture,ginosar2019stylegesture}. \citet{ahuja2022lowresource} propose a strategy that efficiently adapts the source generator to another speaker style using low-resource data. Some works learn a speaker style embedding space with labeled speaker-motion data, enabling gesture style control by sampling from this space \cite{ahuja2020stylegesture,yoon2020trimodalgesture,bhattacharya2021affectivegesture}. \citet{alexanderson2020stylegesture} aimat controlling fine-grained styles, such as gesturing speed and spatial scope, using preprocessed control signal-motion data. Their later work \cite{alexanderson2022diffusiongesture} utilizes a diffusion model for audio-driven motion synthesis, achieving label-based style control by training the model on labeled data. For methods not requiring style labels, \citet{habibie2022motionmatching} propose a motion matching framework to achieve flexible style control. Other studies achieve arbitrary style control by imitating an example given as a video \cite{liu2022hierarchicalgesture} or a motion clip \cite{ghorbani2022zeroeggs,ye2022styleflowgesture,kuriyama2022tokenizedgestures}.  In this work, we utilize a CLIP-based encoder to extract a style embedding from an arbitrary text prompt and incorporate it into the generator via an AdaIN layer, guiding the synthesis of stylized gestures. Our system supports fine-grained multimodal style prompts as opposed to label-based style control. It employs a self-supervised learning scheme and eliminates the need for labeled data. Additionally, we use an autoregressive model rather than a parallel model, making it potentially suitable for real-time applications.
\section{Pitchclass2vec model}\label{sec:model}
% 1. nlp embeddings/chord2vec problems and limitations
% 2. how to overcome them and why
% 3. our encoding
% 4. encoding implementation

Embedding approaches used in natural language processing have obvious limitations when it comes to dealing with musical content, such as musical chords.
While relying on purely syntactical representations has been show to correctly encapsulate some forms of domain knowledge \cite{anzuoni2021historical}, more advanced representations are needed to obtain accurate results when dealing with harmonic progressions \cite{madjiheurem2016chord2vec}.

There are, however, some ambiguous cases in which both vector representations might introduce wrong similarities between chords.
Let us take for instance the chords \textit{C:maj} and \textit{C:maj13}\footnote{In Harte\cite{harte2005symbolic} notation}, whose notes are respectively $\mathcal{C}_{\textit{C:maj}} = \{C, E, G\}$ and $\mathcal{C}_{\textit{C:maj13}} = \{ C, E, G, B, D, A \}$. Both chords' labels only differ by two characters, however the difference between the notes that they are composed of can't be neglected. A method exclusively based on syntactical information would wrongly represent the vectors as similar between each other.
Conversely, only relying on the notes that compose a chord results in ambiguous representations of some particular classes of chords, called \textit{enharmonic} chords.
For instance, the \textit{enharmonic} chords \textit{C:dim} and \textit{Eb:dim} share the exact same set of notes, $\mathcal{C}_{\textit{C:dim}} = \mathcal{C}_{\textit{Eb:dim}} = \{C, Eb, Gb, A\}$ but need to be represented as different chords as they serve different harmonic purposes. \textit{Chord2vec} would wrongly represents both chords as the same exact vector.


In order to overcome the aforementioned limitations, we propose an encoding which requirements can be summarised as follows:
\begin{enumerate*}
    \item it has to be based on the constituent notes of a chord, rather than its label; and
    \item it must take into account the relation between those notes instead of the notes themselves.
\end{enumerate*}

The proposed encoding is grounded on tonal music theory: each chord $c$ is composed of a set of notes $\mathcal{C} \subset \mathcal{N}$, where $\mathcal{N}$ is the set of all notes and $C$ is called the \textit{pitch class} of a chord. An important distinction is represented by the \textit{root} note, which names the chord and plays an important role in its harmonic function.
 
\medskip
\begin{figure}
\centering
\begin{subfigure}{.48\textwidth}
  \centering
  \includegraphics[width=\textwidth]{images/similar_chord_diagram.drawio.png}
  \caption{C:maj and C:maj9 chord embeddings. The final representation is computed from common elements and will hence share some aspects.}
  \label{fig:pitchclass2vec-similarlabel-embedding}
\end{subfigure}
\hfill
\begin{subfigure}{.48\textwidth}
  \centering
  \includegraphics[width=\textwidth]{images/same_dim_chord_diagram.drawio.png}
  \caption{C:dim and Eb:dim chord embeddings. Both chords are composed of the same notes but using mostly different components.}
  \label{fig:pitchclass2vec-enharmonic-embedding}
\end{subfigure}
\caption{Visual reference on pitchclass2vec embedding method.}
\end{figure}

We encode each chord as the Cartesian product $\mathcal{I}_c = \textit{root}_c \times \mathcal{C}_c$ between the \textit{root} note and the \textit{pitch class} of the chord. The vector representation $\mathbf{u}_c$ of a chord $c$ is computed as
\begin{displaymath}
  \mathbf{u}_c = \sum_{i \in \mathcal{I}_c} \mathbf{u}_i
\end{displaymath}
where $\mathbf{u}_i$ is the vector representation of the tuple $x_i \in \mathcal{I}_c$. 
See Figure \ref{fig:pitchclass2vec-enharmonic-embedding} for a visual reference on how \textit{pitchclass2vec} handles \textit{enharmonic} chords and Figure \ref{fig:pitchclass2vec-similarlabel-embedding} on how chords with common components are handled.
This formalization can be seen as an extension of the chord2vec \cite{madjiheurem2016chord2vec} method, in which the chord inner structure is taken into consideration as well.

% mixed encoding: nlp + pitchclass2vec

Nevertheless, the label of a chord has a well-defined semantic. Chords composed of the same set of notes may have different harmonic functions. For example, the chords \textit{G:min7} and \textit{Bb:6}, despite different labels contain the exact same notes: $\mathcal{C}_{\textit{G:min7}} = \mathcal{C}_{\textit{Bb:6}} = \{G, Bb, D, F\}$. 
This problem is particularly evident in datasets containing annotations made by experts, where the choice of label is the result of a meticulous analysis. 
For this reason, we have implemented two different variants of \textit{pitchclass2vec}: 
\begin{enumerate*}[label=(\roman*)]
    \item a variant combining the approach proposed by \textit{word2vec} with \textit{pitchclass2vec}; and 
    \item a variant combining \textit{fasttext} with \textit{pitchclass2vec}.
\end{enumerate*}

In order to obtain mixed embeddings we test different hybrid combinations before passing the new representation to the LSTM model:
\begin{enumerate}[label=(\roman*)]
    \item concatenating the embeddings;
    \item concatenating the embeddings and projecting the result in a $N$-dimensional vector, using a fully connected layer;
    \item projecting the embeddings in the same $N$-dimensional space by using two different fully connected layer and summing the $N$-dimensional vectors;
    \item computing a new representation of each embedding by using two separate LSTM layers and summing the resulting vectors;
    \item computing a new representation of each embedding by using two separate LSTM layers and concatenating the resulting vectors.
\end{enumerate}
None of the combination used proved to be able to outperform the others and we decided to stick to the first simpler and faster approach.

\subsection{Implementation details}
\label{sec:implementation-details}
The model is implemented using \texttt{pytorch}. We train the model on a set of $\approx 16000$ chord progressions (with a total number of over $1M$ chord instances), taken from the Chord Corpus (ChoCo) dataset \cite{deberardinis2022choco}. ChoCo is a chord dataset consistsing of more than $20000$ tracks taken from $18$ different professionally curated datasets. 
All datasets have been parsed in JAMS \cite{humphrey2014jams} format and converted in Harte Notation \cite{harte2005symbolic}.
We train the model for at most $10$ epochs on an \textit{NVIDIA RTX 3090} with batch sizes of $512$ harmonic progressions. 
We manually tune the batch size to efficiently train the model on our available resources.
For each chord we take a window of $4$ context chords as positive examples, $2$ preceding and $2$ succeeding, as it has been done in the original \emph{fasttext} implementation \cite{bojanowski2018fasttext}. Then, we sample $20$ random chords as negative examples. Even though it has been shown that windows of different sizes yields different results depending on the task they are applied to \cite{caselles2018word2vec} here we will rely on a fixed size window to better compare it to the related works.
%For each chord we take a window of $4$ context chords, $2$ preceding and $2$ succeeding, as positive examples and sample $20$ random chords as negative examples.
We subsample our corpus to obtain a more balanced one by removing some of the most frequent chords instances.  We use a factor of $t = 10^{-5}$ as suggested by \cite{mikolov2013word2vec} to allow a faster and more accurate training phase.
The model is trained using a standard training procedure where a binary cross entropy loss between a chord and its positive and negative examples is minimized using Adam optimizer, with fixed learning rate of $0.025$.
We set the embedding dimension to $10$ as the result of manual trials.

\section{Data Generation and Model Training}\label{sec:training}
We now describe our simulation-based approach to training DefGraspNets.
We design a set of 60 object primitive models as a high-level abstraction of real-world geometries grouped into geometric categories (e.g., cuboids, cylinders, ellipsoids, annuli), and instances within each category have different dimensions and aspect ratios. Our dataset also includes a set of 11 of fruits and vegetables (e.g., apples, eggplants, potatoes) based on 3D scans \cite{ybjDataset}. Tetrahedral volume meshes are generated for each deformable object using fTetWild~\cite{ftw}. Triangular surface meshes are generated for the gripper fingers using Onshape.

For each pre-contacted object mesh $M_o$, 100 grasps are generated using an antipodal sampler~\cite{EppnerISRR2019} wherein randomly-sampled surface points define gripper contact points, surface normals define grasp axes, and 4 rotations are regularly drawn about each grasp axis. These 100 grasps correspond to 100 gripper meshes $M_g$. Each grasp is evaluated using the DefGraspSim\cite{Huang2022RAL} simulation framework (built upon Isaac Gym\cite{makoviychuk2021isaac} and the FleX FEM solver\cite{macklin2019non}) with the Franka parallel-jaw gripper. DefGraspSim evaluates the stress and deformation fields of the deformable object during grasping. 

Given an object-grasp pair $(M_o, M_g)$ in DefGraspSim, the gripper applies a linearly increasing amount of force on the object until $F_g^{max} = 15$\units{N} is reached in a zero-gravity environment.\footnote{For the elastic moduli examined ($1e^4 \leq E \leq 1e^7$~\units{Pa}), $15$\units{N} was observed to induce substantial stress and deformation; gravity was ignored due to having negligible effect on stress and deformation compared to contact forces.} This  force was achieved by directly commanding DOF torque applied at the gripper joints. The values of the stress ($\vec{\sigma}$) and deformation fields ($\vec{d}$) at all object vertices are saved over 50 evenly-spaced substeps throughout the entire grasping trajectory. 
Formally, our dataset $D$ is composed of input-output pairs, each consisting of a candidate grasp pose $X_i$ and corresponding set of fields $Y_i$, that is, $D=\{X_i=(M_g, M_o, F_g), Y_i=(\vec{\sigma},\vec{d})\}_{i=1}^N$, where $0 \leq F_g \leq 15$. Dataset $D$ has $N = \#~objects \times 100 \times 50=3.55e5$ unique points. Because our network performs one-step predictions of the final state and is ideal for quasistatic interactions, all unstable grasps involving chaotic dynamics are not included in $D$. 


The values of the stress field $\vec{\sigma}$ at all object vertices are computed as follows: first, the second-order stress tensor at each tetrahedral element of $M_o$ is acquired from DefGraspSim. The stress tensors at each vertex are calculated by averaging the stress tensors at all adjacent elements. Each stress tensor is then converted to the scalar von Mises stress (i.e., the second invariant of the deviatoric stress), which is widely used to quantify whether a material has yielded \cite{timoshenko2010}. The values of the deformation field $\vec{d}$ are defined simply as the distance between the positions of the pre-contacted vertices of $M_o$ and their positions under gripper force $F_g$. 

Contact edges $E^C$ are formed based on the threshold $\epsilon = 5$mm. Our networks are trained with a decaying learning rate from $5e^{-5}$ to $1e^{-6}$ over 25 epochs and a batch size of $1$. A latent size of $128$ and $L=15$ message passing steps are used, where all MLPs have 2 hidden layers. Loss is defined as the sum of the MSE of stress and deformation over all nodes. On a single RTX 3090 GPU, the network trains at approximately 1600 steps per minute.  
%%% Local Variables:
%%% mode: latex
%%% TeX-master: "root"
%%% End:

\section{Grasp Planning}\label{sec:grasp-planning}
We demonstrate DefGraspNets as a grasp planner, where both gradient-free (i.e., evaluation of sampled grasps) and gradient-based refinement methods can be used to find an optimal grasp. We define $Q$ as the optimization objective, which is any backwards pass-differentiable measure of the predicted deformation and/or stress fields (e.g., mean deformation, smooth differentiable approximation of maximum stress implemented in modern deep learning libraries).

\subsection{Evaluation of sampled grasps}
First, DefGraspNets supports online sampling-based grasp planning. For an unseen object, forward passes of DefGraspNets can be used to evaluate $Q$ for 100 random antipodal grasps with parallel batches of size 5 in 7.3 seconds. In comparison, DefGraspSim requires approximately 3 hours to evaluate 100 grasps, which is ${\sim}1500$x slower. 

The best grasp pose is identified as $T^* = \arg\min_{T \in T_s} Q(T; M_o)$, where $T$ is a 6D rigid transformation applied to a constant initial state of the gripper $M_g^{0}$ wherein both fingers are maximally open. Any valid $M_g$ can be fully defined by $T$ and joint states $\vec{p}_g \in \mathbb{R}^2$ that determine how much each finger closes in order to contact $M_o$. These joint states $\vec{p}_g$ are calculated analytically by projecting the vertices of $M_o$ onto the gripper faces, backprojecting the vertices within each face, and computing the minimum perpendicular distance over these vertices (i.e., the minimum contact distance) per finger.


\subsection{Grasp refinement}
Unlike existing deformable object planners, DefGraspNets' differentiability enables gradient-based refinement of a grasp pose to optimize $Q$. Starting from an initial grasp pose $T_{init}$, we perform gradient updates in the direction of $\sfrac{\partial{Q}}{\partial T}$ to achieve a refined $T$ using backtracking line search \cite{NoceWrig06} and simulated annealing \cite{Laarhoven87}. With 12 refinement steps per grasp, refining 100 initial grasps requires approximately $8$ minutes. A comparable time does not exist for DefGraspSim, as it is not differentiable.


\section{Prediction Results}
We test DefGraspNets' predictions of the ranking of grasps with respect to their mean stress and deformation values by quantifying the respective Kendall's $\tau$ rank correlation coefficients ($\tau_{s}$ and $\tau_{d}$).\footnote{Kendall's $\tau$ was chosen over Spearman's $\rho$ for its comparative robustness (i.e., smaller gross error sensitivity).} We answer the following questions for 4 levels of generalization:

\begin{figure*}[ht]
     \centering
     \begin{subfigure}[b]{.3\textwidth}
         \centering
         \includegraphics[scale=0.23, trim={0cm 0cm 0cm 0cm},clip]{figs/prediction/mustard.png}
         \caption{Mustard bottle, $F_g = 12
         \units{N}$, $E = 1e7$}
         \label{fig:pred_mustard}
     \end{subfigure}
     \hfill
     \begin{subfigure}[b]{.3\textwidth}
         \centering
         \includegraphics[scale=0.23, trim={0cm 0cm 0cm 0cm},clip]{figs/prediction/strawberry.png}
         \caption{Strawberry, $F_g = 6\units{N}$, $E = 5e4$}
         \label{fig:pred_strawberry}
     \end{subfigure}
     \hfill
     \begin{subfigure}[b]{.3\textwidth}
         \centering
         \includegraphics[scale=0.23, trim={0cm 0cm 0cm 0cm},clip]{figs/prediction/sphere.png}
         \caption{Sphere, $F_g = 5\units{N}$, $E \in [5e5, 1e6, 5e6]$}
         \label{fig:pred_lemon}
     \end{subfigure}
     \hfill
        \caption{A) Predicted and ground-truth deformation fields for a mustard bottle subject to grasps inducing increasing mean deformation, B) Predicted and ground-truth stress fields for a strawberry subject to grasps inducing increasing maximum stress, and C) Predicted and ground-truth stress fields for a sphere of increasing elastic moduli subject to identical grasps (deformation can be seen in resulting shape).
        }
        \label{fig:predictions}
\end{figure*}

\begin{enumerate}[leftmargin=2ex]
    \item Can DefGraspNets rank unseen grasps when trained on other grasps on the same object? (Ans: Yes. For an 80-20 train-test split over grasps on the same object, we get an average $\tau_{s} = 0.78$ and $\tau_d = 0.66$ over $15000$ unseen $X_i$.)
    \item Can DefGraspNets generalize to unseen elastic moduli $E$ on the same object? (Ans: Yes. For a 7-3 train-test split over unique $E$ for grasps on the same object, we get an average $\tau_{s} = 0.81$ and $\tau_d = 0.72$ over $15000$ unseen $X_i$.)
    \item Can DefGraspNets generalize to unseen primitive objects within the same geometric category? (Ans: Yes. For a 5-1 train-test split over unique objects, we get an average $\tau_{s} = 0.48$ and $\tau_d = 0.54$ over $15000$ unseen $X_i$.)
    \item Can DefGraspNets generalize to unseen real-world objects? (Ans: Yes. Moreover, we generate useful predictions even when training on a small number of objects, as long as the train geometries are relevant to the test geometry as quantified by a low Chamfer distance. See Table~\ref{tab:g4}, which also reports the mean absolute error (\textit{MAE})). 
\end{enumerate}




% We present DefGraspNets' prediction results on 4 different levels of generalization. We report the mean absolute error of deformation and stress predictions for all test grasp states, as well as the average Kendall's $\tau$ coefficient over the mean stress and mean deformation. 


% First, we show how predictions on unseen grasps on the same object improve as training data grows (Table \ref{tab:g1}). Second, we demonstrate generalization to unseen elastic moduli $E$ as the variety of $E$ in the training data increases (Table \ref{tab:g2}). Third, we show that generalization is also achievable over unseen objects within the same geometric categories (Table \ref{tab:g3}). Finally, we demonstrate generalization over unseen real-world  objects that are not in the geometric categories used in the training set (Table \ref{tab:g4}). We show that although performance is best with more training data, performance can also be improved if the training data is more geometrically similar to the real-world object, as quantified by the average Chamfer distance $d_C$ between the test object and the objects in the train set. 

Full visualizations of predicted field quantities for the 4th (i.e., most challenging) generalization level is shown in Fig.~\ref{fig:predictions} on an unseen mustard bottle and unseen strawberry, as well as for the 2nd generalization level on a sphere.

% Across all examples, the regions of high stress and deformation always overlap between the predicted and ground truth cases. 






%%%%%%%%%%%%%%%%%%%%%%%%%%%%%%%%%%%%%%%%%%%%%%%%%%%%%%%%%%%%%%%%%%%%%
% \begin{table*}[!htp]\centering
% \caption{Generalization to unseen grasps on the same object}\label{tab:g1}
% \scriptsize
% \begin{tabular}{lrrrrrrrrrrrrr}\toprule
% & & & & &\multicolumn{4}{c}{\textbf{Test object name}} & & & & \\\cmidrule{6-9}
% &\multicolumn{4}{c}{\textbf{Cuboid 02}} &\multicolumn{4}{c}{\textbf{Flask 01}} &\multicolumn{4}{c}{\textbf{Apple 02}} \\\cmidrule{2-13}
% &\multicolumn{2}{c}{Deformation} &\multicolumn{2}{c}{Stress} &\multicolumn{2}{c}{Deformation} &\multicolumn{2}{c}{Stress} &\multicolumn{2}{c}{Deformation} &\multicolumn{2}{c}{Stress} \\\cmidrule{2-13}
% \textbf{Train-test split} &MAE [m] ↓ &$\tau$ ↑ &MAE [Pa] ↓ &$\tau$ ↑ &MAE [m] ↓ &$\tau$ ↑ &MAE [Pa] ↓ &$\tau$ ↑ &MAE [m] ↓ &$\tau$ ↑ &MAE [Pa] ↓ &$\tau$ ↑ \\\midrule
% 10-20 &7.70e-5 &0.34 &4.54e+2 &0.76 &4.65e+2 &0.28 &1.50e+10 &0.26 & & & & \\
% 20-20 &7.21e-5 &0.22 &4.06e+2 &0.77 &1.61e+2 &0.39 &2.10e+9 &0.35 &3.14e-4 &0.22 &2.14e+3 &0.43 \\
% 40-20 &6.54e-5 &0.39 &3.41e+2 &0.73 &1.53e-4 &0.75 &1.47e+3 &0.63 &1.64e-4 &0.59 &7.64e+2 &0.63 \\
% 80-20 &6.62e-5 &0.56 &3.36e+2 &0.83 &1.37e-4 &0.84 &1.21e+3 &0.79 &1.59e-4 &0.67 &6.25e+2 &0.73 \\
% \bottomrule
% \end{tabular}
% \end{table*}

%%%%%%%%%%%%%%%%%%%%%%%%%%%%%%%%%%%%%%%%%%%%%%%%%%%%%%%%%%%%%%%%%%%%%
% \begin{table*}[!htp]\centering
% \caption{Generalization to unseen elastic moduli $E$ on the same object}\label{tab:g2}
% \scriptsize
% \begin{tabular}{lrrrrrrrrrrrrr}\toprule
% & & & & &\multicolumn{4}{c}{\textbf{Test object name}} & & & & \\\midrule
% &\multicolumn{4}{c}{\textbf{Sphere 01}} &\multicolumn{4}{c}{\textbf{Flask}} &\multicolumn{4}{c}{\textbf{Strawberry 01}} \\
% &\multicolumn{2}{c}{Deformation} &\multicolumn{2}{c}{Stress} &\multicolumn{2}{c}{Deformation} &\multicolumn{2}{c}{Stress} &\multicolumn{2}{c}{Deformation} &\multicolumn{2}{c}{Stress} \\
% \textbf{\# train $E$} &MAE [m] ↓ &$\tau$ ↑ &MAE [Pa] ↓ &$\tau$ ↑ &MAE [m] ↓ &$\tau$ ↑ &MAE [Pa] ↓ &$\tau$ ↑ &MAE [m] ↓ &$\tau$ ↑ &MAE [Pa] ↓ &$\tau$ ↑ \\
% 1 &2.00e-4 &0.30 &7.00e+3 &0.37 &2.00e-3 &0.23 &1.05e+4 &0.58 &4.41e-4 &0.45 &3.04e+3 &0.79 \\
% 3 &2.31e-4 &0.63 &4.40e+3 &0.75 &6.33e-4 &0.63 &7.32e+3 &0.50 &1.03e-3 &0.65 &2.42e+3 &0.84 \\
% 5 &9.28e-5 &0.71 &1.05e+3 &0.83 &1.92e-4 &0.73 &2.41e+3 &0.78 &2.46e-4 &0.73 &1.40e+3 &0.84 \\
% \bottomrule
% \end{tabular}
% \end{table*}

%%%%%%%%%%%%%%%%%%%%%%%%%%%%%%%%%%%%%%%%%%%%%%%%%%%%%%%%%%%%%%%%%%%%%
% \begin{table*}[!htp]\centering
% \caption{Generalization to unseen objects within the same geometric category}\label{tab:g3}
% \scriptsize
% \begin{tabular}{lrrrrrrrrrrrrr}\toprule
% & & & & &\multicolumn{4}{c}{\textbf{Test object name}} & & & & \\\cmidrule{6-9}
% &\multicolumn{4}{c}{\textbf{Cylinder 04}} &\multicolumn{4}{c}{\textbf{Hexagon 01}} &\multicolumn{4}{c}{\textbf{Potato 03}} \\\cmidrule{2-13}
% &\multicolumn{2}{c}{Deformation} &\multicolumn{2}{c}{Stress} &\multicolumn{2}{c}{Deformation} &\multicolumn{2}{c}{Stress} &\multicolumn{2}{c}{Deformation} &\multicolumn{2}{c}{Stress} \\\cmidrule{2-13}
% \textbf{\# train objects in category} &MAE [m] ↓ &$\tau$ ↑ &MAE [Pa] ↓ &$\tau$ ↑ &MAE [m] ↓ &$\tau$ ↑ &MAE [Pa] ↓ &$\tau$ ↑ &MAE [m] ↓ &$\tau$ ↑ &MAE [Pa] ↓ &$\tau$ ↑ \\\midrule
% 1 &8.18e+1 &-0.10 &8.90e+8 &-0.52 &1.47e+2 &-0.07 &2.06e+9 &-0.45 & & & & \\
% 3 &1.67e-4 &0.38 &1.60e+3 &0.21 &2.27e-4 &0.26 &1.31e+3 &0.37 &9.48e-4 &0.23 &9.01e+2 &0.35 \\
% 5 &1.34e-4 &0.50 &9.37e+2 &0.45 &1.95e-4 &0.39 &7.70e+2 &0.49 &5.24e-4 &0.72 &6.87e+2 &0.52 \\
% \bottomrule
% \end{tabular}
% \end{table*}




%%%%%%%%%%%%%%%%%%%%%%%%%%%%%%%%%%%%%%%%%%%%%%%%%%%%%%%%%%%%%%%%%%%%%
\begin{table*}[!htp]\centering
\caption{Generalization to unseen real-world objects. Gray cells denote the best values per column. Train sets each contain only 5 objects; the ``All" group contains all 15. The $d_C$ column measures the best Chamfer distance between the test geometry and the train geometries. Lower $d_C$ implies geometric similarity between the train and test objects, and corresponds to more favorable MAE and $\tau$ during prediction.}\label{tab:g4}
\scriptsize
\begin{tabular}{lrrrrrrrrrrrrrrrr}\toprule
&\multicolumn{5}{c}{\textbf{Mustard bottle}} &\multicolumn{5}{c}{\textbf{Lemon half}} &\multicolumn{5}{c}{\textbf{Strawberry}} \\\cmidrule{2-16}
\multirow{2}{*}{\textbf{Train set}} &\multirow{2}{*}{$d_C$ [mm] ↓} &\multicolumn{2}{c}{Deformation [mm]} &\multicolumn{2}{c}{Stress [kPa]} &\multirow{2}{*}{$d_C$  ↓} &\multicolumn{2}{c}{Deformation} &\multicolumn{2}{c}{Stress} &\multirow{2}{*}{$d_C$  ↓} &\multicolumn{2}{c}{Deformation} &\multicolumn{2}{c}{Stress} \\\cmidrule{3-6}\cmidrule{8-11}\cmidrule{13-16}
& &MAE ↓ &$\tau_d$ ↑ &MAE ↓ &$\tau_s$ ↑ & &MAE ↓ &$\tau_d$ ↑ &MAE ↓ &$\tau_s$ ↑ & &MAE ↓ &$\tau_d$ ↑ &MAE ↓ &$\tau_s$ ↑ \\\midrule
Group 1 &\cellcolor[HTML]{efefef}5.57 &\cellcolor[HTML]{efefef}0.71 &\cellcolor[HTML]{efefef}0.62 &\cellcolor[HTML]{efefef}2.92 &\cellcolor[HTML]{efefef}0.56 &3.57 &4.82 &0.20 &2.15 &0.31 &3.27 &0.42 &0.09 &6.41 &0.58 \\
Group 2 &6.77 &0.74 &0.20 &4.72 &0.45 &\cellcolor[HTML]{efefef}3.30 &3.98 &\cellcolor[HTML]{efefef}0.50 &\cellcolor[HTML]{efefef}1.30 &\cellcolor[HTML]{efefef}0.43 &2.61 &0.30 &0.29 &2.85 &0.54 \\
Group 3 &6.07 &0.73 &-0.31 &4.57 &0.45 &4.50 &4.28 &-0.03 &2.63 &0.19 &\cellcolor[HTML]{efefef}2.46 &0.31 &-0.05 &2.79 &0.64 \\
All &\cellcolor[HTML]{efefef}5.57 &0.73 &0.60 &3.66 &\cellcolor[HTML]{efefef}0.56 &\cellcolor[HTML]{efefef}3.30 &\cellcolor[HTML]{efefef}3.98 &0.43 &1.36 &\cellcolor[HTML]{efefef}0.43 &\cellcolor[HTML]{efefef}2.46 &\cellcolor[HTML]{efefef}0.30 &\cellcolor[HTML]{efefef}0.39 &\cellcolor[HTML]{efefef}2.68 &\cellcolor[HTML]{efefef}0.66 \\
\bottomrule
\end{tabular}
\vspace{-5pt}
\end{table*}
\section{Grasp Planning Results}
We demonstrate DefGraspNets as a grasp planner on $3$ unseen objects (a mustard bottle, a lemon, and a strawberry) from existing datasets~\cite{Calli2015ICAR,TurboSquid} with real-world elastic moduli. First, we perform evaluation of sampled grasps. On each unseen object, $100$ random grasps $T_r$ are generated, and the optimization metric $Q(T)$ is evaluated for each $T \in T_r$ via the forward pass of DefGraspNets. Of the $100$ grasps, we select the $10$ grasps that are predicted to yield the lowest $Q$ (``threshold low" grasps), as well as $10$ grasps that are predicted to yield the highest $Q$ (``threshold high" grasps). We also randomly select $10$ other grasps from the remaining $80$ grasp candidates as a baseline. These $30$ grasps are then evaluated within the ground-truth simulator DefGraspSim. 

DefGraspNets is a reliable predictor of minimal- and maximal-$Q$ grasps on the unseen objects, with $88\%$ of these threshold-low and high grasps belonging to the set of 30 lowest and highest ground-truth-$Q$ grasps, respectively. 

Subsequently, we perform gradient-based grasp refinement on the threshold-low and threshold-high grasps to further reduce and increase $Q$, respectively. For each object, box plots in Fig.~\ref{fig:boxplots} visualize the distribution of ground-truth $Q$ values for $5$ groups of grasps: all sampled grasps, threshold-low grasps, threshold-low grasps after refinement, threshold-high grasps, and threshold-high grasps after refinement. In all cases, not only do threshold-high and low grasps from DefGraspNets yield substantially different ground-truth $Q$ values, but refinement increases their polarity as desired. 

The highest- and lowest-$Q$ grasps generated by the sample-and-refine grasp planning procedure are shown in Fig.~\ref{fig:minmax}. These grasps align with physical reasoning (e.g., the highest-deformation grasps on the bottle and lemon compress the directions of lowest geometric stiffness; the highest-stress grasp on a strawberry concentrates force on minimal area). These grasps are also validated in the real world in Fig.~\ref{fig:real_world}.



\begin{figure}
\centering
\includegraphics[width=\columnwidth]{figs/optimization/boxplots.png}
\caption{Box plots for $5$ groups of grasps for each unseen object: 1) all grasps, 2) threshold low grasps from sampling only, 3) threshold low grasps after refinement, 4) threshold high grasps from sampling only, and 5) threshold high grasps after refinement. The $y$-axis is the ground-truth $Q$ value of these grasps as computed in DefGraspSim.}
\label{fig:boxplots}
%\vspace{-20pt}
\end{figure}

\begin{figure}
\centering
\includegraphics[width=0.83\columnwidth]{figs/optimization/minmax_smaller.png}
\caption{Highest- and lowest-$Q$ grasps for the mustard bottle, lemon, and strawberry generated by the sample-and-refine procedure.}
\label{fig:minmax}
\vspace{-18pt}
\end{figure}


\begin{figure}[h]
\centering
\includegraphics[width=\columnwidth]{figs/real_world/real_world.png}
\caption{Validation of grasps from Fig.~\ref{fig:minmax} using a Franka-based gripper gravitationally loaded under $15\units{N}$. For the bottle and lemon, deformation is measured by proxy (change in volume and weight). For the strawberry, only the highest-$Q$ grasp imparts damage.}
\label{fig:real_world}
\vspace{-14pt}
\end{figure}
\section{Ablation Studies}\label{sec:ablation}
We run several ablation studies on our network architecture design. Table~\ref{tab:ablation} lists key design variables in DefGraspNets, with our selected conditions in bold. We compare our baseline model with 5 other trained models, each of which differ from baseline by exactly one condition.  We compare performance on a fixed test set and report the Kendall's $\tau$ metric for mean $\vec{d}$ and $\vec{\sigma}$. % over $50$ training epochs. 
We address the following questions:
\begin{itemize}[leftmargin=2ex]
    \item Does jointly predicting stress and deformation outperform using two separate networks to predict these quantities? (Ans: The $\tau$ metric is comparable in both cases, likely because stress and deformation are coupled through the equations of elasticity. Thus, training two networks would be strictly disadvantageous computationally, c.f. V1.)
    \item Does one-step prediction outperform multi-step prediction? (Ans: Yes, when predicting deformation. Otherwise, both are comparable when predicting stress, c.f. V2. In MeshGraphNets, multi-step prediction does not accumulate significant deformation errors  because the trajectory of the actuators is exactly controlled. In DefGraspNets, gripper force is commanded; the positions of \textit{both} $M_o$ and $M_g$ are predicted and subject to accumulating errors.) 
    \item Should $F_g$ be normalized by the number of contact edges? (Ans: Yes. This aligns with simulation, in which the total force is the sum of forces at all contact points, c.f. V3.)
    \item Should force features $F_g^C$ be assigned to contact edges or to gripper nodes? (Ans: The network is able to incorporate this information equally well, c.f. V4.)
\end{itemize}



\begin{table}[!htp]\centering
\caption{Ablation study variables and conditions. Our DefGraspNets network conditions are in bold. Best conditions are in gray.}\label{tab:ablation}
\scriptsize
\begin{tabular}{lrrrr}\toprule
\textbf{Variable} &\textbf{Condition} &\textbf{$\tau_d$↑} &\textbf{$\tau_s$↑} \\\midrule
\multirow{3}{*}{V1. Num. outputs} &\textbf{Def. and stress} &\cellcolor[HTML]{efefef}0.61 &0.82 \\
&Def. only &0.57 & \\
&Stress only & &\cellcolor[HTML]{efefef}0.84 \\ \midrule
\multirow{2}{*}{V2. Prediction type} &\textbf{One-step predictions} &\cellcolor[HTML]{efefef}0.61 &\cellcolor[HTML]{efefef}0.82 \\
&Multi-step &0.37 &0.70 \\ \midrule
\multirow{2}{*}{V3. Value of $F_g^W$} &\textbf{Distributed, $\mathbf{\sfrac{F_g}{|E^W|}}$} &\cellcolor[HTML]{efefef}0.61 &\cellcolor[HTML]{efefef}0.82 \\
&Non-distributed $F_g$ &0.33 &0.51 \\ \midrule
\multirow{2}{*}{V4. Assignment of $F_g$} &\textbf{On world edges $E^W$} &\cellcolor[HTML]{efefef}0.61 &\cellcolor[HTML]{efefef}0.82 \\
&On all nodes $V$ &0.58 &0.74 \\
\bottomrule
\end{tabular}
\vspace{-14pt}
\end{table}


\section{Discussion and Future Work}
We present DefGraspNets, a differentiable GNN-based model for FEM simulation of 3D stress and deformation fields. We demonstrate that training DefGraspNets on a diverse set of grasps on primitive geometries enables effective prediction and grasp planning on unseen, real-world geometries. DefGraspNets enables not only fast evaluation of sampled candidate grasps ($1500$x faster than GPU-accelerated FEM), but also gradient-based refinement of these grasps to optimize field quantities (e.g., max stress and mean deformation). We verify the effectiveness of optimized grasps on novel objects both in the ground-truth FEM simulator and in the real world. 

To expand DefGraspNets for use in downstream manipulation tasks such as food preparation or robotic surgery, prediction of additional quantities should be explored. These may include stability during transport and deformation and flow under reorientation and gravity. 
%or fluid dynamics as it exits the bottle. 
Furthermore, as FEM simulators evolve, DefGraspNets can be retrained to predict  soft-soft contact or heterogeneous material responses. 

Developing data augmentation techniques for \textit{meshes} may enable vast dataset scaling from a minimal set of object models, further strengthening our ability to generalize to unseen objects. 
%As our results suggest, generating train objects that are geometrically relevant to test objects is extremely useful. 
In addition, as our network is differentiable, techniques such as Stein variational gradient descent \cite{NIPS2016_b3ba8f1b} and stochastic gradient Langevin dynamics \cite{Welling2011Lan} may allow us to provide probabilistic, multi-modal \textit{distributions} of optimal grasps. Finally, architecture optimization (e.g., sparsity acceleration \cite{Choy2019CVPR}) may lead to even faster performance. 

DefGraspNets contributes the first differentiable approach to deformable grasp planning capable of predicting and optimizing stress and deformation fields on novel objects. We believe this coupling of fast prediction of field quantities with a differentiable model will enable a wide range of users to apply deformable grasp planning to their target domains.


\section{Acknowledgment}
\noindent We thank Miles Macklin and Eric Heiden for simulation expertise; Ankur Handa for network design advice; and Balakumar Sundaralingam and Clemens Eppner for insightful discussions.





\clearpage
\newpage
\bibliographystyle{unsrt}
\bibliography{references}  % .bib

\end{document}

%%% Local Variables:
%%% mode: latex
%%% TeX-master: t
%%% End: