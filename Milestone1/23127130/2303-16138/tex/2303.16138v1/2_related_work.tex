\section{Related Work}
Grasp planning has received significant attention in robotics~\cite{Grupen1991,Murray1994,sahbani2012overview,ciocarlie-2007-eigengrasps,Newbury2022arxis}. Recent works leverage learning-based approaches to enable fast planning and generalization to novel objects~\cite{lenz2015deep,mousavian20196,mahler2017dex,lu-ram2020-MultiFingeredGP,lundell-2021icra-fingan}. We focus on grasp planning for 3D deformable objects. Unlike rope or cloth, 3D deformables have dimensions of a similar magnitude along all 3 spatial axes and can undergo significant deformations along any of them~\cite{Huang2022RAL}.
We review grasp planning for 3D deformables, as well as methods for predicting stress and deformation fields via graph neural networks and differentiable simulation.

\subsection{Grasp planning for deformable objects}

Early works in grasp planning for deformable objects focused on finding \textit{stable} grasps of planar objects, under which the object's strain energy would be maximized without inducing plastic deformation~\cite{Goldberg2005IJRR, Jia2014IJRR}. This has since been extended to the 3D case, where novel time-dependent grasp quality metrics have been proposed to capture the evolution of contact states under deformation~\cite{Song2022RAL,Le2023IROS}.

Grasp planning for \textit{deformation} of thin-walled containers (e.g., boxes, bottles) has also been explored. Given a 3D geometric stiffness map of the object, a minimal deformation grasp can be planned by localizing contact at high-stiffness regions. This map can be generated in simulation, via real-world probing ~\cite{Xu2020ICRA}, or from 2D images of the object via generative adversarial networks~\cite{Makihara2022AR}. Grasp planning for \textit{stress} has also been demonstrated via simulation on quasi-rigid objects using the boundary element method~\cite{Pan2020ICRA}. Finally, grasp planning for additional metrics  can be performed with DefGraspSim, a 3D FEM-based grasp simulation framework \cite{Huang2022RAL}. For every grasp, it evaluates success, stability, stress, deformation, strain energy, and controllability.

These methods vary not only in the planning metric, but also in the type of computation required. Some require FEM simulation of the beginning of the interaction (e.g., just past initial contact)~\cite{Jia2014IJRR,Song2022RAL,Le2023IROS,Xu2020ICRA} or the full interaction~\cite{Pan2020ICRA,Huang2022RAL}, whereas others use neural networks~\cite{Makihara2022AR}. Yet, all of these planners can only evaluate or predict the outcome of a candidate grasp, and cannot optimize grasps through gradient-based methods.

\subsection{Graph neural networks for deformable-object interaction}
Graph neural networks (GNNs) have been used to efficiently learn  dynamics models for granular solids, deformable solids, and fluids  \cite{li2018learning, Sanchez020ICML, Ummenhofer2020ICLR, Pfaff2021ICLR, Shi2022RSS}. Inspiring our work, MeshGraphNets~\cite{Pfaff2021ICLR} used GNNs to learn accurate dynamics for deformable solids
using mesh-based representations, training from an industry-standard FEM solver. It predicted deformation and stress on a 3D deformable plate with kinematically-actuated colliding shapes and achieved evaluation speeds up to two orders of magnitude faster than the solver.
RoboCraft~\cite{Shi2022RSS} used GNNs to  learn how plasticine-like objects with particle representations deform under interaction with a robotic gripper, training from visual input. Whereas MeshGraphNets used forward passes through the networks to predict dynamics, RoboCraft also used backwards passes to perform gradient-based trajectory optimization, molding the plasticine into a desired shape.

Our work also utilizes a GNN as a surrogate simulator for dynamics predictions. Unlike MeshGraphNets, which uses $N$-step rollouts to predict a final state via intermediate steps, DefGraspNets performs direct, one-step predictions of the final state. One-step prediction ensures that gradients are only propagated once through the network rather than over tens or hundreds of steps, mitigating vanishing or exploding gradients \cite{lillicrap2019backpropagation}. Furthermore, we focus on quasistatic rather than dynamic grasping; the ability of multi-step rollouts to predict object and controller dynamics offers limited advantage. Our ablation study verifies that using single-step predictions in our setting performs better than multi-step predictions (c.f. Sec.~\ref{sec:ablation}).

Like RoboCraft, we design our network to include gripper actions in order to perform gradient-based optimization for grasp planning.
Unlike RoboCraft and MeshGraphNets, we use force rather than position commands for our actuators, as force commands are implemented in notable industrial grippers \cite{Lauzier2016Robotiq,Onrobot2019,Schunkgrippers} and are preferable for grasping (as opposed to applications like shape control). Gripper force determines whether the grasp will overcome the object's gravity, and gripper position cannot indicate force without additional knowledge (e.g., contact area, object stiffness). In addition, when grasping stiffer objects, position commands can induce high torques that can damage both the object and gripper. We also generalize our network to different elastic moduli, which was not explored in prior works. 

\subsection{Differentiable simulators}
Differentiable simulators for rigid and deformable bodies allow gradients of output variables (e.g., poses, velocities, or deformation fields of objects) to be computed with respect to input variables (e.g., control inputs or material parameters) \cite{warp2022, freeman2021brax, hu2019difftaichi, heiden2021neuralsim, werling2021fast, murthy2020gradsim, hu2019chainqueen}. Such simulators enable gradient-based optimization for control optimization~\cite{Xu2022ICLR,  murthy2020gradsim, huang2021plasticinelab, hu2019chainqueen, geilinger2020add}, parameter estimation \cite{Heiden2022AutoRob, geilinger2020add, hu2019chainqueen}, and inverse design \cite{xu2021end, hu2019chainqueen}.

There are 4 main strategies to realize a differentiable simulator or equivalent model: 1) finite-differencing a non-differentiable simulator, which has unfavorable $\mathcal{O}(n)$ scaling to an $n$-dimensional input space \cite{baydin2018automatic, margossian2019review}, 2) analytically or automatically differentiating a simulator that smoothly approximates spatial or kinetic discontinuities (e.g., penalty-based contact forces and smooth friction models \cite{Heiden2022AutoRob, geilinger2020add}, which may introduce inaccuracies or require tuning), 3) training a deep network with physically-based loss functions \cite{raissi2019physics, karniadakis2021physics}, which has seen limited use for contact dynamics \cite{pfrommer2020contactnets}, and 4) training a deep network on datasets from a non-differentiable simulator, primarily with graph-based inductive biases \cite{Battaglia2018arxiv, li2018learning, Sanchez020ICML, Pfaff2021ICLR}.

For our application, we aim to simulate robotic grasping of 3D deformable objects. Thus, we focus on gold-standard 3D FEM simulation of deformable objects with contact. For this application, strategy 2 has been explored in a handful of recent works \cite{Heiden2022AutoRob}, including differentiable projective dynamics \cite{Du2021TOG, qiao2021differentiable}. However, such simulators typically execute substantially slower than real-time (especially including a backward pass), and only one has realized differentiable FEM and contact modeled via the full nonlinear complementarity problem (NCP) \cite{horak2019similarities} with both static and dynamic friction \cite{qiao2021differentiable}.

In this work, we explore strategy 4, training for the first time on a GPU-accelerated robotics FEM simulator \cite{makoviychuk2021isaac} that addresses the full NCP \cite{macklin2019non} and has been experimentally validated across multiple studies~\cite{Huang2022RAL,Narang2021IJRR, Narang2021Latent}. To our knowledge, this effort also comprises the first application of such methods to robotic grasping of 3D deformable objects.

Strategy 2 has often been favored over strategy 4 due to the former's potential for generalizing to arbitrary physics \cite{Du2021TOG, qiao2021differentiable}. Nevertheless, we show for the first time that strategy 4, through judicious selection and scaling of training data, can indeed generalize to novel grasps, elastic moduli, in-category objects, and out-of-category objects. Furthermore, the trained networks   can execute 2 to 3 orders of magnitude faster than the reference simulator (i.e., faster than real-time).



%%% Local Variables:
%%% mode: latex
%%% TeX-master: "root"
%%% End:
