\section{Prediction Results}
We test DefGraspNets' predictions of the ranking of grasps with respect to their mean stress and deformation values by quantifying the respective Kendall's $\tau$ rank correlation coefficients ($\tau_{s}$ and $\tau_{d}$).\footnote{Kendall's $\tau$ was chosen over Spearman's $\rho$ for its comparative robustness (i.e., smaller gross error sensitivity).} We answer the following questions for 4 levels of generalization:

\begin{figure*}[ht]
     \centering
     \begin{subfigure}[b]{.3\textwidth}
         \centering
         \includegraphics[scale=0.23, trim={0cm 0cm 0cm 0cm},clip]{figs/prediction/mustard.png}
         \caption{Mustard bottle, $F_g = 12
         \units{N}$, $E = 1e7$}
         \label{fig:pred_mustard}
     \end{subfigure}
     \hfill
     \begin{subfigure}[b]{.3\textwidth}
         \centering
         \includegraphics[scale=0.23, trim={0cm 0cm 0cm 0cm},clip]{figs/prediction/strawberry.png}
         \caption{Strawberry, $F_g = 6\units{N}$, $E = 5e4$}
         \label{fig:pred_strawberry}
     \end{subfigure}
     \hfill
     \begin{subfigure}[b]{.3\textwidth}
         \centering
         \includegraphics[scale=0.23, trim={0cm 0cm 0cm 0cm},clip]{figs/prediction/sphere.png}
         \caption{Sphere, $F_g = 5\units{N}$, $E \in [5e5, 1e6, 5e6]$}
         \label{fig:pred_lemon}
     \end{subfigure}
     \hfill
        \caption{A) Predicted and ground-truth deformation fields for a mustard bottle subject to grasps inducing increasing mean deformation, B) Predicted and ground-truth stress fields for a strawberry subject to grasps inducing increasing maximum stress, and C) Predicted and ground-truth stress fields for a sphere of increasing elastic moduli subject to identical grasps (deformation can be seen in resulting shape).
        }
        \label{fig:predictions}
\end{figure*}

\begin{enumerate}[leftmargin=2ex]
    \item Can DefGraspNets rank unseen grasps when trained on other grasps on the same object? (Ans: Yes. For an 80-20 train-test split over grasps on the same object, we get an average $\tau_{s} = 0.78$ and $\tau_d = 0.66$ over $15000$ unseen $X_i$.)
    \item Can DefGraspNets generalize to unseen elastic moduli $E$ on the same object? (Ans: Yes. For a 7-3 train-test split over unique $E$ for grasps on the same object, we get an average $\tau_{s} = 0.81$ and $\tau_d = 0.72$ over $15000$ unseen $X_i$.)
    \item Can DefGraspNets generalize to unseen primitive objects within the same geometric category? (Ans: Yes. For a 5-1 train-test split over unique objects, we get an average $\tau_{s} = 0.48$ and $\tau_d = 0.54$ over $15000$ unseen $X_i$.)
    \item Can DefGraspNets generalize to unseen real-world objects? (Ans: Yes. Moreover, we generate useful predictions even when training on a small number of objects, as long as the train geometries are relevant to the test geometry as quantified by a low Chamfer distance. See Table~\ref{tab:g4}, which also reports the mean absolute error (\textit{MAE})). 
\end{enumerate}




% We present DefGraspNets' prediction results on 4 different levels of generalization. We report the mean absolute error of deformation and stress predictions for all test grasp states, as well as the average Kendall's $\tau$ coefficient over the mean stress and mean deformation. 


% First, we show how predictions on unseen grasps on the same object improve as training data grows (Table \ref{tab:g1}). Second, we demonstrate generalization to unseen elastic moduli $E$ as the variety of $E$ in the training data increases (Table \ref{tab:g2}). Third, we show that generalization is also achievable over unseen objects within the same geometric categories (Table \ref{tab:g3}). Finally, we demonstrate generalization over unseen real-world  objects that are not in the geometric categories used in the training set (Table \ref{tab:g4}). We show that although performance is best with more training data, performance can also be improved if the training data is more geometrically similar to the real-world object, as quantified by the average Chamfer distance $d_C$ between the test object and the objects in the train set. 

Full visualizations of predicted field quantities for the 4th (i.e., most challenging) generalization level is shown in Fig.~\ref{fig:predictions} on an unseen mustard bottle and unseen strawberry, as well as for the 2nd generalization level on a sphere.

% Across all examples, the regions of high stress and deformation always overlap between the predicted and ground truth cases. 






%%%%%%%%%%%%%%%%%%%%%%%%%%%%%%%%%%%%%%%%%%%%%%%%%%%%%%%%%%%%%%%%%%%%%
% \begin{table*}[!htp]\centering
% \caption{Generalization to unseen grasps on the same object}\label{tab:g1}
% \scriptsize
% \begin{tabular}{lrrrrrrrrrrrrr}\toprule
% & & & & &\multicolumn{4}{c}{\textbf{Test object name}} & & & & \\\cmidrule{6-9}
% &\multicolumn{4}{c}{\textbf{Cuboid 02}} &\multicolumn{4}{c}{\textbf{Flask 01}} &\multicolumn{4}{c}{\textbf{Apple 02}} \\\cmidrule{2-13}
% &\multicolumn{2}{c}{Deformation} &\multicolumn{2}{c}{Stress} &\multicolumn{2}{c}{Deformation} &\multicolumn{2}{c}{Stress} &\multicolumn{2}{c}{Deformation} &\multicolumn{2}{c}{Stress} \\\cmidrule{2-13}
% \textbf{Train-test split} &MAE [m] ↓ &$\tau$ ↑ &MAE [Pa] ↓ &$\tau$ ↑ &MAE [m] ↓ &$\tau$ ↑ &MAE [Pa] ↓ &$\tau$ ↑ &MAE [m] ↓ &$\tau$ ↑ &MAE [Pa] ↓ &$\tau$ ↑ \\\midrule
% 10-20 &7.70e-5 &0.34 &4.54e+2 &0.76 &4.65e+2 &0.28 &1.50e+10 &0.26 & & & & \\
% 20-20 &7.21e-5 &0.22 &4.06e+2 &0.77 &1.61e+2 &0.39 &2.10e+9 &0.35 &3.14e-4 &0.22 &2.14e+3 &0.43 \\
% 40-20 &6.54e-5 &0.39 &3.41e+2 &0.73 &1.53e-4 &0.75 &1.47e+3 &0.63 &1.64e-4 &0.59 &7.64e+2 &0.63 \\
% 80-20 &6.62e-5 &0.56 &3.36e+2 &0.83 &1.37e-4 &0.84 &1.21e+3 &0.79 &1.59e-4 &0.67 &6.25e+2 &0.73 \\
% \bottomrule
% \end{tabular}
% \end{table*}

%%%%%%%%%%%%%%%%%%%%%%%%%%%%%%%%%%%%%%%%%%%%%%%%%%%%%%%%%%%%%%%%%%%%%
% \begin{table*}[!htp]\centering
% \caption{Generalization to unseen elastic moduli $E$ on the same object}\label{tab:g2}
% \scriptsize
% \begin{tabular}{lrrrrrrrrrrrrr}\toprule
% & & & & &\multicolumn{4}{c}{\textbf{Test object name}} & & & & \\\midrule
% &\multicolumn{4}{c}{\textbf{Sphere 01}} &\multicolumn{4}{c}{\textbf{Flask}} &\multicolumn{4}{c}{\textbf{Strawberry 01}} \\
% &\multicolumn{2}{c}{Deformation} &\multicolumn{2}{c}{Stress} &\multicolumn{2}{c}{Deformation} &\multicolumn{2}{c}{Stress} &\multicolumn{2}{c}{Deformation} &\multicolumn{2}{c}{Stress} \\
% \textbf{\# train $E$} &MAE [m] ↓ &$\tau$ ↑ &MAE [Pa] ↓ &$\tau$ ↑ &MAE [m] ↓ &$\tau$ ↑ &MAE [Pa] ↓ &$\tau$ ↑ &MAE [m] ↓ &$\tau$ ↑ &MAE [Pa] ↓ &$\tau$ ↑ \\
% 1 &2.00e-4 &0.30 &7.00e+3 &0.37 &2.00e-3 &0.23 &1.05e+4 &0.58 &4.41e-4 &0.45 &3.04e+3 &0.79 \\
% 3 &2.31e-4 &0.63 &4.40e+3 &0.75 &6.33e-4 &0.63 &7.32e+3 &0.50 &1.03e-3 &0.65 &2.42e+3 &0.84 \\
% 5 &9.28e-5 &0.71 &1.05e+3 &0.83 &1.92e-4 &0.73 &2.41e+3 &0.78 &2.46e-4 &0.73 &1.40e+3 &0.84 \\
% \bottomrule
% \end{tabular}
% \end{table*}

%%%%%%%%%%%%%%%%%%%%%%%%%%%%%%%%%%%%%%%%%%%%%%%%%%%%%%%%%%%%%%%%%%%%%
% \begin{table*}[!htp]\centering
% \caption{Generalization to unseen objects within the same geometric category}\label{tab:g3}
% \scriptsize
% \begin{tabular}{lrrrrrrrrrrrrr}\toprule
% & & & & &\multicolumn{4}{c}{\textbf{Test object name}} & & & & \\\cmidrule{6-9}
% &\multicolumn{4}{c}{\textbf{Cylinder 04}} &\multicolumn{4}{c}{\textbf{Hexagon 01}} &\multicolumn{4}{c}{\textbf{Potato 03}} \\\cmidrule{2-13}
% &\multicolumn{2}{c}{Deformation} &\multicolumn{2}{c}{Stress} &\multicolumn{2}{c}{Deformation} &\multicolumn{2}{c}{Stress} &\multicolumn{2}{c}{Deformation} &\multicolumn{2}{c}{Stress} \\\cmidrule{2-13}
% \textbf{\# train objects in category} &MAE [m] ↓ &$\tau$ ↑ &MAE [Pa] ↓ &$\tau$ ↑ &MAE [m] ↓ &$\tau$ ↑ &MAE [Pa] ↓ &$\tau$ ↑ &MAE [m] ↓ &$\tau$ ↑ &MAE [Pa] ↓ &$\tau$ ↑ \\\midrule
% 1 &8.18e+1 &-0.10 &8.90e+8 &-0.52 &1.47e+2 &-0.07 &2.06e+9 &-0.45 & & & & \\
% 3 &1.67e-4 &0.38 &1.60e+3 &0.21 &2.27e-4 &0.26 &1.31e+3 &0.37 &9.48e-4 &0.23 &9.01e+2 &0.35 \\
% 5 &1.34e-4 &0.50 &9.37e+2 &0.45 &1.95e-4 &0.39 &7.70e+2 &0.49 &5.24e-4 &0.72 &6.87e+2 &0.52 \\
% \bottomrule
% \end{tabular}
% \end{table*}




%%%%%%%%%%%%%%%%%%%%%%%%%%%%%%%%%%%%%%%%%%%%%%%%%%%%%%%%%%%%%%%%%%%%%
\begin{table*}[!htp]\centering
\caption{Generalization to unseen real-world objects. Gray cells denote the best values per column. Train sets each contain only 5 objects; the ``All" group contains all 15. The $d_C$ column measures the best Chamfer distance between the test geometry and the train geometries. Lower $d_C$ implies geometric similarity between the train and test objects, and corresponds to more favorable MAE and $\tau$ during prediction.}\label{tab:g4}
\scriptsize
\begin{tabular}{lrrrrrrrrrrrrrrrr}\toprule
&\multicolumn{5}{c}{\textbf{Mustard bottle}} &\multicolumn{5}{c}{\textbf{Lemon half}} &\multicolumn{5}{c}{\textbf{Strawberry}} \\\cmidrule{2-16}
\multirow{2}{*}{\textbf{Train set}} &\multirow{2}{*}{$d_C$ [mm] ↓} &\multicolumn{2}{c}{Deformation [mm]} &\multicolumn{2}{c}{Stress [kPa]} &\multirow{2}{*}{$d_C$  ↓} &\multicolumn{2}{c}{Deformation} &\multicolumn{2}{c}{Stress} &\multirow{2}{*}{$d_C$  ↓} &\multicolumn{2}{c}{Deformation} &\multicolumn{2}{c}{Stress} \\\cmidrule{3-6}\cmidrule{8-11}\cmidrule{13-16}
& &MAE ↓ &$\tau_d$ ↑ &MAE ↓ &$\tau_s$ ↑ & &MAE ↓ &$\tau_d$ ↑ &MAE ↓ &$\tau_s$ ↑ & &MAE ↓ &$\tau_d$ ↑ &MAE ↓ &$\tau_s$ ↑ \\\midrule
Group 1 &\cellcolor[HTML]{efefef}5.57 &\cellcolor[HTML]{efefef}0.71 &\cellcolor[HTML]{efefef}0.62 &\cellcolor[HTML]{efefef}2.92 &\cellcolor[HTML]{efefef}0.56 &3.57 &4.82 &0.20 &2.15 &0.31 &3.27 &0.42 &0.09 &6.41 &0.58 \\
Group 2 &6.77 &0.74 &0.20 &4.72 &0.45 &\cellcolor[HTML]{efefef}3.30 &3.98 &\cellcolor[HTML]{efefef}0.50 &\cellcolor[HTML]{efefef}1.30 &\cellcolor[HTML]{efefef}0.43 &2.61 &0.30 &0.29 &2.85 &0.54 \\
Group 3 &6.07 &0.73 &-0.31 &4.57 &0.45 &4.50 &4.28 &-0.03 &2.63 &0.19 &\cellcolor[HTML]{efefef}2.46 &0.31 &-0.05 &2.79 &0.64 \\
All &\cellcolor[HTML]{efefef}5.57 &0.73 &0.60 &3.66 &\cellcolor[HTML]{efefef}0.56 &\cellcolor[HTML]{efefef}3.30 &\cellcolor[HTML]{efefef}3.98 &0.43 &1.36 &\cellcolor[HTML]{efefef}0.43 &\cellcolor[HTML]{efefef}2.46 &\cellcolor[HTML]{efefef}0.30 &\cellcolor[HTML]{efefef}0.39 &\cellcolor[HTML]{efefef}2.68 &\cellcolor[HTML]{efefef}0.66 \\
\bottomrule
\end{tabular}
\vspace{-5pt}
\end{table*}