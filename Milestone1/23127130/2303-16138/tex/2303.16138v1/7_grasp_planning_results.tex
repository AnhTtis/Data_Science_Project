\section{Grasp Planning Results}
We demonstrate DefGraspNets as a grasp planner on $3$ unseen objects (a mustard bottle, a lemon, and a strawberry) from existing datasets~\cite{Calli2015ICAR,TurboSquid} with real-world elastic moduli. First, we perform evaluation of sampled grasps. On each unseen object, $100$ random grasps $T_r$ are generated, and the optimization metric $Q(T)$ is evaluated for each $T \in T_r$ via the forward pass of DefGraspNets. Of the $100$ grasps, we select the $10$ grasps that are predicted to yield the lowest $Q$ (``threshold low" grasps), as well as $10$ grasps that are predicted to yield the highest $Q$ (``threshold high" grasps). We also randomly select $10$ other grasps from the remaining $80$ grasp candidates as a baseline. These $30$ grasps are then evaluated within the ground-truth simulator DefGraspSim. 

DefGraspNets is a reliable predictor of minimal- and maximal-$Q$ grasps on the unseen objects, with $88\%$ of these threshold-low and high grasps belonging to the set of 30 lowest and highest ground-truth-$Q$ grasps, respectively. 

Subsequently, we perform gradient-based grasp refinement on the threshold-low and threshold-high grasps to further reduce and increase $Q$, respectively. For each object, box plots in Fig.~\ref{fig:boxplots} visualize the distribution of ground-truth $Q$ values for $5$ groups of grasps: all sampled grasps, threshold-low grasps, threshold-low grasps after refinement, threshold-high grasps, and threshold-high grasps after refinement. In all cases, not only do threshold-high and low grasps from DefGraspNets yield substantially different ground-truth $Q$ values, but refinement increases their polarity as desired. 

The highest- and lowest-$Q$ grasps generated by the sample-and-refine grasp planning procedure are shown in Fig.~\ref{fig:minmax}. These grasps align with physical reasoning (e.g., the highest-deformation grasps on the bottle and lemon compress the directions of lowest geometric stiffness; the highest-stress grasp on a strawberry concentrates force on minimal area). These grasps are also validated in the real world in Fig.~\ref{fig:real_world}.



\begin{figure}
\centering
\includegraphics[width=\columnwidth]{figs/optimization/boxplots.png}
\caption{Box plots for $5$ groups of grasps for each unseen object: 1) all grasps, 2) threshold low grasps from sampling only, 3) threshold low grasps after refinement, 4) threshold high grasps from sampling only, and 5) threshold high grasps after refinement. The $y$-axis is the ground-truth $Q$ value of these grasps as computed in DefGraspSim.}
\label{fig:boxplots}
%\vspace{-20pt}
\end{figure}

\begin{figure}
\centering
\includegraphics[width=0.83\columnwidth]{figs/optimization/minmax_smaller.png}
\caption{Highest- and lowest-$Q$ grasps for the mustard bottle, lemon, and strawberry generated by the sample-and-refine procedure.}
\label{fig:minmax}
\vspace{-18pt}
\end{figure}


\begin{figure}[h]
\centering
\includegraphics[width=\columnwidth]{figs/real_world/real_world.png}
\caption{Validation of grasps from Fig.~\ref{fig:minmax} using a Franka-based gripper gravitationally loaded under $15\units{N}$. For the bottle and lemon, deformation is measured by proxy (change in volume and weight). For the strawberry, only the highest-$Q$ grasp imparts damage.}
\label{fig:real_world}
\vspace{-14pt}
\end{figure}