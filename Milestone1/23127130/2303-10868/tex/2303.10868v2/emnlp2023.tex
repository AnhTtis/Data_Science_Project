% This must be in the first 5 lines to tell arXiv to use pdfLaTeX, which is strongly recommended.
\pdfoutput=1
% In particular, the hyperref package requires pdfLaTeX in order to break URLs across lines.

\documentclass[11pt]{article}

% Remove the "review" option to generate the final version.
% \usepackage[review]{EMNLP2023}
\usepackage{EMNLP2023}

% Standard package includes
\usepackage{times}
\usepackage{latexsym}

% For proper rendering and hyphenation of words containing Latin characters (including in bib files)
\usepackage[T1]{fontenc}
% For Vietnamese characters
% \usepackage[T5]{fontenc}
% See https://www.latex-project.org/help/documentation/encguide.pdf for other character sets

% This assumes your files are encoded as UTF8
\usepackage[utf8]{inputenc}
\usepackage{enumitem}
% This is not strictly necessary and may be commented out.
% However, it will improve the layout of the manuscript,
% and will typically save some space.
\usepackage{microtype}
\usepackage{cleveref}
\usepackage{xcolor}
\usepackage{multirow}
\usepackage{adjustbox}
\usepackage{booktabs}
% This is also not strictly necessary and may be commented out.
% However, it will improve the aesthetics of text in
% the typewriter font.
\usepackage{inconsolata}
\newcommand{\ws}[1]{\textcolor{magenta}{(WWS: #1)}}
\newcommand{\cw}[1]{\textcolor{red}{(qcw: #1)}}
\newcommand{\rc}[1]{\textcolor{purple}{(zrc: #1)}}
\newcommand{\fk}[1]{\textcolor{blue}{#1}}
\newcommand{\xl}[1]{\textcolor{orange}{(dxl: #1)}}

\newcommand{\red}[1]{\textcolor{red}{#1}}

% If the title and author information does not fit in the area allocated, uncomment the following
%
%\setlength\titlebox{<dim>}
%
% and set <dim> to something 5cm or larger.

\title{Retrieving Multimodal Information for Augmented Generation: A Survey}

% Author information can be set in various styles:
% For several authors from the same institution:
% \author{Author 1 \and ... \and Author n \\
%         Address line \\ ... \\ Address line}
% if the names do not fit well on one line use
%         Author 1 \\ {\bf Author 2} \\ ... \\ {\bf Author n} \\
% For authors from different institutions:
% \author{Author 1 \\ Address line \\  ... \\ Address line
%         \And  ... \And
%         Author n \\ Address line \\ ... \\ Address line}
% To start a separate ``row'' of authors use \AND, as in
% \author{Author 1 \\ Address line \\  ... \\ Address line
%         \AND
%         Author 2 \\ Address line \\ ... \\ Address line \And
%         Author 3 \\ Address line \\ ... \\ Address line}

\newcommand*{\affaddr}[1]{#1} % No op here. Customize it for different styles.
\newcommand*{\affmark}[1][*]{\textsuperscript{#1}}
\newcommand*{\email}[1]{\textrm{#1}}

\author{
  Ruochen Zhao\affmark[1]~~
  Hailin Chen\affmark[1]~~
  \textbf{Weishi Wang\affmark[1]}~~
  Fangkai Jiao\affmark[1]~~ \\
  \textbf{Xuan Long Do}\affmark[1]~~
  \textbf{Chengwei Qin\affmark[1]}~~
  \textbf{Bosheng Ding\affmark[1]}~~
  \textbf{Xiaobao Guo\affmark[1]}~~ \\
  \textbf{Minzhi Li \affmark[2]}~~
  \textbf{Xingxuan Li \affmark[1]}~~
  \textbf{Shafiq Joty\affmark[1,3]\thanks{~~Work done while the author is on leave from NTU}}~~ \\
\affaddr{\affmark[1]Nanyang Technological University, Singapore} \\
\affaddr{\affmark[2]National University of Singapore, Singapore} \\
\affaddr{\affmark[3]Salesforce Research} \\
\email{\small{\{ruochen002, hailin001, xuanlong001, weishi001, fangkai002, chengwei003, bosheng001\}@e.ntu.edu.sg}}\\
\email{\small{\{xiaobao001, xingxuan001\}@e.ntu.edu.sg, li.minzhi@u.nus.edu, srjoty@ntu.edu.sg}}\\
}

% We review methods that augments with multimodal retrieval
% How might it help / Why is it important
% Our contributions
\begin{document}
\maketitle
\begin{abstract}
% Gaining popularity
As Large Language Models (LLMs) become popular, there emerged an important trend of using multimodality to augment the LLMs' generation ability, which enables LLMs to better interact with the world.
% importance of survey
However, there lacks a unified perception of at which stage and how to incorporate different modalities.
% Survey content
In this survey, we review methods that assist and augment generative models by retrieving multimodal knowledge, whose formats range from images, codes, tables, graphs, to audio.
Such methods offer a promising solution to important concerns such as factuality, reasoning, interpretability, and robustness. 
% Survey contributions
By providing an in-depth review, this survey is expected to provide scholars with a deeper understanding of the methods' applications and encourage them to adapt existing techniques to the fast-growing field of LLMs.
\end{abstract}

\vspace{-1ex}
\section{Introduction}
\vspace{-1ex}
\section{Introduction}
\label{sec:intro}
\begin{figure}[t]
\begin{center}
    \includegraphics[width=1\linewidth]{figures/teaser.pdf}
\end{center}
\vspace{-0.1in}
\caption{\textbf{{\em Foggy} vs {\em Clear} NeRF.} Our \ournerf gets rid of reconstruction errors manifested as foggy ``floaters" in the density volume without additional input or significant computational overhead. 
%
Below are density profiles along a given ray before and after our geometry correction procedure, where we discard density peaks corresponding to floaters.
}
\label{fig:teaser}
\vspace{-0.2in}
\end{figure}



%The emergence of 
Neural Radiance Fields (NeRFs)~\cite{mildenhall2020nerf}  %and its variants 
have made revolutionary contributions in %photo-realistic 
novel view synthesis~\cite{barron2021mip,barron2022mip}, 
autonomous driving~\cite{rematas2022urban,tancik2022block}, digital human~\cite{hong2022headnerf,zhao2022humannerf}, and 3D content generation~\cite{eg3d,poole2022dreamfusion,lin2022magic3d}.
%by leveraging a multi-layer perceptron (MLP) to implicitly model the mapping from input 5D coordinates (i.e., 3D coordinates $\mathbf{x} = (x,y,z)$ and 2D viewing directions $\mathbf{d}=(\theta,\phi)$) to volume density $\sigma$ and view-dependent emitted radiance color $\mathbf{c} = (r,g,b)$. 
%
%They then use traditional volume rendering mechanisms on the obtained continuous 5D function (i.e., MLP) to generate novel views. 
To date, unfortunately, most NeRF-based methods encounter challenges when tackling large-scale cluttered scenes (e.g., Fig.~\ref{fig:teaser}):
\begin{enumerate}[leftmargin=0.16in, topsep=2pt,itemsep=-1ex,partopsep=1ex,parsep=1ex]
\item Input observations used for NeRF are often too sparse  compared to forward-facing or synthetic looking-inward scenes;
%\item Recovering fine-grained objects within a large volume is challenging for NeRF; %in capturing details accurately.
\item View-dependent visual effects give rise to ambiguity, resulting in a ``foggy" density field as shown in Fig.~\ref{fig:teaser}. 
%
Such artifacts are particularly pronounced in indoor scenes strewn with view-dependent appearances, such as specular highlights, glossy surface reflections from man-made objects. 
\end{enumerate}

Despite attempts to enhance NeRF's rendering quality given suboptimal input, such as using 3D conical frustums~\cite{barron2021mip,barron2022mip}, physically-grounded augmentations~\cite{chen2022aug}, and misalignment correction~\cite{jiang2022alignerf},  these challenges have yet to be fully resolved.
%
Depth supervision~\cite{deng2022depth, wei2021nerfingmvs} or proxy geometry~\cite{xu2021scalable,wu2022scalable} images can help alleviate the challenges in handling large-scale with sparse input, at the expense of %but they come at the cost of requiring 
expensive pre-processing or additional input.
%
Another line of work~\cite{wang2021neus, oechsle2021unisurf, wang2022neuris} achieves better reconstruction of surface geometry by using signed distances instead of volume density as scene representation. However, they sacrifice the ability to synthesize photo-realistic novel views.

%We observe that NeRF has been suffering from foggy ``floater" artifacts in large-scale cluttered scenes.
%
%Such artifacts are particularly pronounced in indoor scenes strewn with view-dependent appearances from man-made objects. 
%
To address the above issues, we propose an extension to NeRF, dubbed as {\bf \ournerf}, which enforces effective {\em appearance} and {\em geometry} constraints conducive to accurate colors and 3D densities estimation. We believe \ournerf can contribute beyond novel view synthesis, such as NeRF object detection~\cite{hu2022nerf}, NeRF object segmentation~\cite{zhi2021place, liu2022unsupervised, fan2022nerf,ren2022neural}, and NeRF registration~\cite{goli2022nerf2nerf}, where the rooms for improvement are substantial if more accurate color and density estimation are available.

Correspondingly, there are two steps in \ournerf. First, for appearance correction, the view-independent and view-dependent color components are predicted from the underlying 3D scene, which is combined to produce the final color estimation (Fig.~\ref{fig:toaster}).
%
The view-independent component (diffuse color and shading) captures the overall scene color, while the view-dependent component (highlights or reflections) captures color variations due to changes in viewing angle.
%
\ournerf then discards these view-dependent appearances in the training views to prevent them from interfering with the density estimation.
%
Second, a simple and effective geometry correction procedure will be performed to further eliminate the foggy ``floaters" or density errors. This geometry correction procedure is based on an assumption in line with traditional ray tracing in computer graphics.
\begin{comment}
% xh: basically copying method
On the other hand, ClearNeRF performs a geometric correction procedure performed on each traced ray during inference to refine the density estimation and better tackle the floater artifacts. 
%
The geometry correction procedure assumes that there should only be one salient peak along each traced ray during NeRF inference. 
Only the salient peak closest to the ray origin (the camera center) corresponds to  true geometry while the others will be manifested as foggy floaters hovering in the density volume. 
%
This assumption is in line with traditional ray tracing in computer graphics where in the absence of noise, only one intersection per ray should be returned to indicate the closest ray-object intersection.
%
\end{comment}
%%%%%%%%%%%
%As shown in Fig.~\ref{fig:teaser}, when reconstructing an indoor scene with sparse input and highly view-dependent objects, NeRF produces severe floating artifacts due to its attempt to explain view-dependent appearances.
%
Experiments verify that our proposed \ournerf can effectively get rid of floater artifacts without additional input.% or significant computational overhead. 


In summary, our contributions include the following:
\begin{itemize}[leftmargin=0.16in, topsep=2pt,itemsep=-1ex,partopsep=1ex,parsep=1ex]
    \item We propose a concise method for decomposing view-independent and view-dependent appearance during NeRF training and eliminate the interference of view-dependent appearance.
    \item We propose a geometric correction procedure performed on each traced ray during inference to refine the density estimation and better tackle the floater artifacts.
    \item Extensive experiments and ablations verify the effectiveness of our core designs and results in improvements over the vanilla NeRF and other state-of-the-art alternatives.
    %without additional computational resources or other inputs.
\end{itemize}





\vspace{-1ex}
\section{Definitions and Background}
\vspace{-1ex}
\section{Related Work}

 %Most MMLA studies so far have primarily focused on developing prototypes and testing the functionality of different combinations of sensors and analytics approaches \citep{shankar2018review, mu2020multimodal, noroozi2020multimodal}. 
In this section, we review the most recent systematic literature reviews on MMLA and related works that have identified several prominent logistical, privacy and ethical challenges that need to be addressed for this promising area to remain relevant and have an actual impact on educational practices.

%MMLA and multimodal data have received increased attention from the learning analytics and educational technology communities as a promising research direction that holds the potential to generate meaningful insights about teaching, and learning in partially and non-computer mediated educational contexts \citep{sharma2020multimodal, chango2022review}. 
% \citep{alwahaby2021evidence, yan2022scalability}. 

%\subsection{Practical Challenges}

%The practical challenges of MMLA are mainly associated with the lack of well-reported and large-scale studies that structurally assess the influence of MMLA innovations on actual educational practice. For example, \citet{alwahaby2021evidence} reviewed 100 MMLA articles and concluded that most of the empirical evidence presented in prior studies remains descriptive, correlational, or anecdotal, with little strong causal evidence regarding the impacts of MMLA innovations on real-world educational practices. 


%Consequently, although the alignment between MMLA innovations and learning design should be one of the foundations for developing MMLA innovations \citep{cukurova2020promise, ochoa_multimodal_2022}, such alignments are rarely considered or reported in the existing literature, as evidenced in recent reviews \citep{sharma2020multimodal, praharaj2021literature}. 
%Likewise, the lack of MMLA studies that closed the LA loop by providing feedback to students or insights to teachers during educational practices instead through post-hoc research-focused interviews or surveys also hindered the understanding of MMLA innovations' actual impacts on learning and teaching outcomes \citep{yan2022scalability}. %Therefore, in-depth insights on aligning MMLA innovations with learning designs, relevant theories, and educational stakeholders are urgently needed to ensure future MMLA studies do not deviate from the ultimate goals of learning analytics \citep{gavsevic2015let}.

\subsection{Logistical Challenges}
Most MMLA studies so far have primarily focused on developing prototypes and testing the functionality of different combinations of sensors and analytics approaches \citep{shankar2018review, mu2020multimodal, noroozi2020multimodal}. Yet, many concerns have been raised regarding the logistical challenges that can emerge when moving from controlled settings to in-the-wild MMLA deployments such as the added intrusiveness of sensing devices and complexity in their installation and orchestration \citep{chua2019technologies}. \citet{yan2022scalability} systematically reviewed these logistical issues and identified a relatively low level of technology readiness regarding existing MMLA innovations, resulting in heavy reliance on the onsite support of researchers or technicians. This  undermines the sustainability of these systems and unnecessarily increases the complexity of the learning situation from the teachers' perspective. While most of the sensing technologies used in MMLA research can be purchased off-the-shelf, implementing these technologies in authentic physical learning spaces often requires extensive technical background for tasks such as physical installation, system integration, and modalities synchronisation \citep{crescenzi2020multimodal, shankar2018review, mu2020multimodal}. There is also a trade-off between data quality and affordability as most of the MMLA innovations that rely on mature sensing technologies, such as location sensors, eye-trackers, and biometric sensors, can be financially unscalable due to the high unit prices \citep{yan2022scalability}. Although low-cost alternatives are emerging \citep[e.g.,][]{ochoa2018rap, saquib2018sensei}, these technologies remain in the prototype and validation stages and often sacrifice accuracy or portability for affordability. 

Likewise, the lack of MMLA studies that have closed the LA loop by providing some form of end-user interface to students or insights to teachers make it harder for educational stakeholders to weigh the benefits against the potential added complexity to their already rich educational ecologies \citep{yan2022scalability}. Although the alignment between MMLA innovations and learning design should be one of the foundations for developing MMLA innovations \citep{cukurova2020promise, ochoa_multimodal_2022}, such alignments are rarely considered or reported in the existing literature, as noted in recent literature reviews \citep{sharma2020multimodal, praharaj2021literature}. This can undermine teacher and student confidence, if they do not understand how the MMLA system aligns with their teaching practices or learning outcomes. 

All of these challenges are hallmarks of emerging HCI infrastructures that must be co-evolved with work practices. This in-the-wild MMLA deployment offered the opportunity to study how both educational and technical stakeholders learnt to work together to address the challenges.%Gaining insights regarding this trade-off between data quality and affordability could benefit educational researchers and practitioners when evaluating their budgets against the type of educational insights they are trying to capture. 

\subsection{Privacy Challenges}
As a research area that benefits from the data collection opportunities enabled by various sensing technologies, the privacy issues surrounding the adoption of MMLA innovations are the focus of critical debate. \citet{crescenzi2020multimodal} emphasised the need to consider the privacy implications of using sensing technologies to generate analytics about children's activity. Such implications have also been identified by students and teachers who have expressed concerns regarding the security of their data \citep{mangaroska2021challenges, kasepalu2021teachers}. These privacy implications of MMLA innovations have been under-investigated in the literature \citep{Alwahaby2022, yan2022scalability, Oviatt2018challenges}. Specifically, while most works published in MMLA  mention that informed consent was obtained from participants, none of the existing works has elaborated on the consenting strategies they adopted, which could contribute valuable insights regarding data security measures for protecting individual privacy and maximising data autonomy (e.g., individuals' autonomy of removing their data from the database) \citep{beardsley2020enhancing}. Additionally, while most of MMLA innovations endeavour to provide dashboards and visualisations for supporting educational practices, privacy issues regarding who has the right to see these visualisations  remain unclear, especially in the contexts of collaborative learning where, in most cases, individuals' personal trace data, even anonymised (e.g., masking students' identity with numbers or colours), could remain identifiable when used for provoking reflections at a group-level, since other students typically have the contextual knowledge to decode anonymised representations \citep{mangaroska2021challenges, Alwahaby2022}. Providing additional empirical evidence on educational stakeholders' perspectives of these privacy-related issues could potential benefit the on-going development of MMLA, and is a particular focus of this study.

\subsection{Ethical Challenges}
Beyond logistical and privcy issues, the potential biases in analytics, and cognitive dissonances that may be caused by the inconsistency between individuals' observations and generated insights, could also undermine the potential benefits of MMLA innovations \citep{ferguson2016guest,Oviatt2018challenges}. Such issues are vital as the accuracy of the existing MMLA-based predictive models and early-warning systems are far from suitable for practical deployment (e.g., rarely above 80\% accuracy), and these models have mostly been developed and evaluated based on relatively small sample sizes (i.e., with n < 50) \citep{yan2022scalability}. These small sample sizes combined with the poor reporting standards found in the existing MMLA literature could also mask potential algorithmic biases that may disadvantage certain minority groups of students as replicating these studies remain difficult without adequately reported methodologies \citep{luzardo2014estimation, yan2022scalability}. Additionally, \citet{Alwahaby2022} also highlighted the significant concerns regarding the need to enhance trust and data transparency within MMLA systems and \citet{yan2021footprints} suggested that more research needs to be done to assess the potential risk of making decisions with incomplete multimodal data.%Additionally, using unsupervised machine learning techniques to cluster and label students may also induce the potential risk of discrimination, where certain labels (e.g., at-risk)  could negatively impact learners' self-esteem and educators' expectations \cite{higgins2002stages}. 
Consequently, understanding the ethical practices of using these analytics is also essential but rarely considered in prior literature \citep{selwyn2019s} and requires the participation of key educational stakeholders such as students and educators \citep{Oviatt2018challenges}.  A large-scale in-the-wild study opens new opportunities to study approaches to these ethical challenges under more authentic conditions than has been reported to date. 

\subsection{Contribution to HCI and Research Question}
Against the literature reviewed above we formulate the following research question (RQ) that guided our study: 

\textit{\textbf{RQ:} What logistical, privacy and ethical challenges emerge from a complex MMLA, in-the-wild study that closes the analytics loop by providing direct feedback to students?}

In addressing this question, the contribution of this paper is a set of lessons learnt regarding how such challenges were, or could have been, addressed in the context of a two-year deployment of a MMLA system in an authentic educational scenario. The implications of this study should assist researchers, developers and designers in making informed decisions about the effective deployment of innovations that involve the use of ubiquitous computing technologies, sensing devices and artificial intelligence (AI) algorithms to augment teaching and learning in physical spaces. 

% Reviews I reckon you already cited in the Scalability paper and which I used in the intro
% \cite{sharma2020multimodal}
% \cite{crescenzi2020multimodal}
% \cite{chua2019technologies}
% \cite{alwahaby2021evidence}

% Note for Jimmie - Other SLR on MMLA reviews to be inlcuded: 
% \cite{noroozi2020multimodal}
% \cite{shankar2018review}
% \cite{praharaj2021literature}
% \cite{mu2020multimodal}
% \cite{yan2022scalability} %of course!
% \cite{chango2022review}


\vspace{-1ex}
\section{Multimodal Retrieval-Augmented Generation}
\vspace{-1ex}

For each modality, there are different retrieval and synthesis procedures, targeted tasks, and challenges. Therefore, we discuss relevant methods by grouping them in terms of modality, including image, code, structured knowledge, audio, and video.

% \red{For each section, we use a different set of keywords to search in both ACL Anthology and Google Scholar and manually filtered the resulting papers. Details on the keywords used and the number of papers can be found in} \Cref{subsec:search}.

\vspace{-1ex}
\subsection{Image}
Incorporating image data with text information has long been a crucial research topic, as a considerable amount of world knowledge is stored exclusively in images. % todo: to cite prev research, and stats on visual information

% todo: make a table for easier comparison
% FIBER, Blip-2
Recent advances on pretrained models shed light on general image-text multi-modal models:
 Flamingo \cite{flamingo} can generate comprehensive captions from input images. FIBER \citep{dou2022coarse} proposes a two-stage vision-language (VL) pre-training strategy benefiting different levels of VL tasks. DALL-E \cite{DALL-E} and Parti \cite{parti} can generate images based on given text instructions. CM3 \cite{CM3} models both text and image for its input and output. Blip-2 \citep{li2023blip} bootstraps language-image pre-training from off-the-shelf frozen vision and language models. 
 
 However, these models require huge computational resources for pre-training and large amounts of model parameters --- as they need to memorize vast world knowledge, such as what chinchillas look like and where they commonly habitat. More critically, such models cannot efficiently deal with new or out-of-domain knowledge. To this end, multiple retrieval-augmented works have been proposed to better incorporate external knowledge from images and text documents. 

%PICa, Re-imagen
 Towards open-domain visual question answering (VQA), RA-VQA \cite{RA-VQA} jointly trains the document retriever and answer generation module by approximately marginalizing predictions over retrieved documents. It first uses existing tools of object detection, image captioning, and optical character recognition (OCR) to convert target images to textual data. Then, it performs dense passage retrieval (DPR)~\citep{dpr} to fetch text documents relevant to target image in the database. Finally, each retrieved document is concatenated with the initial question to generate the final prediction, similar to RAG \cite{lewis2020retrieval}. Besides external documents, PICa \citep{yang2022empirical} and KAT \cite{KAT} also consider LLMs as implicit knowledge bases and extract relevant implicit information from GPT-3. Plug-and-Play \cite{Plug-and-Play} retrieves relevant image patches by using GradCAM \cite{GradCAM} to localize relevant parts based on the initial question. It then performs image captioning on retrieved patches to acquire augmented context. Beyond text-only augmented context, MuRAG \cite{chen-etal-2022-murag} retrieves both text and image data and incorporates images as visual tokens. RAMM \cite{RAMM} retrieves similar biomedical images and captions, then encodes two modalities through different networks. 

 Apart from VQA, RA-transformer \cite{RA-transformer} and Re-ViLM \cite{Re-ViLM} generate more factual captions by retrieving relevant captions. Beyond retrieving images and text documents before generating text, Re-Imagen \citep{chen2022re} leverages a  multi-modal knowledge base to retrieve image-text pairs to facilitate image generation. RA-CM3 \cite{RA-CM3} can generate mixtures of images and text. It shows that retrieval-augmented image generation performs much better on knowledge-intensive generation tasks and opens up new capabilities such as multimodal in-context learning. % todo: RA-CM3 is a great work, should extend it a bit
 

\vspace{-1ex}
\subsection{Code}
Software developers attempt to search for relevant information to improve their productivity from large amounts of available resources such as explanations for unknown terminologies, reusable code patches, and solutions to common programming bugs~\cite{dsf}. Inspired by the progress of deep learning in NLP, a general retrieval-augmented generation paradigm has benefited a wide range of code intelligent tasks including code completion \cite{lu-etal-2022-reacc}, code generation \cite{zhou2022docprompting}, and automatic program repair (APR) \cite{nashidretrieval}. However, these approaches often treat programming languages and natural languages as equivalent sequences of tokens and ignore the rich semantics inherent to source code. To address these limitations, recent research work has focused on improving code generalization performance via multimodal learning, which incorporates additional modalities such as code comments, identifier tags, and abstract syntax trees (AST) into code pretrained models~\cite{codet5,unixcoder,codereviewer}. To this end, multimodal retrieval-augmented generation approach has demonstrated its feasibility in a variety of code-specific tasks, including:

\paragraph{\textbf{Text-to-Code Generation}} 
%ws edit
Numerous research studies have investigated the utilization of relevant codes and associated documents to benefit code generation models. A prominent example is REDCODER \cite{parvez-etal-2021-retrieval-augmented}, which retrieves the top-ranked code snippets or summaries from an existing codebase, and aggregates them with source code sequences to enhance the generation or summarization capabilities. As another such approach, DocPrompting \citep{zhou2022docprompting} uses a set of relevant documentation as in-context prompts to generate corresponding code via retrieval. In addition to these lexical modalities, RECODE~\cite{DBLP:conf/emnlp/HayatiOAYTN18} proposes a syntax-based code generation approach to reference existing subtree from the AST as templates to direct code generation explicitly.
%Original: Numerous research studies have investigated how to retrieve relevant codes and documents to enhance (both end-to-end and syntax-based) code generation models. A prominent example is REDCODER \cite{parvez-etal-2021-retrieval-augmented} which augments the source input code sequence with relevant code snippets or summaries retrieved from a codebase. DocPrompting \citet{zhou2022docprompting} is another such example that retrieves the relevant documentation pieces given a natural language query and generates code based on the query and the retrieved documentation. In addition, RECODE~\cite{DBLP:conf/emnlp/HayatiOAYTN18} is an approach that explores retrieving codes to support syntax-based code generation models. Specifically, it explicitly references existing subtrees from the Abstract Syntax Tree (AST) of codes as templates to guide the code generation models.

%\ws{For your reference, how chatgpt refines this paragraph: Numerous research studies have investigated the utilization of relevant codes and documents to enhance code generation models. A prominent example is REDCODER \cite{parvez-etal-2021-retrieval-augmented}, which entails augmenting the input sequence to the code generation model with relevant code snippets or summaries retrieved from a database. DocPrompting \cite{zhou2022docprompting} is a code generation approach that goes beyond code retrieval to include (1) retrieving relevant documentation pieces based on a natural language query, and (2) generating code based on the query and the retrieved documentation. Additionally, the RECODE approach \cite{DBLP:conf/emnlp/HayatiOAYTN18} is a syntax-based code generation technique that uses existing subtrees from the AST as templates to direct code generation.}

\paragraph{\textbf{Code-to-Text Generation}} 
%ws edit
Retrieval-based code summarization methods have been studied extensively. For example, RACE~\cite{DBLP:conf/emnlp/ShiW0DZHZ022} leverages relevant code differences and their associated commit message to enhance commit message generation. Besides, RACE calculates the semantic similarity between source code differences and retrieved ones to weigh the importance of different input modalities. Another retrieval-based neural approach is Rencos~\cite{10.1145/3377811.3380383}, which retrieves two similar code snippets based on the aspects of syntactic-level similarity and semantic-level similarity of a given query code. 
These similar contexts are then incorporated into the summarization model during the decoding phase.
This idea is further explored by~\citet{liu2021retrievalaugmented}, where retrieved code-summary pairs are used to augment the original code property graph~\cite{DBLP:conf/sp/YamaguchiGAR14} of source code via local attention mechanism. To capture the global semantics for better code structural learning, a global structure-aware self-attention mechanism~\cite{DBLP:conf/emnlp/ZhuLZQZZ19} is further employed.

\paragraph{\textbf{Code Completion}}
%\rc{``Methods that improve'' instead of ``benefiting''?} 
%Methods that improve code completion tasks 
%\rc{tasks?} 
%\cite{mcconnell2004code} by retrieval have 
%\rc{have?} 
%have gained increasing attention in recent years. 
%\citet{hashimoto2018retrieve} introduce the very first trainable approach to retrieve a training example based on the input and then edit 
%\rc{edit?} 
%it for code auto-completion. 
%ReACC \cite{lu-etal-2022-reacc} is another approach that employs both lexical copying and referring to code with semantic similarity to the input to enhance it for the code completion generator. More recently, \citet{ding2022cocomic} introduce CoCoMIC that targets at 
%ws edit
Recent advances in retrieval-based code completion tasks~\cite{mcconnell2004code} have gained increasing attention. Notably,~\citet{hashimoto2018retrieve} adapt the retrieve-and-edit framework to improve the model's performance in code auto-completion tasks. To address practical code completion scenarios, ReACC~\cite{lu-etal-2022-reacc} takes both lexical and semantic information of the unfinished code snippet into account, utilizing a hybrid technique to combine a lexical-based sparse retriever and a semantic-based dense retriever. First, the hybrid retriever searches for a relevant code from the codebase based on the given incomplete code. Then, the unfinished code is concatenated with the retrieval, and an auto-regressive code completion generator will generate the completed code based on them. In order to address project relations, CoCoMIC~\cite{ding2022cocomic} decomposes a code file into four components: \emph{files}, \emph{global variables}, \emph{classes}, and \emph{functions}. It constructs an in-file context graph based on the hierarchical relations among all associated code components, forming a project-level context graph by considering both in-file and cross-file dependencies. Given an incomplete program, CoCoMIC retrieves the most relevant cross-file entities from its project-level context graph and jointly learns the incomplete program and the retrieved cross-file context for code completion.   

%Recently, \citet{ding2022cocomic} introduce CoCoMIC that targets at 
%\rc{to -> at} 
%effectively locating and retrieving 
%\rc{locating and retrieving} 
%relevant cross-file codes 
%\rc{use ``cross-file codes'' instead of ``codes across files''? could be clearer} 
%with the given context, and incorporate cross-file context to jointly learn the in-file and cross-file contexts for code completion.
% \rc{should change the ``context'' to ``contexts''?} for code completion.
%\ws{For your reference, how chatgpt refines this paragraph: Retrieval-based methods for improving code completion tasks \cite{mcconnell2004code} have gained increasing attention in recent years. Notably, Hashimoto et al. \cite{hashimoto2018retrieve} introduced the first trainable approach that retrieves a training example based on the input and then modifies it for code auto-completion. ReACC \cite{lu-etal-2022-reacc} is another approach that employs both lexical copying and retrieval of code with similar semantics to enhance the input for code completion generation. More recently, Ding et al. \cite{ding2022cocomic} proposed CoCoMIC, a method that effectively locates and retrieves relevant cross-file codes given a context and jointly learns the in-file and cross-file contexts for code completion.}

\paragraph{\textbf{Automatic Program Repair (APR)}}
Inspired by the nature that a remarkable portion of commits is comprised of existing code commits~\cite{Martinez2014DoTF}, APR is typically treated as a search problem by traversing the search space of repair ingredients to identify a correct fix~\cite{DBLP:conf/icse/QiMLDW14}, based on a redundancy assumption~\cite{White2019SortingAT} that the target fix can often be reconstructed in the search space. Recent studies have shown that mining relevant bug-fix patterns from existing search space~\cite{simfix} and external repair templates from StackOverflow~\cite{DBLP:conf/wcre/LiuZ18} can significantly benefit APR models.~\citet{joshi2022repair} intuitively rank a collection of bug-fix pairs based on the similarity of error messages to develop few-shot prompts. They incorporate compiler error messages into a large programming language model Codex~\cite{codex} for multilingual APR. CEDAR~\cite{nashidretrieval} further extends this idea to retrieval-based prompts design using relevant code demonstrations, comprising more modalities such as unit test, error type, and error information. Additionally, ~\citet{jin2023inferfix} leverage a static analyzer Infer to extract error type, error location, and syntax hierarchies~\cite{clement2021long} to prioritize the focal context. Then, they retrieve semantically-similar fixes from an existing bug-fix codebase and concatenate the retrieved fixes and focal context to form the instruction prompts for program repair.

\paragraph{\textbf{Reasoning over Codes as Intermediate Steps}} 
%\ws{I have refined this part, please go through it and feel free to correct where I might misunderstand your intention, thanks Long.} \xl{I think they look great. Thanks Weishi!}
While large language models (LLMs) have recently demonstrated their impressive capability to perform reasoning tasks, they are still prone to logical and arithmetic errors~\cite{gao2022pal, chen2022program, madaan2022language}. To mitigate this issue, emerging research works have focused on using LLMs of code (e.g., Codex \cite{codex}) to generate the code commands for solving logical and arithmetic tasks and calling external interpreters to execute the commands to obtain the results. Notably, \citet{gao2022pal} propose to generate Python programs as intermediate reasoning steps and offload the solution step to a Python interpreter. Additionally, \citet{chen2022program} explore generating chain-of-thought (CoT) \cite{wei2022chain} for not only text but also programming language statements as reasoning steps to solve the problem. During the inference phase, answers are obtained via an external interpreter. Similarly, \citet{lyu2023faithful} propose Faithful CoT that first translates the natural language query to a symbolic reasoning chain and then solves the reasoning chain by calling external executors to derive the answer. Another example is \citet{ye2023large}, which utilizes LLMs to decompose table-based reasoning tasks into subtasks, decouples logic and numerical computations in each step through SQL queries by Codex, and calls SQL interpreters to solve them (a process called "parsing-execution-filling").

LLMs of code are also known as good-structured commonsense reasoners, and even better-structured reasoners than LLMs \cite{madaan2022language}. As a result, prior studies have also investigated the idea of transforming structured commonsense generation tasks into code generation problems and employing LLMs of code as the solvers. One such work is CoCoGen \cite{madaan2022language} which converts each training sample (consisting of textual input and the output structure) into a Tree class in Python. The LLMs of code then perform few-shot reasoning over the textual input to generate Python code, and the Python code is then converted back to the original structure for evaluation. Besides, the success of LLMs of code such as Codex in synthesizing computer code also makes them suitable for generating formal codes. Motivated by this, \citet{wu2022autoformalization} propose to prove mathematical theorems by adopting Codex to generate formalized theorems from natural language mathematics for the interactive theorem prover Isabelle \cite{WenzelPN-TPHOLs08}.

%Prior works have attempted to address these reasoning tasks by converting them into code generation problems, and utilizing large language models of code (e.g., Codex \cite{codex}) to derive answers. For instance, CoCoGen \cite{madaan2022language} solves the structured commonsense generation tasks by converting the structure (typically a graph) into semantically equivalent code programs and employing Codex as the task solver. Besides, 

%Recent research has focused on using LLMs of code to output the code commands for solving logical and arithmetic tasks then calling external interpreters to execute the commands and derive the results. Notably, \citet{gao2022pal} propose to generate Python programs as intermediate reasoning steps and offload the solution step to a Python interpreter, while \citet{chen2022program} explore generating chain-of-thought \cite{wei2022chain}(CoT) for both text and programming language statements. During the inference phase, answers are obtained via an external interpreter. Similarly, \citet{lyu2023faithful} propose Faithful CoT, which decomposes reasoning tasks into two stages: (1) translating natural language query to symbolic reasoning chain; (2) solving reasoning chains by calling external executors. Another example is \citet{ye2023large}, which leverages LLMs to decompose table-based reasoning tasks into subtasks, decouples logic and numerical computations in each step through SQL queries by Codex, and calls SQL interpreters to solve them (a process called "parsing-execution-filling"). 
%








\vspace{-1ex}
\subsection{Structured Knowledge}
\vspace{-1ex}
% To increase factual grounding and reduce hallucinations for PLMs, many research papers have attempted to augment generation by incorporating more structured knowledge, such as knowledge graphs, tables, and databases.

An open challenge in generative models is hallucination, where the model is likely to output false information.
% seemingly plausible sentences that do not conform to the ground-truth facts. 
% Researchers have denoted that language models, while only relying on internal knowledge (pre-trained weights), fail to recall accurate details when functioning as a knowledge base in question-answering tasks \citep{ye2022unreliability,creswell2022selection}. 
Thus, A potential solution is to ground generation with retrieved structured knowledge, such as knowledge graphs, tables, and databases.
% , often represents how knowledge from different domains is integrated. They could function as a reliant source of truth to enhance factuality. 

% QA task
\noindent\textbf{Question Answering (QA)}~~~
A natural setting to use knowledge is QA. To augment \emph{Knowledge Base (KB) QA} by extracting the most relevant knowledge, \citet{hu-etal-2022-logical} uses dense retrieval and \citet{liu-etal-2022-uni} uses a cross-encoder ranker. \citet{shu-etal-2022-tiara} employs multi-grained retrieval to extract relevant KB context and uses constrained decoding to control the output space. In \emph{table QA}, \citet{nan-etal-2022-fetaqa} proposes a dataset that requires retrieving relevant tables for answer generation. \citet{pan-etal-2021-cltr} then proposes a method that uses a transformer-based system to retrieve the most relevant tables and locate the correct cells. Moreover, to improve \emph{Video QA}, \citet{hu2023reveal} retrieves from Knowledge Graph (KG) encodings stored in the memory. The most prominent RAG usage remains in \emph{open-domain QA}, where multiple datasets are proposed \citep{lin-etal-2023-fvqa, ramnath-etal-2021-worldly}. For retrieval, \citet{ma-etal-2022-open-domain} verbalizes the KG and then uses dense passage retrieval. \citet{fan2019using, gupta-etal-2018-retrieve} encodes KG information into dense representations. \citet{pramanik2021uniqorn, jin2022heterformer} builds graph embeddings to retrieve question-relevant evidence. \citet{xu-etal-2021-fusing, baek2023knowledge} use semantic similarity and text-matching methods.
{Synthesis can occur at different stages. At the input stage,} \citet{xu-etal-2021-fusing, baek2023knowledge} feed in the retrieved contexts as additional inputs or prompts to the PLM. \citep{ma-etal-2022-open-domain, fan2019using} adapt the generator to accept the context representations as inputs.  {At model inference stage, }an interesting work is \citet{hu-etal-2022-empowering}, which inserts an interaction layer into PLMs to guide an external KG reasoning module.

\noindent\textbf{General text generation}~~~
External knowledge retrieval can improve general text generation to be more factually grounded. \citet{liu-etal-2022-relational} presents a memory-augmented approach to condition an autoregressive language model on a knowledge graph (KG). {During inference, }\citet{tan-etal-2022-tegtok} selects knowledge entries through dense retrieval and then injects them into the input encoding and output decoding stages in pretrained language models (PLMs). For \emph{domain-specific text generation}, \citet{frisoni-etal-2022-bioreader, yang-etal-2021-writing, li2019knowledgedriven} retrieve medical report chunks or report templates {to augment input prompts. Then, they} use self-devised decoders or graph transformers to generate grounded reports. To improve interpretability, RAG could be used to select facts as interpretable reasoning paths \citep{aggarwal-etal-2021-explanations, jansen-ustalov-2019-textgraphs}. 
% For image captioning, \citet{lu-etal-2018-entity} retrieves named entities via hashtags to provide richer contextual information. 
Moreover, RAG is especially useful for low-resource generation tasks, such as question generation \citep{yu-jiang-2021-expanding, xin-etal-2021-enhancing, gu-etal-2019-extract}, document-to-slide \citep{sun-etal-2021-d2s}, table-to-text \citep{su-etal-2021-shot-table}, counterargument generation \citep{jo-etal-2021-knowledge-enhanced}, entity description generation \citep{cheng-etal-2020-ent} and text-based games \citep{murugesan-etal-2021-efficient}.

% To aid LLMs, some papers design task-specific queries to retrieve structured knowledge by fine-tuning. 
Recent research has attempted to reduce hallucinations in LLMs by leveraging external structured knowledge. For example, during fine-tuning, LaMDA \citep{thoppilan2022lamda} learns to consult external knowledge sources before responding to the user, including an information retrieval system that can retrieve knowledge triplets and web URLs. Some papers treat the generative model (often large language models) as black-box and retrieve structured information without fine-tuning. For example, BINDER \citep{cheng2022binding} uses in-context learning to output designed API calls that retrieve question-relevant columns from tables.

\noindent\textbf{Reasoning with knowledge}~~~
By selecting knowledge, reasoning tasks can be solved in a more grounded and interpretable way. 
% \citet{moon-etal-2019-opendialkg} studies the Open-ended Dialog task parallel with a KG corpus. It proposes a model that learns the symbolic transitions of dialog contexts as structured traversals over KG, and predicts natural entities to introduce given previous dialog contexts via a graph path decoder. 
To generate an entailment tree explanation for a given hypothesis, \citet{neves-ribeiro-etal-2022-entailment} retrieves from textual premises iteratively and combines them with generation. \citet{yang-etal-2022-logicsolver} proposes a math reasoner that first retrieves highly-correlated algebraic knowledge and then passes them as prompts to improve the semantic representations for the generation task. With the recent advances in LLMs, \citet{he2022rethinking, li2023chain} retrieve from KG and KB, such as Wikidata, based on reasoning steps obtained from the chain-of-thought (CoT) prompting \citep{wei2022chain}.

\noindent\textbf{Knowledge-grounded dialogue}~~~
Dialogue generation based on relevant tables and knowledge bases has been a practical research application \citep{wu-etal-2020-diverse, li2022opera, nakamura-etal-2022-hybridialogue, gao-etal-2022-comfact, lu-etal-2023-statcan}. To tackle the challenge, \citet{li-etal-2022-knowledge} and \citet{galetzka-etal-2021-space} retrieve relevant knowledge, process it into a dense representation and incorporate it into dialogue generation. On top of dense representations, \citet{gu-etal-2020-filtering} and \citet{jung-etal-2020-attnio} leverage attention mechanisms to flexibly adjust which knowledge to depend on during generation. 
% \citet{} learns a fact-linking model to improve dialogue generation. 
Some methods~\citep{zhang-etal-2021-kers-knowledge, dziri-etal-2021-neural, chen-etal-2020-airconcierge} first generate subgoals or responses and then use them to retrieve relevant knowledge. The retrieved knowledge then helps amend previous responses. 
Besides knowledge, \citet{cai-etal-2019-retrieval} and \citet{wu-etal-2020-improving-knowledge} improve dialogue response generation by retrieving templates or prototype dialogues {to augment inputs}. Recently, \citet{kang2023knowledge} retrieves relevant subgraphs from KGs, and then utilizes contrastive learning to ensure that the generated texts have high similarity to the subgraphs.
% \citet{li2022opera} propose a unified dialog model that learns to query pre-defined databases with belief states, which is a list of triplets.

% As the format of structured knowledge departs from the natural texts seen by LLMs during pre-training, how to effectively retrieve and synthesize it for generation has been an open challenge. \citet{xie-etal-2022-unifiedskg} represent an early attempt, where all formats of knowledge, including tables, triplets, and ontology, are linearized into text format and fed into the LLM without retrieval. Such methods, however, are limited to the acceptable context length of the PLM and are often computationally expensive.


By retrieving from relevant sources, RAG not only improves factuality but also provides the grounding contexts while generating, thus addressing interpretability and robustness concerns. With the potential to handle more information types with recent advances in LLMs \citep{gpt4}, RAG with structured knowledge could be further enhanced. There are still challenges to be addressed. For example, there could be new designs for better retrieval systems that could promote efficient interactions suitable for diverse knowledge bases. Synthesizing this information correctly is also an open challenge, where it is hard to decide which parts need augmenting in the textual outputs.

\vspace{-1ex}
\subsection{Audio}
\vspace{-1ex}

Audio RAG applications include audio data augmentation, music captioning, music and text generation, and speech recognition. It could be a promising future direction \citep{li2022survey}.
% There currently exist several research that use audio information to augment generation. 

\noindent\textbf{Text-audio data augmentation}~~~
For text-audio tasks, one of the most important challenges is the lack of training data on audio-text pairs. Therefore, retrieving audio and textual cues can alleviate the data scarcity problem and improve performance. In audio captioning, which aims at translating the input audio into its description, \citet{koizumi2020audio} retrieves guidance captions similar to the input audio from the training set. Then, the retrieved guidance captions are fed into a PLM to help generate new captions, which improves generation performance. To augment scarce speech translation (ST) data, \citet{zhao-etal-2023-generating} proposes SpokenVocab, a technique to convert machine translation (MT) data to synthetic ST data. To form synthetic speech, SpokenVocab retrieves and stitches audio snippets, corresponding to words in an MT sentence. Experiments show that stitched audio snippets can improve translation quality. \citet{kim2023prefix} leverages a PLM to tackle the data scarcity issue. It retrieves features from the input audio, maps them to continuous vectors using mapping networks, and uses vectors as prefixes for prefix tuning the PLM. With the additional information from retrieved audio, it outperforms previous methods. In text-to-audio generation, \citet{huang2023make} applies audio-text retrieval to get pseudo text prompts, which enhance audio generation in data-scarce scenarios. To augment the argumentation mining (AM) task in political debates, \citet{mestre-etal-2023-augmenting} integrates audio features into PLMs, which improves performance when data is scarce.

\noindent\textbf{Music captioning}~~~
Music captioning is the task of generating a text description or lyrics given the music audio. And RAG is explored to learn better audio-lyric alignment. \citet{manco2021muscaps} proposes the first music audio captioning model, MusCaps. Firstly, a pretrained multimodal encoder obtains audio representations that retrieve musical features in the input. As the pretraining bridges the gap between the audio modality and textual understanding, the method improves task performance. \citet{he2022recap} learns an audio-lyric alignment through contrastive learning, which results in a higher-quality generation of captions for music.
% When audio information is the input for the generation task, retrieval augmentation is explored to learn the audio and lyrics alignment through contrastive learning \citep{he2022recap}, which results in a higher-quality generation of captions for music. Moreover, retrieval of key/value pairs from the external knowledge catalog is used for automatic speech recognition tasks \citep{chan2023using}. 

\noindent\textbf{Music generation}~~~
\citet{royal2020deep} uses deep neural hashing to retrieve music building blocks and then performs generation by using the current music segment to retrieve the next. In automatic speech recognition (ASR), \citet{chan2023using} uses a k-nearest neighbor (KNN) approach to retrieve external knowledge related to the audio and text embeddings. The retrieved knowledge significantly reduces domain adaptation time for ASR.

% In cases where audio information is the output, retrieval is applied in a music generation system with deep neural hashing that encodes the music segments \citep{royal2020deep}. Audio-text retrieval is also applied to produce candidates in the process of pseudo prompt enhancement for text-to-audio generation \citep{huang2023make}. 
% Although there is a limited amount of research work which focuses on retrieval augmented generation tasks involving the audio, it could be a promising future direction \citep{li2022survey}.

The audio modality is closely intertwined with other modalities, such as video. Therefore, recent advancements in using
% in audio-text retrieval techniques  and uses of 
audio features for text-video retrieval \citep{falcon2022feature, mithun2018learning} can benefit RAG tasks involving other modalities. Moreover, although audio-text retrieval has been a long-standing task \citep{liu2015combining, milde-etal-2016-ambient, milde-etal-2016-demonstrating}, exploring recently discovered techniques \citep{hu2022audio, lou2022audio, koepke2022audio} could lead to further improvements.

\vspace{-1ex}
\subsection{Video}
\vspace{-1ex}
Currently, very few works have explored video retrieval for generative tasks, e.g., video captioning. However, the recent studies on dense video representation learning can be useful when developing video knowledge-enhanced generative approaches in the future. \citet{bogolin2022querybank} propose a query bank normalization method for cross-modal text-video retrieval. Besides, Cap4Video \cite{bogolin2022cap4video} and CLIP-ViP \citep{xue2022clip-vip} are data augmentation frameworks that utilize the web-scale pre-trained knowledge to enhance text-video retrieval pre-training. Besides, some works also try to introduce fine-grained interaction between different modalities~\citep{yang23vid2seq,wang2021t2vlad}. However, these methods still own a significant gap to be the foundation of retrieval-augmented generation models due to the cost of building a video index for knowledge search.

\vspace{-1ex}
\section{Future Directions}
\vspace{-1ex}
\subsection{Retrieval Augmented Multimodal Reasoning}
%including both cot and other formats of reasoning (kosmos-1)

\begin{quote}
\small 
\emph{The words of the language, as they are written or spoken, do not seem to play any role in my mechanism of thought. The psychical entities which seem to serve as elements in thought are certain signs and more or less clear images which can be "voluntarily" reproduced and combined. --- Albert Einstein}
\end{quote}

One potential application of multimodal information retrieval is multimodal reasoning. \citet{lu2022learn} first introduce ScienceQA, a large-scale multimodal science question dataset annotated with lectures and explanations. Based on this benchmark, \citet{zhang2023multimodal} propose Multimodal Chain-of-Thought (Multimodal-CoT) which incorporates language and vision modalities into a two-stage (rationale generation and answer inference) framework, surpassing GPT-3.5 by a large margin with a much smaller fine-tuned model. Similar to \citet{zhang2023multimodal}, kosmos-1 \citep{huang2023language} breaks down multimodal reasoning into two steps. It first generates intermediate content as the rationale based on visual information, and then uses the generated rationale to induce the result. However, both methods may have difficulties in understanding certain types of images (e.g., maps), which could be mitigated by retrieving relevant informative image-text pairs. We hope that future work can pay more attention to how to effectively and efficiently combine multimodal reasoning with multimodal retrieval.
% \xl{$citet$ is often counted as plural (as it means many authors) according to my understanding. Please adjust the verbs if needed.}

\subsection{Building a Multimodal Knowledge Index}
% Assigned with Fangkai

In order to facilitate retrieval augmented generation, one of the most fundamental aspects is the building of a multimodal knowledge index. The goal of building a knowledge index is twofold: Firstly, the dense representation should support low storage, dynamic updating of the knowledge base, and accurate search. Secondly, it could enable faster search speed with the help of local sensitive hashing~\citep{data-mining}, which combats scaling and robustness concerns when the knowledge base is scaled up extremely.

Currently, the dense representation for text snippets has been widely studied for documents~\cite{karpukhin-etal-2020-dense,gao-callan-2021-condenser,gao-etal-2021-simcse}, entities~\citep{sciavolino-etal-2021-simple,lee-etal-2021-learning-dense}, and images~\cite{clip}. Besides, there are also many studies optimizing dense representations in an end-to-end manner~\cite{lewis2020retrieval}.
Nevertheless, few works~\citep{chen-etal-2022-murag} have explored building 
a multimodal index at the same time for downstream generation, and are also limited in text and image. How to map a multimodal knowledge index into a unified space is still a long-term challenge.

\subsection{Pre-training combined with multimodal retrieval}

With the goal of better aligning the abilities to handle different modalities in a pre-trained model, there could be future work built on employing retrieval-based approaches during pre-training. Currently, there have been many methods that fine-tune the pre-trained generative model for retrieval. For example, LaMDA \citep{thoppilan2022lamda} can call an external toolset for fine-tuning, including an information retrieval system, a calendar, and a calculator. Similarly, during fine-tuning, Toolformer \citep{schick2023toolformer} augments models with API calls to tools including a question-answering system and a Wikipedia search engine. 

During pretraining, if similar retrieval abilities are leveraged, the generative model would be able to interact with retrieval tools better. Thus, it could output more grounded information, provide relevant contexts to users, and update their information accordingly. When new information comes in, the generative model would be able to effectively retrieve from an up-to-date external base instead of relying solely on pre-trained weights. This advantage also expands to handling robustness in out-of-domain questions.

To incorporate retrieval with pre-training, there remains the challenge of developing appropriate datasets labeled with retrieval-based API calls. To tackle this challenge, LaMDA \citep{thoppilan2022lamda} uses labels developed by human annotators, which could be expensive to collect. Toolformer \citep{schick2023toolformer} uses a sampling and filtering approach for automatic labeling, which is inexpensive but could induce noise and bias. A potential solution is to use a neuro-symbolic approach such as \citet{davoudi2021toward}, which use prototype learning and deep-KNN to find nearest neighbors during training.

\vspace{-1ex}
\section{Conclusions}
\vspace{-1ex}
\section{Discussion and Limitations}

Although we can ablate concepts efficiently for a wide range of object instances, styles, and memorized images, our method is still limited in several ways. First, while our method overwrites a target concept, this does not guarantee that the target concept cannot be generated through a different, distant text prompt. We show an example in \reffig{limitation} (a), where after ablating {\menlo Van Gogh}, the model can still generate {\menlo starry night painting}. However, upon discovery, one can resolve this by explicitly ablating the target concept {\menlo starry night painting}. Secondly, when ablating a target concept, we still sometimes observe slight degradation in its surrounding concepts, as shown in \reffig{limitation} (c). 

\nupur{Our method does not prevent a downstream user with full access to model weights from re-introducing the ablated concept~\cite{ruiz2022dreambooth,kumari2022multi,gal2022image}. Even without access to the model weights, one may be able to iteratively optimize for a text prompt with a particular target concept. Though that may be much more difficult than optimizing the model weights, our work does not guarantee that this is impossible.}

Nevertheless, we believe every creator should have an ``opt-out'' capability. We take a small step towards this goal, creating a computational tool to remove copyrighted images and artworks from large-scale image generative models.


% Entries for the entire Anthology, followed by custom entries
\bibliography{anthology,custom}
\bibliographystyle{acl_natbib}

\appendix

\clearpage
\section{Appendix}
\label{sec:appendix}

\section{Stochastic Allocations of Indivisible Goods}\label{sec:app}
% title: what is the purpose of this section

In this section we consider a particular application of our results, for the problem of \textit{stochastic allocations of indivisible goods}. 
% title: basic model  
The setting postulates a set of $n$ agents $1,\ldots,n,$ and $m$ items, $1,\ldots,m,$ to be distributed amongst the agents.
A \emph{deterministic allocation} of the items to the agents is a mapping $A:[m]\rightarrow [n]$, determining which agent gets each item. 
% Given an allocation $A$, we denote by $A_i=A^{-1}(i)$ - the set of items allocated to $i$ in $A$. 
We denote by $\mathcal{A}$ the set of deterministic allocations. 
A \emph{stochastic allocation}, $d$, is a distribution over the deterministic allocations.  The set of all possible stochastic allocations is: 
\begin{align*}
    \mathcal{D} = \{d \mid p_d \colon \mathcal{A} \to [0,1], \sum_{A \in \mathcal{A}} p_d(A) = 1\}
\end{align*}   
Each agent $j$ is associated with a function $u_j \colon \mathcal{A} \to \mathbb{R}_{\geq 0}$ that describes its utility from a deterministic allocation.
Agents are assumed to care only about their own share (allowing us to use the following abuse of notation in which $u_j$ takes a bundle $b$ of items), their utilities are assumed to be normalized ($u_j(\emptyset) = 0$), monotone ($u_j(b_1) \leq u_j(b_2)$ if $b_1 \subseteq b_2$), and submodular ($u_j(b_1) + u_j(b_2) \geq u_j(b_1 \cup b_2) + u_j(b_1 \cap b_2)$ for any bundles $b_1,b_2$).
We also assume that utilities $(u_i)_{i=1}^n$ are given in the \emph{value oracle model}, meaning that we do not have a direct access to them, but only to an oracle that indicates the value of an agent from a given deterministic allocation.
Lastly, it is assumed that each agent has a positive utility from the set of all items.
% \eden{z1 > 0}
% \eden{should we explain somewhere that submodularity in the context of people's utilities makes a lot of sense? Maybe to explain what it is with diminishing returns?}
% \eden{should we say something about the value-oracle model? if so, where?}

% \eden{maybe to change "when agents have submodular utilities" to: "under these settings"?}
We prove that, in this setting, an approximately-optimal leximin solution with \emph{only} a multiplicative error can be obtained in polynomial time.
Specifically, we prove that a $\frac{1}{3}$-approximation\footnote{Throughout this section, we only discuss multiplicative approximations; so, for brevity, we use the term "$\multApprox$-approximation" to refer to "$(\multApprox,0)$-approximation".} can be obtained deterministically, whereas a $\frac{(e-1)^2}{e^2-e+1} \approx 0.52$-approximation can be obtained w.h.p.
As a reference point, it is worth noting that the problem of maximizing the egalitarian welfare in these settings has been shown to be NP-hard to approximate to a (multiplicative) factor better than than $1-\frac{1}{e} \approx 0.632$ \cite{kawase_max-min_2020}.
% \eden{should we say, as \textcite{kawase_max-min_2020}, that given an approximation algorithm with a multiplicative error of $\multError$ for welfare maximization, we obtain a approximate leximin with a multiplicative error of $\frac{\multError}{1-\multError +\multError^2}$?}.

% \footnote{Recall that a $(\multError,\additiveError)$-approximation has at most a multiplicative error of $\multError$ and at most an additive error of $\additiveError$. Therefore, the lower the parameters, the higher the accuracy.}




Given a stochastic allocation $d$, the expected utility of agent $j$ is given by
\begin{align*}
	E_j(d) = \sum_{A\in \mathcal{A}}p_d(A)\cdot u_j(A)
\end{align*}
The goal is to find a stochastic allocation that maximizes the set of functions $E_1,\ldots,E_n$. 
Formally, we consider the following problem:
\begin{align*}
	\lexmaxmin \quad &\{E_1(d), E_2(x), \dots E_n(d)\} \\
	s.t. \quad  & d \in \mathcal{D}
\end{align*}
% \eden{maybe $\maxlexmin$?} 
\eden{should write something about the output size, as \textcite{kawase_max-min_2020}}
That is, the feasible region is the set of stochastic allocations ($S = \mathcal{D}$) and the objective functions are the expected utilities ($f_i = E_i$ for any $i\in [N]$).

However, we shall see that an $\multApprox$-approximation to leximin is first and foremost an $\multApprox$-approximation to the egalitarian welfare. Therefore, the same hardness result applies to our problem as well.

\begin{lemma}
    If a solution is an $\multApprox$-approximation to leximin, then it is also an $\multApprox$-approximation to the egalitarian welfare.
\end{lemma}

% \eden{I removed the proof for now due to lack of space}
% \begin{proof}
%     Let $d \in \mathcal{D}$ be a stochastic allocation, and assume that it is an $\multApprox$-approximation to leximin. 
%     By definition, there is no solution that is $(\multApprox,0)$-preferred over it --- $d' \nAlphaBetaPreferredParams{\multApprox}{0} d$ for any $d' \in \mathcal{D}$.
%     Suppose by contradiction that $d$ is \emph{not} an $\multApprox$-approximation to the egalitarian welfare, and let $d^*$ be the optimal solution to this problem.
%     This means that the smallest objective value of $d'$ is less than smallest objective value of $d^*$ times $\multApprox$ --- $\valBy{1}{d} < \multApprox \cdot\valBy{1}{d^*}$.
%     But it follows that $d^*\alphaBetaPreferredParams{\multApprox}{0} d $;
%     for $k=1$, the required for $j<k$ is vacuously true 
%     and $\valBy{1}{d^*} > \frac{1}{\multApprox}\valBy{1}{d}$.
%     However, we know that the $d' \nAlphaBetaPreferredParams{\multApprox}{0} d$ for any $d' \in \mathcal{D}$, so it is true in particular for $d^*$. This is a contradiction.
% \end{proof}
As the proof is straightforward, it is omitted.


\textcite{kawase_max-min_2020} present an approximation algorithm that relates the problem of finding a stochastic allocation that approximates the egalitarian welfare, to the problem of finding a \emph{deterministic} allocation that approximates the \emph{utilitarian welfare} (i.e., the sum of utilities):
\begin{align*}
 \max \quad &\sum_{i=1}^n u_i(A)   \;\;
        \quad s.t. \quad   A \in \mathcal{A}  \tag{U1}\label{eq:utilitarian}
\end{align*}
% \erel{Can you write the maximization problem for utilitarian welfare?}

In this paper, we adapt their algorithm and prove the following relation to leximin:
\begin{theorem}
\label{th:app-main}
Suppose we are given a randomized algorithm that returns a deterministic allocation that approximates the utilitarian welfare with multiplicative error $\multError$ (with success probability $p$).
Then, Algorithm \ref{alg:basic-ordered-Outcomes} can be used to obtain a stochastic allocation that approximates leximin with a multiplicative error of at most $\frac{\multError}{1-\multError +\multError^2}$ (with the same probability).
\end{theorem}

\noindent Proving this theorem will yield two immediate results since there are known algorithms to approximate the utilitarian welfare when the agents' utility functions are monotone and submodular.
First, there are deterministic approximation algorithms with a multiplicative error $\frac{1}{2}$ \cite{Fisher1978, Buchbinder2019}, and therefore:
\begin{corollary}
    Algorithm \ref{alg:basic-ordered-Outcomes} can be used to obtain a stochastic allocation that approximates leximin with a multiplicative error at most 
    $
    \frac{0.5}{1-0.5+0.5^2} = 
    \frac{2}{3}$.
\end{corollary}
\noindent Second, there is a randomized approximation algorithm with a multiplicative error of  $\frac{1}{e}$ w.h.p \cite{vondrak_optimal_2008}, and therefore:
\begin{corollary}
    Algorithm \ref{alg:basic-ordered-Outcomes} can be used to obtain a stochastic allocation that approximates leximin with a multiplicative error at most $\frac{e}{e^2-e+1} \approx 0.48$ w.h.p.
\end{corollary}

The proof of Theorem \ref{th:app-main} is provided in Appendix \ref{sec:app-sec-proofs}. 

% This is a section in the appendix.

\end{document}
