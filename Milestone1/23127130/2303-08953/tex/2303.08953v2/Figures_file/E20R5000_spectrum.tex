\begin{figure}[tbhp]
\centering
\subfloat{
    \begin{tikzpicture}
    \begin{axis}[
    width=0.4\textwidth,
    xlabel={Re$(\lambda_i)$}, 
    ylabel={Im$(\lambda_i)$},
    ]
    \addplot [
    only marks,
    hcred,
    mark=*,
    % mark repeat=5,
    ] table [x=eig_real,y=eig_imag] {Figures_data/E20R5000/E20R5000_spectrum.dat};
    \end{axis}
    \end{tikzpicture}
}
\\
\subfloat{
    \begin{tikzpicture}
    \begin{axis}[
    width=0.4\textwidth,
    xlabel={Re$(\lambda_i)$}, 
    ylabel={Im$(\lambda_i)$},
    ]
    \addplot [
    only marks,
    hcred,
    mark=*,
    % mark repeat=5,
    ] table [x=eig_real,y=eig_imag] {Figures_data/E20R5000_Scalar/E20R5000Scalar_spectrum.dat};
    \end{axis}
    \end{tikzpicture}
}
\subfloat{
    \begin{tikzpicture}
    \begin{axis}[
    width=0.4\textwidth,
    xlabel={Re$(\lambda_i)$}, 
    ylabel={Im$(\lambda_i)$},
    ]
    \addplot [
    only marks,
    hcred,
    mark=*,
    % mark repeat=5,
    ] table [x=eig_real,y=eig_imag] {Figures_data/E20R5000_Column/E20R5000Column_spectrum.dat};
    \end{axis}
    \end{tikzpicture}
}
\caption{Eigenvalue spectrum of matrix E20R5000. The three sub-figures show the eigenvalue spectrum of the same matrix but under different preconditioning techniques. Top: no preconditioning. Lower left: scalar scaling. Lower right: column scaling.}
\label{fig:E20R5000_spectrum}
\end{figure}