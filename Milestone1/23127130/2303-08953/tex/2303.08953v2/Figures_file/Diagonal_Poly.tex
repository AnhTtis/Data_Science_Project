\begin{figure}[!tbhp]
\centering
\subfloat{
    \begin{tikzpicture}
    \begin{axis}[
    width=0.7\textwidth,
    ymode=log,
    xmin=1,
    xmax=100,
    ymin=1e-16,
    ymax=1,
    xlabel={iteration, $i$}, 
    ylabel={Relative residual / LOO},
    legend entries={GMRES, Monomial, Newton, Scaled Newton},
    % legend pos={outer north east},
    legend columns=-1,
    legend style={at={(0.5,1.15)},anchor=north},
    ]
    \addplot [mark=o,mark repeat=8] table [x=i,y=GMRES] {Figures_data/Diagonal_Poly/Polynomial_res.dat};
    \addplot [
    hcred, 
    mark=diamond,
    mark repeat=8,
    mark phase=3,
    ] table [x=i,y=Monomial] {Figures_data/Diagonal_Poly/Polynomial_res.dat};
    \addplot [
    hcblue, 
    mark=triangle,
    mark repeat=8,
    mark phase=5,
    ] table [x=i,y=Newton] {Figures_data/Diagonal_Poly/Polynomial_res.dat};
    \addplot [
    hcgreen, 
    mark=square,
    mark repeat=8,
    mark phase=7,
    ] table [x=i,y=SNewton] {Figures_data/Diagonal_Poly/Polynomial_res.dat};
    % \addplot [
    % hcgrey,
    % mark=o,
    % mark repeat=8,
    % mark phase=8,
    % ] table [x=i,y=Chebyshev] {Figures_data/Diagonal_Poly/Polynomial_res.dat};
    \addplot [mark=o,mark repeat=8] table [x=i,y=LOO] {Figures_data/Diagonal/Diagonal_res_LOO_s.dat};
    \addplot [
    hcred, 
    mark=diamond,
    mark repeat=8,
    mark phase=3,
    ] table [x=i,y=Monomial] {Figures_data/Diagonal_Poly/Polynomial_LOO.dat};
    \addplot [
    hcblue, 
    mark=triangle,
    mark repeat=8,
    mark phase=5,
    ] table [x=i,y=Newton] {Figures_data/Diagonal_Poly/Polynomial_LOO.dat};
    \addplot [
    hcgreen, 
    mark=square,
    mark repeat=8,
    mark phase=7,
    ] table [x=i,y=SNewton] {Figures_data/Diagonal_Poly/Polynomial_LOO.dat};
    % \addplot [
    % hcgrey, 
    % mark=o,
    % mark repeat=8,
    % mark phase=8,
    % ] table [x=i,y=Chebyshev] {Figures_data/Diagonal_Poly/Polynomial_LOO.dat};
    \end{axis}
    \end{tikzpicture}
}
\\
\subfloat{
    \begin{tikzpicture}
    \begin{axis}[
    width=0.7\textwidth,
    height=0.3\textwidth,
    xmin=1,
    xmax=100,
    ymin=0,
    ymax=110,
    xlabel={iteration, $i$}, 
    ylabel={block size, $s$},
    ]
    \addplot [
    hcred, 
    mark=diamond,
    mark repeat=6,
    mark phase=2,
    ] table [x=i,y=Monomial] {Figures_data/Diagonal_Poly/Polynomial_s.dat};
    \addplot [
    hcblue, 
    mark=triangle,
    mark repeat=6,
    mark phase=4,
    ] table [x=i,y=Newton] {Figures_data/Diagonal_Poly/Polynomial_s.dat};
    \addplot [
    hcgreen, 
    mark=square,
    mark repeat=6,
    mark phase=6,
    ] table [x=i,y=SNewton] {Figures_data/Diagonal_Poly/Polynomial_s.dat};
    % \addplot [
    % hcgrey, 
    % mark=o,
    % mark repeat=8,
    % mark phase=8,
    % ] table [x=i,y=Chebyshev] {Figures_data/Diagonal_Poly/Polynomial_s.dat};
    \end{axis}
    \end{tikzpicture}
}\;
\\
\subfloat{
    \begin{tikzpicture}
    \begin{axis}[
    ymode=log,
    width=0.45\textwidth,
    height=0.3\textwidth,
    xmin=0,
    xmax=100,
    ymin=1,
    ymax=1e8,
    ytick={1,1e7},
    xlabel={column, $j$}, 
    ylabel={$\kappa(R_{1:j,1:j})$},
    ]
    \addplot [
    hcred,
    mark=diamond,
    ] table [x=i,y=Monomial_ICE] {Figures_data/Diagonal_Poly/Polynomial_cond.dat};
    \addplot [
    hcblue,
    mark=triangle,
    mark repeat=2,
    ] table [x=i,y=Newton_ICE] {Figures_data/Diagonal_Poly/Polynomial_cond.dat};
    \addplot [
    hcgreen,
    mark=square,
    mark repeat=5,
    ] table [x=i,y=SNewton_ICE] {Figures_data/Diagonal_Poly/Polynomial_cond.dat};
    % \addplot [
    % hcgrey,
    % mark=o,
    % mark repeat=2,
    % ] table [x=i,y=Chebyshev_ICE] {Figures_data/Diagonal_Poly/Polynomial_cond.dat};
    \addplot [
    sharp plot, dashed,
    ] coordinates
    {(1,1e7) (100,1e7)};
    \end{axis}
    \end{tikzpicture}
}
\subfloat{
    \begin{tikzpicture}
    \begin{axis}[
    ymode=log,
    width=0.45\textwidth,
    height=0.3\textwidth,
    xmin=0,
    xmax=100,
    ymin=1e-1,
    ymax=1e8,
    ytick={1,1e7},
    xlabel={column, $j$}, 
    ylabel={$\|V_{:,j}\|_2$},
    ]
    \addplot [
    hcred,
    mark=diamond,
    ] table [x=i,y=monomial_v] {Figures_data/Diagonal_Poly/Polynomial_vecnorm.dat};
    \addplot [
    hcblue,
    mark=triangle,
    mark repeat=2,
    ] table [x=i,y=newton_v] {Figures_data/Diagonal_Poly/Polynomial_vecnorm.dat};
    \addplot [
    hcgreen,
    mark=square,
    mark repeat=5,
    ] table [x=i,y=snewton_v] {Figures_data/Diagonal_Poly/Polynomial_vecnorm.dat};
    % \addplot [
    % hcgrey,
    % mark=o,
    % mark repeat=2,
    % ] table [x=i,y=chebyshev_v] {Figures_data/Diagonal_Poly/Polynomial_vecnorm.dat};
    \addplot [
    sharp plot, dashed,
    ] coordinates
    {(1,1e7) (100,1e7)};
    \end{axis}
    \end{tikzpicture}
}
\caption{Diagonal matrix. Top figure: relative residual and LOO errors of GMRES and adaptive $s$-step GMRES using different polynomial bases in the MPK. Middle figure: adapted step size $s$. The scaled Newton polynomials have the largest adapted step size of $s=100$. Lower left figure: incremental condition number of the triangular matrix seen by the first instance of partial CholQR in the first block iteration. Lower right figure: vector-wise norm of the Krylov basis matrix. The scaled Newton polynomial is effective at keeping the vector norm to $O(1)$ and slowing the incremental condition number growth.}
\label{fig:diagonalmatrix_poly}
\end{figure}