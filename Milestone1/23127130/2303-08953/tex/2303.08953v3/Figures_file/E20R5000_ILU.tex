\begin{figure}[tbhp]
\centering
\subfloat{
    \begin{tikzpicture}
    \begin{axis}[
    width=0.7\textwidth,
    ymode=log,
    xmin=1,
    xmax=75,
    ytick={1e-15, 1e-10, 1e-5, 1},
    ymax=1,
    ymin=1e-16,
    xlabel={iteration, $i$}, 
    ylabel={Relative residual / LOO},
    legend entries={GMRES,Monomial,Newton,Scaled Newton},
    legend columns=-1,
    legend style={at={(0.5,1.15)},anchor=north},
    % legend pos=north east
    ]
    \addplot [mark=o,mark repeat=20] table [x=i,y=GMRES] {Figures_data/E20R5000_ILU/E20R5000ILU_res.dat};
    \addplot [
    hcred, 
    mark=diamond,
    mark repeat=20,
    mark phase=6,
    ] table [x=i,y=Monomial] {Figures_data/E20R5000_ILU/E20R5000ILU_res.dat};
    \addplot [
    hcblue, 
    mark=triangle,
    mark repeat=20,
    mark phase=11,
    ] table [x=i,y=Newton] {Figures_data/E20R5000_ILU/E20R5000ILU_res.dat};
    \addplot [
    hcgreen, 
    mark=square,
    mark repeat=20,
    mark phase=16,
    ] table [x=i,y=SNewton] {Figures_data/E20R5000_ILU/E20R5000ILU_res.dat};
    \addplot [mark=o,mark repeat=20] table [x=i,y=GMRES] {Figures_data/E20R5000_ILU/E20R5000ILU_LOO.dat};
    \addplot [
    hcred, 
    mark=diamond,
    mark repeat=20,
    mark phase=6,
    ] table [x=i,y=Monomial] {Figures_data/E20R5000_ILU/E20R5000ILU_LOO.dat};
    \addplot [
    hcblue, 
    mark=triangle,
    mark repeat=20,
    mark phase=11,
    ] table [x=i,y=Newton] {Figures_data/E20R5000_ILU/E20R5000ILU_LOO.dat};
    \addplot [
    hcgreen, 
    mark=square,
    mark repeat=20,
    mark phase=16,
    ] table [x=i,y=SNewton] {Figures_data/E20R5000_ILU/E20R5000ILU_LOO.dat};
    \end{axis}
    \end{tikzpicture}
}\\
\subfloat{
    \begin{tikzpicture}
    \begin{axis}[
    width=0.45\textwidth,
    % height=0.3\textwidth,
    xmin=1,
    xmax=75,
    ymin=0,
    ymax=10,
    xlabel={iteration, $i$}, 
    ylabel={block size, $s$},
    % ylabel shift=5,
    ]
    \addplot [
    hcred, 
    mark=diamond,
    mark repeat=15,
    mark phase=5,
    ] table [x=i,y=Monomial] {Figures_data/E20R5000_ILU/E20R5000ILU_s.dat};
    \addplot [
    hcblue, 
    mark=triangle,
    mark repeat=15,
    mark phase=10,
    ] table [x=i,y=Newton] {Figures_data/E20R5000_ILU/E20R5000ILU_s.dat};
    \addplot [
    hcgreen, 
    mark=square,
    mark repeat=15,
    mark phase=15,
    ] table [x=i,y=SNewton] {Figures_data/E20R5000_ILU/E20R5000ILU_s.dat};
    % \addplot [
    % hcgrey, dashed,
    % mark=*,
    % mark repeat=7,
    % ] table [x=i,y=Chebyshev] {Figures_data/E20R5000_ILU/E20R5000ILU_s.dat};
    \end{axis}
    \end{tikzpicture}
}
\subfloat{
    \begin{tikzpicture}
    \begin{axis}[
    width=0.45\textwidth,
    xlabel={Re$(\lambda_i)$}, 
    ylabel={Im$(\lambda_i)$},
    ]
    \addplot [
    only marks,
    hcred,
    mark=*,
    % mark repeat=5,
    ] table [x=eig_real,y=eig_imag] {Figures_data/E20R5000_ILU/E20R5000ILU_spectrum.dat};
    \end{axis}
    \end{tikzpicture}
}
\caption{E20R5000 matrix with the ILUTP preconditioner. Top figure: relative residual and LOO errors. A rapid acceleration of the convergence is observed due to the effective preconditioning, while the stability of the orthogonalization is still guaranteed. Lower left figure: adapted step size, $s$. Adapted block sizes are generally very small. Lower right figure: eigenvalue spectrum of the preconditioned matrix. One outlying eigenvalue is well-separated from the rest of the eigenvalues.}
\label{fig:E20R5000_ILU}
\end{figure}
