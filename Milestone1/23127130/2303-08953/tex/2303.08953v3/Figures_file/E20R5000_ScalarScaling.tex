\begin{figure}[tbhp]
\centering
\subfloat{
    \begin{tikzpicture}
    \begin{groupplot}[
        group style={
            group name=my fancy plots,
            group size=1 by 2,
            xticklabels at=edge bottom,
            vertical sep=0pt,
        },
        width=0.7\textwidth,
        xmin=0, xmax=600,
    ]
    \nextgroupplot[
        ymode=log,
        ymin=0.92,ymax=1,
        ytick={0.95,1},
        axis x line=top, 
        axis y discontinuity=parallel,
        height=0.3\textwidth,
        legend entries={GMRES,Monomial,Newton,Scaled Newton},
        legend columns=-1,
        legend style={at={(0.42,1.35)},anchor=north},
    ]
    \addplot [mark=o,mark repeat=40] table [x=i,y=GMRES] {Figures_data/E20R5000_Scalar/E20R5000Scalar_res.dat};
    \addplot [
    hcred,
    mark=diamond,
    mark repeat=40,
    mark phase=11,
    ] table [x=i,y=Monomial] {Figures_data/E20R5000_Scalar/E20R5000Scalar_res.dat};   
    \addplot [
    hcblue,
    mark=triangle,
    mark repeat=40,
    mark phase=21,
    ] table [x=i,y=Newton] {Figures_data/E20R5000_Scalar/E20R5000Scalar_res.dat};
    \addplot [
    hcgreen,
    mark=square,
    mark repeat=40,
    mark phase=31,
    ] table [x=i,y=SNewton] {Figures_data/E20R5000_Scalar/E20R5000Scalar_res.dat};
    % \addplot [
    % hcgrey,
    % mark=o,
    % mark repeat=40,
    % mark phase=40,
    % ] table [x=i,y=Chebyshev] {Figures_data/E20R5000/E20R5000_res.dat};
    
    \nextgroupplot[
        ymode=log,
        ymin=1e-16,ymax=1e-13,
        ytick={1e-16,1e-14},
        xtick={150,300,450,600},
        axis x line=bottom,
        height=0.3\textwidth,
        xlabel={iteration, $i$},
        ylabel={Relative residual / LOO},
        ylabel style={at={(ticklabel cs:1)}},
    ]
    \addplot [mark=o,mark repeat=40] table [x=i,y=GMRES] {Figures_data/E20R5000_Scalar/E20R5000Scalar_LOO.dat};
    \addplot [
    hcred, 
    mark=diamond,
    mark repeat=40,
    mark phase=11,
    ] table [x=i,y=Monomial] {Figures_data/E20R5000_Scalar/E20R5000Scalar_LOO.dat};     
    \addplot [
    hcblue, 
    mark=triangle,
    mark repeat=40,
    mark phase=21,
    ] table [x=i,y=Newton] {Figures_data/E20R5000_Scalar/E20R5000Scalar_LOO.dat};  
    \addplot [
    hcgreen, 
    mark=square,
    mark repeat=40,
    mark phase=31,
    ] table [x=i,y=SNewton] {Figures_data/E20R5000_Scalar/E20R5000Scalar_LOO.dat};  
    % \addplot [
    % hcgrey, 
    % mark=o,
    % mark repeat=40,
    % mark phase=40,
    % ] table [x=i,y=Chebyshev] {Figures_data/E20R5000/E20R5000_LOO.dat};  
    \end{groupplot}
    \end{tikzpicture}
}
\\\,\,
\subfloat{
    \begin{tikzpicture}
    \begin{axis}[
    width=0.7\textwidth,
    height=0.3\textwidth,
    xmin=1,
    xmax=600,
    xtick={150,300,450,600},
    ymin=0,
    ymax=120,
    ytick={0, 40, 80, 120},
    xlabel={iteration, $i$}, 
    ylabel={block size, $s$},
    ylabel shift=5,
    ]
    \addplot [
    hcred, 
    mark=diamond,
    mark repeat=30,
    ] table [x=i,y=Monomial] {Figures_data/E20R5000_Scalar/E20R5000Scalar_s.dat};
    \addplot [
    hcblue, 
    mark=triangle,
    mark repeat=30,
    ] table [x=i,y=Newton] {Figures_data/E20R5000_Scalar/E20R5000Scalar_s.dat};
    \addplot [
    hcgreen, 
    mark=square,
    mark repeat=30,
    ] table [x=i,y=SNewton] {Figures_data/E20R5000_Scalar/E20R5000Scalar_s.dat};
    % \addplot [
    % hcgrey, dashed,
    % mark=*,
    % mark repeat=7,
    % ] table [x=i,y=Chebyshev] {Figures_data/E20R5000_Scalar/E20R5000Scalar_s.dat};
    \end{axis}
    \end{tikzpicture}
}
\caption{E20R5000 matrix with scalar scaling. The top sub-figure shows the relative residual and LOO errors of GMRES and adaptive $s$-step GMRES using different polynomial bases in the MPK. The lower sub-figure shows the adapted step size $s$.}
\label{fig:E20R5000_scalar}
\end{figure}
