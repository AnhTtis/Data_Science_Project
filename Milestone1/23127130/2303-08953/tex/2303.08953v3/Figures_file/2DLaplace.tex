\begin{figure}[tbhp]
\centering
\subfloat{
    \begin{tikzpicture}
    \begin{groupplot}[
        group style={
            group name=my fancy plots,
            group size=1 by 2,
            xticklabels at=edge bottom,
            vertical sep=0pt,
        },
        width=0.7\textwidth,
        xmin=0, xmax=500,
    ]
    \nextgroupplot[
        ymode=log,
        ymin=0.05,ymax=1,
        ytick={1e-1,1},
        axis x line=top, 
        axis y discontinuity=parallel,
        height=0.35\textwidth,legend entries={GMRES, Monomial},
        % legend pos=outer north east,
        legend columns=-1,
        legend style={at={(0.5,1.3)},anchor=north},
    ]
    \addplot [mark=o,mark repeat=26,mark phase=14] table [x=i,y=res] {Figures_data/2DLaplace/2DLaplace_res_LOO_s.dat};     
    \addplot [
    hcred,
    mark=diamond,
    mark repeat=26,
    ] table [x=i,y=res_CA] {Figures_data/2DLaplace/2DLaplace_res_LOO_s.dat};   
    
    \nextgroupplot[
        ymode=log,
        ymin=1e-16,ymax=1e-10,
        ytick={1e-15,1e-13,1e-11},
        axis x line=bottom,
        x axis line style={},
        % x axis line style={-stealth},
        height=0.35\textwidth,
        xlabel={iteration, $i$},
        ylabel={Relative residual / LOO},
        ylabel style={at={(ticklabel cs:1)}},
    ]
    \addplot [mark=o,mark repeat=26,mark phase=14] table [x=i,y=LOO] {Figures_data/2DLaplace/2DLaplace_res_LOO_s.dat};
    \addplot [
    hcred, 
    mark=diamond,
    mark repeat=26,
    ] table [x=i,y=LOO_CA] {Figures_data/2DLaplace/2DLaplace_res_LOO_s.dat};
    \end{groupplot}
    \end{tikzpicture}
}
\\
\subfloat{
    \begin{tikzpicture}
    \begin{axis}[
    width=0.45\textwidth,
    height=0.3\textwidth,
    xmin=1,
    xmax=500,
    ymin=0,
    ymax=10,
    xlabel={iteration, $i$}, 
    ylabel={block size, $s$},
    xtick={250,500},
    ]
    \addplot [
    hcred, 
    mark=diamond,
    mark repeat=50,
    ] table [x=i,y=s] {Figures_data/2DLaplace/2DLaplace_res_LOO_s.dat};
    \end{axis}
    \end{tikzpicture}
}
\subfloat{
    \begin{tikzpicture}
    \begin{semilogyaxis}[
    width=0.45\textwidth,
    height=0.3\textwidth,
    % grid=major,
    xmin=1,
    xmax=10,
    ymin=1,
    ymax=1e9,
    xlabel={column, $j$}, 
    ylabel={$\kappa(R_{1:j,1:j})$},
    ytick={1e0, 1e2, 1e4, 1e6, 1e8},
    legend entries={ICE, SVD},
    legend pos=south east,
    ]
    \addplot [
    hcred, 
    mark=o,
    ] table [x=i,y=ICE] {Figures_data/2DLaplace/2DLaplace_cond.dat};
    \addplot [
    hcnavy,
    mark=triangle,
    ] table [x=i,y=SVD] {Figures_data/2DLaplace/2DLaplace_cond.dat};
    \addplot [
    sharp plot, dashed,
    ] coordinates
    {(1,1e7) (10,1e7)};
    \end{semilogyaxis}
    \end{tikzpicture}
}
\caption{2D Laplace matrix of size $N=1.6\times10^\rpy{5}$ using 5-stencil finite difference discretization. The top sub-figure shows the relative residual and LOO errors of GMRES and adaptive s-step GMRES with a monomial basis. Good agreement of the relative residuals between the baseline and the adaptive algorithm can be observed. LOO of the adaptive s-step GMRES is kept near the machine epsilon. The lower left sub-figure shows the adapted block size, $s$. In this case, the adapted step size is constant at $s=6$. The lower right sub-figure shows the incremental condition number of the triangular matrix seen by the first instance of partial CholQR in the first block iteration. The ICE estimate agrees well with the true SVD estimate.}
\label{fig:2Dlaplace}
\end{figure}