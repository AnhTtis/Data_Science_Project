\begin{figure}[tbhp]
\centering
\subfloat{
    \begin{tikzpicture}
    \begin{axis}[
    width=0.7\textwidth,
    ymode=log,
    xmin=1,
    xmax=400,
    ymin=1e-16,
    ymax=1,
    xlabel={iteration, $i$}, 
    ylabel={Relative residual / LOO},
    legend entries={GMRES, Monomial, Newton, Scaled Newton},
    legend columns=-1,
    legend style={at={(0.5,1.15)},anchor=north},
    ]
    \addplot [mark=o,mark repeat=24] table [x=i,y=GMRES] {Figures_data/2DLaplace_ILU/2DLaplaceILU_res.dat};
    \addplot [
    hcred,
    mark=diamond,
    mark repeat=24,
    mark phase=7,
    ] table [x=i,y=Monomial] {Figures_data/2DLaplace_ILU/2DLaplaceILU_res.dat};
    \addplot [
    hcblue,
    mark=triangle,
    mark repeat=24,
    mark phase=13,
    ] table [x=i,y=Newton] {Figures_data/2DLaplace_ILU/2DLaplaceILU_res.dat};
    \addplot [
    hcgreen, 
    mark=square,
    mark repeat=24,
    mark phase=19,
    ] table [x=i,y=SNewton] {Figures_data/2DLaplace_ILU/2DLaplaceILU_res.dat};
    % \addplot [
    % hcgrey, dashed,
    % mark=*,
    % mark repeat=24,
    % ] table [x=i,y=Chebyshev] {Figures_data/2DLaplace_ILU/2DLaplaceILU_res.dat};
    \addplot [mark=o,mark repeat=24] table [x=i,y=GMRES] {Figures_data/2DLaplace_ILU/2DLaplaceILU_LOO.dat};
    \addplot [
    hcred, 
    mark=diamond,
    mark repeat=24,
    mark phase=7,
    ] table [x=i,y=Monomial] {Figures_data/2DLaplace_ILU/2DLaplaceILU_LOO.dat};
    \addplot [
    hcblue, 
    mark=triangle,
    mark repeat=24,
    mark phase=13,
    ] table [x=i,y=Newton] {Figures_data/2DLaplace_ILU/2DLaplaceILU_LOO.dat};
    \addplot [
    hcgreen,
    mark=square,
    mark repeat=24,
    mark phase=19,
    ] table [x=i,y=SNewton] {Figures_data/2DLaplace_ILU/2DLaplaceILU_LOO.dat};
    % \addplot [
    % hcgrey, dashed,
    % mark=*,
    % mark repeat=24,
    % ] table [x=i,y=Chebyshev] {Figures_data/2DLaplace_ILU/2DLaplaceILU_LOO.dat};
    \end{axis}
    \end{tikzpicture}
}
\\
\subfloat{
    \begin{tikzpicture}
    \begin{groupplot}[
        group style={
            group name=my fancy plots,
            group size=1 by 2,
            xticklabels at=edge bottom,
            vertical sep=0pt,
        },
        width=0.7\textwidth,
        xmin=0, xmax=400,
    ]
    \nextgroupplot[
        ymin=375,ymax=400,
        % ytick={1e-1,1},
        axis x line=top, 
        axis y discontinuity=parallel,
        height=0.2\textwidth,
        ytick={390,400},
        % legend entries={GMRES,Adapt. s-step},
    ]
    \addplot [
    hcgreen,
    mark=square,
    mark repeat=20,
    ] table [x=i,y=SNewton] {Figures_data/2DLaplace_ILU/2DLaplaceILU_s.dat};     
    
    \nextgroupplot[
        ymin=0,ymax=25,
        axis x line=bottom,
        x axis line style={},
        height=0.2\textwidth,
        xlabel={iteration, $i$},
        ylabel={block size, $s$},
        ylabel style={at={(ticklabel cs:1)}},
        ytick={0,10,20},
        ylabel shift=6,
    ]
    \addplot [
    hcred,
    mark=diamond,
    mark repeat=20,
    ] table [x=i,y=Monomial] {Figures_data/2DLaplace_ILU/2DLaplaceILU_s.dat};   
    \addplot [
    hcblue,
    mark=triangle,
    mark repeat=20,
    ] table [x=i,y=Newton] {Figures_data/2DLaplace_ILU/2DLaplaceILU_s.dat};
    % \addplot [
    % hcgrey, dashed,
    % mark=*,
    % mark repeat=20,
    % ] table [x=i,y=Chebyshev] {Figures_data/2DLaplace_ILU/2DLaplaceILU_s.dat};
    \end{groupplot}
    \end{tikzpicture}
}
\caption{2D Laplace matrix with ILU(0) preconditioner. Top figure: relative residual and LOO errors of GMRES and adaptive $s$-step GMRES with different bases. Convergence is accelerated by preconditioning and LOO error of the adaptive s-step GMRES is still kept near machine epsilon. Lower figure: adapted step size, $s$. It is worth mentioning that scaled Newton polynomials give a block size of $s=400$, which is equivalent to one block iteration for convergence.}
\label{fig:test_laplace_precond}
\end{figure}