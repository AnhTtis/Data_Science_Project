\paragraph{Related works.}

% Add implicit comparison with the proposed method

There are multiple combinations of the particle filter (PF) with other methods~\cite{Rao, Reconcillation, particle_swarm, particle_genetic, Tutorial, SHARIATI201932}.
% Rao-Blackwellised PF \cite{Rao}, Box PF method ~\cite{Interval}, PF with data reconciliation \cite{Reconcillation}, PF and particle swarm method~\cite{particle_swarm}, PF with genetic algorithm~\cite{particle_genetic}, and many others.
Rao-Blackwellised PF~\cite{Rao} splits state vectors into two parts. 
The first part is processed with the Kalman filter, and the second part is processed with the PF. 
This approach is applicable for high dimensional problems, where the standard particle filter may fail.
The Box PF method~\cite{Interval} describes the state vector distribution as a sum of uniform probability density functions. 
This method decreases the number of particles resulting in the same accuracy as PF.
In study~\cite{Reconcillation} the measurement test criterion and data reconciliation are proposed to derive reliable initial states under sufficient information about measurements.

Other studies are devoted to a combination of the particle filter with nature-inspired optimization methods like particle swarm method~\cite{particle_swarm} or genetic algorithm~\cite{particle_genetic}.
Such combinations incorporate elements of these optimization methods into the particle filter, e.g. stochastic resampling is replaced by the crossover and mutation operations. 
Although these combinations provide better localization accuracy, they are computationally expensive.

There are also various combinations of Kalman and particle filters. 
In~\cite{Tutorial} extended Kalman particle filter is presented.
This filtering algorithm reduces uncertainty in every particle motion due to the additional Kalman-type update for every particle. 
However, 
% inconsistency between the state covariance matrix and process covariance matrix may lead to slow convergence and 
it requires a large number of particles and converges slowly if the state noise model is incorrect.
Another combination is proposed in~\cite{SHARIATI201932} and it is used diagonal process covariance matrices fitted from experimental data.
Therefore, it requires more data and can not treat an arbitrary localization problem.
% However, large weights may be assigned to irrelevant particles due to inconsistency between the state covariance matrix and process covariance matrix. 
% Therefore, the convergence becomes slower or may require a large number of particles.

% Other approach of such combination was proposed in~\cite{SHARIATI201932}, where  proposes a similar approach to \textbf{PFKU}, however, there are some differences.
% In particular, it uses diagonal process covariance matrices fitted from experimental data.
% In contrast, \textbf{PFKU} explicitly estimates covariance matrices that are not diagonal.
% Thus, \textbf{PFKU} is not based on experimental data and is more universal than this competitor.


% The other version of the Kalman and particle filters combination is discussed in ~\cite{SHARIATI201932}, where multiple particles are initiated as in the particle filter, and then the Kalman filter prediction and update steps are used for every particle, followed by the resampling procedure. 
% Here the weights updates are rather straightforward as in the standard particle filter. 
% The approach used in the present paper is almost the same as in the ~\cite{SHARIATI201932}, however, there are some modifications. 
% First, in the paper ~\cite{SHARIATI201932} the covariance matrices are taken from fitting experimental data and they are diagonal. 
% In the present paper, we assume the motion model is known: the standard robot motion based on external speed and heading. 
% Thus there is no need for the covariance matrices to be taken from the experimental data, they are calculated from the first principles and are not supposed to be diagonal. 

% The only factors which are not known are the speed and heading noise level, thus two different values for these noises will be tried.

%It should also be noted Rao–Blackwellized %particle filter~\cite{Rao}. 
% In some specific cases, the probability density function allows the split of the state vector into two parts and treats them sequentially~\cite{Rao}. 
% Then, often the first part can be processed with the Kalman filter, while the second by the particle filter. 
% This approach speeds up computations and is very useful in high dimensions, where standard particle filter doesn't perform well. In the considered test environments it does not give any benefits as both initial location and heading are unknown. 

% The other combination of Kalman and particle filters is suggested in~\cite{KPF_occlusion} avoid the occlusion problem in image processing during object tracking. 
% The Kalman filter is replaced by the particle filter when occlusion occurs and the particle filter is used while the system is stable, and then the Kalman filter is used again. The approach looks promising for occlusion problem reduction purposes, though does not look suitable for the robots' in-door tracking. In~\cite{EPF_EKF, UPF} there are more discussions on other combinations of these two filters. 


% The study~\cite{particle_swarm} suggests a particle filter based on the particle swarm optimization method. 
% By using the PSO algorithm, the particles area around the current state is determined depending on the measurement results. 
% Then, in order to attain diversity and convergence, the particles are distributed through the area. 
% According to the simulation results, the PSO-PF algorithm is more accurate than the standard particle filter, though has higher complexity.
% Another combination of particle filter and nature-inspired optimization method, for instance, genetic algorithm, is considered in~\cite{particle_genetic}. 
% Here the pure stochastic resampling is replaced by adopting crossover and mutation operators, which enhances the particles' diversity and mitigates the sample impoverishment problem.

% Such approaches are very important as allowing particle filter applications in FPGA devices, smartphones, etc.

% There are also approaches based on neural network combination with particle filter~\cite{PFRNN}. Here, the time recurrent neural network is being developed with the $N$ particles: $N$ hidden states and their weights evolving with time. At each time step, modified GRU or LSTM cell is used to update hidden states and weights based on encoded motion and measurement. Then, the differentiable resampling is used to update weights and resample hidden states. This approach reduces the required number of particle by increasing the dimension of a hidden space. The drawbacks of this approach are a long time for training and necessity to have a large dataset. 

% These two filters and their multiple modifications cover the majority of problems, so other methods~\cite{ex_filters}, such as Wiener filter~\cite{Kalman_book}, distribution filters, conjugate analysis approach, differential geometrical approach, and interacting multiple models are rather considered exotic.
