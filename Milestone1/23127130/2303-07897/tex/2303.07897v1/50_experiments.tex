\section{Computational experiments}
\label{sec::experiments}

In this section, we present the description of the experiments for comparison of the proposed multiparticle Kalman filter (MKF) with a standard particle filter in symmetric and non-symmetric environments.
We exclude the Kalman-based filters from our comparison since they require the object's initial state, which is unknown according to our assumption.
Every experiment is conducted on a single NVIDIA Tesla V100 GPU.


\subsection{Test environments}

To evaluate the performance of the proposed method and compare it with the particle filter we use different types of environments.
In particular, we consider the symmetric environments with an increasing number of symmetrical subparts and call them \textit{world $10\times 10$}, \textit{World $18\times 18$}, and \textit{WORLD $27\times 27$}.
The number of beacons is also increased with the number of symmetrical subparts which makes filtering of object states more challenging.
The considered symmetric environments are shown in Figure~\ref{fig::symmetric_envs}.

\begin{figure}[!ht]
\centering
%\begin{adjustbox}{minipage=\linewidth,scale=0.7}
    \begin{subfigure}{0.3\textwidth}
    \centering
    \includegraphics[width=\textwidth]
    %{pics/world_fig_10.pdf}
    {pics/_w10.pdf}
    \caption{\textit{world $10\times 10$}}
    \end{subfigure}
    ~
    \begin{subfigure}{0.3\textwidth}
    \centering
    \includegraphics[width=\textwidth]
    {pics/_w18.pdf}
    \caption{\textit{World $18\times 18$}}
    \end{subfigure}
    ~
    \begin{subfigure}{0.3\textwidth}
    \centering
    \includegraphics[width=\textwidth]
    {pics/_w27.pdf}
    \caption{\textit{WORLD $27\times 27$}}
    \end{subfigure}
\caption{Visualization of symmetrical test environments. Beacons and obstacles are shown as black crosses and grey blocks, respectively.}
\label{fig::symmetric_envs}
%\end{adjustbox}
\end{figure}

To illustrate the effect of symmetry in an environment, we remove a subpart in every environment described above such that they become nonsymmetric.
The nonsymmetric analogs of the aforementioned symmetrical environments are shown in Figure~\ref{fig::nonsym_envs} and we call them \textit{n-world $10\times 10$}, \textit{n-World $18\times 18$}, and \textit{n-WORLD $27\times 27$}, respectively.
\begin{figure}[!ht]
\centering
%\begin{adjustbox}{minipage=\linewidth,scale=0.7}
    \begin{subfigure}{0.3\textwidth}
    \centering
    \includegraphics[width=\textwidth]
    %{pics/world_fig_10.pdf}
    {pics/_w10_1.pdf}
    \caption{\textit{n-world $10\times 10$}}
    \end{subfigure}
    ~
    \begin{subfigure}{0.3\textwidth}
    \centering
    \includegraphics[width=\textwidth]
    {pics/_w18_1.pdf}
    \caption{\textit{n-World $18\times 18$}}
    \end{subfigure}
    ~
    \begin{subfigure}{0.3\textwidth}
    \centering
    \includegraphics[width=\textwidth]
    {pics/_w27_1.pdf}
    \caption{\textit{n-WORLD $27\times 27$}}
    \end{subfigure}
\caption{Visualization of \emph{nonsymmetric} test environments. Beacons and obstacles are shown as black crosses and grey blocks, respectively.}
\label{fig::nonsym_envs}
\end{figure}

One more test environment \emph{Labyrinth} is considered in~\cite{towards} and is shown in Figure~\ref{fig::labyrinth}. 
Compared to the previous environments, \emph{Labyrinth} environment has a lower degree of symmetry and allows solving localization problems accurately and comparatively fast in terms of the number of time steps for the convergence.
Therefore, we also compare the considered methods in this relatively friendly environment. 
% As it was mentioned above some types of the environments may cause poor performance of filters:
% \begin{itemize}
%     \item Environments with few symmetrically located obstacles and beacons.
%     \item Environments with multiple beacons and high noise, so that multiple state may reproduce the same measurements results within noise level.
%     \item Environments with unknown object's initial state. 
% \end{itemize}
% The first part of the experiments is done with three environments with increasing complexity~\cite{PFRNN}: world $10\times 10$, World $18\times 18$, and WORLD $27\times 27$ (see \ref{fig::worlds}). 
% The second part contains same worlds with "slightly broken" symmetry, i.e. one section is removed and few less number of landmarks. It should demonstrate essential improvement of the accuracy of predictions (\ref{fig::worlds}). 
% The last test world is \emph{Labyrinth}, similar to what has been used in~\cite{towards}. In contrary to very symmetric worlds it allows precise coordinate determination and is interesting to find out how many particles are required to both algorithms (particle filter and \textbf{PFKU}) to converge.
% The Kalman filter will not be considered as it shows poor performance at unknown initial conditions. 
% The standard particle filter and \textbf{PFKU} will be compared at different parameters of noise values and number of particles.
\begin{figure}[!ht]
\centering
\includegraphics[width=10cm]{pics/_w_lab.pdf}
\caption{Visualization of the \emph{Labyrinth} environment. Beacons and obstacles are shown as black crosses and grey blocks, respectively.
}
\label{fig::labyrinth}
\end{figure}

% Here start the text from section Test environments

% \begin{figure}[!ht]
% \centering
%     \begin{subfigure}[t]{0.3\textwidth}
%     \centering
%     \includegraphics[width=\textwidth]
%     %{pics/world_fig_10.pdf}
%     {pics/_10_10.pdf}
%     \caption{\textit{world $10\times 10$}}
%     \end{subfigure}
%     ~
%     \begin{subfigure}[t]{0.3\textwidth}
%     \centering
%     \includegraphics[width=\textwidth]
%     {pics/_18_18.pdf}
%     \caption{\textit{World $18\times 18$}}
%     \end{subfigure}
%     ~
%     \begin{subfigure}[t]{0.3\textwidth}
%     \centering
%     \includegraphics[width=\textwidth]
%     {pics/_27_27.pdf}
%     \caption{\textit{WORLD $27\times 27$}}
%     \end{subfigure}
% \caption{Visualization of test environments with 5 nearest beacons. In every world, one of the true (circle) and predicted (cross) trajectories are shown. The last positions are shown by a big circle and cross. It is important to note that in very symmetric environments the last points in the predicted trajectory may differ from the corresponding points in a true trajectory, though they are identical up to symmetry.}
% \label{fig::worlds_tracks}
% \end{figure}


% \begin{figure}[!ht]
% \centering
% %\begin{adjustbox}{minipage=\linewidth,scale=0.7}
%     \begin{subfigure}[t]{0.31\textwidth}
%     \centering
%     \includegraphics[width=\textwidth]
%     {pics/_10_10_1.pdf}
%     \caption{\textit{n-world $10\times 10$}}
%     \end{subfigure}
%     ~
%     \begin{subfigure}[t]{0.31\textwidth}
%     \centering
%     \includegraphics[width=\textwidth]
%     {pics/_18_18_1.pdf}
%     \caption{\textit{n-World $18\times 18$}}
%     \end{subfigure}
%     ~
%     \begin{subfigure}[t]{0.32\textwidth}
%     \centering
%     \includegraphics[width=\textwidth]
%     {pics/_27_27_1.pdf}
%     \caption{\textit{n-WORLD $27\times 27$}}
%     \end{subfigure}
% \caption{Visualization of test environments with 5 nearest beacons. Note, that the predicted trajectories normally converge to true.}
% \label{fig::worlds_tracks1}
% %\end{adjustbox}
% \end{figure}


% \begin{figure}[!ht]
% \centering
% \includegraphics[width=0.5\textwidth]{pics/34_14.pdf}
% \caption{\emph{Labyrinth} environment. At a sufficiently large number of particles predicted trajectory converges to a true one normally in a few steps.}
% %\label{fig::Labyrinth1}
% \end{figure}



% \subsection{Assumptions on matrices}
% The simple calculations provide the following values for measurement covariance ($\mR$), process covariance ($\mQ$), and initial position matrix ($\mP_{t=0})$:

% \begin{align*}
% \mR=\sigma d^2 {\bf 1},
% \end{align*}

% \begin{align*}
% \mQ = \begin{bmatrix}
%        \cos\phi^2\,\sigma u^2 + u^2\,\sin\phi^2\, \sigma \phi^2  & (\sigma u^2 + u^2 \sigma\phi^2) \sin2\phi/2 & -u\,\sin\phi\,\sigma\phi^2 \\
%        ( \sigma u^2 + u^2 \sigma\phi^2) \sin2\phi/2 & \sin\phi^2\,\sigma u^2  + u^2 \, \cos\phi^2\sigma \phi^2 & u \cos\phi\,\sigma\phi^2\\
%        -u\,\sin\phi\,\sigma\phi^2 & u\, \cos\phi\,\sigma\phi^2 & \sigma\phi^2
% \end{bmatrix},
% \end{align*}

% \begin{align*}
% \mP_0=
% \begin{bmatrix}
%            \sigma x_0^2 & 0 & 0\\
%            0 & \sigma y_0^2  & 0\\
%            0 & 0 & \sigma \phi_0^2.
%     \end{bmatrix}
% \end{align*}

% The matrices are build on the simple assumption of known motion model and the only hyperparameters are $\sigma d^2, \sigma u^2, \sigma \phi^2$, $\sigma x_0^2, \sigma y_0^2, \sigma \phi_0^2$.


\subsection{Trajectory generation procedure}

To generate trajectories for an object in the considered environments, the following procedure is used.
The object starts motion from a random location without obstacles and has a randomly chosen direction.
In every step, it moves according to external and known velocity $u \in [0, 0.5]$, and the direction remains the same from the previous step within the noise.
If this movement leads to a collision with an obstacle, the object's direction is changed randomly to avoid collision with another obstacle.
To simulate engine noise, the velocity~$u$ is perturbed by $\eta_r \sim \mathcal{U}[-0.02, 0.02]$ and the direction $\phi$ is also perturbed by $\eta_{\phi} \sim 2\pi \alpha$, where $\alpha \sim \mathcal{U}[-0.01, 0.01]$.
The direction perturbation simulates the uncertainty in the object control system.
In the considered environments, we set the number of time steps in every trajectory $T=100$. 
To make a fair comparison of the considered methods, we generate $10000$ trajectories for testing.
% The externally controlled direction remains the same as before unless object hits the wall; in this case the direction is randomly chosen to ensure that the next location is free. 
% But the true direction is also noisy with noise level $\delta\phi \sim 2\pi\, \mathcal{U}[-0.01, 0.01]$.

\paragraph{Filter input data.}
The input data for every filter algorithm consists of the following parts: ground-truth measurement vector $\rvz^*_t$, external control vector $\rvu_t = [u_{r}, \Delta \phi_t]$.
An element $\Delta \phi_t \neq 0$ only if the collision with obstacle appears.
Note that, the object movement is affected by the additional noise $\eta_{u}$ and $\eta_{\phi}$.
Therefore, the filtering methods have to identify the ground-truth object state including its position and heading.

% The motion is represented by the intended distance to move ahead a new intended heading~\cite{Tutorial}. 
% The resulting motion differs from the input due to noise over distance and angle as described above. 
% The measurements are the distances to few (five in our examples) nearest beacons, which are also noisy with the noise distributed as $\delta\phi \sim U[-0.1, 0.1]$.

% The task is to predict coordinates and headings at any time moment. In all environments, $5$ nearest beacons data are used for the measurements.

%%%%%%%%%%%%%%%%%%%%%%%%%%%%%%%%%%%%%%%%%%%%%%%%%
\subsection{Hyperparameters}
\label{sec:hyperparams}
Before one runs the proposed filtering method, the following hyperparameters have to be set: covariance matrices of motion and measurement noise $\mM$ and $\mR$, and initial state covariance matrix $\mP_{t=0}$. 
These hyperparameters significantly affect the performance of the method and have to be tuned carefully.
According to~\cite{towards}, a filtering approach based on particles works better if the variances of motion and measurement noise exceed the ground-truth variances in measurement devices and the object control system.
However, the excessively large variance of motion and measurement noise may lead to slow convergence.
In the experiments, we use the following ground-truth variances in measurement devices and the object control system: $\mR_0 = \sigma_{d0}^2 \mI$, where $\sigma_{d0}^2 = 0.01$ and $\mM_0 = \mathrm{diag}(\sigma^2_{r0}, \sigma^2_{\phi 0})$, $\sigma_{r0} = 0.02$, $\sigma_{\phi 0} = 0.01 \cdot 2\pi$.
At the same time, to study the robustness of the proposed approach to different scales of motion and measurement variance, we consider the following settings.
The first setup is $\mM = \mM_0$ and $\mR = 2\mR_0$.
The second setup is $\mM = 4\mM_0$ and $\mR = 2\mR_0$.
The initial state covariance matrix $\mP_{t=0} = \mathrm{diag}(\sigma_{x}^2, \sigma_{y}^2, \sigma_{\phi}^2)$ and $\sigma_{x}^2=\sigma_{y}^2=\frac{w\,h}{12}$ and $\sigma_{\phi}^2=\frac{(2\pi)^2}{12}$, where $w, h$ are width and height of environment and factor $1/12$ is used to model the uniform distribution of particle states in the environment.

\subsection{Comparison of multiparticle Kalman filter with particle filter}

In this section, we provide a comparison of the considered filtering methods in the aforementioned environments.
However, before presenting the comparison results we introduce the upper bound of the MSE error that indicates the poor quality of the state estimate.
The na\"ive filtering method just generates uniformly random states of the object in the given environment.
Therefore, we can estimate MSE between randomly generated states and the ground-truth states for the considered environments as $MSE_{random}=\frac{w^2+h^2}{6}$, where~$w$ and~$h$ are the width and height of the environment, respectively.
If a filtering method generates states such that MSE between them and the ground-truth states is larger than $MSE_{random}$, we consider such filtering completely useless and show this threshold in the plots below. 

% An upper bound of MSE is such value, that if the model gives a larger value then it is useless. 
% For example, MSE is given by the model that generates random states.
% It is easy to guess that for rectangular environments with scales $w$ and $h$, the upper boundary for MSE is the mean of the square distance between two random points. 
% It is $MSE_{random}=\frac{w^2+h^2}{6}$.

In the experiments, we compare the proposed filtering method with the classical particle filter (see Algorithm~\ref{alg::PF}).
Kalman filter and its modifications are excluded from the comparison since they do not perform well without knowledge of the initial state. 
% Sometimes, large absolute value elements of a covariance matrix $\mP$ may help to converge, however, it is not always true.
Moreover, the Gaussian distribution of state vectors assumes an elliptical uncertainty region that is irrelevant to the considered environments.
We use both MSE~\eqref{eq::mse_def} and FSE~\eqref{eq::fse_def} loss functions.
Also, we compare the robustness of the considered methods to the scale of motion covariance matrix $\mM$.
In particular, the first setting is $\mM = \mM_0$, which is further referred to as $\Sigma$ in legends.
The second setting is $\mM = 4\mM_0$, which is further referred to as $4\Sigma$ in legends.
Here we denote by $\mM_0$ the ground-truth covariance matrix of the noise from the object control system.
The measurement noise covariance matrix in both settings is $\mR = 2\mR_0$, where $\mR_0$ is the covariance matrix of the noise from a measurement device.
The values for $\mM_0$ and $\mR_0$ used in our simulations are given in Section~\ref{sec:hyperparams}.

The comparison results of the proposed filtering method with the particle filter in terms of the MSE~\eqref{eq::mse_def} are shown in Figures~\ref{fig::MSE} and~\ref{fig::MSE1} for symmetric and non-symmetric environments, respectively.
Both plots show that the proposed filtering method (MKF) requires fewer particles to achieve smaller values of MSE in the considered environments.
Also, one can observe that the filtering process in non-symmetric environments is more accurate and robust than in symmetric environments.
The smaller value of MSE indicates higher accuracy and the robustness is illustrated by the number of particles necessary for the convergence of MSE.
Also, these plots show that the proposed method is less sensitive to the estimate of motion noise than the particle filter.

\begin{figure}[!ht]
    %\centering
    \begin{subfigure}{0.3\textwidth}
    %\centering
    \includegraphics[width=\textwidth]
    {pics/world10_performance.pdf}
    \caption{\textit{world $10 \times 10$}}
    \end{subfigure}
    ~
    \begin{subfigure}{0.3\textwidth}
    % \centering
    \includegraphics[width=\textwidth]
    {pics/world18_performance.pdf}
    \caption{\textit{World $18 \times 18$}}
    \end{subfigure}
    ~
    \begin{subfigure}{0.3\textwidth}
    \centering
    \includegraphics[width=\textwidth]
    {pics/world27_performance.pdf}
    \caption{\textit{WORLD $27 \times 27$}}
    \end{subfigure}
\caption{Dependence of MSE on the number of particles in three symmetric environments. Multiparticle Kalman filter (MKF) demonstrates more accurate filtering of states and requires fewer particles for MSE convergence compared to the particle filter (PF). Our method is also less sensitive to the estimate of the motion noise than the particle filter.}
\label{fig::MSE}
\end{figure}

\begin{figure}[!ht]
    %\centering
    \begin{subfigure}{0.3\textwidth}
    %\centering
    \includegraphics[width=\textwidth]
    {pics/world10_1_performance.pdf}
    \caption{\textit{n-world $10 \times 10$}}
    \end{subfigure}
    ~
    \begin{subfigure}{0.3\textwidth}
    %\centering
    \includegraphics[width=\textwidth]
    {pics/world18_1_performance.pdf}
    \caption{\textit{n-World $18 \times 18$}}
    \end{subfigure}
~
    \begin{subfigure}{0.3\textwidth}
    \centering
    \includegraphics[width=\textwidth]
    {pics/world27_1_performance.pdf}
    \caption{\textit{n-WORLD $27 \times 27$}}
    \end{subfigure}
\caption{Dependence of MSE on the number of particles in three non-symmetric environments. 
Multiparticle Kalman filter (MKF) demonstrates more accurate filtering of states and requires fewer particles for MSE convergence compared to the particle filter (PF). 
Our method is also less sensitive to the estimate of the motion noise than the particle filter.}
\label{fig::MSE1}
\end{figure}

Additional experiments are carried out to evaluate the considered filtering methods in terms of the final state error function~\eqref{eq::fse_def}.
The comparison results are shown in Figures~\ref{fig::Fin} and~\ref{fig::Fin1} for symmetric and non-symmetric environments, respectively.
The final states are computed after $100$ time steps in the considered environments.
These plots demonstrate the same trends that are observed in the analysis of MSE dependence on the number of particles presented in Figures~\ref{fig::MSE} and~\ref{fig::MSE1}.

\begin{figure}[!ht]
    %\centering
    \begin{subfigure}{0.3\textwidth}
    %\centering
    \includegraphics[width=\textwidth]
    {pics/world10fin_performance.pdf}
    \caption{\textit{world $10 \times 10$}}
    \end{subfigure}
    ~
    \begin{subfigure}{0.3\textwidth}
    %\centering
    \includegraphics[width=\textwidth]
    {pics/world18fin_performance.pdf}
    \caption{\textit{World $18 \times 18$}}
    \end{subfigure}
~
    \begin{subfigure}{0.3\textwidth}
    \centering
    \includegraphics[width=\textwidth]
    {pics/world27fin_performance.pdf}
    \caption{\textit{WORLD $27 \times 27$}}
    \end{subfigure}
\caption{Dependence of the final error loss function (FSE) on the number of particles used in the PF and MKF in the considered symmetric environments. MKF provides more accurate filtering of the states and requires fewer particles for convergence of FSE. Our filtering method is also less sensitive to the estimate of the motion noise than the particle filter.}
\label{fig::Fin}
\end{figure}


\begin{figure}[!ht]
    %\centering
    \begin{subfigure}{0.3\textwidth}
    %\centering
    \includegraphics[width=\textwidth]
    {pics/world10_1fin_performance.pdf}
    \caption{\textit{n-world $10 \times 10$}}
    \end{subfigure}
    ~
    \begin{subfigure}{0.3\textwidth}
    %\centering
    \includegraphics[width=\textwidth]
    {pics/world18_1fin_performance.pdf}
    \caption{\textit{n-World $18 \times 18$}}
    \end{subfigure}
~
    \begin{subfigure}{0.31\textwidth}
    \centering
    \includegraphics[width=\textwidth]
    {pics/world27_1fin_performance.pdf}
    \caption{\textit{n-WORLD $27 \times 27$}}
    \end{subfigure}
\caption{Dependence of the final error loss function (FSE) on the number of particles used in the PF and MKF in the considered non-symmetric environments. 
MKF provides more accurate filtering of the states and requires fewer particles for the convergence of FSE. Our method is also less sensitive to the estimate of the motion noise than the particle filter.}
\label{fig::Fin1}
\end{figure}

Last but not least comparison of the particle filter and the multiparticle Kalman filter is performed in the \empty{Labyrinth} environment (see Figure~\ref{fig::labyrinth}).
Figure~\ref{fig::lab_mse_fse} shows that the proposed filtering method outperforms the particle filter in terms of both MSE and FSE quality criteria.
Also, we again observe the smaller number of particles required for the convergence of both loss functions.
The proposed method is more robust with respect to the motion noise level than the particle filter, which is aligned with previous results.

\begin{figure}[!ht]
%\centering
\begin{subfigure}{0.45\textwidth}
\includegraphics[width=\textwidth]
    {pics/world_lab_performance.pdf}
\end{subfigure}
    ~
    \begin{subfigure}{0.45\textwidth}
\includegraphics[width=\textwidth]
    {pics/world_labfin_performance.pdf}
\end{subfigure}
    \caption{Dependence of MSE (left) and FSE (right) values on the number of particles in the \emph{Labyrinth} environment. Our method (MKF) is also less sensitive to the estimate of the motion noise than the particle filter.}
    \label{fig::lab_mse_fse}
\end{figure}

\paragraph{Variance analysis.}
To make the previous plots clear, we do not provide confidence intervals there.
To fill this gap in the reporting comparison results, we summarize the FSE values and the corresponding variance in Table~\ref{tab::variance_comparison}.
This table shows that the MKF provides a more accurate and less variable estimation of the final state for both symmetric and nonsymmetric environments.
This gain is observed uniformly with respect to the considered range of the number of particles.

\begin{table}[!h]
%\fontsize{7pt}{7pt}\selectfont
    \centering
    \caption{FSE values and variance comparison of particle filter (PF) and the proposed multiparticle Kalman filter (MKF). Standard deviation is given in braces near the corresponding mean FSE value. In these simulations, we use $\mM = 4\mM_0$, which is equal to $4\Sigma$ setting.}
    \begin{adjustbox}{width=\columnwidth,center}
    \begin{tabular}{ccccccccccc}
    \toprule
    Number of particles & \multicolumn{2}{c}{$N=100$} & \multicolumn{2}{c}{$N=500$} & \multicolumn{2}{c}{$N=1000$} & \multicolumn{2}{c}{$N=4000$} & \multicolumn{2}{c}{$N=10000$}\\
    \midrule
    Environment &  PF & MKF &  PF & MKF & PF & MKF & PF & MKF & PF & MKF \\
    \cmidrule(lr){1-1} \cmidrule(lr){2-3} \cmidrule(lr){4-5} \cmidrule(lr){6-7} \cmidrule(lr){8-9} \cmidrule(lr){10-11}
        \textit{world 10} & 5.37 (3.53) & 4.86 (3.76) & 5.13 (3.91) & 4.71 (2.78) & 4.93 (3.87) & 4.65 (2.44) & 4.85 (3.31) & 4.57 (2.05) & 4.8 (2.82) & 4.55 (1.98) \\
        \textit{n-world 10} & 3.07 (3.63) & 0.24 (1.26) & 1.91 (3.25) & 0.03 (0.07) & 1.40 (2.88) & 0.43 (1.67) & 0.04 (0.21) & 0.11 (0.70) &0.03 (0.02) & 0.03 (0.02) \\
        \textit{World 18} & 10.10 (6.49) & 9.22 (7.87) & 9.40 (7.72) & 8.41 (7.22) & 8.89 (8.20) & 8.38 (5.94) & 8.47 (7.79) & 8.34 (3.62) & 8.38 (6.64) & 8.34 (3.18) \\
        \textit{n-World 18} & 6.56 (6.94) & 3.87 (6.37) & 4.91 (6.89) & 1.70 (4.54) & 3.91 (6.46) & 1.30 (3.75) & 1.83 (4.86) & 1.13 (2.99) & 1.24 (3.77) & 1.15 (2.94) \\
        \textit{WORLD 27} & 13.41 (7.50) & 11.14 (7.62)  & 11.49 (7.48) & 7.26 (6.33) & 10.19 (7.21) & 5.96 (5.55) & 7.13 (6.27) & 4.89 (4.42) & 5.71 (5.47) & 4.80 (4.23) \\
        \textit{n-WORLD 27} & 11.84 (8.60) & 9.35 (8.41) & 9.97 (8.32) & 5.77 (6.86) & 8.68 (8.03) & 4.34 (5.89) & 5.66 (6.75) & 2.91 (4.39) & 3.95 (5.56) & 2.83 (4.25) \\
        \emph{Labyrinth} & 10.45 (10.11) & 1.83 (5.65) & 6.84 (9.53) & 0.06 (0.68) & 4.83 (8.49) & 0.03 (0.02) & 1.00 (4.23) & 0.03 (0.02) & 0.11 (1.17) & 0.03 (0.02) \\
        \bottomrule
    \end{tabular}
    \end{adjustbox}
    \label{tab::variance_comparison}
\end{table}


% As it is seen in the Figures the particle filter performance essentially depends both on the number of particles and the noise level added. Kalman particle filter requires much less particles to show comparable results both for MSE and final points. Moreover, it shows weak dependence on the noise level, which is a significant circumstance as it doesn't need an additional exploration stage of hyperparameter matching.

% The results for best cases (i.e. $4\Sigma$) for particle and Kalman particle  filters at various particles number are used as summarized in Table \ref{tab::variance_comparison}. In all cases, \textbf{PFKU} outperforms the particle filter in terms of mean final error and its standard deviation. 

\paragraph{Runtime comparison.}
In the previous sections, we demonstrate the performance of the proposed filtering method in terms of the required number of particles for convergence of MSE and FSE and the smaller variance of these quantities compared to the particle filter.
Here, we provide the runtime comparison of the proposed filtering method and the particle filter.
In this experiment, we simulate $10000$  trajectories in the considered environments and provide the total runtime of such a simulation. 
Since the runtime of both compared methods significantly depends on the used number of particles, we consider $2000$ and $5000$ particles in the particle filter simulations and report the resulting FSE values.
Then, we tune the number of particles in the MKF such that the resulting FSE values are the same or slightly smaller than the corresponding FSE in the particle filter simulations.
The measured runtime, FSE, and the numbers of particles are shown in Table~\ref{tab::time_comparison}.
From this Table follows that the proposed multiparticle Kalman filter is typically 3-4 times faster than the particle filter.
This observation indicates that the gain from the reduction of the number of particles dominates the increasing per-iteration complexity of the proposed method.

% Here computations were performed with particle filter at $2000$ and $5000$ number of particles. 
% Later the \textbf{PFKU} has been run at different numbers of particles till the final coordinate reaches the same error or lower. 
% The required number of particles and computational times are compared. 
% The new algorithm needs at least an order of magnitude fewer particles than the standard particle filter and is typical $3-4$ times faster, which compensates for its extra complexity, which has been discussed earlier.


\begin{table}[!h]
%\fontsize{7pt}{7pt}\selectfont
    \centering
    \caption{Comparison of the total runtime of particle filter (PF) and the proposed multiparticle Kalman filter (MKF) to simulate 10000 trajectories in the considered environments. The number of particles required for the MKF is set such that it achieves the same or slightly smaller FSE compared to values from PF simulations.}
    \begin{adjustbox}{width=0.8\columnwidth,center}
    \begin{tabular}{ccccccc}
    \toprule
     & \multicolumn{3}{c}{PF} & \multicolumn{3}{c}{MKF} \\
    \cmidrule(lr){2-4} \cmidrule(lr){5-7} 
    Environment & \# particles & FSE & Time, s & \# particles & FSE & Time, s\\
    \midrule
        \textit{world 10} & 2000 & 5.03 & 156 & 100 & 4.92 & 58 \\
        \textit{n-world 10} & 2000 & 0.87 & 151 & 100 & 0.80 & 46 \\
        \textit{World 18} & 2000 & 9.26 & 186 & 200 & 8.96 & 80 \\
        \textit{n-World 18} & 2000 & 4.87 & 168 & 100 & 4.61 & 50 \\
        \textit{WORLD 27} & 2000 & 11.0 & 204 & 200 & 10.62 & 86 \\
        \textit{n-WORLD 27} & 2000 & 9.58 & 211 & 150 & 9.45 & 70 \\
        \emph{Labyrinth} & 2000 & 4.8 & 139 & 100 & 3.02 & 48 \\
        \midrule
        \textit{world 10} & 5000 & 4.72 & 368 & 150 & 4.84 & 63 \\
        \textit{n-world 10} & 5000 & 0.30 & 334 & 250 & 0.25 & 90 \\
        \textit{World 18} & 5000 & 8.81 & 415 & 200 & 0.20 & 75 \\
        \textit{n-World 18} & 5000 & 3.49 & 412 & 300 & 3.20 & 111 \\
        \textit{WORLD 27} & 5000 & 9.50 & 489 & 400 & 9.00 & 153 \\
        \textit{n-WORLD 27} & 5000 & 7.94 & 473 & 400 & 7.54 & 151 \\
        \emph{Labyrinth} & 5000 & 2.69 & 323 & 150 & 1.86 & 61 \\
        \bottomrule
    \end{tabular}
    \end{adjustbox}
    \label{tab::time_comparison}
\end{table}

% \begin{figure}[!ht]
% %\centering
%     \hspace{2.8cm}
%     \includegraphics[width=7cm, clip]
%     {pics/world_labfin_performance.pdf}
%     \caption{Dependence of an absolute error for a final coordinate on the number of particles in the \emph{Labyrinth}}
%     \label{fig::final_error_lab}
% \end{figure}


% \begin{table}[!h]
% %\fontsize{7pt}{7pt}\selectfont
%     \centering
%     \caption{Comparison of particle filter (PF) and the proposed Kalman particle filter (MKF). Final point mean and standard deviation are shown. $4\Sigma$ level is used for twice bigger noise.}
%     \begin{adjustbox}{width=\columnwidth,center}
%     \begin{tabular}{c|cc|cc|cc|cc|cc}
%     Number of particles & \multicolumn{2}{c|}{$N=100$} & \multicolumn{2}{c|}{$N=500$} & \multicolumn{2}{c|}{$N=1000$} & \multicolumn{2}{c|}{$N=4000$} & \multicolumn{2}{c}{$N=10000$}\\
%     \hline
%     Environment & PF & MKF & PF & MKF & PF & MKF & PF & MKF & PF & MKF \\
%     \hline
%         world 10 & 5.37 (3.53) & 4.86 (3.76) & 5.13 (3.91) & 4.71 (2.78) & 4.93 (3.87) & 4.65 (2.44) & 4.85 (3.31) & 4.57 (2.05) & 4.8 (2.82) & \textbf{4.55 (1.98)} \\
%         World 18 & 10.10 (6.49) & 9.22 (7.87) & 9.40 (7.72) & 8.41 (7.22) & 8.89 (8.20) & 8.38 (5.94) & 8.47 (7.79) & 8.34 (3.62) & 8.38 (6.64) & \textbf{8.34 (3.18)} \\
%         WORLD 27 & 13.41 (7.50) & 11.14 (7.62)  & 11.49 (7.48) & 7.26 (6.33) & 10.19 (7.21) & 5.96 (5.55) & 7.13 (6.27) & 4.89 (4.42) & 5.71 (5.47) & \textbf{4.80 (4.23)} \\
%         world 10 a & 2.17 (3.55) & 0.24 (1.26) & 0.71 (2.21) & 0.03 (0.07) & 0.33 (1.49) & 0.03 (0.02) & 0.04 (0.21) & 0.03 (0.02) &0.03 (0.02) & \textbf{0.03 (0.02)} \\
%         World 18 a & 6.56 (6.94) & 3.87 (6.37) & 4.91 (6.89) & 1.70 (4.54) & 3.91 (6.46) & 1.30 (3.75) & 1.83 (4.86) & 1.13 (2.99) & 1.24 (3.77) & \textbf{1.15 (2.94)} \\
%         WORLD 27 a & 11.84 (8.60) & 9.35 (8.41) & 9.97 (8.32) & 5.77 (6.86) & 8.68 (8.03) & 4.34 (5.89) & 5.66 (6.75) & 2.91 (4.39) & 3.95 (5.56) & \textbf{2.83 (4.25)} \\
%         \emph{Labyrinth} & 10.45 (10.11) & 1.83 (5.65) & 6.84 (9.53) & 0.06 (0.68) & 4.83 (8.49) & 0.03 (0.02) & 1.00 (4.23) & 0.03 (0.02) & 0.11 (1.17) & \textbf{0.03 (0.02)} 
%     \end{tabular}
%     \end{adjustbox}
%     \label{tab::variance_comparison}
% \end{table}

% As it is seen in Figures the particle filter performance essentially depends both on the number of particles and noise level added. Kalman particle filter requires much less particles to show comparable results both for MSE and final points. Moreover, it shows weak dependence on the noise level, which is very important circumstance as doesn't need additional exploration stage of hyperparameter matching.

% The results for best cases (i.e. $4\Sigma$) for particle and Kalman particle  filters at various particles number used as summarized in Table \ref{tab::variance_comparison}. In all cases Kalman particle outperforms particle filter in terms of mean final error and its standard deviation. This algorithm needs about an order of magnitude less particles than standard particle filter, which compensates its extra complexity, which has been discussed earlier.