\begin{table}[t]
% \vspace{-0.2cm}
% \scriptsize
\resizebox{\columnwidth}{!}{
\begin{tabular}[t]{lcccccc }  %p{0.1\textwidth}
\toprule
& \multicolumn{2}{c}{\textbf{Results on MPII}} & \\
\midrule 
\multirow{2}{*}{Method} & Trained with  & \multirow{2}{*}{T} & \multirow{2}{*}{R} & \multirow{2}{*}{D} & FID/& KID/ \\ 
& paired data & & & & H-FID & H-KID \\
\midrule 
Brooks et al.
 & \cmark & \xmark & \xmark  & \cmark & 109.0 / 75.5 & 0.10 / 0.07 \\ 
\midrule
\multirow{6}{*}{\Ours{}}
 & \xmark & \xmark & \cmark  & \xmark & 95.6 / 59.3 & 0.07 / 0.05 \\ 
 & \xmark & \cmark & \xmark  & \xmark & 54.6 / \underline{41.3} & 0.03 / 0.03 \\ 
 & \xmark & \cmark & \cmark  & \xmark & \underline{44.6} / 41.5 & \underline{0.02} / \underline{0.03}  \\ 
 & \xmark & \xmark & \cmark  & \cmark & 95.3 / 58.1 & 0.07 / 0.04 \\ 
 & \xmark & \cmark & \xmark  & \cmark & 72.8 / 136.2 & 0.03 / 0.11 \\ 
 & \xmark & \cmark & \cmark  & \cmark & \textbf{42.6} / \textbf{38.6} & \textbf{0.02} / \textbf{0.02} \\ 
\midrule
& \multicolumn{2}{c}{\textbf{Results on SMART}} & \\
\midrule 
Brooks et al.
 & \cmark & \xmark & \xmark  & \xmark & 175.8 / 114.1 & 0.14 / 0.06 \\ 
\midrule
\multirow{6}{*}{\Ours{}}
 & \xmark & \xmark & \cmark  & \xmark & 85.9 / 121.5 & 0.06 / 0.07 \\ 
 & \xmark & \cmark & \xmark  & \xmark & 162.3 / 113.9 & 0.13 / 0.08 \\ 
 & \xmark & \cmark & \cmark  & \xmark & 94.8 / \underline{99.8} & \underline{0.06} / \underline{0.05} \\ 
 & \xmark & \xmark & \cmark  & \cmark & \textbf{85.5} / 122.2 & 0.06 / 0.07 \\ 
 & \xmark & \cmark & \xmark  & \cmark & 145.9 / 131.5 & 0.10 / 0.07 \\ 
 & \xmark & \cmark & \cmark  & \cmark & \underline{92.4} / \textbf{44.5} & \textbf{0.06} / \textbf{0.03} \\ 
\bottomrule
\end{tabular}
}
\caption{Pose-conditioned generation quality. 
Note that \citet{brooks2022hallucinating} was trained with paired data (images with corresponding 2D keypoints), whereas \Ours{} is trained/finetuned only with images.
\Ours{} can take the background of text (``T") and/or a real image (``R") as conditioning information.}
% In-domain (``D") means the model had access to the dataset during training (for \cite{brooks2022hallucinating}) or finetuning (for \Ours{}). \textbf{Bold} and \underline{underlined} indicate best and next best numbers. 
\vspace{-0.4cm}
\label{table:pose_generation}
\end{table}