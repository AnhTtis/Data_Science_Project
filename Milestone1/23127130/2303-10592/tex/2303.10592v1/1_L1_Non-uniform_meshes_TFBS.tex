\documentclass [11pt,a4paper]{article}
\usepackage{amsfonts,amssymb,amsmath,amsthm}
\usepackage{a4wide}
\usepackage{graphicx}
\input{epsf.sty}
\usepackage{subfigure}
\usepackage{epsfig}
\usepackage{footmisc}
\usepackage{color}
\usepackage{hyperref}
\usepackage{breakurl}
\usepackage{indentfirst}
\usepackage{amscd,enumerate,latexsym}
\usepackage{multirow}
\usepackage{booktabs,bm}
\usepackage{cite}
\newtheorem{theorem}{Theorem}[section]
\newtheorem{lemma}[theorem]{Lemma}%[section]
\newtheorem{definition}[theorem]{Definition}%[section]
\newtheorem{corollary} [theorem]{Corollary}%[section]
%\newtheorem{assumption}[theorem]{Assumption}
%
\newtheorem{example}{Example}
\newtheorem{remark}{Remark}
%\newtheorem{hypothesis}{Hypothesis}
\numberwithin{equation}{section}
\newtheorem{proposition}{Proposition}
\newtheorem{algorithm}{Algorithm}
\newcommand{\be}{\begin{equation}}
\newcommand{\ee}{\end{equation}}
\newcommand\bea{\begin{eqnarray}}
\newcommand\eea{\end{eqnarray}}
\newcommand{\bean}{\begin{eqnarray*}}
\newcommand{\eean}{\end{eqnarray*}}
\def\nn{\nonumber}
\begin{document}
\title{\LARGE{A fast compact difference scheme with unequal time-steps
for the tempered time-fractional Black-Scholes model}}
\author{Jinfeng Zhou\footnotemark[1],~~Xian-Ming Gu$^{\S,}$\footnotemark[1],~~Jinye Shen$^{\S,}
$\footnotemark[1],~~Yong-Liang Zhao\footnotemark[2],~~Hu Li\footnotemark[3]}

\maketitle
\footnotetext[1]
{\footnotesize School of Mathematics, Southwestern University of Finance and Economics, Chengdu,
Sichuan 611130, P.R. China. E-mail: {\tt 2207441009@qq.com}; {\tt guxianming@live.cn},
{\tt guxm@swufe.edu.cn} (corresponding author); {\tt shenjy@swufe.edu.cn} (corresponding author)}
\footnotetext[2]
{\footnotesize School of Mathematical Sciences, Sichuan Normal University, Chengdu, Sichuan 610068, P.R. China. E-mail:
{\tt ylzhaofde@sina.com}}
\footnotetext[3]
{\footnotesize School of Mathematics, Chengdu Normal University, Chengdu, Sichuan 611130, P.R. China. E-mail:
{\tt lihu\_0826@163.com}}


\begin{abstract}
\noindent The Black-Scholes (B-S) equation has been recently extended
as a kind of tempered time-fractional B-S equations, which become an
interesting mathematical model in option pricing. In this study, we
provide a fast numerical method to approximate the solution of the tempered
time-fractional B-S model. To achieve high-order accuracy in space and
overcome the weak initial singularity of the solution, we combine
the compact operator with L1 approximation with nonuniform time
steps to yield the numerical scheme. The convergence of the proposed difference scheme is proved to
be unconditionally stable. Moreover, the kernel function in tempered Caputo fractional
derivative is approximated by sum-of-exponentials, which leads to
a fast unconditional stable compact difference method that reduces the computational cost.
Finally, numerical results demonstrate the effectiveness of the proposed methods.

\end{abstract}


\noindent
{\bf Keywords:} Tempered time-fractional B-S model; nonuniform time steps; exponential transformation; compact difference scheme.

\noindent {\bf AMS subject classifications:} 65M06, 65M50, 26A33, 91G20.

\section{Introduction}
%%%%%%%%%%%%%%%%%%%%%%%%%%%%
%\setcounter{equation}{0}
The Black-Scholes (B-S) model is an important method of option pricing because of its clever combination of option pricing, random fluctuation of underlying asset price and risk-free interest rate \cite{ksendal03}. However, the classic B-S model was put forward by Black and Scholes under a series of assumptions \cite{Staelen}. Through observation and research on the stock market, many scholars found that the essential characteristics and state of the capital market are random fluctuations, which are not completely consistent with these assumptions of the traditional B-S option pricing model. The pricing of the model is different from the actual market price. Therefore, in order to extend the application of B-S equation from the ideal state of stock price to a more realistic state, many scholars have done a lot of work, such as B-S model with transaction cost \cite{Meng2010}, stochastic interest rate model \cite{Nicolas}, jump-diffusion model \cite{Kou02}, etc.

Besides, in order to make Brownian motion reflect more properties such as autocorrelation, long-term memory and incremental correlation, some other scholars began to consider modifying the original partial differential equation of Brownian motion which leads to new option pricing models which are more suitable for the actual financial market. With the discovery of the fractal structure of differential equations in the financial field, more and more scholars began to pay attention to the fractional differential equation in the financial field. In 2000, Wyss \cite{Wyss2000} first applied the idea of fractal to the financial field and deduced the time-fractional B-S (TFBS) option pricing model. Later, Jumariel \cite{Jumarie2008} use the fractional Taylor formula to deduce and demonstrate the TFBS option pricing formula, Cartea \cite{Cartea13} propose to model stock price tick-by-tick data via a non-explosive marked point process where the model equation satisfied by the value of European-style derivatives contains a Caputo fractional derivative in time-to-maturity. Liang et al. \cite{Liang2010} proposed a special TFBS equation based on the real market option price analysis in a fractional transfer system. Magdziarz \cite{Magdziarz9} consider a generalization of this model, which is based on a subdiffusive geometric Brownian motion and captures the subdiffusive characteristics of financial markets. There are also other TFBS models available
at \cite{Chen15,Farhadi,Meng2010}.

With the development of time-fractional B-S model, more and more scholars begin to pay attention to its solution which is hard to be solved in the analytical manner \cite{Liang2010,Chen15,Fadugba}. Thus it is necessary to study efficient numerical methods for such models. Krzy\`{z}anowskia et al. \cite{Grzegorz20} present the weighted finite difference method to solve the subdiffusive B-S model numerically and prove its convergence. In \cite{Zhuang16}, the numerical solution of the TFBS model governing European options is obtained by using the L1 approximation of Caputo fractional derivatives, which can achieve ($2-\alpha$)-order accuracy in time and second-order accuracy in space. Tian et al. \cite{Tian20} proposed three compact finite difference schemes for the TFBS model governing European option pricing where the time fractional derivative is approximated by L1 formula, L2-1$_\sigma$ formula and L1-2 formula, then convergence orders of three compact difference schemes are fourth-order in space and $(2-\alpha)$-, 2-, and $(3-\alpha)$-order in time, respectively. Staelen and Hendy \cite{Staelen} studied an implicit numerical scheme with a temporal accuracy of ($2-\alpha$)-order and spatial accuracy of fourth-order by using the Fourier analysis method. In addition, some other related numerical methods based on differential spatial discrezations
for the TFBS model can be found in
\cite{Dimitrov,Kazmi22,Roul20,Abdi2022,Sarboland,Koleva17,An2021,Akram22,Rezaei21}

%Other about tempered fractional calculus details are available in \cite{Guo2019}.

It is worth noting that the above numerical methods can reach the theoretical convergence
order with the assumption that the exact solution of TFBS model is sufficiently
smooth in the time variable. But in fact, the solutions of time-fractional differential equations
always show weak singularities near the initial time, which makes most of the above numerical
methods to achieve optimal order convergence \cite{Stynes17,Gracia18}. Therefore, in order to overcome
the weak singularity of exact solution, numerical methods with variable-step size introduced
in \cite{Stynes17,Liao2018,Shen2018} have been considered to solve the model problem; see e.g.,
\cite{Cen2018,Luchko09} for details. In order to overcome the difficulty of initial layer, She
et al. \cite{She2021} present modified L1 time discretization is presented based on a change of variable for
solving the TFBS model. However all above the numerical methods need huge storage and
computational cost due to the nonlocal time fractional derivative. In order to reduce the computational cost, Song and Lyu \cite{Kerui21} use the fast sum-of-exponentials (SOE) approximation
of of Caputo fractional derivative \cite{Jiang17,Shen2018} to present a fast numerical method
for TFBS equations, the fast algorithm keeps the second-order accuracy of in time and fourth-order accuracy. Moreover, it reduces the computational computational complexity significantly. The stability of scheme which they proposed is established base on the analysis framework developed in \cite{Liao19}.

When we consider the option pricing in such a stagnated market, the tempered TFBS model does better in estimating the fair price than the classic B-S model, thus Krzyzanowskia
and Magdziarza \cite{Wang2016} proposed the tempered subdiffusive B-S model assumed that the
underlying asset is driven by $\alpha$-stable $\lambda$-tempered inverse subordinator \cite{Cartea2007,Meerschaert}. In this paper, we are also interested in the abovel tempered TFBS model governing
European options:
\begin{equation}
\begin{cases}
\frac{\partial^{\alpha,\lambda}C(S,t)}{\partial t^{\alpha,\lambda}} + \frac{1}{2}\sigma^2S^2
\frac{\partial^2 C(S,t)}{\partial S^2} + \hat{r}S\frac{\partial C(S,t)}{\partial S}
-rC(S,t) = 0,& (S,t)\in(0,+\infty)\times[0,T),\\
C(S,T) = \mu(S),& S\in[0,+\infty),\\
C(0,t) = \phi(t),\quad C(+\infty,t) = \varphi(t),& t\in[0,T),
\end{cases}
\label{eq1.1}
\end{equation}
where $\alpha\in(0,1)$ and $\lambda \geq 0$, $T$ is the expiry time, $
\hat{r} = r - D$ ($r >0$ and $D\geq 0$ are the risk-free rate and the dividend yield,
respectively) and $\sigma~(> 0)$ is the volatility of the returns from
the holding stock price $S$. Here the terminal condition is typically
chosen as $\mu(S)$ for the payoff of the option. The fractional derivative operator in
Eq. (\ref{eq1.1}) is a modified right Riemann-Liouville tempered fractional derivative defined as
\begin{equation}
\frac{\partial^{\alpha,\lambda} C(S,t)}{\partial t^{\alpha,\lambda}}
= \frac{e^{-\lambda (T-t)}}{\Gamma(1 - \alpha)}\frac{\partial}{\partial t}
\int^{T}_t\frac{e^{-\lambda(\xi - T)}C(S,\xi)-C(S,T)}{(\xi - t)^{\alpha}}d\xi,
\end{equation}
where $\alpha = 1$ and $\lambda=0$, the model (\ref{eq1.1}) becomes the classical
B-S model.

Let $t = T - \tau$ and $\alpha\in(0,1)$, then we have
\begin{equation}
\begin{split}
\frac{\partial^{\alpha,\lambda} C(S,t)}{\partial t^{\alpha,\lambda}}
& = \frac{e^{-\lambda (T-t)}}{\Gamma(1 - \alpha)}\frac{\partial}{\partial t}\int^{T}_t\frac{e^{
-\lambda(\xi - T)}C(S,\xi)-C(S,T)}{(\xi - t)^{\alpha}}d\xi\\
& = \frac{e^{-\lambda \tau}}{\Gamma(1 - \alpha)}\frac{-\partial}{\partial \tau}\int^{T}_{T-\tau}
\frac{e^{-\lambda(\xi - T)}C(S,\xi)-C(S,T)}{[\xi - (T - \tau)]^{\alpha}}d\xi\\
& = \frac{-e^{-\lambda \tau}}{\Gamma(1 - \alpha)}\frac{\partial}{\partial \tau}
\int^{\tau}_0\frac{e^{\lambda\eta}C(S,T - \eta)-C(S,T)}{(\tau - \eta)^{\alpha}}d\eta
\end{split}
\end{equation}

Moreover, supposing $e^x = S$ and denoting $U(x,\tau) = C(e^x,T-\tau)$,
then model (\ref{eq1.1}) can be rewritten as
\begin{equation}
\begin{cases}
{}_0D^{\alpha,\lambda}_{\tau}U(x,\tau) =
%\frac{\partial^{,\lambda}C(S,t)}{\partial t^{\alpha}} = ^{C}S^2
\frac{1}{2}\sigma^2\frac{\partial^2 U(x,\tau)}{\partial x^2} +
c\frac{\partial U(x,\tau)}{\partial x} - rU(x,\tau),& (x,\tau)
\in\mathbb{R}\times(0,T],\\
U(x,0) = \kappa(x),& x\in\mathbb{R},\\
U(-\infty,\tau) = \phi(\tau),\quad U(+\infty,\tau) = \varphi(\tau),& \tau\in(0,T],
\end{cases}
\label{eq1.2}
\end{equation}
where $c = \hat{r} - \frac{1}{2}\sigma^2$, $\kappa(x) = \mu(e^x)$ and the tempered fractional derivative reads
\begin{equation*}
\begin{split}
{}_0D^{\alpha,\lambda}_{\tau}U(x,\tau) &= \frac{e^{-\lambda \tau}}{\Gamma(1 - \alpha)}\frac{\partial}
{\partial \tau}\int^{\tau}_0\frac{e^{\lambda \eta}U(x,\eta) - U(x,0)}{(\tau
- \eta)^{\alpha}}d\eta\\
& = \frac{e^{-\lambda\tau}}{\Gamma(1 - \alpha)}\frac{\partial}
{\partial \tau}\int^{\tau}_0\frac{e^{\lambda \eta}U(x,\eta)}{(\tau
- \eta)^{\alpha}}d\eta - \frac{e^{-\lambda\tau}}{\Gamma(1 - \alpha)}\frac{\partial}
{\partial \tau}\int^{\tau}_0\frac{U(x,0)}{(\tau - \eta)^{\alpha}}d\eta\\
& = \frac{e^{-\lambda\tau}}{\Gamma(1 - \alpha)}\frac{\partial}
{\partial \tau}\int^{\tau}_0\frac{e^{\lambda \eta}U(x,\eta)}{(\tau
- \eta)^{\alpha}}d\eta - \frac{e^{-\lambda\tau}\tau^{-\alpha}}{\Gamma(1 - \alpha)}U(x,0)\\
& = \frac{e^{-\lambda\tau}}{\Gamma(1 - \alpha)}\int^{\tau}_0\frac{1}{(\tau
- \eta)^{\alpha}}\cdot\frac{\partial [e^{\lambda \eta}U(x,\eta)]}{\partial \eta}d\eta\\
%& = e^{-\lambda\tau}{}^{C}_0D^{\alpha}_{\tau}\left[e^{\lambda\tau}U(x,\tau)\right], \\
& \triangleq {}^{C}_0\mathbb{D}^{\alpha,\lambda}_{\tau}U(x,\tau)
\end{split}
%n = 0,1,\ldots,N,%
\end{equation*}
with the operator ${}^{C}_0\mathbb{D}^{\alpha,\lambda}_{\tau}$ being the tempered Caputo fractional
derivative \cite{Zhao20}.
%,\quad \alpha\in(0,1).
%{(1 - \alpha)}\int^{t}_0 \frac{\partial
%v(x,s)}{ s}(t - s)^{-\alpha},\qquad \alpha

To solve the above model numerically, it always truncates the original unbounded
domain of variable $x$ in problem (\ref{eq1.2}) into a finite interval $[x_l,x_r]$.
Then the general model we consider is of the following form:
\begin{equation}
\begin{cases}
{}^{C}_0\mathbb{D}^{\alpha,\lambda}_{\tau}U(x,\tau) =
\frac{1}{2}\sigma^2\frac{\partial^2 U(x,\tau)}{\partial x^2} +
c\frac{\partial U(x,\tau)}{\partial x} -rU(x,\tau)+ f(x,\tau),& (x,\tau)
\in\Omega\times(0,T],\\
U(x,0) = \zeta(x),& x\in\Omega,\\
U(x_l,\tau) = \phi(\tau),\quad U(x_r,\tau) = \varphi(\tau),& (x,\tau)\in\partial\Omega\times(0,T],
\end{cases}
\label{eq1.2x}
\end{equation}
where $\Omega = [x_l,x_r]$ and a source term $f(x, \tau)$ is added for the purposes of validation in
Section \ref{sec4} without loss of generality. Meanwhile, the initial condition is
accordingly chosen as $\zeta(x) = \max(K - e^x,0)$ (which is only continuous)
as suggested by the model (\ref{eq1.1}). In addition, the above equation (\ref{eq1.2x})
can be viewed a special case of tempered time-fractional advection-diffusion
equations \cite{Meersc08}.

In fact, the authors of \cite{Wang2016} present a finite difference method that has the $(2 -\alpha)$
and 2 order of accuracy with respects to time and space respectively to the option
pricing in the considered model (\ref{eq1.1}). However, such a study
seems to be less efficient because it overlooks two numerical difficulties of the tempered
TFBS model, i.e., the nonlocal time fractional derivative and the weak initial singularity
of the exact solution. In order to overcome the above numerical difficulties, we first transform
the model (\ref{eq1.2x}) into a tempered time-fractional diffusion-reaction (tTFDR) equation
with homogeneous boundary conditions, which holds the relation of same solution. Then the regularity
of the solution of the transformed tTFDRE equation is investigated by the method of variable separation
that is different from the idea suggested in \cite{Morgado}. Moreover, an implicit difference method
with the compact difference operator in space and the graded L1 formula in time will be derived for solving the
the transformed tTFDR equation. To reduce the computational cost caused by the nonlocal
tempered Caputo derivative, we extend the popular fast SOE approximation \cite{Jiang17,Shen2018}
to reconstruct a fast difference method for the transformed tTFDR equation. At the same time,
the proposed schemes are proved to be unconditional stable and convergent with the $(2 -\alpha)$
and 4 order of accuracy with respects to time and space, respectively. Finally, we report
some numerical experiments to examine the feasibility of the proposed methods.

The rest of the paper is organized as follows. In Section \ref{sec2}, we use the non-uniform
temporal discretization to establish a compact difference scheme for solving
the equivalent tTFDR equation. Moreover, the unconditional stability and the convergence
of $\min\left\{r\alpha,2-\alpha\right\}$-order in time and fourth-order in space for the proposed scheme are well displayed by mathematical introductions. In Section \ref{sec3}, we establish a fast
compact difference scheme that reduces the computational cost and then state the convergence for solving
the equivalent tTFDR equation. Numerical examples are provided
in Section \ref{sec4} to demonstrate the theoretical statement. A brief conclusion is followed in Section
\ref{sec5}.
%%%%%%%%%%%%%%%%%%%%
\section{Construction of the compact difference scheme}
\label{sec2}
%\setcounter{equation}{0}
In this section. We first establish a direct finite difference scheme for solving
the equivalent tTFDR equation. Since it is well-known that the solution of (\ref{eq1.2}) always
has the weak singularity at the initial time (refer to the {\em Appendix \ref{appd}} for details), i.e. the solution will be not smooth enough
near the initial time, then the classical L1 scheme cannot
achieve the optimal convergence order of $(2-\alpha)$. Alternatively, the non-uniform
temporal discretization is proved to be a reliable numerical technique for solving the
equivalent tTFDR equation.
\subsection{The equivalent reformulation of tTFDR equations}
%%%%%%%%%%%%%%%%%%%
In order to simplify the derivation of the high-order spatial discrezation for the model (\ref{eq1.2x}), we first denote $w(x,\tau):= U(x,\tau) - z(x,\tau)$, where
\begin{equation}
z(x,\tau):=\frac{\varphi(\tau) - \phi(\tau)}{x_r - x_l}(x - x_l) + \phi(\tau),
\end{equation}
It is not hard to note that the problem (\ref{eq2.1}) is equivalent to the next equations with homogeneous boundary conditions:
\begin{equation}
\begin{cases}
{}^{C}_0\mathbb{D}^{\alpha,\lambda}_{\tau}w(x,\tau) = \frac{1}{2}\sigma^2\frac{\partial^2 w(x,\tau)}{\partial x^2} +
c\frac{\partial w(x,\tau)}{\partial x} -rw(x,\tau)+ \tilde{f}(x,\tau),& (x,\tau)\in\Omega\times(0,T],\\
w(x,0) = \tilde{\zeta}(x),& x\in\Omega,\\
w(x_l,\tau) = w(x_r,\tau) = 0,&(x,\tau)\in\partial\Omega\times(0,T],
\end{cases}
\label{modifiedxx}
\end{equation}
where $\tilde{f}(x,\tau) = f(x,\tau) + c\frac{\varphi(\tau) - \phi(\tau)}{x_r - x_l} - rz(x,\tau) - {}^{C}_0\mathbb{D}^{\alpha,
\lambda}_{\tau}z(x,\tau)$\footnote{Due to two known functions $\phi(\tau)$ and $\varphi(\tau)$, it is easy to compute
the term ${}^{C}_0\mathbb{D}^{\alpha,\lambda}_{\tau}z(x,\tau)$ in the analytical (or numerical) manner.}
and $\tilde{\zeta}(x) = \zeta(x) - \frac{\varphi(0) - \phi(0)}{x_r - x_l}(x - x_l) -\phi(0)$. On the other hand, we introduce the following functions
%ds^^{}
\begin{equation*}
k(x) = \exp\left(\frac{c(x - x_l)}{\sigma^2}\right),\quad v(x,\tau) = k(x)w(x,\tau),
\end{equation*}
According to some tedious but simple calculations, we transform the equation (\ref{eq1.2x}) into

%Then the problem (\ref{eq1.2x}) can be transformed into^- q v(x,) +{f}{\zeta}\phi^{\ast}(\tau),\varphi^{\ast}(\tau)
\begin{equation}
\begin{cases}
{}^{C}_0\mathbb{D}^{\alpha,\lambda}_{\tau} v(x,\tau) = \frac{1}{2}\sigma^2\frac{\partial^2 v(x,\tau)
}{\partial x^2} - qv(x,\tau) + g(x,\tau), & (x,\tau)\in\Omega\times(0,T],\\
v(x,0) = \sigma(x), & x\in\Omega,\\
v(x_l,\tau) = v(x_r,\tau) = 0, & (x,\tau)\in\partial\Omega\times(0,T],
\end{cases}
\label{eq2.1}
\end{equation}
where $q = \frac{c^2}{2\sigma^2} + r > 0$, $\sigma(x) = k(x)\cdot\tilde{\zeta}(x)$ and $g(x,\tau) = k(x)\cdot\tilde{f}(x,\tau)$.
%\begin{equation}
%%{lll}
%,\quad \phi^{\ast}(\tau) = \phi(\tau),
%\varphi^{\ast}(\tau) =  k(x_r)\varphi(\tau),
%\label{eq2.2}
%\end{equation}
%and
%\begin{equation}
%
%\end{equation}
It is clear that $U(x, \tau)$ is a solution of (\ref{eq1.2x}) if and only if
$v(x, \tau)$ is a solution of (\ref{eq2.1}); refer to \cite{Wang2015,Tian20,Kerui21} for details.
Therefore, in the following, our proposed finite difference methods for the problem (\ref{eq1.1}) are both
based on the above equivalent form (\ref{eq2.1}).
%\begin{equation}
%g(x, = { x} - rv(x,


%%%%%%%%%%%%%%%%%%%%%%%%%%%%%%%
\subsection{Discretization in time on non-uniform steps}

For nonuniform time levels $0 = \tau_0 < \tau_1 < \tau_2 < \cdots
< \tau_M = T$, we denote the $j$-th step size by $\Delta \tau_n:= \tau_n - \tau_{n-1}
$, $x_i = x_l + ih,i = 0, 1, 2,\cdots, N$, where $h = (x_r - x_l)/N$ are space
grid size and $M,N\in\mathbb{N}^{+}$ respectively. Since the grid function $\{v_i|0\leq
i\leq N\}$, then we define the difference operators as follows:
\begin{equation*}
\delta_xv_{i-1/2} = \frac{v_i - v_{i-1}}{h},~~\delta^{2}_xv_i = \frac{v_{i+1} - 2v_i +
v_{i-1}}{h^2},~~\mathcal{H}_xv_i =
\begin{cases}
\left(1 + \frac{h^2}{12}\delta^{2}_x\right)v_i, &1\leq i\leq N-1,\\
v_i, & i=0,~N,
\end{cases}
\end{equation*}

Let $\Pi_{1,n}u$ denote the linear interpolation of a function $u$ with respect
to the nodes $\tau_{n-1}$ and $\tau_n$, The corresponding interpolation error
is denoted by $(\widetilde{\Pi_{1,n}}v)(\tau):= v(\tau) - (\Pi_{1,n}v)(\tau)$,
then it is easy to find that
\begin{equation*}
(\widetilde{\Pi_{1,n}}v)'(\tau) = \frac{\nabla_{\tau}v^n}{\Delta\tau_n},
\end{equation*}

{\bf The $L1$ formula on graded mesh \cite{Stynes17}:}
\begin{align}
D^{\alpha}_{\tau} f(\tau_n)
=\frac{1}{\Gamma(1-\alpha)}\big[a_n^{(n,\alpha)}f(\tau_n)-\sum_{k=1}^{n-1}(a_{k+1}^{(n,\alpha)}
-a_{k}^{(n,\alpha)})f(\tau_k)-a_{1}^{(n,\alpha)}f(\tau_0)\big],
\label{8}
\end{align}
where
\begin{align}\label{191}
a_k^{(n,\alpha)}=\frac{1}{\Delta\tau_k}\int_{\tau_{k-1}}^{\tau_k}\frac{\mathrm{d}s}{(\tau_n-s)^\alpha},\quad k=1, 2, \cdots, n.
\end{align}

\begin{lemma}\cite{Shen2018}\label{le2}
For any ~$\alpha ~(0<\alpha<1)~$ and ~$\{a_{k}^{(n,\alpha)}~(1\leqslant n \leqslant M)\}~$ defined in ~(\ref{191}),~it holds
 $\mathrm{(I)} $  \be 0< a_1^{(n,\alpha)} <  a_2^{(n,\alpha)} <\cdots  < a_n^{(n,\alpha)}, \quad  n=1, 2, \cdots, M,\label{615}\ee\\
 $\mathrm{(II)}$ \be a_1^{(1,\alpha)}> a_1^{(2,\alpha)}>\cdots > a_1^{(N,\alpha)}\ge T^{-\alpha}.\label{615BB}\ee\\
 $ \mathrm{(III)}$There exists a constant $c_1$ such that \be a_{n}^{(n,\alpha)}-a_{n-1}^{(n,\alpha)}\ge  c_1M^\alpha,  \quad n=2, 3, \cdots, M.\label{615AA}\ee
\end{lemma}


\begin{lemma}\cite{Shen2018}\label{le1}
{\it Suppose $|f''(\tau)|\leqslant c_0\tau^{\alpha-2},~0<\tau\leqslant T.$ Then there exists a constant $C$ such that
$$\left|{}_{0}^{C}\mathbb{D}_{\tau}^{\alpha}f(\tau_n)-\mathcal{D}^{\alpha}_{\tau} f(\tau_n)\right|\le C n^{-\min{\{\gamma(1+\alpha), 2-\alpha\}}},\quad n= 1, 2,\ldots, M.$$}
\end{lemma}

At this stage, we recall $u(\tau_n) = e^{\lambda \tau_n}v(\tau_n)$ and obtain
the nonuniform tempered $L1$ formula for tempered Caputo fractional derivative
at the time point $\tau_n$ by $L1$ formula:
\begin{equation}
\begin{split}
{}^{C}_0\mathbb{D}^{\alpha,\lambda}_{\tau}v(\tau_n) &
= \mathcal{D}^{\alpha}_{\tau}v^{n} + \mathcal{R}^{n} = \frac{e^{-\lambda \tau_n}}{\Gamma(1-\alpha)}\sum^{n}_{k= 1}a_k^{(n,\alpha)}(u^k-u^{k-1})+\mathcal{R}^n\\
& = \frac{1}{\Gamma(1-\alpha)}\sum^{n}_{ k=1}a_k^{(n,\alpha)}\left(e^{\lambda(\tau_{k}-\tau_n)}v^{k} - e^{\lambda(\tau_{k-1}-\tau_n)}v^{k-1}\right) + \mathcal{R}^n\\
& = \frac{1}{\Gamma(1-\alpha)}\big[a^{(n,\alpha)}_nv^n - \sum^{n-1}_{k= 1}
(a^{(n,\alpha)}_{k+1} -a_k^{(n,\alpha)})e^{\lambda(\tau_k-\tau_n)}v^k-a_1^{(n,\alpha)}e^{\lambda(\tau_0-\tau_n)}v^0\big]\\
&~+\mathcal{R}^n
\end{split}
\label{eq2.7}
\end{equation}
where $\mathcal{R}^n: = {}^{C}_0\mathbb{D}^{\alpha,\lambda}_{\tau}v(\tau)
\mid_{\tau = \tau_n} - \mathcal{D}^{\alpha}_{\tau}v^n$ is
the truncation error. In addition, if $\lambda \equiv 0$, the above
formula is just the nonuniform L1 formula for approximating the Caputo fractional
derivative \cite{Shen2018}.

Next, we consider Eq. (\ref{eq2.1}) at the point $(x,\tau) = (x_i,\tau_n)$,
then it follows that
\begin{equation}\label{point-problem}
{}^{C}_0\mathbb{D}^{\alpha,\lambda}_{\tau}v(x_i,\tau_n)
= \frac{1}{2}\sigma^2\frac{\partial^2 v(x_i,\tau_n)}{\partial
x^2} - qv(x_i,\tau_n) + g(x_i,\tau_n),
\end{equation}
namely,

\begin{equation*}
\begin{cases}
\mathcal{H}_x\left(\mathcal{D}^{\alpha}_{\tau}V^{n}_i\right) = \frac{1}{2}\sigma^2\delta^{2}_xV^{n}_i
- q\mathcal{H}_xV^{n}_i + \mathcal{H}_xg^{n}_i + (R^{\alpha}_{\tau,x})^{n}_i,& 1\leq i\leq N-1,~1\leq n\leq M,\\
V^{0}_i = \sigma(x_i),& 1\leq i\leq N-1,\\
V^{n}_0 = V^{n}_N = 0,& 0\leq n\leq M.
\end{cases}
\end{equation*}
Now, we omit the small error items and arrive at the following difference schemes

\begin{equation}
\begin{cases}
\mathcal{H}_x\left(\mathcal{D}^{\alpha}_{\tau}v^{n}_i\right) = \frac{1}{2}\sigma^2\delta^{2}_xv^{n}_i
- q\mathcal{H}_xv^{n}_i + \mathcal{H}_xg^{n}_i,& 1\leq i\leq N-1,~~1\leq n\leq M,\\
v^{0}_i = \sigma(x_i),& 1\leq i\leq N-1,\\
v^{n}_0 = v^{n}_N = 0,& 0\leq n\leq M.
\end{cases}
\label{fourth-order-scheme}
\end{equation}


%%%%%%%%%%%%%%%%%%%%%%%%%%%%%%%%%%%%%%%%%%
\subsection{Stability and convergence of the difference scheme}
\setcounter{equation}{0}
In this section, we will give the stability and convergence analysis of the difference scheme \eqref{fourth-order-scheme}.
\begin{theorem}\label{thm1}
Suppose $\{v_i^n\,|\, 0\leqslant i \leqslant N,~0\leqslant n\leqslant M\}~$ is the solution of the difference scheme \eqref{fourth-order-scheme}.
Then, it holds
\begin{align}\label{65}
\|v^k\|_{\infty}\leqslant \|v^0\|_{\infty}+ \Gamma(1-\alpha)\max \limits_{1\leqslant n \leqslant k} \frac{\|g^n\|_\infty}{a_1^{(n,\alpha)}},\quad k=1, 2, \cdots, M,
\end{align}
where
$$\|g^n\|_\infty=\max_{1\le i\le N-1}|g_i^n|.$$
\end{theorem}

\begin{proof} Let $i_0~(1\le i_0\le N-1)$ be an index  such that $|v_{i_0}^n|=\|v^n\|_{\infty}.$ Rewriting the first equality of \eqref{fourth-order-scheme} in the form
\begin{equation}
\begin{split}
\left[\frac{1}{\Gamma(1-\alpha)}a_n^{(n,\alpha)}+q\right]{H_{x}}v_i^n+\frac{\sigma^2}{h^2}v_i^n
&=\frac{\sigma^2}{2h^2}(v_{i-1}^n+v_{i+1}^n)+\frac{1}{\Gamma(1-\alpha)}
\Big[\sum_{k=1}^{n-1}e^{-\lambda(\tau_n-\tau_k)}\cdot\\
&~~~~(a_{k+1}^{(n,\alpha)}-a_k^{(n,\alpha)}){H_{x}}v_i^k
+e^{-\lambda(\tau_n-\tau_0)}a_1^{(n,\alpha)}{H_{x}}v_i^0\Big]\\
&~~~+{H_{x}}g_i^n,\quad 1 \leqslant i
\leqslant N-1,\quad 1\leqslant n \leqslant M,
\end{split}
%\lambda}_j
\label{41}
\end{equation}
we set $i=i_0$ in (\ref{41}) and take the absolute value on the both sides of the equation obtained so that
\begin{equation*}
\begin{split}
\Big[\frac{1}{\Gamma(1-\alpha)}a_n^{(n,\alpha)}+q\Big]\|v^n\|_{\infty}+
\frac{\sigma^2}{h^2}\|v^n\|_{\infty}&\leqslant
\frac{\sigma^2}{h^2}\|v^n\|_{\infty}+\frac{1}{\Gamma(1-\alpha)}\Big[\sum_{k=1}^{n-1}(a_{k+1}^{(n,\alpha)}-a_k^{(n,\alpha)})\|v^k\|_{\infty}
\\
&~~~+a_1^{(n,\alpha)}\|v^0\|_{\infty}\Big]+\|g^n\|_{\infty},\quad
1\leqslant n \leqslant M.
\end{split}
\end{equation*}
This inequality can be rearranged and written as follows,
\begin{equation}
\begin{split}
%                                                                                         g_j \\
%                                                                                         f_j \\
%
%
%&=-\left[
a_n^{(n,\alpha)}\|v^n\|_{\infty}&\leqslant \sum_{k=1}^{n-1}(a_{k+1}^{(n,\alpha)}-a_k^{(n,\alpha)})\|v^k\|_{\infty}
+a_1^{(n,\alpha)}\Big(\|v^0\|_{\infty}\\
&~~~+\frac{\Gamma(1-\alpha)}{a_1^{(n,\alpha)}}\|g^n\|_{\infty}\Big), \quad 1 \leqslant n \leqslant M.
\end{split}
\label{64}
\end{equation}
Next, we apply mathematical induction to prove (\ref{65}) is valid. In fact, setting $n=1$ in (\ref{64}) and noticing  $a_1^{(1,\alpha)}>0,$ we obtain
$$\|v^1\|_\infty \leqslant \|v^0\|_\infty+\frac{\Gamma(1-\alpha)}{a_1^{(1,\alpha)}}\|g^1\|_\infty.$$
Thus (\ref{65}) is valid for $k=1.$ Now, assume that the inequality (\ref{65}) is valid for $1 \leqslant k \leqslant n-1,$ that is,
\begin{align*}
 \|v^k\|_{\infty} \leqslant \|v^0\|_{\infty} +\Gamma(1-\alpha)\max_{1\leqslant n \leqslant k} \frac{\|g^n\|_{\infty}}{a_1^{(n,\alpha)}},\quad k=1,2,\cdots, n-1.
\end{align*}
Then, from (\ref{64}), we have
\begin{align}
a_n^{(n,\alpha)}\|v^n\|_{\infty} &\leqslant \sum_{k=1}^{n-1}(a_{k+1}^{(n,\alpha)}-a_k^{(n,\alpha)})\big(\|v^0\|_{\infty}+\Gamma(1-\alpha)\max_{1 \leqslant n \leqslant
k}\frac{\|g^n\|_\infty}{a_1^{(n,\alpha)}}\big)+a_1^{(n,\alpha)}\big[\|v^0\|_{\infty}\nn\\
&+\Gamma(1-\alpha)\frac{\|g^n\|_{\infty}}{a_1^{(n,\alpha)}}\big] \nn\\
&\leqslant \big[\sum_{k=1}^{n-1}(a_{k+1}^{(n,\alpha)}-a_k^{(n,\alpha)})+a_1^{(n,\alpha)}\big]\big(\|v^0\|_\infty+\Gamma(1-\alpha)\max_{1 \leqslant n \leqslant
n}\frac{\|g^n\|_\infty}{a_1^{(n,\alpha)}}\big)\nn\\
& \leqslant a_n^{(n,\alpha)}(\|v^0\|_\infty+\Gamma(1-\alpha)\max_{1 \leqslant n\leqslant n}\frac{\|g^n\|_\infty}{a_1^{(n,\alpha)}}).
\end{align}
Noticing $a_n^{(n,\alpha)}\neq 0,$ we obtain
\begin{align}
\|v^n\|_\infty \leqslant \|v^0\|_\infty +\Gamma(1-\alpha)\max_{1\leqslant n \leqslant n}\frac{\|g^n\|_\infty}{a_1^{(n,\alpha)}}.
\end{align}
Therefore (\ref{65}) is valid for $k=n$. This completes the proof.\end{proof}

\begin{theorem}\label{thm4} Suppose $\{U_i^n\,|\, 0\le i\le N,~0\le n\le M\}$ is the solution of the problem of \eqref{eq2.1} and $\{u_i^n\,|\,0\le i\le N,~0\le n\le M\}$ is the solution of  the difference
scheme \eqref{fourth-order-scheme}. Let
 $$e_i^n=U_i^n-u_i^n,\quad 0\leqslant i \leqslant N,~0\leqslant n \leqslant M.$$
then  \be \|e^n\|_\infty \leqslant C\left( M^{-\min\{\gamma\alpha, 2-\alpha\}}+h^4\right),\quad 1\le n\le M.\nn \ee
\end{theorem}

\begin{proof}
  The proof of this theorem is similar to that of \cite[Theorem 4.2]{Shen2018}.
\end{proof}

\section{Fast implementation of the compact difference scheme}
\label{sec3}
%\setcounter{equation}{0}
For the scheme (\ref{fourth-order-scheme}), at each time level, it needs to
solve the tridiagonal system of linear algebraic equations which can be solved
by double-sweep method with computational cost $\mathcal{O}(NM + NM^2)$, so
it is meaningful to reduce the computational cost; refer to \cite{Yang16,Luchko09}.
However, there are few fast numerical method for the tempered fractional differential
equations \cite{Guo2019}. Fortunately, we find that the fast SOE approximation of
the Caputo fractional derivative can be extended to approximate the tempered Caputo
derivative. In this section, we first introduce the fast SOE approximation of
the tempered Caputo fractional derivative, then we can reconstruct a fast stable difference
method with the help of the scheme (\ref{fourth-order-scheme}) and the proposed
fast SOE approximation.

\subsection{The construction of fast compact difference scheme}

\begin{lemma} \rm{(\cite{Jiang17})}\label{le3}
For the given $\alpha \in (0,1)$  and tolerance error $\epsilon,$ cut-off time  restriction ~$\delta$~ and final time $T$, there are one positive integer~$M_{exp},$
positive points $\{s_j\,|\,j=1,2, \cdots, M_{exp}\}$ and corresponding positive weights $\{ w_j\,|\, j=1,2, \cdots, M_{exp}\}$ such that
\begin{align}\label{50}
\left|\tau^{-\alpha}-\sum_{j=1}^{M_{exp}}w_j e^{-s_j \tau}\right|\leqslant \epsilon,\quad \forall \tau\in[\delta, T],
\end{align}
where
\begin{align}\label{51}
M_{exp}=\mathcal{O}\left(\big(\mathrm{log}\frac{1}{\epsilon}\big)\big(\mathrm{loglog}\frac{1}{\epsilon}+\mathrm{log}\frac{T}{\delta}\big)+\big(\mathrm{log}\frac{1}{\delta}\big)
\big(\mathrm{loglog}\frac{1}{\epsilon}+\mathrm{log}\frac{1}{\delta}\big)\right).
\end{align}
\end{lemma}

Now we will derive  a fast algorithm for computing the Caputo fractional derivative on the graded mesh. Let $\delta=(\frac{1}{N})^r T.$ Using the linear polynomial
interpolation, we have
\begin{equation}
\begin{split}
{}^{C}_0D^{\alpha,\lambda}_{\tau} f(\tau_n)&=\frac{e^{-\lambda \tau_n}}{\Gamma(1-\alpha)}\Big[\int_0^{\tau_{n-1}}\frac{1}{(\tau_n-s)^\alpha}
\frac{\partial(e^{\lambda s f(s)})}{\partial s}\mathrm{d}s+\int_{\tau_{n-1}}^{\tau_n}\frac{1}{(\tau_n-s)^\alpha}\frac{\partial(e^{\lambda s f(s)})}{\partial s}\mathrm{d}s\Big]\\
& \approx \frac{e^{-\lambda \tau_n}}{\Gamma(1-\alpha)}\Big[\int_0^{\tau_{n-1}}\frac{\partial(e^{\lambda s f(s)})}{\partial s}\sum_{j=1}^{M_{exp}}w_j e^{-s_j(\tau_n-s)}ds\\
&\quad~+\int_{\tau_{n-1}}^{\tau_n}\frac{e^{\lambda \tau_n}f(\tau_n)-e^{\lambda \tau_{n-1}}f(\tau_{n-1})}{\Delta\tau_n}\cdot\frac{1}{(\tau_n-s)^\alpha}\mathrm{d}s\Big]\\
& = \frac{e^{-\lambda \tau_n}}{\Gamma(1-\alpha)}\Big[\sum_{j=1}^{M_{exp}}w_jF_j^n+a_n^{(n,\alpha)}\Big(e^{\lambda \tau_n}f(\tau_n)-e^{\lambda \tau_{n-1}}f(\tau_{n-1})\Big)\Big]\\
& = {}^F D^{\alpha}_{\tau} f(\tau_n),
\label{197}
\end{split}
\end{equation}
where $$F_j^n=\int_0^{\tau_{n-1}}\frac{\partial(e^{\lambda s}f(s))}{\partial s}e^{-s_j(\tau_n-s)}\mathrm{d}s.$$
The integral $F_j^n$ can be evaluated by using a recursive algorithm, one has
\begin{equation}
\begin{split}
F_j^n& = \int_0^{\tau_{n-2}}\frac{\partial(e^{\lambda s}f(s))}{\partial s}e^{-s_j(\tau_n-s)}\mathrm{d}s+\int_{\tau_{n-2}}^{\tau_{n-1}}\frac{\partial(e^{\lambda s}f(s))}{\partial s}e^{-s_j(\tau_n-s)}\mathrm{d}s \\
&\approx e^{-s_j\tau_n}\int_0^{\tau_{n-2}}\frac{\partial(e^{\lambda s}f(s))}{\partial s}e^{-s_j(\tau_{n-1}-s)}\mathrm{d}s\\
&\quad~+\frac{\left[e^{\lambda \tau_{n-1}}f(\tau_{n-1})-e^{\lambda \tau_{n-2}}f(\tau_{n-2})\right]}{\Delta\tau_{n-1}}
\int_{\tau_{n-2}}^{\tau_{n-1}}e^{-s_j(\tau_n-s)}\mathrm{d}s \nn\\
=&e^{-s_j\tau_n}F_j^{n-1}+B_j^n\big[e^{\lambda \tau_{n-1}}f(\tau_{n-1})-e^{\lambda \tau_{n-2}}f(\tau_{n-2})\big],\quad n=2,3,\cdots,
\label{503}
\end{split}
\end{equation}
where
\begin{align}
F_j^1=0,\quad B_j^n=\frac{1}{\Delta\tau_{n-1}}\int_{\tau_{n-2}}^{\tau_{n-1}}e^{-s_j(\tau_n-s)}\mathrm{d}s,\quad 1\leqslant j \leqslant M_{\rm{exp}}.
\label{504}
\end{align}
According to Eqs. (\ref{197})-(\ref{504}), we obtain a two-step recursive  formula approximating the Caputo derivative as  follows
\begin{align}
&{}^F D_t^\alpha f(\tau_n)=\frac{e^{-\lambda \tau_n}}{\Gamma(1-\alpha)}\Big\{\sum_{j=1}^{M_{\rm{exp}}}w_jF_j^n+a_n^{(n,\alpha)}\big[e^{\lambda \tau_n}f(\tau_n)-e^{\lambda \tau_{n-1}}f(\tau_{n-1})\big]\Big\},\quad n\geqslant 1,\label{617}\\
&F_j^n=e^{-s_j\tau_n}F_j^{n-1}+B_j^n\left[e^{\lambda \tau_{n-1}}f(\tau_{n-1})-e^{\lambda \tau_{n-2}}f(\tau_{n-2})\right],\quad n\geqslant 2,\label{618}\\
&F_j^1=0.\label{619}
\end{align}

It is not difficult to know that
\begin{equation}
\begin{split}
{}^F D^{\alpha}_{\tau} f(\tau_n) & = \frac{e^{-\lambda \tau_n}}{\Gamma(1-\alpha)}\Big[\sum_{k=1}^{n-1}\int_{\tau_{k-1}}^{\tau_k}\frac{e^{\lambda \tau_k}f(\tau_k)-e^{\lambda \tau_{k-1}}f(\tau_{k-1})}{\Delta\tau_k}\cdot\sum_{j=1}^{M_{exp}}w_j e^{-s_j
(\tau_n-s)}ds\\
& +\int_{\tau_{n-1}}^{\tau_n}\frac{e^{\lambda \tau_n}f(\tau_n)-e^{\lambda \tau_{n-1}}f(\tau_{n-1})}{\Delta\tau_n}\cdot\frac{1}{(\tau_n-s)^\alpha}\mathrm{d}s\Big]\\
& = \frac{1}{\Gamma(1-\alpha)}\Big\{\sum_{k=1}^{n-1}b_k^{(n,\alpha)}
\big[e^{-\lambda(\tau_n-\tau_k)}f(\tau_k)-e^{\lambda(\tau_n-\tau_{k-1})}f(\tau_{k-1})\big]\\
&\quad~ + a_n^{(n,\alpha)}\big[f(\tau_n)-e^{-\lambda(\tau_n-\tau_{n-1})}f(\tau_{n-1})\big]\Big\}\\
&=\frac{1}{\Gamma(1-\alpha)}\big[b_n^{(n,\alpha)}f(\tau_n)-\sum_{k=1}^{n-1}(b_{k+1}^{(n,\alpha)}
-b_{k}^{(n,\alpha)})e^{\lambda(\tau_k-\tau_n)}f(\tau_k)\\
&\quad~ -b_{1}^{(n,\alpha)}e^{\lambda(\tau_0-\tau_n)}f(\tau_0)\big],\quad 1\leqslant n \leqslant M,\label{63}
\end{split}
\end{equation}
where
\begin{equation}
b_k^{(n,\alpha)}=
\begin{cases}
\sum_{j=1}^{M_{exp}}w_j \frac{1}{\Delta\tau_{k}}\int_{\tau_{k-1}}^{\tau_k}e^{-s_j(\tau_n-s)}\mathrm{d}s,& k=1, 2, \cdots, n-1,\\
a_n^{(n,\alpha)},& k=n.
\end{cases}\label{53}
\end{equation}

\begin{lemma}\label{le4}
For any $\alpha \in (0,1),$ the sequence  $\{b_k^{(n,\alpha)}, n=1,2,\cdots,M\}$ defined in (\ref{53}) satisfies
\be0< b_1^{(n,\alpha)} < b_k^{(n,\alpha)} < \cdots  < b_{n-1}^{(n,\alpha)}.\nn\\ \ee
If $\epsilon \leqslant c_1M^\alpha,$ then
\be  b_{n-1}^{(n,\alpha)}\le b_{n}^{(n,\alpha)}.\nn\\ \ee
\end{lemma}

\begin{lemma}\label{le5}
For any ~$\alpha \in (0,1)$ and ~$|f'(\tau)|\leqslant c_0 \tau^{\alpha-1}, |f''(\tau)|\leqslant c_0 \tau^{\alpha-2},$  we have
$${}_0^CD^{\alpha}_{\tau} f(\tau_n)={}^F D^{\alpha}_{\tau} f(\tau_n)+\mathcal{O}\left(n^{-\min\{\gamma(1+\alpha),2-\alpha\}}+\epsilon \right).\quad n=1,2,\cdots,M.$$
\end{lemma}

Applying \eqref{617}-\eqref{619} to the \eqref{point-problem} in temporal direction and fourth-order in spatial direction, we have
\begin{equation*}%\label{fast-discrete}
\begin{cases}
\mathcal{H}_x\left(\mathcal{D}^{\alpha}_{\tau}V^{n}_i\right) = \frac{1}{2}\sigma^2\delta^{2}_xV^{n}_i
- q\mathcal{H}_xV^{n}_i + \mathcal{H}_xg^{n}_i + (R^{\alpha}_{\tau,x})^{n}_i,\quad 1\leq i\leq N-1,~~1\leq n\leq M,\\
F_{j,i}^n=e^{-s_j \tau_n}F_{j,i}^{n-1}+B_j^n(V_i^{n-1}-V_i^{n-2}),\quad 1\leq j \leq M_{\rm{exp}},~ 1 \leq i \leq N-1,~2\leq n \leq M,\\
F_{j,i}^1=0,\quad j=1,2,\cdots,M_{\rm{exp}},\quad 1\leq i \leq N-1,\\
V^{0}_i = \sigma(x_i),\quad 1\leq i\leq N-1,\\
V^{n}_0 = V^{n}_N = 0,\quad 0\leq n \leq M.
\end{cases}
\end{equation*}
Now, we omit the small error items and arrive at the following fast difference scheme
\begin{equation}
\begin{cases}
\mathcal{H}_x\left(\mathcal{D}^{\alpha}_{\tau}v^{n}_i\right) = \frac{1}{2}\sigma^2\delta^{2}_xv^{n}_i
- q\mathcal{H}_xv^{n}_i + \mathcal{H}_xg^{n}_i,\; 1\leq i\leq N-1,~~1\leq n\leq M,\\
f_{j,i}^n = e^{-s_j \tau_n}f_{j,i}^{n-1}+\!B_j^n(v_i^{n-1}-v_i^{n-2}),~1\leq\! j \leq M_{exp},~1\leq\! i\leq\! N-1,~2\leq\! n \leq M,\\
f_{j,i}^1=0,\quad j=1,2,\cdots,M_{exp},\quad 1\leq i \leq N-1,\\
v^{0}_i = \sigma(x_i),\quad 1\leq i\leq N-1,\\
v^{n}_0 = v^{n}_N = 0,\quad 0\leq n \leq M.
\end{cases}
\label{fast-scheme}
\end{equation}
Similarly, at each time level, the scheme (\ref{fast-scheme}) needs to
solve the tridiagonal system of linear algebraic equations which can be solved
by double-sweep method with computational cost $\mathcal{O}(NM + NMM_{exp})$.
Note that, generally, $M_{exp} < 120$ \cite{Shen2018} and $M$ is large, so that
the scheme (\ref{fast-scheme}) requires lower computational cost than the scheme
(\ref{fourth-order-scheme}).

\subsection{Convergence of the fast compact difference scheme}
\begin{theorem} Suppose $\{U_i^n\,|\,0\le i\le N,~0\le n\le M\}$ is the solution of the problem of \eqref{eq2.1} and $\{u_i^n\,|\,0\le i\le N,~0\le n\le M\}$ is the solution of  the difference
scheme \eqref{fast-scheme}. Let
 $$e_i^n=U_i^n-u_i^n,\quad 0\leqslant i \leqslant N,~0\leqslant n \leqslant M.$$
If $\epsilon \le \min\{c_1M^\alpha, \frac 12T^{-\alpha}\},$ then  \be \|e^n\|_\infty \leqslant C\left(M^{-\min\{\gamma\alpha, 2-\alpha\}}+h^4+\epsilon\right),\quad 1\le n\le M.\nn \ee
\end{theorem}
\begin{proof}
  The proof of this theorem is similar with that of Theorem \ref{thm4}.
\end{proof}


\section{Numerical experiments}
\label{sec4}
%\setcounter{equation}{0}
In this section, the first two examples exhibiting an exact solution are presented
to demonstrate the accuracy of the solution and the order of convergence
of our proposed numerical scheme given in Sections \ref{sec2}--\ref{sec3}. Furthermore, we use
the proposed schemes to price several different European options governed by a
tempered TFBS model, which is one of the most interesting models in the financial market.
All experiments were performed on a Windows 7 (64 bit) PC-11th Gen Inter(R) i7-11700K @3.60GHz,
32 GB of RAM using MATLAB R2021a.

\textbf{Example 1}. Consider the following tempered TFBS model with homogeneous
boundary conditions
\begin{equation}
\begin{cases}
{}^{C}_0D^{\alpha,\lambda}_{\tau}U(x,\tau) = \frac{\sigma^2}{2}
\frac{\partial^2 U(x,\tau)}{\partial x^2} + c \frac{\partial U(x,\tau)
}{\partial x} -rU(x,\tau) + f(x,\tau), &(x,\tau)\in(0,1)\times(0,1],\\
U(x,0) = 5\sin(\pi x), & x\in[0,1],\\
U(0,\tau) = U(1,\tau) = 0, &\tau\in(0,1],
\end{cases}
\label{ex1eqn}
\end{equation}
where the source term
\begin{equation*}
\begin{split}
f(x,\tau) &= 5e^{-\lambda\tau}\Big[\Gamma(1 + \alpha)\sin(\pi x)
+ \frac{\sigma^2}{2}\pi^2\sin(\pi x)(\tau^{\alpha} + 1) - c\pi\cos(\pi
x)(\tau^{\alpha} + 1)\\
&\quad + r\sin(\pi x)(\tau^{\alpha} + 1)\Big]
\end{split}
\end{equation*}
is chosen so that the exact solution of model (\ref{ex1eqn}) is $U(x,\tau) = 5e^{-\lambda \tau}(\tau^{\alpha} + 1)\sin(\pi x)$. Here the parameter values are $r = 0.05, D = 0$ and $\sigma = 0.25$.

At present, Tables \ref{tab1}--\ref{tab4} show the numerical results of fast and direct schemes
for different $\alpha$ with the grading parameter $\gamma$. It is seen that the errors and the
rates of the two methods are almost the same except for the last results under $(\alpha,\gamma)
= (0.3,4)$ in Table \ref{tab1}--\ref{tab2}; hence, the SOE approximation does not lose accuracy
with fitted tolerance $\epsilon$, whereas the direct difference scheme is numerically sensitive
for big mesh grading index $\gamma$ where the coefficients $a^{(n,\alpha)}_k$ of the graded L1
formula (\ref{8}) should be accurately computed; see the (try-before-you-buy) package {\tt Advanpix}\footnote{Multiprecision Computing Toolbox for MATLAB, which is available at \url{https://www.advanpix.com/}.} for a further discussion. Moreover, it is clear that the proposed numerical methods are convergent with $\min\{\gamma\alpha,2-\alpha\}$-order and fourth-order accuracy in time and space, respectively,
which agree well with the theoretical statements. In addition, as seen from the CPU times of fast
and direct schemes when solving Example 1 in these tables. Obviously, the fast difference scheme saves many computational
costs compared with the direct difference scheme.
%\label{KNY2}
%K\widetilde{Y}_k=\widetilde{X}_{k+1}\widetilde{B}_{k,k+1}^T.
%\end{equation}
%                                                                                         f_j \\
%%%%%%%%%%%%%%%%%%%%%%%%%%%%%%%%%%%%%%%%%%%%%%%%%%%%%%% {Table 1} %%%%%%%%%%%%%%%%%%%%%%%%%%%%%%%%%%%%%%%%%%%%%%%%%%%%%%%%%%

\begin{table}[!htpb]
	\begin{center}
		\caption{Spatial convergence order of the direct difference method for Example 1 with $M = \lceil N^{4/\min\{\gamma\alpha,2-\alpha\}}\rceil$ and $\lambda = 1$.}
		\centering
		\begin{tabular}{ccccccc}
			\hline
			$(\alpha,\gamma)$ &$N$ &$e_2(M,N)$ &${\rm Order}_2$ &$e_{\infty}(M,N)$ &${Order}_{\infty}$ &$\mathrm{Time}$ \\
			\hline
			(0.3, 4)          &4   &2.8586e-3  & --            &4.0427e-3         & --            &0.006    \\
			                  &8   &1.8067e-4  & 3.9839        &2.5550e-4         & 3.9839        &0.418    \\
			                  &16  &1.1309e-5  & 3.9978        &1.5994e-5         & 3.9977        &33.752   \\
			                  &32  &1.3414e-5  & -0.2463       &1.8969e-5         & -0.2461       &3358.442 \\
%%%%%%%%%%%%%%%%%%%%%%%%%%%%%%%%%%%%%%%%%%
			\hline
			(0.5, 3)          &6   &1.3187e-3  & --             &1.8649e-3        & --            & 0.007    \\
			                  &12  &8.7238e-5  & 3.9180        &1.2337e-4         & 3.9180        & 0.199    \\
			                  &24  &5.5967e-6  & 3.9623        &7.9150e-6          & 3.9623       & 7.105    \\
			                  &48  &3.5461e-7  & 3.9803        &5.0149e-7          & 3.9803       & 291.118  \\
%%%%%%%%%%%%%%%%%%%%%%%%%%%%%%%%%%%%%%%%%&
			\hline	
              (0.8, 2)        &4   &2.1517e-3  & --            & 3.0372e-3        & --            & 0.006   \\
			                  &8   &1.4188e-4  & 3.9227        & 2.0027e-4        & 3.9206        & 0.417   \\
			                  &16  &8.9448e-6  & 3.9875        & 1.2626e-5        & 3.9875        & 33. 627 \\
			                  &32  &5.5974e-7  & 3.9982        & 7.9200e-7        & 3.9948        &3352.674\\
			\hline
		\end{tabular}
		\label{tab1}
	\end{center}
\end{table}
%%%%%%%%%%%%%%%%%%%%%%%%%%%%%%%%%%%%%%%%%%% {Table 2} %%%%%%%%%%%%%%%%%%%%%%%%%%%%%%%%%%%%%%
\begin{table}[t]
	\begin{center}
		\caption{Spatial convergence order of the fast difference method for Example 1 with $M = \lceil N^{4/\min\{\gamma\alpha,2-\alpha\}}\rceil$ and $\lambda = 1$.}
		\centering
		\begin{tabular}{ccccccc}
			\hline
			 $(\alpha,\gamma)$ & $N$  & $e_2(M,N)$ & ${\rm Order}_2$ & $e_{\infty}(M,N)$ & ${Order}_{\infty}$ & $\mathrm{Time}$ \\
			\hline
			(0.3, 4) &4  & 2.8586e-3  & --         & 4.0427e-3        & --                & 0.029\\
			         &8  & 1.8067e-4  & 3.9839     & 2.5550e-4      & 3.9839         & 0.362  \\
			         &16 & 1.1309e-5  & 3.9978     & 1.5994e-5      & 3.9977        & 4.543 \\
			         &32 & 7.0690e-7  & 3.9998     & 9.9971e-7      & 3.9999         & 56.177\\
			%64 & 256 & OoM & $>5$ hours & (91.0, 92.0) & 1913.149 & (9.0, 5.0) & 358.067 \\
			\hline
			(0.5, 3) &6  & 1.3187e-3  & --            & 1.8649e-3         & --              & 0.028 \\
			         &12 & 8.7238e-5  & 3.9180      & 1.2337e-4     & 3.9180        & 0.207  \\
			         &24 & 5.5967e-6  & 3.9623     & 7.9150e-6      & 3.9623        & 1.581 \\
			         &48 & 3.5461e-7  & 3.9803     & 5.0149e-7      & 3.9803        & 12.115  \\
%			& 256 & OoM & $>5$ hours & (278.0, 544.0) & $\approx 3.1$ hours & (7.0, 5.0) & 354.081 \\
			\hline	
                     (0.8, 2) &4  & 2.1517e-3  & --            & 3.0372e-3         & --        & 0.019 \\
			         &8  & 1.4188e-4  & 3.9227     & 2.0027e-4      & 3.9206     & 0.214\\
			         &16 & 8.9448e-6  & 3.9875     & 1.2626e-5      & 3.9875    & 2.572\\
			         &32 & 5.5974e-7  & 3.9982     & 7.9200e-7      & 3.9948    & 31.116 \\
%			& 256 & OoM & $>5$ hours & (154.0, 333.0) & 6901.452 & (8.0, 5.0) & 359.428 \\
			\hline
		\end{tabular}
		\label{tab2}
	\end{center}
\end{table}

%%%%%%%%%%%%%%%%%%%%%%%%%%%%%%%%%%%%%%%%%%%%%%%%% {Table 3} %%%%%%%%%%%%%%%%%%%%%%%%%%%%%%%%%%%%%%%%%%%

\begin{table}[!htpb]
	\begin{center}
		\caption{Temporal convergence order of the direct difference scheme for Example 1 with $N = \lceil M^{\min\{\gamma\alpha,2-\alpha\}/4}\rceil$ and $\lambda = 1$.}
		\centering
		\begin{tabular}{ccccccc}
			\hline
			%& & \multicolumn{5}{c}{$r = 2$} & \multicolumn{5}{c}{$r = 3$} \\
			%[-2pt] \cmidrule(lr){3-7} \cmidrule(lr){8-12} \\ [-11pt]
			$(\alpha,\gamma)$ & $M$ & $e_2(M,N)$ & ${\rm Order}_2$ & $e_{\infty}(M,N)$ & ${Order}_{\infty}$ & $\mathrm{Time}$ \\
			\hline
			(0.3, 4) &800  & 2.4322e-4  & --      & 3.4397e-4    & --         & 0.265  \\
			         &1600 & 1.0590e-4  & 1.1996  & 1.4977e-4  & 1.1995 & 0.950  \\
			         &3200 & 4.6103e-5  & 1.1998  & 6.5200e-5  & 1.1998 & 3.485  \\
			         &6400 & 2.0069e-5  & 1.1999  & 2.8382e-5  & 1.9999 & 13.250\\
			%64 & 256 & OoM & $>5$ hours & (91.0, 92.0) & 1913.149 & (9.0, 5.0) & 358.067 \\
			\hline
			(0.5, 3) &640  & 1.1153e-4  & --      & 1.5773e-4  & --          & 0.138 \\
			         &1280 & 3.9899e-5  & 1.4831  & 5.6136e-5  & 1.4905 & 0.542 \\
			         &2560 & 1.4239e-5  & 1.4865  & 2.0073e-5  & 1.4837 & 2.076\\
			         &5120 & 5.0732e-6  & 1.4889  & 7.1614e-6  & 1.4869 & 8.112  \\
%			& 256 & OoM & $>5$ hours & (278.0, 544.0) & $\approx 3.1$ hours & (7.0, 5.0) & 354.081 \\
			\hline	
                                              (0.8, 2) &640  & 2.4942e-4  & --          & 3.4921e-4      & --    & 0.173 \\
			         &1280 & 1.1092e-4  & 1.1690  & 1.5606e-4  & 1.1620 & 0.626 \\
			         &2560 & 4.8307e-5  & 1.1993  & 6.8165e-5  & 1.1950 & 2.277  \\
			         &5120 & 2.0748e-5  & 1.2193  & 2.9334e-5  & 1.2164 & 8.578  \\
%			& 256 & OoM & $>5$ hours & (154.0, 333.0) & 6901.452 & (8.0, 5.0) & 359.428 \\
			\hline
		\end{tabular}
		\label{tab3}
	\end{center}
\end{table}

%%%%%%%%%%%%%%%%%%%%%%%%%%%%%% {Table 4} %%%%%%%%%%%%%%%%%%%%%%%%%%%%%%%%%%%%

\begin{table}[t]
	\begin{center}
		\caption{Temporal convergence order of the fast difference scheme for Example 1 with $N = \lceil M^{\min\{\gamma\alpha,2-\alpha\}/4}\rceil$ and $\lambda = 1$.}
		\centering
		\begin{tabular}{ccccccc}
			\hline
			%& & \multicolumn{5}{c}{$r = 2$} & \multicolumn{5}{c}{$r = 3$} \\
			%[-2pt] \cmidrule(lr){3-7} \cmidrule(lr){8-12} \\ [-11pt]
			$(\alpha,\gamma)$ & $M$       & $e_2(M,N)$ & ${\rm Order}_2$ & $e_{\infty}(M,N)$ & ${Order}_{\infty}$ & $\mathrm{Time}$ \\
			\hline
			(0.3, 4) &800  & 2.4322e-4  & --        & 3.4397e-4    & --      & 0.277 \\
			         &1600 & 1.0590e-4  & 1.1996  & 1.4977e-4  & 1.1995 & 0.587  \\
			         &3200 & 4.6103e-5  & 1.1998  & 6.5200e-5  & 1.1998 & 1.251 \\
			         &6400 & 2.0069e-5  & 1.1999  & 2.8382e-5  & 1.9999 & 2.680 \\
			%64 & 256 & OoM & $>5$ hours & (91.0, 92.0) & 1913.149 & (9.0, 5.0) & 358.067 \\
			\hline
			(0.5, 3) &640  & 1.1153e-4  & --      & 1.5773e-4  & --          & 0.173 \\
			         &1280 & 3.9899e-5  & 1.4831  & 5.6136e-5  & 1.4905 & 0.363 \\
			         &2560 & 1.4239e-5  & 1.4865  & 2.0073e-5  & 1.4837 & 0.794 \\
			         &5120 & 5.0732e-6  & 1.4889  & 7.1614e-6  & 1.4869 & 1.693  \\
%			& 256 & OoM & $>5$ hours & (278.0, 544.0) & $\approx 3.1$ hours & (7.0, 5.0) & 354.081 \\
			\hline	
                                              (0.8, 2) &640  & 2.4942e-4  & --      & 3.4921e-4       & --       & 0.127 \\
			         &1280 & 1.1092e-4  & 1.1690  & 1.5606e-4  & 1.1620 & 0.278\\
			         &2560 & 4.8307e-5  & 1.1993  & 6.8165e-5  & 1.1950 & 0.564  \\
			         &5120 & 2.0748e-5  & 1.2193  & 2.9334e-5  & 1.2164 & 1.198 \\
%			& 256 & OoM & $>5$ hours & (154.0, 333.0) & 6901.452 & (8.0, 5.0) & 359.428 \\
			\hline
		\end{tabular}
		\label{tab4}
	\end{center}
\end{table}
%
%
%
%    0 & _k^2B_k^{-1}e_ke_k^T \\
%    0 & 0 \
\vspace{1.5mm}
\textbf{Example 2}. The second example reads the following tempered TFBS model
with nonhomogeneous boundary conditions.
\begin{equation}
\begin{cases}
{}^{C}_0D^{\alpha,\lambda}_{\tau}U(x,\tau) = \frac{\sigma^2}{2}
\frac{\partial^2 U(x,\tau)}{\partial x^2} + c \frac{\partial U(x,\tau)
}{\partial x} -rU(x,\tau) + f(x,\tau), &(x,\tau)\in(0,1)\times(0,1],\\
U(x,0) = x^4 + x^3 + x^2 + 1, & x\in[0,1],\\
U(0,\tau) = e^{-\lambda\tau}(\tau^{\alpha} + 1),\quad U(1,\tau) =
4e^{-\lambda\tau}(\tau^{\alpha} + 1), &\tau\in(0,1],
\end{cases}
\label{array4}
\end{equation}
where the source term is defined as
\begin{equation*}
\begin{split}
f(x,\tau) & = e^{-\lambda\tau}\Big[\Gamma(1 + \alpha)(x^4 + x^3 + x^2 + 1)
- \frac{\sigma^2}{2}(12x^2 + 6x + 2)(\tau^{\alpha}+1) - c(4x^3 \\
&\quad~ + 3x^2 + 2x)(\tau^{\alpha} + 1) + r(x^4 + x^3 + x^2 + 1)(\tau^{\alpha} + 1)\Big]
\end{split}
\end{equation*}
such that the exact solution of model (\ref{array4}) reads $U(x,\tau) = e^{
-\lambda\tau}(\tau^{\alpha}+1)(x^4 + x^3 + x^2 + 1)$. Moreover, the parameter values
are $r = 0.03, D = 0.01$ and $\sigma = 0.45$.

To solve Example 2, we rewrite it into the form of (\ref{modifiedxx}) in order to apply
the proposed schemes (\ref{fourth-order-scheme}) and (\ref{fast-scheme}). By choosing
appropriate mesh gradings, the numerical results of Example 2 are listed in Tables
\ref{tab5}--\ref{tab8}, which demonstrate that the proposed methods work very well (except
for the last result of $(\alpha,\gamma) = (0.3,4)$ in Table \ref{tab5}) with
the temporal $\min\{\gamma\alpha,2-\alpha\}$-order and spatial fourth-order convergence
accuracies for the time-fractional model with nonhomogeneous boundary conditions.
Compared to the direct difference scheme, the fast scheme based on the SOE approximation
does not lose accuracy with fitted tolerance $\epsilon$ and it also can save much computational
cost in terms of the elapsed CPU time.

%%%%%%%%%%%%%%%%%%%%%%%%%%%%%%%%%%% {Table 5} %%%%%%%%%%%%%%%%%%%%%%%%%%%%%%%%%%%%%%%%%%%%%%%%%
\begin{table}[!htpb]
	\begin{center}
		\caption{Spatial convergence order of the direct difference method for Example 2 with $M = \lceil N^{4/\min\{\gamma\alpha,2-\alpha\}}\rceil$ and $\lambda = 1$.}
		\centering
		\begin{tabular}{ccccccc}
			\hline
			%& &  & \multicolumn{5}{c}{$r = 3$} \\
			%[-2pt] \cmidrule(lr){3-7} \cmidrule(lr){8-12} \\ [-11pt]
			 $(\alpha,\gamma)$ & $N$ & $e_2(M,N)$ & ${\rm Order}_2$ & $e_{\infty}(M,N)$ & ${Order}_{\infty}$ & $\mathrm{Time}$ \\
			\hline
			(0.3, 4) & 4  & 6.3463e-4  & --     & 8.5897e-4 & --         & 0.007    \\
			         & 8  & 4.0319e-5  & 3.9764 & 5.5574e-5 & 3.9501 & 0.414    \\
			         & 16 & 2.5247e-6  & 3.9973 & 3.4962e-6 & 3.9920 & 33.678   \\
			         & 32 & 2.2462e-6  & 0.1686 & 3.1668e-6 & 0.1428 & 3352.296\\
			%64 & 256 & OoM & $>5$ hours & (91.0, 92.0) & 1913.149 & (9.0, 5.0) & 358.067 \\
			\hline
			(0.5, 3) & 6  & 2.8825e-4  & --     & 3.9132e-4 & --          & 0.007    \\
			         & 12 & 1.9298e-5  & 3.9008 & 2.6851e-5 & 3.8653 & 0.197    \\
			         & 24 & 1.2434e-6  & 3.9561 & 1.7280e-6 & 3.9578 & 7.134    \\
			         & 48 & 7.8960e-8  & 3.9770 & 1.0966e-7 & 3.9780 & 285.720  \\
%			& 256 & OoM & $>5$ hours & (278.0, 544.0) & $\approx 3.1$ hours & (7.0, 5.0) & 354.081 \\
			\hline	
                                              (0.8, 2) & 4  & 4.7492e-4  & --     & 6.5572e-4 & --     & 0.007    \\
			         & 8  & 3.1178e-5  & 3.9291 & 4.3348e-5 & 3.9190 & 0.418    \\
			         & 16 & 1.9637e-6  & 3.9889 & 2.7691e-6 & 3.9685 & 33.538   \\
			         & 32 & 1.2286e-7  & 3.9985 & 1.7325e-7 & 3.9985 & 3362.851 \\
%			& 256 & OoM & $>5$ hours & (154.0, 333.0) & 6901.452 & (8.0, 5.0) & 359.428 \\
%			\\		
%			(0.9, 1.9) & 16 & 15.910 & 0.075 & (18.0, 72.0) & 1.073 & (4.0, 3.0) & 0.173 \\
%			& 32 & $>5$ hours & 1.910 & (22.0, 202.0) & 13.323 & (4.0, 3.0) & 0.740 \\
%			& 64 & $>5$ hours & 112.491 & (22.0, 644.0) & 229.510 & (4.0, 4.0) & 4.962 \\
%			& 128 & $>5$ hours & $\approx 3.7$ hours & \dag & \dag & (3.0, 4.0) & 36.029 \\
%			& 256 & OoM & $>5$ hours & \dag & \dag & (3.0, 4.0) & 323.964 \\
			\hline
		\end{tabular}
		\label{tab5}
	\end{center}
\end{table}

%%%%%%%%%%%%%%%%%%%%%%%%%%%%%%%%%%%%%%%%%%%%%%%%%%%%%%% {Table 6} %%%%%%%%%%%%%%%%%%%%%%%%%%%%%%%%%%%%%%%%%

\begin{table}[t]
	\begin{center}
		\caption{Spatial convergence order of the fast difference method for Example 2 with $M = \lceil N^{4/\min\{\gamma\alpha,2-\alpha\}}\rceil$ and $\lambda = 1$.}
		\centering
		\begin{tabular}{ccccccc}
			\hline
			%& & \multicolumn{5}{c}{$r = 2$} & \multicolumn{5}{c}{$r = 3$} \\
			%[-2pt] \cmidrule(lr){3-7} \cmidrule(lr){8-12} \\ [-11pt]
			 $(\alpha,\gamma)$ & $N$ & $e_2(M,N)$ & ${\rm Order}_2$ & $e_{\infty}(M,N)$ & ${Order}_{\infty}$ & $\mathrm{Time}$ \\
			\hline
			(0.3, 4) & 4  & 6.3463e-4  & --     & 8.5897e-4 & --         & 0.028   \\
			         & 8  & 4.0319e-5  & 3.9764 & 5.5574e-5 & 3.9501 & 0.358   \\
			         & 16 & 2.5247e-6  & 3.9973 & 3.4962e-6 & 3.9920 & 4.516   \\
			         & 32 & 1.5782e-7  & 3.9998 & 2.1896e-7 & 3.9970 & 55.893 \\
			%64 & 256 & OoM & $>5$ hours & (91.0, 92.0) & 1913.149 & (9.0, 5.0) & 358.067 \\
			\hline
			(0.5, 3) & 6  & 2.8825e-4  & --     & 3.9132e-4 & --          & 0.027    \\
			         & 12 & 1.9298e-5  & 3.9008 & 2.6851e-5 & 3.8653 & 0.202    \\
			         & 24 & 1.2434e-6  & 3.9561 & 1.7280e-6 & 3.9578 & 1.557    \\
			         & 48 & 7.8960e-8  & 3.9770 & 1.0966e-7 & 3.9780 & 11.961  \\
%			& 256 & OoM & $>5$ hours & (278.0, 544.0) & $\approx 3.1$ hours & (7.0, 5.0) & 354.081 \\
			\hline	
                                              (0.8, 2) & 4  & 4.7492e-4  & --     & 6.5572e-4 & --           & 0.019    \\
			         & 8  & 3.1178e-5  & 3.9291 & 4.3348e-5 & 3.9190 & 0.209    \\
			         & 16 & 1.9637e-6  & 3.9889 & 2.7691e-6 & 3.9685 & 2.552   \\
			         & 32 & 1.2286e-7  & 3.9985 & 1.7325e-7 & 3.9985 & 30.974 \\
%			& 256 & OoM & $>5$ hours & (154.0, 333.0) & 6901.452 & (8.0, 5.0) & 359.428 \\
%			\\		
%			(0.9, 1.9) & 16 & 15.910 & 0.075 & (18.0, 72.0) & 1.073 & (4.0, 3.0) & 0.173 \\
%			& 32 & $>5$ hours & 1.910 & (22.0, 202.0) & 13.323 & (4.0, 3.0) & 0.740 \\
%			& 64 & $>5$ hours & 112.491 & (22.0, 644.0) & 229.510 & (4.0, 4.0) & 4.962 \\
%			& 128 & $>5$ hours & $\approx 3.7$ hours & \dag & \dag & (3.0, 4.0) & 36.029 \\
%			& 256 & OoM & $>5$ hours & \dag & \dag & (3.0, 4.0) & 323.964 \\
			\hline
		\end{tabular}
		\label{tab6}
	\end{center}
\end{table}

%%%%%%%%%%%%%%%%%%%%%%%%%%%%%%%%%%%%%%%%%%%%%%%%%%%%%%% {Table 7} %%%%%%%%%%%%%%%%%%%%%%%%%%%%%%%%%%%%%%%%%

\begin{table}[!htpb]
	\begin{center}
		\caption{Temporal convergence order of the direct difference method for Example 2 with $N = \lceil M^{\min\{\gamma\alpha,2-\alpha\}/4}\rceil$ and $\lambda = 1$.}
		\centering
		\begin{tabular}{ccccccc}
			\hline
			%& & \multicolumn{5}{c}{$r = 2$} & \multicolumn{5}{c}{$r = 3$} \\
			%[-2pt] \cmidrule(lr){3-7} \cmidrule(lr){8-12} \\ [-11pt]
			 $(\alpha,\gamma)$ & $M$ & $e_2(M,N)$ & ${\rm Order}_2$ & $e_{\infty}(M,N)$ & ${Order}_{\infty}$ & $\mathrm{Time}$ \\
			\hline
			(0.3, 4) & 800  & 5.4272e-5  & --      & 7.4809e-5   & --          & 0.262 \\
			         & 1600 & 2.3638e-5  & 1.1991 & 3.2779e-5   & 1.1904 & 0.943  \\
			         & 3200 & 1.0292e-5  & 1.1996 & 1.4284e-5   & 1.1984 & 3.441  \\
			         & 6400 & 4.4802e-6  & 1.1999 & 6.2129e-6   & 1.2011 & 13.112 \\
			%64 & 256 & OoM & $>5$ hours & (91.0, 92.0) & 1913.149 & (9.0, 5.0) & 358.067 \\
			\hline
			(0.5, 3) & 640  & 2.4634e-5  & --     & 3.4280e-5   & --          & 0.138 \\
			         & 1280 & 8.8373e-6  & 1.4790 & 1.2276e-5   & 1.4815 & 0.554  \\
			         & 2560 & 3.1598e-6  & 1.4838 & 4.3921e-6   & 1.4829 & 2.065 \\
			         & 5120 & 1.1271e-6  & 1.4872 & 1.5645e-6   & 1.4892 & 8.058  \\
%			& 256 & OoM & $>5$ hours & (278.0, 544.0) & $\approx 3.1$ hours & (7.0, 5.0) & 354.081 \\
			\hline	
                                              (0.8, 2) & 640  & 5.4687e-5  & --     & 7.7152e-5   & --            & 0.170\\
			         & 1280 & 2.3981e-5  & 1.1893 & 3.3781e-5   & 1.1915 & 0.623 \\
			         & 2560 & 1.0464e-5  & 1.1965 & 1.4712e-5   & 1.1992 & 2.263 \\
			         & 5120 & 4.5515e-6  & 1.2010 & 6.3887e-6   & 1.2034 & 8.499\\
%			& 256 & OoM & $>5$ hours & (154.0, 333.0) & 6901.452 & (8.0, 5.0) & 359.428 \\
			\hline
		\end{tabular}
		\label{tab7}
	\end{center}
\end{table}

\begin{table}[t]
	\begin{center}
		\caption{Temporal convergence order of the fast difference method for Example 2 with $N = \lceil M^{\min\{\gamma\alpha,2-\alpha\}/4}\rceil$ and $\lambda = 1$.}
		\centering
		\begin{tabular}{ccccccc}
			\hline
			%& & \multicolumn{5}{c}{$r = 2$} & \multicolumn{5}{c}{$r = 3$} \\
			%[-2pt] \cmidrule(lr){3-7} \cmidrule(lr){8-12} \\ [-11pt]
			 $(\alpha,\gamma)$ & $N$ & $e_2(M,N)$ & ${\rm Order}_2$ & $e_{\infty}(M,N)$ & ${Order}_{\infty}$ & $\mathrm{Time}$ \\
			\hline
			(0.3, 4) & 800  & 5.4272e-5 & --      & 7.4809e-5 & --              & 0.276  \\
			         & 1600 & 2.3638e-5 &1.1991   & 3.2779e-5 & 1.1904         & 0.594 \\
			         & 3200 & 1.0292e-5 & 1.1996 & 1.4284e-5 & 1.1984    & 1.260  \\
			         & 6400 & 4.4802e-6 & 1.1999   & 6.2129e-6 & 1.2011      & 2.694 \\
			%64 & 256 & OoM & $>5$ hours & (91.0, 92.0) & 1913.149 & (9.0, 5.0) & 358.067 \\
			\hline
			(0.5, 3) & 640  & 2.4634e-5 & --   & 3.4280e-5 & --      & 0.169  \\
			         & 1280 & 8.8373e-6 & 1.4790   & 1.2276e-5 & 1.4815      & 0.365 \\
			         & 2560 & 3.1598e-6 & 1.4838 & 4.3921e-6 & 1.4829  & 0.782\\
			         & 5120 & 1.1271e-6 &1.4872    & 1.5645e-6 & 1.4892   & 1.659 \\
%			& 256 & OoM & $>5$ hours & (278.0, 544.0) & $\approx 3.1$ hours & (7.0, 5.0) & 354.081 \\
			\hline	
                                              (0.8, 2) & 640  & 5.4687e-5 & -- & 7.7152e-5          & --               & 0.126 \\
			         & 1280 & 2.3981e-5 & 1.1893  & 3.3781e-5 & 1.1915     & 0.267 \\
			         & 2560 & 1.0464e-5 & 1.1965 & 1.4712e-5 & 1.1992 & 0.569  \\
			         & 5120 & 4.5515e-6 & 1.2010& 6.3887e-6 & 1.2034  & 1.194\\
%			& 256 & OoM & $>5$ hours & (154.0, 333.0) & 6901.452 & (8.0, 5.0) & 359.428 \\
			\hline
		\end{tabular}
		\label{tab8}
	\end{center}
\end{table}

\vspace{1.5mm}
\textbf{Example 3}. Consider the following tempered TFBS model
governing the European put options
\begin{equation}
\begin{cases}
%_k^2e_k^Tf_j\left[
%{c}
%B_k^{-1}e_k\\0\\
%, we have
\frac{\partial^{\alpha,\lambda}C(S,t)}{\partial t^{\alpha,\lambda}} + \frac{1}{2}\sigma^2S^2
\frac{\partial^2 C(S,t)}{\partial S^2} + \hat{r}S\frac{\partial C(S,t)}{\partial S}
-rC(S,t) = 0,& (S,t)\in(0,+\infty)\times[0,T),\\
C(S,T) = \max(K-S,0),& S\in[0,+\infty),\\
C(0,t) = \phi(t),\quad \lim\limits_{S\rightarrow +\infty}C(S,t) = 0,& t\in[0,T),
\end{cases}
\end{equation}
where the parameters are set as $r=0.05$, $K=50$, $\sigma = 0.25$, $D=0$, $T=1~{\rm(year)}$ and $\phi(t) = Ke^{-r(T-t)}$.
In this example, we need to compute the Mittage-Lefflier (M-L) function $E_{1,2-\alpha}(z)$ ($z\in\mathbb{R}$) (numerically)\footnote{Evaluation of the M-L function with 2 parameters: \url{https://www.mathworks.com/matlabcentral/fileexchange/48154-the-mittag-leffler-function}.} when
we transform this model to the equation like Eq. (\ref{modifiedxx}).

We now apply the proposed fast numerical scheme (\ref{fast-scheme}) which is better
than the scheme (\ref{fourth-order-scheme}) to solve
Example 3 for different values of $\alpha$ and $\lambda$. The curves of the
put option price are plotted in Figs. \ref{fig1}--\ref{fig3}, respectively,
where the grading parameter $\gamma$ is chosen as the same as Examples 1--2.
As can be seen from them, the order of fractional derivative $\alpha$ and the
tempered index $\lambda$ have an effect on the prices of European options.

\begin{figure}[!htpb]
\centering
\includegraphics[width=2.0in,height=1.95in]{figure1a}
\includegraphics[width=2.0in,height=1.95in]{figure1b}
\includegraphics[width=2.0in,height=1.95in]{figure1c}
\caption{The put option prices at $\alpha = 0.3$ with
different tempered indices $\lambda$ at $S=[e^{-4},e^{4}]$.}
\label{fig1}
\end{figure}
%%%%
\begin{figure}[!htpb]
\centering
\includegraphics[width=2.0in,height=1.95in]{figure2a}
\includegraphics[width=2.0in,height=1.95in]{figure2b}
\includegraphics[width=2.0in,height=1.95in]{figure2c}
\caption{The put option prices at $\alpha = 0.5$ with
different tempered indices $\lambda$ at $S=[e^{-4},e^{4}]$}
\label{fig2}
\end{figure}
%%%%%%%%%%%%%%%%%
\begin{figure}[!htpb]
\centering
\includegraphics[width=2.0in,height=1.95in]{figure3a}
\includegraphics[width=2.0in,height=1.95in]{figure3b}
\includegraphics[width=2.0in,height=1.95in]{figure3c}
\caption{The put option prices at $\alpha = 0.8$ with
different tempered indices $\lambda$ at $S=[e^{-4},e^{4}]$}
\label{fig3}
\end{figure}
%                                                 \end{array}
%                                               \right]-\hat{\lambda}_j\left[
%                                          \begin{array}{cc}
%                                            X_k &  \\
%                                             & Y_k \\
%                                          \end{array}
%                                        \right]\left[
%                                                 \begin{array}{c}
%                                                   g_j \\
%                                                   f_j \\
%                                                 \end{array}
%                                               \right]\\
%%&=\left(\left[
%                                          %\begin{array}{cc}
%                                           % X_k &  \\
%                                            % & Y_k \\
%                                          %\end{array}
%                                        %\right]\left[
%                                        %  \begin{array}{cc}
%                                         %   0 &B_k^T  \\
%                                         %    B_k& 0 \\
%                                          %\end{array}
%                                        %\right]+\beta_k\left[
%                                         %                \begin{array}{c}
%                                          %                 x_{k+1} \\
%                                           %                0 \\
%                                            %             \end{array}
%                                             %          \right]e_{2k}^T\right)\left[
%                                              %   \begin{array}{c}
%                                               %    g_j \\
%                                                %   f_j \\
%                                                 %\end{array}
%                                               %\right]-\hat{\lambda}_j\left[
%                                          %\begin{array}{cc}
%                                           % X_k &  \\
%                                            % & Y_k \\
%                                          %\end{array}
%                                        %\right]\left[
%                                         %        \begin{array}{c}
%                                          %         g_j \\
%                                           %        f_j \\
%                                            %     \end{array}
%                                             %  \right]\\
%&=\left[
%                                          \begin{array}{cc}
%                                            X_k &  \\
%                                             & Y_k \\
%                                          \end{array}
%                                        \right]\left(\left[
%                                          \begin{array}{cc}
%                                            0 &B_k^T  \\
%                                             B_k& 0 \\
%                                          \end{array}
%                                        \right]\left[
%                                                 \begin{array}{c}
%                                                   g_j \\
%                                                   f_j \\
%                                                 \end{array}
%                                               \right]-\hat{\lambda}_j\left[
%                                                 \begin{array}{c}
%                                                   g_j \\
%                                                   f_j \\
%                                                 \end{array}
%                                               \right]\right)+\beta_ke_k^Tf_j\left[
%                                                         \begin{array}{c}
%                                                           x_{k+1} \\
%                                                           0 \\
%                                                         \end{array}
%                                                       \right]\\
%&=\beta_ke_k^Tf_j\left[
%                                          \begin{array}{cc}
%                                            X_{k+1} &  \\
%                                             & Y_k \\
%                                          \end{array}
%                                        \right]\left[
%                                                 \begin{array}{c}
%                                                   -\beta_kB_k^{-1}e_k \\
%                                                   1 \\
%                                                   0 \\
%                                                 \end{array}
%                                               \right].
%\end{split}
%\end{equation*}
%Easily to know $B_k^{-1}e_k=[0,\cdots,0,-\frac{\beta_{k-1}}{\alpha_{k-1}\alpha_k},\frac{1}{\alpha_k}]^T$, so we can use
%\begin{equation}\label{whcon}
%\|\mathbf{H}\hat{z}_j^+-\hat{\lambda}_j\hat{z}_j^+\|_\mathbf{M}=\beta_k|f_{jk}|\sqrt{1+(\frac{\beta_{k-1}}{\alpha_{k-1}\alpha_k})^2+(\frac{1}{\alpha_k})^2}
%\end{equation}
%as the stopping criterion, here $f_{jk}$ is the $k$-th element of $f_j$.
%
%

%
%\subsection{Convergence of whGKL}
%From the above discussion, we know the weighted harmonic Ritz value $\hat{\lambda}$ of $\mathbf{H}$ is the exact eigenvalue of $\mathbf{A}^{-1}\mathbf{C}$. In fact, the exact eigenvalue $\lambda$ of $\mathbf{H}$ is not far from the weighted harmonic Ritz value $\hat{\lambda}$. In the following, we will give a theorem to show that there exist a small disturbance matrix $E$, such that the $\lambda$ is the exact eigenvalue of $\mathbf{A}^{-1}\mathbf{C}+E$, i.e., $\mathbf{A}^{-1}\mathbf{C}$ must has an eigenvalue $\hat{\lambda}$ near $\lambda$.
%
%First, let's give some assumptions and notations. Assuming $z=\left[
%              \begin{array}{c}
%                u \\
%                v \\
%              \end{array}
%            \right]$ is the exact eigenvector of $\mathbf{H}$ corresponding to $\lambda$, and $\|u\|^2_M=\|v\|^2_K=\frac{1}{2}$, and letting $(\mathbf{X}_k)_\bot$ be an $\mathbf{M}$-orthonormal basis for the $\mathbf{M}$-orthogonal complement of $span\{\mathbf{X}_k\}$, we use
%\begin{equation*}
%\sin\angle_\mathbf{M}(z,span\{\mathbf{X}_k\})=\|(\mathbf{X}_k)_\bot^T\mathbf{M}z\|_2
%\end{equation*}
%to be the distance between $z$ and the projection subspace $span\{\mathbf{X}_k\}$, here $\angle_\mathbf{M}(z,span\{\mathbf{X}_k\})$ is called the $\mathbf{M}$-canonical angle between $z$ and $span\{\mathbf{X}_k\}$, it is induced the $\mathbf{M}$-inner product $(x,y)_\mathbf{M}=y^T\mathbf{M}x$, for $\forall x,y \in\mathbb{R}^n$. Since both $K$ and $M$ are symmetric positive definite, then they have the factorizations
%\begin{equation*}
%K=LL^T,~~~~~~~M=RR^T,
%\end{equation*}
%where both $L$ and $R$ are invertible. If we set $\mathbf{R}=\left[
%                                                               \begin{array}{cc}
%                                                                 R & 0 \\
%                                                                 0 & L \\
%                                                               \end{array}
%                                                             \right]$, then $\mathbf{R}$ is nonsingular, and $\mathbf{M}=\mathbf{R}\mathbf{R}^T$. Hence from (2.7) and (2.8) in \cite{ZhangXL14}, we have
%\begin{equation*}
%\angle_\mathbf{M}(z,span\{\mathbf{X}_k\})=\angle(\mathbf{R}^Tz,span\{\mathbf{R}^T\mathbf{X}_k\}).
%\end{equation*}
%Consequently,
%\begin{equation*}
%\sin\angle_\mathbf{M}(z,span\{\mathbf{X}_k\})=\sin\angle(\mathbf{R}^Tz,span\{\mathbf{R}^T\mathbf{X}_k\})=\|(I-P)\mathbf{R}^Tz\|_2,
%\end{equation*}
%where $P$ is the orthogonal projector onto $span\{\mathbf{R}^T\mathbf{X}_k\}$.
%
%Obviously, $(\mathbf{X}_k)_\bot=\mathbf{X}_{n-k}=\left[
%                                          \begin{array}{cc}
%                                            X_{n-k} &  \\
%                                             & Y_{n-k} \\
%                                          \end{array}
%                                        \right]$. For brevity, we denote $\mathbf{g}=X_k^TMu$, $\mathbf{g}_\bot=X_{n-k}^TMu$, $\mathbf{f}=Y_k^TKv$, $\mathbf{f}_\bot=Y_{n-k}^TKv$, and denote $\varepsilon$ the sin of the $\mathbf{M}$-canonical angle between $z$ and $span\{\mathbf{X}_k\}$, then $$\varepsilon=\sin\angle_\mathbf{M}(z,span\{\mathbf{X}_k\})=\|\left[
%                                                                                    \begin{array}{c}
%                                                                                      \mathbf{g}_\bot \\
%                                                                                      \mathbf{f}_\bot \\
%                                                                                    \end{array}
%                                                                                  \right]
%\|_2,$$
%and $\|\left[
%                                                                                    \begin{array}{c}
%                                                                                      \mathbf{g} \\
%                                                                                      \mathbf{f} \\
%                                                                                    \end{array}
%                                                                                  \right]
%\|_2=\sqrt{1-\varepsilon^2}$.
%
%\begin{theorem}
%Let $(\lambda,z=\left[
%              \begin{array}{c}
%                u \\
%                v \\
%              \end{array}
%            \right]
%)$ is the exact eigenpair of $\mathbf{H}$ with $\|u\|^2_M=\|v\|^2_K=\frac{1}{2}$, and let $\varepsilon$, $\mathbf{g}$, $\mathbf{g}_\bot$, $\mathbf{f}$, $\mathbf{f}_\bot$, $\mathbf{A}$, and $\mathbf{C}$ be defined as in the above. Then there exsit a matrix $E$ satisfying
%\begin{equation}\label{DAniC}
%\|E\|_2\leq\frac{\varepsilon}{\sqrt{1-\varepsilon^2}}(|\lambda|a+b)
%\end{equation}
%and such that $\lambda$ is an eigenvalue of $\mathbf{A}^{-1}\mathbf{C}+E$, where $a=\frac{\beta_k}{\alpha_k\alpha_{k-1}}\sqrt{(\alpha_{k-1})^2+(\beta_{k-1})^2}$, $b=\max\{\frac{\alpha_{k+1}\beta_k}{\alpha_k\alpha_{k-1}}\sqrt{(\alpha_{k-1})^2+(\beta_{k-1})^2}, \beta_k\}$. Furthermore, there exists an eigenvalue $\theta$ of $\mathbf{A}^{-1}\mathbf{C}$ satisfying
%\begin{equation}\label{theAniC}
%|\lambda-\theta|\leq(2\max\{\|K\|_2,\|M\|_2\}+\|E\|_2)^{1-\frac{1}{k}}\|E\|^{\frac{1}{k}}.
%\end{equation}
%\end{theorem}
%
%
%\begin{proof}
%From the relation $\mathbf{H}z-\lambda z=0$, we have $\mathbf{H}^T\mathbf{M}\mathbf{H}z-\lambda\mathbf{H}^T\mathbf{M}z=0$. Note that
%\begin{equation*}
%\left[
%                                          \begin{array}{cc}
%                                            [X_{k},X_{n-k}] &  \\
%                                             & [Y_{k},Y_{n-k}] \\
%                                          \end{array}
%                                        \right]
%\left[
%                                          \begin{array}{cc}
%                                            [X_{k},X_{n-k}]M &  \\
%                                             & [Y_{k},Y_{n-k}]K \\
%                                          \end{array}
%                                        \right]^T=I.
%\end{equation*}
%Then we have
%\begin{equation*}
%\begin{split}
%&\mathbf{X}_k^T\mathbf{H}^T\mathbf{M}\mathbf{H}\left[
%                                          \begin{array}{cc}
%                                            [X_{k},X_{n-k}] &  \\
%                                             & [Y_{k},Y_{n-k}] \\
%                                          \end{array}
%                                        \right]
%\left[
%                                          \begin{array}{cc}
%                                            [X_{k},X_{n-k}]^TM &  \\
%                                             & [Y_{k},Y_{n-k}]^TK \\
%                                          \end{array}
%                                        \right]z\\
%-&\lambda\mathbf{X}_k^T\mathbf{H}^T\mathbf{M}\left[
%                                          \begin{array}{cc}
%                                            [X_{k},X_{n-k}] &  \\
%                                             & [Y_{k},Y_{n-k}] \\
%                                          \end{array}
%                                        \right]
%\left[
%                                          \begin{array}{cc}
%                                            [X_{k},X_{n-k}]^TM &  \\
%                                             & [Y_{k},Y_{n-k}]^TK \\
%                                          \end{array}
%                                        \right]z=0
%\end{split}
%\end{equation*}
%Using $MX_k=Y_kB_k$ and $Y^TKY=I$, the above equation is equivalent to the following one
%\begin{equation*}
%\begin{split}
%&\left[
%   \begin{array}{cc}
%     X_k^TMKMX_k &  \\
%      & Y_k^TKMKY_k \\
%   \end{array}
% \right]\left[
%                                                                                    \begin{array}{c}
%                                                                                      \mathbf{g} \\
%                                                                                      \mathbf{f} \\
%                                                                                    \end{array}
%                                                                                  \right]-\lambda\left[
%                                                                                                   \begin{array}{cc}
%                                                                                                      &X_k^TMKY_k\\
%                                                                                                     Y_k^TKMX_k&\\
%                                                                                                   \end{array}
%                                                                                                 \right]\left[
%                                                                                    \begin{array}{c}
%                                                                                      \mathbf{g} \\
%                                                                                      \mathbf{f} \\
%                                                                                    \end{array}
%                                                                                  \right]\\
%=&\lambda\left[
%           \begin{array}{cc}
%             0 & 0 \\
%             Y_k^TKMX_{n-k} & 0 \\
%           \end{array}
%         \right]\left[
%                                                                                    \begin{array}{c}
%                                                                                      \mathbf{g}_\bot \\
%                                                                                      \mathbf{f}_\bot \\
%                                                                                    \end{array}
%                                                                                  \right]-\left[
%   \begin{array}{cc}
%     X_k^TMKMX_{n-k} &  \\
%      & Y_k^TKMKY_{n-k} \\
%   \end{array}
% \right]\left[
%                                                                                    \begin{array}{c}
%                                                                                      \mathbf{g}_\bot \\
%                                                                                      \mathbf{f}_\bot \\
%                                                                                    \end{array}
%                                                                                  \right]
%\end{split}
%\end{equation*}
%Note the definition of $\mathbf{A}$ and $\mathbf{C}$, then left multiplying the above relation by $\mathbf{A}^{-1}$, and from (\ref{KMYXk}), we can get
%\begin{equation*}
%\begin{split}
%&\mathbf{A}^{-1}\mathbf{C}\left[
%                                                                                    \begin{array}{c}
%                                                                                      \mathbf{g} \\
%                                                                                      \mathbf{f} \\
%                                                                                    \end{array}
%                                                                                  \right]-\lambda\left[
%                                                                                    \begin{array}{c}
%                                                                                      \mathbf{g} \\
%                                                                                      \mathbf{f} \\
%                                                                                    \end{array}
%                                                                                  \right]\\
%=&\mathbf{A}^{-1}\left(\lambda\left[
%           \begin{array}{cc}
%             0 & 0 \\
%             Y_k^TKMX_{n-k} & 0 \\
%           \end{array}
%         \right]\left[
%                                                                                    \begin{array}{c}
%                                                                                      \mathbf{g}_\bot \\
%                                                                                      \mathbf{f}_\bot \\
%                                                                                    \end{array}
%                                                                                  \right]-\left[
%   \begin{array}{cc}
%     X_k^TMKMX_{n-k} &  \\
%      & Y_k^TKMKY_{n-k} \\
%   \end{array}
% \right]\left[
%                                                                                    \begin{array}{c}
%                                                                                      \mathbf{g}_\bot \\
%                                                                                      \mathbf{f}_\bot \\
%                                                                                    \end{array}
%                                                                                  \right]\right)\\
%=&\lambda\left[
%           \begin{array}{cc}
%             \beta_kB_k^{-1}e_ke_1^T & 0 \\
%             0 & 0 \\
%           \end{array}
%         \right]\left[
%                                                                                    \begin{array}{c}
%                                                                                      \mathbf{g}_\bot \\
%                                                                                      \mathbf{f}_\bot \\
%                                                                                    \end{array}
%                                                                                  \right]-\left[
%                                                                                            \begin{array}{cc}
%                                                                                              0 & \alpha_{k+1}\beta_kB_k^{-1}e_ke_1^T \\
%                                                                                              \alpha_{k}\beta_kB_{k}^{-T}e_ke_1^T & 0 \\
%                                                                                            \end{array}
%                                                                                          \right]
%\left[
%                                                                                    \begin{array}{c}
%                                                                                      \mathbf{g}_\bot \\
%                                                                                      \mathbf{f}_\bot \\
%                                                                                    \end{array}
%                                                                                  \right],
%\end{split}
%\end{equation*}
%the last equation is because $\mathbf{A}^{-1}=\left[
%                                                \begin{array}{cc}
%                                                  0 & B_k^{-1} \\
%                                                  B_k^{-T} & 0 \\
%                                                \end{array}
%                                              \right]$.\\
%Normalize $\left[
%                                                                                    \begin{array}{c}
%                                                                                      \mathbf{g} \\
%                                                                                      \mathbf{f} \\
%                                                                                    \end{array}
%                                                                                  \right]$ by $\left[
%                                                                                    \begin{array}{c}
%\mathbf{\hat{g}} \\                                                                                      \mathbf{\hat{f}} \\
%                                                                                    \end{array}
%                                                                                  \right]=\left[
%                                                                                    \begin{array}{c}
%                                                                                      \mathbf{g} \\
%                                                                                      \mathbf{f} \\
%                                                                                    \end{array}
%                                                                                  \right]/\sqrt{1-\varepsilon^2}$, and set
%\begin{equation}\label{rAniC}
%\begin{split}
%r&=\mathbf{A}^{-1}\mathbf{C}\left[
%                                                                                    \begin{array}{c}
%                                                                                      \mathbf{\hat{g}} \\
%                                                                                      \mathbf{\hat{f}} \\
%                                                                                    \end{array}
%                                                                                  \right]-\lambda\left[
%                                                                                    \begin{array}{c}
%                                                                                      \mathbf{\hat{g}} \\
%                                                                                      \mathbf{\hat{f}} \\
%                                                                                    \end{array}
%                                                                                  \right]\\
%&=\left(\lambda\left[
%           \begin{array}{cc}
%             \beta_kB_k^{-1}e_ke_1^T & 0 \\
%             0 & 0 \\
%           \end{array}
%         \right]-\left[
%                                                                                            \begin{array}{cc}
%                                                                                              0 & \alpha_{k+1}\beta_kB_k^{-1}e_ke_1^T \\
%                                                                                              \alpha_{k}\beta_kB_{k}^{-T}e_ke_1^T & 0 \\
%                                                                                            \end{array}
%                                                                                          \right]\right)
%\left[
%                                                                                    \begin{array}{c}
%                                                                                      \mathbf{g}_\bot \\
%                                                                                      \mathbf{f}_\bot \\
%                                                                                    \end{array}
%                                                                                  \right]/\sqrt{1-\varepsilon^2}.
%\end{split}
%\end{equation}
%Since $B_k^{-1}e_k=[0,\cdots,0,-\frac{\beta_{k-1}}{\alpha_{k-1}\alpha_k},\frac{1}{\alpha_k}]^T$ and $B_k^{-T}e_k=\frac{1}{\alpha_k}e_k$, thus taking 2-norm for both sides of (\ref{rAniC}), we have
%\begin{equation}\label{rnorm}
%\|r\|_2\leq\frac{\varepsilon}{\sqrt{1-\varepsilon^2}}(|\lambda|a+b)
%\end{equation}
%with $a$ and $b$ defined in Theorem 1.
%Setting
%$$E=-r\left[
%                                                                                    \begin{array}{c}
%                                                                                      \mathbf{\hat{g}} \\
%                                                                                      \mathbf{\hat{f}} \\
%                                                                                    \end{array}
%                                                                                  \right]^T,$$
%then we get (\ref{DAniC}), and $(\mathbf{A}^{-1}\mathbf{C}+E)\left[
%                                                                                    \begin{array}{c}
%                                                                                      \mathbf{\hat{g}} \\
%                                                                                      \mathbf{\hat{f}} \\
%                                                                                    \end{array}
%                                                                                  \right]=\lambda\left[
%                                                                                    \begin{array}{c}
%                                                                                      \mathbf{\hat{g}} \\
%                                                                                      \mathbf{\hat{f}} \\
%                                                                                    \end{array}
%                                                                                  \right]$.
%Inequality (\ref{theAniC}) is direct from Corollary 2.2 of \cite{Jia04}.
%\end{proof}
%
%Actually, the exact eigenvector
%$z$ and the weighted harmonic Ritz vector $\hat{z}$ are not far away from each other. Theorem will give the sin of the $\mathbf{M}$-canonical angle between $z$ and $\hat{z}$ under the hypothesis that all weighted harmonic Ritz values are well separated.
%Let $\hat{\lambda}$ be the weighted harmonic Ritz value, and $w$ be the corresponding eigenvector of $\mathbf{A}^{-1}\mathbf{C}$ with $\|w\|_2=1$. Let $[w, W_\bot]$ be orthogonal, from the relation $\mathbf{A}^{-1}\mathbf{C}w=\hat{\lambda}w$, it follows that
%\begin{equation*}
%\left[
%  \begin{array}{c}
%    w^T \\
%    W_\bot^T \\
%  \end{array}
%\right]\mathbf{A}^{-1}\mathbf{C}[w~~W_\bot]=\left[
%                                             \begin{array}{cc}
%                                               \hat{\lambda} & w^T\mathbf{A}^{-1}\mathbf{C}W_\bot \\
%                                               0 & G \\
%                                             \end{array}
%                                           \right].
%\end{equation*}
%Consequently, $G$ contains all eigenvalues of $\mathbf{A}^{-1}\mathbf{C}$ other than $\hat{\lambda}$.
%Since all weighted harmonic Ritz values are well separated, the eigenvalues of $G$ are not equal to $\hat{\lambda}$, hence $G-\hat{\lambda}I$ is nonsingular. Define
%\begin{equation*}
%sep(\hat{\lambda},G)=\|(G-\hat{\lambda}I)^{-1}\|_2^{-1}.
%\end{equation*}
%
%\begin{theorem}
%Under the above assumption, if
%\begin{equation*}
%sep(\hat{\lambda},G)-|\hat{\lambda}-\lambda|>0,
%\end{equation*}
%then
%\begin{equation*}
%\begin{split}
%\sin\angle_\mathbf{M}(z,\hat{z})&\leq\left(1+\frac{|\lambda|a+b}{\sqrt{1+\varepsilon^2}sep(\lambda,G)}\right)\varepsilon\\
%&\leq\left(1+\frac{|\lambda|a+b}{\sqrt{1+\varepsilon^2}(sep(\hat{\lambda},G)-|\hat{\lambda}-\lambda|)}\right)\varepsilon,
%\end{split}
%\end{equation*}
%where $a$ and $b$ are the same as in Theorem 1.
%\end{theorem}
%
%\begin{proof}
%First, by the continuity of sep, we get
%\begin{equation}\label{sep}
%sep(\lambda,G)\geq sep(\hat{\lambda},G)-|\hat{\lambda}-\lambda|>0.
%\end{equation}
%We let $\mathbf{\hat{w}}=\left[
%                                                                                    \begin{array}{c}
%                                                                                      \mathbf{\hat{g}} \\
%                                                                                      \mathbf{\hat{f}} \\
%                                                                                    \end{array}
%                                                                                  \right]$, where $\left[
%                                                                                    \begin{array}{c}
%                                                                                      \mathbf{\hat{g}} \\
%                                                                                      \mathbf{\hat{f}} \\
%                                                                                    \end{array}
%                                                                                  \right]$ is from the proof of Theorem 1. Hence $r=\mathbf{A}^{-1}\mathbf{C}\mathbf{\hat{w}}-\lambda\mathbf{\hat{w}}$.
%Since $(\hat{\lambda}, w)$ be the eigenpair of $\mathbf{A}^{-1}\mathbf{C}$, so that $\mathbf{X}_kw$ is the weighted harmonic Ritz value $\hat{z}$ used to approximate $z$. Then it follows from Theorem 3.1 in \cite{Jia04} and Theorem 6.1 in \cite{JiaSte01} that
%\begin{equation}\label{wsin}
%\sin\angle(w,\mathbf{\hat{w}})\leq\frac{\|r\|_2}{sep(\lambda,G)}.
%\end{equation}
%Because $\mathbf{X}_k$ is $\mathbf{M}$-orthonormal, i.e., $(\mathbf{X}_k^T\mathbf{R})(\mathbf{R}^T\mathbf{X}_k)=I$, thus from the definition of $\hat{z}$ and $\mathbf{\hat{w}}$, we have
%\begin{equation*}
%\begin{split}
%\sin\angle(w,\mathbf{\hat{w}})=&\sin\angle(\mathbf{R}^T\mathbf{X}_kw,\mathbf{R}^T\mathbf{X}_k\mathbf{\hat{w}})\\
%=&\sin\angle(\mathbf{R}^T\hat{z},(\mathbf{R}^T\mathbf{X}_k)(\mathbf{X}_k^T\mathbf{R})(\mathbf{R}^Tz))\\
%=&\sin\angle(\mathbf{R}^T\hat{z},P(\mathbf{R}^Tz)),
%\end{split}
%\end{equation*}
%here $P$ is the orthogonal projector onto $span\{\mathbf{R}^T\mathbf{X}_k\}$.
%
%Owing to the following inequality
%\begin{equation*}
%\begin{split}
%\angle(\mathbf{R}^Tz,\mathbf{R}^T\hat{z})\leq&\angle(\mathbf{R}^Tz,P(\mathbf{R}^Tz))+\angle(P(\mathbf{R}^Tz),\mathbf{R}^T\hat{z})\\
%&=\angle(\mathbf{R}^Tz,span\{\mathbf{R}^Tz\})+\angle(P(\mathbf{R}^Tz),\mathbf{R}^T\hat{z}),
%\end{split}
%\end{equation*}
%thus using (\ref{rnorm}) in the proof of Theorem 1, (\ref{wsin}), and the definition of $\varepsilon$, we have
%\begin{equation*}
%\begin{split}
%\sin\angle_\mathbf{M}(z,\hat{z})=&\sin\angle(\mathbf{R}^Tz,\mathbf{R}^T\hat{z})\\
%\leq&\sin\angle(\mathbf{R}^Tz,span\{\mathbf{R}^Tz\})+\sin\angle(P(\mathbf{R}^Tz),\mathbf{R}^T\hat{z})\\
%=&\|(I-P)\mathbf{R}^Tz\|_2+\sin\angle(w,\mathbf{\hat{w}})\\
%\leq&\varepsilon+\frac{\|r\|_2}{sep(\lambda,G)}\\
%\leq&\left(1+\frac{|\lambda|a+b}{\sqrt{1-\varepsilon^2}sep(\lambda,G)}\right)\varepsilon\\
%\leq&\left(1+\frac{|\lambda|a+b}{\sqrt{1-\varepsilon^2}(sep(\hat{\lambda},G)+|\hat{\lambda}-\lambda|)}\right)\varepsilon,
%\end{split}
%\end{equation*}
%the last inequality is from (\ref{sep}).
%\end{proof}
%
%
%\section{Thick restarting}
%With the size of searching space increasing, the storage, computational costs of {\bf whGKL} grows quickly, and its numerical stability is affected gradually. To overcome the difficulty, we will consider a thick-restarting procedure for {\bf whGKL}. Details will be discussed in the following.
%
%Assume {\bf whGKL} runs $m$ steps, and generates $X_{m+1}$, $Y_m$, $B_m$, and $B_{m,m+1}=[B_m, \beta_me_m]$. Let $B_{m,m+1}=\Phi_m\Sigma_m\Psi_{m+1,m}^T$ be the SVD of $B_{m,m+1}$, and $\Phi_{m,k}$ be the first $k$ columns of $\Phi_m$.
%Take QR decomposition,
%\begin{equation}\label{newcyQR}
%\left[
%                        \begin{array}{cc}
%                          B_m^{-1}\Phi_{m,k}\Sigma_k & -\beta_mB_m^{-1}e_m \\
%                          0 & 1 \\
%                        \end{array}
%                      \right]=Q_{k+1}R_{k+1},
%\end{equation}
%where $Q_{k+1}\in\mathbb{R}^{(k+1)\times(k+1)}$ is orthogonal, $R_{k+1}\in\mathbb{R}^{(k+1)\times(k+1)}$ is an upper tridiagonal matrix, whose $(k+1)\times(k+1)$ element is 1.
%
%Let
%\begin{equation}\label{necyXk1}
%\widetilde{X}_{k+1}=\left[
%                           \begin{array}{ccc}
%                             \tilde{x}_1 & \cdots & \tilde{x}_{k+1} \\
%                           \end{array}
%                         \right]=[X_m, x_{m+1}]Q_{k+1},
%\end{equation}
%then $\widetilde{X}_{k+1}^TM\widetilde{X}_{k+1}=I_{k+1}$, and
%\begin{equation*}
%\begin{split}
%M\widetilde{X}_{k+1}&=[MX_m,Mx_{m+1}]\left[
%                        \begin{array}{cc}
%                          B_m^{-1}\Phi_{m,k}\Sigma_k & -\beta_mB_m^{-1}e_m \\
%                          0 & 1 \\
%                        \end{array}
%                      \right]R_{k+1}^{-1}\\
%&=[Y_mB_m,Mx_{m+1}]\left[
%                        \begin{array}{cc}
%                          B_m^{-1}\Phi_{m,k}\Sigma_k & -\beta_mB_m^{-1}e_m \\
%                          0 & 1 \\
%                        \end{array}
%                      \right]R_{k+1}^{-1}\\
%&=[Y_m\Phi_{m,k}\Sigma_k,-\beta_my_m+Mx_{m+1}]R_{k+1}^{-1}.
%\end{split}
%M\widetilde{X}_{k+1}=&[\widetilde{Y}_k,\tilde{y}_{k+1}]\left[
%                                                        \begin{array}{cccc}
%                                                          \sigma_1 &  &  & \tilde{\gamma}_1 \\
%                                                           & \ddots &  & \vdots \\
%                                                           &  & \sigma_k & \tilde{\gamma}_k \\
%                                                          0 & \cdots & 0 & \tilde{\alpha}_{k+1} \\
%                                                        \end{array}
%                                                      \right]R_{k+1}^{-1}\\
%=&\widetilde{Y}_{k+1}\widetilde{B}_{k+1},
%\end{split}
%\end{equation}
%here $\widetilde{Y}_{k+1}=[\widetilde{Y}_k,\tilde{y}_{k+1}]$,
%\begin{equation}\label{necyBk1}
%\widetilde{B}_{k+1}=\left[
%                                                        \begin{array}{cccc}
%                                                          \sigma_1 &  &  & \tilde{\gamma}_1 \\
%                                                           & \ddots &  & \vdots \\
%                                                           &  & \sigma_k & \tilde{\gamma}_k \\
%                                                          0 & \cdots & 0 & \tilde{\alpha}_{k+1} \\
%                                                        \end{array}
%                                                      \right]R_{k+1}^{-1},
%\end{equation}
%obviously, the $(k+1, k+1)$ element of $\widetilde{B}_{k+1}$ is $\tilde{\alpha}_{k+1}$.
%
%Since the SVD of $B_{m,m+1}$ is $B_{m,m+1}=\Phi_m\Sigma_m\Psi_{m+1,m}^T$, then,
%\begin{equation}\label{BSVD1}
%B_{m,m+1}^T\Phi_{m,k}=\Psi_{m+1,k}\Sigma_k,
%\end{equation}
%and
%\begin{equation}\label{BSVD2}
%[B_m,\beta_me_m]\Psi_{m+1,k}=\Phi_{m,k}\Sigma_k.
%\end{equation}
%From (\ref{BSVD2}), there has $[I_m,\beta_mB_m^{-1}e_m]\Psi_{m+1,k}=B_m^{-1}\Phi_{m,k}\Sigma_k$, thus,
%\begin{equation}\label{Phi}
%\begin{split}
%\end{split}
%\end{equation}
%From (\ref{MNX}), we know
%\begin{equation*}
%\widetilde{Y}_k^TKM\widetilde{X}_{k+1}=[I_k,0]\widetilde{B}_{k+1}\triangleq \widetilde{B}_{k,k+1}.
%\end{equation*}
%While from (\ref{KNY1}), we have
%\begin{equation*}
%\widetilde{X}_{k+1}^TKM\widetilde{Y}_k=R_{k+1}\left[
%                      \begin{array}{c}
%                        I_k \\
%                        e_{m+1}^T\Psi_{m+1,k} \\
%                      \end{array}
%                    \right]\Sigma_k.
%\end{equation*}
%Comparing the above two equations, it can be known that
%\begin{equation*}
%\widetilde{B}_{k,k+1}^T=R_{k+1}\left[
%                      \begin{array}{c}
%                        I_k \\
%                        e_{m+1}^T\Psi_{m+1,k} \\
%                      \end{array}
%                    \right]\Sigma_k,
%\end{equation*}
%so, (\ref{KNY1}) can be written as

%
%Next, let's see the vector $K\tilde{y}_{k+1}$. Because
%\begin{equation*}
%\widetilde{X}_{k+1}^TMK\tilde{y}_{k+1}=\widetilde{B}_k^T\widetilde{Y}_{k+1}^TK\tilde{y}_{k+1}
%=\widetilde{B}_k^Te_{k+1}=\tilde{\alpha}_{k+1}e_{k+1},
%\end{equation*}
%thus, we can suppose $K\tilde{y}_{k+1}=\tilde{\alpha}_{k+1}\tilde{x}_{k+1}+\tilde{r}_{k+1}$, here $\tilde{r}_{k+1}$ satisfies $\widetilde{X}_{k+1}^TM\tilde{r}_{k+1}=0$. Therefore
%\begin{equation*}
%\begin{split}

%Let $\tilde{\beta}_{k+1}=\|r_{k+1}\|_M$, and
%\begin{equation}\label{necyxk2}
%\tilde{x}_{k+2}=\tilde{r}_{k+1}/\|\tilde{r}_{k+1}\|_M,
%\begin{remark}
%$\widetilde{B}_{k+1}$ is an upper tridiagonal matrix, it's no longer an upper bidiagonal matrix like $B_{k+1}$ in Algorithm {\bf wGKL} and Algorithm {\bf whGKL}. For simplicity, we still use $X_m$, $Y_m$, $B_m$, and $B_{m,m+1}$ to replace $\widetilde{X}_m$, $\widetilde{Y}_m$, $\widetilde{B}_m$, and $\widetilde{B}_{m,m+1}$, respectively, then equation (\ref{KMYXk}) is still holds in {\bf whGKL-DR}.
%\end{remark}
%
%The following is the weighted harmonic Gloub-Kahan-Lanczos with deflated restarting algorithm {\bf whGKL-DR}.
%
%\begin{algorithm}\label{alg3}{\bf whGKL-DR}\\
%{\bf 1.} Given an initial guess $x_1$ satisfying $\|x_1\|_M=1$, a tolerance $tol$, an integer $k$ the number of approximate eigenvectors that we want to add to the solving subspace, an integer $m$ the dimension of solving subspace, as well as $l$ the desired number of eigenpairs. Set $\beta_0=1$, $y_0=0$;\\
%{\bf 2.} Apply {\bf wGKL} from the current point to generate the rest of $X_{m+1}$, $Y_m$, and $B_{m,m+1}$. if it is the first cycle, the current point is from $x_1$, else from $x_{k+2}$;\\
%{\bf 3.} Compute an SVD of $B_{m,m+1}$ as in (\ref{Bk1svd}), select $l(l\leq k)$ wanted singular values $\sigma_j$, and their associated left singular vectors $\varphi_j$, $j=1,\cdots,l$. Form  $\pm\hat{\lambda}_j=\pm\sigma_j$, $\hat{z}_j=\left[
%                                                         \begin{array}{c}

%\section{Numerical examples}

\section{Conclusions}
\label{sec5}
In this work, we proposed a fast compact difference scheme with unequal time-steps for the tempered TFBS model.
The numerical method is constructed by the compact difference operator and the fast $L1$ formula with nonuniform time mesh. The unconditional stability and convergence of $\min\left\{\gamma\alpha,2-\alpha\right\}$-order in time and fourth-order in space are rigorously proved by mathematical induction. Numerical examples are included and the results indicated that the proposed numerical method works very precise and fast.
%%%%%%%%%%%%%%%%%%%%%%%%%%%%%%%%%%
%%%%%%%%%%%%%%%%%%%%%%%%%%%%
\appendix
\section{Appendix}
\label{appd}
For clarity, the regularity of the solution $v$ of Eq. (\ref{eq2.1}) is discussed
in this appendix. It helps us to find that $v$ is smooth away from $\tau = 0$ but it has
in general a certain singular behaviour at $\tau = 0$.

Motivated by \cite{Stynes17,Wang2022}, we use the separation of variables to construct a classical solution
$v(x,\tau)$ of (\ref{eq2.1}) in the form of an infinite series. Let $\{(\mu_i, \psi_i):
i = 1, 2,\ldots\}$ be the eigenvalues and eigenfunctions for the Sturm-Liouville two-point
boundary value problem
\begin{equation}
\mathcal{L}\psi_i:= -\frac{\sigma^2}{2}\psi''_i + q\psi_i = \mu_i\psi_i~~{\rm on}~(x_l,x_r),
\quad\psi_i(x_l) =\psi_i(x_r)\equiv 0,
\end{equation}
where the eigenfunctions are normalised by requiring $\|\psi_i\|_2 = 1$ for all
$i$. It is well known that $\mu_i > 0$ for all $i$. A standard separation of variables
technique construct an infinite series solution to Eq. (\ref{eq2.1}) with the form
\begin{equation*}
v(x,\tau) = \sum^{\infty}_{i=1}v_i(\tau)\psi_i(x).
\end{equation*}
For clarity, we denote $v_i(\tau) = \langle v(\cdot,\tau),\psi_i(x)\rangle$, $\widetilde{\sigma}_i = \langle\sigma(\cdot),\psi_i(\cdot)\rangle$, $g_i(\tau) = \langle v(\cdot,\tau),\psi_i(\cdot)\rangle$, $Q = \Omega\times(0,T]$, $\bar{Q} = \bar{\Omega}\times(0,T]$ and $\bar{\Omega} = \partial\Omega\cup\Omega$.
Then we have
\begin{equation*}
{}^{C}_0\mathbb{D}^{\alpha,\lambda}_{\tau}v_i(\tau) = -\mu_iv_i(\tau) + g_i(\tau).
\end{equation*}
With the help of Laplace transform, one obtains
\begin{equation*}
(s + \lambda)^{\alpha}\hat{v}_i(s) - (s + \lambda)^{\alpha-1}\widetilde{\sigma}_i =
-\mu_i\hat{v}_i(s) + \hat{g}_i(s),
\end{equation*}
or equivalently \cite[(62)]{Li2015}
\begin{equation}
\hat{v}_i(s) = \frac{(s + \lambda)^{\alpha-1}\widetilde{\sigma}_i
+ \hat{g}_i(s)}{(s + \lambda)^{\alpha} + \mu_i}.
\end{equation}
%%%%%%%%%%%%%%
Hence, using the inverse Laplace transform and two-parameter M-L function
\cite{Ishteva05} defined by
\begin{equation*}
E_{\alpha,\beta}(z) := \sum^{\infty}_{k=0}\frac{z^k}{\Gamma(\alpha z + \beta)},\quad
\Re e(\alpha) > 0,~\Re e(\beta) > 0,~z\in\mathbb{C},
\end{equation*}
we can obtain the following form
\begin{equation}
v(x,\tau) = \sum^{\infty}_{i=1}\left[\widetilde{\sigma}_iG_{i}(\tau)
+ F_{i}(\tau)\right]\psi_i(x)
\label{Eq.A1}
\end{equation}
--see \cite[(2.8)]{Wang2022}--where $G_i(\tau) = e^{-\lambda\tau}E_{\alpha,1}(-\mu_i\tau^{\alpha})$
and $F_i(\tau) = \int^{\tau}_0e^{-\lambda s}s^{\alpha-1}E_{\alpha,\alpha}(-\mu_i s^{\alpha})
g_i(\tau-s)ds$.
%Here we suppose that the eigenvalues sequence
%$\{\mu_i\}^{\infty}_{i=1}$ is non-decreasing, there exists an integer $i_0$ such that
%$a_i > 0$ for all $i \geq i_0$ and $a_i \leq 0$ for all $i < i_0$.

With the help of sectorial operator \cite{Sakamoto}, for each $\nu\in\mathbb{R}$ the fractional power $\mathcal{L}^{\nu}$ of the operator $\mathcal{L}$ is defined
with the domain
\begin{equation*}
H(\mathcal{L}^{\nu}) := \left\{g\in L_2(\Omega): \sum^{\infty}_{i=1}\mu^{2\nu}_i|\langle g,\psi_i\rangle|^2\right\},
\end{equation*}
and the norm
\begin{equation*}
\|g\|_{\mathcal{L}^{\nu}} = \left(\sum^{\infty}_{i=1}\mu^{2\nu}_i|\langle g,\psi_i\rangle|^2\right)^{1/2}.
\end{equation*}
%%%%%%%%%%%%%%%%%%%%
\begin{theorem}
Let $v$ be the solution of Eq. (\ref{eq2.1}). Then for $\forall \epsilon > 0$, there exists the constant $C$ independent of $\tau$ such that the following
conclusions are available:
\begin{itemize}
\item[\romannumeral1)] if $\sigma\in H\left(\mathcal{L}^{\frac{1+\epsilon}{2}}\right), g(\cdot,\tau)\in H\left(\mathcal{L}^{\frac{1+\epsilon}{2}}\right)$ for $\forall \tau\in[0,T]$, then $|v(x,\tau)|\leq C$ for
$(x,\tau)\in \bar{Q}$;
\item[\romannumeral2)] if $\sigma\in H\left(\mathcal{L}^{\frac{3+\epsilon}{2}}\right), g(\cdot,\tau)\in H\left(\mathcal{L}^{\frac{1+\epsilon}{2}}\right)$ for $\forall \tau\in[0,T]$, then
$\|g_{tau}(\cdot,\tau)\|_{\mathcal{L}^{\frac{1+\epsilon}{2}}}\leq C\tau^{-\rho}$ ($\rho < 1)$ for $\forall \tau\in (0,T]$,
then $|v_{\tau}(x,\tau)\leq \tau^{\alpha-1}$ for $\forall (x,\tau)\in Q$.
\end{itemize}
\end{theorem}
%%%%%%%%%%%%%%%
\textbf{Proof}: \romannumeral1) With the help of Eq. (\ref{Eq.A1}), the triangle inequality yields that
\begin{equation}
\begin{split}
|v(x,\tau)| & = \sum^{\infty}_{i=1}|\widetilde{\sigma}_i G_i(\tau) + F_i(\tau)|\cdot |\psi_i(x)|\\
            &\leq \sum^{\infty}_{i=1}\Big[|\widetilde{\sigma}_i e^{-\lambda\tau}E_{\alpha,1}(-\mu_i\tau^{\alpha})|~+\\
            &\quad~\left|\int^{\tau}_0e^{-\lambda s}s^{\alpha-1}E_{\alpha,\alpha}(-\mu_i s^{\alpha})g_i(\tau-s)ds\right|\Big]\cdot|\psi_i(x)|,
\end{split}
\label{eqA.4}
\end{equation}
since $\sigma\in H(\mathcal{L}^{\frac{1+\epsilon}{2}})$, $g(\cdot,\tau)\in H(\mathcal{L}^{\frac{1+\epsilon}{2}})$, we have $\|\sigma\|_{L^{\frac{1+\epsilon}{2}}}\leq C$
and $\|g(\cdot,\tau)\|_{\mathcal{L}^{\frac{1+\epsilon}{2}}}\leq C$. In conclusion, we have $\mu_i\approx i$ and $|\psi_i(x)|\leq C$.

Consider the term in (\ref{eqA.4}). Using the Cauchy-Schwarz inequality and \cite[Lemma 2.4]{Wang2022}, we have
\begin{equation*}
\begin{split}
\sum^{\infty}_{i=1}|\widetilde{\sigma}_ie^{-\lambda\tau}E_{\alpha,1}(-\mu_i\tau^{\alpha})|& \leq C\sum^{\infty}_{i=1}|\widetilde{\sigma}_i|\cdot
|E_{\alpha,1}(-\mu_i\tau^{\alpha})|\\
%&\leq C \left(\sum^{i_0-1}_{i=1} + \sum^{\infty}_{i=i_0}\right)|\widetilde{\sigma}_i|\cdot|E_{\alpha,1}(-a_i\tau^{\alpha})|\\
&\leq C \left(\sum^{\infty}_{i=1}\frac{1}{\mu^{1+\epsilon}_i}\right)^{1/2}\left(\sum^{\infty}_{i=1}\mu^{1+\epsilon}_i|\widetilde{\sigma}_i|^2\right)^{1/2}\\
&\leq C \|\sigma\|_{\mathcal{L}^{\frac{1+\epsilon}{2}}}.
\end{split}
\end{equation*}
%By using the Cauchy-Schwarz inequality and \cite[Lemmas 2.1-2.2, Lemma 2.4]{Wang2022}, one has
%\begin{equation*}
%\begin{split}
%\sum^{\infty}_{i=1}\left|\widetilde{\sigma}_i\lambda\int^{\tau}_0e^{-\lambda s}E_{\alpha,1}(-a_i s^{\alpha})ds\right| & \leq C\sum^{\infty}_{i=1}|\widetilde{\sigma}_i|\left|\int^{\tau}_0 e^{-\lambda s}E_{\alpha,1}(-a_i s^{\alpha})ds\right|\\
%&\leq C\left(\sum^{i_0-1}_{i=1} + \sum^{\infty}_{i=i_0}\right)|\widetilde{\sigma}_i|\tau E_{\alpha,2}(-a_i\tau^{\alpha})\\
%&\leq C\left(\sum^{\infty}_{i=1}\frac{1}{\lambda^{1+\epsilon}_i}\right)^{1/2}\left(\sum^{\infty}_{i=1}\lambda^{1+\epsilon}_i|\widetilde{\sigma}_i|^2\right)^{1/2}\\
%&\leq C\|\sigma\|_{\mathcal{L}^{\frac{1+\epsilon}{2}}}.
%\end{split}
%\end{equation*}
Similarly, we can obtain
\begin{equation*}
\begin{split}
\sum^{\infty}_{i=1}\left|\int^{\tau}_0e^{-\lambda s}s^{\alpha-1}E_{\alpha,\alpha}(-\mu_is^{\alpha})g_i(\tau-s)ds\right|&\leq
C\int^{\tau}_0\left|s^{\alpha-1}\sum^{\infty}_{i=1}E_{\alpha,\alpha}(-\mu_i s^{\alpha})g_i(\tau-s)\right|ds\\
&\leq C\int^{\tau}_{0}s^{\alpha-1}\left(\sum^{\infty}_{i=1}\frac{1}{\mu^{1+\epsilon}_i}(E_{\alpha,\alpha}(-\mu_i s^{\alpha}))^2\right)^{1/2}\\
&\quad~\cdot\left(\sum^{\infty}_{i=1}\mu^{1+\epsilon}_ig^{2}_i(\tau -s)\right)^{1/2}ds\\
&\leq C.
\end{split}
\end{equation*}
Hence the series (\ref{eqA.4}) is absolutely and uniformly convergence on $\bar{Q}$, and
\begin{equation}
|v(x,\tau)|\leq C,~{\rm for~} \forall (x,\tau)\in \bar{Q}.
\end{equation}

\romannumeral2) Differentiating Eq. (\ref{Eq.A1}) term by term with respect to $\tau$ for $(x,\tau)\in Q$ yields
\begin{equation}
\begin{split}
v_{\tau}(x,\tau) &= \sum^{\infty}_{i=1}\Big[-\widetilde{\sigma}_i e^{-\lambda\tau}\mu_i\tau^{\alpha-1}E_{\alpha,\alpha}(-\mu_i\tau^{\alpha}) + e^{-\lambda\tau}\tau^{\alpha-1}E_{\alpha,\alpha}(-\mu_i\tau^{\alpha})g_i(0)~+ \\
&\quad~\int^{\tau}_0e^{-\lambda s}s^{\alpha-1}
E_{\alpha,\alpha}(-\mu_i s^{\alpha})g'_i(\tau -s)ds\Big]\psi_i(x),
\end{split}
\end{equation}
where we use the fact $\frac{dE_{\alpha,1}(-\lambda \tau^{\alpha})}{d\tau} = -\lambda\tau^{\alpha-1}E_{\alpha,\alpha}(-\lambda\tau^{\alpha})$ to differentiate $E_{\alpha,1}(\cdot)$. Based on the proof of \romannumeral1), it is not hard to check that
\begin{equation}
|v_{\tau}(x,\tau)|\leq C\tau^{\alpha-1},~{\rm for~}(x,\tau)\in Q.
\label{KNY1x}
\end{equation}
Moreover, one can show that ${}^{C}_0\mathbb{D}^{\alpha,\lambda}_{\tau}v$ exists and $v$ is the solution of Eq. (\ref{eq2.1}), the maximum principle guarantees
the uniqueness of solution, which completes the proof.\hfill $\Box$
%_{m,k}=X_{m+1}B_{m,m+1}^T\Phi_{m,k}=X_{m+1}\Psi_{m+1,k}\Sigma_k\
%&=X_{m+1}Q_{k+1}R_{k+1}\left[
%                      {c}
%                        I_k \
%                        e_{m+1}^T\Psi_{m+1,k} \\
%                      \end{array}
%                    \right]\Sigma_k\\
%&=\widetilde{X}_{k+1}R_{k+1}\left[
%                      {c}
%                        I_k \\
%                        e_{m+1}^T\Psi_{m+1,k} \\
%                      \end{array}
%                    \right]\Sigma_k.
%{k+1}
%\left[
%                      \begin{array}{c}
%                        I_k \\
%                        e_{m+1}^T\Psi_{m+1,k} \\
%
%                    \right].


At this stage, we can estimate other derivatives $\frac{\partial^2 v}{\partial\tau^2}$ and $\frac{\partial^p v}{\partial x^p}$~($p=1,2,3,4$) on the domain $Q$. At present,
we summarize all the above activity in the following conclusion.
%%                                                                                              u
%%                                                                                              v
%%
%%                                                                                          \right]
\begin{theorem}
Assume that $\sigma, g(\cdot,\tau)\in H(\mathcal{L}^{\frac{5+\epsilon}{2}})$, $g_{\tau}(\cdot,\tau)$ and $g_{\tau\tau}(\cdot,\tau)$ are
in $H(\mathcal{L}^{\frac{5+\epsilon}{2}})$ for each $\tau\in(0,T]$ with
\begin{equation*}
\|g(\cdot,\tau)\|_{\mathcal{L}^{\frac{5+\epsilon}{2}}} + \|g_{\tau}(\cdot,\tau)\|_{\mathcal{L}^{\frac{1+\epsilon}{2}}} + \tau^{\rho}\|g_{\tau\tau}(\cdot,\tau)\|_{\mathcal{L}^{\frac{1+\epsilon}{2}}}
\leq C_1
\end{equation*}
for $\forall \tau\in(0,T], \forall\epsilon > 0$ and the constant $\rho < 1$, where $C_1$ is a constant independent of $\tau$. Then
there exists a constant $C$ such that
\begin{equation}
\begin{cases}
\left|\frac{\partial^{p}v(x,\tau)}{\partial x^p}\right|\leq C,& {\rm for}~p = 0,1,2,3,4,\\
\left|\frac{\partial^{\ell}v(x,\tau)}{\partial \tau^{\ell}}\right| \leq C(1 + \tau^{\alpha - \ell}), &
{\rm for}~\ell = 0,1,2
\end{cases}
\label{MNX}
\end{equation}
for $\forall(x,\tau)\in\bar{\Omega}\times(0,T]$.
\end{theorem}
In short, we shall assume that the solution $v$ of (\ref{eq2.1}) satisfies the bounds (\ref{MNX}). Thus in general, the solution $v$ of (\ref{eq2.1}) will have a weak singularity
along $\tau = 0$. Its presence leads to significant practical and theoretical difficulties in designing and analysing numerical methods for (\ref{eq2.1}). That is also why we consider
the non-uniform temporal discretization in this paper.


% and the projection subspace $span\{\mathbf{X}_k\}$.
%%\noindent

%
%
\medskip
\section*{Acknowledgment}
{\em The authors sincerely would like to thank Dr. Can Li for his
insightful discussions and suggestions. This work was supported
by the Applied Basic Research Program of Sichuan Province (2020YJ0007)
and the Sichuan Science and Technology Program (2022ZYD0006). X.-M.
Gu would like to thank Prof. Dongling Wang for helpful discussions
during his visiting to Xiangtan University.}


\def\refname{\large \bfseries References}
%\end{center}
\begin{thebibliography}{1}
\bibitem{ksendal03}
B. {\O}ksendal, {\em Stochastic Differential Equations: An Introduction with Applications},
Springer Berlin, Heidelberg (2003).

\bibitem{Staelen}
R. H. D. Staelen, A. S. Hendy, Numerically pricing double barrier options
in a time-fractional Black-Scholes model, {\em Comput. Math. Appl.},
74(6) (2017): 1166-1175.

\bibitem{Meng2010}
L. Meng, M. Wang, Comparison of Black-Scholes formula with fractional Black-Scholes
formula in the foreign exchange option market with changing volatility,
{\em Asia Pac. Financ. Mark.}, 17(2) (2010): 99-111.

\bibitem{Nicolas}
P. Nicolas, {\em An Elementary Introduction to Stochastic Interest Rate Modeling}
(2nd ed.), Advanced Series on Statistical Science and Applied Probability, Vol.
16, World Scientific Publishing, Singapore (2012).

\bibitem{Kou02}
S. G. Kou, A jump-diffusion model for option pricing, {\em Manage. Sci.}, 48(8)
(2002): 1086-1101.

\bibitem{Wyss2000}
W. Wyss, The fractional Black-Scholes equation, {\em Fract. Cal. Appl. Anal.},
3(1) (2000): 51-61.

\bibitem{Jumarie2008}
G. Jumarie, Stock exchange fract ional dynamics defined as fractional exponential
growth driven by (usual) Gaussian white noise. Application to fractional Black
Scholes equations, {\em Insur. Math. Econ.}, 42(1) (2008): 271-287.

\bibitem{Cartea13}
\'{A}. Cartea, Derivatives pricing with marked point processes using tick-by-tick data,
{\em Quant. Finance}, 13(1) (2013): 111-123.

\bibitem{Liang2010}
J.-R. Liang, J. Wang, W.-J. Zhang, W.-Y. Qiu, F.-Y. Ren, The solutions to a bi-fractional
Black-Scholes-Merton differential equation, {\em Int. J. Pure Appl. Math.}, 58(1)
(2010): 99-112.

\bibitem{Magdziarz9}
M. Magdziarz, Black-Scholes formula in subdiffusive regime, {\em J. Stat. Phys.},
136 (2009): 553-564.

\bibitem{Chen15}
W. Chen, X. Xu, S.-P. Zhu, Analytically pricing double barrier options based
on a time-fractional Black-Scholes equation, {\em Comput. Math. Appl.}, 69(12)
(2015): 1407-1419.

\bibitem{Farhadi}
A. Farhadi, M. Salehi, G. H. Erjaee, A new version of Black-Scholes equation presented by
time-fractional derivative, {\em Iran. J. Sci. Technol. Trans. A: Sci.}, 42 (2018): 2159-2166.

\bibitem{Fadugba}
S. E. Fadugba, Homotopy analysis method and its applications in the valuation of
European call options with time-fractional Black-Scholes equation, {\em Chaos Solit.
Fractals}, 141 (2022): 110351.

\bibitem{Grzegorz20}
G. Krzy\.{z}anowski, M. Magdziarz, {\L}. P{\l}ociniczak, A weighted finite difference
method for subdiffusive Black-Scholes model, {\em Comput. Math. Appl.}, 80(5) (2020):
653-670.

\bibitem{Zhuang16}
H. Zhang, F. Liu, I. Turner, Q. Yang, Numerical solution of the time fractional
Black-Scholes model governing European options, {\em Comput. Math. Appl.}, 71(9)
(2016): 1772-1783.

\bibitem{Tian20}
Z. Tian, S. Zhai, Z. Weng, Compact finite difference schemes of the
time fractional Black-Scholes model, {\em J. Appl. Anal. Comput.},
10(3) (2020): 904-919.

\bibitem{Dimitrov}
Y. M. Dimitrov, L.G. Vulkov, Three-point compact finite difference scheme on
non-uniform meshes for the time-fractional Black-Scholes equation, {\em AIP
Conf. Proc.}, 1690(1) (2015): 040022. DOI: \href{https://doi.org/10.1063/1.4936729}{10.1063/1.4936729}.


\bibitem{Kazmi22}
K. Kazmi, A second order numerical method for the time-fractional
Black-Scholes European option pricing model, {\em J. Comput. Appl. Math.},
418 (2022): 114647.


\bibitem{Roul20}
P. Roul, A high accuracy numerical method and its convergence for time-fractional
Black-Scholes equation governing European options, {\em Appl. Numer. Math.}, 151
(2020): 472-493.

\bibitem{Abdi2022}
N. Abdi, H. Aminikhah, A. H. Refahi Sheikhani, High-order compact finite difference
schemes for the time-fractional Black-Scholes model governing European options, {\em
Chaos Solit. Fractals}, 162 (2022): 112423.

\bibitem{Koleva17}
M. N. Koleva, L. G. Vulkov, Numerical solution of time-fractional Black-Scholes
equation, {\em Comput. Appl. Math.}, 36 (2017): 1699-1715.

\bibitem{Sarboland}
M. Sarboland, A. Aminataei, On the numerical solution of time fractional
Black-Scholes equation, {\em Int. J. Comput. Math.}, 99(9) (2022): 1736-1753.

\bibitem{An2021}
X. An, F. Liu, M. Zheng, V. V. Anh, I. W. Turner, A space-time spectral method
for time-fractional Black-Scholes equation, {\em Appl. Numer. Math.}, 165 (2021):
152-166.

\bibitem{Akram22}
T. Akram, M. Abbas, K. M. Abualnaja, A. Iqbal, A. Majeed, An efficient numerical
technique based on the extended cubic B-spline functions for solving time fractional
Black-Scholes model, {\em Eng. Comput.}, 38 (2022): 1705-1716.

\bibitem{Rezaei21}
M. Rezaei, A. R. Yazdanian, A. Ashrafi, S. M. Mahmoudi, Numerical pricing based on fractional
Black-Scholes equation with time-dependent parameters under the CEV model: Double barrier
options, {\em Comput. Math. Appl.}, 90 (2021): 104-111.

\bibitem{Stynes17}
M. Stynes, E. O'Riordan, J. L. Gracia, Error analysis of a finite difference
method on graded meshes for a time-fractional diffusion equation, {\em SIAM J.
Numer. Anal.}, 55(2) (2017): 1057-1079.

\bibitem{Gracia18}
J. L. Gracia, E. O'Riordan, M. Stynes, Convergence in positive time for a finite
difference method applied to a fractional convection-diffusion problem, {\em
Comput. Methods Appl. Math.}, 18(1) (2018): 33-42.

\bibitem{Liao2018}
H.-L. Liao, D. Li, J. Zhang, Sharp error estimate of the nonuniform $L1$
formula for linear reaction-subdiffusion equations, {\em SIAM J. Numer.
Anal.}, 56(2) (2018): 1112-1133.

\bibitem{Shen2018}
J.-Y. Shen, Z.-Z. Sun, R. Du, Fast finite difference schemes for the
time-fractional diffusion equation with a weak singularity at the
initial time, {\em East Asia J. Appl. Math.}, 8(4) (2018), pp. 834-858.

\bibitem{Cen2018}
Z. Cen, J. Huang, A. Xu, A. Le, Numerical approximation of a time-fractional
Black-Scholes equation, {\em Comput. Math. Appl.}, 75(8) (2018): 2874-2887.

\bibitem{Luchko09}
M. N. Koleva, L. G. Vulkov, Fast positivity preserving numerical method for
time-fractional regime-switching option pricing problem, in {\em Advanced
Computing in Industrial Mathematics. BGSIAM 2020} (I. Georgiev, H. Kostadinov,
E. Lilkova, eds.), Studies in Computational Intelligence, Vol. 1076. Springer,
Cham (2023): 88-99. DOI: \href{https://doi.org/10.1007/978-3-031-20951-2_9}{10.1007/978-3-031-20951-2\_9}.

\bibitem{She2021}
M. She, L. Li, R. Tang, D. Li, A novel numerical scheme for a time fractional
Black-Scholes equation, {\em J. Appl. Math. Comput.}, 66 (2021): 853-870.

\bibitem{Kerui21}
K. Song, P. Lyu, A high-order and fast scheme with variable time steps for the
time-fractional Black-Scholes equation, {\em Math. Methods Appl. Sci.}, 46(2)
(2023): 1990-2011.

\bibitem{Jiang17}
S. Jiang, J. Zhang, Q. Zhang, Z. Zhang, Fast evaluation of the Caputo fractional
derivative and its applications to fractional diffusion equations, {\em Commun.
Comput. Phys.}, 21(3) (2017): 650-678.

\bibitem{Liao19}
H.-L. Liao, W. McLean, J. Zhang, A discrete Gronwall inequality with
applications to numerical schemes for subdiffusion problems, {\em SIAM
J. Numer. Anal.}, 57(1) (2019): 218-237.

\bibitem{Wang2016}
G. Krzy\.{z}anowski, M. Magdziarz, A tempered subdiffusive Black-Scholes model,
{\em arXiv preprint}, \href{https://arxiv.org/abs/2103.13679}{arXiv:2103.13679},
22 May 2022, 27 pages.

\bibitem{Meerschaert}
M. M. Meerschaert, A. Sikorskii, {\em Stochastic Models for Fractional Calculus}. De Gruyter Studies in Mathematics, Vol. 43, Walter de Gruyter, Berlin/Boston (2012).

\bibitem{Cartea2007}
\'{A}. Cartea, D. del-Castillo-Negrete, Fluid limit of the continuous-time random walk with general L\`{e}vy jump distribution functions, {\em Phys. Rev. E.}, 76(4) (2017): 041105.

\bibitem{Zhao20}
L. Zhao, C. Li, F. Zhao, Efficient diference schemes for the Caputo-tempered
fractional difusion equations based on polynomial interpolation, {\em Commun.
Appl. Math. Comput.}, 3(1) (2021): 1-40. DOI:
\href{https://doi.org/10.1007/s42967-020-00067-5}{10.1007/s42967-020-00067-5}.

\bibitem{Meersc08}
M. M. Meerschaert, Y. Zhang, B. Baeumer, Tempered anomalous diffusion in heterogeneous
systems, {\em Geophys. Res. Lett.}, 35(17) (2008): L17403. DOI: \href{https://doi.org/10.1029/2008GL034899}{10.1029/2008GL034899}.


\bibitem{Morgado}
M. L. Morgado, L. F. Morgado, Modeling transient currents in time-of-flight
experiments with tempered time-fractional diffusion equations, {\em Progr.
Fract. Differ. Appl.}, 6(1) (2020): 43-53. DOI:
\href{https://doi.org/10.18576/pfda/060105}{10.18576/pfda/060105}.

%\bibitem{Gao2015}
%G. H. Gao, H. W. Sun, Three-point combined compact difference schemes for
%time-fractional advection-diffusion equations with smooth solutions, {\em
%J. Comput. Phys.}, 298 (2015): 520-538.

%\bibitem{Huang05}
%F. Huang, F. Liu, The time fractional diffusion equation and the
%advection-dispersion equation, {\em ANZIAM J.}, 46(3) (2005): 317-330.

%\bibitem{Wei2013}
%L. Wei, X. Zhang, Y. He, Analysis of a local discontinuous Galerkin method for
%time-fractional advection-diffusion equations, {\em Int. J. Numer. Methods
%Heat Fluid Flow}, 23(4) (2013): 634-648.

%\bibitem{Zhang18x}
%J. Zhang, X. Zhang, B. Yang, An approximation scheme for the time fractional
%convection-diffusion equation, {\em Appl. Math. Comput.}, 335 (2018): 305-312.

\bibitem{Wang2015}
Y.-M. Wang, A compact finite difference method for solving a class of time
fractional convection-subdiffusion equations, {\em BIT}, 55(4) (2015):
1187-1217.


\bibitem{Yang16}
X. Yang, L. Wu, S. Sun, X. Zhang, A universal difference method for
time-space fractional Black-Scholes equation, {\em Adv. Differ. Equ.},
2016(1) (2016): 71. DOI: \href{https://doi.org/10.1186/s13662-016-0792-8}{10.1186/s13662-016-0792-8}.

\bibitem{Guo2019}
L. Guo, F. Zeng, I. W. Turner, K. Burrage, G. E. Karniadakis, Efficient
multistep methods for tempered fractional calculus: algorithms and
simulations, {\em SIAM J. Sci. Comput.}, 41(4) (2019): A2510-A2535.

\bibitem{Wang2022}
C. Wang, W. Deng, X. Tang, A sharp $\alpha$-robust L1 scheme on graded meshes for
two-dimensional time tempered fractional Fokker-Planck equation, {\em arXiv preprint},
\href{https://arxiv.org/abs/2205.15837v1}{arXiv:2205.15837v1}, May 31, 2022, 25 pages.

\bibitem{Li2015}
C. Li, W. Deng, L. Zhao, Well-posedness and numerical algorithm for the tempered
fractional ordinary differential equations, {\em Discrete Contin. Dyn. Syst. Ser.
B}, 24(4) (2015): 1989-2015.

\bibitem{Ishteva05}
M. Ishteva, {\em Properties and Applications of the Caputo Fractional Operator}. Master
thesis, Department of Mathematics, University of Karlsruhe, Karlsruhe (2005).

\bibitem{Sakamoto}
K. Sakamoto, M. Yamamoto, Initial value/boundary value problems for fractional
diffusion-wave equations and applications to some inverse problems, {\em J. Math.
Anal. Appl.}, 382(1) (201): 426-447.

\end{thebibliography}

\end{document}
