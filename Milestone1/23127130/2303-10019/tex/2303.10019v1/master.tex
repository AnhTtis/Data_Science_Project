% !TEX root = master.tex
% elsearticle docs: https://www.elsevier.com/__data/assets/pdf_file/0008/56843/elsdoc-1.pdf
\documentclass[review, nopreprintline, authoryear]{elsarticle}

\usepackage[a4paper, left=4cm, right=4cm, top=4cm, bottom=4cm]{geometry}

\usepackage[english]{babel}
\usepackage[utf8]{inputenc}
\usepackage[english]{babel}

% Pakete für mathematische Symbole
% Pakete für mathematische Symbole
\usepackage{amsmath}
\usepackage{amstext}
\usepackage{amssymb}
\usepackage{latexsym}
\usepackage{dsfont}
\usepackage{mathrsfs}
% \usepackage{setspace}
% \usepackage{ulem}
\usepackage{bbold}

\usepackage{hyperref}
\hypersetup{
    colorlinks=true,
    unicode=true
}

% See for problem with hyperref and \corref command
% https://tex.stackexchange.com/questions/504814/package-hyperref-warning-token-not-allowed-in-a-pdf-string-pdfdocencoding/559193
\pdfstringdefDisableCommands{%
  \def\corref#1{<#1>}%
}

\usepackage{graphicx}
\usepackage{float}
\usepackage{caption, booktabs}
\usepackage{graphics}
\usepackage{subcaption}
\usepackage{lipsum}
\usepackage{amsmath,amssymb,amsfonts,amsthm}
\usepackage[table, dvipsnames]{xcolor}
\usepackage{mathrsfs}
\usepackage{soul}

% \setlength{\parindent}{50mm} % ver\"andert den Einzug
% \usepackage[pdftex]{graphicx} %notwendig, wenn Grafiken geladen werden sollen
\usepackage{url} % zum korrekten Einf\"ugen von URL mit dem \url Befehl, das Package hyperref erzeugt dazu noch Links

\usepackage{bm}
\usepackage{bbm}
\usepackage{array}

% \usepackage{listings}
% \lstset{language=R,showstringspaces=false,breaklines=true, numbers=left,
% numbersep=2pt, tabsize=2, numberstyle=\tiny\color{gray}} 
\usepackage[noend, linesnumbered,ruled,vlined]{algorithm2e}

% Helper to align equations in algorithm environments
\newcommand{\ali}[2]{\makebox[#1][l]{#2}}
% \usepackage[all]{xy}
%\usepackage{url}
%\usepackage{color}
%\usepackage[usenames,dvipsnames]{color}
\usepackage{tikz}
\newcommand{\mx}[1]{\mathbf{\bm{#1}}} % Matrix command
\newcommand{\vc}[1]{\mathbf{\bm{#1}}} % Vector command


\usepackage{attachfile}






\definecolor{b}{rgb}{0,0,.8}	%%omega-blau
\definecolor{g}{rgb}{0,.6,0}	%%Tau-grün
\definecolor{n}{rgb}{0,0,0}	%%normal-schwarz
\definecolor{h}{rgb}{0.4,0.2,0.2}	%%hint
\definecolor{v}{rgb}{0.2,0.6,0}

\newtheorem{beispiel}{Beispiel}
\newtheorem{gbsp}{Gegenbeispiel}
\newtheorem{definition}{Definition}
\newtheorem{lemma}{Lemma}
\newtheorem{proposition}{Proposition}
\newtheorem{theorem}{Theorem}
\newtheorem{satz}{Satz}
\newtheorem{bem}{Bemerkung}
\newtheorem{bew}{Beweis}
\newtheorem{korollar}{Korollar}


%mathbb alphabet
\newcommand{\A}{{\mathbb A}}
\newcommand{\B}{{\mathbb B}}
%\newcommand{\C}{{\mathbb C}}
%\newcommand{\D}{{\mathbb D}}
\newcommand{\E}{{\mathbb E}}
\newcommand{\F}{{\mathbb F}}
%\newcommand{\G}{{\mathbb G}}
\renewcommand{\H}{{\mathbb H}}
\newcommand{\I}{{\mathbb I}}
\newcommand{\J}{{\mathbb J}}
\newcommand{\K}{{\mathbb K}}
\renewcommand{\L}{{\mathbb L}}
\newcommand{\M}{{\mathbb M}}
\newcommand{\N}{{\mathbb N}}
\renewcommand{\O}{{\mathbb O}}
%\renewcommand{\P}{{\mathbb P}}
\newcommand{\Q}{{\mathbb Q}}
\newcommand{\R}{{\mathbb R}}
%\renewcommand{\S}{{\mathbb S}}
\newcommand{\T}{{\mathbb T}}
%\newcommand{\U}{{\mathbb U}}
\newcommand{\V}{{\mathbb V}}
\newcommand{\W}{{\mathbb W}}
\newcommand{\X}{{\mathbb X}}
\newcommand{\Y}{{\mathbb Y}}
\newcommand{\Z}{{\mathbb Z}}

%mathcal alphabet
\renewcommand{\AA}{{\mathcal{A}}}
\newcommand{\BB}{{\mathcal{B}}}
\newcommand{\CC}{{\mathcal{C}}}
\newcommand{\DD}{{\mathcal{D}}}
\newcommand{\EE}{{\mathcal{E}}}
\newcommand{\FF}{{\mathcal{F}}}
\newcommand{\CG}{{\mathcal{G}}}
\newcommand{\HH}{{\mathcal{H}}}
\newcommand{\II}{{\mathcal{I}}}
\newcommand{\JJ}{{\mathcal{J}}}
\newcommand{\KK}{{\mathcal{K}}}
\newcommand{\LL}{{\mathcal{L}}}
\newcommand{\MM}{{\mathcal{M}}}
\newcommand{\NN}{{\mathcal{N}}}
\newcommand{\OO}{{\mathcal{O}}}
\newcommand{\PP}{{\mathcal{P}}}
\newcommand{\QQ}{{\mathcal{Q}}}
\newcommand{\RR}{{\mathcal{R}}}
\renewcommand{\SS}{{\mathcal{S}}}
\newcommand{\TT}{{\mathcal{T}}}
\newcommand{\UU}{{\mathcal{U}}}
\newcommand{\VV}{{\mathcal{V}}}
\newcommand{\WW}{{\mathcal{W}}}
\newcommand{\XX}{{\mathcal{X}}}
\newcommand{\YY}{{\mathcal{Y}}}
\newcommand{\ZZ}{{\mathcal{Z}}}

%boldsymbols
\newcommand{\bsa}{\boldsymbol a}
\newcommand{\bsb}{\boldsymbol b}
\newcommand{\bsc}{\boldsymbol c}
\newcommand{\bsd}{\boldsymbol d}
\newcommand{\bse}{\boldsymbol e}
\newcommand{\bsf}{\boldsymbol f}
\newcommand{\bsg}{\boldsymbol g}
\newcommand{\bsh}{\boldsymbol h}
\newcommand{\bsi}{\boldsymbol i}
\newcommand{\bsj}{\boldsymbol j}
\newcommand{\bsk}{\boldsymbol k}
\newcommand{\bsl}{\boldsymbol l}
\newcommand{\bsm}{\boldsymbol m}
\newcommand{\bsn}{\boldsymbol n}
\newcommand{\bso}{\boldsymbol o}
\newcommand{\bsp}{\boldsymbol p}
\newcommand{\bsq}{\boldsymbol q}
\newcommand{\bsr}{\boldsymbol r}
\newcommand{\bss}{\boldsymbol s}
\newcommand{\bst}{\boldsymbol t}
\newcommand{\bsu}{\boldsymbol u}
\newcommand{\bsv}{\boldsymbol v}
\newcommand{\bsw}{\boldsymbol w}
\newcommand{\bsx}{\boldsymbol x}
\newcommand{\bsy}{\boldsymbol y}
\newcommand{\bsz}{\boldsymbol z}
\newcommand{\bsA}{\boldsymbol A}
\newcommand{\bsB}{\boldsymbol B}
\newcommand{\bsC}{\boldsymbol C}
\newcommand{\bsD}{\boldsymbol D}
\newcommand{\bsE}{\boldsymbol E}
\newcommand{\bsF}{\boldsymbol F}
\newcommand{\bsG}{\boldsymbol G}
\newcommand{\bsH}{\boldsymbol H}
\newcommand{\bsI}{\boldsymbol I}
\newcommand{\bsJ}{\boldsymbol J}
\newcommand{\bsK}{\boldsymbol K}
\newcommand{\bsL}{\boldsymbol L}
\newcommand{\bsM}{\boldsymbol M}
\newcommand{\bsN}{\boldsymbol N}
\newcommand{\bsO}{\boldsymbol O}
\newcommand{\bsP}{\boldsymbol P}
\newcommand{\bsQ}{\boldsymbol Q}
\newcommand{\bsR}{\boldsymbol R}
\newcommand{\bsS}{\boldsymbol S}
\newcommand{\bsT}{\boldsymbol T}
\newcommand{\bsU}{\boldsymbol U}
\newcommand{\bsV}{\boldsymbol V}
\newcommand{\bsW}{\boldsymbol W}
\newcommand{\bsX}{\boldsymbol X}
\newcommand{\bsY}{\boldsymbol Y}
\newcommand{\bsZ}{\boldsymbol Z}
\newcommand{\bsone}{\boldsymbol 1}
\newcommand{\bstwo}{\boldsymbol 2}
\newcommand{\bsthree}{\boldsymbol 3}
\newcommand{\bsfour}{\boldsymbol 4}
\newcommand{\bsfive}{\boldsymbol 5}
\newcommand{\bssix}{\boldsymbol 6}
\newcommand{\bsseven}{\boldsymbol 7}
\newcommand{\bseight}{\boldsymbol 8}
\newcommand{\bsnine}{\boldsymbol 9}
\newcommand{\bszero}{\boldsymbol 0}
\newcommand{\bsnought}{\boldsymbol 0}
\newcommand{\bsnaught}{\boldsymbol 0}
\newcommand{\bsnull}{\boldsymbol 0}
\newcommand{\bsnil}{\boldsymbol 0}

%greeks
\newcommand{\bsalpha}{\boldsymbol \alpha}
\newcommand{\bsbeta}{\boldsymbol \beta}
\newcommand{\bslambda}{\boldsymbol \lambda}
\newcommand{\bsmu}{\boldsymbol \mu}
\newcommand{\bseps}{\boldsymbol \varepsilon}
\newcommand{\bstheta}{\boldsymbol \theta}
\newcommand{\bssigma}{\boldsymbol \sigma}
\newcommand{\bszeta}{\boldsymbol \zeta}
\newcommand{\bsvarphi}{\boldsymbol \varphi}

\newcommand{\bsGamma}{\boldsymbol \Gamma}
\newcommand{\bsPhi}{\boldsymbol \Phi}
\newcommand{\bsSigma}{\boldsymbol \Sigma}
\newcommand{\bsPi}{\boldsymbol \Pi}


%easier greeks
%capital
\newcommand{\Om}{{\Omega}}
\newcommand{\Ome}{{\Omega}}
\newcommand{\Omeg}{{\Omega}}
%small
\newcommand{\eps}{{\varepsilon}}
\newcommand{\epsi}{{\epsilon}}
\newcommand{\kap}{{\kappa}}

% gen math
\DeclareMathOperator*{\klimsup}{K-lim\,sup}
\DeclareMathOperator*{\kliminf}{K-lim\,inf}
\DeclareMathOperator*{\klim}{K-lim}
\DeclareMathOperator*{\argmin}{arg\,min}
\DeclareMathOperator*{\argmax}{arg\,max}
\DeclareMathOperator*{\epi}{epi}
\DeclareMathOperator{\lev}{lev}
\DeclareMathOperator{\gr}{graph}
\DeclareMathOperator{\conv}{conv}	%convex hull
\DeclareMathOperator{\inter}{int}		%interior
\DeclareMathOperator{\cl}{cl}
\DeclareMathOperator{\bdy}{bdy}
\DeclareMathOperator{\relint}{relint}	%relative interior
\DeclareMathOperator{\relbdy}{relbdy}	%relative boundary
\DeclareMathOperator{\dist}{dist}
\DeclareMathOperator{\LSC}{LSC}
\DeclareMathOperator{\EPI}{EPI}
\DeclareMathOperator{\id}{id}
\DeclareMathOperator{\ind}{\boldsymbol 1 }
\DeclareMathOperator{\gdc}{gdc}
\DeclareMathOperator{\modulo}{mod}
\DeclareMathOperator{\logit}{logit}
\DeclareMathOperator{\spt}{spt}
\DeclareMathOperator{\Tr}{Tr}

%stochastic commands
\DeclareMathOperator{\var}{\V ar}
\DeclareMathOperator{\cov}{\C ov}
\DeclareMathOperator{\cor}{\C or}
\DeclareMathOperator{\corr}{\C orr}
\DeclareMathOperator{\as}{\text{a.s.}}
\newcommand{\MC}{\text{MC}} 
\newcommand{\MLE}{\text{MLE}} 
\newcommand{\MCMLE}{\text{MC-MLE}} 

%general short cuts
\newcommand{\ov}\overline
\newcommand{\what}{\widehat}
\newcommand{\wtilde}{\widetilde}
\newcommand{\ow}{\text{ otherwise}}
\newcommand{\rig}\right
\newcommand{\lef}\left
\newcommand{\nf}\normalfont

%document specific
\newcommand{\bsPP}{\boldsymbol{\mathcal{P}}}
\newcommand{\bsDD}{\boldsymbol{\mathcal{D}}}
\DeclareMathOperator*{\bsQL}{\textbf{QL}}
\DeclareMathOperator*{\QL}{QL}
\DeclareMathOperator*{\CRPS}{CRPS}
\DeclareMathOperator*{\sign}{sign}
\newcommand{\bseta}{\boldsymbol \eta}
\DeclareMathOperator*{\softmax}{\text{SoftMax}}
\DeclareMathOperator*{\bssoftmax}{\textbf{SoftMax}}
\newcommand{\bspsi}{\boldsymbol \psi}
\newcommand{\bs}{\boldsymbol}
% Number of inner knots
\newcommand{\innerknots}{J}
% Oder of the B-Spline basis
\newcommand{\bsorder}{o}

% Hours
\newcommand{\D}{D}
\renewcommand{\d}{d}
% Reduced hours
\newcommand{\Dr}{\widetilde{D}} 

% Quantiles
\renewcommand{\P}{P}
\newcommand{\p}{p}
% Reduced quantiles
\renewcommand{\Pr}{\widetilde{P}} 

% Subscripts for the quantiles
\newcommand{\prob}{\text{pr}}

% Subscripts for the covariates
\newcommand{\mult}{\text{mv}}


\newcommand{\MSP}{\text{MSP}} 
\newcommand{\MVP}{\text{MVP}} 

\newcommand{\MAE}{\text{MAE}} 
\newcommand{\MMAE}{\text{MMAE}} 
\newcommand{\bsMAE}{\text{\textbf{MAE}}} 
\newcommand{\bsMMAE}{\text{\textbf{MMAE}}} 
\newcommand{\RMSE}{\text{RMSE}} 
\newcommand{\MRMSE}{\text{MRMSE}} 
\newcommand{\bsRMSE}{\text{\textbf{RMSE}}} 
\newcommand{\bsMRMSE}{\text{\textbf{MRMSE}}} 

\newcommand{\EXAA}{\text{EXAA}} 

\newcommand{\bsYY}{\boldsymbol \YY}
\newcommand{\DDelta}{\boldsymbol \varDelta}

% Typeset algorithms
% \usepackage{algorithm, algorithmicx, algpseudocode}

% Inline comments
\definecolor{darkgray}{HTML}{6e6e6e}
% \newcommand{\COMMENT}[1]{\State \textcolor{darkgray}{// #1}}

% Align comments in algorithm
% \renewcommand{\Comment}[2][.5\linewidth]{%
%   \leavevmode\hfill\makebox[#1][l]{//~#2}}
% \algnewcommand\algorithmicto{\textbf{to}}
% \algnewcommand\RETURN{\State \textbf{return} }

\definecolor{dcyan}{rgb}{0,0.5,.5}
\newcommand{\FZ}[1]{\color{dcyan} #1 \color{black}}
\definecolor{dgreen}{rgb}{0,0.35,0}
 \newcommand{\JB}[1]{\color{dgreen}{\textbf{ #1 }} \color{black}} 
 

\newcommand{\MS}{\text{MS}} 
\newcommand{\CS}{\text{CS}} 
\newcommand{\MCS}{\text{MCS}} 
\newcommand{\ES}{\text{ES}} 
\newcommand{\VS}{\text{VS}} 
\newcommand{\CES}{\text{CES}} 
\newcommand{\CVS}{\text{CVS}} 
 
\usepackage{tikz}
\usepackage{lineno}
\modulolinenumbers[0]

\begin{document}

% Comment out to disable line numbers
% \linenumbers

\begin{frontmatter}

  \journal{International Journal of Forecasting}

  \title{Multivariate Probabilistic CRPS Learning with an Application to Day-Ahead Electricity Prices}

  %% Group authors per affiliation:
  \author[1]{Jonathan Berrisch\corref{cor1}}
  \ead{jonathan.berrisch@uni-due.de}
  \cortext[cor1]{Corresponding author}

  \author[1]{Florian Ziel}
  \ead{florian.ziel@uni-due.de}

  \address[1]{Chair of Environmental Economics, esp. Economics of Renewable Energy \\ University of Duisburg-Essen \\
    Germany}

  \begin{abstract}
    This paper presents a new method for combining (or aggregating or ensembling) multivariate probabilistic forecasts, taking into account dependencies between quantiles and covariates through a smoothing procedure that allows for online learning. Two smoothing methods are discussed: dimensionality reduction using Basis matrices and penalized smoothing. The new online learning algorithm generalizes the standard CRPS learning framework into multivariate dimensions. It is based on Bernstein Online Aggregation (BOA) and yields optimal asymptotic learning properties. We provide an in-depth discussion on possible extensions of the algorithm and several nested cases related to the existing literature on online forecast combination. The methodology is applied to the forecasting of day-ahead electricity prices, which are 24-dimensional distributional forecasts, and the proposed method yields significant improvements over uniform combination in terms of continuous ranked probability score (CRPS). We discuss the temporal evolution of the weights and hyperparameters and present the results of reduced versions of the preferred model. A fast C++ implementation of all discussed methods is provided in the R-Package profoc.
  \end{abstract}
  \begin{keyword}
    Combination; Aggregation; Ensembling; Online; Multivariate; Probabilistic; Forecasting; Quantile; Time Series; Distribution; Density; Prediction; Splines
    \JEL C15; C18; C21; C22;  C53; C58; G17; Q47
  \end{keyword}
\end{frontmatter}

\newpage
\section{Introduction}~\label{Introduction}

Forecast combination (sometimes referred to as expert aggregation or ensembling) has gained a lot of traction in recent times. We know from theory that combination methods work well to combine different but well performing model classes~\cite{cesa2006prediction}. As ~\cite{gaillard2016additive} pointed out, it is always recommended to use different classes of models, e.g., regression and time series type models, neural network models, decision tree learning models, and other machine learning and artificial intelligence methods.

This paper proposes a novel online updating scheme for combining multivariate probabilistic forecasts, taking into account multivariate dependencies between quantiles and covariates through a simple but flexible smoothing procedure. Therefore, we assume a simple metric or spatial structure on the multivariate dimension, which we have e.g. when forecasting a univariate time series several steps ahead or predicting one-dimensional spatial data.

Online learning algorithms are particularly attractive for forecasting where frequent short-term forecasts are essential for the application domain (e.g. energy, weather, finance, retail). The proposed algorithm generalizes the probabilistic CRPS learning framework presented in \cite{berrisch2021crps}. It is based on exponential weighted averaging (EWA) and yields optimal asymptotic convergence rates with respect to the best individual forecast and the best convex combination of all forecasts \citep{wintenberger2017optimal}.

In forecasting applications, there is already considerable research into combination methods. \citet{bordignon2013combining, nowotarski2014empirical, avci2018managing} combine point-forecasts using various batch methods.~\cite{marcjasz2020probabilistic, Serafin2019averaging} apply batch methods to probabilistic forecast combination. Some authors also applied online learning algorithms for point forecasting \citep{nowotarski2016improving} and probabilistic forecasting \citep{gaillard2016additive, gonzalez2021new}. The work above focuses on developing distinct forecasting models and on combination methods. \citet{gaillard2015forecasting} discusses how model development can be optimized in the framework of aggregation of experts.

In electricity price forecasting, dynamic aggregation techniques, where the combination weights are adjusted based on past performance, tend to perform better than simple constant weight techniques \citep{gaillard2015forecasting, marcjasz2018selection, maciejowska2020pca}. However, they consider multivariate updating schemes that use the same weight for all time series. Most other work in energy forecasting considers all time series to be independent and therefore combines forecasts separately \citep{bordignon2013combining, nowotarski2016improving}.
Both approaches do not take into account possible dependencies between the multivariate space. Consequently, we can expect potential improvements by exploiting this metric structure of electricity prices by considering updating schemes that assign different weights to all neighbouring price forecasts of the day, while also considering possible dependencies between time series. Of course, the same logic applies to other areas of application.

The contributions of this manuscript are manifold:
\begin{itemize}
  \item [i)] We generalize batch and online CRPS learning to multivariate settings.
  \item[ii)] We show how the metric or spatial structure of multivariate data can be considered using two smoothing methods.
  \item[iii)] We discuss three possible strategies for optimizing hyperparameters in online learning settings.
  \item[iv)] We provide a fast open-source implementation of the proposed combination procedure.
  \item[v)] We empirically apply the proposed methods to multivariate probabilistic day-ahead electricity price forecasts.
\end{itemize}

The remainder of this paper is structured as follows. Section~\ref{mv_crps_learn} discusses the general multivariate probabilistic combination setting and discusses CRPS learning using quantile regression. Section~\ref{theor} presents the proposed multivariate generalization of online CRPS learning and summarizes its asymptotic properties. Additionally, we discuss possible extensions of the proposed method. Those extensions to the core algorithm add hyperparameters that have to be specified. Therefore, we elaborate on two possible strategies for hyperparameter tuning in Section~\ref{sec_hyperpar}. Section~\ref{application} continues with an empirical application of the proposed algorithm. We apply the methodology to multivariate probabilistic forecasts of Day-Ahead power prices. We discuss the data, elaborate on the specific algorithms we consider, and present a detailed analysis of the obtained results. Section~\ref{conclusion} discusses the limitations of the discussed approach, presents potential enhancements, and concludes.

\section{Multivariate CRPS Learning}\label{mv_crps_learn}

\subsection{The combination setting}\label{setting}

In this paper, we consider the combination of multivariate probabilistic forecasts. In particular, we consider a setting where the forecasts are given as quantiles of a multivariate distribution. \citet{berrisch2021crps} show that pointwise forecast combinations potentially outperform standard methods where weights are constant over all quantiles of the distribution. We apply this idea to a multivariate setting by computing weights not only depending on the quantile but also on the covariates. First, we discuss batch learning methods and propose a dimension reduction technique that bridges the gap between flexible pointwise and robust constant procedures. Afterward, we show how the proposed online learning algorithm of \citet{berrisch2021crps} can be extended for combining multivariate probabilistic forecasts.

Let $\what{\bsF}_{t,k} = (\what{F}_{t,1},\ldots, \what{F}_{t,K})$ be a set of $K$ multivariate distributions resp. the set of experts that we want to combine. We consider the combination across quantiles (also known as horizontal aggregation):
\begin{equation}
  \wtilde{F}_{t}^{-1} = \sum_{k=1}^K w_{t,k} \what{F}^{-1}_{t,k}
  \label{eq_comb_crps_cw_quant}
\end{equation}
We evaluate the performance using the cumulative CRPS over all covariates. Therefore, the weights shall be chosen to minimize the cumulative CRPS of all covariates. We  can approximate the CRPS by the sum over Quantile Losses ($\QL$)
\begin{align}
  \CRPS(F, y) = \int_{{\R}} {(F(x) - \mathbb{1}\{ x > y \})}^2 dx \approx \frac{2}{P} \sum_{p \in \bsPP}  {\QL}_{p}(F^{-1}(p), y)
  \label{eq_crps_approx}
\end{align}
for an equidistant dense grid $\bsPP = ( p_1,\ldots, p_P )$ with $p_i<p_{i+1}$ and $p_{i+1}-p_i = h$ for all $p$. Clearly, $P\to \infty$ induces $h \to0$, $p_1\to0$, $p_P\to1$ and the approximation converges to the CRPS~\citep{gneiting2011making, gneiting2011quantiles}. This relationship enables us to compute pointwise weights based on pinball scores. We can extend this idea by optimizing weights not only depending on the quantile $\p$ but also on the covariate $\d$:
\begin{equation}
  \wtilde{F}_{t, \d}^{-1}(p) = \sum_{k=1}^K w_{t,k}(\d,p) \what{F}^{-1}_{t,k, \d}(p)
  \label{eq_forecast_F_def_d}
\end{equation}
We are interested in setting $w_{t,k}$ such that the CRPS of $\wtilde{F}_{t,d}$ is minimized.


\subsection{CRPS learning using quantile regression}\label{cprs_learn_using_quant_reg}

Pointwise CRPS learning has the potential to outperform standard CRPS learning methods. However, the best pointwise weights in~\eqref{eq_forecast_F_def_d} have to be estimated. Theoretically, a pointwise approach has to be applied to all probabilities $\p\in(0,1)$ and all marginals $\bsDD = (1, 2, \ldots, \D)$ such that the bivariate weight function $\bsw_{t,k}$ can be specified.
However, we can never evaluate infinitely many values for $\p$. On the same page, the computation may be infeasible if $\D$ is very large. Therefore, we must consider some finite-dimensional representation for the weight functions $\bsw_{t,k}$. A suitable option is representing the weight functions $\bsw_{t,k}$ using a finite-dimensional representation using splines. Bivariate splines are a suitable option in this scenario. We can express them as follows:
\begin{equation}
  f(X_{t,j_1}, X_{t,j_2})=\sum_{l=1}^L \beta_l \bs \varphi_l(X_{t,j_1}, X_{t,j_2}).
  \label{bivariate_basis_functions}
\end{equation}
This is essentially the same as univariate splines with $L$-dimensional parameter vector $(\beta_1,\ldots,\beta_L)'$. However, the support of $f$ is 2-dimensional. Thus, we need much more basis functions $L$ to have a suitable description of $f$. A popular way to describe the bivariate basis function $\bs \varphi_l$ in~\eqref{bivariate_basis_functions} is to assume a tensor structure~\citep{mclean2014functional, wood2017gen}. In the bivariate case, the spline function is a product of two univariate ones. In addition, $\varphi_{1,l}$ and $\varphi_{2,l}$ are usually chosen such that $\varphi_{1,l_1}$ interacts with each of the considered basis functions $\varphi_{2,l_2}$. This, allows to renumerate the problem such that $l=(l_1,l_2)$, and yields
\begin{equation}
  \bs \varphi_{l_1,l_2}(x_1,x_2) = \varphi_{1,l_1}(x_1)\varphi_{2,l_2}(x_2).
  \label{eq_gam_bivariate_basis_decompositon2}
\end{equation}

This can be used to express the bivariate weight function as a product of the $\Pr \times \Dr$ parameter matrix $\bsbeta_{t,k}$ and the bivariate basis represented by $\boldsymbol{\varphi}^{\mult}$ and $\boldsymbol{\varphi}^{\prob}$:
\begin{equation}
  \bs w_{t,k} = \sum_{j=1}^{\Dr} \sum_{l=1}^{\Pr} \beta_{t,j,l,k} \varphi^{\mult}_{j} \varphi^{\prob}_{l} = \bs \varphi^{\mult} \bs \beta_{t,k} {\bs\varphi^{\prob}}'.
\end{equation}
Given $T$ historic forecasts $\what{F}^{-1}_{t,\d,k}(\p)$, the cooresponding realizations $Y_{t,d}$, and the index of covariates $\bsDD = (1,2, \ldots, D)$ we can estimate the $\Dr \times \Pr \times K$-dimensional parameter tensor $\bsbeta_{t}$ by minimizing the corresponding CRPS using~\eqref{eq_crps_approx}:
\begin{align}
  \bsbeta_{t}^{\bsvarphi\text{-CRPS}}
   & = \argmin_{ \bsbeta \in {\R}^{K\times L}} \sum_{i=t-T+1}^t \sum_{d\in\bsDD}
  \int_0^{1}  {\rho}_p\left( Y_{t,d} - \sum_{k=1}^K \sum_{j=1}^{\Dr} \sum_{l=1}^{\Pr} \beta_{t,j,l,k} \varphi^{\mult}_{d, j} \varphi^{\prob}_{l, \p} \what{F}^{-1}_{t,d,k}(\p) \right) \, d\p .
  \label{eq_qr_basis}
\end{align}
The second line uses the shift-invariance of the quantile loss and quantile regression notation $\rho_p(z) = \QL_p(0,z)  = z(p-\mathbb{1}\{z< 0\})$ \citep{koenker2017handbook}.

Still, computing~\eqref{eq_qr_basis} requires the evaluation of all distribution forecasts. As discussed, this is often not possible in practice. If we restrict the evaluation to a grid of probabilities $\bsPP$ problem~\eqref{eq_qr_basis} simplifies with~\eqref{eq_crps_approx} to
\begin{align}
  \bsbeta_{t}^{\bsvarphi\text{-QR}}
  = \argmin_{ \bsbeta \in {\R}^{K\times L}} \sum_{i=t-T+1}^t \sum_{d \in \bsDD} \sum_{p\in\bsPP} \rho_p \left( Y_{t,d} -  \sum_{k=1}^K \sum_{j=1}^{\Dr} \sum_{l=1}^{\Pr} \beta_{t,j,l,k} \varphi^{\mult}_{d, j} \varphi^{\prob}_{l, \p} \what{F}^{-1}_{t,d,k}(\p) \right).
  \label{eq_qr_basis_p}
\end{align}
In general, quantile regression problems can be solved efficiently using linear programming solvers \citep{koenker2017handbook}. However,~\eqref{eq_qr_basis_p} is not a simple quantile regression problem, but a joint quantile regression~\citep{sangnier2016joint, chun2016graphical}. The parameters $\beta_{t,j,l,k}$ are active for multiple quantiles. Thus, adequate estimation requires solving the optimization problem for $\Dr \times \Pr \times K$ parameters, which can be computationally costly if $\Dr$, $\Pr$, and $K$ are large.

However, if we choose both basis $\bs \varphi^\prob$ and $\bs \varphi^\mult$ so that $\bs \varphi_i^\mult=\mathbb{1}{\{d_i\}}$ on $\bsDD$ for $d_i\in \bsDD=(d_1,\ldots, d_D)$ and $\bs \varphi_i^\prob=\mathbb{1}{\{p_i\}}$ on $\bsPP$ for $p_i\in \bsPP=(p_1,\ldots, p_P)$ then~\eqref{eq_qr_basis_p} can be disentangled into $\Dr \times \Pr$ separate quantile regression problems.
This is
\begin{align}
  \bsw^{\text{QR}}_{t,h}(p)
  = \argmin_{ \bsw \in {\R}^K} \sum_{i=t-T+1}^t \rho_p \left( Y_{i,d} -  \sum_{k=1}^K w_{i, d, k} \what{F}^{-1}_{i,d,k}(p) \right)
  \label{eq_qr}
\end{align}
for $p\in \bsPP$ and $\d \in \bsDD$ where $\what{F}^{-1}_{t,h,k}(p)$ are the experts for the $p$-quantile and the $d$-covariate.

Quantile regression~\eqref{eq_qr} will lead to linear optimality on $\bsDD$ and $\bsPP$, as long as standard regularity conditions required for the quantile regression are satisfied \citep{koenker2001quantile}. However, we might assume further restrictions to reduce the estimation risk, e.g., the solution is a convex combination. \citet{taylor1998combining} discussed many related plausible restrictions for quantile combination concerning bias correction, positivity, and affinity, among others.

A potential issue of pointwise algorithms is \textit{quantile crossing}. This problem occurs if we have $\wtilde{F}_{t,d}^{-1}(p_i) > \wtilde{F}_{t,h}^{-1}(p_j)$ for some $p_i,p_j\in(0,1)$ with $p_i<p_j$. In this case, we recommend rearranging the predictions. It will never reduce the forecasting performance regarding the quantile loss~\citep{chernozhukov2010quantile}.

\section{Multivariate Online CRPS Learning}\label{theor}

Batch-learning approaches, like quantile regression, evaluate the entire history for estimating new combination weights, which is computationally costly.
Therefore, we suggest to use online learning methods instead.

Online learning is often called \textit{prediction under expert advice}. In this context, \textit{experts} refer to the models producing the predictions (or predictive distributions) and the person or model that combines the experts' predictions is called \textit{forecaster}. A key element of online learning methods is (cumulative) regret. It is defined as:
\begin{equation}
  R_{t,k} = \sum_{i = 1}^t r_{t,k} =
  \sum_{i = 1}^t \ell(\wtilde{X}_{i},Y_i) - \ell(\what{X}_{i,k},Y_i)
  \label{eq_regret}
\end{equation}
i.e., the cumulative difference between the loss of the expert's predictions $\what{X}_{t,k}$ and the prediction of the forecaster $\wtilde{X}_{t}$ for a loss function $\ell$. $\wtilde{X}_{t,k}$ might be a predicted quantile $\what{F}^{-1}_{t,k}(p)$ of expert $k$ as discussed in the previous section.
$R_{t,k}$ is called regret because it indicates how much the forecaster regrets not following the experts' advice~\citep{cesa2006prediction}.
With \eqref{eq_regret}, we can formulate %one of the most common weighting schemes in online learning, 
the EWA:
\begin{align}
  w_{t,k}^{\text{EWA}} & = Kw_{0,k} \frac{e^{\eta R_{t,k}} }{\sum_{j = 1}^K e^{\eta R_{t,j}}}
  =
  \frac{e^{-\eta \ell(\what{X}_{t,k},Y_t)} w^{\text{EWA}}_{t-1,k} }{\sum_{j = 1}^K e^{-\eta \ell(\what{X}_{t,j},Y_t)} w^{\text{EWA}}_{t-1,j} }
  \label{eq_ewa_general}
\end{align}
where $K$ refers to the number of experts, $w_{0,k}$ to the initial weights of an expert $k$, and $\eta$ to the learning rate, which defines how fast the weights adjust to changes in the regret~\cite{cesa2006prediction}. We can express this aggregation rule in terms of %cumulative regret (first expression of~\ref{eq_ewa_general}) as well as in terms of 
past weights and the loss suffered by the experts (right-hand side of~\ref{eq_ewa_general}). This highlights that there is no need for evaluating the entire history %(in contrast to batch-learning methods) 
when adjusting weights.

EWA yields optimal convergence rates of $\OO(T)$  towards the best expert for exp-concave loss functions~\citet{cesa2006prediction}. It means that the algorithm's performance (in terms of risk) is asymptotically not worse than the performance of the best expert. A more ambitious property that can also be satisfied is the \textit{convex aggregation property} which ensures that the risk of the algorithm is not worse than the risk of the best convex combination of the experts. To satisfy this property, the \textit{gradient trick} is needed ~\citep{devaine2013forecasting}. % It means replacing the loss by a linearized version of the loss $\ell^{\nabla}(x,y) = \ell'(\wtilde{X},y) x$ where $\ell'$ is the subgradient of $\ell$ in its first coordinate evaluated at forecast combination $\wtilde{X}$\footnote{This linearized version is sometimes also called pseudo-loss function~\citep{devaine2013forecasting}.}. 
For exp-concave losses, this gives optimal convergence rate $\OO(\sqrt{T})$ with respect to the best convex combination of the experts \citep{cesa2006prediction}. This property also holds for losses that satisfy some Bernstein condition, such as the MAE, when algorithms like Bernstein Online Aggregation (BOA) are used. These algorithms use regularized updating techniques to improve convergence and stability properties \citep{wintenberger2017optimal}. % and lets us weaken the exp-concavity condition while preserving fast convergence rates. BOA features multiple time-adaptive learning rates, so the speed of weight adjustment can vary over time and across experts~\citep{cesa2006prediction}.

\citet{berrisch2021crps} adapted BOA to probabilistic problems. The new algorithm is called CRPS learning because it optimizes the CRPS of the target distribution using pointwise optimization on a grid of quantiles. That is, the weights are allowed to vary over time % as usual in online learning, but
and over parts of the distribution. CRPS learning still maintains the fast convergence of BOA. This algorithm for combining $\what{\bsX}_{t}=(\what{X}_{t,1},\ldots, \what{X}_{t,K})$ to $\wtilde{X}_{t}=\bsw_{t-1}'\what{\bsX}_{t}$ can be summarized as follows:
\begin{subequations}
  \begin{align}
    \bsr_{t}   & = {\QL}_{\bsPP}^{\nabla}(\wtilde{X}_{t},Y_t)- {\QL}_{\bsPP}^{\nabla}(\what{\bsX}_{t},Y_t)                                                            \label{algo:regret}                                     \\
    \bsE_{t}   & = \max(\bsE_{t-1}, \bsr_{t}^+ + \bsr_{t}^-)                                                                                                                         \label{algo:lr1}                         \\
    \bsV_{t}   & = \bsV_{t-1} + \bsr_t^{ \odot 2}                                                                                                                                    \label{algo:lr2}                         \\
    \bseta_{t} & =\min\left( \left(-\log(\bsw_0) \odot \bsV_t^{\odot -1} \right)^{\odot\frac{1}{2}} ,  \frac{1}{2}\bsE_{t}^{\odot-1}\right)                                          \label{algo:lr3}                         \\
    \bsR_{t}   & = \bsR_{t-1}+  \bsr_{t} \odot \left(  \bsone - \bseta_{t} \odot \bsr_{t} \right)/2 + \bsE_{t} \odot \mathbb{1}\{-2\bseta_{t}\odot \bsr_{t} > 1\}                                     \label{algo:cum_regret} \\
    \bsw_{t}   & = K \bsw_{0} \odot \softmax\left( -  \bseta_{t} \odot \bsR_{t} + \log( \bseta_{t}) \right) \label{algo:weights}
  \end{align}
  \label{algo:boa_general_step}
\end{subequations}
where $\bsx^+$ and $\bsx^-$ denote the elementwise positive and negative parts of $\bsx$ and $\odot$ the elementwise product (Hadamard product).
The learning rate $\bseta_{t}$ determines the weight adjustment speed. It depends on the bound estimator $\bsE_{t}$ and $\bsV_t$, which is an estimator for the variance. The algorithm describes how weights are calculated on a full quantile grid $\bsPP$. First, the instantaneous regret is calculated~\eqref{algo:regret}. Then the learning rate (\ref{algo:lr1}-\ref{algo:lr3}) is adjusted. In \eqref{algo:cum_regret} the cumulative regret is calculated. Afterward, we calculate the weights~\eqref{algo:weights}. Finally, the forecaster uses $\bsw_{t}$ to calculate $\wtilde{X}_{t+1}$ and starts over with~\eqref{algo:regret}.

% Bis hier stark kürzen

% \subsection{Extensions of Online Learning Algorithms}

Several extensions of online learning algorithms were proposed in the literature. However, they can also be applied in standard Batch-Learning settings. The benefits of these extensions have been confirmed in empirical studies. Some extensions, like shrinkage operators, are also valuable to nest specific weighting strategies into the learning algorithm.

\subsection{Smoothing}\label{subsec_smooth}

As mentioned, we apply the general CRPS learning idea to multivariate data. Thereby, we adopt the two weight-smoothing methods that were included in the original CRPS Learning algorithm. The first consists of the dimension reduction method using basis matrices. The approach is analogous to the idea discussed in Section~\ref{cprs_learn_using_quant_reg}. Using a bivariate basis, we can reduce the dimensionality of the instantaneous regret from $\D \times \P$ to $\Dr \times \Pr$:
\begin{align}
  \widetilde{\bsr}_k = \frac{\Dr\Pr}{\D\P} {\bsB^{\mult}}' \bsr_k \bsB^{\prob}
\end{align}
We can use this reduced regret to carry out the online learning algorithm as usual. After obtaining weights on this reduced version of the regret (we refer to them as $\bsbeta_{t,k}$), we can utilize the basis matrices again to obtain weights in our original dimensions of interest $\bsw_{t,k} = {\bsB^\mult} \bsbeta_{t,k} {\bsB^\prob}'$.

This relatively simple method yields a powerful property: It bridges the gap between purely pointwise weight optimization based on quantiles and the constant approach where weights are optimized concerning the CRPS. This means we can move from a setting with low flexibility and low estimation risk to a very flexible one at the price of high estimation risk.

Another option is to smooth the weights using penalized smoothing. This method can be applied after the estimation, i.e., after the updating step. Hereby we consider two sets of bounded basis functions $\bspsi^{\text{\mult}}=(\psi_1,\ldots, \psi_{\D})$ and $\bspsi^{\text{\prob}}=(\psi_1,\ldots, \psi_{\P})$ on $(0,1)$ that we will use for penalized smoothing.

Then the weights can be represented by
\begin{equation}
  w_{t,k} = \bspsi^{\mult} \bsb_{t,k} {\bspsi^{\prob}}'
\end{equation}
with parameter matix $\bsb_{t,k}$. We estimate $\bsb_{t,k}$ by penalized $L_1$- and $L_2$-smoothing which minimizes
\begin{align}
   & \| \bsbeta_{t,\d, k}' \bsvarphi^\prob  - \bsb_{t, \d, k}' \bspsi^{\prob}  \|^2_2 + \lambda^{\prob}  \| \mathcal{D}_{q}  (\bsb_{t, \d, k}' \bspsi^\prob)  \|^2_2 +                       \nonumber \\
   & \| \bsbeta_{t, \p, k}' \bsvarphi^\mult  - \bsb_{t, \p, k}' \bspsi^\mult  \|^2_2 + \lambda^\mult  \| \mathcal{D}_{q}  (\bsb_{t, \p, k}' \bspsi^\mult)  \|^2_2  \label{eq_function_smooth}
\end{align}
for each $k$ given $\bsbeta_{t,k}$ with differential operator $\mathcal{D}_q$ of order $q$. The differential order characterizes the smoothing penalty, and $\lambda\geq 0$ characterizes the roughness penalty. Typically, $q=2$ is considered along with cubic B-Splines to penalize for roughness~\citep{wang2011smoothing, wood2017generalized}. However, we prefer using $q<2$ here. As this smoothes towards constant weights over $\bsPP$ for $\lambda\to \infty$ and not towards a linear relationship between weights and probabilities as for $q=2$. As pointed out in~\cite{berrisch2021crps} there is no argument in CRPS learning that supports shrinkage towards a linear relationship. In contrast, shrinkage towards constant weights yields the non-pointwise CRPS-learning theory of constant weight functions. However, let us remark that the penalized smoothing approach with $\lambda\to \infty$ yields a different result than the simple basis smoothing approach mentioned before with $\bsvarphi = \varphi_1 \equiv 1$.

In applications, we only apply this function bases approach on finite grids of probabilities $\bsPP=(p_1, \ldots, p_\P)$ and a finite number of covariates $\bsDD=(1, \ldots, \D)$. If we consider B-Spline basis functions $\bspsi^\mult$ and $\bspsi^\prob$, then an explicit solution based on ordinary least squares exists for~\eqref{eq_function_smooth}. This explicit solution has a ridge regression representation. The smoothed weights matrix $\bsw_{t,k}$ is then given by
\begin{align}
  \bsw_{t,k}(\bsPP) = & \left(\bsB^\mult\left({\bsB^\mult}'\bsB^\mult +                                                                                                                                                                         \lambda {\bsD_q^\mult}'\bsD_q^\mult\right)^{-1} {\bsB^\mult}'\right) \nonumber           \\
                      & \bsvarphi^\mult\left( \bsDD \right) \bsbeta_{t,k} {\bsvarphi^\prob}\left( \bsPP \right)' \nonumber                                                                                                                                                                                                               \\
                      & \left(\bsB^\prob\left({\bsB^\prob}'\bsB^\prob+ \lambda {\bsD_q^\prob}'\bsD_q^\prob\right)^{-1} {\bsB^\prob}'\right)                                                                                                                                                                                    \nonumber \\
  =                   & \boldsymbol{\mathcal{H}^\mult} \bsvarphi^\mult \bsbeta_{t,k} {\bsvarphi^\prob}' \boldsymbol{\mathcal{H}^\prob}
  \label{eq_smoothing_solution_w}
\end{align}
with basis matrices $\bsB^\mult = \bspsi^\mult\left( \bsDD \right)$ and $\bsB^\prob = \bspsi^\prob\left( \bsPP \right)$, penalty matrices $\bsD^\mult_q$ and $\bsD^\prob_q$. The penalty matrix can be easily computed if equidistant knots are used to create the b-spline basis. Hereinafter, we distinguish $\bsD^S_q$ and $\bsD^G_q$, which refer to the standard equidistant case and the general case where knot placement does not have to be equidistant, respectively. Let $\Delta$ denote the matrix difference operator % (its exact dimensions depend on the context)
\begin{equation}
  \bm{\Delta} =
  \begin{bmatrix}
    -1 & 1                        \\
       & -1 & 1                   \\
       &    & \ddots & \ddots     \\
       &    &        & -1     & 1
  \end{bmatrix}.
\end{equation}
Now, $\bsD^{S}_q$ can be easily computed as $\bsD^{S}_q= \Delta^q \bsI$. The computation of $\bsD^{G}_q$ is more intricate since non-equidistant knots are permitted. The calculation involves an additional weighting step with respect to the non-equidistant distribution of the knots. We elaborate on this topic briefly since the literature is surprisingly scarce~\cite[][Section 2.2]{li2022general}. The required difference matrix can be computed as
\begin{align}
  % \bm{D}_1 & = \bm{W}_1^{-1}\bm{\Delta}                                                             \\
  % \bm{D}_2 & = \bm{W}_2^{-1}\bm{\Delta}\bm{W}_1^{-1}\bm{\Delta}                                     \\
  \bm{D}^{G}_q & = \bm{W}_q^{-1}\bm{\Delta}\bm{W}_{q - 1}^{-1}\bm{\Delta}\cdots\bm{W}_1^{-1}\bm{\Delta} \label{diffmat}
\end{align}
where $W_q$ are weighting matrices that depend on the order of the B-Spline basis, denoted as $\bsorder$, and the knots. Let $\innerknots$ denote the number of inner knots. The dimension of the difference matrices $\Delta$ in~\eqref{diffmat} depend on $W_q$ are $(\innerknots-\bsorder-q)\times(\innerknots-\bsorder)$. The total number of knots will be $\innerknots + 2\bsorder$. We can specify the weighting matrices as:
\begin{align}
  \bm{W}_q = \frac{1}{\bsorder - q}\begin{bmatrix}
                                     t_{\bsorder + 1} - t_{1 + q} \\ & t_{\bsorder + 2} - t_{2 + q} \\ & & \ddots \\ & & & t_{\innerknots+2\bsorder-q} - t_{\innerknots + \bsorder}
                                   \end{bmatrix},
\end{align}
The quantity $\bsorder-q$ represents the lag used to differentiate the knots. If the knots are equidistant, then $\bm{W}_q$ will be proportional to the identity matrix $\bsI$. Therefore, it nets the standard P-Spline, which uses $\bsD^{S}_q= \Delta^q \bsI$. This gives rise to the general P-Spline estimator. However, $\bm{W}_q$ is only proportional to $\bsI$ rather than equal to it. Therefore, scaling needs to be applied for the results of both estimators to coincide. The scaling can be applied to lambda, the penalty matrix, or the difference matrix. To state this formally, let $\bm{P}_q^S = {\bm{D}_q^S}'\bm{D}_q^S$ and $\bm{P}_q^G = {\bm{D}_q^G}'\bm{D}_q^G$ denote the penalty terms of the standard and general P-Spline estimators. The scaling factor with respect to the penalty $\bm{P}_q^G$ is $\left(\Tr\left(\bm{W}_q\right)/(\innerknots+\bsorder-q)\right)^{2q}$ so the following relation holds:
\begin{align}
  \bm{P}_q^S & = \left(\Tr\left(\bm{W}_q\right)/(\innerknots+\bsorder-q)\right)^{2q} \bm{P}_q^G. \label{equi_non_equi_rel}
\end{align}
This is only valid for equidistant B-Splines. For non-equidistant B-Splines, the generalized P-Spline is the only appropriate estimator. We suggest always applying this scaling to ensure the comparability between lambda values in equidistant situations.

For notational brevity, we denote the first and last part of~\eqref{eq_smoothing_solution_w} as $\boldsymbol{\mathcal{H}}^\mult$ and $\boldsymbol{\mathcal{H}}^\prob$, respectively, the so-called hat matrices. Fortunately, they do not depend on time-varying components; therefore, we can compute them prior to the main online learning task, which yields a great reduction in the algorithm's computational complexity.

\subsection{Knot placement for Smoothing Splines}

For both smoothing approaches discussed above, the knots of the B-Spline basis must be placed. A well-established approach is placing plenty of equidistant knots. However, as discussed above, non-equidistant knot placement is valid if the penalty is defined accordingly. We consider equidistant and non-equidistant knots. Thereby, the non-central beta distribution with the following parameterization is used for distributing the knots:
\begin{align}
  \mathcal{B}(x, a, b, c) = \sum_{j=0}^{\infty} e^{-c/2} \frac{\left( \frac{c}{2} \right)^j}{j!} I_x \left( a + j , b \right)
\end{align}
Where $I_x$ is the incomplete beta function, $a$ and $b$ are shape parameters, and $c$ is the non-centrality parameter~\cite{johnson1995continuous}. Algorithm~\ref{algo:knots} describes the knot placement in detail. It returns equidistant knots if $\mu = 0.5$, $\sigma = 1$, $c = 0$ and  the tailweight parameter $\tau = 1$.
\begin{algorithm}[!h]
  \DontPrintSemicolon
  \KwData{Let $\mathcal{B}$ denote the CDF of the beta distribution and $\mu, \sigma, c, \tau, \text{deg}$ be parameters for adjusting the knot placement.}
  \KwResult{Sequence of knots with length $\innerknots + 2\left( \text{deg}+1 \right) = \innerknots + 2\bsorder$}
  \SetAlgoLined
  % \ali{4em}{seq\_i} = sequence(from = 0, to = 1, length = \innerknots+2)\;
  \ali{3.5em}{x} $= (0, 1, \ldots, \innerknots+2)/(\innerknots+2) $\;
  \ali{3.5em}{a} $= 2 \sigma \left(1-\mu \right)$\;
  \ali{3.5em}{b} $= 2 \sigma \mu $\;
  \ali{3.5em}{knots\_c} $= \mathcal{B}\left(\text{x}, \text{a}, \text{b}, \lvert c \rvert \right)$\;
  \If{$c < 0$}{
    knots\_c  $= \text{reverse}\left( 1 - \text{knots\_c} \right)$\;
  }
  \ali{3.5em}{knots\_l} $= |\tau| \left( \text{knots\_c}[2] -\text{knots\_c}[1] \right) \left( -\text{deg}, \ldots, -1 \right)$\;
  \ali{3.5em}{knots\_r} $= |\tau| \left( \text{knots\_c}[\innerknots + 2 ] - \text{knots\_c}[\innerknots + 1 ]\right) \left( 1, \ldots, \text{deg} \right) + 1$\;
  \ali{3.5em}{knots} $= \text{combine}\left( \text{knots\_l, knots\_c, knots\_r} \right)$\;
  % return(knots)\;
  \caption{\label{algo:knots} Knot placement for B-Splines}
\end{algorithm}
Figure~\ref{fig:knots} shows B-Spline Basis' for different knot placements for the inputs of Algorithm~\ref{algo:knots}.
\begin{figure}[!h]
  \centering{
    \begin{subfigure}[a]{\textwidth}
      \centering
      \resizebox{\textwidth}{!}{
        % !TEX encoding = UTF-8 Unicode
\begin{tikzpicture}[x=1pt,y=1pt]
\definecolor{fillColor}{RGB}{255,255,255}
\path[use as bounding box,fill=fillColor,fill opacity=0.00] (0,0) rectangle (578.16,289.08);
\begin{scope}
\path[clip] ( 32.73,186.57) rectangle (198.42,263.47);
\definecolor{drawColor}{gray}{0.92}

\path[draw=drawColor,line width= 0.3pt,line join=round] ( 32.73,199.60) --
	(198.42,199.60);

\path[draw=drawColor,line width= 0.3pt,line join=round] ( 32.73,218.66) --
	(198.42,218.66);

\path[draw=drawColor,line width= 0.3pt,line join=round] ( 32.73,237.73) --
	(198.42,237.73);

\path[draw=drawColor,line width= 0.3pt,line join=round] ( 32.73,256.80) --
	(198.42,256.80);

\path[draw=drawColor,line width= 0.3pt,line join=round] ( 57.94,186.57) --
	( 57.94,263.47);

\path[draw=drawColor,line width= 0.3pt,line join=round] ( 96.36,186.57) --
	( 96.36,263.47);

\path[draw=drawColor,line width= 0.3pt,line join=round] (134.79,186.57) --
	(134.79,263.47);

\path[draw=drawColor,line width= 0.3pt,line join=round] (173.21,186.57) --
	(173.21,263.47);

\path[draw=drawColor,line width= 0.6pt,line join=round] ( 32.73,190.06) --
	(198.42,190.06);

\path[draw=drawColor,line width= 0.6pt,line join=round] ( 32.73,209.13) --
	(198.42,209.13);

\path[draw=drawColor,line width= 0.6pt,line join=round] ( 32.73,228.20) --
	(198.42,228.20);

\path[draw=drawColor,line width= 0.6pt,line join=round] ( 32.73,247.26) --
	(198.42,247.26);

\path[draw=drawColor,line width= 0.6pt,line join=round] ( 38.73,186.57) --
	( 38.73,263.47);

\path[draw=drawColor,line width= 0.6pt,line join=round] ( 77.15,186.57) --
	( 77.15,263.47);

\path[draw=drawColor,line width= 0.6pt,line join=round] (115.58,186.57) --
	(115.58,263.47);

\path[draw=drawColor,line width= 0.6pt,line join=round] (154.00,186.57) --
	(154.00,263.47);

\path[draw=drawColor,line width= 0.6pt,line join=round] (192.42,186.57) --
	(192.42,263.47);
\definecolor{drawColor}{RGB}{155,38,176}

\path[draw=drawColor,line width= 1.1pt,line join=round] ( 40.26,222.98) --
	( 41.80,218.15) --
	( 43.34,213.70) --
	( 44.88,209.64) --
	( 46.41,205.96) --
	( 47.95,202.66) --
	( 49.49,199.75) --
	( 51.02,197.21) --
	( 52.56,195.06) --
	( 54.10,193.30) --
	( 55.63,191.92) --
	( 57.17,190.92) --
	( 58.71,190.30) --
	( 60.24,190.07) --
	( 61.78,190.06) --
	( 63.32,190.06) --
	( 64.86,190.06) --
	( 66.39,190.06) --
	( 67.93,190.06) --
	( 69.47,190.06) --
	( 71.00,190.06) --
	( 72.54,190.06) --
	( 74.08,190.06) --
	( 75.61,190.06) --
	( 77.15,190.06) --
	( 78.69,190.06) --
	( 80.23,190.06) --
	( 81.76,190.06) --
	( 83.30,190.06) --
	( 84.84,190.06) --
	( 86.37,190.06) --
	( 87.91,190.06) --
	( 89.45,190.06) --
	( 90.98,190.06) --
	( 92.52,190.06) --
	( 94.06,190.06) --
	( 95.60,190.06) --
	( 97.13,190.06) --
	( 98.67,190.06) --
	(100.21,190.06) --
	(101.74,190.06) --
	(103.28,190.06) --
	(104.82,190.06) --
	(106.35,190.06) --
	(107.89,190.06) --
	(109.43,190.06) --
	(110.96,190.06) --
	(112.50,190.06) --
	(114.04,190.06) --
	(115.58,190.06) --
	(117.11,190.06) --
	(118.65,190.06) --
	(120.19,190.06) --
	(121.72,190.06) --
	(123.26,190.06) --
	(124.80,190.06) --
	(126.33,190.06) --
	(127.87,190.06) --
	(129.41,190.06) --
	(130.95,190.06) --
	(132.48,190.06) --
	(134.02,190.06) --
	(135.56,190.06) --
	(137.09,190.06) --
	(138.63,190.06) --
	(140.17,190.06) --
	(141.70,190.06) --
	(143.24,190.06) --
	(144.78,190.06) --
	(146.32,190.06) --
	(147.85,190.06) --
	(149.39,190.06) --
	(150.93,190.06) --
	(152.46,190.06) --
	(154.00,190.06) --
	(155.54,190.06) --
	(157.07,190.06) --
	(158.61,190.06) --
	(160.15,190.06) --
	(161.69,190.06) --
	(163.22,190.06) --
	(164.76,190.06) --
	(166.30,190.06) --
	(167.83,190.06) --
	(169.37,190.06) --
	(170.91,190.06) --
	(172.44,190.06) --
	(173.98,190.06) --
	(175.52,190.06) --
	(177.05,190.06) --
	(178.59,190.06) --
	(180.13,190.06) --
	(181.67,190.06) --
	(183.20,190.06) --
	(184.74,190.06) --
	(186.28,190.06) --
	(187.81,190.06) --
	(189.35,190.06) --
	(190.89,190.06);
\definecolor{drawColor}{RGB}{63,81,180}

\path[draw=drawColor,line width= 1.1pt,line join=round] ( 40.26,233.13) --
	( 41.80,237.12) --
	( 43.34,240.16) --
	( 44.88,242.26) --
	( 46.41,243.41) --
	( 47.95,243.62) --
	( 49.49,242.89) --
	( 51.02,241.21) --
	( 52.56,238.59) --
	( 54.10,235.02) --
	( 55.63,230.50) --
	( 57.17,225.05) --
	( 58.71,218.65) --
	( 60.24,211.30) --
	( 61.78,203.99) --
	( 63.32,198.21) --
	( 64.86,193.97) --
	( 66.39,191.27) --
	( 67.93,190.11) --
	( 69.47,190.06) --
	( 71.00,190.06) --
	( 72.54,190.06) --
	( 74.08,190.06) --
	( 75.61,190.06) --
	( 77.15,190.06) --
	( 78.69,190.06) --
	( 80.23,190.06) --
	( 81.76,190.06) --
	( 83.30,190.06) --
	( 84.84,190.06) --
	( 86.37,190.06) --
	( 87.91,190.06) --
	( 89.45,190.06) --
	( 90.98,190.06) --
	( 92.52,190.06) --
	( 94.06,190.06) --
	( 95.60,190.06) --
	( 97.13,190.06) --
	( 98.67,190.06) --
	(100.21,190.06) --
	(101.74,190.06) --
	(103.28,190.06) --
	(104.82,190.06) --
	(106.35,190.06) --
	(107.89,190.06) --
	(109.43,190.06) --
	(110.96,190.06) --
	(112.50,190.06) --
	(114.04,190.06) --
	(115.58,190.06) --
	(117.11,190.06) --
	(118.65,190.06) --
	(120.19,190.06) --
	(121.72,190.06) --
	(123.26,190.06) --
	(124.80,190.06) --
	(126.33,190.06) --
	(127.87,190.06) --
	(129.41,190.06) --
	(130.95,190.06) --
	(132.48,190.06) --
	(134.02,190.06) --
	(135.56,190.06) --
	(137.09,190.06) --
	(138.63,190.06) --
	(140.17,190.06) --
	(141.70,190.06) --
	(143.24,190.06) --
	(144.78,190.06) --
	(146.32,190.06) --
	(147.85,190.06) --
	(149.39,190.06) --
	(150.93,190.06) --
	(152.46,190.06) --
	(154.00,190.06) --
	(155.54,190.06) --
	(157.07,190.06) --
	(158.61,190.06) --
	(160.15,190.06) --
	(161.69,190.06) --
	(163.22,190.06) --
	(164.76,190.06) --
	(166.30,190.06) --
	(167.83,190.06) --
	(169.37,190.06) --
	(170.91,190.06) --
	(172.44,190.06) --
	(173.98,190.06) --
	(175.52,190.06) --
	(177.05,190.06) --
	(178.59,190.06) --
	(180.13,190.06) --
	(181.67,190.06) --
	(183.20,190.06) --
	(184.74,190.06) --
	(186.28,190.06) --
	(187.81,190.06) --
	(189.35,190.06) --
	(190.89,190.06);
\definecolor{drawColor}{RGB}{2,169,243}

\path[draw=drawColor,line width= 1.1pt,line join=round] ( 40.26,190.34) --
	( 41.80,191.19) --
	( 43.34,192.59) --
	( 44.88,194.56) --
	( 46.41,197.08) --
	( 47.95,200.17) --
	( 49.49,203.82) --
	( 51.02,208.03) --
	( 52.56,212.80) --
	( 54.10,218.14) --
	( 55.63,224.03) --
	( 57.17,230.49) --
	( 58.71,237.51) --
	( 60.24,245.09) --
	( 61.78,251.20) --
	( 63.32,252.75) --
	( 64.86,249.72) --
	( 66.39,242.12) --
	( 67.93,229.93) --
	( 69.47,215.39) --
	( 71.00,203.99) --
	( 72.54,195.97) --
	( 74.08,191.34) --
	( 75.61,190.06) --
	( 77.15,190.06) --
	( 78.69,190.06) --
	( 80.23,190.06) --
	( 81.76,190.06) --
	( 83.30,190.06) --
	( 84.84,190.06) --
	( 86.37,190.06) --
	( 87.91,190.06) --
	( 89.45,190.06) --
	( 90.98,190.06) --
	( 92.52,190.06) --
	( 94.06,190.06) --
	( 95.60,190.06) --
	( 97.13,190.06) --
	( 98.67,190.06) --
	(100.21,190.06) --
	(101.74,190.06) --
	(103.28,190.06) --
	(104.82,190.06) --
	(106.35,190.06) --
	(107.89,190.06) --
	(109.43,190.06) --
	(110.96,190.06) --
	(112.50,190.06) --
	(114.04,190.06) --
	(115.58,190.06) --
	(117.11,190.06) --
	(118.65,190.06) --
	(120.19,190.06) --
	(121.72,190.06) --
	(123.26,190.06) --
	(124.80,190.06) --
	(126.33,190.06) --
	(127.87,190.06) --
	(129.41,190.06) --
	(130.95,190.06) --
	(132.48,190.06) --
	(134.02,190.06) --
	(135.56,190.06) --
	(137.09,190.06) --
	(138.63,190.06) --
	(140.17,190.06) --
	(141.70,190.06) --
	(143.24,190.06) --
	(144.78,190.06) --
	(146.32,190.06) --
	(147.85,190.06) --
	(149.39,190.06) --
	(150.93,190.06) --
	(152.46,190.06) --
	(154.00,190.06) --
	(155.54,190.06) --
	(157.07,190.06) --
	(158.61,190.06) --
	(160.15,190.06) --
	(161.69,190.06) --
	(163.22,190.06) --
	(164.76,190.06) --
	(166.30,190.06) --
	(167.83,190.06) --
	(169.37,190.06) --
	(170.91,190.06) --
	(172.44,190.06) --
	(173.98,190.06) --
	(175.52,190.06) --
	(177.05,190.06) --
	(178.59,190.06) --
	(180.13,190.06) --
	(181.67,190.06) --
	(183.20,190.06) --
	(184.74,190.06) --
	(186.28,190.06) --
	(187.81,190.06) --
	(189.35,190.06) --
	(190.89,190.06);
\definecolor{drawColor}{RGB}{0,150,135}

\path[draw=drawColor,line width= 1.1pt,line join=round] ( 40.26,190.06) --
	( 41.80,190.06) --
	( 43.34,190.06) --
	( 44.88,190.06) --
	( 46.41,190.06) --
	( 47.95,190.06) --
	( 49.49,190.06) --
	( 51.02,190.06) --
	( 52.56,190.06) --
	( 54.10,190.06) --
	( 55.63,190.06) --
	( 57.17,190.06) --
	( 58.71,190.06) --
	( 60.24,190.06) --
	( 61.78,191.27) --
	( 63.32,195.49) --
	( 64.86,202.76) --
	( 66.39,213.07) --
	( 67.93,226.41) --
	( 69.47,240.04) --
	( 71.00,247.17) --
	( 72.54,247.49) --
	( 74.08,240.99) --
	( 75.61,227.77) --
	( 77.15,213.94) --
	( 78.69,203.26) --
	( 80.23,195.72) --
	( 81.76,191.33) --
	( 83.30,190.06) --
	( 84.84,190.06) --
	( 86.37,190.06) --
	( 87.91,190.06) --
	( 89.45,190.06) --
	( 90.98,190.06) --
	( 92.52,190.06) --
	( 94.06,190.06) --
	( 95.60,190.06) --
	( 97.13,190.06) --
	( 98.67,190.06) --
	(100.21,190.06) --
	(101.74,190.06) --
	(103.28,190.06) --
	(104.82,190.06) --
	(106.35,190.06) --
	(107.89,190.06) --
	(109.43,190.06) --
	(110.96,190.06) --
	(112.50,190.06) --
	(114.04,190.06) --
	(115.58,190.06) --
	(117.11,190.06) --
	(118.65,190.06) --
	(120.19,190.06) --
	(121.72,190.06) --
	(123.26,190.06) --
	(124.80,190.06) --
	(126.33,190.06) --
	(127.87,190.06) --
	(129.41,190.06) --
	(130.95,190.06) --
	(132.48,190.06) --
	(134.02,190.06) --
	(135.56,190.06) --
	(137.09,190.06) --
	(138.63,190.06) --
	(140.17,190.06) --
	(141.70,190.06) --
	(143.24,190.06) --
	(144.78,190.06) --
	(146.32,190.06) --
	(147.85,190.06) --
	(149.39,190.06) --
	(150.93,190.06) --
	(152.46,190.06) --
	(154.00,190.06) --
	(155.54,190.06) --
	(157.07,190.06) --
	(158.61,190.06) --
	(160.15,190.06) --
	(161.69,190.06) --
	(163.22,190.06) --
	(164.76,190.06) --
	(166.30,190.06) --
	(167.83,190.06) --
	(169.37,190.06) --
	(170.91,190.06) --
	(172.44,190.06) --
	(173.98,190.06) --
	(175.52,190.06) --
	(177.05,190.06) --
	(178.59,190.06) --
	(180.13,190.06) --
	(181.67,190.06) --
	(183.20,190.06) --
	(184.74,190.06) --
	(186.28,190.06) --
	(187.81,190.06) --
	(189.35,190.06) --
	(190.89,190.06);
\definecolor{drawColor}{RGB}{139,195,74}

\path[draw=drawColor,line width= 1.1pt,line join=round] ( 40.26,190.06) --
	( 41.80,190.06) --
	( 43.34,190.06) --
	( 44.88,190.06) --
	( 46.41,190.06) --
	( 47.95,190.06) --
	( 49.49,190.06) --
	( 51.02,190.06) --
	( 52.56,190.06) --
	( 54.10,190.06) --
	( 55.63,190.06) --
	( 57.17,190.06) --
	( 58.71,190.06) --
	( 60.24,190.06) --
	( 61.78,190.06) --
	( 63.32,190.06) --
	( 64.86,190.06) --
	( 66.39,190.06) --
	( 67.93,190.06) --
	( 69.47,191.02) --
	( 71.00,195.30) --
	( 72.54,203.00) --
	( 74.08,214.13) --
	( 75.61,228.60) --
	( 77.15,240.81) --
	( 78.69,247.32) --
	( 80.23,248.14) --
	( 81.76,243.25) --
	( 83.30,232.71) --
	( 84.84,221.00) --
	( 86.37,211.16) --
	( 87.91,203.20) --
	( 89.45,197.11) --
	( 90.98,192.91) --
	( 92.52,190.58) --
	( 94.06,190.06) --
	( 95.60,190.06) --
	( 97.13,190.06) --
	( 98.67,190.06) --
	(100.21,190.06) --
	(101.74,190.06) --
	(103.28,190.06) --
	(104.82,190.06) --
	(106.35,190.06) --
	(107.89,190.06) --
	(109.43,190.06) --
	(110.96,190.06) --
	(112.50,190.06) --
	(114.04,190.06) --
	(115.58,190.06) --
	(117.11,190.06) --
	(118.65,190.06) --
	(120.19,190.06) --
	(121.72,190.06) --
	(123.26,190.06) --
	(124.80,190.06) --
	(126.33,190.06) --
	(127.87,190.06) --
	(129.41,190.06) --
	(130.95,190.06) --
	(132.48,190.06) --
	(134.02,190.06) --
	(135.56,190.06) --
	(137.09,190.06) --
	(138.63,190.06) --
	(140.17,190.06) --
	(141.70,190.06) --
	(143.24,190.06) --
	(144.78,190.06) --
	(146.32,190.06) --
	(147.85,190.06) --
	(149.39,190.06) --
	(150.93,190.06) --
	(152.46,190.06) --
	(154.00,190.06) --
	(155.54,190.06) --
	(157.07,190.06) --
	(158.61,190.06) --
	(160.15,190.06) --
	(161.69,190.06) --
	(163.22,190.06) --
	(164.76,190.06) --
	(166.30,190.06) --
	(167.83,190.06) --
	(169.37,190.06) --
	(170.91,190.06) --
	(172.44,190.06) --
	(173.98,190.06) --
	(175.52,190.06) --
	(177.05,190.06) --
	(178.59,190.06) --
	(180.13,190.06) --
	(181.67,190.06) --
	(183.20,190.06) --
	(184.74,190.06) --
	(186.28,190.06) --
	(187.81,190.06) --
	(189.35,190.06) --
	(190.89,190.06);
\definecolor{drawColor}{RGB}{255,235,58}

\path[draw=drawColor,line width= 1.1pt,line join=round] ( 40.26,190.06) --
	( 41.80,190.06) --
	( 43.34,190.06) --
	( 44.88,190.06) --
	( 46.41,190.06) --
	( 47.95,190.06) --
	( 49.49,190.06) --
	( 51.02,190.06) --
	( 52.56,190.06) --
	( 54.10,190.06) --
	( 55.63,190.06) --
	( 57.17,190.06) --
	( 58.71,190.06) --
	( 60.24,190.06) --
	( 61.78,190.06) --
	( 63.32,190.06) --
	( 64.86,190.06) --
	( 66.39,190.06) --
	( 67.93,190.06) --
	( 69.47,190.06) --
	( 71.00,190.06) --
	( 72.54,190.06) --
	( 74.08,190.06) --
	( 75.61,190.09) --
	( 77.15,191.70) --
	( 78.69,195.87) --
	( 80.23,202.59) --
	( 81.76,211.87) --
	( 83.30,223.68) --
	( 84.84,235.21) --
	( 86.37,244.54) --
	( 87.91,251.69) --
	( 89.45,256.64) --
	( 90.98,259.40) --
	( 92.52,259.98) --
	( 94.06,258.43) --
	( 95.60,256.31) --
	( 97.13,254.23) --
	( 98.67,252.17) --
	(100.21,250.15) --
	(101.74,248.17) --
	(103.28,246.21) --
	(104.82,244.30) --
	(106.35,242.41) --
	(107.89,240.56) --
	(109.43,238.74) --
	(110.96,236.95) --
	(112.50,235.20) --
	(114.04,233.48) --
	(115.58,231.79) --
	(117.11,230.14) --
	(118.65,228.52) --
	(120.19,226.94) --
	(121.72,225.38) --
	(123.26,223.86) --
	(124.80,222.38) --
	(126.33,220.93) --
	(127.87,219.51) --
	(129.41,218.12) --
	(130.95,216.77) --
	(132.48,215.45) --
	(134.02,214.17) --
	(135.56,212.91) --
	(137.09,211.70) --
	(138.63,210.51) --
	(140.17,209.36) --
	(141.70,208.24) --
	(143.24,207.16) --
	(144.78,206.10) --
	(146.32,205.09) --
	(147.85,204.10) --
	(149.39,203.15) --
	(150.93,202.23) --
	(152.46,201.35) --
	(154.00,200.50) --
	(155.54,199.68) --
	(157.07,198.89) --
	(158.61,198.14) --
	(160.15,197.42) --
	(161.69,196.74) --
	(163.22,196.09) --
	(164.76,195.47) --
	(166.30,194.89) --
	(167.83,194.34) --
	(169.37,193.82) --
	(170.91,193.33) --
	(172.44,192.88) --
	(173.98,192.47) --
	(175.52,192.08) --
	(177.05,191.73) --
	(178.59,191.42) --
	(180.13,191.13) --
	(181.67,190.88) --
	(183.20,190.66) --
	(184.74,190.48) --
	(186.28,190.33) --
	(187.81,190.21) --
	(189.35,190.13) --
	(190.89,190.08);
\definecolor{drawColor}{RGB}{255,152,0}

\path[draw=drawColor,line width= 1.1pt,line join=round] ( 40.26,190.06) --
	( 41.80,190.06) --
	( 43.34,190.06) --
	( 44.88,190.06) --
	( 46.41,190.06) --
	( 47.95,190.06) --
	( 49.49,190.06) --
	( 51.02,190.06) --
	( 52.56,190.06) --
	( 54.10,190.06) --
	( 55.63,190.06) --
	( 57.17,190.06) --
	( 58.71,190.06) --
	( 60.24,190.06) --
	( 61.78,190.06) --
	( 63.32,190.06) --
	( 64.86,190.06) --
	( 66.39,190.06) --
	( 67.93,190.06) --
	( 69.47,190.06) --
	( 71.00,190.06) --
	( 72.54,190.06) --
	( 74.08,190.06) --
	( 75.61,190.06) --
	( 77.15,190.06) --
	( 78.69,190.06) --
	( 80.23,190.06) --
	( 81.76,190.06) --
	( 83.30,190.06) --
	( 84.84,190.25) --
	( 86.37,190.76) --
	( 87.91,191.57) --
	( 89.45,192.70) --
	( 90.98,194.14) --
	( 92.52,195.90) --
	( 94.06,197.96) --
	( 95.60,200.06) --
	( 97.13,202.12) --
	( 98.67,204.12) --
	(100.21,206.07) --
	(101.74,207.97) --
	(103.28,209.82) --
	(104.82,211.61) --
	(106.35,213.35) --
	(107.89,215.04) --
	(109.43,216.68) --
	(110.96,218.27) --
	(112.50,219.81) --
	(114.04,221.29) --
	(115.58,222.72) --
	(117.11,224.10) --
	(118.65,225.43) --
	(120.19,226.71) --
	(121.72,227.93) --
	(123.26,229.10) --
	(124.80,230.22) --
	(126.33,231.29) --
	(127.87,232.31) --
	(129.41,233.27) --
	(130.95,234.19) --
	(132.48,235.05) --
	(134.02,235.86) --
	(135.56,236.62) --
	(137.09,237.32) --
	(138.63,237.97) --
	(140.17,238.58) --
	(141.70,239.13) --
	(143.24,239.63) --
	(144.78,240.07) --
	(146.32,240.47) --
	(147.85,240.81) --
	(149.39,241.10) --
	(150.93,241.34) --
	(152.46,241.53) --
	(154.00,241.66) --
	(155.54,241.75) --
	(157.07,241.78) --
	(158.61,241.76) --
	(160.15,241.69) --
	(161.69,241.56) --
	(163.22,241.39) --
	(164.76,241.16) --
	(166.30,240.88) --
	(167.83,240.55) --
	(169.37,240.16) --
	(170.91,239.73) --
	(172.44,239.24) --
	(173.98,238.70) --
	(175.52,238.11) --
	(177.05,237.47) --
	(178.59,236.78) --
	(180.13,236.03) --
	(181.67,235.23) --
	(183.20,234.38) --
	(184.74,233.48) --
	(186.28,232.53) --
	(187.81,231.52) --
	(189.35,230.47) --
	(190.89,229.36);
\definecolor{drawColor}{RGB}{121,84,71}

\path[draw=drawColor,line width= 1.1pt,line join=round] ( 40.26,190.06) --
	( 41.80,190.06) --
	( 43.34,190.06) --
	( 44.88,190.06) --
	( 46.41,190.06) --
	( 47.95,190.06) --
	( 49.49,190.06) --
	( 51.02,190.06) --
	( 52.56,190.06) --
	( 54.10,190.06) --
	( 55.63,190.06) --
	( 57.17,190.06) --
	( 58.71,190.06) --
	( 60.24,190.06) --
	( 61.78,190.06) --
	( 63.32,190.06) --
	( 64.86,190.06) --
	( 66.39,190.06) --
	( 67.93,190.06) --
	( 69.47,190.06) --
	( 71.00,190.06) --
	( 72.54,190.06) --
	( 74.08,190.06) --
	( 75.61,190.06) --
	( 77.15,190.06) --
	( 78.69,190.06) --
	( 80.23,190.06) --
	( 81.76,190.06) --
	( 83.30,190.06) --
	( 84.84,190.06) --
	( 86.37,190.06) --
	( 87.91,190.06) --
	( 89.45,190.06) --
	( 90.98,190.06) --
	( 92.52,190.06) --
	( 94.06,190.06) --
	( 95.60,190.08) --
	( 97.13,190.11) --
	( 98.67,190.16) --
	(100.21,190.23) --
	(101.74,190.32) --
	(103.28,190.43) --
	(104.82,190.55) --
	(106.35,190.69) --
	(107.89,190.86) --
	(109.43,191.04) --
	(110.96,191.23) --
	(112.50,191.45) --
	(114.04,191.69) --
	(115.58,191.94) --
	(117.11,192.21) --
	(118.65,192.50) --
	(120.19,192.81) --
	(121.72,193.14) --
	(123.26,193.49) --
	(124.80,193.85) --
	(126.33,194.24) --
	(127.87,194.64) --
	(129.41,195.06) --
	(130.95,195.50) --
	(132.48,195.96) --
	(134.02,196.43) --
	(135.56,196.93) --
	(137.09,197.44) --
	(138.63,197.97) --
	(140.17,198.52) --
	(141.70,199.09) --
	(143.24,199.67) --
	(144.78,200.28) --
	(146.32,200.90) --
	(147.85,201.54) --
	(149.39,202.20) --
	(150.93,202.88) --
	(152.46,203.58) --
	(154.00,204.30) --
	(155.54,205.03) --
	(157.07,205.78) --
	(158.61,206.56) --
	(160.15,207.34) --
	(161.69,208.15) --
	(163.22,208.98) --
	(164.76,209.82) --
	(166.30,210.69) --
	(167.83,211.57) --
	(169.37,212.47) --
	(170.91,213.39) --
	(172.44,214.33) --
	(173.98,215.28) --
	(175.52,216.26) --
	(177.05,217.25) --
	(178.59,218.26) --
	(180.13,219.29) --
	(181.67,220.34) --
	(183.20,221.41) --
	(184.74,222.49) --
	(186.28,223.60) --
	(187.81,224.72) --
	(189.35,225.86) --
	(190.89,227.02);
\end{scope}
\begin{scope}
\path[clip] ( 32.73,104.16) rectangle (198.42,181.07);
\definecolor{drawColor}{gray}{0.92}

\path[draw=drawColor,line width= 0.3pt,line join=round] ( 32.73,117.19) --
	(198.42,117.19);

\path[draw=drawColor,line width= 0.3pt,line join=round] ( 32.73,136.26) --
	(198.42,136.26);

\path[draw=drawColor,line width= 0.3pt,line join=round] ( 32.73,155.32) --
	(198.42,155.32);

\path[draw=drawColor,line width= 0.3pt,line join=round] ( 32.73,174.39) --
	(198.42,174.39);

\path[draw=drawColor,line width= 0.3pt,line join=round] ( 57.94,104.16) --
	( 57.94,181.07);

\path[draw=drawColor,line width= 0.3pt,line join=round] ( 96.36,104.16) --
	( 96.36,181.07);

\path[draw=drawColor,line width= 0.3pt,line join=round] (134.79,104.16) --
	(134.79,181.07);

\path[draw=drawColor,line width= 0.3pt,line join=round] (173.21,104.16) --
	(173.21,181.07);

\path[draw=drawColor,line width= 0.6pt,line join=round] ( 32.73,107.66) --
	(198.42,107.66);

\path[draw=drawColor,line width= 0.6pt,line join=round] ( 32.73,126.72) --
	(198.42,126.72);

\path[draw=drawColor,line width= 0.6pt,line join=round] ( 32.73,145.79) --
	(198.42,145.79);

\path[draw=drawColor,line width= 0.6pt,line join=round] ( 32.73,164.86) --
	(198.42,164.86);

\path[draw=drawColor,line width= 0.6pt,line join=round] ( 38.73,104.16) --
	( 38.73,181.07);

\path[draw=drawColor,line width= 0.6pt,line join=round] ( 77.15,104.16) --
	( 77.15,181.07);

\path[draw=drawColor,line width= 0.6pt,line join=round] (115.58,104.16) --
	(115.58,181.07);

\path[draw=drawColor,line width= 0.6pt,line join=round] (154.00,104.16) --
	(154.00,181.07);

\path[draw=drawColor,line width= 0.6pt,line join=round] (192.42,104.16) --
	(192.42,181.07);
\definecolor{drawColor}{RGB}{155,38,176}

\path[draw=drawColor,line width= 1.1pt,line join=round] ( 40.26,124.87) --
	( 41.80,112.16) --
	( 43.34,107.67) --
	( 44.88,107.66) --
	( 46.41,107.66) --
	( 47.95,107.66) --
	( 49.49,107.66) --
	( 51.02,107.66) --
	( 52.56,107.66) --
	( 54.10,107.66) --
	( 55.63,107.66) --
	( 57.17,107.66) --
	( 58.71,107.66) --
	( 60.24,107.66) --
	( 61.78,107.66) --
	( 63.32,107.66) --
	( 64.86,107.66) --
	( 66.39,107.66) --
	( 67.93,107.66) --
	( 69.47,107.66) --
	( 71.00,107.66) --
	( 72.54,107.66) --
	( 74.08,107.66) --
	( 75.61,107.66) --
	( 77.15,107.66) --
	( 78.69,107.66) --
	( 80.23,107.66) --
	( 81.76,107.66) --
	( 83.30,107.66) --
	( 84.84,107.66) --
	( 86.37,107.66) --
	( 87.91,107.66) --
	( 89.45,107.66) --
	( 90.98,107.66) --
	( 92.52,107.66) --
	( 94.06,107.66) --
	( 95.60,107.66) --
	( 97.13,107.66) --
	( 98.67,107.66) --
	(100.21,107.66) --
	(101.74,107.66) --
	(103.28,107.66) --
	(104.82,107.66) --
	(106.35,107.66) --
	(107.89,107.66) --
	(109.43,107.66) --
	(110.96,107.66) --
	(112.50,107.66) --
	(114.04,107.66) --
	(115.58,107.66) --
	(117.11,107.66) --
	(118.65,107.66) --
	(120.19,107.66) --
	(121.72,107.66) --
	(123.26,107.66) --
	(124.80,107.66) --
	(126.33,107.66) --
	(127.87,107.66) --
	(129.41,107.66) --
	(130.95,107.66) --
	(132.48,107.66) --
	(134.02,107.66) --
	(135.56,107.66) --
	(137.09,107.66) --
	(138.63,107.66) --
	(140.17,107.66) --
	(141.70,107.66) --
	(143.24,107.66) --
	(144.78,107.66) --
	(146.32,107.66) --
	(147.85,107.66) --
	(149.39,107.66) --
	(150.93,107.66) --
	(152.46,107.66) --
	(154.00,107.66) --
	(155.54,107.66) --
	(157.07,107.66) --
	(158.61,107.66) --
	(160.15,107.66) --
	(161.69,107.66) --
	(163.22,107.66) --
	(164.76,107.66) --
	(166.30,107.66) --
	(167.83,107.66) --
	(169.37,107.66) --
	(170.91,107.66) --
	(172.44,107.66) --
	(173.98,107.66) --
	(175.52,107.66) --
	(177.05,107.66) --
	(178.59,107.66) --
	(180.13,107.66) --
	(181.67,107.66) --
	(183.20,107.66) --
	(184.74,107.66) --
	(186.28,107.66) --
	(187.81,107.66) --
	(189.35,107.66) --
	(190.89,107.66);
\definecolor{drawColor}{RGB}{63,81,180}

\path[draw=drawColor,line width= 1.1pt,line join=round] ( 40.26,163.99) --
	( 41.80,168.51) --
	( 43.34,159.35) --
	( 44.88,143.97) --
	( 46.41,131.29) --
	( 47.95,121.32) --
	( 49.49,114.06) --
	( 51.02,109.52) --
	( 52.56,107.70) --
	( 54.10,107.66) --
	( 55.63,107.66) --
	( 57.17,107.66) --
	( 58.71,107.66) --
	( 60.24,107.66) --
	( 61.78,107.66) --
	( 63.32,107.66) --
	( 64.86,107.66) --
	( 66.39,107.66) --
	( 67.93,107.66) --
	( 69.47,107.66) --
	( 71.00,107.66) --
	( 72.54,107.66) --
	( 74.08,107.66) --
	( 75.61,107.66) --
	( 77.15,107.66) --
	( 78.69,107.66) --
	( 80.23,107.66) --
	( 81.76,107.66) --
	( 83.30,107.66) --
	( 84.84,107.66) --
	( 86.37,107.66) --
	( 87.91,107.66) --
	( 89.45,107.66) --
	( 90.98,107.66) --
	( 92.52,107.66) --
	( 94.06,107.66) --
	( 95.60,107.66) --
	( 97.13,107.66) --
	( 98.67,107.66) --
	(100.21,107.66) --
	(101.74,107.66) --
	(103.28,107.66) --
	(104.82,107.66) --
	(106.35,107.66) --
	(107.89,107.66) --
	(109.43,107.66) --
	(110.96,107.66) --
	(112.50,107.66) --
	(114.04,107.66) --
	(115.58,107.66) --
	(117.11,107.66) --
	(118.65,107.66) --
	(120.19,107.66) --
	(121.72,107.66) --
	(123.26,107.66) --
	(124.80,107.66) --
	(126.33,107.66) --
	(127.87,107.66) --
	(129.41,107.66) --
	(130.95,107.66) --
	(132.48,107.66) --
	(134.02,107.66) --
	(135.56,107.66) --
	(137.09,107.66) --
	(138.63,107.66) --
	(140.17,107.66) --
	(141.70,107.66) --
	(143.24,107.66) --
	(144.78,107.66) --
	(146.32,107.66) --
	(147.85,107.66) --
	(149.39,107.66) --
	(150.93,107.66) --
	(152.46,107.66) --
	(154.00,107.66) --
	(155.54,107.66) --
	(157.07,107.66) --
	(158.61,107.66) --
	(160.15,107.66) --
	(161.69,107.66) --
	(163.22,107.66) --
	(164.76,107.66) --
	(166.30,107.66) --
	(167.83,107.66) --
	(169.37,107.66) --
	(170.91,107.66) --
	(172.44,107.66) --
	(173.98,107.66) --
	(175.52,107.66) --
	(177.05,107.66) --
	(178.59,107.66) --
	(180.13,107.66) --
	(181.67,107.66) --
	(183.20,107.66) --
	(184.74,107.66) --
	(186.28,107.66) --
	(187.81,107.66) --
	(189.35,107.66) --
	(190.89,107.66);
\definecolor{drawColor}{RGB}{2,169,243}

\path[draw=drawColor,line width= 1.1pt,line join=round] ( 40.26,110.39) --
	( 41.80,118.58) --
	( 43.34,132.22) --
	( 44.88,146.86) --
	( 46.41,157.15) --
	( 47.95,163.08) --
	( 49.49,164.65) --
	( 51.02,161.86) --
	( 52.56,154.70) --
	( 54.10,145.07) --
	( 55.63,136.47) --
	( 57.17,128.99) --
	( 58.71,122.64) --
	( 60.24,117.40) --
	( 61.78,113.29) --
	( 63.32,110.30) --
	( 64.86,108.42) --
	( 66.39,107.67) --
	( 67.93,107.66) --
	( 69.47,107.66) --
	( 71.00,107.66) --
	( 72.54,107.66) --
	( 74.08,107.66) --
	( 75.61,107.66) --
	( 77.15,107.66) --
	( 78.69,107.66) --
	( 80.23,107.66) --
	( 81.76,107.66) --
	( 83.30,107.66) --
	( 84.84,107.66) --
	( 86.37,107.66) --
	( 87.91,107.66) --
	( 89.45,107.66) --
	( 90.98,107.66) --
	( 92.52,107.66) --
	( 94.06,107.66) --
	( 95.60,107.66) --
	( 97.13,107.66) --
	( 98.67,107.66) --
	(100.21,107.66) --
	(101.74,107.66) --
	(103.28,107.66) --
	(104.82,107.66) --
	(106.35,107.66) --
	(107.89,107.66) --
	(109.43,107.66) --
	(110.96,107.66) --
	(112.50,107.66) --
	(114.04,107.66) --
	(115.58,107.66) --
	(117.11,107.66) --
	(118.65,107.66) --
	(120.19,107.66) --
	(121.72,107.66) --
	(123.26,107.66) --
	(124.80,107.66) --
	(126.33,107.66) --
	(127.87,107.66) --
	(129.41,107.66) --
	(130.95,107.66) --
	(132.48,107.66) --
	(134.02,107.66) --
	(135.56,107.66) --
	(137.09,107.66) --
	(138.63,107.66) --
	(140.17,107.66) --
	(141.70,107.66) --
	(143.24,107.66) --
	(144.78,107.66) --
	(146.32,107.66) --
	(147.85,107.66) --
	(149.39,107.66) --
	(150.93,107.66) --
	(152.46,107.66) --
	(154.00,107.66) --
	(155.54,107.66) --
	(157.07,107.66) --
	(158.61,107.66) --
	(160.15,107.66) --
	(161.69,107.66) --
	(163.22,107.66) --
	(164.76,107.66) --
	(166.30,107.66) --
	(167.83,107.66) --
	(169.37,107.66) --
	(170.91,107.66) --
	(172.44,107.66) --
	(173.98,107.66) --
	(175.52,107.66) --
	(177.05,107.66) --
	(178.59,107.66) --
	(180.13,107.66) --
	(181.67,107.66) --
	(183.20,107.66) --
	(184.74,107.66) --
	(186.28,107.66) --
	(187.81,107.66) --
	(189.35,107.66) --
	(190.89,107.66);
\definecolor{drawColor}{RGB}{0,150,135}

\path[draw=drawColor,line width= 1.1pt,line join=round] ( 40.26,107.66) --
	( 41.80,107.66) --
	( 43.34,107.66) --
	( 44.88,108.41) --
	( 46.41,110.80) --
	( 47.95,114.84) --
	( 49.49,120.53) --
	( 51.02,127.86) --
	( 52.56,136.84) --
	( 54.10,146.25) --
	( 55.63,153.81) --
	( 57.17,159.46) --
	( 58.71,163.21) --
	( 60.24,165.05) --
	( 61.78,164.99) --
	( 63.32,163.02) --
	( 64.86,159.15) --
	( 66.39,153.37) --
	( 67.93,146.54) --
	( 69.47,140.23) --
	( 71.00,134.48) --
	( 72.54,129.29) --
	( 74.08,124.66) --
	( 75.61,120.58) --
	( 77.15,117.06) --
	( 78.69,114.10) --
	( 80.23,111.70) --
	( 81.76,109.86) --
	( 83.30,108.57) --
	( 84.84,107.84) --
	( 86.37,107.66) --
	( 87.91,107.66) --
	( 89.45,107.66) --
	( 90.98,107.66) --
	( 92.52,107.66) --
	( 94.06,107.66) --
	( 95.60,107.66) --
	( 97.13,107.66) --
	( 98.67,107.66) --
	(100.21,107.66) --
	(101.74,107.66) --
	(103.28,107.66) --
	(104.82,107.66) --
	(106.35,107.66) --
	(107.89,107.66) --
	(109.43,107.66) --
	(110.96,107.66) --
	(112.50,107.66) --
	(114.04,107.66) --
	(115.58,107.66) --
	(117.11,107.66) --
	(118.65,107.66) --
	(120.19,107.66) --
	(121.72,107.66) --
	(123.26,107.66) --
	(124.80,107.66) --
	(126.33,107.66) --
	(127.87,107.66) --
	(129.41,107.66) --
	(130.95,107.66) --
	(132.48,107.66) --
	(134.02,107.66) --
	(135.56,107.66) --
	(137.09,107.66) --
	(138.63,107.66) --
	(140.17,107.66) --
	(141.70,107.66) --
	(143.24,107.66) --
	(144.78,107.66) --
	(146.32,107.66) --
	(147.85,107.66) --
	(149.39,107.66) --
	(150.93,107.66) --
	(152.46,107.66) --
	(154.00,107.66) --
	(155.54,107.66) --
	(157.07,107.66) --
	(158.61,107.66) --
	(160.15,107.66) --
	(161.69,107.66) --
	(163.22,107.66) --
	(164.76,107.66) --
	(166.30,107.66) --
	(167.83,107.66) --
	(169.37,107.66) --
	(170.91,107.66) --
	(172.44,107.66) --
	(173.98,107.66) --
	(175.52,107.66) --
	(177.05,107.66) --
	(178.59,107.66) --
	(180.13,107.66) --
	(181.67,107.66) --
	(183.20,107.66) --
	(184.74,107.66) --
	(186.28,107.66) --
	(187.81,107.66) --
	(189.35,107.66) --
	(190.89,107.66);
\definecolor{drawColor}{RGB}{139,195,74}

\path[draw=drawColor,line width= 1.1pt,line join=round] ( 40.26,107.66) --
	( 41.80,107.66) --
	( 43.34,107.66) --
	( 44.88,107.66) --
	( 46.41,107.66) --
	( 47.95,107.66) --
	( 49.49,107.66) --
	( 51.02,107.66) --
	( 52.56,107.66) --
	( 54.10,107.93) --
	( 55.63,108.97) --
	( 57.17,110.79) --
	( 58.71,113.40) --
	( 60.24,116.79) --
	( 61.78,120.97) --
	( 63.32,125.92) --
	( 64.86,131.67) --
	( 66.39,138.19) --
	( 67.93,144.91) --
	( 69.47,150.70) --
	( 71.00,155.55) --
	( 72.54,159.46) --
	( 74.08,162.43) --
	( 75.61,164.45) --
	( 77.15,165.53) --
	( 78.69,165.67) --
	( 80.23,164.86) --
	( 81.76,163.11) --
	( 83.30,160.42) --
	( 84.84,156.78) --
	( 86.37,152.23) --
	( 87.91,147.54) --
	( 89.45,143.11) --
	( 90.98,138.95) --
	( 92.52,135.04) --
	( 94.06,131.39) --
	( 95.60,128.01) --
	( 97.13,124.88) --
	( 98.67,122.02) --
	(100.21,119.41) --
	(101.74,117.07) --
	(103.28,114.99) --
	(104.82,113.16) --
	(106.35,111.60) --
	(107.89,110.30) --
	(109.43,109.26) --
	(110.96,108.47) --
	(112.50,107.95) --
	(114.04,107.69) --
	(115.58,107.66) --
	(117.11,107.66) --
	(118.65,107.66) --
	(120.19,107.66) --
	(121.72,107.66) --
	(123.26,107.66) --
	(124.80,107.66) --
	(126.33,107.66) --
	(127.87,107.66) --
	(129.41,107.66) --
	(130.95,107.66) --
	(132.48,107.66) --
	(134.02,107.66) --
	(135.56,107.66) --
	(137.09,107.66) --
	(138.63,107.66) --
	(140.17,107.66) --
	(141.70,107.66) --
	(143.24,107.66) --
	(144.78,107.66) --
	(146.32,107.66) --
	(147.85,107.66) --
	(149.39,107.66) --
	(150.93,107.66) --
	(152.46,107.66) --
	(154.00,107.66) --
	(155.54,107.66) --
	(157.07,107.66) --
	(158.61,107.66) --
	(160.15,107.66) --
	(161.69,107.66) --
	(163.22,107.66) --
	(164.76,107.66) --
	(166.30,107.66) --
	(167.83,107.66) --
	(169.37,107.66) --
	(170.91,107.66) --
	(172.44,107.66) --
	(173.98,107.66) --
	(175.52,107.66) --
	(177.05,107.66) --
	(178.59,107.66) --
	(180.13,107.66) --
	(181.67,107.66) --
	(183.20,107.66) --
	(184.74,107.66) --
	(186.28,107.66) --
	(187.81,107.66) --
	(189.35,107.66) --
	(190.89,107.66);
\definecolor{drawColor}{RGB}{255,235,58}

\path[draw=drawColor,line width= 1.1pt,line join=round] ( 40.26,107.66) --
	( 41.80,107.66) --
	( 43.34,107.66) --
	( 44.88,107.66) --
	( 46.41,107.66) --
	( 47.95,107.66) --
	( 49.49,107.66) --
	( 51.02,107.66) --
	( 52.56,107.66) --
	( 54.10,107.66) --
	( 55.63,107.66) --
	( 57.17,107.66) --
	( 58.71,107.66) --
	( 60.24,107.66) --
	( 61.78,107.66) --
	( 63.32,107.66) --
	( 64.86,107.66) --
	( 66.39,107.66) --
	( 67.93,107.79) --
	( 69.47,108.30) --
	( 71.00,109.20) --
	( 72.54,110.48) --
	( 74.08,112.15) --
	( 75.61,114.21) --
	( 77.15,116.65) --
	( 78.69,119.47) --
	( 80.23,122.68) --
	( 81.76,126.27) --
	( 83.30,130.25) --
	( 84.84,134.62) --
	( 86.37,139.35) --
	( 87.91,143.96) --
	( 89.45,148.18) --
	( 90.98,152.03) --
	( 92.52,155.50) --
	( 94.06,158.60) --
	( 95.60,161.31) --
	( 97.13,163.65) --
	( 98.67,165.61) --
	(100.21,167.19) --
	(101.74,168.39) --
	(103.28,169.21) --
	(104.82,169.65) --
	(106.35,169.72) --
	(107.89,169.41) --
	(109.43,168.72) --
	(110.96,167.65) --
	(112.50,166.20) --
	(114.04,164.38) --
	(115.58,162.23) --
	(117.11,160.06) --
	(118.65,157.95) --
	(120.19,155.87) --
	(121.72,153.84) --
	(123.26,151.86) --
	(124.80,149.92) --
	(126.33,148.02) --
	(127.87,146.16) --
	(129.41,144.35) --
	(130.95,142.58) --
	(132.48,140.86) --
	(134.02,139.18) --
	(135.56,137.54) --
	(137.09,135.95) --
	(138.63,134.40) --
	(140.17,132.89) --
	(141.70,131.43) --
	(143.24,130.01) --
	(144.78,128.63) --
	(146.32,127.30) --
	(147.85,126.01) --
	(149.39,124.77) --
	(150.93,123.57) --
	(152.46,122.41) --
	(154.00,121.30) --
	(155.54,120.23) --
	(157.07,119.20) --
	(158.61,118.22) --
	(160.15,117.28) --
	(161.69,116.39) --
	(163.22,115.54) --
	(164.76,114.73) --
	(166.30,113.97) --
	(167.83,113.25) --
	(169.37,112.57) --
	(170.91,111.94) --
	(172.44,111.35) --
	(173.98,110.80) --
	(175.52,110.30) --
	(177.05,109.84) --
	(178.59,109.43) --
	(180.13,109.06) --
	(181.67,108.73) --
	(183.20,108.44) --
	(184.74,108.20) --
	(186.28,108.01) --
	(187.81,107.85) --
	(189.35,107.75) --
	(190.89,107.68);
\definecolor{drawColor}{RGB}{255,152,0}

\path[draw=drawColor,line width= 1.1pt,line join=round] ( 40.26,107.66) --
	( 41.80,107.66) --
	( 43.34,107.66) --
	( 44.88,107.66) --
	( 46.41,107.66) --
	( 47.95,107.66) --
	( 49.49,107.66) --
	( 51.02,107.66) --
	( 52.56,107.66) --
	( 54.10,107.66) --
	( 55.63,107.66) --
	( 57.17,107.66) --
	( 58.71,107.66) --
	( 60.24,107.66) --
	( 61.78,107.66) --
	( 63.32,107.66) --
	( 64.86,107.66) --
	( 66.39,107.66) --
	( 67.93,107.66) --
	( 69.47,107.66) --
	( 71.00,107.66) --
	( 72.54,107.66) --
	( 74.08,107.66) --
	( 75.61,107.66) --
	( 77.15,107.66) --
	( 78.69,107.66) --
	( 80.23,107.66) --
	( 81.76,107.66) --
	( 83.30,107.66) --
	( 84.84,107.66) --
	( 86.37,107.66) --
	( 87.91,107.74) --
	( 89.45,107.94) --
	( 90.98,108.26) --
	( 92.52,108.69) --
	( 94.06,109.25) --
	( 95.60,109.92) --
	( 97.13,110.71) --
	( 98.67,111.62) --
	(100.21,112.64) --
	(101.74,113.78) --
	(103.28,115.04) --
	(104.82,116.42) --
	(106.35,117.92) --
	(107.89,119.53) --
	(109.43,121.27) --
	(110.96,123.12) --
	(112.50,125.09) --
	(114.04,127.17) --
	(115.58,129.35) --
	(117.11,131.48) --
	(118.65,133.54) --
	(120.19,135.52) --
	(121.72,137.44) --
	(123.26,139.27) --
	(124.80,141.04) --
	(126.33,142.72) --
	(127.87,144.34) --
	(129.41,145.88) --
	(130.95,147.35) --
	(132.48,148.75) --
	(134.02,150.07) --
	(135.56,151.32) --
	(137.09,152.49) --
	(138.63,153.59) --
	(140.17,154.62) --
	(141.70,155.58) --
	(143.24,156.46) --
	(144.78,157.26) --
	(146.32,158.00) --
	(147.85,158.66) --
	(149.39,159.24) --
	(150.93,159.75) --
	(152.46,160.19) --
	(154.00,160.56) --
	(155.54,160.85) --
	(157.07,161.07) --
	(158.61,161.21) --
	(160.15,161.29) --
	(161.69,161.28) --
	(163.22,161.21) --
	(164.76,161.06) --
	(166.30,160.84) --
	(167.83,160.54) --
	(169.37,160.17) --
	(170.91,159.73) --
	(172.44,159.21) --
	(173.98,158.62) --
	(175.52,157.95) --
	(177.05,157.22) --
	(178.59,156.40) --
	(180.13,155.52) --
	(181.67,154.56) --
	(183.20,153.53) --
	(184.74,152.42) --
	(186.28,151.24) --
	(187.81,149.99) --
	(189.35,148.66) --
	(190.89,147.26);
\definecolor{drawColor}{RGB}{121,84,71}

\path[draw=drawColor,line width= 1.1pt,line join=round] ( 40.26,107.66) --
	( 41.80,107.66) --
	( 43.34,107.66) --
	( 44.88,107.66) --
	( 46.41,107.66) --
	( 47.95,107.66) --
	( 49.49,107.66) --
	( 51.02,107.66) --
	( 52.56,107.66) --
	( 54.10,107.66) --
	( 55.63,107.66) --
	( 57.17,107.66) --
	( 58.71,107.66) --
	( 60.24,107.66) --
	( 61.78,107.66) --
	( 63.32,107.66) --
	( 64.86,107.66) --
	( 66.39,107.66) --
	( 67.93,107.66) --
	( 69.47,107.66) --
	( 71.00,107.66) --
	( 72.54,107.66) --
	( 74.08,107.66) --
	( 75.61,107.66) --
	( 77.15,107.66) --
	( 78.69,107.66) --
	( 80.23,107.66) --
	( 81.76,107.66) --
	( 83.30,107.66) --
	( 84.84,107.66) --
	( 86.37,107.66) --
	( 87.91,107.66) --
	( 89.45,107.66) --
	( 90.98,107.66) --
	( 92.52,107.66) --
	( 94.06,107.66) --
	( 95.60,107.66) --
	( 97.13,107.66) --
	( 98.67,107.66) --
	(100.21,107.66) --
	(101.74,107.66) --
	(103.28,107.66) --
	(104.82,107.66) --
	(106.35,107.66) --
	(107.89,107.66) --
	(109.43,107.66) --
	(110.96,107.66) --
	(112.50,107.66) --
	(114.04,107.66) --
	(115.58,107.66) --
	(117.11,107.69) --
	(118.65,107.75) --
	(120.19,107.84) --
	(121.72,107.96) --
	(123.26,108.11) --
	(124.80,108.29) --
	(126.33,108.50) --
	(127.87,108.74) --
	(129.41,109.01) --
	(130.95,109.31) --
	(132.48,109.63) --
	(134.02,109.99) --
	(135.56,110.38) --
	(137.09,110.80) --
	(138.63,111.25) --
	(140.17,111.73) --
	(141.70,112.24) --
	(143.24,112.77) --
	(144.78,113.34) --
	(146.32,113.94) --
	(147.85,114.57) --
	(149.39,115.23) --
	(150.93,115.91) --
	(152.46,116.63) --
	(154.00,117.38) --
	(155.54,118.16) --
	(157.07,118.97) --
	(158.61,119.80) --
	(160.15,120.67) --
	(161.69,121.57) --
	(163.22,122.49) --
	(164.76,123.45) --
	(166.30,124.44) --
	(167.83,125.45) --
	(169.37,126.50) --
	(170.91,127.58) --
	(172.44,128.68) --
	(173.98,129.82) --
	(175.52,130.99) --
	(177.05,132.18) --
	(178.59,133.41) --
	(180.13,134.67) --
	(181.67,135.95) --
	(183.20,137.27) --
	(184.74,138.61) --
	(186.28,139.99) --
	(187.81,141.39) --
	(189.35,142.83) --
	(190.89,144.30);
\end{scope}
\begin{scope}
\path[clip] ( 32.73, 21.76) rectangle (198.42, 98.66);
\definecolor{drawColor}{gray}{0.92}

\path[draw=drawColor,line width= 0.3pt,line join=round] ( 32.73, 34.79) --
	(198.42, 34.79);

\path[draw=drawColor,line width= 0.3pt,line join=round] ( 32.73, 53.85) --
	(198.42, 53.85);

\path[draw=drawColor,line width= 0.3pt,line join=round] ( 32.73, 72.92) --
	(198.42, 72.92);

\path[draw=drawColor,line width= 0.3pt,line join=round] ( 32.73, 91.99) --
	(198.42, 91.99);

\path[draw=drawColor,line width= 0.3pt,line join=round] ( 57.94, 21.76) --
	( 57.94, 98.66);

\path[draw=drawColor,line width= 0.3pt,line join=round] ( 96.36, 21.76) --
	( 96.36, 98.66);

\path[draw=drawColor,line width= 0.3pt,line join=round] (134.79, 21.76) --
	(134.79, 98.66);

\path[draw=drawColor,line width= 0.3pt,line join=round] (173.21, 21.76) --
	(173.21, 98.66);

\path[draw=drawColor,line width= 0.6pt,line join=round] ( 32.73, 25.25) --
	(198.42, 25.25);

\path[draw=drawColor,line width= 0.6pt,line join=round] ( 32.73, 44.32) --
	(198.42, 44.32);

\path[draw=drawColor,line width= 0.6pt,line join=round] ( 32.73, 63.39) --
	(198.42, 63.39);

\path[draw=drawColor,line width= 0.6pt,line join=round] ( 32.73, 82.45) --
	(198.42, 82.45);

\path[draw=drawColor,line width= 0.6pt,line join=round] ( 38.73, 21.76) --
	( 38.73, 98.66);

\path[draw=drawColor,line width= 0.6pt,line join=round] ( 77.15, 21.76) --
	( 77.15, 98.66);

\path[draw=drawColor,line width= 0.6pt,line join=round] (115.58, 21.76) --
	(115.58, 98.66);

\path[draw=drawColor,line width= 0.6pt,line join=round] (154.00, 21.76) --
	(154.00, 98.66);

\path[draw=drawColor,line width= 0.6pt,line join=round] (192.42, 21.76) --
	(192.42, 98.66);
\definecolor{drawColor}{RGB}{155,38,176}

\path[draw=drawColor,line width= 1.1pt,line join=round] ( 40.26, 85.82) --
	( 41.80, 64.95) --
	( 43.34, 48.47) --
	( 44.88, 36.38) --
	( 46.41, 28.69) --
	( 47.95, 25.39) --
	( 49.49, 25.25) --
	( 51.02, 25.25) --
	( 52.56, 25.25) --
	( 54.10, 25.25) --
	( 55.63, 25.25) --
	( 57.17, 25.25) --
	( 58.71, 25.25) --
	( 60.24, 25.25) --
	( 61.78, 25.25) --
	( 63.32, 25.25) --
	( 64.86, 25.25) --
	( 66.39, 25.25) --
	( 67.93, 25.25) --
	( 69.47, 25.25) --
	( 71.00, 25.25) --
	( 72.54, 25.25) --
	( 74.08, 25.25) --
	( 75.61, 25.25) --
	( 77.15, 25.25) --
	( 78.69, 25.25) --
	( 80.23, 25.25) --
	( 81.76, 25.25) --
	( 83.30, 25.25) --
	( 84.84, 25.25) --
	( 86.37, 25.25) --
	( 87.91, 25.25) --
	( 89.45, 25.25) --
	( 90.98, 25.25) --
	( 92.52, 25.25) --
	( 94.06, 25.25) --
	( 95.60, 25.25) --
	( 97.13, 25.25) --
	( 98.67, 25.25) --
	(100.21, 25.25) --
	(101.74, 25.25) --
	(103.28, 25.25) --
	(104.82, 25.25) --
	(106.35, 25.25) --
	(107.89, 25.25) --
	(109.43, 25.25) --
	(110.96, 25.25) --
	(112.50, 25.25) --
	(114.04, 25.25) --
	(115.58, 25.25) --
	(117.11, 25.25) --
	(118.65, 25.25) --
	(120.19, 25.25) --
	(121.72, 25.25) --
	(123.26, 25.25) --
	(124.80, 25.25) --
	(126.33, 25.25) --
	(127.87, 25.25) --
	(129.41, 25.25) --
	(130.95, 25.25) --
	(132.48, 25.25) --
	(134.02, 25.25) --
	(135.56, 25.25) --
	(137.09, 25.25) --
	(138.63, 25.25) --
	(140.17, 25.25) --
	(141.70, 25.25) --
	(143.24, 25.25) --
	(144.78, 25.25) --
	(146.32, 25.25) --
	(147.85, 25.25) --
	(149.39, 25.25) --
	(150.93, 25.25) --
	(152.46, 25.25) --
	(154.00, 25.25) --
	(155.54, 25.25) --
	(157.07, 25.25) --
	(158.61, 25.25) --
	(160.15, 25.25) --
	(161.69, 25.25) --
	(163.22, 25.25) --
	(164.76, 25.25) --
	(166.30, 25.25) --
	(167.83, 25.25) --
	(169.37, 25.25) --
	(170.91, 25.25) --
	(172.44, 25.25) --
	(173.98, 25.25) --
	(175.52, 25.25) --
	(177.05, 25.25) --
	(178.59, 25.25) --
	(180.13, 25.25) --
	(181.67, 25.25) --
	(183.20, 25.25) --
	(184.74, 25.25) --
	(186.28, 25.25) --
	(187.81, 25.25) --
	(189.35, 25.25) --
	(190.89, 25.25);
\definecolor{drawColor}{RGB}{63,81,180}

\path[draw=drawColor,line width= 1.1pt,line join=round] ( 40.26, 40.89) --
	( 41.80, 60.90) --
	( 43.34, 75.45) --
	( 44.88, 84.52) --
	( 46.41, 88.14) --
	( 47.95, 86.29) --
	( 49.49, 80.60) --
	( 51.02, 75.02) --
	( 52.56, 69.74) --
	( 54.10, 64.76) --
	( 55.63, 60.07) --
	( 57.17, 55.68) --
	( 58.71, 51.58) --
	( 60.24, 47.78) --
	( 61.78, 44.28) --
	( 63.32, 41.07) --
	( 64.86, 38.16) --
	( 66.39, 35.54) --
	( 67.93, 33.22) --
	( 69.47, 31.20) --
	( 71.00, 29.47) --
	( 72.54, 28.04) --
	( 74.08, 26.90) --
	( 75.61, 26.06) --
	( 77.15, 25.52) --
	( 78.69, 25.27) --
	( 80.23, 25.25) --
	( 81.76, 25.25) --
	( 83.30, 25.25) --
	( 84.84, 25.25) --
	( 86.37, 25.25) --
	( 87.91, 25.25) --
	( 89.45, 25.25) --
	( 90.98, 25.25) --
	( 92.52, 25.25) --
	( 94.06, 25.25) --
	( 95.60, 25.25) --
	( 97.13, 25.25) --
	( 98.67, 25.25) --
	(100.21, 25.25) --
	(101.74, 25.25) --
	(103.28, 25.25) --
	(104.82, 25.25) --
	(106.35, 25.25) --
	(107.89, 25.25) --
	(109.43, 25.25) --
	(110.96, 25.25) --
	(112.50, 25.25) --
	(114.04, 25.25) --
	(115.58, 25.25) --
	(117.11, 25.25) --
	(118.65, 25.25) --
	(120.19, 25.25) --
	(121.72, 25.25) --
	(123.26, 25.25) --
	(124.80, 25.25) --
	(126.33, 25.25) --
	(127.87, 25.25) --
	(129.41, 25.25) --
	(130.95, 25.25) --
	(132.48, 25.25) --
	(134.02, 25.25) --
	(135.56, 25.25) --
	(137.09, 25.25) --
	(138.63, 25.25) --
	(140.17, 25.25) --
	(141.70, 25.25) --
	(143.24, 25.25) --
	(144.78, 25.25) --
	(146.32, 25.25) --
	(147.85, 25.25) --
	(149.39, 25.25) --
	(150.93, 25.25) --
	(152.46, 25.25) --
	(154.00, 25.25) --
	(155.54, 25.25) --
	(157.07, 25.25) --
	(158.61, 25.25) --
	(160.15, 25.25) --
	(161.69, 25.25) --
	(163.22, 25.25) --
	(164.76, 25.25) --
	(166.30, 25.25) --
	(167.83, 25.25) --
	(169.37, 25.25) --
	(170.91, 25.25) --
	(172.44, 25.25) --
	(173.98, 25.25) --
	(175.52, 25.25) --
	(177.05, 25.25) --
	(178.59, 25.25) --
	(180.13, 25.25) --
	(181.67, 25.25) --
	(183.20, 25.25) --
	(184.74, 25.25) --
	(186.28, 25.25) --
	(187.81, 25.25) --
	(189.35, 25.25) --
	(190.89, 25.25);
\definecolor{drawColor}{RGB}{2,169,243}

\path[draw=drawColor,line width= 1.1pt,line join=round] ( 40.26, 25.31) --
	( 41.80, 26.18) --
	( 43.34, 28.11) --
	( 44.88, 31.12) --
	( 46.41, 35.20) --
	( 47.95, 40.34) --
	( 49.49, 46.14) --
	( 51.02, 51.56) --
	( 52.56, 56.56) --
	( 54.10, 61.13) --
	( 55.63, 65.29) --
	( 57.17, 69.03) --
	( 58.71, 72.34) --
	( 60.24, 75.23) --
	( 61.78, 77.71) --
	( 63.32, 79.76) --
	( 64.86, 81.39) --
	( 66.39, 82.60) --
	( 67.93, 83.39) --
	( 69.47, 83.75) --
	( 71.00, 83.70) --
	( 72.54, 83.23) --
	( 74.08, 82.33) --
	( 75.61, 81.01) --
	( 77.15, 79.27) --
	( 78.69, 77.12) --
	( 80.23, 74.64) --
	( 81.76, 72.20) --
	( 83.30, 69.82) --
	( 84.84, 67.51) --
	( 86.37, 65.25) --
	( 87.91, 63.06) --
	( 89.45, 60.93) --
	( 90.98, 58.86) --
	( 92.52, 56.85) --
	( 94.06, 54.91) --
	( 95.60, 53.02) --
	( 97.13, 51.20) --
	( 98.67, 49.44) --
	(100.21, 47.74) --
	(101.74, 46.11) --
	(103.28, 44.53) --
	(104.82, 43.02) --
	(106.35, 41.57) --
	(107.89, 40.18) --
	(109.43, 38.85) --
	(110.96, 37.59) --
	(112.50, 36.38) --
	(114.04, 35.24) --
	(115.58, 34.16) --
	(117.11, 33.14) --
	(118.65, 32.19) --
	(120.19, 31.29) --
	(121.72, 30.46) --
	(123.26, 29.69) --
	(124.80, 28.98) --
	(126.33, 28.33) --
	(127.87, 27.74) --
	(129.41, 27.22) --
	(130.95, 26.76) --
	(132.48, 26.36) --
	(134.02, 26.02) --
	(135.56, 25.74) --
	(137.09, 25.53) --
	(138.63, 25.37) --
	(140.17, 25.28) --
	(141.70, 25.25) --
	(143.24, 25.25) --
	(144.78, 25.25) --
	(146.32, 25.25) --
	(147.85, 25.25) --
	(149.39, 25.25) --
	(150.93, 25.25) --
	(152.46, 25.25) --
	(154.00, 25.25) --
	(155.54, 25.25) --
	(157.07, 25.25) --
	(158.61, 25.25) --
	(160.15, 25.25) --
	(161.69, 25.25) --
	(163.22, 25.25) --
	(164.76, 25.25) --
	(166.30, 25.25) --
	(167.83, 25.25) --
	(169.37, 25.25) --
	(170.91, 25.25) --
	(172.44, 25.25) --
	(173.98, 25.25) --
	(175.52, 25.25) --
	(177.05, 25.25) --
	(178.59, 25.25) --
	(180.13, 25.25) --
	(181.67, 25.25) --
	(183.20, 25.25) --
	(184.74, 25.25) --
	(186.28, 25.25) --
	(187.81, 25.25) --
	(189.35, 25.25) --
	(190.89, 25.25);
\definecolor{drawColor}{RGB}{0,150,135}

\path[draw=drawColor,line width= 1.1pt,line join=round] ( 40.26, 25.25) --
	( 41.80, 25.25) --
	( 43.34, 25.25) --
	( 44.88, 25.25) --
	( 46.41, 25.25) --
	( 47.95, 25.25) --
	( 49.49, 25.29) --
	( 51.02, 25.44) --
	( 52.56, 25.73) --
	( 54.10, 26.13) --
	( 55.63, 26.66) --
	( 57.17, 27.32) --
	( 58.71, 28.10) --
	( 60.24, 29.01) --
	( 61.78, 30.04) --
	( 63.32, 31.20) --
	( 64.86, 32.48) --
	( 66.39, 33.88) --
	( 67.93, 35.42) --
	( 69.47, 37.07) --
	( 71.00, 38.85) --
	( 72.54, 40.76) --
	( 74.08, 42.79) --
	( 75.61, 44.95) --
	( 77.15, 47.23) --
	( 78.69, 49.64) --
	( 80.23, 52.12) --
	( 81.76, 54.50) --
	( 83.30, 56.77) --
	( 84.84, 58.92) --
	( 86.37, 60.96) --
	( 87.91, 62.89) --
	( 89.45, 64.71) --
	( 90.98, 66.41) --
	( 92.52, 68.01) --
	( 94.06, 69.48) --
	( 95.60, 70.85) --
	( 97.13, 72.10) --
	( 98.67, 73.24) --
	(100.21, 74.27) --
	(101.74, 75.19) --
	(103.28, 75.99) --
	(104.82, 76.68) --
	(106.35, 77.25) --
	(107.89, 77.72) --
	(109.43, 78.07) --
	(110.96, 78.31) --
	(112.50, 78.43) --
	(114.04, 78.45) --
	(115.58, 78.35) --
	(117.11, 78.13) --
	(118.65, 77.81) --
	(120.19, 77.37) --
	(121.72, 76.82) --
	(123.26, 76.16) --
	(124.80, 75.38) --
	(126.33, 74.49) --
	(127.87, 73.49) --
	(129.41, 72.37) --
	(130.95, 71.15) --
	(132.48, 69.81) --
	(134.02, 68.35) --
	(135.56, 66.79) --
	(137.09, 65.11) --
	(138.63, 63.32) --
	(140.17, 61.41) --
	(141.70, 59.40) --
	(143.24, 57.36) --
	(144.78, 55.38) --
	(146.32, 53.47) --
	(147.85, 51.62) --
	(149.39, 49.83) --
	(150.93, 48.11) --
	(152.46, 46.45) --
	(154.00, 44.85) --
	(155.54, 43.31) --
	(157.07, 41.84) --
	(158.61, 40.43) --
	(160.15, 39.08) --
	(161.69, 37.79) --
	(163.22, 36.57) --
	(164.76, 35.41) --
	(166.30, 34.31) --
	(167.83, 33.28) --
	(169.37, 32.31) --
	(170.91, 31.40) --
	(172.44, 30.55) --
	(173.98, 29.77) --
	(175.52, 29.05) --
	(177.05, 28.39) --
	(178.59, 27.79) --
	(180.13, 27.26) --
	(181.67, 26.79) --
	(183.20, 26.38) --
	(184.74, 26.04) --
	(186.28, 25.75) --
	(187.81, 25.54) --
	(189.35, 25.38) --
	(190.89, 25.28);
\definecolor{drawColor}{RGB}{139,195,74}

\path[draw=drawColor,line width= 1.1pt,line join=round] ( 40.26, 25.25) --
	( 41.80, 25.25) --
	( 43.34, 25.25) --
	( 44.88, 25.25) --
	( 46.41, 25.25) --
	( 47.95, 25.25) --
	( 49.49, 25.25) --
	( 51.02, 25.25) --
	( 52.56, 25.25) --
	( 54.10, 25.25) --
	( 55.63, 25.25) --
	( 57.17, 25.25) --
	( 58.71, 25.25) --
	( 60.24, 25.25) --
	( 61.78, 25.25) --
	( 63.32, 25.25) --
	( 64.86, 25.25) --
	( 66.39, 25.25) --
	( 67.93, 25.25) --
	( 69.47, 25.25) --
	( 71.00, 25.25) --
	( 72.54, 25.25) --
	( 74.08, 25.25) --
	( 75.61, 25.25) --
	( 77.15, 25.25) --
	( 78.69, 25.25) --
	( 80.23, 25.26) --
	( 81.76, 25.32) --
	( 83.30, 25.43) --
	( 84.84, 25.59) --
	( 86.37, 25.81) --
	( 87.91, 26.07) --
	( 89.45, 26.38) --
	( 90.98, 26.75) --
	( 92.52, 27.16) --
	( 94.06, 27.63) --
	( 95.60, 28.15) --
	( 97.13, 28.72) --
	( 98.67, 29.34) --
	(100.21, 30.01) --
	(101.74, 30.73) --
	(103.28, 31.50) --
	(104.82, 32.33) --
	(106.35, 33.20) --
	(107.89, 34.13) --
	(109.43, 35.10) --
	(110.96, 36.13) --
	(112.50, 37.21) --
	(114.04, 38.34) --
	(115.58, 39.52) --
	(117.11, 40.75) --
	(118.65, 42.03) --
	(120.19, 43.36) --
	(121.72, 44.75) --
	(123.26, 46.18) --
	(124.80, 47.67) --
	(126.33, 49.20) --
	(127.87, 50.79) --
	(129.41, 52.43) --
	(130.95, 54.12) --
	(132.48, 55.86) --
	(134.02, 57.65) --
	(135.56, 59.49) --
	(137.09, 61.39) --
	(138.63, 63.33) --
	(140.17, 65.33) --
	(141.70, 67.37) --
	(143.24, 69.38) --
	(144.78, 71.24) --
	(146.32, 72.98) --
	(147.85, 74.58) --
	(149.39, 76.06) --
	(150.93, 77.39) --
	(152.46, 78.60) --
	(154.00, 79.67) --
	(155.54, 80.61) --
	(157.07, 81.42) --
	(158.61, 82.10) --
	(160.15, 82.64) --
	(161.69, 83.05) --
	(163.22, 83.32) --
	(164.76, 83.47) --
	(166.30, 83.48) --
	(167.83, 83.36) --
	(169.37, 83.11) --
	(170.91, 82.72) --
	(172.44, 82.20) --
	(173.98, 81.55) --
	(175.52, 80.77) --
	(177.05, 79.85) --
	(178.59, 78.80) --
	(180.13, 77.62) --
	(181.67, 76.30) --
	(183.20, 74.86) --
	(184.74, 73.28) --
	(186.28, 71.56) --
	(187.81, 69.72) --
	(189.35, 67.74) --
	(190.89, 65.63);
\definecolor{drawColor}{RGB}{255,235,58}

\path[draw=drawColor,line width= 1.1pt,line join=round] ( 40.26, 25.25) --
	( 41.80, 25.25) --
	( 43.34, 25.25) --
	( 44.88, 25.25) --
	( 46.41, 25.25) --
	( 47.95, 25.25) --
	( 49.49, 25.25) --
	( 51.02, 25.25) --
	( 52.56, 25.25) --
	( 54.10, 25.25) --
	( 55.63, 25.25) --
	( 57.17, 25.25) --
	( 58.71, 25.25) --
	( 60.24, 25.25) --
	( 61.78, 25.25) --
	( 63.32, 25.25) --
	( 64.86, 25.25) --
	( 66.39, 25.25) --
	( 67.93, 25.25) --
	( 69.47, 25.25) --
	( 71.00, 25.25) --
	( 72.54, 25.25) --
	( 74.08, 25.25) --
	( 75.61, 25.25) --
	( 77.15, 25.25) --
	( 78.69, 25.25) --
	( 80.23, 25.25) --
	( 81.76, 25.25) --
	( 83.30, 25.25) --
	( 84.84, 25.25) --
	( 86.37, 25.25) --
	( 87.91, 25.25) --
	( 89.45, 25.25) --
	( 90.98, 25.25) --
	( 92.52, 25.25) --
	( 94.06, 25.25) --
	( 95.60, 25.25) --
	( 97.13, 25.25) --
	( 98.67, 25.25) --
	(100.21, 25.25) --
	(101.74, 25.25) --
	(103.28, 25.25) --
	(104.82, 25.25) --
	(106.35, 25.25) --
	(107.89, 25.25) --
	(109.43, 25.25) --
	(110.96, 25.25) --
	(112.50, 25.25) --
	(114.04, 25.25) --
	(115.58, 25.25) --
	(117.11, 25.25) --
	(118.65, 25.25) --
	(120.19, 25.25) --
	(121.72, 25.25) --
	(123.26, 25.25) --
	(124.80, 25.25) --
	(126.33, 25.25) --
	(127.87, 25.25) --
	(129.41, 25.25) --
	(130.95, 25.25) --
	(132.48, 25.25) --
	(134.02, 25.25) --
	(135.56, 25.25) --
	(137.09, 25.25) --
	(138.63, 25.25) --
	(140.17, 25.25) --
	(141.70, 25.25) --
	(143.24, 25.29) --
	(144.78, 25.40) --
	(146.32, 25.57) --
	(147.85, 25.82) --
	(149.39, 26.13) --
	(150.93, 26.52) --
	(152.46, 26.98) --
	(154.00, 27.50) --
	(155.54, 28.10) --
	(157.07, 28.76) --
	(158.61, 29.50) --
	(160.15, 30.31) --
	(161.69, 31.18) --
	(163.22, 32.13) --
	(164.76, 33.14) --
	(166.30, 34.23) --
	(167.83, 35.38) --
	(169.37, 36.61) --
	(170.91, 37.91) --
	(172.44, 39.27) --
	(173.98, 40.71) --
	(175.52, 42.21) --
	(177.05, 43.79) --
	(178.59, 45.43) --
	(180.13, 47.15) --
	(181.67, 48.93) --
	(183.20, 50.79) --
	(184.74, 52.71) --
	(186.28, 54.71) --
	(187.81, 56.77) --
	(189.35, 58.91) --
	(190.89, 61.11);
\end{scope}
\begin{scope}
\path[clip] (209.80,186.57) rectangle (375.49,263.47);
\definecolor{drawColor}{gray}{0.92}

\path[draw=drawColor,line width= 0.3pt,line join=round] (209.80,199.60) --
	(375.49,199.60);

\path[draw=drawColor,line width= 0.3pt,line join=round] (209.80,218.66) --
	(375.49,218.66);

\path[draw=drawColor,line width= 0.3pt,line join=round] (209.80,237.73) --
	(375.49,237.73);

\path[draw=drawColor,line width= 0.3pt,line join=round] (209.80,256.80) --
	(375.49,256.80);

\path[draw=drawColor,line width= 0.3pt,line join=round] (235.01,186.57) --
	(235.01,263.47);

\path[draw=drawColor,line width= 0.3pt,line join=round] (273.43,186.57) --
	(273.43,263.47);

\path[draw=drawColor,line width= 0.3pt,line join=round] (311.85,186.57) --
	(311.85,263.47);

\path[draw=drawColor,line width= 0.3pt,line join=round] (350.28,186.57) --
	(350.28,263.47);

\path[draw=drawColor,line width= 0.6pt,line join=round] (209.80,190.06) --
	(375.49,190.06);

\path[draw=drawColor,line width= 0.6pt,line join=round] (209.80,209.13) --
	(375.49,209.13);

\path[draw=drawColor,line width= 0.6pt,line join=round] (209.80,228.20) --
	(375.49,228.20);

\path[draw=drawColor,line width= 0.6pt,line join=round] (209.80,247.26) --
	(375.49,247.26);

\path[draw=drawColor,line width= 0.6pt,line join=round] (215.79,186.57) --
	(215.79,263.47);

\path[draw=drawColor,line width= 0.6pt,line join=round] (254.22,186.57) --
	(254.22,263.47);

\path[draw=drawColor,line width= 0.6pt,line join=round] (292.64,186.57) --
	(292.64,263.47);

\path[draw=drawColor,line width= 0.6pt,line join=round] (331.07,186.57) --
	(331.07,263.47);

\path[draw=drawColor,line width= 0.6pt,line join=round] (369.49,186.57) --
	(369.49,263.47);
\definecolor{drawColor}{RGB}{155,38,176}

\path[draw=drawColor,line width= 1.1pt,line join=round] (217.33,226.07) --
	(218.87,224.01) --
	(220.40,222.01) --
	(221.94,220.07) --
	(223.48,218.18) --
	(225.02,216.37) --
	(226.55,214.61) --
	(228.09,212.91) --
	(229.63,211.27) --
	(231.16,209.70) --
	(232.70,208.18) --
	(234.24,206.73) --
	(235.77,205.33) --
	(237.31,204.00) --
	(238.85,202.73) --
	(240.39,201.52) --
	(241.92,200.37) --
	(243.46,199.28) --
	(245.00,198.25) --
	(246.53,197.28) --
	(248.07,196.37) --
	(249.61,195.53) --
	(251.14,194.74) --
	(252.68,194.02) --
	(254.22,193.36) --
	(255.76,192.75) --
	(257.29,192.21) --
	(258.83,191.73) --
	(260.37,191.31) --
	(261.90,190.95) --
	(263.44,190.65) --
	(264.98,190.42) --
	(266.51,190.24) --
	(268.05,190.12) --
	(269.59,190.07) --
	(271.13,190.06) --
	(272.66,190.06) --
	(274.20,190.06) --
	(275.74,190.06) --
	(277.27,190.06) --
	(278.81,190.06) --
	(280.35,190.06) --
	(281.88,190.06) --
	(283.42,190.06) --
	(284.96,190.06) --
	(286.49,190.06) --
	(288.03,190.06) --
	(289.57,190.06) --
	(291.11,190.06) --
	(292.64,190.06) --
	(294.18,190.06) --
	(295.72,190.06) --
	(297.25,190.06) --
	(298.79,190.06) --
	(300.33,190.06) --
	(301.86,190.06) --
	(303.40,190.06) --
	(304.94,190.06) --
	(306.48,190.06) --
	(308.01,190.06) --
	(309.55,190.06) --
	(311.09,190.06) --
	(312.62,190.06) --
	(314.16,190.06) --
	(315.70,190.06) --
	(317.23,190.06) --
	(318.77,190.06) --
	(320.31,190.06) --
	(321.85,190.06) --
	(323.38,190.06) --
	(324.92,190.06) --
	(326.46,190.06) --
	(327.99,190.06) --
	(329.53,190.06) --
	(331.07,190.06) --
	(332.60,190.06) --
	(334.14,190.06) --
	(335.68,190.06) --
	(337.22,190.06) --
	(338.75,190.06) --
	(340.29,190.06) --
	(341.83,190.06) --
	(343.36,190.06) --
	(344.90,190.06) --
	(346.44,190.06) --
	(347.97,190.06) --
	(349.51,190.06) --
	(351.05,190.06) --
	(352.58,190.06) --
	(354.12,190.06) --
	(355.66,190.06) --
	(357.20,190.06) --
	(358.73,190.06) --
	(360.27,190.06) --
	(361.81,190.06) --
	(363.34,190.06) --
	(364.88,190.06) --
	(366.42,190.06) --
	(367.95,190.06);
\definecolor{drawColor}{RGB}{63,81,180}

\path[draw=drawColor,line width= 1.1pt,line join=round] (217.33,230.27) --
	(218.87,232.18) --
	(220.40,233.94) --
	(221.94,235.53) --
	(223.48,236.97) --
	(225.02,238.24) --
	(226.55,239.36) --
	(228.09,240.31) --
	(229.63,241.10) --
	(231.16,241.74) --
	(232.70,242.21) --
	(234.24,242.53) --
	(235.77,242.68) --
	(237.31,242.68) --
	(238.85,242.51) --
	(240.39,242.18) --
	(241.92,241.70) --
	(243.46,241.05) --
	(245.00,240.25) --
	(246.53,239.28) --
	(248.07,238.16) --
	(249.61,236.87) --
	(251.14,235.42) --
	(252.68,233.82) --
	(254.22,232.05) --
	(255.76,230.13) --
	(257.29,228.04) --
	(258.83,225.80) --
	(260.37,223.39) --
	(261.90,220.82) --
	(263.44,218.10) --
	(264.98,215.21) --
	(266.51,212.17) --
	(268.05,208.96) --
	(269.59,205.59) --
	(271.13,202.18) --
	(272.66,199.14) --
	(274.20,196.55) --
	(275.74,194.39) --
	(277.27,192.67) --
	(278.81,191.38) --
	(280.35,190.53) --
	(281.88,190.11) --
	(283.42,190.06) --
	(284.96,190.06) --
	(286.49,190.06) --
	(288.03,190.06) --
	(289.57,190.06) --
	(291.11,190.06) --
	(292.64,190.06) --
	(294.18,190.06) --
	(295.72,190.06) --
	(297.25,190.06) --
	(298.79,190.06) --
	(300.33,190.06) --
	(301.86,190.06) --
	(303.40,190.06) --
	(304.94,190.06) --
	(306.48,190.06) --
	(308.01,190.06) --
	(309.55,190.06) --
	(311.09,190.06) --
	(312.62,190.06) --
	(314.16,190.06) --
	(315.70,190.06) --
	(317.23,190.06) --
	(318.77,190.06) --
	(320.31,190.06) --
	(321.85,190.06) --
	(323.38,190.06) --
	(324.92,190.06) --
	(326.46,190.06) --
	(327.99,190.06) --
	(329.53,190.06) --
	(331.07,190.06) --
	(332.60,190.06) --
	(334.14,190.06) --
	(335.68,190.06) --
	(337.22,190.06) --
	(338.75,190.06) --
	(340.29,190.06) --
	(341.83,190.06) --
	(343.36,190.06) --
	(344.90,190.06) --
	(346.44,190.06) --
	(347.97,190.06) --
	(349.51,190.06) --
	(351.05,190.06) --
	(352.58,190.06) --
	(354.12,190.06) --
	(355.66,190.06) --
	(357.20,190.06) --
	(358.73,190.06) --
	(360.27,190.06) --
	(361.81,190.06) --
	(363.34,190.06) --
	(364.88,190.06) --
	(366.42,190.06) --
	(367.95,190.06);
\definecolor{drawColor}{RGB}{2,169,243}

\path[draw=drawColor,line width= 1.1pt,line join=round] (217.33,190.11) --
	(218.87,190.26) --
	(220.40,190.51) --
	(221.94,190.86) --
	(223.48,191.30) --
	(225.02,191.85) --
	(226.55,192.49) --
	(228.09,193.24) --
	(229.63,194.08) --
	(231.16,195.02) --
	(232.70,196.06) --
	(234.24,197.20) --
	(235.77,198.44) --
	(237.31,199.78) --
	(238.85,201.22) --
	(240.39,202.75) --
	(241.92,204.39) --
	(243.46,206.13) --
	(245.00,207.96) --
	(246.53,209.89) --
	(248.07,211.93) --
	(249.61,214.06) --
	(251.14,216.29) --
	(252.68,218.62) --
	(254.22,221.05) --
	(255.76,223.58) --
	(257.29,226.20) --
	(258.83,228.93) --
	(260.37,231.76) --
	(261.90,234.68) --
	(263.44,237.70) --
	(264.98,240.83) --
	(266.51,244.05) --
	(268.05,247.37) --
	(269.59,250.79) --
	(271.13,253.98) --
	(272.66,255.59) --
	(274.20,255.46) --
	(275.74,253.60) --
	(277.27,250.01) --
	(278.81,244.68) --
	(280.35,237.63) --
	(281.88,228.83) --
	(283.42,218.80) --
	(284.96,210.02) --
	(286.49,202.84) --
	(288.03,197.25) --
	(289.57,193.26) --
	(291.11,190.86) --
	(292.64,190.06) --
	(294.18,190.06) --
	(295.72,190.06) --
	(297.25,190.06) --
	(298.79,190.06) --
	(300.33,190.06) --
	(301.86,190.06) --
	(303.40,190.06) --
	(304.94,190.06) --
	(306.48,190.06) --
	(308.01,190.06) --
	(309.55,190.06) --
	(311.09,190.06) --
	(312.62,190.06) --
	(314.16,190.06) --
	(315.70,190.06) --
	(317.23,190.06) --
	(318.77,190.06) --
	(320.31,190.06) --
	(321.85,190.06) --
	(323.38,190.06) --
	(324.92,190.06) --
	(326.46,190.06) --
	(327.99,190.06) --
	(329.53,190.06) --
	(331.07,190.06) --
	(332.60,190.06) --
	(334.14,190.06) --
	(335.68,190.06) --
	(337.22,190.06) --
	(338.75,190.06) --
	(340.29,190.06) --
	(341.83,190.06) --
	(343.36,190.06) --
	(344.90,190.06) --
	(346.44,190.06) --
	(347.97,190.06) --
	(349.51,190.06) --
	(351.05,190.06) --
	(352.58,190.06) --
	(354.12,190.06) --
	(355.66,190.06) --
	(357.20,190.06) --
	(358.73,190.06) --
	(360.27,190.06) --
	(361.81,190.06) --
	(363.34,190.06) --
	(364.88,190.06) --
	(366.42,190.06) --
	(367.95,190.06);
\definecolor{drawColor}{RGB}{0,150,135}

\path[draw=drawColor,line width= 1.1pt,line join=round] (217.33,190.06) --
	(218.87,190.06) --
	(220.40,190.06) --
	(221.94,190.06) --
	(223.48,190.06) --
	(225.02,190.06) --
	(226.55,190.06) --
	(228.09,190.06) --
	(229.63,190.06) --
	(231.16,190.06) --
	(232.70,190.06) --
	(234.24,190.06) --
	(235.77,190.06) --
	(237.31,190.06) --
	(238.85,190.06) --
	(240.39,190.06) --
	(241.92,190.06) --
	(243.46,190.06) --
	(245.00,190.06) --
	(246.53,190.06) --
	(248.07,190.06) --
	(249.61,190.06) --
	(251.14,190.06) --
	(252.68,190.06) --
	(254.22,190.06) --
	(255.76,190.06) --
	(257.29,190.06) --
	(258.83,190.06) --
	(260.37,190.06) --
	(261.90,190.06) --
	(263.44,190.06) --
	(264.98,190.06) --
	(266.51,190.06) --
	(268.05,190.06) --
	(269.59,190.06) --
	(271.13,190.30) --
	(272.66,191.72) --
	(274.20,194.44) --
	(275.74,198.46) --
	(277.27,203.78) --
	(278.81,210.39) --
	(280.35,218.30) --
	(281.88,227.51) --
	(283.42,237.33) --
	(284.96,244.25) --
	(286.49,247.79) --
	(288.03,247.96) --
	(289.57,244.75) --
	(291.11,238.16) --
	(292.64,228.20) --
	(294.18,217.43) --
	(295.72,208.45) --
	(297.25,201.25) --
	(298.79,195.83) --
	(300.33,192.19) --
	(301.86,190.33) --
	(303.40,190.06) --
	(304.94,190.06) --
	(306.48,190.06) --
	(308.01,190.06) --
	(309.55,190.06) --
	(311.09,190.06) --
	(312.62,190.06) --
	(314.16,190.06) --
	(315.70,190.06) --
	(317.23,190.06) --
	(318.77,190.06) --
	(320.31,190.06) --
	(321.85,190.06) --
	(323.38,190.06) --
	(324.92,190.06) --
	(326.46,190.06) --
	(327.99,190.06) --
	(329.53,190.06) --
	(331.07,190.06) --
	(332.60,190.06) --
	(334.14,190.06) --
	(335.68,190.06) --
	(337.22,190.06) --
	(338.75,190.06) --
	(340.29,190.06) --
	(341.83,190.06) --
	(343.36,190.06) --
	(344.90,190.06) --
	(346.44,190.06) --
	(347.97,190.06) --
	(349.51,190.06) --
	(351.05,190.06) --
	(352.58,190.06) --
	(354.12,190.06) --
	(355.66,190.06) --
	(357.20,190.06) --
	(358.73,190.06) --
	(360.27,190.06) --
	(361.81,190.06) --
	(363.34,190.06) --
	(364.88,190.06) --
	(366.42,190.06) --
	(367.95,190.06);
\definecolor{drawColor}{RGB}{139,195,74}

\path[draw=drawColor,line width= 1.1pt,line join=round] (217.33,190.06) --
	(218.87,190.06) --
	(220.40,190.06) --
	(221.94,190.06) --
	(223.48,190.06) --
	(225.02,190.06) --
	(226.55,190.06) --
	(228.09,190.06) --
	(229.63,190.06) --
	(231.16,190.06) --
	(232.70,190.06) --
	(234.24,190.06) --
	(235.77,190.06) --
	(237.31,190.06) --
	(238.85,190.06) --
	(240.39,190.06) --
	(241.92,190.06) --
	(243.46,190.06) --
	(245.00,190.06) --
	(246.53,190.06) --
	(248.07,190.06) --
	(249.61,190.06) --
	(251.14,190.06) --
	(252.68,190.06) --
	(254.22,190.06) --
	(255.76,190.06) --
	(257.29,190.06) --
	(258.83,190.06) --
	(260.37,190.06) --
	(261.90,190.06) --
	(263.44,190.06) --
	(264.98,190.06) --
	(266.51,190.06) --
	(268.05,190.06) --
	(269.59,190.06) --
	(271.13,190.06) --
	(272.66,190.06) --
	(274.20,190.06) --
	(275.74,190.06) --
	(277.27,190.06) --
	(278.81,190.06) --
	(280.35,190.06) --
	(281.88,190.06) --
	(283.42,190.33) --
	(284.96,192.19) --
	(286.49,195.83) --
	(288.03,201.25) --
	(289.57,208.45) --
	(291.11,217.43) --
	(292.64,228.20) --
	(294.18,238.16) --
	(295.72,244.75) --
	(297.25,247.96) --
	(298.79,247.79) --
	(300.33,244.25) --
	(301.86,237.33) --
	(303.40,227.51) --
	(304.94,218.30) --
	(306.48,210.39) --
	(308.01,203.78) --
	(309.55,198.46) --
	(311.09,194.44) --
	(312.62,191.72) --
	(314.16,190.30) --
	(315.70,190.06) --
	(317.23,190.06) --
	(318.77,190.06) --
	(320.31,190.06) --
	(321.85,190.06) --
	(323.38,190.06) --
	(324.92,190.06) --
	(326.46,190.06) --
	(327.99,190.06) --
	(329.53,190.06) --
	(331.07,190.06) --
	(332.60,190.06) --
	(334.14,190.06) --
	(335.68,190.06) --
	(337.22,190.06) --
	(338.75,190.06) --
	(340.29,190.06) --
	(341.83,190.06) --
	(343.36,190.06) --
	(344.90,190.06) --
	(346.44,190.06) --
	(347.97,190.06) --
	(349.51,190.06) --
	(351.05,190.06) --
	(352.58,190.06) --
	(354.12,190.06) --
	(355.66,190.06) --
	(357.20,190.06) --
	(358.73,190.06) --
	(360.27,190.06) --
	(361.81,190.06) --
	(363.34,190.06) --
	(364.88,190.06) --
	(366.42,190.06) --
	(367.95,190.06);
\definecolor{drawColor}{RGB}{255,235,58}

\path[draw=drawColor,line width= 1.1pt,line join=round] (217.33,190.06) --
	(218.87,190.06) --
	(220.40,190.06) --
	(221.94,190.06) --
	(223.48,190.06) --
	(225.02,190.06) --
	(226.55,190.06) --
	(228.09,190.06) --
	(229.63,190.06) --
	(231.16,190.06) --
	(232.70,190.06) --
	(234.24,190.06) --
	(235.77,190.06) --
	(237.31,190.06) --
	(238.85,190.06) --
	(240.39,190.06) --
	(241.92,190.06) --
	(243.46,190.06) --
	(245.00,190.06) --
	(246.53,190.06) --
	(248.07,190.06) --
	(249.61,190.06) --
	(251.14,190.06) --
	(252.68,190.06) --
	(254.22,190.06) --
	(255.76,190.06) --
	(257.29,190.06) --
	(258.83,190.06) --
	(260.37,190.06) --
	(261.90,190.06) --
	(263.44,190.06) --
	(264.98,190.06) --
	(266.51,190.06) --
	(268.05,190.06) --
	(269.59,190.06) --
	(271.13,190.06) --
	(272.66,190.06) --
	(274.20,190.06) --
	(275.74,190.06) --
	(277.27,190.06) --
	(278.81,190.06) --
	(280.35,190.06) --
	(281.88,190.06) --
	(283.42,190.06) --
	(284.96,190.06) --
	(286.49,190.06) --
	(288.03,190.06) --
	(289.57,190.06) --
	(291.11,190.06) --
	(292.64,190.06) --
	(294.18,190.86) --
	(295.72,193.26) --
	(297.25,197.25) --
	(298.79,202.84) --
	(300.33,210.02) --
	(301.86,218.80) --
	(303.40,228.83) --
	(304.94,237.63) --
	(306.48,244.68) --
	(308.01,250.01) --
	(309.55,253.60) --
	(311.09,255.46) --
	(312.62,255.59) --
	(314.16,253.98) --
	(315.70,250.79) --
	(317.23,247.37) --
	(318.77,244.05) --
	(320.31,240.83) --
	(321.85,237.70) --
	(323.38,234.68) --
	(324.92,231.76) --
	(326.46,228.93) --
	(327.99,226.20) --
	(329.53,223.58) --
	(331.07,221.05) --
	(332.60,218.62) --
	(334.14,216.29) --
	(335.68,214.06) --
	(337.22,211.93) --
	(338.75,209.89) --
	(340.29,207.96) --
	(341.83,206.13) --
	(343.36,204.39) --
	(344.90,202.75) --
	(346.44,201.22) --
	(347.97,199.78) --
	(349.51,198.44) --
	(351.05,197.20) --
	(352.58,196.06) --
	(354.12,195.02) --
	(355.66,194.08) --
	(357.20,193.24) --
	(358.73,192.49) --
	(360.27,191.85) --
	(361.81,191.30) --
	(363.34,190.86) --
	(364.88,190.51) --
	(366.42,190.26) --
	(367.95,190.11);
\definecolor{drawColor}{RGB}{255,152,0}

\path[draw=drawColor,line width= 1.1pt,line join=round] (217.33,190.06) --
	(218.87,190.06) --
	(220.40,190.06) --
	(221.94,190.06) --
	(223.48,190.06) --
	(225.02,190.06) --
	(226.55,190.06) --
	(228.09,190.06) --
	(229.63,190.06) --
	(231.16,190.06) --
	(232.70,190.06) --
	(234.24,190.06) --
	(235.77,190.06) --
	(237.31,190.06) --
	(238.85,190.06) --
	(240.39,190.06) --
	(241.92,190.06) --
	(243.46,190.06) --
	(245.00,190.06) --
	(246.53,190.06) --
	(248.07,190.06) --
	(249.61,190.06) --
	(251.14,190.06) --
	(252.68,190.06) --
	(254.22,190.06) --
	(255.76,190.06) --
	(257.29,190.06) --
	(258.83,190.06) --
	(260.37,190.06) --
	(261.90,190.06) --
	(263.44,190.06) --
	(264.98,190.06) --
	(266.51,190.06) --
	(268.05,190.06) --
	(269.59,190.06) --
	(271.13,190.06) --
	(272.66,190.06) --
	(274.20,190.06) --
	(275.74,190.06) --
	(277.27,190.06) --
	(278.81,190.06) --
	(280.35,190.06) --
	(281.88,190.06) --
	(283.42,190.06) --
	(284.96,190.06) --
	(286.49,190.06) --
	(288.03,190.06) --
	(289.57,190.06) --
	(291.11,190.06) --
	(292.64,190.06) --
	(294.18,190.06) --
	(295.72,190.06) --
	(297.25,190.06) --
	(298.79,190.06) --
	(300.33,190.06) --
	(301.86,190.06) --
	(303.40,190.11) --
	(304.94,190.53) --
	(306.48,191.38) --
	(308.01,192.67) --
	(309.55,194.39) --
	(311.09,196.55) --
	(312.62,199.14) --
	(314.16,202.18) --
	(315.70,205.59) --
	(317.23,208.96) --
	(318.77,212.17) --
	(320.31,215.21) --
	(321.85,218.10) --
	(323.38,220.82) --
	(324.92,223.39) --
	(326.46,225.80) --
	(327.99,228.04) --
	(329.53,230.13) --
	(331.07,232.05) --
	(332.60,233.82) --
	(334.14,235.42) --
	(335.68,236.87) --
	(337.22,238.16) --
	(338.75,239.28) --
	(340.29,240.25) --
	(341.83,241.05) --
	(343.36,241.70) --
	(344.90,242.18) --
	(346.44,242.51) --
	(347.97,242.68) --
	(349.51,242.68) --
	(351.05,242.53) --
	(352.58,242.21) --
	(354.12,241.74) --
	(355.66,241.10) --
	(357.20,240.31) --
	(358.73,239.36) --
	(360.27,238.24) --
	(361.81,236.97) --
	(363.34,235.53) --
	(364.88,233.94) --
	(366.42,232.18) --
	(367.95,230.27);
\definecolor{drawColor}{RGB}{121,84,71}

\path[draw=drawColor,line width= 1.1pt,line join=round] (217.33,190.06) --
	(218.87,190.06) --
	(220.40,190.06) --
	(221.94,190.06) --
	(223.48,190.06) --
	(225.02,190.06) --
	(226.55,190.06) --
	(228.09,190.06) --
	(229.63,190.06) --
	(231.16,190.06) --
	(232.70,190.06) --
	(234.24,190.06) --
	(235.77,190.06) --
	(237.31,190.06) --
	(238.85,190.06) --
	(240.39,190.06) --
	(241.92,190.06) --
	(243.46,190.06) --
	(245.00,190.06) --
	(246.53,190.06) --
	(248.07,190.06) --
	(249.61,190.06) --
	(251.14,190.06) --
	(252.68,190.06) --
	(254.22,190.06) --
	(255.76,190.06) --
	(257.29,190.06) --
	(258.83,190.06) --
	(260.37,190.06) --
	(261.90,190.06) --
	(263.44,190.06) --
	(264.98,190.06) --
	(266.51,190.06) --
	(268.05,190.06) --
	(269.59,190.06) --
	(271.13,190.06) --
	(272.66,190.06) --
	(274.20,190.06) --
	(275.74,190.06) --
	(277.27,190.06) --
	(278.81,190.06) --
	(280.35,190.06) --
	(281.88,190.06) --
	(283.42,190.06) --
	(284.96,190.06) --
	(286.49,190.06) --
	(288.03,190.06) --
	(289.57,190.06) --
	(291.11,190.06) --
	(292.64,190.06) --
	(294.18,190.06) --
	(295.72,190.06) --
	(297.25,190.06) --
	(298.79,190.06) --
	(300.33,190.06) --
	(301.86,190.06) --
	(303.40,190.06) --
	(304.94,190.06) --
	(306.48,190.06) --
	(308.01,190.06) --
	(309.55,190.06) --
	(311.09,190.06) --
	(312.62,190.06) --
	(314.16,190.06) --
	(315.70,190.07) --
	(317.23,190.12) --
	(318.77,190.24) --
	(320.31,190.42) --
	(321.85,190.65) --
	(323.38,190.95) --
	(324.92,191.31) --
	(326.46,191.73) --
	(327.99,192.21) --
	(329.53,192.75) --
	(331.07,193.36) --
	(332.60,194.02) --
	(334.14,194.74) --
	(335.68,195.53) --
	(337.22,196.37) --
	(338.75,197.28) --
	(340.29,198.25) --
	(341.83,199.28) --
	(343.36,200.37) --
	(344.90,201.52) --
	(346.44,202.73) --
	(347.97,204.00) --
	(349.51,205.33) --
	(351.05,206.73) --
	(352.58,208.18) --
	(354.12,209.70) --
	(355.66,211.27) --
	(357.20,212.91) --
	(358.73,214.61) --
	(360.27,216.37) --
	(361.81,218.18) --
	(363.34,220.07) --
	(364.88,222.01) --
	(366.42,224.01) --
	(367.95,226.07);
\end{scope}
\begin{scope}
\path[clip] (209.80,104.16) rectangle (375.49,181.07);
\definecolor{drawColor}{gray}{0.92}

\path[draw=drawColor,line width= 0.3pt,line join=round] (209.80,117.19) --
	(375.49,117.19);

\path[draw=drawColor,line width= 0.3pt,line join=round] (209.80,136.26) --
	(375.49,136.26);

\path[draw=drawColor,line width= 0.3pt,line join=round] (209.80,155.32) --
	(375.49,155.32);

\path[draw=drawColor,line width= 0.3pt,line join=round] (209.80,174.39) --
	(375.49,174.39);

\path[draw=drawColor,line width= 0.3pt,line join=round] (235.01,104.16) --
	(235.01,181.07);

\path[draw=drawColor,line width= 0.3pt,line join=round] (273.43,104.16) --
	(273.43,181.07);

\path[draw=drawColor,line width= 0.3pt,line join=round] (311.85,104.16) --
	(311.85,181.07);

\path[draw=drawColor,line width= 0.3pt,line join=round] (350.28,104.16) --
	(350.28,181.07);

\path[draw=drawColor,line width= 0.6pt,line join=round] (209.80,107.66) --
	(375.49,107.66);

\path[draw=drawColor,line width= 0.6pt,line join=round] (209.80,126.72) --
	(375.49,126.72);

\path[draw=drawColor,line width= 0.6pt,line join=round] (209.80,145.79) --
	(375.49,145.79);

\path[draw=drawColor,line width= 0.6pt,line join=round] (209.80,164.86) --
	(375.49,164.86);

\path[draw=drawColor,line width= 0.6pt,line join=round] (215.79,104.16) --
	(215.79,181.07);

\path[draw=drawColor,line width= 0.6pt,line join=round] (254.22,104.16) --
	(254.22,181.07);

\path[draw=drawColor,line width= 0.6pt,line join=round] (292.64,104.16) --
	(292.64,181.07);

\path[draw=drawColor,line width= 0.6pt,line join=round] (331.07,104.16) --
	(331.07,181.07);

\path[draw=drawColor,line width= 0.6pt,line join=round] (369.49,104.16) --
	(369.49,181.07);
\definecolor{drawColor}{RGB}{155,38,176}

\path[draw=drawColor,line width= 1.1pt,line join=round] (217.33,141.35) --
	(218.87,137.19) --
	(220.40,133.30) --
	(221.94,129.68) --
	(223.48,126.34) --
	(225.02,123.28) --
	(226.55,120.49) --
	(228.09,117.97) --
	(229.63,115.73) --
	(231.16,113.76) --
	(232.70,112.07) --
	(234.24,110.65) --
	(235.77,109.50) --
	(237.31,108.63) --
	(238.85,108.04) --
	(240.39,107.72) --
	(241.92,107.66) --
	(243.46,107.66) --
	(245.00,107.66) --
	(246.53,107.66) --
	(248.07,107.66) --
	(249.61,107.66) --
	(251.14,107.66) --
	(252.68,107.66) --
	(254.22,107.66) --
	(255.76,107.66) --
	(257.29,107.66) --
	(258.83,107.66) --
	(260.37,107.66) --
	(261.90,107.66) --
	(263.44,107.66) --
	(264.98,107.66) --
	(266.51,107.66) --
	(268.05,107.66) --
	(269.59,107.66) --
	(271.13,107.66) --
	(272.66,107.66) --
	(274.20,107.66) --
	(275.74,107.66) --
	(277.27,107.66) --
	(278.81,107.66) --
	(280.35,107.66) --
	(281.88,107.66) --
	(283.42,107.66) --
	(284.96,107.66) --
	(286.49,107.66) --
	(288.03,107.66) --
	(289.57,107.66) --
	(291.11,107.66) --
	(292.64,107.66) --
	(294.18,107.66) --
	(295.72,107.66) --
	(297.25,107.66) --
	(298.79,107.66) --
	(300.33,107.66) --
	(301.86,107.66) --
	(303.40,107.66) --
	(304.94,107.66) --
	(306.48,107.66) --
	(308.01,107.66) --
	(309.55,107.66) --
	(311.09,107.66) --
	(312.62,107.66) --
	(314.16,107.66) --
	(315.70,107.66) --
	(317.23,107.66) --
	(318.77,107.66) --
	(320.31,107.66) --
	(321.85,107.66) --
	(323.38,107.66) --
	(324.92,107.66) --
	(326.46,107.66) --
	(327.99,107.66) --
	(329.53,107.66) --
	(331.07,107.66) --
	(332.60,107.66) --
	(334.14,107.66) --
	(335.68,107.66) --
	(337.22,107.66) --
	(338.75,107.66) --
	(340.29,107.66) --
	(341.83,107.66) --
	(343.36,107.66) --
	(344.90,107.66) --
	(346.44,107.66) --
	(347.97,107.66) --
	(349.51,107.66) --
	(351.05,107.66) --
	(352.58,107.66) --
	(354.12,107.66) --
	(355.66,107.66) --
	(357.20,107.66) --
	(358.73,107.66) --
	(360.27,107.66) --
	(361.81,107.66) --
	(363.34,107.66) --
	(364.88,107.66) --
	(366.42,107.66) --
	(367.95,107.66);
\definecolor{drawColor}{RGB}{63,81,180}

\path[draw=drawColor,line width= 1.1pt,line join=round] (217.33,150.09) --
	(218.87,153.84) --
	(220.40,157.05) --
	(221.94,159.70) --
	(223.48,161.81) --
	(225.02,163.36) --
	(226.55,164.37) --
	(228.09,164.83) --
	(229.63,164.74) --
	(231.16,164.09) --
	(232.70,162.90) --
	(234.24,161.17) --
	(235.77,158.88) --
	(237.31,156.04) --
	(238.85,152.65) --
	(240.39,148.72) --
	(241.92,144.28) --
	(243.46,139.93) --
	(245.00,135.86) --
	(246.53,132.06) --
	(248.07,128.54) --
	(249.61,125.29) --
	(251.14,122.32) --
	(252.68,119.62) --
	(254.22,117.19) --
	(255.76,115.04) --
	(257.29,113.16) --
	(258.83,111.56) --
	(260.37,110.24) --
	(261.90,109.18) --
	(263.44,108.41) --
	(264.98,107.90) --
	(266.51,107.67) --
	(268.05,107.66) --
	(269.59,107.66) --
	(271.13,107.66) --
	(272.66,107.66) --
	(274.20,107.66) --
	(275.74,107.66) --
	(277.27,107.66) --
	(278.81,107.66) --
	(280.35,107.66) --
	(281.88,107.66) --
	(283.42,107.66) --
	(284.96,107.66) --
	(286.49,107.66) --
	(288.03,107.66) --
	(289.57,107.66) --
	(291.11,107.66) --
	(292.64,107.66) --
	(294.18,107.66) --
	(295.72,107.66) --
	(297.25,107.66) --
	(298.79,107.66) --
	(300.33,107.66) --
	(301.86,107.66) --
	(303.40,107.66) --
	(304.94,107.66) --
	(306.48,107.66) --
	(308.01,107.66) --
	(309.55,107.66) --
	(311.09,107.66) --
	(312.62,107.66) --
	(314.16,107.66) --
	(315.70,107.66) --
	(317.23,107.66) --
	(318.77,107.66) --
	(320.31,107.66) --
	(321.85,107.66) --
	(323.38,107.66) --
	(324.92,107.66) --
	(326.46,107.66) --
	(327.99,107.66) --
	(329.53,107.66) --
	(331.07,107.66) --
	(332.60,107.66) --
	(334.14,107.66) --
	(335.68,107.66) --
	(337.22,107.66) --
	(338.75,107.66) --
	(340.29,107.66) --
	(341.83,107.66) --
	(343.36,107.66) --
	(344.90,107.66) --
	(346.44,107.66) --
	(347.97,107.66) --
	(349.51,107.66) --
	(351.05,107.66) --
	(352.58,107.66) --
	(354.12,107.66) --
	(355.66,107.66) --
	(357.20,107.66) --
	(358.73,107.66) --
	(360.27,107.66) --
	(361.81,107.66) --
	(363.34,107.66) --
	(364.88,107.66) --
	(366.42,107.66) --
	(367.95,107.66);
\definecolor{drawColor}{RGB}{2,169,243}

\path[draw=drawColor,line width= 1.1pt,line join=round] (217.33,107.80) --
	(218.87,108.21) --
	(220.40,108.89) --
	(221.94,109.85) --
	(223.48,111.09) --
	(225.02,112.60) --
	(226.55,114.38) --
	(228.09,116.44) --
	(229.63,118.78) --
	(231.16,121.39) --
	(232.70,124.27) --
	(234.24,127.43) --
	(235.77,130.86) --
	(237.31,134.56) --
	(238.85,138.55) --
	(240.39,142.80) --
	(241.92,147.29) --
	(243.46,151.40) --
	(245.00,154.97) --
	(246.53,157.99) --
	(248.07,160.46) --
	(249.61,162.39) --
	(251.14,163.76) --
	(252.68,164.58) --
	(254.22,164.86) --
	(255.76,164.58) --
	(257.29,163.76) --
	(258.83,162.39) --
	(260.37,160.46) --
	(261.90,157.99) --
	(263.44,154.97) --
	(264.98,151.40) --
	(266.51,147.29) --
	(268.05,142.80) --
	(269.59,138.55) --
	(271.13,134.56) --
	(272.66,130.86) --
	(274.20,127.43) --
	(275.74,124.27) --
	(277.27,121.39) --
	(278.81,118.78) --
	(280.35,116.44) --
	(281.88,114.38) --
	(283.42,112.60) --
	(284.96,111.09) --
	(286.49,109.85) --
	(288.03,108.89) --
	(289.57,108.21) --
	(291.11,107.80) --
	(292.64,107.66) --
	(294.18,107.66) --
	(295.72,107.66) --
	(297.25,107.66) --
	(298.79,107.66) --
	(300.33,107.66) --
	(301.86,107.66) --
	(303.40,107.66) --
	(304.94,107.66) --
	(306.48,107.66) --
	(308.01,107.66) --
	(309.55,107.66) --
	(311.09,107.66) --
	(312.62,107.66) --
	(314.16,107.66) --
	(315.70,107.66) --
	(317.23,107.66) --
	(318.77,107.66) --
	(320.31,107.66) --
	(321.85,107.66) --
	(323.38,107.66) --
	(324.92,107.66) --
	(326.46,107.66) --
	(327.99,107.66) --
	(329.53,107.66) --
	(331.07,107.66) --
	(332.60,107.66) --
	(334.14,107.66) --
	(335.68,107.66) --
	(337.22,107.66) --
	(338.75,107.66) --
	(340.29,107.66) --
	(341.83,107.66) --
	(343.36,107.66) --
	(344.90,107.66) --
	(346.44,107.66) --
	(347.97,107.66) --
	(349.51,107.66) --
	(351.05,107.66) --
	(352.58,107.66) --
	(354.12,107.66) --
	(355.66,107.66) --
	(357.20,107.66) --
	(358.73,107.66) --
	(360.27,107.66) --
	(361.81,107.66) --
	(363.34,107.66) --
	(364.88,107.66) --
	(366.42,107.66) --
	(367.95,107.66);
\definecolor{drawColor}{RGB}{0,150,135}

\path[draw=drawColor,line width= 1.1pt,line join=round] (217.33,107.66) --
	(218.87,107.66) --
	(220.40,107.66) --
	(221.94,107.66) --
	(223.48,107.66) --
	(225.02,107.66) --
	(226.55,107.66) --
	(228.09,107.66) --
	(229.63,107.66) --
	(231.16,107.66) --
	(232.70,107.66) --
	(234.24,107.66) --
	(235.77,107.66) --
	(237.31,107.66) --
	(238.85,107.66) --
	(240.39,107.66) --
	(241.92,107.67) --
	(243.46,107.90) --
	(245.00,108.41) --
	(246.53,109.18) --
	(248.07,110.24) --
	(249.61,111.56) --
	(251.14,113.16) --
	(252.68,115.04) --
	(254.22,117.19) --
	(255.76,119.62) --
	(257.29,122.32) --
	(258.83,125.29) --
	(260.37,128.54) --
	(261.90,132.06) --
	(263.44,135.86) --
	(264.98,139.93) --
	(266.51,144.28) --
	(268.05,148.72) --
	(269.59,152.65) --
	(271.13,156.04) --
	(272.66,158.88) --
	(274.20,161.17) --
	(275.74,162.90) --
	(277.27,164.09) --
	(278.81,164.74) --
	(280.35,164.83) --
	(281.88,164.37) --
	(283.42,163.36) --
	(284.96,161.81) --
	(286.49,159.70) --
	(288.03,157.05) --
	(289.57,153.84) --
	(291.11,150.09) --
	(292.64,145.79) --
	(294.18,141.35) --
	(295.72,137.19) --
	(297.25,133.30) --
	(298.79,129.68) --
	(300.33,126.34) --
	(301.86,123.28) --
	(303.40,120.49) --
	(304.94,117.97) --
	(306.48,115.73) --
	(308.01,113.76) --
	(309.55,112.07) --
	(311.09,110.65) --
	(312.62,109.50) --
	(314.16,108.63) --
	(315.70,108.04) --
	(317.23,107.72) --
	(318.77,107.66) --
	(320.31,107.66) --
	(321.85,107.66) --
	(323.38,107.66) --
	(324.92,107.66) --
	(326.46,107.66) --
	(327.99,107.66) --
	(329.53,107.66) --
	(331.07,107.66) --
	(332.60,107.66) --
	(334.14,107.66) --
	(335.68,107.66) --
	(337.22,107.66) --
	(338.75,107.66) --
	(340.29,107.66) --
	(341.83,107.66) --
	(343.36,107.66) --
	(344.90,107.66) --
	(346.44,107.66) --
	(347.97,107.66) --
	(349.51,107.66) --
	(351.05,107.66) --
	(352.58,107.66) --
	(354.12,107.66) --
	(355.66,107.66) --
	(357.20,107.66) --
	(358.73,107.66) --
	(360.27,107.66) --
	(361.81,107.66) --
	(363.34,107.66) --
	(364.88,107.66) --
	(366.42,107.66) --
	(367.95,107.66);
\definecolor{drawColor}{RGB}{139,195,74}

\path[draw=drawColor,line width= 1.1pt,line join=round] (217.33,107.66) --
	(218.87,107.66) --
	(220.40,107.66) --
	(221.94,107.66) --
	(223.48,107.66) --
	(225.02,107.66) --
	(226.55,107.66) --
	(228.09,107.66) --
	(229.63,107.66) --
	(231.16,107.66) --
	(232.70,107.66) --
	(234.24,107.66) --
	(235.77,107.66) --
	(237.31,107.66) --
	(238.85,107.66) --
	(240.39,107.66) --
	(241.92,107.66) --
	(243.46,107.66) --
	(245.00,107.66) --
	(246.53,107.66) --
	(248.07,107.66) --
	(249.61,107.66) --
	(251.14,107.66) --
	(252.68,107.66) --
	(254.22,107.66) --
	(255.76,107.66) --
	(257.29,107.66) --
	(258.83,107.66) --
	(260.37,107.66) --
	(261.90,107.66) --
	(263.44,107.66) --
	(264.98,107.66) --
	(266.51,107.66) --
	(268.05,107.72) --
	(269.59,108.04) --
	(271.13,108.63) --
	(272.66,109.50) --
	(274.20,110.65) --
	(275.74,112.07) --
	(277.27,113.76) --
	(278.81,115.73) --
	(280.35,117.97) --
	(281.88,120.49) --
	(283.42,123.28) --
	(284.96,126.34) --
	(286.49,129.68) --
	(288.03,133.30) --
	(289.57,137.19) --
	(291.11,141.35) --
	(292.64,145.79) --
	(294.18,150.09) --
	(295.72,153.84) --
	(297.25,157.05) --
	(298.79,159.70) --
	(300.33,161.81) --
	(301.86,163.36) --
	(303.40,164.37) --
	(304.94,164.83) --
	(306.48,164.74) --
	(308.01,164.09) --
	(309.55,162.90) --
	(311.09,161.17) --
	(312.62,158.88) --
	(314.16,156.04) --
	(315.70,152.65) --
	(317.23,148.72) --
	(318.77,144.28) --
	(320.31,139.93) --
	(321.85,135.86) --
	(323.38,132.06) --
	(324.92,128.54) --
	(326.46,125.29) --
	(327.99,122.32) --
	(329.53,119.62) --
	(331.07,117.19) --
	(332.60,115.04) --
	(334.14,113.16) --
	(335.68,111.56) --
	(337.22,110.24) --
	(338.75,109.18) --
	(340.29,108.41) --
	(341.83,107.90) --
	(343.36,107.67) --
	(344.90,107.66) --
	(346.44,107.66) --
	(347.97,107.66) --
	(349.51,107.66) --
	(351.05,107.66) --
	(352.58,107.66) --
	(354.12,107.66) --
	(355.66,107.66) --
	(357.20,107.66) --
	(358.73,107.66) --
	(360.27,107.66) --
	(361.81,107.66) --
	(363.34,107.66) --
	(364.88,107.66) --
	(366.42,107.66) --
	(367.95,107.66);
\definecolor{drawColor}{RGB}{255,235,58}

\path[draw=drawColor,line width= 1.1pt,line join=round] (217.33,107.66) --
	(218.87,107.66) --
	(220.40,107.66) --
	(221.94,107.66) --
	(223.48,107.66) --
	(225.02,107.66) --
	(226.55,107.66) --
	(228.09,107.66) --
	(229.63,107.66) --
	(231.16,107.66) --
	(232.70,107.66) --
	(234.24,107.66) --
	(235.77,107.66) --
	(237.31,107.66) --
	(238.85,107.66) --
	(240.39,107.66) --
	(241.92,107.66) --
	(243.46,107.66) --
	(245.00,107.66) --
	(246.53,107.66) --
	(248.07,107.66) --
	(249.61,107.66) --
	(251.14,107.66) --
	(252.68,107.66) --
	(254.22,107.66) --
	(255.76,107.66) --
	(257.29,107.66) --
	(258.83,107.66) --
	(260.37,107.66) --
	(261.90,107.66) --
	(263.44,107.66) --
	(264.98,107.66) --
	(266.51,107.66) --
	(268.05,107.66) --
	(269.59,107.66) --
	(271.13,107.66) --
	(272.66,107.66) --
	(274.20,107.66) --
	(275.74,107.66) --
	(277.27,107.66) --
	(278.81,107.66) --
	(280.35,107.66) --
	(281.88,107.66) --
	(283.42,107.66) --
	(284.96,107.66) --
	(286.49,107.66) --
	(288.03,107.66) --
	(289.57,107.66) --
	(291.11,107.66) --
	(292.64,107.66) --
	(294.18,107.80) --
	(295.72,108.21) --
	(297.25,108.89) --
	(298.79,109.85) --
	(300.33,111.09) --
	(301.86,112.60) --
	(303.40,114.38) --
	(304.94,116.44) --
	(306.48,118.78) --
	(308.01,121.39) --
	(309.55,124.27) --
	(311.09,127.43) --
	(312.62,130.86) --
	(314.16,134.56) --
	(315.70,138.55) --
	(317.23,142.80) --
	(318.77,147.29) --
	(320.31,151.40) --
	(321.85,154.97) --
	(323.38,157.99) --
	(324.92,160.46) --
	(326.46,162.39) --
	(327.99,163.76) --
	(329.53,164.58) --
	(331.07,164.86) --
	(332.60,164.58) --
	(334.14,163.76) --
	(335.68,162.39) --
	(337.22,160.46) --
	(338.75,157.99) --
	(340.29,154.97) --
	(341.83,151.40) --
	(343.36,147.29) --
	(344.90,142.80) --
	(346.44,138.55) --
	(347.97,134.56) --
	(349.51,130.86) --
	(351.05,127.43) --
	(352.58,124.27) --
	(354.12,121.39) --
	(355.66,118.78) --
	(357.20,116.44) --
	(358.73,114.38) --
	(360.27,112.60) --
	(361.81,111.09) --
	(363.34,109.85) --
	(364.88,108.89) --
	(366.42,108.21) --
	(367.95,107.80);
\definecolor{drawColor}{RGB}{255,152,0}

\path[draw=drawColor,line width= 1.1pt,line join=round] (217.33,107.66) --
	(218.87,107.66) --
	(220.40,107.66) --
	(221.94,107.66) --
	(223.48,107.66) --
	(225.02,107.66) --
	(226.55,107.66) --
	(228.09,107.66) --
	(229.63,107.66) --
	(231.16,107.66) --
	(232.70,107.66) --
	(234.24,107.66) --
	(235.77,107.66) --
	(237.31,107.66) --
	(238.85,107.66) --
	(240.39,107.66) --
	(241.92,107.66) --
	(243.46,107.66) --
	(245.00,107.66) --
	(246.53,107.66) --
	(248.07,107.66) --
	(249.61,107.66) --
	(251.14,107.66) --
	(252.68,107.66) --
	(254.22,107.66) --
	(255.76,107.66) --
	(257.29,107.66) --
	(258.83,107.66) --
	(260.37,107.66) --
	(261.90,107.66) --
	(263.44,107.66) --
	(264.98,107.66) --
	(266.51,107.66) --
	(268.05,107.66) --
	(269.59,107.66) --
	(271.13,107.66) --
	(272.66,107.66) --
	(274.20,107.66) --
	(275.74,107.66) --
	(277.27,107.66) --
	(278.81,107.66) --
	(280.35,107.66) --
	(281.88,107.66) --
	(283.42,107.66) --
	(284.96,107.66) --
	(286.49,107.66) --
	(288.03,107.66) --
	(289.57,107.66) --
	(291.11,107.66) --
	(292.64,107.66) --
	(294.18,107.66) --
	(295.72,107.66) --
	(297.25,107.66) --
	(298.79,107.66) --
	(300.33,107.66) --
	(301.86,107.66) --
	(303.40,107.66) --
	(304.94,107.66) --
	(306.48,107.66) --
	(308.01,107.66) --
	(309.55,107.66) --
	(311.09,107.66) --
	(312.62,107.66) --
	(314.16,107.66) --
	(315.70,107.66) --
	(317.23,107.66) --
	(318.77,107.67) --
	(320.31,107.90) --
	(321.85,108.41) --
	(323.38,109.18) --
	(324.92,110.24) --
	(326.46,111.56) --
	(327.99,113.16) --
	(329.53,115.04) --
	(331.07,117.19) --
	(332.60,119.62) --
	(334.14,122.32) --
	(335.68,125.29) --
	(337.22,128.54) --
	(338.75,132.06) --
	(340.29,135.86) --
	(341.83,139.93) --
	(343.36,144.28) --
	(344.90,148.72) --
	(346.44,152.65) --
	(347.97,156.04) --
	(349.51,158.88) --
	(351.05,161.17) --
	(352.58,162.90) --
	(354.12,164.09) --
	(355.66,164.74) --
	(357.20,164.83) --
	(358.73,164.37) --
	(360.27,163.36) --
	(361.81,161.81) --
	(363.34,159.70) --
	(364.88,157.05) --
	(366.42,153.84) --
	(367.95,150.09);
\definecolor{drawColor}{RGB}{121,84,71}

\path[draw=drawColor,line width= 1.1pt,line join=round] (217.33,107.66) --
	(218.87,107.66) --
	(220.40,107.66) --
	(221.94,107.66) --
	(223.48,107.66) --
	(225.02,107.66) --
	(226.55,107.66) --
	(228.09,107.66) --
	(229.63,107.66) --
	(231.16,107.66) --
	(232.70,107.66) --
	(234.24,107.66) --
	(235.77,107.66) --
	(237.31,107.66) --
	(238.85,107.66) --
	(240.39,107.66) --
	(241.92,107.66) --
	(243.46,107.66) --
	(245.00,107.66) --
	(246.53,107.66) --
	(248.07,107.66) --
	(249.61,107.66) --
	(251.14,107.66) --
	(252.68,107.66) --
	(254.22,107.66) --
	(255.76,107.66) --
	(257.29,107.66) --
	(258.83,107.66) --
	(260.37,107.66) --
	(261.90,107.66) --
	(263.44,107.66) --
	(264.98,107.66) --
	(266.51,107.66) --
	(268.05,107.66) --
	(269.59,107.66) --
	(271.13,107.66) --
	(272.66,107.66) --
	(274.20,107.66) --
	(275.74,107.66) --
	(277.27,107.66) --
	(278.81,107.66) --
	(280.35,107.66) --
	(281.88,107.66) --
	(283.42,107.66) --
	(284.96,107.66) --
	(286.49,107.66) --
	(288.03,107.66) --
	(289.57,107.66) --
	(291.11,107.66) --
	(292.64,107.66) --
	(294.18,107.66) --
	(295.72,107.66) --
	(297.25,107.66) --
	(298.79,107.66) --
	(300.33,107.66) --
	(301.86,107.66) --
	(303.40,107.66) --
	(304.94,107.66) --
	(306.48,107.66) --
	(308.01,107.66) --
	(309.55,107.66) --
	(311.09,107.66) --
	(312.62,107.66) --
	(314.16,107.66) --
	(315.70,107.66) --
	(317.23,107.66) --
	(318.77,107.66) --
	(320.31,107.66) --
	(321.85,107.66) --
	(323.38,107.66) --
	(324.92,107.66) --
	(326.46,107.66) --
	(327.99,107.66) --
	(329.53,107.66) --
	(331.07,107.66) --
	(332.60,107.66) --
	(334.14,107.66) --
	(335.68,107.66) --
	(337.22,107.66) --
	(338.75,107.66) --
	(340.29,107.66) --
	(341.83,107.66) --
	(343.36,107.66) --
	(344.90,107.72) --
	(346.44,108.04) --
	(347.97,108.63) --
	(349.51,109.50) --
	(351.05,110.65) --
	(352.58,112.07) --
	(354.12,113.76) --
	(355.66,115.73) --
	(357.20,117.97) --
	(358.73,120.49) --
	(360.27,123.28) --
	(361.81,126.34) --
	(363.34,129.68) --
	(364.88,133.30) --
	(366.42,137.19) --
	(367.95,141.35);
\end{scope}
\begin{scope}
\path[clip] (209.80, 21.76) rectangle (375.49, 98.66);
\definecolor{drawColor}{gray}{0.92}

\path[draw=drawColor,line width= 0.3pt,line join=round] (209.80, 34.79) --
	(375.49, 34.79);

\path[draw=drawColor,line width= 0.3pt,line join=round] (209.80, 53.85) --
	(375.49, 53.85);

\path[draw=drawColor,line width= 0.3pt,line join=round] (209.80, 72.92) --
	(375.49, 72.92);

\path[draw=drawColor,line width= 0.3pt,line join=round] (209.80, 91.99) --
	(375.49, 91.99);

\path[draw=drawColor,line width= 0.3pt,line join=round] (235.01, 21.76) --
	(235.01, 98.66);

\path[draw=drawColor,line width= 0.3pt,line join=round] (273.43, 21.76) --
	(273.43, 98.66);

\path[draw=drawColor,line width= 0.3pt,line join=round] (311.85, 21.76) --
	(311.85, 98.66);

\path[draw=drawColor,line width= 0.3pt,line join=round] (350.28, 21.76) --
	(350.28, 98.66);

\path[draw=drawColor,line width= 0.6pt,line join=round] (209.80, 25.25) --
	(375.49, 25.25);

\path[draw=drawColor,line width= 0.6pt,line join=round] (209.80, 44.32) --
	(375.49, 44.32);

\path[draw=drawColor,line width= 0.6pt,line join=round] (209.80, 63.39) --
	(375.49, 63.39);

\path[draw=drawColor,line width= 0.6pt,line join=round] (209.80, 82.45) --
	(375.49, 82.45);

\path[draw=drawColor,line width= 0.6pt,line join=round] (215.79, 21.76) --
	(215.79, 98.66);

\path[draw=drawColor,line width= 0.6pt,line join=round] (254.22, 21.76) --
	(254.22, 98.66);

\path[draw=drawColor,line width= 0.6pt,line join=round] (292.64, 21.76) --
	(292.64, 98.66);

\path[draw=drawColor,line width= 0.6pt,line join=round] (331.07, 21.76) --
	(331.07, 98.66);

\path[draw=drawColor,line width= 0.6pt,line join=round] (369.49, 21.76) --
	(369.49, 98.66);
\definecolor{drawColor}{RGB}{155,38,176}

\path[draw=drawColor,line width= 1.1pt,line join=round] (217.33, 32.40) --
	(218.87, 25.25) --
	(220.40, 25.25) --
	(221.94, 25.25) --
	(223.48, 25.25) --
	(225.02, 25.25) --
	(226.55, 25.25) --
	(228.09, 25.25) --
	(229.63, 25.25) --
	(231.16, 25.25) --
	(232.70, 25.25) --
	(234.24, 25.25) --
	(235.77, 25.25) --
	(237.31, 25.25) --
	(238.85, 25.25) --
	(240.39, 25.25) --
	(241.92, 25.25) --
	(243.46, 25.25) --
	(245.00, 25.25) --
	(246.53, 25.25) --
	(248.07, 25.25) --
	(249.61, 25.25) --
	(251.14, 25.25) --
	(252.68, 25.25) --
	(254.22, 25.25) --
	(255.76, 25.25) --
	(257.29, 25.25) --
	(258.83, 25.25) --
	(260.37, 25.25) --
	(261.90, 25.25) --
	(263.44, 25.25) --
	(264.98, 25.25) --
	(266.51, 25.25) --
	(268.05, 25.25) --
	(269.59, 25.25) --
	(271.13, 25.25) --
	(272.66, 25.25) --
	(274.20, 25.25) --
	(275.74, 25.25) --
	(277.27, 25.25) --
	(278.81, 25.25) --
	(280.35, 25.25) --
	(281.88, 25.25) --
	(283.42, 25.25) --
	(284.96, 25.25) --
	(286.49, 25.25) --
	(288.03, 25.25) --
	(289.57, 25.25) --
	(291.11, 25.25) --
	(292.64, 25.25) --
	(294.18, 25.25) --
	(295.72, 25.25) --
	(297.25, 25.25) --
	(298.79, 25.25) --
	(300.33, 25.25) --
	(301.86, 25.25) --
	(303.40, 25.25) --
	(304.94, 25.25) --
	(306.48, 25.25) --
	(308.01, 25.25) --
	(309.55, 25.25) --
	(311.09, 25.25) --
	(312.62, 25.25) --
	(314.16, 25.25) --
	(315.70, 25.25) --
	(317.23, 25.25) --
	(318.77, 25.25) --
	(320.31, 25.25) --
	(321.85, 25.25) --
	(323.38, 25.25) --
	(324.92, 25.25) --
	(326.46, 25.25) --
	(327.99, 25.25) --
	(329.53, 25.25) --
	(331.07, 25.25) --
	(332.60, 25.25) --
	(334.14, 25.25) --
	(335.68, 25.25) --
	(337.22, 25.25) --
	(338.75, 25.25) --
	(340.29, 25.25) --
	(341.83, 25.25) --
	(343.36, 25.25) --
	(344.90, 25.25) --
	(346.44, 25.25) --
	(347.97, 25.25) --
	(349.51, 25.25) --
	(351.05, 25.25) --
	(352.58, 25.25) --
	(354.12, 25.25) --
	(355.66, 25.25) --
	(357.20, 25.25) --
	(358.73, 25.25) --
	(360.27, 25.25) --
	(361.81, 25.25) --
	(363.34, 25.25) --
	(364.88, 25.25) --
	(366.42, 25.25) --
	(367.95, 25.25);
\definecolor{drawColor}{RGB}{63,81,180}

\path[draw=drawColor,line width= 1.1pt,line join=round] (217.33, 91.88) --
	(218.87, 91.69) --
	(220.40, 83.31) --
	(221.94, 75.49) --
	(223.48, 68.23) --
	(225.02, 61.54) --
	(226.55, 55.42) --
	(228.09, 49.86) --
	(229.63, 44.87) --
	(231.16, 40.44) --
	(232.70, 36.58) --
	(234.24, 33.28) --
	(235.77, 30.55) --
	(237.31, 28.39) --
	(238.85, 26.79) --
	(240.39, 25.75) --
	(241.92, 25.28) --
	(243.46, 25.25) --
	(245.00, 25.25) --
	(246.53, 25.25) --
	(248.07, 25.25) --
	(249.61, 25.25) --
	(251.14, 25.25) --
	(252.68, 25.25) --
	(254.22, 25.25) --
	(255.76, 25.25) --
	(257.29, 25.25) --
	(258.83, 25.25) --
	(260.37, 25.25) --
	(261.90, 25.25) --
	(263.44, 25.25) --
	(264.98, 25.25) --
	(266.51, 25.25) --
	(268.05, 25.25) --
	(269.59, 25.25) --
	(271.13, 25.25) --
	(272.66, 25.25) --
	(274.20, 25.25) --
	(275.74, 25.25) --
	(277.27, 25.25) --
	(278.81, 25.25) --
	(280.35, 25.25) --
	(281.88, 25.25) --
	(283.42, 25.25) --
	(284.96, 25.25) --
	(286.49, 25.25) --
	(288.03, 25.25) --
	(289.57, 25.25) --
	(291.11, 25.25) --
	(292.64, 25.25) --
	(294.18, 25.25) --
	(295.72, 25.25) --
	(297.25, 25.25) --
	(298.79, 25.25) --
	(300.33, 25.25) --
	(301.86, 25.25) --
	(303.40, 25.25) --
	(304.94, 25.25) --
	(306.48, 25.25) --
	(308.01, 25.25) --
	(309.55, 25.25) --
	(311.09, 25.25) --
	(312.62, 25.25) --
	(314.16, 25.25) --
	(315.70, 25.25) --
	(317.23, 25.25) --
	(318.77, 25.25) --
	(320.31, 25.25) --
	(321.85, 25.25) --
	(323.38, 25.25) --
	(324.92, 25.25) --
	(326.46, 25.25) --
	(327.99, 25.25) --
	(329.53, 25.25) --
	(331.07, 25.25) --
	(332.60, 25.25) --
	(334.14, 25.25) --
	(335.68, 25.25) --
	(337.22, 25.25) --
	(338.75, 25.25) --
	(340.29, 25.25) --
	(341.83, 25.25) --
	(343.36, 25.25) --
	(344.90, 25.25) --
	(346.44, 25.25) --
	(347.97, 25.25) --
	(349.51, 25.25) --
	(351.05, 25.25) --
	(352.58, 25.25) --
	(354.12, 25.25) --
	(355.66, 25.25) --
	(357.20, 25.25) --
	(358.73, 25.25) --
	(360.27, 25.25) --
	(361.81, 25.25) --
	(363.34, 25.25) --
	(364.88, 25.25) --
	(366.42, 25.25) --
	(367.95, 25.25);
\definecolor{drawColor}{RGB}{2,169,243}

\path[draw=drawColor,line width= 1.1pt,line join=round] (217.33, 27.75) --
	(218.87, 35.08) --
	(220.40, 43.31) --
	(221.94, 50.78) --
	(223.48, 57.48) --
	(225.02, 63.41) --
	(226.55, 68.57) --
	(228.09, 72.96) --
	(229.63, 76.58) --
	(231.16, 79.44) --
	(232.70, 81.53) --
	(234.24, 82.84) --
	(235.77, 83.39) --
	(237.31, 83.18) --
	(238.85, 82.19) --
	(240.39, 80.43) --
	(241.92, 77.91) --
	(243.46, 74.81) --
	(245.00, 71.76) --
	(246.53, 68.81) --
	(248.07, 65.95) --
	(249.61, 63.19) --
	(251.14, 60.53) --
	(252.68, 57.97) --
	(254.22, 55.50) --
	(255.76, 53.13) --
	(257.29, 50.85) --
	(258.83, 48.68) --
	(260.37, 46.60) --
	(261.90, 44.61) --
	(263.44, 42.72) --
	(264.98, 40.93) --
	(266.51, 39.24) --
	(268.05, 37.64) --
	(269.59, 36.14) --
	(271.13, 34.74) --
	(272.66, 33.43) --
	(274.20, 32.22) --
	(275.74, 31.11) --
	(277.27, 30.09) --
	(278.81, 29.17) --
	(280.35, 28.35) --
	(281.88, 27.62) --
	(283.42, 27.00) --
	(284.96, 26.46) --
	(286.49, 26.03) --
	(288.03, 25.69) --
	(289.57, 25.45) --
	(291.11, 25.30) --
	(292.64, 25.25) --
	(294.18, 25.25) --
	(295.72, 25.25) --
	(297.25, 25.25) --
	(298.79, 25.25) --
	(300.33, 25.25) --
	(301.86, 25.25) --
	(303.40, 25.25) --
	(304.94, 25.25) --
	(306.48, 25.25) --
	(308.01, 25.25) --
	(309.55, 25.25) --
	(311.09, 25.25) --
	(312.62, 25.25) --
	(314.16, 25.25) --
	(315.70, 25.25) --
	(317.23, 25.25) --
	(318.77, 25.25) --
	(320.31, 25.25) --
	(321.85, 25.25) --
	(323.38, 25.25) --
	(324.92, 25.25) --
	(326.46, 25.25) --
	(327.99, 25.25) --
	(329.53, 25.25) --
	(331.07, 25.25) --
	(332.60, 25.25) --
	(334.14, 25.25) --
	(335.68, 25.25) --
	(337.22, 25.25) --
	(338.75, 25.25) --
	(340.29, 25.25) --
	(341.83, 25.25) --
	(343.36, 25.25) --
	(344.90, 25.25) --
	(346.44, 25.25) --
	(347.97, 25.25) --
	(349.51, 25.25) --
	(351.05, 25.25) --
	(352.58, 25.25) --
	(354.12, 25.25) --
	(355.66, 25.25) --
	(357.20, 25.25) --
	(358.73, 25.25) --
	(360.27, 25.25) --
	(361.81, 25.25) --
	(363.34, 25.25) --
	(364.88, 25.25) --
	(366.42, 25.25) --
	(367.95, 25.25);
\definecolor{drawColor}{RGB}{0,150,135}

\path[draw=drawColor,line width= 1.1pt,line join=round] (217.33, 25.25) --
	(218.87, 25.26) --
	(220.40, 25.41) --
	(221.94, 25.76) --
	(223.48, 26.32) --
	(225.02, 27.08) --
	(226.55, 28.04) --
	(228.09, 29.20) --
	(229.63, 30.57) --
	(231.16, 32.14) --
	(232.70, 33.92) --
	(234.24, 35.90) --
	(235.77, 38.08) --
	(237.31, 40.46) --
	(238.85, 43.05) --
	(240.39, 45.84) --
	(241.92, 48.83) --
	(243.46, 51.95) --
	(245.00, 54.91) --
	(246.53, 57.71) --
	(248.07, 60.34) --
	(249.61, 62.80) --
	(251.14, 65.09) --
	(252.68, 67.21) --
	(254.22, 69.17) --
	(255.76, 70.96) --
	(257.29, 72.58) --
	(258.83, 74.03) --
	(260.37, 75.31) --
	(261.90, 76.43) --
	(263.44, 77.37) --
	(264.98, 78.15) --
	(266.51, 78.76) --
	(268.05, 79.20) --
	(269.59, 79.47) --
	(271.13, 79.58) --
	(272.66, 79.52) --
	(274.20, 79.29) --
	(275.74, 78.89) --
	(277.27, 78.32) --
	(278.81, 77.58) --
	(280.35, 76.68) --
	(281.88, 75.60) --
	(283.42, 74.36) --
	(284.96, 72.95) --
	(286.49, 71.38) --
	(288.03, 69.63) --
	(289.57, 67.72) --
	(291.11, 65.64) --
	(292.64, 63.39) --
	(294.18, 61.09) --
	(295.72, 58.86) --
	(297.25, 56.70) --
	(298.79, 54.62) --
	(300.33, 52.61) --
	(301.86, 50.67) --
	(303.40, 48.80) --
	(304.94, 47.00) --
	(306.48, 45.27) --
	(308.01, 43.61) --
	(309.55, 42.03) --
	(311.09, 40.52) --
	(312.62, 39.08) --
	(314.16, 37.71) --
	(315.70, 36.41) --
	(317.23, 35.18) --
	(318.77, 34.03) --
	(320.31, 32.94) --
	(321.85, 31.93) --
	(323.38, 30.99) --
	(324.92, 30.12) --
	(326.46, 29.32) --
	(327.99, 28.59) --
	(329.53, 27.94) --
	(331.07, 27.36) --
	(332.60, 26.84) --
	(334.14, 26.40) --
	(335.68, 26.03) --
	(337.22, 25.73) --
	(338.75, 25.51) --
	(340.29, 25.35) --
	(341.83, 25.27) --
	(343.36, 25.25) --
	(344.90, 25.25) --
	(346.44, 25.25) --
	(347.97, 25.25) --
	(349.51, 25.25) --
	(351.05, 25.25) --
	(352.58, 25.25) --
	(354.12, 25.25) --
	(355.66, 25.25) --
	(357.20, 25.25) --
	(358.73, 25.25) --
	(360.27, 25.25) --
	(361.81, 25.25) --
	(363.34, 25.25) --
	(364.88, 25.25) --
	(366.42, 25.25) --
	(367.95, 25.25);
\definecolor{drawColor}{RGB}{139,195,74}

\path[draw=drawColor,line width= 1.1pt,line join=round] (217.33, 25.25) --
	(218.87, 25.25) --
	(220.40, 25.25) --
	(221.94, 25.25) --
	(223.48, 25.25) --
	(225.02, 25.25) --
	(226.55, 25.25) --
	(228.09, 25.25) --
	(229.63, 25.25) --
	(231.16, 25.25) --
	(232.70, 25.25) --
	(234.24, 25.25) --
	(235.77, 25.25) --
	(237.31, 25.25) --
	(238.85, 25.25) --
	(240.39, 25.25) --
	(241.92, 25.25) --
	(243.46, 25.27) --
	(245.00, 25.35) --
	(246.53, 25.51) --
	(248.07, 25.73) --
	(249.61, 26.03) --
	(251.14, 26.40) --
	(252.68, 26.84) --
	(254.22, 27.36) --
	(255.76, 27.94) --
	(257.29, 28.59) --
	(258.83, 29.32) --
	(260.37, 30.12) --
	(261.90, 30.99) --
	(263.44, 31.93) --
	(264.98, 32.94) --
	(266.51, 34.03) --
	(268.05, 35.18) --
	(269.59, 36.41) --
	(271.13, 37.71) --
	(272.66, 39.08) --
	(274.20, 40.52) --
	(275.74, 42.03) --
	(277.27, 43.61) --
	(278.81, 45.27) --
	(280.35, 47.00) --
	(281.88, 48.80) --
	(283.42, 50.67) --
	(284.96, 52.61) --
	(286.49, 54.62) --
	(288.03, 56.70) --
	(289.57, 58.86) --
	(291.11, 61.09) --
	(292.64, 63.39) --
	(294.18, 65.64) --
	(295.72, 67.72) --
	(297.25, 69.63) --
	(298.79, 71.38) --
	(300.33, 72.95) --
	(301.86, 74.36) --
	(303.40, 75.60) --
	(304.94, 76.68) --
	(306.48, 77.58) --
	(308.01, 78.32) --
	(309.55, 78.89) --
	(311.09, 79.29) --
	(312.62, 79.52) --
	(314.16, 79.58) --
	(315.70, 79.47) --
	(317.23, 79.20) --
	(318.77, 78.76) --
	(320.31, 78.15) --
	(321.85, 77.37) --
	(323.38, 76.43) --
	(324.92, 75.31) --
	(326.46, 74.03) --
	(327.99, 72.58) --
	(329.53, 70.96) --
	(331.07, 69.17) --
	(332.60, 67.21) --
	(334.14, 65.09) --
	(335.68, 62.80) --
	(337.22, 60.34) --
	(338.75, 57.71) --
	(340.29, 54.91) --
	(341.83, 51.95) --
	(343.36, 48.83) --
	(344.90, 45.84) --
	(346.44, 43.05) --
	(347.97, 40.46) --
	(349.51, 38.08) --
	(351.05, 35.90) --
	(352.58, 33.92) --
	(354.12, 32.14) --
	(355.66, 30.57) --
	(357.20, 29.20) --
	(358.73, 28.04) --
	(360.27, 27.08) --
	(361.81, 26.32) --
	(363.34, 25.76) --
	(364.88, 25.41) --
	(366.42, 25.26) --
	(367.95, 25.25);
\definecolor{drawColor}{RGB}{255,235,58}

\path[draw=drawColor,line width= 1.1pt,line join=round] (217.33, 25.25) --
	(218.87, 25.25) --
	(220.40, 25.25) --
	(221.94, 25.25) --
	(223.48, 25.25) --
	(225.02, 25.25) --
	(226.55, 25.25) --
	(228.09, 25.25) --
	(229.63, 25.25) --
	(231.16, 25.25) --
	(232.70, 25.25) --
	(234.24, 25.25) --
	(235.77, 25.25) --
	(237.31, 25.25) --
	(238.85, 25.25) --
	(240.39, 25.25) --
	(241.92, 25.25) --
	(243.46, 25.25) --
	(245.00, 25.25) --
	(246.53, 25.25) --
	(248.07, 25.25) --
	(249.61, 25.25) --
	(251.14, 25.25) --
	(252.68, 25.25) --
	(254.22, 25.25) --
	(255.76, 25.25) --
	(257.29, 25.25) --
	(258.83, 25.25) --
	(260.37, 25.25) --
	(261.90, 25.25) --
	(263.44, 25.25) --
	(264.98, 25.25) --
	(266.51, 25.25) --
	(268.05, 25.25) --
	(269.59, 25.25) --
	(271.13, 25.25) --
	(272.66, 25.25) --
	(274.20, 25.25) --
	(275.74, 25.25) --
	(277.27, 25.25) --
	(278.81, 25.25) --
	(280.35, 25.25) --
	(281.88, 25.25) --
	(283.42, 25.25) --
	(284.96, 25.25) --
	(286.49, 25.25) --
	(288.03, 25.25) --
	(289.57, 25.25) --
	(291.11, 25.25) --
	(292.64, 25.25) --
	(294.18, 25.30) --
	(295.72, 25.45) --
	(297.25, 25.69) --
	(298.79, 26.03) --
	(300.33, 26.46) --
	(301.86, 27.00) --
	(303.40, 27.62) --
	(304.94, 28.35) --
	(306.48, 29.17) --
	(308.01, 30.09) --
	(309.55, 31.11) --
	(311.09, 32.22) --
	(312.62, 33.43) --
	(314.16, 34.74) --
	(315.70, 36.14) --
	(317.23, 37.64) --
	(318.77, 39.24) --
	(320.31, 40.93) --
	(321.85, 42.72) --
	(323.38, 44.61) --
	(324.92, 46.60) --
	(326.46, 48.68) --
	(327.99, 50.85) --
	(329.53, 53.13) --
	(331.07, 55.50) --
	(332.60, 57.97) --
	(334.14, 60.53) --
	(335.68, 63.19) --
	(337.22, 65.95) --
	(338.75, 68.81) --
	(340.29, 71.76) --
	(341.83, 74.81) --
	(343.36, 77.91) --
	(344.90, 80.43) --
	(346.44, 82.19) --
	(347.97, 83.18) --
	(349.51, 83.39) --
	(351.05, 82.84) --
	(352.58, 81.53) --
	(354.12, 79.44) --
	(355.66, 76.58) --
	(357.20, 72.96) --
	(358.73, 68.57) --
	(360.27, 63.41) --
	(361.81, 57.48) --
	(363.34, 50.78) --
	(364.88, 43.31) --
	(366.42, 35.08) --
	(367.95, 27.75);
\definecolor{drawColor}{RGB}{255,152,0}

\path[draw=drawColor,line width= 1.1pt,line join=round] (217.33, 25.25) --
	(218.87, 25.25) --
	(220.40, 25.25) --
	(221.94, 25.25) --
	(223.48, 25.25) --
	(225.02, 25.25) --
	(226.55, 25.25) --
	(228.09, 25.25) --
	(229.63, 25.25) --
	(231.16, 25.25) --
	(232.70, 25.25) --
	(234.24, 25.25) --
	(235.77, 25.25) --
	(237.31, 25.25) --
	(238.85, 25.25) --
	(240.39, 25.25) --
	(241.92, 25.25) --
	(243.46, 25.25) --
	(245.00, 25.25) --
	(246.53, 25.25) --
	(248.07, 25.25) --
	(249.61, 25.25) --
	(251.14, 25.25) --
	(252.68, 25.25) --
	(254.22, 25.25) --
	(255.76, 25.25) --
	(257.29, 25.25) --
	(258.83, 25.25) --
	(260.37, 25.25) --
	(261.90, 25.25) --
	(263.44, 25.25) --
	(264.98, 25.25) --
	(266.51, 25.25) --
	(268.05, 25.25) --
	(269.59, 25.25) --
	(271.13, 25.25) --
	(272.66, 25.25) --
	(274.20, 25.25) --
	(275.74, 25.25) --
	(277.27, 25.25) --
	(278.81, 25.25) --
	(280.35, 25.25) --
	(281.88, 25.25) --
	(283.42, 25.25) --
	(284.96, 25.25) --
	(286.49, 25.25) --
	(288.03, 25.25) --
	(289.57, 25.25) --
	(291.11, 25.25) --
	(292.64, 25.25) --
	(294.18, 25.25) --
	(295.72, 25.25) --
	(297.25, 25.25) --
	(298.79, 25.25) --
	(300.33, 25.25) --
	(301.86, 25.25) --
	(303.40, 25.25) --
	(304.94, 25.25) --
	(306.48, 25.25) --
	(308.01, 25.25) --
	(309.55, 25.25) --
	(311.09, 25.25) --
	(312.62, 25.25) --
	(314.16, 25.25) --
	(315.70, 25.25) --
	(317.23, 25.25) --
	(318.77, 25.25) --
	(320.31, 25.25) --
	(321.85, 25.25) --
	(323.38, 25.25) --
	(324.92, 25.25) --
	(326.46, 25.25) --
	(327.99, 25.25) --
	(329.53, 25.25) --
	(331.07, 25.25) --
	(332.60, 25.25) --
	(334.14, 25.25) --
	(335.68, 25.25) --
	(337.22, 25.25) --
	(338.75, 25.25) --
	(340.29, 25.25) --
	(341.83, 25.25) --
	(343.36, 25.28) --
	(344.90, 25.75) --
	(346.44, 26.79) --
	(347.97, 28.39) --
	(349.51, 30.55) --
	(351.05, 33.28) --
	(352.58, 36.58) --
	(354.12, 40.44) --
	(355.66, 44.87) --
	(357.20, 49.86) --
	(358.73, 55.42) --
	(360.27, 61.54) --
	(361.81, 68.23) --
	(363.34, 75.49) --
	(364.88, 83.31) --
	(366.42, 91.69) --
	(367.95, 91.88);
\definecolor{drawColor}{RGB}{121,84,71}

\path[draw=drawColor,line width= 1.1pt,line join=round] (217.33, 25.25) --
	(218.87, 25.25) --
	(220.40, 25.25) --
	(221.94, 25.25) --
	(223.48, 25.25) --
	(225.02, 25.25) --
	(226.55, 25.25) --
	(228.09, 25.25) --
	(229.63, 25.25) --
	(231.16, 25.25) --
	(232.70, 25.25) --
	(234.24, 25.25) --
	(235.77, 25.25) --
	(237.31, 25.25) --
	(238.85, 25.25) --
	(240.39, 25.25) --
	(241.92, 25.25) --
	(243.46, 25.25) --
	(245.00, 25.25) --
	(246.53, 25.25) --
	(248.07, 25.25) --
	(249.61, 25.25) --
	(251.14, 25.25) --
	(252.68, 25.25) --
	(254.22, 25.25) --
	(255.76, 25.25) --
	(257.29, 25.25) --
	(258.83, 25.25) --
	(260.37, 25.25) --
	(261.90, 25.25) --
	(263.44, 25.25) --
	(264.98, 25.25) --
	(266.51, 25.25) --
	(268.05, 25.25) --
	(269.59, 25.25) --
	(271.13, 25.25) --
	(272.66, 25.25) --
	(274.20, 25.25) --
	(275.74, 25.25) --
	(277.27, 25.25) --
	(278.81, 25.25) --
	(280.35, 25.25) --
	(281.88, 25.25) --
	(283.42, 25.25) --
	(284.96, 25.25) --
	(286.49, 25.25) --
	(288.03, 25.25) --
	(289.57, 25.25) --
	(291.11, 25.25) --
	(292.64, 25.25) --
	(294.18, 25.25) --
	(295.72, 25.25) --
	(297.25, 25.25) --
	(298.79, 25.25) --
	(300.33, 25.25) --
	(301.86, 25.25) --
	(303.40, 25.25) --
	(304.94, 25.25) --
	(306.48, 25.25) --
	(308.01, 25.25) --
	(309.55, 25.25) --
	(311.09, 25.25) --
	(312.62, 25.25) --
	(314.16, 25.25) --
	(315.70, 25.25) --
	(317.23, 25.25) --
	(318.77, 25.25) --
	(320.31, 25.25) --
	(321.85, 25.25) --
	(323.38, 25.25) --
	(324.92, 25.25) --
	(326.46, 25.25) --
	(327.99, 25.25) --
	(329.53, 25.25) --
	(331.07, 25.25) --
	(332.60, 25.25) --
	(334.14, 25.25) --
	(335.68, 25.25) --
	(337.22, 25.25) --
	(338.75, 25.25) --
	(340.29, 25.25) --
	(341.83, 25.25) --
	(343.36, 25.25) --
	(344.90, 25.25) --
	(346.44, 25.25) --
	(347.97, 25.25) --
	(349.51, 25.25) --
	(351.05, 25.25) --
	(352.58, 25.25) --
	(354.12, 25.25) --
	(355.66, 25.25) --
	(357.20, 25.25) --
	(358.73, 25.25) --
	(360.27, 25.25) --
	(361.81, 25.25) --
	(363.34, 25.25) --
	(364.88, 25.25) --
	(366.42, 25.25) --
	(367.95, 32.40);
\end{scope}
\begin{scope}
\path[clip] (386.87,186.57) rectangle (552.55,263.47);
\definecolor{drawColor}{gray}{0.92}

\path[draw=drawColor,line width= 0.3pt,line join=round] (386.87,199.60) --
	(552.55,199.60);

\path[draw=drawColor,line width= 0.3pt,line join=round] (386.87,218.66) --
	(552.55,218.66);

\path[draw=drawColor,line width= 0.3pt,line join=round] (386.87,237.73) --
	(552.55,237.73);

\path[draw=drawColor,line width= 0.3pt,line join=round] (386.87,256.80) --
	(552.55,256.80);

\path[draw=drawColor,line width= 0.3pt,line join=round] (412.07,186.57) --
	(412.07,263.47);

\path[draw=drawColor,line width= 0.3pt,line join=round] (450.50,186.57) --
	(450.50,263.47);

\path[draw=drawColor,line width= 0.3pt,line join=round] (488.92,186.57) --
	(488.92,263.47);

\path[draw=drawColor,line width= 0.3pt,line join=round] (527.35,186.57) --
	(527.35,263.47);

\path[draw=drawColor,line width= 0.6pt,line join=round] (386.87,190.06) --
	(552.55,190.06);

\path[draw=drawColor,line width= 0.6pt,line join=round] (386.87,209.13) --
	(552.55,209.13);

\path[draw=drawColor,line width= 0.6pt,line join=round] (386.87,228.20) --
	(552.55,228.20);

\path[draw=drawColor,line width= 0.6pt,line join=round] (386.87,247.26) --
	(552.55,247.26);

\path[draw=drawColor,line width= 0.6pt,line join=round] (392.86,186.57) --
	(392.86,263.47);

\path[draw=drawColor,line width= 0.6pt,line join=round] (431.29,186.57) --
	(431.29,263.47);

\path[draw=drawColor,line width= 0.6pt,line join=round] (469.71,186.57) --
	(469.71,263.47);

\path[draw=drawColor,line width= 0.6pt,line join=round] (508.13,186.57) --
	(508.13,263.47);

\path[draw=drawColor,line width= 0.6pt,line join=round] (546.56,186.57) --
	(546.56,263.47);
\definecolor{drawColor}{RGB}{155,38,176}

\path[draw=drawColor,line width= 1.1pt,line join=round] (394.40,227.02) --
	(395.93,225.86) --
	(397.47,224.72) --
	(399.01,223.60) --
	(400.55,222.49) --
	(402.08,221.41) --
	(403.62,220.34) --
	(405.16,219.29) --
	(406.69,218.26) --
	(408.23,217.25) --
	(409.77,216.26) --
	(411.30,215.28) --
	(412.84,214.33) --
	(414.38,213.39) --
	(415.92,212.47) --
	(417.45,211.57) --
	(418.99,210.69) --
	(420.53,209.82) --
	(422.06,208.98) --
	(423.60,208.15) --
	(425.14,207.34) --
	(426.67,206.56) --
	(428.21,205.78) --
	(429.75,205.03) --
	(431.29,204.30) --
	(432.82,203.58) --
	(434.36,202.88) --
	(435.90,202.20) --
	(437.43,201.54) --
	(438.97,200.90) --
	(440.51,200.28) --
	(442.04,199.67) --
	(443.58,199.09) --
	(445.12,198.52) --
	(446.66,197.97) --
	(448.19,197.44) --
	(449.73,196.93) --
	(451.27,196.43) --
	(452.80,195.96) --
	(454.34,195.50) --
	(455.88,195.06) --
	(457.41,194.64) --
	(458.95,194.24) --
	(460.49,193.85) --
	(462.02,193.49) --
	(463.56,193.14) --
	(465.10,192.81) --
	(466.64,192.50) --
	(468.17,192.21) --
	(469.71,191.94) --
	(471.25,191.69) --
	(472.78,191.45) --
	(474.32,191.23) --
	(475.86,191.04) --
	(477.39,190.86) --
	(478.93,190.69) --
	(480.47,190.55) --
	(482.01,190.43) --
	(483.54,190.32) --
	(485.08,190.23) --
	(486.62,190.16) --
	(488.15,190.11) --
	(489.69,190.08) --
	(491.23,190.06) --
	(492.76,190.06) --
	(494.30,190.06) --
	(495.84,190.06) --
	(497.38,190.06) --
	(498.91,190.06) --
	(500.45,190.06) --
	(501.99,190.06) --
	(503.52,190.06) --
	(505.06,190.06) --
	(506.60,190.06) --
	(508.13,190.06) --
	(509.67,190.06) --
	(511.21,190.06) --
	(512.74,190.06) --
	(514.28,190.06) --
	(515.82,190.06) --
	(517.36,190.06) --
	(518.89,190.06) --
	(520.43,190.06) --
	(521.97,190.06) --
	(523.50,190.06) --
	(525.04,190.06) --
	(526.58,190.06) --
	(528.11,190.06) --
	(529.65,190.06) --
	(531.19,190.06) --
	(532.73,190.06) --
	(534.26,190.06) --
	(535.80,190.06) --
	(537.34,190.06) --
	(538.87,190.06) --
	(540.41,190.06) --
	(541.95,190.06) --
	(543.48,190.06) --
	(545.02,190.06);
\definecolor{drawColor}{RGB}{63,81,180}

\path[draw=drawColor,line width= 1.1pt,line join=round] (394.40,229.36) --
	(395.93,230.47) --
	(397.47,231.52) --
	(399.01,232.53) --
	(400.55,233.48) --
	(402.08,234.38) --
	(403.62,235.23) --
	(405.16,236.03) --
	(406.69,236.78) --
	(408.23,237.47) --
	(409.77,238.11) --
	(411.30,238.70) --
	(412.84,239.24) --
	(414.38,239.73) --
	(415.92,240.16) --
	(417.45,240.55) --
	(418.99,240.88) --
	(420.53,241.16) --
	(422.06,241.39) --
	(423.60,241.56) --
	(425.14,241.69) --
	(426.67,241.76) --
	(428.21,241.78) --
	(429.75,241.75) --
	(431.29,241.66) --
	(432.82,241.53) --
	(434.36,241.34) --
	(435.90,241.10) --
	(437.43,240.81) --
	(438.97,240.47) --
	(440.51,240.07) --
	(442.04,239.63) --
	(443.58,239.13) --
	(445.12,238.58) --
	(446.66,237.97) --
	(448.19,237.32) --
	(449.73,236.62) --
	(451.27,235.86) --
	(452.80,235.05) --
	(454.34,234.19) --
	(455.88,233.27) --
	(457.41,232.31) --
	(458.95,231.29) --
	(460.49,230.22) --
	(462.02,229.10) --
	(463.56,227.93) --
	(465.10,226.71) --
	(466.64,225.43) --
	(468.17,224.10) --
	(469.71,222.72) --
	(471.25,221.29) --
	(472.78,219.81) --
	(474.32,218.27) --
	(475.86,216.68) --
	(477.39,215.04) --
	(478.93,213.35) --
	(480.47,211.61) --
	(482.01,209.82) --
	(483.54,207.97) --
	(485.08,206.07) --
	(486.62,204.12) --
	(488.15,202.12) --
	(489.69,200.06) --
	(491.23,197.96) --
	(492.76,195.90) --
	(494.30,194.14) --
	(495.84,192.70) --
	(497.38,191.57) --
	(498.91,190.76) --
	(500.45,190.25) --
	(501.99,190.06) --
	(503.52,190.06) --
	(505.06,190.06) --
	(506.60,190.06) --
	(508.13,190.06) --
	(509.67,190.06) --
	(511.21,190.06) --
	(512.74,190.06) --
	(514.28,190.06) --
	(515.82,190.06) --
	(517.36,190.06) --
	(518.89,190.06) --
	(520.43,190.06) --
	(521.97,190.06) --
	(523.50,190.06) --
	(525.04,190.06) --
	(526.58,190.06) --
	(528.11,190.06) --
	(529.65,190.06) --
	(531.19,190.06) --
	(532.73,190.06) --
	(534.26,190.06) --
	(535.80,190.06) --
	(537.34,190.06) --
	(538.87,190.06) --
	(540.41,190.06) --
	(541.95,190.06) --
	(543.48,190.06) --
	(545.02,190.06);
\definecolor{drawColor}{RGB}{2,169,243}

\path[draw=drawColor,line width= 1.1pt,line join=round] (394.40,190.08) --
	(395.93,190.13) --
	(397.47,190.21) --
	(399.01,190.33) --
	(400.55,190.48) --
	(402.08,190.66) --
	(403.62,190.88) --
	(405.16,191.13) --
	(406.69,191.42) --
	(408.23,191.73) --
	(409.77,192.08) --
	(411.30,192.47) --
	(412.84,192.88) --
	(414.38,193.33) --
	(415.92,193.82) --
	(417.45,194.34) --
	(418.99,194.89) --
	(420.53,195.47) --
	(422.06,196.09) --
	(423.60,196.74) --
	(425.14,197.42) --
	(426.67,198.14) --
	(428.21,198.89) --
	(429.75,199.68) --
	(431.29,200.50) --
	(432.82,201.35) --
	(434.36,202.23) --
	(435.90,203.15) --
	(437.43,204.10) --
	(438.97,205.09) --
	(440.51,206.10) --
	(442.04,207.16) --
	(443.58,208.24) --
	(445.12,209.36) --
	(446.66,210.51) --
	(448.19,211.70) --
	(449.73,212.91) --
	(451.27,214.17) --
	(452.80,215.45) --
	(454.34,216.77) --
	(455.88,218.12) --
	(457.41,219.51) --
	(458.95,220.93) --
	(460.49,222.38) --
	(462.02,223.86) --
	(463.56,225.38) --
	(465.10,226.94) --
	(466.64,228.52) --
	(468.17,230.14) --
	(469.71,231.79) --
	(471.25,233.48) --
	(472.78,235.20) --
	(474.32,236.95) --
	(475.86,238.74) --
	(477.39,240.56) --
	(478.93,242.41) --
	(480.47,244.30) --
	(482.01,246.21) --
	(483.54,248.17) --
	(485.08,250.15) --
	(486.62,252.17) --
	(488.15,254.23) --
	(489.69,256.31) --
	(491.23,258.43) --
	(492.76,259.98) --
	(494.30,259.40) --
	(495.84,256.64) --
	(497.38,251.69) --
	(498.91,244.54) --
	(500.45,235.21) --
	(501.99,223.68) --
	(503.52,211.87) --
	(505.06,202.59) --
	(506.60,195.87) --
	(508.13,191.70) --
	(509.67,190.09) --
	(511.21,190.06) --
	(512.74,190.06) --
	(514.28,190.06) --
	(515.82,190.06) --
	(517.36,190.06) --
	(518.89,190.06) --
	(520.43,190.06) --
	(521.97,190.06) --
	(523.50,190.06) --
	(525.04,190.06) --
	(526.58,190.06) --
	(528.11,190.06) --
	(529.65,190.06) --
	(531.19,190.06) --
	(532.73,190.06) --
	(534.26,190.06) --
	(535.80,190.06) --
	(537.34,190.06) --
	(538.87,190.06) --
	(540.41,190.06) --
	(541.95,190.06) --
	(543.48,190.06) --
	(545.02,190.06);
\definecolor{drawColor}{RGB}{0,150,135}

\path[draw=drawColor,line width= 1.1pt,line join=round] (394.40,190.06) --
	(395.93,190.06) --
	(397.47,190.06) --
	(399.01,190.06) --
	(400.55,190.06) --
	(402.08,190.06) --
	(403.62,190.06) --
	(405.16,190.06) --
	(406.69,190.06) --
	(408.23,190.06) --
	(409.77,190.06) --
	(411.30,190.06) --
	(412.84,190.06) --
	(414.38,190.06) --
	(415.92,190.06) --
	(417.45,190.06) --
	(418.99,190.06) --
	(420.53,190.06) --
	(422.06,190.06) --
	(423.60,190.06) --
	(425.14,190.06) --
	(426.67,190.06) --
	(428.21,190.06) --
	(429.75,190.06) --
	(431.29,190.06) --
	(432.82,190.06) --
	(434.36,190.06) --
	(435.90,190.06) --
	(437.43,190.06) --
	(438.97,190.06) --
	(440.51,190.06) --
	(442.04,190.06) --
	(443.58,190.06) --
	(445.12,190.06) --
	(446.66,190.06) --
	(448.19,190.06) --
	(449.73,190.06) --
	(451.27,190.06) --
	(452.80,190.06) --
	(454.34,190.06) --
	(455.88,190.06) --
	(457.41,190.06) --
	(458.95,190.06) --
	(460.49,190.06) --
	(462.02,190.06) --
	(463.56,190.06) --
	(465.10,190.06) --
	(466.64,190.06) --
	(468.17,190.06) --
	(469.71,190.06) --
	(471.25,190.06) --
	(472.78,190.06) --
	(474.32,190.06) --
	(475.86,190.06) --
	(477.39,190.06) --
	(478.93,190.06) --
	(480.47,190.06) --
	(482.01,190.06) --
	(483.54,190.06) --
	(485.08,190.06) --
	(486.62,190.06) --
	(488.15,190.06) --
	(489.69,190.06) --
	(491.23,190.06) --
	(492.76,190.58) --
	(494.30,192.91) --
	(495.84,197.11) --
	(497.38,203.20) --
	(498.91,211.16) --
	(500.45,221.00) --
	(501.99,232.71) --
	(503.52,243.25) --
	(505.06,248.14) --
	(506.60,247.32) --
	(508.13,240.81) --
	(509.67,228.60) --
	(511.21,214.13) --
	(512.74,203.00) --
	(514.28,195.30) --
	(515.82,191.02) --
	(517.36,190.06) --
	(518.89,190.06) --
	(520.43,190.06) --
	(521.97,190.06) --
	(523.50,190.06) --
	(525.04,190.06) --
	(526.58,190.06) --
	(528.11,190.06) --
	(529.65,190.06) --
	(531.19,190.06) --
	(532.73,190.06) --
	(534.26,190.06) --
	(535.80,190.06) --
	(537.34,190.06) --
	(538.87,190.06) --
	(540.41,190.06) --
	(541.95,190.06) --
	(543.48,190.06) --
	(545.02,190.06);
\definecolor{drawColor}{RGB}{139,195,74}

\path[draw=drawColor,line width= 1.1pt,line join=round] (394.40,190.06) --
	(395.93,190.06) --
	(397.47,190.06) --
	(399.01,190.06) --
	(400.55,190.06) --
	(402.08,190.06) --
	(403.62,190.06) --
	(405.16,190.06) --
	(406.69,190.06) --
	(408.23,190.06) --
	(409.77,190.06) --
	(411.30,190.06) --
	(412.84,190.06) --
	(414.38,190.06) --
	(415.92,190.06) --
	(417.45,190.06) --
	(418.99,190.06) --
	(420.53,190.06) --
	(422.06,190.06) --
	(423.60,190.06) --
	(425.14,190.06) --
	(426.67,190.06) --
	(428.21,190.06) --
	(429.75,190.06) --
	(431.29,190.06) --
	(432.82,190.06) --
	(434.36,190.06) --
	(435.90,190.06) --
	(437.43,190.06) --
	(438.97,190.06) --
	(440.51,190.06) --
	(442.04,190.06) --
	(443.58,190.06) --
	(445.12,190.06) --
	(446.66,190.06) --
	(448.19,190.06) --
	(449.73,190.06) --
	(451.27,190.06) --
	(452.80,190.06) --
	(454.34,190.06) --
	(455.88,190.06) --
	(457.41,190.06) --
	(458.95,190.06) --
	(460.49,190.06) --
	(462.02,190.06) --
	(463.56,190.06) --
	(465.10,190.06) --
	(466.64,190.06) --
	(468.17,190.06) --
	(469.71,190.06) --
	(471.25,190.06) --
	(472.78,190.06) --
	(474.32,190.06) --
	(475.86,190.06) --
	(477.39,190.06) --
	(478.93,190.06) --
	(480.47,190.06) --
	(482.01,190.06) --
	(483.54,190.06) --
	(485.08,190.06) --
	(486.62,190.06) --
	(488.15,190.06) --
	(489.69,190.06) --
	(491.23,190.06) --
	(492.76,190.06) --
	(494.30,190.06) --
	(495.84,190.06) --
	(497.38,190.06) --
	(498.91,190.06) --
	(500.45,190.06) --
	(501.99,190.06) --
	(503.52,191.33) --
	(505.06,195.72) --
	(506.60,203.26) --
	(508.13,213.94) --
	(509.67,227.77) --
	(511.21,240.99) --
	(512.74,247.49) --
	(514.28,247.17) --
	(515.82,240.04) --
	(517.36,226.41) --
	(518.89,213.07) --
	(520.43,202.76) --
	(521.97,195.49) --
	(523.50,191.27) --
	(525.04,190.06) --
	(526.58,190.06) --
	(528.11,190.06) --
	(529.65,190.06) --
	(531.19,190.06) --
	(532.73,190.06) --
	(534.26,190.06) --
	(535.80,190.06) --
	(537.34,190.06) --
	(538.87,190.06) --
	(540.41,190.06) --
	(541.95,190.06) --
	(543.48,190.06) --
	(545.02,190.06);
\definecolor{drawColor}{RGB}{255,235,58}

\path[draw=drawColor,line width= 1.1pt,line join=round] (394.40,190.06) --
	(395.93,190.06) --
	(397.47,190.06) --
	(399.01,190.06) --
	(400.55,190.06) --
	(402.08,190.06) --
	(403.62,190.06) --
	(405.16,190.06) --
	(406.69,190.06) --
	(408.23,190.06) --
	(409.77,190.06) --
	(411.30,190.06) --
	(412.84,190.06) --
	(414.38,190.06) --
	(415.92,190.06) --
	(417.45,190.06) --
	(418.99,190.06) --
	(420.53,190.06) --
	(422.06,190.06) --
	(423.60,190.06) --
	(425.14,190.06) --
	(426.67,190.06) --
	(428.21,190.06) --
	(429.75,190.06) --
	(431.29,190.06) --
	(432.82,190.06) --
	(434.36,190.06) --
	(435.90,190.06) --
	(437.43,190.06) --
	(438.97,190.06) --
	(440.51,190.06) --
	(442.04,190.06) --
	(443.58,190.06) --
	(445.12,190.06) --
	(446.66,190.06) --
	(448.19,190.06) --
	(449.73,190.06) --
	(451.27,190.06) --
	(452.80,190.06) --
	(454.34,190.06) --
	(455.88,190.06) --
	(457.41,190.06) --
	(458.95,190.06) --
	(460.49,190.06) --
	(462.02,190.06) --
	(463.56,190.06) --
	(465.10,190.06) --
	(466.64,190.06) --
	(468.17,190.06) --
	(469.71,190.06) --
	(471.25,190.06) --
	(472.78,190.06) --
	(474.32,190.06) --
	(475.86,190.06) --
	(477.39,190.06) --
	(478.93,190.06) --
	(480.47,190.06) --
	(482.01,190.06) --
	(483.54,190.06) --
	(485.08,190.06) --
	(486.62,190.06) --
	(488.15,190.06) --
	(489.69,190.06) --
	(491.23,190.06) --
	(492.76,190.06) --
	(494.30,190.06) --
	(495.84,190.06) --
	(497.38,190.06) --
	(498.91,190.06) --
	(500.45,190.06) --
	(501.99,190.06) --
	(503.52,190.06) --
	(505.06,190.06) --
	(506.60,190.06) --
	(508.13,190.06) --
	(509.67,190.06) --
	(511.21,191.34) --
	(512.74,195.97) --
	(514.28,203.99) --
	(515.82,215.39) --
	(517.36,229.93) --
	(518.89,242.12) --
	(520.43,249.72) --
	(521.97,252.75) --
	(523.50,251.20) --
	(525.04,245.09) --
	(526.58,237.51) --
	(528.11,230.49) --
	(529.65,224.03) --
	(531.19,218.14) --
	(532.73,212.80) --
	(534.26,208.03) --
	(535.80,203.82) --
	(537.34,200.17) --
	(538.87,197.08) --
	(540.41,194.56) --
	(541.95,192.59) --
	(543.48,191.19) --
	(545.02,190.34);
\definecolor{drawColor}{RGB}{255,152,0}

\path[draw=drawColor,line width= 1.1pt,line join=round] (394.40,190.06) --
	(395.93,190.06) --
	(397.47,190.06) --
	(399.01,190.06) --
	(400.55,190.06) --
	(402.08,190.06) --
	(403.62,190.06) --
	(405.16,190.06) --
	(406.69,190.06) --
	(408.23,190.06) --
	(409.77,190.06) --
	(411.30,190.06) --
	(412.84,190.06) --
	(414.38,190.06) --
	(415.92,190.06) --
	(417.45,190.06) --
	(418.99,190.06) --
	(420.53,190.06) --
	(422.06,190.06) --
	(423.60,190.06) --
	(425.14,190.06) --
	(426.67,190.06) --
	(428.21,190.06) --
	(429.75,190.06) --
	(431.29,190.06) --
	(432.82,190.06) --
	(434.36,190.06) --
	(435.90,190.06) --
	(437.43,190.06) --
	(438.97,190.06) --
	(440.51,190.06) --
	(442.04,190.06) --
	(443.58,190.06) --
	(445.12,190.06) --
	(446.66,190.06) --
	(448.19,190.06) --
	(449.73,190.06) --
	(451.27,190.06) --
	(452.80,190.06) --
	(454.34,190.06) --
	(455.88,190.06) --
	(457.41,190.06) --
	(458.95,190.06) --
	(460.49,190.06) --
	(462.02,190.06) --
	(463.56,190.06) --
	(465.10,190.06) --
	(466.64,190.06) --
	(468.17,190.06) --
	(469.71,190.06) --
	(471.25,190.06) --
	(472.78,190.06) --
	(474.32,190.06) --
	(475.86,190.06) --
	(477.39,190.06) --
	(478.93,190.06) --
	(480.47,190.06) --
	(482.01,190.06) --
	(483.54,190.06) --
	(485.08,190.06) --
	(486.62,190.06) --
	(488.15,190.06) --
	(489.69,190.06) --
	(491.23,190.06) --
	(492.76,190.06) --
	(494.30,190.06) --
	(495.84,190.06) --
	(497.38,190.06) --
	(498.91,190.06) --
	(500.45,190.06) --
	(501.99,190.06) --
	(503.52,190.06) --
	(505.06,190.06) --
	(506.60,190.06) --
	(508.13,190.06) --
	(509.67,190.06) --
	(511.21,190.06) --
	(512.74,190.06) --
	(514.28,190.06) --
	(515.82,190.06) --
	(517.36,190.11) --
	(518.89,191.27) --
	(520.43,193.97) --
	(521.97,198.21) --
	(523.50,203.99) --
	(525.04,211.30) --
	(526.58,218.65) --
	(528.11,225.05) --
	(529.65,230.50) --
	(531.19,235.02) --
	(532.73,238.59) --
	(534.26,241.21) --
	(535.80,242.89) --
	(537.34,243.62) --
	(538.87,243.41) --
	(540.41,242.26) --
	(541.95,240.16) --
	(543.48,237.12) --
	(545.02,233.13);
\definecolor{drawColor}{RGB}{121,84,71}

\path[draw=drawColor,line width= 1.1pt,line join=round] (394.40,190.06) --
	(395.93,190.06) --
	(397.47,190.06) --
	(399.01,190.06) --
	(400.55,190.06) --
	(402.08,190.06) --
	(403.62,190.06) --
	(405.16,190.06) --
	(406.69,190.06) --
	(408.23,190.06) --
	(409.77,190.06) --
	(411.30,190.06) --
	(412.84,190.06) --
	(414.38,190.06) --
	(415.92,190.06) --
	(417.45,190.06) --
	(418.99,190.06) --
	(420.53,190.06) --
	(422.06,190.06) --
	(423.60,190.06) --
	(425.14,190.06) --
	(426.67,190.06) --
	(428.21,190.06) --
	(429.75,190.06) --
	(431.29,190.06) --
	(432.82,190.06) --
	(434.36,190.06) --
	(435.90,190.06) --
	(437.43,190.06) --
	(438.97,190.06) --
	(440.51,190.06) --
	(442.04,190.06) --
	(443.58,190.06) --
	(445.12,190.06) --
	(446.66,190.06) --
	(448.19,190.06) --
	(449.73,190.06) --
	(451.27,190.06) --
	(452.80,190.06) --
	(454.34,190.06) --
	(455.88,190.06) --
	(457.41,190.06) --
	(458.95,190.06) --
	(460.49,190.06) --
	(462.02,190.06) --
	(463.56,190.06) --
	(465.10,190.06) --
	(466.64,190.06) --
	(468.17,190.06) --
	(469.71,190.06) --
	(471.25,190.06) --
	(472.78,190.06) --
	(474.32,190.06) --
	(475.86,190.06) --
	(477.39,190.06) --
	(478.93,190.06) --
	(480.47,190.06) --
	(482.01,190.06) --
	(483.54,190.06) --
	(485.08,190.06) --
	(486.62,190.06) --
	(488.15,190.06) --
	(489.69,190.06) --
	(491.23,190.06) --
	(492.76,190.06) --
	(494.30,190.06) --
	(495.84,190.06) --
	(497.38,190.06) --
	(498.91,190.06) --
	(500.45,190.06) --
	(501.99,190.06) --
	(503.52,190.06) --
	(505.06,190.06) --
	(506.60,190.06) --
	(508.13,190.06) --
	(509.67,190.06) --
	(511.21,190.06) --
	(512.74,190.06) --
	(514.28,190.06) --
	(515.82,190.06) --
	(517.36,190.06) --
	(518.89,190.06) --
	(520.43,190.06) --
	(521.97,190.06) --
	(523.50,190.06) --
	(525.04,190.07) --
	(526.58,190.30) --
	(528.11,190.92) --
	(529.65,191.92) --
	(531.19,193.30) --
	(532.73,195.06) --
	(534.26,197.21) --
	(535.80,199.75) --
	(537.34,202.66) --
	(538.87,205.96) --
	(540.41,209.64) --
	(541.95,213.70) --
	(543.48,218.15) --
	(545.02,222.98);
\end{scope}
\begin{scope}
\path[clip] (386.87,104.16) rectangle (552.55,181.07);
\definecolor{drawColor}{gray}{0.92}

\path[draw=drawColor,line width= 0.3pt,line join=round] (386.87,117.19) --
	(552.55,117.19);

\path[draw=drawColor,line width= 0.3pt,line join=round] (386.87,136.26) --
	(552.55,136.26);

\path[draw=drawColor,line width= 0.3pt,line join=round] (386.87,155.32) --
	(552.55,155.32);

\path[draw=drawColor,line width= 0.3pt,line join=round] (386.87,174.39) --
	(552.55,174.39);

\path[draw=drawColor,line width= 0.3pt,line join=round] (412.07,104.16) --
	(412.07,181.07);

\path[draw=drawColor,line width= 0.3pt,line join=round] (450.50,104.16) --
	(450.50,181.07);

\path[draw=drawColor,line width= 0.3pt,line join=round] (488.92,104.16) --
	(488.92,181.07);

\path[draw=drawColor,line width= 0.3pt,line join=round] (527.35,104.16) --
	(527.35,181.07);

\path[draw=drawColor,line width= 0.6pt,line join=round] (386.87,107.66) --
	(552.55,107.66);

\path[draw=drawColor,line width= 0.6pt,line join=round] (386.87,126.72) --
	(552.55,126.72);

\path[draw=drawColor,line width= 0.6pt,line join=round] (386.87,145.79) --
	(552.55,145.79);

\path[draw=drawColor,line width= 0.6pt,line join=round] (386.87,164.86) --
	(552.55,164.86);

\path[draw=drawColor,line width= 0.6pt,line join=round] (392.86,104.16) --
	(392.86,181.07);

\path[draw=drawColor,line width= 0.6pt,line join=round] (431.29,104.16) --
	(431.29,181.07);

\path[draw=drawColor,line width= 0.6pt,line join=round] (469.71,104.16) --
	(469.71,181.07);

\path[draw=drawColor,line width= 0.6pt,line join=round] (508.13,104.16) --
	(508.13,181.07);

\path[draw=drawColor,line width= 0.6pt,line join=round] (546.56,104.16) --
	(546.56,181.07);
\definecolor{drawColor}{RGB}{155,38,176}

\path[draw=drawColor,line width= 1.1pt,line join=round] (394.40,144.30) --
	(395.93,142.83) --
	(397.47,141.39) --
	(399.01,139.99) --
	(400.55,138.61) --
	(402.08,137.27) --
	(403.62,135.95) --
	(405.16,134.67) --
	(406.69,133.41) --
	(408.23,132.18) --
	(409.77,130.99) --
	(411.30,129.82) --
	(412.84,128.68) --
	(414.38,127.58) --
	(415.92,126.50) --
	(417.45,125.45) --
	(418.99,124.44) --
	(420.53,123.45) --
	(422.06,122.49) --
	(423.60,121.57) --
	(425.14,120.67) --
	(426.67,119.80) --
	(428.21,118.97) --
	(429.75,118.16) --
	(431.29,117.38) --
	(432.82,116.63) --
	(434.36,115.91) --
	(435.90,115.23) --
	(437.43,114.57) --
	(438.97,113.94) --
	(440.51,113.34) --
	(442.04,112.77) --
	(443.58,112.24) --
	(445.12,111.73) --
	(446.66,111.25) --
	(448.19,110.80) --
	(449.73,110.38) --
	(451.27,109.99) --
	(452.80,109.63) --
	(454.34,109.31) --
	(455.88,109.01) --
	(457.41,108.74) --
	(458.95,108.50) --
	(460.49,108.29) --
	(462.02,108.11) --
	(463.56,107.96) --
	(465.10,107.84) --
	(466.64,107.75) --
	(468.17,107.69) --
	(469.71,107.66) --
	(471.25,107.66) --
	(472.78,107.66) --
	(474.32,107.66) --
	(475.86,107.66) --
	(477.39,107.66) --
	(478.93,107.66) --
	(480.47,107.66) --
	(482.01,107.66) --
	(483.54,107.66) --
	(485.08,107.66) --
	(486.62,107.66) --
	(488.15,107.66) --
	(489.69,107.66) --
	(491.23,107.66) --
	(492.76,107.66) --
	(494.30,107.66) --
	(495.84,107.66) --
	(497.38,107.66) --
	(498.91,107.66) --
	(500.45,107.66) --
	(501.99,107.66) --
	(503.52,107.66) --
	(505.06,107.66) --
	(506.60,107.66) --
	(508.13,107.66) --
	(509.67,107.66) --
	(511.21,107.66) --
	(512.74,107.66) --
	(514.28,107.66) --
	(515.82,107.66) --
	(517.36,107.66) --
	(518.89,107.66) --
	(520.43,107.66) --
	(521.97,107.66) --
	(523.50,107.66) --
	(525.04,107.66) --
	(526.58,107.66) --
	(528.11,107.66) --
	(529.65,107.66) --
	(531.19,107.66) --
	(532.73,107.66) --
	(534.26,107.66) --
	(535.80,107.66) --
	(537.34,107.66) --
	(538.87,107.66) --
	(540.41,107.66) --
	(541.95,107.66) --
	(543.48,107.66) --
	(545.02,107.66);
\definecolor{drawColor}{RGB}{63,81,180}

\path[draw=drawColor,line width= 1.1pt,line join=round] (394.40,147.26) --
	(395.93,148.66) --
	(397.47,149.99) --
	(399.01,151.24) --
	(400.55,152.42) --
	(402.08,153.53) --
	(403.62,154.56) --
	(405.16,155.52) --
	(406.69,156.40) --
	(408.23,157.22) --
	(409.77,157.95) --
	(411.30,158.62) --
	(412.84,159.21) --
	(414.38,159.73) --
	(415.92,160.17) --
	(417.45,160.54) --
	(418.99,160.84) --
	(420.53,161.06) --
	(422.06,161.21) --
	(423.60,161.28) --
	(425.14,161.29) --
	(426.67,161.21) --
	(428.21,161.07) --
	(429.75,160.85) --
	(431.29,160.56) --
	(432.82,160.19) --
	(434.36,159.75) --
	(435.90,159.24) --
	(437.43,158.66) --
	(438.97,158.00) --
	(440.51,157.26) --
	(442.04,156.46) --
	(443.58,155.58) --
	(445.12,154.62) --
	(446.66,153.59) --
	(448.19,152.49) --
	(449.73,151.32) --
	(451.27,150.07) --
	(452.80,148.75) --
	(454.34,147.35) --
	(455.88,145.88) --
	(457.41,144.34) --
	(458.95,142.72) --
	(460.49,141.04) --
	(462.02,139.27) --
	(463.56,137.44) --
	(465.10,135.52) --
	(466.64,133.54) --
	(468.17,131.48) --
	(469.71,129.35) --
	(471.25,127.17) --
	(472.78,125.09) --
	(474.32,123.12) --
	(475.86,121.27) --
	(477.39,119.53) --
	(478.93,117.92) --
	(480.47,116.42) --
	(482.01,115.04) --
	(483.54,113.78) --
	(485.08,112.64) --
	(486.62,111.62) --
	(488.15,110.71) --
	(489.69,109.92) --
	(491.23,109.25) --
	(492.76,108.69) --
	(494.30,108.26) --
	(495.84,107.94) --
	(497.38,107.74) --
	(498.91,107.66) --
	(500.45,107.66) --
	(501.99,107.66) --
	(503.52,107.66) --
	(505.06,107.66) --
	(506.60,107.66) --
	(508.13,107.66) --
	(509.67,107.66) --
	(511.21,107.66) --
	(512.74,107.66) --
	(514.28,107.66) --
	(515.82,107.66) --
	(517.36,107.66) --
	(518.89,107.66) --
	(520.43,107.66) --
	(521.97,107.66) --
	(523.50,107.66) --
	(525.04,107.66) --
	(526.58,107.66) --
	(528.11,107.66) --
	(529.65,107.66) --
	(531.19,107.66) --
	(532.73,107.66) --
	(534.26,107.66) --
	(535.80,107.66) --
	(537.34,107.66) --
	(538.87,107.66) --
	(540.41,107.66) --
	(541.95,107.66) --
	(543.48,107.66) --
	(545.02,107.66);
\definecolor{drawColor}{RGB}{2,169,243}

\path[draw=drawColor,line width= 1.1pt,line join=round] (394.40,107.68) --
	(395.93,107.75) --
	(397.47,107.85) --
	(399.01,108.01) --
	(400.55,108.20) --
	(402.08,108.44) --
	(403.62,108.73) --
	(405.16,109.06) --
	(406.69,109.43) --
	(408.23,109.84) --
	(409.77,110.30) --
	(411.30,110.80) --
	(412.84,111.35) --
	(414.38,111.94) --
	(415.92,112.57) --
	(417.45,113.25) --
	(418.99,113.97) --
	(420.53,114.73) --
	(422.06,115.54) --
	(423.60,116.39) --
	(425.14,117.28) --
	(426.67,118.22) --
	(428.21,119.20) --
	(429.75,120.23) --
	(431.29,121.30) --
	(432.82,122.41) --
	(434.36,123.57) --
	(435.90,124.77) --
	(437.43,126.01) --
	(438.97,127.30) --
	(440.51,128.63) --
	(442.04,130.01) --
	(443.58,131.43) --
	(445.12,132.89) --
	(446.66,134.40) --
	(448.19,135.95) --
	(449.73,137.54) --
	(451.27,139.18) --
	(452.80,140.86) --
	(454.34,142.58) --
	(455.88,144.35) --
	(457.41,146.16) --
	(458.95,148.02) --
	(460.49,149.92) --
	(462.02,151.86) --
	(463.56,153.84) --
	(465.10,155.87) --
	(466.64,157.95) --
	(468.17,160.06) --
	(469.71,162.23) --
	(471.25,164.38) --
	(472.78,166.20) --
	(474.32,167.65) --
	(475.86,168.72) --
	(477.39,169.41) --
	(478.93,169.72) --
	(480.47,169.65) --
	(482.01,169.21) --
	(483.54,168.39) --
	(485.08,167.19) --
	(486.62,165.61) --
	(488.15,163.65) --
	(489.69,161.31) --
	(491.23,158.60) --
	(492.76,155.50) --
	(494.30,152.03) --
	(495.84,148.18) --
	(497.38,143.96) --
	(498.91,139.35) --
	(500.45,134.62) --
	(501.99,130.25) --
	(503.52,126.27) --
	(505.06,122.68) --
	(506.60,119.47) --
	(508.13,116.65) --
	(509.67,114.21) --
	(511.21,112.15) --
	(512.74,110.48) --
	(514.28,109.20) --
	(515.82,108.30) --
	(517.36,107.79) --
	(518.89,107.66) --
	(520.43,107.66) --
	(521.97,107.66) --
	(523.50,107.66) --
	(525.04,107.66) --
	(526.58,107.66) --
	(528.11,107.66) --
	(529.65,107.66) --
	(531.19,107.66) --
	(532.73,107.66) --
	(534.26,107.66) --
	(535.80,107.66) --
	(537.34,107.66) --
	(538.87,107.66) --
	(540.41,107.66) --
	(541.95,107.66) --
	(543.48,107.66) --
	(545.02,107.66);
\definecolor{drawColor}{RGB}{0,150,135}

\path[draw=drawColor,line width= 1.1pt,line join=round] (394.40,107.66) --
	(395.93,107.66) --
	(397.47,107.66) --
	(399.01,107.66) --
	(400.55,107.66) --
	(402.08,107.66) --
	(403.62,107.66) --
	(405.16,107.66) --
	(406.69,107.66) --
	(408.23,107.66) --
	(409.77,107.66) --
	(411.30,107.66) --
	(412.84,107.66) --
	(414.38,107.66) --
	(415.92,107.66) --
	(417.45,107.66) --
	(418.99,107.66) --
	(420.53,107.66) --
	(422.06,107.66) --
	(423.60,107.66) --
	(425.14,107.66) --
	(426.67,107.66) --
	(428.21,107.66) --
	(429.75,107.66) --
	(431.29,107.66) --
	(432.82,107.66) --
	(434.36,107.66) --
	(435.90,107.66) --
	(437.43,107.66) --
	(438.97,107.66) --
	(440.51,107.66) --
	(442.04,107.66) --
	(443.58,107.66) --
	(445.12,107.66) --
	(446.66,107.66) --
	(448.19,107.66) --
	(449.73,107.66) --
	(451.27,107.66) --
	(452.80,107.66) --
	(454.34,107.66) --
	(455.88,107.66) --
	(457.41,107.66) --
	(458.95,107.66) --
	(460.49,107.66) --
	(462.02,107.66) --
	(463.56,107.66) --
	(465.10,107.66) --
	(466.64,107.66) --
	(468.17,107.66) --
	(469.71,107.66) --
	(471.25,107.69) --
	(472.78,107.95) --
	(474.32,108.47) --
	(475.86,109.26) --
	(477.39,110.30) --
	(478.93,111.60) --
	(480.47,113.16) --
	(482.01,114.99) --
	(483.54,117.07) --
	(485.08,119.41) --
	(486.62,122.02) --
	(488.15,124.88) --
	(489.69,128.01) --
	(491.23,131.39) --
	(492.76,135.04) --
	(494.30,138.95) --
	(495.84,143.11) --
	(497.38,147.54) --
	(498.91,152.23) --
	(500.45,156.78) --
	(501.99,160.42) --
	(503.52,163.11) --
	(505.06,164.86) --
	(506.60,165.67) --
	(508.13,165.53) --
	(509.67,164.45) --
	(511.21,162.43) --
	(512.74,159.46) --
	(514.28,155.55) --
	(515.82,150.70) --
	(517.36,144.91) --
	(518.89,138.19) --
	(520.43,131.67) --
	(521.97,125.92) --
	(523.50,120.97) --
	(525.04,116.79) --
	(526.58,113.40) --
	(528.11,110.79) --
	(529.65,108.97) --
	(531.19,107.93) --
	(532.73,107.66) --
	(534.26,107.66) --
	(535.80,107.66) --
	(537.34,107.66) --
	(538.87,107.66) --
	(540.41,107.66) --
	(541.95,107.66) --
	(543.48,107.66) --
	(545.02,107.66);
\definecolor{drawColor}{RGB}{139,195,74}

\path[draw=drawColor,line width= 1.1pt,line join=round] (394.40,107.66) --
	(395.93,107.66) --
	(397.47,107.66) --
	(399.01,107.66) --
	(400.55,107.66) --
	(402.08,107.66) --
	(403.62,107.66) --
	(405.16,107.66) --
	(406.69,107.66) --
	(408.23,107.66) --
	(409.77,107.66) --
	(411.30,107.66) --
	(412.84,107.66) --
	(414.38,107.66) --
	(415.92,107.66) --
	(417.45,107.66) --
	(418.99,107.66) --
	(420.53,107.66) --
	(422.06,107.66) --
	(423.60,107.66) --
	(425.14,107.66) --
	(426.67,107.66) --
	(428.21,107.66) --
	(429.75,107.66) --
	(431.29,107.66) --
	(432.82,107.66) --
	(434.36,107.66) --
	(435.90,107.66) --
	(437.43,107.66) --
	(438.97,107.66) --
	(440.51,107.66) --
	(442.04,107.66) --
	(443.58,107.66) --
	(445.12,107.66) --
	(446.66,107.66) --
	(448.19,107.66) --
	(449.73,107.66) --
	(451.27,107.66) --
	(452.80,107.66) --
	(454.34,107.66) --
	(455.88,107.66) --
	(457.41,107.66) --
	(458.95,107.66) --
	(460.49,107.66) --
	(462.02,107.66) --
	(463.56,107.66) --
	(465.10,107.66) --
	(466.64,107.66) --
	(468.17,107.66) --
	(469.71,107.66) --
	(471.25,107.66) --
	(472.78,107.66) --
	(474.32,107.66) --
	(475.86,107.66) --
	(477.39,107.66) --
	(478.93,107.66) --
	(480.47,107.66) --
	(482.01,107.66) --
	(483.54,107.66) --
	(485.08,107.66) --
	(486.62,107.66) --
	(488.15,107.66) --
	(489.69,107.66) --
	(491.23,107.66) --
	(492.76,107.66) --
	(494.30,107.66) --
	(495.84,107.66) --
	(497.38,107.66) --
	(498.91,107.66) --
	(500.45,107.84) --
	(501.99,108.57) --
	(503.52,109.86) --
	(505.06,111.70) --
	(506.60,114.10) --
	(508.13,117.06) --
	(509.67,120.58) --
	(511.21,124.66) --
	(512.74,129.29) --
	(514.28,134.48) --
	(515.82,140.23) --
	(517.36,146.54) --
	(518.89,153.37) --
	(520.43,159.15) --
	(521.97,163.02) --
	(523.50,164.99) --
	(525.04,165.05) --
	(526.58,163.21) --
	(528.11,159.46) --
	(529.65,153.81) --
	(531.19,146.25) --
	(532.73,136.84) --
	(534.26,127.86) --
	(535.80,120.53) --
	(537.34,114.84) --
	(538.87,110.80) --
	(540.41,108.41) --
	(541.95,107.66) --
	(543.48,107.66) --
	(545.02,107.66);
\definecolor{drawColor}{RGB}{255,235,58}

\path[draw=drawColor,line width= 1.1pt,line join=round] (394.40,107.66) --
	(395.93,107.66) --
	(397.47,107.66) --
	(399.01,107.66) --
	(400.55,107.66) --
	(402.08,107.66) --
	(403.62,107.66) --
	(405.16,107.66) --
	(406.69,107.66) --
	(408.23,107.66) --
	(409.77,107.66) --
	(411.30,107.66) --
	(412.84,107.66) --
	(414.38,107.66) --
	(415.92,107.66) --
	(417.45,107.66) --
	(418.99,107.66) --
	(420.53,107.66) --
	(422.06,107.66) --
	(423.60,107.66) --
	(425.14,107.66) --
	(426.67,107.66) --
	(428.21,107.66) --
	(429.75,107.66) --
	(431.29,107.66) --
	(432.82,107.66) --
	(434.36,107.66) --
	(435.90,107.66) --
	(437.43,107.66) --
	(438.97,107.66) --
	(440.51,107.66) --
	(442.04,107.66) --
	(443.58,107.66) --
	(445.12,107.66) --
	(446.66,107.66) --
	(448.19,107.66) --
	(449.73,107.66) --
	(451.27,107.66) --
	(452.80,107.66) --
	(454.34,107.66) --
	(455.88,107.66) --
	(457.41,107.66) --
	(458.95,107.66) --
	(460.49,107.66) --
	(462.02,107.66) --
	(463.56,107.66) --
	(465.10,107.66) --
	(466.64,107.66) --
	(468.17,107.66) --
	(469.71,107.66) --
	(471.25,107.66) --
	(472.78,107.66) --
	(474.32,107.66) --
	(475.86,107.66) --
	(477.39,107.66) --
	(478.93,107.66) --
	(480.47,107.66) --
	(482.01,107.66) --
	(483.54,107.66) --
	(485.08,107.66) --
	(486.62,107.66) --
	(488.15,107.66) --
	(489.69,107.66) --
	(491.23,107.66) --
	(492.76,107.66) --
	(494.30,107.66) --
	(495.84,107.66) --
	(497.38,107.66) --
	(498.91,107.66) --
	(500.45,107.66) --
	(501.99,107.66) --
	(503.52,107.66) --
	(505.06,107.66) --
	(506.60,107.66) --
	(508.13,107.66) --
	(509.67,107.66) --
	(511.21,107.66) --
	(512.74,107.66) --
	(514.28,107.66) --
	(515.82,107.66) --
	(517.36,107.66) --
	(518.89,107.67) --
	(520.43,108.42) --
	(521.97,110.30) --
	(523.50,113.29) --
	(525.04,117.40) --
	(526.58,122.64) --
	(528.11,128.99) --
	(529.65,136.47) --
	(531.19,145.07) --
	(532.73,154.70) --
	(534.26,161.86) --
	(535.80,164.65) --
	(537.34,163.08) --
	(538.87,157.15) --
	(540.41,146.86) --
	(541.95,132.22) --
	(543.48,118.58) --
	(545.02,110.39);
\definecolor{drawColor}{RGB}{255,152,0}

\path[draw=drawColor,line width= 1.1pt,line join=round] (394.40,107.66) --
	(395.93,107.66) --
	(397.47,107.66) --
	(399.01,107.66) --
	(400.55,107.66) --
	(402.08,107.66) --
	(403.62,107.66) --
	(405.16,107.66) --
	(406.69,107.66) --
	(408.23,107.66) --
	(409.77,107.66) --
	(411.30,107.66) --
	(412.84,107.66) --
	(414.38,107.66) --
	(415.92,107.66) --
	(417.45,107.66) --
	(418.99,107.66) --
	(420.53,107.66) --
	(422.06,107.66) --
	(423.60,107.66) --
	(425.14,107.66) --
	(426.67,107.66) --
	(428.21,107.66) --
	(429.75,107.66) --
	(431.29,107.66) --
	(432.82,107.66) --
	(434.36,107.66) --
	(435.90,107.66) --
	(437.43,107.66) --
	(438.97,107.66) --
	(440.51,107.66) --
	(442.04,107.66) --
	(443.58,107.66) --
	(445.12,107.66) --
	(446.66,107.66) --
	(448.19,107.66) --
	(449.73,107.66) --
	(451.27,107.66) --
	(452.80,107.66) --
	(454.34,107.66) --
	(455.88,107.66) --
	(457.41,107.66) --
	(458.95,107.66) --
	(460.49,107.66) --
	(462.02,107.66) --
	(463.56,107.66) --
	(465.10,107.66) --
	(466.64,107.66) --
	(468.17,107.66) --
	(469.71,107.66) --
	(471.25,107.66) --
	(472.78,107.66) --
	(474.32,107.66) --
	(475.86,107.66) --
	(477.39,107.66) --
	(478.93,107.66) --
	(480.47,107.66) --
	(482.01,107.66) --
	(483.54,107.66) --
	(485.08,107.66) --
	(486.62,107.66) --
	(488.15,107.66) --
	(489.69,107.66) --
	(491.23,107.66) --
	(492.76,107.66) --
	(494.30,107.66) --
	(495.84,107.66) --
	(497.38,107.66) --
	(498.91,107.66) --
	(500.45,107.66) --
	(501.99,107.66) --
	(503.52,107.66) --
	(505.06,107.66) --
	(506.60,107.66) --
	(508.13,107.66) --
	(509.67,107.66) --
	(511.21,107.66) --
	(512.74,107.66) --
	(514.28,107.66) --
	(515.82,107.66) --
	(517.36,107.66) --
	(518.89,107.66) --
	(520.43,107.66) --
	(521.97,107.66) --
	(523.50,107.66) --
	(525.04,107.66) --
	(526.58,107.66) --
	(528.11,107.66) --
	(529.65,107.66) --
	(531.19,107.66) --
	(532.73,107.70) --
	(534.26,109.52) --
	(535.80,114.06) --
	(537.34,121.32) --
	(538.87,131.29) --
	(540.41,143.97) --
	(541.95,159.35) --
	(543.48,168.51) --
	(545.02,163.99);
\definecolor{drawColor}{RGB}{121,84,71}

\path[draw=drawColor,line width= 1.1pt,line join=round] (394.40,107.66) --
	(395.93,107.66) --
	(397.47,107.66) --
	(399.01,107.66) --
	(400.55,107.66) --
	(402.08,107.66) --
	(403.62,107.66) --
	(405.16,107.66) --
	(406.69,107.66) --
	(408.23,107.66) --
	(409.77,107.66) --
	(411.30,107.66) --
	(412.84,107.66) --
	(414.38,107.66) --
	(415.92,107.66) --
	(417.45,107.66) --
	(418.99,107.66) --
	(420.53,107.66) --
	(422.06,107.66) --
	(423.60,107.66) --
	(425.14,107.66) --
	(426.67,107.66) --
	(428.21,107.66) --
	(429.75,107.66) --
	(431.29,107.66) --
	(432.82,107.66) --
	(434.36,107.66) --
	(435.90,107.66) --
	(437.43,107.66) --
	(438.97,107.66) --
	(440.51,107.66) --
	(442.04,107.66) --
	(443.58,107.66) --
	(445.12,107.66) --
	(446.66,107.66) --
	(448.19,107.66) --
	(449.73,107.66) --
	(451.27,107.66) --
	(452.80,107.66) --
	(454.34,107.66) --
	(455.88,107.66) --
	(457.41,107.66) --
	(458.95,107.66) --
	(460.49,107.66) --
	(462.02,107.66) --
	(463.56,107.66) --
	(465.10,107.66) --
	(466.64,107.66) --
	(468.17,107.66) --
	(469.71,107.66) --
	(471.25,107.66) --
	(472.78,107.66) --
	(474.32,107.66) --
	(475.86,107.66) --
	(477.39,107.66) --
	(478.93,107.66) --
	(480.47,107.66) --
	(482.01,107.66) --
	(483.54,107.66) --
	(485.08,107.66) --
	(486.62,107.66) --
	(488.15,107.66) --
	(489.69,107.66) --
	(491.23,107.66) --
	(492.76,107.66) --
	(494.30,107.66) --
	(495.84,107.66) --
	(497.38,107.66) --
	(498.91,107.66) --
	(500.45,107.66) --
	(501.99,107.66) --
	(503.52,107.66) --
	(505.06,107.66) --
	(506.60,107.66) --
	(508.13,107.66) --
	(509.67,107.66) --
	(511.21,107.66) --
	(512.74,107.66) --
	(514.28,107.66) --
	(515.82,107.66) --
	(517.36,107.66) --
	(518.89,107.66) --
	(520.43,107.66) --
	(521.97,107.66) --
	(523.50,107.66) --
	(525.04,107.66) --
	(526.58,107.66) --
	(528.11,107.66) --
	(529.65,107.66) --
	(531.19,107.66) --
	(532.73,107.66) --
	(534.26,107.66) --
	(535.80,107.66) --
	(537.34,107.66) --
	(538.87,107.66) --
	(540.41,107.66) --
	(541.95,107.67) --
	(543.48,112.16) --
	(545.02,124.87);
\end{scope}
\begin{scope}
\path[clip] (386.87, 21.76) rectangle (552.55, 98.66);
\definecolor{drawColor}{gray}{0.92}

\path[draw=drawColor,line width= 0.3pt,line join=round] (386.87, 34.79) --
	(552.55, 34.79);

\path[draw=drawColor,line width= 0.3pt,line join=round] (386.87, 53.85) --
	(552.55, 53.85);

\path[draw=drawColor,line width= 0.3pt,line join=round] (386.87, 72.92) --
	(552.55, 72.92);

\path[draw=drawColor,line width= 0.3pt,line join=round] (386.87, 91.99) --
	(552.55, 91.99);

\path[draw=drawColor,line width= 0.3pt,line join=round] (412.07, 21.76) --
	(412.07, 98.66);

\path[draw=drawColor,line width= 0.3pt,line join=round] (450.50, 21.76) --
	(450.50, 98.66);

\path[draw=drawColor,line width= 0.3pt,line join=round] (488.92, 21.76) --
	(488.92, 98.66);

\path[draw=drawColor,line width= 0.3pt,line join=round] (527.35, 21.76) --
	(527.35, 98.66);

\path[draw=drawColor,line width= 0.6pt,line join=round] (386.87, 25.25) --
	(552.55, 25.25);

\path[draw=drawColor,line width= 0.6pt,line join=round] (386.87, 44.32) --
	(552.55, 44.32);

\path[draw=drawColor,line width= 0.6pt,line join=round] (386.87, 63.39) --
	(552.55, 63.39);

\path[draw=drawColor,line width= 0.6pt,line join=round] (386.87, 82.45) --
	(552.55, 82.45);

\path[draw=drawColor,line width= 0.6pt,line join=round] (392.86, 21.76) --
	(392.86, 98.66);

\path[draw=drawColor,line width= 0.6pt,line join=round] (431.29, 21.76) --
	(431.29, 98.66);

\path[draw=drawColor,line width= 0.6pt,line join=round] (469.71, 21.76) --
	(469.71, 98.66);

\path[draw=drawColor,line width= 0.6pt,line join=round] (508.13, 21.76) --
	(508.13, 98.66);

\path[draw=drawColor,line width= 0.6pt,line join=round] (546.56, 21.76) --
	(546.56, 98.66);
\definecolor{drawColor}{RGB}{155,38,176}

\path[draw=drawColor,line width= 1.1pt,line join=round] (394.40, 61.11) --
	(395.93, 58.91) --
	(397.47, 56.77) --
	(399.01, 54.71) --
	(400.55, 52.71) --
	(402.08, 50.79) --
	(403.62, 48.93) --
	(405.16, 47.15) --
	(406.69, 45.43) --
	(408.23, 43.79) --
	(409.77, 42.21) --
	(411.30, 40.71) --
	(412.84, 39.27) --
	(414.38, 37.91) --
	(415.92, 36.61) --
	(417.45, 35.38) --
	(418.99, 34.23) --
	(420.53, 33.14) --
	(422.06, 32.13) --
	(423.60, 31.18) --
	(425.14, 30.31) --
	(426.67, 29.50) --
	(428.21, 28.76) --
	(429.75, 28.10) --
	(431.29, 27.50) --
	(432.82, 26.98) --
	(434.36, 26.52) --
	(435.90, 26.13) --
	(437.43, 25.82) --
	(438.97, 25.57) --
	(440.51, 25.40) --
	(442.04, 25.29) --
	(443.58, 25.25) --
	(445.12, 25.25) --
	(446.66, 25.25) --
	(448.19, 25.25) --
	(449.73, 25.25) --
	(451.27, 25.25) --
	(452.80, 25.25) --
	(454.34, 25.25) --
	(455.88, 25.25) --
	(457.41, 25.25) --
	(458.95, 25.25) --
	(460.49, 25.25) --
	(462.02, 25.25) --
	(463.56, 25.25) --
	(465.10, 25.25) --
	(466.64, 25.25) --
	(468.17, 25.25) --
	(469.71, 25.25) --
	(471.25, 25.25) --
	(472.78, 25.25) --
	(474.32, 25.25) --
	(475.86, 25.25) --
	(477.39, 25.25) --
	(478.93, 25.25) --
	(480.47, 25.25) --
	(482.01, 25.25) --
	(483.54, 25.25) --
	(485.08, 25.25) --
	(486.62, 25.25) --
	(488.15, 25.25) --
	(489.69, 25.25) --
	(491.23, 25.25) --
	(492.76, 25.25) --
	(494.30, 25.25) --
	(495.84, 25.25) --
	(497.38, 25.25) --
	(498.91, 25.25) --
	(500.45, 25.25) --
	(501.99, 25.25) --
	(503.52, 25.25) --
	(505.06, 25.25) --
	(506.60, 25.25) --
	(508.13, 25.25) --
	(509.67, 25.25) --
	(511.21, 25.25) --
	(512.74, 25.25) --
	(514.28, 25.25) --
	(515.82, 25.25) --
	(517.36, 25.25) --
	(518.89, 25.25) --
	(520.43, 25.25) --
	(521.97, 25.25) --
	(523.50, 25.25) --
	(525.04, 25.25) --
	(526.58, 25.25) --
	(528.11, 25.25) --
	(529.65, 25.25) --
	(531.19, 25.25) --
	(532.73, 25.25) --
	(534.26, 25.25) --
	(535.80, 25.25) --
	(537.34, 25.25) --
	(538.87, 25.25) --
	(540.41, 25.25) --
	(541.95, 25.25) --
	(543.48, 25.25) --
	(545.02, 25.25);
\definecolor{drawColor}{RGB}{63,81,180}

\path[draw=drawColor,line width= 1.1pt,line join=round] (394.40, 65.63) --
	(395.93, 67.74) --
	(397.47, 69.72) --
	(399.01, 71.56) --
	(400.55, 73.28) --
	(402.08, 74.86) --
	(403.62, 76.30) --
	(405.16, 77.62) --
	(406.69, 78.80) --
	(408.23, 79.85) --
	(409.77, 80.77) --
	(411.30, 81.55) --
	(412.84, 82.20) --
	(414.38, 82.72) --
	(415.92, 83.11) --
	(417.45, 83.36) --
	(418.99, 83.48) --
	(420.53, 83.47) --
	(422.06, 83.32) --
	(423.60, 83.05) --
	(425.14, 82.64) --
	(426.67, 82.10) --
	(428.21, 81.42) --
	(429.75, 80.61) --
	(431.29, 79.67) --
	(432.82, 78.60) --
	(434.36, 77.39) --
	(435.90, 76.06) --
	(437.43, 74.58) --
	(438.97, 72.98) --
	(440.51, 71.24) --
	(442.04, 69.38) --
	(443.58, 67.37) --
	(445.12, 65.33) --
	(446.66, 63.33) --
	(448.19, 61.39) --
	(449.73, 59.49) --
	(451.27, 57.65) --
	(452.80, 55.86) --
	(454.34, 54.12) --
	(455.88, 52.43) --
	(457.41, 50.79) --
	(458.95, 49.20) --
	(460.49, 47.67) --
	(462.02, 46.18) --
	(463.56, 44.75) --
	(465.10, 43.36) --
	(466.64, 42.03) --
	(468.17, 40.75) --
	(469.71, 39.52) --
	(471.25, 38.34) --
	(472.78, 37.21) --
	(474.32, 36.13) --
	(475.86, 35.10) --
	(477.39, 34.13) --
	(478.93, 33.20) --
	(480.47, 32.33) --
	(482.01, 31.50) --
	(483.54, 30.73) --
	(485.08, 30.01) --
	(486.62, 29.34) --
	(488.15, 28.72) --
	(489.69, 28.15) --
	(491.23, 27.63) --
	(492.76, 27.16) --
	(494.30, 26.75) --
	(495.84, 26.38) --
	(497.38, 26.07) --
	(498.91, 25.81) --
	(500.45, 25.59) --
	(501.99, 25.43) --
	(503.52, 25.32) --
	(505.06, 25.26) --
	(506.60, 25.25) --
	(508.13, 25.25) --
	(509.67, 25.25) --
	(511.21, 25.25) --
	(512.74, 25.25) --
	(514.28, 25.25) --
	(515.82, 25.25) --
	(517.36, 25.25) --
	(518.89, 25.25) --
	(520.43, 25.25) --
	(521.97, 25.25) --
	(523.50, 25.25) --
	(525.04, 25.25) --
	(526.58, 25.25) --
	(528.11, 25.25) --
	(529.65, 25.25) --
	(531.19, 25.25) --
	(532.73, 25.25) --
	(534.26, 25.25) --
	(535.80, 25.25) --
	(537.34, 25.25) --
	(538.87, 25.25) --
	(540.41, 25.25) --
	(541.95, 25.25) --
	(543.48, 25.25) --
	(545.02, 25.25);
\definecolor{drawColor}{RGB}{2,169,243}

\path[draw=drawColor,line width= 1.1pt,line join=round] (394.40, 25.28) --
	(395.93, 25.38) --
	(397.47, 25.54) --
	(399.01, 25.75) --
	(400.55, 26.04) --
	(402.08, 26.38) --
	(403.62, 26.79) --
	(405.16, 27.26) --
	(406.69, 27.79) --
	(408.23, 28.39) --
	(409.77, 29.05) --
	(411.30, 29.77) --
	(412.84, 30.55) --
	(414.38, 31.40) --
	(415.92, 32.31) --
	(417.45, 33.28) --
	(418.99, 34.31) --
	(420.53, 35.41) --
	(422.06, 36.57) --
	(423.60, 37.79) --
	(425.14, 39.08) --
	(426.67, 40.43) --
	(428.21, 41.84) --
	(429.75, 43.31) --
	(431.29, 44.85) --
	(432.82, 46.45) --
	(434.36, 48.11) --
	(435.90, 49.83) --
	(437.43, 51.62) --
	(438.97, 53.47) --
	(440.51, 55.38) --
	(442.04, 57.36) --
	(443.58, 59.40) --
	(445.12, 61.41) --
	(446.66, 63.32) --
	(448.19, 65.11) --
	(449.73, 66.79) --
	(451.27, 68.35) --
	(452.80, 69.81) --
	(454.34, 71.15) --
	(455.88, 72.37) --
	(457.41, 73.49) --
	(458.95, 74.49) --
	(460.49, 75.38) --
	(462.02, 76.16) --
	(463.56, 76.82) --
	(465.10, 77.37) --
	(466.64, 77.81) --
	(468.17, 78.13) --
	(469.71, 78.35) --
	(471.25, 78.45) --
	(472.78, 78.43) --
	(474.32, 78.31) --
	(475.86, 78.07) --
	(477.39, 77.72) --
	(478.93, 77.25) --
	(480.47, 76.68) --
	(482.01, 75.99) --
	(483.54, 75.19) --
	(485.08, 74.27) --
	(486.62, 73.24) --
	(488.15, 72.10) --
	(489.69, 70.85) --
	(491.23, 69.48) --
	(492.76, 68.01) --
	(494.30, 66.41) --
	(495.84, 64.71) --
	(497.38, 62.89) --
	(498.91, 60.96) --
	(500.45, 58.92) --
	(501.99, 56.77) --
	(503.52, 54.50) --
	(505.06, 52.12) --
	(506.60, 49.64) --
	(508.13, 47.23) --
	(509.67, 44.95) --
	(511.21, 42.79) --
	(512.74, 40.76) --
	(514.28, 38.85) --
	(515.82, 37.07) --
	(517.36, 35.42) --
	(518.89, 33.88) --
	(520.43, 32.48) --
	(521.97, 31.20) --
	(523.50, 30.04) --
	(525.04, 29.01) --
	(526.58, 28.10) --
	(528.11, 27.32) --
	(529.65, 26.66) --
	(531.19, 26.13) --
	(532.73, 25.73) --
	(534.26, 25.44) --
	(535.80, 25.29) --
	(537.34, 25.25) --
	(538.87, 25.25) --
	(540.41, 25.25) --
	(541.95, 25.25) --
	(543.48, 25.25) --
	(545.02, 25.25);
\definecolor{drawColor}{RGB}{0,150,135}

\path[draw=drawColor,line width= 1.1pt,line join=round] (394.40, 25.25) --
	(395.93, 25.25) --
	(397.47, 25.25) --
	(399.01, 25.25) --
	(400.55, 25.25) --
	(402.08, 25.25) --
	(403.62, 25.25) --
	(405.16, 25.25) --
	(406.69, 25.25) --
	(408.23, 25.25) --
	(409.77, 25.25) --
	(411.30, 25.25) --
	(412.84, 25.25) --
	(414.38, 25.25) --
	(415.92, 25.25) --
	(417.45, 25.25) --
	(418.99, 25.25) --
	(420.53, 25.25) --
	(422.06, 25.25) --
	(423.60, 25.25) --
	(425.14, 25.25) --
	(426.67, 25.25) --
	(428.21, 25.25) --
	(429.75, 25.25) --
	(431.29, 25.25) --
	(432.82, 25.25) --
	(434.36, 25.25) --
	(435.90, 25.25) --
	(437.43, 25.25) --
	(438.97, 25.25) --
	(440.51, 25.25) --
	(442.04, 25.25) --
	(443.58, 25.25) --
	(445.12, 25.28) --
	(446.66, 25.37) --
	(448.19, 25.53) --
	(449.73, 25.74) --
	(451.27, 26.02) --
	(452.80, 26.36) --
	(454.34, 26.76) --
	(455.88, 27.22) --
	(457.41, 27.74) --
	(458.95, 28.33) --
	(460.49, 28.98) --
	(462.02, 29.69) --
	(463.56, 30.46) --
	(465.10, 31.29) --
	(466.64, 32.19) --
	(468.17, 33.14) --
	(469.71, 34.16) --
	(471.25, 35.24) --
	(472.78, 36.38) --
	(474.32, 37.59) --
	(475.86, 38.85) --
	(477.39, 40.18) --
	(478.93, 41.57) --
	(480.47, 43.02) --
	(482.01, 44.53) --
	(483.54, 46.11) --
	(485.08, 47.74) --
	(486.62, 49.44) --
	(488.15, 51.20) --
	(489.69, 53.02) --
	(491.23, 54.91) --
	(492.76, 56.85) --
	(494.30, 58.86) --
	(495.84, 60.93) --
	(497.38, 63.06) --
	(498.91, 65.25) --
	(500.45, 67.51) --
	(501.99, 69.82) --
	(503.52, 72.20) --
	(505.06, 74.64) --
	(506.60, 77.12) --
	(508.13, 79.27) --
	(509.67, 81.01) --
	(511.21, 82.33) --
	(512.74, 83.23) --
	(514.28, 83.70) --
	(515.82, 83.75) --
	(517.36, 83.39) --
	(518.89, 82.60) --
	(520.43, 81.39) --
	(521.97, 79.76) --
	(523.50, 77.71) --
	(525.04, 75.23) --
	(526.58, 72.34) --
	(528.11, 69.03) --
	(529.65, 65.29) --
	(531.19, 61.13) --
	(532.73, 56.56) --
	(534.26, 51.56) --
	(535.80, 46.14) --
	(537.34, 40.34) --
	(538.87, 35.20) --
	(540.41, 31.12) --
	(541.95, 28.11) --
	(543.48, 26.18) --
	(545.02, 25.31);
\definecolor{drawColor}{RGB}{139,195,74}

\path[draw=drawColor,line width= 1.1pt,line join=round] (394.40, 25.25) --
	(395.93, 25.25) --
	(397.47, 25.25) --
	(399.01, 25.25) --
	(400.55, 25.25) --
	(402.08, 25.25) --
	(403.62, 25.25) --
	(405.16, 25.25) --
	(406.69, 25.25) --
	(408.23, 25.25) --
	(409.77, 25.25) --
	(411.30, 25.25) --
	(412.84, 25.25) --
	(414.38, 25.25) --
	(415.92, 25.25) --
	(417.45, 25.25) --
	(418.99, 25.25) --
	(420.53, 25.25) --
	(422.06, 25.25) --
	(423.60, 25.25) --
	(425.14, 25.25) --
	(426.67, 25.25) --
	(428.21, 25.25) --
	(429.75, 25.25) --
	(431.29, 25.25) --
	(432.82, 25.25) --
	(434.36, 25.25) --
	(435.90, 25.25) --
	(437.43, 25.25) --
	(438.97, 25.25) --
	(440.51, 25.25) --
	(442.04, 25.25) --
	(443.58, 25.25) --
	(445.12, 25.25) --
	(446.66, 25.25) --
	(448.19, 25.25) --
	(449.73, 25.25) --
	(451.27, 25.25) --
	(452.80, 25.25) --
	(454.34, 25.25) --
	(455.88, 25.25) --
	(457.41, 25.25) --
	(458.95, 25.25) --
	(460.49, 25.25) --
	(462.02, 25.25) --
	(463.56, 25.25) --
	(465.10, 25.25) --
	(466.64, 25.25) --
	(468.17, 25.25) --
	(469.71, 25.25) --
	(471.25, 25.25) --
	(472.78, 25.25) --
	(474.32, 25.25) --
	(475.86, 25.25) --
	(477.39, 25.25) --
	(478.93, 25.25) --
	(480.47, 25.25) --
	(482.01, 25.25) --
	(483.54, 25.25) --
	(485.08, 25.25) --
	(486.62, 25.25) --
	(488.15, 25.25) --
	(489.69, 25.25) --
	(491.23, 25.25) --
	(492.76, 25.25) --
	(494.30, 25.25) --
	(495.84, 25.25) --
	(497.38, 25.25) --
	(498.91, 25.25) --
	(500.45, 25.25) --
	(501.99, 25.25) --
	(503.52, 25.25) --
	(505.06, 25.25) --
	(506.60, 25.27) --
	(508.13, 25.52) --
	(509.67, 26.06) --
	(511.21, 26.90) --
	(512.74, 28.04) --
	(514.28, 29.47) --
	(515.82, 31.20) --
	(517.36, 33.22) --
	(518.89, 35.54) --
	(520.43, 38.16) --
	(521.97, 41.07) --
	(523.50, 44.28) --
	(525.04, 47.78) --
	(526.58, 51.58) --
	(528.11, 55.68) --
	(529.65, 60.07) --
	(531.19, 64.76) --
	(532.73, 69.74) --
	(534.26, 75.02) --
	(535.80, 80.60) --
	(537.34, 86.29) --
	(538.87, 88.14) --
	(540.41, 84.52) --
	(541.95, 75.45) --
	(543.48, 60.90) --
	(545.02, 40.89);
\definecolor{drawColor}{RGB}{255,235,58}

\path[draw=drawColor,line width= 1.1pt,line join=round] (394.40, 25.25) --
	(395.93, 25.25) --
	(397.47, 25.25) --
	(399.01, 25.25) --
	(400.55, 25.25) --
	(402.08, 25.25) --
	(403.62, 25.25) --
	(405.16, 25.25) --
	(406.69, 25.25) --
	(408.23, 25.25) --
	(409.77, 25.25) --
	(411.30, 25.25) --
	(412.84, 25.25) --
	(414.38, 25.25) --
	(415.92, 25.25) --
	(417.45, 25.25) --
	(418.99, 25.25) --
	(420.53, 25.25) --
	(422.06, 25.25) --
	(423.60, 25.25) --
	(425.14, 25.25) --
	(426.67, 25.25) --
	(428.21, 25.25) --
	(429.75, 25.25) --
	(431.29, 25.25) --
	(432.82, 25.25) --
	(434.36, 25.25) --
	(435.90, 25.25) --
	(437.43, 25.25) --
	(438.97, 25.25) --
	(440.51, 25.25) --
	(442.04, 25.25) --
	(443.58, 25.25) --
	(445.12, 25.25) --
	(446.66, 25.25) --
	(448.19, 25.25) --
	(449.73, 25.25) --
	(451.27, 25.25) --
	(452.80, 25.25) --
	(454.34, 25.25) --
	(455.88, 25.25) --
	(457.41, 25.25) --
	(458.95, 25.25) --
	(460.49, 25.25) --
	(462.02, 25.25) --
	(463.56, 25.25) --
	(465.10, 25.25) --
	(466.64, 25.25) --
	(468.17, 25.25) --
	(469.71, 25.25) --
	(471.25, 25.25) --
	(472.78, 25.25) --
	(474.32, 25.25) --
	(475.86, 25.25) --
	(477.39, 25.25) --
	(478.93, 25.25) --
	(480.47, 25.25) --
	(482.01, 25.25) --
	(483.54, 25.25) --
	(485.08, 25.25) --
	(486.62, 25.25) --
	(488.15, 25.25) --
	(489.69, 25.25) --
	(491.23, 25.25) --
	(492.76, 25.25) --
	(494.30, 25.25) --
	(495.84, 25.25) --
	(497.38, 25.25) --
	(498.91, 25.25) --
	(500.45, 25.25) --
	(501.99, 25.25) --
	(503.52, 25.25) --
	(505.06, 25.25) --
	(506.60, 25.25) --
	(508.13, 25.25) --
	(509.67, 25.25) --
	(511.21, 25.25) --
	(512.74, 25.25) --
	(514.28, 25.25) --
	(515.82, 25.25) --
	(517.36, 25.25) --
	(518.89, 25.25) --
	(520.43, 25.25) --
	(521.97, 25.25) --
	(523.50, 25.25) --
	(525.04, 25.25) --
	(526.58, 25.25) --
	(528.11, 25.25) --
	(529.65, 25.25) --
	(531.19, 25.25) --
	(532.73, 25.25) --
	(534.26, 25.25) --
	(535.80, 25.25) --
	(537.34, 25.39) --
	(538.87, 28.69) --
	(540.41, 36.38) --
	(541.95, 48.47) --
	(543.48, 64.95) --
	(545.02, 85.82);
\end{scope}
\begin{scope}
\path[clip] ( 32.73,263.47) rectangle (198.42,283.58);
\definecolor{drawColor}{gray}{0.10}

\node[text=drawColor,anchor=base,inner sep=0pt, outer sep=0pt, scale=  1.07] at (115.58,269.12) {$\mu = 0.25$};
\end{scope}
\begin{scope}
\path[clip] (209.80,263.47) rectangle (375.49,283.58);
\definecolor{drawColor}{gray}{0.10}

\node[text=drawColor,anchor=base,inner sep=0pt, outer sep=0pt, scale=  1.07] at (292.64,269.12) {$\mu = 0.5$};
\end{scope}
\begin{scope}
\path[clip] (386.87,263.47) rectangle (552.55,283.58);
\definecolor{drawColor}{gray}{0.10}

\node[text=drawColor,anchor=base,inner sep=0pt, outer sep=0pt, scale=  1.07] at (469.71,269.12) {$\mu = 0.75$};
\end{scope}
\begin{scope}
\path[clip] (552.55,186.57) rectangle (572.66,263.47);
\definecolor{drawColor}{gray}{0.10}

\node[text=drawColor,rotate=-90.00,anchor=base,inner sep=0pt, outer sep=0pt, scale=  1.07] at (558.20,225.02) {$\sigma = 0.25$};
\end{scope}
\begin{scope}
\path[clip] (552.55,104.16) rectangle (572.66,181.07);
\definecolor{drawColor}{gray}{0.10}

\node[text=drawColor,rotate=-90.00,anchor=base,inner sep=0pt, outer sep=0pt, scale=  1.07] at (558.20,142.62) {$\sigma = 1$};
\end{scope}
\begin{scope}
\path[clip] (552.55, 21.76) rectangle (572.66, 98.66);
\definecolor{drawColor}{gray}{0.10}

\node[text=drawColor,rotate=-90.00,anchor=base,inner sep=0pt, outer sep=0pt, scale=  1.07] at (558.20, 60.21) {$\sigma = 4$};
\end{scope}
\begin{scope}
\path[clip] (  0.00,  0.00) rectangle (578.16,289.08);
\definecolor{drawColor}{gray}{0.30}

\node[text=drawColor,anchor=base,inner sep=0pt, outer sep=0pt, scale=  1.07] at ( 38.73,  7.99) {0.00};

\node[text=drawColor,anchor=base,inner sep=0pt, outer sep=0pt, scale=  1.07] at ( 77.15,  7.99) {0.25};

\node[text=drawColor,anchor=base,inner sep=0pt, outer sep=0pt, scale=  1.07] at (115.58,  7.99) {0.50};

\node[text=drawColor,anchor=base,inner sep=0pt, outer sep=0pt, scale=  1.07] at (154.00,  7.99) {0.75};

\node[text=drawColor,anchor=base,inner sep=0pt, outer sep=0pt, scale=  1.07] at (192.42,  7.99) {1.00};
\end{scope}
\begin{scope}
\path[clip] (  0.00,  0.00) rectangle (578.16,289.08);
\definecolor{drawColor}{gray}{0.30}

\node[text=drawColor,anchor=base,inner sep=0pt, outer sep=0pt, scale=  1.07] at (215.79,  7.99) {0.00};

\node[text=drawColor,anchor=base,inner sep=0pt, outer sep=0pt, scale=  1.07] at (254.22,  7.99) {0.25};

\node[text=drawColor,anchor=base,inner sep=0pt, outer sep=0pt, scale=  1.07] at (292.64,  7.99) {0.50};

\node[text=drawColor,anchor=base,inner sep=0pt, outer sep=0pt, scale=  1.07] at (331.07,  7.99) {0.75};

\node[text=drawColor,anchor=base,inner sep=0pt, outer sep=0pt, scale=  1.07] at (369.49,  7.99) {1.00};
\end{scope}
\begin{scope}
\path[clip] (  0.00,  0.00) rectangle (578.16,289.08);
\definecolor{drawColor}{gray}{0.30}

\node[text=drawColor,anchor=base,inner sep=0pt, outer sep=0pt, scale=  1.07] at (392.86,  7.99) {0.00};

\node[text=drawColor,anchor=base,inner sep=0pt, outer sep=0pt, scale=  1.07] at (431.29,  7.99) {0.25};

\node[text=drawColor,anchor=base,inner sep=0pt, outer sep=0pt, scale=  1.07] at (469.71,  7.99) {0.50};

\node[text=drawColor,anchor=base,inner sep=0pt, outer sep=0pt, scale=  1.07] at (508.13,  7.99) {0.75};

\node[text=drawColor,anchor=base,inner sep=0pt, outer sep=0pt, scale=  1.07] at (546.56,  7.99) {1.00};
\end{scope}
\begin{scope}
\path[clip] (  0.00,  0.00) rectangle (578.16,289.08);
\definecolor{drawColor}{gray}{0.30}

\node[text=drawColor,anchor=base east,inner sep=0pt, outer sep=0pt, scale=  1.07] at ( 27.78,185.65) {0.00};

\node[text=drawColor,anchor=base east,inner sep=0pt, outer sep=0pt, scale=  1.07] at ( 27.78,204.72) {0.25};

\node[text=drawColor,anchor=base east,inner sep=0pt, outer sep=0pt, scale=  1.07] at ( 27.78,223.79) {0.50};

\node[text=drawColor,anchor=base east,inner sep=0pt, outer sep=0pt, scale=  1.07] at ( 27.78,242.85) {0.75};
\end{scope}
\begin{scope}
\path[clip] (  0.00,  0.00) rectangle (578.16,289.08);
\definecolor{drawColor}{gray}{0.30}

\node[text=drawColor,anchor=base east,inner sep=0pt, outer sep=0pt, scale=  1.07] at ( 27.78,103.25) {0.00};

\node[text=drawColor,anchor=base east,inner sep=0pt, outer sep=0pt, scale=  1.07] at ( 27.78,122.32) {0.25};

\node[text=drawColor,anchor=base east,inner sep=0pt, outer sep=0pt, scale=  1.07] at ( 27.78,141.38) {0.50};

\node[text=drawColor,anchor=base east,inner sep=0pt, outer sep=0pt, scale=  1.07] at ( 27.78,160.45) {0.75};
\end{scope}
\begin{scope}
\path[clip] (  0.00,  0.00) rectangle (578.16,289.08);
\definecolor{drawColor}{gray}{0.30}

\node[text=drawColor,anchor=base east,inner sep=0pt, outer sep=0pt, scale=  1.07] at ( 27.78, 20.84) {0.00};

\node[text=drawColor,anchor=base east,inner sep=0pt, outer sep=0pt, scale=  1.07] at ( 27.78, 39.91) {0.25};

\node[text=drawColor,anchor=base east,inner sep=0pt, outer sep=0pt, scale=  1.07] at ( 27.78, 58.98) {0.50};

\node[text=drawColor,anchor=base east,inner sep=0pt, outer sep=0pt, scale=  1.07] at ( 27.78, 78.04) {0.75};
\end{scope}
\end{tikzpicture}

      }
      \caption{Basis functions for different location and scale values $\mu$ and $\sigma$ (using $\tau=1$ and $c = 0$)}\label{knots_mu_sigma}

    \end{subfigure}

    \begin{subfigure}[b]{\textwidth}
      \centering
      \resizebox{\textwidth}{!}{
        {% !TEX encoding = UTF-8 Unicode
\begin{tikzpicture}[x=1pt,y=1pt]
\definecolor{fillColor}{RGB}{255,255,255}
\path[use as bounding box,fill=fillColor,fill opacity=0.00] (0,0) rectangle (578.16,289.08);
\begin{scope}
\path[clip] ( 32.73,186.57) rectangle (198.42,263.47);
\definecolor{drawColor}{gray}{0.92}

\path[draw=drawColor,line width= 0.3pt,line join=round] ( 32.73,200.10) --
	(198.42,200.10);

\path[draw=drawColor,line width= 0.3pt,line join=round] ( 32.73,220.17) --
	(198.42,220.17);

\path[draw=drawColor,line width= 0.3pt,line join=round] ( 32.73,240.25) --
	(198.42,240.25);

\path[draw=drawColor,line width= 0.3pt,line join=round] ( 32.73,260.32) --
	(198.42,260.32);

\path[draw=drawColor,line width= 0.3pt,line join=round] ( 57.94,186.57) --
	( 57.94,263.47);

\path[draw=drawColor,line width= 0.3pt,line join=round] ( 96.36,186.57) --
	( 96.36,263.47);

\path[draw=drawColor,line width= 0.3pt,line join=round] (134.79,186.57) --
	(134.79,263.47);

\path[draw=drawColor,line width= 0.3pt,line join=round] (173.21,186.57) --
	(173.21,263.47);

\path[draw=drawColor,line width= 0.6pt,line join=round] ( 32.73,190.06) --
	(198.42,190.06);

\path[draw=drawColor,line width= 0.6pt,line join=round] ( 32.73,210.14) --
	(198.42,210.14);

\path[draw=drawColor,line width= 0.6pt,line join=round] ( 32.73,230.21) --
	(198.42,230.21);

\path[draw=drawColor,line width= 0.6pt,line join=round] ( 32.73,250.28) --
	(198.42,250.28);

\path[draw=drawColor,line width= 0.6pt,line join=round] ( 38.73,186.57) --
	( 38.73,263.47);

\path[draw=drawColor,line width= 0.6pt,line join=round] ( 77.15,186.57) --
	( 77.15,263.47);

\path[draw=drawColor,line width= 0.6pt,line join=round] (115.58,186.57) --
	(115.58,263.47);

\path[draw=drawColor,line width= 0.6pt,line join=round] (154.00,186.57) --
	(154.00,263.47);

\path[draw=drawColor,line width= 0.6pt,line join=round] (192.42,186.57) --
	(192.42,263.47);
\definecolor{drawColor}{RGB}{155,38,176}

\path[draw=drawColor,line width= 1.1pt,line join=round] ( 40.26,251.15) --
	( 41.80,248.08) --
	( 43.34,245.09) --
	( 44.88,242.18) --
	( 46.41,239.34) --
	( 47.95,236.59) --
	( 49.49,233.92) --
	( 51.02,231.32) --
	( 52.56,228.80) --
	( 54.10,226.37) --
	( 55.63,224.01) --
	( 57.17,221.73) --
	( 58.71,219.53) --
	( 60.24,217.41) --
	( 61.78,215.37) --
	( 63.32,213.41) --
	( 64.86,211.53) --
	( 66.39,209.72) --
	( 67.93,208.00) --
	( 69.47,206.35) --
	( 71.00,204.79) --
	( 72.54,203.30) --
	( 74.08,201.89) --
	( 75.61,200.56) --
	( 77.15,199.31) --
	( 78.69,198.14) --
	( 80.23,197.05) --
	( 81.76,196.04) --
	( 83.30,195.11) --
	( 84.84,194.25) --
	( 86.37,193.48) --
	( 87.91,192.78) --
	( 89.45,192.17) --
	( 90.98,191.63) --
	( 92.52,191.17) --
	( 94.06,190.79) --
	( 95.60,190.49) --
	( 97.13,190.27) --
	( 98.67,190.13) --
	(100.21,190.07) --
	(101.74,190.06) --
	(103.28,190.06) --
	(104.82,190.06) --
	(106.35,190.06) --
	(107.89,190.06) --
	(109.43,190.06) --
	(110.96,190.06) --
	(112.50,190.06) --
	(114.04,190.06) --
	(115.58,190.06) --
	(117.11,190.06) --
	(118.65,190.06) --
	(120.19,190.06) --
	(121.72,190.06) --
	(123.26,190.06) --
	(124.80,190.06) --
	(126.33,190.06) --
	(127.87,190.06) --
	(129.41,190.06) --
	(130.95,190.06) --
	(132.48,190.06) --
	(134.02,190.06) --
	(135.56,190.06) --
	(137.09,190.06) --
	(138.63,190.06) --
	(140.17,190.06) --
	(141.70,190.06) --
	(143.24,190.06) --
	(144.78,190.06) --
	(146.32,190.06) --
	(147.85,190.06) --
	(149.39,190.06) --
	(150.93,190.06) --
	(152.46,190.06) --
	(154.00,190.06) --
	(155.54,190.06) --
	(157.07,190.06) --
	(158.61,190.06) --
	(160.15,190.06) --
	(161.69,190.06) --
	(163.22,190.06) --
	(164.76,190.06) --
	(166.30,190.06) --
	(167.83,190.06) --
	(169.37,190.06) --
	(170.91,190.06) --
	(172.44,190.06) --
	(173.98,190.06) --
	(175.52,190.06) --
	(177.05,190.06) --
	(178.59,190.06) --
	(180.13,190.06) --
	(181.67,190.06) --
	(183.20,190.06) --
	(184.74,190.06) --
	(186.28,190.06) --
	(187.81,190.06) --
	(189.35,190.06) --
	(190.89,190.06);
\definecolor{drawColor}{RGB}{63,81,180}

\path[draw=drawColor,line width= 1.1pt,line join=round] ( 40.26,209.24) --
	( 41.80,212.22) --
	( 43.34,215.06) --
	( 44.88,217.76) --
	( 46.41,220.32) --
	( 47.95,222.74) --
	( 49.49,225.02) --
	( 51.02,227.16) --
	( 52.56,229.16) --
	( 54.10,231.02) --
	( 55.63,232.74) --
	( 57.17,234.32) --
	( 58.71,235.77) --
	( 60.24,237.07) --
	( 61.78,238.23) --
	( 63.32,239.25) --
	( 64.86,240.14) --
	( 66.39,240.88) --
	( 67.93,241.48) --
	( 69.47,241.95) --
	( 71.00,242.27) --
	( 72.54,242.45) --
	( 74.08,242.50) --
	( 75.61,242.40) --
	( 77.15,242.17) --
	( 78.69,241.79) --
	( 80.23,241.28) --
	( 81.76,240.62) --
	( 83.30,239.83) --
	( 84.84,238.90) --
	( 86.37,237.82) --
	( 87.91,236.61) --
	( 89.45,235.26) --
	( 90.98,233.76) --
	( 92.52,232.13) --
	( 94.06,230.36) --
	( 95.60,228.45) --
	( 97.13,226.40) --
	( 98.67,224.20) --
	(100.21,221.87) --
	(101.74,219.46) --
	(103.28,217.14) --
	(104.82,214.91) --
	(106.35,212.77) --
	(107.89,210.73) --
	(109.43,208.79) --
	(110.96,206.94) --
	(112.50,205.19) --
	(114.04,203.54) --
	(115.58,201.98) --
	(117.11,200.52) --
	(118.65,199.15) --
	(120.19,197.88) --
	(121.72,196.70) --
	(123.26,195.62) --
	(124.80,194.64) --
	(126.33,193.75) --
	(127.87,192.96) --
	(129.41,192.26) --
	(130.95,191.66) --
	(132.48,191.15) --
	(134.02,190.74) --
	(135.56,190.43) --
	(137.09,190.21) --
	(138.63,190.09) --
	(140.17,190.06) --
	(141.70,190.06) --
	(143.24,190.06) --
	(144.78,190.06) --
	(146.32,190.06) --
	(147.85,190.06) --
	(149.39,190.06) --
	(150.93,190.06) --
	(152.46,190.06) --
	(154.00,190.06) --
	(155.54,190.06) --
	(157.07,190.06) --
	(158.61,190.06) --
	(160.15,190.06) --
	(161.69,190.06) --
	(163.22,190.06) --
	(164.76,190.06) --
	(166.30,190.06) --
	(167.83,190.06) --
	(169.37,190.06) --
	(170.91,190.06) --
	(172.44,190.06) --
	(173.98,190.06) --
	(175.52,190.06) --
	(177.05,190.06) --
	(178.59,190.06) --
	(180.13,190.06) --
	(181.67,190.06) --
	(183.20,190.06) --
	(184.74,190.06) --
	(186.28,190.06) --
	(187.81,190.06) --
	(189.35,190.06) --
	(190.89,190.06);
\definecolor{drawColor}{RGB}{2,169,243}

\path[draw=drawColor,line width= 1.1pt,line join=round] ( 40.26,190.09) --
	( 41.80,190.18) --
	( 43.34,190.34) --
	( 44.88,190.55) --
	( 46.41,190.82) --
	( 47.95,191.15) --
	( 49.49,191.55) --
	( 51.02,192.00) --
	( 52.56,192.52) --
	( 54.10,193.09) --
	( 55.63,193.73) --
	( 57.17,194.43) --
	( 58.71,195.18) --
	( 60.24,196.00) --
	( 61.78,196.88) --
	( 63.32,197.82) --
	( 64.86,198.82) --
	( 66.39,199.88) --
	( 67.93,201.00) --
	( 69.47,202.18) --
	( 71.00,203.43) --
	( 72.54,204.73) --
	( 74.08,206.09) --
	( 75.61,207.52) --
	( 77.15,209.00) --
	( 78.69,210.55) --
	( 80.23,212.15) --
	( 81.76,213.82) --
	( 83.30,215.55) --
	( 84.84,217.33) --
	( 86.37,219.18) --
	( 87.91,221.09) --
	( 89.45,223.06) --
	( 90.98,225.09) --
	( 92.52,227.18) --
	( 94.06,229.33) --
	( 95.60,231.54) --
	( 97.13,233.82) --
	( 98.67,236.15) --
	(100.21,238.54) --
	(101.74,240.92) --
	(103.28,243.06) --
	(104.82,244.95) --
	(106.35,246.60) --
	(107.89,247.99) --
	(109.43,249.14) --
	(110.96,250.04) --
	(112.50,250.69) --
	(114.04,251.09) --
	(115.58,251.25) --
	(117.11,251.15) --
	(118.65,250.81) --
	(120.19,250.22) --
	(121.72,249.38) --
	(123.26,248.29) --
	(124.80,246.96) --
	(126.33,245.37) --
	(127.87,243.54) --
	(129.41,241.46) --
	(130.95,239.13) --
	(132.48,236.55) --
	(134.02,233.73) --
	(135.56,230.65) --
	(137.09,227.33) --
	(138.63,223.76) --
	(140.17,219.95) --
	(141.70,216.25) --
	(143.24,212.78) --
	(144.78,209.57) --
	(146.32,206.60) --
	(147.85,203.87) --
	(149.39,201.39) --
	(150.93,199.15) --
	(152.46,197.16) --
	(154.00,195.42) --
	(155.54,193.92) --
	(157.07,192.67) --
	(158.61,191.66) --
	(160.15,190.90) --
	(161.69,190.38) --
	(163.22,190.11) --
	(164.76,190.06) --
	(166.30,190.06) --
	(167.83,190.06) --
	(169.37,190.06) --
	(170.91,190.06) --
	(172.44,190.06) --
	(173.98,190.06) --
	(175.52,190.06) --
	(177.05,190.06) --
	(178.59,190.06) --
	(180.13,190.06) --
	(181.67,190.06) --
	(183.20,190.06) --
	(184.74,190.06) --
	(186.28,190.06) --
	(187.81,190.06) --
	(189.35,190.06) --
	(190.89,190.06);
\definecolor{drawColor}{RGB}{0,150,135}

\path[draw=drawColor,line width= 1.1pt,line join=round] ( 40.26,190.06) --
	( 41.80,190.06) --
	( 43.34,190.06) --
	( 44.88,190.06) --
	( 46.41,190.06) --
	( 47.95,190.06) --
	( 49.49,190.06) --
	( 51.02,190.06) --
	( 52.56,190.06) --
	( 54.10,190.06) --
	( 55.63,190.06) --
	( 57.17,190.06) --
	( 58.71,190.06) --
	( 60.24,190.06) --
	( 61.78,190.06) --
	( 63.32,190.06) --
	( 64.86,190.06) --
	( 66.39,190.06) --
	( 67.93,190.06) --
	( 69.47,190.06) --
	( 71.00,190.06) --
	( 72.54,190.06) --
	( 74.08,190.06) --
	( 75.61,190.06) --
	( 77.15,190.06) --
	( 78.69,190.06) --
	( 80.23,190.06) --
	( 81.76,190.06) --
	( 83.30,190.06) --
	( 84.84,190.06) --
	( 86.37,190.06) --
	( 87.91,190.06) --
	( 89.45,190.06) --
	( 90.98,190.06) --
	( 92.52,190.06) --
	( 94.06,190.06) --
	( 95.60,190.06) --
	( 97.13,190.06) --
	( 98.67,190.06) --
	(100.21,190.06) --
	(101.74,190.10) --
	(103.28,190.29) --
	(104.82,190.62) --
	(106.35,191.11) --
	(107.89,191.76) --
	(109.43,192.55) --
	(110.96,193.50) --
	(112.50,194.60) --
	(114.04,195.85) --
	(115.58,197.25) --
	(117.11,198.81) --
	(118.65,200.52) --
	(120.19,202.38) --
	(121.72,204.40) --
	(123.26,206.57) --
	(124.80,208.89) --
	(126.33,211.36) --
	(127.87,213.98) --
	(129.41,216.76) --
	(130.95,219.69) --
	(132.48,222.77) --
	(134.02,226.01) --
	(135.56,229.40) --
	(137.09,232.94) --
	(138.63,236.63) --
	(140.17,240.45) --
	(141.70,243.87) --
	(143.24,246.65) --
	(144.78,248.78) --
	(146.32,250.26) --
	(147.85,251.10) --
	(149.39,251.30) --
	(150.93,250.85) --
	(152.46,249.76) --
	(154.00,248.03) --
	(155.54,245.65) --
	(157.07,242.63) --
	(158.61,238.96) --
	(160.15,234.65) --
	(161.69,229.69) --
	(163.22,224.09) --
	(164.76,217.95) --
	(166.30,212.22) --
	(167.83,207.16) --
	(169.37,202.75) --
	(170.91,198.99) --
	(172.44,195.90) --
	(173.98,193.46) --
	(175.52,191.67) --
	(177.05,190.55) --
	(178.59,190.08) --
	(180.13,190.06) --
	(181.67,190.06) --
	(183.20,190.06) --
	(184.74,190.06) --
	(186.28,190.06) --
	(187.81,190.06) --
	(189.35,190.06) --
	(190.89,190.06);
\definecolor{drawColor}{RGB}{139,195,74}

\path[draw=drawColor,line width= 1.1pt,line join=round] ( 40.26,190.06) --
	( 41.80,190.06) --
	( 43.34,190.06) --
	( 44.88,190.06) --
	( 46.41,190.06) --
	( 47.95,190.06) --
	( 49.49,190.06) --
	( 51.02,190.06) --
	( 52.56,190.06) --
	( 54.10,190.06) --
	( 55.63,190.06) --
	( 57.17,190.06) --
	( 58.71,190.06) --
	( 60.24,190.06) --
	( 61.78,190.06) --
	( 63.32,190.06) --
	( 64.86,190.06) --
	( 66.39,190.06) --
	( 67.93,190.06) --
	( 69.47,190.06) --
	( 71.00,190.06) --
	( 72.54,190.06) --
	( 74.08,190.06) --
	( 75.61,190.06) --
	( 77.15,190.06) --
	( 78.69,190.06) --
	( 80.23,190.06) --
	( 81.76,190.06) --
	( 83.30,190.06) --
	( 84.84,190.06) --
	( 86.37,190.06) --
	( 87.91,190.06) --
	( 89.45,190.06) --
	( 90.98,190.06) --
	( 92.52,190.06) --
	( 94.06,190.06) --
	( 95.60,190.06) --
	( 97.13,190.06) --
	( 98.67,190.06) --
	(100.21,190.06) --
	(101.74,190.06) --
	(103.28,190.06) --
	(104.82,190.06) --
	(106.35,190.06) --
	(107.89,190.06) --
	(109.43,190.06) --
	(110.96,190.06) --
	(112.50,190.06) --
	(114.04,190.06) --
	(115.58,190.06) --
	(117.11,190.06) --
	(118.65,190.06) --
	(120.19,190.06) --
	(121.72,190.06) --
	(123.26,190.06) --
	(124.80,190.06) --
	(126.33,190.06) --
	(127.87,190.06) --
	(129.41,190.06) --
	(130.95,190.06) --
	(132.48,190.06) --
	(134.02,190.06) --
	(135.56,190.06) --
	(137.09,190.06) --
	(138.63,190.06) --
	(140.17,190.07) --
	(141.70,190.36) --
	(143.24,191.05) --
	(144.78,192.14) --
	(146.32,193.63) --
	(147.85,195.51) --
	(149.39,197.79) --
	(150.93,200.48) --
	(152.46,203.56) --
	(154.00,207.04) --
	(155.54,210.91) --
	(157.07,215.19) --
	(158.61,219.87) --
	(160.15,224.94) --
	(161.69,230.41) --
	(163.22,236.28) --
	(164.76,242.39) --
	(166.30,247.13) --
	(167.83,250.12) --
	(169.37,251.36) --
	(170.91,250.84) --
	(172.44,248.58) --
	(173.98,244.55) --
	(175.52,238.78) --
	(177.05,231.25) --
	(178.59,221.97) --
	(180.13,212.05) --
	(181.67,203.92) --
	(183.20,197.66) --
	(184.74,193.27) --
	(186.28,190.74) --
	(187.81,190.06) --
	(189.35,190.06) --
	(190.89,190.06);
\definecolor{drawColor}{RGB}{255,235,58}

\path[draw=drawColor,line width= 1.1pt,line join=round] ( 40.26,190.06) --
	( 41.80,190.06) --
	( 43.34,190.06) --
	( 44.88,190.06) --
	( 46.41,190.06) --
	( 47.95,190.06) --
	( 49.49,190.06) --
	( 51.02,190.06) --
	( 52.56,190.06) --
	( 54.10,190.06) --
	( 55.63,190.06) --
	( 57.17,190.06) --
	( 58.71,190.06) --
	( 60.24,190.06) --
	( 61.78,190.06) --
	( 63.32,190.06) --
	( 64.86,190.06) --
	( 66.39,190.06) --
	( 67.93,190.06) --
	( 69.47,190.06) --
	( 71.00,190.06) --
	( 72.54,190.06) --
	( 74.08,190.06) --
	( 75.61,190.06) --
	( 77.15,190.06) --
	( 78.69,190.06) --
	( 80.23,190.06) --
	( 81.76,190.06) --
	( 83.30,190.06) --
	( 84.84,190.06) --
	( 86.37,190.06) --
	( 87.91,190.06) --
	( 89.45,190.06) --
	( 90.98,190.06) --
	( 92.52,190.06) --
	( 94.06,190.06) --
	( 95.60,190.06) --
	( 97.13,190.06) --
	( 98.67,190.06) --
	(100.21,190.06) --
	(101.74,190.06) --
	(103.28,190.06) --
	(104.82,190.06) --
	(106.35,190.06) --
	(107.89,190.06) --
	(109.43,190.06) --
	(110.96,190.06) --
	(112.50,190.06) --
	(114.04,190.06) --
	(115.58,190.06) --
	(117.11,190.06) --
	(118.65,190.06) --
	(120.19,190.06) --
	(121.72,190.06) --
	(123.26,190.06) --
	(124.80,190.06) --
	(126.33,190.06) --
	(127.87,190.06) --
	(129.41,190.06) --
	(130.95,190.06) --
	(132.48,190.06) --
	(134.02,190.06) --
	(135.56,190.06) --
	(137.09,190.06) --
	(138.63,190.06) --
	(140.17,190.06) --
	(141.70,190.06) --
	(143.24,190.06) --
	(144.78,190.06) --
	(146.32,190.06) --
	(147.85,190.06) --
	(149.39,190.06) --
	(150.93,190.06) --
	(152.46,190.06) --
	(154.00,190.06) --
	(155.54,190.06) --
	(157.07,190.06) --
	(158.61,190.06) --
	(160.15,190.06) --
	(161.69,190.06) --
	(163.22,190.06) --
	(164.76,190.15) --
	(166.30,191.13) --
	(167.83,193.20) --
	(169.37,196.38) --
	(170.91,200.65) --
	(172.44,206.01) --
	(173.98,212.47) --
	(175.52,220.03) --
	(177.05,228.69) --
	(178.59,238.44) --
	(180.13,247.36) --
	(181.67,251.32) --
	(183.20,250.17) --
	(184.74,243.92) --
	(186.28,232.55) --
	(187.81,216.19) --
	(189.35,201.67) --
	(190.89,192.97);
\definecolor{drawColor}{RGB}{255,152,0}

\path[draw=drawColor,line width= 1.1pt,line join=round] ( 40.26,190.06) --
	( 41.80,190.06) --
	( 43.34,190.06) --
	( 44.88,190.06) --
	( 46.41,190.06) --
	( 47.95,190.06) --
	( 49.49,190.06) --
	( 51.02,190.06) --
	( 52.56,190.06) --
	( 54.10,190.06) --
	( 55.63,190.06) --
	( 57.17,190.06) --
	( 58.71,190.06) --
	( 60.24,190.06) --
	( 61.78,190.06) --
	( 63.32,190.06) --
	( 64.86,190.06) --
	( 66.39,190.06) --
	( 67.93,190.06) --
	( 69.47,190.06) --
	( 71.00,190.06) --
	( 72.54,190.06) --
	( 74.08,190.06) --
	( 75.61,190.06) --
	( 77.15,190.06) --
	( 78.69,190.06) --
	( 80.23,190.06) --
	( 81.76,190.06) --
	( 83.30,190.06) --
	( 84.84,190.06) --
	( 86.37,190.06) --
	( 87.91,190.06) --
	( 89.45,190.06) --
	( 90.98,190.06) --
	( 92.52,190.06) --
	( 94.06,190.06) --
	( 95.60,190.06) --
	( 97.13,190.06) --
	( 98.67,190.06) --
	(100.21,190.06) --
	(101.74,190.06) --
	(103.28,190.06) --
	(104.82,190.06) --
	(106.35,190.06) --
	(107.89,190.06) --
	(109.43,190.06) --
	(110.96,190.06) --
	(112.50,190.06) --
	(114.04,190.06) --
	(115.58,190.06) --
	(117.11,190.06) --
	(118.65,190.06) --
	(120.19,190.06) --
	(121.72,190.06) --
	(123.26,190.06) --
	(124.80,190.06) --
	(126.33,190.06) --
	(127.87,190.06) --
	(129.41,190.06) --
	(130.95,190.06) --
	(132.48,190.06) --
	(134.02,190.06) --
	(135.56,190.06) --
	(137.09,190.06) --
	(138.63,190.06) --
	(140.17,190.06) --
	(141.70,190.06) --
	(143.24,190.06) --
	(144.78,190.06) --
	(146.32,190.06) --
	(147.85,190.06) --
	(149.39,190.06) --
	(150.93,190.06) --
	(152.46,190.06) --
	(154.00,190.06) --
	(155.54,190.06) --
	(157.07,190.06) --
	(158.61,190.06) --
	(160.15,190.06) --
	(161.69,190.06) --
	(163.22,190.06) --
	(164.76,190.06) --
	(166.30,190.06) --
	(167.83,190.06) --
	(169.37,190.06) --
	(170.91,190.06) --
	(172.44,190.06) --
	(173.98,190.06) --
	(175.52,190.06) --
	(177.05,190.06) --
	(178.59,190.06) --
	(180.13,191.07) --
	(181.67,195.24) --
	(183.20,202.65) --
	(184.74,213.30) --
	(186.28,227.19) --
	(187.81,244.09) --
	(189.35,250.20) --
	(190.89,237.55);
\definecolor{drawColor}{RGB}{121,84,71}

\path[draw=drawColor,line width= 1.1pt,line join=round] ( 40.26,190.06) --
	( 41.80,190.06) --
	( 43.34,190.06) --
	( 44.88,190.06) --
	( 46.41,190.06) --
	( 47.95,190.06) --
	( 49.49,190.06) --
	( 51.02,190.06) --
	( 52.56,190.06) --
	( 54.10,190.06) --
	( 55.63,190.06) --
	( 57.17,190.06) --
	( 58.71,190.06) --
	( 60.24,190.06) --
	( 61.78,190.06) --
	( 63.32,190.06) --
	( 64.86,190.06) --
	( 66.39,190.06) --
	( 67.93,190.06) --
	( 69.47,190.06) --
	( 71.00,190.06) --
	( 72.54,190.06) --
	( 74.08,190.06) --
	( 75.61,190.06) --
	( 77.15,190.06) --
	( 78.69,190.06) --
	( 80.23,190.06) --
	( 81.76,190.06) --
	( 83.30,190.06) --
	( 84.84,190.06) --
	( 86.37,190.06) --
	( 87.91,190.06) --
	( 89.45,190.06) --
	( 90.98,190.06) --
	( 92.52,190.06) --
	( 94.06,190.06) --
	( 95.60,190.06) --
	( 97.13,190.06) --
	( 98.67,190.06) --
	(100.21,190.06) --
	(101.74,190.06) --
	(103.28,190.06) --
	(104.82,190.06) --
	(106.35,190.06) --
	(107.89,190.06) --
	(109.43,190.06) --
	(110.96,190.06) --
	(112.50,190.06) --
	(114.04,190.06) --
	(115.58,190.06) --
	(117.11,190.06) --
	(118.65,190.06) --
	(120.19,190.06) --
	(121.72,190.06) --
	(123.26,190.06) --
	(124.80,190.06) --
	(126.33,190.06) --
	(127.87,190.06) --
	(129.41,190.06) --
	(130.95,190.06) --
	(132.48,190.06) --
	(134.02,190.06) --
	(135.56,190.06) --
	(137.09,190.06) --
	(138.63,190.06) --
	(140.17,190.06) --
	(141.70,190.06) --
	(143.24,190.06) --
	(144.78,190.06) --
	(146.32,190.06) --
	(147.85,190.06) --
	(149.39,190.06) --
	(150.93,190.06) --
	(152.46,190.06) --
	(154.00,190.06) --
	(155.54,190.06) --
	(157.07,190.06) --
	(158.61,190.06) --
	(160.15,190.06) --
	(161.69,190.06) --
	(163.22,190.06) --
	(164.76,190.06) --
	(166.30,190.06) --
	(167.83,190.06) --
	(169.37,190.06) --
	(170.91,190.06) --
	(172.44,190.06) --
	(173.98,190.06) --
	(175.52,190.06) --
	(177.05,190.06) --
	(178.59,190.06) --
	(180.13,190.06) --
	(181.67,190.06) --
	(183.20,190.06) --
	(184.74,190.06) --
	(186.28,190.06) --
	(187.81,190.21) --
	(189.35,198.60) --
	(190.89,219.97);
\end{scope}
\begin{scope}
\path[clip] ( 32.73,104.16) rectangle (198.42,181.07);
\definecolor{drawColor}{gray}{0.92}

\path[draw=drawColor,line width= 0.3pt,line join=round] ( 32.73,117.69) --
	(198.42,117.69);

\path[draw=drawColor,line width= 0.3pt,line join=round] ( 32.73,137.77) --
	(198.42,137.77);

\path[draw=drawColor,line width= 0.3pt,line join=round] ( 32.73,157.84) --
	(198.42,157.84);

\path[draw=drawColor,line width= 0.3pt,line join=round] ( 32.73,177.91) --
	(198.42,177.91);

\path[draw=drawColor,line width= 0.3pt,line join=round] ( 57.94,104.16) --
	( 57.94,181.07);

\path[draw=drawColor,line width= 0.3pt,line join=round] ( 96.36,104.16) --
	( 96.36,181.07);

\path[draw=drawColor,line width= 0.3pt,line join=round] (134.79,104.16) --
	(134.79,181.07);

\path[draw=drawColor,line width= 0.3pt,line join=round] (173.21,104.16) --
	(173.21,181.07);

\path[draw=drawColor,line width= 0.6pt,line join=round] ( 32.73,107.66) --
	(198.42,107.66);

\path[draw=drawColor,line width= 0.6pt,line join=round] ( 32.73,127.73) --
	(198.42,127.73);

\path[draw=drawColor,line width= 0.6pt,line join=round] ( 32.73,147.80) --
	(198.42,147.80);

\path[draw=drawColor,line width= 0.6pt,line join=round] ( 32.73,167.88) --
	(198.42,167.88);

\path[draw=drawColor,line width= 0.6pt,line join=round] ( 38.73,104.16) --
	( 38.73,181.07);

\path[draw=drawColor,line width= 0.6pt,line join=round] ( 77.15,104.16) --
	( 77.15,181.07);

\path[draw=drawColor,line width= 0.6pt,line join=round] (115.58,104.16) --
	(115.58,181.07);

\path[draw=drawColor,line width= 0.6pt,line join=round] (154.00,104.16) --
	(154.00,181.07);

\path[draw=drawColor,line width= 0.6pt,line join=round] (192.42,104.16) --
	(192.42,181.07);
\definecolor{drawColor}{RGB}{155,38,176}

\path[draw=drawColor,line width= 1.1pt,line join=round] ( 40.26,145.84) --
	( 41.80,143.92) --
	( 43.34,142.05) --
	( 44.88,140.23) --
	( 46.41,138.46) --
	( 47.95,136.74) --
	( 49.49,135.07) --
	( 51.02,133.44) --
	( 52.56,131.87) --
	( 54.10,130.35) --
	( 55.63,128.88) --
	( 57.17,127.45) --
	( 58.71,126.08) --
	( 60.24,124.75) --
	( 61.78,123.48) --
	( 63.32,122.25) --
	( 64.86,121.07) --
	( 66.39,119.95) --
	( 67.93,118.87) --
	( 69.47,117.84) --
	( 71.00,116.86) --
	( 72.54,115.93) --
	( 74.08,115.05) --
	( 75.61,114.22) --
	( 77.15,113.44) --
	( 78.69,112.71) --
	( 80.23,112.03) --
	( 81.76,111.39) --
	( 83.30,110.81) --
	( 84.84,110.28) --
	( 86.37,109.79) --
	( 87.91,109.36) --
	( 89.45,108.97) --
	( 90.98,108.64) --
	( 92.52,108.35) --
	( 94.06,108.11) --
	( 95.60,107.93) --
	( 97.13,107.79) --
	( 98.67,107.70) --
	(100.21,107.66) --
	(101.74,107.66) --
	(103.28,107.66) --
	(104.82,107.66) --
	(106.35,107.66) --
	(107.89,107.66) --
	(109.43,107.66) --
	(110.96,107.66) --
	(112.50,107.66) --
	(114.04,107.66) --
	(115.58,107.66) --
	(117.11,107.66) --
	(118.65,107.66) --
	(120.19,107.66) --
	(121.72,107.66) --
	(123.26,107.66) --
	(124.80,107.66) --
	(126.33,107.66) --
	(127.87,107.66) --
	(129.41,107.66) --
	(130.95,107.66) --
	(132.48,107.66) --
	(134.02,107.66) --
	(135.56,107.66) --
	(137.09,107.66) --
	(138.63,107.66) --
	(140.17,107.66) --
	(141.70,107.66) --
	(143.24,107.66) --
	(144.78,107.66) --
	(146.32,107.66) --
	(147.85,107.66) --
	(149.39,107.66) --
	(150.93,107.66) --
	(152.46,107.66) --
	(154.00,107.66) --
	(155.54,107.66) --
	(157.07,107.66) --
	(158.61,107.66) --
	(160.15,107.66) --
	(161.69,107.66) --
	(163.22,107.66) --
	(164.76,107.66) --
	(166.30,107.66) --
	(167.83,107.66) --
	(169.37,107.66) --
	(170.91,107.66) --
	(172.44,107.66) --
	(173.98,107.66) --
	(175.52,107.66) --
	(177.05,107.66) --
	(178.59,107.66) --
	(180.13,107.66) --
	(181.67,107.66) --
	(183.20,107.66) --
	(184.74,107.66) --
	(186.28,107.66) --
	(187.81,107.66) --
	(189.35,107.66) --
	(190.89,107.66);
\definecolor{drawColor}{RGB}{63,81,180}

\path[draw=drawColor,line width= 1.1pt,line join=round] ( 40.26,149.74) --
	( 41.80,151.57) --
	( 43.34,153.29) --
	( 44.88,154.90) --
	( 46.41,156.39) --
	( 47.95,157.78) --
	( 49.49,159.06) --
	( 51.02,160.23) --
	( 52.56,161.28) --
	( 54.10,162.23) --
	( 55.63,163.07) --
	( 57.17,163.80) --
	( 58.71,164.41) --
	( 60.24,164.92) --
	( 61.78,165.32) --
	( 63.32,165.60) --
	( 64.86,165.78) --
	( 66.39,165.85) --
	( 67.93,165.80) --
	( 69.47,165.65) --
	( 71.00,165.39) --
	( 72.54,165.01) --
	( 74.08,164.53) --
	( 75.61,163.94) --
	( 77.15,163.23) --
	( 78.69,162.42) --
	( 80.23,161.49) --
	( 81.76,160.46) --
	( 83.30,159.32) --
	( 84.84,158.06) --
	( 86.37,156.70) --
	( 87.91,155.22) --
	( 89.45,153.64) --
	( 90.98,151.95) --
	( 92.52,150.14) --
	( 94.06,148.23) --
	( 95.60,146.20) --
	( 97.13,144.07) --
	( 98.67,141.82) --
	(100.21,139.47) --
	(101.74,137.06) --
	(103.28,134.73) --
	(104.82,132.50) --
	(106.35,130.37) --
	(107.89,128.33) --
	(109.43,126.39) --
	(110.96,124.54) --
	(112.50,122.79) --
	(114.04,121.13) --
	(115.58,119.58) --
	(117.11,118.11) --
	(118.65,116.74) --
	(120.19,115.47) --
	(121.72,114.30) --
	(123.26,113.22) --
	(124.80,112.23) --
	(126.33,111.34) --
	(127.87,110.55) --
	(129.41,109.86) --
	(130.95,109.25) --
	(132.48,108.75) --
	(134.02,108.34) --
	(135.56,108.03) --
	(137.09,107.81) --
	(138.63,107.69) --
	(140.17,107.66) --
	(141.70,107.66) --
	(143.24,107.66) --
	(144.78,107.66) --
	(146.32,107.66) --
	(147.85,107.66) --
	(149.39,107.66) --
	(150.93,107.66) --
	(152.46,107.66) --
	(154.00,107.66) --
	(155.54,107.66) --
	(157.07,107.66) --
	(158.61,107.66) --
	(160.15,107.66) --
	(161.69,107.66) --
	(163.22,107.66) --
	(164.76,107.66) --
	(166.30,107.66) --
	(167.83,107.66) --
	(169.37,107.66) --
	(170.91,107.66) --
	(172.44,107.66) --
	(173.98,107.66) --
	(175.52,107.66) --
	(177.05,107.66) --
	(178.59,107.66) --
	(180.13,107.66) --
	(181.67,107.66) --
	(183.20,107.66) --
	(184.74,107.66) --
	(186.28,107.66) --
	(187.81,107.66) --
	(189.35,107.66) --
	(190.89,107.66);
\definecolor{drawColor}{RGB}{2,169,243}

\path[draw=drawColor,line width= 1.1pt,line join=round] ( 40.26,107.69) --
	( 41.80,107.78) --
	( 43.34,107.93) --
	( 44.88,108.14) --
	( 46.41,108.42) --
	( 47.95,108.75) --
	( 49.49,109.14) --
	( 51.02,109.60) --
	( 52.56,110.11) --
	( 54.10,110.69) --
	( 55.63,111.32) --
	( 57.17,112.02) --
	( 58.71,112.78) --
	( 60.24,113.60) --
	( 61.78,114.48) --
	( 63.32,115.42) --
	( 64.86,116.41) --
	( 66.39,117.48) --
	( 67.93,118.60) --
	( 69.47,119.78) --
	( 71.00,121.02) --
	( 72.54,122.32) --
	( 74.08,123.69) --
	( 75.61,125.11) --
	( 77.15,126.60) --
	( 78.69,128.14) --
	( 80.23,129.75) --
	( 81.76,131.41) --
	( 83.30,133.14) --
	( 84.84,134.93) --
	( 86.37,136.78) --
	( 87.91,138.69) --
	( 89.45,140.66) --
	( 90.98,142.69) --
	( 92.52,144.78) --
	( 94.06,146.93) --
	( 95.60,149.14) --
	( 97.13,151.41) --
	( 98.67,153.75) --
	(100.21,156.14) --
	(101.74,158.52) --
	(103.28,160.66) --
	(104.82,162.55) --
	(106.35,164.19) --
	(107.89,165.59) --
	(109.43,166.74) --
	(110.96,167.64) --
	(112.50,168.29) --
	(114.04,168.69) --
	(115.58,168.84) --
	(117.11,168.75) --
	(118.65,168.41) --
	(120.19,167.82) --
	(121.72,166.98) --
	(123.26,165.89) --
	(124.80,164.55) --
	(126.33,162.97) --
	(127.87,161.14) --
	(129.41,159.06) --
	(130.95,156.73) --
	(132.48,154.15) --
	(134.02,151.32) --
	(135.56,148.25) --
	(137.09,144.93) --
	(138.63,141.36) --
	(140.17,137.55) --
	(141.70,133.84) --
	(143.24,130.38) --
	(144.78,127.16) --
	(146.32,124.19) --
	(147.85,121.46) --
	(149.39,118.98) --
	(150.93,116.75) --
	(152.46,114.76) --
	(154.00,113.01) --
	(155.54,111.52) --
	(157.07,110.26) --
	(158.61,109.25) --
	(160.15,108.49) --
	(161.69,107.97) --
	(163.22,107.70) --
	(164.76,107.66) --
	(166.30,107.66) --
	(167.83,107.66) --
	(169.37,107.66) --
	(170.91,107.66) --
	(172.44,107.66) --
	(173.98,107.66) --
	(175.52,107.66) --
	(177.05,107.66) --
	(178.59,107.66) --
	(180.13,107.66) --
	(181.67,107.66) --
	(183.20,107.66) --
	(184.74,107.66) --
	(186.28,107.66) --
	(187.81,107.66) --
	(189.35,107.66) --
	(190.89,107.66);
\definecolor{drawColor}{RGB}{0,150,135}

\path[draw=drawColor,line width= 1.1pt,line join=round] ( 40.26,107.66) --
	( 41.80,107.66) --
	( 43.34,107.66) --
	( 44.88,107.66) --
	( 46.41,107.66) --
	( 47.95,107.66) --
	( 49.49,107.66) --
	( 51.02,107.66) --
	( 52.56,107.66) --
	( 54.10,107.66) --
	( 55.63,107.66) --
	( 57.17,107.66) --
	( 58.71,107.66) --
	( 60.24,107.66) --
	( 61.78,107.66) --
	( 63.32,107.66) --
	( 64.86,107.66) --
	( 66.39,107.66) --
	( 67.93,107.66) --
	( 69.47,107.66) --
	( 71.00,107.66) --
	( 72.54,107.66) --
	( 74.08,107.66) --
	( 75.61,107.66) --
	( 77.15,107.66) --
	( 78.69,107.66) --
	( 80.23,107.66) --
	( 81.76,107.66) --
	( 83.30,107.66) --
	( 84.84,107.66) --
	( 86.37,107.66) --
	( 87.91,107.66) --
	( 89.45,107.66) --
	( 90.98,107.66) --
	( 92.52,107.66) --
	( 94.06,107.66) --
	( 95.60,107.66) --
	( 97.13,107.66) --
	( 98.67,107.66) --
	(100.21,107.66) --
	(101.74,107.70) --
	(103.28,107.88) --
	(104.82,108.22) --
	(106.35,108.71) --
	(107.89,109.35) --
	(109.43,110.15) --
	(110.96,111.09) --
	(112.50,112.19) --
	(114.04,113.44) --
	(115.58,114.85) --
	(117.11,116.41) --
	(118.65,118.12) --
	(120.19,119.98) --
	(121.72,121.99) --
	(123.26,124.16) --
	(124.80,126.48) --
	(126.33,128.95) --
	(127.87,131.58) --
	(129.41,134.36) --
	(130.95,137.29) --
	(132.48,140.37) --
	(134.02,143.60) --
	(135.56,146.99) --
	(137.09,150.53) --
	(138.63,154.22) --
	(140.17,158.05) --
	(141.70,161.47) --
	(143.24,164.24) --
	(144.78,166.37) --
	(146.32,167.86) --
	(147.85,168.70) --
	(149.39,168.90) --
	(150.93,168.45) --
	(152.46,167.36) --
	(154.00,165.62) --
	(155.54,163.24) --
	(157.07,160.22) --
	(158.61,156.55) --
	(160.15,152.24) --
	(161.69,147.29) --
	(163.22,141.69) --
	(164.76,135.54) --
	(166.30,129.82) --
	(167.83,124.75) --
	(169.37,120.34) --
	(170.91,116.59) --
	(172.44,113.49) --
	(173.98,111.05) --
	(175.52,109.27) --
	(177.05,108.14) --
	(178.59,107.67) --
	(180.13,107.66) --
	(181.67,107.66) --
	(183.20,107.66) --
	(184.74,107.66) --
	(186.28,107.66) --
	(187.81,107.66) --
	(189.35,107.66) --
	(190.89,107.66);
\definecolor{drawColor}{RGB}{139,195,74}

\path[draw=drawColor,line width= 1.1pt,line join=round] ( 40.26,107.66) --
	( 41.80,107.66) --
	( 43.34,107.66) --
	( 44.88,107.66) --
	( 46.41,107.66) --
	( 47.95,107.66) --
	( 49.49,107.66) --
	( 51.02,107.66) --
	( 52.56,107.66) --
	( 54.10,107.66) --
	( 55.63,107.66) --
	( 57.17,107.66) --
	( 58.71,107.66) --
	( 60.24,107.66) --
	( 61.78,107.66) --
	( 63.32,107.66) --
	( 64.86,107.66) --
	( 66.39,107.66) --
	( 67.93,107.66) --
	( 69.47,107.66) --
	( 71.00,107.66) --
	( 72.54,107.66) --
	( 74.08,107.66) --
	( 75.61,107.66) --
	( 77.15,107.66) --
	( 78.69,107.66) --
	( 80.23,107.66) --
	( 81.76,107.66) --
	( 83.30,107.66) --
	( 84.84,107.66) --
	( 86.37,107.66) --
	( 87.91,107.66) --
	( 89.45,107.66) --
	( 90.98,107.66) --
	( 92.52,107.66) --
	( 94.06,107.66) --
	( 95.60,107.66) --
	( 97.13,107.66) --
	( 98.67,107.66) --
	(100.21,107.66) --
	(101.74,107.66) --
	(103.28,107.66) --
	(104.82,107.66) --
	(106.35,107.66) --
	(107.89,107.66) --
	(109.43,107.66) --
	(110.96,107.66) --
	(112.50,107.66) --
	(114.04,107.66) --
	(115.58,107.66) --
	(117.11,107.66) --
	(118.65,107.66) --
	(120.19,107.66) --
	(121.72,107.66) --
	(123.26,107.66) --
	(124.80,107.66) --
	(126.33,107.66) --
	(127.87,107.66) --
	(129.41,107.66) --
	(130.95,107.66) --
	(132.48,107.66) --
	(134.02,107.66) --
	(135.56,107.66) --
	(137.09,107.66) --
	(138.63,107.66) --
	(140.17,107.67) --
	(141.70,107.96) --
	(143.24,108.65) --
	(144.78,109.73) --
	(146.32,111.22) --
	(147.85,113.11) --
	(149.39,115.39) --
	(150.93,118.07) --
	(152.46,121.15) --
	(154.00,124.63) --
	(155.54,128.51) --
	(157.07,132.79) --
	(158.61,137.46) --
	(160.15,142.53) --
	(161.69,148.01) --
	(163.22,153.88) --
	(164.76,159.98) --
	(166.30,164.73) --
	(167.83,167.72) --
	(169.37,168.95) --
	(170.91,168.44) --
	(172.44,166.17) --
	(173.98,162.15) --
	(175.52,156.37) --
	(177.05,148.84) --
	(178.59,139.56) --
	(180.13,129.65) --
	(181.67,121.52) --
	(183.20,115.26) --
	(184.74,110.86) --
	(186.28,108.34) --
	(187.81,107.66) --
	(189.35,107.66) --
	(190.89,107.66);
\definecolor{drawColor}{RGB}{255,235,58}

\path[draw=drawColor,line width= 1.1pt,line join=round] ( 40.26,107.66) --
	( 41.80,107.66) --
	( 43.34,107.66) --
	( 44.88,107.66) --
	( 46.41,107.66) --
	( 47.95,107.66) --
	( 49.49,107.66) --
	( 51.02,107.66) --
	( 52.56,107.66) --
	( 54.10,107.66) --
	( 55.63,107.66) --
	( 57.17,107.66) --
	( 58.71,107.66) --
	( 60.24,107.66) --
	( 61.78,107.66) --
	( 63.32,107.66) --
	( 64.86,107.66) --
	( 66.39,107.66) --
	( 67.93,107.66) --
	( 69.47,107.66) --
	( 71.00,107.66) --
	( 72.54,107.66) --
	( 74.08,107.66) --
	( 75.61,107.66) --
	( 77.15,107.66) --
	( 78.69,107.66) --
	( 80.23,107.66) --
	( 81.76,107.66) --
	( 83.30,107.66) --
	( 84.84,107.66) --
	( 86.37,107.66) --
	( 87.91,107.66) --
	( 89.45,107.66) --
	( 90.98,107.66) --
	( 92.52,107.66) --
	( 94.06,107.66) --
	( 95.60,107.66) --
	( 97.13,107.66) --
	( 98.67,107.66) --
	(100.21,107.66) --
	(101.74,107.66) --
	(103.28,107.66) --
	(104.82,107.66) --
	(106.35,107.66) --
	(107.89,107.66) --
	(109.43,107.66) --
	(110.96,107.66) --
	(112.50,107.66) --
	(114.04,107.66) --
	(115.58,107.66) --
	(117.11,107.66) --
	(118.65,107.66) --
	(120.19,107.66) --
	(121.72,107.66) --
	(123.26,107.66) --
	(124.80,107.66) --
	(126.33,107.66) --
	(127.87,107.66) --
	(129.41,107.66) --
	(130.95,107.66) --
	(132.48,107.66) --
	(134.02,107.66) --
	(135.56,107.66) --
	(137.09,107.66) --
	(138.63,107.66) --
	(140.17,107.66) --
	(141.70,107.66) --
	(143.24,107.66) --
	(144.78,107.66) --
	(146.32,107.66) --
	(147.85,107.66) --
	(149.39,107.66) --
	(150.93,107.66) --
	(152.46,107.66) --
	(154.00,107.66) --
	(155.54,107.66) --
	(157.07,107.66) --
	(158.61,107.66) --
	(160.15,107.66) --
	(161.69,107.66) --
	(163.22,107.66) --
	(164.76,107.74) --
	(166.30,108.72) --
	(167.83,110.80) --
	(169.37,113.97) --
	(170.91,118.24) --
	(172.44,123.61) --
	(173.98,130.07) --
	(175.52,137.63) --
	(177.05,146.28) --
	(178.59,156.03) --
	(180.13,164.96) --
	(181.67,168.92) --
	(183.20,167.77) --
	(184.74,161.51) --
	(186.28,150.14) --
	(187.81,133.78) --
	(189.35,119.27) --
	(190.89,110.56);
\definecolor{drawColor}{RGB}{255,152,0}

\path[draw=drawColor,line width= 1.1pt,line join=round] ( 40.26,107.66) --
	( 41.80,107.66) --
	( 43.34,107.66) --
	( 44.88,107.66) --
	( 46.41,107.66) --
	( 47.95,107.66) --
	( 49.49,107.66) --
	( 51.02,107.66) --
	( 52.56,107.66) --
	( 54.10,107.66) --
	( 55.63,107.66) --
	( 57.17,107.66) --
	( 58.71,107.66) --
	( 60.24,107.66) --
	( 61.78,107.66) --
	( 63.32,107.66) --
	( 64.86,107.66) --
	( 66.39,107.66) --
	( 67.93,107.66) --
	( 69.47,107.66) --
	( 71.00,107.66) --
	( 72.54,107.66) --
	( 74.08,107.66) --
	( 75.61,107.66) --
	( 77.15,107.66) --
	( 78.69,107.66) --
	( 80.23,107.66) --
	( 81.76,107.66) --
	( 83.30,107.66) --
	( 84.84,107.66) --
	( 86.37,107.66) --
	( 87.91,107.66) --
	( 89.45,107.66) --
	( 90.98,107.66) --
	( 92.52,107.66) --
	( 94.06,107.66) --
	( 95.60,107.66) --
	( 97.13,107.66) --
	( 98.67,107.66) --
	(100.21,107.66) --
	(101.74,107.66) --
	(103.28,107.66) --
	(104.82,107.66) --
	(106.35,107.66) --
	(107.89,107.66) --
	(109.43,107.66) --
	(110.96,107.66) --
	(112.50,107.66) --
	(114.04,107.66) --
	(115.58,107.66) --
	(117.11,107.66) --
	(118.65,107.66) --
	(120.19,107.66) --
	(121.72,107.66) --
	(123.26,107.66) --
	(124.80,107.66) --
	(126.33,107.66) --
	(127.87,107.66) --
	(129.41,107.66) --
	(130.95,107.66) --
	(132.48,107.66) --
	(134.02,107.66) --
	(135.56,107.66) --
	(137.09,107.66) --
	(138.63,107.66) --
	(140.17,107.66) --
	(141.70,107.66) --
	(143.24,107.66) --
	(144.78,107.66) --
	(146.32,107.66) --
	(147.85,107.66) --
	(149.39,107.66) --
	(150.93,107.66) --
	(152.46,107.66) --
	(154.00,107.66) --
	(155.54,107.66) --
	(157.07,107.66) --
	(158.61,107.66) --
	(160.15,107.66) --
	(161.69,107.66) --
	(163.22,107.66) --
	(164.76,107.66) --
	(166.30,107.66) --
	(167.83,107.66) --
	(169.37,107.66) --
	(170.91,107.66) --
	(172.44,107.66) --
	(173.98,107.66) --
	(175.52,107.66) --
	(177.05,107.66) --
	(178.59,107.66) --
	(180.13,108.66) --
	(181.67,112.83) --
	(183.20,120.24) --
	(184.74,130.89) --
	(186.28,144.79) --
	(187.81,161.74) --
	(189.35,171.00) --
	(190.89,166.36);
\definecolor{drawColor}{RGB}{121,84,71}

\path[draw=drawColor,line width= 1.1pt,line join=round] ( 40.26,107.66) --
	( 41.80,107.66) --
	( 43.34,107.66) --
	( 44.88,107.66) --
	( 46.41,107.66) --
	( 47.95,107.66) --
	( 49.49,107.66) --
	( 51.02,107.66) --
	( 52.56,107.66) --
	( 54.10,107.66) --
	( 55.63,107.66) --
	( 57.17,107.66) --
	( 58.71,107.66) --
	( 60.24,107.66) --
	( 61.78,107.66) --
	( 63.32,107.66) --
	( 64.86,107.66) --
	( 66.39,107.66) --
	( 67.93,107.66) --
	( 69.47,107.66) --
	( 71.00,107.66) --
	( 72.54,107.66) --
	( 74.08,107.66) --
	( 75.61,107.66) --
	( 77.15,107.66) --
	( 78.69,107.66) --
	( 80.23,107.66) --
	( 81.76,107.66) --
	( 83.30,107.66) --
	( 84.84,107.66) --
	( 86.37,107.66) --
	( 87.91,107.66) --
	( 89.45,107.66) --
	( 90.98,107.66) --
	( 92.52,107.66) --
	( 94.06,107.66) --
	( 95.60,107.66) --
	( 97.13,107.66) --
	( 98.67,107.66) --
	(100.21,107.66) --
	(101.74,107.66) --
	(103.28,107.66) --
	(104.82,107.66) --
	(106.35,107.66) --
	(107.89,107.66) --
	(109.43,107.66) --
	(110.96,107.66) --
	(112.50,107.66) --
	(114.04,107.66) --
	(115.58,107.66) --
	(117.11,107.66) --
	(118.65,107.66) --
	(120.19,107.66) --
	(121.72,107.66) --
	(123.26,107.66) --
	(124.80,107.66) --
	(126.33,107.66) --
	(127.87,107.66) --
	(129.41,107.66) --
	(130.95,107.66) --
	(132.48,107.66) --
	(134.02,107.66) --
	(135.56,107.66) --
	(137.09,107.66) --
	(138.63,107.66) --
	(140.17,107.66) --
	(141.70,107.66) --
	(143.24,107.66) --
	(144.78,107.66) --
	(146.32,107.66) --
	(147.85,107.66) --
	(149.39,107.66) --
	(150.93,107.66) --
	(152.46,107.66) --
	(154.00,107.66) --
	(155.54,107.66) --
	(157.07,107.66) --
	(158.61,107.66) --
	(160.15,107.66) --
	(161.69,107.66) --
	(163.22,107.66) --
	(164.76,107.66) --
	(166.30,107.66) --
	(167.83,107.66) --
	(169.37,107.66) --
	(170.91,107.66) --
	(172.44,107.66) --
	(173.98,107.66) --
	(175.52,107.66) --
	(177.05,107.66) --
	(178.59,107.66) --
	(180.13,107.66) --
	(181.67,107.66) --
	(183.20,107.66) --
	(184.74,107.66) --
	(186.28,107.66) --
	(187.81,107.75) --
	(189.35,113.00) --
	(190.89,126.35);
\end{scope}
\begin{scope}
\path[clip] ( 32.73, 21.76) rectangle (198.42, 98.66);
\definecolor{drawColor}{gray}{0.92}

\path[draw=drawColor,line width= 0.3pt,line join=round] ( 32.73, 35.29) --
	(198.42, 35.29);

\path[draw=drawColor,line width= 0.3pt,line join=round] ( 32.73, 55.36) --
	(198.42, 55.36);

\path[draw=drawColor,line width= 0.3pt,line join=round] ( 32.73, 75.44) --
	(198.42, 75.44);

\path[draw=drawColor,line width= 0.3pt,line join=round] ( 32.73, 95.51) --
	(198.42, 95.51);

\path[draw=drawColor,line width= 0.3pt,line join=round] ( 57.94, 21.76) --
	( 57.94, 98.66);

\path[draw=drawColor,line width= 0.3pt,line join=round] ( 96.36, 21.76) --
	( 96.36, 98.66);

\path[draw=drawColor,line width= 0.3pt,line join=round] (134.79, 21.76) --
	(134.79, 98.66);

\path[draw=drawColor,line width= 0.3pt,line join=round] (173.21, 21.76) --
	(173.21, 98.66);

\path[draw=drawColor,line width= 0.6pt,line join=round] ( 32.73, 25.25) --
	(198.42, 25.25);

\path[draw=drawColor,line width= 0.6pt,line join=round] ( 32.73, 45.33) --
	(198.42, 45.33);

\path[draw=drawColor,line width= 0.6pt,line join=round] ( 32.73, 65.40) --
	(198.42, 65.40);

\path[draw=drawColor,line width= 0.6pt,line join=round] ( 32.73, 85.47) --
	(198.42, 85.47);

\path[draw=drawColor,line width= 0.6pt,line join=round] ( 38.73, 21.76) --
	( 38.73, 98.66);

\path[draw=drawColor,line width= 0.6pt,line join=round] ( 77.15, 21.76) --
	( 77.15, 98.66);

\path[draw=drawColor,line width= 0.6pt,line join=round] (115.58, 21.76) --
	(115.58, 98.66);

\path[draw=drawColor,line width= 0.6pt,line join=round] (154.00, 21.76) --
	(154.00, 98.66);

\path[draw=drawColor,line width= 0.6pt,line join=round] (192.42, 21.76) --
	(192.42, 98.66);
\definecolor{drawColor}{RGB}{155,38,176}

\path[draw=drawColor,line width= 1.1pt,line join=round] ( 40.26, 40.52) --
	( 41.80, 39.76) --
	( 43.34, 39.01) --
	( 44.88, 38.28) --
	( 46.41, 37.57) --
	( 47.95, 36.88) --
	( 49.49, 36.22) --
	( 51.02, 35.57) --
	( 52.56, 34.94) --
	( 54.10, 34.33) --
	( 55.63, 33.74) --
	( 57.17, 33.17) --
	( 58.71, 32.62) --
	( 60.24, 32.09) --
	( 61.78, 31.58) --
	( 63.32, 31.09) --
	( 64.86, 30.62) --
	( 66.39, 30.17) --
	( 67.93, 29.74) --
	( 69.47, 29.33) --
	( 71.00, 28.93) --
	( 72.54, 28.56) --
	( 74.08, 28.21) --
	( 75.61, 27.88) --
	( 77.15, 27.57) --
	( 78.69, 27.27) --
	( 80.23, 27.00) --
	( 81.76, 26.75) --
	( 83.30, 26.51) --
	( 84.84, 26.30) --
	( 86.37, 26.11) --
	( 87.91, 25.93) --
	( 89.45, 25.78) --
	( 90.98, 25.64) --
	( 92.52, 25.53) --
	( 94.06, 25.44) --
	( 95.60, 25.36) --
	( 97.13, 25.30) --
	( 98.67, 25.27) --
	(100.21, 25.25) --
	(101.74, 25.25) --
	(103.28, 25.25) --
	(104.82, 25.25) --
	(106.35, 25.25) --
	(107.89, 25.25) --
	(109.43, 25.25) --
	(110.96, 25.25) --
	(112.50, 25.25) --
	(114.04, 25.25) --
	(115.58, 25.25) --
	(117.11, 25.25) --
	(118.65, 25.25) --
	(120.19, 25.25) --
	(121.72, 25.25) --
	(123.26, 25.25) --
	(124.80, 25.25) --
	(126.33, 25.25) --
	(127.87, 25.25) --
	(129.41, 25.25) --
	(130.95, 25.25) --
	(132.48, 25.25) --
	(134.02, 25.25) --
	(135.56, 25.25) --
	(137.09, 25.25) --
	(138.63, 25.25) --
	(140.17, 25.25) --
	(141.70, 25.25) --
	(143.24, 25.25) --
	(144.78, 25.25) --
	(146.32, 25.25) --
	(147.85, 25.25) --
	(149.39, 25.25) --
	(150.93, 25.25) --
	(152.46, 25.25) --
	(154.00, 25.25) --
	(155.54, 25.25) --
	(157.07, 25.25) --
	(158.61, 25.25) --
	(160.15, 25.25) --
	(161.69, 25.25) --
	(163.22, 25.25) --
	(164.76, 25.25) --
	(166.30, 25.25) --
	(167.83, 25.25) --
	(169.37, 25.25) --
	(170.91, 25.25) --
	(172.44, 25.25) --
	(173.98, 25.25) --
	(175.52, 25.25) --
	(177.05, 25.25) --
	(178.59, 25.25) --
	(180.13, 25.25) --
	(181.67, 25.25) --
	(183.20, 25.25) --
	(184.74, 25.25) --
	(186.28, 25.25) --
	(187.81, 25.25) --
	(189.35, 25.25) --
	(190.89, 25.25);
\definecolor{drawColor}{RGB}{63,81,180}

\path[draw=drawColor,line width= 1.1pt,line join=round] ( 40.26, 90.24) --
	( 41.80, 90.92) --
	( 43.34, 91.52) --
	( 44.88, 92.03) --
	( 46.41, 92.47) --
	( 47.95, 92.82) --
	( 49.49, 93.10) --
	( 51.02, 93.29) --
	( 52.56, 93.41) --
	( 54.10, 93.44) --
	( 55.63, 93.39) --
	( 57.17, 93.27) --
	( 58.71, 93.06) --
	( 60.24, 92.77) --
	( 61.78, 92.40) --
	( 63.32, 91.95) --
	( 64.86, 91.42) --
	( 66.39, 90.81) --
	( 67.93, 90.12) --
	( 69.47, 89.35) --
	( 71.00, 88.50) --
	( 72.54, 87.57) --
	( 74.08, 86.56) --
	( 75.61, 85.47) --
	( 77.15, 84.30) --
	( 78.69, 83.04) --
	( 80.23, 81.71) --
	( 81.76, 80.30) --
	( 83.30, 78.80) --
	( 84.84, 77.23) --
	( 86.37, 75.57) --
	( 87.91, 73.84) --
	( 89.45, 72.02) --
	( 90.98, 70.13) --
	( 92.52, 68.15) --
	( 94.06, 66.09) --
	( 95.60, 63.96) --
	( 97.13, 61.74) --
	( 98.67, 59.44) --
	(100.21, 57.06) --
	(101.74, 54.65) --
	(103.28, 52.33) --
	(104.82, 50.10) --
	(106.35, 47.96) --
	(107.89, 45.92) --
	(109.43, 43.98) --
	(110.96, 42.13) --
	(112.50, 40.38) --
	(114.04, 38.73) --
	(115.58, 37.17) --
	(117.11, 35.71) --
	(118.65, 34.34) --
	(120.19, 33.07) --
	(121.72, 31.89) --
	(123.26, 30.81) --
	(124.80, 29.83) --
	(126.33, 28.94) --
	(127.87, 28.15) --
	(129.41, 27.45) --
	(130.95, 26.85) --
	(132.48, 26.34) --
	(134.02, 25.93) --
	(135.56, 25.62) --
	(137.09, 25.40) --
	(138.63, 25.28) --
	(140.17, 25.25) --
	(141.70, 25.25) --
	(143.24, 25.25) --
	(144.78, 25.25) --
	(146.32, 25.25) --
	(147.85, 25.25) --
	(149.39, 25.25) --
	(150.93, 25.25) --
	(152.46, 25.25) --
	(154.00, 25.25) --
	(155.54, 25.25) --
	(157.07, 25.25) --
	(158.61, 25.25) --
	(160.15, 25.25) --
	(161.69, 25.25) --
	(163.22, 25.25) --
	(164.76, 25.25) --
	(166.30, 25.25) --
	(167.83, 25.25) --
	(169.37, 25.25) --
	(170.91, 25.25) --
	(172.44, 25.25) --
	(173.98, 25.25) --
	(175.52, 25.25) --
	(177.05, 25.25) --
	(178.59, 25.25) --
	(180.13, 25.25) --
	(181.67, 25.25) --
	(183.20, 25.25) --
	(184.74, 25.25) --
	(186.28, 25.25) --
	(187.81, 25.25) --
	(189.35, 25.25) --
	(190.89, 25.25);
\definecolor{drawColor}{RGB}{2,169,243}

\path[draw=drawColor,line width= 1.1pt,line join=round] ( 40.26, 25.28) --
	( 41.80, 25.37) --
	( 43.34, 25.53) --
	( 44.88, 25.74) --
	( 46.41, 26.01) --
	( 47.95, 26.34) --
	( 49.49, 26.74) --
	( 51.02, 27.19) --
	( 52.56, 27.71) --
	( 54.10, 28.28) --
	( 55.63, 28.92) --
	( 57.17, 29.62) --
	( 58.71, 30.37) --
	( 60.24, 31.19) --
	( 61.78, 32.07) --
	( 63.32, 33.01) --
	( 64.86, 34.01) --
	( 66.39, 35.07) --
	( 67.93, 36.19) --
	( 69.47, 37.37) --
	( 71.00, 38.62) --
	( 72.54, 39.92) --
	( 74.08, 41.28) --
	( 75.61, 42.71) --
	( 77.15, 44.19) --
	( 78.69, 45.74) --
	( 80.23, 47.34) --
	( 81.76, 49.01) --
	( 83.30, 50.74) --
	( 84.84, 52.52) --
	( 86.37, 54.37) --
	( 87.91, 56.28) --
	( 89.45, 58.25) --
	( 90.98, 60.28) --
	( 92.52, 62.37) --
	( 94.06, 64.52) --
	( 95.60, 66.73) --
	( 97.13, 69.01) --
	( 98.67, 71.34) --
	(100.21, 73.73) --
	(101.74, 76.11) --
	(103.28, 78.25) --
	(104.82, 80.14) --
	(106.35, 81.79) --
	(107.89, 83.18) --
	(109.43, 84.33) --
	(110.96, 85.23) --
	(112.50, 85.88) --
	(114.04, 86.28) --
	(115.58, 86.44) --
	(117.11, 86.34) --
	(118.65, 86.00) --
	(120.19, 85.41) --
	(121.72, 84.57) --
	(123.26, 83.48) --
	(124.80, 82.15) --
	(126.33, 80.56) --
	(127.87, 78.73) --
	(129.41, 76.65) --
	(130.95, 74.32) --
	(132.48, 71.74) --
	(134.02, 68.92) --
	(135.56, 65.84) --
	(137.09, 62.52) --
	(138.63, 58.95) --
	(140.17, 55.14) --
	(141.70, 51.44) --
	(143.24, 47.97) --
	(144.78, 44.76) --
	(146.32, 41.79) --
	(147.85, 39.06) --
	(149.39, 36.58) --
	(150.93, 34.34) --
	(152.46, 32.35) --
	(154.00, 30.61) --
	(155.54, 29.11) --
	(157.07, 27.86) --
	(158.61, 26.85) --
	(160.15, 26.09) --
	(161.69, 25.57) --
	(163.22, 25.30) --
	(164.76, 25.25) --
	(166.30, 25.25) --
	(167.83, 25.25) --
	(169.37, 25.25) --
	(170.91, 25.25) --
	(172.44, 25.25) --
	(173.98, 25.25) --
	(175.52, 25.25) --
	(177.05, 25.25) --
	(178.59, 25.25) --
	(180.13, 25.25) --
	(181.67, 25.25) --
	(183.20, 25.25) --
	(184.74, 25.25) --
	(186.28, 25.25) --
	(187.81, 25.25) --
	(189.35, 25.25) --
	(190.89, 25.25);
\definecolor{drawColor}{RGB}{0,150,135}

\path[draw=drawColor,line width= 1.1pt,line join=round] ( 40.26, 25.25) --
	( 41.80, 25.25) --
	( 43.34, 25.25) --
	( 44.88, 25.25) --
	( 46.41, 25.25) --
	( 47.95, 25.25) --
	( 49.49, 25.25) --
	( 51.02, 25.25) --
	( 52.56, 25.25) --
	( 54.10, 25.25) --
	( 55.63, 25.25) --
	( 57.17, 25.25) --
	( 58.71, 25.25) --
	( 60.24, 25.25) --
	( 61.78, 25.25) --
	( 63.32, 25.25) --
	( 64.86, 25.25) --
	( 66.39, 25.25) --
	( 67.93, 25.25) --
	( 69.47, 25.25) --
	( 71.00, 25.25) --
	( 72.54, 25.25) --
	( 74.08, 25.25) --
	( 75.61, 25.25) --
	( 77.15, 25.25) --
	( 78.69, 25.25) --
	( 80.23, 25.25) --
	( 81.76, 25.25) --
	( 83.30, 25.25) --
	( 84.84, 25.25) --
	( 86.37, 25.25) --
	( 87.91, 25.25) --
	( 89.45, 25.25) --
	( 90.98, 25.25) --
	( 92.52, 25.25) --
	( 94.06, 25.25) --
	( 95.60, 25.25) --
	( 97.13, 25.25) --
	( 98.67, 25.25) --
	(100.21, 25.25) --
	(101.74, 25.29) --
	(103.28, 25.48) --
	(104.82, 25.81) --
	(106.35, 26.30) --
	(107.89, 26.95) --
	(109.43, 27.74) --
	(110.96, 28.69) --
	(112.50, 29.79) --
	(114.04, 31.04) --
	(115.58, 32.44) --
	(117.11, 34.00) --
	(118.65, 35.71) --
	(120.19, 37.57) --
	(121.72, 39.59) --
	(123.26, 41.76) --
	(124.80, 44.08) --
	(126.33, 46.55) --
	(127.87, 49.17) --
	(129.41, 51.95) --
	(130.95, 54.88) --
	(132.48, 57.96) --
	(134.02, 61.20) --
	(135.56, 64.59) --
	(137.09, 68.13) --
	(138.63, 71.82) --
	(140.17, 75.64) --
	(141.70, 79.06) --
	(143.24, 81.84) --
	(144.78, 83.97) --
	(146.32, 85.45) --
	(147.85, 86.29) --
	(149.39, 86.49) --
	(150.93, 86.04) --
	(152.46, 84.95) --
	(154.00, 83.22) --
	(155.54, 80.84) --
	(157.07, 77.82) --
	(158.61, 74.15) --
	(160.15, 69.84) --
	(161.69, 64.88) --
	(163.22, 59.28) --
	(164.76, 53.14) --
	(166.30, 47.41) --
	(167.83, 42.35) --
	(169.37, 37.94) --
	(170.91, 34.18) --
	(172.44, 31.08) --
	(173.98, 28.65) --
	(175.52, 26.86) --
	(177.05, 25.74) --
	(178.59, 25.27) --
	(180.13, 25.25) --
	(181.67, 25.25) --
	(183.20, 25.25) --
	(184.74, 25.25) --
	(186.28, 25.25) --
	(187.81, 25.25) --
	(189.35, 25.25) --
	(190.89, 25.25);
\definecolor{drawColor}{RGB}{139,195,74}

\path[draw=drawColor,line width= 1.1pt,line join=round] ( 40.26, 25.25) --
	( 41.80, 25.25) --
	( 43.34, 25.25) --
	( 44.88, 25.25) --
	( 46.41, 25.25) --
	( 47.95, 25.25) --
	( 49.49, 25.25) --
	( 51.02, 25.25) --
	( 52.56, 25.25) --
	( 54.10, 25.25) --
	( 55.63, 25.25) --
	( 57.17, 25.25) --
	( 58.71, 25.25) --
	( 60.24, 25.25) --
	( 61.78, 25.25) --
	( 63.32, 25.25) --
	( 64.86, 25.25) --
	( 66.39, 25.25) --
	( 67.93, 25.25) --
	( 69.47, 25.25) --
	( 71.00, 25.25) --
	( 72.54, 25.25) --
	( 74.08, 25.25) --
	( 75.61, 25.25) --
	( 77.15, 25.25) --
	( 78.69, 25.25) --
	( 80.23, 25.25) --
	( 81.76, 25.25) --
	( 83.30, 25.25) --
	( 84.84, 25.25) --
	( 86.37, 25.25) --
	( 87.91, 25.25) --
	( 89.45, 25.25) --
	( 90.98, 25.25) --
	( 92.52, 25.25) --
	( 94.06, 25.25) --
	( 95.60, 25.25) --
	( 97.13, 25.25) --
	( 98.67, 25.25) --
	(100.21, 25.25) --
	(101.74, 25.25) --
	(103.28, 25.25) --
	(104.82, 25.25) --
	(106.35, 25.25) --
	(107.89, 25.25) --
	(109.43, 25.25) --
	(110.96, 25.25) --
	(112.50, 25.25) --
	(114.04, 25.25) --
	(115.58, 25.25) --
	(117.11, 25.25) --
	(118.65, 25.25) --
	(120.19, 25.25) --
	(121.72, 25.25) --
	(123.26, 25.25) --
	(124.80, 25.25) --
	(126.33, 25.25) --
	(127.87, 25.25) --
	(129.41, 25.25) --
	(130.95, 25.25) --
	(132.48, 25.25) --
	(134.02, 25.25) --
	(135.56, 25.25) --
	(137.09, 25.25) --
	(138.63, 25.25) --
	(140.17, 25.26) --
	(141.70, 25.55) --
	(143.24, 26.24) --
	(144.78, 27.33) --
	(146.32, 28.82) --
	(147.85, 30.70) --
	(149.39, 32.98) --
	(150.93, 35.67) --
	(152.46, 38.75) --
	(154.00, 42.23) --
	(155.54, 46.10) --
	(157.07, 50.38) --
	(158.61, 55.06) --
	(160.15, 60.13) --
	(161.69, 65.60) --
	(163.22, 71.47) --
	(164.76, 77.58) --
	(166.30, 82.32) --
	(167.83, 85.31) --
	(169.37, 86.55) --
	(170.91, 86.03) --
	(172.44, 83.77) --
	(173.98, 79.74) --
	(175.52, 73.97) --
	(177.05, 66.44) --
	(178.59, 57.16) --
	(180.13, 47.24) --
	(181.67, 39.11) --
	(183.20, 32.85) --
	(184.74, 28.46) --
	(186.28, 25.93) --
	(187.81, 25.25) --
	(189.35, 25.25) --
	(190.89, 25.25);
\definecolor{drawColor}{RGB}{255,235,58}

\path[draw=drawColor,line width= 1.1pt,line join=round] ( 40.26, 25.25) --
	( 41.80, 25.25) --
	( 43.34, 25.25) --
	( 44.88, 25.25) --
	( 46.41, 25.25) --
	( 47.95, 25.25) --
	( 49.49, 25.25) --
	( 51.02, 25.25) --
	( 52.56, 25.25) --
	( 54.10, 25.25) --
	( 55.63, 25.25) --
	( 57.17, 25.25) --
	( 58.71, 25.25) --
	( 60.24, 25.25) --
	( 61.78, 25.25) --
	( 63.32, 25.25) --
	( 64.86, 25.25) --
	( 66.39, 25.25) --
	( 67.93, 25.25) --
	( 69.47, 25.25) --
	( 71.00, 25.25) --
	( 72.54, 25.25) --
	( 74.08, 25.25) --
	( 75.61, 25.25) --
	( 77.15, 25.25) --
	( 78.69, 25.25) --
	( 80.23, 25.25) --
	( 81.76, 25.25) --
	( 83.30, 25.25) --
	( 84.84, 25.25) --
	( 86.37, 25.25) --
	( 87.91, 25.25) --
	( 89.45, 25.25) --
	( 90.98, 25.25) --
	( 92.52, 25.25) --
	( 94.06, 25.25) --
	( 95.60, 25.25) --
	( 97.13, 25.25) --
	( 98.67, 25.25) --
	(100.21, 25.25) --
	(101.74, 25.25) --
	(103.28, 25.25) --
	(104.82, 25.25) --
	(106.35, 25.25) --
	(107.89, 25.25) --
	(109.43, 25.25) --
	(110.96, 25.25) --
	(112.50, 25.25) --
	(114.04, 25.25) --
	(115.58, 25.25) --
	(117.11, 25.25) --
	(118.65, 25.25) --
	(120.19, 25.25) --
	(121.72, 25.25) --
	(123.26, 25.25) --
	(124.80, 25.25) --
	(126.33, 25.25) --
	(127.87, 25.25) --
	(129.41, 25.25) --
	(130.95, 25.25) --
	(132.48, 25.25) --
	(134.02, 25.25) --
	(135.56, 25.25) --
	(137.09, 25.25) --
	(138.63, 25.25) --
	(140.17, 25.25) --
	(141.70, 25.25) --
	(143.24, 25.25) --
	(144.78, 25.25) --
	(146.32, 25.25) --
	(147.85, 25.25) --
	(149.39, 25.25) --
	(150.93, 25.25) --
	(152.46, 25.25) --
	(154.00, 25.25) --
	(155.54, 25.25) --
	(157.07, 25.25) --
	(158.61, 25.25) --
	(160.15, 25.25) --
	(161.69, 25.25) --
	(163.22, 25.25) --
	(164.76, 25.34) --
	(166.30, 26.32) --
	(167.83, 28.39) --
	(169.37, 31.57) --
	(170.91, 35.84) --
	(172.44, 41.20) --
	(173.98, 47.66) --
	(175.52, 55.22) --
	(177.05, 63.88) --
	(178.59, 73.63) --
	(180.13, 82.55) --
	(181.67, 86.51) --
	(183.20, 85.36) --
	(184.74, 79.11) --
	(186.28, 67.74) --
	(187.81, 51.38) --
	(189.35, 36.86) --
	(190.89, 28.16);
\definecolor{drawColor}{RGB}{255,152,0}

\path[draw=drawColor,line width= 1.1pt,line join=round] ( 40.26, 25.25) --
	( 41.80, 25.25) --
	( 43.34, 25.25) --
	( 44.88, 25.25) --
	( 46.41, 25.25) --
	( 47.95, 25.25) --
	( 49.49, 25.25) --
	( 51.02, 25.25) --
	( 52.56, 25.25) --
	( 54.10, 25.25) --
	( 55.63, 25.25) --
	( 57.17, 25.25) --
	( 58.71, 25.25) --
	( 60.24, 25.25) --
	( 61.78, 25.25) --
	( 63.32, 25.25) --
	( 64.86, 25.25) --
	( 66.39, 25.25) --
	( 67.93, 25.25) --
	( 69.47, 25.25) --
	( 71.00, 25.25) --
	( 72.54, 25.25) --
	( 74.08, 25.25) --
	( 75.61, 25.25) --
	( 77.15, 25.25) --
	( 78.69, 25.25) --
	( 80.23, 25.25) --
	( 81.76, 25.25) --
	( 83.30, 25.25) --
	( 84.84, 25.25) --
	( 86.37, 25.25) --
	( 87.91, 25.25) --
	( 89.45, 25.25) --
	( 90.98, 25.25) --
	( 92.52, 25.25) --
	( 94.06, 25.25) --
	( 95.60, 25.25) --
	( 97.13, 25.25) --
	( 98.67, 25.25) --
	(100.21, 25.25) --
	(101.74, 25.25) --
	(103.28, 25.25) --
	(104.82, 25.25) --
	(106.35, 25.25) --
	(107.89, 25.25) --
	(109.43, 25.25) --
	(110.96, 25.25) --
	(112.50, 25.25) --
	(114.04, 25.25) --
	(115.58, 25.25) --
	(117.11, 25.25) --
	(118.65, 25.25) --
	(120.19, 25.25) --
	(121.72, 25.25) --
	(123.26, 25.25) --
	(124.80, 25.25) --
	(126.33, 25.25) --
	(127.87, 25.25) --
	(129.41, 25.25) --
	(130.95, 25.25) --
	(132.48, 25.25) --
	(134.02, 25.25) --
	(135.56, 25.25) --
	(137.09, 25.25) --
	(138.63, 25.25) --
	(140.17, 25.25) --
	(141.70, 25.25) --
	(143.24, 25.25) --
	(144.78, 25.25) --
	(146.32, 25.25) --
	(147.85, 25.25) --
	(149.39, 25.25) --
	(150.93, 25.25) --
	(152.46, 25.25) --
	(154.00, 25.25) --
	(155.54, 25.25) --
	(157.07, 25.25) --
	(158.61, 25.25) --
	(160.15, 25.25) --
	(161.69, 25.25) --
	(163.22, 25.25) --
	(164.76, 25.25) --
	(166.30, 25.25) --
	(167.83, 25.25) --
	(169.37, 25.25) --
	(170.91, 25.25) --
	(172.44, 25.25) --
	(173.98, 25.25) --
	(175.52, 25.25) --
	(177.05, 25.25) --
	(178.59, 25.25) --
	(180.13, 26.26) --
	(181.67, 30.43) --
	(183.20, 37.84) --
	(184.74, 48.49) --
	(186.28, 62.38) --
	(187.81, 79.38) --
	(189.35, 91.80) --
	(190.89, 95.17);
\definecolor{drawColor}{RGB}{121,84,71}

\path[draw=drawColor,line width= 1.1pt,line join=round] ( 40.26, 25.25) --
	( 41.80, 25.25) --
	( 43.34, 25.25) --
	( 44.88, 25.25) --
	( 46.41, 25.25) --
	( 47.95, 25.25) --
	( 49.49, 25.25) --
	( 51.02, 25.25) --
	( 52.56, 25.25) --
	( 54.10, 25.25) --
	( 55.63, 25.25) --
	( 57.17, 25.25) --
	( 58.71, 25.25) --
	( 60.24, 25.25) --
	( 61.78, 25.25) --
	( 63.32, 25.25) --
	( 64.86, 25.25) --
	( 66.39, 25.25) --
	( 67.93, 25.25) --
	( 69.47, 25.25) --
	( 71.00, 25.25) --
	( 72.54, 25.25) --
	( 74.08, 25.25) --
	( 75.61, 25.25) --
	( 77.15, 25.25) --
	( 78.69, 25.25) --
	( 80.23, 25.25) --
	( 81.76, 25.25) --
	( 83.30, 25.25) --
	( 84.84, 25.25) --
	( 86.37, 25.25) --
	( 87.91, 25.25) --
	( 89.45, 25.25) --
	( 90.98, 25.25) --
	( 92.52, 25.25) --
	( 94.06, 25.25) --
	( 95.60, 25.25) --
	( 97.13, 25.25) --
	( 98.67, 25.25) --
	(100.21, 25.25) --
	(101.74, 25.25) --
	(103.28, 25.25) --
	(104.82, 25.25) --
	(106.35, 25.25) --
	(107.89, 25.25) --
	(109.43, 25.25) --
	(110.96, 25.25) --
	(112.50, 25.25) --
	(114.04, 25.25) --
	(115.58, 25.25) --
	(117.11, 25.25) --
	(118.65, 25.25) --
	(120.19, 25.25) --
	(121.72, 25.25) --
	(123.26, 25.25) --
	(124.80, 25.25) --
	(126.33, 25.25) --
	(127.87, 25.25) --
	(129.41, 25.25) --
	(130.95, 25.25) --
	(132.48, 25.25) --
	(134.02, 25.25) --
	(135.56, 25.25) --
	(137.09, 25.25) --
	(138.63, 25.25) --
	(140.17, 25.25) --
	(141.70, 25.25) --
	(143.24, 25.25) --
	(144.78, 25.25) --
	(146.32, 25.25) --
	(147.85, 25.25) --
	(149.39, 25.25) --
	(150.93, 25.25) --
	(152.46, 25.25) --
	(154.00, 25.25) --
	(155.54, 25.25) --
	(157.07, 25.25) --
	(158.61, 25.25) --
	(160.15, 25.25) --
	(161.69, 25.25) --
	(163.22, 25.25) --
	(164.76, 25.25) --
	(166.30, 25.25) --
	(167.83, 25.25) --
	(169.37, 25.25) --
	(170.91, 25.25) --
	(172.44, 25.25) --
	(173.98, 25.25) --
	(175.52, 25.25) --
	(177.05, 25.25) --
	(178.59, 25.25) --
	(180.13, 25.25) --
	(181.67, 25.25) --
	(183.20, 25.25) --
	(184.74, 25.25) --
	(186.28, 25.25) --
	(187.81, 25.29) --
	(189.35, 27.39) --
	(190.89, 32.73);
\end{scope}
\begin{scope}
\path[clip] (209.80,186.57) rectangle (375.49,263.47);
\definecolor{drawColor}{gray}{0.92}

\path[draw=drawColor,line width= 0.3pt,line join=round] (209.80,200.10) --
	(375.49,200.10);

\path[draw=drawColor,line width= 0.3pt,line join=round] (209.80,220.17) --
	(375.49,220.17);

\path[draw=drawColor,line width= 0.3pt,line join=round] (209.80,240.25) --
	(375.49,240.25);

\path[draw=drawColor,line width= 0.3pt,line join=round] (209.80,260.32) --
	(375.49,260.32);

\path[draw=drawColor,line width= 0.3pt,line join=round] (235.01,186.57) --
	(235.01,263.47);

\path[draw=drawColor,line width= 0.3pt,line join=round] (273.43,186.57) --
	(273.43,263.47);

\path[draw=drawColor,line width= 0.3pt,line join=round] (311.85,186.57) --
	(311.85,263.47);

\path[draw=drawColor,line width= 0.3pt,line join=round] (350.28,186.57) --
	(350.28,263.47);

\path[draw=drawColor,line width= 0.6pt,line join=round] (209.80,190.06) --
	(375.49,190.06);

\path[draw=drawColor,line width= 0.6pt,line join=round] (209.80,210.14) --
	(375.49,210.14);

\path[draw=drawColor,line width= 0.6pt,line join=round] (209.80,230.21) --
	(375.49,230.21);

\path[draw=drawColor,line width= 0.6pt,line join=round] (209.80,250.28) --
	(375.49,250.28);

\path[draw=drawColor,line width= 0.6pt,line join=round] (215.79,186.57) --
	(215.79,263.47);

\path[draw=drawColor,line width= 0.6pt,line join=round] (254.22,186.57) --
	(254.22,263.47);

\path[draw=drawColor,line width= 0.6pt,line join=round] (292.64,186.57) --
	(292.64,263.47);

\path[draw=drawColor,line width= 0.6pt,line join=round] (331.07,186.57) --
	(331.07,263.47);

\path[draw=drawColor,line width= 0.6pt,line join=round] (369.49,186.57) --
	(369.49,263.47);
\definecolor{drawColor}{RGB}{155,38,176}

\path[draw=drawColor,line width= 1.1pt,line join=round] (217.33,246.82) --
	(218.87,239.81) --
	(220.40,233.25) --
	(221.94,227.17) --
	(223.48,221.54) --
	(225.02,216.37) --
	(226.55,211.67) --
	(228.09,207.43) --
	(229.63,203.66) --
	(231.16,200.34) --
	(232.70,197.49) --
	(234.24,195.10) --
	(235.77,193.17) --
	(237.31,191.71) --
	(238.85,190.71) --
	(240.39,190.17) --
	(241.92,190.06) --
	(243.46,190.06) --
	(245.00,190.06) --
	(246.53,190.06) --
	(248.07,190.06) --
	(249.61,190.06) --
	(251.14,190.06) --
	(252.68,190.06) --
	(254.22,190.06) --
	(255.76,190.06) --
	(257.29,190.06) --
	(258.83,190.06) --
	(260.37,190.06) --
	(261.90,190.06) --
	(263.44,190.06) --
	(264.98,190.06) --
	(266.51,190.06) --
	(268.05,190.06) --
	(269.59,190.06) --
	(271.13,190.06) --
	(272.66,190.06) --
	(274.20,190.06) --
	(275.74,190.06) --
	(277.27,190.06) --
	(278.81,190.06) --
	(280.35,190.06) --
	(281.88,190.06) --
	(283.42,190.06) --
	(284.96,190.06) --
	(286.49,190.06) --
	(288.03,190.06) --
	(289.57,190.06) --
	(291.11,190.06) --
	(292.64,190.06) --
	(294.18,190.06) --
	(295.72,190.06) --
	(297.25,190.06) --
	(298.79,190.06) --
	(300.33,190.06) --
	(301.86,190.06) --
	(303.40,190.06) --
	(304.94,190.06) --
	(306.48,190.06) --
	(308.01,190.06) --
	(309.55,190.06) --
	(311.09,190.06) --
	(312.62,190.06) --
	(314.16,190.06) --
	(315.70,190.06) --
	(317.23,190.06) --
	(318.77,190.06) --
	(320.31,190.06) --
	(321.85,190.06) --
	(323.38,190.06) --
	(324.92,190.06) --
	(326.46,190.06) --
	(327.99,190.06) --
	(329.53,190.06) --
	(331.07,190.06) --
	(332.60,190.06) --
	(334.14,190.06) --
	(335.68,190.06) --
	(337.22,190.06) --
	(338.75,190.06) --
	(340.29,190.06) --
	(341.83,190.06) --
	(343.36,190.06) --
	(344.90,190.06) --
	(346.44,190.06) --
	(347.97,190.06) --
	(349.51,190.06) --
	(351.05,190.06) --
	(352.58,190.06) --
	(354.12,190.06) --
	(355.66,190.06) --
	(357.20,190.06) --
	(358.73,190.06) --
	(360.27,190.06) --
	(361.81,190.06) --
	(363.34,190.06) --
	(364.88,190.06) --
	(366.42,190.06) --
	(367.95,190.06);
\definecolor{drawColor}{RGB}{63,81,180}

\path[draw=drawColor,line width= 1.1pt,line join=round] (217.33,213.45) --
	(218.87,220.03) --
	(220.40,225.86) --
	(221.94,230.94) --
	(223.48,235.27) --
	(225.02,238.84) --
	(226.55,241.67) --
	(228.09,243.74) --
	(229.63,245.06) --
	(231.16,245.63) --
	(232.70,245.44) --
	(234.24,244.51) --
	(235.77,242.82) --
	(237.31,240.38) --
	(238.85,237.20) --
	(240.39,233.25) --
	(241.92,228.62) --
	(243.46,224.04) --
	(245.00,219.76) --
	(246.53,215.76) --
	(248.07,212.05) --
	(249.61,208.63) --
	(251.14,205.50) --
	(252.68,202.65) --
	(254.22,200.10) --
	(255.76,197.84) --
	(257.29,195.86) --
	(258.83,194.17) --
	(260.37,192.78) --
	(261.90,191.67) --
	(263.44,190.85) --
	(264.98,190.32) --
	(266.51,190.08) --
	(268.05,190.06) --
	(269.59,190.06) --
	(271.13,190.06) --
	(272.66,190.06) --
	(274.20,190.06) --
	(275.74,190.06) --
	(277.27,190.06) --
	(278.81,190.06) --
	(280.35,190.06) --
	(281.88,190.06) --
	(283.42,190.06) --
	(284.96,190.06) --
	(286.49,190.06) --
	(288.03,190.06) --
	(289.57,190.06) --
	(291.11,190.06) --
	(292.64,190.06) --
	(294.18,190.06) --
	(295.72,190.06) --
	(297.25,190.06) --
	(298.79,190.06) --
	(300.33,190.06) --
	(301.86,190.06) --
	(303.40,190.06) --
	(304.94,190.06) --
	(306.48,190.06) --
	(308.01,190.06) --
	(309.55,190.06) --
	(311.09,190.06) --
	(312.62,190.06) --
	(314.16,190.06) --
	(315.70,190.06) --
	(317.23,190.06) --
	(318.77,190.06) --
	(320.31,190.06) --
	(321.85,190.06) --
	(323.38,190.06) --
	(324.92,190.06) --
	(326.46,190.06) --
	(327.99,190.06) --
	(329.53,190.06) --
	(331.07,190.06) --
	(332.60,190.06) --
	(334.14,190.06) --
	(335.68,190.06) --
	(337.22,190.06) --
	(338.75,190.06) --
	(340.29,190.06) --
	(341.83,190.06) --
	(343.36,190.06) --
	(344.90,190.06) --
	(346.44,190.06) --
	(347.97,190.06) --
	(349.51,190.06) --
	(351.05,190.06) --
	(352.58,190.06) --
	(354.12,190.06) --
	(355.66,190.06) --
	(357.20,190.06) --
	(358.73,190.06) --
	(360.27,190.06) --
	(361.81,190.06) --
	(363.34,190.06) --
	(364.88,190.06) --
	(366.42,190.06) --
	(367.95,190.06);
\definecolor{drawColor}{RGB}{2,169,243}

\path[draw=drawColor,line width= 1.1pt,line join=round] (217.33,190.21) --
	(218.87,190.64) --
	(220.40,191.36) --
	(221.94,192.38) --
	(223.48,193.68) --
	(225.02,195.27) --
	(226.55,197.15) --
	(228.09,199.31) --
	(229.63,201.77) --
	(231.16,204.52) --
	(232.70,207.55) --
	(234.24,210.88) --
	(235.77,214.49) --
	(237.31,218.39) --
	(238.85,222.58) --
	(240.39,227.06) --
	(241.92,231.78) --
	(243.46,236.12) --
	(245.00,239.88) --
	(246.53,243.06) --
	(248.07,245.66) --
	(249.61,247.68) --
	(251.14,249.13) --
	(252.68,249.99) --
	(254.22,250.28) --
	(255.76,249.99) --
	(257.29,249.13) --
	(258.83,247.68) --
	(260.37,245.66) --
	(261.90,243.06) --
	(263.44,239.88) --
	(264.98,236.12) --
	(266.51,231.78) --
	(268.05,227.06) --
	(269.59,222.58) --
	(271.13,218.39) --
	(272.66,214.49) --
	(274.20,210.88) --
	(275.74,207.55) --
	(277.27,204.52) --
	(278.81,201.77) --
	(280.35,199.31) --
	(281.88,197.15) --
	(283.42,195.27) --
	(284.96,193.68) --
	(286.49,192.38) --
	(288.03,191.36) --
	(289.57,190.64) --
	(291.11,190.21) --
	(292.64,190.06) --
	(294.18,190.06) --
	(295.72,190.06) --
	(297.25,190.06) --
	(298.79,190.06) --
	(300.33,190.06) --
	(301.86,190.06) --
	(303.40,190.06) --
	(304.94,190.06) --
	(306.48,190.06) --
	(308.01,190.06) --
	(309.55,190.06) --
	(311.09,190.06) --
	(312.62,190.06) --
	(314.16,190.06) --
	(315.70,190.06) --
	(317.23,190.06) --
	(318.77,190.06) --
	(320.31,190.06) --
	(321.85,190.06) --
	(323.38,190.06) --
	(324.92,190.06) --
	(326.46,190.06) --
	(327.99,190.06) --
	(329.53,190.06) --
	(331.07,190.06) --
	(332.60,190.06) --
	(334.14,190.06) --
	(335.68,190.06) --
	(337.22,190.06) --
	(338.75,190.06) --
	(340.29,190.06) --
	(341.83,190.06) --
	(343.36,190.06) --
	(344.90,190.06) --
	(346.44,190.06) --
	(347.97,190.06) --
	(349.51,190.06) --
	(351.05,190.06) --
	(352.58,190.06) --
	(354.12,190.06) --
	(355.66,190.06) --
	(357.20,190.06) --
	(358.73,190.06) --
	(360.27,190.06) --
	(361.81,190.06) --
	(363.34,190.06) --
	(364.88,190.06) --
	(366.42,190.06) --
	(367.95,190.06);
\definecolor{drawColor}{RGB}{0,150,135}

\path[draw=drawColor,line width= 1.1pt,line join=round] (217.33,190.06) --
	(218.87,190.06) --
	(220.40,190.06) --
	(221.94,190.06) --
	(223.48,190.06) --
	(225.02,190.06) --
	(226.55,190.06) --
	(228.09,190.06) --
	(229.63,190.06) --
	(231.16,190.06) --
	(232.70,190.06) --
	(234.24,190.06) --
	(235.77,190.06) --
	(237.31,190.06) --
	(238.85,190.06) --
	(240.39,190.06) --
	(241.92,190.08) --
	(243.46,190.32) --
	(245.00,190.85) --
	(246.53,191.67) --
	(248.07,192.78) --
	(249.61,194.17) --
	(251.14,195.86) --
	(252.68,197.84) --
	(254.22,200.10) --
	(255.76,202.65) --
	(257.29,205.50) --
	(258.83,208.63) --
	(260.37,212.05) --
	(261.90,215.76) --
	(263.44,219.76) --
	(264.98,224.04) --
	(266.51,228.62) --
	(268.05,233.29) --
	(269.59,237.44) --
	(271.13,241.00) --
	(272.66,243.99) --
	(274.20,246.40) --
	(275.74,248.23) --
	(277.27,249.48) --
	(278.81,250.15) --
	(280.35,250.25) --
	(281.88,249.77) --
	(283.42,248.71) --
	(284.96,247.07) --
	(286.49,244.86) --
	(288.03,242.06) --
	(289.57,238.69) --
	(291.11,234.74) --
	(292.64,230.21) --
	(294.18,225.54) --
	(295.72,221.15) --
	(297.25,217.06) --
	(298.79,213.25) --
	(300.33,209.74) --
	(301.86,206.51) --
	(303.40,203.57) --
	(304.94,200.92) --
	(306.48,198.56) --
	(308.01,196.49) --
	(309.55,194.70) --
	(311.09,193.21) --
	(312.62,192.01) --
	(314.16,191.09) --
	(315.70,190.46) --
	(317.23,190.13) --
	(318.77,190.06) --
	(320.31,190.06) --
	(321.85,190.06) --
	(323.38,190.06) --
	(324.92,190.06) --
	(326.46,190.06) --
	(327.99,190.06) --
	(329.53,190.06) --
	(331.07,190.06) --
	(332.60,190.06) --
	(334.14,190.06) --
	(335.68,190.06) --
	(337.22,190.06) --
	(338.75,190.06) --
	(340.29,190.06) --
	(341.83,190.06) --
	(343.36,190.06) --
	(344.90,190.06) --
	(346.44,190.06) --
	(347.97,190.06) --
	(349.51,190.06) --
	(351.05,190.06) --
	(352.58,190.06) --
	(354.12,190.06) --
	(355.66,190.06) --
	(357.20,190.06) --
	(358.73,190.06) --
	(360.27,190.06) --
	(361.81,190.06) --
	(363.34,190.06) --
	(364.88,190.06) --
	(366.42,190.06) --
	(367.95,190.06);
\definecolor{drawColor}{RGB}{139,195,74}

\path[draw=drawColor,line width= 1.1pt,line join=round] (217.33,190.06) --
	(218.87,190.06) --
	(220.40,190.06) --
	(221.94,190.06) --
	(223.48,190.06) --
	(225.02,190.06) --
	(226.55,190.06) --
	(228.09,190.06) --
	(229.63,190.06) --
	(231.16,190.06) --
	(232.70,190.06) --
	(234.24,190.06) --
	(235.77,190.06) --
	(237.31,190.06) --
	(238.85,190.06) --
	(240.39,190.06) --
	(241.92,190.06) --
	(243.46,190.06) --
	(245.00,190.06) --
	(246.53,190.06) --
	(248.07,190.06) --
	(249.61,190.06) --
	(251.14,190.06) --
	(252.68,190.06) --
	(254.22,190.06) --
	(255.76,190.06) --
	(257.29,190.06) --
	(258.83,190.06) --
	(260.37,190.06) --
	(261.90,190.06) --
	(263.44,190.06) --
	(264.98,190.06) --
	(266.51,190.06) --
	(268.05,190.13) --
	(269.59,190.46) --
	(271.13,191.09) --
	(272.66,192.01) --
	(274.20,193.21) --
	(275.74,194.70) --
	(277.27,196.49) --
	(278.81,198.56) --
	(280.35,200.92) --
	(281.88,203.57) --
	(283.42,206.51) --
	(284.96,209.74) --
	(286.49,213.25) --
	(288.03,217.06) --
	(289.57,221.15) --
	(291.11,225.54) --
	(292.64,230.21) --
	(294.18,234.74) --
	(295.72,238.69) --
	(297.25,242.06) --
	(298.79,244.86) --
	(300.33,247.07) --
	(301.86,248.71) --
	(303.40,249.77) --
	(304.94,250.25) --
	(306.48,250.15) --
	(308.01,249.48) --
	(309.55,248.23) --
	(311.09,246.40) --
	(312.62,243.99) --
	(314.16,241.00) --
	(315.70,237.44) --
	(317.23,233.29) --
	(318.77,228.62) --
	(320.31,224.04) --
	(321.85,219.76) --
	(323.38,215.76) --
	(324.92,212.05) --
	(326.46,208.63) --
	(327.99,205.50) --
	(329.53,202.65) --
	(331.07,200.10) --
	(332.60,197.84) --
	(334.14,195.86) --
	(335.68,194.17) --
	(337.22,192.78) --
	(338.75,191.67) --
	(340.29,190.85) --
	(341.83,190.32) --
	(343.36,190.08) --
	(344.90,190.06) --
	(346.44,190.06) --
	(347.97,190.06) --
	(349.51,190.06) --
	(351.05,190.06) --
	(352.58,190.06) --
	(354.12,190.06) --
	(355.66,190.06) --
	(357.20,190.06) --
	(358.73,190.06) --
	(360.27,190.06) --
	(361.81,190.06) --
	(363.34,190.06) --
	(364.88,190.06) --
	(366.42,190.06) --
	(367.95,190.06);
\definecolor{drawColor}{RGB}{255,235,58}

\path[draw=drawColor,line width= 1.1pt,line join=round] (217.33,190.06) --
	(218.87,190.06) --
	(220.40,190.06) --
	(221.94,190.06) --
	(223.48,190.06) --
	(225.02,190.06) --
	(226.55,190.06) --
	(228.09,190.06) --
	(229.63,190.06) --
	(231.16,190.06) --
	(232.70,190.06) --
	(234.24,190.06) --
	(235.77,190.06) --
	(237.31,190.06) --
	(238.85,190.06) --
	(240.39,190.06) --
	(241.92,190.06) --
	(243.46,190.06) --
	(245.00,190.06) --
	(246.53,190.06) --
	(248.07,190.06) --
	(249.61,190.06) --
	(251.14,190.06) --
	(252.68,190.06) --
	(254.22,190.06) --
	(255.76,190.06) --
	(257.29,190.06) --
	(258.83,190.06) --
	(260.37,190.06) --
	(261.90,190.06) --
	(263.44,190.06) --
	(264.98,190.06) --
	(266.51,190.06) --
	(268.05,190.06) --
	(269.59,190.06) --
	(271.13,190.06) --
	(272.66,190.06) --
	(274.20,190.06) --
	(275.74,190.06) --
	(277.27,190.06) --
	(278.81,190.06) --
	(280.35,190.06) --
	(281.88,190.06) --
	(283.42,190.06) --
	(284.96,190.06) --
	(286.49,190.06) --
	(288.03,190.06) --
	(289.57,190.06) --
	(291.11,190.06) --
	(292.64,190.06) --
	(294.18,190.21) --
	(295.72,190.64) --
	(297.25,191.36) --
	(298.79,192.38) --
	(300.33,193.68) --
	(301.86,195.27) --
	(303.40,197.15) --
	(304.94,199.31) --
	(306.48,201.77) --
	(308.01,204.52) --
	(309.55,207.55) --
	(311.09,210.88) --
	(312.62,214.49) --
	(314.16,218.39) --
	(315.70,222.58) --
	(317.23,227.06) --
	(318.77,231.78) --
	(320.31,236.12) --
	(321.85,239.88) --
	(323.38,243.06) --
	(324.92,245.66) --
	(326.46,247.68) --
	(327.99,249.13) --
	(329.53,249.99) --
	(331.07,250.28) --
	(332.60,249.99) --
	(334.14,249.13) --
	(335.68,247.68) --
	(337.22,245.66) --
	(338.75,243.06) --
	(340.29,239.88) --
	(341.83,236.12) --
	(343.36,231.78) --
	(344.90,227.06) --
	(346.44,222.58) --
	(347.97,218.39) --
	(349.51,214.49) --
	(351.05,210.88) --
	(352.58,207.55) --
	(354.12,204.52) --
	(355.66,201.77) --
	(357.20,199.31) --
	(358.73,197.15) --
	(360.27,195.27) --
	(361.81,193.68) --
	(363.34,192.38) --
	(364.88,191.36) --
	(366.42,190.64) --
	(367.95,190.21);
\definecolor{drawColor}{RGB}{255,152,0}

\path[draw=drawColor,line width= 1.1pt,line join=round] (217.33,190.06) --
	(218.87,190.06) --
	(220.40,190.06) --
	(221.94,190.06) --
	(223.48,190.06) --
	(225.02,190.06) --
	(226.55,190.06) --
	(228.09,190.06) --
	(229.63,190.06) --
	(231.16,190.06) --
	(232.70,190.06) --
	(234.24,190.06) --
	(235.77,190.06) --
	(237.31,190.06) --
	(238.85,190.06) --
	(240.39,190.06) --
	(241.92,190.06) --
	(243.46,190.06) --
	(245.00,190.06) --
	(246.53,190.06) --
	(248.07,190.06) --
	(249.61,190.06) --
	(251.14,190.06) --
	(252.68,190.06) --
	(254.22,190.06) --
	(255.76,190.06) --
	(257.29,190.06) --
	(258.83,190.06) --
	(260.37,190.06) --
	(261.90,190.06) --
	(263.44,190.06) --
	(264.98,190.06) --
	(266.51,190.06) --
	(268.05,190.06) --
	(269.59,190.06) --
	(271.13,190.06) --
	(272.66,190.06) --
	(274.20,190.06) --
	(275.74,190.06) --
	(277.27,190.06) --
	(278.81,190.06) --
	(280.35,190.06) --
	(281.88,190.06) --
	(283.42,190.06) --
	(284.96,190.06) --
	(286.49,190.06) --
	(288.03,190.06) --
	(289.57,190.06) --
	(291.11,190.06) --
	(292.64,190.06) --
	(294.18,190.06) --
	(295.72,190.06) --
	(297.25,190.06) --
	(298.79,190.06) --
	(300.33,190.06) --
	(301.86,190.06) --
	(303.40,190.06) --
	(304.94,190.06) --
	(306.48,190.06) --
	(308.01,190.06) --
	(309.55,190.06) --
	(311.09,190.06) --
	(312.62,190.06) --
	(314.16,190.06) --
	(315.70,190.06) --
	(317.23,190.06) --
	(318.77,190.08) --
	(320.31,190.32) --
	(321.85,190.85) --
	(323.38,191.67) --
	(324.92,192.78) --
	(326.46,194.17) --
	(327.99,195.86) --
	(329.53,197.84) --
	(331.07,200.10) --
	(332.60,202.65) --
	(334.14,205.50) --
	(335.68,208.63) --
	(337.22,212.05) --
	(338.75,215.76) --
	(340.29,219.76) --
	(341.83,224.04) --
	(343.36,228.62) --
	(344.90,233.25) --
	(346.44,237.20) --
	(347.97,240.38) --
	(349.51,242.82) --
	(351.05,244.51) --
	(352.58,245.44) --
	(354.12,245.63) --
	(355.66,245.06) --
	(357.20,243.74) --
	(358.73,241.67) --
	(360.27,238.84) --
	(361.81,235.27) --
	(363.34,230.94) --
	(364.88,225.86) --
	(366.42,220.03) --
	(367.95,213.45);
\definecolor{drawColor}{RGB}{121,84,71}

\path[draw=drawColor,line width= 1.1pt,line join=round] (217.33,190.06) --
	(218.87,190.06) --
	(220.40,190.06) --
	(221.94,190.06) --
	(223.48,190.06) --
	(225.02,190.06) --
	(226.55,190.06) --
	(228.09,190.06) --
	(229.63,190.06) --
	(231.16,190.06) --
	(232.70,190.06) --
	(234.24,190.06) --
	(235.77,190.06) --
	(237.31,190.06) --
	(238.85,190.06) --
	(240.39,190.06) --
	(241.92,190.06) --
	(243.46,190.06) --
	(245.00,190.06) --
	(246.53,190.06) --
	(248.07,190.06) --
	(249.61,190.06) --
	(251.14,190.06) --
	(252.68,190.06) --
	(254.22,190.06) --
	(255.76,190.06) --
	(257.29,190.06) --
	(258.83,190.06) --
	(260.37,190.06) --
	(261.90,190.06) --
	(263.44,190.06) --
	(264.98,190.06) --
	(266.51,190.06) --
	(268.05,190.06) --
	(269.59,190.06) --
	(271.13,190.06) --
	(272.66,190.06) --
	(274.20,190.06) --
	(275.74,190.06) --
	(277.27,190.06) --
	(278.81,190.06) --
	(280.35,190.06) --
	(281.88,190.06) --
	(283.42,190.06) --
	(284.96,190.06) --
	(286.49,190.06) --
	(288.03,190.06) --
	(289.57,190.06) --
	(291.11,190.06) --
	(292.64,190.06) --
	(294.18,190.06) --
	(295.72,190.06) --
	(297.25,190.06) --
	(298.79,190.06) --
	(300.33,190.06) --
	(301.86,190.06) --
	(303.40,190.06) --
	(304.94,190.06) --
	(306.48,190.06) --
	(308.01,190.06) --
	(309.55,190.06) --
	(311.09,190.06) --
	(312.62,190.06) --
	(314.16,190.06) --
	(315.70,190.06) --
	(317.23,190.06) --
	(318.77,190.06) --
	(320.31,190.06) --
	(321.85,190.06) --
	(323.38,190.06) --
	(324.92,190.06) --
	(326.46,190.06) --
	(327.99,190.06) --
	(329.53,190.06) --
	(331.07,190.06) --
	(332.60,190.06) --
	(334.14,190.06) --
	(335.68,190.06) --
	(337.22,190.06) --
	(338.75,190.06) --
	(340.29,190.06) --
	(341.83,190.06) --
	(343.36,190.06) --
	(344.90,190.17) --
	(346.44,190.71) --
	(347.97,191.71) --
	(349.51,193.17) --
	(351.05,195.10) --
	(352.58,197.49) --
	(354.12,200.34) --
	(355.66,203.66) --
	(357.20,207.43) --
	(358.73,211.67) --
	(360.27,216.37) --
	(361.81,221.54) --
	(363.34,227.17) --
	(364.88,233.25) --
	(366.42,239.81) --
	(367.95,246.82);
\end{scope}
\begin{scope}
\path[clip] (209.80,104.16) rectangle (375.49,181.07);
\definecolor{drawColor}{gray}{0.92}

\path[draw=drawColor,line width= 0.3pt,line join=round] (209.80,117.69) --
	(375.49,117.69);

\path[draw=drawColor,line width= 0.3pt,line join=round] (209.80,137.77) --
	(375.49,137.77);

\path[draw=drawColor,line width= 0.3pt,line join=round] (209.80,157.84) --
	(375.49,157.84);

\path[draw=drawColor,line width= 0.3pt,line join=round] (209.80,177.91) --
	(375.49,177.91);

\path[draw=drawColor,line width= 0.3pt,line join=round] (235.01,104.16) --
	(235.01,181.07);

\path[draw=drawColor,line width= 0.3pt,line join=round] (273.43,104.16) --
	(273.43,181.07);

\path[draw=drawColor,line width= 0.3pt,line join=round] (311.85,104.16) --
	(311.85,181.07);

\path[draw=drawColor,line width= 0.3pt,line join=round] (350.28,104.16) --
	(350.28,181.07);

\path[draw=drawColor,line width= 0.6pt,line join=round] (209.80,107.66) --
	(375.49,107.66);

\path[draw=drawColor,line width= 0.6pt,line join=round] (209.80,127.73) --
	(375.49,127.73);

\path[draw=drawColor,line width= 0.6pt,line join=round] (209.80,147.80) --
	(375.49,147.80);

\path[draw=drawColor,line width= 0.6pt,line join=round] (209.80,167.88) --
	(375.49,167.88);

\path[draw=drawColor,line width= 0.6pt,line join=round] (215.79,104.16) --
	(215.79,181.07);

\path[draw=drawColor,line width= 0.6pt,line join=round] (254.22,104.16) --
	(254.22,181.07);

\path[draw=drawColor,line width= 0.6pt,line join=round] (292.64,104.16) --
	(292.64,181.07);

\path[draw=drawColor,line width= 0.6pt,line join=round] (331.07,104.16) --
	(331.07,181.07);

\path[draw=drawColor,line width= 0.6pt,line join=round] (369.49,104.16) --
	(369.49,181.07);
\definecolor{drawColor}{RGB}{155,38,176}

\path[draw=drawColor,line width= 1.1pt,line join=round] (217.33,143.13) --
	(218.87,138.75) --
	(220.40,134.65) --
	(221.94,130.85) --
	(223.48,127.33) --
	(225.02,124.10) --
	(226.55,121.16) --
	(228.09,118.51) --
	(229.63,116.15) --
	(231.16,114.08) --
	(232.70,112.30) --
	(234.24,110.81) --
	(235.77,109.60) --
	(237.31,108.69) --
	(238.85,108.06) --
	(240.39,107.72) --
	(241.92,107.66) --
	(243.46,107.66) --
	(245.00,107.66) --
	(246.53,107.66) --
	(248.07,107.66) --
	(249.61,107.66) --
	(251.14,107.66) --
	(252.68,107.66) --
	(254.22,107.66) --
	(255.76,107.66) --
	(257.29,107.66) --
	(258.83,107.66) --
	(260.37,107.66) --
	(261.90,107.66) --
	(263.44,107.66) --
	(264.98,107.66) --
	(266.51,107.66) --
	(268.05,107.66) --
	(269.59,107.66) --
	(271.13,107.66) --
	(272.66,107.66) --
	(274.20,107.66) --
	(275.74,107.66) --
	(277.27,107.66) --
	(278.81,107.66) --
	(280.35,107.66) --
	(281.88,107.66) --
	(283.42,107.66) --
	(284.96,107.66) --
	(286.49,107.66) --
	(288.03,107.66) --
	(289.57,107.66) --
	(291.11,107.66) --
	(292.64,107.66) --
	(294.18,107.66) --
	(295.72,107.66) --
	(297.25,107.66) --
	(298.79,107.66) --
	(300.33,107.66) --
	(301.86,107.66) --
	(303.40,107.66) --
	(304.94,107.66) --
	(306.48,107.66) --
	(308.01,107.66) --
	(309.55,107.66) --
	(311.09,107.66) --
	(312.62,107.66) --
	(314.16,107.66) --
	(315.70,107.66) --
	(317.23,107.66) --
	(318.77,107.66) --
	(320.31,107.66) --
	(321.85,107.66) --
	(323.38,107.66) --
	(324.92,107.66) --
	(326.46,107.66) --
	(327.99,107.66) --
	(329.53,107.66) --
	(331.07,107.66) --
	(332.60,107.66) --
	(334.14,107.66) --
	(335.68,107.66) --
	(337.22,107.66) --
	(338.75,107.66) --
	(340.29,107.66) --
	(341.83,107.66) --
	(343.36,107.66) --
	(344.90,107.66) --
	(346.44,107.66) --
	(347.97,107.66) --
	(349.51,107.66) --
	(351.05,107.66) --
	(352.58,107.66) --
	(354.12,107.66) --
	(355.66,107.66) --
	(357.20,107.66) --
	(358.73,107.66) --
	(360.27,107.66) --
	(361.81,107.66) --
	(363.34,107.66) --
	(364.88,107.66) --
	(366.42,107.66) --
	(367.95,107.66);
\definecolor{drawColor}{RGB}{63,81,180}

\path[draw=drawColor,line width= 1.1pt,line join=round] (217.33,152.33) --
	(218.87,156.28) --
	(220.40,159.66) --
	(221.94,162.45) --
	(223.48,164.67) --
	(225.02,166.30) --
	(226.55,167.36) --
	(228.09,167.85) --
	(229.63,167.75) --
	(231.16,167.08) --
	(232.70,165.82) --
	(234.24,163.99) --
	(235.77,161.58) --
	(237.31,158.60) --
	(238.85,155.03) --
	(240.39,150.89) --
	(241.92,146.21) --
	(243.46,141.64) --
	(245.00,137.35) --
	(246.53,133.35) --
	(248.07,129.64) --
	(249.61,126.22) --
	(251.14,123.09) --
	(252.68,120.25) --
	(254.22,117.69) --
	(255.76,115.43) --
	(257.29,113.46) --
	(258.83,111.77) --
	(260.37,110.37) --
	(261.90,109.26) --
	(263.44,108.45) --
	(264.98,107.92) --
	(266.51,107.67) --
	(268.05,107.66) --
	(269.59,107.66) --
	(271.13,107.66) --
	(272.66,107.66) --
	(274.20,107.66) --
	(275.74,107.66) --
	(277.27,107.66) --
	(278.81,107.66) --
	(280.35,107.66) --
	(281.88,107.66) --
	(283.42,107.66) --
	(284.96,107.66) --
	(286.49,107.66) --
	(288.03,107.66) --
	(289.57,107.66) --
	(291.11,107.66) --
	(292.64,107.66) --
	(294.18,107.66) --
	(295.72,107.66) --
	(297.25,107.66) --
	(298.79,107.66) --
	(300.33,107.66) --
	(301.86,107.66) --
	(303.40,107.66) --
	(304.94,107.66) --
	(306.48,107.66) --
	(308.01,107.66) --
	(309.55,107.66) --
	(311.09,107.66) --
	(312.62,107.66) --
	(314.16,107.66) --
	(315.70,107.66) --
	(317.23,107.66) --
	(318.77,107.66) --
	(320.31,107.66) --
	(321.85,107.66) --
	(323.38,107.66) --
	(324.92,107.66) --
	(326.46,107.66) --
	(327.99,107.66) --
	(329.53,107.66) --
	(331.07,107.66) --
	(332.60,107.66) --
	(334.14,107.66) --
	(335.68,107.66) --
	(337.22,107.66) --
	(338.75,107.66) --
	(340.29,107.66) --
	(341.83,107.66) --
	(343.36,107.66) --
	(344.90,107.66) --
	(346.44,107.66) --
	(347.97,107.66) --
	(349.51,107.66) --
	(351.05,107.66) --
	(352.58,107.66) --
	(354.12,107.66) --
	(355.66,107.66) --
	(357.20,107.66) --
	(358.73,107.66) --
	(360.27,107.66) --
	(361.81,107.66) --
	(363.34,107.66) --
	(364.88,107.66) --
	(366.42,107.66) --
	(367.95,107.66);
\definecolor{drawColor}{RGB}{2,169,243}

\path[draw=drawColor,line width= 1.1pt,line join=round] (217.33,107.80) --
	(218.87,108.24) --
	(220.40,108.96) --
	(221.94,109.97) --
	(223.48,111.27) --
	(225.02,112.86) --
	(226.55,114.74) --
	(228.09,116.91) --
	(229.63,119.36) --
	(231.16,122.11) --
	(232.70,125.15) --
	(234.24,128.47) --
	(235.77,132.08) --
	(237.31,135.99) --
	(238.85,140.18) --
	(240.39,144.66) --
	(241.92,149.38) --
	(243.46,153.71) --
	(245.00,157.47) --
	(246.53,160.65) --
	(248.07,163.25) --
	(249.61,165.28) --
	(251.14,166.72) --
	(252.68,167.59) --
	(254.22,167.88) --
	(255.76,167.59) --
	(257.29,166.72) --
	(258.83,165.28) --
	(260.37,163.25) --
	(261.90,160.65) --
	(263.44,157.47) --
	(264.98,153.71) --
	(266.51,149.38) --
	(268.05,144.66) --
	(269.59,140.18) --
	(271.13,135.99) --
	(272.66,132.08) --
	(274.20,128.47) --
	(275.74,125.15) --
	(277.27,122.11) --
	(278.81,119.36) --
	(280.35,116.91) --
	(281.88,114.74) --
	(283.42,112.86) --
	(284.96,111.27) --
	(286.49,109.97) --
	(288.03,108.96) --
	(289.57,108.24) --
	(291.11,107.80) --
	(292.64,107.66) --
	(294.18,107.66) --
	(295.72,107.66) --
	(297.25,107.66) --
	(298.79,107.66) --
	(300.33,107.66) --
	(301.86,107.66) --
	(303.40,107.66) --
	(304.94,107.66) --
	(306.48,107.66) --
	(308.01,107.66) --
	(309.55,107.66) --
	(311.09,107.66) --
	(312.62,107.66) --
	(314.16,107.66) --
	(315.70,107.66) --
	(317.23,107.66) --
	(318.77,107.66) --
	(320.31,107.66) --
	(321.85,107.66) --
	(323.38,107.66) --
	(324.92,107.66) --
	(326.46,107.66) --
	(327.99,107.66) --
	(329.53,107.66) --
	(331.07,107.66) --
	(332.60,107.66) --
	(334.14,107.66) --
	(335.68,107.66) --
	(337.22,107.66) --
	(338.75,107.66) --
	(340.29,107.66) --
	(341.83,107.66) --
	(343.36,107.66) --
	(344.90,107.66) --
	(346.44,107.66) --
	(347.97,107.66) --
	(349.51,107.66) --
	(351.05,107.66) --
	(352.58,107.66) --
	(354.12,107.66) --
	(355.66,107.66) --
	(357.20,107.66) --
	(358.73,107.66) --
	(360.27,107.66) --
	(361.81,107.66) --
	(363.34,107.66) --
	(364.88,107.66) --
	(366.42,107.66) --
	(367.95,107.66);
\definecolor{drawColor}{RGB}{0,150,135}

\path[draw=drawColor,line width= 1.1pt,line join=round] (217.33,107.66) --
	(218.87,107.66) --
	(220.40,107.66) --
	(221.94,107.66) --
	(223.48,107.66) --
	(225.02,107.66) --
	(226.55,107.66) --
	(228.09,107.66) --
	(229.63,107.66) --
	(231.16,107.66) --
	(232.70,107.66) --
	(234.24,107.66) --
	(235.77,107.66) --
	(237.31,107.66) --
	(238.85,107.66) --
	(240.39,107.66) --
	(241.92,107.67) --
	(243.46,107.92) --
	(245.00,108.45) --
	(246.53,109.26) --
	(248.07,110.37) --
	(249.61,111.77) --
	(251.14,113.46) --
	(252.68,115.43) --
	(254.22,117.69) --
	(255.76,120.25) --
	(257.29,123.09) --
	(258.83,126.22) --
	(260.37,129.64) --
	(261.90,133.35) --
	(263.44,137.35) --
	(264.98,141.64) --
	(266.51,146.21) --
	(268.05,150.89) --
	(269.59,155.03) --
	(271.13,158.60) --
	(272.66,161.58) --
	(274.20,163.99) --
	(275.74,165.82) --
	(277.27,167.08) --
	(278.81,167.75) --
	(280.35,167.85) --
	(281.88,167.36) --
	(283.42,166.30) --
	(284.96,164.67) --
	(286.49,162.45) --
	(288.03,159.66) --
	(289.57,156.28) --
	(291.11,152.33) --
	(292.64,147.80) --
	(294.18,143.13) --
	(295.72,138.75) --
	(297.25,134.65) --
	(298.79,130.85) --
	(300.33,127.33) --
	(301.86,124.10) --
	(303.40,121.16) --
	(304.94,118.51) --
	(306.48,116.15) --
	(308.01,114.08) --
	(309.55,112.30) --
	(311.09,110.81) --
	(312.62,109.60) --
	(314.16,108.69) --
	(315.70,108.06) --
	(317.23,107.72) --
	(318.77,107.66) --
	(320.31,107.66) --
	(321.85,107.66) --
	(323.38,107.66) --
	(324.92,107.66) --
	(326.46,107.66) --
	(327.99,107.66) --
	(329.53,107.66) --
	(331.07,107.66) --
	(332.60,107.66) --
	(334.14,107.66) --
	(335.68,107.66) --
	(337.22,107.66) --
	(338.75,107.66) --
	(340.29,107.66) --
	(341.83,107.66) --
	(343.36,107.66) --
	(344.90,107.66) --
	(346.44,107.66) --
	(347.97,107.66) --
	(349.51,107.66) --
	(351.05,107.66) --
	(352.58,107.66) --
	(354.12,107.66) --
	(355.66,107.66) --
	(357.20,107.66) --
	(358.73,107.66) --
	(360.27,107.66) --
	(361.81,107.66) --
	(363.34,107.66) --
	(364.88,107.66) --
	(366.42,107.66) --
	(367.95,107.66);
\definecolor{drawColor}{RGB}{139,195,74}

\path[draw=drawColor,line width= 1.1pt,line join=round] (217.33,107.66) --
	(218.87,107.66) --
	(220.40,107.66) --
	(221.94,107.66) --
	(223.48,107.66) --
	(225.02,107.66) --
	(226.55,107.66) --
	(228.09,107.66) --
	(229.63,107.66) --
	(231.16,107.66) --
	(232.70,107.66) --
	(234.24,107.66) --
	(235.77,107.66) --
	(237.31,107.66) --
	(238.85,107.66) --
	(240.39,107.66) --
	(241.92,107.66) --
	(243.46,107.66) --
	(245.00,107.66) --
	(246.53,107.66) --
	(248.07,107.66) --
	(249.61,107.66) --
	(251.14,107.66) --
	(252.68,107.66) --
	(254.22,107.66) --
	(255.76,107.66) --
	(257.29,107.66) --
	(258.83,107.66) --
	(260.37,107.66) --
	(261.90,107.66) --
	(263.44,107.66) --
	(264.98,107.66) --
	(266.51,107.66) --
	(268.05,107.72) --
	(269.59,108.06) --
	(271.13,108.69) --
	(272.66,109.60) --
	(274.20,110.81) --
	(275.74,112.30) --
	(277.27,114.08) --
	(278.81,116.15) --
	(280.35,118.51) --
	(281.88,121.16) --
	(283.42,124.10) --
	(284.96,127.33) --
	(286.49,130.85) --
	(288.03,134.65) --
	(289.57,138.75) --
	(291.11,143.13) --
	(292.64,147.80) --
	(294.18,152.33) --
	(295.72,156.28) --
	(297.25,159.66) --
	(298.79,162.45) --
	(300.33,164.67) --
	(301.86,166.30) --
	(303.40,167.36) --
	(304.94,167.85) --
	(306.48,167.75) --
	(308.01,167.08) --
	(309.55,165.82) --
	(311.09,163.99) --
	(312.62,161.58) --
	(314.16,158.60) --
	(315.70,155.03) --
	(317.23,150.89) --
	(318.77,146.21) --
	(320.31,141.64) --
	(321.85,137.35) --
	(323.38,133.35) --
	(324.92,129.64) --
	(326.46,126.22) --
	(327.99,123.09) --
	(329.53,120.25) --
	(331.07,117.69) --
	(332.60,115.43) --
	(334.14,113.46) --
	(335.68,111.77) --
	(337.22,110.37) --
	(338.75,109.26) --
	(340.29,108.45) --
	(341.83,107.92) --
	(343.36,107.67) --
	(344.90,107.66) --
	(346.44,107.66) --
	(347.97,107.66) --
	(349.51,107.66) --
	(351.05,107.66) --
	(352.58,107.66) --
	(354.12,107.66) --
	(355.66,107.66) --
	(357.20,107.66) --
	(358.73,107.66) --
	(360.27,107.66) --
	(361.81,107.66) --
	(363.34,107.66) --
	(364.88,107.66) --
	(366.42,107.66) --
	(367.95,107.66);
\definecolor{drawColor}{RGB}{255,235,58}

\path[draw=drawColor,line width= 1.1pt,line join=round] (217.33,107.66) --
	(218.87,107.66) --
	(220.40,107.66) --
	(221.94,107.66) --
	(223.48,107.66) --
	(225.02,107.66) --
	(226.55,107.66) --
	(228.09,107.66) --
	(229.63,107.66) --
	(231.16,107.66) --
	(232.70,107.66) --
	(234.24,107.66) --
	(235.77,107.66) --
	(237.31,107.66) --
	(238.85,107.66) --
	(240.39,107.66) --
	(241.92,107.66) --
	(243.46,107.66) --
	(245.00,107.66) --
	(246.53,107.66) --
	(248.07,107.66) --
	(249.61,107.66) --
	(251.14,107.66) --
	(252.68,107.66) --
	(254.22,107.66) --
	(255.76,107.66) --
	(257.29,107.66) --
	(258.83,107.66) --
	(260.37,107.66) --
	(261.90,107.66) --
	(263.44,107.66) --
	(264.98,107.66) --
	(266.51,107.66) --
	(268.05,107.66) --
	(269.59,107.66) --
	(271.13,107.66) --
	(272.66,107.66) --
	(274.20,107.66) --
	(275.74,107.66) --
	(277.27,107.66) --
	(278.81,107.66) --
	(280.35,107.66) --
	(281.88,107.66) --
	(283.42,107.66) --
	(284.96,107.66) --
	(286.49,107.66) --
	(288.03,107.66) --
	(289.57,107.66) --
	(291.11,107.66) --
	(292.64,107.66) --
	(294.18,107.80) --
	(295.72,108.24) --
	(297.25,108.96) --
	(298.79,109.97) --
	(300.33,111.27) --
	(301.86,112.86) --
	(303.40,114.74) --
	(304.94,116.91) --
	(306.48,119.36) --
	(308.01,122.11) --
	(309.55,125.15) --
	(311.09,128.47) --
	(312.62,132.08) --
	(314.16,135.99) --
	(315.70,140.18) --
	(317.23,144.66) --
	(318.77,149.38) --
	(320.31,153.71) --
	(321.85,157.47) --
	(323.38,160.65) --
	(324.92,163.25) --
	(326.46,165.28) --
	(327.99,166.72) --
	(329.53,167.59) --
	(331.07,167.88) --
	(332.60,167.59) --
	(334.14,166.72) --
	(335.68,165.28) --
	(337.22,163.25) --
	(338.75,160.65) --
	(340.29,157.47) --
	(341.83,153.71) --
	(343.36,149.38) --
	(344.90,144.66) --
	(346.44,140.18) --
	(347.97,135.99) --
	(349.51,132.08) --
	(351.05,128.47) --
	(352.58,125.15) --
	(354.12,122.11) --
	(355.66,119.36) --
	(357.20,116.91) --
	(358.73,114.74) --
	(360.27,112.86) --
	(361.81,111.27) --
	(363.34,109.97) --
	(364.88,108.96) --
	(366.42,108.24) --
	(367.95,107.80);
\definecolor{drawColor}{RGB}{255,152,0}

\path[draw=drawColor,line width= 1.1pt,line join=round] (217.33,107.66) --
	(218.87,107.66) --
	(220.40,107.66) --
	(221.94,107.66) --
	(223.48,107.66) --
	(225.02,107.66) --
	(226.55,107.66) --
	(228.09,107.66) --
	(229.63,107.66) --
	(231.16,107.66) --
	(232.70,107.66) --
	(234.24,107.66) --
	(235.77,107.66) --
	(237.31,107.66) --
	(238.85,107.66) --
	(240.39,107.66) --
	(241.92,107.66) --
	(243.46,107.66) --
	(245.00,107.66) --
	(246.53,107.66) --
	(248.07,107.66) --
	(249.61,107.66) --
	(251.14,107.66) --
	(252.68,107.66) --
	(254.22,107.66) --
	(255.76,107.66) --
	(257.29,107.66) --
	(258.83,107.66) --
	(260.37,107.66) --
	(261.90,107.66) --
	(263.44,107.66) --
	(264.98,107.66) --
	(266.51,107.66) --
	(268.05,107.66) --
	(269.59,107.66) --
	(271.13,107.66) --
	(272.66,107.66) --
	(274.20,107.66) --
	(275.74,107.66) --
	(277.27,107.66) --
	(278.81,107.66) --
	(280.35,107.66) --
	(281.88,107.66) --
	(283.42,107.66) --
	(284.96,107.66) --
	(286.49,107.66) --
	(288.03,107.66) --
	(289.57,107.66) --
	(291.11,107.66) --
	(292.64,107.66) --
	(294.18,107.66) --
	(295.72,107.66) --
	(297.25,107.66) --
	(298.79,107.66) --
	(300.33,107.66) --
	(301.86,107.66) --
	(303.40,107.66) --
	(304.94,107.66) --
	(306.48,107.66) --
	(308.01,107.66) --
	(309.55,107.66) --
	(311.09,107.66) --
	(312.62,107.66) --
	(314.16,107.66) --
	(315.70,107.66) --
	(317.23,107.66) --
	(318.77,107.67) --
	(320.31,107.92) --
	(321.85,108.45) --
	(323.38,109.26) --
	(324.92,110.37) --
	(326.46,111.77) --
	(327.99,113.46) --
	(329.53,115.43) --
	(331.07,117.69) --
	(332.60,120.25) --
	(334.14,123.09) --
	(335.68,126.22) --
	(337.22,129.64) --
	(338.75,133.35) --
	(340.29,137.35) --
	(341.83,141.64) --
	(343.36,146.21) --
	(344.90,150.89) --
	(346.44,155.03) --
	(347.97,158.60) --
	(349.51,161.58) --
	(351.05,163.99) --
	(352.58,165.82) --
	(354.12,167.08) --
	(355.66,167.75) --
	(357.20,167.85) --
	(358.73,167.36) --
	(360.27,166.30) --
	(361.81,164.67) --
	(363.34,162.45) --
	(364.88,159.66) --
	(366.42,156.28) --
	(367.95,152.33);
\definecolor{drawColor}{RGB}{121,84,71}

\path[draw=drawColor,line width= 1.1pt,line join=round] (217.33,107.66) --
	(218.87,107.66) --
	(220.40,107.66) --
	(221.94,107.66) --
	(223.48,107.66) --
	(225.02,107.66) --
	(226.55,107.66) --
	(228.09,107.66) --
	(229.63,107.66) --
	(231.16,107.66) --
	(232.70,107.66) --
	(234.24,107.66) --
	(235.77,107.66) --
	(237.31,107.66) --
	(238.85,107.66) --
	(240.39,107.66) --
	(241.92,107.66) --
	(243.46,107.66) --
	(245.00,107.66) --
	(246.53,107.66) --
	(248.07,107.66) --
	(249.61,107.66) --
	(251.14,107.66) --
	(252.68,107.66) --
	(254.22,107.66) --
	(255.76,107.66) --
	(257.29,107.66) --
	(258.83,107.66) --
	(260.37,107.66) --
	(261.90,107.66) --
	(263.44,107.66) --
	(264.98,107.66) --
	(266.51,107.66) --
	(268.05,107.66) --
	(269.59,107.66) --
	(271.13,107.66) --
	(272.66,107.66) --
	(274.20,107.66) --
	(275.74,107.66) --
	(277.27,107.66) --
	(278.81,107.66) --
	(280.35,107.66) --
	(281.88,107.66) --
	(283.42,107.66) --
	(284.96,107.66) --
	(286.49,107.66) --
	(288.03,107.66) --
	(289.57,107.66) --
	(291.11,107.66) --
	(292.64,107.66) --
	(294.18,107.66) --
	(295.72,107.66) --
	(297.25,107.66) --
	(298.79,107.66) --
	(300.33,107.66) --
	(301.86,107.66) --
	(303.40,107.66) --
	(304.94,107.66) --
	(306.48,107.66) --
	(308.01,107.66) --
	(309.55,107.66) --
	(311.09,107.66) --
	(312.62,107.66) --
	(314.16,107.66) --
	(315.70,107.66) --
	(317.23,107.66) --
	(318.77,107.66) --
	(320.31,107.66) --
	(321.85,107.66) --
	(323.38,107.66) --
	(324.92,107.66) --
	(326.46,107.66) --
	(327.99,107.66) --
	(329.53,107.66) --
	(331.07,107.66) --
	(332.60,107.66) --
	(334.14,107.66) --
	(335.68,107.66) --
	(337.22,107.66) --
	(338.75,107.66) --
	(340.29,107.66) --
	(341.83,107.66) --
	(343.36,107.66) --
	(344.90,107.72) --
	(346.44,108.06) --
	(347.97,108.69) --
	(349.51,109.60) --
	(351.05,110.81) --
	(352.58,112.30) --
	(354.12,114.08) --
	(355.66,116.15) --
	(357.20,118.51) --
	(358.73,121.16) --
	(360.27,124.10) --
	(361.81,127.33) --
	(363.34,130.85) --
	(364.88,134.65) --
	(366.42,138.75) --
	(367.95,143.13);
\end{scope}
\begin{scope}
\path[clip] (209.80, 21.76) rectangle (375.49, 98.66);
\definecolor{drawColor}{gray}{0.92}

\path[draw=drawColor,line width= 0.3pt,line join=round] (209.80, 35.29) --
	(375.49, 35.29);

\path[draw=drawColor,line width= 0.3pt,line join=round] (209.80, 55.36) --
	(375.49, 55.36);

\path[draw=drawColor,line width= 0.3pt,line join=round] (209.80, 75.44) --
	(375.49, 75.44);

\path[draw=drawColor,line width= 0.3pt,line join=round] (209.80, 95.51) --
	(375.49, 95.51);

\path[draw=drawColor,line width= 0.3pt,line join=round] (235.01, 21.76) --
	(235.01, 98.66);

\path[draw=drawColor,line width= 0.3pt,line join=round] (273.43, 21.76) --
	(273.43, 98.66);

\path[draw=drawColor,line width= 0.3pt,line join=round] (311.85, 21.76) --
	(311.85, 98.66);

\path[draw=drawColor,line width= 0.3pt,line join=round] (350.28, 21.76) --
	(350.28, 98.66);

\path[draw=drawColor,line width= 0.6pt,line join=round] (209.80, 25.25) --
	(375.49, 25.25);

\path[draw=drawColor,line width= 0.6pt,line join=round] (209.80, 45.33) --
	(375.49, 45.33);

\path[draw=drawColor,line width= 0.6pt,line join=round] (209.80, 65.40) --
	(375.49, 65.40);

\path[draw=drawColor,line width= 0.6pt,line join=round] (209.80, 85.47) --
	(375.49, 85.47);

\path[draw=drawColor,line width= 0.6pt,line join=round] (215.79, 21.76) --
	(215.79, 98.66);

\path[draw=drawColor,line width= 0.6pt,line join=round] (254.22, 21.76) --
	(254.22, 98.66);

\path[draw=drawColor,line width= 0.6pt,line join=round] (292.64, 21.76) --
	(292.64, 98.66);

\path[draw=drawColor,line width= 0.6pt,line join=round] (331.07, 21.76) --
	(331.07, 98.66);

\path[draw=drawColor,line width= 0.6pt,line join=round] (369.49, 21.76) --
	(369.49, 98.66);
\definecolor{drawColor}{RGB}{155,38,176}

\path[draw=drawColor,line width= 1.1pt,line join=round] (217.33, 39.44) --
	(218.87, 37.69) --
	(220.40, 36.05) --
	(221.94, 34.53) --
	(223.48, 33.12) --
	(225.02, 31.83) --
	(226.55, 30.66) --
	(228.09, 29.60) --
	(229.63, 28.65) --
	(231.16, 27.82) --
	(232.70, 27.11) --
	(234.24, 26.51) --
	(235.77, 26.03) --
	(237.31, 25.66) --
	(238.85, 25.41) --
	(240.39, 25.28) --
	(241.92, 25.25) --
	(243.46, 25.25) --
	(245.00, 25.25) --
	(246.53, 25.25) --
	(248.07, 25.25) --
	(249.61, 25.25) --
	(251.14, 25.25) --
	(252.68, 25.25) --
	(254.22, 25.25) --
	(255.76, 25.25) --
	(257.29, 25.25) --
	(258.83, 25.25) --
	(260.37, 25.25) --
	(261.90, 25.25) --
	(263.44, 25.25) --
	(264.98, 25.25) --
	(266.51, 25.25) --
	(268.05, 25.25) --
	(269.59, 25.25) --
	(271.13, 25.25) --
	(272.66, 25.25) --
	(274.20, 25.25) --
	(275.74, 25.25) --
	(277.27, 25.25) --
	(278.81, 25.25) --
	(280.35, 25.25) --
	(281.88, 25.25) --
	(283.42, 25.25) --
	(284.96, 25.25) --
	(286.49, 25.25) --
	(288.03, 25.25) --
	(289.57, 25.25) --
	(291.11, 25.25) --
	(292.64, 25.25) --
	(294.18, 25.25) --
	(295.72, 25.25) --
	(297.25, 25.25) --
	(298.79, 25.25) --
	(300.33, 25.25) --
	(301.86, 25.25) --
	(303.40, 25.25) --
	(304.94, 25.25) --
	(306.48, 25.25) --
	(308.01, 25.25) --
	(309.55, 25.25) --
	(311.09, 25.25) --
	(312.62, 25.25) --
	(314.16, 25.25) --
	(315.70, 25.25) --
	(317.23, 25.25) --
	(318.77, 25.25) --
	(320.31, 25.25) --
	(321.85, 25.25) --
	(323.38, 25.25) --
	(324.92, 25.25) --
	(326.46, 25.25) --
	(327.99, 25.25) --
	(329.53, 25.25) --
	(331.07, 25.25) --
	(332.60, 25.25) --
	(334.14, 25.25) --
	(335.68, 25.25) --
	(337.22, 25.25) --
	(338.75, 25.25) --
	(340.29, 25.25) --
	(341.83, 25.25) --
	(343.36, 25.25) --
	(344.90, 25.25) --
	(346.44, 25.25) --
	(347.97, 25.25) --
	(349.51, 25.25) --
	(351.05, 25.25) --
	(352.58, 25.25) --
	(354.12, 25.25) --
	(355.66, 25.25) --
	(357.20, 25.25) --
	(358.73, 25.25) --
	(360.27, 25.25) --
	(361.81, 25.25) --
	(363.34, 25.25) --
	(364.88, 25.25) --
	(366.42, 25.25) --
	(367.95, 25.25);
\definecolor{drawColor}{RGB}{63,81,180}

\path[draw=drawColor,line width= 1.1pt,line join=round] (217.33, 91.21) --
	(218.87, 92.53) --
	(220.40, 93.45) --
	(221.94, 93.96) --
	(223.48, 94.06) --
	(225.02, 93.77) --
	(226.55, 93.06) --
	(228.09, 91.95) --
	(229.63, 90.44) --
	(231.16, 88.52) --
	(232.70, 86.20) --
	(234.24, 83.48) --
	(235.77, 80.34) --
	(237.31, 76.81) --
	(238.85, 72.87) --
	(240.39, 68.52) --
	(241.92, 63.81) --
	(243.46, 59.23) --
	(245.00, 54.95) --
	(246.53, 50.95) --
	(248.07, 47.24) --
	(249.61, 43.82) --
	(251.14, 40.69) --
	(252.68, 37.84) --
	(254.22, 35.29) --
	(255.76, 33.03) --
	(257.29, 31.05) --
	(258.83, 29.36) --
	(260.37, 27.97) --
	(261.90, 26.86) --
	(263.44, 26.04) --
	(264.98, 25.51) --
	(266.51, 25.27) --
	(268.05, 25.25) --
	(269.59, 25.25) --
	(271.13, 25.25) --
	(272.66, 25.25) --
	(274.20, 25.25) --
	(275.74, 25.25) --
	(277.27, 25.25) --
	(278.81, 25.25) --
	(280.35, 25.25) --
	(281.88, 25.25) --
	(283.42, 25.25) --
	(284.96, 25.25) --
	(286.49, 25.25) --
	(288.03, 25.25) --
	(289.57, 25.25) --
	(291.11, 25.25) --
	(292.64, 25.25) --
	(294.18, 25.25) --
	(295.72, 25.25) --
	(297.25, 25.25) --
	(298.79, 25.25) --
	(300.33, 25.25) --
	(301.86, 25.25) --
	(303.40, 25.25) --
	(304.94, 25.25) --
	(306.48, 25.25) --
	(308.01, 25.25) --
	(309.55, 25.25) --
	(311.09, 25.25) --
	(312.62, 25.25) --
	(314.16, 25.25) --
	(315.70, 25.25) --
	(317.23, 25.25) --
	(318.77, 25.25) --
	(320.31, 25.25) --
	(321.85, 25.25) --
	(323.38, 25.25) --
	(324.92, 25.25) --
	(326.46, 25.25) --
	(327.99, 25.25) --
	(329.53, 25.25) --
	(331.07, 25.25) --
	(332.60, 25.25) --
	(334.14, 25.25) --
	(335.68, 25.25) --
	(337.22, 25.25) --
	(338.75, 25.25) --
	(340.29, 25.25) --
	(341.83, 25.25) --
	(343.36, 25.25) --
	(344.90, 25.25) --
	(346.44, 25.25) --
	(347.97, 25.25) --
	(349.51, 25.25) --
	(351.05, 25.25) --
	(352.58, 25.25) --
	(354.12, 25.25) --
	(355.66, 25.25) --
	(357.20, 25.25) --
	(358.73, 25.25) --
	(360.27, 25.25) --
	(361.81, 25.25) --
	(363.34, 25.25) --
	(364.88, 25.25) --
	(366.42, 25.25) --
	(367.95, 25.25);
\definecolor{drawColor}{RGB}{2,169,243}

\path[draw=drawColor,line width= 1.1pt,line join=round] (217.33, 25.40) --
	(218.87, 25.83) --
	(220.40, 26.55) --
	(221.94, 27.57) --
	(223.48, 28.87) --
	(225.02, 30.46) --
	(226.55, 32.33) --
	(228.09, 34.50) --
	(229.63, 36.96) --
	(231.16, 39.71) --
	(232.70, 42.74) --
	(234.24, 46.07) --
	(235.77, 49.68) --
	(237.31, 53.58) --
	(238.85, 57.77) --
	(240.39, 62.25) --
	(241.92, 66.97) --
	(243.46, 71.31) --
	(245.00, 75.07) --
	(246.53, 78.25) --
	(248.07, 80.85) --
	(249.61, 82.87) --
	(251.14, 84.32) --
	(252.68, 85.18) --
	(254.22, 85.47) --
	(255.76, 85.18) --
	(257.29, 84.32) --
	(258.83, 82.87) --
	(260.37, 80.85) --
	(261.90, 78.25) --
	(263.44, 75.07) --
	(264.98, 71.31) --
	(266.51, 66.97) --
	(268.05, 62.25) --
	(269.59, 57.77) --
	(271.13, 53.58) --
	(272.66, 49.68) --
	(274.20, 46.07) --
	(275.74, 42.74) --
	(277.27, 39.71) --
	(278.81, 36.96) --
	(280.35, 34.50) --
	(281.88, 32.33) --
	(283.42, 30.46) --
	(284.96, 28.87) --
	(286.49, 27.57) --
	(288.03, 26.55) --
	(289.57, 25.83) --
	(291.11, 25.40) --
	(292.64, 25.25) --
	(294.18, 25.25) --
	(295.72, 25.25) --
	(297.25, 25.25) --
	(298.79, 25.25) --
	(300.33, 25.25) --
	(301.86, 25.25) --
	(303.40, 25.25) --
	(304.94, 25.25) --
	(306.48, 25.25) --
	(308.01, 25.25) --
	(309.55, 25.25) --
	(311.09, 25.25) --
	(312.62, 25.25) --
	(314.16, 25.25) --
	(315.70, 25.25) --
	(317.23, 25.25) --
	(318.77, 25.25) --
	(320.31, 25.25) --
	(321.85, 25.25) --
	(323.38, 25.25) --
	(324.92, 25.25) --
	(326.46, 25.25) --
	(327.99, 25.25) --
	(329.53, 25.25) --
	(331.07, 25.25) --
	(332.60, 25.25) --
	(334.14, 25.25) --
	(335.68, 25.25) --
	(337.22, 25.25) --
	(338.75, 25.25) --
	(340.29, 25.25) --
	(341.83, 25.25) --
	(343.36, 25.25) --
	(344.90, 25.25) --
	(346.44, 25.25) --
	(347.97, 25.25) --
	(349.51, 25.25) --
	(351.05, 25.25) --
	(352.58, 25.25) --
	(354.12, 25.25) --
	(355.66, 25.25) --
	(357.20, 25.25) --
	(358.73, 25.25) --
	(360.27, 25.25) --
	(361.81, 25.25) --
	(363.34, 25.25) --
	(364.88, 25.25) --
	(366.42, 25.25) --
	(367.95, 25.25);
\definecolor{drawColor}{RGB}{0,150,135}

\path[draw=drawColor,line width= 1.1pt,line join=round] (217.33, 25.25) --
	(218.87, 25.25) --
	(220.40, 25.25) --
	(221.94, 25.25) --
	(223.48, 25.25) --
	(225.02, 25.25) --
	(226.55, 25.25) --
	(228.09, 25.25) --
	(229.63, 25.25) --
	(231.16, 25.25) --
	(232.70, 25.25) --
	(234.24, 25.25) --
	(235.77, 25.25) --
	(237.31, 25.25) --
	(238.85, 25.25) --
	(240.39, 25.25) --
	(241.92, 25.27) --
	(243.46, 25.51) --
	(245.00, 26.04) --
	(246.53, 26.86) --
	(248.07, 27.97) --
	(249.61, 29.36) --
	(251.14, 31.05) --
	(252.68, 33.03) --
	(254.22, 35.29) --
	(255.76, 37.84) --
	(257.29, 40.69) --
	(258.83, 43.82) --
	(260.37, 47.24) --
	(261.90, 50.95) --
	(263.44, 54.95) --
	(264.98, 59.23) --
	(266.51, 63.81) --
	(268.05, 68.48) --
	(269.59, 72.63) --
	(271.13, 76.19) --
	(272.66, 79.18) --
	(274.20, 81.59) --
	(275.74, 83.42) --
	(277.27, 84.67) --
	(278.81, 85.34) --
	(280.35, 85.44) --
	(281.88, 84.96) --
	(283.42, 83.90) --
	(284.96, 82.26) --
	(286.49, 80.05) --
	(288.03, 77.25) --
	(289.57, 73.88) --
	(291.11, 69.93) --
	(292.64, 65.40) --
	(294.18, 60.73) --
	(295.72, 56.34) --
	(297.25, 52.25) --
	(298.79, 48.44) --
	(300.33, 44.92) --
	(301.86, 41.70) --
	(303.40, 38.76) --
	(304.94, 36.11) --
	(306.48, 33.75) --
	(308.01, 31.68) --
	(309.55, 29.89) --
	(311.09, 28.40) --
	(312.62, 27.20) --
	(314.16, 26.28) --
	(315.70, 25.65) --
	(317.23, 25.32) --
	(318.77, 25.25) --
	(320.31, 25.25) --
	(321.85, 25.25) --
	(323.38, 25.25) --
	(324.92, 25.25) --
	(326.46, 25.25) --
	(327.99, 25.25) --
	(329.53, 25.25) --
	(331.07, 25.25) --
	(332.60, 25.25) --
	(334.14, 25.25) --
	(335.68, 25.25) --
	(337.22, 25.25) --
	(338.75, 25.25) --
	(340.29, 25.25) --
	(341.83, 25.25) --
	(343.36, 25.25) --
	(344.90, 25.25) --
	(346.44, 25.25) --
	(347.97, 25.25) --
	(349.51, 25.25) --
	(351.05, 25.25) --
	(352.58, 25.25) --
	(354.12, 25.25) --
	(355.66, 25.25) --
	(357.20, 25.25) --
	(358.73, 25.25) --
	(360.27, 25.25) --
	(361.81, 25.25) --
	(363.34, 25.25) --
	(364.88, 25.25) --
	(366.42, 25.25) --
	(367.95, 25.25);
\definecolor{drawColor}{RGB}{139,195,74}

\path[draw=drawColor,line width= 1.1pt,line join=round] (217.33, 25.25) --
	(218.87, 25.25) --
	(220.40, 25.25) --
	(221.94, 25.25) --
	(223.48, 25.25) --
	(225.02, 25.25) --
	(226.55, 25.25) --
	(228.09, 25.25) --
	(229.63, 25.25) --
	(231.16, 25.25) --
	(232.70, 25.25) --
	(234.24, 25.25) --
	(235.77, 25.25) --
	(237.31, 25.25) --
	(238.85, 25.25) --
	(240.39, 25.25) --
	(241.92, 25.25) --
	(243.46, 25.25) --
	(245.00, 25.25) --
	(246.53, 25.25) --
	(248.07, 25.25) --
	(249.61, 25.25) --
	(251.14, 25.25) --
	(252.68, 25.25) --
	(254.22, 25.25) --
	(255.76, 25.25) --
	(257.29, 25.25) --
	(258.83, 25.25) --
	(260.37, 25.25) --
	(261.90, 25.25) --
	(263.44, 25.25) --
	(264.98, 25.25) --
	(266.51, 25.25) --
	(268.05, 25.32) --
	(269.59, 25.65) --
	(271.13, 26.28) --
	(272.66, 27.20) --
	(274.20, 28.40) --
	(275.74, 29.89) --
	(277.27, 31.68) --
	(278.81, 33.75) --
	(280.35, 36.11) --
	(281.88, 38.76) --
	(283.42, 41.70) --
	(284.96, 44.92) --
	(286.49, 48.44) --
	(288.03, 52.25) --
	(289.57, 56.34) --
	(291.11, 60.73) --
	(292.64, 65.40) --
	(294.18, 69.93) --
	(295.72, 73.88) --
	(297.25, 77.25) --
	(298.79, 80.05) --
	(300.33, 82.26) --
	(301.86, 83.90) --
	(303.40, 84.96) --
	(304.94, 85.44) --
	(306.48, 85.34) --
	(308.01, 84.67) --
	(309.55, 83.42) --
	(311.09, 81.59) --
	(312.62, 79.18) --
	(314.16, 76.19) --
	(315.70, 72.63) --
	(317.23, 68.48) --
	(318.77, 63.81) --
	(320.31, 59.23) --
	(321.85, 54.95) --
	(323.38, 50.95) --
	(324.92, 47.24) --
	(326.46, 43.82) --
	(327.99, 40.69) --
	(329.53, 37.84) --
	(331.07, 35.29) --
	(332.60, 33.03) --
	(334.14, 31.05) --
	(335.68, 29.36) --
	(337.22, 27.97) --
	(338.75, 26.86) --
	(340.29, 26.04) --
	(341.83, 25.51) --
	(343.36, 25.27) --
	(344.90, 25.25) --
	(346.44, 25.25) --
	(347.97, 25.25) --
	(349.51, 25.25) --
	(351.05, 25.25) --
	(352.58, 25.25) --
	(354.12, 25.25) --
	(355.66, 25.25) --
	(357.20, 25.25) --
	(358.73, 25.25) --
	(360.27, 25.25) --
	(361.81, 25.25) --
	(363.34, 25.25) --
	(364.88, 25.25) --
	(366.42, 25.25) --
	(367.95, 25.25);
\definecolor{drawColor}{RGB}{255,235,58}

\path[draw=drawColor,line width= 1.1pt,line join=round] (217.33, 25.25) --
	(218.87, 25.25) --
	(220.40, 25.25) --
	(221.94, 25.25) --
	(223.48, 25.25) --
	(225.02, 25.25) --
	(226.55, 25.25) --
	(228.09, 25.25) --
	(229.63, 25.25) --
	(231.16, 25.25) --
	(232.70, 25.25) --
	(234.24, 25.25) --
	(235.77, 25.25) --
	(237.31, 25.25) --
	(238.85, 25.25) --
	(240.39, 25.25) --
	(241.92, 25.25) --
	(243.46, 25.25) --
	(245.00, 25.25) --
	(246.53, 25.25) --
	(248.07, 25.25) --
	(249.61, 25.25) --
	(251.14, 25.25) --
	(252.68, 25.25) --
	(254.22, 25.25) --
	(255.76, 25.25) --
	(257.29, 25.25) --
	(258.83, 25.25) --
	(260.37, 25.25) --
	(261.90, 25.25) --
	(263.44, 25.25) --
	(264.98, 25.25) --
	(266.51, 25.25) --
	(268.05, 25.25) --
	(269.59, 25.25) --
	(271.13, 25.25) --
	(272.66, 25.25) --
	(274.20, 25.25) --
	(275.74, 25.25) --
	(277.27, 25.25) --
	(278.81, 25.25) --
	(280.35, 25.25) --
	(281.88, 25.25) --
	(283.42, 25.25) --
	(284.96, 25.25) --
	(286.49, 25.25) --
	(288.03, 25.25) --
	(289.57, 25.25) --
	(291.11, 25.25) --
	(292.64, 25.25) --
	(294.18, 25.40) --
	(295.72, 25.83) --
	(297.25, 26.55) --
	(298.79, 27.57) --
	(300.33, 28.87) --
	(301.86, 30.46) --
	(303.40, 32.33) --
	(304.94, 34.50) --
	(306.48, 36.96) --
	(308.01, 39.71) --
	(309.55, 42.74) --
	(311.09, 46.07) --
	(312.62, 49.68) --
	(314.16, 53.58) --
	(315.70, 57.77) --
	(317.23, 62.25) --
	(318.77, 66.97) --
	(320.31, 71.31) --
	(321.85, 75.07) --
	(323.38, 78.25) --
	(324.92, 80.85) --
	(326.46, 82.87) --
	(327.99, 84.32) --
	(329.53, 85.18) --
	(331.07, 85.47) --
	(332.60, 85.18) --
	(334.14, 84.32) --
	(335.68, 82.87) --
	(337.22, 80.85) --
	(338.75, 78.25) --
	(340.29, 75.07) --
	(341.83, 71.31) --
	(343.36, 66.97) --
	(344.90, 62.25) --
	(346.44, 57.77) --
	(347.97, 53.58) --
	(349.51, 49.68) --
	(351.05, 46.07) --
	(352.58, 42.74) --
	(354.12, 39.71) --
	(355.66, 36.96) --
	(357.20, 34.50) --
	(358.73, 32.33) --
	(360.27, 30.46) --
	(361.81, 28.87) --
	(363.34, 27.57) --
	(364.88, 26.55) --
	(366.42, 25.83) --
	(367.95, 25.40);
\definecolor{drawColor}{RGB}{255,152,0}

\path[draw=drawColor,line width= 1.1pt,line join=round] (217.33, 25.25) --
	(218.87, 25.25) --
	(220.40, 25.25) --
	(221.94, 25.25) --
	(223.48, 25.25) --
	(225.02, 25.25) --
	(226.55, 25.25) --
	(228.09, 25.25) --
	(229.63, 25.25) --
	(231.16, 25.25) --
	(232.70, 25.25) --
	(234.24, 25.25) --
	(235.77, 25.25) --
	(237.31, 25.25) --
	(238.85, 25.25) --
	(240.39, 25.25) --
	(241.92, 25.25) --
	(243.46, 25.25) --
	(245.00, 25.25) --
	(246.53, 25.25) --
	(248.07, 25.25) --
	(249.61, 25.25) --
	(251.14, 25.25) --
	(252.68, 25.25) --
	(254.22, 25.25) --
	(255.76, 25.25) --
	(257.29, 25.25) --
	(258.83, 25.25) --
	(260.37, 25.25) --
	(261.90, 25.25) --
	(263.44, 25.25) --
	(264.98, 25.25) --
	(266.51, 25.25) --
	(268.05, 25.25) --
	(269.59, 25.25) --
	(271.13, 25.25) --
	(272.66, 25.25) --
	(274.20, 25.25) --
	(275.74, 25.25) --
	(277.27, 25.25) --
	(278.81, 25.25) --
	(280.35, 25.25) --
	(281.88, 25.25) --
	(283.42, 25.25) --
	(284.96, 25.25) --
	(286.49, 25.25) --
	(288.03, 25.25) --
	(289.57, 25.25) --
	(291.11, 25.25) --
	(292.64, 25.25) --
	(294.18, 25.25) --
	(295.72, 25.25) --
	(297.25, 25.25) --
	(298.79, 25.25) --
	(300.33, 25.25) --
	(301.86, 25.25) --
	(303.40, 25.25) --
	(304.94, 25.25) --
	(306.48, 25.25) --
	(308.01, 25.25) --
	(309.55, 25.25) --
	(311.09, 25.25) --
	(312.62, 25.25) --
	(314.16, 25.25) --
	(315.70, 25.25) --
	(317.23, 25.25) --
	(318.77, 25.27) --
	(320.31, 25.51) --
	(321.85, 26.04) --
	(323.38, 26.86) --
	(324.92, 27.97) --
	(326.46, 29.36) --
	(327.99, 31.05) --
	(329.53, 33.03) --
	(331.07, 35.29) --
	(332.60, 37.84) --
	(334.14, 40.69) --
	(335.68, 43.82) --
	(337.22, 47.24) --
	(338.75, 50.95) --
	(340.29, 54.95) --
	(341.83, 59.23) --
	(343.36, 63.81) --
	(344.90, 68.52) --
	(346.44, 72.87) --
	(347.97, 76.81) --
	(349.51, 80.34) --
	(351.05, 83.48) --
	(352.58, 86.20) --
	(354.12, 88.52) --
	(355.66, 90.44) --
	(357.20, 91.95) --
	(358.73, 93.06) --
	(360.27, 93.77) --
	(361.81, 94.06) --
	(363.34, 93.96) --
	(364.88, 93.45) --
	(366.42, 92.53) --
	(367.95, 91.21);
\definecolor{drawColor}{RGB}{121,84,71}

\path[draw=drawColor,line width= 1.1pt,line join=round] (217.33, 25.25) --
	(218.87, 25.25) --
	(220.40, 25.25) --
	(221.94, 25.25) --
	(223.48, 25.25) --
	(225.02, 25.25) --
	(226.55, 25.25) --
	(228.09, 25.25) --
	(229.63, 25.25) --
	(231.16, 25.25) --
	(232.70, 25.25) --
	(234.24, 25.25) --
	(235.77, 25.25) --
	(237.31, 25.25) --
	(238.85, 25.25) --
	(240.39, 25.25) --
	(241.92, 25.25) --
	(243.46, 25.25) --
	(245.00, 25.25) --
	(246.53, 25.25) --
	(248.07, 25.25) --
	(249.61, 25.25) --
	(251.14, 25.25) --
	(252.68, 25.25) --
	(254.22, 25.25) --
	(255.76, 25.25) --
	(257.29, 25.25) --
	(258.83, 25.25) --
	(260.37, 25.25) --
	(261.90, 25.25) --
	(263.44, 25.25) --
	(264.98, 25.25) --
	(266.51, 25.25) --
	(268.05, 25.25) --
	(269.59, 25.25) --
	(271.13, 25.25) --
	(272.66, 25.25) --
	(274.20, 25.25) --
	(275.74, 25.25) --
	(277.27, 25.25) --
	(278.81, 25.25) --
	(280.35, 25.25) --
	(281.88, 25.25) --
	(283.42, 25.25) --
	(284.96, 25.25) --
	(286.49, 25.25) --
	(288.03, 25.25) --
	(289.57, 25.25) --
	(291.11, 25.25) --
	(292.64, 25.25) --
	(294.18, 25.25) --
	(295.72, 25.25) --
	(297.25, 25.25) --
	(298.79, 25.25) --
	(300.33, 25.25) --
	(301.86, 25.25) --
	(303.40, 25.25) --
	(304.94, 25.25) --
	(306.48, 25.25) --
	(308.01, 25.25) --
	(309.55, 25.25) --
	(311.09, 25.25) --
	(312.62, 25.25) --
	(314.16, 25.25) --
	(315.70, 25.25) --
	(317.23, 25.25) --
	(318.77, 25.25) --
	(320.31, 25.25) --
	(321.85, 25.25) --
	(323.38, 25.25) --
	(324.92, 25.25) --
	(326.46, 25.25) --
	(327.99, 25.25) --
	(329.53, 25.25) --
	(331.07, 25.25) --
	(332.60, 25.25) --
	(334.14, 25.25) --
	(335.68, 25.25) --
	(337.22, 25.25) --
	(338.75, 25.25) --
	(340.29, 25.25) --
	(341.83, 25.25) --
	(343.36, 25.25) --
	(344.90, 25.28) --
	(346.44, 25.41) --
	(347.97, 25.66) --
	(349.51, 26.03) --
	(351.05, 26.51) --
	(352.58, 27.11) --
	(354.12, 27.82) --
	(355.66, 28.65) --
	(357.20, 29.60) --
	(358.73, 30.66) --
	(360.27, 31.83) --
	(361.81, 33.12) --
	(363.34, 34.53) --
	(364.88, 36.05) --
	(366.42, 37.69) --
	(367.95, 39.44);
\end{scope}
\begin{scope}
\path[clip] (386.87,186.57) rectangle (552.55,263.47);
\definecolor{drawColor}{gray}{0.92}

\path[draw=drawColor,line width= 0.3pt,line join=round] (386.87,200.10) --
	(552.55,200.10);

\path[draw=drawColor,line width= 0.3pt,line join=round] (386.87,220.17) --
	(552.55,220.17);

\path[draw=drawColor,line width= 0.3pt,line join=round] (386.87,240.25) --
	(552.55,240.25);

\path[draw=drawColor,line width= 0.3pt,line join=round] (386.87,260.32) --
	(552.55,260.32);

\path[draw=drawColor,line width= 0.3pt,line join=round] (412.07,186.57) --
	(412.07,263.47);

\path[draw=drawColor,line width= 0.3pt,line join=round] (450.50,186.57) --
	(450.50,263.47);

\path[draw=drawColor,line width= 0.3pt,line join=round] (488.92,186.57) --
	(488.92,263.47);

\path[draw=drawColor,line width= 0.3pt,line join=round] (527.35,186.57) --
	(527.35,263.47);

\path[draw=drawColor,line width= 0.6pt,line join=round] (386.87,190.06) --
	(552.55,190.06);

\path[draw=drawColor,line width= 0.6pt,line join=round] (386.87,210.14) --
	(552.55,210.14);

\path[draw=drawColor,line width= 0.6pt,line join=round] (386.87,230.21) --
	(552.55,230.21);

\path[draw=drawColor,line width= 0.6pt,line join=round] (386.87,250.28) --
	(552.55,250.28);

\path[draw=drawColor,line width= 0.6pt,line join=round] (392.86,186.57) --
	(392.86,263.47);

\path[draw=drawColor,line width= 0.6pt,line join=round] (431.29,186.57) --
	(431.29,263.47);

\path[draw=drawColor,line width= 0.6pt,line join=round] (469.71,186.57) --
	(469.71,263.47);

\path[draw=drawColor,line width= 0.6pt,line join=round] (508.13,186.57) --
	(508.13,263.47);

\path[draw=drawColor,line width= 0.6pt,line join=round] (546.56,186.57) --
	(546.56,263.47);
\definecolor{drawColor}{RGB}{155,38,176}

\path[draw=drawColor,line width= 1.1pt,line join=round] (394.40,219.97) --
	(395.93,198.60) --
	(397.47,190.21) --
	(399.01,190.06) --
	(400.55,190.06) --
	(402.08,190.06) --
	(403.62,190.06) --
	(405.16,190.06) --
	(406.69,190.06) --
	(408.23,190.06) --
	(409.77,190.06) --
	(411.30,190.06) --
	(412.84,190.06) --
	(414.38,190.06) --
	(415.92,190.06) --
	(417.45,190.06) --
	(418.99,190.06) --
	(420.53,190.06) --
	(422.06,190.06) --
	(423.60,190.06) --
	(425.14,190.06) --
	(426.67,190.06) --
	(428.21,190.06) --
	(429.75,190.06) --
	(431.29,190.06) --
	(432.82,190.06) --
	(434.36,190.06) --
	(435.90,190.06) --
	(437.43,190.06) --
	(438.97,190.06) --
	(440.51,190.06) --
	(442.04,190.06) --
	(443.58,190.06) --
	(445.12,190.06) --
	(446.66,190.06) --
	(448.19,190.06) --
	(449.73,190.06) --
	(451.27,190.06) --
	(452.80,190.06) --
	(454.34,190.06) --
	(455.88,190.06) --
	(457.41,190.06) --
	(458.95,190.06) --
	(460.49,190.06) --
	(462.02,190.06) --
	(463.56,190.06) --
	(465.10,190.06) --
	(466.64,190.06) --
	(468.17,190.06) --
	(469.71,190.06) --
	(471.25,190.06) --
	(472.78,190.06) --
	(474.32,190.06) --
	(475.86,190.06) --
	(477.39,190.06) --
	(478.93,190.06) --
	(480.47,190.06) --
	(482.01,190.06) --
	(483.54,190.06) --
	(485.08,190.06) --
	(486.62,190.06) --
	(488.15,190.06) --
	(489.69,190.06) --
	(491.23,190.06) --
	(492.76,190.06) --
	(494.30,190.06) --
	(495.84,190.06) --
	(497.38,190.06) --
	(498.91,190.06) --
	(500.45,190.06) --
	(501.99,190.06) --
	(503.52,190.06) --
	(505.06,190.06) --
	(506.60,190.06) --
	(508.13,190.06) --
	(509.67,190.06) --
	(511.21,190.06) --
	(512.74,190.06) --
	(514.28,190.06) --
	(515.82,190.06) --
	(517.36,190.06) --
	(518.89,190.06) --
	(520.43,190.06) --
	(521.97,190.06) --
	(523.50,190.06) --
	(525.04,190.06) --
	(526.58,190.06) --
	(528.11,190.06) --
	(529.65,190.06) --
	(531.19,190.06) --
	(532.73,190.06) --
	(534.26,190.06) --
	(535.80,190.06) --
	(537.34,190.06) --
	(538.87,190.06) --
	(540.41,190.06) --
	(541.95,190.06) --
	(543.48,190.06) --
	(545.02,190.06);
\definecolor{drawColor}{RGB}{63,81,180}

\path[draw=drawColor,line width= 1.1pt,line join=round] (394.40,237.55) --
	(395.93,250.20) --
	(397.47,244.09) --
	(399.01,227.19) --
	(400.55,213.30) --
	(402.08,202.65) --
	(403.62,195.24) --
	(405.16,191.07) --
	(406.69,190.06) --
	(408.23,190.06) --
	(409.77,190.06) --
	(411.30,190.06) --
	(412.84,190.06) --
	(414.38,190.06) --
	(415.92,190.06) --
	(417.45,190.06) --
	(418.99,190.06) --
	(420.53,190.06) --
	(422.06,190.06) --
	(423.60,190.06) --
	(425.14,190.06) --
	(426.67,190.06) --
	(428.21,190.06) --
	(429.75,190.06) --
	(431.29,190.06) --
	(432.82,190.06) --
	(434.36,190.06) --
	(435.90,190.06) --
	(437.43,190.06) --
	(438.97,190.06) --
	(440.51,190.06) --
	(442.04,190.06) --
	(443.58,190.06) --
	(445.12,190.06) --
	(446.66,190.06) --
	(448.19,190.06) --
	(449.73,190.06) --
	(451.27,190.06) --
	(452.80,190.06) --
	(454.34,190.06) --
	(455.88,190.06) --
	(457.41,190.06) --
	(458.95,190.06) --
	(460.49,190.06) --
	(462.02,190.06) --
	(463.56,190.06) --
	(465.10,190.06) --
	(466.64,190.06) --
	(468.17,190.06) --
	(469.71,190.06) --
	(471.25,190.06) --
	(472.78,190.06) --
	(474.32,190.06) --
	(475.86,190.06) --
	(477.39,190.06) --
	(478.93,190.06) --
	(480.47,190.06) --
	(482.01,190.06) --
	(483.54,190.06) --
	(485.08,190.06) --
	(486.62,190.06) --
	(488.15,190.06) --
	(489.69,190.06) --
	(491.23,190.06) --
	(492.76,190.06) --
	(494.30,190.06) --
	(495.84,190.06) --
	(497.38,190.06) --
	(498.91,190.06) --
	(500.45,190.06) --
	(501.99,190.06) --
	(503.52,190.06) --
	(505.06,190.06) --
	(506.60,190.06) --
	(508.13,190.06) --
	(509.67,190.06) --
	(511.21,190.06) --
	(512.74,190.06) --
	(514.28,190.06) --
	(515.82,190.06) --
	(517.36,190.06) --
	(518.89,190.06) --
	(520.43,190.06) --
	(521.97,190.06) --
	(523.50,190.06) --
	(525.04,190.06) --
	(526.58,190.06) --
	(528.11,190.06) --
	(529.65,190.06) --
	(531.19,190.06) --
	(532.73,190.06) --
	(534.26,190.06) --
	(535.80,190.06) --
	(537.34,190.06) --
	(538.87,190.06) --
	(540.41,190.06) --
	(541.95,190.06) --
	(543.48,190.06) --
	(545.02,190.06);
\definecolor{drawColor}{RGB}{2,169,243}

\path[draw=drawColor,line width= 1.1pt,line join=round] (394.40,192.97) --
	(395.93,201.67) --
	(397.47,216.19) --
	(399.01,232.55) --
	(400.55,243.92) --
	(402.08,250.17) --
	(403.62,251.32) --
	(405.16,247.36) --
	(406.69,238.44) --
	(408.23,228.69) --
	(409.77,220.03) --
	(411.30,212.47) --
	(412.84,206.01) --
	(414.38,200.65) --
	(415.92,196.38) --
	(417.45,193.20) --
	(418.99,191.13) --
	(420.53,190.15) --
	(422.06,190.06) --
	(423.60,190.06) --
	(425.14,190.06) --
	(426.67,190.06) --
	(428.21,190.06) --
	(429.75,190.06) --
	(431.29,190.06) --
	(432.82,190.06) --
	(434.36,190.06) --
	(435.90,190.06) --
	(437.43,190.06) --
	(438.97,190.06) --
	(440.51,190.06) --
	(442.04,190.06) --
	(443.58,190.06) --
	(445.12,190.06) --
	(446.66,190.06) --
	(448.19,190.06) --
	(449.73,190.06) --
	(451.27,190.06) --
	(452.80,190.06) --
	(454.34,190.06) --
	(455.88,190.06) --
	(457.41,190.06) --
	(458.95,190.06) --
	(460.49,190.06) --
	(462.02,190.06) --
	(463.56,190.06) --
	(465.10,190.06) --
	(466.64,190.06) --
	(468.17,190.06) --
	(469.71,190.06) --
	(471.25,190.06) --
	(472.78,190.06) --
	(474.32,190.06) --
	(475.86,190.06) --
	(477.39,190.06) --
	(478.93,190.06) --
	(480.47,190.06) --
	(482.01,190.06) --
	(483.54,190.06) --
	(485.08,190.06) --
	(486.62,190.06) --
	(488.15,190.06) --
	(489.69,190.06) --
	(491.23,190.06) --
	(492.76,190.06) --
	(494.30,190.06) --
	(495.84,190.06) --
	(497.38,190.06) --
	(498.91,190.06) --
	(500.45,190.06) --
	(501.99,190.06) --
	(503.52,190.06) --
	(505.06,190.06) --
	(506.60,190.06) --
	(508.13,190.06) --
	(509.67,190.06) --
	(511.21,190.06) --
	(512.74,190.06) --
	(514.28,190.06) --
	(515.82,190.06) --
	(517.36,190.06) --
	(518.89,190.06) --
	(520.43,190.06) --
	(521.97,190.06) --
	(523.50,190.06) --
	(525.04,190.06) --
	(526.58,190.06) --
	(528.11,190.06) --
	(529.65,190.06) --
	(531.19,190.06) --
	(532.73,190.06) --
	(534.26,190.06) --
	(535.80,190.06) --
	(537.34,190.06) --
	(538.87,190.06) --
	(540.41,190.06) --
	(541.95,190.06) --
	(543.48,190.06) --
	(545.02,190.06);
\definecolor{drawColor}{RGB}{0,150,135}

\path[draw=drawColor,line width= 1.1pt,line join=round] (394.40,190.06) --
	(395.93,190.06) --
	(397.47,190.06) --
	(399.01,190.74) --
	(400.55,193.27) --
	(402.08,197.66) --
	(403.62,203.92) --
	(405.16,212.05) --
	(406.69,221.97) --
	(408.23,231.25) --
	(409.77,238.78) --
	(411.30,244.55) --
	(412.84,248.58) --
	(414.38,250.84) --
	(415.92,251.36) --
	(417.45,250.12) --
	(418.99,247.13) --
	(420.53,242.39) --
	(422.06,236.28) --
	(423.60,230.41) --
	(425.14,224.94) --
	(426.67,219.87) --
	(428.21,215.19) --
	(429.75,210.91) --
	(431.29,207.04) --
	(432.82,203.56) --
	(434.36,200.48) --
	(435.90,197.79) --
	(437.43,195.51) --
	(438.97,193.63) --
	(440.51,192.14) --
	(442.04,191.05) --
	(443.58,190.36) --
	(445.12,190.07) --
	(446.66,190.06) --
	(448.19,190.06) --
	(449.73,190.06) --
	(451.27,190.06) --
	(452.80,190.06) --
	(454.34,190.06) --
	(455.88,190.06) --
	(457.41,190.06) --
	(458.95,190.06) --
	(460.49,190.06) --
	(462.02,190.06) --
	(463.56,190.06) --
	(465.10,190.06) --
	(466.64,190.06) --
	(468.17,190.06) --
	(469.71,190.06) --
	(471.25,190.06) --
	(472.78,190.06) --
	(474.32,190.06) --
	(475.86,190.06) --
	(477.39,190.06) --
	(478.93,190.06) --
	(480.47,190.06) --
	(482.01,190.06) --
	(483.54,190.06) --
	(485.08,190.06) --
	(486.62,190.06) --
	(488.15,190.06) --
	(489.69,190.06) --
	(491.23,190.06) --
	(492.76,190.06) --
	(494.30,190.06) --
	(495.84,190.06) --
	(497.38,190.06) --
	(498.91,190.06) --
	(500.45,190.06) --
	(501.99,190.06) --
	(503.52,190.06) --
	(505.06,190.06) --
	(506.60,190.06) --
	(508.13,190.06) --
	(509.67,190.06) --
	(511.21,190.06) --
	(512.74,190.06) --
	(514.28,190.06) --
	(515.82,190.06) --
	(517.36,190.06) --
	(518.89,190.06) --
	(520.43,190.06) --
	(521.97,190.06) --
	(523.50,190.06) --
	(525.04,190.06) --
	(526.58,190.06) --
	(528.11,190.06) --
	(529.65,190.06) --
	(531.19,190.06) --
	(532.73,190.06) --
	(534.26,190.06) --
	(535.80,190.06) --
	(537.34,190.06) --
	(538.87,190.06) --
	(540.41,190.06) --
	(541.95,190.06) --
	(543.48,190.06) --
	(545.02,190.06);
\definecolor{drawColor}{RGB}{139,195,74}

\path[draw=drawColor,line width= 1.1pt,line join=round] (394.40,190.06) --
	(395.93,190.06) --
	(397.47,190.06) --
	(399.01,190.06) --
	(400.55,190.06) --
	(402.08,190.06) --
	(403.62,190.06) --
	(405.16,190.06) --
	(406.69,190.08) --
	(408.23,190.55) --
	(409.77,191.67) --
	(411.30,193.46) --
	(412.84,195.90) --
	(414.38,198.99) --
	(415.92,202.75) --
	(417.45,207.16) --
	(418.99,212.22) --
	(420.53,217.95) --
	(422.06,224.09) --
	(423.60,229.69) --
	(425.14,234.65) --
	(426.67,238.96) --
	(428.21,242.63) --
	(429.75,245.65) --
	(431.29,248.03) --
	(432.82,249.76) --
	(434.36,250.85) --
	(435.90,251.30) --
	(437.43,251.10) --
	(438.97,250.26) --
	(440.51,248.78) --
	(442.04,246.65) --
	(443.58,243.87) --
	(445.12,240.45) --
	(446.66,236.63) --
	(448.19,232.94) --
	(449.73,229.40) --
	(451.27,226.01) --
	(452.80,222.77) --
	(454.34,219.69) --
	(455.88,216.76) --
	(457.41,213.98) --
	(458.95,211.36) --
	(460.49,208.89) --
	(462.02,206.57) --
	(463.56,204.40) --
	(465.10,202.38) --
	(466.64,200.52) --
	(468.17,198.81) --
	(469.71,197.25) --
	(471.25,195.85) --
	(472.78,194.60) --
	(474.32,193.50) --
	(475.86,192.55) --
	(477.39,191.76) --
	(478.93,191.11) --
	(480.47,190.62) --
	(482.01,190.29) --
	(483.54,190.10) --
	(485.08,190.06) --
	(486.62,190.06) --
	(488.15,190.06) --
	(489.69,190.06) --
	(491.23,190.06) --
	(492.76,190.06) --
	(494.30,190.06) --
	(495.84,190.06) --
	(497.38,190.06) --
	(498.91,190.06) --
	(500.45,190.06) --
	(501.99,190.06) --
	(503.52,190.06) --
	(505.06,190.06) --
	(506.60,190.06) --
	(508.13,190.06) --
	(509.67,190.06) --
	(511.21,190.06) --
	(512.74,190.06) --
	(514.28,190.06) --
	(515.82,190.06) --
	(517.36,190.06) --
	(518.89,190.06) --
	(520.43,190.06) --
	(521.97,190.06) --
	(523.50,190.06) --
	(525.04,190.06) --
	(526.58,190.06) --
	(528.11,190.06) --
	(529.65,190.06) --
	(531.19,190.06) --
	(532.73,190.06) --
	(534.26,190.06) --
	(535.80,190.06) --
	(537.34,190.06) --
	(538.87,190.06) --
	(540.41,190.06) --
	(541.95,190.06) --
	(543.48,190.06) --
	(545.02,190.06);
\definecolor{drawColor}{RGB}{255,235,58}

\path[draw=drawColor,line width= 1.1pt,line join=round] (394.40,190.06) --
	(395.93,190.06) --
	(397.47,190.06) --
	(399.01,190.06) --
	(400.55,190.06) --
	(402.08,190.06) --
	(403.62,190.06) --
	(405.16,190.06) --
	(406.69,190.06) --
	(408.23,190.06) --
	(409.77,190.06) --
	(411.30,190.06) --
	(412.84,190.06) --
	(414.38,190.06) --
	(415.92,190.06) --
	(417.45,190.06) --
	(418.99,190.06) --
	(420.53,190.06) --
	(422.06,190.11) --
	(423.60,190.38) --
	(425.14,190.90) --
	(426.67,191.66) --
	(428.21,192.67) --
	(429.75,193.92) --
	(431.29,195.42) --
	(432.82,197.16) --
	(434.36,199.15) --
	(435.90,201.39) --
	(437.43,203.87) --
	(438.97,206.60) --
	(440.51,209.57) --
	(442.04,212.78) --
	(443.58,216.25) --
	(445.12,219.95) --
	(446.66,223.76) --
	(448.19,227.33) --
	(449.73,230.65) --
	(451.27,233.73) --
	(452.80,236.55) --
	(454.34,239.13) --
	(455.88,241.46) --
	(457.41,243.54) --
	(458.95,245.37) --
	(460.49,246.96) --
	(462.02,248.29) --
	(463.56,249.38) --
	(465.10,250.22) --
	(466.64,250.81) --
	(468.17,251.15) --
	(469.71,251.25) --
	(471.25,251.09) --
	(472.78,250.69) --
	(474.32,250.04) --
	(475.86,249.14) --
	(477.39,247.99) --
	(478.93,246.60) --
	(480.47,244.95) --
	(482.01,243.06) --
	(483.54,240.92) --
	(485.08,238.54) --
	(486.62,236.15) --
	(488.15,233.82) --
	(489.69,231.54) --
	(491.23,229.33) --
	(492.76,227.18) --
	(494.30,225.09) --
	(495.84,223.06) --
	(497.38,221.09) --
	(498.91,219.18) --
	(500.45,217.33) --
	(501.99,215.55) --
	(503.52,213.82) --
	(505.06,212.15) --
	(506.60,210.55) --
	(508.13,209.00) --
	(509.67,207.52) --
	(511.21,206.09) --
	(512.74,204.73) --
	(514.28,203.43) --
	(515.82,202.18) --
	(517.36,201.00) --
	(518.89,199.88) --
	(520.43,198.82) --
	(521.97,197.82) --
	(523.50,196.88) --
	(525.04,196.00) --
	(526.58,195.18) --
	(528.11,194.43) --
	(529.65,193.73) --
	(531.19,193.09) --
	(532.73,192.52) --
	(534.26,192.00) --
	(535.80,191.55) --
	(537.34,191.15) --
	(538.87,190.82) --
	(540.41,190.55) --
	(541.95,190.34) --
	(543.48,190.18) --
	(545.02,190.09);
\definecolor{drawColor}{RGB}{255,152,0}

\path[draw=drawColor,line width= 1.1pt,line join=round] (394.40,190.06) --
	(395.93,190.06) --
	(397.47,190.06) --
	(399.01,190.06) --
	(400.55,190.06) --
	(402.08,190.06) --
	(403.62,190.06) --
	(405.16,190.06) --
	(406.69,190.06) --
	(408.23,190.06) --
	(409.77,190.06) --
	(411.30,190.06) --
	(412.84,190.06) --
	(414.38,190.06) --
	(415.92,190.06) --
	(417.45,190.06) --
	(418.99,190.06) --
	(420.53,190.06) --
	(422.06,190.06) --
	(423.60,190.06) --
	(425.14,190.06) --
	(426.67,190.06) --
	(428.21,190.06) --
	(429.75,190.06) --
	(431.29,190.06) --
	(432.82,190.06) --
	(434.36,190.06) --
	(435.90,190.06) --
	(437.43,190.06) --
	(438.97,190.06) --
	(440.51,190.06) --
	(442.04,190.06) --
	(443.58,190.06) --
	(445.12,190.06) --
	(446.66,190.09) --
	(448.19,190.21) --
	(449.73,190.43) --
	(451.27,190.74) --
	(452.80,191.15) --
	(454.34,191.66) --
	(455.88,192.26) --
	(457.41,192.96) --
	(458.95,193.75) --
	(460.49,194.64) --
	(462.02,195.62) --
	(463.56,196.70) --
	(465.10,197.88) --
	(466.64,199.15) --
	(468.17,200.52) --
	(469.71,201.98) --
	(471.25,203.54) --
	(472.78,205.19) --
	(474.32,206.94) --
	(475.86,208.79) --
	(477.39,210.73) --
	(478.93,212.77) --
	(480.47,214.91) --
	(482.01,217.14) --
	(483.54,219.46) --
	(485.08,221.87) --
	(486.62,224.20) --
	(488.15,226.40) --
	(489.69,228.45) --
	(491.23,230.36) --
	(492.76,232.13) --
	(494.30,233.76) --
	(495.84,235.26) --
	(497.38,236.61) --
	(498.91,237.82) --
	(500.45,238.90) --
	(501.99,239.83) --
	(503.52,240.62) --
	(505.06,241.28) --
	(506.60,241.79) --
	(508.13,242.17) --
	(509.67,242.40) --
	(511.21,242.50) --
	(512.74,242.45) --
	(514.28,242.27) --
	(515.82,241.95) --
	(517.36,241.48) --
	(518.89,240.88) --
	(520.43,240.14) --
	(521.97,239.25) --
	(523.50,238.23) --
	(525.04,237.07) --
	(526.58,235.77) --
	(528.11,234.32) --
	(529.65,232.74) --
	(531.19,231.02) --
	(532.73,229.16) --
	(534.26,227.16) --
	(535.80,225.02) --
	(537.34,222.74) --
	(538.87,220.32) --
	(540.41,217.76) --
	(541.95,215.06) --
	(543.48,212.22) --
	(545.02,209.24);
\definecolor{drawColor}{RGB}{121,84,71}

\path[draw=drawColor,line width= 1.1pt,line join=round] (394.40,190.06) --
	(395.93,190.06) --
	(397.47,190.06) --
	(399.01,190.06) --
	(400.55,190.06) --
	(402.08,190.06) --
	(403.62,190.06) --
	(405.16,190.06) --
	(406.69,190.06) --
	(408.23,190.06) --
	(409.77,190.06) --
	(411.30,190.06) --
	(412.84,190.06) --
	(414.38,190.06) --
	(415.92,190.06) --
	(417.45,190.06) --
	(418.99,190.06) --
	(420.53,190.06) --
	(422.06,190.06) --
	(423.60,190.06) --
	(425.14,190.06) --
	(426.67,190.06) --
	(428.21,190.06) --
	(429.75,190.06) --
	(431.29,190.06) --
	(432.82,190.06) --
	(434.36,190.06) --
	(435.90,190.06) --
	(437.43,190.06) --
	(438.97,190.06) --
	(440.51,190.06) --
	(442.04,190.06) --
	(443.58,190.06) --
	(445.12,190.06) --
	(446.66,190.06) --
	(448.19,190.06) --
	(449.73,190.06) --
	(451.27,190.06) --
	(452.80,190.06) --
	(454.34,190.06) --
	(455.88,190.06) --
	(457.41,190.06) --
	(458.95,190.06) --
	(460.49,190.06) --
	(462.02,190.06) --
	(463.56,190.06) --
	(465.10,190.06) --
	(466.64,190.06) --
	(468.17,190.06) --
	(469.71,190.06) --
	(471.25,190.06) --
	(472.78,190.06) --
	(474.32,190.06) --
	(475.86,190.06) --
	(477.39,190.06) --
	(478.93,190.06) --
	(480.47,190.06) --
	(482.01,190.06) --
	(483.54,190.06) --
	(485.08,190.07) --
	(486.62,190.13) --
	(488.15,190.27) --
	(489.69,190.49) --
	(491.23,190.79) --
	(492.76,191.17) --
	(494.30,191.63) --
	(495.84,192.17) --
	(497.38,192.78) --
	(498.91,193.48) --
	(500.45,194.25) --
	(501.99,195.11) --
	(503.52,196.04) --
	(505.06,197.05) --
	(506.60,198.14) --
	(508.13,199.31) --
	(509.67,200.56) --
	(511.21,201.89) --
	(512.74,203.30) --
	(514.28,204.79) --
	(515.82,206.35) --
	(517.36,208.00) --
	(518.89,209.72) --
	(520.43,211.53) --
	(521.97,213.41) --
	(523.50,215.37) --
	(525.04,217.41) --
	(526.58,219.53) --
	(528.11,221.73) --
	(529.65,224.01) --
	(531.19,226.37) --
	(532.73,228.80) --
	(534.26,231.32) --
	(535.80,233.92) --
	(537.34,236.59) --
	(538.87,239.34) --
	(540.41,242.18) --
	(541.95,245.09) --
	(543.48,248.08) --
	(545.02,251.15);
\end{scope}
\begin{scope}
\path[clip] (386.87,104.16) rectangle (552.55,181.07);
\definecolor{drawColor}{gray}{0.92}

\path[draw=drawColor,line width= 0.3pt,line join=round] (386.87,117.69) --
	(552.55,117.69);

\path[draw=drawColor,line width= 0.3pt,line join=round] (386.87,137.77) --
	(552.55,137.77);

\path[draw=drawColor,line width= 0.3pt,line join=round] (386.87,157.84) --
	(552.55,157.84);

\path[draw=drawColor,line width= 0.3pt,line join=round] (386.87,177.91) --
	(552.55,177.91);

\path[draw=drawColor,line width= 0.3pt,line join=round] (412.07,104.16) --
	(412.07,181.07);

\path[draw=drawColor,line width= 0.3pt,line join=round] (450.50,104.16) --
	(450.50,181.07);

\path[draw=drawColor,line width= 0.3pt,line join=round] (488.92,104.16) --
	(488.92,181.07);

\path[draw=drawColor,line width= 0.3pt,line join=round] (527.35,104.16) --
	(527.35,181.07);

\path[draw=drawColor,line width= 0.6pt,line join=round] (386.87,107.66) --
	(552.55,107.66);

\path[draw=drawColor,line width= 0.6pt,line join=round] (386.87,127.73) --
	(552.55,127.73);

\path[draw=drawColor,line width= 0.6pt,line join=round] (386.87,147.80) --
	(552.55,147.80);

\path[draw=drawColor,line width= 0.6pt,line join=round] (386.87,167.88) --
	(552.55,167.88);

\path[draw=drawColor,line width= 0.6pt,line join=round] (392.86,104.16) --
	(392.86,181.07);

\path[draw=drawColor,line width= 0.6pt,line join=round] (431.29,104.16) --
	(431.29,181.07);

\path[draw=drawColor,line width= 0.6pt,line join=round] (469.71,104.16) --
	(469.71,181.07);

\path[draw=drawColor,line width= 0.6pt,line join=round] (508.13,104.16) --
	(508.13,181.07);

\path[draw=drawColor,line width= 0.6pt,line join=round] (546.56,104.16) --
	(546.56,181.07);
\definecolor{drawColor}{RGB}{155,38,176}

\path[draw=drawColor,line width= 1.1pt,line join=round] (394.40,126.35) --
	(395.93,113.00) --
	(397.47,107.75) --
	(399.01,107.66) --
	(400.55,107.66) --
	(402.08,107.66) --
	(403.62,107.66) --
	(405.16,107.66) --
	(406.69,107.66) --
	(408.23,107.66) --
	(409.77,107.66) --
	(411.30,107.66) --
	(412.84,107.66) --
	(414.38,107.66) --
	(415.92,107.66) --
	(417.45,107.66) --
	(418.99,107.66) --
	(420.53,107.66) --
	(422.06,107.66) --
	(423.60,107.66) --
	(425.14,107.66) --
	(426.67,107.66) --
	(428.21,107.66) --
	(429.75,107.66) --
	(431.29,107.66) --
	(432.82,107.66) --
	(434.36,107.66) --
	(435.90,107.66) --
	(437.43,107.66) --
	(438.97,107.66) --
	(440.51,107.66) --
	(442.04,107.66) --
	(443.58,107.66) --
	(445.12,107.66) --
	(446.66,107.66) --
	(448.19,107.66) --
	(449.73,107.66) --
	(451.27,107.66) --
	(452.80,107.66) --
	(454.34,107.66) --
	(455.88,107.66) --
	(457.41,107.66) --
	(458.95,107.66) --
	(460.49,107.66) --
	(462.02,107.66) --
	(463.56,107.66) --
	(465.10,107.66) --
	(466.64,107.66) --
	(468.17,107.66) --
	(469.71,107.66) --
	(471.25,107.66) --
	(472.78,107.66) --
	(474.32,107.66) --
	(475.86,107.66) --
	(477.39,107.66) --
	(478.93,107.66) --
	(480.47,107.66) --
	(482.01,107.66) --
	(483.54,107.66) --
	(485.08,107.66) --
	(486.62,107.66) --
	(488.15,107.66) --
	(489.69,107.66) --
	(491.23,107.66) --
	(492.76,107.66) --
	(494.30,107.66) --
	(495.84,107.66) --
	(497.38,107.66) --
	(498.91,107.66) --
	(500.45,107.66) --
	(501.99,107.66) --
	(503.52,107.66) --
	(505.06,107.66) --
	(506.60,107.66) --
	(508.13,107.66) --
	(509.67,107.66) --
	(511.21,107.66) --
	(512.74,107.66) --
	(514.28,107.66) --
	(515.82,107.66) --
	(517.36,107.66) --
	(518.89,107.66) --
	(520.43,107.66) --
	(521.97,107.66) --
	(523.50,107.66) --
	(525.04,107.66) --
	(526.58,107.66) --
	(528.11,107.66) --
	(529.65,107.66) --
	(531.19,107.66) --
	(532.73,107.66) --
	(534.26,107.66) --
	(535.80,107.66) --
	(537.34,107.66) --
	(538.87,107.66) --
	(540.41,107.66) --
	(541.95,107.66) --
	(543.48,107.66) --
	(545.02,107.66);
\definecolor{drawColor}{RGB}{63,81,180}

\path[draw=drawColor,line width= 1.1pt,line join=round] (394.40,166.36) --
	(395.93,171.00) --
	(397.47,161.74) --
	(399.01,144.79) --
	(400.55,130.89) --
	(402.08,120.24) --
	(403.62,112.83) --
	(405.16,108.66) --
	(406.69,107.66) --
	(408.23,107.66) --
	(409.77,107.66) --
	(411.30,107.66) --
	(412.84,107.66) --
	(414.38,107.66) --
	(415.92,107.66) --
	(417.45,107.66) --
	(418.99,107.66) --
	(420.53,107.66) --
	(422.06,107.66) --
	(423.60,107.66) --
	(425.14,107.66) --
	(426.67,107.66) --
	(428.21,107.66) --
	(429.75,107.66) --
	(431.29,107.66) --
	(432.82,107.66) --
	(434.36,107.66) --
	(435.90,107.66) --
	(437.43,107.66) --
	(438.97,107.66) --
	(440.51,107.66) --
	(442.04,107.66) --
	(443.58,107.66) --
	(445.12,107.66) --
	(446.66,107.66) --
	(448.19,107.66) --
	(449.73,107.66) --
	(451.27,107.66) --
	(452.80,107.66) --
	(454.34,107.66) --
	(455.88,107.66) --
	(457.41,107.66) --
	(458.95,107.66) --
	(460.49,107.66) --
	(462.02,107.66) --
	(463.56,107.66) --
	(465.10,107.66) --
	(466.64,107.66) --
	(468.17,107.66) --
	(469.71,107.66) --
	(471.25,107.66) --
	(472.78,107.66) --
	(474.32,107.66) --
	(475.86,107.66) --
	(477.39,107.66) --
	(478.93,107.66) --
	(480.47,107.66) --
	(482.01,107.66) --
	(483.54,107.66) --
	(485.08,107.66) --
	(486.62,107.66) --
	(488.15,107.66) --
	(489.69,107.66) --
	(491.23,107.66) --
	(492.76,107.66) --
	(494.30,107.66) --
	(495.84,107.66) --
	(497.38,107.66) --
	(498.91,107.66) --
	(500.45,107.66) --
	(501.99,107.66) --
	(503.52,107.66) --
	(505.06,107.66) --
	(506.60,107.66) --
	(508.13,107.66) --
	(509.67,107.66) --
	(511.21,107.66) --
	(512.74,107.66) --
	(514.28,107.66) --
	(515.82,107.66) --
	(517.36,107.66) --
	(518.89,107.66) --
	(520.43,107.66) --
	(521.97,107.66) --
	(523.50,107.66) --
	(525.04,107.66) --
	(526.58,107.66) --
	(528.11,107.66) --
	(529.65,107.66) --
	(531.19,107.66) --
	(532.73,107.66) --
	(534.26,107.66) --
	(535.80,107.66) --
	(537.34,107.66) --
	(538.87,107.66) --
	(540.41,107.66) --
	(541.95,107.66) --
	(543.48,107.66) --
	(545.02,107.66);
\definecolor{drawColor}{RGB}{2,169,243}

\path[draw=drawColor,line width= 1.1pt,line join=round] (394.40,110.56) --
	(395.93,119.27) --
	(397.47,133.78) --
	(399.01,150.14) --
	(400.55,161.51) --
	(402.08,167.77) --
	(403.62,168.92) --
	(405.16,164.96) --
	(406.69,156.03) --
	(408.23,146.28) --
	(409.77,137.63) --
	(411.30,130.07) --
	(412.84,123.61) --
	(414.38,118.24) --
	(415.92,113.97) --
	(417.45,110.80) --
	(418.99,108.72) --
	(420.53,107.74) --
	(422.06,107.66) --
	(423.60,107.66) --
	(425.14,107.66) --
	(426.67,107.66) --
	(428.21,107.66) --
	(429.75,107.66) --
	(431.29,107.66) --
	(432.82,107.66) --
	(434.36,107.66) --
	(435.90,107.66) --
	(437.43,107.66) --
	(438.97,107.66) --
	(440.51,107.66) --
	(442.04,107.66) --
	(443.58,107.66) --
	(445.12,107.66) --
	(446.66,107.66) --
	(448.19,107.66) --
	(449.73,107.66) --
	(451.27,107.66) --
	(452.80,107.66) --
	(454.34,107.66) --
	(455.88,107.66) --
	(457.41,107.66) --
	(458.95,107.66) --
	(460.49,107.66) --
	(462.02,107.66) --
	(463.56,107.66) --
	(465.10,107.66) --
	(466.64,107.66) --
	(468.17,107.66) --
	(469.71,107.66) --
	(471.25,107.66) --
	(472.78,107.66) --
	(474.32,107.66) --
	(475.86,107.66) --
	(477.39,107.66) --
	(478.93,107.66) --
	(480.47,107.66) --
	(482.01,107.66) --
	(483.54,107.66) --
	(485.08,107.66) --
	(486.62,107.66) --
	(488.15,107.66) --
	(489.69,107.66) --
	(491.23,107.66) --
	(492.76,107.66) --
	(494.30,107.66) --
	(495.84,107.66) --
	(497.38,107.66) --
	(498.91,107.66) --
	(500.45,107.66) --
	(501.99,107.66) --
	(503.52,107.66) --
	(505.06,107.66) --
	(506.60,107.66) --
	(508.13,107.66) --
	(509.67,107.66) --
	(511.21,107.66) --
	(512.74,107.66) --
	(514.28,107.66) --
	(515.82,107.66) --
	(517.36,107.66) --
	(518.89,107.66) --
	(520.43,107.66) --
	(521.97,107.66) --
	(523.50,107.66) --
	(525.04,107.66) --
	(526.58,107.66) --
	(528.11,107.66) --
	(529.65,107.66) --
	(531.19,107.66) --
	(532.73,107.66) --
	(534.26,107.66) --
	(535.80,107.66) --
	(537.34,107.66) --
	(538.87,107.66) --
	(540.41,107.66) --
	(541.95,107.66) --
	(543.48,107.66) --
	(545.02,107.66);
\definecolor{drawColor}{RGB}{0,150,135}

\path[draw=drawColor,line width= 1.1pt,line join=round] (394.40,107.66) --
	(395.93,107.66) --
	(397.47,107.66) --
	(399.01,108.34) --
	(400.55,110.86) --
	(402.08,115.26) --
	(403.62,121.52) --
	(405.16,129.65) --
	(406.69,139.56) --
	(408.23,148.84) --
	(409.77,156.37) --
	(411.30,162.15) --
	(412.84,166.17) --
	(414.38,168.44) --
	(415.92,168.95) --
	(417.45,167.72) --
	(418.99,164.73) --
	(420.53,159.98) --
	(422.06,153.88) --
	(423.60,148.01) --
	(425.14,142.53) --
	(426.67,137.46) --
	(428.21,132.79) --
	(429.75,128.51) --
	(431.29,124.63) --
	(432.82,121.15) --
	(434.36,118.07) --
	(435.90,115.39) --
	(437.43,113.11) --
	(438.97,111.22) --
	(440.51,109.73) --
	(442.04,108.65) --
	(443.58,107.96) --
	(445.12,107.67) --
	(446.66,107.66) --
	(448.19,107.66) --
	(449.73,107.66) --
	(451.27,107.66) --
	(452.80,107.66) --
	(454.34,107.66) --
	(455.88,107.66) --
	(457.41,107.66) --
	(458.95,107.66) --
	(460.49,107.66) --
	(462.02,107.66) --
	(463.56,107.66) --
	(465.10,107.66) --
	(466.64,107.66) --
	(468.17,107.66) --
	(469.71,107.66) --
	(471.25,107.66) --
	(472.78,107.66) --
	(474.32,107.66) --
	(475.86,107.66) --
	(477.39,107.66) --
	(478.93,107.66) --
	(480.47,107.66) --
	(482.01,107.66) --
	(483.54,107.66) --
	(485.08,107.66) --
	(486.62,107.66) --
	(488.15,107.66) --
	(489.69,107.66) --
	(491.23,107.66) --
	(492.76,107.66) --
	(494.30,107.66) --
	(495.84,107.66) --
	(497.38,107.66) --
	(498.91,107.66) --
	(500.45,107.66) --
	(501.99,107.66) --
	(503.52,107.66) --
	(505.06,107.66) --
	(506.60,107.66) --
	(508.13,107.66) --
	(509.67,107.66) --
	(511.21,107.66) --
	(512.74,107.66) --
	(514.28,107.66) --
	(515.82,107.66) --
	(517.36,107.66) --
	(518.89,107.66) --
	(520.43,107.66) --
	(521.97,107.66) --
	(523.50,107.66) --
	(525.04,107.66) --
	(526.58,107.66) --
	(528.11,107.66) --
	(529.65,107.66) --
	(531.19,107.66) --
	(532.73,107.66) --
	(534.26,107.66) --
	(535.80,107.66) --
	(537.34,107.66) --
	(538.87,107.66) --
	(540.41,107.66) --
	(541.95,107.66) --
	(543.48,107.66) --
	(545.02,107.66);
\definecolor{drawColor}{RGB}{139,195,74}

\path[draw=drawColor,line width= 1.1pt,line join=round] (394.40,107.66) --
	(395.93,107.66) --
	(397.47,107.66) --
	(399.01,107.66) --
	(400.55,107.66) --
	(402.08,107.66) --
	(403.62,107.66) --
	(405.16,107.66) --
	(406.69,107.67) --
	(408.23,108.14) --
	(409.77,109.27) --
	(411.30,111.05) --
	(412.84,113.49) --
	(414.38,116.59) --
	(415.92,120.34) --
	(417.45,124.75) --
	(418.99,129.82) --
	(420.53,135.54) --
	(422.06,141.69) --
	(423.60,147.29) --
	(425.14,152.24) --
	(426.67,156.55) --
	(428.21,160.22) --
	(429.75,163.24) --
	(431.29,165.62) --
	(432.82,167.36) --
	(434.36,168.45) --
	(435.90,168.90) --
	(437.43,168.70) --
	(438.97,167.86) --
	(440.51,166.37) --
	(442.04,164.24) --
	(443.58,161.47) --
	(445.12,158.05) --
	(446.66,154.22) --
	(448.19,150.53) --
	(449.73,146.99) --
	(451.27,143.60) --
	(452.80,140.37) --
	(454.34,137.29) --
	(455.88,134.36) --
	(457.41,131.58) --
	(458.95,128.95) --
	(460.49,126.48) --
	(462.02,124.16) --
	(463.56,121.99) --
	(465.10,119.98) --
	(466.64,118.12) --
	(468.17,116.41) --
	(469.71,114.85) --
	(471.25,113.44) --
	(472.78,112.19) --
	(474.32,111.09) --
	(475.86,110.15) --
	(477.39,109.35) --
	(478.93,108.71) --
	(480.47,108.22) --
	(482.01,107.88) --
	(483.54,107.70) --
	(485.08,107.66) --
	(486.62,107.66) --
	(488.15,107.66) --
	(489.69,107.66) --
	(491.23,107.66) --
	(492.76,107.66) --
	(494.30,107.66) --
	(495.84,107.66) --
	(497.38,107.66) --
	(498.91,107.66) --
	(500.45,107.66) --
	(501.99,107.66) --
	(503.52,107.66) --
	(505.06,107.66) --
	(506.60,107.66) --
	(508.13,107.66) --
	(509.67,107.66) --
	(511.21,107.66) --
	(512.74,107.66) --
	(514.28,107.66) --
	(515.82,107.66) --
	(517.36,107.66) --
	(518.89,107.66) --
	(520.43,107.66) --
	(521.97,107.66) --
	(523.50,107.66) --
	(525.04,107.66) --
	(526.58,107.66) --
	(528.11,107.66) --
	(529.65,107.66) --
	(531.19,107.66) --
	(532.73,107.66) --
	(534.26,107.66) --
	(535.80,107.66) --
	(537.34,107.66) --
	(538.87,107.66) --
	(540.41,107.66) --
	(541.95,107.66) --
	(543.48,107.66) --
	(545.02,107.66);
\definecolor{drawColor}{RGB}{255,235,58}

\path[draw=drawColor,line width= 1.1pt,line join=round] (394.40,107.66) --
	(395.93,107.66) --
	(397.47,107.66) --
	(399.01,107.66) --
	(400.55,107.66) --
	(402.08,107.66) --
	(403.62,107.66) --
	(405.16,107.66) --
	(406.69,107.66) --
	(408.23,107.66) --
	(409.77,107.66) --
	(411.30,107.66) --
	(412.84,107.66) --
	(414.38,107.66) --
	(415.92,107.66) --
	(417.45,107.66) --
	(418.99,107.66) --
	(420.53,107.66) --
	(422.06,107.70) --
	(423.60,107.97) --
	(425.14,108.49) --
	(426.67,109.25) --
	(428.21,110.26) --
	(429.75,111.52) --
	(431.29,113.01) --
	(432.82,114.76) --
	(434.36,116.75) --
	(435.90,118.98) --
	(437.43,121.46) --
	(438.97,124.19) --
	(440.51,127.16) --
	(442.04,130.38) --
	(443.58,133.84) --
	(445.12,137.55) --
	(446.66,141.36) --
	(448.19,144.93) --
	(449.73,148.25) --
	(451.27,151.32) --
	(452.80,154.15) --
	(454.34,156.73) --
	(455.88,159.06) --
	(457.41,161.14) --
	(458.95,162.97) --
	(460.49,164.55) --
	(462.02,165.89) --
	(463.56,166.98) --
	(465.10,167.82) --
	(466.64,168.41) --
	(468.17,168.75) --
	(469.71,168.84) --
	(471.25,168.69) --
	(472.78,168.29) --
	(474.32,167.64) --
	(475.86,166.74) --
	(477.39,165.59) --
	(478.93,164.19) --
	(480.47,162.55) --
	(482.01,160.66) --
	(483.54,158.52) --
	(485.08,156.14) --
	(486.62,153.75) --
	(488.15,151.41) --
	(489.69,149.14) --
	(491.23,146.93) --
	(492.76,144.78) --
	(494.30,142.69) --
	(495.84,140.66) --
	(497.38,138.69) --
	(498.91,136.78) --
	(500.45,134.93) --
	(501.99,133.14) --
	(503.52,131.41) --
	(505.06,129.75) --
	(506.60,128.14) --
	(508.13,126.60) --
	(509.67,125.11) --
	(511.21,123.69) --
	(512.74,122.32) --
	(514.28,121.02) --
	(515.82,119.78) --
	(517.36,118.60) --
	(518.89,117.48) --
	(520.43,116.41) --
	(521.97,115.42) --
	(523.50,114.48) --
	(525.04,113.60) --
	(526.58,112.78) --
	(528.11,112.02) --
	(529.65,111.32) --
	(531.19,110.69) --
	(532.73,110.11) --
	(534.26,109.60) --
	(535.80,109.14) --
	(537.34,108.75) --
	(538.87,108.42) --
	(540.41,108.14) --
	(541.95,107.93) --
	(543.48,107.78) --
	(545.02,107.69);
\definecolor{drawColor}{RGB}{255,152,0}

\path[draw=drawColor,line width= 1.1pt,line join=round] (394.40,107.66) --
	(395.93,107.66) --
	(397.47,107.66) --
	(399.01,107.66) --
	(400.55,107.66) --
	(402.08,107.66) --
	(403.62,107.66) --
	(405.16,107.66) --
	(406.69,107.66) --
	(408.23,107.66) --
	(409.77,107.66) --
	(411.30,107.66) --
	(412.84,107.66) --
	(414.38,107.66) --
	(415.92,107.66) --
	(417.45,107.66) --
	(418.99,107.66) --
	(420.53,107.66) --
	(422.06,107.66) --
	(423.60,107.66) --
	(425.14,107.66) --
	(426.67,107.66) --
	(428.21,107.66) --
	(429.75,107.66) --
	(431.29,107.66) --
	(432.82,107.66) --
	(434.36,107.66) --
	(435.90,107.66) --
	(437.43,107.66) --
	(438.97,107.66) --
	(440.51,107.66) --
	(442.04,107.66) --
	(443.58,107.66) --
	(445.12,107.66) --
	(446.66,107.69) --
	(448.19,107.81) --
	(449.73,108.03) --
	(451.27,108.34) --
	(452.80,108.75) --
	(454.34,109.25) --
	(455.88,109.86) --
	(457.41,110.55) --
	(458.95,111.34) --
	(460.49,112.23) --
	(462.02,113.22) --
	(463.56,114.30) --
	(465.10,115.47) --
	(466.64,116.74) --
	(468.17,118.11) --
	(469.71,119.58) --
	(471.25,121.13) --
	(472.78,122.79) --
	(474.32,124.54) --
	(475.86,126.39) --
	(477.39,128.33) --
	(478.93,130.37) --
	(480.47,132.50) --
	(482.01,134.73) --
	(483.54,137.06) --
	(485.08,139.47) --
	(486.62,141.82) --
	(488.15,144.07) --
	(489.69,146.20) --
	(491.23,148.23) --
	(492.76,150.14) --
	(494.30,151.95) --
	(495.84,153.64) --
	(497.38,155.22) --
	(498.91,156.70) --
	(500.45,158.06) --
	(501.99,159.32) --
	(503.52,160.46) --
	(505.06,161.49) --
	(506.60,162.42) --
	(508.13,163.23) --
	(509.67,163.94) --
	(511.21,164.53) --
	(512.74,165.01) --
	(514.28,165.39) --
	(515.82,165.65) --
	(517.36,165.80) --
	(518.89,165.85) --
	(520.43,165.78) --
	(521.97,165.60) --
	(523.50,165.32) --
	(525.04,164.92) --
	(526.58,164.41) --
	(528.11,163.80) --
	(529.65,163.07) --
	(531.19,162.23) --
	(532.73,161.28) --
	(534.26,160.23) --
	(535.80,159.06) --
	(537.34,157.78) --
	(538.87,156.39) --
	(540.41,154.90) --
	(541.95,153.29) --
	(543.48,151.57) --
	(545.02,149.74);
\definecolor{drawColor}{RGB}{121,84,71}

\path[draw=drawColor,line width= 1.1pt,line join=round] (394.40,107.66) --
	(395.93,107.66) --
	(397.47,107.66) --
	(399.01,107.66) --
	(400.55,107.66) --
	(402.08,107.66) --
	(403.62,107.66) --
	(405.16,107.66) --
	(406.69,107.66) --
	(408.23,107.66) --
	(409.77,107.66) --
	(411.30,107.66) --
	(412.84,107.66) --
	(414.38,107.66) --
	(415.92,107.66) --
	(417.45,107.66) --
	(418.99,107.66) --
	(420.53,107.66) --
	(422.06,107.66) --
	(423.60,107.66) --
	(425.14,107.66) --
	(426.67,107.66) --
	(428.21,107.66) --
	(429.75,107.66) --
	(431.29,107.66) --
	(432.82,107.66) --
	(434.36,107.66) --
	(435.90,107.66) --
	(437.43,107.66) --
	(438.97,107.66) --
	(440.51,107.66) --
	(442.04,107.66) --
	(443.58,107.66) --
	(445.12,107.66) --
	(446.66,107.66) --
	(448.19,107.66) --
	(449.73,107.66) --
	(451.27,107.66) --
	(452.80,107.66) --
	(454.34,107.66) --
	(455.88,107.66) --
	(457.41,107.66) --
	(458.95,107.66) --
	(460.49,107.66) --
	(462.02,107.66) --
	(463.56,107.66) --
	(465.10,107.66) --
	(466.64,107.66) --
	(468.17,107.66) --
	(469.71,107.66) --
	(471.25,107.66) --
	(472.78,107.66) --
	(474.32,107.66) --
	(475.86,107.66) --
	(477.39,107.66) --
	(478.93,107.66) --
	(480.47,107.66) --
	(482.01,107.66) --
	(483.54,107.66) --
	(485.08,107.66) --
	(486.62,107.70) --
	(488.15,107.79) --
	(489.69,107.93) --
	(491.23,108.11) --
	(492.76,108.35) --
	(494.30,108.64) --
	(495.84,108.97) --
	(497.38,109.36) --
	(498.91,109.79) --
	(500.45,110.28) --
	(501.99,110.81) --
	(503.52,111.39) --
	(505.06,112.03) --
	(506.60,112.71) --
	(508.13,113.44) --
	(509.67,114.22) --
	(511.21,115.05) --
	(512.74,115.93) --
	(514.28,116.86) --
	(515.82,117.84) --
	(517.36,118.87) --
	(518.89,119.95) --
	(520.43,121.07) --
	(521.97,122.25) --
	(523.50,123.48) --
	(525.04,124.75) --
	(526.58,126.08) --
	(528.11,127.45) --
	(529.65,128.88) --
	(531.19,130.35) --
	(532.73,131.87) --
	(534.26,133.44) --
	(535.80,135.07) --
	(537.34,136.74) --
	(538.87,138.46) --
	(540.41,140.23) --
	(541.95,142.05) --
	(543.48,143.92) --
	(545.02,145.84);
\end{scope}
\begin{scope}
\path[clip] (386.87, 21.76) rectangle (552.55, 98.66);
\definecolor{drawColor}{gray}{0.92}

\path[draw=drawColor,line width= 0.3pt,line join=round] (386.87, 35.29) --
	(552.55, 35.29);

\path[draw=drawColor,line width= 0.3pt,line join=round] (386.87, 55.36) --
	(552.55, 55.36);

\path[draw=drawColor,line width= 0.3pt,line join=round] (386.87, 75.44) --
	(552.55, 75.44);

\path[draw=drawColor,line width= 0.3pt,line join=round] (386.87, 95.51) --
	(552.55, 95.51);

\path[draw=drawColor,line width= 0.3pt,line join=round] (412.07, 21.76) --
	(412.07, 98.66);

\path[draw=drawColor,line width= 0.3pt,line join=round] (450.50, 21.76) --
	(450.50, 98.66);

\path[draw=drawColor,line width= 0.3pt,line join=round] (488.92, 21.76) --
	(488.92, 98.66);

\path[draw=drawColor,line width= 0.3pt,line join=round] (527.35, 21.76) --
	(527.35, 98.66);

\path[draw=drawColor,line width= 0.6pt,line join=round] (386.87, 25.25) --
	(552.55, 25.25);

\path[draw=drawColor,line width= 0.6pt,line join=round] (386.87, 45.33) --
	(552.55, 45.33);

\path[draw=drawColor,line width= 0.6pt,line join=round] (386.87, 65.40) --
	(552.55, 65.40);

\path[draw=drawColor,line width= 0.6pt,line join=round] (386.87, 85.47) --
	(552.55, 85.47);

\path[draw=drawColor,line width= 0.6pt,line join=round] (392.86, 21.76) --
	(392.86, 98.66);

\path[draw=drawColor,line width= 0.6pt,line join=round] (431.29, 21.76) --
	(431.29, 98.66);

\path[draw=drawColor,line width= 0.6pt,line join=round] (469.71, 21.76) --
	(469.71, 98.66);

\path[draw=drawColor,line width= 0.6pt,line join=round] (508.13, 21.76) --
	(508.13, 98.66);

\path[draw=drawColor,line width= 0.6pt,line join=round] (546.56, 21.76) --
	(546.56, 98.66);
\definecolor{drawColor}{RGB}{155,38,176}

\path[draw=drawColor,line width= 1.1pt,line join=round] (394.40, 32.73) --
	(395.93, 27.39) --
	(397.47, 25.29) --
	(399.01, 25.25) --
	(400.55, 25.25) --
	(402.08, 25.25) --
	(403.62, 25.25) --
	(405.16, 25.25) --
	(406.69, 25.25) --
	(408.23, 25.25) --
	(409.77, 25.25) --
	(411.30, 25.25) --
	(412.84, 25.25) --
	(414.38, 25.25) --
	(415.92, 25.25) --
	(417.45, 25.25) --
	(418.99, 25.25) --
	(420.53, 25.25) --
	(422.06, 25.25) --
	(423.60, 25.25) --
	(425.14, 25.25) --
	(426.67, 25.25) --
	(428.21, 25.25) --
	(429.75, 25.25) --
	(431.29, 25.25) --
	(432.82, 25.25) --
	(434.36, 25.25) --
	(435.90, 25.25) --
	(437.43, 25.25) --
	(438.97, 25.25) --
	(440.51, 25.25) --
	(442.04, 25.25) --
	(443.58, 25.25) --
	(445.12, 25.25) --
	(446.66, 25.25) --
	(448.19, 25.25) --
	(449.73, 25.25) --
	(451.27, 25.25) --
	(452.80, 25.25) --
	(454.34, 25.25) --
	(455.88, 25.25) --
	(457.41, 25.25) --
	(458.95, 25.25) --
	(460.49, 25.25) --
	(462.02, 25.25) --
	(463.56, 25.25) --
	(465.10, 25.25) --
	(466.64, 25.25) --
	(468.17, 25.25) --
	(469.71, 25.25) --
	(471.25, 25.25) --
	(472.78, 25.25) --
	(474.32, 25.25) --
	(475.86, 25.25) --
	(477.39, 25.25) --
	(478.93, 25.25) --
	(480.47, 25.25) --
	(482.01, 25.25) --
	(483.54, 25.25) --
	(485.08, 25.25) --
	(486.62, 25.25) --
	(488.15, 25.25) --
	(489.69, 25.25) --
	(491.23, 25.25) --
	(492.76, 25.25) --
	(494.30, 25.25) --
	(495.84, 25.25) --
	(497.38, 25.25) --
	(498.91, 25.25) --
	(500.45, 25.25) --
	(501.99, 25.25) --
	(503.52, 25.25) --
	(505.06, 25.25) --
	(506.60, 25.25) --
	(508.13, 25.25) --
	(509.67, 25.25) --
	(511.21, 25.25) --
	(512.74, 25.25) --
	(514.28, 25.25) --
	(515.82, 25.25) --
	(517.36, 25.25) --
	(518.89, 25.25) --
	(520.43, 25.25) --
	(521.97, 25.25) --
	(523.50, 25.25) --
	(525.04, 25.25) --
	(526.58, 25.25) --
	(528.11, 25.25) --
	(529.65, 25.25) --
	(531.19, 25.25) --
	(532.73, 25.25) --
	(534.26, 25.25) --
	(535.80, 25.25) --
	(537.34, 25.25) --
	(538.87, 25.25) --
	(540.41, 25.25) --
	(541.95, 25.25) --
	(543.48, 25.25) --
	(545.02, 25.25);
\definecolor{drawColor}{RGB}{63,81,180}

\path[draw=drawColor,line width= 1.1pt,line join=round] (394.40, 95.17) --
	(395.93, 91.80) --
	(397.47, 79.38) --
	(399.01, 62.38) --
	(400.55, 48.49) --
	(402.08, 37.84) --
	(403.62, 30.43) --
	(405.16, 26.26) --
	(406.69, 25.25) --
	(408.23, 25.25) --
	(409.77, 25.25) --
	(411.30, 25.25) --
	(412.84, 25.25) --
	(414.38, 25.25) --
	(415.92, 25.25) --
	(417.45, 25.25) --
	(418.99, 25.25) --
	(420.53, 25.25) --
	(422.06, 25.25) --
	(423.60, 25.25) --
	(425.14, 25.25) --
	(426.67, 25.25) --
	(428.21, 25.25) --
	(429.75, 25.25) --
	(431.29, 25.25) --
	(432.82, 25.25) --
	(434.36, 25.25) --
	(435.90, 25.25) --
	(437.43, 25.25) --
	(438.97, 25.25) --
	(440.51, 25.25) --
	(442.04, 25.25) --
	(443.58, 25.25) --
	(445.12, 25.25) --
	(446.66, 25.25) --
	(448.19, 25.25) --
	(449.73, 25.25) --
	(451.27, 25.25) --
	(452.80, 25.25) --
	(454.34, 25.25) --
	(455.88, 25.25) --
	(457.41, 25.25) --
	(458.95, 25.25) --
	(460.49, 25.25) --
	(462.02, 25.25) --
	(463.56, 25.25) --
	(465.10, 25.25) --
	(466.64, 25.25) --
	(468.17, 25.25) --
	(469.71, 25.25) --
	(471.25, 25.25) --
	(472.78, 25.25) --
	(474.32, 25.25) --
	(475.86, 25.25) --
	(477.39, 25.25) --
	(478.93, 25.25) --
	(480.47, 25.25) --
	(482.01, 25.25) --
	(483.54, 25.25) --
	(485.08, 25.25) --
	(486.62, 25.25) --
	(488.15, 25.25) --
	(489.69, 25.25) --
	(491.23, 25.25) --
	(492.76, 25.25) --
	(494.30, 25.25) --
	(495.84, 25.25) --
	(497.38, 25.25) --
	(498.91, 25.25) --
	(500.45, 25.25) --
	(501.99, 25.25) --
	(503.52, 25.25) --
	(505.06, 25.25) --
	(506.60, 25.25) --
	(508.13, 25.25) --
	(509.67, 25.25) --
	(511.21, 25.25) --
	(512.74, 25.25) --
	(514.28, 25.25) --
	(515.82, 25.25) --
	(517.36, 25.25) --
	(518.89, 25.25) --
	(520.43, 25.25) --
	(521.97, 25.25) --
	(523.50, 25.25) --
	(525.04, 25.25) --
	(526.58, 25.25) --
	(528.11, 25.25) --
	(529.65, 25.25) --
	(531.19, 25.25) --
	(532.73, 25.25) --
	(534.26, 25.25) --
	(535.80, 25.25) --
	(537.34, 25.25) --
	(538.87, 25.25) --
	(540.41, 25.25) --
	(541.95, 25.25) --
	(543.48, 25.25) --
	(545.02, 25.25);
\definecolor{drawColor}{RGB}{2,169,243}

\path[draw=drawColor,line width= 1.1pt,line join=round] (394.40, 28.16) --
	(395.93, 36.86) --
	(397.47, 51.38) --
	(399.01, 67.74) --
	(400.55, 79.11) --
	(402.08, 85.36) --
	(403.62, 86.51) --
	(405.16, 82.55) --
	(406.69, 73.63) --
	(408.23, 63.88) --
	(409.77, 55.22) --
	(411.30, 47.66) --
	(412.84, 41.20) --
	(414.38, 35.84) --
	(415.92, 31.57) --
	(417.45, 28.39) --
	(418.99, 26.32) --
	(420.53, 25.34) --
	(422.06, 25.25) --
	(423.60, 25.25) --
	(425.14, 25.25) --
	(426.67, 25.25) --
	(428.21, 25.25) --
	(429.75, 25.25) --
	(431.29, 25.25) --
	(432.82, 25.25) --
	(434.36, 25.25) --
	(435.90, 25.25) --
	(437.43, 25.25) --
	(438.97, 25.25) --
	(440.51, 25.25) --
	(442.04, 25.25) --
	(443.58, 25.25) --
	(445.12, 25.25) --
	(446.66, 25.25) --
	(448.19, 25.25) --
	(449.73, 25.25) --
	(451.27, 25.25) --
	(452.80, 25.25) --
	(454.34, 25.25) --
	(455.88, 25.25) --
	(457.41, 25.25) --
	(458.95, 25.25) --
	(460.49, 25.25) --
	(462.02, 25.25) --
	(463.56, 25.25) --
	(465.10, 25.25) --
	(466.64, 25.25) --
	(468.17, 25.25) --
	(469.71, 25.25) --
	(471.25, 25.25) --
	(472.78, 25.25) --
	(474.32, 25.25) --
	(475.86, 25.25) --
	(477.39, 25.25) --
	(478.93, 25.25) --
	(480.47, 25.25) --
	(482.01, 25.25) --
	(483.54, 25.25) --
	(485.08, 25.25) --
	(486.62, 25.25) --
	(488.15, 25.25) --
	(489.69, 25.25) --
	(491.23, 25.25) --
	(492.76, 25.25) --
	(494.30, 25.25) --
	(495.84, 25.25) --
	(497.38, 25.25) --
	(498.91, 25.25) --
	(500.45, 25.25) --
	(501.99, 25.25) --
	(503.52, 25.25) --
	(505.06, 25.25) --
	(506.60, 25.25) --
	(508.13, 25.25) --
	(509.67, 25.25) --
	(511.21, 25.25) --
	(512.74, 25.25) --
	(514.28, 25.25) --
	(515.82, 25.25) --
	(517.36, 25.25) --
	(518.89, 25.25) --
	(520.43, 25.25) --
	(521.97, 25.25) --
	(523.50, 25.25) --
	(525.04, 25.25) --
	(526.58, 25.25) --
	(528.11, 25.25) --
	(529.65, 25.25) --
	(531.19, 25.25) --
	(532.73, 25.25) --
	(534.26, 25.25) --
	(535.80, 25.25) --
	(537.34, 25.25) --
	(538.87, 25.25) --
	(540.41, 25.25) --
	(541.95, 25.25) --
	(543.48, 25.25) --
	(545.02, 25.25);
\definecolor{drawColor}{RGB}{0,150,135}

\path[draw=drawColor,line width= 1.1pt,line join=round] (394.40, 25.25) --
	(395.93, 25.25) --
	(397.47, 25.25) --
	(399.01, 25.93) --
	(400.55, 28.46) --
	(402.08, 32.85) --
	(403.62, 39.11) --
	(405.16, 47.24) --
	(406.69, 57.16) --
	(408.23, 66.44) --
	(409.77, 73.97) --
	(411.30, 79.74) --
	(412.84, 83.77) --
	(414.38, 86.03) --
	(415.92, 86.55) --
	(417.45, 85.31) --
	(418.99, 82.32) --
	(420.53, 77.58) --
	(422.06, 71.47) --
	(423.60, 65.60) --
	(425.14, 60.13) --
	(426.67, 55.06) --
	(428.21, 50.38) --
	(429.75, 46.10) --
	(431.29, 42.23) --
	(432.82, 38.75) --
	(434.36, 35.67) --
	(435.90, 32.98) --
	(437.43, 30.70) --
	(438.97, 28.82) --
	(440.51, 27.33) --
	(442.04, 26.24) --
	(443.58, 25.55) --
	(445.12, 25.26) --
	(446.66, 25.25) --
	(448.19, 25.25) --
	(449.73, 25.25) --
	(451.27, 25.25) --
	(452.80, 25.25) --
	(454.34, 25.25) --
	(455.88, 25.25) --
	(457.41, 25.25) --
	(458.95, 25.25) --
	(460.49, 25.25) --
	(462.02, 25.25) --
	(463.56, 25.25) --
	(465.10, 25.25) --
	(466.64, 25.25) --
	(468.17, 25.25) --
	(469.71, 25.25) --
	(471.25, 25.25) --
	(472.78, 25.25) --
	(474.32, 25.25) --
	(475.86, 25.25) --
	(477.39, 25.25) --
	(478.93, 25.25) --
	(480.47, 25.25) --
	(482.01, 25.25) --
	(483.54, 25.25) --
	(485.08, 25.25) --
	(486.62, 25.25) --
	(488.15, 25.25) --
	(489.69, 25.25) --
	(491.23, 25.25) --
	(492.76, 25.25) --
	(494.30, 25.25) --
	(495.84, 25.25) --
	(497.38, 25.25) --
	(498.91, 25.25) --
	(500.45, 25.25) --
	(501.99, 25.25) --
	(503.52, 25.25) --
	(505.06, 25.25) --
	(506.60, 25.25) --
	(508.13, 25.25) --
	(509.67, 25.25) --
	(511.21, 25.25) --
	(512.74, 25.25) --
	(514.28, 25.25) --
	(515.82, 25.25) --
	(517.36, 25.25) --
	(518.89, 25.25) --
	(520.43, 25.25) --
	(521.97, 25.25) --
	(523.50, 25.25) --
	(525.04, 25.25) --
	(526.58, 25.25) --
	(528.11, 25.25) --
	(529.65, 25.25) --
	(531.19, 25.25) --
	(532.73, 25.25) --
	(534.26, 25.25) --
	(535.80, 25.25) --
	(537.34, 25.25) --
	(538.87, 25.25) --
	(540.41, 25.25) --
	(541.95, 25.25) --
	(543.48, 25.25) --
	(545.02, 25.25);
\definecolor{drawColor}{RGB}{139,195,74}

\path[draw=drawColor,line width= 1.1pt,line join=round] (394.40, 25.25) --
	(395.93, 25.25) --
	(397.47, 25.25) --
	(399.01, 25.25) --
	(400.55, 25.25) --
	(402.08, 25.25) --
	(403.62, 25.25) --
	(405.16, 25.25) --
	(406.69, 25.27) --
	(408.23, 25.74) --
	(409.77, 26.86) --
	(411.30, 28.65) --
	(412.84, 31.08) --
	(414.38, 34.18) --
	(415.92, 37.94) --
	(417.45, 42.35) --
	(418.99, 47.41) --
	(420.53, 53.14) --
	(422.06, 59.28) --
	(423.60, 64.88) --
	(425.14, 69.84) --
	(426.67, 74.15) --
	(428.21, 77.82) --
	(429.75, 80.84) --
	(431.29, 83.22) --
	(432.82, 84.95) --
	(434.36, 86.04) --
	(435.90, 86.49) --
	(437.43, 86.29) --
	(438.97, 85.45) --
	(440.51, 83.97) --
	(442.04, 81.84) --
	(443.58, 79.06) --
	(445.12, 75.64) --
	(446.66, 71.82) --
	(448.19, 68.13) --
	(449.73, 64.59) --
	(451.27, 61.20) --
	(452.80, 57.96) --
	(454.34, 54.88) --
	(455.88, 51.95) --
	(457.41, 49.17) --
	(458.95, 46.55) --
	(460.49, 44.08) --
	(462.02, 41.76) --
	(463.56, 39.59) --
	(465.10, 37.57) --
	(466.64, 35.71) --
	(468.17, 34.00) --
	(469.71, 32.44) --
	(471.25, 31.04) --
	(472.78, 29.79) --
	(474.32, 28.69) --
	(475.86, 27.74) --
	(477.39, 26.95) --
	(478.93, 26.30) --
	(480.47, 25.81) --
	(482.01, 25.48) --
	(483.54, 25.29) --
	(485.08, 25.25) --
	(486.62, 25.25) --
	(488.15, 25.25) --
	(489.69, 25.25) --
	(491.23, 25.25) --
	(492.76, 25.25) --
	(494.30, 25.25) --
	(495.84, 25.25) --
	(497.38, 25.25) --
	(498.91, 25.25) --
	(500.45, 25.25) --
	(501.99, 25.25) --
	(503.52, 25.25) --
	(505.06, 25.25) --
	(506.60, 25.25) --
	(508.13, 25.25) --
	(509.67, 25.25) --
	(511.21, 25.25) --
	(512.74, 25.25) --
	(514.28, 25.25) --
	(515.82, 25.25) --
	(517.36, 25.25) --
	(518.89, 25.25) --
	(520.43, 25.25) --
	(521.97, 25.25) --
	(523.50, 25.25) --
	(525.04, 25.25) --
	(526.58, 25.25) --
	(528.11, 25.25) --
	(529.65, 25.25) --
	(531.19, 25.25) --
	(532.73, 25.25) --
	(534.26, 25.25) --
	(535.80, 25.25) --
	(537.34, 25.25) --
	(538.87, 25.25) --
	(540.41, 25.25) --
	(541.95, 25.25) --
	(543.48, 25.25) --
	(545.02, 25.25);
\definecolor{drawColor}{RGB}{255,235,58}

\path[draw=drawColor,line width= 1.1pt,line join=round] (394.40, 25.25) --
	(395.93, 25.25) --
	(397.47, 25.25) --
	(399.01, 25.25) --
	(400.55, 25.25) --
	(402.08, 25.25) --
	(403.62, 25.25) --
	(405.16, 25.25) --
	(406.69, 25.25) --
	(408.23, 25.25) --
	(409.77, 25.25) --
	(411.30, 25.25) --
	(412.84, 25.25) --
	(414.38, 25.25) --
	(415.92, 25.25) --
	(417.45, 25.25) --
	(418.99, 25.25) --
	(420.53, 25.25) --
	(422.06, 25.30) --
	(423.60, 25.57) --
	(425.14, 26.09) --
	(426.67, 26.85) --
	(428.21, 27.86) --
	(429.75, 29.11) --
	(431.29, 30.61) --
	(432.82, 32.35) --
	(434.36, 34.34) --
	(435.90, 36.58) --
	(437.43, 39.06) --
	(438.97, 41.79) --
	(440.51, 44.76) --
	(442.04, 47.97) --
	(443.58, 51.44) --
	(445.12, 55.14) --
	(446.66, 58.95) --
	(448.19, 62.52) --
	(449.73, 65.84) --
	(451.27, 68.92) --
	(452.80, 71.74) --
	(454.34, 74.32) --
	(455.88, 76.65) --
	(457.41, 78.73) --
	(458.95, 80.56) --
	(460.49, 82.15) --
	(462.02, 83.48) --
	(463.56, 84.57) --
	(465.10, 85.41) --
	(466.64, 86.00) --
	(468.17, 86.34) --
	(469.71, 86.44) --
	(471.25, 86.28) --
	(472.78, 85.88) --
	(474.32, 85.23) --
	(475.86, 84.33) --
	(477.39, 83.18) --
	(478.93, 81.79) --
	(480.47, 80.14) --
	(482.01, 78.25) --
	(483.54, 76.11) --
	(485.08, 73.73) --
	(486.62, 71.34) --
	(488.15, 69.01) --
	(489.69, 66.73) --
	(491.23, 64.52) --
	(492.76, 62.37) --
	(494.30, 60.28) --
	(495.84, 58.25) --
	(497.38, 56.28) --
	(498.91, 54.37) --
	(500.45, 52.52) --
	(501.99, 50.74) --
	(503.52, 49.01) --
	(505.06, 47.34) --
	(506.60, 45.74) --
	(508.13, 44.19) --
	(509.67, 42.71) --
	(511.21, 41.28) --
	(512.74, 39.92) --
	(514.28, 38.62) --
	(515.82, 37.37) --
	(517.36, 36.19) --
	(518.89, 35.07) --
	(520.43, 34.01) --
	(521.97, 33.01) --
	(523.50, 32.07) --
	(525.04, 31.19) --
	(526.58, 30.37) --
	(528.11, 29.62) --
	(529.65, 28.92) --
	(531.19, 28.28) --
	(532.73, 27.71) --
	(534.26, 27.19) --
	(535.80, 26.74) --
	(537.34, 26.34) --
	(538.87, 26.01) --
	(540.41, 25.74) --
	(541.95, 25.53) --
	(543.48, 25.37) --
	(545.02, 25.28);
\definecolor{drawColor}{RGB}{255,152,0}

\path[draw=drawColor,line width= 1.1pt,line join=round] (394.40, 25.25) --
	(395.93, 25.25) --
	(397.47, 25.25) --
	(399.01, 25.25) --
	(400.55, 25.25) --
	(402.08, 25.25) --
	(403.62, 25.25) --
	(405.16, 25.25) --
	(406.69, 25.25) --
	(408.23, 25.25) --
	(409.77, 25.25) --
	(411.30, 25.25) --
	(412.84, 25.25) --
	(414.38, 25.25) --
	(415.92, 25.25) --
	(417.45, 25.25) --
	(418.99, 25.25) --
	(420.53, 25.25) --
	(422.06, 25.25) --
	(423.60, 25.25) --
	(425.14, 25.25) --
	(426.67, 25.25) --
	(428.21, 25.25) --
	(429.75, 25.25) --
	(431.29, 25.25) --
	(432.82, 25.25) --
	(434.36, 25.25) --
	(435.90, 25.25) --
	(437.43, 25.25) --
	(438.97, 25.25) --
	(440.51, 25.25) --
	(442.04, 25.25) --
	(443.58, 25.25) --
	(445.12, 25.25) --
	(446.66, 25.28) --
	(448.19, 25.40) --
	(449.73, 25.62) --
	(451.27, 25.93) --
	(452.80, 26.34) --
	(454.34, 26.85) --
	(455.88, 27.45) --
	(457.41, 28.15) --
	(458.95, 28.94) --
	(460.49, 29.83) --
	(462.02, 30.81) --
	(463.56, 31.89) --
	(465.10, 33.07) --
	(466.64, 34.34) --
	(468.17, 35.71) --
	(469.71, 37.17) --
	(471.25, 38.73) --
	(472.78, 40.38) --
	(474.32, 42.13) --
	(475.86, 43.98) --
	(477.39, 45.92) --
	(478.93, 47.96) --
	(480.47, 50.10) --
	(482.01, 52.33) --
	(483.54, 54.65) --
	(485.08, 57.06) --
	(486.62, 59.44) --
	(488.15, 61.74) --
	(489.69, 63.96) --
	(491.23, 66.09) --
	(492.76, 68.15) --
	(494.30, 70.13) --
	(495.84, 72.02) --
	(497.38, 73.84) --
	(498.91, 75.57) --
	(500.45, 77.23) --
	(501.99, 78.80) --
	(503.52, 80.30) --
	(505.06, 81.71) --
	(506.60, 83.04) --
	(508.13, 84.30) --
	(509.67, 85.47) --
	(511.21, 86.56) --
	(512.74, 87.57) --
	(514.28, 88.50) --
	(515.82, 89.35) --
	(517.36, 90.12) --
	(518.89, 90.81) --
	(520.43, 91.42) --
	(521.97, 91.95) --
	(523.50, 92.40) --
	(525.04, 92.77) --
	(526.58, 93.06) --
	(528.11, 93.27) --
	(529.65, 93.39) --
	(531.19, 93.44) --
	(532.73, 93.41) --
	(534.26, 93.29) --
	(535.80, 93.10) --
	(537.34, 92.82) --
	(538.87, 92.47) --
	(540.41, 92.03) --
	(541.95, 91.52) --
	(543.48, 90.92) --
	(545.02, 90.24);
\definecolor{drawColor}{RGB}{121,84,71}

\path[draw=drawColor,line width= 1.1pt,line join=round] (394.40, 25.25) --
	(395.93, 25.25) --
	(397.47, 25.25) --
	(399.01, 25.25) --
	(400.55, 25.25) --
	(402.08, 25.25) --
	(403.62, 25.25) --
	(405.16, 25.25) --
	(406.69, 25.25) --
	(408.23, 25.25) --
	(409.77, 25.25) --
	(411.30, 25.25) --
	(412.84, 25.25) --
	(414.38, 25.25) --
	(415.92, 25.25) --
	(417.45, 25.25) --
	(418.99, 25.25) --
	(420.53, 25.25) --
	(422.06, 25.25) --
	(423.60, 25.25) --
	(425.14, 25.25) --
	(426.67, 25.25) --
	(428.21, 25.25) --
	(429.75, 25.25) --
	(431.29, 25.25) --
	(432.82, 25.25) --
	(434.36, 25.25) --
	(435.90, 25.25) --
	(437.43, 25.25) --
	(438.97, 25.25) --
	(440.51, 25.25) --
	(442.04, 25.25) --
	(443.58, 25.25) --
	(445.12, 25.25) --
	(446.66, 25.25) --
	(448.19, 25.25) --
	(449.73, 25.25) --
	(451.27, 25.25) --
	(452.80, 25.25) --
	(454.34, 25.25) --
	(455.88, 25.25) --
	(457.41, 25.25) --
	(458.95, 25.25) --
	(460.49, 25.25) --
	(462.02, 25.25) --
	(463.56, 25.25) --
	(465.10, 25.25) --
	(466.64, 25.25) --
	(468.17, 25.25) --
	(469.71, 25.25) --
	(471.25, 25.25) --
	(472.78, 25.25) --
	(474.32, 25.25) --
	(475.86, 25.25) --
	(477.39, 25.25) --
	(478.93, 25.25) --
	(480.47, 25.25) --
	(482.01, 25.25) --
	(483.54, 25.25) --
	(485.08, 25.25) --
	(486.62, 25.27) --
	(488.15, 25.30) --
	(489.69, 25.36) --
	(491.23, 25.44) --
	(492.76, 25.53) --
	(494.30, 25.64) --
	(495.84, 25.78) --
	(497.38, 25.93) --
	(498.91, 26.11) --
	(500.45, 26.30) --
	(501.99, 26.51) --
	(503.52, 26.75) --
	(505.06, 27.00) --
	(506.60, 27.27) --
	(508.13, 27.57) --
	(509.67, 27.88) --
	(511.21, 28.21) --
	(512.74, 28.56) --
	(514.28, 28.93) --
	(515.82, 29.33) --
	(517.36, 29.74) --
	(518.89, 30.17) --
	(520.43, 30.62) --
	(521.97, 31.09) --
	(523.50, 31.58) --
	(525.04, 32.09) --
	(526.58, 32.62) --
	(528.11, 33.17) --
	(529.65, 33.74) --
	(531.19, 34.33) --
	(532.73, 34.94) --
	(534.26, 35.57) --
	(535.80, 36.22) --
	(537.34, 36.88) --
	(538.87, 37.57) --
	(540.41, 38.28) --
	(541.95, 39.01) --
	(543.48, 39.76) --
	(545.02, 40.52);
\end{scope}
\begin{scope}
\path[clip] ( 32.73,263.47) rectangle (198.42,283.58);
\definecolor{drawColor}{gray}{0.10}

\node[text=drawColor,anchor=base,inner sep=0pt, outer sep=0pt, scale=  1.07] at (115.58,269.12) {$c = -4$};
\end{scope}
\begin{scope}
\path[clip] (209.80,263.47) rectangle (375.49,283.58);
\definecolor{drawColor}{gray}{0.10}

\node[text=drawColor,anchor=base,inner sep=0pt, outer sep=0pt, scale=  1.07] at (292.64,269.12) {$c = 0$};
\end{scope}
\begin{scope}
\path[clip] (386.87,263.47) rectangle (552.55,283.58);
\definecolor{drawColor}{gray}{0.10}

\node[text=drawColor,anchor=base,inner sep=0pt, outer sep=0pt, scale=  1.07] at (469.71,269.12) {$c = 4$};
\end{scope}
\begin{scope}
\path[clip] (552.55,186.57) rectangle (572.66,263.47);
\definecolor{drawColor}{gray}{0.10}

\node[text=drawColor,rotate=-90.00,anchor=base,inner sep=0pt, outer sep=0pt, scale=  1.07] at (558.20,225.02) {$\tau = 0.25$};
\end{scope}
\begin{scope}
\path[clip] (552.55,104.16) rectangle (572.66,181.07);
\definecolor{drawColor}{gray}{0.10}

\node[text=drawColor,rotate=-90.00,anchor=base,inner sep=0pt, outer sep=0pt, scale=  1.07] at (558.20,142.62) {$\tau = 1$};
\end{scope}
\begin{scope}
\path[clip] (552.55, 21.76) rectangle (572.66, 98.66);
\definecolor{drawColor}{gray}{0.10}

\node[text=drawColor,rotate=-90.00,anchor=base,inner sep=0pt, outer sep=0pt, scale=  1.07] at (558.20, 60.21) {$\tau = 4$};
\end{scope}
\begin{scope}
\path[clip] (  0.00,  0.00) rectangle (578.16,289.08);
\definecolor{drawColor}{gray}{0.30}

\node[text=drawColor,anchor=base,inner sep=0pt, outer sep=0pt, scale=  1.07] at ( 38.73,  7.99) {0.00};

\node[text=drawColor,anchor=base,inner sep=0pt, outer sep=0pt, scale=  1.07] at ( 77.15,  7.99) {0.25};

\node[text=drawColor,anchor=base,inner sep=0pt, outer sep=0pt, scale=  1.07] at (115.58,  7.99) {0.50};

\node[text=drawColor,anchor=base,inner sep=0pt, outer sep=0pt, scale=  1.07] at (154.00,  7.99) {0.75};

\node[text=drawColor,anchor=base,inner sep=0pt, outer sep=0pt, scale=  1.07] at (192.42,  7.99) {1.00};
\end{scope}
\begin{scope}
\path[clip] (  0.00,  0.00) rectangle (578.16,289.08);
\definecolor{drawColor}{gray}{0.30}

\node[text=drawColor,anchor=base,inner sep=0pt, outer sep=0pt, scale=  1.07] at (215.79,  7.99) {0.00};

\node[text=drawColor,anchor=base,inner sep=0pt, outer sep=0pt, scale=  1.07] at (254.22,  7.99) {0.25};

\node[text=drawColor,anchor=base,inner sep=0pt, outer sep=0pt, scale=  1.07] at (292.64,  7.99) {0.50};

\node[text=drawColor,anchor=base,inner sep=0pt, outer sep=0pt, scale=  1.07] at (331.07,  7.99) {0.75};

\node[text=drawColor,anchor=base,inner sep=0pt, outer sep=0pt, scale=  1.07] at (369.49,  7.99) {1.00};
\end{scope}
\begin{scope}
\path[clip] (  0.00,  0.00) rectangle (578.16,289.08);
\definecolor{drawColor}{gray}{0.30}

\node[text=drawColor,anchor=base,inner sep=0pt, outer sep=0pt, scale=  1.07] at (392.86,  7.99) {0.00};

\node[text=drawColor,anchor=base,inner sep=0pt, outer sep=0pt, scale=  1.07] at (431.29,  7.99) {0.25};

\node[text=drawColor,anchor=base,inner sep=0pt, outer sep=0pt, scale=  1.07] at (469.71,  7.99) {0.50};

\node[text=drawColor,anchor=base,inner sep=0pt, outer sep=0pt, scale=  1.07] at (508.13,  7.99) {0.75};

\node[text=drawColor,anchor=base,inner sep=0pt, outer sep=0pt, scale=  1.07] at (546.56,  7.99) {1.00};
\end{scope}
\begin{scope}
\path[clip] (  0.00,  0.00) rectangle (578.16,289.08);
\definecolor{drawColor}{gray}{0.30}

\node[text=drawColor,anchor=base east,inner sep=0pt, outer sep=0pt, scale=  1.07] at ( 27.78,185.65) {0.00};

\node[text=drawColor,anchor=base east,inner sep=0pt, outer sep=0pt, scale=  1.07] at ( 27.78,205.73) {0.25};

\node[text=drawColor,anchor=base east,inner sep=0pt, outer sep=0pt, scale=  1.07] at ( 27.78,225.80) {0.50};

\node[text=drawColor,anchor=base east,inner sep=0pt, outer sep=0pt, scale=  1.07] at ( 27.78,245.87) {0.75};
\end{scope}
\begin{scope}
\path[clip] (  0.00,  0.00) rectangle (578.16,289.08);
\definecolor{drawColor}{gray}{0.30}

\node[text=drawColor,anchor=base east,inner sep=0pt, outer sep=0pt, scale=  1.07] at ( 27.78,103.25) {0.00};

\node[text=drawColor,anchor=base east,inner sep=0pt, outer sep=0pt, scale=  1.07] at ( 27.78,123.32) {0.25};

\node[text=drawColor,anchor=base east,inner sep=0pt, outer sep=0pt, scale=  1.07] at ( 27.78,143.40) {0.50};

\node[text=drawColor,anchor=base east,inner sep=0pt, outer sep=0pt, scale=  1.07] at ( 27.78,163.47) {0.75};
\end{scope}
\begin{scope}
\path[clip] (  0.00,  0.00) rectangle (578.16,289.08);
\definecolor{drawColor}{gray}{0.30}

\node[text=drawColor,anchor=base east,inner sep=0pt, outer sep=0pt, scale=  1.07] at ( 27.78, 20.84) {0.00};

\node[text=drawColor,anchor=base east,inner sep=0pt, outer sep=0pt, scale=  1.07] at ( 27.78, 40.92) {0.25};

\node[text=drawColor,anchor=base east,inner sep=0pt, outer sep=0pt, scale=  1.07] at ( 27.78, 60.99) {0.50};

\node[text=drawColor,anchor=base east,inner sep=0pt, outer sep=0pt, scale=  1.07] at ( 27.78, 81.06) {0.75};
\end{scope}
\end{tikzpicture}
}
      }
      \caption[]{Basis functions for different noncentrality and tailweight values $c$ and $\tau$ (using $\mu = 0.5$ and $\sigma = 1$)}\label{knots_c_tau}
    \end{subfigure}
  }
  \caption{B-Spline functions for selected placements of the knots concerning the inputs of Algorithm~\ref{algo:knots}. The center of both figures shows the default case of equidistant knots.}\label{fig:knots}
\end{figure}

\subsection{Shrinkage operators and Forgetting}\label{subsec_shrink}

Shrinkage operators are well-known in statistical learning theory. They help to reduce the overfitting problem by shrinking a solution. The P-Spline smoothing we discussed above can also be interpreted as a shrinkage operator. However, simple shrinkage operators can also be applied to $\bsbeta_{t}$. We consider three additional shrinkage operators: the fixed share operator $\mathcal{F}$, the soft-thresholding operator $\mathcal{S}$, and the hard-thresholding operator $\mathcal{H}$. They are defined as
\begin{align}
  \mathcal{F}(x;\phi)   & = \phi/K  + (1-\phi) x ,       \\
  \mathcal{S}(x;\nu)    & = \sign(x)||x|-\nu| ,          \\
  \mathcal{H}(x;\kappa) & = x  \mathbb{1}\{|x|>\kappa \}
\end{align}
for some $\phi\in[0,1]$, $\nu\geq0$ and $\kappa\geq0$. The fixed share operator shrinks towards the naive combination. This is preferable if no prior information on the experts' performance is available. For some shrinkage problems, there are theoretical guarantees for improvements~\cite {tu2011markowitz, cesa2012mirror}. Applications in the context of online learning include, e.g., \citet{cesa2012mirror, gonzales2021new}. The thresholding operators $\mathcal{S}$ and $\mathcal{H}$ were also considered in online learning contexts previously~\citep{dalalyan2012sharp, gaillard2017sparse}. Applying thresholds leads to sparse solutions. Both appear in several situations for specific linear model estimators. Most notably, soft-thresholding is the key operator in the coordinate descent algorithm for estimating the lasso~\citep{friedman2007pathwise}. Applying any threshold operator potentially violates affinity constraints (incl. the convexity constraint). Therefore, projections to the desired solution space should be applied.

As mentioned, cumulative regret is a key element in online learning. However, in settings with a long history, there might be structural breaks in the data. These breaks motivate the introduction of the forgetting factor. It means that only a limited amount of the old cumulative regret is considered for adjusting the weights. In other words, the algorithm \textit{forgets} about some part of the past performance. In online learning, usually exponential forgetting is chosen~\cite{guo2018online, messner2019online,ziel2021smoothed}. The Regret with a forgetting factor $\theta\in[0,1]$ is formally defined as
\begin{align}
  R_{t,k}(\theta) & =  (1-\theta) R_{t-1,k} + \ell(\wtilde{F}_{t},Y_t) - \ell(\what{F}_{t,k},Y_t)\label{eq_regret_forget}
\end{align}
where $\theta = 0$ correspons to no forgetting. Optimal values for the forgetting factor $\theta$ are usually close to $0$. The forget should be applied to all hidden state variables in sophisticated online learning procedures like BOA. % In the case of the BOA these regret $\bsR_t$, the range estimate $\bsE_t$, and the variance estimate $\bsV_t$.

\section{Full Model and Hyperparameter Optimization}\label{sec_hyperpar}

Algorithm~\ref{algo:boag_smooth} shows the multivariate online CRPS-Learning algorithm. This includes all extensions discussed in Subsections~\ref{subsec_smooth} and~\ref{subsec_shrink}. Considering all extensions, this algorithm contains five general hyperparameters (the forget rate $\phi$, the parameters of the shrinkage operators $\theta$, $\kappa$, $\nu$) as well as 30 hyperparameters concerning the design of basis and hat matrices.

\begin{algorithm}[!h]
  \DontPrintSemicolon
  Initialization see~\ref{append_init} \;
  \SetAlgoLined
  \For{$t$ in $1,\ldots, T$}{
  \lFor*{$d$ in $1,\ldots, D$}{
  \For{$p$ in $1,\ldots, \P$}{
  $\wtilde{X}_{t,d,p} = \bsw_{t-1,d,p}' \what{\bsX}_{t,d,p}$ \;
  \lFor{$k$ in $1,\ldots, K$}{
  $\bsr^\text{full}_{d,p,k} = QL_p^{\nabla}(\wtilde{X}_{t,d,p},Y_t) - QL_p^{\nabla}(\what{X}_{t,d,p,k},Y_t)$
  }}}
  \For{$k$ in $1,\ldots, K$}{
    $\bsr^\text{red}_k = \frac{\Dr}{\D} \frac{\Pr}{\P} {\bsB^\mult}' \bsr^\text{full}_k \bsB^\prob$ \tcp*{now $\bsr^\text{red}_k$ is $ \Dr \times \Pr$}
  }
  \For{$d$ in $1,\ldots, \Dr$}{
  \For{$p$ in $1,\ldots, \Pr$}{
  \ali{2.5em}{$\bs V_{t,d,p}$}              $= (1-\theta) \bs V_{t-1,d,p} + \left({\bs r^\text{red}_{d,p}}\right)^{ \odot 2}$\;
  \ali{2.5em}{$\bs E_{t,d,p}$}              $= \max\left( (1-\theta) \bs E_{t-1,d,p}, \left|{\bs r^\text{red}_{d,p}} \right| \right) $\;
  \ali{2.5em}{$\bs \eta_{t,d,p}$}  $= \gamma \min\left( \left(-\log(\bs \beta_{0,d,p}) \odot \bs V_{t,d,p}^{\odot -1} \right)^{\odot\frac{1}{2}} , \frac{1}{2} \bs E_{t,d,p}^{\odot-1}\right)$\;
  \ali{2.5em}{$\bs R_{t,d,p}$} $= (1-\theta) \bs R_{t-1,d,p} +  {\bs r^\text{red}_{d,p}} \odot \left( 1 - \bs\eta_{t,d,p} \odot {\bs r^\text{red}_{d,p}} \right)/2 + \phantom{{}===1} \bs E_{t,d,p} \odot  \mathbb{1}\{-2 \bs \eta_{t,d,p} \odot {\bs r^\text{red}_{d,p}} > 1\}  $\;
  \ali{2.5em}{$\bs \beta_{t,d,p}$} $= K \bsbeta_{0,d,p} \odot \softmax\left( -  \bs \eta_{t,d,p} \odot \bs R_{t,d,p} + \log( \bs \eta_{t,d,p}) \right)$\;
  \ali{2.5em}{$\bs \beta_{t,d,p}$} $= \left( \mathcal{F}_\phi \circ \mathcal{H}_\kappa \circ \mathcal{S}_\nu \right) \left(\bs \beta_{t,d,p}\right) $ \;%\tcp*{apply thresholds}
  }
  $\wtilde{\bs X}_{t,d} = \text{Sort} ( \wtilde{\bs X}_{t,d} )$\
  }
  \For{$k$ in $1,\ldots, K$}{
    $\bsw_{t,k}(\bsPP) = \boldsymbol{\mathcal{H}^\mult} \bsB^\mult \bsbeta_{t,k} {\bsB^\prob}' \boldsymbol{\mathcal{H}^\prob}$\;
  }}
  \textbf{end}\;
  \caption{\label{algo:boag_smooth} Smoothed CRPS Bernstein Online Aggregation}
\end{algorithm}
The algorithm is versatile as it handles several special cases discussed in the literature. One such case is the uniform combination, also known as the \textbf{naive} combination. This can be calculated by using the Fixed-Share operator $\mathcal{F}_\phi$ with $\phi=1$, resulting in entirely uniform weights. There are more efficient ways of calculating uniform weights, obviously. However, adjusting the value of $\phi$ allows a smooth transition from the uniform solution to the solution calculated by BOA. Another common special case is constant weights, where each expert receives a specific weight. This can be calculated by setting both basis matrices, $B^\mult$ and $B^\prob$, to the unity Vector of length $\D$ and $\P$, respectively. This leads to weights without variation across marginals and probabilities (\textbf{Constant}). Setting only one of the basis matrices to the unity vector will result in weights that are constant over either covariates (\textbf{Constant Mv}) or probabilities (\textbf{Constant Pr}). Additionally, setting all smoothing matrices to identity produces pointwise weights, and optimizing $\lambda$ in the hat matrices concerning the predictive CRPS results in possibly smoothed weights. These cases are illustrated in Figures~\ref{fig:b.constant.mv}-\ref{fig:smooth}.
\begin{figure*}[!h]
  \centering
  \begin{subfigure}[b]{0.475\textwidth}
    \centering
    \includegraphics[trim={3.5cm 0cm 3.5cm 2cm}, clip, width=\textwidth]{anc/plots/specials/bewa.b.constant.mv_JSU1.pdf}
    \caption[Network2]%
    {{\small Constant weights w.r.t. time (hours)}}
    \label{fig:b.constant.mv}
  \end{subfigure}
  \hfill
  \begin{subfigure}[b]{0.475\textwidth}
    \centering
    \includegraphics[trim={3.5cm 0cm 3.5cm 2cm}, clip, width=\textwidth]{anc/plots/specials/bewa.b.constant.pr_JSU1.pdf}
    \caption[]%
    {{\small Constant weights w.r.t. probabilities}}
    \label{fig:b.constant.pr}
  \end{subfigure}
  \vskip\baselineskip
  \begin{subfigure}[b]{0.475\textwidth}
    \centering
    \includegraphics[trim={3.5cm 0cm 3.5cm 2cm}, clip, width=\textwidth]{anc/plots/specials/bewa.pointwise_JSU1.pdf}
    \caption[]%
    {{\small Optimized pointwise weights}}
    \label{fig:pointwise}
  \end{subfigure}
  \hfill
  \begin{subfigure}[b]{0.475\textwidth}
    \centering
    \includegraphics[trim={3.5cm 0cm 3.5cm 2cm}, clip, width=\textwidth]{anc/plots/best/mix.bewa.sm.fr_JSU1.pdf}
    \caption[]%
    {{\small Optimized smoothed weights}}
    \label{fig:smooth}
  \end{subfigure}
  \caption[ The average and standard deviation of critical parameters ]
  {\small Most recent weighs of JSU1 calculated using different specifications of Algorithm~\ref{algo:boag_smooth}}
  \label{fig:nested_cases}
\end{figure*}
The extensions discussed in Subsections \ref{subsec_smooth} and \ref{subsec_shrink} require the specification of various hyperparameters. There are many possible hyperparameters to choose from, and we do not have any prior information on the best values. This means that it is impossible to test all combinations of these parameters in each iteration of the forecasting task. The latter would be ideal, but it is impractical due to the large amount of computational resources required. As a result, we need to use other, less demanding methods for tuning these hyperparameters. In this paper, we will utilize three approaches for tuning.

The first approach to hyperparameter tuning is using a sophisticated search algorithm based on random forest and optimizing towards the lowest CRPS on a subset of our observations (i.e., a training set). We utilize the R-Package mlrMBO to execute this optimization \citep{mlrMBO}. This approach brings one major advantage: the search algorithm can search the considered space very efficiently by repeatedly reevaluating the objective function. However, once the final set of hyperparameters is selected, it will remain constant throughout the forecasting task. Additionally, the tuning is only executed using a small subset of the dataset. This could be a problem as the chosen set of parameters may not be optimal for the rest of the dataset, especially if there are structural breaks. Hereinafter, we will refer to this approach as \textbf{Bayesian fix} as it fixes the hyperparameters after utilizing a Bayesian search algorithm.

The second approach uses the \texttt{online} function, which is included in the \textit{profoc} R-Package. It implements the proposed algorithm along with an online tuning strategy for the hyperparameters. This strategy considers a random sample of all possible hyperparameter sets, and for each iteration, the combination with the lowest aggregate CRPS is chosen. In contrast to the \textbf{Bayesian fix} approach, we define all possible parameter sets before the learning task. However, this method dynamically selects the parameter set based on past performance, allowing for dynamic adjustments if underlying properties change. The most significant drawback of this approach is that only a random sample of the hyperparameter space is considered. However, this approach has the advantage of adjusting dynamically to changes in the data. Therefore, we will refer to this approach as \textbf{Sampling Online}.

It is also possible to combine both approaches. In this case, mlrMBO optimizes on a subset of the data. \textit{online} uses the parameter combinations that were considered in the mlrMBO optimization. This has the potential to profit from efficient exploration of the hyperparameter space and the ability to adjust to changes in the data dynamically. After this, we will refer to this approach as \textbf{Bayesian Online}.

\section{Application to Multivariate Probabilistic Day-Ahead Power Prices}\label{application}
In day-ahead electricity price forecasting, we consider the price $Y_{t,h}$ at day $t$ and product $h=1, \ldots, H$ of the day. For hourly electricity prices, we have $H=24$, and therefore $h$ is often referred to as \textit{hour}, see~\cite{ziel2018day}.
We consider the forecasts of \citet{marcjasz2022distributional}, which range from 27th Dec. 2018 to 31st Dec. 2020. These are hourly forecasts of eight different models, i.e., neural network specifications. The forecasts are given as distributional parameters for each hour (i.e., $\bsDD = (1, \ldots, 24)$) of all 736 Days. We use those distributional forecasts for calculating quantiles on the equidistant grid of percentiles $\bsPP=(0.01,\dots,0.99)$.

For our learning task, we chose to consider the penalized smoothing approach only. We consider the knot placement and the other extensions discussed in Subsections \ref{subsec_smooth} and \ref{subsec_shrink}. Table~\ref{tab:pars} summarizes the considered hyperparameters. That is, we have a total of 15 tuning parameters to optimize.

\begin{table}[!h]
    \centering
    \resizebox{1\textwidth}{!}{
        \begin{tabular}[t]{l>{}c>{}c>{}l|>{}c>{}c>{}c>{}c}
            \toprule
                                      &                     &                  &                   & \multicolumn{4}{c}{Model Specification}                                           \\
            Description               & Notation            & Range            & Trafo             & Full                                    & Smooth.Forget & Smooth     & Forget     \\
            \midrule
            Forget Regret             & $\theta$            & $-12, \ldots, 2$ & $2^x$             & \checkmark                              & \checkmark    &            & \checkmark \\
            Fixed Share               & $\phi$              & $-15, \ldots, 0$ & $2^x$             & \checkmark                              &               &            &            \\
            Soft Threshold            & $\nu$               & $-15, \ldots, 0$ & $2^x$             & \checkmark                              &               &            &            \\
            Hard Threshold            & $\kappa$            & $-15, \ldots, 0$ & $2^x$             & \checkmark                              &               &            &            \\
            Learning rate adjustment  & $\gamma$            & $-1, \ldots, 1$  & $2^x$             & \checkmark                              &               &            &            \\
            Penalized Smoothing Prob. & $\lambda^\text{pr}$ & $-5, \ldots, 15$ & $2^x$             & \checkmark                              & \checkmark    & \checkmark &            \\
            Penalized Smoothing Mult. & $\lambda^\text{mv}$ & $-5, \ldots, 15$ & $2^x$             & \checkmark                              & \checkmark    & \checkmark &            \\
            Knot Placement Prob.      & $\mu^\text{pr}$     & $-1, \ldots, 1$  & $x^3 / 2.1 + 0.5$ & \checkmark                              &               &            &            \\
            Knot Placement Prob.      & $\sigma^\text{pr}$  & $-1, \ldots, 1$  & $x^3 / 1.1 + 1$   & \checkmark                              &               &            &            \\
            Knot Placement Prob.      & $c^\text{pr}$       & $-3, \ldots, 3$  & $x^3$             & \checkmark                              &               &            &            \\
            Knot Placement Prob.      & $\tau^\text{pr}$    & $-1, \ldots, 1$  & $x^3 / 1.1 + 1$   & \checkmark                              &               &            &            \\
            Knot Placement Mult.      & $\mu^\text{mv}$     & $-1, \ldots, 1$  & $x^3 / 2.1 + 0.5$ & \checkmark                              &               &            &            \\
            Knot Placement Mult.      & $\sigma^\text{mv}$  & $-1, \ldots, 1$  & $x^3 / 1.1 + 1$   & \checkmark                              &               &            &            \\
            Knot Placement Mult.      & $c^\text{mv}$       & $-3, \ldots, 3$  & $x^3$             & \checkmark                              &               &            &            \\
            Knot Placement Mult.      & $\tau^\text{mv}$    & $-1, \ldots, 1$  & $x^3 / 1.1 + 1$   & \checkmark                              &               &            &            \\
            \bottomrule
        \end{tabular}
    }
    \caption{Considered hyperparameters with their ranges and respective transformation functions and three nested specifications (nested in the \textit{Full} specification, which considers all of the above hyperparameters).\label{tab:pars}}
\end{table}


We conduct the forecasting task using the three tuning strategies \textbf{Bayesian fix}, \textbf{Sampling Online}, and  \textbf{Sampling Online} discussed in Section~\ref{sec_hyperpar}. \citet{marcjasz2022distributional} used about half a year of data, resp. the first (182) observations, as a burn-in period for hyperparameters to stabilize. With \textbf{Bayesian fix}, we use these first 182 observations. However, we do not evaluate the forecasts of the first 50 observations due to the elevated estimation uncertainty early in the learning process. We utilize the Krigin learner of MLRmbo to propose eight new points until the budget of 1000 points is exhausted. This is done in parallel and needs around 30 minutes of computation time on our Intel i5-12600K CPU. Then, the best hyperparameter set is used to conduct the whole forecast combination task with all 736 observations. For our final evaluation, we follow \citet{marcjasz2022distributional} again by excluding the first 182 observations.

For \textbf{sampling online}, we first divide the range of each hyperparameter into 16 equidistant values, apply the transformation function (see Table~\ref{tab:pars}), and then randomly sample up to 2500 points from the resulting multivariate hyperparameter space. The online optimization process is then carried out as described in Section~\ref{sec_hyperpar}. As with \textbf{Bayesian fix}, the first 182 observations are excluded from the evaluation. We chose to sample 2500 points so that the \textbf{Sampling Online} optimization process also takes around 30 minutes of computation time.

For \textbf{Bayesian Online}, we run \textbf{Bayesian fix} analogous to the above but with a reduced budget of 750 points to propose. These points are then fed into the \textbf{Sampling Online} optimization. We chose these settings to ensure the computation times remain around 30 minutes.

In addition to tuning all 15 hyperparameters (\textbf{Full}), we examine three subsets of these hyperparameters. The first subset only includes penalized smoothing and forget (\textbf{Smooth.Forget}), the second subset only includes penalized smoothing (\textbf{Smooth}), and the last subset only includes forget (\textbf{Forget}). Note that the time required for computing \textbf{Smooth} \textbf{Forget} is reduced when using \textbf{Sampling Online} as the number of possible parameter combinations does not exceed 2500. The specifications are summarized in Table~\ref{tab:pars}. We also report the performance of the \textbf{naive}, the performance of each expert, and the four special cases shown in Figure~\ref{fig:nested_cases}.

\begin{table}[!h]
    % latex table generated in R 4.0.3 by xtable 1.8-4 package
    % Fri Jan 29 10:45:56 2021
    \centering
    \resizebox{1\textwidth}{!}{
        % latex table generated in R 4.1.0 by xtable 1.8-4 package
        % Wed Jul  7 16:03:20 2021
        \begin{tabular}[t]{>{}l>{}r>{}r>{}r>{}r>{}r>{}r>{}r>{}r}
            \toprule
            JSU1                            & JSU2                            & JSU3                            & JSU4                            & Norm1                           & Norm2                           & Norm3                          & Norm4                           & Naive                                    \\
            \midrule
            \cellcolor[HTML]{EE5250}{1.487} & \cellcolor[HTML]{EE5250}{1.444} & \cellcolor[HTML]{EE5250}{1.499} & \cellcolor[HTML]{F15E4D}{1.374} & \cellcolor[HTML]{EE5250}{1.414} & \cellcolor[HTML]{EE5250}{1.535} & \cellcolor[HTML]{EE5250}{1.42} & \cellcolor[HTML]{EE5250}{1.422} & \cellcolor[HTML]{FFFFFF}{\textbf{1.295}} \\
            \bottomrule
            % \multicolumn{9}{l}{\rule{0pt}{1em}Coloring w.r.t. test statistic: \colorbox[HTML]{66BA6A}{$<$-5} \colorbox[HTML]{7CC168}{-4} \colorbox[HTML]{91C866}{-3} \colorbox[HTML]{B0D363}{-2} \colorbox[HTML]{D8E05E}{-1} \colorbox[HTML]{FFED58}{0} \colorbox[HTML]{FFD145}{1} \colorbox[HTML]{FFB531}{2} \colorbox[HTML]{FC9733}{3} \colorbox[HTML]{F67744}{4} \colorbox[HTML]{EE5250}{$>$5} } \\
        \end{tabular}
    }
    \resizebox{1\textwidth}{!}{
        \begin{tabular}[t]{
                p{0.18\textwidth}
                p{0.22\textwidth}
                p{0.2\textwidth}
                p{0.2\textwidth}
                p{0.2\textwidth}}
            % \toprule
            Description   & Parameter Tuning & \centering BOA                                         & \centering ML-Poly                                   & \centering\arraybackslash EWA                                       \\
            \toprule
            Constant      & -                & \centering \cellcolor[HTML]{F1E85A}{1.2933}            & \centering \cellcolor[HTML]{FFE854}{1.2966}          & \centering\arraybackslash \cellcolor[HTML]{FFBB35}{1.3188}          \\
            Constant PR   & -                & \centering \cellcolor[HTML]{F1E85A}{1.2936}            & \centering \cellcolor[HTML]{FFD94A}{1.3000}          & \centering\arraybackslash \cellcolor[HTML]{F1604C}{1.3432}          \\
            Constant MV   & -                & \centering \cellcolor[HTML]{E4E45C}{1.2918}            & \centering \cellcolor[HTML]{FAEB59}{1.2945}          & \centering\arraybackslash \cellcolor[HTML]{FFCB40}{1.3076}          \\
            Pointwise     & -                & \centering \cellcolor[HTML]{EFE85B}{1.2936}            & \centering \cellcolor[HTML]{FFD346}{1.3010}          & \centering\arraybackslash \cellcolor[HTML]{FFAE2C}{1.3101}          \\
            % MLRBO FR.2 &   & \centering \cellcolor[HTML]{E9E65C}{1.2930}          & \centering \cellcolor[HTML]{FFEC57}{1.2956}          & \centering \cellcolor[HTML]{FFB02D}{1.3096}  \arraybackslash                                                                                                                                                                                                                                              \\
            % MLRBO FR.3    & \centering \cellcolor[HTML]{E9E65C}{1.2930}          & \centering \cellcolor[HTML]{FFEC57}{1.2956}          & \centering \cellcolor[HTML]{FFB02D}{1.3096}  \arraybackslash                                                                                                                                                                                                                                              \\
            % MLRBO FR.4    & \centering \cellcolor[HTML]{E9E65C}{1.2930}          & \centering \cellcolor[HTML]{FFEC57}{1.2956}          & \centering \cellcolor[HTML]{FFB02D}{1.3096}  \arraybackslash                                                                                                                                                                                                                                              \\
            % MLRBO FR.5    & \centering \cellcolor[HTML]{E9E65C}{1.2930}          & \centering \cellcolor[HTML]{FFEC57}{1.2956}          & \centering \cellcolor[HTML]{FFB02D}{1.3096}  \arraybackslash                                                                                                                                                                                                                                              \\
            Full          & Sampling Online  & \centering \cellcolor[HTML]{D0DD5F}{1.2886}            & \centering \cellcolor[HTML]{BBD662}{1.2861}          & \centering\arraybackslash \cellcolor[HTML]{C7DA60}{1.2873}          \\
            Full          & Bayesian Fix     & \centering \cellcolor[HTML]{E6E55C}{1.2922}            & \centering \cellcolor[HTML]{F0E85B}{1.2930}          & \centering\arraybackslash \cellcolor[HTML]{C4D961}{1.2877}          \\
            Full          & Bayesian Online  & \centering \cellcolor[HTML]{DBE15E}{1.2899}            & \centering \cellcolor[HTML]{BAD662}{1.2862}          & \centering\arraybackslash \cellcolor[HTML]{C3D961}{1.2869}          \\
            Smooth.Forget & Sampling Online  & \centering \cellcolor[HTML]{8FC766}{1.2845}            & \centering \cellcolor[HTML]{BBD762}{1.2867}          & \centering\arraybackslash \cellcolor[HTML]{C2D961}{1.2866}          \\
            Smooth.Forget & Bayesian Fix     & \centering \cellcolor[HTML]{D7E05E}{1.2911}            & \centering \cellcolor[HTML]{E0E35D}{1.2911}          & \centering\arraybackslash \cellcolor[HTML]{C1D861}{1.2869}          \\
            Smooth.Forget & Bayesian Online  & \centering \cellcolor[HTML]{8EC766}{\textbf{1.2844}}   & \centering \cellcolor[HTML]{B0D363}{\textbf{1.2854}} & \centering\arraybackslash \cellcolor[HTML]{C0D861}{\textbf{1.2864}} \\
            Forget        & Sampling Online  & \centering \cellcolor[HTML]{96CA66}{1.2855}            & \centering \cellcolor[HTML]{FFEA56}{1.2961}          & \centering\arraybackslash \cellcolor[HTML]{FFAF2C}{1.3114}          \\
            Forget        & Bayesian Fix     & \centering \centering \cellcolor[HTML]{E9E65C}{1.2930} & \centering \cellcolor[HTML]{FFEC57}{1.2956}          & \centering\arraybackslash \cellcolor[HTML]{FFB02D}{1.3096}          \\
            Forget        & Bayesian Online  & \centering \cellcolor[HTML]{97CB65}{1.2856}            & \centering \cellcolor[HTML]{FFEA56}{1.2960}          & \centering\arraybackslash \cellcolor[HTML]{FFAB2A}{1.3111}          \\
            Smooth        & Sampling Online  & \centering \cellcolor[HTML]{DEE25D}{1.2918}            & \centering \cellcolor[HTML]{E5E45C}{1.2917}          & \centering\arraybackslash \cellcolor[HTML]{C9DB60}{1.2877}          \\
            Smooth        & Bayesian Fix     & \centering \cellcolor[HTML]{DDE25D}{1.2917}            & \centering \cellcolor[HTML]{E5E45C}{1.2917}          & \centering\arraybackslash \cellcolor[HTML]{C4D961}{1.2873}          \\
            Smooth        & Bayesian Online  & \centering \cellcolor[HTML]{DFE25D}{1.2920}            & \centering \cellcolor[HTML]{E6E55C}{1.2919}          & \centering\arraybackslash \cellcolor[HTML]{C6DA60}{1.2875}          \\
            % MLRBO Full  & \centering \cellcolor[HTML]{E4E45C}{1.2916}          & \centering \cellcolor[HTML]{F1E85A}{1.2932}          & \centering \cellcolor[HTML]{C0D861}{\textbf{1\arraybackslash.2866}}                                                                                                                                                                                                                                       \\
            % MLRBO Full  & \centering \cellcolor[HTML]{E2E35D}{1.2912}          & \centering \cellcolor[HTML]{EFE85B}{1.2929}          & \centering \cellcolor[HTML]{C6DA60}{1.2875}  \arraybackslash                                                                                                                                                                                                                                              \\
            % MLRBO Full  & \centering \cellcolor[HTML]{E3E45C}{1.2913}          & \centering \cellcolor[HTML]{F8EB59}{1.2942}          & \centering \cellcolor[HTML]{BFD861}{1.2866}  \arraybackslash                                                                                                                                                                                                                                              \\
            % MLRBO Full  & \centering \cellcolor[HTML]{EAE65B}{1.2923}          & \centering \cellcolor[HTML]{F5EA5A}{1.2938}          & \centering \cellcolor[HTML]{C0D861}{1.2868}  \arraybackslash                                                                                                                                                                                                                                              \\
            % MLRBO.sm FR.2 & \centering \cellcolor[HTML]{D6DF5E}{1.2911}          & \centering \cellcolor[HTML]{E0E35D}{1.2911}          & \centering \cellcolor[HTML]{C1D861}{1.2869}  \arraybackslash                                                                                                                                                                                                                                              \\
            % MLRBO.sm FR.3 & \centering \cellcolor[HTML]{D7E05E}{1.2912}          & \centering \cellcolor[HTML]{E0E35D}{1.2911}          & \centering \cellcolor[HTML]{C1D861}{1.2869}  \arraybackslash                                                                                                                                                                                                                                              \\
            % MLRBO.sm FR.4 & \centering \cellcolor[HTML]{D7E05E}{1.2911}          & \centering \cellcolor[HTML]{E0E35D}{1.2911}          & \centering \cellcolor[HTML]{C1D861}{1.2869}  \arraybackslash                                                                                                                                                                                                                                              \\
            % MLRBO.sm FR.5 & \centering \cellcolor[HTML]{D6DF5E}{1.2910}          & \centering \cellcolor[HTML]{E0E35D}{1.2911}          & \centering \cellcolor[HTML]{C0D861}{1.2868}  \arraybackslash                                                                                                                                                                                                                                              \\
            % MLRBO SM.2    & \centering \cellcolor[HTML]{DDE25D}{1.2917}          & \centering \cellcolor[HTML]{E5E45C}{1.2917}          & \centering \cellcolor[HTML]{C4D961}{1.2873}  \arraybackslash                                                                                                                                                                                                                                              \\
            % MLRBO SM.3    & \centering \cellcolor[HTML]{DDE25D}{1.2917}          & \centering \cellcolor[HTML]{E5E45C}{1.2917}          & \centering \cellcolor[HTML]{C5DA61}{1.2873}  \arraybackslash                                                                                                                                                                                                                                              \\
            % MLRBO SM.4    & \centering \cellcolor[HTML]{DDE25D}{1.2917}          & \centering \cellcolor[HTML]{E5E45C}{1.2917}          & \centering \cellcolor[HTML]{C5DA61}{1.2874}  \arraybackslash                                                                                                                                                                                                                                              \\
            % MLRBO SM.5    & \centering \cellcolor[HTML]{DDE25D}{1.2917}          & \centering \cellcolor[HTML]{E4E45C}{1.2917}          & \centering \cellcolor[HTML]{C5DA61}{1.2874}  \arraybackslash                                                                                                                                                                                                                                              \\
            \bottomrule
            \multicolumn{4}{l}{\rule{0pt}{1em}Coloring w.r.t. test statistic: \colorbox[HTML]{66BA6A}{$<$-5} \colorbox[HTML]{7CC168}{-4} \colorbox[HTML]{91C866}{-3} \colorbox[HTML]{B0D363}{-2} \colorbox[HTML]{D8E05E}{-1} \colorbox[HTML]{FFED58}{0} \colorbox[HTML]{FFD145}{1} \colorbox[HTML]{FFB531}{2} \colorbox[HTML]{FC9733}{3} \colorbox[HTML]{F67744}{4} \colorbox[HTML]{EE5250}{$>$5} }
        \end{tabular}
    }
    \caption{CRPS scores (lower = better) of the individual experts and the \textit{naive} combination (top), and different specifications (see Table~\ref{tab:pars}) of Algorithm~\ref{algo:boag_smooth} (bottom). Colored according to the test statistic of a DM-Test comparing to \textit{Naive} (greener means lower test statistic, i.e., better performance compared to \textit{Naive}).}\label{tab:crps}
\end{table}

Table~\ref{tab:crps} summarizes the results. It reports the CRPS of each expert and the \textbf{naive} combination in the top row and the performance of different specifications of Algorithm~\ref{algo:boag_smooth}.
We also report the performance of \textbf{ML-Poly} and \textbf{EWA} weighting schemes as they are popular in the forecast combination literature \citep{gaillard2014second, jore2010combining, dalalyan2012sharp, opschoor2017combining}. We always apply the gradient trick (see. Section~\ref{theor}) However, due to their inferior convergence properties, we do expect them to perform worse than BOA. The table cells are colored according to the test statistic of a Diebol Mariano test, testing whether the forecasts in question significantly differ over the naive model (negative test statistic) or not \citep{diebold2002comparing}. We apply the Diebold Mariano test with the small sample adjustment of \citet{harvey1997testing}.

We see that all individual experts perform significantly worse compared to \textbf{naive}. Further, the best results were obtained with the \textbf{Bayesian Online} approach and using equidistant knots, penalized smoothing, and a forget rate (\textbf{Smooth.Forget}). This solution clearly yields a significant improvement over the naive combination. Comparing this solution with the smaller models \textbf{Smooth} and \textbf{Forget}, we conclude that forgetting contributes to the majority of the observed improvement. The importance of the forget indicates structural changes in the data. Further evidence comes from the fact that the dynamic \textbf{Bayesian Online} optimization generally outperforms parameter optimization using \textbf{Bayesian fix}.

Further, BOA performs best compared to the other considered weighting schemes ML-Poly and EWA. Overall forgetting and smoothing play a crucial role in the performance of the combination. Finally, regarding the hyperparameter tuning, we conclude that the dynamic optimization \textbf{Bayesian Online} and \textbf{Sampling Online} should be preferred to the static Bayesian optimization \textbf{Bayesian Mix}.

% See https://tex.stackexchange.com/a/130078/190318 for the following 2 lines
\let\pgfimageWithoutPath\pgfimage
\renewcommand{\pgfimage}[2][]{\pgfimageWithoutPath[#1]{anc/plots/best/#2}}
\begin{figure}[!h]
  \centering{
    \resizebox{\textwidth}{!}{%
      \fbox{% !TEX encoding = UTF-8 Unicode
\begin{tikzpicture}[x=1pt,y=1pt]
\definecolor{fillColor}{RGB}{255,255,255}
\path[use as bounding box,fill=fillColor,fill opacity=0.00] (0,0) rectangle (578.16,433.62);
\begin{scope}
\path[clip] ( 43.34,384.25) rectangle (486.19,428.12);
\definecolor{drawColor}{gray}{0.92}

\path[draw=drawColor,line width= 0.3pt,line join=round] ( 43.34,398.13) --
	(486.19,398.13);

\path[draw=drawColor,line width= 0.3pt,line join=round] ( 43.34,414.24) --
	(486.19,414.24);

\path[draw=drawColor,line width= 0.3pt,line join=round] (101.03,384.25) --
	(101.03,428.12);

\path[draw=drawColor,line width= 0.3pt,line join=round] (210.69,384.25) --
	(210.69,428.12);

\path[draw=drawColor,line width= 0.3pt,line join=round] (320.65,384.25) --
	(320.65,428.12);

\path[draw=drawColor,line width= 0.3pt,line join=round] (430.61,384.25) --
	(430.61,428.12);

\path[draw=drawColor,line width= 0.6pt,line join=round] ( 43.34,390.07) --
	(486.19,390.07);

\path[draw=drawColor,line width= 0.6pt,line join=round] ( 43.34,406.18) --
	(486.19,406.18);

\path[draw=drawColor,line width= 0.6pt,line join=round] ( 43.34,422.30) --
	(486.19,422.30);

\path[draw=drawColor,line width= 0.6pt,line join=round] ( 46.65,384.25) --
	( 46.65,428.12);

\path[draw=drawColor,line width= 0.6pt,line join=round] (155.41,384.25) --
	(155.41,428.12);

\path[draw=drawColor,line width= 0.6pt,line join=round] (265.97,384.25) --
	(265.97,428.12);

\path[draw=drawColor,line width= 0.6pt,line join=round] (375.32,384.25) --
	(375.32,428.12);

\path[draw=drawColor,line width= 0.6pt,line join=round] (485.89,384.25) --
	(485.89,428.12);
\node[inner sep=0pt,outer sep=0pt,anchor=south west,rotate=  0.00] at ( 43.34, 386.24) {
	\pgfimage[width=442.84pt,height= 39.88pt,interpolate=true]{wplot_prob_17_mix.bewa.sm.fr_ras1}};
\end{scope}
\begin{scope}
\path[clip] ( 43.34,334.87) rectangle (486.19,378.75);
\definecolor{drawColor}{gray}{0.92}

\path[draw=drawColor,line width= 0.3pt,line join=round] ( 43.34,348.75) --
	(486.19,348.75);

\path[draw=drawColor,line width= 0.3pt,line join=round] ( 43.34,364.87) --
	(486.19,364.87);

\path[draw=drawColor,line width= 0.3pt,line join=round] (101.03,334.87) --
	(101.03,378.75);

\path[draw=drawColor,line width= 0.3pt,line join=round] (210.69,334.87) --
	(210.69,378.75);

\path[draw=drawColor,line width= 0.3pt,line join=round] (320.65,334.87) --
	(320.65,378.75);

\path[draw=drawColor,line width= 0.3pt,line join=round] (430.61,334.87) --
	(430.61,378.75);

\path[draw=drawColor,line width= 0.6pt,line join=round] ( 43.34,340.69) --
	(486.19,340.69);

\path[draw=drawColor,line width= 0.6pt,line join=round] ( 43.34,356.81) --
	(486.19,356.81);

\path[draw=drawColor,line width= 0.6pt,line join=round] ( 43.34,372.93) --
	(486.19,372.93);

\path[draw=drawColor,line width= 0.6pt,line join=round] ( 46.65,334.87) --
	( 46.65,378.75);

\path[draw=drawColor,line width= 0.6pt,line join=round] (155.41,334.87) --
	(155.41,378.75);

\path[draw=drawColor,line width= 0.6pt,line join=round] (265.97,334.87) --
	(265.97,378.75);

\path[draw=drawColor,line width= 0.6pt,line join=round] (375.32,334.87) --
	(375.32,378.75);

\path[draw=drawColor,line width= 0.6pt,line join=round] (485.89,334.87) --
	(485.89,378.75);
\node[inner sep=0pt,outer sep=0pt,anchor=south west,rotate=  0.00] at ( 43.34, 336.87) {
	\pgfimage[width=442.84pt,height= 39.88pt,interpolate=true]{wplot_prob_17_mix.bewa.sm.fr_ras2}};
\end{scope}
\begin{scope}
\path[clip] ( 43.34,285.50) rectangle (486.19,329.37);
\definecolor{drawColor}{gray}{0.92}

\path[draw=drawColor,line width= 0.3pt,line join=round] ( 43.34,299.38) --
	(486.19,299.38);

\path[draw=drawColor,line width= 0.3pt,line join=round] ( 43.34,315.49) --
	(486.19,315.49);

\path[draw=drawColor,line width= 0.3pt,line join=round] (101.03,285.50) --
	(101.03,329.37);

\path[draw=drawColor,line width= 0.3pt,line join=round] (210.69,285.50) --
	(210.69,329.37);

\path[draw=drawColor,line width= 0.3pt,line join=round] (320.65,285.50) --
	(320.65,329.37);

\path[draw=drawColor,line width= 0.3pt,line join=round] (430.61,285.50) --
	(430.61,329.37);

\path[draw=drawColor,line width= 0.6pt,line join=round] ( 43.34,291.32) --
	(486.19,291.32);

\path[draw=drawColor,line width= 0.6pt,line join=round] ( 43.34,307.44) --
	(486.19,307.44);

\path[draw=drawColor,line width= 0.6pt,line join=round] ( 43.34,323.55) --
	(486.19,323.55);

\path[draw=drawColor,line width= 0.6pt,line join=round] ( 46.65,285.50) --
	( 46.65,329.37);

\path[draw=drawColor,line width= 0.6pt,line join=round] (155.41,285.50) --
	(155.41,329.37);

\path[draw=drawColor,line width= 0.6pt,line join=round] (265.97,285.50) --
	(265.97,329.37);

\path[draw=drawColor,line width= 0.6pt,line join=round] (375.32,285.50) --
	(375.32,329.37);

\path[draw=drawColor,line width= 0.6pt,line join=round] (485.89,285.50) --
	(485.89,329.37);
\node[inner sep=0pt,outer sep=0pt,anchor=south west,rotate=  0.00] at ( 43.34, 287.49) {
	\pgfimage[width=442.84pt,height= 39.88pt,interpolate=true]{wplot_prob_17_mix.bewa.sm.fr_ras3}};
\end{scope}
\begin{scope}
\path[clip] ( 43.34,236.13) rectangle (486.19,280.00);
\definecolor{drawColor}{gray}{0.92}

\path[draw=drawColor,line width= 0.3pt,line join=round] ( 43.34,250.01) --
	(486.19,250.01);

\path[draw=drawColor,line width= 0.3pt,line join=round] ( 43.34,266.12) --
	(486.19,266.12);

\path[draw=drawColor,line width= 0.3pt,line join=round] (101.03,236.13) --
	(101.03,280.00);

\path[draw=drawColor,line width= 0.3pt,line join=round] (210.69,236.13) --
	(210.69,280.00);

\path[draw=drawColor,line width= 0.3pt,line join=round] (320.65,236.13) --
	(320.65,280.00);

\path[draw=drawColor,line width= 0.3pt,line join=round] (430.61,236.13) --
	(430.61,280.00);

\path[draw=drawColor,line width= 0.6pt,line join=round] ( 43.34,241.95) --
	(486.19,241.95);

\path[draw=drawColor,line width= 0.6pt,line join=round] ( 43.34,258.06) --
	(486.19,258.06);

\path[draw=drawColor,line width= 0.6pt,line join=round] ( 43.34,274.18) --
	(486.19,274.18);

\path[draw=drawColor,line width= 0.6pt,line join=round] ( 46.65,236.13) --
	( 46.65,280.00);

\path[draw=drawColor,line width= 0.6pt,line join=round] (155.41,236.13) --
	(155.41,280.00);

\path[draw=drawColor,line width= 0.6pt,line join=round] (265.97,236.13) --
	(265.97,280.00);

\path[draw=drawColor,line width= 0.6pt,line join=round] (375.32,236.13) --
	(375.32,280.00);

\path[draw=drawColor,line width= 0.6pt,line join=round] (485.89,236.13) --
	(485.89,280.00);
\node[inner sep=0pt,outer sep=0pt,anchor=south west,rotate=  0.00] at ( 43.34, 238.12) {
	\pgfimage[width=442.84pt,height= 39.88pt,interpolate=true]{wplot_prob_17_mix.bewa.sm.fr_ras4}};
\end{scope}
\begin{scope}
\path[clip] ( 43.34,186.75) rectangle (486.19,230.63);
\definecolor{drawColor}{gray}{0.92}

\path[draw=drawColor,line width= 0.3pt,line join=round] ( 43.34,200.63) --
	(486.19,200.63);

\path[draw=drawColor,line width= 0.3pt,line join=round] ( 43.34,216.75) --
	(486.19,216.75);

\path[draw=drawColor,line width= 0.3pt,line join=round] (101.03,186.75) --
	(101.03,230.63);

\path[draw=drawColor,line width= 0.3pt,line join=round] (210.69,186.75) --
	(210.69,230.63);

\path[draw=drawColor,line width= 0.3pt,line join=round] (320.65,186.75) --
	(320.65,230.63);

\path[draw=drawColor,line width= 0.3pt,line join=round] (430.61,186.75) --
	(430.61,230.63);

\path[draw=drawColor,line width= 0.6pt,line join=round] ( 43.34,192.58) --
	(486.19,192.58);

\path[draw=drawColor,line width= 0.6pt,line join=round] ( 43.34,208.69) --
	(486.19,208.69);

\path[draw=drawColor,line width= 0.6pt,line join=round] ( 43.34,224.81) --
	(486.19,224.81);

\path[draw=drawColor,line width= 0.6pt,line join=round] ( 46.65,186.75) --
	( 46.65,230.63);

\path[draw=drawColor,line width= 0.6pt,line join=round] (155.41,186.75) --
	(155.41,230.63);

\path[draw=drawColor,line width= 0.6pt,line join=round] (265.97,186.75) --
	(265.97,230.63);

\path[draw=drawColor,line width= 0.6pt,line join=round] (375.32,186.75) --
	(375.32,230.63);

\path[draw=drawColor,line width= 0.6pt,line join=round] (485.89,186.75) --
	(485.89,230.63);
\node[inner sep=0pt,outer sep=0pt,anchor=south west,rotate=  0.00] at ( 43.34, 188.75) {
	\pgfimage[width=442.84pt,height= 39.88pt,interpolate=true]{wplot_prob_17_mix.bewa.sm.fr_ras5}};
\end{scope}
\begin{scope}
\path[clip] ( 43.34,137.38) rectangle (486.19,181.25);
\definecolor{drawColor}{gray}{0.92}

\path[draw=drawColor,line width= 0.3pt,line join=round] ( 43.34,151.26) --
	(486.19,151.26);

\path[draw=drawColor,line width= 0.3pt,line join=round] ( 43.34,167.37) --
	(486.19,167.37);

\path[draw=drawColor,line width= 0.3pt,line join=round] (101.03,137.38) --
	(101.03,181.25);

\path[draw=drawColor,line width= 0.3pt,line join=round] (210.69,137.38) --
	(210.69,181.25);

\path[draw=drawColor,line width= 0.3pt,line join=round] (320.65,137.38) --
	(320.65,181.25);

\path[draw=drawColor,line width= 0.3pt,line join=round] (430.61,137.38) --
	(430.61,181.25);

\path[draw=drawColor,line width= 0.6pt,line join=round] ( 43.34,143.20) --
	(486.19,143.20);

\path[draw=drawColor,line width= 0.6pt,line join=round] ( 43.34,159.32) --
	(486.19,159.32);

\path[draw=drawColor,line width= 0.6pt,line join=round] ( 43.34,175.43) --
	(486.19,175.43);

\path[draw=drawColor,line width= 0.6pt,line join=round] ( 46.65,137.38) --
	( 46.65,181.25);

\path[draw=drawColor,line width= 0.6pt,line join=round] (155.41,137.38) --
	(155.41,181.25);

\path[draw=drawColor,line width= 0.6pt,line join=round] (265.97,137.38) --
	(265.97,181.25);

\path[draw=drawColor,line width= 0.6pt,line join=round] (375.32,137.38) --
	(375.32,181.25);

\path[draw=drawColor,line width= 0.6pt,line join=round] (485.89,137.38) --
	(485.89,181.25);
\node[inner sep=0pt,outer sep=0pt,anchor=south west,rotate=  0.00] at ( 43.34, 139.37) {
	\pgfimage[width=442.84pt,height= 39.88pt,interpolate=true]{wplot_prob_17_mix.bewa.sm.fr_ras6}};
\end{scope}
\begin{scope}
\path[clip] ( 43.34, 88.01) rectangle (486.19,131.88);
\definecolor{drawColor}{gray}{0.92}

\path[draw=drawColor,line width= 0.3pt,line join=round] ( 43.34,101.89) --
	(486.19,101.89);

\path[draw=drawColor,line width= 0.3pt,line join=round] ( 43.34,118.00) --
	(486.19,118.00);

\path[draw=drawColor,line width= 0.3pt,line join=round] (101.03, 88.01) --
	(101.03,131.88);

\path[draw=drawColor,line width= 0.3pt,line join=round] (210.69, 88.01) --
	(210.69,131.88);

\path[draw=drawColor,line width= 0.3pt,line join=round] (320.65, 88.01) --
	(320.65,131.88);

\path[draw=drawColor,line width= 0.3pt,line join=round] (430.61, 88.01) --
	(430.61,131.88);

\path[draw=drawColor,line width= 0.6pt,line join=round] ( 43.34, 93.83) --
	(486.19, 93.83);

\path[draw=drawColor,line width= 0.6pt,line join=round] ( 43.34,109.94) --
	(486.19,109.94);

\path[draw=drawColor,line width= 0.6pt,line join=round] ( 43.34,126.06) --
	(486.19,126.06);

\path[draw=drawColor,line width= 0.6pt,line join=round] ( 46.65, 88.01) --
	( 46.65,131.88);

\path[draw=drawColor,line width= 0.6pt,line join=round] (155.41, 88.01) --
	(155.41,131.88);

\path[draw=drawColor,line width= 0.6pt,line join=round] (265.97, 88.01) --
	(265.97,131.88);

\path[draw=drawColor,line width= 0.6pt,line join=round] (375.32, 88.01) --
	(375.32,131.88);

\path[draw=drawColor,line width= 0.6pt,line join=round] (485.89, 88.01) --
	(485.89,131.88);
\node[inner sep=0pt,outer sep=0pt,anchor=south west,rotate=  0.00] at ( 43.34,  90.00) {
	\pgfimage[width=442.84pt,height= 39.88pt,interpolate=true]{wplot_prob_17_mix.bewa.sm.fr_ras7}};
\end{scope}
\begin{scope}
\path[clip] ( 43.34, 38.63) rectangle (486.19, 82.51);
\definecolor{drawColor}{gray}{0.92}

\path[draw=drawColor,line width= 0.3pt,line join=round] ( 43.34, 52.51) --
	(486.19, 52.51);

\path[draw=drawColor,line width= 0.3pt,line join=round] ( 43.34, 68.63) --
	(486.19, 68.63);

\path[draw=drawColor,line width= 0.3pt,line join=round] (101.03, 38.63) --
	(101.03, 82.51);

\path[draw=drawColor,line width= 0.3pt,line join=round] (210.69, 38.63) --
	(210.69, 82.51);

\path[draw=drawColor,line width= 0.3pt,line join=round] (320.65, 38.63) --
	(320.65, 82.51);

\path[draw=drawColor,line width= 0.3pt,line join=round] (430.61, 38.63) --
	(430.61, 82.51);

\path[draw=drawColor,line width= 0.6pt,line join=round] ( 43.34, 44.46) --
	(486.19, 44.46);

\path[draw=drawColor,line width= 0.6pt,line join=round] ( 43.34, 60.57) --
	(486.19, 60.57);

\path[draw=drawColor,line width= 0.6pt,line join=round] ( 43.34, 76.69) --
	(486.19, 76.69);

\path[draw=drawColor,line width= 0.6pt,line join=round] ( 46.65, 38.63) --
	( 46.65, 82.51);

\path[draw=drawColor,line width= 0.6pt,line join=round] (155.41, 38.63) --
	(155.41, 82.51);

\path[draw=drawColor,line width= 0.6pt,line join=round] (265.97, 38.63) --
	(265.97, 82.51);

\path[draw=drawColor,line width= 0.6pt,line join=round] (375.32, 38.63) --
	(375.32, 82.51);

\path[draw=drawColor,line width= 0.6pt,line join=round] (485.89, 38.63) --
	(485.89, 82.51);
\node[inner sep=0pt,outer sep=0pt,anchor=south west,rotate=  0.00] at ( 43.34,  40.63) {
	\pgfimage[width=442.84pt,height= 39.88pt,interpolate=true]{wplot_prob_17_mix.bewa.sm.fr_ras8}};
\end{scope}
\begin{scope}
\path[clip] (486.19,384.25) rectangle (506.29,428.12);
\definecolor{drawColor}{gray}{0.10}

\node[text=drawColor,rotate=-90.00,anchor=base,inner sep=0pt, outer sep=0pt, scale=  1.07] at (491.83,406.18) {JSU1};
\end{scope}
\begin{scope}
\path[clip] (486.19,334.87) rectangle (506.29,378.75);
\definecolor{drawColor}{gray}{0.10}

\node[text=drawColor,rotate=-90.00,anchor=base,inner sep=0pt, outer sep=0pt, scale=  1.07] at (491.83,356.81) {JSU2};
\end{scope}
\begin{scope}
\path[clip] (486.19,285.50) rectangle (506.29,329.37);
\definecolor{drawColor}{gray}{0.10}

\node[text=drawColor,rotate=-90.00,anchor=base,inner sep=0pt, outer sep=0pt, scale=  1.07] at (491.83,307.44) {JSU3};
\end{scope}
\begin{scope}
\path[clip] (486.19,236.13) rectangle (506.29,280.00);
\definecolor{drawColor}{gray}{0.10}

\node[text=drawColor,rotate=-90.00,anchor=base,inner sep=0pt, outer sep=0pt, scale=  1.07] at (491.83,258.06) {JSU4};
\end{scope}
\begin{scope}
\path[clip] (486.19,186.75) rectangle (506.29,230.63);
\definecolor{drawColor}{gray}{0.10}

\node[text=drawColor,rotate=-90.00,anchor=base,inner sep=0pt, outer sep=0pt, scale=  1.07] at (491.83,208.69) {Norm1};
\end{scope}
\begin{scope}
\path[clip] (486.19,137.38) rectangle (506.29,181.25);
\definecolor{drawColor}{gray}{0.10}

\node[text=drawColor,rotate=-90.00,anchor=base,inner sep=0pt, outer sep=0pt, scale=  1.07] at (491.83,159.32) {Norm2};
\end{scope}
\begin{scope}
\path[clip] (486.19, 88.01) rectangle (506.29,131.88);
\definecolor{drawColor}{gray}{0.10}

\node[text=drawColor,rotate=-90.00,anchor=base,inner sep=0pt, outer sep=0pt, scale=  1.07] at (491.83,109.94) {Norm3};
\end{scope}
\begin{scope}
\path[clip] (486.19, 38.63) rectangle (506.29, 82.51);
\definecolor{drawColor}{gray}{0.10}

\node[text=drawColor,rotate=-90.00,anchor=base,inner sep=0pt, outer sep=0pt, scale=  1.07] at (491.83, 60.57) {Norm4};
\end{scope}
\begin{scope}
\path[clip] (  0.00,  0.00) rectangle (578.16,433.62);
\definecolor{drawColor}{gray}{0.30}

\node[text=drawColor,anchor=base,inner sep=0pt, outer sep=0pt, scale=  1.07] at ( 46.65, 24.87) {2019-01};

\node[text=drawColor,anchor=base,inner sep=0pt, outer sep=0pt, scale=  1.07] at (155.41, 24.87) {2019-07};

\node[text=drawColor,anchor=base,inner sep=0pt, outer sep=0pt, scale=  1.07] at (265.97, 24.87) {2020-01};

\node[text=drawColor,anchor=base,inner sep=0pt, outer sep=0pt, scale=  1.07] at (375.32, 24.87) {2020-07};

\node[text=drawColor,anchor=base,inner sep=0pt, outer sep=0pt, scale=  1.07] at (485.89, 24.87) {2021-01};
\end{scope}
\begin{scope}
\path[clip] (  0.00,  0.00) rectangle (578.16,433.62);
\definecolor{drawColor}{gray}{0.30}

\node[text=drawColor,anchor=base east,inner sep=0pt, outer sep=0pt, scale=  1.07] at ( 38.39,385.66) {0.1};

\node[text=drawColor,anchor=base east,inner sep=0pt, outer sep=0pt, scale=  1.07] at ( 38.39,401.77) {0.5};

\node[text=drawColor,anchor=base east,inner sep=0pt, outer sep=0pt, scale=  1.07] at ( 38.39,417.89) {0.9};
\end{scope}
\begin{scope}
\path[clip] (  0.00,  0.00) rectangle (578.16,433.62);
\definecolor{drawColor}{gray}{0.30}

\node[text=drawColor,anchor=base east,inner sep=0pt, outer sep=0pt, scale=  1.07] at ( 38.39,336.29) {0.1};

\node[text=drawColor,anchor=base east,inner sep=0pt, outer sep=0pt, scale=  1.07] at ( 38.39,352.40) {0.5};

\node[text=drawColor,anchor=base east,inner sep=0pt, outer sep=0pt, scale=  1.07] at ( 38.39,368.52) {0.9};
\end{scope}
\begin{scope}
\path[clip] (  0.00,  0.00) rectangle (578.16,433.62);
\definecolor{drawColor}{gray}{0.30}

\node[text=drawColor,anchor=base east,inner sep=0pt, outer sep=0pt, scale=  1.07] at ( 38.39,286.91) {0.1};

\node[text=drawColor,anchor=base east,inner sep=0pt, outer sep=0pt, scale=  1.07] at ( 38.39,303.03) {0.5};

\node[text=drawColor,anchor=base east,inner sep=0pt, outer sep=0pt, scale=  1.07] at ( 38.39,319.14) {0.9};
\end{scope}
\begin{scope}
\path[clip] (  0.00,  0.00) rectangle (578.16,433.62);
\definecolor{drawColor}{gray}{0.30}

\node[text=drawColor,anchor=base east,inner sep=0pt, outer sep=0pt, scale=  1.07] at ( 38.39,237.54) {0.1};

\node[text=drawColor,anchor=base east,inner sep=0pt, outer sep=0pt, scale=  1.07] at ( 38.39,253.65) {0.5};

\node[text=drawColor,anchor=base east,inner sep=0pt, outer sep=0pt, scale=  1.07] at ( 38.39,269.77) {0.9};
\end{scope}
\begin{scope}
\path[clip] (  0.00,  0.00) rectangle (578.16,433.62);
\definecolor{drawColor}{gray}{0.30}

\node[text=drawColor,anchor=base east,inner sep=0pt, outer sep=0pt, scale=  1.07] at ( 38.39,188.17) {0.1};

\node[text=drawColor,anchor=base east,inner sep=0pt, outer sep=0pt, scale=  1.07] at ( 38.39,204.28) {0.5};

\node[text=drawColor,anchor=base east,inner sep=0pt, outer sep=0pt, scale=  1.07] at ( 38.39,220.40) {0.9};
\end{scope}
\begin{scope}
\path[clip] (  0.00,  0.00) rectangle (578.16,433.62);
\definecolor{drawColor}{gray}{0.30}

\node[text=drawColor,anchor=base east,inner sep=0pt, outer sep=0pt, scale=  1.07] at ( 38.39,138.79) {0.1};

\node[text=drawColor,anchor=base east,inner sep=0pt, outer sep=0pt, scale=  1.07] at ( 38.39,154.91) {0.5};

\node[text=drawColor,anchor=base east,inner sep=0pt, outer sep=0pt, scale=  1.07] at ( 38.39,171.02) {0.9};
\end{scope}
\begin{scope}
\path[clip] (  0.00,  0.00) rectangle (578.16,433.62);
\definecolor{drawColor}{gray}{0.30}

\node[text=drawColor,anchor=base east,inner sep=0pt, outer sep=0pt, scale=  1.07] at ( 38.39, 89.42) {0.1};

\node[text=drawColor,anchor=base east,inner sep=0pt, outer sep=0pt, scale=  1.07] at ( 38.39,105.53) {0.5};

\node[text=drawColor,anchor=base east,inner sep=0pt, outer sep=0pt, scale=  1.07] at ( 38.39,121.65) {0.9};
\end{scope}
\begin{scope}
\path[clip] (  0.00,  0.00) rectangle (578.16,433.62);
\definecolor{drawColor}{gray}{0.30}

\node[text=drawColor,anchor=base east,inner sep=0pt, outer sep=0pt, scale=  1.07] at ( 38.39, 40.05) {0.1};

\node[text=drawColor,anchor=base east,inner sep=0pt, outer sep=0pt, scale=  1.07] at ( 38.39, 56.16) {0.5};

\node[text=drawColor,anchor=base east,inner sep=0pt, outer sep=0pt, scale=  1.07] at ( 38.39, 72.28) {0.9};
\end{scope}
\begin{scope}
\path[clip] (  0.00,  0.00) rectangle (578.16,433.62);
\definecolor{drawColor}{RGB}{0,0,0}

\node[text=drawColor,anchor=base,inner sep=0pt, outer sep=0pt, scale=  1.33] at (264.76,  8.61) {date};
\end{scope}
\begin{scope}
\path[clip] (  0.00,  0.00) rectangle (578.16,433.62);
\definecolor{drawColor}{RGB}{0,0,0}

\node[text=drawColor,rotate= 90.00,anchor=base,inner sep=0pt, outer sep=0pt, scale=  1.33] at ( 16.52,233.38) {probability};
\end{scope}
\begin{scope}
\path[clip] (  0.00,  0.00) rectangle (578.16,433.62);
\node[inner sep=0pt,outer sep=0pt,anchor=south west,rotate=  0.00] at (522.79,  59.71) {
	\pgfimage[width= 17.34pt,height=325.21pt,interpolate=true]{wplot_prob_17_mix.bewa.sm.fr_ras9}};
\end{scope}
\begin{scope}
\path[clip] (  0.00,  0.00) rectangle (578.16,433.62);
\definecolor{drawColor}{RGB}{0,0,0}

\node[text=drawColor,anchor=base west,inner sep=0pt, outer sep=0pt, scale=  1.07] at (548.14, 55.84) {0.0};

\node[text=drawColor,anchor=base west,inner sep=0pt, outer sep=0pt, scale=  1.07] at (548.14,120.67) {0.1};

\node[text=drawColor,anchor=base west,inner sep=0pt, outer sep=0pt, scale=  1.07] at (548.14,185.49) {0.2};

\node[text=drawColor,anchor=base west,inner sep=0pt, outer sep=0pt, scale=  1.07] at (548.14,250.32) {0.3};

\node[text=drawColor,anchor=base west,inner sep=0pt, outer sep=0pt, scale=  1.07] at (548.14,315.14) {0.4};

\node[text=drawColor,anchor=base west,inner sep=0pt, outer sep=0pt, scale=  1.07] at (548.14,379.97) {0.5};
\end{scope}
\begin{scope}
\path[clip] (  0.00,  0.00) rectangle (578.16,433.62);
\definecolor{drawColor}{RGB}{0,0,0}

\node[text=drawColor,anchor=base west,inner sep=0pt, outer sep=0pt, scale=  1.33] at (522.79,394.48) {weight};
\end{scope}
\begin{scope}
\path[clip] (  0.00,  0.00) rectangle (578.16,433.62);
\definecolor{drawColor}{RGB}{255,255,255}

\path[draw=drawColor,line width= 0.2pt,line join=round] (522.79, 60.25) -- (526.26, 60.25);

\path[draw=drawColor,line width= 0.2pt,line join=round] (522.79,125.07) -- (526.26,125.07);

\path[draw=drawColor,line width= 0.2pt,line join=round] (522.79,189.90) -- (526.26,189.90);

\path[draw=drawColor,line width= 0.2pt,line join=round] (522.79,254.73) -- (526.26,254.73);

\path[draw=drawColor,line width= 0.2pt,line join=round] (522.79,319.55) -- (526.26,319.55);

\path[draw=drawColor,line width= 0.2pt,line join=round] (522.79,384.38) -- (526.26,384.38);

\path[draw=drawColor,line width= 0.2pt,line join=round] (536.67, 60.25) -- (540.14, 60.25);

\path[draw=drawColor,line width= 0.2pt,line join=round] (536.67,125.07) -- (540.14,125.07);

\path[draw=drawColor,line width= 0.2pt,line join=round] (536.67,189.90) -- (540.14,189.90);

\path[draw=drawColor,line width= 0.2pt,line join=round] (536.67,254.73) -- (540.14,254.73);

\path[draw=drawColor,line width= 0.2pt,line join=round] (536.67,319.55) -- (540.14,319.55);

\path[draw=drawColor,line width= 0.2pt,line join=round] (536.67,384.38) -- (540.14,384.38);
\end{scope}
\end{tikzpicture}
}
    }}
  \caption{Temporal evolution of the weights at hour 16 across all 99 probabilities}\label{weights_temporal_vs_prob}
\end{figure}

Figures~\ref{weights_temporal_vs_prob} and~\ref{weights_temporal_vs_hour} provide a more detailed analysis of the proposed model (\textbf{Smooth.Forget}, \textbf{Bayesian Online}). They depict the temporal evolution of the weights for each expert. Thereby, Figure~\ref{weights_temporal_vs_prob} presents the temporal evolution across probabilities for hour 16 of the day. After a brief initial burn-in period, the weights of the eight experts stabilize. There is a higher degree of variability in the center of the distribution. The weights are close to the uniform solution at the tails, with only a few exceptions. JSU4 strongly influences the combined value in the center of the distribution until around April 2020. After that point, the weight of JSU4 decreases, and JSU3 becomes more prominent. Further, there seem to be noticeable changes in the weights around June 2019 and April 2020, suggesting possible structural changes. These structural changes potentially lead to the dynamic hyperparameter tuning \textbf{Sampling Online} and \textbf{Bayesian Online} performing better than \textbf{Bayesian fix} due to the ability of the hyperparameters to adapt the changing data.

Figure~\ref{weights_temporal_vs_hour} shows the temporal evolution of the weights at the median across all 24 hours. However, the high weights for JSU4 (see Figure~\ref{weights_temporal_vs_prob}) are only present in the afternoon and evening. Additionally, the plot reveals structural changes around June 2019 (regarding JSU4 and NORM4) and April 2020 (concerning JSU3, JSU4, NORM3, and NORM4).

Both graphs clearly indicate that the weights vary with time, hours, and quantiles. Thus, a flexible approach like the one proposed in this study seems appropriate. Lastly, the weights show less smoothing across hours than across quantiles.

\begin{figure}[!h]
  \centering{
    \resizebox{\textwidth}{!}{%
      \fbox{% !TEX encoding = UTF-8 Unicode
\begin{tikzpicture}[x=1pt,y=1pt]
\definecolor{fillColor}{RGB}{255,255,255}
\path[use as bounding box,fill=fillColor,fill opacity=0.00] (0,0) rectangle (578.16,433.62);
\begin{scope}
\path[clip] ( 39.86,384.25) rectangle (486.19,428.12);
\definecolor{drawColor}{gray}{0.92}

\path[draw=drawColor,line width= 0.3pt,line join=round] ( 39.86,393.72) --
	(486.19,393.72);

\path[draw=drawColor,line width= 0.3pt,line join=round] ( 39.86,407.01) --
	(486.19,407.01);

\path[draw=drawColor,line width= 0.3pt,line join=round] ( 39.86,420.31) --
	(486.19,420.31);

\path[draw=drawColor,line width= 0.3pt,line join=round] ( 98.00,384.25) --
	( 98.00,428.12);

\path[draw=drawColor,line width= 0.3pt,line join=round] (208.52,384.25) --
	(208.52,428.12);

\path[draw=drawColor,line width= 0.3pt,line join=round] (319.34,384.25) --
	(319.34,428.12);

\path[draw=drawColor,line width= 0.3pt,line join=round] (430.17,384.25) --
	(430.17,428.12);

\path[draw=drawColor,line width= 0.6pt,line join=round] ( 39.86,387.07) --
	(486.19,387.07);

\path[draw=drawColor,line width= 0.6pt,line join=round] ( 39.86,400.37) --
	(486.19,400.37);

\path[draw=drawColor,line width= 0.6pt,line join=round] ( 39.86,413.66) --
	(486.19,413.66);

\path[draw=drawColor,line width= 0.6pt,line join=round] ( 39.86,426.96) --
	(486.19,426.96);

\path[draw=drawColor,line width= 0.6pt,line join=round] ( 43.19,384.25) --
	( 43.19,428.12);

\path[draw=drawColor,line width= 0.6pt,line join=round] (152.80,384.25) --
	(152.80,428.12);

\path[draw=drawColor,line width= 0.6pt,line join=round] (264.23,384.25) --
	(264.23,428.12);

\path[draw=drawColor,line width= 0.6pt,line join=round] (374.45,384.25) --
	(374.45,428.12);

\path[draw=drawColor,line width= 0.6pt,line join=round] (485.88,384.25) --
	(485.88,428.12);
\node[inner sep=0pt,outer sep=0pt,anchor=south west,rotate=  0.00] at ( 39.86, 386.24) {
	\pgfimage[width=446.33pt,height= 39.88pt,interpolate=true]{wplot_mv_0.5_mix.bewa.sm.fr_ras1}};
\end{scope}
\begin{scope}
\path[clip] ( 39.86,334.87) rectangle (486.19,378.75);
\definecolor{drawColor}{gray}{0.92}

\path[draw=drawColor,line width= 0.3pt,line join=round] ( 39.86,344.35) --
	(486.19,344.35);

\path[draw=drawColor,line width= 0.3pt,line join=round] ( 39.86,357.64) --
	(486.19,357.64);

\path[draw=drawColor,line width= 0.3pt,line join=round] ( 39.86,370.94) --
	(486.19,370.94);

\path[draw=drawColor,line width= 0.3pt,line join=round] ( 98.00,334.87) --
	( 98.00,378.75);

\path[draw=drawColor,line width= 0.3pt,line join=round] (208.52,334.87) --
	(208.52,378.75);

\path[draw=drawColor,line width= 0.3pt,line join=round] (319.34,334.87) --
	(319.34,378.75);

\path[draw=drawColor,line width= 0.3pt,line join=round] (430.17,334.87) --
	(430.17,378.75);

\path[draw=drawColor,line width= 0.6pt,line join=round] ( 39.86,337.70) --
	(486.19,337.70);

\path[draw=drawColor,line width= 0.6pt,line join=round] ( 39.86,350.99) --
	(486.19,350.99);

\path[draw=drawColor,line width= 0.6pt,line join=round] ( 39.86,364.29) --
	(486.19,364.29);

\path[draw=drawColor,line width= 0.6pt,line join=round] ( 39.86,377.58) --
	(486.19,377.58);

\path[draw=drawColor,line width= 0.6pt,line join=round] ( 43.19,334.87) --
	( 43.19,378.75);

\path[draw=drawColor,line width= 0.6pt,line join=round] (152.80,334.87) --
	(152.80,378.75);

\path[draw=drawColor,line width= 0.6pt,line join=round] (264.23,334.87) --
	(264.23,378.75);

\path[draw=drawColor,line width= 0.6pt,line join=round] (374.45,334.87) --
	(374.45,378.75);

\path[draw=drawColor,line width= 0.6pt,line join=round] (485.88,334.87) --
	(485.88,378.75);
\node[inner sep=0pt,outer sep=0pt,anchor=south west,rotate=  0.00] at ( 39.86, 336.87) {
	\pgfimage[width=446.33pt,height= 39.88pt,interpolate=true]{wplot_mv_0.5_mix.bewa.sm.fr_ras2}};
\end{scope}
\begin{scope}
\path[clip] ( 39.86,285.50) rectangle (486.19,329.37);
\definecolor{drawColor}{gray}{0.92}

\path[draw=drawColor,line width= 0.3pt,line join=round] ( 39.86,294.97) --
	(486.19,294.97);

\path[draw=drawColor,line width= 0.3pt,line join=round] ( 39.86,308.27) --
	(486.19,308.27);

\path[draw=drawColor,line width= 0.3pt,line join=round] ( 39.86,321.56) --
	(486.19,321.56);

\path[draw=drawColor,line width= 0.3pt,line join=round] ( 98.00,285.50) --
	( 98.00,329.37);

\path[draw=drawColor,line width= 0.3pt,line join=round] (208.52,285.50) --
	(208.52,329.37);

\path[draw=drawColor,line width= 0.3pt,line join=round] (319.34,285.50) --
	(319.34,329.37);

\path[draw=drawColor,line width= 0.3pt,line join=round] (430.17,285.50) --
	(430.17,329.37);

\path[draw=drawColor,line width= 0.6pt,line join=round] ( 39.86,288.33) --
	(486.19,288.33);

\path[draw=drawColor,line width= 0.6pt,line join=round] ( 39.86,301.62) --
	(486.19,301.62);

\path[draw=drawColor,line width= 0.6pt,line join=round] ( 39.86,314.92) --
	(486.19,314.92);

\path[draw=drawColor,line width= 0.6pt,line join=round] ( 39.86,328.21) --
	(486.19,328.21);

\path[draw=drawColor,line width= 0.6pt,line join=round] ( 43.19,285.50) --
	( 43.19,329.37);

\path[draw=drawColor,line width= 0.6pt,line join=round] (152.80,285.50) --
	(152.80,329.37);

\path[draw=drawColor,line width= 0.6pt,line join=round] (264.23,285.50) --
	(264.23,329.37);

\path[draw=drawColor,line width= 0.6pt,line join=round] (374.45,285.50) --
	(374.45,329.37);

\path[draw=drawColor,line width= 0.6pt,line join=round] (485.88,285.50) --
	(485.88,329.37);
\node[inner sep=0pt,outer sep=0pt,anchor=south west,rotate=  0.00] at ( 39.86, 287.49) {
	\pgfimage[width=446.33pt,height= 39.88pt,interpolate=true]{wplot_mv_0.5_mix.bewa.sm.fr_ras3}};
\end{scope}
\begin{scope}
\path[clip] ( 39.86,236.13) rectangle (486.19,280.00);
\definecolor{drawColor}{gray}{0.92}

\path[draw=drawColor,line width= 0.3pt,line join=round] ( 39.86,245.60) --
	(486.19,245.60);

\path[draw=drawColor,line width= 0.3pt,line join=round] ( 39.86,258.89) --
	(486.19,258.89);

\path[draw=drawColor,line width= 0.3pt,line join=round] ( 39.86,272.19) --
	(486.19,272.19);

\path[draw=drawColor,line width= 0.3pt,line join=round] ( 98.00,236.13) --
	( 98.00,280.00);

\path[draw=drawColor,line width= 0.3pt,line join=round] (208.52,236.13) --
	(208.52,280.00);

\path[draw=drawColor,line width= 0.3pt,line join=round] (319.34,236.13) --
	(319.34,280.00);

\path[draw=drawColor,line width= 0.3pt,line join=round] (430.17,236.13) --
	(430.17,280.00);

\path[draw=drawColor,line width= 0.6pt,line join=round] ( 39.86,238.95) --
	(486.19,238.95);

\path[draw=drawColor,line width= 0.6pt,line join=round] ( 39.86,252.25) --
	(486.19,252.25);

\path[draw=drawColor,line width= 0.6pt,line join=round] ( 39.86,265.54) --
	(486.19,265.54);

\path[draw=drawColor,line width= 0.6pt,line join=round] ( 39.86,278.84) --
	(486.19,278.84);

\path[draw=drawColor,line width= 0.6pt,line join=round] ( 43.19,236.13) --
	( 43.19,280.00);

\path[draw=drawColor,line width= 0.6pt,line join=round] (152.80,236.13) --
	(152.80,280.00);

\path[draw=drawColor,line width= 0.6pt,line join=round] (264.23,236.13) --
	(264.23,280.00);

\path[draw=drawColor,line width= 0.6pt,line join=round] (374.45,236.13) --
	(374.45,280.00);

\path[draw=drawColor,line width= 0.6pt,line join=round] (485.88,236.13) --
	(485.88,280.00);
\node[inner sep=0pt,outer sep=0pt,anchor=south west,rotate=  0.00] at ( 39.86, 238.12) {
	\pgfimage[width=446.33pt,height= 39.88pt,interpolate=true]{wplot_mv_0.5_mix.bewa.sm.fr_ras4}};
\end{scope}
\begin{scope}
\path[clip] ( 39.86,186.75) rectangle (486.19,230.63);
\definecolor{drawColor}{gray}{0.92}

\path[draw=drawColor,line width= 0.3pt,line join=round] ( 39.86,196.23) --
	(486.19,196.23);

\path[draw=drawColor,line width= 0.3pt,line join=round] ( 39.86,209.52) --
	(486.19,209.52);

\path[draw=drawColor,line width= 0.3pt,line join=round] ( 39.86,222.82) --
	(486.19,222.82);

\path[draw=drawColor,line width= 0.3pt,line join=round] ( 98.00,186.75) --
	( 98.00,230.63);

\path[draw=drawColor,line width= 0.3pt,line join=round] (208.52,186.75) --
	(208.52,230.63);

\path[draw=drawColor,line width= 0.3pt,line join=round] (319.34,186.75) --
	(319.34,230.63);

\path[draw=drawColor,line width= 0.3pt,line join=round] (430.17,186.75) --
	(430.17,230.63);

\path[draw=drawColor,line width= 0.6pt,line join=round] ( 39.86,189.58) --
	(486.19,189.58);

\path[draw=drawColor,line width= 0.6pt,line join=round] ( 39.86,202.87) --
	(486.19,202.87);

\path[draw=drawColor,line width= 0.6pt,line join=round] ( 39.86,216.17) --
	(486.19,216.17);

\path[draw=drawColor,line width= 0.6pt,line join=round] ( 39.86,229.46) --
	(486.19,229.46);

\path[draw=drawColor,line width= 0.6pt,line join=round] ( 43.19,186.75) --
	( 43.19,230.63);

\path[draw=drawColor,line width= 0.6pt,line join=round] (152.80,186.75) --
	(152.80,230.63);

\path[draw=drawColor,line width= 0.6pt,line join=round] (264.23,186.75) --
	(264.23,230.63);

\path[draw=drawColor,line width= 0.6pt,line join=round] (374.45,186.75) --
	(374.45,230.63);

\path[draw=drawColor,line width= 0.6pt,line join=round] (485.88,186.75) --
	(485.88,230.63);
\node[inner sep=0pt,outer sep=0pt,anchor=south west,rotate=  0.00] at ( 39.86, 188.75) {
	\pgfimage[width=446.33pt,height= 39.88pt,interpolate=true]{wplot_mv_0.5_mix.bewa.sm.fr_ras5}};
\end{scope}
\begin{scope}
\path[clip] ( 39.86,137.38) rectangle (486.19,181.25);
\definecolor{drawColor}{gray}{0.92}

\path[draw=drawColor,line width= 0.3pt,line join=round] ( 39.86,146.85) --
	(486.19,146.85);

\path[draw=drawColor,line width= 0.3pt,line join=round] ( 39.86,160.15) --
	(486.19,160.15);

\path[draw=drawColor,line width= 0.3pt,line join=round] ( 39.86,173.44) --
	(486.19,173.44);

\path[draw=drawColor,line width= 0.3pt,line join=round] ( 98.00,137.38) --
	( 98.00,181.25);

\path[draw=drawColor,line width= 0.3pt,line join=round] (208.52,137.38) --
	(208.52,181.25);

\path[draw=drawColor,line width= 0.3pt,line join=round] (319.34,137.38) --
	(319.34,181.25);

\path[draw=drawColor,line width= 0.3pt,line join=round] (430.17,137.38) --
	(430.17,181.25);

\path[draw=drawColor,line width= 0.6pt,line join=round] ( 39.86,140.21) --
	(486.19,140.21);

\path[draw=drawColor,line width= 0.6pt,line join=round] ( 39.86,153.50) --
	(486.19,153.50);

\path[draw=drawColor,line width= 0.6pt,line join=round] ( 39.86,166.80) --
	(486.19,166.80);

\path[draw=drawColor,line width= 0.6pt,line join=round] ( 39.86,180.09) --
	(486.19,180.09);

\path[draw=drawColor,line width= 0.6pt,line join=round] ( 43.19,137.38) --
	( 43.19,181.25);

\path[draw=drawColor,line width= 0.6pt,line join=round] (152.80,137.38) --
	(152.80,181.25);

\path[draw=drawColor,line width= 0.6pt,line join=round] (264.23,137.38) --
	(264.23,181.25);

\path[draw=drawColor,line width= 0.6pt,line join=round] (374.45,137.38) --
	(374.45,181.25);

\path[draw=drawColor,line width= 0.6pt,line join=round] (485.88,137.38) --
	(485.88,181.25);
\node[inner sep=0pt,outer sep=0pt,anchor=south west,rotate=  0.00] at ( 39.86, 139.37) {
	\pgfimage[width=446.33pt,height= 39.88pt,interpolate=true]{wplot_mv_0.5_mix.bewa.sm.fr_ras6}};
\end{scope}
\begin{scope}
\path[clip] ( 39.86, 88.01) rectangle (486.19,131.88);
\definecolor{drawColor}{gray}{0.92}

\path[draw=drawColor,line width= 0.3pt,line join=round] ( 39.86, 97.48) --
	(486.19, 97.48);

\path[draw=drawColor,line width= 0.3pt,line join=round] ( 39.86,110.77) --
	(486.19,110.77);

\path[draw=drawColor,line width= 0.3pt,line join=round] ( 39.86,124.07) --
	(486.19,124.07);

\path[draw=drawColor,line width= 0.3pt,line join=round] ( 98.00, 88.01) --
	( 98.00,131.88);

\path[draw=drawColor,line width= 0.3pt,line join=round] (208.52, 88.01) --
	(208.52,131.88);

\path[draw=drawColor,line width= 0.3pt,line join=round] (319.34, 88.01) --
	(319.34,131.88);

\path[draw=drawColor,line width= 0.3pt,line join=round] (430.17, 88.01) --
	(430.17,131.88);

\path[draw=drawColor,line width= 0.6pt,line join=round] ( 39.86, 90.83) --
	(486.19, 90.83);

\path[draw=drawColor,line width= 0.6pt,line join=round] ( 39.86,104.13) --
	(486.19,104.13);

\path[draw=drawColor,line width= 0.6pt,line join=round] ( 39.86,117.42) --
	(486.19,117.42);

\path[draw=drawColor,line width= 0.6pt,line join=round] ( 39.86,130.72) --
	(486.19,130.72);

\path[draw=drawColor,line width= 0.6pt,line join=round] ( 43.19, 88.01) --
	( 43.19,131.88);

\path[draw=drawColor,line width= 0.6pt,line join=round] (152.80, 88.01) --
	(152.80,131.88);

\path[draw=drawColor,line width= 0.6pt,line join=round] (264.23, 88.01) --
	(264.23,131.88);

\path[draw=drawColor,line width= 0.6pt,line join=round] (374.45, 88.01) --
	(374.45,131.88);

\path[draw=drawColor,line width= 0.6pt,line join=round] (485.88, 88.01) --
	(485.88,131.88);
\node[inner sep=0pt,outer sep=0pt,anchor=south west,rotate=  0.00] at ( 39.86,  90.00) {
	\pgfimage[width=446.33pt,height= 39.88pt,interpolate=true]{wplot_mv_0.5_mix.bewa.sm.fr_ras7}};
\end{scope}
\begin{scope}
\path[clip] ( 39.86, 38.63) rectangle (486.19, 82.51);
\definecolor{drawColor}{gray}{0.92}

\path[draw=drawColor,line width= 0.3pt,line join=round] ( 39.86, 48.11) --
	(486.19, 48.11);

\path[draw=drawColor,line width= 0.3pt,line join=round] ( 39.86, 61.40) --
	(486.19, 61.40);

\path[draw=drawColor,line width= 0.3pt,line join=round] ( 39.86, 74.70) --
	(486.19, 74.70);

\path[draw=drawColor,line width= 0.3pt,line join=round] ( 98.00, 38.63) --
	( 98.00, 82.51);

\path[draw=drawColor,line width= 0.3pt,line join=round] (208.52, 38.63) --
	(208.52, 82.51);

\path[draw=drawColor,line width= 0.3pt,line join=round] (319.34, 38.63) --
	(319.34, 82.51);

\path[draw=drawColor,line width= 0.3pt,line join=round] (430.17, 38.63) --
	(430.17, 82.51);

\path[draw=drawColor,line width= 0.6pt,line join=round] ( 39.86, 41.46) --
	(486.19, 41.46);

\path[draw=drawColor,line width= 0.6pt,line join=round] ( 39.86, 54.75) --
	(486.19, 54.75);

\path[draw=drawColor,line width= 0.6pt,line join=round] ( 39.86, 68.05) --
	(486.19, 68.05);

\path[draw=drawColor,line width= 0.6pt,line join=round] ( 39.86, 81.34) --
	(486.19, 81.34);

\path[draw=drawColor,line width= 0.6pt,line join=round] ( 43.19, 38.63) --
	( 43.19, 82.51);

\path[draw=drawColor,line width= 0.6pt,line join=round] (152.80, 38.63) --
	(152.80, 82.51);

\path[draw=drawColor,line width= 0.6pt,line join=round] (264.23, 38.63) --
	(264.23, 82.51);

\path[draw=drawColor,line width= 0.6pt,line join=round] (374.45, 38.63) --
	(374.45, 82.51);

\path[draw=drawColor,line width= 0.6pt,line join=round] (485.88, 38.63) --
	(485.88, 82.51);
\node[inner sep=0pt,outer sep=0pt,anchor=south west,rotate=  0.00] at ( 39.86,  40.63) {
	\pgfimage[width=446.33pt,height= 39.88pt,interpolate=true]{wplot_mv_0.5_mix.bewa.sm.fr_ras8}};
\end{scope}
\begin{scope}
\path[clip] (486.19,384.25) rectangle (506.29,428.12);
\definecolor{drawColor}{gray}{0.10}

\node[text=drawColor,rotate=-90.00,anchor=base,inner sep=0pt, outer sep=0pt, scale=  1.07] at (491.83,406.18) {JSU1};
\end{scope}
\begin{scope}
\path[clip] (486.19,334.87) rectangle (506.29,378.75);
\definecolor{drawColor}{gray}{0.10}

\node[text=drawColor,rotate=-90.00,anchor=base,inner sep=0pt, outer sep=0pt, scale=  1.07] at (491.83,356.81) {JSU2};
\end{scope}
\begin{scope}
\path[clip] (486.19,285.50) rectangle (506.29,329.37);
\definecolor{drawColor}{gray}{0.10}

\node[text=drawColor,rotate=-90.00,anchor=base,inner sep=0pt, outer sep=0pt, scale=  1.07] at (491.83,307.44) {JSU3};
\end{scope}
\begin{scope}
\path[clip] (486.19,236.13) rectangle (506.29,280.00);
\definecolor{drawColor}{gray}{0.10}

\node[text=drawColor,rotate=-90.00,anchor=base,inner sep=0pt, outer sep=0pt, scale=  1.07] at (491.83,258.06) {JSU4};
\end{scope}
\begin{scope}
\path[clip] (486.19,186.75) rectangle (506.29,230.63);
\definecolor{drawColor}{gray}{0.10}

\node[text=drawColor,rotate=-90.00,anchor=base,inner sep=0pt, outer sep=0pt, scale=  1.07] at (491.83,208.69) {Norm1};
\end{scope}
\begin{scope}
\path[clip] (486.19,137.38) rectangle (506.29,181.25);
\definecolor{drawColor}{gray}{0.10}

\node[text=drawColor,rotate=-90.00,anchor=base,inner sep=0pt, outer sep=0pt, scale=  1.07] at (491.83,159.32) {Norm2};
\end{scope}
\begin{scope}
\path[clip] (486.19, 88.01) rectangle (506.29,131.88);
\definecolor{drawColor}{gray}{0.10}

\node[text=drawColor,rotate=-90.00,anchor=base,inner sep=0pt, outer sep=0pt, scale=  1.07] at (491.83,109.94) {Norm3};
\end{scope}
\begin{scope}
\path[clip] (486.19, 38.63) rectangle (506.29, 82.51);
\definecolor{drawColor}{gray}{0.10}

\node[text=drawColor,rotate=-90.00,anchor=base,inner sep=0pt, outer sep=0pt, scale=  1.07] at (491.83, 60.57) {Norm4};
\end{scope}
\begin{scope}
\path[clip] (  0.00,  0.00) rectangle (578.16,433.62);
\definecolor{drawColor}{gray}{0.30}

\node[text=drawColor,anchor=base,inner sep=0pt, outer sep=0pt, scale=  1.07] at ( 43.19, 24.87) {2019-01};

\node[text=drawColor,anchor=base,inner sep=0pt, outer sep=0pt, scale=  1.07] at (152.80, 24.87) {2019-07};

\node[text=drawColor,anchor=base,inner sep=0pt, outer sep=0pt, scale=  1.07] at (264.23, 24.87) {2020-01};

\node[text=drawColor,anchor=base,inner sep=0pt, outer sep=0pt, scale=  1.07] at (374.45, 24.87) {2020-07};

\node[text=drawColor,anchor=base,inner sep=0pt, outer sep=0pt, scale=  1.07] at (485.88, 24.87) {2021-01};
\end{scope}
\begin{scope}
\path[clip] (  0.00,  0.00) rectangle (578.16,433.62);
\definecolor{drawColor}{gray}{0.30}

\node[text=drawColor,anchor=base east,inner sep=0pt, outer sep=0pt, scale=  1.07] at ( 34.91,382.66) {0};

\node[text=drawColor,anchor=base east,inner sep=0pt, outer sep=0pt, scale=  1.07] at ( 34.91,395.96) {8};

\node[text=drawColor,anchor=base east,inner sep=0pt, outer sep=0pt, scale=  1.07] at ( 34.91,409.25) {16};

\node[text=drawColor,anchor=base east,inner sep=0pt, outer sep=0pt, scale=  1.07] at ( 34.91,422.55) {24};
\end{scope}
\begin{scope}
\path[clip] (  0.00,  0.00) rectangle (578.16,433.62);
\definecolor{drawColor}{gray}{0.30}

\node[text=drawColor,anchor=base east,inner sep=0pt, outer sep=0pt, scale=  1.07] at ( 34.91,333.29) {0};

\node[text=drawColor,anchor=base east,inner sep=0pt, outer sep=0pt, scale=  1.07] at ( 34.91,346.58) {8};

\node[text=drawColor,anchor=base east,inner sep=0pt, outer sep=0pt, scale=  1.07] at ( 34.91,359.88) {16};

\node[text=drawColor,anchor=base east,inner sep=0pt, outer sep=0pt, scale=  1.07] at ( 34.91,373.17) {24};
\end{scope}
\begin{scope}
\path[clip] (  0.00,  0.00) rectangle (578.16,433.62);
\definecolor{drawColor}{gray}{0.30}

\node[text=drawColor,anchor=base east,inner sep=0pt, outer sep=0pt, scale=  1.07] at ( 34.91,283.92) {0};

\node[text=drawColor,anchor=base east,inner sep=0pt, outer sep=0pt, scale=  1.07] at ( 34.91,297.21) {8};

\node[text=drawColor,anchor=base east,inner sep=0pt, outer sep=0pt, scale=  1.07] at ( 34.91,310.51) {16};

\node[text=drawColor,anchor=base east,inner sep=0pt, outer sep=0pt, scale=  1.07] at ( 34.91,323.80) {24};
\end{scope}
\begin{scope}
\path[clip] (  0.00,  0.00) rectangle (578.16,433.62);
\definecolor{drawColor}{gray}{0.30}

\node[text=drawColor,anchor=base east,inner sep=0pt, outer sep=0pt, scale=  1.07] at ( 34.91,234.54) {0};

\node[text=drawColor,anchor=base east,inner sep=0pt, outer sep=0pt, scale=  1.07] at ( 34.91,247.84) {8};

\node[text=drawColor,anchor=base east,inner sep=0pt, outer sep=0pt, scale=  1.07] at ( 34.91,261.13) {16};

\node[text=drawColor,anchor=base east,inner sep=0pt, outer sep=0pt, scale=  1.07] at ( 34.91,274.43) {24};
\end{scope}
\begin{scope}
\path[clip] (  0.00,  0.00) rectangle (578.16,433.62);
\definecolor{drawColor}{gray}{0.30}

\node[text=drawColor,anchor=base east,inner sep=0pt, outer sep=0pt, scale=  1.07] at ( 34.91,185.17) {0};

\node[text=drawColor,anchor=base east,inner sep=0pt, outer sep=0pt, scale=  1.07] at ( 34.91,198.46) {8};

\node[text=drawColor,anchor=base east,inner sep=0pt, outer sep=0pt, scale=  1.07] at ( 34.91,211.76) {16};

\node[text=drawColor,anchor=base east,inner sep=0pt, outer sep=0pt, scale=  1.07] at ( 34.91,225.05) {24};
\end{scope}
\begin{scope}
\path[clip] (  0.00,  0.00) rectangle (578.16,433.62);
\definecolor{drawColor}{gray}{0.30}

\node[text=drawColor,anchor=base east,inner sep=0pt, outer sep=0pt, scale=  1.07] at ( 34.91,135.80) {0};

\node[text=drawColor,anchor=base east,inner sep=0pt, outer sep=0pt, scale=  1.07] at ( 34.91,149.09) {8};

\node[text=drawColor,anchor=base east,inner sep=0pt, outer sep=0pt, scale=  1.07] at ( 34.91,162.39) {16};

\node[text=drawColor,anchor=base east,inner sep=0pt, outer sep=0pt, scale=  1.07] at ( 34.91,175.68) {24};
\end{scope}
\begin{scope}
\path[clip] (  0.00,  0.00) rectangle (578.16,433.62);
\definecolor{drawColor}{gray}{0.30}

\node[text=drawColor,anchor=base east,inner sep=0pt, outer sep=0pt, scale=  1.07] at ( 34.91, 86.42) {0};

\node[text=drawColor,anchor=base east,inner sep=0pt, outer sep=0pt, scale=  1.07] at ( 34.91, 99.72) {8};

\node[text=drawColor,anchor=base east,inner sep=0pt, outer sep=0pt, scale=  1.07] at ( 34.91,113.01) {16};

\node[text=drawColor,anchor=base east,inner sep=0pt, outer sep=0pt, scale=  1.07] at ( 34.91,126.31) {24};
\end{scope}
\begin{scope}
\path[clip] (  0.00,  0.00) rectangle (578.16,433.62);
\definecolor{drawColor}{gray}{0.30}

\node[text=drawColor,anchor=base east,inner sep=0pt, outer sep=0pt, scale=  1.07] at ( 34.91, 37.05) {0};

\node[text=drawColor,anchor=base east,inner sep=0pt, outer sep=0pt, scale=  1.07] at ( 34.91, 50.34) {8};

\node[text=drawColor,anchor=base east,inner sep=0pt, outer sep=0pt, scale=  1.07] at ( 34.91, 63.64) {16};

\node[text=drawColor,anchor=base east,inner sep=0pt, outer sep=0pt, scale=  1.07] at ( 34.91, 76.93) {24};
\end{scope}
\begin{scope}
\path[clip] (  0.00,  0.00) rectangle (578.16,433.62);
\definecolor{drawColor}{RGB}{0,0,0}

\node[text=drawColor,anchor=base,inner sep=0pt, outer sep=0pt, scale=  1.33] at (263.02,  8.61) {date};
\end{scope}
\begin{scope}
\path[clip] (  0.00,  0.00) rectangle (578.16,433.62);
\definecolor{drawColor}{RGB}{0,0,0}

\node[text=drawColor,rotate= 90.00,anchor=base,inner sep=0pt, outer sep=0pt, scale=  1.33] at ( 16.52,233.38) {hour};
\end{scope}
\begin{scope}
\path[clip] (  0.00,  0.00) rectangle (578.16,433.62);
\node[inner sep=0pt,outer sep=0pt,anchor=south west,rotate=  0.00] at (522.79,  59.71) {
	\pgfimage[width= 17.34pt,height=325.21pt,interpolate=true]{wplot_mv_0.5_mix.bewa.sm.fr_ras9}};
\end{scope}
\begin{scope}
\path[clip] (  0.00,  0.00) rectangle (578.16,433.62);
\definecolor{drawColor}{RGB}{0,0,0}

\node[text=drawColor,anchor=base west,inner sep=0pt, outer sep=0pt, scale=  1.07] at (548.14, 55.84) {0.0};

\node[text=drawColor,anchor=base west,inner sep=0pt, outer sep=0pt, scale=  1.07] at (548.14,120.67) {0.1};

\node[text=drawColor,anchor=base west,inner sep=0pt, outer sep=0pt, scale=  1.07] at (548.14,185.49) {0.2};

\node[text=drawColor,anchor=base west,inner sep=0pt, outer sep=0pt, scale=  1.07] at (548.14,250.32) {0.3};

\node[text=drawColor,anchor=base west,inner sep=0pt, outer sep=0pt, scale=  1.07] at (548.14,315.14) {0.4};

\node[text=drawColor,anchor=base west,inner sep=0pt, outer sep=0pt, scale=  1.07] at (548.14,379.97) {0.5};
\end{scope}
\begin{scope}
\path[clip] (  0.00,  0.00) rectangle (578.16,433.62);
\definecolor{drawColor}{RGB}{0,0,0}

\node[text=drawColor,anchor=base west,inner sep=0pt, outer sep=0pt, scale=  1.33] at (522.79,394.48) {weight};
\end{scope}
\begin{scope}
\path[clip] (  0.00,  0.00) rectangle (578.16,433.62);
\definecolor{drawColor}{RGB}{255,255,255}

\path[draw=drawColor,line width= 0.2pt,line join=round] (522.79, 60.25) -- (526.26, 60.25);

\path[draw=drawColor,line width= 0.2pt,line join=round] (522.79,125.07) -- (526.26,125.07);

\path[draw=drawColor,line width= 0.2pt,line join=round] (522.79,189.90) -- (526.26,189.90);

\path[draw=drawColor,line width= 0.2pt,line join=round] (522.79,254.73) -- (526.26,254.73);

\path[draw=drawColor,line width= 0.2pt,line join=round] (522.79,319.55) -- (526.26,319.55);

\path[draw=drawColor,line width= 0.2pt,line join=round] (522.79,384.38) -- (526.26,384.38);

\path[draw=drawColor,line width= 0.2pt,line join=round] (536.67, 60.25) -- (540.14, 60.25);

\path[draw=drawColor,line width= 0.2pt,line join=round] (536.67,125.07) -- (540.14,125.07);

\path[draw=drawColor,line width= 0.2pt,line join=round] (536.67,189.90) -- (540.14,189.90);

\path[draw=drawColor,line width= 0.2pt,line join=round] (536.67,254.73) -- (540.14,254.73);

\path[draw=drawColor,line width= 0.2pt,line join=round] (536.67,319.55) -- (540.14,319.55);

\path[draw=drawColor,line width= 0.2pt,line join=round] (536.67,384.38) -- (540.14,384.38);
\end{scope}
\end{tikzpicture}
}
    }}
  \caption{Temporal evolution of the weights at the median across all 24 hours}\label{weights_temporal_vs_hour}
\end{figure}

Figure~\ref{fig:best_pars} presents the parameters used by the proposed \textit{online.sm.fr} specification. This approach optimizes three parameters: the forgetting rate and the two smoothing penalties. All parameters need some time to stabilize. However, the chosen burn-in period (marked in grey) seems to suffice for the most part. Further, we observe an increasing forgetting as the learning progresses and a quite consistent level of smoothing across both dimensions. Lastly, we observe that the weights are getting more smoothed across probabilities than across hours (see also Figure~\ref{fig:smooth}, which presents the most recent weights of this solution across hours and probabilities).
\begin{figure}[!h]
  \centering{
    \resizebox{\textwidth}{!}{%
      \fbox{% !TEX encoding = UTF-8 Unicode
\begin{tikzpicture}[x=1pt,y=1pt]
	\definecolor{fillColor}{RGB}{255,255,255}
	\path[use as bounding box,fill=fillColor,fill opacity=0.00] (0,0) rectangle (578.16,289.08);
	\begin{scope}
		\path[clip] ( 51.53,205.60) rectangle (552.55,283.58);
		\definecolor{drawColor}{gray}{0.92}

		\path[draw=drawColor,line width= 0.3pt,line join=round] ( 51.53,215.36) --
		(552.55,215.36);

		\path[draw=drawColor,line width= 0.3pt,line join=round] ( 51.53,239.06) --
		(552.55,239.06);

		\path[draw=drawColor,line width= 0.3pt,line join=round] ( 51.53,262.75) --
		(552.55,262.75);

		\path[draw=drawColor,line width= 0.3pt,line join=round] (133.49,205.60) --
		(133.49,283.58);

		\path[draw=drawColor,line width= 0.3pt,line join=round] (246.58,205.60) --
		(246.58,283.58);

		\path[draw=drawColor,line width= 0.3pt,line join=round] (359.98,205.60) --
		(359.98,283.58);

		\path[draw=drawColor,line width= 0.3pt,line join=round] (473.39,205.60) --
		(473.39,283.58);

		\path[draw=drawColor,line width= 0.6pt,line join=round] ( 51.53,227.21) --
		(552.55,227.21);

		\path[draw=drawColor,line width= 0.6pt,line join=round] ( 51.53,250.91) --
		(552.55,250.91);

		\path[draw=drawColor,line width= 0.6pt,line join=round] ( 51.53,274.60) --
		(552.55,274.60);

		\path[draw=drawColor,line width= 0.6pt,line join=round] ( 77.41,205.60) --
		( 77.41,283.58);

		\path[draw=drawColor,line width= 0.6pt,line join=round] (189.57,205.60) --
		(189.57,283.58);

		\path[draw=drawColor,line width= 0.6pt,line join=round] (303.59,205.60) --
		(303.59,283.58);

		\path[draw=drawColor,line width= 0.6pt,line join=round] (416.38,205.60) --
		(416.38,283.58);

		\path[draw=drawColor,line width= 0.6pt,line join=round] (530.40,205.60) --
		(530.40,283.58);
		\definecolor{fillColor}{gray}{0.93}

		\path[fill=fillColor] ( 74.31,205.60) rectangle (111.49,248.99);

		\path[fill=fillColor] ( 74.31,205.60) rectangle (111.49,239.54);

		\path[fill=fillColor] ( 74.31,205.60) rectangle (111.49,211.22);

		\path[fill=fillColor] ( 74.31,205.60) rectangle (111.49,259.30);

		\path[fill=fillColor] ( 74.31,205.60) rectangle (111.49,211.22);

		\path[fill=fillColor] ( 74.31,205.60) rectangle (111.49,211.22);

		\path[fill=fillColor] ( 74.31,205.60) rectangle (111.49,257.78);

		\path[fill=fillColor] ( 74.31,205.60) rectangle (111.49,280.04);

		\path[fill=fillColor] ( 74.31,205.60) rectangle (111.49,280.04);

		\path[fill=fillColor] ( 74.31,205.60) rectangle (111.49,280.04);

		\path[fill=fillColor] ( 74.31,205.60) rectangle (111.49,280.04);

		\path[fill=fillColor] ( 74.31,205.60) rectangle (111.49,243.16);

		\path[fill=fillColor] ( 74.31,205.60) rectangle (111.49,280.04);

		\path[fill=fillColor] ( 74.31,205.60) rectangle (111.49,280.04);

		\path[fill=fillColor] ( 74.31,205.60) rectangle (111.49,243.16);

		\path[fill=fillColor] ( 74.31,205.60) rectangle (111.49,243.16);

		\path[fill=fillColor] ( 74.31,205.60) rectangle (111.49,243.16);

		\path[fill=fillColor] ( 74.31,205.60) rectangle (111.49,223.74);

		\path[fill=fillColor] ( 74.31,205.60) rectangle (111.49,223.74);

		\path[fill=fillColor] ( 74.31,205.60) rectangle (111.49,256.88);

		\path[fill=fillColor] ( 74.31,205.60) rectangle (111.49,256.88);

		\path[fill=fillColor] ( 74.31,205.60) rectangle (111.49,256.88);

		\path[fill=fillColor] ( 74.31,205.60) rectangle (111.49,256.88);

		\path[fill=fillColor] ( 74.31,205.60) rectangle (111.49,256.88);

		\path[fill=fillColor] ( 74.31,205.60) rectangle (111.49,256.88);

		\path[fill=fillColor] ( 74.31,205.60) rectangle (111.49,256.88);

		\path[fill=fillColor] ( 74.31,205.60) rectangle (111.49,256.88);

		\path[fill=fillColor] ( 74.31,205.60) rectangle (111.49,256.88);

		\path[fill=fillColor] ( 74.31,205.60) rectangle (111.49,256.88);

		\path[fill=fillColor] ( 74.31,205.60) rectangle (111.49,252.05);

		\path[fill=fillColor] ( 74.31,205.60) rectangle (111.49,252.05);

		\path[fill=fillColor] ( 74.31,205.60) rectangle (111.49,252.05);

		\path[fill=fillColor] ( 74.31,205.60) rectangle (111.49,280.04);

		\path[fill=fillColor] ( 74.31,205.60) rectangle (111.49,252.05);

		\path[fill=fillColor] ( 74.31,205.60) rectangle (111.49,252.05);

		\path[fill=fillColor] ( 74.31,205.60) rectangle (111.49,252.05);

		\path[fill=fillColor] ( 74.31,205.60) rectangle (111.49,261.71);

		\path[fill=fillColor] ( 74.31,205.60) rectangle (111.49,261.71);

		\path[fill=fillColor] ( 74.31,205.60) rectangle (111.49,252.05);

		\path[fill=fillColor] ( 74.31,205.60) rectangle (111.49,261.71);

		\path[fill=fillColor] ( 74.31,205.60) rectangle (111.49,261.71);

		\path[fill=fillColor] ( 74.31,205.60) rectangle (111.49,261.71);

		\path[fill=fillColor] ( 74.31,205.60) rectangle (111.49,261.71);

		\path[fill=fillColor] ( 74.31,205.60) rectangle (111.49,261.71);

		\path[fill=fillColor] ( 74.31,205.60) rectangle (111.49,252.05);

		\path[fill=fillColor] ( 74.31,205.60) rectangle (111.49,215.84);

		\path[fill=fillColor] ( 74.31,205.60) rectangle (111.49,223.74);

		\path[fill=fillColor] ( 74.31,205.60) rectangle (111.49,223.74);

		\path[fill=fillColor] ( 74.31,205.60) rectangle (111.49,223.74);

		\path[fill=fillColor] ( 74.31,205.60) rectangle (111.49,223.74);

		\path[fill=fillColor] ( 74.31,205.60) rectangle (111.49,215.84);

		\path[fill=fillColor] ( 74.31,205.60) rectangle (111.49,223.74);

		\path[fill=fillColor] ( 74.31,205.60) rectangle (111.49,215.84);

		\path[fill=fillColor] ( 74.31,205.60) rectangle (111.49,215.84);

		\path[fill=fillColor] ( 74.31,205.60) rectangle (111.49,238.54);

		\path[fill=fillColor] ( 74.31,205.60) rectangle (111.49,238.54);

		\path[fill=fillColor] ( 74.31,205.60) rectangle (111.49,252.05);

		\path[fill=fillColor] ( 74.31,205.60) rectangle (111.49,252.05);

		\path[fill=fillColor] ( 74.31,205.60) rectangle (111.49,238.54);

		\path[fill=fillColor] ( 74.31,205.60) rectangle (111.49,238.54);

		\path[fill=fillColor] ( 74.31,205.60) rectangle (111.49,252.05);

		\path[fill=fillColor] ( 74.31,205.60) rectangle (111.49,252.05);

		\path[fill=fillColor] ( 74.31,205.60) rectangle (111.49,238.54);

		\path[fill=fillColor] ( 74.31,205.60) rectangle (111.49,238.54);

		\path[fill=fillColor] ( 74.31,205.60) rectangle (111.49,238.54);

		\path[fill=fillColor] ( 74.31,205.60) rectangle (111.49,238.54);

		\path[fill=fillColor] ( 74.31,205.60) rectangle (111.49,238.54);

		\path[fill=fillColor] ( 74.31,205.60) rectangle (111.49,247.19);

		\path[fill=fillColor] ( 74.31,205.60) rectangle (111.49,238.54);

		\path[fill=fillColor] ( 74.31,205.60) rectangle (111.49,238.54);

		\path[fill=fillColor] ( 74.31,205.60) rectangle (111.49,238.54);

		\path[fill=fillColor] ( 74.31,205.60) rectangle (111.49,238.54);

		\path[fill=fillColor] ( 74.31,205.60) rectangle (111.49,227.02);

		\path[fill=fillColor] ( 74.31,205.60) rectangle (111.49,236.26);

		\path[fill=fillColor] ( 74.31,205.60) rectangle (111.49,211.22);

		\path[fill=fillColor] ( 74.31,205.60) rectangle (111.49,211.22);

		\path[fill=fillColor] ( 74.31,205.60) rectangle (111.49,236.26);

		\path[fill=fillColor] ( 74.31,205.60) rectangle (111.49,236.26);

		\path[fill=fillColor] ( 74.31,205.60) rectangle (111.49,236.26);

		\path[fill=fillColor] ( 74.31,205.60) rectangle (111.49,247.19);

		\path[fill=fillColor] ( 74.31,205.60) rectangle (111.49,244.77);

		\path[fill=fillColor] ( 74.31,205.60) rectangle (111.49,244.77);

		\path[fill=fillColor] ( 74.31,205.60) rectangle (111.49,215.84);

		\path[fill=fillColor] ( 74.31,205.60) rectangle (111.49,215.84);

		\path[fill=fillColor] ( 74.31,205.60) rectangle (111.49,215.84);

		\path[fill=fillColor] ( 74.31,205.60) rectangle (111.49,215.84);

		\path[fill=fillColor] ( 74.31,205.60) rectangle (111.49,215.84);

		\path[fill=fillColor] ( 74.31,205.60) rectangle (111.49,215.84);

		\path[fill=fillColor] ( 74.31,205.60) rectangle (111.49,215.84);

		\path[fill=fillColor] ( 74.31,205.60) rectangle (111.49,215.84);

		\path[fill=fillColor] ( 74.31,205.60) rectangle (111.49,215.84);

		\path[fill=fillColor] ( 74.31,205.60) rectangle (111.49,211.22);

		\path[fill=fillColor] ( 74.31,205.60) rectangle (111.49,211.22);

		\path[fill=fillColor] ( 74.31,205.60) rectangle (111.49,211.22);

		\path[fill=fillColor] ( 74.31,205.60) rectangle (111.49,211.22);

		\path[fill=fillColor] ( 74.31,205.60) rectangle (111.49,211.22);

		\path[fill=fillColor] ( 74.31,205.60) rectangle (111.49,211.22);

		\path[fill=fillColor] ( 74.31,205.60) rectangle (111.49,211.22);

		\path[fill=fillColor] ( 74.31,205.60) rectangle (111.49,211.22);

		\path[fill=fillColor] ( 74.31,205.60) rectangle (111.49,211.22);

		\path[fill=fillColor] ( 74.31,205.60) rectangle (111.49,221.66);

		\path[fill=fillColor] ( 74.31,205.60) rectangle (111.49,211.22);

		\path[fill=fillColor] ( 74.31,205.60) rectangle (111.49,221.66);

		\path[fill=fillColor] ( 74.31,205.60) rectangle (111.49,221.66);

		\path[fill=fillColor] ( 74.31,205.60) rectangle (111.49,238.54);

		\path[fill=fillColor] ( 74.31,205.60) rectangle (111.49,238.54);

		\path[fill=fillColor] ( 74.31,205.60) rectangle (111.49,238.54);

		\path[fill=fillColor] ( 74.31,205.60) rectangle (111.49,238.54);

		\path[fill=fillColor] ( 74.31,205.60) rectangle (111.49,247.19);

		\path[fill=fillColor] ( 74.31,205.60) rectangle (111.49,247.19);

		\path[fill=fillColor] ( 74.31,205.60) rectangle (111.49,247.19);

		\path[fill=fillColor] ( 74.31,205.60) rectangle (111.49,247.19);

		\path[fill=fillColor] ( 74.31,205.60) rectangle (111.49,247.19);

		\path[fill=fillColor] ( 74.31,205.60) rectangle (111.49,247.19);

		\path[fill=fillColor] ( 74.31,205.60) rectangle (111.49,247.19);

		\path[fill=fillColor] ( 74.31,205.60) rectangle (111.49,247.19);

		\path[fill=fillColor] ( 74.31,205.60) rectangle (111.49,228.36);

		\path[fill=fillColor] ( 74.31,205.60) rectangle (111.49,211.22);

		\path[fill=fillColor] ( 74.31,205.60) rectangle (111.49,211.22);

		\path[fill=fillColor] ( 74.31,205.60) rectangle (111.49,211.22);

		\path[fill=fillColor] ( 74.31,205.60) rectangle (111.49,211.22);

		\path[fill=fillColor] ( 74.31,205.60) rectangle (111.49,211.22);

		\path[fill=fillColor] ( 74.31,205.60) rectangle (111.49,211.22);

		\path[fill=fillColor] ( 74.31,205.60) rectangle (111.49,211.22);

		\path[fill=fillColor] ( 74.31,205.60) rectangle (111.49,211.22);

		\path[fill=fillColor] ( 74.31,205.60) rectangle (111.49,211.22);

		\path[fill=fillColor] ( 74.31,205.60) rectangle (111.49,211.22);

		\path[fill=fillColor] ( 74.31,205.60) rectangle (111.49,211.22);

		\path[fill=fillColor] ( 74.31,205.60) rectangle (111.49,211.22);

		\path[fill=fillColor] ( 74.31,205.60) rectangle (111.49,211.22);

		\path[fill=fillColor] ( 74.31,205.60) rectangle (111.49,211.22);

		\path[fill=fillColor] ( 74.31,205.60) rectangle (111.49,211.22);

		\path[fill=fillColor] ( 74.31,205.60) rectangle (111.49,211.22);

		\path[fill=fillColor] ( 74.31,205.60) rectangle (111.49,211.22);

		\path[fill=fillColor] ( 74.31,205.60) rectangle (111.49,211.22);

		\path[fill=fillColor] ( 74.31,205.60) rectangle (111.49,211.22);

		\path[fill=fillColor] ( 74.31,205.60) rectangle (111.49,211.22);

		\path[fill=fillColor] ( 74.31,205.60) rectangle (111.49,211.22);

		\path[fill=fillColor] ( 74.31,205.60) rectangle (111.49,211.22);

		\path[fill=fillColor] ( 74.31,205.60) rectangle (111.49,211.22);

		\path[fill=fillColor] ( 74.31,205.60) rectangle (111.49,211.22);

		\path[fill=fillColor] ( 74.31,205.60) rectangle (111.49,211.22);

		\path[fill=fillColor] ( 74.31,205.60) rectangle (111.49,211.22);

		\path[fill=fillColor] ( 74.31,205.60) rectangle (111.49,211.22);

		\path[fill=fillColor] ( 74.31,205.60) rectangle (111.49,211.22);

		\path[fill=fillColor] ( 74.31,205.60) rectangle (111.49,211.22);

		\path[fill=fillColor] ( 74.31,205.60) rectangle (111.49,211.22);

		\path[fill=fillColor] ( 74.31,205.60) rectangle (111.49,211.22);

		\path[fill=fillColor] ( 74.31,205.60) rectangle (111.49,211.22);

		\path[fill=fillColor] ( 74.31,205.60) rectangle (111.49,211.22);

		\path[fill=fillColor] ( 74.31,205.60) rectangle (111.49,211.22);

		\path[fill=fillColor] ( 74.31,205.60) rectangle (111.49,211.22);

		\path[fill=fillColor] ( 74.31,205.60) rectangle (111.49,211.22);

		\path[fill=fillColor] ( 74.31,205.60) rectangle (111.49,211.22);

		\path[fill=fillColor] ( 74.31,205.60) rectangle (111.49,211.22);

		\path[fill=fillColor] ( 74.31,205.60) rectangle (111.49,211.22);

		\path[fill=fillColor] ( 74.31,205.60) rectangle (111.49,211.22);

		\path[fill=fillColor] ( 74.31,205.60) rectangle (111.49,211.22);

		\path[fill=fillColor] ( 74.31,205.60) rectangle (111.49,211.22);

		\path[fill=fillColor] ( 74.31,205.60) rectangle (111.49,211.22);

		\path[fill=fillColor] ( 74.31,205.60) rectangle (111.49,211.22);

		\path[fill=fillColor] ( 74.31,205.60) rectangle (111.49,211.22);

		\path[fill=fillColor] ( 74.31,205.60) rectangle (111.49,211.22);

		\path[fill=fillColor] ( 74.31,205.60) rectangle (111.49,211.22);

		\path[fill=fillColor] ( 74.31,205.60) rectangle (111.49,211.22);

		\path[fill=fillColor] ( 74.31,205.60) rectangle (111.49,211.22);

		\path[fill=fillColor] ( 74.31,205.60) rectangle (111.49,211.22);

		\path[fill=fillColor] ( 74.31,205.60) rectangle (111.49,211.22);

		\path[fill=fillColor] ( 74.31,205.60) rectangle (111.49,211.22);

		\path[fill=fillColor] ( 74.31,205.60) rectangle (111.49,211.22);

		\path[fill=fillColor] ( 74.31,205.60) rectangle (111.49,211.22);

		\path[fill=fillColor] ( 74.31,205.60) rectangle (111.49,211.22);

		\path[fill=fillColor] ( 74.31,205.60) rectangle (111.49,211.22);

		\path[fill=fillColor] ( 74.31,205.60) rectangle (111.49,215.84);

		\path[fill=fillColor] ( 74.31,205.60) rectangle (111.49,215.84);

		\path[fill=fillColor] ( 74.31,205.60) rectangle (111.49,215.84);

		\path[fill=fillColor] ( 74.31,205.60) rectangle (111.49,215.84);

		\path[fill=fillColor] ( 74.31,205.60) rectangle (111.49,215.84);

		\path[fill=fillColor] ( 74.31,205.60) rectangle (111.49,215.84);

		\path[fill=fillColor] ( 74.31,205.60) rectangle (111.49,215.84);

		\path[fill=fillColor] ( 74.31,205.60) rectangle (111.49,215.84);

		\path[fill=fillColor] ( 74.31,205.60) rectangle (111.49,215.84);

		\path[fill=fillColor] ( 74.31,205.60) rectangle (111.49,215.84);

		\path[fill=fillColor] ( 74.31,205.60) rectangle (111.49,215.84);

		\path[fill=fillColor] ( 74.31,205.60) rectangle (111.49,215.84);

		\path[fill=fillColor] ( 74.31,205.60) rectangle (111.49,215.84);

		\path[fill=fillColor] ( 74.31,205.60) rectangle (111.49,215.84);

		\path[fill=fillColor] ( 74.31,205.60) rectangle (111.49,215.84);

		\path[fill=fillColor] ( 74.31,205.60) rectangle (111.49,215.84);

		\path[fill=fillColor] ( 74.31,205.60) rectangle (111.49,223.74);

		\path[fill=fillColor] ( 74.31,205.60) rectangle (111.49,227.02);

		\path[fill=fillColor] ( 74.31,205.60) rectangle (111.49,215.84);

		\path[fill=fillColor] ( 74.31,205.60) rectangle (111.49,215.84);

		\path[fill=fillColor] ( 74.31,205.60) rectangle (111.49,215.84);

		\path[fill=fillColor] ( 74.31,205.60) rectangle (111.49,215.84);

		\path[fill=fillColor] ( 74.31,205.60) rectangle (111.49,215.84);

		\path[fill=fillColor] ( 74.31,205.60) rectangle (111.49,215.84);

		\path[fill=fillColor] ( 74.31,205.60) rectangle (111.49,215.84);

		\path[fill=fillColor] ( 74.31,205.60) rectangle (111.49,215.84);

		\path[fill=fillColor] ( 74.31,205.60) rectangle (111.49,215.84);

		\path[fill=fillColor] ( 74.31,205.60) rectangle (111.49,215.84);

		\path[fill=fillColor] ( 74.31,205.60) rectangle (111.49,215.84);

		\path[fill=fillColor] ( 74.31,205.60) rectangle (111.49,215.84);

		\path[fill=fillColor] ( 74.31,205.60) rectangle (111.49,215.84);

		\path[fill=fillColor] ( 74.31,205.60) rectangle (111.49,215.84);

		\path[fill=fillColor] ( 74.31,205.60) rectangle (111.49,215.84);

		\path[fill=fillColor] ( 74.31,205.60) rectangle (111.49,215.84);

		\path[fill=fillColor] ( 74.31,205.60) rectangle (111.49,215.84);

		\path[fill=fillColor] ( 74.31,205.60) rectangle (111.49,215.84);

		\path[fill=fillColor] ( 74.31,205.60) rectangle (111.49,211.22);

		\path[fill=fillColor] ( 74.31,205.60) rectangle (111.49,227.02);

		\path[fill=fillColor] ( 74.31,205.60) rectangle (111.49,227.02);

		\path[fill=fillColor] ( 74.31,205.60) rectangle (111.49,211.22);

		\path[fill=fillColor] ( 74.31,205.60) rectangle (111.49,211.22);

		\path[fill=fillColor] ( 74.31,205.60) rectangle (111.49,211.22);

		\path[fill=fillColor] ( 74.31,205.60) rectangle (111.49,211.22);

		\path[fill=fillColor] ( 74.31,205.60) rectangle (111.49,211.22);

		\path[fill=fillColor] ( 74.31,205.60) rectangle (111.49,211.22);

		\path[fill=fillColor] ( 74.31,205.60) rectangle (111.49,211.22);

		\path[fill=fillColor] ( 74.31,205.60) rectangle (111.49,211.22);

		\path[fill=fillColor] ( 74.31,205.60) rectangle (111.49,211.22);

		\path[fill=fillColor] ( 74.31,205.60) rectangle (111.49,211.22);

		\path[fill=fillColor] ( 74.31,205.60) rectangle (111.49,211.22);

		\path[fill=fillColor] ( 74.31,205.60) rectangle (111.49,211.22);

		\path[fill=fillColor] ( 74.31,205.60) rectangle (111.49,211.22);

		\path[fill=fillColor] ( 74.31,205.60) rectangle (111.49,211.22);

		\path[fill=fillColor] ( 74.31,205.60) rectangle (111.49,211.22);

		\path[fill=fillColor] ( 74.31,205.60) rectangle (111.49,211.22);

		\path[fill=fillColor] ( 74.31,205.60) rectangle (111.49,211.22);

		\path[fill=fillColor] ( 74.31,205.60) rectangle (111.49,227.02);

		\path[fill=fillColor] ( 74.31,205.60) rectangle (111.49,236.26);

		\path[fill=fillColor] ( 74.31,205.60) rectangle (111.49,247.19);

		\path[fill=fillColor] ( 74.31,205.60) rectangle (111.49,247.19);

		\path[fill=fillColor] ( 74.31,205.60) rectangle (111.49,247.19);

		\path[fill=fillColor] ( 74.31,205.60) rectangle (111.49,247.19);

		\path[fill=fillColor] ( 74.31,205.60) rectangle (111.49,247.19);

		\path[fill=fillColor] ( 74.31,205.60) rectangle (111.49,247.19);

		\path[fill=fillColor] ( 74.31,205.60) rectangle (111.49,247.19);

		\path[fill=fillColor] ( 74.31,205.60) rectangle (111.49,247.19);

		\path[fill=fillColor] ( 74.31,205.60) rectangle (111.49,247.19);

		\path[fill=fillColor] ( 74.31,205.60) rectangle (111.49,247.19);

		\path[fill=fillColor] ( 74.31,205.60) rectangle (111.49,247.19);

		\path[fill=fillColor] ( 74.31,205.60) rectangle (111.49,247.19);

		\path[fill=fillColor] ( 74.31,205.60) rectangle (111.49,247.19);

		\path[fill=fillColor] ( 74.31,205.60) rectangle (111.49,247.19);

		\path[fill=fillColor] ( 74.31,205.60) rectangle (111.49,247.19);

		\path[fill=fillColor] ( 74.31,205.60) rectangle (111.49,247.19);

		\path[fill=fillColor] ( 74.31,205.60) rectangle (111.49,247.19);

		\path[fill=fillColor] ( 74.31,205.60) rectangle (111.49,247.19);

		\path[fill=fillColor] ( 74.31,205.60) rectangle (111.49,247.19);

		\path[fill=fillColor] ( 74.31,205.60) rectangle (111.49,247.19);

		\path[fill=fillColor] ( 74.31,205.60) rectangle (111.49,247.19);

		\path[fill=fillColor] ( 74.31,205.60) rectangle (111.49,247.19);

		\path[fill=fillColor] ( 74.31,205.60) rectangle (111.49,252.52);

		\path[fill=fillColor] ( 74.31,205.60) rectangle (111.49,247.19);

		\path[fill=fillColor] ( 74.31,205.60) rectangle (111.49,247.19);

		\path[fill=fillColor] ( 74.31,205.60) rectangle (111.49,247.19);

		\path[fill=fillColor] ( 74.31,205.60) rectangle (111.49,247.19);

		\path[fill=fillColor] ( 74.31,205.60) rectangle (111.49,247.19);

		\path[fill=fillColor] ( 74.31,205.60) rectangle (111.49,247.19);

		\path[fill=fillColor] ( 74.31,205.60) rectangle (111.49,247.19);

		\path[fill=fillColor] ( 74.31,205.60) rectangle (111.49,247.19);

		\path[fill=fillColor] ( 74.31,205.60) rectangle (111.49,247.19);

		\path[fill=fillColor] ( 74.31,205.60) rectangle (111.49,247.19);

		\path[fill=fillColor] ( 74.31,205.60) rectangle (111.49,247.19);

		\path[fill=fillColor] ( 74.31,205.60) rectangle (111.49,247.19);

		\path[fill=fillColor] ( 74.31,205.60) rectangle (111.49,247.19);

		\path[fill=fillColor] ( 74.31,205.60) rectangle (111.49,247.19);

		\path[fill=fillColor] ( 74.31,205.60) rectangle (111.49,247.19);

		\path[fill=fillColor] ( 74.31,205.60) rectangle (111.49,247.19);

		\path[fill=fillColor] ( 74.31,205.60) rectangle (111.49,247.19);

		\path[fill=fillColor] ( 74.31,205.60) rectangle (111.49,247.19);

		\path[fill=fillColor] ( 74.31,205.60) rectangle (111.49,247.19);

		\path[fill=fillColor] ( 74.31,205.60) rectangle (111.49,247.19);

		\path[fill=fillColor] ( 74.31,205.60) rectangle (111.49,247.19);

		\path[fill=fillColor] ( 74.31,205.60) rectangle (111.49,236.26);

		\path[fill=fillColor] ( 74.31,205.60) rectangle (111.49,239.05);

		\path[fill=fillColor] ( 74.31,205.60) rectangle (111.49,247.19);

		\path[fill=fillColor] ( 74.31,205.60) rectangle (111.49,247.19);

		\path[fill=fillColor] ( 74.31,205.60) rectangle (111.49,247.19);

		\path[fill=fillColor] ( 74.31,205.60) rectangle (111.49,247.19);

		\path[fill=fillColor] ( 74.31,205.60) rectangle (111.49,247.19);

		\path[fill=fillColor] ( 74.31,205.60) rectangle (111.49,247.19);

		\path[fill=fillColor] ( 74.31,205.60) rectangle (111.49,247.19);

		\path[fill=fillColor] ( 74.31,205.60) rectangle (111.49,247.19);

		\path[fill=fillColor] ( 74.31,205.60) rectangle (111.49,247.19);

		\path[fill=fillColor] ( 74.31,205.60) rectangle (111.49,247.19);

		\path[fill=fillColor] ( 74.31,205.60) rectangle (111.49,247.19);

		\path[fill=fillColor] ( 74.31,205.60) rectangle (111.49,247.19);

		\path[fill=fillColor] ( 74.31,205.60) rectangle (111.49,247.19);

		\path[fill=fillColor] ( 74.31,205.60) rectangle (111.49,247.19);

		\path[fill=fillColor] ( 74.31,205.60) rectangle (111.49,247.19);

		\path[fill=fillColor] ( 74.31,205.60) rectangle (111.49,247.19);

		\path[fill=fillColor] ( 74.31,205.60) rectangle (111.49,247.19);

		\path[fill=fillColor] ( 74.31,205.60) rectangle (111.49,247.19);

		\path[fill=fillColor] ( 74.31,205.60) rectangle (111.49,247.19);

		\path[fill=fillColor] ( 74.31,205.60) rectangle (111.49,247.19);

		\path[fill=fillColor] ( 74.31,205.60) rectangle (111.49,247.19);

		\path[fill=fillColor] ( 74.31,205.60) rectangle (111.49,247.19);

		\path[fill=fillColor] ( 74.31,205.60) rectangle (111.49,247.19);

		\path[fill=fillColor] ( 74.31,205.60) rectangle (111.49,247.19);

		\path[fill=fillColor] ( 74.31,205.60) rectangle (111.49,252.52);

		\path[fill=fillColor] ( 74.31,205.60) rectangle (111.49,252.52);

		\path[fill=fillColor] ( 74.31,205.60) rectangle (111.49,247.19);

		\path[fill=fillColor] ( 74.31,205.60) rectangle (111.49,247.19);

		\path[fill=fillColor] ( 74.31,205.60) rectangle (111.49,247.19);

		\path[fill=fillColor] ( 74.31,205.60) rectangle (111.49,247.19);

		\path[fill=fillColor] ( 74.31,205.60) rectangle (111.49,247.19);

		\path[fill=fillColor] ( 74.31,205.60) rectangle (111.49,247.19);

		\path[fill=fillColor] ( 74.31,205.60) rectangle (111.49,247.19);

		\path[fill=fillColor] ( 74.31,205.60) rectangle (111.49,247.19);

		\path[fill=fillColor] ( 74.31,205.60) rectangle (111.49,252.52);

		\path[fill=fillColor] ( 74.31,205.60) rectangle (111.49,252.52);

		\path[fill=fillColor] ( 74.31,205.60) rectangle (111.49,252.52);

		\path[fill=fillColor] ( 74.31,205.60) rectangle (111.49,252.52);

		\path[fill=fillColor] ( 74.31,205.60) rectangle (111.49,252.52);

		\path[fill=fillColor] ( 74.31,205.60) rectangle (111.49,252.52);

		\path[fill=fillColor] ( 74.31,205.60) rectangle (111.49,252.52);

		\path[fill=fillColor] ( 74.31,205.60) rectangle (111.49,252.52);

		\path[fill=fillColor] ( 74.31,205.60) rectangle (111.49,252.52);

		\path[fill=fillColor] ( 74.31,205.60) rectangle (111.49,252.52);

		\path[fill=fillColor] ( 74.31,205.60) rectangle (111.49,252.52);

		\path[fill=fillColor] ( 74.31,205.60) rectangle (111.49,252.52);

		\path[fill=fillColor] ( 74.31,205.60) rectangle (111.49,252.52);

		\path[fill=fillColor] ( 74.31,205.60) rectangle (111.49,252.52);

		\path[fill=fillColor] ( 74.31,205.60) rectangle (111.49,252.52);

		\path[fill=fillColor] ( 74.31,205.60) rectangle (111.49,252.52);

		\path[fill=fillColor] ( 74.31,205.60) rectangle (111.49,252.52);

		\path[fill=fillColor] ( 74.31,205.60) rectangle (111.49,252.52);

		\path[fill=fillColor] ( 74.31,205.60) rectangle (111.49,252.52);

		\path[fill=fillColor] ( 74.31,205.60) rectangle (111.49,252.52);

		\path[fill=fillColor] ( 74.31,205.60) rectangle (111.49,252.52);

		\path[fill=fillColor] ( 74.31,205.60) rectangle (111.49,252.52);

		\path[fill=fillColor] ( 74.31,205.60) rectangle (111.49,252.52);

		\path[fill=fillColor] ( 74.31,205.60) rectangle (111.49,252.52);

		\path[fill=fillColor] ( 74.31,205.60) rectangle (111.49,252.52);

		\path[fill=fillColor] ( 74.31,205.60) rectangle (111.49,252.52);

		\path[fill=fillColor] ( 74.31,205.60) rectangle (111.49,252.52);

		\path[fill=fillColor] ( 74.31,205.60) rectangle (111.49,252.52);

		\path[fill=fillColor] ( 74.31,205.60) rectangle (111.49,252.52);

		\path[fill=fillColor] ( 74.31,205.60) rectangle (111.49,252.52);

		\path[fill=fillColor] ( 74.31,205.60) rectangle (111.49,252.52);

		\path[fill=fillColor] ( 74.31,205.60) rectangle (111.49,252.52);

		\path[fill=fillColor] ( 74.31,205.60) rectangle (111.49,252.52);

		\path[fill=fillColor] ( 74.31,205.60) rectangle (111.49,252.52);

		\path[fill=fillColor] ( 74.31,205.60) rectangle (111.49,252.52);

		\path[fill=fillColor] ( 74.31,205.60) rectangle (111.49,252.52);

		\path[fill=fillColor] ( 74.31,205.60) rectangle (111.49,252.52);

		\path[fill=fillColor] ( 74.31,205.60) rectangle (111.49,252.52);

		\path[fill=fillColor] ( 74.31,205.60) rectangle (111.49,252.52);

		\path[fill=fillColor] ( 74.31,205.60) rectangle (111.49,252.52);

		\path[fill=fillColor] ( 74.31,205.60) rectangle (111.49,252.52);

		\path[fill=fillColor] ( 74.31,205.60) rectangle (111.49,252.52);

		\path[fill=fillColor] ( 74.31,205.60) rectangle (111.49,252.52);

		\path[fill=fillColor] ( 74.31,205.60) rectangle (111.49,252.52);

		\path[fill=fillColor] ( 74.31,205.60) rectangle (111.49,252.52);

		\path[fill=fillColor] ( 74.31,205.60) rectangle (111.49,252.52);

		\path[fill=fillColor] ( 74.31,205.60) rectangle (111.49,252.52);

		\path[fill=fillColor] ( 74.31,205.60) rectangle (111.49,252.52);

		\path[fill=fillColor] ( 74.31,205.60) rectangle (111.49,252.52);

		\path[fill=fillColor] ( 74.31,205.60) rectangle (111.49,252.52);

		\path[fill=fillColor] ( 74.31,205.60) rectangle (111.49,252.52);

		\path[fill=fillColor] ( 74.31,205.60) rectangle (111.49,252.52);

		\path[fill=fillColor] ( 74.31,205.60) rectangle (111.49,252.52);

		\path[fill=fillColor] ( 74.31,205.60) rectangle (111.49,252.52);

		\path[fill=fillColor] ( 74.31,205.60) rectangle (111.49,252.52);

		\path[fill=fillColor] ( 74.31,205.60) rectangle (111.49,252.52);

		\path[fill=fillColor] ( 74.31,205.60) rectangle (111.49,252.52);

		\path[fill=fillColor] ( 74.31,205.60) rectangle (111.49,252.52);

		\path[fill=fillColor] ( 74.31,205.60) rectangle (111.49,252.52);

		\path[fill=fillColor] ( 74.31,205.60) rectangle (111.49,252.52);

		\path[fill=fillColor] ( 74.31,205.60) rectangle (111.49,252.52);

		\path[fill=fillColor] ( 74.31,205.60) rectangle (111.49,252.52);

		\path[fill=fillColor] ( 74.31,205.60) rectangle (111.49,252.52);

		\path[fill=fillColor] ( 74.31,205.60) rectangle (111.49,252.52);

		\path[fill=fillColor] ( 74.31,205.60) rectangle (111.49,252.52);

		\path[fill=fillColor] ( 74.31,205.60) rectangle (111.49,252.52);

		\path[fill=fillColor] ( 74.31,205.60) rectangle (111.49,252.52);

		\path[fill=fillColor] ( 74.31,205.60) rectangle (111.49,252.52);

		\path[fill=fillColor] ( 74.31,205.60) rectangle (111.49,252.52);

		\path[fill=fillColor] ( 74.31,205.60) rectangle (111.49,252.52);

		\path[fill=fillColor] ( 74.31,205.60) rectangle (111.49,252.52);

		\path[fill=fillColor] ( 74.31,205.60) rectangle (111.49,252.52);

		\path[fill=fillColor] ( 74.31,205.60) rectangle (111.49,252.52);

		\path[fill=fillColor] ( 74.31,205.60) rectangle (111.49,252.52);

		\path[fill=fillColor] ( 74.31,205.60) rectangle (111.49,252.52);

		\path[fill=fillColor] ( 74.31,205.60) rectangle (111.49,252.52);

		\path[fill=fillColor] ( 74.31,205.60) rectangle (111.49,252.52);

		\path[fill=fillColor] ( 74.31,205.60) rectangle (111.49,252.52);

		\path[fill=fillColor] ( 74.31,205.60) rectangle (111.49,252.52);

		\path[fill=fillColor] ( 74.31,205.60) rectangle (111.49,252.52);

		\path[fill=fillColor] ( 74.31,205.60) rectangle (111.49,252.52);

		\path[fill=fillColor] ( 74.31,205.60) rectangle (111.49,252.52);

		\path[fill=fillColor] ( 74.31,205.60) rectangle (111.49,252.52);

		\path[fill=fillColor] ( 74.31,205.60) rectangle (111.49,252.52);

		\path[fill=fillColor] ( 74.31,205.60) rectangle (111.49,252.52);

		\path[fill=fillColor] ( 74.31,205.60) rectangle (111.49,252.52);

		\path[fill=fillColor] ( 74.31,205.60) rectangle (111.49,252.52);

		\path[fill=fillColor] ( 74.31,205.60) rectangle (111.49,252.52);

		\path[fill=fillColor] ( 74.31,205.60) rectangle (111.49,252.52);

		\path[fill=fillColor] ( 74.31,205.60) rectangle (111.49,252.52);

		\path[fill=fillColor] ( 74.31,205.60) rectangle (111.49,252.52);

		\path[fill=fillColor] ( 74.31,205.60) rectangle (111.49,252.52);

		\path[fill=fillColor] ( 74.31,205.60) rectangle (111.49,252.52);

		\path[fill=fillColor] ( 74.31,205.60) rectangle (111.49,252.52);

		\path[fill=fillColor] ( 74.31,205.60) rectangle (111.49,252.52);

		\path[fill=fillColor] ( 74.31,205.60) rectangle (111.49,252.52);

		\path[fill=fillColor] ( 74.31,205.60) rectangle (111.49,252.52);

		\path[fill=fillColor] ( 74.31,205.60) rectangle (111.49,252.52);

		\path[fill=fillColor] ( 74.31,205.60) rectangle (111.49,252.52);

		\path[fill=fillColor] ( 74.31,205.60) rectangle (111.49,252.52);

		\path[fill=fillColor] ( 74.31,205.60) rectangle (111.49,252.52);

		\path[fill=fillColor] ( 74.31,205.60) rectangle (111.49,252.52);

		\path[fill=fillColor] ( 74.31,205.60) rectangle (111.49,252.52);

		\path[fill=fillColor] ( 74.31,205.60) rectangle (111.49,252.52);

		\path[fill=fillColor] ( 74.31,205.60) rectangle (111.49,252.52);

		\path[fill=fillColor] ( 74.31,205.60) rectangle (111.49,252.52);

		\path[fill=fillColor] ( 74.31,205.60) rectangle (111.49,252.52);

		\path[fill=fillColor] ( 74.31,205.60) rectangle (111.49,252.52);

		\path[fill=fillColor] ( 74.31,205.60) rectangle (111.49,252.52);

		\path[fill=fillColor] ( 74.31,205.60) rectangle (111.49,252.52);

		\path[fill=fillColor] ( 74.31,205.60) rectangle (111.49,252.52);

		\path[fill=fillColor] ( 74.31,205.60) rectangle (111.49,252.52);

		\path[fill=fillColor] ( 74.31,205.60) rectangle (111.49,252.52);

		\path[fill=fillColor] ( 74.31,205.60) rectangle (111.49,252.52);

		\path[fill=fillColor] ( 74.31,205.60) rectangle (111.49,252.52);

		\path[fill=fillColor] ( 74.31,205.60) rectangle (111.49,252.52);

		\path[fill=fillColor] ( 74.31,205.60) rectangle (111.49,252.52);

		\path[fill=fillColor] ( 74.31,205.60) rectangle (111.49,252.52);

		\path[fill=fillColor] ( 74.31,205.60) rectangle (111.49,252.52);

		\path[fill=fillColor] ( 74.31,205.60) rectangle (111.49,252.52);

		\path[fill=fillColor] ( 74.31,205.60) rectangle (111.49,252.52);

		\path[fill=fillColor] ( 74.31,205.60) rectangle (111.49,252.52);

		\path[fill=fillColor] ( 74.31,205.60) rectangle (111.49,252.52);

		\path[fill=fillColor] ( 74.31,205.60) rectangle (111.49,252.52);

		\path[fill=fillColor] ( 74.31,205.60) rectangle (111.49,252.52);

		\path[fill=fillColor] ( 74.31,205.60) rectangle (111.49,252.52);

		\path[fill=fillColor] ( 74.31,205.60) rectangle (111.49,252.52);

		\path[fill=fillColor] ( 74.31,205.60) rectangle (111.49,252.52);

		\path[fill=fillColor] ( 74.31,205.60) rectangle (111.49,252.52);

		\path[fill=fillColor] ( 74.31,205.60) rectangle (111.49,252.52);

		\path[fill=fillColor] ( 74.31,205.60) rectangle (111.49,252.52);

		\path[fill=fillColor] ( 74.31,205.60) rectangle (111.49,252.52);

		\path[fill=fillColor] ( 74.31,205.60) rectangle (111.49,252.52);

		\path[fill=fillColor] ( 74.31,205.60) rectangle (111.49,252.52);

		\path[fill=fillColor] ( 74.31,205.60) rectangle (111.49,252.52);

		\path[fill=fillColor] ( 74.31,205.60) rectangle (111.49,252.52);

		\path[fill=fillColor] ( 74.31,205.60) rectangle (111.49,252.52);

		\path[fill=fillColor] ( 74.31,205.60) rectangle (111.49,252.52);

		\path[fill=fillColor] ( 74.31,205.60) rectangle (111.49,252.52);

		\path[fill=fillColor] ( 74.31,205.60) rectangle (111.49,252.52);

		\path[fill=fillColor] ( 74.31,205.60) rectangle (111.49,252.52);

		\path[fill=fillColor] ( 74.31,205.60) rectangle (111.49,252.52);

		\path[fill=fillColor] ( 74.31,205.60) rectangle (111.49,252.52);

		\path[fill=fillColor] ( 74.31,205.60) rectangle (111.49,252.52);

		\path[fill=fillColor] ( 74.31,205.60) rectangle (111.49,252.52);

		\path[fill=fillColor] ( 74.31,205.60) rectangle (111.49,252.52);

		\path[fill=fillColor] ( 74.31,205.60) rectangle (111.49,252.52);

		\path[fill=fillColor] ( 74.31,205.60) rectangle (111.49,252.52);

		\path[fill=fillColor] ( 74.31,205.60) rectangle (111.49,252.52);

		\path[fill=fillColor] ( 74.31,205.60) rectangle (111.49,252.52);

		\path[fill=fillColor] ( 74.31,205.60) rectangle (111.49,252.52);

		\path[fill=fillColor] ( 74.31,205.60) rectangle (111.49,252.52);

		\path[fill=fillColor] ( 74.31,205.60) rectangle (111.49,252.52);

		\path[fill=fillColor] ( 74.31,205.60) rectangle (111.49,252.52);

		\path[fill=fillColor] ( 74.31,205.60) rectangle (111.49,252.52);

		\path[fill=fillColor] ( 74.31,205.60) rectangle (111.49,252.52);

		\path[fill=fillColor] ( 74.31,205.60) rectangle (111.49,261.15);

		\path[fill=fillColor] ( 74.31,205.60) rectangle (111.49,261.15);

		\path[fill=fillColor] ( 74.31,205.60) rectangle (111.49,261.15);

		\path[fill=fillColor] ( 74.31,205.60) rectangle (111.49,261.15);

		\path[fill=fillColor] ( 74.31,205.60) rectangle (111.49,261.15);

		\path[fill=fillColor] ( 74.31,205.60) rectangle (111.49,261.15);

		\path[fill=fillColor] ( 74.31,205.60) rectangle (111.49,261.15);

		\path[fill=fillColor] ( 74.31,205.60) rectangle (111.49,261.15);

		\path[fill=fillColor] ( 74.31,205.60) rectangle (111.49,261.15);

		\path[fill=fillColor] ( 74.31,205.60) rectangle (111.49,261.15);

		\path[fill=fillColor] ( 74.31,205.60) rectangle (111.49,261.15);

		\path[fill=fillColor] ( 74.31,205.60) rectangle (111.49,261.15);

		\path[fill=fillColor] ( 74.31,205.60) rectangle (111.49,261.15);

		\path[fill=fillColor] ( 74.31,205.60) rectangle (111.49,261.15);

		\path[fill=fillColor] ( 74.31,205.60) rectangle (111.49,261.15);

		\path[fill=fillColor] ( 74.31,205.60) rectangle (111.49,261.15);

		\path[fill=fillColor] ( 74.31,205.60) rectangle (111.49,261.15);

		\path[fill=fillColor] ( 74.31,205.60) rectangle (111.49,261.15);

		\path[fill=fillColor] ( 74.31,205.60) rectangle (111.49,261.15);

		\path[fill=fillColor] ( 74.31,205.60) rectangle (111.49,261.15);

		\path[fill=fillColor] ( 74.31,205.60) rectangle (111.49,261.15);

		\path[fill=fillColor] ( 74.31,205.60) rectangle (111.49,261.15);

		\path[fill=fillColor] ( 74.31,205.60) rectangle (111.49,261.15);

		\path[fill=fillColor] ( 74.31,205.60) rectangle (111.49,261.15);

		\path[fill=fillColor] ( 74.31,205.60) rectangle (111.49,261.15);

		\path[fill=fillColor] ( 74.31,205.60) rectangle (111.49,261.15);

		\path[fill=fillColor] ( 74.31,205.60) rectangle (111.49,261.15);

		\path[fill=fillColor] ( 74.31,205.60) rectangle (111.49,261.15);

		\path[fill=fillColor] ( 74.31,205.60) rectangle (111.49,261.15);

		\path[fill=fillColor] ( 74.31,205.60) rectangle (111.49,261.15);

		\path[fill=fillColor] ( 74.31,205.60) rectangle (111.49,261.15);

		\path[fill=fillColor] ( 74.31,205.60) rectangle (111.49,261.15);

		\path[fill=fillColor] ( 74.31,205.60) rectangle (111.49,261.15);

		\path[fill=fillColor] ( 74.31,205.60) rectangle (111.49,261.15);

		\path[fill=fillColor] ( 74.31,205.60) rectangle (111.49,261.15);

		\path[fill=fillColor] ( 74.31,205.60) rectangle (111.49,261.15);

		\path[fill=fillColor] ( 74.31,205.60) rectangle (111.49,261.15);

		\path[fill=fillColor] ( 74.31,205.60) rectangle (111.49,261.15);

		\path[fill=fillColor] ( 74.31,205.60) rectangle (111.49,261.15);

		\path[fill=fillColor] ( 74.31,205.60) rectangle (111.49,261.15);

		\path[fill=fillColor] ( 74.31,205.60) rectangle (111.49,261.15);

		\path[fill=fillColor] ( 74.31,205.60) rectangle (111.49,261.15);

		\path[fill=fillColor] ( 74.31,205.60) rectangle (111.49,261.15);

		\path[fill=fillColor] ( 74.31,205.60) rectangle (111.49,261.15);

		\path[fill=fillColor] ( 74.31,205.60) rectangle (111.49,261.15);

		\path[fill=fillColor] ( 74.31,205.60) rectangle (111.49,261.15);

		\path[fill=fillColor] ( 74.31,205.60) rectangle (111.49,261.15);

		\path[fill=fillColor] ( 74.31,205.60) rectangle (111.49,261.15);

		\path[fill=fillColor] ( 74.31,205.60) rectangle (111.49,261.15);

		\path[fill=fillColor] ( 74.31,205.60) rectangle (111.49,261.15);

		\path[fill=fillColor] ( 74.31,205.60) rectangle (111.49,261.15);

		\path[fill=fillColor] ( 74.31,205.60) rectangle (111.49,261.15);

		\path[fill=fillColor] ( 74.31,205.60) rectangle (111.49,261.15);

		\path[fill=fillColor] ( 74.31,205.60) rectangle (111.49,261.15);

		\path[fill=fillColor] ( 74.31,205.60) rectangle (111.49,261.15);

		\path[fill=fillColor] ( 74.31,205.60) rectangle (111.49,261.15);

		\path[fill=fillColor] ( 74.31,205.60) rectangle (111.49,261.15);

		\path[fill=fillColor] ( 74.31,205.60) rectangle (111.49,261.15);

		\path[fill=fillColor] ( 74.31,205.60) rectangle (111.49,261.15);

		\path[fill=fillColor] ( 74.31,205.60) rectangle (111.49,261.15);

		\path[fill=fillColor] ( 74.31,205.60) rectangle (111.49,261.15);

		\path[fill=fillColor] ( 74.31,205.60) rectangle (111.49,261.15);

		\path[fill=fillColor] ( 74.31,205.60) rectangle (111.49,261.15);

		\path[fill=fillColor] ( 74.31,205.60) rectangle (111.49,261.15);

		\path[fill=fillColor] ( 74.31,205.60) rectangle (111.49,261.15);

		\path[fill=fillColor] ( 74.31,205.60) rectangle (111.49,261.15);

		\path[fill=fillColor] ( 74.31,205.60) rectangle (111.49,261.15);

		\path[fill=fillColor] ( 74.31,205.60) rectangle (111.49,261.15);

		\path[fill=fillColor] ( 74.31,205.60) rectangle (111.49,261.15);

		\path[fill=fillColor] ( 74.31,205.60) rectangle (111.49,261.15);

		\path[fill=fillColor] ( 74.31,205.60) rectangle (111.49,261.15);

		\path[fill=fillColor] ( 74.31,205.60) rectangle (111.49,261.15);

		\path[fill=fillColor] ( 74.31,205.60) rectangle (111.49,261.15);

		\path[fill=fillColor] ( 74.31,205.60) rectangle (111.49,261.15);

		\path[fill=fillColor] ( 74.31,205.60) rectangle (111.49,261.15);

		\path[fill=fillColor] ( 74.31,205.60) rectangle (111.49,261.15);

		\path[fill=fillColor] ( 74.31,205.60) rectangle (111.49,261.15);

		\path[fill=fillColor] ( 74.31,205.60) rectangle (111.49,261.15);

		\path[fill=fillColor] ( 74.31,205.60) rectangle (111.49,261.15);

		\path[fill=fillColor] ( 74.31,205.60) rectangle (111.49,261.15);

		\path[fill=fillColor] ( 74.31,205.60) rectangle (111.49,261.15);

		\path[fill=fillColor] ( 74.31,205.60) rectangle (111.49,261.15);

		\path[fill=fillColor] ( 74.31,205.60) rectangle (111.49,261.15);

		\path[fill=fillColor] ( 74.31,205.60) rectangle (111.49,261.15);

		\path[fill=fillColor] ( 74.31,205.60) rectangle (111.49,261.15);

		\path[fill=fillColor] ( 74.31,205.60) rectangle (111.49,261.15);

		\path[fill=fillColor] ( 74.31,205.60) rectangle (111.49,261.15);

		\path[fill=fillColor] ( 74.31,205.60) rectangle (111.49,261.15);

		\path[fill=fillColor] ( 74.31,205.60) rectangle (111.49,261.15);

		\path[fill=fillColor] ( 74.31,205.60) rectangle (111.49,261.15);

		\path[fill=fillColor] ( 74.31,205.60) rectangle (111.49,261.15);

		\path[fill=fillColor] ( 74.31,205.60) rectangle (111.49,261.15);

		\path[fill=fillColor] ( 74.31,205.60) rectangle (111.49,261.15);

		\path[fill=fillColor] ( 74.31,205.60) rectangle (111.49,261.15);

		\path[fill=fillColor] ( 74.31,205.60) rectangle (111.49,261.15);

		\path[fill=fillColor] ( 74.31,205.60) rectangle (111.49,261.15);

		\path[fill=fillColor] ( 74.31,205.60) rectangle (111.49,261.15);

		\path[fill=fillColor] ( 74.31,205.60) rectangle (111.49,261.15);

		\path[fill=fillColor] ( 74.31,205.60) rectangle (111.49,261.15);

		\path[fill=fillColor] ( 74.31,205.60) rectangle (111.49,261.15);

		\path[fill=fillColor] ( 74.31,205.60) rectangle (111.49,261.15);

		\path[fill=fillColor] ( 74.31,205.60) rectangle (111.49,261.15);

		\path[fill=fillColor] ( 74.31,205.60) rectangle (111.49,261.15);

		\path[fill=fillColor] ( 74.31,205.60) rectangle (111.49,261.15);

		\path[fill=fillColor] ( 74.31,205.60) rectangle (111.49,261.15);

		\path[fill=fillColor] ( 74.31,205.60) rectangle (111.49,261.15);

		\path[fill=fillColor] ( 74.31,205.60) rectangle (111.49,261.15);

		\path[fill=fillColor] ( 74.31,205.60) rectangle (111.49,261.15);

		\path[fill=fillColor] ( 74.31,205.60) rectangle (111.49,261.15);

		\path[fill=fillColor] ( 74.31,205.60) rectangle (111.49,261.15);

		\path[fill=fillColor] ( 74.31,205.60) rectangle (111.49,261.15);

		\path[fill=fillColor] ( 74.31,205.60) rectangle (111.49,261.15);

		\path[fill=fillColor] ( 74.31,205.60) rectangle (111.49,261.15);

		\path[fill=fillColor] ( 74.31,205.60) rectangle (111.49,261.15);

		\path[fill=fillColor] ( 74.31,205.60) rectangle (111.49,261.15);

		\path[fill=fillColor] ( 74.31,205.60) rectangle (111.49,261.15);

		\path[fill=fillColor] ( 74.31,205.60) rectangle (111.49,261.15);

		\path[fill=fillColor] ( 74.31,205.60) rectangle (111.49,261.15);

		\path[fill=fillColor] ( 74.31,205.60) rectangle (111.49,261.15);

		\path[fill=fillColor] ( 74.31,205.60) rectangle (111.49,261.15);

		\path[fill=fillColor] ( 74.31,205.60) rectangle (111.49,261.15);

		\path[fill=fillColor] ( 74.31,205.60) rectangle (111.49,261.15);

		\path[fill=fillColor] ( 74.31,205.60) rectangle (111.49,261.15);

		\path[fill=fillColor] ( 74.31,205.60) rectangle (111.49,261.15);

		\path[fill=fillColor] ( 74.31,205.60) rectangle (111.49,261.15);

		\path[fill=fillColor] ( 74.31,205.60) rectangle (111.49,261.15);

		\path[fill=fillColor] ( 74.31,205.60) rectangle (111.49,261.15);

		\path[fill=fillColor] ( 74.31,205.60) rectangle (111.49,261.15);

		\path[fill=fillColor] ( 74.31,205.60) rectangle (111.49,261.15);

		\path[fill=fillColor] ( 74.31,205.60) rectangle (111.49,261.15);

		\path[fill=fillColor] ( 74.31,205.60) rectangle (111.49,261.15);

		\path[fill=fillColor] ( 74.31,205.60) rectangle (111.49,261.15);

		\path[fill=fillColor] ( 74.31,205.60) rectangle (111.49,261.15);

		\path[fill=fillColor] ( 74.31,205.60) rectangle (111.49,261.15);

		\path[fill=fillColor] ( 74.31,205.60) rectangle (111.49,261.15);

		\path[fill=fillColor] ( 74.31,205.60) rectangle (111.49,261.15);

		\path[fill=fillColor] ( 74.31,205.60) rectangle (111.49,261.15);

		\path[fill=fillColor] ( 74.31,205.60) rectangle (111.49,261.15);

		\path[fill=fillColor] ( 74.31,205.60) rectangle (111.49,261.15);

		\path[fill=fillColor] ( 74.31,205.60) rectangle (111.49,261.15);

		\path[fill=fillColor] ( 74.31,205.60) rectangle (111.49,261.15);

		\path[fill=fillColor] ( 74.31,205.60) rectangle (111.49,261.15);

		\path[fill=fillColor] ( 74.31,205.60) rectangle (111.49,261.15);

		\path[fill=fillColor] ( 74.31,205.60) rectangle (111.49,261.15);

		\path[fill=fillColor] ( 74.31,205.60) rectangle (111.49,261.15);

		\path[fill=fillColor] ( 74.31,205.60) rectangle (111.49,261.15);

		\path[fill=fillColor] ( 74.31,205.60) rectangle (111.49,261.15);

		\path[fill=fillColor] ( 74.31,205.60) rectangle (111.49,261.15);

		\path[fill=fillColor] ( 74.31,205.60) rectangle (111.49,261.15);

		\path[fill=fillColor] ( 74.31,205.60) rectangle (111.49,261.15);

		\path[fill=fillColor] ( 74.31,205.60) rectangle (111.49,261.15);

		\path[fill=fillColor] ( 74.31,205.60) rectangle (111.49,261.15);

		\path[fill=fillColor] ( 74.31,205.60) rectangle (111.49,261.15);

		\path[fill=fillColor] ( 74.31,205.60) rectangle (111.49,261.15);

		\path[fill=fillColor] ( 74.31,205.60) rectangle (111.49,261.15);

		\path[fill=fillColor] ( 74.31,205.60) rectangle (111.49,261.15);

		\path[fill=fillColor] ( 74.31,205.60) rectangle (111.49,261.15);

		\path[fill=fillColor] ( 74.31,205.60) rectangle (111.49,261.15);

		\path[fill=fillColor] ( 74.31,205.60) rectangle (111.49,261.15);

		\path[fill=fillColor] ( 74.31,205.60) rectangle (111.49,261.15);

		\path[fill=fillColor] ( 74.31,205.60) rectangle (111.49,261.15);

		\path[fill=fillColor] ( 74.31,205.60) rectangle (111.49,261.15);

		\path[fill=fillColor] ( 74.31,205.60) rectangle (111.49,261.15);

		\path[fill=fillColor] ( 74.31,205.60) rectangle (111.49,261.15);

		\path[fill=fillColor] ( 74.31,205.60) rectangle (111.49,261.15);

		\path[fill=fillColor] ( 74.31,205.60) rectangle (111.49,261.15);

		\path[fill=fillColor] ( 74.31,205.60) rectangle (111.49,261.15);

		\path[fill=fillColor] ( 74.31,205.60) rectangle (111.49,261.15);

		\path[fill=fillColor] ( 74.31,205.60) rectangle (111.49,261.15);

		\path[fill=fillColor] ( 74.31,205.60) rectangle (111.49,261.15);

		\path[fill=fillColor] ( 74.31,205.60) rectangle (111.49,261.15);

		\path[fill=fillColor] ( 74.31,205.60) rectangle (111.49,261.15);

		\path[fill=fillColor] ( 74.31,205.60) rectangle (111.49,261.15);

		\path[fill=fillColor] ( 74.31,205.60) rectangle (111.49,261.15);

		\path[fill=fillColor] ( 74.31,205.60) rectangle (111.49,261.15);

		\path[fill=fillColor] ( 74.31,205.60) rectangle (111.49,261.15);

		\path[fill=fillColor] ( 74.31,205.60) rectangle (111.49,261.15);

		\path[fill=fillColor] ( 74.31,205.60) rectangle (111.49,261.15);

		\path[fill=fillColor] ( 74.31,205.60) rectangle (111.49,261.15);

		\path[fill=fillColor] ( 74.31,205.60) rectangle (111.49,261.15);

		\path[fill=fillColor] ( 74.31,205.60) rectangle (111.49,261.15);

		\path[fill=fillColor] ( 74.31,205.60) rectangle (111.49,261.15);

		\path[fill=fillColor] ( 74.31,205.60) rectangle (111.49,261.15);

		\path[fill=fillColor] ( 74.31,205.60) rectangle (111.49,261.15);

		\path[fill=fillColor] ( 74.31,205.60) rectangle (111.49,261.15);

		\path[fill=fillColor] ( 74.31,205.60) rectangle (111.49,261.15);

		\path[fill=fillColor] ( 74.31,205.60) rectangle (111.49,261.15);

		\path[fill=fillColor] ( 74.31,205.60) rectangle (111.49,261.15);

		\path[fill=fillColor] ( 74.31,205.60) rectangle (111.49,261.15);

		\path[fill=fillColor] ( 74.31,205.60) rectangle (111.49,261.15);

		\path[fill=fillColor] ( 74.31,205.60) rectangle (111.49,261.15);

		\path[fill=fillColor] ( 74.31,205.60) rectangle (111.49,261.15);

		\path[fill=fillColor] ( 74.31,205.60) rectangle (111.49,261.15);

		\path[fill=fillColor] ( 74.31,205.60) rectangle (111.49,261.15);

		\path[fill=fillColor] ( 74.31,205.60) rectangle (111.49,261.15);

		\path[fill=fillColor] ( 74.31,205.60) rectangle (111.49,261.15);

		\path[fill=fillColor] ( 74.31,205.60) rectangle (111.49,261.15);

		\path[fill=fillColor] ( 74.31,205.60) rectangle (111.49,261.15);

		\path[fill=fillColor] ( 74.31,205.60) rectangle (111.49,261.15);

		\path[fill=fillColor] ( 74.31,205.60) rectangle (111.49,261.15);

		\path[fill=fillColor] ( 74.31,205.60) rectangle (111.49,261.15);

		\path[fill=fillColor] ( 74.31,205.60) rectangle (111.49,261.15);

		\path[fill=fillColor] ( 74.31,205.60) rectangle (111.49,261.15);

		\path[fill=fillColor] ( 74.31,205.60) rectangle (111.49,261.15);

		\path[fill=fillColor] ( 74.31,205.60) rectangle (111.49,261.15);

		\path[fill=fillColor] ( 74.31,205.60) rectangle (111.49,261.15);

		\path[fill=fillColor] ( 74.31,205.60) rectangle (111.49,261.15);

		\path[fill=fillColor] ( 74.31,205.60) rectangle (111.49,261.15);

		\path[fill=fillColor] ( 74.31,205.60) rectangle (111.49,261.15);

		\path[fill=fillColor] ( 74.31,205.60) rectangle (111.49,261.15);

		\path[fill=fillColor] ( 74.31,205.60) rectangle (111.49,261.15);

		\path[fill=fillColor] ( 74.31,205.60) rectangle (111.49,261.15);

		\path[fill=fillColor] ( 74.31,205.60) rectangle (111.49,261.15);

		\path[fill=fillColor] ( 74.31,205.60) rectangle (111.49,261.15);

		\path[fill=fillColor] ( 74.31,205.60) rectangle (111.49,261.15);

		\path[fill=fillColor] ( 74.31,205.60) rectangle (111.49,261.15);

		\path[fill=fillColor] ( 74.31,205.60) rectangle (111.49,261.15);

		\path[fill=fillColor] ( 74.31,205.60) rectangle (111.49,261.15);

		\path[fill=fillColor] ( 74.31,205.60) rectangle (111.49,261.15);

		\path[fill=fillColor] ( 74.31,205.60) rectangle (111.49,261.15);

		\path[fill=fillColor] ( 74.31,205.60) rectangle (111.49,261.15);

		\path[fill=fillColor] ( 74.31,205.60) rectangle (111.49,261.15);

		\path[fill=fillColor] ( 74.31,205.60) rectangle (111.49,261.15);

		\path[fill=fillColor] ( 74.31,205.60) rectangle (111.49,261.15);

		\path[fill=fillColor] ( 74.31,205.60) rectangle (111.49,261.15);

		\path[fill=fillColor] ( 74.31,205.60) rectangle (111.49,261.15);

		\path[fill=fillColor] ( 74.31,205.60) rectangle (111.49,261.15);

		\path[fill=fillColor] ( 74.31,205.60) rectangle (111.49,261.15);

		\path[fill=fillColor] ( 74.31,205.60) rectangle (111.49,261.15);

		\path[fill=fillColor] ( 74.31,205.60) rectangle (111.49,261.15);

		\path[fill=fillColor] ( 74.31,205.60) rectangle (111.49,261.15);

		\path[fill=fillColor] ( 74.31,205.60) rectangle (111.49,261.15);

		\path[fill=fillColor] ( 74.31,205.60) rectangle (111.49,261.15);

		\path[fill=fillColor] ( 74.31,205.60) rectangle (111.49,261.15);

		\path[fill=fillColor] ( 74.31,205.60) rectangle (111.49,261.15);

		\path[fill=fillColor] ( 74.31,205.60) rectangle (111.49,261.15);

		\path[fill=fillColor] ( 74.31,205.60) rectangle (111.49,261.15);

		\path[fill=fillColor] ( 74.31,205.60) rectangle (111.49,261.15);

		\path[fill=fillColor] ( 74.31,205.60) rectangle (111.49,261.15);

		\path[fill=fillColor] ( 74.31,205.60) rectangle (111.49,261.15);

		\path[fill=fillColor] ( 74.31,205.60) rectangle (111.49,261.15);

		\path[fill=fillColor] ( 74.31,205.60) rectangle (111.49,261.15);

		\path[fill=fillColor] ( 74.31,205.60) rectangle (111.49,261.15);

		\path[fill=fillColor] ( 74.31,205.60) rectangle (111.49,261.15);

		\path[fill=fillColor] ( 74.31,205.60) rectangle (111.49,261.15);

		\path[fill=fillColor] ( 74.31,205.60) rectangle (111.49,261.15);

		\path[fill=fillColor] ( 74.31,205.60) rectangle (111.49,261.15);

		\path[fill=fillColor] ( 74.31,205.60) rectangle (111.49,261.15);

		\path[fill=fillColor] ( 74.31,205.60) rectangle (111.49,261.15);

		\path[fill=fillColor] ( 74.31,205.60) rectangle (111.49,261.15);

		\path[fill=fillColor] ( 74.31,205.60) rectangle (111.49,261.15);

		\path[fill=fillColor] ( 74.31,205.60) rectangle (111.49,261.15);

		\path[fill=fillColor] ( 74.31,205.60) rectangle (111.49,261.15);

		\path[fill=fillColor] ( 74.31,205.60) rectangle (111.49,261.15);

		\path[fill=fillColor] ( 74.31,205.60) rectangle (111.49,261.15);

		\path[fill=fillColor] ( 74.31,205.60) rectangle (111.49,261.15);

		\path[fill=fillColor] ( 74.31,205.60) rectangle (111.49,261.15);

		\path[fill=fillColor] ( 74.31,205.60) rectangle (111.49,261.15);

		\path[fill=fillColor] ( 74.31,205.60) rectangle (111.49,261.15);

		\path[fill=fillColor] ( 74.31,205.60) rectangle (111.49,261.15);

		\path[fill=fillColor] ( 74.31,205.60) rectangle (111.49,261.15);

		\path[fill=fillColor] ( 74.31,205.60) rectangle (111.49,261.15);

		\path[fill=fillColor] ( 74.31,205.60) rectangle (111.49,261.15);

		\path[fill=fillColor] ( 74.31,205.60) rectangle (111.49,261.15);

		\path[fill=fillColor] ( 74.31,205.60) rectangle (111.49,261.15);

		\path[fill=fillColor] ( 74.31,205.60) rectangle (111.49,261.15);

		\path[fill=fillColor] ( 74.31,205.60) rectangle (111.49,261.15);

		\path[fill=fillColor] ( 74.31,205.60) rectangle (111.49,261.15);

		\path[fill=fillColor] ( 74.31,205.60) rectangle (111.49,261.15);
		\definecolor{drawColor}{RGB}{235,63,121}

		\path[draw=drawColor,line width= 1.1pt,line join=round] ( 74.31,246.91) --
		( 74.93,237.46) --
		( 75.55,209.14) --
		( 76.17,257.22) --
		( 76.79,209.14) --
		( 77.41,209.14) --
		( 78.03,255.70) --
		( 78.65,277.96) --
		( 79.27,277.96) --
		( 79.89,277.96) --
		( 80.50,277.96) --
		( 81.12,241.09) --
		( 81.74,277.96) --
		( 82.36,277.96) --
		( 82.98,241.09) --
		( 83.60,241.09) --
		( 84.22,241.09) --
		( 84.84,221.66) --
		( 85.46,221.66) --
		( 86.08,254.81) --
		( 86.70,254.81) --
		( 87.32,254.81) --
		( 87.94,254.81) --
		( 88.56,254.81) --
		( 89.18,254.81) --
		( 89.80,254.81) --
		( 90.42,254.81) --
		( 91.04,254.81) --
		( 91.66,254.81) --
		( 92.28,249.98) --
		( 92.90,249.98) --
		( 93.52,249.98) --
		( 94.14,277.96) --
		( 94.76,249.98) --
		( 95.38,249.98) --
		( 96.00,249.98) --
		( 96.62,259.63) --
		( 97.24,259.63) --
		( 97.86,249.98) --
		( 98.48,259.63) --
		( 99.10,259.63) --
		( 99.72,259.63) --
		(100.33,259.63) --
		(100.95,259.63) --
		(101.57,249.98) --
		(102.19,213.76) --
		(102.81,221.66) --
		(103.43,221.66) --
		(104.05,221.66) --
		(104.67,221.66) --
		(105.29,213.76) --
		(105.91,221.66) --
		(106.53,213.76) --
		(107.15,213.76) --
		(107.77,236.47) --
		(108.39,236.47) --
		(109.01,249.98) --
		(109.63,249.98) --
		(110.25,236.47) --
		(110.87,236.47) --
		(111.49,249.98) --
		(112.11,249.98) --
		(112.73,236.47) --
		(113.35,236.47) --
		(113.97,236.47) --
		(114.59,236.47) --
		(115.21,236.47) --
		(115.83,245.12) --
		(116.45,236.47) --
		(117.07,236.47) --
		(117.69,236.47) --
		(118.31,236.47) --
		(118.93,224.94) --
		(119.55,234.18) --
		(120.16,209.14) --
		(120.78,209.14) --
		(121.40,234.18) --
		(122.02,234.18) --
		(122.64,234.18) --
		(123.26,245.12) --
		(123.88,242.69) --
		(124.50,242.69) --
		(125.12,213.76) --
		(125.74,213.76) --
		(126.36,213.76) --
		(126.98,213.76) --
		(127.60,213.76) --
		(128.22,213.76) --
		(128.84,213.76) --
		(129.46,213.76) --
		(130.08,213.76) --
		(130.70,209.14) --
		(131.32,209.14) --
		(131.94,209.14) --
		(132.56,209.14) --
		(133.18,209.14) --
		(133.80,209.14) --
		(134.42,209.14) --
		(135.04,209.14) --
		(135.66,209.14) --
		(136.28,219.58) --
		(136.90,209.14) --
		(137.52,219.58) --
		(138.14,219.58) --
		(138.76,236.47) --
		(139.38,236.47) --
		(139.99,236.47) --
		(140.61,236.47) --
		(141.23,245.12) --
		(141.85,245.12) --
		(142.47,245.12) --
		(143.09,245.12) --
		(143.71,245.12) --
		(144.33,245.12) --
		(144.95,245.12) --
		(145.57,245.12) --
		(146.19,226.28) --
		(146.81,209.14) --
		(147.43,209.14) --
		(148.05,209.14) --
		(148.67,209.14) --
		(149.29,209.14) --
		(149.91,209.14) --
		(150.53,209.14) --
		(151.15,209.14) --
		(151.77,209.14) --
		(152.39,209.14) --
		(153.01,209.14) --
		(153.63,209.14) --
		(154.25,209.14) --
		(154.87,209.14) --
		(155.49,209.14) --
		(156.11,209.14) --
		(156.73,209.14) --
		(157.35,209.14) --
		(157.97,209.14) --
		(158.59,209.14) --
		(159.21,209.14) --
		(159.82,209.14) --
		(160.44,209.14) --
		(161.06,209.14) --
		(161.68,209.14) --
		(162.30,209.14) --
		(162.92,209.14) --
		(163.54,209.14) --
		(164.16,209.14) --
		(164.78,209.14) --
		(165.40,209.14) --
		(166.02,209.14) --
		(166.64,209.14) --
		(167.26,209.14) --
		(167.88,209.14) --
		(168.50,209.14) --
		(169.12,209.14) --
		(169.74,209.14) --
		(170.36,209.14) --
		(170.98,209.14) --
		(171.60,209.14) --
		(172.22,209.14) --
		(172.84,209.14) --
		(173.46,209.14) --
		(174.08,209.14) --
		(174.70,209.14) --
		(175.32,209.14) --
		(175.94,209.14) --
		(176.56,209.14) --
		(177.18,209.14) --
		(177.80,209.14) --
		(178.42,209.14) --
		(179.04,209.14) --
		(179.65,209.14) --
		(180.27,209.14) --
		(180.89,209.14) --
		(181.51,213.76) --
		(182.13,213.76) --
		(182.75,213.76) --
		(183.37,213.76) --
		(183.99,213.76) --
		(184.61,213.76) --
		(185.23,213.76) --
		(185.85,213.76) --
		(186.47,213.76) --
		(187.09,213.76) --
		(187.71,213.76) --
		(188.33,213.76) --
		(188.95,213.76) --
		(189.57,213.76) --
		(190.19,213.76) --
		(190.81,213.76) --
		(191.43,221.66) --
		(192.05,224.94) --
		(192.67,213.76) --
		(193.29,213.76) --
		(193.91,213.76) --
		(194.53,213.76) --
		(195.15,213.76) --
		(195.77,213.76) --
		(196.39,213.76) --
		(197.01,213.76) --
		(197.63,213.76) --
		(198.25,213.76) --
		(198.87,213.76) --
		(199.48,213.76) --
		(200.10,213.76) --
		(200.72,213.76) --
		(201.34,213.76) --
		(201.96,213.76) --
		(202.58,213.76) --
		(203.20,213.76) --
		(203.82,209.14) --
		(204.44,224.94) --
		(205.06,224.94) --
		(205.68,209.14) --
		(206.30,209.14) --
		(206.92,209.14) --
		(207.54,209.14) --
		(208.16,209.14) --
		(208.78,209.14) --
		(209.40,209.14) --
		(210.02,209.14) --
		(210.64,209.14) --
		(211.26,209.14) --
		(211.88,209.14) --
		(212.50,209.14) --
		(213.12,209.14) --
		(213.74,209.14) --
		(214.36,209.14) --
		(214.98,209.14) --
		(215.60,209.14) --
		(216.22,224.94) --
		(216.84,234.18) --
		(217.46,245.12) --
		(218.08,245.12) --
		(218.70,245.12) --
		(219.31,245.12) --
		(219.93,245.12) --
		(220.55,245.12) --
		(221.17,245.12) --
		(221.79,245.12) --
		(222.41,245.12) --
		(223.03,245.12) --
		(223.65,245.12) --
		(224.27,245.12) --
		(224.89,245.12) --
		(225.51,245.12) --
		(226.13,245.12) --
		(226.75,245.12) --
		(227.37,245.12) --
		(227.99,245.12) --
		(228.61,245.12) --
		(229.23,245.12) --
		(229.85,245.12) --
		(230.47,245.12) --
		(231.09,250.44) --
		(231.71,245.12) --
		(232.33,245.12) --
		(232.95,245.12) --
		(233.57,245.12) --
		(234.19,245.12) --
		(234.81,245.12) --
		(235.43,245.12) --
		(236.05,245.12) --
		(236.67,245.12) --
		(237.29,245.12) --
		(237.91,245.12) --
		(238.53,245.12) --
		(239.15,245.12) --
		(239.76,245.12) --
		(240.38,245.12) --
		(241.00,245.12) --
		(241.62,245.12) --
		(242.24,245.12) --
		(242.86,245.12) --
		(243.48,245.12) --
		(244.10,245.12) --
		(244.72,234.18) --
		(245.34,236.97) --
		(245.96,245.12) --
		(246.58,245.12) --
		(247.20,245.12) --
		(247.82,245.12) --
		(248.44,245.12) --
		(249.06,245.12) --
		(249.68,245.12) --
		(250.30,245.12) --
		(250.92,245.12) --
		(251.54,245.12) --
		(252.16,245.12) --
		(252.78,245.12) --
		(253.40,245.12) --
		(254.02,245.12) --
		(254.64,245.12) --
		(255.26,245.12) --
		(255.88,245.12) --
		(256.50,245.12) --
		(257.12,245.12) --
		(257.74,245.12) --
		(258.36,245.12) --
		(258.98,245.12) --
		(259.59,245.12) --
		(260.21,245.12) --
		(260.83,250.44) --
		(261.45,250.44) --
		(262.07,245.12) --
		(262.69,245.12) --
		(263.31,245.12) --
		(263.93,245.12) --
		(264.55,245.12) --
		(265.17,245.12) --
		(265.79,245.12) --
		(266.41,245.12) --
		(267.03,250.44) --
		(267.65,250.44) --
		(268.27,250.44) --
		(268.89,250.44) --
		(269.51,250.44) --
		(270.13,250.44) --
		(270.75,250.44) --
		(271.37,250.44) --
		(271.99,250.44) --
		(272.61,250.44) --
		(273.23,250.44) --
		(273.85,250.44) --
		(274.47,250.44) --
		(275.09,250.44) --
		(275.71,250.44) --
		(276.33,250.44) --
		(276.95,250.44) --
		(277.57,250.44) --
		(278.19,250.44) --
		(278.81,250.44) --
		(279.42,250.44) --
		(280.04,250.44) --
		(280.66,250.44) --
		(281.28,250.44) --
		(281.90,250.44) --
		(282.52,250.44) --
		(283.14,250.44) --
		(283.76,250.44) --
		(284.38,250.44) --
		(285.00,250.44) --
		(285.62,250.44) --
		(286.24,250.44) --
		(286.86,250.44) --
		(287.48,250.44) --
		(288.10,250.44) --
		(288.72,250.44) --
		(289.34,250.44) --
		(289.96,250.44) --
		(290.58,250.44) --
		(291.20,250.44) --
		(291.82,250.44) --
		(292.44,250.44) --
		(293.06,250.44) --
		(293.68,250.44) --
		(294.30,250.44) --
		(294.92,250.44) --
		(295.54,250.44) --
		(296.16,250.44) --
		(296.78,250.44) --
		(297.40,250.44) --
		(298.02,250.44) --
		(298.64,250.44) --
		(299.25,250.44) --
		(299.87,250.44) --
		(300.49,250.44) --
		(301.11,250.44) --
		(301.73,250.44) --
		(302.35,250.44) --
		(302.97,250.44) --
		(303.59,250.44) --
		(304.21,250.44) --
		(304.83,250.44) --
		(305.45,250.44) --
		(306.07,250.44) --
		(306.69,250.44) --
		(307.31,250.44) --
		(307.93,250.44) --
		(308.55,250.44) --
		(309.17,250.44) --
		(309.79,250.44) --
		(310.41,250.44) --
		(311.03,250.44) --
		(311.65,250.44) --
		(312.27,250.44) --
		(312.89,250.44) --
		(313.51,250.44) --
		(314.13,250.44) --
		(314.75,250.44) --
		(315.37,250.44) --
		(315.99,250.44) --
		(316.61,250.44) --
		(317.23,250.44) --
		(317.85,250.44) --
		(318.47,250.44) --
		(319.08,250.44) --
		(319.70,250.44) --
		(320.32,250.44) --
		(320.94,250.44) --
		(321.56,250.44) --
		(322.18,250.44) --
		(322.80,250.44) --
		(323.42,250.44) --
		(324.04,250.44) --
		(324.66,250.44) --
		(325.28,250.44) --
		(325.90,250.44) --
		(326.52,250.44) --
		(327.14,250.44) --
		(327.76,250.44) --
		(328.38,250.44) --
		(329.00,250.44) --
		(329.62,250.44) --
		(330.24,250.44) --
		(330.86,250.44) --
		(331.48,250.44) --
		(332.10,250.44) --
		(332.72,250.44) --
		(333.34,250.44) --
		(333.96,250.44) --
		(334.58,250.44) --
		(335.20,250.44) --
		(335.82,250.44) --
		(336.44,250.44) --
		(337.06,250.44) --
		(337.68,250.44) --
		(338.30,250.44) --
		(338.91,250.44) --
		(339.53,250.44) --
		(340.15,250.44) --
		(340.77,250.44) --
		(341.39,250.44) --
		(342.01,250.44) --
		(342.63,250.44) --
		(343.25,250.44) --
		(343.87,250.44) --
		(344.49,250.44) --
		(345.11,250.44) --
		(345.73,250.44) --
		(346.35,250.44) --
		(346.97,250.44) --
		(347.59,250.44) --
		(348.21,250.44) --
		(348.83,250.44) --
		(349.45,250.44) --
		(350.07,250.44) --
		(350.69,250.44) --
		(351.31,250.44) --
		(351.93,250.44) --
		(352.55,250.44) --
		(353.17,250.44) --
		(353.79,250.44) --
		(354.41,250.44) --
		(355.03,250.44) --
		(355.65,250.44) --
		(356.27,250.44) --
		(356.89,250.44) --
		(357.51,250.44) --
		(358.13,250.44) --
		(358.74,250.44) --
		(359.36,250.44) --
		(359.98,250.44) --
		(360.60,250.44) --
		(361.22,250.44) --
		(361.84,250.44) --
		(362.46,250.44) --
		(363.08,250.44) --
		(363.70,259.08) --
		(364.32,259.08) --
		(364.94,259.08) --
		(365.56,259.08) --
		(366.18,259.08) --
		(366.80,259.08) --
		(367.42,259.08) --
		(368.04,259.08) --
		(368.66,259.08) --
		(369.28,259.08) --
		(369.90,259.08) --
		(370.52,259.08) --
		(371.14,259.08) --
		(371.76,259.08) --
		(372.38,259.08) --
		(373.00,259.08) --
		(373.62,259.08) --
		(374.24,259.08) --
		(374.86,259.08) --
		(375.48,259.08) --
		(376.10,259.08) --
		(376.72,259.08) --
		(377.34,259.08) --
		(377.96,259.08) --
		(378.57,259.08) --
		(379.19,259.08) --
		(379.81,259.08) --
		(380.43,259.08) --
		(381.05,259.08) --
		(381.67,259.08) --
		(382.29,259.08) --
		(382.91,259.08) --
		(383.53,259.08) --
		(384.15,259.08) --
		(384.77,259.08) --
		(385.39,259.08) --
		(386.01,259.08) --
		(386.63,259.08) --
		(387.25,259.08) --
		(387.87,259.08) --
		(388.49,259.08) --
		(389.11,259.08) --
		(389.73,259.08) --
		(390.35,259.08) --
		(390.97,259.08) --
		(391.59,259.08) --
		(392.21,259.08) --
		(392.83,259.08) --
		(393.45,259.08) --
		(394.07,259.08) --
		(394.69,259.08) --
		(395.31,259.08) --
		(395.93,259.08) --
		(396.55,259.08) --
		(397.17,259.08) --
		(397.79,259.08) --
		(398.41,259.08) --
		(399.02,259.08) --
		(399.64,259.08) --
		(400.26,259.08) --
		(400.88,259.08) --
		(401.50,259.08) --
		(402.12,259.08) --
		(402.74,259.08) --
		(403.36,259.08) --
		(403.98,259.08) --
		(404.60,259.08) --
		(405.22,259.08) --
		(405.84,259.08) --
		(406.46,259.08) --
		(407.08,259.08) --
		(407.70,259.08) --
		(408.32,259.08) --
		(408.94,259.08) --
		(409.56,259.08) --
		(410.18,259.08) --
		(410.80,259.08) --
		(411.42,259.08) --
		(412.04,259.08) --
		(412.66,259.08) --
		(413.28,259.08) --
		(413.90,259.08) --
		(414.52,259.08) --
		(415.14,259.08) --
		(415.76,259.08) --
		(416.38,259.08) --
		(417.00,259.08) --
		(417.62,259.08) --
		(418.24,259.08) --
		(418.85,259.08) --
		(419.47,259.08) --
		(420.09,259.08) --
		(420.71,259.08) --
		(421.33,259.08) --
		(421.95,259.08) --
		(422.57,259.08) --
		(423.19,259.08) --
		(423.81,259.08) --
		(424.43,259.08) --
		(425.05,259.08) --
		(425.67,259.08) --
		(426.29,259.08) --
		(426.91,259.08) --
		(427.53,259.08) --
		(428.15,259.08) --
		(428.77,259.08) --
		(429.39,259.08) --
		(430.01,259.08) --
		(430.63,259.08) --
		(431.25,259.08) --
		(431.87,259.08) --
		(432.49,259.08) --
		(433.11,259.08) --
		(433.73,259.08) --
		(434.35,259.08) --
		(434.97,259.08) --
		(435.59,259.08) --
		(436.21,259.08) --
		(436.83,259.08) --
		(437.45,259.08) --
		(438.07,259.08) --
		(438.68,259.08) --
		(439.30,259.08) --
		(439.92,259.08) --
		(440.54,259.08) --
		(441.16,259.08) --
		(441.78,259.08) --
		(442.40,259.08) --
		(443.02,259.08) --
		(443.64,259.08) --
		(444.26,259.08) --
		(444.88,259.08) --
		(445.50,259.08) --
		(446.12,259.08) --
		(446.74,259.08) --
		(447.36,259.08) --
		(447.98,259.08) --
		(448.60,259.08) --
		(449.22,259.08) --
		(449.84,259.08) --
		(450.46,259.08) --
		(451.08,259.08) --
		(451.70,259.08) --
		(452.32,259.08) --
		(452.94,259.08) --
		(453.56,259.08) --
		(454.18,259.08) --
		(454.80,259.08) --
		(455.42,259.08) --
		(456.04,259.08) --
		(456.66,259.08) --
		(457.28,259.08) --
		(457.90,259.08) --
		(458.51,259.08) --
		(459.13,259.08) --
		(459.75,259.08) --
		(460.37,259.08) --
		(460.99,259.08) --
		(461.61,259.08) --
		(462.23,259.08) --
		(462.85,259.08) --
		(463.47,259.08) --
		(464.09,259.08) --
		(464.71,259.08) --
		(465.33,259.08) --
		(465.95,259.08) --
		(466.57,259.08) --
		(467.19,259.08) --
		(467.81,259.08) --
		(468.43,259.08) --
		(469.05,259.08) --
		(469.67,259.08) --
		(470.29,259.08) --
		(470.91,259.08) --
		(471.53,259.08) --
		(472.15,259.08) --
		(472.77,259.08) --
		(473.39,259.08) --
		(474.01,259.08) --
		(474.63,259.08) --
		(475.25,259.08) --
		(475.87,259.08) --
		(476.49,259.08) --
		(477.11,259.08) --
		(477.73,259.08) --
		(478.34,259.08) --
		(478.96,259.08) --
		(479.58,259.08) --
		(480.20,259.08) --
		(480.82,259.08) --
		(481.44,259.08) --
		(482.06,259.08) --
		(482.68,259.08) --
		(483.30,259.08) --
		(483.92,259.08) --
		(484.54,259.08) --
		(485.16,259.08) --
		(485.78,259.08) --
		(486.40,259.08) --
		(487.02,259.08) --
		(487.64,259.08) --
		(488.26,259.08) --
		(488.88,259.08) --
		(489.50,259.08) --
		(490.12,259.08) --
		(490.74,259.08) --
		(491.36,259.08) --
		(491.98,259.08) --
		(492.60,259.08) --
		(493.22,259.08) --
		(493.84,259.08) --
		(494.46,259.08) --
		(495.08,259.08) --
		(495.70,259.08) --
		(496.32,259.08) --
		(496.94,259.08) --
		(497.56,259.08) --
		(498.17,259.08) --
		(498.79,259.08) --
		(499.41,259.08) --
		(500.03,259.08) --
		(500.65,259.08) --
		(501.27,259.08) --
		(501.89,259.08) --
		(502.51,259.08) --
		(503.13,259.08) --
		(503.75,259.08) --
		(504.37,259.08) --
		(504.99,259.08) --
		(505.61,259.08) --
		(506.23,259.08) --
		(506.85,259.08) --
		(507.47,259.08) --
		(508.09,259.08) --
		(508.71,259.08) --
		(509.33,259.08) --
		(509.95,259.08) --
		(510.57,259.08) --
		(511.19,259.08) --
		(511.81,259.08) --
		(512.43,259.08) --
		(513.05,259.08) --
		(513.67,259.08) --
		(514.29,259.08) --
		(514.91,259.08) --
		(515.53,259.08) --
		(516.15,259.08) --
		(516.77,259.08) --
		(517.39,259.08) --
		(518.00,259.08) --
		(518.62,259.08) --
		(519.24,259.08) --
		(519.86,259.08) --
		(520.48,259.08) --
		(521.10,259.08) --
		(521.72,259.08) --
		(522.34,259.08) --
		(522.96,259.08) --
		(523.58,259.08) --
		(524.20,259.08) --
		(524.82,259.08) --
		(525.44,259.08) --
		(526.06,259.08) --
		(526.68,259.08) --
		(527.30,259.08) --
		(527.92,259.08) --
		(528.54,259.08) --
		(529.16,259.08) --
		(529.78,259.08);
	\end{scope}
	\begin{scope}
		\path[clip] ( 51.53,122.12) rectangle (552.55,200.10);
		\definecolor{drawColor}{gray}{0.92}

		\path[draw=drawColor,line width= 0.3pt,line join=round] ( 51.53,127.40) --
		(552.55,127.40);

		\path[draw=drawColor,line width= 0.3pt,line join=round] ( 51.53,144.90) --
		(552.55,144.90);

		\path[draw=drawColor,line width= 0.3pt,line join=round] ( 51.53,162.39) --
		(552.55,162.39);

		\path[draw=drawColor,line width= 0.3pt,line join=round] ( 51.53,179.89) --
		(552.55,179.89);

		\path[draw=drawColor,line width= 0.3pt,line join=round] ( 51.53,197.38) --
		(552.55,197.38);

		\path[draw=drawColor,line width= 0.3pt,line join=round] (133.49,122.12) --
		(133.49,200.10);

		\path[draw=drawColor,line width= 0.3pt,line join=round] (246.58,122.12) --
		(246.58,200.10);

		\path[draw=drawColor,line width= 0.3pt,line join=round] (359.98,122.12) --
		(359.98,200.10);

		\path[draw=drawColor,line width= 0.3pt,line join=round] (473.39,122.12) --
		(473.39,200.10);

		\path[draw=drawColor,line width= 0.6pt,line join=round] ( 51.53,136.15) --
		(552.55,136.15);

		\path[draw=drawColor,line width= 0.6pt,line join=round] ( 51.53,153.64) --
		(552.55,153.64);

		\path[draw=drawColor,line width= 0.6pt,line join=round] ( 51.53,171.14) --
		(552.55,171.14);

		\path[draw=drawColor,line width= 0.6pt,line join=round] ( 51.53,188.64) --
		(552.55,188.64);

		\path[draw=drawColor,line width= 0.6pt,line join=round] ( 77.41,122.12) --
		( 77.41,200.10);

		\path[draw=drawColor,line width= 0.6pt,line join=round] (189.57,122.12) --
		(189.57,200.10);

		\path[draw=drawColor,line width= 0.6pt,line join=round] (303.59,122.12) --
		(303.59,200.10);

		\path[draw=drawColor,line width= 0.6pt,line join=round] (416.38,122.12) --
		(416.38,200.10);

		\path[draw=drawColor,line width= 0.6pt,line join=round] (530.40,122.12) --
		(530.40,200.10);
		\definecolor{fillColor}{gray}{0.93}

		\path[fill=fillColor] ( 74.31,122.12) rectangle (111.49,180.72);

		\path[fill=fillColor] ( 74.31,122.12) rectangle (111.49,126.58);

		\path[fill=fillColor] ( 74.31,122.12) rectangle (111.49,175.67);

		\path[fill=fillColor] ( 74.31,122.12) rectangle (111.49,196.54);

		\path[fill=fillColor] ( 74.31,122.12) rectangle (111.49,196.55);

		\path[fill=fillColor] ( 74.31,122.12) rectangle (111.49,142.48);

		\path[fill=fillColor] ( 74.31,122.12) rectangle (111.49,126.58);

		\path[fill=fillColor] ( 74.31,122.12) rectangle (111.49,135.74);

		\path[fill=fillColor] ( 74.31,122.12) rectangle (111.49,135.74);

		\path[fill=fillColor] ( 74.31,122.12) rectangle (111.49,135.74);

		\path[fill=fillColor] ( 74.31,122.12) rectangle (111.49,135.74);

		\path[fill=fillColor] ( 74.31,122.12) rectangle (111.49,135.39);

		\path[fill=fillColor] ( 74.31,122.12) rectangle (111.49,135.74);

		\path[fill=fillColor] ( 74.31,122.12) rectangle (111.49,135.74);

		\path[fill=fillColor] ( 74.31,122.12) rectangle (111.49,135.39);

		\path[fill=fillColor] ( 74.31,122.12) rectangle (111.49,135.39);

		\path[fill=fillColor] ( 74.31,122.12) rectangle (111.49,135.39);

		\path[fill=fillColor] ( 74.31,122.12) rectangle (111.49,142.21);

		\path[fill=fillColor] ( 74.31,122.12) rectangle (111.49,141.92);

		\path[fill=fillColor] ( 74.31,122.12) rectangle (111.49,136.17);

		\path[fill=fillColor] ( 74.31,122.12) rectangle (111.49,136.17);

		\path[fill=fillColor] ( 74.31,122.12) rectangle (111.49,136.17);

		\path[fill=fillColor] ( 74.31,122.12) rectangle (111.49,136.17);

		\path[fill=fillColor] ( 74.31,122.12) rectangle (111.49,136.17);

		\path[fill=fillColor] ( 74.31,122.12) rectangle (111.49,136.17);

		\path[fill=fillColor] ( 74.31,122.12) rectangle (111.49,136.17);

		\path[fill=fillColor] ( 74.31,122.12) rectangle (111.49,136.17);

		\path[fill=fillColor] ( 74.31,122.12) rectangle (111.49,136.17);

		\path[fill=fillColor] ( 74.31,122.12) rectangle (111.49,136.17);

		\path[fill=fillColor] ( 74.31,122.12) rectangle (111.49,142.64);

		\path[fill=fillColor] ( 74.31,122.12) rectangle (111.49,142.64);

		\path[fill=fillColor] ( 74.31,122.12) rectangle (111.49,142.64);

		\path[fill=fillColor] ( 74.31,122.12) rectangle (111.49,135.74);

		\path[fill=fillColor] ( 74.31,122.12) rectangle (111.49,142.64);

		\path[fill=fillColor] ( 74.31,122.12) rectangle (111.49,142.64);

		\path[fill=fillColor] ( 74.31,122.12) rectangle (111.49,142.64);

		\path[fill=fillColor] ( 74.31,122.12) rectangle (111.49,145.53);

		\path[fill=fillColor] ( 74.31,122.12) rectangle (111.49,145.53);

		\path[fill=fillColor] ( 74.31,122.12) rectangle (111.49,142.64);

		\path[fill=fillColor] ( 74.31,122.12) rectangle (111.49,145.53);

		\path[fill=fillColor] ( 74.31,122.12) rectangle (111.49,145.53);

		\path[fill=fillColor] ( 74.31,122.12) rectangle (111.49,145.53);

		\path[fill=fillColor] ( 74.31,122.12) rectangle (111.49,145.53);

		\path[fill=fillColor] ( 74.31,122.12) rectangle (111.49,145.53);

		\path[fill=fillColor] ( 74.31,122.12) rectangle (111.49,142.64);

		\path[fill=fillColor] ( 74.31,122.12) rectangle (111.49,143.15);

		\path[fill=fillColor] ( 74.31,122.12) rectangle (111.49,142.21);

		\path[fill=fillColor] ( 74.31,122.12) rectangle (111.49,142.21);

		\path[fill=fillColor] ( 74.31,122.12) rectangle (111.49,142.21);

		\path[fill=fillColor] ( 74.31,122.12) rectangle (111.49,142.21);

		\path[fill=fillColor] ( 74.31,122.12) rectangle (111.49,141.81);

		\path[fill=fillColor] ( 74.31,122.12) rectangle (111.49,142.21);

		\path[fill=fillColor] ( 74.31,122.12) rectangle (111.49,141.25);

		\path[fill=fillColor] ( 74.31,122.12) rectangle (111.49,141.25);

		\path[fill=fillColor] ( 74.31,122.12) rectangle (111.49,140.75);

		\path[fill=fillColor] ( 74.31,122.12) rectangle (111.49,140.75);

		\path[fill=fillColor] ( 74.31,122.12) rectangle (111.49,142.64);

		\path[fill=fillColor] ( 74.31,122.12) rectangle (111.49,142.64);

		\path[fill=fillColor] ( 74.31,122.12) rectangle (111.49,140.75);

		\path[fill=fillColor] ( 74.31,122.12) rectangle (111.49,140.75);

		\path[fill=fillColor] ( 74.31,122.12) rectangle (111.49,142.64);

		\path[fill=fillColor] ( 74.31,122.12) rectangle (111.49,142.64);

		\path[fill=fillColor] ( 74.31,122.12) rectangle (111.49,140.75);

		\path[fill=fillColor] ( 74.31,122.12) rectangle (111.49,140.75);

		\path[fill=fillColor] ( 74.31,122.12) rectangle (111.49,140.75);

		\path[fill=fillColor] ( 74.31,122.12) rectangle (111.49,140.75);

		\path[fill=fillColor] ( 74.31,122.12) rectangle (111.49,140.75);

		\path[fill=fillColor] ( 74.31,122.12) rectangle (111.49,143.01);

		\path[fill=fillColor] ( 74.31,122.12) rectangle (111.49,140.75);

		\path[fill=fillColor] ( 74.31,122.12) rectangle (111.49,140.75);

		\path[fill=fillColor] ( 74.31,122.12) rectangle (111.49,140.75);

		\path[fill=fillColor] ( 74.31,122.12) rectangle (111.49,140.75);

		\path[fill=fillColor] ( 74.31,122.12) rectangle (111.49,142.96);

		\path[fill=fillColor] ( 74.31,122.12) rectangle (111.49,143.10);

		\path[fill=fillColor] ( 74.31,122.12) rectangle (111.49,140.64);

		\path[fill=fillColor] ( 74.31,122.12) rectangle (111.49,140.64);

		\path[fill=fillColor] ( 74.31,122.12) rectangle (111.49,143.10);

		\path[fill=fillColor] ( 74.31,122.12) rectangle (111.49,143.10);

		\path[fill=fillColor] ( 74.31,122.12) rectangle (111.49,143.10);

		\path[fill=fillColor] ( 74.31,122.12) rectangle (111.49,143.01);

		\path[fill=fillColor] ( 74.31,122.12) rectangle (111.49,140.02);

		\path[fill=fillColor] ( 74.31,122.12) rectangle (111.49,140.02);

		\path[fill=fillColor] ( 74.31,122.12) rectangle (111.49,140.61);

		\path[fill=fillColor] ( 74.31,122.12) rectangle (111.49,140.61);

		\path[fill=fillColor] ( 74.31,122.12) rectangle (111.49,140.61);

		\path[fill=fillColor] ( 74.31,122.12) rectangle (111.49,140.61);

		\path[fill=fillColor] ( 74.31,122.12) rectangle (111.49,140.61);

		\path[fill=fillColor] ( 74.31,122.12) rectangle (111.49,140.61);

		\path[fill=fillColor] ( 74.31,122.12) rectangle (111.49,140.61);

		\path[fill=fillColor] ( 74.31,122.12) rectangle (111.49,140.61);

		\path[fill=fillColor] ( 74.31,122.12) rectangle (111.49,140.61);

		\path[fill=fillColor] ( 74.31,122.12) rectangle (111.49,140.64);

		\path[fill=fillColor] ( 74.31,122.12) rectangle (111.49,140.64);

		\path[fill=fillColor] ( 74.31,122.12) rectangle (111.49,140.64);

		\path[fill=fillColor] ( 74.31,122.12) rectangle (111.49,140.64);

		\path[fill=fillColor] ( 74.31,122.12) rectangle (111.49,140.64);

		\path[fill=fillColor] ( 74.31,122.12) rectangle (111.49,140.64);

		\path[fill=fillColor] ( 74.31,122.12) rectangle (111.49,140.64);

		\path[fill=fillColor] ( 74.31,122.12) rectangle (111.49,140.64);

		\path[fill=fillColor] ( 74.31,122.12) rectangle (111.49,140.64);

		\path[fill=fillColor] ( 74.31,122.12) rectangle (111.49,140.67);

		\path[fill=fillColor] ( 74.31,122.12) rectangle (111.49,140.64);

		\path[fill=fillColor] ( 74.31,122.12) rectangle (111.49,140.67);

		\path[fill=fillColor] ( 74.31,122.12) rectangle (111.49,140.67);

		\path[fill=fillColor] ( 74.31,122.12) rectangle (111.49,140.75);

		\path[fill=fillColor] ( 74.31,122.12) rectangle (111.49,140.75);

		\path[fill=fillColor] ( 74.31,122.12) rectangle (111.49,140.75);

		\path[fill=fillColor] ( 74.31,122.12) rectangle (111.49,140.75);

		\path[fill=fillColor] ( 74.31,122.12) rectangle (111.49,143.01);

		\path[fill=fillColor] ( 74.31,122.12) rectangle (111.49,143.01);

		\path[fill=fillColor] ( 74.31,122.12) rectangle (111.49,143.01);

		\path[fill=fillColor] ( 74.31,122.12) rectangle (111.49,143.01);

		\path[fill=fillColor] ( 74.31,122.12) rectangle (111.49,143.01);

		\path[fill=fillColor] ( 74.31,122.12) rectangle (111.49,143.01);

		\path[fill=fillColor] ( 74.31,122.12) rectangle (111.49,143.01);

		\path[fill=fillColor] ( 74.31,122.12) rectangle (111.49,143.01);

		\path[fill=fillColor] ( 74.31,122.12) rectangle (111.49,139.86);

		\path[fill=fillColor] ( 74.31,122.12) rectangle (111.49,139.76);

		\path[fill=fillColor] ( 74.31,122.12) rectangle (111.49,139.76);

		\path[fill=fillColor] ( 74.31,122.12) rectangle (111.49,139.76);

		\path[fill=fillColor] ( 74.31,122.12) rectangle (111.49,139.76);

		\path[fill=fillColor] ( 74.31,122.12) rectangle (111.49,139.76);

		\path[fill=fillColor] ( 74.31,122.12) rectangle (111.49,139.76);

		\path[fill=fillColor] ( 74.31,122.12) rectangle (111.49,139.76);

		\path[fill=fillColor] ( 74.31,122.12) rectangle (111.49,139.76);

		\path[fill=fillColor] ( 74.31,122.12) rectangle (111.49,139.76);

		\path[fill=fillColor] ( 74.31,122.12) rectangle (111.49,139.76);

		\path[fill=fillColor] ( 74.31,122.12) rectangle (111.49,139.76);

		\path[fill=fillColor] ( 74.31,122.12) rectangle (111.49,139.76);

		\path[fill=fillColor] ( 74.31,122.12) rectangle (111.49,139.76);

		\path[fill=fillColor] ( 74.31,122.12) rectangle (111.49,139.76);

		\path[fill=fillColor] ( 74.31,122.12) rectangle (111.49,139.76);

		\path[fill=fillColor] ( 74.31,122.12) rectangle (111.49,139.76);

		\path[fill=fillColor] ( 74.31,122.12) rectangle (111.49,139.76);

		\path[fill=fillColor] ( 74.31,122.12) rectangle (111.49,139.76);

		\path[fill=fillColor] ( 74.31,122.12) rectangle (111.49,139.76);

		\path[fill=fillColor] ( 74.31,122.12) rectangle (111.49,139.76);

		\path[fill=fillColor] ( 74.31,122.12) rectangle (111.49,139.76);

		\path[fill=fillColor] ( 74.31,122.12) rectangle (111.49,139.76);

		\path[fill=fillColor] ( 74.31,122.12) rectangle (111.49,139.76);

		\path[fill=fillColor] ( 74.31,122.12) rectangle (111.49,139.76);

		\path[fill=fillColor] ( 74.31,122.12) rectangle (111.49,139.76);

		\path[fill=fillColor] ( 74.31,122.12) rectangle (111.49,139.76);

		\path[fill=fillColor] ( 74.31,122.12) rectangle (111.49,139.76);

		\path[fill=fillColor] ( 74.31,122.12) rectangle (111.49,139.76);

		\path[fill=fillColor] ( 74.31,122.12) rectangle (111.49,139.76);

		\path[fill=fillColor] ( 74.31,122.12) rectangle (111.49,139.76);

		\path[fill=fillColor] ( 74.31,122.12) rectangle (111.49,139.76);

		\path[fill=fillColor] ( 74.31,122.12) rectangle (111.49,139.76);

		\path[fill=fillColor] ( 74.31,122.12) rectangle (111.49,139.76);

		\path[fill=fillColor] ( 74.31,122.12) rectangle (111.49,139.76);

		\path[fill=fillColor] ( 74.31,122.12) rectangle (111.49,139.76);

		\path[fill=fillColor] ( 74.31,122.12) rectangle (111.49,139.76);

		\path[fill=fillColor] ( 74.31,122.12) rectangle (111.49,139.76);

		\path[fill=fillColor] ( 74.31,122.12) rectangle (111.49,139.76);

		\path[fill=fillColor] ( 74.31,122.12) rectangle (111.49,139.76);

		\path[fill=fillColor] ( 74.31,122.12) rectangle (111.49,139.76);

		\path[fill=fillColor] ( 74.31,122.12) rectangle (111.49,139.76);

		\path[fill=fillColor] ( 74.31,122.12) rectangle (111.49,139.76);

		\path[fill=fillColor] ( 74.31,122.12) rectangle (111.49,139.76);

		\path[fill=fillColor] ( 74.31,122.12) rectangle (111.49,139.76);

		\path[fill=fillColor] ( 74.31,122.12) rectangle (111.49,139.76);

		\path[fill=fillColor] ( 74.31,122.12) rectangle (111.49,139.76);

		\path[fill=fillColor] ( 74.31,122.12) rectangle (111.49,139.76);

		\path[fill=fillColor] ( 74.31,122.12) rectangle (111.49,139.76);

		\path[fill=fillColor] ( 74.31,122.12) rectangle (111.49,142.86);

		\path[fill=fillColor] ( 74.31,122.12) rectangle (111.49,142.69);

		\path[fill=fillColor] ( 74.31,122.12) rectangle (111.49,139.76);

		\path[fill=fillColor] ( 74.31,122.12) rectangle (111.49,139.76);

		\path[fill=fillColor] ( 74.31,122.12) rectangle (111.49,139.76);

		\path[fill=fillColor] ( 74.31,122.12) rectangle (111.49,139.76);

		\path[fill=fillColor] ( 74.31,122.12) rectangle (111.49,139.76);

		\path[fill=fillColor] ( 74.31,122.12) rectangle (111.49,139.76);

		\path[fill=fillColor] ( 74.31,122.12) rectangle (111.49,139.76);

		\path[fill=fillColor] ( 74.31,122.12) rectangle (111.49,139.76);

		\path[fill=fillColor] ( 74.31,122.12) rectangle (111.49,139.76);

		\path[fill=fillColor] ( 74.31,122.12) rectangle (111.49,139.76);

		\path[fill=fillColor] ( 74.31,122.12) rectangle (111.49,139.76);

		\path[fill=fillColor] ( 74.31,122.12) rectangle (111.49,139.76);

		\path[fill=fillColor] ( 74.31,122.12) rectangle (111.49,139.76);

		\path[fill=fillColor] ( 74.31,122.12) rectangle (111.49,139.76);

		\path[fill=fillColor] ( 74.31,122.12) rectangle (111.49,139.76);

		\path[fill=fillColor] ( 74.31,122.12) rectangle (111.49,141.81);

		\path[fill=fillColor] ( 74.31,122.12) rectangle (111.49,139.76);

		\path[fill=fillColor] ( 74.31,122.12) rectangle (111.49,139.76);

		\path[fill=fillColor] ( 74.31,122.12) rectangle (111.49,139.76);

		\path[fill=fillColor] ( 74.31,122.12) rectangle (111.49,140.58);

		\path[fill=fillColor] ( 74.31,122.12) rectangle (111.49,141.15);

		\path[fill=fillColor] ( 74.31,122.12) rectangle (111.49,141.15);

		\path[fill=fillColor] ( 74.31,122.12) rectangle (111.49,142.87);

		\path[fill=fillColor] ( 74.31,122.12) rectangle (111.49,142.79);

		\path[fill=fillColor] ( 74.31,122.12) rectangle (111.49,143.74);

		\path[fill=fillColor] ( 74.31,122.12) rectangle (111.49,143.74);

		\path[fill=fillColor] ( 74.31,122.12) rectangle (111.49,143.74);

		\path[fill=fillColor] ( 74.31,122.12) rectangle (111.49,143.74);

		\path[fill=fillColor] ( 74.31,122.12) rectangle (111.49,143.74);

		\path[fill=fillColor] ( 74.31,122.12) rectangle (111.49,143.74);

		\path[fill=fillColor] ( 74.31,122.12) rectangle (111.49,143.74);

		\path[fill=fillColor] ( 74.31,122.12) rectangle (111.49,143.74);

		\path[fill=fillColor] ( 74.31,122.12) rectangle (111.49,143.74);

		\path[fill=fillColor] ( 74.31,122.12) rectangle (111.49,143.74);

		\path[fill=fillColor] ( 74.31,122.12) rectangle (111.49,143.74);

		\path[fill=fillColor] ( 74.31,122.12) rectangle (111.49,143.74);

		\path[fill=fillColor] ( 74.31,122.12) rectangle (111.49,143.74);

		\path[fill=fillColor] ( 74.31,122.12) rectangle (111.49,143.74);

		\path[fill=fillColor] ( 74.31,122.12) rectangle (111.49,143.74);

		\path[fill=fillColor] ( 74.31,122.12) rectangle (111.49,143.74);

		\path[fill=fillColor] ( 74.31,122.12) rectangle (111.49,143.74);

		\path[fill=fillColor] ( 74.31,122.12) rectangle (111.49,143.74);

		\path[fill=fillColor] ( 74.31,122.12) rectangle (111.49,144.77);

		\path[fill=fillColor] ( 74.31,122.12) rectangle (111.49,142.79);

		\path[fill=fillColor] ( 74.31,122.12) rectangle (111.49,142.79);

		\path[fill=fillColor] ( 74.31,122.12) rectangle (111.49,144.77);

		\path[fill=fillColor] ( 74.31,122.12) rectangle (111.49,144.77);

		\path[fill=fillColor] ( 74.31,122.12) rectangle (111.49,144.77);

		\path[fill=fillColor] ( 74.31,122.12) rectangle (111.49,144.77);

		\path[fill=fillColor] ( 74.31,122.12) rectangle (111.49,144.77);

		\path[fill=fillColor] ( 74.31,122.12) rectangle (111.49,144.77);

		\path[fill=fillColor] ( 74.31,122.12) rectangle (111.49,144.77);

		\path[fill=fillColor] ( 74.31,122.12) rectangle (111.49,144.77);

		\path[fill=fillColor] ( 74.31,122.12) rectangle (111.49,144.77);

		\path[fill=fillColor] ( 74.31,122.12) rectangle (111.49,144.77);

		\path[fill=fillColor] ( 74.31,122.12) rectangle (111.49,144.77);

		\path[fill=fillColor] ( 74.31,122.12) rectangle (111.49,144.77);

		\path[fill=fillColor] ( 74.31,122.12) rectangle (111.49,144.77);

		\path[fill=fillColor] ( 74.31,122.12) rectangle (111.49,144.77);

		\path[fill=fillColor] ( 74.31,122.12) rectangle (111.49,144.77);

		\path[fill=fillColor] ( 74.31,122.12) rectangle (111.49,144.77);

		\path[fill=fillColor] ( 74.31,122.12) rectangle (111.49,144.77);

		\path[fill=fillColor] ( 74.31,122.12) rectangle (111.49,143.50);

		\path[fill=fillColor] ( 74.31,122.12) rectangle (111.49,144.39);

		\path[fill=fillColor] ( 74.31,122.12) rectangle (111.49,143.01);

		\path[fill=fillColor] ( 74.31,122.12) rectangle (111.49,143.01);

		\path[fill=fillColor] ( 74.31,122.12) rectangle (111.49,143.01);

		\path[fill=fillColor] ( 74.31,122.12) rectangle (111.49,143.01);

		\path[fill=fillColor] ( 74.31,122.12) rectangle (111.49,143.01);

		\path[fill=fillColor] ( 74.31,122.12) rectangle (111.49,143.01);

		\path[fill=fillColor] ( 74.31,122.12) rectangle (111.49,143.01);

		\path[fill=fillColor] ( 74.31,122.12) rectangle (111.49,143.01);

		\path[fill=fillColor] ( 74.31,122.12) rectangle (111.49,143.01);

		\path[fill=fillColor] ( 74.31,122.12) rectangle (111.49,143.01);

		\path[fill=fillColor] ( 74.31,122.12) rectangle (111.49,143.01);

		\path[fill=fillColor] ( 74.31,122.12) rectangle (111.49,143.01);

		\path[fill=fillColor] ( 74.31,122.12) rectangle (111.49,143.01);

		\path[fill=fillColor] ( 74.31,122.12) rectangle (111.49,143.01);

		\path[fill=fillColor] ( 74.31,122.12) rectangle (111.49,143.01);

		\path[fill=fillColor] ( 74.31,122.12) rectangle (111.49,143.01);

		\path[fill=fillColor] ( 74.31,122.12) rectangle (111.49,143.01);

		\path[fill=fillColor] ( 74.31,122.12) rectangle (111.49,143.01);

		\path[fill=fillColor] ( 74.31,122.12) rectangle (111.49,143.01);

		\path[fill=fillColor] ( 74.31,122.12) rectangle (111.49,143.01);

		\path[fill=fillColor] ( 74.31,122.12) rectangle (111.49,143.01);

		\path[fill=fillColor] ( 74.31,122.12) rectangle (111.49,143.01);

		\path[fill=fillColor] ( 74.31,122.12) rectangle (111.49,145.64);

		\path[fill=fillColor] ( 74.31,122.12) rectangle (111.49,143.01);

		\path[fill=fillColor] ( 74.31,122.12) rectangle (111.49,143.01);

		\path[fill=fillColor] ( 74.31,122.12) rectangle (111.49,143.01);

		\path[fill=fillColor] ( 74.31,122.12) rectangle (111.49,143.01);

		\path[fill=fillColor] ( 74.31,122.12) rectangle (111.49,143.01);

		\path[fill=fillColor] ( 74.31,122.12) rectangle (111.49,143.01);

		\path[fill=fillColor] ( 74.31,122.12) rectangle (111.49,143.01);

		\path[fill=fillColor] ( 74.31,122.12) rectangle (111.49,143.01);

		\path[fill=fillColor] ( 74.31,122.12) rectangle (111.49,143.01);

		\path[fill=fillColor] ( 74.31,122.12) rectangle (111.49,143.01);

		\path[fill=fillColor] ( 74.31,122.12) rectangle (111.49,143.01);

		\path[fill=fillColor] ( 74.31,122.12) rectangle (111.49,143.01);

		\path[fill=fillColor] ( 74.31,122.12) rectangle (111.49,143.01);

		\path[fill=fillColor] ( 74.31,122.12) rectangle (111.49,143.01);

		\path[fill=fillColor] ( 74.31,122.12) rectangle (111.49,143.01);

		\path[fill=fillColor] ( 74.31,122.12) rectangle (111.49,143.01);

		\path[fill=fillColor] ( 74.31,122.12) rectangle (111.49,143.01);

		\path[fill=fillColor] ( 74.31,122.12) rectangle (111.49,143.01);

		\path[fill=fillColor] ( 74.31,122.12) rectangle (111.49,143.01);

		\path[fill=fillColor] ( 74.31,122.12) rectangle (111.49,143.01);

		\path[fill=fillColor] ( 74.31,122.12) rectangle (111.49,143.01);

		\path[fill=fillColor] ( 74.31,122.12) rectangle (111.49,144.39);

		\path[fill=fillColor] ( 74.31,122.12) rectangle (111.49,143.68);

		\path[fill=fillColor] ( 74.31,122.12) rectangle (111.49,143.01);

		\path[fill=fillColor] ( 74.31,122.12) rectangle (111.49,143.01);

		\path[fill=fillColor] ( 74.31,122.12) rectangle (111.49,143.01);

		\path[fill=fillColor] ( 74.31,122.12) rectangle (111.49,143.01);

		\path[fill=fillColor] ( 74.31,122.12) rectangle (111.49,143.01);

		\path[fill=fillColor] ( 74.31,122.12) rectangle (111.49,143.01);

		\path[fill=fillColor] ( 74.31,122.12) rectangle (111.49,143.01);

		\path[fill=fillColor] ( 74.31,122.12) rectangle (111.49,143.01);

		\path[fill=fillColor] ( 74.31,122.12) rectangle (111.49,143.01);

		\path[fill=fillColor] ( 74.31,122.12) rectangle (111.49,143.01);

		\path[fill=fillColor] ( 74.31,122.12) rectangle (111.49,143.01);

		\path[fill=fillColor] ( 74.31,122.12) rectangle (111.49,143.01);

		\path[fill=fillColor] ( 74.31,122.12) rectangle (111.49,143.01);

		\path[fill=fillColor] ( 74.31,122.12) rectangle (111.49,143.01);

		\path[fill=fillColor] ( 74.31,122.12) rectangle (111.49,143.01);

		\path[fill=fillColor] ( 74.31,122.12) rectangle (111.49,143.01);

		\path[fill=fillColor] ( 74.31,122.12) rectangle (111.49,143.01);

		\path[fill=fillColor] ( 74.31,122.12) rectangle (111.49,143.01);

		\path[fill=fillColor] ( 74.31,122.12) rectangle (111.49,143.01);

		\path[fill=fillColor] ( 74.31,122.12) rectangle (111.49,143.01);

		\path[fill=fillColor] ( 74.31,122.12) rectangle (111.49,143.01);

		\path[fill=fillColor] ( 74.31,122.12) rectangle (111.49,143.01);

		\path[fill=fillColor] ( 74.31,122.12) rectangle (111.49,143.01);

		\path[fill=fillColor] ( 74.31,122.12) rectangle (111.49,143.01);

		\path[fill=fillColor] ( 74.31,122.12) rectangle (111.49,145.64);

		\path[fill=fillColor] ( 74.31,122.12) rectangle (111.49,145.64);

		\path[fill=fillColor] ( 74.31,122.12) rectangle (111.49,143.01);

		\path[fill=fillColor] ( 74.31,122.12) rectangle (111.49,143.01);

		\path[fill=fillColor] ( 74.31,122.12) rectangle (111.49,143.01);

		\path[fill=fillColor] ( 74.31,122.12) rectangle (111.49,143.01);

		\path[fill=fillColor] ( 74.31,122.12) rectangle (111.49,143.01);

		\path[fill=fillColor] ( 74.31,122.12) rectangle (111.49,143.01);

		\path[fill=fillColor] ( 74.31,122.12) rectangle (111.49,143.01);

		\path[fill=fillColor] ( 74.31,122.12) rectangle (111.49,143.01);

		\path[fill=fillColor] ( 74.31,122.12) rectangle (111.49,145.64);

		\path[fill=fillColor] ( 74.31,122.12) rectangle (111.49,145.64);

		\path[fill=fillColor] ( 74.31,122.12) rectangle (111.49,145.64);

		\path[fill=fillColor] ( 74.31,122.12) rectangle (111.49,145.64);

		\path[fill=fillColor] ( 74.31,122.12) rectangle (111.49,145.64);

		\path[fill=fillColor] ( 74.31,122.12) rectangle (111.49,145.64);

		\path[fill=fillColor] ( 74.31,122.12) rectangle (111.49,145.64);

		\path[fill=fillColor] ( 74.31,122.12) rectangle (111.49,145.64);

		\path[fill=fillColor] ( 74.31,122.12) rectangle (111.49,145.64);

		\path[fill=fillColor] ( 74.31,122.12) rectangle (111.49,145.64);

		\path[fill=fillColor] ( 74.31,122.12) rectangle (111.49,145.64);

		\path[fill=fillColor] ( 74.31,122.12) rectangle (111.49,145.64);

		\path[fill=fillColor] ( 74.31,122.12) rectangle (111.49,145.64);

		\path[fill=fillColor] ( 74.31,122.12) rectangle (111.49,145.64);

		\path[fill=fillColor] ( 74.31,122.12) rectangle (111.49,145.64);

		\path[fill=fillColor] ( 74.31,122.12) rectangle (111.49,145.64);

		\path[fill=fillColor] ( 74.31,122.12) rectangle (111.49,145.64);

		\path[fill=fillColor] ( 74.31,122.12) rectangle (111.49,145.64);

		\path[fill=fillColor] ( 74.31,122.12) rectangle (111.49,145.64);

		\path[fill=fillColor] ( 74.31,122.12) rectangle (111.49,145.64);

		\path[fill=fillColor] ( 74.31,122.12) rectangle (111.49,145.64);

		\path[fill=fillColor] ( 74.31,122.12) rectangle (111.49,145.64);

		\path[fill=fillColor] ( 74.31,122.12) rectangle (111.49,145.64);

		\path[fill=fillColor] ( 74.31,122.12) rectangle (111.49,145.64);

		\path[fill=fillColor] ( 74.31,122.12) rectangle (111.49,145.64);

		\path[fill=fillColor] ( 74.31,122.12) rectangle (111.49,145.64);

		\path[fill=fillColor] ( 74.31,122.12) rectangle (111.49,145.64);

		\path[fill=fillColor] ( 74.31,122.12) rectangle (111.49,145.64);

		\path[fill=fillColor] ( 74.31,122.12) rectangle (111.49,145.64);

		\path[fill=fillColor] ( 74.31,122.12) rectangle (111.49,145.64);

		\path[fill=fillColor] ( 74.31,122.12) rectangle (111.49,145.64);

		\path[fill=fillColor] ( 74.31,122.12) rectangle (111.49,145.64);

		\path[fill=fillColor] ( 74.31,122.12) rectangle (111.49,145.64);

		\path[fill=fillColor] ( 74.31,122.12) rectangle (111.49,145.64);

		\path[fill=fillColor] ( 74.31,122.12) rectangle (111.49,145.64);

		\path[fill=fillColor] ( 74.31,122.12) rectangle (111.49,145.64);

		\path[fill=fillColor] ( 74.31,122.12) rectangle (111.49,145.64);

		\path[fill=fillColor] ( 74.31,122.12) rectangle (111.49,145.64);

		\path[fill=fillColor] ( 74.31,122.12) rectangle (111.49,145.64);

		\path[fill=fillColor] ( 74.31,122.12) rectangle (111.49,145.64);

		\path[fill=fillColor] ( 74.31,122.12) rectangle (111.49,145.64);

		\path[fill=fillColor] ( 74.31,122.12) rectangle (111.49,145.64);

		\path[fill=fillColor] ( 74.31,122.12) rectangle (111.49,145.64);

		\path[fill=fillColor] ( 74.31,122.12) rectangle (111.49,145.64);

		\path[fill=fillColor] ( 74.31,122.12) rectangle (111.49,145.64);

		\path[fill=fillColor] ( 74.31,122.12) rectangle (111.49,145.64);

		\path[fill=fillColor] ( 74.31,122.12) rectangle (111.49,145.64);

		\path[fill=fillColor] ( 74.31,122.12) rectangle (111.49,145.64);

		\path[fill=fillColor] ( 74.31,122.12) rectangle (111.49,145.64);

		\path[fill=fillColor] ( 74.31,122.12) rectangle (111.49,145.64);

		\path[fill=fillColor] ( 74.31,122.12) rectangle (111.49,145.64);

		\path[fill=fillColor] ( 74.31,122.12) rectangle (111.49,145.64);

		\path[fill=fillColor] ( 74.31,122.12) rectangle (111.49,145.64);

		\path[fill=fillColor] ( 74.31,122.12) rectangle (111.49,145.64);

		\path[fill=fillColor] ( 74.31,122.12) rectangle (111.49,145.64);

		\path[fill=fillColor] ( 74.31,122.12) rectangle (111.49,145.64);

		\path[fill=fillColor] ( 74.31,122.12) rectangle (111.49,145.64);

		\path[fill=fillColor] ( 74.31,122.12) rectangle (111.49,145.64);

		\path[fill=fillColor] ( 74.31,122.12) rectangle (111.49,145.64);

		\path[fill=fillColor] ( 74.31,122.12) rectangle (111.49,145.64);

		\path[fill=fillColor] ( 74.31,122.12) rectangle (111.49,145.64);

		\path[fill=fillColor] ( 74.31,122.12) rectangle (111.49,145.64);

		\path[fill=fillColor] ( 74.31,122.12) rectangle (111.49,145.64);

		\path[fill=fillColor] ( 74.31,122.12) rectangle (111.49,145.64);

		\path[fill=fillColor] ( 74.31,122.12) rectangle (111.49,145.64);

		\path[fill=fillColor] ( 74.31,122.12) rectangle (111.49,145.64);

		\path[fill=fillColor] ( 74.31,122.12) rectangle (111.49,145.64);

		\path[fill=fillColor] ( 74.31,122.12) rectangle (111.49,145.64);

		\path[fill=fillColor] ( 74.31,122.12) rectangle (111.49,145.64);

		\path[fill=fillColor] ( 74.31,122.12) rectangle (111.49,145.64);

		\path[fill=fillColor] ( 74.31,122.12) rectangle (111.49,145.64);

		\path[fill=fillColor] ( 74.31,122.12) rectangle (111.49,145.64);

		\path[fill=fillColor] ( 74.31,122.12) rectangle (111.49,145.64);

		\path[fill=fillColor] ( 74.31,122.12) rectangle (111.49,145.64);

		\path[fill=fillColor] ( 74.31,122.12) rectangle (111.49,145.64);

		\path[fill=fillColor] ( 74.31,122.12) rectangle (111.49,145.64);

		\path[fill=fillColor] ( 74.31,122.12) rectangle (111.49,145.64);

		\path[fill=fillColor] ( 74.31,122.12) rectangle (111.49,145.64);

		\path[fill=fillColor] ( 74.31,122.12) rectangle (111.49,145.64);

		\path[fill=fillColor] ( 74.31,122.12) rectangle (111.49,145.64);

		\path[fill=fillColor] ( 74.31,122.12) rectangle (111.49,145.64);

		\path[fill=fillColor] ( 74.31,122.12) rectangle (111.49,145.64);

		\path[fill=fillColor] ( 74.31,122.12) rectangle (111.49,145.64);

		\path[fill=fillColor] ( 74.31,122.12) rectangle (111.49,145.64);

		\path[fill=fillColor] ( 74.31,122.12) rectangle (111.49,145.64);

		\path[fill=fillColor] ( 74.31,122.12) rectangle (111.49,145.64);

		\path[fill=fillColor] ( 74.31,122.12) rectangle (111.49,145.64);

		\path[fill=fillColor] ( 74.31,122.12) rectangle (111.49,145.64);

		\path[fill=fillColor] ( 74.31,122.12) rectangle (111.49,145.64);

		\path[fill=fillColor] ( 74.31,122.12) rectangle (111.49,145.64);

		\path[fill=fillColor] ( 74.31,122.12) rectangle (111.49,145.64);

		\path[fill=fillColor] ( 74.31,122.12) rectangle (111.49,145.64);

		\path[fill=fillColor] ( 74.31,122.12) rectangle (111.49,145.64);

		\path[fill=fillColor] ( 74.31,122.12) rectangle (111.49,145.64);

		\path[fill=fillColor] ( 74.31,122.12) rectangle (111.49,145.64);

		\path[fill=fillColor] ( 74.31,122.12) rectangle (111.49,145.64);

		\path[fill=fillColor] ( 74.31,122.12) rectangle (111.49,145.64);

		\path[fill=fillColor] ( 74.31,122.12) rectangle (111.49,145.64);

		\path[fill=fillColor] ( 74.31,122.12) rectangle (111.49,145.64);

		\path[fill=fillColor] ( 74.31,122.12) rectangle (111.49,145.64);

		\path[fill=fillColor] ( 74.31,122.12) rectangle (111.49,145.64);

		\path[fill=fillColor] ( 74.31,122.12) rectangle (111.49,145.64);

		\path[fill=fillColor] ( 74.31,122.12) rectangle (111.49,145.64);

		\path[fill=fillColor] ( 74.31,122.12) rectangle (111.49,145.64);

		\path[fill=fillColor] ( 74.31,122.12) rectangle (111.49,145.64);

		\path[fill=fillColor] ( 74.31,122.12) rectangle (111.49,145.64);

		\path[fill=fillColor] ( 74.31,122.12) rectangle (111.49,145.64);

		\path[fill=fillColor] ( 74.31,122.12) rectangle (111.49,145.64);

		\path[fill=fillColor] ( 74.31,122.12) rectangle (111.49,145.64);

		\path[fill=fillColor] ( 74.31,122.12) rectangle (111.49,145.64);

		\path[fill=fillColor] ( 74.31,122.12) rectangle (111.49,145.64);

		\path[fill=fillColor] ( 74.31,122.12) rectangle (111.49,145.64);

		\path[fill=fillColor] ( 74.31,122.12) rectangle (111.49,145.64);

		\path[fill=fillColor] ( 74.31,122.12) rectangle (111.49,145.64);

		\path[fill=fillColor] ( 74.31,122.12) rectangle (111.49,145.64);

		\path[fill=fillColor] ( 74.31,122.12) rectangle (111.49,145.64);

		\path[fill=fillColor] ( 74.31,122.12) rectangle (111.49,145.64);

		\path[fill=fillColor] ( 74.31,122.12) rectangle (111.49,145.64);

		\path[fill=fillColor] ( 74.31,122.12) rectangle (111.49,145.64);

		\path[fill=fillColor] ( 74.31,122.12) rectangle (111.49,145.64);

		\path[fill=fillColor] ( 74.31,122.12) rectangle (111.49,145.64);

		\path[fill=fillColor] ( 74.31,122.12) rectangle (111.49,145.64);

		\path[fill=fillColor] ( 74.31,122.12) rectangle (111.49,145.64);

		\path[fill=fillColor] ( 74.31,122.12) rectangle (111.49,145.64);

		\path[fill=fillColor] ( 74.31,122.12) rectangle (111.49,145.64);

		\path[fill=fillColor] ( 74.31,122.12) rectangle (111.49,145.64);

		\path[fill=fillColor] ( 74.31,122.12) rectangle (111.49,145.64);

		\path[fill=fillColor] ( 74.31,122.12) rectangle (111.49,145.64);

		\path[fill=fillColor] ( 74.31,122.12) rectangle (111.49,145.64);

		\path[fill=fillColor] ( 74.31,122.12) rectangle (111.49,145.64);

		\path[fill=fillColor] ( 74.31,122.12) rectangle (111.49,145.64);

		\path[fill=fillColor] ( 74.31,122.12) rectangle (111.49,145.64);

		\path[fill=fillColor] ( 74.31,122.12) rectangle (111.49,145.64);

		\path[fill=fillColor] ( 74.31,122.12) rectangle (111.49,145.64);

		\path[fill=fillColor] ( 74.31,122.12) rectangle (111.49,145.64);

		\path[fill=fillColor] ( 74.31,122.12) rectangle (111.49,145.64);

		\path[fill=fillColor] ( 74.31,122.12) rectangle (111.49,145.64);

		\path[fill=fillColor] ( 74.31,122.12) rectangle (111.49,145.64);

		\path[fill=fillColor] ( 74.31,122.12) rectangle (111.49,145.64);

		\path[fill=fillColor] ( 74.31,122.12) rectangle (111.49,145.64);

		\path[fill=fillColor] ( 74.31,122.12) rectangle (111.49,145.64);

		\path[fill=fillColor] ( 74.31,122.12) rectangle (111.49,145.64);

		\path[fill=fillColor] ( 74.31,122.12) rectangle (111.49,145.64);

		\path[fill=fillColor] ( 74.31,122.12) rectangle (111.49,145.64);

		\path[fill=fillColor] ( 74.31,122.12) rectangle (111.49,145.64);

		\path[fill=fillColor] ( 74.31,122.12) rectangle (111.49,145.64);

		\path[fill=fillColor] ( 74.31,122.12) rectangle (111.49,145.64);

		\path[fill=fillColor] ( 74.31,122.12) rectangle (111.49,145.64);

		\path[fill=fillColor] ( 74.31,122.12) rectangle (111.49,145.64);

		\path[fill=fillColor] ( 74.31,122.12) rectangle (111.49,145.64);

		\path[fill=fillColor] ( 74.31,122.12) rectangle (111.49,145.64);

		\path[fill=fillColor] ( 74.31,122.12) rectangle (111.49,145.64);

		\path[fill=fillColor] ( 74.31,122.12) rectangle (111.49,145.64);

		\path[fill=fillColor] ( 74.31,122.12) rectangle (111.49,145.64);

		\path[fill=fillColor] ( 74.31,122.12) rectangle (111.49,145.64);

		\path[fill=fillColor] ( 74.31,122.12) rectangle (111.49,145.64);

		\path[fill=fillColor] ( 74.31,122.12) rectangle (111.49,143.69);

		\path[fill=fillColor] ( 74.31,122.12) rectangle (111.49,143.69);

		\path[fill=fillColor] ( 74.31,122.12) rectangle (111.49,143.69);

		\path[fill=fillColor] ( 74.31,122.12) rectangle (111.49,143.69);

		\path[fill=fillColor] ( 74.31,122.12) rectangle (111.49,143.69);

		\path[fill=fillColor] ( 74.31,122.12) rectangle (111.49,143.69);

		\path[fill=fillColor] ( 74.31,122.12) rectangle (111.49,143.69);

		\path[fill=fillColor] ( 74.31,122.12) rectangle (111.49,143.69);

		\path[fill=fillColor] ( 74.31,122.12) rectangle (111.49,143.69);

		\path[fill=fillColor] ( 74.31,122.12) rectangle (111.49,143.69);

		\path[fill=fillColor] ( 74.31,122.12) rectangle (111.49,143.69);

		\path[fill=fillColor] ( 74.31,122.12) rectangle (111.49,143.69);

		\path[fill=fillColor] ( 74.31,122.12) rectangle (111.49,143.69);

		\path[fill=fillColor] ( 74.31,122.12) rectangle (111.49,143.69);

		\path[fill=fillColor] ( 74.31,122.12) rectangle (111.49,143.69);

		\path[fill=fillColor] ( 74.31,122.12) rectangle (111.49,143.69);

		\path[fill=fillColor] ( 74.31,122.12) rectangle (111.49,143.69);

		\path[fill=fillColor] ( 74.31,122.12) rectangle (111.49,143.69);

		\path[fill=fillColor] ( 74.31,122.12) rectangle (111.49,143.69);

		\path[fill=fillColor] ( 74.31,122.12) rectangle (111.49,143.69);

		\path[fill=fillColor] ( 74.31,122.12) rectangle (111.49,143.69);

		\path[fill=fillColor] ( 74.31,122.12) rectangle (111.49,143.69);

		\path[fill=fillColor] ( 74.31,122.12) rectangle (111.49,143.69);

		\path[fill=fillColor] ( 74.31,122.12) rectangle (111.49,143.69);

		\path[fill=fillColor] ( 74.31,122.12) rectangle (111.49,143.69);

		\path[fill=fillColor] ( 74.31,122.12) rectangle (111.49,143.69);

		\path[fill=fillColor] ( 74.31,122.12) rectangle (111.49,143.69);

		\path[fill=fillColor] ( 74.31,122.12) rectangle (111.49,143.69);

		\path[fill=fillColor] ( 74.31,122.12) rectangle (111.49,143.69);

		\path[fill=fillColor] ( 74.31,122.12) rectangle (111.49,143.69);

		\path[fill=fillColor] ( 74.31,122.12) rectangle (111.49,143.69);

		\path[fill=fillColor] ( 74.31,122.12) rectangle (111.49,143.69);

		\path[fill=fillColor] ( 74.31,122.12) rectangle (111.49,143.69);

		\path[fill=fillColor] ( 74.31,122.12) rectangle (111.49,143.69);

		\path[fill=fillColor] ( 74.31,122.12) rectangle (111.49,143.69);

		\path[fill=fillColor] ( 74.31,122.12) rectangle (111.49,143.69);

		\path[fill=fillColor] ( 74.31,122.12) rectangle (111.49,143.69);

		\path[fill=fillColor] ( 74.31,122.12) rectangle (111.49,143.69);

		\path[fill=fillColor] ( 74.31,122.12) rectangle (111.49,143.69);

		\path[fill=fillColor] ( 74.31,122.12) rectangle (111.49,143.69);

		\path[fill=fillColor] ( 74.31,122.12) rectangle (111.49,143.69);

		\path[fill=fillColor] ( 74.31,122.12) rectangle (111.49,143.69);

		\path[fill=fillColor] ( 74.31,122.12) rectangle (111.49,143.69);

		\path[fill=fillColor] ( 74.31,122.12) rectangle (111.49,143.69);

		\path[fill=fillColor] ( 74.31,122.12) rectangle (111.49,143.69);

		\path[fill=fillColor] ( 74.31,122.12) rectangle (111.49,143.69);

		\path[fill=fillColor] ( 74.31,122.12) rectangle (111.49,143.69);

		\path[fill=fillColor] ( 74.31,122.12) rectangle (111.49,143.69);

		\path[fill=fillColor] ( 74.31,122.12) rectangle (111.49,143.69);

		\path[fill=fillColor] ( 74.31,122.12) rectangle (111.49,143.69);

		\path[fill=fillColor] ( 74.31,122.12) rectangle (111.49,143.69);

		\path[fill=fillColor] ( 74.31,122.12) rectangle (111.49,143.69);

		\path[fill=fillColor] ( 74.31,122.12) rectangle (111.49,143.69);

		\path[fill=fillColor] ( 74.31,122.12) rectangle (111.49,143.69);

		\path[fill=fillColor] ( 74.31,122.12) rectangle (111.49,143.69);

		\path[fill=fillColor] ( 74.31,122.12) rectangle (111.49,143.69);

		\path[fill=fillColor] ( 74.31,122.12) rectangle (111.49,143.69);

		\path[fill=fillColor] ( 74.31,122.12) rectangle (111.49,143.69);

		\path[fill=fillColor] ( 74.31,122.12) rectangle (111.49,143.69);

		\path[fill=fillColor] ( 74.31,122.12) rectangle (111.49,143.69);

		\path[fill=fillColor] ( 74.31,122.12) rectangle (111.49,143.69);

		\path[fill=fillColor] ( 74.31,122.12) rectangle (111.49,143.69);

		\path[fill=fillColor] ( 74.31,122.12) rectangle (111.49,143.69);

		\path[fill=fillColor] ( 74.31,122.12) rectangle (111.49,143.69);

		\path[fill=fillColor] ( 74.31,122.12) rectangle (111.49,143.69);

		\path[fill=fillColor] ( 74.31,122.12) rectangle (111.49,143.69);

		\path[fill=fillColor] ( 74.31,122.12) rectangle (111.49,143.69);

		\path[fill=fillColor] ( 74.31,122.12) rectangle (111.49,143.69);

		\path[fill=fillColor] ( 74.31,122.12) rectangle (111.49,143.69);

		\path[fill=fillColor] ( 74.31,122.12) rectangle (111.49,143.69);

		\path[fill=fillColor] ( 74.31,122.12) rectangle (111.49,143.69);

		\path[fill=fillColor] ( 74.31,122.12) rectangle (111.49,143.69);

		\path[fill=fillColor] ( 74.31,122.12) rectangle (111.49,143.69);

		\path[fill=fillColor] ( 74.31,122.12) rectangle (111.49,143.69);

		\path[fill=fillColor] ( 74.31,122.12) rectangle (111.49,143.69);

		\path[fill=fillColor] ( 74.31,122.12) rectangle (111.49,143.69);

		\path[fill=fillColor] ( 74.31,122.12) rectangle (111.49,143.69);

		\path[fill=fillColor] ( 74.31,122.12) rectangle (111.49,143.69);

		\path[fill=fillColor] ( 74.31,122.12) rectangle (111.49,143.69);

		\path[fill=fillColor] ( 74.31,122.12) rectangle (111.49,143.69);

		\path[fill=fillColor] ( 74.31,122.12) rectangle (111.49,143.69);

		\path[fill=fillColor] ( 74.31,122.12) rectangle (111.49,143.69);

		\path[fill=fillColor] ( 74.31,122.12) rectangle (111.49,143.69);

		\path[fill=fillColor] ( 74.31,122.12) rectangle (111.49,143.69);

		\path[fill=fillColor] ( 74.31,122.12) rectangle (111.49,143.69);

		\path[fill=fillColor] ( 74.31,122.12) rectangle (111.49,143.69);

		\path[fill=fillColor] ( 74.31,122.12) rectangle (111.49,143.69);

		\path[fill=fillColor] ( 74.31,122.12) rectangle (111.49,143.69);

		\path[fill=fillColor] ( 74.31,122.12) rectangle (111.49,143.69);

		\path[fill=fillColor] ( 74.31,122.12) rectangle (111.49,143.69);

		\path[fill=fillColor] ( 74.31,122.12) rectangle (111.49,143.69);

		\path[fill=fillColor] ( 74.31,122.12) rectangle (111.49,143.69);

		\path[fill=fillColor] ( 74.31,122.12) rectangle (111.49,143.69);

		\path[fill=fillColor] ( 74.31,122.12) rectangle (111.49,143.69);

		\path[fill=fillColor] ( 74.31,122.12) rectangle (111.49,143.69);

		\path[fill=fillColor] ( 74.31,122.12) rectangle (111.49,143.69);

		\path[fill=fillColor] ( 74.31,122.12) rectangle (111.49,143.69);

		\path[fill=fillColor] ( 74.31,122.12) rectangle (111.49,143.69);

		\path[fill=fillColor] ( 74.31,122.12) rectangle (111.49,143.69);

		\path[fill=fillColor] ( 74.31,122.12) rectangle (111.49,143.69);

		\path[fill=fillColor] ( 74.31,122.12) rectangle (111.49,143.69);

		\path[fill=fillColor] ( 74.31,122.12) rectangle (111.49,143.69);

		\path[fill=fillColor] ( 74.31,122.12) rectangle (111.49,143.69);

		\path[fill=fillColor] ( 74.31,122.12) rectangle (111.49,143.69);

		\path[fill=fillColor] ( 74.31,122.12) rectangle (111.49,143.69);

		\path[fill=fillColor] ( 74.31,122.12) rectangle (111.49,143.69);

		\path[fill=fillColor] ( 74.31,122.12) rectangle (111.49,143.69);

		\path[fill=fillColor] ( 74.31,122.12) rectangle (111.49,143.69);

		\path[fill=fillColor] ( 74.31,122.12) rectangle (111.49,143.69);

		\path[fill=fillColor] ( 74.31,122.12) rectangle (111.49,143.69);

		\path[fill=fillColor] ( 74.31,122.12) rectangle (111.49,143.69);

		\path[fill=fillColor] ( 74.31,122.12) rectangle (111.49,143.69);

		\path[fill=fillColor] ( 74.31,122.12) rectangle (111.49,143.69);

		\path[fill=fillColor] ( 74.31,122.12) rectangle (111.49,143.69);

		\path[fill=fillColor] ( 74.31,122.12) rectangle (111.49,143.69);

		\path[fill=fillColor] ( 74.31,122.12) rectangle (111.49,143.69);

		\path[fill=fillColor] ( 74.31,122.12) rectangle (111.49,143.69);

		\path[fill=fillColor] ( 74.31,122.12) rectangle (111.49,143.69);

		\path[fill=fillColor] ( 74.31,122.12) rectangle (111.49,143.69);

		\path[fill=fillColor] ( 74.31,122.12) rectangle (111.49,143.69);

		\path[fill=fillColor] ( 74.31,122.12) rectangle (111.49,143.69);

		\path[fill=fillColor] ( 74.31,122.12) rectangle (111.49,143.69);

		\path[fill=fillColor] ( 74.31,122.12) rectangle (111.49,143.69);

		\path[fill=fillColor] ( 74.31,122.12) rectangle (111.49,143.69);

		\path[fill=fillColor] ( 74.31,122.12) rectangle (111.49,143.69);

		\path[fill=fillColor] ( 74.31,122.12) rectangle (111.49,143.69);

		\path[fill=fillColor] ( 74.31,122.12) rectangle (111.49,143.69);

		\path[fill=fillColor] ( 74.31,122.12) rectangle (111.49,143.69);

		\path[fill=fillColor] ( 74.31,122.12) rectangle (111.49,143.69);

		\path[fill=fillColor] ( 74.31,122.12) rectangle (111.49,143.69);

		\path[fill=fillColor] ( 74.31,122.12) rectangle (111.49,143.69);

		\path[fill=fillColor] ( 74.31,122.12) rectangle (111.49,143.69);

		\path[fill=fillColor] ( 74.31,122.12) rectangle (111.49,143.69);

		\path[fill=fillColor] ( 74.31,122.12) rectangle (111.49,143.69);

		\path[fill=fillColor] ( 74.31,122.12) rectangle (111.49,143.69);

		\path[fill=fillColor] ( 74.31,122.12) rectangle (111.49,143.69);

		\path[fill=fillColor] ( 74.31,122.12) rectangle (111.49,143.69);

		\path[fill=fillColor] ( 74.31,122.12) rectangle (111.49,143.69);

		\path[fill=fillColor] ( 74.31,122.12) rectangle (111.49,143.69);

		\path[fill=fillColor] ( 74.31,122.12) rectangle (111.49,143.69);

		\path[fill=fillColor] ( 74.31,122.12) rectangle (111.49,143.69);

		\path[fill=fillColor] ( 74.31,122.12) rectangle (111.49,143.69);

		\path[fill=fillColor] ( 74.31,122.12) rectangle (111.49,143.69);

		\path[fill=fillColor] ( 74.31,122.12) rectangle (111.49,143.69);

		\path[fill=fillColor] ( 74.31,122.12) rectangle (111.49,143.69);

		\path[fill=fillColor] ( 74.31,122.12) rectangle (111.49,143.69);

		\path[fill=fillColor] ( 74.31,122.12) rectangle (111.49,143.69);

		\path[fill=fillColor] ( 74.31,122.12) rectangle (111.49,143.69);

		\path[fill=fillColor] ( 74.31,122.12) rectangle (111.49,143.69);

		\path[fill=fillColor] ( 74.31,122.12) rectangle (111.49,143.69);

		\path[fill=fillColor] ( 74.31,122.12) rectangle (111.49,143.69);

		\path[fill=fillColor] ( 74.31,122.12) rectangle (111.49,143.69);

		\path[fill=fillColor] ( 74.31,122.12) rectangle (111.49,143.69);

		\path[fill=fillColor] ( 74.31,122.12) rectangle (111.49,143.69);

		\path[fill=fillColor] ( 74.31,122.12) rectangle (111.49,143.69);

		\path[fill=fillColor] ( 74.31,122.12) rectangle (111.49,143.69);

		\path[fill=fillColor] ( 74.31,122.12) rectangle (111.49,143.69);

		\path[fill=fillColor] ( 74.31,122.12) rectangle (111.49,143.69);

		\path[fill=fillColor] ( 74.31,122.12) rectangle (111.49,143.69);

		\path[fill=fillColor] ( 74.31,122.12) rectangle (111.49,143.69);

		\path[fill=fillColor] ( 74.31,122.12) rectangle (111.49,143.69);

		\path[fill=fillColor] ( 74.31,122.12) rectangle (111.49,143.69);

		\path[fill=fillColor] ( 74.31,122.12) rectangle (111.49,143.69);

		\path[fill=fillColor] ( 74.31,122.12) rectangle (111.49,143.69);

		\path[fill=fillColor] ( 74.31,122.12) rectangle (111.49,143.69);

		\path[fill=fillColor] ( 74.31,122.12) rectangle (111.49,143.69);

		\path[fill=fillColor] ( 74.31,122.12) rectangle (111.49,143.69);

		\path[fill=fillColor] ( 74.31,122.12) rectangle (111.49,143.69);

		\path[fill=fillColor] ( 74.31,122.12) rectangle (111.49,143.69);

		\path[fill=fillColor] ( 74.31,122.12) rectangle (111.49,143.69);

		\path[fill=fillColor] ( 74.31,122.12) rectangle (111.49,143.69);

		\path[fill=fillColor] ( 74.31,122.12) rectangle (111.49,143.69);

		\path[fill=fillColor] ( 74.31,122.12) rectangle (111.49,143.69);

		\path[fill=fillColor] ( 74.31,122.12) rectangle (111.49,143.69);

		\path[fill=fillColor] ( 74.31,122.12) rectangle (111.49,143.69);

		\path[fill=fillColor] ( 74.31,122.12) rectangle (111.49,143.69);

		\path[fill=fillColor] ( 74.31,122.12) rectangle (111.49,143.69);

		\path[fill=fillColor] ( 74.31,122.12) rectangle (111.49,143.69);

		\path[fill=fillColor] ( 74.31,122.12) rectangle (111.49,143.69);

		\path[fill=fillColor] ( 74.31,122.12) rectangle (111.49,143.69);

		\path[fill=fillColor] ( 74.31,122.12) rectangle (111.49,143.69);

		\path[fill=fillColor] ( 74.31,122.12) rectangle (111.49,143.69);

		\path[fill=fillColor] ( 74.31,122.12) rectangle (111.49,143.69);

		\path[fill=fillColor] ( 74.31,122.12) rectangle (111.49,143.69);

		\path[fill=fillColor] ( 74.31,122.12) rectangle (111.49,143.69);

		\path[fill=fillColor] ( 74.31,122.12) rectangle (111.49,143.69);

		\path[fill=fillColor] ( 74.31,122.12) rectangle (111.49,143.69);

		\path[fill=fillColor] ( 74.31,122.12) rectangle (111.49,143.69);

		\path[fill=fillColor] ( 74.31,122.12) rectangle (111.49,143.69);

		\path[fill=fillColor] ( 74.31,122.12) rectangle (111.49,143.69);

		\path[fill=fillColor] ( 74.31,122.12) rectangle (111.49,143.69);

		\path[fill=fillColor] ( 74.31,122.12) rectangle (111.49,143.69);

		\path[fill=fillColor] ( 74.31,122.12) rectangle (111.49,143.69);

		\path[fill=fillColor] ( 74.31,122.12) rectangle (111.49,143.69);

		\path[fill=fillColor] ( 74.31,122.12) rectangle (111.49,143.69);

		\path[fill=fillColor] ( 74.31,122.12) rectangle (111.49,143.69);

		\path[fill=fillColor] ( 74.31,122.12) rectangle (111.49,143.69);

		\path[fill=fillColor] ( 74.31,122.12) rectangle (111.49,143.69);

		\path[fill=fillColor] ( 74.31,122.12) rectangle (111.49,143.69);

		\path[fill=fillColor] ( 74.31,122.12) rectangle (111.49,143.69);

		\path[fill=fillColor] ( 74.31,122.12) rectangle (111.49,143.69);

		\path[fill=fillColor] ( 74.31,122.12) rectangle (111.49,143.69);

		\path[fill=fillColor] ( 74.31,122.12) rectangle (111.49,143.69);

		\path[fill=fillColor] ( 74.31,122.12) rectangle (111.49,143.69);

		\path[fill=fillColor] ( 74.31,122.12) rectangle (111.49,143.69);

		\path[fill=fillColor] ( 74.31,122.12) rectangle (111.49,143.69);

		\path[fill=fillColor] ( 74.31,122.12) rectangle (111.49,143.69);

		\path[fill=fillColor] ( 74.31,122.12) rectangle (111.49,143.69);

		\path[fill=fillColor] ( 74.31,122.12) rectangle (111.49,143.69);

		\path[fill=fillColor] ( 74.31,122.12) rectangle (111.49,143.69);

		\path[fill=fillColor] ( 74.31,122.12) rectangle (111.49,143.69);

		\path[fill=fillColor] ( 74.31,122.12) rectangle (111.49,143.69);

		\path[fill=fillColor] ( 74.31,122.12) rectangle (111.49,143.69);

		\path[fill=fillColor] ( 74.31,122.12) rectangle (111.49,143.69);

		\path[fill=fillColor] ( 74.31,122.12) rectangle (111.49,143.69);

		\path[fill=fillColor] ( 74.31,122.12) rectangle (111.49,143.69);

		\path[fill=fillColor] ( 74.31,122.12) rectangle (111.49,143.69);

		\path[fill=fillColor] ( 74.31,122.12) rectangle (111.49,143.69);

		\path[fill=fillColor] ( 74.31,122.12) rectangle (111.49,143.69);

		\path[fill=fillColor] ( 74.31,122.12) rectangle (111.49,143.69);

		\path[fill=fillColor] ( 74.31,122.12) rectangle (111.49,143.69);

		\path[fill=fillColor] ( 74.31,122.12) rectangle (111.49,143.69);

		\path[fill=fillColor] ( 74.31,122.12) rectangle (111.49,143.69);

		\path[fill=fillColor] ( 74.31,122.12) rectangle (111.49,143.69);

		\path[fill=fillColor] ( 74.31,122.12) rectangle (111.49,143.69);

		\path[fill=fillColor] ( 74.31,122.12) rectangle (111.49,143.69);

		\path[fill=fillColor] ( 74.31,122.12) rectangle (111.49,143.69);

		\path[fill=fillColor] ( 74.31,122.12) rectangle (111.49,143.69);

		\path[fill=fillColor] ( 74.31,122.12) rectangle (111.49,143.69);

		\path[fill=fillColor] ( 74.31,122.12) rectangle (111.49,143.69);

		\path[fill=fillColor] ( 74.31,122.12) rectangle (111.49,143.69);

		\path[fill=fillColor] ( 74.31,122.12) rectangle (111.49,143.69);

		\path[fill=fillColor] ( 74.31,122.12) rectangle (111.49,143.69);

		\path[fill=fillColor] ( 74.31,122.12) rectangle (111.49,143.69);

		\path[fill=fillColor] ( 74.31,122.12) rectangle (111.49,143.69);

		\path[fill=fillColor] ( 74.31,122.12) rectangle (111.49,143.69);

		\path[fill=fillColor] ( 74.31,122.12) rectangle (111.49,143.69);

		\path[fill=fillColor] ( 74.31,122.12) rectangle (111.49,143.69);

		\path[fill=fillColor] ( 74.31,122.12) rectangle (111.49,143.69);

		\path[fill=fillColor] ( 74.31,122.12) rectangle (111.49,143.69);

		\path[fill=fillColor] ( 74.31,122.12) rectangle (111.49,143.69);

		\path[fill=fillColor] ( 74.31,122.12) rectangle (111.49,143.69);

		\path[fill=fillColor] ( 74.31,122.12) rectangle (111.49,143.69);

		\path[fill=fillColor] ( 74.31,122.12) rectangle (111.49,143.69);

		\path[fill=fillColor] ( 74.31,122.12) rectangle (111.49,143.69);

		\path[fill=fillColor] ( 74.31,122.12) rectangle (111.49,143.69);

		\path[fill=fillColor] ( 74.31,122.12) rectangle (111.49,143.69);

		\path[fill=fillColor] ( 74.31,122.12) rectangle (111.49,143.69);

		\path[fill=fillColor] ( 74.31,122.12) rectangle (111.49,143.69);

		\path[fill=fillColor] ( 74.31,122.12) rectangle (111.49,143.69);

		\path[fill=fillColor] ( 74.31,122.12) rectangle (111.49,143.69);

		\path[fill=fillColor] ( 74.31,122.12) rectangle (111.49,143.69);

		\path[fill=fillColor] ( 74.31,122.12) rectangle (111.49,143.69);

		\path[fill=fillColor] ( 74.31,122.12) rectangle (111.49,143.69);

		\path[fill=fillColor] ( 74.31,122.12) rectangle (111.49,143.69);

		\path[fill=fillColor] ( 74.31,122.12) rectangle (111.49,143.69);

		\path[fill=fillColor] ( 74.31,122.12) rectangle (111.49,143.69);

		\path[fill=fillColor] ( 74.31,122.12) rectangle (111.49,143.69);

		\path[fill=fillColor] ( 74.31,122.12) rectangle (111.49,143.69);

		\path[fill=fillColor] ( 74.31,122.12) rectangle (111.49,143.69);

		\path[fill=fillColor] ( 74.31,122.12) rectangle (111.49,143.69);

		\path[fill=fillColor] ( 74.31,122.12) rectangle (111.49,143.69);

		\path[fill=fillColor] ( 74.31,122.12) rectangle (111.49,143.69);

		\path[fill=fillColor] ( 74.31,122.12) rectangle (111.49,143.69);

		\path[fill=fillColor] ( 74.31,122.12) rectangle (111.49,143.69);

		\path[fill=fillColor] ( 74.31,122.12) rectangle (111.49,143.69);

		\path[fill=fillColor] ( 74.31,122.12) rectangle (111.49,143.69);

		\path[fill=fillColor] ( 74.31,122.12) rectangle (111.49,143.69);

		\path[fill=fillColor] ( 74.31,122.12) rectangle (111.49,143.69);
		\definecolor{drawColor}{RGB}{255,202,39}

		\path[draw=drawColor,line width= 1.1pt,line join=round] ( 74.31,179.80) --
		( 74.93,125.66) --
		( 75.55,174.75) --
		( 76.17,195.62) --
		( 76.79,195.63) --
		( 77.41,141.56) --
		( 78.03,125.66) --
		( 78.65,134.82) --
		( 79.27,134.82) --
		( 79.89,134.82) --
		( 80.50,134.82) --
		( 81.12,134.47) --
		( 81.74,134.82) --
		( 82.36,134.82) --
		( 82.98,134.47) --
		( 83.60,134.47) --
		( 84.22,134.47) --
		( 84.84,141.29) --
		( 85.46,141.00) --
		( 86.08,135.25) --
		( 86.70,135.25) --
		( 87.32,135.25) --
		( 87.94,135.25) --
		( 88.56,135.25) --
		( 89.18,135.25) --
		( 89.80,135.25) --
		( 90.42,135.25) --
		( 91.04,135.25) --
		( 91.66,135.25) --
		( 92.28,141.72) --
		( 92.90,141.72) --
		( 93.52,141.72) --
		( 94.14,134.82) --
		( 94.76,141.72) --
		( 95.38,141.72) --
		( 96.00,141.72) --
		( 96.62,144.61) --
		( 97.24,144.61) --
		( 97.86,141.72) --
		( 98.48,144.61) --
		( 99.10,144.61) --
		( 99.72,144.61) --
		(100.33,144.61) --
		(100.95,144.61) --
		(101.57,141.72) --
		(102.19,142.23) --
		(102.81,141.29) --
		(103.43,141.29) --
		(104.05,141.29) --
		(104.67,141.29) --
		(105.29,140.89) --
		(105.91,141.29) --
		(106.53,140.33) --
		(107.15,140.33) --
		(107.77,139.83) --
		(108.39,139.83) --
		(109.01,141.72) --
		(109.63,141.72) --
		(110.25,139.83) --
		(110.87,139.83) --
		(111.49,141.72) --
		(112.11,141.72) --
		(112.73,139.83) --
		(113.35,139.83) --
		(113.97,139.83) --
		(114.59,139.83) --
		(115.21,139.83) --
		(115.83,142.09) --
		(116.45,139.83) --
		(117.07,139.83) --
		(117.69,139.83) --
		(118.31,139.83) --
		(118.93,142.04) --
		(119.55,142.18) --
		(120.16,139.71) --
		(120.78,139.71) --
		(121.40,142.18) --
		(122.02,142.18) --
		(122.64,142.18) --
		(123.26,142.09) --
		(123.88,139.10) --
		(124.50,139.10) --
		(125.12,139.69) --
		(125.74,139.69) --
		(126.36,139.69) --
		(126.98,139.69) --
		(127.60,139.69) --
		(128.22,139.69) --
		(128.84,139.69) --
		(129.46,139.69) --
		(130.08,139.69) --
		(130.70,139.71) --
		(131.32,139.71) --
		(131.94,139.71) --
		(132.56,139.71) --
		(133.18,139.71) --
		(133.80,139.71) --
		(134.42,139.71) --
		(135.04,139.71) --
		(135.66,139.71) --
		(136.28,139.75) --
		(136.90,139.71) --
		(137.52,139.75) --
		(138.14,139.75) --
		(138.76,139.83) --
		(139.38,139.83) --
		(139.99,139.83) --
		(140.61,139.83) --
		(141.23,142.09) --
		(141.85,142.09) --
		(142.47,142.09) --
		(143.09,142.09) --
		(143.71,142.09) --
		(144.33,142.09) --
		(144.95,142.09) --
		(145.57,142.09) --
		(146.19,138.94) --
		(146.81,138.84) --
		(147.43,138.84) --
		(148.05,138.84) --
		(148.67,138.84) --
		(149.29,138.84) --
		(149.91,138.84) --
		(150.53,138.84) --
		(151.15,138.84) --
		(151.77,138.84) --
		(152.39,138.84) --
		(153.01,138.84) --
		(153.63,138.84) --
		(154.25,138.84) --
		(154.87,138.84) --
		(155.49,138.84) --
		(156.11,138.84) --
		(156.73,138.84) --
		(157.35,138.84) --
		(157.97,138.84) --
		(158.59,138.84) --
		(159.21,138.84) --
		(159.82,138.84) --
		(160.44,138.84) --
		(161.06,138.84) --
		(161.68,138.84) --
		(162.30,138.84) --
		(162.92,138.84) --
		(163.54,138.84) --
		(164.16,138.84) --
		(164.78,138.84) --
		(165.40,138.84) --
		(166.02,138.84) --
		(166.64,138.84) --
		(167.26,138.84) --
		(167.88,138.84) --
		(168.50,138.84) --
		(169.12,138.84) --
		(169.74,138.84) --
		(170.36,138.84) --
		(170.98,138.84) --
		(171.60,138.84) --
		(172.22,138.84) --
		(172.84,138.84) --
		(173.46,138.84) --
		(174.08,138.84) --
		(174.70,138.84) --
		(175.32,138.84) --
		(175.94,138.84) --
		(176.56,141.94) --
		(177.18,141.77) --
		(177.80,138.84) --
		(178.42,138.84) --
		(179.04,138.84) --
		(179.65,138.84) --
		(180.27,138.84) --
		(180.89,138.84) --
		(181.51,138.84) --
		(182.13,138.84) --
		(182.75,138.84) --
		(183.37,138.84) --
		(183.99,138.84) --
		(184.61,138.84) --
		(185.23,138.84) --
		(185.85,138.84) --
		(186.47,138.84) --
		(187.09,140.89) --
		(187.71,138.84) --
		(188.33,138.84) --
		(188.95,138.84) --
		(189.57,139.66) --
		(190.19,140.23) --
		(190.81,140.23) --
		(191.43,141.95) --
		(192.05,141.87) --
		(192.67,142.82) --
		(193.29,142.82) --
		(193.91,142.82) --
		(194.53,142.82) --
		(195.15,142.82) --
		(195.77,142.82) --
		(196.39,142.82) --
		(197.01,142.82) --
		(197.63,142.82) --
		(198.25,142.82) --
		(198.87,142.82) --
		(199.48,142.82) --
		(200.10,142.82) --
		(200.72,142.82) --
		(201.34,142.82) --
		(201.96,142.82) --
		(202.58,142.82) --
		(203.20,142.82) --
		(203.82,143.85) --
		(204.44,141.87) --
		(205.06,141.87) --
		(205.68,143.85) --
		(206.30,143.85) --
		(206.92,143.85) --
		(207.54,143.85) --
		(208.16,143.85) --
		(208.78,143.85) --
		(209.40,143.85) --
		(210.02,143.85) --
		(210.64,143.85) --
		(211.26,143.85) --
		(211.88,143.85) --
		(212.50,143.85) --
		(213.12,143.85) --
		(213.74,143.85) --
		(214.36,143.85) --
		(214.98,143.85) --
		(215.60,143.85) --
		(216.22,142.58) --
		(216.84,143.47) --
		(217.46,142.09) --
		(218.08,142.09) --
		(218.70,142.09) --
		(219.31,142.09) --
		(219.93,142.09) --
		(220.55,142.09) --
		(221.17,142.09) --
		(221.79,142.09) --
		(222.41,142.09) --
		(223.03,142.09) --
		(223.65,142.09) --
		(224.27,142.09) --
		(224.89,142.09) --
		(225.51,142.09) --
		(226.13,142.09) --
		(226.75,142.09) --
		(227.37,142.09) --
		(227.99,142.09) --
		(228.61,142.09) --
		(229.23,142.09) --
		(229.85,142.09) --
		(230.47,142.09) --
		(231.09,144.72) --
		(231.71,142.09) --
		(232.33,142.09) --
		(232.95,142.09) --
		(233.57,142.09) --
		(234.19,142.09) --
		(234.81,142.09) --
		(235.43,142.09) --
		(236.05,142.09) --
		(236.67,142.09) --
		(237.29,142.09) --
		(237.91,142.09) --
		(238.53,142.09) --
		(239.15,142.09) --
		(239.76,142.09) --
		(240.38,142.09) --
		(241.00,142.09) --
		(241.62,142.09) --
		(242.24,142.09) --
		(242.86,142.09) --
		(243.48,142.09) --
		(244.10,142.09) --
		(244.72,143.47) --
		(245.34,142.76) --
		(245.96,142.09) --
		(246.58,142.09) --
		(247.20,142.09) --
		(247.82,142.09) --
		(248.44,142.09) --
		(249.06,142.09) --
		(249.68,142.09) --
		(250.30,142.09) --
		(250.92,142.09) --
		(251.54,142.09) --
		(252.16,142.09) --
		(252.78,142.09) --
		(253.40,142.09) --
		(254.02,142.09) --
		(254.64,142.09) --
		(255.26,142.09) --
		(255.88,142.09) --
		(256.50,142.09) --
		(257.12,142.09) --
		(257.74,142.09) --
		(258.36,142.09) --
		(258.98,142.09) --
		(259.59,142.09) --
		(260.21,142.09) --
		(260.83,144.72) --
		(261.45,144.72) --
		(262.07,142.09) --
		(262.69,142.09) --
		(263.31,142.09) --
		(263.93,142.09) --
		(264.55,142.09) --
		(265.17,142.09) --
		(265.79,142.09) --
		(266.41,142.09) --
		(267.03,144.72) --
		(267.65,144.72) --
		(268.27,144.72) --
		(268.89,144.72) --
		(269.51,144.72) --
		(270.13,144.72) --
		(270.75,144.72) --
		(271.37,144.72) --
		(271.99,144.72) --
		(272.61,144.72) --
		(273.23,144.72) --
		(273.85,144.72) --
		(274.47,144.72) --
		(275.09,144.72) --
		(275.71,144.72) --
		(276.33,144.72) --
		(276.95,144.72) --
		(277.57,144.72) --
		(278.19,144.72) --
		(278.81,144.72) --
		(279.42,144.72) --
		(280.04,144.72) --
		(280.66,144.72) --
		(281.28,144.72) --
		(281.90,144.72) --
		(282.52,144.72) --
		(283.14,144.72) --
		(283.76,144.72) --
		(284.38,144.72) --
		(285.00,144.72) --
		(285.62,144.72) --
		(286.24,144.72) --
		(286.86,144.72) --
		(287.48,144.72) --
		(288.10,144.72) --
		(288.72,144.72) --
		(289.34,144.72) --
		(289.96,144.72) --
		(290.58,144.72) --
		(291.20,144.72) --
		(291.82,144.72) --
		(292.44,144.72) --
		(293.06,144.72) --
		(293.68,144.72) --
		(294.30,144.72) --
		(294.92,144.72) --
		(295.54,144.72) --
		(296.16,144.72) --
		(296.78,144.72) --
		(297.40,144.72) --
		(298.02,144.72) --
		(298.64,144.72) --
		(299.25,144.72) --
		(299.87,144.72) --
		(300.49,144.72) --
		(301.11,144.72) --
		(301.73,144.72) --
		(302.35,144.72) --
		(302.97,144.72) --
		(303.59,144.72) --
		(304.21,144.72) --
		(304.83,144.72) --
		(305.45,144.72) --
		(306.07,144.72) --
		(306.69,144.72) --
		(307.31,144.72) --
		(307.93,144.72) --
		(308.55,144.72) --
		(309.17,144.72) --
		(309.79,144.72) --
		(310.41,144.72) --
		(311.03,144.72) --
		(311.65,144.72) --
		(312.27,144.72) --
		(312.89,144.72) --
		(313.51,144.72) --
		(314.13,144.72) --
		(314.75,144.72) --
		(315.37,144.72) --
		(315.99,144.72) --
		(316.61,144.72) --
		(317.23,144.72) --
		(317.85,144.72) --
		(318.47,144.72) --
		(319.08,144.72) --
		(319.70,144.72) --
		(320.32,144.72) --
		(320.94,144.72) --
		(321.56,144.72) --
		(322.18,144.72) --
		(322.80,144.72) --
		(323.42,144.72) --
		(324.04,144.72) --
		(324.66,144.72) --
		(325.28,144.72) --
		(325.90,144.72) --
		(326.52,144.72) --
		(327.14,144.72) --
		(327.76,144.72) --
		(328.38,144.72) --
		(329.00,144.72) --
		(329.62,144.72) --
		(330.24,144.72) --
		(330.86,144.72) --
		(331.48,144.72) --
		(332.10,144.72) --
		(332.72,144.72) --
		(333.34,144.72) --
		(333.96,144.72) --
		(334.58,144.72) --
		(335.20,144.72) --
		(335.82,144.72) --
		(336.44,144.72) --
		(337.06,144.72) --
		(337.68,144.72) --
		(338.30,144.72) --
		(338.91,144.72) --
		(339.53,144.72) --
		(340.15,144.72) --
		(340.77,144.72) --
		(341.39,144.72) --
		(342.01,144.72) --
		(342.63,144.72) --
		(343.25,144.72) --
		(343.87,144.72) --
		(344.49,144.72) --
		(345.11,144.72) --
		(345.73,144.72) --
		(346.35,144.72) --
		(346.97,144.72) --
		(347.59,144.72) --
		(348.21,144.72) --
		(348.83,144.72) --
		(349.45,144.72) --
		(350.07,144.72) --
		(350.69,144.72) --
		(351.31,144.72) --
		(351.93,144.72) --
		(352.55,144.72) --
		(353.17,144.72) --
		(353.79,144.72) --
		(354.41,144.72) --
		(355.03,144.72) --
		(355.65,144.72) --
		(356.27,144.72) --
		(356.89,144.72) --
		(357.51,144.72) --
		(358.13,144.72) --
		(358.74,144.72) --
		(359.36,144.72) --
		(359.98,144.72) --
		(360.60,144.72) --
		(361.22,144.72) --
		(361.84,144.72) --
		(362.46,144.72) --
		(363.08,144.72) --
		(363.70,142.77) --
		(364.32,142.77) --
		(364.94,142.77) --
		(365.56,142.77) --
		(366.18,142.77) --
		(366.80,142.77) --
		(367.42,142.77) --
		(368.04,142.77) --
		(368.66,142.77) --
		(369.28,142.77) --
		(369.90,142.77) --
		(370.52,142.77) --
		(371.14,142.77) --
		(371.76,142.77) --
		(372.38,142.77) --
		(373.00,142.77) --
		(373.62,142.77) --
		(374.24,142.77) --
		(374.86,142.77) --
		(375.48,142.77) --
		(376.10,142.77) --
		(376.72,142.77) --
		(377.34,142.77) --
		(377.96,142.77) --
		(378.57,142.77) --
		(379.19,142.77) --
		(379.81,142.77) --
		(380.43,142.77) --
		(381.05,142.77) --
		(381.67,142.77) --
		(382.29,142.77) --
		(382.91,142.77) --
		(383.53,142.77) --
		(384.15,142.77) --
		(384.77,142.77) --
		(385.39,142.77) --
		(386.01,142.77) --
		(386.63,142.77) --
		(387.25,142.77) --
		(387.87,142.77) --
		(388.49,142.77) --
		(389.11,142.77) --
		(389.73,142.77) --
		(390.35,142.77) --
		(390.97,142.77) --
		(391.59,142.77) --
		(392.21,142.77) --
		(392.83,142.77) --
		(393.45,142.77) --
		(394.07,142.77) --
		(394.69,142.77) --
		(395.31,142.77) --
		(395.93,142.77) --
		(396.55,142.77) --
		(397.17,142.77) --
		(397.79,142.77) --
		(398.41,142.77) --
		(399.02,142.77) --
		(399.64,142.77) --
		(400.26,142.77) --
		(400.88,142.77) --
		(401.50,142.77) --
		(402.12,142.77) --
		(402.74,142.77) --
		(403.36,142.77) --
		(403.98,142.77) --
		(404.60,142.77) --
		(405.22,142.77) --
		(405.84,142.77) --
		(406.46,142.77) --
		(407.08,142.77) --
		(407.70,142.77) --
		(408.32,142.77) --
		(408.94,142.77) --
		(409.56,142.77) --
		(410.18,142.77) --
		(410.80,142.77) --
		(411.42,142.77) --
		(412.04,142.77) --
		(412.66,142.77) --
		(413.28,142.77) --
		(413.90,142.77) --
		(414.52,142.77) --
		(415.14,142.77) --
		(415.76,142.77) --
		(416.38,142.77) --
		(417.00,142.77) --
		(417.62,142.77) --
		(418.24,142.77) --
		(418.85,142.77) --
		(419.47,142.77) --
		(420.09,142.77) --
		(420.71,142.77) --
		(421.33,142.77) --
		(421.95,142.77) --
		(422.57,142.77) --
		(423.19,142.77) --
		(423.81,142.77) --
		(424.43,142.77) --
		(425.05,142.77) --
		(425.67,142.77) --
		(426.29,142.77) --
		(426.91,142.77) --
		(427.53,142.77) --
		(428.15,142.77) --
		(428.77,142.77) --
		(429.39,142.77) --
		(430.01,142.77) --
		(430.63,142.77) --
		(431.25,142.77) --
		(431.87,142.77) --
		(432.49,142.77) --
		(433.11,142.77) --
		(433.73,142.77) --
		(434.35,142.77) --
		(434.97,142.77) --
		(435.59,142.77) --
		(436.21,142.77) --
		(436.83,142.77) --
		(437.45,142.77) --
		(438.07,142.77) --
		(438.68,142.77) --
		(439.30,142.77) --
		(439.92,142.77) --
		(440.54,142.77) --
		(441.16,142.77) --
		(441.78,142.77) --
		(442.40,142.77) --
		(443.02,142.77) --
		(443.64,142.77) --
		(444.26,142.77) --
		(444.88,142.77) --
		(445.50,142.77) --
		(446.12,142.77) --
		(446.74,142.77) --
		(447.36,142.77) --
		(447.98,142.77) --
		(448.60,142.77) --
		(449.22,142.77) --
		(449.84,142.77) --
		(450.46,142.77) --
		(451.08,142.77) --
		(451.70,142.77) --
		(452.32,142.77) --
		(452.94,142.77) --
		(453.56,142.77) --
		(454.18,142.77) --
		(454.80,142.77) --
		(455.42,142.77) --
		(456.04,142.77) --
		(456.66,142.77) --
		(457.28,142.77) --
		(457.90,142.77) --
		(458.51,142.77) --
		(459.13,142.77) --
		(459.75,142.77) --
		(460.37,142.77) --
		(460.99,142.77) --
		(461.61,142.77) --
		(462.23,142.77) --
		(462.85,142.77) --
		(463.47,142.77) --
		(464.09,142.77) --
		(464.71,142.77) --
		(465.33,142.77) --
		(465.95,142.77) --
		(466.57,142.77) --
		(467.19,142.77) --
		(467.81,142.77) --
		(468.43,142.77) --
		(469.05,142.77) --
		(469.67,142.77) --
		(470.29,142.77) --
		(470.91,142.77) --
		(471.53,142.77) --
		(472.15,142.77) --
		(472.77,142.77) --
		(473.39,142.77) --
		(474.01,142.77) --
		(474.63,142.77) --
		(475.25,142.77) --
		(475.87,142.77) --
		(476.49,142.77) --
		(477.11,142.77) --
		(477.73,142.77) --
		(478.34,142.77) --
		(478.96,142.77) --
		(479.58,142.77) --
		(480.20,142.77) --
		(480.82,142.77) --
		(481.44,142.77) --
		(482.06,142.77) --
		(482.68,142.77) --
		(483.30,142.77) --
		(483.92,142.77) --
		(484.54,142.77) --
		(485.16,142.77) --
		(485.78,142.77) --
		(486.40,142.77) --
		(487.02,142.77) --
		(487.64,142.77) --
		(488.26,142.77) --
		(488.88,142.77) --
		(489.50,142.77) --
		(490.12,142.77) --
		(490.74,142.77) --
		(491.36,142.77) --
		(491.98,142.77) --
		(492.60,142.77) --
		(493.22,142.77) --
		(493.84,142.77) --
		(494.46,142.77) --
		(495.08,142.77) --
		(495.70,142.77) --
		(496.32,142.77) --
		(496.94,142.77) --
		(497.56,142.77) --
		(498.17,142.77) --
		(498.79,142.77) --
		(499.41,142.77) --
		(500.03,142.77) --
		(500.65,142.77) --
		(501.27,142.77) --
		(501.89,142.77) --
		(502.51,142.77) --
		(503.13,142.77) --
		(503.75,142.77) --
		(504.37,142.77) --
		(504.99,142.77) --
		(505.61,142.77) --
		(506.23,142.77) --
		(506.85,142.77) --
		(507.47,142.77) --
		(508.09,142.77) --
		(508.71,142.77) --
		(509.33,142.77) --
		(509.95,142.77) --
		(510.57,142.77) --
		(511.19,142.77) --
		(511.81,142.77) --
		(512.43,142.77) --
		(513.05,142.77) --
		(513.67,142.77) --
		(514.29,142.77) --
		(514.91,142.77) --
		(515.53,142.77) --
		(516.15,142.77) --
		(516.77,142.77) --
		(517.39,142.77) --
		(518.00,142.77) --
		(518.62,142.77) --
		(519.24,142.77) --
		(519.86,142.77) --
		(520.48,142.77) --
		(521.10,142.77) --
		(521.72,142.77) --
		(522.34,142.77) --
		(522.96,142.77) --
		(523.58,142.77) --
		(524.20,142.77) --
		(524.82,142.77) --
		(525.44,142.77) --
		(526.06,142.77) --
		(526.68,142.77) --
		(527.30,142.77) --
		(527.92,142.77) --
		(528.54,142.77) --
		(529.16,142.77) --
		(529.78,142.77);
	\end{scope}
	\begin{scope}
		\path[clip] ( 51.53, 38.63) rectangle (552.55,116.62);
		\definecolor{drawColor}{gray}{0.92}

		\path[draw=drawColor,line width= 0.3pt,line join=round] ( 51.53, 45.05) --
		(552.55, 45.05);

		\path[draw=drawColor,line width= 0.3pt,line join=round] ( 51.53, 62.36) --
		(552.55, 62.36);

		\path[draw=drawColor,line width= 0.3pt,line join=round] ( 51.53, 79.67) --
		(552.55, 79.67);

		\path[draw=drawColor,line width= 0.3pt,line join=round] ( 51.53, 96.98) --
		(552.55, 96.98);

		\path[draw=drawColor,line width= 0.3pt,line join=round] ( 51.53,114.29) --
		(552.55,114.29);

		\path[draw=drawColor,line width= 0.3pt,line join=round] (133.49, 38.63) --
		(133.49,116.62);

		\path[draw=drawColor,line width= 0.3pt,line join=round] (246.58, 38.63) --
		(246.58,116.62);

		\path[draw=drawColor,line width= 0.3pt,line join=round] (359.98, 38.63) --
		(359.98,116.62);

		\path[draw=drawColor,line width= 0.3pt,line join=round] (473.39, 38.63) --
		(473.39,116.62);

		\path[draw=drawColor,line width= 0.6pt,line join=round] ( 51.53, 53.70) --
		(552.55, 53.70);

		\path[draw=drawColor,line width= 0.6pt,line join=round] ( 51.53, 71.01) --
		(552.55, 71.01);

		\path[draw=drawColor,line width= 0.6pt,line join=round] ( 51.53, 88.32) --
		(552.55, 88.32);

		\path[draw=drawColor,line width= 0.6pt,line join=round] ( 51.53,105.63) --
		(552.55,105.63);

		\path[draw=drawColor,line width= 0.6pt,line join=round] ( 77.41, 38.63) --
		( 77.41,116.62);

		\path[draw=drawColor,line width= 0.6pt,line join=round] (189.57, 38.63) --
		(189.57,116.62);

		\path[draw=drawColor,line width= 0.6pt,line join=round] (303.59, 38.63) --
		(303.59,116.62);

		\path[draw=drawColor,line width= 0.6pt,line join=round] (416.38, 38.63) --
		(416.38,116.62);

		\path[draw=drawColor,line width= 0.6pt,line join=round] (530.40, 38.63) --
		(530.40,116.62);
		\definecolor{fillColor}{gray}{0.93}

		\path[fill=fillColor] ( 74.31, 38.63) rectangle (111.49, 71.00);

		\path[fill=fillColor] ( 74.31, 38.63) rectangle (111.49, 94.40);

		\path[fill=fillColor] ( 74.31, 38.63) rectangle (111.49, 88.89);

		\path[fill=fillColor] ( 74.31, 38.63) rectangle (111.49,113.07);

		\path[fill=fillColor] ( 74.31, 38.63) rectangle (111.49, 96.07);

		\path[fill=fillColor] ( 74.31, 38.63) rectangle (111.49, 56.02);

		\path[fill=fillColor] ( 74.31, 38.63) rectangle (111.49, 83.19);

		\path[fill=fillColor] ( 74.31, 38.63) rectangle (111.49, 63.83);

		\path[fill=fillColor] ( 74.31, 38.63) rectangle (111.49, 63.83);

		\path[fill=fillColor] ( 74.31, 38.63) rectangle (111.49, 63.83);

		\path[fill=fillColor] ( 74.31, 38.63) rectangle (111.49, 63.83);

		\path[fill=fillColor] ( 74.31, 38.63) rectangle (111.49, 78.97);

		\path[fill=fillColor] ( 74.31, 38.63) rectangle (111.49, 63.83);

		\path[fill=fillColor] ( 74.31, 38.63) rectangle (111.49, 63.83);

		\path[fill=fillColor] ( 74.31, 38.63) rectangle (111.49, 78.97);

		\path[fill=fillColor] ( 74.31, 38.63) rectangle (111.49, 78.97);

		\path[fill=fillColor] ( 74.31, 38.63) rectangle (111.49, 78.97);

		\path[fill=fillColor] ( 74.31, 38.63) rectangle (111.49, 74.78);

		\path[fill=fillColor] ( 74.31, 38.63) rectangle (111.49, 74.77);

		\path[fill=fillColor] ( 74.31, 38.63) rectangle (111.49, 43.78);

		\path[fill=fillColor] ( 74.31, 38.63) rectangle (111.49, 43.78);

		\path[fill=fillColor] ( 74.31, 38.63) rectangle (111.49, 43.78);

		\path[fill=fillColor] ( 74.31, 38.63) rectangle (111.49, 43.78);

		\path[fill=fillColor] ( 74.31, 38.63) rectangle (111.49, 43.78);

		\path[fill=fillColor] ( 74.31, 38.63) rectangle (111.49, 43.78);

		\path[fill=fillColor] ( 74.31, 38.63) rectangle (111.49, 43.78);

		\path[fill=fillColor] ( 74.31, 38.63) rectangle (111.49, 43.78);

		\path[fill=fillColor] ( 74.31, 38.63) rectangle (111.49, 43.78);

		\path[fill=fillColor] ( 74.31, 38.63) rectangle (111.49, 43.78);

		\path[fill=fillColor] ( 74.31, 38.63) rectangle (111.49, 74.03);

		\path[fill=fillColor] ( 74.31, 38.63) rectangle (111.49, 74.03);

		\path[fill=fillColor] ( 74.31, 38.63) rectangle (111.49, 74.03);

		\path[fill=fillColor] ( 74.31, 38.63) rectangle (111.49, 63.83);

		\path[fill=fillColor] ( 74.31, 38.63) rectangle (111.49, 74.03);

		\path[fill=fillColor] ( 74.31, 38.63) rectangle (111.49, 74.03);

		\path[fill=fillColor] ( 74.31, 38.63) rectangle (111.49, 74.03);

		\path[fill=fillColor] ( 74.31, 38.63) rectangle (111.49, 43.70);

		\path[fill=fillColor] ( 74.31, 38.63) rectangle (111.49, 43.70);

		\path[fill=fillColor] ( 74.31, 38.63) rectangle (111.49, 74.03);

		\path[fill=fillColor] ( 74.31, 38.63) rectangle (111.49, 43.70);

		\path[fill=fillColor] ( 74.31, 38.63) rectangle (111.49, 43.70);

		\path[fill=fillColor] ( 74.31, 38.63) rectangle (111.49, 43.70);

		\path[fill=fillColor] ( 74.31, 38.63) rectangle (111.49, 43.70);

		\path[fill=fillColor] ( 74.31, 38.63) rectangle (111.49, 43.70);

		\path[fill=fillColor] ( 74.31, 38.63) rectangle (111.49, 74.03);

		\path[fill=fillColor] ( 74.31, 38.63) rectangle (111.49, 91.21);

		\path[fill=fillColor] ( 74.31, 38.63) rectangle (111.49, 74.78);

		\path[fill=fillColor] ( 74.31, 38.63) rectangle (111.49, 74.78);

		\path[fill=fillColor] ( 74.31, 38.63) rectangle (111.49, 74.78);

		\path[fill=fillColor] ( 74.31, 38.63) rectangle (111.49, 74.78);

		\path[fill=fillColor] ( 74.31, 38.63) rectangle (111.49, 88.67);

		\path[fill=fillColor] ( 74.31, 38.63) rectangle (111.49, 74.78);

		\path[fill=fillColor] ( 74.31, 38.63) rectangle (111.49, 87.44);

		\path[fill=fillColor] ( 74.31, 38.63) rectangle (111.49, 87.44);

		\path[fill=fillColor] ( 74.31, 38.63) rectangle (111.49, 94.87);

		\path[fill=fillColor] ( 74.31, 38.63) rectangle (111.49, 94.87);

		\path[fill=fillColor] ( 74.31, 38.63) rectangle (111.49, 74.03);

		\path[fill=fillColor] ( 74.31, 38.63) rectangle (111.49, 74.03);

		\path[fill=fillColor] ( 74.31, 38.63) rectangle (111.49, 94.87);

		\path[fill=fillColor] ( 74.31, 38.63) rectangle (111.49, 94.87);

		\path[fill=fillColor] ( 74.31, 38.63) rectangle (111.49, 74.03);

		\path[fill=fillColor] ( 74.31, 38.63) rectangle (111.49, 74.03);

		\path[fill=fillColor] ( 74.31, 38.63) rectangle (111.49, 94.87);

		\path[fill=fillColor] ( 74.31, 38.63) rectangle (111.49, 94.87);

		\path[fill=fillColor] ( 74.31, 38.63) rectangle (111.49, 94.87);

		\path[fill=fillColor] ( 74.31, 38.63) rectangle (111.49, 94.87);

		\path[fill=fillColor] ( 74.31, 38.63) rectangle (111.49, 94.87);

		\path[fill=fillColor] ( 74.31, 38.63) rectangle (111.49, 92.44);

		\path[fill=fillColor] ( 74.31, 38.63) rectangle (111.49, 94.87);

		\path[fill=fillColor] ( 74.31, 38.63) rectangle (111.49, 94.87);

		\path[fill=fillColor] ( 74.31, 38.63) rectangle (111.49, 94.87);

		\path[fill=fillColor] ( 74.31, 38.63) rectangle (111.49, 94.87);

		\path[fill=fillColor] ( 74.31, 38.63) rectangle (111.49, 98.58);

		\path[fill=fillColor] ( 74.31, 38.63) rectangle (111.49, 98.67);

		\path[fill=fillColor] ( 74.31, 38.63) rectangle (111.49,101.25);

		\path[fill=fillColor] ( 74.31, 38.63) rectangle (111.49,101.25);

		\path[fill=fillColor] ( 74.31, 38.63) rectangle (111.49, 98.67);

		\path[fill=fillColor] ( 74.31, 38.63) rectangle (111.49, 98.67);

		\path[fill=fillColor] ( 74.31, 38.63) rectangle (111.49, 98.67);

		\path[fill=fillColor] ( 74.31, 38.63) rectangle (111.49, 92.44);

		\path[fill=fillColor] ( 74.31, 38.63) rectangle (111.49,108.99);

		\path[fill=fillColor] ( 74.31, 38.63) rectangle (111.49,108.99);

		\path[fill=fillColor] ( 74.31, 38.63) rectangle (111.49,102.76);

		\path[fill=fillColor] ( 74.31, 38.63) rectangle (111.49,102.76);

		\path[fill=fillColor] ( 74.31, 38.63) rectangle (111.49,102.76);

		\path[fill=fillColor] ( 74.31, 38.63) rectangle (111.49,102.76);

		\path[fill=fillColor] ( 74.31, 38.63) rectangle (111.49,102.76);

		\path[fill=fillColor] ( 74.31, 38.63) rectangle (111.49,102.76);

		\path[fill=fillColor] ( 74.31, 38.63) rectangle (111.49,102.76);

		\path[fill=fillColor] ( 74.31, 38.63) rectangle (111.49,102.76);

		\path[fill=fillColor] ( 74.31, 38.63) rectangle (111.49,102.76);

		\path[fill=fillColor] ( 74.31, 38.63) rectangle (111.49,101.25);

		\path[fill=fillColor] ( 74.31, 38.63) rectangle (111.49,101.25);

		\path[fill=fillColor] ( 74.31, 38.63) rectangle (111.49,101.25);

		\path[fill=fillColor] ( 74.31, 38.63) rectangle (111.49,101.25);

		\path[fill=fillColor] ( 74.31, 38.63) rectangle (111.49,101.25);

		\path[fill=fillColor] ( 74.31, 38.63) rectangle (111.49,101.25);

		\path[fill=fillColor] ( 74.31, 38.63) rectangle (111.49,101.25);

		\path[fill=fillColor] ( 74.31, 38.63) rectangle (111.49,101.25);

		\path[fill=fillColor] ( 74.31, 38.63) rectangle (111.49,101.25);

		\path[fill=fillColor] ( 74.31, 38.63) rectangle (111.49,100.55);

		\path[fill=fillColor] ( 74.31, 38.63) rectangle (111.49,101.25);

		\path[fill=fillColor] ( 74.31, 38.63) rectangle (111.49,100.55);

		\path[fill=fillColor] ( 74.31, 38.63) rectangle (111.49,100.55);

		\path[fill=fillColor] ( 74.31, 38.63) rectangle (111.49, 94.87);

		\path[fill=fillColor] ( 74.31, 38.63) rectangle (111.49, 94.87);

		\path[fill=fillColor] ( 74.31, 38.63) rectangle (111.49, 94.87);

		\path[fill=fillColor] ( 74.31, 38.63) rectangle (111.49, 94.87);

		\path[fill=fillColor] ( 74.31, 38.63) rectangle (111.49, 92.44);

		\path[fill=fillColor] ( 74.31, 38.63) rectangle (111.49, 92.44);

		\path[fill=fillColor] ( 74.31, 38.63) rectangle (111.49, 92.44);

		\path[fill=fillColor] ( 74.31, 38.63) rectangle (111.49, 92.44);

		\path[fill=fillColor] ( 74.31, 38.63) rectangle (111.49, 92.44);

		\path[fill=fillColor] ( 74.31, 38.63) rectangle (111.49, 92.44);

		\path[fill=fillColor] ( 74.31, 38.63) rectangle (111.49, 92.44);

		\path[fill=fillColor] ( 74.31, 38.63) rectangle (111.49, 92.44);

		\path[fill=fillColor] ( 74.31, 38.63) rectangle (111.49, 94.89);

		\path[fill=fillColor] ( 74.31, 38.63) rectangle (111.49, 94.58);

		\path[fill=fillColor] ( 74.31, 38.63) rectangle (111.49, 94.58);

		\path[fill=fillColor] ( 74.31, 38.63) rectangle (111.49, 94.58);

		\path[fill=fillColor] ( 74.31, 38.63) rectangle (111.49, 94.58);

		\path[fill=fillColor] ( 74.31, 38.63) rectangle (111.49, 94.58);

		\path[fill=fillColor] ( 74.31, 38.63) rectangle (111.49, 94.58);

		\path[fill=fillColor] ( 74.31, 38.63) rectangle (111.49, 94.58);

		\path[fill=fillColor] ( 74.31, 38.63) rectangle (111.49, 94.58);

		\path[fill=fillColor] ( 74.31, 38.63) rectangle (111.49, 94.58);

		\path[fill=fillColor] ( 74.31, 38.63) rectangle (111.49, 94.58);

		\path[fill=fillColor] ( 74.31, 38.63) rectangle (111.49, 94.58);

		\path[fill=fillColor] ( 74.31, 38.63) rectangle (111.49, 94.58);

		\path[fill=fillColor] ( 74.31, 38.63) rectangle (111.49, 94.58);

		\path[fill=fillColor] ( 74.31, 38.63) rectangle (111.49, 94.58);

		\path[fill=fillColor] ( 74.31, 38.63) rectangle (111.49, 94.58);

		\path[fill=fillColor] ( 74.31, 38.63) rectangle (111.49, 94.58);

		\path[fill=fillColor] ( 74.31, 38.63) rectangle (111.49, 94.58);

		\path[fill=fillColor] ( 74.31, 38.63) rectangle (111.49, 94.58);

		\path[fill=fillColor] ( 74.31, 38.63) rectangle (111.49, 94.58);

		\path[fill=fillColor] ( 74.31, 38.63) rectangle (111.49, 94.58);

		\path[fill=fillColor] ( 74.31, 38.63) rectangle (111.49, 94.58);

		\path[fill=fillColor] ( 74.31, 38.63) rectangle (111.49, 94.58);

		\path[fill=fillColor] ( 74.31, 38.63) rectangle (111.49, 94.58);

		\path[fill=fillColor] ( 74.31, 38.63) rectangle (111.49, 94.58);

		\path[fill=fillColor] ( 74.31, 38.63) rectangle (111.49, 94.58);

		\path[fill=fillColor] ( 74.31, 38.63) rectangle (111.49, 94.58);

		\path[fill=fillColor] ( 74.31, 38.63) rectangle (111.49, 94.58);

		\path[fill=fillColor] ( 74.31, 38.63) rectangle (111.49, 94.58);

		\path[fill=fillColor] ( 74.31, 38.63) rectangle (111.49, 94.58);

		\path[fill=fillColor] ( 74.31, 38.63) rectangle (111.49, 94.58);

		\path[fill=fillColor] ( 74.31, 38.63) rectangle (111.49, 94.58);

		\path[fill=fillColor] ( 74.31, 38.63) rectangle (111.49, 94.58);

		\path[fill=fillColor] ( 74.31, 38.63) rectangle (111.49, 94.58);

		\path[fill=fillColor] ( 74.31, 38.63) rectangle (111.49, 94.58);

		\path[fill=fillColor] ( 74.31, 38.63) rectangle (111.49, 94.58);

		\path[fill=fillColor] ( 74.31, 38.63) rectangle (111.49, 94.58);

		\path[fill=fillColor] ( 74.31, 38.63) rectangle (111.49, 94.58);

		\path[fill=fillColor] ( 74.31, 38.63) rectangle (111.49, 94.58);

		\path[fill=fillColor] ( 74.31, 38.63) rectangle (111.49, 94.58);

		\path[fill=fillColor] ( 74.31, 38.63) rectangle (111.49, 94.58);

		\path[fill=fillColor] ( 74.31, 38.63) rectangle (111.49, 94.58);

		\path[fill=fillColor] ( 74.31, 38.63) rectangle (111.49, 94.58);

		\path[fill=fillColor] ( 74.31, 38.63) rectangle (111.49, 94.58);

		\path[fill=fillColor] ( 74.31, 38.63) rectangle (111.49, 94.58);

		\path[fill=fillColor] ( 74.31, 38.63) rectangle (111.49, 94.58);

		\path[fill=fillColor] ( 74.31, 38.63) rectangle (111.49, 94.58);

		\path[fill=fillColor] ( 74.31, 38.63) rectangle (111.49, 94.58);

		\path[fill=fillColor] ( 74.31, 38.63) rectangle (111.49, 94.58);

		\path[fill=fillColor] ( 74.31, 38.63) rectangle (111.49, 91.45);

		\path[fill=fillColor] ( 74.31, 38.63) rectangle (111.49, 87.26);

		\path[fill=fillColor] ( 74.31, 38.63) rectangle (111.49, 94.58);

		\path[fill=fillColor] ( 74.31, 38.63) rectangle (111.49, 94.58);

		\path[fill=fillColor] ( 74.31, 38.63) rectangle (111.49, 94.58);

		\path[fill=fillColor] ( 74.31, 38.63) rectangle (111.49, 94.58);

		\path[fill=fillColor] ( 74.31, 38.63) rectangle (111.49, 94.58);

		\path[fill=fillColor] ( 74.31, 38.63) rectangle (111.49, 94.58);

		\path[fill=fillColor] ( 74.31, 38.63) rectangle (111.49, 96.53);

		\path[fill=fillColor] ( 74.31, 38.63) rectangle (111.49, 96.53);

		\path[fill=fillColor] ( 74.31, 38.63) rectangle (111.49, 96.53);

		\path[fill=fillColor] ( 74.31, 38.63) rectangle (111.49, 96.53);

		\path[fill=fillColor] ( 74.31, 38.63) rectangle (111.49, 96.53);

		\path[fill=fillColor] ( 74.31, 38.63) rectangle (111.49, 96.53);

		\path[fill=fillColor] ( 74.31, 38.63) rectangle (111.49, 96.53);

		\path[fill=fillColor] ( 74.31, 38.63) rectangle (111.49, 96.53);

		\path[fill=fillColor] ( 74.31, 38.63) rectangle (111.49, 96.53);

		\path[fill=fillColor] ( 74.31, 38.63) rectangle (111.49, 88.67);

		\path[fill=fillColor] ( 74.31, 38.63) rectangle (111.49, 96.53);

		\path[fill=fillColor] ( 74.31, 38.63) rectangle (111.49, 96.53);

		\path[fill=fillColor] ( 74.31, 38.63) rectangle (111.49, 96.53);

		\path[fill=fillColor] ( 74.31, 38.63) rectangle (111.49, 98.90);

		\path[fill=fillColor] ( 74.31, 38.63) rectangle (111.49, 98.96);

		\path[fill=fillColor] ( 74.31, 38.63) rectangle (111.49, 98.96);

		\path[fill=fillColor] ( 74.31, 38.63) rectangle (111.49, 90.41);

		\path[fill=fillColor] ( 74.31, 38.63) rectangle (111.49, 90.36);

		\path[fill=fillColor] ( 74.31, 38.63) rectangle (111.49, 98.62);

		\path[fill=fillColor] ( 74.31, 38.63) rectangle (111.49, 98.62);

		\path[fill=fillColor] ( 74.31, 38.63) rectangle (111.49, 98.62);

		\path[fill=fillColor] ( 74.31, 38.63) rectangle (111.49, 98.62);

		\path[fill=fillColor] ( 74.31, 38.63) rectangle (111.49, 98.62);

		\path[fill=fillColor] ( 74.31, 38.63) rectangle (111.49, 98.62);

		\path[fill=fillColor] ( 74.31, 38.63) rectangle (111.49, 98.62);

		\path[fill=fillColor] ( 74.31, 38.63) rectangle (111.49, 98.62);

		\path[fill=fillColor] ( 74.31, 38.63) rectangle (111.49, 98.62);

		\path[fill=fillColor] ( 74.31, 38.63) rectangle (111.49, 98.62);

		\path[fill=fillColor] ( 74.31, 38.63) rectangle (111.49, 98.62);

		\path[fill=fillColor] ( 74.31, 38.63) rectangle (111.49, 98.62);

		\path[fill=fillColor] ( 74.31, 38.63) rectangle (111.49, 98.62);

		\path[fill=fillColor] ( 74.31, 38.63) rectangle (111.49, 98.62);

		\path[fill=fillColor] ( 74.31, 38.63) rectangle (111.49, 98.62);

		\path[fill=fillColor] ( 74.31, 38.63) rectangle (111.49, 98.62);

		\path[fill=fillColor] ( 74.31, 38.63) rectangle (111.49, 98.62);

		\path[fill=fillColor] ( 74.31, 38.63) rectangle (111.49, 98.62);

		\path[fill=fillColor] ( 74.31, 38.63) rectangle (111.49, 96.09);

		\path[fill=fillColor] ( 74.31, 38.63) rectangle (111.49, 90.36);

		\path[fill=fillColor] ( 74.31, 38.63) rectangle (111.49, 90.36);

		\path[fill=fillColor] ( 74.31, 38.63) rectangle (111.49, 96.09);

		\path[fill=fillColor] ( 74.31, 38.63) rectangle (111.49, 96.09);

		\path[fill=fillColor] ( 74.31, 38.63) rectangle (111.49, 96.09);

		\path[fill=fillColor] ( 74.31, 38.63) rectangle (111.49, 96.09);

		\path[fill=fillColor] ( 74.31, 38.63) rectangle (111.49, 96.09);

		\path[fill=fillColor] ( 74.31, 38.63) rectangle (111.49, 96.09);

		\path[fill=fillColor] ( 74.31, 38.63) rectangle (111.49, 96.09);

		\path[fill=fillColor] ( 74.31, 38.63) rectangle (111.49, 96.09);

		\path[fill=fillColor] ( 74.31, 38.63) rectangle (111.49, 96.09);

		\path[fill=fillColor] ( 74.31, 38.63) rectangle (111.49, 96.09);

		\path[fill=fillColor] ( 74.31, 38.63) rectangle (111.49, 96.09);

		\path[fill=fillColor] ( 74.31, 38.63) rectangle (111.49, 96.09);

		\path[fill=fillColor] ( 74.31, 38.63) rectangle (111.49, 96.09);

		\path[fill=fillColor] ( 74.31, 38.63) rectangle (111.49, 96.09);

		\path[fill=fillColor] ( 74.31, 38.63) rectangle (111.49, 96.09);

		\path[fill=fillColor] ( 74.31, 38.63) rectangle (111.49, 96.09);

		\path[fill=fillColor] ( 74.31, 38.63) rectangle (111.49, 96.09);

		\path[fill=fillColor] ( 74.31, 38.63) rectangle (111.49, 98.59);

		\path[fill=fillColor] ( 74.31, 38.63) rectangle (111.49, 98.47);

		\path[fill=fillColor] ( 74.31, 38.63) rectangle (111.49, 92.44);

		\path[fill=fillColor] ( 74.31, 38.63) rectangle (111.49, 92.44);

		\path[fill=fillColor] ( 74.31, 38.63) rectangle (111.49, 92.44);

		\path[fill=fillColor] ( 74.31, 38.63) rectangle (111.49, 92.44);

		\path[fill=fillColor] ( 74.31, 38.63) rectangle (111.49, 92.44);

		\path[fill=fillColor] ( 74.31, 38.63) rectangle (111.49, 92.44);

		\path[fill=fillColor] ( 74.31, 38.63) rectangle (111.49, 92.44);

		\path[fill=fillColor] ( 74.31, 38.63) rectangle (111.49, 92.44);

		\path[fill=fillColor] ( 74.31, 38.63) rectangle (111.49, 92.44);

		\path[fill=fillColor] ( 74.31, 38.63) rectangle (111.49, 92.44);

		\path[fill=fillColor] ( 74.31, 38.63) rectangle (111.49, 92.44);

		\path[fill=fillColor] ( 74.31, 38.63) rectangle (111.49, 92.44);

		\path[fill=fillColor] ( 74.31, 38.63) rectangle (111.49, 92.44);

		\path[fill=fillColor] ( 74.31, 38.63) rectangle (111.49, 92.44);

		\path[fill=fillColor] ( 74.31, 38.63) rectangle (111.49, 92.44);

		\path[fill=fillColor] ( 74.31, 38.63) rectangle (111.49, 92.44);

		\path[fill=fillColor] ( 74.31, 38.63) rectangle (111.49, 92.44);

		\path[fill=fillColor] ( 74.31, 38.63) rectangle (111.49, 92.44);

		\path[fill=fillColor] ( 74.31, 38.63) rectangle (111.49, 92.44);

		\path[fill=fillColor] ( 74.31, 38.63) rectangle (111.49, 92.44);

		\path[fill=fillColor] ( 74.31, 38.63) rectangle (111.49, 92.44);

		\path[fill=fillColor] ( 74.31, 38.63) rectangle (111.49, 92.44);

		\path[fill=fillColor] ( 74.31, 38.63) rectangle (111.49, 87.74);

		\path[fill=fillColor] ( 74.31, 38.63) rectangle (111.49, 92.44);

		\path[fill=fillColor] ( 74.31, 38.63) rectangle (111.49, 92.44);

		\path[fill=fillColor] ( 74.31, 38.63) rectangle (111.49, 92.44);

		\path[fill=fillColor] ( 74.31, 38.63) rectangle (111.49, 92.44);

		\path[fill=fillColor] ( 74.31, 38.63) rectangle (111.49, 92.44);

		\path[fill=fillColor] ( 74.31, 38.63) rectangle (111.49, 92.44);

		\path[fill=fillColor] ( 74.31, 38.63) rectangle (111.49, 92.44);

		\path[fill=fillColor] ( 74.31, 38.63) rectangle (111.49, 92.44);

		\path[fill=fillColor] ( 74.31, 38.63) rectangle (111.49, 92.44);

		\path[fill=fillColor] ( 74.31, 38.63) rectangle (111.49, 92.44);

		\path[fill=fillColor] ( 74.31, 38.63) rectangle (111.49, 92.44);

		\path[fill=fillColor] ( 74.31, 38.63) rectangle (111.49, 92.44);

		\path[fill=fillColor] ( 74.31, 38.63) rectangle (111.49, 92.44);

		\path[fill=fillColor] ( 74.31, 38.63) rectangle (111.49, 92.44);

		\path[fill=fillColor] ( 74.31, 38.63) rectangle (111.49, 92.44);

		\path[fill=fillColor] ( 74.31, 38.63) rectangle (111.49, 92.44);

		\path[fill=fillColor] ( 74.31, 38.63) rectangle (111.49, 92.44);

		\path[fill=fillColor] ( 74.31, 38.63) rectangle (111.49, 92.44);

		\path[fill=fillColor] ( 74.31, 38.63) rectangle (111.49, 92.44);

		\path[fill=fillColor] ( 74.31, 38.63) rectangle (111.49, 92.44);

		\path[fill=fillColor] ( 74.31, 38.63) rectangle (111.49, 92.44);

		\path[fill=fillColor] ( 74.31, 38.63) rectangle (111.49, 98.47);

		\path[fill=fillColor] ( 74.31, 38.63) rectangle (111.49, 91.38);

		\path[fill=fillColor] ( 74.31, 38.63) rectangle (111.49, 92.44);

		\path[fill=fillColor] ( 74.31, 38.63) rectangle (111.49, 92.44);

		\path[fill=fillColor] ( 74.31, 38.63) rectangle (111.49, 92.44);

		\path[fill=fillColor] ( 74.31, 38.63) rectangle (111.49, 92.44);

		\path[fill=fillColor] ( 74.31, 38.63) rectangle (111.49, 92.44);

		\path[fill=fillColor] ( 74.31, 38.63) rectangle (111.49, 92.44);

		\path[fill=fillColor] ( 74.31, 38.63) rectangle (111.49, 92.44);

		\path[fill=fillColor] ( 74.31, 38.63) rectangle (111.49, 92.44);

		\path[fill=fillColor] ( 74.31, 38.63) rectangle (111.49, 92.44);

		\path[fill=fillColor] ( 74.31, 38.63) rectangle (111.49, 92.44);

		\path[fill=fillColor] ( 74.31, 38.63) rectangle (111.49, 92.44);

		\path[fill=fillColor] ( 74.31, 38.63) rectangle (111.49, 92.44);

		\path[fill=fillColor] ( 74.31, 38.63) rectangle (111.49, 92.44);

		\path[fill=fillColor] ( 74.31, 38.63) rectangle (111.49, 92.44);

		\path[fill=fillColor] ( 74.31, 38.63) rectangle (111.49, 92.44);

		\path[fill=fillColor] ( 74.31, 38.63) rectangle (111.49, 92.44);

		\path[fill=fillColor] ( 74.31, 38.63) rectangle (111.49, 92.44);

		\path[fill=fillColor] ( 74.31, 38.63) rectangle (111.49, 92.44);

		\path[fill=fillColor] ( 74.31, 38.63) rectangle (111.49, 92.44);

		\path[fill=fillColor] ( 74.31, 38.63) rectangle (111.49, 92.44);

		\path[fill=fillColor] ( 74.31, 38.63) rectangle (111.49, 92.44);

		\path[fill=fillColor] ( 74.31, 38.63) rectangle (111.49, 92.44);

		\path[fill=fillColor] ( 74.31, 38.63) rectangle (111.49, 92.44);

		\path[fill=fillColor] ( 74.31, 38.63) rectangle (111.49, 92.44);

		\path[fill=fillColor] ( 74.31, 38.63) rectangle (111.49, 87.74);

		\path[fill=fillColor] ( 74.31, 38.63) rectangle (111.49, 87.74);

		\path[fill=fillColor] ( 74.31, 38.63) rectangle (111.49, 92.44);

		\path[fill=fillColor] ( 74.31, 38.63) rectangle (111.49, 92.44);

		\path[fill=fillColor] ( 74.31, 38.63) rectangle (111.49, 92.44);

		\path[fill=fillColor] ( 74.31, 38.63) rectangle (111.49, 92.44);

		\path[fill=fillColor] ( 74.31, 38.63) rectangle (111.49, 92.44);

		\path[fill=fillColor] ( 74.31, 38.63) rectangle (111.49, 92.44);

		\path[fill=fillColor] ( 74.31, 38.63) rectangle (111.49, 92.44);

		\path[fill=fillColor] ( 74.31, 38.63) rectangle (111.49, 92.44);

		\path[fill=fillColor] ( 74.31, 38.63) rectangle (111.49, 87.74);

		\path[fill=fillColor] ( 74.31, 38.63) rectangle (111.49, 87.74);

		\path[fill=fillColor] ( 74.31, 38.63) rectangle (111.49, 87.74);

		\path[fill=fillColor] ( 74.31, 38.63) rectangle (111.49, 87.74);

		\path[fill=fillColor] ( 74.31, 38.63) rectangle (111.49, 87.74);

		\path[fill=fillColor] ( 74.31, 38.63) rectangle (111.49, 87.74);

		\path[fill=fillColor] ( 74.31, 38.63) rectangle (111.49, 87.74);

		\path[fill=fillColor] ( 74.31, 38.63) rectangle (111.49, 87.74);

		\path[fill=fillColor] ( 74.31, 38.63) rectangle (111.49, 87.74);

		\path[fill=fillColor] ( 74.31, 38.63) rectangle (111.49, 87.74);

		\path[fill=fillColor] ( 74.31, 38.63) rectangle (111.49, 87.74);

		\path[fill=fillColor] ( 74.31, 38.63) rectangle (111.49, 87.74);

		\path[fill=fillColor] ( 74.31, 38.63) rectangle (111.49, 87.74);

		\path[fill=fillColor] ( 74.31, 38.63) rectangle (111.49, 87.74);

		\path[fill=fillColor] ( 74.31, 38.63) rectangle (111.49, 87.74);

		\path[fill=fillColor] ( 74.31, 38.63) rectangle (111.49, 87.74);

		\path[fill=fillColor] ( 74.31, 38.63) rectangle (111.49, 87.74);

		\path[fill=fillColor] ( 74.31, 38.63) rectangle (111.49, 87.74);

		\path[fill=fillColor] ( 74.31, 38.63) rectangle (111.49, 87.74);

		\path[fill=fillColor] ( 74.31, 38.63) rectangle (111.49, 87.74);

		\path[fill=fillColor] ( 74.31, 38.63) rectangle (111.49, 87.74);

		\path[fill=fillColor] ( 74.31, 38.63) rectangle (111.49, 87.74);

		\path[fill=fillColor] ( 74.31, 38.63) rectangle (111.49, 87.74);

		\path[fill=fillColor] ( 74.31, 38.63) rectangle (111.49, 87.74);

		\path[fill=fillColor] ( 74.31, 38.63) rectangle (111.49, 87.74);

		\path[fill=fillColor] ( 74.31, 38.63) rectangle (111.49, 87.74);

		\path[fill=fillColor] ( 74.31, 38.63) rectangle (111.49, 87.74);

		\path[fill=fillColor] ( 74.31, 38.63) rectangle (111.49, 87.74);

		\path[fill=fillColor] ( 74.31, 38.63) rectangle (111.49, 87.74);

		\path[fill=fillColor] ( 74.31, 38.63) rectangle (111.49, 87.74);

		\path[fill=fillColor] ( 74.31, 38.63) rectangle (111.49, 87.74);

		\path[fill=fillColor] ( 74.31, 38.63) rectangle (111.49, 87.74);

		\path[fill=fillColor] ( 74.31, 38.63) rectangle (111.49, 87.74);

		\path[fill=fillColor] ( 74.31, 38.63) rectangle (111.49, 87.74);

		\path[fill=fillColor] ( 74.31, 38.63) rectangle (111.49, 87.74);

		\path[fill=fillColor] ( 74.31, 38.63) rectangle (111.49, 87.74);

		\path[fill=fillColor] ( 74.31, 38.63) rectangle (111.49, 87.74);

		\path[fill=fillColor] ( 74.31, 38.63) rectangle (111.49, 87.74);

		\path[fill=fillColor] ( 74.31, 38.63) rectangle (111.49, 87.74);

		\path[fill=fillColor] ( 74.31, 38.63) rectangle (111.49, 87.74);

		\path[fill=fillColor] ( 74.31, 38.63) rectangle (111.49, 87.74);

		\path[fill=fillColor] ( 74.31, 38.63) rectangle (111.49, 87.74);

		\path[fill=fillColor] ( 74.31, 38.63) rectangle (111.49, 87.74);

		\path[fill=fillColor] ( 74.31, 38.63) rectangle (111.49, 87.74);

		\path[fill=fillColor] ( 74.31, 38.63) rectangle (111.49, 87.74);

		\path[fill=fillColor] ( 74.31, 38.63) rectangle (111.49, 87.74);

		\path[fill=fillColor] ( 74.31, 38.63) rectangle (111.49, 87.74);

		\path[fill=fillColor] ( 74.31, 38.63) rectangle (111.49, 87.74);

		\path[fill=fillColor] ( 74.31, 38.63) rectangle (111.49, 87.74);

		\path[fill=fillColor] ( 74.31, 38.63) rectangle (111.49, 87.74);

		\path[fill=fillColor] ( 74.31, 38.63) rectangle (111.49, 87.74);

		\path[fill=fillColor] ( 74.31, 38.63) rectangle (111.49, 87.74);

		\path[fill=fillColor] ( 74.31, 38.63) rectangle (111.49, 87.74);

		\path[fill=fillColor] ( 74.31, 38.63) rectangle (111.49, 87.74);

		\path[fill=fillColor] ( 74.31, 38.63) rectangle (111.49, 87.74);

		\path[fill=fillColor] ( 74.31, 38.63) rectangle (111.49, 87.74);

		\path[fill=fillColor] ( 74.31, 38.63) rectangle (111.49, 87.74);

		\path[fill=fillColor] ( 74.31, 38.63) rectangle (111.49, 87.74);

		\path[fill=fillColor] ( 74.31, 38.63) rectangle (111.49, 87.74);

		\path[fill=fillColor] ( 74.31, 38.63) rectangle (111.49, 87.74);

		\path[fill=fillColor] ( 74.31, 38.63) rectangle (111.49, 87.74);

		\path[fill=fillColor] ( 74.31, 38.63) rectangle (111.49, 87.74);

		\path[fill=fillColor] ( 74.31, 38.63) rectangle (111.49, 87.74);

		\path[fill=fillColor] ( 74.31, 38.63) rectangle (111.49, 87.74);

		\path[fill=fillColor] ( 74.31, 38.63) rectangle (111.49, 87.74);

		\path[fill=fillColor] ( 74.31, 38.63) rectangle (111.49, 87.74);

		\path[fill=fillColor] ( 74.31, 38.63) rectangle (111.49, 87.74);

		\path[fill=fillColor] ( 74.31, 38.63) rectangle (111.49, 87.74);

		\path[fill=fillColor] ( 74.31, 38.63) rectangle (111.49, 87.74);

		\path[fill=fillColor] ( 74.31, 38.63) rectangle (111.49, 87.74);

		\path[fill=fillColor] ( 74.31, 38.63) rectangle (111.49, 87.74);

		\path[fill=fillColor] ( 74.31, 38.63) rectangle (111.49, 87.74);

		\path[fill=fillColor] ( 74.31, 38.63) rectangle (111.49, 87.74);

		\path[fill=fillColor] ( 74.31, 38.63) rectangle (111.49, 87.74);

		\path[fill=fillColor] ( 74.31, 38.63) rectangle (111.49, 87.74);

		\path[fill=fillColor] ( 74.31, 38.63) rectangle (111.49, 87.74);

		\path[fill=fillColor] ( 74.31, 38.63) rectangle (111.49, 87.74);

		\path[fill=fillColor] ( 74.31, 38.63) rectangle (111.49, 87.74);

		\path[fill=fillColor] ( 74.31, 38.63) rectangle (111.49, 87.74);

		\path[fill=fillColor] ( 74.31, 38.63) rectangle (111.49, 87.74);

		\path[fill=fillColor] ( 74.31, 38.63) rectangle (111.49, 87.74);

		\path[fill=fillColor] ( 74.31, 38.63) rectangle (111.49, 87.74);

		\path[fill=fillColor] ( 74.31, 38.63) rectangle (111.49, 87.74);

		\path[fill=fillColor] ( 74.31, 38.63) rectangle (111.49, 87.74);

		\path[fill=fillColor] ( 74.31, 38.63) rectangle (111.49, 87.74);

		\path[fill=fillColor] ( 74.31, 38.63) rectangle (111.49, 87.74);

		\path[fill=fillColor] ( 74.31, 38.63) rectangle (111.49, 87.74);

		\path[fill=fillColor] ( 74.31, 38.63) rectangle (111.49, 87.74);

		\path[fill=fillColor] ( 74.31, 38.63) rectangle (111.49, 87.74);

		\path[fill=fillColor] ( 74.31, 38.63) rectangle (111.49, 87.74);

		\path[fill=fillColor] ( 74.31, 38.63) rectangle (111.49, 87.74);

		\path[fill=fillColor] ( 74.31, 38.63) rectangle (111.49, 87.74);

		\path[fill=fillColor] ( 74.31, 38.63) rectangle (111.49, 87.74);

		\path[fill=fillColor] ( 74.31, 38.63) rectangle (111.49, 87.74);

		\path[fill=fillColor] ( 74.31, 38.63) rectangle (111.49, 87.74);

		\path[fill=fillColor] ( 74.31, 38.63) rectangle (111.49, 87.74);

		\path[fill=fillColor] ( 74.31, 38.63) rectangle (111.49, 87.74);

		\path[fill=fillColor] ( 74.31, 38.63) rectangle (111.49, 87.74);

		\path[fill=fillColor] ( 74.31, 38.63) rectangle (111.49, 87.74);

		\path[fill=fillColor] ( 74.31, 38.63) rectangle (111.49, 87.74);

		\path[fill=fillColor] ( 74.31, 38.63) rectangle (111.49, 87.74);

		\path[fill=fillColor] ( 74.31, 38.63) rectangle (111.49, 87.74);

		\path[fill=fillColor] ( 74.31, 38.63) rectangle (111.49, 87.74);

		\path[fill=fillColor] ( 74.31, 38.63) rectangle (111.49, 87.74);

		\path[fill=fillColor] ( 74.31, 38.63) rectangle (111.49, 87.74);

		\path[fill=fillColor] ( 74.31, 38.63) rectangle (111.49, 87.74);

		\path[fill=fillColor] ( 74.31, 38.63) rectangle (111.49, 87.74);

		\path[fill=fillColor] ( 74.31, 38.63) rectangle (111.49, 87.74);

		\path[fill=fillColor] ( 74.31, 38.63) rectangle (111.49, 87.74);

		\path[fill=fillColor] ( 74.31, 38.63) rectangle (111.49, 87.74);

		\path[fill=fillColor] ( 74.31, 38.63) rectangle (111.49, 87.74);

		\path[fill=fillColor] ( 74.31, 38.63) rectangle (111.49, 87.74);

		\path[fill=fillColor] ( 74.31, 38.63) rectangle (111.49, 87.74);

		\path[fill=fillColor] ( 74.31, 38.63) rectangle (111.49, 87.74);

		\path[fill=fillColor] ( 74.31, 38.63) rectangle (111.49, 87.74);

		\path[fill=fillColor] ( 74.31, 38.63) rectangle (111.49, 87.74);

		\path[fill=fillColor] ( 74.31, 38.63) rectangle (111.49, 87.74);

		\path[fill=fillColor] ( 74.31, 38.63) rectangle (111.49, 87.74);

		\path[fill=fillColor] ( 74.31, 38.63) rectangle (111.49, 87.74);

		\path[fill=fillColor] ( 74.31, 38.63) rectangle (111.49, 87.74);

		\path[fill=fillColor] ( 74.31, 38.63) rectangle (111.49, 87.74);

		\path[fill=fillColor] ( 74.31, 38.63) rectangle (111.49, 87.74);

		\path[fill=fillColor] ( 74.31, 38.63) rectangle (111.49, 87.74);

		\path[fill=fillColor] ( 74.31, 38.63) rectangle (111.49, 87.74);

		\path[fill=fillColor] ( 74.31, 38.63) rectangle (111.49, 87.74);

		\path[fill=fillColor] ( 74.31, 38.63) rectangle (111.49, 87.74);

		\path[fill=fillColor] ( 74.31, 38.63) rectangle (111.49, 87.74);

		\path[fill=fillColor] ( 74.31, 38.63) rectangle (111.49, 87.74);

		\path[fill=fillColor] ( 74.31, 38.63) rectangle (111.49, 87.74);

		\path[fill=fillColor] ( 74.31, 38.63) rectangle (111.49, 87.74);

		\path[fill=fillColor] ( 74.31, 38.63) rectangle (111.49, 87.74);

		\path[fill=fillColor] ( 74.31, 38.63) rectangle (111.49, 87.74);

		\path[fill=fillColor] ( 74.31, 38.63) rectangle (111.49, 87.74);

		\path[fill=fillColor] ( 74.31, 38.63) rectangle (111.49, 87.74);

		\path[fill=fillColor] ( 74.31, 38.63) rectangle (111.49, 87.74);

		\path[fill=fillColor] ( 74.31, 38.63) rectangle (111.49, 87.74);

		\path[fill=fillColor] ( 74.31, 38.63) rectangle (111.49, 87.74);

		\path[fill=fillColor] ( 74.31, 38.63) rectangle (111.49, 87.74);

		\path[fill=fillColor] ( 74.31, 38.63) rectangle (111.49, 87.74);

		\path[fill=fillColor] ( 74.31, 38.63) rectangle (111.49, 87.74);

		\path[fill=fillColor] ( 74.31, 38.63) rectangle (111.49, 87.74);

		\path[fill=fillColor] ( 74.31, 38.63) rectangle (111.49, 87.74);

		\path[fill=fillColor] ( 74.31, 38.63) rectangle (111.49, 87.74);

		\path[fill=fillColor] ( 74.31, 38.63) rectangle (111.49, 87.74);

		\path[fill=fillColor] ( 74.31, 38.63) rectangle (111.49, 87.74);

		\path[fill=fillColor] ( 74.31, 38.63) rectangle (111.49, 87.74);

		\path[fill=fillColor] ( 74.31, 38.63) rectangle (111.49, 87.74);

		\path[fill=fillColor] ( 74.31, 38.63) rectangle (111.49, 87.74);

		\path[fill=fillColor] ( 74.31, 38.63) rectangle (111.49, 87.74);

		\path[fill=fillColor] ( 74.31, 38.63) rectangle (111.49, 87.74);

		\path[fill=fillColor] ( 74.31, 38.63) rectangle (111.49, 87.74);

		\path[fill=fillColor] ( 74.31, 38.63) rectangle (111.49, 87.74);

		\path[fill=fillColor] ( 74.31, 38.63) rectangle (111.49, 87.74);

		\path[fill=fillColor] ( 74.31, 38.63) rectangle (111.49, 87.74);

		\path[fill=fillColor] ( 74.31, 38.63) rectangle (111.49, 87.74);

		\path[fill=fillColor] ( 74.31, 38.63) rectangle (111.49, 87.74);

		\path[fill=fillColor] ( 74.31, 38.63) rectangle (111.49,101.60);

		\path[fill=fillColor] ( 74.31, 38.63) rectangle (111.49,101.60);

		\path[fill=fillColor] ( 74.31, 38.63) rectangle (111.49,101.60);

		\path[fill=fillColor] ( 74.31, 38.63) rectangle (111.49,101.60);

		\path[fill=fillColor] ( 74.31, 38.63) rectangle (111.49,101.60);

		\path[fill=fillColor] ( 74.31, 38.63) rectangle (111.49,101.60);

		\path[fill=fillColor] ( 74.31, 38.63) rectangle (111.49,101.60);

		\path[fill=fillColor] ( 74.31, 38.63) rectangle (111.49,101.60);

		\path[fill=fillColor] ( 74.31, 38.63) rectangle (111.49,101.60);

		\path[fill=fillColor] ( 74.31, 38.63) rectangle (111.49,101.60);

		\path[fill=fillColor] ( 74.31, 38.63) rectangle (111.49,101.60);

		\path[fill=fillColor] ( 74.31, 38.63) rectangle (111.49,101.60);

		\path[fill=fillColor] ( 74.31, 38.63) rectangle (111.49,101.60);

		\path[fill=fillColor] ( 74.31, 38.63) rectangle (111.49,101.60);

		\path[fill=fillColor] ( 74.31, 38.63) rectangle (111.49,101.60);

		\path[fill=fillColor] ( 74.31, 38.63) rectangle (111.49,101.60);

		\path[fill=fillColor] ( 74.31, 38.63) rectangle (111.49,101.60);

		\path[fill=fillColor] ( 74.31, 38.63) rectangle (111.49,101.60);

		\path[fill=fillColor] ( 74.31, 38.63) rectangle (111.49,101.60);

		\path[fill=fillColor] ( 74.31, 38.63) rectangle (111.49,101.60);

		\path[fill=fillColor] ( 74.31, 38.63) rectangle (111.49,101.60);

		\path[fill=fillColor] ( 74.31, 38.63) rectangle (111.49,101.60);

		\path[fill=fillColor] ( 74.31, 38.63) rectangle (111.49,101.60);

		\path[fill=fillColor] ( 74.31, 38.63) rectangle (111.49,101.60);

		\path[fill=fillColor] ( 74.31, 38.63) rectangle (111.49,101.60);

		\path[fill=fillColor] ( 74.31, 38.63) rectangle (111.49,101.60);

		\path[fill=fillColor] ( 74.31, 38.63) rectangle (111.49,101.60);

		\path[fill=fillColor] ( 74.31, 38.63) rectangle (111.49,101.60);

		\path[fill=fillColor] ( 74.31, 38.63) rectangle (111.49,101.60);

		\path[fill=fillColor] ( 74.31, 38.63) rectangle (111.49,101.60);

		\path[fill=fillColor] ( 74.31, 38.63) rectangle (111.49,101.60);

		\path[fill=fillColor] ( 74.31, 38.63) rectangle (111.49,101.60);

		\path[fill=fillColor] ( 74.31, 38.63) rectangle (111.49,101.60);

		\path[fill=fillColor] ( 74.31, 38.63) rectangle (111.49,101.60);

		\path[fill=fillColor] ( 74.31, 38.63) rectangle (111.49,101.60);

		\path[fill=fillColor] ( 74.31, 38.63) rectangle (111.49,101.60);

		\path[fill=fillColor] ( 74.31, 38.63) rectangle (111.49,101.60);

		\path[fill=fillColor] ( 74.31, 38.63) rectangle (111.49,101.60);

		\path[fill=fillColor] ( 74.31, 38.63) rectangle (111.49,101.60);

		\path[fill=fillColor] ( 74.31, 38.63) rectangle (111.49,101.60);

		\path[fill=fillColor] ( 74.31, 38.63) rectangle (111.49,101.60);

		\path[fill=fillColor] ( 74.31, 38.63) rectangle (111.49,101.60);

		\path[fill=fillColor] ( 74.31, 38.63) rectangle (111.49,101.60);

		\path[fill=fillColor] ( 74.31, 38.63) rectangle (111.49,101.60);

		\path[fill=fillColor] ( 74.31, 38.63) rectangle (111.49,101.60);

		\path[fill=fillColor] ( 74.31, 38.63) rectangle (111.49,101.60);

		\path[fill=fillColor] ( 74.31, 38.63) rectangle (111.49,101.60);

		\path[fill=fillColor] ( 74.31, 38.63) rectangle (111.49,101.60);

		\path[fill=fillColor] ( 74.31, 38.63) rectangle (111.49,101.60);

		\path[fill=fillColor] ( 74.31, 38.63) rectangle (111.49,101.60);

		\path[fill=fillColor] ( 74.31, 38.63) rectangle (111.49,101.60);

		\path[fill=fillColor] ( 74.31, 38.63) rectangle (111.49,101.60);

		\path[fill=fillColor] ( 74.31, 38.63) rectangle (111.49,101.60);

		\path[fill=fillColor] ( 74.31, 38.63) rectangle (111.49,101.60);

		\path[fill=fillColor] ( 74.31, 38.63) rectangle (111.49,101.60);

		\path[fill=fillColor] ( 74.31, 38.63) rectangle (111.49,101.60);

		\path[fill=fillColor] ( 74.31, 38.63) rectangle (111.49,101.60);

		\path[fill=fillColor] ( 74.31, 38.63) rectangle (111.49,101.60);

		\path[fill=fillColor] ( 74.31, 38.63) rectangle (111.49,101.60);

		\path[fill=fillColor] ( 74.31, 38.63) rectangle (111.49,101.60);

		\path[fill=fillColor] ( 74.31, 38.63) rectangle (111.49,101.60);

		\path[fill=fillColor] ( 74.31, 38.63) rectangle (111.49,101.60);

		\path[fill=fillColor] ( 74.31, 38.63) rectangle (111.49,101.60);

		\path[fill=fillColor] ( 74.31, 38.63) rectangle (111.49,101.60);

		\path[fill=fillColor] ( 74.31, 38.63) rectangle (111.49,101.60);

		\path[fill=fillColor] ( 74.31, 38.63) rectangle (111.49,101.60);

		\path[fill=fillColor] ( 74.31, 38.63) rectangle (111.49,101.60);

		\path[fill=fillColor] ( 74.31, 38.63) rectangle (111.49,101.60);

		\path[fill=fillColor] ( 74.31, 38.63) rectangle (111.49,101.60);

		\path[fill=fillColor] ( 74.31, 38.63) rectangle (111.49,101.60);

		\path[fill=fillColor] ( 74.31, 38.63) rectangle (111.49,101.60);

		\path[fill=fillColor] ( 74.31, 38.63) rectangle (111.49,101.60);

		\path[fill=fillColor] ( 74.31, 38.63) rectangle (111.49,101.60);

		\path[fill=fillColor] ( 74.31, 38.63) rectangle (111.49,101.60);

		\path[fill=fillColor] ( 74.31, 38.63) rectangle (111.49,101.60);

		\path[fill=fillColor] ( 74.31, 38.63) rectangle (111.49,101.60);

		\path[fill=fillColor] ( 74.31, 38.63) rectangle (111.49,101.60);

		\path[fill=fillColor] ( 74.31, 38.63) rectangle (111.49,101.60);

		\path[fill=fillColor] ( 74.31, 38.63) rectangle (111.49,101.60);

		\path[fill=fillColor] ( 74.31, 38.63) rectangle (111.49,101.60);

		\path[fill=fillColor] ( 74.31, 38.63) rectangle (111.49,101.60);

		\path[fill=fillColor] ( 74.31, 38.63) rectangle (111.49,101.60);

		\path[fill=fillColor] ( 74.31, 38.63) rectangle (111.49,101.60);

		\path[fill=fillColor] ( 74.31, 38.63) rectangle (111.49,101.60);

		\path[fill=fillColor] ( 74.31, 38.63) rectangle (111.49,101.60);

		\path[fill=fillColor] ( 74.31, 38.63) rectangle (111.49,101.60);

		\path[fill=fillColor] ( 74.31, 38.63) rectangle (111.49,101.60);

		\path[fill=fillColor] ( 74.31, 38.63) rectangle (111.49,101.60);

		\path[fill=fillColor] ( 74.31, 38.63) rectangle (111.49,101.60);

		\path[fill=fillColor] ( 74.31, 38.63) rectangle (111.49,101.60);

		\path[fill=fillColor] ( 74.31, 38.63) rectangle (111.49,101.60);

		\path[fill=fillColor] ( 74.31, 38.63) rectangle (111.49,101.60);

		\path[fill=fillColor] ( 74.31, 38.63) rectangle (111.49,101.60);

		\path[fill=fillColor] ( 74.31, 38.63) rectangle (111.49,101.60);

		\path[fill=fillColor] ( 74.31, 38.63) rectangle (111.49,101.60);

		\path[fill=fillColor] ( 74.31, 38.63) rectangle (111.49,101.60);

		\path[fill=fillColor] ( 74.31, 38.63) rectangle (111.49,101.60);

		\path[fill=fillColor] ( 74.31, 38.63) rectangle (111.49,101.60);

		\path[fill=fillColor] ( 74.31, 38.63) rectangle (111.49,101.60);

		\path[fill=fillColor] ( 74.31, 38.63) rectangle (111.49,101.60);

		\path[fill=fillColor] ( 74.31, 38.63) rectangle (111.49,101.60);

		\path[fill=fillColor] ( 74.31, 38.63) rectangle (111.49,101.60);

		\path[fill=fillColor] ( 74.31, 38.63) rectangle (111.49,101.60);

		\path[fill=fillColor] ( 74.31, 38.63) rectangle (111.49,101.60);

		\path[fill=fillColor] ( 74.31, 38.63) rectangle (111.49,101.60);

		\path[fill=fillColor] ( 74.31, 38.63) rectangle (111.49,101.60);

		\path[fill=fillColor] ( 74.31, 38.63) rectangle (111.49,101.60);

		\path[fill=fillColor] ( 74.31, 38.63) rectangle (111.49,101.60);

		\path[fill=fillColor] ( 74.31, 38.63) rectangle (111.49,101.60);

		\path[fill=fillColor] ( 74.31, 38.63) rectangle (111.49,101.60);

		\path[fill=fillColor] ( 74.31, 38.63) rectangle (111.49,101.60);

		\path[fill=fillColor] ( 74.31, 38.63) rectangle (111.49,101.60);

		\path[fill=fillColor] ( 74.31, 38.63) rectangle (111.49,101.60);

		\path[fill=fillColor] ( 74.31, 38.63) rectangle (111.49,101.60);

		\path[fill=fillColor] ( 74.31, 38.63) rectangle (111.49,101.60);

		\path[fill=fillColor] ( 74.31, 38.63) rectangle (111.49,101.60);

		\path[fill=fillColor] ( 74.31, 38.63) rectangle (111.49,101.60);

		\path[fill=fillColor] ( 74.31, 38.63) rectangle (111.49,101.60);

		\path[fill=fillColor] ( 74.31, 38.63) rectangle (111.49,101.60);

		\path[fill=fillColor] ( 74.31, 38.63) rectangle (111.49,101.60);

		\path[fill=fillColor] ( 74.31, 38.63) rectangle (111.49,101.60);

		\path[fill=fillColor] ( 74.31, 38.63) rectangle (111.49,101.60);

		\path[fill=fillColor] ( 74.31, 38.63) rectangle (111.49,101.60);

		\path[fill=fillColor] ( 74.31, 38.63) rectangle (111.49,101.60);

		\path[fill=fillColor] ( 74.31, 38.63) rectangle (111.49,101.60);

		\path[fill=fillColor] ( 74.31, 38.63) rectangle (111.49,101.60);

		\path[fill=fillColor] ( 74.31, 38.63) rectangle (111.49,101.60);

		\path[fill=fillColor] ( 74.31, 38.63) rectangle (111.49,101.60);

		\path[fill=fillColor] ( 74.31, 38.63) rectangle (111.49,101.60);

		\path[fill=fillColor] ( 74.31, 38.63) rectangle (111.49,101.60);

		\path[fill=fillColor] ( 74.31, 38.63) rectangle (111.49,101.60);

		\path[fill=fillColor] ( 74.31, 38.63) rectangle (111.49,101.60);

		\path[fill=fillColor] ( 74.31, 38.63) rectangle (111.49,101.60);

		\path[fill=fillColor] ( 74.31, 38.63) rectangle (111.49,101.60);

		\path[fill=fillColor] ( 74.31, 38.63) rectangle (111.49,101.60);

		\path[fill=fillColor] ( 74.31, 38.63) rectangle (111.49,101.60);

		\path[fill=fillColor] ( 74.31, 38.63) rectangle (111.49,101.60);

		\path[fill=fillColor] ( 74.31, 38.63) rectangle (111.49,101.60);

		\path[fill=fillColor] ( 74.31, 38.63) rectangle (111.49,101.60);

		\path[fill=fillColor] ( 74.31, 38.63) rectangle (111.49,101.60);

		\path[fill=fillColor] ( 74.31, 38.63) rectangle (111.49,101.60);

		\path[fill=fillColor] ( 74.31, 38.63) rectangle (111.49,101.60);

		\path[fill=fillColor] ( 74.31, 38.63) rectangle (111.49,101.60);

		\path[fill=fillColor] ( 74.31, 38.63) rectangle (111.49,101.60);

		\path[fill=fillColor] ( 74.31, 38.63) rectangle (111.49,101.60);

		\path[fill=fillColor] ( 74.31, 38.63) rectangle (111.49,101.60);

		\path[fill=fillColor] ( 74.31, 38.63) rectangle (111.49,101.60);

		\path[fill=fillColor] ( 74.31, 38.63) rectangle (111.49,101.60);

		\path[fill=fillColor] ( 74.31, 38.63) rectangle (111.49,101.60);

		\path[fill=fillColor] ( 74.31, 38.63) rectangle (111.49,101.60);

		\path[fill=fillColor] ( 74.31, 38.63) rectangle (111.49,101.60);

		\path[fill=fillColor] ( 74.31, 38.63) rectangle (111.49,101.60);

		\path[fill=fillColor] ( 74.31, 38.63) rectangle (111.49,101.60);

		\path[fill=fillColor] ( 74.31, 38.63) rectangle (111.49,101.60);

		\path[fill=fillColor] ( 74.31, 38.63) rectangle (111.49,101.60);

		\path[fill=fillColor] ( 74.31, 38.63) rectangle (111.49,101.60);

		\path[fill=fillColor] ( 74.31, 38.63) rectangle (111.49,101.60);

		\path[fill=fillColor] ( 74.31, 38.63) rectangle (111.49,101.60);

		\path[fill=fillColor] ( 74.31, 38.63) rectangle (111.49,101.60);

		\path[fill=fillColor] ( 74.31, 38.63) rectangle (111.49,101.60);

		\path[fill=fillColor] ( 74.31, 38.63) rectangle (111.49,101.60);

		\path[fill=fillColor] ( 74.31, 38.63) rectangle (111.49,101.60);

		\path[fill=fillColor] ( 74.31, 38.63) rectangle (111.49,101.60);

		\path[fill=fillColor] ( 74.31, 38.63) rectangle (111.49,101.60);

		\path[fill=fillColor] ( 74.31, 38.63) rectangle (111.49,101.60);

		\path[fill=fillColor] ( 74.31, 38.63) rectangle (111.49,101.60);

		\path[fill=fillColor] ( 74.31, 38.63) rectangle (111.49,101.60);

		\path[fill=fillColor] ( 74.31, 38.63) rectangle (111.49,101.60);

		\path[fill=fillColor] ( 74.31, 38.63) rectangle (111.49,101.60);

		\path[fill=fillColor] ( 74.31, 38.63) rectangle (111.49,101.60);

		\path[fill=fillColor] ( 74.31, 38.63) rectangle (111.49,101.60);

		\path[fill=fillColor] ( 74.31, 38.63) rectangle (111.49,101.60);

		\path[fill=fillColor] ( 74.31, 38.63) rectangle (111.49,101.60);

		\path[fill=fillColor] ( 74.31, 38.63) rectangle (111.49,101.60);

		\path[fill=fillColor] ( 74.31, 38.63) rectangle (111.49,101.60);

		\path[fill=fillColor] ( 74.31, 38.63) rectangle (111.49,101.60);

		\path[fill=fillColor] ( 74.31, 38.63) rectangle (111.49,101.60);

		\path[fill=fillColor] ( 74.31, 38.63) rectangle (111.49,101.60);

		\path[fill=fillColor] ( 74.31, 38.63) rectangle (111.49,101.60);

		\path[fill=fillColor] ( 74.31, 38.63) rectangle (111.49,101.60);

		\path[fill=fillColor] ( 74.31, 38.63) rectangle (111.49,101.60);

		\path[fill=fillColor] ( 74.31, 38.63) rectangle (111.49,101.60);

		\path[fill=fillColor] ( 74.31, 38.63) rectangle (111.49,101.60);

		\path[fill=fillColor] ( 74.31, 38.63) rectangle (111.49,101.60);

		\path[fill=fillColor] ( 74.31, 38.63) rectangle (111.49,101.60);

		\path[fill=fillColor] ( 74.31, 38.63) rectangle (111.49,101.60);

		\path[fill=fillColor] ( 74.31, 38.63) rectangle (111.49,101.60);

		\path[fill=fillColor] ( 74.31, 38.63) rectangle (111.49,101.60);

		\path[fill=fillColor] ( 74.31, 38.63) rectangle (111.49,101.60);

		\path[fill=fillColor] ( 74.31, 38.63) rectangle (111.49,101.60);

		\path[fill=fillColor] ( 74.31, 38.63) rectangle (111.49,101.60);

		\path[fill=fillColor] ( 74.31, 38.63) rectangle (111.49,101.60);

		\path[fill=fillColor] ( 74.31, 38.63) rectangle (111.49,101.60);

		\path[fill=fillColor] ( 74.31, 38.63) rectangle (111.49,101.60);

		\path[fill=fillColor] ( 74.31, 38.63) rectangle (111.49,101.60);

		\path[fill=fillColor] ( 74.31, 38.63) rectangle (111.49,101.60);

		\path[fill=fillColor] ( 74.31, 38.63) rectangle (111.49,101.60);

		\path[fill=fillColor] ( 74.31, 38.63) rectangle (111.49,101.60);

		\path[fill=fillColor] ( 74.31, 38.63) rectangle (111.49,101.60);

		\path[fill=fillColor] ( 74.31, 38.63) rectangle (111.49,101.60);

		\path[fill=fillColor] ( 74.31, 38.63) rectangle (111.49,101.60);

		\path[fill=fillColor] ( 74.31, 38.63) rectangle (111.49,101.60);

		\path[fill=fillColor] ( 74.31, 38.63) rectangle (111.49,101.60);

		\path[fill=fillColor] ( 74.31, 38.63) rectangle (111.49,101.60);

		\path[fill=fillColor] ( 74.31, 38.63) rectangle (111.49,101.60);

		\path[fill=fillColor] ( 74.31, 38.63) rectangle (111.49,101.60);

		\path[fill=fillColor] ( 74.31, 38.63) rectangle (111.49,101.60);

		\path[fill=fillColor] ( 74.31, 38.63) rectangle (111.49,101.60);

		\path[fill=fillColor] ( 74.31, 38.63) rectangle (111.49,101.60);

		\path[fill=fillColor] ( 74.31, 38.63) rectangle (111.49,101.60);

		\path[fill=fillColor] ( 74.31, 38.63) rectangle (111.49,101.60);

		\path[fill=fillColor] ( 74.31, 38.63) rectangle (111.49,101.60);

		\path[fill=fillColor] ( 74.31, 38.63) rectangle (111.49,101.60);

		\path[fill=fillColor] ( 74.31, 38.63) rectangle (111.49,101.60);

		\path[fill=fillColor] ( 74.31, 38.63) rectangle (111.49,101.60);

		\path[fill=fillColor] ( 74.31, 38.63) rectangle (111.49,101.60);

		\path[fill=fillColor] ( 74.31, 38.63) rectangle (111.49,101.60);

		\path[fill=fillColor] ( 74.31, 38.63) rectangle (111.49,101.60);

		\path[fill=fillColor] ( 74.31, 38.63) rectangle (111.49,101.60);

		\path[fill=fillColor] ( 74.31, 38.63) rectangle (111.49,101.60);

		\path[fill=fillColor] ( 74.31, 38.63) rectangle (111.49,101.60);

		\path[fill=fillColor] ( 74.31, 38.63) rectangle (111.49,101.60);

		\path[fill=fillColor] ( 74.31, 38.63) rectangle (111.49,101.60);

		\path[fill=fillColor] ( 74.31, 38.63) rectangle (111.49,101.60);

		\path[fill=fillColor] ( 74.31, 38.63) rectangle (111.49,101.60);

		\path[fill=fillColor] ( 74.31, 38.63) rectangle (111.49,101.60);

		\path[fill=fillColor] ( 74.31, 38.63) rectangle (111.49,101.60);

		\path[fill=fillColor] ( 74.31, 38.63) rectangle (111.49,101.60);

		\path[fill=fillColor] ( 74.31, 38.63) rectangle (111.49,101.60);

		\path[fill=fillColor] ( 74.31, 38.63) rectangle (111.49,101.60);

		\path[fill=fillColor] ( 74.31, 38.63) rectangle (111.49,101.60);

		\path[fill=fillColor] ( 74.31, 38.63) rectangle (111.49,101.60);

		\path[fill=fillColor] ( 74.31, 38.63) rectangle (111.49,101.60);

		\path[fill=fillColor] ( 74.31, 38.63) rectangle (111.49,101.60);

		\path[fill=fillColor] ( 74.31, 38.63) rectangle (111.49,101.60);

		\path[fill=fillColor] ( 74.31, 38.63) rectangle (111.49,101.60);

		\path[fill=fillColor] ( 74.31, 38.63) rectangle (111.49,101.60);

		\path[fill=fillColor] ( 74.31, 38.63) rectangle (111.49,101.60);

		\path[fill=fillColor] ( 74.31, 38.63) rectangle (111.49,101.60);

		\path[fill=fillColor] ( 74.31, 38.63) rectangle (111.49,101.60);

		\path[fill=fillColor] ( 74.31, 38.63) rectangle (111.49,101.60);

		\path[fill=fillColor] ( 74.31, 38.63) rectangle (111.49,101.60);

		\path[fill=fillColor] ( 74.31, 38.63) rectangle (111.49,101.60);

		\path[fill=fillColor] ( 74.31, 38.63) rectangle (111.49,101.60);

		\path[fill=fillColor] ( 74.31, 38.63) rectangle (111.49,101.60);

		\path[fill=fillColor] ( 74.31, 38.63) rectangle (111.49,101.60);

		\path[fill=fillColor] ( 74.31, 38.63) rectangle (111.49,101.60);

		\path[fill=fillColor] ( 74.31, 38.63) rectangle (111.49,101.60);

		\path[fill=fillColor] ( 74.31, 38.63) rectangle (111.49,101.60);

		\path[fill=fillColor] ( 74.31, 38.63) rectangle (111.49,101.60);

		\path[fill=fillColor] ( 74.31, 38.63) rectangle (111.49,101.60);

		\path[fill=fillColor] ( 74.31, 38.63) rectangle (111.49,101.60);

		\path[fill=fillColor] ( 74.31, 38.63) rectangle (111.49,101.60);

		\path[fill=fillColor] ( 74.31, 38.63) rectangle (111.49,101.60);

		\path[fill=fillColor] ( 74.31, 38.63) rectangle (111.49,101.60);

		\path[fill=fillColor] ( 74.31, 38.63) rectangle (111.49,101.60);

		\path[fill=fillColor] ( 74.31, 38.63) rectangle (111.49,101.60);

		\path[fill=fillColor] ( 74.31, 38.63) rectangle (111.49,101.60);

		\path[fill=fillColor] ( 74.31, 38.63) rectangle (111.49,101.60);

		\path[fill=fillColor] ( 74.31, 38.63) rectangle (111.49,101.60);

		\path[fill=fillColor] ( 74.31, 38.63) rectangle (111.49,101.60);

		\path[fill=fillColor] ( 74.31, 38.63) rectangle (111.49,101.60);

		\path[fill=fillColor] ( 74.31, 38.63) rectangle (111.49,101.60);

		\path[fill=fillColor] ( 74.31, 38.63) rectangle (111.49,101.60);

		\path[fill=fillColor] ( 74.31, 38.63) rectangle (111.49,101.60);

		\path[fill=fillColor] ( 74.31, 38.63) rectangle (111.49,101.60);

		\path[fill=fillColor] ( 74.31, 38.63) rectangle (111.49,101.60);

		\path[fill=fillColor] ( 74.31, 38.63) rectangle (111.49,101.60);

		\path[fill=fillColor] ( 74.31, 38.63) rectangle (111.49,101.60);
		\definecolor{drawColor}{RGB}{102,186,106}

		\path[draw=drawColor,line width= 1.1pt,line join=round] ( 74.31, 69.49) --
		( 74.93, 92.88) --
		( 75.55, 87.37) --
		( 76.17,111.55) --
		( 76.79, 94.56) --
		( 77.41, 54.50) --
		( 78.03, 81.67) --
		( 78.65, 62.31) --
		( 79.27, 62.31) --
		( 79.89, 62.31) --
		( 80.50, 62.31) --
		( 81.12, 77.45) --
		( 81.74, 62.31) --
		( 82.36, 62.31) --
		( 82.98, 77.45) --
		( 83.60, 77.45) --
		( 84.22, 77.45) --
		( 84.84, 73.26) --
		( 85.46, 73.25) --
		( 86.08, 42.26) --
		( 86.70, 42.26) --
		( 87.32, 42.26) --
		( 87.94, 42.26) --
		( 88.56, 42.26) --
		( 89.18, 42.26) --
		( 89.80, 42.26) --
		( 90.42, 42.26) --
		( 91.04, 42.26) --
		( 91.66, 42.26) --
		( 92.28, 72.51) --
		( 92.90, 72.51) --
		( 93.52, 72.51) --
		( 94.14, 62.31) --
		( 94.76, 72.51) --
		( 95.38, 72.51) --
		( 96.00, 72.51) --
		( 96.62, 42.18) --
		( 97.24, 42.18) --
		( 97.86, 72.51) --
		( 98.48, 42.18) --
		( 99.10, 42.18) --
		( 99.72, 42.18) --
		(100.33, 42.18) --
		(100.95, 42.18) --
		(101.57, 72.51) --
		(102.19, 89.69) --
		(102.81, 73.26) --
		(103.43, 73.26) --
		(104.05, 73.26) --
		(104.67, 73.26) --
		(105.29, 87.15) --
		(105.91, 73.26) --
		(106.53, 85.93) --
		(107.15, 85.93) --
		(107.77, 93.35) --
		(108.39, 93.35) --
		(109.01, 72.51) --
		(109.63, 72.51) --
		(110.25, 93.35) --
		(110.87, 93.35) --
		(111.49, 72.51) --
		(112.11, 72.51) --
		(112.73, 93.35) --
		(113.35, 93.35) --
		(113.97, 93.35) --
		(114.59, 93.35) --
		(115.21, 93.35) --
		(115.83, 90.92) --
		(116.45, 93.35) --
		(117.07, 93.35) --
		(117.69, 93.35) --
		(118.31, 93.35) --
		(118.93, 97.06) --
		(119.55, 97.15) --
		(120.16, 99.74) --
		(120.78, 99.74) --
		(121.40, 97.15) --
		(122.02, 97.15) --
		(122.64, 97.15) --
		(123.26, 90.92) --
		(123.88,107.47) --
		(124.50,107.47) --
		(125.12,101.24) --
		(125.74,101.24) --
		(126.36,101.24) --
		(126.98,101.24) --
		(127.60,101.24) --
		(128.22,101.24) --
		(128.84,101.24) --
		(129.46,101.24) --
		(130.08,101.24) --
		(130.70, 99.74) --
		(131.32, 99.74) --
		(131.94, 99.74) --
		(132.56, 99.74) --
		(133.18, 99.74) --
		(133.80, 99.74) --
		(134.42, 99.74) --
		(135.04, 99.74) --
		(135.66, 99.74) --
		(136.28, 99.03) --
		(136.90, 99.74) --
		(137.52, 99.03) --
		(138.14, 99.03) --
		(138.76, 93.35) --
		(139.38, 93.35) --
		(139.99, 93.35) --
		(140.61, 93.35) --
		(141.23, 90.92) --
		(141.85, 90.92) --
		(142.47, 90.92) --
		(143.09, 90.92) --
		(143.71, 90.92) --
		(144.33, 90.92) --
		(144.95, 90.92) --
		(145.57, 90.92) --
		(146.19, 93.37) --
		(146.81, 93.06) --
		(147.43, 93.06) --
		(148.05, 93.06) --
		(148.67, 93.06) --
		(149.29, 93.06) --
		(149.91, 93.06) --
		(150.53, 93.06) --
		(151.15, 93.06) --
		(151.77, 93.06) --
		(152.39, 93.06) --
		(153.01, 93.06) --
		(153.63, 93.06) --
		(154.25, 93.06) --
		(154.87, 93.06) --
		(155.49, 93.06) --
		(156.11, 93.06) --
		(156.73, 93.06) --
		(157.35, 93.06) --
		(157.97, 93.06) --
		(158.59, 93.06) --
		(159.21, 93.06) --
		(159.82, 93.06) --
		(160.44, 93.06) --
		(161.06, 93.06) --
		(161.68, 93.06) --
		(162.30, 93.06) --
		(162.92, 93.06) --
		(163.54, 93.06) --
		(164.16, 93.06) --
		(164.78, 93.06) --
		(165.40, 93.06) --
		(166.02, 93.06) --
		(166.64, 93.06) --
		(167.26, 93.06) --
		(167.88, 93.06) --
		(168.50, 93.06) --
		(169.12, 93.06) --
		(169.74, 93.06) --
		(170.36, 93.06) --
		(170.98, 93.06) --
		(171.60, 93.06) --
		(172.22, 93.06) --
		(172.84, 93.06) --
		(173.46, 93.06) --
		(174.08, 93.06) --
		(174.70, 93.06) --
		(175.32, 93.06) --
		(175.94, 93.06) --
		(176.56, 89.94) --
		(177.18, 85.74) --
		(177.80, 93.06) --
		(178.42, 93.06) --
		(179.04, 93.06) --
		(179.65, 93.06) --
		(180.27, 93.06) --
		(180.89, 93.06) --
		(181.51, 95.01) --
		(182.13, 95.01) --
		(182.75, 95.01) --
		(183.37, 95.01) --
		(183.99, 95.01) --
		(184.61, 95.01) --
		(185.23, 95.01) --
		(185.85, 95.01) --
		(186.47, 95.01) --
		(187.09, 87.15) --
		(187.71, 95.01) --
		(188.33, 95.01) --
		(188.95, 95.01) --
		(189.57, 97.39) --
		(190.19, 97.44) --
		(190.81, 97.44) --
		(191.43, 88.89) --
		(192.05, 88.84) --
		(192.67, 97.10) --
		(193.29, 97.10) --
		(193.91, 97.10) --
		(194.53, 97.10) --
		(195.15, 97.10) --
		(195.77, 97.10) --
		(196.39, 97.10) --
		(197.01, 97.10) --
		(197.63, 97.10) --
		(198.25, 97.10) --
		(198.87, 97.10) --
		(199.48, 97.10) --
		(200.10, 97.10) --
		(200.72, 97.10) --
		(201.34, 97.10) --
		(201.96, 97.10) --
		(202.58, 97.10) --
		(203.20, 97.10) --
		(203.82, 94.57) --
		(204.44, 88.84) --
		(205.06, 88.84) --
		(205.68, 94.57) --
		(206.30, 94.57) --
		(206.92, 94.57) --
		(207.54, 94.57) --
		(208.16, 94.57) --
		(208.78, 94.57) --
		(209.40, 94.57) --
		(210.02, 94.57) --
		(210.64, 94.57) --
		(211.26, 94.57) --
		(211.88, 94.57) --
		(212.50, 94.57) --
		(213.12, 94.57) --
		(213.74, 94.57) --
		(214.36, 94.57) --
		(214.98, 94.57) --
		(215.60, 94.57) --
		(216.22, 97.07) --
		(216.84, 96.95) --
		(217.46, 90.92) --
		(218.08, 90.92) --
		(218.70, 90.92) --
		(219.31, 90.92) --
		(219.93, 90.92) --
		(220.55, 90.92) --
		(221.17, 90.92) --
		(221.79, 90.92) --
		(222.41, 90.92) --
		(223.03, 90.92) --
		(223.65, 90.92) --
		(224.27, 90.92) --
		(224.89, 90.92) --
		(225.51, 90.92) --
		(226.13, 90.92) --
		(226.75, 90.92) --
		(227.37, 90.92) --
		(227.99, 90.92) --
		(228.61, 90.92) --
		(229.23, 90.92) --
		(229.85, 90.92) --
		(230.47, 90.92) --
		(231.09, 86.22) --
		(231.71, 90.92) --
		(232.33, 90.92) --
		(232.95, 90.92) --
		(233.57, 90.92) --
		(234.19, 90.92) --
		(234.81, 90.92) --
		(235.43, 90.92) --
		(236.05, 90.92) --
		(236.67, 90.92) --
		(237.29, 90.92) --
		(237.91, 90.92) --
		(238.53, 90.92) --
		(239.15, 90.92) --
		(239.76, 90.92) --
		(240.38, 90.92) --
		(241.00, 90.92) --
		(241.62, 90.92) --
		(242.24, 90.92) --
		(242.86, 90.92) --
		(243.48, 90.92) --
		(244.10, 90.92) --
		(244.72, 96.95) --
		(245.34, 89.86) --
		(245.96, 90.92) --
		(246.58, 90.92) --
		(247.20, 90.92) --
		(247.82, 90.92) --
		(248.44, 90.92) --
		(249.06, 90.92) --
		(249.68, 90.92) --
		(250.30, 90.92) --
		(250.92, 90.92) --
		(251.54, 90.92) --
		(252.16, 90.92) --
		(252.78, 90.92) --
		(253.40, 90.92) --
		(254.02, 90.92) --
		(254.64, 90.92) --
		(255.26, 90.92) --
		(255.88, 90.92) --
		(256.50, 90.92) --
		(257.12, 90.92) --
		(257.74, 90.92) --
		(258.36, 90.92) --
		(258.98, 90.92) --
		(259.59, 90.92) --
		(260.21, 90.92) --
		(260.83, 86.22) --
		(261.45, 86.22) --
		(262.07, 90.92) --
		(262.69, 90.92) --
		(263.31, 90.92) --
		(263.93, 90.92) --
		(264.55, 90.92) --
		(265.17, 90.92) --
		(265.79, 90.92) --
		(266.41, 90.92) --
		(267.03, 86.22) --
		(267.65, 86.22) --
		(268.27, 86.22) --
		(268.89, 86.22) --
		(269.51, 86.22) --
		(270.13, 86.22) --
		(270.75, 86.22) --
		(271.37, 86.22) --
		(271.99, 86.22) --
		(272.61, 86.22) --
		(273.23, 86.22) --
		(273.85, 86.22) --
		(274.47, 86.22) --
		(275.09, 86.22) --
		(275.71, 86.22) --
		(276.33, 86.22) --
		(276.95, 86.22) --
		(277.57, 86.22) --
		(278.19, 86.22) --
		(278.81, 86.22) --
		(279.42, 86.22) --
		(280.04, 86.22) --
		(280.66, 86.22) --
		(281.28, 86.22) --
		(281.90, 86.22) --
		(282.52, 86.22) --
		(283.14, 86.22) --
		(283.76, 86.22) --
		(284.38, 86.22) --
		(285.00, 86.22) --
		(285.62, 86.22) --
		(286.24, 86.22) --
		(286.86, 86.22) --
		(287.48, 86.22) --
		(288.10, 86.22) --
		(288.72, 86.22) --
		(289.34, 86.22) --
		(289.96, 86.22) --
		(290.58, 86.22) --
		(291.20, 86.22) --
		(291.82, 86.22) --
		(292.44, 86.22) --
		(293.06, 86.22) --
		(293.68, 86.22) --
		(294.30, 86.22) --
		(294.92, 86.22) --
		(295.54, 86.22) --
		(296.16, 86.22) --
		(296.78, 86.22) --
		(297.40, 86.22) --
		(298.02, 86.22) --
		(298.64, 86.22) --
		(299.25, 86.22) --
		(299.87, 86.22) --
		(300.49, 86.22) --
		(301.11, 86.22) --
		(301.73, 86.22) --
		(302.35, 86.22) --
		(302.97, 86.22) --
		(303.59, 86.22) --
		(304.21, 86.22) --
		(304.83, 86.22) --
		(305.45, 86.22) --
		(306.07, 86.22) --
		(306.69, 86.22) --
		(307.31, 86.22) --
		(307.93, 86.22) --
		(308.55, 86.22) --
		(309.17, 86.22) --
		(309.79, 86.22) --
		(310.41, 86.22) --
		(311.03, 86.22) --
		(311.65, 86.22) --
		(312.27, 86.22) --
		(312.89, 86.22) --
		(313.51, 86.22) --
		(314.13, 86.22) --
		(314.75, 86.22) --
		(315.37, 86.22) --
		(315.99, 86.22) --
		(316.61, 86.22) --
		(317.23, 86.22) --
		(317.85, 86.22) --
		(318.47, 86.22) --
		(319.08, 86.22) --
		(319.70, 86.22) --
		(320.32, 86.22) --
		(320.94, 86.22) --
		(321.56, 86.22) --
		(322.18, 86.22) --
		(322.80, 86.22) --
		(323.42, 86.22) --
		(324.04, 86.22) --
		(324.66, 86.22) --
		(325.28, 86.22) --
		(325.90, 86.22) --
		(326.52, 86.22) --
		(327.14, 86.22) --
		(327.76, 86.22) --
		(328.38, 86.22) --
		(329.00, 86.22) --
		(329.62, 86.22) --
		(330.24, 86.22) --
		(330.86, 86.22) --
		(331.48, 86.22) --
		(332.10, 86.22) --
		(332.72, 86.22) --
		(333.34, 86.22) --
		(333.96, 86.22) --
		(334.58, 86.22) --
		(335.20, 86.22) --
		(335.82, 86.22) --
		(336.44, 86.22) --
		(337.06, 86.22) --
		(337.68, 86.22) --
		(338.30, 86.22) --
		(338.91, 86.22) --
		(339.53, 86.22) --
		(340.15, 86.22) --
		(340.77, 86.22) --
		(341.39, 86.22) --
		(342.01, 86.22) --
		(342.63, 86.22) --
		(343.25, 86.22) --
		(343.87, 86.22) --
		(344.49, 86.22) --
		(345.11, 86.22) --
		(345.73, 86.22) --
		(346.35, 86.22) --
		(346.97, 86.22) --
		(347.59, 86.22) --
		(348.21, 86.22) --
		(348.83, 86.22) --
		(349.45, 86.22) --
		(350.07, 86.22) --
		(350.69, 86.22) --
		(351.31, 86.22) --
		(351.93, 86.22) --
		(352.55, 86.22) --
		(353.17, 86.22) --
		(353.79, 86.22) --
		(354.41, 86.22) --
		(355.03, 86.22) --
		(355.65, 86.22) --
		(356.27, 86.22) --
		(356.89, 86.22) --
		(357.51, 86.22) --
		(358.13, 86.22) --
		(358.74, 86.22) --
		(359.36, 86.22) --
		(359.98, 86.22) --
		(360.60, 86.22) --
		(361.22, 86.22) --
		(361.84, 86.22) --
		(362.46, 86.22) --
		(363.08, 86.22) --
		(363.70,100.08) --
		(364.32,100.08) --
		(364.94,100.08) --
		(365.56,100.08) --
		(366.18,100.08) --
		(366.80,100.08) --
		(367.42,100.08) --
		(368.04,100.08) --
		(368.66,100.08) --
		(369.28,100.08) --
		(369.90,100.08) --
		(370.52,100.08) --
		(371.14,100.08) --
		(371.76,100.08) --
		(372.38,100.08) --
		(373.00,100.08) --
		(373.62,100.08) --
		(374.24,100.08) --
		(374.86,100.08) --
		(375.48,100.08) --
		(376.10,100.08) --
		(376.72,100.08) --
		(377.34,100.08) --
		(377.96,100.08) --
		(378.57,100.08) --
		(379.19,100.08) --
		(379.81,100.08) --
		(380.43,100.08) --
		(381.05,100.08) --
		(381.67,100.08) --
		(382.29,100.08) --
		(382.91,100.08) --
		(383.53,100.08) --
		(384.15,100.08) --
		(384.77,100.08) --
		(385.39,100.08) --
		(386.01,100.08) --
		(386.63,100.08) --
		(387.25,100.08) --
		(387.87,100.08) --
		(388.49,100.08) --
		(389.11,100.08) --
		(389.73,100.08) --
		(390.35,100.08) --
		(390.97,100.08) --
		(391.59,100.08) --
		(392.21,100.08) --
		(392.83,100.08) --
		(393.45,100.08) --
		(394.07,100.08) --
		(394.69,100.08) --
		(395.31,100.08) --
		(395.93,100.08) --
		(396.55,100.08) --
		(397.17,100.08) --
		(397.79,100.08) --
		(398.41,100.08) --
		(399.02,100.08) --
		(399.64,100.08) --
		(400.26,100.08) --
		(400.88,100.08) --
		(401.50,100.08) --
		(402.12,100.08) --
		(402.74,100.08) --
		(403.36,100.08) --
		(403.98,100.08) --
		(404.60,100.08) --
		(405.22,100.08) --
		(405.84,100.08) --
		(406.46,100.08) --
		(407.08,100.08) --
		(407.70,100.08) --
		(408.32,100.08) --
		(408.94,100.08) --
		(409.56,100.08) --
		(410.18,100.08) --
		(410.80,100.08) --
		(411.42,100.08) --
		(412.04,100.08) --
		(412.66,100.08) --
		(413.28,100.08) --
		(413.90,100.08) --
		(414.52,100.08) --
		(415.14,100.08) --
		(415.76,100.08) --
		(416.38,100.08) --
		(417.00,100.08) --
		(417.62,100.08) --
		(418.24,100.08) --
		(418.85,100.08) --
		(419.47,100.08) --
		(420.09,100.08) --
		(420.71,100.08) --
		(421.33,100.08) --
		(421.95,100.08) --
		(422.57,100.08) --
		(423.19,100.08) --
		(423.81,100.08) --
		(424.43,100.08) --
		(425.05,100.08) --
		(425.67,100.08) --
		(426.29,100.08) --
		(426.91,100.08) --
		(427.53,100.08) --
		(428.15,100.08) --
		(428.77,100.08) --
		(429.39,100.08) --
		(430.01,100.08) --
		(430.63,100.08) --
		(431.25,100.08) --
		(431.87,100.08) --
		(432.49,100.08) --
		(433.11,100.08) --
		(433.73,100.08) --
		(434.35,100.08) --
		(434.97,100.08) --
		(435.59,100.08) --
		(436.21,100.08) --
		(436.83,100.08) --
		(437.45,100.08) --
		(438.07,100.08) --
		(438.68,100.08) --
		(439.30,100.08) --
		(439.92,100.08) --
		(440.54,100.08) --
		(441.16,100.08) --
		(441.78,100.08) --
		(442.40,100.08) --
		(443.02,100.08) --
		(443.64,100.08) --
		(444.26,100.08) --
		(444.88,100.08) --
		(445.50,100.08) --
		(446.12,100.08) --
		(446.74,100.08) --
		(447.36,100.08) --
		(447.98,100.08) --
		(448.60,100.08) --
		(449.22,100.08) --
		(449.84,100.08) --
		(450.46,100.08) --
		(451.08,100.08) --
		(451.70,100.08) --
		(452.32,100.08) --
		(452.94,100.08) --
		(453.56,100.08) --
		(454.18,100.08) --
		(454.80,100.08) --
		(455.42,100.08) --
		(456.04,100.08) --
		(456.66,100.08) --
		(457.28,100.08) --
		(457.90,100.08) --
		(458.51,100.08) --
		(459.13,100.08) --
		(459.75,100.08) --
		(460.37,100.08) --
		(460.99,100.08) --
		(461.61,100.08) --
		(462.23,100.08) --
		(462.85,100.08) --
		(463.47,100.08) --
		(464.09,100.08) --
		(464.71,100.08) --
		(465.33,100.08) --
		(465.95,100.08) --
		(466.57,100.08) --
		(467.19,100.08) --
		(467.81,100.08) --
		(468.43,100.08) --
		(469.05,100.08) --
		(469.67,100.08) --
		(470.29,100.08) --
		(470.91,100.08) --
		(471.53,100.08) --
		(472.15,100.08) --
		(472.77,100.08) --
		(473.39,100.08) --
		(474.01,100.08) --
		(474.63,100.08) --
		(475.25,100.08) --
		(475.87,100.08) --
		(476.49,100.08) --
		(477.11,100.08) --
		(477.73,100.08) --
		(478.34,100.08) --
		(478.96,100.08) --
		(479.58,100.08) --
		(480.20,100.08) --
		(480.82,100.08) --
		(481.44,100.08) --
		(482.06,100.08) --
		(482.68,100.08) --
		(483.30,100.08) --
		(483.92,100.08) --
		(484.54,100.08) --
		(485.16,100.08) --
		(485.78,100.08) --
		(486.40,100.08) --
		(487.02,100.08) --
		(487.64,100.08) --
		(488.26,100.08) --
		(488.88,100.08) --
		(489.50,100.08) --
		(490.12,100.08) --
		(490.74,100.08) --
		(491.36,100.08) --
		(491.98,100.08) --
		(492.60,100.08) --
		(493.22,100.08) --
		(493.84,100.08) --
		(494.46,100.08) --
		(495.08,100.08) --
		(495.70,100.08) --
		(496.32,100.08) --
		(496.94,100.08) --
		(497.56,100.08) --
		(498.17,100.08) --
		(498.79,100.08) --
		(499.41,100.08) --
		(500.03,100.08) --
		(500.65,100.08) --
		(501.27,100.08) --
		(501.89,100.08) --
		(502.51,100.08) --
		(503.13,100.08) --
		(503.75,100.08) --
		(504.37,100.08) --
		(504.99,100.08) --
		(505.61,100.08) --
		(506.23,100.08) --
		(506.85,100.08) --
		(507.47,100.08) --
		(508.09,100.08) --
		(508.71,100.08) --
		(509.33,100.08) --
		(509.95,100.08) --
		(510.57,100.08) --
		(511.19,100.08) --
		(511.81,100.08) --
		(512.43,100.08) --
		(513.05,100.08) --
		(513.67,100.08) --
		(514.29,100.08) --
		(514.91,100.08) --
		(515.53,100.08) --
		(516.15,100.08) --
		(516.77,100.08) --
		(517.39,100.08) --
		(518.00,100.08) --
		(518.62,100.08) --
		(519.24,100.08) --
		(519.86,100.08) --
		(520.48,100.08) --
		(521.10,100.08) --
		(521.72,100.08) --
		(522.34,100.08) --
		(522.96,100.08) --
		(523.58,100.08) --
		(524.20,100.08) --
		(524.82,100.08) --
		(525.44,100.08) --
		(526.06,100.08) --
		(526.68,100.08) --
		(527.30,100.08) --
		(527.92,100.08) --
		(528.54,100.08) --
		(529.16,100.08) --
		(529.78,100.08);
	\end{scope}
	\begin{scope}
		\path[clip] (552.55,205.60) rectangle (572.66,283.58);
		\definecolor{drawColor}{gray}{0.10}

		\node[text=drawColor,rotate=-90.00,anchor=base,inner sep=0pt, outer sep=0pt, scale=  1.07] at (558.20,244.59) {Forget $\theta$};
	\end{scope}
	\begin{scope}
		\path[clip] (552.55,122.12) rectangle (572.66,200.10);
		\definecolor{drawColor}{gray}{0.10}

		\node[text=drawColor,rotate=-90.00,anchor=base,inner sep=0pt, outer sep=0pt, scale=  1.07] at (558.20,161.11) {Smooth Mv $\lambda^{\text{mv}}$};
	\end{scope}
	\begin{scope}
		\path[clip] (552.55, 38.63) rectangle (572.66,116.62);
		\definecolor{drawColor}{gray}{0.10}

		\node[text=drawColor,rotate=-90.00,anchor=base,inner sep=0pt, outer sep=0pt, scale=  1.07] at (558.20, 77.62) {Smooth Pr $\lambda^{\text{pr}}$};
	\end{scope}
	\begin{scope}
		\path[clip] (  0.00,  0.00) rectangle (578.16,289.08);
		\definecolor{drawColor}{gray}{0.30}

		\node[text=drawColor,anchor=base,inner sep=0pt, outer sep=0pt, scale=  1.07] at ( 77.41, 24.87) {2019-01};

		\node[text=drawColor,anchor=base,inner sep=0pt, outer sep=0pt, scale=  1.07] at (189.57, 24.87) {2019-07};

		\node[text=drawColor,anchor=base,inner sep=0pt, outer sep=0pt, scale=  1.07] at (303.59, 24.87) {2020-01};

		\node[text=drawColor,anchor=base,inner sep=0pt, outer sep=0pt, scale=  1.07] at (416.38, 24.87) {2020-07};

		\node[text=drawColor,anchor=base,inner sep=0pt, outer sep=0pt, scale=  1.07] at (530.40, 24.87) {2021-01};
	\end{scope}
	\begin{scope}
		\path[clip] (  0.00,  0.00) rectangle (578.16,289.08);
		\definecolor{drawColor}{gray}{0.30}

		\node[text=drawColor,anchor=base east,inner sep=0pt, outer sep=0pt, scale=  1.07] at ( 46.58,222.80) {0.00};

		\node[text=drawColor,anchor=base east,inner sep=0pt, outer sep=0pt, scale=  1.07] at ( 46.58,246.50) {0.01};

		\node[text=drawColor,anchor=base east,inner sep=0pt, outer sep=0pt, scale=  1.07] at ( 46.58,270.19) {0.06};
	\end{scope}
	\begin{scope}
		\path[clip] (  0.00,  0.00) rectangle (578.16,289.08);
		\definecolor{drawColor}{gray}{0.30}

		\node[text=drawColor,anchor=base east,inner sep=0pt, outer sep=0pt, scale=  1.07] at ( 46.58,131.74) {0.25};

		\node[text=drawColor,anchor=base east,inner sep=0pt, outer sep=0pt, scale=  1.07] at ( 46.58,149.24) {8.00};

		\node[text=drawColor,anchor=base east,inner sep=0pt, outer sep=0pt, scale=  1.07] at ( 46.58,166.73) {256.00};

		\node[text=drawColor,anchor=base east,inner sep=0pt, outer sep=0pt, scale=  1.07] at ( 46.58,184.23) {8192.00};
	\end{scope}
	\begin{scope}
		\path[clip] (  0.00,  0.00) rectangle (578.16,289.08);
		\definecolor{drawColor}{gray}{0.30}

		\node[text=drawColor,anchor=base east,inner sep=0pt, outer sep=0pt, scale=  1.07] at ( 46.58, 49.30) {0.12};

		\node[text=drawColor,anchor=base east,inner sep=0pt, outer sep=0pt, scale=  1.07] at ( 46.58, 66.61) {1.00};

		\node[text=drawColor,anchor=base east,inner sep=0pt, outer sep=0pt, scale=  1.07] at ( 46.58, 83.92) {8.00};

		\node[text=drawColor,anchor=base east,inner sep=0pt, outer sep=0pt, scale=  1.07] at ( 46.58,101.23) {64.00};
	\end{scope}
	\begin{scope}
		\path[clip] (  0.00,  0.00) rectangle (578.16,289.08);
		\definecolor{drawColor}{RGB}{0,0,0}

		\node[text=drawColor,anchor=base,inner sep=0pt, outer sep=0pt, scale=  1.33] at (302.04,  8.61) {date};
	\end{scope}
\end{tikzpicture}
}
    }}
  \caption{Hyperparameter values of the \textbf{Smooth.Forget} specification with \textbf{Bayesian Online} tuning. The grey area represents the burn-in period which is not used for evaluation.}\label{fig:best_pars}
\end{figure}

\section{Conclusion}\label{conclusion}

This paper proposes a novel method for combining multivariate probabilistic forecasts, considering dependencies between quantiles and marginals through a smoothing procedure. The first discussed smoothing method reduces the dimensionality of the regret using Basis matrices. It bridges the gap between pointwise and constant weight optimization. The second method involves smoothing the weights by penalized smoothing. The proposed method extends the (online) \textit{CRPS learning} algorithm and is based on gradient-based EWA, which yields fast convergence rates. We also discuss possible extensions to the algorithm, such as forgetting and shrinkage operators. We also discuss how non-equidistant knots can be used in penalized smoothing, i.e., how the penalty term needs to adjust.

We apply the proposed methodology to multivariate probabilistic forecasts for day-ahead electricity prices. These are 24-dimensional distributional forecasts from which we extracted an equidistant grid of 99 percentiles. For hyperparameter tuning of Algorithm~\ref{algo:boag_smooth}, we compare two strategies \textbf{Bayesian fix} and \textbf{Sampling Online}.

The best performance is obtained by optimizing \textbf{Sampling Online} with penalized smoothing, and a forget rate. This specification yields a significant improvement over the \textit{naive} combination. The forget rate contributes to the majority of the improvement, indicating structural changes in the data. The second considered tuning strategy using \textbf{Bayesian fix} produced subpar results compared to the \textbf{Sampling Online} strategy. This is likely due to the mentioned structural changes to which the hyperparameters can adjust dynamically when optimizing \textbf{Sampling Online}. The BOA weighting scheme obtained the best results compared to two other popular schemes, ML-Poly and EWA. We also discuss the temporal evolution of the weights, which stabilize quickly after a short burn-in period. However, we observed some structural change in the weights around November 2019 which remains puzzling.

The smoothing methods used in this paper assume a relatively simple metric or spatial structure where we only consider the dependence between adjacent time series. This assumption is reasonable for electricity prices and all time series that we combine over the forecast horizon, as the order is naturally determined by time. In other cases, the adjacency structure may be more complex. For example, in spatial data, the first and last coordinates are often adjacent (e.g. forecasting a meteorological feature along the equator). Other spatial structures (e.g. across the globe) may be even more complex. We can account for these structures by deriving the penalty terms from the respective adjacency matrix, which describes the metric structure of the data. The Laplacian matrix (graph Laplacian, admittance matrix or Kirchhoff matrix) can be calculated from this matrix. It represents the dependencies of a graph. The penalty matrix is the negative of the Laplacian. Therefore, instead of using the identity matrix for the penalty term, we can use the Laplacian matrix. Both approaches will be consistent with the spatial structure considered in this paper. This is a promising topic for future research.

As outlined, we consider potential spatial structures between quantiles and marginals. We do this for the performance of the considered experts. Hence, this approach is particularly suitable for situations where only quantiles of the marginals are available. In practice, this is often satisfied as experts regularly specialize in forecasting one specific marginal. However, ff experts provide the dependency structure between marginals, i.e., by providing joint predictive distributions, combination algorithms that take these dependencies into account should be chosen. However, this is a nontrivial task. Methods that assign uniform weights across the joint distribution can be applied easily. However, by design, these will ignore any variations in experts' performance across the joint distribution. On the other hand, combining marginals using non-uniform weighting strategies like CRPS learning considers the variations mentioned above. However, it needs to be clarified how the dependency structure can be reintroduced into the predictive marginals, i.e., which weights should be chosen for combining the copula. This remains an important topic for future investigation.

Overall, the empirical results approve the proposed algorithm, which accounts for variations across time, distribution, and marginals. The algorithm surpasses the performance of simpler weighting schemes such as the \textit{naive} and several nested cases in terms of overall CRPS. A fast C++ implementation of the proposed algorithm is available in the open-source \textit{R}-Package \textit{profoc} \citep{profoc_package}.

\clearpage

\appendix
\section{Initialization of Algorithm~\ref{algo:boag_smooth}}\label{append_init}


\begin{algorithm}[!h]
  \SetAlgoLined
  \DontPrintSemicolon
  \textbf{input:}
  4-dimensional array of expert predictions $(\what{X})_{t,d,p,k}$ and matrix of prediction targets $\bsY_{t,d}$ for $t=1,\ldots, T$, $d\in \bsDD = (d_1,\ldots,d_D)$,  $p\in \bsPP = (p_1,\ldots,p_P)$, $k=1,\ldots, K$ \;
  \textbf{initialize:} \\
  $\bsw_{0} = 1/K$\;
  $\bsE_{0} = \bsV_0 = \bsR_0 = \bsnull$ \;
  $\bsbeta_0 = \text{pinv}\left(\bsB^{\mult}\right) \bsw_0\left(\bsPP\right) \text{pinv}\left(\bsB^{\prob}\right)'$ \;
  $\boldsymbol{\mathcal{H}^\prob} = \bsB^\prob({\bsB^\prob}' \bsB^\prob + \lambda^{\P} (\alpha \boldsymbol D_1'\boldsymbol D_1 + (1-\alpha) \boldsymbol D_2'\boldsymbol D_2))^{-1} {\bsB^\prob}'$\;
  $\boldsymbol{\mathcal{H}^\mult} = \bsB^\mult({\bsB^\mult}' \bsB^\mult + \lambda^{\D} (\alpha \boldsymbol D_1'\boldsymbol D_1 + (1-\alpha) \boldsymbol D_2'\boldsymbol D_2))^{-1} {\bsB^\mult}'$\;
  \caption{\label{algo:init} Initialization of Algorithm \ref{algo:boag_smooth}}
\end{algorithm}

%  \clearpage
\bibliographystyle{anc/model5-names}

\bibliography{bib}



\end{document}
