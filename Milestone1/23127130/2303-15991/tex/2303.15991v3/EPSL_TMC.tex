%\documentclass[conference]{IEEEtran}
\UseRawInputEncoding
\documentclass[journal]{IEEEtran}

\usepackage{blindtext}
\usepackage{amssymb}
\usepackage{amsmath}
\usepackage{graphicx}
\usepackage{cite}
\usepackage{citesort}
\usepackage{multicol}
%\usepackage{balance}
\usepackage{amsfonts}
\usepackage{color}
 %\usepackage[utf8x]{inputenc}
\usepackage{subfigure}
\usepackage{epstopdf}
%\DeclareGraphicsExtensions{.png,.pdf}
\usepackage{amssymb}
\usepackage{comment}
\usepackage{amsmath}
\allowdisplaybreaks[4]
\usepackage{booktabs}
\usepackage{multirow}
\usepackage{bmpsize}
%\newtheorem{definition}{Definition}
%\newtheorem{theorem}{Theorem}
\usepackage{diagbox}
\usepackage{mathrsfs}
\usepackage{algorithm}
\usepackage{algpseudocode}
\usepackage{amsmath}
\renewcommand{\algorithmicrequire}{\textbf{Input:}}  % Use Input in the format of Algorithm
\renewcommand{\algorithmicensure}{\textbf{Output:}} % Use Output in the format of Algorithm
%\newtheorem{property}{Property}
%\usepackage{mdframed}
\usepackage{lipsum}
\usepackage{stfloats}
%\newmdtheoremenv{theo}{Theorem}
\usepackage{amsmath, bm}
\usepackage{booktabs}
\newtheorem{proposition}{{Proposition}}
%\newenvironment{theorembox} {\begin{theorem}}{\hfill \interlinepenalty500 $\Box$\end{theorem}}

%\newenvironment{lemmabox} {\begin{lemma}}{\hfill \interlinepenalty500 $\Box$\end{lemma}}
%\newtheorem{corollary}{\textbf{Corollary}}
%\newenvironment{corollarybox} {\begin{corollary}}{\hfill \interlinepenalty500 $\Box$\end{corollary}}
%\newtheorem{proof}{\textbf{Proof}}
%\newenvironment{proofbox} {\begin{proof}}{\hfill \interlinepenalty500 $\Box$\end{proof}}





%\newcounter{mytempeqncnt}
\newtheorem{theorem}{\textbf{Theorem}}
\newenvironment{theorembox} {\begin{theorem}}{\hfill \interlinepenalty500 $\Box$\end{theorem}}
\newtheorem{lemma}{\textbf{Lemma}}
\newenvironment{lemmabox} {\begin{lemma}}{\hfill \interlinepenalty500 $\Box$\end{lemma}}
\newtheorem{corollary}{\textbf{Corollary}}
\newenvironment{corollarybox} {\begin{corollary}}{\hfill \interlinepenalty500 $\Box$\end{corollary}}
\newtheorem{proof}{\textbf{Proof}}
%\newenvironment{proofbox} {\begin{proof}}{\hfill \interlinepenalty500 $\Box$\end{proof}}

\makeatletter
\def\ScaleIfNeeded{%
\ifdim\Gin@nat@width>\linewidth \linewidth \else \Gin@nat@width
\fi } \makeatother


\DeclareMathOperator{\vecOp}{\mathrm{vec}}
\DeclareMathOperator{\tr}{\mathrm{tr}}
\DeclareMathOperator{\E}{\mathbb{E}}
\DeclareMathOperator{\var}{\mathbb{V}\mathrm{ar}}
\DeclareMathOperator{\cov}{\mathrm{Cov}}
\DeclareMathOperator{\sgn}{\mathrm{sgn}}
\DeclareMathOperator{\etr}{\mathrm{etr}}
\DeclareMathOperator{\re}{\mathfrak{Re}}
\DeclareMathOperator{\im}{\mathfrak{Im}}

\begin{document}
%\pagestyle{fancyplain}
%
%\pagestyle{fancy}
%\lhead[]%
 %   {\footnotesize Physical layer security}
%\cfoot{}

\title{\huge{Efficient Parallel Split Learning over Resource-constrained Wireless Edge Networks}}
\author{Zheng Lin, Guangyu Zhu, Yiqin Deng,~\IEEEmembership{Member,~IEEE},  Xianhao Chen,~\IEEEmembership{Member,~IEEE}, \\ Yue Gao,~\IEEEmembership{Senior Member,~IEEE}, Kaibin Huang,~\IEEEmembership{Fellow,~IEEE},  and Yuguang Fang,~\IEEEmembership{Fellow,~IEEE}
\thanks{Zheng Lin is with the Department of Electrical Engineering, Fudan University, Shanghai 200438, China (e-mail: zlin20@fudan.edu.cn).}
\thanks{Guangyu Zhu is with the Department of Electrical and Computer Engineering, University of Florida, Gainesville, FL 32611 USA (e-mail: gzhu@ufl.edu).}
\thanks{Yiqin Deng is with the School of Control Science and Engineering, Shandong University, Jinan 250061, Shandong, China (e-mail: yiqin.deng@email.sdu.edu.cn).}
\thanks{Xianhao Chen and Kaibin Huang are with the Department of Electrical and Electronic Engineering, University of Hong Kong, Pok Fu Lam, Hong Kong (e-mail: xchen@eee.hku.hk; huangkb@eee.hku.hk).}

\thanks{Yue Gao is with the School of Computer Science, Fudan University, Shanghai 200438, China (email: gao\underline{~}yue@fudan.edu.cn).}

\thanks{Yuguang Fang is with the Department of Computer Science, City University of Hong Kong, Kowloon, Hong Kong (e-mail: my.fang@cityu.edu.hk).}

\thanks{\textit{(Corresponding author: Xianhao Chen)}}

}

\maketitle

\begin{abstract}
The increasingly deeper neural networks hinder the democratization of privacy-enhancing distributed learning, such as federated learning (FL), to resource-constrained devices. To overcome this challenge, in this paper, we advocate the integration of edge computing paradigm and parallel split learning (PSL), allowing multiple client devices to offload substantial training workloads to an edge server via layer-wise model split. By observing that existing PSL schemes incur excessive training latency and large volume of data transmissions, we propose an innovative PSL framework, namely, efficient parallel split learning (EPSL), to accelerate model training. To be specific, EPSL parallelizes client-side model training and \textit{reduces the dimension of activations' gradients} for back propagation (BP) via \textit{last-layer gradient aggregation}, leading to a significant reduction in server-side training and communication latency. Moreover, by considering the heterogeneous channel conditions and computing capabilities at client devices, we jointly optimize subchannel allocation, power control, and cut layer selection to minimize the per-round latency. Simulation results show that the proposed EPSL framework significantly decreases the training latency needed to achieve a target accuracy compared with the state-of-the-art benchmarks, and the tailored resource management and layer split strategy can considerably reduce latency than the counterpart without optimization.


\end{abstract}

\begin{IEEEkeywords}
Distributed learning, split learning, edge computing, resource management, edge intelligence.
\end{IEEEkeywords}

%========================================================================
\section{Introduction}
With the proliferation of Internet of Things (IoT) devices and advancement of information and communications technology (ICT), networked IoT devices are capable of collecting massive data in a timely fashion. It is predicted in~\cite{report} that the worldwide number of IoT devices is expected to reach 29 billion by 2030, nearly tripling from 2020. The unprecedented amount of data generated from countless devices serves as the fuel to power artificial intelligence (AI), which has gained tremendous success in major sectors, including smart healthcare, natural language processing, and intelligent transportation~\cite{2022Letaief,zhu2020toward,wu2020fedhome,lin2022channel}.

\begin{figure*}[t!]
\centering
\includegraphics[width=15.2 cm]{DCML_comparison.eps}
\vspace{-0cm}
\caption{An illustration of vanilla SL, PSL, SFL and EPSL frameworks, where $\phi$ denotes the ratio of last-layer gradient aggregation. When $\phi = 0$, EPSL is reduced to PSL.
}% \textcolor[rgb]{1.00,0.00,0.00}{edge clouds  at or linked to the SBSs through optical fiber}
\label{DCML_comparison}
\end{figure*}

The conventional machine learning (ML) framework, known as centralized learning (CL), entails gathering and processing the raw data in a central server. Each client device shares local raw data with a central server for model training. However, the lack of trust and privacy regulations deter client devices from sharing their private data, hindering scalable AI deployment. Moreover, it is unrealistic for all client devices to transfer local raw data to a central server due to the large bandwidth consumption. Motivated by this, federated learning (FL)~\cite{konevcny2016federated} is proposed to enable collaborative model training without sharing raw data. Nevertheless, FL exclusively employs on-device training, which places heavy computation burden on client devices. For instance, the popular image recognition model VGG16~\cite{VGG} comprises 138 million parameters (528MB in 32bit float) and requires 15.5G MACs (multiply-add computation) for forward propagation (FP) process, essentially making exclusive on-device training impractical for resource-constrained mobile or IoT devices. One promising solution is to compress the ML model to lower both communication and computing workload~\cite{shi2022toward,chen2022energy,xu2022adaptive}. However, model compression/quantization inevitably induces more inference/learning errors due to the low-precision calculations.

To address the above challenges, split learning (SL)~\cite{vepakomma2018split} has emerged as an effective approach to taking over training load from client devices while still preserving their raw data locally. Specifically, SL allows client devices to offload substantial training workloads to a server via layer-wise model partitioning. In the vanilla SL, the sequential training from one client device to another, however, incurs excessive training latency. To overcome this severe limitation, split federated learning (SFL)~\cite{thapa2022splitfed} and parallel split learning (PSL) (without client-side model synchronization compared with SFL)~\cite{kim2022bargaining,joshi2021splitfed} have been proposed, which enable multiple devices to train in parallel, thus positioning SL as a promising alternative for FL in many situations.

Unfortunately, the existing SFL/PSL frameworks can still lead to significant computing and communication latency. On the one hand, an SL/SFL/PSL server \textit{takes over the main training workloads from multiple clients}. Although training models for a reasonable number of client devices might not be a crucial issue for a sufficiently powerful cloud computing center, it is arguably overwhelming to a resource-constrained edge server as the number of served client devices increases. Particularly, edge computing servers in 5G and beyond can be (small) base stations and access points usually equipped with limited capabilities~\cite{ding2018beef,deng2022actions}. On the other hand, communication latency is a limiting factor due to the large volume of cut-layer data and model exchange involved in SL. Specifically, PSL incurs large volumes of cut-layer data exchange while SFL involves both massive cut-layer data exchange and model exchange, as summarized in Fig.~\ref{DCML_comparison}. To deploy low-latency SL at the network edge, \textit{a more efficient SL is desired}.


In this paper, we propose a novel SL framework, namely, efficient parallel split learning (EPSL), which is particularly useful for resource-constrained edge computing systems. The proposed EPSL framework incorporates parallel model training and model partitioning, which amalgamates the advantages of FL and SL.  The parallel training of client-side model efficiently utilizes idle local computing resources, while model partitioning significantly lowers the computation workload and communication overhead on client devices. More importantly, we design the last-layer gradient aggregation on the server-side model to shrink the dimension of activations' gradients and computation workload on a server, leading to a considerable reduction in training latency compared with the state-of-the-art SL benchmarks, including SFL and PSL. Furthermore, this operation also reduces communication overhead and eliminates the necessity for model exchange. We define aggregation ratio $\phi  \in \left[ {0,1} \right] $  to control the fraction of last-layer activations' gradients to be aggregated, which strikes the balance between learning accuracy and latency reduction. The detailed comparison of FL, vanilla SL, SFL, PSL and EPSL frameworks is illustrated in Table~\ref{table1}, demonstrating that EPSL is more compute- and communication-efficient than existing SL schemes. It is noted that PSL is a special case of EPSL when the aggregation ratio is set to 0.


Moreover, at the wireless edge, the heterogeneous channel conditions and computing capabilities at client devices considerably decrease the efficiency of ML model training due to the straggler effect~\cite{chen2021distributed,xiao2023time,lee2017speeding}. In this regard, the appropriate design of resource management and layer split strategy can preferably allocate more resources to the straggler, thereby boosting learning efficiency. Towards this end, we formulate an optimization problem of subchannel allocation, power control, and cut layer selection for the proposed EPSL, and develop an efficient algorithm to solve it. The major contributions of this paper are summarized as follows.


\begin{itemize}
\item  We propose EPSL, a compute- and communication-efficient PSL framework, which features the key operation of aggregating the last-layer activations' gradients during the BP process to reduce communication and computing overhead for PSL.
\item  We design an effective resource management and layer split strategy that jointly optimizes subchannel allocation, power control, and cut layer selection for EPSL to minimize the per-round latency.
\item  We conduct extensive simulation studies to demonstrate that the proposed EPSL framework significantly decreases the training latency needed to achieve a target accuracy compared with existing SL schemes. Besides, the tailored resource management and layer split strategy can significantly reduce the latency than the counterpart without optimization.
\end{itemize}

The remainder of this paper is organized as follows. Section~\ref{Related_Work} and~\ref{System_Model} introduce the related work and system model. Section~\ref{EPSL_Framework} elaborates on the detailed EPSL framework. We formulate the resource management problem in Section~\ref{PROBLEM_FORMULATION} and offer the corresponding solution approach in Section~\ref{Solution_Approach}. Section~\ref{simulation results} covers the simulation results. Finally, concluding remarks are presented in Section~\ref{conclusion_section}.

%{\emph{Notations:}} In this paper, $\left(\cdot\right)^{\rm{T}}$ is the transpose operator; bold uppercase letter denotes matrices (i.e., ${\bf{W}}$); vectors are represented by bold lowercase letters (i.e., ${\bf{x}}$); normal font presents scalar (i.e., $y$).



\begin{table}[t]\label{table1}
\caption{The Comparison of FL, vanilla SL, SFL, PSL and EPSL Frameworks.}
  \renewcommand{\arraystretch}{1.4}{
  \setlength{\tabcolsep}{2mm}{
\begin{tabular}{cccccc}
\hline
\textbf{Learning framework}                                                              & \textbf{FL} & \textbf{vanilla SL} & \textbf{SFL} & \textbf{PSL} & \textbf{EPSL} \\ \hline
\textbf{\begin{tabular}[c]{@{}c@{}}Partial computation\\ offloading  \end{tabular}} & No          & Yes     & Yes    & Yes          & Yes           \\ \hline
\textbf{\begin{tabular}[c]{@{}c@{}}Parallel computing\end{tabular}}           & Yes    & No  & Yes  & Yes   & Yes      \\ \hline
\textbf{Model exchange}     & Yes & No & Yes & No & No  \\ \hline
\textbf{\begin{tabular}[c]{@{}c@{}}Activations' gradients'\\dimension reduction\end{tabular}} & No & No & No & No & Yes \\ \hline
\textbf{Access to raw data}    & No  & No  & No  & No  & No\\ \hline
\end{tabular}}}
\end{table}
\vspace{1em}


\section{Related Work}\label{Related_Work}

FL has become the most prevalent distributed learning framework due to its advantages in data privacy and parallel model training. However, since FL is resource-hungry, the limited communication and computing capabilities at client devices are the bottlenecks~\cite{shi2022toward}. To mitigate this issue, there are two research directions to be explored: One is to design judicious resource scheduling and transmission schemes, including client selection~\cite{xu2020client,pang2022incentive}, resource allocation~\cite{jiao2020toward,yang2020energy,chen2020joint,dinh2022network}, and hierarchical model aggregation~\cite{liu2020client,chen2022federated}. The other is to integrate model compression frameworks~\cite{shi2022toward,chen2022energy}. However, the latter introduces inevitable approximation errors.


In contrast to FL, SL has recently emerged as a novel privacy-enhancing distributed learning framework mainly targeting resource-constrained devices. It was first proposed by Vepakomma~\textit{et al.}~\cite{vepakomma2018split} for health systems in which privacy concern is of high priority. To enable privacy-preserving deep neural network computation, Kim~\textit{et al.}~\cite{kim2021spatio} develop a novel SL framework with multiple end-systems and verify by experiments that the proposed framework achieves near-optimal accuracy while guaranteeing data privacy. In~\cite{yang2022differentially}, a generic gradient perturbation-based split learning framework is designed to provide provable differential privacy guarantee. The earlier works focus on vanilla SL operating in a sequential fashion, resulting in excessive training latency. Thepa~\textit{et al.}~\cite{thapa2022splitfed} integrate FL and SL to parallelize SL to accelerate model training. Pal~\textit{et al.}~\cite{pal2021server} propose a more scalable PSL framework to address server-side large batch and the backward client decoupling problems by averaging the local gradients at the cut layer. However, our studies show that the training latency in SL can still be substantially reduced without noticeably impacting learning accuracy.

Apart from learning framework development, we will also investigate the resource management and layer split strategy tailored for the proposed SL frameworks. Along this line, Kim~\textit{et al.}~\cite{kim2022bargaining} present the optimal cut layer selection strategy for balancing the energy consumption, training time, and data privacy. To minimize the expected energy and time cost, Yan, Bi and Zhang~\cite{yan2022optimal} devise the joint optimization of model placement and model splitting for split inference. In~\cite{tian2022jmsnas}, a joint model split and neural architecture search approach is proposed to automatically generate and place neural networks over chain or mesh networks. Unfortunately, research on resource management strategy for enhancing SL performance is still in its infancy. It is obvious that resource management and model split are tightly coupled. Reference~\cite{wu2022split} is one of the few studies that considers the integrated design of resource management and cut layer selection, where device clustering is also exploited to reduce communication overhead. However, the algorithm implementation is hampered by the exhaustive search-based cut layer selection��s with high computational complexity.




\section{System Model}\label{System_Model}
\begin{figure}[t!]
\centering
\includegraphics[width=7.2cm]{EPSL.eps}
\vspace{-0.2cm}
\caption{The illustration of EPSL over wireless networks.
}% \textcolor[rgb]{1.00,0.00,0.00}{edge clouds  at or linked to the SBSs through optical fiber}
\label{EPSL}
\end{figure}

As demonstrated in Fig.~\ref{EPSL}, we consider a typical scenario of EPSL framework in wireless networks, which consists of two essential components:

\begin{itemize}
\item \textbf{Client devices:} We assume that each client  possesses an end device with computing capability to perform FP and BP processes for client-side model. The set of participating client devices in the model training is denoted by $\mathcal{C} = \left\{ {1,2,...C} \right\}$, where $C$ is the number of client devices. The local dataset ${\mathcal{D}_i}$ with ${D_i}$ data samples owned by client device $i$ is represented as ${\mathcal{D}_i} = \left\{ {{{\bf{x}}_{i,k}},{y_{i,k}}} \right\}_{k = 1}^{{D_i}}$, where  ${{\bf{x}}_{i,k}} \in {\mathbb{R}^{d \times 1}}$ denotes the $k$-th input data in the local dataset ${\mathcal{D}_i}$ and ${{{y}}_{i,k}} \in {\mathbb{R}^{1}}$ is the label of ${{\bf{x}}_{i,k}}$. Therefore, the total dataset ${\mathcal{D}}$ with $D = \sum\limits_{i = 1}^C {{D_i}} $ data samples is denoted by ${\cal D} =  \cup _{i = 1}^C{{\cal D}_i}$. The client-side model is represented as ${{\bf{W}}_c} \in {\mathbb{R}^{u \times 1}}$, where $u$ denotes the dimension of client-side model's parameters.
\item \textbf{Server:} The server is a central entity with powerful computing capability, which takes charge of server-side model training. The server-side model is denoted by ${{\bf{W}}_s} \in {\mathbb{R}^{r \times 1}}$, where $r$ represents the dimension of server-side model's parameter. The server gathers important network information such as channel status information and device computing capability, to implement the resource management strategy.
\end{itemize}



The global model is represented as ${\bf{W}} = \left[ {{{\bf{W}}_s};{{\bf{W}}_c}} \right] \in {\mathbb{R}^{\left( {r + u} \right) \times 1}}$. For client device $i$, we assume that the smashed data obtained according to ${{{\bf{x}}_{i,k}}}$ is denoted by ${{\bf{s}}_{i,k}} = f\left( {{{\bf{x}}_{i,k}};{{\bf{W}}_c}} \right)\in {\mathbb{R}^{q \times 1}}$, where $q$ denotes smashed data's dimension per data sample, $f\left( {{\bf{x}};{\bf{w}}} \right)$ represents the mapping function reflecting the relationship between input data ${\bf{x}}$ and its predicted value given model parameter ${\bf{w}}$. Likewise, based on the smashed data ${{{\bf{s}}_{i,k}}}$, we denote the predicted value as ${{\hat y}_{i,k}} = f\left( {{{\bf{s}}_{i,k}};{{\bf{W}}_s}} \right)$. Therefore, the local loss function for client device $i$ is ${L_i}\left( {\bf{W}} \right) = \frac{1}{{ {{{D}_i}} }}\sum\limits_{k = 1}^{ {{{D}_i}} } {{L_{i,k}}\left( {{{\hat y}_{i,k}},{y_{i,k}}};{\bf{W}} \right)} $, where ${{L_{i,k}}\left( {{{\hat y}_{i,k}},{y_{i,k}};{\bf{W}}} \right)}$ represents the sample-wise loss function of the predicted value and the label for the $k$-th data sample in the local dataset ${\mathcal{D}_i}$. The global loss function $L\left( {\bf{W}} \right)$ is the weighted average of all participating client devices' local loss functions. The objective of SL is to seek the optimal model parameter ${{\bf{W}}^{\bf{*}}}$ for the following optimization problem to minimize the global loss function:
\begin{align}\label{minimiaze_loss_function}
\mathop {\min }\limits_{\bf{W}} L\left( {\bf{W}} \right) = \mathop {\min }\limits_{\bf{W}} \sum\limits_{i = 1}^C {\frac{{{D_i}}}{D}} {L_i}({\bf{W}}).
\end{align}
To solve~\eqref{minimiaze_loss_function}, the conventional SL employs the sequential model training to find the optimal model parameters, resulting in a dramatic increase in training latency. In view of this, we propose a state-of-the-art EPSL framework, which incorporates parallel model training and model partitioning. For readers' convenience, the important notations in this paper are summarized in Table II.

\begin{table}[t]\label{notation}
\caption{Summary of Important Notations.}
  \renewcommand{\arraystretch}{1.1}{
  \setlength{\tabcolsep}{1mm}{
\begin{tabular}{ll}
\hline
\textbf{Notation}                                                              & ~~~\textbf{Description}  \\ \hline
~~~$\mathcal{C}$ &  ~~~The set of participating client devices     \\
~~~${\mathcal{D}_i}$           &  ~~~The local dataset of client device $i$  \\
~~~$L\left( {\bf{W}} \right)$     &  ~~~The global loss function with model \\
&  ~~~parameter ${{\bf{W}}}$ \\
~~~$\mathcal{T}$     &  ~~~The set of training rounds \\
~~~${\mathcal{B}_i}$     &  ~~~The mini-batch drawn from client \\
&  ~~~device $i$'s local dataset\\
~~~$b$     &  ~~~The mini-batch size \\
~~~${{\bf{W}}_s}$           &  ~~~The server-side model  \\
~~~${{\bf{W}}_{c,i}}$           &  ~~~The client-side model at client device $i$  \\
~~~$\ell_s$/$\ell_c$           &  ~~~The number of server-side/client-side\\
&  ~~~model layers  \\
~~~${{f _s}}$/${{f _i}}$           &  ~~~The computing capability of server/client\\
&  ~~~device $i$  \\
~~~${{\kappa _s}}$/${{\kappa _i}}$            &  ~~~The computing intensity of server/client\\
&  ~~~device $i$  \\
~~~${{B_k}}$            &  ~~~The bandwidth of the $k$-th subchannel  \\
~~~$G_s$/$G_c$           &  ~~~The effective antenna gain of the server/a \\
&  ~~~client device  \\
~~~$\gamma ({F_k},{d_i})$            &  ~~~The average channel gain from client device $i$ to\\
&  ~~~the server  \\
~~~${{p^{DL}}}$            &  ~~~The transmit PSD of the server  \\
~~~${{\sigma ^2}}$            &  ~~~The PSD of the noise  \\
~~~${\rho _j}$/${\varpi _j}$           &  ~~~The FP/BP computation workload of \\
& ~~~propagating the first $j$ layer neural \\
&  ~~~network for one data sample  \\
~~~${\psi _j}$/${\chi _j}$           &  ~~~The data size of the smashed data/activations'\\
&  ~~~gradients at the cut layer $j$\\
~~~$r_i^k$, $\mu_j$, $p_k$            &  ~~~Decision variables (explained in Section~\ref{PROBLEM_FORMULATION})  \\
~~~$T_1$, $T_2$           &  ~~~Auxiliary variables (to linearize problem~\eqref{sub2})  \\\hline
\end{tabular}}}
\end{table}


\section{EPSL Framework}\label{EPSL_Framework}
In this section, we present the proposed EPSL framework. The fundamental idea of EPSL is to parallelize the client-side model training while executing last-layer gradient aggregation to reduce the training latency. Besides computation workload and activations gradients' dimension reduction, the last-layer gradient aggregation eliminates the necessity for model exchange, which facilitates the implementation of EPSL.

Before model training begins, the server initializes the ML model and partitions it into a server-side model and a client-side model via layer-wise model partitioning (the cut layer selection strategy will be elaborated in Section~\ref{Solution_Approach}), then broadcasting the client-side model to all participating client devices. After that,  EPSL is performed in consecutive training rounds until the model converges. As depicted in Fig.~\ref{EPSL}, for training round\footnote{{In each training round, we assume that every client device employs a mini-batch with $b$ data samples from its local dataset for model training.}} $t \in \mathcal{T} = \left\{ {1,2,...,T} \right\}$, the EPSL training procedure consists of the following stages.

\subsubsection{Client-side Model Forward Propagation}
At this stage, all participating client devices perform the FP for client-side models in parallel. Specifically, for client device $i$, a mini-batch ${\mathcal{B}_i} \subseteq {\mathcal{D}_i}$ with $b$ data samples is randomly drawn  from its local dataset. The input data and label of mini-batch are denoted by ${{\bf{X}}_i}\left( t \right) \in {\mathbb{R}^{b \times d}}$ and ${{\bf{y}}_i}\left( t \right) \in {\mathbb{R}^{b \times 1}}$, and ${\bf{W}}_{c,i}\left( t-1 \right)$ represents the client-side model of client device $i$. After feeding the client-side model with the input data from the mini-batch, the cut layer generates the smashed data. The smashed data of client device $i$ is given by
\begin{align}\label{stage_2}
{{\bf{S}}_i}\left( t \right) = f\left( {{{\bf{X}}_i}\left( t \right);{{\bf{W}}_{c,i}}\left( t-1 \right)} \right)\in {\mathbb{R}^{b \times q}}.
\end{align}

\subsubsection{Smashed Data Transmissions}
After the FP process for the client-side model is completed, each participating client device sends the smashed data and corresponding label to the server over the wireless channel. The smashed data collected from all participating client devices is utilized as the input to the server-side model.


\subsubsection{Server-side Model Forward Propagation}
After receiving the smashed data from participating client devices, the server concatenates all smashed data to execute FP process for the server-side model. The concatenated smashed data matrix is denoted by ${\bf{S}}\left( t \right) = \left[ {{{\bf{S}}_1}\left( t \right);{{\bf{S}}_2}\left( t \right);...;{{\bf{S}}_C}\left( t \right)} \right] \in {\mathbb{R}^{bC \times q}}$, and the predicted value is represented as
\begin{align}\label{stage_4}
{\bf{\hat y}}\left( t \right) = f\left( {{\bf{S}}\left( t \right);{{\bf{W}}_s}\left( t-1 \right)} \right) \in {\mathbb{R}^{bC \times 1}}.
\end{align}
\subsubsection{Gradient Aggregation and Server-side Model Back Propagation}


\begin{figure}[t!]
\centering
\includegraphics[width=6.8cm, height=6.2cm]{last-layer_agg.eps}
\vspace{-0cm}
\caption{An example of the last-layer gradient aggregation in EPSL with $\phi=0.5$ (e.g., half of data going through last-layer aggregation), where two client devices with two data samples participate in model training.
}% \textcolor[rgb]{1.00,0.00,0.00}{edge clouds  at or linked to the SBSs through optical fiber}
\label{last-layer_agg}
\end{figure}

At this stage, EPSL executes the last-layer gradient aggregation on the server-side model to shrink the dimension of {activations' gradients and computation workload of server, resulting in a considerable reduction in training latency. The server calculates the last-layer activations' gradients based on the loss function value, and then completes the server-side BP according to these gradients. The aggregation ratio $\phi  \in \left[ {0,1} \right] $  represents the fraction of activations' gradients that are aggregated at the last layer, i.e., $\left\lceil {\phi b} \right\rceil$ last-layer activations' gradients of every client device are aggregated in a client-wise manner before executing BP process, and the remaining $\left( {b - \left\lceil {\phi b} \right\rceil } \right)$ unaggregated gradients are directly back-propagated. One toy example is illustrated in Fig.~\ref{last-layer_agg}. It is worth noting that PSL is a special case of EPSL where $\phi = 0$. For client device $i$, the client-side model is denoted by ${{\bf{W}}_{c,i}}\left( t \right) = \left[ {{\bf{w}}_{c,i}^{{\ell _c}}\left( t \right);{\bf{w}}_{c,i}^{{\ell _c} - 1}\left( t \right);...;{\bf{w}}_{c,i}^1\left( t \right)} \right]$, where ${\bf{w}}_{c,i}^k\left( t \right)$ is the $k$-th layer in the client-side model and $\ell _c$ represents the number of client-side model layers. The server-side model is represented as ${{\bf{W}}_s}\left( t \right) = \left[ {{\bf{w}}_s^{{\ell _s}}\left( t \right);{\bf{w}}_s^{{\ell _s} - 1}\left( t \right);...;{\bf{w}}_s^{ 1}\left( t \right)} \right]$, where ${\bf{w}}_s^{k}$ is the $k$-th layer in the server-side model and $\ell _s$ denotes the number of server-side model layers. Mathematically, the gradients of server-side model is expressed as
\begin{equation}\label{stage_5_1}
{{\bf{G}}_s}\left( t \right) = \left[ {\begin{array}{*{20}{c}}
{{\bf{g}}_s^{{\ell _s}}\left( t \right)}\\
{\!\!\!{\bf{g}}_s^{{\ell _s} - 1}\left( t \right)}\!\!\!\\
{...}\\
{{\bf{g}}_s^1\left( t \right)}
\end{array}} \right],
\end{equation}
where ${\bf{g}}_s^{k}$ is the gradients of the server-side $k$-th layer, which is calculated by

\begin{equation}\label{gradient_k_layer}
{\bf{g}}_s^k \!=\! \!\!\left[ \!{\underbrace {\frac{1}{b},...,\frac{1}{b}}_{\left\lceil {\phi b} \right\rceil },\overbrace {\underbrace {\frac{{{\lambda _1}}}{b},...,\frac{{{\lambda _1}}}{b}}_{b - \left\lceil {\phi b} \right\rceil },...,\underbrace {\frac{{{\lambda _C}}}{b},...,\frac{{{\lambda _C}}}{b}}_{b - \left\lceil {\phi b} \right\rceil }}^{C\left( {b - \left\lceil {\phi b} \right\rceil } \right)}} \right]\!\!\!\left[ {\begin{array}{*{20}{c}}
{f\left( {\overline {{\bf{z}}_{s,1}^k} \left( t \right)} \right)}\\
 \vdots \\
{f\left( {\overline {{\bf{z}}_{s,{\left\lceil {\phi b} \right\rceil } }^k} \left( t \right)} \right)}\\
\!\!\!{f\left( {{\bf{z}}_{s,1,{\left\lceil {\phi b} \right\rceil }  + 1}^k\left( t \right)} \right)}\!\!\!\!\!\\
 \vdots \\
{f\left( {{\bf{z}}_{s,C,b}^k\left( t \right)} \right)}
\end{array}} \right]\!\!,
\end{equation}
 where ${\lambda _i} = \frac{{{D_i}}}{D}$, $f\left( \cdot \right)$ represents the function mapping activations' gradients to weights' gradients based on the standard BP process, $\overline {{\bf{z}}_{s,j}^k} \left( t \right)$ $\left( {j \in \left[ {1,\left\lceil {\phi b} \right\rceil} \right]} \right)$ denotes the $j$-th aggregated activations' gradients of server-side $k$-th layer, ${\bf{z}}_{s,i,{\left\lceil {\phi b} \right\rceil }  + j}^k\left( t \right)$ $\left( {j \in \left[ {1,b - \left\lceil {\phi b} \right\rceil } \right]} \right)$ represents the $j$-th unaggregated activations' gradients of server-side $k$-th layer for client device $i$, According to the BP process,  $\overline{{\bf{z}}^{k}_{s,j}}(t)=\overline{{\bf{z}}^{k+1}_{s,j}}(t) \odot\Delta_{k}^{k+1}$, ${\bf{z}}^{k}_{s,i,{\left\lceil {\phi b} \right\rceil }+j}(t)={\bf{z}}^{k+1}_{s,i,{\left\lceil {\phi b} \right\rceil }+j}(t) \odot \Delta_{k}^{k+1}$, where $\odot$ denotes the element-wise product operation, $\Delta_{k}^{k+1}$ is the derivative of the $(k+1)$-th layer activations with respect to the $k$-th layer activations in the server-side model. Regarding the vectors on the right-hand side of~\eqref{gradient_k_layer}, the first $\left\lceil {\phi b} \right\rceil$ elements correspond to aggregated activations' gradients, and the remaining elements correspond to unaggregated activations' gradients. Obviously, to obtain ${\bf{G}}_s(t)$, the key is to construct $\overline{{{\bf{z}}_{s,j}^{{\ell _s}}}}\left( t \right)$ at the last layer, which is given by
\begin{equation}\label{last_aggregation}
\overline {{\bf{z}}_{s,j}^{{\ell _s}}} \left( t \right) = \sum\limits_{i = 1}^C {{\lambda _i}} {\bf{z}}_{s,i,j}^{{\ell _s}} {\kern 1pt},
\end{equation}
where ${\bf{z}}_{s,i,j}^{{\ell _s}}$ , $\forall {j \in \left[ {1,\left\lceil {\phi b} \right\rceil} \right]} $ is the $j$-th last-layer activations' gradients of client device $i$ for aggregation. Hence, the parameter update of the server-side model is expressed as
\begin{align}\label{stage_5_2}
{{\bf{W}}_s}\left( t \right) \leftarrow {{\bf{W}}_s}\left( t-1 \right) - {\eta}_s {{\bf{G}}_s}\left( t \right),
\end{align}
where ${\eta}_s$ is the learning rate for server-side model update.

\begin{figure}[t!]
     \centering
    \subfigure[]{
         \centering
         \includegraphics[width=1.60 in,height=1.40 in]{epoch_accuracy.eps}

      \label{add_fig_1}

     }
     \subfigure[]{
         \centering
         \includegraphics[width=1.6 in,height=1.4 in]{round_comparison.eps}

      \label{add_fig_2}
    }
%
\caption{The test accuracy (a) and per-round latency (b) of EPSL, PSL, SFL, and vanilla SL on HAM 10000 under IID settings using ResNet-18 with $C=5$ client devices. As shown in the figures, EPSL achieves similar accuracy compared with other benchmarks with the same number of rounds, yet with much shorter per-round latency.}
\label{additional_fig}
\end{figure}


The last-layer gradient aggregation lies in the heart of EPSL, which remarkably decreases the communication and computing overhead in the BP process thereby boosting the efficiency in resource-constrained systems. Moreover, since the server broadcasts the aggregated gradients to all clients, the client-side models learn the shared features from each other, preventing overfitting without requiring model exchange as in FL or SFL. To better interpret EPSL, traditional training (e.g., PSL) conducts BP first and then average the gradients, while our approach averages the last-layer activations' gradients first and then conducts the remaining BP. Due to the non-linearity of activation functions, these two are not equivalent. However, our approach might still be a good approximation as popular activation functions (e.g., ReLU and Sigmoid) are linear or approximately linear in a certain interval. As shown in Fig.~\ref{additional_fig}, despite the significant per-round latency reduction, we find that EPSL with $\phi=0.5$ and $\phi=1$ achieve a similar learning accuracy versus the number of rounds as compared with vanilla SL, SFL and PSL.



%Despite the significant per-round latency reduction, we find that EPSL with $\phi=1$ achieve a similar learning accuracy %versus the number of rounds as compared with SL and SFL when the number of clients is small. When the number of clients is %large, EPSL with $\phi=1$ cannot converge well, because the aggregated local gradients could be too "general" to fit each %local dataset. However, as we decrease aggregation ratio $\phi$, the performance again becomes comparable to SL and SFL. %This demonstrates the effectiveness of last-layer gradient aggregation.


\subsubsection{Aggregated Activations' Gradients Downlink Broadcasting}
When the BP process reaches the cut layer, the server broadcasts the aggregated activations' gradients (namely, the gradients that are generated at the cut layer by performing BP on the last-layer aggregated activations' gradients) to all participating client devices over the shared wireless channel.

\subsubsection{Unaggregated Activations' Gradients Transmissions}
After the aggregated activations' gradients downlink broadcasting is completed, the server sends the unaggregated activations' gradients (i.e., the gradients that are obtained at the cut layer by executing BP on the last-layer unaggregated activations' gradients.) to corresponding client devices over the wireless channel.

\subsubsection{Client-side Model Back Propagation}
At this stage, each client device completes the parameter update of its client-side model based on the received activations' gradients. The client-side gradients for client device $i$ can be calculated
\begin{align}\label{stage_7_1}
{{\bf{G}}_{c,i}}\left( t \right) = \left[ {\begin{array}{*{20}{c}}
{{\bf{g}}_{c,i}^{{\ell _c}}\left( t \right)}\\
{{\bf{g}}_{c,i}^{{\ell _c} - 1}\left( t \right)}\\
{...}\\
{{\bf{g}}_{c,i}^1\left( t \right)}
\end{array}} \right],
\end{align}
where ${\bf{g}}_{c,i}^{k}$ represents the gradients of the client-side $k$-th layer of client device $i$, which is given by
\begin{align}\label{client_k_gradient}
{\bf{g}}_{c,i}^k = \Bigg[ {\underbrace {\frac{1}{b},...,\frac{1}{b}}_b} \Bigg]\left[ {\begin{array}{*{20}{c}}
{f\left( {\overline {{\bf{z}}_{c,1}^k} \left( t \right)} \right)}\\
 \vdots \\
{f\left( {\overline {{\bf{z}}_{c,{\left\lceil {\phi b} \right\rceil } }^k} \left( t \right)} \right)}\\
{f\left( {{\bf{z}}_{c,i,{\left\lceil {\phi b} \right\rceil }  + 1}^k\left( t \right)} \right)}\\
 \vdots \\
{f\left( {{\bf{z}}_{c,i,b}^k\left( t \right)} \right)}
\end{array}} \right],
\end{align}
where $\overline {{\bf{z}}_{c,j}^k} \left( t \right) = \overline {{\bf{z}}_{c,j}^{k + 1}} \left( t \right)\odot\Theta _{k,j}^{k + 1}$ $\left( {j \in \left[ {1,\left\lceil {\phi b} \right\rceil} \right]} \right)$ denotes the $j$-th aggregated activations' gradients of the client-side $k$-th layer for client device $i$, ${\bf{z}}_{c,i,{\left\lceil {\phi b} \right\rceil }  + j}^k\left( t \right) = {\bf{z}}_{c,i,{\left\lceil {\phi b} \right\rceil }  + j}^{k + 1}\left( t \right)\odot\Theta _k^{k + 1}$ $\left( {j \in \left[ {1,b - \left\lceil {\phi b} \right\rceil } \right]} \right)$ represents the $j$-th unaggregated activations' gradients of the client-side $k$-th layer for client device $i$, $\Theta _k^{k + 1}$ is derivative of the $\left(k + 1\right)$-th layer activations with respect to the $k$-th layer activations in the client-side model. $\overline {{\bf{z}}_{c,j}^{{\ell _c}}} \left( t \right)$ and ${{\bf{z}}_{c,i,b}^{\ell _c}\left( t \right)}$ can be calculated by
\begin{align}\label{z_c_j}
\overline {{\bf{z}}_{c,j}^{{\ell _c}}} \left( t \right) = \overline {{\bf{z}}_{s,j}^1} \left( t \right)\odot\Psi,
\end{align}
\begin{align}\label{z_c_i_j}
{\bf{z}}_{c,i,{\left\lceil {\phi b} \right\rceil }  + j}^{{\ell _c}}\left( t \right) = {\bf{z}}_{s,i,{\left\lceil {\phi b} \right\rceil }  + j}^1\odot\Psi,
\end{align}
where $\Psi$ represents derivative of the server-side first layer activations with respect to the client-side ${\ell _c}$-th layer activations. Therefore, for client device $i$, the client-side model is updated through
\begin{align}\label{stage_7_2_}
{{\bf{W}}_{c,i}}\left( t \right) \leftarrow {{\bf{W}}_{c,i}}\left( t-1 \right) - {\eta _c}{{\bf{G}}_{c,i}}\left( t \right),
\end{align}
where ${\eta}_c$ is the learning rate for the client-side model update.

The EPSL training framework is outlined in~\textbf{Algorithm~\ref{EPSL_procedure}}.


\begin{algorithm}[h]
  \caption{The EPSL training framework}\label{EPSL_procedure}
  \begin{algorithmic}[1]
    \Require
        $b$, ${\eta}_c$, ${\eta}_s$, ${\cal C}$, $\cal{D}$, $\phi$, $f_k$ and $B_k$.
    \Ensure
        ${{\bf{W}}^{\bf{*}}}$
    \State Initialization: ${{\bf{W}}}\left( 0 \right)$
    \For{$t=1,2,...T$ }
    \State
    \State /** {Runs} {on} {client} {devices} **/
        \ForAll {client device ${i \in \,{\cal C}}$ in parallel}
        \State ${{\bf{S}}_i}\left( t \right) \leftarrow f\left( {{{\bf{X}}_i}\left( t \right);{{\bf{W}}_{c,i}}\left( t-1 \right)} \right)$
        \State Send $\left( {{\bf{S}}_i\left( t \right),{\bf{y}}_i\left( t \right)} \right)$ to the server
        \EndFor
    \State
    \State /** {Runs} {on} {server} **/
        \State ${\bf{S}}\left( t \right) \leftarrow \left[ {{{\bf{S}}_1}\left( t \right);{{\bf{S}}_2}\left( t \right);...;{{\bf{S}}_C}\left( t \right)} \right] $
        \State ${\bf{y}}\left( t \right) \leftarrow \left[ {{{\bf{y}}_1}\left( t \right);{{\bf{y}}_2}\left( t \right);...;{{\bf{y}}_C}\left( t \right)} \right] $
        \State ${\bf{\hat y}}\left( t \right) \leftarrow f\left( {{\bf{S}}\left( t \right);{{\bf{W}}_s}\left( t-1 \right)} \right) $
        \State Calculate loss function value $L\left( {{\bf{W}}\left( {t - 1} \right)} \right)$
        \State Calculate last-layer aggregated activations' gradients
        \Statex $\;\;\;\;\;\;{\overline {{\bf{z}}_{s,j}^{{\ell _{s}}}} }$
        \State Calculate gradients of server-side model  ${\bf{G}}_s\left( t \right)$
        \State ${{\bf{W}}_s}\left( t \right) \leftarrow {{\bf{W}}_s}\left( t-1 \right) - {\eta}_s {{\bf{G}}_s}\left( t \right)$
        \State Broadcast aggregated activations' gradients in~\eqref{stage_5_1} to
        \Statex $\;\;\;\;\;\;$all client devices
        \State Send unaggregated activations' gradients in~\eqref{stage_5_1} to c-
        \Statex $\;\;\;\;\;\;$orresponding client devices
    \State
    \State /** {Runs} {on} {client} {devices} **/
        \ForAll {client device ${i \in \,{\cal C}}$ in parallel}
        \State Calculate gradients of client-side model ${{\bf{G}}_{c,i}}\left( t \right)$
        \State ${{\bf{W}}_{c,i}}\left( t \right) \leftarrow {{\bf{W}}_{c,i}}\left( t-1 \right) - {\eta _c}{{\bf{G}}_{c,i}}\left( t \right)$
        \EndFor

    \EndFor

  \end{algorithmic}
\end{algorithm}



\section{Resource Management For EPSL}\label{PROBLEM_FORMULATION}
This section presents the formulation of the resource management problem, with the objective of minimizing the per-round latency. Since the server trains the server-side model after collecting the smashed data from all participating client devices, the heterogeneous channel conditions and computing capabilities at the client devices  result in a severe straggler effect. To decrease the excessive training latency, we devise an effective resource management and layer split strategy that jointly optimizes subchannel allocation, power control, and cut layer selection. For clarity, the decision variables and definitions are listed below.

\begin{itemize}
\item ${\bf{r}}$: $r_i^k \in \left\{ {0,1} \right\}$ denotes the subchannel allocation variable, where $r_i^k=1$ indicates that the $k$-th subchannel is allocated to client device $i$, and 0 otherwise. The subchannel allocation vector for client device $i$ is represented as ${{\bf{r }}_i} = \left[ {r _i^1,r _i^2,...,r _i^M} \right]$, where $M$ is the total number of subchannels. Therefore, ${\bf{r}} = \left[ {{{\bf{r}}_1},{{\bf{r}}_2},...,{{\bf{r}}_C}} \right]$ denotes the collection of subchannel allocation decisions.
\item ${\bf{p}}$: $p_k$ represents the transmit power spectral density (PSD) of the $k$-th subchannel. The power control decision is denoted by ${\bf{p}} = \left[ {{p_1},{p_2},...,{p_M}} \right]$.
\item $\bm{\mu}$: $\mu_j \in \left\{ {0,1} \right\}$ is the cut layer selection variable, where $\mu_j=1$ indicates that the $j$-th neural network layer is selected as the cut layer, and 0 otherwise.
$\bm{\mu}  = \left[ {{\mu _1},{\mu _2},...,{\mu _{{\ell_c} + {\ell_s}}}} \right]$ represents the collection of cut layer selection decisions.
\end{itemize}


\subsection{Training Latency Computation}
EPSL is executed in consecutive training rounds until model converges. Without loss of generality, we focus on one training round for analysis. For notational simplicity, index $t$ of training round number is omitted. We conduct the detailed latency analysis of the seven stages in one training round, as follows.



\subsubsection{Client-side Model Forward Propagation Latency}
At this stage, all participating client devices perform the FP process of client-side models in parallel. Let $\Phi_c^{F}\! \left( \bm{\mu}  \right) = \sum\limits_{j = 1}^{{\ell_c} + {\ell_s}} {{\mu _j}} {\rho _j}$ denote the computation workload (in FLOPs) of the client-side FP process for each data sample, where ${\rho _j}$ is the FP computation workload of propagating the first $j$ layer neural network for one data sample. Since each client device randomly draw a mini-batch with $b$ data samples to execute the client-side FP process, the client device $i$'s FP latency is calculated as
\begin{align}\label{client_FP_latency}
T_i^{F} = \frac{{b\,{\kappa _i}\Phi_c^{F}\! \left( \bm{\mu}  \right)}}{{{f_i}}} ,\;\forall i \in \mathcal{C},
\end{align}
where ${{f _i}}$ represents the computing capability (namely, the number of  central processing unit (CPU) cycles per second) of client device $i$, and ${{\kappa _i}}$ is the computing intensity (i.e., the number of CPU cycles required to complete one float-point operation) of client device $i$, which is determined by the CPU architecture.

\subsubsection{Smashed Data Transmission Latency}
After completing the FP process of the client-side models, each participating client device sends the smashed data to the server over the wireless channel. At each training round, we consider frequency-division multiple access for data transmission. Let $\Gamma_s \left( \bm{\mu}  \right) = \sum\limits_{j = 1}^{{\ell_c} + {\ell_s}} {{\mu _j}} {\psi _j}$ represent the data size (in bits) of the smashed data, where ${\psi _j}$ denotes the data size of the smashed data at the cut layer $j$. Therefore, the uplink transmission rate from client device $i$ to the server is given as
\begin{align}\label{smashed_rate}
R_i^U = \sum\limits_{k = 1}^M {r_i^k{B_k}{{\log }_2}\left( {1 + \frac{{{p_k}{G_c}{G_s}\gamma ({F_k},{d_i})}}{{{\sigma ^2}}}} \right)} ,\forall i \in {\cal C},
\end{align}
where ${{B_k}}$ is the bandwidth of the $k$-th subchannel, $G_c$ and $G_s$ denote the effective antenna gain of a client device and the server, $\gamma(F_k, d_i)$ represents the average channel gain from client device $i$ to the server, and ${{F_k}}$, ${d_i}$, and ${{\sigma ^2}}$ represent the center frequency of the $k$-th subchannel, the communication distance between client device $i$ and the server, and the PSD of the noise, respectively. Due to the utilization of the mini-batch with $b$ data samples, the smashed data transmission latency for client device $i$ is determined as
\begin{align}\label{smashed_trans_latency}
T_i^{U} = \frac{{b\Gamma_s \left( \bm{\mu}  \right)}}{{R_i^{U}}},\;\forall i \in \mathcal{C}.
\end{align}

\subsubsection{Server-side Model Forward Propagation Latency}
This stage involves executing the server-side FP process\footnote{{The server can execute FP process for multiple client devices in either a serial or parallel fashion, where the latency in~\eqref{server_FP_latency} would not be affected.}} with smashed data gathered from all participating client devices. Let $\Phi _s^{F}\left( \bm{\mu}  \right) = \!\sum\limits_{j = 1}^{{\ell_c} + {\ell_s}} {{{\mu} _j}} \left( {{\rho _{{\ell_c} + {\ell_s}}} - {\rho _j}} \right)$ denote the computation workload of the server-side FP process for each data sample. Since the server concatenates the smashed data of all participating client devices, the data size of the smashed data is represented as $Cb\Phi _s^{F}\left( \bm{\mu}  \right)$. As a result, the server-side model FP latency can be obtained from
\begin{align}\label{server_FP_latency}
T_s^{F} = \frac{{Cb\,{\kappa _s}\Phi _s^{F}\left( \bm{\mu}  \right)}}{{{f_s}}},
\end{align}
where ${{f _s}}$ denotes the computing capability of the server, and ${{\kappa _s}}$ represents the computing intensity of the server.


\subsubsection{Server-side Model Back Propagation Latency}
After the server-side FP process is completed, EPSL first executes the last-layer BP process to generate the last-layer activations' gradients. Then, EPSL performs last-layer activations' gradient aggregation on the dimension of client devices. In other words, each client device employs $\left\lceil {\phi b} \right\rceil$ out of $b$ of its last-layer activations' gradients for aggregation with other client devices, after which the aggregated activations' gradients go through the BP process. The remaining $\left( {b - \left\lceil {\phi b} \right\rceil } \right)$ unaggregated gradients are directly back-propagated. Let $\Phi _s^{B}\left( \bm{\mu}  \right) = \sum\limits_{j = 1}^{{\ell_c} + {\ell_s}-1} {{\mu _j}} \left( {{\varpi _{{\ell_c} + {\ell_s}-1}} - {\varpi _j}} \right)$ denote the computation workload of the server-side BP process for each data sample (excluding the last layer), where ${\varpi _j}$ is the BP computation workload of propagating the first $j$ layer neural network for one data sample. The computation workload of the last-layer BP process for each data sample is represented as $\Phi _s^L = {\varpi _{{\ell_c} + {\ell_s}}} - {\varpi _{{\ell_c} + {\ell_s} - 1}}$. The computation workload of the last-layer BP is denoted by $Cb\Phi _s^L$, as it is performed on $Cb$ data samples. Similarly, the BP computation workloads for aggregated and unaggregated activations' gradients are represented as $\left\lceil {\phi b} \right\rceil \Phi _s^B\left( \bm{\mu}  \right)$ and $C\left( {b - \left\lceil {\phi b} \right\rceil } \right)\Phi _s^B\left( \bm{\mu}  \right)$, respectively. By ignoring the last-layer aggregation time, the server-side model BP latency is expressed as
\begin{align}\label{server_BP_latency}
T_s^B \!= \!\frac{{\left( {\left\lceil {\phi b} \right\rceil \! + \!C\!\left( {b - \left\lceil {\phi b} \right\rceil } \right)} \right){\kappa _s}\Phi _s^B\left( {\bm{\mu} } \right) + Cb{\kappa _s}\Phi _s^L}}{{{f_s}}}.
\end{align}

\subsubsection{Aggregated Activations' Gradients Downlink Broadcasting Latency}
When the BP process reaches the cut layer, the server broadcasts the aggregated activations' gradients to all participating client devices over the shared wireless channel. Let $\Gamma_g \left( \bm{\mu}  \right) = \sum\limits_{j = 1}^{{\ell_c} + {\ell_s}} {{\mu _j}} {\chi _j}$ represent the data size of activations' gradients, where the data size of activations' gradients at the cut layer $j$ is denoted by ${\chi _j}$. The downlink broadcasting data rate from the server to every client device is the same and equal to
\begin{align}\label{server_trans_rate}
{R^B} = \sum\limits_{k = 1}^M {{B_k}{{\log }_2}\left( {1 + \frac{{{p^{DL}}{G_c}{G_s}{\gamma _w}}}{{{\sigma ^2}}}} \right)} ,
\end{align}
where ${\gamma _w} = {\min _{i \in \mathcal{C}, k = 1,2...M}}\left\{ {\gamma ({F_k},{d_i})} \right\}$ represents the weakest average channel gain from client devices to the server, the transmit PSD of the server is denoted by ${{p^{DL}}}$. Hence, the aggregated activations' gradients downlink broadcasting latency can be obtained by
\begin{align}\label{server_trans_latency}
{T^B} = \frac{{\left\lceil {\phi b} \right\rceil {\Gamma _g}\left( {\bm{\mu} } \right)}}{{{R^B}}}.
\end{align}

\subsubsection{Unaggregated Activations' Gradients Transmission Latency}
After the aggregated activations' gradients downlink broadcasting is completed, the server sends the unaggregated activations' gradients to the corresponding client devices over the wireless channel. The downlink transmission rate from the server to client device $i$ is given as
\begin{align}\label{downlink_rate}
R_i^{D} \!=\!\! \sum\limits_{k = 1}^M {r_i^k{B_k}{{\log }_2}\!\!\left( {1 + \frac{{{p^{DL}}G_c G_s { {{\gamma ({F_k},{d_i})}}}}}{{{\sigma ^2}}}} \right)} ,\forall i \in \mathcal{C}.
\end{align}

Therefore, the unaggregated activations' gradients transmission latency for client device $i$ is determined as
\begin{align}\label{downlink_latency}
T_i^D = \frac{{\left( {b - \left\lceil {\phi b} \right\rceil } \right){\Gamma _g}\left( {\bm{\mu} } \right)}}{{R_i^D}},\;\forall i \in {\cal C}.
\end{align}


\subsubsection{Client-side Model Back Propagation Latency}
At this stage, each client device performs BP on the client-side model based on the received activations' gradients. Let $\Phi _c^{B}\left( \bm{\mu}  \right) = \sum\limits_{j = 1}^{{\ell_c} + {\ell_s}} {{\mu _j}} {\varpi _j}$ represent the computation workload of the client-side BP process for one data sample. As a result, the client-side BP latency  for client device $i$  can be obtained from
\begin{align}\label{client_BP_latency}
T_i^{B} = \frac{{b\,{\kappa _i}\Phi _c^{B}\left( \bm{\mu}  \right)}}{{{f_i}}},\;\forall i \in \mathcal{C}.
\end{align}


\subsection{Resource Management Problem Formulation}
As shown in Fig.~\ref{Time_consumption}, the total latency for one training round can be derived as
\begin{align}\label{total_time}
T\!\left( {{\bf{r}},{\bm{\mu }},{\bf{p}}} \right) \!= \!\max_i\! \left\{ {\;\!\!T_i^F\! \!+\! T_i^U\!} \right\}\!+\! T_s^F \!\!+ \!T_s^B \!\!+ \!T^B\! \!\!+\!\max_i \!\left\{ { T_i^D\!\!+\!T_i^B}\right\},
\end{align}


\begin{figure}[t!]
\centering
\includegraphics[width=8.5cm, height=6.2cm]{Time_consumption.eps}
\vspace{-3.2cm}
\caption{An illustration of EPSL training procedure for one training round.
}% \textcolor[rgb]{1.00,0.00,0.00}{edge clouds  at or linked to the SBSs through optical fiber}
\label{Time_consumption}
\end{figure}

Apparently, the heterogeneous channel conditions and computing capabilities at client devices may result in  a severe straggler issue. The cut layer selection significantly impacts the communication and computing latency, and appropriate radio resource management is crucial to accelerate the training process. Observing these facts, we formulate the following optimization problem to minimize the per-round latency:
\begin{align}\label{time_minimize_problem}
\mathcal{P}:&\mathop {{\rm{min}}}\limits_{{\bf{r}},{\bm{\mu }},{\bf{p}}} T({\bf{r}},{\bm{\mu }},{\bf{p}})   \\
&\mathrm{s.t.} ~\mathrm{C1:}~r_i^k \in \left\{ {0,1} \right\},\quad \forall i \in {\cal C},k = 1,2,...,M, \nonumber \\
&~\mathrm{C2:}~\sum\limits_{i = 1}^C {r_i^k}  = 1,\quad k = 1,2,...,M, \nonumber \\
&~\mathrm{C3:}~{\mu _j} \in \left\{ {0,1} \right\},\quad j = 1,2,...,{\ell_c} + {\ell_s}, \nonumber\\
&~\mathrm{C4:}~\sum\limits_{j = 1}^{{\ell_c} + {\ell_s}} {{\mu _j}}  = 1, \nonumber \\
&~\mathrm{C5:}~\sum\limits_{k = 1}^M {r_i^k{p_k}{B_k} \le {p_i^{\max }}}, \quad \forall i \in {\cal C}, \nonumber\\
&~\mathrm{C6:}~\sum\limits_{k = 1}^M {\sum\limits_{i = 1}^C {r_i^k{p_k}{B_k}} }  \le {p_{th}}, \nonumber\\
&~\mathrm{C7:}~{p_k} \ge 0,\quad k = 1,2,...,M. \nonumber
\end{align}

Constraints C1 and C2 ensure that each subchannel is solely allocated to one client device to avoid co-channel interference; C3 and C4 guarantee the uniqueness of cut layer selection, so that the global model is partitioned into the client-side model and the server-side model; C5 is the transmit power constraint of each client device; C6 illustrates the total uplink transmit power limitation with the threshold $p_{th}$, which is conducive to interference management~\cite{zuo2017power}; C7 ensures that ${p_k}$ is a non-negative value. Note that we consider a stationary network where average link gains vary slowly. Therefore, the cut layer decision, once determined, could last for a long period. We will show the robustness of layer split to channel variation in our simulations. The consideration of mobility for split learning could be left as future work.

Problem~\eqref{time_minimize_problem} is a combinatorial problem with a non-convex mixed-integer nonlinear objective function, which is NP-hard in general. Thus, it is impractical to employ a polynomial-time algorithm to obtain the optimal solution. Therefore, we propose an efficient suboptimal algorithm in the next section.


\section{Solution Approach}\label{Solution_Approach}
To solve problem~\eqref{time_minimize_problem}, we fix the decision variables ${\bm{\mu }}$, ${\bf{p}}$ and consider the subproblem involving subchannel allocation, which is expressed as
\begin{align}\label{sub1}
\mathcal{P}1:&~\mathop {{\rm{min}}}\limits_{{\bf{r}}} T({\bf{r}})   \\
&\mathrm{s.t.} ~\mathrm{C1:}~r_i^k \in \left\{ {0,1} \right\},\quad \forall i \in {\cal C},k = 1,2,...,M, \nonumber \\
&~\mathrm{C2:}~\sum\limits_{i = 1}^C {r_i^k}  = 1,\quad k = 1,2,...,M, \nonumber \\
&~\mathrm{C5:}~\sum\limits_{k = 1}^M {r_i^k{p_k}{B_k} \le {p_i^{\max }}}, \quad \forall i \in {\cal C}, \nonumber\\
&~\mathrm{C6:}~\sum\limits_{k = 1}^M {\sum\limits_{i = 1}^C {r_i^k{p_k}{B_k}} }  \le {p_{th}}. \nonumber
\end{align}

The optimization objective of problem~\eqref{sub1} is still non-convex and contains integer variables,  making it challenging to find the optimal solution. In view of this, we devise a greedy subchannel allocation approach. The core idea of the proposed approach is to allocate subchannel to the straggler with the longest latency of  $T_i^{F}+T_i^{U}$ or $T_i^{D}+T_i^{B}$ (other latency terms are the same for all client devices). This is because the objective function of problem~\eqref{sub1} is determined by the stragglers' latency in the first and last two stages of one training round, and we prioritize allocating subchannel to the straggler with the longer training latency. To be more specific, we first allocate the subchannel with better propagation characteristics (i.e., lower center frequency) to the client device with less powerful computing capability. After each client device attains one channel, the remaining idle channels will be iteratively allocated to the straggler at each iteration until all subchannels are occupied. The procedure of the tailored greedy subchannel allocation approach is described in~\textbf{Algorithm~\ref{greedy-based}}.

\begin{algorithm}[h]
  \caption{Greedy subchannel allocation approach}\label{greedy-based}
  \begin{algorithmic}[1]
    \Require
        $\mathcal{C}$, $F_k$, $B_k$
    \Ensure
        ${\bf{r}^*}$

    \State Initialization: Set  ${\bf{r}} \leftarrow {\bf{0}}$, $\mathcal{A}_1, \mathcal{A}_2 \leftarrow \mathcal{C}$, $\mathcal{K} \leftarrow \left\{ {1,2,...,M} \right\}$
    \For{$j=1,2,...,N$ }
    \State Find $n \leftarrow \arg {\min _{i \in {\cal A}_1}}\left\{ {f_i} \right\}$
    \State Find $m \leftarrow \arg {\min _{k \in \mathcal{K}}}\left\{ {\frac{{{F_k}}}{{{B_k}}}} \right\}$
    \State Let $r_n^m\leftarrow1$, $\mathcal{A}_1 \leftarrow \mathcal{A}_1 - \left\{ n \right\}$, $\mathcal{K}\leftarrow \mathcal{K} - \left\{ m \right\}$
    \State Update $ T_n^{U}$, $ T_n^{D}$
    \EndFor
    \While {$\mathcal{K} \ne \varnothing$}
    \State Find $n_1 \leftarrow \arg {\max _{i \in {\cal A}_2}}\left\{ {T_i^{F} \!+ \!T_i^{U}} \right\}$
    \State Find $n_2 \leftarrow \arg {\max _{i \in {\cal A}_2}}\left\{ {T_i^{D} \!+ \!T_i^{B}} \right\}$
    \State Find $n \leftarrow \arg {\max _{i \in \left\{ {{n_1},{n_2}} \right\}}}\left\{ {T_i^F\! + \!T_i^U\! +\! T_i^D \!+\! T_i^B} \right\}$
    \State Find $m \leftarrow \arg {\min _{k \in \mathcal{K}}}\left\{ {\frac{{{F_k}}}{{{B_k}}}} \right\}$, $r_n^m\leftarrow1$
    \If {${{\bf{r}}_n}$ does not meet C5 in~\eqref{sub1}}
    \State Let $r_n^m\leftarrow0$, $\mathcal{A}_2 \leftarrow \mathcal{A}_2 - \left\{ n \right\}$
    \Else
    \State Update $ T_n^{U}$, $ T_n^{D}$, $\mathcal{K} \leftarrow \mathcal{K} - \left\{ m \right\}$
    \EndIf
    \EndWhile
  \end{algorithmic}
\end{algorithm}


After fixing subchannels according to the aforementioned subchannel allocation approach, we investigate the problem of power control and cut layer selection based on~\eqref{time_minimize_problem}. For notational simplicity, we introduce the substitution variable sets ${\bf{\widetilde p}}\! = \left\{ {{p_{1,1}},{p_{1,2}},...,{p_{C,{M_C}}}} \right\}$ and ${\bf{\widetilde B}} \! = \left\{ {{B_{1,1}},{B_{1,2}},...,{B_{C,{M_C}}}} \right\}$ to represent ${\bf{p}}$ and ${\bf{B}}\!  = \left\{ {{B_1},{B_2}...,{B_M}} \right\}$, respectively. The considered problem is then given by
\begin{align}\label{sub2}
&~\mathop {{\rm{min}}}\limits_{{\bm{\mu }},{{{\bf{\widetilde p}}}}} T({\bm{\mu }},{{\bf{\widetilde p}}})    \\
&\mathrm{s.t.} ~\mathrm{C3:}~{\mu _j} \in \left\{ {0,1} \right\},\quad j = 1,2,...,{\ell_c} + {\ell_s}, \nonumber\\
&~\mathrm{C4:}~\sum\limits_{j = 1}^{{\ell_c} + {\ell_s}} {{\mu _j}}  = 1, \nonumber \\
&~\mathrm{C5:}~\sum\limits_{\xi  = 1}^{{M_i}} {{p_{i,\xi }}{B_{i,\xi }} \le p_i^{\max }}, \quad \forall i \in {\cal C}, \nonumber\\
&~\mathrm{C6:}~\sum\limits_{i = 1}^C {\sum\limits_{\xi  = 1}^{{M_i}} {{p_{i,\xi }}{B_{i,\xi }}} }  \le {p_{th}}, \nonumber\\
&~\mathrm{C7:}~{p_{i,\xi }} \ge 0,\quad \forall i \in {\cal C}, \xi=1,2,...,M_i, \nonumber
\end{align}
where $M_i$ represents the number of subchannels allocated to client device $i$, ${p_{i,\xi }}$ denotes the transmit PSD on the $\xi$-th subchannel for client device $i$, and ${B_{i,\xi }}$ is the bandwidth of $\xi$-th subchannel for client device $i$.

Problem~\eqref{sub2} is a mixed-integer nonlinear programming. The challenges in addressing this problem stem primarily from the nonconvexity of the objective function. In order to linearize the objective function, we introduce the auxiliary variables ${T_1}$ and ${T_2}$ that satisfy ${T_1} \ge \mathop {\max }\limits_i \left\{ {\;T_i^F + T_i^U} \right\}$ and ${T_2} \ge \mathop {\max }\limits_i \left\{ {T_i^D + T_i^B} \right\}$, respectively. The problem~\eqref{sub2} can be converted into
\begin{align}\label{sub2_convert}
&~\mathop {{\rm{min}}}\limits_{{\bm{\mu }},{\bf{\widetilde p}},{T_1},{T_2}} \widetilde{T}({\bm{\mu }},{\bf{\widetilde p}},{T_1},{T_2})\! \\
&\mathrm{s.t.} ~\mathrm{C3:}~{\mu _j} \in \left\{ {0,1} \right\},\quad j = 1,2,...,{\ell_c} + {\ell_s}, \nonumber\\
&~\mathrm{C4:}~\sum\limits_{j = 1}^{{\ell_c} + {\ell_s}} {{\mu _j}}  = 1, \nonumber \\
&~\mathrm{C5:}~\sum\limits_{\xi  = 1}^{{M_i}} {{p_{i,\xi }}{B_{i,\xi }} \le p_i^{\max }}, \quad \forall i \in {\cal C}, \nonumber\\
&~\mathrm{C6:}~\sum\limits_{i = 1}^C {\sum\limits_{\xi  = 1}^{{M_i}} {{p_{i,\xi }}{B_{i,\xi }}} }  \le {p_{th}}, \nonumber\\
&~\mathrm{C7:}~{p_{i,\xi }} \ge 0,\quad \forall i \in {\cal C},\xi=1,2,...,M_i, \nonumber\\
&~\mathrm{C8:}~\frac{{b{\mkern 1mu} {\kappa _i}\!\!\!\sum\limits_{j = 1}^{{\ell_c} + {\ell_s}} {\!\!\!{\mu _j}} {\rho _j}}}{{{f_i}}}\! +\!\frac{{b\!\!\sum\limits_{j = 1}^{{\ell_c} + {\ell_s}} {\!\!{\mu _j}} \psi_j }}{{\sum\limits_{\xi  = 1}^{{M_i}} \!{\!{B_{i,\xi }}} {\rm{lo}}{{\rm{g}}_2}\!\left( {1\! + \!\frac{{{p_{i,\xi }}G_c G_s { {\gamma ({F_k},{d_i})} }}}{{{\sigma ^2}}}} \right)}} \!\le\! {T_1},\nonumber\\
&{\kern 225pt}\forall i \in {\cal C}, \nonumber\\
&~\mathrm{C9:}~\frac{{\left( {b - \left\lceil {\phi b} \right\rceil } \right)\sum\limits_{j = 1}^{{\ell _c} + {\ell _s}} {{\mu _j}} {\chi _j}}}{{R_i^D}} + \frac{{b{\kappa _i}\sum\limits_{j = 1}^{{\ell _c} + {\ell _s}} {{\mu _j}} {\varpi _j}}}{{{f_i}}} \le {T_2},\quad \!\!\forall i \in {\cal C}, \nonumber
\end{align}
where $\widetilde{T}({\bm{\mu }},{\bf{\widetilde p}},{T_1},{T_2})\!={{T_1} + T_s^F + T_s^B + T^B + {T_2}}$, the additional constraint C8 and C9 are established based on ${T_1} \ge \mathop {\max }\limits_i \left\{ {\;T_i^F + T_i^U} \right\}$ and ${T_2} \ge \mathop {\max }\limits_i \left\{ {T_i^D + T_i^B} \right\}$.

The transformed problem~\eqref{sub2_convert} is equivalent to the original problem~\eqref{sub2}, as the optimal solution ${T_1^*}$ and ${T_2^*}$ obtained from problem~\eqref{sub2_convert} must meet $T_1^* = \mathop {\max }\limits_i \left\{ {\;T_i^F + T_i^U} \right\}$ and $T_2^* = \mathop {\max }\limits_i \left\{ {\;T_i^D + T_i^B} \right\}$. Otherwise, there must be ${T'_1}$ and ${T'_2}$ satisfying $\mathop {\max }\limits_i \left\{ {\;T_i^F + T_i^U} \right\} \le {{T'}_1} < T_1^*$ and $\mathop {\max }\limits_i \left\{ {\;T_i^D + T_i^B} \right\} \le {{T'}_2} < T_2^*$ with lower objective function value. Even though the objective function of problem~\eqref{sub2_convert} is linear, the nonconvexity of constraint C8 makes solving the problem challenging. Therefore, we introduce a set of substitution variables $\bm{\theta}   = \left\{ {{\theta _{1,1}},{\theta _{1,2}},...,{\theta _{N,{M_N}}}} \right\}$, which are determined as
\begin{align}\label{substitution variables}
{\theta _{i,\xi }} \!=\! {B_{i,\xi }}{\rm{lo}}{{\rm{g}}_2}\!\!\left( {\!1 \!+ \!\frac{{{p_{i,\xi }}G_c G_s {\gamma ({F_k},{d_i})}}}{{{\sigma ^2}}}}\right)\!.
\end{align}

We derive the feasible region of substitution variables as ${\theta_{i,\xi }} \ge 0$  based on transmit PSD's feasible region ${p_{i,\xi }} \ge 0$. By substituting the variable $\bm{\theta}$ and its corresponding feasible region into the problem~\eqref{sub2_convert}, the further transformed problem is given as
\begin{align}\label{sub2_further_convert}
&~\mathop {{\rm{min}}}\limits_{{\bm{\mu }},{\bm{\theta}},{T_1},{T_2}} \widetilde{T}({\bm{\mu }},{\bm{\theta}},{T_1},{T_2})  \\
&\mathrm{s.t.} ~\mathrm{C3:}~{\mu _j} \in \left\{ {0,1} \right\},\quad j = 1,2,...,{\ell_c} + {\ell_s}, \nonumber\\
&~\mathrm{C4:}~\sum\limits_{j = 1}^{{\ell_c} + {\ell_s}} {{\mu _j}}  = 1, \nonumber \\
&~\mathrm{\widetilde{C}5:}~\sum\limits_{\xi  = 1}^{{M_i}} {{\sigma ^2}{B_{i,\xi }}\frac{{{2^{\frac{{{\theta _{i,\xi }}}}{{{B_{i,\xi }}}}}} - 1}}{{G_c G_s {\gamma ({F_k},{d_i})}}} \le p_i^{\max }},  \quad \forall i \in {\cal C}, \nonumber\\
&~\mathrm{\widetilde{C}6:}~\sum\limits_{i = 1}^C {\sum\limits_{\xi  = 1}^{{M_i}} {{\sigma ^2}{B_{i,\xi }}\frac{{{2^{\frac{{{\theta _{i,\xi }}}}{{{B_{i,\xi }}}}}} - 1}}{{G_c G_s {\gamma ({F_k},{d_i})}}}} }  \le {p_{th}}, \nonumber\\
&~\mathrm{\widetilde{C}7:}~{\theta_{i,\xi }} \ge 0,\quad \forall i \in {\cal C},\xi=1,2,...,M_i, \nonumber\\
&~\mathrm{\widetilde{C}8:}~\!\frac{{b{\kappa _i}\!\!\sum\limits_{j = 1}^{{\ell_c} + {\ell_s}} \!\!\!{{\mu _j}} {\rho _j}}}{{{f_i}}}\! + \!\frac{{b\!\!\sum\limits_{j = 1}^{{\ell_c} + {\ell_s}} \!\!\!{{\mu _j}} \psi_j }}{{\sum\limits_{\xi  = 1}^{{M_i}} {{\theta _{i,\xi }}} }} \!\!\le {T_1},\quad\forall i \in {\cal C}, \nonumber\\
&~\mathrm{C9:}~\frac{{\left( {b - \left\lceil {\phi b} \right\rceil } \right)\sum\limits_{j = 1}^{{\ell _c} + {\ell _s}} {{\mu _j}} {\chi _j}}}{{R_i^D}} + \frac{{b{\kappa _i}\sum\limits_{j = 1}^{{\ell _c} + {\ell _s}} {{\mu _j}} {\varpi _j}}}{{{f_i}}} \le {T_2},\quad\!\! \forall i \in {\cal C}. \nonumber
\end{align}

Although problem~\eqref{sub2_further_convert} is still non-convex, we observe that if we fix the decision variables $\bm{\mu}$, ${T_1}$ and ${T_2}$, it can be written as
\begin{align}\label{sub2_sub1}
\mathcal{P}2:&~\mathop {{\rm{min}}}\limits_{{\bm{\theta}}} \widetilde{T}({\bm{\theta}})    \\
&\mathrm{s.t.} ~\mathrm{\widetilde{C}5:}~\sum\limits_{\xi  = 1}^{{M_i}} {{\sigma ^2}{B_{i,\xi }}\frac{{{2^{\frac{{{\theta _{i,\xi }}}}{{{B_{i,\xi }}}}}} - 1}}{{G_c G_s {\gamma ({F_k},{d_i})})}} \le p_i^{\max }},  \forall i \in {\cal C}, \nonumber\\
&~\mathrm{\widetilde{C}6:}~\sum\limits_{i = 1}^C {\sum\limits_{\xi  = 1}^{{M_i}} {{\sigma ^2}{B_{i,\xi }}\frac{{{2^{\frac{{{\theta _{i,\xi }}}}{{{B_{i,\xi }}}}}} - 1}}{{G_c G_s {\gamma ({F_k},{d_i})}}}} }  \le {p_{th}}, \nonumber\\
&~\mathrm{\widetilde{C}7:}~{\theta_{i,\xi }} \ge 0,\quad \forall i \in {\cal C},\xi=1,2,...,M_i, \nonumber\\
&~\mathrm{\widetilde{C}8:}~\frac{{b{\kappa _i} {\rho _j}}}{{{f_i}}}\! + \!\frac{{b \psi_j }}{{\sum\limits_{\xi  = 1}^{{M_i}} {{\theta _{i,\xi }}} }} \!\!\le {T_1},\;\;\;\,\forall i \in {\cal C}. \nonumber
\end{align}

Problem~\eqref{sub2_sub1} is obviously a convex problem with respect to ${\bm{\theta}}$, which can be efficiently solved through available toolkits such as CVX~\cite{CVX}.

Similarly, if the decision variables ${\bm{\theta}}$, ${T_1}$ and ${T_2}$ are fixed, problem~\eqref{sub2_further_convert} is then converted to
\begin{align}\label{sub2_sub2}
\mathcal{P}3:&~\mathop {{\rm{min}}}\limits_{{\bm{\mu }}} \widetilde{T}({\bm{\mu }})    \\
&\mathrm{s.t.} ~\mathrm{C3:}~{\mu _j} \in \left\{ {0,1} \right\},\quad j = 1,2,...,{\ell_c} + {\ell_s}, \nonumber\\
&~\mathrm{C4:}~\sum\limits_{j = 1}^{{\ell_c} + {\ell_s}} {{\mu _j}}  = 1, \nonumber \\
&~\mathrm{\widetilde{C}8:}~\!\frac{{b{\kappa _i}\!\sum\limits_{j = 1}^{{\ell_c} + {\ell_s}} \!\!\!{{\mu _j}} {\rho _j}}}{{{f_i}}}\! + \!\frac{{b\!\!\sum\limits_{j = 1}^{{\ell_c} + {\ell_s}} \!\!\!{{\mu _j}} \psi_j }}{{\sum\limits_{\xi  = 1}^{{M_i}} {{\theta _{i,\xi }}} }} \!\le {T_1},\quad\forall i \in {\cal C}, \nonumber\\
&~\mathrm{C9:}~\frac{{\left( {b - \left\lceil {\phi b} \right\rceil } \right)\!\!\sum\limits_{j = 1}^{{\ell _c}\! + \!{\ell _s}} {\!\!{\mu _j}} {\chi _j}}}{{R_i^D}} + \frac{{b{\kappa _i}\!\!\sum\limits_{j = 1}^{{\ell _c} + {\ell _s}} {\!\!{\mu _j}} {\varpi _j}}}{{{f_i}}} \le {T_2},\forall i \in {\cal C}. \nonumber
\end{align}

Fortunately, problem~\eqref{sub2_sub2} is a standard mixed integer linear programming (MILP), and the branch-and-bound (B$\&$B) algorithm~\cite{lawler1966branch,chen2022end} is always regarded as an effective method for solving MILP. As a result, we employ the B$\&$B algorithm to address the problem~\eqref{sub2_sub2}. It is noted that the dimension of $\mu_j$ is the number of layers of a neural network, which is typically a small number. For example, the well-known image classification neural network AlexNet~\cite{krizhevsky2017imagenet} and GoogLeNet~\cite{szegedy2015going} contain 8 and 22 neural network layers. Therefore, B$\&$B can solve the problem efficiently.

By fixing the decision variables $\bm{\mu}$ and ${\bm{\theta}}$, we convert problem~\eqref{sub2_further_convert} into a linear programming with respect to ${T_1}$ and ${T_2}$, which is given by
\begin{align}\label{sub2_sub3}
\mathcal{P}4:&~\mathop {{\rm{min}}}\limits_{{{T_1},{T_2}}} \widetilde{T}({T_1},{T_2})   \\
&\mathrm{s.t.} ~\mathrm{\widetilde{C}8:}~\frac{{b{\kappa _i} {\rho _j}}}{{{f_i}}}\! + \!\frac{{b \psi_j }}{{\sum\limits_{\xi  = 1}^{{M_i}} {{\theta _{i,\xi }}} }} \!\!\le {T_1},\;\;\;\,\forall i \in {\cal C}, \nonumber\\
&~\mathrm{C9:}~\frac{{\left( {b - \left\lceil {\phi b} \right\rceil } \right){\chi _j}}}{{R_i^D}} + \frac{{b{\kappa _i}{\varpi _j}}}{{{f_i}}} \le {T_2}, \;\forall i \in {\cal C}. \nonumber
\end{align}




We can easily obtain the optimal solution by observing problem~\eqref{sub2_sub3}, i.e.
\begin{align}\label{sub2_sub3_optimal_1}
&T_1^* = \mathop {\max }\limits_i \Bigg\{ \frac{{b{\kappa _i}{\rho _j}}}{{{f_i}}} + \frac{{b{\psi _j}}}{{\sum\limits_{\xi  = 1}^{{M_i}} {{\theta _{i,\xi }}} }}\Bigg\} \;,\\
&T_2^* = \mathop {\max }\limits_i\left\{ {\frac{{\left( {b - \left\lceil {\phi b} \right\rceil } \right){\chi _j}}}{{R_i^D}} + \frac{{b{\kappa _i}{\varpi _j}}}{{{f_i}}}} \right\},
\end{align}
%\begin{align}\label{sub2_sub3_optimal_2}
%T_2^* = \mathop {\max }\limits_i\left\{ {\frac{{\left( {b - \left\lceil {\phi b} \right\rceil } \right){\chi _j}}}{{R_i^D}} + \frac{{b{\kappa _i}{\varpi _j}}}{{{f_i}}}} \right\},
%\end{align}


As aforementioned, we decompose the original problem~\eqref{time_minimize_problem} into four less complicated subproblems (i.e., $\mathcal{P}1$, $\mathcal{P}2$, $\mathcal{P}3$, $\mathcal{P}4$) based on different decision variables, and present efficient algorithms to address each subproblem. Therefore, we propose a block-coordinate descent (BCD)-based algorithm~\cite{tseng2001convergence} to tackle problem~\eqref{time_minimize_problem}, which is described in~\textbf{Algorithm~\ref{BCD_based}}, where $\bm{r}^{\tau}$, $\bm{\mu}^{\tau}$, $\bm{\theta}^{\tau}$, $T_{1}^{\tau}$, and $T_{2}^{\tau}$ denote $\bm{r}$, $\bm{\mu}$, $\bm{\theta}$, $T_{1}$, and $T_{2}$ at the $\tau$-iteration.

\begin{algorithm}[h]
  \caption{BCD-based Algorithm.}\label{BCD_based}
  \begin{algorithmic}[1]
    \Require
        convergence tolerance $\varepsilon $, iteration index $\tau  = 0$.
    \Ensure
        ${\bf{r}}^*$, ${\bm{\mu }^*}$, ${\bf{p}^*}$
    \State Initialization: ${\bf{r}}^0$, ${\bm{\mu }^0}$, ${{\bm{\theta }}^0}$, $T_1^0$, $T_2^0$
    \Repeat
    \State $\tau \leftarrow \tau+1$
    \State Update ${\bf{r}}^{\tau}$ based on~\textbf{Algorithm~\ref{greedy-based}}
    \State Update ${\bm{\theta }^{\tau}}$ by solving convex problem~\eqref{sub2_sub1}
    \State Update ${\bm{\mu }^{\tau}}$ by solving MILP~\eqref{sub2_sub2}
    \State Update ${T_1^{\tau}}$ and ${T_2^{\tau}}$ according to equation~\eqref{sub2_sub3_optimal_1} and (34)
    \Until $| {\widetilde{T}({\bf{r}}^{\tau}\! ,{\bm{\mu }^{\tau}}\!,{\bm{\theta }^{\tau}}\!,{T_1^{\tau}},{T_2^{\tau}}) - \widetilde{T}({\bf{r}}^{\tau-1}\!\!, {\bm{\mu }^{\tau-1}}\!\!, {\bm{\theta }^{\tau-1}}\!, {T_1^{\tau-1}}\!,{} } $ ${T_2^{\tau-1}} ) | \le \varepsilon $

  \end{algorithmic}
\end{algorithm}


\section{Simulation Results}\label{simulation results}
This section provides numerical results to evaluate the learning performance of the proposed EPSL framework and the effectiveness of the tailored resource management and layer split strategy.
\subsection{Simulation Setup}
In the simulations, we deploy $C$ client devices uniformly distributed within the coverage radius $d_{max}=200$m, with server in its center. $C$ is set to 5 by default unless specified otherwise. The computing capability of each client device is uniformly distributed within $[1, 1.6]\times {10^9}$ cycles/s, and the computing capability of the server is set to $5 \times {10^9}$ cycles/s. The total available bandwidth is 200 MHz, with equal subchannel bandwidth ${B_1} = ... = {B_M} = B =$ 10MHz. The transmit PSD of the server $p^{DL}=-50$dBm/Hz, and the PSD of the noise is set to $-174$dBm/Hz~\cite{hu2019edge}. we consider the maximum transmit power $p_1^{\max } = ... = p_C^{\max } = {p^{\max }}=$31.76dBm, and an equal computing intensity ${\kappa _1} = ... = {\kappa _C} = {\kappa } =\frac{1}{{16}}$ cycles/FLOPs. We adopt the channel model in~\cite{samimi2015probabilistic}, where the average path loss exponents for LoS and NLoS are 2.1 and 3.4, and the standard deviations of shadow fading for LoS and NLoS are 3.6dB and 9.7dB. For readers' convenience, the detailed simulation parameters are summarized in Table III.



\begin{table}[h]\label{table_2}
  \centering
  \caption{Simulation Parameters.}
  \renewcommand{\arraystretch}{1.25}{
  \setlength{\tabcolsep}{0.5mm}{
\begin{tabular}{|c|c|c|c|}
\hline
\textbf{Parameter}          & \textbf{value} & \textbf{Parameter} & \textbf{value}  \\ \hline
$f_s$             & $5 \times {10^9} $cycles/s              & $f_i$                 & $[1, 1.6]\times {10^9} $cycles/s                        \\ \hline
$C$             & 5              & $M$           & 20                    \\ \hline
$p^{DL}$               & $-50$dBm/Hz              & $G_cG_s$                  & 10                        \\ \hline
$\sigma^2$        & $-174$dBm/Hz            & $b$             & 64                       \\ \hline
$\kappa _s$        & $\frac{1}{{32}}$ cycles/FLOPs            & $\kappa$             & $\frac{1}{{16}}$ cycles/FLOPs                       \\ \hline
$d_{max}$        & $200$m            &$B$            &$10$MHz                        \\ \hline
$\eta _c$        & $1.5 \times {10^{-4}} $            &$\eta _s$            &$ {10^{-4}} $                        \\ \hline
$p^{\max }$        & $31.76$dBm            &$p_{th}$            &$36.99$dBm                         \\ \hline
\end{tabular}}}
\end{table}

\begin{figure}[t!]
\centering
\includegraphics[width=6.5cm]{NN_formal.eps}
\vspace{0cm}
\caption{ResNet-18 network structure, where the gray dashed line marks the potential choice of the cut layer.
}% \textcolor[rgb]{1.00,0.00,0.00}{edge clouds  at or linked to the SBSs through optical fiber}
\label{NN_formal}
\end{figure}


\begin{table}[h]
  \centering
  \caption{ResNet-18 Network Parameters.}
  \renewcommand{\arraystretch}{1.2}{
  \setlength{\tabcolsep}{0.5mm}{
\begin{tabular}{|c|c|c|c|c|c|}
\hline
\multirow{2}{*}{\textbf{Layer name}} &
  \multirow{2}{*}{\textbf{NN unit}} &
  \multirow{2}{*}{\textbf{Activation}} &
  \multirow{2}{*}{\textbf{\begin{tabular}[c]{@{}c@{}}Layer size\\ (MB)\end{tabular}}} &
  \multirow{2}{*}{\textbf{\begin{tabular}[c]{@{}c@{}}FP FLOPs\\ (MFLOP)\end{tabular}}} &
  \multirow{2}{*}{\textbf{\begin{tabular}[c]{@{}c@{}}Smashed data\\ (MB)\end{tabular}}} \\
        &         &         &        &        &          \\ \hline
CONV1   & 7$\times$7,64  & BN+ReLU & 0.0364 & 9.8304 & 0.25     \\ \hline
CONV2   & 3$\times$3,64  & BN+ReLU & 0.1411 & 9.5027 & 0.0625   \\ \hline
CONV3   & 3$\times$3,64  & BN      & 0.1414 & 9.4863 & 0.0625   \\ \hline
CONV4   & 3$\times$3,128 & BN      & 0.2827 & 4.7432 & 0.0313   \\ \hline
CONV5   & 3$\times$3,128 & BN      & 0.564  & 9.4618 & 0.0313   \\ \hline
CONV6   & 1$\times$1,128 & BN      & 0.0327 & 0.5489 & 0.0313   \\ \hline
CONV4   & 3$\times$3,128 & BN      & 0.2827 & 4.7432 & 0.0313   \\ \hline
CONV5   & 3$\times$3,128 & BN      & 0.564  & 9.4618 & 0.0313   \\ \hline
CONV6   & 1$\times$1,128 & BN      & 0.0327 & 0.5489 & 0.0313   \\ \hline
CONV7   & 3$\times$3,256 & BN      & 1.1279 & 4.7309 & 0.0156   \\ \hline
CONV8   & 3$\times$3,256 & BN      & 2.2529 & 9.4495 & 0.0156   \\ \hline
CONV9   & 1$\times$1,256 & BN      & 0.1279 & 0.5366 & 0.0156   \\ \hline
CONV10  & 3$\times$3,512 & BN      & 4.5059 & 4.7247 & 0.0078   \\ \hline
CONV11  & 3$\times$3,512 & BN      & 9.0059 & 9.4433 & 0.0078   \\ \hline
CONV12  & 1$\times$1,512 & BN      & 0.5059 & 0.5304 & 0.0078   \\ \hline
MAXPOOL & 3$\times$3,64  & /       & /      & 0.0655 & 0.0625   \\ \hline
AVGPOOL & 2$\times$2     & /       & /      & /      & 0.0020   \\ \hline
FC      & 7       & /       & 0.0137 & 0.0036 & 2.67E-05 \\ \hline
    \end{tabular}}}%
  \label{tab:addlabel}%
\end{table}%

We adopt the image classification datasets MNIST~\cite{lecun1998mnist} and HAM10000~\cite{tschandl2018ham10000} to evaluate EPSL's learning performance. The MNIST dataset contains grayscale images with handwritten digits 0 to 9, whereas HAM10000 dataset includes seven different types of skin images, such as melanoma and dermatofibroma. The MNIST dataset has 60000 training samples and 10000 test samples, and HAM10000 dataset has 8000 training samples and 2015 test samples. Furthermore, we conduct experiments under IID and non-IID data settings. The data samples are shuffled and distributed evenly to all client devices in the IID setting. In the non-IID setting~\cite{chen2022federated,yang2021achieving}, each client devices has training samples only associated with two categories.



We deploy the ResNet-18 network~\cite{he2016deep}, which primarily consists of convolution (CONV) layers, pooling (POOL) layers, and  fully-connected (FC) layers. The network structure and network parameters of the implemented  ResNet-18  are detailed in Fig.~\ref{NN_formal} and Table IV. We resize the input images to 64$\times$64 to fit the neural network's input. The mini-batch size is set to 64.




\subsection{Performance Evaluation of the Proposed EPSL Framework}

In this subsection, we conduct the performance evaluation of the proposed EPSL framework in terms of test accuracy, convergence speed and training latency. To investigate the advantages of the EPSL framework, we compare it with three well-known SL frameworks:

\begin{itemize}
\item \textbf{Vanilla SL:} Vanilla SL sequentially trains the client-side model across all participating client devices. Specifically, model training is solely conducted on a single client device and server. After finishing the training for one client device, the updated client-side model is transferred to the next client device for model training~\cite{vepakomma2018split}.
\item \textbf{SFL:} SFL trains the client-side model on all participating client devices in a parallel fashion, and then transmit the updated client-side model to the server for model aggregation~\cite{thapa2022splitfed}.
\item \textbf{PSL:} PSL trains the shared server-side model using all client devices data, while training client-side models based on their respective data. PSL is the special case of EPSL with $\phi=0$~\cite{kim2022bargaining,joshi2021splitfed}.
\item \textbf{EPSL with phased training (EPSL-PT):} The phased training is a hybrid distributed training framework that first leverages EPSL with $\phi=1$ initially, and then switches to employing PSL (i.e., EPSL with $\phi=0$) for model training. This demostrates the flexibility of the proposed EPSL via appropriate control of $\phi$.
\end{itemize}


\begin{table}[t]\label{table5}
\caption{The Converged Test Accuracy for HAM10000 under IID Setting.}
  \renewcommand{\arraystretch}{1.2}{
  \setlength{\tabcolsep}{2.5mm}{
\begin{tabular}{|l|c|c|c|c|}
\hline
\diagbox{\textbf{\begin{tabular}[c]{@{}c@{}}Learning\\ framework  \end{tabular}}}{\textbf{$\bm{C}$}} & \textbf{5} & \textbf{10} & \textbf{15} \\ \hline
\textbf{\quad \quad \quad vanilla SL} & 77.58$\%$  & 77.65$\%$  & 77.62$\%$   \\ \hline
\textbf{\quad \quad \quad SFL} & 77.37$\%$  & 77.21$\%$  & 77.24$\%$   \\ \hline
\textbf{\quad \quad \quad PSL$($EPSL with $\phi=0)$} & 77.33$\%$  & 77.29$\%$  & 77.12$\%$   \\ \hline
\textbf{\quad \quad \quad EPSL$(\phi=0.5)$} & 77.24$\%$  & 77.11$\%$  & 76.93$\%$  \\ \hline
\textbf{\quad \quad \quad EPSL$(\phi=1)$} & 77.12$\%$  & 76.37$\%$ & 74.29$\%$  \\ \hline
\end{tabular}}}
\end{table}



Table V shows the converged test accuracy for HAM10000 dataset under IID setting, where the accuracy is taken at the 300-th epoch (one epoch means going through all local data) when these learning frameworks all converges. It can be seen that EPSL with $\phi=0.5$ and $\phi=1$ achieve a similar learning accuracy as compared with vanilla SL, SFL and PSL when the number of client devices is small. When the number of client devices is large, EPSL with $\phi=1$ cannot converge well, because the aggregated activations' gradients could be too "general" to fit each local dataset. However, as we decrease aggregation ratio $\phi$, the performance again becomes comparable to vanilla SL, SFL and PSL. This demonstrates the effectiveness of last-layer activations' gradient aggregation and the flexibility of our scheme via appropriate control of $\phi$ (like EPSL-PT that switches to $\phi=0$ when the model is close to the convergence region).


Fig.\ref{test_accuracy_1}-\ref{test_accuracy} present the test accuracy of different SL frameworks on MNIST and HAM10000 datasets using ResNet-18. It is seen that EPSL retains similar test accuracy as vanilla SL, SFL and PSL as the models converge. Furthermore, EPSL demands the lowest time budget to achieve a target accuracy. The reason is twofold: one is that EPSL conducts the last-layer gradient aggregation on the server-side model to drastically reduce the dimension of activations' gradient and the server's computation workload in the BP process,  and the other is that EPSL eliminates the necessity for model exchange. Fig.~\ref{test_accuracy_1b} and Fig.~\ref{test_accuracy_b} show that the convergence speed of the four SL frameworks is slower under non-IID setting than under IID setting.


% As seen in Fig.~\ref{test_accuracy_a} and Fig.~\ref{test_accuracy_1a}, under IID setting, EPSL takes 1142 and 113 seconds to reach the target accuracy, which is lower than SFL and PSL.  Additionally, the sequential training mechanism of SL makes it achieve the target accuracy at 3473 and 273 seconds. Fig.~\ref{test_accuracy_b} and Fig.~\ref{test_accuracy_b} show that the convergence speed of the three SL frameworks is slower under non-IID setting than under IID setting. Under non-IID setting, the vanilla SL, SFL, PSL and EPSL achieve the target accuracy at 3907 and 950, 2129 and 570, 1823 and 510, and 1305 and 364 seconds, respectively.



\begin{figure}[t]
     \centering
    \subfigure[Test accuracy on MNIST under IID setting.]{
         \centering
         \includegraphics[width=2.9 in,height=2.5 in]{IID_MNIST.eps}
      \label{test_accuracy_1a}
     }
     \subfigure[Test accuracy on MNIST under non-IID setting.]{
         \centering
         \includegraphics[width=2.9 in,height=2.5 in]{nonIID_MNIST.eps}
      \label{test_accuracy_1b}
    }
%
\caption{The test accuracy of vanilla SL, SFL, PSL and EPSL on MNIST under IID and non-IID settings using ResNet-18.}
\label{test_accuracy_1}
\end{figure}


Fig.~\ref{client_num} illustrates the total training latency for achieving target accuracy (75.5$\%$) versus the number of client devices under different SL frameworks. The total training latency of vanilla SL increases as the number of client devices grows. This is due to the fact that, with the fixed total available bandwidth, the bandwidth allocated to each client device reduces considerably, resulting in lower uplink transmission rate. Different from vanilla SL, the training latency of SFL, PSL and EPSL gradually decreases with the growth of the number of client devices. The reason is that each client device is assigned with fewer number of data samples while the number of client devices increase. Moreover, SFL, PSL and EPSL employ the parallel client-side model training schemes to accelerate model training, which leads to lower training latency. The EPSL is the SL framework with the lowest training latency, and its training latency decreases faster as the number of client devices increases.  This is because that the last-layer gradient aggregation reduces the data size of the transmitted data, thereby mitigating the adverse impact of the wireless spectrum scarcity on training latency. It demonstrates that EPSL is more capable than vanilla SL, SFL, and PSL of enabling low-latency model training over unfavourable wireless communication conditions.

\begin{figure}[t!]
     \centering
    \subfigure[Test accuracy on HAM10000 under IID setting.]{
         \centering
         \includegraphics[width=2.9 in,height=2.5in]{test_accuracy_IID.eps}
      \label{test_accuracy_a}
     }
     \subfigure[Test accuracy on HAM10000 under non-IID setting.]{
         \centering
         \includegraphics[width=2.9 in,height=2.5 in]{test_accuracy_non_IID.eps}
      \label{test_accuracy_b}
    }
%
\caption{The test accuracy of vanilla SL, SFL, PSL and EPSL on HAM10000 under IID and non-IID settings using ResNet-18.}
\label{test_accuracy}
\end{figure}

Fig.~\ref{dataset_size} shows the total training latency for achieving target accuracy versus the size of training dataset under different SL frameworks. It can be seen that the training latency of vanilla SL, SFL, PSL and EPSL grows as the dataset size increases. The training latency of the vanilla SL is always the highest for different dataset sizes. The reason is that vanilla SL's sequential training scheme results in a significant increase in training latency. The training latency for SFL and PSL is much lower than that for vanilla SL because the parallel training schemes of SFL and PSL are far more effective than sequential training of vanilla SL in accelerating the model training. The EPSL has the lowest total training latency for different dataset sizes. Furthermore,  compared to vanilla SL, SFL and PSL, the training latency for EPSL grows more slowly as the dataset size increases, due to the last-layer gradient aggregation design and the elimination of model exchange. This result demonstrates that the proposed EPSL is better suited for large-scale data model training, which is conducive to widespread implementations of EPSL.



\begin{figure}[t]
\centering
\includegraphics[width=7.3cm, height=6.4cm]{client_num.eps}
\vspace{0cm}
\caption{The total training latency for achieving target accuracy on HAM10000 versus the number of client devices under IID setting with $M$=20 and $D$=8000.
}% \textcolor[rgb]{1.00,0.00,0.00}{edge clouds  at or linked to the SBSs through optical fiber}
\label{client_num}
\end{figure}


\begin{figure}[t]
\centering
\includegraphics[width=7.3cm, height=6.4cm]{dataset_size.eps}
\vspace{0cm}
\caption{The total training latency for achieving target accuracy on HAM10000 versus the training dataset size under IID setting with $M$=20 and $C$=5.
}% \textcolor[rgb]{1.00,0.00,0.00}{edge clouds  at or linked to the SBSs through optical fiber}
\label{dataset_size}
\end{figure}


\subsection{Performance Evaluation of the Proposed Resource Management Strategy}

In this subsection, we evaluate the effectiveness of the tailored resource management and layer split strategy in terms of per-round latency. We compare the proposed resource management strategy with four baselines listed below to assess its benefits:

\begin{itemize}
\item \textbf{Baseline a):} The received signal strength (RSS)-based subchannel allocation approach is adopted, which allocates each subchannel to the client device with the highest RSS in this subchannel. The transmit PSD is set uniformly among client devices and subchannels while the cut layer is randomly selected.
\item \textbf{Baseline b):} The proposed greedy subchannel allocation approach and power control scheme depicted in Section~\ref{Solution_Approach} are adopted. The cut layer is randomly selected.
\item \textbf{Baseline c):} The RSS-based subchannel allocation approach, the proposed cut layer selection strategy and  power control scheme are adopted.
\item \textbf{Baseline d):} The proposed greedy subchannel allocation approach and cut layer selection strategy are adopted. The transmit PSD is set uniformly among client devices and subchannels.
\end{itemize}

\begin{figure}[t!]
\centering
\includegraphics[width=7.3cm, height=6.4cm]{different_subnum_1.eps}
\vspace{0cm}
\caption{The performance for per-round training latency versus the total bandwidth with $\phi = 0.5$, $f_s=5 \times {10^9}$ cycles/s and $d_{max}=200$m.
}% \textcolor[rgb]{1.00,0.00,0.00}{edge clouds  at or linked to the SBSs through optical fiber}
\label{different_subnum_1}
\end{figure}


\begin{figure}[htbp]
\centering
\includegraphics[width=7.3cm, height=6.4cm]{different_server_1.eps}
\vspace{0cm}
\caption{The performance for per-round training latency versus the computing capability of server with $\phi = 0.5$, $B_{total}=200$MHz and $d_{max}=200$m.
}% \textcolor[rgb]{1.00,0.00,0.00}{edge clouds  at or linked to the SBSs through optical fiber}
\label{different_server_1}
\end{figure}

Fig.~\ref{different_subnum_1} shows the performance for per-round latency versus the total bandwidth. It is seen that the training latency decreases as the total bandwidth grows. The reason is that increasing total bandwidth results in higher uplink and downlink transmission rates between client devices and the server. The reduction in training latency achieved by power control becomes slow as the total bandwidth increases due to the considerable decrease in data exchange latency.  In this case, the per-round latency is primarily determined by the computing latency of the client devices and the server. The performance gap between baseline c) and proposed solution is widened with larger total bandwidth, due to that the proposed greedy subchannel allocation approach gives priority to reducing the latency of the straggler. Moreover, we observe that optimizing cut layer selection leads to considerably greater reduction in training latency than optimizing subchannel allocation and power control.


Fig.~\ref{different_server_1} illustrates the performance for per-round latency versus the computing capability of the server. It can be observed that the training latency decreases with the increase of the server computing capability. Furthermore, with a more powerful server, the performance improvements brought by power control and subchannel allocation grow. This is because, in this case, the training latency is primarily limited by the computing latency of the client devices and data exchange latency. Again, we observe that optimizing the cut layer selection brings better performance improvement than power control and subchannel allocation when the server's computing capability varies.


\begin{figure}[t]
\centering
\includegraphics[width=7.3cm, height=6.4cm]{channel_impact_2.eps}
\vspace{0cm}
\caption{The effect of channel variation on the performance of the proposed solution with $\phi = 0.5$, $B_{total}=200$MHz and $d_{max}=200$m.
}% \textcolor[rgb]{1.00,0.00,0.00}{edge clouds  at or linked to the SBSs through optical fiber}
\label{channel_impact}
\end{figure}



Fig.~\ref{channel_impact}  shows the effect of channel variation on the performance of the proposed solution. Our layer split decision remains unchanged for a period (in the simulations, it remains the same until the model converges), and therefore it is important to evaluate how channel variation would impact its performance. Specifically, we compare the random realization of channel model~\cite{samimi2015probabilistic} in each training round and the ideal static channel channel model (i.e. channel gain remain unchanged, which is unrealistic but used as the benchmark). We observe that channel variation has little impact on the performance of the proposed solution, which demonstrates its robustness in dynamic wireless channel conditions.


\section{Conclusions}\label{conclusion_section}

In this paper, we have proposed a novel SL framework, efficient parallel split learning (EPSL), to accelerate the model training. EPSL parallelizes the client-side model training, and lowers the dimension of activations' gradients and server's computation workload by performing the last-layer gradient aggregation, leading to significant reduction in training latency. By considering the heterogeneity in channel conditions and computing capabilities at client devices, we have designed a resource management and layer split strategy to jointly optimize subchannel allocation, power control, and cut layer selection to minimize the training latency for ESPL over the wireless edge networks. While the formulated problem is a mixed-integer nonlinear programming, we decompose it into four less complicated subproblems based on different decision variables, and present an efficient BCD-based algorithm to solve this problem. Simulation results demonstrate that our proposed EPSL framework takes significantly less time budget to achieve a target accuracy than existing benchmarks, and the effectiveness of the tailored resource management and layer split strategy.

This work has demonstrated the potential of integrating EPSL and edge computing paradigm. However, more research attention needs to be paid to addressing label privacy issue (especially for privacy-sensitive data, such as disease diagnosis data) resulting from the label sharing in SL. Furthermore, convergence analysis of EPSL is worth further exploration.

\bibliographystyle{IEEEtran}
\bibliography{NEWmybib}

%\vspace{-33pt}
%\begin{IEEEbiography}[{\includegraphics[width=1in,height=1.25in,clip,keepaspectratio]{zhenglin.jpg}}]{Zheng Lin}
%is a postgraduate student at the Department of Electrical Engineering, Fudan University. His research interest focuses on the edge intelligence and distributed machine learning.
%\end{IEEEbiography}
%
%\vspace{-33pt}
%\begin{IEEEbiography}[{\includegraphics[width=1.2in,height=1.25in,clip,keepaspectratio]{guangyuzhu.jpg}}]{Guangyu Zhu}
%received the B.Eng. degree from Xidian University, Xi��an, China, in 2019. Since 2019, he has been pursuing the Ph.D degree with the Department of Electrical and Computer Engineering, University of Florida, Gainesville, FL, USA. His research interests include machine learning, wireless networks, and edge computing.
%\end{IEEEbiography}
%
%\vspace{-33pt}
%\begin{IEEEbiography}[{\includegraphics[width=1in,height=1.25in,clip,keepaspectratio]{yiqindeng.jpg}}]{Yiqin Deng}(Member, IEEE)
%is currently a Postdoctoral Research Fellow with the School of Control Science and Engineering, Shandong University. She received her B.S. degree in project management from Hunan Institute of Engineering, Xiangtan, China, in 2014, and her M.S. degree in software engineering and her Ph.D. degree in computer science and technology from Central South University, Changsha, China, in 2017 and 2022, respectively. She was a visiting researcher at the University of Florida, Gainesville, from 2019 to 2021. Her research interests include Edge/Fog computing, Internet of Vehicles, and Resource Management.
%\end{IEEEbiography}
%
%\vspace{-33pt}
%\begin{IEEEbiography}[{\includegraphics[width=1in,height=1.25in,clip,keepaspectratio]{xianhaochen.eps}}]{Xianhao Chen}(Member, IEEE)
%is currently an assistant professor with the Department of Electrical and Electronic Engineering, the University of Hong Kong. He obtained the Ph.D. degree from the University of Florida in 2022, and received the B.Eng. degree from Southwest Jiaotong University in 2017. He received 2022 ECE graduate excellence award for research from the University of Florida. His research interests include wireless networking, edge intelligence, and machine learning.
%\end{IEEEbiography}
%
%\vspace{-33pt}
%\begin{IEEEbiography}[{\includegraphics[width=1in,height=1.25in,clip,keepaspectratio]{yuegao.jpg}}]{Yue Gao}(Senior Member, IEEE)
%received a PhD from the Queen Mary University of London (QMUL), U.K., in 2007. He is a Professor at the School of Computer Science, Fudan University, China, and a Visiting Professor at the University of Surrey, U.K.. He worked as a Lecturer, a Senior Lecturer, a Reader and the Chair Professor with QMUL and the University of Surrey, respectively. He has published over 200 peer-reviewed journal and conference papers. His research interests include sparse signal processing, smart antennas and cognitive networks for mobile and satellite systems. He was a co-recipient of the EU Horizon Prize Award on Collaborative Spectrum Sharing in 2016 and an Engineering and Physical Sciences Research Council Fellow in 2017. He is a member of the Board of Governors and a Distinguished Speaker of the IEEE Vehicular Technology Society (VTS), the Chair of the IEEE ComSoc Wireless Communication Technical Committee, and the Past Chair of the IEEE ComSoc Technical Committee on Cognitive Networks. He has been an Editor of several IEEE Transactions and Journals, the Symposia Chair, the Track Chair, and other roles in the organizing committee of several IEEE ComSoC, VTS, and other conferences.
%\end{IEEEbiography}
%
%\vspace{-33pt}
%\begin{IEEEbiography}[{\includegraphics[width=1in,height=1.25in,clip,keepaspectratio]{kaibinhuang.jpeg}}]{Kaibin Huang}(Fellow, IEEE)
%received the B.Eng. and M.Eng. degrees from the National University of Singapore and the Ph.D. degree from The University of Texas at Austin, all in electrical engineering. He is a Professor and an Associate Head at the Dept. of Electrical and Electronic Engineering, The University of Hong Kong, Hong Kong. He received the IEEE Communication Society's 2021 Best Survey Paper, 2019 Best Tutorial Paper, 2019 Asia-Pacific Outstanding Paper, 2015 Asia-Pacific Best Paper Award, and the best paper awards at IEEE GLOBECOM 2006 and IEEE/CIC ICCC 2018. He received the Outstanding Teaching Award from Yonsei University, South Korea, in 2011. He has been named as a Highly Cited Researcher by the Clarivate Analytics in 2019-2022. He is a member of the Engineering Panel of Hong Kong Research Grants Council (RGC)and a RGC Research Fellow. He served as the Lead Chair for the Wireless Communications Symposium of IEEE Globecom 2017 and the Communication Theory Symposium of IEEE GLOBECOM 2014, and the TPC Co-chair for IEEE PIMRC 2017 and IEEE CTW 2023 and 2013. He is also an Executive Editor of IEEE Transactions on Wireless Communications, and an Area Editor for both IEEE Transactions on Machine Learning in Communications and Networking and IEEE Transactions on Green Communications and Networking. Previously, he served on the Editorial Boards for IEEE Journal on Selected Areas in Communications and IEEE Wireless Communication Letters. He has guest edited special issues of IEEE Journal on Selected Areas in Communications, IEEE Journal of Selected Areas in Signal Processing, and IEEE Communications Magazine. He is also a Distinguished Lecturer of the IEEE Communications Society and the IEEE Vehicular Technology Society.
%\end{IEEEbiography}
%
%\vspace{-33pt}
%\begin{IEEEbiography}[{\includegraphics[width=1in,height=1.25in,clip,keepaspectratio]{fang.eps}}]{Yuguang ``Michael'' Fang}(Fellow, IEEE)
%received an MS degree from Qufu Normal University, China in 1987, a PhD degree from Case Western Reserve University in 1994, and a PhD degree from Boston University in 1997. He joined the Department of Electrical and Computer Engineering at University of Florida in 2000 as an assistant professor, then was promoted to associate professor in 2003, full professor in 2005, and distinguished professor in 2019, respectively. Since August 2022, he has been the Chair Professor of Internet of Things with the Department of Computer Science at City University of Hong Kong.
%
%Dr. Fang received many awards including the US NSF CAREER Award (2001), US ONR Young Investigator Award (2002), 2018 IEEE Vehicular Technology Outstanding Service Award, 2019 IEEE Communications Society AHSN Technical Achievement Award, 2015 IEEE Communications Society CISTC Technical Recognition Award, 2014 IEEE Communications Society WTC Recognition Award, the Best Paper Award from IEEE ICNP (2006), 2010-2011 UF Doctoral Dissertation Advisor/Mentoring Award, and 2009 UF College of Engineering Faculty Mentoring Award. He held multiple professorships including the Changjiang Scholar Chair Professorship (2008-2011), Tsinghua University Guest Chair Professorship (2009-2012), NSC Visiting Researcher of National Taiwan University (2007-2008), Invitational Fellowship of Japan Society for the Promotion of Science (2009), University of Florida Foundation Preeminence Term Professorship (2019-2022), University of Florida Research Foundation Professorship (2017-2020, 2006-2009), and University of Florida Term Professorship (2017-2021). He served as the Editor-in-Chief of IEEE Transactions on Vehicular Technology (2013-2017) and IEEE Wireless Communications (2009-2012) and serves/served on several editorial boards of journals including Proceedings of the IEEE (2018-present), ACM Computing Surveys (2017-present), ACM Transactions on Cyber-Physical Systems (2020-present), IEEE Transactions on Mobile Computing (2003-2008, 2011-2016, 2019-present), IEEE Transactions on Communications (2000-2011), and IEEE Transactions on Wireless Communications (2002-2009). He served as the Technical Program Co-Chair of IEEE INFOCOM��2014 and the Technical Program Vice-Chair of IEEE INFOCOM'2005. He has actively engaged with his professional community, serving as a Member-at-Large of the Board of Governors of IEEE Communications Society (2022-2024) and the Director of Magazines of IEEE Communications Society (2018-2019). He is a fellow of ACM, IEEE, and AAAS.
%\end{IEEEbiography}


%
%
%
%
%\vfill

\end{document}
