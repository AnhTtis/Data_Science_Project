%% using aastex version 6.3
%\documentclass[linenumbers,dvipdfmx,trackchanges, twocolumn]{aastex631}
\UseRawInputEncoding 
\documentclass[modern]{aastex63}
%\documentclass[linenumbers,trackchanges, twocolumn]{aastex631}

\usepackage{bm}
\usepackage{amsmath}
\usepackage{color}
\usepackage{graphicx}
\usepackage{ulem}

\newcommand{\vdag}{(v)^\dagger}
\newcommand\aastex{AAS\TeX}
\newcommand\latex{La\TeX}

\newcommand{\bea}{\begin{eqnarray} }
\newcommand{\eea}{\end{eqnarray}}
\def\mbf#1{\mbox{\boldmath ${#1}$}}
\newcommand{\mpc}{$M_\odot$ pc$^{-2}\,$}
\newcommand{\twco}{$^{12}$CO}
\newcommand{\thco}{$^{13}$CO}
%\newcommand{\arho}{\langle\rho\rangle}
\newcommand{\arho}{\rho_0}
\newcommand{\brho}{\rho_m}
\newcommand{\hmol}{H$_2~$}
\newcommand{\oiii}{[O{\scriptsize III }] }
\newcommand{\oii}{[O{\scriptsize II }] }
\newcommand{\oi}{[O{\scriptsize I }] }
\newcommand{\nii}{[N{\scriptsize II }] }
\newcommand{\sii}{[S{\scriptsize II }] }
\newcommand{\hii}{H{\scriptsize II }}
\newcommand{\niiha}{[N{\scriptsize II }]/H$\alpha$ }
\newcommand{\siiha}{[S{\scriptsize II }]/H$\alpha$ }
\newcommand{\oiha}{[O{\scriptsize I }]/H$\alpha$ }
\newcommand{\oiiihb}{[O{\scriptsize III }]/H$\beta$ }

\newcommand{\civ}{C{\scriptsize ~IV }}
\newcommand{\mgii}{Mg{\scriptsize ~II }}
\newcommand{\cii}{C{\scriptsize ~II }}
\newcommand{\ciip}{C{\scriptsize ~II}]}
\newcommand{\niip}{N{\scriptsize ~II }]}
\newcommand{\siiii}{Si{\scriptsize ~III }}
\newcommand{\siiv}{Si{\scriptsize ~IV }}
\newcommand{\hei}{He{\scriptsize ~I}}
\newcommand{\ciiif}{C{\scriptsize ~III }]}
\newcommand{\niv}{N{\scriptsize ~IV}] }

\newcommand{\cloudy}{C{\scriptsize LOUDY} }
%% Reintroduced the \received and \accepted commands from AASTeX v5.2
%\received{March 1, 2021}
%\revised{April 1, 2021}
%\accepted{\today}

%% Command to document which AAS Journal the manuscript was submitted to.
%% Adds "Submitted to " the argument.
%\submitjournal{PSJ}


%%%%%%%%%%%%%%%%%%%%%%%%%%%%%%%%%%%%%%%%%%%%%%%%%%%%%%%%%%%%%%%%%%%%%%%%%%%%%%%%
%%
%% The following section outlines numerous optional output that
%% can be displayed in the front matter or as running meta-data.
%%
%% If you wish, you may supply running head information, although
%% this information may be modified by the editorial offices.
\shorttitle{AGN spectra in a radiation-driven fountain}
\shortauthors{Wada et al.}
%%
%% You can add a light gray and diagonal water-mark to the first page 
%% with this command:
%% \watermark{text}
%% where "text", e.g. DRAFT, is the text to appear.  If the text is 
%% long you can control the water-mark size with:
%% \setwatermarkfontsize{dimension}
%% where dimension is any recognized LaTeX dimension, e.g. pt, in, etc.
%%
%%%%%%%%%%%%%%%%%%%%%%%%%%%%%%%%%%%%%%%%%%%%%%%%%%%%%%%%%%%%%%%%%%%%%%%%%%%%%%%%
%\graphicspath{{./}{figures/}}
%% This is the end of the preamble.  Indicate the beginning of the
%% manuscript itself with \begin{document}.

\begin{document}

%\title{Structures of optical-UV spectra based on the radiation-driven fountain in AGNs}
\title{Multi-phase gas nature in the sub-pc region of the active galactic nuclei II: Optical-UV spectra originated in the ionized gas}
 
%%% begin:list of authors
\author{%
Keiichi Wada
}
\affiliation{Kagoshima University, Graduate School of Science and Engineering, Kagoshima 890-0065, Japan}
\affiliation{Ehime University, Research Center for Space and Cosmic Evolution, Matsuyama 790-8577, Japan}
\affiliation{Hokkaido University, Faculty of Science, Sapporo 060-0810, Japan}
\correspondingauthor{Keiichi Wada}
\email{wada@astrophysics.jp}

\author{
Yuki Kudoh
%Ryota \textsc{Hamamura}\altaffilmark{1}
}%
\affiliation{Kagoshima University, Graduate School of Science and Engineering, Kagoshima 890-0065, Japan}

\author{Tohru Nagao}%
\affiliation{Ehime University, Research Center for Space and Cosmic Evolution, Matsuyama 790-8577, Japan}


\begin{abstract}

Through two-dimensional radiation-hydrodynamical simulations, we investigate the spectral properties of ionized gas irradiated by an active galactic nucleus with 
a supermassive black hole of $10^7 M_\odot$.
For the gas inside the dust-sublimation radius ($r \sim 10^{-2}$ pc), 
we conduct post-process pseudo-three-dimensional 
calculations utilizing the spectral synthesis code \cloudy. We show that we can reproduce various broad emission lines in 
optical and ultraviolet wavelengths. 
The line profiles change depending on the viewing angles even for a small range from the rational axis, i.e., 5-30 degrees; most lines, such as H$\alpha$, are
characterized by a double-peaked profile, reflecting that the emissions are originated in the surface of the rotating disk.
By contrast, high-ionization emission lines such as \civ$\lambda$1549 show a double-peaked profile 
for a nearly face-on view, as these lines derive from the fast outflowing gas from the disk surface.
{Our results suggest that some properties of the bright UV-optical emission lines observed in Seyfert-like AGNs can be 
caused by the radiation-driven fountain flow inside the 
dust sublimation radius.}

\end{abstract}


\keywords{}

%%%%%%%%%%%%%%%%%%%%%%%%%%%%
%
\section{Introduction} \label{sec:intro}
%
%%%%%%%%%%%%%%%%%%%%%%%%%%%%

Type 1 active galactic nuclei (AGNs) are characterized by broad ($\gtrsim 1000$ km s$^{-1}$) emission lines. These include 
Balmer lines, \civ$~\lambda 1549$, and \mgii$~\lambda2798$.
\citep[e.g.,][]{1986ARA&A..24..171O, peterson_1997}. These lines most likely originate in 
the photoionized gas derived from the strong radiation emitted by the AGN \citep[e.g.][]{agn2agn2006, netzer2013}.
The widths of the individual emission lines have often been used to estimate central black hole (BH) mass \citep[e.g.,][]{ferrarese_ford2005, bentz2009}.
However, for a reliable estimate of the masses of supermassive BHs, understanding the geometry and kinematics of 
broad emission line regions (BLRs) and their origins is essential.
In terms of low-ionization gas, BLRs are believed to originate in a rotating disk, but
radial motions such as outflows and inflows are also assumed to be present \citep[e.g.,][and references therein]{gaskell1982, 1985MNRAS.212..425S, Ferland1989ApJ...347..656F, chiang1996, gaskell2009}.

Recently, the spatial structures of BLRs were partially resolved using a near-infrared interferometer in some nearby AGNs
%e.g., 3C 273, IRAS 09149-6206 and NGC 3783 
\citep{2018Natur.563..657G, gravity2020, 2021A&A...648A.117G}.
The results are consistent with the size determined by the  reverberation mapping (RM) technique
 \citep[e.g.,][]{blandford1982, peterson1993, peterson2004ApJ...613..682P, lawther2018, Baskin2018-hj}.
The outer edge of the BLR is  $\sim 1/3 $ of the dust sublimation radius \citep{netzer_laor1993, suganuma2006, netzer2015, Netzer2020-hv, GRAVITY_Collaboration2022-md}. 
The velocity-resolved RM \citep[e.g.,][]{barth2011, almeyda2020} can provide clues about the spatial distribution of BLRs in Seyfert 1 galaxies, such as NGC 5548 \citep[][]{Williams2020-me}. However, it is still unclear whether these two components are or are not physically related.

  %%  dust 
If BLR is extended to the inner edge of the dust torus, the structure and dynamics of the dust sublimation region are critical in understanding the origins of BLRs \citep{Baskin2018-hj}.
 %their dynamics should be greatly affected by the central radiation and the radiation field is also determined by distribution of the material around the AGNs.
  The failed radiatively accelerated dusty outflow (FRADO) is a type of dynamical model \citep{Czerny2010-oz, Naddaf2021-kz}, but the multi-dimensional dynamics of dusty gas, which are essential for the line shape and distribution of ionized gas, is not directly solved in the FRADO model \citep[see also][]{Dorodnitsyn2021-ky}.
 
 %% cloud
%Physical conditions of BLR are other important unsolved problems.
BLR gases are often assumed to be high-density ($\sim 10^{10}-10^{11}$ cm s$^{-3}$) cloudlets.
%which may be
% confined, for example, by magnetic pressure \citep{rees1987} or radiation pressure \citep{baskin2014}.
However, the realistic structures and dynamics of BLR ``clouds'' remain theoretically unclear.
  {Recently,  \citet{matthews2020} demonstrated that biconical disk winds
illuminated by an AGN continuum can produce BLR-like spectra. 
Based on a simple clumpy wind model, they conducted  Monte Carlo radiation transfer calculations, 
and found that broad emission lines with equivalent widths and line ratios comparable to those observed in quasars.
Although they have succeeded in reproducing spectra resembling those of luminous, type-1 AGNs,
their model needs to assume wind properties and geometry with various free parameters. 
As they pointed out in the summary of the paper,  the radiative transfer calculations based on 
hydrodynamic simulations are necessary for the next step.
}
 
 %% importance of hydrodynamics
 The hydrodynamics of dusty gas under central radiation was recently studied in terms of the ``obscuring torus'' on a 1--10 pc scale \citep{wada2012radiation, dorodnitsyn2012, wada2015obscuring, namekata2016, williamson2020}. 
Multi-dimensional radiation-hydrodynamic calculations previously
revealed that outflowing multi-phase gas with dust is formed naturally, 
and the Type 1 and 2 dichotomies in the spectral energy distribution (SED) can thus be naturally explained \citep{schartmann2014}. 
%The radiation-driven outflows are characterized by non-steady outflows, backflows,
%and  turbulent motion, which are sustained by mass inflow through the dense gas disk \citep{wada2012radiation}.
 This dynamical model (``radiation-driven fountain") effectively explains the multi-wavelength observations of 
 the nearby Type 2 Seyfert galaxy, the Circinus galaxy, in many aspects:  molecular and atomic emission and 
 absorption lines in the central 10 pc \citep{izumi2018, wada_fukushige2018, uzuo2021, matsumoto2022, izumi2023},
 the conical shape and line ratio properties of the narrow emission line
region (e.g., \oiii$\lambda$ 5007 \citep{wada_yonekura2018}), and the X-ray spectral energy distribution and lines \citep{buchner2021, ogawa2022}.
%The ``polar" dust emission in the mid-infrared band, which
%is often observed in nearby Seyfert galaxies \citep{hoenig2013, tristram2014, asmus2016, isbell2022}, is also naturally reproduced.

{
As the second paper of the series, we here focus on emission lines derived from the gas inside the dust-sublimation radius ($ < $ 0.02 pc) using a high-spatial-resolution radiation-driven fountain model \citep[][]{kudoh2023} (hearafter Paper I). 
In contrast to \citet{matthews2020}, we investigate gas dynamics in relatively low luminosity AGNs with a moderate black hole (BH) mass, i.e., $10^7 M_\odot$ in this paper. 
This is partly because a larger dynamic range should be necessary  for more luminous quasar-type AGNs associated with more massive BHs, 
and the radiation-driven fountain scheme is most relevant to explain the multi-wavelength properties of Seyfert-type AGNs \citep[e.g.,][]{izumi2023}. 
}

Following \citet{wada_yonekura2018}, we analyze a snapshot of the hydrodynamic simulation using 
the photo-ionized code C{\scriptsize LOUDY} \citep{ferland2017}. 
The line profiles of the hydrogen recombination lines as well as the high-ionization lines are discussed.
This is an attempt to understand the origins of the emission lines of AGNs based on a physics-motivated 
multi-dimensional model.
%%%%%%%%%%%%%%%%%%%%%%%%%%%%%%%%%%%%%%%%%%%%%%%%%%%%%%%%%%%%%%%%%%%%%%%%%%%%%%
%
\section{Numerical Methods}

\subsection{Physical model in gas, dust, and radiation} \label{sec:model}
We use one-snapshot data from a two-dimensional radiation-hydrodynamic simulation in a quasi-steady state (see Paper I in detail) and
calculate the radiative-transfer as a post process (see \S 2.2). Here, we briefly summarize the hydrodynamic model.

We solve the evolution of a dusty gas disk with mass inflow irradiated by a central source in a computational box of $r = 10^{-4} \sim 50$ pc
(Figure \ref{wada_fig: model}). 
This is an extension of the three-dimensional radiation-driven fountain simulations \citep{wada2012radiation,wada2015obscuring} with a higher resolution.  However, we assume an axisymmetric distribution using a cylindrical coordinate.
The included physics here are those of radiative heating by X-ray as well as the radiation force that induces both dusty and ionized gases.
The black hole mass is $M_{\rm BH} = 10^7 M_\odot$ and the Eddington ratio is 0.1 (the bolometric luminosity is $1.25 \times 10^{44}$ erg s$^{-1}$).


The basic equations are
%continuum
\begin{eqnarray}
\frac{\partial \rho}{\partial t} + \bm{\nabla} \cdot \left[\rho \bm{v} \right]  =0,  
\label{eq:mass}
\end{eqnarray}
% equation of motion
\begin{eqnarray}
 \displaystyle \frac{\partial \rho \bm{v}}{\partial t}
 + \bm{\nabla} \cdot \left[ \rho  \mathbf{vv} + {P}_{\rm g} \mathbf{I}  \right]  
 =  \bm{f}_{\rm rad} + \bm{f}_{\rm grav} + \bm{f}_{\rm vis} ,
\label{eq:momentum}
\end{eqnarray}
% energy equation
\begin{eqnarray}
\displaystyle \frac{\partial e}{\partial t}
+ \bm{\nabla} \cdot \left[ \left( e+ P_{\rm g}  \right) \bm{v}  \right]
 = - \rho {\cal L} + \bm{ v} \cdot \bm{f}_{\rm rad} +  \\ \nonumber 
  \bm{ v} \cdot \bm{f}_{\rm grav} + W_{\rm vis},
 \label{eq:energy}
\end{eqnarray}
where total energy density is $e=P_{\rm g}/(\gamma-1)+\rho v^2/2$, and the specific heat ratio adiabatically, i.e.  $\gamma =5/3$. %$\mathbf{I}$ denotes a unit tensor.
${\cal L}$ is the net heating/cooling rate per unit mass.
We adopted the gravitational force, $\bm{f}_{\rm grav} =  -\rho G M_{\rm BH} \bm{e}_{r} /r^2 $, where $G$ denotes the gravitational constant and $r=\sqrt{R^2+z^2}$ is the distance from the center of BH with $M_{\rm BH}= 10^7 M_{\odot}$.
%The density $\rho$ is expressed as sum of the gas and dust densities,
%and $\bm{v}$ is the barycentric velocity, $(\rho_{\text{g}} \bm{v}_{\text{g}} + \rho_{\text{d}} \bm{v}_{\text{d}} ) / \rho$ where indexes d and g denote dust and gas.
The radiation force is $\bm{f}_{\text{rad}} \simeq \int \nabla \cdot F_{\nu} \bm{e}_{r} d \nu$, where $F_{\nu}$ is the radiation flux.
%\citep[e.g.,][]{whalen_norman2006, chan_krolik2017, namekata2016}.  

We assume  the $\alpha$ viscosity ($\alpha = 0.1$) to achieve mass accretion through the disk 
as $\nu_{\rm vis}= \alpha c_s^2/\Omega_{\rm K}$ with the viscosity depends on the sound speed $c_s$ and the Keplerian angular speed $\Omega_{\rm K}$ \citep{shakura_sunyaev1973}.
The viscous force in Equation (\ref{eq:momentum}) and the viscous heating in Equation \ref{eq:energy} are taken from \citet{osuga2005}:

\begin{equation}
  \bm{f}_{\rm vis} \equiv \frac{\bm{e}_{\varphi}}{R^2} \frac{\partial}{\partial R} \left[ R^2 \alpha P_{\rm g} \frac{R^2}{v_{\varphi}} \frac{\partial}{ \partial R } \left( \frac{v_{\varphi}  }{R} \right)  \right], 
\end{equation}
and
\begin{equation}
  W_{\rm vis} \equiv \alpha P_{\rm g} \frac{R}{v_{\varphi} } \left[  R \frac{\partial}{ \partial  R } \left( \frac{v_{\varphi}}{R} \right)  \right]^2.
\end{equation}
We assumed the viscosity parameter $\alpha$ to obtain the gas supply around the disk mid-plane,
\begin{equation}
\alpha=
\left\{\begin{array}{ll}
0.1 \quad &  n > 10^{3} \text{ cm}^{-3} ~\&~ T_{\textrm{g}}< 10^3 \text{ K} \\ 
0.0 \quad & {\rm otherwise} 
\end{array}\right.
\end{equation}



We consider heating by UV and X-ray \citep{maloney1996, meijerink2005, wada2012radiation} and the optically thin radiative cooling \citep{meijerink_spaans2005, wada_papadopoulos2009}.
We assume that dust grains sublimate above a dust sublimation temperature $T_{\text{sub}}=1500$ K.
Dust temperature is generally used in local thermal equilibrium with radiation.

We use the public MHD code CANS+ \citep{matsumoto2019}
\footnote{\url{https://github.com/chiba-aplab/cansplus}} 
with the additional module to evaluate the radiation force and radiative heating/cooling. However, we ignore the magnetic field in the present model.
The number of computational cells in each direction is set to $(N_R, N_z) = (1200, 2304)$.
The cell sizes in the uniform region give high resolutions $\Delta R=\Delta z=5 \times 10^{-5}$ pc for $R< 3.5 \times 10^{-2}$ pc and $|z|< 3.26 \times 10^{-2}$ pc.
On the outside, the cells are stretched to approximately 0.1 pc for a maximum simulation box of $R=|z|=50$ pc (Figure \ref{wada_fig: model}).


\begin{figure}[h]
\centering
%
\includegraphics[width = 8cm]{figure1.pdf}
%\includegraphics[width = 8cm]{BLR_figures/model_setup_v2.pdf}

\caption{Model setup:  The central radiation field of anisotropic and isotropic are assumed for UV and X-ray, respectively. Therefore, 
the dust sublimation region is not spherical (see Paper I in detail). The dusty, cold gas is
supplied through the disk and the outflow is launched from the inner most region of the disk ($r \lesssim 0.1$ pc). $R_s$ is the Schwarzschild radius.}

\label{wada_fig: model}
\end{figure}


\begin{figure}[h]
\centering
%
\includegraphics[width = 9cm]{figure2a.pdf}
\includegraphics[width = 9cm]{figure2b.pdf}
%\includegraphics[width = 9cm]{BLR_figures/xz_lin_0.2pc_vi05_0336.pdf}
%\includegraphics[width = 9cm]{BLR_figures/xz_lin_0.002pc_vi05_0336.pdf}

\caption{Input radiation-hydrodynamic model, where the top and bottom are the same, but for the central 0.2 pc and 0.002 pc regions, respectively. 
See also Paper I.
The left-half panel shows the line-of-sight velocity for the viewing angle, which is 5$^\circ$ from the rotational axis. The right-half panel shows the number density of the gas.
In the gray region near the $z$ axis, the velocity exceeds $10^5$ km s$^{-1}$ of the light speed. This region is a temporal structure derived from the numerical artifact near the
boundary. However, this high velocity region is not used in the spectrum calculations.}

\label{wada_fig: hydro}
\end{figure}




\subsection{Radiative transfer using C{\scriptsize LOUDY}}


We used density, temperature, and velocities in the central $r \le 0.02$ pc in the hydrodynamic simulation described in Section 2.1
(Figure \ref{wada_fig: hydro}), and the data of the 2D cylindrical coordinate
were modified for polar grid cells with $ (N_r, N_\theta) = (400, 41)$ for $ -30 \le \theta \le 30^\circ$, where $\theta$ is the angle from
the equatorial plane.
We then ran the spectral synthesis code C{\scriptsize LOUDY} (version 17.03) \citep{ferland2017}.




%\subsection{\cloudy simulation}
The SED of the central source was derived from 
 \textsf{Cloudy}'s \textsf{AGN} command and is represented as
\begin{eqnarray}
\label{eqn:agncon}
f \left(\nu\right)    =  \nu ^{\alpha _\mathrm{UV} } \exp \left( { - h\nu /kT_\mathrm{BB} } \right)\exp \left( { - kT_\mathrm{IR} /h\nu } \right)\cos{i} \nonumber\\
                         +  a\nu ^{\alpha_\mathrm{X} } \exp \left( { - h\nu /E_1 } \right) \exp \left( { - E_2 /h\nu } \right),
\end{eqnarray}
where 
$\alpha _\mathrm{UV} = -0.5$, $T_\mathrm{BB} = 10^5$~K, 
$\alpha_\mathrm{X} = -0.7$, $a$ is a constant that yields the X-ray-to-UV ratio $\alpha_\mathrm{OX} = -1.4$, $kT_\mathrm{IR} = 0.01$~Ryd,
$E_\mathrm{1} = 300$~keV, $E_\mathrm{2} = 0.1$~Ryd, and $i$ is the angle from the z axis (i.e., rotational axis).
The UV radiation (first term), which derives from the geometrically
thin optically thick disk, 
was assumed to be proportional to $\cos{i}$. 
By contrast, the X-ray component (second term) was assumed to be isotropic (Figure \ref{wada_fig: model}).

We assume that grains are sublimated (i.e., no grains) in the data used in C{\scriptsize LOUDY} ,
and the Solar metallicity is assumed. 
In Paper I, we confirmed that the gas inside $r \sim 0.01$ pc is mostly dust-free.
 The following are parts of the input file for C{\scriptsize LOUDY}:
\begin{verbatim}
abundances "default.abn" no grains
grains ism function sublimation
filling factor 1.0
no molecules
set nend 2000
set continuum resolution 0.2
\end{verbatim}



The transmitted SED, calculated using \cloudy for the inner most cell,
was used as an incident SED for the next outward radial cell,
and this procedure was repeated up to the outer edge (i.e., $r=0.02$ pc) for a given radial ray \citep[see][in detail]{wada_yonekura2018}.
We confirmed that beyond $r > 0.02 $ pc, most emission lines typically seen in BLRs are very weak. 

 
 Upon completion of all \cloudy calculations, we  {\it observed} the system (i.e., all the grid cells within 
 $r = 0.02 $ pc) along the 
 line of sight, assuming the viewing angle $i$ ($i = 0$ means
face-on). For the azimuthal direction, we assumed that the system was axisymmetric. 
In addition, the Doppler-shifted emission lines (in which the velocity of each cell was used) from 
all grid cell spectra were integrated while considering 64 azimuthal directions over $2 \pi$. 
{This is justified that the optical depth of the prominent lines in \cloudy calculations 
are smaller than 0.1, and therefore they are not heavily attenuated by the diffuse media between 
the observer and the surface of the dense disk, where the most emission lines are originated in.
Note that we do not assume optically thin for the radial direction in the \cloudy calculations.}



%%%%%%%%%%%%%%%%%%%%%%%%%
%
\section{Results}
%
%%%%%%%%%%%%%%%%%%%%%%%%%

Figure \ref{wada_fig: spect1} shows the spectrum between 1000\AA  ~and 7000\AA, assuming a viewing angle $i$ of 5$^\circ$ from the rotational axis. 
Strong hydrogen recombination lines as well as \civ$\lambda$1549, \ciiif $\lambda$1909, and \mgii$\lambda$2798, which are
often seen in Type 1 and 2 AGNs, are obtained \footnote{It is unclear why the \hei$\lambda$4471 was stronger than the Balmer lines, which is not usually prominent in observations.}.

Figure \ref{wada_fig: spect2} shows the spectra that include H$\alpha$, H$\beta$, and \mgii. The profiles for all of these lines depend on the
viewing angle; it is wider for $i = 30^\circ$ than $i = 5^\circ$ or $10^\circ$. In H$\alpha$ and \mgii, the line shows a double-peak profile for 
$i = 30^\circ$. This dependence on the viewing angle is a natural consequence in which the emission region for these lines represents the upper or lower surface 
of the rotating disk, as can be seen in the brownish region of the density map shown in Figure \ref{wada_fig: hydro} (bottom panel), 
where $n \gtrsim 10^{11}$ cm$^{-3}$. 


\begin{figure}[t]
\centering
%
\includegraphics[width = 9cm]{figure3.pdf}
%\includegraphics[width = 9cm]{BLR_figures/model77_nr400_vang05-1000-7000.pdf}
\caption{Line spectrum calculated using Cloudy between 1000 and 7000 \AA. \civ$\lambda 1549$, \ciiif$\lambda 1909$, \ciip$\lambda 2326$, \mgii$\lambda 2798$, \hei$\lambda$4471, $\lambda$5876, and the hydrogen recombination lines are all marked.	The viewing angle is assumed to be $5^\circ$.}
% Here CII $\lambda 1335$, 
\label{wada_fig: spect1}
\end{figure}



\begin{figure}[t]
\centering

\includegraphics[width = 7cm]{figure4a.pdf}
\includegraphics[width = 7cm]{figure4b.pdf}
\includegraphics[width = 7cm]{figure4c.pdf}

%\includegraphics[width = 7cm]{BLR_figures/model77_nr100_vang5-10-30-6400-6700.pdf}
%\includegraphics[width = 7cm]{BLR_figures/model77_nr100_vang5-10-30-4800-5000.pdf}
%\includegraphics[width = 7cm]{BLR_figures/model77_nr400_vang5-10-30-mgII.pdf}
\caption{Spectra around H$\alpha$, H$\beta$, and \mgii$\lambda 2798$ for three viewing angles (5$^\circ$, 10$^\circ$ and 30$^\circ$).}


%Figure \ref{wada_fig: spect2} shows line profiles of H$\alpha$, H$\beta$ and \mgii $\lambda 2798$ for  three viewing angles.

\label{wada_fig: spect2}
\end{figure}

\begin{figure}[t]
\centering
\includegraphics[width = 9cm]{figure5.pdf}
%\includegraphics[width = 9cm]{BLR_figures/model78_phi64_nr1-400_vang5-10-30-SiIV-CIV.pdf}

\caption{Line profiles for three viewing angles (5$^\circ$ and 30$^\circ$): \siiv$\lambda 1397$, \niv$\lambda 1487$, \civ$\lambda 1549$. The rest-frame wavelengths are shown
by vertical lines. Note that blue- and redshifted components appear in both lines for small viewing angles.}
\label{wada_fig: spect3}
\end{figure}

We also observed significant differences in the line profiles of the high-ionization lines (\civ$\lambda 1549$ and \siiv$\lambda 1397$) as
compared with the hydrogen recombination lines and \mgii shown in Figure \ref{wada_fig: spect2}.
In Figure \ref{wada_fig: spect3}, \civ shows a double peak profile for $i = 5^\circ$ and a strong systemic component at 
the rest-frame wavelength. However, a single peak appears only around the systemic velocity for $30^\circ$.
\siiv also shows high-velocity components with a systemic component for $i = 5^\circ$. For $i = 30^\circ$, 
the profile is a single peak around the systemic component only. Note that the weaker \niv shows a double-peaked profile for $i = 30^\circ$, which is similar to that of
\mgii.
These results indicate that \civ ~and \siiv ~partially originated in the {\it outflows} located at the upper/lower surface of the disk, which 
are prominently displayed as the high line-of-sight velocity ($> 2000$ km s$^{-1}$) in the velocity map of the central 0.002 pc in Figure \ref{wada_fig: hydro}.

%%%%%%%%%%%%%%%%%%%%%%%%%%%%%%%%%%%%
%
\section{Discussion and conclusion}
%
%%%%%%%%%%%%%%%%%%%%%%%%%%%%%%%%%%%%
Since the 1980s, quasar broad emission lines have been known to blue-shift frequently from the systemic velocity, particularly 
for the high-ionization broad \civ line \citep[e.g.,][]{gaskell1982, wilkes1982}. 
This has also been confirmed for Sloan Digital Sky Survey quasars \citep[e.g.,][]{vanden_berk2001, richards2011, shen2016}.
The origin of the velocity shift remains under debate; it could be caused by outflows and osculation by the disk \citep{gaskell1982, chiang-murray1996}. However,
some observations, including those of velocity-resolved reverberation mapping and line width-time delay relation are not well explained by the disk wind model \citep{gaskell2016}, which implies that
the motion is dominated by gravity \citep{krolik1991}.


%\appendix

As described in this study, we found that the hydrogen recombination lines and most other emission lines show a wider line profile or double-peaked profiles for 
larger viewing angles (i.e., closer to edge-on). This is consistent with the notion that BLRs originate in the rotating disk \citep[see e.g., ][and references therein]{Storchi-Bergmann2016-hc}. However, our results also show emission lines originating in the radiation-driven wind.

In some disk wind models, two spatially distinct components are required to explain the properties of BLR \citep{collin1988}. Accordingly, \citet{yong2020} proposed 
that the velocity-shift between \civ $\lambda 1549$ and \mgii$\lambda 2798$ can be used to infer the orientation of the nucleus. 
This is basically consistent with what we found in Figure \ref{wada_fig: spect3}\footnote{In the spectra of type-1 AGNs, the Si~{\sc iv}$\lambda$1397 line is blended with the O~{\sc iv}]$\lambda$1402 line due to their broad nature. However, in typical situations of BLRs (i.e., with the solar or super-solar metallicity), the flux of O~{\sc iv}]$\lambda$1402 is less than $\sim$30\% of the total flux of the Si~{\sc iv}+O~{\sc iv}] blend (see, e.g., Figure~29 in \citet{nagao2006}). 
Thus we simply write \siiv$\lambda 1397$ to denote the \siiv+O~{\sc iv} blend.}.
In our present analysis, we assumed that we can observe both far and near sides of the outflows (see Fig. \ref{wada_fig: hydro}). 
If the far side outflow (i.e., redshifted component) is obscured by the dense disk, we expect that \civ and \siiv are
blueshifted with respect to the systemic velocity, or it may show an asymmetric profile. 


As \citet{Baldwin1995} indicated in their ``locally optimal cloud'' representation, the emission line spectrum can be reproduced by 
integrating various properties of emitting clouds, and the spectra do not necessarily represent 
physical conditions such as pressure, gas density, or ionization of individual clouds.
They also speculated that a chaotic cloud environment could be the source of the lines, and therefore the lines 
reflect mostly global properties of the clouds. The present results suggest that this rather chaotic depiction is naturally reproduced by the radiation-driven fountain model.

Using the narrow Fe-K$\alpha$ reverberation mapping for a changing-look AGN NGC 3516,  \citet{noda2023} found 
that  the Fe-Kα emitting radius in the type-2
phase is consistent with that of the BLR materials in the
type-1 phase. They claimed a possibility that the BLR materials remained at the same location as in the type-1
phase.  If the BLR material is mainly originated in the surface of the rotating disk as our results suggested, this observational fact 
can be naturally understood. However, farther investigation based on the radiation-hydrodynamic simulations by changing the
AGN luminosity would be necessary.

\begin{acknowledgments}

We thank G. Ferland and the \cloudy team for their regular support. 
Numerical computations of the radiation-hydrodynamic model were performed on a Cray XC50 at the Center for Computational Astrophysics at the National Astronomical Observatory of Japan and using the Fugaku supercomputer at RIKEN. This work was supported by JSPS KAKENHI Grant Number 21H04496.
The work used computational resources of Fugaku provided by RIKEN through the HPCI System Research Project (Project IDs: hp210147, hp210219).

\end{acknowledgments}



\bibliographystyle{aasjournal}
%\bibliography{agn_papers}
\bibliography{Paperpile-AGN-Sep02,agn_papers,agn_papers_takasao}


%% This command is needed to show the entire author+affiliation list when
%% the collaboration and author truncation commands are used.  It has to
%% go at the end of the manuscript.
%\allauthors

%% Include this line if you are using the \added, \replaced, \deleted
%% commands to see a summary list of all changes at the end of the article.
%\listofchanges

\end{document}

