\documentclass[aps,pre,preprint,superscriptaddress,onecolumn]{revtex4-2}

\usepackage{color}

\usepackage{epsfig,amssymb,amsmath,amsthm,amsfonts,amsbsy,mathrsfs}

\usepackage{graphicx}

\usepackage{subfigure}

\usepackage{float}

\usepackage{verbatim}

\usepackage{physics}

\usepackage{epstopdf}

% \usepackage{amsmath}
% \usepackage{amssymb}
% \usepackage{caption}
% \usepackage{subcaption}
% \usepackage[margin=1in, top=1.5in]{geometry}
% \setlength\headheight{20pt}
% \usepackage[slantedGreek]{mathptmx}
% \usepackage{url}
% \usepackage{stmaryrd}
% \usepackage{amsthm}
% \usepackage{pdfpages}

\newcommand{\sgn}{\mbox{\rm sign}}
\newcommand{\im}{\mathrm{i}}
\newcommand{\e}{\varepsilon}
% \newcommand{\grad}{\nabla}
\newcommand{\tf}{\tilde{f}}
\newcommand{\tF}{\tilde{F}}
\renewcommand{\div}{{\rm div}}
\renewcommand{\i}{{\rm \bf i}}
\newcommand{\bx}{{\bf x}}




\begin{document}

\title{Breakdown of Boltzmann-type Models for Self-propelled Rods}
%changed failure to breakdown and I think the title is fine as is. The intro clearly leads to this message so we are ok [OI]


\author{Patrick Murphy\footnote[1]{These authors contributed equally to this project}
\footnote[2]{corresponding authors}}
\affiliation{Department of Bioengineering, Rice University, Houston, TX 77005}
\affiliation{Center for Theoretical Biological Physics, Rice University, Houston, TX 77005}
\author{Misha Perepelitsa$^{\textrm{a},\textrm{b}}$
}
\affiliation{Department of Mathematics, University of Houston, TX 77204}
\email[Corresponding author: ]{mperepel$@$central.uh.edu}
\author{Ilya Timofeyev}
\affiliation{Department of Mathematics, University of Houston, TX 77204},
\author{Matan Lieber-Kotz}
\affiliation{Department of Bioengineering, Rice University, Houston, TX 77005}
\author{Brandon Islas}
\affiliation{Department of Computational and Applied Mathematics, Rice University, Houston, TX 77005}
\author{Oleg A. Igoshin}
\affiliation{Department of Bioengineering, Rice University, Houston, TX 77005}
\affiliation{Center for Theoretical Biological Physics, Rice University, Houston, TX 77005} 
\affiliation{Department of Chemistry, Rice University, Houston, TX 77005}
\affiliation{Department of Biosciences, Rice University, Houston, TX 77005}
\email[Corresponding author: ]{igoshin$@$rice.edu}
\date{\today}




\begin{abstract}
Boltzmann-type models are classically used to infer the collective dynamics of groups of interacting particles. However, their derivation relies on an assumption of "molecular chaos", or statistical independence, that may not be true in all contexts. In this work, we develop a Boltzmann-type model for nematic alignment of self-propelled rods where rod re-orientation occurs upon binary collisions. We identify relevant parameters and derive mean-field equations for the corresponding asymptotic regime. We simplify these equations to a closed-form system of PDEs where rods can only switch between a finite number of orientations. By comparing the numerical solutions of the system to an agent-based model that implements our microscopic alignment rules, we examine the accuracy of the derived kinetic model. The results indicate that Boltzmann-type models fail to replicate the underlying dynamics due to the formation of clusters that violate statistical independence. Additionally, the inclusion of mechanisms that limit cluster formation can help to improve the agreement between the model and agent simulations.
\end{abstract}

\maketitle




% \keywords{alignment of rods, polar alignment; Boltzmann-type model; mean field model.}

% \ccode{AMS Subject Classification: 22E46, 53C35, 57S20}



\section{Introduction}

% introduce models of alignment of rods and background examples
Self-propelled rods are a type of active matter seen across both living and non-living systems \cite{ndlec97,riedel05,narayan07,schaller10,sumino12,balagam14b,balagam15b}. Due to their shape, such systems are intrinsically capable of collective behavior through realignments of rods due to their collisions \cite{baskaran08,baskaran08a} or longer-ranged hydrodynamic interactions when in a fluid \cite{riedel05,baskaran09}.
These interactions lead to macroscopic collective motion such as flocking, clustering, phase changes, and vortexes. Several notable biological examples of collective motion in self-propelled rods are the dynamics of the soil bacterium \textit{Myxococcus xanthus} \cite{kaiser03,wu07,thutupalli15c}, and the behavior of groups of cellular cytoskeletal rods (such as F-actin \cite{schaller10} or microtubules \cite{sumino12}) driven by molecular motors deposited on the surface. Collisions between \textit{M. xanthus} cells often result in the head of the colliding cell reorienting along the length of the cell that was struck (see Fig. \ref{fig:myxo}) \cite{balagam14b}. Microtubule collisions on a 2D surface also exhibit asymmetric realignments upon collision, with the additional possibility of the colliding microtubule stalling until the struck microtubule has passed \cite{sumino12}. 

\begin{figure}[htbp]
\centering
\includegraphics[ width=1.00\textwidth]{images/myxo_collision.png}

\caption{An example of asymmetric alignment during the collision of two \textit{M. xanthus} cells due to cell-substrate forces. Adapted from Balagam 2014 \cite{balagam14b}.
\label{fig:myxo}
}
\end{figure} 


%%Introduction
%%%%%%%%%%%%%%%%%%%%%%%%%%%%%%%%%%%%%%%%%%%

% Active matter and self-propelled rods are subjects of recent research interest.
% Boltzmann models of alignment are classically used to describe many-body interactions
Due to the interest surrounding such systems, researchers have developed various methods of modeling them. In particular, many groups have approached studying the system's emergent behavior by deriving a mean-field description \cite{Bertin,degond18,degond08,Manhart1,baskaran08a,peruani06a,bolley12,hanke_understanding_2013,perepelitsa_mean-field_2022}. A common starting point for such derivations is Boltzmann formalism \cite{Bertin,Manhart1,hanke_understanding_2013,perepelitsa_mean-field_2022}, which starts with explicitly accounting for binary collisions and associated reorientation rules between rod pairs. For example, in Bertin et al (2006) \cite{Bertin}, the alignment of a pair of interacting cells reorients them to the average of their pre-collision orientations with a certain amount of noise added. The authors derive a Boltzmann-type equation for the distribution of cells and, based on that, postulate a reduced hydrodynamic-type model for the first few moments of the cell density distribution. In Hittmer et al (2021) \cite{Manhart1}, the authors consider a model in which a pair of co-oriented cells (rods with angle difference $|\theta_1-\theta_2| < \pi/2$) change their orientation to their average alignment, otherwise cells reverse their orientations in a reversal of motion. For this setip, a Boltzmann-type equation is formulated under the assumption $Nl/L$ is finite, where $N$ is the number of cells, $l$ is the cell length and $L$ is the characteristic length scale. For some special sets of cell orientation, the authors identify the set of equilibrium distributions and derive a hydrodynamic-type model for this set in the limit $Nl/L\to\infty.$

% Boltzmann formalism for self-propelled rods has received some critical investigations into its accuracy.
Over the last decade, the validity of the Boltzmann modeling approach has come into question. Evidence has shown that in some biological systems, the assumption of binary interactions is not strong enough to reproduce the observed dynamics \cite{suzuki_polar_2015}. Additionally, several studies have indicated that rapid cluster formation can lead to a strong violation of the molecular chaos assumption needed in the Boltzmann formalism \cite{hanke_understanding_2013,thuroff_critical_2013}. These studies suggest that biologically-relevant phenomena reliant on cluster formation \cite{peruani06a,peruani_collective_2012,weber_nucleation-induced_2012} may not be accurately modeled with Boltzmann-type equations.



% We investigated a Boltzmann-type model of alignment among self-propelled rods and identify where and why it fails.

The main goals of this paper are twofold: to derive Boltzmann-type equations for the collective motion of self-propelled rods from first principles involving the geometry of collisions, and to investigate the validity of these equations using computational simulations. We assume each rod is rigid and reorients upon collision with another rod, and that there is a refractory period for collisions related to the characteristic time scale of reorientations. These assumptions allow us to derive general equations of motion that can be reduced to intuitive cases when a finite number of orientations are present, resulting in a finite system of PDEs. We will show that the agent models exhibit cluster formation, with the formation rate increasing with higher numbers of rods. This correlates closely to a growing difference in the joint probability distribution for pairs of rods and the product of the marginal distributions, which indicates a loss of statistical independence. The interpretation is that clusters decrease the probability of finding pairs of rods with different orientations in the same local region of space.




%%%%%%%%%%%%%%%%%%%%%%
%%%%%Section 2%%%%%%%%
\section{Derivation of mean-field equation}

% Assumptions behind collision scheme are binary collisions, an internal refractory period based on characteristic time scale of reorientation, and synchronized internal clocks among rods.
Our goal is to derive a tractable mean-field description of a system of colliding self-propelled rods starting from a microscopic description of collisions. To this end, we make several assumptions about rod collisions. First, we assume that collisions are binary events, with the striking rod reorienting to match the struck rod's orientation modulo $\pi$. This results in the nematic alignment of rods. We also assume that reorientation occurs through the head of the rod. Realistically, reorientation does not take place instantaneously but over a period of time with a characteristic timescale given by $l/(2v)$, where $l$ is the length of a rod and $v$ is its velocity. To include this in our model, we assume that all rods have a refractory period of $l/(2v)$ following a collision. As we are focused on deriving a tractable model, we make two additional approximations. The first is that all rods have the same internal clock at time $t = 0$. The second is that if a rod experiences a collision in the next $\Delta t = l/v$ time after the refractory period is over, then it reorients halfway through after $\Delta t / 2$ time. This approximation is valid outside of high densities where the mean free path of rods is less than the length $l$, in which collisions are more likely to occur before $\Delta t / 2$ time has passed.

% Time shift calculation
Based on our assumptions that all rods' internal clocks are synchronized, we can model the dynamics as proceeding in discrete time steps $t_n=n dt,$ with $dt=\frac{l}{v}$. Rods will move with a ballistic motion for $dt/2$ time at speed $v$ in their given orientation, reorient at $t_n+dt/2$ if they would have experienced a collision in $[t_n, t_n+1]$, then move ballistically again for the remaining time until $t_{n+1}$. We fix a spatial location $x$ and two rods with orientations, $\theta$ and $\theta_1$, with heads at $x$. We then look at the sets of collisions in the next $dt$ time that will result in either a gain or a rod with orientation $\theta$ or an equivalent loss. This yields two regions $P_1(x,\theta,\theta_1)$ and $P_2(x,\theta,\theta_1)$ where another rod could be located to cause a collision (see Figure \ref{fig:2}A).

To simplify this setup, we perform a time shift of $\hat{t}_n = t_n + l/(2v)$ so that we can consider a sequence of events where a rod reorients if it collides, then moves ballistically with a single orientation for $dt$ time. This shift in time results in a corresponding shift in the parallelograms $P_1$ and $P_2$ so that they capture collisions that would occur in the time interval $[t_n, t_n+1] = [\hat{t}_n-l/(2v),\hat{t}_n+l/(2v)]$ (see Figure \ref{fig:2}B). We can now use this simplified setup to derive the change in the probability density $f(x,\theta,t)$ over the time interval $[\hat{t}_n,\hat{t}_{n+1}]$.

\begin{figure}[h!]
\centering
    \includegraphics[width=\textwidth]{images/fig2.png}
    \caption{A) Regions of integration for joint distribution at original time points $t_n$. Integrating over $P_1(x,\theta,\theta_1)$ and $P_2(x,\theta,\theta_1)$ yield the sink and source terms, respectively. B) Regions of integration for joint distribution at shifted time points $t_n+l/(2v)$. For the sake of illustration, $\theta_1$ is taken to be $0$.}
    \label{fig:2}
\end{figure} 

The change of the kinetic density function $f$ during $[\hat{t}_n,\hat{t}_{n+1}]$ can be written as
\begin{multline}
\label{Model3:Delta}
\Delta f_n(x,\theta) = f_{n+1}(x,\theta) - \hat{f}_n(x,\theta)
{}+{}\hat{f}_n(x,\theta)-f_n(x,\theta)\\
{}={}f_{n+1}(x,\theta) - f_{n+1}(x+e(\theta)vdt,\theta)
{}+{}\Delta_1 f_n(x,\theta),
\end{multline}
where $\hat{f}_n$ describes the values of $f_n$ after interaction, and 
\[
\Delta_1 f_n(x,\theta) = \hat{f}_n(x,\theta)-f_n(x,\theta).
\]

Consider the change of $f_n$ due to collisions. This quantity is found by integrating the probability of finding a pair of rods at $x$ and $x_1$ with orientations $\theta$ and $\theta_1$ over the parallelograms $P_1$ and $P_2$ illustrated in Figure \ref{fig:2}B. This probability is given by the joint density function (distribution for a pair of cells) $f_2(x,\theta,x_1,\theta_1,t_n) := f_{2,n}(x,\theta,x_1,\theta_1)$, resulting in
\[
\begin{split}
\Delta_1 f_n(x,\theta) =& 
-(N-1)\int_{-\pi}^{\pi}\int_{P_2(x,\theta,\theta_1)}f_{2,n}(x,\theta,x_1,\theta_1)\,dx_1d\theta_1\\
&+(N-1)\int_{\theta-\pi/2}^{\theta+\pi/2}\int_{P_1(x,\theta,\theta_1)}f_{2,n}(x,\theta,x_1,\theta_1)\,dx_1d\theta_1\\
&+(N-1)\int_{\theta-\pi/2}^{\theta+\pi/2}\int_{P_1(x,\theta+\pi,\theta_1)}f_{2,n}(x,\theta,x_1,\theta_1+\pi)\,dx_1d\theta_1\\
&:= -Q^-+Q^+_1+Q^+_2.
\end{split}
\]
We assume that the joint probability distribution $f_{2,n}(x,\theta,x_1,\theta_1)$ of finding a pair of rods at $x$ with orientation $\theta$ and at $x_1$ with orientation $\theta_1$ can be written as the product of the marginal distributions $f_n(x,\theta)$ and $f_n(x_1,\theta_1)$. This first-order moment closure is known as statistical independence, or molecular chaos, as it implies that the presence of one rod does not affect the probability of finding the other at the given position and orientation. With this assumption, we can use
\begin{equation}\label{eqInd}
\begin{split}
    f_{2,n}(x,\theta,x_1,\theta_1)=f_n(x,\theta)f_n(x_1,\theta_1)\\ f_{2,n}(x,\theta,x_1,\theta_1+\pi)=f_n(x,\theta)f_n(x_1,\theta_1+\pi)
\end{split}
\end{equation}
Next, we will simplify the integral terms using a Taylor expansion in $x_1$
\[
f_n(x_1,\theta_1) = f_n(x,\theta_1) + \grad_xf_n(x,\theta)\cdot(x_1-x) + O(|x_1-x|^2),
\]
so that only an integral over $\theta_1$ remains. A geometric computation shows that the areas $|P_1|=|P_2|=l^2|\sin(\theta-\theta_1)|.$
Substituting the expansion into the expression for $\Delta_1 f_n$ and using the integrals
\[
\int_{P_2(x,\theta,\theta_1)} (x_1-x)\,dx_1{}={}\frac{l^3}{2}|\sin(\theta-\theta_1)| \, e(\theta_1)
\]
and
\[
\int_{P_1(x,\theta,\theta_1)} (x_1-x)\,dx_1{}={}\frac{l^3}{2}|\sin(\theta-\theta_1)| \, e(\theta),
\]
we define
\begin{multline}
    Q^- = \int_{-\pi}^\pi l^2|\sin(\theta-\theta_1)|f_n(x,\theta)f_n(x,\theta_1)\,d\theta_1 \\
    {}+{} \int_{-\pi}^\pi\frac{l^3}{2}f_n(x,\theta)e(\theta_1)\cdot\grad_xf_n(x,\theta_1)\,d\theta_1 {}+{}O(l^4),
\end{multline}
\begin{multline}
    Q^+_1 = \int_{\theta-\pi/2}^{\theta+\pi/2} l^2|\sin(\theta-\theta_1)|f_n(x,\theta)f_n(x,\theta_1)\,d\theta_1 \\
    {}+{} \int_{\theta-\pi/2}^{\theta+\pi/2}\frac{l^3}{2}f_n(x,\theta_1)e(\theta)\cdot\grad_xf_n(x,\theta)\,d\theta_1 {}+{}O(l^4),
\end{multline}
\begin{multline}
    Q^+_2 = \int_{\theta-\pi/2}^{\theta+\pi/2} l^2|\sin(\theta-\theta_1)|f_n(x,\theta+\pi)f_n(x,\theta_1)\,d\theta_1 \\
    {}+{} \int_{\theta-\pi/2}^{\theta+\pi/2}\frac{l^3}{2}f_n(x,\theta_1)e(\theta)\cdot\grad_xf_n(x,\theta+\pi)\,d\theta_1 {}+{}O(l^4),
\end{multline}
where $e(\theta)$ is the unit vector in the direction of $\theta$.
Using the expressions above, we can write
\[
\begin{split}
    \Delta_1f_n(x,\theta) &= (N-1)l^2\int_{\theta-\pi/2}^{\theta+\pi/2} |\sin(\theta-\theta_1)|f_n(x,\theta)f_n(x,\theta_1+\pi)\,d\theta_1\\
    &{}-{}
    (N-1)l^2\int_{\theta-\pi/2}^{\theta+\pi/2} |\sin(\theta-\theta_1)|f_n(x,\theta+\pi)f_n(x,\theta_1)\,d\theta_1\\
    &{}-{}\frac{(N-1)l^3}{2}\int_{-\pi}^\pi|\sin(\theta-\theta_1)| f_n(x,\theta)e(\theta_1)\cdot\grad_xf_n(x,\theta_1)\,d\theta_1\\
    &{}+{}\frac{(N-1)l^3}{2}\int_{\theta-\pi/2}^{\theta+\pi/2}|\sin(\theta-\theta_1)|f_n(x,\theta_1)e(\theta)\cdot\grad_x(f_n(x,\theta)+f_n(x,\theta+\pi)\,d\theta_1 {}+{}O(Nl^4).
\end{split}
\]
Dividing \eqref{Model3:Delta} by $dt=\frac{l}{v}$ and assuming $f_n(x,\theta)$, $n = 1,2,\cdots$, are the values of continuously varying function $f(x,t,\theta)$ at times $t=t_n,$ we obtain 
\begin{equation}
\begin{split}
\partial_t f+ve(\theta)\cdot\grad_x f &=
(N-1)lv\int_{\theta-\pi/2}^{\theta+\pi/2} |\sin(\theta-\theta_1)\left(|f(x,t,\theta+\pi)f(x,t,\theta_1)-f(x,t,\theta)f(x,t,\theta_1+\pi\right)\,d\theta_1\\
&+\frac{(N-1)l^2v}{2}\int_{\theta-\pi/2}^{\theta+\pi/2}|\sin(\theta-\theta_1)|f(x,\theta_1)e(\theta)\cdot\grad_x\left(f(x,t,\theta)+f(x,t,\theta+\pi)\right)\,d\theta_1\\
&-\frac{(N-1)l^2v}{2}\int_{\theta-\pi/2}^{\theta+\pi/2}|\sin(\theta-\theta_1)|f(x,t,\theta)e(\theta_1)\cdot\grad_xf(x,t,\theta_1)\,d\theta_1+ O(Nl^3) + O(l).\\
\end{split}
\end{equation}
Finally, we write the kinetic equation as
\begin{equation}\label{eqn:kin1}
\partial_t f+ve(\theta)\cdot\grad_x f =
(N-1)lQ_0 + \frac{(N-1)l^2}{2}Q_1 + O\left(Nl^3\right) + O\left(l\right),
\end{equation}
where
\[
Q_0 {}={}\int_{\theta-\pi/2}^{\theta+\pi/2} |\sin(\theta-\theta_1)|\left(f(x,t,\theta+\pi)f(x,t,\theta_1)-f(x,t,\theta)f(x,t,\theta_1+\pi\right)\,d\theta_1,
\]
and
\[
\begin{split}
Q_1{}&={}\int_{\theta-\pi/2}^{\theta+\pi/2}|\sin(\theta-\theta_1)|f(x,\theta_1)e(\theta)\cdot\grad_x\left(f(x,t,\theta)+f(x,t,\theta+\pi)\right)\,d\theta_1\\
&-\int_{\theta-\pi/2}^{\theta+\pi/2}|\sin(\theta-\theta_1)|f(x,t,\theta)e(\theta_1)\cdot\grad_xf(x,t,\theta_1)\,d\theta_1.
\end{split}
\]



%%%%%%%%%%%%%%%%%%%%%%
%%%%%Section 3%%%%%%%%

%%%%%%%%%%%%%%%%%%%%%%%%%%%%%%%%%%%%%%%%%%%%%%%%%%%%%%%%%%%%%%%%%%%%%%%%%
%%%%%%%%%%%%%%%%%%%%%%%%%%%%%%%%%%%%%%%%%%%%%%%%%%%%%%%%%%%%%%%%%%%%%%%%%
%%%%%%%%%%%%%%%%%%%% SIMULATIONS %%%%%%%%%%%%%%%%%%%%%%%%%%%%%%%%%%%%%%%%
%%%%%%%%%%%%%%%%%%%%%%%%%%%%%%%%%%%%%%%%%%%%%%%%%%%%%%%%%%%%%%%%%%%%%%%%%
%%%%%%%%%%%%%%%%%%%%%%%%%%%%%%%%%%%%%%%%%%%%%%%%%%%%%%%%%%%%%%%%%%%%%%%%%


\section{Numerical solutions and results from agent-based simulations for case study}


\subsection{Model equations for test cases}
\label{sec:two_conservative}

To test the accuracy of our first-order mean-field equations, we compare the behavior of their numerical solutions to agent-based simulations. The agent-based model will implement the alignment dynamics of a finite set of rods colliding based on the physical rules from which the kinetic equations were derived. For our comparisons, we will focus on cases involving two bands of rods with distinct orientations given by angles $\theta_1 = \pi/4$ and $\theta_2 = 3\pi/4$. This ensures no additional orientations beyond the original two appear as a result of collisions. In this case, the term $Q_0$ vanishes.
We now only need to take the mean-field limit $N \to \infty$, $l \to 0$. Since the leading term is $O\left(Nl^2\right)$, we fix $Nl^2$ as a constant $K$ as we take the limit, leaving us with
\begin{equation}\label{eqn:kin2}
\partial_t f+v {\bf e}(\theta)\cdot\grad_x f = \frac{K}{2}Q_1.
\end{equation}

Since we are restricting to two possible orientations, we can substitute $f(x,\theta,t) = \rho_1(x,t)\delta(\theta - \theta_1) + \rho_2(x,t)\delta(\theta - \theta_2)$, yielding a system for $\rho = (\rho_1,\rho_2)$ of the form
\begin{equation}
    \label{Model3:2angles}
    \partial_t\rho + A(\rho)\partial_x\rho + B(\rho)\partial_y\rho {}={} 0
\end{equation}
where the matrices $A$ and $B$ are given by
\[
A{}={}
\left[
\begin{array}{rr}
v\cos(\theta_1)-\frac{K}{2}\cos(\theta_2)\rho_2 &\quad \frac{K}{2}\cos(\theta_1)\rho_1 \\
\frac{K}{2}\cos(\theta_2)\rho_2 &\quad v\cos(\theta_2)-\frac{K}{2}\cos(\theta_1)\rho_1
\end{array}
\right],
\]
\[
B{}={}
\left[
\begin{array}{rr}
v\sin(\theta_1)-\frac{K}{2}\sin(\theta_2)\rho_2 &\quad \frac{K}{2}\sin(\theta_1)\rho_1 \\
\frac{K}{2}\sin(\theta_2)\rho_2 &\quad v\sin(\theta_2)-\frac{K}{2}\sin(\theta_1)\rho_1
\end{array}
\right].
\]

To simplify the mean-field equations to a form easily analyzed, we will assume that all densities vary only in the spatial direction $x$. We use the spatial domain $(x,y) \in [0 \; L] \times [0 \; L]$, $L = 400$, with periodic boundary conditions, set the velocity $v = \sqrt{2}$, and scale the probability density to a number density $\rho$ such that $\sum_i \int \rho_i(x,t) dx = N$, where $N$ is the number of rods that will be used in the agent simulations. In this setting, the system (\ref{Model3:2angles}) can be written in conservation form as
\begin{equation}
    \label{Model3:2angles_sim}
    \begin{split}
    \partial_t\rho_1 + \partial_x\left(\rho_1(1-\frac{l^2}{2}\rho_2)\right)  {}&={}0\\
    \partial_t\rho_2- \partial_x\left(\rho_2(1-\frac{l^2}{2}\rho_1)\right)  {}&={}0.
    \end{split}
\end{equation}
Lastly, we define the dimensionless quantity $\kappa = \dfrac{K}{2 L^2}= \dfrac{N l^2}{2 L^2}$ to capture how the number and length of rods depend on each other. We restrict ourselves to the regime $0 < \kappa < 0.5$ (see Appendix A for the underlying rationale).

The agent-based model is simulated with $N$ rods. The initial locations of the rods will be determined by first scaling the initial conditions of the mean-field equations to create probability densities, then sampling from them. Collisions will then be determined using the schematic in Figure \ref{fig:2}. As the simulation is deterministic after initial rod locations are determined, we will average the profiles over 1000 simulations to remove random fluctuations and recover the mean behavior.

We will consider a test case with two waves moving towards each other over a background density. The linear factor $1-\frac{l^2}{2} \rho_i$ inside the spatial derivative will decrease in the both the left-and right-moving density profiles' speeds when they interact. Since the solutions of this system are 1D in space, we will compare them to agent-based simulations as follows. By looking only at the $x$-coordinates of the set of simulated rods, we can construct a 1D kernel density estimate (KDE) and compare the density profiles for each orientation as the profiles interact.







\subsection{Agent-based model fails to match mean-field model even for large $N$}

%Agent-based model fails to match mean-field model even for large $N$
Our agent-based simulations deviate from the numerical solutions of the mean-field equations, with the latter exhibiting far less change in the density profile shape (Figure \ref{fig:4} A\&B). The peaks of the density profiles for both orientations slow down slightly as they approach each other, but the slowdown and the increase in the peak density are much greater in the agent framework. If this discrepancy is due to not being near the mean-field limit $N \to \infty$, $l \to 0$, increasing the number of simulated rods while keeping $\kappa$ fixed should theoretically improve the agreement. However, we see that this is not the case, with the simulations using higher numbers of rods showing greater discrepancies (Figure \ref{fig:4} C). This suggests the disagreement is due to one or more flaws in the assumptions used to derive the mean-field equations.

\begin{figure}[h!]
\centering
    \includegraphics[width=\textwidth]{images/figures_all2-02.png}
    \caption{A) Numerical solution of the mean-field model for $\kappa = 0.1$. B) Density profile for agent-based simulations for $\kappa = 0.1$ with $N = 2856$ rods. C) Density profile for agent-based simulations for $\kappa = 0.1$ with $N = 45700$ rods. (Inset) The discrepancy at the final time between the mean-field model and the density profile from simulations for various numbers of rods. The results indicate that the agent simulations show much greater changes in the density profile and do not converge to the mean-field model in the mean-field limit $N \to \infty$, $l \to 0$.}
    \label{fig:4}
\end{figure}


\subsection{Discrepancy is linked to collision-driven cluster formation, which violates statistical independence}

%The loss of statistical independence can be theoretically linked with cluster formation.
The key assumption used in deriving the Boltzmann-type equation was the statistical independence of the joint probability distribution $f_2(x_1,\theta_1,x_2\theta_2,t)$, i.e. Eq. (\ref{eqInd}) . This allowed us to express the probability that a pair of rods with different orientations would have a spatial configuration leading to a collision in terms of the probability of each individually occupying the corresponding spatial region. A loss of statistical independence, therefore, has two cases with different interpretations. If the presence of a rod with orientation $\theta_1$ increases the probability of a rod with orientation $\theta_2$ being nearby, this would indicate that rods with orientation $\theta_1$ essentially recruit rods with orientation $\theta_2$. The second case involves the presence of a rod with orientation $\theta_1$ decreasing the probability of a nearby rod having orientation $\theta_2$. Simply put, it would be more likely to see rods with orientation $\theta_1$ clustering together.
%(measured as $(f_2(x,\xi,x_1,\xi_1,t) - f(x,\xi,t)f(x_1,\xi_1,t))/(f(x,\xi,t)f(x_1,\xi_1,t))$ )

%Our agent simulations show clusters form and grow over time, wit the rate of growth increasing for higher kappa.
Since our collision scheme results in co-aligned rods, cluster formation is possible and would be a violation of our assumption of statistical independence. To measure the clustering present in our agent simulations, we used a standard algorithm to group rods. The minimum distance between the heads of rods was picked to be the length of a rod $l$. We then measured the proportion of rods in clusters of size 4 or more (Fig. \ref{fig:5}A). As expected, clusters grew over time in our agent simulations, with the rate of growth increasing with $\kappa$ for fixed $N$. This increase in $\kappa$ corresponds to a greater rod length, which would increase the chance of collisions between rods.

% The loss of statistical independence is calculated using a global average of a coarse-grained approximation.
Since cluster formation is linked to a loss of statistical independence, we next measured the extent of this loss. We first divided our domain into $2^{10}$ square subregions $\Omega_{ij}$, $i,\, j = 1,\dots,2^5$. We then calculated coarse-grained approximations of the joint and marginal distributions in two steps. First, we calculated the proportion of rods for each orientation and the proportion of rod pairs with distinct orientations in each subregion. The local averages $f_2^{ij}(\theta_1,\theta_2,t)$, $f^{ij}(\theta_1,t)$, and $f^{ij}(\theta_2,t)$ in each subregion were then obtained by averaging over every 1000 simulations. We then chose the relative difference between the averages of the joint distribution function and the product of the marginals given by
\begin{equation}
E_r^{ij}(t) = (f_2^{ij}(\theta_1,\theta_2,t) - f^{ij}(\theta_1,t)f^{ij}(\theta_2,t))/(f^{ij}(\theta_1,t)f^{ij}(\theta_2,t))
\end{equation}
as a local estimate of the loss of independence. We finally obtain a global estimate
\begin{equation}
E_r(t) = \langle E_r^{ij}(t) \rangle_{ij}
\end{equation}
by averaging over all subregions.

% The loss of statistical independence is similar to the results of cluster growth.
As shown in Figure \ref{fig:5}B, the metric $E_r(t)$ shows similar trends to the cluster formation. The loss of statistical independence increases over time, with higher values of $\kappa$ showing a greater loss for fixed $N$. The fact that $E_r$ is negative reflects the fact that finding a pair of rods with different orientations is lowered from what is expected if statistical independence held true.


\begin{figure}[h!]
\centering
    \includegraphics[width=\textwidth]{images/figures_all2-03.png}
    \caption{A) The percentage of rods in clusters with more than 4 rods over time. This quantity grows over time as rods collide and co-orient. B) The loss of statistical independence $E_r(t)$ between the two sets of rods with different orientations. The negative values growing over time indicate that it becomes rarer to see nearby rods having different orientations, indicating cluster formation.}
    \label{fig:5}
\end{figure} 

%Data collapse suggests increased rod collision are responsible for increased clustering and loss of statistical independence with higher values of kappa.
Since cluster formation is the result of co-alignment between rods upon collision, it is natural to see if the increase in cluster growth with higher values of $\kappa$ is due to a greater number of collisions. Calculating the mean number of collisions per rod by time $t$ shows that the number does increase as $\kappa$ increases (Fig. \ref{fig:6}A). Furthermore, there is strong evidence of data collapse when plotting both the proportion of rods in clusters (Fig. \ref{fig:6}B) and the loss of statistical independence versus the cumulative number of collisions per rod (Fig. \ref{fig:6}C). The slight difference in the cluster measurement for $\kappa = 0.1$ results from some rods starting in clusters at $t = 0$. The reasons for the deviation in the rescaled $E_r$ cure for $\kappa = 0.1$ is less clear, but could be the result of cluster-cluster interactions once most rods reside in such clusters. The rescaling is based on rod-rod collisions, so cluster-cluster interactions might not fully be accounted for.


\begin{figure}[h!]
\centering
    \includegraphics[width=\textwidth]{images/figures_all2-04.png}
    \caption{A) The total number of collisions experienced on average per rod over time for $\kappa = 0.001$ (blue), $\kappa = 0.01$ (orange), and $\kappa = 0.1$ (purple). B) The fraction of rods in clusters with more than 4 rods versus the mean number of collisions experienced per rod by time $t$. This is plotted for $\kappa = 0.001$ (blue), $\kappa = 0.01$ (orange), and $\kappa = 0.1$ (purple) C) The normalized covariance between the two sets of rods with different orientations versus the mean number of collisions experienced per rod by time $t$. Both figures show some data collapse when compared to Figure \ref{fig:4}, especially for low mean rod collisions per rod.}
    \label{fig:6}
\end{figure} 



\subsection{Agent simulations can better match the mean-field model when clusters are destroyed by the addition of vertical diffusion}

% The introduction of diffusion in the y-direction can destroy clusters without changing the mean-field equations in the x-direction.
While the results of our simulations have indicated close links between cluster formation, loss of independence, and the discrepancy between the equations and agent simulations, they have not shown a strict cause and effect. To illustrate that clustering is the main cause of the discrepancy, we introduce diffusion in $y$-direction to our agent simulations. As the solution to the mean-field equations is assumed constant in the $y$-direction, no modification is needed to them. The addition of this noise will cause rods forming a cluster to slowly drift apart at a rate dependent on the noise strength.

% Vertical diffusion restores agreement between the agent simulations and the mean-field equations up to a point.
We implemented the same sets of agent simulations as before (N = 2856, $\kappa = 0.1$) with the addition of different levels of noise $\sigma$. Agent rods step a random amount drawn from a normal distribution $N(0,\sigma)$ in the $y$-direction every $\Delta t = l/v$. We further nondimensionalize the noise using the scaling $\sigma \rightarrow \sigma/l$. The results of the simulations show that the addition of noise in the $y$-direction improves agreement between the agent simulations and the mean-field equations, with the former now resembling the latter for a sufficient level of noise (Fig. \ref{fig:7}A). Increasing the noise strength from zero improves the discrepancy between the mean-field and agent density profiles (Fig. \ref{fig:7}B) and reduces the proportion of rods in clusters (Fig. \ref{fig:7}C). However, the discrepancy improves only up to a point. Once the strength of the scaled noise exceeds roughly $1/\sqrt{2}$, the discrepancy increases slightly before plateauing. Why this is the case in unclear, however, the reduction in clustering slows down at around the same level of noise.



\begin{figure}[h!]
\centering
    \includegraphics[width=\textwidth]{images/figures_all2-05.png}
    \caption{A) Density profile for agent-based simulations with $y$-directional scaled noise of $0.75$ for $\kappa = 0.1$ with $N = 2856$ rods. The scaled noise is a non-dimensional parameter obtained by scaling the noise strength (which has units of length) by the length of a rod. B) Discrepancy between the mean-field model and the agent-based simulations for different levels of white noise in the $y$-direction. The discrepancy initially drops, but increases slightly once the strength of the noise increases past $l/\sqrt{2}$, the length of a rod projected in the y-direction C) Percentage of rods in clusters with more than 4 rods versus noise in $y$-direction. The results indicate that the agent simulations better approximate the mean-field model when low- to moderate-level noise is added to the vertical motion to help break up clusters.}
    \label{fig:7}
\end{figure} 





\newpage 

% \section{Discussion}
\section{Conclusion}
% We demonstrated the resulting dynamics do not match the mean-field model due to the formation of clusters that violate our assumption of statistical independence. 
In this paper, we have shown how the Boltzmann formalism can fail when applied to collections of self-propelled rods. We explicitly measured the loss of statistical independence that invalidates the classic assumption of molecular chaos that Boltzmann-type equations rely on. Such a loss corresponds to the formation of rod clusters due to alignment from binary rod collisions. This result mirrors that in other studies \cite{hanke_understanding_2013,thuroff_critical_2013}, however, we have built upon this work by showing that mechanisms that destroy or inhibit cluster formation help restore agreement between the mean-field model and agent-based implementations of the microscopic alignment rules. As this discrepancy occurs even in the simple setup we consider, our work highlights the need to extend current methodologies beyond mean-field equations in order to accurately capture the properties of active matter biological systems. 


% Also talk about other works focusing on how Boltzmann model can fail (binary collisions insufficient for actomyosin interactions).

% Possible alternatives to Boltzmann formalism include kinetic modeling of clusters of various sizes and higher-order moment closures in BBGKY hierarchy.

% Hypotheses for why the discrepancy between mean-field equations and agent simulations is not fully restored.


\newpage


% \newpage




\begin{appendix}

%%%%%%%%%Mean Free Path%%%%%%%%%%
\section{Mean free path}

The mean free path is defined as the average distance that a rod moves between collisions. This distance can be computed as the product of the speed $v$ and the time $\tau$ a rod moves between re-orientations. To estimate $\tau,$ we first compute the number of rods with orientation ${\bf e}(\theta)=(\cos\theta,\sin\theta)$ that 
will collide with a given rod (rod 1) in the time $[t^*, t^*+dt]$, were $t^*$ marks the last change in rod 1's orientation. Without loss of generality, we consider rod 1 as moving horizontally to the right, as in Figure \ref{mfp}. Suppose that a coordinate system is chosen so that the rod is at rest. Then the velocity of rods with orientation $\xi$ is described by the vector ${\bf w}{}={}v(\cos\theta-1,\sin\theta).$ The heads of the rods that rod 1 can collide with are located in a parallelogram $P$ formed by vectors ${\bf w}dt$ and $l {\bf e}(\theta).$ The area of this parallelogram equals $|\sin\theta|vldt.$ Using that $Nf(x,\theta,t^*)$ is the local number of rods with a given orientation, we estimate the number of collisions with $\theta$-oriented rods in time $dt$ as $vl N f(x,\theta,t^*)|\sin\theta|dt,$ which is maximized at $\theta=\pi/2.$ 
Thus, the number of collisions over all orientations in $dt$ time can be estimated as
\[
vlNdt\int f(x,\theta,t^*)\,d\theta{}\leq{} n(x) vl dt,
\]
where $n(x,t^*)$ is the local rod number density.
From this, we obtain an upper bound for the frequency of collisions as $n(x,t^*)vl$. The time between collisions and the mean free path are then bounded by
\[
\tau \geq \frac{1}{n(x,t^*)vl},\quad 
d \geq \frac{1}{n(x,t^*)l}.
\]
Note that both $\tau$ and the mean free path $d$ are local quantities depending on $x.$

\begin{figure}[htbp]
\centering
\includegraphics[width=\textwidth]{images/figA2.png}

\caption{Interactions for the estimation of the mean free path. A) rod at $x$ where ${\bf e}(\theta)=(\cos\theta,\sin\theta)$ moves with velocity $v {\bf e}(\theta).$ In a reference coordinates of a horizontally moving rod with velocity $(v,0),$ the rod velocity is ${\bf w}.$ B) a rod moving horizontally to the right, during time interval $[t^*,t^*+dt]$ interacts with rods which have orientation ${\bf e}(\theta)$ whose centers are located in the parallelogram P, formed by vectors ${\bf w}dt$ and $l{\bf e}(\theta).$ The area of the parallelogram equals $|\sin\theta|v l dt.$
\label{mfp}}
\end{figure} 

For the model we consider in this paper, we restrict the mean free path to $d\geq l$. When $d < l$, rods will on average experience collisions at a rate equal to the inverse of the refractory period $\tau$. If $d<l/2$, rods are more likely to experience a collision before moving half their length. This violates our assumption that rods move half their length on average prior to a collision. This in turn would nullify our approximation that the refractory periods of all rods are in sync.

Our assumption that $d\geq l$ puts a restriction on the local density of the form $n(x,t^*)l^2\leq1,$ and subsequently on the parameter $\kappa$ of the form
\begin{equation}
\label{eq:kappa_bound}
\kappa = \frac{Nl^2}{2L^2}\leq \frac{1}{2}l^2\max_x\,n(x,t^*)\leq\frac{1}{2}.
\end{equation}
Here we use that $N/L^2$ is a lower bound on the maximum of $n(x,t^*).$ Note that we cannot have $\kappa = 0$ without the local density being identically 0.











%%%%%%%%%Hyperbolicity Regime%%%%%%%%%%
\section{Domain of hyperbolicity of system \eqref{Model3:2angles}}\label{append:hyper}
Denote the column vector $U=(\rho_1,\rho_2)^T$ and column of the  fluxes $F(U)=\frac{\sqrt{2}}{2}(\rho_1-\kappa\rho_1\rho_2,-\rho_2+\kappa\rho_1\rho_2)^T.$ Then the system of equations \eqref{Model3:2angles} is expressed as
\[
\partial_t U + \grad_x F(U){}={}0.
\]
The system is hyperbolic if the eigenvalues of the gradient matrix 
\[
\grad_U F(U){}={}
\frac{\sqrt{2}}{2}\left[
\begin{array}{cc}
1-\kappa\rho_2 & -\kappa \rho_1\\
\kappa \rho_2 & -1+\kappa\rho_1
\end{array}\right]
\]
are real. The eigenvalues equal
\[
\lambda_\pm = \frac{1}{2}\left(
\kappa(\rho_1-\rho_2) \pm \sqrt{\kappa^2(\rho_1-\rho_2)^2 + 4 - 4\kappa(\rho_1+\rho_2)}\right).
\]
Thus, the system is hyperbolic whenever
\[ 
\kappa^2(\rho_1-\rho_2)^2 + 4 - 4\kappa(\rho_1+\rho_2)\geq0.
\]
The sufficient condition is
\[
\kappa(\rho_1+\rho_2)\leq 1.
\]




%%%%%%%%%Numerical Methods%%%%%%%%%%
\section{Numerical Methods}\label{append:num}

 We express the system in \eqref{Model3:2angles_sim} in flux form
\begin{equation*}
\partial_t U = - \partial_x F, 
\end{equation*}
with $U = (\rho_1, \rho_2)^T$ and $F=A(\rho_1, \rho_2)^T$. In particular,
for two orientations $\pi/4$ and $3\pi/4$ the
fluxes become
$F_1 = \rho_1(1+\rho_2)/\sqrt{2}$ and $F_2 = \rho_2(1+\rho_1)/\sqrt{2}$.
Next, we discretize the equation in space and time using the Lax-Friedrichs method \cite{LeVeque2002}
\[
U_i^{n+1} = U_i^n - \frac{\Delta t}{\Delta x}
[\hat{F}_{i+1/2} - \hat{F}_{i-1/2}]
\]
and define 
\[
\hat{F}_{i+1/2} = \frac{F(U_{i+1}^n) + F(U_i^n)}{2} - \frac{K}{2}(U_{i+1}^n - U_i^n)
\]
where $K=\xi_{1,1}(1+|\xi_1\cdot\xi_2^\perp|)$ is the upper bound on the speed of propagation given $\rho_1, \, \rho_2 <1$. Therefore, the CFL condition for the scheme above is $K\Delta t / \Delta x < 1$, and the additional diffusion introduced by the scheme is proportional to $D = K \Delta x / 2$. The scheme is first order in space and time, but we found that it was sufficient for our simulations.





\end{appendix}


\bibliography{referencesMain}
% \bibliographystyle{Definitions/ws-m3as}



















\end{document}