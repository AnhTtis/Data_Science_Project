








\section{Usage Scenarios} 
\label{section:use_scenario}


This section describes two real-world cases, including a stacked area chart from an online website and a bubble chart on a news website.










\begin{figure*}[!htb]
    \centering
    \includegraphics[width=\textwidth]{image/case_stack_chart.png}
    \caption{Stacked area chart example from an online webpage. 
    Our approach supports (a-c) aligning, (d-f)rescaling, and (g-i) reordering.}
    \label{fig:case_stack_chart}
\end{figure*}


\subsection{Stacked Graph of Slack Software's Message Trends}


While a stacked graph is a widely utilized visualization technique for illustrating temporal changes across multiple categories, it poses challenges related to legibility, comparison, and scalability~\cite{Baur2012touchwave}.
In terms of legibility, the perceptual distortion called the sine illusion effect~\cite{vanderplas2015signs, day1991sine} significantly affects the perception of values.
More specifically, within the context of a stacked area chart, distinct segments positioned atop it often lead to substantial perceptual distortions due to variations in the underlying stacking slopes.
Various algorithms~\cite{byron2008stacked, sinestream2021} have been developed to compute optimized static layouts for stacked graphs, aiming to enhance their legibility.
Furthermore, scalability becomes a concern when numerous categories are present, particularly when small values are challenging to discern.
Moreover, as a stacked graph represents aggregate values through the stacking of visual elements, comparing values across different time points or visual objects becomes intricate for users.
Our proposed approach addresses these aforementioned challenges by providing an effective solution for static stacked graphs.



We extracted the stacked graph that describes the trends of the messages per channel in the Slack software, as shown in \autoref{fig:case_stack_chart} (a); this graph comes from the Preset website\footnote{https://preset.io/blog/2020-09-22-slack-dashboard/}.
The original chart is a ThemeRiver~\cite{havre2002themeriver} without alignment on the $x$-axis.
A user, Martin, wants to uncover insights given this chart.
Our method makes it possible to improve the legibility of the stacked area chart by allowing the flex alignment of the visual objects and allowing the user to change the order of the visual objects.
We set an aligned baseline on the $x$-axis that is composed of two parts, the upwards support constraints of the $x$-axis and the gravity constraints of the $x$-axis.
After setting such new constraints, the visual objects fall, driven by gravity, as shown in \autoref{fig:case_stack_chart} (b), but they stop at the $x$-axis because of the support constraints from the $x$-axis.
The visual objects finally align and stack on the $x$-axis, as shown in \autoref{fig:case_stack_chart} (c), which results in a rearrangement that allows better value retrieval and the recognition of the trend for the total value.

In \autoref{fig:case_stack_chart} (c), the values after June 12th are relatively small compared to the beginning value.
These values only take up a small proportion of the vertical space, which makes it difficult to determine the data values.
Our approach allows Martin to rescale the chart in the y-direction to explore the visualization better, as shown in \autoref{fig:case_stack_chart} (d), (e), and (f).
The rescaling changes the height of the visual objects.
The ticks in the $y$-axis are updated according to the new scale, which is calculated using the D3 scale function.
However, some problems still occur because of stacking.
One problem is the distortion of the stacking area.
The mark on the top is heavily distorted because of different slopes.
As shown in \autoref{fig:case_stack_chart} (f), it is difficult for Martin to determine the trend of the top visual object (deep grey) because of the unalignment.
Martin can directly drag the visual object to the bottom to solve this problem.
The collision order of the visual objects is updated.
Collision constraints and support constraints cause the visual objects to converge at (i).
Consequently, the legibility, comparison, and scalability problems can be handled through these interactions.











\subsection{New York Times Vaccination Rate Bubble Chart}

Bubble charts are extensively discussed in academic papers (e.g., FluxFlow~\cite{fluxflow}) and are also widely used in news media.
Each point in the chart represents a data item.
Rearranging these data items can support various analytical tasks performed by users.

One example is the COVID-19 vaccination rate, which has received significant media attention.
The New York Times\footnote{\url{https://www.nytimes.com/2021/05/12/us/covid-vaccines-vulnerable.html}} published a chart that shows vaccine hesitancy, social vulnerability, and vaccination rates for each county in the US (\autoref{fig:teaser} (a)).
Each circle represents a county, with four different colors representing different regions (yellow: Northeast, blue: West, green: South, red: Midwest) and with the sizes of the circles indicating the population of each county.
In the original chart (\autoref{fig:teaser} (a)), from top to bottom, the counties are classified into four categories: ``low hesitancy and low vulnerability,'' ``high hesitancy and low vulnerability,'' ``low hesitancy and high vulnerability,'' and ``high hesitancy and high vulnerability.''

\begin{figure}[!ht]
    \centering
    \includegraphics[width=\columnwidth]{image/vaccine_chart_horizontal}
    \caption{A bubble chart showing the vaccination rate of counties in the US. 
    Our method adds various interactions to the (a) original chart, supporting a wide range of tasks that involve manipulating the marks in the static visualization. 
    For example, in (a), (c), and (g), the visual objects are driven by different gravity constraints in the x-direction.}
    \label{fig:teaser}
\end{figure}



The chart encodes five data attributes: hesitancy, vulnerability, vaccination rate, population, and region.
Assume a user, Martin, reads the news on this website, and he wants to uncover insights by exploring the visualization.
Our approach can perform various manipulations of the chart to support data exploration.
Martin wants to explore the bubbles in detail. 
He can navigate the chart by zooming in on the $x$-axis, resulting in the view shown in \autoref{fig:teaser} (b).
In \autoref{fig:teaser} (b), he can see the details of specific counties' vaccination rates.
He wants to compare counties from different regions (colors), so he sets new x-gravity constraints in the x-direction that drives different colors to move to different positions.
The optimization process is shown in \autoref{fig:teaser} (c), where the points are moving to a new position.
The converged circles in \autoref{fig:teaser} (g) are rearranged by color; this is a kind of aggregation grouped by regions. 
Circles of the same region and category are aggregated together to estimate the population of each subset using the whole-area size of the group.
In the Northeast and West, most counties are in the low-hesitancy category, while in the South and Midwest, a large number of counties are in the high-hesitancy category.


Martin hypothesizes that counties with larger populations may exhibit higher vaccination rates. As a result, he intends to investigate the correlation between population size and vaccination rate. To achieve this, he establishes distinct y-gravity constraints corresponding to the population (radius). The resulting chart is depicted in \autoref{fig:teaser} (f), revealing a relatively weak correlation between population and vaccination rate. Nevertheless, counties characterized by low vaccination rates tend to be smaller in size. Since Martin resides in a county with limited hesitancy and vulnerability, he expresses a keen interest in the distribution of counties within the same category. He can effortlessly select this category by dragging it onto a new canvas, as showcased in \autoref{fig:teaser} (d). Additionally, he can configure y-gravity constraints for various colors (regions) as shown in \autoref{fig:teaser} (e).

These explorations yield Martin a deeper level of insight compared to the original chart. In conclusion, our approach empowers users to select, filter, arrange, and aggregate marks while retaining the contextual information present in the original chart.
Our model can also be directly applied to many other bubble charts. For instance, the bubble chart in the FluxFlow~\cite{fluxflow} can serve as an input for our model, allowing for rearrangement and recombination. Additionally, changes in Visual Sedimentation~\cite{Huron2013Visual} that do not involve changes in the type of visual elements (such as from bubbles to area sedimentation) can also be represented using various constraints.





























































