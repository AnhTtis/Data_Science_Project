\section{Related Work}
\label{section:background}

We contribute a spatial constraint approach to model visual objects, which is related to the direct manipulation of visual elements, interaction modeling in visualization, and the concept of force-directed layouts.

\subsection{Direct Manipulation for Visual Objects}


Direct manipulation provides an interface with continuous representation, physical actions, and ongoing feedback~\cite{directManipulation}, and it is more intuitive than traditional interactions, as it leverages physical concepts in the interface design.
Some approaches incorporate force metaphors in visualization; for example, the magnet metaphor~\cite{soo2005dust} and attractive and repulsive forces~\cite{rzeszotarski2014kinetica} are utilized to assist users in interacting with multivariate data items.
However, these approaches primarily focus on unit visualization; they treat each point as an individual data item.
Other studies use force metaphors to support visualization tasks.
For instance, Tominski et al.~\cite{tominski2012interaction} employed a force metaphor to facilitate folding interactions for comparisons.
While many prior approaches have integrated physical forces into visualization, where each data item is represented as a point (e.g., in graph layouts or unit visualizations), our approach is more generic; it applies constraints to various types of visual objects (e.g., bars and areas).
Saket et al.~\cite{saket2019investigating} proposed a space for manipulation of existing visualizations, such as resizing, recoloring, and repositioning. 
In Saket et al.'s approach, some operations like recoloring or repositioning actually introduce user-annotated data, which might lead to some changes in the data.
However, The direct manipulations in our model emphasize not altering the underlying data.


\subsection{Interaction Modeling for Visualization}

Visualization authoring tools, such as D3~\cite{bostock2011d3}, Vega-Lite~\cite{satyanarayan2016vegalite}, and ECharts~\cite{li2018echarts}, offer interaction features that enable users to initiate updates of visual objects in response to actions.
Nebula~\cite{chen2021nebula} introduces a grammar for modeling multi-view interactions.
Some other methods~\cite{choi2015visdock, harper2014deconstructing} aim to enhance existing visualizations through interactions.
VisDock~\cite{choi2015visdock} permits users to incorporate interactions (e.g., selection, filtering, navigation, etc.) into established visualizations using code.
Harper and Agrawala~\cite{harper2014deconstructing}, leveraging the characteristics of D3's DOM element specifications, deconstruct existing D3 visualizations by aligning provided data with visual attributes.
Moreover, Harper and Agrawala~\cite{harper2018converting} extract D3 visualizations and transform them into templates for reusability.
These methods demand users to be familiar with the authoring tool and coding of an existing visualization.
In contrast, our approach treats visual objects as physical entities with spatial constraints, making it agnostic to the specific authoring tool kits.



The utilization of physical force metaphors in visualization has been a subject of exploration over the past decades. Huron et al.~\cite{Huron2013Visual} introduced a sedimentation metaphor for visualization.
Saket et al.~\cite{saket2016visualization} reconfigure visualization layouts by directly manipulating visual objects and deducing the mapping of data attributes.
The approach proposed in this method can accommodate pre-existing visualizations and render them interactive, thus enhancing their utility.
In contrast, other methods are tailored to visualizations developed within their specific systems.
The methodology presented in this manuscript introduces a novel interpretation for the positioning of visual objects, their motion dynamics, and the mechanics of visual element movement in pre-existing visualizations.
This manuscript puts forward a scheme to guide the orchestrated motion of these visual elements after manipulation. Unlike other approaches that rely on metaphorical constructs for visualization design, our approach distinguishes itself by orchestrating the autonomous relocation of visual objects to strategically determined positions through the application of forces. Remarkably, these positions inherently encapsulate an effective mode of visual representation. From this perspective, the approach advocated in this study exhibits heightened generality and adaptability.



Interaction+~\cite{lu2017interaction} is also tool-agnostic; it parses the attributes of visual objects and applies additional interaction components to the visualization.
However, Interaction+ primarily targets non-spatial attributes such as color and opacity, which cannot be utilized to update the spatial positions of visual objects.

\subsection{Force-Directed Layout}


Force-directed algorithms have found extensive application in graph layout visualizations~\cite{forcedirected}.
The spring-electrical model, which combines attractive and repulsive forces, was initially introduced by Eades~\cite{eades1984heuristic}.
Building upon Eades's work~\cite{eades1984heuristic}, Fruchterman and Reingold~\cite{fruchterman1991graph} proposed the spring-embedder model.
Furthermore, Kamada and Kawai~\cite{kamada1989algorithm} treated the force-directed layout problem as an optimization challenge.
The force-directed graph layout problem can be reformulated into a constrained energy minimization problem~\cite{dwyer2005dig, gansner2004graph, zheng2018graph}.
Certain approaches~\cite{hu2005efficient, hachul2004drawing} introduce multi-level techniques to reduce computational complexity; these methods yield efficient and high-quality outcomes for drawing large graphs.
Moreover, users might impose customized constraints on graph nodes, such as fixing a node's position or maintaining a constant distance between two nodes.
Numerous approaches~\cite{kamps1995constraint, dwyer2006ipsep} have been developed to address such tailored constraints.

Apart from calculating graph layouts, the concept of constraints has been widely employed for interactive interaction; for instance, it can be harnessed to create a graph interface~\cite{garnet} and specify interactive objects~\cite{carr1994specification}.
In our approach, spatial constraints are utilized to guide visual objects toward equilibrium states.



















































































 




















































