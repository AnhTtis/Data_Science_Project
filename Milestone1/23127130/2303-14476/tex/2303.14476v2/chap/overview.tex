





\section{Method Overview}
\label{section:overview}








User interactions trigger changes in visual representations.
As per Mackinlay~\cite{mackinlay1986APT}, spatial channels (position, size, shape) are deemed the most effective channels.
Consequently, many changes prompted by interaction can be conceptualized as spatial modifications of visual objects.
We can employ Brehmer and Munzner's~\cite{Brehmer2013TaskAbstract} classification of interaction to explore how numerous interactions can be interpreted as spatial changes.
According to Brehmer and Munzner, interaction techniques applied to existing visual objects fall under the category of manipulation, encompassing actions like selecting, navigating, arranging, changing, filtering, and aggregating visual objects.


\begin{itemize}
\item \textbf{Rearrange current visual objects:} A substantial portion of visualization interactions revolves around adjusting the spatial layout of existing visual elements. For example, resizing a logarithmic axis to facilitate navigation within a visualization or transforming a stacked bar chart into a grouped bar chart for altered visual representation. Sorting elements within a visualization also falls into this category.
\item \textbf{Reduce current visual objects:} Certain interactions aim to diminish information content. For instance, actions such as selecting and filtering. Selection doesn't modify the visualization's content, whereas filtering or removing a subset of visual objects does.
\end{itemize}











\begin{figure*}[!ht]
    \centering
    \includegraphics[width=\textwidth]{image/whole_overview}
    \caption{We present a spatially-constrained conceptual model. Building upon this conceptual model, we have implemented a prototype system. These interactions can facilitate various user interaction tasks and accommodate diverse user intentions.
   }
    \label{fig:whole_overview}
\end{figure*}

From this classification, we can observe that changes triggered by many types of interactions can be described as spatial changes.
Therefore, the spatial constraint method can model numerous interactions and the resulting spatial changes.
To support these spatial-related interaction tasks, we propose a spatial constraint method that models the positions of visual objects.
Direct manipulation of visual objects allows the constraints to guide them into a new stable state, resulting in smooth transitions during interactions.
The spatial changes support users' tasks, such as comparison and identification.






As shown in \autoref{fig:whole_overview}, we present a conceptual framework in the form of a spatial constraint model that elucidates spatial relationships among visual objects through spatial constraints. Specifically, it clarifies the rationale behind the existing positions of visual objects and anticipates their transformations when subjected to changes. Visual objects characterized by spatial constraints are suitable for accommodating spatial interactions. Building upon this conceptual foundation, we have developed a prototype system. This system extracts existing visualizations, including the visual objects within them, and deduces latent spatial constraints among these entities. Furthermore, it facilitates the direct manipulation of visual objects, coordinate axes, and spatial constraints based on these inferred constraints.
These direct manipulations empower users with data analysis tasks to execute a diverse array of actions aimed at addressing a spectrum of user intents.

\begin{figure*}[!ht]
    \centering
    \includegraphics[width=\textwidth]{image/pipeline}
    \caption{A chart can be constructed using four fundamental constraints: fixed, collision, support, and gravity constraints. These constraints facilitate the direct manipulation of the visualization, encompassing (a) and (b) the manipulation of visual objects, (c) manipulation of axes, and (d) manipulation of constraints.
    Spatial constraints are translated into forces that prompt positional changes in visual objects, analogous to how forces impact the positions of physical objects. The resultant converged state enables a variety of interactive tasks, including rearrangement, deletion, and navigation.
   }
    \label{fig:pipeline}
\end{figure*}



As shown in \autoref{fig:pipeline}, spatial constraints are established for the visualization, allowing users to manipulate visual objects, axes, and constraints through intuitive actions.
The manipulated control points and positions of constraints serve as inputs for the optimization process, which utilizes encoded forces to alter the positions of visual objects and attain a new stable layout.
The concept of spatial constraints and the four types of atomic constraints are explored in \autoref{section:constraints}, while \autoref{section:manipulation} delves into direct interactions with the visualization supported by spatial constraints.

























































    






