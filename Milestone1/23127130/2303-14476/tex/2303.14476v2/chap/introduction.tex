\section{Introduction}

Interaction plays a crucial role in facilitating a dialogue between users and visualizations, providing alternative representations and different perspectives during the process of data exploration. Despite its importance, many visualizations in the real world are static or have limited interactive capabilities. This issue arises from both the creators' and users' perspectives. On the creators' side, traditional paper-based practices and the specialized knowledge required for visualization programming (using tools like $D^3$~\cite{bostock2011d3}, Vega~\cite{satyanarayan2015reactive}, or ECharts~\cite{li2018echarts}) contribute to the lack of interactivity in visualizations. On the users' side, complex interaction requirements beyond predefined options can also hinder the realization of interactive visualizations.



A comprehensive interaction process compasses both user actions and visualization changes.
Spatial change is one of the most prevalent types of changes, given that the spatial channel is one of the most perceptually effective visual channels~\cite{mackinlay1986APT}.
Therefore, many interactions can be categorized as spatial changes affecting elements within the visualization.
For example, re-encoding a grouped bar chart as a stacked bar chart or changing the stacking order of a ThemeRiver~\cite{havre2002themeriver} both fall under spatial transformations within the visualization, altering the position, shape, or size of visual elements.
Considering the prevalence of static visualizations and their potential benefits from interaction, we propose a spatial constraint model that can facilitate spatial interactions within visualizations.
This model involves defining spatial constraints on visual objects within the visualization, enabling user interactions with the original static visualizations, and supporting a wide range of interactions with them.

In the spatial constraint model, we conceptualize visual objects (such as points, bars, areas, etc.) as entities governed by physical principles. These visual objects are subjected to a set of spatial constraints, which encompass gravity, support, collision, and fixation constraints. These constraints are integrated to facilitate the construction of intricate interactions, forming the foundation for the sequence of visual object updates prompted by interactions. In contrast to conventional tool kits (e.g., D3~\cite{bostock2011d3}) that treat the positioning and shaping of visual objects as outcomes of pre-programmed functions, we adopt a physical perspective to model the spatial channels of visual objects. Crucially, our approach introduces spatial updates that are activated by interactions. From this standpoint, the stable positioning of visual objects emerges from the equilibrium achieved through the interplay of spatial constraints, drawing inspiration from the concept of object equilibrium.

To validate the spatial constraint model, we have developed a prototype system designed for the interactive manipulation of existing visualizations.
Following manipulation, the visualizations are transformed to converge towards new layouts, enhancing users ability to explore the visualization and underlying data.
These interactions facilitate a range of user tasks, such as reordering visual objects through dragging to alter stacking orders or orientations, zooming or dragging scales to reorganize or rescale axes, introducing new constraints, or modifying existing ones.
These operations enable intelligent interactions to support user analysis tasks including navigation, filtering, reordering, re-encoding, and aggregation. Subsequent to user interactions with the visualization, our system maps spatial constraints to forces and propels visual objects toward convergent positions using force-directed optimization. The resultant convergence leads to the emergence of a new visualization layout. Our model is applicable to various common visualization types, such as bar charts, area charts, bubble charts, and line charts.
We demonstrate the efficacy of our approach in supporting diverse interactive tasks for common visualizations through real-world examples and user studies.
We conducted user experiments with participants who have varying levels of expertise in visualization. The experiments demonstrated that the interactions supported by the spatial constraint model are intuitive and easy to understand, and they can facilitate diverse forms of interaction. Therefore, our approach is beneficial for both novice and expert users.
The contributions of this work encompass the following:
\begin{itemize}
\item A spatial constraint model that effectively represents both the positioning and positional changes of visual objects, while also facilitating intuitive transitions after user manipulations of visualizations. 
\item A prototype system grounded in the constraint model, enabling the activation and manipulation of static visualizations by users.
\end{itemize}




























































































 









 











 





































































