








\section{Usage Scenarios} 
\label{section:use_scenario}


This section describes two real-world cases, including a stacked area chart from an online website and a bubble chart on a news website.










\begin{figure*}[!htb]
    \centering
    \includegraphics[width=\linewidth]{image/case_stack_chart.pdf}
    \caption{Stacked area chart example from an online webpage. 
    Our approach supports (a-c) aligning, (d-f)rescaling, and (g-i) reordering.}
    \label{fig:case_stack_chart}
\end{figure*}


\subsection{Stacked Graph of Slack Software's Message Trends}

Although a stacked graph (or stacked area chart) is a popular visualization method that can present the temporal changes in several categories, it has problems in terms of legibility, comparison, and scalability~\cite{Baur2012touchwave}.
On the legibility side, the distortion called the sine illusion effect~\cite{vanderplas2015signs, day1991sine} influences the perceived values. 
Many algorithms~\cite{byron2008stacked, sinestream2021} calculate an optimized static layout of the stacked graphs to increase their legibility.
Moreover, the scalability problem matters when there are many categories, and the small values will be challenging to read.
Furthermore, as a stacked graph shows the sum value by stacking all the visual marks together, it is difficult for the users to compare the values of two different time positions or two visual marks.
Our approach can be applied to a static stacked graph to sufficiently handle the problems described above.

We extracted the stacked graph that describes the trends of the messages per channel in the Slack software, as shown in \autoref{fig:case_stack_chart} (a); this graph comes from the Preset website\footnote{https://preset.io/blog/2020-09-22-slack-dashboard/}.
The original chart is a ThemeRiver~\cite{havre2002themeriver} without alignment on the x-axis.
A user, Martin, wants to uncover insights given this chart.
Our method makes it possible to improve the legibility of the stacked area chart by allowing the flex alignment of the visual marks and allowing the user to change the order of the visual marks.
We set an aligned baseline on the x-axis that is composed of two parts, the upwards support constraints of the x-axis and the gravity constraints of the x-axis.
After setting such new constraints, the visual marks fall, driven by gravity, as shown in \autoref{fig:case_stack_chart} (b), but they stop at the x-axis because of the support constraints from the x-axis.
The visual marks finally align and stack on the x-axis, as shown in \autoref{fig:case_stack_chart} (c), which results in a rearrangement that allows better value retrieval and the recognition of the trend for the total value.

In \autoref{fig:case_stack_chart} (c), the values after June 12th are relatively small compared to the beginning value.
These values only take up a small proportion of the vertical space, which makes it difficult to determine the data values.
Our approach allows Martin to rescale the chart in the y-direction to explore the visualization better, as shown in \autoref{fig:case_stack_chart} (d), (e), and (f).
The rescaling changes the height of the visual marks.
The ticks in the y-axis are updated according to the new scale, which is calculated using the D3 scale function.
However, some problems still occur because of stacking.
One problem is the distortion of the stacking area.
The mark on the top is heavily distorted.
As shown in \autoref{fig:case_stack_chart} (f), it is difficult for Martin to determine the trend of the top visual mark (deep grey) because of the unalignment.
Martin can directly drag the visual mark to the bottom to solve this problem.
The collision order of the visual marks is updated.
Collision constraints and support constraints cause the visual marks to converge at (i).
Consequently, the legibility, comparison, and scalability problems can be handled through these interactions.










\begin{figure}[!ht]
    \centering
    \includegraphics[width=\columnwidth]{image/vaccine_chart_horizontal.pdf}
    \caption{A bubble chart showing the vaccination rate of counties in the US. 
    Our method adds various interactions to the (a) original chart, supporting a wide range of tasks that involve manipulating the marks in the static visualization. 
    For example, in (a), (c), and (g), the visual marks are driven by different gravity constraints in the x-direction.}
    \label{fig:teaser}
\end{figure}


\subsection{New York Times Vaccination Rate Bubble Chart}

Bubble charts are commonly used in the news.
Each point in the chart represents a data item.
Rearranging these data items can support various analytical tasks performed by users.

One example is the COVID-19 vaccination rate, which has received significant media attention.
The New York Times~\footnote{\url{https://www.nytimes.com/2021/05/12/us/covid-vaccines-vulnerable.html}} published a chart that shows vaccine hesitancy, social vulnerability, and vaccination rates for each county in the US(\autoref{fig:teaser} (a)).
Each circle represents a county, with four different colors representing different regions (yellow: Northeast, blue: West, green: South, red: Midwest) and with the sizes of the circles indicating the population of each county.
In the original chart (\autoref{fig:teaser} (a)), from top to bottom, the counties are classified into four categories: ``low hesitancy and low vulnerability,'' ``high hesitancy and low vulnerability,'' ``low hesitancy and high vulnerability,'' and ``high hesitancy and high vulnerability.''

The chart encodes five data attributes: hesitancy, vulnerability, vaccination rate, population, and region.
A user, Martin, reads the news on this website, and he wants to uncover insights by exploring the visualization.
Our approach can perform various manipulations of the chart to support data exploration.
Martin wants to explore the bubbles in detail. 
He can navigate the chart by zooming in on the x-axis, resulting in the view shown in \autoref{fig:teaser} (b).
In \autoref{fig:teaser} (b), he can see the details of specific counties' vaccination rates.
He wants to compare counties from different regions (colors), so he sets new x-gravity constraints in the x-direction that drives different colors to move to different positions.
The optimization process is shown in \autoref{fig:teaser} (c), where the points are moving to a new position.
The converged circles in \autoref{fig:teaser} (g) are rearranged by color; this is a kind of aggregation grouped by regions. 
Circles of the same region and category are aggregated together to estimate the population of each subset using the whole-area size of the group.
In the Northeast and West, most counties are in the low-hesitancy category, while in the South and Midwest, a large number of counties are in the high-hesitancy category.

Martin hypothesizes that the counties with larger populations may have higher vaccination rates, so he wants to observe the correlation between the size and vaccination rate.
He sets different y-gravity constraints according to the population (the radius).
\autoref{fig:teaser} (f) is the resulting chart;
it indicates that there is not a strong correlation between the population and vaccination rate. 
Nevertheless, generally, the counties with low vaccination rates are small.
Since Martin lives in a county with low hesitation and low vulnerability, he is interested in the distribution of counties in the same category.
He can select this category by dragging it to a new canvas, as \autoref{fig:teaser} (d) shows.
Furthermore, he can set the y-gravity constraints for different colors (regions) (\autoref{fig:teaser} (e)).

These explorations give Martin more insights than the original chart.
In summary, our approach allows users to select, filter, arrange, and aggregate the marks without losing the context information in the original chart.























































