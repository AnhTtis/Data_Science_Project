

\section{Manipulation Supported by Constraints}
\label{section:manipulation}

\revision{As shown in \autoref{fig:manipulate_classification}, there are three kinds of direct manipulations that can be classified according to the type of object being manipulated;
there are manipulations of visual objects, axes, and constraints.
These manipulations are denoted as object-level, axis-level, and constraint-level manipulations, respectively.}


\begin{figure*}[!ht]
  \centering
  \includegraphics[width=\textwidth]{image/manipulate_classification.pdf}
  \caption{
      Our work allows users to directly manipulate visual objects, axes, and constraints.
      Our method supports various interaction tasks in visualizations.
  }
  \label{fig:manipulate_classification}
\end{figure*}

\subsection{Manipulating Visual Objects}
\label{sec:manipulate_object}

Object-level manipulations directly change the positions of visual objects.
Manipulations with different directions, distances, and speeds imply various user intentions.
For example, dragging visual objects in the collision direction means changing the stacking or group order, while dragging a visual object out of the current canvas implies moving it onto a new canvas.


\textbf{Changing the stacking direction.}
A set of visual objects can be stacked together in the x-direction (e.g., a grouped bar chart) or y-direction (e.g., a stacked bar chart).
The constraints for changing the stacked direction involve setting the collision direction into another direction.
For example, in \autoref{fig:manipulate_classification} (a), when the collision constraints in the y-direction are changed to the x-direction, the new collision constraints cause the blue bars to move to the left while the downward gravity causes the blue bars to fall.
In a bar chart, the change in the direction supports the comparison of categories (changing from a stacked bar chart to a grouped bar chart) or computes a summary of categories (changing from a grouped bar chart to a stacked bar chart).


\textbf{Changing the stacking order.}
As shown in \autoref{fig:manipulate_classification} (b), the user drags the pink bar above the blue one, which implies that the user wants to change the stack order of these two visual objects.
We recalculate the collision order according to the new positions of these visual objects, i.e., for each tick area, the horizontal position of the pink bar is set to be higher than that of the blue one. 
\revision{As a result, the blue bar is pushed down to the x-axis by the gravitational force, and the pink bar falls and is stacked on top of the blue one.}
\removed{based on the new order, i.e., the spatial position of the read element is larger than the blue one, and the position of the blue element is larger than the yellow one.}

\textbf{Moving to a new canvas.}
When a visual object (a group of visual objects) is dragged off the canvas (\autoref{fig:manipulate_classification} (c)), a new canvas is constructed to hold the visual object.
The new canvas inherits the spatial constraints of the original canvas.
The visual objects of each canvas are independently manipulated and optimized.
Visual objects can be dragged to move them between existing canvases.
Quickly dragging and dropping the visual objects away from the canvas will delete these objects.

\subsection{Manipulating Axes}
\label{sec:manipulate_axes}

\revision{The foundation of axis-level manipulation is changing the axis-related constraints, i.e., the gravity, support, and fixed constraints that correspond to the axis.}
The manipulation of the axis includes zooming in on and dragging ticks, which results in the rescaling and reordering of the axis. 
\revision{There are two axes: axes with continuous (quantitative and temporal) attributes and axes with discrete (categorical) attributes.
We determine the type of axis according to the tick labels on the axis.
An axis is recognized as continuous if the texts can be converted to numbers or time stamps.}
Axis-level manipulation changes the scale or order of the axis.
\revision{For continuous axes, users can zoom in (either by scrolling or pinching) on axes or drag ticks to rescale or reorder the axis.
For discrete axes, users can drag the ticks to change the tick order or rescale by zooming (e.g., changing the widths of bars).}
As the scale or order changes, axes-related constraints are changed accordingly.

\textbf{Rescaling axes.}
As shown in \autoref{fig:manipulate_classification} (d), users can zoom in on a continuous axis to change the scale of the axis.
In our method, users can zoom in on a continuous axis in three ways, namely, by pinching, scrolling, and dragging a tick label.
The scale of the axis changes according to the zooming rate.
According to the changed scale, our model updates the gravity, support, and fixed constraints of the control points in the axis direction.
\removed{when the scale of an axis changes, we reset the gravity constraints and support constraints of the axis direction to the new position and reset the distance of fixed collisions accordingly.}


\textbf{Reordering axes.}
If the axis is discrete (e.g., categorical), users can reorder the axis directly by dragging the ticks, as shown in \autoref{fig:manipulate_classification} (e).
After dragging the ticks, the order of the ticks is recalculated according to the ticks' new positions.
The gravity constraints of a specific tick on the axis are changed accordingly.
Moreover, as dragging each tick to sort the ticks is time-consuming, we also implement a sort function; by right-clicking the axis, the ticks can be sorted according to selected visual objects' attributes (e.g., width, height, left, right, or color).
For example, we can sort the bars of a bar chart according to the height of the bars.


\begin{figure}[htb]
    \centering
    \setlength{\belowcaptionskip}{-10px}
    \includegraphics[width=\columnwidth]{image/case_force_set.pdf}
    \caption{
        Manipulating constraints and setting new constraints for visual objects results in diverse new layouts of visual objects.
        \revision{(a) A bubble chart.
        (c) Constructing an aggregated view by setting support and gravity constraints.}
    }
    \label{fig:force_handle}
\end{figure}


\subsection{Manipulating Constraints}
\label{sec:manipulate_cons}

The manipulation of constraints includes two parts: changing the existing constraints and setting new constraints.
A constraint layer shows the existing constraints of the selected visual objects.

\textbf{Changing constraints.}
The selected visual objects' constraints can be modified directly through dragging.
Our interface presents the top-$k$ gravity and support constraints with the most control points.
Each constraint has a handle for the users to manipulate directly.
When the constraints are dragged, related visual objects' constraints are changed.
\revision{For example, we can transform a stacked area chart into a ThemeRiver~\cite{havre2002themeriver} by dragging all visual objects' gravity constraints to the center, as illustrated in \autoref{fig:manipulate_classification} (f).}
In \autoref{fig:manipulate_classification} (g), the bubbles can be squeezed by changing the support constraints.

\textbf{Setting new constraints.} 
Except for manipulating an existing constraint, users can set new constraints.
For example, \autoref{fig:manipulate_classification} (g) and (f) can be seen as setting new constraints.
Moreover, setting constraints on visual objects can construct diverse visualization layouts.
For example, as shown in~\autoref{fig:force_handle}, the bubble chart is reshaped to a bar chart by setting new support and gravity constraints.

\textbf{Setting groups of constraints.} 
Users can set different gravity or support constraints for different values of a visual attribute when they want to rearrange visual objects according to their attributes.
However, it is time-consuming for users to set the constraints one by one.
We allow users to set a group of different constraints according to a certain attribute (e.g., height, width, color).
The result is that different visual objects with different attributes are separated.
As shown in \autoref{fig:manipulate_classification} (f), we can set different x-gravity constraints for different colors of bubbles.











\subsection{High-Level Interactions}

In subections~\ref{sec:manipulate_object}, \ref{sec:manipulate_axes}, and \ref{sec:manipulate_cons}, we describe low-level manipulations in our model.
These manipulations can compose high-level interactions.
Users can perform various interactions using these manipulations.
We list some interactions and discuss how they are composed by these manipulations.

\textbf{Navigating} changes the users' viewpoints.
Navigating effectively narrows the field of view to allow users to observe details, e.g., in a dense scatterplot or a multi-line chart.
Visualization with continuous axes allows navigation interactions.
In our model, the navigation is performed by rescaling the continuous axes. 


\textbf{Filtering} reduces the number of visual objects. 
\revision{Our method supports filtering by selecting focused visual objects and dragging them to a new canvas or by selecting unfocused visual objects and deleting them.}
Filtering is a generic interaction for common visualizations such as, for example, bar charts, area charts, line charts, and scatterplots.

\textbf{Rearranging} changes the spatial organization of visual objects; it includes reordering, realignment, etc. 
Rearranging is supported by our model on three levels: there is object-level, axis-level, and constraint-level rearranging.
At the object level, users can drag visual objects to reorder, align, and stack them.
For example, users can reorder the categories of a ThemeRiver graph. 
At the axis level, manipulating discrete axes means reordering or sorting the axes.
At the constraint level, flexible constraint settings create a large space for rearrangement.
For example, a user can set support constraints for visual objects to align them.

\textbf{Re-encoding} changes the encoding of the visual objects.
\revision{At the object level, users can transform a grouped bar chart into a stacked bar chart or a stacked area chart into an overlapping area chart. 
At the constraint level, users can set groups of gravity constraints for visual objects according to their size or color, and they can re-encode their positions, as \autoref{fig:manipulate_classification} (h) shows.}

\textbf{Aggregating} changes the granularity of the visual objects by gathering visual objects of the same type.
Users can define different gravity and support constraints at the constraint level for various sets of visual objects, such as those with different colors. 
There is ample flexibility in setting constraints to aggregate certain visual marks. For example, as illustrated in \autoref{fig:force_handle}, by setting collision, support, and gravity constraints, a bubble chart can be transformed into an aggregated bar chart.

\subsection{Direct Manipulation Interface}
\label{section:interface}

The interface allows users to manipulate visual objects, axes, and constraints.
As shown in \autoref{fig:interface}, a visualization is loaded onto a canvas.
Users can directly manipulate the visual objects and axes.
In the top-left corner, four icons represent deleting the canvas, copying the canvas, resetting the canvas to its initial state, and showing the constraints layer.
Clicking the constraints button leads to a new layer showing constraints and buttons that can be used to set new constraints, as shown at the bottom of \autoref{fig:interface}.
The constraints of the selected visual object are shown and can be directly manipulated.
New constraints can be set for the selected visual objects, including support (upward, downward, left, rightward), gravity (vertical and horizontal), and collision constraints. 

\begin{figure}[!ht]
    \centering
    \setlength{\belowcaptionskip}{-10px}
    \includegraphics[width=1.04\columnwidth]{image/interface.pdf}
    \caption{
        The interface of our system.
        Top: direct interactions with visual objects or axis.
        Bottom: direct setting constraints for visual objects directly.
    }
    \label{fig:interface}
\end{figure}
  

