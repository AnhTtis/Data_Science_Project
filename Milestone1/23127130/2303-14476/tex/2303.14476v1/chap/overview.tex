


\section{Method Overview}
\label{section:overview}


User interactions trigger changes in visual representations, with spatial changes being the most important, as noted by Mackinlay~\cite{mackinlay1986APT}.
Many representation changes resulting from interactions can be formulated as the spatial changes of visual objects.
According to Brehmer and Munzner\cite{Brehmer2013TaskAbstract}, interaction techniques for existing visual objects are referred to as manipulation, which includes selecting, navigating, arranging, changing, filtering, and aggregating visual objects.
Most of these manipulations can be formulated as spatial changes (position and shape) of visual objects.

Certain manipulations focus on changing the organization of the current visual objects, such as reordering and alignment.
Changes to the encoding can also be seen as reorganizing visual objects; for instance, it may involve transforming a grouped bar chart into a stacked bar chart.
Navigating, which changes viewpoints by panning and zooming, can be viewed as spatial movement according to the scale change of the axes.
Other interactions aim to reduce the information content, for example, by filtering or deleting a subset of visual objects.
In contrast, some interactions aim to reveal additional information.








To support these user tasks, we propose a \method{} method that models the positions of visual objects with spatial constraints inspired by physical forces.
Direct manipulation of visual objects allows the constraints to drive them into a new stable state, resulting in fluid transitions during interaction.
The spatial changes support users' tasks, such as comparison and identification.

As depicted in Figure~\ref{fig:pipeline}, spatial constraints are constructed for the visualization, and users can manipulate visual objects, axes, and constraints through intuitive actions. 
The manipulated control points and constraints' positions are inputs for the optimization process that uses encoded forces to change the positions of visual objects and reach a new stable layout.
The concept of spatial constraints and the four types of atomic constraints are discussed in Section~\ref{section:constraints}, while Section~\ref{section:manipulation} covers direct interactions with the visualization supported by spatial constraints.

























































    






