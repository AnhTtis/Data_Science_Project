\section{Introduction}



















Interaction plays a critical role as a dialogue between users and visualizations, providing alternative representations and viewpoints during data exploration. Despite its importance, many visualizations in the real world are static or have limited interaction capabilities. This issue arises from both the creators' and users' perspectives. On the creators' side, traditional paper-based habits and the need for expertise in visualization programming with tools such as D3~\cite{bostock2011d3} and Vega~\cite{satyanarayan2015reactive} contribute to the lack of interactivity in visualizations. On the users' side, complex interaction requirements beyond pre-defined options can also hinder the implementation of interactive visualizations.

Interactions trigger changes in visual representations.
Spatial changes are the most common changes, as spatial channels are the most effective channels~\cite{mackinlay1986APT}.
Many interactions can be regarded as spatial changes; for example, re-encoding a grouped bar chart to transform it into a stacked bar chart or changing the stacking order of a ThemeRiver~\cite{havre2002themeriver} are spatial changes.
Considering that many visualizations are static and could benefit from interactions, we propose a \method{} technique to activate static visualizations by adding spatial-based interactions to the original visualizations.
We treat visual objects (points, bars, areas, etc.) as objects that are constrained by physical rules.
The spatial channels of these visual objects are constrained by a set of spatial constraints, including gravity, support, collision, and fixed constraints.
These constraints support the construction of complex interactions and their triggered update processes.
Unlike traditional toolkits (e.g., D3~\cite{bostock2011d3}) that treat the position and shape of a visual object as the calculation results of pre-programmed functions, we provide a physical viewpoint for modeling the spatial channel of visual objects and, more importantly, the spatial updating triggered by interactions.
From our viewpoint, the stable positioning of visual objects results from an equilibrium of the spatial constraints; this is inspired by the idea that a physical object's position is due to an equilibrium of forces.



Following the spatial-constraint model, we propose an interactive system that adds various types of manipulations to static visualizations. Users can drag visual marks to change their stacked order or direction, zoom out to create a new canvas, zoom in on or drag ticks to rescale or reorder axes, and set new constraints or change existing ones. These manipulations can facilitate various intelligent interactions involved in users' analytical tasks, such as navigating, filtering, rearranging, re-encoding, and aggregating.
Our system operates by applying the spatial constraints, which drive the visual objects to gradually move towards converging positions through force-directed optimization, after the user manipulates the visualization.
Our model can be applied to various common visualizations, including bar charts, area charts, bubble charts, and line charts. We demonstrate the effectiveness of our method in supporting various interaction tasks for common visualizations through real-world cases and a user study.






The contributions of this work include the following: 
\begin{itemize}
    \item \revision{A \method{} model} for interactive visual objects that supports an intuitive transition when a user interacts with a visualization.
    \item A prototype system based on the model that allows users to manipulate a static visualization directly.
\end{itemize}














































 









 











 





































































