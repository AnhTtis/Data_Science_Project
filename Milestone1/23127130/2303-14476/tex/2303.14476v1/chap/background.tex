

\section{Related Work}
\label{section:background}
We contribute a \method{} method for modeling visual objects; this method is related to the direct manipulation of visual objects, interaction modeling for visualization, and force-directed layouts.




\begin{figure*}[!ht]
    \centering
    \includegraphics[width=\textwidth]{image/pipeline.pdf}
    \caption{A chart can be constructed using four atomic constraints, namely fixed, collision, support, and gravity constraints.
    These constraints facilitate direct manipulation of the visualization, including (a) and (b) manipulation of visual objects, (c) manipulation of axes, and (d) manipulation of constraints.
    Spatial constraints are expressed as forces that drive the positional changes of visual objects, similar to how forces drive the positional changes of physical objects.
    The eventual converged state enables various interactive tasks such as rearrangement, deletion, navigation, etc.
   }
    \label{fig:pipeline}
\end{figure*}



\subsection{Direct Manipulation for Visual Objects}



Direct manipulation provides an interface with a continuous representation, physical actions, and continuous feedback~\cite{directManipulation}, and it is more intuitive than traditional interactions, as it utilizes physical concepts in the interface.
Some approaches employ force metaphors in visualization; for example, the magnet metaphor~\cite{soo2005dust} and attractive and repulsive forces~\cite{rzeszotarski2014kinetica} are used to help users interact with multivariate data items.
However, these approaches only focus on unit visualization; they treat each point as a data item.
Other works employ force metaphors to support visualization tasks.
For example, Tominski et al.~\cite{tominski2012interaction} utilized a force metaphor to support the folding interaction for comparisons.
Most previous approaches employed physical forces in visualization, where each data item is encoded as a point (e.g., a graph layout or a unit visualization).
Compared with these approaches, our approach is more generic; it applies constraints to generic visual objects (e.g., a bar and an area).





\subsection{Interaction Modeling for Visualization}


\removed{In recent years, some approaches modeled visualization interactions through programming way.}
Visualization authoring tools, including D3~\cite{bostock2011d3}, Vega-Lite~\cite{satyanarayan2016vegalite}, and ECharts~\cite{li2018echarts}, provide interaction functions that allow users to trigger the updates of visual objects according to actions.
Nebula~\cite{chen2021nebula} provides a grammar for modeling the multi-view interactions.
Some other methods~\cite{choi2015visdock, harper2014deconstructing} aim to enhance an existing visualization with interactions.
VisDock~\cite{choi2015visdock} allows users to add interactions (e.g., selecting, filtering, navigation, etc.) to existing visualizations with codes.
Based on the features of D3 specifications of DOM elements, Harper and Agrawala\cite{harper2014deconstructing} deconstructed existing D3 visualizations by matching the given data with visual attributes.
Furthermore, Harper and Agrawala~\cite{harper2018converting} extracted D3 visualizations and converted them into templates to enable them to be reused. 
\revision{These methods require users to know the authoring tool and codes of an existing visualization.
Meanwhile, our approach considers visual objects as physical objects with spatial constraints; it is independent of the authoring tool being used.}

Interaction+~\cite{lu2017interaction} is also independent of the authoring tools; it parses the attributes of visual marks and applies extra interaction add-ons to the visualization.
However, Interaction+ focuses on non-spatial attributes like color and opacity, which can not be used to update the spatial positions of the visual objects.














\subsection{Force-Directed Layout}

Force-directed algorithms have been widely used in graph layout visualizations~\cite{forcedirected}.
The spring-electrical model that combines attractive and repulsive forces was first introduced by Eades~\cite{eades1984heuristic}.
Based on Eades's approach~\cite{eades1984heuristic}, Fruchterman and Reingold~\cite{fruchterman1991graph} proposed the spring-embedder model.
Furthermore, Kamada and Kawai~\cite{kamada1989algorithm} treat the force-directed layout problem as an optimization problem.
The force-directed graph layout problem can be converted into the constrained energy minimization problem~\cite{dwyer2005dig, gansner2004graph, zheng2018graph}.
Some approaches~\cite{hu2005efficient, hachul2004drawing} introduce multi-level methods to lower the computational complexity; these methods provide high-quality and efficient results when large graphs are drawn.
Moreover, users may have customized constraints for the graph nodes, such as fixing a node's position or keeping the distance between two nodes constant.
Many approaches~\cite{kamps1995constraint, dwyer2006ipsep} have been developed to handle such customized constraints.

Except for calculating the graph layout, the constraint concept has been widely used for interactive interaction; for example, it can be used to create a graph interface~\cite{garnet} and specify interactive objects~\cite{carr1994specification}.
In our approach, spatial constraints are used to drive visual objects to equilibrium states.




































 




















































