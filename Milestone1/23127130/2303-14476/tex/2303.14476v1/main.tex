\documentclass[manuscript, review=false, screen=true]{acmart}

\AtBeginDocument{%
  \providecommand\BibTeX{{%
    Bib\TeX}}}

\setcopyright{acmcopyright}
\copyrightyear{2023}
\acmYear{2023}
\acmDOI{XXXXXXX.XXXXXXX}


\acmJournal{JACM}
\acmVolume{0}
\acmNumber{0}
\acmArticle{000}
\acmMonth{00}




% \RequirePackage[
%   datamodel=acmdatamodel,
%   style=acmnumeric,
% ]{biblatex}

% \addbibresource{manuscript.bib}


\usepackage{algorithm}
\usepackage{algpseudocode}
\usepackage{wrapfig}



\usepackage{enumitem}
\setitemize{noitemsep,topsep=1pt,parsep=0pt,partopsep=1pt}

\makeatletter
\algrenewcommand\ALG@beginalgorithmic{\footnotesize}
\makeatother

\renewcommand{\algorithmicrequire}{ \textbf{Input:}} %
\renewcommand{\algorithmicensure}{ \textbf{Output:}} %



\newcommand{\tabincell}[2]{\begin{tabular}{@{}#1@{}}#2\end{tabular}}


\usepackage[normalem]{ulem}

\newcommand{\hint}[1]{\textcolor{red}{#1}}

\newcommand{\revision}[1]{\leavevmode{\color{blue}{#1}}}
\newcommand{\remark}[1]{\textcolor{yellow}{#1}}
\newcommand{\removed}[1]{\leavevmode{\color{red}{\sout{#1}}}}

\def \cleanversion{} %
\ifx\cleanversion\undefined
\else
 \renewcommand{\remark}[1]{} %
 \renewcommand{\removed}[1]{} 
 \renewcommand{\revision}[1]{#1}
\fi

\usepackage{calc}
\newlength\myheight
\newlength\mydepth
\settototalheight\myheight{Xygp}
\settodepth\mydepth{Xygp}
\setlength\fboxsep{0pt}

\newcommand*\inlinegraphics[1]{%
  \settototalheight\myheight{Xygp}%
  \settodepth\mydepth{Xygp}%
  \raisebox{-\mydepth}{\includegraphics[height=\myheight]{#1}}%
}



\def \method{spatial-constraint}


\begin{document}

\title{A Spatial-Constraint Model for Manipulating Static Visualizations}

\author{Can Liu}
\email{can.liu@pku.edu.cn}
\affiliation{%
  \streetaddress{Key Laboratory of Machine Perception (Ministry of Education), School of Intelligence Science and Technology}
  \institution{Peking University}
  \city{Beijing}
  \country{China}
  \postcode{100871}
}

\author{Yu Zhang}
\affiliation{%
  \institution{University of Oxford}
  \city{Oxford}
  \country{UK}}
\email{yu.zhang@cs.ox.ac.uk}

\author{Cong Wu}
\affiliation{%
\streetaddress{Key Laboratory of Machine Perception (Ministry of Education), School of Intelligence Science and Technology}
  \institution{Peking University}
  \city{Beijing}
  \country{China}
}

\author{Chen Li}
\affiliation{%
 \institution{Central Academy of Fine Arts}
 \city{Beijing}
 \country{China}}
 \email{chen.li.cafa@gmail.com}

\author{Xiaoru Yuan}
\affiliation{%
\streetaddress{Corresponding Author, Key Laboratory of Machine Perception (Ministry of Education), School of Intelligence Science and Technology, National Engineering Laboratory for Big Data Analysis and Application}
  \institution{Peking University}
  \state{Beijing}
  \country{China}}





\renewcommand{\shortauthors}{Liu et al.}

\begin{abstract}
    We propose a \method{} approach for modeling spatial-based interactions and enabling interactive visualizations, which involves the manipulation of visualizations through selection, filtering, navigation, arrangement, and aggregation.
    We proposes a system that activates static visualizations by adding intelligent interactions, which is achieved by associating static visual objects with forces.
    Our force-directed technique facilitates smooth animated transitions of the visualizations between different interaction states.
    We showcase the effectiveness of our technique through usage scenarios that involve activating visualizations in real-world settings.
\end{abstract}

\begin{CCSXML}
    <ccs2012>
    <concept>
    <concept_id>10003120.10003123</concept_id>
    <concept_desc>Human-centered computing~Interaction design</concept_desc>
    <concept_significance>500</concept_significance>
    </concept>
    <concept>
    <concept_id>10003120.10003145.10003146</concept_id>
    <concept_desc>Human-centered computing~Visualization techniques</concept_desc>
    <concept_significance>300</concept_significance>
    </concept>
    </ccs2012>
\end{CCSXML}
    
\ccsdesc[500]{Human-centered computing~Interaction design}
\ccsdesc[300]{Human-centered computing~Visualization techniques}

\keywords{intelligent interaction, constraint}

% \received{20 February 2007}
% \received[revised]{12 March 2009}
% \received[accepted]{5 June 2009}

\maketitle

\section{Introduction}
\label{sec:introduction}
% \begin{itemize}
%     % Diffusion of FL
%     \item {\st{Diffusion of FL}}
%     % Security threats to FL
%     \item {\st{Security threats to FL with particular focus on model poisoning}}
%     % Limitations of existing countermeasures
%     \item {\st{Current countermeasures (e.g., KRUM) and their limitations}}
%     % Proposed method and its advantages
%     \item {\st{Intuitive description of the proposed method and its difference (i.e., advantages) w.r.t. state of the art}}
%     % Main contributions
%     \item {\st{Summary of the main contributions of this work}}
%     % Paper's structure and organization
%     \item {\st{Paper's structure and organization}}
% \end{itemize}

% Diffusion of FL
Recently, {\em federated learning} (FL) has emerged as the leading paradigm for training distributed, large-scale, and privacy-preserving machine learning (ML) systems~\cite{mcmahan2017googleai,mcmahan2017aistats}. 
The core idea of FL is to allow multiple edge clients to collaboratively train a shared, global model without disclosing their local private training data.
%Specifically, an FL system consists of a central server and many edge clients; 
A typical FL round involves the following steps: {\em(i)} the server randomly picks some clients and sends them the current, global model; {\em(ii)} each selected client locally trains its model with its own private data; then, it sends the resulting local model to the server;\footnote{Whenever we refer to global/local model, we mean global/local model {\em parameters}.} {\em(iii)} the server updates the global model by computing an \emph{aggregation function}, usually the average (FedAvg), on the local models received from clients.
% \begin{enumerate}
%     \item[{\em(i)}] the server sends the current, global model to the clients and appoints some of them for training;
%     \item[{\em(ii)}] each selected client locally trains its copy of the global model with its own private data; then, it sends the resulting local model back to the server;\footnote{Whenever we refer to global/local model, we mean global/local model {\em parameters}.}
%     \item[{\em(iii)}] the server updates the global model by computing an \emph{aggregation function} on the local models received from clients (by default, the average, also referred to as FedAvg~\cite{mcmahan2017aistats}).
% \end{enumerate}
This process goes on until the global model converges. %(e.g., after a certain number of rounds or other similar stopping criteria).
%\\
% The advantages of FL over the traditional, centralized learning paradigm are undoubtedly clear in terms of flexibility/scalability (clients can join/disconnect from the FL network dynamically), network communications (only model weights\footnote{We will use \textit{parameters} and \textit{weights} interchangeably.} are exchanged between clients and server), and privacy (each client's private training data is kept local at the client's end and not uploaded to the server).
\\
% Security threats to FL
%However, the growing adoption of FL also raises security concerns~\cite{costa2022covert}, particularly about its confidentiality, integrity, and availability.
Although its advantages over standard ML, FL also raises security concerns~\cite{costa2022covert}. %, particularly about its confidentiality, integrity, and availability~\cite{costa2022covert}.
% OLD, LONG VERSION
% Indeed, some work deals with privacy leakage that may expose the local data of some clients~\cite{melis2019sp}. 
% A large body of work, instead, investigates attacks that usually aim to detriment the predictive accuracy of the learned global model. For instance, \emph{data poisoning} attacks achieve this goal by letting an adversary pollute the training set of some corrupt FL clients with maliciously crafted examples~\cite{jagielski2018sp}.
% Similarly, in \emph{model poisoning} the attacker attempts to tweak the global model weights~\cite{bhagoji2019pmlr} by directly perturbing the local model's weights of some infected FL clients before these are sent to the central server for aggregation, usually via so-called Byzantine attacks. 
% It turns out that Byzantine model poisoning attacks severely impact standard FedAvg; therefore, more robust aggregation functions must be designed to make FL systems secure.
Here, we focus on \emph{untargeted model poisoning} attacks~\cite{bhagoji2019pmlr}, where an adversary attempts to tweak the global model weights %\footnote{We will use the terms \textit{parameters} and \textit{weights} interchangeably.} 
by directly perturbing the local model's parameters of some infected clients before these are sent to the central server for aggregation.
In doing so, the adversary aims to jeopardize the global model \textit{indiscriminately} at inference time.
Such model poisoning attacks severely impact standard FedAvg; therefore, more robust aggregation functions must be designed to secure FL systems.
\\
% In this paper, we focus on designing a novel robust aggregation scheme at the server's end to contrast the effect of Byzantine model poisoning attacks.
%
% Current countermeasures and their limitations
%Several countermeasures have been proposed in the literature to combat model poisoning attacks on FL systems.
% Some methods use simple statistics more robust than plain average to smooth the impact of malicious updates (e.g., Trimmed Mean and FedMedian~\cite{yin2018icml}). 
% Other defenses implement outlier detection techniques to discard malicious updates from the aggregation performed at the server's end. Those are either based on heuristics (e.g., Krum/Multi-Krum~\cite{blanchard2017nips} and Bulyan~\cite{mhamdi2018pmlr}) or data-driven approaches (e.g., K-means clustering~\cite{shen2016acm} or DnC via spectral analysis~\cite{shejwalkar2021ndss}). 
% Finally, some strategies rely on a centralized ``source of trust'' to spot potential malicious updates (e.g., FLTrust~\cite{cao2020fltrust}).
% Several countermeasures have been proposed in the literature to combat model poisoning attacks on FL systems, i.e., to discard possible malicious local updates from the aggregation performed at the server's end. 
% These techniques range from simple statistics more robust than plain average (e.g., Trimmed Mean and FedMedian~\cite{yin2018icml}) to outlier detection heuristics (e.g., Krum/Multi-Krum~\cite{blanchard2017nips} and Bulyan~\cite{mhamdi2018pmlr}) or data-driven approaches (e.g., spectral analysis via K-means clustering~\cite{shen2016acm} or spectral analysis), or methods based on ``source of trust'' (e.g., FLTrust~\cite{cao2020fltrust}).
% OLD, LONG VERSION
%Several countermeasures have been proposed in the literature to combat Byzantine model poisoning attacks on FL systems.
% Descriptive statistics
% For example, Trimmed Mean and FedMedian aggregate local model updates using more robust statistics than standard average~\cite{yin2018icml}.
%
% % Heuristics for outlier detection
% Many existing Byzantine-resilient strategies implement some outlier detection heuristics to discard the model updates sent by potentially malicious clients from the input of the aggregation function.
% One of the most popular heuristics is Krum~\cite{blanchard2017nips}.
% This strategy tries to mitigate the impact of Byzantine attacks by selecting as a global model the local model with the smallest sum of Euclidean distances to {\em all} the other local models.
% Although powerful, Krum requires the server to know (or, at least, estimate) the number of malicious FL clients upfront, which is generally impossible in a realistic attack scenario. %
% Moreover, Krum may become ineffective for complex, high-dimensional model parameter spaces due to the curse of dimensionality.
% Bulyan~\cite{mhamdi2018pmlr} tries to overcome this issue by combining Krum with a variant of Trimmed Mean.
% % Data-driven outlier detection
% Other strategies use data-driven outlier detection techniques -- e.g., via K-means clustering~\cite{shen2016acm} -- to spot potential malicious local model updates. 
% %For instance, Shen et al. propose to cluster local model updates with K-means and thus identify outliers.
%
% % Other techniques
% As far as the server is concerned, any local model received can be from a potential malicious client. 
% FLTrust~\cite{cao2020fltrust} assumes the server acts as a client, i.e., trains a local model on an additional {\em trustworthy} dataset at the server's end and compares it against all the local models from other clients. 
% This way, the server can rely on some ``source of trust'' when discarding potentially malicious clients.
%\\
% Limitations of existing Byzantine-resilient strategies
Unfortunately, existing defense mechanisms either rely on simple heuristics (e.g., Trimmed Mean and FedMedian by~\cite{yin2018icml}) or need strong and unrealistic assumptions to work effectively (e.g., foreknowledge or estimation of the number of malicious clients in the FL system, as for Krum/Multi-Krum~\cite{blanchard2017nips} and Bulyan~\cite{mhamdi2018pmlr}, which, however, cannot exceed a fixed threshold).
Furthermore, outlier detection methods using K-means clustering~\cite{shen2016acm} or spectral analysis like DnC~\cite{shejwalkar2021ndss} do not directly consider the temporal evolution of local model updates received.
Finally, strategies like FLTrust~\cite{cao2020fltrust} require the server to collect its own dataset and act as a proper client, thereby altering the standard FL protocol.
\\
% OLD, LONG VERSION
% Overall, existing Byzantine-resilient strategies are either simple heuristics (e.g., FedMedian) or, if they are more complex, they rely on strong and unrealistic assumptions to work effectively (e.g., knowing the number of malicious clients in the FL system in advance, as for Krum and alike).
% Furthermore, data-driven outlier detection methods do not consider the temporary evolution of local model updates received (e.g., K-means clustering). 
% Finally, strategies like FLTrust requires the server to collect its own dataset and act as a proper client, thereby altering the standard FL protocol.
%
% Description of the proposed method
This work introduces a novel pre-aggregation \textit{filter} robust to untargeted model poisoning attacks. Notably, this filter $(i)$ operates without requiring prior knowledge or constraints on the number of malicious clients and $(ii)$ inherently integrates temporal dependencies. 
The FL server can employ this filter as a preprocessing step before applying \textit{any} aggregation function, be it standard like FedAvg or robust like Krum or Bulyan.
Specifically, we formulate the problem of identifying corrupted updates as a multidimensional (i.e., matrix-valued) time series anomaly detection task. 
The key idea is that legitimate local updates, resulting from well-calibrated iterative procedures like stochastic gradient descent (SGD) with an appropriate learning rate, show \textit{higher predictability} compared to malicious updates. This hypothesis stems from the fact that the sequence of gradients (thus, model parameters) observed during legitimate training exhibit regular patterns, as validated in Section~\ref{subsec:intuition}. %until convergence. 
%This regularity may be more pronounced for smooth convex loss functions, but it can still be captured within an appropriate time window, even for more complex and convoluted loss surfaces. 
%We provide evidence of this claim in Appendix~B, where we show that the average mutual information (i.e., ``predictability''), calculated over pairs of legitimate model updates sent at different FL rounds, is significantly higher than the corresponding computation for a malicious client.
\\
Inspired by the matrix autoregressive (MAR) framework for multidimensional time series forecasting~\cite{chen2021je}, we propose the FLANDERS ({\em \textbf{F}ederated \textbf{L}earning meets \textbf{AN}omaly \textbf{DE}tection for a \textbf{R}obust and \textbf{S}ecure}) filter.
The main advantages of FLANDERS over existing strategies like FLDetector~\cite{zhao2020multivariate} are its resilience to large-scale attacks, where $50\%$ or more FL participants are hostile, and the capability of working under realistic non-iid scenarios.
We attribute such a capability to two key factors: $(i)$ FLANDERS works without knowing a priori the ratio of corrupted clients, and $(ii)$ it embodies temporal dependencies between intra- and inter-client updates, quickly recognizing local model drifts caused by evil players. Below, we summarize our main contributions:

\begin{itemize}
\item[{\em(i)}]
We provide empirical evidence that the sequence of models sent by legitimate clients is more predictable than those of malicious participants performing untargeted model poisoning attacks.
\\
\item[{\em(ii)}] 
We introduce FLANDERS, the first pre-aggregation filter for FL robust to untargeted model poisoning based on multidimensional time series anomaly detection.
\\
\item[{\em(iii)}] 
We integrate FLANDERS into Flower,\footnote{\scriptsize{\url{https://flower.dev/}}} a popular FL simulation framework for reproducibility.
\\
\item[{\em(iv)}] 
We show that FLANDERS improves the robustness of the existing aggregation methods under multiple settings: different datasets, client's data distribution (non-iid), models, and attack scenarios.
\\
\item[{\em(v)}] 
We publicly release all the implementation code of FLANDERS along with our experiments.\footnote{\scriptsize{\url{https://anonymous.4open.science/r/flanders_exp-7EEB}}}
\end{itemize}

% Paper's structure and organization
The remainder of the paper is structured as follows. %some related work and the current state-of-the-art solutions to security issues that FL entails. 
Section~\ref{sec:background} covers background and preliminaries. 
In Section~\ref{sec:related}, we discuss related work.
Section~\ref{sec:problem} and Section~\ref{sec:method} describe the problem formulation and the method proposed. % to tackle it. 
Section~\ref{sec:experiments} gathers experimental results. %, and Section~\ref{sec:limitations} discusses some limitations of this work.
Finally, we conclude in Section~\ref{sec:conclusion}.
 %discusses the limitations of this work and draws future research directions.
%reports conclusions and draws perspectives for future research directions.

%%%%%%% OLD %%%%%%%
%to overcome the resilience of Byzantine failures in distributed Stochastic Gradient Descent computations. 
% The strength of Krum is its time complexity, which is linear in the gradient dimension. 
% However, the robustness of the approach is guaranteed for gradient-based learning applications only when the majority of the clients are not compromised. 
% Besides, the aggregation mechanism of Krum, as well as that of similar methods, is robust from a coarse-grained perspective and does not provide solutions to errors and perturbations that may occur at inference time.
%A related approach to~\cite{blanchard2017nips} is the work of Su et al.~\cite{su2016dc}. Here, the authors propose an iterated approximate agreement to tackle a multi-layer scenario attacked by Byzantine agents. 
%However, the method works efficiently on the sole discrete context and it is inapplicable to continuous state environments.
%\gabri{Maybe, we should just talk about the main limitations of existing countermeasures without digging into their details (or, we can just mention Krum as this is the most popular one). I will move the description of all these methods to the Related Work section.}
\section{Background on Network Calculus}
\label{sec: background}


\begin{figure*}[tbh]
\centering
\begin{subfigure}[b]{0.3\textwidth}
    \centering
    \includegraphics[width=\linewidth]{images/in-out.png}
    \caption{Arrival and departure data and their relation with delay $d(t)$ and backlog $b(t)$. For a FIFO system, the delay is the horizontal distance between $R(t)$ and $R^*(t)$ but some other multiplexing techniques may shift the data to a later priority, causing a longer delay.}
    \label{fig: data in-out}
\end{subfigure}
\hfill
\begin{subfigure}[b]{0.35\textwidth}
    \centering
    \includegraphics[width=\linewidth]{images/arrival-service.png}
    \caption{Characteristics of an arrival curve and a service curve. From any point of observation, the arriving data never exceeds its arrival curve; the departure data is also never less than the service curve with respect to the data arrival.}
    \label{fig: arrival-service curves}
\end{subfigure}
\hfill
\begin{subfigure}[b]{0.33\textwidth}
    \centering
    \includegraphics[width=\linewidth]{images/bound.png}
    \caption{Delay and backlog bounds of a system. Backlog is the maximum vertical distance between $\alpha(t)$ and $\beta(t)$; FIFO delay is their maximum horizontal distance; but for arbitrary multiplexing, the delay guarantee is when the system clears its buffer, thus it's the intersection of $\alpha(t)$ and $\beta(t)$.}
    \label{fig: system bounds}
\end{subfigure}
\caption{Network calculus framework. We let $R(t)$ and $R^*(t)$ be the arrival and departure data flow of a system; $\alpha(t)$ be the piecewise linear concave arrival curve and $\beta(t)$ be the piecewise linear convex service curve of a system.}
% \hossein{Better to show piece-wise linear concave arrival curve and piece-wise linear convex service curve instead of token-bucket and rate-latency.}}
\end{figure*}

We recall some of the network calculus essentials for a better understanding of the framework used in Saihu. In the following context, we use the following notation: $\mbb{R}^+$ is the set of non-negative real numbers; $[x]_+$ denotes $\max(0, x)$

The data flow is by convention modeled as a left-continuous wide-sense increasing function $R(t): \mbb{R}^+ \mapsto \mbb{R}^+$ with respect to time $t$~\cite{ncbook2001leboudec}. 

A system $\mcal{S}$ receives arrival data described as a cumulative function $R(t)$ and delivers departure data as another cumulative function $R^*(t)$. Figure~\ref{fig: data in-out} illustrates such a system $\mcal{S}$. The benefit of representing a system like this is that we can observe system backlog and delay with such a model. 

\begin{definition}[Backlog and Delay~\cite{ncbook2001leboudec}]
    The backlog of a system at time~$t$ is
    \begin{equation}
        b(t) = R(t) - R^*(t)
    \end{equation}
    
    The virtual delay of a FIFO system at time $t$ is
    \begin{equation}
        d_{FIFO}(t) = \inf \lbp \tau \geq 0 : R(t) \leq R^*(t+\tau) \rbp
    \end{equation}
\end{definition}



The backlog of a system can be viewed as the vertical distance between $R$ and $R^*$. The FIFO (\textit{First-in First-out}) delay is the horizontal distance between $R$ and $R^*$. One may obtain other delay values if the multiplexing technique is not FIFO.

% \begin{figure}
%     \centering
%     \includegraphics[width=0.9\linewidth]{images/in-out.png}
%     \caption{In/out data flow; delay and backlog}
%     \label{fig: data in-out}
% \end{figure}

Since we are interested in the system guarantee instead of a single instance of data flow, we would like to have general bounds to the arrival and departure data flows. Therefore, we define \textit{arrival curve} and \textit{service curve} as the bounds of arrival and departure data flows.

\begin{definition}[Arrival Curve~\cite{ncbook2001leboudec}]
    Given a wide-sense increasing function $\alpha: \mbb{R}^+ \mapsto \mbb{R}^+$, we say that a flow $R(t)$ is $\alpha$-constrained if and only if for all $s \leq t$:
    \begin{equation}
        R(t) - R(s) \leq \alpha(t-s)
    \end{equation}
    We say $R(t)$ has $\alpha$ as an arrival curve.
\end{definition}

\begin{definition}[Service Curve~\cite{ncbook2001leboudec}]
    Given a wide-sense increasing function $\beta: \mbb{R}^+ \mapsto \mbb{R}^+$ and $\beta(0) = 0$. A system $\mcal{S}$ having $R(t)$ and $R^*(t)$ as its arrival and departure flows. We say $\mcal{S}$ offers a service curve $\beta$ if and only if
    \begin{equation}
        R^*(t) \geq (R \otimes \beta)(t) =: \inf_{s \leq t} \lbp R(s) + \beta(t-s) \rbp
    \end{equation}
    where $\otimes$ denotes the min-plus convolution
\end{definition}

Figure~\ref{fig: arrival-service curves} illustrates the arrival and service curves. Any segment of arrival flow $R(t)$ is constrained by arrival curve $\alpha$ and the output curve $R^*(t)$ is always no less than the curve $R\otimes\beta$. As a result, an arrival curve upper bounds the incoming traffic, and a service curve lower bounds the outgoing traffic.

% \begin{figure}
%     \centering
%     \includegraphics[width=\linewidth]{images/arrival-service.png}
%     \caption{Arrival/Service curve}
%     \label{fig: arrival-service curves}
% \end{figure}

We consider 2 special types of curves throughout this paper, \textit{token-bucket} (or sometimes called \textit{leaky-bucket}) curve and \textit{rate-Latency} curve.

\begin{definition}[Token-bucket and Rate-latency~\cite{ncbook2001leboudec}]
    A token-bucket curve $\gamma_{r,b}$ with arrival rate $r$ and burst $b$ is defined as
    \begin{equation}
        \gamma_{r,b}(t) = b + rt
    \end{equation}

    A rate-latency curve $\beta_{R,T}$ with service rate $R$ and latency $T$ is defined as
    \begin{equation}
        \beta_{R,T}(t) = R \lb t - T \rb_+
    \end{equation}
\end{definition}

A token-bucket curve is determined by a burst $b$ and an arrival rate~$r$. Burst represents the maximum possible data volume that can arrive simultaneously, and arrival rate represents the maximum long-term data rate~\cite{bouillard2022tradeoff}.
A rate-latency curve is determined by a latency~$T$ and a service rate~$R$. Latency represents the time a server needs before starting to process the incoming data, and service rate represents the minimum rate to process data after the initial latency.

With the help of arrival and service curves, we can derive delay and backlog bounds for a system $\mcal{S}$ illustrated in Figure~\ref{fig: system bounds}. Suppose a system $\mcal{S}$ has arrival curve $\alpha$ and service curve~$\beta$, its worst-case backlog $b^*$ is the maximum vertical distance between~$\alpha$ and~$\beta$. Similarly, depending on the multiplexing technique applied to the system, its worst-case delay bound $d^*$ is the maximum horizontal distance between $\alpha$ and $\beta$ if $\mcal{S}$ is a FIFO system. If we don't have any information about its multiplexing technique, referred to as arbitrary multiplexing, the best we can say is that when $\alpha$ and $\beta$ intersect each other, where all data has been delivered out of the system. Consequently, the worst-case delay bound for arbitrary multiplexing is the time required for $\mcal{S}$ to clear its buffer.

% \begin{figure}
%     \centering
%     \includegraphics[width=\linewidth]{images/bound.png}
%     \caption{System delay/backlog bounds}
%     \label{fig: system bounds}
% \end{figure}

While a service curve captures the slowest possible output speed of a system, a link's transmission capacity limits the speed as well. Hence, we model this phenomenon using a \textit{greedy shaper} with a sub-additive function $\sigma: \mbb{R}^+ \mapsto \mbb{R}^+$ concatenated with a server. We consider a concatenation as shown in Figure \ref{fig: system}. By convention we assume $\sigma(0) = 0$ and $\beta(t) \leq \sigma(t), \forall t \in \mbb{R}^+$, meaning that the buffer is cleared at the beginning and the service never exceed its physical limitation. With the above definition, such greedy shaper conserves the service provided by the system due to theorem \ref{thm: shaping}.

\begin{figure}[thb]
    \centering
    \includegraphics[width=0.7\linewidth]{images/system.png}
    \caption{Shaping of departure data. A flow that has an arrival curve $\alpha$ feeds into a server with an arrival data flow $R(t)$. The server having service curve $\beta$ takes $R(t)$ and gives a departure data flow $R^*(t)$ to a shaper with shaping function $\sigma$. The shaper takes $R^*(t)$ and shape the data flow as another departure $D(t)$.}
    \label{fig: system}
\end{figure}


\begin{theorem}[Shaping conserves service \cite{ncbook2001leboudec}]
\label{thm: shaping}
Following the system shown in Figure \ref{fig: system}, we have
\begin{equation}
     D = R^* \otimes \sigma \geq \lp R \otimes \beta \rp \otimes \sigma = R \otimes \lp \beta \otimes \sigma \rp = R \otimes \beta
\end{equation}
\end{theorem}

In the following context, we model the shaping function $\sigma$ as a token-bucket curve $\gamma_{C,L}$ with transmission capacity $C$ and the packet size $L$ to capture the link capacity and packetization~\cite{bouillard2022tradeoff}.

\section{Applications}
\label{sec:apps}
To demonstrate the wide range of usagages of our model, we implement a series of applications:
\begin{enumerate}
	\item Incremental surface \& color reconstruction
	\item 3D saliency detection
	\item Open vocabulary scene understanding
	\item Surface infrared field
	\item 3D style transfer
\end{enumerate}
Originating from our motivation in inspection and service robotics, we implement 1) Incremental surface \& color reconstruction for visualization of robot surroundings.
For robot exploration, we implement 2) 3D saliency detection to indicate the salient regions in maps.
For recovering object-level semantic information in environments, we implement 3) open vocabulary scene understanding to yield the regions containing the objects..
Furthermore, to demonstrate the flexibility, we implement 4) surface infrared fields and 5) 3D style transfer for artistic purposes. 

In~\cref{fig:latent_diff}, we classify those 3 applications into 3 categories: (a) directly obtaining the properties from sensor observation, such as application 1) and 4). (b) processing on sensor data and predict properties, such as application 2), 5). (c) extending (b) to operating beyond latent features, such as application 3).
%Thus, in the following, we discuss about those categories of applications.
% we mainly describe the application 1) (\cref{sec:incremental_reconstruction}) and 3) (\cref{sec:openvoc}).

Application 1) and 4) are in the first one category. Thus, we mainly describe 1) incremental surface \& color reconstruction (\cref{sec:incremental_reconstruction}), while for 4) we can easily exchange color with infrared.
%
For the second with 2) and 5) in~\cref{sec:fabircated_prop}, we mainly describe the usage of fabricated properties.
As the mapping part is redundant to previous category, it will not be detailed.
%
The third category is the application 3) that maps a LIM for high dimensional latent fields.
We demonstrate that this application provides a flexible inference in \cref{sec:openvoc}.


%Afterwards, we evaluate application 1) and 3) in~\cref{sec:exp} and extensively show demonstration for all application in~\cref{sec:exp:extensive_app}.

\section{Spatial Constraints for Visualization}
\label{section:constraints}

We introduce a spatial constraint model to depict the positioning of visual objects within visualizations. We provide an overview of spatial constraints frequently employed in visualizations and establish definitions for \textbf{atomic constraints} to model spatial layouts. We illustrate the applicability of atomic constraints in representing typical visualizations through examples.



\subsection{Modeling Visualization Layouts with Forces}
\label{sec:vis_force_case}


This section introduces a conceptual framework that models visual objects with constraints, using several illustrative scenarios. As depicted in Figure \ref{fig:filter_circle} (a), each bar within the chart can be treated as an area-type object with four control points. When implementing the bar chart programmatically, the position and shape of each bar are determined using predefined algorithms. For instance, if the chart is constructed using a tool like D3~\cite{bostock2011d3}, the programmer assigns attributes (such as \texttt{x}, \texttt{y}, \texttt{width}, and \texttt{height}) to each \texttt{<rect>} element.
However, in situations where certain bars, such as the pink bars, are removed, the programmer must implement a function to readjust the positions of the remaining bars, aligning them with the $x$-axis. Achieving fluid transitions in such cases requires additional significant effort.

In our approach, bars can be conceptualized as physical objects stacked on a supporting surface (ground). All objects are influenced by gravity, naturally descending towards the ground. In this context, the $x$-axis serves as the supporting ground, facilitating intuitive and automated movement of visual objects. As depicted in Figure \ref{fig:filter_circle}, when the pink bars are removed, the blue bars naturally descend to rest upon the $x$-axis, aided by the ``hold-up'' force exerted by the $x$-axis. Consequently, the final spatial arrangement of visual objects results in a simplified bar chart.
Furthermore, beyond external forces impacting visual objects, there are internal forces preserving the shapes of these objects. For instance, the fixed distance between the bottom-left and top-left corner points is maintained both vertically and horizontally.






\begin{figure}[!htb]
    \centering
    \setlength{\belowcaptionskip}{10px}
    \includegraphics[width=.9\columnwidth]{image/filter_circle}
    \caption{The position of the existing visual objects will be updated after some visual objects are removed.}
    \label{fig:filter_circle}
\end{figure}


\autoref{fig:filter_circle} depicts a stacked area chart and a bubble chart. After the removal of the pink area in the stacked area chart, the blue area descends to the "ground". Within the stacked area chart, each point is influenced by multiple horizontal positions (i.e., ticks). Additionally, there are collision forces among the visual objects in the vertical direction to prevent the blue area from overlapping with the pink area. In the bubble chart, collision forces also exist among the bubbles. Simultaneously, all the points are drawn towards the center in the horizontal direction. In these examples, the control points are subject to three types of constraints:
\begin{itemize}
  \item \textbf{Environmental Constraints}: In the bar chart and area chart depicted in \autoref{fig:filter_circle}, each visual object is influenced by both gravity and "hold-up" forces in the vertical direction. These forces are referred to as the \textbf{gravity} and \textbf{support} constraints, respectively.
  
  \item \textbf{Inter-object Constraints}: Vertical collision forces come into play among visual objects, such as the stacked blue and pink bars in \autoref{fig:filter_circle}. These \textbf{collision} constraints ensure that the blue bars are stacked atop the pink bars, preventing overlap. Additionally, for point-like visual objects, collision forces maintain separation between circles.
  
  \item \textbf{Intra-object Constraints}: Visual objects also have \textbf{fixed} relationships among their internal points to preserve their shapes. For instance, fixed vertical distances between control points (e.g., between the top-left and bottom-left corners) maintain a constant bar height.
\end{itemize}
  











\subsection{Control Points}


Three types of visual objects exist: points (\inlinegraphics{image/point_visual_mark}), lines (\includegraphics{image/line_visual_mark}), and areas (\includegraphics{image/area_visual_mark}). Control points serve as the fundamental units for these visual objects. These objects manifest as individual control points, connected sequences of control points forming lines, and enclosed regions defined by control points.
In the context of SVG (Scalable Vector Graphics), control points for lines and areas can be extracted by parsing the endpoints of each segment from an SVG path. For instance, in a bar chart, each bar is defined by its four corner points, which function as control points. Similarly, in a circular chart, each circle is represented as a point with an associated radius.
In a stacked area chart, points encompass the areas of each visual object. Presently, our focus is on 2-D visualization, where a control point occupies a position defined by two dimensions. Formally, a control point is represented as $$P_i = \{(x, y), r\},$$ where $x$ and $y$ denote coordinates along the horizontal and vertical axes in a Cartesian coordinate system. The value of $r$ represents the point's radius.
For line-type and area-type visual objects, the radii of their control points are set to $0$, while a point-type object may possess a non-zero radius. This distinction in radii contributes to the nuanced representation of these various visual elements.



\subsection{Atomic Constraints}

As summarized in subsection~\ref{sec:vis_force_case}, our analysis reveals four fundamental atomic constraints: gravity, support, collision, and fixed constraints. The subsequent paragraphs provide the formal definitions for each of these constraints.

\begin{wrapfigure}{r}{0.16\columnwidth}
    \begin{center}
      \vspace{-0.09\columnwidth}
      \includegraphics[width=0.16\columnwidth]{image/formula_gravity}
      \vspace{-0.03\columnwidth}
    \end{center}
\end{wrapfigure}
\textbf{Gravity constraints} induce an attractive force towards a control point $P$ from a specific position, either horizontally or vertically. Formally, a gravity constraint for a control point $P$ in the $x$ direction is established by minimizing $$f_g\left(x - d\right),$$ where $f_g$ increases as the magnitude of $|x - d|$ increases.
To ensure the continuity and differentiability of gravity constraints for optimization purposes, we define $$f_g \left( t\right) = (t)^2,$$ which represents the simplest function that maintains continuity and differentiability. The accompanying diagram on the right provides an illustrative instance of gravitational attraction towards the control point in the $x$-direction.








\begin{wrapfigure}{r}{0.16\columnwidth}
    \begin{center}
      \vspace{-0.03\columnwidth}
      \includegraphics[width=0.16\columnwidth]{image/formula_constraint}
      \vspace{-0.03\columnwidth}
    \end{center}
\end{wrapfigure}
\textbf{Support constraints} exert controlled influence on the coordinates of a given control point $P$, either compelling them to exceed or remain below specific thresholds along the horizontal or vertical axes. 
An instance is enforcing a point's positioning above a designated threshold. 
This can be observed in \autoref{fig:filter_circle}, where support constraints are enforced on the bars from $x$-axis.
Mathematically, the support constraint for a control point $P$ is defined as
$$x \pm \xi = d,$$
where $\xi$ represents a non-negative slack variable. This constraint ensures that $\xi \geq 0$ in all cases. When the positive sign is selected for $\pm$, as in $$x + \xi = d,$$
control point $P$ is positioned to the left of the distance $d$.











\begin{wrapfigure}{r}{0.16\columnwidth}
  \begin{center}
    \vspace{-0.03\columnwidth}
    \includegraphics[width=0.16\columnwidth]{image/formula_collision_A}
    \vspace{-0.04\columnwidth}
  \end{center}
\end{wrapfigure}
\textbf{Collision constraints} pertain to the spatial relationship between control points. Two types of collision constraints exist: the collision of control points along the vertical or horizontal axes and the collision relationship among point-type visual objects. The former guarantees specific positions relative to each other, positioning control point $A$ to the left, right, above, or below control point $B$.
Mathematically, the constraint in the x-direction is expressed as
$$x_{b} - x_a - d \pm \xi = 0,$$
When the $\pm$ sign is negative, thus yielding
$$x_{b} - x_a - d - \xi = 0,$$
as illustrated in the right figure, Point $B$ is situated to the right of Point $A$, with a minimum distance of $d$ between them.

\begin{wrapfigure}{r}{0.16\columnwidth}
    \begin{center}
      \vspace{-0.06\columnwidth}
      \includegraphics[width=0.16\columnwidth]{image/formula_collision}
      \vspace{-0.08\columnwidth}
    \end{center}
\end{wrapfigure}
The collision constraints among point-type visual objects prevent overlapping of points within a 2D space.
This constraint guarantees that the distance between two points exceeds the sum of their radii. The collision constraint between two points, $P_a$ and $P_b$, can be expressed as
$$\ d\left(P_a, P_b\right) = \sqrt{(x_a - x_b)^2 + (y_a - y_b)^2} \geq r_a + r_b.$$


\begin{wrapfigure}{r}{0.16\columnwidth}
    \begin{center}
      \vspace{-0.04\columnwidth}
      \includegraphics[width=0.16\columnwidth]{image/formula_fixed}
      \vspace{-0.12\columnwidth}
    \end{center}
\end{wrapfigure}
\textbf{Fixed constraints} are related to the fixed distance of control points inside a visual object.
For example, the fixed-distance between corner points of a bar.
Formally, a fixed constraint for points $A$ and $B$ in the x-direction is presented as $$(x_a - x_b - d) = 0.$$


\subsection{Modeling Visualization Layouts with Atomic Constraints}

This subsection offers an introduction to the methodology of representing common visualizations using atomic constraints. \autoref{fig:chart_constraint} showcases a matrix that includes four common visualizations, each accompanied by its corresponding atomic constraints.


Within the conceptual framework, our spatial constraint model can be built upon control points, integrating a range of constraint types, thus facilitating versatile representations. Commencing with visualizations articulated in control point format within a two-dimensional Cartesian coordinate system, we employ a set of visualizations as case studies. These encompass both categorical and quantitative axes.

\begin{itemize}
\item \textbf{Gravity constraints:}
Coordinate axes can apply gravity constraints to their associated positions. For example, in a line chart or scatter plot, the positions of control points are defined by the data they represent along their respective coordinates. Gravity constraints are solely relevant to line charts and scatter plots, with no involvement of collision or fixed constraints. As the coordinate axes are rescaled, the visual objects within these charts experience scale adjustments accordingly.

\item \textbf{Fixed constraints:} Slightly more complex visualizations incorporate fixed constraints, as seen in bar charts and area charts.
Fixed constraints can align with data mappings or pre-established configurations. For instance, in the context of a bar chart, the height corresponds to data values, while the width represents pre-defined settings.

\item \textbf{Support constraints:}
Moreover, in the case of stacked visualizations along an axis, gravity and support forces can be utilized to guarantee the vertical alignment of content along the x-axis (assuming we are discussing visualizations stacked along the x-direction).

\item \textbf{Collision constraints:}
Interactions among visual objects lead to compression and collision effects. For instance, take a stacked area chart, where a collision relationship exists between the visual objects both above and below, requiring collision constraints.
In a grouped bar chart, collision constraints are relevant to the left and right visual objects within a group.



\end{itemize}



Expanding on these insights, we illustrate the process of constraint inference through various representative examples. Specifically, for the purpose of demonstration, we offer instances involving stacked bar charts, grouped bar charts, stacked area charts, and bubble charts. Following this, we broaden our analysis to encompass basic bar charts, area charts, line charts, and scatter plots.
All these instances are rooted in visualizations within a two-dimensional Cartesian coordinate system.






\begin{figure}[!ht]
  \centering
  \includegraphics[width=.8\columnwidth]{image/force}
  \caption{Constraints for exemplary visualizations. Each column corresponds to a visualization type, and each row represents a specific category of atomic constraints.}
  \label{fig:chart_constraint}
\end{figure}

\begin{itemize}
\item \textbf{Stacked bar chart}.
\autoref{fig:chart_constraint} (a) displays a stacked bar chart alongside its four types of atomic constraints.
Each bar adheres to fixed constraints along all four edges.
Vertical collision constraints exist among the stacked bars within a group, ensuring their non-overlapping alignment.
Simultaneously, all bars experience a horizontal gravitational force that pulls them towards the center position of the tick marks (such as ``Tick A'' for the left bars).
Moreover, all control points are subjected to gravitational and supportive constraints originating from the $x$-axis, designated as the \textbf{baseline axis}.
Through the collective interplay of these four constraint types, the visual objects assume their respective positions.



\item \textbf{Grouped bar chart}.
As depicted in \autoref{fig:chart_constraint} (b), the fixed constraints, gravity constraints, and support constraints applicable to a grouped bar chart closely mirror those of a stacked bar chart.
Within a single group, horizontal collision constraints arise among the bars.
The fundamental divergence between a grouped bar chart and a stacked bar chart pertains solely to the orientation of the collision constraints.

\item \textbf{Stacked area chart}.
A stacked area chart typically portrays the temporal evolution of multiple data series.
As depicted in \autoref{fig:chart_constraint} (c), the control points forming a vertical line in a visual object are governed by fixed constraints, ensuring a constant height at a specific x-position.
Each point's x-position is subject to horizontal gravity constraints.
Analogous to a stacked bar chart, gravity constraints and support constraints originating from the baseline axis are also applicable.
Furthermore, collision constraints in the vertical direction prevent visual objects from overlapping.


\item \textbf{Bubble chart}.
\autoref{fig:chart_constraint} (d) illustrates a bubble chart, where bubbles experience compression from the left and right sides while being drawn towards the horizontal line.
The absence of overlap among bubbles is due to collision constraints.
A vertical gravity constraint originates from the center, while support constraints are present on the left and right sides.



\item \textbf{Other visualizations.}
The constraints can also be applied to model other prevalent visualizations. For instance, a simple bar chart exhibits a subset of the constraints found in a stacked bar chart, with the exception of collision forces.
Regarding line charts and scatter plots, collision constraints are not applicable to visual objects within them.
Gravity constraints in both horizontal and vertical directions can effectively underpin the modeling process.

\end{itemize}

In stacked/grouped bar charts and stacked area charts, collision constraints exclusively pertain to the control points of distinct visual objects sharing the same tick value on the $x$-axis; this subset of objects is termed the collision group.
To illustrate, in \autoref{fig:chart_constraint}, stacked or grouped bars occupying identical tick positions on the $x$-axis, or points aligned along the same vertical line within the stacked area chart, form part of a collision group.
The control points within a collision group maintain an order and direction, denoted as the collision order and collision direction, respectively.
For instance, the stacked area chart presented in \autoref{fig:chart_constraint} (c) exhibits a collision order: the blue visual element is positioned above the red one, with a vertical collision direction.




We utilize these examples to demonstrate the efficacy of atomic constraints.
Such constraints can stabilize control points in their present locations and facilitate enhanced interactivity, a topic elaborated upon in ~\autoref{section:manipulation}.




\subsection{Construction of Constraints for Existing Charts}



We present a heuristic algorithm for constructing constraints in an existing visualization based on the analysis of prevalent design patterns.
Initially, we extract the control points from the given chart.
Subsequently, we establish the \textbf{visual object set}, which constitutes a collection of visual objects of the same category.
In the case of area objects (such as bars and areas), the constraint construction involves detecting the baseline axis, calculating collision constraints, and determining fixed constraints.
Regarding point objects (like points and bubbles), we incorporate a point collision detection mechanism.
For line objects (including line charts), since there typically exist no collision constraints among visual elements (as lines are usually not stacked together), we can assign gravity constraints to maintain their current positions.


\textbf{Control point extraction.}
We initiate the process by utilizing a visualization in SVG format to extract the control point positions.
In our methodology, a \texttt{<circle>} element is interpreted as a point visual object.
A line or an open path in the SVG corresponds to a line visual object.
Closed paths and rectangles are parsed as area visual objects.
The absolute coordinates of each control point are calculated.
Subsequently, we compile a roster of visual objects along with their corresponding control points.
A bar encompasses four control points located at its corners.
In contrast, each bubble constitutes an object with a solitary control point positioned at its center.
For area objects, control points are positioned at the termini of each segment within the \texttt{<path>} element of the SVG.



\textbf{Visual object set extraction.}
A visual object set comprises a group of visual objects employed in establishing constraints.
A visualization encompasses a series of data items depicted by a set of visual objects.
An area object set facilitates the creation of diverse chart types, such as stacked and basic area charts, as well as stacked, grouped, and basic bar charts.
For instance, a set of uniformly spaced bars forms a bar chart.
Point sets are employed in crafting scatter plots and bubble charts.
A singular line or an assemblage of lines can be harnessed to craft a line chart.



\begin{figure}[htbp]
    \centering
    \setlength{\belowcaptionskip}{-10px}
    \includegraphics[width=\columnwidth]{image/bar_area_deducing}
    \caption{Baseline axes and ticks of bar charts and area charts.}
    \label{fig:bar_area_deducing}
\end{figure}

\textbf{Baseline axis detector.}
For bar charts and area charts, a baseline axis is established, which enforces gravity and support constraints on the visual objects.
As demonstrated in \autoref{fig:bar_area_deducing}, a shared configuration (d) featuring a baseline axis showcases multiple parallel tick lines perpendicular to the baseline.
By analyzing the positions of control points and visual objects, we identify the baseline.
In both stacked and simple bar charts, the centers of bars are uniformly aligned along the baseline axis, simplifying the computation of tick positions via these visual object centers.
Similarly, in stacked area charts, control points oriented along the baseline axis are uniformly distributed, enabling us to designate these equidistant positions as ticks.
In the scenario of grouped bar charts, the intermediate positions on the scale do not exhibit perfect uniformity. Consequently, we ascertain these positions in conjunction with the arrangement of text (which is uniformly distributed) along the axes.


\begin{figure}[htbp]
  \centering
  \includegraphics[width=\columnwidth]{image/evenly_distribute}
  \caption{Evenly distributed intervals can be extracted using the peaks in the frequency domain even with some noise.}
  \label{fig:evenly_distributed}
\end{figure}


The central step in establishing the baseline axis involves assessing the uniform distribution of visual objects, control points, and text along the vertical or horizontal dimension.
We derive their positions in the x- and y-directions, followed by a transformation into the frequency domain using Fourier analysis.
Illustrated in \autoref{fig:evenly_distributed}, the identification of evenly spaced intervals is facilitated by identifying peaks in the frequency domain.
As shown in \autoref{fig:evenly_distributed}, positions are projected with a 5-pixel interval, accompanied by some degree of noise.
By detecting peaks within the frequency domain, we can extract distribution intervals.
This interval corresponds to a wavelength within the frequency domain, and its calculation is as follows:
$interval = \frac{50}{10} = 5.$
By employing the computed interval to filter out noise, we deduce the positions of ticks.
Upon tick detection, the baseline direction is established.
With the identification of the baseline axis direction, the baseline position can be readily determined using the positions of visual objects along the perpendicular direction.
Subsequently, the support constraints and gravity constraints are ascertained.

\textbf{Calculation of collision constraints.}
Collision constraints solely come into effect for control points of distinct visual objects situated within the same tick area, as depicted in \autoref{fig:bar_area_deducing}.
We ascertain the absence of overlapping by evaluating both the x and y directions of the control points.
In instances where a group of visual objects within a tick area exhibit non-overlapping alignment in a specific direction, we conclude that these elements possess collision constraints with one another.
For instance, in a grouped bar chart, collision constraints are oriented parallel to the baseline axis to ensure bars align in parallel along it.
In contrast, stacked area and stacked bar charts impose collision constraints perpendicular to the baseline axis to ensure elements are orthogonally stacked.


\textbf{Calculation of fixed constraints.}
Fixed constraints are confined to control points situated within the boundaries of the same visual objects and tick intervals. As depicted in \autoref{fig:bar_area_deducing}, we can enforce fixed constraints on visual objects that fall within a shared tick interval. In cases where a visual object is entirely contained within a tick interval, such as a bar in a bar chart, we establish fixed constraints on the two vertices of an edge along each of the four sides, thereby preserving the edge's positioning. In situations where elements extend across multiple tick intervals, fixed constraints are applied to the same element within the corresponding tick interval. For instance, in the context of an area chart, fixed constraints are set on two points at the same x-position perpendicular to the axis.









\textbf{Point collision detector.}
We propose a point collision detection algorithm designed to ascertain whether a collection of point objects is subject to collision constraints. When a series of circles are closely situated, the separation between pairs of circles approaches the summation of their radii, resulting in $Distance(A, B) - r_a - r_b \sim 0$.
Our collision detection algorithm computes the distribution of $Distance(A, B) - r_a - r_b$, with the expected minimum value being approximately 0 if collision constraints exist. In scenarios involving numerous interconnected pairs of circles, the FloodFill algorithm is employed to identify a continuous area of interconnected circles, originating from a specified circle.

For visual object groupings devoid of collision constraints, such as line charts and scatter plots, the gravity constraints can be aligned with the present positions of control points along both the x- and y-axes. In such visualizations, axis scaling can be adjusted to modify the gravity constraints on the control points.





It is noteworthy that the algorithms presented in this section serve as inference algorithms, grounded in prevalent implementation approaches for visualizations. The algorithm is founded on vector visualization and encompasses several established principles, such as aligning multiple labels horizontally or vertically along the axes. It encompasses a range of common visualizations, including bar charts (stacked, grouped, simple), area charts (stacked, overlapped, simple), bubble charts (including collision detection), scatter plots, and line charts.
Nevertheless, this inference method does not represent the sole solution, as alternative reverse engineering approaches~\cite{savva2011revision, poco2017reverse} for extracting visualization results can also be equivalently expressed as outcomes derived from the inference process outlined in this method.



\begin{algorithm}[!htb]
    \caption{Optimization Process: the calculation of the position of control points.}
    \label{alg:optimization}
    \begin{algorithmic}[1]
    \Require 
    \Statex Control points list $P_i$, with position $(P_i.x, P_i.y)$, velocity $(P_i.v_x, P_i.v_y)$, and radius $(P_i.r)$, $P_i.v_x = 0$, $P_i.v_y = 0$;
    \Statex Gravity constraints list $G_i$, $i = 1, 2 \ldots N_g$, $G_i = \{P, d, dim\}$; \Comment{$P$ is the position; $dim$ is the coordinate direction of $G_i$, i.e., $dim \in \{x, y\}$}
    \Statex Support constraints list $S_i$, $i = 1, 2 \ldots N_s$, $S_i = \{P, d, dim, op\}$; \Comment{$op \in \{\leq, \geq\}$.}
    \Statex Fixed constraints list $F_i$, $i = 1, 2 \ldots N_f$, $F_i = \{P_{1}, P_{2}, d, dim\}$; \Comment{$d$ is the distance.}
    \Statex Collision I constraints list $L_{i}$, $i = 1, 2 \ldots N_l$, $L_i = \{P_{1}, P_{2}, d, dim\}$; %
    \Statex Collision II constraints list $C_i$, $i = 1, 2 \ldots N_c$, $C_i = \{P_{1}, P_{2}, d\}$; %
    \Statex alpha; \Comment{The strength of current iteration.}
    
    \For{ $i = 1;\ i \leq N_g;\ i \gets i + 1$}\Comment{Handle gravity constraints.}
    \State $dim = G_i.dim$\Comment{$dim \in \{x, y\}$}
    \State $\epsilon \gets G_i.P[dim] + G_i.P.v[dim] - G_i.distance$
    \State $G_i.P.v[dim] \gets G_i.P.v[dim]  -\epsilon * alpha$; 
    \EndFor

    \For{ $i = 1;\ i \leq N_s;\ i\gets i + 1$}\Comment{Handle support constraints.}
    \State $dim = S_i.dim$
    \If{not $S_i.P[dim] + S_i.P.v[dim]\ S_i.op\ S_i.d$}\Comment{$S_i.op \in \{\leq,\geq\}$}
    \State $S_i.P[dim] \gets S_i.d$
    \State $S_i.P.v[dim] \gets 0$; 
    \EndIf
    \EndFor

    \For{ $i = 1;\ i \leq N_f;\ i\gets i + 1$}\Comment{Handle fixed constraints.}
    \State $dim = F_i.dim$; $d \gets F_i.d$;  $P_1 \gets F_i.P_1$; $P_2 \gets F_i.P_2$
    \State $\epsilon \gets P_{1}[dim] + P_{1}.v[dim] - P_{2}[dim] - P_{2}.v[dim] - d$
    \State $P_{1}[dim] \gets P_{1}[dim] - \epsilon * 0.5$;  $P_{2}[dim] \gets P_{2}[dim]  + \epsilon * 0.5$
    \State $P_{1}.v[dim], P_{2}.v[dim] \gets (P_{1}.v[dim] + P_{2}.v[dim])/2$
    \EndFor
    \For{ $i = 1;\ i \leq N_l;\ i\gets i + 1$}\Comment{Handle collision I constraints.}
    \State $dim \gets L_i[dim]$; $P_1 \gets L_i.P_1$; $P_2 \gets L_i.P_2$
    \State $\epsilon \gets P_{1}[dim] + P_{1}.v[dim] - P_{2}[dim] - P_{2}.v[dim] - L_i.d$
    \If{$\epsilon < 0$}
    \State $P_{1}[dim] \gets P_{1}[dim]  -\epsilon * 0.5$; $P_{2}[dim] \gets P_{2}[dim]  +\epsilon * 0.5$
    \State $v_{avg} \gets P_{1}.v[dim] + P_{2}.v[dim]$
    \State $P_{1}.v[dim] \gets v_{avg}$; $P_{2}.v[dim] \gets v_{avg}$
    \EndIf
    \EndFor
    
    \For{ $i = 1;\ i \leq N_c;\ i\gets i + 1$}\Comment{Handle collision II constraints.}
    \State $P_1 \gets C_i.P_1$; $P_2 \gets C_i.P_2$
    \State $d_x \gets (P_{1}.x + P_{1}.v_x - P_{2}.x - P_{2}.v_x)$
    \State $d_y \gets (P_{1}.y + P_{1}.v_y - P_{2}.y - P_{2}.v_y)$
    \State $d \gets \sqrt{(d_x)^2 + (d_y)^2}$
    \State $\epsilon \gets d - N_i.d$
    \If{$\epsilon < 0$}
    \State $bias \gets (P_{1}.radius) ^ 2 / ((P_{1}.radius) ^ 2 + (P_{2}.radius) ^ 2)$
    \State $P_{1}.v_x \gets P_{1}.v_x - \epsilon * d_x / d * bias$
    \State $P_{1}.v_y \gets P_{1}.v_y - \epsilon * d_y / d * bias$
    \State $P_{2}.v_x \gets P_{2}.v_x - \epsilon * d_x / d * (1- bias)$
    \State $P_{2}.v_y \gets P_{2}.v_y - \epsilon * d_y / d * (1 - bias)$
    \EndIf
    \EndFor
    \For{ $i = 0; i < N_p; i\gets i + 1$}\Comment{Update control points' position.}
    \State $P_i.x \gets P_i.x + P_i.v_x$; $P_i.y \gets P_i.y + P_i.v_y$
    \EndFor
    \end{algorithmic}
\end{algorithm}


\subsection{Optimization Process}

The interaction process involves three key stages: dragging, dropping, and optimization, as depicted in \autoref{fig:pipeline}.
Spatial constraints guide visual objects towards a new equilibrium.
To make the optimization process interactive and continuous, we employ physical forces to encode these constraints.
Once visual objects, control points, and constraints are extracted from the visualization, we convert the constraint conditions into forces that facilitate spatial transformations of the control points.
Each visual object is linked to an ordered sequence of control points, collectively defining point, line, or area-type visual entities based on their arrangement.
Every control point possesses attributes of position and velocity. During each iteration, these control points update their positions and velocities, resulting in the rendering of a new visual object based on the revised control point positions.
We leverage D3's force-directed simulation~\cite{bostock2011d3}, tailoring the forces to accommodate spatial constraint integration.
Further details are provided in Algorithm~\ref{alg:optimization}.
The computational complexity of a force-directed layout is $O(k * (n^2))$, where $k$ denotes the number of iterations, and $n$ signifies the count of control points.
Except for circle collisions, constraints operate solely along horizontal or vertical axes.
Consequently, the influence of each control point on others is limited, curbing the magnitude of $n$ in the complexity equation.
For circle collisions, a quad-tree division is employed before computation, reducing the collision complexity to $O(kn\ log(n))$.
In a visualization featuring 1,000 control points (e.g., a bar chart with 250 bars), the optimization process can converge within a matter of seconds.













































    




















\section{Manipulation Supported by Constraints}
\label{section:manipulation}

\revision{As shown in \autoref{fig:manipulate_classification}, there are three kinds of direct manipulations that can be classified according to the type of object being manipulated;
there are manipulations of visual objects, axes, and constraints.
These manipulations are denoted as object-level, axis-level, and constraint-level manipulations, respectively.}


\begin{figure*}[!ht]
  \centering
  \includegraphics[width=\textwidth]{image/manipulate_classification.pdf}
  \caption{
      Our work allows users to directly manipulate visual objects, axes, and constraints.
      Our method supports various interaction tasks in visualizations.
  }
  \label{fig:manipulate_classification}
\end{figure*}

\subsection{Manipulating Visual Objects}
\label{sec:manipulate_object}

Object-level manipulations directly change the positions of visual objects.
Manipulations with different directions, distances, and speeds imply various user intentions.
For example, dragging visual objects in the collision direction means changing the stacking or group order, while dragging a visual object out of the current canvas implies moving it onto a new canvas.


\textbf{Changing the stacking direction.}
A set of visual objects can be stacked together in the x-direction (e.g., a grouped bar chart) or y-direction (e.g., a stacked bar chart).
The constraints for changing the stacked direction involve setting the collision direction into another direction.
For example, in \autoref{fig:manipulate_classification} (a), when the collision constraints in the y-direction are changed to the x-direction, the new collision constraints cause the blue bars to move to the left while the downward gravity causes the blue bars to fall.
In a bar chart, the change in the direction supports the comparison of categories (changing from a stacked bar chart to a grouped bar chart) or computes a summary of categories (changing from a grouped bar chart to a stacked bar chart).


\textbf{Changing the stacking order.}
As shown in \autoref{fig:manipulate_classification} (b), the user drags the pink bar above the blue one, which implies that the user wants to change the stack order of these two visual objects.
We recalculate the collision order according to the new positions of these visual objects, i.e., for each tick area, the horizontal position of the pink bar is set to be higher than that of the blue one. 
\revision{As a result, the blue bar is pushed down to the x-axis by the gravitational force, and the pink bar falls and is stacked on top of the blue one.}
\removed{based on the new order, i.e., the spatial position of the read element is larger than the blue one, and the position of the blue element is larger than the yellow one.}

\textbf{Moving to a new canvas.}
When a visual object (a group of visual objects) is dragged off the canvas (\autoref{fig:manipulate_classification} (c)), a new canvas is constructed to hold the visual object.
The new canvas inherits the spatial constraints of the original canvas.
The visual objects of each canvas are independently manipulated and optimized.
Visual objects can be dragged to move them between existing canvases.
Quickly dragging and dropping the visual objects away from the canvas will delete these objects.

\subsection{Manipulating Axes}
\label{sec:manipulate_axes}

\revision{The foundation of axis-level manipulation is changing the axis-related constraints, i.e., the gravity, support, and fixed constraints that correspond to the axis.}
The manipulation of the axis includes zooming in on and dragging ticks, which results in the rescaling and reordering of the axis. 
\revision{There are two axes: axes with continuous (quantitative and temporal) attributes and axes with discrete (categorical) attributes.
We determine the type of axis according to the tick labels on the axis.
An axis is recognized as continuous if the texts can be converted to numbers or time stamps.}
Axis-level manipulation changes the scale or order of the axis.
\revision{For continuous axes, users can zoom in (either by scrolling or pinching) on axes or drag ticks to rescale or reorder the axis.
For discrete axes, users can drag the ticks to change the tick order or rescale by zooming (e.g., changing the widths of bars).}
As the scale or order changes, axes-related constraints are changed accordingly.

\textbf{Rescaling axes.}
As shown in \autoref{fig:manipulate_classification} (d), users can zoom in on a continuous axis to change the scale of the axis.
In our method, users can zoom in on a continuous axis in three ways, namely, by pinching, scrolling, and dragging a tick label.
The scale of the axis changes according to the zooming rate.
According to the changed scale, our model updates the gravity, support, and fixed constraints of the control points in the axis direction.
\removed{when the scale of an axis changes, we reset the gravity constraints and support constraints of the axis direction to the new position and reset the distance of fixed collisions accordingly.}


\textbf{Reordering axes.}
If the axis is discrete (e.g., categorical), users can reorder the axis directly by dragging the ticks, as shown in \autoref{fig:manipulate_classification} (e).
After dragging the ticks, the order of the ticks is recalculated according to the ticks' new positions.
The gravity constraints of a specific tick on the axis are changed accordingly.
Moreover, as dragging each tick to sort the ticks is time-consuming, we also implement a sort function; by right-clicking the axis, the ticks can be sorted according to selected visual objects' attributes (e.g., width, height, left, right, or color).
For example, we can sort the bars of a bar chart according to the height of the bars.


\begin{figure}[htb]
    \centering
    \setlength{\belowcaptionskip}{-10px}
    \includegraphics[width=\columnwidth]{image/case_force_set.pdf}
    \caption{
        Manipulating constraints and setting new constraints for visual objects results in diverse new layouts of visual objects.
        \revision{(a) A bubble chart.
        (c) Constructing an aggregated view by setting support and gravity constraints.}
    }
    \label{fig:force_handle}
\end{figure}


\subsection{Manipulating Constraints}
\label{sec:manipulate_cons}

The manipulation of constraints includes two parts: changing the existing constraints and setting new constraints.
A constraint layer shows the existing constraints of the selected visual objects.

\textbf{Changing constraints.}
The selected visual objects' constraints can be modified directly through dragging.
Our interface presents the top-$k$ gravity and support constraints with the most control points.
Each constraint has a handle for the users to manipulate directly.
When the constraints are dragged, related visual objects' constraints are changed.
\revision{For example, we can transform a stacked area chart into a ThemeRiver~\cite{havre2002themeriver} by dragging all visual objects' gravity constraints to the center, as illustrated in \autoref{fig:manipulate_classification} (f).}
In \autoref{fig:manipulate_classification} (g), the bubbles can be squeezed by changing the support constraints.

\textbf{Setting new constraints.} 
Except for manipulating an existing constraint, users can set new constraints.
For example, \autoref{fig:manipulate_classification} (g) and (f) can be seen as setting new constraints.
Moreover, setting constraints on visual objects can construct diverse visualization layouts.
For example, as shown in~\autoref{fig:force_handle}, the bubble chart is reshaped to a bar chart by setting new support and gravity constraints.

\textbf{Setting groups of constraints.} 
Users can set different gravity or support constraints for different values of a visual attribute when they want to rearrange visual objects according to their attributes.
However, it is time-consuming for users to set the constraints one by one.
We allow users to set a group of different constraints according to a certain attribute (e.g., height, width, color).
The result is that different visual objects with different attributes are separated.
As shown in \autoref{fig:manipulate_classification} (f), we can set different x-gravity constraints for different colors of bubbles.











\subsection{High-Level Interactions}

In subections~\ref{sec:manipulate_object}, \ref{sec:manipulate_axes}, and \ref{sec:manipulate_cons}, we describe low-level manipulations in our model.
These manipulations can compose high-level interactions.
Users can perform various interactions using these manipulations.
We list some interactions and discuss how they are composed by these manipulations.

\textbf{Navigating} changes the users' viewpoints.
Navigating effectively narrows the field of view to allow users to observe details, e.g., in a dense scatterplot or a multi-line chart.
Visualization with continuous axes allows navigation interactions.
In our model, the navigation is performed by rescaling the continuous axes. 


\textbf{Filtering} reduces the number of visual objects. 
\revision{Our method supports filtering by selecting focused visual objects and dragging them to a new canvas or by selecting unfocused visual objects and deleting them.}
Filtering is a generic interaction for common visualizations such as, for example, bar charts, area charts, line charts, and scatterplots.

\textbf{Rearranging} changes the spatial organization of visual objects; it includes reordering, realignment, etc. 
Rearranging is supported by our model on three levels: there is object-level, axis-level, and constraint-level rearranging.
At the object level, users can drag visual objects to reorder, align, and stack them.
For example, users can reorder the categories of a ThemeRiver graph. 
At the axis level, manipulating discrete axes means reordering or sorting the axes.
At the constraint level, flexible constraint settings create a large space for rearrangement.
For example, a user can set support constraints for visual objects to align them.

\textbf{Re-encoding} changes the encoding of the visual objects.
\revision{At the object level, users can transform a grouped bar chart into a stacked bar chart or a stacked area chart into an overlapping area chart. 
At the constraint level, users can set groups of gravity constraints for visual objects according to their size or color, and they can re-encode their positions, as \autoref{fig:manipulate_classification} (h) shows.}

\textbf{Aggregating} changes the granularity of the visual objects by gathering visual objects of the same type.
Users can define different gravity and support constraints at the constraint level for various sets of visual objects, such as those with different colors. 
There is ample flexibility in setting constraints to aggregate certain visual marks. For example, as illustrated in \autoref{fig:force_handle}, by setting collision, support, and gravity constraints, a bubble chart can be transformed into an aggregated bar chart.

\subsection{Direct Manipulation Interface}
\label{section:interface}

The interface allows users to manipulate visual objects, axes, and constraints.
As shown in \autoref{fig:interface}, a visualization is loaded onto a canvas.
Users can directly manipulate the visual objects and axes.
In the top-left corner, four icons represent deleting the canvas, copying the canvas, resetting the canvas to its initial state, and showing the constraints layer.
Clicking the constraints button leads to a new layer showing constraints and buttons that can be used to set new constraints, as shown at the bottom of \autoref{fig:interface}.
The constraints of the selected visual object are shown and can be directly manipulated.
New constraints can be set for the selected visual objects, including support (upward, downward, left, rightward), gravity (vertical and horizontal), and collision constraints. 

\begin{figure}[!ht]
    \centering
    \setlength{\belowcaptionskip}{-10px}
    \includegraphics[width=1.04\columnwidth]{image/interface.pdf}
    \caption{
        The interface of our system.
        Top: direct interactions with visual objects or axis.
        Bottom: direct setting constraints for visual objects directly.
    }
    \label{fig:interface}
\end{figure}
  











\section{Usage Scenarios} 
\label{section:use_scenario}


This section describes two real-world cases, including a stacked area chart from an online website and a bubble chart on a news website.










\begin{figure*}[!htb]
    \centering
    \includegraphics[width=\textwidth]{image/case_stack_chart.png}
    \caption{Stacked area chart example from an online webpage. 
    Our approach supports (a-c) aligning, (d-f)rescaling, and (g-i) reordering.}
    \label{fig:case_stack_chart}
\end{figure*}


\subsection{Stacked Graph of Slack Software's Message Trends}


While a stacked graph is a widely utilized visualization technique for illustrating temporal changes across multiple categories, it poses challenges related to legibility, comparison, and scalability~\cite{Baur2012touchwave}.
In terms of legibility, the perceptual distortion called the sine illusion effect~\cite{vanderplas2015signs, day1991sine} significantly affects the perception of values.
More specifically, within the context of a stacked area chart, distinct segments positioned atop it often lead to substantial perceptual distortions due to variations in the underlying stacking slopes.
Various algorithms~\cite{byron2008stacked, sinestream2021} have been developed to compute optimized static layouts for stacked graphs, aiming to enhance their legibility.
Furthermore, scalability becomes a concern when numerous categories are present, particularly when small values are challenging to discern.
Moreover, as a stacked graph represents aggregate values through the stacking of visual elements, comparing values across different time points or visual objects becomes intricate for users.
Our proposed approach addresses these aforementioned challenges by providing an effective solution for static stacked graphs.



We extracted the stacked graph that describes the trends of the messages per channel in the Slack software, as shown in \autoref{fig:case_stack_chart} (a); this graph comes from the Preset website\footnote{https://preset.io/blog/2020-09-22-slack-dashboard/}.
The original chart is a ThemeRiver~\cite{havre2002themeriver} without alignment on the $x$-axis.
A user, Martin, wants to uncover insights given this chart.
Our method makes it possible to improve the legibility of the stacked area chart by allowing the flex alignment of the visual objects and allowing the user to change the order of the visual objects.
We set an aligned baseline on the $x$-axis that is composed of two parts, the upwards support constraints of the $x$-axis and the gravity constraints of the $x$-axis.
After setting such new constraints, the visual objects fall, driven by gravity, as shown in \autoref{fig:case_stack_chart} (b), but they stop at the $x$-axis because of the support constraints from the $x$-axis.
The visual objects finally align and stack on the $x$-axis, as shown in \autoref{fig:case_stack_chart} (c), which results in a rearrangement that allows better value retrieval and the recognition of the trend for the total value.

In \autoref{fig:case_stack_chart} (c), the values after June 12th are relatively small compared to the beginning value.
These values only take up a small proportion of the vertical space, which makes it difficult to determine the data values.
Our approach allows Martin to rescale the chart in the y-direction to explore the visualization better, as shown in \autoref{fig:case_stack_chart} (d), (e), and (f).
The rescaling changes the height of the visual objects.
The ticks in the $y$-axis are updated according to the new scale, which is calculated using the D3 scale function.
However, some problems still occur because of stacking.
One problem is the distortion of the stacking area.
The mark on the top is heavily distorted because of different slopes.
As shown in \autoref{fig:case_stack_chart} (f), it is difficult for Martin to determine the trend of the top visual object (deep grey) because of the unalignment.
Martin can directly drag the visual object to the bottom to solve this problem.
The collision order of the visual objects is updated.
Collision constraints and support constraints cause the visual objects to converge at (i).
Consequently, the legibility, comparison, and scalability problems can be handled through these interactions.











\subsection{New York Times Vaccination Rate Bubble Chart}

Bubble charts are extensively discussed in academic papers (e.g., FluxFlow~\cite{fluxflow}) and are also widely used in news media.
Each point in the chart represents a data item.
Rearranging these data items can support various analytical tasks performed by users.

One example is the COVID-19 vaccination rate, which has received significant media attention.
The New York Times\footnote{\url{https://www.nytimes.com/2021/05/12/us/covid-vaccines-vulnerable.html}} published a chart that shows vaccine hesitancy, social vulnerability, and vaccination rates for each county in the US (\autoref{fig:teaser} (a)).
Each circle represents a county, with four different colors representing different regions (yellow: Northeast, blue: West, green: South, red: Midwest) and with the sizes of the circles indicating the population of each county.
In the original chart (\autoref{fig:teaser} (a)), from top to bottom, the counties are classified into four categories: ``low hesitancy and low vulnerability,'' ``high hesitancy and low vulnerability,'' ``low hesitancy and high vulnerability,'' and ``high hesitancy and high vulnerability.''

\begin{figure}[!ht]
    \centering
    \includegraphics[width=\columnwidth]{image/vaccine_chart_horizontal}
    \caption{A bubble chart showing the vaccination rate of counties in the US. 
    Our method adds various interactions to the (a) original chart, supporting a wide range of tasks that involve manipulating the marks in the static visualization. 
    For example, in (a), (c), and (g), the visual objects are driven by different gravity constraints in the x-direction.}
    \label{fig:teaser}
\end{figure}



The chart encodes five data attributes: hesitancy, vulnerability, vaccination rate, population, and region.
Assume a user, Martin, reads the news on this website, and he wants to uncover insights by exploring the visualization.
Our approach can perform various manipulations of the chart to support data exploration.
Martin wants to explore the bubbles in detail. 
He can navigate the chart by zooming in on the $x$-axis, resulting in the view shown in \autoref{fig:teaser} (b).
In \autoref{fig:teaser} (b), he can see the details of specific counties' vaccination rates.
He wants to compare counties from different regions (colors), so he sets new x-gravity constraints in the x-direction that drives different colors to move to different positions.
The optimization process is shown in \autoref{fig:teaser} (c), where the points are moving to a new position.
The converged circles in \autoref{fig:teaser} (g) are rearranged by color; this is a kind of aggregation grouped by regions. 
Circles of the same region and category are aggregated together to estimate the population of each subset using the whole-area size of the group.
In the Northeast and West, most counties are in the low-hesitancy category, while in the South and Midwest, a large number of counties are in the high-hesitancy category.


Martin hypothesizes that counties with larger populations may exhibit higher vaccination rates. As a result, he intends to investigate the correlation between population size and vaccination rate. To achieve this, he establishes distinct y-gravity constraints corresponding to the population (radius). The resulting chart is depicted in \autoref{fig:teaser} (f), revealing a relatively weak correlation between population and vaccination rate. Nevertheless, counties characterized by low vaccination rates tend to be smaller in size. Since Martin resides in a county with limited hesitancy and vulnerability, he expresses a keen interest in the distribution of counties within the same category. He can effortlessly select this category by dragging it onto a new canvas, as showcased in \autoref{fig:teaser} (d). Additionally, he can configure y-gravity constraints for various colors (regions) as shown in \autoref{fig:teaser} (e).

These explorations yield Martin a deeper level of insight compared to the original chart. In conclusion, our approach empowers users to select, filter, arrange, and aggregate marks while retaining the contextual information present in the original chart.
Our model can also be directly applied to many other bubble charts. For instance, the bubble chart in the FluxFlow~\cite{fluxflow} can serve as an input for our model, allowing for rearrangement and recombination. Additionally, changes in Visual Sedimentation~\cite{Huron2013Visual} that do not involve changes in the type of visual elements (such as from bubbles to area sedimentation) can also be represented using various constraints.






























































\begin{table}[ht!]
% \vspace{-12pt}
\caption{\textbf{User study results on the reconstructed shape and details.} {\name} achieves the best results in coarse shape and details according to human perception compared to prior art~\protect\citesupp{feng2021learningsupp,wood2022densesupp}.}\label{tab:userstudy}
\vspace{2pt}
\resizebox{1\linewidth}{!} (Ours) & 35.06\% (Dense~\citesupp{wood2022densesupp})   & 10.39\% (\citesupp{MICAsupp,deng2019accuratesupp,wu2021synergysupp,guo2020towardssupp})      \\ \midrule[1pt]
Detail         & \textbf{80.52\%} (Ours) & 11.69\% (DECA~\citesupp{feng2021learningsupp})    & 7.79\% (\citesupp{danvevcek2022emocasupp,wang2022faceversesupp,chen2020selfsupp,yang2020facescapesupp})       \\ \bottomrule[1pt]
\end{tabular}
}
\vspace{-10pt}
\end{table}
\section{Discussion and Future Work}
\label{section:discussions}

In this section, we discuss the coverage of the supported interactions and supported visualization and point out directions for future work.

\subsection{Coverage of Supported Interactions}



The manipulations supported for an existing visualization are contingent on the current set of visual objects within that visualization.
There exist two constraints: our methodology does not alter the quantity of control points, and it solely relies on information present in the current visualization without any additional data.
Of the unipolar interactions as categorized by Sedig and Parsons~\cite{sedig2013interaction}, operations such as drilling and blending, which necessitate supplementary data or a change in the control point count, are not compatible with our framework. Conversely, arrangements, assignments, cloning, comparisons, filtering, navigation, transformations, and translations can be partially or fully facilitated within the present framework, as these actions can be translated into alterations in the positions of control points.
In the realm of bipolar interactions, composing and decomposing provoke a change in the number of control points, making them non-applicable within our framework.
Interactions like gathering and discarding, inserting and removing, as well as storing and retrieving, can be executed by manipulating constraints and relocating visual objects between canvases.
In the future, the extensibility of the constraints model could involve permitting changes in control point quantities and associating visual objects with additional information.













\subsection{Coverage of Supported Visualizations}


In the previous sections, we demonstrated some common visualizations, including area charts and bubble charts.
Our approach can be applied to more visualizations.



\textbf{Limitations arising from implementation.}
Currently, our focus in terms of implementation is on non-nested visualizations within the two-dimensional Cartesian coordinate system.
The current implementation of this study effectively encompasses various prevalent visualizations, including scatter plots, line charts, and bar charts (both simple and with stacked or grouped variations).

\begin{itemize}
\item \textbf{2D Cartesian coordinates:}
The prototype system is primarily focused on visualizations within the two-dimensional Cartesian coordinate system.
These visualizations involve visual elements distributed either discretely along axes at fixed frequencies, as seen in heatmaps or Bertin Matrices~\cite{perin2014revisiting}, or more irregularly across continuous axes. We start by extracting both the horizontal and vertical axes. Then, we analyze the distribution patterns of visual elements along these axes.
By integrating these patterns with the axis structures, we are able to derive constraints for the visualizations.

\item \textbf{Exclusion of nested visualizations:}
The current model implementation does not encompass multi-layer nested structures. It does not yet accommodate visualizations that involve glyphs or exhibit nested properties, such as a bar within a grouped bar chart composed of multiple stacked bars. However, this model can accommodate such visualizations in future extensions.

\item \textbf{Deviation from conventional rules:}
Instances of parsing errors predominantly stem from heuristic solving algorithms. These errors might involve misinterpretation of numerical axes or visual arrangement patterns that deviate from common conventions.
\end{itemize}

Potential future extensions may encompass polar coordinate systems. In this context, sector charts could be interpreted as bar charts, and donut charts as stacked bar charts. Such extensions would require the augmentation of parsing algorithms for polar coordinates, as well as the development of coordinate system conversion algorithms.







\textbf{Coverage of visualization types.}
Ideally, at the conceptual level, common visualizations encode spatial channels with attributes that are compatible with our spatial constraints model.
For example, the interactions that change the stacked order/direction will be effective for the visualizations with stacked collisions.
For those visualizations that only have gravity constraints, including line charts and scatter plots, selecting and filtering can be supported by moving visual objects to a new canvas.
Currently, visualizations with a Cartesian coordinate system can be parsed by our system.
The axis type determines what axis operations are supported.
For example, the visualization in \autoref{fig:class_project} shows important moments in the lives of China's university presidents (e.g., being born or obtaining a Ph.D. degree).
We can sort the presidents according to their birthdays to better support comparison tasks.
The results are shown in \autoref{fig:class_project} (b).
We can extract a horizontal axis and a vertical axis (i.e., a visualization with a Cartesian coordinate system), and the continuous axis is taken as the linear scale. 
In the future, we could extend the model to visualizations with parallel axes or polar coordinate systems. 
For parallel axes, spatial constraints can be constructed for every axis.
If we want to support manipulations that change the organization of the axes (e.g., change the axis order of parallel coordinates), additional rules for axis positioning are needed.
For polar coordinate systems, we can conceptually regard the coordinate systems as Cartesian coordinate systems in which the pie chart is stacked in the angle direction.




\begin{figure}[htb]
    \centering
    \includegraphics[width=\columnwidth]{image/sort}
    \caption{(a) The original static visualization presents important moments in the lives of university presidents.
    (b) The view after they are sorted according to their birthdays.
    }
    \label{fig:class_project}
\end{figure}

\textbf{Raster image support.}
Our current prototype system primarily focuses on parsing visualizations in vector formats. However, many static visualizations are not represented in vector formats, such as historical visualizations~\cite{zhang2023oldvis}. For such visualizations, the automatic addition of interactivity would be of great interest. In the future, we envision leveraging object detection and optical character recognition (OCR) techniques to support raster images. Subsequent parsing procedures, akin to the present modeling process, will ensue after control points extraction. This approach holds the potential to significantly broaden the scope of applicable scenarios.




\subsection{Toward interaction authoring toolkit}






The primary innovation of our work lies in enhancing the interactive capabilities of existing visualizations. This approach emphasizes spatial representation within visualizations while disregarding tools used for visualization creation.
The principal challenge in extending this work into a toolkit lies in compatibility. The fundamental unit of interaction and transformation in this paper is the control point. However, if one aims to integrate corresponding plugins into existing toolkits, a reconfiguration of the original visualization implementation in terms of control points is required. Such an implementation may encounter resistance from existing visualization practices; for instance, within visualizations realized using Vega-Lite, our approach necessitates a reorganization of visualizations.

Customized interactive capabilities will be provided, catering to the specific needs of scenarios. The current method offers a comprehensive range of interaction features for existing visualizations. It supports the addition of interactive features to these visualizations. For many toolkit scenarios, the data and toolkit used in visualizations are known. As a manifestation of a toolkit, this paper is a well-suited candidate for augmenting interactivity upon pre-existing visualizations, with provisions for user customization, such as defining required interactions or enabling user-defined settings.

Our model enables the activation of static visualizations through a set of low-level operations involving visual objects, coordinate axes, and constraints. Nevertheless, these low-level interactions might pose a time-consuming obstacle to users in constructing meaningful interactions. To address this issue, for instance, a set of constraints could be established to transform bubbles of different colors into bars within a bar chart, obviating the need for individual constraint configuration, as illustrated in Figure \ref{fig:force_handle}. In the future, incorporating user-defined constraints as novel forms of interaction and preserving them could significantly augment the semantic diversity of interactions.



\subsection{Toward Intelligent Interaction}







This method can be enhanced in several aspects through the application of deep learning techniques.

\textbf{Intelligent constraint deduction.}
During the constraint inference process, it is common to employ heuristic and rule-based algorithms to cover prevalent implementation methodologies. However, due to varying user implementations, inherent uncertainty emerges in the inference process. For instance, within a ThemeRiver visualization, the mapping of data based on relative height versus absolute height may lack certainty. The incorporation of deep learning models can aid in resolving such classification ambiguities.

\textbf{User intent inference through mixed initiatives.}
Concerning the aspect of user intent inference, while achieving complete machine-driven inference is challenging, a subset of classification tasks can be formulated to guide machine decision-making. For instance, one such task involves discerning whether a user rapidly drags a visual element beyond the current canvas – an action open to diverse interpretations of speed across user profiles. To address this, user feedback can be leveraged, enabling deep learning to enhance user exploration efficiency. It could contribute to decisions like creating new canvases, deleting visual objects, among other actions, while not entirely replacing user agency.

\textbf{Combining our model with other interaction techniques.}
Our method facilitates a more intuitive interaction for users to comprehend visualizations. This endeavors to minimize the user's barrier to utilization and comprehension of visualization interactions. In the future, incorporating natural language interaction represents another means of reducing the user's threshold. Our approach can be integrated with natural language interaction techniques, enabling users to simply describe the comparison they wish to make between two parts of a stacked area chart, for example. After natural language processing, our system can then further display the transition of visual objects that clearly illustrates the differences. This approach results in reduced barriers for the user in terms of both articulating their intention and comprehending changes.


























  


























\section{Conclusions}
We consider the phase-extraction problem, and we showed that, given a unitary $U = e^{i\pi H}$ and its inverse $U^{\dag}$, we could implement a block-encoding of $\phi(H)$ for some smooth function $\phi(x)$. The word `smooth' here means existence and continuity of the derivatives: the higher the number of continuous derivatives that a function has, the faster its Fourier sum (and thus the Laurent polynomial on the eigenphases) uniformly converges to that function. We are confident this can have many more applications beyond what is shown in this work. It is also worth remarking that Jackson showed that the convergence rate of a Fourier series is almost-optimal, in the sense that no trigonometric (or, equivalently, complex exponential) series can approximate the desired function faster, up to that $\log d$ factor~\cite[p.\ 21]{jacksonTheoryApproximation1930a}. Also remember that `smoothing' a function, i.e., replacing its derivative with a continuous function, does not give faster convergence for free in general, as its derivative will become steep in the points where we smooth out discontinuities, and this translates to a high Lipschitz constant: a~clear example is given by Eq.~\ref{eq:lipschitz-constant-recurrence-solution}, but in that case, fortunately, nothing depends on the size of the input $N$, and thus does not influence the asymptotic query complexity of Algorithm~\ref{alg:prop-sampling-qsp}, although the constant factor can become large even for $p = 20$. From a theoretical point of view, this work shows that, for any $\eta > 0$, there is an algorithm with query complexity 
$$\Tilde{\bigO}\left(\frac{1}{\bar{c}^{\frac{1}{2} + \eta}} \frac{1}{\epsilon^\eta} \right)$$
solving the proportional-sampling problem. This statement seems to suggest there exists an algorithm which directly solves the problem with $\eta = 0$, and an open question would be to find such algorithm.


It is also interesting to remark that Theorems~\ref{thm:haah-construction},~\ref{thm:haah-completion} indeed allow the construction for any $\phi$, even complex-valued, provided that its absolute value is reciprocal.

One could think that, in Section~\ref{sec:prop-sampling}, instead of using the linear function in the phase-extraction subroutine, we could approximate the square root and then apply the transformation directly on $e^{i \pi c(x)}$. However, in the case of proportional sampling this would be inconvenient, as the derivative of the square root function has a discontinuity with an infinite jump around 0, and we could not choose a constant $\delta$ if we had values of the oracle that are too close to $0$.

\begin{acks}
This work is supported by NSFC No. 62272012. 
\end{acks}

\bibliographystyle{ACM-Reference-Format}
\bibliography{main.bib}

% \printbibliography


\end{document}
