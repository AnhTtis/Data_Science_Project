% CVPR 2023 Paper Template
% based on the CVPR template provided by Ming-Ming Cheng (https://github.com/MCG-NKU/CVPR_Template)
% modified and extended by Stefan Roth (stefan.roth@NOSPAMtu-darmstadt.de)

\documentclass[10pt,twocolumn,letterpaper]{article}

%%%%%%%%% PAPER TYPE  - PLEASE UPDATE FOR FINAL VERSION
% \usepackage[review]{cvpr}      % To produce the REVIEW version
\usepackage{cvpr}              % To produce the CAMERA-READY version
%\usepackage[pagenumbers]{cvpr} % To force page numbers, e.g. for an arXiv version

% Include other packages here, before hyperref.
\usepackage{graphicx}
\usepackage{amsmath}
\usepackage{amssymb}
\usepackage{booktabs}
\usepackage{color} 
\usepackage{multirow} 
\usepackage{algorithm}  
\usepackage{algpseudocode}  
\usepackage{amsmath}  
% \usepackage{authblk}
\renewcommand{\algorithmicrequire}{\textbf{Input:}}  % Use Input in the format of Algorithm  
\renewcommand{\algorithmicensure}{\textbf{Output:}} % Use Output in the format of Algorithm  
% It is strongly recommended to use hyperref, especially for the review version.
% hyperref with option pagebackref eases the reviewers' job.
% Please disable hyperref *only* if you encounter grave issues, e.g. with the
% file validation for the camera-ready version.
%
% If you comment hyperref and then uncomment it, you should delete
% ReviewTempalte.aux before re-running LaTeX.
% (Or just hit 'q' on the first LaTeX run, let it finish, and you
%  should be clear).
\usepackage[pagebackref,breaklinks,colorlinks]{hyperref}


% Support for easy cross-referencing
\usepackage[capitalize]{cleveref}
\crefname{section}{Sec.}{Secs.}
\Crefname{section}{Section}{Sections}
\Crefname{table}{Table}{Tables}
\crefname{table}{Tab.}{Tabs.}


%%%%%%%%% PAPER ID  - PLEASE UPDATE
\def\cvprPaperID{424} % *** Enter the CVPR Paper ID here
\def\confName{CVPR}
\def\confYear{2023}


\begin{document}

%%%%%%%%% TITLE - PLEASE UPDATE
\title{Local-to-Global Panorama Inpainting for Locale-Aware Indoor Lighting Prediction}


\author{Jiayang Bai$^1$, Zhen He$^1$, Shan Yang$^1$, Jie Guo$^1$\footnotemark[1], Zhenyu Chen$^1$, Yan Zhang$^1$, YanwenGuo$^1$\\
$^{1}$Nanjing University, Nanjing, China\\
{\tt\small  \{jybai, hz,yangshan\}@smail.nju.edu.cn guojie@nju.edu.cn}\\
{\tt\small MF21330012@smail.nju.edu.cn \{zhangyannju, ywguo\}@nju.edu.cn}
}

% \author{Jiayang Bai\\
% Nanjing University\\
% {\tt\small jybai@smail.nju.edu.cn}
% % For a paper whose authors are all at the same institution,
% % omit the following lines up until the closing ``}''.
% % Additional authors and addresses can be added with ``\and'',
% % just like the second author.
% % To save space, use either the email address or home page, not both
% \and
% Zhen He\\
% % Nanjing University\\
% {\tt\small hz@smail.nju.edu.cn}
% \and
% Shan Yang\\
% Nanjing University\\
% {\tt\small yangshan@smail.nju.edu.cn}
% \and
% Jie Guo\\
% Nanjing University\\
% {\tt\small guojie@nju.edu.cn}
% \and
% Zhenyu Chen\\
% Nanjing University\\
% {\tt\small MF21330012@smail.nju.edu.cn}
% \and
% Yan Zhang\\
% Nanjing University\\
% {\tt\small zhangyannju@nju.edu.cn}
% \and
% Yanwen Guo\\
% Nanjing University\\
% {\tt\small ywguo@nju.edu.cn}
% }
% \maketitle

\twocolumn[{
\renewcommand\twocolumn[1][]{#1}
\maketitle
\centering
\vspace{-0.4cm}
\includegraphics[width=\textwidth]{figures/methods/teaser_v2.pdf}
\vspace{-0.5cm}
\captionsetup{type=figure}
\caption{We propose a locale-aware indoor illumination prediction method that can generate a full and texture-rich HDR panorama (the third column in each group) at any locale in the scene, enabling spatially-varying and consistent shading after virtual object insertion (the first and second columns).}
\label{fig:teaser}

\vspace{0.7cm}
}]
\footnotetext[1]{Corresponding author.}

%%%%%%%%% ABSTRACT
\begin{abstract}
Predicting panoramic indoor lighting from a single perspective image is a fundamental but highly ill-posed problem in computer vision and graphics. To achieve locale-aware and robust prediction, this problem can be decomposed into three sub-tasks: depth-based image warping, panorama inpainting and high-dynamic-range (HDR) reconstruction, among which the success of panorama inpainting plays a key role. Recent methods mostly rely on convolutional neural networks (CNNs) to fill the missing contents in the warped panorama. However, they usually achieve suboptimal performance since the missing contents occupy a very large portion in the panoramic space while CNNs are plagued by limited receptive fields. The spatially-varying distortion in the spherical signals further increases the difficulty for conventional CNNs.
To address these issues, we propose a \emph{local-to-global strategy} for large-scale panorama inpainting. In our method, a depth-guided local inpainting is first applied on the warped panorama to fill small but dense holes. Then, a transformer-based network, dubbed \emph{PanoTransformer}, is designed to hallucinate reasonable global structures in the large holes. 
To avoid distortion, we further employ cubemap projection in our design of PanoTransformer. The high-quality panorama recovered at any locale helps us to capture spatially-varying indoor illumination with physically-plausible global structures and fine details. 
\end{abstract}

%%%%%%%%% BODY TEXT
\section{Introduction}
\label{sec:introduction}
% \begin{itemize}
%     % Diffusion of FL
%     \item {\st{Diffusion of FL}}
%     % Security threats to FL
%     \item {\st{Security threats to FL with particular focus on model poisoning}}
%     % Limitations of existing countermeasures
%     \item {\st{Current countermeasures (e.g., KRUM) and their limitations}}
%     % Proposed method and its advantages
%     \item {\st{Intuitive description of the proposed method and its difference (i.e., advantages) w.r.t. state of the art}}
%     % Main contributions
%     \item {\st{Summary of the main contributions of this work}}
%     % Paper's structure and organization
%     \item {\st{Paper's structure and organization}}
% \end{itemize}

% Diffusion of FL
Recently, {\em federated learning} (FL) has emerged as the leading paradigm for training distributed, large-scale, and privacy-preserving machine learning (ML) systems~\cite{mcmahan2017googleai,mcmahan2017aistats}. 
The core idea of FL is to allow multiple edge clients to collaboratively train a shared, global model without disclosing their local private training data.
%Specifically, an FL system consists of a central server and many edge clients; 
A typical FL round involves the following steps: {\em(i)} the server randomly picks some clients and sends them the current, global model; {\em(ii)} each selected client locally trains its model with its own private data; then, it sends the resulting local model to the server;\footnote{Whenever we refer to global/local model, we mean global/local model {\em parameters}.} {\em(iii)} the server updates the global model by computing an \emph{aggregation function}, usually the average (FedAvg), on the local models received from clients.
% \begin{enumerate}
%     \item[{\em(i)}] the server sends the current, global model to the clients and appoints some of them for training;
%     \item[{\em(ii)}] each selected client locally trains its copy of the global model with its own private data; then, it sends the resulting local model back to the server;\footnote{Whenever we refer to global/local model, we mean global/local model {\em parameters}.}
%     \item[{\em(iii)}] the server updates the global model by computing an \emph{aggregation function} on the local models received from clients (by default, the average, also referred to as FedAvg~\cite{mcmahan2017aistats}).
% \end{enumerate}
This process goes on until the global model converges. %(e.g., after a certain number of rounds or other similar stopping criteria).
%\\
% The advantages of FL over the traditional, centralized learning paradigm are undoubtedly clear in terms of flexibility/scalability (clients can join/disconnect from the FL network dynamically), network communications (only model weights\footnote{We will use \textit{parameters} and \textit{weights} interchangeably.} are exchanged between clients and server), and privacy (each client's private training data is kept local at the client's end and not uploaded to the server).
\\
% Security threats to FL
%However, the growing adoption of FL also raises security concerns~\cite{costa2022covert}, particularly about its confidentiality, integrity, and availability.
Although its advantages over standard ML, FL also raises security concerns~\cite{costa2022covert}. %, particularly about its confidentiality, integrity, and availability~\cite{costa2022covert}.
% OLD, LONG VERSION
% Indeed, some work deals with privacy leakage that may expose the local data of some clients~\cite{melis2019sp}. 
% A large body of work, instead, investigates attacks that usually aim to detriment the predictive accuracy of the learned global model. For instance, \emph{data poisoning} attacks achieve this goal by letting an adversary pollute the training set of some corrupt FL clients with maliciously crafted examples~\cite{jagielski2018sp}.
% Similarly, in \emph{model poisoning} the attacker attempts to tweak the global model weights~\cite{bhagoji2019pmlr} by directly perturbing the local model's weights of some infected FL clients before these are sent to the central server for aggregation, usually via so-called Byzantine attacks. 
% It turns out that Byzantine model poisoning attacks severely impact standard FedAvg; therefore, more robust aggregation functions must be designed to make FL systems secure.
Here, we focus on \emph{untargeted model poisoning} attacks~\cite{bhagoji2019pmlr}, where an adversary attempts to tweak the global model weights %\footnote{We will use the terms \textit{parameters} and \textit{weights} interchangeably.} 
by directly perturbing the local model's parameters of some infected clients before these are sent to the central server for aggregation.
In doing so, the adversary aims to jeopardize the global model \textit{indiscriminately} at inference time.
Such model poisoning attacks severely impact standard FedAvg; therefore, more robust aggregation functions must be designed to secure FL systems.
\\
% In this paper, we focus on designing a novel robust aggregation scheme at the server's end to contrast the effect of Byzantine model poisoning attacks.
%
% Current countermeasures and their limitations
%Several countermeasures have been proposed in the literature to combat model poisoning attacks on FL systems.
% Some methods use simple statistics more robust than plain average to smooth the impact of malicious updates (e.g., Trimmed Mean and FedMedian~\cite{yin2018icml}). 
% Other defenses implement outlier detection techniques to discard malicious updates from the aggregation performed at the server's end. Those are either based on heuristics (e.g., Krum/Multi-Krum~\cite{blanchard2017nips} and Bulyan~\cite{mhamdi2018pmlr}) or data-driven approaches (e.g., K-means clustering~\cite{shen2016acm} or DnC via spectral analysis~\cite{shejwalkar2021ndss}). 
% Finally, some strategies rely on a centralized ``source of trust'' to spot potential malicious updates (e.g., FLTrust~\cite{cao2020fltrust}).
% Several countermeasures have been proposed in the literature to combat model poisoning attacks on FL systems, i.e., to discard possible malicious local updates from the aggregation performed at the server's end. 
% These techniques range from simple statistics more robust than plain average (e.g., Trimmed Mean and FedMedian~\cite{yin2018icml}) to outlier detection heuristics (e.g., Krum/Multi-Krum~\cite{blanchard2017nips} and Bulyan~\cite{mhamdi2018pmlr}) or data-driven approaches (e.g., spectral analysis via K-means clustering~\cite{shen2016acm} or spectral analysis), or methods based on ``source of trust'' (e.g., FLTrust~\cite{cao2020fltrust}).
% OLD, LONG VERSION
%Several countermeasures have been proposed in the literature to combat Byzantine model poisoning attacks on FL systems.
% Descriptive statistics
% For example, Trimmed Mean and FedMedian aggregate local model updates using more robust statistics than standard average~\cite{yin2018icml}.
%
% % Heuristics for outlier detection
% Many existing Byzantine-resilient strategies implement some outlier detection heuristics to discard the model updates sent by potentially malicious clients from the input of the aggregation function.
% One of the most popular heuristics is Krum~\cite{blanchard2017nips}.
% This strategy tries to mitigate the impact of Byzantine attacks by selecting as a global model the local model with the smallest sum of Euclidean distances to {\em all} the other local models.
% Although powerful, Krum requires the server to know (or, at least, estimate) the number of malicious FL clients upfront, which is generally impossible in a realistic attack scenario. %
% Moreover, Krum may become ineffective for complex, high-dimensional model parameter spaces due to the curse of dimensionality.
% Bulyan~\cite{mhamdi2018pmlr} tries to overcome this issue by combining Krum with a variant of Trimmed Mean.
% % Data-driven outlier detection
% Other strategies use data-driven outlier detection techniques -- e.g., via K-means clustering~\cite{shen2016acm} -- to spot potential malicious local model updates. 
% %For instance, Shen et al. propose to cluster local model updates with K-means and thus identify outliers.
%
% % Other techniques
% As far as the server is concerned, any local model received can be from a potential malicious client. 
% FLTrust~\cite{cao2020fltrust} assumes the server acts as a client, i.e., trains a local model on an additional {\em trustworthy} dataset at the server's end and compares it against all the local models from other clients. 
% This way, the server can rely on some ``source of trust'' when discarding potentially malicious clients.
%\\
% Limitations of existing Byzantine-resilient strategies
Unfortunately, existing defense mechanisms either rely on simple heuristics (e.g., Trimmed Mean and FedMedian by~\cite{yin2018icml}) or need strong and unrealistic assumptions to work effectively (e.g., foreknowledge or estimation of the number of malicious clients in the FL system, as for Krum/Multi-Krum~\cite{blanchard2017nips} and Bulyan~\cite{mhamdi2018pmlr}, which, however, cannot exceed a fixed threshold).
Furthermore, outlier detection methods using K-means clustering~\cite{shen2016acm} or spectral analysis like DnC~\cite{shejwalkar2021ndss} do not directly consider the temporal evolution of local model updates received.
Finally, strategies like FLTrust~\cite{cao2020fltrust} require the server to collect its own dataset and act as a proper client, thereby altering the standard FL protocol.
\\
% OLD, LONG VERSION
% Overall, existing Byzantine-resilient strategies are either simple heuristics (e.g., FedMedian) or, if they are more complex, they rely on strong and unrealistic assumptions to work effectively (e.g., knowing the number of malicious clients in the FL system in advance, as for Krum and alike).
% Furthermore, data-driven outlier detection methods do not consider the temporary evolution of local model updates received (e.g., K-means clustering). 
% Finally, strategies like FLTrust requires the server to collect its own dataset and act as a proper client, thereby altering the standard FL protocol.
%
% Description of the proposed method
This work introduces a novel pre-aggregation \textit{filter} robust to untargeted model poisoning attacks. Notably, this filter $(i)$ operates without requiring prior knowledge or constraints on the number of malicious clients and $(ii)$ inherently integrates temporal dependencies. 
The FL server can employ this filter as a preprocessing step before applying \textit{any} aggregation function, be it standard like FedAvg or robust like Krum or Bulyan.
Specifically, we formulate the problem of identifying corrupted updates as a multidimensional (i.e., matrix-valued) time series anomaly detection task. 
The key idea is that legitimate local updates, resulting from well-calibrated iterative procedures like stochastic gradient descent (SGD) with an appropriate learning rate, show \textit{higher predictability} compared to malicious updates. This hypothesis stems from the fact that the sequence of gradients (thus, model parameters) observed during legitimate training exhibit regular patterns, as validated in Section~\ref{subsec:intuition}. %until convergence. 
%This regularity may be more pronounced for smooth convex loss functions, but it can still be captured within an appropriate time window, even for more complex and convoluted loss surfaces. 
%We provide evidence of this claim in Appendix~B, where we show that the average mutual information (i.e., ``predictability''), calculated over pairs of legitimate model updates sent at different FL rounds, is significantly higher than the corresponding computation for a malicious client.
\\
Inspired by the matrix autoregressive (MAR) framework for multidimensional time series forecasting~\cite{chen2021je}, we propose the FLANDERS ({\em \textbf{F}ederated \textbf{L}earning meets \textbf{AN}omaly \textbf{DE}tection for a \textbf{R}obust and \textbf{S}ecure}) filter.
The main advantages of FLANDERS over existing strategies like FLDetector~\cite{zhao2020multivariate} are its resilience to large-scale attacks, where $50\%$ or more FL participants are hostile, and the capability of working under realistic non-iid scenarios.
We attribute such a capability to two key factors: $(i)$ FLANDERS works without knowing a priori the ratio of corrupted clients, and $(ii)$ it embodies temporal dependencies between intra- and inter-client updates, quickly recognizing local model drifts caused by evil players. Below, we summarize our main contributions:

\begin{itemize}
\item[{\em(i)}]
We provide empirical evidence that the sequence of models sent by legitimate clients is more predictable than those of malicious participants performing untargeted model poisoning attacks.
\\
\item[{\em(ii)}] 
We introduce FLANDERS, the first pre-aggregation filter for FL robust to untargeted model poisoning based on multidimensional time series anomaly detection.
\\
\item[{\em(iii)}] 
We integrate FLANDERS into Flower,\footnote{\scriptsize{\url{https://flower.dev/}}} a popular FL simulation framework for reproducibility.
\\
\item[{\em(iv)}] 
We show that FLANDERS improves the robustness of the existing aggregation methods under multiple settings: different datasets, client's data distribution (non-iid), models, and attack scenarios.
\\
\item[{\em(v)}] 
We publicly release all the implementation code of FLANDERS along with our experiments.\footnote{\scriptsize{\url{https://anonymous.4open.science/r/flanders_exp-7EEB}}}
\end{itemize}

% Paper's structure and organization
The remainder of the paper is structured as follows. %some related work and the current state-of-the-art solutions to security issues that FL entails. 
Section~\ref{sec:background} covers background and preliminaries. 
In Section~\ref{sec:related}, we discuss related work.
Section~\ref{sec:problem} and Section~\ref{sec:method} describe the problem formulation and the method proposed. % to tackle it. 
Section~\ref{sec:experiments} gathers experimental results. %, and Section~\ref{sec:limitations} discusses some limitations of this work.
Finally, we conclude in Section~\ref{sec:conclusion}.
 %discusses the limitations of this work and draws future research directions.
%reports conclusions and draws perspectives for future research directions.

%%%%%%% OLD %%%%%%%
%to overcome the resilience of Byzantine failures in distributed Stochastic Gradient Descent computations. 
% The strength of Krum is its time complexity, which is linear in the gradient dimension. 
% However, the robustness of the approach is guaranteed for gradient-based learning applications only when the majority of the clients are not compromised. 
% Besides, the aggregation mechanism of Krum, as well as that of similar methods, is robust from a coarse-grained perspective and does not provide solutions to errors and perturbations that may occur at inference time.
%A related approach to~\cite{blanchard2017nips} is the work of Su et al.~\cite{su2016dc}. Here, the authors propose an iterated approximate agreement to tackle a multi-layer scenario attacked by Byzantine agents. 
%However, the method works efficiently on the sole discrete context and it is inapplicable to continuous state environments.
%\gabri{Maybe, we should just talk about the main limitations of existing countermeasures without digging into their details (or, we can just mention Krum as this is the most popular one). I will move the description of all these methods to the Related Work section.}

\section{Related work}
\noindent \textbf{Video foundation models.}
With sufficient computational power and an abundant source of data, there have been attempts to build a single large-scale foundation model that can be adapted to diverse downstream tasks.
Along with the success of foundations models in the natural language processing domain~\cite{brown2020language,chen2021evaluating,devlin2019bert} and in computer vision~\cite{bertasius2021space,jia2021scaling,radford2021learning}, video data has become another data type of interest, as it has grown in scale due to numerous internet video-sharing platforms.
Accordingly, several methods to train a video foundation model have been proposed.
Due to the innate multi-modality of video data, \textit{i.e.}, a combination of visual $\cdot$ vocal $\cdot$ textual context, most works have centered around the variations of the cross-modal attention mechanism \cite{akbari2021vatt,bertasius2021space,gabeur2020multi,luo2020univl,neimark2021video,tan2021look,wei2020multi,yang2021taco}.
In addition, as most video data lack proper labels or descriptions, contrastive learning methods were studied to learn meaningful feature representations or enhance video-text alignment in a self-supervised manner \cite{akbari2021vatt,kuang2021video,luo2020univl,yang2021taco}.

More specifically, MERLOT \cite{zellers2021merlot} proposed a multi-modal representation learning method for visual commonsense reasoning, which also performed well in twelve video reasoning tasks.
VATT \cite{akbari2021vatt} introduced a multi-modal learning method via contrastive learning. 
The pre-trained model performed well in a variety of vision tasks from image classification to video action recognition and zero-shot video retrieval.
Another representative work, UniVL \cite{luo2020univl} proposed a straightforward pre-training method with auxiliary loss functions. 
After fine-tuning on a specific task, the pre-trained model performed outstandingly in a wide range of tasks of text-to-video retrieval, action segmentation, action step localization, video sentiment analysis, and video captioning.
Other foundation models for multiple video tasks include \cite{li2020hero,sun2019learning,sun2019videobert,zhu2020actbert,fu2021violet,wang2022all}. 

\noindent \textbf{Auxiliary learning.}
In order to enhance the performance of one or a multitude of primary tasks, auxiliary learning methods can be incorporated.
\cite{ruder2017overview} introduced Multi-task learning (MTL) to the deep neural networks by training a single model with multiple task losses to assist learning on the main task.
Such a method is generally adapted to pre-train the foundation models in the self-supervised manner~\cite{li2020hero,sun2019learning,sun2019videobert,zhu2020actbert,fu2021violet,wang2022all}.
However, these various pretext task losses used in the pre-training phase are ignored in the fine-tuning phase, and only the primary task loss is minimized.

Recently, meta-learning methods have been introduced for auxiliary learning.
\cite{liu2019self,navon2020auxiliary,shu2019meta} proposed a meta-learning method in which the model learns auxiliary tasks to generalize well to unseen data. 
In these settings, a separate subset of data is held out as the primary task, while the others are used as auxiliary tasks that aid the primary task's performance.
Similar methods were adopted for computer vision tasks such as semantic segmentation \cite{xu2021leveraging}.
Other domain applications include navigation tasks with reinforcement learning \cite{ye2021auxiliary}, or self-supervised learning methods on graph data \cite{hwang2020self}.

\begin{figure*}[t]
  \includegraphics[width=\textwidth]{figures/methods/V7.pdf}
  \caption{Architecture of the proposed locale-aware indoor illumination prediction method. The key is a local-to-global panorama inpainting pipeline that recovers both physically-plausible global structures and fine local details from a FOV-limited input image.}
  \label{fig:pipeline}
\end{figure*}
\section{Local-to-global Panorama Inpainting}
\label{section:overview}
Our goal in this paper is to estimate a full HDR environment map $\mathbf{P}_{\text{HDR}}$ at a  locale $\mathbf{R}$ of an FOV-limited LDR image $\mathbf{I}$.
We follow the general framework of Song \emph{et al.}~\cite{song2019neural} which decomposes this ill-posed problem into three sub-tasks: depth-based image warping, LDR panorama inpainting and HDR panorama reconstruction (see Fig. \ref{fig:pipeline}). In the first task, we leverage the recent DPT~\cite{dpt2021} to estimate the depth map $\mathbf{D}$ for $\mathbf{I}$. 
% DPT produces more fine-grained and globally coherent predictions when compared to fully-convolutional architecture. 
Then $\mathbf{D}$ and $\mathbf{I}$ are geometrically warped and transformed to $360^{\circ}$ spherical panoramas centered at the selected locale $\mathbf{R}$, denoting as $\mathbf{\hat{D}}$ and ${\hat{\mathbf{P}}}$ respectively. 
This warping operation is realized through a forward projection using the estimated scene geometry and camera pose. Some regions in $\mathbf{\hat{P}}$ are missing because they do not have a projected pixel in $\mathbf{I}$.
The following task, LDR panorama inpainting, aims to infer a dense panorama from the sparse ${\hat{\mathbf{P}}}$. In this section, we introduce a local-to-global panorama inpainting pipeline that can recover a complete LDR panorama $\mathbf{P}_{G}$ from $\mathbf{\hat{P}}$. 
In the last sub-task, we utilize the state-of-the-art network of ~\cite{LDR2HDR} to reconstruct the HDR panoramas $\mathbf{P}_{\text{HDR}}$ from $\mathbf{P}_{G}$ for inserted objects at $\mathbf{R}$. We detail our local-to-global pipeline in this paper and more details about the other off-the-shelf networks are provided in the supplemental materials.

\subsection{Observation and overview}

\begin{figure}[t]
  \begin{center}
  \renewcommand\tabcolsep{1.0pt}
  \begin{tabular}{ccc}
  \small{Input} &\small{Attention matrix} & \small{Attention map}  \\
    \includegraphics[height=0.24\linewidth, clip]{figures/methods/4_6_attention/6/inpaint/_input_cube_mark.png}&
    \includegraphics[height=0.24\linewidth, clip]{figures/methods/4_6_attention/6/inpaint/block-9_attn_mark.png}&
    \includegraphics[height=0.24\linewidth, clip]{figures/methods/4_6_attention/6/inpaint/patch-286_attn_line.png} 
     \\
    
    \includegraphics[height=0.24\linewidth, clip]{figures/methods/4_6_attention/6/noinpaint/_input_cube_mark.png}&
    \includegraphics[height=0.24\linewidth, clip]{figures/methods/4_6_attention/6/noinpaint/block-9_attn_mark.png}&
    \includegraphics[height=0.24\linewidth, clip]{figures/methods/4_6_attention/6/noinpaint/patch-286_attn_line.png}
  \end{tabular}
  \end{center}
  \caption{From left to right: the input incomplete panoramas with red boxes specifying the selected patches, attention matrices of the 9th transformer block with red lines indicating the attention scores of the selected patches, attention maps of the selected patches.}
  \label{fig:attention}
\end{figure}


For panorama inpainting, previous works \cite{song2019neural} mostly resort to fully convolutional networks. However, CNN-based models achieve suboptimal performance due to large-scale sparse missing contents in the warped panorama and some inherent limitations of convolutional layers. CNNs are good at preserving local structures and modeling rich textures but fail to complete the large hole regions. 
Therefore, previous works can hardly acquire sufficiently broad context as well as significant high-level representations from sparse omnidirectional images. Meanwhile, the distortion in spherical panoramas will further hamper the performance on large-scale inpainting.
    
Compared to CNN models with limited perspective fields, transformer is designed to support long-term interaction via the self-attention module~\cite{vit}. The self-attention mechanism can directly compute the interaction between any pair of patches, naturally capturing long-range dependencies and having a global receptive field at every stage. However, we observe that transformers work poorly on the sparse input which is the case in our task. We compare the attention maps and attention score maps for the selected patch from a sparse panorama and a dense one for illustration in Fig.~\ref{fig:attention}. As we can see, the query patch contains adequate illumination information. Given a dense input, the query patch impacts some regions (e.g. the ground) with red colors in the attention map. However, transformer blocks have trouble recovering the global structure from scattered pixels, thus the lighting can not pass information to invisible patches properly, resulting in the smooth attention map.

\begin{figure}[t]
  \includegraphics[width=\linewidth]{figures/methods/warpholesV2.pdf}
  \caption{Illustration of warping. The warping operation results in missing regions which are categorized as pixel-stretching regions and out-of-view regions.}
  \label{fig:pixel}
\end{figure}

Unfortunately, the problem of sparsity is inevitable in our task due to the limited FOV of our input perspective images. The panorama captured at $\mathbf{R}$ should have a 360$^\circ$ FOV while some parts are out-of-view in $\mathbf{I}$. We mark the invisible regions with blue in Fig.~\ref{fig:pixel}. The missing parts are expected to be restored by a global understanding of the scene. Another factor of the sparsity stems from pixel stretching. During image warping, some regions marked in red in Fig.~\ref{fig:pixel} are actually visible in $\mathbf{I}$, but still have small holes due to pixel stretching. 

Based on the above observations, our main idea is that we first fill pixel-stretching regions according to their neighboring pixels to alleviate the sparsity, and then fill other large holes based on the global understanding of the whole scene. To this end, we propose a novel local-to-global inpainting pipeline, which can be formulated as follows:
\begin{equation}
    \mathbf{P}_{G} = \mathbf{M} \odot \mathbf{\hat{P}}+ (1-\mathbf{M})\odot \mathcal{G}(\mathcal{L}(\mathbf{\hat{P}});\mathcal{L}(\mathbf{M})) 
\label{equ:mask}
\end{equation}
where $\mathbf{M}$ is the binary mask indicating visible pixels in $\mathbf{\hat{P}}$ and $\odot$ denotes element-wise multiplication. 

The key to Eq.~\ref{equ:mask} is a local-to-global panorama inpainting pipeline that applies a local inpainting module $\mathcal{L}$ and a global inpainting module $\mathcal{G}$ sequentially on the warped panorama $\mathbf{\hat{P}}$.
Our local inpainting method aims to fill dense holes in the pixel-stretching regions, according to the depth information. The local inpainting module employs a modified bilateral filtering-based method to remove dense and small holes in the pixel-stretching regions. After that, a global inpainting module based on a novel transformer architecture is developed to extract reliable global features from visible regions and then fill large holes in the out-of-view regions. Our specially-designed transformer architecture, named PanoTransformer, takes a cubemap projection as input to resolve the problem of spatial distortion in the spherical signals.  


\subsection{Depth-guided local inpainting}
\begin{algorithm}[tb]
  \caption{Depth-guided local inpainting}
  \label{code:localinpaint}
  \begin{algorithmic}[1]
    \Require
      The input image, $\mathbf{I}$;
      The estimated depth, $\mathbf{D}$;
      The warped image centered at selected location, $\mathbf{\hat{P}}$;
      The warped depth centered at selected location, $\mathbf{\hat{D}}$;
      The threshold $t$
    \Ensure
      The panorama inpainted locally, $\mathbf{P}_{L}$;
    \State $\mathbf{P}_{L} = \mathbf{\hat{P}}$
    \State $//$ Depth recovery.
    \State $\mathbf{\hat{D}} = \text{morphologyClosing}(\mathbf{\hat{D}})$
    \State $\mathbf{\hat{D}} = \text{bilateralFilter}(\mathbf{\hat{D}})$
    
    \State $//$ Depth-guided inpainting.
    \For{each $d\in \mathbf{\hat{D}}$}
      \State Calculate the pixel coordinate $\mathbf{c}_{p}$ of $d$ ;
      \State $\mathbf{c}_w = \text{PixelToWorldCoordinate}(\mathbf{c}_{p})$
      \State Project $\mathbf{c}_w$ to $\mathbf{D}$ to get its pixel coordinate $\mathbf{c}$;
      \State $d_i = \mathbf{D}[\mathbf{c} ]$
      \If {$\|d_i - d \| < t$}
        \State $\mathbf{P}_{L}[\mathbf{c}_p]=\mathbf{I}[\mathbf{c} ]$
      \EndIf
    \EndFor
  \end{algorithmic}
\end{algorithm}

The local inpainting module aims to alleviate the sparsity caused by pixel stretching. Our local inpainting method contains two basic steps shown in Algorithm~\ref{code:localinpaint}. First, we leverage morphology operations (Line 3) and bilateral filtering (Line 4) to fill holes in the warped depth $\mathbf{\hat{D}}$ as much as possible. The intuition of performing depth recovery is that depth values in the pixel stretching areas vary smoothly, while the warped panorama $\mathbf{\hat{P}}$ may have rich textures. 

Then, we traverse all valid depth values in $\mathbf{\hat{D}}$, and fill missing regions in $\mathbf{\hat{P}}$ with re-projected pixel values from $\mathbf{I}$. Guided by the recovered depth, we only fill those pixel-stretching regions with colors in $\mathbf{I}$. Note that our local inpainting method effectively fills dense but small holes and maintains the local structures.


\subsection{Global inpainting with PanoTransformer}
\label{section:lt}
For global inpainting, we design and train a transformer-based network, named PanoTransformer, to hallucinate contents in out-of-view regions. PanoTransformer can be logically separated to an encoder and a decoder, where the encoder captures long-range distortion-free representations, while the decoder gradually recovers the spatial information for generating accurate pixel-level predictions.

Considering the distortion in the equirectangular projection, PanoTransformer takes the warped cubemap projection as input. To distinguish pixels needed to be regressed, a binary mask is concatenated to the RGB channels in cubemap projection. 
We denote the input as $\mathbf{x}\in \mathbb{R}^{6\times H\times W\times 4}$ where $H$ and $W$ is the height and width of each face in the cubemap projection.

In the encoder, the input cubemap projection $\mathbf{x}$ is first reshaped into a sequence of flattened 2D patches $\mathbf{x}_p\in \mathbb{R}^{N \times(p^2\times 4)}$, where $(p, p)$ is the resolution of each image patch, and $N = \frac{6\times H\times W}{p^2}$ is the resulting number of patches. 
Each patch is then mapped to 1D tokens $z_i\in \mathbb{R}^d$ using a trainable linear projection and augmented with position embeddings to retain positional information.
These tokens are then fed into several transformer blocks. Each transformer block comprises multiheaded self-attention (MHSA)~\cite{NIPS2017_3f5ee243}, layernorm (LN)~\cite{ba2016layer} and MLP blocks.
In $l$-th transformer block, $\boldsymbol{z}^l$ is fed as the input, yielding $\boldsymbol{z}^{l+1}$ as follows:
\begin{equation}
    \boldsymbol{w}^l =\text{MHSA}(\text{LN}(\boldsymbol{z}^l)) +\boldsymbol{z}^l
\end{equation}
\begin{equation}
    \boldsymbol{z}^{l+1} =\text{MLP}(\text{LN}(\boldsymbol{w}^l)) +\boldsymbol{w}^l
\end{equation}

Noted that the token-dimensionality is fixed throughout all layers. Thus the resulting $\boldsymbol{z}$ can be transformed back to a cubemap-shape representation $\boldsymbol{z}\in \mathbb{R}^{6\times H\times W\times C}$ according to the position of the initial patch. $C$ is the channel of the extracted feature. 
These reconstructed priors $\boldsymbol{z}$ contain ample cues of global structure and coarse textures, thanks to the transformer’s strong representation ability and global receptive fields. 
$\boldsymbol{z}$ can be regarded as six images of the cubemap projection. We feed these image-like features to six Residual blocks~\cite{resnet} to replenish texture details. 
\begin{figure}[tb]
  \includegraphics[width=1.0\linewidth]{figures/methods/data_make.pdf}
  \caption{We apply masks from ~\cite{song2019neural} on high-quality panoramas from Matterport3D~\cite{Matterport3D}, SUN360~\cite{DBLP:conf/cvpr/XiaoEOT12} and Laval~\cite{gardner2017learning} to generate pairs of masked input and the ground truth for training.}
  \label{fig:dataset}
\end{figure}

\begin{figure}[tb]
\begin{center}
  \renewcommand\tabcolsep{1.0pt}
  \begin{tabular}{ccccccc}
  \multicolumn{4}{c}{\includegraphics[width=0.52\linewidth, clip]{figures/results/ablation/train_on_neural_dataset/4686_our_mark.png}}&\includegraphics[width=0.13\linewidth,  clip]{figures/results/ablation/train_on_neural_dataset/4686_neural_data_clipcombine.png}&
  \includegraphics[width=0.13\linewidth,  clip]{figures/results/ablation/train_on_neural_dataset/4686_our_clipcombine.png}&
  \includegraphics[width=0.13\linewidth,  clip]{figures/results/ablation/train_on_neural_dataset/4686_gt_clipcombine.png}\\
  \multicolumn{4}{c}{\small{Our result and the input}}&\small{With~\cite{song2019neural}}&
  \small{Ours}&
  \small{Reference}\\
  \end{tabular}
  \end{center}
  \caption{Comparison between models trained with our dataset and the dataset of Song \emph{et al.}~\cite{song2019neural}.}
  \label{fig:nodata}
\end{figure}

\subsection{Dataset}
\label{section:dataset}
Currently, the only dataset that contains paired LDR perspective images and the corresponding HDR panoramas for a diverse set of locales is proposed by Song \emph{et al.}~\cite{song2019neural} based on Matterport3D~\cite{Matterport3D}. Unfortunately, the reconstructed HDR panoramas have obvious artifacts (\emph{e.g.}, stitching seams and broken structures) as we explained in the supplemental materials. This prohibits our model from inferring complete and globally consistent structures seen at each locale.

Considering the above issue, we collect a large scale dataset with high-quality and diverse panoramas from Matterport3D~\cite{Matterport3D}, SUN360~\cite{DBLP:conf/cvpr/XiaoEOT12} and Laval~\cite{gardner2017learning}.
Apart from the panoramas, training PanoTransformer also requires masks to generate the sparse input $\mathbf{\hat{P}}$. As the invisible regions are mainly on the top of panorama, we generate masks from the dataset of Song \emph{et al.}~\cite{song2019neural} instead of generating randomly. These sparse masks are obtained by geometrically warping, fitting to the real-world data distribution. These masks are locally inpainted before feeding to PanoTransformer.
The main difference of our dataset to Song \emph{et al.}'s~\cite{song2019neural} dataset is that our panoramas and masks are unpaired, hence we can randomly apply diversified irregular masks on one panorama to generate various input. Since we focus on the inpainting task, we do not require that the mask and the panorama are physically correlated. Our dataset (some examples are shown in Fig. \ref{fig:dataset}) ensures that the ground-truth panoramas are free from artifacts. In all, we gather 38,929 high-quality panoramas accompanied by randomly selected masks for training and 5,368 for evaluation. We ensure that the scenes used in the evaluation would not appear in the training procedure. 
As shown in Fig. \ref{fig:nodata}, the model trained with our dataset produces much better results than that trained with Song \emph{et al.}'s~\cite{song2019neural} dataset. Note the cluttered structures generated by the latter model due to artifacts in Song \emph{et al.}'s~\cite{song2019neural} dataset. Please refer to the supplemental material for more comparisons.



\subsection{Loss function and training details}
We optimize PanoTransformer by minimizing a pixel-wise reverse Huber loss ~\cite{Laina_2016_3DV} between the predicted panoramas and corresponding ground truth. Since using a standard L1 loss function to learn a binary light mask heavily penalizes even small shifts of a light source position, the reverse Huber loss takes advantage of L1 loss and L2 loss as below:
\begin{equation}
L_B=\left\{
\begin{aligned}
 &|y-\hat{y}|, & |y-\hat{y}| \leq T \\
&\frac{(y-\hat{y})^2+T^2}{2T},  & |y-\hat{y}| \ge T
\end{aligned}
\right.
\label{eq6}
\end{equation}
where $y$ is the ground truth value and $\hat{y}$ is the prediction. The threshold $T$ is set to 0.2 in our experiments.
To generate more realistic details, an extra adversarial loss is also involved in the training process. Our discriminator adopts the same architecture as Patch-GAN~\cite{patchgan}.


We implement our PanoTransformer using the PyTorch framework~\cite{pytorch}. Adam~\cite{kingma2017adam} optimizer is used with the default parameters $\beta_1=0.9$ and $\beta_2 = 0.999$ and an initial learning rate of 0.0001.
PanoTransformer is trained on our dataset for 100 epochs. Training are conducted on two NVIDIA RTX 3090 GPUs with a batch size of 8. 


\begin{figure*}
  \includegraphics[width=\textwidth]{figures/results/evaluateLight/evaluateLight.pdf}
  \caption{Rendering comparison with Gardner \emph{et al.}~\cite{Gardner17}, Neural Illumination~\cite{song2019neural} and EMLight~\cite{zhan2020emlight}. We render two teapots with a matte silver and a mirror material below each recovered/ground-truth illumination map. }
  \label{fig:ball_comparions}
\end{figure*}

\section{Results}
\label{sec:results}

\begin{figure}[t]
    \centering
    \includegraphics[width=\textwidth]{figs/Nexf-3D-compare.png}
    \caption{Comparison of 3D oral reconstruction by different methods from PX imaging. The reconstruction results are shown by maximum projection to compare density details. We could easily find that our method show the best performance with clear density density distributions and teeth boundaries.
    }
    \label{fig:3d_compare}
\end{figure}

\subsection{Comparison of 3D reconstruction with other models}
We compare Oral-NeXF with existing deep-learning-based tomography models, and present the results in Figures~\ref{fig:3d_compare} and \ref{tab:compare}, where we observe that Oral-NeXF achieves the best performance. Oral-3D \cite{oral_3d}, ResCNN \cite{x_to_3d}, and GAN \cite{gan} are trained using paired images generated from the reserved 60 cases. Specifically, GAN is trained using the same encoding-decoding network as ResEncoder and the same discriminator as Oral-3D but without any curve information. Moreover, the NAF \cite{naf} model is trained similarly to our work, but utilizes a trainable hash embedding for position encoding and a 3D attenuation coefficient predictor as the neural field function. As shown in the figures, Oral-NeXF achieves remarkable performance with clear details, without requiring prior expert knowledge or additional patient data.

\subsection{Experiment analysis}
Combining the results presented in Figure~\ref{fig:3d_compare} and Table~\ref{tab:compare}, we observe that Oral-NeXF achieves state-of-the-art performance. In contrast, ResEncoder and GAN can only restore the curved shape by learning from numerous paired images. Oral-3D achieves better performance in shape restoration and detail reconstruction, mainly due to prior knowledge of the dental arch shape information that enables the generator to focus on learning inverse projection. On the other hand, NAF fails to generate a detailed structure and contains much noise in the surroundings. As mentioned earlier, a general neural field function with 3D coordinate input and single-head prediction cannot fit PX imaging. This is also demonstrated in our ablation study.


\subsection{Ablation Study}
\label{sec:ablation}
We conduct an ablation study to evaluate the contribution of each component in Oral-NeXF. We use the letters M, D, and S to denote the experiments: 1) replacing the multi-head field function with a single-head predictor and taking in 3D coordinates as input for the positional encoder; 2) using a fixed sampling rate of $N_s=1$ to generate sample points on projection rays; 3) changing the formula in Equation (\ref{eq:render_discrete}) to a weighted sum function that strictly follows the Beer–Lambert law by taking the voxel intensity as Hounsfield units. As shown in Table~\ref{tab:ablation}, the proposed dynamic sampling method plays the most crucial role in 3D reconstruction. This finding is consistent with the experiment in NeRF, where the model uses a coarse network to predict the particle density distribution for high-resolution generation. The drop in Dice and SSIM also highlights the importance of multi-head prediction and soft rendering in Oral-NeXF.

\begin{table}[tp]
    \centering
    \caption{Evaluation of 3D oral reconstruction by PSNR, SSIM, and Dice.}
    \label{tab:compare}
    \setlength\tabcolsep{2pt}
    \begin{tabular}{p{1.8cm}<{\centering}p{1.8cm}<{\centering}p{2.0cm}<{\centering}p{1.8cm}<{\centering}p{1.8cm}<{\centering}p{1.8cm}<{\centering}}
    \hline
    Method&Oral-3D&ResEncoder
    &GAN&NAF&\textbf{Ours}\cr
    \hline
    PSNR&18.59$\pm$0.70 &18.26$\pm$0.62&16.71$\pm$0.89&18.35$\pm$0.86&18.26$\pm$0.50\cr
    SSIM(\%)&76.88$\pm$1.26 &72.67$\pm$1.56&75.10$\pm$1.46&60.69$\pm$2.69&76.67$\pm$1.72\cr
    Dice(\%)&65.94$\pm$4.24&62.52$\pm$5.56&63.96$\pm$7.03&57.20$\pm$3.94&72.09$\pm$3.63\cr
    \hline
    Overall&78.60&75.49&76.93&65.93&\textbf{80.02}\cr
    \hline
    \end{tabular}
\end{table}

\begin{table}[tp]
    \centering
    \caption{Ablation study by removing each component in Oral-NeXF. M: Multi-head Prediction, D: Dynamic Sampling, S: Soft Rendering}
    \label{tab:ablation}
    \setlength\tabcolsep{2pt}
    \begin{tabular}
     {p{0.8cm}<{\centering}p{0.8cm}<{\centering}p{0.8cm}<{\centering}p{2.2cm}<{\centering}p{2.2cm}<{\centering}p{2.2cm}<{\centering}p{2.2cm}<{\centering}}
    \hline
    M&D&S
    &PSNR&SSIM(\%)&Dice(\%)&Overall\cr
    \hline
    \xmark&\cmark&\cmark&17.12$\pm$0.86&71.28$\pm$3.38&61.03$\pm$6.07&72.64(-7.38)\cr
    \cmark&\xmark&\cmark&13.02$\pm$0.52&50.83$\pm$0.65&31.18$\pm$4.70&49.03(-30.99)\cr
    \cmark&\cmark&\xmark&15.80$\pm$0.38&58.72$\pm$0.90&53.01$\pm$3.88&63.79(-16.43)\cr
    \hline
    \end{tabular}
\end{table}

\section{Conclusion}\label{sec:conclusion}
In this work, we focus on addressing the fundamental challenge of OOD detection tasks, which is how to fully understand the semantic discrepancy between the ID/OOD samples. We reveal that the key to success in the realistic SCOOD task is to allocate as many ID samples in the unlabeled set correctly as possible. To this end, we propose a novel uncertainty-aware optimal transport scheme that introduces class-specific energy scores as guidance for effective label assignment. Experimental results show that our method achieves better performance than previous state-of-the-art methods on SCOOD benchmarks.

\textbf{Limitations.} In addition to temperature scaling, other techniques such as feature clipping applied in ReAct~\cite{sun2021react} also enhance the performance of energy score, so how to obtain an OOD score that best fits the SCOOD task can be further explored. Moreover, a setting highly related to SCOOD has been proposed in \cite{katz2022training} and formulated as a constrained optimization problem. We will also theoretically analyze these practical OOD settings in our feature work.

% \section*{Acknowledgments}
\textbf{Acknowledgments.} 
This work is supported by National Key R\&D Program of China under Grant 2020AAA0105701, National Natural Science Foundation of China (NSFC) under Grants 61872327, Major Special Science and Technology Project of Anhui, National Natural Science Foundation of China (62033012) and Ant Group through Ant Research Intern Program.


%%%%%%%%% REFERENCES
{\small
\bibliographystyle{ieee_fullname}
\bibliography{acmart}
}

\end{document}
