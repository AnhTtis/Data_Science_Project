\documentclass[../main.tex]{subfiles}

\begin{document}

\textit{K. M. Hess,	R. J. Jurek, S. Kitaeff, P. Serra, A. X. Shen,	J. M. van der Hulst, T. Westmeier}\newline
\begin{figure}
	\centering
	\includegraphics[trim={0cm 1cm .cm 0.cm},clip,width=0.95\columnwidth]{images/sofia_reliability-MOD4.pdf}
	\caption{Team SoFiA: Histogram of total detections (light-grey), real galaxies (dark-grey), detections after filtering (red) and real galaxies after filtering (blue) as a function of integrated signal-to-noise ratio from a SoFiA run on the development cube (top panel). The reliability of the original and filtered catalogue is shown as the grey and orange curve, respectively (bottom panel). Parameter space filtering significantly boosts SoFiA's reliability at low SNR. Note that we measure SNR within the actual SoFiA source mask, and the resulting values can not be directly compared with the optimised SNR defined in Section~\ref{snr}.}
	\label{fig_sofia_reliability}
\end{figure}

\noindent Team SoFiA made use of the Source Finding Application (SoFiA; \citealt{2015MNRAS.448.1922S,2021MNRAS.506.3962W}) to tackle the  Challenge. Development version 2.3.1 of the software, dated 22 July 2021,\footnote{\url{https://github.com/SoFiA-Admin/SoFiA-2/tree/11ff5fb01a8e3061a79d47b1ec3d353c429adf33}} was used in the final run submitted to the scoring service. To minimise processing time, 80~instances of SoFiA were run in parallel, each operating on a smaller region ($\approx 11.8~\mathrm{GB}$) of the full cube. The processing time for an individual instance was just under 25~minutes, increasing to slightly more than 2~hours when all 80~instances were launched at once due to overhead from simultaneous file access. The resulting output catalogues were merged and any duplicate detections in areas of overlap between adjacent regions discarded.


After flagging of bright continuum sources $> 7~\mathrm{mJy}$ followed by noise normalisation in each spectral channel, SoFiA's S+C finder was run with a detection threshold of $3.8$ times the noise level, spatial filter sizes of 0, 3 and 6~pixels and spectral filter sizes of 0, 3, 7, 15 and 31~channels. We adopted a linking radius of 2 and a minimum size requirement of 3~pixels/channels. Lastly, reliability filtering was enabled with a reliability threshold of 0.1, an SNR threshold of 1.5 and a kernel scale factor of 0.3.

Based on tests using the development cube, we improved the reliability of the resulting catalogue by removing all detections with $n_{\rm pix} < 700$, $s < -0.00135 \times (n_{\rm pix} - 942)$ or $f > 0.18 \times \mathrm{SNR} + 0.17$, where $n_{\rm pix}$ is the number of pixels within the 3D source mask, $s$ is the skewness of the flux density values within the mask, $f$ is the filling factor of the source mask within its rectangular 3D bounding box, and $\mathrm{SNR}$ is the integrated signal-to-noise ratio of the detection. Detection counts for the original and filtered catalogue from the development cube are shown in Fig.~\ref{fig_sofia_reliability} as a function of SNR. Our final detection rate peaks at $\mathrm{SNR} \approx 3$, with a reliability of close to $1$ down to $\mathrm{SNR} \approx 2$. The filtered catalogue from the full cube contains almost $25,000$ detections, about $23,500$ of which are real, implying a global reliability of 94.2\%.

It should be emphasised that our strategy of first creating a low-reliability catalogue with SoFiA and then removing false positives through additional cuts in parameter space is based on development cube tests and was adopted to maximise our score. This strategy may not work well for real astronomical surveys which are likely to have different requirements for the balance between completeness and reliability than the one mandated by the scoring algorithm.

Lastly, the source parameters measured by SoFiA were converted to the requested physical parameters. As the calculation of disc size and inclination required spatial deconvolution of the source, we adopted a constant disc size of $8.5$ arcsec and an inclination of $57.3$ degrees for all spatially unresolved detections. In addition, statistical noise bias corrections were derived from the development cube and applied to SoFiA's raw measurement of integrated flux, line width and H{\sc i} disc size.

\end{document}