\documentclass[../main.tex]{subfiles}

\begin{document}

\textit{L. Yu, B. Liu , H. Xi, R. Chen, B. Peng}\newline

\noindent The JLRAT team first divided the whole dataset into small cubes of size  $320\times320\times160$ (RA, Dec, frequency) before applying to each cube a CNN containing a fully convolutional layer and a softmax layer. The CNN used 1D spectra from the cube as inputs and produced a masked output of candidate spectral signals. Using the inner product, we computed the correlation in the space domain between each candidate spectrum and known spectra from the SDC2 development cube. The result provided us with a set of 3D cubes, each containing a predicted galaxy with approximate position and size, and accurate line width. A two-dimensional Gaussian function was used to fit the moment zero map with an intensity cutoff at 1 M$_{\odot}$ pc$^{-2}$. The fit produced an ellipse with central position (RA, Dec), major axis and position angle, and the inclination of the galaxy.  The flux integral was obtained by integrating the spectra within the ellipse in both space and frequency.





\end{document}