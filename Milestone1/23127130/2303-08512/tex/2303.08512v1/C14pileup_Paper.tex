\documentclass[submitting]{nst}

\usepackage{subfigure,dcolumn}
\usepackage[T2A,T1]{fontenc}
\usepackage[russian,english]{babel}

\usepackage{listings}
\usepackage{makecell}
\lstloadlanguages{[LaTeX]TeX}
\lstset{language=[LaTeX]TeX,keywordstyle=\color{red},showspaces=true,breaklines=true,breakatwhitespace=true,basicstyle=\small\tt,commentstyle=\color{white},frame=single,framerule=0pt,backgroundcolor=\color{yellow}}


\begin{document}


\title{Discrimination of \textit{pp} solar neutrinos and $^{14}$C pile-up events in a large scale LS detector}\thanks{This work was supported by National Natural Science Foundation of China No. 12005044.}

\author{Guo-Ming Chen}
\affiliation{School of Physical Science and Technology, Guangxi University, Nanning 530004, China.}
\author{Xin Zhang}
\affiliation{Institute of High Energy Physics, Beijing 100049, China.}
\affiliation{University of Chinese Academy of Sciences, Beijing 100049, China.}
\author{Ze-Yuan Yu}
\affiliation{Institute of High Energy Physics, Beijing 100049, China.}
\author{Si-Yuan Zhang}
\affiliation{School of Physical Science and Technology, Guangxi University, Nanning 530004, China.}
\author{Yu Xu}
\affiliation{School of Physics, Sun Yat-Sen University, Guangzhou 510275, China.}
%\author{Cheng-Feng Yang}
%\affiliation{School of Physics, Sun Yat-Sen University, Guangzhou 510275, China.}
\author{Wen-Jie Wu}
\affiliation{Department of Physics and Astronomy, University of California, Irvine, California, USA}
\author{Yao-Guang Wang}
\affiliation{Institute of High Energy Physics, Beijing 100049, China.}
\author{Yong-Bo Huang}
\email[Yong-Bo Huang, ]{huangyb@gxu.edu.cn}
\affiliation{School of Physical Science and Technology, Guangxi University, Nanning 530004, China.}



\begin{abstract}
    As an unique probe, the precision measurement of \textit{pp} neutrinos is important for the study of the Sun's energy mechanism, thermodynamic equilibrium monitoring, and neutrino oscillation in the vacuum dominated region. For a large scale liquid scintillator detector, one of the bottleneck for \textit{pp} neutrino detection comes from the pile-up events of intrinsic $^{14}$C decays. This paper presents a few approaches to discriminate \textit{pp} neutrinos and $^{14}$C pile-up events by considering the difference in their time and spatial distributions. In this work, a Geant4-based Monte Carlo simulation was built, then multivariate analysis and deep learning technology are adopted respectively to investigate the capability of $^{14}$C pile-up reduction. As a result, VGG network and BDTG model show good performance in the discrimination of \textit{pp} solar neutrinos and $^{14}$C double pile-up events, their signal significance can achieve 15.6 and 10.3 using only one day statistics. In this case, the signal efficiency is 42.7\% for the discrimination using VGG network when rejecting 99.81\% $^{14}$C double pile-up events. And the signal efficiency is 51.1\% for the case using BDTG model when rejecting 99.18\% $^{14}$C double pile-up events.
 
\end{abstract}


\keywords{Liquid scintillator detector, pp neutrinos, $^{14}$C pile-up, multivariate analysis, deep Learning}

\maketitle

\section{Introduction}
\label{section:Introduction}
    
    With the development of nuclear physics and astrophysics, we were able to glimpse the Sun's energy mechanism, which comes from the nuclear fusion of light nuclei in the core of the Sun~\cite{Bethe-1938,Bethe-1939,Bahcall-1996pt}. The proton-proton (\textit{pp}) cycle produces $\sim$99\% of the solar energy, and its primary reaction is the fusion of two protons into a deuteron:

    \begin{large}
		\begin{equation} 
		p + p = {^{2}He} + e^{+} + \nu_{e}
		\label{eq:pp}
		\end{equation} 
	\end{large}
    
    In the reaction, large amounts of low-energy neutrinos emitted, named as \textit{pp} neutrinos ($E \textless 0.42$~MeV). In addition, proton–electron–proton (\textit{pep}) process and the secondary reactions in the \textit{pp} cycle with neutrinos emitted as well, they are known as \textit{pep} neutrinos, $^{7}$Be neutrinos, $^{8}$B neutrinos and \textit{hep} (helium-proton) neutrinos, respectively. The remaining energy of the Sun is contributed by the carbon–nitrogen–oxygen (CNO) cycle, with CNO neutrinos emitted. The detection of solar neutrinos is considered to be a direct way of testing theoretical solar models. However, as differences between early observations and theoretical predictions were discovered~\cite{Davis-1968cp, Cleveland-1998nv, GALLEX-1992gcp, GALLEX-1998kcz, Kaether-2010ag, GNO-2005bds, SAGE-2009eeu, Gavrin-2019sok, Kamiokande-II-1989hkh, Kamiokande-1996qmi}, the so-called "solar neutrino problem" has plagued us for more than 30 years. Later, MSW-LMA mechanism~\cite{Wolfenstein-1977ue, Mikheyev-1985zog} is proved to be the standard solution since the solid evidences provided by SNO~\cite{SNO-2001kpb, SNO-2003bmh} and KamLAND~\cite{KamLAND-2002uet}. Currently, the Standard Solar Model (SSM)~\cite{Bahcall-1995bt, Christensen-Dalsgaard-1996hpz, DeglInnocenti-1996uex, Brun-1999dw, Bahcall-2001pf, Serenelli-2009yc} can provide a precise prediction for the flux and energy distribution of the solar neutrinos. As for the detection of solar neutrinos, almost all solar neutrino components are observed~\cite{Borexino-2007kvk, BOREXINO-2014pcl, BOREXINO-2018ohr, BOREXINO-2020aww} and we are expected to go to an era of precise and comprehensive measurement of solar neutrinos in the next decades~\cite{Gann-2021ndb, Xu-2022wcq}.
   
    \textit{pp} neutrinos are strongly related to the predominant energy production of solar and they carry the recent message of the core of the Sun, these characteristics make \textit{pp} neutrinos become an unique messenger for the study of the Sun's energy mechanism and thermodynamic equilibrium monitoring. On the other hand, \textit{pp} neutrinos can be used for the study of neutrino oscillation in the vacuum dominated region. The detection of \textit{pp} neutrinos requires a low threshold ($\sim 200$~keV) and an effective background reduction in the meantime. The first detection of \textit{pp} neutrinos was made by $^{71}Ga$-based radiochemical detectors~\cite{GALLEX-1992gcp, GALLEX-1998kcz, Kaether-2010ag, GNO-2005bds, SAGE-2009eeu, Gavrin-2019sok}. Later, large scale liquid scintillator (LS) detector was successfully applied in Borexino experiment and it provided the best measurement of \textit{pp} neutrinos on $\sim$10\% level~\cite{BOREXINO-2014pcl, BOREXINO-2018ohr} via elastic neutrino-electron scattering. 
    
    According to the experiences from Borexino, intrinsic $^{14}$C decays from the organic liquid scintillator and its associated pile-up events are important internal background for a large scale LS detector. $^{14}$C pile-up events correspond to the case that more than one $^{14}$C decays at different detector positions but take place in the same trigger window. In addition, pile-up can be classified into the following categories according to the multiplicity of $^{14}$C accidental coincidence: double pile-up, threefold pile-up, fourfold pile-up, and so on. Borexino experiment ($\sim$278~ton) takes a lot of efforts in LS purification and makes the $^{14}$C concentration reach to about $2.7 \times 10^{-18}$~g/g. With this  concentration, $^{14}$C double pile-up is about 10\% of the events in the spectral gap between $^{14}$C and $^{210}$Po spectra~\cite{BOREXINO-2014pcl}. 
   
    \begin{table*}[!htb]
	\caption{The event rates (unit:~cpd/kton) of \textit{pp} neutrinos, $^{14}$C single and pile-up events in different $^{14}$C concentrations. In this table, A spherical LS detector ($\sim$12~kton) with 15~m radius is used in calculation, and the time window is 500~ns. The values in bracket indicate the event rates inside the energy range of interest (0.16, 0.25)~MeV, the ratio is about 10\% for both \textit{pp} neutrinos and $^{14}$C double pile-up events.}
	\label{table:tab-eventRate}
		\renewcommand\arraystretch{2.4}%cm
		\setlength{\tabcolsep}{7mm}
		\centering
		\begin{tabular}{c|c|c|c|c}
			\hline
			 Event types & $10^{-18} g/g$ & \makecell[c]{$2.7\times10^{-18} g/g$ \\ (Borexino)} & $5\times10^{-18} g/g$ & $10^{-17} g/g$   \\ \hline
			\textit{pp}-$\nu$ & \makecell[c]{$1.37\times10^{3}$ \\ ($\sim$~$1.37\times10^{2}$~)} & \makecell[c]{$1.37\times10^{3}$  \\ ($\sim$~$1.37\times10^{2}$~)} & \makecell[c]{$1.37\times10^{3}$ \\ ($\sim$~$1.37\times10^{2}$~)} & \makecell[c]{$1.37\times10^{3}$ \\ ($\sim$~$1.37\times10^{2}$~)}     \\ \hline
			$^{14}$C single & $1.43\times10^{7}$ & $3.86\times10^{7}$ & $7.16\times10^{7}$ & $1.43\times10^{8}$ \\ \hline
			$^{14}$C double & \makecell[c]{$2.38\times10^{4}$ \\ ($\sim$~$2.38\times10^{3}$~)} & \makecell[c]{$1.73\times10^{5}$ \\ ($\sim$~$1.73\times10^{4}$~)} & \makecell[c]{$5.94\times10^{5}$ \\ ($\sim$~$5.94\times10^{4}$~)}& \makecell[c]{$2.38\times10^{6}$ \\ ($\sim$~$2.38\times10^{5}$~)} \\ \hline
			$^{14}$C triple & $1.97\times10^{1}$ & $3.88\times10^{2}$ & $2.47\times10^{3}$ & $1.97\times10^{4}$\\ \hline
			\makecell[c]{Signal-to-background\\ ratio:  ($\frac{\textit{pp}-\nu}{^{14}C double}$)} & \makecell[c]{$\sim$~1 : 17 \\} & \makecell[c]{$\sim$~1 : 126 \\} & \makecell[c]{$\sim$~1 : 431   \\  } & \makecell[c]{$\sim$~1: 1727   \\   }  \\ \hline 
		\end{tabular}
	\end{table*}
    
    For a LS detector whose sensitive target mass is $m$ kiloton (kton), the frequency of $^{14}$C single event is: 
    
    \begin{large} 
		\begin{equation} 
		f_{single} [Hz] = \frac{C_{^{14}C} \cdot N_{A} \cdot m}{\tau \cdot M} \times 10^{9}
		\label{eq:frequency_single}
		\end{equation} 
	\end{large}
    
    where $N_{A}$ is Avogadro's constants ($6.023 \times 10^{23}$), and $\tau$, $M$, $C_{^{14}C}$ correspond to $^{14}$C's lifetime, molar mass and its concentration in LS, respectively. 
    
    The frequency of $^{14}$C pile-up events can be calculated in the following:
    
     \begin{large} 
		\begin{equation} 
		%f_{pile-up} [Hz] = n! \prod_{i=1}^{n} f_{single} \cdot \Delta t^{n-1} 
		f_{pile-up} [Hz] = \frac{e^{-f_{single} \cdot \Delta t}}{(n-1)!} \cdot f^{n}_{single} \cdot \Delta t^{n-1} \cdot \varepsilon
		\label{eq:frequency_pile-up}
		\end{equation} 
	\end{large}
    
    Where $n$ ($n \geq 2$) denotes the multiplicity of $^{14}$C accidental coincidence, for example, $n = 2$ represents the case of double $^{14}$C pile-up. $\Delta t$ is the time window for detection and $\varepsilon$ corresponds to the reconstruction efficiency of $^{14}$C pile-up events. 
    
    As the detector mass increases, the dramatic increase in $^{14}$C pile-up events has to be taken into account and rejected effectively. Taking a large spherical LS detector as an example, assuming the radius of the detector is 15~m thus the detector mass is about 12~kton, Table~\ref{table:tab-eventRate} lists the event rate of \textit{pp} neutrinos, $^{14}$C single and pile-up events in different $^{14}$C concentrations, 500~ns time window was used in this calculation and reconstruction efficiency was set to 100\%. Assuming the $^{14}$C concentration of LS is $2.7 \times 10^{-18}$~g/g in the above detector, Fig.~\ref{fig:spectrum_compare} shows the recoil energy spectrum of \textit{pp} neutrinos via elastic neutrino-electron scattering, its calculation can be found in~\cite{Xu-2022wcq}. And the energy spectra of $^{14}$C single, double and triple pile-up events are shown for comparison. For this giant detector, \textit{pp} neutrino signals are totally swamped by $^{14}$C pile-up events of more than two orders of magnitude.
    
    In Table~\ref{table:tab-eventRate}, the values in bracket indicate the event rates inside the energy range of interest, which is from 0.16~MeV to 0.25~MeV for deposited energy by considering the \textit{Q} value of $^{14}$C $\beta$ decay is $\sim156$~keV and the scattered electron of \textit{pp} neutrino is difficult to distinguish from the emitted electron of $^{14}$C single event. The target mass of the above detector ($\sim$12~kton) is $\sim$43 times larger compared to Borexino ($\sim$278~kton), as a result, the signal-to-background ratio of $pp$ neutrinos and $^{14}$C double pile-up events is smaller than 1 : 126 in the case of $^{14}$C concentration at $2.7 \times 10^{-18}$~g/g in this detector, and the signal-to-background ratio will be much more poor if unlucky $^{14}$C concentration was found. On the other hand, since the energy resolution will introduce smearing in the energy spectrum, the energy range of analysis needs to be determined according to the realistic situation.  
    
    
    
    \begin{figure}[!htb]
		\centering
		\includegraphics[width=1\hsize]{figure/FullEnSpec_5e10_18.pdf}
		\caption{The recoil energy spectra of \textit{pp} neutrinos, $^{14}$C single, double and triple pile-up events in a spherical LS detector, whose radius and $^{14}$C concentration are 15~m and $5 \times 10^{-18}$~g/g, respectively. The spectra didn't include the detection effect: energy non-linearity, non-uniformity and resolution. The detection time window was set as 500 ~ns. The contribution from $^{14}$C pile-up with higher order is negligible and not shown.} 
		\label{fig:spectrum_compare} 
	\end{figure}
    
    More neutrino experiments are under construction or being planned, many of them~\cite{JUNO-PPNP, Jinping-2016iiq, DARWIN-2020bnc, Bieger-2021sas} with good potential in \textit{pp} neutrinos detection since they are expected to have a large detector target, well radioactivity control, low detection threshold or good energy resolution. For those experiments with LS detectors of tens of kilotons, like JUNO~\cite{juno-yellowbook,JUNO-PPNP} and LENA~\cite{LENA}, etc, $^{14}$C pile-up makes the detection becomes difficult at low energy region, hence it is necessary to develop an approach for $^{14}$C pile-up discrimination and reduction, especially the discrimination of $^{14}$C double pile-up, since its event rate is much higher than the others accidental coincidences. 
     
    This paper focus on the discrimination of \textit{pp} solar neutrinos and $^{14}$C double pile-up events. As for the discrimination of others accidental coincidence with $^{14}$C multiplicity $\geq$ 3, it's an important topic in the case of poor $^{14}$C concentration, but it is not the subject of this article. The details of our work will be presented as follows: First, we build a LS detector in simulation and investigate the features of detector response for \textit{pp} neutrinos and $^{14}$C double pile-up events (Sec.~\ref{section:Simulation}). Then, we present several approaches for $^{14}$C double pile-up discrimination based on multivariate analysis and deep learning technology (Sec.~\ref{section:Discrimination methods}). In Sec.~\ref{section:Discrimination performance and discussion}, the discrimination performance will be shown and compared. Finally, a further discussion and conclusion will be provided in Sec.~\ref{section:Detector size and C14 concentration}
    
    \begin{figure}[!htb]
        \flushleft
		%\centering
		\includegraphics[width=1.\hsize]{figure/detector.pdf}
		\caption{A schematic view of the detector. Each pixel corresponds to a 20-inch PMT, and its color indicates the ID of each PMT. There are 10650 PMTs in total.} 
		\label{fig:detector} 
	\end{figure}

\section{Detector simulation}
\label{section:Simulation}

    In this work, a spherical LS detector was bulit in Monte Carlo simulation using Geant4 toolkit~\cite{GEANT4-2002zbu}, version 4.10.p02. The radius of the spherical detector is 15~m, and the LS is contained in an acrylic sphere with 10~cm thick. To simplify the simulation, a sensitive optical surface was defined for photons receiving instead of PMT simulation in details. The sensitive optical surface is a sphere outside the acrylic sphere, separated by 1~m thick water. Next, the coverage and the quantum efficiency of photosensor can be easily applied and tuned. In the simulation, the coverage rate is 65\% and it corresponds to about 10650 20-inch photomultipliers (PMTs) uniformly distributed in the sensitive optical surface. Fig.~\ref{fig:detector} is the schematic view of the detector. In the simulation, 30\% averaged quantum efficiency was used for 20-inch PMTs, with 2\% Gaussian relative spread. The LS properties were referenced from~\cite{Zhou-2015gwa, Gao-2013pua, Wurm-2010ad, Zhang-2020mqz, Ding-2015sys, Buck-2015jxa, OKeeffe-2011dex, Yu-2022god}, and comprehensive optical processes were adopted, including quenching, Rayleigh scattering, absorption and re-emission. Table~\ref{table:tab-SimParameters} summarizes the main parameters of PMTs in the simulation, including the transit time spread (TTS), quantum efficiency (QE), dark noise and the resolution of single photoelectron (spe). As a result, about 1100 photoelectrons (PEs) will be observed by 10650 PMTs for a 1 MeV electron fully deposited its kinetic energy in the center of detector, and it corresponds to about 3\% energy resolution. On the other hand, there are about 105 additional PEs will be detected which comes from PMT dark noises in 500~ns time window.  

    \begin{table}
    %\begin{table*}
	\caption{PMT parameters in the simulation.}
	\label{table:tab-SimParameters}
		\renewcommand\arraystretch{1.4}%cm
		\setlength{\tabcolsep}{7mm}
		\centering
		\begin{tabular}{c|c}
			\hline
			 Parameters & Values   \\ \hline
			 PMT Coverage & 65\%   \\ %\hline
			 PMT QE & 30\% $\pm$ 2\% (Gaussian)  \\ %\hline
			 PMT TTS & 3 $\pm$ 0.3~ns (Gaussian)  \\ %\hline
			 PMT dark rate & 20 $\pm$ 3~kHz (Gaussian) \\ %\hline
			 PMT spe resolution & 30\% $\pm$ 3\% (Gaussian)  \\ %\hline
			 Time window & 500~ns   \\ \hline
			 %LS decay length & 25~m   \\ \hline
			 \end{tabular}
	%\end{table*}
    \end{table}

     
    \begin{figure*}[!htb]
        \centering
        \subfigure[]{
            \includegraphics[width=0.45\hsize]{figure/evt_pp_J0-936.pdf}
            \label{fig:MCsample_pp}
        }
        \quad
        \subfigure[]{
            \includegraphics[width=0.45\hsize]{figure/evt_pp_J0-936-DR.pdf}
            \label{fig:MCsample_pp_DR}
        }
        \caption{The PMT hit patterns of a \textit{pp} solar neutrino event. Each pixel corresponds to a fired PMT, and its color indicates the hit time information. The location of the purple star is (-6582.21, -8972.86, 8696.34)~mm, which indicates the position where the physics event deposited its energy (159.94~keV). (a) only physics hits are included, and 172 PEs are observed for a 500~ns time window. (b) Both physics hits and PMT dark noise hits are shown, and 284 PEs are observed for a 500~ns time window, including 112 PEs from PMT dark noise.}
        \label{fig:hitpatterns_pp}
    \end{figure*}

    \begin{figure*}[!htb]
        \centering
        \subfigure[]{
            \includegraphics[width=0.45\hsize]{figure/evt_C14_J19-301.pdf}
            \label{fig:MCsample_C14_double}
        }
        \quad
        \subfigure[]{
            \includegraphics[width=0.45\hsize]{figure/evt_C14_J19-301-DR.pdf}
            \label{fig:MCsample_C14_double_DR}
        }
        \caption{The PMT hit patterns of a $^{14}$C double pile-up event. Each pixel corresponds to a fired PMT, and its color indicates the hit time information. Two purple stars indicate the positions where two $^{14}$C events deposited their energies (71.161~keV and 56.593~keV). Their locations are (-6229.32, -2139.36, 10471.7)~mm and (484.61, -3199.44, 14423.5)~mm, respectively. (a) only physics hits are included, and 173 PEs (107+66) are observed for a 500~ns time window. (b) Both physics hits and PMT dark noise hits are shown, and 273 PEs are observed for a 500~ns time window, including 100 PEs from PMT dark noise. }
        \label{fig:hitpatterns_C14_double}
    \end{figure*}

    To investigate the response features of \textit{pp} neutrinos and $^{14}$C double pile-up events, their MC samples were generated and compared. About 1 million final-state electrons from the elastic neutrino-electron scattering reaction of \textit{pp} neutrinos were uniformly simulated in the LS volume, the spectrum of scattered electrons was referenced from~\cite{Xu-2022wcq}. Since the final-state electrons from the elastic neutrino-electron scattering are similar to the emitted electrons from $^{14}$C $\beta$ decay ($^{14}$C single event), it's difficult to distinguish them in event-by-event level, so an energy cut is needed to focus on a narrow energy region. The same treatment is applied by Borexino. On the other hand, there is about 5\% energy non-linearity~\cite{Yu-2022god, DayaBay-2019fje} for electron whose kinetic energy is around 0.2~MeV in LS and energy resolution is included in the above simulation naturally. As a result, in our analysis, a 255~PEs cut was applied to the total number of photoelectrons of all PMTs by considering $\sim$156~keV end-point energy of $^{14}$C $\beta$ decay ($\sim$150~PEs) and the contribution of PMT dark noise ($\sim$105~PEs). 
    
    After the total PEs cut, a MC sample which includes 100 thousands of \textit{pp} neutrino will be used for the study of discrimination, and they are uniformly distributed in the LS. As for the generation of $^{14}$C double pile-up sample, firstly, a large dataset was produced by simulating 10 millions $^{14}$C single events in the LS via $^{14}$C $\beta$ decay. Next, two $^{14}$C single events were randomly picked up from the dataset and then merged into a double pile-up event. In the merge operation, since the lifetime of $^{14}$C is longer than 8000 years, the time interval of two $^{14}$C single events can be approximately treated as an uniform distribution in a few hundred nanoseconds. Similarly, a 255~PEs cut was applied and 100 thousands of $^{14}$C double pile-up events will be used for our analysis.

    As illustrated in Fig.~\ref{fig:hitpatterns_pp} and Fig.~\ref{fig:hitpatterns_C14_double}, \textit{pp} solar neutrinos and $^{14}$C double pile-up events show different features in their time and spatial distributions. \textit{pp} solar neutrino is a single point-like event whose energy deposition happens in a relatively short time and a small space, hence, only a cluster will be found in its PMT hit pattern. As for $^{14}$C double pile-up event, if two $^{14}$C decay at different detector positions, two clusters will be found. On the other hand, since the hit time distribution of the fired PMTs including both scintillation time, photon's time of flight and the decay time of $^{14}$C, these make the hit time distribution becomes useful for identification studies. Especially for the case that two $^{14}$C decay near each other, their spatial distribution will not be significantly different from single point-like event, but the hit time distribution may still be helpful if the time interval between two $^{14}$C decays is large. An example of this case can be found in Fig.~\ref{fig:hitpatterns_C14_double}. In Sec.~\ref{section:Discrimination methods}, event's spatial information will be extracted and used together with hit time information as input to the discrimination algorithms.
    
\section {Discrimination methods}
\label{section:Discrimination methods}

    The basic idea to develop a discrimination algorithm for \textit{pp} solar neutrinos and $^{14}$C double pile-up events is by utilizing their time and spatial information, which with different characteristics (Sec.~\ref{section:Simulation}). Similar approaches were applied in the discrimination of single-site and multi-site energy depositions in large scale liquid scintillation detectors~\cite{Dunger:2019dfo} and the fast online filtering algorithms for JUNO multi-messenger trigger. In measurement, the cluster structure will be smeared with the interference from the dark noises and the TTS of PMTs. These effects make the identification becomes more difficult and it requires an efficient approach. In this study, a multivariate analysis using the TMVA (Toolkit for Multivariate Data Analysis)~\cite{Hocker:2007ht,Speckmayer:2010zz} is performed, and the widely used algorithm BDTG (Boosted Decision Trees with Gradient boosting) is chosen and investigated. In addition, deep learning technology is also applied based on VGG network. Next, the details of discrimination methods will be presented. 
    
\subsection{TMVA analysis}
\label{subsection: TMVA analysis}

    \begin{figure*}[!htb]
        \centering
        \subfigure[]{
            \includegraphics[width=0.3\hsize]{figure/evt_pp_J0-936-hittime.pdf}
            \label{fig:hittime_pp}
        }
        \quad
        \subfigure[]{
            \includegraphics[width=0.3\hsize]{figure/evt_pp_J0-936-theta.pdf}
            \label{fig:theta_pp}
        }
        \quad
        \subfigure[]{
            \includegraphics[width=0.3\hsize]{figure/evt_pp_J0-936-phi.pdf}
            \label{fig:phi_pp}
        }
        \caption{The hit time, $\theta$ and $\phi$ distribution of a \textit{pp} solar neutrino event, which corresponds to the event in Fig.~\ref{fig:MCsample_pp_DR}. (a) hit time distribution. (b) $\theta$ distribution. (c) $\phi$ distribution. }
        \label{fig:distribution_pp}
    \end{figure*}

    \begin{figure*}[!htb]
        \centering
        \subfigure[]{
            \includegraphics[width=0.3\hsize]{figure/evt_C14_J19-301-hittime.pdf}
            \label{fig:hittime_C14_double}
        }
        \quad
        \subfigure[]{
            \includegraphics[width=0.3\hsize]{figure/evt_C14_J19-301-theta.pdf}
            \label{fig:theta_C14_double}
        }
        \quad
        \subfigure[]{
            \includegraphics[width=0.3\hsize]{figure/evt_C14_J19-301-phi.pdf}
            \label{fig:phi_C14_double}
        }
        \caption{The hit time, $\theta$ and $\phi$ distribution of a $^{14}$C double pile-up event, which corresponds to the event in Fig.~\ref{fig:MCsample_C14_double_DR}. (a) hit time distribution. (b) $\theta$ distribution. (c) $\phi$ distribution.}
        \label{fig:distribution_C14_double}
    \end{figure*}

    \begin{figure*}[!htb]
		\centering
		\subfigure[]{
		    \includegraphics[width=0.45\hsize]{figure/TMVA_var_1.pdf}
		    \label{fig:TMVA_var_1}
		}
		\quad
        \subfigure[]{
            \includegraphics[width=0.45\hsize]{figure/TMVA_var_2.pdf}
		    \label{fig:TMVA_var_2}
        }
        \quad
        \subfigure[]{
            \includegraphics[width=0.45\hsize]{figure/TMVA_var_3.pdf}		    \label{fig:TMVA_var_3}
        }
		\caption{Normalized distributions of the variables of pp solar neutrino and C14 double pileup event. } 
		\label{fig:TMVA_parameterInput_example}
	\end{figure*}

    \begin{figure*}[!htb]
        \centering
        \subfigure[]{
            \includegraphics[width=0.4\hsize]{figure/TMVA_correlation_sig.pdf}
            \label{fig:TMVA_parameterInput_correlation_signal}
        }
        \quad
        \subfigure[]{
            \includegraphics[width=0.4\hsize]{figure/TMVA_correlation_bkg.pdf}
            \label{fig:TMVA_parameterInput_correlation_bkg}
        }
        \caption{Linear correlation matrix for the input variables of \textit{pp} solar neutrinos (a) and $^{14}$C double pile-up events (b).}
        \label{fig:Correlation_matrix}
    \end{figure*}

    TMVA~\cite{Hocker:2007ht,Speckmayer:2010zz} is a powerful tool for multivariate analysis, and it has been successful applied in both the signal and background classification in accelerator physics~\cite{Lampen:2008zza}, the component identification of cosmic rays~\cite{LHAASO:2019qdu} and event reconstruction in LS detector for neutrino experiment~\cite{Qian-2021vnh}. TMVA toolkit hosts a large variety of multivariate classification algorithms, in this paper, TMVA algorithm BDTG is chosen and investigated. To extract input variables, PMT hit pattern was projected on one-dimensional (1-D) plane for hit time, and $\theta$, $\phi$ of each fired PMT in spherical coordinates, respectively. The projection results of Fig.~\ref{fig:MCsample_pp_DR} are shown in Fig.~\ref{fig:distribution_pp} and the projection results of Fig.~\ref{fig:MCsample_C14_double_DR} are shown in Fig.~\ref{fig:distribution_C14_double}. \textit{pp} solar neutrino, which is a single point-like event, only shows one cluster in its distributions, while $^{14}$C double pile-up event with two clusters. 
    
    These 1-D distributions will be used for multivariate analysis. The input variables of TMVA algorithms should be sensitive to the discrimination and contain the characteristics of \textit{pp} solar neutrinos and $^{14}$C double pile-up events. In our analysis, it was found that the hit time information dominate the performance of discrimination, so more variables are extracted from the 1-D distribution of hit time. A total of eighteen variables are used in the TMVA analysis, these variables are marked as $V^{\alpha}_{i}$, where $i = 1, 2, 3$, etc and they correspond to the extracted parameters in each 1-D distribution, and $\alpha = hittime$, $\theta$ or $\phi$, which denotes the variables are from the 1-D distribution of hit time, $\theta$ or $\phi$. Their details as follows:\\
    
    (1) $V^{hittime}_{1}$: Hit number in the first 200~ns;\\
    
    (2) $V^{hittime}_{2}$: The peak position of the highest bin in the first 200~ns;\\
    
    (3) $V^{hittime}_{3}$: The amplitude of the highest bin in the first 200~ns;\\
    
    (4) $V^{hittime}_{4}$: The ratio between the peak amplitude and peak position of the highest bin in the first 200~ns;\\
    
    (5) $V^{hittime}_{5}$: Hit number in (200, 500)~ns;\\
    
    (6) $V^{hittime}_{6}$: The peak position of the highest bin in (200, 500)~ns;\\
    
    (7) $V^{hittime}_{7}$: The amplitude of the highest bin in (200, 500)~ns;\\
    
    (8) $V^{hittime}_{8}$: The ratio between the peak amplitude and peak position of the highest bin in (200, 500)~ns;\\
    
    (9) $V^{hittime}_{9}$: The ratio between the hit number in the first 200~ns and in (200, 500)~ns;
    
    (10) $V^{hittime}_{10}$: The RMS value of the 1-D distribution of hit time;\\
    
    (11) $V^{hittime}_{11}$: The Mean value of the 1-D distribution of hit time;\\
    
    (12) $V^{hittime}_{12}$: The skewness coefficient of the 1-D distribution of hit time;\\
    
    (13) $V^{theta}_{1}$: The RMS value of the 1-D distribution of $\theta$;\\
    
    (14) $V^{theta}_{2}$: The skewness coefficient of the 1-D distribution of $\theta$;\\
    
    (15) $V^{theta}_{3}$: The kurtosis coefficient of the 1-D distribution of $\theta$;\\
    
    (16) $V^{phi}_{1}$: The RMS value of the 1-D distribution of $\phi$;\\
    
    (17) $V^{phi}_{2}$: The skewness coefficient of the 1-D distribution of $\phi$;\\
    
    (18) $V^{phi}_{3}$: The kurtosis coefficient of the 1-D distribution of $\phi$;\\
    
    Fig.~\ref{fig:TMVA_parameterInput_example} shows the normalized distributions of these input variables, and their shape differences are observed by comparing these two types of events. On the other hand, the correlations of the input variables are checked for both \textit{pp} solar neutrinos and $^{14}$C double pile-up events. As shown in Fig.~\ref{fig:Correlation_matrix}, since we have dropped several variables with strong correlation in the previous study, the correlation of the current variables is acceptable, no one greater than 95\%.

    \begin{table*}[!htb]
	\caption{Parameters used in BDTG algorithm.}
	\label{table:tab-BDTG-setting}
		\renewcommand\arraystretch{1.4}%cm
		\setlength{\tabcolsep}{7mm}
		\centering
		\begin{tabular}{c|c|c}
			\hline
			 Configuration option & Setting & Description \\ \hline
			NTrees & 1000 & Number of trees in the forest     \\ %\hline
			MaxDepth & 2 & Max depth of the decision tree allowed     \\ %\hline
			MinNodeSize & 2.5\% & Minimum percentage of training events required in a leaf node     \\ %\hline
			nCuts & 20 & Number of grid points in variable range used in finding optimal cut in node splitting     \\ %\hline
			BoostType & Grad & Boosting type for the trees in the forest     \\ \hline
			%AdaBoostBeta & 0.5 & Learning rate for AdaBoost algorithm     \\ \hline
		\end{tabular}
	\end{table*}
	
    The MC samples of \textit{pp} solar neutrinos and $^{14}$C double pile-up events are divided into two equal parts respectively, one for TMVA training and the other for validation. To improve the performance, several main parameters are tuned in BDTG algorithm, Table~\ref{table:tab-BDTG-setting} shows the settings of the parameters, the other parameters are set to their default values and aren't listed in the tables.
   
\subsection{Deep learning}
\label{subsection: Machine learning}

\iffalse%%%%%hide
    \begin{figure*}[!htb]
		\centering
		\includegraphics[width=1\hsize]{figure/VGG_structure.pdf}
		\caption{VGG structure.} 
		\label{fig:VGG_structrue} 
	\end{figure*}
\fi%%%%%%%hide
    
    Deep learning is widely used in high energy physics research, energy reconstruction, track reconstruction, particle identification and etc. And it has greatly improved the research and development of physics. In this paper, deep learning algorithm VGG convolutional neural networks are used for feature recognition of one-dimensional sequences. The extracted PMT hit patterns is projected into a one-dimensional feature series for hit time, and $\theta$, $\phi$, respectively. To extract their features, one-dimensional convolution kernel is used for the above three series, the pooling layer is used for information compression, and the fully connected layer is used for particle classification. The model structure adopts the architecture of VGG-16, including 13 convolution and pooling modules, 3 fully connected layers, and batch normalize layer and connected neural unit dropout processing.

    On the other hand, in addition to one-dimensional projection using the PMT hit patterns, we also tried two-dimensional projection methods to provide input to deep learning network, including  Mercator projection, sinusoidal projection, and the projection method based on the arrangement of PMTs~\cite{Qian-2021vnh}. However, after the application of two-dimensional projection, it is found that its performance is not improved, but decreased a little bit. Considering the hit number is very small in our analysis, we performed a detailed investigation and comparison, this result can be explained by the fact that the cluster features are much more obvious in the case of one-dimensional projection but they are very discrete in two-dimensional projection.
    
    Finally, one-dimensional projection was adopted to provide input to the above VGG network. We trained the VGG network using Adaptive momentum with a batch size of 256 examples, momentum of 0.9, iniaial learning rate 1e-2. For each 10 epochs, the learning rate decreases by a factor 10. The cross-entropy loss function is used to evaluate the accuracy of the model.

\section {Discrimination performance and discussion} 
\label{section:Discrimination performance and discussion}

\subsection{Discrimination performance of BDTG}
\label{subsection:result_BDTG}

    \begin{figure*}[!htb]
        \centering
        \subfigure[]{
            \includegraphics[width=0.4\hsize]{figure/TMVA_pro_BDTG.pdf}
            \label{fig:response_BDTG}
        }
        \quad
        \subfigure[]{
            \includegraphics[width=0.4\hsize]{figure/Significance_BDTG_1_431.pdf}
            \label{fig:efficiency_BDTG_5e-18}
        }
        \quad
        \subfigure[]{
            \includegraphics[width=0.4\hsize]{figure/BDTG_SigSB.pdf}
            \label{fig:significance_BDTG_differentC14}
        }
        \quad
        \subfigure[]{
            \includegraphics[width=0.4\hsize]{figure/BDTG_S_B_ratio.pdf}
            \label{fig:StoB_afterCut_BDTG_differentC14}
        }
        \caption{Identification performances using BDTG model. (a) Normalized response distributions of BDTG model for the signal and the background. (b) Cut efficiencies as functions of BDTG cut values. The significance (green line) was calculated using 30 days statistics of the signal and the background in the analysis region, and the $^{14}$C concentration of LS is assumed to $5 \times 10^{-18}$~g/g. (c) Significance for different assumptions of $^{14}$C concentration. (d) Signal-to-background ratio after identification in the case of different assumptions of $^{14}$C concentration, 30 days statistics were adopted.}
        \label{fig:BDTG_distribution}
    \end{figure*}
    
    \iffalse%%%%%%%%%%hide
    
    \begin{figure*}[!htb]
        \centering
        \subfigure[]{
            \includegraphics[width=0.35\hsize]{figure/response_Likelihood.PNG}
            \label{fig:response_Likelihood}
        }
        \quad
        \subfigure[]{
            \includegraphics[width=0.35\hsize]{figure/efficiency_Likelihood_5e-18.PNG}
            \label{fig:efficiency_Likelihood_5e-18}
        }
        \quad
        \subfigure[]{
            \includegraphics[width=0.35\hsize]{figure/significance_Likelihood_differentC14.PNG}
            \label{fig:significance_Likelihood_differentC14}
        }
        \quad
        \subfigure[]{
            \includegraphics[width=0.35\hsize]{figure/StoB_afterCut_Likelihood_differentC14.PNG}
            \label{fig:StoB_afterCut_Likelihood_differentC14}
        }
        \caption{Identification performances using Likelihood model. (a) Normalized response distributions of Likelihood model for the signal and the background. (b) Cut efficiencies as functions of Likelihood cut values. The significance (green line) was calculated using 30 days statistics of the signal and the background in the analysis region, and the $^{14}$C concentration of LS is assumed to $5 \times 10^{-18}$~g/g. (c) Significance for different assumptions of $^{14}$C concentration. (d) Signal-to-background ratio after identification in the case of different assumptions of $^{14}$C concentration, 30 days statistics were adopted.}
        \label{fig:Likelihood_distribution}
    \end{figure*}
    
    \fi%%%%%%hide
    
    Fig.~\ref{fig:BDTG_distribution} shows the training results of BDTG model. The network is not overtrained as the response of testing data is consistent to the training data (Fig.~\ref{fig:response_BDTG}. Basically, the signal and the background separated into two parts after training, but there are still some overlapped components, which indicates that their event features are similar so the network failed to distinguish between them. According to a detailed investigation, it was found that one of the main contributions to the failed identification comes from the stacking case of two $^{14}$C that are very close together in both time and space. To optimize the significance: $N_s/\sqrt{N_s+N_b}$ (where $N_s$ and $N_b$ are the number of signal and background after identification), we scanned the cut value on TMVA response and the corresponding efficiencies can be obtained as well. The $^{14}$C concentration of LS is assumed to $5 \times 10^{-18}$~g/g in Fig.~\ref{fig:efficiency_BDTG_5e-18}, the calculation of the significance using one day statistics in the analysis region (160-350~keV) based on the estimation in Table~\ref{table:tab-eventRate}, they are $\sim$1653 for signal and $\sim$712440 for background before the identification. For BDTG model, the significance can reach its maximum value on 10.33 after applied a cut at 0.915, and the signal efficiency and the background rejection efficiency are 51.1\% and 99.18\% in this case. As discussed in Sec.~\ref{section:Introduction}, the signal-to-background ratio of $pp$ neutrinos and $^{14}$C double pile-up events is poor in a large scale LS detector, thus a strict cut is needed to reject the most of background. In this case, 51.1\% is an acceptable value for signal efficiency, and it still corresponds to a much larger statistics of effective \textit{pp} neutrino signal per day compared to most existing experiments.
    
	In Fig.~\ref{fig:significance_BDTG_differentC14}, significance is evaluated using different assumptions of $^{14}$C concentration, while Fig.~\ref{fig:StoB_afterCut_BDTG_differentC14} shows the signal-to-background ratio after identification using BDTG model, and the calculations were based on one day statistics in the case of different $^{14}$C concentrations. As a result, BDTG model shows great performance and it can handle the most of $^{14}$C double pile-up events effectively.
	
	In addition, others TMVA algorithms are also investigated, including Likelihood algorithm and several BDT models (BDT, BDTB, BDTD, BDTF). Many of them show similar performance which indicates good robustness and stability of our analysis. 

\subsection{Discrimination performance of VGG}
\label{subsection:result_VGG}

    Fig.~\ref{fig:VGG_distribution} shows the training results of VGG network. The network is not overtrained as the response of testing data is consistent to the training data (Fig.~\ref{fig:response_VGG}. To optimize the significance, we scanned the cut value on VGG output and the corresponding efficiencies can be obtained as well. The $^{14}$C concentration of LS is assumed to $5 \times 10^{-18}$~g/g in Fig.~\ref{fig:efficiency_VGG_5e-18}, the calculation of the significance using one day statistics in the analysis region (160-350~keV) based on the estimation in Table~\ref{table:tab-eventRate}, they are $\sim$1653 for signal and $\sim$712440 for background before the identification. For VGG network, the significance can reach its maximum value on 15.55 after applied a cut at 0.975, and the signal efficiency and the background rejection efficiency are 42.7\% and 99.81\% in this case. As a result, deep learning method based on VGG network has better performance compared to BDT model.
    
	In Fig.~\ref{fig:significance_VGG_differentC14}, significance is evaluated using different assumptions of $^{14}$C concentration, while Fig.~\ref{fig:StoB_afterCut_VGG_differentC14} shows the signal-to-background ratio after identification using VGG network, and the calculations were based on one day statistics in the case of different $^{14}$C concentrations. As a result, VGG network shows great performance and it can achieve a higher significance compared to the BDTG model, and the signal-to-background ratio has a good improvement.

    \begin{figure*}[!htb]
        \centering
        \subfigure[]{
            \includegraphics[width=0.4\hsize]{figure/G41Dtrain_test20w.pdf}
            \label{fig:response_VGG}
        }
        \quad
        \subfigure[]{
            \includegraphics[width=0.4\hsize]{figure/G4DataSet1D20w_PredictV1_431.pdf}
            \label{fig:efficiency_VGG_5e-18}
        }
        \quad
        \subfigure[]{
            \includegraphics[width=0.4\hsize]{figure/G4DataSet1D20w_PredictV1_SigSB.pdf}
            \label{fig:significance_VGG_differentC14}
        }
        \quad
        \subfigure[]{
            \includegraphics[width=0.4\hsize]{figure/S_B_ratio_Update.pdf}
            \label{fig:StoB_afterCut_VGG_differentC14}
        }
        \caption{Identification performances using VGG model. (a) Normalized response distributions of VGG model for the signal and the background. (b) Cut efficiencies as functions of VGG cut values. The significance (green line) was calculated using one day statistics of the signal and the background in the analysis region, and the $^{14}$C concentration of LS is assumed to $5 \times 10^{-18}$~g/g. (c) Significance for different assumptions of $^{14}$C concentration. (d) Signal-to-background ratio after identification in the case of different assumptions of $^{14}$C concentration, one day statistics were adopted.}
        \label{fig:VGG_distribution}
    \end{figure*}

\iffalse%%hide    
\section {Detector size and C14 concentration [statistics]} 
\label{section:Detector size and C14 concentration}

    \textcolor{red}{(1) Change another detector size and repeat the above analysis, show the s/sqrt(s+b) plot for BDTG and VGG model (the best one in TMVA and machine learning) in the case of different detector size (and 4 C14 concentrations ?); }
    
    \textcolor{red}{(2) Discuss the influence of detector size and C14 concentrations in pp solar neutrino detection, actually, we need to find a balance between their statistics; }
\fi%%%%%%%hide

     
	
\section{Summary and discussion}
\label{section:Summary}
	
    In this paper, we investigate the discrimination of \textit{pp} solar neutrinos and $^{14}$C double pile-up events in a large scale LS detector using both multivariate analysis and deep learning technology. For the discrimination based on VGG network, signal significance can achieve 15.6 using only one day statistics. And the signal efficiency is 42.7\% when rejecting 99.81\% $^{14}$C double pile-up events. As for BDTG model, signal significance can achieve 10.3 using only one day statistics, and the signal efficiency is 51.1\% when rejecting 99.18\% $^{14}$C double pile-up events.

\begin{thebibliography}{99}
\bibitem{Bethe-1938} 
H.A.~Bethe and C.L.~Critchfield, The formation of deuterium by proton combination. Phys. Rev. \textbf{54}, 248-254 (1938). \href{https://doi.org/10.1103/PhysRev.54.248}{https://doi.org/10.1103/PhysRev.54.248}

\bibitem{Bethe-1939} 
H.A.~Bethe, Energy production in stars. Phys. Rev. 55, 434–456 (1939). \href{https://doi.org/10.1103/PhysRev.55.434}{https://doi.org/10.1103/PhysRev.55.434}

\bibitem{Bahcall-1996pt}
J.N.~Bahcall, M.~Fukugita and P.I.~Krastev, How does the Sun shine? Phys. Lett. B \textbf{374}, 1-6 (1996). \href{https://doi.org/10.1016/0370-2693(96)00187-6}{https://doi.org/10.1016/0370-2693(96)00187-6}
%[arXiv:astro-ph/9602065 [astro-ph]].

\bibitem{Davis-1968cp}
R.~Davis, Jr., D.S.~Harmer \textit{et al.}, Search for neutrinos from the sun. Phys. Rev. Lett. \textbf{20}, 1205-1209 (1968). \href{https://doi.org/10.1103/PhysRevLett.20.1205}{https://doi.org/10.1103/PhysRevLett.20.1205}

\bibitem{Cleveland-1998nv}
B.T.~Cleveland, T.~Daily, R.~Davis \textit{et al.}, Measurement of the solar electron neutrino flux with the Homestake chlorine detector. Astrophys. J. \textbf{496}, 505-526 (1998). \href{https://doi.org/10.1086/305343}{https://doi.org/10.1086/305343}

\bibitem{GALLEX-1992gcp}
P.~Anselmann, W.~Hampel, G.~Heusser \textit{et al.}, Solar neutrinos observed by GALLEX at Gran Sasso. Phys. Lett. B \textbf{285}, 376-389 (1992). \href{https://doi.org/10.1016/0370-2693(92)91521-A}{https://doi.org/10.1016/0370-2693(92)91521-A}

\bibitem{GALLEX-1998kcz}
W.~Hampel, J.~Handt, G.~Heusser \textit{et al.}, GALLEX solar neutrino observations: Results for GALLEX IV. Phys. Lett. B \textbf{447}, 127-133 (1999). \href{https://doi.org/10.1016/S0370-2693(98)01579-2}{https://doi.org/10.1016/S0370-2693(98)01579-2}

\bibitem{Kaether-2010ag}
F.~Kaether, W.~Hampel, G.~Heusser \textit{et al.}, Reanalysis of the GALLEX solar neutrino flux and source experiments. Phys. Lett. B \textbf{685}, 47-54 (2010). \href{https://doi.org/10.1016/j.physletb.2010.01.030}{https://doi.org/10.1016/j.physletb.2010.01.030}
%[arXiv:1001.2731 [hep-ex]].

\bibitem{GNO-2005bds}
M.~Altmann, M.~Balata, P.~Belli \textit{et al.}, Complete results for five years of GNO solar neutrino observations. Phys. Lett. B \textbf{616}, 174-190 (2005). \href{https://doi.org/10.1016/j.physletb.2005.04.068}{https://doi.org/10.1016/j.physletb.2005.04.068}
%[arXiv:hep-ex/0504037 [hep-ex]].

\bibitem{SAGE-2009eeu}
J.N.~Abdurashitov, V.N.~Gavrin, V.V.~Gorbachev \textit{et al.}, Measurement of the solar neutrino capture rate with gallium metal. III: Results for the 2002--2007 data-taking period. Phys. Rev. C \textbf{80}, 015807 (2009). \href{https://doi.org/10.1103/PhysRevC.80.015807}{https://doi.org/10.1103/PhysRevC.80.015807}
%[arXiv:0901.2200 [nucl-ex]].

\bibitem{Gavrin-2019sok}
V.~N.~Gavrin, The history, present and future of SAGE (Soviet-American Gallium Experiment). \href{https://doi.org/10.1142/9789811204296\_0002}{https://doi.org/10.1142/9789811204296\_0002}

\bibitem{Kamiokande-II-1989hkh}
K.S.~Hirata, T.~Kajita, T. Kifune \textit{et al.}, Observation of B-8 Solar Neutrinos in the Kamiokande-II Detector. Phys. Rev. Lett. \textbf{63}, 16 (1989). \href{https://doi.org/10.1103/PhysRevLett.63.16}{https://doi.org/10.1103/PhysRevLett.63.16}

\bibitem{Kamiokande-1996qmi}
Y.~Fukuda, T.~Hayakawa, K. Inoue \textit{et al.}, Solar neutrino data covering solar cycle 22, Phys. Rev. Lett. \textbf{77}, 1683-1686 (1996). \href{https://doi.org/10.1103/PhysRevLett.77.1683}{https://doi.org/10.1103/PhysRevLett.77.1683}

\bibitem{Wolfenstein-1977ue}
L.~Wolfenstein, Neutrino Oscillations in Matter. Phys. Rev. D \textbf{17}, 2369-2374 (1978). \href{https://doi.org/10.1103/PhysRevD.17.2369}{https://doi.org/10.1103/PhysRevD.17.2369}

\bibitem{Mikheyev-1985zog}
S.P.~Mikheyev and A.Y.~Smirnov, Resonance Amplification of Oscillations in Matter and Spectroscopy of Solar Neutrinos. Sov. J. Nucl. Phys. \textbf{42}, 913-917 (1985). 

\bibitem{SNO-2001kpb}
Q.R.~Ahmad, R.C.~Allen, J.D.~Anglin \textit{et al.}, Measurement of the rate of $\nu_e+d \to p+p+e^-$ interactions produced by $^8$B solar neutrinos at the Sudbury Neutrino Observatory. Phys. Rev. Lett. \textbf{87}, 071301 (2001). \href{https://doi.org/10.1103/PhysRevLett.87.071301}{https://doi.org/10.1103/PhysRevLett.87.071301}

\bibitem{SNO-2003bmh}
S.N.~Ahmed, A.E.~Anthony, E.W.~Beier \textit{et al.}, Measurement of the total active B-8 solar neutrino flux at the Sudbury Neutrino Observatory with enhanced neutral current sensitivity. Phys. Rev. Lett. \textbf{92}, 181301 (2004). \href{https://doi.org/10.1103/PhysRevLett.92.181301}{https://doi.org/10.1103/PhysRevLett.92.181301}
%[arXiv:nucl-ex/0309004 [nucl-ex]].

\bibitem{KamLAND-2002uet}
K.~Eguchi, S.~Enomoto, K.~Furuno \textit{et al.}, First results from KamLAND: Evidence for reactor anti-neutrino disappearance. Phys. Rev. Lett. \textbf{90}, 021802 (2003). \href{https://doi.org/10.1103/PhysRevLett.90.021802}{https://doi.org/10.1103/PhysRevLett.90.021802}
%[arXiv:hep-ex/0212021 [hep-ex]].

\bibitem{Bahcall-1995bt}
J.N.~Bahcall and M.H.~Pinsonneault, Solar models with helium and heavy element diffusion. Rev. Mod. Phys. \textbf{67}, 781-808 (1995). \href{https://doi.org/10.1103/RevModPhys.67.781}{https://doi.org/10.1103/RevModPhys.67.781}
%[arXiv:hep-ph/9505425 [hep-ph]].

\bibitem{Christensen-Dalsgaard-1996hpz}
J.~Christensen-Dalsgaard, W.~Dappen, S.V.~Ajukov \textit{et al.}, The current state of solar modeling. Science \textbf{272}, 1286-1292 (1996). \href{https://doi.org/10.1126/science.272.5266.1286}{https://doi.org/10.1126/science.272.5266.1286}

\bibitem{DeglInnocenti-1996uex}
S.~Degl'Innocenti, W.A.~Dziembowski, G.~Fiorentini \textit{et al.}, Helioseismology and standard solar models. Astropart. Phys. \textbf{7}, 77-95 (1997). \href{https://doi.org/10.1016/S0927-6505(97)00004-2}{https://doi.org/10.1016/S0927-6505(97)00004-2}
%[arXiv:astro-ph/9612053 [astro-ph]].

\bibitem{Brun-1999dw}
A.S.~Brun, S.~Turck-Chieze and J.~P.~Zahn, Standard solar models in the light of new helioseismic constraints. 2. mixing below the convective zone. Astrophys. J. \textbf{525}, 1032-1041 (1999). \href{https://doi.org/10.1086/307932}{https://doi.org/10.1086/307932}
%[arXiv:astro-ph/9906382 [astro-ph]].

\bibitem{Bahcall-2001pf}
J.N.~Bahcall, The Luminosity constraint on solar neutrino fluxes. Phys. Rev. C \textbf{65}, 025801 (2002). \href{https://doi.org/10.1103/PhysRevC.65.025801}{https://doi.org/10.1103/PhysRevC.65.025801}
%[arXiv:hep-ph/0108148 [hep-ph]].

\bibitem{Serenelli-2009yc}
A.~Serenelli, S.~Basu, J.W.~Ferguson \textit{et al.}, New Solar Composition: The Problem With Solar Models Revisited. Astrophys. J. Lett. \textbf{705}, L123-L127 (2009). \href{https://doi.org/10.1088/0004-637X/705/2/L123}{https://doi.org/10.1088/0004-637X/705/2/L123}
%[arXiv:0909.2668 [astro-ph.SR]].

\bibitem{Borexino-2007kvk}
C.~Arpesella, G.~Bellini, J.~Benziger \textit{et al.}, First real time detection of Be-7 solar neutrinos by Borexino. Phys. Lett. B \textbf{658}, 101-108 (2008). \href{https://doi.org/10.1016/j.physletb.2007.09.054}{https://doi.org/10.1016/j.physletb.2007.09.054}
%[arXiv:0708.2251 [astro-ph]].

\bibitem{BOREXINO-2014pcl}
G.~Bellini, J.~Benziger, D.~Bick \textit{et al.}, Neutrinos from the primary proton\textendash{}proton fusion process in the Sun. Nature \textbf{512}, no.7515, 383-386 (2014). \href{https://doi.org/10.1038/nature13702}\href{https://doi.org/10.1038/nature13702}

\bibitem{BOREXINO-2018ohr}
M.~Agostini, K.~Altenmüller, S.~Appel \textit{et al.}, Comprehensive measurement of $pp$-chain solar neutrinos. Nature \textbf{562}, no.7728, 505-510 (2018). \href{https://doi.org/10.1038/s41586-018-0624-y}{https://doi.org/10.1038/s41586-018-0624-y}

\bibitem{BOREXINO-2020aww}
M.~Agostini, K.~Altenmüller, S.~Appel \textit{et al.}, Experimental evidence of neutrinos produced in the CNO fusion cycle in the Sun. Nature \textbf{587}, 577-582 (2020). \href{https://doi.org/10.1038/s41586-020-2934-0}{https://doi.org/10.1038/s41586-020-2934-0}
%[arXiv:2006.15115 [hep-ex]].

%\bibitem{Bahcall-2005va}
%J.N.~Bahcall, A.M.~Serenelli and S.~Basu, 10,000 standard solar models: a Monte Carlo simulation. Astrophys. J. Suppl. \textbf{165}, 400-431 (2006). \href{https://doi.org/10.1086/504043}{https://doi.org/10.1086/504043}
%[arXiv:astro-ph/0511337 [astro-ph]].

%\bibitem{Vinyoles-2016djt}
%N.~Vinyoles, A.M.~Serenelli, F.L.~Villante \textit{et al.}, A new Generation of Standard Solar Models. Astrophys. J. \textbf{835}, no.2, 202 (2017). \href{https://doi.org/10.3847/1538-4357/835/2/202}{https://doi.org/10.3847/1538-4357/835/2/202}
%[arXiv:1611.09867 [astro-ph.SR]].

\bibitem{Gann-2021ndb}
G.D.O.~Gann, K.~Zuber, D.~Bemmerer \textit{et al.}, The Future of Solar Neutrinos. Ann. Rev. Nucl. Part. Sci. \textbf{71}, 491-528 (2021). \href{https://doi.org/10.1146/annurev-nucl-011921-061243}{https://doi.org/10.1146/annurev-nucl-011921-061243}
%[arXiv:2107.08613 [hep-ph]].

\bibitem{Xu-2022wcq}
X.J.~Xu, Z.~Wang and S.~Chen, Solar neutrino physics. \href{
https://doi.org/10.48550/arXiv.2209.14832}{https://doi.org/10.48550/arXiv.2209.14832}
%[arXiv:2209.14832 [hep-ph]].

\bibitem{JUNO-PPNP}
A. Abusleme \textit{et al.}, JUNO Physics and Detector. Prog. Part. Nucl. Phys. \textbf{123}, 103927 (2022). \href{https://doi.org/10.1016/j.ppnp.2021.103927}{https://doi.org/10.1016/j.ppnp.2021.103927}
%[arXiv:2104.02565 [hep-ex]].

\bibitem{Jinping-2016iiq}
J.F.~Beacom, S.M.~Chen, J.P.~Cheng \textit{et al.}, Physics prospects of the Jinping neutrino experiment. Chin. Phys. C \textbf{41}, no.2, 023002 (2017). \href{https://doi.org/10.1088/1674-1137/41/2/023002}{https://doi.org/10.1088/1674-1137/41/2/023002}
%[arXiv:1602.01733 [physics.ins-det]].

\bibitem{DARWIN-2020bnc}
J.~Aalbers, F.~Agostini, S.E.M.~Ahmed Maouloud \textit{et al.}, Solar neutrino detection sensitivity in DARWIN via electron scattering. Eur. Phys. J. C \textbf{80}, no.12, 1133 (2020). \href{https://doi.org/10.1140/epjc/s10052-020-08602-7}{https://doi.org/10.1140/epjc/s10052-020-08602-7}
%[arXiv:2006.03114 [physics.ins-det]].

\bibitem{Bieger-2021sas}
L.~Bieger, T.~Birkenfeld, D.~Blum \textit{et al.}, Potential for a precision measurement of solar pp neutrinos in the Serappis experiment. Eur. Phys. J. C \textbf{82}, no.9, 779 (2022). \href{https://doi.org/10.1140/epjc/s10052-022-10725-y}{https://doi.org/10.1140/epjc/s10052-022-10725-y}
%[arXiv:2109.10782 [physics.ins-det]].

\bibitem{juno-yellowbook}
F.P. An, G.P An, Q. An \textit{et al.}, Neutrino Physics with JUNO. J. Phys. G \textbf{43}, no.3, 030401 (2016). \href{https://doi.org/10.1088/0954-3899/43/3/030401}{https://doi.org/10.1088/0954-3899/43/3/030401}
%[arXiv:1507.05613 [physics.ins-det]].

\bibitem{LENA}
M. Wurm, Studying neutrino properties in the future LENA experiment. Nucl. Phys. B Proc. Suppl. \textbf{237-238}, 314-316 (2013). \href{https://doi.org/10.1016/j.nuclphysbps.2013.04.114}{https://doi.org/10.1016/j.nuclphysbps.2013.04.114}

\bibitem{GEANT4-2002zbu}
S.~Agostinelli, J.~Allison, K.~Amako \textit{et al.}, GEANT4--a simulation toolkit. Nucl. Instrum. Meth. A \textbf{506}, 250-303 (2003). \href{https://doi.org/10.1016/S0168-9002(03)01368-8}{https://doi.org/10.1016/S0168-9002(03)01368-8}

\bibitem{Zhou-2015gwa}
X.~Zhou, Q.~Liu, M.~Wurm \textit{et al.}, Rayleigh scattering of linear alkylbenzene in large liquid scintillator detectors. Rev. Sci. Instrum. \textbf{86}, no.7, 073310 (2015). \href{https://doi.org/10.1063/1.4927458}{https://doi.org/10.1063/1.4927458}
%[arXiv:1504.00987 [physics.ins-det]].

\bibitem{Gao-2013pua}
L.~Gao, B.x.~Yu, Y.y.~Ding \textit{et al.}, Attenuation length measurements of a liquid scintillator with LabVIEW and reliability evaluation of the device. Chin. Phys. C \textbf{37}, 076001 (2013). \href{https://doi.org/10.1088/1674-1137/37/7/076001}{https://doi.org/10.1088/1674-1137/37/7/076001}
%[arXiv:1305.1471 [physics.ins-det]].

\bibitem{Wurm-2010ad}
M.~Wurm, F.von Feilitzsch, M.~Goeger-Neff \textit{et al.}, Optical Scattering Lengths in Large Liquid-Scintillator Neutrino Detectors. Rev. Sci. Instrum. \textbf{81}, 053301 (2010). \href{https://doi.org/10.1063/1.3397322}{https://doi.org/10.1063/1.3397322}
%[arXiv:1004.0811 [physics.ins-det]].

\bibitem{Zhang-2020mqz}
Y.~Zhang, Z.Y.~Yu, X.Y.~Li \textit{et al.}, A complete optical model for liquid-scintillator detectors. Nucl. Instrum. Meth. A \textbf{967}, 163860 (2020). \href{https://doi.org/10.1016/j.nima.2020.163860}{https://doi.org/10.1016/j.nima.2020.163860}
%[arXiv:2003.12212 [physics.ins-det]].

\bibitem{Ding-2015sys}
X.F.~Ding, L.J.~Wen, X.~Zhou \textit{et al.}, Measurement of the fluorescence quantum yield of bis-MSB. Chin. Phys. C \textbf{39}, no.12, 126001 (2015). \href{https://doi.org/10.1088/1674-1137/39/12/126001}{https://doi.org/10.1088/1674-1137/39/12/126001}
%[arXiv:1506.00240 [physics.ins-det]].

\bibitem{Buck-2015jxa}
C.~Buck, B.~Gramlich and S.~Wagner, Light propagation and fluorescence quantum yields in liquid scintillators. JINST \textbf{10}, no.09, P09007 (2015). \href{https://doi.org/10.1088/1748-0221/10/09/P09007}{https://doi.org/10.1088/1748-0221/10/09/P09007}
%[arXiv:1509.02327 [physics.ins-det]].

\bibitem{OKeeffe-2011dex}
H.M.~O'Keeffe, E.~O'Sullivan and M.C.~Chen, Scintillation decay time and pulse shape discrimination in oxygenated and deoxygenated solutions of linear alkylbenzene for the SNO+ experiment. Nucl. Instrum. Meth. A \textbf{640}, 119-122 (2011). \href{https://doi.org/10.1016/j.nima.2011.03.027}{https://doi.org/10.1016/j.nima.2011.03.027}
%[arXiv:1102.0797 [physics.ins-det]].

\bibitem{Yu-2022god}
M.~Yu, L.~Wen, X.~Zhou \textit{et al.}, Determine Energy Nonlinearity and Resolution of $e^{\pm}$ and $\gamma$ in Liquid Scintillator Detectors by A Universal Energy Response Model. \href{https://doi.org/10.48550/arXiv.2211.02467}{https://doi.org/10.48550/arXiv.2211.02467}

\bibitem{DayaBay-2019fje}
D.~Adey, A.B.~Balantekin, M.~Bishai \textit{et al.}, A high precision calibration of the nonlinear energy response at Daya Bay. Nucl. Instrum. Meth. A \textbf{940}, 230-242 (2019). \href{https://doi.org/10.1016/j.nima.2019.06.031}{https://doi.org/10.1016/j.nima.2019.06.031}
%[arXiv:1902.08241 [physics.ins-det]].

\bibitem{Dunger:2019dfo}
J. Dunger and S.D. Biller, Multi-site Event Discrimination in Large Liquid Scintillation Detectors, Nucl. Instrum. Meth. A \textbf{943}, 162420 (2019). \href{https://doi.org/10.1016/j.nima.2019.162420}{https://doi.org/10.1016/j.nima.2019.162420}
%[arXiv:1904.00440 [physics.ins-det]].

\bibitem{Hocker:2007ht}
A. Hocker, P. Speckmayer, J. Stelzer \textit{et al.}, TMVA - Toolkit for Multivariate Data Analysis, \href{https://doi.org/10.48550/arXiv.physics/0703039}{https://doi.org/10.48550/arXiv.physics/0703039} 

\bibitem{Speckmayer:2010zz}
P. Speckmayer, A. Hocker, J. Stelzer \textit{et al.}, The toolkit for multivariate data analysis, TMVA 4, J. Phys. Conf. Ser. \textbf{219}, 032057 (2010). \href{https://doi.org/10.1088/1742-6596/219/3/032057}{https://doi.org/10.1088/1742-6596/219/3/032057}

\bibitem{Lampen:2008zza}
T. Lampen, F. Garcia, A. Heikkinen \textit{et al.}, Testing TMVA software in b-tagging for the search of MSSM Higgs bosons at the LHC, J. Phys. Conf. Ser. \textbf{119}, 032028 (2008). \href{https://doi.org/10.1088/1742-6596/119/3/032028}{https://doi.org/10.1088/1742-6596/119/3/032028}

\bibitem{LHAASO:2019qdu}
L.Q. Yin, S.S. Zhang, Z. Cao \textit{et al.} [LHAASO], Expected energy spectrum of cosmic ray protons and helium below 4 PeV measured by LHAASO, Chin. Phys. C \textbf{43}, no.7, 075001 (2019). \href{https://doi.org/10.1088/1674-1137/43/7/075001}{https://doi.org/10.1088/1674-1137/43/7/075001}

\bibitem{Qian-2021vnh}
Z.~Qian, V.~Belavin, V.~Bokov \textit{et al.}, Vertex and energy reconstruction in JUNO with machine learning methods. Nucl. Instrum. Meth. A \textbf{1010}, 165527 (2021). \href{https://doi.org/10.1016/j.nima.2021.165527}{https://doi.org/10.1016/j.nima.2021.165527}
%[arXiv:2101.04839 [physics.ins-det]].


\end{thebibliography}



\end{document}