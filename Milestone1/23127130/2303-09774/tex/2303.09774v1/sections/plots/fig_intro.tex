\begin{figure}[t]
\begin{subfigure}[b]{\linewidth}
\centering
\begin{tikzpicture}
\begin{scope}[every node/.style={rectangle,thick, minimum width = 1cm, minimum height = 0.5cm,fill opacity = 0.6,text opacity=1}]
    \node[text width=1.5cm, align=center, fill=PrestoColor,draw=Greyborder] (X) at (-0.4,0) {\scriptsize Base Tables};
    \node[fill=ExampleColor2,draw=Redborder] (A1) at (-0.4, 1.4) {\scriptsize MV A};
    \node[fill=BaseColor,draw=Blueborder] (B1) at (-1,2.8) {\scriptsize MV B};
    \node[fill=ExampleColor1,draw=Greenborder] (C1) at (0.2,2.8) {\scriptsize MV C} ;
\end{scope}

\begin{scope}[every node/.style={rectangle,thick, rounded corners=0.1cm,fill opacity = 0.6,text opacity=1}]
    \node[fill=ExampleColor2,draw=Redborder] (A2) at (-0.4, 0.7) {\scriptsize SQL A};
    \node[fill=BaseColor,draw=Blueborder] (B2) at (-1,2.1) {\scriptsize SQL B};
    \node[fill=ExampleColor1,draw=Greenborder] (C2) at (0.2,2.1) {\scriptsize SQL C} ;
\end{scope}

\begin{scope}[>={Stealth[black]},
              every node/.style={fill=none,circle},
              every edge/.style={draw=black}]
    \path [->] (X) edge node {} (A2);
    \path [->] (A2) edge node {} (A1);
    \path [->] (A1) edge node {} (B2);
    \path [->] (A1) edge node {} (C2);
    \path [->] (B2) edge node {} (B1);
    \path [->] (C2) edge node {} (C1);
\end{scope}

% labels
\begin{scope}[every node/.style={rectangle,thick, anchor=east}]
    \node[align=right] (label1) at (2.6, 3.0) {\small \textbf{Traditional}};
    \node[align=right] (label2) at (1.5, 2.3) {\scriptsize Exec};
    \node[align=right] (label3) at (1.5, 1.8) {\scriptsize I/O} ;
    \node[align=right] (label4) at (2.5, 1.2) {\small \textbf{Ours (\system)}};
    \node[align=right] (label5) at (1.5, 0.6) {\scriptsize Exec};
    \node[align=right] (label6) at (1.5, 0.1) {\scriptsize I/O} ;
    \node[align=right] (label7) at (7, 1.5) {\scriptsize Time};
    \node[align=right] (label8) at (7, -0.3) {\scriptsize Time} ;
\end{scope}

% plot lines
\begin{scope}[>={Stealth[black]},
              every node/.style={fill=none,circle},
              every edge/.style={draw=black}]
    \draw (1.5,2.8) -- (1.5, 1.5);
    \draw (1.5,1.0) -- (1.5, -0.3);
    \draw[->] (1.5,1.5) -- (6.3, 1.5);
    \draw[->] (1.5,-0.3) -- (6.3, -0.3);
\end{scope}

% traditional
\begin{scope}[every node/.style={rectangle, anchor=west, inner sep=0.05cm,fill opacity = 0.6,text opacity=1}]
    \node[fill=ExampleColor2,draw=Redborder] (X) at (1.5,1.8) {\tiny R(data)};
    \node[fill=ExampleColor2,draw=Redborder] (A1) at (2.25, 2.3) {\tiny SQL A};
    \node[fill=ExampleColor2,draw=Redborder] (B1) at (2.9, 1.8) {\tiny W(A)};
    \node[fill=BaseColor,draw=Blueborder] (C1) at (3.45, 1.8) {\tiny R(A)};
    \node[fill=BaseColor,draw=Blueborder] (A1) at (4, 2.3) {\tiny SQL B};
    \node[fill=BaseColor,draw=Blueborder] (B1) at (4.65, 1.8) {\tiny W(B)};
    \node[fill=ExampleColor1,draw=Greenborder] (C1) at (5.2, 1.8) {\tiny R(A)};
    \node[fill=ExampleColor1,draw=Greenborder] (A1) at (5.75, 2.3) {\tiny SQL C};
    \node[fill=ExampleColor1,draw=Greenborder] (C1) at (6.4, 1.8) {\tiny W(C)};
\end{scope}

% traditional label
\begin{scope}[>={Stealth[black]},
              every node/.style={fill=none,circle},
              every edge/.style={draw=black}]
    \draw [decorate,
    decoration = {brace}] (2.9, 2.5) --  (4, 2.5);
    \draw [decorate,
    decoration = {brace}] (4.65, 2.5) --  (5.75, 2.5);
\end{scope}
\node[text width=3cm, align=center] (X) at (4.6,2.9) {\scriptsize Overhead we can address with bounded memory \par};

% ours
\begin{scope}[every node/.style={rectangle, anchor=west, inner sep=0.05cm,fill opacity = 0.6,text opacity=1}]
    \node[fill=ExampleColor2,draw=Redborder] (X) at (1.5,0.0) {\tiny R(data)};
    \node[fill=ExampleColor2,draw=Redborder] (A1) at (2.25, 0.5) {\tiny SQL A};
    \node[fill=ExampleColor2,draw=Redborder] (B1) at (2.9, 0.0) {\tiny W(A)};
    \node[fill=BaseColor,draw=Blueborder] (A1) at (2.9, 0.5) {\tiny SQL B};
    \node[fill=BaseColor,draw=Blueborder] (B1) at (3.55, 0.0) {\tiny W(B)};
    \node[fill=ExampleColor1,draw=Greenborder] (A1) at (3.55, 0.5) {\tiny SQL C};
    \node[fill=ExampleColor1,draw=Greenborder] (C1) at (4.2, 0.0) {\tiny W(C)};
\end{scope}

% ours label
\begin{scope}[>={Stealth[black]},
              every node/.style={fill=none,circle},
              every edge/.style={draw=black}]
    \node[] (y) at (5.2, 0.9) {};
    \node[] (z) at (2.55, 0.7) {};
    \draw[->] (y)  to [bend right = 20] (z);
\end{scope}
\node[text width=2.1cm, align=center] (X) at (6.0,0.5) 
{\scriptsize MV A is kept in memory, eliminating overhead\par};

\end{tikzpicture}
\end{subfigure}
\vspace{-6mm}
\caption{Our intuition:
we can reduce an overall MV refresh time by exploiting existing dependency relationships.
If we know SQL B will depend on the output of SQL A,
    we don't have to wait until the result of SQL A is fully materialized.}
\vspace{-4mm}
% Our proposed problem (\system) addresses the short-circuiting of reads (R) and writes (W) of intermediate tables during MV refresh 
% by using a limited amount of memory .
% through persisting data in spare memory.
\label{fig:intro}
\end{figure}