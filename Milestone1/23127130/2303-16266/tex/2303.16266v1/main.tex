% MAX 7 pages + references

\documentclass{article}
\pdfpagewidth=8.5in
\pdfpageheight=11in
% The file ijcai22.sty is NOT the same as previous years'
\usepackage{ijcai23}

%
\usepackage{times}
\usepackage{amsmath,amssymb,amsfonts}
\usepackage{algorithmic}
\usepackage{algorithm}
\usepackage{array}
\usepackage[caption=false,font=normalsize,labelfont=sf,textfont=sf]{subfig}
\usepackage{textcomp}
\usepackage{stfloats}
\usepackage{url}
\usepackage{verbatim}
\usepackage{graphicx}
\usepackage{multicol}
\usepackage{multirow}
\hyphenation{op-tical net-works semi-conduc-tor IEEE-Xplore}
\usepackage{natbib}
\usepackage[normalem]{ulem}
\definecolor{purple}{rgb}{1, 0, 1}

\newcommand{\ie}{\emph{i.e.,}\xspace}
\newcommand{\eg}{\emph{e.g.,}\xspace}
\newcommand{\abr}{\emph{abbr.}\xspace}
\newcommand{\ea}{\emph{et al.}\xspace}
\newcommand{\gensync}{\emph{GenSync}\xspace}
\newcommand{\colosseum}{\emph{Colosseum}\xspace}
\newcommand{\srep}{\emph{SREP}\xspace} % Set Reconciliation Enhances
\newcommand{\srepsim}{\emph{SREPSim}\xspace}
% Propagation
\newcommand{\esrep}{\emph{E-SREP}\xspace}
\newcommand{\epsrep}{\emph{EP-SREP}\xspace}
\newcommand{\mesrep}{\emph{ME-SREP}\xspace}
\newcommand{\mempoolsync}{\emph{MempoolSync}}

\newcommand{\fref}[1]{Fig.~\ref{#1}}
\newcommand{\tref}[1]{Table~\ref{#1}}
\newcommand{\aref}[1]{Algorithm~\ref{#1}}
\newcommand{\procref}[1]{Procedure~\ref{#1}}
\newcommand{\sref}[1]{Section~\ref{#1}}
\newcommand{\lineref}[1]{line~\ref{#1}}
\newcommand{\appref}[1]{Appendix~\ref{#1}}

% Change \eqref
\LetLtxMacro{\originaleqref}{\eqref}
\renewcommand{\eqref}{Eq.~\originaleqref}

% Theorems and corollaries
\newcounter{theoremcount}
\setcounter{theoremcount}{0}
\DeclareRobustCommand{\theorem}[1]{%
  \refstepcounter{theoremcount}%
  \noindent\textit{\textbf{Theorem \thetheoremcount\label{theorem:#1}: }}%
}
\DeclareRobustCommand{\theoremref}[1]{Theorem~\ref{theorem:#1}}

\DeclareRobustCommand{\proof}{\emph{Proof:}\xspace}
\DeclareRobustCommand{\qqed}{\hfill$\blacksquare$}

\newcounter{corollcount}
\setcounter{corollcount}{0}
\DeclareRobustCommand{\coroll}[1]{%
  \refstepcounter{corollcount}%
  \noindent\textit{\textbf{Corollary \thecorollcount\label{coroll:#1}: }}%
}
\DeclareRobustCommand{\corollref}[1]{Corollary~\ref{coroll:#1}}

\newcounter{lemmacount}
\setcounter{lemmacount}{0}
\DeclareRobustCommand{\lemma}[1]{%
  \refstepcounter{lemmacount}%
  \noindent\textit{\textbf{Lemma \thelemmacount\label{lemma:#1}: }}%
}
\DeclareRobustCommand{\lemmaref}[1]{Lemma~\ref{lemma:#1}}

\newcounter{definitioncount}
\setcounter{definitioncount}{0}
\DeclareRobustCommand{\definition}[1]{%
  \refstepcounter{definitioncount}%
  \noindent\textit{\textbf{Definition \thedefinitioncount\label{definition:#1}: }}%
}
\DeclareRobustCommand{\defref}[1]{Definition~\ref{definition:#1}}

%notes of different authors
\newif\ifnotes
\notestrue
\notesfalse

\newif\ifdiff
\difftrue
\difffalse

\newcommand{\anote}[1]{\ifnotes $\ll$\textsf{\textcolor{purple}{Ari: {#1}}}$\gg$ \fi}
\newcommand{\nnote}[1]{\ifnotes $\ll$\textsf{\textcolor{orange}{Novak: {#1}}}$\gg$ \fi}
\newcommand{\diff}[1]{\ifdiff\textcolor{orange}{#1}\else#1\fi}

%%% Local Variables:
%%% mode: latex
%%% TeX-master: "main"
%%% End:

% Used for displaying a sample figure. If possible, figure files should
% be included in EPS format.
%
% If you use the hyperref package, please uncomment the following line
% to display URLs in blue roman font according to Springer's eBook style:
% \renewcommand\UrlFont{\color{blue}\rmfamily}

\usepackage{xcolor} 
\newcommand\pawel[1]{{\color{red} [\bf PW: #1]}}
\newcommand\lukasz[1]{{\color{blue} [\bf ŁL: #1]}}

\pdfinfo{
/TemplateVersion (IJCAI.2023.0)
}

% Remember, if you use this you must call \IEEEpubidadjcol in the second
% column for its text to clear the IEEEpubid mark.

\begin{document}

\title{Reinforcement learning for optimization of energy trading strategy}

%%% Provide names, affiliations, and email addresses for all authors.

% \author{
% Anonymous Author
% \affiliations
% Anonymous affiliation
% \emails
% anonymous email
% }

% \iffalse
\author{Łukasz Lepak$^1$ \and Paweł Wawrzyński$^2$
\affiliations 
$^1$Warsaw University of Technology \and 
$^2$IDEAS NCBR 
\emails 
$^1$lukasz.lepak.dokt@pw.edu.pl \and $^2$pawel.wawrzynski@ideas-ncbr.pl
}
% \fi

\maketitle

\begin{abstract}
An increasing part of energy is produced from renewable sources by a~large number of small producers. The efficiency of these sources is volatile and, to some extent, random, exacerbating the energy market balance problem. In many countries, that balancing is performed on day-ahead (DA) energy markets. In this paper, we consider automated trading on a~DA energy market by a~medium size prosumer. We model this activity as a Markov Decision Process and formalize a~framework in which a~ready-to-use strategy can be optimized with real-life data. We synthesize parametric trading strategies and optimize them with an~evolutionary algorithm. We also use state-of-the-art reinforcement learning algorithms to optimize a~black-box trading strategy fed with available information from the environment that can impact future prices. \\ 
{\it Keywords:} Reinforcement learning, Automated trading, Energy market
\end{abstract}

\section{Introduction}

In the year 2021, 6.54\% and 3.63\% of global electricity was produced by wind turbines and solar panels, respectively, after these ratios doubled in 5 preceding years \citep{2022ritchie+1}. The power of wind and sunlight reaching the Earth's surface is, to some extent, random. Therefore, while the rise of renewable energy sources presents the prospect of cheap and clean energy, it also exacerbates the problem of balancing power supply and demand. 

In many countries, the main institution that balances volatile electricity supply and demand is a day-ahead energy market. Every day, agents participating in this market place their buy and sell bids separately for every hour between 0 am and 11 pm the next day. Market clearing prices are then designated for each of these hours, and the bids are consequently executed or not, depending on the proposed prices. 

In this paper, we consider an~energy prosumer, who is an agent that (i) consumes electricity, (ii) produces electricity, and (iii) has electricity storage. What is of our interest here is a~strategy for automated trading on a day-ahead energy market on behalf of this agent. 

%In most studies, the problem of automated trading is reduced to the problem of prediction, as buying and selling may be naturally based on predictions: If the price is going to increase, then the action should be to buy, and vice versa. This approach is reasonable in the case of highly liquid markets, where it is always possible to buy/sell at the current market price. However, it does not make sense where there is a~considerable delay between making bids and actual transactions. 
%Therefore, in this paper we take a~different approach and consider placing market bids as a~sequential decision process under uncertainty. We present a~realistic model of such a~process. This model allows to optimize a~trading strategy that is ready for real-life deployment. 
%The contributions of this paper are as follows: 
%\begin{enumerate} 
%\item We model trading on a day-ahead energy market as a~Markov Decision Process (MDP). 
%\item We design a~parametric strategy of automated trading based on the features of this specific MDP. 
%\item We optimize the parameters of the aforementioned strategy in with an evolutionary algorithm and real-life data. The resulting strategy is deployable in the real market. 
%\item We apply reinforcement learning to optimize a~black-box strategy of automated trading. Once again, we use real-life data and produce a~strategy deployable in the real market. 
%\end{enumerate}

In most studies, decision-making in power systems is based solely on the state of this system. We argue that (i) a~useful strategy for operation in the power system needs to be fed with data on the environment and (ii) it needs to be optimized with real-life data. Firstly, reasonable temporal energy allocation needs to be based on the information that makes it possible to anticipate future prices (even if they are not directly predicted). Therefore, the strategy needs to be based on such information. Secondly, the environment that impacts the energy prices (e.g., weather conditions) has its own temporal dynamics that are hardly possible to model but can be replayed from real-life data, which is enough for strategy optimization. 

Based on the above line of thought, this paper contributes as follows: 
\begin{itemize} 
\item We formalize a~framework in which bidding on a day-ahead energy market is a Markov Decision Process, on which behavior can be optimized with real-life data rather than in model-based simulations or with real-life trial-and-error. 
\item We design a~parametric strategy of automated bidding, which is fed with available information that makes it possible to anticipate future prices. 
\item We apply reinforcement learning to optimize the above strategy. 
\end{itemize} 

The rest of the paper is organized as follows: Section~\ref{sec:problem} defines the problem at hand. Section~\ref{sec:related-work} reviews related work. Section~\ref{sec:model} contains the main contribution of this paper: the model of the market, the strategies of automated trading, and the methods of their optimization. Section~\ref{sec:simulations} presents the results of simulation experiments. The last section concludes the paper. 

\section{Problem definition} 
\label{sec:problem} 

\subsection{Day-ahead energy market} 

Details of the day-ahead (DA) energy market are here taken from the Polish market of this kind. When created in 2000, this market was modeled on existing day-ahead energy markets in Western Europe. It is, therefore, typical. 

Every day between 8 am and 10.30 am, an agent participating in the market places a set of bids defined by: (i) [buy or sell] indicator, (ii) price for 1 MWh [PLN], (iii) volume [number of MWh, at least 0.1 MWh], and (iv) an hour of realization [one of 24 between 0 am and 11 pm the next day]. The bids are independent. Based on the bids placed by all agents, the clearing market price for each hour is designated. A~buy bid is accepted when its price is not below the market price for its hour. A~sell bid is accepted when its price is not above the market price for its hour. At each hour of the next day, the agents that realize their sell bids inject the declared volume of electricity into the system and get the market price for it. The agents that realize their buy bids withdraw the declared volume of electricity from the system and pay the market price for it.  

In order to participate in the day-ahead energy market in Poland, every agent has to apply to become a member of this market and pay 2~000 PLN initial fee. Then, market members must pay 2~000 PLN a year to maintain their member status. Each member may choose one of two options of membership. Option 1: A~participant pays a yearly participation fee equal to 50 000 PLN and 0.08 PLN for each MWh traded, making it well-suited for agents with high turnover. Option 2: A~participant pays a yearly participation fee equal to 1 000 PLN and 0.45 PLN for each MWh traded, which may be better for agents making small or occasional bids. %\sout{Day-ahead energy market participants have to pay a yearly participation fee equal to 50 000 PLN (Variant 1) or 1 000 PLN (Variant 2). Also, each participant pays a~small fee proportional to their turnover, equal to 0.08 PLN/MWh (Variant 1) or 0.45 PLN/MWh (Variant 2). } }

\subsection{Prosumer} 

The agent considered here (i) consumes electricity at random but with a~given statistical profile, (ii) produces electricity with means of limited random efficiency, such as a solar panel or wind turbines, (iii) has energy storage with limited capacity and efficiency (it outputs less energy than it inputs). We also assume that the prosumer is large enough to be able to participate in a DA energy market and not large enough for its bids to change the market prices. 

At every hour, the agent may consume, produce, buy and sell some energy. The residual energy is deposited into or taken from the energy storage. If some fraction of the residuum still remains because of the storage being full or empty, this portion is given to or taken from the market operator, and the agent is charged the corresponding penalty fee. 

An example of a~prosumer considered here is a~group (or an aggregator) of households. It cannot be a~single household, though, as the minimum volume of electricity tradeable on the market is 0.1 MWh, which is too much for a~typical single household to consume or produce.

The objective of the prosumer is to maximize its profit (or minimize its costs) by issuing optimal bids on a DA market. Essentially, the agent should buy the energy when its market price is relatively low, keep it in storage, and/or sell it when the market price is relatively high. The agent should also avoid paying penalty fees, thus avoiding having the storage charged or discharged entirely. Note that the problem does not quintessentially change when the prosumer does not produce nor consume electricity because then it becomes a~temporal arbitrator and its profit still non-trivially depends on the strategy of issuing the buy/sell bids. However, if the prosumer does not have the storage, then events at different times are independent of each other, and the objective degenerates to just predicting the prosumer's own production and consumption. 

\section{Related Work} 
\label{sec:related-work} 

\paragraph{Automated trading on the electricity market.} 

%Strategic bidding, energy brokerage (1997, jeden ze starszych) - \citep{lamont1997strategic}
Bidding on a one-day ahead (DA) energy market was presented as an optimization problem by \citet{lamont1997strategic}. 
%Proposed bidding strategies, optimized with genetic algorithm (2001) - \citep{wen2001strategic}
\citet{wen2001strategic} introduced a~catalog of parametric bidding strategies and optimized their parameters with a~genetic algorithm. 
%Bidding strategy, evolutionary programming - \citep{attaviriyanupap2005new}
\citet{attaviriyanupap2005new} proposed other strategies and applied evolutionary programming to optimize them. 
%Mixed integer linear programming, for producers, small test data - \citep{bakirtzis2007electricity}
\citet{bakirtzis2007electricity} analyze selling on a~DA energy market from a~producer point of view and optimize his bidding strategy with mixed integer linear programming. 
%Review of day-ahead bidding strategy models for power producers - \citep{kwon2012optimization}
Bidding strategies applicable by power producers at a~DA energy market were reviewed by \citet{kwon2012optimization}. 
%Microgrid bidding strategy, optimization model proposed, forecasted wind and energy prices - \citep{liu2015bidding}
\citet{liu2015bidding} analyze a microgrid that produces, stores, consumes energy and buys/sells it on a~DA market. The authors use hybrid stochastic/robust optimization and predictions of prices and wind to optimize the bidding strategy. 
%Strategic bidding, virtual power plant - \citep{rahimiyan2015strategic}
\citet{rahimiyan2015strategic} analyze a~microgrid as above, but the strategy they develop also covers bidding on a~real-time (RT) energy market. 
%Stochastic optimization, 1000 prosumers, Iberian market - \citep{iria2017trading}
\citet{iria2017trading} apply stochastic optimization to designate a~strategy of bidding at a DA energy market by an aggregator of prosumers. It is assumed there that the prosumers do not have any batteries but have access to a~real-time (RT) energy market. 
%Clustering + optimization, 1000 prosumers, Iberian market - \citep{iria2019cluster}
\citet{iria2019cluster} further extend the above work with a~clustering of the aggregated prosumers. 
%\citep{prabavathi2015energy} Bidding system for p2p market - \citep{zhang2016bidding}
More general issues related to energy markets, microgrids, and bidding strategies are analyzed in \citep{prabavathi2015energy,zhang2016bidding}. 

Automated trading on an~energy market is a~complex activity that can be modeled as a~parametric transformation of available information into action. The parameters of this transformation can be determined with usual optimization techniques such as evolutionary algorithms. However, as the more complex behavior is expected and the more complex transformation is required, the less effective these techniques become. An~approach specialized for the optimization of complex behavior is reinforcement learning.   

\paragraph{Reinforcement learning and the electricity market.}
%Przeglądowe - \citep{jogunola2020consensus}, \citep{yang2020reinforcement}, \citep{perera2021applications}
With the advent of electricity prosumerism, energy micro-grids, and flexible price-driven energy consumption, there is an increasing need for automated decision-making and control in various activities undertaken by the energy market participants. Strategies for these agents can be optimized with reinforcement learning (RL). Various applications of RL in power systems are reviewed in \citep{jogunola2020consensus,yang2020reinforcement,perera2021applications}. 
%\citep{nanduri2007reinforcement}
\citet{nanduri2007reinforcement} analyze bidding on a DA energy market as a~zero-sum stochastic game played by energy producers willing to exercise their market power and keep their generators productive. RL is used there to optimize their bidding strategy. 
%RL + day-ahead market for EV fleet charging - \citep{vandael2015reinforcement}
\citet{vandael2015reinforcement} analyze bidding on a DA energy market from the point of view of a~flexible buyer (who charges a~fleet of electric vehicles). His strategy is optimized with RL. 
%P2P energy trading with RL - \citep{chen2018indirect}
%Deep Q-Learning on local energy markets (LEM) - \citep{chen2018local}, \citep{jogunola2021trading}
A~number of papers is devoted to peer-to-peer trading with electricity on a~local, event-driven energy market, with RL applied to optimize the behavior of such peers \citep{chen2018indirect,chen2018local,jogunola2021trading,bose2021reinforcement,qiu2021multi}.  
%Arbitrage, real-time energy market - \citep{wang2018energy}
\citet{wang2018energy} use RL to develop a strategy of temporal arbitrage for an agent that operates on a~real-time energy market with an~energy storage. 
%Hour-ahead RL-based model for demand response, a neural network for price prediction - \citep{lu2019demand}
\citet{lu2019demand} use RL and neural price predictions to optimize the scheduling of home appliances of private users. The authors assume that the electricity prices are changing and are known one hour ahead. %LEM - \citep{bose2021reinforcement}
\citet{bose2021reinforcement} analyze a~similar setting in which the users also trade energy with each other.  
%Multi agent RL, P2P day-ahead energy market, distributed RL algorithms - \citep{qiu2021multi} pawel: w tym artykule DA oznacza double-side auction i nie ma tu mowy o day-ahead 
\citet{qiu2021multi} optimize the user strategies in this setting with multi-agent RL. 
%Day-ahead bidding for extending battery life - \citep{dong2021strategic}
\citet{dong2021strategic} use RL to optimize a~strategy of bidding on a~DA energy market by a~battery energy storage system (BESS). The authors address the dynamics of that process only to a~limited extent. Firstly, the criterion of policy optimization is on-day-ahead profit instead of a~long-term profit. Secondly, no information on the environment that could impact future prices is considered, e.g., weather conditions.  

\citet{dong2021strategic} considers simultaneous trading on a~DA and hour-ahead energy markets by an energy storage operator as a~Markov Decision Process. In this MDP, consecutive days are separate episodes, so between-days dynamics of the market are not accounted for. Discrete actions define the parameters of the bids. They are not based on external observations such as weather forecasts. In the current paper, we take into account the between-days dynamics, continuous parameters of the bids, and weather forecasts. These all lead to significantly better performance of our proposed strategy. 

\section{Model}
\label{sec:model} 

\subsection{Markov Decision Process} 

In this section, we model the automated trading on a day-ahead energy market as a~Markov Decision Process (MDP) \citep{2018sutton+1}. This MDP includes the following components: 
\begin{itemize} 
\item Time, $t=1,2,\dots$. Here time instants denote days. 
\item Actions, $\ctrl_t \in \ctrlSpace$. An action is a set of bids in the form 
\Beq
    \ord{ volume, price, type, hour },
\Eeq 
where $type \in \{\Sell, \Buy\}$, $hour\in\{0\text{ am},1\text{ am}, \dots, 11\text{ pm}\}$. 
\item Reward, $r_t \in \real$ is equal to the profit collected during the day. 
\item States of the environment, $\state_t\in\stateSpace$. A state here is a vector that encompasses all the information about the surrounding world that may influence the market prices of electricity and the volume of its production and consumption by the prosumer. Here we divide the coordinates of the state into {\it uncontrollable}, $\state^u_t$, and {\it controllable}, $\state^c_t$, $\state_t = \ord{\state^u_t, \state^c_t}$. The agent does not influence the uncontrollable state coordinates; they may include an~indicator of the day within the week, an~indicator of the month within the year, and weather forecasts. These state coordinates evolve according to a~stationary conditional probability: 
\Beq \label{state^u} 
    \state^u_{t+1} \sim P(\cdot | \state^u_t). 
\Eeq
The controllable state coordinates are directly determined by the actions taken, and the uncontrollable state coordinates, that is
\Beq \label{state^c} 
    \state^c_{t+1} = f(\state^c_t, \ctrl_t, \state^u_t, \state^u_{t+1}), 
\Eeq
where $f$ is known. Here there is only one controllable state coordinate: the storage level. The $f$ function is known because the storage level trivially results from consuming, producing, buying, and selling energy. 
\end{itemize} 
The assumptions that (i) the prosumer is small enough not to impact the market prices, (ii) the uncontrollable states changes according to the stationary rule \eqref{state^u}, (iii) the controllable state evolves according to a~known transition function \eqref{state^c} have the following implication: Based on a~recorder trajectory of uncontrollable states, $(\state^u_t: t=1,\dots,T)$, we can designate a~strategy of selecting actions $\ctrl_t$ based on states $\state_t$ and evaluate this strategy in a~simulation based on $(\state^u_t: t=1,\dots,T)$. This valuation will be an~unbiased estimate of the performance of this strategy deployed in reality. 

Note that the above-defined division of state variables into controllable and uncontrollable is unusual. In a~typical MDP, we assume that the state changes according to 
\Beq 
    \state_{t+1} \sim P_\state(\cdot | \state_t, \ctrl_t), 
\Eeq 
where the conditional probability $P_\state$ is quite difficult to analyze and estimate. Therefore, a~strategy of choosing actions cannot be evaluated with no bias within a~simulation based on~a~model of~$P_\state$.

\subsection{Designing a strategy}

In general, by a~{\it strategy}, $\pi$, we understand a~probability distribution of actions, $\ctrl_t$, conditioned on states, $\state_t$: 
\Beq \label{pi} 
    \ctrl_t \sim \pi(\cdot|\state_t). 
\Eeq 
In some cases, $\pi$ will be a single point distribution, thereby being a~function. 

Let us denote by $l_t\in(0,1)$ the storage level at midnight when the bids defined in action $\ctrl_t$ start to be realized. The action $\ctrl_t$ is selected at 10.30 am on a preceding day. At this moment, $l_t$ is unknown. However, it is known which of the bids placed with $\ctrl_{t-1}$ have been and will be realized. Therefore, $l_t$ can be estimated with a~reasonable accuracy. We will denote this estimate by $\est l_t$. 

\paragraph{Timing-based strategy (Timing).} A simple strategy may be based on an~observation that the market prices are generally low between 0 am - 3 am and high between 5 pm - 8 pm. That leads to actions comprising the following eight bids: 
\Beq \label{pi:timing} 
    \begin{split} 
    & \ord{(\alpha_1 - \alpha_2\est l_t)/4, +\infty, \Buy, 0\text{ am}} \\ 
    & \ord{(\alpha_1 - \alpha_2\est l_t)/4, +\infty, \Buy, 1 \text{ am}} \\ 
    & \ord{(\alpha_1 - \alpha_2\est l_t)/4, +\infty, \Buy, 2 \text{ am}} \\
    & \ord{(\alpha_1 - \alpha_2\est l_t)/4, +\infty, \Buy, 3 \text{ am}} \\ 
    & \ord{(\alpha_1 + \alpha_2\est l_t)/4, 0, \Sell, 5 \text{ pm}} \\ 
    & \ord{(\alpha_1 + \alpha_2\est l_t)/4, 0, \Sell, 6 \text{ pm}} \\
    & \ord{(\alpha_1 + \alpha_2\est l_t)/4, 0, \Sell, 7 \text{ pm}} \\ 
    & \ord{(\alpha_1 + \alpha_2\est l_t)/4, 0, \Sell, 8 \text{ pm}}
    \end{split} 
\Eeq
where $\alpha_1, \alpha_2$ are positive coefficients. The term $\pm\alpha_2\est l_t$ results from the fact that the more we have in the storage, the less we want to buy, and the more we want to sell. The prices ($+\infty$ and $0$) are defined to ensure that the bids will be accepted. 

\paragraph{Opportunistic strategy (Opportunistic).} Another strategy is based on the observation that the prices generally vary, and the best thing to do is to buy when the price is relatively low and sell when it is relatively high while considering the battery level and production capabilities. That leads to the strategy in which, for each hour $h$, there is a pair of bids: 
% \Beq \label{pi:arbiter}
%     \begin{split} 
%     & \ord{\alpha_1, \bar p (\alpha_{2h+4} - \alpha_2\est l_t), \Buy, h} \\ 
%     & \ord{\alpha_1, \bar p (\alpha_{2h+5} + \alpha_3\est l_t), \Sell, h}, 
%     \end{split} 
% \Eeq
\Beq \label{pi:arbiter}
    \begin{split} 
    & \ord{\bar v\exp(\alpha_{4h+5} + \alpha_1 \est l_t), \bar p \exp(\alpha_{4h+7} + \alpha_3\est l_t), \Buy, h} \\ 
    & \ord{\bar v\exp(\alpha_{4h+6} + \alpha_2 \est l_t), \bar p \exp(\alpha_{4h+8} + \alpha_4\est l_t), \Sell, h}, 
    \end{split} 
\Eeq
where $h=0,1,\dots,23$, $\bar p$ is the market energy price averaged over every hour and the preceding 30 days, $\bar v$ is the maximum possible energy generation volume defined by solar and wind installations, and $\alpha_i$ are coefficients. 
%Here, we always try to sell/buy the same quantum $\alpha_1$ of energy but at different prices. These prices are to be optimized with respect to the profit this strategy yields. 
Here we try to sell/buy varied volumes of energy based on our production capabilities and battery level at different prices. These prices and volumes are to be optimized with respect to the profit this strategy yields. 


\subsection{Optimization of strategy} 

Given real data $(\state^u_t: t=1,\dots,T)$, we optimize a~parametric strategy such as \eqref{pi:timing} or \eqref{pi:arbiter} using a~gradient-free optimization method. In this approach, we need to be able to evaluate the strategy for any~given vector of parameters. Here, an evaluation is a~simulation of events over time $t=1,\dots,T$ with the real data, given strategy in use, and calculating the resulting profit. 

\subsection{Black-box strategy and its optimization with reinforcement learning} 

We design a black-box strategy as a~set of 24 pairs of bids 
\Beq \label{pi:black-box} 
    \begin{split}
    & \ord{\bar v \exp(v^B_h), \bar p \exp(y^B_h), \Buy, h} 
    \\
    & \ord{\bar v \exp(v^S_h), \bar p \exp(y^S_h), \Sell, h},
    \end{split} 
\Eeq 
for $h=0\text{ am}, 1\text{ am}, \dots, 11\text{ pm}$. The numbers $v^B_h, y^B_h, v^S_h, y^S_h$ are produced as a~sum of the output of a~zero-mean normal noise, $\xi_t$, and a~neural network, $g$, output: 
\Beq \label{pi:RL} 
    \begin{bmatrix}
    v^B_{0am}  & \dots  & v^B_{11pm} \\ 
    y^B_{0am}  & \dots  & y^B_{11pm} \\ 
    v^S_{0am}  & \dots  & v^S_{11pm} \\
    y^S_{0am}  & \dots  & y^S_{11pm}
    \end{bmatrix}
    = \xi_t + g(\state_t; \theta), 
    \; \xi_t \sim \normal(0, \Sigma). 
\Eeq
The network $g$ is fed with the state $\state_t$ and parameterized by the vector $\theta$ of trained weights. 

The reason to introduce the noise $\xi_t$ into the bids is exploration: By taking different actions under similar circumstances, the trading agent is able to learn to tell good actions from the inferior ones in the current state. 

In order to optimize the strategy \eqref{pi:RL}, we may use any algorithm of on-line reinforcement learning \citep{2001sutton+2} e.g., A2C \citep{2016mnih+many}, PPO \citep{2017schulman+4}, TD3 \citep{fujimoto2018addressing} or SAC \citep{haarnoja2018soft}. A~training consists of a~sequence of simulated trials in which the trajectory of uncontrollable states $(\state^u_t: t=1,\dots,T)$ is just replayed from the data, and the corresponding trajectory of controllable states $(\state^c_t: t=1,\dots,T)$ is designated based on the uncontrollable states, the actions selected and the function $f$ \eqref{state^c}. 

\section{Experimental study} 
\label{sec:simulations}

In this section, we demonstrate the effectiveness of our proposed black-box strategy optimized with reinforcement learning. We compare it to the parametric strategies optimized with the~gradient-free CMA-ES algorithm \citep{hansen2016cma} and the strategy optimized with the FARL algorithm \citep{dong2021strategic}. 

\subsection{Testing environment}

Our experiments are conducted using a~custom environment simulating day-ahead energy market operations (EngMar in short) based on real-life data from the Polish market. This environment allows for customization of various market settings, such as a~bid creation time, a~scale of the bidding prosumer (defined by the number of households), or its solar and wind energy generation capabilities. The environment is based on the OpenAI Gym environment interface, making it compatible with many reinforcement learning libraries, including Stable-Baselines3 \citep{stable-baselines3}, which we use as our source of RL algorithms. 
% The code for the EngMar environment and experiments described below is available at \citep{rl4energytrading}. 

In our experiments, we use real historical data from the following sources:\footnote{We will make both our framework and the data available upon acceptance of the paper.} 
\begin{itemize}
    \item energy prices -- Polish day-ahead energy market (Fixing I).
    \item weather data -- Polish Meteorology and Water Management Institute.
    \item average energy consumption -- Polish Central Statistical Office.
\end{itemize}

As there are no publicly available historical weather forecast datasets for Poland, we generate one by noising actual weather data. For each day in the actual data, we start at 10 am of the previous day. For every hour from 11 am of the previous day to 11 pm of the currently forecasted day, we generate the forecasts as follows:
\Beq
\begin{split}
     & \epsilon_t \sim \normal(0, \frac{\sigma^2}{24}) \\
     & d_t = \sum_{i=1}^{t} \epsilon_t \\
     & x_{t}^{forecast} = x_{t}^{actual} + d_t
\end{split}
\Eeq
where $\sigma$ is an accuracy of a 24-hour forecast, $d_t$ is a deviation for index $t$ and $x_t^{actual}, x_t^{forecast}$ are actual and forecasted weather for index $t$, respectively. For cloudiness, we assume $\sigma = 2$ Oktas, and for wind speed, we assume $\sigma = 1$ m/s. Here, $t = 0$ denotes 10 am of the previous day, and we are interested in $t \in [14, 37]$ - next-day forecasts. Cloudiness forecasts are clipped to be integer values at least zero, while wind speed forecasts are clipped to be at least zero.

We also test the RL agent without the weather forecast data included in the observations. We do this to check if the weather forecasts allow the agent to define better bids, as this information impact future energy production, consumption, and prices. 

Common environment settings used in our experiments are depicted in Table \ref{tab:env-settings}. We set the action scheduling time to match the Polish day-ahead energy market. Battery and solar panel efficiencies reflect the efficiencies of real-life batteries and solar panels. Wind energy and solar energy limits are tuned so that daily energy production in the environment averages around 1 MWh. The number of households is set to 100 to scale the simulation to represent a medium-sized prosumer, an~aggregator, or a small energy generation facility. 

\begin{table}[!h]
\centering
\begin{tabular}{l | r}
Action scheduling time & 10.30 am \\
\hline
Battery capacity & 2 MWh \\
Battery efficiency  & 85\% \\
\hline
Maximum solar energy generation & 0.4 MWh \\
Solar panel efficiency & 20\% \\
Maximum wind energy generation & 0.05 MWh \\
Maximum wind speed for which \\
\quad wind turbines are still operational & 11 m/s \\
Number of households & 100 \\
\end{tabular}
\caption{Parameters of the EngMar environment used for experiments.}
\label{tab:env-settings}
\end{table}

Energy consumption for the given hour ($E_{c}^{h}$) is calculated as follows:
\Beq \label{eq:eng_consumption} 
    E_{c}^{h} = n \cdot E_{c\_avg}^{h} \cdot \left | 1 + \rho \right |
\Eeq
where $E_{c\_avg}^{h}$ is the average energy consumption per one household for the given hour, $n$ is the number of households, and $\rho \sim \normal(0, 0.03)$ allows the resulting energy consumption to differ each day while maintaining the average value. Equation \eqref{eq:eng_consumption} is prepared so that it scales well with the changing number of households. 

Solar energy production for the given hour ($E_{s}^{h}$) is based on cloudiness value from the actual weather data and is calculated as follows:
\Beq \label{eq:solar_gen} 
    E_{s}^{h} = s_{max} \cdot (1 - \frac{c}{8}) \cdot \eta
\Eeq
where $s_{max}$ is the maximum solar energy generation, $c \in \{0, 1, ..., 7, 8\}$ is the cloudiness value in Oktas ($0$ - clear sky, $8$ - heavy overcast) taken from the weather data and $\eta$ is the solar panel efficiency.

Wind energy production for the given hour ($E_{w}^{h}$) is based on the actual wind speed value from the weather data and is calculated as follows:
\Beq \label{eq:wind_gen} 
    E_{w}^{h} = w_{max} \cdot \frac{ws}{ws_{max}} \cdot (ws \leq ws_{max})
\Eeq
where $w_{max}$ is the maximum wind energy generation, $ws$ is the wind speed, $ws_{max}$ is the maximum wind speed for which the wind turbines are still operational, and $(ws \leq ws_{max}) = 1 \text{ when } ws \leq ws_{max} \text{, else } 0$.

During the simulation, it may turn out that the agent has to buy missing energy or sell excess energy immediately. In that case, he is being penalized for such events. Immediate buying is realized for double the current market price, and immediate selling is realized for half the current market price so that the agent has the incentive to better plan his bids instead of relying on instant buys or sells. Also, we do not include market entry and transaction fees, as they are fixed costs independent of the bidding strategy. 

\subsection{Experiments}

\paragraph{Evolutionary algorithm}

The evolutionary algorithm CMA-ES is used to optimize strategies defined by Equations \eqref{pi:timing} and \eqref{pi:arbiter}. It utilizes data from 2016 to 2018 as the training set and data from 2019 as the testing set. 
%\pawel{niepotrzebne} Training is done on a set of intervals from the training set lasting 90 days. These intervals are in the same order for every vector of parameters sampled by the algorithm, so the comparison between them is fair. 
After training, the resulting parameters (mean values) are evaluated on a single testing interval 365 days long. Table \ref{tab:cmaes-settings} presents the parameters used for the CMA-ES algorithm. Customized initialization for parameters $\alpha_{4h+5}, \alpha_{4h+6}$ of Opportunistic strategy as defined in Equation \eqref{pi:arbiter} prevents the initial samples from creating bids with too high volumes, which leads the strategy to the inefficient solution of creating no bids at all. 
\begin{table}[!h]
\centering
\begin{tabular}{l | r}
\multirow{2}{*}{Initial mean value ($\mu$)} & default: $\normal(0, 1)$ \\
&\eqref{pi:arbiter}, $\alpha_{4h+5}, \alpha_{4h+6}$: $\normal(-2, 1)$ \\
Initial sigma ($\sigma$) & 1 \\
\hline
Population size & automatic ($4 + \lfloor 3 \cdot ln(n) \rfloor$) \\ 
Generations & 100 \\
\end{tabular}
\caption{Parameters of the CMA-ES algorithm used for the experiments. $n$ is the number of parameters in the strategy.}
\label{tab:cmaes-settings}
\end{table}


\paragraph{Reinforcement learning}

Reinforcement learning is used to optimize the strategy defined in \eqref{pi:black-box}. It utilizes data from 2016 to the third quarter of 2018 as the training set, data from the fourth quarter of 2018 as the validation set, and data from 2019 as the testing set. The training is done on random intervals from the training set 90 days long, generated randomly. Periodically, evaluation is done on a single validation interval 90 days long. After the training timesteps budget is depleted, the model for which the highest reward on validation interval was achieved is evaluated on the single testing interval 365 days long. We use the A2C algorithm \citep{2016mnih+many} to optimize the strategy of the RL agent. Parameters used for the A2C algorithm are presented in the supplementary material. 

The action space is limited to range the $[-3, 3]$, which allows the agent to define prices and volumes up to $e^3\approx20$ times smaller/larger than the 30-day average price and maximum possible energy generation volume, respectively. 

The observation of the environment's state (117 values) is passed to the agent at bid placing time and contains the following information:
\begin{itemize}
    \item prices of energy at the current day for every hour (24 values) -- these prices result from the bids created the day before; the agent does not know energy prices for the bid currently submitted.
    \item average energy consumption for $n$ households for every hour, $n \cdot E_{c\_avg}^{h}$ (24 values) -- this is statistical data about consumption, the actual consumption data is designated according to \eqref{eq:eng_consumption}.
    \item current relative battery charge (1 value).
    \item estimated relative battery charge at midnight (1 value).
    \item one-hot encoded information about the current month (12 values).
    \item one-hot encoded information about the current day of the week (7 values).
    \item cloudiness and wind speed forecasts for each hour of the next day (48 values).
\end{itemize}

For comparison, we also applied the FARL algorithm \citep{dong2021strategic}, which is a~conceptually different approach to optimize a black-box bidding strategy. We fed FARL with the same training, evaluation, and test data discussed above. Parameters and details of FARL are presented in the supplementary material. 

\subsection{Results}

%. This algorithm *}[h!]
%     \begin{tabular}{c  \includegraphics[width=0.48\textwidth]{images/fig_sample.png} &\!\!\!\!\!
%     \includegraphics[width=0.48\textwidth]{images/fig_sample.png} \\%&\!\!\!\!\!
%     \includegrapics[width=0.48\textwidth]{images/fig_sample.png} &\!\!\!\!\!
%     \includegraphics[width=0.48\textwidth]{images/fig_sample.png}
%     \end{tabular}
%     \caption{Some results.}
%     \label{fig:results}
% \end{figure*}

\begin{figure*}[!htb]
    \centering
    \scalebox{.85}{\includegraphics[width=\textwidth]{images/balance_all.png}}
    \vspace{-1em}
    \caption{Wallet balances achieved on testing data by different strategies. A2Cw denotes the results of the black-box strategy optimized with the A2C algorithm with weather forecasts included in observations, A2Cnw - without weather forecasts. Numbers in the FARL algorithm denote the number of possible discrete actions. }
    \label{fig:balances}
\end{figure*}

\begin{figure*}[!htb]
    \centering
    \begin{tabular}{c c}
    \scalebox{.85}{\includegraphics[width=0.48\textwidth]{images/battery_A2Cw.png}} &\!\!\!\!\!
    \scalebox{.85}{\includegraphics[width=0.48\textwidth]{images/unscheduled_A2Cw.png}} \\%&\!\!\!\!\!
    \scalebox{.85}{\includegraphics[width=0.48\textwidth]{images/amounts_A2Cw.png}} &\!\!\!\!\!
    \scalebox{.85}{\includegraphics[width=0.48\textwidth]{images/prices_A2Cw.png}} \\%&\!\!\!\!\!
    \end{tabular}
    \vspace{-1em}
    \caption{Plots for the black-box strategy trained with the A2C algorithm. {\it Left-top:} Battery level. {\it Right-top:} Unscheduled energy buying/selling. {\it Left-bottom:} Bid volumes. {\it Right-bottom:} Bid prices.}
    \label{fig:a2c}
\end{figure*}

Results of experiments are shown in Figures \ref{fig:balances}, \ref{fig:a2c}, and in the supplementary material. We define the reference balance of the given day as the difference between the energy produced and consumed multiplied by this day's average energy price. We then calculate the sum of these reference balances during the whole simulation. Note that it is moderately difficult to achieve the reference balance. The agent mostly consumes the energy in the evenings, when it is expensive, and produces it when its price is average. Therefore, the agent needs to ensure it buys little energy when it needs it to reach the reference balance.  

Figure \ref{fig:balances} presents balance changes during test simulations, averaged over 5 test runs on different random seeds. Streaks around graphs represent the range of balances achieved by the given strategy. We can see that the black-box strategies optimized with the A2C algorithm achieve the best return on the test simulation, beating our reference balance and the strategies optimized with the CMA-ES and FARL algorithms. Also, the A2C-trained strategy utilizing weather forecasts as part of its observations is able to achieve higher returns than the A2C-trained strategy without those observations. 

In Figures \ref{fig:a2c}, we look into five days from the middle of the testing simulations. For battery levels plots, we show the average relative battery charge with streaks around graphs indicating the range of levels from all testing simulations, while the other plots were taken from a~chosen testing simulation. There is an unscaled market price graph on the bid volumes plot, which allows for easy identification of whether the successful bid was realized when the price was high or low.

It is seen in Figure \ref{fig:a2c} that the strategy trained with RL behaves reasonably: It charges the battery at about 0 am when the prices are low and discharges at about 8 am when they are high. Unscheduled purchases/sells, which are costly, are rare. 

The simple Timing and Opportunistic strategies behave reasonably (we discuss them in more detail in the supplementary material). However, due to their simplicity, they are unable to represent sufficiently complex behavior to respond efficiently to diverse circumstances. 

The bidding strategy developed by the FARL algorithm barely exhibited reasonable behavior. This algorithm is based on {\it Q-Learning} with function approximation applied in a~rather non-standard way: It learns to make a~sequence of $24$ bids, each time only having access to the previous bidding and several other variables. Reasonable bidding based on that information was not possible. 

%The simple Timing strategy based on buying the same amount at night and selling in the afternoon also works reasonably. However, it is not able to adapt sufficiently to the circumstances, hence its performance is noticeably worse than that of A2C. 

%The more elaborate Opportunistic strategy achieves a~better return than its simple Timing counterpart. Its behavior is reasonable: It buys mostly at 0 am (at low prices) and sells mostly at 8 am and 4 pm (at high prices). However, it is still not able to, on average, improves the results over the reference balance. Also, the range of returns of this strategy is noticeably larger than those of other strategies. This strategy usually tries to place selling bids with low prices and high volumes and buying bids with high prices and low volumes. The wide range of returns between runs indicates that the optimization of this strategy is susceptible to getting stuck in local optima. Therefore, it is unlikely that this strategy's globally optimal parameters have been found with the CMA-ES algorithm. 

\subsection{Battery capacity optimization} 

The battery is often the largest part of the prosumer installation cost. Our proposed approach can be readily used to choose the battery from the possible options. It is enough to perform the strategy optimization for each option and compare the incomes with the battery costs. 

Table \ref{tab:a2c-diff-batteries} presents the income gained with the strategy optimized with the A2C algorithm with weather forecasts as input, depending on the battery capacity. A figure with wallet balance changes in testing simulations and additional comments are available in the supplementary material. It is seen that the larger the battery capacity, the larger the income. 

%\lukasz{However, with larger battery capacity, the strategy becomes more speculative, as the agent has more resources to perform bids with higher energy amounts. In some cases, this may lead to losing money on immediate transactions and lowering overall profit. We can see this increased risk when comparing results for capacities 2.0 and 3.0. While the latter has higher mean profit, it also has significantly higher standard deviation, which makes the former safer and still highly profitable option. }

\begin{table}[!h]
\centering
\begin{tabular}{l | r}
Maximum battery capacity (MWh) & Achieved income \\
\hline
1.0 & 39937.62 $\pm$ 1723.72 \\
1.5 & 48245.24 $\pm$ 1182.34 \\
2.0 & 53107.61 $\pm$ 560.78 \\
3.0 & 54770.32 $\pm$ 4760.21 \\
\end{tabular}
\caption{Incomes achieved for different maximum battery capacities with A2C-optimized strategy with weather forecasts. Incomes are averaged over 5 test runes with the same random seeds and are provided with their respective standard deviations.}
\label{tab:a2c-diff-batteries}
\end{table}

\subsection{Discussion} 

The optimal bidding strategy among those analyzed here is based on neural networks trained with reinforcement learning and fed with weather forecasts. Weather impacts the production of energy (e.g., by wind turbines), its consumption (e.g., by air conditioners), and thus its prices. Consequently, optimal bids need to be based on these forecasts. We have tried several RL algorithms. A2C yielded the best performance. The algorithm that was especially disappointing was SAC \citep{haarnoja2018soft}. This algorithm is based on the action-value function with the action (bid parameters in this case) having 96 dimensions. Under these circumstances, the action-value function was impossible to approximate with sufficient accuracy, hence poor performance. 

Parametric strategies with parameters optimized with the CMA-ES evolutionary algorithm behaved worse than those based on neural networks. Even though they were not very complex, their globally optimal parameters were difficult to find for CMA-ES. One can come up with even more elaborate strategies than \eqref{pi:arbiter} (and in fact, we have), but then this strategy will have even more parameters, and their optimal values will be even more difficult to find for any gradient-free optimization algorithm. 

The bidding strategy learned with the FARL algorithm delivered disappointing results, even worse than those achieved with optimized parametric strategies. Its management of available information proved insufficient to effectively map available observations into actions.

\section{Conclusions} 
\label{sec:conclusions} 

In this paper, we proposed a framework for optimization of bidding strategy on a~day-ahead energy market based on simulations and real-life data. We have optimized two parametric strategies with the~state-of-the-art evolutionary algorithm CMA-ES. We have also used reinforcement learning to optimize two strategies that produced bids for this market. One of them was fed weather forecasts, and the other was not. The strategy fed with weather forecasts produced the highest financial return.  

%%%%%%%%%%%%%%%%%%%%%%%%%%%%%%%%%%%%%%%%%%%%%%%%%%%%%%%%%%%%%%%%%%%%%%%%

%%% The acknowledgments section is defined using the "acks" environment
%%% (rather than an unnumbered section). The use of this environment 
%%% ensures the proper identification of the section in the article 
%%% metadata as well as the consistent spelling of the heading.

%%%%%%%%%%%%%%%%%%%%%%%%%%%%%%%%%%%%%%%%%%%%%%%%%%%%%%%%%%%%%%%%%%%%%%%%

%%% The next two lines define, first, the bibliography style to be 
%%% applied, and, second, the bibliography file to be used.

\bibliographystyle{named} 
% This must be in the first 5 lines to tell arXiv to use pdfLaTeX, which is strongly recommended.
\pdfoutput=1
% In particular, the hyperref package requires pdfLaTeX in order to break URLs across lines.

\documentclass[11pt]{article}

% Remove the "review" option to generate the final version.
%\usepackage[review]{ACL2023}
\usepackage{ACL2023}

% Standard package includes
\usepackage{times}
\usepackage{latexsym}

% For proper rendering and hyphenation of words containing Latin characters (including in bib files)
\usepackage[T1]{fontenc}
% For Vietnamese characters
% \usepackage[T5]{fontenc}
% See https://www.latex-project.org/help/documentation/encguide.pdf for other character sets

% This assumes your files are encoded as UTF8
\usepackage[utf8]{inputenc}

% This is not strictly necessary, and may be commented out.
% However, it will improve the layout of the manuscript,
% and will typically save some space.
\usepackage{microtype}

% This is also not strictly necessary, and may be commented out.
% However, it will improve the aesthetics of text in
% the typewriter font.
\usepackage{inconsolata}


% If the title and author information does not fit in the area allocated, uncomment the following
%
%\setlength\titlebox{10cm}
%
% and set <dim> to something 5cm or larger.

%%%%%%%%%%%%%%%%%%%%%%%%%%%%%%%%%%
\usepackage{graphicx}
\usepackage{amsfonts}
\usepackage{amsmath}
\usepackage{bigdelim}
\usepackage{diagbox}
\usepackage{amsthm}
\usepackage{makecell}
\usepackage{mathtools}
\usepackage{booktabs}
\usepackage[shortlabels]{enumitem}
\graphicspath{ {figs/} }

\theoremstyle{remark}
\newtheorem*{question}{Question}

\newcommand{\tk}[1]{\textcolor{blue}{{#1}}}
\newcommand{\sy}[1]{\textcolor{red}{{#1}}}
\newcommand{\mg}[1]{\textcolor{purple}{{#1}}}
\newcommand{\lh}[1]{\textcolor{green}{{#1}}}
\newcommand{\lc}[1]{\textcolor{green}{{#1}}}

% Rounded color box
\definecolor{light_blue}{HTML}{cfdfff}
\usepackage[most]{tcolorbox}
\tcbset{on line, 
        boxsep=1pt, left=0pt,right=0pt,top=0pt,bottom=0pt,
        colframe=white,colback=light_blue,  
        highlight math style={enhanced}
        }

\newcommand{\quash}[1]{}  %Anything in \quash is ignored
\newcommand{\gpt}{\textsc{GPT-2}}
\newcommand{\bert}{\textsc{BERT}}
\newcommand{\bertlarge}{\textsc{BERT-large}}
\newcommand{\mask}{\texttt{[MASK]}}
\newcommand{\cls}{\texttt{[CLS]}}
\newcommand{\sep}{\texttt{[SEP]}}
\newcommand{\mat}{\texttt{mat}}
\newcommand{\id}{\texttt{id}}
\newcommand{\matl}{\texttt{mat}_{\ell \rightarrow \ell'}}
\newcommand{\matattnl}{\texttt{mat\_attn}_{\ell \rightarrow \ell'}}
\newcommand{\matffl}{\texttt{mat\_ffn}_{\ell \rightarrow \ell'}}
\newcommand{\matlnl}{\texttt{mat\_ln1\_ln2}_{\ell \rightarrow \ell'}}
\newcommand{\idl}{\texttt{id}_{\ell \rightarrow \ell'}}
\newcommand{\matlL}{\texttt{mat}_{\ell \rightarrow L}}
\newcommand{\matattnlL}{\texttt{mat\_attn}_{\ell \rightarrow L}}
\newcommand{\matfflL}{\texttt{mat\_ffn}_{\ell \rightarrow L}}
\newcommand{\matlnlL}{\texttt{mat\_ln1\_ln2}_{\ell \rightarrow L}}
\newcommand{\idlL}{\texttt{id}_{\ell \rightarrow L}}

\definecolor{blue(munsell)}{rgb}{0.0, 0.5, 0.69}
%%%%%%%%%%%%%%%%%%%%%%%%%%%%%%%%%%

\title{Jump to Conclusions: Short-Cutting Transformers\\With Linear Transformations}

% Author information can be set in various styles:
% For several authors from the same institution:
% \author{Author 1 \and ... \and Author n \\
%         Address line \\ ... \\ Address line}
% if the names do not fit well on one line use
%         Author 1 \\ {\bf Author 2} \\ ... \\ {\bf Author n} \\
% For authors from different institutions:
% \author{Author 1 \\ Address line \\  ... \\ Address line
%         \And  ... \And
%         Author n \\ Address line \\ ... \\ Address line}
% To start a seperate ``row'' of authors use \AND, as in
% \author{Author 1 \\ Address line \\  ... \\ Address line
%         \AND
%         Author 2 \\ Address line \\ ... \\ Address line \And
%         Author 3 \\ Address line \\ ... \\ Address line}

\author{Alexander Yom Din$^{1}$ ~~~~~ Taelin Karidi$^{1}$ ~~~~~ Leshem Choshen$^{1}$ ~~~~~
Mor Geva$^{2}$ 
\vspace{0.2cm} \\
$^1$Hebrew University of Jerusalem ~~~ $^2$Google Research \\
\small{\texttt{\{alexander.yomdin, taelin.karidi, leshem.choshen\}@mail.huji.ac.il}}, \small{\texttt{pipek@google.com}}}

\quash{
\author{Alexander Yom Din \\
  Hebrew University of Jerusalem \\ \texttt{alexander.yomdin@mail.huji.ac.il} \\\And
  Taelin Karidi \\
  Hebrew University of Jerusalem \\
  \texttt{taelin.karidi@mail.huji.ac.il} \\\And
  Leshem Choshen \\
  Hebrew University of Jerusalem \\ \texttt{leshem.choshen@mail.huji.ac.il} \\\And
  Mor Geva \\
  Google Research \\
  \texttt{pipek@google.com} \\}
}

\begin{document}
\maketitle



\begin{abstract}
% \vspace{-1em}
The diffusion-based generative models have achieved remarkable success in text-based image generation. However, since it contains enormous randomness in generation progress, it is still challenging to apply such models for real-world visual content editing, especially in videos. 
In this paper, we propose \texttt{FateZero}, a zero-shot text-based editing method on real-world videos without per-prompt training or use-specific mask. 
\RM{Specifically, different from a pipeline of two independent inversion and then generation stages, we find the intermediate attention maps during inversions store better structure and motion information. We thus reform them to temporally casual attention and replace them in the generation progress. To further reduce the unnecessary semantic leakage of source video and enhance the editing quality, we then remix the temporally casual attentions via the cross-attention features of the source prompt as the mask.}
To edit videos consistently, we propose several techniques based on the pre-trained models. Firstly, in contrast to the straightforward DDIM inversion technique, our approach captures intermediate attention maps during inversion, which effectively retain both structural and motion information. These maps are directly fused in the editing process rather than generated during denoising. To further minimize semantic leakage of the source video, we then fuse self-attentions with a blending mask obtained by cross-attention features from the source prompt. Furthermore, we have implemented a reform of the self-attention mechanism in denoising UNet by introducing spatial-temporal attention to ensure frame consistency.
Yet succinct, our method is the first one to show the ability of zero-shot text-driven video style and local attribute editing from the trained text-to-image model. We also have a better zero-shot shape-aware editing ability based on the text-to-video model~\cite{tuneavideo}. \RM{Besides video, our unified method also achieves state-of-the-art performance in zero-shot image editing.\chenyang{Need exp or remove the zero-shot image}} Extensive experiments demonstrate our superior temporal consistency and editing capability than previous works.
% The code will be released.
% \chenyang{emphasize: our observation at inversion time} \xiaodong{replacing the bold part to the actual pipeline: \textbf{Specifically, we work on replacing and mixing the attention maps between the inversion and generation since the self-attention map keeps the structure of the original natural image and the cross-attention is semantic-related, after remixing, we replace them in the corresponding generation steps for denoising.}}
% \footnote{Since there is no general video diffusion model is publicly available, we use one-shot video generation method~(Tune-A-Video~\cite{tuneavideo}) as the pretrained video diffusion model for zero-shot video editing\xiaodong{can be removed if we actually zero-shot on video}.}.
\end{abstract}
\section{Introduction}

The ability to reason about plans is critical for performing long-horizon tasks \citep{erol1996hierarchical, sohn2018hierarchical, sharma-etal-2022-skill}, compositional generalization \citep{corona-etal-2021-modular} and generalization to unseen tasks and environments \citep{shridhar2020alfred}.
Consider a simple long-horizon planning scenario where a robot is tasked with preparing a meal and serving it on the table. 
This presents a non-trivial planning problem since the agent needs to understand the sequence of operations required to perform the task and search for the relevant objects in the unfamiliar environment by interacting with various objects. %



Large language models have been recently shown to possess commonsense knowledge about the world such as object affordances and physical dynamics \citep{ouyang2022training,chowdhery2022palm}.
Early approaches considered text based environments and fine-tuned PLMs to predict actions given the history of past observations and actions \citep{jansen-2020-visually,micheli-fleuret-2021-language,yao-etal-2020-keep}.
Recent work has used this ability to reason about plans from text instructions in simulated household environments with simplifying assumptions such as text-only environment observations or feedback \citep{huang2022language,ahn2022can,li2022pre,logeswaran-etal-2022-shot}.


We focus on \emph{visually grounded planning} with PLMs --- the ability to adapt plans based on interaction and visual feedback from the environment.
While PLMs have strong planning commonsense priors, predictions from a PLM may not be directly realizable in the environment since the observation and action spaces are unknown.
This requires \emph{grounding} the PLM in the environment and adapting it to observe visual feedback, which is highly non-trivial.
Some prior works assume the availability of a pre-trained affordance function \citep{ahn2022can} or a success detector \citep{mirchandani2021ella}.
Notably, SayCan \citep{ahn2022can} completely decouples the PLM from observation information by selecting actions that have both high affordability (through a pre-trained affordance model) and high PLM likelihood.
Although this partially addresses the grounding problem, the use of visual feedback for action affordance alone is limited.
Often an agent must choose one of many affordable actions using information from observations.
For example, a driving agent should re-navigate and possibly turn around when encountering a ``road closed'' sign, but both turning around and driving forward are indistinguishable to SayCan because they are both affordable and the PLM is blind to observations.

Another workaround explored in prior work is translating the information in the visual observations to text using a pre-trained captioning system \citep{shridhar2021alfworld,huang2022language}.
However, it can be difficult to faithfully describe an image in words and information is lost in this inherently noisy process, which limits the information available to the planner.



Recent work shows that PLMs can be adapted for various natural language tasks by inserting tunable embeddings or soft prompts at the input of the PLM (also called prompt tuning or prefix tuning)~\citep{li-liang-2021-prefix,lester-etal-2021-power}.
This approach also extends to multi-modal understanding tasks such as image captioning \citep{mokady2021clipcap} and VQA \citep{tsimpoukelli2021multimodal} where images are encoded as soft prompts and finetuned for the target task.
Transformer based architectures have also been successfully applied to offline Reinforcement Learning in recent work \citep{chen2021decision,janner2021offline,li2022pre,reid2022can}.

Taking inspiration from these works, we propose the simple approach of embedding visual observations (`visual prompts') and \textit{directly inserting them as PLM input embeddings}.
The visual encoder and PLM are jointly trained for the target task, an approach we call \textbf{\oursfull}~(\ours).
By teaching the PLM to use observations for planning in an end to end manner, we remove the dependency on external data such as captions and affordability information that was used in prior work.
We show that this simple approach performs better than prior PLM-based planning approaches on two embodied planning benchmarks based on ALFWorld~\citep{shridhar2021alfworld} and Virtualhome~\cite{puig2018virtualhome}.



\section{Related Work}

%Here we summarize prior work on transfer learning and property inference.

%\shortsection{Transfer Learning}
%%Transfer learning reuses features learned by pre-trained models for new tasks, with the pretext that inherent similarities in the generic features will be useful for the downstream tasks and hence reducing their cost of downstream training. Specifically, the downstream model trainer will use a pre-trained upstream model as the starting point for the downstream training, with inclusion of (or replacement with) the task-specific classification layer/module. The downstream model is then trained by either updating all layers of the model (including ones reused from upstream model) or freezing some earlier layers of the reused parts as the ``feature extractor'' and only updating the rest. The latter approach is more popular as the reused feature extractors can already learn useful feature representations and the training cost is also much lower and affordable for individuals with limited computational resources. We study the vulnerability of the latter transfer learning approach in this paper. 


%\shortsection{Transfer Learning} 
Several works have demonstrated risks associated with transfer learning across a variety of attack goals. Wang et al.~\cite{wang2018great} and Yao et al.~\cite{yao2019latent} consider manipulating the upstream model such that the fine-tuned downstream models contain backdoors, misclassifying test inputs that contain predefined backdoor triggers. These transfer manipulations are tailored to their particular attack goals and cannot be applied for the property inference goal considered in this paper. Zou et al.~\cite{zou2020privacy} study the threat of membership inference attacks on transfer learning, but with normally trained upstream models.  
%\dnote{its clear that the goals are different for these attacks, but how similar are the methods?} \ynote{similarity of the methods? more details about the methods? do not know what is expected here}
%In contrast, we investigate the possibility of boosting the effectiveness of property inference by manipulating the upstream model training. % Schuster et al.~\cite{schuster2020humpty} show that the attacker can modify the corpus on which the word embedding is trained such that the downstream NLP models which use that embedding will behave abnormally.

%\shortsection{Property Inference}
The risk of property inference was introduced by Ateniese et al.~\cite{ateniese2015hacking}, % introduces the threat of inferring properties of the training data from pre-trained models, 
and several subsequent works have developed property inference (also known as distribution inference) attacks~\cite{Wang2022GroupPI, suri2022formalizing, Jurez2022BlackBoxAF, Hartmann2022DistributionIR}.
% Ganju et al.~\cite{ganju2018property} and Suri and Evans~\cite{suri2022formalizing} 
These works study property inference against normally trained models, and they launch attacks using a variety of black-box and white-box attacks. All the white-box attacks use meta-classifiers, which take the permutation-invariant representation~\cite{ganju2018property} of the model parameters as the features. We use the state-of-the-art white-box attack~\cite{suri2022formalizing} in our experiments.
%We will use the state-of-the-art white-box method proposed by Ganju et al.~\cite{ganju2018property} and later extended by suri et al.~\cite{suri2022formalizing} in this paper.
%\dnote{do we use these attacks?} 
Melis et al.~\cite{melis2019exploiting} and Zhang et al.~\cite{zhang2021leakage} focus on property inference in distributed training scenarios. In their settings, the attacker is a participant in the global model training and conducts property inference using meta-classifiers that are trained on model outputs or gradients. Similarly, Suri et al.~\cite{suri2022subject} focus on federated learning settings where the attacker is a participant (or the central server) that utilizes black-box attacks for inferring membership of data from particular subjects. %\dnote{if we use black-box attacks, explain which ones, or how ours are related to previous ones} 
For our experiments, We improve the black-box meta-classifier proposed by Zhang et al.~\cite{zhang2021leakage} using the ``query tuning'' technique in Xu et al.~\cite{xu2019detecting}. 

The closest works to ours are Chase et al.~\cite{saeed} and Chaudhari et al.~\cite{Chaudhari2022SNAPEE}, which both consider a scenario where the attacker can manipulate some of the training data of the model to induce a model that significantly increases property inference risk.
% \dnote{it enables precise property inference attacks?}.
These works assume an adversary with the ability to poison the victim's training data, while the adversary in our scenario has no access to the victim's training data, and therefore, their methods are not applicable.
% \dnote{example how different from ours, and why the methods are not applicable}
%Thus, their methods are not applicable to our transfer learning scenario.
%Their methods rely on inducing certain behavior correlated with the properties to be inferred, and thus are not applicable to our transfer learning scenario. \anote{Still a bit unclear why that is the case.}
%
There are also works similar to ours that leverage ``adversarial initializations'' for attack purposes.
% \cite{grosse2019adversarial, boenisch2021curious, wen2022fishing, fowl2021robbing}.
Grosse et al.~\cite{grosse2019adversarial} focus on scenarios where the attacker can control the parameter initialization of a model, and demonstrate that the attacker can use special initializations to damage the performance of the trained model. %This attack is orthogonal to ours.
Other works \cite{boenisch2021curious, wen2022fishing, fowl2021robbing} show that the malicious central server in a federated learning protocol can reconstruct some training samples via falsifying the global model in some training rounds and then analyzing the submitted gradients. These kinds of attacks do not apply to our transfer-learning scenario since the attacker cannot access the downstream gradients, and can only manipulate the upstream training.

\iffalse %%%%%%%%%%%%%%%%%%%%%%%%%%%%%%%%

In this section, we provide the background and also the summary of prior attacks on transfer learning (Section~\ref{sec:transfer_learning}) and property inference (Section~\ref{sec:property_inference}). Then, we introduce the closely related manipulation attacks against machine learning models to boost different privacy risks in Section~\ref{sec:active_inference_attacks}.

%\anote{Do we really need a dedicated section for this? It's barely 2 paragraphs right now.}

%\dnote{the most closely related work to ours are works that attempt to amplify inference attacks by poisoning models, the two most relevant I know of are \url{https://www.computer.org/csdl/proceedings-article/sp/2022/131600b569/1CIO8nmuota} and \url{https://arxiv.org/abs/2204.00032}, but need to look thoroughly for others. We should definitely be describing this and relating it to our work, probably in the introduction. Most of what is here is Background, but should be clear what this section is for (not muddling background and related work)}

\subsection{Transfer Learning} \label{sec:transfer_learning}
Transfer learning reuses features learned by pre-trained models for new tasks, with the pretext that inherent similarities in generic features can be useful for downstream tasks, thus reducing the cost of downstream training. Specifically, the downstream model trainer uses a pre-trained upstream model as the starting point for downstream training, with the inclusion (or replacement) of task-specific classification layers/modules. The downstream model is then trained by either updating all layers of the model (including ones reused from the upstream model) or freezing some earlier layers of the reused parts as the ``feature extractor'' and only updating the rest. The latter approach is more popular as the reused feature extractors can already learn useful feature representations and the training cost is also much lower and affordable for individuals with limited computational resources. We study the vulnerability of the latter transfer learning approach in this paper. 
%mainly in two ways:  1) all the layers (including ones reused from ) and tune the full model; the other one is to freeze some earlier layers of the model as the feature extractor and only tune the rest later layers. The second update strategy could achieve better efficiency since the frozen layers can already produce meaningful feature representations~\cite{wang2018great,yao2019latent}, and we will study the transfer learning using this strategy. 

Recently, various attacks have been proposed for the transfer learning setting, but with different attack goals from ours. Wang et al.~\cite{wang2018great} generate adversarial examples against black-box student models that transfer knowledge from publicly available teacher models without repeated queries. Yao et al.~\cite{yao2019latent} propose to manipulate the upstream model such that the downstream models derived from the upstream model contain backdoors, which would misclassify test inputs that contain some predefined backdoor triggers. Zou et al.~\cite{zou2020privacy} study the threat of membership inference attacks on transfer learning and the upstream models are trained normally. In contrast, we investigate the possibility of boosting the effectiveness of property inference by manipulating the upstream model training. Schuster et al.~\cite{schuster2020humpty} show that the attacker can modify the corpus on which the word embedding is trained such that the downstream NLP models which use that embedding will behave abnormally.

%This additionally allows model trainers to achieve satisfactory performance with limited training samples, leading to reduced computational costs. The most common approach reuses parameters in the earlier layers of the pre-trained model, either by fixing them as the feature extractor or just using them for initialization, to conduct downstream training.

\subsection{Property Inference} \label{sec:property_inference}

\shortsection{Property Inference Attacks} In property inference attacks, the adversary aims to infer some sensitive properties of some data, given a model trained on it. For example, the adversary may be interested in sensitive properties like the presence of people of a specific race in the dataset~\cite{ateniese2015hacking, melis2019exploiting}), or even be curious about the 
the statistics of the training set (e.g, the ratio of people with a specific gender~\cite{saeed, ganju2018property, suri2022formalizing, zhang2021leakage}).


Ateniese et al.~\cite{ateniese2015hacking} were the first to identify the threat of inferring properties of the training data from pre-trained models. Ganju et al.~\cite{ganju2018property} and Suri and Evans~\cite{suri2022formalizing} 
study property inference against normally trained models, and they launch attacks using white-box meta-classifiers, which utilize the permutation-invariance representation~\cite{ganju2018property} of the model parameters, while other works focus on distributed training~\cite{zhang2021leakage} where the attacker is a participant in the global model training and conducts property inference using meta-classifiers trained on model outputs. Similarly, Suri et al.~\cite{suri2022subject} focus on federated learning, where the attacker is a participant (or the central server) that utilizes black-box attacks for inferring membership of data from particular subjects. Chase et al.~\cite{saeed} propose an active property inference attack for data poisoning scenarios, which we will cover and compare to in Section~\ref{sec:active_inference_attacks}.

%The closest work to ours are by Chase et al.~\cite{saeed} and Tramer et al.~\cite{tramer2022truth}. In their work, the attacker can manipulate some of the training data of the model such that a model trained (from scratch) on the poisoned data has an increased inference risk. However, their methods are not applicable to the transfer learning scenario. 
%In this work, we will focus on the property inference in transfer learning scenarios in which the attacker releases the upstream model and infer sensitive properties of the downstream models tuned from that upstream model.
% 

\shortsection{Defenses}
Defending against property inference attacks is an open problem. There are no studies in the current literature on active adversaries, and only a couple on passive ones. Ma et. al.~\cite{ma2021nosnoop} propose a defense against property inference attacks on data batches in the  collaborative learning setting. However, adversaries in the transfer-learning setting do not have access to batch-wise gradients of the downstream trainer. Chen and Ohrimenko~\cite{chen2022protecting} utilize mechanisms that add carefully-crafted noise to features to provide theoretical guarantees against inference adversaries, but focus on query-based access to the underlying dataset, not a machine learning model trained on it. These existing defenses thus do not apply to our threat model.

%propose a framework that reduces property inference to Boolean functions of individual members, posing the ratio of members satisfying the given function in a dataset as the property. These property inference attacks have since then been proposed as distribution inference attacks~\cite{suri2022formalizing}, presenting such attacks as inferring properties of the distributions used to sample datasets, differentiating them from exact inference attacks like dataset inference~\cite{maini2021dataset}. Nearly all property inference attacks use meta-classifiers to perform inference: training models on versions of datasets with and without the target property, followed by training a meta-classifier on top of these classifiers's model representations. These representations can take several forms: using model weights themselves with permutation-invariance~\cite{ganju2018property}, or model activations or logits for a generated set of query points~\cite{xu2019detecting}. However, the capability of such approaches is limited: the most that these attacks have been shown to work is medium-sized convolutional networks on the CelebA dataset~\cite{suri2022formalizing}.


\subsection{Active Privacy Attacks} \label{sec:active_inference_attacks}
% Perhaps the closely related works to ours as ones that proactively enhance the effectiveness of privacy attacks by manipulating the model training process in certain ways~\cite{saeed, melis2019exploiting, nasr2019comprehensive, tramer2022truth}. 
%shown that the adversary can, by using proactive ways, achieve stronger attacks that infer private information from deep learning systems~\cite{nasr2019comprehensive, melis2019exploiting, tramer2022truth, saeed}. In this section, we introduce the ones that are close to ours.

In the decentralized federated learning training, by submitting specially crafted gradients to the central server, malicious agents can increase membership inference risk~\cite{nasr2019comprehensive} and property inference risks~\cite{melis2019exploiting} of other benign agents' training data. However, these attacks do not apply to transfer learning scenario, as the attacker cannot control model gradients of downstream training. In the centralized setting, researchers propose attacks to poison the victim's training data such that the impacts of attribute inference and membership inference~\cite{tramer2022truth} and property inference~\cite{saeed} attacks are amplified on the poisoned model.
The ability to poison the victim's data is a threat model orthogonal to ours, since we have no access to the victim's downstream data. While there is scope to combine such approaches for stronger attacks (albeit with stronger access assumptions), we choose to focus on the scenario with no read/write access to the victim's data.

\fi %%%%%%%%%%%%%%%%%%%%%%%%%%%%%%%%

\section{Linear Shortcut Across Blocks}
\label{sec:layer_jump}

To use a hidden representation from layer $\ell<L$ as a final representation, we propose to cast it using linear regression, while skipping the computation in-between these layers. More generally, this approach can be applied to cast any $\ell$-th hidden representation to any subsequent layer $\ell'>\ell$.


\subsection{Method}
\label{subsec:methodology_linear_shortcut}

Given a source layer $\ell$ and a target layer $\ell'$ such that $0 \leq \ell < \ell' \leq L$, our goal is to learn a mapping
%$A_{\ell', \ell} \in \mathbb{R}^{d_h \times d_h}$
from hidden representations at layer $\ell$ to those at layer $\ell'$. To this end, we first collect a set of corresponding hidden representation pairs $(h^\ell, h^{\ell'})$. Concretely, we run a set $\mathcal{T}$ of input sequences through the model, and for each input $s$, we extract the hidden representations $h_{i_s}^{\ell}, h_{i_s}^{\ell'}$, where $i_s$ is a random position in $s$.
Next, we learn a matrix $A_{\ell', \ell} \in \mathbb{R}^{d_h \times d_h}$ by fitting linear regression over $\mathcal{T}$, i.e., $A_{\ell', \ell}$ is a numerical minimizer for:
$$ A \mapsto \sum_{s \in \mathcal{T}} || A \cdot h_{i_s}^\ell - h_{i_s}^{\ell'} ||^2,$$ 
and define the mapping of a representation $h$ from layer $\ell$ to layer $\ell'$ as:
\begin{equation}
\label{eq:linear_jump}
    \matl{} (h) \coloneqq A_{\ell', \ell} \cdot h.
\end{equation}


\subsection{Baseline}
\label{subsec:baseline}

We evaluate 
% our method against 
the prevalent approach of ``reading'' hidden representations directly, without any transformation. 
Namely, the propagation of a hidden representation from layer $\ell$ to layer $\ell'$ is given by the identity function, dubbed \id{}:

$$ \idl{} (h) \coloneqq h.$$

% Notably, 
This baseline 
assumes that representations at different layers operate in the same linear space.

\subsection{Quality of Fit}
\label{subsec:experiments_r2}

We first evaluate our method by measuring how well the learned linear mappings approximate the representations at the target layer. To this end, we calculate the (coordinate-averaged) $r^2$-score of our mapping's outputs with respect to the representations obtained from a full inference pass, and compare to the same for the \id{} baseline.


\paragraph{Models.}

We use \gpt{} \cite{radford2019language}, a decoder-only auto-regressive LM, with $L = 48$, $d_h = 1600$, and \bert{} \cite{devlin-etal-2019-bert}, an encoder-only model trained with masked language modeling, with $L=24$, $d_h=1024$.
% \footnote{\label{footnote:hf}We use models and data from Huggingface \cite{wolf-etal-2020-transformers,lhoest-etal-2021-datasets}.}
%For masked token prediction, we use a masked LM head pre-trained for our \bert{} model.

% \footnote{Specifically, we use the Huggingface Transformers \cite{wolf-etal-2020-transformers} implementations of all these models.}

%\sy{We use \gpt{} \cite{radford2019language}, a decoder-only auto-regressive LM, coming in four scales; $\texttt{gpt2}$ ($L = 12$, $d_h = 768$), $\texttt{gpt2-medium}$ ($L = 24$, $d_h = 1024$), $\texttt{gpt2-large}$ ($L = 36$, $d_h = 1280$) and $\texttt{gpt2-xl}$ ($L = 48$, $d_h = 1600$). Also, we use \bert{} \cite{devlin-etal-2019-bert}, an encoder-only model trained with masked language modeling, coming in two scales;  \texttt{bert-base-uncased} ($L=12$, $d_h=768$) and \texttt{bert-large-uncased} ($L=24$, $d_h=1024$). For masked token prediction, we use masked LM heads pre-trained for our models. Specifically, we use the Huggingface Transformers \cite{wolf-etal-2020-transformers} implementations of all these models. The plots presented in this section are for $48$-layered \gpt{} and $24$-layered \bert{}.}

%\sy{We use \gpt{} \cite{radford2019language}, a decoder-only auto-regressive LM, in the Huggingface \cite{wolf-etal-2020-transformers} implementation\footnote{\url{https://huggingface.co/gpt2}}, coming in four scales; $\texttt{gpt2}$ ($L = 12$, $d_h = 768$), $\texttt{gpt2-medium}$ ($L = 24$, $d_h = 1024$), $\texttt{gpt2-large}$ ($L = 36$, $d_h = 1280$) and $\texttt{gpt2-xl}$ ($L = 48$, $d_h = 1600$). Also, we use \bert{} \cite{devlin-etal-2019-bert}, an encoder-only model trained with masked language modeling, in the Hugginface implementation, coming in two scales;  \texttt{bert-base-uncased}\footnote{\url{https://huggingface.co/bert-base-uncased}} ($L=12$, $d_h=768$) and \texttt{bert-large-uncased}\footnote{\url{https://huggingface.co/bert-large-uncased}} ($L=24$, $d_h=1024$). For masked token prediction, we use the \texttt{BertForMaskedLM} heads from Huggingface, pretrained for these models. The plots presented in this section are for $48$-layered \gpt{} and $24$-layered \bert{}.}

\paragraph{Data.}
We sample random sentences from Wikipedia,
% \footref{footnote:hf} 
collecting 9,000 (resp. 3,000) sentences for the training set $\mathcal{T}$ (resp. validation set $\mathcal{V}$).\footnote{We use sentences rather than full documents to simplify the analysis.}
%\sy{We use two data sources to evaluate our method. One is Wikiepdia \cite{lhoest-etal-2021-datasets}\footnote{\url{https://huggingface.co/datasets/wikipedia}}; we use \texttt{spaCy}\footnote{\url{https://spacy.io/}} to divide documents into sentences\footnote{We use sentences rather than full documents to simplify the analysis.}\footnote{We pick randomly a Wikipedia document and then pick randomly a sentence ending in a newline character in it. \sy{[maybe this footnote is not needed?]}}, collecting 9,000 (resp. 3,000) random sentences for the training set $\mathcal{T}$ (resp. validation set $\mathcal{V}$). The second is a news article sentences dataset, the 10K English 2020 news sentences corpus
% \footnote{\url{https://downloads.wortschatz-leipzig.de/corpora/eng_news_2020_10K.tar.gz}} from the Leipzig Corpora Collection \cite{goldhahn-etal-2012-building}, which we randomly divide into a training set $\mathcal{T}$ consisting of 9,000 examples and a validation set $\mathcal{V}$ consisting of 1,000 examples.
% We truncate sentences to the maximal token length allowed by the model \mg{do we ever need to truncate? a sentence has about 10 words and the max. input len is thousands} \sy{[I surely did not need to in Leipzig, but discovered (via a transformers runtime warning) that I do need to for some (probably a minority) of the Wikipedia sentences. This probably has to do with that it is not really ``sentences" necessarily, for example, I noticed that it has some listings or something like that (bulleted items)... So some minority might get very long I guess...]}.
For each example $s$, we select a random position $i_s$ and extract the hidden representations $h_{i_s}^{\ell}$ at that position from all the layers.
For \bert{}, we first replace the input token at position $i_s$ with a \mask{} token, as our motivation is interpreting predictions, which are obtained via masked tokens in \bert{} (see \S\ref{subsec:BERT}).
Thus, in this case, the hidden representations we consider
%in the case of \bert{}
are of \mask{} tokens only.
%As we observed highly similar results for the two data sources across all our experiments, throughout the paper we will mainly report results for Wikipedia (except for \S\ref{sec:robustness}, where we cross-validate).


\begin{figure}[t]
\includegraphics[scale=0.2]{figs/r2_scores_48.pdf}
% \includegraphics[width=\columnwidth]{figs/r2_scores_48.pdf}
\caption{The coordinate-averaged $r^2$-score of $\matl{}$ (left) and $\idl{}$ (right) (\gpt{}).}
\label{fig:r2_scores}
\end{figure}


\begin{figure}[t]
\setlength{\belowcaptionskip}{-10pt}
\includegraphics[scale=0.2]{figs/bertmask_r2_scores_24.pdf}
% \includegraphics[width=\columnwidth]{figs/bertmask_r2_scores_24.pdf}
\caption{The coordinate-averaged $r^2$-score of $\matl{}$ (left) and $\idl{}$ (right) (\bert{}).}
\label{fig:bertmask_r2_scores}
\end{figure}



\paragraph{Evaluation.}
For every pair of layers $\ell, \ell'$, such that $0 \leq \ell < \ell' \leq L$, we use the training set $\mathcal{T}$ to fit linear regression as described in \S\ref{subsec:methodology_linear_shortcut}, and obtain a mapping $\matl{}$. 
Next, we evaluate the quality of $\matl{}$ as well as of $\idl{}$ using the $r^2$-coefficient, uniformly averaged over all coordinates. Concretely, we compute the $r^2$-coefficient of each of the predicted representations $\matl{} (h_{i_s}^{\ell})$ and $\idl{} (h_{i_s}^{\ell})$ versus the true representations $h_{i_s}^{\ell'}$
over all $s \in \mathcal{V}$.
%as we vary $s \in \mathcal{V}$.
%for every $s \in \mathcal{V}$.



\paragraph{Results.}
Results for \gpt{} and \bert{} are presented in Figs.~\ref{fig:r2_scores} and~\ref{fig:bertmask_r2_scores}, respectively.
In both models, \mat{} consistently yields better approximations than \id{}, as it obtains higher $r^2$-scores (in blue) across the network. 
This gap between \mat{} and \id{} is especially evident in \bert{}, where \id{} completely fails to map the representations between most layers, suggesting that hidden representations are modified  substantially by every transformer block.
Overall, this highlights the shortcoming of existing practices to inspect representations in the same linear space, and the gains from using our method to approximate future layers.
% in the network.
\section{Linear Shortcut for Language Modeling}
\label{sec:prediction}

We saw that our method approximates future hidden representations substantially better than a naive propagation. 
In this section, we will show that this improvement also translates to better predictive abilities from earlier layers. Specifically, we will use our method to estimate how often intermediate representations encode the final prediction, in the context of two fundamental LM tasks; next token prediction and masked token prediction.

\paragraph{Evaluation Metrics.}
Let $h, h' \in \mathbb{R}^{d_h}$ be a final representation and a substitute final representation obtained by some mapping, and denote by $\delta (h), \delta (h') \in \mathbb{R}^{d_v}$ their corresponding output probability distributions (obtained through projection to the output vocabulary -- see details below). 
We measure the prediction quality of $h'$ with respect to $h$ using two metrics:
\begin{itemize}
[leftmargin=*,topsep=1pt,parsep=1pt]
    \item \textbf{Precision@$k$} ($\uparrow$ is better): This checks whether the token with the highest probability according to $\delta(h')$ appears in the top-$k$ tokens according to $\delta(h)$. Namely, we sort $\delta(h)$ and assign a score of $1$ if $\arg\max(\delta(h'))$ appears in the top-$k$ tokens by $\delta(h)$, and $0$ otherwise.
    
    \item \textbf{Surprisal} ($\downarrow$ is better): We measure the minus log-probability according to $\delta(h)$, of the highest-probability token according to $\delta(h')$. Intuitively, low values mean that the model sees the substitute result as probable and hence not surprising.
\end{itemize}

\noindent We report the average Precision@$k$ and Surprisal over the validation set $\mathcal{V}$.



\subsection{Next Token Prediction}
\label{subsec:next_token_prediction_task}

Auto-regressive LMs output for every position a probability distribution over the vocabulary for the next token. Specifically, the output distribution for every position $i$ is given by $\delta (h_i^L)$, where:
\begin{equation}\label{eq:output_distribution}
    \delta (h) = \texttt{softmax} ( E^\top \cdot h) \in \mathbb{R}^{d_v}
\end{equation}
For some LMs, including \gpt{}, a layer normalization $\texttt{ln\_f}$ is applied to the final layer representation before this conversion (i.e., computing $\delta (\texttt{ln\_f}(h))$ rather than $\delta (h)$).

Recall that our goal is to measure how well this distribution can be estimated from intermediate representations, i.e. estimating $\delta (h_i^L)$ from $\delta (h_i^\ell)$ where $\ell<L$. To this end, we first run examples from the validation set through the model, while extracting for each example $s$ the hidden representation of a random position $i_s$ at every layer. Next, we apply our mappings $\matlL{}$ and the $\idlL{}$ baseline to cast the hidden representations of every layer $\ell$ to final layer substitutes (see \S\ref{sec:layer_jump}). Last, for each layer, we convert its corresponding final-layer substitute to an output distribution (Eq.~\ref{eq:output_distribution}) and compute the average Precision@$k$ (for $k=1,5,10$) and Surprisal scores with respect to the final output distribution, over the validation set.

\paragraph{Results.}
Figs.~\ref{fig:pre} and~\ref{fig:surp} show the average Precision@$k$ and Surprisal scores per layer in $48$-layered \gpt{}, respectively (the plots for the other \gpt{} models are presented in \S\ref{sec:app_scale}). Across all layers, \mat{} outperforms \id{} in terms of both scores, often by a large margin (e.g. till layer $44$ the Precision@$1$ achieved by \mat{} is bigger than that of $\id{}$ by more than $0.2$). 
This shows that linear mappings enable not just better estimation of final layer representations, but also of the predictions they induce. Moreover, the relatively high Precision@$k$ scores of \mat{} in early layers ($0.62$-$0.82$ for $k=10$, $0.52$-$0.74$ for $k=5$, and $0.28$-$0.45$ for $k=1$) suggest that early representations already encode a good estimation of the final prediction. Also, the substantially lower Surprisal scores of \mat{} compared to \id{} imply that our method allows for a more representative reading into the layer-wise prediction-formation of the model than allowed through direct projection to the vocabulary.

\begin{figure}[t]
\centering
\includegraphics[scale=0.4]{figs/pre_48.pdf}
\caption{Precision@$k$ ($k = 1,5, 10$) of $\matlL{}$ and $\idlL{}$ for next token prediction in $48$-layered \gpt{}.}
\label{fig:pre}
\end{figure}

\begin{figure}[t]
\centering
\includegraphics[scale=0.35]{figs/surp_48.pdf}
\caption{Surprisal for $\matlL$ and the baseline $\idlL{}$ ($48$-layered \gpt{} next token prediction task). A 95\% confidence interval surrounds the lines.}
\label{fig:surp}
\end{figure}

\subsection{Masked Token Prediction}
\label{subsec:BERT}

We now conduct the same experiment for the task of masked language modeling, where the model predicts a probability distribution of a masked token in the input rather than the token that follows the input. Unlike next token prediction, where the output distribution is computed from representations of varying input tokens, in masked token prediction the output is always obtained from representations of the same input token (i.e. \texttt{[MASK]}).

For this experiment, we use \bert{}, on top of which we use a pretrained masked language model head $\delta$; given a token sequence $s$, a \mask{} token inside it and its final representation $h$, $\delta (h) \in \mathbb{R}^{d_v}$
 is a probability distribution over tokens giving the model's assessment
 of the likelihood of tokens to be fitting in place of the \mask{} token in $s$.


\begin{figure}[t]
\centering
\includegraphics[scale=0.4]{figs/bertmask_pre_24.pdf}
\caption{Precision@$k$ ($k = 1,5, 10$) for  $\matlL{}$ and the baseline $\idlL{}$ ($24$-layered \bert{} masked token prediction task).}
\label{fig:bertmask_pre}
\end{figure}

\begin{figure}[t]
\centering
\includegraphics[scale=0.35]{figs/bertmask_surp_24.pdf}
\caption{Surprisal for $\matlL{}$ and the baseline $\idlL{}$ ($24$-layered \bert{} masked token prediction task). A 95\% confidence interval surrounds the lines.}
\label{fig:bertmask_surp}
\end{figure}

\paragraph{Results.}
Figs.~\ref{fig:bertmask_pre} and~\ref{fig:bertmask_surp} present the average Precision@$k$ and Surprisal scores per layer in $24$-layered \bert{} (the plots for the $12$-layered \bert{} model are presented in \S\ref{sec:app_scale}), overall showing trends similar to those observed for next token prediction in \gpt{} (\S\ref{subsec:next_token_prediction_task}). This is despite the differences between the two tasks and the considerable architectural differences between \bert{} and \gpt{}.
Notably, the superiority of \mat{} over \id{} in this setting is even more prominent; 
while \mat{}'s precision is between $0.2-0.6$ in the first ten layers (Fig.~\ref{fig:bertmask_pre}), \id{}'s precision for all values of $k$ is close to zero, again strongly indicating that our method allows for better reading into early layer hidden representations. 
More generally, \mat{} improves the Precision@$1$ of \id{} by more than $17\%$ at most layers, and unveils that a substantial amount of predictions ($>25\%$ starting from layer $3$) appear already in the very first layers.
Interestingly, the (rough) divide between the first half of layers and last half of layers for $\id{}$ in Figs.~\ref{fig:bertmask_pre},~\ref{fig:bertmask_surp} seems to align with the two-hump shape of the blue region for $\mat{}$ in Fig.~\ref{fig:bertmask_r2_scores}.

\paragraph{Analysis.}
We manually compare the predictions of our mapping $\matlL{}$ with $\idlL{}$, for a $24$-layered \bert{} model.  Concretely, we select 50 random sentences from the Leipzig dataset. Next, for each layer $\ell$, we manually analyze how many of the top-$5$ tokens according to $\matlL{}$ and $\idlL{}$ fit into context. We consider a token to fit into context if it is grammatically plausible within the sentence (see Tab.~\ref{tab:manual} for concrete examples).
In the resulting $1250$ instances (i.e. $50$ sentences $\times$ $25$ representations), we observe a substantially higher plausibility rate of $85.36\%$ for \mat{} compared to $52.8\%$ for \id{}. In fact, only in less than $4.3\%$ of the instances there are more plausible tokens among the top-$5$ tokens according to \id{} than among the top-$5$ tokens according to \mat{}, further supporting the Surprisal results above.

\begin{table*}
\footnotesize
\setlength{\belowcaptionskip}{-15pt}
\begin{tabular}{p{0.3\linewidth}ccccc}
& $\texttt{id}_{4 \rightarrow 24}$ & $\texttt{mat}_{4 \rightarrow 24}$ & $\texttt{id}_{12 \rightarrow 24}$ & $\texttt{mat}_{12 \rightarrow 24}$ & $\texttt{id}_{24 \rightarrow 24}$ \\ \midrule
\multirow{5}{=}{aldridge had shoulder surgery in \mask{}.} & fellowship & \tcbox{time} & cyclist & \tcbox{2009} & \tcbox{september} \\
& employment & \tcbox{it} & emergencies & \tcbox{2008} & \tcbox{november} \\
& agreement & her & seniors & \tcbox{2010} & \tcbox{december} \\
& \#\#ostal & them & cycling & \tcbox{2006} & \tcbox{august} \\
& \#\#com & work & \tcbox{pennsylvania} & \tcbox{2007} & \tcbox{july} \\ \midrule
\multirow{5}{=}{on your next view you will be asked to \mask{} continue reading.} & \#\#com & be & be & be & \tcbox{please} \\
& accreditation & get & undergo & \tcbox{please} & \tcbox{simply} \\ 
& $	\copyright$ & go & spartans & help & \tcbox{also} \\ 
& fellowship & \tcbox{help} & seniors & \tcbox{simply} & \tcbox{again} \\ 
& summer & have & * & say & \tcbox{immediately} \\ \bottomrule
\end{tabular}
\caption{Examples of top-$5$ predictions at layers $4$, $12$ and $24$, under the mappings $\matlL{}$ and $\idlL{}$, for a $24$-layered \bert{} model. Grammatically plausible predictions (according to a human annotator) are marked in \tcbox{blue}. Note that at layer $24$ the predictions of $\matlL{}$ and $\idlL{}$ are the same (by definition).} 
\label{tab:manual}
\end{table*}

\section{Implication to Early Exiting}
\label{sec:applications}

%The fact that it is often possible to approximate
The possibility of approximating
the final prediction already in the early layers has important implications for efficiency; applying our linear mapping instead of executing transformer blocks of quadratic time complexity, could save a substantial portion of the computation. In this section, we demonstrate this in the context of early exiting.

When 
% performing transformer model inference under 
using an early exit strategy \cite{schwartz-etal-2020-right, xin-etal-2020-deebert, schuster2022confident}, one aims at deciding dynamically at which layer to stop the computation and ``read'' the prediction from the hidden representation of that layer.
More precisely, under a confidence measure paradigm, one decides to stop the computation for a position $i$ at layer $\ell$ based on a confidence criterion, that is derived from casting the hidden representation $h_i^\ell$ as a final-layer representation and converting it to an output probability distribution. Specifically, following \citet{schuster2022confident}, a decision to exit is made if the difference between the highest and the second highest probabilities is bigger than $$ 0.9 \cdot \lambda + 0.1 \cdot {\rm exp} (-4 i / N),$$
where $N$ is the average length of the input until position $i_s$ for $s \in \mathcal{V}$, and $\lambda$ is a hyper-parameter.

\begin{figure}[t]
\setlength{\belowcaptionskip}{-10pt}
\centering
\includegraphics[width=\columnwidth]{figs/ee_gpt2bert.pdf}
\caption{Precision@$1$ with early exit and ``fixed exit'', applied to the $24$-layer \gpt{} for next token prediction (left) and the $24$-layer \bert{} for masked token prediction (right). Varying the confidence parameter $\lambda$, the $x$-coordinate is the average number of layers processed before an early exit decision is reached.}
\label{fig:ee_gpt2bert}
\end{figure}

\quash{
\begin{figure}[t]
\setlength{\belowcaptionskip}{-10pt}
\centering
\includegraphics[scale=0.35]{figs/ee_pre1_24.pdf}
\caption{Precision@$1$ for the various early exit methods, and previous ``fixed exit'' methods for comparison ($24$-layer \gpt{} next token prediction task). Varying the confidence parameter $\lambda$, the $x$-coordinate is the average number of layers processed before an early exit decision is reached.}
\label{fig:ee_pre1}
\end{figure}
}

\paragraph{Experiment.}
We assess the utility of our mapping $\matlL{}$ for early exit as a plug-and-play replacement for $\idlL{}$, through which intermediate representations are cast into final-layer representations.
We use \gpt{} for the next token prediction and \bert{} for masked token prediction (both with 24 layers).
We run each of the models over the validation set examples, while varying the confidence parameter $\lambda$ and using either $\idlL{}$ or $\matlL{}$ for casting intermediate representations.
Furthermore, we compare these early exit variants to the ``fixed exit'' strategy from \S\ref{sec:prediction}, where the computation is stopped after a pre-defined number of layers rather than relying on a dynamic decision.
We evaluate each variant in terms of both prediction's accuracy, using the Precision@$1$ metric (see \S\ref{sec:prediction}), and efficiency, measured as the average number of transformer layers processed during inference.


\paragraph{Results.}
%Figs.~\ref{fig:ee_pre1} and~\ref{fig:bertmask_ee_pre1}
Fig.~\ref{fig:ee_gpt2bert}
plots the average Precision@$1$ score against the average number of layers processed, for $24$-layer \gpt{} and $24$-layer \bert{}. For both models, under an early exit strategy our mapping \mat{} again provides a substantial improvement over \id{}.
For example, aiming at $95\%$ average precision, \mat{} saves $\sim3.3$ ($13.8$\%) layers in \gpt{} compared to only $\sim1.4$ ($5.9$\%) layers by \id{}, and $\sim4.8$ ($20$\%) layers in \bert{} versus $\sim3.5$ ($14.6$\%) layers by \id{}.
These results highlight the potential gains prominent early exit methods can obtain by using our method.
Notably, in both models and for each of the mapping methods, early exit obtains better results than fixed layer exit, as expected. 

\quash{
\begin{figure}[t]
\setlength{\belowcaptionskip}{-10pt}
\centering
\includegraphics[scale=0.35]{figs/bertmask_ee_pre1_24.pdf}
\caption{Precision@$1$ for the various early exit methods, and previous ``fixed exit'' methods for comparison ($24$-layer \bert{} masked token prediction task). Varying the confidence parameter $\lambda$, the $x$-coordinate is the average number of layers processed before an early exit decision is reached.}
\label{fig:bertmask_ee_pre1}
\end{figure}
}
\section{Linear Shortcut Across Sub-Modules}
\label{sec:submodules}

% Our experiments show that
% , despite the commonly-applied simplification by interpretability works, transformer layers do not operate in the same linear space and 
% there is a major gap in approximating future representations using an identity mapping (\S\ref{sec:layer_jump}, \S\ref{sec:prediction}).
% Here, 
In this section, we investigate whether discrepancies across layers result from specific sub-modules or are a general behaviour of all sub-modules in the network.  
This is done by extending our approach to test how well particular components in transformer blocks can be linearly approximated. 


\paragraph{Method.}

Consider \gpt{} for definiteness, then:
% we have 
$$ \texttt{b}_{\ell} = \texttt{b}_{\ell}^{\texttt{ffn}} \circ \texttt{b}_{\ell}^{\texttt{attn}}$$ 
% with
\begin{equation}\label{eq:attn} \texttt{b}^{\texttt{attn}}_{\ell} (H) = \texttt{attn}_{\ell} (\texttt{ln1}_{\ell} (H)) + H,\end{equation} 
where $\texttt{attn}_{\ell}$ is
%a multi-head self-attention
a MHSA
layer and \texttt{ln1} is a layer normalization (LN), and 
$$ \texttt{b}^{\texttt{ffn}}_{\ell} (H) = \texttt{ffn}_{\ell} (\texttt{ln2}_{\ell} (H)) + H,$$  
where $\texttt{ffn}_{\ell}$ is
%a feed-forward network
an FFN
layer and $\texttt{ln2}$ is a LN.
\quash{
Given a block $\texttt{b}_\ell$ and one of its sub-modules $\texttt{ln1}_\ell, \ \texttt{attn}_\ell, \ \texttt{ln2}_\ell$, or $\texttt{ffn}_\ell$, we fit linear regression approximating the output of the sub-module given its input and then use it in order to define mappings, as we now describe.
}
Given a block $\texttt{b}_\ell$ and one of its sub-modules $\texttt{ln1}_\ell, \ \texttt{attn}_\ell, \ \texttt{ln2}_\ell$, or $\texttt{ffn}_\ell$, we fit linear regression approximating the output of the sub-module given its input, and then use it to define mappings $\matattnl{}$, $\matlnl{}$ and $\matffl{}$.
%We provide the definition of $\matattnl{}$ below, and that of the other two in App. \ref{sec:app_submodule_skip_description}.
We provide the formal definitions of these mappings in App. \ref{sec:app_submodule_skip_description}.
\iffalse
\paragraph{$\matattnl{}$.}
%Illustrating this on $\texttt{attn}_\ell$ for definiteness,
For an input $s$, let $v^\ell_{i_s}$ be the vector at position $i_s$ in the output of $\texttt{attn}_\ell (\texttt{ln1}_\ell (H^{\ell - 1}))$. We denote by $A_\ell^{\texttt{attn}} \in \mathbb{R}^{d_h \times d_h}$ the matrix numerically minimizing 
$$ A \mapsto \sum_{s \in \mathcal{T}} || A \cdot \texttt{ln1}_\ell (h^{\ell-1}_{i_s}) - v^\ell_{i_s}||^2,$$
and define an attention sub-module replacement (Eq.~\ref{eq:attn}) by $$
\texttt{b}^{\overline{\texttt{attn}}}_\ell (h) \coloneqq A_{\ell}^{\texttt{attn}} \cdot \texttt{ln1}_\ell (h) + h. $$
We then define a mapping between two layers ${\ell \rightarrow \ell'}$ by:
$$ \matattnl{} (h) \coloneqq $$
$$ \texttt{b}^{\texttt{ffn}}_{\ell'} ( \texttt{b}^{\overline{\texttt{attn}}}_{\ell'} ( \ldots (\texttt{b}^{\texttt{ffn}}_{\ell+1} ( \texttt{b}^{\overline{\texttt{attn}}}_{\ell+1} (h)))\ldots)).$$ 
Namely, when applying each $\ell''$-th block, $\ell < \ell'' \leq \ell'$, we replace its attention sub-module $\texttt{attn}_{\ell''}$ by its linear approximation.
%In an analogous way, we consider the mappings $\matffl{}$ and $\matlnl{}$, where in the latter we perform the linear shortcut both for \texttt{ln1} and for \texttt{ln2} (see~\S\ref{sec:app_submodule_skip_description} for precise descriptions).
Importantly, unlike the original attention module, the approximation $\texttt{b}^{\overline{\texttt{attn}}}_\ell$ operates on each position independently, and therefore applying $\matattnl{}$ disables any contextualization between the layers $\ell$ and $\ell'$. Note that this is not the case for $\matffl{}$ and $\matlnl{}$, which retain the self-attention sub-modules and operate contextually.
\fi

\paragraph{Evaluation.}


We analyze the $24$-layered \gpt{}, and proceed completely analogously to \S\ref{subsec:next_token_prediction_task}, evaluating the Precision@$1$ and Surprisal metrics for the mappings $\matattnlL{}$, $\matfflL{}$ and $\matlnlL{}$.

\begin{figure}[t]
\setlength{\belowcaptionskip}{-0pt}
\centering
%\includegraphics[scale=0.2]
\includegraphics[width=\columnwidth]{figs/parts_presurp_24.pdf}
\caption{Precision@$1$ and Surprisal for the various sub-module linear mappings, and $\matlL{}$ for comparison ($24$-layer \gpt{} next token prediction task). A 95\% confidence interval surrounds the Surprisal lines.}
\label{fig:parts_presurp}
\end{figure}

\quash{
\begin{figure}[t]
\centering
\includegraphics[scale=0.4]{figs/parts_pre1_24.pdf}
\caption{Precision@$1$ for the various sub-module linear shortcut mappings, and the mapping $\matlL{}$ for comparison (\gpt{} next token prediction task).}
\label{fig:parts_pre1}
\end{figure}

\begin{figure}[t]
\centering
\includegraphics[scale=0.35]{figs/parts_surp_24.pdf}
\caption{Surprisal for the various sub-module linear shortcut mappings, and the mapping $\matlL{}$ for comparison (\gpt{} next token prediction task). A 95\% confidence interval surrounds the lines.}
\label{fig:parts_surp}
\end{figure}
}

\paragraph{Results.}
Fig.~\ref{fig:parts_presurp} shows the average Precision@$1$ and Surprisal scores per layer.
From a certain layer (\textasciitilde$7$), all sub-module mappings achieve better results than the full-block mapping $\matlL{}$. Thus, it is not just the cumulative effect of all the sub-modules in the transformer block that is amenable to linear approximation, but also individual sub-modules can be linearly approximated. 
Furthermore, the linear approximation of attention sub-modules is less harmful than that of the FFN or LN sub-modules. 
% Hypothetically, 
A possible reason is that the linear replacement of FFN or LN ``erodes'' the self-attention computation after a few layers. 
Moreover, the good performance of $\matattnlL{}$ suggests that contextualization often exhausts itself in early layers; speculatively, it is only in more delicate cases that the self-attention of late layers adds important information. Last, remark the sharp ascent of the scores for layer normalization in layers $5$-$8$, for which we do not currently see a particular reason. To conclude, we see that the possibility of linear approximation permeates
%the various
transformer components.


\section{Related Work}

Recently, there was a lot of interest in utilizing intermediate representations in transformer-based LMs, both for interpretability and for efficiency.

In the direction of interpretability, one seeks to understand the prediction construction process of the model \cite{tenney-etal-2019-bert, voita-etal-2019-bottom}.

More recent works use mechanistic interpretability and view the inference pass as a residual stream of information \cite{dar2022analyzing,geva-etal-2022-transformer}. Additionally, there are works on probing, attempting to understand what features are stored in the hidden representations \cite{adi2017finegrained, conneau-etal-2018-cram,liu-etal-2019-linguistic}. Our work is different in that it attempts to convert intermediate representations into a final-layer form, which is interpretable by design.

In the direction of efficiency, there is the thread of work on early exit, where computation is cut at a dynamically-decided earlier stage \cite{schwartz-etal-2020-right,xin-etal-2020-deebert,schuster2022confident}. Other works utilize a fixed early stage network to parallelize inference \citep{leviathan2022fast, chen2023accelerating}. However, intermediate representations are directly propagated in these works, which we show is substantially worse than our approach. Moreover, our method requires training considerably less parameters than methods such as \citet{schuster-etal-2021-consistent}, that learn a different output softmax for each intermediate layer.  

More broadly, skipping transformer layers and analyzing the linearity properties of transformer components have been discussed in prior works \cite{Zhao2021of,mickus-etal-2022-dissect,wang-etal-2022-skipbert,lamparth2023analyzing}.


\section{Conclusion and Future Work}

We present a simple and effective method for enhancing utilization of hidden representations in transformer-based LMs, that uses 
pre-fitted context-free and token-uniform linear mappings.
Through a series of experiments on different data sources, model architectures and scales, we show that our method consistently outperforms the prevalent practice of interpreting representations in the final-layer space of the model, yielding better approximations of succeeding representations and the predictions they induce, thus allowing a more faithful interpretation of the model's prediction-formation.
We demonstrate the practicality of our method for improving computation efficiency, saving a substantial amount of compute on top of prominent early exiting approaches. 
Also, by extending our method to sub-modules, 
% more specifically the attention sub-modules, 
we observe that replacing a part of the transformer inference by a non-contextual linear computation often results in a small deterioration of the prediction.
This opens new research directions for improving model efficiency,
% and parallelizability.
% including breaking the computation into several parallelizable tasks.
including breaking the computation into parallel tasks.

\section*{Limitations}

Although we see in this work that there is more linear structure to transformer inference than could be explained solely by the residual connection, we do not elucidate a reason for that. We also do not try to formulate formal criteria according to which to judge, in principle, the quality of ways of short-cutting transformer inference in-between layers. In addition, our experiments cover only English data.


%\section*{Ethics Statement}
%Scientific work published at ACL 2023 must comply with the ACL Ethics Policy.\footnote{\url{https://www.aclweb.org/portal/content/acl-code-ethics}} We encourage all authors to include an explicit ethics statement on the broader impact of the work, or other ethical considerations after the conclusion but before the references. The ethics statement will not count toward the page limit (8 pages for long, 4 pages for short papers).

\section*{Acknowledgements}

We thank Tal Schuster for constructive comments.

% Entries for the entire Anthology, followed by custom entries
\bibliography{anthology,custom}
\bibliographystyle{acl_natbib}

\appendix

\section{Descriptions of $\matattn{}$, $\matff{}$ and $\matln{}$}
\label{sec:app_submodule_skip_description}

Here we detail the definitions of the mappings $\matattnl{}$, $\matffl{}$ and $\matlnl{}$ utilized in \S\ref{sec:submodules}.

\paragraph{Description of $\matattnl{}$.}
%Illustrating this on $\texttt{attn}_\ell$ for definiteness,
For an input $s$, let $v^\ell_{i_s}$ be the vector at position $i_s$ in the output of $\texttt{attn}_\ell (\texttt{ln1}_\ell (H^{\ell - 1}))$. We denote by $A_\ell^{\texttt{attn}} \in \mathbb{R}^{d_h \times d_h}$ the matrix numerically minimizing 
$$ A \mapsto \sum_{s \in \mathcal{T}} || A \cdot \texttt{ln1}_\ell (h^{\ell-1}_{i_s}) - v^\ell_{i_s}||^2,$$
and define an attention sub-module replacement (Eq.~\ref{eq:attn}) by $$
\texttt{b}^{\overline{\texttt{attn}}}_\ell (h) \coloneqq A_{\ell}^{\texttt{attn}} \cdot \texttt{ln1}_\ell (h) + h. $$
We then define a mapping between two layers ${\ell \rightarrow \ell'}$ by:
$$ \matattnl{} (h) \coloneqq $$
$$ \texttt{b}^{\texttt{ffn}}_{\ell'} ( \texttt{b}^{\overline{\texttt{attn}}}_{\ell'} ( \ldots (\texttt{b}^{\texttt{ffn}}_{\ell+1} ( \texttt{b}^{\overline{\texttt{attn}}}_{\ell+1} (h)))\ldots)).$$ 
Namely, when applying each $\ell''$-th block, $\ell < \ell'' \leq \ell'$, we replace its attention sub-module $\texttt{attn}_{\ell''}$ by its linear approximation.
%In an analogous way, we consider the mappings $\matffl{}$ and $\matlnl{}$, where in the latter we perform the linear shortcut both for \texttt{ln1} and for \texttt{ln2} (see~\S\ref{sec:app_submodule_skip_description} for precise descriptions).
Importantly, unlike the original attention module, the approximation $\texttt{b}^{\overline{\texttt{attn}}}_\ell$ operates on each position independently, and therefore applying $\matattnl{}$ disables any contextualization between the layers $\ell$ and $\ell'$. Note that this is not the case for $\matffl{}$ and $\matlnl{}$, which retain the self-attention sub-modules and operate contextually.

\paragraph{Description of $\matffl{}$.}
Let $v^\ell_{i_s}$ be the vector at position $i_s$ in the output of $\texttt{ln2}_{\ell} (\texttt{b}_\ell^{\texttt{attn}} (H^{\ell - 1}))$, for a given input $s$. We denote by $A_\ell^{\texttt{ffn}} \in \mathbb{R}^{d_h \times d_h}$ the matrix numerically minimizing 
$$ A \mapsto \sum_{s \in \mathcal{T}} || A \cdot v^{\ell}_{i_s} - \texttt{ffn}_{\ell} (v^\ell_{i_s})||^2,$$
and define a replacement of the feed-forward sub-module $\texttt{b}_{\ell}^{\texttt{ffn}}$ by $$ \texttt{b}^{\overline{\texttt{ffn}}}_\ell (H) \coloneqq A_{\ell}^{\texttt{ffn}} \cdot \texttt{ln2}_\ell (H) + H.$$
We then define a mapping between two layers ${\ell \rightarrow \ell'}$ by:
$$ \matffl{} (H) \coloneqq $$
$$ \texttt{b}^{\overline{\texttt{ffn}}}_{\ell'} ( \texttt{b}^{\texttt{attn}}_{\ell'} ( \ldots (\texttt{b}^{\overline{\texttt{ffn}}}_{\ell+1} ( \texttt{b}^{\texttt{attn}}_{\ell+1} (H))\ldots)).$$

\paragraph{Description of $\matlnl{}$.}
Let $v^\ell_{i_s}$ be the vector at position $i_s$ in the output of $\texttt{b}^{\texttt{attn}}_{\ell} (H^{\ell - 1})$, for a given input $s$. We denote by $A_\ell^{\texttt{ln1}} \in \mathbb{R}^{d_h \times d_h}$ the matrix numerically minimizing 
$$ A \mapsto \sum_{s \in \mathcal{T}} || A \cdot h^{\ell}_{i_s} - \texttt{ln1}_{\ell} (h^\ell_{i_s})||^2$$ and we denote by $A_\ell^{\texttt{ln2}} \in \mathbb{R}^{d_h \times d_h}$ the matrix numerically minimizing $$ A \mapsto \sum_{s \in \mathcal{T}} || A \cdot v^{\ell}_{i_s} - \texttt{ln2}_{\ell} (v^\ell_{i_s})||^2.$$ We define a replacement of the block $\texttt{b}^{\texttt{attn}}_{\ell}$ by \begin{equation} \texttt{b}^{\overline{\texttt{ln1}}}_\ell (H) \coloneqq \texttt{attn}_{\ell} (A_{\ell}^{\texttt{ln1}} \cdot H) + H\end{equation} and we define a replacement of the block $\texttt{b}^{\texttt{ffn}}_{\ell}$ by \begin{equation} \texttt{b}^{\overline{\texttt{ln2}}}_\ell (H) \coloneqq \texttt{ffn}_{\ell} (A_{\ell}^{\texttt{ln2}} \cdot H) + H.\end{equation}
We then define a mapping between two layers ${\ell \rightarrow \ell'}$ by:
$$ \matlnl{} (H) \coloneqq $$
$$ \texttt{b}^{\overline{\texttt{ln2}}}_{\ell'} ( \texttt{b}^{\overline{\texttt{ln1}}}_{\ell'} ( \ldots (\texttt{b}^{\overline{\texttt{ln2}}}_{\ell+1} ( \texttt{b}^{\overline{\texttt{ln1}}}_{\ell+1} (H))\ldots)).$$


\end{document}


%%%%%%%%%%%%%%%%%%%%%%%%%%%%%%%%%%%%%%%%%%%%%%%%%%%%%%%%%%%%%%%%%%%%%%%%

\end{document}

%%%%%%%%%%%%%%%%%%%%%%%%%%%%%%%%%%%%%%%%%%%%%%%%%%%%%%%%%%%%%%%%%%%%%%%%

