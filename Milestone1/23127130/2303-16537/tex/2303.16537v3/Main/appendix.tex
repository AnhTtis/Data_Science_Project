\section{Algorithms}

We present the details of two fundamental algorithms of \methodname{}. Algorithm ``Construct Element-graph'' shows  the procedure for generating a pruned element-graph \(G_e\) from the input content \(z\). It utilizes LM and ConceptNet to extract relevant nodes, evaluates them to compute a pruning score, and selects the top \(K\) nodes to form \(G_e\).

Algorithm ``Element-graph Interpretation'' processes the pruned element-graph \(G_e\) to identify essential ``Reason-elements'' that enhancing the understanding of \(z\). It uses a graph attention network, which utilizes attention to analyze node interactions, updates node features, and calculates the probability of selecting specific outcomes. This process results in the identification and ranking of nodes to highlight important ``Reason-elements.''


\begin{algorithm} [H]
\DontPrintSemicolon
\KwData{Input content $z$}
\KwResult{Pruned element-graph $G_e$}
\caption{Construct Element-graph}
\label{algo:subgraph}

\Begin{
    $G_k \gets \text{ExtractFromConceptNet}(z)$ \\ \tcp*[r]{Extract the $L$-hop neighbor from ConceptNet}
    \For{each node $v_e$ in $G_k$}{
        % $v_{\text{embedding}} \gets f_{\text{enc}}(v)$ \tcp*[r]{Embedding function of $f_{LM}$} 
     %   $v_{\text{embedding}} \gets z_{emb} \mathbin\Vert v_{emb}$ \tcp*[r]{Concat text embeddings}
        $v_{\text{score}} \gets f_{prob}(z_{emb},v_{emb})$ \tcp*[r]{Compute score for pruning}
    }
    $G_e \gets \text{SelectTopK}(G_k)$ \tcp*[r]{Prune based on top $K$ scores}
    \Return{$G_e$}\;
}
\end{algorithm}


\begin{algorithm}[H]
\DontPrintSemicolon
\KwData{Element-graph \( G_e \)  containing node type embedding \(u_i\) and relation embedding \(r_{ij}\), input \(z\).}
\KwResult{\emph{Reason-elements}}
\caption{Element-graph Interpretation}
\label{algo:element}

\Begin{
        \For{each attention layer \( k \) in graph reasoning network}{
        \For{each node \( v_i \) in \( G_e \)}{
      \(\alpha_{ij}  \gets  \frac{\exp(a(h_{ki},h_{kj},u_i,r_{ij}))}{\sum \limits_{v_l \in \mathcal{N}_i } \exp(a(h_{ki},h_{kl}))}\) \tcp*[r]{Compute attention coefficient \(\alpha_{ij}\)}
     \(  h_{k+1,i}  \gets  f_{\delta}\left(\sum\limits_{v_j\in \mathcal{N}_i}   \alpha_{ij} m(h_{kj},u_i,r_{ij}) \right) + h_{ki} \) \tcp*[r]{Update node feature}
      }
    }
  %  \( \Lambda \gets \text{AttentionPooling}(G_e) \) \tcp*[r]{Attention-based pooling}
   % \( \mathbb{H}^{itp} \gets h_{v_i}^k \parallel \Lambda,\; \mathbb{H}^{LM} \gets f_{LM}(z) \) \tcp*[r]{Forming \(\mathbb{H}^{itp}\) and \(\mathbb{H}^{LM}\)}
  %  \( P(a|q) \propto \exp(\text{MLP}(\mathbb{H}^{LM}, \mathbb{H}^{itp})) \) \tcp*[r]{Probability of choosing an answer}
 \(\mathbb{H}^{LM} \gets f_{LM}(z) \) \tcp*[r]{Forming  \(\mathbb{H}^{LM}\)}
 \( P(y'|q) \propto \exp(\text{MLP}(\mathbb{H}^{LM}, \boldsymbol{h}_K,\boldsymbol{\alpha}_K)) \) \tcp*[r]{Probability of choosing an answer}
 \( \text{ReasonElements} \gets \text{RankNode}(G_e, \boldsymbol{\alpha}_K)) \) \tcp*[r]{Rank nodes based on the attentions}
    

    
    \Return{ReasonElements}
}
\end{algorithm}


\section{Experimental Settings}\label{app:experimental_settings}

We set our GNN module to have 200 dimensions and 5 layers, where a dropout rate of 0.2 is applied to each layer. We train the model on a single NVIDIA A100 GPU with a batch size of 64. The learning rates for the language model and the GNN module are set to $1e-5$ and $1e-3$, respectively. 
We opt for the RAdam optimizer for RoBERTa-large, while employing AdamW for Llama-2-7B.
These settings are adopted in the first part of the evaluation to investigate the performance of the GNN module.


We employ ConceptNet \citep{speer2017conceptnet} as our external knowledge source for CommonsenseQA and OpenBookQA. ConceptNet contains a vast amount of information with 799,273 nodes and 2,487,810 edges, which provides a valuable resource for improving the accuracy of QA systems. We extract the $G_k$ with a hop size of 2, and subsequently prune the obtained graph to retain only the top 200 nodes.


\section{Instruction for Explanation Generation}

\begin{figure}[h!]
\centering
% \resizebox{0.95\textwidth}{!}{%
\begin{stage1}
\textbf{Basis:} Given a LM augmented with a graph attention network to extract key reasoning elements for decision-making. 
The task is \highlight{[TASK\_TYPE]}.

\textbf{Input:} The question is: \highlight{[$q$]}. The Answer Options are: \highlight{[$\mathcal{A}$]}

\textbf{Output:} The model predicted choice \highlight{[$y'$]}. Based on the Ranked Reason-elements: \highlight{[$\mathcal{Q}$]}

\textbf{Explanation (Stage 1):} 
Explain the LM's reasoning process for selecting \highlight{[$y'$]} over the other options. Provide concise explanations for why each reason-element supports \highlight{[$y'$]} as the predicted choice. Focus on the LM's behavior and the significance of the Ranked Reason-elements. Your response should be short and concise.

\textbf{Explanation (Stage 2):} Based on the \highlight{[$E_0$]}, explain why this LM makes the other options less likely \highlight{[$\mathcal{A}\setminus\{y'\}$]}. Your response should be short and concise.
\end{stage1}
% }
\caption{The instructions for explanation generators.}
\label{fig:instructions}
\end{figure}




\section{LM Debugger} \label{app:lm_debugger}


The LM Debugger is an important component of the \methodname{}, designed to simulate the reasoning process of a ``perfect'' LM using the generated explanations. By treating the explanations as a representation of the LM's decision-making process, the debugger evaluates the quality of these explanations across four key dimensions: faithfulness, completeness, minimality, and accuracy. Additionally, it offers advice for improving the actual LM's performance based on the simulated reasoning process.


\subsection{Evaluation Dimensions}

\paragraph{Faithfulness} The LM Debugger evaluates whether the explanation faithfully represents the simulated LM's reasoning process. It checks if the provided rationale aligns with the data-driven and algorithmic processes an simulated LM would use to arrive at its conclusions. This evaluation ensures that the explanation accurately reflects the expected decision-making process of a simulated LM, serving as a benchmark for the actual LM's performance.

\paragraph{Completeness} The LM debugger determines whether the explanation fully captures the breadth of data that an simulated LM would leverage to make a decision. It ensures that no significant computational strategies or data insights that a simulated LM would rely on are omitted. A complete explanation should provide a comprehensive understanding of the simulated LM's decision-making process, including all relevant factors and considerations.

\paragraph{Minimality} The LM Debugger verifies that the explanation includes only the essential computational processes or data insights that an simulated LM would utilize, without unnecessary elaboration or speculative reasoning beyond the expected operational framework. This evaluation helps to maintain the reasoning's conciseness and relevance, focusing on the key factors that directly influence the simulated LM's decision-making process.

\paragraph{Accuracy} The LM Debugger acts as a simulated "perfect" LM, using the generated explanations to assess how well they align with the reasoning process and capabilities of an simulated LM. It evaluates the accuracy of the explanations against the simulated LM's standard. 

\subsection{Advice for Improvement}

Based on the evaluation results, the LM Debugger suggests improvements focusing on enhancing the actual LM's training data diversity, algorithmic transparency, or decision-making accuracy to better align with the simulated ``prefect'' LM. These suggestions aim to mitigate biases and increase the LM's reliability and trustworthiness. 

\subsection{Instruction for LM Debugger}


\begin{figure}[h]
\centering
\begin{tcolorbox}[colback=softGray, colframe=deepBlue, title=Instruction for LM Debugger, fontupper=\fontsize{8pt}{1pt}\selectfont,]

\textbf{System Prompt:} Evaluate the explanation provided for the LM‘s decision-making process in answering the given question. Assess the explanation across four dimensions: faithfulness, completeness, minimality, and accuracy. Assume the role of an LM debugger with expertise in the inner workings of \highlight{[MODEL\_NAME]} and a 100\% accurate \highlight{[MODEL\_NAME]}.

\textbf{Explanation Content:} Given a LM augmented with a graph attention network to extract key reasoning elements for decision-making. The task is \highlight{[TASK\_TYPE]}. The question is \highlight{[$q$]}. The Answer Options are: \highlight{[$\mathcal{A}$]}. The LM's prediction is \highlight{[$y'$]}. Top-ranked reason-elements are \highlight{[$\mathcal{Q}$]}. Explanation:
\highlight{[$E$]}. 

\textbf{Evaluation Criteria:} 

- Faithfulness: Assess if the explanation truly represents the LM's underlying computational and statistical mechanisms. Check if the rationale provided mirrors the data-driven and algorithmic processes the LM uses to arrive at its conclusions.
% Does the explanation accurately reflect the model's reasoning process? Consider if the explanation could independently arrive at the same conclusion as the whole model.

- Completeness: Determine whether the explanation fully captures the breadth of data and algorithms the LM leverages to make a decision. Ensure no significant computational strategies or data insights that the LM relies on are omitted.
% Evaluate whether the explanation encompasses all the elements used by the model to perform the task. Are any critical reasoning elements missing from the explanation?

- Minimality: Verify that the explanation includes only the essential computational processes and data insights the LM utilizes, without unnecessary elaboration or speculative reasoning beyond the LM's actual operational framework.
% Assess if the explanation contains only the necessary elements relevant to the task. Identify any nodes or parts of the explanation that are irrelevant or redundant in justifying the model's decision.

- Accuracy: Evaluate the precision with which the explanation reflects the LM's true capabilities and decision-making process, taking into account the LM's design, training data, and algorithmic functions.
% Based on the results of faithfulness, completeness, and minimality, assess the accuracy of the answer.

\textbf{Advice Instruction:} 
Suggest improvements focusing on enhancing the LM's training data diversity, algorithmic transparency, or decision-making accuracy, to mitigate biases and increase the model's reliability and trustworthiness in varied contexts.

\textbf{Debugging Instruction:}
Please provide a score from 1 to 5 for each dimension, with 1 being the lowest (poor, equivalent to \highlight{[MODEL\_NAME]} having 0\% accuracy) and 5 being the highest (excellent, equivalent to \highlight{[MODEL\_NAME]} having 100\% accuracy). Highlight specific areas where the explanation aligns well or falls short of the evaluation criteria. Your response should be short and concise.



\end{tcolorbox}
\caption{The instructions for LM Debugger.}
\end{figure}

\subsection{Evaluation of Effectiveness}

We demonstrate the effectiveness of the LM Debugger in helping users determine when to trust the model's predictions. We set a threshold score of 3 and focus on the accuracy-related dimensions, specifically ``Faithfulnes'' and ``Accuracy'', which serve as guides for assessing the LM's prediction accuracy.
Figure \ref{fig:lm_debugger} presents the percentage of cases where the LM Debugger correctly identifies the reliability of the model's predictions based on the explanation. The results show that our debugging process effectively assesses the dependability of the model's predictions based on the \methodname{} generated explanation, achieving 100\% accuracy in evaluating faithfulness and accuracy for both correct and incorrect predictions.



\begin{figure}[htbp]
    \centering
        \includegraphics[width=0.4\textwidth]{figure/two_dimensions_chart.pdf}
        \caption{LM Debugger's accuracy in identifying the reliability of the model's predictions based on explanation quality.}
        \label{fig:lm_debugger}
\end{figure}


\section{Details of Element-graph} \label{sec:graph}
Due to space constraints in the main text, we provide a comprehensive description of the node and relations types, alongside the detailed equations for computing their embeddings. 

The node-type $u_i$ are the one-hot vectors of the node types. The type is according to the node's origin form, the input content $z$, question $\{q\}$, answer $\mathcal{A}$, or the node in the KG. 
The $u_i$ is transformed into an embedding through a linear transformation for subsequent calculations.

The relation type $r_{ij}$ is determined using pre-defined templates, which are employed to extract relations from the knowledge triples in the KG~\citep{feng2020scalable}. The embedding $\boldsymbol r_{ij}$ for the relation is computed for subsequent use: 
\begin{equation}
  \boldsymbol r_{ij} = f_{\zeta } (r_{ij}, u_{ij}) = f_{\zeta } (r_{ij}, u_{i}, u_{j}),
\end{equation}
where $f_{\zeta }$ is a two-layer MLP, $u_{ij}$ denotes the concatenation of $u_{i}$ and $u_{j}$.

The node score $v_{score}$ is subsequently utilized in its embedded form, calculated by:
\begin{equation}
    \boldsymbol v_{score} = f_{\rho }(v_{score})
\end{equation}
where $f_{\rho }$ is a two-layer MLP. 


\section{Other Explanation Examples}\label{sec:appendix-example}
We demonstrate the complete explanation example of PathReasoner and ECQA in Table~\ref{app:example}. These methods exhibit in an unclear and intricate manner. Such explanations make it hard for humans to understand the decision-making process behind the model.


\begin{table}[htb]
  \centering
  \resizebox{0.9\columnwidth}{!}{%
    \begin{tabular}{@{}c|l@{}}
      \toprule
      \textbf{Input Questions}                          & \begin{tabular}[c]{@{}l@{}}Q: What is someone doing if he or she is sitting quietly and his or \\her eyes are moving?\\ A. reading B. meditate C. fall asleep D. bunk E. think\end{tabular}                                                                                                                                                                                                                                                                                                                 \\ \midrule
      \textbf{Label}                                    & A. reading                                                                                                                                                                                                                                                                                                                                                                                                                                                                                                                                                                                                                                                                                    \\
      \midrule
                                                        & \textbf{Explanation of Others}                                                                                                                                                                                                                                                                                                                                                                                                                                                                                                                                                                                                                                                                \\ \midrule
      \textbf{\makecell{Path-                                                                                                                                                                                                                                                                                                                                                                                                                                                                                                                                                                                                                                                                                                                           \\Reasoner }} & \begin{tabular}[c]{@{}l@{}}quietly [related to] quiet [at location] a library [used for] reading \\ eyes [used for] reading \\ eyes [form of] eye [related to] glasses [used for] reading \\ sitting [related to] sit [related to] relaxing [has subevent] reading \\\end{tabular} \\ \midrule
      \textbf{\makecell{ECQA \\}} & \begin{tabular}[c]{@{}l@{}}\textbf{Positive examples:} \\ - When we read, our eyes move. \\ - While reading, a person sits quietly, \\ \textbf{Negative examples: }\\ - While meditating, eyes don't move, eyes are closed, \\ - While sleeping, eyes are closed and they don't move, \\ - When a person bunks, he/she doesn't sit quietly, \\ - Eyes don't move when you think about something. \\ \textbf{Explanation:} \\ When we read, our eyes move. \\ While reading, a person sits quietly. \\ While meditating and sleeping, eyes don't move, eyes are closed. \\ When a person bunks, he/she doesn't sit quietly. \\ Eyes don't move when you think about something. \\\end{tabular} \\ \bottomrule
    \end{tabular}%
  }
  \caption{The complete explanation examples of PathReasoner and ECQA.}
  \label{app:example}
\end{table}







\section{Case Study} \label{app:case}

In this section, we will present more explanation examples generated by~\methodname{}, to demonstrate the effectiveness of our method. These questions are randomly selected from the dataset. By sharing these examples, we aim to highlight the broad applicability and robust performance of~\methodname{} across diverse contexts. These examples not only illustrate the adaptability of our approach but also substantiate its validity by demonstrating consistent results. The comparison examples are shown in Figure \ref{fig:correct_ex} and Figure \ref{fig:wrong_exam}.



\definecolor{green}{HTML}{B7C47D}
\definecolor{red}{HTML}{D46155}

\begin{figure}[htb]
\centering
\begin{tcolorbox}[colback=softGray, colframe=green, title=\ding{52} Explanation Example of Correct Prediction, 
coltitle=black,
fonttitle=\fontsize{8pt}{1pt}\selectfont, fontupper=\fontsize{5pt}{1pt}\selectfont,]
\textbf{Input:} 
The question is ``What is someone doing if he or she is sitting quietly and his or her eyes are moving?''\\
The Answer Options are:
A. Reading, B. Meditate, C. Fall Asleep, D. Bunk, E. Think

\textbf{Output (Model Prediction):} “A. Reading”

\textbf{Top-ranked reason-elements:}
1. Quiet chattering mind, 2. Not making sound, 3. Mind focuses, 4. Glasses for people with poor eyesight, 5. War
\vspace{2mm} 
\hrule 
\vspace{2mm} 
\textbf{Stage 1 - Explanation (why-choose):}
The model predicts "Reading" (A) as the activity someone is engaged in when sitting quietly with moving eyes, based on the key reason-elements. Here's how the reason-elements support this conclusion:\\
- \highlight{\textbf{Quiet Chattering Mind \& Not Making Sound:}} Indicate silent, internal focus, typical of reading, fitting the description of someone sitting quietly.\\
- \highlight{\textbf{Mind Focuses:}} Suggests active engagement with content, aligning with reading.\\
- \highlight{\textbf{Glasses for Poor Eyesight:}} Implies a visual task, supporting reading despite being an indirect clue.\\
% - \highlight{\textbf{War:}} Irrelevant to the decision-making process.

Excluding outliers, these elements collectively signal silent, visually focused activity, characteristic of reading.

\textbf{Stage 2 - Explanation (why-not-choose):}
The model's reasoning for making other options less likely is as follows:\\
- \highlight{\textbf{Meditate (B):}} Involves closed or unfocused eyes, unlike reading's active eye movement.\\
- \highlight{\textbf{Fall Asleep (C):}} Entails cessation of eye movement, not alignment with reading.\\
- \highlight{\textbf{Bunk (D):}} Lacks specific association with quiet, focused eye movement.\\
- \highlight{\textbf{Think (E):}} Doesn't necessitate visual engagement, different from reading.

Thus, only reading aligns with the criteria of silent, focused activity involving active visual engagement.

\vspace{2mm} 
\hrule 
\vspace{2mm} 
\textbf{LM Debugger:}

- \highlight{\textbf{Faithfulness:}} 4/5 - Well-aligned with the LM's data-driven reasoning.

- \highlight{\textbf{Completeness:}} 4/5 - Captures key reasons for "Reading" but may miss minor factors distinguishing it from similar activities.

- \highlight{\textbf{Minimality:}} 4/5 - Includes mostly relevant reasoning but is diluted by the unrelated "War."

- \highlight{\textbf{Accuracy:}} 4/5 - Accurately identifies "Reading" as the most likely activity, with minor issues in relevancy and nuanced differentiation.

- \highlight{\textbf{Advice for Improvement:}} Enhance the LM by broadening training data to cover more \textbf{physical activities} for accuracy, and fine-tuning with \textbf{detailed activity datasets} for sharper recognition.
\end{tcolorbox}
\caption{Explanation Example of Correct Prediction.}
\label{fig:correct_ex}
\end{figure}


\begin{figure}[htb]
\centering
\begin{tcolorbox}[colback=softGray, colframe=red, title=\ding{56} Explanation Example of Incorrect Prediction, fonttitle=\fontsize{8pt}{1pt}\selectfont, fontupper=\fontsize{5pt}{1pt}\selectfont,]

\textbf{Input:} 
The question is ``What do parents tell a child to do on the weekend?''\\
The Answer Options are:
A. Study, B. Begin school, C. Go out to play, D. Row boat, E. Clean room

\textbf{Output (Model Prediction):} “E. Clean room”

\textbf{Top-ranked reason-elements:}
1. Place Baseball, 2. Asleep, 3. Subway, 4. Ups, 5. Mentors
\vspace{2mm} 
\hrule 
\vspace{2mm} 
\textbf{Stage 1 - Explanation (why-choose):}
The model's prediction of "E. Clean room" as the task parents tell a child to do on the weekend is influenced by the ranked reasoning elements, each contributing to the rationale in abstract but supportive ways:

- \highlight{\textbf{Place Baseball:}} Suggests activities requiring cleanup, reinforcing the idea of tidying up spaces like a room.

- \highlight{\textbf{Asleep:}} Implies end-of-day tasks, where cleaning up could be a preparatory action before rest.

- \highlight{\textbf{Subway:}} Though less direct, could symbolize daily routines, pointing to regular tasks like room cleaning.

- \highlight{\textbf{Ups:}} Might relate to picking up or organizing items, aligning with cleaning activities.

- \highlight{\textbf{Mentors:}} Reflects the guiding role of parents in instilling responsibility, including maintaining cleanliness.

Together, these elements hint at themes of responsibility, routine, and organization, leading to the conclusion that cleaning a room is a fitting weekend task for a child from a parent's perspective.

\textbf{Stage 2 - Explanation (why-not-choose):}
The model's reasoning for making other options less likely is as follows:\\
- \highlight{\textbf{A. Study}} and \highlight{\textbf{B. Begin school:}} They're associated with cognitive tasks, diverging from the model's focus on physical organization and day-end routines.

- \highlight{\textbf{C. Go out to play:}} Suggests initiating a new activity, not related to the model's emphasis on tidiness and concluding the day with responsible actions.

- \highlight{\textbf{D. Row boat:}} Similar to "C," it indicates starting an activity rather than organizing or cleaning, moving away from the themes of responsibility towards personal spaces.

Thus, the model's logic, centered around ending day activities and maintaining personal spaces, naturally leads to "E. Clean room" as the most logical weekend task for a child, sidelining the other options due to their lack of connection to tidiness and routine responsibility.

\vspace{2mm} 
\hrule 
\vspace{2mm} 
\textbf{LM Debugger:}

- \highlight{\textbf{Faithfulness:}} 1/5 - Misrepresents LM reasoning, attributing abstract thought where only statistical patterns apply.

- \highlight{\textbf{Completeness:}} 2/5 - Omits explanation of LMs' data-driven nature, focusing instead on unrelated elements.

- \highlight{\textbf{Minimality:}} 1/5 -  Includes extraneous elements, veering away from the statistical basis of LM outputs.

- \highlight{\textbf{Accuracy:}} 1/5 - Anthropomorphizes LM's statistical analysis as abstract reasoning, misinterpreting its function.

- \highlight{\textbf{Advice for Improvement:}} Focus on diversifying training datasets with a wider range of language use related to \textbf{children's tasks}, improve decision-making accuracy with \textbf{targeted training on family and child-rearing contexts}.
\end{tcolorbox}
\caption{Explanation Example of Incorrect Prediction.}
\label{fig:wrong_exam}
\end{figure}






\section{Ablation Studies}\label{sec:ablation}
In this section, we examine the impact of different components of~\methodname{} on its performance. We evaluated the effects of the size of the LMs, knowledge components, and interpreting components using the CommonsenseQA IHdev and IHtest datasets. Tables~\ref{ab:PLM},~\ref{ab:model-only} and~\ref{ab:interpreting} summarize the ablation study results.

\begin{table}[h]
  \centering
  \resizebox{0.49\columnwidth}{!}{%
  \begin{tabular}{@{}rcc@{}}
    \toprule
    Method                       & IHdev-Acc. & IHtest-Acc. \\ \midrule
    RoBERTa w/o itp              & 68.63\%    & 64.54\%     \\
    RoBERTa-large w/o itp        & 73.05\%    & 71.96\%     \\
    RoBERTa-large + itp & \textbf{77.97\%}    & \textbf{77.31\%}     \\ \bottomrule
  \end{tabular}
  }
  \caption{Ablation study on the effect of interpreting component on model accuracy.}
  \label{ab:interpreting}
\end{table}

Table~\ref{ab:PLM} shows the impact of the size of LM on~\methodname{}. We evaluate the performance of LMs with two different sizes: 1) RoBERTa-large (with 340 million parameters) and 2) RoBERTa (with 110 million parameters). The results show that using a larger LM leads to significant improvement in performance, with an increase of 11.71\% and 14.30\% in model accuracy on the IHdev dataset and the IHtest dataset, respectively.

Table~\ref{ab:model-only} shows the impact of the knowledge component of~\methodname{}.
We compare the performance of the LM-only model with and without external knowledge from ConceptNet.
\textit{only} means we only use the LM to predict the answer. \textit{+ external knowledge} means the external knowledge is leveraged. We observe that incorporating external knowledge significantly improves the accuracy of the LM prediction, especially on the test set. With external knowledge, the model accuracy on IHdev and IHtest is increased by at least 3.69\% and 7.12\%, respectively.
% This shows that external knowledge plays an important role in enhancing the reasoning ability of the model.

\begin{table}[h]
\centering
\resizebox{0.49\columnwidth}{!}{
\begin{tabular}{@{}rcc@{}}
    \toprule
    LM                    & IHdev-Acc. & IHtest-Acc. \\ \midrule
    RoBERTa               & 66.26\%    & 63.01\%     \\
    RoBERTa-large (final) & \textbf{77.97\%}    & \textbf{77.31\%}     \\\bottomrule
  \end{tabular}
  }
  \caption{Ablation study on the effect of LM size on model accuracy.}
  \label{ab:PLM}
\end{table}



In Table~\ref{ab:interpreting}, we analyze the impact of the interpreting component on LM performance. \textit{w/o itp} indicates that the interpreting component was not incorporated in the prediction, whereas the \textit{+ itp} indicates its presence. We observe that removing the interpreting component leads to a clear decrease in accuracy by at least 4.92\% and 5.35\% on IHdev and IHtest, respectively. Furthermore, comparing the results of \textit{RoBERTa-large only}, \textit{RoBERTa-large + itp}, and \textit{final}, we find that the interpreting component has a greater impact on accuracy than the other components.

\begin{table}[htb]
\centering
\resizebox{0.49\columnwidth}{!}{
\begin{tabular}{@{}rcc@{}}
    \toprule
    Method                        & IHdev-Acc. & IHtest-Acc. \\ \midrule
    RoBERTa only                  & 62.65\%    & 60.27\%     \\
    RoBERTa-large only            & 74.28\%    & 70.19\%     \\
    RoBERTa-large + external knowledge & \textbf{77.97\%}    & \textbf{77.31\%}     \\ \bottomrule
  \end{tabular}
  }
  \caption{Ablation study on the effect of knowledge component on model accuracy.}
  \label{ab:model-only}
\end{table}

The ablation highlights the positive contributions of each component of \methodname{}. Specifically, we find that the interpreting component plays a crucial role in enhancing model accuracy and generalizability on unseen questions.


\subsection{Results of Different Generators} \label{appendix:generators}

\begin{figure*}[htb]
  \begin{center}
    %\framebox[4.0in]{$\;$}
    %\fbox{\rule[-.5cm]{0cm}{4cm} \rule[-.5cm]{4cm}{0cm}}
    \includegraphics[width=1\textwidth]{figure/complete_example.pdf}
  \end{center}
  \caption{The \textit{why-choose} and \textit{why-not-choose} explanations generated by Llama-2-70B, GPT-4 and GPT-3.5. The semantic meanings remain consistently aligned among the explanations generated by the three models.}
  \label{fig:appendix-exam}
\end{figure*}


In this section, we present a comprehensive analysis of the results from different explanation generators: Llama-2-70B, GPT-4-turbo, and GPT-3.5-turbo. We focus on evaluating how each generator interprets and translates the model's decision-making process into human-understandable explanations.

The complete experimental results are presented in Figure \ref{fig:appendix-exam}, where all experiments are conducted under the same settings. The question is collected randomly: 
\begin{itemize}
    \item Question: What is someone doing if he or she is sitting quietly and his or her eyes are moving?
    \item Answer Choices: A. reading, B. meditate, C. fall asleep, D. bunk, E. think.
    \item Model Prediction: A. reading
    \item Ground-truth Answer: A. reading
\end{itemize}

We utilize RoBERTa-large as the LM $f_{LM}$ for this experiment. The $f_{LM}$ correctly predicts the answer as ``A. reading''. Our extracted \textit{reason-elements} are: 1. quiet chattering mind, 2. not making sound, 3. mind focuses, 4. glasses for people with poor eyesight, 5. war.

To further quantify the semantic similarity between explanations of Llama-2, GPT-4, and GPT-3.5, we employ GPT-4 to generate similarity scores. GPT-4's advanced language comprehension abilities make it well-suited for this task, offering a human-like understanding of textual content. The scores reflect the degree of alignment in content among the explanations. The score is on a scale from 0 to 1, where 1 is very similar and 0 is not similar at all.

\begin{figure}[h]
  \begin{center}
    \includegraphics[width=0.45\textwidth]{figure/heatmap.png}
  \end{center}
  \caption{Heatmap of Similarity Scores for Llama-2, GPT-4, and GPT-3.5: Their generated explanations show high consistency in terms of semantic meaning. }
  \label{fig:appendix-heat}
  \vspace{-10pt}
\end{figure}

\textbf{Llama-2 vs. GPT-4:}

Similarity: Both explanations align in focusing on the 'reading' activity, referencing quiet sitting, eye movement, and glasses use.

Similarity Score: 0.85/1 - High similarity in core reasoning and conclusion.

\textbf{Llama-2 vs. GPT-3.5:}

Similarity: Both identify the person as engaged in reading, noting quiet sitting and glasses use.

Similarity Score: 0.75/1 - Similar in conclusion and main points, but GPT-3.5 provides more concise content.

\textbf{GPT-4 vs. GPT-3.5:}

Similarity: Agreement in the conclusion of ``reading'', common elements include quiet posture, eye movement, and glasses use.

Similarity Score: 0.80/1 - Similar key conclusions and elements, but GPT-4 includes more detail.

We illustrate the similarity scores in Figure \ref{fig:appendix-heat}. The color intensities represent the degree of similarity, with darker tones indicating higher congruence. It shows their generated explanations align in semantic meaning. 

% Despite variations in style and detail, the fundamental meanings are consistent across all generators. This consistency highlights the effectiveness of our approach in preserving the accuracy of the explanations.

The explanations generated by the three models are largely consistent in semantic meaning, demonstrating that under our constrained prompt instruction, these models primarily functioned as ``translators''. They convert the reasoning process into human-understandable language. However, it is important to note that the capability of the generator influenced the readability of the explanations. For instance, Llama-2 tends to produce more repetitive language (in red), while GPT-3.5-turbo and GPT-4 show consistency and conciseness. Based on these observations, we recommend using GPT-3.5-turbo or GPT-4 as the explanation generator for optimal clarity.




\section{Material for User Perception Perspective Evaluation}
\subsection{Details of Evaluation Metrics}
\paragraph{Overall Quality} This criterion assesses the effectiveness of the explanations in making the LMs' decision processes understandable to users, providing a comprehensive measure of the explainability's efficacy.

\paragraph{Understandability} This metric evaluates the clarity and coherence of the explanations, determining how easily users can comprehend the model's outputs and underlying reasoning.

\paragraph{Trustworthiness} This assesses users' confidence in the model's outputs and explanations, examining whether the explanations are perceived as reliable, credible, and based on sound logic.

\paragraph{Satisfaction} This captures users' overall contentment with the explanations, considering whether the outputs meet their expectations in terms of quality, relevance, and usefulness.

\paragraph{Detail Sufficiency} This examines whether the explanations provide an adequate level of detail, ensuring they are sufficiently descriptive and comprehensive to fully address the question or task at hand.

\paragraph{Completeness} This evaluates the extent to which the explanations cover the model's decision-making process, verifying that all essential aspects of the model's reasoning are addressed.

\paragraph{Accuracy} In addition to objective measures, this involves a subjective assessment of whether the explanations align with users' knowledge or expectations, confirming the factual correctness and contextual appropriateness of the model's outputs and explanations.

\subsection{Details of Automated Evaluation}\label{appendix:auto-eval}

In this section, we describe the process of using automated evaluators to evaluate the quality of our explanations. We start by randomly selecting an example to demonstrate the evaluation procedure. The evaluators provide their scores and reasons for their ratings. This allows us to verify the validity of their assessments and better understand the basis of their evaluations. We use GPT-3.5-turbo and GPT-4 as our automated evaluators. 


\textbf{Question:} What might happen when not being careful setting a cup on a table?

A. picks up

B. noise

C. decorating the table

D. spillage

E. clutter

\textbf{LM predicted answer:}  D. spillage

\textbf{Ranked Reason-elements:} 1. cold day, 2. careless, 3. past can, 4. quite possible, 5. written document.

\textbf{\textit{Why-choose} explanation:} Since the person is being careless and it's quite possible that they've spilled things in the past, the model predicts that there is a high likelihood of spillage when not being careful setting a cup on a table. 

\textbf{\textit{Why-not-choose} explanation:} The other potential choices, such as picks up, decorating the table, and clutter, were not predicted because those words were not mentioned as significant reasons by the model. Additionally, there was no indication in the reason-words to suggest that those actions would occur as a result of not being careful when setting a cup on a table. The word noise was also not predicted because it does not directly relate to the action of setting a cup on a table, nor were any reason-words related to it.

\textbf{Prompt: }Suppose you are an AI expert. We now require your assistance in evaluating the content validity of explanations that why a language model made certain decisions. Please adhere to our grading rubric, using ``1'' to represent ``Disagree'', ``2'' for ``Neutral'', and ``3'' for ``Agree''. Your grading should be based on the following 8 questions: 

1. This is a good explanation; 

2. I understand this explanation of how the AI model works; 

3. I trust this explanation of how the AI model works; 

4. This explanation of how the AI model works is satisfying; 

5. This explanation of how the AI model works provides sufficient detail; 

% 6. This explanation of how the AI model works contains irrelevant details; 

6. This explanation of how the AI model works seems complete; 

7. This explanation of how the AI model works is accurate.

\textbf{Evaluation results: }

\textbf{GPT-3.5-turbo:} 

1. This is a good explanation: 3 (Agree)

2. I understand this explanation of how the AI model works: 3 (Agree)

3. I trust this explanation of how the AI model works: 3 (Agree)

4.  This explanation of how the AI model works is satisfying: 3 (Agree)

5. This explanation of how the AI model works has sufficient detail: 3 (Agree)

% 6. This explanation of how the AI model works contains irrelevant details: 2 (Neutral)

6. This explanation of how the AI model works seems complete: 3 (Agree)

7. This explanation of how the AI model works is accurate: 3 (Agree) 

Reason: The explanation provided is clear and understandable, providing a valid reasoning for why the predicted answer is ``spillage''. The mention of being careless and the likelihood of spillage based on past experiences supports the prediction that not being careful when setting a cup on a table could result in spillage. However, some of the reason-elements, such as cold day, quite possible, and written document, do not seem directly relevant to the context and do not contribute significantly to the explanation, resulting in a neutral rating for question 6. Nonetheless, the overall explanation is satisfactory and accurate given the available information.




\textbf{GPT-4:} 

1. This is a good explanation: 3 (Agree)

2. I understand this explanation of how the AI model works: 3 (Agree)

3. I trust this explanation of how the AI model works: 3 (Agree)

4. This explanation of how the AI model works is satisfying: 3 (Agree)

5. This explanation of how the AI model works has sufficient detail: 3 (Agree)

% 6. This explanation of how the AI model works contains irrelevant details: 2 (Neutral)

6. This explanation of how the AI model works seems complete: 3 (Agree)

7. This explanation of how the AI model works is accurate: 3 (Agree)

Reason: While the reason-elements (``cold day'', ``careless'', ``past can'', ``quite possible'', ``written document'') don't seem to have a clear relation to the specific context of the question, the explanation itself is still understandable, providing clear reasoning for why the model chose ``spillage'' as the most likely outcome and why the other options were less likely. The explanation is detailed, complete, and aligns with common sense and real-world expectations about what might happen when someone is not careful while setting a cup on a table.







