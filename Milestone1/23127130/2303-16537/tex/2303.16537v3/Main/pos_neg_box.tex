
\definecolor{green}{HTML}{B7C47D}
\definecolor{red}{HTML}{D46155}

\begin{figure}[htb]
\centering
\begin{tcolorbox}[colback=softGray, colframe=green, title=\ding{52} Explanation Example of Correct Prediction, 
coltitle=black,
fonttitle=\fontsize{8pt}{1pt}\selectfont, fontupper=\fontsize{5pt}{1pt}\selectfont,]
\textbf{Input:} 
The question is ``What is someone doing if he or she is sitting quietly and his or her eyes are moving?''\\
The Answer Options are:
A. Reading, B. Meditate, C. Fall Asleep, D. Bunk, E. Think

\textbf{Output (Model Prediction):} “A. Reading”

\textbf{Top-ranked reason-elements:}
1. Quiet chattering mind, 2. Not making sound, 3. Mind focuses, 4. Glasses for people with poor eyesight, 5. War
\vspace{2mm} 
\hrule 
\vspace{2mm} 
\textbf{Stage 1 - Explanation (why-choose):}
The model predicts "Reading" (A) as the activity someone is engaged in when sitting quietly with moving eyes, based on the key reason-elements. Here's how the reason-elements support this conclusion:\\
- \highlight{\textbf{Quiet Chattering Mind \& Not Making Sound:}} Indicate silent, internal focus, typical of reading, fitting the description of someone sitting quietly.\\
- \highlight{\textbf{Mind Focuses:}} Suggests active engagement with content, aligning with reading.\\
- \highlight{\textbf{Glasses for Poor Eyesight:}} Implies a visual task, supporting reading despite being an indirect clue.\\
% - \highlight{\textbf{War:}} Irrelevant to the decision-making process.

Excluding outliers, these elements collectively signal silent, visually focused activity, characteristic of reading.

\textbf{Stage 2 - Explanation (why-not-choose):}
The model's reasoning for making other options less likely is as follows:\\
- \highlight{\textbf{Meditate (B):}} Involves closed or unfocused eyes, unlike reading's active eye movement.\\
- \highlight{\textbf{Fall Asleep (C):}} Entails cessation of eye movement, not alignment with reading.\\
- \highlight{\textbf{Bunk (D):}} Lacks specific association with quiet, focused eye movement.\\
- \highlight{\textbf{Think (E):}} Doesn't necessitate visual engagement, different from reading.

Thus, only reading aligns with the criteria of silent, focused activity involving active visual engagement.

\vspace{2mm} 
\hrule 
\vspace{2mm} 
\textbf{LM Debugger:}

- \highlight{\textbf{Faithfulness:}} 4/5 - Well-aligned with the LM's data-driven reasoning.

- \highlight{\textbf{Completeness:}} 4/5 - Captures key reasons for "Reading" but may miss minor factors distinguishing it from similar activities.

- \highlight{\textbf{Minimality:}} 4/5 - Includes mostly relevant reasoning but is diluted by the unrelated "War."

- \highlight{\textbf{Accuracy:}} 4/5 - Accurately identifies "Reading" as the most likely activity, with minor issues in relevancy and nuanced differentiation.

- \highlight{\textbf{Advice for Improvement:}} Enhance the LM by broadening training data to cover more \textbf{physical activities} for accuracy, and fine-tuning with \textbf{detailed activity datasets} for sharper recognition.
\end{tcolorbox}
\caption{Explanation Example of Correct Prediction.}
\label{fig:correct_ex}
\end{figure}


\begin{figure}[htb]
\centering
\begin{tcolorbox}[colback=softGray, colframe=red, title=\ding{56} Explanation Example of Incorrect Prediction, fonttitle=\fontsize{8pt}{1pt}\selectfont, fontupper=\fontsize{5pt}{1pt}\selectfont,]

\textbf{Input:} 
The question is ``What do parents tell a child to do on the weekend?''\\
The Answer Options are:
A. Study, B. Begin school, C. Go out to play, D. Row boat, E. Clean room

\textbf{Output (Model Prediction):} “E. Clean room”

\textbf{Top-ranked reason-elements:}
1. Place Baseball, 2. Asleep, 3. Subway, 4. Ups, 5. Mentors
\vspace{2mm} 
\hrule 
\vspace{2mm} 
\textbf{Stage 1 - Explanation (why-choose):}
The model's prediction of "E. Clean room" as the task parents tell a child to do on the weekend is influenced by the ranked reasoning elements, each contributing to the rationale in abstract but supportive ways:

- \highlight{\textbf{Place Baseball:}} Suggests activities requiring cleanup, reinforcing the idea of tidying up spaces like a room.

- \highlight{\textbf{Asleep:}} Implies end-of-day tasks, where cleaning up could be a preparatory action before rest.

- \highlight{\textbf{Subway:}} Though less direct, could symbolize daily routines, pointing to regular tasks like room cleaning.

- \highlight{\textbf{Ups:}} Might relate to picking up or organizing items, aligning with cleaning activities.

- \highlight{\textbf{Mentors:}} Reflects the guiding role of parents in instilling responsibility, including maintaining cleanliness.

Together, these elements hint at themes of responsibility, routine, and organization, leading to the conclusion that cleaning a room is a fitting weekend task for a child from a parent's perspective.

\textbf{Stage 2 - Explanation (why-not-choose):}
The model's reasoning for making other options less likely is as follows:\\
- \highlight{\textbf{A. Study}} and \highlight{\textbf{B. Begin school:}} They're associated with cognitive tasks, diverging from the model's focus on physical organization and day-end routines.

- \highlight{\textbf{C. Go out to play:}} Suggests initiating a new activity, not related to the model's emphasis on tidiness and concluding the day with responsible actions.

- \highlight{\textbf{D. Row boat:}} Similar to "C," it indicates starting an activity rather than organizing or cleaning, moving away from the themes of responsibility towards personal spaces.

Thus, the model's logic, centered around ending day activities and maintaining personal spaces, naturally leads to "E. Clean room" as the most logical weekend task for a child, sidelining the other options due to their lack of connection to tidiness and routine responsibility.

\vspace{2mm} 
\hrule 
\vspace{2mm} 
\textbf{LM Debugger:}

- \highlight{\textbf{Faithfulness:}} 1/5 - Misrepresents LM reasoning, attributing abstract thought where only statistical patterns apply.

- \highlight{\textbf{Completeness:}} 2/5 - Omits explanation of LMs' data-driven nature, focusing instead on unrelated elements.

- \highlight{\textbf{Minimality:}} 1/5 -  Includes extraneous elements, veering away from the statistical basis of LM outputs.

- \highlight{\textbf{Accuracy:}} 1/5 - Anthropomorphizes LM's statistical analysis as abstract reasoning, misinterpreting its function.

- \highlight{\textbf{Advice for Improvement:}} Focus on diversifying training datasets with a wider range of language use related to \textbf{children's tasks}, improve decision-making accuracy with \textbf{targeted training on family and child-rearing contexts}.
\end{tcolorbox}
\caption{Explanation Example of Incorrect Prediction.}
\label{fig:wrong_exam}
\end{figure}