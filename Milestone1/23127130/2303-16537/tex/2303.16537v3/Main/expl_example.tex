\definecolor{green}{HTML}{B7C47D}
\definecolor{red}{HTML}{D46155}

\begin{figure}[htb]
\centering

\begin{tcolorbox}[colback=softGray, colframe=deepBlue, title=\ding{52} Explanation Example of Correct Prediction, 
fonttitle=\fontsize{8pt}{1pt}\selectfont, fontupper=\fontsize{5pt}{1pt}\selectfont,]
\textbf{Input:} 
The question is ``What is someone doing if he or she is sitting quietly and his or her eyes are moving?''\\
The Answer Options are:
A. Reading, B. Meditate, C. Fall Asleep, D. Bunk, E. Think

\textbf{Output (Model Prediction):} “A. Reading”

\textbf{Top-ranked reason-elements:}
1. Quiet chattering mind, 2. Not making sound, 3. Mind focuses, 4. Glasses for people with poor eyesight, 5. War
\vspace{2mm} 
\hrule 
\vspace{2mm} 
\textbf{Stage 1 - Explanation (why-choose):}
The model predicts "Reading" (A) as the activity someone is engaged in when sitting quietly with moving eyes, based on the key reason-elements. Here's how the reason-elements support this conclusion:\\
- \highlight{\textbf{Quiet Chattering Mind \& Not Making Sound:}} Indicate silent, internal focus, typical of reading, fitting the description of someone sitting quietly.\\
- \highlight{\textbf{Mind Focuses:}} Suggests active engagement with content, aligning with reading.\\
- \highlight{\textbf{Glasses for Poor Eyesight:}} Implies a visual task, supporting reading despite being an indirect clue.\\
% - \highlight{\textbf{War:}} Irrelevant to the decision-making process.

Excluding outliers, these elements collectively signal silent, visually focused activity, characteristic of reading.

\textbf{Stage 2 - Explanation (why-not-choose):}
The model's reasoning for making other options less likely is as follows:\\
- \highlight{\textbf{Meditate (B):}} Involves closed or unfocused eyes, unlike reading's active eye movement.\\
- \highlight{\textbf{Fall Asleep (C):}} Entails cessation of eye movement, not alignment with reading.\\
- \highlight{\textbf{Bunk (D):}} Lacks specific association with quiet, focused eye movement.\\
- \highlight{\textbf{Think (E):}} Doesn't necessitate visual engagement, different from reading.

Thus, only reading aligns with the criteria of silent, focused activity involving active visual engagement.

\vspace{2mm} 
\hrule 
\vspace{2mm} 
\textbf{LM Debugger:}

- \highlight{\textbf{Faithfulness:}} 4/5 - Well-aligned with the LM's data-driven reasoning.

- \highlight{\textbf{Completeness:}} 4/5 - Captures key reasons for "Reading" but may miss minor factors distinguishing it from similar activities.

- \highlight{\textbf{Minimality:}} 4/5 - Includes mostly relevant reasoning but is diluted by the unrelated "War."

- \highlight{\textbf{Accuracy:}} 4/5 - Accurately identifies "Reading" as the most likely activity, with minor issues in relevancy and nuanced differentiation.

- \highlight{\textbf{Advice for Improvement:}} Enhance the LM by broadening training data to cover more \textbf{physical activities} for accuracy, and fine-tuning with \textbf{detailed activity datasets} for sharper recognition.
\end{tcolorbox}
\caption{Example of \methodname{}'s explanation for a correct LM prediction, with high scores and potential enhancement suggestions from the LM Debugger on key dimensions. }
\vspace{-5mm}

\label{fig:correct_expl}

\end{figure}
