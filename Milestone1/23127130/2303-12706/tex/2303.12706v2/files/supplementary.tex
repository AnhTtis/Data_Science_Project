\documentclass[runningheads]{llncs}
%
\usepackage{graphicx}
\usepackage{xcolor}
\usepackage{caption}
\usepackage{subcaption}
\usepackage{amsmath}
\renewcommand{\labelitemi}{$\bullet$}
\usepackage{array}
\usepackage{amssymb}
\usepackage{float}
\usepackage[export]{adjustbox}
% Used for displaying a sample figure. If possible, figure files should
% be included in EPS format.
%
% If you use the hyperref package, please uncomment the following line
% to display URLs in blue roman font according to Springer's eBook style:
% \renewcommand\UrlFont{\color{blue}\rmfamily}

\begin{document}
\section*{\centering \Large Supplementary Materials}
\begin{table}[!htb]
    \caption{$\alpha$ weightings for the gPoE-normVAE model with $L_{\text{dim}}$=10.}\label{alpha_table}
    \vspace*{0.3cm}
    \begin{subtable}{\linewidth}
      \centering
        \resizebox{\textwidth}{!}{%
        \begin{tabular}{|l|r|r|r|r|r|r|r|r|r|r|}
        \hline
        modality &  latent 0 &  latent 1 &  latent 2 &  latent 3 &  latent 4 &  latent 5 &  latent 6 &  latent 7 &  latent 8 &  latent 9 \\
        \hline
              T1 &  0.422716 &   0.319826 &  0.704999 &    0.4191 &  0.345487 &  0.520485 &  0.464519 &  0.698466 &  0.156988 &  0.632687 \\
             DTI &  0.577284 &   0.680174 &  0.295001 &    0.5809 &  0.654513 &  0.479515 &  0.535481 &  0.301534 &  0.843012 &  0.367313 \\
        \hline
        \end{tabular}}
    \end{subtable}
\end{table}
\begin{figure}[H]
     \centering
     \begin{subfigure}[b]{\textwidth}
         \centering
          \includegraphics[width=\textwidth,trim={0 0.5cm 0 4cm}, scale=1]{figures/UKBB/sig_ratio_deviation_3plots.png}
          \caption{}
          \label{fig:deviations_Dl_Dd}
     \end{subfigure}
        \caption{(a) Significance ratio calculated from $D_{\text{ml}}$, and $D_{\text{mf}}$ and $D_{\text{uf}}$ for the UK Biobank.}\label{fig:deviations_UKBB}
\end{figure}
\begin{figure}
     \centering
     \begin{subfigure}[b]{\textwidth}
         \centering
          \includegraphics[width=\textwidth,trim={0 1cm 0 5cm}]{figures/ADNI/Baseline_vs_proposed_EF_correlations_adjustedforage.png}
          \caption{}
          \label{fig:ADNI_corr_EF}
     \end{subfigure}
     \begin{subfigure}[b]{\textwidth}
         \centering
          \includegraphics[width=\textwidth,trim={0 1cm 0 0cm}]{figures/ADNI/Baseline_vs_proposed_MEM_correlations_adjustedforage.png}
          \caption{}
          \label{fig:ADNI_corr_MEM}
     \end{subfigure}
     \caption{Pearson correlation between $D_{\text{ml}}$ and (a) executive function and (b) memory scores for the T1 normVAE, PoE normVAE and gPoE normVAE models applied to the ADNI dataset.}
\end{figure}
\subsubsection{Product and generalised product of Gaussian's.}
In this section, we provide the derivation for the parameters of the Product of Experts (PoE) and generalised Product of Experts (gPoE) Gaussian product distributions.

\begin{figure}
     \centering
     \begin{subfigure}[b]{0.9\textwidth}
        \begin{minipage}{.5\textwidth}
          \centering
          \includegraphics[width=\textwidth,trim={0 2cm 1cm 3cm}]{figures/ADNI/mean_LMCI_cortical_weighted_mVAE.png}
            \includegraphics[width=\textwidth,trim={0 4cm 1cm 3cm}]{figures/ADNI/mean_LMCI_subcortical_weighted_mVAE.png}
        \end{minipage}%
        \begin{minipage}{.5\textwidth}
          \centering
          \includegraphics[width=0.49\textwidth,trim={0 4cm 3cm 3cm}]{figures/ADNI/FA_mean_LMCI_dti_fa_weighted_mVAE.png}
            \includegraphics[width=0.49\textwidth,trim={0 4cm 3cm 3cm}]{figures/ADNI/MD_mean_LMCI_dti_md_weighted_mVAE.png}
        \end{minipage}
         % \caption{Average $D_{\text{uf}}$ for LMCI cohort}
            \caption{}
          \label{fig:ADNI_LMCI_recon}
     \end{subfigure}
     \begin{subfigure}[b]{0.9\textwidth}
        \begin{minipage}{.5\textwidth}
          \centering
          \includegraphics[width=\textwidth,trim={0 2cm 1cm 3cm}]{figures/ADNI/mean_AD_cortical_weighted_mVAE.png}
            \includegraphics[width=\textwidth,trim={0 4cm 1cm 3cm}]{figures/ADNI/mean_AD_subcortical_weighted_mVAE.png}
        \end{minipage}%
        \begin{minipage}{.5\textwidth}
          \centering
          \includegraphics[width=0.49\textwidth,trim={0 4cm 3cm 3cm}]{figures/ADNI/FA_mean_AD_dti_fa_weighted_mVAE.png}
            \includegraphics[width=0.49\textwidth,trim={0 4cm 3cm 3cm}]{figures/ADNI/MD_mean_AD_dti_md_weighted_mVAE.png}
        \end{minipage}
         \caption{}
          \label{fig:ADNI_AD_recon}
     \end{subfigure}
        \caption{(a) Average $D_{\text{uf}}$ using the gPoE-normVAE ($L_{\text{dim}}$=10) model for the LMCI and (b) AD cohort. The left-hand plots show T1 features and the right-hand plots show DTI features.}\label{fig:deviations_ADNI}
\end{figure}

\textit{Proof.} The probability density of a Gaussian distribution is given by: 

\begin{equation}
p = K \exp\{{-\frac{1}{2}(\textbf{x}-\boldsymbol{\mu})^T\boldsymbol{\Sigma}^{-1}(\textbf{x}-\boldsymbol{\mu})\}} = K\exp{\{\textbf{x}^T\boldsymbol{\Sigma}^{-1}\boldsymbol{\mu}-\frac{1}{2}\textbf{x}^T\boldsymbol{\Sigma}^{-1}\textbf{x}\}}
\end{equation}
where $K$ is a normalisation constant. The product of $N$ Gaussian distributions is a Gaussian of the form: 
\begin{equation}
    \prod_{n=1}^{N} p_n \varpropto \exp{\{\textbf{x}^T \sum_{n=1}^{N} \boldsymbol{\Sigma}^{-1}_{n}\boldsymbol{\mu}_{n}-\frac{1}{2}\textbf{x}^T(\sum_{n=1}^{N}\boldsymbol{\Sigma}^{-1}_{n})\textbf{x}\}}
\end{equation}

where $\boldsymbol{\Sigma}^{-1}\boldsymbol{\mu}=\sum_{n=1}^{N} \boldsymbol{\Sigma}^{-1}_{n}\boldsymbol{\mu}_{n}$ and $\boldsymbol{\Sigma}^{-1} = \sum_{n=1}^{N} \boldsymbol{\Sigma}^{-1}_{n}$. Thus the product Gaussian has a mean $\boldsymbol{\mu} = \frac{\sum_{n=1}^{N}\boldsymbol{\Sigma}^{-1}_{n}\boldsymbol{\mu}_{n}}{\sum_{n=1}^{N} \boldsymbol{\Sigma}^{-1}_{n}}$ and covariance $\boldsymbol{\Sigma}=(\sum_{n=1}^{N} \boldsymbol{\Sigma}^{-1}_{n})^{-1}$. If we have isotropic Gaussian distributions $p_n = \mathcal{N}(\boldsymbol{\mu}_n, \boldsymbol{\sigma}_{n}^{2} \textbf{I})$, the parameters of the product Gaussian become $\boldsymbol{\mu} = \frac{\sum_{n=1}^{N} \boldsymbol{\mu}_{n} / \boldsymbol{\sigma}_{n}^{2}}{\sum_{n=1}^{N} 1 / \boldsymbol{\sigma}_{n}^{2}} \quad \text { and } \quad \boldsymbol{\sigma}^{2} =\frac{1}{\sum_{n=1}^{N} 1 / \boldsymbol{\sigma}_{n}^{2}}$.

Similarly, for a weighted product of $N$ Gaussian distributions, the product Gaussian has the form: 
\begin{equation}
    \prod_{n=1}^{N} p^{\alpha_n}_{n} \varpropto \exp{\{\textbf{x}^T \sum_{n=1}^{N} \alpha_n\boldsymbol{\Sigma}^{-1}_{n}\boldsymbol{\mu}_{n}-\frac{1}{2}\textbf{x}^T(\sum_{n=1}^{N}\alpha_n\boldsymbol{\Sigma}^{-1}_{n})\textbf{x}\}}.
\end{equation}
For isotropic Gaussian distributions, the weighted product Gaussian has mean $
\boldsymbol{\mu} = \frac{\sum_{n=1}^{N} \boldsymbol{\mu}_{n}\boldsymbol{\alpha}_{n} / \boldsymbol{\sigma}_{n}^{2}}{\sum_{n=1}^{N} \boldsymbol{\alpha}_{n} / \boldsymbol{\sigma}_{n}^{2}}$ and variance $ \boldsymbol{\sigma}^{2} = \sum_{n=1}^{N} \frac{1}{ \boldsymbol{\alpha}_{n} / \boldsymbol{\sigma}_{n}^{2}}
$.

\end{document}