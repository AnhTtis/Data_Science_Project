\section{Hyperparameter Sensitivity}
\label{appendix:hyper}
\noindent\textbf{Experiment Settings.} To explore the impact of the hyperparameters and the structure of LOP, we implement the LOP with different values of $B$ and different architectures. 
Table \ref{tab:ablation_normal} report the AP and precision of PointRCNN when our defense is deployed with different settings under the normal circumstance. Fig.\ref{fig:ablation_atk} report the defense effectiveness and the precision of PointRCNN when our defense is deployed with different settings under attacks.

%%%%%%% BEGIN Ablation
\begin{figure*}[t]
    \centering
    \includegraphics[width=0.9\textwidth]{figs/PointRCNN_ablation.png}
    \caption{The ASR of different attacks and the precision of PointRCNN when deploying the LOP with different values of $B$ and different model structures on PointRCNN.
    % \lyf{The precision and ASR metrics are represented in triangles and circles respectively, while the structures of LOP are distinguished in colors.}
    % The blue horizontal dotted lines in (a), (b) and (c) all represent the precision of PointRCNN on cars in the normal circumstances.
    }
    \label{fig:ablation_atk}
\end{figure*}


\begin{table}[ht]
  \centering
  \vspace{-0.2in}
  \caption{The AP and precision of PointRCNN equipped with the LOP with different $B$ and different model structures.}
\scalebox{0.7}{
    \begin{tabular}{lcccc}
    \toprule
          % & \multicolumn{4}{c}{PointRCNN} \\
\cmidrule{2-5}          & AP    & \multicolumn{3}{c}{Precision} \\
\cmidrule(lr){2-2} \cmidrule(lr){3-5}          & Car   & Car   & Pedestrian & Cyclist \\
    \midrule
    w/o. defense  & 75.13\% & 75.04\% & 47.08\% & 56.87\% \\
    \midrule
    Ours(PointNet, B=0.2) & 75.96\% & 76.49\% & 50.70\% & 58.66\% \\
    Ours(PointNet, B=0.3) & 76.21\% & 77.15\% & \textbf{50.94\%} & 58.93\% \\
    Ours(PointNet, B=0.4) & \textbf{76.50\%} & 78.05\% & 50.49\% & 61.59\% \\
    Ours(PointNet, B=0.5) & 76.49\% & 79.29\% & 49.63\% & 61.92\% \\
    Ours(PointNet, B=0.6) & 76.37\% & \textbf{80.03\%} & 49.43\% & \textbf{63.87\%} \\
    \midrule
    Ours(DGCNN, B=0.2) & 76.44\% & 77.56\% & 51.04\% & 61.03\% \\
    Ours(DGCNN, B=0.3) & 76.66\% & 78.36\% & \textbf{51.31\%} & 61.72\% \\
    Ours(DGCNN, B=0.4) & 76.58\% & 79.44\% & 51.21\% & 63.74\% \\
    Ours(DGCNN, B=0.5) & 76.77\% & 80.75\% & 49.59\% & 65.14\% \\
    Ours(DGCNN, B=0.6) & \textbf{76.84\%} & \textbf{81.52\%} & 49.42\% & \textbf{67.24\%} \\
    \bottomrule
    \end{tabular}%
}
%   \vspace{-0.1in}
  \label{tab:ablation_normal}%
\end{table}%
%%%%%%%% END Ablation

\noindent\textbf{Results \& Analysis.}  
As we can see from Table \ref{tab:ablation_normal}, the choice of LOP's structure has limited influence on the performance of PointRCNN under the normal circumstance. The differences between their precision are at most $1.49\%$, $0.72\%$ and $3.37\%$ on cars, pedestrians and cyclists, and the differences between their AP are at most $0.48\%$ on cars.
In contrast, the value of $B$ greatly affects the performance of PointRCNN.
Normally, the higher value of $B$ is realted with better performance of PointRCNN with the LOP: the precision of PointRCNN on cars and on cyclists increase with a larger $B$, while the change of AP on cars is always less than $2\%$. 
% However, the LOP with a lower $B$ may bring PointRCNN better performance on detecting pedestrians, which is mainly because the larger precision of PointRCNN on pedestrians. As we can see from Fig.\ref{fig:ablation_atk}, although the LOP with the structure of DGCNN performs slightly better than the LOP with the structure of PointNet, the differences of the precision are always less than $3.48\%$, and the differences of the ASR are less than $4.00\%$ in most cases, a relatively small gap. Similarly, a higher $B$ always brings the better performance and the better defense effectiveness. For example, when $B=0.6$, the precision of PointRCNN with the LOP on cars increase \lyf{$10.79\%$ on} average, and the ASR of these appearing attacks decreases by $26.56\%$ on average compared with the results of $B=0.2$.

In fact, the point-wise PC model also performs well in other downstream tasks such as classification and semantic segmentation, which means the key features extracted by them is general enough to handle different CV tasks \cite{charles2017PN,charles2017PN++,yue2019DGCNN}. Thus, the LOP with different structures can both perform well in recognizing the components of real objects. However, in the pipeline of our proposed defense the value of $B$ directly determines whether a predicted object is preserved or eliminated. Therefore, the value of $B$ affects the performance of 3D object detectors equipped with the LOP.

% Besides, as a trade-off, the increase in $B$ always causes the degradation in recall in the normal circumstances and under attacks. However, the degradation is limited by $10\%$ on average. Based on our discussion in Section \ref{sec:Limitation}, we consider it as a reasonable trade-off because the slight decrease in recall has limited influence on the normal function of ADS in the real word. In summary, the effectiveness of our proposed defense is insensitive to different choices of the model structure of LOP, while the value of $B$ does play an important role on the contrary. Regardless of the performance of the 3D object detectors in the normal circumstances and the acceptable trading of recall, the LOP with a higher $B$ can always bring the 3D object detector larger improvement.