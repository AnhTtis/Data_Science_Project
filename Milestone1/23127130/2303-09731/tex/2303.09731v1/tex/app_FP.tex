\section{A Case Study of the False Positives}
\label{appendix:FP}
Specifically, we investigate the origin of the false positives with the three factors: the size, the depth and the density of the detected objects. In our analysis, we randomly select $1000$ sample frames of point clouds from the validation set of KITTI, and calculate the three factors above for the contained objects.

%%%%%%%% BEGIN FP
\begin{figure}[t]
    \centering
    \includegraphics[width=0.45\textwidth]{figs/FP_analysis.png}
    \caption{The Visualization of false positive detected by our LOP, where the solid boxes refer to the remained real vehicles after LOP's detection, and dashed boxes refer to the real vehicles erased by LOP.}
    \label{fig:FP_analysis}
\end{figure}
%%%%%%%% END FP

Indeed, the distribution of the false positives and the true positives are similar to one another in terms of the size and the density, while non-trivial differences exist in their depth. Specifically, false negatives are relatively more likely to exist when the depth is larger: the ratio between the false positive with depth less than $10$m and those with depth more than $10$m is $1:7.37$, while the ratio between the false positive with depth less than $20$m and those with depth more than $20$m is $1:5.05$. Furthermore, the number of false positive with depth less than $10$m is only $1.60\%$ of the total real vehicles, and the number of false positive with depth less than $10$m is only $2.21\%$ of the total real vehicles. The results indicate that our LOP may not recognize the forged obstacles well in some cases due to its imprecision on distant vehicles. 

For better intuition, we visualize one sample scenario in Fig.\ref{fig:FP_analysis}. As we can see, compared with the false positives (marked in dotted lines), the true positives usually have a lower depth and its point density were unevenly distributed and dense in some specific parts. Therefore, the true positives usually have less pillars which however contain sufficient points for LOP to analyze. Therefore, they were easier to be detected with the sufficient information contained in the dense pillars. In contrast, the pillars in false positives are almost evenly sparse and hence would increase the uncertainty of our LOP's prediction. The visualization explains why LOP is limited in recognizing some distant crafted objects. 

Finally, due to the existence of the MOT module, which tracks the detected objects and supply the trajectories of nearby objects for further route planning, the self-driving system will keep refreshing the driving plan and correct the mis-prediction of distant objects when it comes nearby. Moreover, MOT would prevent the self-driving system from ignoring a distant object only if LOP misses a distant object in several consecutive frames, the possibility of which is less than $1\%$ according to our calculation. Therefore, the negative influence of LOP on the normal performance of a detector can be further limited.