Specifically, we selected 5 different fake PCs based on the previous experiments which are located in front of the self-driving vehicle in 10 meters and can successfully forged the perception module of Apollo for at least one frame. Meanwhile, we use Apollo to record 3 different traces of the self-driving vehicle moving under manually control. Then, we inject these fake PCs into all the frames of these traces to generate 15 different fully poisoned traces. During the experiments, we run Apollo 6.0.0 system with and without LOP on these traces and visualize the future routes generated by the planning module of Apollo (which is also shown as a green rectangle in front of the self-driving vehicle) in the Dreamview, and assume there is a car moving behind the self-driving vehicle at a constant speed. Based on the generated future route, we can determine whether the self-driving vehicles will suddenly brake if they are controlled by the Apollo system, and consider the brake as a potential crash.

As shown in the part (a) of Fig.\ref{fig:sys_level_apollo}, the generated future route is extended to the crossroads, which means the self-driving vehicle will moving normally and stop at a red light under the instructions of the Apollo with our LOP. Therefore, we consider this situation as ``no accident''. Meanwhile, as shown in the part (b) of Fig.\ref{fig:sys_level_apollo}, the generated future route disappears for a while, which means the self-driving vehicle will falsely stop in the middle of the road and hit by the potential car moving behind it. Therefore, we consider this situation as ``car crash''. The crash rate we report in the Section 5.4 are calculated based on the number of ``car crash'' in the 15 poisoned traces above.