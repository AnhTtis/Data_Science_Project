\section{Details of The System-level Evaluation}
\label{appendix:system}

\begin{figure}[t]
    \centering
    \includegraphics[width=0.45\textwidth]{figs/SysLevelTest2.png}
    \caption{The Dreamview of Apollo without and with our LOP while the attack is aiming at interrupting the route planning.}
    \label{fig:sys_level_apollo}
\end{figure}

Specifically, to perform these experiments, we first select 5 different fake PCs which are located in front of the self-driving vehicle and can successfully forge the perception module of Apollo or those 3D object detectors for at least 1 frame in the previous experiments. 
% Meanwhile, we use Apollo to record the traces of the self-driving vehicle moving in 3 different maps. 
Then, we inject the selected fake PCs into 15 testing traces to conduct the appearing attack against the self-driving vehicle. 

We use the Dreamview to visualize the generated future routes of Apollo 6.0.0, which are shown as the green rectangles in front of the self-driving vehicle and represent the moving trajectories of the self-driving vehicle under the Apollo's control. Based on these routes and the planning module, we define that the self-driving vehicle is harsh braking when the planning module guide the self-driving vehicle decelerate to 0 km/h in less than 1 second.

As shown in the part (a) of Fig.\ref{fig:sys_level_apollo}, the generated future route is extended to the crossroads, which means the self-driving vehicle will move normally and stop at a red light under the instructions of the Apollo with our LOP. Therefore, we consider this situation as ``normal''. Meanwhile, as shown in the part (b) of Fig.\ref{fig:sys_level_apollo}, the generated future route disappears for a while, and the planning module guide the self-driving vehicle stop in a certain place, which means the self-driving vehicle will falsely brake in the middle of the road. Therefore, we consider this situation as a ``harsh braking''. We calculate the proportion of the ``harsh braking'' in the 15 poisoned traces as the harsh braking rate, and report it in Section \ref{sec:Experiment:results:real}.

% Specifically, we record the trace of the self-diving car and calculate the crash rate based on the planning route shown in the Dreamview. As shown in Fig.\ref{fig:sys_level_apollo}, the Apollo without LOP will be misled by the attacks and the planning module will falsely suggest to stop behind the forged car, which may further lead to the car crash between the self-driving car and the vehicles behind it. At the same time, the Apollo with LOP can identify the forged car and ignore it in the perception module, so its planning module can correctly generating the route during the whole trace and avoid the accidents.

