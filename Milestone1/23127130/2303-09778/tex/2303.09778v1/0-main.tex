\documentclass[sigconf]{acmart}


\usepackage{amsmath}
\usepackage{bm}
\usepackage{mathrsfs}
\usepackage{amsthm}
\usepackage{amsfonts}
\usepackage{subfigure}
\usepackage{multirow}
\usepackage{url}
% \usepackage{tabularray}
% \UseTblrLibrary{booktabs}
\usepackage{colortbl}
\usepackage[ruled,linesnumbered,vlined]{algorithm2e}
\usepackage{makecell}
\usepackage{enumitem}
\usepackage{diagbox}
\usepackage{float}




%% \BibTeX command to typeset BibTeX logo in the docs
\AtBeginDocument{%
  \providecommand\BibTeX{{%
    \normalfont B\kern-0.5em{\scshape i\kern-0.25em b}\kern-0.8em\TeX}}}

%% Rights management information.  This information is sent to you
%% when you complete the rights form.  These commands have SAMPLE
%% values in them; it is your responsibility as an author to replace
%% the commands and values with those provided to you when you
%% complete the rights form.


\copyrightyear{2023}
\acmYear{2023}
\setcopyright{acmlicensed}\acmConference[WWW '23]{Proceedings of the ACM Web Conference 2023}{April 30-May 4, 2023}{Austin, TX, USA}
\acmBooktitle{Proceedings of the ACM Web Conference 2023 (WWW '23), April 30-May 4, 2023, Austin, TX, USA}
\acmPrice{15.00}
\acmDOI{10.1145/3543507.3583453}
\acmISBN{978-1-4503-9416-1/23/04}

%%
%% end of the preamble, start of the body of the document source.
\begin{document}

%%commands
\newcommand{\framework}{SE-GSL}
\newcommand\negimp[1]{\textcolor[RGB]{255,180,0}{#1}}
\newcommand\posimp[1]{\textcolor[RGB]{0,190,0}{#1}}
\theoremstyle{definition}
\newtheorem{define}{Definition}[]
%%
%% The "title" command has an optional parameter,
%% allowing the author to define a "short title" to be used in page headers.
\title{\framework: A General and Effective Graph Structure Learning Framework through Structural Entropy Optimization}

% \author{
% Dongcheng Zou$^1$,
% Xiang Huang$^1$,
% Hao Peng$^1$,
% Renyu Yang$^1$,
% Jia Wu$^2$,
% Jianxin Li$^1$,
% Chunyang Liu$^1$, and 
% Philip S. Yu$^3$
% }
% \affiliation{
% \institute{
% $^1$Beihang University;
% $^2$Macquarie University;
% $^3$University of Illinois Chicago
% }
% }
% \email{
% {zoudongcheng, huang.xiang, penghao, renyu.yang, lijx}@buaa.edu.cn, 
% jia.wu@mq.edu.au, 
% liuchunyang@didiglobal.com, 
% psyu@uic.edu
% }


\author{Dongcheng Zou}
\affiliation{%
  \institution{Beihang University}
  \city{Beijing}
  \country{China}
}
\email{zoudongcheng@buaa.edu.cn}

\author{Hao Peng}
\affiliation{%
  \institution{Beihang University}
  \city{Beijing}
  \country{China}
}
\email{penghao@buaa.edu.cn}
\authornote{Corresponding author}


\author{Xiang Huang}
\affiliation{%
  \institution{Beihang University}
  \city{Beijing}
  \country{China}
}
\email{huang.xiang@buaa.edu.cn}



\author{Renyu Yang}
\affiliation{%
  \institution{Beihang University}
  \city{Beijing}
  \country{China}
}
\email{renyu.yang@buaa.edu.cn}


\author{Jianxin Li}
\affiliation{%
  \institution{Beihang University}
  \city{Beijing}
  \country{China}
}
\email{lijx@buaa.edu.cn}

\author{Jia Wu}
\affiliation{%
  \institution{Macquarie University}
  \city{Sydney}
  %\state{NSW}
  \country{Australia}
}
\email{jia.wu@mq.edu.au}


\author{Chunyang Liu}
\affiliation{%
  \institution{Didi Chuxing}
  \city{Beijing}
  \country{China}
}
\email{liuchunyang@didiglobal.com}

\author{Philip S. Yu}
\affiliation{%
  \institution{University of Illinois Chicago}
  \city{Chicago}
  %\state{IL}
  \country{USA}
}
\email{psyu@uic.edu}


\renewcommand{\shortauthors}{Zou and Peng, et al.}


%%
%% The abstract is a short summary of the work to be presented in the article.
\begin{abstract}
Graph Neural Networks (GNNs) are de facto solutions to structural data learning. However, it is susceptible to low-quality and unreliable structure, which has been a norm rather than an exception in real-world graphs. Existing graph structure learning (GSL) frameworks still lack robustness and interpretability.
% in noisy or heterophily graphs. 
This paper proposes a general GSL framework, \framework{}, through structural entropy and the graph hierarchy abstracted in the encoding tree. Particularly, we exploit the one-dimensional structural entropy to maximize embedded information content when auxiliary neighbourhood attributes is fused to enhance the original graph. 
A new scheme of constructing optimal encoding trees is proposed to minimize the uncertainty and noises in the graph whilst assuring proper community partition in hierarchical abstraction. 
We present a novel sample-based mechanism for restoring the graph structure via node structural entropy distribution. It increases the connectivity among nodes with larger uncertainty in lower-level communities.
\ \framework\ is compatible with various GNN models and enhances the robustness towards noisy and heterophily structures. 
Extensive experiments show significant improvements in the effectiveness and robustness of structure learning and node representation learning.
\end{abstract}






%%
%% The code below is generated by the tool at http://dl.acm.org/ccs.cfm.
%% Please copy and paste the code instead of the example below.
%%

\begin{CCSXML}
<ccs2012>
   <concept>
       <concept_id>10010147.10010178</concept_id>
       <concept_desc>Computing methodologies~Artificial intelligence</concept_desc>
       <concept_significance>500</concept_significance>
       </concept>
   <concept>
       <concept_id>10002950.10003624.10003633.10010917</concept_id>
       <concept_desc>Mathematics of computing~Graph algorithms</concept_desc>
       <concept_significance>300</concept_significance>
       </concept>
   <concept>
       <concept_id>10002951.10003227.10003351</concept_id>
       <concept_desc>Information systems~Data mining</concept_desc>
       <concept_significance>300</concept_significance>
       </concept>
 </ccs2012>
\end{CCSXML}

\ccsdesc[500]{Computing methodologies~Artificial intelligence}
\ccsdesc[300]{Mathematics of computing~Graph algorithms}
\ccsdesc[300]{Information systems~Data mining}

%%
%% Keywords. The author(s) should pick words that accurately describe
%% the work being presented. Separate the keywords with commas.
\keywords{Graph structure learning, structural entropy, graph neural network}


\maketitle

\section{Introduction}


Recent years have witnessed the rise of human digitization~\cite{habermannDeepCapMonocularHuman2020,alexanderCREATINGPHOTOREALDIGITAL,pengNeuralBodyImplicit2021,alldieckDetailedHumanAvatars2018, rajANRArticulatedNeural2020}. This technology greatly impacts the entertainment, education, design, and engineering industry.
There is a well-developed industry solution for this task.
High-fidelity reconstruction of humans can be achieved either with full-body laser scans~\cite{saitoSCANimateWeaklySupervised2021}, dense synchronized multi-view cameras~\cite{xiangModelingClothingSeparate2021a,xiangDressingAvatarsDeep2022a}, or light stages~\cite{alexanderCREATINGPHOTOREALDIGITAL}.
However, these settings are expensive and tedious to deploy and consist of a complex processing pipeline, preventing the technology's democratization.

Another solution is to view the problem as inverse rendering and learn digital humans directly from custom-collected data.
Traditional approaches directly optimize explicit mesh representation~\cite{loperSMPLSkinnedMultiperson2015, fangRMPERegionalMultiperson2018, pavlakosExpressiveBodyCapture2019} which suffers from the problems of smooth geometry and coarse textures~\cite{prokudinSMPLpixNeuralAvatars2020,alldieckVideoBasedReconstruction2018}. Besides, they require professional artists to design human templates, rigging, and unwrapped UV coordinates.
Recently, with the help of volumetric-based implicit representations~\cite{mildenhallNeRFRepresentingScenes2020, parkDeepSDFLearningContinuous2019, meschederOccupancyNetworksLearning2019} and neural rendering~\cite{laineModularPrimitivesHighPerformance2020, liuSoftRasterizerDifferentiable2019, thiesDeferredNeuralRendering2019}, 
one can easily digitize a quality-plausible human avatar from video footage~\cite{jiangNeuManNeuralHuman2022,wengHumanNeRFFreeviewpointRendering}.
Particularly, volumetric-based implicit representations~\cite{mildenhallNeRFRepresentingScenes2020, pengNeuralBodyImplicit2021} can reconstruct scenes or objects with much higher fidelity against previous neural renderer~\cite{thiesDeferredNeuralRendering2019,prokudinSMPLpixNeuralAvatars2020}, and is more user-friendly as it does not need any human templates, pre-set rigging, or UV coordinates.
Captured visual footage and corresponding skeleton tracking are enough for training.
However, better reconstructions and more friendly usability are at the expense of the following factors.
1) \textbf{Inefficiency:}
They require longer optimization times (typically tens of hours or days) and inference slowly.
Volume rendering~\cite{mildenhallNeRFRepresentingScenes2020,lombardiNeuralVolumesLearning2019} formulates images by querying the densities and colors of millions of spatial coordinates. 
In the training stage, due to memory constraints, only a small fraction of points are sampled which leads to slow convergence speed.
2) \textbf{Entangled representations}:
The geometry, materials, and motion dynamics are entangled in the neural networks. 
Due to the implicit nature of neural nets, one can hardly edit one property without touching the others~\cite{yuanNeRFEditingGeometryEditing2022a,liuEditingConditionalRadiance2021}.
3) \textbf{Graphics incompatibility}:
Volume rendering is incompatible with the current popular graphic pipeline,
which renders triangular/quadrilateral meshes efficiently with the rasterization technique.
Many downstream applications require mesh rasterization in their workflow (\eg, editing~\cite{foundationBlenderOrgHome}, simulation~\cite{benderPositionBasedSimulationMethods2015}, real-time rendering~\cite{akenine2019real}, ray-tracing~\cite{waldRTXRayTracing}).
Although there are approaches~\cite{lorensenMarchingCubesHigh,labelleIsosurfaceStuffingFast2007} can convert volumetric fields into meshes, the gaps from discrete sampling degrade the output quality in terms of both meshes and textures.


To address these issues, we present \textbf{EMA}, a method based on \textbf{E}fficient \textbf{M}eshy neural fields to reconstruct animatable human \textbf{A}vatars.
Our method enjoys flexibility from implicit representations and efficiency from explicit meshes, yet still maintains high-fidelity reconstruction quality.
Given video sequences and the corresponding pose tracking, our method digitizes humans in terms of canonical triangular meshes, physically-based rendering (PBR) materials, and skinning weights \textit{w.r.t.} skeletons.
We jointly learn the above components via inverse rendering~\cite{laineModularPrimitivesHighPerformance2020,chenDIBRLearningPredict2021,chenLearningPredict3D2019} in an end-to-end manner.
Each of them is derived from a separate neural field, which relaxes the requirements of a preset human template, rigging, or UV coordinates.
Specifically, we predict a canonical mesh out of a signed distance field (SDF) by differentiable marching tetrahedra~\cite{shenDeepMarchingTetrahedra2021,gaoGET3DGenerativeModel,gaoLearningDeformableTetrahedral2020,munkbergExtractingTriangular3D2022}, then we extend the marching tetrahedra~\cite{shenDeepMarchingTetrahedra2021} for spatial-varying materials by utilizing a neural field to predict PBR materials \textit{on the mesh surfaces} after rasterization~\cite{munkbergExtractingTriangular3D2022,hasselgrenShapeLightMaterial2022,laineModularPrimitivesHighPerformance2020}.
To make the canonical mesh animatable, we take another neural field to model the forward linear blend skinning for the meshes. 
Given a posed skeleton, the canonical mesh is then transformed into the corresponding poses.
Finally, we shade the mesh with a rasterization-based differentiable renderer~\cite{laineModularPrimitivesHighPerformance2020} and train our models with a photo-metric loss.
After training, we export the mesh with materials and discard the neural fields.

\looseness=-1
There are several merits of our method design.
1) \textbf{Efficiency}:
Powered by efficient mesh rendering, our method can render in real-time.
Besides, the training speed is boosted as well, 
since we compute loss holistically on the whole image and the gradients only flow on the mesh surface. In contrast, volume rendering takes limited pixels for loss computation and back-propagates the gradients in the whole space.
Our method only needs about an hour of training and minutes of optimization are enough for plausible avatar reconstruction.
2) \textbf{Disentangled representations}:
Our shape, materials, and motion modules are disentangled naturally by design, which facilitates editing. 
Besides, Canonical meshes with forward skinning modeling handle the out-of-distribution poses better.
3) \textbf{Graphics compatibility}:
Our derived mesh representation is compatible with 
the prominent graphic pipeline, which leads to instant downstream applications (\eg, the shape and materials can be edited directly in design software~\cite{foundationBlenderOrgHome}).
To further improve reconstruction quality, we additionally optimize image-based environment lights and non-rigid motions.


We conduct extensive experiments on standards benchmarks H36M~\cite{ionescuHuman36MLarge2014b} and ZJU-MoCap~\cite{pengNeuralBodyImplicit2021}.
Our method achieves very competitive performance for novel view synthesis, generalizes better for novel poses, 
and significantly improves both training time and inference speed against previous arts.
Our research-oriented code reaches real-time inference speed (100+ FPS for rendering $512\times512$ images).
We in addition showcase applications including novel pose synthesis, material editing, and relighting.
\section{Preliminaries}\label{sec:prelim}
This section formally reviews the basic concepts of Graph, Graph Neural Networks (GNNs), Graph Structure Learning (GSL), and Structural Entropy. 
Important notations are given in Appendix ~\ref{appendix:notations}.

\subsection{Graph and Graph Structure Learning}
\noindent \textbf{Graph and Community}. 
% \subsubsection{Graph and Communities}
Let $G = \{V,E,X\}$ denote a graph, where $V$ is the set of $n$ vertices\footnote{A vertex is defined in the graph and a node in the tree.}, $E \subseteq V \times V$ is the edge set, and $X \in \mathbb{R}^{n\times d}$ refers to the vertex attribute set.
$\mathrm {A} \in \mathbb{R}^{n \times n}$ denotes the adjacency matrix of $G$, where $\mathrm {A}_{ij}$ is referred to as the weight of the edge between vertex $i$ and vertex $j$ in $G$. 
Particularly, if $G$ is unweighted, $\mathrm {A} \in \{0,1\}^{n \times n}$ and $\mathrm {A}_{ij}$ only indicate the existence of the edges. 
In our work, we only consider the undirected graph, where $\mathrm {A}_{ij} = \mathrm {A}_{ji}$. 
For any vertex $v_i$, the degree of $v_i$ is defined as $d(v_i) = \sum_{j}\mathrm {A}_{ij}$, and $D = \mathrm {diag}(d(v_1),d(v_2),\dots,d(v_n))$ refers to the degree matrix.

Suppose that $\mathcal{P} =
\{P_1, P_2,\dots, P_L\}$ is a partition of $V$. 
Each $P_i$ is called a \textit{community} (aka. module or cluster), representing a group of vertices with commonality.
% such that the density of edges between vertices is higher than the density with the rest of the graph. 
Due to the grouping nature of a real-world network, each community of the graph can be hierarchically split into multi-level \textit{sub-communities}. 
Such \textit{hierarchical community} partition (i.e., hierarchical \textit{semantic}) of a graph can be intrinsically abstracted as the encoding tree~\cite{li2016structural,li2018decoding}, and each tree node represents a specific community. 
Take Fig.~\ref{fig:intro} as an example: at a high abstraction (semantic) level, the entire graph can be categorized as two coarse-grained communities, i.e., teachers (T) and students (S). 
Students can be identified as sub-communities like S.1 and S.2, as per the class placement scheme.














\noindent \textbf{Graph Structure Learning (GSL)}. 
For any given graph $G$, the goal of GSL~\cite{zhu2021deep} is to simultaneously learn an optimal graph structure $G^*$ optimized for a specific downstream task and the corresponding graph representation $Z$. 
In general, the objective of GSL can be summarized as $\mathcal{L}_{gsl} = \mathcal{L}_{task}(Z,Y) + \alpha \mathcal{L}_{reg}(Z,G^*,G)$,
% \begin{equation}\label{eq:LGSL}
% \end{equation}
where $\mathcal{L}_{task}$ refers to a task-specific objective with respect to the learned representation $Z$ and the ground truth $Y$.
$\mathcal{L}_{reg}$ imposes constraints on the learned graph structure and representations, and $\alpha$ is a hyper-parameter.




\subsection{Structural Entropy}
Different from information entropy (aka. Shannon entropy) that measures the uncertainty of probability distribution in information theory ~\cite{shannon1948mathematical}, \textit{structural entropy}~\cite{li2016structural} measures the structural system diversity, e.g., the uncertainty embedded in a graph.








\noindent \textbf{Encoding Tree}. 
Formally, the encoding tree $\mathcal{T}$ of graph $G=(V, E)$ holds the following properties:
\textbf{(1)} The root node $\lambda$ in $\mathcal{T}$ has a label $T_\lambda = V$, $V$ represents the set of all vertices in $G$.
\textbf{(2)} Each non-root node $\alpha$ has a label $T_\alpha \subset V$. 
Furthermore, if $\alpha$ is a leaf node, $T_\alpha$ is a singleton with one vertex in $V$.
\textbf{(3)} For each non-root node $\alpha$, its parent node in $T$ is denoted as $\alpha^-$.
\textbf{(4)} For each non-leaf node $\alpha$, its $i$-th children node is denoted as $\alpha^{\left \langle i \right \rangle }$ ordered from left to right as $i$ increases.
\textbf{(5)} For each non-leaf node $\alpha$, assuming the number of children $\alpha$ is $N$, all vertex subset $T_{\alpha^{\left \langle i \right \rangle}}$ form a partition of $T_\alpha$, written as $T_\alpha = {\textstyle \bigcup_{i=1}^{N}} T_{\alpha^{\left \langle i \right \rangle}}$ and ${\textstyle \bigcap_{i=1}^{N}} T_{\alpha^{\left \langle i \right \rangle}} = \varnothing$.
If the encoding tree's height is restricted to $K$, we call it \textit{$K$-level} encoding tree. 
Entropy measures can be conducted on different encoding trees.








\noindent \textbf{One-dimensional Structural Entropy}. \label{prelim:1dse}
In a single-level encoding tree $\mathcal{T}$, its structural entropy degenerates to the unstructured Shannon entropy, which is formulated as:
\begin{equation}\label{eq:H1}
H^1(G) = -\sum_{v \in V}{\frac{d_v}{vol(G)}\log_{2}{\frac{d_v}{vol(G)}}},
\end{equation}
where $d_v$ is the degree of vertex $v$, and $vol(G)$ is the sum of the degrees of all vertices in $G$.
According to the fundamental research~\cite{li2016structural}, one-dimensional structural entropy $H^1(G)$ measures the uncertainty of vertex set $V$ in $G$, which is also the upper bound on the amount of information embedded in $G$.

\noindent \textbf{High-dimensional Structural Entropy}. 
For the encoding tree $\mathcal{T}$, we define high-dimensional structural entropy of $G$ as:
\begin{equation}\label{eq:HK}
H^K(G) = \min_{\forall \mathcal{T}:height(\mathcal{T}) \le K}\{H^{\mathcal{T}}(G)\},
\end{equation}
\begin{equation}\label{eq:HT}
% \vspace{-0.1em}
H^{\mathcal{T}}(G) = \sum_{\alpha \in \mathcal{T},\alpha \ne \lambda} {H^{\mathcal{T}}(G;\alpha)} = -\sum_{\alpha \in \mathcal{T},\alpha \ne \lambda} {\frac{g_\alpha}{vol(G)}\log_{2}{\frac{\mathcal{V}_{\alpha}}{\mathcal{V}_{\alpha^-}}}},
\end{equation}
where $g_\alpha$ is the sum weights of the cut edge set $[T_\alpha,T_\alpha/T_\lambda]$, i.e., all edges connecting vertices inside $T_\alpha$ with vertices outside $T_\alpha$. $\mathcal{V}_\alpha$ is the sum of degrees of all vertices in $T_\alpha$. 
$H^{\mathcal{T}}(G;\alpha)$ is the structural entropy of node $\alpha$ and $H^{\mathcal{T}}(G)$ is the structural entropy of $\mathcal{T}$. 
$H^K(G)$ is the $K$-dimensional structural entropy, with the optimal encoding tree of $K$-level .%% 2-related work
\section{Our Approach}\label{sec:framework}

\begin{figure*}[t]
  \centering
  \includegraphics[width=0.98\textwidth]{framework.pdf}
  \caption{The overall architecture of\ \framework.}
%   \FIXME{change $G^{(k)}$ to $G_{knn}^{(k)}$}
  \label{fig:framework}
  \Description{The overall architecture of SE-GSL.}
\end{figure*}







This section presents the architecture of \framework{}, then elaborate on how we enhance the graph structure learning by structural entropy-based optimization of the hierarchical encoding tree.


\subsection{Overview of ~\framework}
Fig.~\ref{fig:framework} depicts the overall pipeline. 
At the core of ~\framework{} is the structure optimization procedure that transforms and enhances the graph structure. 
More specifically, it encompasses multi-stages: graph structure enhancement, hierarchical encoding tree generation, and sampling-based structure reconstruction before an iterative representation optimization.

First, the original topological information is integrated with vertex attributes and the neighborhood in close proximity. 
Specifically, we devise a similarity-based edge reweighting mechanism and incorporate $k$-NN graph structuralization to provide auxiliary edge information. The most suitable $k$ is selected under the guidance of the one-dimensional structural entropy maximization strategy (\S~\ref{step1}). 
Upon the enhanced graph, we present a hierarchical abstraction mechanism to further suppress the edge noise and reveal the high-level hierarchical community structure (encoding tree) (\S~\ref{step2}). 
A novel sampling-based approach is designed to build new graph topology from the encoding tree, particularly by restoring the edge connectivity from the tree hierarchy (\S~\ref{step3}). 
The core idea is to weaken the association between high-level communities whilst establishing dense and extensive connections within low-level communities. 
To this end, we transform the node structural entropy into probability, rejecting the deterministic threshold. 
Through multi-iterative stochastic sampling, it is more likely to find favorable graph structures for GNNs. 
Afterward, the rebuilt graph will be fed into the downstream generic GNN encoders. To constantly improve both the node representation and the graph structure, the optimization pipeline is iterated for multiple epochs.
% -- the optimized node representation and graph structure will go into the next epoch as the new input. 
The training procedure of\ \framework{}\ is summarized in Appendix~\ref{appendix:overall algorithm}. 





%介绍1维结构熵图强化
\subsection{Graph Structure Enhancement}\label{step1}

To fully incorporate vertex attributes and neighborhood information in the graph structure, we perform feature fusion and edge reweighting so that the topological structure, together with the informative vertex adjacent similarity, can be passed on to the encoding tree generator. 
To begin with, we calculate the pair-wise similarity matrix $S \in \mathbb{R}^{|V|\times |V|}$ among vertices in graph $G$. 
To better depict the linear correlation between two vertex attributes, we take the Pearson correlation coefficient (PCC) as the similarity measure in the experiments, i.e.,
\begin{equation}\label{eq:pcc}
S_{ij}=\mathrm{PCC}(x_i,x_j)=\frac{E((x_i-u_i)(x_j-u_j))}{\sigma_i\sigma_j},
\end{equation}
where $x_i$ and $x_j \in \mathbb{R}^{1 \times d}$ are the attribute vectors of vertices $i$ and $j$, respectively. $u_i$ and $\sigma_i$ denote the mean value and variance  of $x_i$, and $E(\cdot)$ is the dot product function.
Based on $S$, we can intrinsically construct the $k$-NN graph $G_{knn} = \{V,E_{knn}\}$ where each edge in $E_{knn}$ represents a vertex and its $k$ nearest neighbors (e.g., the edges in red in Fig ~\ref{fig:framework}). We fuse $G_{knn}$ and the original $G$ to $G_{f} = \{V,E_{f} = E \cup E_{knn}\}$.  

A key follow-up step is pinpointing the most suitable number $k$ of nearest neighbors. An excessive value of $k$ would make $G_{f}$ over-noisy and computationally inefficient, while a small $k$ would result in insufficient information and difficulties in hierarchy extraction. As outlined in \S~\ref{prelim:1dse}, a larger one-dimensional structural entropy indicates more information that $G_{f}$ can potentially hold. Hence, we aim to maximize the one-dimensional structural entropy $H^1(G_{f})$ to guide the selection of $k$ for larger encoding information. In practice, we gradually increase the integer parameter $k$, generate the corresponding $G^{(k)}_{f}$ and compute  $H^1(G^{(k)}_{f})$. Observably, when $k$ reaches a threshold $k_m$, $H^1(G^{(k)}_{f})$ comes into a plateau without noticeable increase. This motivates us to regard this critical point $k_m$ as the target parameter. The $k$-selector algorithm is depicted in Appendix ~\ref{appendix:1dse algorithm}. 
In addition, the edge $e_{ij}$ between $v_i$ and $v_j$ is reweighted as: 
\begin{equation}\label{eq:reweighted}
    \omega_{ij}=S_{ij}+M, \quad M = \frac{1}{2|V|} \cdot \frac{1}{|E|}\sum_{1<i,j<n}{S_{ij}}, 
\end{equation}
where $M$ is a modification factor that amplifies the trivial edge weights and thus makes the $k$-selector more sensitive to noises.



\subsection{Hierarchical Encoding Tree Generation}\label{step2}

Our methodology of abstracting the fused graph into a hierarchy is inspired by the structural entropy theory~\cite{li2016structural,li2018decoding}. We intend to minimize the graph uncertainty (i.e., edge noises) and maximize the knowledge embedded (e.g., optimal partition) in the high-dimensional hierarchy. Correspondingly, the objective of structural entropy minimization is to find out an encoding tree $\mathcal{T}$ that minimizes $H^\mathcal{T}(G_f)$ defined in Eq.~\ref{eq:HT}.
Due to the difficulty in graph semantic complexity quantification, we restrict the optimization objective to the $K$-level tree with a hyperparameter $K$. The optimal $K$-dimensional encoding tree is represented as:
\begin{equation}\label{eq:T}
\mathcal{T^*} = \mathop{\arg\min}\limits_{\forall\mathcal{T}:height(\mathcal{T})\le K}(H^\mathcal{T}(G)).
\end{equation}

To address this optimization problem, we design a greedy-based heuristic algorithm to approximate $H^K(G)$. To assist the greedy heuristic, we define two basic operators:


\begin{define}
\label{def:CBop}
\textbf{Combining operator:}~
Given an encoding tree $\mathcal{T}$ for $G=(V,E)$, let $\alpha$ and $\beta$ be two nodes in $\mathcal{T}$ sharing the same parent $\gamma$. 
The combining operator $\mathrm{CB}_{\mathcal{T}}(\alpha,\beta)$ updates the encoding tree as: $\gamma \gets \delta^-; \delta \gets \alpha^-; \delta \gets \beta^-.$ 
A new node $\delta$ is inserted between $\gamma$ and its children $\alpha, \beta$.
\end{define}
\begin{define}
\label{def:LFop}
\textbf{Lifting operator:}~
Given an encoding tree $\mathcal{T}$ for $G=(V,E)$, let $\alpha$, $\beta$ and $\gamma$ be the nodes in $\mathcal{T}$, satisfying $\beta^-=\gamma$ and $\alpha^-=\beta$.
The lifting operator $\mathrm{LF}_{\mathcal{T}}(\alpha,\beta)$ updates the encoding tree as:  $\gamma \gets \alpha^-; \mathrm{IF:}T_{\beta}=\varnothing, \mathrm{THEN:}\mathrm{drop}(\beta).$
The subtree rooted at $\alpha$ is lifted by placing itself as $\gamma$'s child. If no more children exist after lifting, $\beta$ will be deleted from $\mathcal{T}$.
\end{define}



In light of the high-dimensional structural entropy minimization principle~\cite{li2018decoding}, we first build a full-height binary encoding tree by greedily performing the combining operations. 
Two children of the root are combined to form a new partition iteratively until the structural entropy is no longer reduced. 
To satisfy the height restriction, we further squeeze the encoding tree by lifting subtrees to higher levels. 
To do so, we select and conduct lifting operations between a non-root node and its parent node that can reduce the structural entropy to the maximum.
This will be repeated until the encoding tree height is less than $K$ and the structural entropy can no longer be decreased.
Eventually, we obtain an encoding tree with a specific height $K$ with minimal structural entropy. The pseudo-code is detailed in Appendix~\ref{appendix:kdse algorithm}.






\subsection{Sample-based Graph Reconstruction}
\label{step3}

The subsequent step is to restore the topological structure from the hierarchy whilst guaranteeing the established hierarchical semantics in optimal encoding tree $\mathcal{T}^*$. 
The key to graph reconstruction is determining which edges to augment or weaken. 
Intuitively, the nodes in real-life graphs in different communities tend to have different labels. The work~\cite{zhu2020cagnn} has demonstrated the effectiveness of strengthening intra-cluster edges and reducing inter-cluster edges in a cluster-awareness approach to refine the graph topology. However, for hierarchical communities, simply removing cross-community edges will undermine the integrity of the higher-level community. Adding edges within communities could also incur additional edge noises to lower-level partitioning. 

We optimize the graph structure with community preservation by investigating the structural entropy of \textit{deduction} between two interrelated nodes as the criterion of edge reconstruction:

\begin{define}
\label{def:deduct-se}
\textbf{Structural entropy of deduction:}~
Let $\mathcal{T}$ be an encoding tree of $G$. We define the structural entropy of the deduction from non-leaf node $\lambda$ to its descendant $\alpha$ as:
\begin{equation}\label{eq:deduct-se}
% H^{\mathcal{T}}(G;(\lambda,\alpha]) = \sum_{\beta,\lambda \subset \beta \subseteq \alpha}{H^{\mathcal{T}}(G;\beta)}.
H^{\mathcal{T}}(G;(\lambda,\alpha]) = \sum_{\beta,T_\alpha \subseteq T_\beta \subset T_\lambda}{H^{\mathcal{T}}(G;\beta)}.
% \vspace{-0.4em}
\end{equation}
 This node structure entropy definition exploits the hierarchical structure of the encoding tree and offers a generic measure of top-down deduction to determine a community or vertex in the graph.
\end{define}

From the viewpoint of message passing, vertices with higher structural entropy of deduction produce more diversity and uncertainty and thus require more supervisory information.
Therefore, such vertices need expanded connection fields during the graph reconstruction to aggregate more information via extensive edges.
To achieve this, we propose an analog sampling-based graph reconstruction method. 
The idea is to explore the node pairs at the leaf node level (the lowest semantic level) and stochastically generate an edge for a given pair of nodes with a certain probability associated with the deduction structural entropy. 

Specifically, for a given $\mathcal{T}$, assume the node $\delta$ has a set of child nodes $\{ \delta^{\left \langle 1 \right \rangle},\delta^{\left \langle 2 \right \rangle},\dots,\delta^{\left \langle n \right \rangle}\} $. The probability of the child $\delta^{\left \langle i \right \rangle}$ is defined as: 
$P(\delta^{\left \langle i \right \rangle}) = \sigma_\delta(H^{\mathcal{T}}(G_{f};(\lambda,\delta^{\left \langle i \right \rangle}]))$,
% \begin{equation}\label{eq:prob}
% P(\delta^{\left \langle i \right \rangle}) = \sigma_\delta(H^{\mathcal{T}}(G_{f};(\lambda,\delta^{\left \langle i \right \rangle}])),
% \end{equation}
where $\lambda$ is the root of $\mathcal{T}$ and $\sigma_\delta(\cdot)$ represents a distribution function. Take $\mathrm{softmax}$ function as an example, the probability of $\delta^{\left \langle i \right \rangle}$ can be calculated as: 
\begin{equation}\label{eq:prob-softmax}
% \vspace{-0.4em}
P(\delta^{\left \langle i \right \rangle}) = \frac{\mathrm{exp}(H^{\mathcal{T}}(G_{f};(\lambda,\delta^{\left \langle i \right \rangle}]))}
{ {\textstyle \sum_{j=1}^{n}}{\mathrm{exp}(H^{\mathcal{T}}(G_{f};(\lambda,\delta^{\left \langle j \right \rangle}]))} }.
\end{equation}

The probability of a non-root node can be acquired recursively. To generate new edges, we sample leaf node pairs by a top-down approach with a single sampling flow as follows: 

\noindent \textbf{(1)} For the encoding tree (or subtree) with root node $\delta$, two different child nodes $\delta^{\left \langle i \right \rangle}$ and $\delta^{\left \langle j \right \rangle}$ are selected by sampling according to $P(\delta^{\left \langle i \right \rangle})$ and $P(\delta^{\left \langle j \right \rangle})$. Let $\delta_1 \gets \delta^{\left \langle i \right \rangle}$ and $\delta_2 \gets \delta^{\left \langle j \right \rangle}$
\textbf{(2)} If $\delta_1$ is a non-leaf node, we perform sampling once on the subtree rooted at $\delta_1$ to get $\delta_1^{\left \langle i \right \rangle}$, then update $\delta_1 \gets \delta_1^{\left \langle i \right \rangle}$. The same is operated on $\delta_2$.
\textbf{(3)} After recursively performing step (2), we sample two leaf nodes $\delta_1$ and $\delta_2$, while adding the edge connecting vertex $v_1 = T_{\delta_1}$ and $v2 = T_{\delta_2}$ into the edge set $E'$ of graph $G'$. 
To establish extensive connections at all levels, multiple samplings are performed on all encoding subtrees. 
For each subtree rooted at $\delta$, we conduct independent samplings for $\theta \times n$ times, where $n$ is the number of $\delta$'s children, and $\theta$ is a hyperparameter that positively correlated with the density of reconstructed graph. 
For simplicity, we adopt a uniform $\theta$ for all subtrees. 
% If it is necessary to precisely control the sparsity within communities and the connectivity between communities, 
Separately setting and tuning $\theta$ of each semantic level for precise control is also feasible.

\subsection{Time Complexity of \ \framework{}}\label{timecomplexity}
The overall time complexity is $O(n^2+n+n\log ^2n)$, in which $n$ is the number of nodes. 
Separately, in \S ~\ref{step1}, the time complexity of calculating similarity matrix is $O(n^2)$ and of $k$-selector is $O(n)$. 
According to ~\cite{li2016structural}, the optimization of a high-dimensional encoding tree in \S ~\ref{step2} costs the time complexity of $O(n\log ^2n)$.
As for the sampling process in \S ~\ref{step3}, the time complexity can be proved as $O(2n)$.
We report the time cost of \ \framework{} in Appendix~\ref{appendix:baseline}.
%% 3-purposed method
%%%%%%%%%%%%%%%%%%%%%%%%%%%%%%%%%%%%%%%%%%%%%%%%%

\begin{table*}[t!]
\centering
\caption{{Main Results on OV-COCO and OV-LVIS:} We evaluate box AP with IoU threshold 0.5 ($\mathrm{mAP^{50}}$) on OV-COCO, and box AP ($\mathrm{mAP^{box}}$) and mask AP ($\mathrm{mAP^{mask}}$) on OV-LVIS. Note that $\mathrm{mAP_{novel}}$ and $\mathrm{mAP}$ indicate the performance of zero-shot and the entire of categories, respectively. Lastly, latency implies inference time per image in seconds for OV-LVIS.}
 \vspace*{-0.25cm}
\label{tab:main}
\resizebox{1.0\linewidth}{!}{%
\begin{tabular}{@{}lrrrrrrrrrrr@{}}
\toprule 
& & & & \multicolumn{3}{c}{OV-COCO}  & \multicolumn{5}{c}{OV-LVIS}  \\
\cmidrule(lr){5-7} \cmidrule(lr){8-12}  
{Methods}& Backbone & CLIP & Res. & $\mathrm{mAP^{50}_{novel}}$ & $\mathrm{mAP^{50}_{base}}$ & $\mathrm{mAP^{50}}$ & $\mathrm{mAP^{box}_{novel}}$ & $\mathrm{mAP^{box}}$ & $\mathrm{mAP^{mask}_{novel}}$ & $\mathrm{mAP^{mask}}$ & Latency\,(s)\\ 
\midrule
\multicolumn{1}{r}{\normalsize {\sffamily DETR-based}} \vspace*{0.1cm} \\
% OWL-ViT & ViT-L/14 & ViT-L/14 & 25.6 & 34.7 & - & - \\ 
OV-DETR~\cite{zang2022open} & RN50 & ViT-B/32 & 1333  & 29.4 & 61.0 & 52.7 & 18.0 & 27.4 & 17.4 & 26.6 & 12.28 \\
\textbf{Prompt-OVD} & ViT-B/16 & ViT-L/14 & 840 & \textbf{30.6} & \textbf{63.5} & \textbf{54.9} & \textbf{29.4} & \textbf{33.0} & \textbf{23.1} & 24.2 & 0.58 \\
\cmidrule(lr){1-12}
% & %(memory update / robust learning) &
\multicolumn{1}{r}{\normalsize {\sffamily RCNN-based}}&  \multicolumn{4}{r}{\normalsize \!\!\!\!\!\!\!\!{\sffamily(Latency Range: 0.40 -- 0.70 seconds)}}\vspace*{0.1cm} \\ 
% ViLD-text & 10.1 & 24.9 & 5.9 & 49.3 \\
Detic~\cite{zhou2022detic}     & RN50 & ViT-B/32 & 1333 & 27.8 & 51.0 & 45.0 & 23.6 & 30.4 & 21.4 & \textbf{26.9} & 0.47 \\
ViLD~\cite{guopen2022vild}      & RN50 & ViT-B/32 & 1333 & 27.6 & 59.6 & 51.3& 16.7 & 27.8 & 16.6 & 25.5 & 0.48  \\
F-VLM~\cite{kuoopen2023fvlm}     & RN50 & RN50 & 1024 & 28.0 & 43.7 & 39.6& 20.3 & 27.8 & 18.6 & 24.2 & 0.50 \\
DetPro~\cite{du2022detpro}    & RN50 & ViT-B/32 & 1333  & - & -& - & 20.8 & 28.4 & 19.8 & 25.9 & 0.67\\
% F-VLM~\cite{kuoopen2023fvlm}    & RN50x4 & RN50x4 & 1024 & - & - & 26.3 & 28.5 & 0.72\\ 
% \cmidrule(lr){1-9} 
\bottomrule
\end{tabular}%
}
\vspace*{-0.1cm}
\end{table*}

\begin{table*}[t!]
% \begin{wraptable*}{r}{3cm}

\parbox{0.3\linewidth}{
\centering
\caption{Latency change by modifying OV-DETR to \algname{}.}
\vspace*{-0.25cm}
\label{tab:inference_study}
\resizebox{1.0\linewidth}{!}{
\begin{tabular}{@{}llr@{}}
\toprule 
& {Modification}& Latency (s)\\ 
\midrule
 & OV-DETR & 12.28\\
% \cmidrule{1-3}
\,\,\,(1)\!\! & ResNet $\xrightarrow{}$ ViT & 12.36\\ 
\,\,\,(2)\!\! & ViT $\xrightarrow{}$ ViTDet & 8.75\\ 
\,\,\,(3)\!\! & Prompt-based Decoding & 2.89\\
\,\,\,(4)\!\! & Ensemble with CLIP & 3.03\\
\,\,\,(5)\!\! & RoI Pruning ($\epsilon=0.3)$ & 0.58\\
\bottomrule
\label{table:modification}
\vspace*{-0.4cm}
\end{tabular}%
}}
{\color{white} \,}
\hfill
\parbox{0.33\linewidth}{
\centering
\caption{Performance with varying $\alpha$ when fixing $\beta=0.4$.}
\vspace*{-0.3cm}
\label{tab:alpha}
\resizebox{1.0\linewidth}{!}{%
\begin{tabular}{@{}lrrrr@{}}
\toprule 
{$\alpha$}& $\mathrm{mAP^{box}_{novel}}$ & $\mathrm{mAP^{box}}$ & $\mathrm{mAP^{mask}_{novel}}$ & $\mathrm{mAP^{mask}}$ \\ 
\midrule
% & %(memory update / robust learning) &
% \multicolumn{1}{r}{\footnotesize {\sffamily RCNN-based}} \\
% ViLD-text & 10.1 & 24.9 & 5.9 & 49.3 \\
0.0 & 28.1 & 30.8 & 21.9 & 22.4 \\
0.1 & 28.7 & 32.3 & 22.6 & 23.6\\
% \rowcolor{LightCyan}
\textbf{0.2} & \textbf{29.4} & \textbf{33.0} & \textbf{23.1} & \textbf{24.2}  \\
0.3 & 29.5 & 32.2 & 23.2 & 23.7 \\
0.4 & 29.7 & 30.6 & 23.3 & 22.5 \\
0.5 & 29.9 & 28.8 & 23.5 & 21.2\\
1.0 & 30.5 & 14.5 & 24.0 & 10.8 \\
\bottomrule
\label{table:alpha_search}
\vspace*{-0.4cm}
\end{tabular}%
}}
{\color{white} \,}
\hfill
\parbox{0.33\linewidth}{
\centering
\caption{Performance with varying $\beta$ when fixing $\alpha=0.2$.}
\vspace*{-0.3cm}
\label{tab:beta}
\resizebox{1.0\linewidth}{!}{
\begin{tabular}{@{}lrrrr@{}}
\toprule 
{$\beta$}& $\mathrm{mAP^{box}_{novel}}$ & $\mathrm{mAP^{box}}$ & $\mathrm{mAP^{mask}_{novel}}$ & $\mathrm{mAP^{mask}}$ \\ 
\midrule
% & %(memory update / robust learning) &
% \multicolumn{1}{r}{\footnotesize {\sffamily RCNN-based}} \\
% ViLD-text & 10.1 & 24.9 & 5.9 & 49.3 \\
0.0 & 15.9 & 30.0 & 12.1 & 21.7\\
0.1 & 22.8 & 31.4 & 17.9 & 22.9 \\
0.2 & 29.0 & 32.7 & 22.5 & 23.9 \\
0.3 & 29.3 & 33.0 & 23.0 & 24.1 \\
\textbf{0.4} & \textbf{29.3} & \textbf{33.0} & \textbf{23.1} & \textbf{24.2} \\
0.5 & 28.6 & 33.0 & 22.4 & 24.1 \\
1.0 & 19.6 & 31.5 & 15.3 & 22.9 \\
\bottomrule
\label{table:beta_search}
\vspace*{-0.4cm}
\end{tabular}%
}
}


\vspace*{-0.35cm}
\end{table*}

\section{Evaluation} 

\noindent\textbf{Datasets.} We evaluate our approach on two popularly used benchmark datasets, namely OV-COCO and OV-LVIS, each of which is modified from MS-COCO~\cite{lin2014microsoft} and LVIS\,(v1)~\cite{gupta2019lvis}; OV-COCO has 121K images with 64 classes while OV-LVIS has 100K images with 1,203 classes.
Following OV-DETR~\cite{zang2022open}, COCO is split into 17 novel classes and 48 base classes. LVIS is split into three categories: 337 novel classes, and 866 common or base classes based on the number of training images. Note that we refer to the two datasets as OV-COCO and OV-LVIS, respectively, and only base classes are used for training. 


\smallskip\smallskip
\noindent\textbf{Algorithms.} We compare \algname{} with an end-to-end OVD detection model named OV-DETR\,\cite{zang2022open} (baseline) and four two-stage OVD models. However, it should be noted that the two-stage OVD models are based on Mask-RCNN or allow the use of external large-scale data, which makes a fair comparison with the end-to-end Transformer-based detectors difficult. Therefore, to ensure a fair comparison, we follow two criteria: (1) the results should be obtained by only using base categories in training, i.e., the restricted OVD setup and (2) the models' inference speed should be in the range of 0.4~--~0.7 seconds/image, which is similar to that of our proposed framework. 

%To validate the effectiveness of our method, we chose OV-DETR and four Mask-RCNN-based methods as baselines. Since Mask-RCNN-based approaches have a completely different model architecture from ours, we chose these methods as baselines based on two criteria for a fair comparison: (i) using only the dataset for base categories during training (not allowing to use the additional dataset), and (ii) having inference times between 0.4 and 0.7 seconds, which are similar to Prompt-OVD.

\smallskip\smallskip
\noindent\textbf{Implementation.} The proposed \algname{} builds upon Deformable DETR\,\cite{zhu2020deformable}, similar to OV-DETR. However, we merge the independent ResNet backbone and Transformer encoder into a single ViT encoder using ViTDet~\cite{li2022exploring}. As a result, our architecture is a purely Transformer encoder-decoder structure, following recent fully Transformer detection pipeline\,\cite{litransformer, songvidt}. 

For training, we initialize the backbone weights with a plain ViT backbone that has been pre-trained as Masked Autoencoders on ImageNet-1K. The entire model is then trained end-to-end for 50 epochs with a batch size of 32, a weight decay of 1e-4, and an AdamW optimizer. We set the initial learning rate to 2e-4 while using an image size of 840$\times$840. We implement and test all algorithms using PyTorch on eight NVIDIA V100 GPUs. 

For inference, there are three hyperparameters: the weights $\alpha$ and $\beta$ for ensembling in Eq.\,\eqref{eq:ensemble}; the threshold $\epsilon$ for RoI pruning in Eq.\,\eqref{eq:pruning}. The former weights are set to be $(0.2, 0.35)$ and $(0.2, 0.4)$ for OV-COCO and OV-LVIS, while the latter pruning threshold is set to 0.125 and 0.3 for OV-COCO and OV-LVIS. As for the ensemble for classification, we leverage the CLIP model that uses ViT-L/14 as its image encoder with a image size of 336$\times$336, which adds very little computational overhead with our RoI-based masked attention and RoI pruning. The detailed analysis of the hyperparameters and additional overhead due to using CLIP is provided in Section \ref{sec:abl_study}. 

In addition, we need to incorporate the mask head into our model for the evaluation on OV-LVIS, as RPN-based two-stage methods have reported both box and mask APs. Following the recent literature\,\cite{dong2021solq, song2022extendable}, we extend our DETR-based detector using SOLQ\,\cite{dong2021solq}, which can perform a joint training of object detection and instance segmentation by simply adding a unified query representation module. The mask vector size is set to be 1,024 while keeping remaining hyperparameters to be the same as SOLQ.

\smallskip\smallskip
\noindent\textbf{Evaluation Metrics.} We evaluate the detection accuracy of our method following exactly the same metrics used in prior OVD studies\,\cite{guopen2022vild, zang2022open, zhong2022regionclip, minderer2022owlvit}. Specifically, for OV-COCO, we use $\mathrm{mAP^{50}}$ which is a measure of the box average precision\,(AP) with an IoU threshold of 0.5. On the other hand, for OV-LVIS, we use both box mAP ($\mathrm{mAP^{box}}$) and mask mAP ($\mathrm{mAP^{mask}}$) obtained by the joint learning of object detection and instance segmentation, respectively. 

Inference time is also a crucial metric for practical applications. To compare the efficiency of different models, we compute the inference time of all methods using the same hardware environment, consisting of a single NVIDIA V100 GPU and six Intel(R) Xeon(R) Gold 5120 CPUs. To ensure an accurate inference time measurement, we calculate the average time of 100 iterations after five initial iterations, using a batch size of 1.

%\noindent\textbf{Implementation.} To enhance the performance of our model, we opt for the DETR~\cite{carion2020end} architecture based on OV-DETR~\cite{zang2022open}, and we replace the backbone and encoder with ViT-DET~\cite{li2022exploring}. The initial weights of the backbone start from the MAE pre-trained model. Our model is trained for 50 epochs, with a 32 batch size, a weight decay of 1e-4, an optimizer AdamW~\cite{loshchilov2017decoupled}, and an initial learning rate of 2e-4, while using an image size of 840x840. It is worth noting that the training process is carried out using 8 NVIDIA A100 GPUs.

%When performing inference, we use the values of ($\alpha$, $\beta) = (0.2, 0.35)$ and (0.2, 0.4) for OV-COCO and OV-LVIS, respectively, to ensemble with CLIP. In addition, we set the values of $\epsilon$ to 0.125 and 0.3 for OV-COCO and OV-LVIS, respectively, for RoI pruning. We limit the number of detections per image to a maximum of 300 and 1500, and the temperature is set to 0.065 and 0.01 for OV-COCO and OV-LVIS, respectively. Finally, we measure the inference time of all methods using the same hardware environment consisting of a NVIDIA V100 GPU and 6 Intel(R) Xeon(R) Gold 5120 CPUs. To ensure accurate inference time measurements, we calculate the average time of 100 iterations after five initial iterations using a batch size of 1.


%\smallskip\smallskip
%\noindent\textbf{Instance Segmentation.} In order to calculate the mask mAP for OV-LVIS, we need to incorporate the mask head into our model. Since the mask head in DETR is a FPN-style network, and our backbone cannot combine with it, we adopt the SOLQ method~\cite{dong2021solq}, which segments objects by jointly learning the unified query representation for three tasks (classification, localization, and segmentation). We utilize the 1024 dimension of the mask vector, while keeping all other parameters the same as SOLQ.

%\smallskip\smallskip
%\noindent\textbf{Evaluation Metrics.} We follow the same metrics as earlier open vocabulary studies. Specifically, for OV-COCO, they use a metric called $\mathrm{mAP^{50}}$ which measures the box AP with an IoU threshold of 0.5. For OV-LVIS, the metrics use both mask mAP ($\mathrm{mAP^{mask}}$) and box mAP ($\mathrm{mAP^{box}}$). 


\subsection{Main Experiment}

We present a comprehensive comparison of \algname{} with other five OVD methods in terms of detection accuracy and speed. To ensure a fair comparison, we only include the results of RCNN-based methods that can operate at a similar inference speed as ours. Table \ref{tab:main} summarizes the results of \algname{} and other five OVD methods.

In general, \algname{} outperforms the previous end-to-end OVD method, OV-DETR, on both datasets. Notably, \algname{}'s inference speed is {$21.2$ times} faster than OV-DETR, thanks to its prompt-based decoding approach. Refer to Section \ref{sec:inference_speed} for an in-depth comparison of efficiency with OV-DETR. %
%
Moreover, Prompt-OVD exhibits superior performance in terms of box mAP, even compared to RCNN-based OVD methods. These results support the effectiveness of our design that utilizes ViT-based CLIP with RoI-based mask attention and RoI pruning, improving the overall performance. Further investigation of the two techniques can be found in Section~\ref{sec:masked_attention} and ~\ref{sec:pruning}.

Although we observe a larger gap between box mAP and mask mAP (29.4 $\mathrm{mAP^{box}_{novel}} \rightarrow 23.1 \mathrm{mAP^{mask}_{novel}})$, this is a result of inheriting the limitation of SOLQ\,\cite{dong2021solq}. Specifically, the vector encoding of 2D segmentation masks using discrete cosine transformation loses object details compared to the conventional FPN-style mask head\,\cite{song2022extendable}.  
%
Furthermore, despite using a larger ViT-B/16 backbone than ResNet-50, \algname{} exhibits comparable inference time to RCNN-based methods, thanks to its simple encoding-decoding pipeline. Therefore, our results demonstrate that \algname{} shows a potential of the end-to-end Transformer-based framework for OVD. 

In Appendices B and C, we discuss potential enhancements to our method and present the results of our experiments on using image queries other than text queries for open-vocabulary object detection, respectively.


%Prompt-OVD performs better than both Mask-RCNN-based and DETR-based models for all metrics on OV-LVIS. Although Prompt-OVD significantly outperforms the other models in terms of box mAP, F-VLM and Detic achieve the highest scores for mask mAP in novel and entire categories, respectively. We believe that SOLQ has limitations in instance segmentation because it only uses 1024 mask vectors that differ from the FPN-style mask head. Consequently, Prompt-OVD has a larger gap between box mAP and mask mAP than models that utilize the FPN-style mask head.

%Also, Table~\ref{tab:main} shows that Prompt-OVD outperforms all other baselines on OV-COCO. It is noteworthy that Prompt-OVD surpasses the others in terms of box $\mathrm{mAP^{50}}$ for both novel and overall categories. We believe that the RoI pruning and score fusion with CLIP may be crucial in improving the overall performance, and we investigate the effects of these techniques in the Section~\ref{sec:abl_study}.


%\smallskip\smallskip
%\noindent\textbf{Inference Time.} Table~\ref{tab:main} also shows the inference time per image on OV-LVIS. Surprisingly, despite using a larger backbone compared to Mask-RCNN-based baselines, there is little difference in inference time. Furthermore, Prompt-OVD reduces the inference time for OV-DETR, which is the method based on DETR, by up to $\mathsf{24x}$.


% \begin{table*}[t!]
% \centering
% \caption{Main Results on OV-LVIS}
% \vspace*{-0.2cm}
% \label{tab:main_lvis}
% \resizebox{1.0\linewidth}{!}{%
% \begin{tabular}{@{}lrrrrrrrr@{}}
% \toprule 
% % & & & \multicolumn{2}{c}{OV-LVIS} & \multicolumn{2}{c}{OV-COCO} \\
% % \cmidrule(lr){4-5} \cmidrule(lr){6-7}  
% {Methods}& Backbone & CLIP & Res. & $\mathrm{mAP^{box}_{novel}}$ & $\mathrm{mAP^{box}}$ & $\mathrm{mAP^{mask}_{novel}}$ & $\mathrm{mAP^{mask}}$ & s/img\\ 
% \midrule
% \multicolumn{1}{r}{\footnotesize {\sffamily DETR-based}} \\
% % OWL-ViT & ViT-L/14 & ViT-L/14 & 25.6 & 34.7 & - & - \\ 
% OV-DETR~\cite{zang2022open} & RN50 & ViT-B/32 & 1024 & 18.0 & 27.4 & 17.4 & 26.6 & 12.28\\
% \textbf{Prompt-OVD} & ViT-B/16 & ViT-L/14 & 840 & \textbf{29.4} & \textbf{33.0} & \textbf{23.1} & 24.2 & 0.58\\
% \cmidrule(lr){1-9}

% % & %(memory update / robust learning) &
% \multicolumn{1}{r}{\footnotesize {\sffamily RCNN-based}} \\
% % ViLD-text & 10.1 & 24.9 & 5.9 & 49.3 \\
% Detic~\cite{zhou2022detic}     & CenterNet2 & ViT-B/32 & 1333 & 23.6 & 30.4 & 21.4 & \textbf{26.9} & 0.47 \\
% ViLD~\cite{guopen2022vild}      & RN50 & ViT-B/32 & 1333 & 16.7 & 27.8 & 16.6 & 25.5 & 0.48 \\
% F-VLM~\cite{kuoopen2023fvlm}     & RN50 & RN50 & 1024 & 20.3 & 27.8 & 18.6 & 24.2 & 0.50 \\

% DetPro~\cite{du2022detpro}    & RN50 & ViT-B/32 & 1333 & 20.8 & 28.4 & 19.8 & 25.9 & 0.67 \\

% % F-VLM~\cite{kuoopen2023fvlm}    & RN50x4 & RN50x4 & 1024 & - & - & 26.3 & 28.5 & 0.72\\ 


% % \cmidrule(lr){1-9} 

% \bottomrule
% \end{tabular}%
% }
% % \vspace*{-0.5em}
% \end{table*}



% \begin{table}[t!]
% \centering
% \caption{Main Results on OV-COCO}
% \vspace*{-0.2cm}
% \label{tab:main_coco}
% \resizebox{0.9\linewidth}{!}{%
% \begin{tabular}{@{}lrrr@{}}
% \toprule 
% % & & & \multicolumn{2}{c}{OV-LVIS} & \multicolumn{2}{c}{OV-COCO} \\
% % \cmidrule(lr){4-5} \cmidrule(lr){6-7}  
% {Methods}& $\mathrm{mAP^{novel}_{50}}$ & $\mathrm{mAP^{base}_{50}}$ & $\mathrm{mAP_{50}}$ \\ 
% \midrule
% % & %(memory update / robust learning) &
% \multicolumn{1}{r}{\footnotesize {\sffamily DETR-based}} \\
% % OWL-ViT & ViT-L/14 & ViT-L/14 & 25.6 & 34.7 & - & - \\ 
% OV-DETR~\cite{zang2022open} & 29.4 & 61.0 & 52.7 \\
% \textbf{Prompt-OVD} & \textbf{30.6}& \textbf{63.5} &\textbf{54.9} \\
% \cmidrule(lr){1-4}

% \multicolumn{1}{r}{\footnotesize {\sffamily RCNN-based}} \\
% % ViLD-text & 10.1 & 24.9 & 5.9 & 49.3 \\
% Detic~\cite{zhou2022detic}      &  27.8 & 51.0 & 45.0 \\
% ViLD~\cite{guopen2022vild}      &  27.6 & 59.6 & 51.3 \\
% F-VLM~\cite{kuoopen2023fvlm}     &  28.0 & 43.7 & 39.6 \\

% \bottomrule
% \end{tabular}%
% }
% \end{table}
\begin{figure*}[t]
\begin{center}
\includegraphics[width=16.7cm]{figures/ablation_study.pdf}
\end{center}
\vspace*{-0.5cm}
\begin{subfigure}{0.3\textwidth}
\label{fig:gt}
\caption{GT}
\end{subfigure}
\begin{subfigure}{0.15\textwidth}
\label{fig:rpn}
\caption{RPN}
\end{subfigure}
\begin{subfigure}{0.31\textwidth}
\caption{No pruning ($\epsilon = 0$)}
\label{fig:noprune}
\end{subfigure}
\begin{subfigure}{0.16\textwidth}
\caption{Pruning ($\epsilon =0.3$)}
\label{fig:prune}
\end{subfigure}
\vspace*{-0.4em}
\caption{Box predictions: (a) ground-truth boxes, (b) boxes estimated by RPN from DetPro~\cite{du2022detpro}, (c)--(d) boxes estimated by \algname{} without and with RoI pruning. Due to numerous predicted boxes in (b) and (c), we limit the number of boxes to 40 for better visualization. Red and green boxes are the ground-truth of novel and base classes, while blue and white ones represent predicted boxes that are either in close proximity or not in close proximity to the ground-truth, respectively.}
\label{fig:pruning_abl}
\vspace*{-0.4cm}
\end{figure*}


\subsection{Main Ablation Study}
\label{sec:abl_study}

\subsubsection{Inference Speed Up from OV-DETR} 
\label{sec:inference_speed} 
Table \ref{table:modification} summarizes the change in inference speed when replacing each design component of OV-DETR with our proposed ones on OV-LVIS. (1) Despite having more parameters, using ViT-B/16 incurs little additional latency, as it rather reduces the computational burden for the multi-scale deformable attention of the DETR encoder. (2) The use of local attention and the replacement of the DETR encoder with a simple feature pyramid network, as suggested by \cite{li2022exploring, song2022extendable}, result in a meaningful reduction in latency. (3) The primary speedup comes from replacing the decoding of OV-DETR with prompt-based decoding. (4) The ensemble with ViT-based CLIP only adds very little latency thanks to our efficient RoI-based masked attention. (6) RoI pruning also significantly contributes to speeding up by reducing the number of RoI candidates for detection and segmentation, without sacrificing detection accuracy. Overall, \algname{} speeds up inference by $21.2$ times over OV-DETR.

\begin{table}[t!]
\centering

\caption{Performance between RoI Align and RoI-based Masked Attention (RMA) on OV-LVIS.}% We set $\epsilon$ as 0.3 while pruning.} % If RoI pruning is employed, $\epsilon$ is assigned a value of 0.3; otherwise, it should be set to 0.0.
\vspace*{-0.3cm}
\label{tab:roi_pool}
\resizebox{1.0\linewidth}{!}{%
\begin{tabular}{@{}lrrrrrr@{}}
\toprule 
{RoI Proc.}& $\mathrm{mAP^{box}_{novel}}$ & $\mathrm{mAP^{box}}$ & $\mathrm{mAP^{mask}_{novel}}$ & $\mathrm{mAP^{mask}}$ & Latency\\ 
\midrule
% & %(memory update / robust learning) &
% \multicolumn{1}{r}{\footnotesize {\sffamily RCNN-based}} \\
% ViLD-text & 10.1 & 24.9 & 5.9 & 49.3 \\
Naive & 12.8 & 28.3 & 9.9 & 20.3 & 13.51 \\
\hspace{3mm}+Pruning & 13.7 & 28.3 & 10.2 & 20.4 & 1.85\\
Align~\cite{he2017mask} & 24.2 & 31.7 & 19.1 & 23.0 & 3.06\\
\hspace{3mm}+Pruning & 26.2 & 32.0 & 20.8 & 23.4 & 0.60 \\
RMA (ours) & 26.6 & 32.5 & 21.0 & 23.8 & 3.03 \\
\hspace{3mm}+Pruning & \textbf{29.4} & \textbf{33.0} & \textbf{23.1} & \textbf{24.2} & 0.58 \\


\bottomrule
\end{tabular}%
}
\vspace*{-0.3cm}
\label{tab:rma_analysis}
\end{table}

%\vspace*{-0.18cm}
\subsubsection{Ensemble Coefficient}
%\label{sec:hyperparams}
We investigate the influence of the ensembling weights $\alpha$ and $\beta$ for base and novel classes. We vary the value of each hyperparameter while keeping the other constant, as summarized in Tables \ref{tab:alpha} and \ref{tab:beta}, with a fixed RoI pruning threshold $\epsilon$ of 0.3. In general, the overall performance increases and then reaches at their maximum values when $\alpha=0.2$ and $\beta=0.4$. Using extreme values 0.0 or 1.0 for either coefficient results in significantly worse performance than using a more balanced ensemble. Moreover, the results show that the ensemble is more effective for novel classes than base classes, as evidenced by the significant impact of $\beta$ on the results of $\mathrm{mAP_{novel}}$. This suggests that the knowledge from CLIP has a greater positive impact on novel classes than on base classes. We set the values of $\alpha$ and $\beta$ to 0.2 and 0.4 for all experiments.
%

%Also, for getting optimal probability weights $\alpha$ and $\beta$, we report the performance change as $\alpha$ and $\beta$ increase in Table~\ref{tab:alpha} and ~\ref{tab:beta}, respectively. Table~\ref{tab:alpha} illustrates that while the zero-shot performance ($\mathrm{mAP_{novel}}$) increases monotonically, the overall performance ($\mathrm{mAP}$) increases and peaks at $\alpha=0.2$, then decreases. Interestingly, Table~\ref{tab:beta} demonstrates that both zero-shot and overall performances increase and reach a maximum value when $\beta=0.4$. Therefore, we anticipate that the knowledge from CLIP have a greater positive impact on new classes than the base classes, as the optimal value of $\beta$ is larger than that of $\alpha$.
%Note that we set the values of $\alpha$ and $\beta$ to 0.2 and 0.4, respectively, for all the experiments.



\subsubsection{RoI-based Masked Attention} 
\label{sec:masked_attention}
We conduct a comparison between our RoI-based masked attention method with both the naive approach and the commonly used RoI Align method, as summarized in Table \ref{tab:rma_analysis}. The naive approach implies that CLIP infers all the cropped images of RoIs according to Eq.~\eqref{eq:iter_infer}. To apply the RoI Align method to CLIP's ViT encoder, we reconstruct its patch tokens into a 2D feature map prior to the final Transformer layer. Compared to the naive approach, our RoI-based masked attention method has a significantly smaller computational overhead. Additionally, the efficiency and effectiveness of our method can be further improved by utilizing RoI pruning, which removes background RoIs. In contrast, RoI Align shows substantially lower $\mathrm{mAP^{box}_{novel}}$ and $\mathrm{mAP^{box}}$ than our method, as it is not optimized for the Transformer structure. Surprisingly, the naive crop method did not perform well, likely due to resizing a small object to be too large. Therefore, using masked attention is a more appropriate approach for Transformers than others. 

%To validate our proposed RoI-based masked attention, we compare it with RoI Align, which is utilized by Mask-RCNN in Table~\ref{tab:roi_pool}. Masked attention overcomes RoI Align for all the metrics with a large margin. Moreover, we analyze the optimal usage of the attention layer number in Table~\ref{tab:blk_num}, and using the last layer has the best results compared to the others for all the metrics. Through those emperical studies, therefore, we select RoI-based masked attention with the last attention layer instead of RoI align. 
\label{sec:eval_RMA}
\begin{table}[t!]
\centering
\caption{Performance trade-off with varying $\epsilon$ on OV-LVIS.}
\vspace*{-0.2cm}
\label{tab:roi_thres}
\resizebox{1.0\linewidth}{!}{%
\begin{tabular}{@{}lrrrrr@{}}
\toprule 
{$\epsilon$}& $\mathrm{mAP^{box}_{novel}}$ & $\mathrm{mAP^{box}}$ & $\mathrm{mAP^{mask}_{novel}}$ & $\mathrm{mAP^{mask}}$ & Latency (s)\\ 
\midrule
% & %(memory update / robust learning) &
% \multicolumn{1}{r}{\footnotesize {\sffamily RCNN-based}} \\
% ViLD-text & 10.1 & 24.9 & 5.9 & 49.3 \\
0.0 & 26.6 & 32.5 & 21.0 & 23.8 & 3.03\\
0.1 & 27.7 & 32.8 & 21.9 & 24.0 & 1.38\\
0.2 & 28.3 & 33.0 & 22.3 & 24.1 & 0.86\\
% \rowcolor{LightCyan}
\textbf{0.3} & \textbf{29.4} & \textbf{33.0} & \textbf{23.1} & \textbf{24.2} & 0.58\\
0.4 & 28.3 & 31.9 & 22.5 & 23.4 & 0.43\\
0.5 & 25.3 & 28.1 & 19.8 & 20.8 & 0.37\\
\bottomrule
\end{tabular}%
}
\vspace*{-0.2cm}
\end{table}



\subsubsection{Box Regression over RPN}
%\label{sec:eval_prune}
% tab2에서 masked attention과 align과의 비교. 
% block number 마지막으로 선택한 이유. 
%\vspace*{-0.1cm}
We validate the effectiveness of \algname{} in terms of box regression compared with the existing RPN-based method. Figure \ref{fig:pruning_abl} compares their estimated bounding boxes based on the ground-truth ones. %Specifically, Figure \ref{fig:pruning_abl}(a) shows the base and novel objects with their ground-truth bounding boxes. 
Figure \ref{fig:pruning_abl}(b) is an example of the scenario where the RPN method fails to accurately localize objects, i.e., a  missing box for carriage (novel class) and two deviated boxes from the ground-truth for the bread (novel class) and plate (base class). In contrast, \algname{} successfully localizes all base and novel objects with high recall using the prompt-guided decoding, as shown in Figure \ref{fig:pruning_abl}(c). Despite the presence of background or inaccurate boxes in the box candidates, \algname{} successfully covers all the ground-truth boxes with its predictions. Furthermore, as seen in Figure \ref{fig:pruning_abl}(d), RoI pruning effectively excludes such irrelevant boxes from the detection process.


%In order to validate the effectiveness of RoI pruning, we compare the proposals before and after pruning in Fig.~\ref{fig:pruning_abl}. Prior to pruning, there are numerous false positive boxes that did not match the ground truth, and those might negatively affect the performance due to score fusion with CLIP. However, pruning significantly reduces the number of non-matching boxes. In addition, since object detection consumes their time for handling lots of proposals, RoI pruning improves both performance and inference speed. We believe that RoI pruning narrow the gap in inference time between our method and the baseline despite using a larger backbone. 
%\vspace*{-0.3cm}
\subsubsection{RoI Pruning Threshold}
\label{sec:pruning}
%\vspace*{-0.1cm}
We investigate the trade-off between detection performance and computational efficiency by varying the pruning threshold $\epsilon$, as summarized in Table \ref{tab:roi_thres}. As the threshold increases, fewer bounding boxes are retained, as the number of boxes with object scores greater than the threshold decreases. For instance, when the threshold is set to be 0.0, which means RoI pruning is not applied, the detection accuracy deteriorates due to the inclusion of background boxes, also resulting in a high latency. On the contrary, the detection accuracy improves as $\epsilon$ increases within the reasonable range of 0.0~--~0.3, but deteriorates with a larger threshold of 0.4~--~0.5. That is, as $\epsilon$ increases, more false positive boxes begins to be excluded, leading to improved performance. However, further increasing the threshold leads to the removal of true positive boxes, causing performance degradation. Therefore, we set the value of $\epsilon$ to 0.3 for all experiments.

%, we analyze the trend of performance and inference time as $\epsilon$ increases and report the results in Table~\ref{tab:roi_thres}. It shows that latency decreases as $\epsilon$ increases, since the number of boxes of which score is bigger than $\epsilon$ decreases. Especially, the inference time is drastically higher when not using RoI prunning ($\epsilon=0.0$). On the contrary, the performance increases as $\epsilon \in [0.0, 0.3]$, and decreases as $\epsilon \in [0.3, 0.5]$. We expect that, as $\epsilon$ increases, it drops more false positive boxes to improve the performance at $\epsilon \in [0.0, 0.3]$, then it causes degradation of performance by dropping true positive boxes at $\epsilon \in [0.3, 0.5]$. Hence, we select $\epsilon$ as 0.3, which has the best performance with valid inference time. 


\begin{table}[t!]
\centering
\caption{Performance when using different CLIP models for ensembling on OV-LVIS.}
\vspace*{-0.25cm}
\label{tab:clip_arch}
\resizebox{1.0\linewidth}{!}{%
\begin{tabular}{@{}lrrrrr@{}}
\toprule 
{CLIP}& $\mathrm{mAP^{box}_{novel}}$ & $\mathrm{mAP^{box}}$ & $\mathrm{mAP^{mask}_{novel}}$ & $\mathrm{mAP^{mask}}$ & Latency (s)\\ 
\midrule
% & %(memory update / robust learning) &
% \multicolumn{1}{r}{\footnotesize {\sffamily RCNN-based}} \\
% ViLD-text & 10.1 & 24.9 & 5.9 & 49.3 \\
None     & 15.4 & 28.1 & 11.7 & 20.3 & 0.54 \\
ViT-B/32 & 20.5 & 30.1 & 16.4 & 21.9 & 0.56 \\
ViT-B/16 & 22.3 & 31.1 & 17.5 & 22.6 & 0.57\\
\textbf{ViT-L/14} & \textbf{29.4} & \textbf{33.0} & \textbf{23.1} & \textbf{24.2} & 0.58 \\


\bottomrule
\end{tabular}%
}
\label{tab:clip_size}
\vspace{-1.2em}
\end{table}


\smallskip\smallskip
\subsection{Additional Design Choice}
%\vspace*{-0.1cm}
\noindent We explore two supplementary design choices to utilize the ViT-based CLIP model in a manner that achieves the best balance between OVD performance and inference speed. %Following this, we further examine the impact of utilizing different initial weights for the backbone. %on the overall performance of \algname{}.
We provide more supplementary analysis on applying RoI-based masked attention to different attention layers and using different pre-trained ViT backbones with \algname{} in Appendix D.

%\vspace*{-0.3cm}
\subsubsection{CLIP Model Size} 
%\vspace*{-0.1cm}
% We conduct experiments to examine the impact of classification ensemble using CLIP with respect to its model size. 
Table \ref{tab:clip_size} summarizes the performance obtained after the ensemble with three different sizes of ViT-based CLIP models, including the scenario where CLIP is not used at all. We observe that without using the ensemble technique, the zero-shot performance is very poor compared to when CLIP is employed. However, as the size of the CLIP model increases, both zero-shot and overall performance improve, albeit with slightly higher inference speed. The difference in inference time is negligible due to our proposed efficient techniques; RoI-based masked attention and RoI pruning. Therefore, we conclude that the benefits obtained from using a larger CLIP encoder with ViT-L/14 outweigh the minimal increase in inference time.

%Since we expect that the zero-shot results ($\mathrm{mAP_{novel}}$) could be influenced by the utilization of the CLIP model for ensembling, we report the performance on OV-LVIS in Table~\ref{tab:clip_arch} as the CLIP model is varied. Specifically, the zero-shot performance is noticeably worse when not utilizing CLIP for ensembling compared to when CLIP is used. As the size of the CLIP model increases, both zero-shot and overall performance increase, and the inference time also increases slightly. However, the difference in inference time is negligible in comparison to the performance benefits gained. As such, we can conclude that boxes from DETR architecture contains not only base classes but also novel classes, and CLIP can more contribute the classification for unseen classes as the size increases without increasing latency much.

%\vspace*{-0.3cm}
\subsubsection{CLIP Input Resolution} 
%$\vspace*{-0.1cm}
Another design consideration is the resolution of input to the ViT-based CLIP. To evaluate the performance trade-off between detection accuracy and latency of CLIP, we vary the input image size and report the results in Table \ref{tab:image_size}. Similar to the model size, the latency change by input resolution is negligible thanks to our efficient methodological design. This indicates that we can use a variety of image resolutions without sacrificing the latency of \algname{}. However, the best image resolution for the ViT-L/14 encoder is $336 \times 336$, which is the original input size used to train the CLIP model, while the detection accuracy for base and novel classes drops with a larger $672\times672$. The $336 \times 336$  provides a good balance between detection accuracy and inference time, so it is the recommended input resolution. %for the proposed \algname{} framework. 

%Here, we evaluate the performance trade-off between detection accuracy and latency with of CLIP by varying the input image size, and report the results in Table~\ref{tab:clip_img_res}. Although there is a significant difference in performance, the latency is negligible when using image resolutions of 168 and 336. However, when compared to image resolutions of 336 and 672, the performance difference is minimal, but there is a significant gap in inference time. As a result, we choose an image resolution of 336 when feeding images into the CLIP model. 



%e analyze the performance is affected by using different initial weights for the backbone, specifically ImageNet and MAE. Table~\ref{tab:pretrained_model} displays the performance results when training the backbone using these two pretrained models. As a result, starting from the MAE pretrained model yields better performance than starting from the ImageNet pretrained model. Therefore, we use the trained model that starts from the MAE pretrained model for all experiments.







\begin{table}[t!]
\centering
\caption{Performance when using different input image resolution for ViT-based CLIP on OV-LVIS.}
\vspace*{-0.25cm}
\label{tab:clip_img_res}
\resizebox{1.0\linewidth}{!}{%
\begin{tabular}{@{}lrrrrr@{}}
\toprule 
{img. res.}& $\mathrm{mAP^{box}_{novel}}$ & $\mathrm{mAP^{box}}$ & $\mathrm{mAP^{mask}_{novel}}$ & $\mathrm{mAP^{mask}}$ & Latency \\ 
\midrule
% & %(memory update / robust learning) &
% \multicolumn{1}{r}{\footnotesize {\sffamily RCNN-based}} \\
% ViLD-text & 10.1 & 24.9 & 5.9 & 49.3 \\

168$\times$168 & 28.5 & 30.1 & 20.9 & 22.0 & 0.57 \\ 
\textbf{336$\times$336} & \textbf{29.4} & \textbf{33.0} & \textbf{23.1} & \textbf{24.2} & 0.58 \\ 
672$\times$672 & 29.2 & 33.0 & 23.0 & 24.2 & 0.83 \\


\bottomrule
\end{tabular}%
}
\vspace*{-0.35cm}
\label{tab:image_size}
\end{table}


% \begin{table}[t!]
% \centering
% \caption{Latency change as modifying OV-DETR to Prompt-OVD on OV-LVIS.}
% \label{tab:inference_study}
% \resizebox{0.7\linewidth}{!}{%
% \begin{tabular}{@{}llr@{}}
% \toprule 
% & {Modification}& Latency (s)\\ 
% \midrule
%  & OV-DETR & 12.28\\
% % \cmidrule{1-3}
% (1) & ResNet $\xrightarrow{}$ ViT & \\ 
% (2) & ViT $\xrightarrow{}$ ViTDET & \\ 
% (3) & Encoder $\xrightarrow{}$ FPN & \\
% (4) & prompt decode & \\
% (5) & + Ensemble with CLIP & \\


% \bottomrule
% \end{tabular}%
% }

% \end{table}



\begin{figure*}[htb]
  \centering
  \includegraphics[width=0.99\textwidth]{bars.pdf}
  \caption{Results of ~\framework{} with different encoding tree heights.}
  \label{fig:K-bar}
  \Description{Results with different encoding tree heights.}
\end{figure*}



\section{Results and Analysis}\label{sec:evalu}
In this section, we demonstrate the efficacy of \ \framework{} on semi-supervised node classification (\S ~\ref{sec:exp:overall}, followed by micro-benchmarks that investigate the detailed effect of the submodules on the overall performance and validate the robustness of ~\framework{} when tackling random perturbations (\S ~\ref{sec:exp:micro}). For better interpretation, we visualize the change of structural entropy and graph topology (\S ~\ref{sec:exp:int}).


\subsection{Node Classification}
\label{sec:exp:overall}



\subsubsection{Comparison with baselines}
We compare the node classification performance of ~\framework~ with ten baseline methods on nine benchmark datasets.
Table ~\ref{tab:performance comparison} shows the average accuracy and the standard deviation. 
Note that the results of H$_2$GCN (except PT and TW) and Geom-GCN are from the reported value in original papers ( - for not reported), while the rest are obtained based on the execution of the source code provided by their authors under our experimental settings. Our observations are three-fold: 
% \textbf{(1)} While all GNN methods can achieve satisfactory results on citation networks, graph structural learning frameworks perform significantly better than conventional GNN methods on WebKB, Wiki, and actor co-occurrence networks due to the high heterophily of these networks.
% The reason for this phenomenon is these conventional GNN models agggregate to much classification-irrelevant information from disassortative neighborhoods. 
% In contrast, graph structural learning can optimize the neighborhood topology to achieve better results.

% \textbf{(1)} While all GNN methods can achieve satisfactory results on citation networks, the specialized graph learning frameworks perform significantly better on WebKB, Wiki and actor co-occurence networks due to the heterophily challenge. 


\noindent \textbf{(1)} \framework~ achieves optimal results on 5 datasets, runner-up results on 8 datasets, and advanced results on all datasets. The accuracy can be improved up to 3.41\% on Pubmed, 3.00\% on Cora, and 2.92\% on Citeseer compared to the baselines. This indicates that our design can effectively capture the inherent and deep structure of the graph and hence the classification improvement. 


\noindent \textbf{(2)} \framework~ shows significant improvement on the datasets with heteropily graphs, e.g., up to 37.97\% and 27.13\% improvement against Wisconsin and Texas datasets, respectively. This demonstrates the importance of the graph structure enhancement that can contribute to a more informative and robust node representation.


\noindent \textbf{(3)} While all GNN methods can achieve satisfactory results on citation networks, the graph learning/high-order neighborhood awareness frameworks substantially outperform others on the WebKB datasets and the actor co-occurrence networks, which is highly disassortative. This is because these methods optimize local neighborhoods for better information aggregation. Our method is one of the top performers among them due to the explicit exploitation of the global structure information in the graph hierarchical semantics.




\subsubsection{Comparison base on different backbones}
Table~\ref{tab:backbone comparison} shows the mean classification accuracy of ~\framework{} with different backbone encoders.
Observably, ~\framework{} upon GCN and GAT overwhelmingly outperforms its backbone model, with an accuracy improvement of up to 31.04\% and 27.48\%, respectively. 
This indicates the iterative mechanism in the ~\framework{} pipeline can alternately optimize the node representation and graph structure.
We also notice that despite the lower improvement, ~\framework{} variants based on GraphSAGE and APPNP perform relatively better compared to those on GCN and GAT.
This is most likely due to the backbone model itself being more adapted to handle disassortative settings on graphs.
% We also notice that ~\framework~ based on GraphSAGE has the lowest improvement. This is most likely due to the weak adaptability of the backbone model itself to disassortative settings.

\begin{table}[t]
    \renewcommand{\arraystretch}{1.05}
    \setlength{\abovecaptionskip}{0.15cm}
    \setlength{\belowcaptionskip}{-0.25cm}
    \caption{Classification accuracy(\%) of ~\framework{} and corresponding backbones. Wisc. is short for Wisconsin.}%mean relative 
    \label{tab:backbone comparison}
    \centering
    % \scalebox{0.9}{
    % \setlength{\tabcolsep}{1mm}{
        \begin{tabular}{l|ccccc}
        \hline
        Method & Actor & TW & Texas & Wisc. & Improvement\\
        \hline
        \framework$_{GCN}$   & 35.03 & 66.88 & 75.68 & 79.61 & $\uparrow$ 5.20$\sim$31.04\\
        % Chebnet         &
        % \framework(Chebnet)
        % \midrule
        % SGC             &
        % \framework(SGC)
        % \framework
        % (SAGE)            & 36.34 & 66.92 & \textbf{81.62} & \textbf{86.27} & $\uparrow$ 0.25$\sim$3.79\\
          \framework$_{SAGE}$& 36.20 & 66.92 & \textbf{82.49} & \textbf{86.27} & $\uparrow$ 0.25$\sim$6.79\\
        \framework$_{GAT}$   & 32.46  & 63.57 & 74.59 & 78.82 & $\uparrow$ 4.69$\sim$27.48\\
        % \framework(APPNP) & \textbf{36.62} & 71.45 & 81.28 & 83.14 & 13.34\%\\
        \framework$_{APPNP}$ & \textbf{36.34} & \textbf{66.99} & 81.28 & 83.14 & $\uparrow$ 2.01$\sim$12.16\\
        \hline
        \end{tabular}
    % }
    % }
\end{table}


\begin{table}[t]
    % \renewcommand{\arraystretch}{0.75}
    \setlength{\abovecaptionskip}{0.15cm}
    \setlength{\belowcaptionskip}{-0.25cm}
    \caption{The $k$ selection for each iteration in structural optimization. Bolds represent the $k$ selection when the accuracy reaches maximum.}\label{tab:k-NN comparison}
    \centering
    % \scalebox{0.9}{
    \setlength{\tabcolsep}{2mm}{
        \begin{tabular}{l|ccccccccc}
        \hline
        Iteration & 1 & 2 & 3 & 4 & 5 & 6 & 7 & 8 & 9\\
        \hline
        {Cora}   & 22 & \bf{22} & 19 & 22 & 21 & 22 & 20 & 21 & 20\\
        {Actor}  & 23   & 15 & 15 & 15 & 14 & 15 & 14 & \bf{14} & 15\\
        {TW} & 50 & 16 & 16 & \bf{17} & 15 & 17 & 27 & 16 & 16 \\
        {Wisconsin} & 21 & 16 & \bf{11} & 16 & 14 & 13 & 16 & 13 & 11 \\
        {Texas}  & 21 & 13 & 13 & \bf{13} & 13 & 10 & 14 & 10 & 14 \\
        \hline
        \end{tabular}
    }
\end{table}

\subsection{Micro-benchmarking}
\label{sec:exp:micro}

\subsubsection{Effectiveness of~$k$-selector}
This subsection evaluate how the one-dimensional structural entropy guides the $k$-selector in \S ~\ref{step1}.
Table ~\ref{tab:k-NN comparison} showcases the selected parameter $k$ in each iteration with \framework$_{GCN}$. 
Noticeably, as the iterative optimization proceeds, the optimal parameter $k$ converges to a certain range, indicating the gradual stabilization of the graph structure and node representation. The disparity of parameter $k$ among different datasets also demonstrates the necessity of customizing $k$ in different cases rather than using $k$ as a static hyperparameter.




\subsubsection{Impact of the encoding tree's height $K$}
We evaluate all four variants of ~\framework~ on the website network datasets, and the encoding tree height $K$ involved in \S ~\ref{step2} varies from 2 to 4.
As shown in Fig. ~\ref{fig:K-bar}, there is a huge variation in the optimal tree heights among different datasets. For example, in the variants based on GAT, GCN, and APPNP, the best results can be targeted at $K=3$ in Texas and at $K=4$ in Cornell and Wisconsin. By contrast, in ~\framework$_{SAGE}$,  $K=2$ can enable the best accuracy of 86.27\%. This weak correlation between the best $K$ and the model performance is worth investigating further, which will be left as future work. 


\begin{figure}[tb]
  \centering
  \includegraphics[width=0.48\textwidth]{ptb_exp.pdf}
  \caption{Robustness of ~\framework~ against random noises.}
  \label{fig:pertubation}
  \Description{Results of perturbation experiment.}
\end{figure}


\begin{figure*}[tb]
  \centering
  \includegraphics[width=0.99\textwidth]{sedecrease.pdf}
  \caption{The normalized structural entropy changes during the training of ~\framework$_{GAT}$ with 2-dimensional structural entropy on (a) Texas, (b) Cornell, and (c) Wisconsin. The structure is iterated every 200 epochs. By comparison, (d) shows the entropy changes on Wisconsin without the graph reconstruction strategy.}
  \label{fig:sedecrease}
  \Description{Visualization of structural entropy and acc. variation.}
\end{figure*}


\begin{figure}[tb]
  \centering
  \includegraphics[width=0.49\textwidth]{topovisual.pdf}
  \caption {The visualized evolution of the graph structure on Cora (a,b,c) and Citeseer (d,e,f). The corresponding Structural Entropy (SE) is also shown.}
  \label{fig:topovisual}  
  \Description{Visualization of topology evolution.}
\end{figure}



\subsubsection{Sensitivity to perturbations}
We introduce random edge noises into Cora and Citeseer, with perturbation rates of 0.2, 0.4, 0.6, 0.8, and 1. As shown in Fig.~\ref{fig:pertubation}(a), ~\framework{} outperforms baselines in both GCN and GAT cases under most noise settings. For instance, ~\framework$_{GCN}$ achieves up to 8.09\% improvement against the native GCN when the perturbation rate is 0.8; by contrast, improvements by GCN-Jaccard and GCN-DropEdge are merely 6.99\% and 5.77\%, respectively. A similar phenomenon is observed for most cases in the Citeseer dataset (Fig.~\ref{fig:pertubation}(b)), despite an exception when compared against GCN-Jaccard. Nevertheless, our approach is still competitive and even better than GCN-Jaccard at a high perturbation rate. 

\subsection{Interpretation of Structure Evolution}
\label{sec:exp:int}



\subsubsection{Structural entropy variations analysis}
We evaluate how the structural entropy changes during the training of ~\framework$_{GAT}$ with 2-dimensional structural entropy on WebKB datasets. For comparison, we visualize the entropy changes on Wisconsin without the structure learning. In the experiment setting, both the graph structure and the encoding tree are updated once at each iteration (i.e., 200 GNN epochs), and within one iteration, the structural entropy is only affected by edge weights determined by the similarity matrix. For comparison, we normalize the structural entropy by $ \textstyle{\frac{H^{\mathcal{T}}(G)}{H^1(G)}}$.

As shown in Fig.~\ref{fig:sedecrease}(a)-(c), as the accuracy goes up, the normalized structural entropy constantly decreases during the iterative graph reconstruction, reaching the minimums of 0.7408 in Texas, 0.7245 in Cornell, and 0.7344 in Wisconsin. This means the increasing determinism of the overall graph structure and the reduced amount of information required to determine a vertex. 
Interestingly, if our graph reconstruction mechanism is disabled (as shown in Fig.~\ref{fig:sedecrease}(d)), the normalized structural entropy keeps rising from 0.7878, compared with Fig.~\ref{fig:sedecrease}(c). Accordingly, the final accuracy will even converge to 55.34\%, a much lower level. 

Such a comparison also provides a feasible explanation for the rising trend of the normalized structural entropy within every single iteration. 
This stems from the smoothing effect during the GNN training. 
As the node representation tends to be homogenized, the graph structure will be gradually smoothed, leading to a decrease in the one-dimensional structural entropy thus the normalized structural entropy increases.
% We speculate that the continuous rise in structural entropy may be a feasible explanation for the over-smoothing of graph neural networks, which requires further study.
% In summary, this experiment well explains the robustness of our framework.


\subsubsection{Visualization}
Fig.~\ref{fig:topovisual} visualizes the topology of the original Cora and Citeseer graph and of the 2nd and 5th iterations.
The vertex color indicates the class it belongs to, and the layout denotes connecting relations. Edges are hidden for clarity. As the iteration continues, much clearer clustering manifests -- few outliers and more concentrated clusters.  
Vertices with the same label are more tightly connected due to the iterative graph reconstruction scheme. This improvement hugely facilitates the interpretability of the GSL and the node representation models. 

pan2021information\section{Related Work}\label{sec:relate}

\noindent {\textbf{Graph structure learning and neighborhood optimization.}}
The performance of GNNs is heavily dependent on task-relevant links and neighborhoods.
Graph structure learning explicitly learns and adjusts the graph topology, and our \framework{} is one of them.
GDC~\cite{gasteiger_diffusion_2019} reconnects graphs through graph diffusion to aggregate from multi-hop neighborhoods.
Dropedge~\cite{rong2019dropedge}, Neuralsparse ~\cite{zheng2020robust} contribute to graph denoising via edge-dropping strategy while failing to renew overall structures.
LDS-GNN~\cite{franceschi2019learning} models edges by sampling graphs from the Bernoulli distribution. Meanwhile, we consider linking the structural entropy, which is more expressive of graph topology, to the sampling probability.
GEN~\cite{wang2021graph}, IDGL~\cite{chen2020iterative} explore the structure from the node attribute space by the $k$-NN method. Differently, instead of directly using attribute similarity, we regenerate edges from the hierarchical abstraction of graphs to avoid inappropriate metrics.
Besides adjusting the graph structure, methods to optimize aggregation are proposed with results on heterophily graphs.
MixHop~\cite{abu2019mixhop} learns the aggregation parameters for neighborhoods of different hops through a mixing network, while H$_2$GCN~\cite{zhu2020beyond} 
% iteratively concatenates the multi-hop neighborhood features for the final node embeddings.
identifies higher-order neighbor-embedding separation and intermediate representation combination, for adapting to heterophily graphs.
Geom-GCN~\cite{pei2019geom} aggregates messages over both the original graph and latent space by a designed geometric scheme.


\noindent {\textbf{Structural entropy with neural networks.}}
Structural information principles~\cite{li2016structural}, defining encoding trees and structural entropy, were first used in bioinformatic structure analysis~\cite{li2016three,li2018decoding}. 
Existing work mainly focuses on network analysis, node clustering and community detection\cite{li2017resistance,liu2019rem,pan2021information}.
As an advanced theory on graphs and hierarchical structure, structural information theory has great potential in combination with neural networks. 
SR-MARL~\cite{zeng2023effective} applies structural information principles to hierarchical role discovery in multi-agent reinforcement learning, thereby boosting agent network optimization.
SEP~\cite{wu2022structural} provides a graph pooling scheme based on optimal encoding trees to address local structure damage and suboptimal problem. It essentially uses structural entropy minimization for a multiple-layer coarsened graph.
MinGE~\cite{luo2021graph} and MEDE~\cite{yang2023wsdm} estimate the node embedding dimension of GNNs via structural entropy, which introduces both attribute entropy and structure entropy as objective.
Although these works exploit structural entropy to mine the latent settings of neural networks and GNNs, how to incorporate this theory in the optimization process is still understudied, and we are among the first attempts.
%% note 
%Although plenty of works have attributed a natural hierarchical structure to graphs, an approach to the most justified and explainable partition has been the angle of discussion. 
\section{Conclusion}
In this paper, we present the first comprehensive analysis to reveal the weaknesses of state-of-the-art Vietnamese language models. Our experiments show that while Vietnamese language models demonstrate good lexical and grammatical abilities in Vietnamese, they show inferior performances when questions require high-level semantic knowledge to successfully identify the unanswerability. This general result from our analysis shows that the inferior performances of Vietnamese language models on Machine Reading Comprehension task are mainly due to its inferior ability in grasping the big ``picture'' of the given context.

Besides, our analysis also show that Vietnamese MRC benchmarks overestimate the comprehension skills of models in some language aspects, so  state-of-the-art performances on MRC benchmarks does not accurately reflect the progress of Vietnamese Machine Reading Comprehension.



\begin{acks}
This paper was supported by the National Key R\&D Program of China through grant 2021YFB1714800, NSFC through grant 62002007, S\&T Program of Hebei through grant 20310101D, Natural Science Foundation of Beijing Municipality through grant 4222030, CCF-DiDi GAIA Collaborative Research Funds for Young Scholars, the Fundamental Research Funds for the Central Universities, Xiaomi Young Scholar Funds for Beihang University, and in part by NSF under grants III-1763325, III-1909323,  III-2106758, and SaTC-1930941. 
\end{acks}


%%
%% The next two lines define the bibliography style to be used, and
%% the bibliography file.
\bibliographystyle{ACM-Reference-Format}
\bibliography{sample-base}

%%
%% If your work has an appendix, this is the place to put it.
\clearpage
\appendix
% \newpage
% \newpage
\section{Appendix}\label{sec:append}
\subsection{Glossary of notations}
\label{appendix:notations}
In Table ~\ref{tab:notations}, we summarize the notations used in our work. 

\begin{table}[ht]
    \setlength{\abovecaptionskip}{0.25cm}
    \setlength{\belowcaptionskip}{-0.25cm}
    \caption{Glossary of Notations.}\label{tab:notations}
    % \vspace{-2.5mm}
    \centering
    \scalebox{1.0}{
        \begin{tabular}{l|l}
        \hline
        Notation & Description \\
        \hline
        $G;A;S$ & Graph; Adjacency matrix; Similarity matrix. \\
        % $A^{G}_{ij}$ & The weight of the edge between $v_i$ and $v_j$ in $G$.\\
        $v;e;x$ & Vertex; Edge; Vertex attribute. \\
        $V;E;X$ & Vertex set; Edge set; Attribute set. \\
        $|V|;|E|$ & The number of vertices and edges. \\
        $\mathcal{P};P_i$ & The partition of $V$; A community.\\
        $D;d(v_i)$ & The degree matrix; The degree of vertex $v_i$. \\
        $e_{ij}$ & The edge between $v_i$ and $v_j$. \\
        $w_{ij}$ & The weight of edge $e_{ij}$. \\
        $vol(G)$ & The volume of graph $G$, i.e., degree sum in $G$. \\
        $G^{(k)}_{knn}$ & The $k$-NN graph with parameter $k$.\\
        $G_{f}$  & Fusion graph.\\
        $G^{(k)}_{f}$ & The fusion graph with parameter $k$.\\
        \hline
        $\mathcal{T}$ & Encoding tree. \\
        $\mathcal{T}^*$ & The optimal encoding tree. \\
        % $\mathcal{T}^{(K)}$ & The encoding tree with height $K$. \\
        $\lambda$ & The root node of an encoding tree. \\
        $\alpha$ & A non-root node of an encoding tree. \\
        $\alpha^-$ & The parent node of $\alpha$. \\
        $\alpha^{\left \langle i \right \rangle}$ & the $i$-th child of $\alpha$.\\
        $T_\lambda$ & The label of $\lambda$, i.e., node set $V$. \\
        $T_\alpha$ & The label of $\alpha$, i.e., a subset of $V$.\\
        $\mathcal{V}_\alpha$ & Volume of graph $G$. \\
        $g_a$ & the sum weights of cut edge set $[T_\alpha,T_\alpha/T_\lambda]$. \\
        $N(\mathcal{T})$ & The number of non-root node in $\mathcal{T}$.\\
        \hline
        $H^\mathcal{T}(G)$ & Structural entropy of $G$ under $\mathcal{T}$.\\
        $H^K(G)$ & $K$-dimensional structural entropy.\\
        $H^1(G)$ & One-dimensional structural entropy.\\
        $H^\mathcal{T}(G;\alpha)$ & Structural entropy of node $\alpha$ in $\mathcal{T}$.\\
        $H^{\mathcal{T}}(G;(\lambda,\alpha])$ & Structural entropy of a deduction from $\lambda$ to $\alpha$.\\
        \hline
        \end{tabular}
    }
% \vspace{-6.5mm}
\end{table}

% \vspace{-0.8em}
\subsection{Dataset and Time Costs of \ \framework{}}
\label{appendix:dataset}
Our framework \framework~ is evaluated on nine graph datasets. the statistics of these datasets are shown in Table~\ref{tab:statistics}. The time costs of \ \framework{} on all datasets are shown in Table~\ref{tab:time comparasion}.
% \vspace{-0.2em}
\begin{table}[ht]
    \setlength{\abovecaptionskip}{0.25cm}
    \setlength{\belowcaptionskip}{-0.25cm}
    \caption{Statistics of benchmark datasets.}\label{tab:statistics}
    \centering
    % \vspace{-0.2em}
    \scalebox{1.0}{
        \begin{tabular}{c|ccccc}
        \hline
        Dataset & Nodes & Edges & Classes & Features & homophily\\
        \hline
        Cora & 2708 & 5429 & 7 & 1433 & 0.83\\
        Citeseer & 3327 & 4732 & 6 & 3703 & 0.71\\
        Pubmed & 19717 & 44338 & 3 & 500 & 0.79\\
        \hline
        % Chameleon & 2277 & 36101 & 5 & 2325 & 0.25\\
        % Squirrel & 5201 & 217073 & 5 & 2089 & 0.22\\
        PT & 1912 & 31299 & 2 & 3169 & 0.59\\
        TW & 2772 & 63462 & 2 & 3169 & 0.55\\
        \hline
        Actor & 7600 & 33544 & 5 & 931 & 0.24\\
        \hline
        Cornell & 183 & 295 & 5 & 1703 & 0.30\\
        Texas & 183 & 309 & 5 & 1703 & 0.11\\
        Wisconsin & 251 & 499 & 5 & 1703 & 0.21\\
        \hline
        \end{tabular}
    }
% \vspace{-4.5mm}
\end{table}

\begin{itemize}[leftmargin=*]
    \item \textbf{Citation networks}~\cite{yang2016revisiting,welling2016semi}. Cora, Citeseer, and Pubmed are benchmark datasets of citation networks. Nodes represent paper, and edges represent citation relationships in these networks. The features are bag-of-words representations of papers, and labels denote their academic fields.
    % \item \textbf{Wikipedia networks}~\cite{rozemberczki2021multi,pei2019geom}. Wikipedia dataset contain three page to page networks, Chameleon, Squirrel and Crocodile, which are originally designed for website traffic regression. In ~\cite{pei2019geom}, the author package monthly traffic into 5 categories, providing Chameleon and Squirrel for node classification task. In both networks, vertices are web pages, edges are hyper-links and features are nouns that characterize the page.
    \item \textbf{Social networks}~\cite{rozemberczki2021multi}.
    TW and PT are two subsets of Twitch Gamers dataset~\cite{rozemberczki2021twitch}, designed for binary node classification tasks, where nodes correspond to users and links to mutual friendships. 
    The features are liked games, location, and streaming habits of the users. 
    The labels denote whether a streamer uses explicit language (Taiwanese and Portuguese).
    \item \textbf{WebKB networks}~\cite{getoor2005link}. 
    Cornell, Texas, and Wisconsin are three subsets of WebKB, where nodes are web pages, and edges are hyperlinks. The features are the bag-of-words representation of pages. The labels denote categories of pages, including student, project, course, staff, and faculty.
    % The WebKB dataset consists of 877 scientific publications classified into one of five classes. The citation network consists of 1608 links. Each publication in the dataset is described by a 0/1-valued word vector indicating the absence/presence of the corresponding word from the dictionary. The dictionary consists of 1703 unique words. 
    \item \textbf{Actor co-occurrence network}~\cite{tang2009social}. This dataset is a subgraph of the film-director-actor-writer network, in which nodes represent actors, edges represent co-occurrence relation, node features are keywords of the actor, and labels are the types of actors.
    % \item \textbf{Karateclub dataset}
\end{itemize}
% \vspace{-0.5em}

% \vspace{-1.0em}
\subsection{Baselines}
\label{appendix:baseline}
% Extensive baselines are used for comparison, which is briefly described as follows\footnote{For GCN, GAT, GraphSAGE, and APPNP layers, we adopt implementation from DGL library~\cite{wang2019deep}:https://github.com/dmlc/dgl }:
Baselines are briefly described as follows\footnote{For GCN, GAT, GraphSAGE, and APPNP layers, we adopt implementation from DGL library~\cite{wang2019deep}:https://github.com/dmlc/dgl }:
\begin{itemize}[leftmargin=*]
% \vspace{-1.5mm}
    \item \textbf{GCN}~\cite{welling2016semi} 
    is the most popular GNN, which defines the first-order approximation of a localized spectral filter on graphs.
    \item \textbf{GAT}~\cite{velivckovic2017graph}  
    introduces a self-attention mechanism to important scores for different neighbor nodes.
    \item \textbf{GraphSAGE}~\cite{hamilton2017inductive} 
    is an inductive framework that leverages node features to generate embeddings by sampling and aggregating features from the local neighborhood. 
    \item \textbf{APPNP}~\cite{gasteiger2019predict} 
    combines GCN with personalized PageRank. 
    \item \textbf{GCNII}\footnote{https://github.com/chennnM/GCNII}~\cite{chen2020simple}
    employs residual connection and identity mapping.
    \item \textbf{Grand}\footnote{https://github.com/THUDM/GRAND}~\cite{feng2020graph}
    purposes a random propagation strategy for data augmentation, and uses consistency regularization to optimize.
    \item \textbf{Mixhop}\footnote{https://github.com/samihaija/mixhop}~\cite{abu2019mixhop} aggregates mixing neighborhood information.
    \item \textbf{Geom-GCN}\footnote{https://github.com/graphdml-uiuc-jlu/geom-gcn}~\cite{pei2019geom}
    exploits geometric relationships to capture long-range dependencies within structural neighborhoods. Three variant of Geom-GCN is used for comparison.
    \item \textbf{GDC}\footnote{https://github.com/gasteigerjo/gdc}~\cite{gasteiger_diffusion_2019}
    refines graph structure based on diffusion kernels.
    % \item \textbf{Pro-GNN}\footnote{https://github.com/DSE-MSU/DeepRobust}~\cite{jin2020graph}
    \item \textbf{GEN}\footnote{https://github.com/BUPT-GAMMA/Graph-Structure-Estimation-Neural-Networks}~\cite{wang2021graph} 
    estimates underlying meaningful graph structures.
    \item \textbf{H$_2$GCN}\footnote{https://github.com/GemsLab/H2GCN}~\cite{zhu2020beyond} combine multi-hop neighbor-embeddings for adapting to both heterophily and homophily graph settings.
    \item \textbf{DropEdge}\footnote{https://github.com/DropEdge/DropEdge}~\cite{rong2019dropedge}
    randomly removes edges from the input graph for over-fitting prevention. 
    \item \textbf{Jaccard}\footnote{https://github.com/DSE-MSU/DeepRobust}~\cite{wu2019adversarial}
    prunes the edges connecting nodes with small Jaccard similarity.
\end{itemize} 

% \footnote{The implementation provided by CogDL~\cite{cen2021cogdl} is adopted for GCNII, GDC and Grand: https://github.com/thudm/cogdl}
% For SGC, GCN~\cite{welling2016semi}, Chebnet, GAT~\cite{velivckovic2017graph},GraphSAGE and APPNP, we adopt the implementations from the Deep Graph Learning library ~\cite{wang2019deep}.
% For the remaining baselines

% we utilize two-layer GNN encoders, and comparing with our \framework with them as the backbones. 

\begin{table*}[htb!]
    \renewcommand{\arraystretch}{0.95}
    \setlength{\abovecaptionskip}{0.25cm}
    \setlength{\belowcaptionskip}{-0.25cm}
    \caption{Comparison of training time(hr.) of achieving the best performance based on GPU.}
    % \vspace{-2.5mm}
    \label{tab:time comparasion}
    \centering
    % \scalebox{1.0}{
    \setlength{\tabcolsep}{3.6mm}{
        \begin{tabular}{l|ccccccccc}
        \hline
      Method & {Cora} & {Citeseer} & {Pubmed} & {PT} & {TW} & {Actor} & {Cornell} & {Texas} & {Wisconsin} \\
        \cline{1-1}
        \hline     % cora    cite    pub     pt    tw    act      cor      tex     wis
        \framework$_{GCN}$  &  0.071 & 0.213 & 4.574 & 0.178 &  0.374
                     & 1.482 & 0.006 & 0.008 & 0.009 \\
        \framework$_{SAGE}$   & 0.074 & 0.076 & 4.852 & 0.169 &  0.214
                     & 0.817 &  0.006 & 0.007 & 0.009 \\
        \framework$_{GAT}$   & 0.071 & 0.180 & 4.602 & 0.172 &  0.329
                     & 1.273 & 0.006 & 0.008 & 0.009 \\
        \framework$_{APPNP}$   & 0.073 & 0.215 & 4.854 & 0.138 &  0.379
                     & 1.367 & 0.010 & 0.011 & 0.013 \\
        \hline
        \end{tabular}
    }
    % \vspace{-2.5mm}
\end{table*}

% \vspace{-1.0em}
\subsection{Overall algorithm of \framework}
The overall algorithm of \framework~ is shown in Algorithm ~\ref{algorithm:training}. Note that, if choose to retain the connection from the previous iteration, to ensure that the number of edges remains stable during the training, a percentage of edges in the reconstructed graph with low similarity will be dropped in each iteration.

\label{appendix:overall algorithm}
\begin{algorithm}[htb!]
\SetAlgoRefName{1}
\SetAlgoVlined
\KwIn{a graph $G=(V,E)$, features $X$, labels $Y_L$, mode $\in {True,False}$\\
iterations $\eta$, encoding tree height $K$, hyperparameter $\theta$}
\KwOut{optimized graph $G'=(V,E')$, prediction $Y_P$, GNN parameters $\Theta$ }
Initialize $\Theta$;\\
\For{$i=1$ to $\eta$}{
    Update $\Theta$ by classification loss $\mathcal{L}_{cls}(Y_L,Y_P)$;\\
    Getting node representation $X'=\mathrm{GNN}(X)$;\\
    Initialize $k=1$ for $k$-NN structuralization;\\
    Create fusion map $G_{f}$ according to Algorithm~\ref{algorithm:kselector};\\
    Create $K$-dimensional encoding tree $\mathcal{T}^*$ according to Algorithm~\ref{algorithm:KDimSEMinimize};\\
    \For{each non-root node $\alpha$ in $\mathcal{T}^*$}{
        Calculate $H^{\mathcal{T}^*}(G_{f};(\lambda,\alpha])$ through Eq.~\ref{eq:deduct-se};\\
        Assign probability $P(\alpha)$ to $\alpha$ through Eq.~\ref{eq:prob-softmax};\\
    }
    \For{each subtree rooted at $\alpha$ in $\mathcal{T}^*$}{
        Assuming $\alpha$ has $n$ children, set $t = \theta \times n$;\\ 
        \For{$j=1$ to $t$}{
            Sample a node pair $(v_m,v_n)$ according to \S~\ref{step3};\\
            Adding edge $e_{mn}$ to $G'$;\\
        }
    }
    \If{mode}{
        Let $E' = E \cup E'$, where $E'$ and $E$ are the edge set of $G'$ and $G$, respectively;\\
        Drop a percentage of edges in $G'$;
    }
    Update graph structure $G \gets G'$;
    Update node representation: $X \gets X'$;\\
}
Get prediction $Y_P$;\\
Return $G'$, $Y_P$ and $\Theta$;\\
\caption{Model training for~\framework}
\label{algorithm:training}
\end{algorithm}

% \vspace{-1.0em}
\subsection{Algorithm of one-dimensional structural entropy guided graph enhancement}
\label{appendix:1dse algorithm}
The $k$-selector is designed for choosing an optimal $k$ for $k$-NN structuralization under the guidance of one-dimensional structural entropy maximization. The algorithm of $k$-selector and fusion graph construction is shown in Algorithm~\ref{algorithm:kselector}.

\begin{algorithm}[htb!]
\SetAlgoRefName{2}
\SetAlgoVlined
\KwIn{a graph $G=(V,E)$, node representation $X$}
\KwOut{fusion graph $G_{f}$ }
Calculate $S \in \mathbb{R}^{|V|\times |V|}$ via Eq.~\ref{eq:pcc};\\
\For{$k=2$ to $|V|-1$}{
    Generate $G_{knn}$ by $S$;\\
    Generate $G^{(k)}_{f} = \{V,E_{f} = E \cup E_{knn} \}$;\\
    Reweight $G^{(k)}_{f}$ via Eq.~\ref{eq:reweighted};\\
    Calculate $H^1(G^{(k)}_{f})$ via Eq.~\ref{eq:H1};\\
    \If{$H^1(G^{(k)}_{f})$ reaches the maximal optima}{
        $G_{f} \gets G^{(k)}_{f}$;\\
        Return $G_{f}$;
    }
}
\caption{$k$-selector and fusion graph construction}
\label{algorithm:kselector}
\end{algorithm}

% \vspace{-1.0em}
\subsection{Algorithm of high-dimensional structural entropy minimization}
\label{appendix:kdse algorithm}
% \vspace{-0.5em}
% The high-dimensional structural entropy minimization is a heuristic algorithm that adjusts encoding tree structure by greedily executing specific operators. The pseudo-code is shown in Algorithm~\ref{algorithm:KDimSEMinimize}.
The pseudo-code of the high-dimensional structural entropy minimization algorithm is shown in Algorithm~\ref{algorithm:KDimSEMinimize}.
%%伪代码
\begin{algorithm}[htb!]
\SetAlgoRefName{3}
\SetAlgoVlined
\KwIn{a graph $G=(V,E)$, the height of encoding tree $k>1$}
\KwOut{Optimal high-dimensional encoding tree $\mathcal{T}^*$}
//Initialize an encoding tree $\mathcal{T}$ with height $1$ and root $\lambda$\\
Create root node $\lambda$;\\
\For{$v_i \in V$}{
    Create node $\alpha_i$. Let $T_{\alpha_i} = v_i$;\\
    $v_i^- = \lambda$;\\
}
//Generation of binary encoding tree\\
%% problem
\While{$\lambda$ has more than $2$ children}{
    Select $\alpha_i$ and $\alpha_j$ in $\mathcal{T}$, 
    condition on $\alpha_i^-=\alpha_j^-=\lambda$ and $\mathop{\arg\max}\limits_{\alpha_i,\alpha_j}(H^\mathcal{T}(G)-H^\mathcal{T}_{\mathrm{CB}(\alpha_{i},\alpha_{j})}(G))$;\\
    $ \mathrm{CB}(\alpha_{i},\alpha_{j})$ according to Definition ~\ref{def:CBop};\\
}
//Squeezing of encoding tree\\
\While{$\mathrm{height}(\mathcal{T}) > K$}{
    Select non-root node $\alpha$ and $\beta$ in $\mathcal{T}$, 
    condition on $\alpha^-=\beta$ and $\mathop{\arg\max}\limits_{\alpha,\beta}(H^\mathcal{T}(G)-H^\mathcal{T}_{\mathrm{LF}(\alpha,\beta)}(G))$;\\
    $ \mathrm{LF}(\alpha,\beta)$ according to Definition ~\ref{def:LFop};\\
}
Return $\mathcal{T}^* \gets \mathcal{T}$;
\caption{K-dimensional structural entropy minimization}
\label{algorithm:KDimSEMinimize}
\end{algorithm}



 
\end{document}
\endinput
%%
%% End of file `sample-sigconf.tex'.
