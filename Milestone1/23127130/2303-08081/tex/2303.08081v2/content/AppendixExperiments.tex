\section{Experiments on real data}
In this section, we extend the prediction task of the main body of the paper.  The methodology used follows the same structure, we start by creating a distribution shift by training the model $f_\theta$ in California in 2014 and evaluating it in the rest of the states in 2018, creating a geopolitical and temporal shift. The model $g_\theta$ is trained each time on each state using only the $X^{ood}$ in the absence of the label, and its performance is evaluated by a 50/50 random train-test split. As models we use a gradient boosting decision tree\cite{xgboost,catboost} as estimator $f_\theta$, approximating the Shapley values by TreeExplainer \cite{lundberg2020local2global}, and using logistic regression for the \textit{Explanation Shift Detector}.

\subsection{ACS Employment}
The goal of this task is to predict whether an individual, between the ages of 16 and 90, is employed. For this prediction task the AUC of all the other states (except PR18) falls below $0.60$, indicating not high OOD explanations. For the most OOD state, PR18, the \enquote{Explanation Shift Detector} finds that the model has shifted due to features such as Citizenship or Military Service.

\begin{figure*}[ht]
\centering
\includegraphics[width=.49\textwidth]{images/AUC_OOD_ACSEmployment.png}\hfill
\includegraphics[width=.49\textwidth]{images/feature_importance_ACSEmployment.png}
\caption{In the left figure, comparison of the performance of \textit{Explanation Shift Detector}, in different states for the ACS Employment dataset. For this dataset, most of the states have the same OOD detection AUC, except for PR18. This difference in the model behaviour is due to features such as Citizenship and Military Service. Features such as difficulties in hearing or seeing, do not play a role in the OOD model behaviour.}
\label{fig:xai.employment}
\end{figure*}

\subsection{ACS Income}
 The goal of this task is to predict whether an individual's income is above $\$50,000$, only includes individuals above the age of 16, and report an income of at least $\$100$. This dataset can serve as a comparable replacement to the UCI Adult dataset.

 For this prediction task the results are different from the previous two cases, the state with the highest OOD score is $KS18$, with the \enquote{Explanation Shift Detector} highlighting features as Place of Birth, Race or Working Hours Per Week. The closest state to ID is CA18, where there is only a temporal shift without any geospatial distribution shift.
\begin{figure*}[ht]
\centering
\includegraphics[width=.49\textwidth]{images/AUC_OOD_ACSIncome.png}\hfill
\includegraphics[width=.49\textwidth]{images/feature_importance_ACSIncome.png}
\caption{In the left figure, comparison of the performance of \textit{Explanation Shift Detector}, in different states for the ACS Income prediction task.  In the left figure, we can see how the state with the highest OOD AUC detection is KS18 and not PR18 as in other prediction tasks, this difference with respect to the other prediction task can be attributed to \enquote{Place of Birth}, whose feature attributions the model finds to be more different than in CA14.}
\label{fig:xai.income}
\end{figure*}


\subsection{ACS Mobility}

The goal of this task is to predict whether an individual had the same residential address one year ago, only including individuals between the ages of 18 and 35. The goal of this filtering is to increase the prediction task difficulty, staying at the same address base rate is above $90\%$ for the population~\cite{ding2021retiring}.

The results of this experiment present a similar behaviour as the ACS Income prediction task (cf. Section \ref{fig:xai.income}), where the in-land states of the US are in an AUC range of $0.55-0.70$ and is the state of PR18 who achieves a higher OOD AUC. The features driving this behaviour are Citizenship for PR18 and Ancestry(Census record of your ancestors' lives with details like where they lived, who they lived with, and what they did for a living) for the other states. 

\begin{figure*}[ht]
\centering
\includegraphics[width=.49\textwidth]{images/AUC_OOD_ACSMobility.png}\hfill
\includegraphics[width=.49\textwidth]{images/feature_importance_ACSMobility.png}
\caption{In the left figure, comparison of the performance of \textit{Explanation Shift Detector}, in different states for the ACS Mobility dataset. Except for PR18, all the other states fall below an AUC of OOD detection of $0.70$. If we look at the features driving this difference is due to the Citizenship and the Ancestry relationship. For the other states protected social attributes such as Race or Marital status play an important role.}
\label{fig:xai.mobility}
\end{figure*}
