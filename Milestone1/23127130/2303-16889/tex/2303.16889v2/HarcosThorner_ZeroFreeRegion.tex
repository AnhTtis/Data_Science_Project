\documentclass[12pt]{amsart}
%\textwidth=360pt \textheight=615pt
\usepackage[margin=1.1in]{geometry}
%\documentclass[11pt,letterpaper]{amsart}
\usepackage{amssymb,amsfonts,amsmath,mathtools}
\usepackage{color}
\usepackage{soul}

%\usepackage{bbold}
%\usepackage[all,cmtip]{xy}
\usepackage{enumerate}
\usepackage{mathrsfs}
%\usepackage{comment}
\usepackage[unicode]{hyperref}
\usepackage[capitalise]{cleveref}
%\usepackage{floatrow}
%\usepackage{todonotes}
%\usepackage{url}
%\urlstyle{sf}
\usepackage[normalem]{ulem}
\let\C\undefined
\usepackage{constants}
\usepackage{bbm}

\hypersetup{colorlinks=true, citecolor=blue, linkcolor=blue, urlcolor=blue, pdfstartview=FitH, pdfauthor=Gergely Harcos and Jesse Thorner, pdftitle=A new zero-free region for Rankin--Selberg L-functions}

%----Table of Contents-----

%----Theorem Environments----
\newtheorem{theorem}{Theorem}[section]
\newtheorem{corollary}[theorem]{Corollary}
\newtheorem{hypothesis}[theorem]{Hypothesis}
\newtheorem{proposition}[theorem]{Proposition}
\newtheorem{lemma}[theorem]{Lemma}

\newtheorem{conjecture}[theorem]{Conjecture}
\newtheorem{question}[theorem]{Question}
\newtheorem{definition}[theorem]{Definition}
\newtheorem{question*}{Question}
\newtheorem{problem*}{Problem}
\newtheorem*{hypothesis*}{Hypothesis}
\newtheorem*{definition*}{Definition}

\theoremstyle{definition}
\newtheorem{definitions}[theorem]{Definitions}
\newtheorem{construction}[theorem]{Construction}
\newtheorem{example}[theorem]{Example}
\newtheorem{examples}[theorem]{Examples}
\newtheorem{notation}[theorem]{Notation}
\newtheorem{notations}[theorem]{Notations}
\newtheorem{addendum}[theorem]{Addendum}

\theoremstyle{remark}
\newtheorem*{remark}{Remark}
\newtheorem*{remarks}{Remarks}

\numberwithin{equation}{section}

\crefname{figure}{Figure}{Figures}
%MATH ENVIRONMENTS
\theoremstyle{plain}
\newtheorem*{theorem*}{Theorem}
\newtheorem*{lemma*}{Lemma}
\crefname{theorems}{Theorem}{Theorems}
\crefname{corollaries}{Corollary}{Corollaries}
\newtheorem*{corollary*}{Corollary}
\crefname{corollaries*}{Corollary}{Corollaries}
\crefname{lemma}{Lemma}{Lemmata}
\crefname{proposition}{Proposition}{Propositions}
\crefname{conjectures}{Conjecture}{Conjectures}
\newtheorem*{conjonjecture*}{Conjecture}
\crefname{conjonjectures*}{Conjecture}{Conjectures}
\crefname{definitions}{Definition}{Definitions}
\crefname{hypotheses}{Hypothesis}{Hypotheses}

\newcommand{\creflastconjunction}{, and\nobreakspace}

\renewcommand{\hat}{\widehat}

\newcommand{\Sym}{\mathrm{Sym}}

\allowdisplaybreaks

\newcommand{\Z}{\mathbb{Z}}
\newcommand{\CC}{\mathbb{C}}
\newcommand{\R}{\mathbb{R}}
\newcommand{\Q}{\mathbb{Q}}
\newcommand{\F}{\mathbb{F}}
\newcommand{\kd}{\mathfrak{d}}
\newcommand{\Ad}{\mathrm{Ad}}
\newcommand{\V}{\mathfrak{v}}
\newcommand{\pp}{\mathfrak{P}}
\newcommand{\f}{\mathfrak{f}}
%\renewcommand{\v}{\mathfrak{v}}
%\newcommand{\threebythree}[9]{\Big[\begin{matrix}#1&#2&#3\\#4&#5&#6\\#7&#8&#9\end{matrix}\Big]}
%\newcommand{\twobytwo}[4]{\Big[\begin{matrix}#1&#2\\#3&#4\end{matrix}\Big]}
\newcommand{\add}[2]{\displaystyle\sum_{#1}^{#2}}
%\newcommand{\abs}[1]{\displaystyle\Big| #1 \Big|}
\newcommand{\re}{\mathrm{Re}} %Gergely changed this to be in harmony with \Re
\newcommand{\im}{\mathrm{Im}} %Gergely changed this to be in harmony with \Re
\renewcommand{\O}{\cO}

\newcommand{\kb}{\mathfrak{b}}

\newcommand{\Sw}{\mathrm{Sw}}
\newcommand{\Ar}{\mathrm{Ar}}

\newcommand{\fb}[1]{\textcolor{red}{#1}}
\DeclareMathOperator{\As}{As}
\newcommand{\m}{\mathfrak{m}}
\newcommand{\GL}{\mathrm{GL}}
\newcommand{\PSL}{\mathrm{PSL}}
\newcommand{\SO}{\mathrm{SO}}
\newcommand{\SU}{\mathrm{SU}}
\newcommand{\Sp}{\mathrm{Sp}}
\newcommand{\PGL}{\mathrm{PGL}}
\newcommand{\SL}{\mathrm{SL}}
\newcommand{\Zgp}{\mathrm{Z}}
\newcommand{\N}{\mathrm{N}}
\newcommand{\A}{\mathbb{A}}
\newcommand{\Run}{\underset{s=1}{\text{Res}}\;}
\newcommand{\Gal}{\mathrm{Gal}}

\newcommand{\spec}{\mathrm{spec}}

\makeatletter
\DeclareFontFamily{U} {MnSymbolF}{}
\DeclareSymbolFont{symbolsMN}{U}{MnSymbolF}{m}{n}
\SetSymbolFont{symbolsMN}{bold}{U}{MnSymbolF}{b}{n}
\DeclareFontShape{U}{MnSymbolF}{m}{n}{
 <-6> MnSymbolF5
 <6-7> MnSymbolF6
 <7-8> MnSymbolF7
 <8-9> MnSymbolF8
 <9-10> MnSymbolF9
 <10-12> MnSymbolF10
 <12-> MnSymbolF12}{}
\DeclareFontShape{U}{MnSymbolF}{b}{n}{
 <-6> MnSymbolF-Bold5
 <6-7> MnSymbolF-Bold6
 <7-8> MnSymbolF-Bold7
 <8-9> MnSymbolF-Bold8
 <9-10> MnSymbolF-Bold9
 <10-12> MnSymbolF-Bold10
 <12-> MnSymbolF-Bold12}{}
\DeclareMathSymbol{\tbigtimes}{\mathop}{symbolsMN}{2}
\newcommand*{\bigtimes}{%
 \DOTSB
 \tbigtimes
 \slimits@ 
}
\makeatother

\renewcommand{\tilde}{\widetilde}

\newcommand{\ka}{\mathfrak{a}}

\newcommand{\cB}{\mathcal{B}}
\renewcommand{\bar}{\overline}
%\newcommand{\Lbrace}{\Big\lbrace}
%\newcommand{\Rbrace}{\Big\rbrace}
\renewcommand{\b}{\mathfrak{b}}

\newcommand{\cC}{\mathscr{C}(\pi)}
\newcommand{\Class}{\mathrm{Cl}}

\newcommand{\cD}{\mathcal{D}(\pi)}

\renewcommand{\epsilon}{\varepsilon}
\newcommand{\cE}{\mathcal{E}}
\newcommand{\bE}{\mathbb{E}}
\newcommand{\ke}{\mathfrak{e}}
\newcommand{\cF}{\mathcal{F}}
\newcommand{\bF}{\mathbb{F}}
\newcommand{\kf}{\mathfrak{f}}
\newcommand{\kF}{\mathfrak{F}}

\renewcommand{\H}{\mathbb{H}}
\newcommand{\cH}{\mathcal{H}}
\newcommand{\kH}{\mathbb{H}}
\newcommand{\HQ}{\mathcal{H}}
\newcommand{\kn}{\mathfrak{n}}
\newcommand{\bI}{\mathbb{I}}

\newcommand{\kj}{\mathfrak{j}}

\newcommand{\cK}{\mathcal{K}}

\newcommand{\RSC}{\mathrm{RSC}_n}

%\newcommand{\cL}{\log Y } %% Gergely changed this definition below
\newcommand{\sL}{\mathscr{L}}
\newcommand{\Li}{\mathrm{Li}}

\renewcommand{\Im}{\mathrm{Im}}
\newcommand{\Ind}{\mathrm{Ind}}
\newcommand{\smod}[1]{\, (\mathrm{mod}^*{\, #1} )}
\renewcommand{\pmod}[1]{\, (\mathrm{mod} {\, #1})}
\newcommand{\cN}{\mathcal{N}}

\newcommand{\cP}{\mathcal{P}}
\renewcommand{\Pr}{\mathbb{P}}

\newcommand{\cO}{\mathcal{O}}
\newcommand{\ord}{\mathrm{ord}\,}

\newcommand{\kq}{\mathfrak{q}}
\newcommand{\kp}{\mathfrak{p}}
\newcommand{\cQ}{\mathcal{Q}}

\newcommand{\cR}{\mathcal{R}}
\newcommand{\kr}{\mathfrak{r}}
\renewcommand{\Re}{\mathrm{Re}}
\renewcommand{\GL}{\mathrm{GL}}
\newcommand{\reg}{\mathrm{reg}}
\def\Res{\mathop{\mathrm{Res}}}
\def\Reg{\mathop{\mathrm{Reg}}}

\newcommand{\bs}{\mathbf{s}}
\def\sgn{\mathop{\mathrm{sgn}}}
\newcommand{\stab}{{\rm stab}}
\newcommand{\cS}{\mathcal{S}}
\newcommand{\supp}{\mathrm{supp}\,}
\newcommand{\sumP}{\sideset{}{'}\sum}
\newcommand{\sumS}{\sideset{}{^\star}\sum}
\newcommand{\sumD}{\sideset{}{^\dagger}\sum}
\newcommand{\sech}{{\hspace{0.05in}\rm sech}}

\newcommand{\bt}{\mathbf{t}}
\newcommand{\cT}{\mathcal{T}}
\newcommand{\Tr}{ \textbf{Tr}}

\newcommand{\cU}{\mathcal{U}}

\newcommand{\kv}{\mathfrak{v}}
\newcommand{\cV}{\mathcal{V}}
\newcommand{\vol}{\text{\rm vol}}

\newcommand{\cW}{\mathcal{W}}
\newcommand{\kw}{\mathfrak{w}}
\newcommand{\wkarrow}{\overset{\text{wk-}\ast}{\to}}

\newcommand{\bx}{\mathbf{x}}
\newcommand{\X}{\mathcal{X}}

\newcommand{\by}{\mathbf{y}}
\newcommand{\cY}{\mathcal{Y}}

\newcommand{\bz}{\mathbf{z}}
\newcommand{\cZ}{\mathcal{Z}}

\newcommand{\cL}{\mathcal{L}}

\DeclareMathAlphabet{\mathpzc}{OT1}{pzc}{m}{it}

\renewcommand{\pmod}[1]{\,(\mathrm{mod}\,\,#1)}

\newconstantfamily{abcon}{symbol=c}

\newcommand{\gergely}[1]{\color{red} {\bf Gergely:} #1 \color{black}}
\newcommand{\gh}[1]{\color{red} #1 \color{black}}
\newcommand{\jesse}[1]{\color{blue} {\bf Jesse:} #1 \color{black}}
\newcommand{\jt}[1]{\color{blue} #1 \color{black}}

\makeatletter
\let\@wraptoccontribs\wraptoccontribs
\makeatother

\title[A new zero-free region for Rankin--Selberg $L$-functions]{A new zero-free region for Rankin--Selberg $L$-functions}

\author{Gergely Harcos}
\address{Alfr{\'e}d R{\'e}nyi Institute of Mathematics, POB 127, Budapest H-1364, Hungary}
\email{\href{mailto:gharcos@renyi.hu}{gharcos@renyi.hu}}

\author{Jesse Thorner}
\address{Department of Mathematics, University of Illinois, Urbana, IL 61801, USA}
\email{\href{mailto:jesse.thorner@gmail.com}{jesse.thorner@gmail.com}}

%%%%%%%%%%%%%%%%%%%%%%%%%%%%%%%%%%%%%%%%%%%%%%%%%%%%%%%%%%%%%%%%%%%%%

\begin{document}

\begin{abstract}
Let $\pi$ and $\pi'$ be cuspidal automorphic representations of $\mathrm{GL}(n)$ and $\mathrm{GL}(n')$ with unitary central characters. We establish a new zero-free region for all $\mathrm{GL}(1)$-twists of the Rankin--Selberg $L$-function $L(s,\pi\times\pi')$, generalizing Siegel's celebrated work on Dirichlet $L$-functions. As an application, we prove a Siegel--Walfisz bound for the Dirichlet coefficients of $-L'(s,\pi\times\pi')/L(s,\pi\times\pi')$.
\end{abstract}

\thanks{The first author was supported by the R\'enyi Int\'ezet Lend\"ulet Automorphic Research Group and NKFIH (National Research, Development and Innovation Office) grant K~143876.}

\maketitle

\section{Introduction and statement of results}
\label{sec:intro}

\subsection{Introduction and the main result}
In 1896, Hadamard and de la Vall{\'e}e Poussin independently proved that the Riemann zeta function $\zeta(s)$ does not vanish in the half-plane $\Re(s)\geq 1$. This statement is equivalent to the asymptotic form of the prime number theorem. In 1899, de la Vall{\'e}e Poussin established the classical zero-free region for $\zeta(s)$, which was quickly extended to Dirichlet $L$-functions. In particular, let $\chi\pmod{q}$ be a Dirichlet character. There exists an absolute and effectively computable constant $\Cl[abcon]{ZetaZFR}>0$ such that $L(s,\chi)$ has at most one zero $\beta$ (necessarily real and simple) in the region
\[
\Re(s)\geq 1-\Cr{ZetaZFR}/\log(q(|\im(s)|+3)).
\]
If $\beta$ exists, then $\chi$ is quadratic. In this case, Siegel's lower bound on $L(1,\chi)$ implies that for any $\epsilon>0$, there exists an ineffective constant $\Cl[abcon]{SiegelDirichlet}=\Cr{SiegelDirichlet}(\epsilon)>0$ such that
\begin{equation}
\label{eqn:Siegel_Dirichlet}
L(\sigma,\chi)\neq 0,\qquad \sigma\geq 1-\Cr{SiegelDirichlet}q^{-\epsilon}.	
\end{equation}
See \cite{Siegel,Walfisz} for the original references.

The method of de la Vall{\'e}e Poussin can be modified to establish a zero-free region for many $L$-functions. Specifically, let $F$ be a number field, let $\mathbb{A}_F$ be the ring of adeles over $F$, and let $\mathfrak{F}_{n}$ be the set of cuspidal automorphic representations $\pi$ of $\GL_{n}(\mathbb{A}_F)$ whose central character $\omega_{\pi}$ is unitary. Given $\pi\in\mathfrak{F}_{n}$, let $\tilde{\pi}\in\mathfrak{F}_{n}$ be the contragredient and $L(s,\pi)$ be the $L$-function of $\pi$. Let $C(\pi)\geq 3$ denote the analytic conductor, which measures the arithmetic and spectral complexity of $\pi$. Given $(\pi,\chi)\in\mathfrak{F}_{n}\times\mathfrak{F}_{1}$, let $\pi\otimes\chi$ be the cuspidal automorphic representation $g\mapsto\pi(g)\chi(\det g)$. For convenience, we introduce the subset $\mathfrak{F}_n^*$ consisting of $\pi\in \mathfrak{F}_n$ for which $\omega_\pi$ is trivial on the diagonally embedded positive reals. For each $\pi\in\mathfrak{F}_n$, there exist unique $\pi^*\in\mathfrak{F}_n^*$ and $t_{\pi}\in\R$ such that
\[
\pi=\pi^*\otimes|\cdot|^{it_{\pi}},\qquad L(s,\pi)=L(s+it_{\pi},\pi^*).
\]

Given $(\pi,\pi')\in\mathfrak{F}_n\times\mathfrak{F}_{n'}$, let $L(s,\pi\times\pi')$ be the associated Rankin--Selberg $L$-function, whose basic properties were established by Jacquet, Piatetski-Shapiro, and Shalika~\cite{JPSS,JS1,JS2}. If $(\pi,\pi')\in\mathfrak{F}_n^*\times\mathfrak{F}_{n'}^*$, then $L(s,\pi\times\pi')$ is holomorphic away from a possible pole at $s=1$, which occurs if and only if $\pi'=\tilde{\pi}$. If $\pi'\in\mathfrak{F}_1^*$ is trivial, then $L(s,\pi\times\pi')=L(s,\pi)$. 

In 1981, Shahidi~\cite{Shahidi} proved that $L(s,\pi\times\pi')$ does not vanish in the half-plane $\Re(s)\geq 1$. In some cases, the method of de la Vall{\'e}e Poussin can extend this zero-free region. Brumley~\cite[Theorem~A.1]{Humphries} and Humphries and Thorner~\cite[Theorem~2.1]{HumphriesThorner} (see also Moreno~\cite{Moreno}) proved that there exists an effectively computable constant $\Cl[abcon]{ZFR2_std}=\Cr{ZFR2_std}(n,n',[F:\Q])>0$ with the following property. If $(\pi,\pi')\in\mathfrak{F}_n^*\times\mathfrak{F}_{n'}^*$, and
\begin{equation}
\label{eqn:special_star}
\pi=\tilde\pi\qquad\textup{or}
\qquad\pi'=\tilde\pi'\qquad\textup{or}
\qquad\pi'=\tilde\pi,
\end{equation}
then $L(s,\pi\times\pi')$ has at most one zero $\beta$ (necessarily real and simple) in the region
\[
\Re(s)\geq 1-\Cr{ZFR2_std}/\log(C(\pi)C(\pi')(|\im(s)|+3)).
\]
If the exceptional zero $\beta$ exists, then
\begin{equation}
\label{eqn:SZself-dual}
\pi=\tilde\pi\quad\textup{and}\quad\pi'=\tilde\pi'\qquad\textup{or}\qquad\pi'=\tilde\pi.
\end{equation}
If $\pi'\in\mathfrak{F}_1^*$ is trivial, then we recover the standard zero-free region for $L(s,\pi)$.

There are limited (though important) cases where \eqref{eqn:SZself-dual} holds and exceptional zeros have been precluded altogether. For example, if $\pi'\in\mathfrak{F}_1^*$ is trivial and $\pi\in\mathfrak{F}_2^*\cup\mathfrak{F}_3^*$, then the exceptional zero does not exist \cite{Banks,HoffsteinLockhart,HoffsteinRamakrishnan}. See \cite{Luo,RamakrishnanWang} for examples when
$\pi'\in\mathfrak{F}_2^*$ and $\pi\in\mathfrak{F}_2^*\cup\mathfrak{F}_3^*$.
Generalizing \eqref{eqn:Siegel_Dirichlet}, Jiang--L{\"u}--Thorner--Wang~\cite[Section~4]{JLTW} and Humphries--Thorner \cite[Theorem 2.4]{HumphriesThorner2} proved that if $\chi\in\mathfrak{F}_1^*$ is quadratic, then for all $\epsilon>0$, there exists an ineffective constant $\Cl[abcon]{Siegel_pre}=\Cr{Siegel_pre}(\pi,\epsilon)>0$ such that
\[
L(\sigma,\pi\otimes\chi)\neq 0,\qquad L(\sigma,\pi\otimes(\tilde{\pi}\otimes\chi))\neq 0,\qquad \sigma\geq 1-\Cr{Siegel_pre}C(\chi)^{-\epsilon}.
\]
See also Molteni \cite{Molteni}.

It is unclear how to execute the method of de la Vall{\'e}e Poussin for all $(\pi,\pi')\in\mathfrak{F}_n^*\times\mathfrak{F}_{n'}^*$ that do not satisfy \eqref{eqn:special_star}. In this setting, Brumley (\cite{Brumley}, \cite[Theorem~A.1]{Lapid}) made the first uniform improvement over \cite{Shahidi}, proving that for all $\epsilon>0$, there exists an effectively computable constant $\Cl[abcon]{ZFR_Brumley}=\Cr{ZFR_Brumley}(n,n',F,\epsilon)>0$ such that $L(s,\pi\times\pi')\neq 0$ when
\begin{equation}
\label{eqn:Brumley_ZFR}
\Re(s)\geq 1-\Cr{ZFR_Brumley}\left((C(\pi)C(\pi'))^{n+n'-1+\epsilon}(|\im(s)|+1)^{n'n[F:\Q](1-\frac{1}{n+n'})+\epsilon}\right)^{-1}.
\end{equation}
See also Zhang~\cite{Zhang}, who halved the exponent of $|\im(s)|+1$ in \eqref{eqn:Brumley_ZFR}, and Sarnak~\cite{Sarnak} and Gelbart--Lapid~\cite{GelbartLapid}, who established zero-free regions using Eisenstein series and the Maa{\ss}--Selberg relations.

In this paper, we develop a new method for establishing zero-free regions for $L$-functions. We apply our method to extend Siegel's celebrated result \eqref{eqn:Siegel_Dirichlet} to every $\GL_1$-twist of $L(s,\pi\times\pi')$, where $(\pi,\pi')\in\mathfrak{F}_n\times\mathfrak{F}_{n'}$. Our proof relies crucially on the group structure of $\mathfrak{F}_1$, not just $\mathfrak{F}_1^*$. We substantially improve \eqref{eqn:Brumley_ZFR} in the $\mathrm{GL}_1$-twist aspect, but the dependence on $\pi$ and $\pi'$ is no longer effective.

\begin{theorem}
\label{thm:Siegel}
Let $\pi\in\mathfrak{F}_n$ and $\pi'\in\mathfrak{F}_{n'}$. For all $\epsilon>0$, there exists an ineffective constant $\Cl[abcon]{ZFR2}=\Cr{ZFR2}(\pi,\pi',\epsilon)>0$ such that if $\chi\in\mathfrak{F}_1$, then
\begin{equation}
\label{eqn:finalbound2}
|L(\sigma,\pi\times(\pi'\otimes\chi))|\geq\Cr{ZFR2}C(\chi)^{-\epsilon},\qquad \sigma\geq 1-\Cr{ZFR2}C(\chi)^{-\epsilon}.
\end{equation}
\end{theorem}

\begin{remark}
By replacing $\chi$ with $\chi|\cdot|^{it}$, the bound \eqref{eqn:finalbound2} becomes
\begin{equation}
\label{eqn:ZFR}
|L(\sigma+it,\pi\times(\pi'\otimes\chi))|\geq\Cr{ZFR2}C(it,\chi)^{-\epsilon},\qquad \sigma\geq 1-\Cr{ZFR2}C(it,\chi)^{-\epsilon},
\end{equation}
where $C(it,\chi)\ll C(\chi)(|t|+1)^{[F:\Q]}$. In particular, there exists an ineffective constant $\Cl[abcon]{C777}=\Cr{C777}(\pi,\pi',\epsilon)>0$ such that
\[|L(\sigma+it,\pi\times\pi')|\geq\Cr{C777}(|t|+1)^{-\epsilon},\qquad \sigma\geq 1-\Cr{C777}(|t|+1)^{-\epsilon}.	\]
This $t$-aspect lower bound can be used for bounding Eisenstein series $E(g,\varphi,s)$ on $\GL_{n+n'}$ coming from cusp forms $\varphi$ on the Levi factor $\GL_{n}\times\GL_{n'}$ of $\GL_{n+n'}$. See \cite[Corollary~2]{GelbartLapid} and its proof for details.
\end{remark}

\subsection{An application}
\label{subsec:application}
Let $\cO_F$ be the ring of integers of $F$, and define the norm $\N=\N_{F/\Q}$ on the nonzero ideals $\ka$ of $\cO_F$ by $\N\ka=|\cO_F/\ka|$. Let $\kq$ be a nonzero ideal of $\cO_F$, and let $I(\kq)$ be the group of fractional ideals that are coprime to $\kq$. Let $P(\kq)$ be the subgroup of $I(\kq)$ consisting of principal fractional ideals $(\alpha)$ such that $\alpha$ is totally positive and $\alpha\equiv 1\pmod{\kq}$. The narrow ray class group $\mathrm{Cl}(\kq)$ is the finite abelian quotient $I(\kq)/P(\kq)$. Let $\mathscr{P}(\kq)$ be the set of primitive ray class characters that induce the characters of $\mathrm{Cl}(\kq)$.

If there exists $u\in\R$ such that $(\pi,\pi')\in\mathfrak{F}_n\times\mathfrak{F}_{n'}$ satisfies $\pi'=\tilde{\pi}\otimes|\cdot|^{iu}$, then we define $\mathcal{M}_{\pi\times\pi'}(x)=x^{1-iu}/(1-iu)$. For all other pairs $(\pi,\pi')\in\mathfrak{F}_n\times\mathfrak{F}_{n'}$, we define $\mathcal{M}_{\pi\times\pi'}(x)=0$. We also define $\Lambda_{\pi\times\pi'}(\ka)$ by the Dirichlet series identity
\begin{equation}
\label{eqn:Lambda_def}
\sum_{\ka}\frac{\Lambda_{\pi\times\pi'}(\ka)}{\N\ka^s}=-\frac{L'}{L}(s,\pi\times\pi'),\qquad\Re(s)>1.
\end{equation}
For a class $\mathcal{C}\in \mathrm{Cl}(\kq)$ and $x\geq 2$, the bound
\begin{equation}
\label{eqn:Edef}
\mathcal{E}_{\pi\times\pi'}(x;\kq,\mathcal{C}):=\sum_{\substack{\N\ka\leq x \\ \ka\in\mathcal{C}}}\Lambda_{\pi\times\pi'}(\ka)-\frac{1}{|\mathrm{Cl}(\kq)|}\sum_{\chi\in\mathscr{P}(\kq)}\overline{\chi}(\mathcal{C})\mathcal{M}_{\pi\times(\pi'\otimes\chi)}(x)=o(x)
\end{equation}
is equivalent to the fact that if $\chi\in\mathscr{P}(\kq)$ and $\Re(s)\geq 1$, then $L(s,\pi\times(\pi'\otimes\chi))\neq 0$.

Unconditionally, Brumley's narrow zero-free region \eqref{eqn:Brumley_ZFR} suffices to prove that there exists an effectively computable constant $\Cl[abcon]{Brumley_PNTAP1}=\Cr{Brumley_PNTAP1}(n,n',[F:\Q])>0$ such that if $\N\kq\leq(\log x)^{\Cr{Brumley_PNTAP1}}$, then $\mathcal{E}_{\pi\times\pi'}(x;\kq,\mathcal{C})\ll_{\pi,\pi'}x(\log x)^{-\Cr{Brumley_PNTAP1}}$. Before the present work, the bound $\mathcal{E}_{\pi\times\pi'}(x;\kq,\mathcal{C})\ll_{\pi,\pi',A}x(\log x)^{-A}$ for all $A>0$ was only known in limited cases, as a consequence of the zero-free regions proved in \cite{Humphries,HumphriesThorner2,HumphriesThorner,JLTW}. We prove the following result using \cref{thm:Siegel}.
\begin{theorem}
\label{thm:PNTAP}
Let $\pi\in\mathfrak{F}_n$, $\pi'\in\mathfrak{F}_{n'}$, $x\geq 2$, and $A>0$. For a nonzero ideal $\kq$ of $\cO_F$, let $\mathrm{Cl}(\kq)$ be the associated ray class group, and let $\mathcal{C}\in\mathrm{Cl}(\kq)$. If $\mathcal{E}_{\pi\times\pi'}(x;\kq,\mathcal{C})$ is given by \eqref{eqn:Edef} and $\N\kq\leq(\log x)^A$, then
\[
\mathcal{E}_{\pi\times\pi'}(x;\kq,\mathcal{C})\ll_{\pi,\pi',A}x(\log x)^{-A}.
\]
The implied constant is ineffective.
\end{theorem}

\begin{remark}
Let $q\geq 1$ be an integer. If $F=\Q$, $\kq=(q)$, and $q$ is coprime to the conductors of $\pi$ and $\pi'$, then \cref{thm:PNTAP} states that if $a\in\Z$ and $\gcd(a,q)=1$, then
\[
\sum_{\substack{m\leq x \\ m\equiv a\pmod{q}}}\Lambda_{\pi\times\pi'}(m)=\frac{\mathcal{M}_{\pi\times\pi'}(x)}{\varphi(q)}+O_{\pi,\pi',A}\left(\frac{x}{(\log x)^A}\right),	\qquad q\leq (\log x)^A,
\]
where $\varphi(q)$ is Euler's totient function. This generalizes the Siegel--Walfisz theorem \cite{Siegel,Walfisz}.
\end{remark}

\section{Properties of $L$-functions}
\label{sec:Properties}

Let $F$ be a number field with adele ring $\A_F$. Let $D_F$ be the absolute discriminant of $F$, $\cO_F$ the ring of integers of $F$, and $\N=\N_{F/\Q}$ the norm defined on nonzero ideals $\ka$ of $\cO_F$ by $\N\ka=|\cO_F/\ka|$. For a place $v$ of $F$, let $v\mid\infty$ (resp. $v\nmid\infty$) denote that $v$ is archimedean (resp. non-archimedean), and let $F_v$ be the corresponding completion of $F$. Each $v\nmid\infty$ corresponds with a prime ideal $\kp$ of $\cO_F$.

\subsection{Standard $L$-functions}
\label{subsec:standard}

Let $\mathfrak{F}_n$ be the set of cuspidal automorphic representations $\pi$ of $\GL_n(\A_F)$ whose central character $\omega_\pi$ is unitary, and let $\mathfrak{F}_n^*$ be the subset of those $\pi$'s for which $\omega_\pi$ is trivial on the diagonally embedded positive reals. Every $\pi\in\mathfrak{F}_n$ can be written uniquely as
\begin{equation}
\label{eqn:pidecomp}
\pi=\pi^*\otimes|\cdot|^{it_{\pi}},\qquad \pi^*\in\mathfrak{F}_n^*,\quad t_{\pi}\in\R.
\end{equation}

If $\pi\in\mathfrak{F}_n^*$, then for each place $v$, there exists an irreducible admissible representation $\pi_v$ of $\GL_n(F_v)$, with $\pi_v$ ramified for at most finitely many $v$, such that $\pi$ is a restricted tensor product $\otimes_v \pi_v$. When $v\nmid\infty$ and $\kp$ corresponds with $v$, then we write $\pi_v$ and $\pi_{\kp}$ interchangeably. For each prime ideal $\kp$, there exist $n$ Satake parameters $(\alpha_{j,\pi}(\kp))_{j=1}^{n}$ such that the standard $L$-function $L(s,\pi)$ of $\pi$ is the absolutely convergent Euler product
\[
L(s,\pi)=\prod_{\kp}L(s,\pi_{\kp}) = \prod_{\kp}\prod_{j=1}^n \frac{1}{1-\alpha_{j,\pi}(\kp)\N\kp^{-s}},\qquad\Re(s)>1.
\]
For $v\mid \infty$, define
\[
\Gamma_v(s)=\begin{cases}
\pi^{-s/2}\Gamma(s/2)&\mbox{if $F_v=\R$,}\\
2(2\pi)^{-s}\Gamma(s)&\mbox{if $F_v=\mathbb{C}$.}
\end{cases}
\]
There exist $n$ Langlands parameters $(\mu_{j,\pi}(v))_{j=1}^{n}$ such that
\[
L(s,\pi_{v})=\prod_{j=1}^n \Gamma_v(s+\mu_{j,\pi}(v)).
\]

Let $\kq_{\pi}$ denote the conductor of $\pi$, and let $\mathbbm{1}\in\mathfrak{F}_1^*$ denote the trivial representation (whose $L$-function is the Dedekind zeta function $\zeta_F(s)$). Let $\delta_{\pi}=1$ if $\pi=\mathbbm{1}$, and $\delta_{\pi}=0$ otherwise. If
\[
L_{\infty}(s,\pi)=\prod_{v\mid\infty}L(s,\pi_v),
\]
then the completed $L$-function
\[
\Lambda(s,\pi)=(s(1-s))^{\delta_{\pi}}(D_F^n\N\kq_{\pi})^{s/2}L_{\infty}(s,\pi)L(s,\pi)
\]
is entire of order $1$. There exists $W(\pi)\in\CC$ of modulus $1$ such that $\Lambda(s,\pi)$ satisfies the functional equation
\[
\Lambda(s,\pi)=W(\pi)\Lambda(1-s,\tilde\pi)=W(\pi)\overline{\Lambda(1-\bar{s},\pi)}.
\]
The nontrivial zeros of $L(s,\pi)$ are the zeros of $\Lambda(s,\pi)$, and they lie in the critical strip $0<\Re(s)<1$. It is conjectured (GRH) that they actually lie on the critical line $\Re(s)=1/2$. Finally, the analytic conductor is $C(\pi)=C(0,\pi)$, where
\[
C(it,\pi) = D_F^n\N\kq_{\pi}\prod_{v\mid\infty}\prod_{j=1}^n (|\mu_{j,\pi}(v)+it|+3)^{[F_v:\R]}.
\]

More generally, these quantities are defined for all $\pi\in\mathfrak{F}_n$, and we have in particular
\[
L(s,\pi)=L(s+it_{\pi},\pi^*),\qquad C(it,\pi)=C(it+it_{\pi},\pi^*).
\]

\subsection{Rankin--Selberg $L$-functions}
\label{subsec:RS}

Let $\pi\in\mathfrak{F}_n^*$ and $\pi'\in\mathfrak{F}_{n'}^*$. For each $\kp\mid\kq_{\pi}\kq_{\pi'}$, there exist complex numbers $\alpha_{j,j',\pi\times\pi'}(\kp)$ with $1\leq j\leq n$ and $1\leq j'\leq n$ such that if
\[
L(s,\pi_{\kp}\times\pi_{\kp}')=\begin{cases}
\prod_{j=1}^n \prod_{j'=1}^{n'}(1-\alpha_{j,\pi}(\kp)\alpha_{j',\pi'}(\kp)\N\kp^{-s})^{-1}&\mbox{if $\kp\nmid\kq_{\pi}\kq_{\pi'}$,}\\
\prod_{j=1}^n \prod_{j'=1}^{n'}(1-\alpha_{j,j',\pi\times\pi'}(\kp)\N\kp^{-s})^{-1}&\mbox{if $\kp\mid\kq_{\pi}\kq_{\pi'}$,}
\end{cases}
\]
then the Rankin--Selberg $L$-function $L(s,\pi\times\pi')$ equals the absolutely convergent product
\[
L(s,\pi\times\pi')=\prod_{\kp}L(s,\pi_{\kp}\times\pi_{\kp}')=\sum_{\ka}\frac{\lambda_{\pi\times\pi'}(\ka)}{\N\ka^s},\qquad \Re(s)>1.
\]
By \cite[Lemma~3.1]{JLW}, we have the bound
\begin{equation}
\label{eqn:JLW_decouple}
|\lambda_{\pi\times\pi'}(\ka)|\leq \sqrt{\lambda_{\pi\times\tilde{\pi}}(\ka)\lambda_{\pi'\times\tilde{\pi}'}(\ka)}.	
\end{equation}

Let $\kq_{\pi\times\pi'}$ be the conductor of $L(s,\pi\times\pi')$. For each $v\mid\infty$, $1\leq j\leq n$, and $1\leq j'\leq n'$, there exists a Langlands parameter $\mu_{j,j',\pi\times\pi'}(v)$ such that if
\begin{equation}
\label{eqn:L1}
L_{\infty}(s,\pi\times\pi')=\prod_{v\mid\infty}\prod_{j=1}^n \prod_{j=1}^{n'}\Gamma_v(s+\mu_{j,j',\pi\times\pi'}(v)),\qquad \delta_{\pi\times\pi'}=\begin{cases}
	1&\mbox{if $\pi'=\tilde{\pi}$,}\\
	0&\mbox{otherwise,}
\end{cases}
\end{equation}
then the completed $L$-function
\begin{equation}
\label{eqn:L2}
\Lambda(s,\pi\times\pi')=(s(1-s))^{\delta_{\pi\times\pi'}}(D_F^{n'n}\N\kq_{\pi\times\pi'})^{s/2} L_{\infty}(s,\pi\times\pi')L(s,\pi\times\pi')
\end{equation}
is an entire function of order 1. There exists $W(\pi\times\pi')\in\CC$ of modulus $1$ such that $\Lambda(s,\pi\times\pi')$ satisfies the functional equation
\begin{equation}
\label{eqn:FE}
\Lambda(s,\pi\times\pi')=W(\pi\times\pi')\Lambda(1-s,\tilde{\pi}\times\tilde{\pi}')=W(\pi\times\pi')\overline{\Lambda(1-\bar{s},\pi\times\pi')}.
\end{equation}

The absolute convergence of the Euler product representation of $L(s,\pi\times\pi')$ ensures that $\re(\mu_{j,j',\pi\times\pi'}(v))\geq -1$. The nontrivial zeros of $L(s,\pi\times\pi')$ are the zeros of $\Lambda(s,\pi\times\pi')$, and they lie in the critical strip $0<\Re(s)<1$. It is conjectured (GRH) that they actually lie on the critical line $\Re(s)=1/2$. Finally, the analytic conductor is $C(\pi\times\pi')=C(0,\pi\times\pi')$, where
\begin{equation}
\label{eqn:C}
C(it,\pi\times\pi')=D_F^{n'n}\N\kq_{\pi\times\pi'}\prod_{v\mid\infty}\prod_{j=1}^n \prod_{j=1}^{n'}
(|\mu_{j,j',\pi\times\pi'}(v)+it|+3)^{[F_v:\R]}.
\end{equation}
We have the bounds \cite[Lemma~A.2]{Humphries}
\begin{equation}
\label{eqn:BH}
C(it,\pi\times\pi')\ll_{n,n'} C(\pi\times\pi') (|t|+1)^{n'n[F:\Q]}\ll_{n,n'} C(\pi)^{n'}C(\pi')^n (|t|+1)^{n'n[F:\Q]}.
\end{equation}

More generally, for $\pi\in\mathfrak{F}_{n}$ and $\pi'\in\mathfrak{F}_{n'}$, we have
\[
L(s,\pi\times\pi') = L(s+it_{\pi}+it_{\pi'},\pi^*\times\pi'^*),\qquad 
C(it,\pi\times\pi')=C(it+it_{\pi}+it_{\pi'},\pi^*\times\pi'^*).
\]

\subsection{Local bounds}

Let $\pi\in\mathfrak{F}_n$. At each prime ideal $\kp$ of $\cO_F$ and each $v\mid \infty$, there exists $\theta_n\in[0,1/2-1/(n^2+1)]$ such that \cite{LRS,MullerSpeh}
\begin{equation}
\label{eqn:GRC1}
|\alpha_{j,\pi}(\kp)|\leq \N\kp^{\theta_n},\qquad \re(\mu_{j,\pi}(v))\geq -\theta_n.
\end{equation}
Using the explicit descriptions of $\alpha_{j,j',\pi\times\pi'}(\kp)$ in \cite[Appendix]{SoundararajanThorner} and $\mu_{j,j',\pi\times\pi'}(\kp)$ in \cite[Section~3]{MullerSpeh}, we find that if $\pi\in\mathfrak{F}_n$ and $\pi'\in\mathfrak{F}_{n'}$, then
\begin{equation}
\label{eqn:GRC2}
|\alpha_{j,j',\pi\times\pi'}(\kp)|\leq \N\kp^{\theta_n+\theta_{n'}},\qquad \re(\mu_{j,j',\pi\times\pi'}(v))\geq -\theta_n-\theta_{n'}.
\end{equation}


\subsection{Isobaric sums}
\label{subsec:isobaric}

The Langlands theory of Eisenstein series associates to any $r$-tuple 
$(\pi_1,\dotsc,\pi_r)\in\mathfrak{F}_{n_1}\times\dotsb\times \mathfrak{F}_{n_r}$ an automorphic representation of $\GL_{n_1+\dotsb+n_r}(\A_F)$, the isobaric sum, denoted $\Pi=\pi_1\boxplus\dotsb\boxplus\pi_r$. The contragredient is $\tilde{\Pi}=\tilde{\pi}_1\boxplus\dotsb\boxplus\tilde{\pi}_r$, and $L(s,\Pi)=\prod_{j=1}^r L(s,\pi_j)$.

\begin{lemma}[{\cite[Lemma~a]{HoffsteinRamakrishnan}}]
\label{lem:nonneg}
If $\Pi$ is an isobaric sum, then the Rankin--Selberg $L$-function
\[
L(s,\Pi\times\tilde{\Pi})=\prod_{j=1}^r \prod_{k=1}^{r} L(s,\pi_j\times\tilde{\pi}_k)=\sum_{\ka}\frac{\lambda_{\Pi\times\tilde{\Pi}}(\ka)}{\N\ka^s},\qquad \Re(s)>1
\]
has nonnegative Dirichlet coefficients, as does its logarithm.
\end{lemma}

\subsection{Convexity}

Our proofs require strong bounds for Rankin--Selberg $L$-functions and their derivatives. We state our next lemma for $\pi\in\mathfrak{F}_n^*$ and $\pi'\in\mathfrak{F}_{n'}^*$, and this suffices in light of the discussion in \cref{subsec:RS}.

\begin{lemma}
\label{lem:Li1}
For $\pi\in\mathfrak{F}_n^*$ and $\pi'\in\mathfrak{F}_{n'}^*$, consider the holomorphic function
\[
\cL(s,\pi\times\pi')=\left(\frac{s-1}{s+1}\right)^{\delta_{\pi\times\pi'}}L(s,\pi\times\pi'),\qquad\Re(s)>-1.
\]
If $j\geq 0$, $\sigma\geq 0$, $t\in\R$, and $\epsilon>0$, then
\[
\cL^{(j)}(\sigma+it,\pi\times\pi')\ll_{n,n',[F:\Q],j,\epsilon}C(it,\pi\times\pi')^{\max(1-\sigma,0)/2+\epsilon}.
\]
In particular, if $\delta_{\pi\times\pi'}=0$ or $|t|>1$, then
\[
L^{(j)}(\sigma+it,\pi\times\pi')\ll_{n,n',[F:\Q],j,\epsilon}C(it,\pi\times\pi')^{\max(1-\sigma,0)/2+\epsilon}.
\]
\end{lemma}

\begin{proof}
Without loss of generality, $0<\epsilon<1$. It suffices to show for all $\sigma\geq -\epsilon/2$ that
\begin{equation}
\label{eqn:Lbound}
\cL(\sigma+it,\pi\times\pi')\ll_{n,n',[F:\Q],\epsilon}C(it,\pi\times\pi')^{\max(1-\sigma,0)/2+3\epsilon/4}.
\end{equation}
Indeed, assuming this statement and using the bound
\[
|\cL^{(j)}(s,\pi\times\pi')|\leq j!(2/\epsilon)^j\sup_{|z-s|=\epsilon/2}|\cL(z,\pi\times\pi')|
\]
that follows from Cauchy's formula and the triangle inequality for complex line integrals, we readily obtain the required conclusion (for all $j\geq 0$, $\sigma\geq 0$, and $t\in\R$).

It remains to prove \eqref{eqn:Lbound} for all $\sigma\geq -\epsilon/2$. Assume first that $\sigma\geq 1+\epsilon/2$. It follows from 
\eqref{eqn:JLW_decouple} and the Cauchy--Schwarz inequality that
\[
|L(\sigma+it,\pi\times\pi')|\leq(L(\sigma,\pi\times\tilde\pi)L(\sigma,\pi'\times\tilde\pi'))^{1/2}.
\]
The right-hand side is decreasing in $\sigma$, hence by \cite[Theorem~2]{Li} we infer that
\[
\log|L(\sigma+it,\pi\times\pi')|\ll_{n,n',[F:\Q],\epsilon}
\frac{\log C(\pi\times\tilde\pi)}{\log\log C(\pi\times\tilde\pi)}+\frac{\log C(\pi'\times\tilde\pi')}{\log\log C(\pi'\times\tilde\pi')}.
\]
Then, using \cite[Lemma~2.1]{SoundararajanThorner}, we conclude that
\begin{equation}
\label{eqn:bound_on_1-line}
L(\sigma+it,\pi\times\pi')\ll_{n,n',[F:\Q],\epsilon}C(it,\pi\times\pi')^{\epsilon/4}.
\end{equation}
Concerning the application of \cite[Lemma~2.1]{SoundararajanThorner}, three remarks are in order. First, this result is meant for 
$F=\Q$, but a generalization to an arbitrary number field $F$ is straightforward (with extra constants depending on $[F:\Q]$ in the exponents). Second, \cite{SoundararajanThorner} tacitly assumes that each central character is trivial on the diagonally embedded positive reals, but this assumption is not used in the proof of \cite[Lemma~2.1]{SoundararajanThorner}. Third, we rely on the fact that $C(it,\pi\times\pi')$ in \eqref{eqn:bound_on_1-line} equals $C(\pi\times(\pi'\otimes|\cdot|^{it}))$.

Finally, we verify \eqref{eqn:Lbound} in the strip $-\epsilon/2\leq\sigma< 1+\epsilon/2$. We express the boundary value $L(-\epsilon/2+it,\pi\times\tilde\pi)$ from $L(1+\epsilon/2-it,\pi\times\tilde\pi)$ using the functional equation \eqref{eqn:FE} and the definitions \eqref{eqn:L1}--\eqref{eqn:L2}. Then, applying \cite[Lemma~3.2]{Harcos} and recalling \eqref{eqn:C}, we obtain from \eqref{eqn:bound_on_1-line} the bound
\begin{equation}
\label{eqn:bound_on_0-line}
L(-\epsilon/2+it,\pi\times\pi')\ll_{n,n',[F:\Q],\epsilon} C(it,\pi\times\pi')^{1/2+3\epsilon/4}.
\end{equation}
We interpolate between \eqref{eqn:bound_on_1-line} and \eqref{eqn:bound_on_0-line} using the Phragm{\'e}n--Lindel{\"o}f principle, and the lemma follows.
\end{proof}

\subsection{A weak Brun--Titchmarsh bound}

Let $\pi\in\mathfrak{F}_n$ and $\pi'\in\mathfrak{F}_{n'}$. Recall the Dirichlet series definition of $\Lambda_{\pi\times\pi'}(\ka)$ in \eqref{eqn:Lambda_def}. 

\begin{lemma}
\label{lem:BT}
Let $(\pi,\pi')\in\mathfrak{F}_{n}\times\mathfrak{F}_{n'}$. There exists an effectively computable constant $\Cl[abcon]{BTrange}=\Cr{BTrange}(n,n',[F:\Q])>0$ such that if
\[X\geq\max(C(\pi\times\tilde{\pi}),C(\pi'\times\tilde{\pi}'))^{\Cr{BTrange}}
\qquad\text{and}\qquad 1\leq T\leq X^{1/(16\max(n,n')^2[F:\Q])},
\]
then
\[
\sum_{X<\N\ka\leq Xe^{1/T}}|\Lambda_{\pi\times\pi'}(\ka)|\ll_{n,n',[F:\Q]} \frac{X}{T}.
\]
\end{lemma}

\begin{proof}
By \cite[Proposition~A.1]{SoundararajanThorner}, we have the bound
\begin{equation}
\label{eqn:Brumley_decouple}
|\Lambda_{\pi\times\pi'}(\ka)|\leq \sqrt{\Lambda_{\pi\times\tilde{\pi}}(\ka)\Lambda_{\pi'\times\tilde{\pi}'}(\ka)}.	
\end{equation}
Combined with the Cauchy--Schwarz inequality, \eqref{eqn:Brumley_decouple} yields
\begin{equation}
\label{eqn:BTCS}
\sum_{X<\N\ka\leq Xe^{1/T}}|\Lambda_{\pi\times\pi'}(\ka)|\leq \left(\sum_{X<\N\ka\leq Xe^{1/T}}\Lambda_{\pi\times\tilde{\pi}}(\ka)\right)^{1/2}\left(\sum_{X<\N\ka\leq Xe^{1/T}}\Lambda_{\pi'\times\tilde{\pi}'}(\ka)\right)^{1/2}.
\end{equation}

By \cite[Proposition~4.1]{HumphriesThorner}, there exist absolute and effectively computable constants $\Cl[abcon]{BT1}>0$ and $\Cl[abcon]{BT2}>0$ such that if $\log\log C(\pi\times\tilde{\pi})\geq \Cr{BT1}n^4[F:\Q]^2$, $X\geq e^{\Cr{BT2}n^4[F:\Q]^2}C(\pi\times\tilde{\pi})^{32n^2[F:\Q]}$, and $1\leq T\leq X^{1/(16n^2[F:\Q])}$, then
\[
\sum_{X<\N\ka\leq Xe^{1/T}}\Lambda_{\pi\times\tilde{\pi}}(\ka)\ll n^2[F:\Q]\frac{X}{T}.
\]
A careful inspection of the proof of \cite[Proposition~4.1]{HumphriesThorner} shows if the lower bound on $C(\pi\times\tilde{\pi})$ is removed, then the dependency on $n$ and $[F:\Q]$ in the implied constant and the range of $X$ become worse, which is not a concern here. We conclude that there exists an effectively computable constant $\Cr{BTrange}=\Cr{BTrange}(n,[F:\Q])>0$ such that if $X\geq C(\pi\times\tilde{\pi})^{\Cr{BTrange}}$ and $1\leq T\leq X^{1/(16n^2[F:\Q])}$, then
\[
\sum_{X<\N\ka\leq Xe^{1/T}}\Lambda_{\pi\times\tilde{\pi}}(\ka)\ll_{n,[F:\Q]} \frac{X}{T}.
\]
To finish, we similarly bound the second factor on the right-hand side of \eqref{eqn:BTCS}.
\end{proof}

\begin{corollary}
\label{cor:BT}
Let $(\pi,\pi',\chi)\in\mathfrak{F}_{n}\times\mathfrak{F}_{n'}\times\mathfrak{F}_1$. If
\[
X\geq 1\qquad\text{and}\qquad X\geq Y\geq X^{1-1/(16\max(n,n')^2[F:\Q])},
\]
then
\begin{equation}
\label{eqn:corBTbound}
\sum_{X<\N\ka\leq X+Y}|\Lambda_{\pi\times(\pi'\otimes\chi)}(\ka)|\ll_{\pi,\pi'}Y.
\end{equation}
\end{corollary}

\begin{proof}
We define $X_0=\max(C(\pi\times\tilde{\pi}),C(\pi'\times\tilde{\pi}'))^{\Cr{BTrange}}$. If $X\geq X_0$, then \eqref{eqn:corBTbound} follows from \cref{lem:BT} upon setting $T=X/Y$. If $X<X_0$, then \eqref{eqn:corBTbound} follows from \eqref{eqn:Brumley_decouple} and the Cauchy--Schwarz inequality.
\end{proof}

\section{The key proposition}

The goal of this section is to prove the key \Cref{prop:P1}.  We begin with a finiteness statement on $\GL_1$-twists.

\begin{lemma}
\label{lem:twists}
Let $\pi\in\mathfrak{F}_n$ and $\pi'\in\mathfrak{F}_{n'}$. There are finitely many $\chi\in\mathfrak{F}_1^*$ such that $L(s,\pi\times(\pi'\otimes\chi))$ has a pole.
\end{lemma}

\begin{proof} Without loss of generality, $\pi\in\mathfrak{F}_n^*$ and $\pi'\in\mathfrak{F}_{n'}^*$. Let $\chi\in\mathfrak{F}_1^*$ be 
such that $L(s,\pi\times(\pi'\otimes\chi))$ has a pole. Then $\pi'\otimes\chi=\tilde\pi$, hence $n=n'$, and $\chi$ can only ramify where $\pi$ or $\pi'$ ramifies. Moreover, $\chi^n$ is uniquely determined by the equation
$\omega_{\pi'}\chi^n=\bar{\omega_\pi}$. Therefore, it suffices to show that if $S$ is a finite set of places containing all archimedean places and
\[
G=\prod_{v\in S}{F_v^\times}^n\prod_{v\not\in S}\cO_{F_v}^\times,
\]
then $F^\times G$ has finite index in $\A_F^\times$. Let $\A_F^1$ be the group of ideles of norm $1$, and let $G^1=\A_F^1\cap G$. We have that
\[
\A_F^\times/(F^\times G)\cong\A_F^1/(F^\times G^1)\cong(\A_F^1/F^\times)/(F^\times G^1/F^\times),
\]
where $G$ is open in $\A_F^\times$, $G^1$ is open in $\A_F^1$, and $F^\times G^1/F^\times$ is open in $\A_F^1/F^\times$.
As $\A_F^1/F^\times$ is compact, the right-hand side is finite, and so is the left-hand side.
\end{proof}

We proceed with two lemmata on residues.

\begin{lemma}
\label{lem:residue}
Let $f_0(s),\dotsc,f_m(s)$ be $m+1$ complex functions that are holomorphic in an open neighborhood of $s_0\in\CC$. If there exists $c\geq 0$ such that $|f_j(s_0)|=c$ for all $j\in\{0,\ldots,m-1\}$, then
\[
\mathop{\mathrm{Res}}_{s=s_0}\frac{f_0(s)\dotsb f_m(s)}{(s-s_0)^m}
\]
equals $c$ times a $\CC$-linear combination of monomials of the derivative values $f_j^{(k)}(s_0)$ for $(j,k)\in\{0,\dotsc,m\}\times\{0,\dotsc,m-1\}$. The monomials in the linear combination, and the modulus of each coefficient in the linear combination depend at most on $m$.
\end{lemma}

\begin{proof}
Expanding $f_0(s),\dotsc,f_m(s)$ into Taylor series around $s_0$, we obtain
\[
\mathop{\mathrm{Res}}_{s=s_0}\frac{f_0(s)\dotsb f_m(s)}{(s-s_0)^m} = 
\sum_{\substack{k_0,\dotsc,k_m\geq 0\\ k_0+\dotsb+k_m=m-1}}\prod_{j=0}^m\frac{f_j^{(k_j)}(s_0)}{k_j!}.
\]
Since $k_0+\dotsb+k_m=m-1$, at least one of $k_0,\dotsc,k_{m-1}$ equals zero. Consequently, the $j$-product above has a factor of $f_j(s_0)$ for some $j\in\{0,\dots,m-1\}$, and the result follows.
\end{proof}

We apply \cref{lem:residue} to study the residues of an auxiliary $L$-function. For $(\pi,\pi',\chi)\in\mathfrak{F}_n\times\mathfrak{F}_{n'}\times\mathfrak{F}_1$, we define
\[
\Pi=\pi\boxplus\pi\otimes\chi\boxplus\tilde\pi'\boxplus\tilde\pi'\otimes\bar{\chi}
\qquad\text{and}\qquad
D(s)=L(s,\Pi\times\tilde{\Pi}).
\]
We have the factorization
\begin{equation}
\label{eqn:D_def}
\begin{aligned}
D(s)=&~L(s,\pi\times\tilde\pi)^2 L(s,\pi'\times\tilde\pi')^2 L(s,\pi\times(\pi'\otimes\chi))^2 L(s,\tilde\pi\times(\tilde\pi'\otimes\bar{\chi}))^2\\
&\cdot L(s,\pi\times(\tilde\pi\otimes\chi)) L(s,\pi'\times(\tilde\pi'\otimes\chi)) L(s,\tilde\pi\times\tilde\pi') L(s,\pi\times(\pi'\otimes\chi^2))\\
&\cdot L(s,\pi\times(\tilde\pi\otimes\bar{\chi})) L(s,\pi'\times(\tilde\pi'\otimes\bar{\chi})) L(s,\pi\times\pi') L(s,\tilde\pi\times(\tilde\pi'\otimes\bar{\chi}^2)).
\end{aligned}
\end{equation}
Let
\begin{equation}
\label{eqn:Qdef}
Q=(C(\pi)C(\pi'))^{2(n+n')}C(\chi)^{(n+n')^2}.
\end{equation}

\begin{lemma}
\label{lem:residue_bounds}
Let $(\pi,\pi',\chi)\in\mathfrak{F}_n\times\mathfrak{F}_{n'}\times\mathfrak{F}_1$, $x>1$, $\epsilon\in(0,1)$, and $\beta\in(1-\epsilon/2,1)$. Recall the notations \eqref{eqn:pidecomp}, \eqref{eqn:D_def}, \eqref{eqn:Qdef}, and let $\mathcal{S}$ be the set of poles of $D(s)$. Assume that the following $L$-functions are entire:
\begin{equation}
\label{eqn:Lfns}
L(s,\pi\times\pi'),\qquad L(s,\pi\times(\pi'\otimes\chi)),\qquad L(s,\pi\times(\pi'\otimes\chi^2)).
\end{equation}
If $\pi\otimes\chi^*=\pi$ or $\pi'\otimes\chi^*=\pi'$, then assume also that $|t_\chi|>1$. We have that
\[
\sum_{s_0\in\mathcal{S}}\mathop{\mathrm{Res}}_{s=s_0}D(s)x^{s-\beta}\Gamma(s-\beta)
\ll_{n,n',[F:\Q],\beta,\epsilon}|L(1,\pi\times(\pi'\otimes\chi))|(Qx)^{\epsilon}.
\]
\end{lemma}

\begin{proof} 
First, assume that $\pi\otimes\chi^*=\pi$ and $\pi'\otimes\chi^*=\pi'$.  Then $|t_{\chi}|>1$ by hypothesis, and $\mathcal{S}=\{1,1-it_\chi,1+it_\chi\}$. For each choice of $s_0\in\mathcal{S}$, let $m$ be the order of the pole of $D(s)$ at $s=s_0$. Specifically, $m=4$ for $s_0=1$, and $m=2$ for $s_0=1\pm it_\chi$. Consider the decomposition
\begin{equation}
\label{eqn:decomposition}
(s-s_0)^m D(s)x^{s-\beta}\Gamma(s-\beta)=f_0(s)\dotsb f_m(s),
\end{equation}
where
\begin{itemize}
\item $f_0(s)=f_1(s)=L(s+it_\chi,\pi\times\pi')$ and $f_2(s)=f_3(s)=L(s-it_\chi,\tilde{\pi}\times\tilde{\pi}')$ for $s_0=1$;
\item $f_0(s)=L(s,\tilde\pi\times\tilde\pi')$ and $f_1(s)=L(s+2it_\chi,\pi\times\pi')$ for $s_0=1-it_\chi$;
\item $f_0(s)=L(s,\pi\times\pi')$ and $f_1(s)=L(s-2it_\chi,\tilde{\pi}\times\tilde{\pi}')$ for $s_0=1+it_\chi$.
\end{itemize}
These three cases correspond to the three lines in \eqref{eqn:D_def}, with $f_0(s),\dotsc,f_{m-1}(s)$ occurring as $m$ factors on the relevant line. Now we apply \cref{lem:residue} in conjunction with \cref{lem:Li1} and \eqref{eqn:BH}. The functions $f_0(s),\dotsc,f_m(s)$ defined above are holomorphic in the open disk $|s-s_0|<1-\beta$. Moreover,
\[
|f_j(s_0)|=|L(1,\pi\times(\pi'\otimes\chi))|,\qquad j\in\{0,\dots,m-1\}.
\]
We are finished upon noting that, at the point $s_0$, the $k$-th derivative of $s\mapsto x^{s-\beta}$ is bounded by $x^{\epsilon/2}(\log x)^k$, while
\[
\left|\Gamma^{(k)}(s_0-\beta)\right|\leq\int_0^\infty r^{-\beta}|\log r|^k e^{-r}\,dr.
\]

Second, assume that exactly one of $\pi\otimes\chi^*=\pi$ and $\pi'\otimes\chi^*=\pi'$ holds true.  Then $|t_{\chi}|>1$ by hypothesis, and $\mathcal{S}=\{1,1-it_\chi,1+it_\chi\}$. For each choice of $s_0\in\mathcal{S}$, let $m$ be the order of the pole of $D(s)$ at $s=s_0$. Specifically, $m=4$ for $s_0=1$, and $m=1$ for $s_0=1\pm it_\chi$. Consider the decomposition \eqref{eqn:decomposition}, where
\begin{itemize}
\item $f_0(s)=f_1(s)=L(s+it_\chi,\pi\times\pi')$ and $f_2(s)=f_3(s)=L(s-it_\chi,\tilde{\pi}\times\tilde{\pi}')$ for $s_0=1$;
\item $f_0(s)=L(s,\tilde\pi\times\tilde\pi')$ for $s_0=1-it_\chi$;
\item $f_0(s)=L(s,\pi\times\pi')$ for $s_0=1+it_\chi$.
\end{itemize}
From here we proceed exactly as in the previous case.

Finally, assume that $\pi\otimes\chi^*\neq\pi$ and $\pi'\otimes\chi^*\neq\pi'$. Then $\mathcal{S}=\{1\}$ by hypothesis.  The order of the pole of $D(s)$ at $s=1$ is 4. We consider the decomposition \eqref{eqn:decomposition}, where 
\[
f_0(s)=f_1(s)=L(s,\pi\times(\pi'\otimes\chi))\qquad\text{and}\qquad f_2(s)=f_3(s)=L(s,\tilde\pi\times(\tilde\pi'\otimes\bar{\chi})).
\]
These four factors occur on the first line of \eqref{eqn:D_def}, and we finish as in the previous two cases.
\end{proof}

We use \cref{lem:nonneg,lem:Li1,lem:residue_bounds} to prove the following conditional lower bound on $L(1,\pi\times(\pi'\otimes\chi))$ that will play a central role in our proof of \cref{thm:Siegel}.

\begin{proposition}
\label{prop:P1}
Let $(\pi,\pi',\chi)\in\mathfrak{F}_n\times\mathfrak{F}_{n'}\times\mathfrak{F}_1$, $\epsilon\in(0,1/2)$, and $\beta\in(1-\epsilon/8,1)$. 
Assume that the $L$-functions in \eqref{eqn:Lfns} are entire.
If $L(\beta,\pi\times\pi')=0$, then
\begin{equation}
\label{eqn:main_lower}
|L(1,\pi\times(\pi'\otimes\chi))|\gg_{\pi,\pi',\beta,\epsilon}C(\chi)^{-(n+n')^2\epsilon}.
\end{equation}
\end{proposition}

\begin{proof}
Recall the notations \eqref{eqn:pidecomp} and \eqref{eqn:Qdef}. If $\pi\otimes\chi^*=\pi$ or $\pi'\otimes\chi^*=\pi'$, then we may assume that $|t_\chi|>1$, because
there are finitely many choices for $\chi^*$ by \Cref{lem:twists}, and both sides of \eqref{eqn:main_lower} are positive continuous functions of $t_\chi\in\R$. Subject to this constraint, let $D(s)$ be as in \eqref{eqn:D_def}, let $\mathcal{S}$ be the set of its poles, and let $x>1$ be a parameter to be chosen later. By the initial assumptions, $D(\beta)=0$. Hence $\mathcal{S}$ is also the set of poles of $D(s)x^{s-\beta}\Gamma(s-\beta)$ in the half-plane $\Re(s)>0$, and we shall use this below.

If $\lambda_{D}(\ka)$ is the $\ka$-th Dirichlet coefficient of $D(s)$, then $\lambda_{D}(\ka)\geq 0$ by \cref{lem:nonneg}. Since $\lambda_{D}(\cO_F)=1$, we have by the residue theorem
\begin{align*}
\frac{1}{e}\leq \sum_{\ka}\frac{\lambda_{D}(\ka)}{\N\ka^{\beta}}e^{-\frac{\N\ka}{x}}
&=\frac{1}{2\pi i}\int_{1-i\infty}^{1+i\infty}D(s+\beta)x^s\Gamma(s)\,ds\\
&=\sum_{s_0\in\mathcal{S}}\mathop{\mathrm{Res}}_{s=s_0}D(s)x^{s-\beta}\Gamma(s-\beta)
+\frac{1}{2\pi i}\int_{1/2-i\infty}^{1/2+i\infty}D(s)x^{s-\beta}\Gamma(s-\beta)\,ds.
\end{align*}
We estimate the sum over $\mathcal{S}$ by \cref{lem:residue_bounds} (with $\epsilon$ replaced by $\epsilon/4$), and the last integral by
\cref{lem:Li1} (with $\epsilon$ replaced by $\epsilon/16$) combined with \eqref{eqn:BH} and Stirling's formula. We conclude that
\begin{equation}
\label{eqn:previousbound}
1\ll_{n,n',[F:\Q],\beta,\epsilon} \left(|L(1,\pi\times(\pi'\otimes\chi))|+Qx^{-1/2}\right)(Qx)^{\epsilon/4}.	
\end{equation}
At this point, we choose
\[x = \max\left(1,Q^{2}|L(1,\pi\times(\pi'\otimes\chi))|^{-2}\right).\]
If $x=1$, then \eqref{eqn:main_lower} is trivial. Otherwise, $x=Q^{2}|L(1,\pi\times(\pi'\otimes\chi))|^{-2}>1$, and \eqref{eqn:previousbound} yields
\eqref{eqn:main_lower} after solving for $|L(1,\pi\times(\pi'\otimes\chi))|$:
\[|L(1,\pi\times(\pi'\otimes\chi))|\gg_{n,n',[F:\Q],\beta,\epsilon} Q^{-3\epsilon/(4-2\epsilon)}>Q^{-\epsilon}.\qedhere\]
\end{proof}

\section{Proof of \cref{thm:Siegel} for $\sigma=1$.}

In this section, we prove that there exists an ineffective constant $\Cl[abcon]{ZFR3}=\Cr{ZFR3}(\pi,\pi',\epsilon)>0$ such that for any $\chi\in\mathfrak{F}_1$ we have
\begin{equation}
\label{eqn:finalbound3}
|L(1,\pi\times(\pi'\otimes\chi))|\geq\Cr{ZFR3}C(\chi)^{-\epsilon}.
\end{equation}
Without loss of generality, we may assume that $\epsilon\in(0,1/2)$. We define
\[
\epsilon'=\frac{\epsilon}{8(n+n')^2}.
\]

\subsection{An initial reduction}
\label{subsec:inital_reduction}

We shall assume that $L(s,\pi\times(\pi'\otimes\chi))$ is holomorphic in the open disk $|s-1|<1$, for otherwise \eqref{eqn:finalbound3} is clear. Moreover, if $L(s,\pi\times(\pi'\otimes\chi))$ has no zero in the half-plane $\Re(s)>1-\epsilon'$, then standard methods produce a bound much stronger than \eqref{eqn:finalbound3}. Indeed, assume this zero-free region, and let $\sigma\in[1,2]$. Let us work with a parameter $x\geq 2$ (to be chosen later) and the Mellin transform pair
\[\phi(r)=\max(1-r,0),\qquad \hat\phi(w)=1/(w^2+w).\]
Proceeding as in the proof of \cite[Proposition~5.16]{IK}, but shifting the contour only to $\Re(w)=-\epsilon'/2$, we infer that
\[
-\frac{L'}{L}(\sigma,\pi\times(\pi'\otimes\chi))=
\sum_{\ka}\frac{\Lambda_{\pi\times(\pi'\otimes\chi)}(\ka)}{\N\ka^{\sigma}}\phi\left(\frac{\N\ka}{x}\right)\\
+O_{\pi,\pi',\epsilon}\left(x^{-\epsilon'/2}\log C(\chi)\right).
\]

Since $C(\chi)\geq 3$ and $0<\epsilon'<1/64$, the choice $x=(\log C(\chi))^{2/\epsilon'}$ satisfies $x\geq 2$ and $x^{-\epsilon'/2}\log C(\chi)=1$. Integrating the resulting approximation from $\sigma=1$ to $\sigma=2$, and applying the triangle inequality, we infer
\begin{align*}
|\log L(1,\pi\times(\pi'\otimes\chi))|&\leq
\sum_{2\leq \N\ka\leq x}\frac{|\Lambda_{\pi\times(\pi'\otimes\chi)}(\ka)|}{\N\ka\log\N\ka}+O_{\pi,\pi',\epsilon}(1)\\
&\ll_{\pi,\pi',\epsilon} \sum_{k=1}^{\lfloor \log x\rfloor}\frac{1}{ke^k}\sum_{e^{k-1}<\N\ka\leq e^k}|\Lambda_{\pi\times(\pi'\otimes\chi)}(\ka)|+1.
\end{align*}
By \cref{cor:BT}, we conclude the bound
\[
|\log L(1,\pi\times(\pi'\otimes\chi))|\ll_{\pi,\pi',\epsilon}\log\log x\ll_{\pi,\pi',\epsilon}\log\log\log C(\chi).
\]
This is much stronger than \eqref{eqn:finalbound3}.

\subsection{Overall strategy.}
\label{subsec:strategy}

By the work in \cref{subsec:inital_reduction}, we only need to prove \eqref{eqn:finalbound3} for those characters $\chi\in\mathfrak{F}_1$ for which there exists $(\beta,\gamma)\in(1-\epsilon',1)\times\R$ such that
\[L(\beta+i\gamma,\pi\times(\pi'\otimes\chi))=0.\]
Keeping this in mind, we shall prove \eqref{eqn:finalbound3} in three steps. Each step relies on an application of 
\cref{prop:P1} and allows us to verify \eqref{eqn:finalbound3} for a larger \emph{subgroup} of characters $\chi\in\mathfrak{F}_1$ than before. Let us introduce the notation
\[\mathfrak{F}_1^{(j)}=\{\chi\in\mathfrak{F}_1:{\chi^*}^j=1\},\]
and note the chain of subgroups $\mathfrak{F}_1^{(1)}\leq\mathfrak{F}_1^{(2)}\leq\mathfrak{F}_1$.

In the first step, we prove \eqref{eqn:finalbound3} for $\chi\in\mathfrak{F}_1^{(1)}$. In the second step, we extend the validity of \eqref{eqn:finalbound3} to $\chi\in\mathfrak{F}_1^{(2)}$. In the third step, we prove \eqref{eqn:finalbound3} in full generality.
For the discussion below, let us assume that $L(s,\pi\times(\pi'\otimes\chi))$ is entire. At each application of \cref{prop:P1}, we 
rely on the ``existence of bad zero'' hypothesis that we paraphrase as follows. For the character $\chi\in\mathfrak{F}_1$ under consideration, there exists $(\beta,\eta)\in(1-\epsilon',1)\times\mathfrak{F}_1^{(1)}$ such that 
\[L(\beta,\pi\times(\pi'\otimes\chi\eta))=0.\]
However, we cannot use $(\beta,\eta)$ directly because the implied constants are not allowed to depend on $\chi$. Instead, we use that each step operates with a given subgroup $G\in\{\mathfrak{F}_1^{(1)},\mathfrak{F}_1^{(2)},\mathfrak{F}_1\}$, and by the above hypothesis there exists $(\beta,\psi)\in(1-\epsilon',1)\times G$ such that 
\begin{equation}
\label{eqn:siegelhypothesis}
\text{$L(s,\pi\times(\pi'\otimes\psi))$ is entire and vanishes at $s=\beta$.}
\end{equation}
This \emph{weaker} hypothesis follows from the previous one (upon setting $\psi=\chi\eta$). Its advantage 
lies in the observation that since $G$ is fixed, we can select an admissible pair $(\beta,\psi)$ solely in terms of the triple $(\pi,\pi',\epsilon)$. Such selection is possible via the axiom of choice or a more constructive approach based on some well-ordering of the admissible pairs.

\subsection{The case of $\chi^*$ trivial.} In this subsection, we prove \eqref{eqn:finalbound3} for all $\chi$ lying in the subgroup 
$G=\mathfrak{F}_1^{(1)}$. Let us write $\chi=|\cdot|^{it}\in G$ with $t\in\R$. If $L(s,\pi\times\pi')$ has a pole, then there exists $u\in\R$ such that $\pi'=\tilde\pi\otimes|\cdot|^{iu}$, hence \eqref{eqn:finalbound3} holds in the stronger form
\[
L(1+it,\pi\times\pi')\gg_{\pi,\pi'}1/\log(|t|+3)
\]
by appealing to the zero-free region in \cite[Theorem~2.1]{HumphriesThorner}. Therefore, we shall assume that $L(s,\pi\times\pi')$ is entire. Furthermore, as explained in \cref{subsec:strategy}, we can assume that \eqref{eqn:siegelhypothesis} holds for some $(\beta,\psi)\in(1-\epsilon',1)\times G$ depending only on $(\pi,\pi',\epsilon)$.

Consider the automorphic representations
\begin{equation}
\label{eqn:changeofvariable}
\pi''=\pi'\otimes\psi\in\mathfrak{F}_{n'}\qquad\text{and}\qquad\chi'=\bar\psi\chi\in G.
\end{equation}
It follows that
\[L(s,\pi\times(\pi''\otimes\chi'))=L(s,\pi\times(\pi'\otimes\chi))\qquad\text{and}\qquad C(\chi')\asymp_{\pi,\pi',\epsilon}C(\chi),\]
and \eqref{eqn:finalbound3} is equivalent to
\begin{equation}
\label{eqn:finalbound4}
L(1,\pi\times(\pi''\otimes\chi'))\gg_{\pi,\pi'',\epsilon}C(\chi')^{-\epsilon}.
\end{equation}
But \eqref{eqn:finalbound4} follows readily from \cref{prop:P1}, upon noting that $L(\beta,\pi\times\pi'')=0$ and the following $L$-functions are entire:
\begin{equation}
\label{eqn:Lfns2}
L(s,\pi\times\pi''),\qquad L(s,\pi\times(\pi''\otimes\chi')),\qquad L(s,\pi\times(\pi''\otimes\chi'^2)).
\end{equation}
Indeed, these $L$-functions are shifts of $L(s,\pi\times\pi')$, which is entire by assumption.

We have shown that \eqref{eqn:finalbound3} holds for all $\chi\in\mathfrak{F}_1^{(1)}$. Consequently, \eqref{eqn:finalbound3} also holds for all $\chi$ in any \emph{fixed coset} of $\mathfrak{F}_1^{(1)}$ within $\mathfrak{F}_1$. We shall use this principle in the next subsection.

\subsection{The case of $\chi^*$ quadratic.} In this subsection, we prove \eqref{eqn:finalbound3} for all $\chi$ lying in the subgroup 
$G=\mathfrak{F}_1^{(2)}$. If $L(s,\pi\times(\pi'\otimes\chi))$ has a pole, then by \cref{lem:twists}, $\chi$ lies in finitely many cosets of $\mathfrak{F}_1^{(1)}$ (depending only on $(\pi,\pi')$), hence \eqref{eqn:finalbound3} holds by the concluding remark of the previous subsection. Therefore, we shall assume that $L(s,\pi\times(\pi'\otimes\chi))$ is entire.

As before, we also assume that \eqref{eqn:siegelhypothesis} holds for some $(\beta,\psi)\in(1-\epsilon',1)\times G$ depending only on $(\pi,\pi',\epsilon)$, and we need to prove \eqref{eqn:finalbound4} with the notation \eqref{eqn:changeofvariable}. But \eqref{eqn:finalbound4} follows readily from \cref{prop:P1}, upon noting that $L(\beta,\pi\times\pi'')=0$ and the $L$-functions in \eqref{eqn:Lfns2} are entire. Indeed, the first two $L$-functions in \eqref{eqn:Lfns2} are entire by assumption, while the third $L$-function is a shift of the first one due to $\chi\in G$.

We have shown that \eqref{eqn:finalbound3} holds for all $\chi\in\mathfrak{F}_1^{(2)}$. Consequently, \eqref{eqn:finalbound3} also holds for all $\chi$ in any \emph{fixed coset} of $\mathfrak{F}_1^{(2)}$ within $\mathfrak{F}_1$. We shall use this principle in the next subsection.

\subsection{The general case.} In this subsection, we prove \eqref{eqn:finalbound3} in general. As before, we assume that $L(s,\pi\times(\pi'\otimes\chi))$ is entire, and \eqref{eqn:siegelhypothesis} holds for some $(\beta,\psi)\in(1-\epsilon',1)\times \mathfrak{F}_1$ depending only on $(\pi,\pi',\epsilon)$. We need to prove \eqref{eqn:finalbound4} with the notation \eqref{eqn:changeofvariable}. 

If $L(s,\pi\times(\pi''\otimes\chi'^2))$ has a pole, then by \cref{lem:twists}, $\chi'$ lies in finitely many cosets of $\mathfrak{F}_1^{(2)}$ depending only on $(\pi,\pi'')$, hence \eqref{eqn:finalbound4} holds by the concluding remark of the previous subsection. Therefore, we shall assume that $L(s,\pi\times(\pi''\otimes\chi'^2))$ is entire. But then \eqref{eqn:finalbound4} follows readily from \cref{prop:P1}, because $L(\beta,\pi\times\pi'')=0$ and the $L$-functions in \eqref{eqn:Lfns2} are entire by assumption.

\section{Finishing the proof of \cref{thm:Siegel}}

Now that we have unconditionally proved \eqref{eqn:finalbound3} for all $\pi$, $\pi'$, $\chi$, and $\epsilon>0$, we will use it to finish the proof of \cref{thm:Siegel}. As before, by continuity and nonvanishing arguments, we can assume that $L(s,\pi\times(\pi'\otimes\chi))$ is holomorphic in the open disk $|s-1|<1$.

\subsection{The case of $\sigma<1$.}\label{subsub}
We begin with the straightforward bound
\[
|L(1,\pi\times(\pi'\otimes\chi))-L(\sigma,\pi\times(\pi'\otimes\chi))|\leq(1-\sigma)\sup_{\kappa\in[\sigma,1]}|L'(\kappa,\pi\times(\pi'\otimes\chi))|.
\]
For $|L(1,\pi\times(\pi'\otimes\chi))|$, we have the lower bound \eqref{eqn:finalbound3}. On the other hand, by \cref{lem:Li1} and \eqref{eqn:BH}, there exists a constant $\Cl[abcon]{ZFRN2}=\Cr{ZFRN2}(\pi,\pi',\epsilon)>0$ such that
\[
|L'(\kappa,\pi\times(\pi'\times\chi))|\leq\Cr{ZFRN2}C(\chi)^{\epsilon/2},\qquad\kappa\geq 1-\frac{\epsilon}{2nn'}.
\]
Define
\[
\Cl[abcon]{ZFRN3}=\Cr{ZFRN3}(\pi,\pi',\epsilon)=\min\left(\frac{\Cr{ZFR3}}{1+\Cr{ZFRN2}},\frac{\epsilon}{2nn'}\right).
\]
If $1-\Cr{ZFRN3}C(\chi)^{-\epsilon}<\sigma<1$, then
\begin{align*}
|L(\sigma,\pi\times(\pi'\otimes\chi))|
&\geq |L(1,\pi\times(\pi'\otimes\chi))|-|L(1,\pi\times(\pi'\otimes\chi))-L(\sigma,\pi\times(\pi'\otimes\chi))|\\
&\geq |L(1,\pi\times(\pi'\otimes\chi))|-\Cr{ZFRN3}C(\chi)^{-\epsilon}\cdot\Cr{ZFRN2}C(\chi)^{\epsilon/2}\\
&\geq (\Cr{ZFR3}- \Cr{ZFRN2}\Cr{ZFRN3})C(\chi)^{-\epsilon/2}\\
&\geq \Cr{ZFRN3} C(\chi)^{-\epsilon/2}.
\end{align*}

\subsection{The case of $\sigma>1$.}

It follows from \eqref{eqn:GRC1} and \eqref{eqn:GRC2} that
\begin{equation}
\label{eqn:edge2}
L(\sigma,\pi\times(\pi'\otimes\chi))\asymp_{n,n'}1,\qquad\sigma\geq 3.
\end{equation}
To prove \eqref{eqn:finalbound2} in the strip $1<\sigma<3$, we interpolate between \eqref{eqn:finalbound3} and \eqref{eqn:edge2} by applying Phragm{\'e}n--Lindel{\"o}f principle to $1/L(s,\pi\times\pi')$.

\section{Proof of \cref{thm:PNTAP}}

Let $x\geq 2$, $A>0$, and $y=x(\log x)^{-A}$. Consider the Mellin transform pair
\begin{equation}
\label{eqn:hatphi}
\begin{aligned}
\phi(r)&=\mathbf{1}_{(0,x]}(r)+\mathbf{1}_{(x,x+y]}(r)\frac{x+y-r}{y},\qquad r>0;\\
\hat\phi(s)&=\int_{0}^{\infty}\phi(r)r^{s-1}\,dr=\frac{(x+y)^{s+1}-x^{s+1}}{y(s^2+s)},\qquad\Re(s)>0.
\end{aligned}
\end{equation}
Clearly,
\begin{align*}
\sum_{\substack{\N\ka\leq x \\ \ka\in\mathcal{C}}}\Lambda_{\pi\times\pi'}(\ka) = \sum_{\substack{ \ka\in\mathcal{C}}}\Lambda_{\pi\times\pi'}(\ka)\phi(\N\ka)+O\left(\sum_{x<\N\ka\leq x+y}|\Lambda_{\pi\times\pi'}(\ka)|\right).
\end{align*}
By \cref{cor:BT}, the error term above is $O_{\pi,\pi'}(y)$.

Let $\widehat{\mathrm{Cl}(\kq)}$ be the group of characters of $\mathrm{Cl}(\kq)$. By character orthogonality, we have that
\[
\sum_{\substack{\ka\in\mathcal{C}}}\Lambda_{\pi\times\pi'}(\ka)\phi(\N\ka)=\frac{1}{|\mathrm{Cl}(\kq)|}\sum_{\psi\in\widehat{\mathrm{Cl}(\kq)}}\bar{\psi}(\mathcal{C})\mathcal\sum_{\ka}\Lambda_{\pi\times\pi'}(\ka)\psi(\ka)\phi(\N\ka).
\]
Using \eqref{eqn:GRC1} and \eqref{eqn:GRC2}, we find that if $\chi\in\mathscr{P}(\kq)$ is the primitive ray class character that induces $\psi$, then
\begin{align*}
\sum_{\ka}\Lambda_{\pi\times\pi'}(\ka)\psi(\ka)\phi(\N\ka)-\sum_{\ka}\Lambda_{\pi\times(\pi'\otimes\chi)}(\ka)\phi(\N\ka)&\ll n'n\sum_{\kp\mid\kq\kq_{\pi}\kq_{\pi'}}\sum_{\ell\leq \frac{\log (2x)}{\log\N\kp}}\N\kp^{\ell(\theta_n+\theta_{n'})}\log\N\kp\\
&\ll_{\pi,\pi'} x^{1-\frac{1}{n^2+1}-\frac{1}{(n')^2+1}}(\log x)\log\N\kq.
\end{align*}
In light of the hypothesis $\N\kq\leq (\log x)^A$ and our choice of $y$, it follows that
\[
\sum_{\substack{\N\ka\leq x \\ \ka\in\mathcal{C}}}\Lambda_{\pi\times\pi'}(\ka)=\frac{1}{|\mathrm{Cl}(\kq)|}\sum_{\chi\in\mathscr{P}(\kq)}\overline{\chi}(\mathcal{C})\sum_{\ka}\Lambda_{\pi\times(\pi'\otimes\chi)}(\ka)\phi(\N\ka)+O_{\pi,\pi',A}(y).
\]

Recall the definition of $\mathcal{E}_{\pi\times\pi'}(x;\kq,\mathcal{C})$ from \eqref{eqn:Edef}. Applying Mellin inversion, we obtain
\begin{multline*}
\mathcal{E}_{\pi\times\pi'}(x;\kq,\mathcal{C})=\frac{1}{|\mathrm{Cl}(\kq)|}\sum_{\chi\in\mathscr{P}(\kq)}\frac{\overline{\chi}(\mathcal{C})}{2\pi i}\int_{3-i\infty}^{3+i\infty}-\frac{L'}{L}(s,\pi\times(\pi'\otimes\chi))\,\hat\phi(s)\,ds\\
-\frac{1}{|\mathrm{Cl}(\kq)|}\sum_{\chi\in\mathscr{P}(\kq)}\overline{\chi}(\mathcal{C})\mathcal{M}_{\pi\times(\pi'\otimes\chi)}(x)+O_{\pi,\pi',A}(y).
\end{multline*}
Define $\epsilon=1/(3A+2)$. By \eqref{eqn:ZFR}, \cref{lem:Li1}, \eqref{eqn:BH}, and the bound $C(\chi)\ll_{[F:\Q]}\N\kq$, there exists an ineffective constant $\Cl[abcon]{C7777}=\Cr{C7777}(\pi,\pi',\epsilon)>0$ such that on the piecewise smooth parametric curve
\[
\mathscr{C}(t)=1-\Cr{C7777}(\N\kq(|t|+1))^{-\epsilon}+it,\qquad t\in\R,
\]
there holds
\begin{equation}
\label{eqn:ZFR_twisted2}
\frac{L'}{L}(\mathscr{C}(t),\pi\times(\pi'\otimes\chi))\ll_{\pi,\pi',\epsilon}(\N\kq(|t|+1))^{2\epsilon}.
\end{equation}
Note that $\mathscr{C}'(t)\ll_{\pi,\pi',\epsilon} 1$ for $t\neq 0$. We deform the line of integration to $\mathscr{C}$. 

If there exists $u\in\R$ such that $\pi'\otimes\chi=\tilde{\pi}\otimes|\cdot|^{iu}$, then the integrand has a pole at $s=1-iu$ with residue
\[
\hat\phi(1-iu)=\int_0^x r^{-iu}\,dr + \int_x^{x+y}\phi(r)r^{-iu}\,dr=\mathcal{M}_{\pi\times(\pi'\otimes\chi)}(x)+O(y).
\]
If no such $u$ exists, then $L(s,\pi\times(\pi'\otimes\chi))$ is entire, and $\mathcal{M}_{\pi\times(\pi'\otimes\chi)}(x)=0$. Therefore, by the residue theorem,
\begin{align*}
\mathcal{E}_{\pi\times\pi'}(x;\kq,\mathcal{C})&=\frac{1}{|\mathrm{Cl}(\kq)|}\sum_{\chi\in\mathscr{P}(\kq)}\frac{\overline{\chi}(\mathcal{C})}{2\pi i}\int_{\mathscr{C}}-\frac{L'}{L}(s,\pi\times(\pi'\otimes\chi))\,\hat\phi(s)\,ds+O_{\pi,\pi',A}(y)\\
&\ll_{\pi,\pi',A}\max_{\chi\in\mathscr{P}(\kq)}\left|\int_{\mathscr{C}}-\frac{L'}{L}(s,\pi\times(\pi'\otimes\chi))\,\hat\phi(s)\,ds\right|+y.
\end{align*}
Therefore, we bound the integrand via \eqref{eqn:hatphi} and \eqref{eqn:ZFR_twisted2}:
\[
\mathcal{E}_{\pi\times\pi'}(x;\kq,\mathcal{C})
\ll_{\pi,\pi',A}\frac{x^2}{y}\N\kq^{2\epsilon}\int_{-\infty}^{\infty}x^{-\Cr{C7777}(\N\kq(|t|+1))^{-\epsilon}}(|t|+1)^{2\epsilon-2}\,dt+y.
\]
Finally, we express the last integral in terms of the new variable $r=(\N\kq(|t|+1))^{-\epsilon}\log x$, and estimate it in a straightforward fashion, using that $\epsilon\in(0,1/2$):
\[
\mathcal{E}_{\pi\times\pi'}(x;\kq,\mathcal{C})
\ll_{\pi,\pi',A}\frac{x^2}{y}(\log x)^{2-1/\epsilon}\N\kq \int_0^{\N\kq^{-\epsilon}\log x}e^{-\Cr{C7777} r}r^{-3+1/\epsilon}\,dr+y.
\]
Our choices of $y$ and $\epsilon$ and our range of $\N\kq$ ensure that $\mathcal{E}_{\pi\times\pi'}(x;\kq,\mathcal{C})
\ll_{\pi,\pi',A} y$, as desired.

\bibliographystyle{abbrv}
\bibliography{HarcosThornerZFR}

\end{document}