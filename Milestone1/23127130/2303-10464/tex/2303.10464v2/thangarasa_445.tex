% \documentclass{uai2023}
\documentclass[accepted]{uai2023}
\usepackage[american]{babel}

\usepackage{natbib} % has a nice set of citation styles and commands
    \bibliographystyle{plainnat}
    \renewcommand{\bibsection}{\subsubsection*{References}}

\usepackage{hyperref}
\usepackage{xurl}
\usepackage{booktabs}       % professional-quality tables
\usepackage{amsfonts}       % blackboard math symbols
\usepackage{nicefrac}       % compact symbols for 1/2, etc.
\usepackage{microtype}      % microtypography
\usepackage{xcolor}         % colors
\usepackage{multirow}
\usepackage{multicol}
\usepackage{subcaption}
% For theorems and such
\usepackage{siunitx} % for proper typesetting of numbers and units
\usepackage{amsmath}
\usepackage{amssymb}
\usepackage{mathtools}
\usepackage{amsthm}
\usepackage{caption}
\usepackage{enumitem}
\usepackage{pdfpages}

\MakeRobust{\ref}% avoid expanding it when in a textual label

\makeatletter
\newcommand{\labeltext}[2]{%
  \@bsphack \csname phantomsection\endcsname % in case hyperref is used
  \def\@currentlabel{#1}{\label{#2}}%
  \@esphack }
\makeatother

\title{SPDF: Sparse Pre-training and Dense Fine-tuning for Large Language Models}

\author[1]{\href{mailto:vithu@cerebras.net}{Vithursan Thangarasa}{}}
\author[1]{Abhay~Gupta}
\author[1]{William~Marshall}
\author[*]{Tianda~Li}
\author[1]{Kevin~Leong} 
\author[*]{\\Dennis DeCoste}
\author[1]{Sean Lie}
\author[1]{Shreyas Saxena}
% Add affiliations after the authors
\affil[1]{%
    Cerebras Systems Inc.\\
    Sunnyvale, California, USA }
% \affil[2]{% Second Affiliation\\
%     Address\\
%     …
% }
% \affil[3]{% Another Affiliation\\
%     Address\\
%     …
%   }
% \author{% Vithursan Thangarasa$^{\text{1}}$, Abhay Gupta$^{\text{1}}$, William
%   Marshall$^{\text{1}}$,\\
%   \textbf{Tianda Li$^{*}$}, \textbf{Kevin Leong$^{\text{1}}$}, \textbf{Dennis
%   DeCoste}\thanks{Work done while at Cerebras Systems.}\hspace{3pt},
%   \textbf{Sean Lie$^{\text{1}}$}, \textbf{Shreyas Saxena$^{\text{1}}$} \\
%   ${}^{1}$Cerebras Systems, Sunnyvale, California \\
%   ${}^{1}$\texttt{\{vithu, shreyas.saxena\}@cerebras.net}\\
% }
\begin{document}


\maketitle

% 

Over the past few years, there has been a significant amount of research focused on studying the ReLU activation function, with the aim of achieving neural network convergence through over-parametrization. However, recent developments in the field of Large Language Models (LLMs) have sparked interest in the use of exponential activation functions, specifically in the attention mechanism.

Mathematically, we define the neural function $F: \R^{d \times m} \times  \mathbb{R}^d \rightarrow \mathbb{R}$ using an exponential activation function. Given a set of data points with labels $\{(x_1, y_1), (x_2, y_2), \dots, (x_n, y_n)\} \subset \mathbb{R}^d \times \mathbb{R}$ where $n$ denotes the number of the data. Here $F(W(t),x)$ can be expressed as $F(W(t),x) := \sum_{r=1}^m a_r \exp(\langle w_r, x \rangle)$, where $m$ represents the number of neurons, and $w_r(t)$ are weights at time $t$. It's standard in literature that $a_r$ are the fixed weights and it's never changed during the training. We initialize the weights $W(0) \in \mathbb{R}^{d \times m}$ with random Gaussian distributions, such that $w_r(0) \sim \mathcal{N}(0, I_d)$ and initialize $a_r$ from random sign distribution for each $r \in [m]$.

Using the gradient descent algorithm, we can find a weight $W(T)$ such that $\| F(W(T), X) - y \|_2 \leq \epsilon$ holds with probability $1-\delta$, where $\epsilon \in (0,0.1)$ and $m = \Omega(n^{2+o(1)}\log(n/\delta))$. To optimize the over-parametrization bound $m$, we employ several tight analysis techniques from previous studies [Song and Yang arXiv 2019, Munteanu, Omlor, Song and Woodruff ICML 2022]. 

 
 \section{Introduction}
\label{sec:introduction}
% \begin{itemize}
%     % Diffusion of FL
%     \item {\st{Diffusion of FL}}
%     % Security threats to FL
%     \item {\st{Security threats to FL with particular focus on model poisoning}}
%     % Limitations of existing countermeasures
%     \item {\st{Current countermeasures (e.g., KRUM) and their limitations}}
%     % Proposed method and its advantages
%     \item {\st{Intuitive description of the proposed method and its difference (i.e., advantages) w.r.t. state of the art}}
%     % Main contributions
%     \item {\st{Summary of the main contributions of this work}}
%     % Paper's structure and organization
%     \item {\st{Paper's structure and organization}}
% \end{itemize}

% Diffusion of FL
Recently, {\em federated learning} (FL) has emerged as the leading paradigm for training distributed, large-scale, and privacy-preserving machine learning (ML) systems~\cite{mcmahan2017googleai,mcmahan2017aistats}. 
The core idea of FL is to allow multiple edge clients to collaboratively train a shared, global model without disclosing their local private training data.
%Specifically, an FL system consists of a central server and many edge clients; 
A typical FL round involves the following steps: {\em(i)} the server randomly picks some clients and sends them the current, global model; {\em(ii)} each selected client locally trains its model with its own private data; then, it sends the resulting local model to the server;\footnote{Whenever we refer to global/local model, we mean global/local model {\em parameters}.} {\em(iii)} the server updates the global model by computing an \emph{aggregation function}, usually the average (FedAvg), on the local models received from clients.
% \begin{enumerate}
%     \item[{\em(i)}] the server sends the current, global model to the clients and appoints some of them for training;
%     \item[{\em(ii)}] each selected client locally trains its copy of the global model with its own private data; then, it sends the resulting local model back to the server;\footnote{Whenever we refer to global/local model, we mean global/local model {\em parameters}.}
%     \item[{\em(iii)}] the server updates the global model by computing an \emph{aggregation function} on the local models received from clients (by default, the average, also referred to as FedAvg~\cite{mcmahan2017aistats}).
% \end{enumerate}
This process goes on until the global model converges. %(e.g., after a certain number of rounds or other similar stopping criteria).
%\\
% The advantages of FL over the traditional, centralized learning paradigm are undoubtedly clear in terms of flexibility/scalability (clients can join/disconnect from the FL network dynamically), network communications (only model weights\footnote{We will use \textit{parameters} and \textit{weights} interchangeably.} are exchanged between clients and server), and privacy (each client's private training data is kept local at the client's end and not uploaded to the server).
\\
% Security threats to FL
%However, the growing adoption of FL also raises security concerns~\cite{costa2022covert}, particularly about its confidentiality, integrity, and availability.
Although its advantages over standard ML, FL also raises security concerns~\cite{costa2022covert}. %, particularly about its confidentiality, integrity, and availability~\cite{costa2022covert}.
% OLD, LONG VERSION
% Indeed, some work deals with privacy leakage that may expose the local data of some clients~\cite{melis2019sp}. 
% A large body of work, instead, investigates attacks that usually aim to detriment the predictive accuracy of the learned global model. For instance, \emph{data poisoning} attacks achieve this goal by letting an adversary pollute the training set of some corrupt FL clients with maliciously crafted examples~\cite{jagielski2018sp}.
% Similarly, in \emph{model poisoning} the attacker attempts to tweak the global model weights~\cite{bhagoji2019pmlr} by directly perturbing the local model's weights of some infected FL clients before these are sent to the central server for aggregation, usually via so-called Byzantine attacks. 
% It turns out that Byzantine model poisoning attacks severely impact standard FedAvg; therefore, more robust aggregation functions must be designed to make FL systems secure.
Here, we focus on \emph{untargeted model poisoning} attacks~\cite{bhagoji2019pmlr}, where an adversary attempts to tweak the global model weights %\footnote{We will use the terms \textit{parameters} and \textit{weights} interchangeably.} 
by directly perturbing the local model's parameters of some infected clients before these are sent to the central server for aggregation.
In doing so, the adversary aims to jeopardize the global model \textit{indiscriminately} at inference time.
Such model poisoning attacks severely impact standard FedAvg; therefore, more robust aggregation functions must be designed to secure FL systems.
\\
% In this paper, we focus on designing a novel robust aggregation scheme at the server's end to contrast the effect of Byzantine model poisoning attacks.
%
% Current countermeasures and their limitations
%Several countermeasures have been proposed in the literature to combat model poisoning attacks on FL systems.
% Some methods use simple statistics more robust than plain average to smooth the impact of malicious updates (e.g., Trimmed Mean and FedMedian~\cite{yin2018icml}). 
% Other defenses implement outlier detection techniques to discard malicious updates from the aggregation performed at the server's end. Those are either based on heuristics (e.g., Krum/Multi-Krum~\cite{blanchard2017nips} and Bulyan~\cite{mhamdi2018pmlr}) or data-driven approaches (e.g., K-means clustering~\cite{shen2016acm} or DnC via spectral analysis~\cite{shejwalkar2021ndss}). 
% Finally, some strategies rely on a centralized ``source of trust'' to spot potential malicious updates (e.g., FLTrust~\cite{cao2020fltrust}).
% Several countermeasures have been proposed in the literature to combat model poisoning attacks on FL systems, i.e., to discard possible malicious local updates from the aggregation performed at the server's end. 
% These techniques range from simple statistics more robust than plain average (e.g., Trimmed Mean and FedMedian~\cite{yin2018icml}) to outlier detection heuristics (e.g., Krum/Multi-Krum~\cite{blanchard2017nips} and Bulyan~\cite{mhamdi2018pmlr}) or data-driven approaches (e.g., spectral analysis via K-means clustering~\cite{shen2016acm} or spectral analysis), or methods based on ``source of trust'' (e.g., FLTrust~\cite{cao2020fltrust}).
% OLD, LONG VERSION
%Several countermeasures have been proposed in the literature to combat Byzantine model poisoning attacks on FL systems.
% Descriptive statistics
% For example, Trimmed Mean and FedMedian aggregate local model updates using more robust statistics than standard average~\cite{yin2018icml}.
%
% % Heuristics for outlier detection
% Many existing Byzantine-resilient strategies implement some outlier detection heuristics to discard the model updates sent by potentially malicious clients from the input of the aggregation function.
% One of the most popular heuristics is Krum~\cite{blanchard2017nips}.
% This strategy tries to mitigate the impact of Byzantine attacks by selecting as a global model the local model with the smallest sum of Euclidean distances to {\em all} the other local models.
% Although powerful, Krum requires the server to know (or, at least, estimate) the number of malicious FL clients upfront, which is generally impossible in a realistic attack scenario. %
% Moreover, Krum may become ineffective for complex, high-dimensional model parameter spaces due to the curse of dimensionality.
% Bulyan~\cite{mhamdi2018pmlr} tries to overcome this issue by combining Krum with a variant of Trimmed Mean.
% % Data-driven outlier detection
% Other strategies use data-driven outlier detection techniques -- e.g., via K-means clustering~\cite{shen2016acm} -- to spot potential malicious local model updates. 
% %For instance, Shen et al. propose to cluster local model updates with K-means and thus identify outliers.
%
% % Other techniques
% As far as the server is concerned, any local model received can be from a potential malicious client. 
% FLTrust~\cite{cao2020fltrust} assumes the server acts as a client, i.e., trains a local model on an additional {\em trustworthy} dataset at the server's end and compares it against all the local models from other clients. 
% This way, the server can rely on some ``source of trust'' when discarding potentially malicious clients.
%\\
% Limitations of existing Byzantine-resilient strategies
Unfortunately, existing defense mechanisms either rely on simple heuristics (e.g., Trimmed Mean and FedMedian by~\cite{yin2018icml}) or need strong and unrealistic assumptions to work effectively (e.g., foreknowledge or estimation of the number of malicious clients in the FL system, as for Krum/Multi-Krum~\cite{blanchard2017nips} and Bulyan~\cite{mhamdi2018pmlr}, which, however, cannot exceed a fixed threshold).
Furthermore, outlier detection methods using K-means clustering~\cite{shen2016acm} or spectral analysis like DnC~\cite{shejwalkar2021ndss} do not directly consider the temporal evolution of local model updates received.
Finally, strategies like FLTrust~\cite{cao2020fltrust} require the server to collect its own dataset and act as a proper client, thereby altering the standard FL protocol.
\\
% OLD, LONG VERSION
% Overall, existing Byzantine-resilient strategies are either simple heuristics (e.g., FedMedian) or, if they are more complex, they rely on strong and unrealistic assumptions to work effectively (e.g., knowing the number of malicious clients in the FL system in advance, as for Krum and alike).
% Furthermore, data-driven outlier detection methods do not consider the temporary evolution of local model updates received (e.g., K-means clustering). 
% Finally, strategies like FLTrust requires the server to collect its own dataset and act as a proper client, thereby altering the standard FL protocol.
%
% Description of the proposed method
This work introduces a novel pre-aggregation \textit{filter} robust to untargeted model poisoning attacks. Notably, this filter $(i)$ operates without requiring prior knowledge or constraints on the number of malicious clients and $(ii)$ inherently integrates temporal dependencies. 
The FL server can employ this filter as a preprocessing step before applying \textit{any} aggregation function, be it standard like FedAvg or robust like Krum or Bulyan.
Specifically, we formulate the problem of identifying corrupted updates as a multidimensional (i.e., matrix-valued) time series anomaly detection task. 
The key idea is that legitimate local updates, resulting from well-calibrated iterative procedures like stochastic gradient descent (SGD) with an appropriate learning rate, show \textit{higher predictability} compared to malicious updates. This hypothesis stems from the fact that the sequence of gradients (thus, model parameters) observed during legitimate training exhibit regular patterns, as validated in Section~\ref{subsec:intuition}. %until convergence. 
%This regularity may be more pronounced for smooth convex loss functions, but it can still be captured within an appropriate time window, even for more complex and convoluted loss surfaces. 
%We provide evidence of this claim in Appendix~B, where we show that the average mutual information (i.e., ``predictability''), calculated over pairs of legitimate model updates sent at different FL rounds, is significantly higher than the corresponding computation for a malicious client.
\\
Inspired by the matrix autoregressive (MAR) framework for multidimensional time series forecasting~\cite{chen2021je}, we propose the FLANDERS ({\em \textbf{F}ederated \textbf{L}earning meets \textbf{AN}omaly \textbf{DE}tection for a \textbf{R}obust and \textbf{S}ecure}) filter.
The main advantages of FLANDERS over existing strategies like FLDetector~\cite{zhao2020multivariate} are its resilience to large-scale attacks, where $50\%$ or more FL participants are hostile, and the capability of working under realistic non-iid scenarios.
We attribute such a capability to two key factors: $(i)$ FLANDERS works without knowing a priori the ratio of corrupted clients, and $(ii)$ it embodies temporal dependencies between intra- and inter-client updates, quickly recognizing local model drifts caused by evil players. Below, we summarize our main contributions:

\begin{itemize}
\item[{\em(i)}]
We provide empirical evidence that the sequence of models sent by legitimate clients is more predictable than those of malicious participants performing untargeted model poisoning attacks.
\\
\item[{\em(ii)}] 
We introduce FLANDERS, the first pre-aggregation filter for FL robust to untargeted model poisoning based on multidimensional time series anomaly detection.
\\
\item[{\em(iii)}] 
We integrate FLANDERS into Flower,\footnote{\scriptsize{\url{https://flower.dev/}}} a popular FL simulation framework for reproducibility.
\\
\item[{\em(iv)}] 
We show that FLANDERS improves the robustness of the existing aggregation methods under multiple settings: different datasets, client's data distribution (non-iid), models, and attack scenarios.
\\
\item[{\em(v)}] 
We publicly release all the implementation code of FLANDERS along with our experiments.\footnote{\scriptsize{\url{https://anonymous.4open.science/r/flanders_exp-7EEB}}}
\end{itemize}

% Paper's structure and organization
The remainder of the paper is structured as follows. %some related work and the current state-of-the-art solutions to security issues that FL entails. 
Section~\ref{sec:background} covers background and preliminaries. 
In Section~\ref{sec:related}, we discuss related work.
Section~\ref{sec:problem} and Section~\ref{sec:method} describe the problem formulation and the method proposed. % to tackle it. 
Section~\ref{sec:experiments} gathers experimental results. %, and Section~\ref{sec:limitations} discusses some limitations of this work.
Finally, we conclude in Section~\ref{sec:conclusion}.
 %discusses the limitations of this work and draws future research directions.
%reports conclusions and draws perspectives for future research directions.

%%%%%%% OLD %%%%%%%
%to overcome the resilience of Byzantine failures in distributed Stochastic Gradient Descent computations. 
% The strength of Krum is its time complexity, which is linear in the gradient dimension. 
% However, the robustness of the approach is guaranteed for gradient-based learning applications only when the majority of the clients are not compromised. 
% Besides, the aggregation mechanism of Krum, as well as that of similar methods, is robust from a coarse-grained perspective and does not provide solutions to errors and perturbations that may occur at inference time.
%A related approach to~\cite{blanchard2017nips} is the work of Su et al.~\cite{su2016dc}. Here, the authors propose an iterated approximate agreement to tackle a multi-layer scenario attacked by Byzantine agents. 
%However, the method works efficiently on the sole discrete context and it is inapplicable to continuous state environments.
%\gabri{Maybe, we should just talk about the main limitations of existing countermeasures without digging into their details (or, we can just mention Krum as this is the most popular one). I will move the description of all these methods to the Related Work section.}
% \section{Method}
\label{sec:method}

% \ml{``Inconsistent'' to ``large variation''}

% In this section, we propose our methods based on the observations in Section \ref{sec:motivation}.
In this section, we propose two techniques to further enhance the strong baseline to capture the variation of activation distributions better.
We first introduce spatial re-scaling to adapt the network to pixel-to-pixel variation.
We then propose channel-wise shifting and re-scaling to better capture the channel-to-channel variation.
Meanwhile, as both of the two methods are image-dependent, the image-to-image variation can be captured naturally.
By combining the two methods with our strong baseline, we build our enhanced BNN for SR, named EBSR.

% Because the activation distributions among pixels, channels and images have large variations \red{**are highly inconsistent} in SR networks, we introduce spatial re-scaling to adapt to pixel-wise variations and channel shift and re-scaling to adapt to channel-wise variations. And both of them are image-dependent to adapt to image-wise variations, which means during inference our network re-scales and shifts the distributions of activations flexibly for different input images. Based on these methods, we build an enhanced binary neural network for image super-resolution (EBSR).

% According to [3], the difference of activation magnitudes indicates different scaling factors are needed for each pixel.

\subsection{Spatial Re-scaling}
% It is better to use different scaling factors for different pixels to reduce the quantization error and retain more detailed information for image super-resolution. 

% \ml{In the main method, we do not need to introduce the previous works but can focus on introducing our own method. Channel rescaling in Real-to-binary Net is not relevant in this context.}

% Re-scaling the output of binary convolutions was proposed at the birth of BNN in XNOR-Net \cite{rastegari2016xnor} to reduce quantization error and improve accuracy for image classification tasks.
% It is computed as below:
% \begin{equation}
% \mathcal{A} * \mathcal{W} \approx(\operatorname{sign}(\mathcal{A}) \circledast \operatorname{sign}(\mathcal{W})) \odot \mathcal{K} \alpha
% \label{eq:xnor-net rescale}
% \end{equation}
% where $\circledast$ denotes the binary convolution and $\odot$ denotes the element-wise multiplication.
% $\mathcal{A}$, $\mathcal{W}$, $\alpha$, and $\mathcal{K}$ denote the activation, weight, weight scaling factor, and activation scaling factor, respectively.
%  Later in XNOR-Net++ \cite{bulat2019xnor}, Bulat et al. fuse the activation and weight scaling factors into a single one that is learned end-to-end based on gradients and this improves the classification accuracy on ImageNet dataset.

% % It is computed as Eq.~\ref{eq:xnor-net rescale}, where $\circledast$ denotes 
% %  the binary convolution and $\odot$ denotes the element-wise multiplication. The binary convolution of $\mathcal{A}$ and $\mathcal{W}$ is rescaled by the weight scaling factor $\alpha$ and the activation scaling factor $\mathcal{K}$, both of which are calculated analytically.


% \zc{Similarly, you should explain the meaning of A, W and the operators $\circledast$ in the formula}
% Then in Real-to-binary Net \cite{martinez2020training}, Martinez et al. used a data-driven channel re-scaling module that takes the pre-convolution activations as input to predict the activation scaling factor. Unlike that in XNOR-Net++ \cite{bulat2019xnor}, these scaling factors are not fixed during inference but rather inferred from data. By doing this, they further improved the classification accuracy on ImageNet over XNOR-Net++. 
As is shown in Figure \ref{fig:pixel}, activation distributions have large pixel-to-pixel variation in SR networks
and the difference of activation magnitudes indicates different scaling factors are preferred for different pixels.
Inspired by \cite{martinez2020training}, we propose spatial re-scaling to better adapt the network to the spatial variation
of activation distributions in SR networks.
% fit the various pixel-wise distributions in SR networks.
We take the real-valued activations $A$ before convolution as input and predict pixel-wise scaling factors $S(A)$, which re-scale the binary convolution output. Spatial re-scaling process can be formulated as follows:
\begin{equation}
A * W \approx(\operatorname{sign}(A) \circledast \operatorname{sign}(W)) \odot \alpha \odot S(A)
\label{eq:spatial rescale}
\end{equation}
where $\circledast$ denotes 
the binary convolution and $\odot$ denotes the element-wise multiplication. $A$, $W$, $\alpha$, and $S\left(A\right)$ denote real-valued activations, weights, the scaling factor of weights, and the spatial-wise scaling factor of activations respectively. $S\left(A\right) \in \mathbb{R}^{1\times H\times W}$ can be calculated with a convolution and a sigmoid function.
% as $\sigma\left( CONV\left(A\right)\right)$. 
As shown in Figure \ref{fig:method}(a), real-valued activations first go through a convolution layer,
which has an input channel of $C$ and an output channel of 1, 
and then pass through a sigmoid function to produce the scaling factors $S(A)$ along the spatial dimension.
During inference, the scaling factor will change dynamically according to different input feature maps.
By re-scaling binary convolution output using $S(A)$, we can reduce the quantization error and the original pixel-wise information in FP activation
will be preserved much better.
Spatial re-scaling leads to a large PSNR improvement of 0.24 dB (from 30.30 dB to 31.54 dB) on Set5 and 0.22 dB (from 25.09 dB to 25.31 dB)
on Urban100 compared with our strong baseline. 

\subsection{Channel-wise Shifting and Re-scaling}

\begin{table}[!tb]
\centering
\caption{Comparison between whether to fuse channel-wise shifting and re-scaling or not based on our baseline with spatial re-scaling. }
\label{tab:fusing}

\scalebox{0.65}{
\begin{tabular}{c|cc|cc|cc}
\hline
\multirow{2}{*}{Method}     & \multirow{2}{*}{OPs} & \multirow{2}{*}{Params} & \multicolumn{2}{c|}{Set5} & \multicolumn{2}{c}{Urban100} \\ \cline{4-7} 
                            &                      &                         & PSNR        & SSIM        & PSNR          & SSIM         \\ \hline
Baseline + spatial re-scale & 2.16G                & 0.05M                   & 31.54       & 0.883       & 25.31         & 0.759        \\
+ channel-wise shift and re-scale             & 2.34G                & 0.09M                   & 31.61       & 0.885       & 25.35         & 0.761        \\
+ w/ fusing                   & 2.27G                & 0.08M                   & \textbf{31.64}       & \textbf{0.885}       & \textbf{25.36}         & \textbf{0.761}        \\ \hline
\end{tabular}
}
\end{table}

In SR networks, activation distributions exhibit larger channel-to-channel variation (Figure \ref{fig:chl}).
Both the mean and magnitude of the activation distributions vary significantly across channels.
% Thus we use channel-wise shifting and re-scaling to adapt to various channel-wise distributions. 
\cite{martinez2020training} has proposed the data-driven channel re-scaling, 
but our method differs from them in further introducing data-driven thresholds to handle the channel-wise variation of both mean and magnitude.
Since the blocks to generate the scaling factors and thresholds are very similar, we further propose to fuse them into one module.
% and fusing channel-wise shifting and re-scaling into one module.
We evaluate the effect of fusing the two blocks in Table \ref{tab:fusing}.
With channel-wise shifting and re-scaling fused, our models have fewer operations and parameters overhead and slightly higher performance.

For the specific process, we take the real-valued activations as input and predict different thresholds and scaling factors for each channel. They are also image dependent, e.g., $\beta_{i}$ in Eq.\ref{eq:act_binarize} is no longer fixed during inference but generated according to different input feature maps. Channel-wise shifting and re-scaling can be formulated as follows:
\begin{equation}
A * W \approx(\operatorname{sign}(A-C_s(A)) \circledast \operatorname{sign}(W)) \odot \alpha \odot C_r(A)
\label{eq:channel-wise_shift_and_rescale}
\end{equation}
where $\circledast$ denotes 
the binary convolution and $\odot$ denotes the element-wise multiplication. $C_s(A), C_r(A) \in \mathbb{R}^{C\times1\times1}$ denote the channel-wise threshold and scaling factor, respectively. 
We show the block diagram in Figure \ref{fig:method}(b).
The real-valued input feature map is first squeezed to a ${C\times1\times1}$ vector by a global average pooling (GAP) layer.
The subsequent fully connected layers and ReLU learn the channel-wise information and output a ${2C\times1\times1}$ vector.
Then the ${2C\times1\times1}$ vector is split into two ${C\times1\times1}$ vectors.
We use the first $C$ channels as the channel-wise bias and pass the last $C$ channels through a sigmoid layer 
as the channel-wise scaling factor, which are used to shift the real-valued activations and re-scale the binary convolution output, respectively. 


% \ml{We can mention previously, channel-wise re-scale has been proposed. We propose to fuse them. Add the comparison between fuse v.s. no fuse.}

\begin{figure}[!tbp]%
  \centering
    \includegraphics[width=0.4\textwidth]{fig/methods.png}
  
% \subfloat[channel-wise shifting\&re-scale]{
%     \label{subfig:channel-wise shifting and re-scale}
%     \includegraphics[width=0.2\textwidth]{fig/chl shift and rescale.png}
%   }

  \caption{Block diagram for spatial re-scaling, and channel-wise shifting and re-scaling.} 
  % Input A is the real-valued activation tensor and C, H, and W denote its dimension. GAP stands for global average pooling. The reduction ratio r is set to 16 for a better trade-off between the performance and the number of operations and parameters.}
  \label{fig:method}
\end{figure}


\subsection{Network Structure}

Combining the spatial re-scaling and the channel-wise shifting and re-scaling methods, we construct the enhanced convolution layer (E-Conv).
Then we build our EBSR model based on E-Conv.
In Figure \ref{fig:E-conv}, we compare the binary convolution layer used in the baseline network and our proposed E-Conv.
We use spatial and channel-wise scaling factors to re-scale the binary convolution output,
and use channel-wise shifting to learn appropriate thresholds for each channel before binarization.
The scaling factors and threshold used in E-Conv are learnable and depend on the real-valued input activations.
In this way, our proposed EBSR can adapt to pixel-to-pixel, channel-to-channel, and image-to-image variations
to reduce the large binarization error and preserve more details.
% In this way, our proposed E-Conv reduces the large quantization error caused by binarization and keeps the original information of input feature maps to a large extent.


\begin{figure}[!tb]%
  \centering

    \includegraphics[width=0.5\textwidth]{fig/E-conv.png}

  \caption{Comparison of (a) the binary convolution layer with a skip connection used in our baseline network and (b) the proposed E-Conv.}
  \label{fig:E-conv}
\end{figure}


Figure \ref{fig:network} shows the basic block based on the E-Conv and our EBSR composed of the basic blocks. Following existing works, the convolution layers in the head and tail modules are not binarized. We choose the lightweight EDSR which has 16 basic blocks and 64 channels, and EDSR which has 32 basic blocks and 256 channels as our backbones, which correspond to EBSR-light and EBSR, respectively.

\begin{figure}[!tb]%
  \centering
  {
    \includegraphics[width=0.35\textwidth]{fig/network.png}
  }
  
  \caption{The structure of our proposed EBSR.  Convolution layers in purple are real-valued vanilla 3x3 convolutions.}
  \label{fig:network}
\end{figure} \section{Experimental Results}
\label{sec:experiments}
\subsection{Training Details}
\cite{Kalantari2017DeepHD} provides the first dataset specifically designed for multi-exposure HDR fusion under large motion. It consists of 74 training sets, which we use to supervise the training of our model. We crop the input images to patches of size \(256 \times 256\) at a step size of 64. This totally generates 20128 training samples. To augment training samples, we randomly rotate and flip the training images. The training adopts Adam optimizer. The learning rate is initialized to \(10^{-4}\) and is reduced to \(10^{-5}\) after 20 epochs. It is observed that 40 epochs are sufficient for the training to converge.    

\subsection{Numerical Evaluation}
We numerically measure the performance of our method on the 15 test sets of \cite{Kalantari2017DeepHD}, by Peak Signal-to-Noise Ratio (PSNR) and Structure Similarity, computed in both tonemapping domain (-\(\mu\)) and HDR linear domain (-L). Visual difference metric HDR-VDP-2 is also adopted, where the parameters are set as same as in previous works \cite{wu2018end} and \cite{niu2021hdrgan}. 

Table \ref{table_metrics} compares our model with state-of-the-art models. For \cite{yan2020nonlocal} and \cite{xiong2021hierarchical}, we use the results reported in the publications. Note that \cite{sen2012robust} and \cite{hu2013hdr} are not machine learning based methods. Moreover,  \cite{Kalantari2017DeepHD} and \cite{wu2018end} apply optical flow and homography transformation to preprocess the input images respectively, and hence entail extra computation. 

Table \ref{table_metrics} shows that our method outperforms competing method in terms of PSNR-L, SSIM-$\mu$, SSIM-L and HDR-VDP-2. It ranks the second best in PSNR-$\mu$, being slightly (0.1dB) inferior to \cite{xiong2021hierarchical}. Note that \cite{xiong2021hierarchical} utilizes a pretrained model to detect ghosting regions for training, whereas our method does not require any pretrained model. The high PSNR and SSIM scores varify that our model has strong HDR reconstruction ability and can accurately restore the radiance and structure of the scene in both tonemapping domain and HDR linear domain. Furthermore, its high performance in term of HDR-VDP-2\cite{mantiuk2011hdr} performance indicates that our method can generate HDR image visually close to the target image.

\begin{table*}[ht]
\centering
\begin{tabular}{l|c|c|c|c|c}
\hline
& PSNR-$\mu$ & PSNR-L & SSIM-$\mu$ & SSIM-L & HDR-VDP-2 \\
\hline
\bfseries Sen & 40.97 & 38.36 & 0.9830 & 0.9746 & 60.60\\
\hline
\bfseries Hu  & 35.65 & 30.80 & 0.9725 & 0.9491 & 58.34\\
\hline
\bfseries Kalantari & 42.69 & 41.22 & 0.9888 & 0.9845 & 65.05\\
\hline
\bfseries DeepHDR& 41.99 & 41.22 & 0.9878 & 0.9859 & \underline{65.91}\\
\hline
\bfseries AHDR & 43.62 & 41.03 & 0.9900  &\underline{0.9883} & 63.85 \\
\hline 
\bfseries NHDRRNet& 42.414 & - & 0.9887 & - & 61.21 \\
\hline 
\bfseries HDR-GAN &43.92 & \underline{41.57} &\underline{0.9905} &0.9865 & 65.45\\
\hline 
\bfseries HFNet & \textbf{44.28} & 41.47 & - & - & - \\
\hline 
\bfseries Ours & \underline{44.18} & \textbf{42.19}&\textbf{0.9912} & \textbf{0.9883}& \textbf{67.07} \\
\hline
\end{tabular}
\caption{Numerical performance of the proposed model, evaluated on the dataset by Kalantari-Ramamoorthi. The best and second best results for each metric are marked in \textbf{bold} and \underline{underlined}, respectively}
\label{table_metrics}
\end{table*}

\subsection{Visual Performance Evaluation}

\begin{figure*}[!htb]
\centering
\includegraphics[width=\textwidth]{experiments/kalantari_test.png}
\caption{Visual comparison on the test set of Kalantari-Ramamoorthi dataset. Zoom-in views of reconstruction by each method are presented on the saturated regions that contain moving objects. Our network built with gated Swin Transformer yields noticeably better visual results than other methods.}
\label{fig_kalantari_test}
\end{figure*}
Fig. \ref{fig_kalantari_test} present the visual performance of our method and comparable methods on two examples from \cite{Kalantari2017DeepHD}. We present the zoom-in views of two challenging cases, where large saturated regions contain substantial non-rigid motion in the reference image. The two patch-based methods do not reconstruct the missing details in the saturated regions, as they heavily rely on the details provided by the reference image for registration. Image reconstructed by the optical flow based method \cite{Kalantari2017DeepHD} suffers motion blur artifacts. This is because the convolutions of DeepHDR and HDR-GAN have limited receptive fields, and hence are hampered to repair missing content in misaligned regions by aligned regions. The gating mechanism of AHDR is only applied to low-level features, so the high-level outliers may deteriorate the HDR fusion. In contrast to comparable methods, our model remarkably overcomes the ghosting artifacts.

\begin{figure}[ht]
\centering
\includegraphics[width=\columnwidth]{experiments/sen_test.pdf}
\caption{Visual performance comparison on example images from the dataset by Sen et al. Zoom in views on challenging areas are presented. Although the ground truth is unavailable, it can be clearly observed that our method visually performs better than comparable methods.}
\label{sen_test}
\end{figure}

\begin{figure}[ht]
\centering
\includegraphics[width=\columnwidth]{experiments/tursun_test.pdf}
\caption{Visual performance comparison on example images from the dataset by Tursun et al. Compared to state of the art methods, our method suffers less ghosting artifact.}
\label{tursun_test}
\end{figure}

Fig.\ref{sen_test} and Fig.\ref{tursun_test} present visual performance of our method on two examples from benchmark datasets \cite{sen2012robust} and \cite{tursun2016objective}. As these test datasets   do not provide ground truth image. we mark the visual difference on the results generated by different methods. It can be seen that our method suffers less artifacts than other methods in various scenes with various motion patterns, achieving better visual results. Our method creates high-quality HDR more robustly and generalizes well. 

\subsection{Ablation Study}

\begin{table}[h]
\centering
\resizebox{\columnwidth}{!}{
\begin{tabular}{l|c|c|c|c|c}
\hline
                         & PSNR-$\mu$ & PSNR-l & SSIM-$\mu$ & SSIM-l & HDR-VDP-2 \\ \hline
restormer(w/o ssim loss) & 44.00  & 41.5   & 0.9906 & 0.9873 & 64.72  \\ \hline
Ours(w/o ssim loss)      & 44.07  & 41.83  & 0.9909 & 0.9879 &  64.78  \\ \hline
Ours                     & 44.18  & 42.19  & 0.9912 & 0.9883 & 67.07      \\ \hline
\end{tabular}
}
\caption{Experimental results of ablation study. We compare using Gated Swin Transformer v.s. Gated Transformer, and the combined loss function v.s. the traditional $l_{1}$ norm loss function.}
\label{table_ablation_block_loss}
\end{table}

We verify various components of our method, including Swin Transformer, loss function, and gating mechanism by ablation study.

\subsubsection{Ablation Study on Block Design}
Our model has similar architecture to Restormer, which uses modified Transformer, whereas we use modified Swin Transformer as the building unit. For comparison, we replace the residual modules in each block in our model with multiple transformer layers as in Restormer, with same number of transformer layers. Table \ref{table_ablation_block_loss} presents the results, which show that using Swin Transformer achieves superior performance in all measures. The reason is that the attention module of Restormer is computed channel-wise, but forgoes the cross-exposure spatial dependency to repair the non-aligned area. 

\subsubsection{Ablation Study on Loss Function}
We trained our model under different loss function configurations, as shown in \ref{table_ablation_block_loss}. The results validate that the SSIM loss benefits detail reconstruction.

\subsubsection{Ablation Study on Gating Mechanism}
\begin{table}[h]
\resizebox{\columnwidth}{!}{
\begin{tabular}{l|c|c|c|c|c}
\hline
           & PSNR-$\mu$ & PSNR-l & SSIM-$\mu$ & SSIM-l & HDR-VDP-2 \\ \hline
w/o gating & 43.14  & 41.03  & 0.9904 & 0.9868 &     64.88      \\ \hline
one gating & 43.44  & 41.42  & 0.9909 & 0.9882 &    67.13   \\ \hline
Ours       & 43.61  & 41.74  & 0.9909 & 0.9881 & 66.96     \\ \hline
\end{tabular}
}
\caption{Ablation experimental results to verify the effectiveness of the gating mechanism}
\label{table_ablation_gating}
\end{table}

The gating mechanism is an important component in our model. Ablation study is conducted in the gating mechanism as follows.

\textbf{w/o gating}: The gating mechanism is not used in the feed forward network of all transformer layers in the model, that it, our GST unit degenerate to the vanilla Swin Transformer.

\textbf{one gating}: The gating mechanism is only used in the first Swin Transformer layers subsequent to the embedding layer, but not used for other layers. 

 Table \ref{table_ablation_gating} shows the results of the ablation experiments, where the model is trained for 20 epochs. By removing the gating mechanism, the network relies on self-attention for image alignment, resulting in the lowest performance. On top of it, adding gates to low level layers notably improves the HDR reconstruction. Furthermore, by integrating the gating mechanism with all Swin Transformer layers, the model effectively inpaints information in non-aligned regions and obtains the highest HDR reconstruction results, thus validates the effectiveness of the gating mechanism in our model.

% \section{Related work}
% There is extensive recent work on speeding up analytical queries due to the need for consistent execution times in the face of the explosive growth in the volume of available data.
% In this section, we divide existing work into two categories: maintaining data freshness in MVs (\Cref{sec:server_side}) and optimizations for minimizing ad-hoc query latency (\Cref{sec:client_side}).

% \subsection{Maintaining Data Freshness in MVs}
% \label{sec:server_side}
% There exists a variety of data warehousing applications aimed at supporting low-latency analytical queries on fresh data.
% In particular, these applications require efficiency in the propagation of newly ingested data into downstream MVs.
 
\mypara{Efficient MV Refresh}
Incremental view maintenance (IVM) aims to update MVs to reflect newly ingested data, taking advantage of already computed results to perform the update in a manner more efficient than computing from scratch (full refresh)
~\cite{ahmad2012dbtoaster,mcsherry2013differential,armbrust2013generalized,zeng2016iolap, palpanas2002incremental, griffin1995incremental, agiwal2021napa, braun2015analytics}. 
There is an abundance of work in IVM, including incremental updates on duplicate values~\cite{griffin1995incremental}, non-distributive aggregate functions~\cite{palpanas2002incremental}, higher-order views~\cite{ahmad2012dbtoaster}, and sliding windows~\cite{braun2015analytics}. 
More recent works also investigate the scalability aspect of IVM, proposing scale-independent updates~\cite{armbrust2013generalized} and sampled views~\cite{zeng2016iolap}. Since \system is applicable to arbitrary SQL statements, \system is orthogonal to and is fully compatible with existing IVM techniques.

\mypara{MV Refresh Scheduling}
There exist works on scheduling the refresh of a MV set focusing on resolving cyclic dependencies~\cite{folkert2005optimizing}, minimizing weighted average staleness~\cite{golab2009scheduling}, and prioritizing MVs with the highest speedups on predicted future queries~\cite{ahmed2020automated}.
\system's scheduling to speed up the end-to-end refresh of the MV set is not addressed in existing works.

\mypara{DAG Workflow Scheduling}
The execution of workloads consisting of individual jobs with acyclic dependencies is a well-studied topic~\cite{apacheoozie,sparkdag,marchal2018parallel,bathie2020revisiting,baruah2022ilp}; many of these techniques can be applied to MV refresh runs studied in this paper.
Existing workflow scheduling systems such as Apache Oozie~\cite{apacheoozie}, Apache Airflow~\cite{airflow}, and Spark DAG scheduler~\cite{sparkdag} automate the execution of user-defined workflows following a topological order.
There are recent works aimed at finding more optimal execution orders in terms of peak memory usage~\cite{marchal2018parallel, bathie2020revisiting} and execution time on parallel platforms~\cite{baruah2022ilp}.
While \system is designed for use with MV refresh runs/workloads, our technique on joint scheduling and optimization can be reasonably applied to general workloads as a possible future direction.

% \paragraph{Incremental MV indexing}
% Update-optimized indices such as the log-structured merge-trees (LSM)~\cite{o1996log} are used for indexing MVs due to frequent updates induced by data ingestion~\cite{gupta2016mesa,agiwal2021napa}.
% \system is orthogonal to indexing: \system is capable of efficiently performing MV refresh runs regardless of whether the individual MVs are indexed or not.

% \subsection{Ad-hoc Query Latency Reduction}
% \label{sec:client_side}

% The minimization of ad-hoc analytical query response times is a well-studied topic due to latency being negatively correlated with the productivity of a data analyst during a data analysis session~\cite{liu2014effects}.
% Sessions are commonly conducted within visualization systems that contain a variety of optimization techniques to ensure that query response times fall within a certain latency tolerance.

% \mypara{Data prefetching}
% Data is often loaded into memory on a by-need basis in visualization systems to minimize interference with user-issued query computations~\cite{mani2017effective,xin2021enhancing,galakatos2017revisiting, yan2020auto, battle2016dynamic, crotty2016case, jalaparti2018netco}. 
% Query-time data retrieval can be significantly expedited by anticipating the data usage of the user in future queries and pre-loading the data into memory during the downtime between user queries (`think time'). SMART~\cite{mani2017effective} prefetches data for modified versions of current user-issued queries with different filters and dimensions. A-WARE~\cite{crotty2016case} maintains a sample store constantly refined through ingesting data based on speculations of future plots.
% ForeCache~\cite{battle2016dynamic} uses an SVM to predict the user's current analysis phase and accordingly prefetches data tiles partitioned based on different numerical values. NetCo predicts future queries via log analysis, and solves an ILP formulation to prefetch data to maximize the number of SLO-meeting queries~\cite{jalaparti2018netco}.
% In the case of MV refresh workloads, `think time' is nonexistent as individual MVs are refreshed back-to-back, rendering data prefetching techniques non-applicable.

\mypara{Intermediate Data Caching}
Some existing data visualization systems cache user-defined variables to support the typical incremental construction of data visualizations~\cite{zgraggen2016progressive, eichmann2020idebench} during data analysis sessions~\cite{jupyter, rstudio, colab}. 
Recent work proposes a management scheme for these cached variables under a memory constraint that greedily keeps variables with the highest estimated time savings based on predicted future user behavior via neural networks~\cite{xin2021enhancing}.
While useful for data visualization, a greedy approach to memory management fails to achieve satisfactory results compared to \system.

\mypara{Intermediate Result Reuse}

There exist works on storing intermediate results from computations to speedup future computations~\cite{yang2018intermediate, dursun2017revisiting, nagel2013recycling, michiardi2019memory, galakatos2017revisiting}.
Studied topics include the identification of reuse opportunities by finding overlaps in computation graphs of successive jobs~\cite{yang2018intermediate, michiardi2019memory},
selective storage under a space constraint with heuristics such as reuse probability~\cite{dursun2017revisiting}, expected savings~\cite{yang2018intermediate}, and recompute-storage cost difference~\cite{nagel2013recycling},
and rewriting incoming jobs to take advantage of stored intermediates~\cite{galakatos2017revisiting}.
These works share similarity with \system in their selection of items to store under a memory constraint, however, \system's problem setting requires it to uniquely consider the joint (re)ordering of job executions along with the selection of items.

% work that considers both job execution (re)order as well as intermediate result caching with a bounded amount of memory. but notably lack the joint aspect of \system and cannot be used to achieve immediate speedup on an incoming MV refresh run if no intermediates are stored beforehand. 

\mypara{Incremental Query Processing} Incremental processing (IQP) is useful for cases where not all data required for a query is immediately available. Similar to online aggregation~\cite{hellerstein1997online}, initial results of a query are computed on a subset of required data and progressively refined as the rest of the required data arrives in a predictable pattern~\cite{tang2019intermittent,wangtempura}. Tang et al. propose a dynamic programming formulation to pick intermediate states to store in memory given a limited memory budget~\cite{tang2019intermittent}. Tempura rewrites the query plan for more efficient execution based on predicted data arrival patterns~\cite{wangtempura}. While similarities exist between the problem setting of IQP and \system, such as management of bounded memory, \system notably includes additional joint optimization for the order of MV updates.

% \paragraph{Sampling}
% Sampling has seen wide use in visualization systems for reducing the computation time of ad-hoc queries by computing an approximate result over a subset of data as exact results are not always required by the user~\cite{crotty2016case, mani2017effective, zgraggen2014panoramicdata, kraska2021northstar, galakatos2017revisiting, kandula2016quickr}. 
% Commonly studied topics in sampling for ad-hoc queries include complex query sampling~\cite{kandula2016quickr}, rare event aggregation~\cite{kraska2021northstar, galakatos2017revisiting}, and maintaining consistency between related sampled visualizations~\cite{zgraggen2014panoramicdata}.
% Sampling server-side at the MV level compromises the assumptions of downstream applications and is thus not considered in \system.

% \paragraph{Progressive visualization}
% The latency tolerance for time-consuming queries can be circumvented by presenting a partially-computed visualization to the user within the tolerance, which is then incrementally refined until it is fully accurate~\cite{rahman2017ve, zgraggen2016progressive, crotty2015vizdom, kraska2021northstar, kamat2017infiniviz}.
% Example plots which benefit from progressive visualization include bar charts~\cite{kamat2017infiniviz} and heatmaps~\cite{rahman2017ve}.
% Similar to sampling, study on this topic is orthogonal to \system as pushing out partially-updated MVs compromises downstream assumptions. \section{Conclusion}\label{sec:conclusion}
In this work, we focus on addressing the fundamental challenge of OOD detection tasks, which is how to fully understand the semantic discrepancy between the ID/OOD samples. We reveal that the key to success in the realistic SCOOD task is to allocate as many ID samples in the unlabeled set correctly as possible. To this end, we propose a novel uncertainty-aware optimal transport scheme that introduces class-specific energy scores as guidance for effective label assignment. Experimental results show that our method achieves better performance than previous state-of-the-art methods on SCOOD benchmarks.

\textbf{Limitations.} In addition to temperature scaling, other techniques such as feature clipping applied in ReAct~\cite{sun2021react} also enhance the performance of energy score, so how to obtain an OOD score that best fits the SCOOD task can be further explored. Moreover, a setting highly related to SCOOD has been proposed in \cite{katz2022training} and formulated as a constrained optimization problem. We will also theoretically analyze these practical OOD settings in our feature work.

% \section*{Acknowledgments}
\textbf{Acknowledgments.} 
This work is supported by National Key R\&D Program of China under Grant 2020AAA0105701, National Natural Science Foundation of China (NSFC) under Grants 61872327, Major Special Science and Technology Project of Anhui, National Natural Science Foundation of China (62033012) and Ant Group through Ant Research Intern Program.

% \chapter*{Acknowledgement}
\addcontentsline{toc}{chapter}{Acknowledgement}
The authors thank Andrzej Kupsc, Sergey Barsuk, Olivier Callot and Wolfgang K{\"u}hn for their contribution on the CDR draft.
%The authors thank the international review committee XXX for their great effort in reading the CDR draft and providing valuable suggestions. 
The STCF working group thanks all 
the colleagues in the world-wide community for many profitable discussions
and expresses gratitude to the Hefei Comprehensive National Science Center for their strong support.  This work is supported by: international 
partnership program of the Chinese Academy of Sciences Grant No. 211134KYSB20200057.

\begin{abstract}
    The pre-training and fine-tuning paradigm has contributed to a number of
    breakthroughs in Natural Language Processing (NLP). Instead of directly
    training on a downstream task, language models are first pre-trained on
    large datasets with cross-domain knowledge (e.g., Pile, MassiveText, etc.)
    and then fine-tuned on task-specific data (e.g., natural language
    generation, text summarization, etc.). Scaling the model and dataset size
    has helped improve the performance of LLMs, but unfortunately, this also
    lead to highly prohibitive computational costs. Pre-training LLMs often
    require orders of magnitude more FLOPs than fine-tuning and the model
    capacity often remains the same between the two phases. To achieve training
    efficiency~w.r.t training FLOPs, we propose to decouple the model capacity
    between the two phases and introduce Sparse Pre-training and Dense
    Fine-tuning (SPDF). In this work, we show the benefits of using unstructured
    weight sparsity to train only a subset of weights during pre-training
    (Sparse Pre-training) and then recover the representational capacity by
    allowing the zeroed weights to learn (Dense Fine-tuning). We demonstrate
    that we can induce up to 75\% sparsity into a 1.3B parameter GPT-3 XL model
    resulting in a 2.5x reduction in pre-training FLOPs, without a significant
    loss in accuracy on the downstream tasks relative to the dense baseline. By
    rigorously evaluating multiple downstream tasks, we also establish a
    relationship between sparsity, task complexity and dataset size. Our work
    presents a promising direction to train large GPT models at a fraction of
    the training FLOPs using weight sparsity, while retaining the benefits of
    pre-trained textual representations for downstream tasks.
    \footnotetext[0]{\textsuperscript{*}Work done while at Cerebras Systems.}
\end{abstract}

\section{Introduction}
Large language models (LLMs) have contributed to significant advances in natural
language understanding (NLU) and natural language generation (NLG) due to the
introduction of pre-training methods~\citep{Devlin2019BERTPO,
Radford2018ImprovingLU} on massive unannotated datasets (e.g.,
Pile~\citep{gao2020pile}, MassiveText~\citep{rae2021gopher}, etc.). While
scaling the model and dataset size has improved the quality of
LLMs~\citep{wei2022emergent}, it has also substantially increased the
computational cost of pre-training. For instance, GPT-3
175B~\citep{brown2020gpt3} is estimated to cost millions of dollars to
train~\citep{li_2022}. Various techniques have been proposed to reduce the
computational cost of training LLMs, including sparse
attention~\citep{dao2022flashattention, jaszczur2021sparse}, improved
optimization techniques~\citep{tang20221bitadam} and sequence-level curriculum
learning~\citep{li2022the}. While these methods can help reduce computation
time, weight sparsity is one promising technique orthogonal to the above
methods. Here, a subset of model parameters are set to zero, reducing the FLOPs
required during training. 

Despite recent advances in sparse training~\citep{hoefler2022sparsity}, it has
yet to be widely adopted by practitioners. First, it is difficult and expensive
to find the optimal sparsity pattern~\citep{frankle2018lottery, ma2022effective}
that can maintain the same level of accuracy as dense models. Second,
unstructured sparsity can be difficult to accelerate on hardware architectures
optimized for dense computation~\citep{sara2020lottery}. In this work, we show
how we can leverage weight sparsity to reduce training FLOPs, and then recover
the lost representational capacity by shifting to dense weight matrices when
fine-tuning on downstream tasks. In addition, while specialized software kernels
have been developed to achieve inference acceleration with unstructured
sparsity~\citep{gale2020sparse, neural_magic_2021, elsen2019sparse,
Ashby2019ExploitingUS, Wang2021SparseDNNFS}, recent work has shown that we can
realize the gains of unstructured weight sparsity on specialized hardware (e.g.,
Cerebras CS-2~\citep{lie_2022, lie_2021}) when training LLMs. For
example,~\citet{lie_2021} shows the measured speedup for a matrix multiplication
kernel~w.r.t to the sparsity level on a single GPT-3 layer (see Appendix C for
more details). Therefore, as unstructured sparse training techniques continue to
become co-designed with the hardware, we can expect the FLOP reduction to
translate into performance and wall-clock speedups.

Prior work on sparsifying LLMs focus on reducing
training~\citep{chen2022pixelated, dao2022monarch} or inference
FLOPs~\citep{chen2020lth}, while matching standard dense training.
\citet{chen2022pixelated} and~\citet{dao2022monarch} replace dense matrices with
butterfly-based structured sparse weight matrices to reduce a model's size and
accelerate pre-training on block-oriented hardware (e.g.,
GPUs~\citep{krashinsky_2020}, TPUs~\citep{xin2020sparsetpu}). Training with
structured sparsity requires maintaining a regular sparse structure, which can
reduce expressivity at higher sparsity levels. This is a well-known constraint
observed when imposing structured sparsity in dense weight
matrices~\citep{zhou2021learning, jiang2022exposing}. The recent innovations in
hardware architectures aim to facilitate the widespread use and adoption of
unstructured weight sparsity, enabling the ability to achieve higher compression
ratios while attaining practical speedups~w.r.t wall-clock time. Our work
focuses on pre-training with unstructured weight sparsity to reduce the FLOPs
for training language models.

In the recent NLP literature, it is common to first pre-train, then fine-tune a
language model. Fine-tuning pre-trained LLMs on downstream tasks leads to
significantly better accuracy than the zero or few-shot
settings~\citep{alt-etal-2019-fine, ouyang2022training}. The pre-training phase
takes significantly longer compared to fine-tuning on a much smaller dataset to
learn the domain-specific task. In the standard setup, the model size and
capacity is generally kept the same between the two phases. We propose to break
this assumption and show the benefits of modifying the model capacity between
pre-training and fine-tuning with weight sparsity. First, we pre-train a sparse
GPT model to reduce computational training FLOPs. Then, during the fine-tuning
phase, we densify the GPT model, allowing the zeroed weights to learn and
increase the modelling capacity to more accurately learn the downstream task.

While previous work has explored sparse-to-dense training to mitigate the
difficulties of sparse-to-sparse training~\citep{dao2022monarch} and improve the
accuracy of dense models~\citep{han2017dsd}, we perform fully sparse
pre-training and only transition to dense weight matrices during fine-tuning. We
refer to this framework as Sparse Pre-training and Dense Fine-tuning (SPDF) and
demonstrate the ability of the sparse pre-trained model to transfer effectively
to different downstream tasks (e.g., natural language generation and text
summarization). The main contributions of our work are:

\begin{enumerate}
    \item We propose Sparse Pre-training and Dense Fine-tuning (SPDF) as a new
    framework to reduce the FLOPs required during the pre-training phase, while
    maintaining accuracy on downstream tasks.

    \item We demonstrate that we can train GPT-3 XL, at 75\% sparsity, reducing
    the overall training FLOPS by 2.5x, while retaining the benefits of
    pre-trained textual representations in LLMs across a majority of tasks and
    evaluation metrics.

    % \item We show that on the WikiText-103 dataset, GPT-2 Small with 90\%
    % sparsity only sees a 0.36 point increase in perplexity over its dense
    % counterpart, while obtaining a 2.4x FLOP reduction over existing
    % structured sparse setups.
    
    \item We establish a correlation between the optimal sparsity level during
    pre-training and the fine-tuning dataset size and task difficulty.
\end{enumerate}

\section{Methodology}
\label{sec:method}

This section presents our method to reduce pre-training FLOPs using unstructured
weight sparsity. We first explain our intuition and hypotheses, followed by our
methodology for the SPDF framework.

\subsection{Intuition and Hypotheses}
\label{sec:hypotheses}
Prior works have shown that overparameterization of neural networks improves
optimization and generalizability~\citep{Soltanolkotabi2019, neyshabur2018the,
pmlr-v97-allen-zhu19a}, but leads to an increase in compute
cost~\citep{brown2020gpt3}. Recent work on the Lottery Ticket
Hypothesis~\cite{frankle2018lottery} demonstrates that overparameterized dense
networks contain sparse subnetworks which can be trained to the same accuracy as
their dense counterparts, as long as one initializes the training with a good
sparsity mask (``lottery ticket''). However, the process of searching for highly
quality sparse subnetworks is computationally
expensive~\citep{frankle2018lottery, ma2022effective}. Existing sparse training
methods~\citep{evci2020rigl, mocanu2018, jayakumar2020top} aim to discover the
winning lottery ticket (i.e., optimal sparsity mask) in a single training run,
but often fall short of the dense model's accuracy. 

In our framework, we mitigate the loss in representational power due to
difficulties in sparse optimization~\citep{Evci2019TheDO}, by transitioning to
fully dense weight matrices during the fine-tuning phase. Even though we perform
dense fine-tuning, the computational costs associated with fine-tuning are
significantly lower than the cost of pre-training LLMs. Therefore, our method
targets the phase which dominates the training FLOPs (i.e., pre-training). Based
on recent theoretical findings and empirical studies on overparameterization and
sparse neural networks, we lay out a set of hypotheses which we aim to study in
our work through extensive experimental evaluation:

\paragraph*{\normalfont\textit{Hypothesis 1\labeltext{1}{hyp:one}: High degrees
of weight sparsity can be used during the pre-training phase of LLMs while
preserving the downstream accuracy with dense fine-tuning.}} \mbox{}

Inducing sparsity during pre-training may cause a loss in representational power
due to difficulties in sparse optimization and inability to discover optimal
sparsity masks~\citep{Evci2019TheDO}. To mitigate these challenges, we aim to
increase the representational power by allowing the zeroed weights to grow
during fine-tuning (i.e., dense fine-tuning). 

Additionally, note the full capacity of the pre-trained model is often not
required to generalize on the simpler downstream task, when using sparsity
during pre-training~\citep{Ding2022DeltaTA}.
\citet{aghajanyan-etal-2021-intrinsic} investigate this phenomenon from a
different angle and show pre-trained language models can learn a large set of
NLP tasks with only a few parameters.
% They show that the pre-training minimizes the intrinsic
% dimension~\citep{li2018measuring} when later fine-tuning various downstream
% tasks. 
This indicates that the full parameterization of the model is not needed to
generalize well across downstream fine-tuning tasks. Hence, we can exploit
weight sparsity during pre-training while retaining important textual
representations despite the model's lower representational capacity.

\paragraph*{\normalfont \textit{Hypothesis 2\labeltext{2}{hyp:two}: The
performance of the sparse pre-trained model is correlated with the dataset size
and degree of difficulty in the downstream task.}} \label{hyp:two} \mbox{}

\citet{liu2023sparsity} evaluate sparse networks on a diverse set of tasks with
varying degrees of difficulty and show a strong correlation between a model's
ability to be sparsified and the task difficulty. Hence, we hypothesize that
models trained on complex tasks with high sparsity levels can suffer more from
sparse training and experience a greater drop in performance compared to simpler
tasks. We also note that small fine-tuning datasets may trigger
over-fitting~\citep{Li2021ImprovedRA}. Therefore, we hypothesize that larger
datasets can allow the sparse model to improve its generalization error on the
task, and recover from training with high sparsity.

\paragraph*{\normalfont\textit{Hypothesis 3\labeltext{3}{hyp:three}: As we
 increase the size of the language model, larger models become more amenable to
 higher levels of sparsity during pre-training.}} \label{hyp:three} \mbox{}

Existing work~\citep{liu2022the} has shown that the quality of a network trained
with random static sparsity (even at high sparsity levels) improves quickly to
match its dense counterpart as the network grows wider and deeper. Also, larger
models tend to have a smaller intrinsic
dimension~\citep{aghajanyan-etal-2021-intrinsic}, which suggests that all
parameters are not required to represent the average NLP task. Therefore, we
expect the gap in downstream performance between the sparse pre-trained model
and its dense counterpart to grow smaller as the size of the model increases.  

\subsection{Sparse Pre-training and Dense Fine-tuning}
Our training procedure consists of two phases. The first phase involves
pre-training a sparse language model on a large corpus of text in an
unsupervised manner. Here, we induce unstructured weight sparsity into the
neural network to reduce the pre-training FLOPs. This is followed by a dense
fine-tuning stage, where we expand the representational capacity of the model by
allowing zeroed weights to learn, and adapt to a discriminative task with
labeled data.

\paragraph*{Unsupervised Dense Pre-training} While our proposed framework is
agnostic to the training objective, we focus on autoregressive language modeling
as our motivating use case. In an autoregressive language model, the sequence
generation process is modeled as a Markov chain, where the token to be predicted
depends on all the previous tokens~\citep{bengio2003pretrain}. Hence, the
standard approach is to learn the probability distribution over sequences of
tokens from an unsupervised pre-training corpus. Given an unsupervised
pre-training corpus of tokens~$\mathcal{U} = \{u_1,
u_2,\ldots,u_{|\mathcal{U}|}\}$,  where $|\mathcal{U}|$ is the total number of
tokens. We aim to maximize the likelihood using the language modeling objective
formulated as follows,
\begin{equation}
    \label{eq:gpt}
    \mathcal{L}(\mathcal{U}) = \sum_{i=1}^{|\mathcal{U}|}\log(p(u_i | u_{i-k},\ldots,u_{i-1}, \theta)),\notag
\end{equation}

where $k$ is the size of the context window, and the conditional probability $p$
is modeled using a neural network with parameters $\theta \in \mathbb{R}^{N}$.
The parameters of the $l^{th}$ layer $\in L$ total layers are denoted as
$\theta_l$, along with the total number of parameters represented as $N_l$. We
note that the network parameters $\theta$ are considered to be dense.

\paragraph*{Unsupervised Sparse Pre-training} To induce sparsity into the
$l^{th}$ layer, we drop $s_l \in (0, 1)$ of its connections, where $s_l$ to
refer to the sparsity of layer $l$. This results in a total of $(1-s_l)N_l$
parameters. Finally, the overall sparsity of a sparse subnetwork is defined as
the ratio of zeroes to the total number of parameters in the original dense
network, i.e., $S = \frac{\sum_l^L s_lN_l}{N}$. In our sparse training setup, we
apply a binary sparsity mask $m \in \{0,1\}^{|\theta|}$ on the initial
parameters $\theta^0$, such that its initialization is $m \odot \theta^{0}$.
Here, the values 0 and 1 in the mask denote inactive (i.e., zero) and active
(i.e., non-zero) weights, respectively. As a result, the sparse language model
minimizes the following objective instead,
\begin{equation}
    \label{eq:gpt}
    \mathcal{L}(\mathcal{U}) = \sum_{i=1}^{|\mathcal{U}|}\log(p(u_i | u_{i-k},\ldots,u_{i-1}, m \odot \theta)).
\end{equation}

In our work, we focus solely on static sparsity (i.e., $m$ remains fixed
throughout training) and the weights are randomly pruned at initialization.
Specifically, we remove weights in each layer $l \in L$ randomly to the target
sparsity $s_l$. Although several works have explored generating different
layer-wise sparsity ratios at initialization (e.g.,
Erd\"os-R\'enyi-Kernel~\cite{evci2020rigl}, Ideal Gas
Quota~\citep{chen2022sparsity}, SNIP~\citep{lee2018snip},
GraSP~\citep{Wang2020Picking}, SynFlow~\citep{tanaka2020synflow}, etc.), we
focus on the simplest setup, which is uniform sparsity~\citep{gale2019state}. In
uniform sparsity, each sparsified layer is pruned to the same target sparsity
level. 

For the language model, we use GPT~\citep{radfordGPT2, brown2020gpt3} in our
experiments, which is a variant of the
Transformer~\citep{vaswani2017transformers}. We train the network with objective
shown in Eq.~\ref{eq:gpt} and AdamW~\citep{loshchilov2017decoupled} optimizer on
an unsupervised pre-training dataset for a total of $j$ iterations, arriving at
parameters $\theta^j$. Then, we adapt (i.e., fine-tune) the final pre-trained
autoregressive language model $p_{m \odot \theta}$ to the supervised target
task.

\begin{figure}[!t]
    \centering
    \includegraphics[keepaspectratio=true, width=\linewidth]{./figures/spdf_framework.pdf}
    \caption{Sparse Pre-training and Dense Fine-tuning (SPDF) framework. In this
    framework, we sparsify a dense network and perform sparse pre-training
    followed by dense fine-tuning (green connections indicate newly activated
    weights). We use SPDF to pre-train large GPT models at a fraction of the
    training FLOPs using weight sparsity, and still retain the benefits on
    downstream tasks with dense fine-tuning.  }
    \label{fig:spdf}
    \vspace{-0.1in}
\end{figure}

\paragraph*{Dense Fine-tuning} 
Following~\citet{hu2022lora} and~\citet{Li2021PrefixTuningOC}, each downstream
fine-tuning task is represented by a training dataset consisting of
context-target pairs defined as: $\mathcal{Z} = \{(x_1 , y_1),(x_2 ,
y_2),\ldots,(x_{|x|}, y_{|y|})\}$, where both $x$ and $y$ are sequences of
tokens. For example, in structured data-to-text (e.g.,
E2E~\citep{novikona2017e2e}), $x$ corresponds to a linearized data table and $y$
a textual description; in text summarization (e.g., Curation
Corpus~\citep{curationcorpusbase:2020}), $x$ is the content of an article and
$y$ is its summary. 

We initialize the start of dense fine-tuning to the final pre-trained parameters
$\theta^j$ and during fine-tuning are updated to $\theta^j + \Delta\theta$. For
each downstream task,  we learn a different set of parameters with the
task-specific parameter increment $\Delta\theta$ whose dimension
$|\Delta\theta|$ equals $|\theta|$. Other works have explored more parameter
efficient approaches to reduce the size of the task-specific parameters for the
purpose of deploying fine-tuned models~\citep{ben-zaken-etal-2022-bitfit,
pmlr-v97-houlsby19a,hu2022lora}.  However, in our approach, we focus on reducing
the pre-training FLOPs with unstructured weight sparsity and perform dense
fine-tuning to mitigate the challenges of sparse optimization by increasing
representational power of the network. In the dense fine-tuning phase, we
essentially remove the sparsity mask $m$ to allow the inactive weights to grow.
More specifically, we increase the representational capacity in $\theta^j$ by
reviving all $\sum_{l}^{L}s_l {\cdot} N_l$ inactive weights, where all newly
activated weights are initialized to 0. We evaluated other initializations like
scaled normal distribution, but this did not lead to better results. Finally,
the network is updated in a dense manner with the objective shown below,
\begin{equation}
    \mathcal{L}(\mathcal{Z}) = \sum_{(x,y)\in\mathcal{Z}}\sum_{t=1}^{|\boldsymbol{y}|}\log(p(y_t | (x_{1},\ldots,x_{t-1}), \theta^j + \Delta\theta)).\notag
\end{equation}

The generic Sparse Pre-training and Dense Fine-Tuning (SPDF) framework,
illustrated in Figure~\ref{fig:spdf}, consists of the following three steps:
\begin{enumerate}
    \item Sparsify a given dense network to some target sparsity level, $s_l$,
    at each sparsifiable layer.
    \item \textit{Pre-train} the sparse model following the same training
    schedule as the original dense model.
    \item \textit{Fine-tune} the pre-trained sparse network on a given
    downstream task in a dense manner by allowing the zeroed weights to learn.
\end{enumerate}


\section{Experimental Setup and Results}
\label{sec:results}
First, we provide details on our pre-training settings for GPT-2 Small (125M)
and GPT-3 XL (1.3B), as well as our setups for the downstream fine-tuning tasks.
Then, we compare sparse pre-training and sparse fine-tuning against sparse
pre-training and dense fine-tuning to highlight the benefits of fine-tuning in a
dense manner. Next, we validate our hypotheses (refer to
Section~\ref{sec:hypotheses}) by evaluating SPDF across several tasks in natural
language generation and text summarization. Following this, we compare the
parameter subspaces between the pre-trained and fine-tuned models. Last, we
present the advantages in training efficiency w.r.t total training FLOPs when
using SPDF versus standard dense pre-training and dense fine-tuning. 

All GPT models are pre-trained and fine-tuned using the Cerebras CS-2, taking
advantage of its ability to accelerate training with unstructured sparsity. At
present, the specialized kernels of Cerebras CS-2 are designed to facilitate
training with static unstructured sparsity. Consequently, the results presented
in this section do not include the utilization of dynamic sparse training
methods (e.g., SET~\citep{mocanu2018}, RigL~\citep{evci2020rigl}, etc). In
Appendix C, we emphasize the possible advantages achieved through unstructured
weight sparsity on the Cerebras CS-2. We provide measured speedup results
compared to theoretical speedup across different sparsity levels for a GPT-3
layer's 12k~$\times$ 12k matrix multiplication (MatMul)~\citep{lie_2022}.

\begin{figure*}[!t]
    \centering
    \includegraphics[keepaspectratio=true, width=0.82\linewidth]{./figures/gpt2_small_sparse_pretrain_dense_ft.pdf}
    \caption{ Comparison of sparse-to-dense vs sparse-to-sparse pre-training and
    fine-tuning with GPT-2 Small on E2E, WebNLG and DART. Across tasks dense
    fine-tuning noticeably outperforms sparse fine-tuning, especially at 75\%
    sparsity. }
    \label{fig:sparseft}
    \vspace{-0.1in}
\end{figure*}

\paragraph{Flop Optimal Pre-training via Chinchilla Scaling Law}
It was previously conventional in the literature to train all large language
models (e.g., GPT-3~\citep{brown2020gpt3}, Gopher~\citep{rae2021gopher},
Jurassic~\citep{J1WhitePaper}, etc.) on approximately 300B tokens of data. More
recently, Chinchilla~\citep{hoffmann2022an} shows how parameters and data should
be scaled equally as compute budget increases, which leads to significant gains
in FLOP efficiency. In our pre-training setup, we follow Chinchilla's scaling
law which suggests that we need approximately 20 tokens per parameter. Thus, for
GPT-2 Small, a model with 125M parameters needs to be pre-trained on 2.5B
tokens. Then, for GPT-3 XL, a model which has 1.3B parameters, needs to be
pre-trained on 26B tokens. Unless stated otherwise, we pre-train our sparse GPT
models from scratch on the Pile dataset~\citep{gao2020pile} across sparsity
levels $S \in \{50\%, 75\%\}$.


\paragraph{Fine-tuning on Downstream Tasks}  We studied dense fine-tuning on
several downstream tasks in natural language generation and text summarization.
We follow~\citet{hu2022lora} in using the three standard natural language
generation benchmark datasets (i.e., E2E~\citep{novikona2017e2e},
WebNLG~\citep{gardent2017webnlg} and DART~\citep{nan2021dart}). In addition, we
fine-tune on Curation Corpus~\citep{curationcorpusbase:2020} according to the
details described in~\citep{rae2021gopher}. We fine-tune all parameters of the
pre-trained GPT models and evaluate the final fine-tuning performance using the
official evaluation scripts. More details on the hyperparameters can be found in
Appendix A.

\subsection{Details on the Fine-tuning Datasets}

Our work uses four fine-tuning datasets to investigate the efficacy of our SPDF
framework. These datasets were chosen for studying the effect of sparse
pre-training on different sizes and types of data, along with the varying degree
of difficulty in the tasks.

\textbf{End-2-End (E2E) NLG challenge} dataset contains approximately 45k
training, 4.6k validation, and 4.6k test examples with 8 distinct fields from
the restaurant domain. The goal of the task is to generate natural language
descriptions in the restaurant domain from meaning representations. We use the
official evaluation script, which reports BLEU~\citep{kishore2002bleu},
NIST~\citep{belz2006comparing}, METEOR~\citep{alon2007meteor},
ROUGE-L~\citep{lin2004rouge}, and CIDEr~\citep{ramakrishna2015cider}.
% It contains multiple test references for one source table, and the average
% output length is 22.9.

\textbf{WebNLG} dataset consists of 18k training, 2.2k validation, and 2.4k test
examples, where the input is a sequence of (subject, property, object) triples.
In the training and validation splits, the input describes entities from 9
distinct DBpedia categories. The test set contains 15 different domains where 10
are present only in the training data. Here, the test data is split into two
parts, where categories seen in the train set are in the first half, while the
second half consists of 5 unseen categories. We use the official evaluation
script, which reports BLEU, METEOR and TER~\citep{snover2006study}. The WebNLG
dataset is the smallest of the three NLG tasks we evaluate on.

\textbf{DART} is an open domain DAta-Record-to-Text (i.e., table-to-text)
dataset, with a similar input format to WebNLG. It consists of 62.6k training,
6.9k validation, and 12.5k test examples from several sources:
WikiSQL~\citep{zhong2018seqsql},
WikiTableQuestions~\citep{pasupat2015compositional}, Cleaned
E2E\footnote{\url{https://github.com/tuetschek/e2e-cleaning}}, and WebNLG
2017\footnote{\url{https://gitlab.com/shimorina/webnlg-dataset/-/tree/master/webnlg_challenge_2017}}
and applies some manual or automated conversion. We use the official evaluation
script and report BLEU, METEOR and TER. The DART dataset is considered to be the
most challenging NLG task out of the three we evaluate.

\textbf{Curation Corpus} is a recently introduced dataset comprised of 40,000
bespoke text summaries of finance articles for the task of text summarization.
We follow the instructions in the Curation Corpus GitHub
repository\footnote{\url{https://github.com/CurationCorp/curation-corpus}} to
download approximately 40k article summary pairs. After filtering examples where
either the article or the summary are empty, we are left with 39,911 examples.
Following~\citet{marfurt2021sentence}, we split them into train/validation/test
sets as 80/10/10 to arrive at split sizes of 31,929/3,991/3,991.


\subsection{Sparse Fine-tuning vs Dense Fine-tuning} In this section, we first
empirically establish the need for dense fine-tuning to help mitigate the
difficulties of sparse-to-sparse training (i.e., sparse pre-training followed by
sparse fine-tuning). In Figure~\ref{fig:sparseft}, we compare dense fine-tuning
against sparse fine-tuning on GPT-2 Small and show that across all three NLG
tasks (i.e., E2E, WebNLG and DART), dense fine-tuning helps reduce the drop in
BLEU score relative to the respective dense baselines. For example, the 75\%
sparse GPT-2 Small model on WebNLG observes a delta of -1.48 and -0.78 in the
BLEU scores, when sparse fine-tuning and dense fine-tuning, respectively. This
suggests that fully sparse end-to-end pre-training and fine-tuning can prevent
the model from generalizing well on downstream tasks. However, we can mitigate
the difficulties of poor generalizability due to sparse-only training by
transitioning from sparse to dense matrices during the fine-tuning phase.
Although dense fine-tuning consumes more FLOPs compared to sparse fine-tuning,
the overall fine-tuning FLOPs relative to pre-training, still remains
insignificant  (discussed further in Section~\ref{sec:spdf_train_eff}). 

% We also note that even though the fine-tuning datasets are small, we perform
% early stopping to ensure dense fine-tuning does not overfit.

\begin{table}[!t]
    \caption{Downstream accuracy of GPT-2 Small and GPT-3 XL across various
    tasks (i.e., E2E, WebNLG, DART and Curation Corpus) at sparsity levels 50\%
    and 75\% during pre-training. In the metric column, the direction of the
    arrow indicates better result (e.g., up indicates higher is better).}
    \label{tab:alltasks}
    % \makebox[\linewidth]{
    \centering
    \resizebox{\linewidth}{!}{
    \begin{tabular}{cc|ccc|c}
        \toprule
        \multirow{3}{*}{Model} &
        \multirow{3}{*}{\begin{tabular}[c]{@{}c@{}}Pre-Train\\
        Sparsity\end{tabular}} & \multirow{2}{*}{E2E} & \multirow{2}{*}{WebNLG}
        & \multirow{2}{*}{DART} &
        \multirow{2}{*}{\begin{tabular}[c]{@{}c@{}}Curation \\
        Corpus\end{tabular}}  \\ 
        & &                      &                         & & \\ \cmidrule{3-6}
        & & \multicolumn{3}{c|}{BLEU$\uparrow$} & PPL$\downarrow$ \\ \midrule
        \multirow{3}{*}{\begin{tabular}[c]{@{}c@{}}GPT-2 \\ Small\end{tabular}}
        & 0\% & 67.49\textsubscript{$\pm$0.60} & 63.42\textsubscript{$\pm$0.26}
        & 46.30\textsubscript{$\pm$0.16} & 13.38\textsubscript{$\pm$0.02} \\
                              & 50\% &      67.39\textsubscript{$\pm$0.38}     &  
                              63.10\textsubscript{$\pm$0.13}    &
                              45.74\textsubscript{$\pm$0.10}&
                              15.09\textsubscript{$\pm$0.04} \\
                              & 75\% &   66.50\textsubscript{$\pm$0.85}       &
                              62.64\textsubscript{$\pm$0.22}  &
                              44.97\textsubscript{$\pm$0.11} &
                              17.14\textsubscript{$\pm$0.01}            \\
    \midrule \multirow{3}{*}{GPT-3 XL} & 0\% & 68.10\textsubscript{$\pm$0.60} &
        63.62\textsubscript{$\pm$0.23} & 47.71\textsubscript{$\pm$0.11}&
        8.28\textsubscript{$\pm$0.01}   \\
                              & 50\% &    67.98\textsubscript{$\pm$0.63}       &
                              
                              63.47\textsubscript{$\pm$0.21}       &
                              47.10\textsubscript{$\pm$0.13} &
                              9.21\textsubscript{$\pm$0.02}            \\
                              & 75\% &     67.66\textsubscript{$\pm$0.59}      &
                              
                              63.06\textsubscript{$\pm$0.11}        &
                              46.96\textsubscript{$\pm$0.08}&
                              11.03\textsubscript{$\pm$0.02} \\
                              \bottomrule             
    \end{tabular}
    % }
    }
\end{table}

\subsection{SPDF on Natural Language Generation and Text Summarization} 

We perform an extended study on SPDF to further investigate its effectiveness on
a diverse set of fine-tuning tasks, when using sparse pre-trained GPT-2 Small
and GPT-3 XL models. In this section, we focus on natural language generation
(i.e., E2E, WebNLG, and DART) and text summarization (i.e., Curation Corpus)
tasks and refer to Table~\ref{tab:alltasks} for all the discussion points. We
note that in Appendix B, we provide evaluation scores on all the metrics used to
officially evaluate E2E, WebNLG and DART, respectively.

First, we validate Hypothesis~\ref{hyp:one} that high degrees of weight sparsity
can be induced during pre-training. Our results indicate that in most settings,
we can pre-train these GPT models with up to 75\% sparsity without significant
degradation across all NLG tasks. On the 75\% sparse GPT-3 XL model, we observe
deltas of -0.44, -0.56, and -0.75 in the BLEU scores for E2E, WebNLG and DART,
respectively. In addition, the 50\% sparse GPT-2 Small model observes deltas of
-0.10, -0.32, and -0.56 in the BLEU scores for E2E, WebNLG and DART,
respectively. Overall, our findings show that these GPT models can be
pre-trained with 50\%-75\% sparsity without losing significant accuracy on these
downstream tasks.

Second, we validate Hypothesis~\ref{hyp:two} that the performance of the sparse
pre-trained model is correlated with the difficulty of the fine-tuning task.
E2E, WebNLG and DART are NLG tasks which focus on mapping structured data
content to a text describing this content. The Curation Corpus task focuses on
summarizing the text description. While both tasks involve generating
semantically coherent natural language, the summarization tasks are more
difficult, since it require understanding of long sequences and compressing the
sequence without loss of information. On the E2E, WebNLG and DART tasks, GPT-3
XL can be pre-trained up to 75\% sparsity without a significant drop in BLEU
score, as discussed previously. In contrast, on Curation Corpus, GPT-3 XL
pre-trained at 75\% sparsity loses 2.75 perplexity. In general, all data-to-text
NLG tasks obtain a lower degradation compared to the more difficult Curation
Corpus summarization task at higher levels of sparsity.

Finally, we validate Hypothesis~\ref{hyp:three} that as the size of the model
increases, it becomes more amenable to higher sparsity levels. We analyze the
relative drop in performance between the dense baseline and its sparse variants
for GPT-2 Small and GPT-3 XL. This trend is clearly evident on the more
difficult Curation Corpus task at 75\% sparsity, where relative to the dense
baseline, the larger GPT-3 XL model has a perplexity delta of +2.75 compared to
a worse +3.76 delta observed in the smaller GPT-2 Small model. Similarly, on the
DART task, the most challenging NLG task out of the three we evaluated, the
delta in the BLEU score is -1.33 and -0.75 for GPT-2 Small and GPT-3 XL,
respectively. These observations indicate that as the size of the language model
increases, it suffers less on downstream task performance when training with
high sparsity.

\subsection{Pre-training vs Fine-tuning Parameter Subspaces}
In this section we analyze the parameter subspaces of the pre-trained model and
its fine-tuned parameters  across all layers to further understand (a) the
behaviour of dense and spare pre-trained representations when fine-tuned, and
(b) the effect of scaling the model size on parameter subspaces between the two
phases. Inspired by~\citet{RadiyaDixit2020HowFC}, we measure the angular
distance (i.e., cosine distance) between the pre-trained model parameters and
its fine-tuned parameters on a given downstream task. Specifically, in all
layers of the language model, we inspect the four weight matrices in the
self-attention module; $W_Q$ (query), $W_K$ (key), $W_V$ (value) and $W_D$
(attention output projection) and the two in the MLP module; $W_I$
(intermediate) and $W_O$ (MLP output projection). In this analysis we focus on
DART, the most difficult NLG task, and report the cosine distances for all
modules in each layer of the dense and 75\% sparse pre-trained GPT-2 Small and
GPT-3 XL. 

First, we aim to understand the behaviour of the parameter subspaces of the
dense and sparse pre-trained models when fine-tuned. In GPT-2 Small (see
Figure~\ref{fig:gpt2_dart_subspaces}) and GPT-3 XL (see
Figure~\ref{fig:gpt3xl_dart_subspaces}), we observe that the dense pre-trained
parameters and its fine-tuned parameters have very small cosine distances in
almost all modules across each layer, whereas the 75\% sparse model has larger
cosine distances in certain modules (e.g., $W_D$ and $W_O$) across all layers.
Here, the dense model's fine-tuned parameters require less change in the
parameter subspace relative to the pre-trained parameters, while the sparse
model requires more movement in certain modules to learn the downstream task.
This indicates that pre-trained models which learn high quality textual
representations need less movement in the parameter subpsace to adapt to the
downstream task. Although the sparse model has less representational capacity in
its pre-trained parameters, it is capable of adapting certain modules through
dense fine-tuning to learn the downstream task and stay competitive with the
dense model's performance.

\begin{figure}[!t]
    \centering
    \begin{subfigure}[b]{0.78\linewidth}
       \includegraphics[keepaspectratio=true, width=1\linewidth]{./figures/gpt2small_dart_cos_0.00_all.pdf}
    \end{subfigure}
    
    \begin{subfigure}[b]{0.78\linewidth}
       \includegraphics[keepaspectratio=true, width=1\linewidth]{./figures/gpt2small_dart_cos_0.75_all.pdf}
    \end{subfigure}
    
    \caption[GPT-2 Small Pre-trained vs. DART Weights]{The angular distances in
    parameter subspaces between dense (top) and 75\% sparse (bottom) pre-trained
    and fine-tuned DART weights for GPT-2 Small.}
    \label{fig:gpt2_dart_subspaces} 
\end{figure}

\begin{figure}[!t]
    \centering
    \makebox[\linewidth]{
    \begin{subfigure}[b]{\linewidth}
       \includegraphics[width=1\linewidth]{./figures/gpt3xl_dart_cos_0.00_all.pdf}
       
    \end{subfigure}
    }
    \begin{subfigure}[b]{\linewidth}
       \includegraphics[width=1\linewidth]{./figures/gpt3xl_dart_cos_0.75_all.pdf}
    \end{subfigure}
    
    \caption[GPT-3 XL Pre-trained vs. DART Weights]{The angular distances in
    parameter subspaces between dense (top) and 75\% sparse (bottom) pre-trained
    and fine-tuned DART weights for GPT-3 XL.}
    \label{fig:gpt3xl_dart_subspaces} 
    \vspace{-0.1in}
\end{figure}


Next, we study the effect of model size and the parameter subspaces of the
pre-trained and fine-tuned parameters. Evidently, in
Figure~\ref{fig:gpt3xl_dart_subspaces}, we observe that the dense pre-trained
GPT-3 XL model has very small cosine distances across all modules in almost each
layer, in comparison to GPT-2 Small. This suggests that as we increase the
modeling capacity of the language model, only a few model parameter updates
traverse a very short distance in the parameter space. This results in the
pre-trained and fine-tuned weights being highly close across all modules in
almost each layer. The larger language model is more capable of learning high
quality representations, thus requires less movement in the fine-tuning
parameter subspace. Even at 75\% sparsity, the GPT-3 XL model requires
significantly less change to the pre-trained parameters compared to GPT-2 Small
in order to perform competitively well with the dense model. Given that many
layers experience a very small change in the parameter subspace, we leave the
investigation of freezing these modules during the fine-tuning phase for future
work.

\begin{table*}[!ht]
    \caption{Total FLOPs along with the associated theoretical speedup~w.r.t the
    dense baseline (in brackets) for each of the evaluated fine-tuning tasks on
    GPT-2 Small and GPT-3 XL.  The reported training FLOPs includes both
    pre-training and dense fine-tuning FLOPs. GPT-3 XL 75\% SPDF provides
    $\approx$ 2.5x FLOP reduction over end-to-end dense training.}
    \label{tab:flops_gpt2_gpt3xl}
    % \makebox[\linewidth]{
    \centering \resizebox{0.78\linewidth}{!}{
    \begin{tabular}{cc|cccc}
        \toprule
    \multirow{2}{*}{Model} &
                           \multirow{2}{*}{\begin{tabular}[c]{@{}c@{}}Pre-Train\\
                           Sparsity\end{tabular}} &
                           \multicolumn{4}{c}{Pre-training + Fine-tuning FLOPs
                           ($\times 10^{18}$)}         \\ \cmidrule{3-6} & & E2E
                           & WebNLG       & DART         & Curation Corpus \\
                           \midrule \multirow{3}{*}{GPT-2 Small}           & 0\%
                           & 2.48 (1.00x) & 2.48 (1.00x) & 2.45 (1.00x) & 2.44
                           (1.00x)  \\
                                      & 50\% & 1.84 (1.34x) & 1.82 (1.35x) &
                           1.84 (1.34x) & 1.81 (1.35x) \\
                & 75\% & 1.52 (1.64x)       & 1.49 (1.65x)       & 1.52 (1.64x)
                & 1.48 (1.65x) \\ 
    \midrule \multirow{3}{*}{GPT-3 XL}           & 0\% & 236.62 (1.00x) & 236.62
    (1.00x) & 236.33 (1.00x) & 236.32 (1.00x) \\
               & 50\% & 142.40 (1.66x)       & 142.10 (1.66x) & 142.01 (1.66x) &
               142.40 (1.66x) \\
            & 75\% & 95.29 (2.48x)       & 94.98 (2.49x)        & 95.29 (2.48x)
            & 94.90 (2.49x) \\
\bottomrule      
    \end{tabular}
    }
    \vspace{-0.1in}
\end{table*}

\subsection{SPDF Training Efficiency}
\label{sec:spdf_train_eff}
We compare the standard dense pre-training followed by dense fine-tuning
framework to SPDF and highlight the potential FLOP reduction we can achieve. In
Table~\ref{tab:flops_gpt2_gpt3xl}, we report the total FLOPs (i.e., both the
forward and backward propagations) needed for pre-training and dense fine-tuning
GPT-2 Small and GPT-3 XL models on each of the tasks we evaluated. We note that
in the GPT-2 Small model, the percentage of attention and vocab embeddings FLOPs
account for approximately 13.3\% and 27\% of the total FLOPs, respectively.
Therefore, at 75\%, we achieve approximately 1.65x FLOP reduction over the dense
baseline. However, in the larger GPT-3 XL, the percentage of attention and vocab
embeddings FLOPs account for 13.3\% and 6.8\%, respectively. As a result, at the
GPT-3 XL scale, SPDF provides almost 2.5x FLOP reduction over the dense baseline
when pre-training with 75\% sparsity. The trend of FLOP reduction relative to
the dense baseline continues to increase with larger models, so the potential
gains from sparse pre-training improves as model size grows. We also emphasize
that the total fine-tuning FLOPs is a small fraction of the total pre-training
FLOPs. In Appendix A.4, we provide details on how the total pre-training and
fine-tuning FLOPs for GPT-2 Small and GPT-3 XL were calculated.

\section{Related Work}

\paragraph{Zero-Shot vs. Fine-tuning}
% The majority of fine-tuning literature has been established on \textit{dense
% fine-tuning} of deep learning models to specific downstream tasks. Fine-tuning
% a pre-trained language model has become the de facto standard for doing
% transfer learning~\citep{sanh2021multitask, ouyang2022training}. The empirical
% success of these methods has led to the development of even larger language
% models~\citep{smith2022using, chowdhery2022palm}. 
Recent works have shown that large language models can achieve reasonable
performance without any parameter
updates~\citep{brown2020gpt3,chowdhery2022palm,rae2021gopher, smith2022using},
often referred to as the zero-shot or few-shot setting. When no parameters are
fine-tuned, framing a target task in terms of the pre-training objective enables
zero-shot or few-shot learning to use a task-specific prompt and a few examples
of a task~\citep{brown2020gpt3}. However, while such few-shot learning is simple
using such large models, there are alternative methods to obtain similar task
accuracy using smaller models~\citep{schick2021smallllm}. In recent
work,~\citet{lamda2022} demonstrate that while scaling the size of LaMDA can
improve quality, combining scaling with fine-tuning can improve the model across
all metrics including quality, safety and groundness.~\citet{irene2021} show
that fine-tuning also helps update language model behaviour to mitigate harmful
outputs, which is highly critical for real-world deployment of LLMs (e.g.,
ChatGPT~\citep{openai_chatgpt_2022}, Bard~\citep{pichai_2023}, etc.). To achieve
the best performance in practice, fine-tuning will continue to be the modus
operandi when using pre-trained LLMs. Hence, our work focuses on pre-training
and fine-tuning language models across a diverse set of tasks, including natural
language generation and text summarization.

\paragraph*{Efficient Fine-tuning} While most large-scale models such as
GPT~\citep{brown2020gpt3, smith2022using} or T5~\citep{raffel2022transfer} are
trained dense, there are works~\citep{houlsby2019parameter, li2021prefix,
zaken2021bitfit, hu2022lora} that explore using limited capacity (tuning a few
layers or subset of parameters) in the pre-trained models to fine-tune on
downstream tasks. These works are indicative that the total modeling capacity is
unnecessary for fine-tuning on downstream tasks. Our work draws some inspiration
from these works for exploiting the limited capacity of models for final tasks.
However, we choose to reduce FLOPs for pre-training (significantly more training
FLOPs than fine-tuning) and then add all the modeling capacity back during
fine-tuning. This allows us to train large models efficiently and yet retain
accuracies comparable to dense baselines. Although we do not explore efficient
fine-tuning in our study, we leave the exploration of using alternative sparsity
schedules~\citep{zhu2018toprune, liu2021sparsegranet}, adapting a subset of
parameters during fine-tuning~\citep{Ding2022DeltaTA} and imposing low-rank
structures~\citep{hu2022lora} for future work.

\paragraph{Weight Sparsification Techniques}
Many unstructured weight sparsification techniques have been proposed in the
literature for training neural networks~\citep{hoefler2022sparsity}, which can
be categorized as static sparsity and dynamic sparsity. Static sparsity methods
have a fixed sparsity structure (i.e., sparsity mask) determined at
initialization~\citep{lee2018snip,Wang2020Picking}. In contrast, dynamic sparse
training (DST) methods iteratively prune (drop) and add (regrow) weights during
training~\citep{mocanu2018, evci2020rigl, jayakumar2020top,
shaoyi2022betterrigl} to find the best possible sparse subnetwork while
retaining accuracy comparable to dense baselines. Although, dynamic sparse
training methods can help achieve Pareto improvements in terms of number of
training FLOPs to accuracy, we leave this for future work. Inspired
by~\citep{li2022the}, which shows that scaling the size of CNNs closes the gap
between a randomly pruned sparse network and its dense counterpart, we focus our
study on language models with static sparsity. 
% In addition, most static sparse methods maintain a constant modeling capacity
% throughout training, but, in our work, we increase the capacity by
% pre-training sparse models and fine-tuning dense. 
While~\citet{dao2022monarch} demonstrate the benefits of sparse-to-dense
training, they mainly apply it during pre-training and instead, focus their
studies on dense-to-sparse fine-tuning similar to other efficient fine-tuning
efforts. In our work, we show that sparse pre-training followed by dense
fine-tuning on downstream tasks can be on par with the accuracy of a dense
pre-trained model on many tasks, while significantly lowering overall training
FLOPS.

\section{Conclusion and Future Work}

In this work, we introduced Sparse Pre-training and Dense Fine-tuning (SPDF) to
reduce the computational FLOPs of training GPT models using weight sparsity. To
the best of our knowledge, this is the first time a large GPT model has been
pre-trained with high sparsity (50\%-75\%) without significant loss in
downstream task metrics. In our work, we only use simple static sparsity, which
is arguably the most naïve way to induce sparsity in neural networks. As for
future work, there are several natural directions for improving our results on
even larger models, including dynamic sparsity methods, better optimization
techniques for sparse training, and architectures amenable to sparse training.
Moreover, to limit the computational cost of our study, we trained our GPT
models following the Chinchilla scaling law. Although the Chinchilla
pre-training schedule has been shown to be FLOP-optimal for dense models, we
plan to investigate how well it transfers to sparse models. Our future work will
also investigate sparse scaling outside the Chinchilla dense scaling laws.
Regardless, we see the tremendous promise of unstructured weight sparsity to
accelerate the training of LLMs, enabled by the recent advances in deep learning
hardware accelerators.

\begin{contributions} % will be removed in pdf for initial submission 
    % (without ‘accepted’ option in \documentclass) so you can already fill it
    % to test with the ‘accepted’ class option
We provide a summary of each author’s contributions:
\begin{itemize}
    \item Vithursan Thangarasa led the effort for training/evaluation of large scale GPT
    models on the Cerebras CS-2, evaluated the technique in different FLOP efficient
    training setups, brought up multiple downstream tasks, analyzed the parameter
    subspaces, and wrote the manuscript.
    
    \item Abhay Gupta helped with pre-training GPT models on the CS-2 and ran reference
    models to validate our training and fine-tuning setup.
    
    \item William Marshall brought up various downstream tasks on the CS-2 and assisted in
    running fine-tuning experiments.
    
    \item Tianda Li assisted William Marshall and Vithursan Thangarasa with running
    fine-tuning experiments.
    
    \item Kevin Leong assisted Abhay Gupta with pre-training GPT models on the CS-2 and
    provided crucial help in debugging issues.
    
    \item Dennis DeCoste conceived the original key idea.
    
    \item Sean Lie coordinated the bring up of GPT on CS-2 and was involved in
    experimental validation and analysis.
    
    \item Shreyas Saxena advised the entire effort, brought up the initial proof of
    concept and experimented with different sparsity techniques.
    
    \item Shreyas Saxena and Sean Lie frequently met with Vithursan Thangarasa to discuss
    the work and helped revise several iterations of the manuscript.
\end{itemize}

\end{contributions}

\section{Acknowledgements}
We thank Anshul Samar, Dimitrios Sinodinos, and Joel Hestness, for helpful edits
and suggestions that improved the clarity of our manuscript.

\clearpage
\bibliography{refs}


%%%%%%%%%%%%%%%%%%%%%%%%%%%%%%%%%%%%%%%%%%%%%%%%%%%%%%%%%%%%%%%%%%%%%%%%%%%%%%%
%%%%%%%%%%%%%%%%%%%%%%%%%%%%%%%%%%%%%%%%%%%%%%%%%%%%%%%%%%%%%%%%%%%%%%%%%%%%%%%
% APPENDIX
%%%%%%%%%%%%%%%%%%%%%%%%%%%%%%%%%%%%%%%%%%%%%%%%%%%%%%%%%%%%%%%%%%%%%%%%%%%%%%%
%%%%%%%%%%%%%%%%%%%%%%%%%%%%%%%%%%%%%%%%%%%%%%%%%%%%%%%%%%%%%%%%%%%%%%%%%%%%%%%
% \clearpage \newpage \appendix \section{Appendix for Proofs}

\paragraph{Proof of Theorem \ref{thm:main}.}

\begin{proof}
\label{proof:main}
Our proof has two steps. In Step 1, we will show that SimCLR is equivalent to minimizing the cross entropy loss defined in Eqn.~(\ref{eqn:cross-entropy}). 
In Step 2, we will show  that minimizing the cross-entropy loss 
is equivalent to spectral clustering on $\bfpi$. 
Combining the two steps together, we have proved our theorem. 

\textbf{Step 1: } SimCLR is equivalent to minimizing the cross entropy loss.

The cross-entropy loss takes expectation over 
$\bfW_\bfX\sim \mathbb{P}(\cdot ; \bfpi)$, 
which means $\bfW_\bfX$ has exactly one non-zero entry in each row $i$. By Lemma~\ref{lem:multinomial}, we know every row $i$ of $\bfW_\bfX$ is independent of other rows. Moreover, 
$\bfW_{\bfX,i}\sim \mathcal{M}(1, \bfpi_i/\sum_j \bfpi_{i,j})=\mathcal{M}(1, \bfpi_i)$, because $\bfpi_i$ itself is a probability distribution.
Similarly, we know $\bfW_\bfZ$ also has the row-independent property by sampling over $\mathbb{P}(\cdot;\bfK_\bfZ)$.
Therefore, by Lemma~\ref{lem:cross_split}, we know Eqn.~(\ref{eqn:cross-entropy}) is equivalent to:
\[
 -\sum_{i=1}^n \mathbb{E}_{\bfW_{\bfX,i}}[\log \mathbb{P}(\bfW_{\bfZ,i}=\bfW_{\bfX,i};\bfK_\bfZ)],
\]

This expression takes expectation over $\bfW_{\bfX,i}$ for the given row $i$. Notice that 
$\bfW_{\bfX,i}$ has exactly one non-zero entry, which equals $1$ (same for $\bfW_{\bfZ,i}$). 
As a result
we expand the above expression to be:
\begin{equation}
 -\sum_{i=1}^n \sum_{j\neq i} \Pr(\bfW_{\bfX,i,j}=1)\log \Pr(\bfW_{\bfZ,i,j}=1).
\label{eqn:detailed-expansion}    
\end{equation}


By Lemma~\ref{lem:multinomial}, $\Pr(\bfW_{\bfZ,i,j}=1)=\bfK_{\bfZ,i,j}/\|\bfK_{\bfZ,i}\|_1$ for $j\neq i$. Recall that $\bfK_\bfZ=(k(\bfZ_i-\bfZ_j))_{(i,j)\in[n]^2}$, which means 
$\bfK_{\bfZ,i,j}/\|\bfK_{\bfZ,i}\|_1=\frac{\exp(-\|\bfZ_i-\bfZ_j\|^2/{2\tau})}{\sum_{k\neq i}
\exp(-\|\bfZ_i-\bfZ_k\|^2/{2\tau})
}$ for $j\neq i$, when $k$ is the Gaussian kernel with variance $\tau$. 

Notice that $\bfZ_i=f(\bfX_i)$, so we know
\begin{equation}
-\log \Pr(\bfW_{\bfZ,i,j}=1)=
-\log \frac{\exp(-\|f(\bfX_i)-f(\bfX_j)\|^2/{2\tau})}{\sum_{k\neq i}
\exp(-\|f(\bfX_i)-f(\bfX_k)\|^2/{2\tau}),
}
\label{eqn:infonce-equivalence}    
\end{equation}


The right hand side is exactly the InfoNCE loss defined in Eqn.~(\ref{eqn:infonce}).
Inserting Eqn.~(\ref{eqn:infonce-equivalence}) into Eqn.~(\ref{eqn:detailed-expansion}), we get the SimCLR algorithm, which first samples augmentation pairs $(i,j)$ with $\Pr(\bfW_{\bfX,i,j}=1)$ for each row $i$, and then optimize the InfoNCE loss. 

\textbf{Step 2: } minimizing the cross entropy loss 
is equivalent to spectral clustering on $\bfpi$.


By Lemma~\ref{lem:convert_to_spectral}, we may further convert the loss to 
\begin{equation}
\label{eqn:main-theorem-repul-attr}
\min_{\bfZ}
-\sum_{(i,j)\in [n]^2} \mathbf{P}_{i,j}
\log k (\bfZ_i-\bfZ_j)+\log \mathbf{R}(\bfZ).
\end{equation}
Since $k$ is the Gaussian kernel, this reduces to \[
\min_\bfZ \mathrm{tr}(\bfZ^\top \mathbf{L}(\bfpi) \bfZ)
+\log \mathbf{R}(\bfZ),
\]

where we use the fact that $\mathbb{E}_{\bfW_\bfX\sim \mathbb{P}(\cdot; \bfpi)}[\mathbf{L}(\bfW_\bfX)]
=\mathbf{L}(\bfpi)
$, because the Laplacian operator is linear and $
\mathbb{E}_{\bfW_\bfX\sim \mathbb{P}(\cdot; \bfpi)}(\bfW_\bfX)=\bfpi
$.
\end{proof}

\paragraph{Proof of Theorem \ref{thm:clip}.}
\begin{proof}
Since $\bfW_\bfX\sim \mathbb{P}(\cdot;\bfpi_{\mathbf{A}, \mathbf{B}})$, we know 
$\bfW_\bfX$ has exactly one non-zero entry in each row, denoting the pair that got sampled. 
A notable difference compared to the previous proof is we now have $n_\mathcal{A}+n_\mathcal{B}$ objects in our graph. CLIP deals with this by taking a mini-batch of size $2N$, 
such that $n_\mathcal{A}=n_\mathcal{B}=N$, and adding the $2N$ InfoNCE losses together. We label the objects in $\mathcal{A}$ as $[n_\mathcal{A}]$, and the objects in $\mathcal{B}$ as $\{n_\mathcal{A}+1, \cdots, n_\mathcal{A}+n_\mathcal{B}\}$. 

Notice that $\bfpi_{\mathbf{A}, \mathbf{B}}$ is a bipartite graph, so the edges of objects in $\mathcal{A}$ will only connect to object in $\mathcal{B}$ and vice versa. We can define the similarity matrix in $\cZ$ as $\bfK_\bfZ$, 
where $\bfK_\bfZ(i, j+n_\mathcal{A})=\bfK_\bfZ(j+n_\mathcal{A},i)= k(\bfZ_i-\bfZ_j)$ for $i\in [n_\mathcal{A}], j\in [n_\mathcal{B}]$, and otherwise we set $\bfK_\bfZ(i,j)=0$. 
The rest is same as the previous proof. 
\end{proof}

\paragraph{Proof of Theorem \ref{thm:exponential}.}

\begin{proof}
\label{proof:exponential}
Since the objective function consists of a linear term combined with an entropy regularization, which is a strongly concave function, the maximization problem is a convex optimization problem. Owing to the implicit constraints provided by the entropy function, the problem is equivalent to having only the equality constraint. We then introduce the Lagrangian multiplier $\lambda$ and obtain the following relaxed problem:

$$
\widetilde{E}(\boldsymbol{\alpha})=\psi_{1}-\sum_{i=1}^n \alpha_{i} \psi_{i}+\tau \sum_{i=1}^n \alpha_{i}\log \alpha_{i}+\lambda\left(\boldsymbol{\alpha}^{\top} \mathbf{1}_n-1\right).
$$

As the relaxed problem is unconstrained, taking the derivative with respect to $\alpha_{i}$ yields

$$
\frac{\partial \widetilde{E}(\boldsymbol{\alpha})}{\partial \alpha_{i}}=-\psi_{i}+\tau\left(\log \alpha_{i}+\alpha_{i} \frac{1}{\alpha_{i}}\right)+\lambda=0.
$$

Solving the above equation implies that $\alpha_{i}$ takes the form
$
\alpha_{i}=\exp \left(\frac{1}{\tau} \psi_{i}\right) \exp \left(\frac{-\lambda}{\tau}-1\right).
$ Since $\alpha_{i}$ lies on the probability simplex, the optimal $\alpha_{i}$ is explicitly given by
$
\alpha^{*}_{i}=\frac{\exp \left(\frac{1}{\tau} \psi_{i}\right)}{\sum_{i^{\prime}=1}^n \exp \left(\frac{1}{\tau} \psi_{i^{\prime}}\right)} .
$ Substituting the optimal point into the objective function, we obtain
$$
\begin{aligned}
E\left(\boldsymbol{\alpha}^*\right)  &=\psi_1-\sum_{i=1}^n \frac{\exp \left(\frac{1}{\tau} \psi_{i}\right)}{\sum_{i^{\prime}=1}^n \exp \left(\frac{1}{\tau} \psi_{i^{\prime}}\right)} \psi_{i}+\tau \sum_{i=1}^n \frac{\exp \left(\frac{1}{\tau} \psi_{i}\right)}{\sum_{i^{\prime}=1}^n \exp \left(\frac{1}{\tau} \psi_{i^{\prime}}\right)}\log \frac{\exp \left(\frac{1}{\tau} \psi_{i}\right)}{\sum_{i^{\prime}=1}^n \exp \left(\frac{1}{\tau} \psi_{i^{\prime}}\right)} \\
& =\psi_1 - \tau \log \left(\sum_{i=1}^n \exp \left(\frac{1}{\tau} \psi_{i}\right)\right).
\end{aligned}
$$
Thus, the Lagrangian dual function is given by
\begin{equation*}
-E\left(\boldsymbol{\alpha}^*\right)= -\tau \log \frac{\exp \left(\frac{1}{\tau} \psi_{1}\right)}{\sum_{i=1}^n \exp \left(\frac{1}{\tau} \psi_{i}\right)}.\qedhere
\end{equation*}
\end{proof}



\section{More on Experiments} \label{section: experiment_details}

\paragraph{CIFAR-10 and CIFAR-100} CIFAR-10 ~\citep{krizhevsky2009learning} and CIFAR-100 ~\citep{krizhevsky2009learning} are well-known classic image classification datasets. Both CIFAR-10 and CIFAR-100 contain a total of 60k $32 \times 32$ labeled images of different classes, with 50k for training and 10k for testing. CIFAR-10 is similar to CIFAR-100, except there are 10 different classes in CIFAR-10 and 100 classes in CIFAR-100.

\paragraph{TinyImageNet} TinyImageNet ~\citep{le2015tiny} is a subset of ImageNet ~\citep{deng2009imagenet}. There are 200 different object classes in TinyImageNet, with 500 training images, 50 validation images, and 50 test images for each class. All the images in TinyImageNet are colored and labeled with a size of $64 \times 64$.

\textbf{Pseudo-code.} Algorithm \ref{alg:Training Procedure} presents the pseudo-code for our empirical training procedure.

\begin{algorithm}[!htbp]
\caption{Training Procedure}
\label{alg:Training Procedure}
\begin{algorithmic}[1]
\REQUIRE trainable encoder network $f$, batch size $N$, augmentation strategy \textit{aug}, loss function $L$ with hyperparameters \textit{args}
\FOR {sampled minibatch ${x_i}_{i=1}^N$}
\FORALL{$i \in { 1, ..., N }$}
\STATE draw two augmentations $t_i = \textit{aug}\left(x_i\right) $, $t_i' = \textit{aug}\left(x_i\right) $
\STATE $z_i = f\left(t_i\right)$, $z_i' = f\left(t_i'\right)$
\ENDFOR
\STATE compute loss $\mathcal{L} = L(N, z, z', \textit{args})$
\STATE update encoder network $f$ to minimize $\mathcal{L}$
\ENDFOR
\STATE \textbf{Return} encoder network $f$
\end{algorithmic}
\end{algorithm}

We also provide the pseudo-code for our core loss function used in the training procedure in Algorithm \ref{alg:Core loss}. The pseudo-code is almost identical to SimCLR's loss function, with the exception of an extra parameter $\gamma$.

\begin{algorithm}[!htbp]
\caption{Core loss function $\mathcal{C}$}
\label{alg:Core loss}
\begin{algorithmic}[1]
\REQUIRE batch size $N$, two encoded minibatches $z_1, z_2$, $\gamma$, temperature $\tau$
\STATE $z = \textit{concat}\left(z_1, z_2\right)$
\FOR {$i \in {1, ..., 2N }, j \in {1, ..., 2N}$ }
\STATE $s_{i,j} = \Vert z_i - z_j \Vert_2^{\gamma}$
\ENDFOR
\STATE \textbf{define} $l(i, j)$ \textbf{as} $l(i, j) = - \log \frac{exp\left(s_{i,j}/\tau \right)}{\sum_{k=1}^{2N} \mathbf{1}{[k \ne i]} exp\left(s{i, j} / \tau \right)} $
\STATE \textbf{Return} $\frac{1}{2N} \sum_{k=1}^N\left[l(i, i+N) + l(i+N, i)\right]$
\end{algorithmic}
\end{algorithm}

Utilizing the core loss function $\mathcal{C}$, we can define all kernel loss functions used in our experiments in Table \ref{table: loss definition}. For all $z_i \in z$ with even dimensions $n$, we define $z_{L_i} = z_i\left[0:n/2\right]$ and $z_{R_i} = z_i\left[n/2:n\right]$.

\begin{table}[ht]
\centering
\begin{tabular}{{@{}l|l@{}}}
Kernel  &  Loss function \\ \midrule
Laplacian & $\mathcal{C}\left(N, z, z', \gamma=1, \tau\right)$\\ \midrule
Sum       & $\lambda * \mathcal{C}\left(N, z, z', \gamma=1, \tau_1\right) + (1-\lambda) * \mathcal{C}\left(N, z, z', \gamma=2, \tau_2\right)$  \\ \midrule
Concatenation Sum&$\lambda * \mathcal{C}\left(N, z_L, z'_L, \gamma=1, \tau_1\right) + (1-\lambda) * \mathcal{C}\left(N, z_R, z'_R, \gamma=2, \tau_2\right)$\\ \midrule
$\gamma = 0.5$ & $\mathcal{C}\left(N, z, z', \gamma=0.5, \tau\right)$          \\ 

\end{tabular}

\caption{Definition of kernel loss functions in our experiments}
\label {table: loss definition}
\end{table}

\textbf{Baselines.} We reproduce the SimCLR algorithm using PyTorch Lightning~\citep{PytorchLightning}.

\textbf{Encoder details.}
The encoder $f$ consists of a backbone network and a projection network. We employ ResNet50~\citep{ResNet} as the backbone and a 2-layer MLP (connected by a batch normalization~\citep{ioffe2015batch} layer and a ReLU \cite{nair2010rectified} layer) with hidden dimensions 2048 and output dimensions 128 (or 256 in the concatenation kernel case).

\textbf{Encoder hyperparameter tuning.}
For each encoder training case, we randomly sample 500 hyperparameter groups (sample details are shown in Table \ref{table: Hyperparameter sample}) and train these samples simultaneously using Ray Tune ~\citep{RayTune}, with the ASHA scheduler~\citep{li2018massively}. Ultimately, the hyperparameter group that maximizes the online validation accuracy (integrated in PyTorch Lightning) within 5000 validation steps is chosen for the given encoder training case.

\begin{table}[ht]
\centering

\begin{tabular}{@{}l|l|l@{}}
\midrule
Hyperparameter  & Sample Range & Sample Strategy \\ \midrule
start learning rate & $\left[10^{-2}, 10\right]$ & log uniform \\ \midrule
$\lambda$       & $\left[0, 1\right]$ & uniform \\ \midrule
$\tau$, $\tau_1$, $\tau_2$ & $\left[0, 1\right]$ & log uniform \\ \midrule
\end{tabular}

\caption{Hyperparameters sample strategy}
\label {table: Hyperparameter sample}
\end{table}

\textbf{Encoder training.} 
We train each encoder using the LARS optimizer~\citep{LARSOptimizer}, LambdaLR Scheduler in PyTorch, momentum 0.9, weight decay $10^{-6}$, batch size 256, and the aforementioned hyperparameters for 400 epochs on a single A-100 GPU.

\textbf{Image transformation.} The image transformation strategy, including augmentation, is identical to the default transformation strategy provided by PyTorch Lightning.

\textbf{Linear evaluation.}
The linear head is trained using the SGD optimizer with a cosine learning rate scheduler, batch size 64, and weight decay $10^{-6}$ for 100 epochs. The learning rate starts at $0.3$ and ends at $0$.

\textbf{Moco Experiments.} We also tested our method based on MoCo~\citep{he2019moco}. The results are summarized in Table \ref{tab:results-moco}. Here we choose ResNet18~\citep{ResNet} as the backbone and set a temperature of $0.1$ as default. For our simple sum kernel, we set $\lambda=0.8$. The results show that our method outperforms the original MoCo method.

\begin{table}[thb]
\centering
\caption{MoCo Experiment Results on CIFAR-10 and CIFAR-100.}
\label{tab:results-moco}
\resizebox{\textwidth}{!}{%
\begin{tabular}{@{}c|ccc|ccc@{}}
\toprule
\multirow{3}{*}{Method} & \multicolumn{3}{c|}{CIFAR-10} & \multicolumn{3}{c}{CIFAR-100} \\ \cmidrule(lr){2-4} \cmidrule(lr){5-7} 
                        & 200 epochs & 400 epochs    & 1000 epochs   & 200 epochs & 400 epochs & 1000 epochs         \\ \midrule
MoCo (repro.)         & $76.41 \pm 0.12$    & $80.01 \pm 0.15$          & $84.45 \pm 0.08$    & $\mathbf{47.02 \pm 0.11}$ & $52.50 \pm 0.07$ & $57.62 \pm 0.15$            \\
\midrule
Laplacian Kernel        & ${78.09 \pm 0.10}$    & $\mathbf{83.85 \pm 0.09}$          & $\mathbf{88.34 \pm 0.16}$    & $46.12 \pm 0.22$   & $53.44 \pm 0.17$ & $59.10 \pm 0.14$        \\
Simple Sum Kernel & $\mathbf{78.12 \pm 0.15}$   & $83.23 \pm 0.18$ & $87.50 \pm 0.20$ & $46.65 \pm 0.06$ & $\mathbf{53.62 \pm 0.19}$ & $\mathbf{59.83 \pm 0.12}$\\
\bottomrule
\end{tabular}
}
\end{table}



\section{More Experiments on Synthetic Data}


Consider a scenario with $n$ clusters, each containing $k$ vertices. Let the probability of vertices $u$ and $v$ from the same cluster belonging to $\bfpi$ be $p$. Conversely, for vertices $u$ and $v$ from different clusters, let the probability of belonging to $\pi$ be $q$. We generate the graph $\bfpi$ randomly, based on $p$ and $q$. We experiment with values of $k=100$ and $n=6$ for ease of visualization, embedding all points in a two-dimensional space. Each vertex's initial position originates from a normal distribution. In each iteration, we sample a subgraph of $\bfpi$ uniformly, ensuring each vertex has an out-degree of $1$. We then optimize the corresponding vectors using InfoNCE loss with an SGD optimizer and iterate until convergence. Our experimental setup consists of an SGD learning rate of $1$, an InfoNCE loss temperature of $0.5$, and a batch size of $50$. We evaluate two scenarios with different $p$ and $q$ values: $p=1$, $q=0$, and $p=0.75$, $q=0.2$. The results of these experiments are visualized in Figure \ref{fig:vis-spectral-cluster}. The obtained embeddings exhibit the hallmark pattern of spectral clustering of graph $\bfpi$.

\begin{figure}[!tb]
\centering
\subfigure{
\includegraphics[width=1\textwidth]{Figures/cluster_pi.png}
\label{fig:vis-cluster}
}
\subfigure{
\includegraphics[width=1\textwidth]{Figures/noised_cluster_pi.png}
\label{fig:vis-noised-cluster}
}
\caption{Visualizations of the optimization process using InfoNCE Loss on the vectors corresponding to $\bfpi$. Points of identical color belong to the same cluster within $\bfpi$. To showcase the internal structure of $\bfpi$, we randomly select 10 vertices from each cluster to display the edge distribution of $\bfpi$.}
\label{fig:vis-spectral-cluster}
\end{figure}


\clearpage
\includepdf[pages=-]{thangarasa_445-supp.pdf}

\end{document}
