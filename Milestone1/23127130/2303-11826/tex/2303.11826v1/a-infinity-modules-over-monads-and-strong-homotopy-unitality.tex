\documentclass[12pt,a4paper,reqno]{amsart}
\title[$\Ainfty$-structures in monoidal DG cats \& strong
homotopy unitality]{$\Ainfty$-structures in monoidal DG categories and \\  strong homotopy unitality}
\author{Rina Anno}
\email{ranno@math.ksu.edu}
\address{Department of Mathematics \\
Kansas State University \\
138 Cardwell Hall \\
Manhattan, KS 66506\\
USA}
\author{Sergey Arkhipov}
\email{hippie@math.au.dk}
\address{Matematisk Institut, Aarhus Universitet, Ny Munkegade, DK-8000,
Aarhus C, Denmark}
\author{Timothy Logvinenko} 
\email{LogvinenkoT@cardiff.ac.uk} 
\address{School of Mathematics\\ 
Cardiff University\\
Senghennydd Road,\\
Cardiff, CF24 4AG\\
UK}
\usepackage{amsmath,amsfonts,amssymb,amsthm,epsfig,amscd,latexsym,comment}
\usepackage{caption}
\usepackage{MnSymbol}
\usepackage{tikz}
\usetikzlibrary{cd}
\usepackage{graphicx}
\usepackage{array}
\usepackage{subfigure}
\usepackage{leftidx}
\usepackage{xparse}% http://ctan.org/pkg/xparse

\usepackage[colorlinks=true, pdfpagemode=none, pdfmenubar=false, linkcolor=blue, citecolor=blue, urlcolor=blue]{hyperref}

\let\amsamp=&

\begingroup
\catcode`\&=13
\gdef\smallampmatrix{%
  \begingroup
  \let&=\amsamp
  \begin{smallmatrix}%
}
\gdef\endsmallampmatrix{\end{smallmatrix}\endgroup}
\endgroup

\addtolength{\voffset}{-1cm}
\addtolength{\textheight}{2cm}
\addtolength{\hoffset}{-1cm}
\addtolength{\textwidth}{2cm}

\DeclareMathOperator{\codim}{codim}
\DeclareMathOperator{\krn}{Ker}
\DeclareMathOperator{\adj}{Adj}
\DeclareMathOperator{\cok}{Coker}
\DeclareMathOperator{\chr}{char}
\DeclareMathOperator{\img}{Im}
\DeclareMathOperator{\iden}{Id}
\DeclareMathOperator{\obj}{Ob}
\DeclareMathOperator{\mor}{Mor}
\DeclareMathOperator{\homm}{Hom}
\DeclareMathOperator{\shhomm}{{\it\mathcal{H}om\rm}}
\DeclareMathOperator{\eend}{End}
\DeclareMathOperator{\seend}{\it \mathcal{E}nd}
\DeclareMathOperator{\autm}{Aut}
\DeclareMathOperator{\AbGp}{\bf AbGp}
\DeclareMathOperator{\rat}{Rat}
\DeclareMathOperator{\dmn}{Dom}
\DeclareMathOperator{\gl}{GL}
\DeclareMathOperator{\gsl}{SL}
\DeclareMathOperator{\gsp}{Sp}
\DeclareMathOperator{\Supp}{Supp}
\DeclareMathOperator{\picr}{Pic}
\DeclareMathOperator{\tot}{Tot}
\DeclareMathOperator{\fof}{fof}
\DeclareMathOperator{\divr}{Div}
\DeclareMathOperator{\cl}{Cl}
\DeclareMathOperator{\car}{Car}
\DeclareMathOperator{\ratf}{Rat}
\DeclareMathOperator{\spec}{Spec}
\DeclareMathOperator{\relspec}{{\mathit{Spec}}}
\DeclareMathOperator{\proj}{Proj\;}
\DeclareMathOperator{\rproj}{\bf Proj\;}
\DeclareMathOperator{\hilb}{Hilb}
\DeclareMathOperator{\diag}{diag}
\DeclareMathOperator{\ord}{ord}
\DeclareMathOperator{\reg}{{reg}}
\DeclareMathOperator{\given}{given}
\DeclareMathOperator{\com}{com}
\DeclareMathOperator{\ext}{Ext}
\DeclareMathOperator{\res}{Res}
\DeclareMathOperator{\irr}{Irr}
\DeclareMathOperator{\tor}{Tor}
\DeclareMathOperator{\supp}{Supp}
\DeclareMathOperator{\supr}{sup}
\DeclareMathOperator{\ann}{Ann}
\DeclareMathOperator{\eval}{ev}
\DeclareMathOperator{\trace}{tr}
\DeclareMathOperator{\composition}{cmps}
\DeclareMathOperator{\action}{act}
\DeclareMathOperator{\val}{\it val}
\DeclareMathOperator{\fract}{frac}
\DeclareMathOperator{\ramm}{Ram}
\DeclareMathOperator{\prdiv}{div}
\DeclareMathOperator{\nil}{nil}
\DeclareMathOperator{\qcohcat}{QCoh}
\DeclareMathOperator{\cohcat}{Coh}
\DeclareMathOperator{\modd}{\bf Mod}
\DeclareMathOperator{\nod}{\bf Nod}
\DeclareMathOperator{\lder}{\bf L}
\DeclareMathOperator{\rder}{\bf R}
\DeclareMathOperator{\ldertimes}{\overset{\lder}{\otimes}}
\DeclareMathOperator{\rderhom}{\rder\homm}
\DeclareMathOperator{\rdershom}{\rder\shhomm}
\DeclareMathOperator{\rank}{rk}
\DeclareMathOperator{\ptfield}{\bf k}
\DeclareMathOperator{\ass}{Ass}
\DeclareMathOperator{\prjdim}{proj\;dim}
\DeclareMathOperator{\depth}{depth}
\DeclareMathOperator{\cxampl}{coh-amp}
\DeclareMathOperator{\torampl}{Tor-amp}
\DeclareMathOperator{\tordim}{Tor-dim}
\DeclareMathOperator{\nonorth}{N}
\DeclareMathOperator{\frg}{\bf Frg}
\DeclareMathOperator{\proobjet}{pro}
\DeclareMathOperator{\indobjet}{ind}
\DeclareMathOperator{\id}{Id}
\DeclareMathOperator{\hex}{Hex}
\DeclareMathOperator{\Except}{Exc}
\DeclareMathOperator{\sinksource}{SS}
\DeclareMathOperator{\fintype}{\mathcal{F}\mathcal{T}}
\DeclareMathOperator{\dperf}{DP}
\DeclareMathOperator{\vectspaces}{\bf Vect}
\DeclareMathOperator{\cone}{Cone}
\DeclareMathOperator{\opp}{{opp}}
\DeclareMathOperator{\fg}{{\it fg}}
\DeclareMathOperator{\qrep}{\it \mathcal{Q}r}
\DeclareMathOperator{\hproj}{\mathcal{P}}
\DeclareMathOperator{\hinj}{\mathcal{I}}
\DeclareMathOperator{\bhproj}{\bar{\mathcal{P}}}
\DeclareMathOperator{\acyc}{\it \mathcal{A}c}
\DeclareMathOperator{\freemod}{{\mathcal{F}\emph{ree}}}
\DeclareMathOperator{\intmod}{{\it \mathcal{I}nt}}
\DeclareMathOperator{\semifree}{\mathcal{S}\mathcal{F}}
\DeclareMathOperator{\sffg}{\mathcal{S}\mathcal{F}_{\fg}}
\DeclareMathOperator{\perf}{{\it \mathcal{P}erf}}
\DeclareMathOperator{\hperf}{{\it h\mathcal{P}erf}}
\DeclareMathOperator{\hmtpy}{{Ho}}
\DeclareMathOperator{\Morita}{{Mrt}}
\DeclareMathOperator{\triag}{{Tr}}
\DeclareMathOperator{\tria}{{Tria}}
\DeclareMathOperator{\twcx}{{Tw}}
\DeclareMathOperator{\twbicx}{{Twbi}}
\DeclareMathOperator{\pretriag}{{Pre\text{-}Tr}}
\DeclareMathOperator{\DGFun}{{DGFun}}
\DeclareMathOperator{\DGFuntwocat}{{\bf DGFun}}
\DeclareMathOperator{\Fun}{{Fun}}
\DeclareMathOperator{\exfun}{{ExFun}}
\DeclareMathOperator{\TPair}{{\bf TPair}}
\DeclareMathOperator{\alg}{{\bf Alg}}
\DeclareMathOperator{\cts}{{cts}}
\DeclareMathOperator{\conv}{{Conv}}
\DeclareMathOperator{\strict}{{strict}}
\DeclareMathOperator{\unit}{{unit}}
\DeclareMathOperator{\counit}{{counit}}
\DeclareMathOperator{\fmcatweak}{{\it \mathcal{F}\mathcal{M}uk}}
\DeclareMathOperator{\fmcatstrict}{{\it \mathcal{F}\mathcal{M}uk}_{str}}
\DeclareMathOperator{\flag}{Fl}
\DeclareMathOperator{\grass}{Gr}
\DeclareMathOperator{\partition}{Prt}
\DeclareMathOperator{\braidgp}{Br}
\DeclareMathOperator{\gbrcat}{\mathcal{G}\mathcal{B}\emph{r}}
\DeclareMathOperator{\web}{{\mathcal{W}\emph{eb}}}
\DeclareMathOperator{\forget}{{\bf Forg}}
\DeclareMathOperator{\free}{{\bf Free}}
\DeclareMathOperator{\barfrg}{{\mathcal{F}\overline{\emph{org}}}}
\DeclareMathOperator{\DGMod}{\it{\mathcal{D}\mathcal{G}\mathcal{M}od}}
\DeclareMathOperator{\DGModtwocat}{{\bf{DGMod}}}
\DeclareMathOperator{\BarModtwocat}{\bf{DG\overline{Mod}}}
\DeclareMathOperator{\yoneda}{Yoneda}
\DeclareMathOperator{\eilmoor}{{\mathcal{E}}}
\DeclareMathOperator{\coeilmoor}{{co\mathcal{E}}}
\DeclareMathOperator{\kleisli}{{\mathcal{K}\mathcal{S}}}
\DeclareMathOperator{\cokleisli}{{co\mathcal{K}\mathcal{S}}}
\DeclareMathOperator{\cxrow}{{\mathcal{C}xrow}}
\DeclareMathOperator{\cxcol}{{\mathcal{C}xcol}}

% Appendix stuff

\DeclareMathOperator{\ima}{Im}
\DeclareMathOperator{\Hom}{Hom}
\DeclareMathOperator{\aut}{Aut}
\DeclareMathOperator{\gal}{Gal}
\DeclareMathOperator{\coker}{coker}

\begin{document}

\def\bv{\mathbf{v}}
\def\kgc_{K^*_G(\mathbb{C}^n)}
\def\kgchi_{K^*_\chi(\mathbb{C}^n)}
\def\kgcf_{K_G(\mathbb{C}^n)}
\def\kgchif_{K_\chi(\mathbb{C}^n)}
\def\gpic_{G\text{-}\picr}
\def\gcl_{G\text{-}\cl}
\def\trch_{{\chi_{0}}}
\def\regring{{R}}
\def\regrep{{V_{\text{reg}}}}
\def\givrep{{V_{\text{giv}}}}
\def\lbar{{(\mathbb{Z}^n)^\vee}}
\def\genpx_{{p_X}}
\def\genpy_{{p_Y}}
\def\genpcn_{p_{\mathbb{C}^n}}
\def\gnat{gnat}
\def\twalg{{\regring \rtimes G}}
\def\L{{\mathcal{L}}}
\def\gcd{\mbox{gcd}}
\def\lcm{\mbox{lcm}}
\def\tf{{\tilde{f}}}
\def\tD{{\tilde{D}}}
\def\A{{\mathcal{A}}}
\def\B{{\mathcal{B}}}
\def\C{{\mathcal{C}}}
\def\D{{\mathcal{D}}}
\def\E{{\mathcal{E}}}
\def\F{{\mathcal{F}}}
\def\H{{\mathcal{H}}}
\def\L{{\mathcal{L}}}
\def\M{{\mathcal{M}}}
\def\N{{\mathcal{N}}}
\def\R{{\mathcal{R}}}
\def\T{{\mathcal{T}}}
\def\RF{{\mathcal{R}\mathcal{F}}}
\def\barA{{\bar{\mathcal{A}}}}
\def\barAi{{\bar{\mathcal{A}}_1}}
\def\barAj{{\bar{\mathcal{A}}_2}}
\def\barB{{\bar{\mathcal{B}}}}
\def\barC{{\bar{\mathcal{C}}}}
\def\barD{{\bar{\mathcal{D}}}}
\def\barT{{\bar{\mathcal{T}}}}
\def\barM{{\bar{\mathcal{M}}}}
\def\Aopp{{\A^{\opp}}}
\def\Bopp{{\B^{\opp}}}
\def\Copp{{\C^{\opp}}}
\def\aA{\leftidx{_{a}}{\A}}
\def\bA{\leftidx{_{b}}{\A}}
\def\Aa{{\A_a}}
\def\Ea{E_a}
\def\aE{\leftidx{_{a}}{E}{}}
\def\Eb{E_b}
\def\bE{\leftidx{_{b}}{E}{}}
\def\Fa{F_a}
\def\aF{\leftidx{_{a}}{F}{}}
\def\Fb{F_b}
\def\bF{\leftidx{_{b}}{F}{}}
\def\aM{\leftidx{_{a}}{M}{}}
\def\aN{\leftidx{_{a}}{N}{}}
\def\aMb{\leftidx{_{a}}{M}{_{b}}}
\def\aNb{\leftidx{_{a}}{N}{_{b}}}
\def\rfRFa{\leftidx{_{\RF}}{\RF}{_{\A}}}
\def\aRFrf{\leftidx{_{\A}}{\RF}{_{\RF}}}
\def\biAMA{\leftidx{_{\A}}{M}{_{\A}}}
\def\biAMC{\leftidx{_{\A}}{M}{_{\C}}}
\def\biCMA{\leftidx{_{\C}}{M}{_{\A}}}
\def\biCMC{\leftidx{_{\C}}{M}{_{\C}}}
\def\biALA{\leftidx{_{\A}}{L}{_{\A}}}
\def\biALC{\leftidx{_{\A}}{L}{_{\C}}}
\def\biCLA{\leftidx{_{\C}}{L}{_{\A}}}
\def\biCLC{\leftidx{_{\C}}{L}{_{\C}}}
\def\Na{{N_a}}
\def\vectk{{\vectspaces\text{-}k}}
\def\vectkfg{{\vectspaces_{\text{fg}}\text{-}k}}
\def\modk{{\modd\text{-}k}}
\def\Amod{{\A\text{-}\modd}}
\def\modA{{\modd\text{-}\A}}
\def\modbar{{\overline{\modd}}}
\def\modbarA{{\overline{\modd}\text{-}\A}}
\def\modbarAopp{{\overline{\modd}\text{-}\Aopp}}
\def\modB{{\modd\text{-}\B}}
\def\modC{{\modd\text{-}\C}}
\def\modD{{\modd\text{-}\D}}
\def\modbarB{{\overline{\modd}\text{-}\B}}
\def\modbarC{{\overline{\modd}\text{-}\C}}
\def\modbarD{{\overline{\modd}\text{-}\D}}
\def\modbarBopp{{\overline{\modd}\text{-}\Bopp}}
\def\kmodk{{k\text-\modd\text{-}k}}
\def\bikmodk{{k_\bullet\text-\modd\text{-}k_\bullet}}
\def\AmodA{{\A\text{-}\modd\text{-}\A}}
\def\AmodM{{\A\text{-}\modd\text{-}\M}}
\def\AmodB{{\A\text{-}\modd\text{-}\B}}
\def\AmodT{{\A\text{-}\modd\text{-}\T}}
\def\BmodB{{\B\text{-}\modd\text{-}\B}}
\def\BmodA{{\B\text{-}\modd\text{-}\A}}
\def\DmodD{{\D\text{-}\modd\text{-}\D}}
\def\MmodA{{\M\text{-}\modd\text{-}\A}}
\def\MmodM{{\M\text{-}\modd\text{-}\M}}
\def\TmodA{{\T\text{-}\modd\text{-}\A}}
\def\TmodT{{\T\text{-}\modd\text{-}\T}}
\def\AmodbarA{\A\text{-}{\overline{\modd}\text{-}\A}}
\def\AmodbarB{\A\text{-}{\overline{\modd}\text{-}\B}}
\def\AmodbarC{\A\text{-}{\overline{\modd}\text{-}\C}}
\def\AmodbarD{\A\text{-}{\overline{\modd}\text{-}\D}}
\def\AmodbarM{\A\text{-}{\overline{\modd}\text{-}\M}}
\def\AmodbarT{\A\text{-}{\overline{\modd}\text{-}\T}}
\def\BmodbarA{\B\text{-}{\overline{\modd}\text{-}\A}}
\def\BmodbarB{\B\text{-}{\overline{\modd}\text{-}\B}}
\def\BmodbarC{\B\text{-}{\overline{\modd}\text{-}\C}}
\def\BmodbarD{\B\text{-}{\overline{\modd}\text{-}\D}}
\def\CmodbarA{\C\text{-}{\overline{\modd}\text{-}\A}}
\def\CmodbarB{\C\text{-}{\overline{\modd}\text{-}\B}}
\def\CmodbarC{\C\text{-}{\overline{\modd}\text{-}\C}}
\def\CmodbarD{\C\text{-}{\overline{\modd}\text{-}\D}}
\def\DmodbarA{\D\text{-}{\overline{\modd}\text{-}\A}}
\def\DmodbarB{\D\text{-}{\overline{\modd}\text{-}\B}}
\def\DmodbarC{\D\text{-}{\overline{\modd}\text{-}\C}}
\def\DmodbarD{\D\text{-}{\overline{\modd}\text{-}\D}}
\def\TmodbarA{\T\text{-}{\overline{\modd}\text{-}\A}}
\def\MmodbarA{\M\text{-}{\overline{\modd}\text{-}\A}}
\def\MmodbarM{\M\text{-}{\overline{\modd}\text{-}\M}}
\def\modbarT{{\overline{\modd}\text{-}T}}
\def\freeA{{\free\text{-}A}}
\def\freeAS{{\free_S\text{-}A}}
\def\freepfA{{\free_{pf}\text{-}A}}
\def\Afree{{A\text{-}\free}}
\def\AfreeA{{A\text{-}\free\text{-}A}}
\def\AfreeB{{A\text{-}\free\text{-}B}}
\def\sfA{{\semifree(\A)}}
\def\sfB{{\semifree(\B)}}
\def\sffgA{{\sffg(\A)}}
\def\sffgB{{\sffg(\B)}}
\def\hprojA{{\hproj(\A)}}
\def\hprojB{{\hproj(\B)}}
\def\qrepA{{\qrep(\A)}}
\def\qrepB{{\qrep(\B)}}
\def\opp{{\text{opp}}}
\def\Aperf{{\A^{\text{pf}}}}
\def\hperfA{{\hperf(\A)}}
\def\hperfB{{\hperf(\B)}}
\def\barperf{{\it\mathcal{P}\overline{er}f}}
\def\barperfA{{\barperf\text{-}\A}}
\def\barperfB{{\barperf\text{-}\B}}
\def\barperfC{{\barperf\text{-}\C}}
\def\qrhpr{{\hproj^{qr}}}
\def\qrhprA{{\qrhpr(\A)}}
\def\qrhprB{{\qrhpr(\B)}}
\def\qrsf{{\semifree^{qr}}}
\def\qrsf{{\semifree^{qr}}}
\def\qrsfA{{\qrsf(\A)}}
\def\qrsfB{{\qrsf(\B)}}
\def\Aperfsf{{\semifree^{\A\text{-}\perf}(\AbimB)}}
\def\Bperfsf{{\semifree^{\B\text{-}\perf}(\AbimB)}}
\def\Aprfhpr{{\hproj^{\A\text{-}\perf}(\AbimB)}}
\def\Bprfhpr{{\hproj^{\B\text{-}\perf}(\AbimB)}}
\def\Aqrhpr{{\hproj^{\A\text{-}qr}(\AbimB)}}
\def\Bqrhpr{{\hproj^{\B\text{-}qr}(\AbimB)}}
\def\Aqrsf{{\semifree^{\A\text{-}qr}(\AbimB)}}
\def\Bqrsf{{\semifree^{\B\text{-}qr}(\AbimB)}}
\def\modAopp{{\modd\text{-}\Aopp}}
\def\modBopp{{\modd\text{-}\Bopp}}
\def\AmodA{{\A\text{-}\modd\text{-}\A}}
\def\AmodB{{\A\text{-}\modd\text{-}\B}}
\def\AmodC{{\A\text{-}\modd\text{-}\C}}
\def\BmodA{{\B\text{-}\modd\text{-}\A}}
\def\BmodB{{\B\text{-}\modd\text{-}\B}}
\def\BmodC{{\B\text{-}\modd\text{-}\C}}
\def\CmodA{{\C\text{-}\modd\text{-}\A}}
\def\CmodB{{\C\text{-}\modd\text{-}\B}}
\def\CmodC{{\C\text{-}\modd\text{-}\C}}
\def\CmodD{{\C\text{-}\modd\text{-}\D}}
\def\AbimA{{\A\text{-}\A}}
\def\AbimC{{\A\text{-}\C}}
\def\AbimM{{\A\text{-}\M}}
\def\BbimA{{\B\text{-}\A}}
\def\BbimB{{\B\text{-}\B}}
\def\BbimC{{\B\text{-}\C}}
\def\BbimD{{\B\text{-}\D}}
\def\CbimA{{\C\text{-}\A}}
\def\CbimB{{\C\text{-}\B}}
\def\CbimC{{\C\text{-}\C}}
\def\DbimA{{\D\text{-}\A}}
\def\DbimB{{\D\text{-}\B}}
\def\DbimC{{\D\text{-}\C}}
\def\DbimD{{\D\text{-}\D}}
\def\MbimA{{\M\text{-}\A}}
\def\AhprA{{\hproj\left(\AbimA\right)}}
\def\BhprB{{\hproj\left(\BbimB\right)}}
\def\AhprB{{\hproj\left(\AbimB\right)}}
\def\BhprA{{\hproj\left(\BbimA\right)}}
\def\AbarA{{\overline{\A\text{-}\A}}}
\def\AbarB{{\overline{\A\text{-}\B}}}
\def\BbarA{{\overline{\B\text{-}\A}}}
\def\BbarB{{\overline{\B\text{-}\B}}}
\def\QAbimB{{Q\A\text{-}\B}}
\def\AbimB{{\A\text{-}\B}}
\def\AbimC{{\A\text{-}\C}}
\def\AonebimB{{\A_1\text{-}\B}}
\def\AtwobimB{{\A_2\text{-}\B}}
\def\BbimA{{\B\text{-}\A}}
\def\MddA{{M^{\tilde{\A}}}}
\def\MddB{{M^{\tilde{\B}}}}
\def\MhdA{{M^{h\A}}}
\def\MhdB{{M^{h\B}}}
\def\NhdB{{N^{h\B}}}
\def\Cat{{\it \mathcal{C}at}}
\def\twoCat{{\it 2\text{-}\;\mathcal{C}at}}
\def\DGCat{{DG\text{-}Cat}}
\def\HoDGCat{{\hmtpy(\DGCat)}}
\def\HoDGCatV{{\hmtpy(\DGCat_\mathbb{V})}}
\def\MoDGCat{{Mo(\DGCat)}}
\def\tr{{tr}}
\def\pretr{{pretr}}
\def\kctr{{kctr}}
\def\PreTrCat{{\DGCat^\pretr}}
\def\KcTrCat{{\DGCat^\kctr}}
\def\HoPretrCat{{\hmtpy(\PreTrCat)}}
\def\HoKcTrCat{{\hmtpy(\KcTrCat)}}
\def\Aquasirep{{\A\text{-}qr}}
\def\QAquasirep{{Q\A\text{-}qr}}
\def\Bquasirep{{\B\text{-}qr}} 
\def\lderA{{\tilde{\A}}} 
\def\lderB{{\tilde{\B}}} 
\def\adjunit{{\text{adj.unit}}}
\def\adjcounit{{\text{adj.counit}}}
\def\degzero{{\text{deg.0}}}
\def\degone{{\text{deg.1}}}
\def\degminusone{{\text{deg.-$1$}}}
\def\barzeta{{\overline{\zeta}}}
\def\Ract{{R {\action}}}
\def\barRact{{\overline{\Ract}}}
\def\actL{{{\action} L}}
\def\baractL{{\overline{\actL}}}
\def\Ainfty{{A_{\infty}}}
\def\moddinf{{{\bf Mod}_{\infty}}}
\def\nodd{{{\bf Nod}}}
\def\noddinf{{{\bf Nod}_{\infty}}}
\def\perfinf{{{\bf Perf}_{\infty}}}
\def\conodd{{{\bf coNod}}}
\def\comodd{{{\bf coMod}}}
\def\conoddinf{{{\bf coNod}_{\infty}}}
\def\conodddgstrict{{{\bf coNod}_{dg}^{strict}}}
\def\conodddg{{{\bf coNod}_{dg}}}
\def\conodddghu{{{\bf coNod}_{dg}^{hu}}}
\def\conoddinfshu{{{\bf coNod}_{\infty}^{shu}}}
\def\noddinfstr{{{\bf Nod}^{\text{strict}}_{\infty}}}
\def\noddinfA{{\noddinf\A}}
\def\noddinfB{{\noddinf\B}}
\def\perfinfA{{\perfinf\A}}
\def\perfinfB{{\perfinf\B}}
\def\nodA{{\noddinf\text{-}A}}
\def\nodstrA{{\nodd\text{-}A}}
\def\nodstrhuA{{\nodd^{hu}\text{-}A}}
\def\nodhuA{{\noddinfhu\text{-}A}}
\def\nodhupfA{{\noddinfhupf\text{-}A}}
\def\nodB{{\noddinf\text{-}B}}
\def\nodAbic{{\left(\nodA\right)^{\text{bicat}}}}
\def\Anodbic{{\left(\Anod\right)^{\text{bicat}}}}
\def\nodBbic{{\left(\nodB\right)^{\text{bicat}}}}
\def\Bnodbic{{\left(\Bnod\right)^{\text{bicat}}}}
\def\repA{{A^{\modA}}}
\def\nodrepA{{\noddinf\text{-}\repA}}
\def\pretrA{{A^{\pretriag\A}}}
\def\nodpretrA{{\noddinf\text{-}\pretrA}}
\def\twcxA{{A^{\twcxub\A}}}
\def\nodtwcxA{{\noddinf\text{-}\twcxA}}
\def\twcxrepA{{A^{\twcxub\modA}}}
\def\nodtwcxrepA{{\noddinf\text{-}\twcxrepA}}
\def\Anod{{A\text{-}\noddinf}}
\def\Bnod{{B\text{-}\noddinf}}
\def\AnodA{{A\text{-}\noddinf\text{-}A}}
\def\AnodB{{A\text{-}\noddinf\text{-}B}}
\def\noddinfAB{{\noddinf\AbimB}}
\def\noddinfBA{{\noddinf\BbimA}}
\def\noddinfu{{({\bf Nod}_{\infty})_u}}
\def\noddinfuA{{(\noddinfA)_u}}
\def\noddinfhu{{{\bf Nod}_{\infty}^{hu}}}
\def\noddinfhupf{{{\bf Nod}_{\infty}^{hupf}}}
\def\noddinfhuA{{(\noddinfA)_{hu}}}
\def\noddinfhuB{{(\noddinfB)_{hu}}}
\def\noddinfdg{{({\bf Nod}_{\infty})_{dg}}}
\def\noddinfdgA{{(\noddinfA)_{dg}}}
\def\noddinfdgAA{{(\noddinf\AbimA)_{dg}}}
\def\noddinfdgAB{{(\noddinf\AbimB)_{dg}}}
\def\noddinfdgB{{(\noddinfB)_{dg}}}
\def\moddinf{{\modd_{\infty}}}
\def\moddinfA{{\modd_{\infty}\A}}
\def\naug{{\text{na}}}
\def\infbar{B_\infty}
\def\infbarl{{B^{l}_\infty}}
\def\infbarr{{B^{r}_\infty}}
\def\infbarbi{{B^{bi}_\infty}}
\def\infbarnaug{{B^{\naug}_\infty}}
\def\infcobar{{CB_\infty}}
\def\infbarres{{\bar{B}_\infty}}
\def\infcobarres{{C\bar{B}_\infty}}
\def\infbarA{{B^A_\infty}}
\def\infbarB{{B^B_\infty}}
\def\inftimes{{\overset{\infty}{\otimes}}}
\def\infhom{{\overset{\infty}{\homm}}}
\def\barhom{{\rm H\overline{om}}}
\def\barend{{\overline{\eend}}}
\def\bartimes{{\;\overline{\otimes}}}
\def\bartimesA{{\;\overline{\otimes}_\A\;}}
\def\bartimesB{{\;\overline{\otimes}_\B\;}}
\def\bartimesC{{\;\overline{\otimes}_\C\;}}
\def\triaA{{\tria \A}}
\def\TPairdg{{\TPair^{dg}}}
\def\algA{{\alg(\A)}}
\def\Ainfty{{A_{\infty}}}
\def\gpmu{{\boldsymbol{\mu}}}
\def\odd{{\text{odd}}}
\def\even{{\text{even}}}
\def\twcxub{\twcx^{\pm}}
\def\twcxmns{\twcx^{-}}
\def\twcxpls{\twcx^{+}}
\def\twbicxub{\twbicx^{\pm}}
\def\twbicxmns{\twbicx^{-}}
\def\twbicxpls{\twbicx^{+}}
\def\twbios{\twbicx_{os}}
\def\twbiosub{{\twbicx_{os}^{\pm}}}
\def\twbiosmns{{\twbicx_{os}^{-}}}
\def\twbiospls{{\twbicx_{os}^{+}}}
\def\pretriagub{{\pretriag^{\pm}}}
\def\pretriagmns{{\pretriag^{-}}}
\def\pretriagpls{{\pretriag^{+}}}
\def\eilmoordg{\eilmoor^{\text{dg}}}
\def\hprojemdg{\hproj^{\text{dg}}}
\def\hperfemdg{\hperf^{\text{dg}}}
\def\coeilmoordg{\coeilmoor^{\text{dg}}}
\def\kleisliA{{\kleisli(A)}}
\def\kleisliAS{{\kleisli_S(A)}}
\def\kleisliwk{\kleisli^{\text{wk}}}
\def\eilmoorwk{\eilmoor^{\text{wk}}}
\def\eilmoorwkpf{\eilmoor^{\text{wk,perf}}}
\def\eilmoorwmor{\eilmoor^{\text{dg,wkmor}}}
\def\coeilmoorwk{\coeilmoor^{\text{wk}}}
\def\coeilmoorwmor{\coeilmoor^{\text{dg,wkmor}}}
\def\dnat{{d_{\text{nat}}}}
\def\bareta{{\tilde{\eta}}}
\def\barepsilon{{\tilde{\epsilon}}}
\def\infalg{{\alg_\infty}}
\def\enhcatkc{{\bf EnhCat_{kc}}}
\def\enhcatkcdg{{\bf EnhCat_{kc}^{dg}}}


\def\conodA{{\conoddinf\text{-}A}}
\def\conodB{{\conoddinf\text{-}B}}
\def\conodC{{\conoddinf\text{-}C}}
\def\conodstrC{{\conodd\text{-}C}}
\def\Aconod{{A\text{-}\conoddinf}}
\def\Cconod{{C\text{-}\conoddinf}}
\def\CconodD{{C\text{-}\conoddinf\text{-}D}}
\def\conodhuA{{{\bf coNod}_{\infty}^{hu}\text{-}A}}
\def\conodshuA{{{\bf coNod}_{\infty}^{shu}\text{-}A}}
\def\conodhuB{{{\bf coNod}_{\infty}^{hu}\text{-}B}}
\def\conodshuB{{{\bf coNod}_{\infty}^{shu}\text{-}B}}
\def\conodhuC{{{\bf coNod}_{\infty}^{hu}\text{-}C}}
\def\conodshuC{{{\bf coNod}_{\infty}^{shu}\text{-}C}}
\def\cokleisliC{{\cokleisli(C)}}
\def\cokleisliCS{{\cokleisli_S(C)}}

\theoremstyle{definition}
\newtheorem{defn}{Definition}[section]
\newtheorem*{defn*}{Definition}
\newtheorem{exmpl}[defn]{Example}
\newtheorem*{exmpl*}{Example}
\newtheorem{exrc}[defn]{Exercise}
\newtheorem*{exrc*}{Exercise}
\newtheorem*{chk*}{Check}
\newtheorem{remark}[defn]{Remark}
\newtheorem*{remark*}{Remark}
\theoremstyle{plain}
\newtheorem{theorem}{Theorem}[section]
\newtheorem*{theorem*}{Theorem}
\newtheorem{conj}[defn]{Conjecture}
\newtheorem*{conj*}{Conjecture}
\newtheorem{question}[defn]{Question}
\newtheorem*{question*}{Question}
\newtheorem{prps}[defn]{Proposition}
\newtheorem*{prps*}{Proposition}
\newtheorem{cor}[defn]{Corollary}
\newtheorem*{cor*}{Corollary}
\newtheorem{lemma}[defn]{Lemma}
\newtheorem*{claim*}{Claim}
\newtheorem{Specialthm}{Theorem}
\renewcommand\theSpecialthm{\Alph{Specialthm}}
\numberwithin{equation}{section}
\renewcommand{\textfraction}{0.001}
\renewcommand{\topfraction}{0.999}
\renewcommand{\bottomfraction}{0.999}
\renewcommand{\floatpagefraction}{0.9}
\setlength{\textfloatsep}{5pt}
\setlength{\floatsep}{0pt}
\setlength{\abovecaptionskip}{2pt}
\setlength{\belowcaptionskip}{2pt}

\begin{abstract}
We define $\Ainfty$-structures -- algebras, coalgebras, modules, and comodules -- in 
an arbitrary monoidal DG category or bicategory 
by rewriting their definitions in terms of unbounded twisted complexes. We develop 
new notions of strong homotopy unitality and bimodule homotopy unitality  
to work in this level of generality. For a strong homotopy unital $\Ainfty$-algebra we 
construct Free-Forgetful homotopy adjunction, its Kleisli category, and its derived category 
of modules. Analogous constructions for $\Ainfty$-coalgebras require  bicomodule homotopy 
counitality.  We define  homotopy adjunction for $\Ainfty$-algebra and 
$\Ainfty$-coalgebra and show such pair to be derived module-comodule equivalent. 
As an application,  we obtain the notions of an $\Ainfty$-monad and 
of an enhanced exact monad. We also show that for any adjoint triple $(L,F,R)$
of functors between enhanced triangulated categories the adjunction monad $RF$
and the adjunction comonad $LF$ are derived module-comodule equivalent. 
\end{abstract}

\maketitle

\section{Introduction}
\label{section-introduction}

The notion of an $\Ainfty$-algebra originated in the work of Stasheff
\cite{Stasheff-HomotopyAssociativityOfHSpacesIandII} on higher
homotopy associativity. It comes from wanting to relax the associativity 
condition on the multiplication of a
differentially graded (DG) algebra $A$ to only hold ``up to homotopy'', 
i.e.~to a boundary of the differential.
An $\Ainfty$-algebra is a graded vector space $A$ equipped 
with operations $m_i\colon A^i \rightarrow A$ where $m_1$ is the differential, 
$m_2$ the multiplication, $m_3$ the homotopy up to which associativity
holds, and $m_{i \geq 4}$ are the higher homotopies describing the interaction 
of $m_1, \dots, m_{i-1}$. These and their generalisations 
were actively studied since, especially in the context of homological 
mirror symmetry \cite{Kadeishvili-OnTheTheoryOfHomologyOfFiberSpaces}
\cite{Fukaya-MorseHomotopyAinftyCategoryAndFloerHomologies}
\cite{Kontsevich-HomologicalAlgebraOfMirrorSymmetry}
\cite{Keller-IntroductionToAInfinityAlgebrasAndModules}
\cite{Fukaya-FloerHomologyAndMirrorSymmetryII} 
\cite{Lyubashenko-CategoryOfAinftyCategories}
\cite{Lefevre-SurLesAInftyCategories}\cite{KontsevichSoibelman-NotesOnAInftyAlgebrasAInftyCategoriesAndNoncommutativeGeometry}. 

A key feature of this standard definition of $\Ainfty$-algebra 
is that it takes place in the monoidal category of graded vector spaces.
Existing generalisations are achieved by slight modifications 
of this monoidal category: introducing an extra grading by objects of
a set yields $\Ainfty$-categories. 

In this paper, we rewrite the definitions of $\Ainfty$-algebras and
their (bi)modules, and $\Ainfty$-coalgebras and (bi)comodules, 
in the language of twisted complexes
\cite{BondalKapranov-EnhancedTriangulatedCategories}. This decouples
them from the differential $m_1$ so that they work in any
monoidal DG category $\A$. New definitions make implicit use of 
the internal differential provided in $\A$ by Yoneda embedding.  
We can now talk about $\Ainfty$-algebras in the monoidal DG
categories of DG functors and of enhanced exact functors, yielding 
the new notions of an \em $\Ainfty$-monad \rm and \em enhanced exact
monad\rm.  

We study modules over such $\Ainfty$-algebras and 
construct their derived categories. First, we develop a new 
notion of \em strong homotopy unitality. \rm There are several existing 
notions of homotopy unitality for $\Ainfty$-algebras 
\cite{Fukaya-MorseHomotopyAinftyCategoryAndFloerHomologies}\cite{Lyubashenko-HomotopyUnitalAinftyAlgebras}\cite{Lefevre-SurLesAInftyCategories}\cite{KontsevichSoibelman-NotesOnAInftyAlgebrasAInftyCategoriesAndNoncommutativeGeometry}. 
Our notion arises naturally in our generality, and 
in the classical setting reduces to a small fraction of the data 
fixed in \cite{Fukaya-MorseHomotopyAinftyCategoryAndFloerHomologies}\cite{KontsevichSoibelman-NotesOnAInftyAlgebrasAInftyCategoriesAndNoncommutativeGeometry}.
It turns out to be enough to define the derived category $D(A)$ of
such $\Ainfty$-algebra $A$ as the homotopy category of 
(the triangulated cocomplete hull of) its homotopy unital $\Ainfty$-modules. 
It also lets us construct \em Free-Forgetful homotopy adjunction \rm 
for such modules. We define the \em Kleisli category \rm of $A$, a usual 
$\Ainfty$-category
\cite{Fukaya-MorseHomotopyAinftyCategoryAndFloerHomologies}\cite{Lefevre-SurLesAInftyCategories},
and prove it is derived equivalent 
to $A$ for strongly homotopy unital $A$. All these constructions
work for $\Ainfty$-coalgebras, only with 
stronger notions of \em bimodule homotopy counitality \rm for coalgebras
and \em strong homotopy counitality \rm for comodules. 
Moreover, the derived category of an $\Ainfty$-coalgebra 
is only well-behaved if the free comodules generated by perfect objects 
of $\A$ are perfect. 

Finally, we define $\Ainfty$-algebra $A$ and
$\Ainfty$-coalgebra $C$ being homotopy adjoint. Such 
$C$ and $A$ are homotopy adjoint as objects of $\A$ and have 
a coherent system of higher homotopies describing how the adjunction 
interacts with their operations. For a strong homotopy
unital $A$ and bimodule homotopy unital $C$ which are homotopy
adjoint, we prove \em derived module-comodule equivalence
\rm $D_c(A) \simeq D_c(C)$ by relating their Kleisli and co-Kleisli
categories via Free-Forgetful and Forgetful-Free homotopy adjunctions. 
Finally, we show that for any homotopy adjoint triple $(L,F,R)$ 
in $\A$  the algebra $RF$ and the coalgebra $LF$ are homotopy adjoint. This
lets us construct derived module-comodule equivalence $D_c(RF)
\simeq D_c(LF)$ for any adjoint triple $(L,F,R)$ of 
functors between enhanced triangulated categories.  

This paper was motivated by the following perspective. In non-derived algebra 
we can view algebras and coalgebras as monoids and comonoids in 
the abelian category of modules over the ground field or ring $k$. 
The hierarchy is completed by defining modules and comodules over 
the algebras and coalgebras. It turns out that this whole three level tower of constructions works just as well in any monoidal abelian category. For example, in the category of endofunctors of an abelian category. One arrives thus at
the notion of monads and comonads, and of modules and comodules over them. One important class of comonads arises when gluing quasicoherent sheaves over an affine covering of a scheme, that is -- it is built into the procedure of Zariski 
or flat or {\'e}tale descent. 

Difficulties apppear on several levels when one tries to repeat these
constructions in the setting of derived categories and of their DG enhancements. The substitutions for algebras, coalgebras, modules and comodules in general monoidal DG categories
(especially, in the functor categories) do not allow naive definitions mimicking the non-derived setting. This led some to do derived algebra and, in particular, the generalisations of functor adjunctions and of
Barr-Beck theorem using stable $\infty$-categories \cite{Lurie-HigherToposTheory}\cite{Lurie-DAG1StableInfinityCategories}. One example is the desired notion of gluing data up to homotopy for complexes of quasicoherent sheaves
on an affine/{\'e}tale/flat covering of a scheme. While the datum of gluing with higher homotopies that provide corrections is easy to write down, a comparison of glued categories with the derived category of quasicoherent sheaves on the scheme itself lacked both a setting and precise statements of theorems. As a remedy one referred to 
Barr-Beck-Lurie theorem \cite[\S3.4]{Lurie-DAG2NoncommutativeAlgebra} and homotopy descent for stable $\infty$-categories --- a topological machinery of a great level of abstraction that glues $\infty$-categories as a whole but makes it difficult to talk about models for glued objects and for glued
morphisms. 

The present article is devoted to creation of the setting that would fill the existing gap: we complete the hierarchy of a monoidal
category, algebras/coalgebras, modules/comodules in the seting of DG and $\Ainfty$-enhanced triangulated categories, thus allowing for explicit computations. 
We will apply it to proving a version of Barr-Beck
theorem for DG categories in a future work.

We now describe our results in more detail. We first note
that though they are formulated for an arbitrary monoidal DG
category $\A$, they all work equally well for $\A$ being any DG
bicategory. However, the language of bicategories is less intuitive, 
so we chose to write this paper in the language of monoidal
categories.  

\subsection{$\Ainfty$-structures in arbitrary DG monoidal categories}

Fix a ground field or
commutative ring $k$ and recall the usual definition of $\Ainfty$-algebra:
\begin{defn}[\cite{Lefevre-SurLesAInftyCategories}, D{\'e}finition 1.2.1.1]
\label{defn-intro-usual-Ainfty-algebra}
An \em $\Ainfty$-algebra \rm $(A,m_\bullet)$ is a graded $k$-module
$A$ and a collection $\left\{ m_i \right\}_{i \geq 1}$ of degree 
$2-i$ graded maps $A^i \rightarrow A$ satisfying
\begin{equation}
\label{eqn-intro-Ainfty-algebra-definition-equalities}
\forall\;\; i \geq 1
\quad \quad 
\sum_{j+k+l = i} (-1)^{jk + l} m_{j+1+l} \circ (\id^j \otimes m_k
\otimes \id^l) = 0.
\end{equation}
\end{defn}
The operations $m_i$ define a $k$-linear map 
$\bigoplus_{i \geq 0} (A[1])^i \rightarrow A$ from the free tensor 
coalgebra $\bigoplus_{i \geq 0} A^i[i]$. By the universal property of
free coalgebras this induces a unique coderivation of  
$\bigoplus_{i \geq 0} A^i[i]$. The condition
\eqref{eqn-intro-Ainfty-algebra-definition-equalities} is equivalent
to this coderivation squaring to zero, and thus defining a
differential on $\bigoplus_{i \geq 0} A^i[i]$. The resulting 
DG coalgebra is the \em bar-construction \rm of $(A,m_\bullet)$. 

Let $\modk$ be the DG category of DG $k$-modules. Tensor product
$\otimes_k$ makes it into a monoidal DG category. 
Definition \ref{defn-intro-usual-Ainfty-algebra} then reads:
an $\Ainfty$-algebra is an object $A$ and a collection $\left\{ m_i
\right\}_{i \geq 2}$ of morphisms $m_i\colon A^i \rightarrow A$ in
$\modk$ such that if we set $m_1$ to be the internal differential of $A$
the conditions \eqref{eqn-intro-Ainfty-algebra-definition-equalities} are 
satisfied. This doesn't generalise naively to an arbitrary monoidal
DG category $\A$ because its objects have no internal differentials. 
Even if we use Yoneda embedding $\A \rightarrow \modA$ to obtain
internal differentials, these are not compatible with the monoidal 
operation of $\A$: we can no longer express the internal differentials of 
$A^i$ in terms of internal differential $m_1$ of $A$. We can 
not hope to restate Definition \ref{defn-intro-usual-Ainfty-algebra}
with equations involving only $m_1$ and $\left\{ m_i
\right\}_{i \geq 2}$. To make matters worse, internal differentials are not morphisms in $\modA$ as they do not commute with
$\A$-action. 

Think instead of Definition \ref{defn-intro-usual-Ainfty-algebra} as saying
that the coderivation of $\bigoplus_{i \geq 0} A^i[i]$ induced by $\left\{ m_i \right\}_{i \geq 2}$ modifies its 
natural differential in $\modA$ to a new one. Now, the language for modifying 
the existing differential of the sum of shifted representable $\A$-modules 
exists - it is the language of twisted complexes over $\A$
\cite{BondalKapranov-EnhancedTriangulatedCategories}. One of its  
advantages is that it is phrased entirely in terms of the category $\A$, 
hiding the Yoneda embedding and the internal differentials into the
woodwork. We can work with twisted complexes over non-small
$\A$ where $\modA$ is not well-defined. However, 
the usual twisted complexes are bounded, while
due to the infinite number of terms in $\bigoplus_{i \geq 0} A^i[i]$ we 
need unbounded ones. We develop their theory  
in the companion paper \cite{AnnoLogvinenko-UnboundedTwistedComplexes},
and thus can define: 
\begin{defn}
\label{defn-intro-ainfty-algebra-a-monoidal-category}
Let $\A$ be a monoidal DG category. An \em $\Ainfty$-algebra
$(A,m_\bullet)$ in $\A$ \rm is an object $A \in \A$ and 
a collection $\left\{m_i\right\}_{i \geq 2}$ of degree $2-i$
morphisms $A^i \rightarrow A$ whose 
(non-augmented) bar-construction $\infbarnaug(A)$ is 
a twisted complex over $\A$:
\begin{tiny}
\begin{equation}
\label{eqn-intro-nonaugmented-bar-construction-of-A-m_i}
\begin{tikzcd}[column sep = 2.5cm]
\dots
\ar{r}[']{\begin{smallmatrix}A^3 m_2  - A^2m_2A + \\ + Am_2A^2 - m_2 A^3 \end{smallmatrix}}
\ar[bend left=25]{rr}[description]{A^2m_3 + Am_3A +  m_3 A^2}
\ar[bend left=30]{rrr}[description]{Am_4 - m_4A}
\ar[bend left=35]{rrrr}[description]{m_5}
&
A^4
\ar{r}[']{A^2m_2 - A m_2 A + m_2 A^2}
\ar[bend left=25]{rr}[description]{-Am_3 - m_3A}
\ar[bend left=30]{rrr}[description]{m_4}
& 
A^3
\ar{r}[']{Am_2 - m_2A}
\ar[bend left=25]{rr}[description]{m_3}
&
A^2
\ar{r}[']{m_2}
&
\underset{\degzero}{A}.
\end{tikzcd}
\end{equation}
\end{tiny}
Here the differentials are
\begin{equation}
\label{eqn-intro-differentials-in-non-aug-bar-construction}
d_{(i+k)i} := (-1)^{(i-1)(k+1)} \sum_{j = 0}^{i-1} (-1)^{jk}
\id^{i-j-1}
\otimes m_{k+1} \otimes \id^{j}. 
\end{equation} 
\end{defn}

We give this definition of $\Ainfty$-algebras for an arbitrary 
DG monoidal category $\A$ in \S\ref{section-ainfty-structures-in-monoidal-dg-categories} along with a similar definition of their morphisms. 
We also define the DG categories $\nodA$ and $\Anod$ of right and left 
$\Ainfty$-$A$-modules and the DG category $\AnodB$ of
$\Ainfty$-$A$-$B$-bimodules. In each instance, we first write down the
corresponding bar-construction, and then the definition asks for it to 
be a twisted complex. For morphisms, we define their composition and 
differential by composing and differentiating their bar-constructions 
as twisted complexes. 

We then study $\Ainfty$-algebras and their modules in this generality. 
In \S\ref{section-Yoneda-embedding-of-A-modules-into-repA-modules}-\ref{section-twisted-complexes-of-ainfty-modules} we construct an analogue of 
Yoneda embedding for $\nodA$ which acts as the target 
for convolutions of twisted complexes over it. In 
\S\ref{section-the-homotopy-lemma} we prove the Homotopy Lemma:
an $\Ainfty$-module $(E,p_\bullet)$ is null-homotopic in
$\nodA$ if and only if $E$ is null-homotopic in $\A$. 
Hence a morphism $f_\bullet$ in
$\nodA$ is a homotopy equivalence if and only if its first component
is a homotopy equivalence in $\A$. In the standard setting of $\A = \modk$ 
for a field $k$, this is the well-known result that 
all $\Ainfty$-quasi-isomorphisms of $\Ainfty$-modules are 
homotopy equivalences, 
cf.~\cite[Prop.~2.4.1.1]{Lefevre-SurLesAInftyCategories}. 

In \S\ref{section-bar-construction-as-a-complex-of-ainfty-A-modules}
we show that the bar-construction $\infbar(E, p_\bullet)$ of an
$\Ainfty$-$A$-module $(E,p_\bullet)$ originally defined 
as a twisted complex over $\A$ admit a natural lift
to a twisted complex over $\nodA$. The object $E$ lift to the module
$(E, p_\bullet)$, while the objects $EA^i$ lift to free 
$\Ainfty$-$A$-modules $EA^i$ defined in 
\S\ref{section-free-modules-and-bimodules-over-Ainfty-algebra}. The
Homotopy Lemma then shows that if $(E,p_\bullet)$ is $H$-unital, 
that is if $\infbar(E, p_\bullet)$ is null-homotopic in $\A$, then 
it is also null-homotopic in $\nodA$. We thus get a functorial 
\em bar-resolution \rm $\infbarres$ which resolves any $H$-unital 
module by free modules, cf. \S\ref{section-bar-resolution}. 

\subsection{Strong homotopy unitality for $\Ainfty$-algebras} 
Several notions of homotopy unitality exist for usual $\Ainfty$-algebras. 
\em $H$-unitality \rm asks for 
the bar-construction to be null-homotopic and 
\em (weak) homotopy unitality \rm asks for a unital structure 
to exist in the homotopy category \cite[\S4]{Lefevre-SurLesAInftyCategories}
\cite{Lyubashenko-CategoryOfAinftyCategories}. For $\Ainfty$-algebras, 
a more powerful notion is given in
\cite{FukayaOhOhtaOno-LagrangianIntersectionFloerTheoryAnomalyAndObstruction}\cite{KontsevichSoibelman-NotesOnAInftyAlgebrasAInftyCategoriesAndNoncommutativeGeometry}.
It fixes a very large coherent system of higher homotopies up to which the
unitality holds. These notions were all proved to be equivalent for 
$k$ a field \cite{Lefevre-SurLesAInftyCategories} or any  
commutative ring \cite{LyubashenkoManzyuk-UnitalAinftyCategories}.  

The unitality notion of
\cite{FukayaOhOhtaOno-LagrangianIntersectionFloerTheoryAnomalyAndObstruction}
is easily defined in our generality. However, as we no longer know it
to be equivalent to the simpler notions above, proving something to be
unital in that sense could be difficult. In
\S\ref{section-strong-homotopy-unitality} we introduce two new notions: 
\em strong homotopy unitality \rm 
and \em bimodule homotopy unitality\rm. Both
are small truncations of the structure in
\cite{FukayaOhOhtaOno-LagrangianIntersectionFloerTheoryAnomalyAndObstruction}
and are easier to construct, while still being 
enough to make our theory work. The authors do not yet know
whether in our generality there are analogues of the results in
\cite{LyubashenkoManzyuk-UnitalAinftyCategories} showing that
these simpler notions imply the existence of the full structure in
\cite{FukayaOhOhtaOno-LagrangianIntersectionFloerTheoryAnomalyAndObstruction}. 

The motivation of our definitions is as follows. 
An $\Ainfty$-algebra $(A,m_\bullet)$ is weakly homotopy unital 
if there exists a morphism $\eta\colon \id \rightarrow A$ in $\A$ 
such that 
\begin{equation}
\label{eqn-intro-weak-homotopy-unitality-conditions}
m_2 \circ \eta A  = \id_A + d h^r 
\quad \text{ and } \quad 
m_2 \circ A \eta  = \id_A + d h^l, 
\end{equation}
for some degree $-1$ endomorphisms $h^l, h^r$ of $A$ in $\A$. 
Now, for a strict algebra $(A,m_2)$ the maps
$m_2 \circ \eta A$ and $m_2 \circ A \eta$ are strict morphisms of 
right and left $A$-modules, respectively. It is natural to ask for
the homotopies between these and $\id_A$ to also be morphisms of left 
and right $A$-modules:
\begin{defn}[Defn.~\ref{defn-strong-homotopy-unitality-for-ainfty-algebras}]
$\Ainfty$-algebra $(A,m_i)$ is \em strongly homotopy unital \rm if there 
exists a unit morphism $\eta\colon \id \rightarrow A$ in $\A$ and
degree $-1$ endomorphisms $h^r_{\bullet}$ and $h^l_{\bullet}$ of $A$
in $\nodA$ and $\Anod$, respectively, such that 
\begin{equation}
\label{eqn-intro-strong-homotopy-unitality-conditions}
\mu_2 \circ {\eta}A = \id_A + d h^r_{\bullet}  \text{ in } \nodA
\quad \text{ and } \quad 
\mu_2 \circ A{\eta} = \id_A + d h^l_{\bullet} \text{ in } \Anod. 
\end{equation}
\end{defn}
The maps $\mu_2$ are natural lifts of
the operation $m_2\colon A^2 \rightarrow A$ from $\A$ into $\nodA$ and 
$\Anod$, see Defn.~\ref{defn-morphisms-pi_i-and-mu_i}. 
Thus our definition simply lifts the weak homotopy unitality conditions 
\eqref{eqn-intro-weak-homotopy-unitality-conditions} from $\A$ to $\nodA$
and $\Anod$. Instead of just the homotopies $h^l$
and $h^r$, we ask for two systems of homotopies $h^l_\bullet$
and $h^r_\bullet$ to exist. 

\em Bimodule homotopy unitality \rm asks 
for a degree $0$ morphism $\bareta_{\bullet\bullet}\colon \id_\A
\rightarrow A$ with a prescribed differential
to exist in the category $\AnodA$ of $\Ainfty$-$A$-$A$-bimodules, 
see Defn.~\ref{def-bimodule-homotopy-unitality}. However, the
following theorem explains it better than its definition:
\begin{theorem}[Theorem
\ref{theorem-conditions-for-bimodule-homotopy-unitality}]
\label{theorem-intro-conditions-for-bimodule-homotopy-unitality}
$(A,m_\bullet)$ is bimodule homotopy unital iff there exist 
\begin{itemize}
\item a closed degree $0$ morphism $\eta\colon \id_{\A} \rightarrow A$ in
$\A$,  
\item a degree $-1$ morphism $h^l_\bullet\colon A \rightarrow A$
in $\Anod$,
\item a degree $-1$ morphism $h^r_\bullet\colon A \rightarrow A$ 
in $\nodA$,
\item a degree $-2$ morphism $\kappa_{\bullet\bullet} \colon A^2
\rightarrow A$ in $\AnodA$, 
\end{itemize}
such that in these categories 
\begin{align*}
dh^r_\bullet &= \id_{\A} - \mu_2 \circ \eta{A}, \\
dh^l_\bullet &= \id_{\A} - \mu_2 \circ A\eta, \\
d\kappa_{\bullet\bullet} &= \mu_2 \circ (h^l_\bullet{A} -
{A}h^r_\bullet) - \mu_3 \circ A \eta A.
\end{align*}
\end{theorem}
In other words, the data of $\bareta_{\bullet\bullet}$ comprises
the homotopy unit $\eta$, strong homotopy unitality systems 
$h^l_\bullet$ and $h^r_\bullet$, and the system
$\kappa_{\bullet\bullet}$ of higher homotopies describing the
interaction of $\eta$, $h^l_\bullet$ and $h^r_\bullet$
with the operations $m_i$ of $A$. 
 
\subsection{Strong homotopy unitality for $\Ainfty$-modules} 

Existing notions of homotopy unitality for $\Ainfty$-modules 
\cite[\S4]{Lefevre-SurLesAInftyCategories} are: a (right) $\Ainfty$-$A$-module 
$(E,p_\bullet)$ is \em $H$-unital \rm if
its bar-construction is null-homotopic and \em (weakly) homotopy
unital \rm if  
\begin{equation}
\label{eqn-intro-weak-homotopy-unitality-for-modules}
E \xrightarrow{E\eta} EA \xrightarrow{p_2} E = \id_E + dh \quad \quad \text{ in } \A 
\end{equation}
for some degree $-1$ endomorphism $h$ of $E$ in $\A$.
Here $\eta$ is the homotopy unit of $A$. 

We define a new notion of \em strong homotopy unitality \rm for modules 
by lifting \eqref{eqn-intro-weak-homotopy-unitality-for-modules}
from $\A$ to $\nodA$. However, here the lift is trickier as 
we must lift $EA$ not to the free $\Ainfty$-module $EA$, but to the whole
bar-resolution $\infbarres(E,p_\bullet)$. This requires
$A$ to be bimodule homotopy unital. We use its unitality
structure $\bareta_{\bullet\bullet}$ to cook up 
$$ \chi\colon (E,p_\bullet) \rightarrow \infbarres(E,p_\bullet) 
$$
which lifts $E\eta$ in \eqref{eqn-intro-weak-homotopy-unitality-for-modules}
to $\nodA$, see \S\ref{section-unitality-conditions-for-A-modules}. 
The lift of $p_2$ is the bar-resolution 
$$ \rho\colon \infbarres(E,p_\bullet) \rightarrow (E,p_\bullet). $$
We then define:
\begin{defn}
\label{defn-intro-strong-homotopy-unitality}
A module $(E,p_\bullet) \in \nodA$ is \em strongly homotopy 
unital \rm if there exists a degree $-1$ endomorphism $h_\bullet$ 
of $(E,p_\bullet)$ in $\nodA$  such that 
$$ (E,p_\bullet) \xrightarrow{\chi} \infbarres(E,p_\bullet)
\xrightarrow{\rho} (E,p_\bullet) \quad = \quad \id + dh_\bullet 
\quad \quad \text{ in } \nodA. $$
\end{defn}

Unlike for $\Ainfty$-algebras, we can prove all these notions 
of homotopy unitality for $\Ainfty$-modules to be equivalent: 
\begin{theorem}[Theorem
\ref{theorem-tfae-unitality-conditions-for-A-modules}]
\label{theorem-intro-tfae-unitality-conditions-for-A-modules}
Let $A$ be strongly homotopy unital and let $(E,p_\bullet)$ be an 
$\Ainfty$-$A$-module. The following are equivalent:
\begin{enumerate}
\item 
$(E,p_\bullet)$ is homotopy unital, 
\item 
$(E,p_\bullet)$ is $H$-unital. 
\end{enumerate}
If $A$ is bimodule homotopy unital, these are further equivalent to:
\begin{enumerate}
\setcounter{enumi}{2} 
\item 
$(E,p_\bullet)$ is strongly homotopy unital. 
\end{enumerate}
\end{theorem}

\subsection{Free-Forgetful homotopy adjunction and Kleisli category} 

For any strict and strictly unital algebra $A$ in $\A$ we have 
Free-Forgetful adjunction. For any $E \in \A$ and $(F,q) \in \modA$ 
we have a natural isomorphism
$$ \homm_{\A}(E,F) \xrightarrow{\sim} \homm_{\modA}(EA, (F,q)). $$
For any strongly homotopy unital $\Ainfty$-algebra in $\A$
we construct in \S\ref{section-free-forgetful-homotopy-adjunction} 
a Free-Forgetful homotopy adjunction. For any $E \in \A$
and $(F,q_\bullet) \in \nodhuA$, the category of homotopy unital 
$\Ainfty$-$A$-modules, we have a natural homotopy equivalence
\begin{equation}
\label{eqn-intro-free-forgetful-homotopy-adjunction}
\homm_{\A}(E,F) \xrightarrow{\sim} \homm_{\nodhuA}(EA, (F,q_\bullet)).
\end{equation}

This motivates the following definition in
\S\ref{section-kleisli-category}:
\begin{defn}
\label{defn-intro-kleisli-category-of-an-ainfty-algebra}
Let $A$ be an $\Ainfty$-algebra in a monoidal DG category $\A$. Define
its \em Kleisli category \rm $\kleisliA$ to be the usual $\Ainfty$-category
\cite{Lefevre-SurLesAInftyCategories} defined by 
\begin{itemize}
\item Its objects are the objects of $\A$. 
\item For any $E,F \in \A$ the $\homm$-complex between them is
\begin{equation}
\homm_{\kleisliA}(E,F) := \homm_{\A}(E,FA).
\end{equation}
\item For any $E_1, E_2, \dots, E_{n+1} \in \A$ and any $\alpha_i \in 
\homm_{\kleisliA}(E_i, E_{i+1})$ define 
\begin{equation}
\label{eqn-intro-defn-of-ainfty-structure-on-kleisli-category}
m_n^{\kleisliA}(\alpha_1, \dots, \alpha_n) := 
E_1 \xrightarrow{\alpha_1} E_2A \xrightarrow{\alpha_2 A} \dots
\xrightarrow{\alpha_n A^{n-1}} E_{n+1}A^n \xrightarrow{E_{n+1}m_n^A}
E_{n+1}A.
\end{equation}
\end{itemize}
\end{defn}

We construct an $\Ainfty$-functor from it to the DG category 
of free $A$-modules:
$$ f_\bullet\colon \kleisliA \rightarrow \freeA $$
which sends any $E \in \A$ to $EA \in \freeA$ 
and whose first component on morphism spaces is the Free-Forgetful 
adjunction map \eqref{eqn-intro-free-forgetful-homotopy-adjunction}. 
It follows that for strongly homotopy unital $A$ the functor
$f_\bullet$ is a quasi-equivalence 
(Theorem \ref{theorem-ainfty-quasi-equivalence-from-kleisli-to-free}). 

As explained in \S\ref{section-examples-dg-and-ainfty-categories},
for a usual $\Ainfty$-category $A$, which in 
our setting is an $\Ainfty$-algebra in 
the monoidal DG category $\A = \modd\text{-}k_S$ of DG $k$-modules
with an extra grading by a set $S$, the Kleisli category of this
$\Ainfty$-algebra recovers $A$. Thus our 
$\Ainfty$-functor $f_\bullet$ can be seen as the generalisation of 
the Yoneda embedding of a usual $\Ainfty$-category into its
DG category of $\Ainfty$-modules \cite[\S7.1]{Lefevre-SurLesAInftyCategories}. 

\subsection{Derived category}

In \S\ref{section-the-derived-category-the-general-case} we define 
the \em derived category \rm of an $\Ainfty$-algebra $(A,m_\bullet)$  
in a monoidal DG category $\A$. By the Homotopy Lemma there 
is no need to invert quasi-isomorphisms. Following
\cite[\S4]{Lefevre-SurLesAInftyCategories}, we want 
the derived category $D(A)$ to be the homotopy category of the 
category $\nodhuA$ of $H$-unital $\Ainfty$-$A$-modules. 
However, we expect the derived category to be triangulated and
cocomplete, while for an arbitrary $\A$
the homotopy category $H^0(\nodhuA)$ is not apriori either.  
We need to fix an embedding of $\A$
into a cocomplete closed monoidal pretriangulated DG category $\B$. 
We also assume that $H^0(\A)$ is compactly generated within
$H^0(\B)$. When $\A$ is small, we can always take $\B = \modA$. When 
$\A$ is not small, we must make a different choice. In many applications 
we take $\B$ to be $\A$ itself, for example in the classical setting of $\A =
\modk$.  

Let $\noddinf\text{-}A^\B$ be the category of $\Ainfty$-$A$-modules in
$\B$. Its homotopy category is triangulated and cocomplete, so we define $D(A)$
to be the cocomplete triangulated hull of $\nodhuA$ in 
$H^0(\noddinf\text{-}A^\B)$.  When $A$ is $H$-unital, 
$D(A)$ is the cocomplete triangulated hull of $\freeA$ in 
$H^0(\noddinf\text{-}A^\B)$. When $A$ is strongly
homotopy unital, the compact derived category $D_c(A)$ admits a
particularly nice description:
\begin{theorem}[Theorem
\ref{theorem-compact-derived-category-of-A-is-that-of-frees-and-kleisli}]
\label{theorem-intro-compact-derived-category-of-A-is-that-of-frees-and-kleisli}
Let $A$ be a strongly homotopy unital $\Ainfty$-algebra in a monoidal 
DG category $\A$. Let $S$ be a set of compact generators of $H^0(\A)$. Then 
$$ D_c(A) \simeq D_c(\freeAS) \simeq D_c(\kleisliAS). $$
\end{theorem}
Note that $D_c(\freeAS)$ and $D_c(\kleisliAS)$ are the usual compact
derived categories of the DG category $\freeAS$ and 
the $\Ainfty$-category $\kleisliAS$. This shows
that $D_c(A)$ is independent of the choice of $\B$. Indeed, 
the choice of $\B$ matters when we start taking infinite direct sums, 
while everything in $D_c(A)$ 
can be obtained from $\freeA$ via bounded twisted complexes 
and homotopy direct summands. Note that there is no hope  
of extending this equivalence to the full derived categories
as $D(\freeAS)$ and $D(\kleisliAS)$ do not depend on $\B$, 
while $D(A)$ does.  

\subsection{$\Ainfty$-coalgebras}

In \S\ref{section-ainfty-coalgebras-and-comodules} we translate 
the definitions and results we obtained for
$\Ainfty$-algebras, modules and bimodules in \S\ref{section-ainfty-structures-in-monoidal-dg-categories}-\S\ref{section-the-derived-category}
to $\Ainfty$-coalgebras, comodules, and
bicomodules. Most translate straightforwardly and we give
all the key definitions. However there are some subtleties. The chief
is that the Homotopy Lemma fails for $\Ainfty$-comodules, 
see the introduction to
\S\ref{section-ainfty-coalgebras-and-comodules}. However, it turns 
out that it still holds for strongly homotopy counital
$\Ainfty$-comodules, see Lemma
\ref{lemma-coalgebra-analogue-of-the-homotopy-lemma}. 

The bottom line: working only with bicomodule homotopy counital 
$\Ainfty$-coalgebras and with strongly homotopy counital
$\Ainfty$-comodules over them, we have 
the analogues of all the definitions and results in 
\S\ref{section-ainfty-structures-in-monoidal-dg-categories}-\S\ref{section-strong-homotopy-unitality}. Consequently, we define 
the \em derived category of an $\Ainfty$-coalgebra $(C,
\Delta_\bullet)$ \rm as the cocomplete triangulated hull of $\conodshuC$ in 
$H^0(\conoddinf\text{-}C^\B)$. Here $\conodshuC$ is the category of 
strongly homotopy unital $\Ainfty$-$C$-comodules. We have 
for $D(C)$ the analogues of all the results in 
\S\ref{section-the-derived-category} provided the following additional 
assumption holds: free comodules $EC$ generated by compact $E \in \A$ 
are compact in $D(C)$. In particular we have:
\begin{theorem}[Theorem
\ref{theorem-compact-derived-category-of-bimodule-homotopy-unital-C-is-that-of-frees-and-cokleisli}]
Let $C$ be a bicomodule homotopy counital $\Ainfty$-coalgebra in a
monoidal DG category $\A$ and $S$ be a set of compact generators 
of $H^0(\A)$ in $H^0(\B)$.
If for every perfect $E \in \A$, the free $\Ainfty$-comodule $EC$ is
also perfect, then 
$$ D_c(C) \simeq D_c(\free_S\text{-}C) \simeq D_c(\cokleisliCS). $$
\end{theorem}

The authors are aware that for usual DG and $\Ainfty$-coalgebras there
are subtleties involved in constructing their derived categories.
These are explored at length by Positselksi in
\cite{Positselski-TwoKindsOfDerivedCategoriesKoszulDualityAndComoduleContramoduleCorrespondence}. We do not yet know how our definition of the derived 
category via strongly homotopy counital comodules relates, 
when applied to usual $\Ainfty$-coalgebras, to the definitions in
\cite{Positselski-TwoKindsOfDerivedCategoriesKoszulDualityAndComoduleContramoduleCorrespondence}.
We discuss this briefly in \S\ref{section-comparison-to-known-constructions}. 

\subsection{Module-comodule correspondence}

In \S\ref{section-homotopy-adjoint-Ainfty-algebras-and-coalgebras} we
define an $\Ainfty$-algebra $(A,m_\bullet)$ and an 
$\Ainfty$-coalgebra $(C,\Delta_\bullet)$ being \em homotopy adjoint
as $\Ainfty$-algebra and $\Ainfty$-coalgebra\rm. In brief, we first ask 
for $A$ and $C$ to be adjoint as objects of $\A$ with 
the unit $\eta_1\colon \id \rightarrow CA$ and counit 
$\epsilon_1\colon AC \rightarrow \id$. We then also ask for 
coherent systems of higher homotopies $\eta_i\colon \id \rightarrow C^iA$ 
and $\epsilon_i\colon A^iC \rightarrow \id$ which describe 
how $\eta_1$ and $\epsilon_1$ interact with the operations
$m_\bullet$ and $\Delta_\bullet$. See
Defn.~\ref{defn-homotopy-adjoint-ainfty-coalgebra-and-algebra} for
full details. 

A non-trivial example of such structure is provided by any homotopy adjoint
triple $(L,F,R)$ of objects in $\A$: adjunction units and counits 
make $RF$ into a strongly homotopy unital 
strict algebra and $LF$ into a strongly homotopy counital 
strict coalgebra. These are homotopy adjoint in the sense above, see
Prop.~\ref{prps-LF-and-RF-are-homotopy-adjoint-in-a-monoidal-category}. 

Finally, we prove that any such pair are derived module-comodule equivalent: 
\begin{theorem}[Theorem \ref{theorem-module-comodule-correspondence-in-a-monoidal-dg-category}]
Let $(A,m_\bullet)$ be a strongly homotopy unital $\Ainfty$-algebra 
in a DG monoidal category $\A$. Let $(C,\Delta_\bullet)$
be a bicomodule homotopy counital $\Ainfty$-coalgebra. 
If $A$ and $C$ are homotopy adjoint as in 
Defn.~\ref{defn-homotopy-adjoint-ainfty-coalgebra-and-algebra},
then
$$ D_c(C) \simeq D_c(A). $$ 
\end{theorem}
The proof works by showing that if $C$ has a left homotopy adjoint in
$\A$, then any $\Ainfty$-coalgebra $(C,\Delta_\bullet)$ satisfies the
condition that free modules generated by perfect objects are perfect. 
Hence $D_c(C) \simeq D_c(\cokleisliC)$, while we also have $D_c(A) \simeq
D_c(\kleisliA)$. The homotopy adjunction of $A$ and $C$ as objects 
of $\A$ gives a homotopy equivalence on the
$\homm$-spaces of $\kleisliA$ and $\cokleisliC$. 
We then use the higher homotopies $\eta_{i}$ and $\epsilon_{i}$ to
extend this to an $\Ainfty$-functor and hence
an $\Ainfty$-quasi-equivalence $\kleisliA \simeq \cokleisliC$. 

\subsection{Examples and applications}

In \S\ref{section-examples-and-applications} we give examples and
applications of the theoretical results described above for
$\Ainfty$-algebras and coalgebras in an arbitrary monoidal DG category 
or DG bicategory $\A$. In
\S\ref{section-examples-associative-algebras}-\S\ref{section-examples-dg-and-ainfty-categories}
we study in detail how the classical notions of associative algebras, 
DG- and $\Ainfty$-algebras, and DG- and $\Ainfty$-categories fit into
our setting. 

In \S\ref{section-classical-and-ainfty-monads} we set
$\A$ to be the DG bicategory $\DGFuntwocat$ of small DG categories, DG
functors and DG natural transformations. Our theory gives 
a new notion of an \em $\Ainfty$-monad \rm structure $(T,m_\bullet)$
on a DG endofunctor $T$ of a DG category $\C$. We define
$\Ainfty$-modules over such $T$ and its \em weak Eilenberg-Moore
category $\eilmoorwk_T$ \rm which is just its DG category of homotopy
unital $\Ainfty$-modules. In particular, this gives the right notion
of the weak Eilenberg-Moore category for usual DG monads. Finally, in 
\S\ref{section-ainfty-modules-over-the-identity-functor} we look at
$\Ainfty$-modules over the identity endofunctor -- and 
show that these are the \em $\Ainfty$-idempotents \rm which appear in 
\cite[\S4]{GorskyHogancampWedrich-DerivedTracesOfSoergelCategories}. 

Finally, in \S\ref{section-enhanced-monads} we 
set $\A$ to be the bicategory $\enhcatkc$ 
of Karoubi complete enhanced triangulated categories, exact
functors, and natural transformation. Our theory gives 
a new notion of an \em enhanced exact monad \rm over an 
enhanced triangulated category and its \em weak Eilenberg-Moore
category\rm. In \S\ref{section-adjunction-monads-and-comonads}
for any adjoint pair $(F,R)$ of enhanced exact functors 
we construct explicitly enhanced monad and comonad structures 
on $RF$ and $FR$ which are strict and bimodule homotopy
unital and counital (Theorem
\ref{theorem-bimodule-homotopy-unitality-for-enhanced-adjunction-monads-and-comonads}). This leads to the following important result:
\begin{theorem}[Theorem
\ref{theorem-module-comodule-correspondence-for-enhanced-adjoint-triples}]
\label{theorem-intro-module-comodule-correspondence-for-enhanced-adjoint-triples}

Let $\C$ and $\D$ be enhanced triangulated categories, 
$F\colon \C \rightarrow \D$ be an enhanced exact functor, and 
$L,R\colon \D \rightarrow \C$ its left and right adjoints. 

Then the enhanced adjunction monad $RF$ and the enhanced adjunction monad $LF$ 
are derived module-comodule equivalent:
$$ D_c(RF) \simeq D_c(LF). $$
\end{theorem}
We will use this theorem as the key ingredient in our DG Barr-Beck 
construction, in other words - for homotopy descent and codescent. 
In a sense, it is an analogue of Positselski's derived comodule-contramodule 
correspondence
\cite[\S5]{Positselski-TwoKindsOfDerivedCategoriesKoszulDualityAndComoduleContramoduleCorrespondence}.
It would be interesting to give a direct comparison between these two results.

\em Acknowledgements: \rm We would like to thank Lino Amorim,
Alexander Efimov, Dmitri Kaledin, and Leonid Positselski 
for useful discussion. The first author would like to thank Kansas State
University for providing a stimulating research environment while
working on this paper. The third author would like to offer similar
thanks to Cardiff University. The second and the third author would
like to also thank Max-Planck-Institut f{\"u}r Mathematik Bonn. 

\section{Preliminaries}
\label{section-preliminaries}

\subsection{DG and $\Ainfty$-categories}

The main technical language of this paper is that of DG categories, 
their modules, and bimodules. For a brief introduction to these
and to the notation used in this paper we direct the reader to 
a survey in \cite{AnnoLogvinenko-SphericalDGFunctors}, \S2-4. 
Other sources of note are \cite{Keller-DerivingDGCategories},
\cite{Toen-TheHomotopyTheoryOfDGCategoriesAndDerivedMoritaTheory},
\cite{Toen-LecturesOnDGCategories}, and 
\cite{LuntsOrlov-UniquenessOfEnhancementForTriangulatedCategories}.

We summarise the key notions relevant to this paper. Throughout 
the paper we work in a fixed universe $\mathbb{U}$ of sets 
containing an infinite set. We also fix the base field or commutative 
ring $k$ we work over. 

We define $\modk$ to be the category of $\mathbb{U}$-small complexes of 
$k$-modules. It is a cocomplete closed symmetric monoidal category
with monoidal operation $\otimes_k$ and unit $k$. A \em DG category \rm 
is a category enriched over $\modk$. In particular, any DG category 
is locally small. 

If a DG category $\A$ is small, we write $\modA$ for the DG category
of (right) $\A$-modules. These are functors $\Aopp \rightarrow \modk$, 
so $\modA = \DGFun(\Aopp,\modk)$. Note that if $\A$ is not small, then 
$\modA$ isn't a DG category in the above sense - its morphism spaces 
do not lie in $\modk$. 

For an introduction to the usual $\Ainfty$-categories we recommend
\cite{Keller-AInfinityAlgebrasModulesAndFunctorCategories}, 
for a comprehensive technical text -- \cite{Lefevre-SurLesAInftyCategories}
and \cite{Lyubashenko-CategoryOfAinftyCategories}.  
For the summary of the technical details relevant to this paper, see
\cite[\S2]{AnnoLogvinenko-BarCategoryOfModulesAndHomotopyAdjunctionForTensorFunctors}.
In particular, in the examples in
\S\ref{section-enhanced-monads}-\S\ref{section-adjunction-monads-and-comonads}
we make extensive use of bar categories of modules and bimodules and 
the homotopy adjunctions in them introduced in \cite[\S3-4]{AnnoLogvinenko-BarCategoryOfModulesAndHomotopyAdjunctionForTensorFunctors}.

\subsection{Unbounded twisted complexes}
\label{section-unbounded-twisted-complexes}

To define $\Ainfty$-structures in an arbitrary monoidal DG category or 
DG bicategory, we rewrite them in 
\S\ref{section-ainfty-structures-in-monoidal-dg-categories} in the
language of twisted complexes. These were introduced in their bounded version 
by Bondal and Kapranov \cite{BondalKapranov-EnhancedTriangulatedCategories} as a tool to study DG enhancements of triangulated categories.  
In this paper we need the unbounded version introduced in
\cite{AnnoLogvinenko-UnboundedTwistedComplexes}. We 
give the key definitions below, and refer to 
\cite{AnnoLogvinenko-UnboundedTwistedComplexes} for all the technical details. 

As explained in \cite[\S3]{AnnoLogvinenko-UnboundedTwistedComplexes},
to define unbounded twisted complexes over a DG category $\A$ we 
need an embedding of $\A$ into another DG category $\B$ which admits 
countable direct sums and shifts. If $\A$ is small the standard choice
is $\B = \modA$, but as we see in this paper there are circumstances 
which warrant other choices. 

\begin{defn}
\label{defn-unbounded-twisted-complexes} 
Let $\A$ be a DG category with a fully faithful embedding into 
a DG category $\B$ which has countable direct sums and shifts. 

An \em unbounded twisted complex \rm over $\A$ relative to $\B$ consists of 
\begin{itemize}
\item $\forall\; i \in \mathbb{Z}$, an object $a_i$ of $\A$,
\item $\forall\; i,j \in \mathbb{Z}$, 
a degree $i - j + 1$ morphism $\alpha_{ij}\colon a_i \rightarrow a_j$ in $\A$,
\end{itemize}
satisfying 
\begin{itemize}
\item $\sum \alpha_{ij}$ is an endomorphism of $\bigoplus_{i \in
\mathbb{Z}} a_i[-i]$ in $\B$, 
\item The twisted complex condition 
\begin{equation}
\label{eqn-the-twisted-complex-condition}
(-1)^j d\alpha_{ij} + \sum_k \alpha_{kj} \circ \alpha_{ik} = 0. 
\end{equation}
\end{itemize}

Define \em DG category $\twcxub_\B(\A)$ of unbounded twisted complexes 
over $\A$ relative to $\B$ \rm by setting
\begin{equation}
\label{eqn-the-hom-complex-of-unbounded-twisted-complexes-naive}
\homm^\bullet_{\twcxub_{\B}(\A)}\bigl((a_i, \alpha_{ij}),(b_i,
\beta_{ij})\bigr) := \homm^\bullet_{\B}(\bigoplus_{k \in \mathbb{Z}} a_k[-k],
 \bigoplus_{l \in \mathbb{Z}} b_l[-l]) 
\end{equation}
where each $f \in \homm^q_\A(a_k, b_l)$ has degree $q + l - k$ and
\begin{equation}
\label{eqn-the-hom-complex-of-twisted-complexes-differential}
 df := (-1)^l d_{\A} f + \sum_{m \in
\mathbb{Z}}\left( \beta_{lm} \circ f - (-1)^{q + l -k} f 
\circ \alpha_{mk} \right).
\end{equation}
\end{defn}

We have a fully faithful convolution functor 
$$ \conv\colon\; \twcxub_{\B}(\A) \rightarrow \modB. $$
We also have a natural embedding
$$ \A \hookrightarrow \twcxub_{\B}(\A) $$
where the objects of $\A$ are sent to twisted complexes
concentrated in degree $0$.  

We say that DG category $\B$ 
\em admits convolutions of unbounded twisted complexes \rm
if $\B  \hookrightarrow \twcxub_{\B}(\B)$ is an equivalence. 
For any such $\B$ the convolution functor $\twcxub_{\B}(\A)
\rightarrow \modB$ for any $\A \subseteq \B$ takes values in $\B$. 

Where the choice of $\B$ is not relevant we write 
$\twcxub(\A)$ for $\twcxub_{\B}(\A)$. We 
define $\twcxpls(\A)$ and $\twcxmns(\A)$ to be the full subcategories
of $\twcxub(\A)$ consisting of all
bounded above and all bounded below twisted complexes, respectively. 

We say that a twisted complex $(a_i, \alpha_{ij})$ is \em one-sided \rm 
if $\alpha_{ij} = 0$ for $j \leq i$. We define $\pretriagub(\A)$,
$\pretriagpls(\A)$, and $\pretriagmns(\A)$ to be the full
subcategories of $\twcxub(\A)$, $\twcxpls(\A)$, and $\twcxmns(\A)$
consisting of one-sided twisted complexes. 

\subsection{The setting}
\label{section-the-setting}

Throughout the paper we work in a general setting of an arbitrary
monoidal DG category $\A$: 
\begin{defn}
A \em monoidal DG category \rm is a DG category $\C$ equipped with
\begin{itemize}
\item a \em monoidal operation, \rm functor 
$$ \C \otimes_k \C \rightarrow \C, $$
\item a \em unit \rm object $1_\C \in \C$, 
\item natural (in each of three arguments) \em associator \rm isomorphisms 
\begin{equation}
\alpha\colon (L \otimes M) \otimes N  
\xrightarrow{\sim} L \otimes (M \otimes N)
\quad \quad 
\forall\; L,M,N \in \C. 
\end{equation}
\item natural \em unitor \rm isomorphisms 
\begin{equation}
\rho\colon M \otimes 1_\C  \xrightarrow{\sim} M
\quad \quad \text{ and } \quad \quad 
\lambda\colon 1_\C \otimes M \xrightarrow{\sim} M 
\end{equation}
\end{itemize}
which make the following diagrams commute:
\begin{itemize}
\item For all $L,M,N \in \C$ the diagram  
\begin{small}
\begin{equation}
\label{eqn-associativity-coherence-diagram}
\begin{tikzcd}
((L \otimes M) \otimes N ) \otimes O 
\ar{r}{ \alpha \otimes \id }
\ar{d}{ \alpha }
&
(L \otimes (M \otimes N)) \otimes O 
\ar{r}{ \alpha } 
&
L \otimes ((M \otimes N) \otimes O)
\ar{d}{\id \otimes \alpha}
\\
(L \otimes M) \otimes (N \otimes O)
\ar{rr}{\alpha}
&
&
L \otimes (M \otimes (N \otimes O))
\end{tikzcd}
\end{equation}
\end{small}
\item For all $L,M \in \C$ the diagram 
\begin{equation}
\label{eqn-unitality-coherence-diagram}
\begin{tikzcd}
(L \otimes 1_\C) \otimes M 
\ar{rr}{\alpha}
\ar{dr}[']{\rho \otimes \id}
&&
L \otimes (1_\C \otimes M)
\ar{dl}{\id \otimes \lambda}
\\
&
L \otimes M. 
& 
\end{tikzcd}
\end{equation}
\end{itemize}
\end{defn}
In this paper we use $\otimes$ to denote the monoidal operation in $\A$. 

A monoidal category is a special case of a more general notion of a
bicategory: it is a bicategory with one object.  See
\cite{Benabou-IntroductionToBicategories} for a general introduction
to bicategories and
\cite{GyengeKoppensteinerLogvinenko-TheHeisenbergCategoryOfACategory}
for definitions of enriched bicategories and DG bicategories.  The
language of bicategories is less intuitive, so it was
a conscious expository choice of the authors to present our results in the
language of monoidal categories. All the results in this paper 
work equally well when $\A$ is a DG bicategory. We make use
of this in several places throughout the paper, particularly in 
\S\ref{section-examples-and-applications} where we give examples
and applications for specific instances of $\A$. 

Another such expository choice is the use of $\modA$ throughout 
\S\ref{section-ainfty-structures-in-monoidal-dg-categories}-\ref{section-strong-homotopy-unitality}. There $\modA$ is used as the ambient category $\B$
containing $\A$ which is necessary to define unbounded twisted
complexes over $\A$, cf.~\S\ref{section-unbounded-twisted-complexes}. 
Indeed, in those sections $\modA$ and the Yoneda embedding $\A
\hookrightarrow \modA$ can be replaced by a monoidal embedding of $\A$
into any monoidal DG category $\B$ which is closed, cocomplete (closed
under small direct sums) and admits convolutions of twisted
complexes. In
\S\ref{section-ainfty-structures-in-monoidal-dg-categories}-\ref{section-strong-homotopy-unitality}
this ambient category $\B$ is only used as a target for the
convolutions of twisted complexes over $\A$, so using $\modA$ and 
the Yoneda embedding would be more familiar to most readers.

Since we only consider one-sided bounded below or bounded above
twisted complexes over $\A$, the condition that $\B$ admits
convolutions of twisted complexes can be relaxed to $\B$
being strongly pretriangulated. This is because 
any cocomplete strongly pretriangulated
category admits convolutions of one-sided bounded below and bounded
above twisted complexes. 

When $\A$ is not small, we mean by $\modA$ the module category of $\A$ 
taken in any appropriate enlargement $\mathbb{V}$ of the universe
$\mathbb{U}$ to make $\A$ small. Since in \S\ref{section-ainfty-structures-in-monoidal-dg-categories}-\ref{section-strong-homotopy-unitality}
we only work with $\modA$ as a target for the convolution of twisted 
complexes over $\A$, the choice of $\mathbb{V}$ doesn't matter. 

Indeed, from \S\ref{section-the-derived-category} onwards, we switch 
to denoting the ambient category of $\A$ by $\B$, since we 
make a more extensive use of it to construct derived categories 
of $\Ainfty$-algebras and 
$\Ainfty$-coalgebras. We also make additional assumptions
on it which are described in the introduction to 
\S\ref{section-the-derived-category}. When $\A$ is small, all these
assumptions are still satisfied by $\modA$. 

When $\A$ is not small, one shouldn't just enlarge the universe 
to still use $\modA$. This would make all objects of $\A$ compact, 
and yield a wrong derived category. Indeed, this is the reason 
for the formalism of an arbitrary ambient category $\B$ satisfying
a list of assumptions. For example when $\A = \modC$ for small $\C$, 
we can set $\B$ to be $\A$ itself.  

\section{$\Ainfty$-structures in monoidal DG categories via twisted complexes}
\label{section-ainfty-structures-in-monoidal-dg-categories}

\subsection{$\Ainfty$-algebras}
\label{section-Ainfty-algebras-in-a-monoidal-category}

One of the central ideas of this paper is that unbounded twisted 
complexes can be used to reformulate the definitions 
of $\Ainfty$-algebras and modules 
\cite[\S2]{Lefevre-SurLesAInftyCategories} in a more 
convenient way. 

In particular, it gets rid of the necessity to work explicitly with 
the operation $m_1$, i.e. the differential, which simplifies a lot 
of formulas. Traditionally, $\Ainfty$-algebra formalism was defined for 
objects in the DG category $\modk$ of DG complexes of $k$-modules 
with its natural monoidal structure given by the tensor product of complexes
\cite[\S2]{Lefevre-SurLesAInftyCategories}. To define
$\Ainfty$-categories, one extends this formalism from $\modk$ to 
$k_S$-$\modd$-$k_S$ for some set $S$. Here $k_S$ is the category whose
object set is $S$ and whose only morphisms are $k$-multiples of the
identity morphisms. In other words, it is an $S$-indexed coproduct of $k$. 
See \cite[\S5]{Lefevre-SurLesAInftyCategories} for further details. 

In $k_A$-$\modd$-$k_A$, the internal differential of each object, that is
-- its differential as a complex of $k$-modules, exists 
as a degree $1$ endomorphism of the object. 
It can therefore be a part of the definition of an
$\Ainfty$-algebra or $\Ainfty$-module in $\modd$-$k_A$. Note that
this is no longer true if we replace $k_A$ by an arbitraty monoidal 
DG category $\A$. In $\modA$, the internal differentials of objects
do not appear as their degree $1$ endomorphisms. Moreover, if we
wanted to try and set up $\Ainfty$-formalism to work in $\A$ itself, 
its objects do not possess an internal differential. 

The language of twisted complexes solves both of these problems. 
It implicitly embeds the objects of $\A$ into $\modA$ as
$\Hom$-complexes of $\A$. These do have an internal differential: 
the differential $d_\A$ of $\A$. The twisted complex
condition \eqref{eqn-the-twisted-complex-condition} involves $d_\A$ and 
makes it possible to define an $\Ainfty$-algebra 
or module structure on an object $a \in \A$ while referring explicitly
only to operations $\left\{ m_i \right\}_{i \geq 2}$: 

\begin{defn}
\label{defn-algebra-bar-construction-in-a-monoidal-category}
Let $\A$ be a monoidal DG category, let $A \in \A$ and let 
$\left\{m_i\right\}_{i \geq 2}$ be a collection of degree $2-i$
morphisms $A^i \rightarrow A$. 

The \em (non-augmented) bar-construction $\infbarnaug(A)$ \rm of $A$ 
is the collection of objects $A^{i+1}$ for all $i \geq 0$ each placed
in degree $-i$ and of degree $k-1$ maps 
$d_{(i+k)i}\colon A^{i+k} \rightarrow A^i$ 
defined by
\begin{equation}
\label{eqn-differentials-in-non-aug-bar-construction}
d_{(i+k)i} := (-1)^{(i-1)(k+1)} \sum_{j = 0}^{i-1} (-1)^{jk} \id^{i-j-1}
\otimes m_{k+1} \otimes \id^{j}. 
\end{equation} 

\begin{tiny}
\begin{equation}
\label{eqn-nonaugmented-bar-construction-of-A-m_i}
\begin{tikzcd}[column sep = 2.5cm]
\dots
\ar{r}[']{\begin{smallmatrix}A^3 m_2  - A^2m_2A + \\ + Am_2A^2 - m_2 A^3 \end{smallmatrix}}
\ar[bend left=25]{rr}[description]{A^2m_3 + Am_3A +  m_3 A^2}
\ar[bend left=30]{rrr}[description]{Am_4 - m_4A}
\ar[bend left=35]{rrrr}[description]{m_5}
&
A^4
\ar{r}[']{A^2m_2 - A m_2 A + m_2 A^2}
\ar[bend left=25]{rr}[description]{-Am_3 - m_3A}
\ar[bend left=30]{rrr}[description]{m_4}
& 
A^3
\ar{r}[']{Am_2 - m_2A}
\ar[bend left=25]{rr}[description]{m_3}
&
A^2
\ar{r}[']{m_2}
&
\underset{\degzero}{A}
\end{tikzcd}
\end{equation}
\end{tiny}

\end{defn}

The sign twist in this definition can be read as follows: we first 
set the maps in the bar-construction to be the sums of the applications, 
from right to left, of the appropriate maps $m_k$. These sums are
sign-alternating for even $k$, and regular for odd $k$. 
We then further twist each map by $(-1)^{ij}$ where $i$ and $j$ are 
the degrees of its source and its target. Note that since the degree of 
such map is $i - j + 1$, this can be equivalently described as twisting 
every odd-degree differential by $(-1)^{n}$, where $n$ is the degree of 
either its source or its target. 

\begin{defn}
\label{defn-ainfty-algebra-in-a-monoidal-category}
Let $\A$ be a monoidal DG category. An \em $\Ainfty$-algebra $(A,m_i)$ \rm 
in $\A$ is an object $A \in \A$ equipped 
with operations $m_i \colon A^i \rightarrow A$ for all $i \geq 2$
which are degree $2-i$ morphisms in $\A$ such that their non-augmented 
bar-construction $\infbarnaug(A)$ is a twisted complex over $\A$. 
\end{defn}

Next, we compare 
Defn.~\ref{defn-ainfty-algebra-in-a-monoidal-category} 
with the standard definition of an $\Ainfty$-algebra 
\cite[D{\'e}finition 1.2.1.1]{Lefevre-SurLesAInftyCategories}. 
The latter was stated for graded categories, but for the convenience 
of comparison we state it for DG categories. As mentioned before, 
it only works when the monoidal DG category $\A$ is $\modk$ or, more
generally, $k_S$-$\modd$-$k_S$, with the monoidal structure
given by the tensor product over $k$. 

An $\Ainfty$-algebra in such $\A$ is an object $A \in \A$
equipped with degree $2-k$ morphisms $m_k\colon A^{k} \rightarrow A$ 
for all $k \geq 1$ which satisfy the following. The morphism
$m_1$ is the internal differential of $A$, and 
\begin{equation}
\label{eqn-Ainfty-algebra-definition-equalities}
\forall\;\; i \geq 1
\quad \quad 
\sum_{j+k+l = i} (-1)^{jk + l} m_{j+1+l} \circ (\id^j \otimes m_k \otimes \id^l) = 0 
\end{equation}

A key result \cite[Lemme 1.2.2.1]{Lefevre-SurLesAInftyCategories}
shows that this definition is equivalent to the following. 
By the universal property of free coalgebras, the morphism  
$$ \bigoplus_{i \geq 1} (A[1])^i \rightarrow A $$
induced by $\sum_{i \geq 1} m_i$ defines a coderivation 
$$ b\colon 
\bigoplus_{i \geq 1} (A[1])^i \rightarrow \bigoplus_{i \geq 1} (A[1])^i. $$
Operations $(m_k)_{k \geq 1}$ define $\Ainfty$-algebra structure on
$A$ if and only if $b$ is a differential on the free coalgebra 
$\bigoplus_{i \geq 1} (A[1])^i$, i.e. $b^2 = 0$. 

The coderivation of $\bigoplus_{i \geq 1} (A[1])^i$
induced by $m_1$ is its internal differential. 
On the other hand, we defined the non-augmented 
bar-construction \eqref{eqn-nonaugmented-bar-construction-of-A-m_i}
so as to ensure the following: 

\begin{lemma}
\label{lemma-total-endomorphism-of-nonaug-bar-constr-is-a-coderivation-induced-by-m_i}
Let $\A$ be a monoidal DG category, let $A \in \A$ and let 
$m_k\colon A^k \rightarrow A$ be a collection of degree $2-k$
of morphisms in $\A$. In $\modA$, let 
$$ \mu \colon  \bigoplus_{i \geq 1} A^i[i-1] \rightarrow
\bigoplus_{i \geq 1} A^i[i-1] $$
be the sum of all the differentials in the non-augmented bar-construction 
\eqref{eqn-nonaugmented-bar-construction-of-A-m_i}. 

Denote by $w\colon A[1] \rightarrow A$ the degree $1$ map given 
by $\id_A$. For any $i \geq 0$, we can view $w^i$ as an isomorphism 
$(A[1])^i \xrightarrow{\sim} A^i[i]$. The resulting isomorphism 
in $\modA$
$$ \sum_{i\geq 1} w^i\colon \bigoplus_{i \geq 1} (A[1])^i \rightarrow
\bigoplus_{i \geq 1} A^i[i] $$
identifies the coderivation of $\bigoplus (A[1])^i$ induced by 
$\sum_{i \geq 2} m_i$ with $\mu[1]$. 
\end{lemma}
Here and below we use the fact that 
the monoidal structure on $\A$ induces the monoidal structure on $\modA$
\cite[\S4.5]{GyengeKoppensteinerLogvinenko-TheHeisenbergCategoryOfACategory}.

\begin{proof}
Recall the definition of the coderivation of $\bigoplus_{i\geq 1} (A[1])^i$
induced by $\sum_{i \geq 2} m_i$ 
\cite[\S1.2.2]{Lefevre-SurLesAInftyCategories}. 
Let $s\colon A \rightarrow A[1]$ be the degree $-1$ map defined by
$\id_A$. It is the inverse of $w$, however observe that 
$s^i \circ w^i = w^i \circ s^i = (-1)^{i(i-1)/2} \id_{A^k}$. 
Define $b_k\colon (A[1])^k \rightarrow A[1]$ by 
$b_k = - s \circ m_j \circ w^k$. We then have a degree $1$ map in $\A$
$$ \sum_{i \geq 2} b_i\colon \bigoplus (A[1])^i \rightarrow A[1]. $$
By the universal property of free coalgebras 
\cite[Lemme 1.1.2.2]{Lefevre-SurLesAInftyCategories}
this map corresponds to the degree $1$ coderivation
$$ b\colon \bigoplus_{i \geq 1} (A[1])^i \rightarrow \bigoplus_{i \geq
1} (A[1])^i $$ 
whose components are the maps 
$(A[1])^{i+k} \rightarrow (A[1])^i$ with $k \geq 1$ given by 
$$ \sum_{j = 0}^{i-1} \id^{i-j-1} \otimes b_{k+1} \otimes \id^j. $$ 

The isomorphism $\sum w^i$ and its inverse $\sum (-1)^{i(i-1)/2} s^i$
identify $b$ with the endomorphism of $\bigoplus_{i \geq 1} A^i[i]$ 
whose components are the maps $A^{i+k}[i+k] \rightarrow A^{i}[i]$
with $k \geq 1$ given by 
\begin{small}
\begin{align*}
& \sum_{j = 0}^{i-1} (-1)^{(i+k)(i + k - 1)/2} w^i \circ (\id^{i-j-1} \otimes
b_{k+1} \otimes \id^j) \circ s^{i+k} = 
\\  
= & 
\sum_{j = 0}^{i-1} (-1)^{(i+k)(i + k - 1)/2 + 1 } w^i \circ 
(\id^{i-j-1} \otimes (s \circ m_{k+1} \circ w^{k+1}) \otimes \id^j) 
\circ s^{i+k} = 
\\
= &
\sum_{j = 0}^{i-1} (-1)^{(i+k)(i + k - 1)/2 + 1 + j } 
(w^{i-j-1} \otimes \id \otimes w^j) \circ 
(\id^{i-j-1} \otimes (m_{k+1} \circ w^{k+1}) \otimes \id^j) 
\circ s^{i+k} = 
\\
= &
\sum_{j = 0}^{i-1} (-1)^{(i+k)(i + k - 1)/2 + 1 + j + (i-1)(k+1) + j(k+1)} 
(\id^{i-j-1} \otimes m_{k+1} \otimes \id^j) 
\circ
w^{i+k} \circ s^{i+k} = 
\\
= &
\sum_{j = 0}^{i-1} (-1)^{1 + (i-1)(k+1) + jk)} 
(\id^{i-j-1} \otimes m_{k+1} \otimes \id^j). 
\end{align*}
\end{small}
This only differs from the formula 
\eqref{eqn-differentials-in-non-aug-bar-construction}
defining the differentials in the non-augmented bar-construction 
by a minus sign, whence the resulting endomorphism of 
$\bigoplus_{i \geq 1} A^i[i]$ equals $\mu[1] = -\mu$. 
\end{proof}

Our Defn.~\ref{defn-ainfty-algebra-in-a-monoidal-category} asks 
for \eqref{eqn-nonaugmented-bar-construction-of-A-m_i} to be a twisted
complex, that is –– adding $\mu$ to the internal differential of 
$\bigoplus_{i \geq 1} A^i[i-1]$ gives 
a new differential on the latter. This is equivalent to the condition
that adding $\mu[1]$ to the internal differential of 
$\bigoplus_{i \geq 0} A^i[i]$ also gives a new differential on the latter. 
The isomorphism $\sum_{i \geq 1} w^i$ identifies the internal
differential of $\bigoplus_{i \geq 1} A^i[i]$ with the internal
differential of $\bigoplus_{1 \geq 1} (A[1])^i$, i.e with the
coderivation induced by $m_1$. On the other hand, it identifies 
$\mu[1]$ with the coderivation induced by $\sum_{k \geq 2} m_k$.  
We thus obtain:
\begin{cor}
For $\A$ either $\modk$ or $k_S$-$\modd$-$k_S$, the following
are equivalent:
\begin{enumerate}
\item Our Defn.~\ref{defn-ainfty-algebra-in-a-monoidal-category}
\item The usual definition \cite[D{\'e}finition
1.2.1.1]{Lefevre-SurLesAInftyCategories} of an $\Ainfty$-algebra. 
\item The coderivation induced by $\sum_{i \geq 1} m_i$ on 
the free coalgebra $\bigoplus_{i \geq 0} (A[1])^i$ is a differential. 
\end{enumerate}
\end{cor}

When $\A$ is an arbitrary monoidal DG category, its objects do 
not apriori have internal differentials. This can be fixed by applying 
Yoneda embedding $\A \hookrightarrow \modA$ which sends $A$ to 
$\homm_\A(-,A)$. The latter does have an internal differential
given by the differential of $\A$. We thus have a degree one 
map $m_1$ from $\homm_\A(-,A)$ to itself. However this map doesn't 
respect the right action of $\A$ by pre-composition, and so  
isn't a morphism in $\modA$, but only in $\modd\text{-}k_\A$. 

In this more general context of an arbitrary monoidal category, 
our Defn.~\ref{defn-ainfty-algebra-in-a-monoidal-category} is still
equivalent to the condition that adding the coderivation induced 
by $\sum_{i \geq 2} m_i$ to the natural differential of free coalgebra 
$\bigoplus_{i \geq 1} (A[1])^i$ gives a new differential on the
latter. Thus the $\Ainfty$-structures on $A$ still correspond
bijectively to differentials on $\bigoplus_{i \geq 1} (A[1])^i$, 
generalising \cite[Lemme 1.2.2.1]{Lefevre-SurLesAInftyCategories}. 

But what about the original definition in terms of the explicit
equalities \eqref{eqn-Ainfty-algebra-definition-equalities} which 
$(m_k)_{k \geq 1}$ have to satisfy? One might be tempted to think 
that our Defn.~\ref{defn-ainfty-algebra-in-a-monoidal-category}
is equivalent to asking for the internal differential $m_1$ of $A$
and the higher operations $(m_k)_{k \geq 2}$ to satisfy 
\eqref{eqn-Ainfty-algebra-definition-equalities} in some appropriate
category. This is not so, and the main reason is that
for an arbitrary monoidal DG category 
$\A$ it is no longer true that 
the coderivation of $\bigoplus_{i \geq 1} (A[1])^i$
induced by $m_1$ is its internal differential in $\modA$. It is
no longer possible to express the internal differential of each $A^k$ 
in terms of $m_1$, so it is no longer possible to restate our 
Defn.~\ref{defn-ainfty-algebra-in-a-monoidal-category} as a collection
of equalities which involve only $m_1$ and $(m_k)_{k \geq 2}$. 

In fact, it is impossible to even write down the equalities 
\eqref{eqn-Ainfty-algebra-definition-equalities}
for an arbitrary monoidal structure on $\A$. This is because  
$m_1$ is not a morphism in $\modA$, and so is not, apriori,
compatible with this monoidal structure. 
For example, we can't even define the operation $\id \otimes m_1$ on 
the module $\homm_\A(-,A^2)$, since arbitrary map $B \rightarrow A^2$ in 
$\A$ doesn't have to be of 
the form $f \otimes g$ for some
$f,g\colon A \rightarrow B$. Similarly, we can't define any of
the maps 
$\id^{i-l-1} \otimes m_1 \otimes \id^l: A^{i} \rightarrow A^{i}$ 
in \eqref{eqn-Ainfty-algebra-definition-equalities}. 

However, even though individual maps $\id^{i-l-1} \otimes m_1 \otimes \id^l$
do not exist, we can still make sense of their sum
$$ \sum_{i=0}^{i-1} \id^{i-l-1} \otimes m_1 \otimes \id^l. $$
When $\A = \modk$, this is the natural differential $d_{\A}$ on 
$\homm_\A(-,A^i)$. Hence the sum of all terms involving $m_1$ in 
\eqref{eqn-Ainfty-algebra-definition-equalities} equals $d_\A m_i$.
Rewriting \eqref{eqn-Ainfty-algebra-definition-equalities} to 
replace the former with the latter yields those twisted
complex conditions for \eqref{eqn-nonaugmented-bar-construction-of-A-m_i}   
which involve the differentials of the arrows labelled with $m_i$.  
This is the way to compare  
Defn.~\ref{defn-ainfty-algebra-in-a-monoidal-category} with the 
original definition \cite[D{\'e}finition
1.2.1.1]{Lefevre-SurLesAInftyCategories} of an $\Ainfty$-algebra:

\begin{prps}
\label{prps-comparing-new-and-old-defns-of-ainfty-algebra}
Let $\A$ be a monoidal DG category, let $A \in \A$ and let 
$(m_i)_{i \geq 2}$  be a collection of degree $2-i$
morphisms $A^i \rightarrow A$ in $\A$. 

\begin{enumerate}
\item 
\label{item-defining-equalities-of-Ainfty-are-subset-of-twisted-complex-conditions}
Take the defining equalities 
\eqref{eqn-Ainfty-algebra-definition-equalities} in the original 
definition of $\Ainfty$-algebra and replace all terms involving $m_1$ with $d_\A m_i$:
\begin{equation}
\label{eqn-defining-equalities-for-new-definition-of-Ainfinity-algebra}
\forall\; i \geq 2 \quad\quad 
d_\A m_i + \sum_{\begin{smallmatrix}j+k+l = i, \\ k \geq
2\end{smallmatrix}} (-1)^{jk+l} m_{j+1+l} \circ \left(\id^j \otimes
m_k \otimes \id^l\right) = 0. 
\end{equation}
The resulting equalities are a subset of twisted complex conditions  \eqref{eqn-the-twisted-complex-condition} for non-augmented bar
construction \eqref{eqn-nonaugmented-bar-construction-of-A-m_i}. It consists of those conditions which involve differentiating the twisted differentials $m_i\colon A^k \rightarrow A$.
 
\item 
\label{item-twcx-condition-on-m-i-imply-the-rest}
If the equalities \eqref{eqn-defining-equalities-for-new-definition-of-Ainfinity-algebra} hold, then so do the rest of the twisted complex conditions for \eqref{eqn-nonaugmented-bar-construction-of-A-m_i}. 

Thus  $(A,m_i)$ is an $\Ainfty$-algebra in
the sense of Defn.~\ref{defn-ainfty-algebra-in-a-monoidal-category} if and only if $m_i$ satisfy the equalities 
\eqref{eqn-defining-equalities-for-new-definition-of-Ainfinity-algebra}. 
\end{enumerate} 
\end{prps}

\begin{proof}
\underline{\eqref{item-defining-equalities-of-Ainfty-are-subset-of-twisted-complex-conditions}}:

Let $i \geq 2$. Taking the formula \eqref{eqn-differentials-in-non-aug-bar-construction} for the twisted differentials in the non-augmented bar-construction \eqref{eqn-nonaugmented-bar-construction-of-A-m_i} and using it to write out the twisted complex condition \eqref{eqn-the-twisted-complex-condition} for the twisted differential $m_i: A^i \rightarrow A$ in \eqref{eqn-nonaugmented-bar-construction-of-A-m_i} we obtain:
\begin{small}
\begin{align*}
& d_\A m_i + \sum_{p = 2}^{i-1} m_p \circ 
\left(
(-1)^{(p-1)(i-p+1)}
\sum_{q = 0}^{p-1}
(-1)^{q(i-p)} \id^{p-q-1} \otimes m_{i-p+1} \otimes \id^q 
\right) = 0 \; \Leftrightarrow
\\
\Leftrightarrow \;\; & d_\A m_i + 
\sum_{p = 2}^{i-1}
\sum_{q = 0}^{p-1}
(-1)^{(p-q+1)(i-p+1)+ q}
 m_p \circ 
 \id^{p-q-1} \otimes m_{i-p+1} \otimes \id^q 
  = 0.
\end{align*}
\end{small}
A change of summation indices $l = q$, $j = p-q-1$, and $k = i - j - l = i - p + 1$ transforms the above into the equality 
\eqref{eqn-defining-equalities-for-new-definition-of-Ainfinity-algebra}, as desired. 

\underline{\eqref{item-twcx-condition-on-m-i-imply-the-rest}}:

% The twisted complex condition for a twisted differential
% $d_{(i+n)i}\colon A^{i+n} \rightarrow A^i$ is
% \begin{equation}
% (-1)^i d_\A \left(d_{(i+n)(i)}\right) + \sum_{k+l = n} d_{(i+k)i} \circ d_{(i+k+l)(i+k)} = 0.
% \end{equation}
% Applying the formula \eqref{eqn-differentials-in-non-aug-bar-construction}
% we have 
% \begin{scriptsize}
% \begin{align*}
% & \sum_{k+l = n} d_{(i+k)i} \circ d_{(i+k+l)(i+k)} = 
% \\
% = &\sum_{k+l = n} \left((-1)^{(i-1)(k+1)} \sum_{p = 0}^{i-1}
% (-1)^{pk} \id^{i-p-1} \otimes m_{k+1} \otimes \id^p\right)
% \circ 
% \left((-1)^{(i+k-1)(l+1)}\sum_{q = 0}^{i+k-1}
% (-1)^{ql} \id^{i+k-q-1} \otimes m_{l+1} \otimes \id^q\right) =
% \\
% = 
% & (-1)^{(i-1)n}
% \sum_{k+l = n} \sum_{p = 0}^{i-1} \sum_{q = 0}^{i+k-1}
% (-1)^{k(l+1) + pk + ql}
% \left(\id^{i-p-1} \otimes m_{k+1} \otimes \id^p\right)
% \circ 
% \left(\id^{i+k-q-1} \otimes m_{l+1} \otimes \id^q\right).
% \end{align*}
% \end{scriptsize}
% In this sum, there are two types of summands:
% \begin{enumerate}
% \item The summands where $q < p$ or $q > p + k$. 
% 
% In this case, the target of $m_{l+1}$ lies outside the source of $m_{k+1}$. 
% The summands with $q < p$ are those where the target of $m_{l+1}$
% occurs to the right of the source of $m_{k+1}$ and those with 
% $q > p + k$ are those where it occurs to the left.  
% We claim that every summand with $q < p$ gets cancelled out
% by a summand with $q > p+k+1$, and thus the total contribution
% of these two types of summands is zero. 
% 
% Indeed, for any $a,b,c$ with $a + 1 + b + 1 + c = i$, 
% the square 
% \begin{equation}
% \begin{tikzcd}[column sep = 2.5cm, row sep = 1.5cm]
% A^{a} A^{k+1} A^{b} A^{l+1} A^c 
% \ar{r}{\id^{a+k+1+b} m_{l+1} \id^c}
% \ar{d}{\id^{a} m_{k+1} \id^{b+l+1+c}}
% &
% A^{a}A^{k+1} A^{b} A^1 A^{c}
% \ar{d}{\id^{a} m_{k+1} \id^{b+1+c}}
% \\
% A^{a}A^1 A^{b} A^{l+1} A^c 
% \ar{r}{\id^{a+1+b} m_{l+1} \id^c}
% &
% A^{a}A^1 A^{b} A^1 A^c.
% \end{tikzcd}
% \end{equation}
% commutes up to the sign twist $(-1)^{(1-k)(1-l)}$.
% The coefficient of 
% $$
% \left(\id^{a} \otimes m_{k+1} \otimes \id^{b+1+c}\right) 
% \circ
% \left(\id^{a+k+1+b} \otimes m_{l+1} \otimes \id^c\right)
% $$
% in the sum above is $ (-1)^{k(l+1) + (b+1+c)k + cl}$,
% while the coefficient of 
% $$
% \left(\id^{a+1+b} \otimes m_{l+1} \otimes \id^c \right)
% \circ
% \left(\id^{a} \otimes m_{k+1} \otimes \id^{b+l+1+c}\right)
% $$
% is 
% $ (-1)^{l(k+1) + cl + (b+l+1+c)k}$. 
% Hence the two summands cancel each other out. 
% \end{enumerate}

Let $\mu$ be the total endomorphism of 
$\bigoplus_{i \geq 1} A^i[i-1]$ defined by the sum of all the maps in 
the non-augmented bar-construction 
\eqref{eqn-nonaugmented-bar-construction-of-A-m_i}. The twisted
complex condition \eqref{eqn-the-twisted-complex-condition} for each 
map $A^i \rightarrow A^j$ in \eqref{eqn-nonaugmented-bar-construction-of-A-m_i}
is precisely the condition that the $A^i[i-1] \rightarrow A^j[j-1]$ component
of 
\begin{equation}
\label{eqn-twisted-complex-condition-on-mu}
d_\A (\mu) + \mu \circ \mu 
\end{equation}
vanishes. Thus, we need to show that if all $A^k[k-1] \rightarrow A$
components of 
\eqref{eqn-twisted-complex-condition-on-mu} vanish, 
then all the remaining components vanish as well. 

By Lemma  
\ref{lemma-total-endomorphism-of-nonaug-bar-constr-is-a-coderivation-induced-by-m_i},
natural isomorphisms $w^i\colon (A[1])^i \rightarrow A^i[i]$ identify 
$\mu[1]$ with the coderivation $b$ of $\bigoplus (A[1])^i$ induced by 
$\sum_{i \geq 2} m_i$. By the universal property of tensor coalgebras, 
any coderivation of $\bigoplus (A[1])^i$ is uniquely determined by its
$(A[1])^k \rightarrow A[1]$ components. In particular, this is true of
$d_\A (b) + b \circ b$. Hence, it is also true of 
$d_\A (\mu[1]) + \mu[1] \circ \mu[1]$, and thus of
\eqref{eqn-twisted-complex-condition-on-mu}. 
\end{proof}

We next define morphisms of $\Ainfty$-algebras in $\A$ in a similar
way:

\begin{defn} 
\label{defn-the-bar-construction-of-a-morphism-of-ainfty-algebras}
Let $(A, m_k)$ and $(B, n_k)$ be $\Ainfty$-algebras in $\A$. 
Let $(f_i)_{i \geq 1}$ be a collection of degree $1 - i$ 
morphisms $A^i \rightarrow B$. 

The \em bar-construction \rm $\infbar(f_\bullet)$ is the morphism 
$\infbarnaug(A) \rightarrow \infbarnaug(B)$ in $\pretriagmns(\A)$
whose $A^{i+k} \rightarrow B^i$ component is 
$$ \sum_{t_1 + \dots + t_i = i + k} 
(-1)^{\sum_{l=2}^{i}(1-t_l)\sum_{n=1}^l t_n}
f_{t_1}\otimes\ldots \otimes f_{t_i}. $$
\begin{small}
\begin{equation*}
\begin{tikzcd}[column sep = 2.6cm, row sep = 3cm]
\dots
\ar{r}
&
A^4
\ar{r}
\ar{d}[description, pos = 0.80]{f_1f_1f_1f_1}
\ar{dr}[description, pos = 0.60]{f_1f_1f_2 - f_1 f_2 f_1 + f_2 f_1 f_1}
\ar{drr}[description, pos = 0.30]{f_1f_3 + f_2 f_2 + f_3 f_1}
\ar{drrr}[description, pos = 0.10]{f_4}
&
A^3
\ar{r}
\ar{d}[description, pos = 0.80]{f_1f_1f_1}
\ar{dr}[description, pos = 0.60]{-f_1f_2 + f_2 f_1}
\ar{drr}[description, pos = 0.30]{f_3}
&
A^2
\ar{r}
\ar{d}[description, pos = 0.80]{f_1f_1}
\ar{dr}[description, pos = 0.60]{f_2}
&
A
\ar{d}[description, pos = 0.80]{f_1}
\\
\dots
\ar{r}
&
B^4
\ar{r}
&
B^3
\ar{r}
&
B^2
\ar{r}
&
B.
\end{tikzcd}
\end{equation*}
\end{small}
\end{defn}
\begin{defn}
\label{defn-morphism-of-ainfty-algebras}
A morphism $f_\bullet\colon (A,m_k) \rightarrow (B, n_k)$ of 
$\Ainfty$-algebras is a collection 
$(f_i)_{i \geq 1}$ of degree $1 - i$ morphisms $A^i \rightarrow B$ 
whose bar-construction is a closed degree $0$ morphism of twisted
complexes. 
\end{defn}

As before, this definition translates into the language of twisted
complexes the corresponding morphism of tensor coalgebras. 
Recall that the universal property of tensor coalgebras ensures 
that for any object $V$ and any coalgebra $C$ in $\A$, the coalgebra
morphisms $f\colon C \rightarrow \bigoplus_{k \geq 1} V^k$ in $\modA$ 
are in bijective correspondence with all morphisms $g\colon C \rightarrow V$. 
The correspondence sends any $f$ to its restriction to a morphism 
$C \rightarrow V$, and conversely sends any $g$ to 
$\sum_i f^i \circ \Delta^{(i)}$. 

On the level of graded $\A$-modules, 
$\conv \infbarnaug (A)$ is the same as $\bigoplus_{i} A^i[i-1]$. 
We then have:

\begin{lemma}
\label{lemma-ainfty-moprhisms-of-ainfty-algebras-identified-with-coalgebra-morphisms}
Let $\A$ be a monoidal DG category, 
let $(A, m_k)$ and $(B, n_k)$ be $\Ainfty$-algebras in $\A$, 
and let $(f_i)_{i \geq 1}$ be a collection of degree $1 - i$ 
morphisms $A^i \rightarrow B$. 

Then, on the level of graded $\A$-modules, 
the isomorphisms $\sum_{i\geq 1} w^i$
identify the coalgebra morphism induced by $\sum_{i \geq 1} f_i$ 
$$ \bigoplus (A[1])^i \rightarrow \bigoplus (B[1])^i $$
with $\conv \infbarnaug (f_\bullet)[1]$
$$ \bigoplus_{i} A^i[i] \rightarrow \bigoplus_{i} B^i[i]. $$
\end{lemma}
\begin{proof}
Similar to the proof of Lemma 
\ref{lemma-total-endomorphism-of-nonaug-bar-constr-is-a-coderivation-induced-by-m_i}. 
\end{proof}

\begin{prps}
\label{prps-defining-equalities-of-Ainfty-morphism}
Let $\A$ be a monoidal DG category,
let $(A, m_k)$ and $(B, n_k)$ be $\Ainfty$-algebras in $\A$, 
and let $(f_i)_{i \geq 1}$ be a collection of degree $1 - i$ morphisms 
$A^i \rightarrow B$.

\begin{enumerate}
\item 
\label{item-defining-equalities-of-Ainfty-morphism-are-subset-of-twisted-complex-conditions}
Take the defining equalities in the  
definition of a morphism of $\Ainfty$-algebras in
\cite[Defn.~1.2.1.2]{Lefevre-SurLesAInftyCategories} 
and replace all terms involving $m_1$ with $d_\A f_i$:
\begin{align}
\label{eqn-defining-equalities-for-new-definition-of-Ainfinity-algebra-morphism}
\forall\; i \geq 1 \quad\quad 
d_\A f_i & + 
\sum_{i_1 + \dots + i_r = i} (-1)^{\sum_{2\leq u < r}\left( (1-i_u)
\sum_{1 \leq v \leq u} i_v \right)} m_r \circ \left(f_{i_1} \otimes \dots
\otimes f_{i_r}\right) - 
\\
\nonumber
& -  
\sum_{\begin{smallmatrix}j+k+l = i, \\ k \geq 2\end{smallmatrix}}
(-1)^{jk+l} f_{j+1+l} \circ \left(\id^j \otimes m_k \otimes
\id^l\right) = 0. 
\end{align}
The resulting equalities are a subset of the set of equalities which 
equate each component $A^i \rightarrow B^j$ 
of the twisted complex map $d\infbar(B_\bullet)$ with zero. 
This subset consists of the equalities for the components $A^i \rightarrow B$.
 
\item 
\label{item-closed-condition-on-f-i-imply-the-rest}
If the equalities 
\eqref{eqn-defining-equalities-for-new-definition-of-Ainfinity-algebra-morphism}
hold, then so do the rest of the equalities for the components of
$d\infbar(B_\bullet)$ to be zero. 

Thus $f_\bullet$ is a morphism of $\Ainfty$-algebra in
the sense of Defn.~\ref{defn-morphism-of-ainfty-algebras}
if and only if $f_i$ satisfy the equalities 
\eqref{eqn-defining-equalities-for-new-definition-of-Ainfinity-algebra-morphism}. 
\end{enumerate} 
\end{prps}
\begin{proof}
Similar to the proof of
Prps.~\ref{prps-comparing-new-and-old-defns-of-ainfty-algebra}, with
the following difference. 
In \eqref{item-closed-condition-on-f-i-imply-the-rest} we can not
apply the universal property of tensor coalgebra morphisms to the
differential of the coalgebra morphism 
$F: \bigoplus (A[1])^i \rightarrow \bigoplus (B[1])^i$ induced by
$\sum_{i\geq 1} f_i$. This is because $dF$ is not a coalgebra
morphism. It is, however, an $(F,F)$-coderivation, Therefore
by the universal property of the coderivations of tensor
colagebras \cite[Lemme~1.1.2.2]{Lefevre-SurLesAInftyCategories} 
the map $dF$ is uniquely determined by its $(A[1])^k \rightarrow B[1]$ 
components. 
\end{proof}

\subsection{$\Ainfty$-modules}
\label{section-Ainfty-modules-in-a-monoidal-category}

We define left and right $\Ainfty$-modules using the same ideas 
we used to define $\Ainfty$-algebras in
\S\ref{section-Ainfty-modules-in-a-monoidal-category}: we define the 
left and right module bar-constructions so as to
ensure that analogues of Lemmas 
\ref{lemma-total-endomorphism-of-nonaug-bar-constr-is-a-coderivation-induced-by-m_i}
and 
\ref{lemma-ainfty-moprhisms-of-ainfty-algebras-identified-with-coalgebra-morphisms}
hold. That is, the totalisations of bar-constructions can be identified with 
the coderivations and morphisms of the corresponding free 
comodules induced by the same data. 

\begin{defn}
\label{defn-right-module-bar-construction-in-a-monoidal-category}
Let $(A,m_i)$ be an $\Ainfty$-algebra in a monoidal DG category $\A$. 
Let $E \in \A$ and let $\left\{p_i\right\}_{i \geq 2}$ be a collection 
of degree $2-i$ morphisms $E \otimes A^{i-1} \rightarrow E$. 

The \em right module bar-construction $\infbar(E)$ \rm of $(E,p_i)$
comprises objects $E \otimes A^{i}$ for $i \geq 0$ 
placed in degree $-i$ and degree $1-k$ maps 
$E \otimes A^{i+k-1} \rightarrow E \otimes A^{i-1}$ 
defined by
\begin{scriptsize}
\begin{equation}
\label{eqn-differentials-in-right-module-bar-construction}
d_{(i+k)i} := (-1)^{(i-1)(k+1)} 
\left(\sum_{j = 0}^{i-2} \left( (-1)^{jk} \id^{i-j-1} \otimes m_{k+1} \otimes
\id^{j}\right) + (-1)^{(i-1)k}p_{k+1}\otimes \id^{i-1} \right). 
\end{equation} 
\end{scriptsize}

\begin{tiny}
\begin{equation}
\label{eqn-right-module-bar-construction-of-A-m_i}
\begin{tikzcd}[column sep = 2.4cm]
\dots
\ar{r}[']{\begin{smallmatrix}EA^2 m_2  - EAm_2A + \\ + Em_2A^2 - p_2 A^3 \end{smallmatrix}}
\ar[bend left=25]{rr}[description]{EA m_3 + Em_3A + p_3 A^2 }
\ar[bend left=30]{rrr}[description]{Em_4 - p_4A}
\ar[bend left=35]{rrrr}[description]{p_5}
&
EA^3
\ar{r}[']{EAm_2 - E m_2 A + p_2 A^2}
\ar[bend left=25]{rr}[description]{-Em_3 - p_3A}
\ar[bend left=30]{rrr}[description]{p_4}
& 
EA^2
\ar{r}[']{Em_2 - p_2A}
\ar[bend left=25]{rr}[description]{p_3}
&
EA
\ar{r}[']{p_2}
&
\underset{\degzero}{E}.
\end{tikzcd}
\end{equation}
\end{tiny}
\end{defn}

\begin{defn}
\label{defn-left-module-bar-construction-in-a-monoidal-category}
For $E \in \A$ and a collection 
$\left\{p_i\right\}_{i \geq 2}$ of degree $2-i$
morphisms $ A^{i-1}\otimes E \rightarrow E$,
its \em left module bar-construction $\infbar(E)$ \rm
comprises objects $A^{i}\otimes E$ for all $i \geq 0$ placed
in degree $-i$ and degree $1-k$ maps 
$A^{i+k-1}\otimes E \rightarrow A^{i-1}\otimes E$ 
defined by
\begin{equation}
\label{eqn-differentials-in-left-module-bar-construction}
d_{(i+k)i} := (-1)^{(i-1)(k+1)} 
\left(\sum_{j = 1}^{i-1} \left( (-1)^{jk} \id^{i-j-1} \otimes m_{k+1} \otimes
\id^{j}\right) + \id^{i-1} \otimes p_{k+1} \right). 
\end{equation} 

\begin{tiny}
\begin{equation}
\label{eqn-left-module-bar-construction-of-A-m_i}
\begin{tikzcd}[column sep = 2.4cm]
\dots
\ar{r}[']{\begin{smallmatrix}A^3 p_2  - A^2m_2E + \\ + Am_2AE - m_2 A^2E \end{smallmatrix}}
\ar[bend left=25]{rr}[description]{A^2 p_3 + Am_3E + m_3 AE }
\ar[bend left=30]{rrr}[description]{Ap_4 - m_4E}
\ar[bend left=35]{rrrr}[description]{p_5}
&
A^3E
\ar{r}[']{A^2p_2 - A m_2 E + m_2 AE}
\ar[bend left=25]{rr}[description]{-Ap_3 - m_3E}
\ar[bend left=30]{rrr}[description]{p_4}
& 
A^2E
\ar{r}[']{Ap_2 - m_2E}
\ar[bend left=25]{rr}[description]{p_3}
&
AE
\ar{r}[']{p_2}
&
\underset{\degzero}{E}.
\end{tikzcd}
\end{equation}
\end{tiny}
\end{defn}

Note that the defining formulas
\eqref{eqn-differentials-in-right-module-bar-construction} 
of the right module and the left module bar-constructions can be
obtained from the analogous 
formula \eqref{eqn-differentials-in-non-aug-bar-construction}
for the algebra bar-construction by replacing each $m_i$ with $p_i$ 
if its domain involves $E$. This will also be true for the bimodule 
bar-construction formula. 

\begin{defn}
Let $\A$ be a monoidal DG category and let $(A,m_i)$ be an
$\Ainfty$-algebra in $\A$. A \em right (resp. left) $\Ainfty$-module 
$(E, p_i)$ over $A$ \rm is an object $E \in \A$ and a collection  
$\left\{p_i\right\}_{i \geq 2}$ of degree $2-i$
morphisms $E \otimes A^{i-1} \rightarrow E$ (resp. $A^{i-1} \otimes E
\rightarrow E$) such that $\infbar(E)$ is a twisted complex. 
\end{defn} 

\begin{defn} 
\label{defn-right-module-bar-constructions-of-an-Ainfty-morphism}
Let $\A$ be a monoidal DG category and let $(A,m_i)$ be an
$\Ainfty$-algebra in $\A$.
Let $(E, p_k)$ and $(F, q_k)$ be right $\Ainfty$-modules over $A$ in $\A$. 

A \em degree $j$ morphism \rm $f_\bullet\colon (E,p_k) \rightarrow (F, q_k)$ of 
right $\Ainfty$-$A$-modules is a collection $(f_i)_{i \geq 1}$ of 
degree $j - i + 1$ morphisms $E \otimes A^{i-1} \rightarrow F$. Its
\em bar-construction \rm $\infbar(f_\bullet)$ is 
the morphism $\infbar(E) \rightarrow \infbar(F)$ in $\pretriagmns(\A)$
whose components are  
$$
E \otimes A^{i+k-1} \rightarrow F \otimes A^{i-1}\colon \;
(-1)^{j(i-1)}  f_{k+1}\otimes \id^{i-1}. $$
We illustrate the case when $f_\bullet$ is of odd degree:
\begin{small}
\begin{equation*}
\begin{tikzcd}[column sep = 2.6cm, row sep = 3cm]
\dots
\ar{r}
&
EA^3
\ar{r}
\ar{d}[description, pos = 0.85]{-f_1A^3}
\ar{dr}[description, pos = 0.65]{f_2 A^2}
\ar{drr}[description, pos = 0.30]{-f_3 A} 
\ar{drrr}[description, pos = 0.10]{f_4}
&
EA^2
\ar{r}
\ar{d}[description, pos = 0.85]{f_1 A^2}
\ar{dr}[description, pos = 0.65]{-f_2A}
\ar{drr}[description, pos = 0.30]{f_3}
&
EA
\ar{r}
\ar{d}[description, pos = 0.85]{-f_1A}
\ar{dr}[description, pos = 0.65]{f_2}
&
E
\ar{d}[description, pos = 0.85]{f_1}
\\
\dots
\ar{r}
&
FA^3
\ar{r}
&
FA^2
\ar{r}
&
FA
\ar{r}
&
F.
\end{tikzcd}
\end{equation*}
\end{small}
\end{defn}

The corresponding definition for the left $\Ainfty$-modules differs only
in signs:
\begin{defn} 
\label{defn-left-module-bar-constructions-of-an-Ainfty-morphism}
Let $\A$ be a monoidal DG category and let $(A,m_i)$ be an
$\Ainfty$-algebra in $\A$.
Let $(E, p_k)$ and $(F, q_k)$ be left $\Ainfty$-modules over $A$ in $\A$. 

A degree $j$ morphism $f_\bullet\colon (E,p_k) \rightarrow (F, q_k)$ of 
left $\Ainfty$-$A$-modules is a collection $(f_i)_{i \geq 1}$ of 
degree $j - i + 1$ morphisms $A^{i-1} \otimes E \rightarrow F$. 
Its \em bar-construction \rm $\infbar(f_\bullet)$ is 
the morphism $\infbar(E) \rightarrow \infbar(F)$ in $\pretriagmns(\A)$
whose components are 
$$ A^{i+k-1} \otimes E \rightarrow A^{i-1} \otimes F \colon \;
(-1)^{(j+k)(i-1)}  \id^{i-1} \otimes f_{k+1}. $$
\end{defn}

We define the DG categories of left and right modules over $A$ in the
unique way which makes the left and right module bar-constructions 
into faithful DG functors from these categories to $\pretriagmns(\A)$:

\begin{defn}
Let $\A$ be a monoidal DG category and $A$ be an $\Ainfty$-algebra in
$\A$. Define the \em DG category $\nodA$ of
right $\Ainfty$-$A$-modules in $\A$ \rm by:
\begin{itemize}
\item Its objects are right $\Ainfty$-$A$-modules in $\A$,
\item For any $E,F \in \obj \noddinf A$, the complex 
$\homm^\bullet_{\noddinf A}(E,F)$
consists of $\Ainfty$-morphisms $f_\bullet\colon E \rightarrow F$
with their natural grading. The differential and
the composition is defined by composing the corresponding 
twisted complex morphisms. 
\item The identity morphism of $E \in \noddinf A$ is the morphism 
$(f_\bullet)$ with $f_1 = \id_E$ and $f_{\geq 2} = 0$ whose
corresponding twisted complex morphism is $\id_{\infbar(E)}$. 
\end{itemize}
The \em DG category $\Anod$ of left $\Ainfty$-$A$-modules 
in $\A$ \rm is defined analogously.  
\end{defn}

For brevity, we write $\infhom_{rA}(-,-)$ and $\infhom_{lA}(-,-)$
for the $\homm$-spaces in $\nodA$ and $\Anod$, respectively. 

\subsection{$\Ainfty$-bimodules}
\label{section-Ainfty-bimodules-in-a-monoidal-category}

The definitions for $\Ainfty$-bimodules are similar to those for left
and right $\Ainfty$-modules, but we define the bimodule bar-construction
to be a bigraded collection of objects and morphisms whose
totalisation can be identified with the coderivations and the
morphisms of the corresponding free bi-comodules. The bar-construction
of an $\Ainfty$-bimodule is therefore a twisted bicomplex, see
\cite{AnnoLogvinenko-UnboundedTwistedComplexes}, \S 4.

Note that the indexing convention is necessarily different. 
With algebras and modules, it is traditional for $m_k$ and $p_k$ 
to denote the operations from $k$ factors to $1$ factor, 
e.g.~$p_k\colon E\otimes A^{k-1} \rightarrow A$. 
However, with bindexing $p_{ij}$ can 
only denote an operation $A^i \otimes M \otimes B^j \rightarrow M$,
whose domain involves $i + j + 1$ factors. In particular, $p_{i,j}$
is then a map of degree $1 - i - j$, and not $2 - i - j$, as one 
might have expected. 

\begin{defn}
\label{defn-bimodule-bar-construction-in-a-monoidal-category}
Let $(A,m_i)$ and $(B,n_i)$ be an $\Ainfty$-algebras 
in a monoidal DG category $\A$. 
Let $M \in \A$ and let $\left\{p_{ij}\right\}_{i + j \geq 1}$ be a collection 
of degree $1-i-j$ morphisms $A^i \otimes M \otimes B^{j} \rightarrow M$. 
The \em bimodule bar-construction $\infbar(M)$ \rm 
comprises objects $A^i \otimes M \otimes B^{j}$ with $i + j \geq 0$ 
placed in bidegree $-i,-j$ and degree $1+p+q-k-l$ maps 
$A^{p} \otimes M \otimes B^{q} \rightarrow A^i \otimes M \otimes B^j$ 
best defined as the components of the maps 
$$ A^{p} \otimes M \otimes B^{q} \longrightarrow 
\bigoplus_{i + j = p + q - k} A^i \otimes M \otimes B^j $$
defined by
\begin{equation}
\label{eqn-differentials-in-the-bimodule-bar-construction}
(-1)^{(i+j)(k+1)} 
\sum_{r = 0}^{i+j} (-1)^{rk} \id^{i + j - r} \otimes (pmn)_{k+1}
\otimes \id^{r},
\end{equation} 
where $(pm)_{k+1}$ denotes the unique operation ––– 
either $p_{s,t}$ with $s+t = k$ or $m_{k+1}$ or $n_{k+1}$ --- 
that can be applied to the corresponding length $k+1$ factor
of $A^{p} \otimes M \otimes B^{q}$. 

\begin{tiny}
\begin{equation*}
\begin{tikzcd}[row sep=3cm, column sep = 3.5cm]
A^2MB^2
\ar{r}[description]{A^2Mn_2 - A^2p_{01}B}
\ar{d}[pos = 0.75, description]{Ap_{10}B^2 - m_2MB^2}
\ar[bend left=10]{rr}[pos=0.8, description]{A^2p_{02}}
\ar[bend right=35]{dd}[pos=0.85, description]{p_{20}B^2}
\ar{rd}[pos = 0.7, description]{Ap_{11}B}
\ar{rdd}[pos = 0.86, description]{-p_{21}B}
\ar{rrd}[pos = 0.8, description]{Ap_{12}}
\ar[bend right=10]{rrdd}[pos = 0.8, description]{p_{22}}
&
A^2MB
\ar{r}[pos = 0.7, description]{A^2p_{01}}
\ar{d}[pos = 0.78, description]{-Ap_{10}B + m_2MB}
\ar[bend left=35]{dd}[pos = 0.86, description]{-p_{20}B}
\ar{rd}[pos = 0.7, description]{-Ap_{11}}
\ar{rdd}[pos=0.75, description]{p_{21}}
&
A^2 M
\ar{d}[pos = 0.6, description]{Ap_{10} - m_2M}
\ar[bend left=35]{dd}[pos=0.85, description]{p_{20}}
\\
AMB^2
\ar{r}[pos = 0.65, description]{AMn_2 - Ap_{01}B}
\ar{d}[description]{p_{10}B^2}
\ar[bend left=10]{rr}[pos = 0.83, description]{-Ap_{02}}
\ar{rd}[pos = 0.83, description]{-p_{11}B}
\ar{rrd}[pos = 0.75, description]{p_{12}}
&
AMB
\ar{r}[pos = 0.75, description]{Ap_{01}}
\ar{d}[pos = 0.62, description]{-p_{10	}B}
\ar{rd}[description]{p_{11}}
&
AM
\ar{d}[description]{p_{10}}
\\
MB^2
\ar{r}[pos = 0.75, description]{Mn_2 - p_{01}B}
\ar[bend right = 10]{rr}[pos = 0.9, description]{p_{02}}
&
MB
\ar{r}[pos = 0.55, description]{p_{01}}
&
M
\end{tikzcd}
\end{equation*}
\end{tiny}
\end{defn}

Again note that the defining formula
\eqref{eqn-differentials-in-the-bimodule-bar-construction}
for the differentials in the bimodule bar-construction is
almost identical to the analogous 
formula \eqref{eqn-differentials-in-non-aug-bar-construction}
for the algebra bar-construction, only with 
the bimodule operations $p_{s,t}$ with $s+t = k$ being used 
along with the algebra operations $m_{k+1}$ and $n_{k+1}$ as approriate, 
and the formula now defining simultaneously the differentials 
from a given $A^{p} \otimes M \otimes B^q$ to 
all $A^{i} \otimes M \otimes B^j$ with $i + j = p + q - k$.   

\begin{defn}
An \em $\Ainfty$-bimodule \rm over $\Ainfty$-algebras $(A,m_i)$ and $(B,n_i)$ 
in a monoidal DG category $\A$ is
an object $M \in \A$ and a collection $\left\{p_{ij}\right\}_{i + j \geq 1}$ 
of degree $1-i-j$ morphisms $A^i \otimes M \otimes B^{j} \rightarrow M$
such that $\infbar(M)$ is a twisted complex. 
\end{defn}

The remaining definitions are then analogous to those for left and
right modules:

\begin{defn} 
Let $\A$ be a monoidal DG category and let $(A,m_i)$ and $(B, n_i)$ be 
$\Ainfty$-algebras in $\A$. Let $(M, p_{ij})$ and $(N, q_{ij})$ be
$\Ainfty$-$A$-$B$-bimodules. 

A \em degree $k$ morphism \rm $f_{\bullet\bullet}\colon 
(M,p_{ij}) \rightarrow (N, q_{ij})$ of 
$\Ainfty$-$A$-$B$-bimodules is a collection $(f_{lm})_{l + m \geq 0}$ of 
degree $k - l - m$ morphisms $A^l \otimes M \otimes B^{m} \rightarrow N$. 
Its \em bar-construction \rm $\infbar(f_{\bullet\bullet})$ is 
the morphism $\infbar(M) \rightarrow \infbar(N)$ in $\twbiosmns(\A)$
whose components are  
$$
A^{i+l} \otimes M \otimes B^{j+m} \rightarrow A^i \otimes N \otimes B^{j}\colon \;
(-1)^{i(l+m) + k (i+j)}  \id^{i} \otimes f_{l,m} \otimes \id^{j}. $$
We illustrate the case when $f_{\bullet\bullet}$ is of odd degree:
\begin{equation}
\begin{tikzcd}[row sep = 1.25cm]
AMB
\ar[dotted]{r}
\ar[dotted]{d}
\ar[dotted]{dr}
\ar[bend right = 15]{ddrr}[pos = 0.6, description]{Af_{00}B}
\ar[bend right = 20]{dddrrr}[pos = 0.52, description]{f_{11}}
\ar[bend right = 17]{dddrr}[pos = 0.525, description]{- f_{10}B}
\ar[bend left = 10]{ddrrr}[pos = 0.6, description]{Af_{01}}
&
AM
\ar[dotted]{d}
\ar[bend left = 25]{ddrr}[pos=0.5, description]{-Af_{00}}
\ar[bend left = 27]{dddrr}[pos=0.45, description]{f_{10}}
&
&
\\
MB
\ar[dotted]{r}
\ar[bend right = 25]{ddrr}[pos=0.35, description]{-f_{00}B~}
\ar[bend right = 25]{ddrrr}[pos=0.35, description]{f_{01}}
&
M
\ar[bend left = 17]{ddrr}[pos = 0.3, description]{f_{00}}
&
&
\\
&
&
ANB
\ar[dotted]{r}
\ar[dotted]{d}
\ar[dotted]{dr}
&
AN
\ar[dotted]{d}
\\
&
&
NB
\ar[dotted]{r}
&
N.
\end{tikzcd}
\end{equation}

\end{defn}

Again, we define the DG category of $\Ainfty$- $A$-$B$-bimodules in the
unique way which makes the bimodule bar-construction
into a faithful DG functor from this category to $\twbiosmns(\A)$:

\begin{defn}
Let $(A,m_i)$ and $(B,n_i)$ be $\Ainfty$-algebras in a monoidal DG category 
$\A$. Define the \em DG category $\AnodB$ of
$\Ainfty$-$A$-$B$-bimodules in $\A$ \rm by:
\begin{itemize}
\item Its objects are $\Ainfty$-$A$-$B$-bimodules in $\A$,
\item For any $M,N \in \obj A\text{-}\noddinf\text{-}B$, the complex 
$\homm^\bullet_{A\text{-}\noddinf\text{-}B}(M,N)$
consists of $\Ainfty$-morphisms $f_{\bullet\bullet}\colon M \rightarrow N$
with their natural grading. The differential and
the composition is defined by composing the corresponding 
twisted complex morphisms. 
\item The identity morphism of $M \in A\text{-}\noddinf\text{-}B$ is 
the morphism $(f_{\bullet\bullet})$ with 
$f_{00} = \id_M$ and $f_{ij} = 0$ for $i + j \geq 1$ whose
corresponding twisted complex morphism is $\id_{\infbar(M)}$. 
\end{itemize}
\end{defn}

Next, we show that the category of $\Ainfty$-bimodules is isomorphic
to the category of left $\Ainfty$-modules in the category of right
$\Ainfty$-modules and to the category of right $\Ainfty$-modules in 
the category of left $\Ainfty$-modules.  

First, we need to explain how such setup fits into our framework 
of definitions. If $(A,m_i)$ is an $\Ainfty$-algebra in a monoidal 
DG category $\A$, then DG category $\nodA$ doesn't have a natural 
monoidal structure. 
\begin{defn}
Define the DG bicategory 
\begin{equation}
\nodAbic
:= 
\quad 
\begin{tikzcd}[column sep = 5em]
\bullet
\ar{r}{\nodA}
&
\bullet 
\ar[out=30, in=-30,loop,distance=6em]{}{\A}
\end{tikzcd}
\end{equation}
whose $1$-composition is given by the monoidal structure on $\A$ and
by the functor
$$ \A \otimes \nodA \rightarrow \nodA $$
which sends $(Q, (E,p_i))$ to $(Q \otimes E, \id \otimes p_i)$. 
Define similarly the DG bicategory
\begin{equation}
\Anodbic
:= 
\quad 
\begin{tikzcd}[column sep = 5em]
\bullet 
\ar[out=150, in=-150,loop,distance=6em]{}[']{\A}
\ar{r}{\Anod}
&
\bullet
\end{tikzcd}
\end{equation} 
\end{defn}

Any other $\Ainfty$-algebra in $\A$ is also an $\Ainfty$-algebra in $\nodAbic$
and $\Anodbic$. We can therefore define:
\begin{defn}
Let $(A,m_i)$ and $(B,n_i)$ be $\Ainfty$-algebras in 
a monoidal DG category $\A$. 
Define $(\Anod)\text{-}B$ to be the DG category of right 
$\Ainfty$-$B$-modules in $\Anodbic$ whose underlying $1$-morphisms 
lie in $\Anod$. Similarly, define $A\text{-}\left(\nodB\right)$ 
to be the DG category of left $\Ainfty$-$A$-modules in $\nodBbic$ 
whose underlying $1$-morphisms lie in $\nodB$.   
\end{defn}

In  \cite[Defn. 4.2]{AnnoLogvinenko-UnboundedTwistedComplexes} we define the functors 
\begin{align*}
\cxrow\colon 
\twcx_\B^{\pm}\left(\twcx_\B^{\pm}\left(\A\right)\right) 
& \rightarrow 
\twbicx_\B^{\pm}(\A), \\\
\cxcol\colon 
\twcx_\B^{\pm}\left(\twcx_\B^{\pm}\left(\A\right)\right) 
& \rightarrow 
\twbicx_\B^{\pm}(\A), 
\end{align*}
that make a bicomplex out of a twisted complex of twisted complexes, interpreting
the complexes (with a necessary sign twist) as rows or columns of the bicomplex respectively.
Then we have the following:

\begin{theorem}
\label{theorem-ainfty-bimodules-are-ainfty-modules-in-cat-of-ainfty-modules}
Let $(A,m_i), (B,n_i)$ be $\Ainfty$-algebras in 
a monoidal DG category $\A$. 
There exist isomorphisms of DG categories 
\begin{equation}
(\Anod)\text{-}B 
\simeq 
\AnodB 
\simeq
A\text{-}\left(\nodB\right) 
\end{equation}
intertwining the functors $\cxcol \circ \infbarA \circ \infbarB$, 
$\infbar^{A\text{-}B}$, and $\cxrow \circ \infbarB \circ \infbarA$ into 
$\twbiosmns \A$. 
\end{theorem}
\begin{proof}
An element of $(\Anod)\text{-}B$ is a left $\Ainfty$-$A$-module $(M,p_i)$ 
and a collection $\left\{q_{j\bullet}\right\}_{j \geq 2}$
of degree $2-j$ left $\Ainfty$-module morphisms 
$M \otimes B^{j-1} \rightarrow M$ such that the right 
$B$-module bar-construction $\infbar^B M$ is a twisted complex of 
left $\Ainfty$-$A$-modules:
\begin{tiny}
\begin{equation*}
\begin{tikzcd}[column sep = 2.4cm]
\dots
\ar{r}[']{\begin{smallmatrix}MB^2 n_2  - MBn_2B + \\ + Mn_2B^2 -
q_{2\bullet} B^3 \end{smallmatrix}}
\ar[bend left=25]{rr}[description]{MB n_3 + Mn_3B + q_{3\bullet} B^2 }
\ar[bend left=30]{rrr}[description]{Mn_4 - q_{4\bullet}B}
\ar[bend left=35]{rrrr}[description]{q_{5\bullet}}
&
MB^3
\ar{r}[']{MBn_2 - M n_2 B + q_{2\bullet} B^2}
\ar[bend left=25]{rr}[description]{-Mn_3 - q_{3\bullet}B}
\ar[bend left=30]{rrr}[description]{q_{4\bullet}}
& 
MB^2
\ar{r}[']{Mn_2 - q_{2\bullet}B}
\ar[bend left=25]{rr}[description]{q_{3\bullet}}
&
MB
\ar{r}[']{q_{2\bullet}}
&
\underset{\degzero}{M}.
\end{tikzcd}
\end{equation*}
\end{tiny}
Applying the functor $\infbar^A$ of left $A$-module bar-construction
to this twisted complex we obtain an element of $\pretriagmns \pretriagmns \A$. 
Applying further the equivalence 
$$ \cxcol\colon \pretriagmns \pretriagmns \A \rightarrow \twbiosmns \A $$
of \cite[Prop. 4.3]{AnnoLogvinenko-UnboundedTwistedComplexes},
we obtain a one-sided
bounded above twisted bicomplex over $\A$:
\begin{tiny}
\begin{equation*}
\begin{tikzcd}[row sep=3cm, column sep = 3.5cm]
A^2MB^2
\ar{r}[description]{A^2Mn_2 - A^2q_{21}B}
\ar{d}[pos = 0.75, description]{Ap_2B^2 - m_2MB^2}
\ar[bend left=10]{rr}[pos=0.8, description]{A^2q_{31}}
\ar[bend right=35]{dd}[pos=0.85, description]{p_3B^2}
\ar{rd}[pos = 0.7, description]{Aq_{22}B}
\ar{rdd}[pos = 0.86, description]{-q_{23}B}
\ar{rrd}[pos = 0.8, description]{Aq_{32}}
\ar[bend right=10]{rrdd}[pos = 0.8, description]{q_{33}}
&
A^2MB
\ar{r}[pos = 0.7, description]{A^2q_{21}}
\ar{d}[pos = 0.78, description]{- Ap_2B + m_2MB}
\ar[bend left=35]{dd}[pos = 0.86, description]{-p_3B}
\ar{rd}[pos = 0.7, description]{-Aq_{22}}
\ar{rdd}[pos=0.75, description]{q_{23}}
&
A^2 M
\ar{d}[pos = 0.6, description]{Ap_2 - m_2M}
\ar[bend left=35]{dd}[pos=0.85, description]{p_3}
\\
AMB^2
\ar{r}[pos = 0.65, description]{AMn_2 - Aq_{21}B}
\ar{d}[description]{p_2B^2}
\ar[bend left=10]{rr}[pos = 0.83, description]{-Aq_{31}}
\ar{rd}[pos = 0.83, description]{-q_{22}B}
\ar{rrd}[pos = 0.75, description]{q_{32}}
&
AMB
\ar{r}[pos = 0.75, description]{Aq_{21}}
\ar{d}[pos = 0.62, description]{-p_2B}
\ar{rd}[description]{q_{22}}
&
AM
\ar{d}[description]{p_2}
\\
MB^2
\ar{r}[pos = 0.75, description]{Mn_2 - q_{21}B}
\ar[bend right = 10]{rr}[pos = 0.9, description]{q_{31}}
&
MB
\ar{r}[pos = 0.55, description]{q_{21}}
&
M
\end{tikzcd}
\end{equation*}
\end{tiny}
It is readily seen to be the bimodule bar-construction of
its differentials into $M$. 

Conversely, given an arbitrary $\Ainfty$-$A$-$B$-bimodule $(M,p_{ij})$ 
the $j$-th column of the bimodule bar-construction $\infbar^{A\text{-}B} M$
becomes, after a sign-twist by $(-1)^j$, the left $A$-module bar-construction
of its differentials $p_{\bullet0} B^j$ into $M \otimes B^j$. Viewing 
the whole of $\infbar^{A\text{-}B} M$ a twisted complex of columns, 
we see that all differentials in it are also left $A$-module bar
constructions. Finally, this twisted complex of columns is itself a
right $B$-module bar-construction of its differentials into the $0$-th
column. 

We conclude that the fully faithful functors 
$$ \cxcol \circ \infbar^A \circ \infbar^B\colon 
(\Anod)\text{-}B \rightarrow \twbiosmns \A, $$
$$ \infbar^{A \text{-} B}\colon 
\AnodB \rightarrow \twbiosmns \A, $$
have the same image in $\twbiosmns \A$ and hence we have
an isomorphism 
$$ (\Anod)\text{-}B \rightarrow \AnodB. $$

The other isomorphism is obtained considering
fully faithful functors $\infbar^{A \text{-} B}$ and
$$\cxrow \circ \infbar^B \circ \infbar^A\colon 
A\text{-}(\nodB) \rightarrow \twbiosmns \A.$$
Their images in $\twbiosmns \A$ do not coincide, however they 
are identified by the sign-twisting automorphism of $\twbiosmns \A$
which sends a twisted bicomplex $(a_{ij}, \alpha_{ijkl})$ to 
$(a_{ij}, (-1)^{ij+kl} \alpha_{ijkl})$
and a morphism $(f_{ijkl})$ of twisted bicomplexes to
$((-1)^{ij+kl} f_{ijkl})$. 
\end{proof}


\subsection{Free modules and bimodules}
\label{section-free-modules-and-bimodules-over-Ainfty-algebra}

\begin{defn}
\label{defn-free-Ainfty-modules}
Let $(A,m_i)$ be an $\Ainfty$-algebra in a monoidal DG category $\A$.
Let $E \in \A$. The \em free right $\Ainfty$-$A$-module generated by 
$E$ \rm is the module $(E \otimes A, p_i)$ where 
$ p_i\colon E \otimes A^i \rightarrow E \otimes A $
is the map $\id_E \otimes m_i$. 
\end{defn}

To see that this does indeed define an $\Ainfty$-module, observe that
$$\infbar(E \otimes A) = E \otimes \infbarnaug(A).$$
Hence if $\infbarnaug(A)$ is a twisted complex, then 
so is $\infbar(E \otimes A)$. 

We can view any $A^i$ with $i \geq 1$ as the free right 
$\Ainfty$-$A$-module generated by $A^{i-1}$.  

\begin{defn}
Let $(A,m_i)$ be an $\Ainfty$-algebra in a monoidal DG category $\A$.
The \em category $\freeA$ of free right $\Ainfty$-$A$-modules \rm
is the full subcategory of $\nodA$ consisting of free modules. We
thus have a fully faithful inclusion 
\begin{equation}
\label{eqn-inclusion-of-free-A-into-noddinf-A} 
\freeA \hookrightarrow \nodA.
\end{equation}

The \em free module \rm functor 
$$ \free\colon \A \rightarrow \freeA, $$
sends each $E \in \A$ to $E \otimes A \in \freeA$ and each 
$\alpha \in \homm_\A(E,F)$ to the morphism in $\homm_{\freeA}(E
\otimes A, F \otimes A)$ whose first component is $\alpha \otimes \id$
and whose higher components are zero. 
We often implicitly use 
\eqref{eqn-inclusion-of-free-A-into-noddinf-A} to
consider $\free$ as a functor $\A \rightarrow \nodA$. 

The \em forgetful \rm functor 
$$ \forget\colon \nodA \rightarrow \A $$
sends any $(E,p_\bullet) \in \nodA$ to $E \in \A$ and 
any $f_\bullet \in
\homm_{\freeA}((E,p_\bullet),(F,q_\bullet))$ to 
$f_1 \in \homm_\A(E,F)$. 
\end{defn}

The notions of free left $\Ainfty$-$A$-modules and of free
$\Ainfty$-$A$-$B$-bimodules, the DG
categories $\Afree$ and $\AfreeB$ they form, and the corresponding
free and forgetful functors are defined analogously. 

\subsection{Yoneda embedding}
\label{section-Yoneda-embedding-of-A-modules-into-repA-modules}

The notions and results of sections \S\ref{section-Yoneda-embedding-of-A-modules-into-repA-modules}-\ref{section-bar-resolution} apply both to right 
and to left $\Ainfty$-$A$-modules. We state them only for the right modules, the case of the left modules is similar.  

The standard Yoneda embedding of $\nodA$ 
is the fully faithful functor 
\begin{equation}
\label{eqn-the-standard-yoneda-for-nodA}
\nodA \rightarrow \modd\text{-}\left(\nodA\right)
\end{equation}
which sends $(E,p_\bullet)$ to 
$\infhom_{rA}(-, (E,p_\bullet))$. We now define a more economical 
version of this, where the Yoneda embedding is only used 
to embed $E$ into $\modA$. 

In \cite[\S4.5]{GyengeKoppensteinerLogvinenko-TheHeisenbergCategoryOfACategory}
it was shown that if $\A$ is a monoidal DG category, then there is a
natural monoidal structure on $\modA$ such that the Yoneda embedding
\begin{equation}
\label{eqn-monoidal-Yoneda-embedding-A-into-modA}
\Upsilon\colon \A \hookrightarrow \modA
\end{equation}
is strong monoidal. We thus view $\A$ as a full 
monoidal subcategory of $\modA$. 

An $\Ainfty$-algebra $A$ in $\A$ can therefore be also 
viewed as an $\Ainfty$-algebra in $\modA$. Where the difference matters, 
we write $A$ and $\repA$ for the corresponding algebras in $\A$ and
$\modA$, respectively:

\begin{defn}
\label{defn-yoneda-embedding-into-repA-modules}
Let $(A,m_\bullet)$ be an $\Ainfty$-algebra in a monoidal DG category $\A$.
Define the \em Yoneda embedding \rm 
\begin{equation}
\label{eqn-yoneda-embedding-into-T^*-modules}
\Upsilon\colon \nodA \rightarrow \nodrepA
\end{equation}
to be the fully faithful functor induced by the monoidal Yoneda embedding
\eqref{eqn-monoidal-Yoneda-embedding-A-into-modA}. It sends any
object $(E,p_\bullet)$ to $(\Upsilon(E), \Upsilon(p_\bullet))$ and any 
morphism $f_\bullet$ to $\Upsilon(f_\bullet)$. 
\end{defn}

Unless specifically mentioned otherwise, 
in the context of $\Ainfty$-$A$-modules we mean by Yoneda embedding 
the functor defined in Definition
\ref{defn-yoneda-embedding-into-repA-modules} as opposed to the
standard Yoneda embedding
\eqref{eqn-the-standard-yoneda-for-nodA}. 

The following is both a direct analogue of the classical result
for modules over ordinary monads in \cite[\S5]{Street-TheFormalTheoryOfMonads} 
and its homotopy generalisation:

\begin{prps}
\label{prps-yoneda-embedding-gives-both-fiber-and-homotopy-fiber-square}
The following is both a fiber square of DG-categories and a homotopy 
fiber square thereof: 
\begin{equation}
\begin{tikzcd}[column sep = 2cm]
\label{eqn-yoneda-embedding-of-Ainfty-A-modules-as-homotopy-fiber-square}
\nodA
\ar[hookrightarrow]{r}{\yoneda} 
\ar{d}{\forget}
&
\nodrepA
\ar{d}{\forget}
\\
\A
\ar[hookrightarrow]{r}{\yoneda} 
&
\modA.
\end{tikzcd}
\end{equation}
\end{prps}
\begin{proof}
For a fiber square, as the Yoneda embedding of $\A$ into
$\modA$ is fully faithful, it suffices to show that $\nodA$ is the
full subcategory of $\nodrepA$ consisting of $(E,p_\bullet)$ with $E
\in \A$. Since $\A$ is a full monoidal subcategory of $\modA$, 
for any such $(E,p_\bullet)$ its structure morphisms 
$p_i\colon E\otimes A^{i-1} \rightarrow E$ all lie in $\A$, and so 
do all components of all $\Ainfty$-morphisms of such modules.  

To show that 
\eqref{eqn-yoneda-embedding-of-Ainfty-A-modules-as-homotopy-fiber-square}
is a homotopy fiber square, it suffices to show that 
\begin{equation}
\label{eqn-the-forgetful-functor-from-nodrepA-to-modA}
\forget\colon \nodrepA \rightarrow \modA, 
\end{equation}
is a fibration. This is because 
the category of small DG-categories with the Dwyer-Kan
model structure due to Tabuada
\cite{Tabuada-UneStructureDeCategorieDeModelesDeQuillenSurLaCategorieDesDG-Categories}
is right proper, and in a right proper model category any pushout
along a fibration is also the homotopy pushout, see 
\cite[IX, \S4.1]{BousfieldKan-HomotopyLimitsCompletionsAndLocalizations}
\cite[Prop.~1.19]{Barwick-OnLeftAndRightModelCategoriesAndLeftAndRightBousfieldLocalizations}. 

Now, a DG-functor is a fibration if
it is surjective on complexes of morphisms and reflects homotopy
equivalences, see 
\cite[Defn.~2.1]{Toen-TheHomotopyTheoryOfDGCategoriesAndDerivedMoritaTheory}. 
The functor \eqref{eqn-the-forgetful-functor-from-nodrepA-to-modA}
satisfies the former as it has a right inverse
which sends any $E \in \modA$ to $(E,p_\bullet)$ with 
$p_i = 0$ for all $i$
and sends any $f\colon E \rightarrow F$ to $(f_\bullet)$ with 
$f_1 = f$ and $f_i = 0$ for $i > 1$. 
It satisfies the latter as given $(E,p_\bullet) \in \nodrepA$ 
and a homotopy equivalence 
$\phi\colon E \xrightarrow{\sim} F$ in $\modA$, 
we can transfer the $\Ainfty$-structure along $\phi$ 
(\cite{AnnoLogvinenko-UnboundedTwistedComplexes}, Theorem A.2)
to construct $(F,q_\bullet)$ homotopy equivalent to $(E, p_\bullet)$. 
\end{proof}

An useful consequence of the universal property of a homotopy fiber square is: 
\begin{cor}
Let $(E,p_\bullet) \in \nodrepA$. It is homotopy equivalent to
some $(F,q_\bullet) \in \nodA$ if and only if $E \in \modA$ is homotopy
equivalent to $F \in \A$. 
\end{cor}

\subsection{Twisted complexes of $\Ainfty$-modules}
\label{section-twisted-complexes-of-ainfty-modules}

We now fix our conventions regarding twisted 
complexes of $\Ainfty$-modules. As explained in
\S\ref{section-unbounded-twisted-complexes} unbounded 
twisted complexes over a DG category $\A$ need to be defined relative 
to its fully faithful embedding into some DG category closed 
under shifts and infinite direct sums. 

Our conventions are: we define $\twcxub \A$ and $\twcxub \modA$ 
relative to $\modA$. Thus twisted complexes in $\twcxub \A$ and
$\twcxub \modA$ can have infinite number of differentials and/or morphism 
components emerge from a single object, but only if their sum still defines 
a morphism in $\modA$. Since $\modA$ admits change of differential, 
the convolution functor $\twcxub \modA \hookrightarrow \modA$ 
is an equivalence. 

Next, we define $\twcxub \nodA$. 
The naive definition gives us a convolution into 
$\modd\text{-}\left(\nodA\right)$, but it isn't the category we want 
to work with. We have Yoneda embedding $\nodA \hookrightarrow \nodrepA$, 
and $\nodrepA$ admits convolutions of unbounded twisted complexes
\cite[Cor.~5.14]{AnnoLogvinenko-UnboundedTwistedComplexes}. We thus define 
both $\twcxub \nodA$ and $\twcxub \nodrepA$ relative to $\nodrepA$, 
while natural embedding $\nodrepA \hookrightarrow \twcxub \nodrepA$
is an equivalence. On the other hand, $\nodA \hookrightarrow \twcxub
\nodA$ is an equivalence if and only if $\A \hookrightarrow \twcxub \A$
is, and the same is true for the subcategories $\twcxpls$ or $\pretriagmns$, 
\cite[Cor.~5.13]{AnnoLogvinenko-UnboundedTwistedComplexes}. 

\subsection{The homotopy lemma}
\label{section-the-homotopy-lemma}
In this section we give an analogue of the classical theorem that all
$\Ainfty$-quasi-isomorphisms of usual $\Ainfty$-modules 
are homotopy equivalences, 
cf.~\cite[Prop.~2.4.1.1]{Lefevre-SurLesAInftyCategories}. 

In our language, the usual $\Ainfty$-modules correspond to the case 
$\A = \modk$. Recall that such $\Ainfty$-module $E$ is said to be
\em acyclic \rm if its underlying complex of vector spaces is acyclic. 

In our full generality, for arbitrary $\A$, it is not immediately 
obvious how one should define acyclicity since there are no underlying 
complexes of vector spaces. However, a complex of vector spaces is
acyclic if and only if it is null-homotopic. Thus, when $\A = \modk$
an $\Ainfty$-$A$-module is acyclic if and only if 
its underlying object of $\A$ is null-homotopic. We thus define:

\begin{defn}
\label{defn-acyclicity-of-modules-and-twisted-complexes}
An $\Ainfty$-$A$-module $(E,p_\bullet)$ is \em acyclic \rm if 
$E$ is null-homotopic~in~$\A$. A twisted complex 
in $\twcxub\nodA$ is \em acyclic \rm if its underlying
twisted complex in $\twcxub\A$ is null-homotopic.  
\end{defn}

We then have: 

\begin{lemma}[The homotopy lemma]~
\label{lemma-the-homotopy-lemma-for-nodA}
\begin{enumerate}
\item 
\label{item-acyclic-if-and-only-if-acyclic-in-A}
An $\Ainfty$-$A$-module $(E,p_\bullet)$ is null-homotopic  
if and only if it is acyclic. 
\item 
\label{item-complex-acyclic-if-and-only-if-acyclic-in-pretriagA}
A twisted complex in $\twcxub\nodA$ is
null-homotopic if and only if it is acyclic. 
\item 
\label{item-homotopy-equivalence-if-and-only-if-one-in-A}
An closed degree zero $\Ainfty$-morphism $f_\bullet\colon (E,p_\bullet) \rightarrow (F,q_\bullet)$
is a homotopy equivalence in $\nodA$ if and only $f_1\colon E \rightarrow F$
is a homotopy equivalence in $\A$. 
\end{enumerate}
\end{lemma}
\bf NB: \rm 
When $\A = \modk$, this shows that an $\Ainfty$-morphism 
$f_\bullet\colon E \rightarrow F$ is a homotopy equivalence 
if and only if $f_1$ is a homotopy equivalence in $\modk$. 
In $\modk$ a morphism is a homotopy equivalence if and only if 
it is a quasi-isomorphism. We thus recover the statement
that a morphism of the usual $\Ainfty$-modules is a homotopy equivalence 
if and only if it is a quasi-isomorphism, 
cf.~\cite[Prop.~2.4.1.1]{Lefevre-SurLesAInftyCategories}. 

\begin{proof}
\eqref{item-acyclic-if-and-only-if-acyclic-in-A}:
The ``only if'' implication is clear. For the ``if'' implication, let
$h$ be the contracting homotopy of $E$ in $\A$:
a degree $-1$ endomorphism of $E$ with $dh = \id_E$. It suffices 
to construct the contracting homotopy $h_\bullet$ of $(E,p_\bullet)$ in 
$\nodA$. In other words, we need to find $h_\bullet$ such that 
$$ d(h_1, h_2, \dots , h_n, \dots) = (\id_E, 0, 0, \dots, 0, \dots). $$

By our definition of $\nodA$, 
the differential of any $x_\bullet: (E,p_\bullet) \rightarrow (E,p_\bullet)$ is
computed by differentiating the map $\infbar x_\bullet$ of twisted complexes over $\A$:
\begin{scriptsize}
\begin{equation}
\begin{tikzcd}[row sep=2.5cm, column sep = 2.5cm]
EA^3
\ar[dashed, bend left=25]{rrr}[description]{p_4}
\ar[dashed, bend left=20]{rr}[description]{-Em_3 - p_3A}
\ar{r}{EAm_2 - Em_2A + p_2A^2}
\ar{d}[description, pos=0.6]{-x_1A^3}
\ar{dr}[description, near start]{x_2A^2}
\ar{drr}[description, near start]{-x_3A}
\ar{drrr}[description, near start]{x_4}
&
EA^2
\ar[dashed, bend left=20]{rr}[description]{p_3}
\ar{r}{Em_2 - p_2A}
\ar{d}[description, pos=0.6]{x_1A^2}
\ar{dr}[description, near start]{ -x_2A}
\ar{drr}[description, near start]{x_3}
&
EA
\ar{r}{p_2}
\ar{d}[description, pos=0.6]{-x_1A}
\ar{dr}[description, near start]{x_2}
&
E 
\ar{d}[description, pos=0.6]{x_1}
\\
EA^3
\ar[dashed, bend right=25]{rrr}[description]{p_4}
\ar[dashed, bend right=20]{rr}[description]{-Em_3 - p_3A}
\ar{r}{EAm_2 - Em_2A + p_2A^2}
& 
EA^2
\ar[dashed, bend right=20]{rr}[description]{p_3}
\ar{r}{Em_2 - p_2A}
&
EA
\ar{r}{p_2}
&
\underset{\degzero}{E}.
\end{tikzcd}
\end{equation}
\end{scriptsize}
Note, that the first $n$ terms of $dx_\bullet$ 
are completely determined by $x_1, \dots, x_n$. Define 
$$ d(x_1, \dots, x_n) := \bigl( (dx_\bullet)_1, \dots, (dx_\bullet)_n \bigr). $$ 
In this notation, the diagram above makes it clear that for any 
$x_{n+1}\colon EA^n \rightarrow E$ 
\begin{align*}
d(0,\dots, 0, x_{n+1}) & = (0,\dots, 0, dx_{n+1}). 
\end{align*}

We now proceed by induction. Suppose we've found $h_1, \dots, h_n$ such that 
\begin{equation*}
d(h_1, h_2, \dots, h_n) = (\id_E, 0, \dots, 0)  
\end{equation*}
We start with $n = 1$ where we set $h_1 = h$. 
For the induction step, observe that  
$$ d(h_1, h_2, \dots, h_n,0) = (\id_E, 0, \dots,0, x_{n+1}), $$
for some $x_{n+1}\colon EA^n \rightarrow E$. The map $(\id_E, 0, \dots, x_{n+1})$ 
is a boundary and the map $(\id_E, 0, \dots, 0)$ is closed. 
So $(0,\dots, 0, x_{n+1})$ must also be closed. We conclude that 
that $dx_{n+1} = 0$, and hence $d(h \circ
x_{n+1})= x_{n+1}$. Set 
$$ h_{n+1} = - h \circ x_{n+1}. $$
We then have 
\begin{align*}
d(h_1, \dots, h_n, h_{n+1}) & = 
d(h_1, \dots, h_n, 0) + 
d(0, \dots, 0 , h_{n+1}) = \\
& = (\id_E, 0, \dots, 0, x_{n+1}) + 
(0,0,\dots, 0, -x_{n+1}) \\ 
& = (\id_E, 0, \dots, 0, 0). 
\end{align*}

\eqref{item-complex-acyclic-if-and-only-if-acyclic-in-pretriagA}:

By \cite{AnnoLogvinenko-UnboundedTwistedComplexes}, Prop.~5.12
we have a fully faithful functor
$$ \Phi\colon \twcxub\left(\nodA\right) \hookrightarrow \nodtwcxA. $$
By construction, $\Phi$  commutes with the forgetful functors
into $\twcxub \A$. The desired assertion follows 
by applying \eqref{item-acyclic-if-and-only-if-acyclic-in-A} 
to $\nodtwcxA$ or rather to $\nodtwcxrepA$. 

\eqref{item-homotopy-equivalence-if-and-only-if-one-in-A}:
The morphism $f_\bullet\colon (a,p_\bullet) \rightarrow
(b,q_\bullet)$ is a homotopy equivalence if and only if \begin{equation}
\label{eqn-two-step-twisted-complex-in-nodA}
(a,p_\bullet) \xrightarrow{f_\bullet} \underset{\degzero}{(b,q_\bullet)},
\end{equation}
is null-homotopic in $\pretriag \nodA$. 
The desired assertion follows by applying 
\eqref{item-complex-acyclic-if-and-only-if-acyclic-in-pretriagA}
to the twisted complex \eqref{eqn-two-step-twisted-complex-in-nodA}. 
\end{proof}

\subsection{Bar-construction as a complex of $\Ainfty$-$A$-modules}
\label{section-bar-construction-as-a-complex-of-ainfty-A-modules}

We now upgrade the bar-construction of an $\Ainfty$-$A$-module 
from a twisted complex of objects of $\A$ to a twisted complex of 
$\Ainfty$-$A$-modules. 

We follow the method detailed in 
\cite[\S2.10]{AnnoLogvinenko-BarCategoryOfModulesAndHomotopyAdjunctionForTensorFunctors}:
\begin{defn}
\label{defn-morphisms-pi_i-and-mu_i}
Let $(A,m_\bullet)$ be an $\Ainfty$-algebra in a monoidal DG category $\A$.
Let $(E,p_\bullet)$ be a right $\Ainfty$-$A$-module. 

For any $i \geq 1$ consider $E \otimes A^i$ as the free $A$-module  
$E \otimes A^{i-1} \otimes A$.
Define a morphism of right $\Ainfty$-$A$-modules
$$ \pi_i\colon E \otimes A^{i-1} \rightarrow (E, p_\bullet) $$ 
by setting each $(\pi_i)_j$ to be the map
$$ E \otimes A^{i-1} \otimes A^{j-1} \xrightarrow{(-1)^{(i-1)(j-1)}p_{i + j - 1}} E. $$

For left $\Ainfty$-$A$-modules, the definition is similar,  but with $(\pi_i)_j = p_{i+j-1}$. 
\end{defn}

When $E$ is a free module $F \otimes A$, 
we write instead of $\pi_{i}$
$$ \id \otimes \mu_i \colon F \otimes A^i \rightarrow F \otimes A $$ 
for the corresponding $\Ainfty$-morphism because its
$j$-th component is $(-1)^{(i-1)(j-1)}\id \otimes m_{i+j-1}$. 
When $F = \id_\A$, we further write $\mu_i\colon  A^i \rightarrow A$
for this $\Ainfty$-morphism. 

On the other hand, given a morphism $\alpha \colon E \rightarrow F$
in $\A$, by abuse of notation we write $\alpha \otimes \id$ for
the right $\Ainfty$-$A$-module morphism 
$$ \free(\alpha)\colon E \otimes A \rightarrow F \otimes A. $$
It is the strict $\Ainfty$-morphism whose first component is $\alpha
\otimes \id$. 

\begin{prps}
\label{prps-bar-construction-as-a-complex-of-Ainfty-modules}
Let $(A,m_\bullet)$ be an $\Ainfty$-algebra in a monoidal DG category $\A$. Let $(E,p_\bullet)$ be a right $\Ainfty$-$A$-module. 

Take the right module bar-construction of $(E,p_\bullet)$ and lift it from $\A$ to $\nodA$ as follows:
\begin{itemize}
	\item Replace its objects by $(E,p_\bullet)$ and free $\Ainfty$-modules $E \otimes A^i$ with $ i > 0$.  
	\item For differentials, in the summands where the operation $m_{k}$ or $p_{k}$ doesn't involve the rightmost copy of $A$ replace this operation by its image under the free functor. 
	\item In those summands where it does --- replace it by $\mu_{k}$ or $\pi_{k}$, respectively.  
\end{itemize}

Then the result is a twisted complex of right $\Ainfty$-$A$-modules:
\begin{tiny}
\begin{equation}
\label{eqn-right-module-bar-construction-Ainfty-version}
\begin{tikzcd}[column sep = 2.4cm]
\dots
\ar{r}[']{\begin{smallmatrix}EA^2 \mu_2  - EAm_2A + \\ + Em_2A^2 - p_2 A^3 \end{smallmatrix}}
\ar[bend left=25]{rr}[description]{EA \mu_3 + Em_3A + p_3 A^2 }
\ar[bend left=30]{rrr}[description]{E\mu_4 - p_4A}
\ar[bend left=35]{rrrr}[description]{\pi_5}
&
EA^3
\ar{r}[']{EA\mu_2 - E m_2 A + p_2 A^2}
\ar[bend left=25]{rr}[description]{-E\mu_3 - p_3A}
\ar[bend left=30]{rrr}[description]{\pi_4}
& 
EA^2
\ar{r}[']{E\mu_2 - p_2A}
\ar[bend left=25]{rr}[description]{\pi_3}
&
EA
\ar{r}[']{\pi_2}
&
\underset{\degzero}{E}.
\end{tikzcd}
\end{equation}
\end{tiny}
\end{prps}

\begin{proof}
See 
\cite[\S2.10]{AnnoLogvinenko-BarCategoryOfModulesAndHomotopyAdjunctionForTensorFunctors}. 
\end{proof}

We then upgrade $\infbar$ from a functor
into $\pretriagmns(\A)$ to a functor into $\pretriagmns(\nodA)$:
\begin{defn}
\label{defn-bar-construction-as-functor-into-pretriagmns-nodA}
\begin{enumerate}
\item For any object $(E, p_\bullet) \in \nodA$ 
define
$$ \infbar(E,p_\bullet) \in \pretriagmns(\nodA) $$
to be the twisted complex 
\eqref{eqn-right-module-bar-construction-Ainfty-version}. 

\item For any morphism $f_\bullet\colon (E, p_\bullet) \rightarrow  
(F, q_\bullet)$ in $\nodA$ define
$$ \infbar(f_\bullet) \in
\homm_{\pretriagmns(\nodA)}\left(\infbar(E,p_\bullet),
\infbar(F,q_\bullet)\right) $$
to be the lift of the bar-construction of $f_\bullet$ from $\A$ to $\nodA$ where in each component where $f_k$ doesn't involve the rightmost $A$ we replace $f_k$ by its image under the free functor, and in each component where it does we replace it by the morphism $f_{\bullet + k}\colon E \otimes A^k \rightarrow (F,q_\bullet)$ defined by $ (f_{\bullet + i})_j = f_{j+i}$. 

As usual, we illustrate the case when $f_\bullet$ is of odd degree:
\end{enumerate}
\begin{scriptsize}
\begin{equation}
\label{eqn-bar-complex-map-corresponding-to-Ainfty-morphism-Ainfty-version}
\begin{tikzcd}[row sep=2.5cm, column sep = 2.0cm]
\dots
\ar{r}{}
& 
EA^3
\ar{r}
\ar{d}[description, pos=0.6]{-f_1 A^3}
\ar{dr}[description, near start]{f_2 A^2}
\ar{drr}[description, near start]{-f_3A}
\ar{drrr}[description, near start]{f_{\bullet + 3}}
& 
EA^2
\ar{r}
\ar{d}[description, pos=0.6]{f_1A^2}
\ar{dr}[description, near start]{-f_2A}
\ar{drr}[description, near start]{f_{\bullet +2}}
&
EA
\ar{r}
\ar{d}[description, pos=0.6]{-f_1A}
\ar{dr}[description, near start]{f_{\bullet+1}}
&
(E,p_\bullet) 
\ar{d}[description, pos=0.6]{f_{\bullet}}
\\
\dots
\ar{r}{}
& 
FA^3
\ar{r}
& 
FA^2
\ar{r}
&
FA
\ar{r}
&
\underset{\degzero}{(F,q_\bullet)},
\end{tikzcd}
\end{equation}
\end{scriptsize}
\end{defn}

\begin{prps}
The assignments in 
Definition \ref{defn-bar-construction-as-functor-into-pretriagmns-nodA}
define the DG-functor
\begin{equation}
\infbar: \nodA \rightarrow \pretriagmns(\nodA)
\end{equation}
such that the following diagram of functors commutes:
\begin{equation}
\label{eqn-two-infbar-interwoven-by-forgetful-functor-nodA-to-A}
\begin{tikzcd} 
\nodA
\ar{r}{\infbar}
\ar{dr}[']{\infbar}
& 
\pretriagmns(\nodA)
\ar{d}{\forget}
\\
&
\pretriagmns(\A).
\end{tikzcd}
\end{equation}
\end{prps}
\begin{proof}
See 
\cite[\S2.10]{AnnoLogvinenko-BarCategoryOfModulesAndHomotopyAdjunctionForTensorFunctors}.  
\end{proof}

\subsection{Bar-resolution}
\label{section-bar-resolution}

Having realised the bar-construction of an $\Ainfty$-$A$-module as 
a twisted complex in $\nodA$, we arrive naturally at the notion of the 
bar-resolution. It is the free part of the bar-construction. 
Thus, when the bar-construction is acyclic, it gives a resolution 
of an $\Ainfty$-$T$-module by free modules:

\begin{defn}
\label{defn-bar-resolution-functor-for-Ainfty-A-modules}
Define the \em bar-resolution \rm DG-functor
\begin{equation}
\infbarres\colon 
\nodA \rightarrow \pretriagmns(\freeA)
\end{equation}
by setting $\infbarres(E,p_\bullet)$ to be the shifted subcomplex 
$\infbar(E,p_\bullet)_{\deg \leq -1}[-1]$
of $\infbar(E, p_\bullet)$
\begin{equation}
\label{eqn-bar-resolution-of-Ainfty-A-module}
\begin{tikzcd}[column sep = 2.75cm]
\dots
\ar{r}[']{\begin{smallmatrix}- EA^2 \mu_2  + EAm_2A - \\ - Em_2A^2 + p_2 A^3 \end{smallmatrix}}
\ar[bend left=25]{rr}[description]{- EA \mu_3 - Em_3A - p_3 A^2 }
\ar[bend left=30]{rrr}[description]{-E\mu_4 + p_4A}
&
EA^3
\ar{r}[']{- EA\mu_2 + E m_2 A - p_2 A^2}
\ar[bend left=25]{rr}[description]{E\mu_3 + p_3A}
& 
EA^2
\ar{r}[']{-E\mu_2 + p_2A}
&
\underset{\degzero}{EA}
\end{tikzcd}
\end{equation}
and setting $\infbarres(f_\bullet)$ to be the
corresponding restriction of the twisted complex map 
$\infbar(f_\bullet)$ sign-twisted by $(-1)^{\deg(f)}$. 
We illustrate the odd degree case:
\begin{equation}
\label{eqn-bar-resolution-of-Ainfty-morphism}
\begin{tikzcd}[row sep=2.5cm, column sep = 2.0cm]
\dots
\ar{r}{}
& 
EA^3
\ar{r}
\ar{d}[description, pos=0.75]{f_1 A^3}
\ar{dr}[description, pos = 0.5]{-f_2 A^2}
\ar{drr}[description, near start]{f_3A}
& 
EA^2
\ar{r}
\ar{d}[description, pos=0.75]{- f_1A^2}
\ar{dr}[description, pos = 0.5]{f_2A}
&
EA
\ar{d}[description, pos=0.75]{f_1A}
\\
\dots
\ar{r}
& 
FA^3
\ar{r}
& 
FA^2
\ar{r}
&
\underset{\degzero}{FA}
\end{tikzcd}
\end{equation}
This defines a DG-functor because
$\infbar$ maps $\Ainfty$-$A$-modules to one-sided
twisted complexes and $\Ainfty$-morphisms thereof 
to one-sided maps of twisted complexes. 
\end{defn}

By definition, we have a natural inclusion 
$\freeA \hookrightarrow \nodA$, 
and therefore a natural inclusion  
\begin{equation}
\label{eqn-inclusion-of-pretriagmns-freeA-into-pretriagmns-nodA}
\pretriagmns(\freeA) \hookrightarrow \pretriagmns(\nodA). 
\end{equation}
Using
\eqref{eqn-inclusion-of-pretriagmns-freeA-into-pretriagmns-nodA}, 
we can consider the bar-resolution to be the functor
\begin{equation*}
\infbarres\colon \nodA \rightarrow \pretriagmns(\nodA). 
\end{equation*}
By abuse of notation, write  
\begin{equation*}
\id_{\nodA}\colon \nodA \hookrightarrow
\pretriagmns(\nodA), 
\end{equation*}
for the tautological fully faithful inclusion which sends each element of 
$\nodA$ to itself considered as a twisted complex concentrated 
in degree $0$. 
\begin{defn}
\label{defn-natural-tranformation-rho-Ainfty-version}
Define the homotopy natural transformation 
\begin{equation}
\label{eqn-natural-tranformation-rho-Ainfty-version}
\rho\colon\infbarres \rightarrow \id_{\nodA}
\end{equation}
of DG-functors $\nodA \rightarrow \pretriagmns(\nodA)$
by setting $\rho(E,p_\bullet)$ to consist of the components
of $\infbar(E,p_\bullet)$ which were discarded to obtain 
$\infbarres(E, p_\bullet)$, that is:
\begin{equation}
\label{eqn-natural-transformation-infbarres-into-identity-Ainfty-version}
\begin{tikzcd}[row sep=1.5cm, column sep = 1.5cm]
\dots
\ar{r}{}
& 
EA^3
\ar{r}
\ar{drr}[description]{\pi_4}
& 
EA^2
\ar{r}
\ar{dr}[description]{\pi_3}
&
EA
\ar{d}[description]{\pi_2}
\\
& 
& 
&
\underset{\degzero}{(E,p_\bullet)}. 
\end{tikzcd}
\end{equation}
\end{defn}

By calling $\rho$ a homotopy natural transformation we mean that it
gives a natural transformation of the induced functors on the homotopy 
categories. To see this, note that for any $f_\bullet\colon
(E,p_\bullet) \rightarrow (F, q_\bullet)$ in $\nodA$ 
we can rewrite the twisted complex morphism $\infbar(f_\bullet)$
depicted on 
\eqref{eqn-bar-complex-map-corresponding-to-Ainfty-morphism-Ainfty-version}
as
\begin{equation}
\label{eqn-infbar-f_bullet-rewritten-as-square-with-bar-resolution-Ainfty-version}
\begin{tikzcd}
\infbarres(E,p_\bullet) 
\ar{r}{\rho}
\ar{d}[description]{(-1)^{\deg(f)} \infbarres(f_\bullet)}
\ar{dr}[description]{f_{\bullet + \bullet}}
&
(E,p_\bullet)
\ar{d}[description]{f_\bullet}
\\
\infbarres(F, q_\bullet)
\ar{r}{\rho}
&
(F,q_\bullet),
\end{tikzcd}
\end{equation}
where $f_{\bullet + \bullet}$ is a twisted complex
morphism whose 
$EA^i \rightarrow (b,q_\bullet)$ component is $f_{\bullet + i}$.  
If $f_\bullet$ is closed of degree $0$, then so is $\infbar(f_\bullet)$. 
We then see from 
\eqref{eqn-infbar-f_bullet-rewritten-as-square-with-bar-resolution-Ainfty-version}
that $\rho$ is natural on $f_\bullet$ up to the homotopy given 
by $f_{\bullet + \bullet}$. Note, that this is also true when 
$df_\bullet$ is strict, and thus $df_{\bullet + \bullet} = 0$. 

\begin{prps}
\label{prps-natural-transformation-rho-is-acyclic-on-H-unital-modules-Ainfty-version}
The homotopy natural transformation
\eqref{eqn-natural-tranformation-rho-Ainfty-version}
is a homotopy equivalence on objects of $\nodA$
whose bar-construction is acyclic. 
\end{prps}

Recall that we say that a twisted complex over $\nodA$ is acyclic 
if its underlying twisted complex over $\A$ is null-homotopic. 

\begin{proof}
For any $(E,p_\bullet) \in \nodA$, 
the total complex of the $\pretriagmns(\nodA)$ map
$\rho(E,p_\bullet)$
is $\infbar(E,p_\bullet)$. 
By the Homotopy Lemma (Lemma \ref{lemma-the-homotopy-lemma-for-nodA})
if $\infbar (E,p_\bullet) \in \pretriagmns(\nodA)$ is acyclic, then 
it is null-homotopic, and hence $\rho(E,p_\bullet)$ is a homotopy equivalence. 
\end{proof}

\section{Strong homotopy unitality}
\label{section-strong-homotopy-unitality}

Throughout this section, let $\A$ be a monoidal DG category. 
Let $(A,m_i)$ be an $\Ainfty$-algebra in $\A$. In examples, 
we also use a strict non-unital algebra $(T,\mu)$ in $\A$. 

\subsection{Unitality conditions for $A$} 
\label{section-unitality-conditions-for-algebras}

The simplest notions of unitality one can ask for in the case of an 
$\Ainfty$-algebra are strict unitality and homotopy unitality. 
These translate to our setting without a change:

\begin{defn}
\label{defn-strict-and-homotopy-unitality}
We say that $A$ is \em strictly unital \rm if there exists a unit
morphism $\eta\colon \id \rightarrow A$ in $\A$ 
such that:
\begin{enumerate}
\item $m_2 \circ (\eta \otimes \id) = m_2 \circ (\id \otimes \eta) = \id_A$. 
\item $m_i \circ (\id^j \otimes \eta \otimes \id^k) = 0$ for $i \geq
3$ and all appropriate $j$ and $k$. 
\end{enumerate}
 
We say that $A$ is \em weakly homotopy unital \rm if $(A, m_2)$ is a unital  
algebra in $H^0(\A)$. Explicitly, this means
that there exists a unit morphism $\eta$ in $\A$ as above 
such that 
\begin{equation}
\label{eqn-weak-homotopy-unitality-conditions}
m_2 \circ (\eta \otimes \id) = \id_A + d h^r 
\quad \text{ and } \quad 
m_2 \circ (\id \otimes {\eta}) = \id_A + d h^l, 
\end{equation}
for some degree $-1$ endomorphisms $h^l, h^r$ of $A$ in $\A$. 
\end{defn}

The following notion is based on the one introduced  
in \cite[Defn.~4.1.2.5]{Lefevre-SurLesAInftyCategories}:

\begin{defn}
We say that $A$ is \em $H$-unital \rm if its non-augmented bar
construction  $\infbarnaug A$ is null-homotopic in $\pretriagmns \A$. 
\end{defn}

It is clear that strict unitality implies weak 
homotopy unitality with $h^l = h ^r = 0$.
To see that strict unitality implies $H$-unitality, we consider
the standard contracting homotopy of the twisted complex
$\infbarnaug A$ whose components are the 
maps $\eta \otimes \id^i\colon A^i \rightarrow A^{i+1}$. 
However, when $\eta$ is only a weak homotopy unit, it doesn't seem 
to be possible to cook up such natural contracting homotopy 
with it.

\begin{exmpl}
\label{exmpl-homotopy-unitality-for-a-strict-algebra-1}
We give a running example where we illustrate the notions and the formulas of this section for the case of a strict algebra. Thus let here and below $(T,\mu)$ be an associative, non-unital algebra in $\A$. We view it as an $\Ainfty$-algebra with $m_2 = \mu$ and $m_i = 0$ for $i \geq 3$.

Since higher $m_i$ vanish,  the strict unitality of Defn.~\ref{defn-strict-and-homotopy-unitality} is equivalent to the usual algebra unitality: 
we must have a unit $\eta: \id \rightarrow T$ in $\A$
such that the compositions
\begin{equation}
T \xrightarrow{{\eta}T} T^2 \xrightarrow{\mu} T
\quad \text{ and } \quad
T \xrightarrow{T{\eta}} T^2 \xrightarrow{\mu} T
\end{equation}
are both $\id_T$.  For homotopy unitality, these compositions must be $\id_T + d h^l$ and $\id_T + dh^r$ for some homotopies $h^l$ and $h^r$ in $\homm^{-1}_{\A}(T,T)$. 

The non-augmented bar-construction of $T$ is the twisted complex
\begin{small}
\begin{equation}
\label{eqn-algebra-bar-construction-of-T}
\infbarnaug(T) := \quad 
... \rightarrow T^4 \xrightarrow{T^2 \mu  - T \mu T + \mu T^2} 
\rightarrow T^3 \xrightarrow{T{\mu}-{\mu}T} T^2 \xrightarrow{\mu}
\underset{\degzero}{T} 
\quad\quad \in \pretriagmns (\A). 
\end{equation}
\end{small}
For $H$-unitality of $T$, this  twisted complex must be null-homotopic. There must exist a contracting homotopy: a degree $-1$ endomorphism $\beta$ with $d\beta = \id_{\infbarnaug(T)}$. 

If $T$ is strictly unital, then the standard contracting homotopy of $\infbarnaug(T)$ is 
\begin{equation*}
\beta : = 
\begin{tikzcd}[column sep = 2.25cm]
\dots
\ar{r}
&
T^4
\ar{r}{T^2 \mu  - T \mu T + \mu T^2}
\ar{dl}[description]{- \eta T^4}
&
T^3
\ar{r}{T\mu  - \mu T}
\ar{dl}[description]{\eta T^3}
&
T^2
\ar{r}{\mu}
\ar{dl}[description]{- \eta T^2}
&
T 
\ar{dl}[description]{\eta T}
\\
\dots
\ar{r}
&
T^4
\ar{r}[']{T^2 \mu  - T \mu T + \mu T^2}
&
T^3
\ar{r}[']{T\mu - \mu T}
&
T^2
\ar{r}[']{\mu}
&
T
\end{tikzcd}
\end{equation*}
If $T$ is only weakly homotopy unital, there seems to be no natural way to modify $\beta$ to ensure that $d\beta$ is still $\id_T$  
\end{exmpl}

In \cite[Cor.~4.1.2.7]{Lefevre-SurLesAInftyCategories} it is shown 
that weak homotopy unitality implies $H$-unitality for the usual
$\Ainfty$-algebras. In \S\ref{section-Ainfty-algebras-in-a-monoidal-category}
we've shown that this translates in our context to the situation 
where $\A = \kmodk$ with the monoidal structure given by the tensor
product. Unfortunately, the proof in loc.~cit. relies on the existence of 
minimal models which uses the fact that in $k$-$\modd$-$k$
every bimodule is homotopic to the sum of its cohomologies. 
This already doesn't hold when $\A = \BmodB$ 
over an arbitrary DG category $\B$. There is no way to generalise that proof
for an arbitrary monoidal DG category $\A$, and the authors believe that
in this generality weak homotopy unitality doesn't imply $H$-unitality. 

To fix this we introduce the notion of strong homotopy unitality:

\begin{defn}
\label{defn-strong-homotopy-unitality-for-ainfty-algebras}
$\Ainfty$-algebra $(A,m_i)$ is \em strongly homotopy unital \rm if there 
exists a unit morphism $\eta\colon \id \rightarrow A$ in $\A$ and
degree $-1$ endomorphisms $h^r_{\bullet}$ and $h^l_{\bullet}$ of $A$, 
respectively, such that 
\begin{equation}
\label{eqn-strong-homotopy-unitality-conditions}
\mu_2 \circ {\eta}A = \id_A + d h^r_{\bullet}  \text{ in } \nodA
\quad \text{ and } \quad 
\mu_2 \circ A{\eta} = \id_A + d h^l_{\bullet} \text{ in } \Anod. 
\end{equation}
Here $\mu_2$ denotes the $\nodA$
and $\Anod$ morphisms $A^2 \rightarrow A$ of Defn.~\ref{defn-morphisms-pi_i-and-mu_i}. 
\end{defn}

Some motivation for this definition can be gleaned from considering the case of a strict algebra:

\begin{exmpl}
\label{exmpl-homotopy-unitality-for-a-strict-algebra-2}

We continue from Example \ref{exmpl-homotopy-unitality-for-a-strict-algebra-1}
where we considered the case of $A = T$ for a strict non-unital algebra
$(T,\mu)$ in $\A$. 

For any morphisms $\eta\colon \id \rightarrow T$ in $\A$ 
the morphisms $\mu \circ \eta T$ and $\mu \circ T \eta$ are naturally 
morphisms of right and of left $T$-modules in $\A$, respectively. 
It is natural to ask for $h^r$ and $h^l$ to also be such morphisms. 
However, the homotopies of a homotopy unit shouldn't be strict, but only 
$\Ainfty$-morphisms of $T$-modules.

Explicitly, this translates into the following. The 
non-augmented bar-construction $\infbarnaug T$ of $T$ as an $\Ainfty$-algebra
\begin{equation}
\label{eqn-left-bar-construction-of-T}
... \rightarrow T^4 \xrightarrow{T^2 \mu - T \mu T + \mu T^2} 
T^3 \xrightarrow{T{\mu}-{\mu}T} T^2 \xrightarrow{\mu} T. 
\end{equation}
coincides with its bar-construction as a left and as a right
$\Ainfty$-$T$-module. We refere to these as $\infbarl T$ and $\infbarr
T$.

Since $m_{\geq 3} = 0$, the morphism $\mu_2$ in $\noddinf\text{-}T$
is just $\mu$ viewed as a strict $\Ainfty$-morphism. 
Let $h^r_{\bullet}$ be a degree $-1$ endomorphism of $T$ in $\noddinf\text{-}T$ such that 
\begin{equation}
\label{eqn-exmpl-eta-T-strong-homotopy-unitality-condition}
\mu \circ {\eta}T = \id_T + d h^r_{\bullet}. 
\end{equation}
The endomorphism of $\infbarr T$ induced by $h^r_{\bullet}$ is 
\begin{equation}
\label{eqn-exmpl-h^r-on-the-bar-construction}
\begin{tikzcd}[column sep=2cm, row sep=2cm]
... 
\ar{r}
&
T^3 
\ar{r}{T{\mu} - {\mu}T} 
\ar{d}[description, pos = 0.3]{h^r_{1}T^2}
\ar{dr}[description, pos = 0.4]{- h^r_{2}T}
\ar{drr}[description, pos = 0.6]{h^r_{3}}
&
T^2
\ar{r}{\mu} 
\ar{d}[description, pos = 0.3]{- h^r_{1}T}
\ar{dr}[description, pos = 0.4]{h^r_{2}}
&
T 
\ar{d}[description, pos = 0.3]{h^r_{1}}
\\
... 
\ar{r}
&
T^3 
\ar{r}{T{\mu} - {\mu}T} 
&
T^2
\ar{r}{\mu} 
&
T.
\end{tikzcd}
\end{equation}
The condition \eqref{eqn-exmpl-eta-T-strong-homotopy-unitality-condition}
means that differentiating 
\eqref{eqn-exmpl-h^r-on-the-bar-construction} 
as a morphism of twisted complexes over $\A$ we get
\begin{equation}
\label{eqn-d-h^l-on-the-bar-construction}
\begin{tikzcd}[column sep=2cm, row sep=2cm]
... 
\ar{r}
&
T^3 
\ar{r}{T{\mu} - {\mu}T} 
\ar{d}[description]{{\mu}T^2 \circ {\eta}T^3 - \id}
&
T^2
\ar{r}{\mu} 
\ar{d}[description]{{\mu}T \circ {\eta}T^2 - \id}
&
T 
\ar{d}[description]{\mu \circ {\eta}T - \id}
\\
... 
\ar{r}
&
T^3 
\ar{r}{T{\mu} - {\mu}T} 
&
T^2
\ar{r}{\mu} 
&
T.
\end{tikzcd}
\end{equation}
In other words, 
in $\A$ we have
\begin{align}
\nonumber
dh^r_{{1}} &= \mu \circ {\eta}T - \id_T , \\
\label{eqn-h^l-condition}
dh^r_{{2}} &= \mu \circ h^r_{{1}}T -  h^r_{{1}} \circ \mu, \\
\nonumber
dh^r_{{3}} &= \mu \circ h^r_{{2}}T - h^r_{{2}} \circ (T{\mu}-{\mu}T), \\
\nonumber
&\dots
\end{align}
Similarly, we have
\begin{align}
\nonumber
dh^l_{{1}} &= \mu \circ T{\eta} - \id_T, \\
\label{eqn-h^r-condition}
dh^l_{{2}} &= \mu \circ T h^l_{{1}} -  h^l_{{1}} \circ \mu, \\
\nonumber
dh^l_{{3}} &= - \mu \circ T h^l_{{2}} + h^l_{{2}} \circ (T{\mu}-{\mu}T), \\
\nonumber
&\dots
\end{align}
\end{exmpl}

The existence of either $h^l_\bullet$ or $h^r_\bullet$
is enough for us to be able to construct 
the contracting homotopy of $\infbarnaug A$ in the same way as 
in the strictly unital case:
\begin{lemma}
\label{lemma-strong-homotopy-unital-implies-H-unital}
If $A$ is strongly homotopy unital, then it is $H$-unital.  
\end{lemma}
\begin{proof}
As a twisted complex, non-augmented algebra bar-construction $\infbarnaug A$ coincides with right module bar-construction $\infbarr A$. It suffices therefore to show that $\infbarr A$ is null-homotopic in $\pretriagmns \A$. 

Consider the following endomorphism of $\infbarr A$ in $\pretriagmns \A$:
\begin{small}
\begin{equation*}
\beta : = 
\begin{tikzcd}[column sep = 3.25cm]
\cdots\quad
A^4
\ar{r}[description]{A^2m_2 - A m_2 A + m_2 A^2}
\ar[bend left=15]{rr}[description]{-Am_3 - m_3A}
\ar[bend left=16]{rrr}[description]{m_4}
& 
A^3
\ar{r}[description]{Am_2 - m_2A}
\ar[bend left=15]{rr}[description]{m_3}
\ar{dl}[description]{\eta A^3}
&
A^2
\ar{r}[description]{m_2}
\ar{dl}[description]{- \eta A^2}
&
\underset{\degzero}{A}
\ar{dl}[description]{\eta A}
\\
\cdots\quad
A^4
\ar{r}[description]{A^2m_2 - A m_2 A + m_2 A^2}
\ar[bend right=15]{rr}[description]{-Am_3 - m_3A}
\ar[bend right=16]{rrr}[description]{m_4}
& 
A^3
\ar{r}[description]{Am_2 - m_2A}
\ar[bend right=15]{rr}[description]{m_3}
&
A^2
\ar{r}[description]{m_2}
&
\underset{\degzero}{A}. 
\end{tikzcd}
\end{equation*}
\end{small}
It can be readily checked that
\begin{equation*}
d\beta : = 
\begin{tikzcd}[column sep = 3.25cm, row sep  = 2.5cm]
\cdots\quad
A^4
\ar{r}
\ar[shift left = 0.25cm]{d}[description, pos = 0.82]{m_2A^3 \circ \eta A^4}
\ar{dr}[description, pos = 0.82]{- m_3A^2 \circ \eta A^4}
\ar{drr}[description, pos = 0.82]{m_4A \circ \eta A^4}
\ar{drrr}[description, pos = 0.8]{- m_5 \circ \eta A^4}
&
A^3
\ar{r}
\ar{d}[description, pos = 0.68]{m_2A^2 \circ \eta A^3}
\ar{dr}[description, pos = 0.7]{- m_3A \circ \eta A^3}
\ar{drr}[description, pos = 0.68]{m_4 \circ \eta A^3}
&
A^2
\ar{r}
\ar{d}[description, pos = 0.37]{m_2 A \circ \eta A^2}
\ar{dr}[description, pos=0.52]{- m_3 \circ \eta A^2}
&
A 
\ar{d}[description, pos = 0.4]{m_2 \circ \eta A}
\\
\cdots\quad
A^4
\ar{r}
&
A^3
\ar{r}
&
A^2
\ar{r}
&
A,
\end{tikzcd}
\end{equation*}
thus it is the bar-construction of  morphism $\mu_2 \circ \eta A$ in $\nodA$. 

Since $A$ is strongly homotopy unital, we have in $\nodA$
$$ \mu_2 \circ \eta A = \id_A + dh^r_{\bullet}. $$
Hence in $\pretriagmns \A$ we have
$$ d\beta = \id_{\infbarr A} + d\left(\infbar
h^r_{\bullet}\right), $$
whence $\infbarr A$ is contractible, as desired. 
\end{proof}

We will see in the next section, that the condition of strong homotopy
unitality is enough to ensure that for $\Ainfty$-$A$-modules
$H$-unitality is equivalent to ordinary homotopy unitality. However, to have 
a good notion of strong homotopy unitality for $A$-modules, we need
the following stronger notion for $A$ itself:

\begin{defn}
\label{def-bimodule-homotopy-unitality}
Let $(A,m_i)$ be an $\Ainfty$-algebra in a monoidal DG category $\A$. 
We say that $A$ is \em bimodule homotopy unital \rm if there is 
a degree $0$ morphism 
$$ \bareta_{\bullet\bullet}\colon \id_\A \rightarrow A, $$
in $\AnodA$ such that
\begin{equation}
\label{eqn-bimodule-homotopy-unitality-condition}
(d\bareta)_{\bullet\bullet}
= \left\{
 (d\bareta)_{10} = \id_A,  (d\bareta)_{01} = \id_A
\right\}. 
\end{equation}
Here the $\Ainfty$-$A$-$A$-bimodule structures on $A$ and $\id_\A$ are
given by the natural action of $A$ on itself with $p_{ij} = m_{i+j+1}$, 
and the zero action of $A$ on $\id_\A$ with $p_{ij} = 0$. 
\end{defn}

\begin{exmpl}
\label{exmpl-homotopy-unitality-for-a-strict-algebra-3}

We continue illustrating the case $A = (T,\mu)$, a strict algebra in $\A$.

The bimodule bar-construction of $T$  as an $\Ainfty$ $T$-$T$-bimodule is the bicomplex 
\begin{equation}
\infbarbi(T) = 
\label{eqn-bar-bicomplex-of-T}
\begin{tikzcd}[column sep = 1.5cm, row sep = 1.5cm]
&
...
\ar{d}
&
...
\ar{d}
&
...
\ar{d}
\\
...
\ar{r}
&
T^5 
\ar{r}{T^3{\mu} - T^2{\mu}T}
\ar{d}[description]{ T{\mu}T^2 - {\mu}T^3}
&
T^4 
\ar{r}{T^2{\mu}}
\ar{d}[description]{ -(T{\mu}T - {\mu}T^2)}
& 
T^3
\ar{d}[description]{T{\mu} - {\mu}T}
\\
...
\ar{r}
&
T^4 
\ar{r}{T^2{\mu} - T{\mu}T}
\ar{d}[description]{{\mu}T^2}
&
T^3 
\ar{r}{T{\mu}}
\ar{d}[description]{- {\mu}T}
& 
T^2
\ar{d}[description]{\mu}
\\
...
\ar{r}
&
T^3 
\ar{r}{T{\mu} - {\mu}T}
&
T^2 
\ar{r}{\mu}
& 
T.  
\end{tikzcd}
\end{equation}
Its rows are bar-constructions of $T^i$ as left $T$-modules and its
columns are sign-twisted bar-constructions of $T^i$ as right
$T$-modules. By Theorem 
\ref{theorem-ainfty-bimodules-are-ainfty-modules-in-cat-of-ainfty-modules}
it can be viewed as the bar-construction of $\infbarl T$ as a right 
$\Ainfty$-$T$-module or as the sign-twisted bar-construction of 
$\infbarr T$ as a left $\Ainfty$-$T$-module.  

Similarly, the bimodule bar-construction of $\id_{\A}$ is the bicomplex 
\begin{equation}
\label{eqn-bar-bicomplex-of-Id}
\infbarbi(\id) = 
\begin{tikzcd}[column sep = 1.5cm, row sep = 1.5cm, execute at end picture={
 \draw[dotted, line width=0.8pt]
    ([yshift=-10,xshift=-15\pgflinewidth]ID.south west) --   
    ([yshift=-10,xshift=15\pgflinewidth]ID.south east) -- 
    ([yshift=10,xshift=15\pgflinewidth]ID.north east) --
    ([yshift=10,xshift=-15\pgflinewidth]ID.north west) --
    ([yshift=-10,xshift=-15\pgflinewidth]ID.south west);
\draw[dotted, line width=0.8pt]
    ([yshift=-10,xshift=-15\pgflinewidth]TH-W.south west) --   
    ([yshift=-10,xshift=15\pgflinewidth]TH-E.south east) -- 
    ([yshift=10,xshift=15\pgflinewidth]TH-E.north east) --
    ([yshift=17,xshift=-15\pgflinewidth]TH-W.north west);
\draw[dotted, line width=0.8pt]
    ([yshift=10,xshift=-14\pgflinewidth]TV-N.north west)--
    ([yshift=-10,xshift=-15\pgflinewidth]TV-S.south west) --   
    ([yshift=-10,xshift=15\pgflinewidth]TV-S.south east) -- 
    ([yshift=10,xshift=15\pgflinewidth]TV-N.north east);
\draw[dotted, line width=0.8pt]
    ([yshift=-10,xshift=-15\pgflinewidth]T2-SW.south west) --   
    ([yshift=-10,xshift=12\pgflinewidth]T2-SE.south east) -- 
    ([yshift=10,xshift=14\pgflinewidth]T2-NE.north east);
  }
]
&
...
\ar{d}
&
...
\ar{d}
&
|[alias=T2-NE]|
...
\ar{d}
&
|[alias=TV-N]|
...
\ar{d}
\\
...
\ar{r}
&
T^6 
\ar{r}{T^4{\mu} - T^3{\mu}T}
\ar{d}[description]{T{\mu}T^3 - {\mu}T^4}
&
T^5 
\ar{r}{T^3{\mu}}
\ar{d}[description]{-T{\mu}T^2 + {\mu}T^3}
&
T^4 
\ar{r}{0}
\ar{d}[description]{T{\mu}T - {\mu}T^2}
& 
T^3
\ar{d}[description]{-T{\mu}+{\mu}T}
\\
...
\ar{r}
&
T^5 
\ar{r}{T^3{\mu} - T^2 {\mu}T }
\ar{d}[description]{{\mu}T^3}
&
T^4 
\ar{r}{{T^2\mu}}
\ar{d}[description]{-{\mu}T^2}
&
T^3 
\ar{r}{0}
\ar{d}[description]{{\mu}T}
& 
T^2
\ar{d}[description]{-\mu}
\\
|[alias=T2-SW]|
...
\ar{r}
&
T^4 
\ar{r}{T^2{\mu} - T{\mu}T}
\ar{d}[description]{0}
&
T^3 
\ar{r}{{T\mu}}
\ar{d}[description]{0}
&
|[alias=T2-SE]|
T^2 
\ar{r}{0}
\ar{d}[description]{0}
& 
|[alias=TV-S]|
T
\ar{d}[description]{0}
\\
|[alias=TH-W]|
...
\ar{r}
&
T^3
\ar{r}{{T\mu - \mu T}}
&
T^2 
\ar{r}{{\mu}}
&
|[alias=TH-E]|
T 
\ar{r}{0}
& 
|[alias=ID]|
\id_\A.  
\end{tikzcd}
\end{equation}
It splits up as a direct sum of four twisted complexes
which can be viewed as:
\begin{itemize}
\item $\id_\A$ concentrated in degree $(0,0)$,
\item $\infbarl(T)[1]$ concentrated in column $0$, 
\item $\infbarr(T)[1]$ concentrated in row $0$ with every 
differential sign-twisted by $-1$,
\item $\infbarbi(T^2)[1,1]$. 
\end{itemize}
Recall that the shift functor $[1]$ also sign-twists every 
differential by $1$. 
\end{exmpl}


\begin{theorem}
\label{theorem-conditions-for-bimodule-homotopy-unitality}
$A$ is bimodule homotopy unital if and only if there exist
\begin{itemize}
\item a degree $0$ morphism $\eta\colon \id_{\A} \rightarrow A$ in
$\A$,  
\item a degree $-1$ morphism $h^l_\bullet\colon A \rightarrow A$
in $\Anod$,
\item a degree $-1$ morphism $h^r_\bullet\colon A \rightarrow A$ 
in $\nodA$,
\item a degree $-2$ morphism $\kappa_{\bullet\bullet} \colon A^2
\rightarrow A$ in $\AnodA$, 
\end{itemize}
such that in these categories 
\begin{align*}
d\eta &= 0, \\
dh^r_\bullet &= \id_{\A} - \mu_2 \circ \eta{A}, \\
dh^l_\bullet &= \id_{\A} - \mu_2 \circ A\eta, \\
d\kappa_{\bullet\bullet} &= \mu_2 \circ (h^l_\bullet{A} -
{A}h^r_\bullet) - \mu_3 \circ A \eta A.
\end{align*}
\end{theorem}

Here in each category, $\eta{A}$, $A{\eta}$,
$h^l_\bullet{A}$, ${A}h^r_\bullet$ and $A\eta{A}$ denote 
$\Ainfty$-morphisms whose only non-zero components they specify. 
E.g.~$A\eta{A}$ is a degree $0$ morphism $A^2 \rightarrow A^3$ in $\A$,
however in $\AnodA$ it denotes the degree $2$ 
morphism $\id_A \rightarrow A^3$ whose only non-zero component is
the $(1,1)$-th component $A\eta{A}$. Moreover, in each
respective category, $\mu_k\colon A^k \rightarrow A$ is the
$\Ainfty$-morphism defined in 
\S\ref{section-bar-construction-as-a-complex-of-ainfty-A-modules}. 
The bimodule version is defined by $(\mu_k)_{ij} = (-1)^{(k-1)j} m_{i+j+1}$, 
and left/right module versions are obtained from it by setting $i$ or 
$j$ to be zero. 

\begin{proof}
A degree $0$ morphism 
$\bareta_{\bullet\bullet} \colon \id \rightarrow A$ 
in the category of $\Ainfty$ $A$-$A$-bimodules is 
a collection of degree $2-i-j$ natural transformations
$$ \bareta_{ij}\colon A^{i-1} A^{j-1} \rightarrow A \quad \quad i,j \geq 1. $$

Let $\infbarbi(\bareta_{\bullet\bullet})$ be the induced map of
twisted bicomplexes. Similar to what we saw in Example
\ref{exmpl-homotopy-unitality-for-a-strict-algebra-3}, 
$\infbarbi(\id_\A)$ breaks up as a direct sum of four
twisted bicomplexes: 
\begin{itemize}
\item $\id_\A$ concentrated in degree $(0,0)$, 
\item $\infbarl(A)[1]$ concentrated in column $0$,
\item $\widehat{\infbarr(A)}[1]$ concentrated in row $0$, 
\item $\infbarbi(A^2)[1,1]$. 
\end{itemize}
Here $\widehat{(-)}$ is the automorphism of the DG category of twisted
complexes which sign-twists every differential $\alpha_{ij}$ and every 
morphism component $f_{ij}$ by $(-1)^{i+j}$. It is isomorphic to the 
identity functor via the isomorphism $(E, \alpha_{ij}) \simeq 
\widehat{(E, \alpha_{ij})}$ whose components are the maps 
$(-1)^{i} \id\colon E_i \rightarrow E_i$. 

Since $\widehat{\infbarr(A)} \simeq \infbarr(A)$,  
$\infbarbi(\id_\A)$ is isomorphic 
to the total complex of the following twisted bicomplex 
of twisted bicomplexes: 
\begin{equation}
\label{eqn-bar-bicomplex-of-Id-subcomplexes-ainfty}
\begin{tikzcd}
\infbarbi(A^2) 
\ar{r}{0}
\ar{d}{0}
&
\infbarl(A) 
\ar{d}{0}
\\
\infbarr(A)
\ar{r}{0}
& 
\id_\A
\end{tikzcd}
\end{equation}
In the rest of the proof we use this to implicitly identify 
$\infbarbi(\bareta_{\bullet\bullet})$ with
\eqref{eqn-bar-bicomplex-of-Id-subcomplexes-ainfty}. 

For any $\Ainfty$-bimodule morphism 
$f_{\bullet\bullet}\colon \id \rightarrow A$
write $f^0$, $f^r$, $f^l$, and $f^{bi}$ for the restrictions of the
twisted bicomplex map $\infbarbi(f_{\bullet\bullet})$
to the four elements of
$\eqref{eqn-bar-bicomplex-of-Id-subcomplexes-ainfty}$. 
These filter through the subcomplexes of $\infbarbi(A)$ given by 
the degree $(0,0)$ element, the $0$th row, the $0$th column, and 
the whole of $\infbarbi(A)$. We identify the first three with 
$A$, $\infbarr(A)$, and $\infbarl(A)$, respectively. 
The restriction of $\infbarbi(f_{\bullet\bullet})$
to each of $\id_\A$, $\infbarr(A)$, $\infbarl(A)$, and $\infbarbi(A^2)$
is given by the respective bar-construction of 
the morphisms of $f_{\bullet\bullet}$ which originate in 
that twisted complex plus the corrections which come from the 
bimodule bar-constructions of those morphisms of 
of $\bareta_{\bullet\bullet}$ which originate in the higher 
degrees of the bicomplex.  

Viewing $\bareta_{\bullet\bullet}$ as a collection
of maps from the elements of $\infbarbi(\id_\A)$ to $A$, 
use the identification of  $\infbarbi(\id_\A)$ and 
$\eqref{eqn-bar-bicomplex-of-Id-subcomplexes-ainfty}$
to write $\eta$, $h^r_\bullet$, $h^l_\bullet$, and
$\kappa_{\bullet\bullet}$ for the restrictions
of $\bareta_{\bullet\bullet}$ to 
$\id_\A$, $\infbarr(A)$, $\infbarl(A)$, and $\infbarbi(A^2)$. 
In other words, $\eta = \bareta_{00}$, $h^r_i = (-1)^{i-1}
\bareta_{0i}$, $h^l_i = \bareta_{i0}$, and $\kappa_{ij} = 
\bareta_{(i+1)(j+1)}$. 

Conversely we use the identification of 
$\eqref{eqn-bar-bicomplex-of-Id-subcomplexes-ainfty}$
with $\infbarbi(\id_\A)$, to view any one of
$\eta$, $h^r_\bullet$, $h^l_\bullet$, or $\kappa_{\bullet\bullet}$
as a collection of maps from the elements of
$\infbarbi(\id_\A)$ to $A$, 
and thus as an $\Ainfty$-bimodule morphism $\id_\A \rightarrow A$.
In this sense
\begin{equation}
\label{eqn-decomposition-of-bareta}
\bareta_{\bullet\bullet} = \eta + h^r_\bullet + h^l_\bullet + 
\kappa_{\bullet\bullet}. 
\end{equation}
We then write e.g.~$\eta^r$ or $(h^r_\bullet)^{bi}$ for 
the corresponding component of its bar-construction. 
For example, $(h^r_\bullet)^0 = (h^r_\bullet)^l = 0$, 
while $(h^r_\bullet)^r = \infbarr(h^r_\bullet)$, and similarly for
$h^l_\bullet$. Also  
\begin{equation}
\label{eqn-bareta-infbarl-A-second-component-r}
\eta^r = 
\begin{tikzcd}
\dots 
\ar{r}
&
A^3  
\ar{r}
\ar{dl}[description]{\eta A^3}
&
A^2  
\ar{r}
\ar{dl}[description]{-\eta A^2}
&
A  
\ar{dl}[description]{\eta A}
\\
\dots 
\ar{r}
&
A^3  
\ar{r}
&
A^2  
\ar{r}
&
A, 
\end{tikzcd}
\end{equation}
\begin{equation}
\label{eqn-bareta-infbarl-A-second-component-l}
\eta^l = 
\begin{tikzcd}
\dots 
\ar{r}
&
A^3  
\ar{r}
\ar{dl}[description]{A^3 \eta}
&
A^2  
\ar{r}
\ar{dl}[description]{A^2 \eta}
&
A  
\ar{dl}[description]{A \eta}
\\
\dots 
\ar{r}
&
A^3  
\ar{r}
&
A^2  
\ar{r}
&
A. 
\end{tikzcd}
\end{equation}

With this notation in mind, we have:
\begin{itemize}
\item $(\bareta_{\bullet\bullet})^0$ 
is the degree $0$ morphism $\eta\colon \id_\A \rightarrow A$ itself. 
\item $(\bareta_{\bullet\bullet})^{r}$
is the degree $-1$ morphism $ \infbarr(h^r_\bullet) + \eta^r$. 
\item $(\bareta_{\bullet\bullet})^l$
is the degree $-1$ morphism $ \infbarl(h^l_\bullet) + \eta^l$. 
\item $(\bareta_{\bullet\bullet})^{bi}$
is the degree $-2$ morphism $\infbarbi(\kappa_{\bullet \bullet}) 
+ (h^l_\bullet)^{bi} + (h^{r}_{\bullet})^{bi} +
\eta^{bi}$. 
\end{itemize}

The condition that $(d\bareta)_{\bullet\bullet}$ has only two 
non-zero components
$$ (d\bareta)_{10} = \id_A \quad\text{ and }\quad (d\bareta)_{01} =
\id_A, $$
is therefore equivalent to:
\begin{itemize}
\item $d\left((\bareta_{\bullet\bullet})^{0}\right) = 0$. This
simply means that $d\eta = 0$ in $\A$. 

\item $d((\bareta_{\bullet\bullet})^l) = \infbarl(\id^l_A)$,where
$\id^l_A$ is the identity map of $A$ in $\Anod$. By above we
have
$$ d((\bareta_{\bullet\bullet})^l) =  \infbarl(dh^l_\bullet) +
d(\eta^l) $$
and while $\eta^l$ is not a left-module 
bar-construction of something, we have $d(\eta^l) = \infbarl(\mu_2 \circ
A\eta)$. We conclude that this condition is equivalent to  
\begin{equation}
\label{eqn-proof-bimodule-homotopy-unitality-l-condition}
dh^l_\bullet = \id^l_A - \mu_2 \circ A\eta \quad \quad \text{ in }
\quad \Anod. 
\end{equation}

\item $d((\bareta_{\bullet\bullet})^r) = \infbarr(\id^r_A)$, 
where $\id^r_A$ is the identity map of $A$ in $\nodA$. 
Similarly, this condition is equivalent to 
\begin{equation}
\label{eqn-proof-bimodule-homotopy-unitality-r-condition}
dh^r_\bullet = \id^r_A - \mu_2 \circ \eta{A}
\quad \quad \text{ in } \quad \nodA. 
\end{equation}

\item $d((\bareta_{\bullet\bullet})^{bi}) = (\id^{l}_T)^{bi} + 
(\id^{r}_T)^{bi}$. 
From \eqref{eqn-decomposition-of-bareta} we have
$$ d((\bareta_{\bullet\bullet})^{bi}) = \infbarbi(d\kappa_{\bullet
\bullet})
+ d(h^{l}_\bullet)^{bi} + d(h^{r}_{\bullet})^{bi} + d\eta^{bi}. $$
We have $d\eta^{bi} =  (d_{bimod}\eta)^{bi}$. Now
note that $(d_{{bimod}}\eta)_{00} = d\eta = 0$ and 
$$ (d_{\text{bimod}}\eta)_{ij} = m_{i+j+1} \circ A^i \eta A^j 
\quad \quad i + j \geq 1. $$
Split $d_{{bimod}}\eta$ up into the summands
originating in $\infbarr(A)$, $\infbarl(A)$, and $\infbarbi(A^2)$.
We can write these summands as $\mu_2 \circ {\eta}A - 
\mu_3 \circ A{\eta}A$, $\mu_2 \circ A{\eta} - \mu_3 \circ A{\eta}A$, 
and $\mu_3 \circ A{\eta}A$, respectively. Indeed, we have componentwise 
\begin{align*}
(\mu_2 \circ {\eta}A)_{ij} & = m_{i+j+1} \circ A^i (\eta A) A^{j-1},
\quad \quad i \geq 0, j \geq 1,
 \\
(\mu_2 \circ {A\eta})_{ij} & = m_{i+j+1} \circ A^{i-1} (A\eta) A^j, 
\quad \quad i \geq 1, j \geq 0,
\\
(\mu_3 \circ A{\eta}A)_{ij} & = m_{i+j+1} \circ A^{i-1}(A{\eta}A) A^{j-1},
\quad \quad i \geq 1, j \geq 1,
\end{align*}
with the remaining components being zero. Thus 
$$ d_{{bimod}}\eta = \mu_2 \circ {\eta}A + \mu_2 \circ A{\eta}   
- \mu_3 \circ A{\eta}A. $$

Next, split $d_{\text{bimod}} h^l_\bullet$ into the summands
originating in $\infbarl(A)$ and  $\infbarbi(A^2)$. The former 
has the same components as $\Anod$ morphism $d_{{lmod}} h^l_\bullet$ 
which equals $\id_A^{l} - \mu_2 \circ A{\eta}$ by
\eqref{eqn-proof-bimodule-homotopy-unitality-l-condition}. We can 
therefore write this summand in $\AnodA$ as  
$\id_A^{l} - \mu_2 \circ A{\eta} + \mu_3 \circ A{\eta}A$. 
As before, $\mu_3 \circ A{\eta}A$ is needed to cancel out 
the components of $\mu_2 \circ A{\eta}$ which originate outside $\infbarl(A)$.  
On the other hand, the summand of $d_{\text{bimod}} h^l_\bullet$ 
originating in $\infbarbi(A^2)$ can be rewritten as 
$\mu_2 \circ h^l_\bullet{A}$. Thus 
\begin{align*}
d_{{bimod}} h^l_\bullet
 = \id_A^{l} - \mu_2 \circ A{\eta} + \mu_3 \circ A{\eta}A + \mu_2 \circ h^l_\bullet{A}, 
\end{align*}
and computing $d_{{bimod}} h^r_\bullet$ in a similar way we obtain
\begin{align*}
d_{{bimod}} h^r_\bullet
&=  \id_A^{r} - \mu_2 \circ {\eta}A + \mu_3 \circ A{\eta}A - \mu_2 \circ {A}h^r_\bullet. 
\end{align*}

We conclude that the condition that 
$d((\bareta_{\bullet\bullet})^{bi}) = (\id^{l}_T)^{bi} + (\id^{r}_T)^{bi}$
is equivalent to 
$$ \infbarbi(d\kappa_{\bullet \bullet}) = \left(\mu_2 \circ
(h^l_\bullet{A} - {A}h^r_\bullet) + \mu_3 \circ A \eta A\right)^{bi}. $$
As $\mu_2 \circ (h^l_\bullet{A} - {A}h^r_\bullet) + \mu_3 \circ A \eta A$ 
originates in $\infbarbi(A^2)$, we have 
$$ \infbarbi \left( \mu_2 \circ (h^l_\bullet{A} - {A}h^r_\bullet) + \mu_3
\circ A \eta A\right) = \left( \mu_2 \circ (h^l_\bullet{A} - {A}h^r_\bullet)
+ \mu_3 \circ A \eta A \right)^{bi}, $$
and therefore the above is equivalent to 
$$ d\kappa_{\bullet\bullet} = \mu_2 \circ (h^l_\bullet{A} -
{A}h^r_\bullet) + \mu_3 \circ A \eta A \quad \quad \text{ in } \AnodA. $$
\end{itemize}
\end{proof}

\begin{cor}
Bimodule homotopy unitality implies strong homotopy unitality. 
\end{cor}

Very explicitly, we give below the first few of the relations in $\A$ 
that the components $\eta_{ij}\colon A^{i+j} \rightarrow A$ of 
$\Ainfty$-bimodule morphism $\eta_{\bullet\bullet} \id_\A \rightarrow A$
have to satisfy in order to fit the definition of bimodule homotopy unitality:
\begin{align*}
d\bareta_{00} &= 0, \\
d\bareta_{01} + m_2 \circ \bareta_{00}A &= \id_A, \\
d\bareta_{10} + m_2 \circ A\bareta_{00} &= \id_A, \\
d\bareta_{02} + m_2 \circ \bareta_{01}A - \bareta_{01} \circ m_2 + m_3
\circ \bareta_{00}A^2 &= 0, \\
d\bareta_{11} + m_2 \circ (\bareta_{10}A - A \bareta_{01}) + m_3 \circ
A \bareta_{00}A &= 0, \\
d\bareta_{20} - m_2 \circ A \bareta_{10} + \bareta_{10}\circ m_2 + m_3
\circ A^2 \bareta_{00} &= 0, \\
d\bareta_{03} + m_2 \circ \bareta_{02}A - \bareta_{02} \circ (Am_2 -
m_2 A) + m_3 \circ \bareta_{01}A^2 + \bareta_{01} \circ m_3 + m_4
\circ \bareta_{00}A^3 &= 0, \\
d\bareta_{12} + 
m_2 \circ (\bareta_{11}A + A\bareta_{02}) - \bareta_{11}\circ Am_2
+ m_3 \circ (\bareta_{10}A - A\bareta_{01})A 
+ m_4 \circ A\bareta_{00}A^2 &= 0, \\
d\bareta_{21} 
+ m_2 \circ (\bareta_{20}A + A\bareta_{11}) - \bareta_{11}\circ m_2 A
- m_3 \circ A (\bareta_{10}A - A \bareta_{01})
+ m_4 \circ A^2 \bareta_{00}A 
& = 0, \\
d\bareta_{30} 
+ m_2 \circ A\bareta_{20}  + \bareta_{20} \circ \left( A m_2 - m_2 A\right) 
+ m_3 \circ A^2 \bareta_{10}+ \bareta_{10} \circ m_3 
+ m_4 \circ A^3 \bareta_{00} 
&= 0, 
\\
\dots &\dots \dots
\end{align*}

\subsection{Unitality conditions for $A$-modules} 
\label{section-unitality-conditions-for-A-modules}

The notions defined in the previous section have analogues 
for $\Ainfty$-$A$-modules:
\begin{defn}
Let $(E,p_\bullet)$ be a right $\Ainfty$-$A$-module. 
\begin{itemize}
\item $(E,p_\bullet)$ is \em $H$-unital \rm 
if its bar-construction $\infbar(E,p_\bullet)$ is acyclic. 
We denote by $\nodhuA$ to be the full subcategory of $\nodA$ consisting 
of $H$-unital modules. 
\end{itemize}

If $A$ is strongly homotopy unital, then: 
\begin{itemize}
\item  $(E,p_\bullet)$ is \em strictly
unital \rm if in $\A$ the composition
$$ EA^i \xrightarrow{EA^j\eta A^{i-j}} EA^{i+1} \xrightarrow{p_{i+2}} E $$
equals $\id_E$ for $i = 0$ and $0$ otherwise. We write
$\moddinf\text{-}A$ for the full subcategory of $\nodA$ consisting of
strictly unital modules. 

\item $(E,p_\bullet)$ is \em homotopy unital \rm if $(E,p_2)$ is
a strictly unital $H^0(A)$-module in $H^0(\A)$. Explicitly, it means
that in $\A$ there exists a degree $-1$ endomorphism $h$ of $E$ such 
that 
\begin{equation}
\label{item-homotopy-unitality-condition-for-A-modules}
E \xrightarrow{E\eta} EA \xrightarrow{p_2} E = \id_E + dh. 
\end{equation}
\end{itemize}
These notions are defined in the same way for left $\Ainfty$-$A$-modules.
\end{defn}

It is more difficult to define the notion of strong homotopy
unitality for $\Ainfty$-$A$-modules. One might be tempted to simply ask for 
there to exist such $h_\bullet$ that 
$$ (E,p_\bullet) \xrightarrow{E\eta} EA \xrightarrow{\pi_2} (E, p_\bullet) = \id_E + dh_\bullet $$
in $\noddinf(A)$. This turns out to be a bad notion: it is not, for
example, stable under homotopy equivalences. Also, note that $\eta$ 
isn't closed as an $\Ainfty$-morphism. Since $\pi_2$ is, we would have
to have $d(E\eta) \circ \pi_2 = 0$, a somewhat
unnatural condition. 

It turns out that for a good notion we need to replace $EA$ by
the whole bar-resolution $\infbarres(E,p_\bullet)$. However, we then 
need to lift the closed degree zero map 
\begin{equation}
\chi : = 
\begin{tikzcd}
& & & 
E 
\ar{d}[description]{E\eta}
\\
\dots 
\ar{r}
\ar[bend left=20]{rr}
\ar[bend left=20]{rrr}
&
EA^3
\ar{r}
\ar[bend left=20]{rr}
&
EA^2 
\ar{r}
&
EA 
\end{tikzcd}
\end{equation}
in $\pretriagmns(\A)$ to a closed degree zero map
$\chi$ in $\pretriagmns (\noddinf(A))$:
\begin{equation}
\chi = 
\begin{tikzcd}[row sep=1.3cm]
& & & 
(E,p_\bullet)
\ar{d}[description]{\chi_{{1}\bullet}}
\ar{dl}[description]{\chi_{{2}\bullet}}
\ar{dll}[description]{\chi_{{3}\bullet}}
\ar[phantom]{dlll}[description]{\cdots}
\\
\dots 
\ar{r}
\ar[bend left=20]{rr}
\ar[bend left=20]{rrr}
&
EA^3 
\ar{r}
\ar[bend left=20]{rr}
&
EA^2
\ar{r}
&
EA, 
\end{tikzcd}
\end{equation}
which we could then use to define strong homotopy unitality for $\Ainfty$-$A$-modules. 

Strong homotopy unitality of $A$ is not sufficient for this, 
but bimodule homotopy unitality is.
View $A$ and $\id_\A$ as $\Ainfty$ $A$-$A$-bimodules with the
natural action on the former and the zero action on the latter. 
The bimodule homotopy unitality structure on $A$ is an 
$\AnodA$-morphism $\bareta_{\bullet\bullet}\colon \id_\A \rightarrow A$
with $d\bareta_{\bullet\bullet} = \id^l_A + \id^r_A$. 
Here $\id^l_A$ and $\id^r_A$ are the $\Ainfty$-bimodule 
maps $\id_\A \rightarrow A$ whose only component is $\id_A$ 
in degree $(1,0)$ and $(0,1)$, respectively. 
See \S\ref{section-unitality-conditions-for-algebras} for more
details. 

With this, we can construct $\chi$ as follows.  
Write $\delta_i\colon \id_A \rightarrow
A^i$ for the morphism of right $\Ainfty$-$A$-modules whose sole component 
is $\id_{A^i}$. Write 
$$ \Delta\colon \id_\A \rightarrow \infbarl(\id_\A) $$
for the closed degree $0$ 
map $\sum (-1)^i\delta_i$ of twisted complexes of right
$\Ainfty$-$A$-modules. 

We now make implicit use of the category isomorphism 
$\AnodA \simeq A\text{-}(\nodA)$ of Theorem  
\ref{theorem-ainfty-bimodules-are-ainfty-modules-in-cat-of-ainfty-modules}
to view $\id_\A$ and $\A$ as left $\Ainfty$-$A$-modules in 
the category of right $\Ainfty$-$A$-modules and 
$\bareta_{\bullet\bullet}$ as a morphism thereof. 
We thus define
\begin{equation}
\label{eqn-bicomplex-version-of-chi-underlying}
\chi^0 : = \id_\A \xrightarrow{\Delta}
\infbarl(\id_\A)
\xrightarrow{\infbarl(\bareta_{\bullet\bullet})}
\infbarl(A) \quad \quad \text{ in } \pretriagmns\left(\nodA\right).
\end{equation}

\begin{lemma}
$\chi^0$ is a degree $0$ closed morphism. 
\end{lemma}
\begin{proof}
It is of degree $0$ because both $\Delta$ and $\bareta_{\bullet\bullet}$ 
were defined to be of degree $0$.  
Since $\Delta$ is a closed map, to show
$\chi^0$ to be closed it remains to show that 
$$ \id_\A \xrightarrow{\Delta}
\infbarl(\id_\A)
\xrightarrow{\infbarl(d \bareta_{\bullet\bullet})} \infbarl(A)
 \; = 0 \quad \quad \quad \text{ in } \pretriagmns\left(\nodA\right). 
$$

We have  
$d\bareta_{\bullet\bullet} = \id^l_A + \id^r_A$ in $\AnodA$. 
% In $A\text{-}(\nodA)$ this becomes a morphism 
% $f_\bullet\colon \id_A \rightarrow A$ which has two components: $f_1 =
% \delta_1$ and $f_2 = \id_A$. Its left module bar construction 
% $\infbarl(f_\bullet)$ is the morphism of twisted complexes whose 
% components are maps $(-1)^i A^i \delta_1$ and $\id_{A^i}$ from each $A^i$ in
% $\infbarl(\id_\A)$. Its composition with $\Delta = \sum (-1)^i
% \delta_i$ is zero, because  
% $$ A^i \delta_1 \circ \delta_i = \delta_{i+1}. $$
The isomorphism 
of Theorem  
\ref{theorem-ainfty-bimodules-are-ainfty-modules-in-cat-of-ainfty-modules}
intertwines $\infbarbi$ with sign-twisted 
$\cxrow \circ \infbarr \circ \infbarl$, so it remains to show
that $\infbarbi(\id^l_A + \id^r_A)$ composes to zero with
$\cxrow \circ \infbarr\left((-1)^{ij + kl}\Delta\right)$. 
The latter is the diagonal map which takes $\infbarr(\id_\A)$, 
viewed as a bicomplex concentrated in the row $0$, and maps its
elements to the diagonals of $\infbarbi(\id_\A)$ using the sign-twisted
identity maps $(-1)^{i}\id\colon A^{i+j} \rightarrow A^iA^j$. 
It composes to zero with the bar-construction
of any $\Ainfty$-bimodule morphism 
$\id_A \rightarrow A$ whose components
in every diagonal sum to zero with alternating signs. 
In particular, with that of $\id^l_A + \id^r_A$. 
\end{proof}


Let $(E,p_\bullet) \in \nodA$. Let $(E,0) \in \nodA$ be
$E$ with the zero action of $A$. 
We have canonically
$$ (E,0) \simeq E \otimes \id_\A, $$
$$ \infbarres (E,0) \simeq E \otimes \infbarres(\id_\A). $$
Twisted complexes $\infbarres(\id_\A)$ and 
$\infbarl(A)$ differ only in signs. We identify
them with the isomorphism $\phi = \sum_i (-1)^{i+1}\id_{A^i}$
and thus have a closed degree $0$ map 
\begin{equation}
\label{eqn-bicomplex-version-of-chi-zero} 
E(\phi\circ \chi^0) := 
(E,0) \xrightarrow{E\chi^0}
E\infbarl(A)
\xrightarrow{E\phi}
\infbarres(E,0) \quad \text{ in } \pretriagmns\left(\nodA\right). 
\end{equation}
We also have non-closed degree $0$ maps 
$\id\colon (E,p_\bullet) \rightarrow (E,0)$ and 
$\id\colon (E,0) \rightarrow (E,p_\bullet)$
whose only non-zero component is $\id_E$. 

\begin{defn}
\label{defn-a-infinity-version-of-map-chi}
For any $(E,p_\bullet) \in \nodA$ 
define a degree $0$ map  
$$ \chi \colon (E,p_\bullet) \xrightarrow{\id}
(E,0) \xrightarrow{E(\phi\circ \chi^0)}
\infbarres(E,0) \xrightarrow{\infbarres(\id)}
\infbarres(E,p_\bullet) 
\quad \quad
\text{ in } \pretriagmns\left(\nodA\right).
$$ 
\end{defn}

Explicitly, we can write $\chi$ in terms of $\bareta_{\bullet\bullet}$
as follows. The components of $\chi$ are 
$$ \chi_{i\bullet}\colon (E,p_\bullet) \rightarrow EA^i \quad \quad
\text{ in } \nodA, $$
and we have
\begin{align*}
\chi_{ij} & = 0 \text{ if } i>j,\\
\chi_{ij} & = (-1)^{(i-1)j}EA^{i-1}\sum_{k=0}^{j-i} (-1)^k \eta_{k,j-i-k}.
\end{align*}
The first few terms are:
\begin{align*}
\chi_{11} & = E\bareta_{00}, \\
\chi_{12} & = E(\bareta_{01} - \bareta_{10}), \\
\chi_{13} & = E(\bareta_{02} - \bareta_{11} + \bareta_{20}), \\
\dots \\
\chi_{22} & = EA \bareta_{00}, \\
\chi_{23} & = - EA (\bareta_{01} - \bareta_{10}), \\
\chi_{24} & = EA (\bareta_{02} - \bareta_{11} + \bareta_{20}), \\
\dots \\
\chi_{33} & = EA^2\bareta_{00}, \\
\chi_{34} & = EA^2(\bareta_{01} - \bareta_{10}), \\
\chi_{35} & = EA^2(\bareta_{02} - \bareta_{11} + \bareta_{20}), \\
\dots
\end{align*}

\begin{prps}
The maps $\chi$ in
Defn.~\ref{defn-a-infinity-version-of-map-chi}
define a closed, degree zero natural transformation
of functors $\nodA \rightarrow \pretriagmns(\nodA)$:
$$ \chi\colon \id \rightarrow \infbarres. $$
\end{prps}
\begin{proof}
Let $f_{\bullet}\colon (E,p_\bullet) \rightarrow (F, q_\bullet)$ be
a morphism in $\nodA$ and  
$f_\bullet\colon (E,0) \rightarrow (F,0)$ be the morphism with the same components. Consider the following diagram:
\begin{equation}
\begin{tikzcd}[column sep = 1.9cm]
(E,p_\bullet) 
\ar{r}{\id}
\ar{d}[description]{f_\bullet}
&
(E,0) 
\ar{r}{E(\phi \circ \chi^0)}
\ar{d}[description]{f_\bullet}
&
\infbarres(E,0) 
\ar{r}{\infbarres(\id)}
\ar{d}[description]{\infbarres(f_\bullet)}
&
\infbarres(E,p_\bullet) 
\ar{d}[description]{\infbarres(f_\bullet)}
\\
(F,q_\bullet) 
\ar{r}{\id}
&
(F,0) 
\ar{r}{F(\phi \circ \chi^0)}
&
\infbarres(F,0) 
\ar{r}{\infbarres(\id)}
&
\infbarres(F,q_\bullet).
\end{tikzcd}
\end{equation}
The left and the right squares commute trivially. To see that the central
square commutes, we compute directly the upper-right and the lower-left
compositions around its perimeter. These are twisted complex maps 
$(E,0) \rightarrow \infbarres(F,0)$ and thus comprise 
$\Ainfty$-morphisms $(E,0) \rightarrow FA^i$ for $i \geq 1$. 
In the lower-left composition, the non-zero components 
of $(E,0) \rightarrow FA^i$ are the morphisms  
$$ EA^{i+j-1} \rightarrow  EA^{i} \quad \quad j \geq 0 $$
in $\A$ given by
$$ \sum_{k = 0}^{j} 
(-1)^{(i-1)(i+j-k)}FA^{i-1}\left(\sum_{l = 0}^{j-k} (-1)^l \bareta_{l,j-k-l}\right) 
\circ  
(-1)^{\deg(f)(i+j-k+1)} f_{k+1} A^{i-1} A^{j-k}. $$
In the upper-right composition, the same components 
of $(E,0) \rightarrow FA^i$ are non-zero, but now they are given by
$$ \sum_{k = 0}^{j} 
(-1)^{\deg(f)(i+1)} f_{k+1} A^{i-1} A
\circ
(-1)^{(i+k-1)(i+j)}E A^k A^{i-1}\left(\sum_{l = 0}^{j-k} (-1)^l
\bareta_{l,j-k-l}\right).
$$
These two expressions are equal by DG bifunctoriality of monoidal
operation: for any morphisms $g$ and $h$ in $\A$ we have
$$ (g \otimes \id) \circ (\id \otimes h) = g \otimes h =
(-1)^{\deg(g)\deg(h)} (\id \otimes h) \circ (g \otimes \id). $$
Thus the perimeter commutes, and $\chi$ is a natural transformation. 

Next, note that
$$ d(\id)\colon (E,p_\bullet) \rightarrow (E,0), $$
$$ d(\id)\colon (E,0) \rightarrow (E,p_\bullet), $$
are the maps $- \pi_1 \circ \id $ and $\id \circ \pi_1$ where
$\pi_1\colon (E,p_\bullet) \rightarrow (E,p_\bullet)$ is the map 
defined in \S\ref{section-bar-construction-as-a-complex-of-ainfty-A-modules} 
whose $EA^i \rightarrow E$ component is $p_{i+1}$ for $i \geq 1$ and
$0$ for $i = 0$. Thus
\begin{align*}
d(\chi_{(E,p_\bullet)}) = 
& \infbarres(\pi_1) \circ \infbarres(\id) \circ 
E(\phi \circ \chi^0) \circ \id
\; - 
\\
& - \;  0 \; - 
\\
& - \;
\infbarres(\id) \circ
E(\phi \circ \chi^0) \circ
\id \circ \pi_1 = \infbarres(\pi_1) \circ \chi_{(E,p_\bullet)} -
\chi_{(E,p_\bullet)} \circ \pi_1. 
\end{align*}
This is zero since $\chi$ is a natural transformation. 
\end{proof}

We now define a good notion of strong homotopy unitality for
$\Ainfty$-$A$-modules:
\begin{defn}
\label{defn-strong-homotopy-unitality}
Let $A$ be bimodule homotopy unital. 
We say that a module $(E,p_\bullet) \in \nodA$ is \em strongly homotopy 
unital \rm if there exists a degree $-1$ endomorphism $h_\bullet$ 
of $(E,p_\bullet)$ in $\nodA$  such that 
$$ (E,p_\bullet) \xrightarrow{\chi} \infbarres(E,p_\bullet)
\xrightarrow{\rho} (E,p_\bullet) \quad = \quad \id + dh_\bullet. $$
\end{defn}

\begin{lemma}
\label{lemma-strong-homotopy-unitality-stable-under-homotopy-equivalences}
Strong homotopy unitality is stable under homotopy equivalences.  
\end{lemma}
\begin{proof}
Our definition of strong homotopy unitality asks that in 
the homotopy category $\rho \circ \chi(E,p_\bullet) =
\id_{(E,p_\bullet)}$. Since $\chi$ is natural transformation and $\rho$ is a
homotopy natural transformation, their composition $\rho \circ \chi$ is also 
a homotopy natural transformation. In any category, 
the set of objects on which any
given natural transformation of the identity functor evaluates to $\id$ 
is closed under isomorphisms. Applying this to $H^0(\nodA)$, we obtain 
the desired assertion. 
\end{proof}

The following is a surprisingly non-trivial result which makes use of 
all of the higher homotopies provided by the bimodule homotopy
unitality of $A$:

\begin{lemma}
\label{lemma-free-modules-are-strong-homotopy-unital}
If $A$ is bimodule homotopy unital, then free $A$-modules are strongly
homotopy unital. 
\end{lemma} \begin{proof}
Let $E \in \A$. The composition $\rho\circ\chi$ for $EA$ is simply
the same composition for $A$ multiplied by $E$ on the left,
so it is enough to prove that $A$ as a right module over itself
is strongly homotopy unital.

Consider the bar-construction of $\chi^0$. It is a closed map
of degree $0$ between twisted complexes
$\infbarr(\id_\A)\to\infbarr(\infbarl(A))$.
Note that we can present $\infbarr(\id_\A)$ as the trivial twisted
complex
$$
\widehat{\infbarr(A)} \xrightarrow{0} \id_\A.
$$
Similarly, we can present $\infbarr(\infbarl(\id_A))$  and $\infbarr(\infbarl(A))$ respectively as
\begin{align*}
\infbarr(\infbarl(A^2))[1] \oplus \widehat{\infbarr(A) }
&\xrightarrow{0}
\infbarl(A)[1] \oplus \id_\A,\\
\infbarr(\infbarres(A)) 
&\xrightarrow{\infbarr(\rho)}
\infbarr(A).
\end{align*}
Then the bar-construction of $\chi^0$ becomes
\begin{equation}
\label{eqn-decomposition-of-delta-bareta}
\begin{tikzcd}
\widehat{\infbarr(A)} 
\ar{r}{0}
\ar{d}[description]{\Delta_{1}}
\ar{dr}[description]{\Delta_{10}}
&
\id_{\A}
\ar{d}[description]{\Delta_0}
\\
\infbarbi(A^2)[1] \oplus \widehat{\infbarr(A)} 
\ar{r}{0}
\ar{d}[description]{\eta_{1}}
\ar{dr}[description]{\eta_{10}}
& 
\infbarl(A)[1] \oplus \id_\A 
\ar{d}[description]{\eta_0}
\\
\infbarr(\infbarres(A))
\ar{r}{\infbarr(\rho)}
&
\underset{\degzero}{\infbarr(A)}. 
\end{tikzcd}
\end{equation}
Here $\Delta_i$ and $\eta_i$ are the components of
$\infbarr(\Delta)$ and $\infbarr(\infbarl(\bareta_{\bullet\bullet}))$,
respectively. 

Now observe that the composition 
$$
\eta_1\circ\Delta_1\circ \phi\colon \infbarr(A) \to \infbarr(\infbarres(A))
$$
has the same components as $\infbarr(\chi)$. On the other hand, we have
$$
d(\eta_0\circ\Delta_{10}+\eta_{10}\circ\Delta_1)+\infbarr(\rho)\circ\eta_1\circ\Delta_1 = 0.
$$
Since $\eta_0\circ\Delta_{10}\circ\phi$ equals $(\bareta_{\bullet\bullet})^l$ composed
with the isomorphism of twisted complexes $\infbarr(A)\simeq\infbarl(A)$ whose components are
negative identity maps, we have $d(\eta_0\circ\Delta_{10}\circ\phi)=-\infbarr(\id_A)$. Then
$$
\infbarr(\chi) - \infbarr(\id)+d(\eta_{10}\circ\Delta_1\circ\phi)=0,
$$
which is what we want.

\end{proof}

%end of edits

We can now prove the main theorem of this section which 
can be compared to similar results for usual 
$\Ainfty$-modules, cf.~\cite[\S4.1.3]{Lefevre-SurLesAInftyCategories}.

\begin{theorem}
\label{theorem-tfae-unitality-conditions-for-A-modules}
Let $A$ be strongly homotopy unital and let $(E,p_\bullet)$ be an 
$\Ainfty$-$A$-module. The following are equivalent:
\begin{enumerate}
\item 
\label{item-A-module-is-homotopy-unital}
$(E,p_\bullet)$ is homotopy unital, 
\item 
\label{item-A-module-is-H-unital}
$(E,p_\bullet)$ is $H$-unital. 
\end{enumerate}
If $A$ is bimodule homotopy unital, these are further equivalent to:
\begin{enumerate}
\setcounter{enumi}{2} 
\item 
\label{item-A-module-is-strong-homotopy-unital}
$(E,p_\bullet)$ is strongly homotopy unital. 
\end{enumerate}
\end{theorem}

We need two preliminary results. The first constructs a specific 
contracting homotopy of the bimodule bar-construction of $A$ viewed,
via Theorem
\ref{theorem-ainfty-bimodules-are-ainfty-modules-in-cat-of-ainfty-modules},
as the left module bar-construction $ \infbarl A$ of $A$ as 
a left $A$-$\Ainfty$-module in $\nodA$:
$$ 
\infbarbi A  \simeq \cxrow \left( \infbarr \left( \infbarl A\right)\right). 
$$
The bar-construction $\infbarl A$ is a twisted complex over $\nodA$, 
so by Homotopy Lemma 
it is acyclic if and only if its underlying twisted complex over $\A$
is acyclic. This underlying twisted complex coincides with $\infbarnaug A$, 
the non-augmented algebra bar-construction of $A$. Thus 
$\infbarl A$ is acyclic if $A$ is $H$-unital.
By Lemma \ref{lemma-strong-homotopy-unital-implies-H-unital}, 
$A$ is $H$-unital if it is strongly homotopy unital. 
 

However, we need more than the existence of some contracting
homotopy of $\infbarl A$. To prove the crucial Chi-Rho Lemma
below, we need a specific contracting homotopy of 
$\infbarl A$ constructed out of the bimodule homotopy unit of $A$:

\begin{lemma}
\label{lemma-the-contracting-homotopy-zeta-of-infbarl-a}
Let $A$ be bimodule homotopy unital with the unit
$\bareta_{\bullet\bullet}: \id \rightarrow A$. Let $\delta_{ij}\colon
A^i \rightarrow A^j$ be the single component morphism 
$\id_{A_j}$ if $j \geq i \geq 0$ and $0$ otherwise. 
Let $\zeta\colon \infbarl A \rightarrow \infbarl \id$ be the map 
whose $ij$-th component is $(-1)^{ij}\delta_{-(i-1),-j}$:
\begin{equation}
\begin{tikzcd}[column sep=2cm, row sep=1.5cm]
\dots
&
A^4  
\ar{r}
\ar[bend left=15]{rr}[description]{-A\mu_3 - m_3A}
\ar[bend left=20]{rrr}[description]{\mu_4}
& 
A^3
\ar{r}[description]{A\mu_2 - m_2 A}
\ar[bend left=15]{rr}[description]{\mu_3}
\ar{dl}[description]{\delta_{33}}
&
A^2 
\ar{r}[description]{\mu_2}
\ar{dl}[pos = 0.30, description]{\delta_{22}}
\ar{dll}[description]{-\delta_{23}}
&
A
\ar{dl}[pos = 0.30, description]{\delta_{11}}
\ar{dll}[description]{\delta_{12}}
\ar{dlll}[pos = 0.70, description]{\delta_{13}}
\\
\dots
&
A^3_0
\ar{r}[description]{-Am_2 + m_2A}
\ar[bend right=15]{rr}[description]{-m_3}
&
A^2_0
\ar{r}[description]{-m_2}
&
A_0
&
\id_A. 
\end{tikzcd}
\end{equation}
Here $A^i_0$ denotes $A^i$ as 
a right $\Ainfty$-module with the zero action of $A$. 

Then the map 
$$ \infbarl{A} \xrightarrow{\zeta} \infbarl{\id_A}
\xrightarrow{\bareta} \infbarl{A} $$
is a contracting homotopy of $\infbarl{A}$.  
\end{lemma}
\begin{proof}
It can be readily checked that $\zeta$ is a closed morphism. We then
have 
$$ d(\bareta \circ \zeta) = (d\bareta) \circ \zeta. $$
By definition of bimodule homotopy unitality, $(d\bareta)$ viewed
as a morphism of left module bar-constructions in $\nodA$ is
\begin{equation}
\begin{tikzcd}[column sep=2cm, row sep=1.5cm]
\dots
&
A^3_0
\ar{r}[description]{-Am_2 + m_2A}
\ar[bend left=15]{rr}[description]{-m_3}
\ar{d}[description]{-\delta_{34}}
\ar{dr}[description]{\delta_{33}}
&
A^2_0
\ar{r}[description]{-m_2}
\ar{d}[description]{\delta_{23}}
\ar{dr}[description]{\delta_{22}}
&
A_0
\ar{dr}[description]{\delta_{11}}
\ar{d}[description]{-\delta_{12}}
&
\id_A,
\ar{d}[description]{\delta_{01}}
\\
\dots
&
A^4  
\ar{r}
\ar[bend right=15]{rr}[description]{-A\mu_3 - m_3A}
\ar[bend right=20]{rrr}[description]{\mu_4}
& 
A^3
\ar{r}[description]{A\mu_2 - m_2 A}
\ar[bend right=15]{rr}[description]{\mu_3}
&
A^2 
\ar{r}[description]{\mu_2}
&
A.
\end{tikzcd}
\end{equation}

Since $\delta_{jk} \circ \delta_{ij} = (-1)^{(i-j)(j-k)} \delta_{ik}$ and 
$\delta_{ii} \circ \delta_{ii} = \delta_{ii} = \id_{A^i}$, we see that 
$$d\bareta \circ \zeta = \id_{\infbarl A}.$$
\end{proof}

Let $(E,p_\bullet)$ be an $\Ainfty$-$A$-module. Recall that
the composition $\rho \chi$ being homotopic to $\id_{(E,p_\bullet)}$ is
the definition of strong homotopy unitality. We now prove that, 
on the contrary, the composition $\chi \rho$ is always homotopic 
to $\id_{\infbarres(E,p_\bullet)}$:
 
\begin{lemma}[Chi-Rho Lemma]
\label{lemma-chi-rho-lemma}
Let $A$ be bimodule homotopy unital and let $(E,p_\bullet) \in \nodA$. There is a degree $-1$ endomorphism 
$\xi$ of $\infbarres (E,p_\bullet)$ with 
$$d\xi = \id_{\infbarres(E,p_\bullet)} - \chi \circ \rho. $$ 
\end{lemma}
\begin{proof}
Recall the isomorphism  
$$ E\infbarl(A) \xrightarrow{E\phi} \infbarres(E,0) $$
of twisted complexes over $\nodA$ used in the construction of the map $\chi$. 
Write 
$$ E(\zeta\circ\bareta)\colon \infbarres(E,0) \rightarrow \infbarres(E,0) $$
for the degree $-1$ morphism induced via the isomorphism $E\phi$
from the contracting homotopy $\zeta\circ\bareta$ of $\infbarl(A)$. 
Recall further the nonclosed degree $0$ maps 
\begin{align*}
\id\colon &(E,p_\bullet) \rightarrow (E,0), \\
\id\colon &(E,0) \rightarrow (E,p_\bullet), 
\end{align*}
whose only non-zero component is $\id_E$. Their differentials are 
$\pi_1 \circ \id$ and $\id \circ \pi_1$ where $\pi_1: (E,p_\bullet)
\rightarrow (E,p_\bullet)$ is the map
defined in \S\ref{section-bar-construction-as-a-complex-of-ainfty-A-modules} 
whose $EA^i \rightarrow E$ component is $p_{i+1}$ for $i \geq 1$ and
$0$ for $i = 0$.

Set $\xi$ to be the composition 
\begin{equation}
\infbarres(E,p_\bullet)
\xrightarrow{ \infbarres(\id)} 
\infbarres(E,0)
\xrightarrow{E(\zeta\circ\bareta)}
\infbarres(E,0)
\xrightarrow{\infbarres(\id) }
\infbarres(E,p_\bullet). 
\end{equation}
We then have
\begin{align*}
d(\xi) = 
\infbarres(\pi_1)\xi + \id - \xi\infbarres(\pi_1) = 
\id - [\xi,\infbarres(\pi_1)]. 
\end{align*}
One readily verifies that $[\xi,\infbarres(\pi_1)] = \chi\rho$, whence
the desired assertion. 
\end{proof}

When $A$ is not bimodule homotopy unital, but only strong homotopy
unital, we do not have the whole of $\bareta$, but we have $\eta$,
$h^l_\bullet$, and $h^r_\bullet$. 
We therefore do not have the whole of the contracting homotopy
$\zeta \circ \bareta$ of Lemma
\ref{lemma-the-contracting-homotopy-zeta-of-infbarl-a} however we have
its first component $\zeta_1$ which involves only $\eta$ and
$h^l_\bullet$. In this context, by the first component we mean the image
under the forgetful functor 
$$ \forget\colon \pretriagmns \nodA \rightarrow \pretriagmns \A. $$
Thus $\zeta_1$ is a contracting homotopy of $\forget(\infbarl(A))$, 
i.e. the bar-construction of $A$ as a left $\Ainfty$-module in $\A$, and 
not $\nodA$. Indeed, it is a version of the contracting homotopy of 
$\infbarnaug(A) = \infbarl(A) = \infbarr(A)$ given in Lemma 
\ref{lemma-strong-homotopy-unital-implies-H-unital} which uses $h^l$ 
instead of $h^r$. 

We thus still obtain a version of Chi-Rho Lemma for the first components: 
\begin{cor}
\label{cor-chi-rho-lemma-for-the-first-components}
Let $A$ be strongly homotopy unital and let $(E,p_\bullet) \in \nodA$. 
Consider the following maps in $\pretriagmns \A$ 
between $E$ and $\forget(\infbarres(E,p_\bullet))$: 
\begin{equation}
E\eta  : = 
\begin{tikzcd}
& & & 
E 
\ar{d}[description]{E\eta}
\\
\dots 
\ar{r}
\ar[bend left=20]{rr}
\ar[bend left=20]{rrr}
&
EA^3
\ar{r}
\ar[bend left=20]{rr}
&
EA^2 
\ar{r}
&
EA 
\end{tikzcd}
\end{equation}
\begin{equation}
\forget(\rho) = 
\begin{tikzcd}[row sep=1.3cm]
\dots 
\ar{r}
\ar[bend left=20]{rr}
\ar[bend left=20]{rrr}
\ar[phantom]{drrr}[description]{\cdots}
&
EA^3 
\ar{r}
\ar[bend left=20]{rr}
\ar{drr}[description]{p_4}
&
EA^2
\ar{r}
\ar{dr}[description]{p_3}
&
EA,
\ar{d}[description]{p_2}
\\
& & & 
E.
\end{tikzcd}
\end{equation}
There is a degree $-1$ endomorphism 
$\xi_1$ of $\forget(\infbarres (E,p_\bullet))$ with 
$$d\xi_1 = \id_{\forget(\infbarres(E,p_\bullet))} - E\eta \circ \forget(\rho). $$ 
\end{cor}

\begin{proof}[Proof of Theorem~\ref{theorem-tfae-unitality-conditions-for-A-modules}]
$\eqref{item-A-module-is-homotopy-unital} 
\Leftrightarrow 
\eqref{item-A-module-is-H-unital}:$

Observe that $\infbar(E,p_\bullet)$ is the total complex of the
twisted complex map $\forget(\rho)$.  Thus $(E,p_\bullet)$ is $H$-unital if 
and only if $\forget(\rho)$ is a homotopy equivalence. By the 
first component version of Chi-Rho Lemma
(Cor.~\ref{cor-chi-rho-lemma-for-the-first-components}), 
$\forget(\rho)$ has a left homotopy inverse $E\eta$. Thus
$\forget(\rho)$ is a homotopy equivalence if and only if $E\eta$ is
also a right homotopy inverse of $\forget(\rho)$. Since 
$\forget(\rho) \circ E\eta = p_2 \circ \eta$, 
$E\eta$ is a right homotopy inverse of $\forget(\rho)$
if and only if $p_2 \circ \eta$ is homotopic to $\id_E$, 
which is the definition of $(E,p_\bullet)$ being homotopy unital. 


$\eqref{item-A-module-is-strong-homotopy-unital} 
\Leftrightarrow 
\eqref{item-A-module-is-H-unital}:$

This is similar, but with $A$ bimodule homotopy unital we can 
witness the firepower of our fully armed and operational 
Chi-Rho Lemma (Lemma \ref{lemma-chi-rho-lemma}). 


By Homotopy Lemma, $\forget(\rho)$ is a homotopy equivalence if and
only if $\rho$ is. Thus $(E,p_\bullet)$ is $H$-unital if and only if
$\rho$ is a homotopy equivalence. By Chi-Rho Lemma, $\rho$
has a left homotopy inverse $\chi$. 
Thus $\rho$ is a homotopy equivalence if and only if $\chi$ is also 
a right homotopy inverse of $\rho$, which is the definition of
strong homotopy unitality. 
\end{proof}

\subsection{Free-Forgetful homotopy adjunction}
\label{section-free-forgetful-homotopy-adjunction}

In this section we give what we consider one of the main applications
of strong homotopy unitality. We show that for $A$ strongly homotopy
unital the functors
defined in \S\ref{section-free-modules-and-bimodules-over-Ainfty-algebra}
\begin{equation*}
\free\colon \A \rightarrow \nodA  
\end{equation*}
\begin{equation*}
\forget\colon \nodA \rightarrow \A  
\end{equation*}
become homotopy adjoint if we restrict to the subcategory
$\nodhuA$ of homotopy unital modules. Note that as $\forget$ 
discards all the higher components of
$\Ainfty$-morphisms, there is no hope for a genuine adjunction of 
$\free$ and $\forget$. 

\begin{defn}
\label{defn-free-forgetful-adjunction-unit-counit}
Let $A$ be a strongly homotopy unital $\Ainfty$-algebra in a monoidal DG
category $\A$. Define the natural transformation 
$$ \id_{A} \xrightarrow{\unit} \forget \free $$ 
by setting for any $E \in \A$ the map
$\unit_{E}$ to be
$$ E \xrightarrow{E\eta} EA. $$
Define the homotopy natural transformation 
$$ \free \forget \xrightarrow{\counit} \id_{\nodA} $$
by setting for any $(E,p_\bullet) \in \nodA$ the map
$\counit_{(E,p_\bullet)}$ to be 
$$ EA \xrightarrow{\pi_2} (E,p_\bullet). $$
\end{defn}

We first show that the counit map above is indeed a homotopy natural
transformation. Recall that $\pi_2$ is one of the maps $\pi_i: EA^i
\rightarrow (E,p_\bullet)$ defined
in \S\ref{section-bar-construction-as-a-complex-of-ainfty-A-modules}. 
\begin{lemma}
Let $A$ be an $\Ainfty$-algebra in a monoidal DG category $\A$. 
Then $\pi_2\colon EA \rightarrow (E,p_\bullet)$ defines a homotopy 
natural transformation $\free\forget \rightarrow \id_{\nodA}$. 
\end{lemma}
\begin{proof}
We need to show that for any closed degree $0$ morphism
$f_\bullet\colon (E,p_\bullet) \rightarrow (F,q_\bullet)$ in $\nodA$
the following square commutes up to homotopy:
\begin{equation}
\label{eqn-pi_2-homotopy-natural-transformation-square}
\begin{tikzcd}
EA 
\ar{r}{\pi_2}
\ar{d}{f_1 A}
&
(E,p_\bullet)
\ar{d}{f_\bullet}
\\
FA
\ar{r}{\pi_2}
&
(F,q_{\bullet}).  
\end{tikzcd}
\end{equation}

Observe that the square \eqref{eqn-pi_2-homotopy-natural-transformation-square}
is the restriction of the square
\eqref{eqn-infbar-f_bullet-rewritten-as-square-with-bar-resolution-Ainfty-version}
of one-sided twisted complexes to the leading term of each complex. 
We saw in \S\ref{section-bar-resolution} that the square 
\eqref{eqn-infbar-f_bullet-rewritten-as-square-with-bar-resolution-Ainfty-version} is a presentation of 
$\infbar(f_\bullet)$ 
and thus for closed $f$ it commutes up to the homotopy 
given by $f_{\bullet+\bullet}: \infbarres(E,p_\bullet) \rightarrow
(F,q_\bullet)$. It follows that
\eqref{eqn-pi_2-homotopy-natural-transformation-square} commutes up
to the homotopy given by $f_{\bullet+1}$, 
the restriction of $f_{\bullet+\bullet}$ to the leading terms. 
\end{proof}

\begin{theorem}
\label{theorem-free-forgetful-homotopy-adjunction}
Let $A$ be a strongly homotopy unital $\Ainfty$-algebra in a monoidal DG
category $\A$. The functors $\free$ and $\forget$ with the unit and
the counit maps of Definition \ref{defn-free-forgetful-adjunction-unit-counit}
are a homotopy adjoint pair of functors $\A \leftrightarrows \nodhuA$. 
\end{theorem}
\begin{proof}
Let $E \in A$. The composition 
$$ \free \xrightarrow{\free(\unit)} \free\forget\free
\xrightarrow{\counit} \free $$
evaluated at $E$ is a $\nodA$ morphism
$$ EA \xrightarrow{E \eta A} EA^2 \xrightarrow{E \mu_2} EA. $$
This morphism is homotopic to $\id_{EA}$ since 
$$  A \xrightarrow{\eta A} A^2 \xrightarrow{\mu_2} A $$
is homotopic to $\id_A$. It is in 
the definition of strong homotopy unitality of $A$.

Let $(E,p_\bullet) \in \nodhuA$. The composition 
$$ \forget \xrightarrow{\unit} \forget\free\forget
\xrightarrow{\forget(\counit)} \forget $$
evaluated at $(E,p_{\bullet})$ is an $\A$ morphism 
$$ E \xrightarrow{E\eta} EA \xrightarrow{p_2} E. $$
It being homotopic to $\id_E$ is the definition of 
homotopy unitality of $(E, p_\bullet)$. 
\end{proof}

\subsection{Kleisli category}
\label{section-kleisli-category}

Given an $\Ainfty$-algebra $A$ in a monoidal DG category $\A$ we can 
construct a (non-unital) $\Ainfty$-category in the sense of 
\cite[\S5]{Lefevre-SurLesAInftyCategories} in the same way 
as we construct the Kleisli category of a monad:

\begin{defn}
\label{defn-kleisli-category-of-an-ainfty-algebra}
Let $A$ be an $\Ainfty$-algebra in a monoidal DG category $\A$. Define
its \em Kleisli category \rm $\kleisliA$ to be the $\Ainfty$-category
(in the classical sense of \cite{Lefevre-SurLesAInftyCategories})
defined by the following data:
\begin{itemize}
\item Its objects are the objects of $\A$. 
\item For any $E,F \in \A$ the $\homm$-complex between them is
\begin{equation}
\homm_{\kleisliA}(E,F) := \homm_{\A}(E,FA).
\end{equation}
\item For any $E_1, E_2, \dots, E_{n+1} \in \A$ and any $\alpha_i \in 
\homm_{\kleisliA}(E_i, E_{i+1})$ define 
\begin{equation}
\label{eqn-defn-of-ainfty-structure-on-kleisli-category}
m_n^{\kleisliA}(\alpha_1, \dots, \alpha_n) := 
E_1 \xrightarrow{\alpha_1} E_2A \xrightarrow{\alpha_2 A} \dots
\xrightarrow{\alpha_n A^{n-1}} E_{n+1}A^n \xrightarrow{E_{n+1}m_n^A}
E_{n+1}A.
\end{equation}
\end{itemize}
\end{defn}

\begin{lemma}
\label{lemma-ainfty-category-kleisliA-is-well-defined}
The $\Ainfty$-category $\kleisliA$ in Definition
\ref{defn-kleisli-category-of-an-ainfty-algebra}
is well-defined. 
\end{lemma} 
\begin{proof}
We need to show that the operations $m_i^{\kleisliA}$
satisfy the equations
\eqref{eqn-defining-equalities-for-new-definition-of-Ainfinity-algebra}. 
These can be reduced to the same equations for $m_i^A$ as follows. 
Consider the equation
\begin{equation}
\label{eqn-dm3-equation-for-the-kleisli-category}
d m^{\kleisliA}_3 + m^{\kleisliA}_2 \circ \left( \id \otimes
m^{\kleisliA}_2 - m^{\kleisliA}_2 \otimes \id \right) = 0.  
\end{equation}
For any 
$$ E_1 \xrightarrow{\alpha_1} E_2 \xrightarrow{\alpha_2} E_3
\xrightarrow{\alpha_3} E_4 \quad \in \quad \kleisliA $$
the morphism $m^{\kleisliA}_2 \circ
\left( \id \otimes m^{\kleisliA}_2\right) (\alpha_3,
\alpha_2, \alpha_1)$ in $\kleisliA$ is defined by the composition in $\A$ which
forms the upper right perimeter of the following commutative diagram:
\begin{equation}
\begin{tikzcd}[column sep = 2cm]
E_1
\ar{r}{\alpha_1}
&
E_2A
\ar{r}{\alpha_2 A}
&
E_3A^2
\ar{r}{E_3 m^A_2}
\ar{d}{\alpha_3 A^2}
&
E_3 A
\ar{d}{\alpha_3 A}
\\
& 
&
E_4 A^3
\ar{r}{E_4 A m^A_2}
&
E_4 A^2
\ar{d}{E_4 m^A_2}
\\
& & &
E_4 A.
\end{tikzcd}
\end{equation}
The diagram commutes by the bifunctoriality of the monoidal operation
of $\A$, thus we conclude that 
$$ 
m^{\kleisliA}_2 \circ \left( \id \otimes m^{\kleisliA}_2\right) 
(\alpha_3, \alpha_2, \alpha_1)
=  
\id_{E_4} \otimes \left(m^A_2 \circ \left( \id \otimes
m^A_2\right)\right)
\circ \left(\alpha_3 A^2 \circ \alpha_2 A \circ \alpha_1\right). 
$$
Similarly, we have 
$$
m^{\kleisliA}_2 \circ \left( m^{\kleisliA}_2  \otimes \id \right) 
(\alpha_3, \alpha_2, \alpha_1)
=  
\id_{E_4} \otimes \left(
m^A_2 \circ \left( m^A_2 \otimes \id \right)\right) 
\circ \left(\alpha_3 A^2 \circ \alpha_2 A \circ \alpha_1\right), 
$$
and 
$$ 
dm^{\kleisliA}_3 (\alpha_3, \alpha_2, \alpha_1) = 
(\id_{E_4} \otimes dm^{\A}_3) \circ (\alpha_3 A^2 \circ \alpha_2 A
\circ \alpha_1). 
$$
We conclude that \eqref{eqn-dm3-equation-for-the-kleisli-category} applied 
to $(\alpha_3, \alpha_2, \alpha_1)$ is $\id_{E_4}$ tensored
with the analogous equation for $\A$ and pre-composed with
$\alpha_3 A^2 \circ \alpha_2 A \circ \alpha_1$. 

The remaining equations 
\eqref{eqn-defining-equalities-for-new-definition-of-Ainfinity-algebra}
reduce similarly to those for $m^A_i$. 
\end{proof}

When $A$ is homotopy unital, so is clearly $\kleisliA$:

\begin{defn}
Let $A$ be a homotopy unital $\Ainfty$-algebra in a monoidal DG
category $\A$. Define a homotopy unital structure on $\kleisliA$ by setting
for any $E \in \A$ the unit morphism 
\begin{equation}
\id_E \in \homm_{\kleisliA}(E,E) 
\end{equation}
to be the morphism corresponding to the morphism
$E\eta\colon E \rightarrow EA$ in $\A$. Here $\eta$ is the unit
morphism $\id_\A \rightarrow A$ of $A$. 
\end{defn}

When $A$ is strongly homotopy unital, the Free-Forgetful adjunction
ensures that $\kleisliA$ is quasi-equivalent to the DG category 
$\freeA$:

\begin{defn}
\label{defn-kleisli-to-free-Ainfty-functor}
Let $A$ be an $\Ainfty$-algebra in a monoidal DG category $\A$. 
Define an $\Ainfty$-functor
$$ f_\bullet\colon \kleisliA \rightarrow \freeA $$
by setting 
$$ f_{\text{obj}}\colon \obj(\kleisliA) \rightarrow \obj(\freeA)$$
be the map $E \mapsto EA$ and by setting 
$$ f_i: \homm_{\kleisliA}(E_i, E_{i+1}) \otimes_k 
\dots \otimes_k \homm_{\kleisliA}(E_1 , E_2) \rightarrow
\homm_{\freeA}(E_1A, E_{i+1}A) $$
to be the map which sends any $\alpha_i \otimes \dots \otimes
\alpha_1$ to the following morphism in $\nodA$: 
$$ E_1 A \xrightarrow{\alpha_1A} E_2 A^2 \xrightarrow{\alpha_2A^2}
\dots \xrightarrow{\alpha_i A^i} E_{i+1} A^{i+1} \xrightarrow{E_{i+1} \mu_{i+1}}
E_{i+1} A. $$
\end{defn}

\begin{theorem}
\label{theorem-ainfty-quasi-equivalence-from-kleisli-to-free}
The $\Ainfty$-functor 
$ f_\bullet\colon \kleisliA \rightarrow \freeA $
of Defn.~\ref{defn-kleisli-to-free-Ainfty-functor} is
well-defined. If $A$ is strongly homotopy unital, it is a quasi-equivalence. 
\end{theorem}
\begin{proof}
To show that $f_\bullet$ is well defined we need to show that its
bar-construction $\infbar(f_\bullet)$ is a closed morphism of twisted
complexes. By
Prps.~\ref{prps-defining-equalities-of-Ainfty-morphism}
it suffices to show that the equations
\eqref{eqn-defining-equalities-for-new-definition-of-Ainfinity-algebra-morphism}
hold for $f_\bullet$. Similarly to the proof of Lemma
\ref{lemma-ainfty-category-kleisliA-is-well-defined}, the $i$-th equation 
\eqref{eqn-defining-equalities-for-new-definition-of-Ainfinity-algebra-morphism}
for $f_\bullet$ reduces to the 
$(i+1)$-st equation for $m^A_{\bullet}$. To be more precise, it
reduces to the $\nodA$-lift of this equation which is obtained
by replacing the twisted complex $\infbarnaug(A)$ in $\A$ with 
the twisted complex $\infbar(A)$ in $\nodA$ as described in 
Prps.~\ref{prps-bar-construction-as-a-complex-of-Ainfty-modules}. 

To show that $f_\bullet$ is a quasi-equivalence, we need to show that
$f_{obj}$ is quasi-essentially surjective and $f_1$ is a quasi-isomorphism 
on all $\homm$-complexes. For the former, $f_{obj}$ is actually
bijective. For the latter, under the identification 
of $\homm_{\kleisliA}(E,F)$ with $\homm_{\A}(E, FA)$, the map
\begin{equation}
\label{eqn-f_1-on-homm-spaces-from-A-to-free-A}
f_1\colon \homm_{\A}(E, FA) \rightarrow \homm_{\freeA}(EA,FA)
\end{equation}
sends any $\alpha\colon E \rightarrow FA$ to the composition 
$ EA \xrightarrow{\alpha A} FA^2 \xrightarrow{F\mu_2} FA. $
Its first composant is the functor $\free$ applied to $\alpha$ and
the second is the counit of the Free-Forgetful adjunction. Thus 
\eqref{eqn-f_1-on-homm-spaces-from-A-to-free-A} is the Free-Forgetful
adjunction map. By Theorem \ref{theorem-free-forgetful-homotopy-adjunction}
it is a quasi-isomorphism since $A$ is strongly homotopy unital. 
\end{proof}

\section{The derived category}
\label{section-the-derived-category}

So far, the results of this paper were phrased in terms of a
small monoidal DG category $\A$ and its Yoneda embedding into $\modA$. 
Recall, as explained in \S\ref{section-the-setting}, that they
work just as well for a non-small $\A$ if we replace $\modA$ 
with a cocomplete closed monoidal DG category $\B$ admits convolutions of twisted complexes. 
We made a conscious choice to write down the previous sections in
terms of $\modA$, instead of a general $\B$, in order to make it more
familiar for the reader. We could do that, because any particulars 
specific to $\modA$ and the Yoneda embedding were not relevant 
to our considerations so far.  

In this section, this is no longer the case. To describe the compact
objects in the derived category $D(A)$ we need to look at 
those objects of $\A$ which are compact in $H^0(\B)$. For a small $\A$ 
and $\B = \modA$ this is the whole of $\A$, however we also want to
consider  the case when $\A = \modC$ for a small DG category $\C$, and 
$\B = \A$. Thus we need to consider the case where not 
all objects of $\A$ are compact in $H^0(\B)$, which doesn't make sense 
when $\B = \modA$. 

Therefore, we switch our exposition style in this section to work with 
a general ambient category $\B$. We assume as before that it is
closed monoidal, cocomplete, and admits convolutions of twisted complexes, 
and assume that $\A$ is its full monoidal subcategory. 
We make a further assumption:
 $\A$ contains a set of compact objects which generate it in $H^0(\B)$. 
Then by \cite[Theorem 5.3]{Keller-DerivingDGCategories} all compact objects 
of $H^0(\A)$ also form a set, since they lie in the Karoubi-completion 
of the triangulated hull of the set of generators. Note, that when 
$\A$ is small all these assumptions are satisfied for $\B = \modA$. 

To consider $\Ainfty$-structures in $\B$, we set the ambient category 
of $\B$ to be $\B$ itself. Any $\Ainfty$-algebra $(A, \mu_i)$ in $\A$ is 
then also an $\Ainfty$-algebra in $\B$. We then write e.g.
$\noddinf\text{-}A^\B$ for the category of right $\Ainfty$-$A$-modules
in $\B$ and $\noddinfhu\text{-}A^\B$ for its full subcategory
consisting of $H$-unital modules. 

\subsection{The general case}
\label{section-the-derived-category-the-general-case}

If $A$ is a unital algebra over $k$, then the unbounded derived 
category $D(A)$ of $A$ is the category of unbounded complexes of $A$-modules 
localised by quasi-isomorphisms. One can show that the smallest cocomplete  
triangulated subcategory of $D(A)$ containing free module $A$ is the whole
of $D(A)$. 

Complexes of $A$-modules are DG $A$-modules. The category
$\modd\text{-}A$ of DG modules over $A$ is pre-triangulated, 
and the localisation by quasi-isomorphisms is achieved by taking
the Verdier quotient by $\acyc$, the subcategory of acyclic
modules. Thus when $A$ is a DG-algebra we set $D(A)$ 
to be $H^0(\modd\text{-}A)/\acyc$. Again, $D(A)$ is the cocomplete
triangulated hull of the free module $A$. 

Let $A$ be a classical $\Ainfty$-algebra considered in 
\cite[\S1]{Lefevre-SurLesAInftyCategories}. In our more general
setting this corresponds to the case $\A = \modk$ with the monoidal
structure given by tensor product of complexes. An elegant approach to
defining the derived category is given in
\cite[\S4]{Lefevre-SurLesAInftyCategories}. First, for $A$ strictly
unital, we define $D(A)$ to be $H^0(\moddinf\text{-}A)$, where 
$\modd_\infty\text{-}A$ is the full subcategory of $\nodA$ comprising
strictly unital $A$-modules. Note that there is no need to quotient
out the acyclics, as the Homotopy Lemma ensures that all
quasi-isomorphisms are already homotopy equivalences. 
Then, for a general non-unital $A$, we consider its augmentation $A^+
:= A \oplus k$ to a strictly unital algebra. The functor 
$(-) \inftimes_{A^+} k$ of extension of scalars along 
the augmentation projection  $A^+ \rightarrow k$ 
induces the exact functor
\begin{equation}
D(A^+) \xrightarrow{(-) \inftimes_{A^+} k} D(k).
\end{equation}
The derived category $D(A)$ is then defined to be the kernel of this
morphism. The logic is that we want to split $D(A^+)$ into $D(A)$ 
and the derived category $D(k)$ of the augmented unit. 

Indeed, the restriction of scalars gives a natural isomorphism from 
$\moddinf\text{-}A^+$ to $\nodA$. The functor $(-) \inftimes_{A^+} k$
can be readily seen to send any $(E,p_\bullet) \in
\moddinf\text{-}A^+$ to its bar-construction as an $A$-module. Thus
$D(A)$, its kernel, can be identified with $H^0(\nodhuA)$, where
$\nodhuA$ is the full subcategory of $\nodA$ consisting of
$H$-unital modules. Moreover, the functor of restriction of scalars 
along $A^+ \rightarrow k$ is quasi-fully faithful if and only if $A$
is $H$-unital. Since it is the right adjoint of 
$(-) \inftimes_{A^+} k$, it follows that when $A$ is $H$-unital
we have a canonical semi-orthogonal decomposition 
\begin{equation}
D(A^+) = \left<D(k), D(A)\right>. 
\end{equation}

Most of these considerations extend readily to our more general setting of
$\A$ being an arbitary DG monoidal category. We use the 
obvious generalisation of $\Ainfty$-tensor product to our setting, 
and we need $\A$ to admit finite direct sums in order for us 
to be able to augment $A$ with the unit $\id_\A$ of $\A$. 
 
However, apriori $H^0(\nodA)$ and $H^0(\nodhuA)$ are neither triangulated nor
cocomplete. Indeed, 
by \cite{AnnoLogvinenko-UnboundedTwistedComplexes}, Cor.~5.13,
$\nodA$ is pretriangulated if and only if $\A$ is. It is 
cocomplete if $\A$ is and its monoidal 
structure commutes with small direct sums. 
Since $H$-unitality is stable under taking cones, shifts, and 
small direct sums, $\nodhuA$ is also pretriangulated 
and cocomplete if $\A$ is. 

\begin{defn}
\label{defn-triangulated-cocomlete-hull}
Let $\T$ be a cocomplete triangulated category and let $W \subset
\mathcal{T}$ be a subset of its objects. The \em triangulated
cocomplete hull \rm $\left< E \right>_{tr,cc}$ of $W$ is the smallest
cocomplete triangulated subcategory of $\T$ containing $W$. 
\end{defn}

We would like $D(A)$ to be a triangulated, cocomplete hull of $H^0(\nodhuA)$.
Taking triangulated hull is canonical on DG level, in the sense that
it is independent of the ambient category. The completion with respect
to small direct sums is not, as explained in 
\S\ref{section-unbounded-twisted-complexes}. 
Fortunately, having fixed a monoidal embedding of $\A$ into a cocomplete 
strongly pretriangulated closed monoidal category $\B$ 
we have the induced embedding $\nodA \hookrightarrow \noddinf\text{-}A^\B$. 
The latter is cocomplete since the monoidal structure of $\B$ commutes with small direct sums and strongly pretriangulated since $\B$ is.
We thus define:

\begin{defn}
\label{defn-the-derived-category-of-an-ainfty-algebra}
Let $A$ be an $\Ainfty$-algebra in a monoidal DG category $\A$.
The \em (unbounded) derived category \em $D(A)$ of $A$ is the triangulated 
cocomplete hull of $H^0(\nodhuA)$ in $H^0(\noddinf\text{-}A^\B)$. 
The \em compact derived category \rm $D_c(A)$ is the full subcategory 
of $D(A)$ comprising its compact objects. 
\end{defn}

We take unbounded twisted complexes over $\nodA$ relative to 
$\noddinf\text{-}A^\B$. Since $\noddinf\text{-}A^\B$ is cocomplete
and closed under convolutions of bounded twisted complexes, it is
closed under convolutions of one-sided twisted complexes bounded from 
above or below.  

Note that taking small direct sums and convolutions of bounded 
twisted complexes preserves $H$-unitality. Thus
$H^0(\noddinfhu\text{-}A^\B)$ is triangulated and cocomplete. 
As it contains $\nodhuA$, it therefore also contains $D(A)$. 

\begin{defn}
Let $A$ be an $\Ainfty$-algebra in a monoidal DG category $\A$.
A module $(E, p_\bullet) \in \nodhuA$ is \em perfect \rm 
if its image in $D(A)$ is a compact object. Let $\nodhupfA$
denote the full subcategory of $\nodhuA$ comprising perfect modules. 
\end{defn}

In this generality we can't say much else, e.g. we don't even
know whether free $A$-modules lie in $D(A)$. 

\subsection{$H$-unital case}
\label{section-the-derived-category-the-h-unital-case}

When $A$ is $H$-unital, we can describe its derived category 
in terms of free modules. 

We introduce some notation first: 
given any subset $C$ of the objects of $\B$ we denote by 
$\noddinfhu\text{-}A^C$ the full subcategory of
$H^0(\noddinfhu\text{-}A^\B)$ consisting of modules whose
underlying objects of $\B$ belong to $C$.

\begin{prps}
\label{prps-derived-category-of-A-is-triangulated-cocomplete-hull-of-freeA}
Let $A$ be an $H$-unital $\Ainfty$-algebra in a monoidal DG
category $\A$. Then 
$$ \left< \freeA \right>_{tr,cc} = D(A) =
H^0(\noddinfhu\text{-}A^{\left<\A\right>_{tr,cc}}). $$
\end{prps}
\begin{proof}
For any $E \in \A$ we have 
$$ \infbar(EA) = E \otimes \infbarnaug(A).$$
Since $A$ is $H$-unital, $\infbarnaug(A)$ is null-homotopic, 
and hence so is $\infbar(EA)$. Thus $\free(A)$ is contained in 
$\nodhuA$ and therefore in $\noddinfhu\text{-}A^{\left<\A\right>_{tr,cc}}$. 
Taking 
triangulated cocomplete hulls we obtain the inclusions
$$ \left< \freeA \right>_{tr,cc} \subseteq D(A) \subseteq
H^0(\noddinfhu\text{-}A^{\left<\A\right>_{tr,cc}}). $$

Conversely, any $(E,p_{\bullet}) \in \noddinfhu\text{-}A^{\left<\A\right>_{tr,cc}}$
is homotopy equivalent to its bar-resolution
$\infbarres(E,p_{\bullet})$ by
Prop.~\ref{prps-natural-transformation-rho-is-acyclic-on-H-unital-modules-Ainfty-version}.
The latter is a bounded above one-sided twisted complex of free modules $EA^i$.
Since the object $E$ lies in $\left<\A\right>_{tr,cc}$, so does each 
object $EA^{i-1}$ for any $i \geq 1$. Each free module $EA^i$ lies
therefore in $\free\left(\left<\A\right>_{tr,cc}\right)$ and so in 
$\left< \freeA \right>_{tr,cc}$. Hence the convolution of a bounded
above one-sided twisted complex of $EA^i$ also lies in 
$\left< \freeA \right>_{tr,cc}$. Thus the inclusions above are
equalities. 
\end{proof}

We can make the following definition:

\begin{defn}
Let $A$ be an $H$-unital $\Ainfty$-algebra in a monoidal DG
category $\A$. Let $\freepfA$ be the full subcategory of $\freeA$
comprising perfect modules. 
\end{defn}

However, to give similar descriptions of $D(A)$ and $D_c(A)$ in terms
of $\freepfA$ we need to know whether the perfect free $A$-modules 
generate everything. We assumed that $\A$ is generated by its compact 
objects in $H^0(\B)$, but at the present level of generality we lack 
the means necessary to relate compactness of $\Ainfty$-$A$-modules to 
the compactness of their underlying objects of $\A$. 

\subsection{Strongly homotopy unital case}
\label{section-the-derived-category-the-strong-homotopy-unital-case}

When $A$ is strongly homotopy unital, we have the Free-Forgetful
homotopy adjunction at our disposal and can achieve 
our best description of the derived category of $A$. 

\begin{defn}
An object $E \in \A$ is \em perfect \rm if it is a compact object 
in $H^0(\B)$. We denote by $\Aperf$ the full subcategory of $\A$
comprising the perfect objects.  
\end{defn}

Recall that by our assumptions, the compact objects of $\A$ form a
set, so $\Aperf$ is a small category. 

\begin{lemma}
\label{lemma-perfect-in-nodhuaA-iff-perfect-in-A}
Let $A$ be a strongly homotopy unital $\Ainfty$-algebra in 
a monoidal DG category $\A$. For any $E \in \A$, the free module 
$EA \in \nodhuA$ is perfect if and only if $E \in \A$ is perfect.
\end{lemma}
\begin{proof}
As $A$ is strongly homotopy unital as an $\Ainfty$-algebra in $\A$, 
it is also strongly homotopy unital as a one in $\B$. The Free-Forgetful
homotopy adjunction of Theorem \ref{theorem-free-forgetful-homotopy-adjunction}
gives a genuine adjunction on the level of the homotopy categories:
$$ \homm_{H^0(\noddinfhu\text{-}A^{\B})}(EA, -) \simeq 
\homm_{H^0(\B)}(E, \forget(-)). $$

By definition, $D(A)$ is a full subcategory of
$H^0(\noddinfhu\text{-}A^{\B})$ and, as $A$ is strongly homotopy unital, 
the free functor filters through this full subcategory. Since the 
forgetful functor commutes with small direct sums, it follows that 
$\homm_{D(A)}(EA,-)$ commutes with them if and only if
$\homm_{H^0(\B)}(E,-)$ does. 
\end{proof}
 
Let now $S \subset \Aperf$ be any set of generators of $H^0(\A)$ in 
$H^0(\B)$. By this we mean that the cocomplete triangulated hull of 
$S$ in $H^0(\B)$ contains $H^0(\A)$. Let us denote by $\freeAS$
the full subcategory of $\freeA$ comprising the objects of $\free(S)$. 
In other words, it is the full subcategory of $\nodA$ comprising
the free modules on objects $S$. 
  
\begin{prps}
\label{prps-perfect-iff-lies-in-hperf-of-perfect-generator-frees}
Let $A$ be a strongly homotopy unital $\Ainfty$-algebra in 
a monoidal DG category $\A$. Let $(E,p_\bullet) \in \nodhuA$. 
Let $S$ be a set of compact generators of $H^0(\A)$ in $H^0(\B)$. 
Then $\freeAS$ is a set of compact generators of $D(A)$ and
the following are equivalent:
\begin{enumerate}
\item 
\label{item-A-infinity-module-is-perfect}
$(E,p_\bullet)$ is perfect. 
\item 
\label{item-A-infinity-module-lies-in-hperf-of-perfect-generator-frees}
$(E,p_\bullet)$ is a homotopy direct summand of something in 
$\pretriag(\freeAS))$. 
\end{enumerate}
If these equivalent conditions hold, then furthermore $(E,p_\bullet)$
is a homotopy direct summand of a finite truncation of its bar resolution
$\infbarres(E,p_\bullet)$. 
\end{prps}
\begin{proof}

Since $S$ generates $\A$ in $H^0(\B)$. 
It follows that $\free(S)$ generates $\free(\A)$ in 
$H^0(\noddinf\text{-}A^{\B})$. The objects of $\freeA$ and $\freeAS$
are the same as those of $\free(\A)$ and $\free(S)$, and 
$\freeA$ generates $D(A)$ by Prop.~\ref{prps-derived-category-of-A-is-triangulated-cocomplete-hull-of-freeA}. Thus $\freeAS$ is a set of compact
generators of $D(A)$. 

$\eqref{item-A-infinity-module-is-perfect} \Rightarrow
\eqref{item-A-infinity-module-lies-in-hperf-of-perfect-generator-frees}$:

By \cite[Theorem 5.3]{Keller-DerivingDGCategories} the compact objects
of $D(A)$ lie in the Karoubi completion of the triangulated hull of 
$\freeAS$ in $D(A)$. Hence any perfect $(E,p_\bullet)$ is a homotopy
direct summand of something in $\pretriag(\freeAS)$. 

$\eqref{item-A-infinity-module-lies-in-hperf-of-perfect-generator-frees}
\Rightarrow \eqref{item-A-infinity-module-is-perfect}$:

By Lemma \ref{lemma-perfect-in-nodhuaA-iff-perfect-in-A} the objects
of $\freeAS$ are perfect in $\nodhuA$. If $(E,p_\bullet)$ is a homotopy 
direct summand of something in $\pretriag(\freeAS))$, its image in $D(A)$ 
lies in the Karoubi completion of the triangulated hull of $\freeAS$. 
It is therefore also compact, since both taking triangulated hull and 
taking Karoubi completion preserves compactness. Hence $(E,p_\bullet)$ is
perfect. 

The final assertion that $(E,p_\bullet)$
is a homotopy direct summand of a finite truncation of its bar resolution
$\infbarres(E,p_\bullet)$ follows by the same argument in the
proof of \cite[Theorem 5.3]{Keller-DerivingDGCategories}. That
proof uses arbitrary resolution of a compact object, and the bar resolution 
is an instance of one. 
\end{proof}

Recall that in \S\ref{section-kleisli-category} we have defined the
Kleisli category $\kleisliA$ of $A$ which is an $\Ainfty$-category in 
the classical sense \cite{Lefevre-SurLesAInftyCategories}. Let 
$\kleisliAS$ denote its full subcategory comprising the objects of
$S$ and let $D_c(\kleisliAS)$ denote its compact derived category 
in the usual sense. We have:
\begin{theorem}
\label{theorem-compact-derived-category-of-A-is-that-of-frees-and-kleisli}
Let $A$ be a strongly homotopy unital $\Ainfty$-algebra in a monoidal 
DG category $\A$. Let $S$ be a set of compact generators of $H^0(\A)$ 
in $H^0(\B)$. Then 
$$ D_c(A) \simeq D_c(\freeAS) \simeq D_c(\kleisliAS). $$
\end{theorem}

Note that we can take $S$ to be the set of all compact objects in
$H^0(\A)$, as by assumption these generate $H^0(\A)$. 
\begin{proof}
The second equivalence $D_c(\freeAS) \simeq D_c(\kleisliAS)$ is induced 
by the $\Ainfty$-quasi-equivalence $\kleisliAS \rightarrow \freeAS$
constructed in Theorem
\ref{theorem-ainfty-quasi-equivalence-from-kleisli-to-free}. 
 
To construct the first equivalence, we note that 
$$D_c(\freeAS) = H^0(\hperf(\freeAS)).$$ 
By the universal property of $\hperf$ the fully faithful embedding 
$$\freeAS \hookrightarrow \noddinf\text{-}A^{\B}$$
induces a fully faithful embedding 
$$ \hperf(\freeAS) \hookrightarrow \noddinf\text{-}A^{\B}$$
since the target category is pre-triangulated and homotopy Karoubi complete. 
On the homotopy level we have therefore a fully faithful embedding
$$ D_c(\freeAS) \hookrightarrow D(A) $$
whose essential image is the Karoubi complete triangulated hull of 
$\freeAS$. Since the proof of  
Prop.~\ref{prps-perfect-iff-lies-in-hperf-of-perfect-generator-frees}
showed that $D_c(A)$ is the Karoubi complete triangulated hull
of $\freeAS$ in $D(A)$, we conclude that the embedding above restricts
to an equivalence $D_c(\freeAS) \simeq D_c(A)$.   
\end{proof}

Theorem \ref{theorem-compact-derived-category-of-A-is-that-of-frees-and-kleisli}
shows that, unsurprisingly, the $D_c(A)$ is 
independent of the choice of the ambient category $\B$ where 
we take infinite direct sums. Since this is clearly not true for
$D(A)$, we can't hope to relate it to $D(\freeAS) \simeq D(\kleisliAS)$ for a general $\B$. However, we can when $\A$ is small
and $\B = \modA$:

\begin{prps}
\label{prps-when-B-is-modA-we-get-the-classical-derived-category}
Let $A$ be a strongly homotopy unital $\Ainfty$-algebra in a small monoidal 
DG category $\A$. Let $\B = \modA$ with the induced monoidal
structure. 
$$ D(A) \simeq D(\freeA) \simeq D(\kleisliA). $$
\end{prps}
\begin{proof}
As before, by Theorem
\ref{theorem-ainfty-quasi-equivalence-from-kleisli-to-free} 
we have $D(\freeA) \simeq D(\kleisliA)$. 
Thus it suffices to demonstrate that $D(A) \simeq D(\kleisliA)$.

We first construct a functor
$$ f_\bullet\colon \noddinf\text{-}A^{\modA} \rightarrow
\noddinf\text{-}\kleisliA. $$
The monoidal structure induced on $\modA$ from $\A$, cf. 
\cite[\S4.5]{GyengeKoppensteinerLogvinenko-TheHeisenbergCategoryOfACategory}, 
has the property that for any $E \in \modA$ and $a \in \A$ the product
$E \homm_\A(-,a)$ is the $\A$-module 
$E \otimes_\A \homm_\A(-,-a),$
where $ \homm_\A(-,-a)$ is viewed as $\A$-$\A$-bimodule. Thus 
any $(E,p_\bullet) \in \noddinf\text{-}A^{\modA}$ is the data of
$\A$-module $E$ and structure morphisms
$$ p_i\colon  E \otimes_\A \homm_\A(-,-A) \otimes_\A \dots \otimes_\A \homm_\A(-,-A) \rightarrow E. $$
Furthermore, taking product in $\modA$ with 
the operations $m_i\colon A^i \rightarrow A$ corresponds to tensoring
over $\A$ with the operations defined 
in \eqref{eqn-defn-of-ainfty-structure-on-kleisli-category}
$$ m_i\colon \homm_\A(-,-A) \otimes_\A \dots \otimes_\A \homm_\A(-,-A)
\rightarrow \homm_\A(-,-A). $$
These are the $\Ainfty$-operations of $\kleisliA$ under the identification  
$$ \homm_\A(-,-A) \simeq \homm_{\kleisliA}(-,-). $$
In other words, $\noddinf\text{-}A^{\modA}$ is isomorphic to 
the category of $\Ainfty$-modules in $\modA$ over bimodule $\Ainfty$-algebra
$\homm_\A(-,-A)$ in $\AmodA$. 

We now observe that the forgetful functors 
$$ \forget\colon \AmodA \rightarrow k_\A\text{-}\modd\text{-}k_\A, $$
$$ \forget\colon \modA \rightarrow \modd\text{-}k_\A, $$
are strict monoidal with respect to $\otimes_\A$ and $\otimes_k$ and 
commute with infinite direct sums.  By definition, 
$\noddinf\text{-}\kleisliA$ is the category of $\Ainfty$-modules in 
$\modd\text{-}k_\A$ over the $\Ainfty$-algebra $\kleisliA$ in 
$k_\A\text{-}\modd\text{-}k_\A$. This algebra is the image under $\forget$ 
of the bimodule $\Ainfty$-algebra $\homm_\A(-,-A)$ in $\AmodA$. 
Hence the forgetful functors induce a functor
$$  \forget_\bullet\colon \noddinf\text{-}A^{\modA} \rightarrow
\noddinf\text{-}\kleisliA,$$ 
which commutes with infinite direct sums. 

We now claim that $\forget_\bullet$ is quasi-fully faithful on $\freeA$.
Indeed, the following diagram commutes:
\begin{equation}
\begin{tikzcd}[column sep = 2cm]
& 
\kleisliA 
\ar[hook]{d}{\text{Yoneda}}
\ar{ld}[']{\text{Theorem
}\ref{theorem-ainfty-quasi-equivalence-from-kleisli-to-free}}
\\
\freeA 
\ar{r}[']{\forget_\bullet}
&
\noddinf\text{-}\kleisliA, 
\end{tikzcd}
\end{equation}
where the Yoneda embedding of a (classical) $\Ainfty$-category 
is as in \cite[\S7]{Lefevre-SurLesAInftyCategories}. 
The functor constructed in Theorem
\ref{theorem-ainfty-quasi-equivalence-from-kleisli-to-free} is a
quasi-equivalence since $A$ is strongly homotopy unital. So to show
that $\forget_\bullet$ is quasi-faithful on $\freeA$ it suffices to show
that the Yoneda embedding is quasi-fully faithful. This holds because
translated into our framework, $\Ainfty$-category $\kleisliA$ is the 
Kleisli category of the $\Ainfty$-algebra $\kleisliA$ in
$k_\A\text{-}\modd\text{-}k_\A$. The Yoneda embedding is then
the composition of the $\Ainfty$-functor of Theorem
\ref{theorem-ainfty-quasi-equivalence-from-kleisli-to-free} 
from Kleisli category to the free $\Ainfty$-modules with the inclusion of
these into all $\Ainfty$-modules. It is therefore quasi-fully faithful 
since $A$, and hence $\kleisliA$, are strongly homotopy unital. 

Thus $\forget_\bullet$ gives an exact functor
$H^0(\noddinf\text{-}A^{\modA}) \rightarrow
H^0(\noddinf\text{-}\kleisliA)$ which is fully faithful on $\freeA$,
sends $\freeA$ to $\kleisliA$, and commutes with infinite direct sums. 
It gives therefore an equivalence of cocomplete triangulated hulls of 
$\freeA$ and $\kleisliA$ which are $D(A)$ and $D(\kleisliA)$, respectively. 
\end{proof}

Another common case worth considering separately is when $\B = \A$:
\begin{prps}
\label{prps-when-B-is-A-we-get-the-???}
Let $A$ be a strongly homotopy unital $\Ainfty$-algebra in a monoidal 
DG category $\A$. Suppose that $\A$ is cocomplete, strongly
pre-triangulated, its monoidal structure is closed, 
and $H^0(\A)$ is compactly generated. In other words, it
satisfies our assumptions on $\B$, and we set $\B = \A$.
Then 
\begin{align*}
D(A) &=  H^0(\nodhuA) \\
D_c(A) &\subseteq 
H^0(\noddinfhu\text{-}A^{\A^{pf}})
\end{align*}
\end{prps}
\begin{remark}
\label{remark-not-all-perfect-object-modules-are-perfect}
The inclusion in the second assertion can be strict. A good
example is the skyscraper sheaf of a singular point on an affine
variety. To fit this into our framework, we set $k = \mathbb{C}$, 
$\A = \modk$, $A = k[x^2, xy, y^2]$, and consider the strict $A$-module 
$A/(x^2, xy, y^2)$. 
Its underlying object of $\modk$ is $k$ concentrated in degree $0$ which 
is certainly compact in $D(k)$. Thus it lies in
$H^0(\noddinfhu\text{-}A^{\A^{pf}})$. On the other hand, as explained
in detail in 
\S\ref{section-examples-associative-algebras}, the derived category 
of $A$ as an $\Ainfty$-algebra in $\modk$ coincides with the usual
derived category of $A$ as an associative algebra. Thus 
$A/(x^2, xy, y^2)$ is not compact object in $D(A)$. 
\end{remark}

\begin{proof}
The first assertion follows from the second equality in
Prop.~\ref{prps-derived-category-of-A-is-triangulated-cocomplete-hull-of-freeA}.

For the compact derived category, since $A$ is strong homotopy unital
the modules in $\free\text{-}(\A^{pf})$ are perfect. 
By Prop.~\ref{prps-perfect-iff-lies-in-hperf-of-perfect-generator-frees} every 
perfect module in $\nodhuA$ is a homotopy direct summand of a bounded twisted
complex of $\free\text{-}(\A^{pf})$. Therefore its underlying object
of $\A$ is a homotopy direct summand of a bounded twisted complex of
perfect objects and thus perfect itself. 
\end{proof}

\subsection{Strict algebra case}
\label{section-the-derived-category-strict-algebra-case}

In \S\ref{section-the-derived-category-the-strong-homotopy-unital-case}
we have obtained our best descriptions of the derived category $D(A)$
and its compact part $D_c(A)$ for $A$ strongly homotopy unital. We
believe it is the biggest generality in which such descriptions exist. 
In this section we examine the case when $A$ is furthermore a strict algebra. 
In such case we can consider the category of strict modules over $A$:

\begin{defn}
\label{defn-strict-algebra-strict-module-categories}
Let $(A,\mu)$ be a strict algebra in a monoidal DG category $\A$. 
Define $\nodstrA$ to be the subcategory of $\nodA$ comprising 
strict $A$-modules with strict morphisms between them whose
differentials are also strict. Define $\nodstrhuA$ to be 
the full subcategory of $\nodstrA$ comprising the $H$-unital modules. 
If $A$ is strictly unital, denote by $\modd\text{-}A$ the full
subcategory of $\nodstrhuA$ comprising strictly unital modules. 
\end{defn}

The objects of $\nodstrA$ are $A$-modules in 
the usual sense: pairs $(E,p)$ of an object $E \in \A$ and 
a structure morphism $p\colon EA \rightarrow A$ of $\A$ 
satisfying $p\circ (pA - E\mu) = 0$. A morphism $(E,p) \rightarrow
(F,q)$ is a morphism $E \rightarrow F$ in $\A$ which commutes with
structure morphisms. The composition and the differentials are those 
of $\A$.  

By definition, the functor $\free\colon \A \rightarrow \nodhuA$ filters 
through $\nodstrhuA$. For any $(E,p) \in \nodstrhuA$ 
the adjunction counit $\free \circ \forget \rightarrow \id_{\nodhuA}$ is 
the strict morphism  $EA \rightarrow (E,p)$ given by $p$. We now
verify that the Free-Forgetful homotopy adjunction of 
\S\ref{section-free-forgetful-homotopy-adjunction} restricts 
from $\nodhuA$ to its non-full subcategory $\nodstrhuA$:

\begin{lemma}
\label{lemma-free-forgetful-homotopy-adjunction-restricts-to-strict-modules}
Let $A$ be a strict, strongly homotopy unital algebra 
in a monoidal DG category $\A$. The functors 
$$ \free, \forget \colon \A \leftrightarrows \nodstrhuA $$
are homotopy adjoint with the unit is given by $E \xrightarrow{E\eta} EA$ 
for any $E \in \A$ and the counit by $EA \xrightarrow{p} (E,p)$. 
\end{lemma}
\begin{proof}
Let $E \in \A$. The composition 
$$ \free \xrightarrow{\free(\unit)} \free\forget\free
\xrightarrow{\counit} \free $$
evaluated at $E$ is a $\nodstrA$ morphism
$$ EA \xrightarrow{E \eta A} EA^2 \xrightarrow{E \mu} EA. $$
As $A$ is strongly homotopy unital, it is homotopy unital
and so $\mu \circ {\eta}A$ is homotopic to $\id_A$. Hence 
the composition above is homotopic to $\id_{EA}$. 

Let $(E,p) \in \nodstrhuA$. Since $A$ is strongly homotopy unital, 
$(E,p)$ is homotopy unital. The composition 
$$ \forget \xrightarrow{\unit} \forget\free\forget
\xrightarrow{\forget(\counit)} \forget $$
evaluated at $(E,p)$ is an $\A$ morphism 
$$ E \xrightarrow{E\eta} EA \xrightarrow{p} E. $$
It being homotopic to $\id_E$ is the definition of 
homotopy unitality of $(E, p)$. 
\end{proof}

Recall that we say that an object of $\nodA$ (resp. 
$\noddinfhu\text{-}A^{\B}$) is acyclic if its
underlying object of $\A$ (resp. $\B$) is null-homotopic. 
Given any subcategory $\C$ of $\noddinfhu\text{-}A^{\B}$, 
we denote by $\acyc_\C$ the full subcategory of $\C$ comprising
acyclic objects. Where no confusion is possible, we simply write
$\acyc$ for $\acyc_\C$. 

\begin{defn}
Let $A$ be a strict algebra in a monoidal DG category $\A$.
Let $(E,p) \in \nodstrA$. We say that $(E,p)$ is \em h-projective \rm 
if $\homm_{\nodstrA}^\bullet(E,F)$ is acyclic for any $(F,q) \in 
\acyc_{\nodstrA}$. 
\end{defn}
\begin{lemma}
Let $A$ be a strict, strongly homotopy unital algebra in a monoidal 
DG category $\A$. Any free $A$-module is $h$-projective. 
\end{lemma}
\begin{proof}
Let $E \in \A$. 
Let $(F,q) \in \nodstrA$ be an acyclic module, thus $F$ is
null-homotopic in $A$. By Free-Forgetful 
adjunction we have a quasi-isomorphism 
$$ \homm^\bullet_{\nodstrA}(EA, (F,q)) \rightarrow 
\homm^\bullet_{\A}(E, F). $$
The RHS is null-homotopic in $\modk$, and hence acyclic. 
\end{proof}

The bar-resolution $\infbarres$ resolves the objects and morphisms of 
$\nodhuA$ by those of $\nodstrhuA$. This allows us to describe 
the derived category $D(A)$ of $A$ as a Verdier quotient of $H^0(\nodstrhuA)$:
\begin{theorem}
\label{theorem-derived-category-as-localisation-in-the-strict-case}
Let $A$ be a strict, strongly homotopy unital algebra in a monoidal 
DG category $\A$. Then we have an exact equivalence
$$ D(A) \simeq H^0(\nodd^{hu}\text{-}A^{\left<\A\right>_{tr,cc}}) / \acyc. $$
\end{theorem}
\begin{proof}
The same argument using the bar-resolution as in 
Prop.~\ref{prps-derived-category-of-A-is-triangulated-cocomplete-hull-of-freeA}
shows that
$$ \left< \freeA \right>_{tr,cc}^{strict}
=
H^0(\nodd^{hu}\text{-}A^{\left<\A\right>_{tr,cc}}) , $$
where on the LHS we take the triangulated cocomplete hull 
in $H^0(\nodd\text{-}A^\B)$. 

Consider the non-full inclusion 
$$ 
\nodd^{hu}\text{-}A^{\left<\A\right>_{tr,cc}}
\hookrightarrow 
\noddinfhu\text{-}A^{\left<\A\right>_{tr,cc}}. $$
Note that it respects direct sums: in both categories  
$\bigoplus (E_i, p_i) = (\bigoplus E_i, \sum p_i)$. 
By Prop.~\ref{prps-derived-category-of-A-is-triangulated-cocomplete-hull-of-freeA} the homotopy category of the RHS is $D(A)$, so we obtain an exact
non-full inclusion of triangulated categories
$$
H^0(\nodd^{hu}\text{-}A^{\left<\A\right>_{tr,cc}})
\hookrightarrow 
D(A).
$$
The source category is cocomplete and the inclusion respects direct
sums, thus its essential image is triangulated and cocomplete.
As the image contains $\freeA$, it must be the whole of $D(A)$. 
By Homotopy Lemma, the inclusion kills all acyclic modules. It
therefore induces an essentially surjective exact functor 
$$
H^0(\nodd^{hu}\text{-}A^{\left<\A\right>_{tr,cc}})/\acyc
\longrightarrow 
D(A).
$$
To show it to be fully faithful, it is enough to show it on a set 
of compact generators. By above, the free modules generate
$H^0(\nodd^{hu}\text{-}A^{\left<\A\right>_{tr,cc}})$ and thus its
Verdier quotient. Since $\A^{pf}$ generates $\A$ in $H^0(\B)$, compact 
free modules generate all free modules in $H^0(\nodd\text{-}A^\B)$. Since 
free modules are $h$-projective, $\homm$-spaces between them in the
Verdier quotient are the same as in 
$H^0(\nodd^{hu}\text{-}A^{\left<\A\right>_{tr,cc}})$ itself. 

It suffices now to show that for any $E,F \in \A$ the natural inclusion 
$$\homm_{\nodstrA}(EA,FA)
\hookrightarrow \homm_{\nodA}(EA,FA)$$
is a quasi-isomorphism.  By Lemma 
\ref{lemma-free-forgetful-homotopy-adjunction-restricts-to-strict-modules}
the Free-Forgetful homotopy adjunction restricts from
$\Ainfty$-modules to strict modules. Hence it gives quasi-isomorphisms 
from $\homm_{\nodstrA}(EA,FA)$ and 
$\homm_{\nodA}(EA,FA)$
to $\homm_{\A}(EA,F)$
which intertwines their natural inclusion. 
\end{proof}

\section{$\Ainfty$-coalgebras and comodules} 
\label{section-ainfty-coalgebras-and-comodules}


Most of the definitions and results for $\Ainfty$-algebras, modules, 
and their derived categories given in
\S\ref{section-ainfty-structures-in-monoidal-dg-categories}-\S\ref{section-the-derived-category}
can be duplicated for $\Ainfty$-coalgebras and comodules. 
In this section, we give the first few as an example, 
and leave the rest to the reader.  

A major subtlety is that with $\Ainfty$-algebras and their modules 
the bar-constructions (see Definitions
\ref{defn-algebra-bar-construction-in-a-monoidal-category},
\ref{defn-the-bar-construction-of-a-morphism-of-ainfty-algebras},
\ref{defn-right-module-bar-construction-in-a-monoidal-category},
\ref{defn-left-module-bar-construction-in-a-monoidal-category},
\ref{defn-right-module-bar-constructions-of-an-Ainfty-morphism}, and
\ref{defn-left-module-bar-constructions-of-an-Ainfty-morphism})  
have only a finite number of arrows coming out of any term of the
complex. The sum of all the maps always gives a morphism between the direct
sums of the terms of the complexs. Thus the definitions of
$\Ainfty$-algebras, modules, and their morphisms are
independent of the ambient category $\B$ used to define 
our unbounded twisted complexes. 

This is no longer the case for the cobar-construction. The condition
that the cobar-construction is a twisted complex -- over $\B$ -- now
involves a non-trivial condition that the sum of all its maps filters 
through the infinite direct sum -- in $\B$ -- 
of the terms of the complex. Thus what is and isn't an
$\Ainfty$-coalgebra or an $\Ainfty$-comodule depends on the choice of 
the ambient category $\B$. 
In this section, we always assume that a choice of $\B$ satisfying 
the assumptions in \S\ref{section-the-setting} and 
\S\ref{section-the-derived-category} is fixed and
"twisted complex" means a twisted complex in $\twbicx^{\pm}_\B(\A)$.

A consequence of the above is that the Homotopy Lemma 
(Lemma \ref{lemma-the-homotopy-lemma-for-nodA}) doesn't work for
$\Ainfty$-comodules. The reason is that the contracting homotopy
our method constructs doesn't apriori satisfy the above condition 
that its cobar-construction is a valid map of twisted complexes in 
$\twbicx^{\pm}_\B(\A)$, that is -- the sum of all its terms filters 
in $\B$ through the infinite direct sum of the terms of the target
complex. We resolve some of the resulting complications by working
only with strongly homotopy unital modules. This, in turn, 
makes the notion of bicomodule homotopy unitality,
analogous to that of bimodule homotopy unitality for
$\Ainfty$-algebras, very relevant. 

\subsection{$\Ainfty$-coalgebras}
\label{section-Ainfty-coalgebras-in-a-monoidal-category}
 
As in \S\ref{section-Ainfty-algebras-in-a-monoidal-category}, 
we first define the notion of a cobar-construction of a
collection of operations, and then say that these operations define
an $\Ainfty$-coalgebra if the cobar-construction is a twisted complex:

\begin{defn}
\label{defn-coalgebra-cobar-construction-in-a-monoidal-category}
Let $\A$ be a monoidal DG category, let $C \in \A$ and let 
$\left\{\Delta_i\right\}_{i \geq 2}$ be a collection of degree $2-i$
morphisms $C \rightarrow C^i$. 

Their \em (non-augmented) cobar-construction $\infbarnaug(C)$ \rm  
comprises objects $C^{i+1}$ for all $i \geq 0$ each placed
in degree $i$ and degree $k-1$ maps 
$d_{i(i+k)}\colon C^{i} \rightarrow C^{i+k}$ 
defined by
\begin{equation}
\label{eqn-differentials-in-non-aug-cobar-construction}
d_{i(i+k)} := (-1)^{(i-1)(k+1)} \sum_{j = 0}^{i-1} (-1)^{jk} 
\id^{i-j-1}\otimes \Delta_{k+1} \otimes \id^{j}. 
\end{equation} 


\begin{tiny}
\begin{equation}
\label{eqn-nonaugmented-cobar-construction-of-A-m_i}
\begin{tikzcd}[column sep = 2.5cm]
\underset{\degzero}{C}
\ar{r}[']{\Delta_2}
\ar[bend left=25]{rr}[description]{\Delta_3}
\ar[bend left=30]{rrr}[description]{\Delta_4}
\ar[bend left=35]{rrrr}[description]{\Delta_5}
&
C^2
\ar{r}[']{C\Delta_2-\Delta_2C}
\ar[bend left=25]{rr}[description]{-C\Delta_3 - \Delta_3C}
\ar[bend left=30]{rrr}[description]{C\Delta_4-\Delta_4C}
& 
C^3
\ar{r}[']{C^2\Delta_2-C\Delta_2C+\Delta_2C^2}
\ar[bend left=25]{rr}[description]{C^2\Delta_3+C\Delta_3C+\Delta_3C^2}
&
C^4
\ar{r}[']{\begin{smallmatrix}C^3 \Delta_2  - C^2\Delta_2C + \\ + C\Delta_2C^2 - \Delta_2 C^3 \end{smallmatrix}}
&
\dots
\end{tikzcd}
\end{equation}
\end{tiny}

\end{defn}


\begin{defn}
\label{defn-ainfty-coalgebra-in-a-monoidal-category}
Let $\A$ be a monoidal DG category. An \em $\Ainfty$-coalgebra 
$(C,\Delta_i)$ \rm in $\A$ is an object $C \in \A$ equipped 
with degree $2-i$ morphisms $\Delta_i \colon C \rightarrow C^i$ 
for all $i \geq 2$ whose non-augmented 
cobar-construction $\infbarnaug(C)$ is a twisted complex over $\A$. 
\end{defn}

We define morphisms of $\Ainfty$-coalgebras in $\A$ in a similar way:

\begin{defn} 
Let $(C, \Delta_k)$ and $(D, \Delta'_k)$ be $\Ainfty$-coalgebras in $\A$. 
Let $(f_i)_{i \geq 1}$ be a collection of degree $1-i$ morphisms 
$C \rightarrow D^i$. 

The \em cobar-construction \rm $\infbar(f_\bullet)$ is 
a collection of morphisms $C^{i} \rightarrow D^{i+k}$ defined by 
$$ \sum_{t_1 + \dots + t_i = i + k}
(-1)^{\sum_{l=2}^{i}(1-t_l)\sum_{n=1}^l t_n
} 
f_{t_1}\otimes\ldots \otimes f_{t_i} $$
We illustrate the first few components:
\begin{small}
\begin{equation*}
\begin{tikzcd}[column sep = 2.6cm, row sep = 3cm]
C
\ar{r}
\ar{d}[description, pos = 0.80]{f_1}
\ar{dr}[description, pos = 0.60]{f_2}
\ar{drr}[description, pos = 0.30]{f_3}
\ar{drrr}[description, pos = 0.10]{f_4}
&
C^2
\ar{r}
\ar{d}[description, pos = 0.80]{f_1f_1}
\ar{dr}[description, pos = 0.60]{-f_1f_2 + f_2 f_1}
\ar{drr}[description, pos = 0.30]{f_1f_3 + f_2 f_2 + f_3 f_1}
&
C^3
\ar{r}
\ar{d}[description, pos = 0.80]{f_1f_1f_1}
\ar{dr}[description, pos = 0.60]{f_1f_1f_2 - f_1 f_2 f_1 + f_2 f_1 f_1}
&
C^4
\ar{r}
\ar{d}[description, pos = 0.80]{f_1f_1f_1f_1}
&
\dots
\\
D
\ar{r}
&
D^2
\ar{r}
&
D^3
\ar{r}
&
D^4
\ar{r}
&
\dots
\end{tikzcd}
\end{equation*}
\end{small}
\end{defn}


\subsection{$\Ainfty$-comodules}
\label{section-Ainfty-comodules-in-a-monoidal-category}

We define left and right $\Ainfty$-comodules using the same ideas 
we used to define $\Ainfty$-modules in
\S\ref{section-Ainfty-modules-in-a-monoidal-category}.

\begin{defn}
\label{defn-right-comodule-cobar-construction-in-a-monoidal-category}
Let $(C,\Delta_i)$ be an $\Ainfty$-coalgebra in a monoidal DG category $\A$. 
Let $G \in \A$ and let $\left\{r_i\right\}_{i \geq 2}$ be a collection 
of degree $2-i$ morphisms $G \rightarrow G\otimes C^{i-1}$. 

The \em right comodule cobar-construction $\infbar(G)$ \rm of $(G,r_i)$
comprises objects $G \otimes C^{i}$ for $i \geq 0$ 
placed in degree $i$ and degree $1-k$ maps 
$G \otimes C^{i-1} \rightarrow G \otimes C^{i+k-1}$ 
defined by
\begin{scriptsize}
\begin{equation}
\label{eqn-differentials-in-right-comodule-cobar-construction}
d_{i(i+k)} := (-1)^{(i-1)(k+1)} 
\left(\sum_{j = 0}^{i-2} \left( (-1)^{jk} \id^{i-j-1} \otimes \Delta_{k+1} \otimes
\id^{j}\right) + (-1)^{(i-1)k}r_{k+1}\otimes \id^{i-1} \right). 
\end{equation} 
\end{scriptsize}

\begin{tiny}
\begin{equation}
\label{eqn-right-comodule-cobar-construction-of-A-m_i}
\begin{tikzcd}[column sep = 2.4cm]
\underset{\degzero}{G}
\ar{r}[']{r_2}
\ar[bend left=25]{rr}[description]{r_3}
\ar[bend left=30]{rrr}[description]{r_4}
\ar[bend left=35]{rrrr}[description]{r_5}
&
GC
\ar{r}[']{G\Delta_2 - r_2C}
\ar[bend left=25]{rr}[description]{-G\Delta_3 - r_3A}
\ar[bend left=30]{rrr}[description]{G\Delta_4 - r_4A}
& 
GC^2
\ar{r}[']{GC\Delta_2 - G \Delta_2 C + r_2 C^2}
\ar[bend left=25]{rr}[description]{GC \Delta_3 + G\Delta_3C + r_3 C^2 }
&
GC^3
\ar{r}[']{\begin{smallmatrix}GC^2 \Delta_2  - GC\Delta_2C + \\ + G\Delta_2C^2 - r_2 C^3 \end{smallmatrix}}
&
\dots
\end{tikzcd}
\end{equation}
\end{tiny}

For $G \in \A$ and a collection 
$\left\{r_i\right\}_{i \geq 2}$ of degree $2-i$
morphisms $ G\rightarrow C^{i-1}\otimes G$,
its \em left comodule cobar-construction $\infbar(G)$ \rm
comprises objects $C^{i}\otimes G$ for all $i \geq 0$ placed
in degree $i$ and degree $1-k$ maps 
$C^{i-1}\otimes G \rightarrow C^{i+k-1}\otimes G$ 
defined by
\begin{equation}
\label{eqn-differentials-in-left-comodule-cobar-construction}
d_{i(i+k)} := (-1)^{(i-1)(k+1)} 
\left(\sum_{j = 1}^{i-1} \left( (-1)^{jk} \id^{i-j-1} \otimes \Delta_{k+1} \otimes
\id^{j}\right) + \id^{i-1} \otimes r_{k+1} \right). 
\end{equation} 

\begin{tiny}
\begin{equation}
\label{eqn-left-comodule-cobar-construction-of-A-m_i}
\begin{tikzcd}[column sep = 2.4cm]
\underset{\degzero}{G}
\ar{r}[']{r_2}
\ar[bend left=25]{rr}[description]{r_3}
\ar[bend left=30]{rrr}[description]{r_4}
\ar[bend left=35]{rrrr}[description]{r_5}
&
GC
\ar{r}[']{Cr_2 - \Delta_2G}
\ar[bend left=25]{rr}[description]{-Cr_3 - \Delta_3G}
\ar[bend left=30]{rrr}[description]{Cr_4 - \Delta_4G}
& 
GC^2
\ar{r}[']{C^2r_2 - C \Delta_2 G + \Delta_2 CG}
\ar[bend left=25]{rr}[description]{C^2 r_3 + C\Delta_3G + \Delta_3 CG }
&
CG^3
\ar{r}[']{\begin{smallmatrix}C^3 r_2  - C^2\Delta_2G + \\ + C\Delta_2CG - \Delta_2 C^2G \end{smallmatrix}}
&
\dots
\end{tikzcd}
\end{equation}
\end{tiny}

\end{defn}

\begin{defn}
Let $\A$ be a monoidal DG category and let $(C,\Delta_i)$ be an
$\Ainfty$-coalgebra in $\A$. A \em right (resp. left) $\Ainfty$-comodule 
$(G, r_i)$ over $C$ \rm is an object $G \in \A$ and a collection  
$\left\{r_i\right\}_{i \geq 2}$ of degree $2-i$
morphisms $G \rightarrow G \otimes C^{i-1}$ (resp. 
$G \rightarrow C^{i-1} \otimes G$) such that $\infbar(G)$ is a twisted complex. 
\end{defn} 

\begin{defn} 
Let $\A$ be a monoidal DG category and let $(C,\Delta_i)$ be an
$\Ainfty$-coalgebra in $\A$.
Let $(G, r_k)$ and $(H, s_k)$ be right $\Ainfty$-comodules over $C$ in $\A$. 

A \em degree $j$ morphism \rm $f_\bullet\colon (G, r_k) \rightarrow (H, s_k)$ of 
right $\Ainfty$-$C$-comodules is a collection $(f_i)_{i \geq 1}$ of 
degree $j - i + 1$ morphisms $G\rightarrow H \otimes C^{i-1}$. Its
\em cobar-construction \rm $\infbar(f_\bullet)$ is 
the morphism $\infbar(G) \rightarrow \infbar(H)$ in $\pretriagpls(\A)$
whose components are  
$$
G \otimes C^{i-1} \rightarrow H \otimes C^{i+k-1}\colon \;
(-1)^{j(i-1)}  f_{k+1}\otimes \id^{i-1}. $$
We illustrate the case when $f_\bullet$ is of odd degree:
\begin{small}
\begin{equation*}
\begin{tikzcd}[column sep = 2.6cm, row sep = 3cm]
G
\ar{r}
\ar{d}[description, pos = 0.85]{f_1}
\ar{dr}[description, pos = 0.65]{f_2}
\ar{drr}[description, pos = 0.30]{f_3} 
\ar{drrr}[description, pos = 0.10]{f_4}
&
GC
\ar{r}
\ar{d}[description, pos = 0.85]{-f_1C}
\ar{dr}[description, pos = 0.65]{-f_2 C}
\ar{drr}[description, pos = 0.30]{-f_3 C}
&
GC^2
\ar{r}
\ar{d}[description, pos = 0.85]{f_1 C^2}
\ar{dr}[description, pos = 0.65]{f_2 C^2}
&
GC^3
\ar{r}
\ar{d}[description, pos = 0.85]{-f_1 C^3}
&
\dots
\\
H
\ar{r}
&
HC
\ar{r}
&
HC^2
\ar{r}
&
HC^3
\ar{r}
&
\dots
\end{tikzcd}
\end{equation*}
\end{small}
\end{defn}

The corresponding definition for the left $\Ainfty$-modules differs only
in signs:
\begin{defn} 
Let $\A$ be a monoidal DG category and let $(C,\Delta_i)$ be an
$\Ainfty$-coalgebra in $\A$.
Let $(G, r_k)$ and $(H, s_k)$ be left $\Ainfty$-comodules over $C$ in $\A$. 

A degree $j$ morphism $f_\bullet\colon (G, r_k) \rightarrow (H, s_k)$ of 
left $\Ainfty$-$C$-comodules is a collection $(f_i)_{i \geq 1}$ of 
degree $j - i + 1$ morphisms $ G \rightarrow C^{i-1} \otimes H$. 
Its \em cobar-construction \rm $\infbar(f_\bullet)$ is 
the morphism $\infbar(G) \rightarrow \infbar(H)$ in $\pretriagpls(\A)$
whose components are 
$$ A^{i-1} \otimes G \rightarrow C^{i+k-1} \otimes H \colon \;
(-1)^{(j+k)(i-1)}  \id^{i-1} \otimes f_{k+1}. $$
\end{defn}

We define the DG categories of left and right comodules over $C$ in the
unique way which makes the left and right module cobar-constructions 
into faithful DG functors from these categories to $\pretriagpls(\A)$:

\begin{defn}
Let $\A$ be a monoidal DG category and $C$ be an $\Ainfty$-coalgebra in
$\A$. Define the \em DG category $\conodC$ of
right $\Ainfty$-$C$-comodules in $\A$ \rm by:
\begin{itemize}
\item Its objects are right $\Ainfty$-$C$-comodules in $\A$,
\item For any $G, H \in \obj \conodC$, the complex 
$\homm^\bullet_{\conodC}(G, H)$
consists of $\Ainfty$-morphisms $f_\bullet\colon G \rightarrow H$
with their natural grading. The differential and
the composition is defined by composing the corresponding 
twisted complex morphisms. 
\item The identity morphism of $G \in \conodC$ is the morphism 
$(f_\bullet)$ with $f_1 = \id_G$ and $f_{\geq 2} = 0$ whose
corresponding twisted complex morphism is $\id_{\infbar(G)}$. 
\end{itemize}
The \em DG category $\Cconod$ of left $\Ainfty$-$C$-comodules 
in $\A$ \rm is defined analogously.  
\end{defn}

For brevity, we write $\infhom_{rC}(-,-)$ and $\infhom_{lC}(-,-)$
for the $\homm$-spaces in $\conodC$ and $\Cconod$, respectively. 


\subsection{Bicomodules}
\label{section-Ainfty-bicomodules-in-a-monoidal-category}

We define $\Ainfty$-bicomodules similarly to the 
way we defined 
definition of $\Ainfty$-bimodules in 
\S\ref{section-Ainfty-bimodules-in-a-monoidal-category}

\begin{defn}
\label{defn-bicomodule-bar-construction-in-a-monoidal-category}
Let $(C,\Delta_i)$ and $(D,\Theta_i)$ be an $\Ainfty$-coalgebras 
in a monoidal DG category $\A$. 
Let $M \in \A$ and let $\left\{r_{ij}\right\}_{i + j \geq 1}$ be a collection 
of degree $1-i-j$ morphisms $M \rightarrow C^i \otimes M \otimes D^{j}$. 
The \em bicomodule cobar-construction $\infbar(M)$ \rm 
comprises objects $C^i \otimes M \otimes D^{j}$ with $i + j \geq 0$ 
placed in bidegree $i,j$ and degree $1+p+q-i-j$ maps 
$$ C^{i} \otimes M \otimes D^{j} \longrightarrow 
\bigoplus_{p + q = i + j + k} C^p \otimes M \otimes D^q $$
defined by
\begin{equation}
\label{eqn-differentials-in-the-bimodule-cobar-construction}
(-1)^{(i+j)(k+1)} 
\sum_{r = 0}^{i+j} (-1)^{rk} \id^{i + j - r} \otimes (\Delta r\Theta)_{k+1}
\otimes \id^{r},
\end{equation} 
where $(\Delta r \Theta)_{k+1}$ denotes the unique operation --- 
either $r_{s,t}$ with $s+t = k$ or $\Delta_{k+1}$ or $\Theta_{k+1}$ --- 
that can be applied to the corresponding factor
of $C^{i} \otimes M \otimes D^{j}$. 

\begin{tiny}
\begin{equation*}
\begin{tikzcd}[row sep=3cm, column sep = 3.5cm]
M
\ar{r}[pos = 0.7, description]{r_{01}}
\ar{d}[description]{r_{10}}
\ar[bend left=10]{rr}[pos=0.8, description]{r_{02}}
\ar[bend right=35]{dd}[pos=0.85, description]{r_{20}}
\ar{rd}[pos = 0.7, description]{r_{11}}
\ar{rdd}[pos = 0.86, description]{r_{21}}
\ar{rrd}[pos = 0.8, description]{r_{12}}
\ar[bend right=10]{rrdd}[pos = 0.8, description]{r_{22}}
&
MD
\ar{r}[pos = 0.5, description]{M\Theta_2 - r_{01}D} 
\ar{d}[pos = 0.62, description]{-r_{10	}D}
\ar[bend left=35]{dd}[pos = 0.86, description]{-r_{20}D}
\ar{rd}[pos = 0.7, description]{-r_{11}D}
\ar{rdd}[pos=0.8, description]{-r_{21}D}
&
 M D^2
\ar{d}[description]{r_{10}D^2}
\ar[bend left=35]{dd}[pos=0.85, description]{r_{20}D^2}
\\
CM
\ar{r}[pos = 0.7, description]{Cr_{01}}
\ar{d}[pos = 0.35, description]{Cr_{10} - \Delta_2M}
\ar[bend left=10]{rr}[pos = 0.83, description]{-Cr_{02}}
\ar{rd}[pos = 0.83, description]{-Cr_{11}}
\ar{rrd}[pos = 0.75, description]{Cr_{12}}
&
CMD
\ar{r}[pos = 0.55, description]{CM\Theta_2 - Cr_{01}D}
\ar{d}[pos = 0.35, description]{-Cr_{10}D + \Delta_2MD}
\ar{rd}[description]{Cr_{11}D}
&
CMD^2
\ar{d}[pos = 0.3, description]{Cr_{10}B^2 - \Delta_2MD^2}
\\
C^2 M
\ar{r}[pos = 0.7, description]{C^2r_{01}}
\ar[bend right = 10]{rr}[pos = 0.8, description]{C^2 r_{02}}
&
C^2 MD
\ar{r}[description]{C^2M\Theta_2 - C^2r_{01}D}
&
C^2 M D^2
\end{tikzcd}
\end{equation*}
\end{tiny}
\end{defn}

\begin{defn}
An \em $\Ainfty$-bicomodule \rm over $\Ainfty$-coalgebras $(C,\Delta_i)$ and $(D,\Theta_i)$ 
in a monoidal DG category $\A$ is
an object $M \in \A$ and a collection $\left\{r_{ij}\right\}_{i + j \geq 1}$ 
of degree $1-i-j$ morphisms $ M\rightarrow C^i \otimes M \otimes D^{j}$
such that $\infbar(M)$ is a twisted complex. 
\end{defn}

The remaining definitions are then analogous to those for left and
right modules:


\begin{defn} 
Let $\A$ be a monoidal DG category and let $(C,\Delta_i)$ and $(D, \Theta_i)$ be 
$\Ainfty$-coalgebras in $\A$. Let $(M, r_{ij})$ and $(N, s_{ij})$ be
$\Ainfty$-$C$-$D$-bicomodules. 

A \em degree $k$ morphism \rm $f_{\bullet\bullet}\colon 
(M, r_{ij}) \rightarrow (N, s_{ij})$ of 
$\Ainfty$-$C$-$D$-bicomodules is a collection $(f_{lm})_{l + m \geq 0}$ of 
degree $k - l - m$ morphisms $M \rightarrow C^l \otimes  N \otimes D^{m}$. 
Its \em cobar-construction \rm $\infbar(f_{\bullet\bullet})$ is 
the morphism $\infbar(M) \rightarrow \infbar(N)$ in $\twbiospls(\A)$
whose components are  
$$
C^{i} \otimes M \otimes D^{j} \rightarrow C^{i+l} \otimes N \otimes D^{j+m}\colon \;
(-1)^{i(l+m) + k (i+j)}  \id^{i} \otimes f_{l,m} \otimes \id^{j}. $$
\end{defn}

Again, we define the DG category of $\Ainfty$- $C$-$D$-bicomodules in the
unique way which makes the bimodule bar-construction
into a faithful DG functor from this category to $\twbiospls(\A)$:

\begin{defn}
Let $(C,\Delta_i)$ and $(D,\Theta_i)$ be $\Ainfty$-coalgebras in a monoidal DG category 
$\A$. The \em DG category $\CconodD$ of
$\Ainfty$-$C$-$D$-bicomodules in $\A$ \rm is:
\begin{itemize}
\item Its objects are $\Ainfty$-$C$-$D$-bicomodules in $\A$,
\item For any $M,N \in \obj \CconodD$, the complex 
$\homm^\bullet_{\CconodD}(M,N)$
consists of $\Ainfty$-morphisms $f_{\bullet\bullet}\colon M \rightarrow N$
with their natural grading. The differential and
the composition is defined by composing the corresponding 
twisted complex morphisms. 
\item The identity morphism of $M \in \CconodD$ is 
the morphism $(f_{\bullet\bullet})$ with 
$f_{00} = \id_M$ and $f_{ij} = 0$ for $i + j \geq 1$ whose
corresponding twisted complex morphism is $\id_{\infbar(M)}$. 
\end{itemize}
\end{defn}

\subsection{Notions of homotopy counitality}
\label{section-notions-of-homotopy-counitality}

We define the notions of
\begin{itemize}
\item $H$-counitality, strong homotopy counitality, and bicomodule homotopy
counitality for $\Ainfty$-coalgebras,
\item $H$-counitality, homotopy counitality, and 
strong homotopy counitality for $\Ainfty$-comodules,
\end{itemize}
analogously to the way they are defined for algebras and modules
in \S\ref{section-unitality-conditions-for-algebras}-\ref{section-unitality-conditions-for-A-modules}. 

As mentioned above, the Homotopy Lemma 
(Lemma \ref{lemma-the-homotopy-lemma-for-nodA}) doesn't hold for
$\Ainfty$-comodules. Thus we do not get an analogue of Theorem
\ref{theorem-tfae-unitality-conditions-for-A-modules}: its proof
relies on Homotopy Lemma to show that $H$-unitality implies strong homotopy
unitality. Indeed, recall that the Chi-Rho Lemma (Lemma \ref{lemma-chi-rho-lemma})
shows that an $\Ainfty$-$A$-module $(E,p_\bullet)$ is strongly homotopy unital 
if and only if the canonical map 
$$ \rho\colon \infbarres(E,p_\bullet) \rightarrow (E,p_\bullet) $$
is a homotopy equivalence in $\nodA$. On the other hand, $H$-unitality
means that $\forget(\rho)$ is a homotopy equivalence in $\A$. 
In absence of the Homotopy Lemma, the two are not equivalent. 

Thus for $\Ainfty$-comodules, we only get that:
\begin{prps}
Let $(C,\Delta_\bullet)$ be a strongly homotopy counital coalgebra 
in a monoidal DG category $\A$ and let $(G, r_\bullet)$ 
be an $\Ainfty$-$C$-comodule. Then $(G, r_\bullet)$ is homotopy counital iff 
it is $H$-counital. 
\end{prps}
For strongly homotopy counital $\Ainfty$-coalgebras and their homotopy
counital comodules we have the \em Forgetful-Free homotopy adjunction\rm, 
analogous to the Free-Forgetful homotopy adjunction for algebras and modules
described in \S\ref{section-free-forgetful-homotopy-adjunction}. Note 
that the direction of the adjunction has changed. 

For any $\Ainfty$-coalgebra, we can define its co-Kleisli category
similarly to the way we defined the Kleisli category of an
$\Ainfty$-algebra: 
\begin{defn}
Let $(C,\Delta_\bullet)$ be a strongly homotopy counital $\Ainfty$-coalgebra 
in a monoidal DG category $\A$. Define its \em co-Kleisli category 
$\cokleisliC$ \em to be the $\Ainfty$-category
(in the classical sense of \cite{Lefevre-SurLesAInftyCategories})
defined by the following data:
\begin{itemize}
\item Its objects are the objects of $\A$. 
\item For any $G, H \in \A$ the $\homm$-complex between them is
\begin{equation}
\homm_{\cokleisliC}(G, H) := \homm_{\A}(GC, H).
\end{equation}
\item For any $G_1, G_2, \dots, G_{n+1} \in \A$ and any $\alpha_i \in 
\homm_{\cokleisliC}(G_i, G_{i+1})$ define 
\begin{equation}
\label{eqn-defn-of-ainfty-structure-on-cokleisli-category}
m_n^{\cokleisliC}(\alpha_1, \dots, \alpha_n) := 
G_1C \xrightarrow{G_{1}\Delta_n} G_1C^n \xrightarrow{\alpha_1C^{n-1}} G_2C^{n-1} \xrightarrow{\alpha_2 C^{n-2}} \dots \xrightarrow{\alpha_n} G_{n+1}.
\end{equation}
\end{itemize}
\end{defn}
We then have the natural $\Ainfty$-functor 
$$ f_\bullet\colon \cokleisli(C) \rightarrow \free (C) $$
defined analogously to Definition
\ref{defn-kleisli-to-free-Ainfty-functor} and when $C$ is strong
homotopy counital, this functor is a quasi-equivalence.

However, when constructing the derived category of an
$\Ainfty$-coalgebra $C$, the notion of $H$-counitality of its
$\Ainfty$-comodules is not enough. The reason is that, as explained in 
\S\ref{section-the-derived-category-the-general-case}, we want
the derived category to be the cocomplete triangulated hull of 
the free comodules. For $\Ainfty$-algebras, any $H$-unital module was
homotopy equivalent to its bar-resolution, and hence had a resolution by
free modules. 

This is no longer true for $\Ainfty$-coalgebras. $H$-counitality of a 
comodule means that its cobar-construction is null-homotopic as 
a twisted complex over $\A$. Without Homotopy Lemma, this is no longer 
equivalent to its cobar-construction being null-homotopic as 
a twisted complex over $\conodC$. The latter is what we need to have 
the cobar-resolution by free modules. We therefore introduce a new 
notion:
\begin{defn}
Let $(C,\Delta_\bullet)$ be a coalgebra 
in a monoidal DG category $\A$. We say that $(G, r_\bullet) \in \conodC$ 
is {\it strongly $H$-counital} if its cobar-construction is null-homotopic 
as a twisted complex over $\conodC$. We denote the category of strongly 
$H$-counital comodules over $C$ by $\conodshuC$.
\end{defn}
This immediately fixes the above mentioned place in the proof of Theorem
\ref{theorem-tfae-unitality-conditions-for-A-modules} that couldn't be
translated to $\Ainfty$-comodules in absence of the Homotopy Lemma:
\begin{lemma}
Let $(C,\Delta_\bullet)$ be a bicomodule homotopy counital coalgebra 
in a monoidal DG category $\A$. Then any $(G, r_\bullet) \in \conodC$
is strongly homotopy counital iff it is strongly $H$-counital. 
\end{lemma}
The analogue of Lemma 
\ref{lemma-free-modules-are-strong-homotopy-unital}
then holds:
\begin{lemma}
$(C,\Delta_\bullet)$ be a bicomodule homotopy counital coalgebra 
in a monoidal DG category $\A$. Then free $C$-comodules are 
strongly homotopy counital.
\end{lemma}
Note that this implies that every comodule which has resolution by
frees is strongly homotopy counital. Thus, as intended, a comodule is 
strongly homotopy counital if and only if it has a free resolution. 

Finally, for the strongly homotopy counital modules we actually have
an analogue of the Homotopy Lemma: 
\begin{lemma}
\label{lemma-coalgebra-analogue-of-the-homotopy-lemma}
Let $(C,\Delta_\bullet)$ be a bicomodule homotopy counital coalgebra 
in a monoidal DG category $\A$.
\begin{enumerate}
\item A comodule in $\conodshuC$ is acyclic iff it is null-homotopic.
\item A twisted complex in $\twcxub(\conodshuC)$ is acyclic iff it is null-homotopic.
\item A morphism of comodules $f_\bullet$ is a homotopy equivalence in $\conodshuC$
iff its component $f_1$ is a homotopy equivalence in $\A$.
\end{enumerate}
\end{lemma}
\begin{proof}
Suppose that a comodule $(G, r_\bullet)\in \conodshuC$ is acyclic. By forgetful-free adjunction, 
for every free comodule $HC$ we have
\begin{equation*}
\homm_C((G, r_\bullet), HC) \simeq \homm_\A(G, H) \simeq 0
\end{equation*}
since $G$ is null-homotopic in $\A$. Since free comodules are generators for $D(C)$,
this means that $(G, r_\bullet)\simeq 0$ in $D(C)=H^0(\conodshuC)$, meaning $(G, r_\bullet)$ 
is null-homotopic in $\conodshuC$. The rest is proved similarly to 
Lemma \ref{lemma-the-homotopy-lemma-for-nodA}.
\end{proof}

Thus the bottom line is: provided that one works with bicomodule
homotopy counital $\Ainfty$-coalgebras and their strongly $H$-counital/strongly
homotopy counital comodules, we have 
the analogues of all the results and definitions 
in \S\ref{section-ainfty-structures-in-monoidal-dg-categories} and 
\S\ref{section-strong-homotopy-unitality}. 

\subsection{The derived category}
\label{section-the-derived-category-of-ainfty-coalgebra}

As explained in \S\ref{section-notions-of-homotopy-counitality}, when 
defining the derived category of an $\Ainfty$-coalgebra
$(C,\Delta_\bullet)$, we need to work with strongly $H$-counital comodules, 
instead of just $H$-counital ones. With that in mind, the definitions
and results of
\S\ref{section-the-derived-category-the-general-case}-\ref{section-the-derived-category-the-h-unital-case} translate
straightforwardly. In particular, we define the unbounded derived category 
$D(C)$ as the cocomplete triangulated hull of $H^0(\conodshuC)$
in $H^0(\conodC^\B)$, and the derived category $D_c(C)$ as its compact part. 
As explained in \S\ref{section-notions-of-homotopy-counitality}, 
the derived category $D(C)$ is then generated by free $C$-comodules.

When trying to translate   
\S\ref{section-the-derived-category-the-strong-homotopy-unital-case}
to comodules and describe $D_c(C)$ in terms of the free modules 
generated by the compact generators of $\A$, and hence in terms of the
co-Kleisli category of $C$, we run into a difficulty. For algebras, 
the free module generated by a perfect object of $\A$
was also perfect. This is immediate from Free-Forgetful homotopy
adjunction. For comodules, we have Forgetful-Free homotopy
adjunction. It doesn't allow us to relate the morphisms 
from a free comodule in $\conoddinf\text{-}C$ to those from 
its generating objects in $\A$. 
Thus for a perfect $G\in\A$ the free comodule $GC$ is not, in general, perfect.

Hence, apriori, there is no good description of $D_c(C)$
along the lines of 
\S\ref{section-the-derived-category-the-strong-homotopy-unital-case}
even if $C$ is bimodule homotopy counital. Indeed, even if it is  
strictly counital. 

\subsection{Coalgebras with perfect free comodules}

We can impose an additional condition on our
$\Ainfty$-coalgebra $(C,\Delta_\bullet)$: that
free comodules generated by perfect objects of $\A$ are perfect. 
If this condition holds, all the arguments in 
\S\ref{section-the-derived-category-the-strong-homotopy-unital-case}
translate straightforwardly to coalgebras and comodules. 
In particular, we obtain the following:
\begin{theorem}
\label{theorem-compact-derived-category-of-bimodule-homotopy-unital-C-is-that-of-frees-and-cokleisli}
Let $C$ be a bicomodule homotopy counital $\Ainfty$-coalgebra in a
monoidal DG category $\A$ and $S$ be a set of compact generators 
of $H^0(\A)$ in $H^0(\B)$. 

If for every perfect $G \in \A$, the free $\Ainfty$-comodule $GC$ is
also perfect, then 
$$ D_c(C) \simeq D_c(\free_S\text{-}C) \simeq D_c(\cokleisliCS). $$
Here $\free_S\text{-}C$ denotes the full subcategory of the DG
category of free $C$-comodules comprising the free comodules
generated by the objects of $S$. Moreover, $\cokleisliCS$ 
denotes the full subcategory of $\cokleisliC$ comprising all objects of $S$.
\end{theorem}

It turns that for this condition of perfect objects generating perfect
comodules, it is enough for there to exist a homotopy right adjoint
to $C$ as an object of $\A$: 
\begin{lemma}
\label{lemma-has-right-adjoint-implies-perfect-objects-generate-perfect-comodules}
Let $C$ be a bicomodule homotopy counital $\Ainfty$-coalgebra in a
monoidal DG category $\A$. Suppose there exists $A \in \A$ which 
is homotopy right adjoint to $C$ as an object of $\A$. 

Then for any perfect $G\in \A$ the free $\Ainfty$-comodule 
$GC \in \conoddinfshu\text{-}C$ is perfect.  
\end{lemma}
\begin{proof}
Let $\left\{ H_i \right\}_{i \in I}$ be a collection 
in $\conoddinfshu$-$C^\B$. Then $\bigoplus_{i \in I} H_i$ is also
strong $H$-counital, and thus homotopy equivalent to its cobar-resolution. 
We therefore have a natural homotopy equivalence in $\modk$
\begin{small}
\begin{equation*}
\begin{tikzcd}[column sep=0.5cm]
\homm_{C^\B}(GC,
\ar{d}{\sim}
&
\bigoplus_{i \in I} H_i)
&
&
&
\\
\homm_{C^\B}\Bigl(GC, 
&
(\bigoplus_{i \in I} H_i)C
\ar{r}
\ar[bend left=10]{rr}
\ar[bend left=12.5]{rrr}
&
(\bigoplus_{i \in I} H_i)C^2
\ar{r}
\ar[bend left=10]{rr}
&
(\bigoplus_{i \in I} H_i)C^3
\ar{r}
&
\dots
\Bigr)
\end{tikzcd}
\end{equation*}
\end{small}
For brevity, we write $\homm_{C^\B}$ for $\homm_{\conoddinf\text{-}C^\B}$. 
The bottom expression is isomorphic to the following twisted
complex over $\modk$
\begin{scriptsize}
\begin{equation*}
\begin{tikzcd}[column sep=0.5cm]
\homm_{C^\B}\Bigl(GC, (\bigoplus_{i \in I} H_i)C\Bigr)
\ar{r}
\ar[bend left=10]{rr}
\ar[bend left=12.5]{rrr}
&
\homm_{C^\B}\Bigl(GC, (\bigoplus_{i \in I} H_i)C^2\Bigr)
\ar{r}
\ar[bend left=10]{rr}
&
\homm_{C^\B}\Bigl(GC, (\bigoplus_{i \in I} H_i)C^3\Bigr)
\ar{r}
&
\dots
\end{tikzcd}
\end{equation*}
\end{scriptsize}
Similarly, $\bigoplus_{i \in I} \homm_{\C^\B}(GC, H_i)$ is homotopy
equivalent to 
\begin{scriptsize}
\begin{equation*}
\begin{tikzcd}[column sep=0.5cm]
\bigoplus_{i \in I} \homm_{C^\B}\Bigl(GC, H_iC\Bigr)
\ar{r}
\ar[bend left=10]{rr}
\ar[bend left=12.5]{rrr}
&
\bigoplus_{i \in I} \homm_{C^\B}\Bigl(GC,  H_iC^2\Bigr)
\ar{r}
\ar[bend left=10]{rr}
&
\bigoplus_{i \in I} \homm_{C^\B}\Bigl(GC, H_iC^3\Bigr)
\ar{r}
&
\dots
\end{tikzcd}
\end{equation*}
\end{scriptsize}
Under these identifications, the natural inclusion map 
\begin{equation}
\label{eqn-bigoplus-homm-to-homm-bigoplus-for-EC}
\homm_{C^\B}(GC, \bigoplus_{i \in I} H_i)
\hookrightarrow 
\bigoplus_{i \in I} \homm_{C^\B}(GC, H_i)
\end{equation}
corresponds to the closed degree zero map of twisted complexes
\begin{scriptsize}
\begin{equation*}
\begin{tikzcd}[column sep=0.5cm, row sep = 1cm]
\homm_{C^\B}\Bigl(GC, (\bigoplus_{i \in I} H_i)C\Bigr)
\ar{r}
\ar[bend left=10]{rr}
\ar[bend left=12.5]{rrr}
&
\homm_{C^\B}\Bigl(GC, (\bigoplus_{i \in I} H_i)C^2\Bigr)
\ar{r}
\ar[bend left=10]{rr}
&
\homm_{C^\B}\Bigl(GC, (\bigoplus_{i \in I} H_i)C^3\Bigr)
\ar{r}
&
\dots
\\
\bigoplus_{i \in I} \homm_{C^\B}\Bigl(GC, H_iC\Bigr)
\ar{r}
\ar[bend left=10]{rr}
\ar[bend left=12.5]{rrr}
\ar{u}
&
\bigoplus_{i \in I} \homm_{C^\B}\Bigl(GC, H_iC^2\Bigr)
\ar{r}
\ar[bend left=10]{rr}
\ar{u}
&
\bigoplus_{i \in I}\homm_{C^\B}\Bigl(GC, H_iC^3\Bigr)
\ar{r}
\ar{u}
&
\dots
\end{tikzcd}
\end{equation*}
\end{scriptsize}
whose components are postcompositions with 
inclusions $H_i C^j \hookrightarrow \bigoplus_{i \in I} H_i C^j$. 

Thus, to show that 
\eqref{eqn-bigoplus-homm-to-homm-bigoplus-for-EC} is a homotopy 
equivalence and hence $GC$ is perfect, it suffices to establish 
that for all $j \geq 1$ the map 
$$
\bigoplus_{i \in I}\homm_{C^\B}\Bigl(GC,H_iC^j\Bigr)
\hookrightarrow 
\homm_{C^\B}\Bigl(GC,\bigoplus_{i \in I}H_iC^j\Bigr)
$$
is a homotopy equivalence. By Forgetful-Free homotopy adjunction for 
strongly homotopy counital comodules, this is equivalent to the map 
$$
\bigoplus_{i \in I}\homm_{\B}\Bigl(GC,H_iC^{j-1}\Bigr)
\hookrightarrow 
\homm_{\B}\Bigl(GC,\bigoplus_{i \in I}H_iC^{j-1}\Bigr)
$$
being a homotopy equivalence. By the homotopy adjunction of $C$ and
$A$ this is further equivalent to 
$$
\bigoplus_{i \in I}\homm_{\B}\Bigl(G, H_iC^{j-1}A\Bigr)
\hookrightarrow 
\homm_{\B}\Bigl(G,\bigoplus_{i \in I}H_iC^{j-1}A\Bigr)
$$
being a homotopy equivalence. This holds since $G$ was assumed 
to be perfect.  
\end{proof}

\subsection{Comparison to known constructions}
\label{section-comparison-to-known-constructions}

Let $C$ be a counital DG coalgebra in the classical sense,
meaning that $\A=\B=\modk$.
Positselski in
\cite{Positselski-TwoKindsOfDerivedCategoriesKoszulDualityAndComoduleContramoduleCorrespondence}
shows that the quotient of the homotopy
category of all $C$-comodules by the acyclic comodules is
equivalent to the homotopy category of injective $C$-comodules,
and calls this the derived category of $C$. He then argues that
this category is not invariant under quasi-isomorphisms of coalgebras,
citing a counterexample by D. Kaledin. We cannot prove that the
category $D(C)$ that we define is equivalent to this classical definition,
and we strongly suspect that this is indeed not the case.

On the side of the Verdier quotient, it turns out that because
free comodules are $h$-injective rather than $h$-projective,
Theorem \ref{theorem-derived-category-as-localisation-in-the-strict-case} does not hold. Following its proof  however we
 can still construct an essentially surjective exact functor 
$$
H^0({\bf coNod}^{hu}\text{-}C)/ \acyc \to D(C),
$$
but we can no longer show it is fully faithful.

On the other hand, since in the absence of Homotopy Lemma
we can no longer prove strict $H$-counitality even for strictly
counital comodules over strictly counital coalgebras, we suspect that
not all strictly counital comodules over strictly counital coalgebras
are strictly $H$-counital. This suggests that our $D(C)$ is not the homotopy
category of injective comodules either, but that of strictly $H$-unital
injective comodules.

We therefore ask the following questions: 
\begin{enumerate}
\item Is the category $D(C)$ that we define, in general, equivalent to 
the category Positselski defines in
\cite[2.4]{Positselski-TwoKindsOfDerivedCategoriesKoszulDualityAndComoduleContramoduleCorrespondence}?
\item Is it equivalent for those coalgebras $C$ whose finitely
generated free comodules are perfect? 
\item What conditions one can impose on $C$ for our $D(C)$ to 
be equivalent to the one in \cite[2.4]{Positselski-TwoKindsOfDerivedCategoriesKoszulDualityAndComoduleContramoduleCorrespondence}?
\item For arbitrary $\Ainfty$-coalgebras $C$ and $D$, does $C$ and $D$
being $\Ainfty$-quasi-equivalent imply that 
the categories $D(C)$ and $D(D)$ are equivalent? Note that we have
established this for those coalgebras whose finitely generated free
comodules are perfect. 
\end{enumerate}

\section{Module-comodule correspondence}
\label{section-module-comodule-correspondence}
 
In this section we show that if a bicomodule homotopy counital 
$\Ainfty$-coalgebra $C$ and a strongly homotopy unital $\Ainfty$-algebra $A$ 
are homotopy adjoint, their derived categories of modules and comodules,
respectively, are equivalent.

\subsection{Homotopy adjoint $\Ainfty$-algebras and coalgebras}
\label{section-homotopy-adjoint-Ainfty-algebras-and-coalgebras}
 
We first need to make clear what does it mean for an
$\Ainfty$-algebra and an $\Ainfty$-coalgebra to be homotopy adjoint. 
Clearly, they need to be homotopy adjoint as objects of $\A$. The 
question is how must this adjunction interact with the algebra and 
the coalgebra structures. We propose the following definition, which 
we then try and motivate for the reader:

\begin{defn}
\label{defn-homotopy-adjoint-ainfty-coalgebra-and-algebra}
Let $(C,\Delta_\bullet)$ and $(A,m_\bullet)$ be an $\Ainfty$-coalgebra
and $\Ainfty$-algebra in a monoidal DG category $\A$. We say that 
$(C,\Delta_\bullet)$ and $(A,m_\bullet)$ are \em homotopy adjoint
as $\Ainfty$-coalgebra and $\Ainfty$-algebra \rm if there exist 
\begin{itemize}
\item A collection $\left\{ \eta_i\colon \id \rightarrow C^i A
\right\}_{i \geq 1}$ of degree $1 - i$ morphisms in $\A$, 
\item A collection $\left\{ \epsilon_i\colon A^iC \rightarrow \id \right\}_{i \geq 1}$ of degree $1 - i$ morphisms in $\A$,  
\end{itemize}
such that 
\begin{enumerate}
\item $\eta_1\colon \id \rightarrow CA$ and $\epsilon_1 \colon AC
\rightarrow \id$ are a counit and a unit of a homotopy adjunction of 
$C$ and $A$ as objects of $\A$, 
\item The following is a twisted complex over $\A$: 
\begin{small} 
\begin{equation}
\label{eqn-the-twisted-complex-for-eta-i}
\begin{tikzcd}[column sep=3cm]
\id
\ar{r}{\eta_1}
\ar[bend left=20]{rr}{\eta_2}
\ar[bend left=30]{rrr}{\eta_3}
&
CA
\ar{r}{C^2m_2\circ C\eta_1A -}[']{-\Delta_2A}
\ar[bend right=20,']{rr}{- C^3m_2\circ C\eta_2A - C^3m_3\circ C^2\eta_1A^2\circ C\eta_1A - \Delta_3A }
&
C^2A
\ar{r}{C^3m_2\circ C\eta_1A -}[']{-C\Delta_2A + \Delta_2CA}
&
\ldots
\end{tikzcd}
\end{equation}
\end{small} 
The differentials $\id \to C^iA$ are $\eta_i$
and those $C^{i-1}A \rightarrow C^{k+i-1}A$ are: 
$$
(-1)^{(i-1)(k+1)} 
\left(\sum_{j = 1}^{i-1} \left( (-1)^{jk} C^{i-j-1} \Delta_{k+1}C^{j-1}A\right) + C^{i-1} r_{k+1} \right)
$$
where 
$$
r_{k+1} = \sum\limits_{k_1+\ldots + k_n=k} 
C^k m_{n+1} \circ C^{k_1+\ldots + k_{n-1}}\eta_{k_n} A^{n} \circ \ldots \circ C^{k_1}\eta_{k_2}A \circ \eta_{k_1}A \  \colon A \to C^kA.
$$
\item The following is a twisted complex over $\A$:
\begin{small}
\begin{equation}
\begin{tikzcd}[column sep=3cm]
\ldots
\ar{r}{A^2\epsilon_1 C \circ A^3\Delta_2 -}[']{- Am_2 C + m_2 AC}
\ar[bend right=20,']{rr}{-A\epsilon_1 C \circ A^2\epsilon_1 C \circ A^3\Delta_3 - A\epsilon_2 C \circ A^3\Delta_2 - m_3 C}
\ar[bend left=30]{rrr}{\epsilon_3}
&
A^2C
\ar{r}{A\epsilon_1 C \circ A^2\Delta_2 - m_2 C}
\ar[bend left=20]{rr}{\epsilon_2 }
&
AC
\ar{r}{\epsilon_1}
&
\id.
\end{tikzcd}
\end{equation}
\end{small}
The differentials $A^i C \rightarrow \id$ are $\epsilon_i$
and those $A^{k+i-1}C \rightarrow A^{i-1}C$ are:
$$
(-1)^{(i-1)(k+1)} 
\left(\sum_{j = 1}^{i-1} \left( (-1)^{jk} A^{i-j-1}
m_{k+1}A^{j-1}C\right) + A^{i-1} p_{k+1} \right)
$$
where 
$$
p_{k+1} =\sum\limits_{k_1+\ldots+k_n=k}
\epsilon_{k_n}C \circ A^{k_n}\epsilon_{k_{n-1}}C^2 \circ \ldots \circ
A^{i-k_1}\epsilon_{k_1}C^{n-1} \circ A^i \Delta_{n}
\colon A^k C \to C.
$$
\end{enumerate}
\end{defn}

To motivate this definition, consider the situation where $C$
and $A$ are a strict colagebra and a strict algebra. In other words
$ m_{\geq 3} = \Delta_{\geq 3} = 0$ and we write $\Delta$ 
for the comultiplication $\Delta_2$ of $C$ and $\mu$ for the
multiplication $m_2$ of $A$. 

Suppose $C$ and $A$ are genuinely adjoint as objects of $\A$ with 
unit $\eta\colon \id \rightarrow CA$ and counit $\epsilon\colon AC
\rightarrow \id$. Then the composition 
\begin{equation}
\label{eqn-homotopy-adjunction-algebra-coalgebra-strict-case-comultiplication-on-A}
q\colon A \xrightarrow{\eta{A}} CA^2 \xrightarrow{C \mu} CA
\end{equation}
gives $A$ the structure of (strict) left $C$-comodule if and only
if the following diagram commutes:
\begin{equation}
\label{eqn-homotopy-adjunction-algebra-coalgebra-strict-case-condition-1}
\begin{tikzcd}
A 
\ar{r}{q} 
&
CA
\ar{d}[']{C\eta{A}}
\ar{dr}{\Delta{A}}
&
\\
&  
C^2A^2
\ar{r}[']{C^2\mu}
&
C^2 A.
\end{tikzcd}
\end{equation}
Moreover, if this condition is satisfied, then the strict
associativity of $\mu$ can be seen to imply that $A$ is a 
$C$-comodule-$A$-module: it is simultaneously a
left $C$-comodule and a right $A$-module and the coaction 
commutes with the action. 

Similarly, the composition 
\begin{equation}
p\colon AC \xrightarrow{A{\Delta}} AC^2 \xrightarrow{\epsilon{C}} C
\end{equation}
gives $C$ the structure of (strict) left $A$-module if and 
only if the following diagram commutes
\begin{equation}
\label{eqn-homotopy-adjunction-algebra-coalgebra-strict-case-condition-2}
\begin{tikzcd}
A^2 C
\ar{r}{A^2\Delta} 
\ar{dr}{\mu{C}}
&
A^2C^2
\ar{d}{A\epsilon{C}} 
&
&
\\
&
AC
\ar{r}{p}
&
C.
\end{tikzcd}
\end{equation}
If it does, then $C$ is automatically an $A$-module-$C$-comodule. 

Suppose our monoidal DG category $\A$ admits kernels and cokernels,
and it is thus possible to define the functor of tensor product over
$A$ and the functor of cotensor product over $C$. It can then be verified 
that cotensoring with the $C$-comodule-$A$-module $A$ and 
tensoring with the $A$-module-$C$-comodule $C$ gives mutually inverse
equivalences $\comodd\text{-}C \leftrightarrows \modd\text{-}A$ 
of categories of strict strictly counital
$C$-comodules and strict strictly unital $A$-modules. 

This suggests that in the strict case to be adjoint as an algebra and 
a coalgebra, $A$ and $C$ have to be adjoint as objects of $\A$ and 
the unit and counit of the adjunction have to be compatible with the
multiplication of $A$ and the comultiplication of $C$ in the way
specified by the conditions
\eqref{eqn-homotopy-adjunction-algebra-coalgebra-strict-case-condition-1}
and \eqref{eqn-homotopy-adjunction-algebra-coalgebra-strict-case-condition-2}. 

To extend this to $\Ainfty$-structures, we have to weaken all
the structures involved to homotopy ones. It turns out it
is enough to assume that $A$ and $C$ are naively homotopy adjoint as
objects of $\A$. However, the conditions
\eqref{eqn-homotopy-adjunction-algebra-coalgebra-strict-case-condition-1}
and
\eqref{eqn-homotopy-adjunction-algebra-coalgebra-strict-case-condition-2}
on compability of the unit and the counit of the adjunction with the
multiplication of $A$ and the comultiplication of $C$ have to be
upgraded to $\Ainfty$-versions. 

The following consideration helps to understand how to perform this
upgrade. We gave $A$ an ad-hoc structure 
\eqref{eqn-homotopy-adjunction-algebra-coalgebra-strict-case-comultiplication-on-A}
of a left $C$-comodule. The condition
\eqref{eqn-homotopy-adjunction-algebra-coalgebra-strict-case-condition-1}
ensured that this coaction of $C$ commuted with natural right action
of $A$, i.e. that $A$ was an $C$-comodule-$A$-module. Instead
of $A$, let us consider extension by $A$ of $\id$ with trivial
coaction of $C$ and trivial action of $A$. In other words, let us take
$A \oplus \id$ with the following natural structure of right $A$-module
\begin{equation}
\label{eqn-natural-A-action-on-A-oplus-id}
\begin{tikzcd}
A  
\ar{dr}[description]{\id}
\ar[phantom]{r}{\oplus}
&
A^2 
\ar{d}[description]{\mu}
\\
\id
\ar[phantom]{r}{\oplus}
&
A.
\end{tikzcd}
\end{equation}
and lift his to a structure of a $C$-comodule-$A$-module which 
would have $\id$ as a quotient and $A$ as a subcomodule-module. 
Note that since cotensoring with the trivial
comodule-module $\id$ is the zero functor, the derived functor of
cotensoring with $A \oplus \id$ is the same as that with its
subcomodule-module $A$.  

However, the data defining the $C$-coaction
on $A \oplus \id$ is more interesting. By above, it has two components: 
\begin{equation}
\label{eqn-C-coaction-on-A-oplus-id}
\begin{tikzcd}
\id 
\ar{dr}[description]{\eta'}
\ar[phantom]{r}{\oplus}
&
A 
\ar{d}[description]{q'}
\\
C
\ar[phantom]{r}{\oplus}
&
CA.
\end{tikzcd}
\end{equation}
The condition that $\eta'$ and $q'$ must satisfy for this to be a
coassociative coaction is that the following two diagrams must commute:
\begin{equation}
\label{eqn-strict-case-condition-for-C-coaction}
\begin{tikzcd}
\id 
\ar{r}{\eta'} 
&
CA
\ar{d}[']{C\eta'{A}}
\ar{dr}{\Delta{A}}
&
\\
&  
C^2A^2
\ar{r}[']{C^2\mu}
&
C^2 A.
\end{tikzcd}
\quad\quad
\begin{tikzcd}
A 
\ar{r}{q'} 
&
CA
\ar{d}[']{C\eta'{A}}
\ar{dr}{\Delta{A}}
&
\\
&  
C^2A^2
\ar{r}[']{C^2\mu}
&
C^2 A.
\end{tikzcd}
\end{equation}
Note that $A$ is a subcomodule of $A \oplus \id$ and the component
$q'$ is its $C$-coaction. 

On the other hand, for $C$-coaction \eqref{eqn-C-coaction-on-A-oplus-id}
to commute with $A$-action 
\eqref{eqn-natural-A-action-on-A-oplus-id} the following diagram
must commute:
\begin{equation}
\label{eqn-strict-case-condition-for-C-coaction-and-A-action-to-commute}
\begin{tikzcd}[column sep = 2cm, row sep = 2cm]
A \oplus A^2
\ar{r}{\left(\begin{smallmatrix} 0  \amsamp 0 \\ \eta'A \amsamp q'A\end{smallmatrix}\right)}
\ar{d}[']{\left(\begin{smallmatrix} 0  \amsamp 0 \\ \id  \amsamp \mu\end{smallmatrix}\right)}
&
CA \oplus CA^2 
\ar{d}{\left(\begin{smallmatrix} 0  \amsamp  0 \\ C\id  \amsamp C\mu\end{smallmatrix}\right)}
\\
\id \oplus A 
\ar{r}{\left(\begin{smallmatrix} 0  \amsamp  0 \\ \eta' \amsamp q'\end{smallmatrix}\right)}
&
C \oplus CA.  
\end{tikzcd}
\end{equation}
This amounts to two conditions:
\begin{align}
q' = C\mu \circ \eta' A, 
\\
q' \circ \mu = C\mu \circ q' A. 
\end{align}
The first of these prescribes $q'$ in terms of $\eta'$. If it holds,
then the second condition holds automatically by strict associativity
of $A$. Note that the formula for $q'$ in terms of $\eta'$ is 
precisely the one we used above to define $C$-coaction $q$ on 
$A$ in terms of the adjunction unit $\eta$. Note also 
that when $q' = C\mu \circ \eta' A$ the commutation of the left diagram 
in \eqref{eqn-strict-case-condition-for-C-coaction}
implies the commutation of the right one. 

Finally, if tensoring over $A$ and cotensoring over $C$ are
well defined in $\A$, the condition that 
cotensoring with the resulting $C$-comodule-$A$-module $A \oplus \id$ 
defines a category equivalence $\comodd\text{-}C \rightarrow \modd\text{-}A$ 
is then equivalent to $C$ and $A$ being adjoint as objects of $\A$
with $\eta'$ being the adjunction unit. 

Similar considerations apply to constructing an
$A$-module-$C$-comodule structure on $\id \oplus C$. We summarise: 
any $C$-comodule-$A$-module structure on $A \oplus \id$ where 
\begin{itemize}
\item $A$ is a subcomodule-module, and the corresponding quotient is
$\id$ with the zero comodule-module structure, 
\item the $A$-module structure is the natural one given by
$A^2 \oplus A \xrightarrow{\left(\begin{smallmatrix} 0 & 0 \\ \mu & \id 
\end{smallmatrix}\right)} A \oplus \id$
\end{itemize}
is completely determined by the single map $\eta'\colon \id \rightarrow CA$ 
which makes the diagram 
\begin{equation}
\label{eqn-strict-case-condition-for-C-coaction-on-id-plus-A}
\begin{tikzcd}
\id 
\ar{r}{\eta'} 
&
CA
\ar{d}[']{C\eta'{A}}
\ar{dr}{\Delta{A}}
&
\\
&  
C^2A^2
\ar{r}[']{C^2\mu}
&
C^2 A.
\end{tikzcd}
\end{equation}
commute. Similarly, any $A$-module-$C$-comodule structure on $\id
\oplus C$ where 
\begin{itemize}
\item $C$ is a quotient submodule-comodule, and the corresponding
kernel is $\id$ with the zero module-comodule structure, 
\item the $C$-comodule structure is the natural one given by
$\id \oplus C \xrightarrow{\left(\begin{smallmatrix} 0 & \id \\ 0 & \Delta
\end{smallmatrix}\right)} C \oplus C^2$, 
\end{itemize}
is completely determined by a single map $\epsilon' \colon AC
\rightarrow \id$ which makes the diagram 
\begin{equation}
\label{eqn-strict-case-condition-for-A-action}
\begin{tikzcd}
A^2 C
\ar{r}{A^2\Delta} 
\ar{dr}{\mu{C}}
&
A^2C^2
\ar{d}{A\epsilon'{C}} 
&
&
\\
&
AC
\ar{r}{\epsilon'}
&
\id,
\end{tikzcd}
\end{equation}
commute. Finally, if tensoring over $A$ and cotensoring over $C$ are
well defined in $\A$, such $\id \oplus A$ and $C \oplus \id$ define mutually
inverse equivalences $\comodd\text{-}C \leftrightarrows \modd\text{-}A$
if and only $\eta'$ and $\epsilon'$ define 
an adjunction of $C$ and $A$ as objects of $\A$. 

Our Definition \ref{defn-homotopy-adjoint-ainfty-coalgebra-and-algebra} 
generalises this to non-strict case, where
$(C,\Delta_\bullet)$ and $(A,\mu_\bullet)$ are arbitrary 
$\Ainfty$-coalgebra and $\Ainfty$-algebra. It turns 
out that the structure of $\Ainfty$-$A$-module-$C$-comodule 
structure on $\id \oplus C$ such that
\begin{itemize}
\item $C$ is a quotient $\Ainfty$-submodule-comodule, and the corresponding
kernel is $\id$ with the zero $\Ainfty$-module-comodule structure, 
\item the $\Ainfty$-$C$-comodule structure is given by
$\id \oplus C \xrightarrow{\left(\begin{smallmatrix} 0 & \id \\ 0 &
\Delta_\bullet \end{smallmatrix}\right)} C \oplus C^2$, 
\end{itemize}
is determined by a collection of maps 
$\eta_i \colon \id \rightarrow C^iA$ of degree $1-i$ such that 
\eqref{eqn-the-twisted-complex-for-eta-i} is a twisted complex. Note
that then $d\eta_2$ is the homotopy up to which 
the condition \eqref{eqn-strict-case-condition-for-C-coaction}
on $\eta_1$ holds. Similary, the $\Ainfty$-$C$-module-$A$-comodule 
structure on $A$ satisfying analogous conditions is  
determined by a collection of maps $\epsilon_i \colon  A^iC \rightarrow \id$
of degree $1-i$ which fit into a twisted complex analogous to 
\eqref{eqn-the-twisted-complex-for-eta-i}.

Even in the strict case, we might still need higher $\eta_i$ and $\epsilon_i$ 
to establish the homotopy adjunction of $C$ and $A$ in the sense of 
Definition \ref{defn-homotopy-adjoint-ainfty-coalgebra-and-algebra}.
This happens when $\eta_1$ and $\epsilon_1$, the unit and counit 
of the (homotopy) adjunction of $C$ and $A$ as the objects of $\A$
only make the diagrams \eqref{eqn-strict-case-condition-for-C-coaction}
and \eqref{eqn-strict-case-condition-for-A-action} commute up to
homotopy. An example is given by any homotopy adjoint triple 
of objects of $\A$:
\begin{prps}
\label{prps-LF-and-RF-are-homotopy-adjoint-in-a-monoidal-category}
Let $\A$ be a monoidal DG category. Let $L,F,R \in \A$ and assume 
that $(L,F)$ and $(F,R)$ are homotopy adjoint pairs with unit and
counit pairs $(\eta_l,\epsilon_l)$ and $(\eta_r, \epsilon_r)$. 
Define
$$ \Delta := LF \xrightarrow{L\eta_lF} LFLF, $$
$$ \mu := RFRF \xrightarrow{R\epsilon_rF} RF. $$

Then $(LF, \Delta)$ and $(RF,\mu)$ are a strict coalgebra and a strict 
algebra in $\A$ which are strongly homotopy (co)unital and 
homotopy adjoint in the sense of 
Definition \ref{defn-homotopy-adjoint-ainfty-coalgebra-and-algebra}.
\end{prps}
\begin{proof}
The associativity of $\mu$ and the coassociativity of $\Delta$ follows from $\A$
being a monoidal category. 

Let us denote the homotopies in homotopy adjunctions of $(L, F, R)$ as follows:
\begin{align*}
F \to FRF \to F \quad & = \quad \id + dh^r_F\\
R \to RFR \to R \quad & = \quad \id + dh^r_R\\
F \to FLF \to F \quad & = \quad \id + dh^l_F\\
L \to LFL \to L \quad & = \quad \id + dh^l_L.
\end{align*}
Homotopy unitality of $(RF, \mu)$ 
means that the maps $\mu\circ RF \eta_r$  and $\mu\circ \eta_r RF$ are homotopic to identity.
Indeed, they have the form $\id + Rh^r_F$ and $\id+dh^r_R F$.
Moreover, the homotopy $Rh^r_F$ commutes with the left action of $RF$ 
and the homotopy $h^r_R F$ commutes with the right action of $RF$, 
thus they are morphisms of left and right modules respectively, 
which makes $(RF, \mu)$ strongly homotopy unital.
Strong homotopy counitality of $(LF, \Delta)$ is shown analogously.

Finally, the morphisms $\eta_i$ and $\epsilon_i$ from Definition \ref{defn-homotopy-adjoint-ainfty-coalgebra-and-algebra}
have the form:
\begin{align*}
\eta_i=RFLh_FLh_F\ldots Lh_FLF \circ R\eta_l\eta_l\ldots \eta_l F \circ \eta_r \colon\quad &
\id \to RFLFLF\ldots LF\\
\epsilon_i=\epsilon_l \circ L \epsilon_r\epsilon_r\ldots \epsilon_r F \circ LFRh_F^lRh^l_F....Rh_F^lRF \colon\quad &
LFRFRF\ldots RF \to \id.
\end{align*}

\end{proof}

\begin{remark}
Note that $(RF, \mu)$ and $(LF, \Delta)$ are not generally bimodule homotopy unital
(resp. bicomodule homotopy counital) with the above structure. For that, we would require e.g.
$Rh^r_F$ and $h^r_RF$ to be homotopic, which we don't have for an arbitrary
homotopy adjunction.
\end{remark}

\subsection{Derived module-comodule correspondence}
\label{section-derived-module-comodule-correspondence-for-homotopy-adjoint-algebras-and-coalgebras}

If one took time to set up the formalism of
$\Ainfty$-tensor and cotensor products and $\Ainfty$-modules-comodules,
the above approach should rigorously construct derived 
equivalence 
$D_c(C) \simeq D_c(A)$ as mutually inverse derived functors of 
cotensoring with $A \oplus \id$ and tensoring with $\id \oplus C$.
However, once we have 
Definition \ref{defn-homotopy-adjoint-ainfty-coalgebra-and-algebra},
there is a simpler approach via the Kleisli and the co-Kleisli
categories:

\begin{theorem}[Module-Comodule Correspondence]
\label{theorem-module-comodule-correspondence-in-a-monoidal-dg-category}
Let $(A,m_\bullet, \eta, h^r_\bullet, h^l_\bullet)$ be a strong homotopy
unital $\Ainfty$-algebra in a DG monoidal category $\A$. Let
$(C,\Delta_\bullet, \barepsilon_{\bullet\bullet})$ be a bimodule
homotopy unital $\Ainfty$-coalgebra in $\A$. 

If $C$ is homotopy left adjoint to $A$ in the sense of 
Definition \ref{defn-homotopy-adjoint-ainfty-coalgebra-and-algebra}, 
then 
$$ D_c(C) \simeq D_c(A). $$ 
\end{theorem}
\begin{proof}
Let $S$ be any set of compact generators of $H^0(\A)$ in $H^0(\B)$. 
Since $A$ is strongly homotopy unital, by Theorem
\ref{theorem-compact-derived-category-of-A-is-that-of-frees-and-kleisli}
we have 
$$ D_c(A) \simeq D_c(\kleisli_S(A)). $$
On the other hand, $A$ is a homotopy right adjoint of $C$ as an object of 
$\A$, so by Lemma 
\ref{lemma-has-right-adjoint-implies-perfect-objects-generate-perfect-comodules}
free $\Ainfty$-$C$-comodules generated by perfect objects of $\A$ are
perfect. Thus coalgebra analogues of all the results in 
\S\ref{section-the-derived-category-the-strong-homotopy-unital-case}
hold for $C$. In particular, by 
Theorem.~\ref{theorem-compact-derived-category-of-bimodule-homotopy-unital-C-is-that-of-frees-and-cokleisli}
we have 
$$ D_c(C) \simeq D_c(\cokleisliCS). $$

It therefore suffices to show that 
$$  D_c(\kleisli_S(A)) \simeq  D_c(\cokleisliCS). $$
For this it suffices to show that $\kleisli_S(A)$ and $\cokleisliCS$
are $\Ainfty$-quasi-equivalent. Indeed, by their
respective definitions the set of objects of both $\kleisli_S(A)$ 
and $\cokleisliCS$ is $\A$. Define an $\Ainfty$-functor
$$ E_\bullet\colon \cokleisliCS \rightarrow \kleisli_S(A) $$
by setting 
$$ E_{obj}\colon \obj(\cokleisliCS) \rightarrow \obj(\kleisli_S(A)) $$
to be the identity map and setting 
\begin{align*}
E_n\colon\; 
 \homm_{\cokleisliCS}(F_n, F_{n+1}) \otimes_k \dots \otimes_k
\homm_{\cokleisliCS}(F_1 , F_2)  
\; \longrightarrow \; \homm_{\kleisli_S(A)}(F_1, F_{n+1}) 
\end{align*}
to be the map which sends any 
$$ \alpha_n \otimes \dots \otimes \alpha_1 $$
where  
$$ \alpha_i \in \homm_{\cokleisliCS}(F_i, F_{i+1}) = \homm_{\A}(F_iC,
F_{i+1}) $$
to the map in 
$$ \homm_{\kleisli_S(A)}(F_1, F_{n+1}) = \homm_{\A}(F_1, F_{n+1}A) $$
given by
$$ F_1 \xrightarrow{F_1 \eta_n} F_1C^nA \xrightarrow{\alpha_1 C^{n-1}A}
F_2C^{n-1}A \xrightarrow{\alpha_2 C^{n-2}A} \dots
\xrightarrow{\alpha_{n-1}A^2}
F_nCA \xrightarrow{\alpha_nA} F_{n+1}A. $$
It can readily checked that the defining conditions on $\eta_i$ in
Definition \ref{defn-homotopy-adjoint-ainfty-coalgebra-and-algebra}
imply that $E_\bullet$ is indeed an $\Ainfty$-functor. 

Now observe that 
$$ E_1\colon \homm_{\cokleisliCS}(F_1, F_2) \rightarrow 
\homm_{\kleisli_S(A)}(F_1, F_{2}) $$
is the map 
$$ \homm_{A}(F_1C, F_2) \xrightarrow{(-)A \circ \eta_1}  \homm_{A}(F_1,
F_2A). $$
It is thus a quasi-isomorphism for all $F_1, F_2 \in \A$ by the
assumption that $\eta_1$ is the unit of the homotopy adjunction of $C$ and $A$. 
We conclude that $E$ is an $\Ainfty$-quasi-equivalence, since it is
is identity on objects, while on morphism spaces its first component
is a quasi-isomorphism. 
\end{proof}

\section{Examples and applications}
\label{section-examples-and-applications}

\subsection{Associative algebras}
\label{section-examples-associative-algebras}

Let $k$ be a field. 
Let $\A$ be the category $\vectk$ of vector spaces over $k$ with 
the monoidal operation $\otimes_k$. 
It is a $k$-linear monoidal category which we consider as 
a monoidal DG category concentrated in degree $0$. 

An $\Ainfty$-algebra $(A,m_i)$ in $\A$ 
in the sense of our Defn.~\ref{defn-ainfty-algebra-in-a-monoidal-category} 
is a vector space $A$ and a single linear map $\mu\colon A \otimes_k A
\rightarrow A$. This is because each $m_i\colon A^i \rightarrow A$
has to be of degree $2 - i$, and all morphisms in $\A$ are of degree $0$. 
The bar-construction is  
$$ \infbar(A) = \quad \quad 
\dots \rightarrow A^4 \xrightarrow{A^2 \mu - A \mu A +
\mu A^2 } A^3 \xrightarrow{A \mu - \mu A} A^2 \xrightarrow{\mu} A, $$
the usual bar-construction of $A$. The twisted complex condition  in
Defn.~\ref{defn-ainfty-algebra-in-a-monoidal-category} is the
condition that the bar-construction is a complex of vector spaces, 
i.e. $d^2 = 0$. 
Prop.~\ref{prps-comparing-new-and-old-defns-of-ainfty-algebra} tells
us that it is sufficient to check this just for the first two
differentialls: we have to have 
$$ \mu \circ (A\mu - \mu{A}) = 0. $$
Thus $\Ainfty$-algebras in $\vectk$ are non-unital associative
$k$-algebras $(A,\mu)$. All the unitality conditions listed in 
\S\ref{section-strong-homotopy-unitality} are equivalent to the strict
unitality: the existence of a morphism $\eta\colon k \rightarrow A$ 
such that $\mu \circ \eta{A} = \mu \circ A\eta = \id_A$. Given such 
morphism, $(A,\mu, \eta(1_k))$ is a unital associative $k$-algebra 
in the classical sense. 

Similarly, an $\Ainfty$-module $(E,p_\bullet)$ over $A$ in $\vectk$ is 
a vector space $E$ with a linear map $\pi\colon E \otimes_k A
\rightarrow E$ such that the bar-construction $\infbar(E,p_\bullet)$ 
is a complex of vector spaces:
\begin{equation}
\infbar(E,p_\bullet) =
\quad \quad 
\dots 
\rightarrow 
EA^3 
\xrightarrow{EA\mu - E\mu{A} + \pi A^2}
EA^2
\xrightarrow{E\mu - \pi{A}}
EA
\xrightarrow{\pi}
E, 
\end{equation}
Again, one only needs to check that the composition of 
the first two differentials, whence
the objects of $\nodA$ are the non-unital $A$-modules in 
the usual sense.  

A degree $j$ morphism $(E,\pi) \rightarrow (F,\kappa)$ in $\nodA$
is a linear map $f: E \otimes A^{j} \rightarrow F$. These are
differentiated and composed by differentiating and composing
their bar-constructions as maps of complexes $\infbar(E,\pi)
\rightarrow \infbar(F,\kappa)$. The bar-construction $\infbar(f)$
is the map of complexes whose components are
$(-1)^{j(i-1)} fA^{i-1}$, e.g. 
\begin{small}
\begin{equation}
\label{eqn-homm-complexes-in-nodA-for-A-associative-algebra}
\begin{tikzcd}[column sep = 2.2cm]
\dots
\ar{r}
&
EA^3
\ar{r}{EA\mu - E\mu{A} + \pi A^2}
\ar{dr}[description]{fA^2}
& 
EA^2
\ar{r}{E\mu - \pi A}
\ar{dr}[description]{-fA}
&
EA
\ar{r}{\pi}
\ar{dr}[description]{f}
&
E
\\
\dots
\ar{r}
&
FA^3
\ar{r}[']{FA\mu - F\mu{A} + \kappa A^2}
& 
FA^2
\ar{r}[']{F\mu - \kappa A}
&
FA
\ar{r}[']{\kappa}
&
F.
\end{tikzcd}
\end{equation}
\end{small}

Since $\A$ only had morphisms of degree $0$, the homotopy unital 
$A$-modules are strictly unital $A$-modules. Thus $\nodhuA$ 
is the category of unital $A$-modules in the usual sense with 
the morphism complexes 
\eqref{eqn-homm-complexes-in-nodA-for-A-associative-algebra}. 
It is well-known that for unital $E$ and $F$ each $\homm$-complex  
\eqref{eqn-homm-complexes-in-nodA-for-A-associative-algebra}
of complexes of vector spaces is isomorphic to the following 
$\homm$-complex of complexes $\A$-modules: 
$$ \homm_{\nodA}(E,F) \simeq
\homm_{\modd\text{-}A}(\bar{B}^A_\infty(E), F). $$
Here $\bar{B}^A_\infty(E)$ is the bar-resolution of $E$ in
$\modd\text{-}A$: 
$$ \bar{B}^A_\infty(E,\pi)\colon 
\rightarrow 
EA^3 
\xrightarrow{- EA\mu + E\mu{A} - \pi A^2}
EA^2
\xrightarrow{- E\mu + \pi{A}}
EA. 
$$
The isomorphism sends the morphism given by a linear
map $E \otimes_k A^{j} \rightarrow F$ to the $A$-module morphism 
$E \otimes_k A^{j+1} \rightarrow F$ which corresponds to this linear 
map under the Free-Forgetful adjunction. The adjunction ensures that 
this is a bijection, so it only remains to check that it is compatible 
with composition and differentiation. The former is clear, while for 
the latter the contribution to the differential of 
\eqref{eqn-homm-complexes-in-nodA-for-A-associative-algebra} made
by the composition through $FA$ accounts for the extra term 
present in the differentials of $\bar{B}^A_\infty(E)$ as compared 
to $\infbar(E)$. 

Let $\modd\text{-}A$ be the category of unital $A$-modules
in the usual sense. Since the bar-resolution 
is a projective resolution of $E$ in $\modd\text{-}A$, the cohomologies 
of $\homm_{\modd\text{-}A}(\bar{B}^A_\infty(E), F)$
are $\ext^i_{\modd\text{-}A}(E,F)$. Thus the homotopy 
category $H^0(\nodhuA)$ is $\modd\text{-}A$ and the graded
homotopy category $H^\bullet(\nodhuA)$ is the graded $\ext$-category of 
$\modd\text{-}A$. 

The $A$-modules free in the sense of our
\S\ref{section-free-modules-and-bimodules-over-Ainfty-algebra}
are those free in the usual sense: 
$V \otimes_k A$ for a vector space $V$. The category $\freeA$
is the DG category of free $A$-modules with $\Ainfty$-morphism 
complexes \eqref{eqn-homm-complexes-in-nodA-for-A-associative-algebra}. 
The Kleisli category $\kleisliA$ is the category of
vector spaces with morphism spaces given by 
$\homm_{\vectk}(E, F \otimes_k A)$. We view it as 
a DG category concentrated in degree $0$. By classical Free-Forgetful 
adjunction, it isomorphic to the category of free 
$A$-modules with ordinary $A$-module morphisms. 
Under this identification, the functor 
$\kleisliA \rightarrow \freeA$ of Theorem 
\ref{lemma-ainfty-category-kleisliA-is-well-defined} 
is the DG functor which sends any ordinary $A$-module morphism to
itself considered as a strict $\Ainfty$-morphism. It being 
a quasi-equivalence can be deduced directly from $H^\bullet(\nodhuA)$
being the graded $\ext$-category of $\modd\text{-}A$: 
since free modules are projective, their higher $\ext$s vanish.  

Category $\vectk$ is not small. Even if we used its small 
subcategory $\vectkfg$ of finite-dimensional vector spaces, 
setting $\B = \modA$ would make $\B$ a category of DG $\vectkfg$-modules 
which is needlessly large. Instead, we take $\B = \modk$, 
the category of DG $k$-modules. It contains $\vectk$ as 
the full subcategory of DG modules concentrated in degree $0$. 
It is, moreover, cocomplete, admits convolutions 
of twisted complexes, and is closed monoidal with the monoidal operation
$\otimes_k$. Thus it satisfies all the assumptions
in \S\ref{section-the-setting} and \S\ref{section-the-derived-category}. 

The DG category $\noddinf\text{-}A^{\modk}$ of $\Ainfty$-$A$-modules  
in $\modk$ in the sense of our 
\S\ref{section-ainfty-structures-in-monoidal-dg-categories}
is the DG category of non-unital $\Ainfty$-$A$-modules in 
\cite[\S7]{Lefevre-SurLesAInftyCategories}.  The derived
category $D(A)$ of $A$ in the sense of our
\S\ref{section-the-derived-category} is the cocomplete
triangulated hull of $\nodhuA$, the unital $A$-modules, in 
$H^0(\noddinf\text{-}A^{\modk})$. Since the co-complete triangulated
hull of $\vectk$ in $\modk$ is the whole of $\modk$, by
Prop.~\ref{prps-derived-category-of-A-is-triangulated-cocomplete-hull-of-freeA}
the derived category $D(A)$ is $H^0(\noddinfhu\text{-}A^{\modk})$. 

The category $\vectk$ is generated in $H^0(\modk)$ by the
one-dimensional vector space $k$, a compact object. 
Indeed, the whole of $H^0(\modk)$ is generated by $k$. Thus we can take 
the set $S$ of compact generators of $\vectk$ to consist of a single
object $k$. Then $\freeAS$ is the full subcategory of $\nodA$
consisting of the free module $A$. The corresponding
Kleisli category $\kleisliAS$ consists of a single
object with the endomorphism space $\homm_{\vectk}(k,A) = A$, 
i.e. the algebra $A$ considered as a single-object category. 

Prop.~\ref{prps-perfect-iff-lies-in-hperf-of-perfect-generator-frees}
tells us that $D(A)$ in the sense of our \S\ref{section-the-derived-category} 
is the cocomplete triangulated hull of $\freeAS$ in 
$H^0(\noddinf\text{-}A^{\modk})$. Since $\freeAS$ is just 
the free module $A$ and since $\noddinf\text{-}A^{\modk}$ coincides with 
the category of non-unital $\Ainfty$-$A$-modules of 
\cite[\S2]{Lefevre-SurLesAInftyCategories}, our definition of $D(A)$
coincides with the definition of the derived category of an $\Ainfty$-algebra 
in \cite[\S4.1.2]{Lefevre-SurLesAInftyCategories}. 

Since $A$ is a strict algebra, we can consider as in
\S\ref{section-the-derived-category-strict-algebra-case}
its category $\nodstrA$ of strict $A$-modules in $\vectk$ 
with strict morphisms with strict differentials. 
This is the category of usual non-unital $A$-modules with usual 
$A$-module morphisms. It contains as a full subcategory 
the category $\modd\text{-}A$ of (strictly) unital $A$-modules.  
The category $\modd\text{-}A^{\modk}$ is the
DG category of complexes of unital $A$-modules and 
$H^0(\modd\text{-}A^{\modk})$ is the usual homotopy category of 
complexes of unital $A$-modules. As $A$ is strictly unital,
the bar-resolution is a resolution by strictly unital modules, 
so the proof of Theorem 
\ref{theorem-derived-category-as-localisation-in-the-strict-case}
shows that $D(A)$ as in our \S\ref{section-the-derived-category} 
is equivalent to the Verdier quotient
of $H^0(\modd\text{-}A^{\modk})$ by the acyclics 
of our
Defn.~\ref{defn-acyclicity-of-modules-and-twisted-complexes}. 
These are the complexes of $A$-modules 
null-homotopic as complexes of $k$-modules. Since $k$ is a field,
homotopy equivalences are the same as quasi-isomorphisms.  
Hence $D(A)$ is the localisation of the category of complexes of 
$A$-modules by quasi-isomorphisms. Thus our 
definitions in \S\ref{section-the-derived-category} 
agree with the classical theory. 

Finally, most of the above works just as well when $k$ is a commutative
ring and we take $\A$ to be the category of unital $k$-modules. This
is not covered by \cite{Lefevre-SurLesAInftyCategories} as it assumes
that $k$ is a field and uses minimal models, a technique which works
only over a field, to prove some of its key results e.g. Lemme 4.1.3.7. 
Our theory works happily with $k$ being a
commutative ring.  Its output, however, is different. 

Firstly, our free $A$-modules $E \otimes_k A$ are not free $A$-modules
in the usual sense, unless $E$ is a free $k$-module. They are
pullbacks from $\modk$ to $\modd\text{-}A$ and can only be viewed as
``relatively free over $k$''. In particular, they are no longer
projective as $A$-modules, and thus the $A$-module bar-resolution
$\bar{B}^A_\infty(E)$ is no longer a projective resolution of $E$.
Thus while $H^0(\nodhuA)$ is still $\modd\text{-}A$, it is no longer
true that $H^\bullet(\nodhuA)$ is the graded $\ext$-category of
$\modd\text{-}A$. 

More generally, the derived category $D(A)$ of $A$ in the sense 
our \S\ref{section-the-derived-category} is no longer the classical
derived category of $A$, that is –– a localisation of the category 
of complexes of $A$-modules by quasi-isomorphisms. Instead, it is the
localisation of this category by those maps of complexes of
$A$-modules which are are homotopy equivalences as maps of complexes
of $k$-modules.

This is not an oversight, but the intention. To work in the generality 
of $\Ainfty$-algebras and modules in an arbitrary monoidal DG category
$\A$ (and its ambient category $\B$), we can only work with the
notions we can read off on that level. The notions of homotopy
equivalence and being hullhomotopic exist in any such $\A$, while
those of (the usual) acyclicity and quasi-isomorphisms only exist
when $\A = \modk$ or one of its variants. Indeed, Homotopy Lemma
and Theorem \ref{theorem-derived-category-as-localisation-in-the-strict-case}
show that the derived category of an $\Ainfty$-algebra $A$ in
monoidal DG category $\A$ can be viewed as the category of
homotopy unital $\Ainfty$-$A$-modules considered up to 
homotopy equivalences in $\A$. Working with $\Ainfty$-morphisms
reduces homotopy equivalences of $A$-modules to the homotopy 
equivalences in the base category $\A$. 

To obtain the needed derived category, one must ensure that $\A$
has the right class of morphisms as homotopy equivalences. For
example, in the above example of $A$ being an associative algebra 
over a commutative ring $k$, we can choose $\A$ to be 
the monoidal category of projective $A$-modules and $\B$ to be 
the monoidal DG category of $h$-projective complexes of $A$-modules. 
Then $D(A)$ is the usual derived category 
of complexes of $A$-modules up to quasi-isomorphisms. 

\subsection{DG and $\Ainfty$-algebras}
\label{section-examples-dg-and-ainfty-algebras}

Let $k$ be a commutative ring. Let $\A = \modk$, the category
of DG modules over $k$ with the monoidal operation $\otimes_k$ and
unit $k$.

The $\Ainfty$-algebras, modules, and bimodules in $\A$ in the sense
of our \S\ref{section-ainfty-structures-in-monoidal-dg-categories}
are the usual $\Ainfty$-algebras, modules, and bimodules 
in the sense of \cite{Keller-IntroductionToAInfinityAlgebrasAndModules}, 
\cite{Lefevre-SurLesAInftyCategories}. 
If $A$ is an $\Ainfty$-algebra in $\modk$, then our $\nodA$ is 
the DG category of non-unital $\Ainfty$-modules in 
\cite[\S5.1]{Lefevre-SurLesAInftyCategories}. 

Strict $\Ainfty$-algebras, modules and bimodules are the usual 
(non-unital) DG algebras, modules, and bimodules. Strict module/bimodule 
morphisms with strict differentials are the usual DG module/bimodule 
morphisms. If $A$ is a strict algebra in $\modk$ then our $\nodstrA$ is 
the usual DG category of (non-unital) DG-modules over $A$, see 
\cite[\S2]{AnnoLogvinenko-SphericalDGFunctors},
\cite{Keller-DerivingDGCategories}, or
\cite{Toen-LecturesOnDGCategories}. 

Let $A$ be a strongly homotopy unital $\Ainfty$-algebra in $\modk$. 
The free $A$-modules in the sense of our
\S\ref{section-free-modules-and-bimodules-over-Ainfty-algebra}
are the DG modules $E \otimes_k A$ with $E \in \modk$. When $k$ is a
field, such modules are semifree: they admit an exhaustive filtration 
whose quotients are free modules \cite[Lemma
2.16]{AnnoLogvinenko-BarCategoryOfModulesAndHomotopyAdjunctionForTensorFunctors},
and thus $h$-projective. When $k$ is only a ring, they can be thought
of as $k$-$h$-projective: they have the $h$-projectivity property 
with respect to the $A$-modules null-homotopic as $k$-modules, as
opposed to all $A$-modules with zero cohomology. 

$\freeA$ is the DG category of free modules $E \otimes_k A$
with $\Ainfty$-morphism complexes. The Kleisli category
$\kleisliA$ is the $\Ainfty$-category with the same objects as 
$\modk$ but $\homm$-complexes $\homm^\bullet_{\modk}(E, F \otimes_k A)$
between $E,F \in \modk$. 

The category $\modk$ is cocomplete, admits a change of differential
and is closed monoidal with the monoidal operation $\otimes_k$.
Thus it satisfies all our assumptions 
in \S\ref{section-the-setting} and \S\ref{section-the-derived-category}
and we can set $\B = \A = \modk$. Let the set $S$ of compact generators of
$H^0(\modk)$ to be $\left\{ k \right\}$. 

The category $\kleisliAS$ is a single-object $\Ainfty$-category 
whose single endomorphism space is $\homm_{\modk}(k,k \otimes_k A) = A$. 
It is the $\Ainfty$-algebra $A$ viewed as a single-object 
$\Ainfty$-category.  
$\freeAS$ is a single-object DG category consisting of 
right $\Ainfty$-$A$-module $A$ and its
$\Ainfty$-endomorphisms. The functor 
$f_\bullet\colon \kleisliAS \rightarrow \freeAS$ of Theorem 
\ref{lemma-ainfty-category-kleisliA-is-well-defined} is
the $\Ainfty$-functor which sends single object $k$ 
of $\kleisliA$ to the free module $A$ and whose component
$$ f_n\colon \homm_{\kleisliA}(k, k)^{\otimes_k n} = 
A^{\otimes_k n} \rightarrow \homm_{\nodA}(A,A) $$
sends any $a_1 \otimes \dots \otimes a_n \in A^{\otimes_k n}$
to the $\Ainfty$-endomorphism of $\alpha_\bullet$ of $A$ whose 
component $\alpha_m\colon A^m \rightarrow A$ is the map
$$ b_1 \otimes \dots \otimes b_m \rightarrow 
m_{n+m}(a_1 \otimes \dots \otimes a_n \otimes b_1 \otimes \dots
\otimes b_m). $$
It is a special case of the $\Ainfty$-Yoneda embedding 
constructed in \cite[\S7.1]{Lefevre-SurLesAInftyCategories}. 

Since $\A = \B = \modk$, 
by Prop.~\ref{prps-when-B-is-modA-we-get-the-classical-derived-category}
our derived category $D(A)$ is $H^0(\nodhuA)$. It thus coincides 
with the usual notion of the derived category of an $\Ainfty$-algebra 
defined in \cite[\S4.1.2]{Lefevre-SurLesAInftyCategories}. Since 
$\kleisliAS$ is $A$ viewed as a classical $\Ainfty$-category, 
we have $D(A) \simeq D(\kleisliAS)$. Note that by Theorem 
\ref{theorem-compact-derived-category-of-A-is-that-of-frees-and-kleisli}
we always have $D_c(A) \simeq D_c(\kleisliAS) \simeq D_c(\freeAS)$. 
The coincidence of large derived categories, however, is specific to
the setup in question. 
It can be explained thus: since $\B = \modk$ our construction
takes infinite direct sums in $\modk$, which coincides with the
classical theory.  

A strict algebra $A$ in $\modk$ is a non-unital DG-algebra. 
$\nodstrA$ is the category of non-unital DG-modules over $A$. 
If $A$ is strongly homotopy unital, by Theorem
\ref{theorem-derived-category-as-localisation-in-the-strict-case} 
the derived category $D(A)$ is equivalent to the localisation of
$H^0(\nodstrhuA)$ by the $\A$-homotopy equivalences, i.e.
$\Ainfty$-morphisms whose first component is a homotopy equivalence 
in $\modk$. When $k$ is a field these are the same as quasi-isomorphisms. 

If $A$ is a strictly unital DG algebra, let $\modd\text{-}A$ be
the usual DG category of strictly unital DG modules over $A$. It can be 
viewed as the full subcategory of $\nodstrA$ comprising strictly
unital modules. The bar-resolution is a resolution by free modules, 
and thus by strictly unital ones. The proof of 
Theorem \ref{theorem-derived-category-as-localisation-in-the-strict-case} then 
shows that $D(A)$ is equivalent to the localisation of 
$H^0(\modA)$ by $\A$-homotopy equivalences. When $k$ is a field, 
these are quasi-isomorphisms and we get therefore the classical 
derived category of a strictly unital DG-algebra $A$. We also get
another proof that for DG algebras the $\Ainfty$ derived category
coincides with the DG derived category.

When $k$ is not a field, we can change our monoidal DG category 
$\A$ from $\modk$ to its monoidal subcategory comprising $h$-projective 
modules. Then $\A$-homotopy equivalences would be the same as
quasi-isomorphisms, whence $D(A)$ would be the localisation 
by quasi-isomorphisms, instead of $\A$-homotopy equivalences. 

\subsection{DG and $\Ainfty$-categories}
\label{section-examples-dg-and-ainfty-categories}

Classical DG categories \cite[\S2]{AnnoLogvinenko-SphericalDGFunctors}
\cite{Keller-DerivingDGCategories}
\cite{Toen-LecturesOnDGCategories} and $\Ainfty$-categories \cite{Keller-IntroductionToAInfinityAlgebrasAndModules} 
\cite{Lefevre-SurLesAInftyCategories} fit into our framework very
similarly to DG algebras and $\Ainfty$-algebras, however one needs to 
use a bicategorical setup. See \cite{Benabou-IntroductionToBicategories} 
for an introduction to bicategories. 

Let $k$ be a commutative ring. For any small set $U$ let $k_U$ be the
$k$-linear hull of $U$: the object set is $U$
and the morphism spaces $\homm_{k_U}(s,t)$ are $k$ if $s = t$ and 
$0$ otherwise. Let $\A$ be the following DG bicategory
$\bikmodk$:
\begin{itemize}
\item The objects are categories $k_U$ for each small set $U$. 
\item The $1$-morphism DG categories are defined by
$$ \homm_{\bikmodk}(k_U, k_T) = k_U\text{-}\modd\text{-}k_T \quad
\quad \forall\; \text{ small sets } U,T.$$ 
\item $1$-composition is given by $\otimes_{k}$. 
\item For any small set $U$ the identity $1$-morphism is the diagonal
bimodule $k_U$.
\item Associator and unitor isomorphisms are the maps induced by the universal 
property of tensor product.  
\end{itemize}
By abuse of notation, we write $k$ for $k_U$ when $U$ is a singleton set. 
We further write $\modd\text{-}k_U$ and $k_U\text{-}\modd$ for 
$k\text{-}\modd\text{-}k_U$ and $k_U\text{-}\modd\text{-}k$, respectively. 

An $\Ainfty$-algebra $A$ in some 
$k_U\text{-}\modd\text{-}k_U$ is a
usual $\Ainfty$-category \cite[\S5.1]{Lefevre-SurLesAInftyCategories}
with the object set $U$. Denote this $\Ainfty$-category by $\C$.  
The usual DG category of right $\Ainfty$-$\C$-modules
\cite[\S5.2]{Lefevre-SurLesAInftyCategories} 
is our category $\nodd^{k,k_U}_\infty\text{-}A$ of right $\Ainfty$-$A$-modules in
$k\text{-}\modd\text{-}k_U$. Similarly, the usual left
$\Ainfty$-$\C$-modules are our category $A\text{-}\nodd^{k_U,k}_\infty$ of left $\Ainfty$-$A$-modules in
$k_U\text{-}\modd\text{-}k$. Given another
$\Ainfty$-algebra $B$ in some $k_T\text{-}\modd\text{-}k_T$
corresponding to a usual $\Ainfty$-category $\D$, the usual 
category of $\Ainfty$-$\C$-$\D$-bimodules is 
our category $A\text{-}\noddinf\text{-}B$ of
$\Ainfty$-$A$-$B$-bimodules in $k_U\text{-}\modd\text{-}k_T$. 
Finally, note that we can e.g. consider the category 
$\nodd^{k_T,k_U}_\infty\text{-}A$ of right $\Ainfty$-$A$-modules in any 
$k_T\text{-}\modd\text{-}k_U$, but the result is just a Cartesian 
product of $|T|$ copies of $\nodd^{k,k_U}_\infty\text{-}A$. We therefore write 
$\nodA$ for $\nodd^{k,k_U}_\infty\text{-}A$ and only consider it 
as the category of right $\Ainfty$-$A$-modules. 

As before, strict algebras and strict modules and bimodules in
$\bikmodk$ correspond to the usual (non-unital) DG categories and DG
modules and bimodules. 

The $1$-morphism DG categories are closed monoidal, cocomplete, 
admit a change of differential. 
Each $H^0(k_U\text{-}\modd\text{-}k_T)$ is compactly 
generated by representable bimodules $k^{u,t} =
\homm_{k_U\text{-}\modd\text{-}k_T}(-, (u,t))$ whose only fiber 
is $k$ over $(u,t)$. We can therefore set $\A = \B$ and take 
the set $S$ of the compact generators of $\modd\text{-}k_U$ 
to be $\left\{ k^u \;\middle|\; u \in U \right\}$. 

The category $\kleisliAS$ is the $\Ainfty$-category with 
objects $U$, morphism spaces 
$$ \homm_{\kleisliAS}(u,v) = \homm_{k_U}(k^u, k^v \otimes_k A)
= \leftidx{_v}{A}{_{u}},$$ and $\Ainfty$-operations given by those of $A$. 
Thus it is the $\Ainfty$-category $\C$ which corresponds 
to $A$. The category $\freeAS$ is the full DG subcategory of $\nodA$
comprising the modules corresponding to the representable 
$\Ainfty$-$\C$-modules. The functor 
$f_\bullet\colon \kleisliAS \rightarrow \freeAS$ of Theorem 
\ref{lemma-ainfty-category-kleisliA-is-well-defined} is the 
$\Ainfty$-Yoneda embedding of $\C$ constructed in 
\cite[\S7.1]{Lefevre-SurLesAInftyCategories}.

The rest of our theory applies just as in the case
of DG and $\Ainfty$-algebras detailed in
\S\ref{section-examples-dg-and-ainfty-algebras}. By 
Prop.~\ref{prps-derived-category-of-A-is-triangulated-cocomplete-hull-of-freeA}
for any $H$-unital $A$ the derived category $D(A)$ produced by our theory 
coincides with the $\Ainfty$-derived category of the corresponding 
$\Ainfty$-category $\C$ defined in 
\cite[\S4.1.2]{Lefevre-SurLesAInftyCategories}. If $k$ is a field,
then for a strict algebra $A$ this coincides with the usual DG derived
category of the corresponding DG category $\C$. When $k$ is not a field,  
$D_{dg}(\C)$ is the localisation of $\modC$ by quasi-isomorphisms, 
while $D(A)$ - only its localisation by $\modk$-homotopy equivalences.  
This suggests that if $k$ is not a field, then even for
$\Ainfty$-algebras $A$ we should either further localise $D(A)$ by
quasi-isomorphisms, or change the DG bicategory $\bikmodk$ to its
$2$-full subcategory comprising all $h$-projective bimodules. 

\subsection{Weak Eilenberg-Moore category and $\Ainfty$-monads}
\label{section-classical-and-ainfty-monads}

In
\S\ref{section-examples-associative-algebras}-\ref{section-examples-dg-and-ainfty-categories}
we gave examples of how known results and constructions 
fit into the theory we introduced in 
\S\ref{section-ainfty-structures-in-monoidal-dg-categories}-\S\ref{section-the-derived-category}.
From now on, we give applications of our theory to obtaining the
results and constructions that are, to our knowledge, new. 

Let $\A$ be the strict DG $2$-category $\DGFuntwocat$ of small DG categories, 
DG functors, and DG natural transformations. Let $\C$ be a small DG category. 
A strictly unital algebra $(T,\mu,\eta)$ in 
$\DGFun(\C,\C)$ is a DG monad $T: \C \rightarrow \C$ with multiplication 
$\mu\colon T^2 \rightarrow T$ and unit $\eta\colon \id_\C \rightarrow
T$, see \cite[\S{VI.1}]{MacLane-CategoriesfortheWorkingMathematician}. 

The category $T\text{-}\modd$ in the sense of our
Defn.~\ref{defn-strict-algebra-strict-module-categories}
is the well-known (strict) \em Eilenberg-Moore \rm category 
$\eilmoor_T$ of $T$ 
\cite[\S{VI.2}]{MacLane-CategoriesfortheWorkingMathematician}. 
Its objects are $T$-modules (sometimes known,
unfortunately, as $T$-algebras): pairs $(a,p)$ where $a \in \C$
and $p$ is a structure map $Ta \rightarrow a$ in $\C$ such that
the following diagrams commute
\begin{equation}
\label{eqn-strict-monad-conditions}
\begin{tikzcd}
T^2 a 
\ar{r}{Tp} 
\ar{d}{\mu}
&
Ta
\ar{d}{p}
\\
Ta
\ar{r}{p}
&
a,
\end{tikzcd}
\quad\quad\text{ and }\quad\quad
\begin{tikzcd}
a 
\ar{r}{\eta}
\ar{dr}[']{\id}
&
Ta
\ar{d}{p}
\\
&
a.
\end{tikzcd}
\end{equation}
Its morphisms $(a,p) \rightarrow (b,q)$ are maps $f\colon a \rightarrow b$ 
in $\C$ which intertwine $p$ and $q$. 

It has long been desired to define the weak Eilenberg-Moore
category of a DG monad where the conditions above would only hold 
up to homotopy. For example, when $T$ is viewed 
as a DG enhancement of an exact monad $H^0(T)$, this weak 
Eilenberg-Moore category of $T$ would be a DG refinement 
of the Eilenberg-Moore category of $H^0(T)$. 

Naively, one could try to define the objects of weak Eilenberg-Moore
category to be pairs $(a,p)$ for which 
\eqref{eqn-strict-monad-conditions} only hold up to homotopy and its
morphisms $(a,p) \rightarrow (b,q)$ to be maps $f\colon a \rightarrow
b$ intertwine structure morphisms up to homotopy. This fails for the
usual reasons – we can't define the composition of morphisms. Given
three modules $(a,p), (b,q),$ and $(c,r)$ and two maps $f\colon a
\rightarrow b$ and $g\colon b \rightarrow c$ which intertwine 
$p,q$ and $q,r$, respectively, up to homotopy, their composition $gf$
doesn't, apriori, intertwine $p,r$. Indeed, if $fp - q(Tf) = dh$ and
$gq - r(Tg) = dj$, then $gfp - r(Tg)(Tf) = g(dh) - (dj)(Tf)$, and this
is not, apriori, a boundary. 

Conceptually, the way to fix this is clear – one shouldn't
consider just the first homotopy $h$ up to which $f$ intertwines $p,q$ but 
all the higher homotopies too. In other words, we need the whole 
$\Ainfty$-structure of homotopies up to which a given map 
intertwines the structure morphisms. 
However, the usual definition of $\Ainfty$-structures 
\cite[\S4.1.2]{Lefevre-SurLesAInftyCategories} requires the underlying
DG category to be $\modk$, while here it is $\DGFun(k,\C)
\simeq \C$ and $\DGFun(\C,\C)$. This was the starting point for 
the authors which led to development of the theory presented in this paper. 

With it, we define as per
\S\ref{section-ainfty-structures-in-monoidal-dg-categories}
the DG category $T\text{-}\noddinf$ of left $\Ainfty$-$T$-modules in $\C$ and
their $\Ainfty$-morphisms. As $T$ is strictly
unital, we have as per \S\ref{section-strong-homotopy-unitality}
the equivalence of the notions of $H$-unitality, homotopy unitality, 
and strong homotopy unitality of $\Ainfty$-$T$-modules. 
We therefore define: 
\begin{defn}
Let $\C$ be a DG category and $(T,\mu,\eta)$ be a monad in $\C$. The 
\em weak Eilenberg-Moore category $\eilmoorwk_T$ \rm is the 
DG category $T\text{-}\noddinfhu$ of homotopy unital $\Ainfty$-$T$-modules 
in $\C$. 
\end{defn}

Explicitly, an $\Ainfty$-$T$-module $(a,p_\bullet)$ is an object
$a \in \C$ and a collection of degree $2-i$ 
morphisms $p_i\colon T^{i-1} a \rightarrow a$ such that the maps 
$d_{i(i-k)}\colon T^i a \rightarrow T^{i-k} a$
given by
\begin{align}
d_{i(i-1)} = \left( \sum_{j =
1}^{i-1} (-1)^{i-j} T^{j-1}\mu T^{i-j-1} \right)
 + T^{i-1} p_2 
\end{align}
and for $k > 1$
\begin{align}
d_{i(i-k)} = (-1)^{(i-k)(k+1)} T^{i-k} p_{k+1}, 
\end{align}
define a twisted complex over $\C$
whose objects are $\{ T^i a \}_{i \geq 0}$ with $T^i a$ in degree $-i$: 
\begin{equation}
\label{eqn-bar-construction-of-Ainfty-T-module}
\begin{tikzcd}[column sep = 2.0cm]
\dots
\ar{r}{}
\ar[dashed, bend left=30]{rrrr}[description]{p_5}
\ar[dashed, bend left=25]{rrr}[description]{Tp_4}
& 
T^3a
\ar{r}[']{\mu T - T\mu + T^2 p_2}
\ar[dashed, bend left=25]{rrr}[description]{p_4}
\ar[dashed, bend left=20]{rr}[description]{(- T p_3)}
& 
T^2 a
\ar[dashed, bend left=20]{rr}[description]{p_3}
\ar{r}[']{-\mu + T p_2}
&
Ta
\ar{r}[']{p_2}
&
\underset{\degzero}{a}. 
\end{tikzcd}
\end{equation}
This twisted complex is the \em bar-construction \rm $\infbar(a,p_\bullet)$
of $(a,p_\bullet)$.  

A degree $n$ morphism $(a,p_\bullet) \rightarrow (b,q_\bullet)$ is 
a collection $f_\bullet$ of degree $1-i+n$ morphisms $f_i\colon
T^{i-1}a \rightarrow b$ in $\C$. These are composed and differentiated
by the means of their bar-constructions. The bar-construction 
$\infbar(f_\bullet)$ is the twisted complex morphism 
$\infbar(a,p_\bullet) \rightarrow \infbar(b,q_\bullet)$ whose
components are the maps
$$ T^{i+k} a \rightarrow T^{i} b \quad \quad \text{ given by } \quad 
(-1)^{i(k+\deg_{f_\bullet})} T^{i} f_{k+1} \quad \quad k \geq 0. $$
We illustrate the case when $f_\bullet$ is of odd degree:
\begin{equation}
\label{eqn-bar-construction-of-ainfty-morphism-of-modules-over-monad}
\begin{tikzcd}[row sep=2.5cm, column sep = 2.0cm]
\dots
\ar{r}{}
& 
T^3a
\ar{r}
\ar{d}[description, pos=0.6]{-T^3f_1}
\ar{dr}[description, near start]{T^2f_2}
\ar{drr}[description, near start]{-Tf_3}
\ar{drrr}[description, near start]{f_4}
& 
T^2 a
\ar{r}
\ar{d}[description, pos=0.6]{T^2f_1}
\ar{dr}[description, near start]{ Tf_2}
\ar{drr}[description, near start]{f_3}
&
Ta
\ar{r}
\ar{d}[description, pos=0.6]{-Tf_1}
\ar{dr}[description, near start]{f_2}
&
a 
\ar{d}[description, pos=0.6]{f_1}
\\
\ar{r}{}
& 
T^3b
\ar{r}
& 
T^2 b
\ar{r}
&
Tb
\ar{r}
&
\underset{\degzero}{b}.
\end{tikzcd}
\end{equation} 

The Kleisli category $\kleisli(T)$ in the sense of our
\S\ref{section-kleisli-category} is the usual Kleisli category of the 
monad $(T,\mu, \eta)$ \cite{Kleisli-EveryStandardConstructionIsInducedByAPairOfAdjointFunctors}\cite[\S{VI.5}]{MacLane-CategoriesfortheWorkingMathematician}. It is the DG category whose objects are those of 
$\C$ and whose morphism complexes are defined as
$$ \homm^\bullet_{\kleisli(T)}(a,b) = \homm^\bullet_\C (a,Tb). $$
The composition is defined by setting for any $f\colon a \rightarrow
Tb$ and $g\colon b \rightarrow Tc$ their composition $gf$ in
$\kleisli(T)$ to be the morphism corresponding to
$$ a \xrightarrow{f} Tb \xrightarrow{Tg} T^2c \xrightarrow{{\mu}c} Tc. $$
The identity $1_a$ in $\kleisli(T)$ is the morphism
corresponding to $ a \xrightarrow{{\eta}a} Ta$ in $\C$. 

The classical Free-Forgetful adjunction 
$\C \leftrightarrows \eilmoor_T$ extends to the homotopy adjunction 
$\C \leftrightarrows \eilmoorwk_T$ constructed in 
\S\ref{section-free-forgetful-homotopy-adjunction}. The classical
adjunction yields a fully faithful embedding
$\kleisli(T) \hookrightarrow \eilmoor_T$ whose image is 
the full subcategory $T\text{-}\free^{strict}$ comprising free modules
and their strict morphisms with strict differentials. 
Our category $T\text{-}\free$ is the full subcategory of $\eilmoorwk_T$ 
comprising free modules and all their $\Ainfty$-morphisms. 
The quasi-fully faithful $\Ainfty$-functor 
$\kleisli(T) \rightarrow \free\text{-}T$ of Theorem 
\ref{lemma-ainfty-category-kleisliA-is-well-defined} is  
the DG functor given by the composition of the equivalence 
$\kleisli(T) \xrightarrow{\sim} T\text{-}\free^{strict}$ 
with the natural non-full inclusion 
$T\text{-}\free^{strict} \hookrightarrow T\text{-}\free$. 

More generally, we get new notions of
$\Ainfty$-monads and their $\Ainfty$-modules over DG categories.
The former are the $\Ainfty$-algebras in $\DGFuntwocat$ 
as per \S\ref{section-ainfty-structures-in-monoidal-dg-categories}:
\begin{defn}
\label{defn-ainfty-monad-over-dg-category}
Let $\C$ be a DG category. An \em $\Ainfty$-monad $(T,m_\bullet)$ \rm is 
\begin{itemize}
\item a DG functor $T: \C \rightarrow \C$, 
\item a collection of degree $2-i$ natural transformations 
$m_i\colon T^i \rightarrow T$ for $i \geq 2$,
\end{itemize}
whose bar-construction $\infbarnaug(T)$
(see Defn.~\ref{eqn-nonaugmented-bar-construction-of-A-m_i}) is
a twisted complex over $\DGFun(\C,\C)$: 
\begin{tiny}
\begin{equation}
\label{eqn-nonaugmented-bar-construction-of-T-m_i}
\begin{tikzcd}[column sep = 2.5cm]
\dots
\ar{r}[']{\begin{smallmatrix}T^3 m_2  - T^2m_2T + \\ + Tm_2T^2 - m_2 T^3 \end{smallmatrix}}
\ar[bend left=25]{rr}[description]{T^2m_3 + Tm_3T +  m_3 T^2}
\ar[bend left=30]{rrr}[description]{Tm_4 - m_4T}
\ar[bend left=35]{rrrr}[description]{m_5}
&
T^4
\ar{r}[']{T^2m_2 - T m_2 T + m_2 T^2}
\ar[bend left=25]{rr}[description]{-Tm_3 - m_3T}
\ar[bend left=30]{rrr}[description]{m_4}
& 
T^3
\ar{r}[']{Tm_2 - m_2T}
\ar[bend left=25]{rr}[description]{m_3}
&
T^2
\ar{r}[']{m_2}
&
\underset{\degzero}{T}
\end{tikzcd}
\end{equation}
\end{tiny}
\end{defn}
By Prop.~\ref{prps-defining-equalities-of-Ainfty-morphism}
the twisted complex condition in 
Defn.~\ref{defn-ainfty-monad-over-dg-category}
is equivalent to the following equalities holding:
\begin{equation}
d_{\DGFun(\C,\C)} m_i + \sum_{\begin{smallmatrix}j+k+l = i, \\ k \geq
2\end{smallmatrix}} (-1)^{jk+l} m_{j+1+l} \circ \left(\id^j \otimes
m_k \otimes \id^l\right) = 0. 
\end{equation}
As explained in \S\ref{section-Ainfty-algebras-in-a-monoidal-category}, 
these are $m_1$-free analogues of the defining equations for the usual 
$\Ainfty$-algebras \cite[Defn.~1.2.1.2]{Lefevre-SurLesAInftyCategories}. 

Similarly, we define $\Ainfty$-modules over such $\Ainfty$-monad 
to be the left $\Ainfty$-modules over it in $\DGFun(k,\C) \simeq \C$ 
as per \S\ref{section-ainfty-structures-in-monoidal-dg-categories}: 
\begin{defn}
\label{defn-ainfty-module-over-ainfty-monad}
Let $\C$ be a DG category and let $(T,m_\bullet): \C \rightarrow \C$
be an $\Ainfty$-monad.
An \em $\Ainfty$-module $(a,p_\bullet)$ \rm over $(T,m_\bullet)$ is:
\begin{itemize}
\item an object $a \in \C$,
\item a collection of degree $2-i$ morphisms 
$m_{i} \colon T^{i-1} a \rightarrow a$ for $i \geq 2$,
\end{itemize}
whose bar-construction $\infbarnaug(a,p_\bullet)$
(see Defn.~\ref{defn-left-module-bar-construction-in-a-monoidal-category}) is
a twisted complex over $\C$: 
\begin{tiny}
\begin{equation}
\label{eqn-ainfty-module-bar-construction-of-a-p_i}
\begin{tikzcd}[column sep = 2.5cm]
\dots
\ar{r}[']{\begin{smallmatrix}T^3 p_2  - T^2m_2T + \\ + Tm_2T^2 - m_2 T^3 \end{smallmatrix}}
\ar[bend left=25]{rr}[description]{T^2p_3 + Tm_3T +  m_3 T^2}
\ar[bend left=30]{rrr}[description]{Tp_4 - m_4T}
\ar[bend left=35]{rrrr}[description]{p_5}
&
T^3e
\ar{r}[']{T^2p_2 - T m_2 T + m_2 T^2}
\ar[bend left=25]{rr}[description]{-Tp_3 - m_3T}
\ar[bend left=30]{rrr}[description]{p_4}
& 
T^2a
\ar{r}[']{Tp_2 - m_2T}
\ar[bend left=25]{rr}[description]{p_3}
&
Ta
\ar{r}[']{p_2}
&
\underset{\degzero}{a}
\end{tikzcd}
\end{equation}
\end{tiny}

An \em $\Ainfty$-morphism 
$f_\bullet\colon (a,p_\bullet) \rightarrow (b,q_\bullet)$  
of $\Ainfty$-$T$-modules \rm is a collection of degree $1-i$ 
morphisms $T^{i-1}a \rightarrow b$ in $\C$. The composition and the
differential of such morphisms are defined by the means of their
bar-constructions, see
Defn.~\ref{defn-right-module-bar-constructions-of-an-Ainfty-morphism}
and \ref{defn-left-module-bar-constructions-of-an-Ainfty-morphism}
We denote the resulting DG category of $\Ainfty$-modules over $T$ by
$T\text{-}\noddinf$. 
\end{defn}

To consider such $\Ainfty$-monads over $\C$ as enhancements of genuine
monads on $H^0(\C)$, we need a notion of homotopy unitality. This is
provided by \S\ref{section-strong-homotopy-unitality}:
\begin{defn}
Let $\C$ be a DG category. An $\Ainfty$-monad 
$(T,m_\bullet): \C \rightarrow \C$ is \em strongly homotopy unital \rm 
if there exist morphisms:
\begin{itemize}
\item $\eta\colon \id_\C \rightarrow T$ in $\DGFun(\C,\C)$, 
\item $h^r_\bullet\colon T \rightarrow T$ in the category 
$\nodd^{\C,\C}_{\infty}\text{-}T$ of right $\Ainfty$-$T$-modules in
$\DGFun(\C,\C)$, 
\item $h^l_\bullet\colon T \rightarrow T$ in the category 
$T\text{-}\nodd^{\C,\C}_{\infty}$ of left $\Ainfty$-$T$-modules in
$\DGFun(\C,\C)$, 
\end{itemize}
such that 
\begin{itemize}
\item $\mu_2 \circ \eta T = \id_T + dh^r_\bullet$ in 
$\nodd^{\C,\C}_{\infty}\text{-}T$, 
\item $\mu_2 \circ T \eta = \id_T + dh^l_\bullet$ in 
$T\text{-}\nodd^{\C,\C}_{\infty}$,
\end{itemize}
where $\mu_2$ denotes the $\Ainfty$-lift of the operation $m_2\colon
T^2 \rightarrow T$, see 
\S\ref{section-bar-construction-as-a-complex-of-ainfty-A-modules}.
\end{defn}

By Theorem \ref{theorem-tfae-unitality-conditions-for-A-modules},
if $T$ is strongly homotopy unital then for $\Ainfty$-$T$-modules
it suffices to use (weak) homotopy unitality: 
$(a,p_\bullet)$ is homotopy unital if $(a,p_2)$ is a strictly unital 
$H^0(T)$-module in $H^0(\C)$. Indeed, homotopy unital modules 
are then $H$-unital, and have a bar-resolution by free $T$-modules. 
We thus define:
\begin{defn}
Let $\C$ be a DG category and let $(T,m_\bullet): \C \rightarrow \C$
be an strongly homotopy unital $\Ainfty$-monad. The \em weak Eilenberg-Moore 
category $\eilmoorwk_T$ \rm is the DG category $T\text{-}\noddinfhu$
of homotopy unital $\Ainfty$-$T$-modules in $\C$. 
\end{defn}

One important application of this new notion of 
the weak Eilenberg-Moore category of a DG/$\Ainfty$-monad is
to fix the well-known problem of the (strong) Eilenberg-Moore category 
of an exact monad over a triangulated category not being 
necessary triangulated:
\begin{theorem}
Let $C$ be a triangulated category and let $t: C \rightarrow C$ be 
an exact monad. Let $\C$ be a pretriangulated DG category such that
$H^0(\C) \simeq C$ and let $T\colon \C \rightarrow \C$ be a strong
homotopy unital $\Ainfty$-monad such that $H^0(T) = t$. In other
words, let $\C$ and $T$ be DG/$\Ainfty$-enhancements of $C$ and $t$.  

The homotopy category $H^0(\eilmoorwk_T)$ of the weak
Eilenberg-Moore category of $T$ is a triangulated category
and we have a natural functor 
$$ F\colon H^0(\eilmoorwk_T) \rightarrow \eilmoor_t $$ 
given by $(a,p_\bullet) \mapsto (a,p_1)$ and $f_\bullet \mapsto f_1$
which is not necessarily either full or faithful. 
\end{theorem}
\begin{proof}
Since $\C$ is pretriangulated,  so is
$T\text{-}\noddinf$ by 
\cite{AnnoLogvinenko-UnboundedTwistedComplexes}, Cor. 5.13. 
Since acyclicity is stable under taking
convolutions of bounded twisted complexes, so is $H$-unitality. 
Since $T$ is strongly homotopy unital, $H$-unitality is equivalent 
to homotopy unitality by 
Theorem \ref{theorem-tfae-unitality-conditions-for-A-modules}. 
We conclude that $\eilmoorwk_T$ is a pretriangulated subcategory 
of $T\text{-}\noddinf$. Hence $H^0(\eilmoorwk_T)$ is
triangulated, as desired. 

The fact that the functor $F$ is well-defined follows readily from 
the definitions. To see that the resulting functor is a priori 
neither full nor faithful, let $k$ be a field and let $\C = \modk$, 
the DG category of DG $k$-modules.  Let $R$ be the polynomial algebra
$k[x,y]$ and let $T = (-) \otimes_k R$. We are thus in the setup of 
\S\ref{section-examples-dg-and-ainfty-algebras}. As explained there, 
$H^0(\eilmoorwk_T)$ is the usual derived category $D(R)$
of complexes of $R$-modules. 


On the other hand, $H^0(\C)$ is 
$\modd^{gr}\text{-}k$ the category of graded $k$-modules. 
The monad $H^0(T)$ is tensoring with $R$ in every
degree, thus $\eilmoor_t$ is the category 
$\modd^{gr}\text{-}R$ of graded $R$-modules. 
The functor $F\colon D(R) \rightarrow \modd^{gr}\text{-}R$  
is taking cohomologies of the complex. It is well-known to
be neither full nor faithful: 
\begin{equation}
\begin{tikzcd}
k[x,y]
\ar{r}{x}
\ar{d}{x=0}
&
\underset{\degzero}{k[x,y]}
\\
k[y],
&
\end{tikzcd}
\end{equation}
is a non-trivial morphism in $D(R)$ which is zero on the level of
cohomologies. On the other hand, the complexes
\begin{align*}
k[x,y] \oplus k[x,y] &\xrightarrow{x \oplus y}
\underset{\degzero}{k[x,y]} \\
k[x,y] &\xrightarrow{0} \underset{\degzero}{k}
\end{align*}
are non-isomorphic in $D(R)$ while having the same cohomologies,
.e. the same image under $F$. Any morphism of complexes
lifting the identity map of that image would have to be a
quasi-isomorphism, i.e. an isomorphism in $D(R)$.  
\end{proof}

The $1$-morphism categories of $\DGFuntwocat$ are not pretriangulated
and cocomplete. For a small DG category $\C$ the
DG category $\DGFun(k,\C) \simeq \C$ isn't cocomplete. To construct
the derived category of an $\Ainfty$-monad, we need to embed 
$\DGFuntwocat$ into a bicategory $\B$ satisfying the assumptions
in \S\ref{section-the-setting} and \S\ref{section-the-derived-category}. 
In spirit of the standard Yoneda embedding $\C \hookrightarrow \modC$, 
we want to embed each $1$-morphism category
$\DGFun(\C,\D)$ into the continuous DG functor category 
$\DGFun_{cts}(\modC, \modD)$. By
\cite[Theorem 7.2]{Toen-TheHomotopyTheoryOfDGCategoriesAndDerivedMoritaTheory} 
every such continous functor is homotopy equivalent to tensoring with
some $\C$-$\D$ bimodule. Thus we work with the following bicategory:
\begin{defn}
Let $\DGModtwocat$ be the following bicategory:
\begin{itemize}
\item The objects are small DG categories.  
\item The $1$-morphism DG categories are defined by
$$ \homm_{\DGModtwocat}(\C,\D) = \C{-}\modd\text{-}\D \quad
\quad \forall\; \text{ small DG-categories } \C,\D.$$ 
\item $1$-composition is given by the tensor product of bimodules. 
\item For any $\C$ its identity $1$-morphism is the diagonal
bimodule $\C$.
\item Associator and unitor isomorphisms are the maps induced by the universal 
property of tensor product.  
\end{itemize}
\end{defn}

There is an obvious $2$-fully faithful $2$-functor 
$$ \DGFuntwocat \hookrightarrow \DGModtwocat $$
which is identity on objects. On $1$-morphism categories it is the
functor 
$$ \leftidx{_{(-)}}{\D} \colon \DGFun(\C,\D) \rightarrow \CmodD $$
which sends any $F\colon \C \rightarrow \D$ to 
the extension of scalars bimodule $\leftidx{_F}{\D} = \homm_\D(-,F-)$.
This $2$-functor is $2$-fully faithful because $\CmodD$
is equivalent to $\DGFun(\C,\modD)$ and under this identification 
$\leftidx{_{(-)}}{\D}$ is the functor of post-composition 
with the Yoneda embedding. Such functor is clearly fully faithful.  

$1$-morphism categories of $\DGModtwocat$ are strongly
pretriangulated and cocomplete. Its $1$-composition is closed
by the bimodule Tensor-Hom adjunction. All objects of
$\DGFun(\C,\D)$ are compact in $\CmodD$, and thus $\DGFun(\C,\D)$ 
is tautologically compactly generated in $H^0(\CmodD)$. It therefore 
satisfies all assumptions in \S\ref{section-the-setting} and
\S\ref{section-the-derived-category} and we can set $\B =
\DGModtwocat$. 

An $\Ainfty$-monad $T$ in $\DGFun(\C,\C)$ 
is an $\Ainfty$-algebra $\leftidx{_T}{\C}$ in $\CmodC$. It  
can be further viewed as an $\Ainfty$-monad structure on $T^*: \modC
\rightarrow \modC$. We consider modules over $T$ in 
$\DGFun(k,\C) \simeq \C$. The corresponding $1$-morphism category 
in $\DGModtwocat$ is $k\text{-}\modd\text{-}\C \simeq \modC$ and 
the $2$-functor above becomes the Yoneda embedding $\C \rightarrow \modC$. 
Thus $\Ainfty$-modules over $T$ in $\DGModtwocat$ correspond
to $\Ainfty$-modules over $T^*$ in $\modC$. 

The rest of the theory in \S\ref{section-the-derived-category}
can now be applied in this setup. In particular, by 
Prop.~\ref{prps-when-B-is-modA-we-get-the-classical-derived-category}
we have 
$$ D(T) \simeq D(\kleisli(T)), $$
where on the left is the derived category of $T$ in the sense of
\S\ref{section-the-derived-category} and on the right is the usual 
derived category of $\Ainfty$-category $\kleisli(T)$. For ordinary
DG monads, $\kleisli(T)$ is a DG category. 

In particular, $D_c(T) \simeq D_c(\kleisli(T))$. By
Prop.~\ref{prps-perfect-iff-lies-in-hperf-of-perfect-generator-frees}, 
as objects of $\C$ are compact in $H^0(\modC)$, $D_c(T)$ contains 
$H^0(\eilmoorwk_T)$. The resulting fully-faithful embedding 
$$ H^0(\eilmoorwk_T) \hookrightarrow D_c(\kleisli(T)) $$
is an instance of the well-known embedding of the Eilenberg-Moore
category of a monad into the category of modules over its Kleisli
category \cite[\S5]{Street-TheFormalTheoryOfMonads}. The above equivalence
$D(T) \simeq D(\kleisli(T))$ is therefore its generalisation.  

Finally, there is a version of this whole theory of $\Ainfty$-monads
and their modules for big categories. There 
we set both $\A$ and $\B$ to be the strict $2$-category of strongly
pretriangulated, cocomplete DG categories and continuous functors
between them. 

\subsection{$\Ainfty$-modules over the identity and strong
homotopy idempotents}
\label{section-ainfty-modules-over-the-identity-functor}

The notion of an $\Ainfty$-module over a DG monad described in
\S\ref{section-classical-and-ainfty-monads}
gives non-trivial results even when applied to the trivial monad, 
i.e. the identity functor. 

Recall, for an additive category $\C$ the Eilenberg-Moore category 
$\eilmoor_{\id_\C}$ is $\C$.  
On the other hand, the non-unital modules over $\id_\C$ are the idempotents 
of $\C$. However the non-unital Eilenberg-Moore category of 
$\id_\C$ is not the Karoubi-comletion of $\C$. Given $a,b \in \C$
with idempotents $p$ and $q$, the morphisms in the former are
$f\colon a \rightarrow b$ such that $fp = qf$ and in the latter ––
such that $fp = qf = f$. 

Let $\C$ be a DG category. $\Ainfty$-modules 
over the identity monad $\id_C$ in the sense of our 
\S\ref{section-ainfty-structures-in-monoidal-dg-categories} are 
the \em $\Ainfty$-idempotents \rm which appear in 
\cite[\S4]{GorskyHogancampWedrich-DerivedTracesOfSoergelCategories}. 
These are collections $(a,p_\bullet)$ with $a \in \C$ and 
degree $2-i$ endomorphisms $p_i: a \rightarrow a$ for $i \geq 2$ 
such that the bar-construction below is a twisted complex:
\begin{equation}
\label{eqn-bar-construction-of-Ainfty-Id-module}
\begin{tikzcd}[column sep = 2.0cm]
\dots
\ar{r}{p_2 - \id_a}
\ar[dashed, bend left=30]{rrrr}[description]{p_5}
\ar[dashed, bend left=25]{rrr}[description]{p_4}
& 
a
\ar{r}[']{p_2}
\ar[dashed, bend left=25]{rrr}[description]{p_4}
\ar[dashed, bend left=20]{rr}[description]{(- p_3)}
& 
a
\ar[dashed, bend left=20]{rr}[description]{p_3}
\ar{r}[']{p_2 - \id_a}
&
a
\ar{r}[']{p_2}
&
\underset{\degzero}{a}. 
\end{tikzcd}
\end{equation}

Note that the twisted complex conditions include 
$dp_3 = (p_2 - \id_a) \circ p_2 = p_2^2 - p_2$. 
In other words, $p_2$ is an idempotent of $a$ in $H^0(\C)$, and $p_3$ 
is the homotopy up to which the equation $p_2^2 - p_2$ holds in $\C$.
Similarly, $p_i$ for $i \geq 4$ are the higher homotopies involved.

The monad $\id_\C$ is strictly unital. For any $a \in \C$, the free module 
$(a,\id)$ is strictly unital. The strict morphisms between the free
modules are just those of their underlying objects in $\C$. The
remaining $\Ainfty$-morphisms are homotopy equivalent to them by
Free-Forgetful homotopy adjunction. In other words, the non-full 
inclusion $\C \rightarrow \free\text{-}\id_\C$ is a quasi-equivalence. 

A module $(a,p_\bullet)$ is homotopy unital if $p_2$ is
homotopy equivalent to $\id_a$. Such $p_2$ is a homotopy 
equivalence and therefore by the Homotopy Lemma so is the closed 
degree zero morphism $\pi_2\colon (a,\id) \rightarrow  (a,p_\bullet)$. 
In other words, $(a,p_\bullet)$ is homotopy equivalent to $a$ itself. 
Hence the weak Eilenberg-Moore category $\eilmoorwk_{\id_\C}$ is
quasi-equivalent to $\C$. 

The Kleisli category of $\id_\C$ is $\C$ itself and the
$\Ainfty$-functor $\kleisli(\id_\C) \rightarrow \id_\C\text{-}\free$ 
coincides with the natural embedding $\C \rightarrow \id_\C\text{-}\free$. 
By Prop.~\ref{prps-when-B-is-modA-we-get-the-classical-derived-category}, 
unsurprisingly, $D(\id_\C)$ is equivlent to $D(\C)$. 

It is more useful to consider the category $\id_\C\text{-}\noddinf$ of all 
(non-unital) $\Ainfty$-modules over $\id_\C$. Its objects are
$\Ainfty$-idempotents of $\C$, however it is not the universal homotopy 
Karoubi completion of $\C$ constructed in 
\cite[\S4]{GorskyHogancampWedrich-DerivedTracesOfSoergelCategories}.
In the latter, the morphisms are all twisted complex morphisms
between the respective bar-constructions, while in $\id_\C\text{-}\noddinf$
it's only those twisted complex morphisms which are bar-constructions
of $\Ainfty$-morphism data. The morphisms constructed in
\cite[\S4]{GorskyHogancampWedrich-DerivedTracesOfSoergelCategories}
which show that $(a,p_\bullet)$ is the homotopy image of the homotopy
idempotent $p_2\colon a \rightarrow a$ are not bar-constructions of
$\Ainfty$-morphisms. 

Indeed, DG functors preserve homotopy idempotents. Suppose
$(a,p_\bullet)$ is an image of a homotopy idempotent $p_2 \colon 
a \rightarrow a$ in $\id_\C\text{-}\noddinf$. Applying $\forget$, $a$
is an image of the homotopy idempotent $p_2\colon a \rightarrow a$ in
$\C$. Hence either $a$ is a non-trivial retract of itself in $\C$,
or $p_2$ is a homotopy equivalence in $\C$. Thus we see that we do
not, in general, get $(a,p_\bullet)$ to be the homotopy image of $p_2$. 

Indeed, the bar-resolution $\infbarres(a,p_\bullet)$ of $(a,p_\bullet)$
is a twisted complex over $\id_\C\text{-}\noddinf$ which is the image
of the following twisted complex over $\C$:  
\begin{equation}
\label{eqn-bar-resolution-of-Ainfty-Id-module}
\begin{tikzcd}[column sep = 2.0cm]
\dots
\ar{r}{-p_2}
\ar[dashed, bend left=30]{rrrr}[description]{p_5}
\ar[dashed, bend left=25]{rrr}[description]{-p_4}
& 
a
\ar{r}[']{-p_2 + \id_a}
\ar[dashed, bend left=25]{rrr}[description]{-p_4}
\ar[dashed, bend left=20]{rr}[description]{(- p_3)}
& 
a
\ar[dashed, bend left=20]{rr}[description]{p_3}
\ar{r}[']{-p_2}
&
a
\ar{r}[']{-p_2 + \id_a}
&
\underset{\degzero}{a}. 
\end{tikzcd}
\end{equation}
It can be readily checked that in $\pretriagmns \C$ it is 
the homotopy image of the homotopy idempotent $p_2$. Indeed, since
all twisted complex maps are allowed, we can use the maps constructed 
in \cite[\S4]{GorskyHogancampWedrich-DerivedTracesOfSoergelCategories}. 

We see that $\infbarres(a,p_\bullet)$ is always the homotopy image of
$p_2$, while, in general, $(a,p_\bullet)$ isn't. This illustrates our 
Theorem \ref{theorem-tfae-unitality-conditions-for-A-modules}
which states in this case that the natural map 
$\infbarres(a,p_\bullet) \xrightarrow{\rho} (a,p_\bullet)$
is a homotopy equivalence if and only if $(a, p_\bullet)$ is homotopy 
unital, and thus $p_2$ is a homotopy equivalence. 

We conclude that:
\begin{theorem}
Let $\C$ be a DG category. The full subcategory of $\pretriagmns \C$
supported on the image of $\id_\C\text{-}\noddinf$ under the
bar-resolution functor $\infbarres(-)$ is the universal homotopy
Karoubi completion of $\C$. 
\end{theorem}

\subsection{Enhancing monads over enhanced triangulated categories}
\label{section-enhanced-monads}

Another application of our theory in
\S\ref{section-ainfty-structures-in-monoidal-dg-categories}-\ref{section-the-derived-category} is to enhancing exact monads over triangulated categories. 
We now give a brief overview of the notion of a DG enhanced triangulated 
category, see \cite{BondalKapranov-EnhancedTriangulatedCategories}
\cite[\S1]{LuntsOrlov-UniquenessOfEnhancementForTriangulatedCategories}
\cite[\S3]{AnnoLogvinenko-SphericalDGFunctors}, \cite[\S4.4]{GyengeKoppensteinerLogvinenko-TheHeisenbergCategoryOfACategory} for the details. 

Let $C$ be a triangulated category. A \em DG enhancement \rm of $C$
is a pretriangulated DG category $\C$ with an exact equivalence 
$C \simeq H^0(\C)$. A \em Morita enhancement \rm of $C$ is a small DG
category $\C$ with an exact equivalence $C \simeq D_c(\C)$. Note that
for any $\C$ its compact derived category $D_c(\C)$ is Karoubi
complete, so Morita enhancements can only be used for Karoubi complete
triangulated categories. 

We work with the Morita enhancement framework and thus define:
\begin{defn}
A \em (Morita) enhanced triangulated category $\C$ \rm is a small
DG category. Its underlying triangulated category is the compact
derived category $D_c(\C)$. 
\end{defn}
We consider Morita enhancements up to \em Morita equivalences\rm,
the DG functors $F\colon \C \rightarrow \D$ such that 
$F^*\colon D_c(\C) \rightarrow D_c(\D)$ is an equivalence. 
Enhanced triangulated categories form naturally the
$1$-category $\mor(\DGCat^1)$ which is the localisation 
of the category $\DGCat^1$ of small DG categories by Morita 
equivalences \cite{Tabuada-InvariantsAdditifsDeDGCategories}. 

A fundamental result by To{\"e}n describes morphisms 
$\C \rightarrow \D$ in $\mor(\DGCat^1)$ as being in bijective 
correspondence with the isomorphism classes of $\D$-perfect 
bimodules in $D(\C\text{-}\D)$
\cite[Theorem 7.2]{Toen-TheHomotopyTheoryOfDGCategoriesAndDerivedMoritaTheory}. We thus define: 
\begin{defn}
Let $\C$ and $\D$ be enhanced triangulated categories. An \em enhanced
exact functor $\C \rightarrow \D$ \rm is a $\D$-perfect bimodule
$M \in \CmodD$. Its 
underlying exact functor $D_c(\C) \rightarrow D_c(\D)$ is 
$(-) \ldertimes_\C M$. 
\end{defn}
Enhanced natural tranformations are then morphisms between the
corresponding modules in $D(\C\text{-}\D)$. Together, these form 
a bicategory: 
\begin{defn}
The \em bicategory $\enhcatkc$ of Karoubi complete enhanced
triangulated categories \rm comprises: 
\begin{enumerate}
\item The \em objects \rm are all small DG categories, 
\item For any objects $\C,\D$, 
the \em $1$-morphism category \rm $\enhcatkc(\C,\D)$ is
$D_{\D\text{-}\perf}(\C\text{-}\D)$, 
\item For any objects $\C,\D,\E$, the \em
$1$-composition functor \em is 
\begin{align*}
\enhcatkc(\D,\E) \times \enhcatkc(\C,\D) &\rightarrow \enhcatkc(\C,\E) \\
(M,N) &\mapsto N \ldertimes M, \\
(f,g) &\mapsto g \ldertimes f, 
\end{align*}
where $M,N$ are objects and $f,g$ are morphisms. 
\item For any object $\C$, its \em identity
$1$-morphism \rm is the diagonal bimodule $\C$,
\item The \em associator \rm isomorphism is the natural isomorphism
$$(L \ldertimes_{\C} M) \ldertimes_{\D} N \simeq L \ldertimes_{\C} (M
\ldertimes_{\D} N).$$
\item  The \em unitor \rm isomorphisms are the natural isomorphisms 
$$\C \ldertimes_\C M \simeq M \simeq M \ldertimes_\D \D. $$ 
\end{enumerate}
\end{defn}

As explained in \ref{section-classical-and-ainfty-monads}, to define
an enhanced monad it is not enough to fix an algebra structure on 
a $1$-endomorphism $T$ in $\enhcatkc$. This doesn't give enough data
to construct an Eilenberg-Moore category which would be enhanced 
triangulated. We need a DG enhancement of $\enhcatkc$ and a lift of $T$ 
to an $\Ainfty$-algebra structure in this enhancement. 

The following DG enhancement of $\enhcatkc$ was constructed 
in \cite[\S4.4]{GyengeKoppensteinerLogvinenko-TheHeisenbergCategoryOfACategory}
with the notions developed in
\cite{AnnoLogvinenko-BarCategoryOfModulesAndHomotopyAdjunctionForTensorFunctors}. 
The \em bar category $\CmodbarD$ of bimodules \em is
a DG category isomorphic the category of DG $\C$-$\D$-bimodules 
with $\Ainfty$-morphisms. However, it has an intrinsic
definition and is simpler to work with. We refer the reader to 
\cite[\S3]{AnnoLogvinenko-BarCategoryOfModulesAndHomotopyAdjunctionForTensorFunctors} for the definition and the technical details. 
\begin{defn}[Defn.~4.19,
\cite{GyengeKoppensteinerLogvinenko-TheHeisenbergCategoryOfACategory}]
The \em homotopy unital DG bicategory $\enhcatkcdg$ of 
Karoubi complete enhanced triangulated categories \rm comprises: 
\begin{enumerate}
\item The \em objects \rm are all small DG categories, 
\item For any objects $\C,\D$, 
the \em $1$-morphism category \rm $\enhcatkc(\C,\D)$ is
the full subcategory of $\CmodbarD$ comprising
$\D$-perfect bimodules,
\item For any objects $\C,\D,\E$, the \em
$1$-composition functor \em is 
\begin{align*}
\enhcatkcdg(\D,\E) \otimes \enhcatkcdg(\C,\D) &\rightarrow \enhcatkcdg(\C,\E) \\
(M,N) &\mapsto N \bartimes M, \\
(f,g) &\mapsto (-1)^{|f||g|} g \bartimes f, 
\end{align*}
where $M,N$ are objects and $f,g$ are morphisms. 
\item For any object $\C$, its \em identity
$1$-morphism \rm is the diagonal bimodule $\C$,
\item The \em associator \rm isomorphism is the natural isomorphism
$$(L \bartimes_{\C} M) \bartimes_{\D} N \simeq L \bartimes_{\C} (M
\bartimes_{\D} N).$$
\item  The \em unitor \rm morphisms are the natural homotopy equivalences 
$$\C \bartimes_\C M \xrightarrow{\alpha_\C} M 
\quad \quad \text{and} \quad \quad 
M \bartimes_\D \D \xrightarrow{\alpha_\D} M, $$ 
defined in
\cite[\S3.3]{AnnoLogvinenko-BarCategoryOfModulesAndHomotopyAdjunctionForTensorFunctors}. 
\end{enumerate}
\end{defn}

$\enhcatkcdg$ is only homotopy unital: the unitor
morphisms $\alpha_\bullet$ are homotopy equivalences, and not
isomorphisms. But they have natural homotopy inverses $\beta_\bullet$ which are moreover their genuine inverses on the right, see 
\cite[\S3.3]{AnnoLogvinenko-BarCategoryOfModulesAndHomotopyAdjunctionForTensorFunctors}. It can be readily checked that this weakening doesn't
affect the theory of $\Ainfty$-structures developed in \S\ref{section-ainfty-structures-in-monoidal-dg-categories}-\S\ref{section-the-derived-category}.  Thus we can set the abstract monoidal DG category/bicategory $\A$ in 
this theory to be $\enhcatkcdg$ and define $\Ainfty$-structures on its
$1$-morphisms. The only additional subtlety is that when working with
the unital structures in \S\ref{section-strong-homotopy-unitality} one
must never suppress the identity $1$-morphism which is the source of 
the unit $2$-morphism $\eta\colon \id_\A \rightarrow A$ of the
$\Ainfty$-algebra in question. For example, when defining strong
homotopy unitality in Defn.~\ref{defn-strong-homotopy-unitality}, 
$h^r_\bullet$ is a degree $-1$ morphism $\id_\A \circ_1 A \rightarrow A$
in $\nodA$ satisfying $\mu_2 \circ_2 (\eta \circ_1 A) = \alpha +
dh^r_\bullet$ where $\alpha$ is the unitor $\id_\A
\circ_1 A \rightarrow A$, and similarly for $h^l_\bullet$. 

\begin{defn}
The \em homotopy unital DG bicategory
$\BarModtwocat$ of bar categories of bimodules \rm is 
defined identically to $\enhcatkc$ above, only with 
the $1$-morphism category $\homm_{\BarModtwocat}(\C,\D)$ being
the whole of $\CmodbarD$. 
\end{defn}

For any small DG categories $\C$ and $\D$, the bar category of
bimodules $\CmodbarD$ is pretriangulated and cocomplete. The bar
tensor product $\bartimes$ has a genuine right adjoint $\barhom$, see 
\cite[Prop.~3.14]{AnnoLogvinenko-BarCategoryOfModulesAndHomotopyAdjunctionForTensorFunctors}.
The homotopy category $H^0(\CmodbarD)$ is isomorphic to the big
derived category $D(\C\text{-}\D)$ of $\C$-$\D$-bimodules. 
It is generated by the representable bimodules. Since representable
bimodules are $\D$-perfect, the compact $1$-morphisms of
$\enhcatkcdg$ 
generate all $1$-morphisms of $H^0(\BarModtwocat)$. In particular, they 
generate all $1$-morphisms of $H^0(\enhcatkcdg)$. 
We conclude that $\BarModtwocat$ 
satisfies our assumptions in \S\ref{section-the-setting} and 
\S\ref{section-the-derived-category}. Thus we set $\B$ to be $\BarModtwocat$
to define the derived categories of $\Ainfty$-algebras in
$\enhcatkcdg$. 

We can now define:
\begin{defn}
Let $\C$ be an enhanced triangulated category. An \em enhanced exact
monad \rm over $\C$ is a strongly homotopy unital $\Ainfty$-algebra in 
$\CmodbarC$. 

Explicitly, it is a collection 
$(T,p_\bullet, \eta, h^l_\bullet, h^r_\bullet)$ where:
\begin{itemize}
\item $T \in \CmodbarC$ is an enhanced exact functor $\C \rightarrow \C$,
\item $p_\bullet$ is a collection of degree $2-i$ morphisms 
$p_i\colon T^i \rightarrow T$ in $\CmodbarC$ such that their 
bar-construction \eqref{eqn-nonaugmented-bar-construction-of-A-m_i} is 
a twisted complex.
\item $\eta$ is a morphism $\C \rightarrow T$ in $\CmodbarC$, 
\item $h^l_\bullet$ and $h^r_\bullet$ are the degree $-1$ 
morphisms $T \rightarrow T$ in the categories of left and 
right $\Ainfty$-$T$-modules in $\CmodbarC$ such that the strong
homotopy unitality conditions 
\eqref{eqn-strong-homotopy-unitality-conditions} hold. 
\end{itemize}

The underlying exact monad of $(T,p_\bullet)$ is 
$t\colon D_c(\C) \rightarrow D_c(\C)$ given by $(-) \bartimes_\C T$
with the operation $t^2 \rightarrow t$ given by $p_2$ and 
the unit $\id \rightarrow t$ given by $\eta$.  
\end{defn}
\begin{defn}
Let $\C$ be an enhanced triangulated category and $T$ an enhanced
monad over $\C$. The \em Eilenberg-Moore category $\eilmoor_T$ of 
$T$ \rm is the enhanced triangulated category 
$\noddinfhu\text{-}T^{\barperfC}$ of homotopy 
unital $\Ainfty$-modules over $T$ in $\barperfC$. 
\end{defn}

The DG category $\barperfC$ is pretriangulated and
Karoubi-complete. Indeed $H^0(\barperfC) = D_c(\C)$. Hence, 
when considered as enhanced triangulated category, its underlying 
triangulated category $D_c(\barperfC)$ is its homotopy category 
$H^0(\barperfC)$. Moreover, we have:
\begin{lemma}
Let $\C$ be an enhanced triangulated category and $T$ be an enhanced
exact monad over $\C$. 

DG category $\noddinfhu\text{-}T^{\barperfC}$ is 
pretriangulated and homotopy Karoubi complete. Its homotopy category
is the triangulated category underlying $\eilmoor_T$. 
\end{lemma}
\begin{proof}
Since $\barperfC$ is pretriangulated, so is 
$\noddinf\text{-}T^{\barperfC}$ by 
\cite{AnnoLogvinenko-UnboundedTwistedComplexes}, Cor.~5.13. 
Since $\noddinf\text{-}T^{\modbarC}$ is cocomplete, it is homotopy
Karoubi complete. Hence every homotopy idempotent of 
$\noddinf\text{-}T^{\barperfC}$ splits in
$\noddinf\text{-}T^{\modbarC}$. But if $(b,q_\bullet)$ is a homotopy
direct summand of some $(a,p_\bullet) \in
\noddinf\text{-}T^{\barperfC}$, then $b$ is homotopy direct summand 
of $a$ in $\modbarC$. Since $a$ is perfect, 
$(b, q_\bullet)$ lies in $\noddinf\text{-}T^{\barperfC}$. 

Thus $\noddinf\text{-}T^{\barperfC}$ is pretriangulated and
homotopy Karoubi complete. Hence so is $\noddinfhu\text{-}T^{\barperfC}$, 
since $H$-unitality is stable under taking convolutions of twisted complexes 
and homotopy direct summands. Hence 
$D_c(\noddinfhu\text{-}T^{\barperfC})$ 
the triangulated Karoubi-complete hull of
$\noddinfhu\text{-}T^{\barperfC}$ in 
$D(\noddinfhu\text{-}T^{\barperfC})$ is just 
$H^0(\noddinfhu\text{-}T^{\barperfC})$. 
\end{proof}

Let $t\colon D_c(\C) \rightarrow D_c(\C)$ be the exact monad 
$(-) \ldertimes T$ underlying $T$. Recall that the category 
$\eilmoor_{t}$ is not apriori triangulated or Karoubi complete.
We have a natural functor 
$$ D_c(\eilmoor_T) \rightarrow \eilmoor_{t} $$
which sends  
every $(a,p_\bullet) \in H^0(\noddinf^{\barperfC}_{hu}\text{-}T)$ to
$(a,p_2)$ and every $f_\bullet$ to $f_1$.

The category $\barperfC$ generates the whole of $H^0(\modbarC)$. 
By Prop.~\ref{prps-derived-category-of-A-is-triangulated-cocomplete-hull-of-freeA},
the derived category $D(T)$ of $T$ is $H^0(\noddinfhu\text{-}T^{\modbarC})$.  
By above, the objects of $D_c(\eilmoor_T)$ are the homotopy unital 
$\Ainfty$-$T$-modules $(a,p_\bullet)$ with $a$ a perfect $\C$-module. 
However, not every such $(a,p_\bullet)$ is
necessarily a perfect $T$-module, that is –– a compact object in $D(T)$. 
For a counterexample, see
Remark.~\ref{remark-not-all-perfect-object-modules-are-perfect}.

\begin{defn}
Let $\C$ be an enhanced triangulated category and $T$ be an enhanced
exact monad over $\C$. The \em perfect Eilenberg-Moore category
$\eilmoor^{pf}_T$ of $T$ \rm is the full subcategory of $\eilmoor_T$ 
consisting of perfect $\Ainfty$-$T$-modules. 
\end{defn}
 
As often with the subcategories of perfect objects, the perfect 
Eilenberg-Moore category is nicer to deal with than the full one. 
We now demonstrate that $\eilmoor^{pf}_T$ comprises all perfect
$\Ainfty$-$T$-modules in $\modbarC$ and hence enhances the compact
derived category of $T$:
\begin{theorem}
Let $\C$ be an enhanced triangulated category and $T$ be an enhanced
exact monad over $\C$. 

As a DG category, $\eilmoor^{pf}_T$ is pretriangulated and 
homotopy Karoubi complete. It contains the free $T$-modules and 
the full inclusion 
$$ \free\text{-}T \hookrightarrow \eilmoor^{pf}_T  $$
is an equivalence of enhanced triangulated categories. The underlying 
triangulated category is further equivalent to $D_c(T)$ and $D_c(\kleisli(T))$. 
\end{theorem}
\begin{proof}
Since $T$ is a right-perfect $\C$-$\C$-bimodule, bar tensoring with it on
the right sends perfects to perfects. Thus for any $a \in \barperfC$
the $\C$-module $Ta$ also lies in $\barperfC$. Hence free $T$-modules 
lie in $\eilmoor^{pf}_T$.

By definition, $\eilmoor^{pf}_T$ lies in 
the full subcategory $\noddinfhupf\text{-}T^{\modbarC}$ 
of $\noddinfhu\text{-}T^{\modbarC}$ consisting of perfect modules. 
By Prop.~\ref{prps-perfect-iff-lies-in-hperf-of-perfect-generator-frees}
every perfect $(a,p_\bullet) \in \noddinfhupf\text{-}T^{\modbarC}$ is 
a homotopy direct summand of something in $\pretriag(\free\text{-}T)$.
Hence $a$ is a homotopy direct summand of something in
$\pretriag(\barperfC)$ and thus itself perfect. 
Therefore $(a,p_\bullet)$ lies in 
$\eilmoor^{pf}_T$, and so $\eilmoor^{pf}_T = 
\noddinfhupf\text{-}T^{\modbarC}$. In particular, its homotopy
category is the triangulated Karoubi-complete hull of free modules.  

The remaining assertions now follow by Theorem 
\ref{theorem-compact-derived-category-of-A-is-that-of-frees-and-kleisli}. 
\end{proof}

\subsection{Adjunction monads and comonads}
\label{section-adjunction-monads-and-comonads}

In this section we continue working in the setting of enhanced Karoubi
complete triangulated categories of 
\S\ref{section-enhanced-monads}. We show that for any enhanced
functor its adjunction monad and comonad can be enhanced by 
strictly (co)associative but bimodule homotopy unital algebra and
coalgebra in $\enhcatkcdg$. We then show that therefore for any 
adjoint triple $(L,F,R)$ the monad $RF$ and the comonad $LF$ are
derived module-comodule equivalent. 

Let $\C$ and $\D$ be enhanced triangulated categories, i.e. objects in
$\enhcatkcdg$. Let $F\colon \C \rightarrow \D$ be an enhanced exact
functor, i.e. a $1$-morphism in $\enhcatkcdg$. 
By definition, $F$ is some right-perfect bimodule $M \in \CmodbarD$. 

Let $F$ have an enhanced right adjoint $R\colon \D \rightarrow \C$. 
By this we mean, that $F$ and $R$ are $2$-categorically adjoint in 
the homotopy bicategory $\enhcatkc$ of $\enhcatkcdg$. Their
underlying exact functors $f$ and $r$ are then also adjoint. 
Since $2$-categorical adjoints are unique, such $R$ is unique 
in $\enhcatkcdg$ up to homotopy equivalence.

In \cite[Theorem
4.1,Prop.~4.6]{AnnoLogvinenko-BarCategoryOfModulesAndHomotopyAdjunctionForTensorFunctors}
we showed that such $R$ exists if and only if the bar dual 
$M^{\barD} : = \barhom_{\D}(M,\D)$
is also a right-perfect $\D$-$\C$-bimodule. Moreover, in such case, 
up to homotopy equivalence, we can take $R = M^{\barD}$, the
adjunction counit $FR \xrightarrow{\counit} \id_\D$ to be the evaluation map 
\begin{equation}
\label{eqn-counit-of-F,R-adjunction}
\barhom_{\D}(M,\D) \bartimes_\C M \xrightarrow{\eval} \D,
\end{equation}
and the adjunction unit $\id_\C \xrightarrow{\unit} RF$ to be the composition 
\begin{equation}
\label{eqn-unit-of-F,R-adjunction}
\C \xrightarrow{\action} \barhom_{\D}(M,M) \xrightarrow{\zeta}
M \bartimes_\D \barhom_{\D}(M,\D),
\end{equation}
of the action map and any homotopy inverse $\zeta$ of the natural map 
$$ M \bartimes_\D \barhom_{\D}(M,\D) \xrightarrow{\eta} \barhom_{\D}(M,M). $$
See \cite[\S4.2]{AnnoLogvinenko-BarCategoryOfModulesAndHomotopyAdjunctionForTensorFunctors} for the definitions and technical details. 

Since $F$ and $R$ are homotopy adjoint, the operations 
$$ RFRF \xrightarrow{R\counit{F}} RF 
\quad \quad \text{ and } \quad \quad 
   \id \xrightarrow{unit} RF, $$
give the composition $RF$ the structure of a stricly unital algebra 
in the homotopy bicategory $\enhcatkc$. Similarly, the operations
$$ FR \xrightarrow{F\unit{R}} FRFR
\quad \quad \text{ and } \quad \quad 
FR \xrightarrow{\counit} \id, $$
give the composition $FR$ the structure of strictly unital coalgebra
in $\enhcatkc$. On the level of underlying exact functors these
become the adjunction monad $rf$ and comonad $fr$ of the adjunction 
$(f,r)$.  

As explained in \S\ref{section-enhanced-monads}, the above is not
enough to give $RF$ and $FR$ the structure of an enhanced monad and
comonad. We need to equip $RF$ and $FR$ with full $\Ainfty$- and 
strong/bimodule homotopy unitality structures in $\enhcatkcdg$ which 
descend to the above algebra and coalgebra structures in 
$\enhcatkc$. 

To simplify the computations involved, we modify 
a little the above setup. In
\cite[\S3.3]{AnnoLogvinenko-BarCategoryOfModulesAndHomotopyAdjunctionForTensorFunctors}
we have constructed natural quasi-inverse homotopy equivalences 
\begin{equation*}
\begin{tikzcd}
M \bartimes_\D \D 
\ar[shift left = 0.75]{r}{\alpha}
&
M
\ar[shift left = 0.75]{l}{\beta}
\end{tikzcd}
\quad \text{ and } \quad 
\begin{tikzcd}
M 
\ar[shift left = 0.75]{r}{\gamma}
&
\barhom_\D(\D,M).
\ar[shift left = 0.75]{l}{\delta}
\end{tikzcd}
\end{equation*}
We replace the bimodule $M$ by a homotopy equivalent bimodule 
$\barhom_\D(\D,M)$. This doesn't change the isomorphism class of the
corresponding $1$-morphism in $\enhcatkc$, and
hence doesn't change the enhanced functor. Henceforth we 
use $F$ to denote the $1$-morphism of $\enhcatkcdg$ defined by
$\barhom_\D(\D,M)$. 

Next, we construct the multiplication of $RF$ and 
the comultiplication of $FR$ which descend to the same
$2$-morphisms in $\enhcatkc$, under the identification of $M$ 
and $\barhom_\D(\D,M)$ provided by $\gamma$, as 
the homotopy adjunction multiplication and comultiplication above. 
For this, we need to make the following choice. By
\cite[Lemma
3.36]{AnnoLogvinenko-BarCategoryOfModulesAndHomotopyAdjunctionForTensorFunctors},
the map 
$$ \composition \colon\quad \barhom_\D(\D,M) \bartimes_\D \barhom_\D(M,\D)
\rightarrow \barhom_\D(M,M), $$
is a homotopy equivalence. We choose, once and for all, 
a homotopy inverse 
$$ \zeta\colon\quad \barhom_\D(M,M) \rightarrow \barhom_\D(\D,M)
\bartimes_\D \barhom_\D(M,\D) $$
and degree $-1$ endomorphisms
$$ \omega\colon\quad \barhom_\D(M,M) \rightarrow \barhom_\D(M,M), $$ 
$$ \omega'\colon\quad 
\barhom_\D(\D,M) \bartimes_\D \barhom_\D(M,\D)
\rightarrow 
\barhom_\D(\D,M) \bartimes_\D \barhom_\D(M,\D)$$
such that $d\omega = \composition \circ \zeta - \id$ and 
$d\omega' = \id - \zeta \circ \composition$. 

\begin{defn}
\label{defn-multiplication-and-comultiplication-for-RF-and-FR}
Let $\C$ and $\D$ be $DG$-categories and let $M \in \CmodD$ be
a right-perfect bimodule such that $M^{\barD}$ is also right-perfect. 
Let $F$ and $R$ be $\barhom_\D(\D,M)$ and $\barhom_{\D}(M,\D)$ considered 
as $1$-morphisms in $\enhcatkcdg$. 

Define $2$-morphism 
\begin{equation}
\mu\colon
\quad
RFRF
\quad \longrightarrow \quad
RF
\end{equation}
to be the $\CmodC$ map 
\begin{equation*}
\begin{tikzcd}
 \barhom_{\D}(\D,M) \bartimes_\D \barhom_{\D}(M,\D) \bartimes_\C \barhom_{\D}(\D,M)
\bartimes_\D \barhom_{\D}(M,\D) 
\ar{d}{\id \bartimes \composition^2}
\\
\barhom_{\D}(\D,M) \bartimes_\D \barhom_{\D}(M,\D),
\end{tikzcd}
\end{equation*} 
where $\composition$ is the map of taking the composition in $\modbarD$,
see
\cite[Defn.~3.11]{AnnoLogvinenko-BarCategoryOfModulesAndHomotopyAdjunctionForTensorFunctors}. 

Define $2$-morphism 
\begin{equation}
\Delta\colon \quad
FR 
\quad \longrightarrow \quad 
FRFR
\end{equation}
to be the following composition in $\DmodD$: 
\begin{equation*}
\begin{tikzcd}
\barhom_{\D}(M,\D) \bartimes_\C \barhom_\D(\D,M) 
\ar{d}{\beta \bartimes \id}
\\
\barhom_{\D}(M,\D) \bartimes_\C \C \bartimes_\C \barhom_\D(\D,M) 
\ar{d}{\id \bartimes \action \bartimes \id}
\\
\barhom_{\D}(M,\D) \bartimes_\C \barhom_{\D}(M,M) \bartimes_\C \barhom_\D(\D,M)
\ar{d}{\id \bartimes \zeta \bartimes \id}
\\
\barhom_{\D}(M,\D) \bartimes_\C \barhom_\D(\D,M) \bartimes_\D \barhom_{\D}(M,\D) 
\bartimes_\C M. 
\end{tikzcd}
\end{equation*}
\end{defn}
\begin{prps}
\label{prps-RF-LF-associativity-coassociativity}
Let $\C$ and $\D$ be $DG$-categories and let $M \in \CmodbarD$ be
a right-perfect bimodule such that $M^{\barD}$ is also right-perfect. 
Let $F$ and $R$ be $\barhom_\D(\D,M)$ and $\barhom_{\D}(M,\D)$ considered 
as $1$-morphisms in $\enhcatkcdg$. 

The $2$-morphisms $\mu\colon RFRF \rightarrow RF$ and $\Delta\colon FR
\rightarrow FRFR$ in Defn.~\ref{defn-multiplication-and-comultiplication-for-RF-and-FR}
are a strictly associative multiplication and 
a strictly coassociative comultiplication. 
\end{prps}
\begin{proof}
The multiplication $\mu\colon RFRF \rightarrow RF$ is strictly
associative because the composition in $\modbarD$ is strictly
associative: both $\mu \circ RF\mu$ and
$\mu \circ \mu{RF}$ are the same $2$-morphism $RFRFRF \rightarrow RF$
given by the following composition in $\CmodbarC$:
\begin{scriptsize}
\begin{equation*}
\begin{tikzcd}
\barhom_{\D}(\D,M) \bartimes_\D \barhom_{\D}(M,\D) \bartimes_\C\barhom_{\D}(\D,M)
\bartimes_\D \barhom_{\D}(M,\D) \bartimes_\C \barhom_{\D}(\D,M)
\bartimes_\D \barhom_{\D}(M,\D) 
\ar{d}{\id \bartimes \composition^4}
\\
\barhom_{\D}(\D,M) \bartimes_\D \barhom_{\D}(M,\D). 
\end{tikzcd}
\end{equation*}
\end{scriptsize}

The comultiplication $\Delta\colon FR \rightarrow FRFR$ is strictly
coassociative because the bar-complex $\barC$ is a strictly
coassociative coalgebra in $\CmodC$, see \cite[\S2.11]{AnnoLogvinenko-BarCategoryOfModulesAndHomotopyAdjunctionForTensorFunctors}.
Indeed, note that for any two bimodules $E$ and $F$ the bar category
morphisms 
$$ \beta \bartimes \id, \; \id \bartimes \beta\colon \quad  E \bartimes_C F \rightarrow E \bartimes_\C \C \bartimes_\C F $$
are equal and correspond to $\CmodC$ morphism applying 
the comultiplication $\barC \rightarrow \barC \otimes \barC$ to the copy 
of $\barC$ in betweed $E$ and $F$. We thus have a canonical
map 
$$ E \bartimes_C F \xrightarrow{\beta^n}  E \bartimes_C \C^{\bartimes
n} \bartimes_C F $$
which equals to the composition of any $n$ applications of $\beta$ to
any of the tensor factors involved. The corresponding non-bar morphism
applies $n$-tuple comultiplication to the copy $\barC$ in the middle 
of $E$ and $F$. 
 
In these terms, both $FR\Delta \circ \Delta$ and $\Delta{FR} \circ \Delta$ are the
same $2$-morphism $FR \rightarrow FRFRFR$ given by the $\DmodbarD$ 
morphism 
\begin{scriptsize}
\begin{equation*}
\begin{tikzcd}
\barhom_{\D}(M,\D) \bartimes_\C  M 
\ar{d}{\beta^2}
\\
\barhom_{\D}(M,\D) \bartimes_\C \C \bartimes_\C \C \bartimes_\C M 
\ar{d}{\id \bartimes \action \bartimes \action \bartimes \id}
\\
\barhom_{\D}(M,\D) \bartimes_\C \barhom_{\D}(M,M) \bartimes_\C \barhom_{\D}(M,M)\bartimes_\C M 
\ar{d}{\id \bartimes \zeta \bartimes \zeta \bartimes \id}
\\
\barhom_{\D}(M,\D) \bartimes_\C  \barhom_{\D}(\D,M) \bartimes_\D
\barhom_{\D}(M,\D) \bartimes_\C  \barhom_{\D}(\D,M) \bartimes_\D
\barhom_{\D}(M,\D) \bartimes_\C  \barhom_{\D}(\D,M). 
\end{tikzcd}
\end{equation*}
\end{scriptsize}
\end{proof}

We next show that $RF$ is bimodule homotopy unital, 
see Defn.~\ref{def-bimodule-homotopy-unitality} and more generally
\S\ref{section-unitality-conditions-for-algebras}. We also 
show that $FR$ is bicomodule homotopy counital. Moreover, 
most of the higher components of the bimodule homotopy unit
$\bareta_{\bullet\bullet}$ of $RF$ and bicomodule homotopy counit
$\barepsilon_{\bullet\bullet}$ of $FR$ are zero. 

\begin{remark}
\label{remark-bimodule-homotopy-unitality-via-four-morphisms-for-RF}
By Theorem \ref{theorem-conditions-for-bimodule-homotopy-unitality}, 
bimodule homotopy unitality of $RF$ is equivalent to the existence
of following four morphisms: 
\begin{itemize}
\item $\eta\colon \id \rightarrow RF$ in $\CmodbarC$ with $d\eta = 0$, 
\item $h^r_\bullet\colon \id RF \rightarrow RF$ in $\noddinf\text{-}RF$
such that $d h^r_\bullet = \alpha{F} - \mu \circ \eta RF$, 
\item $h^l_\bullet\colon RF\id \rightarrow RF$ in $RF\text{-}\noddinf$
such that $d h^l_\bullet = R{\alpha} - \mu \circ RF \eta$, 
\item $\kappa_{\bullet\bullet}\colon RF{\id}RF \rightarrow RF$ in  
$RF\text{-}\noddinf\text{-}RF$ such that $d \kappa_{\bullet \bullet} = \mu\circ( h^l_\bullet RF - RF h^r_\bullet).$
\end{itemize}
Recall that $\alpha$ denotes the unitor homotopy equivalences
of $\enhcatkcdg$, which were constructed explicitly
in \cite[\S3.3]{AnnoLogvinenko-BarCategoryOfModulesAndHomotopyAdjunctionForTensorFunctors}. The conditions on the four morphisms are simpler than in 
the original definition due to $(RF,\mu)$ being a strict algebra. 
The bimodule homotopy unit $\bareta_{\bullet}$ is composed of
these four as follows: $\eta = \bareta_{00}$, $h^r_i = (-1)^{i-1}
\bareta_{0i}$, $h^l_i = \bareta_{i0}$, and $\kappa_{ij} = 
\bareta_{(i+1)(j+1)}$. 
\end{remark}
\begin{remark}
\label{remark-bicomodule-homotopy-counitality-via-four-morphisms-for-FR}
By the coalgebra analogue of 
Theorem \ref{theorem-conditions-for-bimodule-homotopy-unitality}, 
bicomodule homotopy counitality of $FR$ is equivalent to the existence
of the following four morphisms:
\begin{itemize}
\item $\epsilon\colon FR \rightarrow \id$ in $\DmodbarD$ with
$d\epsilon = 0$, 
\item $g^r_\bullet\colon FR \rightarrow \id FR$ in $\conoddinf\text{-}FR$
such that $d g^r_\bullet = \beta{R} - \epsilon FR \circ \Delta$, 
\item $g^l_\bullet\colon FR \rightarrow FR\id$ in $FR\text{-}\conoddinf$
such that $d g^l_\bullet = {F}\beta - FR \epsilon \circ \Delta$, 
\item $\lambda_{\bullet\bullet}\colon FR \rightarrow FR{\id}FR$ in  
$FR\text{-}\conoddinf\text{-}FR$ with 
$d \lambda_{\bullet \bullet} = ( g^l_\bullet FR - FR g^r_\bullet)\circ
\Delta.$
\end{itemize}
Here $\beta$ are the natural homotopy equivalences homotopy inverse
to the unitors of $\enhcatkcdg$, which were constructed explicitly
in \cite[\S3.3]{AnnoLogvinenko-BarCategoryOfModulesAndHomotopyAdjunctionForTensorFunctors}.
\end{remark}

To construct these morphisms, we need the following auxilliary definition:

\begin{defn}
\label{defn-eta-epsilon-chi-chi'-lambda}
Let $\C$ and $\D$ be $DG$-categories and let $M \in \CmodbarD$ be
a right-perfect bimodule such that $M^{\barD}$ is also right-perfect. 
Let $F$ and $R$ be $\barhom_{\D}(\D,M)$ and $\barhom_{\D}(M,\D)$ 
considered as $1$-morphisms in $\enhcatkcdg$. 

We define the following $2$-morphisms in $\enhcatkcdg$:
\begin{itemize}
\item $\eta\colon \id \rightarrow RF$ is the degree $0$ morphism 
\begin{align*}
\C \xrightarrow{\action} \barhom_\D(M,M) \xrightarrow{\zeta}
\barhom_{\D}(\D,M) \bartimes_\D \barhom_{\D}(M,\D).
\end{align*}
\item $\epsilon\colon FR \rightarrow \id$ is the degree $0$ morphism 
\begin{align*}
\barhom_{\D}(M,\D) \bartimes_\C \barhom_{\D}(\D,M)  
\xrightarrow{\composition}
\barhom_{\D}(\D,\D) 
\xrightarrow{\delta}
\D.
\end{align*}
\item $\chi\colon F\id \rightarrow F$ and 
$\chi'\colon \id{R} \rightarrow R$ are 
the degree $-1$ morphisms
\begin{equation*}
\begin{tikzcd}
\C \bartimes_\C \barhom_{\D}(\D,M) 
\ar{d}{\action \bartimes \id} 
\\ 
\barhom_{\D}(M,M) \bartimes_\C \barhom_{\D}(\D,M)
\ar{d}{\omega \bartimes \id}
\\ 
\barhom_{\D}(M,M) \bartimes_\C \barhom_{\D}(\D,M)
\ar{d}{\composition}
\\
\barhom_{\D}(\D,M)
\end{tikzcd}
\quad \text{ and } \quad
\begin{tikzcd}
\barhom_{\D}(M, \D) \bartimes_\C \C
\ar{d}{\id \bartimes \action}
\\
\barhom_{\D}(M, \D) \bartimes_\C \barhom_{\D}(M,M) 
\ar{d}{\id \bartimes \omega}
\\
\barhom_{\D}(M, \D) \bartimes_\C \barhom_{\D}(M,M) 
\ar{d}{\composition}
\\
\barhom_{\D}(M, \D). 
\end{tikzcd}
\end{equation*}

\item 
$\xi\colon F \rightarrow \id F$ is the degree $-1$ morphism 
\begin{tiny}
\begin{equation*}
\begin{tikzcd}
\barhom_\D(\D,M)  
\ar{d}{\beta}
\\
\C \bartimes_\C \barhom_\D(M,\D)  
\ar{d}{\action \bartimes \id }
\\
\barhom_\D(M,M) \bartimes_\C \barhom_\D(\D,M) 
\ar{d}{\zeta \bartimes \id }
\\
\barhom_\D(\D,M) \bartimes_\D \barhom_\D(M,\D) \bartimes_\C \barhom_\D(\D,M) 
\ar{d}{\id \bartimes \composition} 
\\
\barhom_\D(\D,M) \bartimes_\D \barhom_\D(\D,\D)  
\ar{d}{\id \bartimes \delta}
\\
\barhom_\D(\D,M) \bartimes_\D \D
\ar{d}{\theta}
\\
\barhom_\D(\D,M) \bartimes_\D \D
\end{tikzcd}
\quad + \quad
\begin{tikzcd}
\barhom_\D(\D,M)  
\ar{d}{\beta}
\\
\C \bartimes_\C \barhom_\D(M,\D)  
\ar{d}{\action \bartimes \id }
\\
\barhom_\D(M,M) \bartimes_\C \barhom_\D(\D,M) 
\ar{d}{\zeta \bartimes \id }
\\
\barhom_\D(\D,M) \bartimes_\D \barhom_\D(M,\D) \bartimes_\C \barhom_\D(\D,M) 
\ar{d}{\id \bartimes \composition} 
\\
\barhom_\D(\D,M) \bartimes_\D \barhom_\D(\D,\D)  
\ar{d}{\id \bartimes \kappa}
\\
\barhom_\D(\D,M) \bartimes_\D \barhom_\D(\D,\D)  
\ar{d}{\composition}
\\
\barhom_\D(\D,M) 
\ar{d}{\beta}
\\
\barhom_\D(\D,M) \bartimes_\D \D
\end{tikzcd}
\end{equation*}
\end{tiny}
Here $\theta$ and $\kappa$ are the natural maps constructed in 
\cite[\S3.3]{AnnoLogvinenko-BarCategoryOfModulesAndHomotopyAdjunctionForTensorFunctors}
such that $d\theta = \id - \beta \circ \alpha$ and $d\kappa = \gamma \circ
\delta - \id$. 

\item 
$\xi'\colon R \rightarrow R\id$ is the degree $-1$ morphism 
\begin{tiny}
\begin{equation*}
\begin{tikzcd}
\barhom_\D(M,\D)  
\ar{d}{\beta}
\\
\barhom_\D(M,\D) \bartimes_\C \C 
\ar{d}{\id \bartimes \action }
\\
\barhom_\D(M,\D) \bartimes_\C 
\barhom_\D(M,M) 
\ar{d}{\id \bartimes \zeta }
\\
\barhom_\D(M,\D) \bartimes_\C 
\barhom_\D(\D,M) \bartimes_\D \barhom_\D(M,\D)  
\ar{d}{\composition \bartimes \id}
\\
\barhom_\D(\D,\D) \bartimes_\D \barhom_\D(M,\D)  
\ar{d}{\delta \bartimes \id}
\\
\D \bartimes_\D \barhom_\D(M,\D)
\ar{d}{\theta}
\\
\D \bartimes_\D \barhom_\D(M,\D).
\end{tikzcd}
\quad + \quad
\begin{tikzcd}
\barhom_\D(M,\D)  
\ar{d}{\beta}
\\
\barhom_\D(M,\D) \bartimes_\C \C 
\ar{d}{\id \bartimes \action }
\\
\barhom_\D(M,\D) \bartimes_\C 
\barhom_\D(M,M) 
\ar{d}{\id \bartimes \zeta }
\\
\barhom_\D(M,\D) \bartimes_\C 
\barhom_\D(\D,M) \bartimes_\D \barhom_\D(M,\D)  
\ar{d}{\composition \bartimes \id}
\\
\barhom_\D(\D,\D) \bartimes_\D \barhom_\D(M,\D)  
\ar{d}{\kappa \bartimes \id}
\\
\barhom_\D(\D,\D) \bartimes_\D \barhom_\D(M,\D)  
\ar{d}{\composition}
\\
\barhom_\D(M,\D)
\ar{d}{\beta}
\\
D \bartimes_\D \barhom_\D(M,\D).
\end{tikzcd}
\end{equation*}
\end{tiny}

\item $\lambda\colon FR \rightarrow FR{\id}FR$
is the degree $-2$ morphism in $\DmodbarD$ defined as follows. 
First, let $\nu$ be the degree $-2$ morphism 
\begin{small}
\begin{equation*}
\begin{tikzcd}
\barhom_\D(M,\D) \bartimes_\C \barhom_\D(\D,M) 
\ar{d}{\beta^2}
\\
\barhom_\D(M,\D) \bartimes_\C 
\C \bartimes_\C \C
\bartimes_\C \barhom_\D(\D,M) 
\ar{d}{\id \bartimes \action \bartimes \action \bartimes \id}
\\
\barhom_\D(M,\D) \bartimes_\C 
\barhom_\D(M,M) \bartimes_\C \barhom_\D(M,M) 
\bartimes_\C \barhom_\D(\D,M) 
\ar{d}{\id \bartimes \omega \bartimes \omega \bartimes \id}
\\
\barhom_\D(M,\D) \bartimes_\C 
\barhom_\D(M,M) \bartimes_\C \barhom_\D(M,M) 
\bartimes_\C \barhom_\D(\D,M) 
\ar{d}{\id \bartimes \composition \bartimes \id}
\\
\barhom_\D(M,\D) \bartimes_\C 
\barhom_\D(M,M) 
\bartimes_\C \barhom_\D(\D,M). 
\end{tikzcd}
\end{equation*}
\end{small}
Its differential $d\nu$ is the composition 
of the closed degree $-1$ morphism  
\begin{equation}
\label{eqn-h^lFR-RFh^r-circ-Delta}
FR \xrightarrow{\Delta}
FRFR \xrightarrow{
FR(\beta \circ \chi \circ \beta + \xi)R - F(\beta \circ \chi' \circ
\beta + \xi')FR }
FR{\id}FR,
\end{equation}
with the homotopy equivalence 
$F\bigl(\composition \circ (\alpha \bartimes \id)\bigr)R$:
\begin{small}
\begin{equation*}
\barhom_\D(M,\D) \bartimes_\C 
\left(
\begin{tikzcd}
\barhom_\D(\D,M) 
\bartimes_\D \D \bartimes_\D
\barhom_\D(M,\D) \ar{d}{\alpha \bartimes \id}
\\
\barhom_\D(\D,M) 
\bartimes_\D
\barhom_\D(M,\D) 
\ar{d}{\composition}
\\
\barhom_\D(M,M) 
\end{tikzcd}
\right)
\bartimes_\C \barhom_\D(\D,M).
\end{equation*}
\end{small}
In \cite[Lemma
4.8(1)]{AnnoLogvinenko-BarCategoryOfModulesAndHomotopyAdjunctionForTensorFunctors}
we showed that a homotopy lift of a composition of 
a closed morphism with a homotopy equivalence induces
a homotopy lift of the morphism itself.  
We define $\lambda$ to be the homotopy lift of
\eqref{eqn-h^lFR-RFh^r-circ-Delta} induced by $\nu$. 
Indeed,  
the homotopy equivalence $\composition \circ (\alpha \bartimes \id)$
has a homotopy inverse $(\beta \bartimes \id) \circ \zeta$ and 
$$\upsilon := \quad
(\beta \bartimes \id) \circ \omega' \circ (\alpha \bartimes \id)
+ \theta \bartimes \id$$
is a homotopy between $\id$ and their composition. 
We thus define
\begin{align}
\label{eqn-lambda-definition}
\lambda := \quad F\Bigl(\upsilon\Bigr)R \circ \eqref{eqn-h^lFR-RFh^r-circ-Delta} 
+ 
F\Bigl(\composition \circ (\alpha \bartimes \id)\Bigr)R \circ \nu. 
\end{align}
\end{itemize}
\end{defn}

We then have:

\begin{prps}
\label{prps-RF-LF-unitality-counitality}
Let $\C$ and $\D$ be $DG$-categories and $M \in \CmodbarD$ be
a right-perfect bimodule with $M^{\barD}$ also right-perfect. 
Let $F$ and $R$ be $\barhom_\D(\D,M)$ and $\barhom_{\D}(M,\D)$ considered 
as $1$-morphisms in $\enhcatkcdg$. 
Let $\mu$,$\Delta$ be as in 
Defn.~\ref{defn-multiplication-and-comultiplication-for-RF-and-FR}
and $\eta$, $\epsilon$, $\chi$, $\chi'$, and $\lambda$ be as in 
Defn.~\ref{defn-eta-epsilon-chi-chi'-lambda}. Finally, let $\beta$ be
the natural homotopy inverses of the unitor homotopy equivalences
$\alpha$ of $\enhcatkcdg$ constructed in
\cite[\S3.3]{AnnoLogvinenko-BarCategoryOfModulesAndHomotopyAdjunctionForTensorFunctors}. 

Setting 
$(\eta,h^r_\bullet, h^l_\bullet, \kappa_{\bullet\bullet})$
in Remark \ref{remark-bimodule-homotopy-unitality-via-four-morphisms-for-RF}
to be $(\unit,-\chi'{F}, -R\chi, 0)$ defines 
a bimodule homotopy unital structure on the strict algebra $(RF,\mu)$. 
Setting $(\epsilon, g^r_\bullet, g^l_\bullet,
\lambda_{\bullet\bullet})$ 
in Remark \ref{remark-bicomodule-homotopy-counitality-via-four-morphisms-for-FR}
to be $(\counit, -(\beta \circ \chi \circ \beta + \xi){R}, -{F}(\beta \circ
\chi' \circ \beta +\xi'), \lambda)$ a bicomodule homotopy 
counital structure on the strict coalgebra $(FR,\Delta)$.  
\end{prps}
Here we view $\CmodbarC$ morphisms $\chi'F$, $R\chi$ and
$\DmodbarD$ morphisms $(\beta \circ \chi + \xi){R}$, ${F}(\beta \circ \chi'+\xi')$ as strict 
$\Ainfty$-morphisms in $\noddinf\text{-}RF$, $RF\text{-}\noddinf$,
$\conoddinf\text{-}FR$, and $FR\text{-}\conoddinf$. 
\begin{proof}
The maps $\eta$ and $\epsilon$ are closed of degree $0$ 
by their definitions in Defn.~\ref{defn-eta-epsilon-chi-chi'-lambda}. 
Hence conditions $d\eta = 0$ and $d\epsilon = 0$ hold.  

Recall the definition of the comultiplication $\Delta$ in
Defn.~\ref{defn-multiplication-and-comultiplication-for-RF-and-FR}. 
Since $\beta \bartimes \id$ in it can be equally written as $\id \bartimes \beta$, we can write $\Delta$ both as a morphism 
$F(R \rightarrow RFR)$ and $(F \rightarrow FRF)R$. It follows
that $(\beta \circ {\chi} \circ \beta + \xi)R$ and $F(\beta \circ
\chi' \circ \beta + \xi')$ are strict morphisms of right and
left $FR$-comodules, respectively. We conclude that the differentials of 
$-(\beta \circ \chi \circ \beta + \xi)R$ and 
$-F(\beta \circ \chi' \circ \beta + \xi')$ as $\Ainfty$-morphisms of right 
and left $FR$-comodules are just their differentials as morphisms 
in $\DmodbarD$. 

Similarly, by the definition of the multiplication $\mu$  in
Defn.~\ref{defn-multiplication-and-comultiplication-for-RF-and-FR}
it is a morphism $(RFR \rightarrow R)F$, whence $R\chi$ 
is a strict morphisms of left $RF$-modules. Finally, 
we check by hand that $\chi'F$ 
is a strict morphism of right $RF$-modules. Indeed, 
both $\mu \circ \chi'FRF$ and $\chi'F \circ \mu$ equal the following
morphism ${\id}RFRF \rightarrow RF$ in $\CmodbarC$:
\begin{scriptsize}
\begin{equation}
\begin{tikzcd}
\barhom_\D(\D,M) \bartimes_\D \barhom_\D(M,\D) \bartimes_\C \barhom_\D(\D,M)
\bartimes_\D \barhom_\D(M,D) \bartimes_\C \C
\ar{d}{\id^4 \bartimes (\omega \circ \action)}
\\
\barhom_\D(\D,M) \bartimes_\D \barhom_\D(M,\D) \bartimes_\C \barhom_\D(\D,M)
\bartimes_\D \barhom_\D(M,D) \bartimes_\C \barhom_\D(M,M) 
\ar{d}{\id \bartimes \composition^3}
\\
\barhom_\D(\D,M) \bartimes_\D \barhom_\D(M,\D). 
\end{tikzcd}
\end{equation}
\end{scriptsize}
We conclude that the differentials of $-\chi'F$ and $ - \chi R$ 
as $\Ainfty$-morphisms of right and left $FR$-modules
are the same as their differentials as morphisms in $\CmodbarC$. 

It can now be readily verified that we have in $\CmodbarC$ and $\DmodbarD$ 
\begin{align}
\id{RF} \xrightarrow{\eta{RF}} RFRF \xrightarrow{\mu} RF \quad &= 
\alpha{F} + d(\chi'{F}), \\
RF\id \xrightarrow{{RF}\eta} RFRF \xrightarrow{\mu} RF \quad &= 
R\alpha + d(R\chi)), \\
FR \xrightarrow{\Delta} FRFR \xrightarrow{\epsilon{FR}} \id{FR} \quad &=
\beta{R} + d((\beta \circ \chi \circ \beta){R} + \xi{R}), \\
\label{eqn-Delta-FR-epsilon}
FR \xrightarrow{\Delta} FRFR \xrightarrow{{FR}\epsilon} FR\id \quad &=
F\beta + d(F(\beta \circ \chi' \circ \beta) + F\xi). 
\end{align}
We verify \eqref{eqn-Delta-FR-epsilon}, the others are similar.
On its LHS we have $FR\epsilon \circ \Delta$:
\begin{tiny}
\begin{equation*}
\begin{tikzcd}
\barhom_\D(M,\D) \bartimes_\C \barhom_\D(\D,M) 
\ar{d}{\id \bartimes \beta = \beta \bartimes \id}
\\
\barhom_\D(M,\D) \bartimes_\C 
\C \bartimes_\C
\barhom_\D(\D,M) 
\ar{d}{\id \bartimes \action \bartimes \id}
\\
\barhom_\D(M,\D) \bartimes_\C 
\barhom_\D(M,M) \bartimes_\C
\barhom_\D(\D,M) 
\ar{d}{\id \bartimes \zeta \bartimes \id}
\\
\barhom_\D(M,\D) \bartimes_\C 
\barhom_\D(\D,M) \bartimes_\D \barhom_\D(M,\D) \bartimes_\C
\barhom_\D(\D,M) 
\ar{d}{\composition \bartimes \id^2}
\\
\barhom_\D(\D,\D) \bartimes_\D \barhom_\D(M,\D) \bartimes_\C
\barhom_\D(\D,M) 
\ar{d}{\delta \bartimes \id^2}
\\
\D \bartimes_\D \barhom_\D(M,\D) \bartimes_\C
\barhom_\D(\D,M).
\end{tikzcd}
\end{equation*}
\end{tiny}
Observe that the first summand of $F\xi'$ is the postcomposition
of this with $F\theta$. Since $d\theta = \id - \beta \circ \alpha$, 
the first summand of $F\xi'$ gives a homotopy from
$FR\epsilon \circ \Delta$ to $F\beta \circ F\alpha \circ 
FR\epsilon \circ \Delta$, i.e. to the postcomposition of the above
with 
\begin{tiny}
\begin{equation*}
\begin{tikzcd}
\D \bartimes_\D \barhom_\D(M,\D) \bartimes_\C
\barhom_\D(\D,M)
\ar{d}{\alpha \otimes \id}
\\
\barhom_\D(M,\D) \bartimes_\C \barhom_\D(\D,M)
\ar{d}{\beta \otimes \id}
\\
D \bartimes_\D \barhom_\D(M,\D) \bartimes_\C
\barhom_\D(\D,M). 
\end{tikzcd}
\end{equation*}
\end{tiny}
This can be further rewritten as
\begin{tiny}
\begin{equation*}
\begin{tikzcd}
\D \bartimes_\D \barhom_\D(M,\D) \bartimes_\C
\barhom_\D(\D,M)
\ar{d}{\gamma \otimes \id^2}
\\ 
\barhom_\D(\D,\D) \bartimes_\D \barhom_\D(M,\D) \bartimes_\C
\barhom_\D(\D,M)
\ar{d}{\composition \otimes \id}
\\
\barhom_\D(M,\D) \bartimes_\C \barhom_\D(\D,M)
\ar{d}{\beta \otimes \id}
\\
\bartimes_\D \barhom_\D(M,\D) \bartimes_\C \barhom_\D(\D,M). 
\end{tikzcd}
\end{equation*}
\end{tiny}
Since $d\kappa = \gamma \circ \delta - \id$, the second summand of 
$F\xi'$ gives the homotopy from $F\beta \circ F\alpha \circ 
FR\epsilon \circ \Delta$ to the composition  
\begin{tiny}
\begin{equation*}
\begin{tikzcd}
\barhom_\D(M,\D) \bartimes_\C \barhom_\D(\D,M) 
\ar{d}{\id \bartimes \beta = \beta \bartimes \id}
\\
\barhom_\D(M,\D) \bartimes_\C 
\C \bartimes_\C
\barhom_\D(\D,M) 
\ar{d}{\id \bartimes \action \bartimes \id}
\\
\barhom_\D(M,\D) \bartimes_\C 
\barhom_\D(M,M) \bartimes_\C
\barhom_\D(\D,M) 
\ar{d}{\id \bartimes \zeta \bartimes \id}
\\
\barhom_\D(M,\D) \bartimes_\C 
\barhom_\D(\D,M) \bartimes_\D \barhom_\D(M,\D) \bartimes_\C
\barhom_\D(\D,M) 
\ar{d}{\composition^2 \bartimes \id}
\\
\barhom_\D(M,\D) \bartimes_\C \barhom_\D(\D,M)
\ar{d}{\beta \bartimes \id}
\\
\bartimes_\D \barhom_\D(M,\D) \bartimes_\C \barhom_\D(\D,M).
\end{tikzcd}
\end{equation*}
\end{tiny}
Since $d \omega = \composition \circ \zeta - \id$, $F(\beta \circ
\chi' \circ \beta)$ gives the homotopy from this to 
\begin{tiny}
\begin{equation*}
\begin{tikzcd}
\barhom_\D(M,\D) \bartimes_\C \barhom_\D(\D,M) 
\ar{d}{\id \bartimes \beta = \beta \bartimes \id}
\\
\barhom_\D(M,\D) \bartimes_\C 
\C \bartimes_\C
\barhom_\D(\D,M) 
\ar{d}{\id \bartimes \action \bartimes \id}
\\
\barhom_\D(M,\D) \bartimes_\C 
\barhom_\D(M,M) \bartimes_\C
\barhom_\D(\D,M) 
\ar{d}{\composition \bartimes \id}
\\
\barhom_\D(M,\D) \bartimes_\C \barhom_\D(\D,M)
\ar{d}{\beta \bartimes \id}
\\
\bartimes_\D \barhom_\D(M,\D) \bartimes_\C \barhom_\D(\D,M).
\end{tikzcd}
\end{equation*}
\end{tiny}
Since $\composition \circ \id \bartimes \action \circ \beta = \id$,
the above is just $F\beta$, as desired. 
 
By above, it follows that
\begin{itemize}
\item  
$d h^r_\bullet = \id - \mu \circ \eta RF$ in $\noddinf\text{-}RF$, 
\item 
$d h^l_\bullet = \id - \mu \circ RF \eta$ in $RF\text{-}\noddinf$,
\item
$d g^r_\bullet = \id - \epsilon FR \circ \Delta$ in
$\conoddinf\text{-}FR$, 
\item
$d g^l_\bullet = \id - FR \epsilon \circ \Delta$ in 
$FR\text{-}\conoddinf$. 
\end{itemize}


Finally, we verify the conditions 
\begin{itemize}
\item $\mu\circ( h^l_\bullet RF - RF h^r_\bullet) = 0$,
\item $(g^l_\bullet FR - FR g^r_\bullet)\circ \Delta = d \lambda_{\bullet \bullet}.$
\end{itemize}
For the former, it is sufficient to note that by associativity of the
composition in $\modbarD$ both $\mu \circ R\chi{RF}$ and 
$\mu \circ RF\chi'F$ equal the same $\CmodbarC$ composition
\begin{tiny}
\begin{equation*}
\begin{tikzcd}
\barhom_\D(\D,M) \bartimes_\D \barhom_\D(M,\D) \bartimes_\C 
\C 
\bartimes_\C \barhom_\D(\D,M) \bartimes_\D \barhom_\D(M,\D)
\ar{d}{\id^2 \bartimes (\omega \circ \action) \bartimes \id^2}
\\
\barhom_\D(\D,M) \bartimes_\D \barhom_\D(M,\D) \bartimes_\C 
\barhom_\D(M,M)
\bartimes_\C \barhom_\D(\D,M) \bartimes_\D \barhom_\D(M,\D)
\ar{d}{\id \bartimes \composition^3}
\\
\barhom_\D(\D,M) \bartimes_\D \barhom_\D(M,\D)
\end{tikzcd}
\end{equation*}
\end{tiny}

For $(g^l_\bullet FR - FR g^r_\bullet)\circ\Delta =
d\lambda_{\bullet\bullet}$,
by definition of $\lambda$ we have in $\DmodbarD$
$$ d\lambda = (-F(\beta \circ \chi' \circ \beta + \xi')FR + FR(\beta
\circ \chi \circ \beta + \xi)R) \circ \Delta. $$
Next, observe that both summands in the definition \eqref{eqn-lambda-definition}
of $\lambda$ are of the form 
$$ FR \xrightarrow{F\beta = \beta R} F{\id}R \xrightarrow{ 
F(...)R} FR{\id}FR $$
and any such morphism is readily seen to be a morphism of $FR$-$FR$
bicomodules. Thus the differential of $\lambda$ as an
$\Ainfty$-morphism of $FR$-$FR$-bicomodules is the same as its
differential as a morphism in $\DmodbarD$. It follows that 
$$d\lambda_{\bullet\bullet} = (g^l_\bullet FR - FR g^r_\bullet)\circ\Delta, $$
as desired. 
\end{proof}

We thus have the first main result of this section.  It tells us that
for any adjunction of enhanced exact functors, the adjunction monad
and comonad of their underlying exact functors can be enhanced by an
enhanced monad and comonad which are strictly (co)associative and
bi(co)module homotopy (co)unital:
\begin{theorem}
\label{theorem-bimodule-homotopy-unitality-for-enhanced-adjunction-monads-and-comonads}
Let $\C$ and $\D$ be enhanced triangulated categories, 
$F\colon \C \rightarrow \D$ be an enhanced exact functor, and 
$R\colon \D \rightarrow \C$ its right adjoint. That is, 
$\C$ and $\D$ are objects in $\enhcatkcdg$, and $F$ and $R$ are
$2$-categorically homotopy adjoint $1$-morphisms. 

Let $M \in \CmodbarD$ be the bimodule defining $F$. 
Replace $F$ and $R$ by homotopy equivalent $1$-morphisms 
$\barhom_\D(\D,M)$ and $\barhom_{\D}(M,\D)$. 
In other words, choose
certain equivalent enhancements of the same pair of adjoint exact functors. 

$RF$ with the multiplication of  
Prps.~\ref{prps-RF-LF-associativity-coassociativity} and the unital 
structure of Prps.~\ref{prps-RF-LF-unitality-counitality} is a
strictly associative, bimodule homotopy unital enhanced monad. 
$FR$ with the comultiplication of  
Prps.~\ref{prps-RF-LF-associativity-coassociativity} and the counital 
structure of Prps.~\ref{prps-RF-LF-unitality-counitality} is a
strictly coassociative, bicomodule homotopy counital enhanced comonad. 
\end{theorem}

It follows that for an adjoint triple $(L,F,R)$ of enhanced functors
monad $RF$ and comonad $LF$ are derived module-comodule equivalent:

\begin{theorem}[Module-comodule correspondence for enhanced adjoint
triples] 
\label{theorem-module-comodule-correspondence-for-enhanced-adjoint-triples}

Let $\C$ and $\D$ be enhanced triangulated categories, 
$F\colon \C \rightarrow \D$ be an enhanced exact functor, and 
$L,R\colon \D \rightarrow \C$ its left and right adjoints. That is, 
$\C$ and $\D$ are objects in $\enhcatkcdg$, and $L$, $F$, and $R$ are
$2$-categorically homotopy adjoint $1$-morphisms. 

Let $M \in \CmodbarD$ be the bimodule defining $F$. 
Replace $L$, $F$ and $R$ by homotopy equivalent $1$-morphisms 
$\barhom_\C(M,\C)$, $\barhom_\D(\C,M)$ and $\barhom_{\D}(M,\D)$. 
In other words, choose certain equivalent enhancements of the same 
pair of adjoint exact functors.

Equip $RF$ with the structure of a strongly homotopy unital strict
algebra of 
Prop.~\ref{prps-LF-and-RF-are-homotopy-adjoint-in-a-monoidal-category}
and equip $LF$ with the structure of a bimodule homotopy counital
strict coalgebra of Theorem
\ref{theorem-bimodule-homotopy-unitality-for-enhanced-adjunction-monads-and-comonads}.
Then $RF$ and $LF$ are derived module-comodule equivalent:
$$ D_c(RF) \simeq D_c(LF). $$
\end{theorem}
\begin{proof}
By
\cite[Prop.~4.6]{AnnoLogvinenko-BarCategoryOfModulesAndHomotopyAdjunctionForTensorFunctors},
our replacements for $L$, $F$, and $R$ form a homotopy adjoint triple of
$1$-morphisms in $\enhcatkcdg$. By
Prop.~\ref{prps-LF-and-RF-are-homotopy-adjoint-in-a-monoidal-category}
it follows that $RF$ and $LF$ are homotopy adjoint as an algebra and a 
coalgebra. Since $RF$ is strongly homotopy unital and $LF$ is bimodule
homotopy counital, they are derived module-comodule equivalent by Theorem
\ref{theorem-module-comodule-correspondence-in-a-monoidal-dg-category}. 
\end{proof}

\bibliography{references}
\bibliographystyle{amsalpha}
\end{document}
`
