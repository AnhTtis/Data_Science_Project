\documentclass[12pt]{article}

\textheight=23.2cm
\textwidth=17.3cm

\oddsidemargin=-0.3cm
\evensidemargin=-0.3cm
\topmargin=-1.5cm

%\pdfoutput=1

%\setlength{\textwidth}{45zw}
%\setlength{\textheight}{38\baselineskip}
%\setlength{\oddsidemargin}{-0.2cm}
%\setlength{\topmargin}{0.7cm}
%\iftombow
%  \addtolength{\topmargin}{-1in}
%\else
%  \addtolength{\topmargin}{-1truein}
%\fi
\usepackage[dvipdfmx]{graphicx}
\usepackage{subcaption}
%\usepackage{graphicx}
%\usepackage{feynmp}
\usepackage{listings,jvlisting}
\usepackage{amsmath,amssymb}
\usepackage{bm}
\usepackage{graphicx, color}
\usepackage{wrapfig}
\usepackage{cite}
\usepackage{braket}
\DeclareMathOperator{\Tr}{Tr}
%\usepackage{hhline}
%\setcounter{tocdepth}{3}
\begin{document}
\title{
\begin{flushright}
\ \\*[-80pt]
\begin{minipage}{0.2\linewidth}
\normalsize
%arXiv:YYMM.NNNN \\
HUPD-2302 \\*[50pt]
\end{minipage}
\end{flushright}
{\Large \bf
$W$ Boson Mass and Grand Unification\\
via the Type-$\rm{I\hspace{-.01em}I}$ Seesaw-like Mechanism
}\\*[20pt]}


\author{
\centerline{
Yusuke~Shimizu $^{1,2}$\footnote{yu-shimizu@hiroshima-u.ac.jp}~and ~Shonosuke~Takeshita $^{1}$\footnote{shonosuke@hiroshima-u.ac.jp}}
\\*[20pt]
\centerline{
\begin{minipage}{\linewidth}
\begin{center}
$^1${\it \normalsize
Physics Program, Graduate School of Advanced Science \\ and Engineering,~Hiroshima~University, \\
Higashi-Hiroshima~739-8526,~Japan \\*[5pt]
$^2${\it \normalsize
Core of Research for the Energetic Universe, Hiroshima University, \\
Higashi-Hiroshima 739-8526, Japan}
}
\end{center}
\end{minipage}}
\\*[50pt]}

\date{
\centerline{\small \bf Abstract}
\begin{minipage}{0.9\linewidth}
\medskip
\medskip
\small
We propose an SU(5) GUT model added two pairs of $\mathbf{10}$ representation for the vector-like fermions to the minimal SU(5) GUT model.
By a real $\mathrm{SU(2)_L}$ triplet coming from the $\mathbf{24}$ representation Higgs getting the VEVs, the $W$ boson mass anomaly reported by the CDF collaboration can be explained.
The vector-like quark doublet acquire the mass through the Yukawa interaction of a real triplet via the type-$\rm{I\hspace{-.01em}I}$ seesaw-like mechanism in addition to that of $\mathbf{24}$ representation Higgs.
We assume that the mass for the vector-like quark doublet is expressed in terms of a real triplet mass.
By combining the constraints of the vector-like quark doublet mass with that of the heavy Higgs boson masses, we can obtain the allowed mass ranges for the vector-like quark doublet and a real triplet scalar which are very narrow.
Therefore, our model can be tested by the search of these particles in the near future.
In addition, in the case that included the new contribution, the SM gauge couplings unify successfully at $M_{\mathrm{GUT}}\thickapprox6.76\times10^{15}$~GeV and our model is testable by the future Hyper-Kamiokande experiment expected as $\tau_p(p\to\pi^0{e^+})<1.0\times10^{35}$~years.
\end{minipage}
}

\begin{titlepage}
\maketitle
\thispagestyle{empty}
\end{titlepage}
\newpage
%------------------------------------------------------------------------------%
%--------------------------------   Introduction   --------------------------------%
%------------------------------------------------------------------------------%
\section{Introduction}
\label{sec:Intro}

The Grand Unified Theory~(GUT) is one of the attractive theories to approach the new physics beyond the Standard Model~(SM).
The GUT embed the SM gauge groups, $\mathrm{SU(3)_C\times{SU(2)_L}\times{U(1)_Y}}$ into a large simple group.
Therefore, the GUT unify strong, weak, and electromagnetic forces.
In addition to the unification, the GUT can explain the quantisation of the electric charge.
Furthermore, the GUT predicts the proton decay since the SM quarks and leptons are unified into irreducible representations of the GUT group.
Then, the GUT is testable by the proton decay search.
The minimal GUT model is the SU(5) one~\cite{Georgi:1974sy} that is proposed by Georgi and Grashow in 1974.
In the minimal SU(5) GUT model, the SM matter fields and gauge fields are unified into $\mathbf{\bar{5}}$, $\mathbf{10}$, and $\mathbf{24}$ adjoint representations, respectively.
The scalar fields of this model are into $\mathbf{5}$ and $\mathbf{24}$ reprsentations and these fields realize the spontaneous symmetry breaking by taking the vacuum expectation values~(VEVs).
However, the minimal SU(5) GUT model is ruled out on the following two reasons:
First, it fails to unify the SM gauge couplings successfully at high energy only by the SM fields.
Second, the proton lifetime predicted by the minimal SU(5) GUT model is $\tau_p(p\to\pi^{0}e^+)\approx10^{30}$-$10^{31}$~years~\cite{Georgi:1974yf} and it is inconsistent with the current experimental result reported by Super-Kamiokande experiment as $\tau_p(p\to\pi^{0}e^+)>2.4\times10^{34}$~years~\cite{Super-Kamiokande:2020wjk}.

Last year, the CDF collaboration has reported a new result of the $W$ boson mass~\cite{CDF:2022hxs}, $M_W^{\mathrm{CDF}}=80.4335\pm0.0094$~GeV.
The average result of combining the CDF results with previous results from LEP2, Tevatron, LHC, and LHCb experiments is $M_W^{\mathrm{exp}}=80.4133\pm0.0080$~GeV~\cite{deBlas:2022hdk}.
This value is 6.5 $\sigma$ derivations away from the SM prediction as $M_W^{\mathrm{SM}}=80.3500\pm0.0056$~GeV~\cite{deBlas:2022hdk}.
Then, physics beyond the SM is required to explain this shift of the $W$ boson mass.
One of the candidates for explaining the $W$ boson mass anomaly is adding a real $\mathrm{SU(2)_L}$ triplet scalar with zero hypercharge~\cite{Ross:1975fq, Gunion:1989ci, Lynn:1990zk, Blank:1997qa, Forshaw:2003kh, Chen:2006pb, Chankowski:2006hs, Chivukula:2007koj, Bandyopadhyay:2020otm, FileviezPerez:2022lxp, Wu:2022uwk}.
By the $\mathrm{SU(2)_L}$ triplet getting the VEVs, this contributes to the $W$ boson mass at tree level.
In the minimal SU(5) model, the real $\mathbf{24}$ representation Higgs contains the real $\mathrm{SU(2)_L}$ triplet scalar with zero hypercharge.
Then, the SU(5) GUT can explain the $W$ boson mass anomaly by a real triplet coming from the $\mathbf{24}$ representation Higgs getting the VEVs~\cite{Evans:2022dgq, Senjanovic:2022zwy, Calibbi:2022wko}.
The other candidates for explaining the $W$ boson mass anomaly in the framework of SO(10) GUT~\cite{Fritzsch:1974nn} or type-$\rm{I\hspace{-.01em}I}$ seesaw mechanism~\cite{Magg:1980ut, Cheng:1980qt, Lazarides:1980nt, Mohapatra:1980yp} are shown in Refs.~\cite{Lazarides:2022spe, Chao:2022blc}.

In this paper, we propose an SU(5) GUT model added two pairs of $\mathbf{10}$ representation for the vector-like fermions to the minimal SU(5) GUT model.
By a real $\mathrm{SU(2)_L}$ triplet coming from the $\mathbf{24}$ representation Higgs getting the VEVs, the $W$ boson mass anomaly reported by the CDF collaboration can be explained.
The vector-like quark doublet acquire the mass through the Yukawa interaction of a real triplet via the type-$\rm{I\hspace{-.01em}I}$ seesaw-like mechanism in addition to that of $\mathbf{24}$ representation Higgs.
We assume that the mass for the vector-like quark doublet is expressed in terms of a real triplet mass
\footnote{In Ref.~\cite{Evans:2022dgq}, authors mentioned that the SU(5) GUT model added one pair of $\mathbf{10}$ representation for the vector-like fermions to the minimal SU(5) GUT model is the minimal one. In this paper, we assume that the mass for the vector-like quark doublet is expressed in terms of a real triplet mass through the Yukawa interaction of a real triplet. Then, we add two pairs of $\mathbf{10}$ representation for the vector-like fermions to the minimal SU(5) GUT model to unify the SM gauge couplings successfully.}.
Since a real triplet and heavy Higgs boson masses are almost the same, we can get the constraints about the vector-like quark doublet mass by considering the constraints of the heavy Higgs bosons.
As a result of combining the constraints of the vector-like quark doublet mass with that of the heavy Higgs boson masses, we can obtain the allowed mass ranges for the vector-like quark doublet and a real triplet scalar which are very narrow.
Therefore, our model can be tested by the search of these particles in the near future.
We set benchmark about the mass eigenvalues for the relevant particles and solve the renormalization group equation~(RGE) including the contributions of new particles.
In the case that included the new contributions, the SM gauge couplings unify successfully at $M_{\mathrm{GUT}}\thickapprox6.76\times10^{15}$~GeV and the value for the unified gauge couplings is $\alpha_{\mathrm{GUT}}=\alpha_1=\alpha_2=\alpha_3\thickapprox1/34.7$.
In addition, the SM Higgs quartic coupling is the positive value at all energy scales and hence the SM Higgs potential is stabilized.
Our model expects the proton lifetime for its decay mediated by the SU(5) gauge bosons as $\tau_p(p\to\pi^0{e^+})\approx{7.19\times10^{34}}$~years and it is testable by the future proton decay search, for example, the Hyper-Kamiokande experiment expected as $\tau_p(p\to\pi^0{e^+})<1.0\times10^{35}$~years~\cite{Dealtry:2019ldr}.

This paper is organaized as follows.
In section~\ref{sec:Wmass}, we discuss the model contained a real $\mathrm{SU(2)_L}$ triplet scalar with zero hypercharge and derive the mass eigenvalues for the physical scalars.
We also obtain the real triplet VEVs to explain the $W$ boson mass anomaly.
In section~\ref{sec:themodel}, we propose the extension of the minimal SU(5) GUT model and consider the allowed mass ranges of the new particles.
Section~\ref{sec:protondecay} is shown our results for gauge unification and proton lifetime and we discuss the testability of our model.
Section~\ref{sec:Summary} is devoted to summary.

%------------------------------------------------------------------------------%
%---------------------------------The W boson mass----------------------------------------%
%------------------------------------------------------------------------------%
\section{The $W$ boson mass}
\label{sec:Wmass}

In this section, we discuss the scalar sector with the SM Higgs $H$ and a real $\mathrm{SU(2)_L}$ triplet scalar with zero hypercharge $T$.
The $H$ and $T$ are given by 
\begin{equation}
\label{eq:Scalar}
    H=
    \begin{pmatrix}
       \phi^{+}\\
       \phi^0\\
    \end{pmatrix},\quad
    T=\frac{1}{2}
    \begin{pmatrix}
        T^0&\sqrt{2}T^{+}\\
        \sqrt{2}T^{-}&T^0\\
    \end{pmatrix}.
\end{equation}
The Lagrangian for the scalar sector is 
\begin{equation}
\label{eq:scalarlagrangian}
\mathcal{L}_{\mathrm{scalar}}\supset(D_\mu{H})^{\dagger}(D_\mu{H})+\Tr(D_\mu{T})^{\dagger}(D_\mu{T})-V(\Phi,T),
\end{equation}
where
\begin{equation}
\label{eq:tripletcovariant}
    D_\mu{T}=\partial_\mu{T}+\mathrm{i}g_2[W_\mu,T].
\end{equation}
Here, $W_\mu$ and $g_2$ are the $\mathrm{SU(2)_L}$ gauge bosons and coupling, respectively.
The most general scalar potential is given by
\begin{equation}
\begin{aligned}
\label{eq:scalarpotential}
    V(H,T)=&-m_h^2{H}^\dagger{H}+\lambda_{0}({H}^\dagger{H})^2+M_T^2\Tr{T}^2+\lambda_1\Tr{T}^4+\lambda_2(\Tr{T}^2)^2\\
&+\alpha(H^\dagger{H})\Tr{T}^2+\beta{H}^\dagger{T^2}{H}+\mu{H}^\dagger{T}{H}.
\end{aligned}    
\end{equation}
The $H$ and $T$ can get the VEVs at the minimum of the potential.
Then, we can parametrize the scalars as,
\begin{equation}
\label{eq:scalarparameter}
    H=
    \begin{pmatrix}
        \phi^{+}\\
        (v_h+h^0+\mathrm{i}G^0)/\sqrt{2}\\
    \end{pmatrix},\quad
    T=\frac{1}{2}
    \begin{pmatrix}
        v_T+t^0&\sqrt{2}t^{+}\\
        \sqrt{2}t^{-}&-v_T-t^0\\
    \end{pmatrix},
\end{equation}
where $v_h$ and $v_T$ are the VEVs of the scalar fields, $H$ and $T$, respectively.
By using Eqs.~\eqref{eq:scalarpotential} and \eqref{eq:scalarparameter}, the minimization conditions for the scalar potential are written as
\begin{align}
    \label{eq:minimizationh}
    m_h^2&=\lambda_0{v_h^2}+\frac{A}{2}v_h^2-\frac{\mu}{2}v_T,\\
    \label{eq:minimizationT}
        M_T^2&=\frac{\mu{v_h^2}}{4v_T}-\frac{A}{2}v_h^2-\frac{B}{2}v_T^2,
\end{align}
where 
\begin{equation}
    A=\alpha+\frac{\beta}{2},\quad{B}=\lambda_1+2\lambda_2.
\end{equation}
The real scalar mass matrix in the $(h^0,t^0)$ basis is 
\begin{equation}
\label{eq:realscalarmatrix}
    \mathcal{M}_0^2=
    \begin{pmatrix}
        2\lambda_0{v_h^2}&{A}v_h{v_T}-\frac{\mu{v_h}}{2}\\
        {A}v_h{v_T}-\frac{\mu{v_h}}{2}&Bv_T^2+\frac{\mu{v_h^2}}{4v_T}
    \end{pmatrix}.
\end{equation}
And the charged scalar mass matrix in the $(\phi^{\pm},T^{\pm})$ basis is
\begin{equation}
    \mathcal{M}_{\pm}^2=
    \begin{pmatrix}
        \mu{v_T}&\frac{\mu{v_T}}{2}\\
        \frac{\mu{v_T}}{2}&\frac{\mu{v_h^2}}{4v_T}\\
    \end{pmatrix}.
\end{equation}
The mass eigenstates are written in terms of the gauge eigenstates as
\begin{align}
    \begin{pmatrix}
        h\\
        H\\
    \end{pmatrix}&=
    \begin{pmatrix}
        \cos{\theta_0}&\sin{\theta_0}\\
        -\sin{\theta_0}&\cos{\theta_0}\\
    \end{pmatrix}
    \begin{pmatrix}
        h^0\\
        t^0\\
    \end{pmatrix},\\
    \begin{pmatrix}
        H^{\pm}\\
        G^{\pm}\\
    \end{pmatrix}&=
    \begin{pmatrix}
        -\sin{\theta_+}&\cos{\theta_+}\\
        \cos{\theta_+}&\sin{\theta_+}\\
    \end{pmatrix}
    \begin{pmatrix}
        \phi^{\pm}\\
        T^{\pm}\\
    \end{pmatrix},
\end{align}
where the mixing angles between the mass and gauge eigenstates are 
\begin{align}
\label{eq:mixinganglereal}
    \tan{2\theta_0}&=\frac{4{v_h}v_T(-\mu+2A{v_T})}{8\lambda_0{v_h^2}v_T-4B{v_T^3}-\mu{v_h^2}},\\
\label{eq:mixinganglecharged}    
    \tan{2\theta_+}&=\frac{4{v_h}v_T}{4{v_T^2}-v_h^2}.
\end{align}
In the limit $v_h\gg{v_T}$, we get $\theta_0\ll{1}$ in Eq.~\eqref{eq:mixinganglereal}.
Then, the mass eigenvalues for the physical scalars are
\begin{align}
\label{eq:SMHiggs}
    M_h^2&=2\lambda_0{v_h^2},\\
\label{eq:HeavyHiggs}    
    M_H^2&=B{v_T^2}+\frac{\mu{v_h^2}}{4{v_T}},\\
\label{eq:ChargedHiggs}    
    M_{H^\pm}^2&=\mu{v_T}\left(1+\frac{v_h^2}{4{v_T^2}}\right).
\end{align}
In Eqs.~\eqref{eq:SMHiggs}-\eqref{eq:ChargedHiggs}, the $h$ becomes the SM-like Higgs and in the limit $v_h\gg{v_T}$, we find $M_{H^\pm}={M_H}={M_T}\approx\mu{v_h^2}/4{v_T}$.

The VEVs of the real $\mathrm{SU(2)_L}$ triplet scalar contribute to the $W$ boson mass.
Then, at tree level, the $W$ boson mass is given by
\begin{equation}
\label{eq:wbosoneq}
    M_W^2=(M_W^{\mathrm{SM}})^2+g_2^2{v_T^2},
\end{equation}
where $M_W^{\mathrm{SM}}$ is the $W$ boson mass in the SM.
The VEVs of the real $\mathrm{SU(2)_L}$ triplet scalar does not contribute to the $Z$ boson mass.
Then, we assume the SM expectation value for the $Z$ boson mass.
The average result for the $W$ boson mass obtained by combining the CDF results with previous results from LEP2, Tevatron, LHC, and LHCb experiments~\cite{deBlas:2022hdk} is
\begin{equation}
    M_W^{\mathrm{exp}}=80.4133\pm{0.0080}~\text{GeV}.
\end{equation}
This value derivates from the SM prediction~\cite{deBlas:2022hdk} by $6.5~\sigma$,
\begin{equation}
    M_W^{\mathrm{SM}}=80.3500{\pm}0.0056~\text{GeV}.
\end{equation}
In order to explain the derivation by using Eq.~\eqref{eq:wbosoneq}, the $v_T$ is
\begin{equation}
    v_T=4.85~\text{GeV}.
\end{equation}
Here, we use $g_2(\mu=M_Z)=0.657452$~\cite{deBlas:2022hdk, ParticleDataGroup:2022pth}.
Then, by using Eq.~\eqref{eq:mixinganglecharged}, we can get the mixing angle for the charged scalar sector as
\begin{equation}
    \theta_{+}=-0.0394.
\end{equation}
We can determine the couplings to the charged Higgs and predict its branching ratios by using the $v_T$ and $\theta_+$.
See details for Ref.~\cite{FileviezPerez:2022lxp}.

%------------------------------------------------------------------------------%
%-------------------------------Model-------------------------------%
%------------------------------------------------------------------------------%
\section{The SU(5) GUT model}
\label{sec:themodel}
In previous section, we discuss the $W$ boson mass anomaly with the SM Higgs and a real $\mathrm{SU(2)_L}$ triplet scalar with zero hypercharge.
In this section, we build an SU(5) GUT model to explain the $W$ boson mass anomaly.
The matter contents for the minimal SU(5) GUT model~\cite{Georgi:1974sy} are $\bar{5}_L^{i}$ and $10_L^{i}$ for the SM matter fields, $5_H$ and $24_H$ for the scalar sector, and $A_\mu$ for the gauge field, respectively.
The $24_H$ is composed of,
\begin{equation}
24_{H}\sim\Phi(8,1,0)\oplus\Phi(1,3,0)\oplus\Phi(1,1,0)\oplus\Phi(3,2,-\frac{5}{6})\oplus\Phi(\bar{3},2,\frac{5}{6}).\\
\end{equation}
Then, the $24_H$ includes a real $\mathrm{SU(2)_L}$ triplet scalar with zero hypercharge and the minimal SU(5) GUT can explain the $W$ boson mass anomaly by a real triplet coming from the $24_H$ getting the VEVs.
However, this model have problems.
Even if a real triplet exists at low energy and contribute to the RGE, the SM gauge couplings
do not unify successfully and this is excluded by the proton decay search~\cite{Evans:2022dgq}.
Then, this model is needed some extension.

We propose the SU(5) GUT model that is a minimal SU(5) GUT model adding two pairs of $\mathbf{10}$ representation for the vector-like fermions.
We denote the vector-like fermions as $10_{L,R}^{4,5}$ and the decompositions of these are
\begin{equation}
    10_{L,R}^{4,5}=Q(3,2,1/6)\oplus{U^c}(\Bar{3},1,-2/3)\oplus{E^c}(1,1,1).
\end{equation}
We impose a $\mathbb{Z}_3$ symmetry in order not to mix the SM, 4th vector-like, and 5th vector-like fermions.
Then, we can write the Yukawa interactions for the SM and vector-like fermions sector as
\begin{align}
\label{eq:SM_Lagrangian}
\mathcal{L}_{\mathrm{SM}}\supset\sum_{i,j=1}^{3}Y^{ij}_{1}5_H{10}_L^i{10_L^j}+\sum_{i,j=1}^{3}Y^{ij}_{2}5_H^*\bar{5}_L^i{10}_L^j+\mathrm{h.c.},\\
\label{eq:VL_Lagrangian}
\mathcal{L}_{\mathrm{VL}}\supset\overline{10}_L^4[{Y}_{10}^{4}{24}_H+M^4_{10}]10_{R}^4+\overline{10}_L^5[{Y}_{10}^5{24}_H+M_{10}^5]10_{R}^5+\mathrm{h.c.},
\end{align}
where $Y_{1,2}^{ij}$ are the Yukawa couplings for the SM sector, $Y_{10}^{4,5}$ are the Yukawa couplings for the vector-like fermions sector, and $M_{10}^{4,5}$ are the mass for 4th and 5th vector-like fermions, respectively.
In the minimal SU(5) GUT model, the mass relations among down-type quark and charged lepton are inconsistent with the experimental results.
The simple way to correct the mass relations is to add the non-renormalizable terms~\cite{Dorsner:2005fq, Dorsner:2006hw, Ellis:1979fg}, for example, $Y_u^{ij}24_H{5_H}10^i_L{10^j_L}/\Lambda$ where $\Lambda$ is the cut-off scale.
Another way is to add the $\mathbf{45}$ representation Higgs~\cite{Georgi:1979df, Kalyniak:1982pt, Eckert:1983bn}.
In addition, the scalars composed of the $\mathbf{45}$ representation Higgs help to acheive the gauge unification~\cite{FileviezPerez:2007bcw, Dorsner:2007fy, FileviezPerez:2016sal, Boucenna:2017fna, FileviezPerez:2019fku, FileviezPerez:2019ssf, Shimizu:2022wsk}.

In our model, a real triplet coming from the $24_H$ get the VEVs to explain the $W$ boson mass anomaly.
Then, vector-like quark doublet acquire the mass through the Yukawa interaction of a real triplet via the type-$\rm{I\hspace{-.01em}I}$ seesaw-like mechanism as
\begin{equation}
\label{eq:VLRTYukawa}
    \mathcal{L}_{\mathrm{VL-RT}}\supset{Y_{10}^4}T\Bar{Q}^4_{L}{Q^4_R}+{Y_{10}^5}T\Bar{Q}^5_{L}{Q^5_R}+\mathrm{h.c.},
\end{equation}
where $T$ is a real triplet coming from the $24_H$ and $Q_{L,R}^{4,5}$ is vector-like quark doublet in the $10_{L,R}^{4,5}$.
The spontaneous symmetry breaking occurs at the following steps.
At first, the $24_H$ breaks the SU(5) symmetry by taking the VEVs as $\langle24_H\rangle=V/2\sqrt{15}~\mathrm{Diag}(-2,-2,-2,3,3)$.
Next, the $5_H$ realizes the electroweak symmetry breaking by taking the VEVs as $\langle{5_H}\rangle=(0,0,0,0,v_h/\sqrt{2})$.
By using Eq.~\eqref{eq:minimizationT}, we derive the VEVs of a real triplet as $v_T\approx\mu{v_h^2}/4M_T^2$.
Then, after spontaneous symmetry breaking, we can acquire the mass eigenvalues of vector-like fermion as follows:
\begin{align}
\label{eq:quarkdoubletVmass4}
    M_Q^4&=M^4_{10}-\frac{Y^4_{10}}{4\sqrt{15}}V+Y^4_{10}\frac{\mu{v_h^2}}{8M_T^2},\\
\label{eq:upquarkVmass4}    
    M_U^4&=M^4_{10}+\frac{Y^4_{10}}{\sqrt{15}}V,\\
\label{eq:electronVmass4}    
    M_E^4&=M^4_{10}-\frac{3Y^4_{10}}{2\sqrt{15}}V,\\
\label{eq:quarkdoubletVmass5}
    M_Q^5&=M^5_{10}-\frac{Y^5_{10}}{4\sqrt{15}}V+Y^5_{10}\frac{\mu{v_h^2}}{8M_T^2},\\
\label{eq:upquarkVmass5}    
    M_U^5&=M^5_{10}+\frac{Y^5_{10}}{\sqrt{15}}V,\\
\label{eq:electronVmass5}    
    M_E^5&=M^5_{10}-\frac{3Y^5_{10}}{2\sqrt{15}}V.
\end{align}
Here, the $M^4_{10}$ and $Y^4_{10}$ are free parameters.
Then, by choosing the value of $M^4_{10}$ and $Y^4_{10}$ appropriately, we can express the mass eigenvalue for the 4th vector-like quark doublet as $M^4_Q=Y^4_{10}{\mu}{v_h^2}/(8M_T^2)$, that is, we assume that $M^4_{10}=Y^4_{10}V/(4\sqrt{15})$.
In that case, the 4th vector-like right-handed up quark and electron has the same mass as $M_U=M_E=5Y^4_{10}V/(4\sqrt{15})$.
Here, the 4th vector-like quark doublet mass is expressed in terms of a real triplet mass.
In previous section, we find $M_{H^\pm}={M_H}={M_T}\approx\mu{v_h^2}/4{v_T}$.
Then, we can get the constraints about the 4th vector-like quark doublet mass by considering the constraints of the heavy Higgs bosons.
We can get the upper bound for the mass of heavy Higgs bosons from the perturbative unitarity of the $WW$ scattering cross-section~\cite{Chivukula:2007koj} as follows:
\begin{equation}
\label{eq:Higgsupper}
    M_{H,H^{\pm}}\lesssim\frac{2\sqrt{\pi}v_h^2}{v_T}.
\end{equation}
In order to explain the $W$ boson mass, we fix the VEVs for real triplet as $v_T=4.85$~GeV.
Then, we can get the upper bound for the mass of heavy Higgs bosons as $M_{H,H^{\pm}}\lesssim~44.3$~TeV.
In Eqs.~\eqref{eq:HeavyHiggs} and \eqref{eq:ChargedHiggs}, we derive the mass of the heavy Higgs bosons as $M_H^2={M_{H^{\pm}}^2}\approx\mu{v_h^2}/4{v_T}$.
Then, the upper bound of the trilinear coupling $\mu$ is given by
\begin{equation}
    \mu<\frac{16\pi{v_h^2}}{v_T}\approx{6.28\times{10}^2}~\text{TeV}.
\end{equation}
In our assumption, the 4th vector-like quark doublet mass is $M^4_Q\approx{{Y^4_{10}}\mu{v_h^2}}/(8M_T^2)$ and then we can derive the upper bound among the 4th vector-like quark doublet and a real $\mathrm{SU(2)_L}$ triplet scalar mass as
\begin{equation}
\label{eq:massupperbound}
    M^4_{Q}\times{(M_T)^2}<4.76\times{Y^4_{10}}~\text{(TeV)}^3.
\end{equation}
The mass for the new particles is restricted from this upper bound.
Next, we consider the experimental constraints for the new particles.
The lower bound of the mass for vector-like quark is 1660~GeV~\cite{CMS:2018dcw}.
We also get the bound of the heavy neutral Higgs boson mass as $M_{H}>1400$~GeV~\cite{ATLAS:2019tpq}.
In our model, the real triplet mass and heavy neutral and charged Higgs bosons have approximately the same mass.
Then, we need to consider the bound of the charged Higgs boson mass as $M_{H^+}>1000$~GeV~\cite{ATLAS:2021upq}.
By combining the upper bound among the vector-like quark doublet and a real $\mathrm{SU(2)_L}$ triplet scalar mass in Eq.~\eqref{eq:massupperbound} with the bound for each particle, we can obtain the allowed mass range for the vector-like quark and a real $\mathrm{SU(2)_L}$ triplet scalar.
However, the upper bound can be determined by the Yukawa coupling $Y^4_{10}$.
If the $Y^4_{10}$ is small, the upper bound is inconsistent with the experimental constraints.
In order to derive the allowed mass ranges for new particles, we assume that the $Y^4_{10}$ is exactly one.
First, we consider the upper bound of the mass for a real $\mathrm{SU(2)_L}$ triplet scalar.
In order to obtain this upper bound, we combine the upper bound in Eq.~\eqref{eq:massupperbound} with the bound for the vector-like quark mass.
Then, we can get the upper bound of the mass for a real $\mathrm{SU(2)_L}$ triplet scalar as $M_{T}<1693$~GeV.
On the other hand, by combining the upper bound in Eq.~\eqref{eq:massupperbound} with the bound for the heavy neutral or charged Higgs boson masses, we can obtain the upper bound of the mass for vector-like quark as $M^4_{Q}<2428$~GeV~(the heavy neutral Higgs boson case) or $M^4_{Q}<4759$~GeV~(the charged Higgs boson case).
As the result, we can get the allowed mass range for the 4th vector-like and a real triplet scalar as $1660<M^4_Q<2428$~GeV and $1400<M_T<1693$~GeV~(the heavy neutral Higgs boson case) or $1660<M^4_Q<4759$~GeV and $1000<M_T<1693$~GeV~(the charged Higgs boson case), respectively.
In the next section, we use these mass ranges for the discussion of the proton decay and gauge unification.

%------------------------------------------------------------------------------%
%-------------------------Proton decay and gauge unification-----------------------%
%------------------------------------------------------------------------------%
\section{Proton decay and gauge unification}
\label{sec:protondecay}

In this section, we evaluate the contribution of the new particles to the RGE and estimate the proton lifetime.
In the previous section, we obtain the masses for the vector-like fermion from Eq.~\eqref{eq:quarkdoubletVmass4} to Eq.~\eqref{eq:electronVmass5}.
By assuming that $M^4_{10}=Y^4_{10}V/(4\sqrt{15})$, the 4th vector-like quark doublet mass is written as $M^4_Q=Y^4_{10}{\mu}{v_h^2}/(8M_T^2)$ and the 4th vector-like right-handed up quark and electron have the same mass as $M^4_U=M^4_E=5Y^4_{10}V/(4\sqrt{15})$.
In addition, we obtain the allowed mass ranges for the 4th vector-like quark doublet and a real triplet scalar.
In the case that the $Y^4_{10}$ is exactly one, the 4th vector-like right-handed up quark and electron have the same GUT scale mass and the allowed mass range for the 4th vector-like quark doublet and a real triplet scalar are given by $1660<M^4_Q<2428$~GeV and $1400<M_T<1693$~GeV~(the heavy neutral Higgs boson case) or $1660<M^4_Q<4759$~GeV and $1000<M_T<1693$~GeV~(the charged Higgs boson case), respectively.

In our model, the candidates for contributing to the RGE are the vector-like fermions, the real triplet scalar in the $24_H$, and the color octet scalar in the $24_H$.
In our analysis of solving the RGE, we fix the masses such that the 4th vector-like quark doublet have mass $M^4_Q=2000$~GeV, the 5th vector-like right-handed up quark have mass $M^5_U=5.0\times{10}^{8}$~GeV, the real triplet scalar and color octet scalar have masses $M_T=M_{H_8}=1500$~GeV, and the other particles have tha GUT scale masses, respectively.
We consider the contributions of the SM particles at 2-loop level~\cite{Machacek:1983tz, Machacek:1983fi, Machacek:1984zw} and the new particles contributions at 1-loop level.
Note that in the case that added single generation of vector-like fermions, that is, added $\mathbf{\bar{5}}$ and $\mathbf{10}$ representations for the vector-like fermions, the proton lifetime estimated in these mass ranges is not within the expected limit for proton lifetime by the Hyper-Kamiokande experiment, which means that the proton lifetime is inconsistent with the current experimental result or exceed that expected by the Hyper-Kamiokande experiment.

The left panel of Fig.~\ref{fig:SMcoupling} is shown our result for the running of the SM gauge couplings by solving the RGE.
We define the SM gauge couplings for $\mathrm{U(1)_Y}$, $\mathrm{SU(2)_L}$, $\mathrm{SU(3)_C}$ as $\alpha_{i}\equiv{g_i}/4\pi$~$(i=1$-$3)$, respectively.
The running of the SM gauge couplings for only the SM particles is depicted by the black dashed line.
The red dashed line is shown the running of the SM gauge couplings including the contribution of the new particles.
In the case included the new contribution, the SM gauge couplings unify successfully at $M_{\mathrm{GUT}}\thickapprox6.76\times10^{15}$~GeV and the value for the unified gauge couplings is $\alpha_{\mathrm{GUT}}=\alpha_1=\alpha_2=\alpha_3\thickapprox1/34.7$.
By using these values, we estimate the proton lifetime for its decay mediated by the SU(5) gauge bosons approximately~\cite{Nath:2006ut},
\begin{equation}
\label{eq:protonlifetime}
     \tau_p(p\to\pi^0{e^+})\approx{\frac{1}{\alpha^2_\mathrm{GUT}}}\frac{M^4_\mathrm{GUT}}{m^5_p}\approx{7.19\times10^{34}}~\mathrm{years},
\end{equation}
where $m_p=0.938$~GeV is the proton mass~\cite{ParticleDataGroup:2022pth}.
This is consistent with the current experimental result for the proton lifetime given by the Super-Kamiokande experiment as $\tau_p(p\to\pi^0{e^+})>2.4\times10^{34}$~years~\cite{Super-Kamiokande:2020wjk}.
In addition, it is within the expected limit for proton lifetime by the Hyper-Kamiokande experiment, $\tau_p(p\to\pi^0{e^+})<1.0\times10^{35}$~years~\cite{Dealtry:2019ldr}.
Then, it is testable by the Hyper-Kamiokande experiment.
The proton decay can also be mediated by the color triplet scalar field contained in the $5_H$.
The Super-Kamiokande experiment excludes that the colored Higgs is lighter than $\mathcal{O}(10^{11})$~GeV~\cite{Nath:2006ut}.
By considering the quartic cross terms between the $5_H$ and $24_H$, foe example, $(5_H^\dagger5_H)(24_H^\dagger24_H)$, the colored Higgs mass can be greater than $\mathcal{O}(10^{11})$~GeV.

\begin{figure}[htbp]
\begin{minipage}[b]{0.5\linewidth}
    \centering
    \includegraphics[keepaspectratio,scale=0.65]{Unification_VLRT_20230309.pdf}
    \subcaption{}
    \label{fig:unification}
\end{minipage}  
\begin{minipage}[b]{0.5\linewidth}
    \centering
    \includegraphics[keepaspectratio,scale=0.65]{Lambda_VLRT_20230309.pdf}
    \subcaption{}
    \label{fig:lambda}
\end{minipage}  
\caption{The plot shows our results of running the SM couplings by solving the RGE for the fixed masses such that the 4th vector-like quark doublet have mass $M^4_Q=2000$~GeV, the 5th vector-like right-handed up quark have mass $M^5_U=5.0\times{10}^{8}$~GeV, the real triplet scalar and color octet scalar have masses $M_T=M_{H_8}=1500$~GeV, and the other particles have the GUT scale masses, respectively.
In both figures, the red (black) dashed line shows the running of the SM gauge couplings with (without) the contribution of the new particles.
In Fig.~\eqref{fig:unification}, the SM gauge couplings unify successfully at $M_{\mathrm{GUT}}\thickapprox6.76\times10^{15}$~GeV and the value for the unified gauge couplings is $\alpha_{\mathrm{GUT}}=\alpha_1=\alpha_2=\alpha_3\thickapprox1/34.7$.
In Fig.~\eqref{fig:lambda}, the SM Higgs quartic coupling is the positive value at all energy scales and hence the SM Higgs potential is stabilized.
}
\label{fig:SMcoupling}
\end{figure}

The right panel of Fig.~\ref{fig:SMcoupling} is shown our result for the running of the SM Higgs quartic coupling $\lambda$ by solving the RGE.
The running of the SM Higgs quartic coupling for only the SM particles is depicted by the black dashed line.
The red dashed line is shown the running of the SM Higgs quartic coupling including the contribution of the new particles.
In addition, the horizontal green line depicts $\lambda=0$.
In the case that includes the new contribution, the SM Higgs quartic coupling is the positive value at all energy scales and hence the SM Higgs potential is stabilized.

In the left panel of Fig.~\ref{fig:MQMT}, the green region denotes the range for the real triplet scalar (color octet scalar) mass $M_T(M_{H_8})$ and expected proton lifetime to achieve the unification of the SM gauge couplings with an accuracy of 1~\% or less.
We define the accuracy of the unification as a percentage difference between the energy scale of unifying the SM gauge couplings $\alpha_1,~\alpha_2$ and $\alpha_2,~\alpha_3$.
The blue and black shaded regions are shown the excluded mass range for the real triplet scalar that except for $1400<M_T<1693$~GeV.
The yellow shaded region is shown the bound for the color octet scalar mass as $M_{H_8}>1$~TeV~\cite{Hayreter:2017wra, Miralles:2019uzg}.
The excluded region for proton lifetime by Super-Kamiokande experiment is the gray shaded one as $\tau_p(p\to\pi^0{e^+})>2.4\times10^{34}$~years~\cite{Super-Kamiokande:2020wjk} and the red dashed line depicts the expected limit for proton lifetime by the Hyper-Kamiokande experiment, $\tau_p(p\to\pi^0{e^+})<1.0\times10^{35}$~years~\cite{Dealtry:2019ldr}.

In the right panel of Fig.~\ref{fig:MQMT}, the green and purple region denotes the range for the 4th vector-like quark doublet mass $M^4_Q$ and the real triplet scalar (color octet scalar) mass $M_T(M_{H_8})$ to achieve the unification of the SM gauge couplings with an accuracy of 1~\% or less.
The green region is within the expected limit for proton lifetime by the Hyper-Kamiokande experiment and the purple one exceed this.
The red and brown shaded region and black and blue shaded region show the excluded mass range for the 4th vector-like quark doublet and the real triplet scalar that except for $1400<M_T<1693$~GeV and $1660<M^4_Q<2428$~GeV, respectively.
The yellow shaded region shows the bound for the color octet scalar mass as $M_{H_8}>1$~TeV~\cite{Hayreter:2017wra, Miralles:2019uzg}.
The all region to achieve the unification of the SM gauge couplings with an accuracy of 1~\% or less within the allowed mass ranges for each particle is testable by the Hyper-Kamiokande experiment.
In addition, since the allowed mass ranges for the 4th vector-like quark doublet and a real triplet scalar are very narrow, our model can be tested by the search of these particles in the near future.

\begin{figure}[htbp]
\begin{minipage}[b]{0.5\linewidth}
    \centering
    \includegraphics[keepaspectratio,scale=0.65]{Mt_tau_VLRT_20230320.pdf}
    \subcaption{}
    \label{fig:MTtau}
\end{minipage}  
\begin{minipage}[b]{0.5\linewidth}
    \centering
    \includegraphics[keepaspectratio,scale=0.65]{MQ_MT_20230320.pdf}
    \subcaption{}
    \label{fig:MQMTN}
\end{minipage}  
\caption{The green region in Fig.~\eqref{fig:MTtau} denotes the range for the real triplet scalar (color octet scalar) mass $M_T(M_{H_8})$ and expected proton lifetime to achieve the unification of the SM gauge couplings with an accuracy of 1~\% or less.
In both figure, the blue, black, and yellow shaded regions show the excluded mass range for the real triplet scalar that except for $1400<M_T<1693$~GeV and the bound for the color octet scalar mass as $M_{H_8}>1$~TeV~\cite{Hayreter:2017wra, Miralles:2019uzg}, respectively.
In Fig.~\eqref{fig:MTtau}, the excluded region for proton lifetime by Super-Kamiokande experiment is the gray shaded one as $\tau_p(p\to\pi^0{e^+})>2.4\times10^{34}$~years~\cite{Super-Kamiokande:2020wjk} and the red dashed line depicts the expected limit for proton lifetime by the Hyper-Kamiokande experiment, $\tau_p(p\to\pi^0{e^+})<1.0\times10^{35}$~years~\cite{Dealtry:2019ldr}.
The green and purple region in Fig.~\eqref{fig:MQMTN} denotes the range for the 4th vector-like quark doublet mass $M^4_Q$ and the real triplet scalar (color octet scalar) mass $M_T(M_{H_8})$ to achieve the unification of the SM gauge couplings with an accuracy of 1~\% or less.
The green region is within the expected limit for proton lifetime by the Hyper-Kamiokande experiment and the purple one exceed this.
In Fig.~\eqref{fig:MQMTN}, the red and brown shaded regions show the excluded mass range for the 4th vector-like quark doublet that except for $1660<M^4_Q<2428$~GeV.}
\label{fig:MQMT}
\end{figure}

\clearpage

%------------------------------------------------------------------------------%
%------------------------Summary and Discussions--------------------------------%
%------------------------------------------------------------------------------%
\section{Summary}
\label{sec:Summary}

We have proposed the SU(5) GUT model added two pair of $\mathbf{10}$ representation for the vector-like fermions to the minimal SU(5) GUT model.
By a real $\mathrm{SU(2)_L}$ triplet coming from the $\mathbf{24}$ representation Higgs getting the VEVs, the $W$ boson mass anomaly reported by the CDF collaboration can be explained.
The vector-like quark doublet acquire the mass through the Yukawa interaction of a real triplet via the type-$\rm{I\hspace{-.01em}I}$ seesaw-like mechanism in addition to that of $\mathbf{24}$ representation Higgs.
We assumed that the mass for the vector-like quark doublet is expressed in terms of a real triplet mass.
Since a real triplet and heavy Higgs boson masses are almost the same, we could get the constraints about the vector-like quark doublet mass by considering the constraints of the heavy Higgs bosons.
As a result of combining the constraints of the vector-like quark doublet mass with that of the heavy Higgs boson masses, we could obtain the allowed mass ranges for the vector-like quark doublet and a real triplet scalar.
In the case that the Yukawa coupling for the vector-like quark doublet is exactly one, the allowed mass ranges for the 4th vector-like quark doublet and a real triplet scalar are given by $1660<M^4_Q<2428$~GeV and $1400<M_T<1693$~GeV~(the heavy neutral Higgs boson case) or $1660<M^4_Q<4759$~GeV and $1000<M_T<1693$~GeV~(the charged Higgs boson case), respectively.
The allowed mass ranges for the vector-like quark doublet and a real triplet scalar are very narrow.
Therefore, our model can be tested by the search of these particles in the near future.
We have set benchmark about the mass eigenvalues for the relevant particles and solve the RGE including the contributions of new particles.
In the case that included the new contributions, the SM gauge couplings unify successfully at $M_{\mathrm{GUT}}\thickapprox6.76\times10^{15}$~GeV and the value for the unified gauge couplings is $\alpha_{\mathrm{GUT}}=\alpha_1=\alpha_2=\alpha_3\thickapprox1/34.7$.
In addition, the SM Higgs quartic coupling is the positive value at all energy scales and hence the SM Higgs potential is stabilized.
Our model expects the proton lifetime for its decay mediated by the SU(5) gauge bosons as $\tau_p(p\to\pi^0{e^+})\approx{7.19\times10^{34}}$~years and it is testable by the future proton decay search, for example, the Hyper-Kamiokande experiment expected as $\tau_p(p\to\pi^0{e^+})<1.0\times10^{35}$~years.

%-------- acknowledgement -------%
%\vspace{1cm}
%\noindent
%{\large \bf Acknowledgement}
%\vspace{1mm}

%\newpage 
%---------------------------------------------------------%
%--------------- Appendix --------------------------------%
%---------------------------------------------------------%
%\appendix
%\section*{Appendix}
%\section{The minimal SU(5) Model}


%------------------------------------------------------------------------------%
%--------------------------    References    --------------------------------------%
%------------------------------------------------------------------------------%
%\newpage
\begin{thebibliography}{99}

%\cite{Georgi:1974sy}
\bibitem{Georgi:1974sy}
H.~Georgi and S.~L.~Glashow,
%``Unity of All Elementary Particle Forces,''
Phys. Rev. Lett. \textbf{32} (1974), 438-441
%doi:10.1103/PhysRevLett.32.438
%5521 citations counted in INSPIRE as of 06 Mar 2023

%\cite{Georgi:1974yf}
\bibitem{Georgi:1974yf}
H.~Georgi, H.~R.~Quinn and S.~Weinberg,
%``Hierarchy of Interactions in Unified Gauge Theories,''
Phys. Rev. Lett. \textbf{33} (1974), 451-454
%doi:10.1103/PhysRevLett.33.451
%1974 citations counted in INSPIRE as of 19 Dec 2022

%\cite{Super-Kamiokande:2020wjk}
\bibitem{Super-Kamiokande:2020wjk}
A.~Takenaka \textit{et al.} [Super-Kamiokande],
%``Search for proton decay via $p\to e^+\pi^0$ and $p\to \mu^+\pi^0$ with an enlarged fiducial volume in Super-Kamiokande I-IV,''
Phys. Rev. D \textbf{102} (2020) no.11, 112011
%doi:10.1103/PhysRevD.102.112011
[arXiv:2010.16098 [hep-ex]].
%52 citations counted in INSPIRE as of 19 Dec 2022

%\cite{CDF:2022hxs}
\bibitem{CDF:2022hxs}
T.~Aaltonen \textit{et al.} [CDF],
%``High-precision measurement of the $W$          boson mass with the CDF II detector,''
Science \textbf{376} (2022) no.6589, 170-176
%doi:10.1126/science.abk1781
%332 citations counted in INSPIRE as of 07 Mar 2023

%\cite{deBlas:2022hdk}
\bibitem{deBlas:2022hdk}
J.~de Blas, M.~Pierini, L.~Reina and L.~Silvestrini,
%``Impact of the Recent Measurements of the Top-Quark and W-Boson Masses on Electroweak Precision Fits,''
Phys. Rev. Lett. \textbf{129} (2022) no.27, 271801
%doi:10.1103/PhysRevLett.129.271801
[arXiv:2204.04204 [hep-ph]].
%115 citations counted in INSPIRE as of 07 Mar 2023

%\cite{Ross:1975fq}
\bibitem{Ross:1975fq}
D.~A.~Ross and M.~J.~G.~Veltman,
%``Neutral Currents in Neutrino Experiments,''
Nucl. Phys. B \textbf{95} (1975), 135-147
%doi:10.1016/0550-3213(75)90485-X
%299 citations counted in INSPIRE as of 07 Mar 2023

%\cite{Gunion:1989ci}
\bibitem{Gunion:1989ci}
J.~F.~Gunion, R.~Vega and J.~Wudka,
%``Higgs triplets in the standard model,''
Phys. Rev. D \textbf{42} (1990), 1673-1691
%doi:10.1103/PhysRevD.42.1673
%343 citations counted in INSPIRE as of 07 Mar 2023

%\cite{Lynn:1990zk}
\bibitem{Lynn:1990zk}
B.~W.~Lynn and E.~Nardi,
%``Radiative corrections in unconstrained SU(2) x U(1) and the top mass problem,''
Nucl. Phys. B \textbf{381} (1992), 467-500
%doi:10.1016/0550-3213(92)90486-U
%30 citations counted in INSPIRE as of 07 Mar 2023

%\cite{Blank:1997qa}
\bibitem{Blank:1997qa}
T.~Blank and W.~Hollik,
%``Precision observables in SU(2) x U(1) models with an additional Higgs triplet,''
Nucl. Phys. B \textbf{514} (1998), 113-134
%doi:10.1016/S0550-3213(97)00785-2
[arXiv:hep-ph/9703392 [hep-ph]].
%85 citations counted in INSPIRE as of 07 Mar 2023

%\cite{Forshaw:2003kh}
\bibitem{Forshaw:2003kh}
J.~R.~Forshaw, A.~Sabio Vera and B.~E.~White,
%``Mass bounds in a model with a triplet Higgs,''
JHEP \textbf{06} (2003), 059
%doi:10.1088/1126-6708/2003/06/059
[arXiv:hep-ph/0302256 [hep-ph]].
%42 citations counted in INSPIRE as of 07 Mar 2023

%\cite{Chen:2006pb}
\bibitem{Chen:2006pb}
M.~C.~Chen, S.~Dawson and T.~Krupovnickas,
%``Higgs triplets and limits from precision measurements,''
Phys. Rev. D \textbf{74} (2006), 035001
%doi:10.1103/PhysRevD.74.035001
[arXiv:hep-ph/0604102 [hep-ph]].
%57 citations counted in INSPIRE as of 07 Mar 2023

%\cite{Chankowski:2006hs}
\bibitem{Chankowski:2006hs}
P.~H.~Chankowski, S.~Pokorski and J.~Wagner,
%``(Non)decoupling of the Higgs triplet effects,''
Eur. Phys. J. C \textbf{50} (2007), 919-933
%doi:10.1140/epjc/s10052-007-0259-x
[arXiv:hep-ph/0605302 [hep-ph]].
%45 citations counted in INSPIRE as of 07 Mar 2023

%\cite{Chivukula:2007koj}
\bibitem{Chivukula:2007koj}
R.~S.~Chivukula, N.~D.~Christensen and E.~H.~Simmons,
%``Low-energy effective theory, unitarity, and non-decoupling behavior in a model with heavy Higgs-triplet fields,''
Phys. Rev. D \textbf{77} (2008), 035001
%doi:10.1103/PhysRevD.77.035001
[arXiv:0712.0546 [hep-ph]].
%25 citations counted in INSPIRE as of 07 Mar 2023

%\cite{Bandyopadhyay:2020otm}
\bibitem{Bandyopadhyay:2020otm}
P.~Bandyopadhyay and A.~Costantini,
%``Obscure Higgs boson at Colliders,''
Phys. Rev. D \textbf{103} (2021) no.1, 015025
%doi:10.1103/PhysRevD.103.015025
[arXiv:2010.02597 [hep-ph]].
%31 citations counted in INSPIRE as of 07 Mar 2023

%\cite{FileviezPerez:2022lxp}
\bibitem{FileviezPerez:2022lxp}
P.~Fileviez Perez, H.~H.~Patel and A.~D.~Plascencia,
%``On the W mass and new Higgs bosons,''
Phys. Lett. B \textbf{833} (2022), 137371
%doi:10.1016/j.physletb.2022.137371
[arXiv:2204.07144 [hep-ph]].
%57 citations counted in INSPIRE as of 07 Mar 2023

%\cite{Wu:2022uwk}
\bibitem{Wu:2022uwk}
J.~Wu, D.~Huang and C.~Q.~Geng,
%``$W$-Boson Mass Anomaly from a General $SU(2)_{L}$ Scalar Multiplet,''
[arXiv:2212.14553 [hep-ph]].
%1 citations counted in INSPIRE as of 07 Mar 2023

%\cite{Evans:2022dgq}
\bibitem{Evans:2022dgq}
J.~L.~Evans, T.~T.~Yanagida and N.~Yokozaki,
%``W boson mass anomaly and grand unification,''
Phys. Lett. B \textbf{833} (2022), 137306
%doi:10.1016/j.physletb.2022.137306
[arXiv:2205.03877 [hep-ph]].
%13 citations counted in INSPIRE as of 08 Mar 2023

%\cite{Senjanovic:2022zwy}
\bibitem{Senjanovic:2022zwy}
G.~Senjanovi\'c and M.~Zantedeschi,
%``SU(5) grand unification and W-boson mass,''
Phys. Lett. B \textbf{837} (2023), 137653
%doi:10.1016/j.physletb.2022.137653
[arXiv:2205.05022 [hep-ph]].
%11 citations counted in INSPIRE as of 08 Mar 2023

%\cite{Calibbi:2022wko}
\bibitem{Calibbi:2022wko}
L.~Calibbi and X.~Gao,
%``Lepton flavor violation in minimal grand unified type II seesaw models,''
Phys. Rev. D \textbf{106} (2022) no.9, 095036
%doi:10.1103/PhysRevD.106.095036
[arXiv:2206.10682 [hep-ph]].
%1 citations counted in INSPIRE as of 25 Mar 2023

%\cite{Fritzsch:1974nn}
\bibitem{Fritzsch:1974nn}
H.~Fritzsch and P.~Minkowski,
%``Unified Interactions of Leptons and Hadrons,''
Annals Phys. \textbf{93} (1975), 193-266
%doi:10.1016/0003-4916(75)90211-0
%2080 citations counted in INSPIRE as of 25 Mar 2023

%\cite{Magg:1980ut}
\bibitem{Magg:1980ut}
M.~Magg and C.~Wetterich,
%``Neutrino Mass Problem and Gauge Hierarchy,''
Phys. Lett. B \textbf{94} (1980), 61-64
%doi:10.1016/0370-2693(80)90825-4
%1145 citations counted in INSPIRE as of 20 Mar 2023

%\cite{Cheng:1980qt}
\bibitem{Cheng:1980qt}
T.~P.~Cheng and L.~F.~Li,
%``Neutrino Masses, Mixings and Oscillations in SU(2) x U(1) Models of Electroweak Interactions,''
Phys. Rev. D \textbf{22} (1980), 2860
%doi:10.1103/PhysRevD.22.2860
%1224 citations counted in INSPIRE as of 20 Mar 2023

%\cite{Lazarides:1980nt}
\bibitem{Lazarides:1980nt}
G.~Lazarides, Q.~Shafi and C.~Wetterich,
%``Proton Lifetime and Fermion Masses in an SO(10) Model,''
Nucl. Phys. B \textbf{181} (1981), 287-300
%doi:10.1016/0550-3213(81)90354-0
%1624 citations counted in INSPIRE as of 20 Mar 2023

%\cite{Mohapatra:1980yp}
\bibitem{Mohapatra:1980yp}
R.~N.~Mohapatra and G.~Senjanovic,
%``Neutrino Masses and Mixings in Gauge Models with Spontaneous Parity Violation,''
Phys. Rev. D \textbf{23} (1981), 165
%doi:10.1103/PhysRevD.23.165
%2891 citations counted in INSPIRE as of 20 Mar 2023

%\cite{Lazarides:2022spe}
\bibitem{Lazarides:2022spe}
G.~Lazarides, R.~Maji, R.~Roshan and Q.~Shafi,
%``Heavier W boson, dark matter, and gravitational waves from strings in an SO(10) axion model,''
Phys. Rev. D \textbf{106} (2022) no.5, 5
%doi:10.1103/PhysRevD.106.055009
[arXiv:2205.04824 [hep-ph]].
%16 citations counted in INSPIRE as of 25 Mar 2023

%\cite{Chao:2022blc}
\bibitem{Chao:2022blc}
W.~Chao, M.~Jin, H.~J.~Li and Y.~Q.~Peng,
%``Axion-like Dark Matter from the Type-II Seesaw Mechanism,''
[arXiv:2210.13233 [hep-ph]].
%1 citations counted in INSPIRE as of 25 Mar 2023

%\cite{Dealtry:2019ldr}
\bibitem{Dealtry:2019ldr}
T.~Dealtry [Hyper-Kamiokande],
%``Hyper-Kamiokande,''
[arXiv:1904.10206 [hep-ex]].
%4 citations counted in INSPIRE as of 08 Mar 2023

%\cite{ParticleDataGroup:2022pth}
\bibitem{ParticleDataGroup:2022pth}
R.~L.~Workman \textit{et al.} [Particle Data Group],
%``Review of Particle Physics,''
PTEP \textbf{2022} (2022), 083C01
%doi:10.1093/ptep/ptac097
%725 citations counted in INSPIRE as of 08 Mar 2023

%\cite{Dorsner:2005fq}
\bibitem{Dorsner:2005fq}
I.~Dorsner and P.~Fileviez Perez,
%``Unification without supersymmetry: Neutrino mass, proton decay and light leptoquarks,''
Nucl. Phys. B \textbf{723} (2005), 53-76
%doi:10.1016/j.nuclphysb.2005.06.016
[arXiv:hep-ph/0504276 [hep-ph]].
%157 citations counted in INSPIRE as of 08 Mar 2023

%\cite{Dorsner:2006hw}
\bibitem{Dorsner:2006hw}
I.~Dorsner, P.~Fileviez Perez and G.~Rodrigo,
%``Fermion masses and the UV cutoff of the minimal realistic SU(5),''
Phys. Rev. D \textbf{75} (2007), 125007
%doi:10.1103/PhysRevD.75.125007
[arXiv:hep-ph/0607208 [hep-ph]].
%49 citations counted in INSPIRE as of 08 Mar 2023

%\cite{Ellis:1979fg}
\bibitem{Ellis:1979fg}
J.~R.~Ellis and M.~K.~Gaillard,
%``Fermion Masses and Higgs Representations in SU(5),''
Phys. Lett. B \textbf{88} (1979), 315-319
%doi:10.1016/0370-2693(79)90476-3
%262 citations counted in INSPIRE as of 08 Mar 2023

%------------------------------------------------------------------------------%
%------------------------45representation Higgs--------------------------------%
%------------------------------------------------------------------------------%

%\cite{Georgi:1979df}
\bibitem{Georgi:1979df}
H.~Georgi and C.~Jarlskog,
%``A New Lepton - Quark Mass Relation in a Unified Theory,''
Phys. Lett. B \textbf{86} (1979), 297-300
%doi:10.1016/0370-2693(79)90842-6
%919 citations counted in INSPIRE as of 08 Mar 2023

%\cite{Kalyniak:1982pt}
\bibitem{Kalyniak:1982pt}
P.~Kalyniak and J.~N.~Ng,
%``Symmetry Breaking Patterns in SU(5) With Nonminimal Higgs Fields,''
Phys. Rev. D \textbf{26} (1982), 890
%doi:10.1103/PhysRevD.26.890
%17 citations counted in INSPIRE as of 08 Mar 2023

%\cite{Eckert:1983bn}
\bibitem{Eckert:1983bn}
P.~Eckert, J.~M.~Gerard, H.~Ruegg and T.~Schucker,
%``Minimization of the SU(5) Invariant Scalar Potential for the Fortyfive-dimensional Representation,''
Phys. Lett. B \textbf{125} (1983), 385-388
%doi:10.1016/0370-2693(83)91308-4
%19 citations counted in INSPIRE as of 08 Mar 2023

%\cite{FileviezPerez:2007bcw}
\bibitem{FileviezPerez:2007bcw}
P.~Fileviez Perez,
%``Renormalizable adjoint SU(5),''
Phys. Lett. B \textbf{654} (2007), 189-193
%doi:10.1016/j.physletb.2007.07.075
[arXiv:hep-ph/0702287 [hep-ph]].
%114 citations counted in INSPIRE as of 08 Mar 2023

%\cite{Dorsner:2007fy}
\bibitem{Dorsner:2007fy}
I.~Dorsner and I.~Mocioiu,
%``Predictions from type II see-saw mechanism in SU(5),''
Nucl. Phys. B \textbf{796} (2008), 123-136
%doi:10.1016/j.nuclphysb.2007.12.004
[arXiv:0708.3332 [hep-ph]].
%54 citations counted in INSPIRE as of 25 Mar 2023

%\cite{FileviezPerez:2016sal}
\bibitem{FileviezPerez:2016sal}
P.~Fileviez Perez and C.~Murgui,
%``Renormalizable SU(5) Unification,''
Phys. Rev. D \textbf{94} (2016) no.7, 075014
%doi:10.1103/PhysRevD.94.075014
[arXiv:1604.03377 [hep-ph]].
%37 citations counted in INSPIRE as of 19 Mar 2023

%\cite{Boucenna:2017fna}
\bibitem{Boucenna:2017fna}
S.~M.~Boucenna and Q.~Shafi,
%``Axion inflation, proton decay, and leptogenesis in $SU(5)\times U(1)_{PQ}$,''
Phys. Rev. D \textbf{97} (2018) no.7, 075012
%doi:10.1103/PhysRevD.97.075012
[arXiv:1712.06526 [hep-ph]].
%21 citations counted in INSPIRE as of 08 Mar 2023

%\cite{FileviezPerez:2019fku}
\bibitem{FileviezPerez:2019fku}
P.~Fileviez P\'erez, C.~Murgui and A.~D.~Plascencia,
%``The QCD Axion and Unification,''
JHEP \textbf{11} (2019), 093
%doi:10.1007/JHEP11(2019)093
[arXiv:1908.01772 [hep-ph]].
%22 citations counted in INSPIRE as of 08 Mar 2023

%\cite{FileviezPerez:2019ssf}
\bibitem{FileviezPerez:2019ssf}
P.~Fileviez P\'erez, C.~Murgui and A.~D.~Plascencia,
%``Axion Dark Matter, Proton Decay and Unification,''
JHEP \textbf{01} (2020), 091
%doi:10.1007/JHEP01(2020)091
[arXiv:1911.05738 [hep-ph]].
%26 citations counted in INSPIRE as of 08 Mar 2023

%\cite{Shimizu:2022wsk}
\bibitem{Shimizu:2022wsk}
Y.~Shimizu and S.~Takeshita,
%``Mass Relations, Unification, and Proton Decay in the Mirror GUT Model,''
[arXiv:2212.13064 [hep-ph]].
%0 citations counted in INSPIRE as of 08 Mar 2023

%\cite{CMS:2018dcw}
\bibitem{CMS:2018dcw}
A.~M.~Sirunyan \textit{et al.} [CMS],
%``Search for single production of vector-like quarks decaying to a top quark and a W boson in proton-proton collisions at $\sqrt{s} =$ 13 TeV,''
Eur. Phys. J. C \textbf{79} (2019), 90
%doi:10.1140/epjc/s10052-019-6556-3
[arXiv:1809.08597 [hep-ex]].
%56 citations counted in INSPIRE as of 08 Mar 2023

%\cite{ATLAS:2019tpq}
\bibitem{ATLAS:2019tpq}
G.~Aad \textit{et al.} [ATLAS],
%``Search for heavy neutral Higgs bosons produced in association with $b$-quarks and decaying into $b$-quarks at $\sqrt{s}=13$ TeV with the ATLAS detector,''
Phys. Rev. D \textbf{102} (2020) no.3, 032004
%doi:10.1103/PhysRevD.102.032004
[arXiv:1907.02749 [hep-ex]].
%64 citations counted in INSPIRE as of 09 Mar 2023

%\cite{ATLAS:2021upq}
\bibitem{ATLAS:2021upq}
G.~Aad \textit{et al.} [ATLAS],
%``Search for charged Higgs bosons decaying into a top quark and a bottom quark at $ \sqrt{\mathrm{s}} $ = 13 TeV with the ATLAS detector,''
JHEP \textbf{06} (2021), 145
%doi:10.1007/JHEP06(2021)145
[arXiv:2102.10076 [hep-ex]].
%93 citations counted in INSPIRE as of 09 Mar 2023

%\cite{Machacek:1983tz}
\bibitem{Machacek:1983tz}
M.~E.~Machacek and M.~T.~Vaughn,
%``Two Loop Renormalization Group Equations in a General Quantum Field Theory. 1. Wave Function Renormalization,''
Nucl. Phys. B \textbf{222} (1983), 83-103
%doi:10.1016/0550-3213(83)90610-7
%787 citations counted in INSPIRE as of 09 Mar 2023

%\cite{Machacek:1983fi}
\bibitem{Machacek:1983fi}
M.~E.~Machacek and M.~T.~Vaughn,
%``Two Loop Renormalization Group Equations in a General Quantum Field Theory. 2. Yukawa Couplings,''
Nucl. Phys. B \textbf{236} (1984), 221-232
%doi:10.1016/0550-3213(84)90533-9
%697 citations counted in INSPIRE as of 09 Mar 2023

%\cite{Machacek:1984zw}
\bibitem{Machacek:1984zw}
M.~E.~Machacek and M.~T.~Vaughn,
%``Two Loop Renormalization Group Equations in a General Quantum Field Theory. 3. Scalar Quartic Couplings,''
Nucl. Phys. B \textbf{249} (1985), 70-92
%doi:10.1016/0550-3213(85)90040-9
%575 citations counted in INSPIRE as of 09 Mar 2023

%\cite{Nath:2006ut}
\bibitem{Nath:2006ut}
P.~Nath and P.~Fileviez Perez,
%``Proton stability in grand unified theories, in strings and in branes,''
Phys. Rept. \textbf{441} (2007), 191-317
%doi:10.1016/j.physrep.2007.02.010
[arXiv:hep-ph/0601023 [hep-ph]].
%461 citations counted in INSPIRE as of 08 Mar 2023

%\cite{Hayreter:2017wra}
\bibitem{Hayreter:2017wra}
A.~Hayreter and G.~Valencia,
%``LHC constraints on color octet scalars,''
Phys. Rev. D \textbf{96} (2017) no.3, 035004
%doi:10.1103/PhysRevD.96.035004
[arXiv:1703.04164 [hep-ph]].
%31 citations counted in INSPIRE as of 09 Mar 2023

%\cite{Miralles:2019uzg}
\bibitem{Miralles:2019uzg}
V.~Miralles and A.~Pich,
%``LHC bounds on colored scalars,''
Phys. Rev. D \textbf{100} (2019) no.11, 115042
%doi:10.1103/PhysRevD.100.115042
[arXiv:1910.07947 [hep-ph]].
%20 citations counted in INSPIRE as of 09 Mar 2023

\end{thebibliography}
\end{document}