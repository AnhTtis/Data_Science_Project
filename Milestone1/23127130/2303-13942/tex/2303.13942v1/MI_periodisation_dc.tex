
\documentclass[11pt,twocolumn]{article}


\usepackage[english,greek]{babel}
\usepackage[utf8x]{inputenc}


\usepackage[font=small]{caption}
\usepackage{amsmath,amsthm,amssymb,stmaryrd,bigints,mathabx}

\usepackage[pdftex]{graphicx}
\usepackage{subcaption}
%\usepackage{caption,subcaption,mathabx,titletoc,title}
\usepackage[titletoc,title]{appendix}
%\usepackage[all]{xy}
\usepackage{datetime,color,cancel,listings,authblk}
%\usepackage[notcite,notref]{showkeys}
\usepackage[hyperfootnotes=false]{hyperref}
\hypersetup{
  colorlinks = true,
  linkcolor = black,
  citecolor = blue,
  urlcolor = black
}
%\usepackage[usenames,dvipsnames]{xcolor}
\usepackage[margin=2.0cm]{geometry}
%\usepackage[margin=2.0cm]{geometry}

%\usepackage[square]{natbib}
%\setcitestyle{square}


\usepackage{float} 


\floatstyle{ruled}
\newfloat{algorithm}{thp}{lop}
\floatname{algorithm}{Algortithm}



\newcommand{\E}{\mathbb{E}}
\newcommand{\Hop}{\mathbb{H}}
\newcommand{\R}{\mathbb{R}}
\newcommand{\N}{\mathbb{N}}
\newcommand{\C}{\mathbb{C}}
\newcommand{\beq}{\begin{equation}}
\newcommand{\ee}{\end{equation}}
\newcommand{\bac}{\begin{array}{c}}
\newcommand{\ea}{\end{array}}
\newcommand{\bal}{\begin{aligned}}
\newcommand{\eal}{\end{aligned}}
\newcommand{\F}{\mathcal{F}}
\newcommand{\T}{\mathbb{S}}
\newcommand{\real}{\operatorname{Re}}
\newcommand{\imag}{\operatorname{Im}}
\newcommand{\alhfill}{\qquad \qquad \qquad \qquad}
\newcommand{\red}{\color{red}}
\newcommand{\BFI}{\textrm{BFI}}

\newcommand{\Expe}{\operatorname{E}}


\newcommand{\vertiii}[1]{{\left\vert\kern-0.25ex\left\vert\kern-0.25ex\left\vert #1 
    \right\vert\kern-0.25ex\right\vert\kern-0.25ex\right\vert}}



\begin{document}
\newtheorem{theorem}{Theorem}[section]
\newtheorem{lemma}[theorem]{Lemma}
\newtheorem{remark}[theorem]{Remark}
\newtheorem{observation}[theorem]{Observation}
\newtheorem{definition}[theorem]{Definition}
\newtheorem{example}[theorem]{Example}
\newtheorem{corollary}[theorem]{Corollary}
\newtheorem{assumption}{Assumption}
\newtheorem{property}{Property}

\selectlanguage{english}

\title{Bifurcation length in the numerical simulation of the modulation instability}

\author{Agissilaos G. Athanassoulis}
\author{Irene Kyza}

\affil{Department of Mathematics, University of Dundee}



\maketitle



\begin{abstract} 
The modulation instability (MI) is a well known feature of the focusing nonlinear Schr\"odinger equation (NLS), namely that plane wave solutions on the real line are linearly unstable. Simulations of the MI  typically use a large computational domain of length $L$ equipped with periodic boundary conditions.  We show for the first time that for $L$ smaller than a bifurcation length $L_c$ the MI is completely suppressed in the periodized problem, and the abrupt bifurcation is also demonstrated numerically. 
 Moreover, when the NLS is used to model  water waves with typical wavelength $\lambda_0,$ the bifurcation lengthscale is  $L_c=O(\lambda_0^2).$ Not accounting for that is artificially removing  MI-related events in simulations.


%The modulation instability (MI) is a well known feature of the focusing nonlinear Schr\"odinger equation (NLS), namely that plane wave solutions on the real line are linearly unstable. Simulations of the MI  typically use numerical solutions on a large computational domain of length $L,$ equipped with periodic boundary conditions.  In this paper, the shortest length $L$ for which the instability exists is computed, and the abrupt bifurcation is also demonstrated numerically. Crucially, $L$ depends only on the coefficients of the equation and the amplitude of the plane wave -- but not on the initial inhomogeneity. 
%If the computational domain is even slightly shorter than the bifurcation length, then 
%the  inhomogeneity is amplified much less or not at all. It is thus clearly seen that periodizing the NLS with perturbed plane wave initial data on an interval much larger than the support of the initial inhomogeneity does not necessarily give a qualitatively correct approximation of the corresponding solution on the real line. It follows that in more complex problems where MI is often considered, such as the evolution of two crossing plane waves, narrowband wavefields etc, the intrinsic length scales needed for the instability to manifest must be systematically investigated and used to inform any simulations.
\end{abstract}

\medskip 

\noindent {\bf Keywords: } Nonlinear Schr\"odinger equation, Modulation instability, bifurcation, scientific computing


%\tableofcontents

\section{Introduction}

The modulation instability (MI) is a fundamental feature of nonlinear waves. It has been re-discovered several times \cite{Zakharov2009}, and inspired a broad range of studies in  physics, mathematics and beyond. However,  a  fundamental aspect regarding the numerical simulation of the MI is not  clearly understood. The numerical solution of the NLS on the real line involves two key approximations: first the original problem is localized in space, i.e. substituted with the same equation on a bounded computational domain. Typically periodic boundary conditions are used, for two reasons: they preserve the translation invariant physics of the problem, and allow for solutions that don't decay at infinity, like perturbed plane waves which are relevant for the MI. Secondly, the periodized problem is discretized, and solved in the computer. Traditional Numerical Analysis studies how well the solution of a discretized problem approximates the solution of its continuous counterpart. However, the continuous counterpart is the periodized problem, not the original one on the real line. In this paper we show that there exists a critical length $L_c$, depending on the original equation, so that periodizing on an interval  shorther than $L_c$ abruptly stabilizes the MI. That is, for the NLS on a short enough torus, plane wave solutions are linearly stable. Striking examples of that can be seen in Table \ref{tab:1} and in Figures \ref{fig:a0}, \ref{fig:a_}, \ref{fig:ab}. 


\begin{table}
\begin{center}
\begin{tabular}{ |c|cccc| } 
 \hline
 & $N=0.98$ & $N=1.3$ & $N=3$ & $N=10$ \\
 \hline
$j=1$ & $0.0359$ & $2.59$ & $3.08$ & $3.03$ \\
$j=2$ & $0.0393$ & $2.59$ & $3.09$ & $3.03$ \\
$j=3$ & $0.149$ & $2.65$ & $3.17$ & $3.12$ \\
$j=4$ & $0.25$ & $2.69$ & $3.22$ & $3.18$ \\
$j=5$ & $0.03$ & $2.57$ & $3.05$ & $2.96$ \\
\hline
\end{tabular}
\end{center}
\caption{Maximum value of the inhomogeneity %$\mathop{\mathrm{max}}\limits_{(x,t)\in [-\frac{L}2,\frac{L}2]\times [0,10] } |\delta(x,t)|$ 
$\mathop{\mathrm{max}}\limits_{x,t} |\delta(x,t)|$ for $x\in [-L/2,L/2]$ and $t\in [0,10],$ with $L=NL_c$ (cf. equation \eqref{eq:L1}) and initial condition $\delta_j(x,0)$ (cf. equation \eqref{eq:ICinh}). See Section \ref{sec:3} for more details on the computation. } \label{tab:1}
\end{table}



Thus, no matter how sophisticated and accurate the discretization, the overall dynamics  present in a simulation may be radically different from the intended problem on the real line because of  periodization on a too short interval. For weaker nonlinearities the critical lengthscale becomes longer, and combined with the lack of awareness of the bifurcation it is possible that this artificial stabilization may  be affecting state of the art simulations. We focus on water waves, where the problem is intrinsically weakly nonlinear (thus the bifurcation length quite large). This issue becomes increasingly important as numerical solutions are more and more heavily used to shed light to complex nonlinear wave problems -- and most of these numerical solutions in fact discretize a periodized version of the NLS.

It should further be noted that the bifurcation lengthscale {does not depend on the initial inhomogeneity}. Even if  the initial inhomogeneity is well contained in the computational domain, and a sufficiently fine computational mesh is used, this is no guarantee that the results approximate the solution of the corresponding problem on the real line -- not even before the numerical solution reaches the boundary of the computational domain. %A careful comparison of Figures \ref{fig:a_} and \ref{fig:b} highlights that (note the different $x-$axis limits and different range of values in the colorbar -- the initial condition used in both simulations is seen in Figure \ref{fig:a0}).



\begin{figure}
\begin{center}
\includegraphics[width=0.49\textwidth]{Fig1a} %\, \includegraphics[width=0.49\textwidth]{Fig1b} 
\end{center}
\caption{The initial inhomogeneity $\delta_2(x,0)$ (cf. equation \eqref{eq:nls4}) periodized on a domain of length $0.98L_c.$ The initial inhomogeneity is well resolved, both in terms of number of points used and in terms of the extent of its support.  Compare however with Figures \ref{fig:a_}, \ref{fig:ab}.}
\label{fig:a0}
\end{figure}

In  Section \ref{sec:bac} we  discuss  papers that are related with the existence of the bifurcation lengthscale $L_c$. We also discuss  the ramifications for  MI in water waves more broadly.
In Section \ref{sec:1} we summarize the derivation of the classical MI for the NLS for completeness and clarity. In Section \ref{sec:2} we proceed to perform an analogous linear stability analysis for the periodized NLS. %and it is possible that other researchers have independently worked it out for themselves,   
%We are not aware of this analysis existing in the literature, nor of its ramifications being accounted for. %Indeed, to the best of our knowledge, the bifurcation  lengthscale $L_c$ that is found here does not typically show up when numerical simulations of the MI are discussed.
In Section \ref{sec:3} we use numerical simulation to explore the fully nonlinear behaviour of the problem.
For the classical MI, our finding is that any computation aiming to simulate full space should use a computational domain at the very least $3$ bifurcation lengthscales $L_c$ long, and preferably $10 L_c$ or more. %The crucial thing here is that $L_1$ {\em does not depend on the initial inhomogeneity}, but only on the coefficients of the NLS and the amplitude of the plane wave. In fact for smaller plane waves (and thus weaker nonlinear interactions), the domain required for the MI to manifest is {\em larger}.
%It is thus not inconceivable that in some cases the MI may have been unknowingly suppressed due to short computational domains -- especially when considered in a somewhat more complicated setting.
In Section \ref{sec:conc} we summarize the findings and compute the scaling of $L_c$ for water waves problems.

\section{State of the art}\label{sec:bac}


The complete suppression of the MI is a striking phenomenon, and its impact shows up indirectly in several works. In \cite{Cousins2015}  the role of lengthscales in the stability of complex wavefields was investigated. Reduced dynamics for a wavetrain of characteristic length $L$ were derived with  implicit boundary conditions  of linear dynamics away from the wavetrain. %Moreover, it was not stability under small perturbations that was examined, but rather the nonlinear evolution of $O(1)$ perturbations. While technically this is not the linear stability analysis for the periodized NLS, it is still broadly related. 
It was found that shorter lengthscales are fundamentally more stable, and would require unrealistically large-amplitude waves to be substantially amplified. In some cases, even complete stabilization for lengthsacles shorter than a critical length was found. Analogous results were found in 2D in \cite{Tang2022}.



%The MI has been used as a prototype for analogous instabilities that arise in more complicated settings. One example is that of ``crossing seas'': a model for two  wavesystems coming from different directions in the same region and interacting is a system two coupled NLS-type equations \cite{Gramstad2011,Gramstad2018a,Hammack2005}. This system is known to exhibit MI, i.e. to be linearly unstable when each wavesystem is taken to be a plane wave.  In \cite{Gramstad2018a} it is found that the maximum growth rate of instabilities is found to exhibit a strong dependence on the angle between the two wavesystems, but this dependence is apparently not reflected in the numerical simulations. It is conceivable that analogous bifurcation lengthscales exist in the periodized crossing-seas system as well, and only when these are taken into account would simulations  capture the MI.

In the experimental paper \cite{Dematteis2019} a wavelength $\lambda_0=3.5m$ is reported, and the tank is $74.5 \lambda_0$ long (excluding the sloping part). Rogue waves with well-defined profiles are recovered. The same profiles were recovered in \cite{Dematteis2017,Athanassoulis2017} with two different sets of methods.

The onset of MI is also considered for more general wavefields  -- not only plane waves. It was shown in \cite{Alber1978} that, for gaussian sea states with known power spectrum, an instability condition based on the shape of the spectrum can be derived. This instability condition and the onset of generalized MI when power spectra become narrow enough has been demonstrated e.g. in \cite{Onorato2003,Gramstad2017,Ribal2013a,Athanassoulis2018,Athanassoulis2023}. 
Numerical investigation of this predicted sudden onset of MI in realistic sea states has been complicated, as it involves many physical and numerical parameters, one of which is the size $L$ of the computational domain.  In \cite{Onorato2003}  $L=100\lambda_0$ was used. In \cite{Ribal2013a} it is reported in Appendix A that after several attempts a lengthscale that works out to $L=15 \sqrt{2} L_c$ was found to be sufficient (along with a length of $\sqrt{2} L_c$ for the distance in the two-space correlation). More recently, in \cite{Athanassoulis2023} it was found that $L$ as large as $L=126\lambda_0$ was required before the onset of the generalized MI could be observed. 
%In contrast, in   \cite{Janssen2003} it was reported that {\em ``The numerical simulations provide no convincing evidence of a bifurcation''} (p.866). Apparently there is no explicit discussion of the computational domain size in \cite{Janssen2003}.  It is possible that lack of awareness of a bifurcation lengthscale $L_c$ can lead to short computational domains and artificial stabilization of the MI.

On the other hand, there are many papers using computational domains of $10-20$ wavelengths $\lambda_0,$ with some explicitly reporting that linear effects dominate the MI in their simulations, or that the onset of  the MI is not found. Indeed, ``several wavelengths $\lambda_0$'' is the default for many in the water waves community, since $\lambda_0$ is generally thought to be the fundamental lengthscale.  However, the bifurcation length is $L_c=O(\lambda_0^2)$ as computed in detail in Section \ref{sec:conc}. Since the interesting wavelengths $\lambda_0$ are {\em in the  hundreds  of meters}, this explains e.g. the $70-130$ wavelengths that were found to be required for the MI to manifest in the various studies mentioned here.

Beyond works looking specifically at the MI, there is now huge activity on extreme events and rogue waves. {\em The widespread use of the $10-20 \lambda_0$ rule of thumb makes
 it entirely possible that many numerical experiments systematically underestimate MI-related events, because they are using computational domains that artificially stabilizes the MI.}
%It is remarkable that in many papers where numerical investigation of the MI is undertaken, there is no explicit discussion of how large was the computational domain used, nor how it was determined that the length was sufficient. More broadly, there is no standard reference for a systematic approach in choosing the computation length when a particular kind of boundary conditions are used. It must be noted that ad hoc length scales of $7-20$ wavelengths are widely used with periodic boundary conditions in the water waves community. On the other hand, there are experimental and numerical works that report requiring as many as  $70-130$ wavelengths in order to simulate MI-related events.
%Overall, there is evidence for lengthscales of several $L_c$ being necessary to see MI in simulations and experiments. At the same time, there is a lack of a clear explanation in the literature of why this is needed, along with a widespread default of  measuring the computational domain the length in wavelengths $\lambda_0$ instead of $L_c.$ This is significant because  

This work provides for the first time a systematic justification of the lengthscales required to simulate the classical MI, and a starting point for calibrating the numerical simulation of more general problems. Furthermore, this approach  can be developed and applied to crossing seas,  generalized MI and other more complex problems to quantify precisely the lengthscales required there.


\section{Stability analysis on the real line and universality of the MI} 
\label{sec:1}

Consider the NLS
\begin{equation}\label{eq:nls1}
	i \partial_t u + p \Delta u + q |u|^2u=0
\end{equation}
for a wavefunction $u=u(x,t)$ on $x\in\mathbb{R}$ and $t\in [0,T],$ with boundary conditions of   $\mathop{\sup}\limits_{x\in\mathbb{R}}|u(x,t)|< \infty$ for all $t\in [0,T].$ These boundary conditions in particular allow for plane waves.   The NLS is known to be
 well-posed in Zhidkov spaces \cite{Gallo2004,Zhidkov1987} (essentially spaces of smooth, globally bounded functions); this is the natural framework for this discussion. Also, since we are dealing with the focusing NLS, we will assume without loss of generality that $p,q>0.$


The NLS \eqref{eq:nls1} admits the exact solution
\begin{equation}
	w(x,t):= A e^{iqA^2 t},
\end{equation}
which is the simplest plane wave solution with amplitude $A>0$.
To study the stability of this plane wave solution one can consider whether small perturbations grow. This leads to the perturbed initial value problem
\begin{equation}\label{eq:nls2}
\begin{array}{c}
	i \partial_t u + p \Delta u + q |u|^2u=0,  \\
u(x,0) = A(1+\delta_0(x))
\end{array}
\end{equation}
for $x\in\mathbb{R}$ and $t\in [0,T]$
and some initial perturbation $\delta_0(x)$ which is small in an appropriate sense, $\delta_0=o(1)$. Using the substitution
\begin{equation}\label{eq:4}
	u(x,t) = A e^{iqA^2 t}(1+\delta(x,t)) 
\end{equation}
one readily computes that problem \eqref{eq:nls2} is equivalent to
\begin{equation}\label{eq:nls3}
\begin{array}{c}
i\delta_t + p\Delta \delta + q A^2 (\delta + \bar\delta) +\qquad \qquad \quad  \\
\quad \qquad \qquad + q A^2 (\delta + \bar\delta) \delta + q A^2 |\delta|^2 (1+\delta),
\\
\delta(x,0) = \delta_0(x).
\end{array}
\end{equation}
Dropping higher order terms, 
 we obtain the linearized problem for the perturbation, namely
\begin{equation}\label{eq:linearised}
\begin{array}{c}
i\delta_t + p\Delta \delta + q A^2 (\delta + \bar\delta),
\qquad 
\delta(x,0) = \delta_0(x).
\end{array}
\end{equation}
By expanding equation \eqref{eq:linearised} into its real and imaginary parts, and denoting 
\begin{equation}\label{eq:rimparts}
\delta(x,t) = \alpha(x,t) + i \beta(x,t),
\end{equation}
we eventually obtain the system
\begin{equation}\label{eq;sol0}
%\left\{
%\begin{array}{c}
%-\partial_t \beta	+p \Delta \alpha + 2qA^2 \alpha=0,\\
%\partial_t \alpha + p \Delta \beta =0
%\end{array}
%\right\} \iff 
%\left\{
\begin{array}{c}
\partial_{tt}\beta + \left( p^2 \Delta\Delta +2pq A^2\Delta \right) \beta=0, \\
\partial_t \alpha + p\Delta \beta=0.
\end{array}
%\right\}
\end{equation}
This now can be solved explicitly with separation of variables, leading to the construction of the modes
\begin{equation}\label{eq:solbetaR}
\begin{array}{c}
\beta_\zeta(x,t) = e^{i [\zeta x + \omega(\zeta) t]} + c.c., \quad \zeta \in \mathbb{R},\\ 
 \omega^2(\zeta) = \zeta^2 [p^2 \zeta^2 - 2pq A^2].
\end{array}
\end{equation}
(Here c.c. stands for complex conjugate.)
More general solutions can be formed by superpositions of these modes, 
$\beta(x,t) = \int\limits P(\zeta) \beta_{\zeta}(x,t) d\zeta.$ 
The instability is due to the fact that, 
\begin{equation}\label{eq:cond}
\begin{array}{l}
|\zeta|<A\sqrt{2\frac{q}p}  \implies 
%\implies \omega^2(\zeta) = \zeta^2 [p^2 (2\pi \zeta)^2 - 2pq A^2] < 0 \implies 
\omega(\zeta) =\pm i |\omega(\zeta)|
\end{array}
\end{equation}
and thus the corresponding modes $\beta_\zeta(x,t)$ of equation \eqref{eq:solbetaR}  contain an exponentially growing component.
%\begin{equation}
%\beta_1(x,t) = \int\limits_{\zeta=0}^{A\sqrt{2q/p}} P(\zeta) \cos(\zeta x) e^{2\pi t \zeta\sqrt{p^2(2\pi \zeta)^2-2pqA^2}} d\zeta 
%\end{equation}
Therefore the solution $\delta$ of the linearized equation \eqref{eq:linearised} generally grows exponentially in time, i.e. the plane wave solution of the NLS is linearly unstable. Moreover the unstable wavenumbers and their rate of growth follow from this analysis.

In this  problem there are three parameters, $p,q,A.$ However by rescaling the problem according to
$\tau = qA^2 \, t,$ $\chi = A\sqrt{\frac{q}p} \, x,$ $U(\chi,\tau) = \frac{1}A \, u(x,t),$
the equation is mapped to the ``canonical'' NLS
$i \partial_{\tau} U +  \partial_{\chi\chi} U +  |U|^2U=0,$
which has the plane wave solution $W(\chi,\tau)=e^{i\tau}.$ That is, the exact values of $p,q,A$ don't play an important role and the nature of the modulation instability is the same for any $p,q,A>0.$ This widely understood universality of the MI may have contributed to  an implicit expectation that the periodized problem enjoys similar properties. 

\section{Stability analysis on the circle}\label{sec:2}

Now let us consider the NLS equation \eqref{eq:nls1} for $x\in [-\frac{L}2,\frac{L}2],$ 
\begin{equation}\label{eq:nls2a}
\begin{array}{c}
i \partial_t u + p \Delta u + q |u|^2u=0, \\ 
x\in [-\frac{L}2,\frac{L}2], \,\, t\in [0,T], 
\end{array}
\end{equation}
equipped with periodic boundary conditions, 
\begin{equation}\label{eq:nls2ab}
\begin{array}{c}
\left\{\begin{array}{c}
u(-\frac{L}2,t)=u(\frac{L}2,t),  \\
	\partial_x u(-\frac{L}2,t)=\partial_x u(\frac{L}2,t),
\end{array}
\right\}
\quad \forall t \in [0,T].
\end{array}
\end{equation}
The steps described above in equations \eqref{eq:nls1}-\eqref{eq;sol0} were carried out with boundary conditions of boundedness at infinity; however, one readily checks that each step is equally valid with the periodic boundary conditions on $[-L/2,L/2].$ Indeed, the plane wave is still a solution, the algebra leading to \eqref{eq:nls3} is the same, and finally the linearization and separation of real and imaginary parts are the same. So now we have to solve the problem 
\begin{equation}\label{eq:s7856rgh}
\begin{array}{c}
\partial_{tt}\beta + \left( p^2 \Delta\Delta +2pq A^2\Delta \right) \beta=0, \\ 
\beta(-\frac{L}2,t)=\beta(\frac{L}2,t), \\ 
\partial_x\beta(-\frac{L}2,t)=\partial_x\beta(\frac{L}2,t).
\end{array}
\end{equation}
This is the first time that the parameter $L$ really comes into play, and separation of variables now leads to the {\em discrete modes}
\begin{equation}
\begin{array}{c}
\displaystyle 
	\beta_n(x,t) = e^{2\pi i(\frac{ n x}L +\omega_n t)} + c.c., \\ (2\pi \omega_n)^2 = (\frac{2\pi n}L)^2 [p^2(\frac{2\pi n}L)^2 -2pqA^2].
\end{array}
\end{equation}
While this is analogous to equation \eqref{eq:solbetaR}, there is a very important difference: {\em depending on the values of $p,q,A,L,$ there may not be even one unstable mode}. Indeed, $\omega_0=0$ and for any $0\neq n \in \mathbb{Z}$ we have 
\begin{equation}
\omega_n^2<0 %\iff (\frac{2\pi n}L)^2 -2\frac{q}pA^2<0 
\iff L> \frac{2\pi |n|}A \sqrt{\frac{p}{2q}}.
\end{equation}
That is, the computational domain $L$ has to be larger than
\begin{equation}\label{eq:L1}
L_c:=\frac{2\pi \sqrt{p}}{ A \sqrt{{2q}} }	
\end{equation}
otherwise  the MI is {\em completely suppressed}. We could say that the wavenumbers $\zeta\in(0,\zeta^*)$ are still unstable for $\zeta^*=A \sqrt{2q / p}$ just as in the real line. Indeed, any discrete wavenumbers $2\pi n/L>0$ falling in $(0,\zeta^*)$ would be unstable -- but the difference on the torus is that now a smallest positive wavenumber $2\pi/L$ exists. Thus if $2\pi/L>\zeta^*,$ there are no unstable modes in the problem -- something that was impossible on the real line. (The symmetric argument holds for negative wavenumbers.) So the numerical value of $L_c$ is unsurprisingly $2\pi/\zeta^*,$ but the change in the nature of MI from $\mathbb{R}$ to the torus (i.e. the capacity for complete elimination of the instability) is due to the discretization of the spectrum.

Observe moreover that the lengthscale $L_c$
 does not depend on the initial inhomogeneity $\delta_0(x),$ and becomes larger when the  nonlinearity becomes weaker (i.e. when $q,A>0$ decrease). 


\begin{figure}
\begin{center}
\includegraphics[width=0.49\textwidth]{autofig_2_1}%{Fig2_1} 
\end{center}
\caption{Space-time plot of $|\delta_2|$ when propagated on $0.98L_c.$}
\label{fig:a_}
\end{figure}

\begin{figure}
\begin{center}
\includegraphics[width=0.49\textwidth]{autofig_2_2}%{Fig2_2} 
\end{center}
\caption{Space-time plot of $|\delta_2|$ when propagated on $1.3L_c.$}
\label{fig:ab}
\end{figure}



\section{Numerical investigation in the number of unstable modes} \label{sec:3}

In what follows we present numerical solutions of the problem
\begin{equation}\label{eq:nls4}
\begin{array}{c}
i \partial_t u + p \Delta u + q |u|^2u=0, \\  u(-\frac{L}2,t)=u(\frac{L}2,t), \, \partial_x u(-\frac{L}2,t)=\partial_x u(\frac{L}2,t),\\
u(x,0)=A(1+\delta_j(x))
\end{array}
\end{equation}
for the initial inhomogeneities 
\begin{equation}\label{eq:ICinh}
\begin{array}{rl}
\delta_1(x) =& A(1+0.03 \cdot \mathrm{sech}(15x) \cos(5x)),\\  
\delta_2(x) =& A(1+0.03 \cdot \mathrm{sech}(15x)), \\
\delta_3(x) =& A(1+0.03 \cdot e^{-3x^2}),\\
\delta_4(x) =& A(1+0.03 \cdot e^{-x^4}), \\
\delta_5(x) =& A(1+0.06 \cdot x\cdot e^{-x^4}), \\
\delta_6(x) =& A(1+0.03 \cdot \mathrm{sech}(0.33x)), 
\end{array}
\end{equation}
and $p=q=A=1.$  The bifurcation length in this case is $L_c\approx 4.45;$ the solutions will be computed for $L=0.98 L_c,$ $L=1.3 L_c,$ $L=3L_c$ and $L=10 L_c.$  The computation time is $t\in [0,10].$ The numerical scheme used is a relaxation in time with mass and energy conservation \cite{Besse2004,Besse2021}, with second-order finite differences in space. The full code is included in the supplement.
The inhomogeneity $\delta(x,t)$ for $t\in[0,10]$ is defined as 
\begin{equation}
%u(x,t)=w(x,t)(1+\delta(x,t)) \iff	
\delta(x,t) := (u(x,t) - Ae^{i q A^2 t})\frac{1}Ae^{-i q A^2 t}
\end{equation}
consistently with equation \eqref{eq:4}.



The inhomogeneities $\delta_1$ and $\delta_2$ are localized $\mathrm{sech}$ bumps, and are quite similar to each other. $\delta_3$ and $\delta_4$ are also bumps with different profiles, and $\delta_5$ is a localised wave. $\delta_6$ is a version of $\delta_2$ dilated to have $45$ times wider support, longer than $L_c.$ Its support is larger than $L_c,$ so it is not useful to demonstrate the bifurcation -- rather it is included to highlight  the  coherent structure that emerges on long enough domains (more details below). 

In Table \ref{tab:1} the maximum moduli of the inhomogeneities are recorded when propagated on intervals of different lengths. There is a clear confirmation of the abrupt bifurcation predicted in Section \ref{sec:2}, namely an abrupt change in behaviour from $L=0.98L_c$ to $L=1.3L_c.$ 


\begin{figure}
\begin{center}
\includegraphics[width=0.49\textwidth]{autofig_2_4}%{Fig2_4} 
\end{center}
\caption{Space-time plot of $|\delta_2|$  for $L=10L_c.$ }
\label{fig:b}
\end{figure}


\begin{figure}
\begin{center}
%\includegraphics[width=0.49\textwidth]{Fig4_1} \,
\includegraphics[width=0.49\textwidth]{autofig_4_4} 
\end{center}
\caption{Space-time plot of $|\delta_4|$ for $L=10L_c.$ }
\label{fig:c}
\end{figure}


\begin{figure}
\begin{center}
\includegraphics[width=0.49\textwidth]{autofig_6_4}%{Fig3_4} 
\end{center}
\caption{Space-time plot of  $|\delta_6|$  for $L=10L_c.$ }
\label{fig:bb}
\end{figure}


On the larger intervals $L=3L_c,$ $L=10L_c,$ coherent structures emerge. These are the sign that the MI has been resolved in a manner qualitatively similar to what would happen on the real line (that is, until the structures reach close to the boundary of the computational domain) \cite{Biondini2016,Biondini2017a}. These structures all involve localised maxima supported on neighbourhoods of size roughly $L_c$ each, arranged homogeneously in a space-time cone. Moreover, the simulation of $\delta_6$ shows that these blobs
can also result from the inhomogeneity {\em shrinking,} not only expanding. All initial inhomogeneities examined here, when propagated on a large enough interval, give rise to qualitatively similar coherent structures, 
clearly seen in  Figures \ref{fig:b}, \ref{fig:c} and \ref{fig:bb}. 
To clearly see even one local maximum, we observe that a computational domain of at least $3L_c$ is required. This seems to allow for a qualitatively correct view of the tip of the cone for a short time (of course the solution no longer corresponds to the real line when the cone comes to close to the boundary). 

When the length of the domain becomes $L<L_c,$ we don't merely see a periodised or slightly off version of the cone, but something completely different:   the amplification is small or non-existent and there is no spatial pattern at all, cf. Figures \ref{fig:a_}, \ref{fig:cminus}. Indeed, for all initial data, when $L=0.98L_c,$ the inhomogeneity initially disperses rapidly and then slowly grows in homogeneous way over the whole computational domain. 

The question of a universal pattern in the MI \cite{Biondini2016,Biondini2017a} is relevant to universal profiles of Rogue waves \cite{Dematteis2017,Dematteis2019} although that discussion is beyond the scope of this paper.  In any case, the numerical results confirm that {the coherent structures emerging in the nonlinear phase of the MI exhibit the lengthscale $L_c$ independently of the initial inhomogeneity and its width}. Thus, {\em a computational domain of at least a few $L_c$ is crucial when simulations of the MI are carried out with periodic boundary conditions}. In particular, taking a domain much larger than the effective support of the initial inhomogeneity  is no guarantee that the numerical solution does look like what would happen on the real line, not even before the solution reaches the boundary.







\begin{figure}
\begin{center}
\includegraphics[width=0.49\textwidth]{autofig_4_1} 
\end{center}
\caption{Space-time plot of $|\delta_4|$ for $L=0.98L_c.$ }
\label{fig:cminus}
\end{figure}

%\begin{figure}[h!]
%\begin{center}
%\includegraphics[width=0.49\textwidth]{untitled} 
%\end{center}
%\caption{Space-time plot of $|\delta_5|$ for $L=10L_c.$ }
%\label{fig:d}
%\end{figure}


%\section{Stability analysis for crossing seas}


\section{Conclusions} \label{sec:conc}



The finding here is that periodizing the NLS can completely remove the MI, when an interval shorter than the bifurcation length $L_c$ is used. Moreover, this can present as a very abrupt bifurcation, where a small change in the length of the domain can lead to a huge difference in the dynamics. %The bifurcation length $L_c$ does not depend on the initial inhomogeneity, and the MI is seen to be essentially a non-local phenomenon.


For water waves with typical wavenumber $k_0$ (equivalently typical wavelength $\lambda_0=2\pi/k_0$), the coefficients are $p=\sqrt{g}/(8k_0^{3/2}),$ $q=\sqrt{g}k_0^{5/2}/2$ \cite{Chiang2005} leading to 
\begin{equation}\label{eq:l11}
L_c =  \frac{1}{4\pi\sqrt{2}} \frac{\lambda_0^2}{A}.
\end{equation} 
So for large wavelengths and weak nonlinearities, requirung a computational domain of many wavelengths is not surprising. Note that the waves that would be dangerous for ships in the ocean and  carry most of the surface wave energy have wavelengths in the hundreds of meters. Moreover, a spectrum that is barely narrow enough to trigger MI in the sense of \cite{Alber1978}, exhibits a weak MI -- which would likely function like a small $A$ in equation \eqref{eq:l11}. The combination of these factors  make a bifurcation length scale $L_c$ in the hundreds of wavelengths likely. This should be accounted for in simulations of extreme events and rogue waves. %The lengthscale of $15\sqrt{2} L_c$ reported as necessary in \cite{Ribal2013a} seems to confirm this. %This does not seem to be widely understood in the literature. 

%In summary, the MI is seen to be essentially a non-local phenomenon, and an intrinsic bifurcation lengthscale required for its onset is found for the first time. A systematic quantification of such these bifurcation lengthscales for more complex  problems, such as crossing seas and realistic sea states described by measured power spectra, is a compelling next step with direct bearing on our understanding of rogue waves.

%
%
%Remarkably, smaller amplitudes $A$ lead to weaker nonlinear interactions that require larger computational domains to give rise to instability at all.
%For example, for $\lambda_0=120m$ and $A=7m,$  a computational domain of $L>$
%Note that for long gravity waves (e.g. $\lambda_0$ in the hundreds of meters) this may realistically lead to a large number of wavelengths required to even have a {\em single} unstable mode present in a problem with periodic boundary conditions. For $n$ unstable modes, $L\approx n\lambda_0^2 /(2\pi \,\sqrt{8} \, A)$ would be required. 
%


\medskip
%
\noindent {\bf Acknowledgment.} The authors would like to thank Prof. G. A. Athanassoulis, Prof. T. Sapsis, Dr. O. Gramstad and Dr. T. Tang for helpful discussions.

\bibliographystyle{siam}
\bibliography{MyCollection.bib}




\end{document}


