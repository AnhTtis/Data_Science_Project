
\arxivFig{\begin{figure*}[t!]}
\osaFig{\begin{figure}[t!]}


\centering\includegraphics[width=\textwidth]{figures/concept_v2.png}
\caption{
(a) Experiment: Light from a superluminescent diode is filtered by a 3.1nm spectral filter centered at 810 nm, modulated by a random phase pattern displayed on a spatial light modulator (SLM$_1$) and coupled into a multi-mode fiber (MMF) which is the complex medium of interest. The output speckle field from the MMF is projected onto SLM$_2$, which displays another random phase pattern, and is then detected on a CMOS camera which is placed in the Fourier plane of SLM$_2$. (b) The phase patterns on the SLM are constructed in a specific macro-pixel basis with varying pixel size based on the incident field distribution. Intensities of the output modes are recorded at a given set of points at the camera, which enclose an area corresponding to the MMF core. (c) The physics-informed neural network consists of two input layers encoding phase patterns on SLM$_1$ and SLM$_2$, and a single output layer encoding the intensity pattern of the output speckle in the given basis. The hidden layer $T$ denotes the complex transformation between SLM$_1$ and SLM$_2$ in the macro-pixel bases, while $F$ is a known layer corresponding to the $2f$-lens system between SLM$_2$ and the camera.}
\label{fig:concept}

\arxivFig{\end{figure*}}
\osaFig{\end{figure}}

