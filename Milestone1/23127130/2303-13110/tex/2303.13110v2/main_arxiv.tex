% CVPR 2022 Paper Template
% based on the CVPR template provided by Ming-Ming Cheng (https://github.com/MCG-NKU/CVPR_Template)
% modified and extended by Stefan Roth (stefan.roth@NOSPAMtu-darmstadt.de)

\documentclass[10pt,twocolumn,letterpaper]{article}

%%%%%%%%% PAPER TYPE  - PLEASE UPDATE FOR FINAL VERSION
% \usepackage[review]{cvpr}      % To produce the REVIEW version
\usepackage{cvpr}              % To produce the CAMERA-READY version
%\usepackage[pagenumbers]{cvpr} % To force page numbers, e.g. for an arXiv version

% Include other packages here, before hyperref.
\usepackage[accsupp]{axessibility}  % Improves PDF readability for those with disabilities.
\usepackage{graphicx}
\usepackage{amsmath}
\usepackage{amssymb}
\usepackage{booktabs}
\usepackage{multirow}
\usepackage{paralist}
\usepackage{array}
\usepackage[normalem]{ulem}
\usepackage[accsupp]{axessibility}
\usepackage{soul}
\usepackage{url}
% \usepackage{hyperref}

%% Jinhee added
\usepackage{kotex}

\usepackage{subcaption}

% It is strongly recommended to use hyperref, especially for the review version.
% hyperref with option pagebackref eases the reviewers' job.
% Please disable hyperref *only* if you encounter grave issues, e.g. with the
% file validation for the camera-ready version.
%
% If you comment hyperref and then uncomment it, you should delete
% ReviewTempalte.aux before re-running LaTeX.
% (Or just hit 'q' on the first LaTeX run, let it finish, and you
%  should be clear).
\usepackage[pagebackref,breaklinks,colorlinks,bookmarks=false]{hyperref}


% Support for easy cross-referencing
\usepackage[capitalize]{cleveref}
\crefname{section}{Sec.}{Secs.}
\Crefname{section}{Section}{Sections}
\Crefname{table}{Table}{Tables}
\crefname{table}{Tab.}{Tabs.}

\usepackage{algorithm}
\usepackage[noend]{algpseudocode}

\usepackage{chngcntr}


%%%%%%%%% PAPER ID  - PLEASE UPDATE
\def\cvprPaperID{8858} % *** Enter the CVPR Paper ID here
\def\confName{CVPR}
\def\confYear{2023}

%%%% OUR OWN MACROS
\definecolor{purple}{rgb}{1, 0, 1}

\newcommand{\ie}{\emph{i.e.,}\xspace}
\newcommand{\eg}{\emph{e.g.,}\xspace}
\newcommand{\abr}{\emph{abbr.}\xspace}
\newcommand{\ea}{\emph{et al.}\xspace}
\newcommand{\gensync}{\emph{GenSync}\xspace}
\newcommand{\colosseum}{\emph{Colosseum}\xspace}
\newcommand{\srep}{\emph{SREP}\xspace} % Set Reconciliation Enhances
\newcommand{\srepsim}{\emph{SREPSim}\xspace}
% Propagation
\newcommand{\esrep}{\emph{E-SREP}\xspace}
\newcommand{\epsrep}{\emph{EP-SREP}\xspace}
\newcommand{\mesrep}{\emph{ME-SREP}\xspace}
\newcommand{\mempoolsync}{\emph{MempoolSync}}

\newcommand{\fref}[1]{Fig.~\ref{#1}}
\newcommand{\tref}[1]{Table~\ref{#1}}
\newcommand{\aref}[1]{Algorithm~\ref{#1}}
\newcommand{\procref}[1]{Procedure~\ref{#1}}
\newcommand{\sref}[1]{Section~\ref{#1}}
\newcommand{\lineref}[1]{line~\ref{#1}}
\newcommand{\appref}[1]{Appendix~\ref{#1}}

% Change \eqref
\LetLtxMacro{\originaleqref}{\eqref}
\renewcommand{\eqref}{Eq.~\originaleqref}

% Theorems and corollaries
\newcounter{theoremcount}
\setcounter{theoremcount}{0}
\DeclareRobustCommand{\theorem}[1]{%
  \refstepcounter{theoremcount}%
  \noindent\textit{\textbf{Theorem \thetheoremcount\label{theorem:#1}: }}%
}
\DeclareRobustCommand{\theoremref}[1]{Theorem~\ref{theorem:#1}}

\DeclareRobustCommand{\proof}{\emph{Proof:}\xspace}
\DeclareRobustCommand{\qqed}{\hfill$\blacksquare$}

\newcounter{corollcount}
\setcounter{corollcount}{0}
\DeclareRobustCommand{\coroll}[1]{%
  \refstepcounter{corollcount}%
  \noindent\textit{\textbf{Corollary \thecorollcount\label{coroll:#1}: }}%
}
\DeclareRobustCommand{\corollref}[1]{Corollary~\ref{coroll:#1}}

\newcounter{lemmacount}
\setcounter{lemmacount}{0}
\DeclareRobustCommand{\lemma}[1]{%
  \refstepcounter{lemmacount}%
  \noindent\textit{\textbf{Lemma \thelemmacount\label{lemma:#1}: }}%
}
\DeclareRobustCommand{\lemmaref}[1]{Lemma~\ref{lemma:#1}}

\newcounter{definitioncount}
\setcounter{definitioncount}{0}
\DeclareRobustCommand{\definition}[1]{%
  \refstepcounter{definitioncount}%
  \noindent\textit{\textbf{Definition \thedefinitioncount\label{definition:#1}: }}%
}
\DeclareRobustCommand{\defref}[1]{Definition~\ref{definition:#1}}

%notes of different authors
\newif\ifnotes
\notestrue
\notesfalse

\newif\ifdiff
\difftrue
\difffalse

\newcommand{\anote}[1]{\ifnotes $\ll$\textsf{\textcolor{purple}{Ari: {#1}}}$\gg$ \fi}
\newcommand{\nnote}[1]{\ifnotes $\ll$\textsf{\textcolor{orange}{Novak: {#1}}}$\gg$ \fi}
\newcommand{\diff}[1]{\ifdiff\textcolor{orange}{#1}\else#1\fi}

%%% Local Variables:
%%% mode: latex
%%% TeX-master: "main"
%%% End:



\begin{document}

%%%%%%%%% TITLE - PLEASE UPDATE
% \title{Cell-Tissue Aligned Dataset}
\title{OCELOT: Overlapped Cell on Tissue Dataset for Histopathology}
% Catfish: Cell-Tissue-Aligned Dataset for \\Detection and Segmentation in Histopathology

% % \author{%
%   Jeongun Ryu$^{1}$\thanks{: Equal contribution}\quad Jaewoong Shin$^{1}$\footnotemark[1]\quad Hae Beom Lee$^{1}$\footnotemark[1]\quad Sung Ju Hwang $^{1,2}$\\

\author{
Jeongun Ryu\thanks{: Equal contribution} \hspace{0.4mm}
Aaron Valero Puche\footnotemark[1] \hspace{0.4mm} 
JaeWoong Shin\footnotemark[1] \hspace{0.4mm}
Seonwook Park \hspace{0.4mm}
Biagio Brattoli \hspace{0.4mm}
Jinhee Lee 
\\
Wonkyung Jung \hspace{0.4mm}
Soo Ick Cho \hspace{0.4mm}
Kyunghyun Paeng \hspace{0.4mm}
Chan-Young Ock \hspace{0.4mm}
Donggeun Yoo \hspace{0.4mm}
Sérgio Pereira
\\[1mm]
Lunit Inc. \\[0.5mm]
\tt\small \{rjw0205, aaron.valero, jwoong.shin, spark, biagio, jinhee.lee,
\\
\tt\small wkjung, sooickcho, khpaeng, ock.chanyoung, dgyoo, sergio\}@lunit.io
\vspace{-3mm}
}
% For a paper whose authors are all at the same institution,
% omit the following lines up until the closing ``}''.
% Additional authors and addresses can be added with ``\and'',
% just like the second author.
% To save space, use either the email address or home page, not both
\maketitle

%%%%%%%%% ABSTRACT
\vspace{-3mm}
\begin{abstract}


\begin{abstract}
% \vspace{-1em}
The diffusion-based generative models have achieved remarkable success in text-based image generation. However, since it contains enormous randomness in generation progress, it is still challenging to apply such models for real-world visual content editing, especially in videos. 
In this paper, we propose \texttt{FateZero}, a zero-shot text-based editing method on real-world videos without per-prompt training or use-specific mask. 
\RM{Specifically, different from a pipeline of two independent inversion and then generation stages, we find the intermediate attention maps during inversions store better structure and motion information. We thus reform them to temporally casual attention and replace them in the generation progress. To further reduce the unnecessary semantic leakage of source video and enhance the editing quality, we then remix the temporally casual attentions via the cross-attention features of the source prompt as the mask.}
To edit videos consistently, we propose several techniques based on the pre-trained models. Firstly, in contrast to the straightforward DDIM inversion technique, our approach captures intermediate attention maps during inversion, which effectively retain both structural and motion information. These maps are directly fused in the editing process rather than generated during denoising. To further minimize semantic leakage of the source video, we then fuse self-attentions with a blending mask obtained by cross-attention features from the source prompt. Furthermore, we have implemented a reform of the self-attention mechanism in denoising UNet by introducing spatial-temporal attention to ensure frame consistency.
Yet succinct, our method is the first one to show the ability of zero-shot text-driven video style and local attribute editing from the trained text-to-image model. We also have a better zero-shot shape-aware editing ability based on the text-to-video model~\cite{tuneavideo}. \RM{Besides video, our unified method also achieves state-of-the-art performance in zero-shot image editing.\chenyang{Need exp or remove the zero-shot image}} Extensive experiments demonstrate our superior temporal consistency and editing capability than previous works.
% The code will be released.
% \chenyang{emphasize: our observation at inversion time} \xiaodong{replacing the bold part to the actual pipeline: \textbf{Specifically, we work on replacing and mixing the attention maps between the inversion and generation since the self-attention map keeps the structure of the original natural image and the cross-attention is semantic-related, after remixing, we replace them in the corresponding generation steps for denoising.}}
% \footnote{Since there is no general video diffusion model is publicly available, we use one-shot video generation method~(Tune-A-Video~\cite{tuneavideo}) as the pretrained video diffusion model for zero-shot video editing\xiaodong{can be removed if we actually zero-shot on video}.}.
\end{abstract}
\end{abstract}

%%%%%%%%% BODY TEXT

\section{Introduction}

The ability to reason about plans is critical for performing long-horizon tasks \citep{erol1996hierarchical, sohn2018hierarchical, sharma-etal-2022-skill}, compositional generalization \citep{corona-etal-2021-modular} and generalization to unseen tasks and environments \citep{shridhar2020alfred}.
Consider a simple long-horizon planning scenario where a robot is tasked with preparing a meal and serving it on the table. 
This presents a non-trivial planning problem since the agent needs to understand the sequence of operations required to perform the task and search for the relevant objects in the unfamiliar environment by interacting with various objects. %



Large language models have been recently shown to possess commonsense knowledge about the world such as object affordances and physical dynamics \citep{ouyang2022training,chowdhery2022palm}.
Early approaches considered text based environments and fine-tuned PLMs to predict actions given the history of past observations and actions \citep{jansen-2020-visually,micheli-fleuret-2021-language,yao-etal-2020-keep}.
Recent work has used this ability to reason about plans from text instructions in simulated household environments with simplifying assumptions such as text-only environment observations or feedback \citep{huang2022language,ahn2022can,li2022pre,logeswaran-etal-2022-shot}.


We focus on \emph{visually grounded planning} with PLMs --- the ability to adapt plans based on interaction and visual feedback from the environment.
While PLMs have strong planning commonsense priors, predictions from a PLM may not be directly realizable in the environment since the observation and action spaces are unknown.
This requires \emph{grounding} the PLM in the environment and adapting it to observe visual feedback, which is highly non-trivial.
Some prior works assume the availability of a pre-trained affordance function \citep{ahn2022can} or a success detector \citep{mirchandani2021ella}.
Notably, SayCan \citep{ahn2022can} completely decouples the PLM from observation information by selecting actions that have both high affordability (through a pre-trained affordance model) and high PLM likelihood.
Although this partially addresses the grounding problem, the use of visual feedback for action affordance alone is limited.
Often an agent must choose one of many affordable actions using information from observations.
For example, a driving agent should re-navigate and possibly turn around when encountering a ``road closed'' sign, but both turning around and driving forward are indistinguishable to SayCan because they are both affordable and the PLM is blind to observations.

Another workaround explored in prior work is translating the information in the visual observations to text using a pre-trained captioning system \citep{shridhar2021alfworld,huang2022language}.
However, it can be difficult to faithfully describe an image in words and information is lost in this inherently noisy process, which limits the information available to the planner.



Recent work shows that PLMs can be adapted for various natural language tasks by inserting tunable embeddings or soft prompts at the input of the PLM (also called prompt tuning or prefix tuning)~\citep{li-liang-2021-prefix,lester-etal-2021-power}.
This approach also extends to multi-modal understanding tasks such as image captioning \citep{mokady2021clipcap} and VQA \citep{tsimpoukelli2021multimodal} where images are encoded as soft prompts and finetuned for the target task.
Transformer based architectures have also been successfully applied to offline Reinforcement Learning in recent work \citep{chen2021decision,janner2021offline,li2022pre,reid2022can}.

Taking inspiration from these works, we propose the simple approach of embedding visual observations (`visual prompts') and \textit{directly inserting them as PLM input embeddings}.
The visual encoder and PLM are jointly trained for the target task, an approach we call \textbf{\oursfull}~(\ours).
By teaching the PLM to use observations for planning in an end to end manner, we remove the dependency on external data such as captions and affordability information that was used in prior work.
We show that this simple approach performs better than prior PLM-based planning approaches on two embodied planning benchmarks based on ALFWorld~\citep{shridhar2021alfworld} and Virtualhome~\cite{puig2018virtualhome}.



% \section{Related Work}

\subsection{Pattern discovery on systematic AI errors}

Systematic errors, sometimes coined as blind spots or unknown-unknowns \cite{BeatMachineChallengingHumans}, refer to model's failure over a group of instances that share similar semantics. There are various approaches for discovering such patterns, including algorithmic, human, or hybrid techniques.

A number of studies have shown that fully algorithmic techniques can help automatically discover unknown-unknowns \cite{lakkaraju2017identifying, coveragebasedutility}. Recently, several studies have also been proposed to advance the methods towards discovering automatic slices or subclasses that are semantically coherent \cite{DominoDiscoveringSystematicErrorsCrossModal, SpotlightGeneralMethodDiscoveringSystematic}, or to propose a framework for evaluating blindspot discovery methods in a unified manner \cite{EvaluatingSystemicErrorDetectionMethods}.

On the other hand, researchers have also explored how human intelligence can identify blind spots where automatic techniques alone do not work. Several studies \cite{BeatMachineChallengingHumans, ContradictMachineHybridApproach, HybridHumanAIWorkflowsUnknown, InvestigatingHumanMachineComplementarity} demonstrated that a well-designed crowdsourcing study can detect problematic instances. Hybrid workflows to leverage the abilities of both humans and machines \cite{HybridHumanAIWorkflowsUnknown, lakkaraju2017identifying, han2021iterative, chung2018unknownexamples} have also been explored throughout several studies in proposing collaborative human-AI workflow \cite{HybridHumanAIWorkflowsUnknown} or generating text descriptions \cite{han2021iterative} about spurious patterns.

While these studies demonstrate how human intelligence plays a significant role, tool support is still lacking to guide practitioners to inspect, identify, and mitigate systematic errors. In our study, we provide a workflow and systematic support for inspecting which systematic errors are attributed to interpretable concepts.

\subsection{Visual analytics for ML diagnostics}
Visual analytics tools in recent years have evolved to offer interactive ways for inspecting the machine learning process. In general, these tools aim to better visualize the predictive results in a model-agnostic manner or present the structure of the model in a model-specific way. Model-agnostic approaches propose to better visualize machine learning results regardless of model types. Many visualizations among them are largely designed on the grounds of confusion matrix as tree or flow diagram \cite{shen2020designing, VisualizingSurrogateDecisionTrees}, comparative visual design \cite{ManifoldModelAgnosticFrameworkInterpretation, ExplainExploreVisualExplorationMachine, olson2021contrastive, kaul2021improvingcounterfactuals, krause2017workflow}, radial \cite{VisualMethodsAnalyzingProbabilistic} or multi-axes based layout \cite{SquaresSupportingInteractivePerformance}. On the other hand, model-specific inspections also gained attention to support the inspection of a deep neural network inside its layers, neurons, or activations \cite{liu2017analyzingtraining, ShapeShopUnderstandingDeepLearning, TopoActVisuallyExploringShape, DeepVIDDeepVisualInterpretation}.

Visual analytic tools can also help inspect and explain the potential cause of systematic failures such as a shifted or skewed distribution of the training examples termed as out-of-distribution \cite{OoDAnalyzerInteractiveAnalysisOutofDistribution}, covariate or concept shift \cite{DiagnosingConceptDriftVisual} or machine biases \cite{FairVisVisualAnalyticsDiscovering, FairSightVisualAnalyticsFairness, WhatIfToolInteractiveProbing}. The OoD analyzer \cite{OoDAnalyzerInteractiveAnalysisOutofDistribution} presented a grid-based layout to visualize the distributional differences in training and test sets. The problem of concept drift was tackled and presented as visualizations in a 2D heatmap visualization \cite{DiagnosingConceptDriftVisual} or distribution-based scatterplot \cite{ConceptExplorerVisualAnalysisConcept}. Other interactive tools such as Deblinder \cite{DiscoveringValidatingAIErrorsCrowdsourced}, SEAL \cite{SEALInteractiveToolSystematicError}, or Error Analysis \cite{erroranalysis} have recently been proposed to mitigate systematic errors with subclass labeling or user-generated report. Compared to previous work, our study aims to promote a human-in-the-loop workflow consisting of tasks to identify biased patterns and their association/attribution aspects with the perspective of spurious associations.

% Recent visualization studies also proposed how to better explain them with counterfactuals [BF-1], or to present them in a form of report [BF-3]. 


\subsection{Understanding model with concept interpretability}

The XAI methods to explain the behavior of black box models \cite{InterpretabilityFeatureAttributionQuantitative, AutomaticConceptbasedExplanations, BayesianCaseModelGenerative, ConceptWhiteningInterpretable2020} have been recently expanded to a concept-level sensitivity. The method called TCAV (Testing Concept Activation Vector) \cite{InterpretabilityFeatureAttributionQuantitative} provides a post-hoc method to explain the global influence of a concept in a pre-trained model. ACE (Automatic Concept Extraction) \cite{AutomaticConceptbasedExplanations} was proposed to identify and filter interpretable concepts from the meaningful clusters of segments on the basis of TCAV. In \cite{ConceptWhiteningInterpretable2020}, Concept Whitening (CW) purposefully alters batch normalization layers to a concept whitening layer to learn an interpretable latent space. Especially, the whitening step in this method points out that the concept space needs to be preprocessed to better align concept vectors.

These concept-level interpretability methods, however, require the human ability to observe and extract semantically meaningful concepts \cite{AutomaticConceptbasedExplanations}. There are various ways to identify and extract concepts in collaboration with humans and systems \cite{AutomaticConceptbasedExplanations, NeuroCartographyScalableAutomaticVisualSummarization, zhao2021humanintheloopextraction, DASHVisualAnalyticsDebiasingImage,  ConceptExplorerVisualAnalysisConcept, ProtoSteerSteeringDeepSequence, AnchorVizFacilitatingSemanticData, ConceptVectorTextVisualAnalytics, VisualConceptProgrammingVisualAnalytics}. ConceptExtract \cite{zhao2021humanintheloopextraction} aimed to support concept extraction and classification in a human-in-the-loop workflow and visual tools. In DASH \cite{kwon2022dash}, problematic biases from irrelevant concepts can be identified through observations from users, which were proposed to be mitigated through random image generation using GAN techniques. ConceptExplainer \cite{ConceptExplorerVisualAnalysisConcept} was designed to explore the concept associations focusing on validating conceptual overlapping between classes, especially serving as a concept exploration tool for non-expert users. In \cite{VisualConceptProgrammingVisualAnalytics}, a self-supervised technique was proposed to automatically extract visual vocabulary to allow experts to refine the labeled data and understand the concepts.

Unlike existing work, our study proposes an interactive workflow of exploring concepts for the purpose of inspecting systematic errors and spurious concept associations behind them. Similar to \cite{WhatDidMyAILearn}, our human-in-the-loop workflow aims to promote the sensemaking of practitioners specifically in the problem of systematic errors where they can iteratively work on subsetting, contrasting patterns in instances, and hypothesizing spurious associations.


% All these methods including.. share the idea of defining a concept vector with a group of semantically coherent segments. While we take the approach of pre-processing steps on concept space in [] and sensitivity, we expand the utility of concept exploration towards inspecting model's false behaviors. In our study, we demonstrate that using concept interpretability can help not only interpreting the concept association towards misclassificaitons, and tracing back ... then removing the biases to further improve the quality of classification.






  While  submodular optimization problems are generally NP-hard, the celebrated greedy algorithm \cite{nemhauser1978analysis} attains a $(1-1/e)$ approximation ratio for  submodular maximization subject to uniform matroids and a $1/2$ approximation ratio for general matroid constraints. As discussed in the introduction, the  continuous greedy algorithm \cite{calinescu2011maximizing} restores the $(1-1/e)$ approximation ratio by lifting the discrete problem to the continuous domain via the multilinear relaxation. %It is worth to mention here that the multilinear relaxation is a DR-submodular function, a.k.a. a continuous function with the diminishing returns property.

Stochastic submodular maximization, in which the objective is expressed as an expectation, has gained a lot of interest in the recent years \cite{asadpour2008stochastic, zhang2022stochastic, chen2018online}. Karimi et al. \cite{karimi2017stochastic} use a concave relaxation method that achieves the $(1-1/e)$ approximation guarantee, but only  for the class of submodular coverage functions. Hassani et al.~\cite{hassani2017gradient} provide projected gradients methods for the general case of stochastic submodular problems that achieve $1/2$ approximation guarantee.  Mokhtari et al. \cite{mokhtari2020stochastic} propose stochastic  conditional gradient methods for solving both minimization and maximization  stochastic submodular optimization problems. Their method for maximization, Stochastic Continous Greedy (SCG) can be interpreted as a stochastic variant of the continuous greedy algorithm \cite{vondrak2008optimal, calinescu2011maximizing} and achieves a tight $(1-1/e)$ approximation guarantee for monotone and submodular functions. %However, all these methods suffer from two sources of randomness (one comes from sampling the objective function and the other comes from estimating the multilinear relaxation via sampling its inputs).

Our work builds upon and relies on the approach by  \"{O}zcan et al.~\cite{ozcan2021submodular}, who studied ways of accelerating the computation of gradients via a polynomial estimator. Extending on the work of Mahdian et al.~\cite{mahdian2020kelly},  \"{O}zcan et al. show that submodular functions that can be written as compositions of (a) an analytic function and (b) a multilinear function can be arbitrarily well approximated via Taylor polynomials; in turn, this gives rise to a method for approximating their multilinear relaxation in a closed form, without sampling. We leverage this method in the context of stochastic submodular optimization, showing that it can also be applied in combination with SCG of Mokhtari et al.~\cite{mokhtari2020stochastic}: this eliminates one of the two sources of randomness, thereby reducing variance at the expense of added bias. From a technical standpoint, this requires controlling the error introduced by the bias of the polynomial estimator, while simultaneously accounting for the variance inherent in SCG, due to sampling instances.   %: this eliminates the latter source of randomness by utilizing the properties of deep submodular models that result from composition over multiple layers. In order to do so, we combine the stochastic continuous greedy algorithm proposed by Mokthari et al. \cite{mokhtari2020stochastic} with the deterministic estimator proposed by
\section{\ours}
\label{sec:dataset}
\begin{figure}[t]
    \centering
    \hfill
        \includegraphics[width=\linewidth]{figures/data_sample.pdf}
    \hfill
    \vskip -7mm
    \caption{\textbf{A sample from the \ours dataset.} Each sample of the dataset consists of two input patches and the corresponding annotations. \textbf{Left} shows the large FoV patch $x_{l}$ with tissue segmentation annotation $y_{l}^{t}$, where green denotes the cancer area. \textbf{Right} shows the small FoV patch $x_{s}$ with cell point annotation $y_{s}^{c}$, where blue and yellow dots denote \textit{tumor} and \textit{background} cells, respectively. The red box indicates the size and location of the $x_{s}$ with respect to the $x_{l}$.}
    \label{fig:data_sample}
    \vspace{-0.3cm}
\end{figure}

In this section, we introduce \ours, a histopathology dataset specifically built to enable the development of methods that leverage cell and tissue relationships. Each sample of the \ours dataset $\mathcal{D}$ is composed of six components,

\begin{equation}
\mathcal{D} = \left\{\left(x_{s}, y_s^{c}, x_l, y_l^{t}, c_x, c_y\right)_i\right\}_{i=1}^{N}
\end{equation}

\noindent where $x_s, x_l$ are the small and large FoV patches extracted from the WSI, $y_s^{c}, y_l^{t}$ refer to the corresponding cell and tissue annotations, respectively, and $c_x, c_y$ are the relative coordinates of the center of $x_s$ within $x_l$. We drop the sample index $i$ for simplicity. \autoref{fig:data_sample} shows the visualization of a sample in \ours. More details about the dataset including data collection and statistics can be found in the following sub-sections. The dataset is publicly available at \href{https://lunit-io.github.io/research/publications/ocelot/}{https://lunit-io.github.io/research/publications/ocelot/}.

\subsection{Data Collection}
\label{ssec:data-collect}
We collect 306 TCGA~\cite{HUTTER2018283} WSIs from a total of 6 different organs: \textit{kidney}, \textit{head-and-neck}, \textit{prostate}, \textit{stomach}, \textit{endometrium}, and \textit{bladder}. 
From each of the WSIs, we select 1 to 3 large Regions of Interest (ROIs) for the tissue segmentation task. 
Finally, for the cell detection task, we randomly choose a smaller ROI that is fully contained within the larger tissue ROI. 
As a result, \ours includes 673 paired patches from 6 organs. The numbers of WSIs and pairs of patches per organ are detailed in \autoref{tab:dst_size}.


Some natural image datasets, such as ImageNet~\cite{deng2009imagenet} or Pascal VOC~\cite{everingham2010pascal}, include thousands of annotated images. However, annotating histopathology images is more challenging and expensive due to the scarcity of expert pathologists~\cite{c3det}. Furthermore, acquiring dense annotations for cell detection and tissue segmentation is especially time-demanding compared to higher-level tasks such as image classification. Nonetheless, in \autoref{tab:label_stats_new}, we observe that \ours is roughly double the size of the recent TIGER dataset with respect to the annotated tissue area and the number of annotated cells. 



\vspace{-4mm}
\paragraph{Patch configuration.} 
Cell detection tasks benefit from fine-grained spatial information to better capture detailed cell properties (e.g. border, shape, color, and opacity). In contrast, tissue segmentation requires a larger context to enable a better understanding of the overall structural information. Therefore, we define the FoV sizes of $x_s$ (cell detection) and $x_l$ (tissue segmentation) as 1024 $\times$ 1024 and 4096 $\times$ 4096 pixels, respectively, at a resolution of 0.2 Microns-per-Pixel (MPP). Finally, the large FoV patches and tissue annotations ($x_l$, $y_l^{t}$) are down-sampled by a factor of 4, resulting in a size of 1024 $\times$ 1024 pixels. 


\vspace{-4mm}
\paragraph{Annotation.} 
All cell-tissue pairs of patches are annotated by board-certified pathologists. Cells are labeled as points, with associated 2D coordinates and class labels. We denote the annotations in a given cell-level patch, $x_s$, as $y_s^{c}$, and consider two classes: Tumor Cell (\textit{TC}) and Background Cell (\textit{BC})\footnote{\textit{BC} includes any of the following cell categories: lymphocyte, macrophage, fibroblast, endothelial, or other remaining cell types.}.
\textit{TC} and \textit{BC} class ratios are 35.01\% and 64.99\%, respectively.
Regarding the tissue patches, $x_l$, pathologists annotate the pixel-wise segmentation maps $y_l^{t}$ with either Cancer Area (\textit{CA}) or Background (\textit{BG}) labels. A minority of pixels where the tissue class was uncertain were labeled as Unknown (\textit{UNK}). \textit{BG}, \textit{CA}, and \textit{UNK} class ratios are 55.77\%,  40.17\%, and 4.06\%, respectively. The amount of annotated cells and tissue pixels, per data split, can be found in the \supple. The detection of \textit{TC} and \textit{BC} has clinical relevance. For example, tumor purity \cite{azimi2017breast}, computed as the tumor/non-tumor cell ratio in a WSI, has a correlation with cancer prognosis \cite{mao2018low, zhang2017tumor, gong2020tumor}.

\vspace{-4mm}
\paragraph{Dataset splits.} The dataset is divided into three subsets: \textit{training}, \textit{validation}, and \textit{test}, following a $6$:$2$:$2$ ratio. To prevent information leaking among the data subsets, we randomly split the dataset per WSI, so that different patches from the same WSI are not included in multiple subsets. We maintain consistent cancer-type ratios in each subset.

\begin{table}
\small

\centering
\setlength{\tabcolsep}{0.2em}
{
\begin{tabular}{lrrrrrr}
\toprule
\multirow{2}{*}{\textbf{Organs}} & \multicolumn{3}{c}{\textbf{\# Slides}} & \multicolumn{3}{c}{\textbf{\# Patch Pairs}} \\ \cmidrule(lr){5-7} \cmidrule(lr){2-4} 
& \textbf{Train} & \textbf{Val} & \textbf{Test} &  \textbf{Train} & \textbf{Val} & \textbf{Test} \\ \midrule
Kidney & 48 & 15 & 18 & 125 & 41 & 41 \\ 
Head-neck & 13 & 5 & 6 & 27 & 9 & 10 \\ 
Prostate & 26 & 12 & 10 & 50 & 17 & 16 \\ 
Stomach & 15 & 6 & 5 & 36 & 12 & 12 \\ 
Endometrium & 38 & 13 & 13 & 86 & 29 & 25 \\ 
Bladder & 35 & 14 & 14 & 82 & 29 & 26 \\ \midrule
\textbf{Total} & 175 & 65 & 66 & 406 & 137 & 130 \\ \bottomrule
\end{tabular}
}
\vskip -2mm
\caption{\textbf{Dataset size per organ and data subset.}}
\label{tab:dst_size}
\vspace{-4mm}
\end{table}
\section{Empirical Analysis}
\label{sec:motivation}
The motivation for considering the cell-tissue relationships for the development of cell detection models stem from the biological and hierarchical arrangement of cells and tissues. These insights are further corroborated by two main empirical observations described in the following \autoref{sec-class-corr} and \autoref{poc-exp}.


\subsection{Interrelation between cell and tissue classes} \label{sec-class-corr}
\begin{table}[t]
\small
\addtolength{\leftskip} {-2cm}
\addtolength{\rightskip}{-2cm}
\setlength{\tabcolsep}{0.3em}
\centering
% \resizebox{\linewidth}{!}
{
\begin{tabular}{lrrclrr}
\cmidrule[\heavyrulewidth]{1-3}\cmidrule[\heavyrulewidth]{5-7}
\multirow{2}[2]{*}{\textbf{Cell}}   & \multicolumn{2}{c}{\textbf{Tissue}}                                       & \hspace{2em} & \multirow{2}[2]{*}{\textbf{Cell}} & \multicolumn{2}{c}{\textbf{Tissue}} \\ \cmidrule(l){2-3}\cmidrule(l){6-7}
                                    & \multicolumn{1}{c}{\textit{CA}}   & \multicolumn{1}{c}{\textit{non-CA}}   &  &                                   & \multicolumn{1}{c}{\textit{ST}} & \multicolumn{1}{c}{\textit{non-ST}} \\ \cmidrule{1-3}\cmidrule{5-7}
\textit{TC}\enspace                 & 67.7K             & 5.4K          & & \multirow{2}{*}{\textit{LC}\enspace}    & \multirow{2}{*}{45.4K}   & \multirow{2}{*}{5.4K\enspace} \\
\textit{BC}                         & 6.4K              & 35.2K         & &                                         &                                   &                               \\ \cmidrule[\heavyrulewidth]{1-3}\cmidrule[\heavyrulewidth]{5-7}
\multicolumn{3}{c}{(a) \ours} && \multicolumn{3}{c}{(b) TIGER} \\
\end{tabular}
}
\caption{\textbf{Cell counts based on the tissue class.}
Each value stands for the number of cells located inside the tissue area. \textit{TC}, \textit{BC}, \textit{LC}, \textit{CA}, and \textit{ST} stand for Tumor Cell, Background Cell, Lymphocyte Cell, Cancer Area tissue, and Stroma tissue, respectively. 
}
\label{tab:stat}
\end{table}

\begin{table}[t]
\addtolength{\leftskip} {-2cm}
\addtolength{\rightskip}{-2cm}
\setlength{\tabcolsep}{0.3em}
\centering
{
\begin{tabular}{rrcrr}
\cmidrule[\heavyrulewidth]{1-2}\cmidrule[\heavyrulewidth]{4-5}
\multicolumn{1}{c}{\textit{CA}} & \multicolumn{1}{c}{\textit{non-CA}} &\hspace{2em} & \multicolumn{1}{c}{\textit{ST}} & \multicolumn{1}{c}{\textit{non-ST}} \\
\cmidrule{1-2}\cmidrule{4-5}
40.17\% & \enspace59.83\% && 30.79\% & \enspace69.21\% \\
\cmidrule[\heavyrulewidth]{1-2}\cmidrule[\heavyrulewidth]{4-5} 
\multicolumn{2}{c}{(a) \ours} && \multicolumn{2}{c}{(b) TIGER} \\
\end{tabular}
}
\caption{\textbf{Pixel ratio among tissue classes.} 
\textit{CA}, and \textit{ST} stand for Cancer Area tissue, and Stroma tissue, respectively.}
\label{tab:stat2}

\vspace{-4mm}
\end{table}


We empirically observe the interrelation between specific cell and tissue classes by counting the amount of each annotated cell type within each tissue region, as observed in \autoref{tab:stat}. Indeed, we verify that in \ours, around 93\% of \textit{TC} are located within \textit{CA} and 85\% of \textit{BC} are found outside of the \textit{CA} tissue. Note that \textit{CA} is not the majority tissue class (\autoref{tab:stat2}), therefore, we conclude that there is, in fact, a relationship between the cell and tissue classes. We observe a similar phenomenon in the TIGER dataset when considering the \textit{LC} and \textit{ST} classes (\autoref{tab:stat} and \autoref{tab:stat2}).



In practice, pathologists classify cells by taking into account such interrelationships, since it is difficult to classify isolated cells, without considering the larger context of the tissues. 
As depicted in \autoref{fig:intro}, pathologists first need to visualize the structure of the tissues at a larger FoV. Then, they zoom in and consider the previously observed context along with the fine-grained details of each individual cell and nearby neighborhood, thus considering the cell-tissue dependencies. 
Inspired by the behavior of pathologists, we expect a cell detection model to also benefit from understanding the tissue structure from a broader viewpoint.

\subsection{Tissue-label Leaking Model} \label{poc-exp}
In the previous section, we observe a strong relationship between certain cell and tissue classes. To further validate the hypothesis that a cell detection model can leverage information from the tissue structure, we design an exploratory experiment where the tissue annotation is provided as an extra input to a cell detection model. To this end, we first crop the corresponding cell patch region from the tissue annotation $y_l^t$, and upsample it to match the size of the cell patch, $x_s$; the cropped tissue annotation is denoted as $y_s^t$. Finally, we concatenate $x_s$ and $y_s^t$ at the channels dimension and use this data to train a cell detection model. We denote this model as a \textit{Tissue-label leaking} model and illustrate it in \autoref{fig:poc}. Note that this model is not appropriate for real-world scenarios as the tissue labels are unknown at inference time, and is explored for the purpose of empirical analysis.


When we compare the performance of the \textit{tissue-label leaking} model with the standard cell detection model on the \ours dataset, we observe a significant improvement in terms of mean F1-score\footnote{True positive (TP), false positive (FP), and false negative (FN) counts are determined following \cite{SwiderskaChadaj2019LearningTD}. If a detected cell is within a valid distance ($\approx 3\mu m$) from an annotated cell and the cell class matches, it is counted as a TP, otherwise an FP. If an annotated cell is not detected, it is counted as an FN. Then, the mean F1 score across classes is computed.} performance of \textbf{+7.69} and \textbf{+9.76} in the validation and test sets, respectively.
Detailed results can be found in \autoref{tab:main}. Taking these results into consideration, we conclude that there is significant room to improve the cell detection model, which can be achieved by combining the tasks of cell detection and tissue segmentation. 

\begin{figure}[t]
    \centering
    %\includegraphics[width=\linewidth]{figures/fig_3_label_leaking_new.pdf}
    \includegraphics[width=0.9\linewidth]{figures/fig3_dg.pdf}
    % \vskip -3mm
    \caption{  
        \textbf{Tissue label leaking model.} 
        This model receives the cell patch $x_s$ along with the corresponding tissue labels as input. 
        The region corresponding to the cell patch in the tissue patch annotation is cropped, upsampled, and concatenated to the cell patch. $\oplus$ denotes channel-wise concatenation.
    }
    \label{fig:poc}
    \vspace{-0.3cm}
\end{figure}
\section{Method}
\label{sec:approach}
In this section, we propose to utilize cell-tissue relationships through multi-task learning. 
First, we propose a set of approaches inspired by the \textit{tissue-label leaking} model described in \autoref{poc-exp}. These models replace the annotated tissue labels with predictions from an auxiliary tissue segmentation branch.
Second, we design a bi-directional information-sharing approach that shares features in both tissue-to-cell and cell-to-tissue directions. The proposed approaches are described in \autoref{subsec:pred-to-x} and \autoref{subsec:arch-search}

\subsection{Preliminary} \label{subsec:preliminary}
To deal with cell-tissue sample pairs, i.e., $(x_s, y_s^c)$ and $(x_l, y_l^t)$, we build a dual-branch architecture containing separate networks for the cell and tissue tasks. Similarly to \cite{SwiderskaChadaj2019LearningTD}, we define cell detection as a segmentation task. Specifically, the cell labels are provided as a segmentation map by drawing a fixed-radius circle centered on each cell point annotation and filled with the corresponding class label.
At inference time, we find local peaks within the predicted cell probability maps and output them as point predictions.
More details are provided in the supplementary material.
Treating cell detection as a segmentation task enables us to use the same architecture for both cell and tissue branches, which largely simplifies the training and tuning of the model and reduces the range of possible decisions, such as neural network architecture, or hyper-parameters.
We use DeepLabV3+~\cite{deeplabv3plus2018} as a base architecture for both branches and single-task models. 

\subsection{Tissue-prediction Injection Models} \label{subsec:pred-to-x}
These models are a simple and practical extension of the \textit{tissue-label leaking} model, where we inject the predicted tissue probabilities into the cell detection branch instead of leaking the tissue labels. We consider only one injection point in the cell detection branch, but explore four possible alternatives: (a) at the input (\textit{Pred-to-input}), (b) after the encoder (\textit{Pred-to-inter-1}), (c) after the ASPP module (\textit{Pred-to-inter-2}), and (d) after the decoder (\textit{Pred-to-output}). \autoref{fig:pred_to_x} depicts this family of models, denominated as \textit{Tissue-prediction injection}.
Since the tissue and cell patches represent different regions, we need to align the content between the tissue and cell feature maps before concatenation. Therefore, we crop the cell corresponding region from the tissue predictions, upsample them, and, finally, concatenate them in the channel dimensions of the feature maps of the cell branch, as illustrated in \autoref{fig:tissue_to_cell}.

\begin{figure}[t]
	\centering
    \includegraphics[width=\linewidth]{figures/fig4_font.png}
	\caption{\textbf{Tissue-prediction injection model} injects the tissue segmentation prediction into 1 out of 4 locations of the cell branch: (a) input, (b) after encoder, (c) after ASPP, and (d) after decoder.}
	\label{fig:pred_to_x}
    \vspace{-0.3cm}
\end{figure}

\subsection{Cell-Tissue Feature Sharing Model} \label{subsec:arch-search}
\textit{Tissue-prediction injection} models share the tissue prediction in a single location and direction, i.e., tissue-to-cell. To enable a more diverse and flexible cell-tissue information flow, we also explore bi-directional feature map sharing from cell-to-tissue (\autoref{fig:info_exchange}, left) and tissue-to-cell (\autoref{fig:info_exchange}, right). Considering these two operations, we conduct an architecture search procedure to find the optimal feature map sharing configuration between both branches. To limit the search space, we constrain it to only 3 positions in the architecture: after the encoder, after the ASPP module, or after the decoder. Furthermore, we also exclusively allow the branches to inject feature maps at the same depth or position.
Finally, we consider only the best-performing model among the $4^3$ candidates, which is presented in \autoref{fig:our_model}. 
We name this model as \textit{cell-tissue feature sharing}.


\begin{figure}[t]
    \centering
    \hfill
    \begin{subfigure}{0.44\linewidth}
        \includegraphics[width=\linewidth]{figures/tissue_to_cell_v2.png}
        \caption{Tissue to Cell}
        \label{fig:tissue_to_cell}
        \end{subfigure}
    \hfill
    \begin{subfigure}{0.48\linewidth}
        \includegraphics[width=\linewidth]{figures/cell_to_tissue_v2.png}
        \caption{Cell to Tissue}
        \label{fig:cell_to_tissue}
    \end{subfigure}
    \hfill
	\caption{Information is shared between cell and tissue branches via channel-wise concatenation preceded by a shallow convolutional layer with $3\times3$ kernel size. Cropping and upsampling (in \autoref{fig:tissue_to_cell}) or downsampling and zero-padding (in \autoref{fig:cell_to_tissue}) is applied to match the patch sizes and pixel-alignment between two feature maps from different FoVs. The cell and tissue feature maps are represented in orange and blue, respectively. The red contour denotes the cell patch-associated region in the tissue patch, and the gray regions represent zero padding.}
    \label{fig:info_exchange}
\end{figure}

\begin{figure}[t]
	\centering
    \includegraphics[width=\linewidth]{figures/arch_search_new2.pdf}
	\caption{\textbf{Cell-Tissue Feature Sharing Model} has two branches for tissue segmentation and cell detection.  Information exchange occurs multiple times between the two branches, indicated by the red vertical arrows. Details regarding the information exchange procedures are described in \autoref{fig:info_exchange}.}
	\label{fig:our_model}
\vspace{-4mm}
\end{figure}

  We evaluate Alg.~\ref{alg: SCG}%the Stochastic Continuous Greedy (SCG) algorithm
  , %described in \ref{alg: SCG}, 
  with sampling and polynomial estimators over two well-known problem instances (influence maximization and facility location%, and data summarization
  ) with real and synthetic %different graph settings
  datasets. We summarize these setups in Tab.~\ref{tab:datasets}. For a more detailed overview of the datasets and experiment parameters, please refer to App.~\ref{app:exps}\deleted{of the supplement}. Our code \replaced{is publicly accessible}{will be public once the submission is reviewed}.\footnote{\url{https://github.com/neu-spiral/StochSubMax}}
  
\begin{wraptable}{r}{6cm}
% \begin{table}[t]
\vspace*{-25pt}
\begin{center}
    \begin{tabular}{|c|c|ccc|cc|}
    \hline
    \thead{instance} & \thead{dataset} & \thead{$|z|$} & \thead{$|S|$} & \thead{$|E|$} & \thead{m} & \thead{k} \\%& \thead{$f^*$}\\
    \hline
    % \thead{IM} & \texttt{GreedyTricker} & 1 & 12 & 13 & 2 & 1 \\%& 0.6\\
    \thead{IM} & \texttt{SBPL} & 20 & 400 & 914 & 4 & 1 \\%& 0.06\\
    % \thead{IM} & \texttt{SyntheticBipartiteUniform} & 100 & 200 & 400 & 4 & 2 & 0.35\\
    \thead{IM} & \texttt{ZKC} & 20 & 34 & 78 & 2 & 3 \\%& -\\
    \thead{FL} & \texttt{MovieLens} & 4000 & 6041 & 256 & 10 & 2 \\%& -\\
    % \thead{SM} & \texttt{MovieLens} & - & - & - & - & - & -\\
    % \thead{SM} & \texttt{Twitter} & - & 42104 & - & 30 & 2 & -\\
    \hline
    \end{tabular}
    %\vspace*{-10pt}
\caption{{Datasets and Experiment Parameters.}}\label{tab:datasets}\end{center}
\vspace*{-25pt}
% \end{table}
\end{wraptable}
 
\noindent\textbf{Algorithms.} We compare the performance of different estimators. These estimators are: (a) sampling estimator (SAMP) with $N = 1, 10, 20, 100$ %, 1000$
and (b) polynomial estimator (POLY) with $L = 1, 2%, 3
$. %We also vary the number of iterations $T$ of the SCG algorithm where $T = 100, 200, 500, 1000, 2000$.

% \begin{table*}[t] \label{tab:final_estimates}
% \caption{Obtained utilities under different estimators}
% \resizebox{\textwidth}{!}{
%     \centering
%     \begin{tabular}{l r r r r r r} 
%     \hline
%     $\texttt{dataset}$ & SAMP1 & SAMP10 & SAMP20 & SAMP100 & POLY1 & POLY2\\% & \multicolumn{2}{c}{POLY3} \\
%     \hline
%     % \texttt{GreedyTricker} & 0.538 & 0.557 & 0.553 & 0.546 & \textbf{0.571} & 0.546 \\% & - & - \\
%     \hline
%     \texttt{SyntheticBipartitePowerLaw} & 0.049 & 0.049 & 0.049 & - & 0.061 & \textbf{0.062} \\
%     \hline
%     \texttt{ZKC} & 0.324 & 0.326 & 0.324 & 0.327 & 0.318 & \textbf{0.332} \\
%     \hline
%     \texttt{MovieLens} & 0.031 & 0.031 & 0.031 & 0.031 & \textbf{0.051} & - \\
%     \hline
% \end{tabular}}
% \end{table*}

\noindent\textbf{Metrics.} We evaluate the performance of the estimators with their clock running time and via %$f^*$, where $f^* = \max f(\mathbf{y})$ is 
the maximum result ($\max f(\mathbf{y})$) obtained using the best available estimator for a given setting.

\noindent\textbf{Results.} The trajectory of the utility obtained at each iteration of the stochastic continuous greey algorithm $f(\mathbf{y})$ is plotted as a function of time in Fig.~\ref{fig:CGiters}. %In Fig.~\ref{fig:GreedyTricker_loglog} POLY1 outperforms other estimators including the sampling estimators in terms of utility. Moreover, POLY1 is more than $10$ times faster than SAMP20 while it runs in comparable time to SAMP1. 
In Fig.~\ref{fig:SBPL_loglog}, we observe that polynomial estimators outperforms sampling estimators in terms of utility. Moreover, POLY1 runs $10$ times faster than SAMP20 and runs in comparable time to SAMP1. In Fig.~\ref{fig:ZKC_loglog}, POLY2 outperforms all estimators whereas POLY1 slightly underperforms. Finally, in Fig.~\ref{fig:MovieLens_loglog} we observe that POLY1 consistently outperforms sampling estimators.

The final outcomes of the objective functions of the estimators are reported as a function of time in Fig.~\ref{fig:final_estimates}. In Fig.~\ref{fig:SBPL_paretolog} and~\ref{fig:ZKC_paretolog}, POLY2 outperforms other estimators in terms of utility. Again in Fig.~\ref{fig:SBPL_paretolog}, POLY1 outperforms sampling estimators in terms of utility and runs in comparable time to SAMP1 while in Fig.~\ref{fig:MovieLens_paretolog}, POLY1 outperforms sampling estimators both in terms of time and utility. %Highest objective value is highlighted for each example. Based on this table, we can conclude that the polynomial estimators are better choices than the sampling estimators.
Ideally, we would expect the performance of the estimators to improve as the degree of the polynomial or the number of samples increase. The examples where this is not always the case can be explained by the stochastic nature of the problem.


\begin{figure}[t]
\centering
% \subfigure[\texttt{GreedyTricker}]{
% \begin{minipage}{0.46\linewidth}
% \centering
% \includegraphics[width=1\linewidth]{images/GreedyTricker_logtime.eps}
% \centering
% \label{fig:GreedyTricker_loglog}\vspace*{-10pt}
% \end{minipage}
% }
\subfigure[\texttt{SyntheticBipartitePowerLaw}]{
\begin{minipage}{0.31\linewidth}
\centering
\includegraphics[width=1\linewidth]{images/SyntheticBipartitePowerLaw_logtime.eps}
\centering
\label{fig:SBPL_loglog}\vspace*{-10pt}
\end{minipage}
}
% \subfigure[\texttt{SyntheticBipartiteUniform}]{
% \begin{minipage}{0.45\linewidth}
% \centering
% \includegraphics[width=1\linewidth]{images/RB100_uniform_100_100_400_k_2_100_FW_logtime.eps}
% \label{fig:FLsynth1_loglog}\vspace*{-10pt}
% \end{minipage}
% }
\subfigure[\texttt{ZKC}]{
\begin{minipage}{0.31\linewidth}
\centering
\includegraphics[width=1\linewidth]{images/zkc_logtime.eps}
\label{fig:ZKC_loglog}\vspace*{-10pt}
\end{minipage}
}
\subfigure[\texttt{MovieLens}]{
\begin{minipage}{0.31\linewidth}
\centering
\includegraphics[width=1\linewidth]{images/MovieLens_logtime.eps}
\centering
\label{fig:MovieLens_loglog}
\end{minipage}
}
% \subfigure[\texttt{Twitter}]{
% \begin{minipage}{0.45\linewidth}
% \centering
% \includegraphics[width=1\linewidth]{images/emptyplot.png}
% \label{fig:6}
% \end{minipage}
% }
\vspace*{-10pt}
\caption{Trajectory of the FW algorithm. Utility of the function at the current $\vc{y}$ as a function of time is marked for every %$10$th 
iteration.} 
 \vspace*{-13pt}
\label{fig:CGiters}
\end{figure}

\begin{figure}[t]
\centering
% \subfigure[\texttt{GreedyTricker}]{
% \begin{minipage}{0.45\linewidth}
% \centering
% \includegraphics[width=1\linewidth]{images/IM_fooler_bipartite_N_5_k_1_100_FW_paretolog.eps}
% \label{fig:IMsynth1_paretolog}\vspace*{-12pt}
% \end{minipage}
% }
\subfigure[\texttt{SBPL}]{
\begin{minipage}{0.30\linewidth}
\centering
\includegraphics[width=1\linewidth]{images/IM_RB20powerlaw_200_200_914_k_1_100_FW_paretolog.eps}
\label{fig:SBPL_paretolog}\vspace*{-12pt}
\end{minipage}
}
% \subfigure[\texttt{SyntheticBipartiteUniform}]{
% \begin{minipage}{0.45\linewidth}
% \centering
% \includegraphics[width=1\linewidth]{images/RB100uniform_100_100_400_k_2_100_FW_paretolog.eps}
% \label{fig:FLsynth1_paretolog}\vspace*{-12pt}
% \end{minipage}
% }
\subfigure[\texttt{ZKC}]{
\begin{minipage}{0.30\linewidth}
\centering
\includegraphics[width=1\linewidth]{images/ZKC_paretolog.eps}
\label{fig:ZKC_paretolog}\vspace*{-12pt}
\end{minipage}
}
\subfigure[\texttt{MovieLens}]{
\begin{minipage}{0.30\linewidth}
\centering
\includegraphics[width=1\linewidth]{images/MovieLens_paretolog.eps}
\label{fig:MovieLens_paretolog}\vspace*{-12pt}
\end{minipage}
}
% \subfigure[\texttt{Twitter}]{
% \begin{minipage}{0.45\linewidth}
% \centering
% \includegraphics[width=1\linewidth]{images/emptyplot.png}
% \label{fig:SMsynth1_paretolog}\vspace*{-12pt}
% \end{minipage}
% }
\vspace*{-10pt}
\caption{Comparison of different estimators on different problems. Blue lines represent the performance of the POLY estimators and the marked points correspond to POLY1 and POLY2 %, POLY3 
respectively. Orange lines represent the performance of the SAMP estimators and the marked points correspond to SAMP1, SAMP10, SAMP20, SAMP100 respectively.}
%\vspace*{-15pt}
\label{fig:final_estimates}
\end{figure}
This work presented a simulation approach that centers around finding iteratively an approximation of the evolution of the algebraic variables in the power system \glspl{DAE}. The approximation of the dynamic state evolutions by NNs, instead of classical explicit numerical integration schemes, allows larger time-steps to be realized while being fast to execute. This work aimed at providing a proof of concept, it is foreseeable that future work on this method shares many typical questions with established \gls{DAE} solvers, hence, by applying various existing techniques the computational performance and scalability of the approach should improve significantly.


% %%%%%%%%% REFERENCES
{\small
\bibliographystyle{ieee_fullname}
\bibliography{egbib}
}

\clearpage

\appendix
\onecolumn
\addcontentsline{toc}{section}{Appendices}

\clearpage

\section{Detailed Local Data Distribution}
\label{sec:dataset_summary}

We adopt a Latent Dirichlet Allocation (LDA) strategy for Non-IID setting \cite{fedma, moon}, where each client $k$ is assigned the partition of classes by sampling $\mathbf{p}_k \sim Dir(\alpha \cdot \mathds{1})$, where $\mathds{1} \in \, \mathbb{R}^C$.
$\alpha$ is a concentration parameter that controls the local heterogeneity level. The smaller $\alpha$, the more heterogeneous data distribution.
Since we consider a fairness issue in the FAL framework, the total number of samples should be equally partitioned for all clients.
Therefore, we made a doubly stochastic matrix $P = [\tilde{\mathbf{p}}_1, \dots, \tilde{\mathbf{p}}_K]^\top$ by scaling $\mathbf{p}_k$ to $\tilde{ \mathbf{p}}_k$, when the number of client and class are same (i.e., $P$ is a square matrix).
Note that we set the sum of columns and rows to the proper values for a non-square matrix.
We visualized the examples of CIFAR-10 when the clients $K=10$ in Figure \ref{fig:data_dist}.

\begin{figure}[h!]
\centering
\begin{minipage}{0.94\linewidth}
    \begin{subfigure}[b]{0.32\linewidth}
    \includegraphics[width=\linewidth]{figure/data_distribution/rho1/dir0.1_rho1.pdf}
    \caption{\small {$\rho$ = 1 and $\alpha$ = 0.1}}
    \end{subfigure}
    \hfill
    \begin{subfigure}[b]{0.32\linewidth}
    \includegraphics[width=\linewidth]{figure/data_distribution/rho1/dir1_rho1.pdf}
    \caption{\small {$\rho$ = 1 and $\alpha$ = 1.0}}
    \end{subfigure}
    \hfill
    \begin{subfigure}[b]{0.32\linewidth}
    \includegraphics[width=\linewidth]{figure/data_distribution/rho1/dir10000_rho1.pdf}
    \caption{\small {$\rho$ = 1 and $\alpha$ = $\infty$}}
    \end{subfigure}
    \begin{subfigure}[b]{0.32\linewidth}
    \centering
    \includegraphics[width=\linewidth]{figure/data_distribution/rho5/dir0.1_rho5.pdf}
    \caption{\small {$\rho$ = 5 and $\alpha$ = 0.1}}
    \end{subfigure}
    \hfill
    \begin{subfigure}[b]{0.32\linewidth}
    \centering
    \includegraphics[width=\linewidth]{figure/data_distribution/rho5/dir1_rho5.pdf}
    \caption{\small {$\rho$ = 5 and $\alpha$ = 1.0}}
    \end{subfigure}
    \hfill
    \begin{subfigure}[b]{0.32\linewidth}
    \centering
    \includegraphics[width=\linewidth]{figure/data_distribution/rho5/dir10000_rho5.pdf}
    \caption{\small {$\rho$ = 5 and $\alpha$ = $\infty$}}
    \end{subfigure}
    \begin{subfigure}[b]{0.32\linewidth}
    \centering
    \includegraphics[width=\linewidth]{figure/data_distribution/rho10/dir0.1_rho10.pdf}
    \caption{\small {$\rho$ = 10 and $\alpha$ = 0.1}}
    \end{subfigure}
    \hfill
    \begin{subfigure}[b]{0.32\linewidth}
    \centering
    \includegraphics[width=\linewidth]{figure/data_distribution/rho10/dir1_rho10.pdf}
    \caption{\small {$\rho$ = 10 and $\alpha$ = 1.0}}
    \end{subfigure}
    \hfill
    \begin{subfigure}[b]{0.32\linewidth}
    \centering
    \includegraphics[width=\linewidth]{figure/data_distribution/rho10/dir10000_rho10.pdf}
    \caption{\small {$\rho$ = 10 and $\alpha$ = $\infty$}}
    \end{subfigure}
    \begin{subfigure}[b]{0.32\linewidth}
    \centering
    \includegraphics[width=\linewidth]{figure/data_distribution/rho20/dir0.1_rho20.pdf}
    \caption{\small {$\rho$ = 20 and $\alpha$ = 0.1}}
    \end{subfigure}
    \hfill
    \begin{subfigure}[b]{0.32\linewidth}
    \centering
    \includegraphics[width=\linewidth]{figure/data_distribution/rho20/dir1_rho20.pdf}
    \caption{\small {$\rho$ = 20 and $\alpha$ = 1.0}}
    \end{subfigure}
    \hfill
    \begin{subfigure}[b]{0.32\linewidth}
    \centering
    \includegraphics[width=\linewidth]{figure/data_distribution/rho20/dir10000_rho20.pdf}
    \caption{\small {$\rho$ = 20 and $\alpha$ = $\infty$}}
    \end{subfigure}
\end{minipage}
\caption{Visualization of the client data distribution on CIFAR-10. Each color represents a different class. The higher $\rho$ denotes the more global imbalanced distribution. The higher $\alpha$ denotes the more locally balanced data.}
\label{fig:data_dist}
\end{figure}

\clearpage

\section{Detailed Analysis Results}
\label{sec:detail_analysis}


\subsection{Detailed Matrices for Data Counts and Accuracy}
\label{sec:detail_analysis_matrices}

We summarized the detailed matrices for the combinations of $\rho$ = $\{$1, 5, 10, 20$\}$ and $\alpha$ = $\{$0.1, 1.0, $\infty \}$.

\begin{figure*}[!h]
\centering
\begin{minipage}{0.8\linewidth}
    \centering
    \begin{subfigure}[b]{0.32\linewidth}
    \centering
    \includegraphics[width=\linewidth]{figure/analysis/obs2/appendix/rho1_alpha01_total.pdf}
    \caption{$\rho$ = 1 and $\alpha$ = 0.1}         
    \end{subfigure}
    \hfill
    \centering
    \begin{subfigure}[b]{0.32\linewidth}
    \centering
    \includegraphics[width=\linewidth]{figure/analysis/obs2/appendix/rho1_alpha1_total_v.pdf}
    \caption{$\alpha$ = 1.0}       
    \end{subfigure}
    \hfill
    \begin{subfigure}[b]{0.32\linewidth}
    \centering
    \includegraphics[width=\linewidth]{figure/analysis/obs2/appendix/rho1_alpha_inf_total.pdf}
    \caption{$\alpha$ = $\infty$}     
    \end{subfigure}
    \vspace*{-0.2cm}
\end{minipage}
\caption{Matrices of data count (top) and class-wise accuracy (down) when $\rho$ = 1.}
\vspace*{-0.2cm}
\label{fig:cnt_acc_rho1}
\end{figure*}

\vspace{-10pt}
\begin{figure*}[!h]
\centering
\begin{minipage}{0.8\linewidth}
    \centering
    \begin{subfigure}[b]{0.32\linewidth}
    \centering
    \includegraphics[width=\linewidth]{figure/analysis/obs2/appendix/rho5_alpha01_total.pdf}
    \caption{$\alpha$ = 0.1}         
    \end{subfigure}
    \hfill
    \centering
    \begin{subfigure}[b]{0.32\linewidth}
    \centering
    \includegraphics[width=\linewidth]{figure/analysis/obs2/appendix/rho5_alpha1_total.pdf}
    \caption{$\alpha$ = 1.0}       
    \end{subfigure}
    \hfill
    \begin{subfigure}[b]{0.32\linewidth}
    \centering
    \includegraphics[width=\linewidth]{figure/analysis/obs2/appendix/rho5_alpha_inf_total.pdf}
    \caption{$\alpha$ = $\infty$}     
    \end{subfigure}
    \vspace*{-0.2cm}
\end{minipage}
\caption{Matrices of data count (top) and class-wise accuracy (down) when $\rho$ = 5.}
\vspace*{-0.2cm}
\label{fig:cnt_acc_rho5}
\end{figure*}

\newpage

\begin{figure*}[!h]
\centering
\begin{minipage}{0.8\linewidth}
    \centering
    \begin{subfigure}[b]{0.32\linewidth}
    \centering
    \includegraphics[width=\linewidth]{figure/analysis/obs2/appendix/rho10_alpha01_total.pdf}
    \caption{$\alpha$ = 0.1}         
    \end{subfigure}
    \hfill
    \centering
    \begin{subfigure}[b]{0.32\linewidth}
    \centering
    \includegraphics[width=\linewidth]{figure/analysis/obs2/appendix/rho10_alpha1_total.pdf}
    \caption{$\alpha$ = 1.0}       
    \end{subfigure}
    \hfill
    \begin{subfigure}[b]{0.32\linewidth}
    \centering
    \includegraphics[width=\linewidth]{figure/analysis/obs2/appendix/rho10_alpha_inf_total.pdf}
    \caption{$\alpha$ = $\infty$}     
    \end{subfigure}
    \vspace*{-0.2cm}
\end{minipage}
\caption{Matrices of data count (top) and class-wise accuracy (down) when $\rho$ = 10.}
\vspace*{-0.2cm}
\label{fig:cnt_acc_rho10}
\end{figure*}


\begin{figure*}[!h]
\centering
\begin{minipage}{0.8\linewidth}
    \centering
    \begin{subfigure}[b]{0.32\linewidth}
    \centering
    \includegraphics[width=\linewidth]{figure/analysis/obs2/appendix/rho20_alpha_01_total.pdf}
    \caption{$\alpha$ = 0.1}         
    \end{subfigure}
    \hfill
    \centering
    \begin{subfigure}[b]{0.32\linewidth}
    \centering
    \includegraphics[width=\linewidth]{figure/analysis/obs2/appendix/rho20_alpha_1_total.pdf}
    \caption{$\alpha$ = 1.0}       
    \end{subfigure}
    \hfill
    \begin{subfigure}[b]{0.32\linewidth}
    \centering
    \includegraphics[width=\linewidth]{figure/analysis/obs2/appendix/rho20_alpha_inf_total.pdf}
    \caption{$\alpha$ = $\infty$}     
    \end{subfigure}
    \vspace*{-0.2cm}
\end{minipage}
\caption{Matrices of data count (top) and class-wise accuracy (down) when $\rho$ = 20.}
\vspace*{-0.2cm}
\label{fig:cnt_acc_rho20}
\end{figure*}

\newpage
\subsection{Detailed Earth Mover Distance}
\label{sec:detail_analysis_emd}
Table \ref{tab:detail_emd} \ summarizes the detailed local and global EMD for the combinations of $\rho$ = $\{$1, 5, 10, 20$\}$ and $\alpha$ = $\{$0.1, 1.0, $\infty \}$.

\begin{table}[H]
\centering
\small
\renewcommand*{\arraystretch}{0.85}
\addtolength{\tabcolsep}{1pt}
\resizebox{0.8\linewidth}{!}{
\begin{tabular}{cc|c|ccccc|ccccc}
\toprule
\multirow{2}{*}{$\rho$} &
  \multirow{2}{*}{$\alpha$} &
  \multirow{2}{*}{model} &
  \multicolumn{5}{c|}{Local EMD $(\downarrow)$} &
  \multicolumn{5}{c}{Global EMD $(\downarrow)$} \\
\cmidrule(l{2pt}r{2pt}){4-8} \cmidrule(l{2pt}r{2pt}){8-13}
                   &                           &   & 10\%  & 20\%  & 30\%  & 40\%  & 50\%  & 10\%  & 20\%  & 30\%  & 40\%  & 50\%  \\
                   \midrule
\multirow{2}{*}{1} & \multirow{2}{*}{0.1}      & G & 0.632 & 0.638 & 0.641 & 0.643 & 0.646 & 0.019 & 0.064 & 0.086 & 0.095 & 0.091 \\
                   &                           & L & 0.632 & 0.597 & 0.592 & 0.595 & 0.601 & 0.019 & 0.050 & 0.050 & 0.046 & 0.055 \\ \midrule
\multirow{2}{*}{1} & \multirow{2}{*}{1.0}      & G & 0.297 & 0.297 & 0.300 & 0.300 & 0.300 & 0.017 & 0.066 & 0.079 & 0.084 & 0.083 \\
                   &                           & L & 0.297 & 0.248 & 0.232 & 0.235 & 0.241 & 0.017 & 0.053 & 0.065 & 0.068 & 0.074 \\ \midrule
\multirow{2}{*}{1} & \multirow{2}{*}{$\infty$} & G & 0.049 & 0.077 & 0.070 & 0.065 & 0.061 & 0.014 & 0.070 & 0.066 & 0.063 & 0.060 \\
                   &                           & L & 0.049 & 0.042 & 0.054 & 0.059 & 0.066 & 0.014 & 0.025 & 0.044 & 0.053 & 0.062 \\  \midrule
% Rho = 5

\multirow{2}{*}{5} & \multirow{2}{*}{0.1}      & G & 0.662 & 0.663 & 0.666 & 0.666 & 0.669 & 0.211 & 0.201 & 0.196 & 0.194 & 0.195 \\ 
                   &                           & L & 0.662 & 0.628 & 0.627 & 0.628 & 0.634 & 0.211 & 0.232 & 0.232 & 0.236 & 0.228 \\ \midrule
\multirow{2}{*}{5} & \multirow{2}{*}{1.0}      & G & 0.402 & 0.391 & 0.387 & 0.388 & 0.389 & 0.206 & 0.188 & 0.180 & 0.173 & 0.169 \\
                   &                           & L & 0.402 & 0.309 & 0.306 & 0.306 & 0.341 & 0.206 & 0.200 & 0.201 & 0.196 & 0.196 \\ \midrule
\multirow{2}{*}{5} & \multirow{2}{*}{$\infty$} & G & 0.213 & 0.190 & 0.178 & 0.168 & 0.165 & 0.206 & 0.185 & 0.174 & 0.162 & 0.163 \\
                   &                           & L & 0.213 & 0.179 & 0.176 & 0.180 & 0.180 & 0.206 & 0.176 & 0.173 & 0.178 & 0.180 \\ \midrule

\multirow{2}{*}{10} & \multirow{2}{*}{0.1}      & G & 0.692 & 0.685 & 0.687 & 0.685 & 0.685 & 0.280 & 0.268 & 0.267 & 0.265 & 0.267 \\
                    &                           & L & 0.692 & 0.652 & 0.650 & 0.654 & 0.660 & 0.280 & 0.270 & 0.277 & 0.282 & 0.281 \\ \midrule
\multirow{2}{*}{10} & \multirow{2}{*}{1.0}      & G & 0.491 & 0.463 & 0.459 & 0.456 & 0.455 & 0.297 & 0.263 & 0.247 & 0.244 & 0.242 \\
                    &                           & L & 0.491 & 0.408 & 0.402 & 0.405 & 0.415 & 0.297 & 0.256 & 0.257 & 0.255 & 0.255 \\ \midrule
\multirow{2}{*}{10} & \multirow{2}{*}{$\infty$} & G & 0.315 & 0.240 & 0.229 & 0.223 & 0.222 & 0.303 & 0.237 & 0.226 & 0.222 & 0.221 \\
                    &                           & L & 0.315 & 0.238 & 0.237 & 0.239 & 0.240 & 0.303 & 0.237 & 0.234 & 0.237 & 0.239 \\ \midrule

\multirow{2}{*}{20} & \multirow{2}{*}{0.1}      & G & 0.692 & 0.680 & 0.676 & 0.674 & 0.677 & 0.377 & 0.300 & 0.294 & 0.294 & 0.298 \\
                    &                           & L & 0.692 & 0.641 & 0.633 & 0.636 & 0.644 & 0.377 & 0.304 & 0.326 & 0.321 & 0.323 \\ \midrule
\multirow{2}{*}{20} & \multirow{2}{*}{1.0}      & G & 0.481 & 0.455 & 0.450 & 0.448 & 0.448 & 0.374 & 0.311 & 0.300 & 0.295 & 0.292 \\
                    &                           & L & 0.481 & 0.448 & 0.437 & 0.431 & 0.437 & 0.374 & 0.354 & 0.342 & 0.303 & 0.304 \\ \midrule
\multirow{2}{*}{20} & \multirow{2}{*}{$\infty$} & G & 0.371 & 0.298 & 0.284 & 0.274 & 0.276 & 0.368 & 0.294 & 0.282 & 0.271 & 0.272 \\
                    &                           & L & 0.371 & 0.313 & 0.293 & 0.290 & 0.289 & 0.368 & 0.309 & 0.287 & 0.288 & 0.289 \\
                   
                   
                   
                   \bottomrule
\end{tabular}}
    \caption{Local and global EMD on CIFAR-10 for 12 combinations of $\rho$ = $\{$1, 5, 10, 20$\}$ and $\alpha$ = $\{$0.1, 1.0, $\infty \}$.}
    \label{tab:detail_emd}
\end{table}

\clearpage
\section{Pseudo Algorithm of LoGo}
\label{sec:pseudo_algorithm}


Algorithm\,\ref{alg:logo} is the overall pipeline of the FAL framework.
Specifically, we summarize the detailed pseudocode of our \algname{} algorithm.

\begin{algorithm}[h]
\small
\caption{FAL framework with \algname{} algorithm}
\label{alg:logo}
\textbf{Input}: initialized parameter $\Theta$; unlabeled data $U^{\scaleto{1}{4pt}}$; sampling strategy $\mathcal{A}$; labeling budget $B$; clients number $K$; AL round $R$; \\
\textbf{Output}: trained parameter $\Theta^{\scaleto{R*}{4pt}}$ \\
\\
\textbf{\# Alternating AL and FL Procedure}
\begin{algorithmic}[1]
\FOR{$k=1, \dots, K$}
\STATE Randomly sample $L_{\scaleto{k}{4pt}}^{\scaleto{1}{4pt}}= \{ x_{\scaleto{1}{4pt}}, \dots, x_{\scaleto{B}{4pt}} \}$ from $U_{\scaleto{k}{4pt}}^{\scaleto{1}{4pt}}$, and $U_k^{2} = U_k^1 \setminus L_k^1$
\STATE Get the labeled set $D_{\scaleto{k}{4pt}}^{\scaleto{1}{4pt}}$\, from the oracles
\ENDFOR
\STATE $\Theta^{\scaleto{1*}{4pt}}=$\,\texttt{FedAvg}\,($\Theta$, $D^{\scaleto{1}{4pt}}, K$) \\
\, \\
\FOR{$r=2, \dots, R$}
\FOR{$k=1, \dots, K$}
\STATE $D^{r}_k, \,U_k^{r+1}=$\,\,\texttt{LoGo}\,($\Theta^{(r-1)*}$, $D^{r-1}_{\scaleto{k}{4pt}}, U_k^r$)
\ENDFOR
\STATE $\Theta^{r*}=$\,\texttt{FedAvg}\,($\Theta$, $D^{r}, K$)
\ENDFOR
\end{algorithmic}
\, \\
\textbf{Function}\,\,\texttt{LoGo}:
\begin{algorithmic}[1] 
\STATE \textbf{\# Macro Step}
\STATE 
Train a local-only model $\Theta^{(r-1)}_{k*}$ from the scratch only using $D_k^{r-1}$
\STATE  For each $x\in U_k^r$, calculate the gradient embedding $g_{\hat{y}}^x$  by Eq.\,\eqref{eq:gradient}
\STATE Cluster $U_k^r$ into $B$ clusters($\mathcal{C}_1,...,\mathcal{C}_B$) by Eq.\,\eqref{eq:kmeans} \\
\, \\
\STATE \textbf{\# Micro Step}
\STATE 
$L_k^r= \emptyset $
\FOR{$\mathcal{C}_{\scaleto{i}{4pt}}=\mathcal{C}_{\scaleto{1}{4pt}}, \dots, \mathcal{C}_{\scaleto{B}{4pt}}$}
\STATE $L_k^r = L_k^r \cup \{ \mathcal{A}(\mathcal{C}_{\scaleto{i}{4pt}}, \Theta^{(r-1)*}, 1) \}$
\STATE $D_k^r = D_k^{r-1} \cup D_k^r$\, and\, $U_{\scaleto{k}{4pt}}^{\scaleto{r+1}{4pt}} = U_{\scaleto{k}{4pt}}^{\scaleto{r}{3pt}} \setminus L_k^r$
\ENDFOR 
\STATE \textbf{return}  $D_k^r$, \,$U_k^{r+1}$
\end{algorithmic}
\, \\
\textbf{Function}\,\,\texttt{FedAvg}:
\begin{algorithmic}[1]
\FOR{$\,FL\,\,round$\,}
\STATE Distribute $\Theta$ to the all client
\FOR{$k = 1, \dots, K$}
\STATE Train $\Theta_k$ on $D_k^r$ by minimizing $\mathbb{E}_{D_k^r}[\ell(x,y; \Theta_k)]$
\ENDFOR
\STATE $\Theta = (\sum_k \Theta_k) / K$
\ENDFOR
\STATE \textbf{return} $\Theta$
\end{algorithmic}

\end{algorithm}

\clearpage
\section{Experimental Settings}
\label{sec:exp_settings}

\subsection{Datasets}
We mainly experimented on two natural image datasets (CIFAR-10\footnote{https://www.cs.toronto.edu/~kriz/cifar.html}, SVHN\footnote{http://ufldl.stanford.edu/housenumbers}) and three medical image datasets\footnote{https://medmnist.com/} (PathMNIST, DermaMNIST, OrganAMNIST). 
Table \ref{tab:dataset_summray} provides a summary of the five datasets.
For the details of partitioning data to each client, please refer to Appendix\,\ref{sec:dataset_summary}.

\begin{table*}[t]
\caption{Details of the evaluated datasets.}
\label{tab:dataset-summary}
\vskip 0.15in
\centering
\begin{tabular}{@{}lllllll@{}}
\toprule
             & \begin{tabular}[c]{@{}l@{}}Number of\\ Series\end{tabular} & \begin{tabular}[c]{@{}l@{}}Time Steps\\ per Series\end{tabular} & Period  & Sampling & \begin{tabular}[c]{@{}l@{}}Number of\\ Features\end{tabular} & Data Split [\%] \\ \midrule
\solarSmall  & 8                                                          & 2,000                                                            & 01--03 2018    & 60m      & 8                                                            & 60/15/25   \\ 
\solarOneY   & 50                                                         & 8,760                                                            & 2019    & 60m      & 8                                                            & 60/15/25   \\ 
\solarThreeY & 50                                                         & 26,304                                                           & 2018--20 & 60m      & 8                                                            & 34/33/33   \\ \midrule
\airTen      & 12                                                         & 35,064                                                           & 2013--17 & 60m      & 11                                                           & 34/33/33   \\ 
\airTwenty   & 12                                                         & 35,064                                                           & 2013--17 & 60m      & 11                                                           & 34/33/33   \\ \midrule
\sapflux     & 24                                                         & \begin{tabular}[c]{@{}l@{}}15,000--\\ 20,000\end{tabular}          & 2008--16 & varying  & 10                                                           & 34/33/33   \\ \midrule
\hydro       & 531                                                        & 9,862                                                          & 1981--2008 & 24h      & 41                                                           & 34/33/33   \\ \bottomrule
\end{tabular}
\end{table*}


\subsection{Implementation Details}
For the FL training pipeline, we set the number of FL rounds to 100 and local update epochs to 5.
We used a SGD optimizer with the initial learning rate of 0.01 and the momentum of 0.9.
The learning rate was decayed by 0.1 at half and three-quarters of federated learning rounds to ensure convergence, and we used a random horizontal flipping as data augmentation.
For training local-only models, we trained the model using the aforementioned settings for 50 epochs. However, the training was terminated if the training accuracy reached 99\%.
It should be noted that we averaged the classification accuracy of the last 5 epochs in each round and repeated all experiments with four different seeds.
All algorithms were implemented using PyTorch 1.11.0 and executed using NVIDIA RTX 3080 GPUs.




\subsection{Experimental Categories}
A total of six categories were considered in the evaluation:
\begin{enumerate}
\item {`Query selector'} of whether to use a local-only or global model with the six compared strategies. 
\item  {`Heterogeneity level'} of varying degree of class imbalance. We adopt a Latent Dirichlet Allocation (LDA) \cite{moon} strategy. 
For example, the smaller $\alpha$, the more heterogeneous the data distribution. 
\item  {`Imbalance ratio'} of used datasets. 
We classified five datasets for evaluation based on the imbalance ratio $\rho$.
CIFAR-10 and PathMNIST belong to a low imbalance ratio ($\rho<2$), and SVHN, DermaMNIST, and OrganAMNIST belong to a high imbalance ratio ($\rho\geq 2$).
\item  {`Model architecture.'} We employed four layers of convolution neural network for a base architecture and also experimented with ResNet-18 \cite{resnet} and MobileNet \cite{mobilenet}. 
\item  `{Budget size}' for labeling. We tested small (1\%), medium (5\%), and large (20\%) budget sizes for each round. 
\item  {`Model initialization'} of either learning from scratch (random) or from the checkpoint of the previous AL round (continue). \looseness=-1
\end{enumerate}

\newpage
\subsection{Combination of experimental settings}
We compared our algorithms and baselines in 38 comprehensive experimental settings, which are the combinations of the aforementioned six categories.
All the experimental combinations we performed are summarized in Table \ref{tab:setting_summary}.

\begin{table*}[h!]
\centering
\small
\renewcommand*{\arraystretch}{1}
\addtolength{\tabcolsep}{1pt}
\resizebox{0.75\linewidth}{!}{
\begin{tabular}{cccccc}
\toprule
Query Selelctor   & Dir($\alpha$) & Data Type     & Model Arch.   & Budget Size & Model Init. \\ 
\midrule
Global          & 0.1 & CIFAR-10     & 4CNN   & 5\%         & Random     \\ 
Global          & 0.1 & SVHN        & 4CNN   & 5\%         & Random     \\ 
Global          & 0.1 & PathMNIST   & 4CNN   & 5\%         & Random     \\ 
Global        & 0.1  & OrganAMNIST & 4CNN & 5\%         & Random     \\ 
Global        & 0.1  & DermaMNIST & 4CNN   & 5\%         & Random     \\ 
Global        & 1   & CIFAR-10      & 4CNN   & 5\%         & Random     \\ 
Global        & 1   & SVHN         & 4CNN    & 5\%         & Random     \\ 
Global        & $\infty$   & CIFAR-10     & 4CNN   & 5\%         & Random     \\ 
Global        & $\infty$   & SVHN        & 4CNN   & 5\%         & Random     \\ 
Global        & 0.1  & CIFAR-10      & 4CNN  & 5\%         & Continue   \\ 
Global        & 0.1  & SVHN         & 4CNN   & 5\%         & Continue   \\ 
Global        & 0.1  & CIFAR-10      & ResNet-18 & 5\%         & Random     \\ 
Global      & 0.1    & SVHN       & ResNet-18  & 5\%         & Random     \\ 
Global        & 0.1  & CIFAR-10      & MobileNet  & 5\%         & Random     \\ 
Global        & 0.1  & SVHN         & MobileNet  & 5\%         & Random     \\ 
Global        & 0.1  & CIFAR-10      & 4CNN   & 1\%         & Random     \\ 
Global        & 0.1  & SVHN         & 4CNN   & 1\%         & Random     \\ 
Global        & 0.1  & CIFAR-10      & 4CNN   & 20\%        & Random     \\ 
Global        & 0.1  & SVHN         & 4CNN   & 20\%        & Random     \\ 
Local-only    & 0.1  & CIFAR-10      & 4CNN  & 5\%         & Random     \\ 
Local-only    & 0.1  & SVHN         & 4CNN   & 5\%         & Random     \\ 
Local-only    & 0.1  & PathMNIST    & 4CNN   & 5\%         & Random     \\ 
Local-only    & 0.1  & OrganAMNIST  & 4CNN   & 5\%         & Random     \\ 
Local-only   & 0.1   & DermaMNIST   & 4CNN & 5\%         & Random     \\ 
Local-only      & 1  & CIFAR-10    & 4CNN    & 5\%         & Random     \\ 
Local-only     & 1   & SVHN       & 4CNN   & 5\%         & Random     \\ 
Local-only     & $\infty$  & CIFAR-10     & 4CNN   & 5\%         & Random     \\ 
Local-only     & $\infty$  & SVHN       & 4CNN  & 5\%         & Random     \\ 
Local-only      & 0.1 & CIFAR-10    & 4CNN   & 5\%         & Continue   \\ 
Local-only     & 0.1  & SVHN        & 4CNN  & 5\%         & Continue   \\ 
Local-only      & 0.1 & CIFAR-10    & ResNet-18    & 5\%         & Random     \\ 
Local-only     & 0.1  & SVHN        & ResNet-18 & 5\%         & Random     \\ 
Local-only     & 0.1  & CIFAR-10     & MobileNet  & 5\%         & Random     \\ 
Local-only     & 0.1  & SVHN         & MobileNet  & 5\%         & Random     \\ 
Local-only     & 0.1  & CIFAR-10      & 4CNN  & 1\%         & Random     \\ 
Local-only     & 0.1  & SVHN         & 4CNN   & 1\%         & Random     \\ 
Local-only     & 0.1  & CIFAR-10      & 4CNN   & 20\%        & Random     \\ 
Local-only     & 0.1  & SVHN        & 4CNN   & 20\%        & Random     \\ 
\bottomrule
\end{tabular}}
\caption{Summary of the entire experimental combinations.}
\label{tab:setting_summary}
\end{table*}



\clearpage
\section{Computational Cost of Query Selection}
\label{sec:computational_cost}
In Table\,\ref{tab:time_cost}, we measured the wallclock time for various combinations of the algorithm, query selector, and labeling ratio.
We confirmed that as the percentage of labeled data increases, the time required to measure the importance score with the global model decreases due to the reduced amount of unlabeled data.
Conversly, the local-only model takes more time as it requires training on a larger number of labeled samples.
Our LoGo algorithm shows a comparable computational cost to the baselines that use the local-only model\,(L) for query selection.
Note that we used a simple Entropy sampling within LoGo algorithm to measure the uncertainty, and the only possible bottleneck is \textit{k}-means clustering in the Macro step.


\begin{table}[h!]
\centering
\addtolength{\tabcolsep}{-1pt}
\renewcommand*{\arraystretch}{1.05}
\resizebox{0.7\linewidth}{!}{
\begin{tabular}{l|cc|cc|cc|cc|cc|c}
\toprule
                   & \multicolumn{2}{c|}{\!\!\!Entropy\!\!\!} & \multicolumn{2}{c|}{Coreset} & \multicolumn{2}{c|}{BADGE} & \multicolumn{2}{c|}{GCNAL} & \multicolumn{2}{c|}{ALFA-Mix} & LoGo \\
 \cmidrule(l{2pt}r{2pt}){2-3} \cmidrule(l{2pt}r{2pt}){4-5} \cmidrule(l{2pt}r{2pt}){6-7} \cmidrule(l{2pt}r{2pt}){8-9} \cmidrule(l{2pt}r{2pt}){10-11} \cmidrule(l{2pt}r{2pt}){12-12}
\multirow{-2.5}{*}{Query ratio} & G & L & G & L & G & L & G & L & G & L & G,\,L \\ \midrule
5\%\,$\rightarrow$\,10\% & 5.99 \!  & \! 8.85 & 7.32 & 10.24 & 14.43 \!  & \! 17.36 & 8.20 & 11.13 & 13.88 \! & 20.87 & 17.10 \\ \hline
40\%\,$\rightarrow$\,45\% & 4.17 \!  & \! 33.59 & 7.02 & 33.99 & 10.01 \!  & \! 39.11 & 8.11 & 35.46 & 11.94 \! & 41.99 & 37.42 \\ \hline
75\%\,$\rightarrow$\,80\% & 3.95 \!  & \! 59.57 & 6.72 & 58.98 & 3.95 \!  & \! 62.62 & 7.71 & 60.26 & 10.46 \! & 65.16 & 56.81 \\
\bottomrule
\end{tabular}
}
\caption{Computational cost on CIFAR-10 with 4 layers of CNN. We averaged the query selection time\,(sec.) of all 10 clients, measured on a RTX 3090 GPU.}
\label{tab:time_cost}
\end{table}

\section{LoGo with Various FL Methods}
\label{sec:various_fl_algo}
We have further experimented with two federated learning algorithms, FedProx\cite{fedprox} and SCAFFOLD\cite{scaffold}, in conjunction with AL strategies.
Specifically, we compared our LoGo with baselines that demonstrated Top-1 or Top-2 performance more than once in Table\,\ref{tab:acc_comparision}. The experimental configurations are same to those used in Table\,\ref{tab:acc_comparision}.
As summarized in Table\,\ref{tab:compare_fedprox_scaffold}, LoGo consistently outperforms the baselines for both federated learning algorithms.
This observation suggests that LoGo is an orthogonal selection algorithm that can be integrated with any federated learning algorithm, having potential to improve the performance in various applications.

\begin{table}[h]
    \small
    \centering
    \renewcommand*{\arraystretch}{0.97}
    \resizebox{0.7\linewidth}{!}{
    \begin{tabular}{l|l|c|ccc|ccc}
        \toprule
         & &  & \multicolumn{3}{c|}{CIFAR-10} & \multicolumn{3}{c}{SVHN} \\
         \cmidrule(l{2pt}r{2pt}){4-6} \cmidrule(l{2pt}r{2pt}){7-9}
        \multirow{-2.5}{*}{FL\,algo.} & \multirow{-2.5}{*}{Method} & \multirow{-2.5}{*}{Model} & 20\% & 40\% & 60\% & 20\% & 30\% & 40\% \\
        \midrule
         \multirow{7}{*}{FedProx} & & G & 62.89 & 67.52 & 70.38 & 82.22 & 84.34 & 85.42  \\
         & \multirow{-2}{*}{Entropy} & L  & \underline{65.72} & \underline{70.57} & \underline{72.42} &82.08 & 83.73 & 85.30  \\
         \cline{2-9}
         & & G & 64.16 & 68.62 & 70.82 & \underline{83.09} & \textbf{84.65} & 85.84  \\
         & \multirow{-2}{*}{BADGE} & L  & 65.54 & 70.56 & 72.30 &81.99 & 84.17 & 85.17  \\
         \cline{2-9}
         & & G & 63.77 & 68.34 & 70.78 & 82.63 & 84.48 & \underline{85.94}  \\
         & \multirow{-2}{*}{ALFA-Mix} & L  & 63.44 & 67.83 & 70.31 &80.71 & 82.81 & 84.22  \\
         \cline{2-9}
          & \textbf{LoGo} & G, L & \textbf{65.79} & \textbf{70.61} & \textbf{72.61} & \textbf{83.12} & \underline{84.61} & \textbf{86.09}  \\
         \midrule
         \multirow{7}{*}{SCAFFOLD} & & G & 65.58 & 70.37 & 72.52 & 82.75 & 85.69 & 86.48  \\
         & \multirow{-2}{*}{Entropy} & L  & 67.96 & \underline{72.67} & \underline{74.06} & 83.24 & 84.30 & 85.82  \\
         \cline{2-9}
         & & G & 66.33 & 70.68 & 72.79 & 83.80 & 84.72 & \textbf{86.93}  \\
         & \multirow{-2}{*}{BADGE} & L  & \underline{68.27} & 72.52 & 73.79 & 83.40 & 84.61 & 86.16  \\
         \cline{2-9}
         & & G & 66.11 & 70.50 & 72.55 & \underline{84.11} & \textbf{85.72} & 86.14  \\
         & \multirow{-2}{*}{ALFA-Mix} & L  & 66.11 & 70.00 & 71.91 & 82.15 & 82.89 & 84.74  \\
         \cline{2-9}
          & \textbf{LoGo} & G, L & \textbf{68.33} & \textbf{72.77} & \textbf{74.48} & \textbf{84.29} & \underline{85.70} & \underline{86.73}  \\
         \bottomrule
    \end{tabular}}
    \caption{Classification accuracy on two benchmarks with FedProx\,($\mu$\,=\,0.01) and SCAFFOLD. We compared to three overwhelming baselines and averaged three random seeds. \textbf{Bold} and \underline{underline} mean Top-1 and Top-2, respectively.}
    \label{tab:compare_fedprox_scaffold}
    \vspace{-0.37cm}
\end{table}



\clearpage
\section{Detailed Experimental Results}
\label{sec:exp_detail_results}

In this Section, \ref{sec:detail_comp_matrix} summarizes all the comparison matrices results based on six categories: query selector, heterogeneity level, imbalance ratio, model architecture, budget size, and model initialization in Figure\,\ref{fig:bar_graph}. Figure\,\ref{fig:comp_selector}--\ref{fig:comp_model_init} are breakdowns of the matrix in Figure\,\ref{fig:comparision_matrix} into six categories.
\ref{sec:detail_comp_performance} provides comprehensive line plots for 38 experimental settings.
It can be seen that \algname{} overwhelms the baselines in most cases at both each category and detailed experimental setting level.


\subsection{Detailed Penalty Comparision Matrix}
\label{sec:detail_comp_matrix}

A maximum value of each matrix corresponds to Table\,\ref{tab:setting_summary}, and the bar plots in Figure\,\ref{fig:bar_graph} are calculated from these matrices.

\begin{figure*}[htb]
\centering
    \begin{subfigure}[b]{0.3\linewidth}
    \raggedleft
    \includegraphics[width=\linewidth]{figure/experiments/comparison/query_selector/global_comp.pdf}
    \caption{Global}
    \end{subfigure}
    \hspace{10pt}
    \begin{subfigure}[b]{0.3\linewidth}
    \raggedright
    \includegraphics[width=\linewidth]{figure/experiments/comparison/query_selector/local.pdf}
    \caption{Local-only}
    \end{subfigure}
    \caption{Pairwise penalty matrix for a query selector category. The maximum value of both matrices is 19.}
    \label{fig:comp_selector}
\end{figure*}

\vspace{-15pt}
\begin{figure*}[htb]
    \centering
    \begin{subfigure}[b]{0.3\linewidth}
    \includegraphics[width=\linewidth]{figure/experiments/comparison/dir/dir0_1.pdf}
    \caption{$\alpha = 0.1$}
    \end{subfigure}
    \hfill
    \begin{subfigure}[b]{0.3\linewidth}
    \includegraphics[width=\linewidth]{figure/experiments/comparison/dir/dir1_0.pdf}
    \caption{$\alpha = 1.0$}
    \end{subfigure}
    \hfill
    \begin{subfigure}[b]{0.3\linewidth}
    \includegraphics[width=\linewidth]{figure/experiments/comparison/dir/dir10_0.pdf}
    \caption{$\alpha = \infty$}
    \end{subfigure}
    \caption{Pairwise penalty matrix for a heterogeneity level category. The maximum value of three matrices is 30, 4, and 4.}
    \label{fig:comp_hetero}
\end{figure*}


\vspace{-15pt}
\begin{figure*}[htb]
    \centering
    \begin{subfigure}[b]{0.3\linewidth}
    \raggedleft
    \includegraphics[width=\linewidth]{figure/experiments/comparison/rho/rho_under2.pdf}
    \caption{$\rho <$ 2}
    \end{subfigure}
    \hspace{5pt}
    \begin{subfigure}[b]{0.3\linewidth}
    \raggedright
    \includegraphics[width=\linewidth]{figure/experiments/comparison/rho/rho_over2.pdf}
    \caption{$\rho \ge$ 2}
    \end{subfigure}
    \caption{Pairwise penalty matrix for imbalance ratio category. The maximum value of two matrices is 18 and 20, respectively.}
    \label{fig:comp_data_type}
\end{figure*}


\vspace{-15pt}
\begin{figure*}[!t]
    \centering
    \begin{subfigure}[b]{0.3\linewidth}
    \includegraphics[width=\linewidth]{figure/experiments/comparison/architecture/4cnn.pdf}
    \caption{Four Convolutional Neural Network}
    \end{subfigure}
    \hfill
    \begin{subfigure}[b]{0.3\linewidth}
    \includegraphics[width=\linewidth]{figure/experiments/comparison/architecture/resnet.pdf}
    \caption{ResNet-18}
    \end{subfigure}
    \hfill
    \begin{subfigure}[b]{0.3\linewidth}
    \includegraphics[width=\linewidth]{figure/experiments/comparison/architecture/mobilenet.pdf}
    \caption{MobileNet}
    \end{subfigure}
    \caption{Pairwise penalty matrix for a model architecture category. The maximum value of three matrices is 30, 4, and 4.}
    \label{fig:comp_model_arch}
\end{figure*}

\vspace{-15pt}
\begin{figure*}[!t]
    \centering
    \begin{subfigure}[b]{0.3\linewidth}
    \includegraphics[width=\linewidth]{figure/experiments/comparison/budget_size/budget_1.pdf}
    \caption{Budget 1\%}
    \end{subfigure}
    \hfill
    \begin{subfigure}[b]{0.3\linewidth}
    \includegraphics[width=\linewidth]{figure/experiments/comparison/budget_size/budget_5.pdf}
    \caption{Budget 5\%}
    \end{subfigure}
    \hfill
    \begin{subfigure}[b]{0.3\linewidth}
    \includegraphics[width=\linewidth]{figure/experiments/comparison/budget_size/budget_20.pdf}
    \caption{Budget 20\%}
    \end{subfigure}
    \caption{Pairwise penalty matrix for a budget size category. The maximum value of three matrices is 4, 30, and 4, respectively.}
    \label{fig:comp_budget_size}
\end{figure*}

% \vspace{-60pt}
\vspace{-15pt}
\begin{figure*}[!t]
    \centering
    \begin{subfigure}[b]{0.3\linewidth}
    \raggedleft
    \includegraphics[width=\linewidth]{figure/experiments/comparison/model_init/random.pdf}
    \caption{Random initialization}
    \end{subfigure}
    \hspace{10pt}
    \begin{subfigure}[b]{0.3\linewidth}
    \raggedright
    \includegraphics[width=\linewidth]{figure/experiments/comparison/model_init/continue.pdf}
    \caption{Continue initialization}
    \end{subfigure}
    \caption{Pairwise penalty matrix for a model initialization category. The maximum value of two matrices is 34 and 4.}
    \label{fig:comp_model_init}
\end{figure*}



\clearpage

\subsection{Detailed Performance Comparision}
\label{sec:detail_comp_performance}

For the line plots, we note that `Random' and `Ours' are independent of the query selector type.

\begin{figure*}[h!]
    \centering
    \includegraphics[width=0.8\linewidth]{figure/experiments/appendix/cifar10_total.pdf}
    \caption{Test accuracy on CIFAR-10, four layers of CNN, $\alpha=0.1$,  medium budget size\,(5\%), and random initialization. }
    \label{fig:app_cifar10}
\end{figure*}

\vspace{-15pt}
\begin{figure*}[h!]
    \centering
    \includegraphics[width=0.8\linewidth]{figure/experiments/appendix/svhn_total.pdf}
    \caption{Test accuracy on SVHN, four layers of CNN, $\alpha=0.1$,  medium budget size\,(5\%), and random initialization.}
    \label{fig:app_svhn}
\end{figure*}

\vspace{-15pt}
\begin{figure*}[!h]
    \centering
    \includegraphics[width=0.8\linewidth]{figure/experiments/appendix/path_total.pdf}
    \caption{Test accuracy on PathMNIST, four layers of CNN, $\alpha=0.1$,  medium budget size\,(5\%), and random initialization.}
    \label{fig:app_pathmnist}
\end{figure*}

\vspace{-15pt}
\begin{figure*}[!h]
    \centering
    \includegraphics[width=0.8\linewidth]{figure/experiments/appendix/derma_total.pdf}
    \caption{Test accuracy on DermaMNIST, four layers of CNN, $\alpha=0.1$,  medium budget size\,(5\%), and random initialization.}
    \label{fig:app_dermamnist}
\end{figure*}


\vspace{-15pt}
\begin{figure*}[!h]
    \centering
    \includegraphics[width=0.8\linewidth]{figure/experiments/appendix/organ_total.pdf}
    \caption{Test accuracy on OrganAMNIST, four layers of CNN, $\alpha=0.1$,  medium budget size\,(5\%), and random initialization.}
    \label{fig:app_organmnist}
\end{figure*}


\vspace{-15pt}
\begin{figure*}[!h]
    \centering
    \includegraphics[width=0.8\linewidth]{figure/experiments/appendix/cifar10_total_dir1.pdf}
    \caption{Test accuracy on CIFAR-10, four layers of CNN, \textbf{$\alpha=1.0$},  medium budget size\,(5\%), and random initialization.}
    \label{fig:app_cifar_dir1}
\end{figure*}

\vspace{-15pt}
\begin{figure*}[!h]
    \centering
    \includegraphics[width=0.8\linewidth]{figure/experiments/appendix/svhn_total_dir1.pdf}
    \caption{Test accuracy on SVHN, four layers of CNN, \textbf{$\alpha=1.0$},  medium budget size\,(5\%), and random initialization.}
    \label{fig:app_svhn_dir1}
\end{figure*}

\vspace{-15pt}
\begin{figure*}[!h]
    \centering
    \includegraphics[width=0.8\linewidth]{figure/experiments/appendix/cifar10_total_dir10.pdf}
    \caption{Test accuracy on CIFAR-10, four layers of CNN, \textbf{$\alpha=\infty$},  medium budget size\,(5\%), and random initialization.}
    \label{fig:app_cifar_dir10}
\end{figure*}



\vspace{-15pt}
\begin{figure*}[!h]
    \centering
    \includegraphics[width=0.8\linewidth]{figure/experiments/appendix/svhn_total_dir10.pdf}
    \caption{Test accuracy on SVHN, four layers of CNN, \textbf{$\alpha=\infty$},  medium budget size\,(5\%), and random initialization.}
    \label{fig:app_svhn_dir10}
\end{figure*}



\vspace{-15pt}
\begin{figure*}[!h]
    \centering
    \includegraphics[width=0.8\linewidth]{figure/experiments/appendix/cifar10_total_mobilenet.pdf}
    \caption{Test accuracy on CIFAR-10, MobileNet, $\alpha=0.1$,  medium budget size\,(5\%), and random initialization.}
    \label{fig:app_cifar10_mobile}
\end{figure*}



\vspace{-15pt}
\begin{figure*}[!h]
    \centering
    \includegraphics[width=0.8\linewidth]{figure/experiments/appendix/svhn_total_mobilenet.pdf}
    \caption{Test accuracy on SVHN, MobileNet, $\alpha=0.1$,  medium budget size\,(5\%), and random initialization.}
    \label{fig:app_svhn_mobile}
\end{figure*}



\vspace{-15pt}
\begin{figure*}[!h]
    \centering
    \includegraphics[width=0.8\linewidth]{figure/experiments/appendix/cifar10_resnet_total.pdf}
    \caption{Test accuracy on CIFAR-10, ResNet-18, $\alpha=0.1$,  medium budget size\,(5\%), and random initialization.}
    \label{fig:app_cifar10_resnet}
\end{figure*}



\vspace{-15pt}
\begin{figure*}[!h]
    \centering
    \includegraphics[width=0.8\linewidth]{figure/experiments/appendix/svhn_resnet_total.pdf}
    \caption{Test accuracy on SVHN, ResNet-18, $\alpha=0.1$,  medium budget size\,(5\%), and random initialization.}
    \label{fig:app_svhn_resnet}
\end{figure*}


\vspace{-15pt}
\begin{figure*}[!h]
    \centering
    \includegraphics[width=0.8\linewidth]{figure/experiments/appendix/cifar10_budget1_total.pdf}
    \caption{Test accuracy on CIFAR-10, four layers of CNN, $\alpha=0.1$,  small budget size\,(1\%), and random initialization.}
    \label{fig:app_cifar10_budget1}
\end{figure*}


\vspace{-15pt}
\begin{figure*}[!h]
    \centering
    \includegraphics[width=0.8\linewidth]{figure/experiments/appendix/svhn_budget1_total.pdf}
    \caption{Test accuracy on SVHN, four layers of CNN, $\alpha=0.1$,  small budget size\,(1\%), and random initialization.}
    \label{fig:app_svhn_budget1}
\end{figure*}


\vspace{-15pt}
\begin{figure*}[!h]
    \centering
    \includegraphics[width=0.8\linewidth]{figure/experiments/appendix/cifar10_total_budget20.pdf}
    \caption{Test accuracy on CIFAR-10, four layers of CNN, $\alpha=0.1$,  large budget size\,(20\%), and random initialization.}
    \label{fig:app_cifar10_budget20}
\end{figure*}


\vspace{-15pt}
\begin{figure*}[!h]
    \centering
    \includegraphics[width=0.8\linewidth]{figure/experiments/appendix/svhn_total_budget20.pdf}
    \caption{Test accuracy on SVHN, four layers of CNN, $\alpha=0.1$,  large budget size\,(20\%), and random initialization.}
    \label{fig:app_svhn_budget20}
\end{figure*}

\newpage
\vspace{-15pt}
\begin{figure*}[!h]
    \centering
    \includegraphics[width=0.8\linewidth]{figure/experiments/appendix/cifar10_total_continue.pdf}
    \caption{Test accuracy on CIFAR-10, four layers of CNN, $\alpha=0.1$,  medium budget size\,(5\%), and continue initialization.}
    \label{fig:app_cifar10_cont}
\end{figure*}



\vspace{-15pt}
\begin{figure*}[t]
    \centering
    \includegraphics[width=0.8\linewidth]{figure/experiments/appendix/svhn_total_continue.pdf}
    \caption{Test accuracy on SVHN, four layers of CNN, $\alpha=0.1$,  medium budget size\,(5\%), and continue initialization.}
    \label{fig:app_svhn_cont}
\end{figure*}



\end{document}
