\section{Related Work}
\label{sec:related_work}

\subsection{Datasets for Cell detection and Tissue segmentation Tasks}
\label{sec:ct_datasets}
In recent years, numerous datasets have been released for tackling cell detection. Some of those works only target a single organ \cite{TNBC, GRAHAM2019101563}, while others consider multiple ones \cite{CPM17, MoNuSeg, MoNuSAC, PanNuke, NuCLS}. The availability of these datasets enables the CPATH community to push forward the development and improvement of cell detection models \cite{GRAHAM2019101563, ZHAO2020101786, liu2019nuclei, lal2021nucleisegnet, li2019dual, qu2020weakly}. In addition, tissue segmentation datasets have also been proposed for prostate \cite{NIR2018167, 8853320, PANDA}, colorectal \cite{tissue_ds_1}, brain \cite{miccai2014}, and multiple organs \cite{ADP}. Some examples of tissue segmentation works can be found in \cite{tissue_ds_2, chan2019histosegnet, zhu2021multi, chen2016dcan, qian2022transformer, li2016gland}. 
The dataset in \cite{digestpath} is composed of a cell detection subset and a tissue segmentation subset. However, the subsets are annotated independently and from different patient groups, and, therefore, there are no overlaps between the cell and tissue data.
Because of the lack of overlapping data in the aforementioned datasets, it is difficult to build an end-to-end framework to learn cell-tissue relationships by jointly training on the cell and tissue tasks.


The recently released TIGER dataset~\cite{tiger} contains both cell and tissue annotations to study tumor-infiltrating lymphocytes \cite{SALGADO2015259} in H\&E breast cancer WSIs.
All the cell-annotated areas exist inside the tissue-annotated area, however, this work does not propose nor initiate any effort toward the integration of both cell and tissue tasks. 

\begin{table}[t]
\small
\centering
\setlength{\tabcolsep}{0.2em}
\begin{tabular}{lrrcl}
\toprule

\textbf{Dataset} & \multicolumn{1}{c}{\textbf{Tissue Area}} & \multicolumn{1}{c}{\textbf{\# Cell}} & \hspace{0.5em} & \multicolumn{1}{c}{\textbf{Organs}}\\ 
\midrule

\ours & \quad4.267$cm^2$ & \quad114.7K & \hspace{0.5em} &multiple\\

TIGER & \quad2.536$cm^2$ & \quad50.8K & \hspace{0.5em} & breast\\


\midrule

\end{tabular}
\vskip -2mm
\caption{\textbf{Dataset comparison} in terms of physical tissue annotated area and total cell counting per dataset.}
\label{tab:label_stats_new}
\vspace{-3mm}
\end{table}
To further promote the development of methods that leverage the cell-tissue relationship for the task of cell detection, we propose \ours, which is designed to capture the hierarchical relationship between cells and tissues, especially in the tumor environment. 
OCELOT contains roughly two times more cell and tissue annotations than TIGER (see \autoref{tab:label_stats_new}).
Additionally, the data was collected from multiple organs to enable the investigation of the generalizability of cell-tissue relationships over various cancer types.
In the end, we utilized both \ours and TIGER to reveal the power of cell-tissue relationship for cell detection in \autoref{sec:experiments}.


\subsection{Leveraging Large Field of View}
\label{sec:fov}
Some studies \cite{KAMNITSAS201761, awmfovnet, VANRIJTHOVEN2021101890, SCHMITZ2021101996, HO2021101866, 10.1007/978-3-030-59722-1_37} extract a large FoV region as an additional input to improve detection/segmentation performance on smaller FoV regions.
\cite{KAMNITSAS201761} proposes a dual pathway 3D CNN for brain lesion segmentation, where each pathway receives both small and large center-shared FoV regions as input. 
Similar studies are also conducted in the CPATH domain for tissue segmentation \cite{awmfovnet, VANRIJTHOVEN2021101890, SCHMITZ2021101996, HO2021101866} and cell detection \cite{10.1007/978-3-030-59722-1_37}.
To fuse different FoV patches, \cite{awmfovnet} introduces a weighting mechanism and \cite{VANRIJTHOVEN2021101890} proposes a multi-scale merging block composed of convolution and concatenation. 
Nevertheless, no inter-task relationship is considered in the previously mentioned methods. In contrast, our models take advantage of the large contextual information while learning the cell-tissue relationship via multi-task objectives at different FoVs.

\subsection{Leveraging Cell-Tissue Relationships}
\label{sec:ct_methods}
To our understanding, there is no study that considers cell-tissue relationships for cell detection or tissue segmentation tasks.
On the other hand, a few efforts have attempted to link tissue and cell for image classification using graph-based methods \cite{cgcnet, wsptcgcn, hactnet}.
Such studies represent the tissue structure as a graph of detected cells, based on the medical knowledge that cells form tissues.
\cite{hactnet} explicitly considers the cell-tissue relationship for the task of breast cancer subtyping, by an interaction between tissue-level and cell-level graphs with a cell-to-tissue hierarchy. 
However, these methods treat the cell/tissue graph generation as a pre-processing step, by using the inference output of independently pre-trained cell detection and tissue segmentation models. In contrast, we directly target the improvement of cell detection by utilizing the cell-tissue relationship. 
