\section{\ours}
\label{sec:dataset}
\begin{figure}[t]
    \centering
    \hfill
        \includegraphics[width=\linewidth]{figures/data_sample.pdf}
    \hfill
    \vskip -7mm
    \caption{\textbf{A sample from the \ours dataset.} Each sample of the dataset consists of two input patches and the corresponding annotations. \textbf{Left} shows the large FoV patch $x_{l}$ with tissue segmentation annotation $y_{l}^{t}$, where green denotes the cancer area. \textbf{Right} shows the small FoV patch $x_{s}$ with cell point annotation $y_{s}^{c}$, where blue and yellow dots denote \textit{tumor} and \textit{background} cells, respectively. The red box indicates the size and location of the $x_{s}$ with respect to the $x_{l}$.}
    \label{fig:data_sample}
    \vspace{-0.3cm}
\end{figure}

In this section, we introduce \ours, a histopathology dataset specifically built to enable the development of methods that leverage cell and tissue relationships. Each sample of the \ours dataset $\mathcal{D}$ is composed of six components,

\begin{equation}
\mathcal{D} = \left\{\left(x_{s}, y_s^{c}, x_l, y_l^{t}, c_x, c_y\right)_i\right\}_{i=1}^{N}
\end{equation}

\noindent where $x_s, x_l$ are the small and large FoV patches extracted from the WSI, $y_s^{c}, y_l^{t}$ refer to the corresponding cell and tissue annotations, respectively, and $c_x, c_y$ are the relative coordinates of the center of $x_s$ within $x_l$. We drop the sample index $i$ for simplicity. \autoref{fig:data_sample} shows the visualization of a sample in \ours. More details about the dataset including data collection and statistics can be found in the following sub-sections. The dataset is publicly available at \href{https://lunit-io.github.io/research/publications/ocelot/}{https://lunit-io.github.io/research/publications/ocelot/}.

\subsection{Data Collection}
\label{ssec:data-collect}
We collect 306 TCGA~\cite{HUTTER2018283} WSIs from a total of 6 different organs: \textit{kidney}, \textit{head-and-neck}, \textit{prostate}, \textit{stomach}, \textit{endometrium}, and \textit{bladder}. 
From each of the WSIs, we select 1 to 3 large Regions of Interest (ROIs) for the tissue segmentation task. 
Finally, for the cell detection task, we randomly choose a smaller ROI that is fully contained within the larger tissue ROI. 
As a result, \ours includes 673 paired patches from 6 organs. The numbers of WSIs and pairs of patches per organ are detailed in \autoref{tab:dst_size}.


Some natural image datasets, such as ImageNet~\cite{deng2009imagenet} or Pascal VOC~\cite{everingham2010pascal}, include thousands of annotated images. However, annotating histopathology images is more challenging and expensive due to the scarcity of expert pathologists~\cite{c3det}. Furthermore, acquiring dense annotations for cell detection and tissue segmentation is especially time-demanding compared to higher-level tasks such as image classification. Nonetheless, in \autoref{tab:label_stats_new}, we observe that \ours is roughly double the size of the recent TIGER dataset with respect to the annotated tissue area and the number of annotated cells. 



\vspace{-4mm}
\paragraph{Patch configuration.} 
Cell detection tasks benefit from fine-grained spatial information to better capture detailed cell properties (e.g. border, shape, color, and opacity). In contrast, tissue segmentation requires a larger context to enable a better understanding of the overall structural information. Therefore, we define the FoV sizes of $x_s$ (cell detection) and $x_l$ (tissue segmentation) as 1024 $\times$ 1024 and 4096 $\times$ 4096 pixels, respectively, at a resolution of 0.2 Microns-per-Pixel (MPP). Finally, the large FoV patches and tissue annotations ($x_l$, $y_l^{t}$) are down-sampled by a factor of 4, resulting in a size of 1024 $\times$ 1024 pixels. 


\vspace{-4mm}
\paragraph{Annotation.} 
All cell-tissue pairs of patches are annotated by board-certified pathologists. Cells are labeled as points, with associated 2D coordinates and class labels. We denote the annotations in a given cell-level patch, $x_s$, as $y_s^{c}$, and consider two classes: Tumor Cell (\textit{TC}) and Background Cell (\textit{BC})\footnote{\textit{BC} includes any of the following cell categories: lymphocyte, macrophage, fibroblast, endothelial, or other remaining cell types.}.
\textit{TC} and \textit{BC} class ratios are 35.01\% and 64.99\%, respectively.
Regarding the tissue patches, $x_l$, pathologists annotate the pixel-wise segmentation maps $y_l^{t}$ with either Cancer Area (\textit{CA}) or Background (\textit{BG}) labels. A minority of pixels where the tissue class was uncertain were labeled as Unknown (\textit{UNK}). \textit{BG}, \textit{CA}, and \textit{UNK} class ratios are 55.77\%,  40.17\%, and 4.06\%, respectively. The amount of annotated cells and tissue pixels, per data split, can be found in the \supple. The detection of \textit{TC} and \textit{BC} has clinical relevance. For example, tumor purity \cite{azimi2017breast}, computed as the tumor/non-tumor cell ratio in a WSI, has a correlation with cancer prognosis \cite{mao2018low, zhang2017tumor, gong2020tumor}.

\vspace{-4mm}
\paragraph{Dataset splits.} The dataset is divided into three subsets: \textit{training}, \textit{validation}, and \textit{test}, following a $6$:$2$:$2$ ratio. To prevent information leaking among the data subsets, we randomly split the dataset per WSI, so that different patches from the same WSI are not included in multiple subsets. We maintain consistent cancer-type ratios in each subset.

\begin{table}
\small

\centering
\setlength{\tabcolsep}{0.2em}
{
\begin{tabular}{lrrrrrr}
\toprule
\multirow{2}{*}{\textbf{Organs}} & \multicolumn{3}{c}{\textbf{\# Slides}} & \multicolumn{3}{c}{\textbf{\# Patch Pairs}} \\ \cmidrule(lr){5-7} \cmidrule(lr){2-4} 
& \textbf{Train} & \textbf{Val} & \textbf{Test} &  \textbf{Train} & \textbf{Val} & \textbf{Test} \\ \midrule
Kidney & 48 & 15 & 18 & 125 & 41 & 41 \\ 
Head-neck & 13 & 5 & 6 & 27 & 9 & 10 \\ 
Prostate & 26 & 12 & 10 & 50 & 17 & 16 \\ 
Stomach & 15 & 6 & 5 & 36 & 12 & 12 \\ 
Endometrium & 38 & 13 & 13 & 86 & 29 & 25 \\ 
Bladder & 35 & 14 & 14 & 82 & 29 & 26 \\ \midrule
\textbf{Total} & 175 & 65 & 66 & 406 & 137 & 130 \\ \bottomrule
\end{tabular}
}
\vskip -2mm
\caption{\textbf{Dataset size per organ and data subset.}}
\label{tab:dst_size}
\vspace{-4mm}
\end{table}