\section{Case Studies}
\label{sec:caseStudies}

\ifdefined\ArxivVersion
We apply Saihu to common network topologies: Families of tandem, ring, and mesh networks, used for numerical evaluation in the Panco research~\cite{bouillard2022tradeoff}. Also, it is applied to an industrial-size network that is a test configuration provided by Airbus \cite{Grieu2004AnalyseE} and is used as a benchmark in several research papers \cite{tabatabaee2021deficit, plpdrr, charara2006methodsAFDX}. Please see~\ref{sec: numerical results}.

This helped us develop Saihu and ensure its applicability to a variety of cases. Note that the obtained results are not the main purpose of this paper, but are only provided for an interested reader in~\ref{sec: numerical results}. Specifically, the comparison of delay bounds obtained by different methods is not the main purpose of this paper. Rather, Saihu significantly reduces the effort to make such a comparison by providing a general interface to apply three frequently used worst-case delay analyses in a single shot. Hence, it solves the non-trivial problem of finding the best, smallest delay among the supported methods.

\fi

\ifdefined\SoftwareXVersion
{\color{blue}
We apply Saihu to common network typologies: Families of tandem, ring, and mesh networks, used for numerical evaluation in the Panco research~\cite{bouillard2022tradeoff}. Also, it is applied to an industrial-size network that is a test configuration provided by Airbus \cite{Grieu2004AnalyseE} and is used as a benchmark in several research papers \cite{tabatabaee2021deficit, plpdrr, charara2006methodsAFDX}. Please see Appendix A of the technical report~\cite{tsai2023saihucommoninterfaceworstcase}.

This helped us to develop Saihu and to ensure its applicability to a variety of cases. Note that the obtained results are not the main purpose of this paper, but are only provided for an interested reader in Appendix A of the technical report~\cite{tsai2023saihucommoninterfaceworstcase}. Specifically, the comparison of delay bounds obtained by different methods is not the main purpose of this paper. Rather, Saihu significantly reduces the effort to make such a comparison by providing a general interface to apply three frequently used worst-case delay analyses in a single shot. Hence, it solves the non-trivial problem of finding the best, smallest delay among the supported methods.

}
\fi