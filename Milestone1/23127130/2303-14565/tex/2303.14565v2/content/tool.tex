\section{Included Tools}
\label{sec: included tools}

Saihu currently includes 3 tools:  xTFA, DNC, and Panco. 
Fig.~\ref{fig: supported methods} summarizes supported methods for each tool.

\begin{center}
    \begin{tabular}{|c|c|c|c|}
        \hline
        Method\textbackslash Tool & DNC & xTFA & Panco \\
        \hline\hline
        TFA  & V & V & V \\
        SFA  & V &   & V \\
        PLP  &   &   & V \\
        ELP  &   &   & V \\
        PMOO & V &   &   \\
        LUDB & V &   &   \\
        \hline
    \end{tabular}
    \captionof{table}{Supported methods are marked with a ``V''
    }
    \label{fig: supported methods}
\end{center}

\begin{itemize}
    \item \textbf{xTFA}~\cite{thoma2022analyse} is developed in Python and supports a more advanced TFA. For its input, an XML file describes the network (cf. Sec.~\ref{sec: physical network xml}). 
    xTFA supports analyzing networks with cyclic dependency and multicast flows.

    \item \textbf{DiscoDNC}~\cite{bondorf2014discodnc} is developed in Java and partially uses linear programming with \textit{CPLEX}~\cite{cplex2009v12} for LUDB. It supports TFA, SFA, PMOO, and LUDB. A network is defined through its own Java classes. Saihu uses the information from an output port network to create a network in DNC syntax internally. Moreover, with DNC, one cannot manually set shaping with FIFO multiplexing but only with arbitrary multiplexing. Also, DNC does not support networks with cyclic dependency and does not support multicast flows (see Section~\ref{sec: system model}).

    \item \textbf{Panco}~\cite{bouillard2022tradeoff} is developed in Python and uses linear programming. So, it requires \textit{lpsolve}~\cite{lpsolve} to execute TFA, SFA, PLP, and ELP. A network is described with its own Python classes. Saihu internally creates the network in Panco syntax from the information of an output port network. All methods of Panco except for ELP support networks with cyclic dependencies. Panco  does not support multicast flows (see Section~\ref{sec: system model}).
\end{itemize}
