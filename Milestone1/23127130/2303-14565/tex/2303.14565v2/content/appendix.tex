\section{Numerical Results}
\label{sec: numerical results}

We apply Saihu to common network typologies. We provide the numerical results we obtained for these networks. This shows the applicability of Saihu to a variety of cases. Note that the obtained results are not the main purpose of this paper, but are provided for an interested reader.

\subsection{Interleave Tandem Network}

An interleave tandem network is illustrated in Figure~\ref{fig: interleave}. Consider a network of $n$ servers, indexed from~$0$ to~$n-1$.
An interleave tandem network has all its servers chained in a line. One flow~$f_0$ goes through all servers from~$s_0$ to~$s_{n-1}$.  The flow~$f_i$ is $s_{i-1} \rightarrow s_{i}$ for $i \in [1,n-1]$. Every flow has the same arrival curve and the same maximum packet length at the source, defined by function arguments \texttt{burst}, \texttt{arrival\_rate}, and \texttt{max\_packet\_length}. Every server has the same service curve and transmission capacity, defined by \texttt{latency}, \texttt{service\_rate}, and \texttt{capacity}.

\begin{figure}[htb]
    \centering
    \includegraphics[width=0.7\linewidth]{images/interleave.png}
    \caption{Interleave tandem network}
    \label{fig: interleave}
\end{figure}

We apply Saihu to an interleave tandem network of 8 servers with the parameters shown in Table~\ref{tab: interleave experiment parameters}; see Figure~\ref{fig: interleave8 report}. Due to the limitations of each method, the interleave tandem network is the only type of network in our examples that is compatible with all the methods. Such restriction does not relate to Saihu. Rather, it is the complexity of ELP. One can already see it in Figure~\ref{fig: interleave8 time}. ELP requires minutes to run while most methods return their analysis results within several milliseconds. The choice of applying ELP to networks with only less than $10$~nodes is also suggested in the original paper~\cite{bouillard2022tradeoff}.

\begin{table}[h]
    \centering
    \begin{tabular}{c|c|c|c|c|c}
        burst & arrival rate & max packet size & latency & service rate & capacity \\
        \hline
        32 bits & 64 bps & 256 bits & 10 ms & 1 kbps & 1 kbps
    \end{tabular}
    \caption{Parameters of the example interleave tandem network with 8 servers}
    \label{tab: interleave experiment parameters}
\end{table}

Even with such a small-scale network, the delay-bound estimations and the runtimes of individual methods can noticeably differ. Such an example shows the importance of Saihu, where the researchers can try out all the available methods to find the best bound relatively easily.

\begin{figure*}[thb]
\centering
\begin{subfigure}[b]{0.75\textwidth}
    \centering
    \includegraphics[width=\linewidth]{images/interleave8_delay.png}
    \caption{Flow end-to-end delay}
    \label{fig: interleave8 e2e delay}
\end{subfigure}
\hfill
\begin{subfigure}[b]{0.2\textwidth}
    \centering
    \includegraphics[width=\linewidth]{images/interleave8_time.png}
    \caption{Execution time}
    \label{fig: interleave8 time}
\end{subfigure}
\caption{Analysis report of interleave tandem network with 8 servers}
\label{fig: interleave8 report}
\end{figure*}

\subsection{Ring Network}

A ring network is illustrated in Figure~\ref{fig: ring}. There are $n$~flows and $n$~servers. The path of flow~$i$ is $s_i \rightarrow s_{i+1} \rightarrow \cdots \rightarrow s_{i+n-1\mod n}$ for $0 \leq i \leq n-1$. A ring network is completely symmetrical with identical flows and servers. An arrival curve of every flow is defined by \texttt{burst}, \texttt{arrival\_rate}, and \texttt{max\_packet\_length}. Also, a service curve of every server is defined by \texttt{latency}, \texttt{service\_rate}, and \texttt{capacity}.

\begin{figure}[htb]
    \centering
    \includegraphics[width=0.3\linewidth]{images/ring.png}
    \caption{Ring network}
    \label{fig: ring}
\end{figure}

We apply Saihu to a ring network of $10$~servers with the parameters shown in Table~\ref{tab: ring experiment parameters}; see Figure~\ref{fig: ring report}. As the ring network has cyclic dependencies, we can only apply the methods PLP and xTFA.

\begin{table}[h]
    \centering
    \begin{tabular}{c|c|c|c|c|c}
        burst & arrival rate & max packet size & latency & service rate & capacity \\
        \hline
        32 bits & 64 bps & 256 bits & 10 ms & 10 kbps & 10 kbps
    \end{tabular}
    \caption{Parameters of the example ring network with $10$ servers}
    \label{tab: ring experiment parameters}
\end{table}

\begin{figure*}[thb]
\centering
\begin{subfigure}[b]{0.47\textwidth}
    \centering
    \includegraphics[width=\linewidth]{images/ring10_delay.png}
    \caption{Flow end-to-end delay (only the first 5 flows)}
    \label{fig: ring e2e delay}
\end{subfigure}
\begin{subfigure}[b]{0.3\textwidth}
    \centering
    \includegraphics[width=\linewidth]{images/ring10_time.png}
    \caption{Execution time}
    \label{fig: ring time}
\end{subfigure}
\caption{Analysis report of a ring network with $10$ servers}
\label{fig: ring report}
\end{figure*}

\subsection{Mesh Network}

A mesh network is illustrated in Figure~\ref{fig: mesh}. All flows start from either~$s_0$ or~$s_1$. The flows go through all $2^{(n-1)/2}$ possible combinations of servers towards the right, e.g. $s_0 \rightarrow s_2 \rightarrow \cdots$ and $s_1 \rightarrow s_2 \rightarrow \cdots$ are both in the network. All servers have the same service curve and capacity except~$s_{n-1}$, that has the doubled service rate. All flows have identical service curves and maximum packet length.

\begin{figure}[htb]
    \centering
    \includegraphics[width=0.4\linewidth]{images/mesh.png}
    \caption{Mesh network. Only parts of the flows from~$s_0$ are shown. There is a flow for every possible path from~$s_0$ or~$s_1$ to~$s_{n-1}$.}
    \label{fig: mesh}
\end{figure}

We apply Saihu to a mesh network of $13$~servers with the parameters shown in Table~\ref{tab: mesh experiment parameters}. We do not apply ELP because the number of flows is too large to handle. One can see the report in Figure~\ref{fig: mesh report}. Note that the complexity of SFA, LUDB, and PLP can still become intractable in a small-size network.

\begin{table}[h]
    \centering
    \begin{tabular}{c|c|c|c|c|c}
        burst & arrival rate & max packet size & latency & service rate & capacity \\
        \hline
        32 bits & 64 bps & 256 bits & 10 ms & 20 Mbps & 200 Mbps
    \end{tabular}
    \caption{Parameters of the example mesh network with $13$ servers}
    \label{tab: mesh experiment parameters}
\end{table}

\begin{figure*}[thb]
\centering
\begin{subfigure}[b]{0.75\textwidth}
    \centering
    \includegraphics[width=\linewidth]{images/mesh13_delay.png}
    \caption{Flow end-to-end delay (only the first 5 flows)}
    \label{fig: mesh e2e delay}
\end{subfigure}
\begin{subfigure}[b]{0.22\textwidth}
    \centering
    \includegraphics[width=\linewidth]{images/mesh13_time.png}
    \caption{Execution time}
    \label{fig: mesh time}
\end{subfigure}
\caption{Analysis report of a mesh network with $13$ servers}
\label{fig: mesh report}
\end{figure*}

\subsection{Industrial-Sized Network}
\label{sec: fixed topology random}

Saihu is applied to an industrial-size network topology provided by \textit{Airbus} as a test configuration~\cite{tabatabaee2021deficit, charara2006methodsAFDX}; please see Figure~\ref{fig: industrial network}.

\begin{figure}[htb]
    \centering
    \includegraphics[width=0.5\linewidth]{images/industrial_net.png}
    \caption{Industrial network from Airbus test configuration~\cite{charara2006methodsAFDX}. Each connection represents a potential link and the boxes with $S$ combined with a number represent switches.}
    \label{fig: industrial network}
\end{figure}

We tested the network by randomly generating $120$ flows within this network. The parameters are selected uniformly randomly from the range shown in Table~\ref{tab: industrial experiment parameters}; see Figure~\ref{fig: industrial report}. Due to the limited space here, we present the first $5$~flows out of~$120$. We only apply the methods xTFA and PLP to this network because of the cyclic dependencies.

As the number of flows increases, the complexity of PLP grows significantly. We understand real-world networks often contain thousands of flows, but we are constrained by the limitation of our computational resources. Nevertheless, Saihu provides a convenient interface for researchers. Even when the analysis options are limited by the network characteristics or the computational resources, Saihu helps researchers apply a suitable analysis method to their network of interest with relative ease.

\begin{table}[htb]
    \centering
    \footnotesize
    \begin{tabular}{c|c|c|c|c|c|c}
        & burst & arrival rate & max packet size & latency & service rate & capacity \\
        \hline
        min & 10 bytes & 64 bps & 128 bytes & 2 $\mu s$ & 1 Mbps & 256 Mbps \\
        \hline
        max & 1024 bytes & 1 kbps & 128 bytes & 200 ms & 100 Mbps & 256 Mbps
    \end{tabular}
    \caption{Parameters of the example Airbus network with size~$120$. The parameters of each flow and node is selected from the range uniformly randomly.}
    \label{tab: industrial experiment parameters}
\end{table}

\begin{figure*}
\centering
\begin{subfigure}[b]{0.5\textwidth}
    \centering
    \includegraphics[width=\linewidth]{industrial_delay.png}
    \caption{Flow end-to-end delay (only first 5 flows)}
    \label{fig: industrial e2e delay}
\end{subfigure}
\hfill
\begin{subfigure}[b]{0.4\textwidth}
    \centering
    \includegraphics[width=0.8\linewidth]{industrial_time.png}
    \caption{Execution time}
    \label{fig: industrial time}
\end{subfigure}
\caption{Example report from the topology of Figure~\ref{fig: industrial network} with 120 randomly generated flows.}
\label{fig: industrial report}
\end{figure*}
