

\section{System Model}
\label{sec: system model}

\begin{figure}[thb]
    \centering
    \includegraphics[width=0.75\textwidth]{images/device.png}
    \caption{Device model}
    \label{fig: device}
\end{figure} 

Devices represent switches or routers that compose the network of interest; they consist of input ports, output ports, and switching fabric. Fig.~\ref{fig: device} shows one such device. Each packet enters a device via an input port and is stored in a packetizer. A packetizer releases a packet only when the entire packet is received. Then, the packet goes through a switching fabric, which transmits the packet to a specific output port based on the static route of the packet; the packet is either buffered in a FIFO (First-In-First-Out) queue and then is serialized on the output line at the transmission rate of the line or exits the network via a terminal port (i.e., a sink). 

A flow is a stream of packets generated from the same source, following the same path, and destined for the same sink.  We assume that flows are statically assigned to a path. A path consists of a source, a sequence of devices (with the corresponding input port and output port), and a sink (see Fig.~\ref{fig: physical network}). For each flow~$i$, we let $l^{\scriptsize\textrm{max}}_i$ and $l^{\scriptsize\textrm{min}}_i$ denote the maximum and minimum packet size, respectively. Every flow is constrained at the source by an arrival curve which we assume to be piece-wise linear and concave. Such an arrival curve, say $\alpha$, can be described by a collection of $m$~rates~$r_1, r_2, ..., r_m$ and bursts $b_1, b_2, ...,b_m$ such that $\alpha(t) = \min_{k=1:m}(r_kt + b_k)$; each function $t \mapsto r_kt + b_k$ is called a token-bucket function with rate~$r_k$ and burst~$b_k$. The long-term arrival rate of a flow is $\min_k r_k$. Note that the parameters $(m, r_{1:m}, b_{1:m})$ may differ for every flow. 

A flow can be either \textit{unicast} (one source, one destination) or \textit{multicast} (one source, multiple destinations). In the latter case, traffic can be split at one or several intermediate devices. 
%
For the tools that do not support multicast flows, we replace every multicast flow with $p$ sub-paths by $p$ unicast flows with the same arrival curve constraint at the source; this increases the traffic inside the network, and thus delay bounds that we obtain are valid but might be less good than those obtained by tools that natively support multicast flows.  

The service offered to the aggregation of all flows of interest at an output port is represented by a service curve, which we assume to be piece-wise linear and convex. Such a service curve, say $\beta$, can be described by a collection of $n$~rates $R_1, R_2, ..., R_n$ and latencies $T_1, T_2, ...,T_n$ such that $\beta(t) = \max_{k=1:n}(R_k[t - T_k]^+)$, with the notation $[x]^+ = \max(x,0)$; each function $t \to R_k[t - T_k]^+$ is called a rate-latency function with rate~$R_k$ and latency~$T_k$. The long-term service rate of the output port is defined as $\max_k R_k$. Note that the parameters $(n, R_{1:n}, T_{1:n})$ may differ at every output port.

We assume that the network is locally stable, namely, at every output port, the aggregate long-term arrival rate (equal to the sum of the long-term arrival rates of all flows using the output port) is less than the long-term service rate. This is a necessary condition for the existence of finite delay bounds; it is also sufficient in feed-forward networks, but not in networks that have cyclic dependencies \cite[Chapter 12]{bouillard_deterministic_2018}.
