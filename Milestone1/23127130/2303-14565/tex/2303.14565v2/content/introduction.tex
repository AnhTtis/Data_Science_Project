\section{Introduction}
\label{sec: introduction}

Time-sensitive networks are used for real-time applications in various automation systems and require bounds on worst-case delays (cf. IEEE-TSN and IETF-Detnet). In such networks, communication flows are required to be constrained at their sources by traffic envelopes, also called arrival curves. The computation of good delay bounds is typically done using network calculus, which abstracts the service offered by a network element by a function called a service curve~\cite{le_boudec_network_2001,bouillard_deterministic_2018,nancy}. Finding the best delay bound, given the arrival curves of flows at the sources and the service curves offered by the nodes, is an NP-hard problem. 
Therefore, several methods were developed to find good delay bounds. Frequently used methods are Total Flow Analysis (TFA)~\cite{thoma2022analyse}, Separate Flow Analysis (SFA)~\cite{bouillard_deterministic_2018}, Pay Multiplexing Only Once (PMOO)~\cite{PMOO}, 
Least Upper Delay Bound (LUDB)~\cite{scheffler2021fifo}, Polynomial Linear Programming (PLP), and Exponential Linear Programming (ELP)~\cite{bouillard2022tradeoff}.
All methods provide valid delay bounds but differ in their design and implementation, and it is not trivial to identify the best, smallest bound among them. Therefore, for every application case, it is interesting to compare different methods and find the smallest delay bound.



The existing worst-case delay analysis tools, such as xTFA~\cite{thoma2022analyse}, DiscoDNC~\cite{scheffler2021fifo,bondorf2014discodnc}, and Panco~\cite{bouillard2022tradeoff} (see~\cite{listtoolWiki} for more tools), support some of the frequently used methods. These tools altogether cover most of the widely recognizable methods within the community. As of today, despite the great potential of utilizing multiple tools, users must implement multiple pieces of code with different syntaxes for each of them, which is impractical and error-prone. 

We present Saihu, \textbf{S}uperimposed worst-case delay \textbf{A}nalysis \textbf{I}nterface for \textbf{H}uman-friendly \textbf{U}sage, to simplify the whole process.
Users can execute analyses and compare the results from each tool easily with a single interface and simple commands. Saihu provides a general interface that enables defining the networks in one XML or JSON file and executing all tools simultaneously without any modification; it automatically generates input for each tool respectively and executes the analyses on them. Saihu can produce analysis results in formatted reports and offer automatic network generation for certain types of networks. Therefore, with its straightforward syntax and ease of execution, Saihu simplifies the worst-case bounds comparisons in time-sensitive networks.
Its design is modular and supports the addition of new tools. 
Fig.~\ref{fig: pipeline} illustrates the design of Saihu with its data flow.


\begin{figure}[tbh]
\centering
\includegraphics[width=0.85\linewidth]{pipeline.png}
\caption{Data flow of Saihu}
\label{fig: pipeline}
\end{figure}
