\section{Impact}
\label{sec: impact}

Worst-case delay analysis tools are essential to make theoretical network studies applicable to real-world problems. With each tool specializing in particular methods, the aggregation of these tools has great potential to find the best delay bound of a network. Saihu makes executing and comparing these analysis tools easy by encapsulating them into a single interface while hiding the complexity within. It turns a previously tedious multi-platform programming task into a straightforward ``write your problem down and press run." By simplifying the user's action to analyze a network, Saihu not only eases the work of existing researchers but also makes these tools accessible to a broader audience by requiring a more approachable level of programming skill. 

Saihu also serves as a platform for potential network analysis tool developers due to its cross-platform design. Since Saihu provides unified input/output formats, one can extend Saihu to include more tools by parsing one of the unified inputs and feeding the analysis results into the unified output class. Future developers can bridge their new tool to Saihu and allow existing Saihu users to access their tool easily without forcing people to learn a new framework. 