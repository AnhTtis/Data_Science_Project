%% From https://www.elsevier.com/journals/softwarex/2352-7110/guide-for-authors
%%
%% Copyright 2007, 2008, 2009 Elsevier Ltd
%% 
%% This file is part of the 'Elsarticle Bundle'.
%% ---------------------------------------------
%% 
%% It may be distributed under the conditions of the LaTeX Project Public
%% License, either version 1.2 of this license or (at your option) any
%% later version.  The latest version of this license is in
%%    http://www.latex-project.org/lppl.txt
%% and version 1.2 or later is part of all distributions of LaTeX
%% version 1999/12/01 or later.
%% 
%% The list of all files belonging to the 'Elsarticle Bundle' is
%% given in the file `manifest.txt'.
%% 

%% Template article for Elsevier's document class `elsarticle'
%% with numbered style bibliographic references
%% SP 2008/03/01

\documentclass[preprint,12pt, a4paper]{elsarticle}

%% Use the option review to obtain double line spacing
%% \documentclass[authoryear,preprint,review,12pt]{elsarticle}

%% For including figures, graphicx.sty has been loaded in
%% elsarticle.cls. If you prefer to use the old commands
%% please give \usepackage{epsfig}

%% The amssymb package provides various useful mathematical symbols
\usepackage{amssymb}
\usepackage{amsmath}
\usepackage{hyperref}
%% The amsthm package provides extended theorem environments
%% \usepackage{amsthm}

%% The lineno packages adds line numbers. Start line numbering with
%% \begin{linenumbers}, end it with \end{linenumbers}. Or switch it on
%% for the whole article with \linenumbers.
%\usepackage{lineno}

\journal{SoftwareX}

\usepackage{subcaption}
\usepackage{listings}

\usepackage{color}
\definecolor{gray}{rgb}{0.4,0.4,0.4}
\definecolor{darkblue}{rgb}{0.0,0.0,0.8}
\definecolor{cyan}{rgb}{0.0,0.6,0.6}
\definecolor{orange}{rgb}{0.8,0.4,0.0}
\definecolor{lightgreen}{rgb}{0.0,0.8,0.3}
\definecolor{codegreen}{rgb}{0,0.6,0}
\definecolor{codegray}{rgb}{0.5,0.5,0.5}
\definecolor{codepurple}{rgb}{0.3,0,0.52}
\definecolor{codepink}{rgb}{0.8, 0.35, 0.7}
\definecolor{lightblue}{rgb}{0.0,0.3,0.7}

% define XML
\lstset{
  basicstyle=\small\ttfamily\color{black},
  columns=fullflexible,
  showstringspaces=false,
  commentstyle=\color{gray}\upshape,
  frame=single,
  captionpos=b
}
\lstdefinelanguage{XML}{
    moredelim=[s][\color{cyan}]{\ }{=},
    moredelim=[s][\color{black}]{>}{<},
    morestring=[b]",
    morecomment=[s]{?}{?},
    morecomment=[s]{!--}{--},
    stringstyle=\color{orange},
    identifierstyle=\color{darkblue},
    keywordstyle=\color{cyan},
    morekeywords={xmlns,version,type}
}

% \lstdefinestyle{xml-style}{language=XML, extendedchars=true,  belowcaptionskip=5pt, xleftmargin=1.8em, xrightmargin=0.5em, numbers=left, numberstyle=\small\ttfamily\bf, frame=single, breaklines=true, breakatwhitespace=true, breakindent=0pt, emph={}, emphstyle=\color{red}, basicstyle=\small\ttfamily, columns=fullflexible, showstringspaces=false, commentstyle=\color{gray}\upshape,
%     morestring=[b]",
%     morecomment=[s]{<?}{?>},
%     morecomment=[s]{<!--}{-->},
%     stringstyle=\color{orange},
%     identifierstyle=\color{darkblue},
%     keywordstyle=\color{cyan},
%     morekeywords={xmlns,version,type}
% }


% define json
\newcommand{\jsonkey}{\small\color{darkblue}}
\newcommand{\jsonvalue}{\small\color{orange}}
\newcommand{\jsonnumber}{\small\color{cyan}}
\newcommand{\jsonbool}{\small\color{codegreen}}

% switch used as state variable
\makeatletter
\newif\ifisvalue@json
\newif\ifisarray@jsonArray

\lstdefinelanguage{json}{
    tabsize             = 4,
    showstringspaces    = false,
    keywords            = {false,true},
    alsoletter          = 0123456789.,
    morestring          = [s]{"}{"},
    stringstyle         = \jsonkey\ifisvalue@json\jsonvalue\fi\ifisarray@jsonArray\jsonvalue\fi,
    keywordstyle        = \jsonbool,
    MoreSelectCharTable = \lst@DefSaveDef{`:}\colon@json{\enterMode@json},
    MoreSelectCharTable = \lst@DefSaveDef{`,}\comma@json{\exitMode@json{\comma@json}} \ifisarray@jsonArray\array@json{\enterMode@json}\fi,
    MoreSelectCharTable = \lst@DefSaveDef{`\{}\bracket@json{\exitMode@json{\bracket@json}},
    MoreSelectCharTable = \lst@DefSaveDef{`\{}\bracket@jsonArray{\exitMode@jsonArray{\bracket@jsonArray}},
    MoreSelectCharTable = \lst@DefSaveDef{`[}\lb@jsonArray{\enterMode@jsonArray{\lb@jsonArray}},
    MoreSelectCharTable = \lst@DefSaveDef{`]}\rb@jsonArray{\exitMode@jsonArray{\rb@jsonArray}},
    basicstyle          = \small\ttfamily
}

% enter "value" mode after encountering a colon
\newcommand\enterMode@json{%
    \colon@json%
    \ifnum\lst@mode=\lst@Pmode%
        \global\isvalue@jsontrue%
    \fi
}

% leave "value" mode: either we hit a comma, or the value is a nested object
\newcommand\exitMode@json[1]{#1\global\isvalue@jsonfalse}

\lst@AddToHook{Output}{%
    \ifisvalue@json%
        \ifnum\lst@mode=\lst@Pmode%
            \def\lst@thestyle{\jsonnumber}%
        \fi
    \fi
    %override by keyword style if a keyword is detected!
    \lsthk@DetectKeywords% 
}

% enter "value" mode after encountering a colon
\newcommand\enterMode@jsonArray{%
    \lb@jsonArray%
    \ifnum\lst@mode=\lst@Pmode%
        \global\isarray@jsonArraytrue%
    \fi
}

% leave "value" mode: either we hit a comma, or the value is a nested object
\newcommand\exitMode@jsonArray[1]{#1\global\isarray@jsonArrayfalse}

\lst@AddToHook{Output}{%
    \ifisarray@jsonArray%
        \ifnum\lst@mode=\lst@Pmode%
            \def\lst@thestyle{\jsonnumber}%
        \fi
    \fi
    %override by keyword style if a keyword is detected!
    \lsthk@DetectKeywords% 
}



% define python
\newcommand{\lstsetblack}{
    \exitMode@json{}
    \exitMode@jsonArray{}
}
\lstdefinestyle{pythonstyling}{
language=Python,
keywords={from,import},
basicstyle=\small\ttfamily\color{black},
morekeywords={self},              % Add keywords here
keywordstyle=\color{codepink},
identifierstyle=\color{lightblue},
emph={MyClass,__init__},          % Custom highlighting
emphstyle=\color{darkblue},    % Custom highlighting style
stringstyle=\color{orange},
commentstyle=\color{codegreen},
frame=single,                         % Any extra options here
showstringspaces=false
}

\graphicspath{{images/}}


\usepackage[normalem]{ulem}

% Control the version of paper, comment out one to use the other
\newcommand{\ArxivVersion}{0}
% \newcommand{\SoftwareXVersion}{0}

\begin{document}

\begin{frontmatter}

%% Title, authors and addresses

%% use the tnoteref command within \title for footnotes;
%% use the tnotetext command for theassociated footnote;
%% use the fnref command within \author or \address for footnotes;
%% use the fntext command for theassociated footnote;
%% use the corref command within \author for corresponding author footnotes;
%% use the cortext command for theassociated footnote;
%% use the ead command for the email address,
%% and the form \ead[url] for the home page:
%% \title{Title\tnoteref{label1}}
%% \tnotetext[label1]{}
%% \author{Name\corref{cor1}\fnref{label2}}
%% \ead{email address}
%% \ead[url]{home page}
%% \fntext[label2]{}
%% \cortext[cor1]{}
%% \address{Address\fnref{label3}}
%% \fntext[label3]{}

\newcommand{\jylb}[1]{\textcolor{red}{JYLB:#1}}
\newcommand{\hossein}[1]{\textcolor{blue}{Hossein:#1}}
\newcommand{\stephan}[1]{\textcolor{green}{Stéphan:#1}}

\title{Saihu: A Common Interface of Worst-Case Delay Analysis Tools for Time-Sensitive Networks}
% \footnotemark
%% use optional labels to link authors explicitly to addresses:
%% \author[label1,label2]{}
%% \address[label1]{}
%% \address[label2]{}

\author{Chun-Tso Tsai\corref{cor1}}
\ead{chun-tso.tsai@epfl.ch}

\author{Seyed Mohammadhossein Tabatabaee\corref{cor2}}
\ead{hossein.tabatabaee@epfl.ch}

\author{Stéphan Plassart\corref{cor3}}
\ead{stephan.plassart@epfl.ch}

\author{Jean-Yves Le Boudec\corref{cor4}}
\ead{jean-yves.leboudec@epfl.ch}

\cortext[cor1]{Corresponding author}


\address{School of Computer and Communication Sciences, \\École Polytechnique Fédérale de Lausanne}

\begin{abstract}
   Time-sensitive networks, as in the context of IEEE-TSN and IETF-Detnet, require bounds on worst-case delays. Various network analysis tools compute such bounds; these tools are based on different methods and provide delay bounds that are all valid but may differ; furthermore, it is generally not known which tool will provide the best bound. To obtain the best possible bound, users need to implement multiple pieces of code with a different syntax for every tool, which is impractical and error-prone. To address this issue, we present Saihu, a Python interface that integrates the three most frequently used worst-case network analysis tools: xTFA, DiscoDNC, and Panco. They altogether implement six analysis methods. Saihu provides a general interface that enables defining a network in a single file and executing all tools simultaneously without any modification. Saihu further exports analysis results as formatted reports automatically and allows quick generation of certain types of networks. With its simplified steps of execution, Saihu reduces the burden on users and makes it accessible for anyone working with time-sensitive networks. An introductory video is available at \url{https://youtu.be/MiOhLay8Kr4}.
\end{abstract}

\begin{keyword}
Worst-case Delay Analysis \sep Network Analysis Interface \sep Network Calculus \sep Time-Sensitive Networking
\end{keyword}

\end{frontmatter}


\section{Introduction}
\label{sec:introduction}
% \begin{itemize}
%     % Diffusion of FL
%     \item {\st{Diffusion of FL}}
%     % Security threats to FL
%     \item {\st{Security threats to FL with particular focus on model poisoning}}
%     % Limitations of existing countermeasures
%     \item {\st{Current countermeasures (e.g., KRUM) and their limitations}}
%     % Proposed method and its advantages
%     \item {\st{Intuitive description of the proposed method and its difference (i.e., advantages) w.r.t. state of the art}}
%     % Main contributions
%     \item {\st{Summary of the main contributions of this work}}
%     % Paper's structure and organization
%     \item {\st{Paper's structure and organization}}
% \end{itemize}

% Diffusion of FL
Recently, {\em federated learning} (FL) has emerged as the leading paradigm for training distributed, large-scale, and privacy-preserving machine learning (ML) systems~\cite{mcmahan2017googleai,mcmahan2017aistats}. 
The core idea of FL is to allow multiple edge clients to collaboratively train a shared, global model without disclosing their local private training data.
%Specifically, an FL system consists of a central server and many edge clients; 
A typical FL round involves the following steps: {\em(i)} the server randomly picks some clients and sends them the current, global model; {\em(ii)} each selected client locally trains its model with its own private data; then, it sends the resulting local model to the server;\footnote{Whenever we refer to global/local model, we mean global/local model {\em parameters}.} {\em(iii)} the server updates the global model by computing an \emph{aggregation function}, usually the average (FedAvg), on the local models received from clients.
% \begin{enumerate}
%     \item[{\em(i)}] the server sends the current, global model to the clients and appoints some of them for training;
%     \item[{\em(ii)}] each selected client locally trains its copy of the global model with its own private data; then, it sends the resulting local model back to the server;\footnote{Whenever we refer to global/local model, we mean global/local model {\em parameters}.}
%     \item[{\em(iii)}] the server updates the global model by computing an \emph{aggregation function} on the local models received from clients (by default, the average, also referred to as FedAvg~\cite{mcmahan2017aistats}).
% \end{enumerate}
This process goes on until the global model converges. %(e.g., after a certain number of rounds or other similar stopping criteria).
%\\
% The advantages of FL over the traditional, centralized learning paradigm are undoubtedly clear in terms of flexibility/scalability (clients can join/disconnect from the FL network dynamically), network communications (only model weights\footnote{We will use \textit{parameters} and \textit{weights} interchangeably.} are exchanged between clients and server), and privacy (each client's private training data is kept local at the client's end and not uploaded to the server).
\\
% Security threats to FL
%However, the growing adoption of FL also raises security concerns~\cite{costa2022covert}, particularly about its confidentiality, integrity, and availability.
Although its advantages over standard ML, FL also raises security concerns~\cite{costa2022covert}. %, particularly about its confidentiality, integrity, and availability~\cite{costa2022covert}.
% OLD, LONG VERSION
% Indeed, some work deals with privacy leakage that may expose the local data of some clients~\cite{melis2019sp}. 
% A large body of work, instead, investigates attacks that usually aim to detriment the predictive accuracy of the learned global model. For instance, \emph{data poisoning} attacks achieve this goal by letting an adversary pollute the training set of some corrupt FL clients with maliciously crafted examples~\cite{jagielski2018sp}.
% Similarly, in \emph{model poisoning} the attacker attempts to tweak the global model weights~\cite{bhagoji2019pmlr} by directly perturbing the local model's weights of some infected FL clients before these are sent to the central server for aggregation, usually via so-called Byzantine attacks. 
% It turns out that Byzantine model poisoning attacks severely impact standard FedAvg; therefore, more robust aggregation functions must be designed to make FL systems secure.
Here, we focus on \emph{untargeted model poisoning} attacks~\cite{bhagoji2019pmlr}, where an adversary attempts to tweak the global model weights %\footnote{We will use the terms \textit{parameters} and \textit{weights} interchangeably.} 
by directly perturbing the local model's parameters of some infected clients before these are sent to the central server for aggregation.
In doing so, the adversary aims to jeopardize the global model \textit{indiscriminately} at inference time.
Such model poisoning attacks severely impact standard FedAvg; therefore, more robust aggregation functions must be designed to secure FL systems.
\\
% In this paper, we focus on designing a novel robust aggregation scheme at the server's end to contrast the effect of Byzantine model poisoning attacks.
%
% Current countermeasures and their limitations
%Several countermeasures have been proposed in the literature to combat model poisoning attacks on FL systems.
% Some methods use simple statistics more robust than plain average to smooth the impact of malicious updates (e.g., Trimmed Mean and FedMedian~\cite{yin2018icml}). 
% Other defenses implement outlier detection techniques to discard malicious updates from the aggregation performed at the server's end. Those are either based on heuristics (e.g., Krum/Multi-Krum~\cite{blanchard2017nips} and Bulyan~\cite{mhamdi2018pmlr}) or data-driven approaches (e.g., K-means clustering~\cite{shen2016acm} or DnC via spectral analysis~\cite{shejwalkar2021ndss}). 
% Finally, some strategies rely on a centralized ``source of trust'' to spot potential malicious updates (e.g., FLTrust~\cite{cao2020fltrust}).
% Several countermeasures have been proposed in the literature to combat model poisoning attacks on FL systems, i.e., to discard possible malicious local updates from the aggregation performed at the server's end. 
% These techniques range from simple statistics more robust than plain average (e.g., Trimmed Mean and FedMedian~\cite{yin2018icml}) to outlier detection heuristics (e.g., Krum/Multi-Krum~\cite{blanchard2017nips} and Bulyan~\cite{mhamdi2018pmlr}) or data-driven approaches (e.g., spectral analysis via K-means clustering~\cite{shen2016acm} or spectral analysis), or methods based on ``source of trust'' (e.g., FLTrust~\cite{cao2020fltrust}).
% OLD, LONG VERSION
%Several countermeasures have been proposed in the literature to combat Byzantine model poisoning attacks on FL systems.
% Descriptive statistics
% For example, Trimmed Mean and FedMedian aggregate local model updates using more robust statistics than standard average~\cite{yin2018icml}.
%
% % Heuristics for outlier detection
% Many existing Byzantine-resilient strategies implement some outlier detection heuristics to discard the model updates sent by potentially malicious clients from the input of the aggregation function.
% One of the most popular heuristics is Krum~\cite{blanchard2017nips}.
% This strategy tries to mitigate the impact of Byzantine attacks by selecting as a global model the local model with the smallest sum of Euclidean distances to {\em all} the other local models.
% Although powerful, Krum requires the server to know (or, at least, estimate) the number of malicious FL clients upfront, which is generally impossible in a realistic attack scenario. %
% Moreover, Krum may become ineffective for complex, high-dimensional model parameter spaces due to the curse of dimensionality.
% Bulyan~\cite{mhamdi2018pmlr} tries to overcome this issue by combining Krum with a variant of Trimmed Mean.
% % Data-driven outlier detection
% Other strategies use data-driven outlier detection techniques -- e.g., via K-means clustering~\cite{shen2016acm} -- to spot potential malicious local model updates. 
% %For instance, Shen et al. propose to cluster local model updates with K-means and thus identify outliers.
%
% % Other techniques
% As far as the server is concerned, any local model received can be from a potential malicious client. 
% FLTrust~\cite{cao2020fltrust} assumes the server acts as a client, i.e., trains a local model on an additional {\em trustworthy} dataset at the server's end and compares it against all the local models from other clients. 
% This way, the server can rely on some ``source of trust'' when discarding potentially malicious clients.
%\\
% Limitations of existing Byzantine-resilient strategies
Unfortunately, existing defense mechanisms either rely on simple heuristics (e.g., Trimmed Mean and FedMedian by~\cite{yin2018icml}) or need strong and unrealistic assumptions to work effectively (e.g., foreknowledge or estimation of the number of malicious clients in the FL system, as for Krum/Multi-Krum~\cite{blanchard2017nips} and Bulyan~\cite{mhamdi2018pmlr}, which, however, cannot exceed a fixed threshold).
Furthermore, outlier detection methods using K-means clustering~\cite{shen2016acm} or spectral analysis like DnC~\cite{shejwalkar2021ndss} do not directly consider the temporal evolution of local model updates received.
Finally, strategies like FLTrust~\cite{cao2020fltrust} require the server to collect its own dataset and act as a proper client, thereby altering the standard FL protocol.
\\
% OLD, LONG VERSION
% Overall, existing Byzantine-resilient strategies are either simple heuristics (e.g., FedMedian) or, if they are more complex, they rely on strong and unrealistic assumptions to work effectively (e.g., knowing the number of malicious clients in the FL system in advance, as for Krum and alike).
% Furthermore, data-driven outlier detection methods do not consider the temporary evolution of local model updates received (e.g., K-means clustering). 
% Finally, strategies like FLTrust requires the server to collect its own dataset and act as a proper client, thereby altering the standard FL protocol.
%
% Description of the proposed method
This work introduces a novel pre-aggregation \textit{filter} robust to untargeted model poisoning attacks. Notably, this filter $(i)$ operates without requiring prior knowledge or constraints on the number of malicious clients and $(ii)$ inherently integrates temporal dependencies. 
The FL server can employ this filter as a preprocessing step before applying \textit{any} aggregation function, be it standard like FedAvg or robust like Krum or Bulyan.
Specifically, we formulate the problem of identifying corrupted updates as a multidimensional (i.e., matrix-valued) time series anomaly detection task. 
The key idea is that legitimate local updates, resulting from well-calibrated iterative procedures like stochastic gradient descent (SGD) with an appropriate learning rate, show \textit{higher predictability} compared to malicious updates. This hypothesis stems from the fact that the sequence of gradients (thus, model parameters) observed during legitimate training exhibit regular patterns, as validated in Section~\ref{subsec:intuition}. %until convergence. 
%This regularity may be more pronounced for smooth convex loss functions, but it can still be captured within an appropriate time window, even for more complex and convoluted loss surfaces. 
%We provide evidence of this claim in Appendix~B, where we show that the average mutual information (i.e., ``predictability''), calculated over pairs of legitimate model updates sent at different FL rounds, is significantly higher than the corresponding computation for a malicious client.
\\
Inspired by the matrix autoregressive (MAR) framework for multidimensional time series forecasting~\cite{chen2021je}, we propose the FLANDERS ({\em \textbf{F}ederated \textbf{L}earning meets \textbf{AN}omaly \textbf{DE}tection for a \textbf{R}obust and \textbf{S}ecure}) filter.
The main advantages of FLANDERS over existing strategies like FLDetector~\cite{zhao2020multivariate} are its resilience to large-scale attacks, where $50\%$ or more FL participants are hostile, and the capability of working under realistic non-iid scenarios.
We attribute such a capability to two key factors: $(i)$ FLANDERS works without knowing a priori the ratio of corrupted clients, and $(ii)$ it embodies temporal dependencies between intra- and inter-client updates, quickly recognizing local model drifts caused by evil players. Below, we summarize our main contributions:

\begin{itemize}
\item[{\em(i)}]
We provide empirical evidence that the sequence of models sent by legitimate clients is more predictable than those of malicious participants performing untargeted model poisoning attacks.
\\
\item[{\em(ii)}] 
We introduce FLANDERS, the first pre-aggregation filter for FL robust to untargeted model poisoning based on multidimensional time series anomaly detection.
\\
\item[{\em(iii)}] 
We integrate FLANDERS into Flower,\footnote{\scriptsize{\url{https://flower.dev/}}} a popular FL simulation framework for reproducibility.
\\
\item[{\em(iv)}] 
We show that FLANDERS improves the robustness of the existing aggregation methods under multiple settings: different datasets, client's data distribution (non-iid), models, and attack scenarios.
\\
\item[{\em(v)}] 
We publicly release all the implementation code of FLANDERS along with our experiments.\footnote{\scriptsize{\url{https://anonymous.4open.science/r/flanders_exp-7EEB}}}
\end{itemize}

% Paper's structure and organization
The remainder of the paper is structured as follows. %some related work and the current state-of-the-art solutions to security issues that FL entails. 
Section~\ref{sec:background} covers background and preliminaries. 
In Section~\ref{sec:related}, we discuss related work.
Section~\ref{sec:problem} and Section~\ref{sec:method} describe the problem formulation and the method proposed. % to tackle it. 
Section~\ref{sec:experiments} gathers experimental results. %, and Section~\ref{sec:limitations} discusses some limitations of this work.
Finally, we conclude in Section~\ref{sec:conclusion}.
 %discusses the limitations of this work and draws future research directions.
%reports conclusions and draws perspectives for future research directions.

%%%%%%% OLD %%%%%%%
%to overcome the resilience of Byzantine failures in distributed Stochastic Gradient Descent computations. 
% The strength of Krum is its time complexity, which is linear in the gradient dimension. 
% However, the robustness of the approach is guaranteed for gradient-based learning applications only when the majority of the clients are not compromised. 
% Besides, the aggregation mechanism of Krum, as well as that of similar methods, is robust from a coarse-grained perspective and does not provide solutions to errors and perturbations that may occur at inference time.
%A related approach to~\cite{blanchard2017nips} is the work of Su et al.~\cite{su2016dc}. Here, the authors propose an iterated approximate agreement to tackle a multi-layer scenario attacked by Byzantine agents. 
%However, the method works efficiently on the sole discrete context and it is inapplicable to continuous state environments.
%\gabri{Maybe, we should just talk about the main limitations of existing countermeasures without digging into their details (or, we can just mention Krum as this is the most popular one). I will move the description of all these methods to the Related Work section.}
\section{Proposed Framework: {\ourmodel}}
\label{model}


In this section, we introduce a novel self-supervised co-training framework {\ourmodel}.
The proposed framework is illustrated in Figure~\ref{fig:intro_model} and works in three phases.
Phase one automatically generates two sets of pseudo labels.
We use a combination of off-the-shelf pre-trained POS and NER taggers, knowledge graph, and GPT-2 scorer for generating the first set of pseudo labels automatically without any hand-crafted rules for matching the slot values.
The other set of pseudo labels is acquired through a zero-shot slot filling model~\cite{liu2020coach}, trained on the out-of-domain dataset.
It is critical to emphasize that both sets of labels are noisy and incomplete which poses serious challenges to training effective models for the task of open-domain slot filling.
Phase two fine-tunes the pre-trained BERT to the slot filling task that effectively transfers the knowledge from the pre-trained language model~(LM) to overcome the issue of label incompleteness to some extent. 
Further, we employ the early stopping technique to minimize the noise in the labels.
The output of this phase is two BERT models that can generate soft labels for self-supervision during co-training in phase three.
Phase three leverages the fine-tuned models and further trains them in an iterative fashion.
Specifically, the proposed peer training approach facilitates high-confidence soft label selection for the other peer to perform training. This phase progressively reduces the noise in the labels and enables effective model fitting. 



\subsection{Phase One: Automatic Label Generation}
To acquire the first set of labels, we perform the following steps.
First of all, off-the-shelf trained POS and NER taggers are used to predict initial estimates of the slot values irrespective of the slot types. Then, the type information of the slot values is queried from the KG and the slot value is tagged for the most appropriate slot in the target domain.
This approach, however, produces low recall. 
To expand the candidate slot values, we generate n-grams of the natural language text and employ a partial matching scheme to query the KG for type information (e.g., \myspecial{Jason} \myspecial{Aldean} = \myspecial{American} \myspecial{singer}) of the n-grams if the entry exists.
This process generates multiple overlapping hypotheses about the slot values.
We replace a span of text that corresponds to a slot value by its type information and a GPT-2 based scorer (see Section~\ref{sec:nlpmodels}) is used to select the best candidate based on the fluency of the text.
Naturally, if a token (or span of tokens) is replaced by its type, the sentence should score higher as compared to the case where an inappropriate substitution is performed. 
We select the best hypothesis if the score is greater than the threshold.
Intuitively, the candidate selection threshold can automatically be searched based on a small validation set from the target domain, making the label generation process fully automatic. 
The other set of noisy labels is acquired by the zero-shot slot filling model~\cite{liu2020coach} that has been trained using an out-of-domain dataset. It is important to highlight that the zero-shot slot filling model does not require any labeled in-domain training example. 
To summarize the automatic label generation phase, both sets of labels are acquired in a fully automatic fashion without any hand-crafting.


In contrast to previous work in weak supervision~\cite{ren2015clustype,he2017autoentity,fries2017swellshark,giannakopoulos2017unsupervised} that obtains a single set of noisy labels and then propose techniques to overcome the challenge of fitting an effective model to the noisy labels, we acquire two sets of complementary labels.
The choice of these two sets of labels is guided by the intuition that they should be complementary and the models trained on these sets of labels should be able to share complementary information with the other to improve the performance in the later phases of the framework.
Essentially, the first set of labels carries information from external knowledge sources, whereas the labels generated through the pre-trained zero-shot slot filling model capture how the slot values are mentioned in other domains.
%
To further elaborate on the motivation and our process for the first set of labels (i.e., labels using KG and other NLP models), the pre-trained LMs have been shown to have a great deal of knowledge~\cite{petroni2019language}, thus should be capable of generating automatic labels with no need of external KG. 
To the best of our knowledge, there exists no work that shows that accurate token-level automatic labeling (e.g., slot filling task) is possible with pre-trained LMs. 
Moreover, such approaches would require heavy prompting in each new target domain, whereas our label generation process is fully automatic and only relies on the readily-available pre-trained NLP models and external KG.

\subsection{Phase Two: LM-assisted Weak Supervision}
Since we do not have access to dataset $\{(\mathbf{X}_n,\mathbf{Y}_n)\}_{n=1}^N$ with true ground-truth labels.
We use pseudo labels generated in phase one, $\{(\mathbf{X}_n,\mathbf{D}_n)\}_{n=1}^N$, to learn 
$f_{m,c}(\cdot; \cdot)$ that outputs the probability of the $m$-th token to take on class $c$. 
We learn $f_{m,c}(\cdot; \cdot)$ by minimizing the following loss over the noisy dataset $\{(\mathbf{X}_n,\mathbf{D}_n)\}_{n=1}^N$: 
$$
\hat\theta = \argmin_{\theta}\frac{1}{N}\sum_{n=1}^{N} \ell(\mathbf{D}_n, f(\mathbf{X}_{n}; \theta)),
\label{eq:stage1}
$$
where $\ell(\mathbf{D}_n, f(\mathbf{X}_{n}; \theta)) = \frac{1}{M} \sum_{m=1}^{M} -\log{f_{m,d_{n, m}}(\mathbf{X}_{n}; \theta)}$. 
We employ the pre-trained multilingual BERT with token-level classification head that uses Adam optimizer \cite{kingma2014adam,Liu2019} with early stopping and multiple random initializations. 


Since slot filling task is similar to the MLM training objective of the BERT, we employ pre-trained BERT as the backbone model.
That is, MLM's goal is to predict the masked tokens using bidirectional contexts. Similarly, slot filling tries to predict the label for a token leveraging both left and right contexts simultaneously, which makes the pre-trained BERT an ideal model of choice that greatly facilitates minimizing incomplete labels.
It is important to highlight that our automatically generated labels are not only incomplete but also potentially wrong.
The training strategies employed in this phase minimize the noise in the label to some extent. 
Specifically, early stopping can provide a strong regularization and would not let the model overfit to the noisy labels, especially wrong labels. 
Moreover, early stopping does not let the model forget the knowledge in the pre-trained model.
Similarly, multiple random initializations enforce robustness. 
Since the model is fine-tuned on the noisy labels, averaging the predictions of multiple models for each token ensures that wrong labels end up with low probabilities and true labels consistently achieve high probabilities.
Using the above-mentioned strategies, we train two slot filling models, which we call the peers. The peer one is trained on the first set of pseudo labels that were generated using POS and NER taggers, KG, and the GPT-2 scorer in phase one. Similarly, peer two is trained using the predictions of the zero-shot slot filling model~\cite{liu2020coach}.
Both models have the same architecture and follow the same training procedures.

\begin{table*}[t!]
\centering
\caption{Dataset statistics.}
\vspace{-7pt}
\label{tab:dataset}
\begin{tabular}{lccccc}
\toprule
\textbf{Dataset}  & \textbf{Dataset Size} & \textbf{Vocab. Size} & \textbf{Avg. Length} & \textbf{\# of Domains} & \textbf{\# of Slots} \\ \hline
\textbf{SGD}      & 188K                  & 33.6K                & 13.8                 & 20                     & 240                  \\
\textbf{MultiWoZ} & 67.4K                 & 10.5K                & 13.3                 & 8                      & 61 \\
\bottomrule
\end{tabular}
\vspace{-7pt}
\end{table*}

\subsection{Phase Three: Self-supervised Co-training}
We introduce an iterative peer training algorithm where both peers generate high-confidence soft labels for training the other peer in the next iteration. 
Theoretically, these peers can be anything, but in this work, 
we explore two of the most promising directions that have shown the promise to minimize the need for manual labeling for the task: zero-shot learning and distant supervision.
This phase uses a self-supervised co-training scheme to exploit the patterns of slot values from other domains through the labels generated by the zero-shot filling model (i.e., peer two)~\cite{liu2020coach} as well as utilize the knowledge in external KGs and pre-trained models via labels provided by the peer one.
Specifically, we initialize the peers trained in phase two and use their pseudo labels to kick-start training in this phase.
Specifically, peer one $f_{m,c}(\cdot; \theta_{\textrm{p1}})$ would generate labels $\{\tilde{\mathbf{Y}}^{(t)}_n = [\tilde{y}_{n,1}^{(t)}, ..., \tilde{y}_{n,m}^{(t)}]\}_{n=1}^{N}$ for peer two $f_{m,c}(\cdot; \theta_{\textrm{p2}})$ at the $t$-th iteration by:
$$
\tilde{y}_{n,m}^{(t)} = \argmax_{c}{f_{m,c}(\mathbf{X}_n; \theta_{\textrm{p1}}^{(t)})}. 
\label{eq:pseudo}
$$

Based on these labels, the peer two can be fine-tuned by: 
$$
\hat\theta_{\textrm{p2}}^{(t+1)} = \argmin_{\theta}\frac{1}{N}\sum_{n=1}^N \ell(\tilde{\mathbf{Y}}_n^{(t)}, f(\mathbf{X}_{n}; \theta)).
\label{eq:self_train1}
$$

Similarly, peer two $f_{m,c}(\cdot; \theta_{\textrm{p2}})$ would generate pseudo labels for peer one $f_{m,c}(\cdot; \theta_{\textrm{p1}})$ that are used to fine-tune peer one. 
We also notice that it is beneficial to stop early during this phase as well, to improve the model fitting and gradually reduce the noise associated with the automatically generated labels.
Since pseudo labels are refined gradually in an iterative way, both peers can benefit from the knowledge contained within the labels of the other while avoiding overfitting.
Furthermore, as an alternative to pseudo labels, we also generate soft labels that are used for confidence re-weighting. 
The high-confidence soft label selection strategy enables better model fitting and efficient learning via better quality of the automatic labels.
Specifically, for the given $m$-th token in the $n$-th training example, the probability for all classes $C$ is $[f_{m,1}(\mathbf{X}_n;\theta),...,f_{m,C}(\mathbf{X}_n;\theta)]$. 
Following ~\cite{xie2016unsupervised}, at $t$-th iteration, peer one generates soft labels, $\{\mathbf{S}_n^{(t)} = [\mathbf{s}_{n,m}^{(t)}]_{m=1}^M \}_{n=1}^N$, as given below:
$$
\mathbf{s}_{n,m}^{(t)} = [s_{n,m,c}^{(t)}]_{c=1}^{C} = \Bigg[  \frac{f_{m,c}^2(\mathbf{X}_n;\theta_{\textrm{peer1}}^{(t)})/p_{c}}{\sum_{c'=1}^C f_{m,c'}^2(\mathbf{X}_n;\theta_{\textrm{peer1}}^{(t)})/p_{c'}}\Bigg]_{c=1}^{C}
\label{eq:soft}
$$ 
where $p_{c} = \sum_{n=1}^N \sum_{m=1}^M f_{m,c}(\mathbf{X}_n;\theta_{\textrm{p1}}^{(t)})$ computes the frequency of the tokens for the $c$-th class. 
Then, peer two $f(\cdot; \theta_{\textrm{p2}}^{(t+1)})$ is fine-tuned by:
$$
\theta_{\textrm{p2}}^{(t+1)} = \argmin_{\theta} \frac{1}{N} \sum_{n=1}^{N} \ell_{\rm KL}(\mathbf{S}_n^{(t)}, f(\mathbf{X}_{n}; \theta)),
$$
where $\ell_{\rm KL}(\cdot,\cdot)$ is the KL-divergence-based loss:
$$
\ell_{\rm KL}(\mathbf{S}_n^{(t)}, f(\mathbf{X}_{n}; \theta))=\frac{1}{M}\sum_{m=1}^M\sum_{c=1}^C - s_{n,m,c}^{(t)} \log f_{m,c}(\mathbf{X}_{n}; \theta).
\label{eq:klloss}
$$

Moreover, we also investigate selecting tokens that have high confidence. 
For instance, we pick high-confidence tokens from the $m$-th input example at the $t$-th iteration by  
$
H^{(t)}_n = \{m : \max_{c} s_{n,m,c}^{(t)} > \epsilon \},
$
where $\epsilon\in [0,1]$ is a threshold that can be searched based on a small validation set. 
Then, peer two $f(\cdot; \theta_{\textrm{p2}}^{(t+1)})$ is fine-tuned by:
$$
\theta_{\textrm{p2}}^{(t+1)} %&= \argmin_{\theta} \frac{1}{N} \sum_{n=1}^{N} \ell_{\rm S-KL}(\bS_n^{(t)}, f(\bX_{n}; \theta)) \\
= \argmin_{\theta} \frac{1}{N|H^{(t)}_n|}\sum_{n=1}^{N} \sum_{m\in H^{(t)}_n}\sum_{c=1}^C - s_{n,m,c}^{(t)} \log f_{m,c}(\mathbf{X}_{n}; \theta).
$$

This phase improves the robustness to effectively fit the model for tokens with high confidence. 
Both peers keep sharing information and their confidence by producing soft labels for their counterparts until they approximate to the true labels while employing early stopping and scheduled learning rates.
It is important to remind that phase three is the most important phase that progressively reduces noise from the labels to a great extent and enables superior performance for the task of open-domain slot filling.
\section{Included Tools}
\label{sec: included tools}

Saihu currently includes 3 tools, namely xTFA, DNC, and Panco. A summary of the supported method and tool pairs is listed in Figure~\ref{fig: supported methods}.

\begin{figure}[!thb]
    \centering
    \begin{tabular}{|c|c|c|c|}
        \hline
        Method\textbackslash Tool & DNC & xTFA & Panco \\
        \hline\hline
        TFA & V & V & V \\
        SFA & V &   & V \\
        PLP &   &   & V \\
        ELP &   &   & V \\
        PMOO& V &   &   \\
        TMA & V &   &   \\
        \hline
    \end{tabular}
    \caption{Supported methods for each tool. A check ``V'' on it means the tool supports the corresponding method.}
    \label{fig: supported methods}
\end{figure}

\textbf{xTFA} \cite{thoma2022analyse}:
xTFA is short for \textit{experimental modular TFA}, which is developed in Python and supports a more advanced TFA (Total Flow Analysis). It takes an XML file as its input for network description, which we will discuss in more detail in Section \ref{sec: physical network xml}. xTFA supports analyzing networks with cyclic dependency and multicast flows, i.e. a flow having multiple paths or potential splits. In other tools, a multicast flow will be treated as separated flows with the same arrival bounds.

\textbf{DNC}~\cite{bondorf2014discodnc}:
DNC is developed in Java and supports TFA, SFA (Separate Flow Analysis), PMOO (Pay Multiplexing Only Once)~\cite{bondorf2014discodnc}, and TMA (Tandem Matching Analysis)~\cite{scheffler2021fifo}. There's no specific input description file for DNC, one has to define the network as a Java script if they use DNC directly. Saihu uses the information from an output port network to create a network in DNC syntax internally. Moreover, with DNC, one cannot manually set shaping with FIFO multiplexing but only with arbitrary multiplexing. Also, no analysis methods in DNC are capable of solving networks with cyclic dependency.

\textbf{Panco}~\cite{bouillard2022tradeoff}:
Panco is developed in Python and supports TFA, SFA, PLP (Polynomial size Linear Program), and ELP (Exponential size Linear Program). Since all its methods are implemented as linear programs, it requires \textit{lpsolve}~\cite{lpsolve}. Same as DNC, Panco doesn't have a specific input description file, Saihu internally creates the network in Panco syntax from the information of an output port network. All the methods of Panco except ELP support networks with cyclic dependency. 


\section{Software Description}
\label{sec: software description}
To execute analyses with Saihu, we roughly divide the tasks into 3 parts: describe a network to be analyzed; execute analyses with individual tools; and export reports back to the user. We will go through these 3 parts one by one.

\subsection{Network Description File}
\label{sec: network description file}
As mentioned in Section~\ref{sec: system model}, Saihu allows the user to write a network in either a \textit{physical network} or an \textit{output port network} format. While xTFA takes a physical network as an XML file and the others take an output port network as a JSON file, one can choose the format they prefer to define a network as Saihu automatically converts a file when needed. 

\subsubsection{Option 1: Defining Physical Network in XML}
\label{sec: physical network xml}
A physical network is written as an XML file according to the xTFA specification. It should at least contains 4 kinds of information: \textbf{General network information}, \textbf{Servers}, \textbf{Links}, and \textbf{Flows}.

Let's take the implementation of Figure \ref{fig: physical network} as an example. First, every entry should be enclosed in one element \texttt{<elements>} as shown in Listing~\ref{lst: xml elements}.

\begin{lstlisting}[language=XML,caption={Examples of XML file. All network entries must inside an element \texttt{<elements>}.},
label={lst: xml elements}]
<?xml version="1.0" encoding="UTF-8"?>
<elements>
    <!-- All entries -->
</elements>
\end{lstlisting}

A physical network must have exactly one \texttt{network} element to define general information across the network as its attributes, an example is shown in Listing~\ref{lst: xml network}. In this example, \texttt{name} is the name of the network, \texttt{technology} is a series of analysis parameters concatenated by the plus sign, and a default value of \texttt{minimum-packet-size}. 

\begin{lstlisting}[language=XML,caption={Example of general network information. Contains \texttt{name}, \texttt{technology} used, and default values for other elements.},label={lst: xml network}]
<network name="demo" technology="FIFO+IS" 
    minimum-packet-size="4B"/>
\end{lstlisting}

The attribute \texttt{technology} takes the following values. More may be found from~\cite{thoma2022analyse}.
\begin{itemize}[leftmargin=1em]
    \item \texttt{FIFO}: FIFO multiplexing. It can be \texttt{ARBITRARY} for arbitrary multiplexing or left blank for tool default.
    \item \texttt{IS}: Input shaping. Consider the shaping effect.
    \item \texttt{PK}: Packetizer.
    \item \texttt{CEIL}: Fix precision when calculating network with cyclic dependency (used only in xTFA.)
\end{itemize}
One can also define some default values that possibly appear in other elements. For example, a \texttt{minimum-packet-size} is usually defined as an attribute of a \texttt{flow} element, while this value defined in the \texttt{network} element will be used as the default value if it's not defined in a \texttt{flow} element.

Second, the servers of the network can be defined as either a \texttt{station} or a \texttt{switch}, as shown in Listing~\ref{lst: xml server}. Although they are very different physically, in our tools they both mean data processing units or possible sources/sinks of a data flow.

\begin{lstlisting}[language=XML,caption={Stations and switches. Name and possibly the default values to all its ports.},label={lst: xml server}]
<station name="src0"/>
<station name="src1"/>
<station name="src2"/>
<switch name="s0" service-latency="10us"
    service-rate="4Mbps"/>
<switch name="s1" service-latency="10us" 
    service-rate="4Mbps"/>
<station name="sink0"/>
<station name="sink1"/>
\end{lstlisting}

Both a station and a switch represent a physical node. The name of each node will be used to define flow paths and links. The \texttt{service-latency} and \texttt{service-rate} define a rate-latency service curve. The service parameters defined at this level serve as default values for all the links attached as outputs of this node.

Third, one must connect physical nodes with \texttt{link}s, as shown in Listing \ref{lst: xml link}. Saihu considers output ports as processing units, so the physical link \texttt{from} a physical node \texttt{to} another node has to be defined, along with the input/output ports used by the link. For example, the link \texttt{lk:s0-s1} connects from the output port \texttt{o0} of switch \texttt{s0} to the input port \texttt{i0} of switch \texttt{s1}.

\begin{lstlisting}[language=XML,caption={Links connecting ports.},label={lst: xml link}]
<link name="lk:src0-s0" from="src0" to="s0" 
    fromPort="o0" toPort="i0"/>
<link name="lk:src1-s0" from="src1" to="s0"
    fromPort="o0" toPort="i1"/>
<link name="lk:src2-s0" from="src2" to="s1" 
    fromPort="o0" toPort="i1"/>
<link name="lk:s0-s1" from="s0" to="s1"
    fromPort="o0" toPort="i0" 
    transmission-capacity="10Mbps"/>
<link name="lk:s1-sink0" from="s1" to="sink0" 
    fromPort="o0" toPort="i0" 
    transmission-capacity="10Mbps"/>
<link name="lk:s1-sink1" from="s1" to="sink1" 
    fromPort="o1" toPort="i0" 
    transmission-capacity="10Mbps"/>
\end{lstlisting}

If the service of an output port needs to be considered in an analysis, one must define the service curve at the link that is directly attached to the output port. The \texttt{transmission-capacity} of the link can also be specified to consider line shaping. If no values are defined, the system tries to apply the default values defined at the upper levels, i.e. \texttt{switch/station} and \texttt{network}. Furthermore, if no values are found across all levels, the link is considered a dummy one and the output port attached to it will not be considered.

Finally, one must define \texttt{flow}s for the network as shown in Listing \ref{lst: xml flow}. Each flow is defined by a \texttt{flow} element. The paths of a flow are defined by \texttt{target} elements, where each node it traverses is listed as \texttt{path} elements with its \texttt{node} attribute indicating the name of the physical node. In this format, multicast of a flow is possible by defining multiple \texttt{target} elements within the same flow.

\begin{lstlisting}[language=XML,caption={Flows. Must have a name and its arrival curve parameters along with its paths.},label={lst: xml flow}]
<flow name="f0" arrival-curve="leaky-bucket" 
    lb-burst="10B" lb-rate="10kbps" 
    maximum-packet-size="50B" source="src0">
    <target>
        <path node="s0"/>
        <path node="s1"/>
        <path node="sink0"/>
    </target>
</flow>
<flow name="f1" arrival-curve="leaky-bucket" 
    lb-burst="10B" lb-rate="10kbps" 
    maximum-packet-size="50B" source="src1">
    <target>
        <path node="s0"/>
        <path node="s1"/>
        <path node="sink1"/>
    </target>
</flow>
<flow name="f2" arrival-curve="leaky-bucket" 
    lb-burst="10B" lb-rate="10kbps" 
    maximum-packet-size="50B" source="src2">
    <target>
        <path node="s1"/>
        <path node="sink0"/>
    </target>
</flow>
\end{lstlisting}

A flow element must have \texttt{name} and the arrival curve specified as its attributes. The keywords \texttt{arrival-curve}, \texttt{lb-burst} and \texttt{lb-rate} define a leaky-bucket curve at the source of the flow. Other parameters like the \texttt{maximum-packet-size} and \texttt{minimum-packet-size} can be also defined to consider packetization. Furthermore, as the definition represents a physical network, each flow must have a data \texttt{source} that is an actual physical node. All the output ports involved in its path, including the output port of the source, will be analyzed by Saihu.


\subsubsection{Option 2: Defining Output Port Network in JSON}
While the XML file syntax is provided by xTFA, we design this JSON format ourselves in order to write an output port network in a concise way.
The file should at least contains 3 kinds of information: \textbf{General network information}, \textbf{Servers}, and \textbf{Flows}. Let's take the implementation of Figure \ref{fig: output port network} as an example. First, all entries must be enclosed as a single JSON object (one \{\} to enclose all attributes.)

A \texttt{network} object is required to define general network information but only the \texttt{name} attribute is necessary. An example is shown in Listing \ref{lst: json network}.


\begin{lstlisting}[language=json,caption={Network information. Contains some general information and default values or units used throughout the file.},label={lst: json network}]
"network": {
    "name": "demo",
    "packetizer": false,
    "multiplexing": "FIFO",
    "analysis_option": ["IS"],
    "time_unit": "us",
    "data_unit": "B",
    "rate_unit": "Mbps",
    "min_packet_length": "4B"
}
\end{lstlisting}
\lstsetblack

The 3 keywords \texttt{packetizer}, \texttt{multiplexing}, and \texttt{analysis\_option} are unique to the \texttt{network} object. \texttt{packetizer} is equivalent to the keyword \texttt{PK} in XML file; \texttt{multiplexing} can be either \texttt{FIFO} or \texttt{ARBITRARY}; and \texttt{analysis\_option} takes other keywords defined in \texttt{technology} mentioned in Section \ref{sec: physical network xml}.

Except for the network options, default values for servers and flows can also be defined at the network level. In the above example, we set the default time/data/rate units to be microsecond/byte/megabits-per-second across the file as well as the minimum packet length being 4 bytes.

Second, we need to define the \texttt{servers} for the network. Some may argue the term \textit{server} instead of \textit{output port} as we discussed in Section~\ref{sec: output port network}. The term \textit{server} is a general term for a processing unit, and one can treat it as a black box that provides service.

The \texttt{servers} is presented as a JSON array, each object in this array is a server. Each server must at least have a \texttt{name}, and its service curve can be missing only when there exists a default value in \texttt{network} attribute. 
The parameters can be expressed in either a \textit{string} or a \textit{number}. A string is written as a number followed by a unit. For example, \texttt{"10us"} means 10 microseconds, and \texttt{"50Mbps"} means 50 megabits per second. If it's directly written as a number, the unit defined in the closest level is used. For example, the time unit defined in server \texttt{s1-o0} is microsecond, so the latency 10 is read as 10 microseconds. 

The object \texttt{service\_curve} takes multiple rate-latency curves and uses the maximum among all these curves as its service curve. Rates and latencies are written as arrays, each pair of rate and latency values with the same index is a rate-latency curve. For example, in server \texttt{s0-o0}, the service curve has 2 segments defined by 2 rate-latency curves, one with a latency of 10 microseconds and a rate of 4 megabits per second, and the other with a latency of 1000 microseconds and a rate of 50 megabits per second.

Notice that in an output port network definition, we don't manually define links. The topology of the network is considered to be the \textit{graph induced by flows}, i.e. a connection from server \textit{A} to \textit{B} exists only when there is at least one flow travels through \textit{B} from \textit{A}. Therefore, the transmission capacity of the link attached to an output port is directly defined on a server with the keyword \texttt{capacity}.

\begin{lstlisting}[language=json,caption={Servers. A list that contains many servers, each with name and service parameters.},label={lst: json server}]
"servers": [
    {
        "name": "s0-o0",
        "service_curve": {
            "latencies": ["10us", 1000],
            "rates": [4, "50Mbps"]
        },
        "capacity": 100
    },
    {
        "name": "s1-o0",
        "service_curve": {
            "latencies": [10, "1ms"],
            "rates": [4, 50]
        },
        "capacity": 100,
        "time_unit": "us"
    },
    {
        "name": "s1-o1",
        "service_curve": {
            "latencies": [10],
            "rates": ["4Mbps"]
        },
        "capacity": 100
    }
]
\end{lstlisting}
\lstsetblack

Finally, the \texttt{flows} are defined in a similar manner as servers as shown in Listing~\ref{lst: json flow}. Each object must have at least a \texttt{name} and a \texttt{path}. A path is represented as an array of server names, and the order in the list represents the sequence that the flow visits.

The representation of values and units is the same as servers, either being a string of a number with units, or a number that uses the default unit.
The arrival curve at the source of a flow is defined by multiple token-bucket curves and taken as the minimum among all these curves. Similar to the service curve of a server, each pair of a burst and a rate value represent a token-bucket curve. For example, the arrival curve of \texttt{f0} has one token-bucket curve of burst 10 bytes and rate 10 kilobits per second, and the other curve of burst 2 kilobytes and rate 0.5 megabits per second.

\begin{lstlisting}[language=json,caption={Flows. A list contains many flows. Each flow contains name, path, and parameters of the arrival data.},label={lst: json flow}]
"flows": [
    {
        "name": "f0",
        "path": ["s0-o0", "s1-o0"],
        "arrival_curve": {
            "bursts": [10, "2kB"],
            "rates": ["10kbps", 0.5]
        },
        "max_packet_length": 50,
        "rate_unit": "kbps"
    },
    {
        "name": "f1",
        "path": ["s0-o0", "s1-o1"],
        "arrival_curve": {
            "bursts": ["10B"],
            "rates": ["10kbps"]
        },
        "max_packet_length": 50
    },
    {
        "name": "f2",
        "path": ["s1-o0"],
        "arrival_curve": {
            "bursts": [10],
            "rates": ["10kbps"]
        },
        "max_packet_length": "50B",
        "min_packet_length": "4B"
    }
]
\end{lstlisting}
\lstsetblack

\subsection{Tool Usage}
In this section, we briefly introduce how to use Saihu to execute analyses. One would only need to import one file, i.e. \textit{interface.py}, to use Saihu given that the project is installed correctly. 
The simplest way to use Saihu is shown in Listing~\ref{lst: simple example}. Once a network description file is available as either an XML or a JSON file, one can execute the following example to do the analysis.

\begin{lstlisting}[style=pythonstyling,caption={Simple example to use Saihu.},label={lst: simple example}]
from interface import TSN_Analyzer
analyzer = TSN_Analyzer("demo.json")
analyzer.analyze_all()
analyzer.export("demo")
\end{lstlisting}

The basic procedure to use Saihu is as follows: 1. initialize the analyzer with a target network description file; 2. execute the analysis with some tools; 3. export the results into reports. 

The supported tools and methods are listed in Figure~\ref{fig: supported methods}. 
To switch between different tools, one uses different functions with names like \texttt{analyze\_xxx}, where \texttt{analyze\_all} means to use all the available tools. To switch methods, one gives different input arguments to each analysis function. An example is provided in Listing~\ref{lst: switch tool and method}. One can execute multiple analyses and all the results will be stored in the internal buffer of the analyzer until the analyzer exports them into reports. 
% The default setting is to try to execute \textit{TFA, SFA} and \textit{PLP}.

\begin{lstlisting}[style=pythonstyling,caption={Execute different tools and methods.},label={lst: switch tool and method}]
analyzer.analyze_dnc()
analyzer.analyze_xtfa("TFA")
analyzer.analyze_panco(methods=["SFA", "PLP"])
\end{lstlisting}


\subsection{Analysis Reports}
\label{sec: analysis reports}

\begin{figure*}[tbh]
\centering
\begin{subfigure}[b]{0.3\textwidth}
    \centering
    \includegraphics[width=\linewidth]{e2e_delay.png}
    \caption{Flow end-to-end delay}
    \label{fig: e2e delay}
\end{subfigure}
\hfill
\begin{subfigure}[b]{0.3\textwidth}
    \centering
    \includegraphics[width=\linewidth]{server_delay.png}
    \caption{Server delay}
    \label{fig: server delay}
\end{subfigure}
\hfill
\begin{subfigure}[b]{0.3\textwidth}
    \centering
    \includegraphics[width=0.9\linewidth]{exec_time.png}
    \caption{Execution time}
    \label{fig: exec time}
\end{subfigure}
\hfill
\begin{subfigure}[b]{0.3\textwidth}
    \centering
    \includegraphics[width=0.8\linewidth]{report_topo.png}
    \caption{Network topology}
    \label{fig: network topology}
\end{subfigure}
\hfill
\begin{subfigure}[b]{0.3\textwidth}
    \centering
    \includegraphics[width=0.6\linewidth]{report_path.png}
    \caption{Flow paths}
    \label{fig: flow path}
\end{subfigure}
\hfill
\begin{subfigure}[b]{0.3\textwidth}
    \centering
    \includegraphics[width=0.9\linewidth]{report_util.png}
    \caption{Link utilization}
    \label{fig: link utilization}
\end{subfigure}
\caption{Human-friendly report. It is written as a Markdown file. The analysis results include the flow end-to-end delay, server delay, and execution time. The values are listed in tables for each tool and method as shown in (a)(b)(c). The units are adjusted accordingly. It also contains input information like the network topology, paths of flows, and link utilization.}
\label{fig: human friendly report}
\end{figure*}

Saihu can generate 2 reports, a \textit{human-friendly report} and a \textit{machine-friendly report}. A human-friendly report is written as a Markdown file that lists all the essential information. An example is shown in Figure~\ref{fig: human friendly report}. The analysis results are listed in 3 sections: per-flow end-to-end delay, per-server delay, and execution time. The delay bounds are presented in tables where each row is a flow or a server, and each column is a method executed by a tool. The last column contains the minimum result obtained in the current round of analysis. The execution time of each method by each tool is also listed for comparison.

Other than the analysis results, the report also contains some information about the user inputs, but in a more formatted manner. They are network topology, flow paths, and link utilization. Network topology is shown as a graph induced by flows. i.e. It's a directed graph where each node is an output port, and an edge from node A to B exists if there's at least one flow traversing from A to B. Flow paths are the same as the network description file, it can serve as a reassurance of user's input. Link utilization is computed node-wise, it is defined as the ratio between the aggregated arrival rate at a node and its service rate. e.g. If 2 flows have arrival rates of $2kbps$ and $3kbps$ respectively filling into a node, which has a service rate of $10kbps$. The link utilization at the node is therefore $(2+3)/10=0.5$.

A \textit{machine-friendly report} stores only the execution outputs, namely the per-flow end-to-end delay, per-server delay, and execution time. It's written in JSON format for easy parsing from other programs. An example is presented in Listing \ref{lst: machine friendly report}.
\begin{lstlisting}[language=json,caption={Machine-friendly report. The flow end-to-end delays, server delays, and execution time are listed in pure numbers. The units these numbers use are also listed as one entry.},label={lst: machine friendly report}]
{
    "name": "demo",
    "flow_e2e_delay": {
        "f0": {
            "Panco_TFA": 100.12500000000001,
            "Panco_PLP": 80.05,
            "DNC_TFA": 100.00375,
            "xTFA_TFA": 99.32394489448944
        },
...
    "server_delay": {
        "s0-o0": {
            "Panco_TFA": 50.0,
            "DNC_TFA": 50.0,
            "xTFA_TFA": 50.0
        },
...
    "execution_time": {
        "Panco_TFA": 62.70909309387207,
        "Panco_PLP": 243.81709098815918,
        "DNC_TFA": 32.0,
        "xTFA_TFA": 81.46500587463379
    },
    "units": {
        "flow_delay": "us",
        "server_delay": "us",
        "execution_time": "ms"
    }
}
\end{lstlisting}
\lstsetblack

In order to let users parse the information easily, Saihu prints the results in numbers, accompanied by the units used in each section. Note that the human-friendly report always contains only 3 decimal digits according to the smallest value in the table, while there's no such rounding for the machine-friendly report. As a result, one should read the machine-friendly report if they require very precise results.


\subsection{Network Generation}
\label{sec: network generation}
Saihu provides a series of functions to allow users to generate certain types of networks into a network description file. Currently, Saihu supports the generation of interleave tandem, mesh, and ring network. They contain specific topologies and routing rules for the flow paths. Users have the freedom to choose the size of the network (number of servers), the service parameters of servers, and the flow parameters of data arrival. The way these parameters are used within each type of network is specified in the respective sections.

Other than the predetermined routing rules, Saihu also provides a function to generate an arbitrary number of flows with random routing. This is particularly suitable for testing the possible traffic with a specific topology. More details will be shown in Section~\ref{sec: fixed topology random}.

\subsubsection{Interleave Tandem Network}
Suppose we wish to generate a network of $n$ servers, indexed from $0$ to $n-1$.
An interleave tandem network has all its servers chained in a line. One flow $f_0$ goes through all servers from $s_0$ to $s_{n-1}$.  The flow $f_i$ is $s_{i-1} \rightarrow s_{i}$ for $i \in [1,n-1]$. Illustrated by Figure \ref{fig: interleave}. All the flows have identical arrival curves and maximum packet length at the source, defined by function arguments \texttt{burst}, \texttt{arrival\_rate}, and \texttt{max\_packet\_length}. Likewise, All the servers have identical service curves and transmission capacity, defined by \texttt{latency}, \texttt{service\_rate}, and \texttt{capacity}.

\begin{figure}
    \centering
    \includegraphics[width=\linewidth]{images/interleave.png}
    \caption{Interleave tandem network}
    \label{fig: interleave}
\end{figure}

\subsubsection{Ring Network}
A ring network is illustrated in Figure \ref{fig: ring}. There are $n$ flows and $n$ servers. The path of flow $i$ is $s_i \rightarrow s_{i+1} \rightarrow \cdots \rightarrow s_{i+n-1\mod n}$ for $0 \leq i \leq n-1$. A ring network is completely symmetrical with all its flows and servers being identical. All flows are defined by \texttt{burst}, \texttt{arrival\_rate}, and \texttt{max\_packet\_length}. Similarly, all servers are defined by \texttt{latency}, \texttt{service\_rate}, and \texttt{capacity}.

\begin{figure}
    \centering
    \includegraphics[width=0.6\linewidth]{images/ring.png}
    \caption{Ring network}
    \label{fig: ring}
\end{figure}

\subsubsection{Mesh Network}
A mesh network is illustrated in Figure~\ref{fig: mesh}. All flows start from either $s_0$ or $s_1$. The flows go through all $2^{(n-1)/2}$ possible combinations of servers towards the right. e.g. $s_0 \rightarrow s_2 \rightarrow \cdots$ and $s_1 \rightarrow s_2 \rightarrow \cdots$ are both in the network. All servers have the same service curve and capacity except $s_{n-1}$ has the doubled service rate. All flows have identical service curves and maximum packet length.

\begin{figure}
    \centering
    \includegraphics[width=0.9\linewidth]{images/mesh.png}
    \caption{Mesh network. Only parts of the flows from $s_0$ are shown. There is a flow for every possible path from $s_0$ or $s_1$ to $s_{n-1}$.}
    \label{fig: mesh}
\end{figure}

\subsubsection{Fixed-Topology Random Routing Network}
\label{sec: fixed topology random}

It's also possible to randomly generate a network defined as a JSON file with a fixed topology of switches. Users have the freedom to decide the number of flows, the topology of switches, and the service/arrival parameters. Each flow randomly routes from one switch to another without repeating the visited switch. We will show more details and use it as an example in Section~\ref{sec: example}.

% A user provides a fixed number of flows to be generated and a connection table along with service parameters and data arrival parameters, then a possible configuration of the network will be generated. A user can also specify the parameters in a range so that they will be uniformly generated within the range.

% We use Figure \ref{fig: industrial network} as an example in Section \ref{sec: example}.

\subsection{Extension}
\label{sec: extension}

As we showed in Figure~\ref{fig: pipeline}, Saihu uses XML/JSON files as a common input and a general information container class as a common output for all the tools. This means to incorporate more tools into Saihu, one only needs to allow the new tool to parse one of the network description formats and feed the analysis results into the information container class. By doing so, they can keep other parts of Saihu untouched and only need to manage one tool at a time.

Moreover, it's also possible to include more network description formats. Because the two formats Saihu uses currently can convert to each other, one only has to make sure a new format can be converted to and from one of the formats Saihu supports. We believe this approach can help more people contribute to and expand Saihu in the future.
\begin{table}[]
    \centering
        \caption{Zero-shot task performance of \texttt{base/large} models after parameter-efficient training.$LwA$/$DA$ indicates adapter types, corresponding to (rows h/f in Table \ref{tab:ablations}). }
    \label{tab:performance}
    \begin{adjustbox}{max width=\textwidth}
        \begin{tabular}{lccccccccc}
% \hline & \multicolumn{4}{c}{ Pre-training Task } & & \multicolumn{2}{c}{ Zero-Shot Performance } \\
% \cline { 2 - 5 } \cline { 6 - 7 } 
\toprule
% \multicolumn{4}{c}{Components} &  \multicolumn{5}{c}{Flickr30k True 0-shot (1k test set)} \\
\multicolumn{4}{c}{Model (591k Training Pairs)} &  & \multicolumn{2}{c}{Flickr} & & \multicolumn{2}{c}{ImageNet V2} \\
\cmidrule(l{0.5em}r{0.5em}){2-4}  \cmidrule(l{0.5em}r{0.5em}){5-8} \cmidrule(l{0.5em}r{0.5em}){9-10}
& Configuration & \# Trainable & \% Trained &  TR@1 & IR@1 & TR@5 & IR@5 & Acc-1 & Acc-5 \\
\midrule
 % (a) &   LilT-tiny & 736.45 K & 7.37\% & 15.7 & 12.4 & 37.4 & 31.56 & 5.61 & 14.54  \\
 % (c) &   LilT-small & 5.19 M & 10.28\% & 37.6 & 27.38 & 66.9 & 54.96 & 10.92 & 23.99  \\
 (a) &   LilT$_{DA}$-base & 14.65 M & 7.51\% & 47.6 & 34.46 & 74.1 & 64.92 & 12.94 & 28.39  \\
 (b) &   LilT$_{DA}$-large & 25.92 M & 4.06\% & 57.6 & 42.18 & 82.2 & 72.38 & 13.97 & 30.89  \\ \midrule
  (c) &   LilT$_{LwA}$-base & 14.67 M & 7.01\% & 56.8 & 41.7 & 81.1 & 70.74 & 12.18 & 27.78  \\
   (d) &   LilT$_{LwA}$-large & 51.18 M & 7.43\% & 63.5 & 50.7 & 88.5 & 79.14 & 14.05 & 31.31 \\
 \midrule
 % (b) & LiT-tiny & 4.45 M & 44.57\% & 25.0 & 16.62 & 49.2 & 40.06 & 10.06 & 22.15  \\
 % (g) & LiT-small & 28.73 M & 56.98\% & 37.3 & 26.46 & 66.6 & 54.78 & \textbf{15.14} & 29.16  \\
 (e) & LiT-base & 109.28 M & 56.01\% & 44.1 & 29.64 & 72.1 & 59.94 & 15.0 & 29.44 \\
 % (d) & CLIP-tiny & 9.99 M & 100.0\% & 34.6 & 25.1 & 62.0 & 51.8 & 7.46 & 18.43  \\
 % (h) & CLIP-small & 50.42 M & 100.0\% & 50.4 & 38.34 & 76.8 & 66.56 & 10.64 & 24.42  \\
 (f) & CLIP-base & 195.13 M & 100.0\% & 56.1 & 44.3 & 81.7 & 71.98 & 12.29 & 28.44  \\
\bottomrule
% (e) & $\checkmark$ & & $\checkmark$ & & & $-$ & $-$ & & & &\\
% (f) & $\checkmark$ & & & $\checkmark$ & & $-$ & $-$ & & & &\\
% (g) & & $\checkmark$ & $\checkmark$ & & & $-$ & $-$ & & & &\\
% \hline (h) & $\checkmark$ & & $\checkmark$ & $\checkmark$ & & $-$ & $-$ & & & &\\
% (i) & $\checkmark$ & $\checkmark$ & & $\checkmark$ & & $-$ & $-$ & & & &\\
% (j) & $\checkmark$ & $\checkmark$ & $\checkmark$ & & & $-$ & $-$ & & & &\\
% (k) & & $\checkmark$ & $\checkmark$ & $\checkmark$ & & $-$ & $-$ & & & &\\
% (k) & & $\checkmark$ & $\checkmark$ & $\checkmark$ & & $-$ & $-$ & & & &\\
% \hline (l) & $\checkmark$ & $\checkmark$ & $\checkmark$ & $\checkmark$ & & $-$ & $-$ & & & &\\
% \hline
\end{tabular}
    \end{adjustbox}
\end{table}
\section{Impact}
\label{sec: impact}

Worst-case delay analysis tools are essential to make theoretical network studies applicable to real-world problems. With each tool specializing in particular methods, the aggregation of these tools has great potential to find the best delay bound of a network. Saihu makes executing and comparing these analysis tools easy by encapsulating them into a single interface while hiding the complexity within. It turns a previously tedious multi-platform programming task into a straightforward ``write your problem down and press run." By simplifying the user's action to analyze a network, Saihu not only eases the work of existing researchers but also makes these tools accessible to a broader audience by requiring a more approachable level of programming skill. 

Saihu also serves as a platform for potential network analysis tool developers due to its cross-platform design. Since Saihu provides unified input/output formats, one can extend Saihu to include more tools by parsing one of the unified inputs and feeding the analysis results into the unified output class. Future developers can bridge their new tool to Saihu and allow existing Saihu users to access their tool easily without forcing people to learn a new framework. 
\section{Conclusion}\label{sec:conclusion}
In this work, we focus on addressing the fundamental challenge of OOD detection tasks, which is how to fully understand the semantic discrepancy between the ID/OOD samples. We reveal that the key to success in the realistic SCOOD task is to allocate as many ID samples in the unlabeled set correctly as possible. To this end, we propose a novel uncertainty-aware optimal transport scheme that introduces class-specific energy scores as guidance for effective label assignment. Experimental results show that our method achieves better performance than previous state-of-the-art methods on SCOOD benchmarks.

\textbf{Limitations.} In addition to temperature scaling, other techniques such as feature clipping applied in ReAct~\cite{sun2021react} also enhance the performance of energy score, so how to obtain an OOD score that best fits the SCOOD task can be further explored. Moreover, a setting highly related to SCOOD has been proposed in \cite{katz2022training} and formulated as a constrained optimization problem. We will also theoretically analyze these practical OOD settings in our feature work.

% \section*{Acknowledgments}
\textbf{Acknowledgments.} 
This work is supported by National Key R\&D Program of China under Grant 2020AAA0105701, National Natural Science Foundation of China (NSFC) under Grants 61872327, Major Special Science and Technology Project of Anhui, National Natural Science Foundation of China (62033012) and Ant Group through Ant Research Intern Program.


\section*{Declaration of competing interest}
The authors declare that they have no known competing financial interests or personal relationships that could have appeared to influence the work reported in this paper.

% \section*{Acknowledgements}
% \label{}

% Optionally thank people and institutes you need to acknowledge. 
% \section*{References}

%% The Appendices part is started with the command \appendix;
%% appendix sections are then done as normal sections
%% \appendix

%% \section{}
%% \label{}

%% References:
%% If you have bibdatabase file and want bibtex to generate the
%% bibitems, please use
%%
\bibliographystyle{elsarticle-num} 
\bibliography{Ref.bib}

%% else use the following coding to input the bibitems directly in the
%% TeX file.

% \begin{thebibliography}{00}

%% \bibitem{label}
%% Text of bibliographic item

% \bibitem{}

% \end{thebibliography}

% \section*{Illustrative Examples}
% Optional : you may include one explanatory  video that will appear next to your article, in the right hand side panel. (Please upload any video as a single supplementary file with your article. Only one MP4 formatted, with 50MB maximum size, video is possible per article. Recommended video dimensions are 640 x 480 at a maximum of 30 frames / second. Prior to submission please test and validate your .mp4 file at  \url{http://elsevier-apps.sciverse.com/GadgetVideoPodcastPlayerWeb/verification} . This tool will display your video exactly in the same way as it will appear on ScienceDirect. )


% \section*{Required Metadata}
% \label{}

\section*{Current code version}
\label{}

% Ancillary data table required for subversion of the codebase. Kindly replace examples in right column with the correct information about your current code, and leave the left column as it is.

\begin{table}[!h]
\begin{tabular}{|l|p{6.5cm}|p{6.5cm}|}
\hline
\textbf{Nr.} & \textbf{Code metadata description} & \textbf{Information} \\
\hline
C1 & Current code version & v1 \\
\hline
C2 & Permanent link to code/repository used for this code version & {\small \href{https://github.com/adfeel220/Saihu-TSN-Analysis-Tool-Integration}{https://github.com/adfeel220/Saihu-TSN-Analysis-Tool-Integration}} \\
\hline
C3  & Permanent link to Reproducible Capsule & \\
\hline
C4 & Legal Code License   & MIT License \\
\hline
C5 & Code versioning system used & Git \\
\hline
C6 & Software code languages, tools, and services used & Python, Java (by DiscoDNC), lpsolve (by Panco), and CPLEX (by \textit{LUDB} option of DiscoDNC)\\
\hline
C7 & Compilation requirements, operating environments \& dependencies & Python packages: \texttt{xTFA}, \texttt{Panco}, Python packages \texttt{numpy}, \texttt{networkx}, \texttt{matplotlib}, \texttt{mdutils}, and \texttt{pulp}. Java package: \texttt{DiscoDNC}\\
\hline
C8 & If available Link to developer documentation/manual & 
\href{https://github.com/adfeel220/Saihu-TSN-Analysis-Tool-Integration/blob/main/README.md}{https://github.com/adfeel220/Saihu-TSN-Analysis-Tool-Integration/blob/main/README.md}\\
\hline
C9 & Support email for questions & \href{mailto:chun-tso.tsai@epfl.ch}{chun-tso.tsai@epfl.ch}\\
\hline
\end{tabular}
\caption{Code metadata}
\label{} 
\end{table}

% \section*{Current executable software version}
% \label{}

% Ancillary data table required for sub version of the executable software: (x.1, x.2 etc.) kindly replace examples in right column with the correct information about your executables, and leave the left column as it is.

% \begin{table}[!h]
% \begin{tabular}{|l|p{6.5cm}|p{6.5cm}|}
% \hline
% \textbf{Nr.} & \textbf{(Executable) software metadata description} & \textbf{Please fill in this column} \\
% \hline
% S1 & Current software version & For example 1.1, 2.4 etc. \\
% \hline
% S2 & Permanent link to executables of this version  & For example: $https://github.com/combogenomics/$ $DuctApe/releases/tag/DuctApe-0.16.4$ \\
% \hline
% S3  & Permanent link to Reproducible Capsule & \\
% \hline
% S4 & Legal Software License & List one of the approved licenses \\
% \hline
% S5 & Computing platforms/Operating Systems & For example Android, BSD, iOS, Linux, OS X, Microsoft Windows, Unix-like , IBM z/OS, distributed/web based etc. \\
% \hline
% S6 & Installation requirements \& dependencies & \\
% \hline
% S7 & If available, link to user manual - if formally published include a reference to the publication in the reference list & For example: $http://mozart.github.io/documentation/$ \\
% \hline
% S8 & Support email for questions & \\
% \hline
% \end{tabular}
% \caption{Software metadata (optional)}
% \label{} 
% \end{table}

\ifdefined\ArxivVersion
    \clearpage
    \appendix
    \section{Appendix for Proofs}

\paragraph{Proof of Theorem \ref{thm:main}.}

\begin{proof}
\label{proof:main}
Our proof has two steps. In Step 1, we will show that SimCLR is equivalent to minimizing the cross entropy loss defined in Eqn.~(\ref{eqn:cross-entropy}). 
In Step 2, we will show  that minimizing the cross-entropy loss 
is equivalent to spectral clustering on $\bfpi$. 
Combining the two steps together, we have proved our theorem. 

\textbf{Step 1: } SimCLR is equivalent to minimizing the cross entropy loss.

The cross-entropy loss takes expectation over 
$\bfW_\bfX\sim \mathbb{P}(\cdot ; \bfpi)$, 
which means $\bfW_\bfX$ has exactly one non-zero entry in each row $i$. By Lemma~\ref{lem:multinomial}, we know every row $i$ of $\bfW_\bfX$ is independent of other rows. Moreover, 
$\bfW_{\bfX,i}\sim \mathcal{M}(1, \bfpi_i/\sum_j \bfpi_{i,j})=\mathcal{M}(1, \bfpi_i)$, because $\bfpi_i$ itself is a probability distribution.
Similarly, we know $\bfW_\bfZ$ also has the row-independent property by sampling over $\mathbb{P}(\cdot;\bfK_\bfZ)$.
Therefore, by Lemma~\ref{lem:cross_split}, we know Eqn.~(\ref{eqn:cross-entropy}) is equivalent to:
\[
 -\sum_{i=1}^n \mathbb{E}_{\bfW_{\bfX,i}}[\log \mathbb{P}(\bfW_{\bfZ,i}=\bfW_{\bfX,i};\bfK_\bfZ)],
\]

This expression takes expectation over $\bfW_{\bfX,i}$ for the given row $i$. Notice that 
$\bfW_{\bfX,i}$ has exactly one non-zero entry, which equals $1$ (same for $\bfW_{\bfZ,i}$). 
As a result
we expand the above expression to be:
\begin{equation}
 -\sum_{i=1}^n \sum_{j\neq i} \Pr(\bfW_{\bfX,i,j}=1)\log \Pr(\bfW_{\bfZ,i,j}=1).
\label{eqn:detailed-expansion}    
\end{equation}


By Lemma~\ref{lem:multinomial}, $\Pr(\bfW_{\bfZ,i,j}=1)=\bfK_{\bfZ,i,j}/\|\bfK_{\bfZ,i}\|_1$ for $j\neq i$. Recall that $\bfK_\bfZ=(k(\bfZ_i-\bfZ_j))_{(i,j)\in[n]^2}$, which means 
$\bfK_{\bfZ,i,j}/\|\bfK_{\bfZ,i}\|_1=\frac{\exp(-\|\bfZ_i-\bfZ_j\|^2/{2\tau})}{\sum_{k\neq i}
\exp(-\|\bfZ_i-\bfZ_k\|^2/{2\tau})
}$ for $j\neq i$, when $k$ is the Gaussian kernel with variance $\tau$. 

Notice that $\bfZ_i=f(\bfX_i)$, so we know
\begin{equation}
-\log \Pr(\bfW_{\bfZ,i,j}=1)=
-\log \frac{\exp(-\|f(\bfX_i)-f(\bfX_j)\|^2/{2\tau})}{\sum_{k\neq i}
\exp(-\|f(\bfX_i)-f(\bfX_k)\|^2/{2\tau}),
}
\label{eqn:infonce-equivalence}    
\end{equation}


The right hand side is exactly the InfoNCE loss defined in Eqn.~(\ref{eqn:infonce}).
Inserting Eqn.~(\ref{eqn:infonce-equivalence}) into Eqn.~(\ref{eqn:detailed-expansion}), we get the SimCLR algorithm, which first samples augmentation pairs $(i,j)$ with $\Pr(\bfW_{\bfX,i,j}=1)$ for each row $i$, and then optimize the InfoNCE loss. 

\textbf{Step 2: } minimizing the cross entropy loss 
is equivalent to spectral clustering on $\bfpi$.


By Lemma~\ref{lem:convert_to_spectral}, we may further convert the loss to 
\begin{equation}
\label{eqn:main-theorem-repul-attr}
\min_{\bfZ}
-\sum_{(i,j)\in [n]^2} \mathbf{P}_{i,j}
\log k (\bfZ_i-\bfZ_j)+\log \mathbf{R}(\bfZ).
\end{equation}
Since $k$ is the Gaussian kernel, this reduces to \[
\min_\bfZ \mathrm{tr}(\bfZ^\top \mathbf{L}(\bfpi) \bfZ)
+\log \mathbf{R}(\bfZ),
\]

where we use the fact that $\mathbb{E}_{\bfW_\bfX\sim \mathbb{P}(\cdot; \bfpi)}[\mathbf{L}(\bfW_\bfX)]
=\mathbf{L}(\bfpi)
$, because the Laplacian operator is linear and $
\mathbb{E}_{\bfW_\bfX\sim \mathbb{P}(\cdot; \bfpi)}(\bfW_\bfX)=\bfpi
$.
\end{proof}

\paragraph{Proof of Theorem \ref{thm:clip}.}
\begin{proof}
Since $\bfW_\bfX\sim \mathbb{P}(\cdot;\bfpi_{\mathbf{A}, \mathbf{B}})$, we know 
$\bfW_\bfX$ has exactly one non-zero entry in each row, denoting the pair that got sampled. 
A notable difference compared to the previous proof is we now have $n_\mathcal{A}+n_\mathcal{B}$ objects in our graph. CLIP deals with this by taking a mini-batch of size $2N$, 
such that $n_\mathcal{A}=n_\mathcal{B}=N$, and adding the $2N$ InfoNCE losses together. We label the objects in $\mathcal{A}$ as $[n_\mathcal{A}]$, and the objects in $\mathcal{B}$ as $\{n_\mathcal{A}+1, \cdots, n_\mathcal{A}+n_\mathcal{B}\}$. 

Notice that $\bfpi_{\mathbf{A}, \mathbf{B}}$ is a bipartite graph, so the edges of objects in $\mathcal{A}$ will only connect to object in $\mathcal{B}$ and vice versa. We can define the similarity matrix in $\cZ$ as $\bfK_\bfZ$, 
where $\bfK_\bfZ(i, j+n_\mathcal{A})=\bfK_\bfZ(j+n_\mathcal{A},i)= k(\bfZ_i-\bfZ_j)$ for $i\in [n_\mathcal{A}], j\in [n_\mathcal{B}]$, and otherwise we set $\bfK_\bfZ(i,j)=0$. 
The rest is same as the previous proof. 
\end{proof}

\paragraph{Proof of Theorem \ref{thm:exponential}.}

\begin{proof}
\label{proof:exponential}
Since the objective function consists of a linear term combined with an entropy regularization, which is a strongly concave function, the maximization problem is a convex optimization problem. Owing to the implicit constraints provided by the entropy function, the problem is equivalent to having only the equality constraint. We then introduce the Lagrangian multiplier $\lambda$ and obtain the following relaxed problem:

$$
\widetilde{E}(\boldsymbol{\alpha})=\psi_{1}-\sum_{i=1}^n \alpha_{i} \psi_{i}+\tau \sum_{i=1}^n \alpha_{i}\log \alpha_{i}+\lambda\left(\boldsymbol{\alpha}^{\top} \mathbf{1}_n-1\right).
$$

As the relaxed problem is unconstrained, taking the derivative with respect to $\alpha_{i}$ yields

$$
\frac{\partial \widetilde{E}(\boldsymbol{\alpha})}{\partial \alpha_{i}}=-\psi_{i}+\tau\left(\log \alpha_{i}+\alpha_{i} \frac{1}{\alpha_{i}}\right)+\lambda=0.
$$

Solving the above equation implies that $\alpha_{i}$ takes the form
$
\alpha_{i}=\exp \left(\frac{1}{\tau} \psi_{i}\right) \exp \left(\frac{-\lambda}{\tau}-1\right).
$ Since $\alpha_{i}$ lies on the probability simplex, the optimal $\alpha_{i}$ is explicitly given by
$
\alpha^{*}_{i}=\frac{\exp \left(\frac{1}{\tau} \psi_{i}\right)}{\sum_{i^{\prime}=1}^n \exp \left(\frac{1}{\tau} \psi_{i^{\prime}}\right)} .
$ Substituting the optimal point into the objective function, we obtain
$$
\begin{aligned}
E\left(\boldsymbol{\alpha}^*\right)  &=\psi_1-\sum_{i=1}^n \frac{\exp \left(\frac{1}{\tau} \psi_{i}\right)}{\sum_{i^{\prime}=1}^n \exp \left(\frac{1}{\tau} \psi_{i^{\prime}}\right)} \psi_{i}+\tau \sum_{i=1}^n \frac{\exp \left(\frac{1}{\tau} \psi_{i}\right)}{\sum_{i^{\prime}=1}^n \exp \left(\frac{1}{\tau} \psi_{i^{\prime}}\right)}\log \frac{\exp \left(\frac{1}{\tau} \psi_{i}\right)}{\sum_{i^{\prime}=1}^n \exp \left(\frac{1}{\tau} \psi_{i^{\prime}}\right)} \\
& =\psi_1 - \tau \log \left(\sum_{i=1}^n \exp \left(\frac{1}{\tau} \psi_{i}\right)\right).
\end{aligned}
$$
Thus, the Lagrangian dual function is given by
\begin{equation*}
-E\left(\boldsymbol{\alpha}^*\right)= -\tau \log \frac{\exp \left(\frac{1}{\tau} \psi_{1}\right)}{\sum_{i=1}^n \exp \left(\frac{1}{\tau} \psi_{i}\right)}.\qedhere
\end{equation*}
\end{proof}



\section{More on Experiments} \label{section: experiment_details}

\paragraph{CIFAR-10 and CIFAR-100} CIFAR-10 ~\citep{krizhevsky2009learning} and CIFAR-100 ~\citep{krizhevsky2009learning} are well-known classic image classification datasets. Both CIFAR-10 and CIFAR-100 contain a total of 60k $32 \times 32$ labeled images of different classes, with 50k for training and 10k for testing. CIFAR-10 is similar to CIFAR-100, except there are 10 different classes in CIFAR-10 and 100 classes in CIFAR-100.

\paragraph{TinyImageNet} TinyImageNet ~\citep{le2015tiny} is a subset of ImageNet ~\citep{deng2009imagenet}. There are 200 different object classes in TinyImageNet, with 500 training images, 50 validation images, and 50 test images for each class. All the images in TinyImageNet are colored and labeled with a size of $64 \times 64$.

\textbf{Pseudo-code.} Algorithm \ref{alg:Training Procedure} presents the pseudo-code for our empirical training procedure.

\begin{algorithm}[!htbp]
\caption{Training Procedure}
\label{alg:Training Procedure}
\begin{algorithmic}[1]
\REQUIRE trainable encoder network $f$, batch size $N$, augmentation strategy \textit{aug}, loss function $L$ with hyperparameters \textit{args}
\FOR {sampled minibatch ${x_i}_{i=1}^N$}
\FORALL{$i \in { 1, ..., N }$}
\STATE draw two augmentations $t_i = \textit{aug}\left(x_i\right) $, $t_i' = \textit{aug}\left(x_i\right) $
\STATE $z_i = f\left(t_i\right)$, $z_i' = f\left(t_i'\right)$
\ENDFOR
\STATE compute loss $\mathcal{L} = L(N, z, z', \textit{args})$
\STATE update encoder network $f$ to minimize $\mathcal{L}$
\ENDFOR
\STATE \textbf{Return} encoder network $f$
\end{algorithmic}
\end{algorithm}

We also provide the pseudo-code for our core loss function used in the training procedure in Algorithm \ref{alg:Core loss}. The pseudo-code is almost identical to SimCLR's loss function, with the exception of an extra parameter $\gamma$.

\begin{algorithm}[!htbp]
\caption{Core loss function $\mathcal{C}$}
\label{alg:Core loss}
\begin{algorithmic}[1]
\REQUIRE batch size $N$, two encoded minibatches $z_1, z_2$, $\gamma$, temperature $\tau$
\STATE $z = \textit{concat}\left(z_1, z_2\right)$
\FOR {$i \in {1, ..., 2N }, j \in {1, ..., 2N}$ }
\STATE $s_{i,j} = \Vert z_i - z_j \Vert_2^{\gamma}$
\ENDFOR
\STATE \textbf{define} $l(i, j)$ \textbf{as} $l(i, j) = - \log \frac{exp\left(s_{i,j}/\tau \right)}{\sum_{k=1}^{2N} \mathbf{1}{[k \ne i]} exp\left(s{i, j} / \tau \right)} $
\STATE \textbf{Return} $\frac{1}{2N} \sum_{k=1}^N\left[l(i, i+N) + l(i+N, i)\right]$
\end{algorithmic}
\end{algorithm}

Utilizing the core loss function $\mathcal{C}$, we can define all kernel loss functions used in our experiments in Table \ref{table: loss definition}. For all $z_i \in z$ with even dimensions $n$, we define $z_{L_i} = z_i\left[0:n/2\right]$ and $z_{R_i} = z_i\left[n/2:n\right]$.

\begin{table}[ht]
\centering
\begin{tabular}{{@{}l|l@{}}}
Kernel  &  Loss function \\ \midrule
Laplacian & $\mathcal{C}\left(N, z, z', \gamma=1, \tau\right)$\\ \midrule
Sum       & $\lambda * \mathcal{C}\left(N, z, z', \gamma=1, \tau_1\right) + (1-\lambda) * \mathcal{C}\left(N, z, z', \gamma=2, \tau_2\right)$  \\ \midrule
Concatenation Sum&$\lambda * \mathcal{C}\left(N, z_L, z'_L, \gamma=1, \tau_1\right) + (1-\lambda) * \mathcal{C}\left(N, z_R, z'_R, \gamma=2, \tau_2\right)$\\ \midrule
$\gamma = 0.5$ & $\mathcal{C}\left(N, z, z', \gamma=0.5, \tau\right)$          \\ 

\end{tabular}

\caption{Definition of kernel loss functions in our experiments}
\label {table: loss definition}
\end{table}

\textbf{Baselines.} We reproduce the SimCLR algorithm using PyTorch Lightning~\citep{PytorchLightning}.

\textbf{Encoder details.}
The encoder $f$ consists of a backbone network and a projection network. We employ ResNet50~\citep{ResNet} as the backbone and a 2-layer MLP (connected by a batch normalization~\citep{ioffe2015batch} layer and a ReLU \cite{nair2010rectified} layer) with hidden dimensions 2048 and output dimensions 128 (or 256 in the concatenation kernel case).

\textbf{Encoder hyperparameter tuning.}
For each encoder training case, we randomly sample 500 hyperparameter groups (sample details are shown in Table \ref{table: Hyperparameter sample}) and train these samples simultaneously using Ray Tune ~\citep{RayTune}, with the ASHA scheduler~\citep{li2018massively}. Ultimately, the hyperparameter group that maximizes the online validation accuracy (integrated in PyTorch Lightning) within 5000 validation steps is chosen for the given encoder training case.

\begin{table}[ht]
\centering

\begin{tabular}{@{}l|l|l@{}}
\midrule
Hyperparameter  & Sample Range & Sample Strategy \\ \midrule
start learning rate & $\left[10^{-2}, 10\right]$ & log uniform \\ \midrule
$\lambda$       & $\left[0, 1\right]$ & uniform \\ \midrule
$\tau$, $\tau_1$, $\tau_2$ & $\left[0, 1\right]$ & log uniform \\ \midrule
\end{tabular}

\caption{Hyperparameters sample strategy}
\label {table: Hyperparameter sample}
\end{table}

\textbf{Encoder training.} 
We train each encoder using the LARS optimizer~\citep{LARSOptimizer}, LambdaLR Scheduler in PyTorch, momentum 0.9, weight decay $10^{-6}$, batch size 256, and the aforementioned hyperparameters for 400 epochs on a single A-100 GPU.

\textbf{Image transformation.} The image transformation strategy, including augmentation, is identical to the default transformation strategy provided by PyTorch Lightning.

\textbf{Linear evaluation.}
The linear head is trained using the SGD optimizer with a cosine learning rate scheduler, batch size 64, and weight decay $10^{-6}$ for 100 epochs. The learning rate starts at $0.3$ and ends at $0$.

\textbf{Moco Experiments.} We also tested our method based on MoCo~\citep{he2019moco}. The results are summarized in Table \ref{tab:results-moco}. Here we choose ResNet18~\citep{ResNet} as the backbone and set a temperature of $0.1$ as default. For our simple sum kernel, we set $\lambda=0.8$. The results show that our method outperforms the original MoCo method.

\begin{table}[thb]
\centering
\caption{MoCo Experiment Results on CIFAR-10 and CIFAR-100.}
\label{tab:results-moco}
\resizebox{\textwidth}{!}{%
\begin{tabular}{@{}c|ccc|ccc@{}}
\toprule
\multirow{3}{*}{Method} & \multicolumn{3}{c|}{CIFAR-10} & \multicolumn{3}{c}{CIFAR-100} \\ \cmidrule(lr){2-4} \cmidrule(lr){5-7} 
                        & 200 epochs & 400 epochs    & 1000 epochs   & 200 epochs & 400 epochs & 1000 epochs         \\ \midrule
MoCo (repro.)         & $76.41 \pm 0.12$    & $80.01 \pm 0.15$          & $84.45 \pm 0.08$    & $\mathbf{47.02 \pm 0.11}$ & $52.50 \pm 0.07$ & $57.62 \pm 0.15$            \\
\midrule
Laplacian Kernel        & ${78.09 \pm 0.10}$    & $\mathbf{83.85 \pm 0.09}$          & $\mathbf{88.34 \pm 0.16}$    & $46.12 \pm 0.22$   & $53.44 \pm 0.17$ & $59.10 \pm 0.14$        \\
Simple Sum Kernel & $\mathbf{78.12 \pm 0.15}$   & $83.23 \pm 0.18$ & $87.50 \pm 0.20$ & $46.65 \pm 0.06$ & $\mathbf{53.62 \pm 0.19}$ & $\mathbf{59.83 \pm 0.12}$\\
\bottomrule
\end{tabular}
}
\end{table}



\section{More Experiments on Synthetic Data}


Consider a scenario with $n$ clusters, each containing $k$ vertices. Let the probability of vertices $u$ and $v$ from the same cluster belonging to $\bfpi$ be $p$. Conversely, for vertices $u$ and $v$ from different clusters, let the probability of belonging to $\pi$ be $q$. We generate the graph $\bfpi$ randomly, based on $p$ and $q$. We experiment with values of $k=100$ and $n=6$ for ease of visualization, embedding all points in a two-dimensional space. Each vertex's initial position originates from a normal distribution. In each iteration, we sample a subgraph of $\bfpi$ uniformly, ensuring each vertex has an out-degree of $1$. We then optimize the corresponding vectors using InfoNCE loss with an SGD optimizer and iterate until convergence. Our experimental setup consists of an SGD learning rate of $1$, an InfoNCE loss temperature of $0.5$, and a batch size of $50$. We evaluate two scenarios with different $p$ and $q$ values: $p=1$, $q=0$, and $p=0.75$, $q=0.2$. The results of these experiments are visualized in Figure \ref{fig:vis-spectral-cluster}. The obtained embeddings exhibit the hallmark pattern of spectral clustering of graph $\bfpi$.

\begin{figure}[!tb]
\centering
\subfigure{
\includegraphics[width=1\textwidth]{Figures/cluster_pi.png}
\label{fig:vis-cluster}
}
\subfigure{
\includegraphics[width=1\textwidth]{Figures/noised_cluster_pi.png}
\label{fig:vis-noised-cluster}
}
\caption{Visualizations of the optimization process using InfoNCE Loss on the vectors corresponding to $\bfpi$. Points of identical color belong to the same cluster within $\bfpi$. To showcase the internal structure of $\bfpi$, we randomly select 10 vertices from each cluster to display the edge distribution of $\bfpi$.}
\label{fig:vis-spectral-cluster}
\end{figure}


\fi

\end{document}
\endinput
%%
%% End of file `SoftwareX_article_template.tex'.
