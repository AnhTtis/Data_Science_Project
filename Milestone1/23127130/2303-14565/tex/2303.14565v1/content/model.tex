
\begin{figure*}[thb]
    \centering
    \includegraphics[width=0.6\textwidth]{images/device.png}
    \caption{Device model. Without loss of generality, we use the indexing 0 and 1 but this model can have arbitrarily many input/output ports as well as the corresponding parameters.}
    % \hossein{I suggest removing the shapers $\sigma$}}
    \label{fig: device}
\end{figure*}

\section{System Model}
\label{sec: system model}


Consider a device model defined in Figure~\ref{fig: device} as any device in the network, and the devices are connected to each other by several transmission links with their transmission capacity. 

A data flow with its source constrained by an arrival curve $\alpha$ goes into an input port of the device. The input port contains a packetizer, which stores the incoming bits in its buffer and releases them only when the whole packet is received. The effect of the packetizer and the capacity of the transmission link is characterized by a shaping function $\sigma$ \cite[page 49]{ncbook2001leboudec} that captures the transmission capacity of the link and the incoming packet length. 

The switch fabric routes the data flow to different output ports according to the routing table. The scheduler with service curve $\beta$ inside the output port provides a service with a multiplexing rule. It then feeds the data to the next transmission link and the departure curve becomes the arrival curve for the next device.

A network is composed of many devices mentioned above, interconnected by many transmission links. We refer to these devices as \textbf{servers} in the network. A network also has \textbf{flows}. Each flow has an arrival curve from its source and imposes the data across many servers with a fixed path. 
We assume all arrival curves are piece-wise linear concave functions and all service curves are piece-wise linear convex functions for all devices. In addition, we assume the aggregated data arrival rate never exceeds the service rate anywhere in the network.

% Generally, a network must at least be composed of flows and servers. \textbf{Flows} are where data is generated and traveling, and \textbf{Servers} are where data is processed and sent to the next node.

Within the scope of this paper, we consider 2 representations where a network can be defined. Namely the \textit{Physical Network} and the \textit{Output Port Network}. 



\begin{figure*}
\centering
\begin{subfigure}[b]{0.63\textwidth}
    \centering
    \includegraphics[width=\linewidth]{phy_net.png}
    \caption{
    Example physical network. 3 flows naming as \texttt{f\#} have their respective source and sink stations \texttt{src\#} and \texttt{sink\#}. 2 switches \texttt{s\#} have their input/output ports \texttt{i\#/o\#}. Each flow has its dedicated data path going from a source to a sink.
    }
    \label{fig: physical network}
\end{subfigure}
\hfill
\begin{subfigure}[b]{0.35\textwidth}
    \centering
    \includegraphics[width=0.7\linewidth]{op_net.png}
    \caption{Example output port network. Only consider the output ports as servers.}
    \label{fig: output port network}
\end{subfigure}
\caption{Example physical and output port networks}
\label{fig: physical and op network}
\end{figure*}



\subsection{Physical Network}
% \begin{figure}[tbh]
% \centering
% \includegraphics[width=\linewidth]{phy_net.png}
% \caption{
% Example Physical Network. 3 flows naming as \texttt{f\#} have their respective source and sink stations \texttt{src\#} and \texttt{sink\#}. 2 switches \texttt{s\#} have their input/output ports \texttt{i\#/o\#}. Each flow has its dedicated data path going from a source to a sink.
% }
% \label{fig: physical network}
% \end{figure}

A physical network represents the physical connections between multiple switches and stations, as well as flows that travel through different input and output ports of switches. The idea of a physical network model is to represent the view of a real-world network without requiring too much re-interpretation from the researcher.

An example is given as Figure \ref{fig: physical network}. Suppose we are interested in the delay bounds of server \texttt{s0} and \texttt{s1}. Each flow $i$ is constrained by an arrival curve $\alpha_i$ at the source, it is also characterized by a path that indicates all the physical nodes it travels. Each output port $j$ provides a service curve $\beta_j$, and is attached with a designated link with a shaping function $\sigma_j$. By giving all the necessary parameters to each network component, we can analyze the delay and backlog experienced at each port.


\subsection{Output Port Network}
\label{sec: output port network}

An output port network is an abstraction form to define a network. An example is given as Figure~\ref{fig: output port network}. The idea is to only consider the output ports that have at least one flow that travels through. We assume the resource competition happens mainly at the output ports even if we provide full information as a physical network, so we consider only the used output ports. By doing so, we greatly simplify the complexity to describe a network.

Similar to the physical network, an output port network requires an arrival curve $\alpha_i$ for each flow $i$. The service curve $\beta_j$ as well as the shaper attached to the link $\sigma_j$ of port $j$ are both defined for each node. As we put all the necessary information on output ports only, we can derive the same delay and backlog bounds for each port.

\subsection{Conversion of the 2 Formats}
Although we can obtain the bounds for each output port in both formats, this doesn't mean that the two formats are equivalent. A physical network can map to a unique expression of an output port network, but not conversely since most of the original information is lost. 
When Saihu converts an output port network to a physical network, we use the assumption that each output port is attached to a unique switch. Although it's not a one-to-one mapping, but any possible mapping provides a valid delay bound for our tools. As a result, we choose the simplest assumption to keep the conversion easy.
% This may not be correct all the time but since the information about switches is lost in the output port network, we use this assumption to keep the result correct.

% \begin{figure}[tbh]
% \centering
% \includegraphics[width=0.7\linewidth]{op_net.png}
% \caption{Example Output Port Network}
% \label{fig: output port network}
% \end{figure}
