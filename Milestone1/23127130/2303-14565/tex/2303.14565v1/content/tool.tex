\section{Included Tools}
\label{sec: included tools}

Saihu currently includes 3 tools, namely xTFA, DNC, and Panco. A summary of the supported method and tool pairs is listed in Figure~\ref{fig: supported methods}.

\begin{figure}[!thb]
    \centering
    \begin{tabular}{|c|c|c|c|}
        \hline
        Method\textbackslash Tool & DNC & xTFA & Panco \\
        \hline\hline
        TFA & V & V & V \\
        SFA & V &   & V \\
        PLP &   &   & V \\
        ELP &   &   & V \\
        PMOO& V &   &   \\
        TMA & V &   &   \\
        \hline
    \end{tabular}
    \caption{Supported methods for each tool. A check ``V'' on it means the tool supports the corresponding method.}
    \label{fig: supported methods}
\end{figure}

\textbf{xTFA} \cite{thoma2022analyse}:
xTFA is short for \textit{experimental modular TFA}, which is developed in Python and supports a more advanced TFA (Total Flow Analysis). It takes an XML file as its input for network description, which we will discuss in more detail in Section \ref{sec: physical network xml}. xTFA supports analyzing networks with cyclic dependency and multicast flows, i.e. a flow having multiple paths or potential splits. In other tools, a multicast flow will be treated as separated flows with the same arrival bounds.

\textbf{DNC}~\cite{bondorf2014discodnc}:
DNC is developed in Java and supports TFA, SFA (Separate Flow Analysis), PMOO (Pay Multiplexing Only Once)~\cite{bondorf2014discodnc}, and TMA (Tandem Matching Analysis)~\cite{scheffler2021fifo}. There's no specific input description file for DNC, one has to define the network as a Java script if they use DNC directly. Saihu uses the information from an output port network to create a network in DNC syntax internally. Moreover, with DNC, one cannot manually set shaping with FIFO multiplexing but only with arbitrary multiplexing. Also, no analysis methods in DNC are capable of solving networks with cyclic dependency.

\textbf{Panco}~\cite{bouillard2022tradeoff}:
Panco is developed in Python and supports TFA, SFA, PLP (Polynomial size Linear Program), and ELP (Exponential size Linear Program). Since all its methods are implemented as linear programs, it requires \textit{lpsolve}~\cite{lpsolve}. Same as DNC, Panco doesn't have a specific input description file, Saihu internally creates the network in Panco syntax from the information of an output port network. All the methods of Panco except ELP support networks with cyclic dependency. 
