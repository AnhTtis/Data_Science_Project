
\section{Example}
\label{sec: example}

\begin{figure}
    \centering
    \includegraphics[width=0.8\linewidth]{images/industrial_net.png}
    \caption{Industrial network \cite{tabatabaee2021deficit}}
    \label{fig: industrial network}
\end{figure}


\begin{figure*}
\centering
\begin{subfigure}[b]{0.3\textwidth}
    \centering
    \includegraphics[width=\linewidth]{example_flow_delay.png}
    \caption{Flow end-to-end delay}
    \label{fig: example e2e delay}
\end{subfigure}
\hfill
\begin{subfigure}[b]{0.3\textwidth}
    \centering
    \includegraphics[width=\linewidth]{example_server_delay.png}
    \caption{Server delay}
    \label{fig: example server delay}
\end{subfigure}
\hfill
\begin{subfigure}[b]{0.3\textwidth}
    \centering
    \includegraphics[width=0.7\linewidth]{example_time.png}
    \caption{Execution time}
    \label{fig: example time}
\end{subfigure}
\hfill
\begin{subfigure}[b]{0.3\textwidth}
    \centering
    \includegraphics[width=0.8\linewidth]{example_topo.png}
    \caption{Network topology}
    \label{fig: example topology}
\end{subfigure}
\hfill
\begin{subfigure}[b]{0.3\textwidth}
    \centering
    \includegraphics[width=\linewidth]{example_path.png}
    \caption{Flow paths}
    \label{fig: example path}
\end{subfigure}
\hfill
\begin{subfigure}[b]{0.3\textwidth}
    \centering
    \includegraphics[width=0.9\linewidth]{example_util.png}
    \caption{Link utilization}
    \label{fig: example link util}
\end{subfigure}
\caption{Human-friendly report from the execution of Listing~\ref{lst: example generate}. We only list the top 5 entries of each section due to limited space in this paper.}
\label{fig: example report}
\end{figure*}

Since we have demonstrated writing a network description file and a small example in Section~\ref{sec: network description file}, we present an example where we generate a network based on the topology shown in Figure~\ref{fig: industrial network} for example. This network topology is provided by Airbus as a test configuration~\cite{tabatabaee2021deficit}. We demonstrate an example code for network generation, analyzing, and result exporting is shown in Listing~\ref{lst: example generate}.


\begin{lstlisting}[style=pythonstyling,caption={Example code with network generation},label={lst: example generate}]
# To use Saihu interface / network generation
from interface import TSN_Analyzer
from src.netscript.net_gen \
    import generate_fix_topology_network
# Switch connection: a switch written as `key' 
#    can route directly to its `values'
connections = {
    "S1": ["S2", "S3", "S8"],
    "S2": ["S1", "S4", "S8"],
    "S3": ["S1", "S4", "S5", "S7", "S8"],
    "S4": ["S2", "S3", "S6", "S7", "S8"],
    "S5": ["S3", "S6", "S7"],
    "S6": ["S4", "S5", "S7"],
    "S7": ["S3", "S4", "S5", "S6"],
    "S8": ["S1", "S2", "S3", "S4"]
}
# Generate a network with 50 flows, connection
# defined above, and parameters.
# Save as "industry50.json"
generate_fix_topology_network(
    num_flows=50,
    connections=connections,
    burst=("10B", "1024B"),
    arrival_rate=("200bps", "20kbps"),
    max_packet_length="128B",
    latency=("2us", "200ms"),
    service_rate=("1Mbps", "50Mbps"),
    capacity="256Mbps",
    save_dir="industry50.json"
)
# Initialize the tool with the generated network
analyzer = TSN_Analyzer("industry50.json")
# Analyze with specific methods by each tool
analyzer.analyze_xtfa()
analyzer.analyze_panco(methods="PLP")
analyzer.analyze_dnc(methods=["TFA","SFA"])
# Export a json and a markdown report
analyzer.export(
    result_json="ind50_result.json",
    report_md="ind50_result.md")
\end{lstlisting}

First, we give the topology using a dictionary \texttt{connections}. Each key-value pair indicates all the links that can travel from one switch to another. For example, the entry \texttt{"S1": ["S2", "S3", "S8"]} means it's possible to go to \texttt{"S2"}, \texttt{"S3"}, and \texttt{"S8"} from \texttt{S1}.

Second, the arguments of the generation function include the number of flows one wishes to generate given the defined topology. As for the service and data arrival parameters, the function allows the user to define either a range as a tuple or a number. If it's defined as a range, the actual value will be chosen within the range uniformly randomly. For example, the arrival rate of a flow is decided between \texttt{200bps} and \texttt{20kbps} whenever a flow is generated. In case it's a number, the same number is assigned across all servers or flows.

Third, we analyze with xTFA using its default setting (TFA), with Panco using PLP, and with DNC using TFA and SFA. One will see the following message upon execution as shown in Figure~\ref{fig: example execution}. One may notice that although we ask DNC to execute, but Saihu skips it automatically since the generated network contains cyclic dependency.

\begin{figure}
    \centering
    \includegraphics[width=\linewidth]{images/example_execution.png}
    \caption{Terminal message when executing Listing \ref{lst: example generate}}
    \label{fig: example execution}
\end{figure}

Figure~\ref{fig: example report} provides the analysis result from this example execution. Since the parameters are randomly generated, the numerical results differ from each other. Moreover, we found the PLP finds a tighter delay bound for flows with much more computation time.

