% original source code: https://www.overleaf.com/latex/templates/style-and-template-for-preprints-arxiv-bio-arxiv/fxsnsrzpnvwc

\documentclass{article}

% \usepackage[latin1]{inputenc}
\usepackage[british]{babel}
\usepackage[all]{xy}
\usepackage{amscd}
\usepackage{amssymb}
\usepackage{amsthm}
\usepackage{enumitem}
\usepackage{mathrsfs,bbm}
\usepackage{xcolor,graphicx}
\usepackage{graphics}
\usepackage{soul}
\usepackage{comment}
\usepackage[all]{xy}
\usepackage{amscd}
\usepackage{amssymb,amsmath,latexsym}
\usepackage{amsthm}
\usepackage{enumitem}
\usepackage{mathrsfs,bbm}
\usepackage{dsfont}
\usepackage{tikz-cd}
\usepackage[T1]{fontenc}
\usepackage[utf8]{inputenc}  
 %
%%%%%%%%%%%%%%%%%%%%%%%%%%%%%%%%%%
%pagestyle
%%%%%%%%%%%%%%%%%%%%%%%%%%%%%%%%%%
%\pagestyle{plain}
\textwidth=430pt
\headsep=.7cm
\evensidemargin=15pt
\oddsidemargin=15pt
\leftmargin=0cm
\rightmargin=0cm
%%
%%%%%%%%%%%%%%%%%%%%%%%
\newcommand*\fixitem {\item[]%
  \refstepcounter{enumi}\hskip-\leftmargin\labelenumi\hskip\labelsep}
\newtheorem*{mainthm}{Main Theorem}
\newtheorem*{mainthm1}{Theorem}
\newtheorem*{maincor}{Corollary}
\usepackage[colorlinks=true]{hyperref}
\DeclareMathOperator{\Forall}{\forall}
\DeclareMathOperator{\Exists}{\exists}
\DeclareMathOperator{\ord}{ord}
\newcommand{\phiD}{\varphi_D}
\newcommand{\phiDI}{\varphi_{\mathbf{D}_I}}
\newcommand{\phiDIj}{\varphi_{\mathbf{D}_I (j)}}
\newcommand{\phiH}{\varphi_H}
\newcommand{\phiTimes}{\phiD \otimes \phiH}
\newcommand{\phiTimesDI}{\varphi_{\mathbf{D}_I} \otimes \phiH}
\newcommand{\R}{\mathscr{A}}
\newcommand{\X}{\mathscr{X}}
\newcommand{\Xf}{\mathscr{X}_{(k_0 ,i)}[r_0]}
\newcommand{\Xfr}{\mathscr{X}_{(k_0,i)}[r]}
\newcommand{\hotimes}{\widehat{\otimes}}
\newcommand{\C}{\mathbb{C}_p}
\newcommand{\V}{\mathscr{V}}
\newcommand{\B}{\mathscr{B}}
\newcommand{\dualD}{\mathfrak{D}}
\newcommand{\Dg}{\mathbf{D}}
\newcommand{\DD}{\mathcal{D}^0}
\newcommand{\DDg}{\mathcal{D}}
\newcommand{\DV}{\mathcal{D}}
\newcommand{\W}{\mathscr{W}_N}
\newcommand{\Ao}{\mathbf{A}^\circ}
\newcommand{\AoK}{\mathbf{A}^\circ_{\K}}
\newcommand{\AK}{\mathbf{A}_{/\K}}
\newcommand{\OOO}{\mathscr{A}^\circ}
\newcommand{\K}{\mathcal{K}} 
\newcommand{\OK}{\mathcal{O}_{\K}}
\newcommand{\varprojlog}[1]{\underleftarrow{\log\!^{#1}}}
\newcommand{\T}{\mathscr{T}}
\newcommand{\TT}{\mathbf{T}}
\newcommand{\VV}{\mathbf{V}}
\newcommand{\HH}{\mathcal{H}}
\newcommand{\hh}{\mathcal{H}^+}
\newcommand{\HG}[2]{\mathcal{H}_{#1}(#2)}
\newcommand{\hhl}{\mathcal{H}^{+,[l]}}
\newcommand{\hhj}{\mathcal{H}^{+,[j]}}
\newcommand{\hhjj}{\mathcal{H}^{+,[l,l']}}
\newcommand{\GS}{G_{\mathbb{Q},S}}
\newcommand{\Rf}{R_{(k_0 ,i)}[r_0]}
\newcommand{\Rfr}{R_{(k_0 ,i)}[r]}
\newcommand{\parT}{\langle T\rangle}
\newcommand{\Zf}{Z_{(k_0 ,i)}[r_0]}
\newcommand{\Zfr}{\mathscr{Z}_{(k_0 ,i)}[r]}
\newcommand{\ZFf}{\mathscr{Z}_{(k_0 ,i)}[r_0]}
\newcommand{\ZFfr}{\mathscr{Z}_{(k_0 ,i)}[r]}
\newcommand{\ZF}{\mathscr{Z}}

\usepackage{arxiv}

\usepackage[utf8]{inputenc} % allow utf-8 input
\usepackage[T1]{fontenc}    % use 8-bit T1 fonts
\usepackage{hyperref}       % hyperlinks
\usepackage{booktabs}       % professional-quality tables
\usepackage{amsfonts}       % blackboard math symbols
\usepackage{nicefrac}       % compact symbols for 1/2, etc.
\usepackage{microtype}      % microtypography
\usepackage{doi}

% Use lorum ipsum temporary
\usepackage{lipsum}

\newcommand\TBD[1]{\textcolor{red}{TBD: #1}}
\newcommand{\hossein}[1]{\textcolor{blue}{Hossein:#1}}

\usepackage{listings}

\usepackage{color}
\definecolor{gray}{rgb}{0.4,0.4,0.4}
\definecolor{darkblue}{rgb}{0.0,0.0,0.8}
\definecolor{cyan}{rgb}{0.0,0.6,0.6}
\definecolor{orange}{rgb}{0.8,0.4,0.0}
\definecolor{lightgreen}{rgb}{0.0,0.8,0.3}
\definecolor{codegreen}{rgb}{0,0.6,0}
\definecolor{codegray}{rgb}{0.5,0.5,0.5}
\definecolor{codepurple}{rgb}{0.3,0,0.52}
\definecolor{codepink}{rgb}{0.8, 0.35, 0.7}
\definecolor{lightblue}{rgb}{0.0,0.3,0.7}

% define XML
\lstset{
  basicstyle=\small\ttfamily\color{black},
  columns=fullflexible,
  showstringspaces=false,
  commentstyle=\color{gray}\upshape,
  frame=single,
  captionpos=b
}
\lstdefinelanguage{XML}{
    moredelim=[s][\color{cyan}]{\ }{=},
    moredelim=[s][\color{black}]{>}{<},
    morestring=[b]",
    morecomment=[s]{?}{?},
    morecomment=[s]{!--}{--},
    stringstyle=\color{orange},
    identifierstyle=\color{darkblue},
    keywordstyle=\color{cyan},
    morekeywords={xmlns,version,type}
}

% \lstdefinestyle{xml-style}{language=XML, extendedchars=true,  belowcaptionskip=5pt, xleftmargin=1.8em, xrightmargin=0.5em, numbers=left, numberstyle=\small\ttfamily\bf, frame=single, breaklines=true, breakatwhitespace=true, breakindent=0pt, emph={}, emphstyle=\color{red}, basicstyle=\small\ttfamily, columns=fullflexible, showstringspaces=false, commentstyle=\color{gray}\upshape,
%     morestring=[b]",
%     morecomment=[s]{<?}{?>},
%     morecomment=[s]{<!--}{-->},
%     stringstyle=\color{orange},
%     identifierstyle=\color{darkblue},
%     keywordstyle=\color{cyan},
%     morekeywords={xmlns,version,type}
% }


% define json
\newcommand{\jsonkey}{\small\color{darkblue}}
\newcommand{\jsonvalue}{\small\color{orange}}
\newcommand{\jsonnumber}{\small\color{cyan}}
\newcommand{\jsonbool}{\small\color{codegreen}}

% switch used as state variable
\makeatletter
\newif\ifisvalue@json
\newif\ifisarray@jsonArray

\lstdefinelanguage{json}{
    tabsize             = 4,
    showstringspaces    = false,
    keywords            = {false,true},
    alsoletter          = 0123456789.,
    morestring          = [s]{"}{"},
    stringstyle         = \jsonkey\ifisvalue@json\jsonvalue\fi\ifisarray@jsonArray\jsonvalue\fi,
    keywordstyle        = \jsonbool,
    MoreSelectCharTable = \lst@DefSaveDef{`:}\colon@json{\enterMode@json},
    MoreSelectCharTable = \lst@DefSaveDef{`,}\comma@json{\exitMode@json{\comma@json}} \ifisarray@jsonArray\array@json{\enterMode@json}\fi,
    MoreSelectCharTable = \lst@DefSaveDef{`\{}\bracket@json{\exitMode@json{\bracket@json}},
    MoreSelectCharTable = \lst@DefSaveDef{`\{}\bracket@jsonArray{\exitMode@jsonArray{\bracket@jsonArray}},
    MoreSelectCharTable = \lst@DefSaveDef{`[}\lb@jsonArray{\enterMode@jsonArray{\lb@jsonArray}},
    MoreSelectCharTable = \lst@DefSaveDef{`]}\rb@jsonArray{\exitMode@jsonArray{\rb@jsonArray}},
    basicstyle          = \small\ttfamily
}

% enter "value" mode after encountering a colon
\newcommand\enterMode@json{%
    \colon@json%
    \ifnum\lst@mode=\lst@Pmode%
        \global\isvalue@jsontrue%
    \fi
}

% leave "value" mode: either we hit a comma, or the value is a nested object
\newcommand\exitMode@json[1]{#1\global\isvalue@jsonfalse}

\lst@AddToHook{Output}{%
    \ifisvalue@json%
        \ifnum\lst@mode=\lst@Pmode%
            \def\lst@thestyle{\jsonnumber}%
        \fi
    \fi
    %override by keyword style if a keyword is detected!
    \lsthk@DetectKeywords% 
}

% enter "value" mode after encountering a colon
\newcommand\enterMode@jsonArray{%
    \lb@jsonArray%
    \ifnum\lst@mode=\lst@Pmode%
        \global\isarray@jsonArraytrue%
    \fi
}

% leave "value" mode: either we hit a comma, or the value is a nested object
\newcommand\exitMode@jsonArray[1]{#1\global\isarray@jsonArrayfalse}

\lst@AddToHook{Output}{%
    \ifisarray@jsonArray%
        \ifnum\lst@mode=\lst@Pmode%
            \def\lst@thestyle{\jsonnumber}%
        \fi
    \fi
    %override by keyword style if a keyword is detected!
    \lsthk@DetectKeywords% 
}



% define python
\newcommand{\lstsetblack}{
    \exitMode@json{}
    \exitMode@jsonArray{}
}
\lstdefinestyle{pythonstyling}{
language=Python,
keywords={from,import},
basicstyle=\small\ttfamily\color{black},
morekeywords={self},              % Add keywords here
keywordstyle=\color{codepink},
identifierstyle=\color{lightblue},
emph={MyClass,__init__},          % Custom highlighting
emphstyle=\color{darkblue},    % Custom highlighting style
stringstyle=\color{orange},
commentstyle=\color{codegreen},
frame=single,                         % Any extra options here
showstringspaces=false
}


\graphicspath{{images/}}


\title{Saihu\footnotemark: A Common Interface of Worst-Case Delay Analysis Tools for Time-Sensitive Networks}

%\date{September 9, 1985}	% Here you can change the date presented in the paper title
\date{} 					% Or removing it

\author{
	{Chun-Tso Tsai} \\
	School of Computer and Communication Sciences\\
	École Polytechnique Fédérale de Lausanne\\
%	Taipei, Taiwan 10617 \\
	\href{mailto:chun-tso.tsai@epfl.ch}{\texttt{chun-tso.tsai@epfl.ch}}
	%% examples of more authors
\And
	{Seyed Mohammadhossein Tabatabaee} \\
	School of Computer and Communication Sciences\\
	École Polytechnique Fédérale de Lausanne\\
	\href{mailto:hossein.tabatabaee@epfl.ch}{\texttt{hossein.tabatabaee@epfl.ch}}
 \And
	{Stéphan Plassart} \\
	School of Computer and Communication Sciences\\
	École Polytechnique Fédérale de Lausanne\\
	\href{mailto:stephan.plassart@epfl.ch}{\texttt{stephan.plassart@epfl.ch}}
 \And
	{Jean-Yves Le Boudec} \\
	School of Computer and Communication Sciences\\
	École Polytechnique Fédérale de Lausanne\\
	\href{mailto:jean-yves.leboudec@epfl.ch}{\texttt{jean-yves.leboudec@epfl.ch}}
}


% Uncomment to override  the `A preprint' in the header
%\renewcommand{\headeright}{Technical Report}
%\renewcommand{\undertitle}{Technical Report}
%\renewcommand{\shorttitle}{\textit{arXiv} Template}

%%% Add PDF metadata to help others organize their library
%%% Once the PDF is generated, you can check the metadata with
%%% $ pdfinfo template.pdf
\hypersetup{
pdftitle={Saihu: A Common Interface of Worst-Case Delay Analysis Tools for Time-Sensitive Networks},
pdfsubject={SAIHU},
pdfauthor={Chun-Tso Tsai, Seyed Mohammadhossein Tabatabaee, Stéphan Plassart, Jean-Yves Le Boudec},
pdfkeywords={Worst-case Delay Analysis; Network Analysis Interface; Network Calculus; Time-Sensitive Networking},
}

\begin{document}
\twocolumn[
\maketitle
    \begin{abstract}  
   Time-sensitive networks, as in the context of IEEE-TSN and IETF-Detnet, require bounds on worst-case delays. Various network analysis tools compute such bounds, but these tools are based on different methods and provide different valid delay bounds. Hence, it is essential to identify the best, smallest bounds. As of today, users must implement multiple pieces of code with different syntaxes for each tool, as each tool is implemented by different groups and uses different programming languages and syntaxes. This results in a significant amount of mechanical actions from users and being error-prone. In this paper, we present \emph{Saihu}, a Python interface that integrates xTFA (supports TFA), DiscoDNC (supports LUDB, PMOO, SFA), and Panco (supports PLP and ELP), the three most frequently used  worst-case network analysis tools. Saihu provides a general interface that enables defining the networks in a single XML or JSON file and executing all tools simultaneously without any adjustment for individual tools. Saihu also exports analysis results into formatted reports automatically, and it offers automatic network generation for certain types of networks. Therefore, with its straightforward syntax and ease of execution, Saihu reduces the burden on users and makes it a valuable tool for anyone working with time-sensitive networks.  Lastly, we modularize the package to incorporate more tools in the future.

    
        % This paper presents Saihu, a Python interface that integrates multiple network analysis tools for obtaining bounds on worst-case delays. With Saihu, users can access multiple analysis tools and methods using straightforward syntax and easy execution: Saihu integrates three worst-case network analysis tools based on the network calculus framework, namely \textit{xTFA}, \textit{DiscoDNC}, and \textit{Panco}. While previously users are required to implement multiple pieces of code with different syntax for each tool respectively, Saihu provides a general interface for all tools and minimizes the required mechanical actions from users.
        % The main contribution of Saihu is that it allows users to define their networks via a single XML or JSON file and execute all tools at once without adjustment for individual tools. Saihu can also export analysis results into formatted reports automatically. Most importantly, Saihu demands only a very limited amount of commands from users.
        % ; fourth, it allows automatic network generation for certain types of networks; finally, We modularize the package so that it can incorporate more tools in the future.
    \end{abstract}
    % Keywords
    \keywords{Worst-case Delay Analysis \and Network Analysis Interface \and Network Calculus \and Time-Sensitive Networking}
    \vspace{0.6cm}
]

% title footnote specified after make title
\footnotetext{SAIHU stands for ``\textbf{S}uperimposed worst-case delay \textbf{A}nalysis \textbf{I}nterface for \textbf{H}uman-friendly \textbf{U}sage.'' }

\section{Introduction}
\label{sec:introduction}
% \begin{itemize}
%     % Diffusion of FL
%     \item {\st{Diffusion of FL}}
%     % Security threats to FL
%     \item {\st{Security threats to FL with particular focus on model poisoning}}
%     % Limitations of existing countermeasures
%     \item {\st{Current countermeasures (e.g., KRUM) and their limitations}}
%     % Proposed method and its advantages
%     \item {\st{Intuitive description of the proposed method and its difference (i.e., advantages) w.r.t. state of the art}}
%     % Main contributions
%     \item {\st{Summary of the main contributions of this work}}
%     % Paper's structure and organization
%     \item {\st{Paper's structure and organization}}
% \end{itemize}

% Diffusion of FL
Recently, {\em federated learning} (FL) has emerged as the leading paradigm for training distributed, large-scale, and privacy-preserving machine learning (ML) systems~\cite{mcmahan2017googleai,mcmahan2017aistats}. 
The core idea of FL is to allow multiple edge clients to collaboratively train a shared, global model without disclosing their local private training data.
%Specifically, an FL system consists of a central server and many edge clients; 
A typical FL round involves the following steps: {\em(i)} the server randomly picks some clients and sends them the current, global model; {\em(ii)} each selected client locally trains its model with its own private data; then, it sends the resulting local model to the server;\footnote{Whenever we refer to global/local model, we mean global/local model {\em parameters}.} {\em(iii)} the server updates the global model by computing an \emph{aggregation function}, usually the average (FedAvg), on the local models received from clients.
% \begin{enumerate}
%     \item[{\em(i)}] the server sends the current, global model to the clients and appoints some of them for training;
%     \item[{\em(ii)}] each selected client locally trains its copy of the global model with its own private data; then, it sends the resulting local model back to the server;\footnote{Whenever we refer to global/local model, we mean global/local model {\em parameters}.}
%     \item[{\em(iii)}] the server updates the global model by computing an \emph{aggregation function} on the local models received from clients (by default, the average, also referred to as FedAvg~\cite{mcmahan2017aistats}).
% \end{enumerate}
This process goes on until the global model converges. %(e.g., after a certain number of rounds or other similar stopping criteria).
%\\
% The advantages of FL over the traditional, centralized learning paradigm are undoubtedly clear in terms of flexibility/scalability (clients can join/disconnect from the FL network dynamically), network communications (only model weights\footnote{We will use \textit{parameters} and \textit{weights} interchangeably.} are exchanged between clients and server), and privacy (each client's private training data is kept local at the client's end and not uploaded to the server).
\\
% Security threats to FL
%However, the growing adoption of FL also raises security concerns~\cite{costa2022covert}, particularly about its confidentiality, integrity, and availability.
Although its advantages over standard ML, FL also raises security concerns~\cite{costa2022covert}. %, particularly about its confidentiality, integrity, and availability~\cite{costa2022covert}.
% OLD, LONG VERSION
% Indeed, some work deals with privacy leakage that may expose the local data of some clients~\cite{melis2019sp}. 
% A large body of work, instead, investigates attacks that usually aim to detriment the predictive accuracy of the learned global model. For instance, \emph{data poisoning} attacks achieve this goal by letting an adversary pollute the training set of some corrupt FL clients with maliciously crafted examples~\cite{jagielski2018sp}.
% Similarly, in \emph{model poisoning} the attacker attempts to tweak the global model weights~\cite{bhagoji2019pmlr} by directly perturbing the local model's weights of some infected FL clients before these are sent to the central server for aggregation, usually via so-called Byzantine attacks. 
% It turns out that Byzantine model poisoning attacks severely impact standard FedAvg; therefore, more robust aggregation functions must be designed to make FL systems secure.
Here, we focus on \emph{untargeted model poisoning} attacks~\cite{bhagoji2019pmlr}, where an adversary attempts to tweak the global model weights %\footnote{We will use the terms \textit{parameters} and \textit{weights} interchangeably.} 
by directly perturbing the local model's parameters of some infected clients before these are sent to the central server for aggregation.
In doing so, the adversary aims to jeopardize the global model \textit{indiscriminately} at inference time.
Such model poisoning attacks severely impact standard FedAvg; therefore, more robust aggregation functions must be designed to secure FL systems.
\\
% In this paper, we focus on designing a novel robust aggregation scheme at the server's end to contrast the effect of Byzantine model poisoning attacks.
%
% Current countermeasures and their limitations
%Several countermeasures have been proposed in the literature to combat model poisoning attacks on FL systems.
% Some methods use simple statistics more robust than plain average to smooth the impact of malicious updates (e.g., Trimmed Mean and FedMedian~\cite{yin2018icml}). 
% Other defenses implement outlier detection techniques to discard malicious updates from the aggregation performed at the server's end. Those are either based on heuristics (e.g., Krum/Multi-Krum~\cite{blanchard2017nips} and Bulyan~\cite{mhamdi2018pmlr}) or data-driven approaches (e.g., K-means clustering~\cite{shen2016acm} or DnC via spectral analysis~\cite{shejwalkar2021ndss}). 
% Finally, some strategies rely on a centralized ``source of trust'' to spot potential malicious updates (e.g., FLTrust~\cite{cao2020fltrust}).
% Several countermeasures have been proposed in the literature to combat model poisoning attacks on FL systems, i.e., to discard possible malicious local updates from the aggregation performed at the server's end. 
% These techniques range from simple statistics more robust than plain average (e.g., Trimmed Mean and FedMedian~\cite{yin2018icml}) to outlier detection heuristics (e.g., Krum/Multi-Krum~\cite{blanchard2017nips} and Bulyan~\cite{mhamdi2018pmlr}) or data-driven approaches (e.g., spectral analysis via K-means clustering~\cite{shen2016acm} or spectral analysis), or methods based on ``source of trust'' (e.g., FLTrust~\cite{cao2020fltrust}).
% OLD, LONG VERSION
%Several countermeasures have been proposed in the literature to combat Byzantine model poisoning attacks on FL systems.
% Descriptive statistics
% For example, Trimmed Mean and FedMedian aggregate local model updates using more robust statistics than standard average~\cite{yin2018icml}.
%
% % Heuristics for outlier detection
% Many existing Byzantine-resilient strategies implement some outlier detection heuristics to discard the model updates sent by potentially malicious clients from the input of the aggregation function.
% One of the most popular heuristics is Krum~\cite{blanchard2017nips}.
% This strategy tries to mitigate the impact of Byzantine attacks by selecting as a global model the local model with the smallest sum of Euclidean distances to {\em all} the other local models.
% Although powerful, Krum requires the server to know (or, at least, estimate) the number of malicious FL clients upfront, which is generally impossible in a realistic attack scenario. %
% Moreover, Krum may become ineffective for complex, high-dimensional model parameter spaces due to the curse of dimensionality.
% Bulyan~\cite{mhamdi2018pmlr} tries to overcome this issue by combining Krum with a variant of Trimmed Mean.
% % Data-driven outlier detection
% Other strategies use data-driven outlier detection techniques -- e.g., via K-means clustering~\cite{shen2016acm} -- to spot potential malicious local model updates. 
% %For instance, Shen et al. propose to cluster local model updates with K-means and thus identify outliers.
%
% % Other techniques
% As far as the server is concerned, any local model received can be from a potential malicious client. 
% FLTrust~\cite{cao2020fltrust} assumes the server acts as a client, i.e., trains a local model on an additional {\em trustworthy} dataset at the server's end and compares it against all the local models from other clients. 
% This way, the server can rely on some ``source of trust'' when discarding potentially malicious clients.
%\\
% Limitations of existing Byzantine-resilient strategies
Unfortunately, existing defense mechanisms either rely on simple heuristics (e.g., Trimmed Mean and FedMedian by~\cite{yin2018icml}) or need strong and unrealistic assumptions to work effectively (e.g., foreknowledge or estimation of the number of malicious clients in the FL system, as for Krum/Multi-Krum~\cite{blanchard2017nips} and Bulyan~\cite{mhamdi2018pmlr}, which, however, cannot exceed a fixed threshold).
Furthermore, outlier detection methods using K-means clustering~\cite{shen2016acm} or spectral analysis like DnC~\cite{shejwalkar2021ndss} do not directly consider the temporal evolution of local model updates received.
Finally, strategies like FLTrust~\cite{cao2020fltrust} require the server to collect its own dataset and act as a proper client, thereby altering the standard FL protocol.
\\
% OLD, LONG VERSION
% Overall, existing Byzantine-resilient strategies are either simple heuristics (e.g., FedMedian) or, if they are more complex, they rely on strong and unrealistic assumptions to work effectively (e.g., knowing the number of malicious clients in the FL system in advance, as for Krum and alike).
% Furthermore, data-driven outlier detection methods do not consider the temporary evolution of local model updates received (e.g., K-means clustering). 
% Finally, strategies like FLTrust requires the server to collect its own dataset and act as a proper client, thereby altering the standard FL protocol.
%
% Description of the proposed method
This work introduces a novel pre-aggregation \textit{filter} robust to untargeted model poisoning attacks. Notably, this filter $(i)$ operates without requiring prior knowledge or constraints on the number of malicious clients and $(ii)$ inherently integrates temporal dependencies. 
The FL server can employ this filter as a preprocessing step before applying \textit{any} aggregation function, be it standard like FedAvg or robust like Krum or Bulyan.
Specifically, we formulate the problem of identifying corrupted updates as a multidimensional (i.e., matrix-valued) time series anomaly detection task. 
The key idea is that legitimate local updates, resulting from well-calibrated iterative procedures like stochastic gradient descent (SGD) with an appropriate learning rate, show \textit{higher predictability} compared to malicious updates. This hypothesis stems from the fact that the sequence of gradients (thus, model parameters) observed during legitimate training exhibit regular patterns, as validated in Section~\ref{subsec:intuition}. %until convergence. 
%This regularity may be more pronounced for smooth convex loss functions, but it can still be captured within an appropriate time window, even for more complex and convoluted loss surfaces. 
%We provide evidence of this claim in Appendix~B, where we show that the average mutual information (i.e., ``predictability''), calculated over pairs of legitimate model updates sent at different FL rounds, is significantly higher than the corresponding computation for a malicious client.
\\
Inspired by the matrix autoregressive (MAR) framework for multidimensional time series forecasting~\cite{chen2021je}, we propose the FLANDERS ({\em \textbf{F}ederated \textbf{L}earning meets \textbf{AN}omaly \textbf{DE}tection for a \textbf{R}obust and \textbf{S}ecure}) filter.
The main advantages of FLANDERS over existing strategies like FLDetector~\cite{zhao2020multivariate} are its resilience to large-scale attacks, where $50\%$ or more FL participants are hostile, and the capability of working under realistic non-iid scenarios.
We attribute such a capability to two key factors: $(i)$ FLANDERS works without knowing a priori the ratio of corrupted clients, and $(ii)$ it embodies temporal dependencies between intra- and inter-client updates, quickly recognizing local model drifts caused by evil players. Below, we summarize our main contributions:

\begin{itemize}
\item[{\em(i)}]
We provide empirical evidence that the sequence of models sent by legitimate clients is more predictable than those of malicious participants performing untargeted model poisoning attacks.
\\
\item[{\em(ii)}] 
We introduce FLANDERS, the first pre-aggregation filter for FL robust to untargeted model poisoning based on multidimensional time series anomaly detection.
\\
\item[{\em(iii)}] 
We integrate FLANDERS into Flower,\footnote{\scriptsize{\url{https://flower.dev/}}} a popular FL simulation framework for reproducibility.
\\
\item[{\em(iv)}] 
We show that FLANDERS improves the robustness of the existing aggregation methods under multiple settings: different datasets, client's data distribution (non-iid), models, and attack scenarios.
\\
\item[{\em(v)}] 
We publicly release all the implementation code of FLANDERS along with our experiments.\footnote{\scriptsize{\url{https://anonymous.4open.science/r/flanders_exp-7EEB}}}
\end{itemize}

% Paper's structure and organization
The remainder of the paper is structured as follows. %some related work and the current state-of-the-art solutions to security issues that FL entails. 
Section~\ref{sec:background} covers background and preliminaries. 
In Section~\ref{sec:related}, we discuss related work.
Section~\ref{sec:problem} and Section~\ref{sec:method} describe the problem formulation and the method proposed. % to tackle it. 
Section~\ref{sec:experiments} gathers experimental results. %, and Section~\ref{sec:limitations} discusses some limitations of this work.
Finally, we conclude in Section~\ref{sec:conclusion}.
 %discusses the limitations of this work and draws future research directions.
%reports conclusions and draws perspectives for future research directions.

%%%%%%% OLD %%%%%%%
%to overcome the resilience of Byzantine failures in distributed Stochastic Gradient Descent computations. 
% The strength of Krum is its time complexity, which is linear in the gradient dimension. 
% However, the robustness of the approach is guaranteed for gradient-based learning applications only when the majority of the clients are not compromised. 
% Besides, the aggregation mechanism of Krum, as well as that of similar methods, is robust from a coarse-grained perspective and does not provide solutions to errors and perturbations that may occur at inference time.
%A related approach to~\cite{blanchard2017nips} is the work of Su et al.~\cite{su2016dc}. Here, the authors propose an iterated approximate agreement to tackle a multi-layer scenario attacked by Byzantine agents. 
%However, the method works efficiently on the sole discrete context and it is inapplicable to continuous state environments.
%\gabri{Maybe, we should just talk about the main limitations of existing countermeasures without digging into their details (or, we can just mention Krum as this is the most popular one). I will move the description of all these methods to the Related Work section.}
\section{Background on Network Calculus}
\label{sec: background}


\begin{figure*}[tbh]
\centering
\begin{subfigure}[b]{0.3\textwidth}
    \centering
    \includegraphics[width=\linewidth]{images/in-out.png}
    \caption{Arrival and departure data and their relation with delay $d(t)$ and backlog $b(t)$. For a FIFO system, the delay is the horizontal distance between $R(t)$ and $R^*(t)$ but some other multiplexing techniques may shift the data to a later priority, causing a longer delay.}
    \label{fig: data in-out}
\end{subfigure}
\hfill
\begin{subfigure}[b]{0.35\textwidth}
    \centering
    \includegraphics[width=\linewidth]{images/arrival-service.png}
    \caption{Characteristics of an arrival curve and a service curve. From any point of observation, the arriving data never exceeds its arrival curve; the departure data is also never less than the service curve with respect to the data arrival.}
    \label{fig: arrival-service curves}
\end{subfigure}
\hfill
\begin{subfigure}[b]{0.33\textwidth}
    \centering
    \includegraphics[width=\linewidth]{images/bound.png}
    \caption{Delay and backlog bounds of a system. Backlog is the maximum vertical distance between $\alpha(t)$ and $\beta(t)$; FIFO delay is their maximum horizontal distance; but for arbitrary multiplexing, the delay guarantee is when the system clears its buffer, thus it's the intersection of $\alpha(t)$ and $\beta(t)$.}
    \label{fig: system bounds}
\end{subfigure}
\caption{Network calculus framework. We let $R(t)$ and $R^*(t)$ be the arrival and departure data flow of a system; $\alpha(t)$ be the piecewise linear concave arrival curve and $\beta(t)$ be the piecewise linear convex service curve of a system.}
% \hossein{Better to show piece-wise linear concave arrival curve and piece-wise linear convex service curve instead of token-bucket and rate-latency.}}
\end{figure*}

We recall some of the network calculus essentials for a better understanding of the framework used in Saihu. In the following context, we use the following notation: $\mbb{R}^+$ is the set of non-negative real numbers; $[x]_+$ denotes $\max(0, x)$

The data flow is by convention modeled as a left-continuous wide-sense increasing function $R(t): \mbb{R}^+ \mapsto \mbb{R}^+$ with respect to time $t$~\cite{ncbook2001leboudec}. 

A system $\mcal{S}$ receives arrival data described as a cumulative function $R(t)$ and delivers departure data as another cumulative function $R^*(t)$. Figure~\ref{fig: data in-out} illustrates such a system $\mcal{S}$. The benefit of representing a system like this is that we can observe system backlog and delay with such a model. 

\begin{definition}[Backlog and Delay~\cite{ncbook2001leboudec}]
    The backlog of a system at time~$t$ is
    \begin{equation}
        b(t) = R(t) - R^*(t)
    \end{equation}
    
    The virtual delay of a FIFO system at time $t$ is
    \begin{equation}
        d_{FIFO}(t) = \inf \lbp \tau \geq 0 : R(t) \leq R^*(t+\tau) \rbp
    \end{equation}
\end{definition}



The backlog of a system can be viewed as the vertical distance between $R$ and $R^*$. The FIFO (\textit{First-in First-out}) delay is the horizontal distance between $R$ and $R^*$. One may obtain other delay values if the multiplexing technique is not FIFO.

% \begin{figure}
%     \centering
%     \includegraphics[width=0.9\linewidth]{images/in-out.png}
%     \caption{In/out data flow; delay and backlog}
%     \label{fig: data in-out}
% \end{figure}

Since we are interested in the system guarantee instead of a single instance of data flow, we would like to have general bounds to the arrival and departure data flows. Therefore, we define \textit{arrival curve} and \textit{service curve} as the bounds of arrival and departure data flows.

\begin{definition}[Arrival Curve~\cite{ncbook2001leboudec}]
    Given a wide-sense increasing function $\alpha: \mbb{R}^+ \mapsto \mbb{R}^+$, we say that a flow $R(t)$ is $\alpha$-constrained if and only if for all $s \leq t$:
    \begin{equation}
        R(t) - R(s) \leq \alpha(t-s)
    \end{equation}
    We say $R(t)$ has $\alpha$ as an arrival curve.
\end{definition}

\begin{definition}[Service Curve~\cite{ncbook2001leboudec}]
    Given a wide-sense increasing function $\beta: \mbb{R}^+ \mapsto \mbb{R}^+$ and $\beta(0) = 0$. A system $\mcal{S}$ having $R(t)$ and $R^*(t)$ as its arrival and departure flows. We say $\mcal{S}$ offers a service curve $\beta$ if and only if
    \begin{equation}
        R^*(t) \geq (R \otimes \beta)(t) =: \inf_{s \leq t} \lbp R(s) + \beta(t-s) \rbp
    \end{equation}
    where $\otimes$ denotes the min-plus convolution
\end{definition}

Figure~\ref{fig: arrival-service curves} illustrates the arrival and service curves. Any segment of arrival flow $R(t)$ is constrained by arrival curve $\alpha$ and the output curve $R^*(t)$ is always no less than the curve $R\otimes\beta$. As a result, an arrival curve upper bounds the incoming traffic, and a service curve lower bounds the outgoing traffic.

% \begin{figure}
%     \centering
%     \includegraphics[width=\linewidth]{images/arrival-service.png}
%     \caption{Arrival/Service curve}
%     \label{fig: arrival-service curves}
% \end{figure}

We consider 2 special types of curves throughout this paper, \textit{token-bucket} (or sometimes called \textit{leaky-bucket}) curve and \textit{rate-Latency} curve.

\begin{definition}[Token-bucket and Rate-latency~\cite{ncbook2001leboudec}]
    A token-bucket curve $\gamma_{r,b}$ with arrival rate $r$ and burst $b$ is defined as
    \begin{equation}
        \gamma_{r,b}(t) = b + rt
    \end{equation}

    A rate-latency curve $\beta_{R,T}$ with service rate $R$ and latency $T$ is defined as
    \begin{equation}
        \beta_{R,T}(t) = R \lb t - T \rb_+
    \end{equation}
\end{definition}

A token-bucket curve is determined by a burst $b$ and an arrival rate~$r$. Burst represents the maximum possible data volume that can arrive simultaneously, and arrival rate represents the maximum long-term data rate~\cite{bouillard2022tradeoff}.
A rate-latency curve is determined by a latency~$T$ and a service rate~$R$. Latency represents the time a server needs before starting to process the incoming data, and service rate represents the minimum rate to process data after the initial latency.

With the help of arrival and service curves, we can derive delay and backlog bounds for a system $\mcal{S}$ illustrated in Figure~\ref{fig: system bounds}. Suppose a system $\mcal{S}$ has arrival curve $\alpha$ and service curve~$\beta$, its worst-case backlog $b^*$ is the maximum vertical distance between~$\alpha$ and~$\beta$. Similarly, depending on the multiplexing technique applied to the system, its worst-case delay bound $d^*$ is the maximum horizontal distance between $\alpha$ and $\beta$ if $\mcal{S}$ is a FIFO system. If we don't have any information about its multiplexing technique, referred to as arbitrary multiplexing, the best we can say is that when $\alpha$ and $\beta$ intersect each other, where all data has been delivered out of the system. Consequently, the worst-case delay bound for arbitrary multiplexing is the time required for $\mcal{S}$ to clear its buffer.

% \begin{figure}
%     \centering
%     \includegraphics[width=\linewidth]{images/bound.png}
%     \caption{System delay/backlog bounds}
%     \label{fig: system bounds}
% \end{figure}

While a service curve captures the slowest possible output speed of a system, a link's transmission capacity limits the speed as well. Hence, we model this phenomenon using a \textit{greedy shaper} with a sub-additive function $\sigma: \mbb{R}^+ \mapsto \mbb{R}^+$ concatenated with a server. We consider a concatenation as shown in Figure \ref{fig: system}. By convention we assume $\sigma(0) = 0$ and $\beta(t) \leq \sigma(t), \forall t \in \mbb{R}^+$, meaning that the buffer is cleared at the beginning and the service never exceed its physical limitation. With the above definition, such greedy shaper conserves the service provided by the system due to theorem \ref{thm: shaping}.

\begin{figure}[thb]
    \centering
    \includegraphics[width=0.7\linewidth]{images/system.png}
    \caption{Shaping of departure data. A flow that has an arrival curve $\alpha$ feeds into a server with an arrival data flow $R(t)$. The server having service curve $\beta$ takes $R(t)$ and gives a departure data flow $R^*(t)$ to a shaper with shaping function $\sigma$. The shaper takes $R^*(t)$ and shape the data flow as another departure $D(t)$.}
    \label{fig: system}
\end{figure}


\begin{theorem}[Shaping conserves service \cite{ncbook2001leboudec}]
\label{thm: shaping}
Following the system shown in Figure \ref{fig: system}, we have
\begin{equation}
     D = R^* \otimes \sigma \geq \lp R \otimes \beta \rp \otimes \sigma = R \otimes \lp \beta \otimes \sigma \rp = R \otimes \beta
\end{equation}
\end{theorem}

In the following context, we model the shaping function $\sigma$ as a token-bucket curve $\gamma_{C,L}$ with transmission capacity $C$ and the packet size $L$ to capture the link capacity and packetization~\cite{bouillard2022tradeoff}.

\section{Proposed Framework: {\ourmodel}}
\label{model}


In this section, we introduce a novel self-supervised co-training framework {\ourmodel}.
The proposed framework is illustrated in Figure~\ref{fig:intro_model} and works in three phases.
Phase one automatically generates two sets of pseudo labels.
We use a combination of off-the-shelf pre-trained POS and NER taggers, knowledge graph, and GPT-2 scorer for generating the first set of pseudo labels automatically without any hand-crafted rules for matching the slot values.
The other set of pseudo labels is acquired through a zero-shot slot filling model~\cite{liu2020coach}, trained on the out-of-domain dataset.
It is critical to emphasize that both sets of labels are noisy and incomplete which poses serious challenges to training effective models for the task of open-domain slot filling.
Phase two fine-tunes the pre-trained BERT to the slot filling task that effectively transfers the knowledge from the pre-trained language model~(LM) to overcome the issue of label incompleteness to some extent. 
Further, we employ the early stopping technique to minimize the noise in the labels.
The output of this phase is two BERT models that can generate soft labels for self-supervision during co-training in phase three.
Phase three leverages the fine-tuned models and further trains them in an iterative fashion.
Specifically, the proposed peer training approach facilitates high-confidence soft label selection for the other peer to perform training. This phase progressively reduces the noise in the labels and enables effective model fitting. 



\subsection{Phase One: Automatic Label Generation}
To acquire the first set of labels, we perform the following steps.
First of all, off-the-shelf trained POS and NER taggers are used to predict initial estimates of the slot values irrespective of the slot types. Then, the type information of the slot values is queried from the KG and the slot value is tagged for the most appropriate slot in the target domain.
This approach, however, produces low recall. 
To expand the candidate slot values, we generate n-grams of the natural language text and employ a partial matching scheme to query the KG for type information (e.g., \myspecial{Jason} \myspecial{Aldean} = \myspecial{American} \myspecial{singer}) of the n-grams if the entry exists.
This process generates multiple overlapping hypotheses about the slot values.
We replace a span of text that corresponds to a slot value by its type information and a GPT-2 based scorer (see Section~\ref{sec:nlpmodels}) is used to select the best candidate based on the fluency of the text.
Naturally, if a token (or span of tokens) is replaced by its type, the sentence should score higher as compared to the case where an inappropriate substitution is performed. 
We select the best hypothesis if the score is greater than the threshold.
Intuitively, the candidate selection threshold can automatically be searched based on a small validation set from the target domain, making the label generation process fully automatic. 
The other set of noisy labels is acquired by the zero-shot slot filling model~\cite{liu2020coach} that has been trained using an out-of-domain dataset. It is important to highlight that the zero-shot slot filling model does not require any labeled in-domain training example. 
To summarize the automatic label generation phase, both sets of labels are acquired in a fully automatic fashion without any hand-crafting.


In contrast to previous work in weak supervision~\cite{ren2015clustype,he2017autoentity,fries2017swellshark,giannakopoulos2017unsupervised} that obtains a single set of noisy labels and then propose techniques to overcome the challenge of fitting an effective model to the noisy labels, we acquire two sets of complementary labels.
The choice of these two sets of labels is guided by the intuition that they should be complementary and the models trained on these sets of labels should be able to share complementary information with the other to improve the performance in the later phases of the framework.
Essentially, the first set of labels carries information from external knowledge sources, whereas the labels generated through the pre-trained zero-shot slot filling model capture how the slot values are mentioned in other domains.
%
To further elaborate on the motivation and our process for the first set of labels (i.e., labels using KG and other NLP models), the pre-trained LMs have been shown to have a great deal of knowledge~\cite{petroni2019language}, thus should be capable of generating automatic labels with no need of external KG. 
To the best of our knowledge, there exists no work that shows that accurate token-level automatic labeling (e.g., slot filling task) is possible with pre-trained LMs. 
Moreover, such approaches would require heavy prompting in each new target domain, whereas our label generation process is fully automatic and only relies on the readily-available pre-trained NLP models and external KG.

\subsection{Phase Two: LM-assisted Weak Supervision}
Since we do not have access to dataset $\{(\mathbf{X}_n,\mathbf{Y}_n)\}_{n=1}^N$ with true ground-truth labels.
We use pseudo labels generated in phase one, $\{(\mathbf{X}_n,\mathbf{D}_n)\}_{n=1}^N$, to learn 
$f_{m,c}(\cdot; \cdot)$ that outputs the probability of the $m$-th token to take on class $c$. 
We learn $f_{m,c}(\cdot; \cdot)$ by minimizing the following loss over the noisy dataset $\{(\mathbf{X}_n,\mathbf{D}_n)\}_{n=1}^N$: 
$$
\hat\theta = \argmin_{\theta}\frac{1}{N}\sum_{n=1}^{N} \ell(\mathbf{D}_n, f(\mathbf{X}_{n}; \theta)),
\label{eq:stage1}
$$
where $\ell(\mathbf{D}_n, f(\mathbf{X}_{n}; \theta)) = \frac{1}{M} \sum_{m=1}^{M} -\log{f_{m,d_{n, m}}(\mathbf{X}_{n}; \theta)}$. 
We employ the pre-trained multilingual BERT with token-level classification head that uses Adam optimizer \cite{kingma2014adam,Liu2019} with early stopping and multiple random initializations. 


Since slot filling task is similar to the MLM training objective of the BERT, we employ pre-trained BERT as the backbone model.
That is, MLM's goal is to predict the masked tokens using bidirectional contexts. Similarly, slot filling tries to predict the label for a token leveraging both left and right contexts simultaneously, which makes the pre-trained BERT an ideal model of choice that greatly facilitates minimizing incomplete labels.
It is important to highlight that our automatically generated labels are not only incomplete but also potentially wrong.
The training strategies employed in this phase minimize the noise in the label to some extent. 
Specifically, early stopping can provide a strong regularization and would not let the model overfit to the noisy labels, especially wrong labels. 
Moreover, early stopping does not let the model forget the knowledge in the pre-trained model.
Similarly, multiple random initializations enforce robustness. 
Since the model is fine-tuned on the noisy labels, averaging the predictions of multiple models for each token ensures that wrong labels end up with low probabilities and true labels consistently achieve high probabilities.
Using the above-mentioned strategies, we train two slot filling models, which we call the peers. The peer one is trained on the first set of pseudo labels that were generated using POS and NER taggers, KG, and the GPT-2 scorer in phase one. Similarly, peer two is trained using the predictions of the zero-shot slot filling model~\cite{liu2020coach}.
Both models have the same architecture and follow the same training procedures.

\begin{table*}[t!]
\centering
\caption{Dataset statistics.}
\vspace{-7pt}
\label{tab:dataset}
\begin{tabular}{lccccc}
\toprule
\textbf{Dataset}  & \textbf{Dataset Size} & \textbf{Vocab. Size} & \textbf{Avg. Length} & \textbf{\# of Domains} & \textbf{\# of Slots} \\ \hline
\textbf{SGD}      & 188K                  & 33.6K                & 13.8                 & 20                     & 240                  \\
\textbf{MultiWoZ} & 67.4K                 & 10.5K                & 13.3                 & 8                      & 61 \\
\bottomrule
\end{tabular}
\vspace{-7pt}
\end{table*}

\subsection{Phase Three: Self-supervised Co-training}
We introduce an iterative peer training algorithm where both peers generate high-confidence soft labels for training the other peer in the next iteration. 
Theoretically, these peers can be anything, but in this work, 
we explore two of the most promising directions that have shown the promise to minimize the need for manual labeling for the task: zero-shot learning and distant supervision.
This phase uses a self-supervised co-training scheme to exploit the patterns of slot values from other domains through the labels generated by the zero-shot filling model (i.e., peer two)~\cite{liu2020coach} as well as utilize the knowledge in external KGs and pre-trained models via labels provided by the peer one.
Specifically, we initialize the peers trained in phase two and use their pseudo labels to kick-start training in this phase.
Specifically, peer one $f_{m,c}(\cdot; \theta_{\textrm{p1}})$ would generate labels $\{\tilde{\mathbf{Y}}^{(t)}_n = [\tilde{y}_{n,1}^{(t)}, ..., \tilde{y}_{n,m}^{(t)}]\}_{n=1}^{N}$ for peer two $f_{m,c}(\cdot; \theta_{\textrm{p2}})$ at the $t$-th iteration by:
$$
\tilde{y}_{n,m}^{(t)} = \argmax_{c}{f_{m,c}(\mathbf{X}_n; \theta_{\textrm{p1}}^{(t)})}. 
\label{eq:pseudo}
$$

Based on these labels, the peer two can be fine-tuned by: 
$$
\hat\theta_{\textrm{p2}}^{(t+1)} = \argmin_{\theta}\frac{1}{N}\sum_{n=1}^N \ell(\tilde{\mathbf{Y}}_n^{(t)}, f(\mathbf{X}_{n}; \theta)).
\label{eq:self_train1}
$$

Similarly, peer two $f_{m,c}(\cdot; \theta_{\textrm{p2}})$ would generate pseudo labels for peer one $f_{m,c}(\cdot; \theta_{\textrm{p1}})$ that are used to fine-tune peer one. 
We also notice that it is beneficial to stop early during this phase as well, to improve the model fitting and gradually reduce the noise associated with the automatically generated labels.
Since pseudo labels are refined gradually in an iterative way, both peers can benefit from the knowledge contained within the labels of the other while avoiding overfitting.
Furthermore, as an alternative to pseudo labels, we also generate soft labels that are used for confidence re-weighting. 
The high-confidence soft label selection strategy enables better model fitting and efficient learning via better quality of the automatic labels.
Specifically, for the given $m$-th token in the $n$-th training example, the probability for all classes $C$ is $[f_{m,1}(\mathbf{X}_n;\theta),...,f_{m,C}(\mathbf{X}_n;\theta)]$. 
Following ~\cite{xie2016unsupervised}, at $t$-th iteration, peer one generates soft labels, $\{\mathbf{S}_n^{(t)} = [\mathbf{s}_{n,m}^{(t)}]_{m=1}^M \}_{n=1}^N$, as given below:
$$
\mathbf{s}_{n,m}^{(t)} = [s_{n,m,c}^{(t)}]_{c=1}^{C} = \Bigg[  \frac{f_{m,c}^2(\mathbf{X}_n;\theta_{\textrm{peer1}}^{(t)})/p_{c}}{\sum_{c'=1}^C f_{m,c'}^2(\mathbf{X}_n;\theta_{\textrm{peer1}}^{(t)})/p_{c'}}\Bigg]_{c=1}^{C}
\label{eq:soft}
$$ 
where $p_{c} = \sum_{n=1}^N \sum_{m=1}^M f_{m,c}(\mathbf{X}_n;\theta_{\textrm{p1}}^{(t)})$ computes the frequency of the tokens for the $c$-th class. 
Then, peer two $f(\cdot; \theta_{\textrm{p2}}^{(t+1)})$ is fine-tuned by:
$$
\theta_{\textrm{p2}}^{(t+1)} = \argmin_{\theta} \frac{1}{N} \sum_{n=1}^{N} \ell_{\rm KL}(\mathbf{S}_n^{(t)}, f(\mathbf{X}_{n}; \theta)),
$$
where $\ell_{\rm KL}(\cdot,\cdot)$ is the KL-divergence-based loss:
$$
\ell_{\rm KL}(\mathbf{S}_n^{(t)}, f(\mathbf{X}_{n}; \theta))=\frac{1}{M}\sum_{m=1}^M\sum_{c=1}^C - s_{n,m,c}^{(t)} \log f_{m,c}(\mathbf{X}_{n}; \theta).
\label{eq:klloss}
$$

Moreover, we also investigate selecting tokens that have high confidence. 
For instance, we pick high-confidence tokens from the $m$-th input example at the $t$-th iteration by  
$
H^{(t)}_n = \{m : \max_{c} s_{n,m,c}^{(t)} > \epsilon \},
$
where $\epsilon\in [0,1]$ is a threshold that can be searched based on a small validation set. 
Then, peer two $f(\cdot; \theta_{\textrm{p2}}^{(t+1)})$ is fine-tuned by:
$$
\theta_{\textrm{p2}}^{(t+1)} %&= \argmin_{\theta} \frac{1}{N} \sum_{n=1}^{N} \ell_{\rm S-KL}(\bS_n^{(t)}, f(\bX_{n}; \theta)) \\
= \argmin_{\theta} \frac{1}{N|H^{(t)}_n|}\sum_{n=1}^{N} \sum_{m\in H^{(t)}_n}\sum_{c=1}^C - s_{n,m,c}^{(t)} \log f_{m,c}(\mathbf{X}_{n}; \theta).
$$

This phase improves the robustness to effectively fit the model for tokens with high confidence. 
Both peers keep sharing information and their confidence by producing soft labels for their counterparts until they approximate to the true labels while employing early stopping and scheduled learning rates.
It is important to remind that phase three is the most important phase that progressively reduces noise from the labels to a great extent and enables superior performance for the task of open-domain slot filling.
\section{Included Tools}
\label{sec: included tools}

Saihu currently includes 3 tools, namely xTFA, DNC, and Panco. A summary of the supported method and tool pairs is listed in Figure~\ref{fig: supported methods}.

\begin{figure}[!thb]
    \centering
    \begin{tabular}{|c|c|c|c|}
        \hline
        Method\textbackslash Tool & DNC & xTFA & Panco \\
        \hline\hline
        TFA & V & V & V \\
        SFA & V &   & V \\
        PLP &   &   & V \\
        ELP &   &   & V \\
        PMOO& V &   &   \\
        TMA & V &   &   \\
        \hline
    \end{tabular}
    \caption{Supported methods for each tool. A check ``V'' on it means the tool supports the corresponding method.}
    \label{fig: supported methods}
\end{figure}

\textbf{xTFA} \cite{thoma2022analyse}:
xTFA is short for \textit{experimental modular TFA}, which is developed in Python and supports a more advanced TFA (Total Flow Analysis). It takes an XML file as its input for network description, which we will discuss in more detail in Section \ref{sec: physical network xml}. xTFA supports analyzing networks with cyclic dependency and multicast flows, i.e. a flow having multiple paths or potential splits. In other tools, a multicast flow will be treated as separated flows with the same arrival bounds.

\textbf{DNC}~\cite{bondorf2014discodnc}:
DNC is developed in Java and supports TFA, SFA (Separate Flow Analysis), PMOO (Pay Multiplexing Only Once)~\cite{bondorf2014discodnc}, and TMA (Tandem Matching Analysis)~\cite{scheffler2021fifo}. There's no specific input description file for DNC, one has to define the network as a Java script if they use DNC directly. Saihu uses the information from an output port network to create a network in DNC syntax internally. Moreover, with DNC, one cannot manually set shaping with FIFO multiplexing but only with arbitrary multiplexing. Also, no analysis methods in DNC are capable of solving networks with cyclic dependency.

\textbf{Panco}~\cite{bouillard2022tradeoff}:
Panco is developed in Python and supports TFA, SFA, PLP (Polynomial size Linear Program), and ELP (Exponential size Linear Program). Since all its methods are implemented as linear programs, it requires \textit{lpsolve}~\cite{lpsolve}. Same as DNC, Panco doesn't have a specific input description file, Saihu internally creates the network in Panco syntax from the information of an output port network. All the methods of Panco except ELP support networks with cyclic dependency. 


\section{Software Description}
\label{sec: software description}
To execute analyses with Saihu, we roughly divide the tasks into 3 parts: describe a network to be analyzed; execute analyses with individual tools; and export reports back to the user. We will go through these 3 parts one by one.

\subsection{Network Description File}
\label{sec: network description file}
As mentioned in Section~\ref{sec: system model}, Saihu allows the user to write a network in either a \textit{physical network} or an \textit{output port network} format. While xTFA takes a physical network as an XML file and the others take an output port network as a JSON file, one can choose the format they prefer to define a network as Saihu automatically converts a file when needed. 

\subsubsection{Option 1: Defining Physical Network in XML}
\label{sec: physical network xml}
A physical network is written as an XML file according to the xTFA specification. It should at least contains 4 kinds of information: \textbf{General network information}, \textbf{Servers}, \textbf{Links}, and \textbf{Flows}.

Let's take the implementation of Figure \ref{fig: physical network} as an example. First, every entry should be enclosed in one element \texttt{<elements>} as shown in Listing~\ref{lst: xml elements}.

\begin{lstlisting}[language=XML,caption={Examples of XML file. All network entries must inside an element \texttt{<elements>}.},
label={lst: xml elements}]
<?xml version="1.0" encoding="UTF-8"?>
<elements>
    <!-- All entries -->
</elements>
\end{lstlisting}

A physical network must have exactly one \texttt{network} element to define general information across the network as its attributes, an example is shown in Listing~\ref{lst: xml network}. In this example, \texttt{name} is the name of the network, \texttt{technology} is a series of analysis parameters concatenated by the plus sign, and a default value of \texttt{minimum-packet-size}. 

\begin{lstlisting}[language=XML,caption={Example of general network information. Contains \texttt{name}, \texttt{technology} used, and default values for other elements.},label={lst: xml network}]
<network name="demo" technology="FIFO+IS" 
    minimum-packet-size="4B"/>
\end{lstlisting}

The attribute \texttt{technology} takes the following values. More may be found from~\cite{thoma2022analyse}.
\begin{itemize}[leftmargin=1em]
    \item \texttt{FIFO}: FIFO multiplexing. It can be \texttt{ARBITRARY} for arbitrary multiplexing or left blank for tool default.
    \item \texttt{IS}: Input shaping. Consider the shaping effect.
    \item \texttt{PK}: Packetizer.
    \item \texttt{CEIL}: Fix precision when calculating network with cyclic dependency (used only in xTFA.)
\end{itemize}
One can also define some default values that possibly appear in other elements. For example, a \texttt{minimum-packet-size} is usually defined as an attribute of a \texttt{flow} element, while this value defined in the \texttt{network} element will be used as the default value if it's not defined in a \texttt{flow} element.

Second, the servers of the network can be defined as either a \texttt{station} or a \texttt{switch}, as shown in Listing~\ref{lst: xml server}. Although they are very different physically, in our tools they both mean data processing units or possible sources/sinks of a data flow.

\begin{lstlisting}[language=XML,caption={Stations and switches. Name and possibly the default values to all its ports.},label={lst: xml server}]
<station name="src0"/>
<station name="src1"/>
<station name="src2"/>
<switch name="s0" service-latency="10us"
    service-rate="4Mbps"/>
<switch name="s1" service-latency="10us" 
    service-rate="4Mbps"/>
<station name="sink0"/>
<station name="sink1"/>
\end{lstlisting}

Both a station and a switch represent a physical node. The name of each node will be used to define flow paths and links. The \texttt{service-latency} and \texttt{service-rate} define a rate-latency service curve. The service parameters defined at this level serve as default values for all the links attached as outputs of this node.

Third, one must connect physical nodes with \texttt{link}s, as shown in Listing \ref{lst: xml link}. Saihu considers output ports as processing units, so the physical link \texttt{from} a physical node \texttt{to} another node has to be defined, along with the input/output ports used by the link. For example, the link \texttt{lk:s0-s1} connects from the output port \texttt{o0} of switch \texttt{s0} to the input port \texttt{i0} of switch \texttt{s1}.

\begin{lstlisting}[language=XML,caption={Links connecting ports.},label={lst: xml link}]
<link name="lk:src0-s0" from="src0" to="s0" 
    fromPort="o0" toPort="i0"/>
<link name="lk:src1-s0" from="src1" to="s0"
    fromPort="o0" toPort="i1"/>
<link name="lk:src2-s0" from="src2" to="s1" 
    fromPort="o0" toPort="i1"/>
<link name="lk:s0-s1" from="s0" to="s1"
    fromPort="o0" toPort="i0" 
    transmission-capacity="10Mbps"/>
<link name="lk:s1-sink0" from="s1" to="sink0" 
    fromPort="o0" toPort="i0" 
    transmission-capacity="10Mbps"/>
<link name="lk:s1-sink1" from="s1" to="sink1" 
    fromPort="o1" toPort="i0" 
    transmission-capacity="10Mbps"/>
\end{lstlisting}

If the service of an output port needs to be considered in an analysis, one must define the service curve at the link that is directly attached to the output port. The \texttt{transmission-capacity} of the link can also be specified to consider line shaping. If no values are defined, the system tries to apply the default values defined at the upper levels, i.e. \texttt{switch/station} and \texttt{network}. Furthermore, if no values are found across all levels, the link is considered a dummy one and the output port attached to it will not be considered.

Finally, one must define \texttt{flow}s for the network as shown in Listing \ref{lst: xml flow}. Each flow is defined by a \texttt{flow} element. The paths of a flow are defined by \texttt{target} elements, where each node it traverses is listed as \texttt{path} elements with its \texttt{node} attribute indicating the name of the physical node. In this format, multicast of a flow is possible by defining multiple \texttt{target} elements within the same flow.

\begin{lstlisting}[language=XML,caption={Flows. Must have a name and its arrival curve parameters along with its paths.},label={lst: xml flow}]
<flow name="f0" arrival-curve="leaky-bucket" 
    lb-burst="10B" lb-rate="10kbps" 
    maximum-packet-size="50B" source="src0">
    <target>
        <path node="s0"/>
        <path node="s1"/>
        <path node="sink0"/>
    </target>
</flow>
<flow name="f1" arrival-curve="leaky-bucket" 
    lb-burst="10B" lb-rate="10kbps" 
    maximum-packet-size="50B" source="src1">
    <target>
        <path node="s0"/>
        <path node="s1"/>
        <path node="sink1"/>
    </target>
</flow>
<flow name="f2" arrival-curve="leaky-bucket" 
    lb-burst="10B" lb-rate="10kbps" 
    maximum-packet-size="50B" source="src2">
    <target>
        <path node="s1"/>
        <path node="sink0"/>
    </target>
</flow>
\end{lstlisting}

A flow element must have \texttt{name} and the arrival curve specified as its attributes. The keywords \texttt{arrival-curve}, \texttt{lb-burst} and \texttt{lb-rate} define a leaky-bucket curve at the source of the flow. Other parameters like the \texttt{maximum-packet-size} and \texttt{minimum-packet-size} can be also defined to consider packetization. Furthermore, as the definition represents a physical network, each flow must have a data \texttt{source} that is an actual physical node. All the output ports involved in its path, including the output port of the source, will be analyzed by Saihu.


\subsubsection{Option 2: Defining Output Port Network in JSON}
While the XML file syntax is provided by xTFA, we design this JSON format ourselves in order to write an output port network in a concise way.
The file should at least contains 3 kinds of information: \textbf{General network information}, \textbf{Servers}, and \textbf{Flows}. Let's take the implementation of Figure \ref{fig: output port network} as an example. First, all entries must be enclosed as a single JSON object (one \{\} to enclose all attributes.)

A \texttt{network} object is required to define general network information but only the \texttt{name} attribute is necessary. An example is shown in Listing \ref{lst: json network}.


\begin{lstlisting}[language=json,caption={Network information. Contains some general information and default values or units used throughout the file.},label={lst: json network}]
"network": {
    "name": "demo",
    "packetizer": false,
    "multiplexing": "FIFO",
    "analysis_option": ["IS"],
    "time_unit": "us",
    "data_unit": "B",
    "rate_unit": "Mbps",
    "min_packet_length": "4B"
}
\end{lstlisting}
\lstsetblack

The 3 keywords \texttt{packetizer}, \texttt{multiplexing}, and \texttt{analysis\_option} are unique to the \texttt{network} object. \texttt{packetizer} is equivalent to the keyword \texttt{PK} in XML file; \texttt{multiplexing} can be either \texttt{FIFO} or \texttt{ARBITRARY}; and \texttt{analysis\_option} takes other keywords defined in \texttt{technology} mentioned in Section \ref{sec: physical network xml}.

Except for the network options, default values for servers and flows can also be defined at the network level. In the above example, we set the default time/data/rate units to be microsecond/byte/megabits-per-second across the file as well as the minimum packet length being 4 bytes.

Second, we need to define the \texttt{servers} for the network. Some may argue the term \textit{server} instead of \textit{output port} as we discussed in Section~\ref{sec: output port network}. The term \textit{server} is a general term for a processing unit, and one can treat it as a black box that provides service.

The \texttt{servers} is presented as a JSON array, each object in this array is a server. Each server must at least have a \texttt{name}, and its service curve can be missing only when there exists a default value in \texttt{network} attribute. 
The parameters can be expressed in either a \textit{string} or a \textit{number}. A string is written as a number followed by a unit. For example, \texttt{"10us"} means 10 microseconds, and \texttt{"50Mbps"} means 50 megabits per second. If it's directly written as a number, the unit defined in the closest level is used. For example, the time unit defined in server \texttt{s1-o0} is microsecond, so the latency 10 is read as 10 microseconds. 

The object \texttt{service\_curve} takes multiple rate-latency curves and uses the maximum among all these curves as its service curve. Rates and latencies are written as arrays, each pair of rate and latency values with the same index is a rate-latency curve. For example, in server \texttt{s0-o0}, the service curve has 2 segments defined by 2 rate-latency curves, one with a latency of 10 microseconds and a rate of 4 megabits per second, and the other with a latency of 1000 microseconds and a rate of 50 megabits per second.

Notice that in an output port network definition, we don't manually define links. The topology of the network is considered to be the \textit{graph induced by flows}, i.e. a connection from server \textit{A} to \textit{B} exists only when there is at least one flow travels through \textit{B} from \textit{A}. Therefore, the transmission capacity of the link attached to an output port is directly defined on a server with the keyword \texttt{capacity}.

\begin{lstlisting}[language=json,caption={Servers. A list that contains many servers, each with name and service parameters.},label={lst: json server}]
"servers": [
    {
        "name": "s0-o0",
        "service_curve": {
            "latencies": ["10us", 1000],
            "rates": [4, "50Mbps"]
        },
        "capacity": 100
    },
    {
        "name": "s1-o0",
        "service_curve": {
            "latencies": [10, "1ms"],
            "rates": [4, 50]
        },
        "capacity": 100,
        "time_unit": "us"
    },
    {
        "name": "s1-o1",
        "service_curve": {
            "latencies": [10],
            "rates": ["4Mbps"]
        },
        "capacity": 100
    }
]
\end{lstlisting}
\lstsetblack

Finally, the \texttt{flows} are defined in a similar manner as servers as shown in Listing~\ref{lst: json flow}. Each object must have at least a \texttt{name} and a \texttt{path}. A path is represented as an array of server names, and the order in the list represents the sequence that the flow visits.

The representation of values and units is the same as servers, either being a string of a number with units, or a number that uses the default unit.
The arrival curve at the source of a flow is defined by multiple token-bucket curves and taken as the minimum among all these curves. Similar to the service curve of a server, each pair of a burst and a rate value represent a token-bucket curve. For example, the arrival curve of \texttt{f0} has one token-bucket curve of burst 10 bytes and rate 10 kilobits per second, and the other curve of burst 2 kilobytes and rate 0.5 megabits per second.

\begin{lstlisting}[language=json,caption={Flows. A list contains many flows. Each flow contains name, path, and parameters of the arrival data.},label={lst: json flow}]
"flows": [
    {
        "name": "f0",
        "path": ["s0-o0", "s1-o0"],
        "arrival_curve": {
            "bursts": [10, "2kB"],
            "rates": ["10kbps", 0.5]
        },
        "max_packet_length": 50,
        "rate_unit": "kbps"
    },
    {
        "name": "f1",
        "path": ["s0-o0", "s1-o1"],
        "arrival_curve": {
            "bursts": ["10B"],
            "rates": ["10kbps"]
        },
        "max_packet_length": 50
    },
    {
        "name": "f2",
        "path": ["s1-o0"],
        "arrival_curve": {
            "bursts": [10],
            "rates": ["10kbps"]
        },
        "max_packet_length": "50B",
        "min_packet_length": "4B"
    }
]
\end{lstlisting}
\lstsetblack

\subsection{Tool Usage}
In this section, we briefly introduce how to use Saihu to execute analyses. One would only need to import one file, i.e. \textit{interface.py}, to use Saihu given that the project is installed correctly. 
The simplest way to use Saihu is shown in Listing~\ref{lst: simple example}. Once a network description file is available as either an XML or a JSON file, one can execute the following example to do the analysis.

\begin{lstlisting}[style=pythonstyling,caption={Simple example to use Saihu.},label={lst: simple example}]
from interface import TSN_Analyzer
analyzer = TSN_Analyzer("demo.json")
analyzer.analyze_all()
analyzer.export("demo")
\end{lstlisting}

The basic procedure to use Saihu is as follows: 1. initialize the analyzer with a target network description file; 2. execute the analysis with some tools; 3. export the results into reports. 

The supported tools and methods are listed in Figure~\ref{fig: supported methods}. 
To switch between different tools, one uses different functions with names like \texttt{analyze\_xxx}, where \texttt{analyze\_all} means to use all the available tools. To switch methods, one gives different input arguments to each analysis function. An example is provided in Listing~\ref{lst: switch tool and method}. One can execute multiple analyses and all the results will be stored in the internal buffer of the analyzer until the analyzer exports them into reports. 
% The default setting is to try to execute \textit{TFA, SFA} and \textit{PLP}.

\begin{lstlisting}[style=pythonstyling,caption={Execute different tools and methods.},label={lst: switch tool and method}]
analyzer.analyze_dnc()
analyzer.analyze_xtfa("TFA")
analyzer.analyze_panco(methods=["SFA", "PLP"])
\end{lstlisting}


\subsection{Analysis Reports}
\label{sec: analysis reports}

\begin{figure*}[tbh]
\centering
\begin{subfigure}[b]{0.3\textwidth}
    \centering
    \includegraphics[width=\linewidth]{e2e_delay.png}
    \caption{Flow end-to-end delay}
    \label{fig: e2e delay}
\end{subfigure}
\hfill
\begin{subfigure}[b]{0.3\textwidth}
    \centering
    \includegraphics[width=\linewidth]{server_delay.png}
    \caption{Server delay}
    \label{fig: server delay}
\end{subfigure}
\hfill
\begin{subfigure}[b]{0.3\textwidth}
    \centering
    \includegraphics[width=0.9\linewidth]{exec_time.png}
    \caption{Execution time}
    \label{fig: exec time}
\end{subfigure}
\hfill
\begin{subfigure}[b]{0.3\textwidth}
    \centering
    \includegraphics[width=0.8\linewidth]{report_topo.png}
    \caption{Network topology}
    \label{fig: network topology}
\end{subfigure}
\hfill
\begin{subfigure}[b]{0.3\textwidth}
    \centering
    \includegraphics[width=0.6\linewidth]{report_path.png}
    \caption{Flow paths}
    \label{fig: flow path}
\end{subfigure}
\hfill
\begin{subfigure}[b]{0.3\textwidth}
    \centering
    \includegraphics[width=0.9\linewidth]{report_util.png}
    \caption{Link utilization}
    \label{fig: link utilization}
\end{subfigure}
\caption{Human-friendly report. It is written as a Markdown file. The analysis results include the flow end-to-end delay, server delay, and execution time. The values are listed in tables for each tool and method as shown in (a)(b)(c). The units are adjusted accordingly. It also contains input information like the network topology, paths of flows, and link utilization.}
\label{fig: human friendly report}
\end{figure*}

Saihu can generate 2 reports, a \textit{human-friendly report} and a \textit{machine-friendly report}. A human-friendly report is written as a Markdown file that lists all the essential information. An example is shown in Figure~\ref{fig: human friendly report}. The analysis results are listed in 3 sections: per-flow end-to-end delay, per-server delay, and execution time. The delay bounds are presented in tables where each row is a flow or a server, and each column is a method executed by a tool. The last column contains the minimum result obtained in the current round of analysis. The execution time of each method by each tool is also listed for comparison.

Other than the analysis results, the report also contains some information about the user inputs, but in a more formatted manner. They are network topology, flow paths, and link utilization. Network topology is shown as a graph induced by flows. i.e. It's a directed graph where each node is an output port, and an edge from node A to B exists if there's at least one flow traversing from A to B. Flow paths are the same as the network description file, it can serve as a reassurance of user's input. Link utilization is computed node-wise, it is defined as the ratio between the aggregated arrival rate at a node and its service rate. e.g. If 2 flows have arrival rates of $2kbps$ and $3kbps$ respectively filling into a node, which has a service rate of $10kbps$. The link utilization at the node is therefore $(2+3)/10=0.5$.

A \textit{machine-friendly report} stores only the execution outputs, namely the per-flow end-to-end delay, per-server delay, and execution time. It's written in JSON format for easy parsing from other programs. An example is presented in Listing \ref{lst: machine friendly report}.
\begin{lstlisting}[language=json,caption={Machine-friendly report. The flow end-to-end delays, server delays, and execution time are listed in pure numbers. The units these numbers use are also listed as one entry.},label={lst: machine friendly report}]
{
    "name": "demo",
    "flow_e2e_delay": {
        "f0": {
            "Panco_TFA": 100.12500000000001,
            "Panco_PLP": 80.05,
            "DNC_TFA": 100.00375,
            "xTFA_TFA": 99.32394489448944
        },
...
    "server_delay": {
        "s0-o0": {
            "Panco_TFA": 50.0,
            "DNC_TFA": 50.0,
            "xTFA_TFA": 50.0
        },
...
    "execution_time": {
        "Panco_TFA": 62.70909309387207,
        "Panco_PLP": 243.81709098815918,
        "DNC_TFA": 32.0,
        "xTFA_TFA": 81.46500587463379
    },
    "units": {
        "flow_delay": "us",
        "server_delay": "us",
        "execution_time": "ms"
    }
}
\end{lstlisting}
\lstsetblack

In order to let users parse the information easily, Saihu prints the results in numbers, accompanied by the units used in each section. Note that the human-friendly report always contains only 3 decimal digits according to the smallest value in the table, while there's no such rounding for the machine-friendly report. As a result, one should read the machine-friendly report if they require very precise results.


\subsection{Network Generation}
\label{sec: network generation}
Saihu provides a series of functions to allow users to generate certain types of networks into a network description file. Currently, Saihu supports the generation of interleave tandem, mesh, and ring network. They contain specific topologies and routing rules for the flow paths. Users have the freedom to choose the size of the network (number of servers), the service parameters of servers, and the flow parameters of data arrival. The way these parameters are used within each type of network is specified in the respective sections.

Other than the predetermined routing rules, Saihu also provides a function to generate an arbitrary number of flows with random routing. This is particularly suitable for testing the possible traffic with a specific topology. More details will be shown in Section~\ref{sec: fixed topology random}.

\subsubsection{Interleave Tandem Network}
Suppose we wish to generate a network of $n$ servers, indexed from $0$ to $n-1$.
An interleave tandem network has all its servers chained in a line. One flow $f_0$ goes through all servers from $s_0$ to $s_{n-1}$.  The flow $f_i$ is $s_{i-1} \rightarrow s_{i}$ for $i \in [1,n-1]$. Illustrated by Figure \ref{fig: interleave}. All the flows have identical arrival curves and maximum packet length at the source, defined by function arguments \texttt{burst}, \texttt{arrival\_rate}, and \texttt{max\_packet\_length}. Likewise, All the servers have identical service curves and transmission capacity, defined by \texttt{latency}, \texttt{service\_rate}, and \texttt{capacity}.

\begin{figure}
    \centering
    \includegraphics[width=\linewidth]{images/interleave.png}
    \caption{Interleave tandem network}
    \label{fig: interleave}
\end{figure}

\subsubsection{Ring Network}
A ring network is illustrated in Figure \ref{fig: ring}. There are $n$ flows and $n$ servers. The path of flow $i$ is $s_i \rightarrow s_{i+1} \rightarrow \cdots \rightarrow s_{i+n-1\mod n}$ for $0 \leq i \leq n-1$. A ring network is completely symmetrical with all its flows and servers being identical. All flows are defined by \texttt{burst}, \texttt{arrival\_rate}, and \texttt{max\_packet\_length}. Similarly, all servers are defined by \texttt{latency}, \texttt{service\_rate}, and \texttt{capacity}.

\begin{figure}
    \centering
    \includegraphics[width=0.6\linewidth]{images/ring.png}
    \caption{Ring network}
    \label{fig: ring}
\end{figure}

\subsubsection{Mesh Network}
A mesh network is illustrated in Figure~\ref{fig: mesh}. All flows start from either $s_0$ or $s_1$. The flows go through all $2^{(n-1)/2}$ possible combinations of servers towards the right. e.g. $s_0 \rightarrow s_2 \rightarrow \cdots$ and $s_1 \rightarrow s_2 \rightarrow \cdots$ are both in the network. All servers have the same service curve and capacity except $s_{n-1}$ has the doubled service rate. All flows have identical service curves and maximum packet length.

\begin{figure}
    \centering
    \includegraphics[width=0.9\linewidth]{images/mesh.png}
    \caption{Mesh network. Only parts of the flows from $s_0$ are shown. There is a flow for every possible path from $s_0$ or $s_1$ to $s_{n-1}$.}
    \label{fig: mesh}
\end{figure}

\subsubsection{Fixed-Topology Random Routing Network}
\label{sec: fixed topology random}

It's also possible to randomly generate a network defined as a JSON file with a fixed topology of switches. Users have the freedom to decide the number of flows, the topology of switches, and the service/arrival parameters. Each flow randomly routes from one switch to another without repeating the visited switch. We will show more details and use it as an example in Section~\ref{sec: example}.

% A user provides a fixed number of flows to be generated and a connection table along with service parameters and data arrival parameters, then a possible configuration of the network will be generated. A user can also specify the parameters in a range so that they will be uniformly generated within the range.

% We use Figure \ref{fig: industrial network} as an example in Section \ref{sec: example}.

\subsection{Extension}
\label{sec: extension}

As we showed in Figure~\ref{fig: pipeline}, Saihu uses XML/JSON files as a common input and a general information container class as a common output for all the tools. This means to incorporate more tools into Saihu, one only needs to allow the new tool to parse one of the network description formats and feed the analysis results into the information container class. By doing so, they can keep other parts of Saihu untouched and only need to manage one tool at a time.

Moreover, it's also possible to include more network description formats. Because the two formats Saihu uses currently can convert to each other, one only has to make sure a new format can be converted to and from one of the formats Saihu supports. We believe this approach can help more people contribute to and expand Saihu in the future.
 \begin{figure}[t]
        \centering
            \begin{subfigure}{0.48\linewidth}
            \centering
            {\includegraphics[width=0.95\linewidth]{figures/qual_wo.png}}
            \caption{\method w/o TA}\label{fig:quala}
           \end{subfigure}
           \begin{subfigure}{0.48\linewidth}
            \centering
            {\includegraphics[width=0.95\linewidth]{figures/qual_w.png}}
            \caption{\method w/ TA}\label{fig:qualb}
           \end{subfigure}\vspace{-7pt}
        \caption{Visualization of attention maps (a) without temporal adaptation (TA) and (b) with temporal adaptation for the action 'Spinning [something] that quickly stops spinning' in SSv2~\cite{ssv2}.}\vspace{-5pt}
    \label{fig:qual}
    \end{figure}
\section{Conclusion}\label{sec:conclusion}
In this work, we focus on addressing the fundamental challenge of OOD detection tasks, which is how to fully understand the semantic discrepancy between the ID/OOD samples. We reveal that the key to success in the realistic SCOOD task is to allocate as many ID samples in the unlabeled set correctly as possible. To this end, we propose a novel uncertainty-aware optimal transport scheme that introduces class-specific energy scores as guidance for effective label assignment. Experimental results show that our method achieves better performance than previous state-of-the-art methods on SCOOD benchmarks.

\textbf{Limitations.} In addition to temperature scaling, other techniques such as feature clipping applied in ReAct~\cite{sun2021react} also enhance the performance of energy score, so how to obtain an OOD score that best fits the SCOOD task can be further explored. Moreover, a setting highly related to SCOOD has been proposed in \cite{katz2022training} and formulated as a constrained optimization problem. We will also theoretically analyze these practical OOD settings in our feature work.

% \section*{Acknowledgments}
\textbf{Acknowledgments.} 
This work is supported by National Key R\&D Program of China under Grant 2020AAA0105701, National Natural Science Foundation of China (NSFC) under Grants 61872327, Major Special Science and Technology Project of Anhui, National Natural Science Foundation of China (62033012) and Ant Group through Ant Research Intern Program.


% \clearpage
\bibliographystyle{IEEEtran}
\bibliography{./Ref.bib}

\end{document}