{
\onecolumn
\normalsize

\ \\
\textbf{Responses to the reviewers}

\ \\
\textbf{Reviewer 1}
\begin{description}\itemsep.8em
\item[R1C1]
\textbf{
The paper considers the legalization problem in designs where cells can occupy multiple row heights, though the experimental results show only double row heights. The paper proposes two heuristics: one is based on chain movements and the other one is legalization while preserving the order. \\
Overall, the paper feels a bit unready in the sense that the two heuristics do not establish themselves clearly in terms of their applicability and their ability to fully legalize a placement because [12] is still needed.
}
\item[R1A1]
Thanks very much for the comments.

\item[R1C2]
\textbf{
- the chain movement algorithm does not guarantee legality, so as a result algorithm from [12] is used to fix placement, which is funny given that the proposed algorithm is supposed for detailed placement.
}
\item[R1A2]
{\color{blue}{
That is really a concern. 
If the input design is legal, the chain move algorithm can maintain legality, as mentioned in Section~\ref{sec:ndp}. 
Maybe we can run two flows. 
One is chain move + \cite{PLACE_DAC2016_Chow} + double-row placement; 
the other is \cite{PLACE_DAC2016_Chow} + chain move + double-row placement. 
Then compare the quality of two flows. 
But it might be better to do it in journal extension. 
}}

\item[R1C3]
\textbf{
- ordered double-row placement is introduced as some sort of post-processing step that is done in the last step of Algorithm 1. If that algorithm is doing what it is supposed to do (i.e., legalization), should it not be used in lieu of [12] to clean up the results from the chain movement algorithm?
}
\item[R1A3]
In general the algorithm can also be applied to resolve small amount of overlaps for legalization, 
but the computation effort turns out to an issue for layouts with large amount of overlaps due to its quadratic relation with maximum displacement. 
Therefore, we adopt it as an incremental optimization technique for legal designs. 
We have added this in the last paragraph of Section~\ref{sec:ext_to_typical_case}. 

\item[R1C4]
\textbf{
- The ordered double row placement can be considered as heuristic or optimal depending on the cell order case. It would have better if the authors outlined the complexity of the problem in the general case. Also, how does the algorithm behave, in terms of optimality, for the general case. 
}
\item[R1A4]
The general double-row placement problem without ordering constraints is very difficult, 
since the general single-row placement placement problem is already known as NP-hard \cite{PLACE_TCAD1989_Chowdhury}. 
We have added this in the last paragraph before Section~\ref{sec:nested_shortest_path}. 

\end{description}

\ \\
\textbf{Reviewer 2}
\begin{description}\itemsep.8em
\item[R2C1]
\textbf{
This paper proposes a multiple-row detailed placement of heterogeneous-sized cells. The proposed method consists of several steps including Chain move and
double-row placement. The efficiency of the proposed method is confirmed empirically.
}
\item[R2A1]
Thanks very much for the comments. 

\item[R2C2]
\textbf{
This method is interesting to the reviewer. But, the reviewer wonders whether the consistency is held during Ordered Double-Row Placement. For example, for the case in Fig. \ref{fig:double_row_place},
cell g and j are spread out of the rows. But, the movement of them seems to be free from the such irregularity. The reviewer would like to make it clear.
}
\item[R2A2]
In the case of Fig.~\subref*{fig:double_row_place3} where cells $e$, $g$, and $j$ spread out of the rows, 
their movements must keep the order within the two rows and not cause any overlap in the other rows. 
We have added this in the last paragraph before Section~\ref{sec:nested_shortest_path}. 

\end{description}

\ \\
\textbf{Reviewer 3}
\begin{description}\itemsep.8em
\item[R3C1]
\textbf{
The paper proposes procedures for detailed placement which includes multiple-row cells to optimize wire length and cell and pin density. The procedures include a new chain-based legalization, and a dynamic programming technique for improving wire length.
}
\item[R3A1]
Thanks very much for the comments. 

\item[R3C2]
\textbf{
Please include comparison with [12] in the simulation results as well as discuss in what ways this work is different or improves upon [12].
}
\item[R3A2]
{\color{blue}{
The work from \cite{PLACE_DAC2016_Chow} is a legalization algorithm that tries to remove overlaps with minimum displacement, 
while this work tries to optimize wirelength and density from globally placed designs. 
In general, a legalization algorithm that minimizes displacement usually degrades wirelength. 
It is not fair to compare our algorithm with \cite{PLACE_DAC2016_Chow}. 
It might be better to do it in journal extension. 
}}

\end{description}

\ \\
\textbf{Reviewer 4}
\begin{description}\itemsep.8em
\item[R3C1]
\textbf{
This paper proposes a detailed placement for multi-row heterogeneous-sized cells. The authors first optimize the wirelength, cell, and pin density for both single-row height cells and multi-row height cells. A nested dynamic programming based method is proposed for wirelength optimization. At last, they achieve good results compared with the recent work. \\
The results are good. And the method is also simple and reasonable.
}
\item[R3A1]
Thanks very much for the comments. 

\end{description}
}

