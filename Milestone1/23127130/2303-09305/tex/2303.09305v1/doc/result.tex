% \vspace{-.05in}
\section{Experimental Results}
\label{sec:Results}

% We implemented our algorithm in C++ and Python with PyTorch for GPU-accelerated gradient computation like \cite{PLACE_TCAD2020_Lin}.
% We conducted experiments on a Linux machine with an Intel Xeon Silver 4214 CPU (2.20 GHz and 24 cores), 251 GB RAM, and one NVIDIA TITAN RTX GPU.
% We tested our work on both the ISPD 2017 clock-aware FPGA placement contest benchmarks \cite{BENCH_ISPD2017_PLACE} and industrial benchmarks.

We implemented our GPU-accelerated placer in C++ and Python along with the open-source machine learning framework Pytorch for fast gradient back propagation\cite{PLACE_TCAD2020_Lin}.
% The experiments are conducted on a Linux machine with 251 GB RAM, one Intel Xeon Silver 4214 CPU (2.20GHz and 24 cores), and one NVIDIA TITIAN GPU.
We conduct experiments on a Ubuntu 18.04 LTS platform that consists of an Intel(R) Xeon(R) Silver 4214 CPU @ 2.20GHz (24 cores), one NVIDIA TITAN GPU,
and 251GB memory.
We demonstrate the effectiveness and efficiency of our proposed algorithm on both academic benchmarks \cite{BENCH_ISPD2017_PLACE} and industrial benchmarks
from the three most concerning aspects of routed wirelength (RWL), runtime (RT), and timing.

\subsection{Evaluation on Academic Benchmarks}
% \tabRef{tab:sota_comparison_ispd2017} summarizes the statistics of ISPD 2017 benchmarks
% as well as the comparison with the state-of-the-art placers,
% \texttt{UTPlaceF 2.0} \cite{PLACE_TODAES2018_Li_UTPlaceF2},
% \texttt{RippleFPGA} \cite{PLACE_ICCAD2017_Pui_RippleFPGA},
% \texttt{UTPlaceF 2.X} \cite{PLACE_FPGA2019_Li},
% and \texttt{NTUfplace} \cite{PLACE_TCAD2020_Chen}.
%The number of instances varies from 400K to 900K with 32--58 clock nets.
%These placers are the top three winners of the ISPD 2017 Clock-Aware Placement Contest.
% All results of the other placers are from their original publications.
% It needs to be mentioned that we do not compare with \texttt{elfPlace} because its algorithm cannot handle clock constraints (see \tabRef{tab:FPGAPlacers}).
% We compare the routed wirelength by the patched Xilinx Vivado v2016.4 for ISPD 2017 and the runtime of different placers.

%so all placement in \tabRef{tab:sota_comparison} are done with a CC value of 24 (TODO:Standardize the term?).
%Further more, the runtimes detailed here are attained with different machine setups, so they are not suited for direct comparison.
%
%use the routed wirelength reported by Xilinx Vivado v2016.4 (with ISPD2017 specific patches) to evaluate our quality.

% \tabRef{tab:sota_comparison_ispd2017} summarizes the statistics of ISPD 2017 benchmarks
% as well as the comparison with the state-of-the-art placers,
% \texttt{UTPlaceF 2.0} \cite{PLACE_TODAES2018_Li_UTPlaceF2},
% \texttt{RippleFPGA} \cite{PLACE_ICCAD2017_Pui_RippleFPGA},
% \texttt{UTPlaceF 2.X} \cite{PLACE_FPGA2019_Li},
% and \texttt{NTUfplace} \cite{PLACE_TCAD2020_Chen}.

% \tabRef{tab:sota_comparison_ispd2016} summarizes the statistics of ISPD 2016 benchmarks 
% as well as the the comparison with the state-of-the-art placers,
% \texttt{UTPlaceF}~\cite{PLACE_TCAD2018_Li},
% \texttt{RippleFPGA}~\cite{TCAD18_RippleFPGA_Chen},
% \texttt{GPlace 3.0}~\cite{TODAES18_GPlace3_Abuowaimer}
% \texttt{UTPlaceF-NEP}~\cite{PLACE_TCAD2019_Li_UTPLACEF_DL}, 
% and \texttt{elfPlace} (GPU)~\cite{PLACE_TCAD2021_Meng}.

The statistics of the ISPD 2017 academic benchmark are summarized in \tabRef{tab:sota_comparison_ispd2017}.
The number of instances varies from 400K to 900K with 32--58 clock nets.
We do not evaluate the timing performance because on this benchmark we have no access to the timing information of the device.
We compare the routed wirelength reported by patched Xilinx Vivado v2016.4 and placement runtime with four state-of-the-art placers,
\texttt{UTPlaceF 2.0} \cite{PLACE_TODAES2018_Li_UTPlaceF2},
\texttt{RippleFPGA} \cite{PLACE_ICCAD2017_Pui_RippleFPGA},
\texttt{UTPlaceF 2.X} \cite{PLACE_FPGA2019_Li},
and \texttt{NTUfplace} \cite{PLACE_TCAD2020_Chen}.
All the results of these placers are from their original placers.
We do not compare the results with \texttt{elfPlace} because its algorithm cannot handle clock constraints (see \tabRef{tab:FPGAPlacers}).

\iffalse
\begin{table*}
\begin{tabular}{|c||c|c|c|c|c|c|c|c|c|}
	\hline
    Design & \#LUT & \#FF & \#DSP & \#BRAM & \#SHIFT & \#LRAM & \#CARRY & \#Clock & \#CtrlSet \\
	\hline
	\hline
    IND01	& 1k & 1K & 1K & 1K & 1K & 1K & 1K & 123 & 456 \\
    IND02	& 1k & 1K & 1K & 1K & 1K & 1K & 1K & 123 & 456 \\
	\hline
\end{tabular}
\label{tab:IndustryBenchmarksCellStat}
\caption{Industry Benchmarks Cell Statistics XX * XX site, XX DSP column, XX RAM column, etc. \color{red}{Should we explain packing rules? Patent problems with SHIFT?}}
\end{table*}

\begin{table}[tb]
	\caption{ISPD 2017 Contest Benchmarks Statistics}
	\label{tab:ispd2017_statistics}
	\begin{tabular}{|c|c|c|c|c|c|}
	\hline
	Design     & \#LUT & \#FF & \#BRAM & \#DSP & \#Clock \\
	\hline
	\hline
	CLK-FPGA01 & 211K  & 324K & 164   & 75                & 32      \\
	CLK-FPGA02 & 230K  & 280K & 236   & 112               & 35      \\
	CLK-FPGA03 & 410K  & 481K & 850   & 395               & 57      \\
	CLK-FPGA04 & 309K  & 372K & 467   & 224               & 44      \\
	CLK-FPGA05 & 393K  & 469K & 798   & 150               & 56      \\
	CLK-FPGA06 & 425K  & 511K & 872   & 420               & 58      \\
	CLK-FPGA07 & 254K  & 309K & 313   & 149               & 38      \\
	CLK-FPGA08 & 212K  & 257K & 161   & 75                & 32      \\
	CLK-FPGA09 & 231K  & 358K & 236   & 112               & 35      \\
	CLK-FPGA10 & 327K  & 506K & 542   & 255               & 47      \\
	CLK-FPGA11 & 300K  & 468K & 454   & 224               & 44      \\
	CLK-FPGA12 & 277K  & 430K & 389   & 187               & 41      \\
	CLK-FPGA13 & 339K  & 405K & 570   & 262               & 47      \\
	\hline
	\end{tabular}
\end{table}

\begin{table}[htbp]
	\centering
	\caption{Industry Benchmarks Statistics}
	\label{tab:industry_statistics}
	  \begin{tabular}{|c|c|c|c|c|c|c|}
	  \hline
	  Design & \#LUT & \#LUTRAM & \#DFF & \#CLA & \#DSP & \#BRAM \\
	  \hline\hline
	  IND01 & 16771 & 9     & 10588 & 2093  & 13    & 0 \\
	  IND02 & 11421 & 6     & 10067 & 335   & 24    & 0 \\
	  IND03 & 109472 & 0     & 11696 & 0     & 0     & 0 \\
	  IND04 & 29277 & 218   & 17096 & 1187  & 16    & 0 \\
	  IND05 & 63667 & 28576 & 191214 & 3848  & 928   & 64 \\
	  IND06 & 111860 & 0     & 64949 & 4063  & 0     & 21 \\
	  IND07 & 39667 & 26225 & 156235 & 2874  & 768   & 89 \\
	  \hline
	  \end{tabular}%
  \end{table}%
\fi

\iffalse
% Please add the following required packages to your document preamble:
% \usepackage{multirow}
\begin{table*}[tb]
	\centering
	\caption{
\colorstuff{blue}{
  Routed Wirelength (×$10^3$) and Runtime (Seconds) Comparison on ISPD 2016 Benchmarks. }}
	\label{tab:sota_comparison_ispd2016}
  \resizebox{\textwidth}{!}{
  %\begin{threeparttable}
\begin{tabular}{|c|c|cc|cc|cc|cc|cc|cc|}
\hline
  \multirow{2}{*}{Design} & \multirow{2}{*}{\#LUT/\#FF/\#BRAM/\#DSP} & \multicolumn{2}{c|}{\texttt{UTPlaceF} \cite{PLACE_TCAD2018_Li}} & \multicolumn{2}{c|}{\texttt{RippleFPGA} \cite{TCAD18_RippleFPGA_Chen}} & \multicolumn{2}{c|}{\texttt{GPlace 3.0} \cite{TODAES18_GPlace3_Abuowaimer}} & \multicolumn{2}{c|}{\texttt{UTPlaceF-NEP} \cite{PLACE_TCAD2019_Li_UTPLACEF_DL}} & \multicolumn{2}{c|}{\texttt{elfPlace} (GPU) \cite{PLACE_TCAD2021_Meng}} & \multicolumn{2}{c|}{Ours (GPU)} \\
                        &                                         & WL             & RT           & WL              & RT            & WL              & RT            & WL               & RT             & WL              & RT             & WL             & RT           \\ \hline \hline
FPGA01                  & 50K/55K/0/0                             & 357            & 162          & 350             & 32            & 356             & 83            & 340              & 58             & 317             & 30             & 321            & 35           \\
FPGA02                  & 100K/66K/100/100                        & 642            & 273          & 682             & 58            & 644             & 158           & 653              & 89             & 581             & 45             & 581            & 56           \\
FPGA03                  & 250K/170K/600/500                       & 3215           & 778          & 3251            & 209           & 3101            & 587           & 3139             & 336            & 2865            & 110            & 2868          & 113          \\
FPGA04                  & 250K/172K/600/500                       & 5410           & 768          & 5492            & 286           & 5403            & 630           & 5331             & 349            & 4853            & 107            & 4871          & 110          \\
FPGA05                  & 250K/174K/600/500                       & 9660           & 973          & 9909            & 334           & 10507           & 736           & 10045            & 381            & 9206            & 111            & 9192          & 121          \\
FPGA06                  & 350K/352K/1000/600                      & 6488           & 1649         & 6145            & 518           & 5820            & 1189          & 5801             & 596            & 5699            & 201            & 5714          & 237          \\
FPGA07                  & 350K/355K/1000/600                      & 10105          & 1647         & 9577            & 558           & 9509            & 1277          & 9356             & 597            & 8740            & 195            & 8620          & 202          \\
FPGA08                  & 500K/216K/600/500                       & 7879           & 1642         & 8088            & 412           & 8126            & 1400          & 8298             & 273            & 7676            & 157            & 7414          & 174          \\
FPGA09                  & 500K/366K/1000/600                      & 12369          & 2483         & 11376           & 662           & 11711           & 1848          & 11633            & 346            & 10650           & 214            & 10644         & 252          \\
FPGA10                  & 350K/600K/1000/600                      & 8795           & 3043         & 6972            & 1002          & 6836            & 1794          & 6317             & 353            & 6068            & 228            & 5983          & 249          \\
FPGA11                  & 480K/363K/1000/400                      & 10196          & 2044         & 10918           & 628           & 10260           & 1709          & 10476            & 309            & 10432           & 178            & 10413         & 209          \\
FPGA12                  & 500K/602K/600/500                       & 7755           & 2934         & 7240            & 847           & 7224            & 2263          & 6835             & 418            & 6484            & 223            & 6550          & 284          \\ \hline \hline
Ratio                   &                                         & 1.14           & 8.18         & 1.11            & 2.43          & 1.09            & 6.02          & 1.07             & 2.11           & 1.00            & 0.89           & 1.00           & 1.00         \\ \hline
\end{tabular}%
  %  \begin{tablenotes}
  %    Results for other placers are from the most recent publication \cite{PLACE_TCAD2021_Meng}, collected from a Linux machine with Intel Xeon Gold 6230 CPU (2.10GHz and 20 physical cores) and 64 GiB RAM.
  %  \end{tablenotes}
  %\end{threeparttable}
  }
\end{table*}
\fi

% Table generated by Excel2LaTeX from sheet 'Sheet2'
\begin{table*}[tb]
	\centering
	\caption{Routed Wirelength (×$10^3$) and Runtime (Seconds) Comparison on ISPD 2017 Benchmarks.}
	\label{tab:sota_comparison_ispd2017}
  \resizebox{\textwidth}{!}{
\begin{tabular}{|c|cc|cc|cc|cc|cc|cc|}
\hline
  \multirow{2}{*}{Design} & \multirow{2}{*}{\#LUT/\#FF/\#BRAM/\#DSP} & \multirow{2}{*}{\#Clock} & \multicolumn{2}{c|}{\texttt{UTPlaceF 2.0} \cite{PLACE_TODAES2018_Li_UTPlaceF2}} & \multicolumn{2}{c|}{\texttt{RippleFPGA} \cite{PLACE_ICCAD2017_Pui_RippleFPGA}} & \multicolumn{2}{c|}{\texttt{UTPlaceF 2.X} \cite{PLACE_FPGA2019_Li}} & \multicolumn{2}{c|}{\texttt{NTUfplace} \cite{PLACE_TCAD2020_Chen}} & \multicolumn{2}{c|}{Ours (GPU)} \\
                        &                                         &                          & RWL              & RT              & RWL             & RT             & RWL    & \multicolumn{1}{c|}{RT}   & RWL             & RT            & RWL              & RT            \\ \hline \hline
\texttt{CLK-FGPA01}              & 211K/324K/164/75                        & 32                       & 2208            & 532             & 2011           & 288            & 2092  & \multicolumn{1}{c|}{180}  & 2039           & 698           & 1868           & 136           \\
\texttt{CLK-FGPA02}              & 230K/280K/236/112                       & 35                       & 2279            & 513             & 2168           & 266            & 2194  & \multicolumn{1}{c|}{179}  & 2149           & 710           & 2011           & 130           \\
\texttt{CLK-FGPA03}              & 410K/481K/850/395                       & 57                       & 5353            & 1039            & 5265           & 583            & 5109  & \multicolumn{1}{c|}{343}  & 4901           & 1704          & 4755           & 215           \\
\texttt{CLK-FGPA04}              & 309K/372K/467/224                       & 44                       & 3698            & 711             & 3607           & 380            & 3600  & \multicolumn{1}{c|}{242}  & 3614           & 1148          & 3338           & 162           \\
\texttt{CLK-FGPA05}              & 393K/469K/798/150                       & 56                       & 4692            & 939             & 4660           & 569            & 4556  & \multicolumn{1}{c|}{323}  & 4417           & 1540          & 4154           & 208           \\
\texttt{CLK-FGPA06}              & 425K/511K/872/420                       & 58                       & 5589            & 1066            & 5737           & 591            & 5432  & \multicolumn{1}{c|}{346}  & 5122           & 2210          & 4918           & 229           \\
\texttt{CLK-FGPA07}              & 254K/309K/313/149                       & 38                       & 2444            & 845             & 2326           & 304            & 2324  & \multicolumn{1}{c|}{201}  & 2320           & 795           & 2145           & 141           \\
\texttt{CLK-FGPA08}              & 212K/257K/161/75                        & 32                       & 1886            & 529             & 1778           & 247            & 1807  & \multicolumn{1}{c|}{169}  & 1803           & 588           & 1648           & 120           \\
\texttt{CLK-FGPA09}              & 231K/358K/236/112                       & 35                       & 2601            & 842             & 2530           & 327            & 2507  & \multicolumn{1}{c|}{197}  & 2436           & 717           & 2248           & 144           \\
\texttt{CLK-FGPA10}              & 327K/506K/542/255                       & 47                       & 4464            & 974             & 4496           & 512            & 4229  & \multicolumn{1}{c|}{286}  & 4339           & 1597          & 3839           & 200           \\
\texttt{CLK-FGPA11}              & 300K/468K/454/224                       & 44                       & 4183            & 1068            & 4190           & 455            & 3936  & \multicolumn{1}{c|}{265}  & 3964           & 1618          & 3626           & 183           \\
\texttt{CLK-FGPA12}              & 277K/430K/389/187                       & 41                       & 3369            & 774             & 3388           & 409            & 3236  & \multicolumn{1}{c|}{247}  & 3179           & 849           & 2938           & 168           \\
\texttt{CLK-FGPA13}              & 339K/405K/570/262                       & 47                       & 3816            & 1172            & 3833           & 441            & 3723  & \multicolumn{1}{c|}{270}  & 3680           & 985           & 3404           & 181           \\ 
  \hline \hline
Ratio                   &                                         &                          & 1.142           & 4.943           & 1.117          & 2.379          & 1.096 & 1.453                     & 1.079          & 6.575         & 1.000           & 1.000         \\ \hline
\end{tabular}%
  }
  \end{table*}%

% \begin{table*}[tb]
% 	\centering
%   \caption{HPWL ($\times 10^3$) and Runtime (Seconds) Comparison with Different Techniques on Industry Benchmarks.}
% 	\label{tab:industry_results}%
% 	\resizebox{\textwidth}{!}{
% \begin{tabular}{|c|ccc|cc|cc|cc|cc|}
% \hline
% \multirow{2}{*}{Design} & \multirow{2}{*}{\#LUT/\#FF/\#BRAM/\#DSP} & \multirow{2}{*}{\#Distributed RAM+\#SHIFT} & \multirow{2}{*}{\#CARRY} & \multicolumn{2}{c|}{\begin{tabular}[c]{@{}c@{}}w/o precond \\ or chain align\\ (GPU)\end{tabular}} & \multicolumn{2}{c|}{\begin{tabular}[c]{@{}c@{}}w/o precond\\ (GPU)\end{tabular}} & \multicolumn{2}{c|}{\begin{tabular}[c]{@{}c@{}}w/o chain align\\ (GPU)\end{tabular}} & \multicolumn{2}{c|}{\begin{tabular}[c]{@{}c@{}}Ours\\ (GPU)\end{tabular}} \\
%                         &                                         &                           &                        & WL                                               & RT                                              & WL                                      & RT                                     & WL                                        & RT                                       & WL                                  & RT                                  \\ \hline \hline
% IND01                   & 17K/11K/0/13                            & 9                         & 2K                     & 87                                               & 123                                             & 76                                      & 131                                    & 76                                        & 44                                       & 65                                  & 45                                  \\
% IND02                   & 11K/10K/0/24                            & 6                         & 335                    & 86                                               & 75                                              & 78                                      & 80                                     & 83                                        & 54                                       & 77                                  & 54                                  \\
% IND03                   & 109K/12K/0/0                            & 0                         & 0                      & 611                                              & 55                                              & 600                                     & 56                                     & 602                                       & 58                                       & 597                                 & 59                                  \\
% IND04                   & 29K/17K/0/16                            & 218                       & 1K                     & 200                                              & 181                                             & 186                                     & 156                                    & 174                                       & 84                                       & 165                                 & 83                                  \\
% IND05                   & 64K/191K/64/928                         & 29K                       & 4K                     & 1228                                            & 173                                             & 1152                                   & 187                                    & 1040                                     & 100                                      & 998                                 & 97                                  \\
% IND06                   & 112K/65K/21/0                           & 0                         & 4K                     & 967                                              & 166                                             & 926                                     & 203                                    & 861                                       & 82                                       & 792                                 & 84                                  \\
% IND07                   & 40K/156K/89/768                         & 26K                       & 3K                     & diverge                                          & diverge                                         & diverge                                 & diverge                                & 761                                       & 81                                       & 745                                 & 83                                  \\ \hline \hline
% Ratio                   &                                         &                           &                        & 1.193                                            & 1.841                                           & 1.107                                   & 1.935                                  & 1.066                                     & 0.996                                    & 1.000                               & 1.000                               \\ \hline
% \end{tabular}
% 	}
%   \end{table*}%

% Table generated by Excel2LaTeX from sheet 'Sheet5'
\begin{table*}[tb]
  \centering
  \caption{Routed Wirelength ($\times 10^3$), WNS($\times 10^3$ps), TNS ($\times 10^5$ps) and Runtime (Seconds) Comparison between Conference Version and Our Algorithm on Industry Benchmarks.}
  \label{tab:industry_timing_results}
  \resizebox{\textwidth}{!}{
    \begin{tabular}{|c|cccc|cccc|cccc|}
      \hline
      \multirow{2}[2]{*}{Design} & \multirow{2}[2]{*}{\#LUT/\#FF/\#BRAM/\#DSP} & \multicolumn{1}{c}{\multirow{2}[2]{*}{\#Distributed \newline{}RAM+\#SHIFT}} & \multirow{2}[2]{*}{\#CARRY} & \multicolumn{1}{c|}{\multirow{2}[2]{*}{Clock\newline{}Period}} & \multicolumn{4}{c|}{Conference Version \cite{PLACE_DAC22_Mai}} & \multicolumn{4}{c|}{Ours (GPU)} \\
          &       &       &       &       & RWL   & WNS   & TNS   & RT    & RWL   & WNS   & TNS   & RT \\
      \hline \hline
      \texttt{IND01} & 17K/11K/0/13 & 9     & 2K    & 5     & 90    & -3.941 & -2.422 & 45    & 90    & -1.751 & -1.323 & 70 \\
      \texttt{IND02} & 11K/10K/0/24 & 6     & 335   & 2     & 102   & -4.240 & -19.733 & 54    & 117   & -2.938 & -18.148 & 76 \\
      \texttt{IND03} & 109K/12K/0/0 & 0     & 0     & 3     & 1028  & -2.666 & -18.467 & 59    & 1031  & -2.764 & -16.836 & 144 \\
      \texttt{IND04} & 29K/17K/0/16 & 218   & 1K    & 5     & 279   & -5.399 & -27.915 & 83    & 283   & -6.382 & -21.112 & 88 \\
      \texttt{IND05} & 64K/191K/64/928 & 29K   & 4K    & 10    & 2305  & -10.306 & -3.009 & 97    & 2312  & -4.558 & -2.354 & 218 \\
      \texttt{IND06} & 112K/65K/21/0 & 0     & 4K    & 15    & 1585  & -10.987 & -106.384 & 84    & 1585  & -6.502 & -51.922 & 193 \\
      \texttt{IND07} & 40K/156K/89/768 & 26K   & 3K    & 4     & 1498  & -6.302 & -20.585 & 83    & 1505  & -6.039 & -21.030 & 265 \\
      \hline \hline
    Ratio &       &       &       &       & 1.000  & 1.000  & 1.000  & 1.000  & 1.025  & 0.764  & 0.775  & 2.029 \\
      \hline
    \end{tabular}%
    }
\end{table*}%

% Table generated by Excel2LaTeX from sheet 'Sheet3'
\begin{table*}[htbp]
  \centering
  \caption{Routed Wirelength($\times 10^3$), WNS ($\times 10^3$ ps), TNS ($\times 10^5$ ps), and Runtime (Seconds) Comparison with Different Techniques on Industry Benchmarks.}
  \label{tab:industry_wl_results}%
  \resizebox{\textwidth}{!}{
    \begin{tabular}{|c|cccc|cccc|cccc|cccc|}
    \hline
    \multirow{3}[2]{*}{Design} & \multicolumn{4}{c|}{\multirow{2}[1]{*}{w/o precond or chain align\newline{}(GPU)}} & \multicolumn{4}{c|}{\multirow{2}[1]{*}{w/o precond\newline{}(GPU)}} & \multicolumn{4}{c|}{\multirow{2}[1]{*}{w/o chain align\newline{}(GPU)}} & \multicolumn{4}{c|}{\multirow{2}[1]{*}{Ours\newline{}(GPU)}} \\
          & \multicolumn{4}{c|}{}         & \multicolumn{4}{c|}{}         & \multicolumn{4}{c|}{}         & \multicolumn{4}{c|}{} \\
          & RWL   & WNS   & TNS   & RT    & RWL   & WNS   & TNS   & RT    & RWL   & WNS   & TNS   & RT    & RWL   & WNS   & TNS   & RT \\
    \hline \hline
      \texttt{IND01} & 103   & -3.491 & -2.849 & 53    & 94    & -3.041 & -1.826 & 66    & 108   & -2.581 & -143.009 & 74    & 90    & -1.751 & -1.323 & 70 \\
      \texttt{IND02} & 124   & -4.580 & -28.426 & 148   & 180   & -4.827 & -43.930 & 64    & 118   & -6.449 & -6029.390 & 73    & 117   & -2.938 & -18.148 & 76 \\
      \texttt{IND03} & 1021  & -3.080 & -17.301 & 123   & 1021  & -3.080 & -17.301 & 126   & 1030  & -2.604 & -1776.630 & 125   & 1031  & -2.764 & -16.836 & 144 \\
      \texttt{IND04} & 377   & -9.384 & -28.074 & 88    & 392   & -8.808 & -31.798 & 90    & 290   & -4.288 & -610.918 & 76    & 283   & -6.382 & -21.112 & 88 \\
      \texttt{IND05} & 2290  & -4.502 & -3.702 & 137   & diverge & diverge & diverge & diverge & 2260  & -4.870 & -238.678 & 153   & 2312  & -4.558 & -2.354 & 218 \\
      \texttt{IND06} & 1558  & -50.647 & -105.558 & 109   & 1576  & -11.450 & -113.733 & 107   & 1580  & -15.977 & -5320.100 & 114   & 1585  & -6.502 & -51.922 & 193 \\
      \texttt{IND07} & 1547  & -9.470 & -31.747 & 186   & 1393  & -8.145 & -20.816 & 155   & diverge & diverge & diverge & diverge & 1505  & -6.039 & -21.030 & 265 \\
    \hline\hline
    Ratio & 1.076 & 2.355 & 1.599 & 0.922 & 1.147 & 1.497 & 1.586 & 0.801 & 1.034 & 1.468 & 1.298 & 0.837 & 1.000 & 1.000 & 1.000 & 1.000 \\
    \hline
    \end{tabular}%
  }
\end{table*}%

\iffalse
\begin{table*}[tb]
  \centering
  \caption{Routed Wirelength($\times 10^3$), Routed Net Ratio(\%) and Runtime (Seconds) Comparison with Different Techniques on Industry Benchmarks.}
  \label{tab:industry_wl_results}
	  \begin{tabular}{|c|ccc|ccc|ccc|ccc|}
	  \hline
	  \multirow{3}[2]{*}{Design} & \multicolumn{3}{c|}{\multirow{2}[1]{*}{w/o precond or chain align\newline{}(GPU)}} & \multicolumn{3}{c|}{\multirow{2}[1]{*}{w/o precond\newline{}(GPU)}} & \multicolumn{3}{c|}{\multirow{2}[1]{*}{w/o chain align\newline{}(GPU)}} & \multicolumn{3}{c|}{\multirow{2}[1]{*}{Ours\newline{}(GPU)}} \\
			& \multicolumn{3}{c|}{} & \multicolumn{3}{c|}{} & \multicolumn{3}{c|}{} & \multicolumn{3}{c|}{} \\
			& WL    & RNR   & RT    & WL    & RNR   & RT    & WL    & RNR   & RT    & WL    & RNR   & RT \\
	  \hline \hline
	  \texttt{IND01} &       &       &       &       &       &       &       &       &       &       &       &  \\
	  \texttt{IND02} &       &       &       &       &       &       &       &       &       &       &       &  \\
	  \texttt{IND03} &       &       &       &       &       &       &       &       &       &       &       &  \\
	  \texttt{IND04} &       &       &       &       &       &       &       &       &       &       &       &  \\
	  \texttt{IND05} &       &       &       &       &       &       &       &       &       &       &       &  \\
	  \texttt{IND06} &       &       &       &       &       &       &       &       &       &       &       &  \\
	  \texttt{IND07} &       &       &       &       &       &       &       &       &       &       &       &  \\
	  \hline \hline
	  Ratio &       &       &       &       &       &       &       &       &       &       &       &  \\
	  \hline
	  \end{tabular}%
\end{table*}%
\fi

\iffalse
\begin{table}[tb]
	\centering
	\caption{Normalized Routed Wirelength and Placement Runtime Comparison for Dynamically Adjusted Preconditioning Technique.}
  \begin{threeparttable}
	  \begin{tabular}{|c|cc|cc|}
		\hline
	  \multirow{2}[2]{*}{Design} & \multicolumn{2}{c|}{w/ \cite{PLACE_TCAD2021_Meng} Precond.} & \multicolumn{2}{c|}{Ours} \\
			& WLR   & RTR   & WLR   & RTR \\
	  \hline \hline
	  CLK-FPGA01 & 1.002 & 0.79  & 1.000 & 1.00 \\
	  CLK-FPGA02 & *     & *     & 1.000 & 1.00 \\
	  CLK-FPGA03 & *     & *     & 1.000 & 1.00 \\
	  CLK-FPGA04 & *     & *     & 1.000 & 1.00 \\
	  CLK-FPGA05 & *     & *     & 1.000 & 1.00 \\
	  CLK-FPGA06 & *     & *     & 1.000 & 1.00 \\
	  CLK-FPGA07 & 1.002 & 1.01  & 1.000 & 1.00 \\
	  CLK-FPGA08 & 0.987 & 0.62  & 1.000 & 1.00 \\
	  CLK-FPGA09 & 0.999 & 1.07  & 1.000 & 1.00 \\
	  CLK-FPGA10 & *     & *     & 1.000 & 1.00 \\
	  CLK-FPGA11 & *     & *     & 1.000 & 1.00 \\
	  CLK-FPGA12 & 1.004 & 1.37  & 1.000 & 1.00 \\
	  CLK-FPGA13 & *     & *     & 1.000 & 1.00 \\
	  \hline
	  \end{tabular}%
    \begin{tablenotes}
    \item[*] denotes diverged during global placement.
    \end{tablenotes}
  \end{threeparttable}
	\label{tab:precondition_comparison}%
\end{table}%

\begin{table*}[tb]
\begin{tabular}{|c||c|c|c|c|}
\hline
\multirow{2}{*}{Design} &  \multicolumn{4}{c|}{CC=24} \\
\cline{2-5} & WL & WLR & RT & RTR \\
\hline
\hline
CLK-FPGA01 & 1 & 2 & 3 & 4 \\
\hline
\end{tabular}
\caption{Runtime \& = Comparisons between different values of CC \& Half Column Count}
\end{table*}
\fi

% \textcolor{red}{(Jing) Some reviewers comment that the comparison with ElfPlace is missing. Is it necessary to explain the differences?}
% We can see that our placer consistently achieves better routed wirelength than other placers,
% i.e., 14.2\% smaller than \texttt{UTPlaceF 2.0}, 11.7\% smaller than \texttt{RippleFPGA}, 9.6\% smaller than \texttt{UTPlaceF 2.X}, and 7.9\% smaller than \texttt{NTUfplace} on average, respectively.
% For some benchmarks like \texttt{CLK-FPGA10},
% our wirelength is even 15.1\%, 15.9\%, and 11.8\% better than that of the three baseline placers, respectively.
%It needs to be mentioned that \texttt{UTPlaceF 2.X} is a follow-up work to \texttt{UTPlaceF 2.0}, which relaxes the bounding-box clock constraints to real clock tree constraints.
%It enables larger solution space for clock feasibility,
%which means intuitively it should be easier to find better solutions.
%As \texttt{UTPlaceF 2.X} targets at a relaxed solution space, which intuitively should be easier to find better solutions.
%However, even in such an unfair comparison, we can achieve 9.6\% better routed wirelength than \texttt{UTPlaceF 2.X}, demonstrating the effectiveness of our algorithm.
% Meanwhile, our placer is the fastest one with GPU acceleration,
% specifically, 4.94$\times$ faster than \texttt{UTPlaceF 2.0}, 2.38$\times$ faster than \texttt{RippleFPGA}, 1.45$\times$ faster than \texttt{UTPlaceF 2.X}, and 6.58$\times$ faster than \texttt{NTUfplace}.

The experimental results show that our placement algorithm consistently achieves better routed wirelengths than other placers.
Specifically, our placer achieves 14.2\% smaller routed wirelength than \texttt{UTPlaceF 2.0},
11.7\% smaller than \texttt{RippleFPGA},
9.6\% smaller than \texttt{UTPlaceF 2.X},
and 7.9\% smaller than \texttt{NTUfplace} on average, respectively.
For some benchmarks, like \texttt{CLK-FPGA06}, which is a large design with about 925K cells and 58 clock nets,
our routed wirelength is even 13.6\%, 6.7\%, 10.5\% and 4.1\% better than that of baseline placers, respectively.
It needs to be mentioned that \texttt{UTPlaceF 2.X} is a follow-up work to \texttt{UTPlaceF 2.0}, which relaxes the clock region bounding box constraints to clock tree constraints.
It allows for a larger solution space for clock routing feasibility and thus should yield better results.
However, even in this unfair comparison, our routed wirelength is still 9.6\% better than \texttt{UTPlaceF 2.X}, exhibiting the efficacy of our algorithm.
Besides, our GPU-accelerated placer is the fastest one with 4.94$\times$, 2.38$\times$, 1.45$\times$, and 6.58$\times$ speedup over other placers, respectively.
These experiments demonstrate the effectiveness and efficiency of our proposed algorithms.

\subsection{Evaluation on Industry Benchmarks}
% We further validate our placer on industrial benchmarks,
% which contain a full set of heterogeneous instances including distributed RAM, SHIFT, and CARRY (see \tabRef{tab:industry_timing_results}).
% \tabRef{tab:industry_timing_results} compared the routed WNS and TNS between the conference version \cite{PLACE_DAC22_Mai} and ours.
% To the best of our knowledge,
% no known academic placers can fully handle such a set of instances with SLICEL-SLICEM heterogeneity yet.
We further evaluate our placer on industrial benchmarks which consist of a
comprehensive instance set and an industrial FPGA architecture from real-world
industry scenarios, including SHIFT, distributed RAM, and CARRY (see
\tabRef{tab:industry_timing_results}).
%
Most previous FPGA placers \cite{ TCAD18_RippleFPGA_Chen, PLACE_ICCAD2016_Ryan,
PLACE_TCAD2018_Li, PLACE_TCAD2021_Meng, TODAES18_GPlace3_Abuowaimer,
PLACE_ICCAD2017_Pui_RippleFPGA, PLACE_TODAES2018_Li_UTPlaceF2,
PLACE_FPGA2019_Li, PLACE_TCAD2020_Chen, PLACE_DATE2021_Lin } cannot fully
handle such an instance set with SLICEL-SLICEM heterogeneity.
%
To better validate the effectiveness of our placer, we leverage a high-quality
FPGA router to evaluate the placement algorithms more precisely \cite{ROUTE_ASPDAC2023_Wang}.
% To the best of our knowledge, no known academic placers can fully handle such
% an instance set with SLICEL-SLICEM heterogeneity yet.

From \tabRef{tab:industry_timing_results}, we can see the comparison between the conference version \cite{PLACE_DAC22_Mai} and our placer.
Our placer can achieve 23.6\% better WNS and 22.5\% better TNS, respectively, with minor routed wirelength degeneration, exhibiting the effectiveness of our placer.
In some benchmarks, such as \texttt{IND06}, which is one of the most congestion benchmarks,
our WNS and TNS are 40.8\%  and 51.2\% better than the baseline, respectively.
These experiments demonstrate that our placer can effectively optimize timing even on congested benchmarks.
The experiments also show that our placer requires more time  to converge on industrial benchmarks. 
This is because optimizing timing requires additional iterations, which needs further optimization in the future.

\subsection{Ablation Study for Optimization Techniques}
% To better understand the performance of our placer,
% We implement a high-quality FGPA router to evaluate the routed wirelength and the timing situation
% and validate the effectiveness of our placer by disabling specific optimization techniques as following.
% To the best of our knowledge,
% no known academic placers can fully handle such a set of instances with SLICEL-SLICEM heterogeneity yet.
% We cannot evaluate the routed wirelength either due to the incompatibility with the Vivado patch for ISPD 2017 benchmarks.
% Thus, we evaluate half-perimeter wirelength (HPWL) and validate the effectiveness of our placer by disabling specific optimization techniques as following.
To better understand the performance of our placer,
we perform an ablation study and validate the effectiveness of our proposed methods by disabling optimization techniques as follows (see \tabRef{tab:industry_wl_results}).
\begin{itemize}
\item Disable the iterative carry chain alignment correction in global placement and only align chains once in legalization.
\item Disable the dynamic preconditioner and use the default preconditioner in \cite{PLACE_TCAD2021_Meng, PLACE_TCAD2018_Cheng}.
\item Disable both techniques as the baseline.
\end{itemize}
% We can see from \tabRef{tab:industry_wl_results} that the dynamically-adjusted preconditioner (\secRef{sec:preconditioning})
% stabilizes the global placement iterations and enables better solution quality at convergence,
% while the default preconditioner \cite{PLACE_TCAD2021_Meng, PLACE_TCAD2018_Cheng} diverges on one design.
% The proposed preconditioner also enables faster convergence, reducing the runtime by more than 80\%.
% Moreover, the iterative carry chain alignment correction helps achieve 6.6\% better wirelength with minor runtime overhead.
% Moreover, the iterative carry chain alignment correction helps achieve 3.4\% better routed wirelength as well as 46.8\% better worst negative slacks.
% This experiment validates the efficiency and effectiveness of the proposed preconditioning and carry chain alignment techniques.
We can see from \tabRef{tab:industry_wl_results} that both the iterative carry chain alignment correction technique
and dynamically-adjusted precondition technique helps stabilizes the global placement convergence and enables better
placement solution at convergence.
Without either of these two techniques, the placer fails to converge on one design.
The dynamically-adjusted preconditioner helps achieve 14.7\% better routed wirelength, 49.7\% better WNS, and 58.6\% better TNS
compared with the default preconditioner \cite{PLACE_TCAD2021_Meng, PLACE_TCAD2018_Cheng}.
Moreover, the iterative carry chain alignment correction helps optimize the routed wirelength, WNS,
and TNS by 3.4\%, 46.8\%, and 29.8\% with minor runtime overhead.
These experiments validate the effectiveness and efficacy of the proposed carry chain alignment and preconditioning technique.

\subsection{Runtime Breakdown}

% We further report the runtime breakdown of our algorithm in \figRef{fig:runtime_breakdown}.
% With GPU acceleration, unlike most placers, global placement is no longer the runtime bottleneck.
% The core steps of global placement, i.e., the calculation of wirelength and electrostatic density, as well as their gradients, take only 43\% of the global placement runtime.
% Legalization is the most time consuming part, taking 53\% of the total runtime.
% The clock constraint related parts are relatively fast.
%The clock penalty term and the clock network planning take around 2\% of the total time,
%proving that our algorithm achieves clock legality without sacrificing efficiency.
% Miscellaneous parts, such as disk IO, parsing, and data initialization, take up almost the same time as global placement,
% which need further optimization in the future.
To analyze the time consumption of our placer, we further exhibit the runtime breakdown of our algorithm on one of the largest benchmarks as shown in \figRef{fig:runtime_breakdown}.
%
With the help of GPU acceleration, global placement is no longer the most time-consuming part.
The core forward and backward propagation of the global placement, i.e., the computation of wirelength and electrostatic density, as well as their gradients,
only take up 26\% and 17\% of the global placement runtime, respectively.
With the great reduction in global placement runtime,
legalization becomes the new runtime bottleneck, taking 53\% of the total placement time.
Meanwhile, other miscellaneous parts, including IOs, parsing, database establishment, etc., take up also the
same time as that of global placement.

\begin{figure}[tb]
    \centering
    \includegraphics[width=0.8\linewidth]{figs/runtime_breakdown2.pdf}
    \caption{Runtime breakdown on \texttt{CLK-FPGA13}.
    Similar distributions are observed on other benchmarks.
    }
    \label{fig:runtime_breakdown}
    \vspace{-.25in}
\end{figure}
