\begin{abstract}
%In recent advances, electrostatic-based placement algorithms for FPGA have demonstrated promising performance.
%Some recent works take clock feasibility constraints into account, but none of them are electrostatics-based.
%Modern field-programmable gate arrays (FPGA) often contain limited clock resources, which must be considered to produce clock routable placement results.

% Modern field-programmable gate arrays (FPGA) placement is challenged with limited clock resources.
% Recent advances in electrostatics-based placement algorithms demonstrate promising performance when ignoring the clock feasibility constraints.
% This work proposes an electrostatics-based placement algorithm that considers clock feasibility,
% with a robust clock region assignment algorithm and a smooth quadratic penalty for guiding instances to clock-feasible regions.
% Experiments on the ISPD 2017 benchmarks demonstrate that this work can achieve better routed-wirelength and efficiency compared with other state-of-the-art placers.

% Modern field-programmable gate arrays (FPGA) usually consist of heterogeneous CLBs like LUTs and LUTRAMs, but few academic FPFA placers have considered the SLICEL-SLICEM heterogeneity.
% Due to the limited clock routing resources, FPGA placement is also challenging when considering clock feasibility.
% Recent advances in electrostatics-based placement algorithms demonstrate promising performance when ignoring the clock feasibility constraints.
% In this work, we propose an electrostatics-based placement algorithm that considers the SLICEL-SLICEM heterogeneity and the clock feasibility. We propose a proper modeling method to handle the heterogeneous CLBs, a robust clock region assignment algorithm, and a smooth quadratic penalty method to guide instances to clock-feasible regions.
% Experiments on the academic and industrial benchmarks demonstrate that this work can achieve better routed wirelength and efficiency than other state-of-the-art placers.

% Modern field-programmable gate arrays (FPGAs) contain heterogeneous resources, including CLB, DSP, BRAM, IO, etc.
% A Configurable Logic Block (CLB) slice is further categorized to SLICEL and SLICEM, which can be configured as specific combinations of instances in \{LUT, FF, distributed RAM, SHIFT, CARRY\}.
% Such kind of heterogeneity challenges the existing FPGA placement algorithms.
% Meanwhile, limited clock routing resources also lead to complicated clock constraints, causing difficulties in achieving clock-feasible placement solutions.
% Traditional FPGA placers seldom consider the
% In this work, we propose a heterogeneous FPGA placement framework considering SLICEL-SLICEM heterogeneity based on a multi-electrostatic formulation with timing awareness and clock feasibility.
% We support a comprehensive set of the aforementioned instance types
% with a uniform algorithm for wirelength, routability, timing and clock optimization.
% Experimental results on both academic and industrial benchmarks demonstrate that we outperform the state-of-the-art placers in both quality and efficiency.

% (113 words)
% Modern field-programmable gate arrays (FPGAs) contain heterogeneous resources, including CLB, DSP, BRAM, IO, etc.
% A configurable logic block (CLB) slice is further categorized to SLICEL and SLICEM, which can be configured as specific combinations of instances in \{LUT, FF, distributed RAM, SHIFT, CARRY\}.
% Such kind of SLICEL-SLICEM heterogeneity challenges existing FPGA placement algorithms.
% Meanwhile, limited clock routing resources also cause difficulties in achieving clock feasible placement solutions.
% In this work, we propose a multi-electrostatic FPGA placement framework considering the aforementioned SLICEL-SLICEM heterogeneity and clock feasibility, with a uniform algorithm for wirelength, routability, and clock optimization.
% Experimental results on academic and industrial benchmarks demonstrate that we outperform the SOTA placers in quality and efficiency.

% In the overall FGPA design flow, placement is a critical step that mainly determines the quality of the final design.
% With the increasingly complicated FPGA architecture, modern FPGA placement problem becomes a mixed optimization
% problem with multiple objectives, including wirelength, routability, timing closure and clock feasibility.
When modern FPGA architecture becomes increasingly complicated, modern FPGA
placement is a mixed optimization problem with multiple objectives, including
wirelength, routability, timing closure, and clock feasibility.
% Modern FPGAs design contain heterogeneous instances including LUT, FF, CARRY,
% DSP, BRAM, IO, etc.
% SLICE on modern FPGAs can be classified into SLICEL and SLICEM,
% which can be further configured as the particular combinations of \{LUT, FF, distclock-feasibleHIFT, CARRY\}.
Typical FPGA devices nowadays consist of heterogeneous SLICEs like SLICEL and
SLICEM. The resources of a SLICE can be configured to \{LUT, FF, distributed RAM,
SHIFT, CARRY\}.
% Such kind of heterogeneity challenges existing FPGA placement algorithms.
% Besides, with various constraints (e.g., timing, clock routing, chain alignment, etc.) imposed by advanced circuit designs,
% placing heterogeneous instances while satisfying the advanced constraints has become more challenging.
Besides such heterogeneity, advanced FPGA architectures also bring complicated
constraints like timing, clock routing, carry chain alignment, etc. The above
heterogeneity and constraints impose increasing challenges to FPGA placement
algorithms. 

In this work, we propose a multi-electrostatic FPGA placer considering
the aforementioned SLICEL-SLICEM heterogeneity under timing, clock routing and
carry chain alignment constraints. We first propose an effective SLICEL-SLICEM
heterogeneity model with a novel electrostatic-based density formulation.
%
We also design a dynamically adjusted preconditioning and carry chain alignment
technique to stabilize the optimization convergence. 
%
We then propose a timing-driven net weighting scheme to incorporate timing
optimization.
%
% To achieve effective clock routing violation elimination with minor quality
% degradation, we construct a instance-to-clock-region mapping considering the
% resource capacity of clock regions and perturbation to the placement, and
% propose a quadratic clock penalty function that eliminate the clock violation
% in continuous global placement engine.
% Finally, putting the aforementioned techniques together, 
Finally, we put forward a nested Lagrangian relaxation-based placement
framework to incorporate the optimization objectives of wirelength,
routability, timing, and clock feasibility. 
%
Experimental results on both academic and industrial benchmarks demonstrate
that our placer outperforms the state-of-the-art placers in quality and
efficiency. 
\end{abstract}
