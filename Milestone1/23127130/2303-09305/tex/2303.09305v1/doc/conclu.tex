\section{Conclusion}
\label{sec:Conclusion}
% In this paper, we propose a heterogeneous FPGA placement algorithm considering SLICEL-SLICEM heterogeneity, timing closure and clock feasibility.
In this paper, we present a heterogeneous FPGA placement algorithm that
considers the heterogeneity of SLICELs and SLICEMs, as well as  timing
closure and clock feasibility.
% Based on a new electrostatic formulation, we design a uniform optimization
% paradigm for wirelength, routability, timing optimization and clock
% feasibility supporting heterogeneous instance types, including LUT, FF, BRAM,
% DSP, distributed RAM, SHIFT, and CARRY.
% A new electrostatic formulation and a nested Lagrangian paradigm have been
% prposed to design a uniform optimization paradigm in which wirelength, routing,
% timing optimization and clock feasibility can be optimized uniformly for
% heterogeneous instance types, such as LUT, FF, BRAM, DSP, distributed RAM,
% SHIFT, and CARRY. 
A new electrostatic formulation and a nested Lagrangian paradigm have been
proposed to achieve uniform optimization of wirelength, routability, timing, and
clock feasibility for heterogeneous instance types, including LUT, FF, BRAM,
DSP, distributed RAM, SHIFT, and CARRY.
% We further propose a dynamically-adjusted preconditioner, a timing-aware
% net-weighting scheme, and a smooth clock penalization technique to ensure the
% placement converging to high-quality solutions.
Additionally, we propose a dynamically adjusted preconditioner, a timing-driven net-weighting scheme, and a smooth clock penalization technique in order to ensure that the placement is convergent to high-quality solutions.
% Experiments on ISPD 2017 contest benchmarks demonstrate that our placer can achieve 14.2\%, 11.7\%, 9.6\%, and 7.9\% better routed wirelength than the state-of-the-art placers like \texttt{UTPlaceF 2.0}, \texttt{RippleFPGA}, \texttt{UTPlaceF 2.X}, and \texttt{NTUfplace}, respectively, with 1.5-6$\times$ speedup leveraging GPU acceleration.
On the ISPD 2017 contest benchmark, experiments have revealed that our placer
can achieve 14.2\%, 11.7\%, 9.6\%, and 7.9\% better routed wirelengths compared
to the state-of-the-art placers, \texttt{UTPlaceF 2.0},
\texttt{RippleFPGA}, \texttt{UTPlaceF 2.X}, and \texttt{NTUfplace},
respectively, at 1.5-6$\times$ speedup leveraging GPU acceleration.
We conducted experiments on industrial benchmarks to prove that our algorithms
are capable of achieving 23.6\% better WNS and 22.5\% better TNS with about 2\% increase in routed wirelength.

% \section*{Acknowledge}
%
% This work was supported in part by the National Science Foundation of China (Grant No. 62034007 and No. 62141404) and the 111 Project (B18001).

