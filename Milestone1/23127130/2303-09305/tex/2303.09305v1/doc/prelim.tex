%\vspace{-.05in}
\section{Preliminaries}
\label{sec:Preliminary}

% In this section, we introduce the FPGA architecture and the background of multi-electrostatics-based FPGA placement. 
%We also discuss the clock architecture and the clock constraints for placement.
In this section, we primarily focus on the architecture of FPGAs and the methodology of multi-electrostatics-based FPGA placement.

% * outline
% - device architecture
%   - SLICEL-SLICEM heterogeneity
%   - TIming constraints
%   - Clock Constraints 

\subsection{Device Architecture}
% In this work, we target \textit{Xilinx UltraScale} series~\cite{DI_ULTRASCALE, DI_ULTRASCALE_CLB}, e.g., 
% \textit{UltraScale VU095}, as a representative FPGA architecture (illustrated in \figRef{fig:fpga_arch}). 
% A simplified version of this architecture is also used in ISPD 2016\&2017 FPGA placement contests 
% with a subset of instance types, i.e., \{LUT, FF, BRAM, DSP\}~\cije{BENCH_ISPD2016_PLACE,BENCH_ISPD2017_PLACE}. 
%As \figRef{fig:fpga_arch} shows, the \textcolor{blue}{simplified} architecture consists of CLBs, block RAMs (BRAMs), DSPs, and IOs. 
In this work, we use the \textit{Xilinx UltraScale} family~\cite{DI_ULTRASCALE, DI_ULTRASCALE_CLB},e.g., the \textit{UltraScale VU095} as a model FPGA design (illustrated in \figRef{fig:fpga_arch}).
The ISPD 2016 and 2017 FPGA placement challenges employed a condensed version of this architecture with a limited number of instance types, including LUT, FF, BRAM, and DSP~\cite{BENCH_ISPD2016_PLACE, BENCH_ISPD2017_PLACE}. 

\subsubsection{SLICEL-SLICEM heterogeneity}
\label{sec:clb_heterogeneity}
% Apart from regular LUTs, the architecture also has other LUT-like instances: distributed RAM and shift registers (SHIFT). 
% CLBs have two categories of slices, SLICEL and SLICEM, the structures of which are shown in \figRef{fig:CLB_slice}. 
% Because SLICEL and SLICEM have slightly difference on logic resources, they support different configurations. 
% Besides, different from other regular instances like DSP and BRAM, these LUT-like instances owns another asymmetrical placement constraints, i.e., 
% LUTs, FFs, and CARRYs can be placed in both SLICEL and SLICEM, 
% whereas distributed RAMs and SHIFTs can only be placed in SLICEM. 
% Moreover, if a SLICEM is configured as LUTs, then it cannot be used as distributed RAMs or SHIFTs; vice versa. 
Shift registers (SHIFT) and distributed RAMs are two additional LUT-like instances of the architecture in addition to regular LUTs. 
Slices in CLBs fall into two categories: SLICEL and SLICEM, whose architectures are depicted in \figRef{fig:CLB_slice}. 
Due to the modest differences in logic resources, SLICEL and SLICEM support various configurations.
% LUTs, FFs, and CARRYs, on the other hand, can be placed in both SLICEL and SLICEM, unlike SHIFT and distributed RAMs, which can only 
% be placed in SLICEM, making them different from other typical instances like DSPs and BRAMs. 
LUTs, FFs, and CARRYs can be placed in both SLICEL and SLICEM. 
%
On the other hand, SHIFT and distributed RAMs can only be placed in SLICEM, which sets them apart from other common instances such as DSPs and BRAMs. 
%
Additionally, a SLICEM cannot be used as SHIFTs or distributed RAMs if it is configured as LUTs; vice versa. 

\iffalse
\textcolor{blue}{
We also need to consider carry chain alignment, as CLB slices also contain resources for CARRY instances. 
A carry chain consists of CARRY instances connected by cascading wires from lower to upper bits in sequence. 
CARRY instances in the chain need to be aligned in a column of CLB slices in proper order according to the cascading wires. 
}
\fi
\subsubsection{Carry Chain Alignment Constraints}
% We also need to consider carry chain alignment for CARRY instances as a placement constraint.
% The CARRY instances are placed in CLB slices, 
% and a carry chain consists of a series of sequential CARRY instances that are connected by cascading wires from lower to upper bits. 
% The alignment constraint imposes that the CARRY instances in a chain need to be placed in the same column, and in successive CLB slices in proper order according to the cascading wires. 
As a placement constraint for CARRY instances, we also need to take carry chain alignment into account. 
A carry chain is made up of several consecutive CARRY instances connected by cascaded wires from the lower bits to the upper bits, 
and the CARRY instances are arranged in CLB slices.
According to the alignment constraint, each CARRY instance in a chain must be positioned in a single column and in subsequent slices in the correct sequence for the cascading wires.
%Carry chains  are widely used to improve the performance of the arithmetic circuits. 
%On the other hand, the CARRY resource is located in the CLB sites and arranged by column on the FPGA layout. 
%Correspondingly, placing a carry chain in successive CLBs in a column can benefit the subsequent routing and timing, since this shortens the wirelength and the cascaded wires can be connected directly without detouring to the switching boxes.

%Our target FPGA device is based on \textit{Xilinx UltraScale VU095}~\cite{DI_ULTRASCALE}, 
%which is also the device used in both ISPD 2016 and ISPD 2017 FPGA placement contest \cite{BENCH_ISPD2016_PLACE,BENCH_ISPD2017_PLACE}. This architecture consists of these site resources: configurable logic blocks (CLBs), DSP slices, block RAMs (BRAMs), and IOs. Some CLB's LUT resource can also be configured as distributed RAM or shift register (SHIFT). The cell instances supported by the architecture consists of LUTs, distributed RAMs, SHIFTs, FFs, DSPs and BRAMs.

\subsubsection{Clock Constraints}
\label{sec:clock_constraint}

% The target FPGA device is divided into $5 \times 8$ rectangular-shaped clock regions (CRs), as shown in \figRef{fig:fpga_arch}
% \footnote{
% Due to the page limit, \figRef{fig:fpga_arch} only contains part of the whole $5 \times 8$ CRs, but is sufficient for illustration.
% }.
% Each CR consists of columns of site resources. Each CR can be further subdivided into pairs of lower and upper half columns (HCs) of half-clock-region height horizontally.  
% The width of each HC is the same as that of two site columns except for some corner cases, as shown in \figRef{fig:fpga_arch}.
The target FPGA device owns $5 \times 8$ rectangular-shaped clock regions (CRs) in a grid manner, as shown in \figRef{fig:fpga_arch}. 
\footnote{
\figRef{fig:fpga_arch} only contains part of the whole $5 \times 8$ CRs, but is sufficient for illustration.
}.
Each CR is made up of columns of site resources and can be further horizontally subdivided into pairs of lower and upper half columns (HCs) of half-clock-region height. 
Except for a few corner cases, the width of each HC is the same as that of two site columns, as shown in \figRef{fig:fpga_arch}.
All the clock sinks within a half column are driven by the same leaf clock tracks.

\iffalse
The clock routing architecture consists of a two-layer network of routing tracks: A routing network and a distribution network. 
Both networks contain 24 horizontal and 24 vertical clocking tracks, and each CR has exactly 24 
horizontal routing (HR) clock tracks, 24 vertical routing (VR) clock tracks, 
24 horizontal distribution (HD) clock tracks, and 24 vertical distribution (HD) clock tracks.
More detailed clock architecture can be found in \cite{BENCH_ISPD2017_PLACE}.
\fi

% The clock architecture imposes two clock constraints on placement, \emph{the clock region constraint}, and \emph{the half column constraint}, as shown in \figRef{fig:cr_con}.
% The clock region constraint restricts the clock demand of 
% each clock region to be at most 24, 
% where the clock demand is defined as the total number of clock nets whose bounding boxes intersect with the clock region. 
% The half clock region constraint restricts the number of clock nets within the half column to be at most 12. 
The clock routing architecture imposes two clock constraints on placement, i.e., the \emph{clock region constraint} and the \emph{half column constraint}, as shown in \figRef{fig:cr_con}.
The clock region constraint limits each clock region's clock demand to a maximum  of 24 clock nets, where the clock demand is the total number of clock nets whose bounding boxes intersect with the clock region. 
The half-column constraint limits the number of clock nets within the half-column to a maximum of 12.

\iffalse

\subsection{Analytical Placement}
\label{sec:ana_pl}
At the global placement stage, an analytical placer minimizes the wirelength with density constraint, as formulated in Problem~\eqref{eq:analytical_prob}.
\begin{equation}
        \min_{\boldsymbol{x}, \boldsymbol{y}} \sum\limits_{e \in E}\text{WL}_e(\boldsymbol{x}, \boldsymbol{y})
        \quad \text{s.t. }\mathcal{D}(\boldsymbol{x}, \boldsymbol{y}) \leq \hat{\mathcal{D}}, 
        \label{eq:analytical_prob}
\end{equation}
where $\boldsymbol{x}$ and $\boldsymbol{y}$ denote the instances location vector, $E$ denotes the set of nets, $WL_e(\cdot)$ is the wirelength function for net $e \in E$, $\mathcal{D}(\cdot)$ is the instance density for a certain position on the layout, and $\hat{\mathcal{D}}$ denotes the instance density threshold defined by users.

The multiplier methods are a certain class of algorithms for solving a constrained optimization problem 
that replaces a constrained optimization problem with a series of unconstrained optimization problems by adding a penalty term to the objective function. In particular, at the global placement stage, the optimization problem can be converted as \eqRef{eq:analytical_multipler}.
\begin{equation}
    \min_{\boldsymbol{x}, \boldsymbol{y}} \sum\limits_{e \in E} \text{WL}_e(\boldsymbol{x}, \boldsymbol{y}) + \lambda \mathcal{D}(\boldsymbol{x}, \boldsymbol{y}), 
    \label{eq:analytical_multipler}
\end{equation}
where $\mathcal{D}(\cdot)$ denotes the density penalty that disperses the instances and $\lambda$ denotes the density multiplier. As the multiplier $\lambda$ increases, the density constraints are incrementally tightened, and we can achieve a smoother transition from wirelength optimization to wirelength and density co-optimization until finally the solution fully satisfies the density constraints.

\subsection{Electrostatics Based Placement}
Electrostatics-based placers are among the state-of-the-art family of global placement algorithms for ASICs that model the layout and instances as an electrostatic system and cast the placement density penalty to the electric potential energy of the electrostatic system. It formulates the global placement problem as Problem~\eqref{eq:eplace}.
\begin{equation}
    \min_{\boldsymbol{x},\boldsymbol{y}} \widetilde{W}(\boldsymbol{x},\boldsymbol{y}) \quad \text{s.t. } \Phi(\boldsymbol{x},\boldsymbol{y}) = 0, 
    \label{eq:eplace}
\end{equation}
where $\Phi(\cdot)$ denotes the electric potential energy of the electrostatic system.  $\widetilde{W}(\cdot)=\sum_{e \in E}\widetilde{W}_e(\cdot)$ denotes the total half perimeter wirelength (HPWL) of all nets approximated by the differentiable weighted-averaged (WA) wirelength model \cite{PLACE_DAC2011_Hsu}. 
\begin{equation}
    \widetilde{W}_e^x(\boldsymbol{x}, \boldsymbol{y})=\frac{\sum_{i \in e} x_{i} e^{\frac{x_{i}}{\gamma}}}{\sum_{i \in e} e^{\frac{x_{i}}{\gamma}}}-\frac{\sum_{i \in e} x_{i} e^{-\frac{x_{i}}{\gamma}}}{\sum_{i \in e} e^{-\frac{x_{i}}{\gamma}}}. 
    \label{eq:wawl}
\end{equation}
The WA wirelength on $x$-direction is defined as \eqRef{eq:wawl}, and that on $y$-direction is defined in a similar way.
$\gamma$ is a parameter that controls the smoothness and precision of the HPWL approximation. A larger $\gamma$ indicates a smoother but less accurate approximation of HPWL. Problem ~\eqref{eq:eplace} can be solved using the multiplier method, by formulating the constrained optimization problem ~\eqref{eq:eplace} as unconstrained problem ~\eqref{eq:eplace_unconstrained}:
\begin{equation}
    \min_{\boldsymbol{x},\boldsymbol{y}} \widetilde{W}(\boldsymbol{x},\boldsymbol{y})+\lambda \Phi(\boldsymbol{x},\boldsymbol{y}), 
    \label{eq:eplace_unconstrained}
\end{equation}
We progressively increase the density multiplier $\lambda$ and solve the resulting optimization subproblem, smoothly transiting from wirelength optimization to wirelength and density co-optimization, until finally reaching a solution that fully satisfies the density constraints.
\fi

\subsection{Multi-Electrostatics based FPGA Placement}
\label{sec:multi_electrostatics_placement}

%Different from ASICs, one major challenge of FPGA placement is its heterogeneity. Each cell/instance has a different resource type, and each resource type are constrained to limited locations.
%As for FPGA placement, each instance has its resource type like LUT, FF, DSP, and BRAM.
%As different resource types are only compatible with specified sites on the FPGA layout, each instance is only assignable to certain positions according to their resource category.

% As shown in \figRef{fig:electrostatics_analogy}, electrostatics-based placement models each instance as an electric particle in an electrostatic system. 
As shown in \figRef{fig:electrostatics_analogy}, electrostatics-based placement models each instance as an electric particle in an electrostatic system. 
% It was first proposed in ASIC placement \cite{PLACE_TODAES2015_Lu}, based on the fundamental physical insight that balanced charge
As firstly stated in the ASIC placement \cite{PLACE_TODAES2015_Lu}, \emph{minimizing potential energy} can
resolve density overflow in the layout.
% distribution in an electrostatic system contributes to a low potential energy, so
% \emph{minimizing potential energy can resolve density overflow and help spread
% instances in the layout}. 
The principle is based on the fundamental physical insight that a balanced charge
distribution in an electrostatic system contributes to low potential energy, so
\emph{minimizing potential energy can resolve density overflow and help spread
instances in the layout}. 
% This approach is then extended to multiple electrostatic fields, allowing multiple resource types in FPGA placement such as LUT, FF, DSP, and BRAM to be handled \cite{PLACE_TCAD2021_Meng}. 
We are also extending this approach to the use of multiple electrostatic fields, which will enable multiple types of resource to be handled in FPGA placement, such as LUTs, FFs, DSPs, and BRAMs.
% The multi-electrostatic formulation for LUT and DSP resources is shown in \figRef{fig:multi_electrostatics_exp}.
\figRef{fig:multi_electrostatics_exp} illustrates a multi-electrostatic formulation of LUTs and DSPs. 
% We can reduce density overflow by minimizing the total potential energy of multiple fields, because low energy means balanced distribution of instances.
In order to reduce density overflow, we must minimize the total potential energy of multiple fields because low energy means a balanced distribution of instances. 
% The issue can be stated as follows.
The issue can be summarized as follows.
% Electrostatics-based placement models each instance as an electric particle in an electrostatic system, as illustrated in \figRef{fig:electrostatics_analogy}. 
% It is originally proposed in ASIC placement \cite{PLACE_TODAES2015_Lu}, 
% leveraging the basic physical insight that balanced charge distribution in an electrostatic system contributes to low potential energy, 
% so 
% \emph{
% minimizing the potential energy can resolve density overflow and help spread instances in the layout.
%   }
% This approach is then extended to multiple electrostatic fields such that multiple resource types in FPGA placement like LUT, FF, DSP, and BRAM can be handled \cite{PLACE_TCAD2021_Meng}. 
% \figRef{fig:multi_electrostatics_exp} illustrates the multi-electrostatic formulation for LUT and DSP resources. 
% By minimizing the total potential energy of multiple fields, we can reduce the density overflow, as low energy means balanced distribution of instances. 
% The problem can be written as, 
\begin{equation}
    \min_{\boldsymbol{x}, \boldsymbol{y}} \widetilde{W}(\boldsymbol{x}, \boldsymbol{y}) \quad \text { s.t. } \Phi_s(\boldsymbol{x}, \boldsymbol{y}) = 0,
    \label{eq:elfplace}
\end{equation}
where $\widetilde{W}(\cdot)$ is the wirelength objective, 
$\boldsymbol{x}, \boldsymbol{y}$ are instance locations, 
$S$ denotes the field type set, 
and $\Phi_s(\cdot)$ is the electric potential energy for field type $s \in S$.
Formally, we constrain the target potential energy of each field type to be zero, though the energy is usually non-negative.
% We formally constrain the target energy $\Phi_s(\boldsymbol{x}, \boldsymbol{y})$ to 0, as the energy is usually non-negative.  
The constraints can be further relaxed to the objective and guide the instances to spread out. 
Practically, we stop the optimization when the energy is small enough; or equivalently, the density overflow is low enough.
Notice that the formulation in \figRef{fig:multi_electrostatics_exp} assumes that one instance occupies the resources of only one field, 
which cannot handle the complicated SLICEL-SLICEM heterogeneity shown in \secRef{sec:clb_heterogeneity}. 
%Under Formulation~\eqref{eq:elfplace}, different resource types of instances do not interact with other resource types in terms of the potential energy terms.
%Especially, The purpose of introducing instance areas $\mathcal{A}$ as parameters of $\Phi_s(\cdot;\cdot)$ is that area adjustment strategy is leveraged to estimate the routing demand of each instance. 
%Therefore, the routing demand is reflected on the adjusted instance areas, which will further make efforts on the density penalty term.

\subsection{Timing Optimization}
\label{sec:timing_optimization}
Timing-driven placement imposes more concern about timing closure than the total wirelength objective in wirelength-driven placement.
Worst negative slack (WNS) and total negative slack (TNS) are two widely adopted timing metrics. 
WNS is the maximum negative slack among all timing paths in the design, and TNS is the sum of all negative slacks of timing endpoints.
Thus, WNS and TNS are used to evaluate the timing performance of a design from the worst and the global view respectively,  and the smaller the WNS and TNS are, the worse the timing performance is. 
The timing-driven placement problem can be formulated as follows. 
 \begin{subequations}
   \label{eq:timing_placement}
  \begin{align}
    \min_{\boldsymbol{x}, \boldsymbol{y}} & \quad \mathcal{T}(\boldsymbol{x}, \boldsymbol{y}), \\
    \text{s.t.} & \quad \rho_s(\boldsymbol{x}, \boldsymbol{y}) \leq \hat{\rho}_s,\quad \forall s \in S,
  \end{align}
\end{subequations}
%
where $S$ is the instance type set, $\rho_s(\cdot)$ denotes the density for
instance type $s \in S$, and  $\hat{\rho}_s$ represents the target density for
instance type $s \in S$. 
%
The objective function $\mathcal{T}(\cdot)$ can be WNS, TNS, or the weighted sum of both.
%
Improving TNS requires collaborative optimization of all timing paths, and is
therefore suitable for the global placement stage. 
%
On the other hand, WNS is more suitable for the detailed placement stage, as it
only considers the worst timing path. 
%
It is worth noting that directly solving \eqRef{eq:timing_placement} is very
difficult, because the delay model generally has strong discrete and non-convex
properties \cite{PLACE_ISPD2002_Kahng}.
%
Therefore, we draw on the two intuitive elements of wirelength-driven
placement and static timing analysis, i.e., net weights and slacks, to tackle
this problem. 
% In this work, we focus on wirelength, timing and routability optimization, considering SLICEL-SLICEM heterogeneity and clock constraints. 
% We define the FPGA placement problem as follows.


\subsection{Problem Formulation}
\tabRef{tab:FPGAPlacers} summarizes the characteristics of the published
state-of-the-art FPGA placers. In recent years, modern FPGA placers mainly
resort to quadratic programming-based approaches~\cite{ PLACE_DAC2015_ShengYen,
PLACE_TCAD2018_Li, PLACE_ICCAD2016_Pui_RippleFPGA, PLACE_ICCAD2016_Ryan,
PLACE_TCAD2019_Li_UTPLACEF_DL, PLACE_TODAES2018_Li_UTPlaceF2,
PLACE_FPGA2019_Li, PLACE_ICCAD2017_Pui_RippleFPGA,
PLACE_ICCAD21_Liang_AMFPlacer} and nonlinear optimization-based
approaches~\cite{ PLACE_TCAD2021_Meng, PLACE_ICCAD2017_Kuo_NTUfplace,
PLACE_TCAD2020_Chen} for the best trade-off between quality and efficiency.
Among them, the current state-of-the-art quality is achieved by nonlinear approaches \texttt{elfPlace}~\cite{PLACE_TCAD2021_Meng} 
and \texttt{NTUfPlace}~\cite{PLACE_TCAD2020_Chen}, whose instance density models are derived from a multi-electrostatics system and 
a hand-crafted bell-shaped field system. 
However, most existing FPGA placers only consider a simplified FPGA architecture, i.e., LUT, FF, DSP, and BRAM,
ignoring the commonplace SLICEL-SLCIEM heterogeneity in real FPGA architectures~\cite{PLACE_TCAD2018_Li,
PLACE_ICCAD2016_Pui_RippleFPGA,
PLACE_ICCAD2016_Ryan,PLACE_TCAD2019_Li_UTPLACEF_DL, PLACE_TCAD2021_Meng,
PLACE_TODAES2018_Li_UTPlaceF2, PLACE_FPGA2019_Li,
PLACE_ICCAD2017_Pui_RippleFPGA, PLACE_ICCAD2017_Kuo_NTUfplace,
PLACE_TCAD2020_Chen}. 
Among these placers, only a few placers utilize the parallelism that the GPU provides \cite{PLACE_TCAD2021_Meng},
and few consider timing and clock feasibility in practice \cite{
 PLACE_ICCAD2017_Pui_RippleFPGA, PLACE_ICCAD2017_Kuo_NTUfplace, PLACE_TCAD2020_Chen, PLACE_TCAD2019_Li_UTPLACEF_DL, PLACE_TODAES2018_Li_UTPlaceF2
}.

In this work, we aim at optimizing wirelength, timing, and routability while cooperating with SLICEL-SLICEM heterogeneity, alignment feasibility, and clock constraints.
We define the FPGA placement problem as follows.

\begin{myproblem}[FPGA Placement]
  % \textit{Given a netlist consisting of instances in \{LUT, FF, DSP, BRAM, distributed RAM, SHIFT, CARRY\}, 
  % generate a feasible FPGA placement solution with optimized wirelength, timing and routability, satisfying the requirement for alignment feasibility and the clock constraints.}

  \textit{Taking as input a netlist consisting of LUTs, FFs, DSPs, BRAMs,
distributed RAMs, SHIFTs, and CARRYs, generate a plausible FPGA placement
solution with optimized wirelength, timings, and routings, satisfying the
  requirements for alignment feasibility and meeting the clock constraints.}
\end{myproblem}

\begin{figure}[h]
 \vspace{-.2in}
  \includegraphics[width=.4\textwidth]{figs/electrostatics_analogy.pdf}
  \caption{
    Analogy between placement for a single resource type and an electrostatic system \cite{PLACE_TCAD2015_Lu}. 
  }
  \label{fig:electrostatics_analogy}
\vspace{-.2in}
\end{figure}

\begin{figure*}[tb]
  \includegraphics[width=.98\textwidth]{figs/multi_electrostatics_exp2.pdf}
  \caption{
    % Example of the multi-electrostatic formulation for LUT and DSP resources, 
    % corresponding to two electric fields, respectively. 
    % For each field, unavailable columns are considered as occupied when computing the initial charge density. 
    % Take LUT as an example. If a LUT instance is not placed at a SLICEL column or there are overlaps between LUT instances, it can cause density overflow and lead to imbalanced density distribution of the field, 
    % eventually resulting in high electric potential energy.  
    % Thus, minimizing the energy can help resolve density overflow and spread instances in the layout. 
    % Note that if we encounter density underflow ($<1.0$), we can insert fillers with positive charges for each field to fill the empty spaces to avoid imbalanced density distribution \cite{PLACE_TODAES2015_Lu, PLACE_TCAD2021_Meng}. 
    % As a result, only density overflow will cause high potential energy. 
    % FF and BRAM can be handled in the same way by adding two more fields. 
  An example of a multi-electrostatic formulation for LUT and DSP resources,
  which correspond to two electric fields. 
  Unavailable columns are treated as occupied when calculating the initial charge density for each field. 
  Take DSP as an example. 
  Density overflow can occur if a DSP instance is not put in a DSP column or if there are overlaps between DSP instances, 
  resulting in an uneven density distribution of the field and, finally, excessive electric
  potential energy. 
  As a result, limiting energy in the layout can assist in the resolution of density overflow and spread cases.
  If we face density underflow ($<1.0$), we can insert fillers with positive charges for each field to fill the vacant spaces, \cite{PLACE_TODAES2015_Lu, PLACE_TCAD2021_Meng}.
  As a result, only density overflow will provide considerable potential energy.
  We may handle them in the same way by including FF and BRAM fields.
    % An illustration of the multi-electrostatic formulation for LUT and DSP resources, which correspond to two electric fields, respectively.
    % When computing the initial charge density for each field, unavailable columns are considered as occupied. 
    % Consider DSP as an example. If a DSP instance is not placed in a DSP column or there are overlaps between DSP instances, density overflow can occur, resulting in 
    % an imbalanced density distribution of the field, and eventually, high electric potential energy. 
    % As a result, minimizing energy can aid in the resolution of density overflow and spread instances in the layout. 
    % To avoid imbalanced density distribution, if we encounter density underflow ($<1.0$), we can insert fillers with positive charges for each field to fill the empty spaces \cite{PLACE_TODAES2015_Lu, PLACE_TCAD2021_Meng}.
    % Therefore, only density overflow will generate significant potential energy. 
    % By adding FF and BRAM fields, we can handle them in the same way.
  }
  \label{fig:multi_electrostatics_exp}
  \vspace{-.1in}
\end{figure*}

