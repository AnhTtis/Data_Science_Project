% =============================================================
%         setting for IEEE format paper (normal mode)
%
%   Author      : Bei Yu
%   Last Update : 07/2016
% =============================================================


%\usepackage[noend]{algpseudocode}
\usepackage{algpseudocode}                                 % new algorithm package
\usepackage{algorithm}
\usepackage{graphicx}                                      % include pdf figures
\usepackage{amsmath}
\usepackage{amssymb}
\usepackage{amsfonts}
\usepackage{booktabs}
\usepackage{multirow}
\usepackage[subrefformat=parens,farskip=0pt,justification=centering]{subfig}
\usepackage{color}
\usepackage{cite}                                          % more citations in one bracket
\usepackage{comment}                                       % use comment
\usepackage{soul}                                          % use highlight command \hl{}
\soulregister\cite7
\soulregister\ref7
\soulregister\pageref7
\usepackage{amsthm}
\usepackage{etoolbox}                                      % commands \newtoggle, \toggletrue, \iftoggle
\usepackage{url}
\usepackage{nth}                                           % nth command
\usepackage{bm}                                            % bm command
\usepackage{balance}
\usepackage{threeparttable}
\usepackage[bookmarks=false]{hyperref}                     %
\hypersetup{
    colorlinks = true,
    citecolor  = blue,
    linkcolor  = blue,
    urlcolor   = blue,
}
\usepackage{filecontents}                                  % support to pgfplots
\usepackage{pgfplots}
\usepackage{pgfplotstable}
\pgfplotsset{compat=newest}
\setlength{\columnsep}{18pt}

% =============================================================
%                   Theorem Definitions
% =============================================================
\newtheorem{myproblem}{\textbf{Problem}}
\newtheorem{mydefinition}{\textbf{Definition}}
\newtheorem{mytheorem}{\textbf{Theorem}}
\newtheorem{mylemma}{\textbf{Lemma}}
\newtheorem{myclaim}{\textbf{Claim}}
\newtheorem{myapplication}{\textbf{Application}}
\newtheorem{myconjecture}{Conjecture}
\newtheorem{myprop}{Property}

\algrenewcommand\textproc{\texttt}
\newcommand{\tabincell}[2]{
    \begin{tabular}{@{}#1@{}}#2\end{tabular}
}

% long line in algorithm
% e.g.: \Statex[4] ...;
\makeatletter
\let\OldStatex\Statex
\renewcommand{\Statex}[1][3]{%
  \setlength\@tempdima{\algorithmicindent}%
  \OldStatex\hskip\dimexpr#1\@tempdima\relax
}
\makeatother


% =============================================================
%          Definitions to support latexdiff
% =============================================================
%DIF PREAMBLE EXTENSION ADDED BY LATEXDIFF
%DIF UNDERLINE PREAMBLE %DIF PREAMBLE
\RequirePackage[normalem]{ulem} %DIF PREAMBLE
\RequirePackage{color}\definecolor{RED}{rgb}{1,0,0}\definecolor{BLUE}{rgb}{0,0,1} %DIF PREAMBLE
\providecommand{\DIFadd}[1]{{\protect\color{blue}{#1}}}                           %DIF PREAMBLE
\providecommand{\DIFdel}[1]{{\protect\color{red}\sout{#1}}}                       %DIF PREAMBLE
%DIF SAFE PREAMBLE %DIF PREAMBLE
\providecommand{\DIFaddbegin}{}                                                   %DIF PREAMBLE
\providecommand{\DIFaddend}{}                                                     %DIF PREAMBLE
\providecommand{\DIFdelbegin}{}                                                   %DIF PREAMBLE
\providecommand{\DIFdelend}{}                                                     %DIF PREAMBLE
%DIF FLOATSAFE PREAMBLE %DIF PREAMBLE
\providecommand{\DIFaddFL}[1]{\DIFadd{#1}}                                        %DIF PREAMBLE
\providecommand{\DIFdelFL}[1]{\DIFdel{#1}}                                        %DIF PREAMBLE
\providecommand{\DIFaddbeginFL}{}                                                 %DIF PREAMBLE
\providecommand{\DIFaddendFL}{}                                                   %DIF PREAMBLE
\providecommand{\DIFdelbeginFL}{}                                                 %DIF PREAMBLE
\providecommand{\DIFdelendFL}{}                                                   %DIF PREAMBLE
%DIF END PREAMBLE EXTENSION ADDED BY LATEXDIFF

% ==== Local new commands
\newcommand{\calH}{\mathcal{H}}
\newcommand{\calN}{\mathcal{N}}
\newcommand{\calO}{\mathcal{O}}
\newcommand{\calP}{\mathcal{P}}
\newcommand{\calV}{\mathcal{V}}
\newcommand{\calS}{\mathcal{S}}
\newcommand{\calD}{\mathcal{D}}
\newcommand{\bignorm}[1]{\left\lVert#1\right\rVert}
\newcommand{\norm}[1]{\lVert#1\rVert}
\renewcommand{\vec}[1]{\boldsymbol{#1}}
\newcommand{\m}[1]{\mathbf{#1}}
\newcommand{\mx}{\m{x}}
\newcommand{\my}{\m{y}}
\newcommand{\mz}{\m{z}}
\newcommand{\bigmax}[2]{\max\left({#1}, {#2}\right)}
\newcommand{\bigmin}[2]{\min\left({#1}, {#2}\right)}
\newcommand{\bigparenthesis}[1]{\left({#1}\right)}
\newcommand{\bigbracket}[1]{\left[{#1}\right]}
\newcommand{\bigbrace}[1]{\left\{{#1}\right\}}
\DeclareMathOperator*{\argmin}{argmin}

% \theoremstyle{plain}
% \newtheorem{mytheorem}{\textbf{Theorem}}
% \newtheorem{mylemma}{\textbf{Lemma}}
% \newtheorem{myclaim}{\textbf{Claim}}

% \theoremstyle{definition}
% \newtheorem{mydefinition}{\textbf{Definition}}
% \newtheorem{myproblem}{\textbf{Problem}}

% ==== Reference Style ========
\newcommand{\eqRef}[1]{Eq.~\eqref{#1}}
\newcommand{\tabRef}[1]{TABLE~\ref{#1}}
\newcommand{\figRef}[1]{Fig.~\ref{#1}}
\newcommand{\secRef}[1]{Section~\ref{#1}}
\newcommand{\algRef}[1]{Algorithm~\ref{#1}}

% === Colorization tools ======
\newcommand{\colorstuff}[2]{\color{#1}{#2}\color{black}}

% ==== Logs:
%  09/2022: add refrence style
%  07/2016: support to pgfplots
%  07/2016: threeparttable
%  04/2016: definitions supporting latexdiff
%  04/2016: remove bookmarks in hyperref
%  02/2016: tabincell
%  02/2016: hyperref
%  12/2015: copy from "ieee_conference"
%
