% =============================================================
%            settign for ACM format paper
%
%   Author      : Bei Yu
%   Last Update : 12/2014
% =============================================================

\settopmatter{printacmref=false}          % no ACM Reference Format
\fancyhead{}                              % no header

% ==== page margin settings
\geometry{twoside=true, head=12pt,
	paperwidth=9.2in, paperheight=11.8in,
	includeheadfoot, columnsep=1.8pc,
	top=45pt, bottom=45pt,
    left=51pt, right=51pt,
    %inner=54pt, outer=54pt,
	marginparwidth=1.8pc,heightrounded
}%
% \geometry{twoside=true, head=13pt,
	% paperwidth=8.5in, paperheight=11in,
	% includeheadfoot, columnsep=2pc,
	% top=45pt, bottom=55pt,
    % left=54pt, right=54pt,
    % %inner=54pt, outer=54pt,
	% marginparwidth=2pc,heightrounded
% }%
%\geometry{twoside=true, head=13pt,
%paperwidth=8.5in, paperheight=11in,
%includeheadfoot, columnsep=2pc,
%top=57pt, bottom=73pt, inner=54pt, outer=54pt,
%marginparwidth=2pc,heightrounded
%}%
\iffalse
\textfloatsep          = 12pt plus 1pt minus 7pt            % set space between figure and text
\floatsep              = 5pt plus 1pt minus 7pt
\fi
%\intextsep             = 0pt plus 1pt minus 7pt           % set space above and below the texts
%\dbltextfloatsep       = 0pt plus 1pt minus 7pt
%\dblfloatsep           = 0pt plus 1pt minus 7pt
%\abovedisplayskip      = 0pt plus 1pt minus 7pt           % set space around equations
%\belowdisplayskip      = 0pt plus 1pt minus 7pt
%\abovedisplayshortskip = 0pt plus 1pt minus 7pt
%\belowdisplayshortskip = 0pt plus 1pt minus 7pt

% ==== Packages
\usepackage{soul}
\usepackage{graphicx}
\usepackage{amsmath}
% \usepackage{amssymb}
\newcommand\hmmax{0} % default 3
\newcommand\bmmax{0} % default 4
\usepackage{mathtools}
\usepackage{comment}
\usepackage[subrefformat=parens,labelformat=parens]{subfig}
\captionsetup[subfigure]{labelformat=simple}               % avoid "double brackets" in sub-figure caption
\renewcommand\thesubfigure{(\alph{subfigure})}             % "Fig.~1b"-->"Fig.1(b)"
\usepackage{bm}
\usepackage{multirow}
\usepackage{threeparttable,booktabs}
\usepackage{tikz}
\usepackage{balance}
\usepackage{courier}                                       % courier font, used in \texttt
\usepackage{cleveref}                                      % smart citation
\usepackage[mathcal]{eucal}
%\usepackage[]{algpseudocode}                               % algorithm package
\usepackage[noend]{algpseudocode}
\algrenewcommand\textproc{\texttt}
\makeatletter\let\float@addtolists\relax\makeatother
% \usepackage{algorithm}
% \renewcommand{\algorithmicrequire}{\textbf{Input:} }       % Use Input in the format of Algorithm
% \renewcommand{\algorithmicensure} {\textbf{Output:}}       % Use Output in the format of Algorithm
\usepackage[ruled,linesnumbered]{algorithm2e}
\usepackage{filecontents}                                  % support to pgfplots
\usepackage{pgfplots}
\usepackage{pgfplotstable}
\pgfplotsset{compat=newest}
\usepackage{siunitx}
\sisetup{product-units = single}

% ==== Local new commands
\newcommand{\calH}{\mathcal{H}}
\newcommand{\calN}{\mathcal{N}}
\newcommand{\calO}{\mathcal{O}}
\newcommand{\calP}{\mathcal{P}}
\newcommand{\calV}{\mathcal{V}}
\newcommand{\calS}{\mathcal{S}}
\newcommand{\calD}{\mathcal{D}}
\newcommand{\bignorm}[1]{\left\lVert#1\right\rVert}
\newcommand{\norm}[1]{\lVert#1\rVert}
\renewcommand{\vec}[1]{\boldsymbol{#1}}
\newcommand{\m}[1]{\mathbf{#1}}
\newcommand{\mx}{\m{x}}
\newcommand{\my}{\m{y}}
\newcommand{\mz}{\m{z}}
\newcommand{\bigmax}[2]{\max\left({#1}, {#2}\right)}
\newcommand{\bigmin}[2]{\min\left({#1}, {#2}\right)}
\newcommand{\bigparenthesis}[1]{\left({#1}\right)}
\newcommand{\bigbracket}[1]{\left[{#1}\right]}
\newcommand{\bigbrace}[1]{\left\{{#1}\right\}}
\DeclareMathOperator*{\argmin}{argmin}

\theoremstyle{plain}
\newtheorem{mytheorem}{\textbf{Theorem}}
\newtheorem{mylemma}{\textbf{Lemma}}
\newtheorem{myclaim}{\textbf{Claim}}

\theoremstyle{definition}
\newtheorem{mydefinition}{\textbf{Definition}}
\newtheorem{myproblem}{\textbf{Problem}}

% ==== Reference Style ========
\newcommand{\eqRef}[1]{Eq.~\eqref{#1}}
\newcommand{\tabRef}[1]{Table~\ref{#1}}
\newcommand{\figRef}[1]{Figure~\ref{#1}}
\newcommand{\secRef}[1]{Section~\ref{#1}}
\newcommand{\algRef}[1]{Algorithm~\ref{#1}}

% === Colorization tools ======
\newcommand{\colorstuff}[2]{\color{#1}{#2}\color{black}}

% ==== spacing control on caption =====
\usepackage[skip=1pt]{caption}            % set space between figure and caption
\setlength{\belowcaptionskip}{-1.0mm}
\captionsetup[table]{aboveskip=5pt}       % reduce space around table caption
\captionsetup[table]{belowskip=2pt}

