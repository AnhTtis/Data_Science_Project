\documentclass{amsart}
\usepackage{enumitem,kantlipsum}

\voffset=-1.4mm
\oddsidemargin=17pt \evensidemargin=17pt
\headheight=9pt     \topmargin=26pt
\textheight=576pt   \textwidth=440.8pt
\parskip=0pt plus 4pt

\usepackage[utf8]{inputenc}
\usepackage{physics}
\usepackage{amssymb}
\usepackage{amsmath}
\usepackage{amsfonts}
\usepackage{amsthm}
\usepackage{mathtools}
\usepackage{tikz}
\usepackage{tikz-cd}
\usepackage{algorithmic}
\usepackage{asymptote}
\usepackage{url}
\usepackage{amsmath,amssymb}
\makeatletter
\newsavebox\myboxA
\newsavebox\myboxB
\newlength\mylenA




\newtheorem{theorem}{Theorem}[section]
\newtheorem{corollary}{Corollary}[theorem]
\newtheorem{lemma}[theorem]{Lemma}
%\newtheorem{remark}[theorem]{Remark}
%\newtheorem{example}[theorem]{Example}
\newtheorem{proposition}[theorem]{Proposition}
\newtheorem{conjecture}[theorem]{Conjecture}
\newtheorem{definition}[theorem]{Definition}

\usepackage[OT2,T1]{fontenc}
\DeclareSymbolFont{cyrletters}{OT2}{wncyr}{m}{n}
\DeclareMathSymbol{\Sha}{\mathalpha}{cyrletters}{"58}
\DeclarePairedDelimiter\ceil{\lceil}{\rceil}
\DeclarePairedDelimiter\floor{\lfloor}{\rfloor}

\theoremstyle{definition}
\newtheorem{alg}[theorem]{Algorithm}
\newtheorem{example}[theorem]{Example}
\newtheorem{remark}[theorem]{Remark}
\newtheorem{defn}[theorem]{Definition}
\newtheorem{conv}[theorem]{Convention}
\newtheorem{obs}[theorem]{Observation}
\newtheorem{notation}[theorem]{Notation}
\newtheorem{question}[theorem]{Question}
\newtheorem{prob}[theorem]{Problem}

\newcommand*\xoverline[2][0.75]{%
    \sbox{\myboxA}{$\m@th#2$}%
    \setbox\myboxB\null% Phantom box
    \ht\myboxB=\ht\myboxA%
    \dp\myboxB=\dp\myboxA%
    \wd\myboxB=#1\wd\myboxA% Scale phantom
    \sbox\myboxB{$\m@th\overline{\copy\myboxB}$}%  Overlined phantom
    \setlength\mylenA{\the\wd\myboxA}%   calc width diff
    \addtolength\mylenA{-\the\wd\myboxB}%
    \ifdim\wd\myboxB<\wd\myboxA%
       \rlap{\hskip 0.5\mylenA\usebox\myboxB}{\usebox\myboxA}%
    \else
        \hskip -0.5\mylenA\rlap{\usebox\myboxA}{\hskip 0.5\mylenA\usebox\myboxB}%
    \fi}
\makeatother
\newcommand{\Z}{\mathbb{Z}}
\newcommand{\ab}{\underline{a}}
\newcommand{\ml}{\mu_l}
\newcommand{\smvee}{\raise0.4ex\hbox{$\scriptscriptstyle\vee$}}
\newcommand{\R}{\mathbb{R}}
\newcommand{\overbar}[1]{\mkern 2mu\overline{\mkern-2mu#1\mkern-2mu}\mkern 2mu}
\newcommand{\F}{\mathbb{F}}
\newcommand{\ag}{\mathcal{A}_g}
\newcommand{\mg}{\mathcal{M}_g}
\newcommand{\OO}{\mathcal{O}}
\newcommand{\p}{\mathfrak{p}}
\newcommand{\q}{\mathfrak{q}}
\newcommand{\PP}{\mathcal{P}}
\newcommand{\M}{\mathcal{M}}
\newcommand{\LL}{\mathcal{L}}
\newcommand{\Q}{\mathbb{Q}}
\newcommand{\ewt}{\text{ewt}}
%\newcommand{\Tr}{\text{Tr}}
\newcommand{\gen}{\text{gen}}
\newcommand{\Spec}{\text{Spec}}
\newcommand{\okgm}{\OO_K[
\gamma]}
\newcommand{\Ab}{\underline{A}}
\newcommand{\f}{|o|}
\newcommand{\coker}{\text{coker}}
\newcommand{\V}{\vee}
\newcommand{\im}{\text{Im}}
\newcommand{\apk}{\alpha_{\p,K}}
\newcommand{\C}{\mathbb{C}}
\newcommand{\fpb}{\overline{\mathbb{F}}_p}
\newcommand{\sigp}{\sigma_p}
\newcommand{\E}{\mathbb{E}}
\newcommand{\Po}{\mathbb{P}^1}
\newcommand{\Pt}{\mathbb{P}^2}
\newcommand{\mb}{\overline{\M_{\mu_m}}}
%\newcommand{\mg}{\overline{\M}_g}
\newcommand{\mml}{\mathcal{M}_{\mu_l}}
\newcommand{\mmg}{\mathcal{M}_{G}}
\newcommand{\mt}{\tilde{\M_{\mu_m}}}
\newcommand{\zt}{\tilde{Z}(\gamma)}
\newcommand{\zto}{\tilde{Z^o}(\gamma)}
\newcommand{\ztone}{\tilde{Z}(\gamma_1)}
\newcommand{\zttwo}{\tilde{Z}(\gamma_2)}
\newcommand{\zb}{\Bar{Z}(\gamma)}
\newcommand{\zbo}{\overline{Z^o}(\gamma)}
\newcommand{\zboo}{\overline{Z^o}(\gamma_1)}
\newcommand{\zbot}{\overline{Z^o}(\gamma_2)}
\newcommand{\lra}{(l,r,\underline{a})}
\newcommand{\zlr}{\mathcal{M}(l,r,\underline{a})}
\newcommand{\zgr}{\mathcal{M}(G,r,\underline{a})}
\newcommand{\gra}{(G,r,\underline{a})}
\newcommand{\zso}{Z^o(\gamma_1)}
\newcommand{\zst}{Z^o(\gamma_2)}
\newcommand{\zsth}{Z^o(\gamma_3)}
\newcommand{\z}{Z(\gamma)}
\newcommand{\zo}{Z(\gamma_1)}
\newcommand{\ztwo}{Z(\gamma_2)}
\newcommand{\zp}{Z(\gamma')}
\newcommand{\zog}{Z^o(\gamma)}
\newcommand{\zogp}{Z^o(\gamma')}
\newcommand{\zi}{Z(\gamma_i)}
\newcommand{\zti}{\tilde{Z}(\gamma_i)}
\newcommand{\zbone}{\overline{Z}(\gamma_1)}
\newcommand{\zbi}{\overline{Z}(\gamma_i)}
\newcommand{\ztth}{\tilde{Z}(\gamma_3)}
\newcommand{\zbth}{\overline{Z}(\gamma_3)}
\newcommand{\shmf}{\text{Sh}(\ml,\underline{f})}
\newcommand{\shgf}{\text{Sh}(G,\underline{f})}
\newcommand{\zgth}{Z(\gamma_3)}
\newcommand{\ztil}{\tilde{Z}}
\newcommand{\zts}{\tilde{Z^o}}
\newcommand{\zbs}{\overline{Z^o}}
\newcommand{\go}{\gamma_1}
\newcommand{\gt}{\gamma_2}
\newcommand{\gth}{\gamma_3}
\newcommand{\gi}{\gamma_i}
\newcommand{\ts}{\tau^*}
\newcommand{\wt}{w_{\tau}}
\newcommand{\tp}{\tau'}
\newcommand{\wts}{w_{\tau^*}}
\newcommand{\tps}{\tau'^*}
\newcommand{\ti}{\tau_i}
\newcommand{\dr}{H^1_{dR}(A)}
%\newcommand{\cr}{H^1_{crys}(A)}
\newcommand{\eis}{\eta_{i,s}}
\newcommand{\et}{e_{\tau}}
\newcommand{\tis}{\ti^*}
\newcommand{\tet}{\tau_{e_{\tau}}}
\newcommand{\ty}{\tau_1}
\newcommand{\te}{\tau_2}
\newcommand{\psid}{\check{\psi}}
\newcommand{\phid}{\check{\phi}}
\newcommand{\qd}{\check{Q}_{\tau_1}}
\newcommand{\qpd}{\check{Q}_{p\tau_1}}
\newcommand{\mvt}{M_V^{\top}}
\newcommand{\mf}{M_F}
\newcommand{\BB}{\mathcal{B}_1}
\newcommand{\hy}{H^1(C,\OO_C)}
\newcommand{\hyd}{H^0(C,\Omega_C)}
\newcommand{\sy}{x_1+x_2+x_3}
\newcommand{\se}{x_1x_2+x_2x_3+x_1x_3}
\newcommand{\sth}{x_1x_2x_3}
\newcommand{\sfr}{x_1x_2x_3+x_1x_2x_4+x_1x_3x_4+x_2x_3x_4}
\newcommand{\pby}{\floor*{p\langle\frac{ia_1}{l}\rangle}}
\newcommand{\pbr}{\floor*{p\langle\frac{ia_r}{l}\rangle}}
\newcommand{\pbj}{\floor*{p\langle\frac{ia_j}{l}\rangle}}
\newcommand{\pbs}{\floor*{p\langle\frac{p^i\tau a_s}{l}\rangle}}
\newcommand{\pis}{\floor*{p\langle\frac{i a_s}{l}\rangle}}
\newcommand{\pbc}{\floor*{p\langle\frac{ia_c}{l}\rangle}}
\newcommand{\pbcm}{\floor*{p\langle\frac{ia_{c-1}}{l}\rangle}}
\newcommand{\pbct}{\floor*{p\langle\frac{ia_{c-2}}{l}\rangle}}
\newcommand{\pbcp}{\floor*{p\langle\frac{ia_{c+1}}{l}\rangle}}
\newcommand{\pbk}{\floor*{p\langle\frac{ia_k}{l}\rangle}}
\newcommand{\pba}{\floor*{p\langle\frac{ia_a}{l}\rangle}}
\newcommand{\pfy}{\floor*{p\frac{ia_1}{l}}}
\newcommand{\pfk}{\floor*{p\frac{ia_k}{l}}}
\newcommand{\fy}{\floor*{\frac{ia_1}{l}}}
\newcommand{\fr}{\floor*{\frac{ia_r}{l}}}
\newcommand{\fk}{\floor*{\frac{ia_k}{l}}}
\newcommand{\fky}{\floor*{\frac{ka_1}{l}}}
\newcommand{\fkr}{\floor*{\frac{ka_r}{l}}}
\newcommand{\pfr}{\floor*{p\frac{ia_r}{l}}}
\newcommand{\pfj}{\floor*{p\frac{ia_j}{l}}}
\newcommand{\pib}{p^ib}
\newcommand{\pibs}{p^ib^*}
\newcommand{\pfibs}{\floor*{p\langle\frac{p^iba_s}{l}\rangle}}
\newcommand{\pfibc}{\floor*{p\langle\frac{p^iba_c}{l}\rangle}}

\newcommand{\ci}{c_i(s_i-pj)}
\newcommand{\Ci}{C_i(s_i-pj)}
\newcommand{\XI}{X_i(s_i-pj)}
\newcommand{\fq}{r_{i,j,k}q_{r-j',k}}
\newcommand{\fkj}{r_{i,j,k}}
\newcommand{\qkj}{q_{r-j',k}}
\newcommand{\qpit}{Q_{p^i\tau}}
\newcommand{\qpip}{Q_{p^{i+1}\tau}}
\newcommand{\qpitsd}{\check{Q}_{p^i\ts}}
\newcommand{\qpiptsd}{\check{Q}_{p^{i+1}\ts}}
\newcommand{\qpitd}{\check{Q}_{p^i\tau}}
\newcommand{\qpiptd}{\check{Q}_{p^{i+1}\tau}}
\newcommand{\qi}{Q_i}
\newcommand{\qid}{Q_i^{\smvee}}
\newcommand{\qt}{Q_\tau}
\newcommand{\qtd}{Q_\tau^{\smvee}}
\newcommand{\qis}{Q_{\tau_i^*}}
\newcommand{\qisd}{Q_{i^*}^{\smvee}}
\newcommand{\qts}{Q_{\ts}}
\newcommand{\qtsd}{Q_{\ts}^{\smvee}}
\newcommand{\qpi}{Q_{pi}}
\newcommand{\qpid}{Q_{pi}^{\smvee}}
\newcommand{\qpt}{Q_{p\tau}}
\newcommand{\qptd}{Q_{p\tau}^{\smvee}}
\newcommand{\qpis}{Q_{pi^*}}
\newcommand{\qpisd}{Q_{pi^*}^{\smvee}}
\newcommand{\qpts}{Q_{p\ts}}
\newcommand{\qptsd}{Q_{p\ts}^{\smvee}}
\newcommand{\ocu}{\OO_C(U)}
\newcommand{\ocv}{\OO_C(V)}
\newcommand{\ocuv}{\OO_C(U \cap V)}
\newcommand{\zij}{\zeta_{i,j}}
\newcommand{\zpij}{\zeta_{pi,j'}}
\newcommand{\zpisk}{\zeta_{pi^*,k}}
\newcommand{\zpiskd}{\check{\zeta_{pi^*,k}}}
\newcommand{\zijd}{\check{\zeta_{i,j}}}
\newcommand{\zpijd}{\check{\zeta_{pi,j'}}}
\newcommand{\oij}{\omega_{i,j}}
\newcommand{\opij}{\omega_{pi,j}}
\newcommand{\opisjp}{\omega_{pi^*,j'}}
\newcommand{\opisk}{\omega_{pi^*,k}}
\newtheorem{innercustomthm}{Theorem}
\newcommand{\phii}{\phi_{\tau_i}}
\newcommand{\psii}{\psi_{\tau_i}}
\newcommand{\psiip}{\psi_{\tau_i}'}
\newcommand{\vi}{v_i}
\newcommand{\vip}{v_{pi}}
\newcommand{\vis}{v_{i^*}}
\newcommand{\vips}{v_{pi^*}}
\newcommand{\pbf}{(x-x_1)^{\pby} \dots (x-x_r)^{\pbr}}
\newcommand{\pff}{(x-x_1)^{\pfy} \dots (x-x_r)^{\pfr}}
\newcommand{\tot}{\tau_{\OO,2}}
\newcommand{\fpi}{f(p^i\tau)}
\newcommand{\fpim}{f(p^{i-1}\tau)}
\newcommand{\fpits}{f(p^i\ts)}
\newcommand{\fpitsp}{f(p^{i+1}\ts)}
\newcommand{\fpitsm}{f(p^{i-1}\ts)}
\newcommand{\fii}{f(i)}
\newcommand{\fis}{f(\tau_i^*)}
\newcommand{\fou}{f_{\OO,u}}
\newcommand{\foum}{f_{\OO,u-1}}
\newcommand{\ft}{f(\tau)}
\newcommand{\fts}{f(\ts)}
\newcommand{\pit}{p^i\tau}
\newcommand{\pits}{p^i\ts}
\newcommand{\shgfb}{\shgf_{\fpb}}
\newcommand{\pipt}{p^{i+1}\tau}
\newcommand{\pipts}{p^{i+1}\ts}
\newcommand{\aip}{A_{|\OO|-1}\circ \dots \circ A_{0}}
\newcommand{\goppi}{\pi_{\tau}(G_{\tau,|\OO|-1}(\dots G_{\tau,1}(G_{\tau,0}(M_{\tau}))))}
\newcommand{\gop}{G_{\tau,|\OO|-1}(\dots G_{\tau,1}(G_{\tau,0}(M_{\tau})))}
\newcommand{\gopp}{\pi_{\tau'}(G_{\tau',|\OO|-1}(\dots G_{\tau',1}(G_{\tau',0}(M_{\tau'}))))}
\newcommand{\gjp}{G_{\tau,j}(\dots G_{\tau,1}(G_{\tau,0}(M_{\tau})))}
\newcommand{\pjt}{p^j\tau}
\newcommand{\pjts}{p^j\ts}
\newcommand{\Span}{\text{Span}}
\newcommand{\mo}{|\OO|}
\newcommand{\aof}{A_{\mo-1} \circ \dots \circ A_{0}}
\newcommand{\aoi}{A_{\mo-1} \circ \dots \circ A_i}
\newcommand{\aiz}{A_{i-1} \circ \dots \circ A_0}
\newcommand{\hopf}{H_{\tp,\mo-1} \circ \dots \circ H_{\tp,0}}
\newcommand{\hoppf}{H_{\tpp,\mo-1} \circ \dots \circ H_{\tpp,0}}
%\newcommand{\hoppf}{H_{\tau'',\mo-1} \circ \dots \circ H_{\tau'',0}}
\newcommand{\hopi}{H_{\tp,\mo-1} \circ \dots \circ H_{\tp,i}}
\newcommand{\hpiz}{H_{\tp,\mo-i-1} \circ \dots \circ H_{\tp,0}}
\newcommand{\tpp}{\tau''}
\newcommand{\pip}{\pi_{\tp}}
\newcommand{\pipp}{\pi_{\tau''}}
\newcommand{\lof}{L_{\tau,\mo-1} \circ \dots \circ L_{\tau,0}}
\newcommand{\piot}{p^{i_0}\tau}
\newcommand{\ajio}{A_{j} \circ A_{j-1} \circ \dots \circ A_{i_0}}
\newcommand{\ljio}{L_{\tau,j} \circ \dots \circ L_{\tau,i_0}}
\newcommand{\hjmiopp}{H_{\tpp,j-1-\iz} \circ \dots \circ H_{\tpp,0}}
\newcommand{\hjmio}{H_{\tau,j-1} \circ \dots \circ H_{\tau,i_0}}
\newcommand{\mpiot}{M_{\piot}}
\newcommand{\mpjpl}{M_{p^{j+1}\tau}}
\newcommand{\pjpl}{p^{j+1}\tau}
\newcommand{\iz}{i_0}
\newcommand{\pizp}{p^{\iz+1}\tau}
\newcommand{\mpizp}{M_{\pizp}}
\newcommand{\au}{\underline{a}}
\newcommand{\n}{\text{new}}
\newcommand{\ord}{\text{ord}}
\newcommand{\mm}{\textbf{m}}
\newcommand{\aaa}{\textbf{a}}
\newcommand{\gal}{\text{Gal}}
\newcommand{\galq}{\gal(\overline{\Q}/\Q)}
\newcommand{\fb}{\underline{f}}
\newcommand{\tgn}{\mathcal{T}_{G/N}}
\newcommand{\tgh}{\mathcal{T}_{G/H}}
\newcommand{\fgn}{f_{G/N}}
\newcommand{\fgh}{f_{G/H}}
\newcommand{\ogn}{\OO_{G/N}}
\newcommand{\og}{\OO_G}
\newcommand{\ogh}{\OO_{G/H}}
\newcommand{\rhot}{\tilde{\tau}}
\newcommand{\Hom}{\text{Hom}}
\newcommand{\ort}{\OO_{\rhot}}
\newcommand{\orho}{\OO_{\tau}}
\newcommand{\tg}{\mathcal{T}_G}
\newcommand{\qpun}{\Q_p^{nr}}
\newcommand{\qpal}{\overline{\Q_p}}
\newcommand{\tgk}{\mathcal{T}_{G/K}}
\newcommand{\ogk}{\OO_{G/K}}
\newcommand{\kij}{\chi_{i,j}}
\newcommand{\ct}{C_{\underline{x}}}
\newcommand{\mlra}{\mathcal{M}(l,r,\au)}
\newcommand{\mgra}{\mathcal{M}(G,r,\au)}
\newcommand{\mgb}{\overline{\mathcal{M}_G}}
\newcommand{\mG}{\mathcal{M}_G}
\newcommand{\crt}{C_{\rho,\underline{x}}}
\newcommand{\tb}{\underline{x}}
\newcommand{\ub}{\underline{b}}
\newcommand{\otau}{\OO_{\tau}}
\newcommand{\xb}{\underline{x}}
\newcommand{\xbt}{\Tilde{\xb}}
\newcommand{\fpn}{\mathbb{F}_{p^n}}
\newcommand{\vs}{\text{V}}
\newcommand{\vpl}{\text{V}^{+}}
\newcommand{\vm}{\text{V}^{-}}
\newcommand{\mgrab}{\mgra_{\fpb}}
\newcommand{\tguh}{\tg^H}
\newcommand{\tghn}{\tg^{H,\text{new}}}
\newcommand{\tgun}{\tg^{\text{new}}}
\newcommand{\MM}{\mathcal{M}}
\newcommand{\m}{\mathfrak{m}}
\newcommand{\qb}{\Bar{\Q}^*}
\newcommand{\tgdhn}{\tgh^{\text{new}}}
\newcommand{\gb}{\gamma_b}
\newcommand{\mts}{\mathcal{M}(\gamma_{t,s})}
\newcommand{\gts}{\gamma_{t,s}}
\newcommand{\shgts}{\text{Sh}(\ml,\gts)}
\newcommand{\qph}{\Hat{\overline{\Q}}_p}
\newenvironment{customthm}[1]
  {\renewcommand\theinnercustomthm{#1}\innercustomthm}
  {\endinnercustomthm}

\newtheorem{innercustomcor}{Corollary}
\usepackage{enumitem,kantlipsum}

\newcommand{\FF}{\mathcal{F}}
\usepackage{amsmath}% http://ctan.org/pkg/amsmath
\usepackage{kbordermatrix}% http://www.hss.caltech.edu/~kcb/TeX/kbordermatrix.sty
\renewcommand{\kbldelim}{(}% Left delimiter
\renewcommand{\kbrdelim}{)}% Right delimiter





\begin{document}

\title{Abelian covers of $\mathbb{P}^1$ of $p$-ordinary Ekedahl-Oort type}



\author{Yuxin Lin, Elena Mantovan, Deepesh Singhal}
\date{}
\maketitle 
\begin{abstract}
%In this paper, we consider the following question. 
Given a monodromy datum for abelian covers of $\Po$ %branched at at least four points 
and a prime $p$ of good reduction, there is a natural lower bound for the Ekedahl-Oort type, and the Newton polygon, at $p$ of the curves in the family. In this paper, we investigate whether such a lower bound is sharp. In particular, we prove sharpeness when the number of branching points is at most five and $p$ sufficiently large. Our result is a generalization of \cite[Theorem 6.1]{Irene} by Bouw, which proves the analogous statement for the $p$-rank, and it relies on the notion of Hasse-Witt triple introduced in \cite{Moonen algorithm} by Moonen.  
\end{abstract}

\section{Introduction}
This paper is motivated by the arithmetic Schottky problem in positive characteristics, as it investigates which mod $p$ discrete invariants of abelian varieties occur for Jacobians  of smooth curves. We restrict our attention to the case of Jacobians of abelian covers of $\Po$.
%we consider $\ag$, the moduli space of principally polarized dimension $g$ abelian varieties, and $\mg$, the moduli space of genus $g$ smooth curves.  We have the Torelli morphism $T:\mg \to \ag$ such that on geometric points, $T$ takes an isomorphism class of curves to its Jacobian. We know that $T$ is injective by Torelli's theorem. The problem asks for a characterization of the image of Torelli morphism inside $\ag$. We restrict our attention to the special fiber of this morphism at a prime $p$. Then $\mg$ and $\ag$ become schemes defined over $\fpb$, and we hope to characterize the image of subsets of $\mg$ consisting of curves of abelian covers of $\Po$. 


Let $G$ be a finite abelian group and a prime $p$ not dividing $ |G|$. We consider          
            $\mmg$, the moduli space of $G$-covers of $\Po$. On each irreducible component of $\mmg$, the monodromy datum $(G,r,\ab)$ of the covers is constant, and we denote such component by $\mathcal{M}(G,r,\ab)$ (in the notation $(G,r, \ab)$, $r$ denotes the number of branched points and $\ab$ %=(\ab_1,\ab_2, \dots, \ab_r )$ with $\ab_k \in G$ is the inertia type). 
%(in the notation $(G,r, \ab)$, $r$ denotes the number of branched points and $\ab$ %=(\ab_1,\ab_2, \dots, \ab_r )$ with $\ab_k \in G$ is 
the inertia type). 
%Then for the class of $[C]$ parameterized by points of $\zgr$, $C$ is a $G$-cover of $\Po$ branched at $r$ points with monodromy at $k^{th}$ point being defined by $\ab_k$. 
By construction, the image of $\mgra$ under the Torelli map $T$ is contained in a special subvariety of the Siegel variety. We denote by $\shgf$  the smallest PEL-type Shimura variety containing $T(\mgra)$, its Shimura datum $(G,\underline{f})$ depending on the monodromy datum $\gra$.  
%Here $\underline{f}$ is a combinatorial datum that encodes the action of group algebra of $G$ on the parametrized abelian variety. For $\shgf$ that contains $T(\mgra)$, $\fb$ is uniquely determined by $\gra$. 

%=(f(\rho))_{\rho}$, such that for each $\rho \in \Hat{G}$, $f(\rho)$ is the dimension of the eigenspace corresponding to $\rho$ of the parameterized abelian variety. When we fix the monodromy datum $(G,r,\ab)$, the resulting signature $(f(\rho))_{\rho}$ of $J(\ct)$ is uniquely determined and only depend on $(G,r,\ab)$. It can be computed via the formula provided by \cite[Equation 2.4]{Second Paper}. Consequently, $T$ will map $\mgra$ into the $\shgf$, with $\underline{f}$ determined by $(G,r,\ab)$.
%
%
The inclusion of $T(\zgr)$ inside $\shgf$ give rise to natural lower bounds for the Ekedahl-Oort types and the Newton polygons %at primes $p$ not dividing $|G|$ 
occurring for the Jacobians of curves parametrized by $\mathcal{M}(G,r,\ab)$. %Both $\mgra$ and $\shgf$ are defined over some number fields $K$, so they can be extended to a scheme (or stack) defined over $\OO_K[\frac{1}{|G|}]$. 
More precisely, let $p$ is a prime not dividing $|G|$. Then, both $\mgra$ and $\shgf$ have good reduction at $p$, and the $p$-rank, Ekedahl–Oort and Newton stratifications of $\mgra_{\fpb}$ are induced from those on $\shgf_{\fpb}$. In particular, the maximal $p$-rank and the lowest Ekedahl–Oort type and the Newton polygon occurring on $\shgf_{\fpb}$  are respectively upper and lower bounds for those occurring on $\mgra_{\fpb}$. For example, by \cite[Theorem 1.6.3]{Wedhorn}, when $p$ is not totally split in the reflex field of $\shgf_{\fpb}$, the ordinary stratum of $\shgf_{\fpb}$ is empty, and hence so is that of $\mgra_{\fpb}$.  It is natural to ask whether these bounds are sharp. More precisely, given a monodromy datum $\gra$ and a prime $p\nmid |G|$, one may ask when the intersection of $T(\zgr)$ with each of the maximal open strata of $\shgf_{\fpb}$ is non-empty. When $r\leq 3$, $\dim\zgr=\dim\shgf=0$ and $T(\zgr)=\shgf$, hence the statement is trivial. On the other hand, by the Coleman--Oort Conjecture, if $r\geq 4$, $\dim\zgr < \dim\shgf$ except in finitely many instances (see \cite{Moonen Oort}). 

For $p$ sufficiently large, the sharpness of the $p$-rank bound follows from \cite[Theorem 6.1]{Irene}. 
%(note that problem studied in \cite{Irene} is more general than the one described above). 
%implies that the intersection of $T(\zgra)$ with the maximal $p$-rank stratum  is always non-empty. 
By \cite[Theorem 1.3.7]{Moonen EO type formula}, the maximal open Ekedahl–Oort and Newton strata of $\shgf_{\fpb}$ agree. Hence, in the following, we focus on the sharpness of the lower bound for Ekedahl-Oort types, and positively answer our question when $p$ is sufficiently large and the number of branched points is at most $5$.  As the maximal Newton stratum is (in general, properly) contained in the maximal $p$-rank stratum, our results is a refinement of \cite[Theorem 6.1]{Irene} for covers of $\Po$ branched at at most 5 points.


%we can look at their special fibre over $\fpb$, then the Torelli morphism maps $\mgra_{\fpb}$ to $\shgf_{\fpb}$. On $\shgf_{\fpb}$ we have the stratification by the discrete invariants, such as Newton polygons, $p$-rank and Ekedahl–Oort type. By comparing their dimensions, we see that $\shgf_{\fpb}$ is much larger than $T(\zgr_{\fpb})$. Therefore, it is natural to wonder, for each discrete invariant, which of the stratum on $\shgf_{\fpb}$ has non-empty intersection with $T(\mgra_{\fpb})$. In the case of $p$-rank, Irene shows in \cite[Theorem 6.1]{Irene} that when $p$ is sufficiently large, for every family $\mgra$, $T(\mgrab)$ has non-empty intersection with the maximal $p$-rank stratum in $\shgfb$. This shows that the lower bound on $p$-rank  coming from $\shgfb$ is sharp on $T(\mgrab)$.

%Our setting is a generalization of Irene's, because we are interested in the Newton polygon at $p$. Since $p$-rank counts the multiplicity of the $0$ slope in the Newton polygon at $p$, the Newton strata is a refinement of the $p$-rank strata. Li, Mantovan, Prius and Tang show in \cite{Second Paper} that there is a natural lower bound for the Newton polygons that can occur for $\shgfb$, and this lower bound is called the $\mu$-ordinary Newton Polygon. When $(G,r,\au)$ is one of the special families in \cite{special family}, this lower bound is sharp for the dimension reason. We want to investigate whether this lower bound is sharp on $T(\mgrab)$ when $\dim(\mgrab)<\dim(\shgfb)$. That is, given a monodromy datum $\gra$ such that $p \nmid |G|$, does there always exists $\underline{x} \in \zgr(\fpb)$ such that the Newton Polygon of $J(\ct)$ is $\mu$-ordinary. To solve this problem, we adapt the notion of Ekedahl-Oort type, which is another discrete invariant that categorizes the isomorphism class of the $p$ kernel of the abelian variety. While the Newton stratum and E-O stratum are in general quite different, in \cite[Theorem 1.3.7]{Moonen EO type formula}, Moonen shows that the $\mu$-ordinary Newton stratum coincides with the largest Ekedahl–Oort stratum in $\shgf$. The Ekedahl-Oort type of the largest Ekedahl-Oort stratum is called $p$-ordinary.

%which can be seen either via dimension count, or the fact that there are Newton polygons appearing in $\shgf$ that do not appear in $T(\zgr)$.  Indeed, for each $\Ab$ parameterized by a $\fpb$ point of $\shgf$, we can associated to it the Newton Polygon at $p$, which we denote as $\text{NP}(\Ab)$. This is a discrete invariant that categorizes the isogeny class of $p$-divisible group of $\Ab$. While all symmetric convex polygons with endpoints $(0, 0)$ and $(2g, g)$ and integral breaking point can occur as Newton polygons of $\Ab$ in $\ag$, Li, Mantovan, Prius and Tang show in \cite{Second Paper} that there is a natural lower bound for the Newton polygons that can occur for $\Ab \in T(\zgr)$. The Newton Polygon which they show is a lower bound is called the $\mu$-ordinary Newton Polygon for monodromy datum $\gra$. When $G$ is cyclic, the formula for $\mu$-ordinary polygon is given in \cite[proposition 4.3]{Second Paper}.


%\begin{theorem}\cite[Theorem 1.3.7]{Moonen EO type formula}\label{mu ordinary=p ordinary}
%An abelian variety parameterized by $\shgf$ achieves $p$-ordinary Ekedahl–Oort type if and only if its Newton polygon is $\mu$-ordinary.
%\end{theorem}
%Therefore, our problem becomes to finding a curve in the family $\mgra$ whose Jacobian has the $p$-ordinary Ekedahl–Oort type. We prove that when there are at most five branching point, such curve exists. In fact, they cut out an open dense subset in $\mgrab$

More precisely, we prove the following result. For $p\nmid |G|$, we refer to the lowest Ekedahl--Oort type and Newton polygon occurring on $\shgf_{\fpb}$ as $(G,\underline{f})$-ordinary.
\begin{theorem}\label{abelian cover}
Let $(G,r,\underline{a})$ be a monodromy datum for abelian $G$-covers of $\Po$, branched at $r$ points, and $p$ a rational prime.  Assume $r\leq 5 $ and 
%\begin{enumerate}
 %   \item $4 \leq r\leq 5$,
  %  \item $\sum_{k=1}^r \au_k=0$ in $G$,
   % \item $(\au_1, \dots, \au_r )$ generate $G$,
    %\item 
    $p>|G|(r-2)$. 
%\end{enumerate}
Then the Ekedahl–Oort type and Newton polygon of the generic $G$-cover of $\Po$ over $\fpb$, with monodromy datum $(G,r,\underline{a})$, are $(G,\underline{f})$-ordinary. % Consequently, the Newton Polygon   % an open and dense subset $U$ in $\mgra_\fpb$, such that for any $\xb \in U$, $J(\ct)$ achieves $p$-ordinary Ekedahl–Oort type. Consequently, the Newton Polygon of $J(\ct)$ at $p$ is $\mu$-ordinary.
\end{theorem}

The condition  $p>|G|(r-2)$ is the same as in \cite[Theorem 6.1]{Irene}. 
%By combining the above result with the constructions in \cite{clutching argument}, we also extend the statement to (infinitely many) abelian monodromy data with $r\geq 6$.
%
The restriction $r\leq 5$ is due to the limitation of our method.  For $r\geq 6$, we identify (infinitely many) cyclic monodromy data for which we can verify the statement by combining  Theorem \ref{abelian cover}  with \cite[Theorem 4.5]{clutching argument} (see Proposition \ref{two t plus n}). 

%Among the above four conditions, condition $2$ and $3$ are standard for $\gra$ to be monodromy datum. Condition $4$ is similar to that of Irene's. Condition $1$ is the most restrictive, and it is due to the limitation to carry out our method. 
%In particular, when $G$ is a cyclic group of order $l$, we get the following corollary about cyclic covers of $\Po$.
%\begin{corollary}\label{cyclic case}
%Suppose we are given monodromy datum $\lra=(l,r,a_1, \dots, a_r)$ and a prime $p$, such that $4 \leq r \leq 5$, $a_1 + \dots +a_r \equiv 0 \pmod{l}$, $\gcd(a_1, \dots, a_r,l)=1$ and $p>l(r-2)$. Then there exists $ \xb \in \zlr$ for which $J(\ct)$ achieves $p$-ordinary Ekedahl–Oort type. Consequently, the Newton Polygon of $J(\ct)$ is $\mu$-ordinary.
%\end{corollary}

We describe our strategy. 
In \cite{Irene}, Bouw considers ramified abelian prime-to-$p$ covers of curves, and the upper bound for the $p$-rank coming from the rank of the Hasse--Witt matrix (the Hasse--Witt invariant of a curve over $\fpb$). In \cite[Theorem 6.1]{Irene}, she proves that for the generic curve in the family the Hasse--Witt invariant agrees with the $p$-rank. When specialized to covers of $\Po$, the Hasse--Witt invariant agrees with the $(G,\underline{f})$-ordinary $p$-rank. Recently, in \cite{Moonen algorithm}, Moonen introduces the notion of Hasse--Witt triple for a smooth curve over $\fpb$, as a generalization of the Hasse--Witt invariant, and proves that it is equivalent to Ekedahl--Oort type of the Jacobian of the curve. He also gives an algorithm for explicitly computing the Hasse--Witt triple of a smooth curve given as a complete intersection.  In this paper, following \cite{Irene}, we prove that for the generic curve in the family the Hasse--Witt triple is $(G,\underline{f})$-ordinary.

By Lemma \ref{equivalence two}, we first reduce the statement to certain types of cyclic covers. After adapting Moonen's notion of Hasse--Witt triple to cyclic covers of $\Po$, 
in Theorem \ref{numerical criteria}, we give an explicit numerical criterium for the Ekedahl--Oort type of such a curve to be $(G,\underline{f})$-ordinary as (finitely many) rank conditions on the Hasse-Witt triple. That is, we describe the $(G,\underline{f})$-ordinary stratum in $\zgr$ as the complement to the vanishing locus of certain minors of (iterations of) the extended Hasse--Witt matrix.  When $r\leq 5$, we verify that these minors do not trivially vanish. The restriction $r\leq 5$ limits the number and complexity of the rank conditions, which grows with the number $r$ of branched points. 
%we extend the result to $r\geq 6$ under some strong restrictions on the monodromy datum.



%and the cyclic cover whose Jacobian fails to be of $p$-ordinary E-O type corresponds to the points on $\mgrab$ that lies in the vanishing locus of some determinant. 


%By adaptin Moonen's algorithm notion of Hasse-Witt triple to parameterize the E-O type, and we compute the coefficients of the matrices in Hasse-Witt triple. We then develop Theorem \ref{numerical criteria} as combinatorial criterion to detect when is the Dieudonne module attached to the Jacobian has $p$-ordinary E-O type. Our criterion is a rank condition on the Hasse-Witt triples, and the cyclic cover whose Jacobian fails to be of $p$-ordinary E-O type corresponds to the points on $\mgrab$ that lies in the vanishing locus of some determinant. 

The paper is organized as follows. 
In section \ref{pre}, we recall the notion of $(G,\underline{f})$-ordinary Ekedahl--Oort type and Moonen's notion of Hasse-Witt triple. In section \ref{reduction argument}, we reduce the statement to the case when $G$ is a cyclic group. In Section \ref{duality}, we introduce the Hasse-Witt triple of $J(C)$, for $C$ a cyclic cover of $\Po$. In section \ref{combi}, we described a criterion for $(G,\underline{f})$-ordinaryness in terms of the rank of extended Hasse--Witt matrix. In section \ref{max monomial}, we explicitly compute certain minors of the extended Hasse--Witt matrix.
%find the maximal monomial in each entry of $\phi,\psi$ under lexicographical order. This will be needed to show the non-vanishing of a certain determinant. 
 %following a strategy similar to that \cite{Irene}, and 
In Section \ref{Section 0,1 in signature}, by Theorem \ref{numerical criteria}, we prove Theorem \ref{abelian cover} some additional conditions.
%an important case where the assumption for Theorem \ref{numerical criteria} is satisfied.
In Section \ref{new part mu ordinary},  we deduce Theorem \ref{abelian cover} from the special instances.
In Section \ref{list}, by combining Theorem \ref{abelian cover} with \cite[Theorem 4.5]{clutching argument}, we construct (infinitely many) cyclic monodromy data $\gra$ with $r \geq 6$ such that the intersection of $T(\mgra)$  with the $(G,\underline{f})$-ordinary locus of $\shgf_{\fpb}$ is non empty.



\section{Preliminaries}\label{pre}



\subsection{The Hurwitz space $\mgra$} 
We briefly recall the definition of the Hurwitz space of abelian cover. We refer to \cite{moduli functor G cover} for more complete description on the construction of the moduli functor.

Fix a finite abelian group $G$ and a positive integer $r \geq 3$. 
\begin{definition}\label{monodromy datum}
Consider $\au=(\au_1,\au_2,\dots, \au_r)\in G^r$. 
   The $r$-tuple $\au$ is called an inertia type
   for $G$ if it satisfies the following properties:
\begin{enumerate}
    \item $\au_i \neq 0$ in $G$
    \item $\au_1, \dots, \au_r$ generate $G$
    \item $\sum_{i=1}^r \au_i = 0$ in $G$.
\end{enumerate}
A monodromy datum for $G$ is given as $(G,r,\au)$ where $\au\in G^r$ is an inertia type for $G$. 

Two monodromy data $(G,r,\au)$ and $(G,s,\ub)$ are equivalent if $r=s$ and $\au, \ub \in G^r$ are in the same orbit under ${\rm Aut}(G) \times \text{Sym}_r$. 
\end{definition}

Let $\mg$ be the moduli space of smooth projective genus $g$ curve and $\overline{\mg}$ be its Delign-Mumford compactification, so it is the moduli space of stable curves of genus $g$. While both are algebraic stacks defined over $\Q$, they have a model defined over some open subset of $\Spec(\Z)$.

Let $p$ be a rational prime, $p \nmid |G|$. We use $e$ to denote the exponent of $G$ and consider schemes over $\Z[\frac{1}{e}, \zeta_e]$ where $\zeta_e$ is a primitive $e^{\text{th}}$ root of unity. Let $\mgb$ be the moduli functor on the category of schemes over $\Z[\frac{1}{e}, \zeta_e]$ that classifies admissible stable $G$-covers of $\Po$, and denote by $\mG$  the smooth locus of $\mgb$. Both $\mG$ and $\mgb$ have good reduction modulo $p$. Within each irreducible component of $\mG$, the monodromy datum of the parameterized curves is constant.  Conversely, given a monodromy datum $(G,r,\au)$, the substack $\mgra$ of $\mG$ parametrizing $G$-covers with monodromy $(G,r,\au)$ is irreducible. 
%That is, $\mgra$ classifies the smooth $G$ covers of $\Po$ branched at $r$ points, with the local monodromy $\au_i$ at the $i^{th}$ branched points. 

\subsection{The Shimura variety $\shgf$}\label{Shimura}  We briefly recall the construction of the PEL type moduli spaace $\shgf$. We refer to \cite[Section 3.2, 3.3]{Second Paper} for more details.

Let $\mathcal{A}_g$ denote the moduli space of principally polarized abelian varieties of dimension $g$, and $T:\mathcal{M}_g\to \mathcal{A}_g$ the Torelli morphism. 
For $\gra$ a monodromy datum as in Definition \ref{monodromy datum},  we denote by $S\gra$ the largest closed, reduced and irreducible substack of $\mathcal{A}_g$ containing $T(\mgra)$ such that the action of $\Z[G]$ on the Jacobian  of the
universal family of curves over $\mgra$ extends to the universal abelian scheme over $S\gra$. By \cite{delignmostow},  $S\gra$  is an irreducible component of a PEL type moduli space, which we denote by $\shgf$  (see also \cite[Section 3]{Second Paper}).  The moduli space $\shgf$ has a canonical model over $\Z[\frac{1}{e}, \zeta_e]$, and good reduction at all primed $p\nmid |G|$.

We recall the definition of the PEL datum associated with $\shgf$. 
Let $\Q[G]$ denote the $\Q$-group algebra of $G$,  and $*$ the involution on $\Q[G]$ induced from the group homomoprhism $g\mapsto g^{-1}$.  Fix $\xb \in \mgra(\C)$, and denote by $C=\ct$  the associated curve over $\C$. Let  $V=H^1(C,\Q)$ be the first Betti cohomology group of $C$; the action of $G$ on $C$ induces a structure of $\Q[G]$-vector space on $V$. We denote by $\langle\cdot,\cdot\rangle$ the standard skew Hermitian form on $V$, and by $h$ the Hodge structure on $V$. 
The Shimura datum of $\shgf$ is defined by the PEL-datum $(\Q[G], *, V, \langle\cdot, \cdot\rangle, h)$, and is independent of the choice of $x$.  
We denote by $\fb$  the {\em signature} of the group of similitudes $\mathcal{G}=GU(V, \langle\cdot, \cdot\rangle )$ associated with $\shgf$.
We denote the Hodge structure $h$ of $V$ by $V\otimes_\Q\C=V^+\oplus V^-$, where $V^+=H^0(C,\Omega^1)$, via the Betti-de Rham comparison isomorphism.
Let $\tg$ be the group of characters $\Hom(G,\C^*)$. We define $\fb:\tg\to \Z$  as  $f(\tau)=\dim( {V}^{+}_{\tau})$, where for each $\tau\in \tg$ we denote by $V_\tau^+$ the subspace of $V^+$ of weight $\tau$. By definition, $f(\tau^*)=\dim V^-_\tau$, and for each $\tau\in \tg$ the pair $(f(\tau),f(\tau^*))$ is the signature of the group $GU(V, \langle\cdot, \cdot\rangle )$ at the real place underlying $\tau$ and $\tau^*$.
The signature $\fb$ can be computed explicitly from the monodromy datum $\gra$ via the Hurwitz--Chevalley-Weil formula (see \cite[Theorem 2.10]{Chevalley}).

With abuse of notation, in the following we denote by $(G,\fb)$ the Shimura datum of $\shgf$.



%\begin{comment}

%Given a monodromy datum $\gra$, it determines the signature $\fb=(f(\tau))_{\tau}$, which is a datum that encodes the action of $\Q[G] \otimes_\Q \C$ on the cohomology of parameterized curve. It turns out that $\fb$ is constant on $\mgra$, and it can be computed as follows. Let $\tg$ be the %group of characters $\Hom(G,\C^*)$. For $\ct$ parameterized by $\xb \in \mgra(\C)$, denote $\ct$ as $C$ and denote $H^1(\ct,\Q)$ as $V$. The action of $G$ on $C$ makes the first Betti cohomology $H^1(C,\Q)$ into a $\Q[G]$ module, so $\text{V}=H^1(C,\Q) \otimes_{\Q} \C$ is a $\Q[G] \otimes_{\Q} %\C$ module. Since $\Q[G] \otimes_{\Q} \C \cong \prod_{\tau \in \tg}\C_{\tau}$, $\text{V} \cong \bigoplus_{\tau \in \tg}\text{V}_{\tau}$, where $g \in G$ acts on $\text{V}_{\tau}$ via scalar multiplication by $\tau(g)$.
%On the other hand, we have the Hodge decomposition $H^1(C,\Q) \otimes_{\Q} \C \cong H^1(C,\OO) \oplus H^0(C,\Omega)$. We denote  $H^0(C,\Omega)$ by $\text{V}^{+}$ and $H^1(C,\OO)$ by $\text{V}^{-}$. Since both are $\Q[G] \otimes_{\Q} \C$ modules, we have decomposition
%\begin{align*}
%\text{V}^{+} &\cong \bigoplus_{\tau \in \tg}\text{V}^{+}_{\tau}, &\text{V}^{-} &\cong \bigoplus_{\tau \in \tg}\text{V}^{-}_{\tau},&\vs_\tau=&\vpl_\tau \oplus \vm_\tau.  \end{align*}
%We have the duality that ${\text{V}^{+}_{\tau}}^{\smvee}=\text{V}^{-}_{\check{\tau}}$. The signature $f_G: \tg \to \Z$ is given by $f(\tau)=\dim(\text{V}^{+}_{\tau})$. That is, $\fb$ determines the isomorphism class of $H^0(\ct,\Omega)_\tau$. Moreover, given the monodromy datum $\gra$, there is %an explicit formula to compute $f(\tau)$ for $\tau \in \tg$ as given in \cite[Theorem 2.10]{Chevalley}.

%For any $C \to \Po$ represented by a point of $\mgra$, its Jacobian $J(C)$ is equipped with a canonical principle polarization. The action of $G$ on $C$ induced extra endomorphism on $J(C)$ by the action of $\Z[G]$. Thus, the image of $\mgra$ under the Torelli map naturally lies inside a PEL-%type Shimura variety $\text{Sh}(G)$. It turns out that the smallest PEL-type Shimura variety containing $T(\mgra)$ is $\shgf$, which parameterize principal polarized abelian variety with extra endomorphism induced by $\Z[G]$ and signature type $\fb$. For an explicit construction of $\text{Sh}G)$ %and $\shgf$, see \cite[Section 3.2, 3.3]{Second Paper}.
%\end{comment}


 \subsubsection{The $(G,\underline{f})$-ordinary stratum at unramified primes}

Let $p$ be a prime not dividing $|G|$. Then $p$ is a prime of good reduction for $\shgf$ and by \cite{viehmann-wedhorn} both the Ekedahl--Oort and Newton stratification of $\shgf_{\fpb}$ are well understood. 

The Newton polygon is a discrete invariant that classifies the isogeny class of the $p$-divisible group of a polarized abelian variety over $\fpb$, and is known to induce a stratification on $\mathcal{A}_{g,\fpb}$. By \cite{viehmann-wedhorn}, the Newton polygons corresponding to non-empty strata in $\shgf_{\fpb}$ are in one-to-one correspondent with the elements in the associated Kottwitz set at $p$, its natural partial order agreeing with specialization on $\shgf_{\fpb}$. In \cite{Kottwitz}, this set is denoted by $B(\mathcal{G}_{\Q_p},\mu)$, where $\mathcal{G}=GU(V, \langle\cdot, \cdot\rangle )$ is the group of similitudes associated with $\shgf$ and $\mu=\mu_h$ is the $p$-adic cocharacter induced by the Hodge structure $h$.
By \cite{Rapoport-RIcharts},\cite{Wedhorn}, there is a unique maximal element / lowest polygon in $B(\mathcal{G}_{\Q_p},\mu_h)$, corresponding to the unique open (and dense) Newton stratum in  $\shgf_{\fpb}$; this is known as the $\mu$-ordinary polygon at $p$ and in our context can  be computed explicitly from the splitting behaviour of p in the group algebra $\Q[G]$ and the signature $\fb$ (for example, it is ordinary if $p$ is totally split in  $\Q[G]$).  

 The Ekedahl-Oort type is a discrete invariant that classifies the isomorphism class of the $p$-kernel of a polarized abelian variety over $\fpb$, and also induces a stratification on  $\mathcal{A}_{g,\fpb}$.  By \cite{viehmann-wedhorn}, the Ekedahl--Oort types corresponding to non-empty strata in $\shgf_{\fpb}$ are in one-to-one correspondent with certain elements in the Weyl group of the reductive group $\mathcal{G}$, their dimension equal to the lenght of the element in the Weyl group.
In particular, there is a unique element of maximal length, corresponding to the unique open (and dense) Ekedahl--Oort stratum in  $\shgf_{\fpb}$. The Ekedahl--Oort type corresponding to the maximal element is called $p$-ordinary. 

By \cite[Theorem 1.3.7]{Moonen EO type formula}, the $p$-ordinary Ekedahl--Oort stratum and
$\mu$-ordinary Newton stratum of $\shgf_{\fpb}$ agree.
As its definition depends on the Shimura datum $(G,\underline{f})$ and the prime $p$, we refer to it as the $(G,\underline{f})$-ordinary stratum at $p$, and denote the associated Newton polygon by $\mu_p(G,\fb)$.  
An explicit formula for the polygon $\mu_p(G,\fb)$ is given   \cite[Proposition 4.3]{Second Paper}, as a special case of that in \cite{Moonen EO type formula}. 
We briefly recall some aspects of its construction.

 
% Newton polygon is a discrete invariant on $\shgfb$ that classifies the isogeny class of $p$ divisible group of the abelian variety. For an abelian variety $A$ defined over $\fpb$, there exists a model of $A$ defined over $\fpn$, which we denote as $A_0$. Let $\text{W}(\fpn)$ denote the ring of Witt vector of length $n$, which we can think of as the ring generated by Teichmüller lift of root of unity in $\fpn$. Then we can lift $A_0$ to be defined over $\text{W}(\fpn)$ and consider its first crystalline cohomology $H^1(A_0/\text{W}(\fpn))$. Since $H^1(A_0/\text{W}(\fpn))$ only depends on $A$, we denote it as $\text{W}(A)$. $\text{W}(A)$ has the action of Frobenius, and the Newton Polygon of $A$, which we denote as $\text{NP}(A)$, is the Newton polygon of characteristic polynomial of Frobenius on $H^1(A_0/\text{W}(\fpn))$. For more details, see \cite[Section 2.3, 4.1]{Second Paper}. We use $B(G,\fb)$ to denote the Kottiwitz set of $\shgf$, which is the set of Newton Polygons occurring in $\shgf$. The unique maximum element of $B(G,\fb)$ is called the $\mu$-ordinary Newton polygon, which we denote as $\mu(G,\fb)$. The formula of $\mu_p(G,\fb)$ is given in \cite[Proposition 4.3]{Second Paper}.

 \subsubsection{The $(G,\underline{f})$-ordinary polygon}
Given the rational prime $p$, we fix an algebraic closure $\overline{\Q}_p$ of $\Q_p$ and an isomorphism $\iota:\Hat{\overline{\Q}}_p\simeq \C$.
We denote by $\overline{\Q}^{\rm un}_p$ the maximal unramified subfield of $\overline{\Q}_p$, and by $\fpb$ its residue field. 
Since $p\nmid |G|$, $\iota$ induces an isomorphism $\tg\simeq \Hom(G,\overline{\Q}^{\rm un}_p)$. 
Let $\sigp$ be the Frobenius element in $\gal(\fpb/\mathbb{F}_p)$, then $\sigp$ lifts to an element of $\gal(\overline{\Q}^{\rm un}_p/\Q_p)$, and we consider the action of $\sigp$ on $\tg$ by composition, that is $\tau^{\sigp}(x)=\sigp(\tau(x))=\tau(x)^p$.
This action partitions $\tg$ into Frobenius orbits, and we denote the set of Frobenius orbits of $\tg$ by $\og$. For $\tau \in \tg$, we use $\orho$ to denote the Frobenius orbit of $\tau$. 
%
The Frobenius orbits in $\og$ are naturally in one-to-one correspondence with the simple factors of $\Q_p[G]$.  From the decomposition into simple factors of $\Q[G]$, $\Q[G] \cong \prod_H K_H $ where $H$ varies among the subgroup of $G$ such that $G/H$ is cyclic, we deduce
\begin{equation}\label{simplefactors}
    \begin{split}
       \Q_p[G] &\cong \prod_{\substack{H \leq G \\ G/H \text{ cyclic }}}K_H \otimes_{\Q}\Q_p \cong  \prod_{\substack{H \leq G \\ G/H \text{ cyclic }}}\prod_{\substack{\orho\\
       \ker(\tau_{\vert G})=H }}K_{\orho},\\
    \end{split}
\end{equation}
where each Frobenius orbit $\OO$ corresponds to a prime $\p$ above $p$ in $K_H$, for $H=\ker(\tau_{\vert G})$,  and $K_{\OO}$ is the completion of $K_H$ at this prime. 



Let $C \to \Po$ be an abelian cover parameterized by a point $\xb \in \mgra(\fpb)$, we denote its Jacobian by $J(C)$. Let $\text{W}=\text{W}(J(C))$ denote the Diedonn\'e module of the abelian variety $J(C)/\fpb$, and $\text{NP}(\text{W})$ its Newton Polygon at $p$. 
Then, the structure of $\Z_p[G]$-module on  $\text{W}$ induces a decomposition up to isogeny  $\text{W} \sim \bigoplus_{\OO \in \og}\text{W}_{\OO}$, and hence an equality of Newton polygon  $\text{NP}(\text{W})=\bigoplus_{\OO \in \og}\text{NP}(\text{W}_{\OO})$. 

From the formula  given in \cite[Proposition 4.3]{Second Paper}, the Newton polygon $\mu_p(G, \fb)$ also decomposes as $\mu_p (G,\fb)=\bigoplus_{\OO \in \og}\mu(\OO)$, where for each orbit $\OO$ the polygon $\mu(\OO)$ only depends on the values $(f(\tau))_{\tau \in \OO}$. Furthermore, $\text{NP}(\text{W})= \mu_p(G, \fb)$ if and only if $\text{NP}(\text{W}_{\OO})=\mu(\OO)$, for each Frobenius orbit $\OO$.




\subsection{Hasse--Witt triples and Ekedahl--Oort types}\label{dieu}
% Moonen in \cite[Theorem 1.3.7]{Moonen EO type formula}, shows that the Newton polygon of $A$ at $p$ is $\mu$-ordinary if and only if the Diudonne module has $p$-ordinary Ekedahl–Oort type.
%We want to investigate whether or not the image of Torelli map intersects the $\mu$-ordinary Newton polygon stratum in $\shgf$. This is equivalent to determining whether the image intersects the $p$-ordinary Ekedahl–Oort stratum.
%In this subsection, we give some preliminaries about the Ekedahl–Oort type, which is a discrete invariant of abelian varieties defined over $\fpb$. See \cite{Moonen EO type formula, Moonen group scheme} by Moonen for more details.

%\subsubsection{Hasse-Witt triple}\label{Hasse Witt triple} 
Recall $\sigma$ denotes the Frobenius of $\fpb$.
Let $A$ be a principally polarized abelian variety of dimension $g$, defined over $\fpb$. Its Ekedahl-Oort type encodes the isomorphism class of $A[p]$, or equivalently the isomorphism class of the associated polarized mod $p$ Diedonn\'e module $(M,F,V,b)$, where
\begin{itemize}
    \item $M=H^1_{dR}(A/\fpb)$;
    \item $F: M \to M$ is the $\sigma$- linear map on $M$ induced by the Frobenius of $A$;
    \item $b: M \times M \to \fpb$ is the pairing induced by the polarization of $A$;
    \item $V: M \to M$ is the unique $\sigma^{-1}$-linear operator satisfying $b(F(x),y)=b(x,V(y))^{p}$.
\end{itemize}
%The quadruple $(M,F,V,b)$ is called the polarized mod $p$ Dieudonné module associated to $A$, and the Ekedahl-Oort type is the isomorphism class of this mod $p$ Dieudonné module. 

In \cite{Moonen algorithm}, Moonen establishes a equivalence of category between the polarized mod $p$ Dieudonné modules and Hasse-Witt triples, where he defines a Hasse-Witt triple $(Q,\phi,\psi)$ as follows:
%According to Moonen (*, cite reference), the isomorphism class of $\mod p$ Dieudonne module is uniquely determined by the associated Hasse-Witt triple $(Q,\phi,\psi)$. By a Hasse-Witt triple, we mean the following:
\begin{itemize}
    \item $Q$ is a finite dimensional vector space over $\fpb$;
    \item $\phi: Q \to Q$ is a $\sigma$-linear map;
    \item $\psi: \ker(\phi) \to \im(\phi)^{\perp}$ is a $\sigma$-linear isomorphism, where $\im(\phi)^{\perp}\subseteq Q^{\smvee} ={\rm Hom}_{\fpb} (Q,\fpb)$ is the subspace $\im(\phi)^{\perp}=\{\lambda \in Q^{\smvee}: \lambda(\phi(q))=0, \forall q \in Q$\}.
\end{itemize}
Under Moonen's equivalence of category,  the polarized mod $p$ Dieudonne module $(F,M,V,b)$ corresponding to a Hasse-Witt triple $(Q,\phi,\psi)$ is given by: 
\begin{itemize}
    \item $M = Q \oplus Q^{\smvee}$;
    \item $F: M \to M$ is defined as follows: set $R_1=\ker(\phi)$, choose $R_0$ a compliment of $R_1$ in $Q$, and write $M=(R_0\oplus R_1)\oplus Q^{\smvee}$, then $F(x+y,z)=(\phi(x),\psi(y))$, for any $x\in R_0,$ $y\in R_1$, $z\in Q^{\smvee}$;
    \item $b:M\times M\to \fpb$ is defined by $b((q,\lambda),(q',\lambda'))= \lambda'(q)-\lambda(q')$, for any $q,q'\in Q$ and $\lambda,\lambda'\in Q^{\smvee}$.
    \item  $V:M\to M$ is uniquely determined by $b(F(x),y)=b(x,V(y))^p$.
\end{itemize}


%\subsubsection{Hasse-Witt triple of cover of $\Po$}\label{Hasse Witt triple} 

%In \cite{Moonen algorithm}, Moonen given an explicit algorithm for computing the Hasse--Witt triple of (the Jacobian of) a complete intersection curve $C$ defined over $\fpb$. 
%Adapting  \cite[proof of Proposition 3.11 and formula (3.11.3)]{Moonen algorithm} to the special case of $C$ a $G$-cover of $\Po$, $\pi: C \to \Po$, using the coordinate charts $U_1=\pi^{-1}(\Po-\{\infty\})$ and $U_2=\pi^{-1}(\Po-\{0\})$,  we explicitly compute the Hasse-Witt triple $(Q,\phi,\psi)$ of $C$ as follows:
%\begin{itemize}
 %   \item $Q=\hy$, and $Q^{\smvee}=\hyd$;
  %  \item $\phi:\hy \to \hy$ given by the Hasse-Witt matrix;
   % \item $\psi: \ker(\phi) \to \im(\phi)^\perp$ given by $\psi(\alpha)=(df_{1,{\alpha}},-df_{2,{\alpha}}),$
    %where $(df_{1,{\alpha}},-df_{2,{\alpha}})$ denotes the global $1$-form on $C$ which restricts to $df_{1,{\alpha}}$ on $U_1$ and to $-df_{2,{\alpha}}$ on $U_2$, and $f_{1,{\alpha}} \in \OO_C(U_1)$ and $f_{2,\alpha} \in \OO_C(U_2)$ satisfy $\alpha^p=f_{1,{\alpha}}+f_{2,{\alpha}}$ 
    %(for $\alpha \in \ker(\phi)$, $f_{1,\alpha}$ and $f_{2,\alpha}$ exist, and by construction $df_{1,{\alpha}}$ and $-df_{2,{\alpha}}$ agree on $U_1\cap U_2$).
%\end{itemize}
%Then $(M,F,V,b)$ obtained from the Hasse-Witt triple via the correspondence is isomorphic to the polarized mod $p$ Dieudonne module associated to $J(C)$.

\subsubsection{The $(G,\fb)$-ordinary Ekedal--Oort type}
We recall the definition of the $p$-ordinary Ekedahl--Oort type for $\shgf$. Since it depends on the Shimura datum $(G,\fb)$ we also refer to it as the $(G,\fb)$-ordinary Ekedahl--Oort type at $p$.

Recall the identification $\tg=\Hom(G,\C^*) \cong \Hom (G,\overline{\Q}_p^{\rm un})$; it induces an isomorphism $\tg\cong \Hom(G,\fpb^*)$.  %, and $\fb: \tg \to \Z$ a signature. 
Let $A$ be an abelian variety  over $\fpb$ corresponding to a point of  $\shgf$, and denote its mod $p$ Dieudonné module by $(M, F,V,b)$. The action of $G$ on $A$ induces a structure of $\fpb[G]$-module on $M$. Hence, the Dieudonné module $M$ decomposes as 
%Now, since $p \nmid |G|$, the image of any character of $G$ is a root of unity with order prime to $p$.
%Let $\sigp$ be the Frobenius element in $\gal(\fpb/\mathbb{F}_p)$, then $\sigp$ acts on $\tg$ by $\tau^{\sigp}(x)=\sigp(\tau(x))=\tau(x)^p$.
% since, $\rho(x)$ is a root of unity whose order is not divisible by $p$.
%This action partitions $\tg$ into Frobenius orbits, and we denote the set of Frobenius orbits of $\tg$ by $\og$. For $\tau \in \tg$, we use $\orho$ to denote the Frobenius orbit of $\tau$. 
%We have a decomposition of $M$:
$$M=\bigoplus_{\tau\in\tg} M_\tau=\bigoplus_{\OO\in\og}M_{\OO},\text{ where } M_{\OO}=\bigoplus_{\tau\in \OO}M_{\tau}, $$
and  for each $\tau \in\tg$,  $M_\tau$ is the $\tau$-isotypic component of $M$. That is, each $g \in G$ acts on $M_\tau$ via multiplication by $\tau(g)$. For $\tau\in \tg$, $g(\tau)=\dim_{\fpb} (M_{\tau})$ depends only on the Frobenius orbit that $\tau$; hence we write  $g(\OO)=g(\tau)$, for all $\tau \in \OO$. 

For simplicity, given $\tau\in\tg$, we write $\tau^{\sigp}$ as $p\tau$ and its orbit $\OO_{\tau}=\{\tau,p\tau,\dots,p^{|\OO_{\tau}|-1}\tau\}$. Since $p(p^{|\OO_\tau|-1}\tau)=\tau$, we also write $p^{|\OO_\tau|-1}\tau$ as $\frac{\tau}{p}$.  
Then, $F$ maps $M_{\tau}$ to $M_{p\tau}$, and $V$ maps $M_{\tau}$ to $M_{\frac{\tau}{p}}$. 

Recall, for $\tau\in\tg$, $\tau^*\in\tg$ is defined as $\tau^*(x)=\tau(x)^{-1}$. Each orbit $\OO$ we denote by $\OO^*$ its conjugate orbit ${\OO}^*=\{\tau^*\mid \tau\in \OO\}$. The polarization $b:M \times M \to \fpb$ identifies $M^{\smvee}$ with $M$, $M_\OO^{\smvee}$ with $M_{\OO^*}$ and $M_{\tau}^{\smvee}$ with $M_{\tau^*}$.

\begin{definition}
    
The $(G,f)$-ordinary mod $p$ Diedonn\'e module $(M,F,V,b)$ is given as follows. 
%
%Set $M=\oplus_{\tau\in \tg } M_\tau$, where each $\M_\tau$ is a $\fpb$-vector space of dimension $g(\tau)=g(\OO_\tau)$. 
Let $\{e_{\tau,j}\mid 1\leq j\leq {g(\OO_\tau)}\}$ be a $\fpb$-basis of $M_\tau$, and denote the dual basis on $M_\tau^{\smvee}$ by $\{\check{e}_{\tau,j}\mid 1\leq j\leq {g(\OO_\tau)}\}$. 
%For each $1\leq i\leq g(\OO_\tau)$, we denote the $\fpb$ span of $\{e_{\tau,1},\dots,e_{\tau,i}\}$ as $M_{\tau,i}$. 
Then the polarization $b$ on $M$, in terms of the induced isomorphisms $M_\tau^{\smvee}\simeq M_{\tau^*}$  for $\tau\in\tg$, is given by $\check{e}_{\tau,j}\mapsto e_{\tau^*,g(\OO)+1-j}$, for $1\leq j\leq g(\OO_\tau)$.
The action of $F$ and $V$ on $M$, when restricted to $M_\tau$ for $\tau\in\tg$, are given by
%\begin{definition}
    %The Dieudonné module $(F,V,M,b)$ with action of $G$ and signature $\fb$ is said to $p$-ordinary if its associated coset representatives $w_{\tau}$ is of maximal length for every $\tau$. Equivalently, $\ker(F)\cap M_{\tau,f(\tau^*)}=\{0\}$ for every $\tau$. 
%\end{definition}
%\begin{theorem}\cite{??}\label{Thm: mu ordinary p-ordinary}
%Suppose $A$ is parameterized by $\shgf$. Then $A$ is $p$-ordinary if and only if its associated Newton polygon is $\mu$-ordinary.
%\end{theorem}
%When the $(G,\fb)$ are fixed, $\tau\in T_G$, the permutation of maximal length is $\wt$ given by as follows: for $k\leq \fts$, $\wt(k)=f(\tau)+k$ and for $k>\fts$, $\wt(k)=k-\fts$.
    %$$\wt(j) = \begin{cases} j+\ft &\text{ if } j \leq g(\OO)-\ft \\ j+\ft-g(\OO) &\text{ if } j>g(\OO)-\ft \end{cases}$$
%Consequently, $(F,V,M,b)$ is $p$-ordinary if there is a basis $\{e_{\tau,j}: 1 \leq j \leq g(\OO)\}$ of $M_\tau$, such that 
%Then $F$ and $V$ acts as
\begin{align}
F(e_{\tau,j})&=\begin{cases}
    e_{p\tau,j} &\text{ if } j \leq \fts\\
    0 &\text{ if } j \geq \fts+1,
    \end{cases}
&
V(e_{p\tau,j_1})&=\begin{cases}
    0 &\text{ if } j_1 \leq f(\tau^*)\\
    e_{\tau,j_1} &\text{ if } j_1 \geq f(\tau^*)+1.
    \end{cases}
\end{align}

\end{definition}


\begin{remark}
Under Moonen's equivalence, the $(G,f)$-ordinary Hasse--Witt triple $(Q, \phi,\psi)$ is defined as follows.
Let $Q=\ker(F)^{\smvee}\subseteq M$ and define $Q_{\tau}=Q\cap M_{\tau}$, for each $\tau\in\tg$. Write $Q_{\tau^*}^{\smvee}=(Q_{\tau^*})^{\smvee}$ (in general $Q_{\tau^*}^{\smvee}$ is not $Q^{\smvee}\cap M_{\tau^*}$).
Then $M_{\tau}=Q_{\tau} \oplus Q_{\tau^*}^{\smvee}$, %and we denote by $\pi_{\tau}$ the projection from $M_{\tau}$ to $Q_{\tau}$, whose kernel is $Q_{\tau^*}^{\smvee}$.
where the set $\{e_{\tau,i_{\tau,1}},\dots, e_{\tau,i_{\tau,f(\ts)}}\}$ is a basis of $Q_{\tau}$ and $\{e_{\tau,j_{\tau,1}},\dots, e_{\tau,j_{\tau,f(\tau)}}\}$ is a basis of $Q_{\tau^*}^{\smvee}$.
%
%When specializing to the abelian varieties coming from the Jacobian of a smooth curve, the associated Dieudonné module can be classified by the notion of Hasse-Witt triple as defined by Moonen in \cite{Moonen algorithm}. The Hasse-Witt triple can be computed explicitly in terms of the cohomology of the curve. In this subsection, we give some preliminaries for Hasse-Witt triples of the jacobians of curves that arise as cyclic cover of $\Po$. See \cite{Moonen algorithm} for more details.
%
%Now for a monodromy datum $\gra$ whose signature is $\fb$, we know that $T(\mgrab)$ lies inside $\shgfb$. Then, for each $\ct$ parameterized by $\xb \in \mgra(\fpb)$, it has the associated Hasse-Witt triple $(Q,\phi,\psi)$ and mod $p$ Dieudonne module $(F,M,V,b)$, both of which have decomposition into $\tau$ isotypic parts. If we restrict the action of $F$ to $M_{\tau}$, then $F$ sends $Q_{\ts}^{\smvee}$ to zero. To present $V^{-1}$, we define $V':M\to M$ as $V'(x)=F(\check{x})^{\smvee}$,  where $F(\check{x})^{\smvee}$ means to take the dual of $x$, apply $F$, and take the dual of the resulting vector. 
With respect to this choice of bases for $\qt,\qtsd$, for all $\tau\in\tg$, the matrix of $F$ %(respectively $V'$) 
restricted to $M_\tau$, that is $F: M_\tau=Q_{\tau} \oplus Q_{\tau^*}^{\smvee}\to M_{p\tau}=\qpt \oplus \qptsd$ is %described in Equation \ref{Eqn: Ftau V'tau}.
\begin{align}\label{Eqn: Ftau V'tau}
F_{\tau}= \begin{bmatrix}
   % & \qt & \qtsd \\
    %\qpt & 
    \phi_\tau & 0\\
    %\qptsd & 
    \psi_\tau & 0 \\
\end{bmatrix},%, &
%V'_{\tau} &= \kbordermatrix{   & \qt & \qtsd \\  \qpt & 0 & \psi_{\ts}\\    \qptsd & 0 & \phi_{\ts}  }.
\end{align}
where $\phi_\tau$ (respectively $\psi_\tau$) is the matrix of $\phi$ (respectively $\psi$) restricted to $Q_{\tau}$, that is $\phi_\tau:Q_\tau\to\qpt$ (respectively $\psi_\tau:\qt \to \qptsd$).
%it parameterizes the curve $\ct$, which is the normalization of the curve with affine equation
%$$C':y^l=(x-x_1)^{a(1)}(x-x_2)^{a(2)} \dots (x-x_r)^{a(r)}.$$
%Notice that when $l \neq a(1)+ \dots +a(r)$, $C'$ has singularity at $[0:1:0]$. And if $a(i)>1$, $C'$ also has singularity at $[x_i:0:1]$. We denote $C$ to be its normalization. Then, its Jacobian $J(C)$ is an abelian variety defined over $\fpb$ and thus has the associated mod $p$ Dieudonné module $(F,V,M,b)$ as described in section \ref{dieu}.
\end{remark}






%Let $Q=\ker(F)^{\smvee}$ and $Q_{\tau}=Q\cap M_{\tau}$. Then we have $M_{\tau}=Q_{\tau} \oplus Q_{\tau^*}^{\smvee}$. Note that by $Q_{\tau^*}^{\smvee}$ we mean $(Q_{\tau^*})^{\smvee}$, which is ditinct from $Q^{\smvee}\cap M_{\tau^*}$. Let $\pi_{\tau}$ be the projection from $M_{\tau}$ to $Q_{\tau}$, whose kernel is $Q_{\tau^*}^{\smvee}$.


\subsubsection{Elements in the Weyl group}

\newcommand{\EO}{Ekedahl--Oort }

\newcommand{\Sym}{{{\rm Sym}}}
By \cite{viehmann-wedhorn}, the \EO types associated with non-empty strata of $\shgf_{\fpb}$ are in one-to-one correspondence with certain elements in the Weyl group of the reductive group $\mathcal{G}$. We recall this construction.

Consider the set $${\rm Weyl}(G,f)=\prod_{\tau \in \tg}\Sym_{g(\OO)}/W_{f,\tau},$$
where, for each $\tau\in\tg$, $W_{f,\tau}=\Sym\{1, \dots,f(\tau)\} \times \Sym\{f(\tau)+1, \dots, g(\OO)\}.$
%
Then, the non-empty \EO strata of $\shgf_{\fpb}$ are in one-to-one correspondence with the cosets in ${\rm Weyl}(G,f)$ defined by elements
$w=(\wt\mid \tau\in\tg)\in \prod_{\tau \in \tg} \Sym_{g(\OO_\tau)}$ %/ W_{f,\tau} $ 
satisfying $w_{\tau^*}(j)=g(\OO)+1-w_{\tau} (g(\OO)+1-j)$. In particular, the open Ekedahl–Oort stratum corresponds to the unique element $w$ such that the permutations $\wt$ have maximum length, for all $\tau\in\tg$. 
%The modulo $p$ Dieudonn\'e modules in the corresponding isomorphism class are called $p$-ordinary. Since the notion depends on the Shimura datum $(G,f)$, in the following we also refer to it as $(G,f)$-ordinary at $p$.

The \EO-type of $A$ is defined in terms of the canonical filtration of $M$, of length $2g=\dim(M)$,  obtained by repeatedly applying $F$ and $V^{-1}$ to $M$. By projecting the filtration to $M_{\tau}$, we obtain a filtration of $M_\tau$, of length $g(\OO_\tau)=\dim(M_{\tau})$,
$0\subsetneq M_{\tau,1} \subsetneq M_{\tau,2} \subsetneq \dots, \subsetneq M_{\tau,g(\OO)}=M_{\tau}.$
%
To each $\tau$, we associate a permutation $w_{\tau}\in \Sym_{g(\OO)}$ as follows. 
For $1\leq j\leq g(\OO)$, denote $\eta_{\tau,j}=\dim(\ker(F)\cap M_{\tau,j})$. Note that $\eta_{\tau,j}\leq \eta_{\tau,j+1}\leq \eta_{\tau,j}+1$. 
 Recall that by definition, the signature $f$ satisfies $f(\tau)=\dim(\ker(F)\cap M_{\tau})$ and $f(\tau)+f(\tau^*)=g(\OO)$.
We deduce that $0\leq \eta_{\tau,j}\leq f(\tau)$. We record the $k^{th}$ position at which there is a jump by $j_{\tau,k}$. We obtain $1\leq j_{\tau,1}<j_{\tau,2}<\dots<j_{\tau,f(\tau)}\leq g(\OO)$ satisfying $\eta_{\tau,j_{\tau,k}}=\eta_{\tau,j_{\tau,k}-1}+1$. Let $i_{\tau,1}<\dots<i_{\tau,f(\tau^*)}$ be the remaining indices, they satisfy  $\eta_{\tau,i_{\tau,k}}=\eta_{\tau,i_{\tau,k}-1}$. 

\begin{definition}
The \EO type of $A$ is the coset in ${\rm Weyl}(G,f)$ of the element $w=(w_\tau\mid \tau\in\tg)$ where  $w_{\tau} \in \Sym_{g(\OO_\tau)}$is given by
$$w_{\tau}(j_{\tau,k})=k \text{ and } w_{\tau}(i_{\tau,k})=f(\tau)+k.$$
\end{definition}
By definition, $w_\tau\in \Sym_{g(\OO_\tau)}$ satisfies the property 
\begin{equation}\label{wordproperty}
 \text{ if } j'< j \text{ and } \wt(j') > \wt(j) \text{ then }\wt(j) \leq f(\tau) < \wt(j')
\end{equation}
%Let $W_{f,\tau}=\Sym\{1, \dots,f(\tau)\} \times \Sym\{f(\tau)+1, \dots, g(\OO)\}$, and consider the set $\Sym_{g(\OO)}/W_{f,\tau}$. 
Furthermore, $\wt$ is the unique element in its coset in $S_{g(\OO)}/W_{f,\tau}$ that satisfies (\ref{wordproperty}).


\begin{lemma}\cite[2.3.4]{Moonen dimension formula}\label{length}
The permutation $w_{\tau}\in \Sym_{g(\OO)}$ has length
$$\sum_{k=1}^{f(\tau)}j_{\tau,k}-w_{\tau}(j_{\tau,k})=\sum_{k=1}^{f(\tau^*)} w_{\tau}(i_{\tau,k})-i_{\tau,k}.$$
Moreover, this quantity is maximized if and only if $\ker(F)\cap M_{\tau,f(\tau^*)}=\{0\}$.
\end{lemma}
\begin{remark}
The condition $\ker(F)\cap M_{\tau,f(\tau^*)}=\{0\}$ is equivalent to the equalities $j_{\tau,k}=\fts+k$ for $1\leq k\leq f(\tau)$,  and  $i_{\tau,k}=k$ for $1\leq k\leq \fts$. We deduce the $w_{\tau}$ has maximal length if and only if $w_{\tau^*}$ has maximal length, if and only if  
\[\wt(k)=\begin{cases} f(\tau)+k \text{ for } k\leq \fts,\\   k-\fts \text{ for } k>\fts.\end{cases} \]
\end{remark}



%%%%%%%%%%%%%%%%%%%

%With $G$ and $\fb$ being fixed, the Dieudonné module has additional structure given by the action of $F$ and $V$, and the structure can be parametrized by the $|G|$ tuple $(\wt)_{\tau \in \tg}$. Given $(\wt)_{\tau \in \tg}$, consider the following standard Dieudonné module $M((\wt)_{\tau})$: for each $\tau$, let $e_{\tau,1},\dots,e_{\tau,g(\OO)}\in M_{\tau}$ be a $\fpb$ basis of $M_\tau$. For each $1\leq i\leq g(\OO)$, we denote the $\fpb$ span of $\{e_{\tau,1},\dots,e_{\tau,i}\}$ as $M_{\tau,i}$. The duality of $M$ is given by $\check{e}_{\tau,j}=e_{\tau^*,g(\OO)+1-j}$. The set $\{e_{\tau,i_{\tau,1}},\dots, e_{\tau,i_{\tau,f(\ts)}}\}$ is a basis of $Q_{\tau}$ and $\{e_{\tau,j_{\tau,1}},\dots, e_{\tau,j_{\tau,f(\tau)}}\}$ is a basis of $Q_{\tau^*}^{\smvee}$.
%The action of $F$ and $V$ on $M_{\tau}$ is determined by the words $(\wt)_{\tau \in \tg}$ as follows:
%\begin{align*}
%F(e_{\tau,j})&=\begin{cases}
%    e_{p\tau,w_{\tau}(j)-f(\tau)} &\text{ if } w_{\tau}(j) \geq f(\tau)+1\\
%    0 &\text{ if } w_{\tau}(j) \leq f(\tau),
%    \end{cases}
%&
%V(e_{p\tau,j_1})&=\begin{cases}
%    0 &\text{ if } j_1 \leq f(\tau^*)\\
%    e_{\tau,w_{\tau}^{-1}(j_1-f(\tau^*))} &\text{ if } j_1 \geq f(\tau^*).
%   \end{cases}
%\end{align*}
%This can alternately be stated as
%\begin{align*}
%&F(e_{\tau,j})=e_{p\tau,\wt(j)-f(\tau)} &\text{and }& V(\check{e}_{p\tau,\wt(j)-f(\tau)})=\check{e}_{\tau,j}, & \text{if }& w_{\tau}(j) \geq f(\tau)+1,  \\
%&F(e_{\tau,j})=0 &\text{and }& V(\check{e}_{p\tau,\wt(j)+g(o)-f(\tau)})=0, &\text{if }& w_{\tau}(j) \leq f(\tau).
%\end{align*}

%It turns out that each mod $p$ Dieudonné module $(F,V,M,b)$ is isomorphic to $M((\wt)_{\tau})$, in which $((\wt)_{\tau})$ is the word obtained from $M$ by the algorithm described earlier in this subsection. Thus, the tuple of permutations $(\wt)_{\tau \in \tg}$ determines the isomorphism class of the Dieudonné module.
%That is, With $G$ and $\fb$ fixed, there is a one to one correspondence 
%\begin{align*}
 %  &\{ \text{ isomorphism classes of Dieudonné module } (F,V,M,b)\} \iff \\
%&\Big\{ (\wt)_{\tau \in \tg} \in \prod_{\tau \in \tg}W_{X,\tau} / S_{g(\OO)}\mid w_{\tau^*}(j)=g(\OO)+1-w_{\tau}(g(\OO)+1-j)\Big\}. 
%\end{align*}



%%%%%%%%%%%%%%%
%By \cite{viehmannwedhorn}, the non-empty \EO strata of $\shgf_{\fpb}$ are in one-to-one correspondence with the elements $w=(\wt)_{\tau \in \tg} \in \prod_{\tau \in \tg} \Sym_{g(\OO_\tau)}/ W_{f,\tau} $ satisfying $w_{\tau^*}(j)=g(\OO)+1-w_{\tau} (g(\OO)+1-j)$. In particular, the open Ekedahl–Oort stratum corresponds to the unique element $w$ such that the permutations $\wt$ have maximum length, for all$\tau\in\tg$. 
%The modulo $p$ Dieudonn\'e modules in the corresponding isomorphism class are called $p$-ordinary. Since the notion depends on the Shimura datum $(G,f)$, in the following we also refer to it as $(G,f)$-ordinary at $p$.

%We explicitly describe the $(G,f)$-ordinary mod $p$ Diedonn\'e module $(M,F,V,b)$. 

%Set $M=\oplus_{\tau\in \tg } M_\tau$, where each $\M_\tau$ is a $\fpb$-vector space of dimension $g(\tau)=g(\OO_\tau)$. 
%Let $\{e_{\tau,1},\dots,e_{\tau,g(\OO_\tau)}\}$ be a $\fpb$-basis of $M_\tau$, and denote the dual basis on $M_\tau^{\smvee}$ by $\{\check{e}_{\tau,j}\}$. 
%For each $1\leq i\leq g(\OO_\tau)$, we denote the $\fpb$ span of $\{e_{\tau,1},\dots,e_{\tau,i}\}$ as $M_{\tau,i}$. 
%Then the polarization $b$ on $M$, in terms of the induced isomorphisms $M_\tau^{\smvee}\simeq M_{\tau^*}$, is given by $\check{e}_{\tau,j}\mapsto e_{\tau^*,g(\OO)+1-j}$, for all $\tau\in\tg$ and $1\leq j\leq g(\OO_\tau)$.
%The action of $F$ and $V$ on $M$, when restricted to $M_\tau$ for $\tau\in\tg$, are given by

%\begin{definition}
    %The Dieudonné module $(F,V,M,b)$ with action of $G$ and signature $\fb$ is said to $p$-ordinary if its associated coset representatives $w_{\tau}$ is of maximal length for every $\tau$. Equivalently, $\ker(F)\cap M_{\tau,f(\tau^*)}=\{0\}$ for every $\tau$. 
%\end{definition}
%\begin{theorem}\cite{??}\label{Thm: mu ordinary p-ordinary}
%Suppose $A$ is parameterized by $\shgf$. Then $A$ is $p$-ordinary if and only if its associated Newton polygon is $\mu$-ordinary.
%\end{theorem}
%When the $(G,\fb)$ are fixed, $\tau\in T_G$, the permutation of maximal length is $\wt$ given by as follows: for $k\leq \fts$, $\wt(k)=f(\tau)+k$ and for $k>\fts$, $\wt(k)=k-\fts$.
    %$$\wt(j) = \begin{cases} j+\ft &\text{ if } j \leq g(\OO)-\ft \\ j+\ft-g(\OO) &\text{ if } j>g(\OO)-\ft \end{cases}$$
%Consequently, $(F,V,M,b)$ is $p$-ordinary if there is a basis $\{e_{\tau,j}: 1 \leq j \leq g(\OO)\}$ of $M_\tau$, such that 
%Then $F$ and $V$ acts as













\section{Reduction from abelian cover to cyclic cover}\label{reduction argument}
%In the previous sections we have been solely considering the family of cyclic covers of $\Po$.

In this section, we reduce the proof of Theorem \ref{abelian cover} to the cyclic case. More precesily, we show that an abelian $G$-cover of $\Po$ is $(G,\fb)$-ordinary if its cyclic quotients are. 
%As a result, to prove Theorem \ref{abelian cover} for abelian covers, it will suffice to prove it for certain Frobenius orbits of cyclic covers.

%We fix a finite abelian group $G$ such that $p\nmid |G|$. 
Recall the identification $\tg= \Hom(G,\C^*) =\Hom(G,\overline{\Q}^*)\simeq \Hom(G,\fpb^*)$, and consider the action of $\galq$ on $\tg$ by composition on the left. 
%Therefore, $\galq$ acts on $\tg$ and $\tgh$, such that for $\rho \in \tg$ and $\sigma \in \galq$, we have $\rho^{\sigma}(x)=\sigma(\rho(x))$. The $\galq$ action partitions $\tg$ into Galois orbit. On the other hand, 
For any subgroup $H \leq G$, we denote $\tg^H= \{ \tau \in \tg | H \subseteq \ker(\tau)\}$ and $\tg^{H,\text{new}}=\{\tau \in \tg | H=\ker(\tau)\}$. If $H=\{1\}$, we write $\tgun=\{\tau \in \tg| \ker(\tau)=\{1\}\}$. Consider the partition 
$$\tg=\bigcup_{\substack{H \leq G,\\ G/H \text{cyclic}}}\tguh=\coprod_{\substack{H \leq G,\\ G/H \text{cyclic}}}\tghn$$
Then the action of $\galq$ on $\tg$ preserve the partition, and for each $H$, with $G/H$ cyclic,  $\galq$ acts transitively on $\tghn$. 

%\begin{lemma}\label{galois orbit subgroup}
%$\galq$ acts transitively on $\tghn$.
%\end{lemma}
%\begin{proof}
% Suppose $\rho_1, \rho \in \tg$ both kernel $H$ and that $|G/H| \cong \Z/m\Z$. Choose $g \in G/H$ that generates $G/H$. Then $\rho(g)$ and $\rho_1(g)$ are $m^{th}$ root of unity, say $\rho(g)=\zeta_m$ and $\rho_1(g)=\zeta_m^a$ where $\gcd(a,m)=1$. Now there is a unique $\sigma_1\in \gal(\Q(\zeta_m)/\Q)$ that sends $\zeta_m$ to $\zeta_m^a$. This can be extended to $\sigma\in\galq$, so we have $\rho_1=\rho^{\sigma}$.
%\end{proof}

%\begin{lemma}\label{galois orbit subfield}
%   The subfields $K$ of the group algebra $\Q[G]$ are in one to one correspondence with $\galq$ orbits of $\tg$.
%\end{lemma}
%\begin{proof}
%Denote $R=\Q[G]$. Given a character $\tau\in \tg$, we obtain a ring homomorphism $\phi_{\tau}:R\to \overline{\Q}$. Denote its kernel as $\m_{\tau}=\ker(\phi_{\tau})$. First, we will show that this only depends on the Galois orbit of $\tau$. Consider $\sigma\in\gal(\overline{\Q}/\Q)$, then $\sum a_g g\in\m_{\tau}$ if and only if $\sum a_g \tau(g)=0$, which happens if and only if $\sum a_g \tau^{\sigma}(g)=0$, that is, $\sum a_g g\in\m_{\tau^{\sigma}}$. Therefore, the maximal ideal $\m_{\tau}$ only depends on the Galois orbit of $\tau$.

%Next, we will show that distinct Galois orbits give distinct maximal ideals. Suppose $\m_{\tau_1}=\m_{\tau_2}$, then $\phi_{\tau_1}(R)\cong R/\m_{\tau_1}\cong \phi_{\tau_2}(R)$. Extend this isomorphism to an automorphism $\sigma$ of $\overline{\Q}$. It follows that $\tau_2=\tau_1^{\sigma}$.

%Finally we show that all maximal ideals of $R$ are obtained as $\m_{\tau}$. Let $\m$ be a maximal ideal of $R$. Let $i$ be an embedding of $R/\m$ into $\C$. Define a character $\tau$ as $\tau(g)=i(g\pmod{\m})$. We see that $\m=\m_{\tau}$.
%\end{proof}

%From Lemma \ref{galois orbit subgroup} and Lemma \ref{galois orbit subfield}, we can index 
Recall the decomposition of $\Q[G]$ into the simple factors (\ref{simplefactors}),
%of $\Q[G]$ by the subgroups $H$ for which $G/H$ is cyclic. That is,
$$\Q[G] \cong \prod_{\substack{H \leq G,\\ G/H \text{ cyclic } }} K_H.$$
%Since $J(C)$ is a $\Z[G]$ module and $\Q[G]$ decomposes into products of fields $K$ corresponding to $\galq$ orbit of characters, $J(C)$ also decomposes, up to isogeny, to a direct sum of $\OO_{K}$ modules. Thus, we can write 
We denote the induced decomposition of $J(C)$ up to isogeny as
$$J(C) \sim \bigoplus_{\substack{H \leq G,\\ G/H \text{ cyclic } }} J(C)_{H}.$$
%In particular, when $G$ itself is cyclic, $H=\{1\}$ is among the subgroup such that $G/H$ cyclic. The component of $J(C)$ corresponding to $H=\{1\}$ is called the new part of the Jacobian.
\begin{definition}\label{definition of new part}
When $G$ a cyclic group, we call $J(C)^{\n}=J(C)_{\{1\}}$ the new part of the Jacobian.
\end{definition}

By construction, for any subgroup $H$ of $G$, with $G/H$ cyclic, after identifying $T_{G/H}\simeq T^H_G$, we have
$$J(C/H)\simeq \bigoplus_{H\leq H'} J(C)_{H'}.$$
where the signature of the $G/H$-cover $C/H$ is $\fb_{G/H}=\fb_{\vert {T}^H_{G}}$ and the signature of the new part $J(C/H)^{\n}$ of $J(C/H)$ is $\fb_{G/H}^{\n}=\fb_{\vert T^{H,\n}_G}$.

From the formula computing  $\mu$-ordinary polygons, and the decomposition $\mu_p(G,\fb)=\oplus_{\OO\in\OO_G} \mu(\OO)$ arising from (\ref{simplefactors}), we deduce 
$$\mu_p(G,\fb)= \bigoplus_{\substack{H \leq G,\\ G/H \text{ cyclic } }} \mu_p(K_H,\fb_{G/H}^{\n}) \text{ where }
\mu_p(K_H,\fb_{G/H}^{\n})= \bigoplus_{\substack{\OO \in \OO_{G}\\ \OO\subseteq T_{G/H}^{\n}}}\mu(\OO)$$
and 
$$\mu_p(G/H, \fb_{G/H})=\bigoplus_{\substack{\OO \in \OO_{G/H}}} \mu(\OO) \text{ where } \OO_{G/H}\simeq \{\OO\in\OO_G\mid \OO\subseteq T_G^H\}$$ 
In the following, we refer to the Newton polygon $\mu_p(K_H,\fb_{G/H}^{\n})$ as the $(K_H,\fb_{G/H}^{\n})$-ordinary polygon at $p$. By definition, it is the $\mu$-ordinary polygon at $p$ of a PEL type Shimura variety parametrizing abelian varieties with an action of the field $K_H$, and signature $\fb_{G/H}^{\n}$.

We deduce the following statement.
%Next, we explain how to compute the signature $\fb$ for the monodromy datum $(G,r,\au)$. Notice that $\fb$ does not depend on the point we choose for $\mgra$, so we might take any $\xb \in \mgra(\C)$ and consider $\ct$, which is now defined over $\C$. 
%The action of $G$ on $\ct$ makes the first Betti cohomology $H^1(\ct,\Q)$ into a $\Q[G]$ module, so $V(\ct)=H^1(\ct,\Q) \otimes_{\Q} \C$ is a $\Q[G] \otimes_{\Q} \C$ module. Since $\Q[G] \otimes_{\Q} \C \cong \prod_{\rho \in \tg}\C_{\rho}$, $V(\ct) \cong \bigoplus_{\rho \in T_G}V(\ct)_{\rho}$, where $g \in G$ acts on $V(\ct)_{\rho}$ via scalar multiplication by $\rho(g)$. 
%On the other hand, we have the Hodge decomposition $H^1(\ct,\Q) \otimes_{\Q} \C \cong H^1(\ct,\OO) \oplus H^0(\ct,\Omega)$. We denote  $H^0(\ct,\Omega)$ by $V^{+}(\ct)$ and $H^1(\ct,\OO)$ by $V^{-}(\ct)$. Since both are $\Q[G] \otimes_{\Q} \C$ modules, we have decomposition $V^{+}(\ct) \cong \bigoplus_{\rho \in \tg}V^{+}(\ct)_{\rho}$ and $V^{-}(\ct) \cong \bigoplus_{\rho \in \tg}V^{-}(\ct)_{\rho}$. We have the duality that $V^{+}(\ct)_{\rho}^{\smvee}=V^{-}(\ct)_{\check{\rho}}$. The signature $f_G: T_G \to \Z$ is given by $f(\rho)=\dim(V^+(\ct)_{\rho})$. 

%\begin{example}
%For $\xb\in \mgra$, let $\xbt$ be the reduction mod $p$. Then $V^+(C)$ is the complex analog of $Q^{\smvee}$, since $V^+(C)=H^0(C_{\xb},\Omega_C)$ and $Q^{\smvee}=H^0(C_{\xbt},\Omega_C)$. Similarly $V^-(C)$ is the complex analog of $Q$. The duality $V^{+}(C)_{\rho}^{\smvee}=V^{-}(C)_{\check{\rho}}$ is the analog of $(Q^{\smvee})_{\tau}=(Q_{\tau^*})^{\smvee}$.
%\end{example}
%\begin{example}
%In the case that $G$ is cyclic of order $l$, the characters are indexed by $i \in \Z/l\Z$, and the notation here corresponds the notations in section \ref{duality} as follows:  $V^{-}(C)_{\tau(i)^{\smvee}}=H^1(C,\OO_C)_{i^*}=Q_{i^*}$ with basis $\zeta_{i^*,j}$, $V^{+}(C)_{\tau(i)}=H^0(C,\Omega_C)_{i^*}=Q_{i^*}^{\smvee}$ with basis $\omega_{i^*,j}$, both with dimension $f(i)$, and the duality is given by residue pairing, as stated in Section \ref{duality}. On the other hand, $V(C)_{\tau(i)}=M_i=Q_i \oplus Q_{i^*}^{\smvee}$ has dimension $f(i^*)+f(i)=g(\OO)$ and is spanned by $\zeta_{i,j}$ and $\omega_{i^*,k}$. 
%\end{example}

%For $H \leq G$, $\tgh = \Hom(G/H,\qb)$. We have the inflation map $i: \tgh \to \tg$ that maps $\tau$ to $\rhot$, such that $\rhot(g)=\tau(gH)$. Then $i$ induces an bijection of $\tgh$ with $\tg^H$, as well as $\tghn$ with $\tgdhn$. Thus we have $\tg \cong \coprod_{\substack{H \leq G,\\ G/H \text{cyclic}}}\tgdhn$.
%We will now show that the signature map is compatible with inflation of characters.

%\begin{lemma}\label{inflation character}
%Let $H \leq G$ be a subgroup. Let $f_G:\tg \to \Z$ be the signature of $C$ and $\fgh: \tgh \to \Z$ be the signature of $C/H$. Then the following diagram commutes.
%\begin{equation*}\label{figure inflation of character}
%\begin{tikzcd}%[row sep=huge]
 %   \tgh \arrow[r,hook,"i"] \arrow{dr}[left]{\fgh} & T_{G} \arrow[d,"f_{G}"]\\ & \Z
%\end{tikzcd}
%\end{equation*}
%\end{lemma}
%\begin{proof}
%For $H \subseteq N \subseteq G$, we have $C/H \to \Po$ a cover with Galois group $G/H$, and $C/N$ a intermediate cover with Galois group $G/N$.
%We have the relation that $\text{V}^+(C/H)=\text{V}^+(C)^{H}$, where $\text{V}^+(C)^{H}$ denotes the subspace of $\text{V}^+(C)$ fixed by $H$. 
%Consequently, for $\tau \in \tgh$, $\text{V}^+(\ct/H)_{\tau}=(\text{V}^+(\ct)^{H})_{\tau}$. Let $\rhot=i(\tau)$ in $T_{G}$, then $ H\subseteq \ker(\rhot)$, so $(\text{V}^+(C)^{H})_{\tau}=\text{V}^+(C/H)_{\rhot}$. Therefore,
%\begin{equation*}
 %   \begin{split}
 %     \fgh(\tau)&=\dim(\text{V}^+(C/H)_\tau)    =\dim((\text{V}^+(C)^{H})_\tau)     =\dim(\text{V}^+(C)_{\rhot})    =f_G(\rhot).\qedhere
 %   \end{split}
%\end{equation*}
%\end{proof}


%Now, since $p \nmid |G|$, $p$ is unramified in any subfield of $\Q[G]$. Consequently, the image of any character of $G$ is a root of unity with order prime to $p$. We can fix an isomorphism $\iota: \C \to \C_p$ where $\C_p=\Hat{\xoverline{\Q_p}}$, which is the completion of an algebraic closure of $\Q_p$. Then $\tg$ can be identified as $\hom(G,\C_p)$, and for any $\rho \in \tg$, the image of $
%\rho$ is in $\Q_p^{nr}$, the maximal unramified extension of $\Q_p$ in $\xoverline{\Q_p}$.
%The Frobenius orbits of $\tg$ are determined by the action of the Frobenius automorphism $\sigma_p$ of $\gal(\Q_p^{un}/\Q_p)$ on $\tg$. Therefore, the partition of $\tg$ into Frobenius orbits is a refinement of the partition of $\tg$ into Galois orbits. That is, each Galois orbit is a finite union of Frobenius orbits. The inflation map $i$ induces the inclusion of Frobenius orbit $i: \ogh \to \og$. If $\OO_\tau$ is a Frobenius orbit of $\tgh$, then $i(\OO_\tau)=\OO_{\rhot}$ is a Frobenius orbit in $\tg$. 

%Since the signature map is compatible with inflation of character, we have the compatibility of $\mu$-ordinary component over each Frobenius orbit: 
%\begin{lemma}\label{mu ordinary polygon coincide}
%Let $\OO_{\tau} \in \ogh$ be a Frobenius orbit in $\tgh$ and $i(\OO_\tau)=\OO_{\rhot} \in \og$ be the image of inflation. Let $\mu(\OO_\tau)$ and $\mu(\OO_{\rhot})$ be the $\mu$-ordinary Newton Polygon at corresponding Frobenius orbits. Then $\mu(\OO_\tau)=\mu(\OO_{\rhot} )$.
%\end{lemma}
%\begin{proof}
%By the formula of $\mu$-ordinary Newton Polygon, the slopes and multiplicities are determined by $(f(\rho))_{\rho \in \orho}$. But $\fgh|_{\orho}=f_G|_{\ort}\circ i$ by Lemma \ref{inflation character}.
%\end{proof}
%If $H \leq G$ such that $G/H$ is cyclic, Then for the $G$ cover $C \to \Po$, $C/H \to \Po$ is an intermediate cyclic cover with Galois group $G/H$. We have the identification of $\text{W}(C/H)$ as the fixed subspace of $\text{W}(C)$ by $H$. That is, $\text{W}(C/H)=\text{W}(C)^{H}$.
%\begin{lemma}\label{mu ordinariness inflate}
%Let $\rho \in \tgh$ and $\rhot$ be its inflation to $\tg$. Then $\text{W}(C/H)_{\orho}$ is $\mu$-ordinary if and only if $\text{W}(C)_{\ort}$ is $\mu$-ordinary.
%\end{lemma}
%\begin{proof}
%By Lemma \ref{mu ordinary polygon coincide}, we have $\mu(\orho)=\mu(\ort)$. On the other hand, $\text{W}(C/H)_{\orho}=\text{W}(C)_{\orho}^{H}=W(C)_{\ort}$. Moreover, the action of Frobenius commutes with the quotienting of curves. Therefore, the characteristic polynomial of Frobenius on $\text{W}(C/H)_{\orho}$ is the same as that of $\text{W}(C)_{\ort}$. Thus, one Newton polygon is the same as $\mu(\orho)$ if and only if the other is. 
%\end{proof}

\begin{lemma}\label{equivalence two}
Let $G$ be an abelian group, and $p$ a prime $p\nmid |G|$. For $C \to \Po$ a $G$-cover of $\Po$ defined over $\fpb$, 
%parameterized by $\xb \in \mgra(\fpb)$. Then 
the following are equivalent:
\begin{enumerate}
    \item $J(C)$ is $(G,\fb)$-ordinary;
    \item $J(C/H)$ is $(G/H,\fb_{G/H})$-ordinary, for all $H \leq G$ with $G/H$ cyclic; 
    \item $J(C/H)^{\n}$ is $(K_H, \fb_{G/H}^{\n})$-ordinary, for all $H \leq G$ with $G/H$ cyclic.
\end{enumerate}
\end{lemma}
%\begin{proof}
%$1. \implies  2.$ Suppose $J(C)$ is $\mu$-ordinary. Then for each Frobenius orbit $\OO \in \og$, $\text{NP}(\text{W}(C)_{\OO})$ coincides with $\mu(\orho)$, so $\text{W}(C)_{\OO}$ is $\mu$-ordinary. Now, fix a subgroup $H \leq G$ such that $G/H$ is cyclic. For $\tau \in \tgdhn$, let $\rhot$ be its inflation to $\tg$. Since $\text{W}(C)_{\OO_{\rhot}}$ is $\mu$-ordinary, by Lemma \ref{mu ordinariness inflate}, we know that $\text{W}(C/H)_{\OO_{\tau}}$ is $\mu$-ordinary. Since this is true for all $\tau \in \tgdhn$, we conclude that $J(C/H)^\n$ is $\mu$-ordinary.

%$2. \implies 1.$ Suppose we have $J(C/H)^\n$ be $\mu$-ordinary for all $H$ for which $G/H$ is cyclic. Then for each $\OO_{\rhot} \in \og$, we pick $H=\ker(\rhot)$. Since $\rhot \in \tghn$, there exists $\tau \in\tgdhn$ such that $i(\tau)=\rhot$. Since $G/H$ is cyclic, $J(C/H)^\n$ is $\mu$-ordinary, so we know that $\text{W}(C/H)_{\orho}$ is $\mu$-ordinary. Therefore, $\text{W}(C)_{\ort}$ is $\mu$-ordinary. Since this is true for all $\rhot \in \tg$, $J(C)$ is $\mu$-ordinary.    
%\end{proof}


%The following Theorem will be proven in Section \ref{new part mu ordinary}.
%\begin{theorem}\label{new part mu ord}
 %   Suppose we are given a monodromy datum $(l,r,\au)$ and a prime $p$ such that $1 \leq a_k \leq l-1$, $\sum_{k=1}^r a_k \equiv 0 \pmod{l}$, $\gcd(a_1, \dots, a_r,l)=1$, $ r \leq 5$ and $p \nmid l, p > l(r-2)$.
  %  Then there exists a non-empty open subset $U \subseteq \mlra$ such that for all $\xb \in U$, $J(\ct)^\n$ is $\mu$-ordinary.
%\end{theorem}

%\begin{proof}[Reduction of Theorem \ref{abelian cover} assuming to Theorem \ref{new part mu ord}]
Since $\mu$-ordinariness is an open condition, the above statement suffices to reduce the proof of Theorem \ref{abelian cover} to the case of $G$ a cyclic group.

\begin{example}\label{Moonen Oort}
We cite an example from \cite{Moonen Oort} to illustrate how to compute signature of the quotient curve and how does the reduction argument works.
Consider the monodromy datum $(G,r,a)=(\Z/2\Z \times \Z/6\Z, 4, (1,0),(1,1),(0,2),(0,3))$. Then there are 12 characters $\kij$ partitioned into $8$ Galois orbit, as follows:
\begin{equation*}
(\chi_{0,0}),(\chi_{0,1},\chi_{0,5}),
(\chi_{0,2},\chi_{0,4}),(\chi_{0,3}),
(\chi_{1,0}),(\chi_{1,1},\chi_{1,5}),
(\chi_{1,2},\chi_{1,4}),(\chi_{1,3}).
\end{equation*}
Let $\zeta_6$, be a primitive sixth root of unity, then $\chi_{i,j}(a,b)=\zeta_6^{3ai+bj}$. 
Denote $C_{\kij}=C/\ker(\kij)$. The $8$ Galois orbits lead to $8$ quotient curves, we illustrate $3$ of them.
\begin{enumerate}
    \item For Galois orbit $(\chi_{1,2},\chi_{1,4})$. Let $\rho=\chi_{1,2}$, $H=\ker(\rho)=\{(0,0),(0,3)\}$. Now $C_{\rho}$ has monodromy datum $(6,3,(3,5,4))$. This is because $|\rho(G)|=6$ and $\zeta_6^{\ub_i}=\rho(\au_i)$ gives $\ub_i=(3,5,4,0)$. So $C_\rho$ is the normalization of $y^6=(x-x_1)^3(x-x_2)^5(x-x_3)^4$.
    The characters whose kernel contains $\ker(\rho)$ are $\{\chi_{1,2},\chi_{0,4},\chi_{1,0},\chi_{0,2},\chi_{1,4},\chi_{0,0}\}$, and they appear as inflation of the characters of $G/H$. We can compute their signature via
    $$f_G(\chi_{i,2i})=f_{G/H}(i^{-1}(\chi_{i,2i}))=f_{G/H}(\rho^i)=\langle\frac{-3i}{6}\rangle+\langle\frac{-5i}{6}\rangle+\langle\frac{-4i}{6}\rangle-1.$$
    Then the signature of $\{\chi_{1,2},\chi_{0,4},\chi_{1,0},\chi_{0,2},\chi_{1,4},\chi_{0,0}\}$ is $\{0,0,0,0,1,0\}$.
    \item For Galois orbit $(\chi_{0,1},\chi_{0,5})$. Let $\rho=\chi_{0,1}$, so $H=\ker(\rho)=\{(0,0),(1,0)\}$. Then $C_\rho$ has monodromy datum $(6,3,(1,2,3))$. So $C_\rho$ is the normalization of $y^6=(x-x_2)(x-x_3)^2(x-x_4)^3$.
    \item For Galois orbit $(\chi_{1,1},\chi_{1,5})$. Let $\rho=\chi_{1,1}$, $H=\ker(\rho)=\{(0,0),(1,3)\}$. Then $C_\rho$ has monodromy datum $(6,4,(3,4,2,3))$. So $C_\rho$ is the normalization of $y^6=(x-x_1)^3(x-x_2)^4(x-x_3)^2(x-x_4)^3$.
\end{enumerate}
Therefore, assuming $J(C/H)^{\n}$ is generically $\mu$-ordinary for any $H$ with $G/H$ cyclic, Hence, by Lemma \ref{equivalence two} we deduce that $J(C)$ is generically $\mu$-ordinary.
\end{example}


%Then we see from Theorem \ref{new part mu ord} that each $J(C/H)^{\n}$ is generically $\mu$-ordinary. In other words, for each of the eight subgroup $H \subseteq G$ such that $G/H$ is cyclic, there is a dense open subset $U_H \subseteq \mathcal{M}(m_H,N_H,\ub_H)$ such that for $\xb=(x_1,x_2,x_3,x_4) \in U_H$, $J(\ct/H)^{\n}$ is $\mu$-ordinary. Then for $\xb \in \cap_{H \subseteq G, G/H \text{ cyclic }}U_H$, $J(\ct/H)$ is $\mu$-ordinary for all such $H$, so $J(\ct)$ is $\mu$-ordinary by Lemma \ref{equivalence two}.


%\section{Jacobian of abelian cover is $\mu$-ordinary}\label{final part}
%The argument above can be generalised to show that Theorem \ref{new part mu ord} implies Theorem \ref{abelian cover}.
%Let $G$ be a finite abelian group. Consider monodromy datum $(G,r,\underline{a})$ and a prime $p$ satisfying the following condition:
%\begin{enumerate}
 %   \item $r\leq 5$ 
 %   \item $\sum_{k=1}^r \au(k)=0$ in $G$.
  %  \item $(\au(1), \dots, \au(r))$ generate $G$.
  %  \item $p \not\;\mid |G|$, and $p>l(r-1)$, where $l$ is the exponent of $G$.
%\end{enumerate}
%We want to show that there exists a curve $\xb \in \mgra$ such that $J(\ct)$ is  $\mu$-ordinary. By Lemma \ref{equivalence two}, $J(\ct)$ is $\mu$-ordinary iff for every $H \subseteq G$ such that $G/H$ is cyclic, $J(C/H)^\n$ is $\mu$-ordinary.

%\begin{proof}[Proof of Theorem \ref{abelian cover} assuming Theorem \ref{new part mu ord}]
  %  Fix the monodromy $\gra$ and consider $H \leq G$ such that $G/H$ is cyclic. Pick $\tau \in \tghn$. Then $\ct/H \cong C_{\tau,\underline{x}}$, with monodromy datum $(l_H,r_H,\ub_H)$, where $l_H=|\tau(G)|=|G/H|$, $\ub_H=(b_1,\dots,b_r)$ such that $\tau(\au_i)=\zeta_l^{b_i}$, and $r_H$ is the number of non-zero $b_i$. The datum $(l_H,r_H,\ub_H)$ satisfies the condition for Theorem \ref{new part mu ord}, so there exists a non-empty, open $U_H \subseteq \mathcal{M}(l_H,r_H,\ub_H)$ such that for every $\xb \in U_H$, $J(C_{\tau,\xb})^{\n}$ is $\mu$-ordinary.

 %   Now, for each $H \leq G$, there is a surjective map $\pi_H: \mgra \to \mathcal{M}(l_H,r_H,\ub_H)$. Therefore, $\pi_H^{-1}(U_H) \subseteq \mgra$ is non-empty and open. Let $U=\bigcap_{\substack{H \leq G \\ G/H \text{cyclic}}} \pi_H^{-1}(U_H)$. Since $\mgra$ is irreducible, $U$ is non-empty. Then given $\xb \in U$, $\pi_H(\xb) \in U_H$ for every $H \leq G$ for which $G/H$ is cyclic. Therefore, $J(\ct/H)^{\n}$ is $\mu$-ordinary for each $H$ by Theorem \ref{new part mu ord}. Therefore, $J(\ct)$ is $\mu$-ordinary by Lemma \ref{equivalence two}.
%\end{proof}






\section{The Hasse--Witt triple of cyclic covers of $\Po$}\label{duality}

The goal of this section is to explicitely compute the Hasse--Witt triple of a cyclic cover of $\Po$, adapting to our context the methods in \cite{Moonen algorithm}.  

\begin{notation}\label{new part mu ord}
Let $G$ be a cyclic group of size $l$. By fixing an identification $G=\Z/l \Z$,
we may write a monodromy datum for $G$ as $(l,r,\au)$ where $\au=(a_1,\dots a_r) \in \Z^r$ satisfy
\begin{enumerate}
\item $1 \leq a_k \leq l-1$, for all $1\leq k\leq r$;
\item $\sum_{k=1}^r a_k \equiv 0 \pmod{l}$; 
\item $\gcd(a_1, \dots, a_r,l)=1$. 
\end{enumerate}
Let $\zeta_l=e^{2\pi i/l}\in\C$ is a primitive root of $l$. 
We also identify $\tg=\Z/l\Z=\{b\in \Z\mid 0\leq b\leq l-1\}$ by $\tau_b(a)=\zeta_{l}^{ba}.$ Under this identification,  $b\in \tg^{\n}$ if and only if $\gcd(b,l)=1$.


\end{notation}


We assume $p > l(r-2)$ and $r\leq 5$. As the statement of the main theorem trivially holds for $r\leq 3$, we are really considering the cases of $4\leq r\leq 5$. The assumption $p>l(r-2)$ is the same as in \cite[Theorem]{Irene}

\subsubsection{Hasse-Witt triple of cover of $\Po$}\label{Hasse Witt triple} 

In \cite{Moonen algorithm}, Moonen given an explicit algorithm for computing the Hasse--Witt triple of (the Jacobian of) a complete intersection curve $C$ defined over $\fpb$. 
Adapting  \cite[proof of Proposition 3.11 and formula (3.11.3)]{Moonen algorithm} to the special case of $C$ a cover of $\Po$, $\pi: C \to \Po$, using the coordinate charts $U_1=\pi^{-1}(\Po-\{\infty\})$ and $U_2=\pi^{-1}(\Po-\{0\})$,  we explicitly compute the Hasse-Witt triple $(Q,\phi,\psi)$ of $C$ as follows:
\begin{itemize}
    \item $Q=\hy$, and $Q^{\smvee}=\hyd$;
    \item $\phi:\hy \to \hy$ given by the Hasse-Witt matrix;
    \item $\psi: \ker(\phi) \to \im(\phi)^\perp$ given by $\psi(\alpha)=(df_{1,{\alpha}},-df_{2,{\alpha}}),$
    where $(df_{1,{\alpha}},-df_{2,{\alpha}})$ denotes the global $1$-form on $C$ which restricts to $df_{1,{\alpha}}$ on $U_1$ and to $-df_{2,{\alpha}}$ on $U_2$, and $f_{1,{\alpha}} \in \OO_C(U_1)$ and $f_{2,\alpha} \in \OO_C(U_2)$ satisfy $\alpha^p=f_{1,{\alpha}}+f_{2,{\alpha}}$ 
    (for $\alpha \in \ker(\phi)$, $f_{1,\alpha}$ and $f_{2,\alpha}$ exist, and by construction $df_{1,{\alpha}}$ and $-df_{2,{\alpha}}$ agree on $U_1\cap U_2$).
\end{itemize}
%Then $(M,F,V,b)$ obtained from the Hasse-Witt triple via the correspondence is isomorphic to the polarized mod $p$ Dieudonne module associated to $J(C)$.

%We have reduced the proof of the Theorem \ref{abelian cover} to proving Theorem \ref{new part mu ord}. Moreover, Moonen in \cite[Proposition 5.1]{Second Paper} proved the special case of Theorem \ref{new part mu ord} when $r\leq 3$. Suppose we have a monodromy datum $(\mu_l,r,\au)$, where $G=\mu_l$ is the group of the $l^{th}$ roots of unity, $\au_1,\dots,\au_r$ generate $\mu_l$ and $\au_i\neq 1$. Then by choosing a primitive $l^{th}$ root of unity $\zeta_l$, we can identify $\mu_l$ with $\Z/l\Z$. Under this identification we get $1 \leq a_1,\dots,a_r\leq l-1$ such that $\au_i=\zeta_l^{a_i}$. The condition that $\au_i$ generate $\mu_l$ becomes $\gcd(a_1,\dots,a_r,l)=1$. We can also index the elements of $\tg$ by $b\in \Z/l\Z$ as follows: define $\tau_b\in\tg$ as $\tau_b(a)=\zeta_l^{ba}$. Note that $\tau_b^*=\tau_{-b}$ and ${\tau_b}^{\sigp}=\tau_{pb}$. Therefore the monodromy data takes the form $(l,r,a_1,\dots,a_r)$.
%From this point on, we fix a monodromy datum for cyclic cover $(l,r,a_1,\dots,a_r)$ and prime $p$ that satisfies the condition in Theorem \ref{new part mu ord} and has $4\leq r\leq 5$. Note that $\ker(\tau_b)=\{1\}$ if and only if $\gcd(b,l)=1$. Therefore we restrict our attention to Frobenius orbits of $\tau_b$ with $\gcd(b,l)=1$.
%By \cite[Theorem 1.3.7]{Moonen EO type formula}, we want to show that for the $\tau \in \tg^{\text{new}}$, we have that $\wt$ are of maximal length. 

Let $(l, r,\au)$ be a cyclic monodromy datum, and consider the curve $C$ over $k=\fpb$ which is the normalization of the affine equation $$C':y^l=(x-x_1)^{a(1)} \cdots (x-x_r)^{a(r)},$$ where we assume $x_1, \dots , x_r\in k$ distinct.
%$\xb \in \mlra$. 
In the following, we sometimes write $f(x,y)=0$ forthe affine equation of $C'$. 
Our goal is to  compute the coefficients of $\phi$ and $\psi$ with regard to an explicit choice of bases for $\hy$ and $\hyd$.   Our starting point is the computation of the Hasse--Witt matrix of $C$ in \cite{Irene}. If we denote the Hasse--Witt triple of $C$ by $(Q,\phi,\psi)$, then its Hasse--Witt matrix is the matrix of $\phi$ with respect to a choice of basis of $Q$.

Consider the affine charts $U=\pi^{-1}(\Po-\{\infty\})$ and $V=\pi^{-1}(\Po-\{0\})$, we compute $\hy$ via C{e}ch cohomology. For $1 \leq i \leq l-1$, let 
$$v_i=\frac{y^i}{(x-x_1)^{\fy} \dots (x-x_r)^{\fr}}.$$
Note that $v_i$ only depends on the congruence class of $i$ mod $l$. The rings of regular functions over $U$, $V$ and $U\cap V$ are given by
\begin{align*}
\OO_C(U) &\cong \bigoplus_{i=1}^{l-1}k[x]v_i,
&
\OO_C(V) &\cong \bigoplus_{i=1}^{l-1}k\Big[\frac{1}{x}\Big]x^{-f(i^*)}v_i,
&
\OO_C(U \cap V)&\cong \bigoplus_{i=1}^{l-1}k\Big[x,\frac{1}{x}\Big]v_i.
\end{align*}
We deduce
\begin{equation*}
    \begin{split}
        \hy &= \frac{\OO_C(U \cap V)}{\ocu \oplus \ocv} 
        = \bigoplus_{i=1}^{l-1} \bigoplus_{j=1}^{f({\tau_i}^*)}k \big\{\zij\big\} \text{ with }\zij=\frac{v_i}{x^j},
    \end{split}
\end{equation*}
that is  $\{ \zij=\frac{v_i}{x^j}|1\leq i\leq l-1, 1\leq j\leq f(\tau_i) \}$ is a basis of $\hy$ as a $k$-vector space.


Then, by \cite[Section 5]{Dual basis}, $\{\omega_{i,j}=\frac{x^{j-1}}{v_i}dx : 1 \leq i \leq l-1, 1 \leq j \leq f({\tau_i}^*)\}$ is a $k$-basis for $\hyd$, satisfying
\begin{equation*}
    \langle\zij,\omega_{i',j'}\rangle = 
    \begin{cases}
    l & \text{if $i=i',j=j'$}  \\
    0 & \text{otherwise},
    \end{cases}
\end{equation*}
where  $\langle\, ,\, \rangle$ denotes  the duality between $\hy$ and $\hyd$, given by,
$$\langle f,\omega\rangle=\sum_{P \in C} \Res_P f\omega, \text{ for }  f \in \hy, \omega \in \hyd.$$


By definition, in the Hasse--Witt triple of $C$, we have $Q=\hy$ and $Q^{\smvee}=\hyd$. For each $1\leq i\leq l-1$, the $\tau_i$-isotypic component of $Q$ and $Q^{\smvee}$ are   $$Q_{\tau_i}=\bigoplus_{j=1}^{f({\tau_i}^*)}k\big\{\zij \big\} \text{ and } Q^{\smvee}_{\tau_i}=(Q^{\smvee})_{\tau_i}=  (Q_{\tau_i^*})^{\smvee}= \bigoplus_{j=1}^{f(\tau)}k\{\omega_{i,j}\}.$$
%Notice that $Q_{\tau_i}^{\smvee}=(Q_{\tau_i})^{\smvee}=$, that is, it is the isotypic subspace of $\tau_{i}^*$ in $Q^{\smvee}$.

For $1\leq i\leq l-1$, recall that $\phi_{\tau_i}$ denotes the restriction of $\phi$ to $Q_{\tau_i}$, that is, $\phi_{\tau_i}:Q_{\tau_i}\to Q_{\tau_{pi}}$. %, $\psi_{\tau_i}: Q_{\tau_i}\to Q_{\tau_{pi}}^{\smvee} $.
%We compute the coefficients of the matrices of $\phi_{\tau_i},\psi_{\tau_i}$ with respect to $\zij$ and $\oij$.

\begin{proposition}\cite[Lemma 5.1]{Irene}\label{coefficient of phii}
The coefficient of $\zpij$ in $\phi_{\tau_i}(\zij)$ is given by
$$(-1)^N \sum_{n_1+ \dots + n_r=N}\binom{\pby}{n_1} \dots \binom{\pbr}{n_r} x_1^{n_1} \dots x_r^{n_r},$$
where $N=\sum_{k=1}^r \pbk -(jp-j')=p(f({\tau_i}^*)+1)-(f(p{\tau_i}^*)+1)-jp+j'$.
\end{proposition}

We proceed to compute $\psi_{\tau_i}$, the restriction of $\psi$ to $Q_{\tau_i}$, that is  $\psi_{\tau_i}: \ker(\phi_{\tau_i})\subseteq Q_{\tau_i}\to {\rm Im}(\phi_{\tau_i})^\perp\subseteq Q_{\tau_{pi}}^{\smvee} $.
In \cite{Moonen algorithm}, to reconstruct the mod $p$ Diendonn\'e module of $C$, the map $\psii$ is extended to
%Now, we want to compute the coefficient of $\psi_{\tau_i}$. Notice $\psi_{\tau_i}:\ker(\phi_{\tau_i}) \to Q_{p\tau_i^*}^{\smvee}$ is only defined on $\ker(\phi_{\tau_i})$. It is extended to 
$Q_{\tau_i}$ by choosing a complement $R_0$ to $R_1=\ker(\phi_{\tau_i})$ in $Q_{\tau_i}$ and setting $\psi_{\tau_i}$ to vanish on $R_0$ (different choices of $R_0$ yield different extensions, but give rise to isomorphic Dieudonn\'e modules). 
Here, in order to carry out our computations, instead of working with a basis of $\ker(\phi_{\tau_i})$ and an explicit complement, we introduce and compute the coefficients of another linear map $\psi_{\tau_i}': Q_{\tau_i} \to Q_{p\tau_i^*}^{\smvee}$. We show that new linear map $\psi_{\tau_i}'$ agrees with $\psi_{\tau_i}$ on $\ker(\phi_{\tau_i})$, and that $\psi_{\tau_i}'=\psi_{\tau_i}$ if certain auxiliary conditions hold (see Proposition \ref{When is psii' correct}). 
Note that that if ${\psi_{\ti}}_{\vert \ker(\phi_{\tau_i})}=\psi_{\tau_i}$ then the equality $\psi_{\tau_i}'=\psi_{\tau_i}$ is equivalent to $\qi=\ker(\psi_{\tau_i}')\oplus \ker(\phi_{\tau_i})$.

We define $\psi_{\tau_i}': Q_{\tau_i} \to Q_{p\tau_i^*}^{\smvee}$ as follows. 
For $\alpha \in Q_{\tau_i}$, write
$\alpha^p=f_{1}+\phi(\alpha)+f_{2}$,
where $f_{1} \in \ocu$ and $f_{2} \in \ocv$, then %Then we define $\psi_{\tau_i}'$ such that
\begin{equation*}
\begin{split}
    \psi_{\tau_i}'(\alpha) & = \sum_{k=1}^{f(p\tau_i)}\langle df_{1}, \zpisk\rangle\opisk 
    =\sum_{k=1}^{f(p\tau_i)}( \text{coefficient of $\opisk$ in $df_{1}$})\opisk.
\end{split}
\end{equation*}
By definition (see \ref{Hasse Witt triple}), if $\alpha \in \ker(\phi_{\tau_i})$, then $\psi_{\tau_i}'(\alpha)=\psi_{\tau_i}(\alpha)$.
\begin{proposition}\label{coefficient of psii}
Let $1\leq i\leq l-1$. Assume $\gcd(i,l)=1$. Then, the coefficient of $\opisjp$ in $\psiip(\zij)$ is 
$$-\sum_{k=1}^r \pbk r_{i,j,k}q_{r-j',k},$$
where $q_{r-j',k}$ is the coefficient of $x^{j'-1}$ in $\frac{(x-x_1) \dots (x-x_r)}{(x-x_k)}$,
%(that is,  the symmetric polynomial of degree $r-j'$, missing $x_k$, multiplied by $(-1)^{r-j'}$), 
and $r_{i,j,k}$ is the residue of $\frac{\zij^p}{v_{pi}(x-x_k)}$ at $x=0$, that is 
    $$r_{i,j,k}=(-1)^N\sum_{n_1+ \dots +n_r=N}\binom{\pby}{n_1} \dots \binom{\pbk -1}{n_k} \dots \binom{\pbr}{n_r} x_1^{n_1}\dots x_r^{n_r},$$
where $N=p(f(\tau_i^*)+1)-(f(p\tau_i^*)+1)-pj$.
%and $q_{r-j',k}$ %is the coefficient of $x^{j'-1}$ in $\frac{(x-x_1) \dots (x-x_r)}{(x-x_k)}$. Equivalently, it 

\end{proposition}
\begin{proof}
Let $\zij^p=f_{1,j}+\phi(\zij)+f_{2,j}$, where $f_{1,j} \in \ocu$ and $f_{2,j} \in \ocv$. We have
$$\zij^p=\vip \frac{(x-x_1)^{\pby} \dots (x-x_r)^{\pbr}}{x^{pj}}.$$
Let
$$h(x)=\left[\frac{\pbf}{x^{pj}}\right]_{\text{deg $\geq 0$}}.$$
Then $f_{1,j}=h(x)\vip$, and we want to find the coefficient of $\opisjp$ in $d(h(x)\vip)$.
Since we are in characteristic $p$, we see that
\begin{equation*}
    \begin{split}
        d(\vip)&=y^{pi}d\left(\frac{1}{(x-x_1)^{\pfy}\dots (x-x_r)^{\pfr}}\right) \\
        &=\frac{y^{pi}}{(x-x_1)^{\pfy}\dots (x-x_r)^{\pfr}}\bigg(\frac{-\pfy}{(x-x_1)}+ \dots +\frac{-\pfr}{(x-x_r)}\bigg)dx \\
        &=v_{pi}\bigg(\frac{-\pfy}{(x-x_1)}+ \dots +\frac{-\pfr}{(x-x_r)}\bigg)dx \\
        &= v_{pi}\bigg(\frac{-\pby}{(x-x_1)}+ \dots +\frac{-\pbr}{(x-x_r)}\bigg)dx.
    \end{split}
\end{equation*}
The last equality follows from 
$$\pfk-\pbk = \floor*{p\fk}=p\fk \equiv 0 \pmod{p}.$$
We claim the following, which we will prove later:
\begin{equation}\label{eqn: *}
d(h(x))=\sum_{k=1}^r \pbk \frac{h(x)-r_{i,j,k}}{x-x_k} dx. \tag{*}  
\end{equation}
Assuming this for now, we see that
\begin{equation*}
    \begin{split}
    d(f_{1,j}) &= d(\vip)h(x)+\vip d(h(x)) \\
    &=- \vip \sum_{k=1}^r \pbk \frac{h(x)}{x-x_k} dx + \vip \sum_{k=1}^r \pbk \frac{h(x)-r_{i,j,k}}{x-x_k} dx\\
    &=-\vip \sum_{k=1}^r \pbk \frac{r_{i,j,k}}{x-x_k} dx.
    \end{split}
\end{equation*}
Since $\gcd(i,l)=1$, we have $\vip\vips=(x-x_1) \dots (x-x_r)$ and hence $\vip dx = \frac{(x-x_1) \dots (x-x_r)}{\vips} dx$. Thus,
$$d(f_{1,j})=- \sum_{k=1}^r \pbk r_{i,j,k}\frac{(x-x_1) \dots (x-x_r)}{(x-x_k)} \frac{dx}{\vips}.$$
Therefore, the coefficient of $\opisjp$ is the coefficient of $x^{j'-1}$ in $-\sum_{k=1}^r \pbk r_{i,j,k}\frac{(x-x_1) \dots (x-x_r)}{(x-x_k)}$, which completes the proof of the proposition assuming (*).

Next, we prove (*).
We let 
$$h_1(x)=\frac{\zij^p}{\vip}=\frac{\pbf}{x^{pj}}.$$
Then, $h(x)=\left[ h_1(x)\right]_{\text{deg $ \geq 0$}}$, and so $d(h(x))=[d(h_1(x))]_{\text{deg $ \geq 0$}}$.
From the expression of $h_1(x)$, we know that $d(h_1(x))=\sum_{k=1}^r \pbk \frac{h_1(x)}{x-x_k} dx$. Thus, 
$$d(h(x))=[d(h_1(x))]_{\text{deg $\geq 0$}}=\sum_{k=1}^r \pbk \left[\frac{h_1(x)}{x-x_k}\right]_{\text{deg $ \geq 0$}} dx.$$
Let $h_{1,k}(x)=\left[\frac{h_1(x)}{x-x_k}\right]_{\text{deg $ \geq 0$}}$.
Then, we have
$$\frac{h_1(x)}{x-x_k}=h_{1,k}(x)+\frac{r_{i,j,k}}{x}+g_{1,k}(x).$$
Where $g_{1,k}(x)$ consists of terms of degree $\leq -2$. Notice that $r_{i,j,k}$ is the recidue at $0$ of $\frac{h_1(x)}{x-x_k}=\frac{\zij^p}{\vip (x-x_k)}$. So $h_1(x)=(x-x_k)h_{1,k}(x)+r_{i,j,k}-\frac{x_kr_{i,j,k}}{x}+(x-x_k)g_{1,k}(x)$ and hence $h(x)=(x-x_k)h_{1,k}(x)+ r_{i,j,k}$. Consequently,
$h_{1,k}(x)=\frac{h(x)-r_{i,j,k}}{x-x_k}$.
Then, 
\begin{equation*}
    \begin{split}
        d(h(x))
        =\sum_{k=1}^r \pbk h_{1,k}(x) dx 
        = \sum_{k=1}^r \pbk \frac{h(x)-r_{i,j,k}}{x-x_k} dx.
    \end{split}
\end{equation*}
This completes the proof of (*). For the alternate expression of $r_{i,j,k}$, note that $r_{i,j,k}$ is the coefficient of $x^{-1}$ in $\frac{\zij^p}{\vip (x-x_k)}$. This means that $r_{i,j,k}$ is the coefficient of $x^{pj-1}$ in $\frac{\pbf}{(x-x_k)}$.
\end{proof}






\begin{proposition}\label{When is psii' correct}
Let $1\leq i\leq l-1$.  Assume either  $f(p\tau_i^*)=0$ or
$f(\tau_i^*)=g(\tau_i)$.

Then, the map $\psi_{\tau_i}'=\psi_{\tau_i}$, that is  
$Q_{\tau_i}=\ker( \psi_{\tau_i})\oplus \ker(\phi_{\tau_i}').$\end{proposition}
\begin{proof} 
%Assume $\psi_{\tau_i}'$ is injective on $\ker(\phi_{\tau_i})$ (which we shall prove Lemma \ref{psii' injective}). %Then, in order to prove the statement,  it suffices to show that the rank of  $\psi_{\tau_i}'$ is at most $\dim_k \ker(\phi_{\tau_i})$.

Assume $f(p\tau_i^*)=0$. Then $Q_{p\tau_i}=\{0\}$ and since $\phi_{\tau_i}:Q_{\tau_i}\to Q_{p\tau_i}$, we deduce that $\ker(\phi_{\tau_i})=Q_{\tau_i}$. %In particular, $\psi_{\tau_i}'$ is defined on all of $Q_{\tau_i}$.

Assume $f(\tau_i^*)=g(\tau_i)$. Recall $g(\tau_i)=g(\tau_{pi})$, since $\tau_i,\tau_{pi}$ are in the same orbit under Frobenius.   We show that $\dim {\rm Im}(\psi_{\tau_i}') \leq \dim \ker(\phi_{\tau_i})$. On one hand, $$\dim(\ker(\phi_{\tau_i}))=\dim(Q_{\tau_i})-\dim({\rm Im}(\phi_{\tau_i}))\geq f(\tau_i^*)-\dim(Q_{p\tau_i})= g(\tau_i)-  f(p\tau_i^*)=f(p\tau_i).$$
On the other hand, the inclusion $\psi_{\tau_i}'(Q_{\tau_i}) \subseteq Q_{p\tau_i^*}^{\smvee}$ 
implies $\dim {\rm Im} (\psi_{\tau_i}')\leq \dim(Q_{p\tau_i^*}^{\smvee})= f(p\tau_i)$. Hence, $\dim(\psi_{\tau_i}'(Q_{\tau_i}))\leq f(p\tau_i)\leq \dim(\ker(\phi_{\tau_i}))$.

%Now implies that 
In both instances, in order to conclude, it suffices to verify that 
$\ker(\phi_{\tau_i})\cap\ker(\psi'_{\tau_i})=\{0\},$ which we prove  in Lemma \ref{psii' injective}.

%, then $\ker(\phi_{\tau_i})+\ker(\psi'_{\tau_i})=\ker(\phi_{\tau_i})\oplus\ker(\psi'_{\tau_i})$. Moreover,
%$$\dim(\ker(\phi_{\tau_i})\oplus\ker(\psi'_{\tau_i})) =\dim(\ker(\phi_{\tau_i})) +\dim(Q_{\tau_i})- \dim(\psi'_{\tau_i}(Q_{\tau_i}))\geq \dim(Q_{\tau_i}).$$
%This means that $\ker(\phi_{\tau_i})\oplus\ker(\psi'_{\tau_i})=Q_{\tau_i}$ and hence $\psi_{\tau_i}'$ is a valid extension of $\psi_{\tau_i}$.
\end{proof}



\begin{lemma}\label{psii' injective}
For each $1\leq i\leq l-1$, $\ker(\phi_{\tau_i})\cap\ker(\psi'_{\tau_i})=\{0\}.$
%the $\psi_{\tau_i}'$ is injective when restricted to $\ker(\phi_{\tau_i})$.
\end{lemma}


\begin{proof}%[Proof of Lemma \ref{psii' injective}]

Suppose $\alpha \in \ker(\phi_{\tau_i}) \cap \ker(\psi_{\tau_i}')$. Say
$\alpha=c_1\frac{v_i}{x}+c_2\frac{v_i}{x^2}+ \dots +c_{f(\tau_i^*)}\frac{v_i}{x^{f(\tau_i^*)}}$.
Then,
\begin{equation*}
    \begin{split}
      \alpha^p&=c_1^p\frac{v_i^p}{x^p}+c_2^p\frac{v_i^p}{x^{2p}}+ \dots +c_{f(\tau_i^*)}^p\frac{v_i^p}{x^{pf(\tau_i^*)}} \\
      &=v_{pi}(x-x_1)^{\pby} \dots (x-x_r)^{\pbr}\left(\frac{c_1x^{f(\tau_i^*)-1}+c_2x^{f(\tau_i^*)-2}+ \dots +c_{f(\tau_i^*)}}{x^{f(\tau_i^*)}} \right)^p.
    \end{split}
\end{equation*}
Let 
\begin{align*}
    h(x)&=c_1x^{f(\tau_i^*)-1}+c_2x^{f(\tau_i^*)-2}+ \dots +c_{f(\tau_i^*)}, \\
    g(x)&=\left[\frac{(x-x_1)^{\pby}\dots (x-x_r)^{\pbr}h(x)^p}{x^{pf(\tau_i^*)}}\right]_{\deg \geq 0}, \\
    q(\frac{1}{x})&=\left[\frac{(x-x_1)^{\pby}\dots (x-x_r)^{\pbr}h(x)^p}{x^{pf(\tau_i^*)}}\right]_{\deg \leq -f(p\tau_i^*)-1}.
\end{align*}
Since $\alpha \in \ker(\phi_{\tau_i})$,  $\alpha^p=g(x)v_{pi}+q(\frac{1}{x})v_{pi}$. Thus, $0=d(g(x)v_{pi})+d(q(\frac{1}{x})v_{pi})$.
Since, $\psi(\alpha)=0$, we have $d(g(x)v_{pi})=0$ and $d(q(\frac{1}{x})v_{pi})=0$. We want to conclude that $h(x)=0$, which would imply $\alpha=0$. Assume for the sake of contradiction that $h(x) \neq 0$. 

First, we show that $g(x)\neq 0$ and $q(\frac{1}{x})\neq 0$. Let $b$ be the smallest integer for which coefficient of $x^{b}$ is non-zero in $h(x)$ and $c$ be the largest such integer. Therefore $0\leq b\leq c\leq f(\tau_i^*)-1$. Now, in $h(x)^p$, the highest power of $x$ is $x^{cp}$ and lowest power of $x$ is $x^{bp}$. Therefore, in $\frac{(x-x_1)^{\pby}\dots (x-x_r)^{\pbr}h(x)^p}{x^{pf(\tau_i^*)}}$, the highest power of $x$ is $x^{p(f(\tau_i^*)+1)-f(p\tau_i^*)-1+pc-pf(\tau_i^*) }=x^{p(c+1)-1-f(p\tau_i^*)}$, and the power is strictly positive. Therefore, $g(x)$ is non-zero. On the other hand, the lowest power of $x$ is $x^{bp-pf(\tau_i^*)}$. Since $b \leq f(\tau_i^*) -1$, the power is negative. Therefore, $q(\frac{1}{x})$ is non-zero.

Now, say $pi \equiv k \pmod{l}$. Then, we have $v_{pi}=v_k=\frac{y^k}{(x-x_1)^{\fky} \dots (x-x_r)^{\fkr}}$.
Let 
$$g_1(x)=\frac{g(x)}{(x-x_1)^{\fky}\dots (x-x_r)^{\fkr}}=\frac{g(x)v_k}{y^k}.$$
Then, $g(x)v_{k}=y^kg_1(x)$, so $d(y^kg_1(x))=0$.

On the other hand, if we write the affine equation of the curve as $y^l=f(x)$, then we have $ly^{l-1}dy=f'(x)dx$. Consequently,
\begin{equation*}
    \begin{split}
        d(y^kg_1(x)) &=ky^{k-1}g_1(x)dy+y^kg_1'(x)dx =\frac{ky^{k-1}g_1(x)f'(x)}{ly^{l-1}}dx +y^{k}g_1'(x) dx \\
        &=\frac{dx}{y^{l-k}}\left(\frac{k}{l}f'(x)g_1(x)+y^lg_1'(x)\right) =\frac{dx}{y^{l-k}}\left(\frac{k}{l}f'(x)g_1(x)+f(x)g_1'(x)\right).
    \end{split}
\end{equation*}
Since $d(y^kg_1(x))=0$, this implies that  $\frac{k}{l}f'(x)g_1(x)+f(x)g_1'(x)=0$. Multiplying by $lf(x)^{k-1}g_1(x)^{l-1}$, we get that
$kf(x)^{k-1}f'(x)g_1(x)^l+lf(x)^{k}g_1(x)^{l-1}g_1'(x)=0$. Therefore
$\left(f(x)^kg_1(x)^l\right)'=0$.
So $f(x)^kg_1(x)^l=s(x)^p$ for some $s(x)$ rational function. Since, $g_1(x)=\frac{g(x)}{(x-x_1)^{\fky}\dots (x-x_r)^{\fkr}}$, we see that
\begin{equation}\label{eq1}
   (x-x_1)^{l\langle\frac{ka_1}{l}\rangle} \dots (x-x_r)^{l\langle\frac{ka_r}{l}\rangle}g(x)^l=\prod(x-\beta_s)^{\pm p}.
\end{equation}
Where $\beta_s$ are the zeros and poles of $s(x)$. 
Similarly, for $q(\frac{1}{x})$, we get that
\begin{equation}\label{eq2}
(x-x_1)^{l\langle\frac{ka_1}{l}\rangle} \dots (x-x_r)^{l\langle\frac{ka_r}{l}\rangle}q(\frac{1}{x})^l=\prod(x-\gamma_s)^{\pm p}.
\end{equation}
Let $a(x)=x^{pf(\tau_i^*)}q(\frac{1}{x})$, so $a(x)$ is a polynomial with $\deg(a(x)) \leq pf(\tau_i^*)-f(p\tau_i^*)-1$.

Let $t_j$ to be the power of $x-x_j$ in $g(x)$, and $m_j$ be the power of $(x-x_j)$ in $a(x)$. From the above two equations we have
\begin{align*}
l\langle\frac{ka_j}{l}\rangle+t_jl &\equiv 0 \pmod{p},
&
l\langle\frac{ka_j}{l}\rangle+m_jl &\equiv 0 \pmod{p}.
\end{align*}
Thus, $t_j \equiv m_j \pmod{p}$.
From Equations \ref{eq1} and \ref{eq2}, we know that for any $\alpha \not \in \{x_1, \dots, x_r\}$, $v_{(x-\alpha)}(g(x)) \equiv 0 \pmod{p}$ and $v_{(x-\alpha)}(a(x)) \equiv 0 \pmod{p}$. Let $n_j= \min(m_j, t_j)$, we have that $n_j \equiv m_j \equiv t_j \pmod{p}$.

Now, recall that 
$$(x-x_1)^{\pby} \dots (x-x_r)^{\pbr} h(x)^p=x^{pf(\tau_i^*)}g(x)+a(x).$$
We divide both sides of the equation by $(x-x_1)^{n_1} \dots (x-x_r)^{n_r}$.
The right hand-side of the equation becomes a $p^{th}$ power. Therefore, the left hand-side of the equation must also become a $p^{th}$ power. Therefore, $n_j \equiv \pbj \pmod{p}$. Moreover, $n_j \geq 0$ as both $a(x)$ and $g(x)$ are polynomials. So $m_j,t_j \geq \pbj$. Therefore, 
$$p(f(\tau_i^*)+1)-f(p\tau_i^*)-1=\sum_{j=1}^r \pbj \leq \sum_{j=1}^r m_j \leq \deg(a(x)) \leq pf(\tau_i^*)-f(p\tau_i^*)-1.$$
Which is a contradiction. Thus, $h(x) =0$, which means that $\alpha=0$. This shows that $\psi_{\tau_i}'$ is injective on $\ker(\phi_{\tau_i})$ as desired. 
\end{proof}















\section{A criterium of $p$-ordinariness for Hasse--Witt triples}\label{combi}

Fix a monodromy datum $(G,r,\au)$ and consider the associated Shimura datum $(G,\fb)$ as defined in Section \ref{Shimura}. Assume $p$ is a prime not dividing $|G|$. In this section, we give an explicit criterion for  Hasse--Witte triples of covers of $\Po$ with monodromy $(G,r,\au)$ to be $(G,\fb)$-ordinary. This result is obtained as a special case of \cite[Theorem 1.3.7]{Moonen EO type formula}, which states that a mod $p$ Diedonn\'e module is $p$-ordinary if for each $\tau\in\tg$ the associated word $\wt$ has maximal length.  The new criterium is stated in terms of the maps $\phi_{\tau},\psi_{\tau}$ instead of the words $w_{\tau}$, which makes it better suited to our setting.

%For the rest of this section, we fix a Frobenius orbit $\OO\in \og$, and consider $\tau\in\OO$.

Recall notations from Section \ref{dieu}. Let $(M,F,V,b)$ be the mod $p$ Diedonn\'e module with a $(G,\fb)$-structure (that is, an action of $G$ of signature $\fb$). Denote by ${e_{\tau,1},\dots, e_{\tau,g(\OO)}}$ a basis of $M_{\tau}$; for $1\leq k\leq g(\OO)$, $M_{\tau,k}$ is the span of $e_{\tau,1},\dots, e_{\tau,k}$, and  $f(\tau)=\dim(\ker(F)\cap M_{\tau})$. For suitable
increasing sequences $1\leq j_{\tau,1}<j_{\tau,2}<\dots<j_{\tau,f(\tau)}\leq g(\OO)$ and $1\leq i_{\tau,1}<\dots<i_{\tau,f(\tau^*)}\leq g(\OO)$ satisfying $\{j_{\tau,1},\dots,j_{\tau,f(\tau)},i_{\tau,1},\dots,i_{\tau,f(\tau^*)}\}=\{1,\dots,g(\OO)\}$ (see Section \ref{dieu}), the word $w_{\tau}$ associated to $(M,F,V,b)$ is given by $$w_{\tau}(j_{\tau,k})=k \text{ and }w_{\tau}(i_{\tau,k})=f(\tau)+k.$$ Then the mod $p$ Diedonn\'e module is $(G,\fb)$-ordinary, and the word $w_{\tau}$ maximal for all $\tau$, if $i_{\tau,t}=t$ and $j_{\tau,t}=f(\tau^*)+t$, for all $1\leq t\leq f(\tau^*)$, or equivalently if $\ker(F)\cap M_{\tau,f(\tau^*)}=\{0\}$.
The set $\{e_{\tau,i_{\tau,1}},\dots, e_{\tau,i_{\tau,f(\ts)}}\}$ is a basis of $Q_{\tau}$ and $\{e_{\tau,j_{\tau,1}},\dots, e_{\tau,j_{\tau,f(\tau)}}\}$ is a basis of $Q_{\tau^*}^{\smvee}$. We have $M_{\tau}=Q_{\tau}\oplus Q_{\ts}^{\smvee}$.
The action of $F$ and $V$ on $M_{\tau}$ is determined by the words $(\wt)_{\tau \in \tg}$ as follows:
\begin{align*}
F(e_{\tau,j})&=\begin{cases}
    e_{p\tau,w_{\tau}(j)-f(\tau)} &\text{ if } w_{\tau}(j) \geq f(\tau)+1\\
    0 &\text{ if } w_{\tau}(j) \leq f(\tau)
    \end{cases},
&
V(e_{p\tau,j_1})&=\begin{cases}
    0 &\text{ if } j_1 \leq f(\tau^*)\\
    e_{\tau,w_{\tau}^{-1}(j_1-f(\tau^*))} &\text{ if } j_1 \geq f(\tau^*)+1.
    \end{cases}
\end{align*}
Further recall that $\pi_{\tau}$ was the projection from $M_{\tau}$ to $Q_{\tau}$ with kernel $Q_{\tau^*}^{\smvee}$. Finally, we had defined $V': M\to M$ as $V'(x)=F(\check{x})^{\smvee}$. Given $\tau\in\OO$ and $1\leq j_2\leq j_3\leq g(\OO)$, we denote $M_{\tau,j_2,j_3}=\Span\{e_{\tau,j_2}, \dots, e_{\tau,j_3}\}$. For a Frobenius orbit $\OO$, we define $s(\OO)=\#\{\fts\mid \tau\in \OO\}$. Further, denote $\{\fts\mid \tau\in \OO\}=\{f_{\OO,1}<f_{\OO,2}<\dots< f_{\OO,s(\OO)}\}$. Pick $\tau_{\OO,u}\in\OO$, such that $f(\tau_{\OO,u}^*)=f_{\OO,u}$.



\begin{lemma}\label{Lem: 2 technical lemma w tau}
Given $\tau\in\OO$ and $1\leq k\leq g(\OO)$, there exists $k' \in \mathbb{N}$ such that $1\leq k'\leq g(\OO)$ and
$$\{w_{\tau}(j)-f(\tau)\mid 1\leq j\leq k, w_{\tau}(j)> f(\tau)\} =\{1,2,\dots,k'\}.$$
Further, we have $k'\leq \min(k,f(\tau^*))$.
Moreover, if $w_{\tau}$ is maximal then we have $k'=\min(k,f(\tau^*))$.
\end{lemma}
\begin{proof}
Firstly, note that $\{j\mid w_{\tau}(j)> f(\tau)\}=\{i_{\tau,1},\dots,i_{\tau,f(\tau^*)}\}$.
Let $k'=\max\{a \mid 1\leq a\leq f(\tau^*),  i_{\tau,a}\leq k\}$. Set $k'=0$ if $k<i_{\tau,1}$. We see that $\{j\mid 1\leq j\leq k, w_{\tau}(j)> f(\tau)\}=\{i_{\tau,1},\dots,i_{\tau,k'}\}$.
Therefore,
\[\{w_{\tau}(j)-f(\tau)\mid 1\leq j\leq k, w_{\tau}(j)> f(\tau)\} =\{1,2,\dots,k'\}.\]
It is clear that $k'\leq f(\tau^*)$.
Further, since $a\leq i_{\tau,a}$, we have $k'\leq i_{\tau,k'}\leq k$. This shows that $k'\leq \min(k,f(\tau^*))$.
If $w_{\tau}$ is maximal, then we have $i_{\tau,a}=a$. Therefore,
\begin{itemize}
    \item if $f(\tau^*)\leq k$, then we have $k'=f(\tau^*)$.
    \item On the other hand, if $f(\tau^*)\geq k$, then we have $k'=k$.
\end{itemize}
We see that in both cases we have $k'=\min(k,f(\tau^*))$.
\end{proof}
Consequently, we have the following lemma.
\begin{lemma}\label{F}
Given $\tau\in\OO$ and $1\leq k\leq g(\OO)$, let $l=\dim(F(M_{\tau,k}))$. Then $F(M_{\tau,k})=M_{p\tau,l}$ and $l\leq \min(k,f(\tau^*))$. Moreover, if $w_{\tau}$ is maximal, then $l= \min(k,f(\tau^*))$.
\end{lemma}
\begin{proof}
We know that $F(M_{\tau,k})$ is spanned by $\{F(e_{\tau,1}),\dots,F(e_{\tau,k})\}$. Therefore, it is spanned by
$$\{e_{p\tau,w_{\tau}(j)-f(\tau)}\mid 1\leq j\leq k, w_{\tau}(j)\geq f(\tau)+1\}.$$
This set is also linearly independent and hence forms a basis of $F(M_{\tau,k})$. Consider the $k'$ given by Lemma \ref{Lem: 2 technical lemma w tau}, we see that $F(M_{\tau,k})=M_{p\tau,k'}$. Since $l=\dim(F(M_{\tau,k}))$, we see that $l=k'$. Therefore, Lemma \ref{Lem: 2 technical lemma w tau} tells us that $l\leq \min(k,f(\tau^*))$ and if $w_{\tau}$ is maximal then $l=\min(k,f(\tau^*))$.
\end{proof}


\begin{lemma}\label{FV'}
The following hold:
\begin{enumerate}
    \item $\ker(V')=Q$,
    \item $\rm{Im}(F) \cap \rm{Im}(V')=\{0\}$,
    \item If $W\subseteq M_{\tau}$ is spanned by some of the $e_{\tau,j}$, then $(F+V')(W)=F(W)\oplus V'(W)$.
\end{enumerate}
\end{lemma}
\begin{proof}
1) For $x \in M$, we have $V'(x)=0$ if and only if $F(\check{x})=0$. This is equivalent to $\check{x}\in \check{Q}$, which happens if and only if $x\in Q$. Therefore, $\ker(V')=Q$.

2) Notice that $\im(V')\subseteq \im(F)^{\smvee}$ and $\im(F)\cap \im(F)^{\smvee}=\{0\}$.

3) Notice that
\begin{equation*}
    (F+V')(e_{\tau,j})=\begin{cases}
     F(e_{\tau,j})  &\text{if }  \wt(j)>f(\tau)\\
     V'(e_{\tau,j})  &\text{if } \wt(j)\leq f(\tau).
    \end{cases}
\end{equation*}
This implies that $(F+V')(W)=F(W)+V'(W)$. The result follows 2).
\end{proof}


\iffalse
\begin{lemma}
 Also, given any $x\in \check{Q}$, we have $V(V'(x))=x$. Finally, given a subspace $W\subseteq M_{\tau}$ which is spanned by some vectors of the standard basis $\{e_{\tau,j}\}$, we have $V^{-1}(W)\cap M_{p\tau}=F(M_{\tau})\oplus V'(W)$.
\end{lemma}
\begin{proof}
Now consider $x \in \Check{Q}$, we want to show that $V(V'(x))=x$. Notice that $\check{Q}$ is spanned by $\check{e}_{\tau,j}$ such that $w_{\tau}(j) \geq f(\tau)+1$. Moreover, for such $\check{e}_{\tau,j}$, we have
\begin{equation*}
    \begin{split}
    V(V'(\check{e}_{\tau,j})) &=V(F(e_{\tau,j})^{\smvee})
    =V(\check{e}_{p\tau,w_{\tau}(j)-f(\tau)})
    =\check{e}_{\tau,j}.
    \end{split}
\end{equation*}
This shows that for any $x\in \check{Q}$, we have $V(V'(x))=x$.

Next, consider $W \subseteq M_\tau$. Firstly, since $W$ has a basis consisting of $e_{\tau,j}$ we see that $W=(W\cap Q_{\tau})\oplus(W\cap Q_{\tau^*}^{\smvee})$. Let $W_1=W\cap Q_{\tau}$ and $W_2=W\cap Q_{\tau^*}^{\smvee}$. We see that
$$V(F(M_{\tau})+V'(W))=V(V'(W))=V(V'(W_2))=W_2\subseteq W.$$
This means that $F(M_{\tau})+V'(W)\subseteq V^{-1}(W)\cap M_{p\tau}$. Next consider $y\in V^{-1}(W)\cap M_{p\tau}$. So $V(y)\in W\cap \im(V)=W_2$, and hence $V(V'(V(y)))=V(y)$. This means that $V'(V(y))-y\in \ker(V)=\im(F)$. Therefore, $y= (y-V'(V(y)))+V'(V(y))\in F(M_{\tau})+V'(W)$. We have shown that $V^{-1}(W)\cap M_{p\tau}=F(M_{\tau})+ V'(W)$.
Finally, since $\im(F) \cap \im(F)^{\smvee}= \{0\}$, the sum is a direct sum. 
\end{proof}
\fi





\begin{lemma}\label{V'stream}
If $\wt$ is maximal and $\fts \leq j_1<j_2\leq j_3$, then we have $V'(M_{\tau,j_1})=M_{p\tau,\fts+1,j_1}$, and $V'(M_{\tau,j_2,j_3})=M_{p\tau,j_2,j_3}$. Moreover, for any $1\leq j\leq g(\OO)$ then we have $(F+V')(M_{\tau,j})=M_{p\tau,j}$.
\end{lemma}
\begin{proof}
Since the word $w_{\tau}$ is maximal, we have $i_{\tau,t}=t$ for $1\leq t\leq \fts$ and $j_{\tau,t}=f(\tau^*)+t$ for $1\leq t\leq f(\tau)$.
Now, for $1\leq t\leq f(\ts)$, we have $F(e_{\tau,t})=e_{p\tau,t}$ and $V'(e_{\tau,t})=0$. Whereas for $\fts<t\leq j$, we have $F(e_{\tau,t})=0$ and $V'(e_{\tau,t})=e_{p\tau,t}$.
The result follows.
\end{proof}

%\begin{lemma}\label{Lem: F,V' in F+V'}
%\end{lemma}





Now for $\tau \in \OO$ and $0\leq i\leq |\OO|-1$, we define $H_{\tau,i}: M_{p^i\tau} \to M_{p^{i+1}\tau}$
as follows:
\begin{equation*}
    H_{\tau,i}(x)=\begin{cases}
    F(x) &\text{if } f(p^i\tau^*)\geq f(\tau^*)\\
    F(x)+V'(x) &\text{if } f(p^i\tau^*)< f(\tau^*).
    \end{cases}
\end{equation*}


\begin{proposition}\label{criteriasecond}
Suppose we are given $\tau\in \OO$ such that
\begin{enumerate}
    \item $\dim(\pi_{\tau}\circ H_{\tau,|\OO|-1}\circ\dots H_{\tau,1}\circ H_{\tau,0}(M_{\tau}))=f(\tau^*)$.
    \item For every $\tau'\in\OO$ with $f(\tau'^*)<f(\tau^*)$, we have $w_{\tau'}$ maximal.
\end{enumerate}
Then we have the following:
\begin{enumerate}
    \item For each $0\leq i\leq |\OO|-1$, we have
    $H_{\tau,i}\circ \dots H_{\tau,1}\circ H_{\tau,0}(M_{\tau})=M_{p^{i+1}\tau,f(\tau^*)}$.
    \item $w_{\tau}$ is maximal.
    \item Given any $\tau'\in\OO$ with $f(\tau'^*)=f(\tau^*)$, we have $\dim(\pi_{\tau'} \circ H_{\tau',|\OO|-1} \circ \dots \circ H_{\tau',0}(M_{\tau'}))=f(\tau^*)$.
    \item Given any other $\tau'\in\OO$ with $f(\tau'^*)=f(\tau^*)$, $w_{\tau'}$ is also maximal.
\end{enumerate}
\end{proposition}
\begin{proof}
1) We prove this by induction on $i$. First, consider the base case $i=0$. We have $H_{\tau,0}=F$. Since $\dim(F(M_{\tau}))=\fts$, by Lemma \ref{F}, we have $F(M_{\tau})=M_{p\tau,\fts}$. Next, suppose that for some $i\geq 1$, we have $H_{\tau,i-1}\circ \dots H_{\tau,1}\circ H_{\tau,0}(M_{\tau})=M_{p^{i}\tau,f(\tau^*)}$.
\begin{itemize}
    \item Case 1: $f(p^{i}\ts)\geq \fts$. Then $H_{\tau,i}=F$. Let $l=\dim(F(M_{p^{i}\tau,f(\tau^*)}))$. By Lemma \ref{F}, we have $F(M_{p^{i}\tau,f(\tau^*)})=M_{p^{i+1}\tau,l}$. Clearly we have $l=\dim(F(M_{p^{i}\tau,f(\tau^*)}))\leq \dim(M_{p^{i}\tau,f(\tau^*)})=\fts$. Further,
    \begin{equation*}
        \begin{split}
         \fts&=\dim(\pi_{\tau}\circ H_{\tau,|\OO|-1}\circ\dots H_{\tau,1}\circ H_{\tau,0}(M_{\tau}))\\
         &=\dim(\pi_{\tau}\circ H_{\tau,|\OO|-1}\circ\dots \circ H_{\tau,i+1}(M_{p^{i+1}\tau,l}))
         \leq \dim(M_{p^{i+1}\tau,l})=l.
        \end{split}
    \end{equation*}
    Therefore, $l=\fts$.
    \item Case 2: $f(p^{i}\ts)< \fts$. Then $H_{\tau,i}=F+V'$ and $w_{p^i\tau}$ is maximal. Therefore by Lemma \ref{V'stream}, we have
    $$H_{\tau,i}\circ \dots H_{\tau,1}\circ H_{\tau,0}(M_{\tau})=(F+V')(M_{p^{i}\tau,f(\tau^*)})=M_{p^{i+1}\tau,\fts}.$$
\end{itemize}
This completes the induction step and hence completes the proof of 1).

2) From part 1), we know that $H_{\tau,|\OO|-1}\circ \dots H_{\tau,1}\circ H_{\tau,0}(M_{\tau})=M_{\tau,f(\tau^*)}$. Thus we know that $\dim(\pi_{\tau}(M_{\tau,\fts}))=\fts$. This means that $M_{\tau,\fts}\cap \ker(\pi_{\tau})=\{0\}$, that is, $M_{\tau,\fts}\cap \ker(F)=\{0\}$. Therefore $w_{\tau}$ is of maximal length.

3) Suppose $\tau'\in \OO$ satisfies $f(\tau')=\fts$. Say $\tau'=p^i\tau$, for some $1\leq i\leq |\OO|-1$.
\begin{itemize}
    \item For $0 \leq j \leq |\OO|-i-1$: $H_{\tp,j}=H_{\tau,j+i}$.
    \item For $|\OO|-i \leq j \leq |\OO|-1: H_{\tp,j}=H_{\tau, j+i-|\OO|}$.
\end{itemize}
Since $H_{\tp,0}=F$, we have $$H_{\tp,0}(M_{\tp})=F(M_{\tp})=M_{p\tp,\fts}=H_{\tau,i}\circ \dots H_{\tau,1}\circ H_{\tau,0}(M_{\tau}).$$
This implies that
$$H_{\tau',|\OO|-i-1}\circ \dots H_{\tau',1}\circ H_{\tau',0}(M_{\tau'}) =H_{\tau,|\OO|-1}\circ \dots H_{\tau,i+1}\circ H_{\tau,i}\circ \dots H_{\tau,1}\circ H_{\tau,0}(M_{\tau})=M_{\tau,\fts}.$$
Since $\wt$ is maximal, from Lemma \ref{F}, we know that $F(M_{\tau,\fts})=M_{p\tau,\fts}=F(M_{\tau})$. This means that $H_{\tau,0}(M_{\tau,\fts})=H_{\tau,0}(M_{\tau})$. Now, we see that
\begin{equation*}
    \begin{split}
    H_{\tau',|\OO|-1}\circ \dots H_{\tau',1}\circ H_{\tau',0}(M_{\tau'}) &=H_{\tau',|\OO|-1}\circ \dots\circ H_{\tau',|\OO|-i}(M_{\tau,\fts})\\ &= H_{\tau,i-1}\circ \dots\circ H_{\tau,0}(M_{\tau,\fts})\\
    &= H_{\tau,i-1}\circ \dots\circ H_{\tau,0}(M_{\tau}).\\
    %&=M_{p^i\tau,\fts}=M_{\tau',\fts}.
    \end{split}
\end{equation*}
Next, note that since $\ker(\pi_{\tau'})=Q_{\tau'^*}^{\smvee}=\ker(F_{\tau'})$, all subspaces $W\subseteq M_{\tau'}$ satisfy $\dim(\pi_{\tau'}(W))=\dim(F(W))$. Therefore, we see that
\begin{equation*}
    \begin{split}
    \dim(\pi_{\tau'} \circ H_{\tau',|\OO|-1} \circ \dots \circ H_{\tau',0}(M_{\tau'}))
    &= \dim(F(H_{\tau,i-1}\circ \dots \circ H_{\tau,0}(M_{\tau})))\\
    &=\dim(H_{\tau,i}\circ H_{\tau,i-1}\circ \dots\circ H_{\tau,0}(M_{\tau}))\\
    &=\dim(M_{p^{i+1}\tau,\fts})=\fts.
    \end{split}
\end{equation*}

4) This follows from part 3) and part 2).
\end{proof}



%\begin{lemma}\label{nointersection}
%If $W_1,W_2$ are sub-spaces spanned by the standard basis $\{e_{\tau,j}\}$ and if $W_1 \cap W_2 =\{0\}$, then $F(W_1) \cap F(W_2)= \{0\}$ and $V'(W_1) \cap V'(W_2)=\{0\}$
%\end{lemma}
%\begin{proof}
%Notice that $F$ or $V'$ never map $e_{\tau,j_1}$ and $e_{\tau,j_2}$ to the same $e_{p\tau,j_3}$.
%\end{proof}

Given $\tau\in \OO$ and $0\leq i\leq |\OO|-1$, we define $L_{\tau,i}$ to be a map from  $M_{p^i\tau}$ to $M_{p^{i+1}\tau}$ as follows.
\begin{equation*}
    L_{\tau,i}(x)=\begin{cases}
    F(x) &\text{if } f(p^i\tau^*)\geq f(\tau^*)\\
    V'(x) &\text{if } f(p^i\tau^*)< f(\tau^*).
    \end{cases}
\end{equation*}


\begin{proposition}\label{criteriathird}
Given $2\leq u\leq s(\OO)$, let $\tau=\tau_{\OO,u}$ and $\tau'=\tau_{\OO,u-1}$. Suppose we have the following:
\begin{enumerate}
    \item $\dim(\pi_{\tau}\circ L_{\tau,|\OO|-1}\circ \dots \circ L_{\tau,0}(M_{\tau}))= f_{\OO,u}-f_{\OO,u-1}.$
    \item $\dim(\pi_{\tau'}\circ H_{\tau',|\OO|-1} \circ \dots \circ H_{\tau',0}(M_{\tau}))=f_{\OO,u-1}.$
    \item Whenever any $\tau''\in \OO$ satisfies $f(\tau''^*)<f_{\OO,u}$, then $w_{\tau''}$ is maximal.
\end{enumerate}
Then we have
$\dim(\pi_{\tau}\circ H_{\tau,|\OO|-1}\circ \dots \circ H_{\tau,0}(M_{\tau}))=f(\tau^*)=f_{\OO,u}.$
\end{proposition}
From the Proposition \ref{criteriasecond} and Proposition \ref{criteriathird}, we can immediately derive the following theorem that will ultimately be used by us.

\begin{theorem}\label{numerical criteria}
Suppose we have an orbit $\OO$ and $1\leq u\leq s(\OO)$ such that
\begin{enumerate}
    \item $\dim(\pi_{\tau_{\OO,1}}\circ L_{\tau_{\OO,1},|\OO|-1} \circ \dots \circ L_{\tau_{\OO,1},0}(M_{\tau_{\OO,1}}))=f_{\OO,1}.$
    \item For each $2\leq j\leq u$, we have
    $$\dim(\pi_{\tau_{\OO,j}}\circ L_{\tau_{\OO,j},|\OO|-1} \circ \dots \circ L_{\tau_{\OO,j},0}(M_{\tau_{\OO,j}}))=f_{\OO,j}-f_{\OO,j-1}.$$
\end{enumerate}
Then all $\tau\in \OO$ with $f(\tau^*)\leq f_{\OO,u}$ satisfy $w_{\tau}$ is maximal.    
\end{theorem}
\begin{proof}[Proof of Proposition \ref{criteriathird}]
For the fixed $\tau=\tau_{\OO,u}$, and for $0 \leq i \leq |\OO|-1$, we define maps $A_i: M_{\pit} \to M_{\pipt}$ such that
\begin{equation*}
    A_i=\begin{cases}
      F+V' & \text{ if } \fpits<\foum \\
      F & \text{ if } \fpits \geq \foum.
    \end{cases}
\end{equation*}
Note that the definition of $A_i$ is similar to $H_{\tau,i}$ but the cutting point is $f_{\OO,u-1}$ instead of $f_{\OO,u}$.
From Lemma \ref{FV'}, we see that
$$\pi_\tau\circ A_{|\OO|-1}\circ \dots \circ A_0(M_\tau)+ \pi_{\tau}\circ L_{\tau,|\OO|-1} \circ \dots \circ L_{\tau,0}(M_{\tau}) \subseteq \pi_{\tau}\circ H_{\tau,|\OO|-1}\circ \dots \circ H_{\tau,0}(M_{\tau}).$$
We know that $\dim(\pi_{\tau}\circ L_{\tau,|\OO|-1}\circ \dots \circ L_{\tau,0}(M_{\tau}))= \fou-\foum$. We will be done, if we show the following claims:
\begin{enumerate}
    \item  $\dim(\pi_\tau \circ A_{|\OO|-1}\circ \dots \circ A_0(M_\tau)) \geq \foum$
    \item $\pi_\tau \circ A_{|\OO|-1} \circ \dots \circ A_0(M_\tau)$ intersect trivially with $\pi_{\tau}\circ L_{\tau,|\OO|-1} \circ \dots \circ L_{\tau,0}(M_{\tau})$. 
\end{enumerate}
Proof of claim (1): Suppose $\tp=\pit$. Then, $\fpits=f(\tps)=\foum$. We have the following relation of $A_i$ and $H_{\tp,i}$:
\begin{itemize}
    \item For $0 \leq j \leq i-1$: $A_j=H_{\tp,j+|\OO|-i}$,
    \item For $i \leq j \leq |\OO|-1: A_j=H_{\tp,j-i}$.
\end{itemize}
Then, $A_{i-1}\circ \dots \circ A_0(M_\tau)=H_{\tp,|\OO|-1} \circ \dots \circ H_{\tp, \mo-i}(M_\tau)$. So it contains $H_{\tp,|\OO|-1} \circ \dots \circ H_{\tp, 0}(M_{\tp})$. By Proposition \ref{criteriasecond}, we know that $H_{\tp,|\OO|-1} \circ \dots \circ H_{\tp, 0}(M_{\tp})=M_{\tau',f(\tau'^*)}$. Therefore, $\pi_\tau \circ A_{|\OO|-1}\circ \dots \circ A_0(M_\tau)$ contains $\pi_\tau \circ A_{|\OO|-1}\circ \dots \circ A_i(M_{\tau',f(\tau'^*)})$. Next, note that
$$\pi_\tau \circ A_{|\OO|-1}\circ \dots \circ A_i(M_{\tau',f(\tau'^*)})= \pi_\tau \circ H_{\tau',|\OO|-i-1}\circ \dots \circ H_{\tau',0}(M_{\tau',f(\tau'^*)}).$$
Since $w_{\tau'}$ is maximal, we know by Lemma \ref{F} that $F(M_{\tau',f(\tau'^*)})=M_{p\tau',f(\tp^*)}=F(M_{\tau'})$. Now $H_{\tau',0}=F$, so we have
$$\pi_\tau \circ H_{\tau',|\OO|-i-1}\circ \dots \circ H_{\tau',0}(M_{\tau',f(\tau'^*)}) =\pi_\tau \circ H_{\tau',|\OO|-i-1}\circ \dots \circ H_{\tau',0}(M_{\tau'}).$$
Finally recall that for any subspace $W\subseteq M_{\tau}$, we have $\dim(\pi_{\tau}(W))=\dim(F(W))$. Therefore, by another application of Proposition~\ref{criteriasecond} we have
\begin{equation*}
    \begin{split}
    \dim(\pi_\tau \circ H_{\tau',|\OO|-i-1}\circ \dots \circ H_{\tau',0}(M_{\tau'}))
    &=\dim(F \circ H_{\tau',|\OO|-i-1}\circ \dots \circ H_{\tau',0}(M_{\tau'}))\\
    &=\dim(H_{\tau',|\OO|-i} \circ H_{\tau',|\OO|-i-1}\circ \dots \circ H_{\tau',0}(M_{\tau'}))\\
    &=\dim(M_{p^{|\OO|-i+1}\tau',f(\tp^*)})=f(\tps)=f_{\OO,u-1}.
    \end{split}
\end{equation*}
This completes the proof of claim (1).\\
Proof of claim (2):
We want to show that $\pi_{\tau}\circ\aof(M_{\tau})$ intersects trivially with $\pi_{\tau}\circ\lof(M_{\tau})$.

Let $i_0=\max\{i' \mid  0 \leq i' \leq \mo-1, f(p^{i'}\ts)=\foum\}$. So $p^{i_0}\tau$ is the last spot where the signature is $\foum$. Denote $\tpp=p^{i_0}\tau$. Then from Proposition \ref{criteriasecond}, we know that 
$\dim(\pipp \circ \hoppf(M_{\tpp}))=\foum$.
We will be done if we show that $\pi_{\tau} \circ A_{|\OO|-1} \circ \dots \circ A_{i_0}(M_{\piot})$ and $\pi_{\tau} \circ L_{\tau,|\OO|-1} \circ \dots \circ L_{\tau,i_0}(M_{\piot})$ intersect trivially. By Proposition \ref{criteriasecond}, we know that
$$A_{|\OO|-1} \circ \dots \circ A_{i_0}(M_{\piot})= H_{\tau'',|\OO|-1-i_0}\circ\dots\circ H_{\tau'',0}(M_{\tau''})=M_{\tau,f_{\OO,u-1}}.$$
Now, we want to show the following claim.\\
Claim $(2')$: For all $j$ such that $i_0 \leq j \leq \mo-1$, there exists $j_2,j_3$ depending on $j$ such that $\foum <j_2\leq j_3$ and $L_{\tau,j} \circ \dots \circ L_{\tau,i_0}(M_{\piot})=M_{p^{j+1}\tau,j_2,j_3}$.\\
Proof of claim $(2')$: We proceed by induction on $j$. The base case is when $j=\iz$. Then, $L_{\tau,\iz}=V'$ and $w_{\piot}$ is maximal. So, by Lemma \ref{V'stream} we have $L_{\tau,\iz}(\mpiot)=M_{\pizp, \foum+1,g(\OO)}$. So take $j_2=\foum+1$, $j_3=g(\OO)$.\\
Now, assume the claim for $j-1$ and let $j_2',j_3'$ be the corresponding indices. By the maximality of $\iz$, $f(\pjts) \neq \foum$. We want to find the $j_2,j_3$ for $j$. 
\begin{itemize}
    \item Case 1: If $f(\pjts) < \foum$.
    Then we know that $w_{\pjt}$ is maximal and $L_{\tau,j}=V'$. Then by Lemma \ref{V'stream} we have
    $$\ljio(\mpiot)=V'(M_{p^j\tau,j_2,j_3})=M_{p^{j+1}\tau,j_2,j_3}.$$
    So we just take $j_2=j_2'$ and $j_3=j_3'$.
    \item Case 2: If $f(\pjts) \geq \fou$.
    Then $L_{\tau,j}=F$. From the inductive hypothesis we have  $L_{\tau,j-1} \circ \dots \circ L_{\tau,\iz}(M_{\piot})=M_{p^j\tau,j_2',j_3'}$. So we have, $\ljio(\mpiot) =F(M_{p^j\tau,j_2',j_3'})$.
    Now, let $j_2=1+\dim(F(M_{\pjt,j_2'-1}))$ and $j_3=\dim(F(M_{\pjt,j_3'}))$. Then $F(M_{p^j\tau,j_2',j_3'})=M_{p^{j+1}\tau,j_2.j_3}$. We still need to show that $\foum<j_2\leq j_3$. By Proposition \ref{criteriasecond}
    \begin{equation*}
        \begin{split}
        F(M_{\pjt,\foum})
        &=F(H_{\tau'',j-1-i_0}\circ\dots\circ H_{\tau'',i_0}(M_{\tau''}))\\
        &= H_{\tau'',j-i_0}\circ H_{\tau'',j-1-i_0}\circ\dots\circ H_{\tau'',i_0}(M_{\tau''})
        = M_{p^{j+1}\tau,\foum}.
        \end{split}
    \end{equation*}
    Now, since $\foum \leq j_2'-1$, we have $$j_2-1=\dim(F(M_{\pjt,j_2'-1})) \geq \dim(F(M_{\pjt,\foum}))=\foum.$$ So $\foum<j_2\leq j_3$ as desired.
\end{itemize}
This finishes the proof for claim $(2')$. Now, take $j=|\OO|-1$, we know that there exists $\foum<j_2\leq j_3$ such that $L_{\tau,\mo-1} \circ \dots \circ L_{\tau,i_0}(M_{\piot})=M_{\tau,j_2,j_3}$.
We have seen that $A_{\mo-1} \circ \dots \circ A_{i_0}(M_{\piot})=M_{\tau,\foum}$ and $L_{\tau,\mo-1} \circ \dots \circ L_{\tau,i_0}(M_{\piot})=M_{\tau,j_2,j_3}$. Therefore, when we apply $\pi_{\tau}$, they intersect trivially. This completes the proof of claim $(2)$ and hence of the proposition.
\end{proof}




\section{Maximal monomials in entries of matrix $\psi_{\tau_i}'$}\label{max monomial}

In section \ref{duality}, we computed the entries of $\psiip$ as polynomials in terms of $x_1, \dots, x_r$. In this section, we compute the maximal monomials of these entries. 

Recall Notation \ref{new part mu ord}: $(l,r,a_1,\dots,a_r)$ is a monodromy datum, with  $4\leq r\leq 5$, 
$p>l(r-2)$ is a prime ,  $\tau_i\in\tg$ with $\gcd(i,l)=1$.
We reorder the entries in $\au$ such that $a_1+a_2 \not \equiv 0 \pmod{l}$, and hence also $a_3+\dots+a_r \not \equiv 0 \pmod{l}$. (We can always find $a_{i_1}, a_{i_2}$ such that $a_{i_1}+a_{i_2} \not \equiv 0 \pmod{l}$. Otherwise, pick any three $a_i,a_j,a_k$. Since each pair sums up to zero mod $ l$, we have that $a_i \equiv -a_i \pmod{l}$. So $\frac{l}{2}\mid a_i$ for each $1 \leq i \leq r$. This contradicts the fact that $\gcd(a_1, \dots, a_r, l)=1$.)
 \begin{definition}\label{define c C N}
 For every $1 \leq i \leq l-1$ and $N \in \mathbb{N}$, we define the following quantities:
 \begin{align*}
     c(i,N)&=\min\{c : \pby + \dots +\pbc >N\}, \\
 C(i,N)&=N-\pby- \dots -\pbcm, \text{where $c=c(i,N)$},\\
  X(i,N)&=x_1^{\pby} \dots x_{c-1}^{\lfloor{p\langle\frac{ia_{c-1}}{l}\rangle\rfloor}}x_c^{C}, \text{where }c=c(i,N), C=C(i,N),\\
 s_i&=\sum_{k=1}^r \pbk. \\
 \end{align*}
 Note that $0 \leq C(i,N)< \pbc$.
 \end{definition}

\begin{lemma}\label{cineqr}
For $j \geq 1$, we have $\ci \neq r$.
\end{lemma}
\begin{proof}
Assume for the sake of contradiction that $\ci=r$. Then $\sum_{k=1}^{r-1}\pbk \leq s_i-pj \leq s_i-p$. So $p \leq \pbr$, which is impossible. 
\end{proof}

\begin{lemma}
Given $1\leq i\leq l-1$ with $\gcd(i,l)=1$, let $\OO$ be the Frobenius orbit that $\tau_i$ is in. Then $g(\OO)=r-2$. Consequently, when $r \leq 5$, for any $j\geq 1$ and $1\leq j'\leq f(p\tau_i)$ we have $\ci \leq r-j'+2$.
\end{lemma}
\begin{proof}
We have
\[g(\OO)=f(\tau_i^*)+f(\tau_i)=-2+\sum_{k=1}^{r}\langle\frac{ia_k}{l}\rangle +\langle\frac{-ia_k}{l}\rangle =r-2.\]
By Lemma \ref{cineqr}, we know that $\ci \leq r-1 \leq 4$. On the other hand, $j' \leq g(\OO)=r-2$, so $r-j'+2 \geq r-(r-2)+2=4$. Therefore, $\ci \leq r-j'+2$.
\end{proof}

\begin{lemma}\label{sum of pbk not zero}
Given $3\leq t\leq r$, we have $\sum_{k=t}^{r}\pbk\not\equiv 0\pmod{p}$. Moreover, for $2 \leq t \leq 3$, we also have $1+\sum_{k=t}^{r}\pbk\not\equiv 0\pmod{p}$. 
\end{lemma}
\begin{proof}
We use the fact that $\langle a+b\rangle=\langle a\rangle+\langle b\rangle$ or $\langle a\rangle+\langle b\rangle-1$. Therefore, we have
   \begin{equation*}
       \begin{split}
           \sum_{k=t}^r \pbk &\equiv -\sum_{k=t}^r \langle\frac{pia_k}{l}\rangle \pmod{p}
           \equiv -\big\langle\frac{pi(a_{t}+ \dots + a_r)}{l}\big\rangle-s \pmod{p},
       \end{split}
   \end{equation*}
    for some $0 \leq s \leq (r-t+1)-1 = r-t$. Let $a$ be the residue of $pi(a_{t}+ \dots + a_r) \pmod{l}$. First we show that for $3\leq t\leq r$, we have $-\frac{a}{l}-s \not \equiv 0 \pmod{p}$.
   \begin{itemize}
       \item If $a=0$. Then since $\sum_{k=t}^r \langle\frac{pia_k}{l}\rangle= \frac{a}{l}+s$ is non-zero, $s$ is non-zero. Then $1 \leq s \leq r-t\leq r-3$. Since $p>(r-2)l>s$, $\frac{a}{l}+s=s$ can not be congruent to $0$ mod $p$.
       \item If $a \neq 0$.  If $\frac{a}{l}+s \equiv 0 \pmod{p}$, then $a+ls \equiv 0 \pmod{p}$. On the other hand, $1\leq a+ls \leq (s+1)l \leq (r-2)l$. Since $p>(r-2)l$, this can not happen. Thus, this is non-zero mod $p$.
   \end{itemize}
We next show that for $2\leq t\leq 3$, we have $-\big\langle\frac{pi(a_{t}+ \dots + a_r)}{l}\big\rangle-(s-1) \not \equiv 0 \pmod{p}$.
\begin{itemize}
    \item If $a=0$. Then we get that $\sum_{k=t}^r a_k \equiv 0 \pmod{l}$, so $\sum_{k=1}^{t-1}a_k \equiv 0 \pmod{l}$. But $t-1 \leq 2$, and we have $a_1 \not \equiv 0 \pmod{l}$, and $a_1+a_2 \not \equiv 0 \pmod{l}$. This is a contradiction, therefore $a\neq0$.
    \item If $a \neq 0$. Then $1 \leq a \leq l-1$. Suppose $-\frac{a}{l}-(s-1) \equiv 0 \pmod{p}$, then $p\mid(a+(s-1)l)$. But notice that $(s-1)l+a<sl\leq (r-2)l$. Since $p>(r-2)l$, this cannot happen. So, this is nonzero mod $p$. \qedhere
\end{itemize}
\end{proof}


We consider the lexicographical order on the monomials such that $x_1>x_2> \dots > x_r$. Given a polynomial $h(x_1,\dots,x_r)$, we denote its maximal monomial as $m(h(x_1,\dots,x_r))$ and the maximal monomial with its coefficient as $M(h(x_1,\dots,x_r))$.
The $(j',j)^{th}$ entry of $\phii$ is denoted as $\phii(j',j)$ and the $(j',j)^{th}$ entry of $\psiip$ is denoted as $\psiip(j',j)$. Bouw in \cite[Lemma 6.5]{Irene} showed that $m(\phii(j',j))=X(i,s_i-pj+j')$.
Now we want to compute the maximal monomial of $\psiip(j',j)$.
We denote $m_i(j',j)=m(\psii'(j',j))$.
Recall that
$$\psii'(j',j)=-\sum_{k=1}^r \pbk r_{i,j,k}q_{r-j',k},$$
where
$$r_{i,j,k}=(-1)^{s_i-pj}\sum_{n_1+ \dots +n_r=s_i-pj}\binom{\pby}{n_1} \dots \binom{\pbk -1}{n_k} \dots \binom{\pbr}{n_r}x_1^{n_1} \dots x_r^{n_r}.$$
\begin{proposition}\label{maximal monomial}
Assume $\gcd(i,l)=1$ and denote $c=\ci$. Then the maximal monomial in $\psii'(j',j)$ is
\begin{equation*}
    m_i(j',j) = 
       \begin{cases}
      x_1x_2 \dots x_{r-j'}\XI & \text{if $\ci < r-j'+1$}\\
     x_1x_2 \dots x_{r-j'-1}x_c\XI & \text{if $\ci \geq r-j'+1$}.\\
    \end{cases}
\end{equation*}
\end{proposition}
\begin{remark}
This expression is true for general $r$, but we are only going to prove it for $\ci \leq r-j'+2$. This will cover all the cases when $r \leq 5$.
\end{remark}
\begin{proof}
We fix $j',j$ such that $1\leq j\leq \fis$ and $1\leq j'\leq f(p\tau_i)$.
First we look at the maximal monomial in $q_{r-j',k}$.
\begin{itemize}
    \item If $ k \geq r-j'+1$, then $M(q_{r-j',k})=(-1)^{r-j'}x_1 \dots x_{r-j'}$.
    \item If $k \leq r-j'$, then $M(q_{r-j',k})=(-1)^{r-j'}x_1 \dots x_{r-j'} \frac{x_{r-j'+1}}{x_k}$. 
\end{itemize} 
Next, look at the maximal monomial in $\fkj$ 
\begin{itemize}
      \item If $k \geq \ci+1$, then $M(\fkj)=(-1)^{s_i-pj}\binom{\pbc}{C_i(s_i-pj)}\XI$.
      \item If $k=c_i(s_i-pj)$, then $M(\fkj)=(-1)^{s_i-pj}\binom{\pbc -1}{C_i(s_i-pj)}X_i(s_i-pj)$.
      \item If $k \leq c_i(s_i-pj)-1$, then $M(\fkj)=(-1)^{s_i-pj}\binom{\pbc}{C_i(s_i-pj)+1}X_i(s_i-pj)\frac{x_c}{x_k}$.
\end{itemize}
Then, depending on the relation of $r-j'$ and $c_i(s_i-pj)$, we have different expressions for the maximal monomial $m_i(j',j)$.

First consider the case when $\ci \leq r-j'$.
Then, when $k \geq r-j'+1$, both $q_{r-j',k}$ and $\fkj$ get maximized. Then 
   $m_i(j',j)=x_1x_2 \dots x_{r-j'}\XI$
   with coefficient
   $$(-1)^{r-j'+s_i-pj}\binom{\pbc}{\Ci} \sum_{k=r-j'+1}^r \pbk.$$
   Now we want to show that this coefficient is not $0$ (in characteristic $p$). The binomial coefficient is non-zero. Since $j' \leq r-2$, $r-j'+1 \geq 3$, so by Lemma \ref{sum of pbk not zero}, $\sum_{k=r-j'+1}^r \pbk \not \equiv 0 \pmod{p}$.
   
   Next, we deal with the cases when $\ci=r-j'+1$ and $\ci=r-j'+2$. In each case, we show that the given monomial has non-zero coefficient, and we show that any larger monomial has zero coefficient. 
\begin{lemma}\label{coefficient 1}
         If $r-j'+e=c_i(s_i-pj)$ for $e \geq 1$,
        then the coefficient of the monomial\\
        $x_1\dots x_{r-j'-1}x_c\XI$ in $\psi_{\tau_i}'(j',j)$ is $\binom{\pbc}{\Ci}\bigg(\sum_{k=c+1}^r \pbk\bigg)\left(\frac{1+\sum_{k=r-j'}^r \pbk}{1+\Ci}\right)$, which is not zero mod $p$.
\end{lemma}
\begin{proof}
  First, we show that the coefficient is indeed the expression given in the lemma. We are looking at $X=\XI x_1 \dots x_{r-j'-1}x_{c}$. We want to compute the coefficient of $X$ in each term of the  sum $\sum_{k=1}^r \pbk \fkj q_{r-j'.k}$.
  Next, we write down the power of each $x_s$ in $X$
  \begin{equation*}
      v_{x_s}(X) = 
      \begin{cases}
            \pis +1  & \text{ if } s \leq r-j'-1 \\
            \pis & \text{ if }  r-j' \leq s \leq c-1 \\
            \Ci + 1 & \text{ if }  s=c \\
            0 & \text{ if } s \geq c+1.
      \end{cases}
  \end{equation*}
Notice that for $k \leq c-1$, $\fkj q_{r-j',k}$ does not contribute to $X$. By this we mean that no monomial in $\fkj q_{r-j',k}$ is $X$. This is because the power of $x_k$ in any monomial in $\fkj$ is at most $\pbk-1$, and the power of $x_k$ in any monomials of $q_{r-j',k}$ is zero.


Next, for $k=c$, the summands in $q_{r-j',k}$ that contribute must not contain any $x_t$ for $t>c$ and must contain $x_1x_2 \dots x_{r-j'-1}$. So then, there are $e$ monomials in $q_{r-j',k}$ can contribute: $x_1 \dots x_{r-j'-1}x_{a}$ for $r-j' \leq a \leq c-1$. If $\qkj$ contributes $x_1 \dots x_{r-j'-1}x_a$, then we are looking for the coefficient of $\frac{x_c\XI}{x_{a}}$ in $\pbc\fkj$, which is $\binom{\pba}{\pba-1}\binom{\pbc -1}{\Ci +1}\pbc$.

Finally, for $k \geq c+1$, the summands in $q_{r-j',k}$ that contribute must not contain any $x_t$ for $t>c$ and must contain $x_1x_2 \dots x_{r-j'-1}$. So then, there are $e+1$ monomials in $q_{r-j',k}$ can contribute: $x_1 \dots x_{r-j'-1}x_{a}$ for $r-j' \leq a \leq c$.
\begin{itemize}
    \item If $\qkj$ contributes $x_1 \dots x_{r-j'-1}x_a$ for $r-j'\leq a\leq c-1$, then we are looking for the coefficient of $\frac{x_c\XI}{x_{a}}$ in $\pbk\fkj$, which is 
      $\binom{\pba}{\pba-1}
      \binom{\pbc}{\Ci +1}\pbk$.
    \item If $\qkj$ contributes $x_1 \dots x_{r-j'-1}x_c$, then we are looking for the coefficient of $\XI$ in $\pbk\fkj$, which is $\binom{\pbc}{\Ci}\pbk$.
\end{itemize}
Thus, the overall coefficient of $X$ in $\psii'(j',j)$ is as given, 
\begin{equation*}
       \begin{split}
           &\pbc \binom{\pbc -1}{\Ci+1} \sum_{a=r-j'}^{c-1} \binom{\pba}{\pba -1}\\
           &+\binom{\pbc}{\Ci+1} \sum_{k=c+1}^r \sum_{a=r-j'}^{c-1} \binom{\pba}{\pba -1}\pbk +\binom{\pbc}{\Ci} \sum_{k=c+1}^r \pbk.
       \end{split}
   \end{equation*}
  Now, we need to show the above coefficient is non-zero mod $p$. We use the identities
  \begin{align*}
    \binom{m}{n+1}&=\frac{m-n}{n+1}\binom{m}{n},
    &
    \binom{m-1}{n}&=\frac{m-n}{m}\binom{m}{n}.
  \end{align*}
  First, we have 
 \begin{equation*}
     \begin{split}
     &\pbc \binom{\pbc -1}{\Ci+1} \sum_{a=r-j'}^{c-1} \binom{\pba}{\pba -1}\\
    &+\binom{\pbc}{\Ci+1} \sum_{k=c+1}^r \sum_{a=r-j'}^{c-1} \binom{\pba}{\pba -1}\pbk \\
    &=\sum_{a=r-j'}^{c-1} \pba \binom{\pbc}{\Ci+1}\left(\frac{\pbc -1 -\Ci}{\pbc}\pbc + \sum_{k=c+1}^r \pbk\right)\\
    &=- \binom{\pbc}{\Ci+1} \sum_{a=r-j'}^{c-1} \pba.
     \end{split}
 \end{equation*}
So then the overall coefficient is
  \begin{equation*}
      \begin{split}
        & -\binom{\pbc}{\Ci+1} \sum_{a=r-j'}^{c-1} \pba+ \binom{\pbc}{\Ci}\sum_{k=c+1}^r \pbk \\
        &= \binom{\pbc}{\Ci}\left(-\frac{\pbc-\Ci}{\Ci+1} \sum_{a=r-j'}^{c-1}\pba +\sum_{k=c+1}^r \pbk \right) \\
        &= \binom{\pbc}{\Ci}\bigg(\sum_{k=c+1}^r \pbk\bigg) \left(\frac{\sum_{a=r-j'}^{c-1}\pba}{\Ci +1}+1\right) \\
        &=\binom{\pbc}{\Ci}\bigg(\sum_{k=c+1}^r \pbk\bigg)\left(\frac{1+\sum_{a=r-j'}^r \pba}{1+\Ci}\right).
      \end{split}
  \end{equation*}
  Notice that since $c+1=r-j'+e+1 \geq r-g(\OO)+1+e \geq 4$, by Lemma \ref{sum of pbk not zero}, we have that $\sum_{k=c+1}^r \pbk \not \equiv 0 \pmod{p}$. At the same time, $r-j' \geq r-g(\OO) \geq 2$. Also, by Lemma \ref{cineqr}, we have that $r-j'=c-e \leq (r-1)-e \leq 3$. Therefore, $2 \leq r-j' \leq 3$. By Lemma \ref{sum of pbk not zero}, we know that $1+\sum_{a=r-j'}^r \pbk \neq 0 \pmod{p}$. Therefore, the coefficient is not zero modulo $p$.
  \end{proof}
  Next, we need to show that this is the maximal monomial whose coefficient is non-zero. So any larger monomial has coefficient $0$. 
  \begin{lemma}\label{vanishing of larger 1}
  If $r-j'+1=\ci$, then let $X=\frac{x_1 \dots x_{r-j'} \XI x_c }{x_{c-1}}=x_1\dots x_{c-2}x_c\XI$.
   For any monomial $Y$ such that $\XI x_1 \dots x_{r-j'} \geq Y> X$, the coefficient of $Y$ in $\psii'(j',j)$ is $0$.
  \end{lemma}
  \begin{proof}
  Since $\psii'(j',j)$ is a homogeneous polynomial in $x_1,\dots,x_r$ we can assume $\deg(Y)=\deg(X)=s_i-pj+r-j'$. So for $1 \leq s \leq c-2, v_{x_s}(Y)= \pis +1$.
  For $s=c-1$, we have
  $\pbcm \leq v_{x_{c-1}}(Y) \leq \pbcm +1$.
  If $v_{x_{c-1}}(Y)=\pbcm=v_{x_{c-1}}(X)$, then since $Y>X$,  $\Ci+1 \leq v_{x_c}(Y)$. But since the degree of $Y$ is $s_i-pj+c-1$, so $v_{x_c}(Y)=\Ci +1$, so $X=Y$, contradiction.
  Thus, $v_{x_{c-1}}(Y) = \pbcm +1$.
  
  Now, denote $v_{x_c}(Y)=\Ci - m$, for $1\leq a\leq r-c$ denote $v_{x_{c+a}}(Y)=m_a$ and finally denote $t=r-c$.
  Since $\deg(X)=\deg(Y)$, we must have $ m_1+ \dots +m_{r-c}=m$. Now,  
  $Y=\XI x_1 \dots x_{c-1}\frac{x_{c+1}^{m_1}\dots x_{r}^{m_{r-c}}}{x_c^m}$.
  Then, the only term that can make contribution to $Y$ in $\qkj$ is $x_1 \dots x_{c-1}$. So we are looking for the coefficient of $\XI\frac{x_{c+1}^{m_1}\dots x_{r}^{m_{r-c}}}{x_c^m}$ in $\pbk\fkj$. Summing over such coefficient for $k \geq c$, the coefficient is
  \begin{equation*}
  \begin{split}
      &\binom{\pbc -1}{\Ci -m}\binom{\pbcp}{m_1} \dots \binom{\pbr}{m_{r-c}}\pbc \\
      &+\sum_{k=c+1}^r \binom{\pbc}{\Ci-m}\binom{\pbcp}{m_1} \dots \binom{\pbk -1}{m_{k-c}} \dots \binom{\pbr}{m_{r-c}}\pbk, \\
          &=\binom{\pbc}{\Ci-m}\prod_{a=1}^{r-c} \binom{\lfloor p\langle\frac{ia_{c+a}}{l}\rangle\rfloor}{m_a}\\
          &\times\left( \frac{\pbc -\Ci +m}{\pbc}\pbc + \sum_{k=c+1}^r \frac{\pbk-{m_{k-c}}}{\pbk}\pbk\right) \\
          &=\binom{\pbc}{\Ci-m}\binom{\pbcp}{m_1}\dots \binom{\pbr}{m_{r-c}} \left(\sum_{k=c}^r \pbk -\Ci +m- \sum_{k=c+1}^r m_{j-c}\right) \\
          &\equiv  0\pmod{p}.
      \end{split}
  \end{equation*}
   Thus for $r-j'+1=\ci$, any larger monomial has coefficient $0$. We have shown that $m_i(j',j)$ is the desired monomial.
  \end{proof}

 Finally, for $r-j'+2=\ci$, we show that any monomial larger than 
 $\XI x_1 \dots x_{r-j'-1}x_c$ has coefficient zero.
 \begin{lemma}
Suppose $r-j'+2=\ci$, let $X=\XI x_1 \dots x_{c-3}x_c$. For any monomial $Y$ such that $ X<Y \leq \XI x_1 \dots x_{c-2}$, the coefficient of $Y$ in $\psii'(j',j)$ is zero.
 \end{lemma}
 \begin{proof}
 Since $\psii'(j',j)$ is a homogeneous polynomial in $x_1,\dots,x_r$, we can assume $\deg(Y)=\deg(X)=s_i-pj+r-j'$.
 For $1 \leq s \leq c-3$, we have $v_{x_s}(Y)=\pis+1$. Hence $\pbct\leq v_{x_{c-2}}(Y)\leq \pbct+1$.
\begin{itemize}
    \item If $v_{x_{c-2}}(Y)=\pbct +1$. Suppose $v_{x_{c-1}}(Y)=\pbcm -m$ and for $1\leq a\leq r-c+1$ denote $m_a=v_{x_{c-1+a}}(Y)\geq 0$. Also denote $m_1'=m_1-\Ci$. Since $\deg(X)=\deg(Y)$, we have $m_1'+ m_2+ \dots +m_{r-c+1}=m$.
Now  , $Y=\XI x_1 \dots x_{c-2}\frac{x_{c}^{m_1'}\dots x_{r}^{m_{r-c+1}}}{x_{c-1}^m}$.
  
  Then the contribution from $q_{r-j',k}$ must be $x_1 \dots x_{c-2}$. So, the contribution of $r_{i,j,k}$ must be
  $\XI\frac{x_c^{m_1'} \dots x_r^{m_{r-c+1}}}{x_{c-1}^m}$. Therefore, coefficient of $Y$ in $\psii'(j',j)$ is
  \begin{equation*}
  \begin{split}
      &\pbcm \binom{\pbcm-1}{\pbcm-m}\binom{\pbc}{m_1} \dots \binom{\pbr}{m_{r-c+1}}\\
      &+\sum_{k=c}^r\pbk \binom{\pbcm}{\pbcm-m}\binom{\pbc}{m_1} \dots\binom{\pbk-1}{m_{k-c+1}} \binom{\pbr}{m_{r-c+1}}.\\
   \end{split}
   \end{equation*}
   Which simplifies to 
   \begin{equation*}
   \begin{split}
      &\binom{\pbcm}{\pbcm-m}\prod_{a=c}^{r} \binom{\pba}{m_{a-(c-1)}} \times\bigg(\pbcm \frac{m}{\pbcm}+  \sum_{k=c}^r \pbk \frac{\pbk-m_{k-c+1}}{\pbk} \bigg).
    \end{split}
  \end{equation*}
  The summation in the last bracket is $0$, since
  $$m-(m_1+\dots+m_{r-c+1})+\sum_{k=c}^r\pbk
    =-\Ci+\sum_{k=c}^r\pbk\equiv 0\pmod{p}.$$
 
  \item If $v_{x_{c-2}}(Y)=\pbct$. Then $\pbcm\leq v_{x_{c-1}}(Y)$.

  Now if $v_{x_{c-1}}(Y)=\pbcm$. Then $X<Y$ implies $\Ci+1 \leq v_{x_{c}}(Y)$. But since $\deg(Y)=deg(X)$, this implies $Y=X$. This is a contradiction, hence $v_{x_{c-1}}(Y)>\pbcm$. Next, since $Y \leq \XI x_1 \dots x_{c-2}$, we have $v_{x_{c-1}}(Y)\leq \pbcm+1$.
  On the other hand, if $v_{x_{c-1}}(Y)\geq \pbcm+2$, then coefficient of $Y$ is zero. Thus, $v_{x_{c-1}}(Y)=\pbcm+1$.
  
  Next, for $1\leq a\leq r+c-1$ denote $m_a=v_{x_{c-1+a}}(Y)$. Let $m_1'=m_1-\Ci$.
  Note that $m_1+ \dots m_{r-c+1}=\Ci$.
  So, $Y=\XI x_1 \dots x_{c-3}x_{c-1}x_{c}^{m_1'}\dots x_{r}^{m_{r-c+1}}$.
  The contribution from $q_{r-j',k}$ can only be $x_1 \dots x_{c-3}x_{c-1}$. So, the contribution of $r_{i,j,k}$ must be
  $\XI x_{c}^{m_1'}\dots x_{r}^{m_{r-c+1}}$.
  So then the coefficient of $Y$ in $\psi_i'(j',j)$ is
  \begin{equation*}
      \begin{split}
          & \sum_{k=c}^r \binom{\pbc}{m_1} \dots \binom{\pbk-1}{m_{k-c+1}} \dots \binom{\pbr}{m_{r-c+1}}\pbk \\
          &= \binom{\pbc}{m_1} \dots \binom{\pbr}{m_{r-c+1}}\left(\sum_{k=c}^r\frac{\pbk-m_{k-c-1}}{\pbk} \pbk \right) \\
          & =\binom{\pbc}{m_1} \dots \binom{\pbr}{m_{r-c+1}}\left(\sum_{k=c}^r\pbk-(m_1+ \dots +m_t)\right) \\
          &\equiv 0\pmod{p}.
      \end{split}
  \end{equation*}
\end{itemize}
So both cases, the coefficient of monomials larger than $X$ is $0$.
 \end{proof}
Thus, we have shown that the maximal monomial of $\psii'(j',j)$ is the given monimal in all the cases.
\end{proof}





\section{A ordinariness result in special instances}\label{Section 0,1 in signature}



Recall %that we have fixed a monodromy datum $(l,r,a_1,\dots,a_r)$ and a prime $p$ satisfying the conditions of 
Notation \ref{new part mu ord}, with  $4\leq r\leq 5$. For $\xb \in \mlra(\fpb)$, let $\ct$ be the normalization of $y^l=(x-x_1)^{a_1} \dots (x-x_r)^{a_r}$. Given a Frobenius orbit $\OO$ in $\og$ and $\tau \in \OO$, let $\wt(\ct)$ be the Weyl coset representative at character $\tau$ of the Dieudonne module associated to $C_{\tb}$.


%smooth curves in the family as follows. $\mathcal{M}(l,r,\au)$ is the open subset of $\fpb^r$ such that $\xb=(x_1, \dots x_r)$ satisfies $x_i \neq x_j$ for $i \neq j$. In other words, $\mathcal{M}(l,r,\au)$ is $\fpb^r$ minus the weak diagonal. Then,  
%For each $\xb \in  \mathcal{M}(l,r,\au)$, we associated $\ct$ which is the normalization of $y^l=(x-x_1)^{a_1} \dots (x-x_r)^{a_r}$, and for each $\tau \in \OO$, we have $\wt(C_t)$ to be the Weyl coset representatives at character $\tau$ of the Dieudonne module associated to $C_{\tb}$. 
\begin{definition}\label{generically maximal}
We say that $\wt$ is generically maximal if there exists a non-empty open subset $U$ of $\mgra(\fpb)$ such that for every $\tb \in U$, $\wt(\ct)$ is maximal.  
\end{definition}

The goal of this section is the following result.

\begin{theorem}\label{signature 1}
Suppose $\gcd(b,l)=1$, $\OO$ is the Frobenius orbit of $\tau_b$ and $\FF(\OO)$ contains $\{0,1\}$, then for all $\tau\in \OO$ such that $f(\ts)=0$ or $1$, $\wt$ is generically maximal.
\end{theorem}
Note that the above Theorem immediately implies the following corollary.
\begin{corollary}
Suppose $\gcd(b,l)=1$, $\OO$ is the Frobenius orbit of $\tau_b$ and $\FF(\OO)$ contains $\{r-2,r-3\}$, then for any $\tau \in \OO$ such that $\fts =r-2$ or $r-3$, $\wt$ is generically maximal.
\end{corollary}
\begin{proof}
If $\FF(\OO)$ contains $\{r-2,r-3\}$, then $\FF(\overline{\OO})$ contains $\{0,1\}$, and for any $\tau \in \OO$ such that $\fts =r-2$ or $r-3$, $\ft=0$ or $1$. Therefore, $\wts$ is generically maximal. Therefore, $\wt$ is generically maximal. 
\end{proof}
To prove Theorem \ref{signature 1}, we will use Theorem \ref{numerical criteria} with $u=2$. By assumption, we have $f_{\OO,1}=0$, $f_{\OO,2}=1$. In order to apply Theorem \ref{numerical criteria}, we need to check the following two conditions:
\begin{enumerate}
    \item $\dim(\pi_{\tau_{\OO,1}}\circ L_{\tau_{\OO,1},|\OO|-1} \circ \dots \circ L_{\tau_{\OO,1},0}(M_{\tau_{\OO,1}}))=f_{\OO,1}=0;$
    \item 
    $\dim(\pi_{\tau_{\OO,2}}\circ L_{\tau_{\OO,2},|\OO|-1} \circ \dots \circ L_{\tau_{\OO,2},0}(M_{\tau_{\OO,2}}))=f_{\OO,2}-f_{\OO,1}=1.$
\end{enumerate}
The first condition is trivially satisfied since $\pi_{\tau_{\OO,1}}:M_{\tau_{\OO,1}}\to Q_{\tau_{\OO,1}}$ and $\dim(Q_{\tau_{\OO,1}})=f(\tau_{\OO,1}^*)=0$. Denote $\tau=\tau_{\OO,2}$, so $f(\ts)=1$. Say $\tau=\tau_b$, then by the choice of Frobenius orbit, we know that $\gcd(b,l)=1$. Since $\dim(Q_{\tau})=1$, it suffices to show that the composite map
$$\pi_{\tau} \circ L_{\tau,|\OO|-1} \circ \dots \circ L_{\tau,0}: Q_{\tau} \to Q_{\tau}$$
has rank at least $1$ for $(x_1,\dots, x_r)$ outside a proper closed subset of $\mlra(\fpb)$. Equivalently, we need to show that the composite matrix is not identically zero as a polynomial in $\fpb[x_1, \dots, x_r]$. 

Recall that for $0 \leq i \leq |\OO|-1$, we have 
\begin{equation*}
    L_{\tau,i} = \begin{cases}
           F_{p^i\tau} &\text{if } f(p^i\tau^*)\geq 1\\
           V'_{p^i\tau} &\text{if } f(p^i\tau^*)=0.
    \end{cases}
\end{equation*}
Consider the map $\check{\phi}_{p^i\tau^*}: Q_{p^i\tau^*}^{\smvee} \to Q_{p^{i+1}\tau^*}^{\smvee}$, defined as $\check{\phi}_{p^i\tau^*}(x)=\phi_{p^i\tau^*}(\check{x})^{\smvee}$. Also consider the map $\check{\psi}_{p^i\tau^*}: Q_{p^i\tau^*}^{\smvee} \to Q_{p^{i+1}\tau^*}$, defined as $\check{\psi}_{p^i\tau^*}(x)=\psi_{p^i\tau^*}(\check{x})^{\smvee}$. Recall that we have the following basis:
\begin{align*}
&\{\zeta_{p^i\tau^*,j}:1 \leq j \leq f(p^i\tau) \} \text{ for } Q_{p^i\tau^*},
&
&\{\zeta_{p^{i+1}\tau^*,j}: 1 \leq j \leq f(p^{i+1}\tau) \} \text{ for } Q_{p^{i+1}\tau^*},\\
& \{\omega_{p^i\tau^*,j}: 1 \leq j \leq f(p^i\tau) \} \text{ for } Q_{p^i\tau^*}^{\smvee},
&
&\{\omega_{p^{i+1}\tau^*,j}: 1 \leq j \leq f(p^{i+1}\tau) \} \text{ for } Q_{p^{i+1}\tau^*}^{\smvee}.
\end{align*}
Under these basis, $\check{\phi}_{p^i\tau^*}$ and $\phi_{p^i\tau^*}$ have identical matrices and $\check{\psi}_{p^i\tau^*}, \psi_{p^i\tau^*}$ have identical matrices.
It turns out that the products of $L_{\tau,i}$ can be reduced to products of $\phi$ and $\psi$. For $0\leq i\leq |\OO|-1$, we denote
\begin{equation*}
    A_{i}=
    \begin{cases}
    \phi_{p^i\tau}: Q_{p^i\tau}\to Q_{p^{i+1}\tau} &\text{if } f(p^i\tau^*)\geq 1, f(p^{i+1}\tau^*)\geq 1,\\
    \check{\phi}_{p^i\tau^*}: Q_{p^i\tau^*}^{\smvee} \to Q_{p^{i+1}\tau^*}^{\smvee}  &\text{if } f(p^i\tau^*)=0, f(p^{i+1}\tau^*)=0,\\
    \psi_{p^i\tau}: Q_{p^i\tau}\to Q_{p^{i+1}\tau^*}^{\smvee} &\text{if } f(p^i\tau^*)\geq 1, f(p^{i+1}\tau^*)=0,\\
    \check{\psi}_{p^i\tau^*}: Q_{p^i\tau^*}^{\smvee} \to Q_{p^{i+1}\tau^*} &\text{if } f(p^i\tau^*)=0, f(p^{i+1}\tau^*)\geq 1.
    \end{cases}
\end{equation*}

\begin{proposition}\label{sequence}
The composition
$A_{|\OO|-1}\circ \dots \circ A_{0} :Q_{\tau}\to Q_{\tau}$
is well defined, moreover it gives the same map as
$\pi_{\tau}\circ L_{\tau,|\OO|-1}\circ \dots \circ L_{\tau,0} :Q_{\tau}\to Q_{\tau}$.
\end{proposition}
\begin{proof}
Firstly note that for $F_{p^i\tau}=\phi_{p^i\tau}+\psi_{p^i\tau}$ and $V'_{p^i\tau}=\check{\phi}_{p^i\tau^*}+\check{\psi}_{p^i\tau^*}$.
Also notice that for $0\leq i\leq |\OO|-2$:
\begin{itemize}
    \item If $f(p^i\tau^*)\geq 1, f(p^{i+1}\tau^*)\geq 1$, then $L_{\tau,i}=F_{p^i\tau}$ and $L_{\tau,i+1}=F_{p^{i+1}\tau}$. Moreover, $F_{p^{i+1}\tau}\circ F_{p^{i}\tau}= F_{p^{i+1}\tau}\circ \phi_{p^i\tau}$.
    \item If $f(p^i\tau^*)=0, f(p^{i+1}\tau^*)=0$, then $L_{\tau,i}=V'_{p^i\tau}$ and $L_{\tau,i+1}=V'_{p^{i+1}\tau}$. Moreover, $V'_{p^{i+1}\tau}\circ V'_{p^{i}\tau}=  V'_{p^{i+1}\tau}\circ \check{\phi}_{p^i\tau^*}$.
    \item If $f(p^i\tau^*)\geq 1, f(p^{i+1}\tau^*)=0$, then $L_{\tau,i}=F_{p^i\tau}$ and $L_{\tau,i+1}=V'_{p^{i+1}\tau}$. Moreover, $V'_{p^{i+1}\tau}\circ F_{p^{i}\tau}= V'_{p^{i+1}\tau} \circ \psi_{p^i\tau}$.
    \item If $f(p^i\tau^*)=0, f(p^{i+1}\tau^*)\geq 1$, then $L_{\tau,i}=V'_{p^i\tau}$ and $L_{\tau,i+1}=F_{p^{i+1}\tau}$. Moreover, $F_{p^{i+1}\tau}\circ V'_{p^{i}\tau}= F_{p^{i+1}\tau}\circ \check{\psi}_{p^i\tau^*}$.
\end{itemize}
Moreover, for $i=|\OO|-1$, if $f(p^{|\OO|-1}\ts)\geq 1$, then $L_{\tau,|\OO|-1}=F_{p^{|\OO|-1}\tau}$. Therefore $\pi_{\tau}\circ L_{\tau,|\OO|-1}=\phi_{p^{|\OO|-1}\tau}$. On the other hand, if $f(p^{|\OO|-1}\ts)=0$, then $L_{\tau,|\OO|-1}=V'_{p^{|\OO|-1}\tau}$. Therefore $\pi_{\tau}\circ L_{\tau,|\OO|-1}=\check{\psi}_{p^{|\OO|-1}\ts}$.
\end{proof}
 

\begin{lemma}\label{can use psii}
For $i$ such that $A_i=\psi_{\pit}$, $\psi'_{\pit}$ defines an valid extension of $\psi_{\pit}$ to $\qpit$. 
For $i$ such that $A_i=\check{\psi}_{\pits}$, $\psi'_{\pit^*}$ defines an valid extension of $\psi_{\pit^*}$ to $Q_{\pits}$. 
\end{lemma}
\begin{proof}
If $A_i=\psi_{\pit}$, then $f(p^{i+1}\ts)=0$, then by Proposition \ref{When is psii' correct}, $\psi_{p^i\tau}'$ defines the valid extension of $\psi_{\pit}$. If $A_i=\check{\psi}_{\pits}$, then $f(\pits)=0$ meaning $f((\pits)^*)=g(\OO)$. Thus by Proposition \ref{When is psii' correct}, $\psi_{\pits}'$ defines the valid extension of $\psi_{\pits}$.
\end{proof}
Thus, we can always use $\psiip$ when testify the criteria. Now notice that the matrix product $\aip$ is a $1 \times 1$ matrix, and we want to show that it is not identically zero. Let $\mathfrak{J}$ be the family of all functions $J:\{0,1, \dots, |\OO|\} \to \mathbb{N}$
such that $J(0)=J(|\OO|)=1$, and for each $i$, $1 \leq J(i) \leq a(i)$, where 
\begin{equation*}
    a(i) = 
    \begin{cases}
    f(\pits) & \text{ if $A_i=\phi_{\pit}$ or $\check{\psi}_{\pits}$} \\
    f(\pit) & \text{ if $A_i=\psi_{\pit}$ or $\check{\phi}_{\pits}$}.\\
    \end{cases} 
\end{equation*}
Note that $A_i$ is a $a(i+1)\times a(i)$ matrix.
Let $A_i(j',j)$ denote the $(j',j)^{th}$ entry of the matrix $A_i$, and let $m_i(j',j)$ be the maximal monomial in $A_i(j',j)$.
For $J \in \mathfrak{J}$, let $R_{J,i}=A_i(J(i+1),J(i))$ and $T_{J,i}=m_i(J(i+1),J(i))$. Also let $R_{J}=\prod_{i=0}^{|\OO|-1}R_{J,i}^{p^{|\OO|-i-1}}$ and $T_J=\prod_{i=0}^{|\OO|-1}T_{J,i}^{p^{|\OO|-i-1}}$ ($A_i$ are $\sigma$-linear maps). So $T_{J}$ is the maximal monomial of $R_J$. Then $\aip=\sum_{J \in \mathfrak{J}}R_J$ as a polynomial in $\fpb[x_1,\dots,x_r]$.
Therefore, the non-vanishing the $\aip$ will follow, if we show that for every $J_1, J_2 \in \mathfrak{J}$, if $T_{J_1}=T_{J_2}$ then $J_1=J_2$.

Now we fix $J_1,J_2\in \mathfrak{J}$ such that $T_{J_1}=T_{J_2}$. The proof of Theorem \ref{signature 1} will be complete if we show that $J_1=J_2$.
For $1 \leq s \leq r$, $0 \leq i \leq |\OO|-1$, we let
$\eis=v_{x_s}(T_{J_1,i})-v_{x_s}(T_{J_2,i})$.
Since $T_{J_1}=T_{J_2}$, for each $1 \leq s \leq r$, we have that
$$\sum_{i=0}^{|\OO|-1}p^{|\OO|-i-1}\eis=v_{x_s}(T_{J_1})-v_{x_s}(T_{J_2})=0.$$


\begin{proposition}\label{eise0}
For each $1 \leq i \leq l-1$ and $1 \leq s \leq r$, we have $\eis=0$.
\end{proposition}
\begin{proof}
If we show that $|\eis|<p$ for each $i,s$, then by the uniqueness of base $p$ representation, we will have that $\eis=0$ for each $i,s$. Recall that $\tau=\tau_b$, so $p^i\tau=\tau_{p^ib}$.
\begin{itemize}
     \item Case 1: If $A_i=\phi_{\pit}$ or $A_i=\check{\phi}_{\pit^*}$, for notational convenience we define
     \begin{align*}
    N(J,i)&=
         \begin{cases}
          s(\pib)-pJ(i)+J(i+1) &\text{if } A_i=\phi_{\pit}\\
          s(\pib^*)-pJ(i)+J(i+1) &\text{if } A_i=\check{\phi}_{\pit^*},
         \end{cases}
    &
    c(J,i)&=
         \begin{cases}
         c(\pib,N(J,i)) & \text{if }A_i=\phi_{\pit}\\ 
         c(\pibs,N(J,i)) & \text{if } A_i=\check{\phi}_{\pit^*},
         \end{cases}\\
    C(J,i)&=
         \begin{cases}
         C(\pib,N(J,i)) & \text{if }A_i=\phi_{\pit}\\ 
         C(\pibs,N(J,i)) & \text{if } A_i=\check{\phi}_{\pit^*},
         \end{cases}
    &
     X(J,i)&=
         \begin{cases}
         X(\pib, N(J,i)) & \text{if }A_i=\phi_{\pit}\\ X(\pibs, N(J,i)) & \text{if } A_i=\check{\phi}_{\pit^*}.
         \end{cases}
     \end{align*}
     Where $c,C,N,s,X$ are as defined in Definition \ref{define c C N}. Then, $T_{J,i}=m_i(J(i+1),J(i))=X(J,i)$, and we have the following expression for $\eis=v_{x_s}(T_{J_1,i})-v_{x_s}(T_{J_2,i})$:
         \begin{equation*}
 \eis =
    \begin{cases}
      0 & \text{if $c(J_1,i)>s,c(J_2,i)>s$}\\
      C(J_1,i)-\pfibs & \text{if $c(J_1,i)=s,c(J_2,i)>s$}\\
      \pfibs-C(J_2,i)& \text{if $c(J_1,i)>s,c(J_2,i)=s$} \\
      C(J_1,i)-C(J_2,i) & \text{if $c(J_1,i)=c(J_2,i)=s$}\\
      -\pfibs & \text{if $c(J_1,i)<s,c(J_2,i)>s$}\\
       -C(J_2,i) & \text{if $c(J_1,i)<s,c(J_2,i)=s$}\\
       0 & \text{if $c(J_1,i)<s,c(J_2,i)<s$}\\
       \pfibs & \text{if $c(J_1,i)>s,c(J_2,i)<s$}\\
       C(J_1,i) & \text{if $c(J_1,i)=s,c(J_2,i)<s$}.
    \end{cases}       
\end{equation*}
In all the cases, $C(J,i) < \pfibc \leq \floor*{p\frac{l-1}{l}}$. When $p>(r-2)l$, $C(J,i)<p-(r-2) \leq p-2<p$. So, $|\eis|<p$.
\item Case 2: If $A_i=\psi_{\pit}$ or $\check{\psi}_{\pits}$. Then we define the following quantities:
\begin{align*}
N(J,i)&=
         \begin{cases}
          s(\pib)-pJ(i) &\text{if } A_i=\psi_{\pit}\\
          s(\pibs)-pJ(i) &\text{if } A_i=\check{\psi}_{\pit^*},
         \end{cases}
&
c(J,i)&=
         \begin{cases}
         c(\pib,N(J,i)) & \text{if }A_i=\psi_{\pit}\\ c(\pibs,N(J,i)) & \text{if } A_i=\check{\psi}_{\pit^*},
         \end{cases}\\
C(J,i)&=
         \begin{cases}
         C(\pib,N(J,i)) & \text{if }A_i=\psi_{\pit}\\ C(\pibs,N(J,i)) & \text{if } A_i=\check{\psi}_{\pit^*},
         \end{cases}
&
X(J,i)&=
         \begin{cases}
         X(\pib,N(J,i)) & \text{if }A_i=\psi_{\pit}\\ X(\pibs,N(J,i)) & \text{if } A_i=\check{\psi}_{\pit^*}.
         \end{cases}
\end{align*}
\end{itemize}
Then we have the following expression for $m_i(J(i+1),J(i))$:
\begin{equation*}
    m_i(J(i+1),J(i))=
    \begin{cases}
    x_1 \dots x_{r-J(i+1)}X(J,i) & \text{if } r-J(i+1) \geq c(J,i) \\
    x_1 \dots x_{r-J(i+1)-1}x_{c(J,i)}X(J,i) & \text{if } c(J,i) \geq r-J(i+1)+1.\\
    \end{cases}
\end{equation*}
Then, for $t\in\{1,2\}$, we define
     \begin{equation*}
     \alpha_{i,s,t}= \begin{cases} v_{x_s}(x_1 \dots x_{r-J_t(i+1)}) & \text{ if } c(J_t,i) \leq r-J_t(i+1)\\
     v_{x_s}(x_1 \dots x_{r-J_t(i+1)-1}x_{c(J_t,i)}) & \text{if } c(J_t,i) \geq r-J_t(i+1)+1.\\
     \end{cases}  
     \end{equation*}
      So then, $\eis=a_{i,s}+b_{i,s}$, where $a_{i,s}=\alpha_{i,s,1}-\alpha_{i,s,2}$ and $b_{i,s}=v_{x_s}(X(J_1,i))-v_{x_s}(X(J_2,i))$.
     For $a_{i,s}$, we know that $|a_{i,s}|\leq 1$.
     \iffalse
         \begin{equation*}
 a_{i,s} =
    \begin{cases}
      -1 & \text{if $J_1(p^i)>5-s, J_2(p^i) \leq 5-s$}\\
      0 & \text{if $J_1(p^i)>5-s, J_2(p^i) > 5-s$}\\
      0 & \text{if $J_1(p^i)<5-s, J_2(p^i) < 5-s$}\\
      1 & \text{if $J_1(p^i)\leq 5-s, J_2(p^i) > 5-s$}\\
    \end{cases}       
\end{equation*}
     \fi
We have the following expression for $b_{i,s}$
         \begin{equation*}
 b_{i,s} =
    \begin{cases}
      0 & \text{if $c(J_1,i)>s,c(J_2,i)>s$}\\
      C(J_1,i)-\pfibs & \text{if $c(J_1,i)=s,c(J_2,i)>s$}\\
      \pfibs-C(J_2,i)& \text{if $c(J_1,i)>s,c(J_2,i)=s$} \\
      C(J_1,i)-C(J_2,i) & \text{if $c(J_1,i)=c(J_2,i)=s$}\\
      -\pfibs & \text{if $c(J_1,i)<s,c(J_2,i)>s$}\\
       -C(J_2,i) & \text{if $c(J_1,i)<s,c(J_2,i)=s$}\\
       0 & \text{if $c(J_1,i)<s,c(J_2,i)<s$}\\
       \pfibs & \text{if $c(J_1,i)>s,c(J_2,i)<s$}\\
       C(J_1,i) & \text{if $c(J_1,i)=s,c(J_2,i)<s$}.
    \end{cases}       
\end{equation*}
Notice that since $p>(r-2)l$, then $p(1-\frac{1}{l}) \leq p-2$. Then, $C(J,i)<\pfibs \leq p-2$. Consequently, $|b_{i,s}| \leq p-2$. Thus, 
\begin{equation*}
    \begin{split}
        |\eis| & = |a_{i,s}+b_{i,s}|  \leq |a_{i,s}|+|b_{i,s}|   \leq 1+p-2  \leq p-1.
    \end{split}
\end{equation*}
Since $|\eis| < p$ in all cases, we can conclude that $\eis=0$.
\end{proof}
\begin{proposition}\label{TjoeTjt}
For each $0 \leq i \leq |\OO|$, we have $J_1(i)=J_2(i)$. Consequently, $J_1=J_2$.
\end{proposition}
\begin{proof}
We do backward induction on $i$. The base case is given as $J_1(|\OO|)=J_2(|\OO|)=1$. Suppose we know that $J_1(i+1)=J_2(i+1)$, we want to show that $J_1(i)=J_2(i)$.
\begin{itemize}
     \item Case 1: if $A_i=\phi_{\pit}$ or $\phid_{\pits}$. Then we further break into two sub-cases:
     \begin{itemize}
         \item case 1A: $c(J_1,i)=c(J_2,i)$. Take  $s=c(J_1,i)=c(J_2,i)$, we know that $\eis=0$. From the proof of Proposition \ref{eise0}, we know that $\eis=C(J_1,i)-C(J_2,i)$. Therefore, we see that $C(J_1,i)=C(J_2,i)$, which implies that $N(J_1,i)=N(J_2,i)$. Since $J(i)$ is a linear combination of $N(J,i)$ and $J(i+1)$, we have $J_1(i)=J_2(i)$.
         \item case 1B: $c(J_1,i) \neq c(J_2,i)$. We will show that this sub-case can not happen. Assume without loss of generality that $c(J_1,i) < c(J_2,i)$. Then take $s=c(J_1,i)< c(J_2,i)$, we get that $\eis=C(J_1,i)-\pfibs<0$, which contradicts the fact that $\eis=0$.
     \end{itemize}
     So, if $A_i=\phi_{\pit}$ or $\phid_{\pits}$ then $J_1(i+1)=J_2(i+1)$ implies $J_1(i)=J_2(i)$.
     \item Case 2: If $A_i=\psi_{\pit}$ or $\psid_{\pits}$. We again get two sub-cases.
     \begin{itemize}
         \item case 2A: If $c(J_1,i)=c(J_2,i)$, then $\alpha_{i,s,1}=\alpha_{i,s,2}$, as $\alpha_{i,s,t}$ only depends on $J(i+1)$ and $c(J,i)$. Now take $s=c(J_1,i)=c(J_2,i)$ and notice that $\eta_{i,s}=0$ implies that $C(J_1,i)=C(J_2,i)$. Since $c(J_1,i)=c(J_2,i)$ and $C(J_1,i)=C(J_2,i)$, we see that $N(J_1,i)=N(J_2,i)$ and hence $J_1(i)=J_2(i)$.
         \item case 2B: If $c(J_1,i) \neq c(J_2,i)$. We will show that this sub-case cannot happen.
         Assume without loss of generality that $c(J_1,i)<c(J_2,i)$. Denote $c_1=c(J_1,i)$ and $c_2=c(J_2,i)$.
         
         First take $s=c_1<c_2$, we get that $b_{i,c_1}=C(J_1,i)-\pbs<0$. Therefore $a_{i,c_1}>0$, this means that $\alpha_{i,c_1,1}=1$ and $\alpha_{i,c_1,2}=0$. From definition of $\alpha$, this implies that $c_1>r-J_2(i+1)$. Therefore $c_2>r-J_1(i+1)$ and $c_2\neq c_1$, so $\alpha_{i,c_2,1}=0$. Next, since $c_2>r-J_2(i+1)$, we see that $\alpha_{i,c_2,2}=1$. This means that $a_{i,c_2}=-1$. At the same time  $b_{i,c_2}=-C(J_2,i)\leq 0$. This implies $\eta_{i,c_2}\leq -1$ which is a contradiction. Therefore, this sub-case cannot happen.
        \end{itemize}
\end{itemize}
We have shown that in all cases, $J_1(i+1)=J_2(i+1)$ implies $J_1(i)=J_2(i)$. Therefore, by backward induction we see that $J_1=J_2$. As noted earlier this completes the proof of Theorem \ref{signature 1}.
\end{proof}



\section{Proof of main result}\label{new part mu ordinary}
Recall Notation \ref{new part mu ord} with $4\leq r\leq 5$.  The goal of this section is to show that there exists a non-empty open subset $U \subseteq \mlra$ such that for all $\xb \in U$, $J(\ct)^\n$ is $\mu$-ordinary.  As noted earlier this will complete the proof of Theorem~\ref{abelian cover}.
By \cite[Theorem 1.3.7]{Moonen EO type formula}, it is enough to show that for each $\tau\in \tg$ with $\ker(\tau)=\{1\}$, $\wt$ is generically maximal.

\begin{lemma}\label{smallest signature}
Given a Frobenius orbit $\OO$, if $\tau\in \OO$ satisfies $f(\tau^*)=f_{\OO,1}$, then $w_{\tau}$ is generically maximal.
\end{lemma}
\begin{proof}
We apply Theorem \ref{numerical criteria} with $u=1$. We only need to check that $\dim(\pi_{\tau_{\OO,1}}\circ L_{\tau_{\OO,1},|\OO|-1} \circ \dots \circ L_{\tau_{\OO,1},0}(M_{\tau_{\OO,1}}))=f_{\OO,1}$ holds outside of a proper closed subset of $\mlra(\fpb)$. Note that $L_{\tau_{\OO,1},i}=F_{p^i\tau_{\OO,1}}$, so the condition is reduced to showing that $\pi_{\tau_{\OO,1}}\circ F^{|\OO|}: Q_{\tau_{\OO,1}}\to Q_{\tau_{\OO,1}}$ is an isomorphism. The main theorem in \cite[Section 6]{Irene} shows that the determinant of this map is a non-zero polynomial in $\fpb[x_1, \dots, x_r]$. Therefore, the condition only fails on the vanishing locus of the determinant, which is a proper closed subset.
\end{proof}

\begin{corollary}\label{Largest signature}
Given a Frobenius orbit $\OO$, if $\tau\in \OO$ satisfies $f(\tau^*)=f_{\OO,s(\OO)}$, then $w_{\tau}$ is generically maximal.
\end{corollary}
\begin{proof}
If $f(\tau^*)=f_{\OO,s(\OO)}$, then $f((\tau^*)^*)=f_{\overline{\OO},1}$. Therefore, by Lemma \ref{smallest signature}, we know that $w_{\tau^*}$ is maximal. Hence $w_{\tau}$ is also generically maximal.
\end{proof}
Now, we deal with the case when $r=4$. Let $\OO$ be a Frobenius orbit such that for every $\tau \in \OO$, $\ker(\tau)=\{1\}$. Therefore, $g(\OO)=r-2=2$. Hence we have
$\FF(\overline{\OO})=\{2-a\mid a\in \FF(\OO)\}$.







\begin{theorem}\label{r=4 main result}
Assume $r=4$.
Suppose we are given a Frobenius orbit $\OO$ such that $\ker(\tau)=1$ for every $\tau \in \OO$. Then for every $\tau \in \OO$, $w_{\tau}$ is generically maximal.
\end{theorem}
\begin{proof}
We make cases based on $\FF(\OO)$.
\begin{itemize}
    \item Case 1: $s(\OO)=1$. Then $\FF(\OO)=\{f_{\OO,1}\}$. Then for any $\tau\in\OO$, we have $f(\tau^*)=f_{\OO,1}$. It follows from Lemma \ref{smallest signature} that $w_{\tau}$ is generically maximal.
    \item Case 2: $s(\OO)=2$. Then $\FF(\OO)=\{f_{\OO,1},f_{\OO,2}\}$. For $\tau\in\OO$, with $f(\tau^*)=f_{\OO,1}$, it follows from Lemma \ref{smallest signature} that $w_{\tau}$ is generically maximal. For $\tau\in\OO$, with $f(\tau^*)=f_{\OO,2}$, it follows from Corollary \ref{Largest signature} that $w_{\tau}$ is generically maximal.
    \item Case 3: $s(\OO)=3$. Then $\FF(\OO)=\{0,1,2\}$.
    For $\tau\in\OO$, with $f(\tau^*)\in\{0,1\}$, it follows from Theorem \ref{signature 1} that $w_{\tau}$ is generically maximal. For $\tau\in\OO$, with $f(\tau^*)=2$, it follows from Corollary \ref{Largest signature} that $w_{\tau}$ is generically maximal.\qedhere
\end{itemize}
\end{proof}
Next, we deal with the case when $r=5$. Let $\OO$ be a Frobenius orbit such that for every $\tau \in \OO$, $\ker(\tau)=\{1\}$. Therefore, $g(\OO)=r-2=3$. Hence we have
$\FF(\overline{\OO})=\{3-a\mid a\in \FF(\OO)\}$.




\begin{theorem}\label{rfivemax}
Assume $r=5$.
Given a Frobenius orbit $\OO$ such that $\ker(\tau)=\{1\}$ for every $\tau \in \OO$. Then for every $\tau \in \OO$, $w_{\tau}$ is generically maximal.
\end{theorem}
\begin{proof}
We make cases based on $\FF(\OO)$.
\begin{itemize}
    \item Case 1: $s(\OO)=1$. This case follows from Lemma \ref{smallest signature}.
    \item Case 2: $s(\OO)=2$. This follows from Lemma \ref{smallest signature} and Corollary \ref{Largest signature}.
    \item Case 3: $\FF(\OO)=\{0,1,2\}$. This follows from Theorem \ref{signature 1} and Corollary \ref{Largest signature}.
    \item Case 4: $\FF(\OO)=\{0,1,3\}$. This follows from Theorem \ref{signature 1} and Corollary \ref{Largest signature}.
    \item Case 5: $\FF(\OO)=\{0,2,3\}$. In this case $\FF(\overline{\OO})=\{0,1,3\}$. Therefore, by Case 4, for all $\tau\in \overline{\OO}$, we have $w_{\tau}$ is generically maximal. Now given $\tau\in \OO$, we have $\tau^*\in \overline{\OO}$, so $w_{\tau^*}$ is generically maximal and hence $w_{\tau}$ is also generically maximal.
    \item Case 6: $\FF(\OO)=\{1,2,3\}$. In this case $\FF(\overline{\OO})=\{0,1,2\}$. Therefore, by Case 3, for all $\tau\in \overline{\OO}$, we have $w_{\tau}$ is generically maximal. Now given $\tau\in \OO$, we have $\tau^*\in \overline{\OO}$, so $w_{\tau^*}$ is generically maximal and hence $w_{\tau}$ is also generically maximal.
    \item Case 7: $\FF(\OO)=\{0,1,2,3\}$. For $f(\tau^*)\in\{0,1,3\}$, by Theorem \ref{signature 1} and Corollary \ref{Largest signature}, we know that $w_{\tau}$ is generically maximal. Next, notice that $\FF(\overline{\OO})=\{0,1,2,3\}$. Now, given $\tau\in \OO$ with $f(\tau^*)=2$, we see that $\tau^*\in \overline{\OO}$ with $f((\tau^*)^*)=1$. Therefore, by Theorem \ref{signature 1}, we see that $w_{\tau^*}$ is generically maximal. Hence, $w_{\tau}$ is also generically maximal.\qedhere
\end{itemize}
\end{proof}


\begin{remark}
    
Note that the restriction $r\leq 5$ does not imply a bound on the genus of the covers. That is, our result applies to families of curves with arbitrarily large genus. For example, given a cyclic monodromy datum $\lra$, 
the genus of the curves parameterized by $\mlra$ is:
$$g=g(l,r,\au)=1 +
\frac{(r-2)l-\sum_{j=1}^r \gcd(a(j),l)}{2}\geq 1+\frac{(r-2)l-r}{2}.$$
In particular,  $g\lra$ can grow linearly with $l$. 

Also, while the restriction $r\leq 5$ implies $\dim(\mlra)=r-3\leq 2$, it does not imply a bound on the dimension of the ambient Shimura variety $\shgf$. That is, our result applies to families of curves in ambient Shimura varieties of arbitrarily large dimension.   For example,
    consider the cylic monodromy datum $(l,4,(1,1,1,l-3))$, where $3 \nmid l$. Then $\dim(\mlra)=1$, while  
    %the dimension of $\shgf$  is 
    %$$    \dim(\shmf) = 
    %\begin{cases}
     %   \sum_{i=1}^{l-1}f(\ti)f(\tis)+\frac{f(\tau_{k})f(\tau_k)+1}{2} & \text{if } l=2k \\
      %  \sum_{i=1}^{l-1}f(\ti)f(\tis) & \text{if }l \text{ is odd} 
    %\end{cases}$$
    %In particular, $\dim(\shmf)$ is at least $\sum_{i=1}^{l-1}f(\ti)f(\tis)$. For $\au=(1,1,1,l-3)$,
    $\dim(\shmf)\geq \floor{\frac{l+1}{3}}$, which grows linearly with $l$. 
    %Thus, for large $l$, $\dim(\mlra) \ll \dim(\shmf)$, and by Theorem \ref{abelian cover} we still have that $T(\mlra)$ intersects the $\mu$-ordinary locus of $\shmf$ non-trivially for any $p>2l$.

\end{remark}


\section{Family of $\mlra$ with $r$ arbitrarily large}\label{list}
In this section, by adapting the results from \cite {clutching argument}, we list (infinitely many) instances of cyclic monodromy data with $r\geq 6$ for which the statement of Theorem \ref{abelian cover} holds.



\begin{definition}\cite[Definition 3.3, 4.2]{clutching argument}
    Let $\gamma_1=(l,r_1,\au_1)$ and $\gamma_2=(l,r_2,\au_2)$ be a pair of monodromy data of cyclic cover of $\Po$. Fix $p \nmid l$, and denote $\mu_l$ as $G$. We use $f_1$ (respectively $f_2$) to denote the signature determined by $\gamma_1$(respectively $\gamma_2$). 
    \begin{itemize}
        \item $(\gamma_1,\gamma_2)$ is said to be admissible if $\au_1(r_1) \equiv -\au_2(r_2) \pmod l$.
        \item $(\gamma_1,\gamma_2)$ is said to be balanced if for every Frobenius orbit $\OO \in \og$, for every $\omega, \tau \in \OO$, we have
     $$ f_1(\omega)> f_1(\tau) \implies f_2(\omega)\geq f_2(\tau)$$
    \end{itemize}
\end{definition}
Note that %the admissible condition is easy to control. For the balanced condition,
$(\go,\gt)$ is balanced if either $\go,\gt$ define the same signature, or if $f_1$ is constant on Frobenius orbits.  

 Given an admissible pair $(\go,\gt)$, following \cite[Definition 3.5]{clutching argument}, we define the monodromy datum $\gth=(l,r_3,\au_3)$ where $r_3=r_1+r_2-2$ and
$$\au_3(i)=\begin{cases}
         \au_1(i) &\text{ if } 1 \leq i \leq r_1-1, \\
         \au_2(i-r_1+1) &\text{ if } r_1 \leq i \leq r_1+r_2-2.
    \end{cases}$$
Then the boundary of $\mathcal{M}(l, r_3,\au_3)$ contains the image of $\mathcal{M}(l, r_1, \au_1)\times \mathcal{M}(l, r_2, \au_2)$ under clutching morphism.

Let $p$ be a prime, $p\nmid l$. For $1\leq i\leq 3$, we write  $\mathcal{M}(\gamma_i)=\mathcal{M}(l,r_i,\au_i)$ over ${\fpb}$, and denote by $\mathcal{M}(\gamma_i)[\mu-{\rm ord}]$ the locus whose image under Torelli map is $\mu$-ordinary in $\text{Sh}(G,\underline{f_i})$.


\begin{theorem}\cite[Theorem 4.5]{clutching argument}\label{clutching argument}
    Suppose $(\go,\gt)$ is a pair of monodromy data that is admissible and balanced. If $\mathcal{M}(\gamma_1)[\mu-{\rm ord}]$ and $\mathcal{M}(\gamma_2)[\mu-{\rm ord}]$ are non-empty, then $\mathcal{M}(\gamma_3)[\mu-{\rm ord}]$ is also non-empty.
\end{theorem}

We apply \cite[Theorem 4.5]{clutching argument} to construct a system of monodromy data $\lra$ with $r$ arbitrarily large, satisfying  $\mlra[\mu-{\rm ord}]\neq \emptyset$, for $p$ sufficiently large. 

\begin{proposition}\label{general}
     Let $\gb=(l, n, \underline{b})$ be a cyclic monodromy datum. For any $t,s\geq 1$, and any $1\leq a_i\leq l-1$, with $\gcd(a_1,l)=1$, $1\leq i\leq t$, define the monodromy datum $\gamma_{t,s}=(l,r_{t,s},\au_{t,s})$, where $r_{t,s}=2t+ns$,  and
    $$\au_{t,s}(i)=
        \begin{cases}
        a_{j} & \text{ if } 1 \leq j \leq t, i=2j-1,\\
        l-a_{j} & \text{ if } 1 \leq j \leq t, i=2j,\\
        b_k & \text{ if } 2t+(k-1)s+1\leq i\leq 2t+ks\\
        \end{cases}.$$
    Assume $p>2l$ is a prime.  If $\MM(\gb)[\mu-{\rm ord}]$ is non-empty then $\MM(\gamma_{t,s})[\mu-{\rm ord}]$ is non-empty.
\end{proposition}
\begin{proof}
    First we show by induction on $s$ that $\gamma_{1,s}$ satisfies $\MM(\gamma_{1,s})[\mu-{\rm ord}] \neq \emptyset$. The base case is $\gamma_{1,1}$. Consider $\gamma_1=\gb$, $\gamma_2=(l,4,b_1,-b_1,a_1,-a_1)$. Then $f_{\gamma_2}$ is constant on Frobenius orbits,. Indeed, $$f_{\gamma_2}(\ti)=-1+{\langle\frac{ia_1}{l}\rangle}+{\langle\frac{-ia_1}{l}\rangle}+{\langle\frac{ib_1}{l}\rangle}+{\langle\frac{-ib_1}{l}\rangle}$$
    and ${\langle\frac{ia_1}{l}\rangle}+{\langle\frac{-ia_1}{l}\rangle}=0$ iff $ia_1 \equiv 0 \pmod{l}$ and it is $1$ otherwise. Note that if $\tau_i,\tau_j$ are in the same Frobenius orbit, then either both $ia_1$, $ja_1$ are divisible by $l$ or neither of them are. Thus $f_{\gamma_2}$ is constant on the Frobenius orbits, so $\gamma_1,\gamma_2$ are admissible and balanced. Thus, by \cite[Theorem 4.5]{clutching argument}, $\gamma_3=\gamma_{1,1}$ has the property that $\MM(\gamma_{1,1})[\mu-{\rm ord}]$ is non-empty. 

    Next, as induction hypothesis suppose $\MM(\gamma_i)[\mu-{\rm ord}] \neq \emptyset$. Let $\gamma_1=\gamma_{1,s-1}$ and $\gamma_2=\gamma_{1,1}$. By inductive hypothesis, $(\go,\gt)$ are admissible and balanced. They are admissible because both contain $a_1, -a_1$. They are balanced, because for any $i$, $f_{\gamma_1}(\ti)+1-({\langle\frac{ia_1}{l}\rangle}+{\langle\frac{-ia_1}{l}\rangle})=(s-1)(f_{\gamma_2}(\ti)+1-({\langle\frac{ia_1}{l}\rangle}+{\langle\frac{-ia_1}{l}\rangle}))$. Therefore, $\gamma_3=\gamma_{1,s}$ satisfies that $\MM(\gamma_3)[\mu-{\rm ord}] \neq \emptyset$.

    Finally, we induct on $t$. The base case is given above. Let $\gamma_1=\gamma_{t-1,s}$ and $\gamma_2=(l,4,a_1,-a_1,a_t, -a_t)$. By induction gypothesis we know that $\MM(\gamma_3)[\mu-{\rm ord}] \neq \emptyset$. As above, we see that $f_{\gt}$ is constant on Frobenius orbits. So $(\go,\gt)$ are admissible and balanced. Therefore, $\gth=\gamma_{t,s}$ has the property that $\MM(\gth)[\mu-{\rm ord}] \neq \emptyset$.
\end{proof}
By Theorem \ref{abelian cover} and Proposition \ref{general} applied to $3\leq n\leq 5$, we have the following  families of cyclic monodromy data that satisfy the desired property:
\begin{proposition}\label{two t plus n}
    Let $l\geq 2$, $3\leq n\leq 5$, and $p$ a prime $p>\max(2l,(n-2)l)$.  For $t,s\geq 1$, and $1\leq b_1,\dots,b_n,a_1,\dots,a_t\leq l-1$ such that $\gcd(b_1,\dots,b_n,l)=1$, $\gcd(a_i,l)=1$ and $b_1+\dots+b_n\equiv 0\pmod{l}$, set $\gamma_{t,s}=(l,r_{t,s},\au_{t,s})$ with $r_{t,s}=2t+ns$ and
    $$\au_{t,s}(i)=
        \begin{cases}
        a_{j} & \text{ if } 1 \leq j \leq t, i=2j-1, \\
        l-a_{j} & \text{ if } 1 \leq j \leq t, i=2j,\\
        b_k & \text{ if } 2t+(k-1)s+1\leq i\leq 2t+ks.
        \end{cases}$$
    Then $\mathcal{M}(\gamma_{t,s})[\mu-{\rm ord}]$ is non-empty.
\end{proposition}
\begin{remark}
    %We compute some invariants of these families. 
    Fix $l\geq 2$ and $3 \leq n \leq 5$. Then given a $t$-tuple $(a_1, \dots, a_t)$ and a $n$-tuple $(b_1, \dots, b_n)$, consider the monodromy datum $\gts$ and the family $\mathcal{M}(\gts)$ from Proposition \ref{two t plus n}. Assume $\gcd(a_j,l)=1$ and $\gcd(b_k,l)=1$ for $1 \leq j \leq t$ and $1 \leq k \leq n$. Then 
$$\dim(\mts)=r_{t,s}-3=2t+ns-3,$$
%In particular, $\dim(\mts)$ grows linearly with $t$ and $s$. 
and the genus of the curve in the family $\mts$ is
$$g(\gts) = \frac{2t+ns-2}{2}(l+1)-\frac{2t+ns}{2},$$
%which grows linearly with $t$ and $s$ and $l$. 
In particular, the construction yields families of arbitrarily large dimension parametrizing curves with  arbitrarily large genus which satisfy the statement of Theorem \ref{abelian cover}.

For comparison, we also compute $\dim( \text{Sh}(\ml,\fb_{\gts}))$. Let  $\gb=(l,n,(b_1, \dots,b_n))$, and denote $d_b=\sum_{i=1}^n f_{\gb}(\ti)f_{\gb}(\tis)$. When $l$ is odd, $d_b=\dim(\text{Sh}(\ml,f_{\gb}))$, and when $l$ is even, $d_b \leq \dim(\text{Sh}(\ml,f_{\gb}))$. Note that $\gb$ does not depend on $t,s$. Then, 
\begin{equation*}
        %\begin{split}
          \dim(\shgts) \geq \sum_{i=1}^{l-1}f(\ti)f(\tis) =(l-1)(t+s-1)(t+(n-1)s-1)+s^2(d_b+(l-1)(n-2)).
         % \end{split}
    \end{equation*}
%So we have $\dim( \text{Sh}(\ml,\fb_{\gts}))$ at least $(l-1)(t+s-1)(t+(n-1)s-1)+s^2(d_b+(l-1)(n-2))$. 
Note that the lower bound grows linearly in $l$ and quadratically in $s,t$. 
%Thus, when $s,t \gg 0$, We have $\dim(\mts) \ll_{b,n} \dim(\text{Sh}(\ml,\fb_{\gts}))$ while $\mts[\mu]$ still non-empty for $p>(n-2)l$.
\end{remark}
%\section{Dimension}
%Theorem \ref{abelian cover} produces large families of cyclic covers of $\Po$, whose genus is arbitrarily large, while $\dim(T(\mlra))<\dim(\shmf)$. The genus of the curves parameterized by $\mlra$ is given as follows \cite[2.2]{Second Paper}
%$$g=g(l,r,\au)=1 +
%\frac{(r-2)l-\sum_{j=1}^r \gcd(a(j),l)}{2}$$
%So it is bounded below by $1+\frac{(r-2)l-r}{2}$. When $r$ is fixed, $g\lra$ can be arbitrarily large for large $l$. So the theorem gives families of $\mu$ ordinary curves with unbounded genus. 

%For monodromy datum $\lra$, $\dim(\mlra)=r-3$, while we can easily find tuples $\au$ such that the corresponding Shimura variety $\shmf$ has arbitrarily large dimension. For example, consider $r=4$, pick any $l$ such that $3 \nmid l$, and consider the monodromy datum $\lra=(l,4,(1,1,1,l-3))$. The corresponding Shimura variety $\shmf$ has dimension given by the following formula:
%$$    \dim(\shmf) = 
    %\begin{cases}
        %\sum_{i=1}^{l-1}f(\ti)f(\tis)+\frac{f(\tau_{k})f(\tau_k)+1}{2} & \text{if } l=2k \\
        %\sum_{i=1}^{l-1}f(\ti)f(\tis) & \text{if }l \text{ is odd} 
    %\end{cases}$$
    %Where $f(\ti)=\sum_{j=1}^r {\langle\frac{-ia_j}{l}\rangle}-1$ for $i \in \Z/l\Z-\{0\}$. So when $\lra=(l,4,(1,1,1,l-3))$, $\dim(\shmf)=\floor{\frac{l+1}{3}}$ while $\dim(\mlra)=1$. In particular, when $l$ grows to arbitrarily large, $\dim(\shmf)$ is much larger then $\dim(\mlra)$, yet $T(\mlra)$ still intersects the $\mu$-ordinary locus of $\shmf$ for $p$ sufficiently large. 

    %When $r \geq 6$, we fix an $n$ such that $3 \leq n \leq 5$. Then for the families $\mathcal{M}(\gamma_{t,s})$, we know that 
     %$$g(\gts)=1 +\frac{(2t+ns-2)l-2\sum_{j=1}^t \gcd(a_j,l)-n\sum_{k=1}^s\gcd(b_k,l)}{2}$$
    %$$\dim(\mts)=r_{t,s}-3=2t+ns-3$$
    %In particular, $g(\gts) \geq \frac{2t+ns-2}{2}l+1-\frac{2t+ns}{2}$, and $\dim(\mts)$ grows linearly with $t$ and $s$. 
    
    %Let $\text{Sh}(\ml,\fb_{\gts})$ denote the Shimura variety with signature type determined by $\gts$. Let $\gamma_b=(l,n,(b_1,\dots,b_n))$ and denote $\dim(\text{Sh}(\ml,\gamma_b))$ as $d_{b}$. We choose $a_j,b_k$ such that $\gcd(a_i,l)=\gcd(b_j,l)=1$. Then
    %\begin{equation*}
        %\begin{split}
          %f_{\gts}(\ti)&=\sum_{j=1}^t  {\langle\frac{-ia_j}{l}\rangle}+{\langle\frac{ia_j}{l}\rangle}+s\Big(\sum_{k=1}^n {\langle\frac{-ib_k}{l}\rangle}\Big)-1 \\  &=t+s(\alpha_i+1)-1
        %\end{split}
    %\end{equation*}
    %\begin{equation*}
        %\begin{split}
          %\sum_{i=1}^{l-1}f(\ti)f(\tis)&=\sum_{i=1}^{l-1}(t+s(\alpha_i+1)-1)(t+s(\alpha_i+1)-1)\\
          %&=\sum_{i=1}^{l-1}(t+s-1)^2+s(t+s-1)(\alpha_i+\alpha_{i^*})+s^2\alpha_i\alpha_{i^*}\\
          %&=\sum_{i=1}^{l-1}(t+s-1)^2+s(t+s-1)(n-2)+s^2(f_b(\ti)f_b(\tis)+(n-1)) \\
          %&=(t+s-1)^2(l-1)+s(t+s-1)(n-2)(l-1)+s^2(d_b+(l-1)(n-1)) \\
          %&=(l-1)(t+s-1)(t+(n-1)s-1)+s^2(d_b+(l-1)(n-2))
          %\end{split}
    %\end{equation*}
    %So with fixed $\ub$, we have 
    %$\dim( \text{Sh}(\ml,\fb_{\gts}))$ bounded below by $(l-1)(t+s-1)(t+(n-1)s-1)+s^2(d_b+(l-1)(n-2))$ which grows quadratically with $s,t$ and grows linearly with $l$. When $n=3,4,5$, when $s,t \gg 0$, We have $\dim(\mts) \ll \dim(\text{Sh}(\ml,\fb_{\gts}))$ while $\mts[\mu]$ still non-empty for $p>(n-2)l$.
\begin{thebibliography}{}
\bibitem{Irene}
I.I. Bouw, The p-Rank of Ramified Covers of Curves Composition Mathematica, 126, 295–322 (2001). https://doi.org/10.1023/A:1017513122376

\bibitem{Moonen algorithm}
B. Moonen,
Computing discrete invariants of varieties in positive characteristic. I. Ekedahl-Oort types of curves. (2022).
arXiv:2202.08050

\bibitem{Gonzalez}
J. Gonzales,
Hasse-Witt matrices for the Fermat curves of prime degree,
Tohoku Mathematical Journal, Second Series, 49 (2), 149-163, 1997.
10.2748/tmj/1178225144

\bibitem{Moonen EO type formula}
B. Moonen,
Serre-Tate theory for moduli spaces of PEL type.
Annales scientifiques de l'École Normale Supérieure, Serie 4, Volume 37 (2004) no. 2, pp. 223-269. doi : 10.1016/j.ansens.2003.04.004.

\bibitem{Moonen group scheme}
B. Moonen,
Group Schemes with Additional Structures and Weyl Group Cosets. In: Faber, C., van der Geer, G., Oort, F. (eds) Moduli of Abelian Varieties.
Progress in Mathematics, vol 195 (2001). Birkhäuser, Basel. https://doi.org/10.1007/978-3-0348-8303-0-10

\bibitem{Dual basis}
J. Stienstra, M. van der Put,
On p-adic monodromy, Mathematische Zeitschrift 208.2 (1991): 309-326. http://eudml.org/doc/174319.

\bibitem{Second Paper}
W. Li, E. Mantovan, R. Pries and Y. Tang,
Newton polygons arising from special families of cyclic covers of the projective line, Research in Number Theory, 5, 12 (2019). 10.1007/s40993-018-0149-3.

\bibitem{Moonen dimension formula}
B. Moonen,
A dimension formula for Ekedahl-Oort strata,
Annales de l’institut Fourier 54.3 (2004): 666-698. http://eudml.org/doc/116122.

\bibitem{Moonen Oort}
B. Moonen, F. Oort,
The Torelli locus and special subvarieties.  In Handbook of Moduli: Volume II, pages 549–94.
International Press, Boston, MA, 2013.

\bibitem{clutching argument}
Li, W., Mantovan, E., Pries, R., Tang, Y. (2018). Newton polygon stratification of the Torelli locus in PEL-type Shimura varieties. arXiv: Number Theory.

\bibitem{special family}
Ben Moonen, Special subvarieties arising from families of cyclic covers of the projective line, Doc. Math. 15 (2010),
793–819. MR 2735989

\bibitem{Chevalley}
Paola Frediani, Alessandro Ghigi, Matteo Penegini, Shimura Varieties in the Torelli Locus via Galois Coverings, International Mathematics Research Notices, Volume 2015, Issue 20, 2015, Pages 10595–10623, https://doi.org/10.1093/imrn/rnu272

\bibitem{Wedhorn}
Torsten Wedhorn,
Ordinariness in good reductions of shimura varieties of PEL-type,
Annales Scientifiques de l’École Normale Supérieure,
Volume 32, Issue 5,
1999,
Pages 575-618,
ISSN 0012-9593,
https://doi.org/10.1016/S0012-9593(01)80001-X.

\bibitem{delignmostow}
Deligne, P., Mostow, G.D. Monodromy of hypergeometric functions and non-lattice integral monodromy. Publications Mathématiques de l’Institut des Hautes Scientifiques 63, 5–89 (1986). https://doi.org/10.1007/BF02831622

\bibitem{viehmann-wedhorn}
Viehmann, E., Wedhorn, T. Ekedahl–Oort and Newton strata for Shimura varieties of PEL type. Math. Ann. 356, 1493–1550 (2013). https://doi.org/10.1007/s00208-012-0892-z

\bibitem{Kottwitz}
Kottwitz, Robert E. Isocrystals with additional structure. Compositio Mathematica, Volume 56 (1985) no. 2, pp. 201-220. 

\bibitem{Rapoport-RIcharts}
 M. Rapoport and M. Richartz, On the classification and specialization of F-isocrystals with additional structure, Compositio Math. 103 (1996), no. 2, 153–181. MR 1411570

 \bibitem{moduli functor G cover}
 Achter, J.D., Pries, R. The integral monodromy of hyperelliptic and trielliptic curves. Math. Ann. 338, 187–206 (2007). https://doi.org/10.1007/s00208-006-0072-0
\end{thebibliography}
\end{document}