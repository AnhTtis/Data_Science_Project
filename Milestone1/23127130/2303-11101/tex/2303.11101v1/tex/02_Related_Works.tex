\section{Related Works}\label{sec:related_works}

\setlength{\tabcolsep}{1.6mm}{
\renewcommand\arraystretch{1.1}
\begin{table}[ht]
  \centering
  \scalebox{0.9}{
  \begin{tabular}{llcccc}
    \toprule
    &\multirow{2}*{Methods} & \multirow{2}*{Sal.} &   \multicolumn{2}{c}{VOC} & MS~COCO \\
    \cmidrule(r){4-5}\cmidrule(r){6-6}
    &&&\texttt{val}&\texttt{test}&\texttt{val}\\
    \hline
    \multirow{13}*{\rotatebox{90}{ResNet-50}}
    &IRN~\cite{irn}          \tiny{CVPR'19}     &              & 63.5       & 64.8          & 42.0  \\
    &LayerCAM~\cite{layercam}\tiny{TIP'21}      &              & 63.0       & 64.5          & -     \\
    &AdvCAM~\cite{advcam}    \tiny{CVPR'21}     &              & 68.1       & 68.0          & 44.2  \\
    &RIB~\cite{rib}          \tiny{NeurIPS'21}  &              & 68.3       & 68.6          & 44.2  \\
    &ReCAM~\cite{recam}      \tiny{CVPR'22}     &              & 68.5       & 68.4          & 42.9  \\
    % \rowcolor{Gray}
    &\cellcolor{Gray}IRN+\texttt{LPCAM}    &\cellcolor{Gray} & \cellcolor{Gray}68.6    & \cellcolor{Gray}68.7      & \cellcolor{Gray}44.5  \\
    &SIPE~\cite{sipe}        \tiny{CVPR'22}     &              & 68.8       & 69.7          & 40.6  \\
    &OOD~\cite{ood}+Adv      \tiny{CVPR'22}     &              & 69.8       & 69.9          & -     \\
    &AMN~\cite{amn}          \tiny{CVPR'22}     &              & 69.5       & 69.6          & 44.7  \\
    &\cellcolor{Gray}AMN+\texttt{LPCAM}    &\cellcolor{Gray} & \cellcolor{Gray}70.1    &\cellcolor{Gray} 70.4      & \cellcolor{Gray}45.5  \\ 
    &ESOL~\cite{esol}        \tiny{NeurIPS'22}  &              & 69.9$^*$   & 69.3$^*$      & 42.6  \\
    &CLIMS~\cite{clims}      \tiny{CVPR'22}     &              & 70.4$^*$   & 70.0$^*$      & -     \\
    &EDAM~\cite{edam}        \tiny{CVPR'21}     &\checkmark    & 70.9$^*$   & 71.8$^*$      & -     \\
    &\cellcolor{Gray}EDAM+\texttt{LPCAM}  &\cellcolor{Gray}\checkmark & \cellcolor{Gray}71.8$^*$ &\cellcolor{Gray} 72.1$^*$& \cellcolor{Gray}42.1\\
    \hline
    \multirow{9}*{\rotatebox{90}{WideResNet-38}}
    &Spatial-BCE~\cite{sbce} \tiny{ECCV'22}     &              & 70.0       & 71.3      & 35.2  \\
    &BDM~\cite{bdm}          \tiny{ACMMM'22}    &\checkmark    & 71.0       & 71.0      & 36.7  \\ 
    &RCA~\cite{rca}+OOA      \tiny{CVPR'22}     &\checkmark    & 71.1       & 71.6      & 35.7  \\
    &RCA~\cite{rca}+EPS      \tiny{CVPR'22}     &\checkmark    & 72.2       & 72.8      & 36.8  \\
    &HGNN~\cite{hgnn}        \tiny{ACMMM'22}    &\checkmark         & 70.5$^*$   & 71.0$^*$  & 34.5  \\ 
    &EPS~\cite{eps}          \tiny{CVPR'21}     &\checkmark         & 70.9$^*$   & 70.8$^*$  & -     \\
    &RPIM~\cite{rpim}        \tiny{ACMMM'22}    &\checkmark         & 71.4$^*$   & 71.4$^*$  & -     \\ 
    &L2G~\cite{l2g}          \tiny{CVPR'22}     &\checkmark         & 72.1$^*$   & 71.7$^*$  & 44.2  \\
    \hline
    \multirow{2}*{\rotatebox{90}{\small{DeiT-S}}}
    &MCTformer~\cite{mctformer}    \tiny{CVPR'22}     &                 & 71.9$^{\dag}$  & 71.6$^{\dag}$   & 42.0  \\
    &\cellcolor{Gray}MCTformer+\texttt{LPCAM}      &\cellcolor{Gray} & \cellcolor{Gray}72.6$^{\dag}$  & \cellcolor{Gray}72.4$^{\dag}$  &\cellcolor{Gray} 42.8 \\
    \bottomrule
  \end{tabular}}
  \vspace{-2mm}
  \caption{The mIoU results (\%) based on DeepLabV2 on VOC and MS~COCO. The side column shows three backbones of multi-label classification model. ``Sal.'' denotes using saliency maps. * denotes the segmentation model is pre-trained on MS~COCO. $^\dag$ denotes the segmentation model is pre-trained on VOC.
  }
  \vspace{-6mm}
  \label{table_related}
\end{table}
}



\subsection{Self-Supervised Learning}

After Oord \etal\cite{oord2018representation} proposed an InfoNCE loss, contrastive learning algorithms began to show remarkable improvements in representation learning\cite{chen2020simple, he2020momentum, grill2020bootstrap, caron2020unsupervised, chen2021exploring, li2021efficient, caron2021emerging}. 
While a large-scale open-set enhances the generalization of learned representation \cite{jaiswal2020survey, tian2021divide, cole2022does}, recent literature has pointed out the distribution mismatch between pretraining and fine-tuning datasets \cite{el2021large, tian2021divide, goyal2021self}.
Particularly, El \etal\cite{el2021large} claimed that pretraining on ImageNet may not always be effective on the target task from different domains.
Tian \etal\cite{tian2021divide} found that pretraining with uncurated data, a more realistic scenario, deteriorates the target task performance.
Although the motivations coincide with ours, their proposed methods are focused on a single data scheme: a denoising autoencoder\cite{el2021large} only for fine-grained dataset, and distillation from expert models\cite{tian2021divide} or a novel architecture\cite{goyal2021self} for uncurated open-sets.
In contrast, we propose an explicit sampling strategy from an open-set, which becomes more effective by augmenting fine-grained dataset with well-matched samples, as well as achieving robustness to the distribution discrepancy or curation of the open-set.


\subsection{Coreset Selection from Open-Set}
In an OpenSSL problem, we denote an open-set as the additional unlabeled pretraining set that includes instances either from relevant or irrelevant domain to target dataset. The assumption of available open-set is also common in other research fields, such as open-set recognition\cite{scheirer2012toward, bendale2016towards, chen2021adversarial, vaze2021open}, webly supervised learning\cite{chen2015webly, li2020mopro, sun2021webly}, open-set\cite{oliver2018realistic, chen2020semi, saito2021openmatch,killamsetty2021retrieve} or open-world\cite{bendale2015towards, cao2021open, boult2019learning} semi-supervised learning, and open-set annotation\cite{ning2022active}, although its detailed meaning varies in each field. We summarize the details in Table\,\ref{tab:related_works}.
Especially, our OpenSSL task is to recognize coreset from a large-scale open-set, without exploiting any label information.
From this perspective, recent coreset selection approaches give us a good intuition.
Existing studies find the representative subset of the unlabeled set for active selection\cite{wei2015submodularity, coreset}, or find the subset of current task data to avoid catastrophic forgetting for continual learning\cite{aljundi2019gradient, yoon2021online, tiwari2022gcr}.
In the meantime, several works on self-supervised learning have developed a novel loss function that leverages hard negative samples, \ie, hard negative mining\cite{robinson2020contrastive, wang2021understanding}, which shares similar concepts to our coreset.
Our problem setup further requires an effective algorithm that takes into consideration of the distribution discrepancy in the open-set.