\subsection{Qualitative Evaluation}
\label{subsec:analysis_of_simcore}

\begin{figure}[!t]
    \centering
    \begin{subfigure}[b]{0.48\linewidth}
    \begin{subfigure}[b]{0.32\linewidth}
    \centering
    \includegraphics[width=\linewidth]{figs/feat_dist/os_aircraft.pdf}
    \end{subfigure}
    \begin{subfigure}[b]{0.32\linewidth}
    \centering
    \includegraphics[width=\linewidth]{figs/feat_dist/x_aircraft.pdf}
    \end{subfigure}
    \begin{subfigure}[b]{0.32\linewidth}
    \centering
    \includegraphics[width=\linewidth]{figs/feat_dist/simcore_aircraft.pdf}
    \end{subfigure}
    \caption{Aircraft}
    \end{subfigure}
    %
    \hfill
    %
    \begin{subfigure}[b]{0.48\linewidth}
    \begin{subfigure}[b]{0.32\linewidth}
    \centering
    \includegraphics[width=\linewidth]{figs/feat_dist/os_cars.pdf}
    \end{subfigure}
    \begin{subfigure}[b]{0.32\linewidth}
    \centering
    \includegraphics[width=\linewidth]{figs/feat_dist/x_cars.pdf}
    \end{subfigure}
    \begin{subfigure}[b]{0.32\linewidth}
    \centering
    \includegraphics[width=\linewidth]{figs/feat_dist/simcore_cars.pdf}
    \end{subfigure}
    \caption{Cars}
    \end{subfigure}

    \begin{subfigure}[b]{0.48\linewidth}
    \begin{subfigure}[b]{0.32\linewidth}
    \centering
    \includegraphics[width=\linewidth]{figs/feat_dist/os_pets.pdf}
    \end{subfigure}
    \begin{subfigure}[b]{0.32\linewidth}
    \centering
    \includegraphics[width=\linewidth]{figs/feat_dist/x_pets.pdf}
    \end{subfigure}
    \begin{subfigure}[b]{0.32\linewidth}
    \centering
    \includegraphics[width=\linewidth]{figs/feat_dist/simcore_pets.pdf}
    \end{subfigure}
    \caption{Pet}
    \end{subfigure}
    %
    \hfill
    %
    \begin{subfigure}[b]{0.48\linewidth}
    \begin{subfigure}[b]{0.32\linewidth}
    \centering
    \includegraphics[width=\linewidth]{figs/feat_dist/os_cub.pdf}
    \end{subfigure}
    \begin{subfigure}[b]{0.32\linewidth}
    \centering
    \includegraphics[width=\linewidth]{figs/feat_dist/x_cub.pdf}
    \end{subfigure}
    \begin{subfigure}[b]{0.32\linewidth}
    \centering
    \includegraphics[width=\linewidth]{figs/feat_dist/simcore_cub.pdf}
    \end{subfigure}
    \caption{Birds}
    \end{subfigure}
    \vspace{-5pt}
    \caption{Feature distribution map of \textit{OS}\,(left), $X$\,(middle), and the coreset sampled by SimCore\,(right).}
    \label{fig:feat_dist}
\end{figure}

\begin{figure*}[!t]
\begin{subfigure}[b]{0.49\textwidth}
    \raggedleft
    \includegraphics[width=\textwidth]{figs/pet_k1_total.pdf}
\end{subfigure}
\tikz{\draw[-,black, densely dashed, thick](0,1.05) -- (0, 4.55);}
\hfill
\begin{subfigure}[b]{0.49\textwidth}
    \raggedright
    \includegraphics[width=\textwidth]{figs/birds_k1_total.pdf}
\end{subfigure}
\hfill
\begin{subfigure}[b]{0.49\textwidth}
    \vspace{5pt}
    \raggedleft
    \includegraphics[width=\textwidth]{figs/pet_k100_total.pdf}
    \caption{$\textit{OS}_{\text{SimCore}}$ with $X$ = Pet and \{$k$ = 1\,(top), $k$ = 100\,(bottom)\}}
    \label{fig:coreset_vis_a}
\end{subfigure}
\tikz{\draw[-,black, densely dashed, thick](0, 1.55) -- (0, -2.55);}
\hfill
\begin{subfigure}[b]{0.49\textwidth}
    \vspace{5pt}
    \raggedright
    \includegraphics[width=\textwidth]{figs/birds_k100_total.pdf}
    \caption{$\textit{OS}_\text{SimCore}$ with $X$ = Birds and \{$k$ = 1\,(top), $k$ = 100\,(bottom)\}}
    \label{fig:coreset_vis_b}
\end{subfigure}
\caption{Visualization of the sampled coreset by SimCore method. We plotted histograms for the top-20 classes with the largest number of samples and visualized one example image per top-12 classes. We highlighted with orange the coreset classes that look dissimilar to the target data. $k$ is the number of centroids in $k$-means clustering to reduce the complexity of SimCore. The x-axis and y-axis of histograms denote the class index and the number of samples, respectively.}
\label{fig:coreset_vis}
\vspace{-5pt}
\end{figure*}


\paragraph{Feature distribution analysis:}
To analyze how the SimCore algorithm samples the coreset in the latent space, we visualized the feature distribution on a unit ring by Gaussian kernel density estimation in $\mathbb{R}^2$\,\cite{wang2020understanding} (implementation details in Appendix\,\ref{appx:implementation_details_feat_dist}).
The results in Figure\,\ref{fig:feat_dist} are interesting, as SimCore actually samples the instances that are closely embedded to the target data. 
In addition, through a comparison of the occupied areas of \textit{OS} and $X$, we confirmed the distribution similarity of the open-set to each target dataset, indicating the sampling ratios by SimCore in Table\,\ref{tab:main_exp} are reasonable.


\vspace{-12pt}
\paragraph{Coreset visualization:}
We visualized which instances from the open-set are actually sampled by our SimCore algorithm. To this end, we displayed in Figure\,\ref{fig:coreset_vis} the ground-truth labels of the coreset samples when the target dataset is Pet or Birds. For comparison, we also displayed the coreset by SimCore with $k=1$, using a single centroid.
We used the open-set as ImageNet, so the ground-truth labels of coreset samples correspond to the ImageNet classes.
Note that Pet contains 12 cat breeds and 25 dog breeds\cite{pet}, and Birds contains 200 bird species\cite{birds}. 

For the Pet dataset (Figure\,\ref{fig:coreset_vis_a}), SimCore with $k=1$ sampled mostly animal images, but included data somewhat irrelevant to cats and dogs. The second most class is giant panda, the third is koala, and the eighth is guenon, a kind of monkey. On the contrary, SimCore with $k=100$ sampled mostly cat or dog images, with up to top-20 classes, each being a breed of either cat or dog. Interestingly, eight out of the top-10 classes were those that overlap with the Pet class labels, such as Persian cat, Saint Bernard, etc. 

For the Birds dataset (Figure\,\ref{fig:coreset_vis_b}), SimCore with $k=1$ sampled a lot of irrelevant images, such as French horn, trombone, bullet train, admiral, hard disc, maillot, etc. On the contrary, SimCore with $k=100$ sampled only the bird species up to the top-20 classes, including bulbul, chickadee, brambling, bee eater, house finch, goldfinch, etc. 


\begin{table}[!t]
    \small
    \centering
    \addtolength{\tabcolsep}{-3.5pt}
    \renewcommand*{\arraystretch}{1.1}
    \resizebox{\linewidth}{!}{
    \begin{tabular}{l|lccc|lccc}
    \toprule
    & \multicolumn{4}{c|}{framework: \textit{Open-Set Semi-Sup.}} & \multicolumn{4}{c}{framework: \textit{Webly Sup.}} \\
    \hline
    pretrain & method & \!\!Aircraft\!\! & Cars & Birds & method & \!\!Aircraft\!\! & Cars & Birds \\
    \hline
    SimCore & FT\,(50\%) & 73.5 & 80.1 & 57.4 & FT\,(100\%) & 84.3 & \textbf{89.3} & 70.6 \\
    \hline
    \xmark & SelfTrain & 51.9 & 55.5 & 35.7 & CoTeach & 79.3 & 51.7 & 70.4 \\
    SimCore & SelfTrain & \textbf{78.1} & \textbf{81.3} & \textbf{59.1} & CoTeach & \textbf{89.8} & 57.0 & \textbf{78.9} \\
    \hline
    \xmark & OpenMatch & 70.1 & 70.2 & 52.3 & DivideMix & 82.2 & 54.4 & 74.5 \\
    SimCore & OpenMatch & \textbf{83.5} & \textbf{89.5} & \textbf{66.4} & DivideMix & \textbf{86.5} & 56.5 & \textbf{80.0} \\
    \bottomrule
    \end{tabular}
    }
    \vspace{-5pt}
    \caption{Comparisons with Self-Training\,\cite{su2021realistic} and OpenMatch\,\cite{saito2021openmatch} in the OpenSemi framework, and Co-teaching\,\cite{han2018co} and DivideMix\,\cite{li2020dividemix} in the WeblySup framework.}
    \label{tab:opensemi_weblysup}
\end{table}
