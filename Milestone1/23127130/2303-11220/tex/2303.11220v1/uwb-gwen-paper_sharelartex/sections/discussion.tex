% !TeX root = ../main.tex

\section{Discussion}\label{sec:discussion}
\noindent 
Our measurements have shown that the advertised accuracy of $\pm$\SI{10}{\centi\metre} 
is not reproducible. Even when both \ac{uwb} antennas were directed at each other a \ac{mae} of more than \SI{10}{\centi\metre} is normal. Additionally, all smartphones are unreliable in certain positions and environments. They failed to measure the distance at positions directed away from the remote device. 
However, the overall performance is promising: The \ac{mae} is below \SI{20}{\centi\meter} and at shorter distances, the devices succeeded to measure the distance at every position we evaluated. 
In this section, we detail how these results may affect \ac{uwb}-based entry systems and which considerations have to be taken when implementing such systems. 

\subsection{Overall performance}

\paragraph*{Accuracy and Reliability}
Our measurements identified that internal components of a smartphone shield the low power \ac{uwb} signals, resulting in strong directional differences. In a real-world scenario, this influences the overall user experience: Depending on the position of a smartphone in a pocket or bag, the device is not able to perform \ac{uwb} ranging. 
If successful, measurements can have an increased distance by several meters because the devices rely on reflected signals, which travel a longer path. 
These issues can delay the time until a distance-based action is triggered, e.g., when approaching \ac{uwb}--cars with iPhone support have the ability to activate their headlights in order to welcome the driver~\cite{incExploreUWBbasedCar}.
In addition, future smart home interactions might fail to recognize that the user is present and as a result turn off lights or heating in these areas. 
At short distances we measured a high reliability of all devices and a \ac{mae} of less than \SI{20}{\centi\meter}.


\paragraph*{Software support}
All smartphones, which we evaluated offer an API to perform \ac{uwb} ranging to other smartphones or \ac{iot} devices. Apple was the first manufacturer to open up their ranging system to \ac{fira} compatible chipsets (e.g., Qorvo and NXP)\cite{NearbyInteractionApple}. Google also integrated an Android-wide API for \ac{uwb}, which should also be compatible with \ac{fira} chips~\cite{googleinc.UltrawidebandUWBCommunication}. Unfortunately, the documentation lacks important parts and Google does not disclose how to perform ranging with non-Android devices. Moreover, the stability of the API varies largely. To start ranging successfully it was required to perform around ten attempts on all Android devices and finally, our Google Pixel devices suddenly failed to perform \ac{uwb} ranging completely. 

Overall, the user experience may be weakened, but the mean measured distances over ten or more consecutive measurements have proven to normalize the results. If the software issues are resolved, the current consumer-market implementations are in a good shape to be used for distance estimations in most smart home scenarios and for secure entry systems, if some precautions are made (see \cref{sec:discussion_recommendations}).

\subsection{Comparison to previous research}\label{ssec:comparison_related_work}


% Things to compare: Error, Accuracy, Standard deviation, Antenna directionality
% Papers to look at: 

Several researchers analyzed the accuracy and ranging errors of different \ac{uwb} devices. None of these devices were actual consumer hardware, most of them focused on the Decawave (now Qorvo) devices. 
In this section, we compare the results of our research on consumer hardware and the Qorvo DW3000 with previous results. 

% UWB Ranging Accuracy by Malajner 2015 - DW1000
% Measurements in LoS testing different configurations of the channel, data rate, preamble, prf  
% Distances between 0 and 30m (Outdoors)
% Root mean square Error: 28.19cm  - 54.04cm non-calibrated | 4.34cm - 14.03cm in calibrated mode 

\paragraph{DW1000} Malajner et al.~\cite{malajnerUWBRangingAccuracy2015}  analyzed the ranging accuracy in \ac{los} of the DW1000, the predecessor of the DW3000. 
Depending on the configuration of the chip, they identified a \ac{rmse} between \SI{28.19}{\centi\meter} and \SI{54.04}{\centi\meter} for uncalibrated devices. For calibrated devices, the \ac{rmse} was lower at a range of \SIrange{4.34}{14.04}{\centi\meter}. 

The \ac{rmse} for an uncalibrated DW3000 over all our measurements was \SI{16.03}{\centi\meter}, which includes all positions and all environments. For comparison, the Google Pixel has a \ac{rmse} of \SI{14.98}{\centi\meter},the Samsung Galaxy S21 Ultra \SI{15.65}{\centi\meter}, and the iPhone \SI{17.68}{\centi\meter}. All of our results are in a similar range when compared to the older DW1000.


% Comparing Decawave and Bespoon UWB location systems - DW1000, Bespoon phone
% Testing LoS up to 100m (Outdoors)
% Standard deviation LoS: Decawave 5.5cm, Bespoon 11cm 
% Mean error: Decawave: 2.6cm, Bespoon: 0.35cm
% NLoS 0 - 14m  (Office)
% Mean Error: Decawave 34cm, BeSpoon: 50cm
% Std Deviation: Decawave 35cm, BeSpoon 61cm

\paragraph{BeSpoon phone} Jimenez et al.~\cite{jimenezComparingDecawaveBespoon2016} compared the BeSpoon phone\footnote{The BeSpoon phone was a prototype hardware demonstrating that it is feasible to implement \ac{uwb} in smartphones.} with the DW1000.  
In \ac{los} measurements, the BeSpoon reached a mean error of \SI{2.6}{\centi\meter} and a \ac{sd} of \SI{11}{\centi\meter}. In \ac{nlos}, the mean error was \SI{50}{\centi\meter} and the \ac{sd} \SI{61}{\centi\meter}. Using the mean error is non-optimal as the errors can be positive or negative, which is why we measure the \ac{mae} in this work. When comparing the mean standard deviation over all our measurements, the DW3000 had \SI{14.03}{\centi\meter}, the Pixel \SI{13.04}{\centi\meter}, the Galaxy S21 Ultra \SI{13.63}{\centi\meter}, and the iPhone \SI{15.02}{\centi\meter}. We can see that modern consumer hardware operates at similar levels, as our measurements include both \ac{nlos} (blocked by smartphone internals) and \ac{los} scenarios. 

% On the Energy Consumption and Ranging Accuracy of Ultra-Wideband Physical Interfaces by Fluearto - DW3000, 3dB Access
% LoS up to 51m
% Mean error: 3dB access 4.8cm, Decawave: 4.1cm
% Standard deviation: 3db 8.4cm, Decawave: 3.2cm
% NLoS different obstructions (walls, body, pillars)
% Mean error: 3db 62.5cm, Decawave 34cm 
% SD: 3db 3db 104cm, 35cm 

\paragraph{DW3000 and 3dB Access} Fluearto et al. \cite{flueratoruEnergyConsumptionRanging2020} compared the DW3000 with a \ac{uwb}--\ac{lrp} development kit from 3db Access. In \ac{los}, the mean error of the 3dB Access device was \SI{4.8}{\centi\meter} whereas the mean error of the DW3000 was \SI{4.1}{\centi\meter}. They measured \acp{sd} of \SI{8.4}{\centi\meter} and \SI{4.1}{\centi\meter}, respectively. In \ac{nlos}, the mean error increased to \SI{62.5}{\centi\meter} and \SI{34}{\centi\meter}. The \ac{sd} also increased to \SI{104}{\centi\meter} and \SI{35}{\centi\meter}. Our results are in the same range as the ones measured by Fluearto et al.


% Experimental Evaluation of IEEE 802.15.4z UWB Ranging Performance under Interference - DW3000
% Moving DW3000 mounted on a helmet in a storage hall. LoS and NLoS conditions. 
% Special blocking walls 
% Errors are generally less than 50cm in all orientations 
\paragraph{DW3000 in motion} Tiemann et al. \cite{tiemannExperimentalEvaluationIEEE2022a} performed an experimental evaluation of the DW3000 while walking through a storage hall with many obstructions. Overall, the measurement errors are less than \SI{50}{\centi\meter} throughout the whole measurement series.
% NOTE: Die folgenden beiden Sätze sind Beispiele wie man das noch vergleichen könnte mit unseren Ergebnissen.
In our measurement campaign, the DW3000 performed similarly well with only a few larger outliers.
Our measurements on consumer hardware, however, had larger outliers reaching multiple meters.
 


\subsection{Security implications}

The already mentioned shielding combined with reflections resulted in an enlarged measured distance in different environments. At a distance of \SI{5}{\metre}, the measured distances on all smartphones resulted in range of \SI{0}{\metre} to \SI{7.79}{\metre}. An enlargement of the measured distance can affect the convenience of a \ac{pke} system, but not the security. 

The Samsung devices show a lower \ac{mae} and better accuracy than the iPhones, but the Samsung devices repeatedly measured a distance of \SI{0}{\meter} in the Lab environment. Such strong reductions can affect the security of \ac{uwb}-based entry systems. 
The Apple iPhone and Google Pixel also created distance reductions of up to \SI{-3}{\meter} when ranging outside. This might not be enough to provoke the unlock action from a \SI{5}{\meter} distance. However, these reductions are unexpected, since the devices had no interference and should, therefore, be able to measure accurately.  

None of these distance reductions have been consistent and happened as outliers. Following our recommendations in \cref{sec:discussion_recommendations} mitigates these problems. 

Previous research presented an attack on iPhones in a \ac{los} setup that resulted in reproducible distance reductions~\cite{279984}. It is to be researched if the attack's success-rate increases when both devices are no longer in \ac{los}, use different orientations as used in our experimental setup, or operate in environments with strong interferences. 
Especially on the iPhone, we discovered that the \ac{sd} increases when the device is no longer directed to its peer (see \cref{fig:lab_polar}).

\subsection{Recommendations for implementing a secure passive keyless entry system}\label{sec:discussion_recommendations}

\ac{uwb} is the most promising technology when it comes to secure distance measurements at a consumer hardware level. Known attacks only have a low success rate and in general, the produced ranging errors are less than \SI{20}{\centi\metre}. Nevertheless, there are important things to consider when implementing such a system. 

Since this technology directly interferes with the real world by sending pulse-based signals, any obstructions, the positioning of the devices, and other wireless devices can influence the accuracy of distance measurements. This has been demonstrated by our study and previous work~\cite{tiemannExperimentalEvaluationIEEE2022a}.

To implement a secure entry system we give four recommendations: 
\subsubsection{Two-way ranging}
The implementation should always use the IEEE 802.15.4z \acf{ds-twr} ranging algorithm. With \ac{ds-twr} ranging, potential clock drifts are eliminated, and both sides calculate a distance measurement without the necessity to send the measured distance from one device to the other.
If a smartphone sends a command to \textit{open} to a smart lock, the measured distance should also be validated by the lock before granting access. 

\subsubsection{Use mean values}
Before a trusted distance can be estimated, at least ten to $15$ ranging cycles should be performed. 
Using a sliding window of ten measurements it is possible to calculate a mean distance over all measured distances. Only the mean value should be used for the decision of granting access or not. In a security context, the device should never perform the distance-based action on the first measurement that is in the desired range. 
Our study has shown that accidental distance reductions of \SI{3}{\meter} to \SI{5}{\meter} can happen. According to our tests, in most scenarios, ten measurements can be performed in less than \SI{2}{\second}. 

\subsubsection{Drop clearly wrong measurements}
When placing two devices right next to each other it is possible to create negative values for a distance measurement. These values are very rare and should not be considered in any parts of the algorithm. 

\subsubsection{Detect potential attacks}
The only currently known attack on \ac{uwb} \ac{hrp} IEEE 802.15.4z systems has a low success rate and cannot be controlled accurately~\cite{279984}. An iPhone under attack has reported several measurements with up to \SI{-2}{\meter}. Therefore, negative values should also be counted to detect potential attacks on an \ac{uwb} entry system. 
If reported distances fluctuate strongly or the system detects several strongly negative values, the distance measurement should be suspended, and the user should be informed. 