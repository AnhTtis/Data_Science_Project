% !TeX root = ../main.tex

\section{Conclusion}\label{sec:conclusion}
\noindent 
In our study, we evaluated  \ac{uwb} distance measurements of three modern smartphones with \ac{uwb} and a development kit from Qorvo with a focus on accuracy and reliability. We designed, built and used a novel measurement setup called \ac{gwen} to evaluate changes in the antenna position on performed measurements.

Our results have shown that in most cases the measured distance is less accurate than advertised by the manufacturers\cite{qorvoinc.DW3000DataSheet2020,nxpsemiconductorsSR040UltraWidebandTransceiver2021}. Nevertheless, the devices produce errors in a reasonable range with a mean absolute error of less than \SI{20}{\centi\meter}. The results are on average good enough to perform distance estimation and to decide whether a smart lock or \ac{pkes} of a car should open. 
However, all smartphones shared the weakness that measurements can fail depending on the position of the antenna, e.g., when the antenna no longer faces the direction of the remote device. These issues decrease the user experience and may lower the adoption of people. 
From our results, we derived four recommendations to improve the security of \ac{pke} systems. 
