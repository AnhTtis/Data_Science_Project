% !TeX root = ../main.tex

\section{Introduction}\label{sec:introduction}
\IEEEPARstart{W}{ireless} communication is the upcoming norm of interaction between users' phones and other devices such as cars and IoT devices. Without the need to interact with their phone or any other key, a user will be able to unlock their car when approaching and start it up once inside.
At home, smart devices react to our presence. Smart locks open doors when an authorized person is nearby, and smart thermostats regulate temperature based on the presence of owners and their phones. These are only a few examples of the use cases for wireless communication based on \ac{uwb}, the newest wireless physical layer in smartphones, which allows two devices to perform \ac{tof}-based distance measurements. The first car that supports this feature using \ac{uwb} on iPhones was released in 2022~\cite{bmwagBMWAnnouncesBMW}.
However, while a smart light that does not turn on is merely an annoyance, many of the applications like unlocking doors and cars need to be both reliant and accurate to avoid theft or blocked access for the owners.

Different applications have varying needs when it comes to the properties of distance measurements. In general, we can divide them into active and passive use cases. In an active use case, a person wants to actively measure the distance to a certain point, e.g., to find a lost item. Since users are already interacting with their phones in this scenario, they can try to enhance the distance measurements deliberately by holding the smartphone differently or moving around to get a good signal. This is not the case in passive use cases like \ac{pke} and smart home automation systems, which are the focus of our work. In these systems, the smartphones start interactions automatically and the user should not be required to optimize the measurements, as this would defy the idea of a passive system. Instead, the systems need to rely only on the existing measurements and it is currently unclear how measurements behave when a \ac{uwb} smartphone is positioned sub-optimal.

There are two principal use cases of passive distance measurements, which we consider in this work:

\begin{enumerate}
    \item \textbf{A passive keyless entry system} implemented in a car or a smart lock requires accurate distance measurements with an error of less than \SI{50}{\centi\meter}. If the error is larger, the locks may open unpredictably. This may result in an unauthorized person gaining access. 
    Additionally, when the user is closer than \SI{50}{\centi\meter}, the measurements need to be performed reliably and fail at less than \SI{10}{\percent} of cases to ensure a good user experience.
    
    \item \textbf{A smart thermostat, speaker, or light} can offer the ability to change its state depending on the presence of a person in the room. These systems can rely on measurements with accuracy in the order of meters. However, they require reliable measurements at distances of \SIrange{3}{10}{\meter}, because stopping the music or turning off the heating while the person is still present can be disruptive to the user experience. 
To achieve high reliability, good software support for apps on a smartphone is essential. Smart home apps need to be able to connect to devices and start \ac{uwb} ranging seamlessly and reliably without disturbing the user and without failing.
\end{enumerate}


Manufacturers of \ac{uwb} chipsets advertise their devices as producing errors of less than $\pm \SI{10}{\centi\meter}$~\cite{nxpsemiconductorsSR040UltraWidebandTransceiver2021,qorvoinc.DW3000DataSheet2020}. However, at of the time of writing, no publicly accessible evaluation of smartphones with \ac{uwb} capabilities exists. 
We do not know how accurate and reliable \ac{uwb} in a smartphone is.
This raises a couple of issues that need to be addressed: 
\begin{itemize}
    \item \textbf{Reliability and Accuracy of \ac{uwb} technology.} People using \ac{uwb} as a \ac{pke} system need to know if they can rely on accurate measurements. Without this, people will be less likely to adopt the technology and more likely to resist its implementation. A positive evaluation will provide peace of mind to people using a \ac{uwb}-smartphone as a car key.
    \item \textbf{Implementation of UWB measurements.} Manufacturers of IoT devices need to understand how accurately and reliably measurements are reported by smartphones to implement dependable algorithms.
    \item \textbf{Security.} By design, UWB is secure against relay and signal amplification attacks. However, if paired with highly inaccurate and strongly fluctuating measurements the security gain is minimal. It, therefore, needs to be evaluated how current implementations are behaving. Many \ac{pke} systems of cars without \ac{uwb} have been successfully attacked~\cite{staatAnalogPhysicalLayerRelay2022,ibrahimKeyAirHacking2019} and it is questioned if \ac{uwb} \ac{tof} measurements can solve the existing problems.
\end{itemize}

Our measurements and their evaluation address these concerns by providing data and a reproducible testbed to analyze future devices. The goal of this paper is to evaluate \ac{uwb} in smartphones and to analyze if the measurements meet the requirements defined above of \ac{pke} systems and IoT smart home devices. To achieve this, we design and implement a GWEn testbed that allows the rotation of a \ac{dut} by 360º in two axes. A manual on how to build our GWEn testbed is available online and the required software is open-source~\cite{krollmannGWEnGimbalBased2022}. Researchers can assemble it to reproduce our measurements or evaluate new devices.

Utilizing our testbed, we perform reproducible \ac{uwb} measurements with one smartphone each from Apple, Google, and Samsung and a DW3000 development kit from Qorvo. All measurements are performed in three different environments: Outside to measure a ground truth, in our lab that resembles an office environment used to evaluate smart home automation, and in a parking garage to evaluate \ac{pke} in cars. Each environment represents a real-world scenario in which \ac{uwb} can be used for distance measurements. We present our results and cover the differences between each environment. All our measurement data is available online~\cite{heinrich_alexander_2023_7702153}. In addition, we discuss if \ac{uwb} in smartphones can perform well enough to achieve the defined goals in accuracy and reliability, we compare our results against previous work which focused on \ac{uwb} development kits, and we give recommendations for implementing a secure \ac{pke} system.  

% \TODO{Annemarie suggested removing the next paragraph because it reveals too many results. I (aheinrich) like it though, and it's been done like that in the suggested paper. }

Our evaluation reveals the following:
\begin{enumerate}
    \item \ac{uwb} measurements are not reliable. Our measurements show that \ac{uwb} antennas embedded in smartphones are highly directional. This directionality is caused by the internals of a smartphone (battery, main board, display, etc.) which shield the signal in one direction. 
    \item Measurements are accurate enough for \ac{pke} systems. On average the devices measure an error of less than \SI{21}{\centi\meter}. This is independent of the smartphone’s orientation. 
    \item \ac{uwb} measurements produce outliers of several meters, which influences the accuracy of the measurements. However, the overall standard deviation of our measurements is generally less than \SI{25}{\centi\meter}. 
    \item The software integration of \ac{uwb} into the Android OS is unstable, especially on Google Pixel smartphones. Samsung's devices achieve better overall stability, notably when using their internal \ac{uwb} API.
\end{enumerate}


% Sounds like a repetition 
% quick comment from me (Annemarie): if you decide to go with the contributions and not the evaluation, we also added open source/open data as a key contribution in our paper, maybe you also want to do that as well?

Our key contributions are: 
\begin{enumerate}
    \item{We are the first to perform measurements with \ac{uwb} capable Apple, Samsung, and Google smartphones.}
    \item{We design and build \ac{gwen}, a reproducible 3D-printed testbed that can rotate a \ac{dut} in two axes around the \ac{uwb} antenna for full \SI{360}{\degree}.}
    \item{We present and evaluate our measurements on accuracy and reliability to show strengths and weaknesses of \ac{uwb} ranging.}
    \item{We give recommendations to manufacturers to implement secure entry systems.}
    \item All our measurement data, that we evaluate in this paper is available online. 
    \item A manual on how to build a \ac{gwen} testbed and the required software are open-source.
\end{enumerate}

%% Old introduction 


% \IEEEPARstart{I}{n} 2022, BMW released the first car that supports the Digital Car Key 3.0 for \ac{pkes} using \ac{uwb} iPhones~\cite{bmwagBMWAnnouncesBMW} as the digital key.
% This feature allows owners to unlock their car when approaching, without the need to interact with their iPhone. Once inside, the car can be started without requiring a key-fob or unlock-code. 
% %\Ac{uwb} as a new physical layer in smartphones allows performing centimeter accurate distance measurements \cite{malajner\ac{uwb}RangingAccuracy2015,nxpsemiconductorsSR040UltraWidebandTransceiver2021}. This technology should ensure that cars do not open until the owner and their iPhone in the required distance~\cite{appleincExplore\ac{uwb}basedCar}. 
% Android and iOS both offer capabilities to perform \ac{uwb} ranging with \ac{fira} compliant \ac{iot} devices. These capabilities allow new interactions based on the users' distance or presence: e.g., smart locks open when an authorized person wishes to enter, lights can turn on based on the presence of people, and smart thermostats can automatically cool down unused rooms or offices. 

% Most available \ac{pke} and presence detection systems are based on an insecure and inaccurate \ac{rssi}-based distance estimation that measures the signal strength of incoming signals to determine the distance to a key-fob or smartphone. 
% Attackers can manipulate estimated distances by relaying or amplifying the legitimate signal, to subsequently gain access to restricted areas or cause increased energy consumption in smart offices/homes~\cite{ibrahimKeyAirHacking2019}. 
% Even systems based on \ac{ble} or \ac{nfc}, as used by Tesla and many available smart locks, have been demonstrated to be vulnerable to such attacks~\cite{staatAnalogPhysicalLayerRelay2022}.

% The physical layer of \ac{uwb} prevents relay and signal amplification attacks~\cite{singhUWBPulseReordering2019}.
% \Ac{uwb} uses very short pulses on a high bandwidth that allow an accurate time of arrival measurement when the signal is received. A relayed signal cannot be transmitted faster than the speed-of-light, and therefore always adds a delay and increases the distance between the owner and the car. 
% Nevertheless, current research has shown that other attacks are possible, even though they have a very low success rate of reducing the distance in only $4\%$ of all measurements~\cite{279984}. 

% \subsection{Motivation}
% Since phones now become a replacement for the default car key, the accuracy in distance measurement of these systems is of utmost importance. Many \ac{uwb} accuracy evaluations are done using \ac{los} optimal conditions between both peers, rely on \ac{uwb} development kits, and do not consider positioning of the devices~\cite{jimenezComparingDecawaveBespoon2016,jimenezruizComparingUbisenseBeSpoon2017,flueratoruEnergyConsumptionRanging2020,flueratoruHighAccuracyRangingLocalization2022}.
% Since \ac{uwb} has moved to the consumer market, these conditions cannot be met in the real-world. A phone can be positioned in different orientations when placed inside a bag or pocket. 
% Therefore, it is important to evaluate the accuracy of the currently available \ac{uwb} phones to identify which measurement errors are to be expected and how manufacturers have to design their system to ensure that the phone can measure the distance reliably and securely.
% Theoretical studies have shown that multipath effects can influence \ac{uwb} measurements and lead to distance reductions~\cite{singhSecurityAnalysisIEEE2021a}. If handled incorrectly, such reductions may inadvertently unlock a car and lead to increased energy consumption or misbehavior of \ac{iot} devices. 
% It is to be evaluated if implementations in consumer smartphones are affected by this flaw. 

% Manufacturers of \ac{uwb} chipsets promise an accuracy with errors of less than $\pm$\SI{10}{\centi\meter}~\cite{nxpsemiconductorsSR040UltraWidebandTransceiver2021,qorvoinc.DW3000DataSheet2020}. So far these promises have not been validated for consumer devices. 
% Our goal is to provide a first evaluation of the accuracy of \ac{uwb} capable smartphones, ensuring that already deployed systems cannot be manipulated and aiding manufacturers to build a secure and reliable system. 

% \subsection{Contributions}
% Our key contributions are: 
% \begin{enumerate}
%     \item{We are the first to perform accuracy measurements with \ac{uwb} capable Apple, Samsung, and Google smartphones.}
%     \item{We build \ac{gwen}, a reproducible 3D-printed testbed that can rotate a \ac{dut} in two axes around the \ac{uwb} antenna for full \SI{360}{\degree}.}
%     \item{We present and evaluate our measurements to show strengths and weaknesses of \ac{uwb} ranging.}
%     \item{We give recommendations for manufacturers to implement secure entry systems.}
% \end{enumerate}

% All measurements and data collections with \ac{gwen} are fully automated and allow reproducible tests for a range of devices. 
% We measure our test devices in three different environments, in two distances and by performing a full \SI{360}{\degree} rotation around two axes. 
% All our presented measurement data is available open-source, and we provide a manual to construct \ac{gwen} by using a 3D-printer and off-the-shelf parts~\cite{krollmannGWEnGimbalBased2022}. 

% This paper is structured as follows: \Cref{sec:background} gives a brief introduction into the topic of \ac{uwb} ranging and related work. \Cref{sec:gwen} introduces \ac{gwen}, demonstrates its workings, and how we are able to make measurements reproducible for other researchers. In \cref{sec:eval_gwen} we evaluate \ac{gwen} and demonstrate that its measurements are reliable and introduce only minimal interference. In \cref{sec:experimental_setup} we explain the setup for all our experiments. \Cref{sec:results} covers the results of our accuracy measurements. In \cref{sec:discussion} we discuss these results and conclude the paper in \cref{sec:conclusion}. 