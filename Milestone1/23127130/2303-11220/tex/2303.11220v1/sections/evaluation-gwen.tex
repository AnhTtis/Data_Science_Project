% !TeX root = ../main.tex

\section{Evaluation of GWEn}\label{sec:eval_gwen}
\noindent 
As described in \cref{ssec:gwen_manufacturing}, a 1 cm thick \ac{pla} wall has a negligible effect on the signal.
In addition to the parts printed from \ac{pla}, \ac{gwen} also contains motors, ball bearings and shafts.
These parts consist of metal that interferes with the signal between the two devices.
This is especially the case when the tower is directly between the two antennas.
Therefore, before evaluating the data, it must be ensured that the measurement is not affected by \ac{gwen}'s test setup and procedure.

\subsection{Maximum error}

\begin{figure}[!t]
  \centering
  \begin{subfigure}{.48\linewidth}
    \centering
    \includegraphics[width=.98\linewidth]{figures/eval/gwen/20220620122246_IMG_4130.JPG}
    \caption{\raggedright{} LOS measurement without obstructions at a base rotation of $\theta=\SI{0}{\degree}$.}\label{fig:t1}
  \end{subfigure}%
  \hfill
  \begin{subfigure}{.48\linewidth}
    \centering
    \includegraphics[width=.98\linewidth]{figures/eval/gwen/20220620122440_IMG_4131.JPG}
    \caption{\raggedright{} NLOS measurement with the tower between the devices at a base rotation of $\theta=\SI{0}{\degree}$.}\label{fig:t2}
  \end{subfigure}
  \caption{Exemplary test setup for evaluating the influence of the tower on \ac{gwen}. The real installation also included the wiring, power supplies, and ensuring the correct spacing. This has been omitted in this picture to simplify matters.}\label{fig:gwen_arm_test}
\end{figure}

To evaluate the influence of \ac{gwen} on the recordings, we performed two tests using a pair of DW3000 \ac{uwb} devices. 
First, we placed both devices facing each other with no obstructions between them in clear \ac{los} (see \cref{fig:t1}). 
Second, we rotated the base of \ac{gwen} to $\theta=\SI{180}{\degree}$ such that the tower is now directly between both devices (see \cref{fig:t2}). We also turned the DW3000 again to ensure that they have the same orientation as in the first test.  For both tests, we ensured that the distance between the DW3000 \ac{uwb} devices is equal. 

In each test, we performed $10,000$ \ac{uwb} measurements. 
To assess the influence of \ac{gwen}, we calculated the mean measured distance and the standard deviation for both measurements.
In \ac{los}, without the tower, a mean value of \SI{35}{\centi\metre} with a standard deviation of \SI{1.3}{\centi\metre} was measured. 
With the tower, \SI{36}{\centi\metre} and \SI{1.7}{\centi\metre} were determined, respectively.

% TODO: Falls gewünscht kann ich (fp) mit statistischen Tests überprüfen, ob die Mittelwerte und Varianzen von LoS und NLoS sich statistisch signifikant unterscheiden.
It is evident that the presence of the arm has increased the distance by \SI{1}{\centi\metre}.
Also, the standard deviation increased by \SI{0.4}{\centi\metre}. The expected accuracy and precision of these devices are claimed to be \SI{10}{\centi\metre}~\cite{qorvoinc.DWM3000Qorvo,qorvoinc.DW3000DataSheet2020} and therefore \ac{gwen} does not add a large enough error to disturb our measurements.

\subsection{Reproducibility}
We assess the reproducibility of measurements by using an experimental setup with two DW$3000$.
One board was our \ac{dut} mounted to \ac{gwen} and the other board was mounted on a tripod. Both were placed \SI{50}{\centi\metre} apart. 
\ac{gwen} rotated a full \SI{360}{\degree} at the base ($\theta$) and \SI{180}{\degree} at the arm ($\phi$) at each base position. At each position, the DW$3000$s performed $100$ \ac{uwb} distance measurements.

The same full measurement cycle was then performed three times in succession. 
Subsequent examination of the three records showed that the mean and standard deviation for all records were \SI{0.48}{\meter} and \SI{0.11}{\meter}, respectively.
\Cref{fig:m_comp} shows for each measurement the plot for a full \SI{360}{\degree} rotation of the base $\theta$ with a fixed arm angle of $\phi=\SI{90}{\degree}$.
All graphs have very similar shapes, and we can see similar results when comparing different arm angles. 

\begin{figure}[!t]
    \centering
    \begin{subfigure}{.16\textwidth}
      \centering
      % \includegraphics[width=\linewidth]{figures/eval/gwen/polar_90_T1.pdf}
      \scalebox{0.32}{\input{figures/eval/gwen/polar_90_T1.pgf}}
      \caption{Recording 1}\label{fig:m1}
    \end{subfigure}%
    \begin{subfigure}{.16\textwidth}
      \centering
      % \includegraphics[width=\linewidth]{figures/eval/gwen/polar_90_T2.pdf}
      \scalebox{0.32}{\input{figures/eval/gwen/polar_90_T2.pgf}}
      \caption{Recording 2}\label{fig:m2}\end{subfigure}%
    \begin{subfigure}{.16\textwidth}
      \centering
      % \includegraphics[width=\linewidth]{figures/eval/gwen/polar_90_T3.pdf}
      \scalebox{0.32}{\input{figures/eval/gwen/polar_90_T3.pgf}}
      \caption{Recording 3}\label{fig:m3}
    \end{subfigure}
    \caption{Comparison of three 360° measurements made directly one after the other without changing the setup. Arm at 90°. The arm angle of $\phi=$\SI{90}{\degree} was chosen arbitrarily, other arm angles led to similar results.}\label{fig:m_comp}
\end{figure} 

Finally, we tore down the measurement setup and set it up again one week later. We measured with the same parameters and compared the results. 
\Cref{fig:m_comp_2} shows two plots one from the first recording described above and one from a fourth recording made one week later.
\Cref{fig:r1} and \Cref{fig:r4} both show a \SI{360}{\degree} rotation of the base ($\theta$) with the arm at $\phi=\SI{90}{\degree}$.
The general shape of the two measurements is similar, but we can see some deviations, which could have been caused by minor changes in the environment. 
The mean value is \SI{0.47}{\centi\metre}, \SI{1}{\centi\metre} less than the first measurements, but the standard deviation remained the same.

All tests combined show that \ac{gwen} only marginally influences \ac{uwb} measurements and can therefore be used to evaluate \ac{uwb} capable devices. 

\begin{figure}[!t]
  \centering
  \begin{subfigure}{.16\textwidth}
    \centering
    % \includegraphics[width=\linewidth]{figures/eval/gwen/polar_90_T1.pgf}
    \scalebox{0.32}{\input{figures/eval/gwen/polar_90_T1.pgf}}
    \caption{Recording 1}\label{fig:r1}
  \end{subfigure}%
  \begin{subfigure}{.16\textwidth}
    \centering
    % \includegraphics[width=\linewidth]{figures/eval/gwen/polar_90_T4.pgf}
    \scalebox{0.32}{\input{figures/eval/gwen/polar_90_T4.pgf}}
    \caption{Recording 4}\label{fig:r4}
  \end{subfigure}
  \caption{Comparison of two \SI{360}{\degree} base rotation ($\theta$) measurements made one week apart. With arm rotation at $\phi=\SI{90}{\degree}$.}\label{fig:m_comp_2}
\end{figure}
