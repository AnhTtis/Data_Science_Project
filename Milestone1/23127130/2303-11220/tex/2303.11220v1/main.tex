\documentclass[journal]{IEEEtran}
\usepackage{cite}
\usepackage[inline]{enumitem}
\usepackage{amsmath,amsfonts}
\usepackage{stix}
\usepackage{algorithmic}
\usepackage{graphicx}
\usepackage{array}
\usepackage{textcomp}
\usepackage{stfloats}
\usepackage{url}
\usepackage{verbatim}
\usepackage{textcomp}
\usepackage[hidelinks]{hyperref}
\usepackage[noabbrev, capitalize]{cleveref}
\usepackage[detect-all]{siunitx}
\usepackage{tabularx}
\usepackage{booktabs}
\usepackage{layouts}
\usepackage{subcaption}
\usepackage[numbers]{natbib}
\usepackage[nolist,nohyperlinks]{acronym}

% Break Urls correctly 
\def\UrlBreaks{\do\/\do-}
% \newcommand\BIBentryALTinterwordstretchfactor{0.1}

% The preceding line is only needed to identify funding in the first footnote. If that is unneeded, please comment it out.

% for subfigures
\usepackage{caption}
\usepackage{subcaption}

% for includesvg
\usepackage{svg}

% for tikzpicture
\usepackage{tikz}
\usetikzlibrary{calc}
\usetikzlibrary{decorations.pathreplacing,calligraphy}
\usetikzlibrary{shapes.callouts}
\usetikzlibrary{patterns,angles,quotes}
\usetikzlibrary{dsp,chains}
\usepackage{tikzsymbols}
\definecolor{seemoo_organge}{RGB}{206, 101, 0}
% for sequencediagram
\usepackage{pgf-umlsd}

\hyphenation{op-tical net-works semi-conduc-tor IEEE-Xplore}
\def\BibTeX{{\rm B\kern-.05em{\sc i\kern-.025em b}\kern-.08em
    T\kern-.1667em\lower.7ex\hbox{E}\kern-.125emX}}
\usepackage{balance}


\renewcommand{\mess}[4][0]{
  \stepcounter{seqlevel}
  \path
  (#2)+(0,-\theseqlevel*\unitfactor-0.7*\unitfactor) node (mess from) {};
  \addtocounter{seqlevel}{#1}
  \path
  (#4)+(0,-\theseqlevel*\unitfactor-0.7*\unitfactor) node (mess to) {};
  \draw[->,>=angle 60] (mess from) -- (mess to) node[midway, above]
  {#3};    
}
% new command for adding a distance option for threads as well
% \newthread[color][distance]{id}{title}
\RequirePackage{xargs}
\renewcommandx{\newthread}[4][1=gray!30, 2=0.2]{
  \newinst[#2]{#3}{#4}
  \stepcounter{threadnum}
  \node[below of=inst\theinstnum,node distance=0.8cm] (thread\thethreadnum) {};
  \tikzstyle{threadcolor\thethreadnum}=[fill=#1]
  \tikzstyle{instcolor#3}=[fill=#1]
}

\newenvironment{conditions}
  {\par\vspace{\abovedisplayskip}\noindent\begin{tabular}{>{$}l<{$} @{${}={}$} l}}
  {\end{tabular}\par\vspace{\belowdisplayskip}}

  
\newif\ifsubmission{}
\submissionfalse{}

\ifsubmission{}
	\newcommand{\TODO}[1]{}
\else
	\newcommand{\TODO}[1]{{\color{red}TODO: #1}}
\fi

\def\BibTeX{{\rm B\kern-.05em{\sc i\kern-.025em b}\kern-.08em
    T\kern-.1667em\lower.7ex\hbox{E}\kern-.125emX}}

% use same footnote multiple times
\makeatletter
\newcommand\footnoteref[1]{\protected@xdef\@thefnmark{\ref{#1}}\@footnotemark}
\makeatother


\makeatletter
\newcommand{\linebreakand}{%
  \end{@IEEEauthorhalign}
  \hfill\mbox{}\par
  \mbox{}\hfill\begin{@IEEEauthorhalign}
}
\makeatother

% \sisetup{range-phrase=\text{--}}

\newcommand \titlestring{Smartphones with UWB: Evaluating the Accuracy and Reliability of UWB Ranging}

\begin{document}

\title{\titlestring{}}

\author{
  Alexander Heinrich, Sören Krollmann, Florentin Putz, Matthias Hollick
  \thanks{
    A. Heinrich*, S. Krollmann*, F. Putz, and M. Hollick are with the Secure Mobile Networking Lab in the Computer Science department at TU Darmstadt, Pankratiusstr. 2, 64289 Darmstadt, Germany (e-mail: aheinrich@seemoo.de, skrollmann@seemoo.de, fputz@seemoo.de, mhollick@seemoo.de). \\
    * Both authors contributed equally to this work.
  }
}

% \author{IEEE Publication Technology Department
% \thanks{Manuscript created October, 2020; This work was developed by the IEEE Publication Technology Department. This work is distributed under the \LaTeX \ Project Public License (LPPL) ( http://www.latex-project.org/ ) version 1.3. A copy of the LPPL, version 1.3, is included in the base \LaTeX \ documentation of all distributions of \LaTeX \ released 2003/12/01 or later. The opinions expressed here are entirely that of the author. No warranty is expressed or implied. User assumes all risk.}}

\markboth{}
{Alexander Heinrich, Sören Krollmann \MakeLowercase{\textit{(et al.)}}:
\titlestring{}}

\maketitle

\begin{abstract} 
  More and more consumer devices implement the IEEE \ac{uwb} standard to perform distance measurements for sensitive tasks such as keyless entry and startup of modern cars, to find lost items using coin-sized trackers, and for smart payments. While \ac{uwb} promises the ability to perform time-of-flight centimeter-accurate distance measurements between two devices, the accuracy and reliability of the implementation in up-to-date consumer devices have not been evaluated so far. In this paper, we present the first evaluation of \ac{uwb} smartphones from Apple, Google, and Samsung, focusing on accuracy and reliability in passive keyless entry and smart home automation scenarios. To perform the measurements for our analysis, we build a custom-designed testbed based on a \ac{gwen}, which allows us to create reproducible measurements. All our results, including all measurement data and a manual to reconstruct a \ac{gwen} are published online. We find that the evaluated devices can measure the distance with an error of less than \SI{20}{\centi\meter}, but fail in producing reliable measurements in all scenarios. Finally, we give recommendations on how to handle measurement results when implementing a passive keyless entry system.
\end{abstract}

% \begin{IEEEkeywords}
%     Device Security, Device-to-Device Communication, Secure Communications, Mobile and Ubiquitous Systems, Cyber-Physical Systems 
% \end{IEEEkeywords}

\section{Introduction}
\label{sec:introduction}
% \begin{itemize}
%     % Diffusion of FL
%     \item {\st{Diffusion of FL}}
%     % Security threats to FL
%     \item {\st{Security threats to FL with particular focus on model poisoning}}
%     % Limitations of existing countermeasures
%     \item {\st{Current countermeasures (e.g., KRUM) and their limitations}}
%     % Proposed method and its advantages
%     \item {\st{Intuitive description of the proposed method and its difference (i.e., advantages) w.r.t. state of the art}}
%     % Main contributions
%     \item {\st{Summary of the main contributions of this work}}
%     % Paper's structure and organization
%     \item {\st{Paper's structure and organization}}
% \end{itemize}

% Diffusion of FL
Recently, {\em federated learning} (FL) has emerged as the leading paradigm for training distributed, large-scale, and privacy-preserving machine learning (ML) systems~\cite{mcmahan2017googleai,mcmahan2017aistats}. 
The core idea of FL is to allow multiple edge clients to collaboratively train a shared, global model without disclosing their local private training data.
%Specifically, an FL system consists of a central server and many edge clients; 
A typical FL round involves the following steps: {\em(i)} the server randomly picks some clients and sends them the current, global model; {\em(ii)} each selected client locally trains its model with its own private data; then, it sends the resulting local model to the server;\footnote{Whenever we refer to global/local model, we mean global/local model {\em parameters}.} {\em(iii)} the server updates the global model by computing an \emph{aggregation function}, usually the average (FedAvg), on the local models received from clients.
% \begin{enumerate}
%     \item[{\em(i)}] the server sends the current, global model to the clients and appoints some of them for training;
%     \item[{\em(ii)}] each selected client locally trains its copy of the global model with its own private data; then, it sends the resulting local model back to the server;\footnote{Whenever we refer to global/local model, we mean global/local model {\em parameters}.}
%     \item[{\em(iii)}] the server updates the global model by computing an \emph{aggregation function} on the local models received from clients (by default, the average, also referred to as FedAvg~\cite{mcmahan2017aistats}).
% \end{enumerate}
This process goes on until the global model converges. %(e.g., after a certain number of rounds or other similar stopping criteria).
%\\
% The advantages of FL over the traditional, centralized learning paradigm are undoubtedly clear in terms of flexibility/scalability (clients can join/disconnect from the FL network dynamically), network communications (only model weights\footnote{We will use \textit{parameters} and \textit{weights} interchangeably.} are exchanged between clients and server), and privacy (each client's private training data is kept local at the client's end and not uploaded to the server).
\\
% Security threats to FL
%However, the growing adoption of FL also raises security concerns~\cite{costa2022covert}, particularly about its confidentiality, integrity, and availability.
Although its advantages over standard ML, FL also raises security concerns~\cite{costa2022covert}. %, particularly about its confidentiality, integrity, and availability~\cite{costa2022covert}.
% OLD, LONG VERSION
% Indeed, some work deals with privacy leakage that may expose the local data of some clients~\cite{melis2019sp}. 
% A large body of work, instead, investigates attacks that usually aim to detriment the predictive accuracy of the learned global model. For instance, \emph{data poisoning} attacks achieve this goal by letting an adversary pollute the training set of some corrupt FL clients with maliciously crafted examples~\cite{jagielski2018sp}.
% Similarly, in \emph{model poisoning} the attacker attempts to tweak the global model weights~\cite{bhagoji2019pmlr} by directly perturbing the local model's weights of some infected FL clients before these are sent to the central server for aggregation, usually via so-called Byzantine attacks. 
% It turns out that Byzantine model poisoning attacks severely impact standard FedAvg; therefore, more robust aggregation functions must be designed to make FL systems secure.
Here, we focus on \emph{untargeted model poisoning} attacks~\cite{bhagoji2019pmlr}, where an adversary attempts to tweak the global model weights %\footnote{We will use the terms \textit{parameters} and \textit{weights} interchangeably.} 
by directly perturbing the local model's parameters of some infected clients before these are sent to the central server for aggregation.
In doing so, the adversary aims to jeopardize the global model \textit{indiscriminately} at inference time.
Such model poisoning attacks severely impact standard FedAvg; therefore, more robust aggregation functions must be designed to secure FL systems.
\\
% In this paper, we focus on designing a novel robust aggregation scheme at the server's end to contrast the effect of Byzantine model poisoning attacks.
%
% Current countermeasures and their limitations
%Several countermeasures have been proposed in the literature to combat model poisoning attacks on FL systems.
% Some methods use simple statistics more robust than plain average to smooth the impact of malicious updates (e.g., Trimmed Mean and FedMedian~\cite{yin2018icml}). 
% Other defenses implement outlier detection techniques to discard malicious updates from the aggregation performed at the server's end. Those are either based on heuristics (e.g., Krum/Multi-Krum~\cite{blanchard2017nips} and Bulyan~\cite{mhamdi2018pmlr}) or data-driven approaches (e.g., K-means clustering~\cite{shen2016acm} or DnC via spectral analysis~\cite{shejwalkar2021ndss}). 
% Finally, some strategies rely on a centralized ``source of trust'' to spot potential malicious updates (e.g., FLTrust~\cite{cao2020fltrust}).
% Several countermeasures have been proposed in the literature to combat model poisoning attacks on FL systems, i.e., to discard possible malicious local updates from the aggregation performed at the server's end. 
% These techniques range from simple statistics more robust than plain average (e.g., Trimmed Mean and FedMedian~\cite{yin2018icml}) to outlier detection heuristics (e.g., Krum/Multi-Krum~\cite{blanchard2017nips} and Bulyan~\cite{mhamdi2018pmlr}) or data-driven approaches (e.g., spectral analysis via K-means clustering~\cite{shen2016acm} or spectral analysis), or methods based on ``source of trust'' (e.g., FLTrust~\cite{cao2020fltrust}).
% OLD, LONG VERSION
%Several countermeasures have been proposed in the literature to combat Byzantine model poisoning attacks on FL systems.
% Descriptive statistics
% For example, Trimmed Mean and FedMedian aggregate local model updates using more robust statistics than standard average~\cite{yin2018icml}.
%
% % Heuristics for outlier detection
% Many existing Byzantine-resilient strategies implement some outlier detection heuristics to discard the model updates sent by potentially malicious clients from the input of the aggregation function.
% One of the most popular heuristics is Krum~\cite{blanchard2017nips}.
% This strategy tries to mitigate the impact of Byzantine attacks by selecting as a global model the local model with the smallest sum of Euclidean distances to {\em all} the other local models.
% Although powerful, Krum requires the server to know (or, at least, estimate) the number of malicious FL clients upfront, which is generally impossible in a realistic attack scenario. %
% Moreover, Krum may become ineffective for complex, high-dimensional model parameter spaces due to the curse of dimensionality.
% Bulyan~\cite{mhamdi2018pmlr} tries to overcome this issue by combining Krum with a variant of Trimmed Mean.
% % Data-driven outlier detection
% Other strategies use data-driven outlier detection techniques -- e.g., via K-means clustering~\cite{shen2016acm} -- to spot potential malicious local model updates. 
% %For instance, Shen et al. propose to cluster local model updates with K-means and thus identify outliers.
%
% % Other techniques
% As far as the server is concerned, any local model received can be from a potential malicious client. 
% FLTrust~\cite{cao2020fltrust} assumes the server acts as a client, i.e., trains a local model on an additional {\em trustworthy} dataset at the server's end and compares it against all the local models from other clients. 
% This way, the server can rely on some ``source of trust'' when discarding potentially malicious clients.
%\\
% Limitations of existing Byzantine-resilient strategies
Unfortunately, existing defense mechanisms either rely on simple heuristics (e.g., Trimmed Mean and FedMedian by~\cite{yin2018icml}) or need strong and unrealistic assumptions to work effectively (e.g., foreknowledge or estimation of the number of malicious clients in the FL system, as for Krum/Multi-Krum~\cite{blanchard2017nips} and Bulyan~\cite{mhamdi2018pmlr}, which, however, cannot exceed a fixed threshold).
Furthermore, outlier detection methods using K-means clustering~\cite{shen2016acm} or spectral analysis like DnC~\cite{shejwalkar2021ndss} do not directly consider the temporal evolution of local model updates received.
Finally, strategies like FLTrust~\cite{cao2020fltrust} require the server to collect its own dataset and act as a proper client, thereby altering the standard FL protocol.
\\
% OLD, LONG VERSION
% Overall, existing Byzantine-resilient strategies are either simple heuristics (e.g., FedMedian) or, if they are more complex, they rely on strong and unrealistic assumptions to work effectively (e.g., knowing the number of malicious clients in the FL system in advance, as for Krum and alike).
% Furthermore, data-driven outlier detection methods do not consider the temporary evolution of local model updates received (e.g., K-means clustering). 
% Finally, strategies like FLTrust requires the server to collect its own dataset and act as a proper client, thereby altering the standard FL protocol.
%
% Description of the proposed method
This work introduces a novel pre-aggregation \textit{filter} robust to untargeted model poisoning attacks. Notably, this filter $(i)$ operates without requiring prior knowledge or constraints on the number of malicious clients and $(ii)$ inherently integrates temporal dependencies. 
The FL server can employ this filter as a preprocessing step before applying \textit{any} aggregation function, be it standard like FedAvg or robust like Krum or Bulyan.
Specifically, we formulate the problem of identifying corrupted updates as a multidimensional (i.e., matrix-valued) time series anomaly detection task. 
The key idea is that legitimate local updates, resulting from well-calibrated iterative procedures like stochastic gradient descent (SGD) with an appropriate learning rate, show \textit{higher predictability} compared to malicious updates. This hypothesis stems from the fact that the sequence of gradients (thus, model parameters) observed during legitimate training exhibit regular patterns, as validated in Section~\ref{subsec:intuition}. %until convergence. 
%This regularity may be more pronounced for smooth convex loss functions, but it can still be captured within an appropriate time window, even for more complex and convoluted loss surfaces. 
%We provide evidence of this claim in Appendix~B, where we show that the average mutual information (i.e., ``predictability''), calculated over pairs of legitimate model updates sent at different FL rounds, is significantly higher than the corresponding computation for a malicious client.
\\
Inspired by the matrix autoregressive (MAR) framework for multidimensional time series forecasting~\cite{chen2021je}, we propose the FLANDERS ({\em \textbf{F}ederated \textbf{L}earning meets \textbf{AN}omaly \textbf{DE}tection for a \textbf{R}obust and \textbf{S}ecure}) filter.
The main advantages of FLANDERS over existing strategies like FLDetector~\cite{zhao2020multivariate} are its resilience to large-scale attacks, where $50\%$ or more FL participants are hostile, and the capability of working under realistic non-iid scenarios.
We attribute such a capability to two key factors: $(i)$ FLANDERS works without knowing a priori the ratio of corrupted clients, and $(ii)$ it embodies temporal dependencies between intra- and inter-client updates, quickly recognizing local model drifts caused by evil players. Below, we summarize our main contributions:

\begin{itemize}
\item[{\em(i)}]
We provide empirical evidence that the sequence of models sent by legitimate clients is more predictable than those of malicious participants performing untargeted model poisoning attacks.
\\
\item[{\em(ii)}] 
We introduce FLANDERS, the first pre-aggregation filter for FL robust to untargeted model poisoning based on multidimensional time series anomaly detection.
\\
\item[{\em(iii)}] 
We integrate FLANDERS into Flower,\footnote{\scriptsize{\url{https://flower.dev/}}} a popular FL simulation framework for reproducibility.
\\
\item[{\em(iv)}] 
We show that FLANDERS improves the robustness of the existing aggregation methods under multiple settings: different datasets, client's data distribution (non-iid), models, and attack scenarios.
\\
\item[{\em(v)}] 
We publicly release all the implementation code of FLANDERS along with our experiments.\footnote{\scriptsize{\url{https://anonymous.4open.science/r/flanders_exp-7EEB}}}
\end{itemize}

% Paper's structure and organization
The remainder of the paper is structured as follows. %some related work and the current state-of-the-art solutions to security issues that FL entails. 
Section~\ref{sec:background} covers background and preliminaries. 
In Section~\ref{sec:related}, we discuss related work.
Section~\ref{sec:problem} and Section~\ref{sec:method} describe the problem formulation and the method proposed. % to tackle it. 
Section~\ref{sec:experiments} gathers experimental results. %, and Section~\ref{sec:limitations} discusses some limitations of this work.
Finally, we conclude in Section~\ref{sec:conclusion}.
 %discusses the limitations of this work and draws future research directions.
%reports conclusions and draws perspectives for future research directions.

%%%%%%% OLD %%%%%%%
%to overcome the resilience of Byzantine failures in distributed Stochastic Gradient Descent computations. 
% The strength of Krum is its time complexity, which is linear in the gradient dimension. 
% However, the robustness of the approach is guaranteed for gradient-based learning applications only when the majority of the clients are not compromised. 
% Besides, the aggregation mechanism of Krum, as well as that of similar methods, is robust from a coarse-grained perspective and does not provide solutions to errors and perturbations that may occur at inference time.
%A related approach to~\cite{blanchard2017nips} is the work of Su et al.~\cite{su2016dc}. Here, the authors propose an iterated approximate agreement to tackle a multi-layer scenario attacked by Byzantine agents. 
%However, the method works efficiently on the sole discrete context and it is inapplicable to continuous state environments.
%\gabri{Maybe, we should just talk about the main limitations of existing countermeasures without digging into their details (or, we can just mention Krum as this is the most popular one). I will move the description of all these methods to the Related Work section.}

\section{Background on Network Calculus}
\label{sec: background}


\begin{figure*}[tbh]
\centering
\begin{subfigure}[b]{0.3\textwidth}
    \centering
    \includegraphics[width=\linewidth]{images/in-out.png}
    \caption{Arrival and departure data and their relation with delay $d(t)$ and backlog $b(t)$. For a FIFO system, the delay is the horizontal distance between $R(t)$ and $R^*(t)$ but some other multiplexing techniques may shift the data to a later priority, causing a longer delay.}
    \label{fig: data in-out}
\end{subfigure}
\hfill
\begin{subfigure}[b]{0.35\textwidth}
    \centering
    \includegraphics[width=\linewidth]{images/arrival-service.png}
    \caption{Characteristics of an arrival curve and a service curve. From any point of observation, the arriving data never exceeds its arrival curve; the departure data is also never less than the service curve with respect to the data arrival.}
    \label{fig: arrival-service curves}
\end{subfigure}
\hfill
\begin{subfigure}[b]{0.33\textwidth}
    \centering
    \includegraphics[width=\linewidth]{images/bound.png}
    \caption{Delay and backlog bounds of a system. Backlog is the maximum vertical distance between $\alpha(t)$ and $\beta(t)$; FIFO delay is their maximum horizontal distance; but for arbitrary multiplexing, the delay guarantee is when the system clears its buffer, thus it's the intersection of $\alpha(t)$ and $\beta(t)$.}
    \label{fig: system bounds}
\end{subfigure}
\caption{Network calculus framework. We let $R(t)$ and $R^*(t)$ be the arrival and departure data flow of a system; $\alpha(t)$ be the piecewise linear concave arrival curve and $\beta(t)$ be the piecewise linear convex service curve of a system.}
% \hossein{Better to show piece-wise linear concave arrival curve and piece-wise linear convex service curve instead of token-bucket and rate-latency.}}
\end{figure*}

We recall some of the network calculus essentials for a better understanding of the framework used in Saihu. In the following context, we use the following notation: $\mbb{R}^+$ is the set of non-negative real numbers; $[x]_+$ denotes $\max(0, x)$

The data flow is by convention modeled as a left-continuous wide-sense increasing function $R(t): \mbb{R}^+ \mapsto \mbb{R}^+$ with respect to time $t$~\cite{ncbook2001leboudec}. 

A system $\mcal{S}$ receives arrival data described as a cumulative function $R(t)$ and delivers departure data as another cumulative function $R^*(t)$. Figure~\ref{fig: data in-out} illustrates such a system $\mcal{S}$. The benefit of representing a system like this is that we can observe system backlog and delay with such a model. 

\begin{definition}[Backlog and Delay~\cite{ncbook2001leboudec}]
    The backlog of a system at time~$t$ is
    \begin{equation}
        b(t) = R(t) - R^*(t)
    \end{equation}
    
    The virtual delay of a FIFO system at time $t$ is
    \begin{equation}
        d_{FIFO}(t) = \inf \lbp \tau \geq 0 : R(t) \leq R^*(t+\tau) \rbp
    \end{equation}
\end{definition}



The backlog of a system can be viewed as the vertical distance between $R$ and $R^*$. The FIFO (\textit{First-in First-out}) delay is the horizontal distance between $R$ and $R^*$. One may obtain other delay values if the multiplexing technique is not FIFO.

% \begin{figure}
%     \centering
%     \includegraphics[width=0.9\linewidth]{images/in-out.png}
%     \caption{In/out data flow; delay and backlog}
%     \label{fig: data in-out}
% \end{figure}

Since we are interested in the system guarantee instead of a single instance of data flow, we would like to have general bounds to the arrival and departure data flows. Therefore, we define \textit{arrival curve} and \textit{service curve} as the bounds of arrival and departure data flows.

\begin{definition}[Arrival Curve~\cite{ncbook2001leboudec}]
    Given a wide-sense increasing function $\alpha: \mbb{R}^+ \mapsto \mbb{R}^+$, we say that a flow $R(t)$ is $\alpha$-constrained if and only if for all $s \leq t$:
    \begin{equation}
        R(t) - R(s) \leq \alpha(t-s)
    \end{equation}
    We say $R(t)$ has $\alpha$ as an arrival curve.
\end{definition}

\begin{definition}[Service Curve~\cite{ncbook2001leboudec}]
    Given a wide-sense increasing function $\beta: \mbb{R}^+ \mapsto \mbb{R}^+$ and $\beta(0) = 0$. A system $\mcal{S}$ having $R(t)$ and $R^*(t)$ as its arrival and departure flows. We say $\mcal{S}$ offers a service curve $\beta$ if and only if
    \begin{equation}
        R^*(t) \geq (R \otimes \beta)(t) =: \inf_{s \leq t} \lbp R(s) + \beta(t-s) \rbp
    \end{equation}
    where $\otimes$ denotes the min-plus convolution
\end{definition}

Figure~\ref{fig: arrival-service curves} illustrates the arrival and service curves. Any segment of arrival flow $R(t)$ is constrained by arrival curve $\alpha$ and the output curve $R^*(t)$ is always no less than the curve $R\otimes\beta$. As a result, an arrival curve upper bounds the incoming traffic, and a service curve lower bounds the outgoing traffic.

% \begin{figure}
%     \centering
%     \includegraphics[width=\linewidth]{images/arrival-service.png}
%     \caption{Arrival/Service curve}
%     \label{fig: arrival-service curves}
% \end{figure}

We consider 2 special types of curves throughout this paper, \textit{token-bucket} (or sometimes called \textit{leaky-bucket}) curve and \textit{rate-Latency} curve.

\begin{definition}[Token-bucket and Rate-latency~\cite{ncbook2001leboudec}]
    A token-bucket curve $\gamma_{r,b}$ with arrival rate $r$ and burst $b$ is defined as
    \begin{equation}
        \gamma_{r,b}(t) = b + rt
    \end{equation}

    A rate-latency curve $\beta_{R,T}$ with service rate $R$ and latency $T$ is defined as
    \begin{equation}
        \beta_{R,T}(t) = R \lb t - T \rb_+
    \end{equation}
\end{definition}

A token-bucket curve is determined by a burst $b$ and an arrival rate~$r$. Burst represents the maximum possible data volume that can arrive simultaneously, and arrival rate represents the maximum long-term data rate~\cite{bouillard2022tradeoff}.
A rate-latency curve is determined by a latency~$T$ and a service rate~$R$. Latency represents the time a server needs before starting to process the incoming data, and service rate represents the minimum rate to process data after the initial latency.

With the help of arrival and service curves, we can derive delay and backlog bounds for a system $\mcal{S}$ illustrated in Figure~\ref{fig: system bounds}. Suppose a system $\mcal{S}$ has arrival curve $\alpha$ and service curve~$\beta$, its worst-case backlog $b^*$ is the maximum vertical distance between~$\alpha$ and~$\beta$. Similarly, depending on the multiplexing technique applied to the system, its worst-case delay bound $d^*$ is the maximum horizontal distance between $\alpha$ and $\beta$ if $\mcal{S}$ is a FIFO system. If we don't have any information about its multiplexing technique, referred to as arbitrary multiplexing, the best we can say is that when $\alpha$ and $\beta$ intersect each other, where all data has been delivered out of the system. Consequently, the worst-case delay bound for arbitrary multiplexing is the time required for $\mcal{S}$ to clear its buffer.

% \begin{figure}
%     \centering
%     \includegraphics[width=\linewidth]{images/bound.png}
%     \caption{System delay/backlog bounds}
%     \label{fig: system bounds}
% \end{figure}

While a service curve captures the slowest possible output speed of a system, a link's transmission capacity limits the speed as well. Hence, we model this phenomenon using a \textit{greedy shaper} with a sub-additive function $\sigma: \mbb{R}^+ \mapsto \mbb{R}^+$ concatenated with a server. We consider a concatenation as shown in Figure \ref{fig: system}. By convention we assume $\sigma(0) = 0$ and $\beta(t) \leq \sigma(t), \forall t \in \mbb{R}^+$, meaning that the buffer is cleared at the beginning and the service never exceed its physical limitation. With the above definition, such greedy shaper conserves the service provided by the system due to theorem \ref{thm: shaping}.

\begin{figure}[thb]
    \centering
    \includegraphics[width=0.7\linewidth]{images/system.png}
    \caption{Shaping of departure data. A flow that has an arrival curve $\alpha$ feeds into a server with an arrival data flow $R(t)$. The server having service curve $\beta$ takes $R(t)$ and gives a departure data flow $R^*(t)$ to a shaper with shaping function $\sigma$. The shaper takes $R^*(t)$ and shape the data flow as another departure $D(t)$.}
    \label{fig: system}
\end{figure}


\begin{theorem}[Shaping conserves service \cite{ncbook2001leboudec}]
\label{thm: shaping}
Following the system shown in Figure \ref{fig: system}, we have
\begin{equation}
     D = R^* \otimes \sigma \geq \lp R \otimes \beta \rp \otimes \sigma = R \otimes \lp \beta \otimes \sigma \rp = R \otimes \beta
\end{equation}
\end{theorem}

In the following context, we model the shaping function $\sigma$ as a token-bucket curve $\gamma_{C,L}$ with transmission capacity $C$ and the packet size $L$ to capture the link capacity and packetization~\cite{bouillard2022tradeoff}.


% !TeX root = ../main.tex

\makeatletter
\AC@reset{gwen}
\makeatother

\section{GWEn}\label{sec:gwen}
\noindent 
% \begin{figure}[!t]
%     \centering
%     \begin{tikzpicture}
 
    % Include the image in a node
    \node [
        above right,
        inner sep=0] (image) at (0,0) {\includegraphics[width=.9\linewidth]{figures/GWEn_Full.PNG}};
     
    % Create scope with normalized axes
    \begin{scope}[
        x={($0.1*(image.south east)$)},
        y={($0.1*(image.north west)$)}]
        
        % % Grid
        % \draw[lightgray,step=1] (image.south west) grid (image.north east);
        
        % % Axes' labels
        % \foreach \x in {0,1,...,10} { \node [below] at (\x,0) {\x}; }
        % \foreach \y in {0,1,...,10} { \node [left] at (0,\y) {\y};}

        % Labels

        \draw[latex-, thick,seemoo_organge] (8.5,6) -- ++(0.5,0.5)
        node[above,black,fill=white]{\tiny Stepper Motor};

        \draw[latex-, thick,seemoo_organge] (3.7,0.7) -- ++(0.5,0.5)
            node[right,black,fill=white]{\tiny Power Input};

        \draw[latex-, thick,seemoo_organge] (5.5,2.6) -- ++(0.5,-0.5)
            node[below,black,fill=white]{\tiny Rotation Plate};

        \draw[latex-, thick,seemoo_organge] (6.3,9.5) -- ++(-0.5,0.5)
            node[left,black,fill=white]{\tiny Rotation Arm};

        \draw[latex-, thick,seemoo_organge] (2,6.5) -- ++(-0.5,-0.5)
            node[below,black,fill=white]{\tiny Holding Plate};

        \draw[latex-, thick,seemoo_organge] (5.7,6.6) -- ++(-0.5,-0.5)
            node[below,black,fill=white]{\tiny Holding Plate Support};
            
        \draw[latex-, thick,seemoo_organge] (3,3.5) edge (4,4)
            (5.5,3.5) -- (4,4)
            node[above,black,fill=white]{\tiny Mounting Plates};

        \draw[latex-, thick,seemoo_organge] (2.5,8.5) edge (2,9)
            (2.6,7.45) -- (2,9)
            node[above,black,fill=white]{\tiny Subject Clamps};
        
        \draw[thick,seemoo_organge] (2.25,2.7) rectangle (7.65,0) 
            node[above left,white,fill=seemoo_organge]{\tiny Base};

        \draw[thick,seemoo_organge] (6.5,3.15) rectangle (7.9,10) 
            node[below left,white,fill=seemoo_organge]{\tiny Tower};

        \draw[thick,seemoo_organge] (1.1,3) rectangle (0,10) 
            node[below right,white,fill=seemoo_organge]{\tiny Supports};

        \draw[latex-, thick,seemoo_organge] (0.3,8.5) -- ++(-0.5,-0.5)
            node[below,black,fill=white]{\tiny Plug};

    \end{scope} 
\end{tikzpicture}
%     \caption{$3$D rendering of fully assembled GWEn with all available parts.}\label{fig:GWEn_complete}
% \end{figure}
The main goal of this work is to determine the ranging accuracy and reliability of \ac{uwb} smartphones in a standardized, repeatable, reproducible, and automated way.
This enables a systematic evaluation and comparison of our \acp{dut}.

To accomplish this, we develop, build, and evaluate a \ac{gwen}. \Cref{fig:gwen_rotation_axes} shows a labeled build of \ac{gwen}. 
It is equipped with two rotation axes, one in the base and one in the arm as depicted in \cref{fig:gwen_rotation_axes}.
Throughout this paper, the angle $\phi$ denotes the arm rotation around the x-axis and the angle $\theta$ denotes the base rotation around the y-axis. 

\subsection{Measurements with GWEn}
\Ac{gwen} is a multipurpose tool for conducting wireless measurements and evaluating antennas at a small scale. 

\begin{figure}[!t]
    \centering
    % \input{figures/tikz/gwen_3d_rendering_axes.tex}
    \includegraphics[width=\linewidth]{figures/GWEN_3d_render_axes_notation.pdf}
    \caption{$3$D rendering of an assembled build of GWEn, highlighting the two rotation axes in red. Both axes can rotate independently for a full 360° each.}\label{fig:gwen_rotation_axes}
\end{figure}

Each axis can rotate the full 360° making it possible to position the \ac{dut}, attached to the holding plate, in any desired orientation with a resolution of $\sim$0.01°.
A variety of devices can be attached to the holding plate, allowing all kinds of wireless transmission measurements or antenna evaluations. 
For an optimal measurement, the antenna of the \ac{dut} must be as close as possible to the intersection of the two rotation axes.
The tower on the rotation plate can be moved further outward or inward to accommodate devices of different lengths.
The holding plate on the tower can be moved further up or down to adapt \ac{gwen} to devices of different thicknesses.
To be able to extract the measurement data from the \ac{dut}, it is connected to \ac{gwen} via a USB connection.

\ac{gwen} can be controlled over a web interface that is reachable via its own Wi-Fi hotspot. Furthermore, it's also possible to create custom Python scripts that allow different movement patterns. 

To create meaningful and repeatable measurements, we have developed a workflow for \ac{gwen} measurements:
\begin{enumerate}
    \item \ac{gwen} setup: This includes setting up \ac{gwen} at the test location, mounting the \ac{dut} and setting up the measurement parameters.
    \Cref{ssec:general_setup} goes into more detail on the specific setup used in our work.
    \item Start the measurement.
    From now on, \ac{gwen} will automatically move the \ac{dut} to previously specified orientations, take the desired number of measurements, and store them accordingly.
    \item A measurement between two iPhones with the settings described in \cref{ssec:general_setup} takes $\sim$90 minutes. No monitoring of the system is required during this period. 
    \item Download the measurement.
    After finishing the measurements, the recordings can be downloaded as zip files.
    Our evaluation software directly works on the generated zip files and generates graphs to evaluate the performance. 
\end{enumerate}

\subsection{Manufacturing}\label{ssec:gwen_manufacturing}
We designed \ac{gwen} to be $3$D-printed using a fused deposition modeling $3$D printer.
Besides belts, screws, bearings and the electrical components all parts are $3$D-printed.
We use \ac{pla} as a printing material. \Ac{pla} has only very little influence on electromagnetic signals \cite{boussatourDielectricCharacterizationPolylactic2018}.
\SI{1}{\centi\metre} of solid \ac{pla} adds about \SI{0.6}{\centi\metre} of measured distance to an \ac{uwb} signal.
We evaluate the interference introduced by \ac{gwen} with more detail in \cref{sec:eval_gwen}.

$3$D-printing is a low-cost and widely available option that allows easy reproduction of desired parts.
Furthermore, it enables modifications, as new parts can be designed and $3$D-printed rapidly.
It takes about $4.5$ days to print all parts on a Prusa i3 MK3S+~\cite{prusaresearcha.s.OriginalPrusaI3}. 
However, this time depends heavily on the printer and the print settings.
Assembling \ac{gwen} and soldering up the electrical components requires additional $\sim$80 hours of manual work.

The parts that cannot be $3$D-printed are off-the-shelf parts.
In total, the material cost for \ac{gwen} was $\sim$410~€.
We released the $3$D files, a list with all additional components required, detailed build instructions, as well as wiring diagrams and source code in our Zenodo repository~\cite{krollmannGWEnGimbalBased2022}.


The software of \ac{gwen} consists of three individual parts:
\begin{enumerate*}
    \item ~the~\textit{web interface} written in Python,
    \item ~the~\textit{measurement software} written in Python, and
    \item ~the~\textit{hardware controller} written in C/C++.
\end{enumerate*}
This modularity allows anyone to rebuild \ac{gwen} as well as expand and adapt it to individual needs.


\subsection{Measurement recording}\label{ssec:sources}

We programmed \ac{gwen} to interact with a variety of devices (see \cref{tab:available_uwb}) and process their \ac{uwb} measurements.
Since the devices come from different manufacturers and run with different software, \ac{gwen} offers the possibility to add new communication options with the help of \textit{sources}.
Each \textit{source} is a plugin written in Python that has to be defined once and provides functions for \ac{gwen} to extract the necessary data from the device.
For smartphones, we use device logs to extract ranging data from the device (see \cref{ssec:dev_conf}).
The \textit{source} to be used for each measurement can be specified at the start of a measurement.

At the end of each measurement, \ac{gwen} creates a recording file.
It contains a JSON file with the setup settings used to create the recording as well as all made distance measurements ordered by their respective base and arm angle.
It is furthermore possible to add additional files, e.g., a complete log of the measurement.
All measurement data is backed up constantly and can be restored. 





% !TeX root = ../main.tex

\section{Evaluation of GWEn}\label{sec:eval_gwen}
\noindent 
As described in \cref{ssec:gwen_manufacturing}, a 1 cm thick \ac{pla} wall has a negligible effect on the signal.
In addition to the parts printed from \ac{pla}, \ac{gwen} also contains motors, ball bearings and shafts.
These parts consist of metal that interferes with the signal between the two devices.
This is especially the case when the tower is directly between the two antennas.
Therefore, before evaluating the data, it must be ensured that the measurement is not affected by \ac{gwen}'s test setup and procedure.

\subsection{Maximum error}

\begin{figure}[!t]
  \centering
  \begin{subfigure}{.48\linewidth}
    \centering
    \includegraphics[width=.98\linewidth]{figures/eval/gwen/20220620122246_IMG_4130.JPG}
    \caption{\raggedright{} LOS measurement without obstructions at a base rotation of $\theta=\SI{0}{\degree}$.}\label{fig:t1}
  \end{subfigure}%
  \hfill
  \begin{subfigure}{.48\linewidth}
    \centering
    \includegraphics[width=.98\linewidth]{figures/eval/gwen/20220620122440_IMG_4131.JPG}
    \caption{\raggedright{} NLOS measurement with the tower between the devices at a base rotation of $\theta=\SI{0}{\degree}$.}\label{fig:t2}
  \end{subfigure}
  \caption{Exemplary test setup for evaluating the influence of the tower on \ac{gwen}. The real installation also included the wiring, power supplies, and ensuring the correct spacing. This has been omitted in this picture to simplify matters.}\label{fig:gwen_arm_test}
\end{figure}

To evaluate the influence of \ac{gwen} on the recordings, we performed two tests using a pair of DW3000 \ac{uwb} devices. 
First, we placed both devices facing each other with no obstructions between them in clear \ac{los} (see \cref{fig:t1}). 
Second, we rotated the base of \ac{gwen} to $\theta=\SI{180}{\degree}$ such that the tower is now directly between both devices (see \cref{fig:t2}). We also turned the DW3000 again to ensure that they have the same orientation as in the first test.  For both tests, we ensured that the distance between the DW3000 \ac{uwb} devices is equal. 

In each test, we performed $10,000$ \ac{uwb} measurements. 
To assess the influence of \ac{gwen}, we calculated the mean measured distance and the standard deviation for both measurements.
In \ac{los}, without the tower, a mean value of \SI{35}{\centi\metre} with a standard deviation of \SI{1.3}{\centi\metre} was measured. 
With the tower, \SI{36}{\centi\metre} and \SI{1.7}{\centi\metre} were determined, respectively.

% TODO: Falls gewünscht kann ich (fp) mit statistischen Tests überprüfen, ob die Mittelwerte und Varianzen von LoS und NLoS sich statistisch signifikant unterscheiden.
It is evident that the presence of the arm has increased the distance by \SI{1}{\centi\metre}.
Also, the standard deviation increased by \SI{0.4}{\centi\metre}. The expected accuracy and precision of these devices are claimed to be \SI{10}{\centi\metre}~\cite{qorvoinc.DWM3000Qorvo,qorvoinc.DW3000DataSheet2020} and therefore \ac{gwen} does not add a large enough error to disturb our measurements.

\subsection{Reproducibility}
We assess the reproducibility of measurements by using an experimental setup with two DW$3000$.
One board was our \ac{dut} mounted to \ac{gwen} and the other board was mounted on a tripod. Both were placed \SI{50}{\centi\metre} apart. 
\ac{gwen} rotated a full \SI{360}{\degree} at the base ($\theta$) and \SI{180}{\degree} at the arm ($\phi$) at each base position. At each position, the DW$3000$s performed $100$ \ac{uwb} distance measurements.

The same full measurement cycle was then performed three times in succession. 
Subsequent examination of the three records showed that the mean and standard deviation for all records were \SI{0.48}{\meter} and \SI{0.11}{\meter}, respectively.
\Cref{fig:m_comp} shows for each measurement the plot for a full \SI{360}{\degree} rotation of the base $\theta$ with a fixed arm angle of $\phi=\SI{90}{\degree}$.
All graphs have very similar shapes, and we can see similar results when comparing different arm angles. 

\begin{figure}[!t]
    \centering
    \begin{subfigure}{.16\textwidth}
      \centering
      % \includegraphics[width=\linewidth]{figures/eval/gwen/polar_90_T1.pdf}
      \scalebox{0.32}{\input{figures/eval/gwen/polar_90_T1.pgf}}
      \caption{Recording 1}\label{fig:m1}
    \end{subfigure}%
    \begin{subfigure}{.16\textwidth}
      \centering
      % \includegraphics[width=\linewidth]{figures/eval/gwen/polar_90_T2.pdf}
      \scalebox{0.32}{\input{figures/eval/gwen/polar_90_T2.pgf}}
      \caption{Recording 2}\label{fig:m2}\end{subfigure}%
    \begin{subfigure}{.16\textwidth}
      \centering
      % \includegraphics[width=\linewidth]{figures/eval/gwen/polar_90_T3.pdf}
      \scalebox{0.32}{\input{figures/eval/gwen/polar_90_T3.pgf}}
      \caption{Recording 3}\label{fig:m3}
    \end{subfigure}
    \caption{Comparison of three 360° measurements made directly one after the other without changing the setup. Arm at 90°. The arm angle of $\phi=$\SI{90}{\degree} was chosen arbitrarily, other arm angles led to similar results.}\label{fig:m_comp}
\end{figure} 

Finally, we tore down the measurement setup and set it up again one week later. We measured with the same parameters and compared the results. 
\Cref{fig:m_comp_2} shows two plots one from the first recording described above and one from a fourth recording made one week later.
\Cref{fig:r1} and \Cref{fig:r4} both show a \SI{360}{\degree} rotation of the base ($\theta$) with the arm at $\phi=\SI{90}{\degree}$.
The general shape of the two measurements is similar, but we can see some deviations, which could have been caused by minor changes in the environment. 
The mean value is \SI{0.47}{\centi\metre}, \SI{1}{\centi\metre} less than the first measurements, but the standard deviation remained the same.

All tests combined show that \ac{gwen} only marginally influences \ac{uwb} measurements and can therefore be used to evaluate \ac{uwb} capable devices. 

\begin{figure}[!t]
  \centering
  \begin{subfigure}{.16\textwidth}
    \centering
    % \includegraphics[width=\linewidth]{figures/eval/gwen/polar_90_T1.pgf}
    \scalebox{0.32}{\input{figures/eval/gwen/polar_90_T1.pgf}}
    \caption{Recording 1}\label{fig:r1}
  \end{subfigure}%
  \begin{subfigure}{.16\textwidth}
    \centering
    % \includegraphics[width=\linewidth]{figures/eval/gwen/polar_90_T4.pgf}
    \scalebox{0.32}{\input{figures/eval/gwen/polar_90_T4.pgf}}
    \caption{Recording 4}\label{fig:r4}
  \end{subfigure}
  \caption{Comparison of two \SI{360}{\degree} base rotation ($\theta$) measurements made one week apart. With arm rotation at $\phi=\SI{90}{\degree}$.}\label{fig:m_comp_2}
\end{figure}


% !TeX root = ../main.tex

\section{Experimental setup}\label{sec:experimental_setup}
\noindent 
In this section, we detail how we set up our measurement device \ac{gwen}, how we configured each \ac{dut}, and how we were able to collect distance measurement data from \ac{uwb} over a USB connection. 

We evaluated each device listed in \cref{tab:available_uwb} in three different environments to see how multipath effects and reflections might change the results.


\subsection{Measurement Setup}\label{ssec:general_setup}


\begin{figure}[!t]
    \centering
    \begin{tikzpicture}
 
    % Include the image in a node
    \node [
        above right,
        inner sep=0] (image) at (0,0) {\includegraphics[width=.6\linewidth]{figures/GWEn_Overview.pdf}};
     
    % Create scope with normalized axes
    \begin{scope}[
        x={($0.1*(image.south east)$)},
        y={($0.1*(image.north west)$)}]
        
        % Grid
        % \draw[lightgray,step=1] (image.south west) grid (image.north east);
        
        % % Axes' labels
        % \foreach \x in {0,1,...,10} { \node [below] at (\x,0) {\x}; }
        % \foreach \y in {0,1,...,10} { \node [left] at (0,\y) {\y};}

        % Labels

        \draw[latex-, thick,seemoo_organge] (8.2,1) -- ++(-0.5,0.5)
        node[above,black,fill=white]{\scriptsize Tripod};

        \draw[latex-, thick,seemoo_organge] (3.5,4.1) -- ++(0.5,0.5)
        node[above,black,fill=white]{\scriptsize Device under test};

        \draw[latex-, thick,seemoo_organge] (8.7,4.4) -- ++(-0.5,0.5)
        node[above,black,fill=white]{\scriptsize Remote Device};

        \draw[latex-, thick,seemoo_organge] (3.5,0.5) -- ++(0.5,-0.5)
            node[right,black,fill=white]{\scriptsize GWEn};

        % arrows
        %\draw [stealth-](1.3,4.6) -- (3.4,7.4);
        \draw [latex-, line width=2pt](1.3,4.6) -- (2.3,5.93);
        \draw [-latex, line width=2pt](2.75,6.53) -- (3.4,7.4);

        \draw [latex-, line width=3pt, draw=white](3.42,3.7) -- (5.25,3.7);
        \draw [latex-, line width=2pt, ](3.5,3.7) -- (5.25,3.7);
        \draw [-latex, line width=2pt](6.45,3.7) -- (8.6,3.7);

    \end{scope} 
\end{tikzpicture}
    \caption{Setup of GWEn with a control PC and a remote device.}\label{fig:GWEn_setup}
\end{figure}


\Cref{fig:GWEn_setup} shows the hardware setup for a typical measurement.
The setup consists of \ac{gwen}, two devices to test, a laptop, and a tripod.
The \ac{dut} is mounted on \ac{gwen}, which is placed on a camera tripod such that the \ac{dut} remains at a height of \SI{80}{\centi\metre}. This height should model the distance of a jeans' pockets to the ground.  
The \ac{dut} has been placed such that its antennas were in the intersection of the rotation axis of \ac{gwen} (see \cref{fig:GWEn_setup_top_view}).
The remote device is mounted on a tripod and a desired distance between them is set. 
The distance between both devices is measured using a laser meter or a measurement tape. 
The arm is set to $\phi=\SI{0}{\degree}$ by using an internal sensor that detects when the motor has rotated to the correct position. The base is aligned such that the tower is located at an angle of \SI{90}{\degree} as shown in \cref{fig:GWEn_setup_top_view}, which we define as $\theta=\SI{0}{\degree}$ (see \cref{fig:GWEn_setup_top_view}).

The rotation axes of \ac{gwen} and according to labels are plotted in \cref{fig:gwen_rotation_axes} and detailed in \cref{sec:gwen}.
For all our experiments \ac{gwen} performs these steps: 
\begin{enumerate}
    \item Start at arm rotation $\phi=\SI{0}{\degree}$ and base rotation $\theta=\SI{0}{\degree}$
    \item Rotate the arm $\phi$ by \SI{10}{\degree} 
    \item When the arm reaches $\phi=\SI{180}{\degree}$, rotate the base $\theta$ by \SI{10}{\degree}
    \item Rotate the arm $\phi$ by \SI{-10}{\degree} 
    \item When the arm reaches $\phi=\SI{0}{\degree}$, rotate the base $\theta$ by \SI{10}{\degree}
    \item Continue with step 2 until the base reaches $\theta=\SI{360}{\degree}$
\end{enumerate}

At each position, we measure the distance and collect the measurements, before \ac{gwen} continues with the next step. 

Each measurement for each pair of devices has been conducted in the same way in the same environments to ensure comparability.
Once the measurement is started all operations for the measurement happen on \ac{gwen} and the external laptop is no longer needed. For \ac{uwb} measurements, the ranging has to be started on the \ac{dut} and the remote device in advance. 

% !TeX root = ../../../main.tex
\begin{figure}[!t]
    \centering
    \begin{tikzpicture}
        
        % Include the image in a node
        \node [
            above right,
            inner sep=0] (image) at (0,0) {\includegraphics[width=.4\textwidth]{figures/GWEn_orientation.pdf}};
         
        % Create scope with normalized axes
        \begin{scope}[
            x={($0.1*(image.south east)$)},
            y={($0.1*(image.north west)$)}]

            \tikzset{every path/.style={line width=1.5pt}};

            \coordinate (d) at (0.5,5.2);
            
            % % Grid
            % \draw[lightgray,step=1] (image.south west) grid (image.north east);
            
            % % Axes' labels
            % \foreach \x in {0,1,...,10} { \node [below] at (\x,0) {\x}; }
            % \foreach \y in {0,1,...,10} { \node [left] at (0,\y) {\y};}

            % Labels

            \draw[latex-, seemoo_organge] (9.5,3.7) -- ++(-0.5,-0.5)
                node[left,black,fill=white]{\scriptsize Remote UWB device};
            \draw[latex-, seemoo_organge] (1.85,5.2) -- ++(-0.5,-0.5)
                node[below,black,fill=white]{\scriptsize UWB antenna position};

            \draw [latex-latex, seemoo_organge]
            (9.42,5.2) coordinate (a) -- (1.85,5.2) coordinate (b) node[midway,above, black] {\scriptsize distance};
            \draw[latex-latex, seemoo_organge, dashed] (b) -- (1.85,8.5) coordinate (c);
            
            \pic[draw=seemoo_organge, -latex, angle eccentricity=1.2, angle radius=1.2cm] {angle=c--b--d};

            \node[] at (0.5,8.5) {$\theta$};
            
        \end{scope} 
    \end{tikzpicture}
    \caption{Top view of GWEn' setup that shows the base start position of $\theta=\SI{0}{\degree}$. In this position the arm is rotated to $\phi=0^\circ$.}\label{fig:GWEn_setup_top_view}
\end{figure}

\paragraph*{Antenna locations}
We identified the location of the \ac{uwb} transmission antenna in smartphones by using an oscilloscope and manually searching for the location with the strongest signal.
All manufacturers placed the antennas close to the rear cameras.
Since they all offer multiple receive antennas to determine the \ac{aoa} the receive antennas are often next to the transmission antenna~\cite{amaldevUWBTechApple2021}. We validated our claims by closely inspecting online device tear-downs~\cite{amaldevUWBTechApple2021,ifixitSamsungGalaxyS212021,dixonIPhone12122020,hughjeffreysPixelProTeardown2021}. All \ac{uwb} antennas are marked in \cref{fig:smartphone_antenna_placement,fig:dw3000_annotated}. 

\begin{figure}[!t]
    \includegraphics[width=\linewidth]{./figures/UWB_Antenna_location.pdf}
    \caption{Left to right: Apple iPhone 12 Pro, Samsung Galaxy S21 Ultra, and Google Pixel 6. Pro. The \ac{uwb} antenna locations are marked in black.}\label{fig:smartphone_antenna_placement}
\end{figure}

\begin{figure}[!t]
    \centering 
    \includegraphics[width=\linewidth]{figures/eval/devices/DW3000_annotated_2.pdf}
    \caption{A DW3000EVB attached to a STM32 Nucleo board.}\label{fig:dw3000_annotated}
\end{figure}

\subsection{Evaluation environments}

\begin{figure*}
    \centering
    \begin{subfigure}[t]{0.32\textwidth}
        \centering
        \includegraphics[width=\textwidth]{figures/environments/outside.jpg}
        \caption{Outside}\label{fig:outside_picture}
    \end{subfigure}
    \hfill
    \begin{subfigure}[t]{0.32\textwidth}
        \centering
        \includegraphics[width=\textwidth]{figures/environments/lab.jpg}
        \caption{Lab}\label{fig:lab_picture}
    \end{subfigure}
    \hfill
    \begin{subfigure}[t]{0.32\textwidth}
        \centering
        \includegraphics[width=\textwidth]{figures/environments/garage.jpg}
        \caption{Garage}\label{fig:garage_picture}
    \end{subfigure}
    \hfill

    \caption{Our three measurement environments.}\label{fig:environment_pictures}

\end{figure*}

We decided on three environments: outside, our lab, and a parking garage. A photo of each environment is shown in \cref{fig:environment_pictures}.  

\paragraph{Outside}
A location outside with no reflecting objects nearby is a good environment to measure a ground truth with a low amount of multipath interference. The soil-based ground cannot reflect signals well. 
There were no nearby Wi-Fi signals that could have interfered with our measurements.
The only nearby Wi-Fi signal was emitted by \ac{gwen} itself and limited to the \SI{2.4}{\giga \hertz} band, which does not interfere with our tested \ac{uwb} channels. The setup is similar to the one used by~\cite{malajnerUWBRangingAccuracy2015,jimenezComparingDecawaveBespoon2016}. 

\paragraph{Lab} 
Our lab offers a range of difficulties for \ac{uwb} measurements that can also occur in an office environment. Our measurement location was close to a glass whiteboard, several computers, monitors, and office furniture. 
These office environments are likely locations for future application of \ac{uwb}--enabled locks that can replace the need for special key fobs. 

\paragraph{Public parking garage}
A \ac{uwb}-based \ac{pke} system is already deployed in new car models. A parking garage is a natural environment to evaluate this. 
Nearby cars are reflecting \ac{uwb} pulses and might cause collisions between them. Identifying the correct first path might become difficult as demonstrated in~\cite{279984,singhSecurityAnalysisIEEE2021a}. 

\subsection{Device configuration}\label{ssec:dev_conf}
All devices need to be configured in a way that they start measuring the distance using \ac{uwb}. We, therefore, explain the different configuration options and how we initiated the measurements. 
As \ac{uwb} is a rather new technology in consumer devices, there is no fully open \ac{api} available by any smartphone manufacturer, which would have allowed fine-grained configuration of the \ac{uwb} chip. Therefore, the configuration of \ac{uwb} parameters, like channel, preamble, data rate and \ac{sts} may not be changed. In many cases, we were also not able to extract the used parameters after the measurement. 

\paragraph{Apple}
The Apple \textit{NearbyInteraction} framework~\cite{appleinc.NearbyInteractionApplea} can be used by any iOS app and can be configured to measure either the distance to another iPhone, Apple Watch or to a compatible \ac{uwb} third-party chip. We use an \textbf{iPhone 12 Pro} as the \ac{dut} and an \textbf{iPhone 12 mini} as the remote device. Both devices utilize the same generation of Apple's U1 chipset. 

\paragraph{Samsung}
Samsung and Google both implement the \ac{fira} protocol and would be theoretically compatible with the other chipsets and the iPhone. Unfortunately, no documentation on how to initialize ranging with any other device is not available publicly. 

The Samsung Galaxy S21 Ultra that we used for testing comes with an installed \textit{UWBTest} app. This app is hidden and can only be launched by opening the telephone app and typing \texttt{*\#UWBTEST\#} in the phone number field. 
We use this app to run our \ac{uwb} measurements between two Samsung devices because this has been the most reliable option on Android. 
We cannot determine which channel or preamble is used here. A \textbf{Samsung Galaxy S21 Ultras} is used as the \ac{dut} and the remote device.

\paragraph{Google}
So far, there are only the Google Pixel 6 Pro and Pixel 7 Pro available that integrate a \ac{uwb} chip. The only supported feature with \ac{uwb} is Android Nearby Share~\cite{googleinc.HowWeRe2022} and the Android 12 \ac{uwb} API~\cite{googleinc.UltrawidebandUWBCommunication}. 
We developed a small test application that is able to perform \ac{uwb} ranging between compatible Android devices. Unfortunately, this \ac{uwb} API needs up to $10$ and may fail to measure the distance completely. All our measurements are conducted using Android 12. 
We configured the app to use channel 9 and the preamble code 11 for all our measurements. 
A \textbf{Google Pixel 6 Pro} is used as the \ac{dut} and the remote device. Additionally, we also performed all measurements with a \textbf{Samsung Galaxy S21 Ultra} as the remote device. Both measurements are compared in \cref{ssec:different_remote_device}. 


\paragraph{Qorvo}
The Qorvo DWM3000EVB is a board that can be attached to an STM development board. By using the SDK provided by Qorvo, it's possible to write programs that can execute simple ranging measurements. Their sample code already provides most features needed for our evaluation. We configured the devices to use a static \ac{sts} and perform \ac{ds-twr} on channel 9 using preamble code 11. 

\subsection{Measurement result extraction}

Every device that we researched is able to generate results for the measurements and accessing them works differently on every device. As introduced in \cref{sec:gwen}, \ac{gwen} connects to all devices via USB to record measurement data while performing the measurement. We implemented separate parsers for each device that we support. 
If possible, we use raw measurements which are not enhanced by optimization algorithms. 

\paragraph{Apple} Apple devices often use extensive logging but omit sensitive data from the logs. To allow developers to debug their apps, certain logs can be activated using a \textit{debug profile}. We use the \textit{AirTag debug profile} to enable rich logs from \texttt{nearbyd}, a process which handles all \ac{uwb} related tasks~\cite{appleinc.ProfilesLogsBug}. 
With a USB connection to \ac{gwen}, we can fetch the logs and extract relevant measurement data. 
We extract raw measurements, as user-facing distance measurements are enhanced by custom machine learning algorithms.

\paragraph{Samsung} Samsung's devices also offer the option to increase the log verbosity by activating \textit{verbose vendor logging} in the developer settings. Then the smartphone logs all distance measurement data in a raw measurement and a calibrated measurement. Depending on the internal state the calibrated measurement may not be available.
\Ac{gwen} extracts both measurements using the \texttt{adb logcat} command line interface. 

\paragraph{Google} Google's smartphones log the \ac{uwb} distance measurements by default. Unfortunately, the logs are less verbose than Apple's or Samsung's. We, therefore, decide to use a custom logging format.  
We expect that all distance measurements on Google smartphones are raw measurements. Furthermore, we did not find any hints of a specially calibrated result or any machine learning-based enhancements. 

\paragraph{Qorvo} We mounted the DWM3000EVB to an STM32 development board. By programming the STM32 board, we are able to create a serial connection between the board and \ac{gwen}. Two-way communication allows \ac{gwen} to instruct the board to perform new measurements and to receive measurement results from the board. All results are raw results without any calibration. We tested the effect of calibrating the chips briefly, but we could not identify larger differences if compared to not calibrated devices.  

\section{Results}
\label{results}

\begin{figure*}[ht]
    \centering
    \includegraphics[scale=0.15,trim={0 2.5cm 0 5cm},clip]{images/aoi-single_burst}
    \caption{The time average peak Age of Information with burst and \gls{soa} loss values against the dynamic reliability logic for different network topologies.}
    \label{fig:aoi_burst}\vspace{-0.4cm}
\end{figure*}


This paper focuses on both transport layer and application layer metrics to determine the feasibility of dynamic reliability. For this, we have selected the session packet volume, as transmitted, retransmitted, lost and backlogged packets as \glspl{kpi} for the transport layer; while focusing on the \gls{aoi} for the application layer. The \gls{aoi} was chosen as a crucial indicator for the freshness of packets in real-time applications. More specifically, this work adopts the time average peak \gls{aoi} equation \cite{aoi_equation} depicted in Eq. \ref{aoi}, where $\Delta(r_{i+1})$ is the $i$th update at the time it was received at the server, for a session time period of $\tau$.

\begin{equation}
    \label{aoi}
    \gls{aoi}_\tau = \frac{1}{n-1}\sum_{i=1}^{n-1} \Delta(r_{i+1})
\end{equation}

We include a comparison between the vanilla QUIC implementation which does not enjoy the dynamic reliability extension, with a number of dynamic reliability policies. The tests were run a number of times for statistical significance, with the mean value of vanilla implementation used as a baseline for comparison. The topology utilised both random loss and bursty loss to explore the bounds of dynamic reliability. The \gls{soa} loss in the figures correspond to the loss values presented in Table. \ref{tab:path_char}, for ease of comparison between bursty and random loss scenarios.

\subsection{Transport-Layer KPIs}

To analyse the performance gain at the transport layer due to dynamic reliability, the volume of transmitted and backlogged packets is examined. The figures are in the form of boxplots, which take the vanilla implementation as a benchmark, depicted as the red dashed line.

As seen in Fig. \ref{fig:sent_burst}, the loss plays a crucial role in the performance of the reliability policies. The policies under random loss did incredibly well for the networks with a larger capacity, namely \gls{mmwave} and Sub-6~GHz, whereas for burst loss, the lower network capacities had a larger packet reduction. With the increase in burst loss, the behaviour of the set split reliable policies became unpredictable, if a reliable assignment happened to coincide with a burst loss, the number of transmitted packets increases, and vice versa. On the other hand, in smarter policies, such as Loss-Aware, the performance lightly matched the vanilla baseline, as the reliable assignment dominated the session to compensate for a higher burst loss. Not only that but, the burst loss also impacted the variance of the transmitted packets for the policies.

Unsurprisingly, the unreliable focused policy, 80-20 split, outperformed other policies for all topologies in random and bursty loss scenarios, with an approximate reduction of 80\%. That being said, the majority of the policies reduced the transmitted packets on the link by approximately 70\% for random loss, while the reduction started at $\approx 15\%$ and decreased as the loss increased for the burst loss scenario.

The retransmitted and lost packets, not shown due to space limitations, followed the same trend as the transmitted packets for the random loss scenarios. However, for the burst loss scenarios, the larger capacity networks had a lower reduction in the retransmitted and lost packets. This can be seen as a favorable outcome since the lower capacity networks are scarce on resources. It is important to note that the Loss-Aware policy mimicked the vanilla approach as the burst loss increased, signifying the overwhelming appointment of reliable packets in adapting to the harsh burst loss conditions.
 
Alternatively, Fig. \ref{fig:backlog_burst} clearly shows a stark comparison between the policies and loss scenario in the reduction of the backlogged packets. The Loss-Aware policy for random loss scenario reduced the backlogged packets by up to 50\%, beating all other policies by approximately 30\%. Furthermore, it is clear that the unreliability focused policies resulted in the lowest backlog for the session. In comparison, we notice that the burst loss and the backlogged frequency have a positive correlation, where the maximum reduction of the backlogged packets for the policies is at most 20\%. Much like the transmitted packets, the probability of a burst loss occurrence plays a vital role in the number of retransmissions sent and by extension the number of backlogged packets. Thus, we can conclude that the stress placed on the buffer is a result of the reliable packets which is tightly coupled with the congestion on the session. Whereas, unreliable focused policies did not encounter such a phenomenon regardless if it was experiencing a burst loss.


\subsection{Application-Layer KPIs}

The feasibility of dynamic reliability for real-time applications can be determined by the \gls{aoi}, with comparison across different topologies and policies. If we take a strict approach and consider anything below $10$~ms is real-time \cite{real-time}, then all the reliability policies passed that requirement, which is attractive for real-time applications, as shown in Fig. \ref{fig:aoi_burst}. Utilising the median as an estimate of the runs, the policies in the WLAN and Sub-6~GHz topology with random loss floated around $4-5$~ms with negligible difference, while the \gls{aoi} for \gls{mmwave} was $\approx 2-3$~ms. It is clear that the \gls{aoi} and the network capacity have a negative correlation, as the network capacity decreases, the \gls{aoi} increases. The same correlation is extended to the bursty loss scenarios, where \gls{mmwave} dominated the other topologies. That being said, it is crucial to note that the \gls{aoi} for the reliability policies is often slightly better than or equal to the \gls{aoi} of the vanilla implementation, proving that dynamic reliability reduces the congestion of the session at no cost to the \gls{aoi}.


We provide some comments on the growth conditions which constituted the majority of our analysis in sections \ref{sec:Hmixing} and \ref{sec:Hsigma}. In the simplest cases of Lemma \ref{lemma:unstableGrowth}, growth was established in an analogous fashion to the old one-step expansion condition (\ref{eq:oldOneStepExpansion}), finding the relevant Jacobians $M_j$ and checking that their expansion factors $K(M_j)$ satisfy
\begin{equation}
    \label{eq:discussionOneStep}
    \sum_j \frac{1}{K(M_j)} <1.
\end{equation}
For the more complicated cases, the inductive method used to establish growth near the accumulation points in Lemma \ref{lemma:unstableGrowth} and the weakened one-step expansion condition (\ref{eq:oneStep}) both address the same fundamental issue: the splitting of unstable curves by singularities into an unbounded number of small components. They circumvent this obstacle in rather different ways, however. While (\ref{eq:oneStep}) generalises (\ref{eq:discussionOneStep}) to ensure an growth of unstable curves `on average' (see \cite{chernov_statistical_2009} for a precise statement), our inductive method is a more direct adaptation of (\ref{eq:discussionOneStep}), using it to generate contradictory geometric conditions which a hypothetical non-growing unstable curve must satisfy. It may be possible to prove Theorem \ref{sec:Hmixing} using (\ref{eq:oneStep}) as the basis for growth. Since we required (\ref{eq:oneStep}) anyway for proving Theorem \ref{thm:HsigmaExp}, this could potentially condense our analysis, but only to a minor extent. A convenience of the method used in section \ref{sec:Hmixing} is that, by way of the `simple intersection' property, it naturally gives geometric information on the images of manifolds, useful for proving the property \textbf{(M)} of Theorem \ref{thm:katok-strelcyn}.

We expect that essentially analogous analysis can be applied to establish mixing properties in a wide class of piecewise linear non-uniformly hyperbolic maps, including those (like the OTM) which sit on the boundary of ergodicity and beyond. While we have relied on the precise partition structure of $H_\sigma$, its fundamental feature (self-similar sequences of elements $A^k$, sharing boundaries with its neighbours $A^{k-1},A^{k+1}$ and accumulating onto some point $p$) is quite typical to return map systems. See, for example, those of various stadium billiards \cite{chernov_chaotic_2006,chernov_improved_2008,chernov_statistical_2009} and LTMs \cite{springham_polynomial_2014}. Indeed, the same method can be used to prove the Bernoulli property for non-monotonic LTMs \cite{myers_hill_mixing_2022}, where monotonicity of the manifold images cannot be assumed and the classical argument \cite{sturman_mathematical_2006} fails. The OTM is the pointwise limit of these maps as the boundary shrinks to null measure. It further has utility in proving growth conditions for maps which are uniformly hyperbolic but possess regions $A_j$ where the hyperbolicity is very weak, signified by $K(M_j) \approx 1$, so that (\ref{eq:discussionOneStep}) fails. Typically this leads to suboptimal bounds on mixing windows, see e.g. \cite{wojtkowski_model_1981,przytycki_ergodicity_1983,myers_hill_family_2022}. The map $H_{(\eta,\eta)}$ for $\eta \approx 1/2$ is another example, possessing weak hyperbolicity over $A_2, A_3$. Letting $\varepsilon = |\eta-1/2|>0$, there is an upper bound $N = N(\varepsilon)$ on escape times from the intersections $A_2\cap \sigma, A_3 \cap \sigma$. The growth lemma then follows by applying the inductive step roughly $N$ times and can be established for arbitrarily small $\varepsilon$, opening the door to establishing optimal mixing windows.

The above gives two examples of piecewise linear perturbations to $H$ where mixing with respect to Lebesgue is preserved and our methods can be applied. Nonlinear perturbations to the shear profiles complicate the analysis in several ways. Firstly as the map's Jacobians takes on a broader range of values, cone invariance becomes an increasingly harder condition to establish. Cones must be widened, giving looser bounds on expansion factors, which may already be weak due to new regions of weaker stretching. This, together with the change from polygonal to curvilinear return time partition elements and nonlinear local manifolds, adds some complexity to showing growth conditions. This does not rule out certain (small) nonlinear perturbations however. There is some leeway in the inequalities which govern cone invariance and growth of local manifolds, the latter of which is not too dissimilar from the piecewise linear setting (see Lemmas \ref{lemma:piecewiseApprox}, \ref{lemma:componentLength}). Certain small perturbations would not alter the \emph{topological} structure of the return time partition, i.e. which elements share boundaries, the key information needed for setting up the induction. Finally while the partition elements would no longer be polygonal, only coarse geometric information is required for verifying each inductive step. Following the above, a potential perturbation could be to replace the linear portions of each shear by a cubic, perturbing the tent profile
\[  f(t) = \begin{cases} 2t & 0 \leq t \leq 1/2, \\ 2(1-t) & 1/2 \leq t \leq 1 ,\end{cases} \]
of the OTM shears to
\[  f_a(t) = \begin{cases} \frac{1}{8} t \left(16 - a + 6at - 8at^{2} \right) & 0 \leq t \leq 1/2, \\ \frac{1}{8}\left(1-t\right)\left( 16 - a + 6a\left(1-t\right) - 8a\left(1-t\right)^{2}\right)  & 1/2 \leq t \leq 1, \end{cases}   \]
for $a>0$. For small enough $a$ the gradient range $f'(t)$ is restricted to small neighbourhoods of $\{ 2, -2\}$ and the escape time partition retains a similar structure. We illustrate this in Figure \ref{fig:perturbations}, showing escapes from the square $S_3$ under the map $G \circ F$, equivalent to escapes from the perturbed $A_3$ under the $G \circ F$, but with a cleaner geometry for comparison. When $a$ is too large the analogy to the OTM breaks down. At $a=16$ the map is twice differentiable everywhere and features a new source of slowed mixing, the Jacobian is the identity at the corner points $x,y \in \{  0, 1/2 \}$ giving locally parabolic behaviour (visible in the escape time partition). 

\begin{figure}
    \centering
    \includegraphics[width=0.24 \linewidth]{0.png}
    \includegraphics[width=0.24 \linewidth]{4.png}
    \includegraphics[width=0.24 \linewidth]{8.png}
    \includegraphics[width=0.24 \linewidth]{16.png}
    \caption{Partition of escape times from $S_3$ under the mapping $F \circ G$ for $a= 0,4,8,16$. }
    \label{fig:perturbations}
\end{figure}

\section{Conclusion}\label{sec:conclusion}
In this work, we focus on addressing the fundamental challenge of OOD detection tasks, which is how to fully understand the semantic discrepancy between the ID/OOD samples. We reveal that the key to success in the realistic SCOOD task is to allocate as many ID samples in the unlabeled set correctly as possible. To this end, we propose a novel uncertainty-aware optimal transport scheme that introduces class-specific energy scores as guidance for effective label assignment. Experimental results show that our method achieves better performance than previous state-of-the-art methods on SCOOD benchmarks.

\textbf{Limitations.} In addition to temperature scaling, other techniques such as feature clipping applied in ReAct~\cite{sun2021react} also enhance the performance of energy score, so how to obtain an OOD score that best fits the SCOOD task can be further explored. Moreover, a setting highly related to SCOOD has been proposed in \cite{katz2022training} and formulated as a constrained optimization problem. We will also theoretically analyze these practical OOD settings in our feature work.

% \section*{Acknowledgments}
\textbf{Acknowledgments.} 
This work is supported by National Key R\&D Program of China under Grant 2020AAA0105701, National Natural Science Foundation of China (NSFC) under Grants 61872327, Major Special Science and Technology Project of Anhui, National Natural Science Foundation of China (62033012) and Ant Group through Ant Research Intern Program.


% Acknowledgements 
\section*{Acknowledgment}
\noindent This work has been funded by the German Federal Ministry of Education and Research and the Hessen State Ministry for Higher Education, Research and the Arts within their joint support of the National Research Center for Applied Cybersecurity ATHENE and by the LOEWE initiative (Hesse, Germany) within the emergenCITY center.


\bibliographystyle{IEEEtran} 
\bibliography{main.bib}
   
 
% \section{Appendix for Proofs}

\paragraph{Proof of Theorem \ref{thm:main}.}

\begin{proof}
\label{proof:main}
Our proof has two steps. In Step 1, we will show that SimCLR is equivalent to minimizing the cross entropy loss defined in Eqn.~(\ref{eqn:cross-entropy}). 
In Step 2, we will show  that minimizing the cross-entropy loss 
is equivalent to spectral clustering on $\bfpi$. 
Combining the two steps together, we have proved our theorem. 

\textbf{Step 1: } SimCLR is equivalent to minimizing the cross entropy loss.

The cross-entropy loss takes expectation over 
$\bfW_\bfX\sim \mathbb{P}(\cdot ; \bfpi)$, 
which means $\bfW_\bfX$ has exactly one non-zero entry in each row $i$. By Lemma~\ref{lem:multinomial}, we know every row $i$ of $\bfW_\bfX$ is independent of other rows. Moreover, 
$\bfW_{\bfX,i}\sim \mathcal{M}(1, \bfpi_i/\sum_j \bfpi_{i,j})=\mathcal{M}(1, \bfpi_i)$, because $\bfpi_i$ itself is a probability distribution.
Similarly, we know $\bfW_\bfZ$ also has the row-independent property by sampling over $\mathbb{P}(\cdot;\bfK_\bfZ)$.
Therefore, by Lemma~\ref{lem:cross_split}, we know Eqn.~(\ref{eqn:cross-entropy}) is equivalent to:
\[
 -\sum_{i=1}^n \mathbb{E}_{\bfW_{\bfX,i}}[\log \mathbb{P}(\bfW_{\bfZ,i}=\bfW_{\bfX,i};\bfK_\bfZ)],
\]

This expression takes expectation over $\bfW_{\bfX,i}$ for the given row $i$. Notice that 
$\bfW_{\bfX,i}$ has exactly one non-zero entry, which equals $1$ (same for $\bfW_{\bfZ,i}$). 
As a result
we expand the above expression to be:
\begin{equation}
 -\sum_{i=1}^n \sum_{j\neq i} \Pr(\bfW_{\bfX,i,j}=1)\log \Pr(\bfW_{\bfZ,i,j}=1).
\label{eqn:detailed-expansion}    
\end{equation}


By Lemma~\ref{lem:multinomial}, $\Pr(\bfW_{\bfZ,i,j}=1)=\bfK_{\bfZ,i,j}/\|\bfK_{\bfZ,i}\|_1$ for $j\neq i$. Recall that $\bfK_\bfZ=(k(\bfZ_i-\bfZ_j))_{(i,j)\in[n]^2}$, which means 
$\bfK_{\bfZ,i,j}/\|\bfK_{\bfZ,i}\|_1=\frac{\exp(-\|\bfZ_i-\bfZ_j\|^2/{2\tau})}{\sum_{k\neq i}
\exp(-\|\bfZ_i-\bfZ_k\|^2/{2\tau})
}$ for $j\neq i$, when $k$ is the Gaussian kernel with variance $\tau$. 

Notice that $\bfZ_i=f(\bfX_i)$, so we know
\begin{equation}
-\log \Pr(\bfW_{\bfZ,i,j}=1)=
-\log \frac{\exp(-\|f(\bfX_i)-f(\bfX_j)\|^2/{2\tau})}{\sum_{k\neq i}
\exp(-\|f(\bfX_i)-f(\bfX_k)\|^2/{2\tau}),
}
\label{eqn:infonce-equivalence}    
\end{equation}


The right hand side is exactly the InfoNCE loss defined in Eqn.~(\ref{eqn:infonce}).
Inserting Eqn.~(\ref{eqn:infonce-equivalence}) into Eqn.~(\ref{eqn:detailed-expansion}), we get the SimCLR algorithm, which first samples augmentation pairs $(i,j)$ with $\Pr(\bfW_{\bfX,i,j}=1)$ for each row $i$, and then optimize the InfoNCE loss. 

\textbf{Step 2: } minimizing the cross entropy loss 
is equivalent to spectral clustering on $\bfpi$.


By Lemma~\ref{lem:convert_to_spectral}, we may further convert the loss to 
\begin{equation}
\label{eqn:main-theorem-repul-attr}
\min_{\bfZ}
-\sum_{(i,j)\in [n]^2} \mathbf{P}_{i,j}
\log k (\bfZ_i-\bfZ_j)+\log \mathbf{R}(\bfZ).
\end{equation}
Since $k$ is the Gaussian kernel, this reduces to \[
\min_\bfZ \mathrm{tr}(\bfZ^\top \mathbf{L}(\bfpi) \bfZ)
+\log \mathbf{R}(\bfZ),
\]

where we use the fact that $\mathbb{E}_{\bfW_\bfX\sim \mathbb{P}(\cdot; \bfpi)}[\mathbf{L}(\bfW_\bfX)]
=\mathbf{L}(\bfpi)
$, because the Laplacian operator is linear and $
\mathbb{E}_{\bfW_\bfX\sim \mathbb{P}(\cdot; \bfpi)}(\bfW_\bfX)=\bfpi
$.
\end{proof}

\paragraph{Proof of Theorem \ref{thm:clip}.}
\begin{proof}
Since $\bfW_\bfX\sim \mathbb{P}(\cdot;\bfpi_{\mathbf{A}, \mathbf{B}})$, we know 
$\bfW_\bfX$ has exactly one non-zero entry in each row, denoting the pair that got sampled. 
A notable difference compared to the previous proof is we now have $n_\mathcal{A}+n_\mathcal{B}$ objects in our graph. CLIP deals with this by taking a mini-batch of size $2N$, 
such that $n_\mathcal{A}=n_\mathcal{B}=N$, and adding the $2N$ InfoNCE losses together. We label the objects in $\mathcal{A}$ as $[n_\mathcal{A}]$, and the objects in $\mathcal{B}$ as $\{n_\mathcal{A}+1, \cdots, n_\mathcal{A}+n_\mathcal{B}\}$. 

Notice that $\bfpi_{\mathbf{A}, \mathbf{B}}$ is a bipartite graph, so the edges of objects in $\mathcal{A}$ will only connect to object in $\mathcal{B}$ and vice versa. We can define the similarity matrix in $\cZ$ as $\bfK_\bfZ$, 
where $\bfK_\bfZ(i, j+n_\mathcal{A})=\bfK_\bfZ(j+n_\mathcal{A},i)= k(\bfZ_i-\bfZ_j)$ for $i\in [n_\mathcal{A}], j\in [n_\mathcal{B}]$, and otherwise we set $\bfK_\bfZ(i,j)=0$. 
The rest is same as the previous proof. 
\end{proof}

\paragraph{Proof of Theorem \ref{thm:exponential}.}

\begin{proof}
\label{proof:exponential}
Since the objective function consists of a linear term combined with an entropy regularization, which is a strongly concave function, the maximization problem is a convex optimization problem. Owing to the implicit constraints provided by the entropy function, the problem is equivalent to having only the equality constraint. We then introduce the Lagrangian multiplier $\lambda$ and obtain the following relaxed problem:

$$
\widetilde{E}(\boldsymbol{\alpha})=\psi_{1}-\sum_{i=1}^n \alpha_{i} \psi_{i}+\tau \sum_{i=1}^n \alpha_{i}\log \alpha_{i}+\lambda\left(\boldsymbol{\alpha}^{\top} \mathbf{1}_n-1\right).
$$

As the relaxed problem is unconstrained, taking the derivative with respect to $\alpha_{i}$ yields

$$
\frac{\partial \widetilde{E}(\boldsymbol{\alpha})}{\partial \alpha_{i}}=-\psi_{i}+\tau\left(\log \alpha_{i}+\alpha_{i} \frac{1}{\alpha_{i}}\right)+\lambda=0.
$$

Solving the above equation implies that $\alpha_{i}$ takes the form
$
\alpha_{i}=\exp \left(\frac{1}{\tau} \psi_{i}\right) \exp \left(\frac{-\lambda}{\tau}-1\right).
$ Since $\alpha_{i}$ lies on the probability simplex, the optimal $\alpha_{i}$ is explicitly given by
$
\alpha^{*}_{i}=\frac{\exp \left(\frac{1}{\tau} \psi_{i}\right)}{\sum_{i^{\prime}=1}^n \exp \left(\frac{1}{\tau} \psi_{i^{\prime}}\right)} .
$ Substituting the optimal point into the objective function, we obtain
$$
\begin{aligned}
E\left(\boldsymbol{\alpha}^*\right)  &=\psi_1-\sum_{i=1}^n \frac{\exp \left(\frac{1}{\tau} \psi_{i}\right)}{\sum_{i^{\prime}=1}^n \exp \left(\frac{1}{\tau} \psi_{i^{\prime}}\right)} \psi_{i}+\tau \sum_{i=1}^n \frac{\exp \left(\frac{1}{\tau} \psi_{i}\right)}{\sum_{i^{\prime}=1}^n \exp \left(\frac{1}{\tau} \psi_{i^{\prime}}\right)}\log \frac{\exp \left(\frac{1}{\tau} \psi_{i}\right)}{\sum_{i^{\prime}=1}^n \exp \left(\frac{1}{\tau} \psi_{i^{\prime}}\right)} \\
& =\psi_1 - \tau \log \left(\sum_{i=1}^n \exp \left(\frac{1}{\tau} \psi_{i}\right)\right).
\end{aligned}
$$
Thus, the Lagrangian dual function is given by
\begin{equation*}
-E\left(\boldsymbol{\alpha}^*\right)= -\tau \log \frac{\exp \left(\frac{1}{\tau} \psi_{1}\right)}{\sum_{i=1}^n \exp \left(\frac{1}{\tau} \psi_{i}\right)}.\qedhere
\end{equation*}
\end{proof}



\section{More on Experiments} \label{section: experiment_details}

\paragraph{CIFAR-10 and CIFAR-100} CIFAR-10 ~\citep{krizhevsky2009learning} and CIFAR-100 ~\citep{krizhevsky2009learning} are well-known classic image classification datasets. Both CIFAR-10 and CIFAR-100 contain a total of 60k $32 \times 32$ labeled images of different classes, with 50k for training and 10k for testing. CIFAR-10 is similar to CIFAR-100, except there are 10 different classes in CIFAR-10 and 100 classes in CIFAR-100.

\paragraph{TinyImageNet} TinyImageNet ~\citep{le2015tiny} is a subset of ImageNet ~\citep{deng2009imagenet}. There are 200 different object classes in TinyImageNet, with 500 training images, 50 validation images, and 50 test images for each class. All the images in TinyImageNet are colored and labeled with a size of $64 \times 64$.

\textbf{Pseudo-code.} Algorithm \ref{alg:Training Procedure} presents the pseudo-code for our empirical training procedure.

\begin{algorithm}[!htbp]
\caption{Training Procedure}
\label{alg:Training Procedure}
\begin{algorithmic}[1]
\REQUIRE trainable encoder network $f$, batch size $N$, augmentation strategy \textit{aug}, loss function $L$ with hyperparameters \textit{args}
\FOR {sampled minibatch ${x_i}_{i=1}^N$}
\FORALL{$i \in { 1, ..., N }$}
\STATE draw two augmentations $t_i = \textit{aug}\left(x_i\right) $, $t_i' = \textit{aug}\left(x_i\right) $
\STATE $z_i = f\left(t_i\right)$, $z_i' = f\left(t_i'\right)$
\ENDFOR
\STATE compute loss $\mathcal{L} = L(N, z, z', \textit{args})$
\STATE update encoder network $f$ to minimize $\mathcal{L}$
\ENDFOR
\STATE \textbf{Return} encoder network $f$
\end{algorithmic}
\end{algorithm}

We also provide the pseudo-code for our core loss function used in the training procedure in Algorithm \ref{alg:Core loss}. The pseudo-code is almost identical to SimCLR's loss function, with the exception of an extra parameter $\gamma$.

\begin{algorithm}[!htbp]
\caption{Core loss function $\mathcal{C}$}
\label{alg:Core loss}
\begin{algorithmic}[1]
\REQUIRE batch size $N$, two encoded minibatches $z_1, z_2$, $\gamma$, temperature $\tau$
\STATE $z = \textit{concat}\left(z_1, z_2\right)$
\FOR {$i \in {1, ..., 2N }, j \in {1, ..., 2N}$ }
\STATE $s_{i,j} = \Vert z_i - z_j \Vert_2^{\gamma}$
\ENDFOR
\STATE \textbf{define} $l(i, j)$ \textbf{as} $l(i, j) = - \log \frac{exp\left(s_{i,j}/\tau \right)}{\sum_{k=1}^{2N} \mathbf{1}{[k \ne i]} exp\left(s{i, j} / \tau \right)} $
\STATE \textbf{Return} $\frac{1}{2N} \sum_{k=1}^N\left[l(i, i+N) + l(i+N, i)\right]$
\end{algorithmic}
\end{algorithm}

Utilizing the core loss function $\mathcal{C}$, we can define all kernel loss functions used in our experiments in Table \ref{table: loss definition}. For all $z_i \in z$ with even dimensions $n$, we define $z_{L_i} = z_i\left[0:n/2\right]$ and $z_{R_i} = z_i\left[n/2:n\right]$.

\begin{table}[ht]
\centering
\begin{tabular}{{@{}l|l@{}}}
Kernel  &  Loss function \\ \midrule
Laplacian & $\mathcal{C}\left(N, z, z', \gamma=1, \tau\right)$\\ \midrule
Sum       & $\lambda * \mathcal{C}\left(N, z, z', \gamma=1, \tau_1\right) + (1-\lambda) * \mathcal{C}\left(N, z, z', \gamma=2, \tau_2\right)$  \\ \midrule
Concatenation Sum&$\lambda * \mathcal{C}\left(N, z_L, z'_L, \gamma=1, \tau_1\right) + (1-\lambda) * \mathcal{C}\left(N, z_R, z'_R, \gamma=2, \tau_2\right)$\\ \midrule
$\gamma = 0.5$ & $\mathcal{C}\left(N, z, z', \gamma=0.5, \tau\right)$          \\ 

\end{tabular}

\caption{Definition of kernel loss functions in our experiments}
\label {table: loss definition}
\end{table}

\textbf{Baselines.} We reproduce the SimCLR algorithm using PyTorch Lightning~\citep{PytorchLightning}.

\textbf{Encoder details.}
The encoder $f$ consists of a backbone network and a projection network. We employ ResNet50~\citep{ResNet} as the backbone and a 2-layer MLP (connected by a batch normalization~\citep{ioffe2015batch} layer and a ReLU \cite{nair2010rectified} layer) with hidden dimensions 2048 and output dimensions 128 (or 256 in the concatenation kernel case).

\textbf{Encoder hyperparameter tuning.}
For each encoder training case, we randomly sample 500 hyperparameter groups (sample details are shown in Table \ref{table: Hyperparameter sample}) and train these samples simultaneously using Ray Tune ~\citep{RayTune}, with the ASHA scheduler~\citep{li2018massively}. Ultimately, the hyperparameter group that maximizes the online validation accuracy (integrated in PyTorch Lightning) within 5000 validation steps is chosen for the given encoder training case.

\begin{table}[ht]
\centering

\begin{tabular}{@{}l|l|l@{}}
\midrule
Hyperparameter  & Sample Range & Sample Strategy \\ \midrule
start learning rate & $\left[10^{-2}, 10\right]$ & log uniform \\ \midrule
$\lambda$       & $\left[0, 1\right]$ & uniform \\ \midrule
$\tau$, $\tau_1$, $\tau_2$ & $\left[0, 1\right]$ & log uniform \\ \midrule
\end{tabular}

\caption{Hyperparameters sample strategy}
\label {table: Hyperparameter sample}
\end{table}

\textbf{Encoder training.} 
We train each encoder using the LARS optimizer~\citep{LARSOptimizer}, LambdaLR Scheduler in PyTorch, momentum 0.9, weight decay $10^{-6}$, batch size 256, and the aforementioned hyperparameters for 400 epochs on a single A-100 GPU.

\textbf{Image transformation.} The image transformation strategy, including augmentation, is identical to the default transformation strategy provided by PyTorch Lightning.

\textbf{Linear evaluation.}
The linear head is trained using the SGD optimizer with a cosine learning rate scheduler, batch size 64, and weight decay $10^{-6}$ for 100 epochs. The learning rate starts at $0.3$ and ends at $0$.

\textbf{Moco Experiments.} We also tested our method based on MoCo~\citep{he2019moco}. The results are summarized in Table \ref{tab:results-moco}. Here we choose ResNet18~\citep{ResNet} as the backbone and set a temperature of $0.1$ as default. For our simple sum kernel, we set $\lambda=0.8$. The results show that our method outperforms the original MoCo method.

\begin{table}[thb]
\centering
\caption{MoCo Experiment Results on CIFAR-10 and CIFAR-100.}
\label{tab:results-moco}
\resizebox{\textwidth}{!}{%
\begin{tabular}{@{}c|ccc|ccc@{}}
\toprule
\multirow{3}{*}{Method} & \multicolumn{3}{c|}{CIFAR-10} & \multicolumn{3}{c}{CIFAR-100} \\ \cmidrule(lr){2-4} \cmidrule(lr){5-7} 
                        & 200 epochs & 400 epochs    & 1000 epochs   & 200 epochs & 400 epochs & 1000 epochs         \\ \midrule
MoCo (repro.)         & $76.41 \pm 0.12$    & $80.01 \pm 0.15$          & $84.45 \pm 0.08$    & $\mathbf{47.02 \pm 0.11}$ & $52.50 \pm 0.07$ & $57.62 \pm 0.15$            \\
\midrule
Laplacian Kernel        & ${78.09 \pm 0.10}$    & $\mathbf{83.85 \pm 0.09}$          & $\mathbf{88.34 \pm 0.16}$    & $46.12 \pm 0.22$   & $53.44 \pm 0.17$ & $59.10 \pm 0.14$        \\
Simple Sum Kernel & $\mathbf{78.12 \pm 0.15}$   & $83.23 \pm 0.18$ & $87.50 \pm 0.20$ & $46.65 \pm 0.06$ & $\mathbf{53.62 \pm 0.19}$ & $\mathbf{59.83 \pm 0.12}$\\
\bottomrule
\end{tabular}
}
\end{table}



\section{More Experiments on Synthetic Data}


Consider a scenario with $n$ clusters, each containing $k$ vertices. Let the probability of vertices $u$ and $v$ from the same cluster belonging to $\bfpi$ be $p$. Conversely, for vertices $u$ and $v$ from different clusters, let the probability of belonging to $\pi$ be $q$. We generate the graph $\bfpi$ randomly, based on $p$ and $q$. We experiment with values of $k=100$ and $n=6$ for ease of visualization, embedding all points in a two-dimensional space. Each vertex's initial position originates from a normal distribution. In each iteration, we sample a subgraph of $\bfpi$ uniformly, ensuring each vertex has an out-degree of $1$. We then optimize the corresponding vectors using InfoNCE loss with an SGD optimizer and iterate until convergence. Our experimental setup consists of an SGD learning rate of $1$, an InfoNCE loss temperature of $0.5$, and a batch size of $50$. We evaluate two scenarios with different $p$ and $q$ values: $p=1$, $q=0$, and $p=0.75$, $q=0.2$. The results of these experiments are visualized in Figure \ref{fig:vis-spectral-cluster}. The obtained embeddings exhibit the hallmark pattern of spectral clustering of graph $\bfpi$.

\begin{figure}[!tb]
\centering
\subfigure{
\includegraphics[width=1\textwidth]{Figures/cluster_pi.png}
\label{fig:vis-cluster}
}
\subfigure{
\includegraphics[width=1\textwidth]{Figures/noised_cluster_pi.png}
\label{fig:vis-noised-cluster}
}
\caption{Visualizations of the optimization process using InfoNCE Loss on the vectors corresponding to $\bfpi$. Points of identical color belong to the same cluster within $\bfpi$. To showcase the internal structure of $\bfpi$, we randomly select 10 vertices from each cluster to display the edge distribution of $\bfpi$.}
\label{fig:vis-spectral-cluster}
\end{figure}


\begin{acronym}
    \acro{LDA}{\emph{Latent Dirichlet Allocation}}
    \acro{CMT}{\emph{Conference Management Toolkit}}
    \acro{TPMS}{\emph{The Toronto Paper Matching System}}
    \acro{MCMF}{\emph{MinCost-MaxFlow}}
\end{acronym}

% \begin{IEEEbiographynophoto}{Jane Doe}
% Biography text here without a photo.
% \end{IEEEbiographynophoto}

\begin{IEEEbiography}[{\includegraphics[width=1in,height=1.25in,clip,keepaspectratio]{figures/biographies/aheinrich.jpg}}]{Alexander Heinrich}
  received the B.Sc., M.Sc.\ degrees from the Technical University of Darmstadt, Darmstadt, Germany, in 2017 and 2019, respectively.

  He is a Ph.D. candidate with the Secure Mobile Networking Lab, Technical University of Darmstadt. His research is on the security and privacy of wireless protocols with a focus on location-aware systems. 

  Mr. Heinrich received several best demo awards and the best paper award at the ACM Conference on Security and Privacy in Wireless and Mobile Networks.  
\end{IEEEbiography}


\begin{IEEEbiography}[{\includegraphics[width=1in,height=1.25in,clip,keepaspectratio]{figures/biographies/skrollmann.jpg}}]{Sören Krollmann}
  received the M.Sc.\ degree from  Technical University of Darmstadt, Darmstadt, Germany in 2022. 
  
  He worked as a scientific assistant with the Secure Mobile Networking Lab, Technical University of Darmstadt. His research focused on distance measurement using Ultra-Wide Band technology. 
  
  Mr. Krollmann currently works in industry as a cyber security consultant and supports customers on their way to more security.
\end{IEEEbiography}

\begin{IEEEbiographynophoto}{Florentin Putz}
  received the B.Sc.\ and M.Sc.\ degrees from the Technical University of Darmstadt, Darmstadt, Germany, in 2016 and 2019, respectively.

  He is a Ph.D. candidate with the Secure Mobile Networking Lab, TU Darmstadt. His research focuses on deployable and usable security mechanisms for smartphones and the Internet of Things.
  
  Mr.\ Putz received several awards, including the Best Student Award from TU Darmstadt's electrical engineering department as well as the KuVS Award 2020 for the best master's thesis in communications and distributed systems in Germany.
\end{IEEEbiographynophoto}

\begin{IEEEbiography}[{\includegraphics[width=1in,height=1.25in,clip,keepaspectratio]{figures/biographies/mhollick.jpg}}]{Matthias Hollick}
  received his Ph.D.\ degree from TU Darmstadt, Darmstadt, Germany, in 2004.
  
  He is currently heading the Secure Mobile Networking Lab, Computer Science Department, Technical University of Darmstadt. He has been researching and teaching at TU Darmstadt, Universidad Carlos III de Madrid, Getafe, Spain, and the University of Illinois at Urbana–Champaign, Champaign, IL, USA. His research focus is on resilient, secure, privacy-preserving, and quality- of-service-aware communication for mobile and
  wireless systems and networks.  
\end{IEEEbiography}

\end{document}


