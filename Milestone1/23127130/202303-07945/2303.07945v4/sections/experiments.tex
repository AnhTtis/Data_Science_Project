\section{Experiments}

\begin{figure*}
\vspace{0.8em}
\begin{center}
\makebox[0.12\textwidth]{\colorbox{pink}{\textbf{Training video}} A woman is on the swing.}\\

\includegraphics[width=0.11\textwidth]{figures/swing/training/00000.pdf}
\includegraphics[width=0.11\textwidth]{figures/swing/training/00001.pdf}
\includegraphics[width=0.11\textwidth]{figures/swing/training/00002.pdf}
\includegraphics[width=0.11\textwidth]{figures/swing/training/00003.pdf}
\includegraphics[width=0.11\textwidth]{figures/swing/training/00004.pdf}
\includegraphics[width=0.11\textwidth]{figures/swing/training/00005.pdf}
\includegraphics[width=0.11\textwidth]{figures/swing/training/00006.pdf}
\includegraphics[width=0.11\textwidth]{figures/swing/training/00007.pdf}


\makebox[0.12\textwidth]{\colorbox{yellow}{\textbf{Edit-A-Video (Ours)}} A \textcolor{blue}{\textbf{watercolor}} painting that a woman is on the swing.}\\

\includegraphics[width=0.11\textwidth]{figures/swing/watercolor/00000.pdf}
\includegraphics[width=0.11\textwidth]{figures/swing/watercolor/00001.pdf}
\includegraphics[width=0.11\textwidth]{figures/swing/watercolor/00002.pdf}
\includegraphics[width=0.11\textwidth]{figures/swing/watercolor/00003.pdf}
\includegraphics[width=0.11\textwidth]{figures/swing/watercolor/00004.pdf}
\includegraphics[width=0.11\textwidth]{figures/swing/watercolor/00005.pdf}
\includegraphics[width=0.11\textwidth]{figures/swing/watercolor/00006.pdf}
\includegraphics[width=0.11\textwidth]{figures/swing/watercolor/00007.pdf}

\makebox[0.12\textwidth]{\colorbox{yellow}{\textbf{Edit-A-Video (Ours)}} A woman is on the swing, \textcolor{blue}{\textbf{Van Gogh}} style.}\\

\includegraphics[width=0.11\textwidth]{figures/swing/vangogh/00000.pdf}
\includegraphics[width=0.11\textwidth]{figures/swing/vangogh/00001.pdf}
\includegraphics[width=0.11\textwidth]{figures/swing/vangogh/00002.pdf}
\includegraphics[width=0.11\textwidth]{figures/swing/vangogh/00003.pdf}
\includegraphics[width=0.11\textwidth]{figures/swing/vangogh/00004.pdf}
\includegraphics[width=0.11\textwidth]{figures/swing/vangogh/00005.pdf}
\includegraphics[width=0.11\textwidth]{figures/swing/vangogh/00006.pdf}
\includegraphics[width=0.11\textwidth]{figures/swing/vangogh/00007.pdf}

\makebox[0.12\textwidth]{\colorbox{yellow}{\textbf{Edit-A-Video (Ours)}} A \textcolor{blue}{\textbf{man}} is on the swing}\\

\includegraphics[width=0.11\textwidth]{figures/swing/man/00000.pdf}
\includegraphics[width=0.11\textwidth]{figures/swing/man/00001.pdf}
\includegraphics[width=0.11\textwidth]{figures/swing/man/00002.pdf}
\includegraphics[width=0.11\textwidth]{figures/swing/man/00003.pdf}
\includegraphics[width=0.11\textwidth]{figures/swing/man/00004.pdf}
\includegraphics[width=0.11\textwidth]{figures/swing/man/00005.pdf}
\includegraphics[width=0.11\textwidth]{figures/swing/man/00006.pdf}
\includegraphics[width=0.11\textwidth]{figures/swing/man/00007.pdf}


\makebox[0.12\textwidth]{\colorbox{yellow}{\textbf{Edit-A-Video (Ours)}} A \textcolor{blue}{\textbf{Iron Man}} is on the swing}\\

\includegraphics[width=0.11\textwidth]{figures/swing/iron_man/00000.pdf}
\includegraphics[width=0.11\textwidth]{figures/swing/iron_man/00001.pdf}
\includegraphics[width=0.11\textwidth]{figures/swing/iron_man/00002.pdf}
\includegraphics[width=0.11\textwidth]{figures/swing/iron_man/00003.pdf}
\includegraphics[width=0.11\textwidth]{figures/swing/iron_man/00004.pdf}
\includegraphics[width=0.11\textwidth]{figures/swing/iron_man/00005.pdf}
\includegraphics[width=0.11\textwidth]{figures/swing/iron_man/00006.pdf}
\includegraphics[width=0.11\textwidth]{figures/swing/iron_man/00007.pdf}

\makebox[0.12\textwidth]{\colorbox{green}{\textbf{Tune-A-Video}} A \textcolor{blue}{\textbf{Iron Man}} is on the swing}\\

\includegraphics[width=0.11\textwidth]{figures/swing/tav/00000.pdf}
\includegraphics[width=0.11\textwidth]{figures/swing/tav/00001.pdf}
\includegraphics[width=0.11\textwidth]{figures/swing/tav/00002.pdf}
\includegraphics[width=0.11\textwidth]{figures/swing/tav/00003.pdf}
\includegraphics[width=0.11\textwidth]{figures/swing/tav/00004.pdf}
\includegraphics[width=0.11\textwidth]{figures/swing/tav/00005.pdf}
\includegraphics[width=0.11\textwidth]{figures/swing/tav/00006.pdf}
\includegraphics[width=0.11\textwidth]{figures/swing/tav/00007.pdf}

\makebox[0.12\textwidth]{\colorbox{green}{\textbf{SDEdit}} A \textcolor{blue}{\textbf{Iron Man}} is on the swing}\\

\includegraphics[width=0.11\textwidth]{figures/swing/sdedit/00000.pdf}
\includegraphics[width=0.11\textwidth]{figures/swing/sdedit/00001.pdf}
\includegraphics[width=0.11\textwidth]{figures/swing/sdedit/00002.pdf}
\includegraphics[width=0.11\textwidth]{figures/swing/sdedit/00003.pdf}
\includegraphics[width=0.11\textwidth]{figures/swing/sdedit/00004.pdf}
\includegraphics[width=0.11\textwidth]{figures/swing/sdedit/00005.pdf}
\includegraphics[width=0.11\textwidth]{figures/swing/sdedit/00006.pdf}
\includegraphics[width=0.11\textwidth]{figures/swing/sdedit/00007.pdf}

\makebox[0.12\textwidth]{\colorbox{green}{\textbf{Video-P2P}} A \textcolor{blue}{\textbf{Iron Man}} is on the swing}\\

\includegraphics[width=0.11\textwidth]{figures/swing/videoptp/00000.pdf}
\includegraphics[width=0.11\textwidth]{figures/swing/videoptp/00001.pdf}
\includegraphics[width=0.11\textwidth]{figures/swing/videoptp/00002.pdf}
\includegraphics[width=0.11\textwidth]{figures/swing/videoptp/00003.pdf}
\includegraphics[width=0.11\textwidth]{figures/swing/videoptp/00004.pdf}
\includegraphics[width=0.11\textwidth]{figures/swing/videoptp/00005.pdf}
\includegraphics[width=0.11\textwidth]{figures/swing/videoptp/00006.pdf}
\includegraphics[width=0.11\textwidth]{figures/swing/videoptp/00007.pdf}

\caption{\textbf{Qualitative Results} Edit-A-Video outperforms in editing compared to other baselines.}
\label{fig:main_result}
\end{center}
\end{figure*}

\begin{table}[t]
\small{
\caption{\textbf{Qualitative Comparisons to Baselines} We measure the overall human preference score (User Score (O)) and automatic metric scores for the comparisons to baselines.}
\begin{center}
% \vspace{0.5em}
\begin{tabular}{|c|c|c|c|c|}
\hline
Method & User Score (O) ($\uparrow$) & Text Alignment ($\uparrow$) & LPIPS ($\downarrow$) & PSNR ($\uparrow$) \\
\hline\hline
Edit-A-Video (Ours) & $3.80\pm0.10$ & $30.2688$ & $0.2625$ & $20.0992$ \\
Tune-A-Video & $3.46\pm0.10$ & $30.0514$ & $0.4482$ & $14.5753$ \\
SDEdit & $3.40\pm0.10$ & $28.4203$ & $0.2711$ & $20.4767$ \\
Video-P2P & $3.66\pm0.10$ & $30.0842$ & $0.3047$ &$17.5760$ \\
\hline
\end{tabular}
\end{center}
\vspace{-1.0em}
\label{tab:comparisons}
}
\end{table}

\subsection{Implementation Details}
We implement our method based on the stable-diffusion-v1-4\footnote{Stable Diffusion: \href{https://github.com/CompVis/stable-diffusion}{https://github.com/CompVis/stable-diffusion}}, publicly available TTI model~\citep{rombach2022high}.
We finetune only the latent diffusion model in TTI model on $8$ frame $512\times512$ video for $300$ steps in temporal modeling and $500$ steps in inversion, while fixing the autoencoder to encode each frame independently.
At inversion and sampling, we use $50$ step DDIM sampler and set the classifier-free guidance scale to $7.5$.
From the analysis introduced in Sec.~\ref{hparams}, we use cross-attention injection duration as $0.2$, spatio-temporal attention injection duration as $0.5$, and temporal attention injection duration as $0.8$. We set TC blending threshold ($\tau$) as $0.25$ and show the effects of the value in Sec.~\ref{ablation}.

For the quantitative evaluation, we edit a total $100$ $<$text, video$>$ pairs (four captions for each of $25$ videos), and the videos are collected from the web and DAVIS dataset~\citep{davis} as other works~\citep{esser2023structure, bar2022text2live}.
We set two of the four sentences for each video to change the style including the background, and the other two sentences to change the object.
We compare our model to baselines by human preference study, which we refer to as User Score (O), and automatic evaluation metrics.
In the human preference study, we ask $62$ users to grade the overall quality score of the edited video on a scale of $1-5$ considering three aspects: background preservation, text alignment, and video realism.
We include detailed explanations in the Supplementary Materials.

For the detailed analysis, we evaluate all models with three automatic metrics, one for editing performance and two for background preservation.
We measure the text alignment for the editing performance, which estimates how much the editing reflects the target text by averaging the cosine similarity between CLIP embedding of target text and CLIP image embeddings of all frames.
We further measure the distance between the source video and the target video for the background preservation by LPIPS~\citep{zhang2018unreasonable} and PSNR following the previous works~\citep{esser2023structure,mokady2022null,hertz2022prompt}.


\subsection{Baseline Comparisons}
\label{baseline_comparison}
We compare our method with three baselines quantitatively and qualitatively:
(1) \textit{Tune-A-Video}: generating the video from target prompt after tuning the inflated 3D model with source text-video pair.
(2) \textit{SDEdit}: Based on Tune-A-Video, editing the video with another method, SDEdit~\citep{meng2021sdedit} which injects the noise to video until intermediate timestep $t_{0}=25$ among $50$ steps following \cite{meng2021sdedit} and denoises from it conditioned on target prompt.
(3) \textit{Video-P2P}: concurrent single video editing method, which inflates the 2D model by replacing self-attention with first-frame attention, where the attention matrix of the current frame is calculated only based on the first frame.


\textbf{Quantitative Results}
We perform the user evaluation to let the participants grade the scores on the edited videos. 
Owing to the proposed techniques, Edit-A-Video achieves superior performance compared to baselines with statistical significance (p-value $<$ $0.05$ from the Wilcoxon signed-rank test), as shown in table~\ref{tab:comparisons}.
We further measure automatic evaluation metrics to support the user score, which are also included in table~\ref{tab:comparisons}.
Since Tune-A-Video generates entire frames of video corresponding to the target text, the synthesized sample accurately reflects the target text. However, we observe that Tune-A-Video modifies even the background to be preserved, which is in line with the results obtained from LPIPS and PSNR measurements.
SDEdit, another baseline, preserves the contents of the source video, yet shows the lowest text alignment score, which indicates that it does not reflect the target text faithfully.
Unlike the aforementioned two baselines, Video-P2P preserves the property that is independent of the target editing to some extent while reflecting the target text. 
Edit-A-Video exhibited superior performance compared to Video-P2P in all metrics. 
Through these results, we confirm that Edit-A-Video is capable of editing video correspond to target text while preserving details that should be remained.

\textbf{Qualitative Results}
Fig.~\ref{fig:main_result} presents qualitative results of ours and baselines.
Tune-A-Video fails to maintain the background content, such as the color of the brick on the floor, while SDEdit fails to generate samples that reflect the target text. 
Although Video-P2P generates higher-quality samples than previous baselines, its imprecise blending mask leads to unintended changes in regions that should not be edited, such as the positioning of the legs.
Compared to these baselines, Edit-A-Video generates samples that preserve the property that should remain unedited and align accurately with the target prompt.
More examples are included in the Supplementary Materials.

\begin{table}[t]
\small{
\caption{\textbf{TC Blending Ablation Study} We demonstrate TC Blending's impact through subjective scores (User Scores) and automatic metrics. User Score (O) represents overall editing quality, while User Score (P) assesses non-target content preservation, including the background. Mask IoU measures IoU between the foreground mask from the saliency detector and the blending mask.
}
\begin{center}
\begin{tabular}{|c|c|c|c|c|c|}
\hline
Method & User Score (O) ($\uparrow$) & User Score (P) ($\uparrow$) & LPIPS 
($\downarrow$) & PSNR ($\uparrow$) & Mask IoU ($\uparrow$) \\
\hline\hline
Edit-A-Video & $3.80\pm0.10$ & $4.13\pm0.13$ & $0.2625$ & $20.0992$ & $0.3805$ \\
w/o TC-Bld & $3.69\pm0.10$ & $3.96\pm0.13$ & $0.2723$ & $19.8628$ & $0.2371$ \\

\hline
\end{tabular}
\end{center}
\vspace{-0.5em}
\label{tab:ablation}
}
\vspace{-0.5em}
\end{table}

\subsection{Ablation}
\label{ablation}


\begin{figure}[t]
\begin{center}
\begin{minipage}{0.45\textwidth}
\centering
\makebox[0.12\textwidth]{\colorbox{yellow}{\textbf{Baseline w/o TC Blending}}}\\
\rotatebox{90}{\parbox{0.20\textwidth}{\centering threshold \\ 0.10}}
\includegraphics[width=0.20\textwidth]{figures/ablation/base_mask_ablation_0.1/blending_1.pdf}
\includegraphics[width=0.20\textwidth]{figures/ablation/base_mask_ablation_0.1/blending_3.pdf}
\includegraphics[width=0.20\textwidth]{figures/ablation/base_mask_ablation_0.1/blending_5.pdf}
\includegraphics[width=0.20\textwidth]{figures/ablation/base_mask_ablation_0.1/blending_7.pdf}

\rotatebox{90}{\parbox{0.20\textwidth}{\centering ~ \\ ~}}
\includegraphics[width=0.20\textwidth]{figures/ablation/base_mask_ablation_0.1/result_1.pdf}
\includegraphics[width=0.20\textwidth]{figures/ablation/base_mask_ablation_0.1/result_3.pdf}
\includegraphics[width=0.20\textwidth]{figures/ablation/base_mask_ablation_0.1/result_5.pdf}
\includegraphics[width=0.20\textwidth]{figures/ablation/base_mask_ablation_0.1/result_7.pdf}

\rotatebox{90}{\parbox{0.20\textwidth}{\centering \textbf{threshold \\ 0.25}}}
\includegraphics[width=0.20\textwidth]{figures/ablation/base_mask_ablation_0.25/blending_1.pdf}
\includegraphics[width=0.20\textwidth]{figures/ablation/base_mask_ablation_0.25/blending_3.pdf}
\includegraphics[width=0.20\textwidth]{figures/ablation/base_mask_ablation_0.25/blending_5.pdf}
\includegraphics[width=0.20\textwidth]{figures/ablation/base_mask_ablation_0.25/blending_7.pdf}

\rotatebox{90}{\parbox{0.20\textwidth}{\centering ~ \\ ~}}
\includegraphics[width=0.20\textwidth]{figures/ablation/base_mask_ablation_0.25/result_1.pdf}
\includegraphics[width=0.20\textwidth]{figures/ablation/base_mask_ablation_0.25/result_3.pdf}
\includegraphics[width=0.20\textwidth]{figures/ablation/base_mask_ablation_0.25/result_5.pdf}
\includegraphics[width=0.20\textwidth]{figures/ablation/base_mask_ablation_0.25/result_7.pdf}

\rotatebox{90}{\parbox{0.20\textwidth}{\centering threshold \\ 0.40}}
{
\includegraphics[width=0.20\textwidth]{figures/ablation/base_mask_ablation_0.4/blending_1.pdf}
\includegraphics[width=0.20\textwidth]{figures/ablation/base_mask_ablation_0.4/blending_3.pdf}
\includegraphics[width=0.20\textwidth]{figures/ablation/base_mask_ablation_0.4/blending_5.pdf}
\includegraphics[width=0.20\textwidth]{figures/ablation/base_mask_ablation_0.4/blending_7.pdf}

\rotatebox{90}{\parbox{0.20\textwidth}{\centering ~ \\ ~}}
\includegraphics[width=0.20\textwidth]{figures/ablation/base_mask_ablation_0.4/result_1.pdf}
\includegraphics[width=0.20\textwidth]{figures/ablation/base_mask_ablation_0.4/result_3.pdf}
\includegraphics[width=0.20\textwidth]{figures/ablation/base_mask_ablation_0.4/result_5.pdf}
\includegraphics[width=0.20\textwidth]{figures/ablation/base_mask_ablation_0.4/result_7.pdf}
}
\end{minipage}\hfill
\begin{minipage}{0.45\textwidth}
\centering
\makebox[0.12\textwidth]{\colorbox{green}{\textbf{Edit-A-Video}}}\\
\rotatebox{90}{\parbox{0.20\textwidth}{\centering threshold \\ 0.10}}
\includegraphics[width=0.20\textwidth]{figures/ablation/mask_ablation_0.1/blending_1.pdf}
\includegraphics[width=0.20\textwidth]{figures/ablation/mask_ablation_0.1/blending_3.pdf}
\includegraphics[width=0.20\textwidth]{figures/ablation/mask_ablation_0.1/blending_5.pdf}
\includegraphics[width=0.20\textwidth]{figures/ablation/mask_ablation_0.1/blending_7.pdf}

\rotatebox{90}{\parbox{0.20\textwidth}{\centering ~ \\ ~}}
\includegraphics[width=0.20\textwidth]{figures/ablation/mask_ablation_0.1/result_1.pdf}
\includegraphics[width=0.20\textwidth]{figures/ablation/mask_ablation_0.1/result_3.pdf}
\includegraphics[width=0.20\textwidth]{figures/ablation/mask_ablation_0.1/result_5.pdf}
\includegraphics[width=0.20\textwidth]{figures/ablation/mask_ablation_0.1/result_7.pdf}

\rotatebox{90}{\parbox{0.20\textwidth}{\centering \textbf{threshold \\ 0.25}}}
\includegraphics[width=0.20\textwidth]{figures/ablation/mask_ablation_0.25/blending_1.pdf}
\includegraphics[width=0.20\textwidth]{figures/ablation/mask_ablation_0.25/blending_3.pdf}
\includegraphics[width=0.20\textwidth]{figures/ablation/mask_ablation_0.25/blending_5.pdf}
\includegraphics[width=0.20\textwidth]{figures/ablation/mask_ablation_0.25/blending_7.pdf}

\rotatebox{90}{\parbox{0.20\textwidth}{\centering ~ \\ ~}}
\includegraphics[width=0.20\textwidth]{figures/ablation/mask_ablation_0.25/result_1.pdf}
\includegraphics[width=0.20\textwidth]{figures/ablation/mask_ablation_0.25/result_3.pdf}
\includegraphics[width=0.20\textwidth]{figures/ablation/mask_ablation_0.25/result_5.pdf}
\includegraphics[width=0.20\textwidth]{figures/ablation/mask_ablation_0.25/result_7.pdf}

\rotatebox{90}{\parbox{0.20\textwidth}{\centering threshold \\ 0.40}}
{
\includegraphics[width=0.20\textwidth]{figures/ablation/mask_ablation_0.4/blending_1.pdf}
\includegraphics[width=0.20\textwidth]{figures/ablation/mask_ablation_0.4/blending_3.pdf}
\includegraphics[width=0.20\textwidth]{figures/ablation/mask_ablation_0.4/blending_5.pdf}
\includegraphics[width=0.20\textwidth]{figures/ablation/mask_ablation_0.4/blending_7.pdf}

\rotatebox{90}{\parbox{0.20\textwidth}{\centering ~ \\ ~}}
\includegraphics[width=0.20\textwidth]{figures/ablation/mask_ablation_0.4/result_1.pdf}
\includegraphics[width=0.20\textwidth]{figures/ablation/mask_ablation_0.4/result_3.pdf}
\includegraphics[width=0.20\textwidth]{figures/ablation/mask_ablation_0.4/result_5.pdf}
\includegraphics[width=0.20\textwidth]{figures/ablation/mask_ablation_0.4/result_7.pdf}
}
\end{minipage}
\caption{\textbf{Qualitative Ablation Studies} We visualize the editing results for the target text ``A Spider Man is skiing". The samples on the left are based on the masking threshold without the proposed blending method applied, and the samples on the right are the results with the proposed blending method applied.}
\label{fig:mask_thres_ablation}
\end{center}
\vspace{-2.0em}
\end{figure}

In this section, we analyze the TC Blending mask's impact through an ablation study.
We evaluate the editing targets of 50 out of the 100 pairs of $<$text, video$>$ used in Sec~\ref{baseline_comparison}, focusing specifically on object editing.
We evaluate the proposed blending method's effectiveness using two human preference scores and three automatic metrics.
The human preference scores, User Score (O) and User Score (P), are rated on a 5-point scale, measuring overall editing quality and non-target region preservation.
For automatic metrics, we use LPIPS, PSNR, and a newly proposed metric, Mask Intersection over Union (Mask IoU).
Mask IoU quantifies the overlap between the blending mask and the target object region, calculated as the Intersection over Union (IoU) between these two regions.
To capture the target object region, we use a publicly available saliency detector~\citep{zheng2022mccl}.

As shown in table~\ref{tab:ablation}, Edit-A-Video achieves higher user scores than the model without TC Blending (p-value $<$ $0.01$ from the Wilcoxon signed-rank test).
Furthermore, through various automatic metrics and qualitative results in Fig~\ref{fig:mask_thres_ablation}, we confirm that TC Blending is effective in accurate target object masking and background preservation.

We also demonstrate the effect of mask threshold value $\tau$. 
In Fig~\ref{fig:mask_thres_ablation}, $\tau$ controls the editing region size. 
Without TC Blending, adjusting the threshold fails to capture sharp object regions effectively, causing abrupt frame-wise mask changes and undesirable artifacts. 
In contrast, Edit-A-Video achieves sharp and smooth masks by properly setting the threshold. These results confirm that our proposed TC Blending mitigates background inconsistency.
