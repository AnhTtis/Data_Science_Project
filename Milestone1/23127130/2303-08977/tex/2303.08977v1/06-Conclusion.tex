\section{Conclusion}
\label{Sec:Conclusion}
In this paper, we introduce DeblurSR, a novel event-based motion deblurring approach based on the spiking representation. DeblurSR builds upon the same biological principles followed by the event camera design. Experiments show that DeblurSR outperforms state-of-the-art approaches in deblurring quality and can be easily extended to video super resolution. In the future, we plan to modify DeblurSR and allow different pixels to have a different number of parametric segments. This requires a non-trivial redesign of the prediction network to handle the heterogeneity. Another possible direction is to design a motion deblurring algorithm without using neuromorphic events, which likely involves theoretical modeling of motion ambiguities. 

\begin{figure}[t]
    \centering
    \includegraphics[width=\columnwidth]{Figures/sr.png}
    \begin{tabularx}{\columnwidth}{Y Y Y Y Y}
        (a) & (b) & (c) & (d) & (e)
    \end{tabularx}
    \caption{Visual Comparison Between Different Super-Resolution Methods. (a) Ground Truth; (b) EDI~\cite{pan2019bringing}; (c) eSL-Net~\cite{wang2020event}; (d) E-CIR~\cite{song2022cir}; (e) Ours.} 
    \label{fig:sr_compare}
\end{figure}