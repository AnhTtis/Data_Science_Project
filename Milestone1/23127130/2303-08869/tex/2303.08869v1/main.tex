\documentclass{article}

% Language setting
% Replace `english' with e.g. `spanish' to change the document language
\usepackage[english]{babel}

% Set page size and margins
% Replace `letterpaper' with `a4paper' for UK/EU standard size
\usepackage[letterpaper,top=2cm,bottom=2cm,left=3cm,right=3cm,marginparwidth=1.75cm]{geometry}

% Useful packages
\usepackage{amsmath}
\usepackage{graphicx}
\usepackage[colorlinks=true, allcolors=blue]{hyperref}

\title{CMB Hotspots}
%\author{You}

\begin{document}
\maketitle

\begin{abstract}
Abstract.
\end{abstract}

\section{Profile}
We choose Newtonian gauge to describe metric perturbations,
\begin{align}
ds^2 = -(1+2\Psi)dt^2+a^2(t)(1+2\Phi)\delta_{ij}dx^{i}dx^{j}.    
\end{align}
To describe temperature fluctuations of the CMB corresponding to Fourier mode $\vec{k}$, pointing to direction $\hat{n}$ in the sky,
\begin{align}
\theta(\vec{k},\hat{n},\eta_0) = \sum_{l}i^l (2l+1)\mathcal{P}_l(\hat{k}\cdot\hat{n})\theta_l(k,\eta_0). 
\end{align}
Here the multipole $\theta_l(k,\eta_0)$ depends on the primordial perturbation $\zeta(\vec{k})$ and a transfer function $T_l(k)$ as,
\begin{align}
\theta_l(k,\eta_0) = T_l(k)\zeta(\vec{k}).   
\end{align}
Importantly, for our scenario $T_l(k)$ itself can be computed exactly as in $\Lambda$CDM. It is given via the expression~\cite{??},
\begin{align}
\theta_l(k,\eta_0)& \simeq \left(\theta_0(k,\eta_{\rm rec})+\Psi(k,\eta_{\rm rec})\right)j_l(k(\eta_0-\eta_{\rm rec}))+ \int_{0}^{\eta_0}d\eta e^{-\tau} \left(\Psi'(k,\eta)-\Phi'(k,\eta)\right)j_l(k(\eta_0-\eta))  \\
& + 3\theta_1(k,\eta_{\rm rec})\left(j_{l-1}(k(\eta_0-\eta_{\rm rec}))-(l+1)\dfrac{j_{l}(k(\eta_0-\eta_{\rm rec}))}{k(\eta_0-\eta_{\rm rec})}\right)\\
& \equiv \left(f_{\rm SW}(k,l,\eta_0) + f_{\rm ISW}(k,l,\eta_0) + f_{\rm Dopp}(k,l,\eta_0)\right)\zeta(\vec{k}).
\end{align}
Regardless of $\zeta(\vec{k})$ is, we can compute $f_{\rm SW}(k,l,\eta_0)$, $f_{\rm ISW}(k,l,\eta_0)$ and $f_{\rm Dopp}(k,l,\eta_0)$ as in $\Lambda$CDM. While the above is valid for any $\zeta(\vec{k})$, we now focus on the primordial perturbation due to massive particles which is given by,
\begin{align}
\langle\zeta_{\rm HS} (\vec{k}) \rangle = e^{-i\vec{k}\cdot\vec{x}_{\rm HS}}\frac{f(k\eta_*)}{k^3}.
\end{align}
Here the profile function is given by,
\begin{align}
f(x) = \frac{gH^2}{\dot{\phi}_0}(\text{Si}(x)-\sin(x)).  
\end{align}
We parametrize the distance to the hotspot as,
\begin{align}
\vec{x}_0-\vec{x}_{\rm HS} = -(\eta_0 - \eta_{\rm HS})\hat{n}_{\rm HS}.   
\end{align}
Here $\vec{x}_0$ and $\vec{x}_{\rm HS}$ parametrize our and the hotspot locations respectively. $\eta$ is the conformal time and $\hat{n}_{\rm HS}$ points to the direction of the hotspot. $\eta_{\rm HS}$ denotes the location of the hotspot in conformal time. In the earlier paper, we took the hotspot to be on the CMB surface and hence set $\eta_{\rm HS} = \eta_{\rm rec} \approx 280~$Mpc.

As derived earlier, the temperature due to the hotspot is given by (dropping $\eta_0$),
\begin{align}
\theta(\vec{x}_0,\hat{n})=\int \frac{d^3\vec{k}}{(2\pi)^3}e^{i\vec{k}\cdot(\vec{x}_0-\vec{x}_{\rm HS})}\sum_{l}i^l (2l+1) \mathcal{P}_l(\hat{k}\cdot\hat{n})\left(f_{\rm SW}(k,l)+f_{\rm ISW}(k,l)+f_{\rm Dopp}(k,l)\right)\frac{f(k\eta_*)}{k^3}. 
\end{align}
Here $\hat{n}$ denotes the direction of observation. $f_{\rm SW}(k,l)$ and $f_{\rm ISW}(k,l)$ are extracted from the transfer function. Using the plane wave expansion,
\begin{align}
e^{-i\vec{k}\cdot\vec{r}}=\sum_{\ell=0}^\infty (-i)^l (2l+1) j_l(kr) \mathcal{P}_l(\hat{k}\cdot\hat{r}), \end{align}
the relation,
\begin{align}
\mathcal{P}_l(\hat{k}\cdot\hat{n})=\frac{4\pi}{(2l+1)}\sum_{m=-l}^{l}Y_{lm}(\hat{n})Y_{lm}^*(\hat{k}),    
\end{align}
we get,
\begin{align}\label{eq.thprofile}
\theta(\vec{x}_0,\hat{n},\eta_{\rm HS}) = \frac{1}{2\pi^2} \int_0^\infty \frac{dk}{k}\sum_{l}j_{l}(k(\eta_0-\eta_{\rm HS}))(2l+1)\mathcal{P}_l(\hat{n}\cdot\hat{n}_{\rm HS})\left(f_{\rm SW}(k,l)+f_{\rm ISW}(k,l)+f_{\rm Dopp}(k,l)\right) f(k\eta_*).   
\end{align}
Note $\theta(\vec{x}_0,\hat{n},\eta_{\rm HS})$ depends on $\eta$, the distance to the hotspot, whereas in the earlier paper we assumed $\eta_{\rm HS}=\eta_{\rm rec}$, i.e., hotspots to be on the surface of recombination.

\subsection{Central Temperature}
To get the temperature of the central pixel, we set $\hat{n}=\hat{n}_{\rm HS}$, implying $\mathcal{P}_l(\hat{n}\cdot\hat{n}_{\rm HS})\approx 1$,
\begin{align}\label{eq.thcentral}
\theta_{\rm center}(\vec{x}_0,\eta_{\rm HS}) = \frac{1}{2\pi^2} \int_0^\infty \frac{dk}{k}\sum_{l}j_{l}(k(\eta_0-\eta_{\rm HS}))(2l+1)\left(f_{\rm SW}(k,l)+f_{\rm ISW}(k,l)+f_{\rm Dopp}(k,l)\right) f(k\eta_*).   
\end{align}
We show the central temperature in Fig.~\ref{fig:T_central}.
\begin{figure}[h]
    \centering
    \includegraphics[width=\textwidth]{central_temp.pdf}
    \caption{Central Temperature as a function of the location of the hotspot. Blue: Sachs-Wolfe contribution, Orange: Integrated Sachs-Wolfe contribution. $\eta_{\rm HS}$ and $T_{\rm central}$ are in Mpc and $\mu K$ respectively.}
    \label{fig:T_central}
\end{figure}
\section{Angular Size of Hotspots}
First we relate $N_{\rm side}$ to angular size of a pixel $\theta_{\rm pix}$. This can be done by equating the total number of pixels in the sky,
\begin{align}
12 N_{\rm side}^2 = 4\pi/\theta_{\rm pix}^2.
\end{align}
As an example, for $N_{\rm side}=256$, this gives $\theta_{\rm pix}=0.004$. This computation makes no reference to $\ell_{\rm max}$ or $\eta_*$ etc.

Now to determine what angular size the separation has, first focus on the case of the previous paper. There, we had a distance of $\eta_*$ on the CMB surface. Denoting the distance to the CMB by $\eta_0$, we see the angular size of the $\eta_*$ distance is,
\begin{align}
\theta_{\rm separation} = \sqrt{4\pi}\eta_*/\eta_0.    
\end{align}
This is similar to applying the $s=r\theta$ relation for arcs. The factor $\sqrt{4\pi}$ can be justified by saying when $\eta_*=\eta_0$, i.e., a horizon-sized spot, then the squared-size should indeed be $4\pi$ so that the spot spans the entire sky. Then to get the pixel size of the spot, we use 
\begin{align}
N_{\rm spot size in pixel} =  \theta_{\rm separation}/\theta_{\rm pix}.
\end{align}
For $\eta_*=160$~Mpc, we get $\theta_{\rm separation}=0.04$, so roughly 10 pixel-size spot.

Now let us come to the present paper where the spots can be anywhere. The distance between two spots are given by,
\begin{align}
d^2 = r_1^2 + r_2^2 - 2 r_1 r_2 \cos\psi \approx (r_1-r_2)^2 + r_1r_2\psi^2.    
\end{align}
Here $\psi$ is the angle between the two spots and is assumed to be $\ll 1$ for the last relation. In this small angle approximation we see that it is only $\psi$ that determines the angular size of corresponding to the `3D' distance $d$. In particular,
\begin{align}
\theta_{\rm separation} = \theta_{\rm pix}/\psi.    
\end{align}
For this expression to be valid, we need to have $\psi > \theta_{\rm pix}$. As a special case, let us set $r_1=r_2=\eta_0$. Then $d=\eta_0\psi$ and $\theta_{\rm separation} = \theta_{\rm pix}\eta_0/d$. This is exactly the previous relation when we use $d=\eta_*$. 

More generally, if we fix $r_1, r_2, d$ then we can solve for $\psi$ and use that to determine $\theta_{\rm separation}$.
\end{document}