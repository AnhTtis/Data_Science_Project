\section{Methodology}
\label{sec:methodology}

%details of your experimental setup, hardware, inputs, benchmarks, accuracy metrics, etc.

%To evaluate our framework we first, through a series of microbenchmarks, assess the energy savings and speedup achieved through different approximate neural networking operations. We also examine whether mutual order (in terms of accuracy and speedup) of the approximate NN configurations calculated by ApproxTuner remains valid when the configurations get executed on a mobile device. Second, we confirm that the softmax layer confidence correlates with the inference accuracy in case of a MobileNet deep learning network trained to recognize human activity. Finally, in a 21-user study we test how Mobiprox performs in a real world HAR application and comparatively evaluate how the 3 adaptation engines perform in terms of energy saved vs. inference accuracy.

We evaluate Mobiprox first through a series of experiments aiming to assess the energy savings and speedup achieved through different approximate NN operations, over a selection of networks. Mobiprox is primarily targeting dynamic mobile environments and inference from time-series sensor data. Thus, we include two NN architectures (MobileNet and ResNet50) trained on UCI HAR dataset~\cite{anguita2013public}. We also assess the energy savings brought by Mobiprox for the standard CIFAR-10 image recognition dataset. We implement all networks in PyTorch.

MobileNet~\cite{howard2017mobilenets} is traditionally used in computer vision tasks and as input takes $32\times32$ 3-channel images. To used it for human activity recognition, we transform the 3-axial body acceleration, 3-axial total acceleration, 3-axial body rotation time-series data contained in the UCI-HAR dataset to fit the required input tensor size. For each feature vector set (body acceleration, total acceleration, body rotation), we first vertically stack the x, y, and z axis data into an a $8\times 128$ matrix. Vectors for each axis are repeated multiple times, as shown in Figure~\ref{fig:signal-image-in}. Following this transformation, the final $32~\times~32~\times~3$ signal image is constructed by vertically re-stacking consecutive 32 elements wide blocks (Figure~\ref{fig:signal-image-out}) and has the following property: neighboring features in the signal image carry information that is neighboring in the time domain\footnote{This does not hold at the borders of each 32-row block, as visible from color changes shown in figure~\ref{fig:signal-image-out}.}.

%To use the UCI-HAR data set with MobileNet, a transformation of sensor data is required. In our signal images, image channels represent different sensors (body acceleration, total body acceleration, body rotation). Figure~\ref{fig:signal-image} shows how we arranged the feature vectors within each image channel. Feature vectors for axes $x$, $y$ and $z$ were vertically stacked into a $8\times 128$ matrix. Feature vectors for each axis were repeated multiple times, as shown in Figure~\ref{fig:signal-image-in}. Following a matrix transformation, the final $32~\times~32~\times~3$ signal image (Figure~\ref{fig:signal-image-out}) has the following property: neighbouring features in the signal image carry information that is neighbouring in the time domain\footnote{Due to signal image construction technique, this does not hold at the borders of each 32-row block, as visible from colour changes shown in figure~\ref{fig:signal-image-out}.}. This property is highlighted in~\cite{jiang2015human}, which inspired our signal image composition technique. The importance of this aspect and caveats of such signal image composition are detailed in section~\ref{sec:mnUci-tuning}.

\begin{figure}
    \newcommand{\w}{0.9\linewidth}
    \newcommand{\sw}{0.4\linewidth}
    \centering
    
    \begin{subfigure}{\sw}
        \centering
        \includegraphics[width=\w]{figures/signal-image.pdf}
        \caption{Matrix of stacked feature vectors with size $128~\times~8~\times~3$.}
        \label{fig:signal-image-in}
    \end{subfigure}
    \hspace{1cm}
    \begin{subfigure}{\sw}
        \centering
        \includegraphics[width=\w]{figures/signal-image-final.pdf}
        \caption{Final signal image with size $32~\times~32~\times~3$.}
        \label{fig:signal-image-out}
    \end{subfigure}
    \caption{Signal image composition. Horizontal blocks of data in figure (a) are arranged vertically in figure (b).}
    \label{fig:signal-image}
\end{figure}

%We implemented the MobileNet-V2 network architecture (\mnUci{}) using PyTorch. The NN was trained on the UCI-HAR training data set composed of 7352 data points (signal images). We tested our model on the test data set containing 2947 data points. The model achieved a classification accuracy of 90\% on the test data set.
%We also investigated per-class accuracies with a confusion matrix shown in Figure~\ref{fig:mnuci-confusion}.

% \begin{figure}
%     \centering
%     \includegraphics[width=0.8\linewidth]{figures/mobilenet_uci-har_0.90.pth.pdf}

%     \caption{Confusion matrix for \mnUci{}.}
%     \label{fig:mnuci-confusion}
% \end{figure}


\newcommand{\MPM}{Monsoon Power Monitor}
\newcommand{\tboard}{ASUS TinkerBoard S}


\textbf{Measurement setup.} \blue{Mobiprox is fully compliant with consumer off-the-shelf Android devices. Yet, modern unibody smartphones do not allow for batteries to be easily removed, precluding the use of high-accuracy power metering, which is essential for evaluating the energy savings enabled by Mobiprox. Therefore, when energy consumption is examined,} we use 
{\tboard}\footnote{\tboard: \url{https://tinker-board.asus.com/product/tinker-board-s.html}} development board running Android 7 OS. \blue{We power} the board through the high-frequency \MPM{}. Our Python-based profiler using ADB runs compiled approximated NNs on the board. The approximation's main, yet not the only (as we will see in Section~\ref{sec:evaluation_microbenchmarks}) impact on the energy consumption stems from the decreased data processing time.  The profiling for each network is, thus, executed on a predefined fraction of the data in 10 batches for UCI-HAR networks and in 8 batches for CIFAR-10 (detailed in Table~\ref{tab:data-fracs}). This was done to 
    \textit{i)} reduce overall time requirement,
    and
    \textit{ii)} obtain more robust measurements by measuring each batch separately.
We measure the energy consumption of each batch of each approximation configuration using the PowerTool\footnote{PowerTool: \url{https://www.msoon.com/hvpm-software-download}} software on the energy-consumption traces, and then report the mean and standard deviation of each configuration's energy consumption. \blue{While each Android device has its own specific power profile, the experimental measurements we perform are still relevant to identify the ordering of the approximate configurations in the QoS loss -- Speedup space. This ordering directly impacts the adaptation strategies and, at least over the devices we had access to (Samsung Galaxy S21, Samsung Galaxy M21, Xiaomi Pocophone F1, and ASUS Tinkerboard S), the ordering remains the same regardless of the device specifics.}

%[r7] Indeed the power profile is different for each smartphone model, however, according to our testing on three widely different devices, the relative ordering of the configurations remains the same with respect to the QoS loss -- Speedup space. This means, that, for instance, \emph{“achieve the highest possible accuracy with the least amount of energy” }strategy, such as the one used in the evaluation section, works on all devices, despite the configurations being profiled on only one device. We note that strategies that rely on absolute performance numbers, such as \emph{“ensure that the execution runs 20\% faster when a user is in a vehicle”} require that the profiling device is of the same architecture (in a narrow sense) as the client device. To avoid this, in the future we plan to integrate a runtime performance measurement method based on heartbeats \emph{(H. Hoffmann, J. Eastep, M. D. Santambrogio, J. E. Miller, and A. Agarwal. Application Heartbeats: A Generic Interface for Specifying Program Performance and Goals in Autonomous Computing Environments. In 7th International Conference on Autonomic Computing, ICAC, 2010.)} that will ensure re-profiling of the configurations on a given execution platform.

%To assess the efficacy of our approximations, we have modified ApproxHPVM's profiler to use ADB\footnote{Android Debug Bridge: \url{https://developer.android.com/studio/command-line/adb}} to run compiled approximated neural networks on the {\tboard}\footnote{\tboard: \url{https://tinker-board.asus.com/product/tinker-board-s.html}} development board. We used the \MPM{} as the board's power supply, which allowed measuring energy consumption accurately. 

%According to the \tboard{}'s specification, the board requires a 5V power supply capable of providing current of up to 3A. However, our configuration of the \MPM{} at 5V was not sufficient for the board. We have detected issues that were not present when using a conventional power supply -- poor classification accuracy of various neural networks and an overall unpredictable behaviour of our software. These issues were resolved after increasing the voltage of the \MPM{} to 5.25V.

%We ran the HPVM profiler on our selection of neural networks.
%Profiling for each neural network was executed on a predefined fraction of the data in 8 batches for CIFAR-10 networks and in 10 batches for UCI-HAR.
%These fractions are shown in Table~\ref{tab:data-fracs}.


\begin{table}
    \centering
    \caption{Data-set sizes used for energy profiling.}
    \begin{tabular}{llcc}
        \toprule
        Neural Network & Data set & No. images & Batch size \\
        \midrule
        \mnUci{}    & UCI-HAR  & 1450 & 145 \\
        \resUci{}   & UCI-HAR  & 250  & 50  \\
        \vggCifar{} & CIFAR-10 & 200  & 25  \\
        \anCifar{}  & CIFAR-10 & 800  & 100 \\
        \mnCifar{}  & CIFAR-10 & 800  & 100 \\
        \bottomrule
    \end{tabular}
    \label{tab:data-fracs}
\end{table}

%When we obtained energy-consumption traces given by the \MPM{}, we used the PowerTool\footnote{PowerTool: \url{https://www.msoon.com/hvpm-software-download}} software to 

\textbf{Real-world human activity recognition traces.} We assess the expected energy savings Mobiprox brings in real-world environments by taking a recent trace of human activity obtained through a body-mounted mobile sensing platform~\cite{electronics10232958}. The dataset contains  traces of 21 participants (13m, 8f), with an average age of 29 (std. dev 12) years. The traces consist of the acceleration and angular velocity in all three axes sampled at 50 Hz from an UDOO Neo Full board\footnote{UDOO NEO FULL: \url{https://shop.udoo.org/en/udoo-neo-full.html}}, a compact IoT embedded computing device equipped with an accelerometer and a digital gyroscope, strapped to each participant's waist. 

In this study, which took place at a university campus, the participants performed the six activities in a row. First the static ones -- sitting, standing still, and lying -- for 2 minutes each. Then, the dynamic activities -- walking up and down a hallway (summing up to two to three minutes for each participant) and walking down and up the stairs (about 45 seconds in each direction, the duration being limited by the total number of stairs). These six activities are the same as the ones present in the original UCI-HAR dataset, yet, the devices used, the environment, the participants, and the experimenters differ. Thus, by using the network built on a completely different UCI-HAR dataset, without re-training on the new data, we hope to obtain a realistic picture of how Mobiprox would perform in the wild. 

\blue{\textbf{Real-world spoken keyword recognition traces.} We also examine the savings Mobiprox brings in real-world environments by considering the problem of a spoken keyword recognition from microphone recordings. For this we use the Google Speech Commands (GSC) v0.01~\cite{warden2018speech}, a dataset containing 65,000 one-second long utterances of 30 short words by thousands of different speakers. Interested in the recognition of keywords in realistic situations, where a word has to be spotted in sound segments that may also contain words we are not interested in as well as recordings of silence, we follow the approach presented in ~\cite{tang2017honk} and use twelve classes for ten selected keywords (yes, no, up, down, left, right, on, off, stop and go) and two extra classes: ``unknown'' (for the remaining 20 words in the dataset) and ``silence''. In Section~\ref{sec:evaluation} we evaluate Mobiprox's ability to bring energy savings when a compact NN is used for on-device spoken keyword recognition within the GSC-based trace.}

%The code accompanying~\cite{tang2017honk} already contains the DL model parameters obtained through training on the 80\% of the GSC dataset (validation on 10\%), and we reuse these parameters in our network instance. This model is then funneled to Mobiprox's on-server and later on-device tuning on the ASUS Tinkerboard S to obtain a 10-point Pareto front of approximate configurations of the network. For tuning we use a half of the 10\% of the GSC that was not used for the training/validation.


\textbf{\blue{Live smartphone-based human activity recognition.}} We recruit ten users (all students or staff at our institution, 7 female/3 male) and record a 10-minute trace of indoor human activity from each user. The users are given an option of conducting six activities \blue{-- sitting, standing still, lying, walking, going up the stairs and going down the stairs --}  at their leisure. We use a Samsung Galaxy M21 smartphone attached to the user’s waist in the portrait orientation with the screen facing forward. The trace contains timestamped accelerometer and gyroscope sample taken at 50Hz. In our evaluation we use these traces to examine how Mobiprox adapts mobile DL approximation in an unscripted situation. In addition, in parallel to trace collection the phone runs an actual adaptation algorithm and approximated network inference. This is done to assess whether real-time approximation is feasible on a consumer-grade mobile device. 

%We chose this platform as its processing hardware corresponds to that found in today's low-end smartphones, yet, the platform itself runs Linux enabling a quick prototyping of DL pipelines. We sample the acceleration and angular velocity in all three axes from the board's sensors with a frequency of 50 Hz using the Neo.GPIO library~\cite{neogpio}.

%The traces experiments took place in a university campus building at our institution's premises. All volunteers performed the six activities in a row, first the static ones (sitting in a chair, standing still and lying down) for 2 minutes each. Then, they performed the dynamic activities: walking up and down a hallway (summing up to a total time of two to three minutes for each participant) and walking down and up the stairs (about 45 seconds in each direction, the duration being limited by the total number of stairs).





%The next step was to conduct a study to evaluate the performance of Mobiprox in terms of inference accuracy vs. energy savings on a human activity detection task using real-world data. Our study involved 21 volunteers, 13 male and 8 female participants, with an average age of 29 and a standard deviation of 12 years.  Given that we initially validated the Mobiprox framework in-lab on CNNs trained on the UCI HAR dataset, we ensure that our study is conducted under similar conditions to those described in the original study. Consequently, the study participants were equipped with a UDOO Neo Full board battery-powered and waist-mounted to accurately replicate the positioning of the smartphone in the original UCI HAR experiment. In addition, we instructed the study participants to perform the same physical activities with the ones present in the original UCI HAR experiment. 

%The UDOO Neo Full board is a compact IoT embedded computing device equipped with a 3-axis accelerometer and a digital gyroscope. We chose this platform as its processing hardware corresponds to that found in today's low-end smartphones, yet, the platform itself runs Linux enabling a quick prototyping of DL pipelines. We sample the acceleration and angular velocity in all three axes from the board's sensors with a frequency of 50 Hz using the Neo.GPIO library~\cite{neogpio}.

%The study experiments took place in a university campus building at our institution's premises. All volunteers performed the six activities in a row, first the static ones (sitting in a chair, standing still and lying down) for 2 minutes each. Then, they performed the dynamic activities: walking up and down a hallway (summing up to a total time of two to three minutes for each participant) and walking down and up the stairs (about 45 seconds in each direction, the duration being limited by the total number of stairs).