%%
%% This is file `sample-acmlarge.tex',
%% generated with the docstrip utility.
%%
%% The original source files were:
%%
%% samples.dtx  (with options: `acmlarge')
%% 
%% IMPORTANT NOTICE:
%% 
%% For the copyright see the source file.
%% 
%% Any modified versions of this file must be renamed
%% with new filenames distinct from sample-acmlarge.tex.
%% 
%% For distribution of the original source see the terms
%% for copying and modification in the file samples.dtx.
%% 
%% This generated file may be distributed as long as the
%% original source files, as listed above, are part of the
%% same distribution. (The sources need not necessarily be
%% in the same archive or directory.)
%%
%% The first command in your LaTeX source must be the \documentclass command.
%\documentclass[acmlarge,anonymous]{acmart}
\documentclass[acmlarge]{acmart}
%\documentclass[manuscript,screen,review]{acmart}
%\documentclass[sigconf, 10pt]{acmart}
%% NOTE that a single column version is required for 
%% submission and peer review. This can be done by changing
%% the \doucmentclass[...]{acmart} in this template to 
%% \documentclass[manuscript,screen,review]{acmart}
%% 
%% To ensure 100% compatibility, please check the white list of
%% approved LaTeX packages to be used with the Master Article Template at
%% https://www.acm.org/publications/taps/whitelist-of-latex-packages 
%% before creating your document. The white list page provides 
%% information on how to submit additional LaTeX packages for 
%% review and adoption.
%% Fonts used in the template cannot be substituted; margin 
%% adjustments are not allowed.
%%
%\pagenumbering{arabic}
%\settopmatter{printfolios=true}
%\usepackage{enumitem}
\usepackage{subcaption}
\usepackage{algorithm}
\usepackage{algorithmic}
\usepackage[inline]{enumitem}
\floatname{algorithm}{\footnotesize Algorithm}
\usepackage[section]{placeins}

%% \BibTeX command to typeset BibTeX logo in the docs
\AtBeginDocument{%
  \providecommand\BibTeX{{%
    \normalfont B\kern-0.5em{\scshape i\kern-0.25em b}\kern-0.8em\TeX}}}

%% Rights management information.  This information is sent to you
%% when you complete the rights form.  These commands have SAMPLE
%% values in them; it is your responsibility as an author to replace
%% the commands and values with those provided to you when you
%% complete the rights form.
\setcopyright{acmcopyright}
\copyrightyear{2022}
\acmYear{2022}
\acmDOI{10.1145/1122445.1122456}


%%
%% These commands are for a JOURNAL article.
\acmConference[Conference acronym 'XX]{Make sure to enter the correct
  conference title from your rights confirmation emai}{June 03--05,
  2018}{Woodstock, NY}
\acmPrice{15.00}
\acmISBN{978-1-4503-XXXX-X/18/06}


%%
%% Submission ID.
%% Use this when submitting an article to a sponsored event. You'll
%% receive a unique submission ID from the organizers
%% of the event, and this ID should be used as the parameter to this command.
%%\acmSubmissionID{123-A56-BU3}

%%
%% The majority of ACM publications use numbered citations and
%% references.  The command \citestyle{authoryear} switches to the
%% "author year" style.
%%
%% If you are preparing content for an event
%% sponsored by ACM SIGGRAPH, you must use the "author year" style of
%% citations and references.
%% Uncommenting
%% the next command will enable that style.
%%\citestyle{acmauthoryear}


\newcommand{\hashim}[1]{\textcolor{cyan}{Hashim: #1}}

\newcommand{\sasa}[1]{{\color{blue} #1}}

\newcommand{\veljko}[1]{{\color{olive} #1}}

\newcommand{\blue}[1]{{\color{black} #1}}

%%
%% end of the preamble, start of the body of the document source.
\begin{document}

%%
%% The "title" command has an optional parameter,
%% allowing the author to define a "short title" to be used in page headers.
\title{Mobiprox: Supporting Dynamic Approximate Computing on Mobiles}

%%
%% The "author" command and its associated commands are used to define
%% the authors and their affiliations.
%% Of note is the shared affiliation of the first two authors, and the
%% "authornote" and "authornotemark" commands
%% used to denote shared contribution to the research.

\author{Matevž Fabjančič}
\email{webmaster@marysville-ohio.com}
\affiliation{%
  \institution{Faculty of Computer and Information Science, University of Ljubljana}
  \country{Slovenia}
}

\author{Octavian Machidon}
\email{webmaster@marysville-ohio.com}
\affiliation{%
  \institution{Faculty of Computer and Information Science, University of Ljubljana}
  \country{Slovenia}
}

\author{Hashim Sharif}
\affiliation{%
 \institution{University of Illinois at Urbana-Champaign}
 \city{Urbana-Champaign}
 \state{Illinois}
 \country{United States}}

 \author{Yifan Zhao}
\affiliation{%
 \institution{University of Illinois at Urbana-Champaign}
 \city{Urbana-Champaign}
 \state{Illinois}
 \country{United States}}

 \author{Saša Misailović}
\affiliation{%
 \institution{University of Illinois at Urbana-Champaign}
 \city{Urbana-Champaign}
 \state{Illinois}
 \country{United States}}
 
\author{Veljko Pejović}
\email{webmaster@marysville-ohio.com}
\affiliation{%
  \institution{Faculty of Computer and Information Science, University of Ljubljana}
  \country{Slovenia}
}
\affiliation{%
  \institution{Department of Computer Systems, Institute ``Jožef Stefan''}
  \country{Slovenia}
}




%%
%% By default, the full list of authors will be used in the page
%% headers. Often, this list is too long, and will overlap
%% other information printed in the page headers. This command allows
%% the author to define a more concise list
%% of authors' names for this purpose.
\renewcommand{\shortauthors}{Fabjančič et al.}

%%
%% The abstract is a short summary of the work to be presented in the
%% article.
\begin{abstract}

Runtime-tunable context-dependent network compression would make mobile deep learning adaptable to often varying resource availability, input ``difficulty'', or user needs. The existing compression techniques significantly reduce the memory, processing, and energy tax of deep learning, yet, the resulting models tend to be permanently impaired, sacrificing the inference power for reduced resource usage. The existing tunable compression approaches, on the other hand, require expensive re-training, seldom provide mobile-ready implementations, and do not support arbitrary strategies for adapting the compression. 

In this paper we present Mobiprox, a framework enabling flexible-accuracy on-device deep learning. Mobiprox implements tunable approximations of tensor operations and enables runtime adaptation of individual network layers. A profiler and a tuner included with Mobiprox identify the most promising neural network approximation configurations leading to the desired inference quality with the minimal use of resources. Furthermore, we develop control strategies that depending on contextual factors, such as the input data difficulty, dynamically adjust the approximation level of a model. \blue{We implement Mobiprox in Android OS and through experiments in diverse mobile domains, including human activity recognition and spoken keyword detection, demonstrate that it can save up to 15\% system-wide energy with a minimal impact on the inference accuracy.}
%run it on a recent trace of smartphone sensors' data, and show that it can save up to 15\% of the system-level energy with only 2\% of the inference accuracy loss, when used for continuous human activity recognition.

%The affordances of deep learning are often counterbalanced by the burden that deep learning imposes on resource-constrained edge devices. While a range of compression techniques significantly reduce the memory, processing, and energy tax of deep learning, the resulting models tend to be permanently impaired, sacrificing the inference power for reduced resource usage. Mobile computing, on the other hand, is highly dynamic and while compressed models perform sufficiently well most of the time, occasions when high-accuracy models are necessary may arise. In this paper we present Mobiprox, a framework enabling flexible-accuracy on-device deep learning. Mobiprox implements tunable approximations of basic deep learning operations and enables runtime-adaptable approximation of individual neural network layers. Mobiprox includes a profiler that, in concert with an external tuner, identifies the most promising neural network approximation configurations leading to the desired inference quality with the minimal use of a mobile’s resources. Finally, with Mobiprox we develop a suite of control algorithms that depending on the context-dependent factors, such as the input data’s difficulty, dynamically adapt the approximation level of a deep learning model. Through a 21-person user study we show Mobiprox can save up to 15\% of the system-level energy with only 2\% of the inference accuracy loss, when used for continuous human activity recognition from on-body sensor data.  
 
% \hashim{The last part of the abstract talks about the novel component. The "context-dependent" runtime tuning to me is the new exciting idea here. The start of the abstract focuses on dynamically tunable approximations - this is less novel given other systems have done it (ApproxTuner, Green, MCDNN, Dynamic pruning techniques) }
 
\end{abstract}


%%
%% The code below is generated by the tool at http://dl.acm.org/ccs.cfm.
%% Please copy and paste the code instead of the example below.
%%
\begin{CCSXML}
<ccs2012>
   <concept>
       <concept_id>10003120.10003138</concept_id>
       <concept_desc>Human-centered computing~Ubiquitous and mobile computing</concept_desc>
       <concept_significance>500</concept_significance>
       </concept>
   <concept>
       <concept_id>10010147.10010257.10010293.10010294</concept_id>
       <concept_desc>Computing methodologies~Neural networks</concept_desc>
       <concept_significance>500</concept_significance>
       </concept>
 </ccs2012>
\end{CCSXML}

\ccsdesc[500]{Human-centered computing~Ubiquitous and mobile computing}
\ccsdesc[500]{Computing methodologies~Neural networks}

%%
%% Keywords. The author(s) should pick words that accurately describe
%% the work being presented. Separate the keywords with commas.
\keywords{approximate computing, context-awareness, mobile deep learning, ubiquitous computing}


%%
%% This command processes the author and affiliation and title
%% information and builds the first part of the formatted document.
\maketitle

\newcommand{\mnCifar}{\texttt{mobilenet\_cifar10}}
\newcommand{\mnUci}{\texttt{mobilenet\_uci-har}}
\newcommand{\anCifar}{\texttt{alexnet2\_cifar10}}
\newcommand{\vggCifar}{\texttt{vgg16\_cifar10}}
\newcommand{\resUci}{\texttt{resnet50\_uci-har}}


\section{Introduction}
\label{sec:introduction}
% \begin{itemize}
%     % Diffusion of FL
%     \item {\st{Diffusion of FL}}
%     % Security threats to FL
%     \item {\st{Security threats to FL with particular focus on model poisoning}}
%     % Limitations of existing countermeasures
%     \item {\st{Current countermeasures (e.g., KRUM) and their limitations}}
%     % Proposed method and its advantages
%     \item {\st{Intuitive description of the proposed method and its difference (i.e., advantages) w.r.t. state of the art}}
%     % Main contributions
%     \item {\st{Summary of the main contributions of this work}}
%     % Paper's structure and organization
%     \item {\st{Paper's structure and organization}}
% \end{itemize}

% Diffusion of FL
Recently, {\em federated learning} (FL) has emerged as the leading paradigm for training distributed, large-scale, and privacy-preserving machine learning (ML) systems~\cite{mcmahan2017googleai,mcmahan2017aistats}. 
The core idea of FL is to allow multiple edge clients to collaboratively train a shared, global model without disclosing their local private training data.
%Specifically, an FL system consists of a central server and many edge clients; 
A typical FL round involves the following steps: {\em(i)} the server randomly picks some clients and sends them the current, global model; {\em(ii)} each selected client locally trains its model with its own private data; then, it sends the resulting local model to the server;\footnote{Whenever we refer to global/local model, we mean global/local model {\em parameters}.} {\em(iii)} the server updates the global model by computing an \emph{aggregation function}, usually the average (FedAvg), on the local models received from clients.
% \begin{enumerate}
%     \item[{\em(i)}] the server sends the current, global model to the clients and appoints some of them for training;
%     \item[{\em(ii)}] each selected client locally trains its copy of the global model with its own private data; then, it sends the resulting local model back to the server;\footnote{Whenever we refer to global/local model, we mean global/local model {\em parameters}.}
%     \item[{\em(iii)}] the server updates the global model by computing an \emph{aggregation function} on the local models received from clients (by default, the average, also referred to as FedAvg~\cite{mcmahan2017aistats}).
% \end{enumerate}
This process goes on until the global model converges. %(e.g., after a certain number of rounds or other similar stopping criteria).
%\\
% The advantages of FL over the traditional, centralized learning paradigm are undoubtedly clear in terms of flexibility/scalability (clients can join/disconnect from the FL network dynamically), network communications (only model weights\footnote{We will use \textit{parameters} and \textit{weights} interchangeably.} are exchanged between clients and server), and privacy (each client's private training data is kept local at the client's end and not uploaded to the server).
\\
% Security threats to FL
%However, the growing adoption of FL also raises security concerns~\cite{costa2022covert}, particularly about its confidentiality, integrity, and availability.
Although its advantages over standard ML, FL also raises security concerns~\cite{costa2022covert}. %, particularly about its confidentiality, integrity, and availability~\cite{costa2022covert}.
% OLD, LONG VERSION
% Indeed, some work deals with privacy leakage that may expose the local data of some clients~\cite{melis2019sp}. 
% A large body of work, instead, investigates attacks that usually aim to detriment the predictive accuracy of the learned global model. For instance, \emph{data poisoning} attacks achieve this goal by letting an adversary pollute the training set of some corrupt FL clients with maliciously crafted examples~\cite{jagielski2018sp}.
% Similarly, in \emph{model poisoning} the attacker attempts to tweak the global model weights~\cite{bhagoji2019pmlr} by directly perturbing the local model's weights of some infected FL clients before these are sent to the central server for aggregation, usually via so-called Byzantine attacks. 
% It turns out that Byzantine model poisoning attacks severely impact standard FedAvg; therefore, more robust aggregation functions must be designed to make FL systems secure.
Here, we focus on \emph{untargeted model poisoning} attacks~\cite{bhagoji2019pmlr}, where an adversary attempts to tweak the global model weights %\footnote{We will use the terms \textit{parameters} and \textit{weights} interchangeably.} 
by directly perturbing the local model's parameters of some infected clients before these are sent to the central server for aggregation.
In doing so, the adversary aims to jeopardize the global model \textit{indiscriminately} at inference time.
Such model poisoning attacks severely impact standard FedAvg; therefore, more robust aggregation functions must be designed to secure FL systems.
\\
% In this paper, we focus on designing a novel robust aggregation scheme at the server's end to contrast the effect of Byzantine model poisoning attacks.
%
% Current countermeasures and their limitations
%Several countermeasures have been proposed in the literature to combat model poisoning attacks on FL systems.
% Some methods use simple statistics more robust than plain average to smooth the impact of malicious updates (e.g., Trimmed Mean and FedMedian~\cite{yin2018icml}). 
% Other defenses implement outlier detection techniques to discard malicious updates from the aggregation performed at the server's end. Those are either based on heuristics (e.g., Krum/Multi-Krum~\cite{blanchard2017nips} and Bulyan~\cite{mhamdi2018pmlr}) or data-driven approaches (e.g., K-means clustering~\cite{shen2016acm} or DnC via spectral analysis~\cite{shejwalkar2021ndss}). 
% Finally, some strategies rely on a centralized ``source of trust'' to spot potential malicious updates (e.g., FLTrust~\cite{cao2020fltrust}).
% Several countermeasures have been proposed in the literature to combat model poisoning attacks on FL systems, i.e., to discard possible malicious local updates from the aggregation performed at the server's end. 
% These techniques range from simple statistics more robust than plain average (e.g., Trimmed Mean and FedMedian~\cite{yin2018icml}) to outlier detection heuristics (e.g., Krum/Multi-Krum~\cite{blanchard2017nips} and Bulyan~\cite{mhamdi2018pmlr}) or data-driven approaches (e.g., spectral analysis via K-means clustering~\cite{shen2016acm} or spectral analysis), or methods based on ``source of trust'' (e.g., FLTrust~\cite{cao2020fltrust}).
% OLD, LONG VERSION
%Several countermeasures have been proposed in the literature to combat Byzantine model poisoning attacks on FL systems.
% Descriptive statistics
% For example, Trimmed Mean and FedMedian aggregate local model updates using more robust statistics than standard average~\cite{yin2018icml}.
%
% % Heuristics for outlier detection
% Many existing Byzantine-resilient strategies implement some outlier detection heuristics to discard the model updates sent by potentially malicious clients from the input of the aggregation function.
% One of the most popular heuristics is Krum~\cite{blanchard2017nips}.
% This strategy tries to mitigate the impact of Byzantine attacks by selecting as a global model the local model with the smallest sum of Euclidean distances to {\em all} the other local models.
% Although powerful, Krum requires the server to know (or, at least, estimate) the number of malicious FL clients upfront, which is generally impossible in a realistic attack scenario. %
% Moreover, Krum may become ineffective for complex, high-dimensional model parameter spaces due to the curse of dimensionality.
% Bulyan~\cite{mhamdi2018pmlr} tries to overcome this issue by combining Krum with a variant of Trimmed Mean.
% % Data-driven outlier detection
% Other strategies use data-driven outlier detection techniques -- e.g., via K-means clustering~\cite{shen2016acm} -- to spot potential malicious local model updates. 
% %For instance, Shen et al. propose to cluster local model updates with K-means and thus identify outliers.
%
% % Other techniques
% As far as the server is concerned, any local model received can be from a potential malicious client. 
% FLTrust~\cite{cao2020fltrust} assumes the server acts as a client, i.e., trains a local model on an additional {\em trustworthy} dataset at the server's end and compares it against all the local models from other clients. 
% This way, the server can rely on some ``source of trust'' when discarding potentially malicious clients.
%\\
% Limitations of existing Byzantine-resilient strategies
Unfortunately, existing defense mechanisms either rely on simple heuristics (e.g., Trimmed Mean and FedMedian by~\cite{yin2018icml}) or need strong and unrealistic assumptions to work effectively (e.g., foreknowledge or estimation of the number of malicious clients in the FL system, as for Krum/Multi-Krum~\cite{blanchard2017nips} and Bulyan~\cite{mhamdi2018pmlr}, which, however, cannot exceed a fixed threshold).
Furthermore, outlier detection methods using K-means clustering~\cite{shen2016acm} or spectral analysis like DnC~\cite{shejwalkar2021ndss} do not directly consider the temporal evolution of local model updates received.
Finally, strategies like FLTrust~\cite{cao2020fltrust} require the server to collect its own dataset and act as a proper client, thereby altering the standard FL protocol.
\\
% OLD, LONG VERSION
% Overall, existing Byzantine-resilient strategies are either simple heuristics (e.g., FedMedian) or, if they are more complex, they rely on strong and unrealistic assumptions to work effectively (e.g., knowing the number of malicious clients in the FL system in advance, as for Krum and alike).
% Furthermore, data-driven outlier detection methods do not consider the temporary evolution of local model updates received (e.g., K-means clustering). 
% Finally, strategies like FLTrust requires the server to collect its own dataset and act as a proper client, thereby altering the standard FL protocol.
%
% Description of the proposed method
This work introduces a novel pre-aggregation \textit{filter} robust to untargeted model poisoning attacks. Notably, this filter $(i)$ operates without requiring prior knowledge or constraints on the number of malicious clients and $(ii)$ inherently integrates temporal dependencies. 
The FL server can employ this filter as a preprocessing step before applying \textit{any} aggregation function, be it standard like FedAvg or robust like Krum or Bulyan.
Specifically, we formulate the problem of identifying corrupted updates as a multidimensional (i.e., matrix-valued) time series anomaly detection task. 
The key idea is that legitimate local updates, resulting from well-calibrated iterative procedures like stochastic gradient descent (SGD) with an appropriate learning rate, show \textit{higher predictability} compared to malicious updates. This hypothesis stems from the fact that the sequence of gradients (thus, model parameters) observed during legitimate training exhibit regular patterns, as validated in Section~\ref{subsec:intuition}. %until convergence. 
%This regularity may be more pronounced for smooth convex loss functions, but it can still be captured within an appropriate time window, even for more complex and convoluted loss surfaces. 
%We provide evidence of this claim in Appendix~B, where we show that the average mutual information (i.e., ``predictability''), calculated over pairs of legitimate model updates sent at different FL rounds, is significantly higher than the corresponding computation for a malicious client.
\\
Inspired by the matrix autoregressive (MAR) framework for multidimensional time series forecasting~\cite{chen2021je}, we propose the FLANDERS ({\em \textbf{F}ederated \textbf{L}earning meets \textbf{AN}omaly \textbf{DE}tection for a \textbf{R}obust and \textbf{S}ecure}) filter.
The main advantages of FLANDERS over existing strategies like FLDetector~\cite{zhao2020multivariate} are its resilience to large-scale attacks, where $50\%$ or more FL participants are hostile, and the capability of working under realistic non-iid scenarios.
We attribute such a capability to two key factors: $(i)$ FLANDERS works without knowing a priori the ratio of corrupted clients, and $(ii)$ it embodies temporal dependencies between intra- and inter-client updates, quickly recognizing local model drifts caused by evil players. Below, we summarize our main contributions:

\begin{itemize}
\item[{\em(i)}]
We provide empirical evidence that the sequence of models sent by legitimate clients is more predictable than those of malicious participants performing untargeted model poisoning attacks.
\\
\item[{\em(ii)}] 
We introduce FLANDERS, the first pre-aggregation filter for FL robust to untargeted model poisoning based on multidimensional time series anomaly detection.
\\
\item[{\em(iii)}] 
We integrate FLANDERS into Flower,\footnote{\scriptsize{\url{https://flower.dev/}}} a popular FL simulation framework for reproducibility.
\\
\item[{\em(iv)}] 
We show that FLANDERS improves the robustness of the existing aggregation methods under multiple settings: different datasets, client's data distribution (non-iid), models, and attack scenarios.
\\
\item[{\em(v)}] 
We publicly release all the implementation code of FLANDERS along with our experiments.\footnote{\scriptsize{\url{https://anonymous.4open.science/r/flanders_exp-7EEB}}}
\end{itemize}

% Paper's structure and organization
The remainder of the paper is structured as follows. %some related work and the current state-of-the-art solutions to security issues that FL entails. 
Section~\ref{sec:background} covers background and preliminaries. 
In Section~\ref{sec:related}, we discuss related work.
Section~\ref{sec:problem} and Section~\ref{sec:method} describe the problem formulation and the method proposed. % to tackle it. 
Section~\ref{sec:experiments} gathers experimental results. %, and Section~\ref{sec:limitations} discusses some limitations of this work.
Finally, we conclude in Section~\ref{sec:conclusion}.
 %discusses the limitations of this work and draws future research directions.
%reports conclusions and draws perspectives for future research directions.

%%%%%%% OLD %%%%%%%
%to overcome the resilience of Byzantine failures in distributed Stochastic Gradient Descent computations. 
% The strength of Krum is its time complexity, which is linear in the gradient dimension. 
% However, the robustness of the approach is guaranteed for gradient-based learning applications only when the majority of the clients are not compromised. 
% Besides, the aggregation mechanism of Krum, as well as that of similar methods, is robust from a coarse-grained perspective and does not provide solutions to errors and perturbations that may occur at inference time.
%A related approach to~\cite{blanchard2017nips} is the work of Su et al.~\cite{su2016dc}. Here, the authors propose an iterated approximate agreement to tackle a multi-layer scenario attacked by Byzantine agents. 
%However, the method works efficiently on the sole discrete context and it is inapplicable to continuous state environments.
%\gabri{Maybe, we should just talk about the main limitations of existing countermeasures without digging into their details (or, we can just mention Krum as this is the most popular one). I will move the description of all these methods to the Related Work section.}

\section{Related work}
% There is extensive recent work on speeding up analytical queries due to the need for consistent execution times in the face of the explosive growth in the volume of available data.
% In this section, we divide existing work into two categories: maintaining data freshness in MVs (\Cref{sec:server_side}) and optimizations for minimizing ad-hoc query latency (\Cref{sec:client_side}).

% \subsection{Maintaining Data Freshness in MVs}
% \label{sec:server_side}
% There exists a variety of data warehousing applications aimed at supporting low-latency analytical queries on fresh data.
% In particular, these applications require efficiency in the propagation of newly ingested data into downstream MVs.
 
\mypara{Efficient MV Refresh}
Incremental view maintenance (IVM) aims to update MVs to reflect newly ingested data, taking advantage of already computed results to perform the update in a manner more efficient than computing from scratch (full refresh)
~\cite{ahmad2012dbtoaster,mcsherry2013differential,armbrust2013generalized,zeng2016iolap, palpanas2002incremental, griffin1995incremental, agiwal2021napa, braun2015analytics}. 
There is an abundance of work in IVM, including incremental updates on duplicate values~\cite{griffin1995incremental}, non-distributive aggregate functions~\cite{palpanas2002incremental}, higher-order views~\cite{ahmad2012dbtoaster}, and sliding windows~\cite{braun2015analytics}. 
More recent works also investigate the scalability aspect of IVM, proposing scale-independent updates~\cite{armbrust2013generalized} and sampled views~\cite{zeng2016iolap}. Since \system is applicable to arbitrary SQL statements, \system is orthogonal to and is fully compatible with existing IVM techniques.

\mypara{MV Refresh Scheduling}
There exist works on scheduling the refresh of a MV set focusing on resolving cyclic dependencies~\cite{folkert2005optimizing}, minimizing weighted average staleness~\cite{golab2009scheduling}, and prioritizing MVs with the highest speedups on predicted future queries~\cite{ahmed2020automated}.
\system's scheduling to speed up the end-to-end refresh of the MV set is not addressed in existing works.

\mypara{DAG Workflow Scheduling}
The execution of workloads consisting of individual jobs with acyclic dependencies is a well-studied topic~\cite{apacheoozie,sparkdag,marchal2018parallel,bathie2020revisiting,baruah2022ilp}; many of these techniques can be applied to MV refresh runs studied in this paper.
Existing workflow scheduling systems such as Apache Oozie~\cite{apacheoozie}, Apache Airflow~\cite{airflow}, and Spark DAG scheduler~\cite{sparkdag} automate the execution of user-defined workflows following a topological order.
There are recent works aimed at finding more optimal execution orders in terms of peak memory usage~\cite{marchal2018parallel, bathie2020revisiting} and execution time on parallel platforms~\cite{baruah2022ilp}.
While \system is designed for use with MV refresh runs/workloads, our technique on joint scheduling and optimization can be reasonably applied to general workloads as a possible future direction.

% \paragraph{Incremental MV indexing}
% Update-optimized indices such as the log-structured merge-trees (LSM)~\cite{o1996log} are used for indexing MVs due to frequent updates induced by data ingestion~\cite{gupta2016mesa,agiwal2021napa}.
% \system is orthogonal to indexing: \system is capable of efficiently performing MV refresh runs regardless of whether the individual MVs are indexed or not.

% \subsection{Ad-hoc Query Latency Reduction}
% \label{sec:client_side}

% The minimization of ad-hoc analytical query response times is a well-studied topic due to latency being negatively correlated with the productivity of a data analyst during a data analysis session~\cite{liu2014effects}.
% Sessions are commonly conducted within visualization systems that contain a variety of optimization techniques to ensure that query response times fall within a certain latency tolerance.

% \mypara{Data prefetching}
% Data is often loaded into memory on a by-need basis in visualization systems to minimize interference with user-issued query computations~\cite{mani2017effective,xin2021enhancing,galakatos2017revisiting, yan2020auto, battle2016dynamic, crotty2016case, jalaparti2018netco}. 
% Query-time data retrieval can be significantly expedited by anticipating the data usage of the user in future queries and pre-loading the data into memory during the downtime between user queries (`think time'). SMART~\cite{mani2017effective} prefetches data for modified versions of current user-issued queries with different filters and dimensions. A-WARE~\cite{crotty2016case} maintains a sample store constantly refined through ingesting data based on speculations of future plots.
% ForeCache~\cite{battle2016dynamic} uses an SVM to predict the user's current analysis phase and accordingly prefetches data tiles partitioned based on different numerical values. NetCo predicts future queries via log analysis, and solves an ILP formulation to prefetch data to maximize the number of SLO-meeting queries~\cite{jalaparti2018netco}.
% In the case of MV refresh workloads, `think time' is nonexistent as individual MVs are refreshed back-to-back, rendering data prefetching techniques non-applicable.

\mypara{Intermediate Data Caching}
Some existing data visualization systems cache user-defined variables to support the typical incremental construction of data visualizations~\cite{zgraggen2016progressive, eichmann2020idebench} during data analysis sessions~\cite{jupyter, rstudio, colab}. 
Recent work proposes a management scheme for these cached variables under a memory constraint that greedily keeps variables with the highest estimated time savings based on predicted future user behavior via neural networks~\cite{xin2021enhancing}.
While useful for data visualization, a greedy approach to memory management fails to achieve satisfactory results compared to \system.

\mypara{Intermediate Result Reuse}

There exist works on storing intermediate results from computations to speedup future computations~\cite{yang2018intermediate, dursun2017revisiting, nagel2013recycling, michiardi2019memory, galakatos2017revisiting}.
Studied topics include the identification of reuse opportunities by finding overlaps in computation graphs of successive jobs~\cite{yang2018intermediate, michiardi2019memory},
selective storage under a space constraint with heuristics such as reuse probability~\cite{dursun2017revisiting}, expected savings~\cite{yang2018intermediate}, and recompute-storage cost difference~\cite{nagel2013recycling},
and rewriting incoming jobs to take advantage of stored intermediates~\cite{galakatos2017revisiting}.
These works share similarity with \system in their selection of items to store under a memory constraint, however, \system's problem setting requires it to uniquely consider the joint (re)ordering of job executions along with the selection of items.

% work that considers both job execution (re)order as well as intermediate result caching with a bounded amount of memory. but notably lack the joint aspect of \system and cannot be used to achieve immediate speedup on an incoming MV refresh run if no intermediates are stored beforehand. 

\mypara{Incremental Query Processing} Incremental processing (IQP) is useful for cases where not all data required for a query is immediately available. Similar to online aggregation~\cite{hellerstein1997online}, initial results of a query are computed on a subset of required data and progressively refined as the rest of the required data arrives in a predictable pattern~\cite{tang2019intermittent,wangtempura}. Tang et al. propose a dynamic programming formulation to pick intermediate states to store in memory given a limited memory budget~\cite{tang2019intermittent}. Tempura rewrites the query plan for more efficient execution based on predicted data arrival patterns~\cite{wangtempura}. While similarities exist between the problem setting of IQP and \system, such as management of bounded memory, \system notably includes additional joint optimization for the order of MV updates.

% \paragraph{Sampling}
% Sampling has seen wide use in visualization systems for reducing the computation time of ad-hoc queries by computing an approximate result over a subset of data as exact results are not always required by the user~\cite{crotty2016case, mani2017effective, zgraggen2014panoramicdata, kraska2021northstar, galakatos2017revisiting, kandula2016quickr}. 
% Commonly studied topics in sampling for ad-hoc queries include complex query sampling~\cite{kandula2016quickr}, rare event aggregation~\cite{kraska2021northstar, galakatos2017revisiting}, and maintaining consistency between related sampled visualizations~\cite{zgraggen2014panoramicdata}.
% Sampling server-side at the MV level compromises the assumptions of downstream applications and is thus not considered in \system.

% \paragraph{Progressive visualization}
% The latency tolerance for time-consuming queries can be circumvented by presenting a partially-computed visualization to the user within the tolerance, which is then incrementally refined until it is fully accurate~\cite{rahman2017ve, zgraggen2016progressive, crotty2015vizdom, kraska2021northstar, kamat2017infiniviz}.
% Example plots which benefit from progressive visualization include bar charts~\cite{kamat2017infiniviz} and heatmaps~\cite{rahman2017ve}.
% Similar to sampling, study on this topic is orthogonal to \system as pushing out partially-updated MVs compromises downstream assumptions.

\section{Notation and Preliminaries}\label{sec_prel}
Let $\mathbb{Z}_{>0}$ denote the set of positive integers and let $\mathbb{Z}_{[a,b]}$ denote the set of integers in the interval $[a,b]$. The $m\times m$ identity matrix is denoted by $I_m$ and its columns by $e_i$ for $i\in\mathbb{Z}_{[1,m]}$. We use $\mathbf{0}$ to denote a vector or a matrix of zeros of appropriate dimensions. For a sequence $\{z_k\}_{k=0}^{N-1}$ with $z_k\in\mathbb{R}^\eta$, we denote its stacked vector as $z = \begin{bmatrix}z_0^\top &z_1^\top & \dots & z_{N-1}^\top\end{bmatrix}^\top$ and a stacked window of it as $z_{[l,j]} = \begin{bmatrix}z_l^\top &z_{l+1}^\top & \dots & z_{j}^\top\end{bmatrix}^\top$ with $0\leq l<j$.\par
Persistence of excitation of a sequence and its extension to multiple sequences \cite{vanWaarde20} are defined as follows.
\begin{definition} The sequence \(\{z_k\}_{k=0}^{N-1}\), $z_k\in\mathbb{R}^{\eta}$, is said to be persistently exciting of order \(L\) if \(\textup{rank}(\mathscr{H}_{L}(z))=\eta L\), where $\mathscr{H}_L(z) = \begin{bmatrix}
		z_{[0,L-1]} & z_{[1,L]} & \cdots & z_{[N-L,N-1]}
	\end{bmatrix}$.
	\label{def_PE}
\end{definition}
\begin{definition}[\cite{vanWaarde20}]\label{def_cPE}
	The sequences $\{z_k^{(j)}\}_{k=0}^{N_j-1}$, with $z_k^{(j)}\in\mathbb{R}^\eta$ and $j\in\mathbb{Z}_{[1,r]}$, are said to be \textit{collectively persistently exciting} of order $L$ if rank$(\mathcal{H}_L(\mathscr{Z}))=\eta L$, where $\mathscr{Z} = \begin{bmatrix}
		(z^{(1)})^\top & \cdots & (z^{(r)})^\top
	\end{bmatrix}^\top,$ and
	\begin{equation*}
		\mathcal{H}_L(\mathscr{Z}) = \begin{bmatrix}
			\mathscr{H}_L(z^{(1)}) & \cdots & \mathscr{H}_L(z^{(r)})
		\end{bmatrix}.
	\end{equation*}
\end{definition}

\section{Mobiprox Framework}
\label{sec:mobiprox}

%\sasa{this is a good motivation -- may be emphasized in the intro more.}
%\textbf{Data scientists are not mobile system experts.} Our guiding vision is that deep learning modeling should be disentangled from system-level performance optimization. Mobiprox aims to support efficient on-device execution of any pre-trained mobile network architecture based on convolutional and fully-connected layers. Furthermore, we do not require that a developer knows which optimizations (in our case -- execution approximations) are available on the device. Still, we give a developer an option of (dynamically) setting an operational point along \textit{the inference accuracy -- resource usage} trade-off line, yet, in the limit case, the developer need not even set this point, but merely let Mobiprox tune the execution according to its internal approximation adaptation algorithms.


%\subsection{Overview}
%\label{sec:mobiprox-overview}
Mobiprox, our novel framework for enabling dynamic approximation of mobile DL, is sketched in Figure~\ref{fig:amc-pipeline}. An Android app compiled with Mobiprox can use an arbitrary runtime approximation adaptation strategy for its DL models (e.g., ``run low quality network when battery is low'', ``run high quality inference when user is at a specific location'', etc.). To achieve this, Mobiprox operates with approximation configurations, i.e. combinations of per-layer approximations of a pre-trained DL model. Mobiprox first uses ApproxTuner to examine the impact of different configurations on the inference accuracy and the speedup. Each approximation configuration yields a point in the accuracy--speedup space, and Mobiprox identifies the most promising configurations that form the Pareto front in this space and then profiles their actual performance on the mobile platform using the novel \textit{HPVM Profilier for Android}. Mobiprox's Android-based \textit{OpenCL runtime} then enables execution of and dynamic switching between approximation configurations on a mobile device. Using the \textit{JNI interface library} generated by the framework, the mobile application can control the approximation level of the NN. Finally, as part of Mobiprox, we also devise \textit{Approximation adaptation strategies }that leverage the generated trade-off curves to match the required and delivered quality of computation, thus enable energy-efficient DL on mobile devices. 
%\hashim{This last point is the new contribution in Mobiprox IMO. It's buried too deep in the para. I would suggest starting with it and reversing where you talk about ApproxTuner at the end - you want to say ApproxTuner is just a tool you use for approximation-selection - and you could use any other approximation tool similarly. The key capability of context-dependent optimization in Mobiprox seems agnostic of the choice of approximation-tuning tool used. }

%Our solution harnesses ApproxTuner (AT), a compiler and optimiser that investigates the effect of different  per-operation approximations on the inference accuracy of a neural network. In addition, AT identifies energy-accuracy trade-off points achieved by different sets of approximations applied to layers of a neural network. Yet, AT does not natively support mobile devices, nor does it provide any guidance on which approximation trade-off point to operate on at a particular time, which is crucial, having in mind that user requirements and available resources may vary in the mobile context.

%In Mobiprox we addresses these limitations by first extending ApproxTuner's runtime with a tensor operation back-end for mobile devices that is capable of executing approximated neural network inference; second, we devise new tools for profiling approximated neural networks, so their true performance on Android devices can be characterised; finally, we devise new adaptation methods that leverage generated approximation trade-off curves to match the required and the delivered quality of computation, thus enabling energy-efficient DL on mobile devices. 




\begin{figure}[t]
    \centering
    \includegraphics[width=\linewidth]{figures/system-diagram.png}
    \caption{Mobiprox overview. OpenCL run-time supports running the inference binary (controlled either directly from the C code, or via JNI from the main Java/Kotlin app) with a varying level of approximation. The HPVM Profiler for Android helps us chart the \textit{approximation -- resource usage} space, so that the Approximation adaptation strategy wihtin the Android app can set the approximation level dynamically at runtime. Main Mobiprox modules are colored green, while the supporting pre-existing modules are grayed out.}
    \label{fig:amc-pipeline}
\end{figure}

\subsection{Charting approximation space} 
\label{sec:mobiprox:charting}
%Neural Network approximation using ApproxHPVM and ApproxTuner}

%\sasa{the next 3 paragraphs can be condensed. Put more emphasis on what is new in Mobiprox -- I like the discussion starting from "However, the above method does not readily translate...". It's better to present first your method and then say why the existing one did not work, for readers' attention. }

Each of the approximation techniques described in Section~\ref{sec:preliminaries-acts} exposes one or more \textbf{approximation knobs} that can change the level of approximation and thus adjust the accuracy and the execution time (consequently the energy efficiency) of a tensor operation. These knobs are \ttit{offset} and \ttit{stride} for convolution perforation and filter sampling, and an indicator \texttt{\_fp16} of whether an operation is executed using half-precision quantization. An \textbf{approximation configuration} is a set of pairs $\langle \mathrm{Op.}, \mathrm{KnobValue} \rangle$ for all operations in a given NN. Each of the configurations leads to a single \textbf{trade-off point} on an speedup-accuracy trade-off curve. 

%An \textbf{HPVM approximation configuration file} is a collection of approximation configurations, or a \textbf{trade-off curve}. This file is read at the initialisation stage of an ApproxHPVM-transformed neural network.

%Table~\ref{tab:knobs} shows the available approximation knobs for these approximation techniques. The behaviour of each knob is defined by its knob name and its parameters. For the three described approximations, only parameters \ttit{stride} and \ttit{offset} define the approximation. Parameter~\ttit{stride} defines the order of approximations. Parameter~\ttit{offset} defines the index of the first omitted row or column in the case of row and column perforations respectively and the index of the first eliminated component in the case of filter sampling. Additionally, each operation can be run at reduced precision with 16-bit floating point data (indicated by \texttt{\_fp16} extension).

% \begin{table}[H]
%     \caption{Convolution approximation knobs and their parameters. Parameters \ttit{stride} and \ttit{offset} define the approximation; each operation can be run at reduced precision with 16-bit floating point data (indicated by \texttt{\_fp16} extension).}
%     \begin{center}
%     \begin{tabular}{lll}
%         \toprule
%         Approximation type & Knob name & Knob parameters \\
%         \midrule
%         Row perforation     & {{\tt perf}[{\tt\_fp16}]} & \ttit{stride}, \texttt{1}, \ttit{offset} \\
%         Column  perforation & {{\tt perf}[{\tt\_fp16}]} & \texttt{1}, \ttit{stride}, \ttit{offset} \\
%         Filter sampling     & {{\tt samp}[{\tt\_fp16}]} & \ttit{stride}, \ttit{offset}, \textit{unused} \\
%         \bottomrule
%         \end{tabular}
%     \end{center}
% \label{tab:knobs}
% \end{table}

%Once a neural network is compiled using ApproxHPVM, 
The tuner heuristically searches the space of possible approximations and determines a Pareto frontier of approximation configurations that maximise the execution speedup at different \textbf{quality of service (QoS) loss} points. This loss is a real number defined as a difference between the classification accuracy, over a representative validation dataset, of a non-approximated and an approximated DL model. %Two different methods of determining the approximated model's loss are offered: 1) {empirical tuning} that determines the loss by testing the inference accuracy of a full approximated DL model, and a faster 2) {predictive tuning} where the impact of an approximation configuration is estimated through a compositional error model that analytically combines the accuracy loss observed at individual network layers. 


%However, the amount of different configurations is too large to search exhaustively, thus, OpenTuner\cite{opentuner} is used to heuristically search the space of supported approximation parameters. 
%The goal of the tuning is to maximise the execution speedup 
% 
%\newcommand{\qos}{\mathrm{QoS}\relax}
%\newcommand{\qosLoss}{\qos_\mathrm{loss}\relax}
%\newcommand{\qosBase}{\qos_\mathrm{base}\relax}
%\newcommand{\qosApprox}{\qos_\mathrm{approx}\relax}
% 
%\begin{equation}
%    \label{eq:qosloss}
%    \qosLoss = \qosBase - \qosApprox,    
%\end{equation}
% 
%$\qosLoss = \qosBase - \qosApprox$,  where $\qosBase$ is the classification accuracy of a non-approximated model and $\qosApprox$ is the classification accuracy of an approximated model. 

%ApproxTuner offers two different methods of determining $\qosApprox$. When using \textbf{empirical tuning}, $\qosApprox$ is determined by executing a complete neural network inference using the ApproxHPVM-compiled network. However, such inference is computationally expensive. ApproxTuner speeds up the optimisation metric evaluation by introducing \textbf{predictive tuning}. With predictive tuning, the impact of all \textit{approximation knobs} on $\qosApprox$ is measured separately for each layer in the neural network. Different combinations of approximation knob values yield approximation configurations and the search for optimal approximation configurations is guided by error composition models, which define how to combine per-layer information of approximations into the final $\qosApprox$. An example of such a composition model would be a linear combination of QoS losses contributed by approximation knobs used in an approximation configuration. 

However, the method described above for determining the optimal approximation configurations does not readily translate to mobile devices. The mobile platform is substantially different from the server used for fast heuristic-based approximation configuration profiling. The specifics of GPU-based execution (e.g., CUDA vs OpenCL), heterogeneous CPUs with fewer cores, and other factors mean that the results of the configuration search performed on a server are a rather poor representation of the actual approximated NN performance on a mobile. In Figure~\ref{fig:combined-pareto-android}, on the example of a MobileNetV2 model used for HAR (detailed in Section~\ref{sec:methodology}), we show the actual on-mobile-device speedup and QoS loss achieved by the approximation configurations that ApproxTuner identified as the most promising. While the Pareto points obtained on a server generally remain relevant, the achieved on-device speedup is about 50\% lower on the mobile.

% \begin{figure}[!htb]
%     \newcommand{\w}{0.45\linewidth}
%     \centering
%     \begin{subfigure}{\w}
%         \centering
%         \includegraphics[width=\linewidth]{figures/pareto-all.pdf}
%         \caption{Pareto frontier (blue line) of generated approximation configurations (red dots) as determined by tuning on a server.}
%         \label{fig:combined-pareto}
%     \end{subfigure}
%     \hspace{1cm}
%     \begin{subfigure}{\w}
%         \centering
%         \includegraphics[width=\linewidth]{figures/pareto-android.pdf}
%         \caption{Configurations from Figure~\ref{fig:combined-pareto} Pareto frontier profiled on Android ASUS TinkerBoard S. Configurations that defy the Pareto front principle are marked as outliers.
%         }%\sasa{The difference between the predicted and obtained speedups on android. Hashim can chime in here}}
%         \label{fig:combined-pareto-android}
%     \end{subfigure}
%     \caption{Comparison of the achieved speedup and the resulting QoS (inference accuracy) loss for approximation configurations selected by the on-server tuning with the same configurations ran on a mobile platform. Note the different scaling of the y-axis.}
%     \label{fig:pareto}
% \end{figure}


\begin{figure} 
    \centering
  \subfloat[Tuning on server\label{fig:combined-pareto}]{%
       \includegraphics[width=\linewidth]{figures/pareto-all.pdf}}
    \hfill
  \subfloat[Tuning on mobile\label{fig:combined-pareto-android}]{%
        \includegraphics[width=\linewidth]{figures/pareto-android.pdf}}
    
  \caption{Comparison of the achieved speedup and the resulting QoS (inference accuracy) loss for approximation configurations selected by the on-server tuning with the same configurations ran on a mobile platform. Note the different scaling of the y-axis.}
  \label{fig:pareto} 
\end{figure}


% NOTE: Taken out in the IMWUT submission only, as the figure does not agree with this text:
%
%Second, with Mobiprox we aim to support a range of neural networks used for diverse tasks on mobile devices, such those that rely on processing multimodal sensor data. ApproxTuner, on the other hand, predominantly targets computer vision tasks. This, unfortunately, limits the applicability of its tuning. We compare the predicted and the actual performance of configurations identified by ApproxTuner for two neural networks of the same architecture (MobileNetV2) but used for different tasks -- image recognition and HAR from accelerometer data. With the same tuning algorithm we achieved vastly different results. In the case of images, ApproxTuner gives accurate performance predictions. However, tuning on the HAR network yields approximations exhibiting an order of magnitude larger QoS losses than predicted. 
%

%
%In Figure~\ref{fig:approxtuner-compare-ucihar-cifar10} we compare the predicted and the actual performance of configurations identified by ApproxTuner's predictive tuning for two neural networks of the same architecture (MobileNetV2) but used for different tasks -- image recognition ({\mnCifar}) and human activity recognition from accelerometer data ({\mnUci}). With the same tuning algorithm we achieved vastly different results. In the case of {\mnCifar}, ApproxTuner achieves accurate predictions. However, predictive tuning {\mnUci} yielded approximations exhibiting an order of magnitude larger QoS losses than predicted.

% \begin{figure}[!htb]
%      \newcommand{\w}{\linewidth}
%     \newcommand{\sw}{0.43\linewidth}
%     \centering
%     \begin{subfigure}{\sw}
%         \centering
%         \includegraphics[width=\w]{figures/tuning/mncifar-pred-p1.pdf}
%         \caption{{\mnCifar}}
%         \label{fig:conf-cifar10-p1}
%     \end{subfigure}
%     \hspace{1cm}
%     \begin{subfigure}{\sw}
%         \centering
%         \includegraphics[width=\w]{figures/tuning/mnuci-pred-p1.pdf}
%         \caption{{\mnUci}}
%         \label{fig:conf-ucihar-p1}
%     \end{subfigure}
%     \caption{Visualisations of \textbf{predictive tuning} for two neural networks. Points represent energy-accuracy trade-off points (approximation configurations).\sasa{I still don't think we should keep this plot in the paper. Fine to say in the words only what was said as the justification of empirical tuning}}
%     \label{fig:approxtuner-compare-ucihar-cifar10}
%     %\sasa{This plot should go away -- the second uci-har plot does not show a good result. }%octavian: I removed the left-sided subfigures in each case, they were confusing. We think leaving this might be of use to discuss how Mobiprox behaves on image data vs. time-domain data%
% \end{figure}

Mobiprox therefore introduces a novel configuration identification approach. First, we perform tuning on a computer cluster to identify candidate approximation configurations. Then, we develop an Android-based profiler (described in Section~\ref{sec:hpvm-profiler-android}) that runs each candidate configuration on a mobile device and obtains a realistic picture of the approximated neural network performance. The resulting picture of the speedup -- QoS loss space charted by these configurations is then used to guide the dynamic adaptation of the approximation. As a final result, the profiler creates a file listing configurations that will be switched during the mobile app runtime (according to a strategy, e.g. from Section~\ref{sec:strategies}), yet\textit{ only a single network model definition gets deployed on a mobile}.

%Table~\ref{tab:knobs} shows the available approximation knobs for these approximation techniques. The behaviour of each knob is defined by its knob name and its parameters. For the three described approximations, only parameters \ttit{stride} and \ttit{offset} define the approximation. Parameter~\ttit{stride} defines the order of approximations. Parameter~\ttit{offset} defines the index of the first omitted row or column in the case of row and column perforations respectively and the index of the first eliminated component in the case of filter sampling. Additionally, each operation can be run at reduced precision with 16-bit floating point data (indicated by \texttt{\_fp16} extension).


\subsection{Mobiprox -- Android implementation}

Mobiprox, as a concept, is not tied to a particular mobile platform. Yet, amassing 75\% of the smartphone market share Android is the most common mobile deep learning platform and that stands to gain the most from dynamically adaptable approximation, thus, in this section we develop a full Mobiprox compilation pipeline targeting Android devices. 

\subsubsection{Mobiprox Android Compiler}
\label{sec:ndk-integration}

\newcommand{\ndkversion}{21.4.7075529}

Mobile application development with Mobiprox involves compiling the tuning binary and the inference binary (Figure~\ref{fig:amc-pipeline}). While the tuning binary is confined to the server environment and is handled by the ApproxHPVM compilation pipeline, the inference binary is cross-compiled from a server to a mobile (Android). We implement a mechanism for turn-taking between ApproxHPVM and Android NDK LLVM compiler toolchains (Algorithm~\ref{alg:llvm-compile}). We enable this by clearly partitioning the compilation steps and harnessing the fact that LLVM-based compilers apply transformations to an intermediate representation termed LLVM-IR. Note that ApproxHPVM extends LLVM-IR by defining HPVM-IR to which approximation-related transformations are applied. This clear division allows us to use Android NDK for generating the initial LLVM-IR suitable for Android applications and for generating the machine code containing approximate NN operations suitable for mobile GPUs in the final compilation step, while using HPVM-IR transformations for the internal part of the compilation pipeline to insert the description of the desired approximate tensor operations.

 %However, it is crucial that versions of both compilers match. This limits advancements of both compilers -- using newer Android NDK versions or upgrading the base of ApproxHPVM would likely result in compilation errors. %Furthermore, we deem our usage of both compilers case-specific and experimental.


%-- ApproxHPVM and the Android NDK -- to compile neural networks into libraries suitable for execution on an Android device. The ApproxHPVM compiler is implemented as a collection of patches and extensions to the LLVM source code. After these patches are applied, the build process is similar to how LLVM is built.

%At the time of writing, LLVM 9.0.0 is used as the foundation. Since Android NDK's adaptation of LLVM (from now on Android-LLVM) is an open-source project\footnote{Android Open Source Project: \url{https://source.android.com/}} we were leaning towards replacing LLVM 9.0.0 with Android-LLVM. This would enable ApproxHPVM to compile high-level descriptions of neural networks directly into native libraries for Android. However, due to the vast size of both LLVM 9.0.0 and Android-LLVM, this approach did not seem feasible.

%Instead, we chose a different approach.
%Since Android NDK version {\ndkversion}  is based on LLVM 9.0.9, we adapted the HPVM compilation pipeline to use \textbf{both} compilers (ApproxHPVM and Android-LLVM), each at different steps during compilation. This was made possible by clear partitioning of the compilation steps: LLVM-based compilers apply transformations to an intermediate representation called LLVM-IR. HPVM and ApproxHPVM extend LLVM-IR by defining HPVM-IR, on which their transformations are applied. We present this partitioning with Algorithm~\ref{alg:llvm-compile}.

% \begin{figure}
% \caption{ApproxHPVM compilation. Compilers used at each step are shown within curly braces.}
% \label{alg:llvm-compile}
% \newcommand{\IRL}{\mathrm{IR}_\mathrm{LLVM}}
% \newcommand{\IRH}{\mathrm{IR}_\mathrm{HPVM}}

% \begin{algorithmic}[1]
% \STATE $\IRL \gets$ Transform source code into LLVM-IR \COMMENT{\textbf{Android LLVM}}
% \STATE $\IRH \gets$ Transform $\IRL$ code into HPVM-IR \COMMENT{\textbf{ApproxHPVM}}
% \label{alg:llvm-compile:for}
% \FOR{\textbf{each} IR transformation $T_i$ of the compiler} 
% \STATE $\IRH \gets T_i(\IRH)$ \COMMENT{\textbf{ApproxHPVM}}
% \ENDFOR
% \STATE $\IRL \gets$ Transform $\IRH$ into LLVM-IR \COMMENT{\textbf{ApproxHPVM}}
% \label{alg:llvm-compile:endfor}
% \STATE Compile $\IRL$ to machine code \COMMENT{\textbf{Android LLVM}}
% \end{algorithmic}
% \end{figure}


% \begin{figure}[!t]
% \newcommand{\IRL}{\mathrm{IR}_\mathrm{LLVM}}
% \newcommand{\IRH}{\mathrm{IR}_\mathrm{HPVM}}
%  \removelatexerror
%   \begin{algorithm}[H]
% \caption{ApproxHPVM compilation. Compilers used at each step are shown within curly braces.}
% \STATE: $\IRL \gets$ Transform source code into LLVM-IR \COMMENT{\textbf{Android LLVM}}
% \STATE $\IRH \gets$ Transform $\IRL$ code into HPVM-IR \COMMENT{\textbf{ApproxHPVM}}
% \label{alg:llvm-compile:for}
% \FOR{\textbf{each} IR transformation $T_i$ of the compiler} 
% \STATE $\IRH \gets T_i(\IRH)$ \COMMENT{\textbf{ApproxHPVM}}
% \ENDFOR
% \STATE $\IRL \gets$ Transform $\IRH$ into LLVM-IR \COMMENT{\textbf{ApproxHPVM}}
% \label{alg:llvm-compile:endfor}
% \STATE Compile $\IRL$ to machine code \COMMENT{\textbf{Android LLVM}}
%   \end{algorithm}
% \end{figure}

\begin{algorithm}[h]
\SetAlgoLined
\caption{Mobiprox compilation. Compilers used at each step are shown in comments.}
\label{alg:llvm-compile}
\newcommand{\IRL}{IR_{LLVM}}
\newcommand{\IRH}{{IR}_{HPVM}}
\begin{scriptsize}
   
$\IRL \gets$ Transform source code into LLVM-IR \tcp*{\textbf{Android LLVM}}
$\IRH \gets$ Transform $\IRL$ into HPVM-IR \tcp*{\textbf{ApproxHPVM}}
\For{\textbf{each} IR transformation $T_i$ of the compiler} {
$\IRH \gets T_i(\IRH)$ \tcp*{\textbf{ApproxHPVM}}
}
$\IRL \gets$ Transform $\IRH$ into LLVM-IR \tcp*{\textbf{ApproxHPVM}}
Compile $\IRL$ to machine code \tcp*{\textbf{Android LLVM}}
\end{scriptsize}
\end{algorithm}

%This clear partitioning allows us to use Android NDK for generating LLVM-IR and the final compilation step (generating machine code), while using ApproxHPVM's LLVM transformations for the internal part of the compilation pipeline (e.g. inserting approximate tensor operations). However, it is crucial that versions of both compilers match. This limits advancements of both compilers -- using newer Android NDK versions or upgrading the base of ApproxHPVM would likely result in compilation errors. %Furthermore, we deem our usage of both compilers case-specific and experimental.
%Ideally, we would be able to use a single compiler for the whole compilation process. 



% LLVM IR, compiler compatibility, required adaptations.

%\subsubsection{HPVM Runtime for Android}
%\label{sec:android-hpvm-rt}

%The next step in the HPVM compilation pipeline (before machine code generation) is linking the LLVM IR with the HPVM Runtime library (\texttt{hpvm-rt}). This library implements the logic for initialising HPVM, executing HPVM's dataflow graphs on heterogeneous hardware and cleaning up resources upon program termination. The Mobiprox pipeline does not use this library directly, however, ApproxHPVM relies on the functionality implemented by this library.

%To enable successful compilation for Android devices, we had to enable the use of this library on Android devices. This was accomplished by defining patches to the original source code of \texttt{hpvm-rt}, which fixed incompatibilities between LLVM-IR generated by Android-LLVM and HPVM itself. However, following the release of ApproxHPVM and increments of both the LLVM used in HPVM and the version of Android-LLVM, these patches are no longer required.
%Instructions and required source files for building the library for Android are published on GitHub\footnote{HPVM RT for Android: \url{https://github.com/MatevzFa/hpvm-rt-android}}.

\subsubsection{OpenCL Tensor Runtime for Android}
\label{sec:android-rt}

A core component of Mobiprox Android is a Tensor Runtime, which implements tuneable approximable tensor operations for NN inference. The existing support for approximate NN operations for Nvidia CUDA GPUs~\cite{sharif2019approxhpvm} is not suitable for mobiles, which seldom host such hardware. Instead, Mobiprox implements an own tensor runtime using OpenCL, an open standard for GPU-accelerated computation which is available on a wide variety of hardware, including mobile platforms.

%Our first version of the HPVM Tensor Runtime for Android was based on \textbf{ARM ComputeLibrary}\footnote{ARM ComputeLibrary:  \url{https://github.com/ARM-software/ComputeLibrary}}. It is a library optimised for ARM Mali graphics processing units. We chose it as it implements a wide variety of machine learning functions in OpenCL. Furthermore, it is an open source library, which made it easier to implement approximate variants of convolution algorithms. A key optimisation in ARM ComputeLibrary revolves around memory management:
%
%\begin{itemize}[itemsep=-0.2em, topsep=0.2em]
%    \item As described in section~\ref{sec:cnn}, convolutions are implemented as a matrix multiplication of transformed input images and kernel matrices. This introduces additional memory requirements into a neural network computation algorithm.
    
%\item In GPU accelerated algorithms, memory access optimisation is crucial. To make it possible to use vectorised memory reads regardless of matrix size, ARM ComputeLibrary adds padding to all matrices.
%\end{itemize}
%
%The library reduces these additional memory requirements by reusing already allocated blocks of memory that are no longer needed. Thus, it requires knowledge on relationships between NN operations for  successful memory management. However, NN operations in the HPVM Tensor Runtime are implemented as independent operations, which makes aforementioned optimisations impossible without redesigning ApproxHPVM's compilation process.

To enable an enhanced control over low-level concepts (such as memory allocation), we implemented the tensor runtime for Android using CLBlast~\cite{clblast}, an OpenCL implementation of basic linear algebra subprograms (BLAS). However, this library is not intended for deep learning: it does not implement operations commonly used in NNs. Therefore we extended CLBlast with the following operators: \textit{i)} Point-wise tensor addition, \textit{ii)} Bias addition, \textit{iii)} Activation functions (ReLU, clipped ReLU, $tanh$), \textit{iv)} FP-16 -- FP-32 tensor conversion, \textit{v)} Batch normalisation, \textit{vi)} Pooling ($min$, $max$, $average$), \textit{vii)} Convolution approximations operators optimized with tiling and vectorization: \textit{Image-to-Column} ($im2col$) transformations with row perforation, column perforation, and filter sampling, \textit{Kernel-to-Row} ($kn2row$) transformation with filter sampling, and \textit{Interpolation} of missing values in convolution perforation. % \textit{viii}, \textit{ix}, \textit{x} 
%     \item Point-wise tensor addition,
%     \item Bias addition,
%     \item Activation functions (ReLU, clipped ReLU, $tanh$),
%     \item FP-16 -- FP-32 tensor conversion,
%     \item Batch normalisation,
%     \item Pooling ($min$, $max$, $average$),
%     \item Convolution approximations operators optimised with tiling and vectorisation:
%     \begin{itemize}
%         \item \textit{Image-to-Column} ($im2col$) transformations with row perforation, column perforation, and filter sampling,
%         \item \textit{Kernel-to-Row} ($kn2row$) transformation with filter sampling,
%         \item Interpolation of missing values in convolution perforation.
%     \end{itemize}
% \end{itemize}
%
%We validate our implementation of the HPVM Tensor Runtime by comparing the results of running each operation with the results of running the same operation through the reference implementation targeting desktop CPUs. 
Finally, during the mobile app compilation Java Native Interface (JNI) is exposed, enabling the tensor runtime initialization and destruction, NN inference invocation, and dynamic approximation configuration loading.

\subsubsection{HPVM Profiler for Android}
\label{sec:hpvm-profiler-android}

To assess the speedups and consequently the energy efficiency of approximated NNs we implement a profiler tool. The profiler, in the form of a Python library, for a given NN binary measures the accuracy, softmax confidence, and execution time of NN inference on a given test dataset. Due to a high discrepancy between the speedup observed on a mobile device and on a server for the same approximated network (Figure~\ref{fig:pareto}), the profiler uses the Android Debug Bridge (ADB)~\cite{adb} to run measurements on an actual Android device and to transfer the profiling information files back to the host machine for analysis.

%The ApproxTuner project features a profiler tool (\texttt{hpvm-profiler}) which is used to measure speedups and consequently the energy efficiency of compiled neural networks. The profiler is implemented as a Python library that enables selecting an inference binary for profiling and recording profiling information (accuracy, execution time) while executing NN inference on a test data set. It does so by running the inference binary in a sub-process, waiting for its termination, and parsing profiling information files produced by the inference binary. However, the library is designed to run on the same device as the inference binary, and is therefore not usable in our use-case of running the neural networks on Android devices.

%To run profiling in the same manner on Android, Mobiprox implements an adapted version of the profiler (\texttt{hpvm-profiler-android}) that uses Android Debug Bridge (ADB)~\cite{adb} to run measurements on the target Android device and transfer profiling information files back to the host machine for analysis.

% \subsection{Case study: MobileNet-V2 approximation using ApproxTuner for HAR classification}

% \sasa{this can be a section on its own --- also consider moving it to the front instead of related work... }

% We use ApproxTuner to determine a pareto-frontier of approximation configurations for a MobileNet-V2 convolutional neural network trained to perform Human Activity Recognition in Pytorch.

% \subsubsection{MobileNet for Human Activity Recognition}
% \label{sec:mnUci}

% MobileNet~\cite{howard2017mobilenets} is a class of neural network architectures proposed in 2017 that focuses on efficient computer vision applications in embedded systems. MobileNet was chosen for our classification since it was designed to be efficient when executed on power-limited devices and is therefore challenging to optimise further. In addition, it has been used for human activity recognition before~\cite{machidon2021queen}.

% Being designed for computer vision applications, MobileNet is natively works with $32\times32$ 3-channel images. However, human activity recognition data sets, such as the UCI-HAR data set~\cite{anguita2013public}, are commonly constructed on time-domain sensor data (3-axial acceleration, 3-axial rotation). Each data point in the UCI-HAR data set represents a 2.5 second time frame of 128 sensor readings per axis ($x$, $y$, $z$) for each of the 3 sensors (body acceleration, total body acceleration, body rotation). %A more in-depth description of these sensors is presented in section~\ref{sec:android-signalimage}.

% To use the UCI-HAR data set with MobileNet, a transformation of sensor data is required. In our signal images, image channels represent different sensors (body acceleration, total body acceleration, body rotation). Figure~\ref{fig:signal-image} shows how we arranged the feature vectors within each image channel. Feature vectors for axes $x$, $y$ and $z$ were vertically stacked into a $8\times 128$ matrix. Feature vectors for each axis were repeated multiple times, as shown in Figure~\ref{fig:signal-image-in}. Following a matrix transformation, the final $32~\times~32~\times~3$ signal image (Figure~\ref{fig:signal-image-out}) has the following property: neighbouring features in the signal image carry information that is neighbouring in the time domain\footnote{Due to signal image construction technique, this does not hold at the borders of each 32-row block, as visible from colour changes shown in figure~\ref{fig:signal-image-out}.}. This property is highlighted in~\cite{jiang2015human}, which inspired our signal image composition technique. The importance of this aspect and caveats of such signal image composition are detailed in section~\ref{sec:mnUci-tuning}.

% \begin{figure}
%     \newcommand{\w}{\linewidth}
%     \newcommand{\sw}{0.45\linewidth}
%     \centering
    
%     \begin{subfigure}{0.47\w}
%         \centering
%         \includegraphics[width=\w]{figures/signal-image.pdf}
%         \caption{Matrix of stacked feature vectors with size $128~\times~8~\times~3$.}
%         \label{fig:signal-image-in}
%     \end{subfigure}
%     \hspace{1cm}
%     \begin{subfigure}{0.38\w}
%         \centering
%         \includegraphics[width=\w]{figures/signal-image-final.pdf}
%         \caption{Final signal image with size $32~\times~32~\times~3$.}
%         \label{fig:signal-image-out}
%     \end{subfigure}
%     \caption{Signal image composition. Horizontal blocks of data in figure (a) are arranged vertically in figure (b).}
%     \label{fig:signal-image}
% \end{figure}

% We implemented the MobileNet-V2 network architecture (\mnUci{}) using PyTorch. The NN was trained on the UCI-HAR training data set composed of 7352 data points (signal images). We tested our model on the test data set containing 2947 data points. The model achieved a classification accuracy of 90\% on the test data set.
% We also investigated per-class accuracies with a confusion matrix shown in Figure~\ref{fig:mnuci-confusion}.

% \begin{figure}
%     \centering
%     \includegraphics[width=3.5in]{figures/mobilenet_uci-har_0.90.pth.pdf}
%     \hspace{1.3cm}
%     \caption{Confusion matrix for \mnUci{}.}
%     \label{fig:mnuci-confusion}
% \end{figure}


% \begin{itemize}
%     \item Tuning a MobileNet for HAR task turns out to be a unique task
%     \item Hypotesis: CNNs on signal images behave differently as RGB images. RGB images are locally smooth in all directions, singal images are not.
%     \item Show how kernels look like in the trained MobileNET on UCI-HAR and on CIFAR-10.
% \end{itemize}

% \subsubsection{Approximating MobileNet for HAR using ApproxTuner}
% \label{sec:mnUci-tuning}

% To use an approximated neural network for the HAR classification task, we used ApproxTuner to determine approximation configurations suitable for this particular neural network. As described in section~\ref{sec:approxtuner-about}, ApproxTuner offers two avenues of determining approximation configurations. \textit{Predictive} tuning is based on knowledge about the behaviour of various NN approximations that is obtained as a pre-processing step to the iterative optimisation, while \textit{empirical} tuning is evaluating configurations during the iterative optimisation.

% We identified a unique property of our NN ({\mnUci}) compared to image classification networks used in evaluation of ApproxTuner. In our case, predictive tuning predicted accuracy losses that poorly resemble real accuracy losses. In Figure~\ref{fig:approxtuner-compare-ucihar-cifar10}, we compare these tuning results with those for {\mnCifar}. We can observe that with the same tuning algorithm, we achieved vastly different results. In the case of {\mnCifar}, ApproxTuner achieved accurate predictions, which resulted in valid approximation configurations. However, predictive tuning {\mnUci}  yielded approximations that yield QoS losses of up to 60\%, which makes these approximations unusable.


% To confirm our hypothesis that prediction models poorly resemble behaviour of NN {\mnUci}, we used ApproxTuner's \textit{empirical tuning}. Figure~\ref{fig:approxtuner-ucihar-empirical} shows results of this tuning approach. We can see that with empirical tuning ApproxTuner successfully finds. Furthermore, in Figure~\ref{fig:filters} we take a look at the filters of both neural networks. We can see that component values in convolution filters of {\mnCifar} transition smoothly, while in {\mnUci} the differences between neighbouring components are greater. Additionally, we can observe similarity between neighbouring components among the horizontal dimension in the case of {\mnUci}, which we believe is a consequence of the signal-image composition described in section~\ref{sec:mnUci}.

% \begin{figure}[!htb]
%     \newcommand{\sw}{0.75\linewidth}
%     \newcommand{\w}{\linewidth}
%     \centering
%     \includegraphics[width=\sw]{figures/tuning/mnuci-empirical.pdf}
%     \caption{Empirical tuning visualisation for {\mnUci}.}
%     \label{fig:approxtuner-ucihar-empirical}
% \end{figure}


% \begin{figure}[!htb]
%     \newcommand{\w}{\linewidth}
%     \newcommand{\sw}{0.45\linewidth}
%     \centering
%     \begin{subfigure}{0.42\w}
%         \centering
%         \includegraphics[width=\w]{figures/weights-mobilenet_cifar10-conv0.pdf}
%         \caption{\mnCifar}
%     \end{subfigure}
%     \hspace{1cm}
%     \begin{subfigure}{0.42\w}
%         \centering
%         \includegraphics[width=\w]{figures/weights-mobilenet_uci-har-conv0.pdf}
%         \caption{\mnUci}
%     \end{subfigure}
%     \caption{Visualisation of first 16 filters of the first convolutional layer in MobileNet. Filters were globally normalised to values on interval $[0, 1]$, for each network separately. }
%     \label{fig:filters}
% \end{figure}



% We believe that these are the properties that make approximations such as filter sampling unfeasible. The idea behind filter sampling is that neighbouring values in CNN filters and images are similar. This allows for accurate interpolation of missing values. However, neither of these assumptions hold for {\mnUci} (see section~\ref{sec:mnUci}).



\section{Approximation adaptation strategies}
\label{sec:strategies}

%\sasa{This is a very interesting section (from my perspective). It would be nice if the context of the adaptation within the system and its importance was pointed a bit more prominently. Another suggestion would be to extract 'the interface' for the adaptation strategies (e.g., the objective they try to accomplish, the prediction problem, inputs) -- if this is extracted at the beginning of the section, then the presentation of each may be guided by these bullet points. }
%\hashim{100\% agreed!! This is the new novel contribution of this paper and its getting lost by it being too late in the paper. In my opinion if the story revolved around these application-specific/context-specific strategies for tuning, the rest of the technical details on how you use Approxtuner would make more sense given you motivated why you want to dynamic tuning. }

%\sasa{This section needs some work to highlight the novelty -- consider writing a small algorithm here for how it does runtime adaptation. This is probably what you want to advertise the most}

Mobiprox's key strength is its support for context-based runtime adaptation of mobile deep learning approximation. The framework itself deliberately does not prescribe the adaptation strategy allowing a developer to implement an arbitrary set of rules driven by energy needs (e.g. ``use higher approximation when battery level falls below 10\%''), the purpose of use (e.g. ``use more accurate HAR models when a user is exercising''), or even business models (e.g. ``use input-adaptable approximation for premium users''). Algorithm~\ref{alg:adapt-RT} shows the generic integration of the adaptation strategy within Mobiprox. Programming such strategies is trivial, yet, one can envision a more challenging-to-achieve goal, such as ``minimize the energy usage without sacrificing the inference accuracy''. In this section \blue{we harness the natural temporal dependence of the instances of sensed data that is characteristic in many mobile computing applications, and devise three strategies demonstrating that a widely applicable goal of energy minimization can be met with Mobiprox.}


\renewcommand{\algorithmicrequire}{\textbf{Parameter:}}

\begin{algorithm}[!htb]
\caption{Real-time approximation adaptation}
\label{alg:adapt-RT}

\begin{algorithmic}[1]
\REQUIRE{adaptationStrategy}
\COMMENT{designed to satisfy a certain objective}
\WHILE{true}
    \STATE $\mathrm{inputs}=\mathrm{gatherInputs()}$ \COMMENT{e.g. sensor data; user req.}
    \STATE $\mathrm{approxConfig}=\mathrm{adaptationStrategy(inputs)}$
    \STATE $\mathrm{load(approxConfig)}$
    \STATE $\mathrm{\textbf{with}\ approxConfig\ \textbf{do}\ inference(inputs)}$
\ENDWHILE
\end{algorithmic}
\end{algorithm}


\subsection{Naive}

%One of the key contributions of Mobiprox is the ability to dynamically adapt the approximation configuration during runtime, targeting computation and energy efficiency while at the same time maximizing the inference accuracy. The goal is to select the more ``aggresive'' approximation configurations when the classification difficulty of the input is lower, and switch to more accurate and computationally expensive configurations when the input is more difficult to correctly classify (and more approximated configurations fail). Determining the ``difficulty'' of an input for a neural network (the likelihood that a particular approximate configuration would yield a correct inference result or not for a given input) is not a trivial task, since no ground truth information is available in the case of a deployed deep learning model. Consequently, we explored several strategies for designing an adaptation engine capable of choosing the most suitable approximation configuration for each inference datapoint.

The idea behind all the approximation strategies we present in this section is that inputs that are less difficult to classify can be processed with more ``aggresive'' energy-saving approximation configurations, whereas more difficult-to-classify inputs require computationally more expensive, more accurate configurations. Determining the ``difficulty'' of an input for a NN (i.e. the likelihood that a certain network will not be able to produce a correct inference result for that input) is not a easy task, in particular since no ground truth information is available in the case of a deployed model. 

Our baseline naive adaptation strategy \blue{assumes that the target class does not change throughout time and} uses a simple, heuristic model to predict the correctness of classification $\hat{C} \in \{0,1\}$. After each prediction $\hat{P}_{t}$ at time $t$, the model checks if the network predicted the same class as prediction $\hat{P}_{t-1}$. If the predictions are the same, the model assumes that the current approximation configuration of the network is yielding correct results ($\hat{C}=1$), signaling Mobiprox to approximate more and select a more ``aggressive'' approximation configuration. In the other case, when the current and previous predictions differ, the model assumes the current configuration is giving incorrect predictions and consequently Mobiprox switches to a ``milder'' approximation configuration\footnote{For all adaptation strategies we experiment with two options for moving to more aggressive approximations -- linear and exponential -- while moving to milder approximations is always done in the exponential fashion.}.


%\subsection{Kalman filtering}

%\sasa{The motivation was not clear here for what Kalman filter does}
% Our naive adaptation strategy considers only two consecutive predictions $\hat{P}_{t}$ and $\hat{P}_{t-1}$ to make a binary prediction of the correctness of classification $\hat{C}$ either 0 (wrong) or 1 (correct). This leads to two main drawbacks: high sensitivity to wrong predictions, as a single incorrect prediction by the network would downgrade to a less approximated configuration, and predicting the correctness as discrete values.

% To address these drawbacks, we develop an adaptation strategy that relies on a Kalman filter. Kalman filters~\cite{10.1115/1.3662552, 10.1115/1.3658902} are the statistically optimal sequential estimation procedure for dynamic systems~\cite{LOUKA20082348}. The Kalman filter allows for the recursive estimation of an unknown state $x_{t}$ based on observation values $y$ up to time $t$, by recursively combining observations with recent predictions with weights that minimize the corresponding biases~\cite{LOUKA20082348}. 

% We design the Kalman filter using a high measurement uncertainty value of $1.0$. This makes the Kalman gain small, causing the filter to suppress single events of differences between predictions $\hat{P}_{t}$ and $\hat{P}_{t-1}$. By filtering a sequence
% of discrete values $\hat{C}$, correctness predictions are now values $\hat{C}_{K}\in [0,1]$. We increase approximations when $\hat{C}_{K} > 0.95$ and reduce approximations when $\hat{C}_{K} < 0.5$. Threshold $0.95$ is chosen to counteract floating-point arithmetic errors, and threshold $0.5$ is chosen as the average of valid values of $\hat{C}$.


\subsection{State-driven}

%\sasa{This was an interesting one.}
Many mobile sensing domains deal with the recognition of states that do not vary rapidly over time: human physiological signals do not change erratically, people have conversations, not random utterances, movement is continuous in space, etc. Our state-driven adaptation strategy is based on the observation that rapid variations, especially in human behavior, are rare (e.g.~\cite{jabla2019balancing, reyes2016transition}).

%takes as an inspiration the problem of human activity recognition for which the following is true:

% \begin{itemize}
%     \item Natural human mobility is not characterised by rapid variations~\cite{jabla2019balancing, reyes2016transition}, so a user's physical activity is changing at a much slower rate than the sampling period of the activity classification;
%     \item Certain sequences of activities are intuitively less common or infeasible (e.g. walking downstairs and then lying) while others are very common and to be expected in everyday life (e.g. running followed by walking)
% \end{itemize}

Starting from this assumptions we implement an adaptation algorithm that adjusts the approximation configuration based on the reliability of classification, which in turn is determined by looking at a subset of the most recent predictions made by the network. After each inference, a vote is cast on the measure of reliability $V$, which is increased by $1$ if all previous $N$ predictions are equal, and decreased by $1$ otherwise. The functionality of this approach is described in detail in Algorithm~\ref{alg:adapt-SM}. %\sasa{Would be good to include the algo!}

In this algorithm, $V_{L}$ refers to the number of required votes that need to be cast consecutively in order to change the approximation configuration -- this parameter avoids the situation where the configuration is changed at every inference point. %For the experiments described in this work, we used $V_{L}=2$. 
The second parameter $N$ defines the capacity of the FIFO memory $M$. A larger memory would increase the robustness of the algorithm to classification errors (since it will consider a larger subset of previous predictions), but at the same time would hinder switching to more approximate configurations after a change in the observed/modeled phenomenon.


\begin{algorithm}[!htb]
\caption{State-driven adaptation engine}
\label{alg:adapt-SM}

\begin{algorithmic}[1]
\STATE $M = [ \; ]$ \COMMENT{FIFO memory with maximal capacity $N$}
\STATE $V = 0$  \COMMENT{Reliability index, always on interval $[-V_L, V_L]$}
\WHILE{$p$ = nextPrediction()}
    \STATE $\mathrm{push}(M, p)$
    \IF{$\mathrm{len}(M) < N$}
        \STATE continue
    \ENDIF
    
    \IF{all predictions in $M$ are equal}
        \STATE $V = \mathrm{max}(0, V) + 1$
    \ELSE
        \STATE $V = \mathrm{min}(0, V) - 1$
    \ENDIF
    
    \IF{$V \leq -V_L$}
        \STATE Approximate \textit{less}
    \ELSIF{$V \geq V_L$}
        \STATE Approximate \textit{more}
    \ENDIF
\ENDWHILE
\end{algorithmic}
\end{algorithm}

\subsection{Confidence-driven}

% Finally, we exploit the classifier's confidence as a proxy to accuracy. We are using the Softmax confidence of the network, which in a Deep Learning classifier is represented by the normalized classification scores generated by the network's final layer (the Softmax layer). In the case of a $N$-class classification task, this is represented by an $N$-dimensional vector $z$ with the probability scores for all classes. For any class $i$, its Softmax confidence can be computed as:

% \begin{equation}
%     \sigma_i(z) = \frac{e^{z_i}}{\sum_{j=1}^{N} e^{z_j}}
% \end{equation}


Finally, we exploit the classifier's confidence as a proxy to accuracy. Recent research shows that the softmax layer probability values can accurately reflect the actual confidence of the classifier~\cite{mahmoud2021optimizing}. However, Guo et al.~\cite{guo2017calibration} point out that to achieve a high correlation between the softmax confidence and the expected inference accuracy, calibration is needed. 

Hence, we perform calibration by applying the temperature scaling during softmax confidence calculation. More specifically, for a $N$-class classification task where the $N$-dimensional vector $z$ contains class scores, for any class $i$, its calibrated softmax confidence is computed as:
 
\begin{equation}
    \sigma_i(z; T) = \frac{e^{z_i/T}}{\sum_{j=1}^{N} e^{z_j/T}}
\end{equation}
 
Where $T>0$ is a scalar temperature parameter, which ``softens'' the softmax (raises the output entropy) when $T>1$ and is optimized with respect to negative log likelihood on the validation dataset~\cite{guo2017calibration}. The goal is to tune the value of $T$ such that the confidence value for the datapoints classified with $p$ accuracy is as close a possible to $p$. Ideally, $T$ should be distinctly optimized for every approximation configuration. However, to avoid the computational impact on the tuning process, we apply temperature scaling by tuning a single instance of $T$ to already approximated NNs. 

\newcommand{\cCorrect}{C_{\mathrm{+}}^{(i)}}
\newcommand{\cWrong}{C_{\mathrm{-}}^{(i)}}
\newcommand{\tMore}{T_{\mathrm{more}}}
\newcommand{\tLess}{T_{\mathrm{less}}}
\newcommand{\CMore}{C_{\mathrm{more}}^{(i)}}
\newcommand{\CLess}{C_{\mathrm{less}}^{(i)}}

Our adaptation strategy then uses the calibrated softmax confidence to identify incorrect classifications. Our Android profiler (Section~\ref{sec:hpvm-profiler-android}) also reports per-class confidence averages for correct ($\cCorrect{}$) and incorrect ($\cWrong{}$) predictions and adds this information to approximation configuration files. The algorithm is then driven by a hysteresis outlined by two thresholds $\CLess$ and $\CMore$, where $\cWrong{}>\CLess>\CMore>\cCorrect{}$. If the classification confidence of the predicted class of the immediately preceding instance is higher than $\CMore$, the algorithm moves towards more aggressive approximation. If it is lower than $\CLess$, the algorithm moves towards less approximated configuration. We empirically find that the values of $\CLess$ halfway and $\CMore$ three-quarters-way between $\cWrong{}$ and $\cCorrect{}$, respectively, perform well in our experiments.


% \begin{figure}[!htb]
%     \newcommand{\sw}{0.48\linewidth}
%     \newcommand{\w}{\linewidth}
%     \newcommand{\hs}{\hspace{0cm}}
%     \newcommand{\vs}{\vspace{1cm}}
%     \centering
%     \begin{subfigure}{\sw}
%         \centering
%         \includegraphics[width=\w]{figures/confidence/combined.android-profiled.pareto.confidence.pdf}
%         \caption{Average over all classes}
%         \label{fig:confidences-average}
%     \end{subfigure}
%     \hs
%     \begin{subfigure}{\sw}
%         \centering
%         \includegraphics[width=\w]{figures/confidence/combined.android-profiled.pareto.confidence-class0.pdf}
%         \caption{Walking}
%         \label{fig:confidences-walking}
%     \end{subfigure}
    
%     \vs
    
%     \begin{subfigure}{\sw}
%         \centering
%         \includegraphics[width=\w]{figures/confidence/combined.android-profiled.pareto.confidence-class1.pdf}
%         \caption{Walking upstairs}
%         \label{fig:confidences-upstairs}
%     \end{subfigure}
%     \hs
%     \begin{subfigure}{\sw}
%         \centering
%         \includegraphics[width=\w]{figures/confidence/combined.android-profiled.pareto.confidence-class5.pdf}
%         \caption{Lying}
%         \label{fig:confidences-lying}
%     \end{subfigure}
%     \vs
%     \caption{SoftMax Confidence values of predicted class at various approximation levels (indicated by the QoS loss, defined by equation ~\ref{eq:qosloss}). Figures show confidence values when predictions were correct, wrong and average over both correct and wrong predictions.}
%     \label{fig:confidences}
% \end{figure}


\section{Method}
\label{s:method}

We consider the 3D euclidean space $\Real^3$ with points $p=(x,y,z)\in\Real^3$. We discretize the unit cube $\gC=[0,1]^3$ with a 3D voxel grid $\gG=\set{p_I}$, with nodes $p_I$ indexed by $I=(i,j,k)$, $i,j,k\in [n]=\set{1,\ldots,n}$, \ie, $p_I=(x_{ijk},y_{ijk},z_{ijk})$. We denote by $h=n^{-1}$, and by $N=n^3$ the total number of nodes.   
We represent our reconstructed surface as a zero level of a scalar function $f$ defined over the cube $\gC$. $f$ is defined by prescribing its values at the grid's nodes $f_I\in\Real$ and trilinear interpolating in each voxel. We will denote by $f(p)$ the interpolated value at point $p$. 

Given an input point cloud consisting of $m$ points $q_k\in\Real^3$ with or without (unit norm) normals $n_k\in \Real^3$, $k\in [m]$, our goal is to compute $f$ so that its zero level set approximates the unknown surface, \ie, 
\begin{equation}
    \gS = \set{p\in\gC \ \vert \ f(p)=0}.
\end{equation}
Our approach to compute $f$ is to minimize a loss function of the form
\begin{equation}
    \gL = \gL_{\text{data}} + \gL_{\text{prior}}
\end{equation}
where 
\begin{equation}\label{e:loss_data}
    \gL_{\text{data}} = \frac{\lambda_{\text{p}}}{m}\sum_{k=1}^m \abs{f(q_k)}^2 + \frac{\lambda_{\text{n}}}{m}\sum_{k=1}^m \norm{\nabla f(q_k) - n_k}^2
\end{equation}
where $\norm{\cdot}$ is the standard euclidean norm in $\Real^3$, $\nabla f(p) \in \Real^3$ is the gradient of $f$ sampled at point $p$. Note that $\nabla f$ is defined in interior of voxels, which is generically where the input points $q_k$ resides. $\gL_{\text{data}}$ is the standard data loss encouraging the zero level to pass through the input points $q_k$, and its normals (defined by gradients of $f$) to coincide with input normals $n_k$. 

The prior, $\gL_{\text{prior}}$, is the main contribution of this work, where we combine two novel losses,
\begin{equation}
    \gL_{\text{prior}} = \lambda_{\text{v}} \gL_{\text{viscosity}} + \lambda_{\text{c}} \gL_{\text{coarea}}
\end{equation}
Intuitively, the viscosity loss optimizes for a smooth Signed Distance Function (SDF) solutions, avoiding auxiliary bad minima of the Eikonal equation, while the coarea loss strives to minimize the area of the zero level surface. Our loss has $4$ hyper-parameters $\lambda_{\text{p}},\lambda_{\text{n}},\lambda_{\text{v}},\lambda_{\text{c}}$. We provide more details on these priors next. 


\subsection{Viscosity Loss}\label{ss:viscosity_loss}
The goal of the viscosity loss is to make $f$ approximate an SDF over $\gC$. Given boundary conditions asking $f$ to vanish on some closed compact surface $\gS$, the SDF solves the Eikonal equation PDE, \ie, $\norm{\nabla f(p)}=1$, in a certain well defined sense (viscosity). This motivated some previous work to directly optimize the Eikonal loss \citep{gropp2020implicit,sitzmann2020implicit}
\begin{equation}\label{e:loss_eikonal}
    \gL_{\text{eikonal}} = \int_\gC \Big (\norm{\nabla f(p)}-1\Big )^2 dp
\end{equation}
\begin{wrapfigure}[14]{r}{0.28\textwidth}\vspace{-15pt}
  \begin{center}
    \includegraphics[width=0.25\textwidth]{figs/illustrations/eikonl_1d.png}
  \end{center}
  \caption{Two global minimizers of the Eikonal loss over a grid in 1D. Top solution is not an SDF. }\label{fig:eikonal_1d}
\end{wrapfigure}
Unfortunately, the Eikonal loss has many undesirable minima which are not SDFs. Figure \ref{fig:eikonal_1d} shows a 1D example: both depicted solutions (denoted $f$) vanish at the input points $q_1,q_2$ (black points) and globally minimize the Eikonal loss over the grid (grid points are shown in blue). The INR works mentioned above use neural networks for representing $f$ which injects an inductive bias avoiding these bad minima, however on grids, minimizing \eqref{e:loss_eikonal} cannot avoid these solutions. See, \eg, middle column in Figure \ref{fig:teaser}. 

More classical Eikonal solvers do work with grids however use mostly fast marching or sweeping methods \citep{osher1988fronts,sethian1996fast,zhao2005fast,chacon2012fast}. Namely, use a special discretization of the Eikonal equation favoring the viscosity  solution of the Eikonal \cite{rouy1992viscosity}, and update node values according to a moving front \cite{sethian1996fast}. Since this discretization is up-wind (will only propagate values in one direction) and requires choosing the maximal among its solution, its success in adaptation to a loss is not clear. 

We use a different approach to build a loss favoring SDF solutions over grids motivated by the vanishing viscosity method \cite{crandall1983viscosity}. Namely, adding to the Eikonal PDE a small perturbation of the Laplacian of $f$ (denoted by $\Delta f$), \ie, $\norm{\nabla f(p)}-1 - \eps\Delta f(p)=0$, makes the PDE semi-linear elliptic \citep{calder2018lecture}, and hence with suitable boundary conditions it is uniquely solvable inside $\gS$ with a smooth solution, approaching the viscosity positive distance function to the boundary as $\eps\too 0$. Similarly, for $1-\norm{\nabla f(p)} - \eps \Delta f(p)=0$ the solution approaches the negative distance function inside the domain. 
Motivated by the vanishing viscosity principle we suggest the following viscosity loss:
\begin{equation}\label{e:loss_viscosity_eikonal}
\gL_{\text{viscosity}} = \int_\gC \Big((\norm{\nabla f (p)}-1)\mathrm{sign}(f(p)) - \eps \Delta f(p)\Big)^2 dp
\end{equation}
We discretize this loss over the grid $\gG$ by replacing the first order derivatives and second order derivatives with symmetric finite  differences, \ie,
\begin{align*}
    D_x f_I=D_x f_{i,j,k} = \frac{f_{i+1,j,k}-f_{i-1,j,k}}{2h}, \quad D^2_x f_I = D^2_x f_{i,j,k}=\frac{f_{i+1,j,k}-2f_{i,j,k}+f_{i-1,j,k}}{h^2}
\end{align*}
and similarly for $D_y$ and $D_z$. We use these discrete operators to approximate the gradient $\widehat{\nabla} f(p_I) = (D_x f_I, D_y f_I, D_z f_I)$ and Laplacian $\widehat{\Delta}f(p_I) = D_x^2f_I + D_y^2 f_I + D_z^2 f_I$. The discretized viscosity loss now takes the form
\begin{equation}
    \widehat{\gL}_{\text{viscosity}} = \frac{1}{N}\sum_{I} \Big((\|\widehat{\nabla} f (p_I)\|-1)\mathrm{sign}(f(p_I)) - \eps \widehat{\Delta} f(p_I)\Big)^2
\end{equation}



\subsection{Coarea loss}\label{ss:coarea_loss}
The coarea loss is approximating the area of the zero level set, and therefore incorporating it in the optimization pushes the reconstructed surface to be economic in area. 

First, similarly to  \citep{yariv2021volume} we use the centered Laplace CDF
\begin{equation}
   \Psi\beta(s)= \begin{cases}
   \frac{1}{2}\exp\parr{\frac{s}{\beta}} & s\leq 0 \\ 1-\frac{1}{2}\exp\parr{-\frac{s}{\beta}} & s\geq  0
   \end{cases}
\end{equation} to transform the SDF $f$ to a smooth approximation of the indicator function:
\begin{equation}
    \chi_\beta(p)=\Psi\beta (-f(p))
\end{equation}
As $\beta\too 0$, $\chi_\beta$ converges to an indicator function leading to $1$ inside $\gS$ and $0$ outside. The coarea loss is defined as 
\begin{equation}
    \gL_{\text{coarea}} = \int_\gC \norm{\nabla \chi_\beta (p)} dp
\end{equation}
To understand why this loss approximates the area of $\gS$ we can use the coarea formula \citep{rindler2018calculus}:
\begin{equation}\label{e:coarea}
    \int_\gC \norm{\nabla \chi_\beta(p)}dp = \int_{-\infty}^{\infty} \mathrm{area}(\chi_\beta^{-1}(s))ds,
\end{equation}
where $\chi_\beta^{-1}(s)=\set{p\ \vert \ \chi_\beta(p)=s}$ is the preimage of the value $s$. Since $\chi_x(p)\in [0,1]$ the r.h.s.~integral can be restricted to the interval $[0,1]$, and therefore the coarea loss averages the area of the level sets of $\chi_\beta$. Next,  $$\chi_\beta^{-1}(s)= \set{p\ \vert \ \Psi\beta (-f(p)) = s } = \{p\ \vert \ f(p) = -\Psi\beta^{-1} (s) \} = f^{-1}(-\Psi\beta^{-1} (s)),$$
\begin{wrapfigure}[11]{r}{0.28\textwidth}\vspace{-20pt}
  \begin{center}
  \includegraphics[width=0.25\textwidth]{figs/semi.png}
  \end{center}
  \caption{Reconstruction of a semisphere point cloud (white dots) without (left) and with (right) coarea loss. }\label{fig:coarea_semisphere}
\end{wrapfigure}

which shows that the level set $s\in (0,1)$ of $\chi_\beta$ is the level set $-\Psi\beta^{-1}(s)$ of the SDF $f$. As $\beta\too 0$, $-\Psi\beta^{-1}(s)\too 0$ for all $s\in (0,1)$ (and uniformly in $(\eps,1-\eps)$ for fixed $\eps>0$). Therefore the average of the level set area (\ie, the r.h.s.~of \eqref{e:coarea}) converges to the area of $f^{-1}(0)=\gS$. Figure \ref{fig:teaser} (right) shows how removing the coarea loss introduces an extraneous zero level set, and hence results in an undesired surface part. Figure \ref{fig:coarea_semisphere} shows a comparison of a reconstruction of semisphere with and without coarea. In the experiments section we provide more ablation tests with the coarea and viscosity losses.

To discretize the coarea loss we let $w_I$ denote the centers of grid's voxels, and note that $\nabla \chi_\beta(w_I) = \Phi_\beta(-f(w_I))\nabla f(w_I)$, where 
\begin{equation*}
    \Phi_\beta(s) = \frac{1}{2\beta}\exp\parr{\frac{\abs{s}}{\beta}}
\end{equation*}
is the PDF of the Laplace distribution, and $\nabla f(w_I)$ is computed as a linear combination of the voxel's corner values $f_{I_1},\ldots,f_{I_8}$, see more details in the Appendix. We end up with the discretized loss:
\begin{equation}
    \widehat{\gL}_{\text{coarea}} = \frac{1}{N}\sum_{I}\Phi_\beta(-f(w_I))\norm{\nabla f(w_I)}
\end{equation}
This loss is usually incorporated with a small hyper-parameter $\lambda_{\text{c}}$ with the purpose of eliminating redundant surface parts.



 \section{Benchmarks and Evaluation}
\label{sec:eval}

We evaluate \krakenSpace to answer the following set of questions:
\begin{itemize}
\item How much improvement does partial evaluation and our implemented compiler optimizations give \kraken? %(\S \ref{sec:eval2})
\item How much faster is our purely functional f-expr language, \krakenSpace, compared to other implementations of fexprs? %(\S \ref{sec:eval1} - \ref{sec:eval2})
\item How does \kraken's performance, with its fexprs, compare to macros? %(\S \ref{sec:eval1}, \S \ref{sec:eval3})
\item How do the different partial evaluation mechanisms/optimizations in \krakenSpace contribute towards reduction in overall runtime?
%\item What does \krakenSpace do internally when we create a data structure and evaluate it for some function? (\S \ref{sec:casestudy})
\end{itemize}

\textbf{Experimental Setup}: 
We ran these experiments in a reproducible Nix environment on a NixOS install \cite{10.1145/1411203.1411255} (Kernel 6.0.0) on a laptop with 8 cores / 16 threads and 64 GB of RAM.
Our code contains the scripts and Nix Flakes needed to reproduce the exact set of dependencies to run our tests.
%The code can be found at \url{https://github.com/limvot/kraken}.

The Kraken benchmarks were run using both the Wasmtime and WAVM WebAssembly engines for most benchmarks.
The Wasmtime WebAssembly engine is one of the most popular, developed by the Bytecode Alliance itself, and uses the CraneLift code generation backend.
The WAVM WebAssembly engine is interesting for its use of LLVM, and it often produces the fastest code on benchmarks but has a higher startup time.
We eliminated the Cfold Wasmtime benchmark due to problems running out of stack space (a known property of the Cfold benchmark).

\textbf{Benchmarks}: 
To showcase the capability of Kraken, we created benchmarks that are commonly implemented in functional languages and have been used as benchmarks in other papers \cite{reinking2021perceus, 10.1145/3547646}.
The benchmarks are
\begin{itemize}
\item Fib - Calculating the nth Fibonacci number
\item RB-Tree - Inserting n items into a red-black tree, then traversing the tree to sum its values
\item Deriv - Computing a symbolic derivative of a large expression
\item Cfold - Constant-folding a large expression
\item NQueens - Placing n number of queens on the board such that no two queens are diagonal, vertical, or horizontal from each other
\end{itemize}
All benchmarks besides Fibonacci use the fexpr version of match for pattern matching in \kraken, which is equivalent to the macro version in NewLisp. We also RB-Tree using NewLisp's~\cite{mueller2018newlisp} version of fexpr match. We modified the sizes of the problems presented to the benchmark to account for the longer running times of some of the less-optimized implementations.
The code for Kraken and NewLisp is very similar, and we should note that it is very unidiomatic NewLisp.
Our goal was not to compare Kraken and NewLisp as implementation languages for Red-Black Trees, but to stress test a single reasonably complex fexpr/macro, namely pattern matching.
% \textbf{Comparison with other languages}: We evaluated \krakenSpace against a language that contains f-exprs, as well as against itself with various optimizations disabled. The only other language we could find which contains a real f-expr mechanism is NewLisp~\cite{mueller2018newlisp} and so we ported \kraken's benchmark implementation to NewLisp.

%The six state-of-the-art languages are Java 17.0.1, Swift 5.4.2, Koka 2.3.2, C++, Haskell 8.10.7, and OCaml 4.12.
%The language choices were taken directly from Perceus reference-counting paper \cite{reinking2021perceus}.
%The Fibonacci benchmark additionally tests Python 3.9.11 and Chez Scheme 9.5.4.
%Koka, Ocaml and Haskell are good comparison points as statically-typed, compiled, functional programming languages, while Chez Scheme is a good comparison point as a mature and industrial strength dynamically-typed Scheme implementation known for its performance. 
%\subsection{Basic Level Comparison}
\subsection{The Effect of Partial Evaluation on Eval Calls}

\begin{table}[h]
\caption{Number of eval calls with no partial evaluation for Fexprs}
	\begin{tabular}{||c | c c c c c ||} 
		\hline
		&Evals & Eval w1 Calls & Eval w0 Calls & Comp Dyn & Comp Dyn\\ 
        & & & & w1 Calls & w0 Calls\\ [0.5ex] 
		\hline\hline
		Cfold 5 & 10897376 & 2784275 & 879066  & 1 & 0 \\ 
		\hline
		  Deriv 2  & 11708558 & 2990090 & 946500 & 1 & 0 \\ 
        \hline
		  NQueens 7 & 13530241 & 3429161 & 1108393 & 1 & 0 \\ 
    \hline
		  Fib 30 & 119107888 & 30450112 & 10770217 & 1 & 0 \\ 
    \hline
		  RB-Tree 10 & 5032297 & 1291489 & 398104 & 1 & 0 \\ 
		\hline
	\end{tabular}
    \label{npe:calls}
 \end{table}

As mentioned before, using fexprs without partial evaluation will prelude optimization and cause a massive amount of repeated work. Table \ref{npe:calls} and Table \ref{pe:calls} show the number of calls to the \krakenSpace runtime's eval function, the number of times the runtime's eval function executed a call to an applicative with wrap\_level=1, the number of times the runtime's eval function executed a call to an operative with wrap\_level=0, the number of compiled dynamic calls to applicatives with wrap\_level=1, and the number of compiled dynamic calls to operatives with wrap\_level=0.
These are shown for \krakenSpace test cases with partial evaluation turned off and turned on. 
\begin{table}[h]
\caption{Number of eval calls in Partially Evaluated Fexprs}
	\begin{tabular}{||c | c c c c c ||} 
		\hline
		&Evals & Eval w1 Calls & Eval w0 Calls & Comp Dyn & Comp Dyn\\ 
        & & & & w1 Calls & w0 Calls\\ [0.5ex] 
		\hline\hline
		Cfold 5 & 0 & 0 & 0  & 0 & 0 \\ 
		\hline
		  Deriv 2  & 0 & 0 & 0 & 2 & 0 \\ 
        \hline
		  NQueens 7 & 0 & 0 & 0 & 0 & 0 \\ 
    \hline
		  Fib 30 & 0 & 0 & 0 & 0 & 0 \\ 
    \hline
		  RB-Tree 10 & 0 & 0 & 0 & 10 & 0 \\ 
		\hline
	\end{tabular}
    \label{pe:calls}
 \end{table}

\begin{table}[h]
\caption{Number of calls to the runtime's eval function for RB-Tree. The table shows the non-partial evaluation numbers -> partial evaluation numbers.}
	\begin{tabular}{||c | c c c c c ||} 
		\hline
		&Evals & Eval w1 Calls & Eval w0 Calls & Comp Dyn & Comp Dyn\\ 
        & & & & w1 Calls & w0 Calls\\ [0.5ex] 
		\hline\hline
		  RB-Tree 7 & 2952848 -> 0 & 757932 -> 0 & 233513 -> 0 & 1 -> 7 & 0 -> 0\\ 
        \hline
		  RB-Tree 8 & 3532131 -> 0 & 906548 -> 0 & 279379 -> 0 & 1 -> 8 & 0 -> 0\\ 
        \hline
		  RB-Tree 9 & 4278001 -> 0 & 1097965 -> 0 & 3383831 -> 0 & 1 -> 9 & 0 -> 0\\ 
		\hline
	\end{tabular}
    \label{pe:rb}
    \vspace{-4mm}
 \end{table}

Without partial evaluation, no compilation can be done because it is impossible to tell if arguments to calls will be evaluated. In all benchmarks, partial evaluation removed all calls to the runtime's eval function, resulting in a completely compiled program. Looking at RB-Tree, there are over a million calls to combiners with wrap level 1 (normal functions), and 398,000 calls to combiners with wrap level 0 (operatives replacing macros). This massive blowup in the number of calls is due to the repeated and exponential re-execution of macro-like-combiners in the definition of other macro-like-combiners, as discussed in the Introduction.

The non-partially-evaluated benchmarks show 1 compiled dynamic call to an applicative (its the first call into eval) and 0 compiled dynamic calls to operatives, because there is no compilation at all. For the partially evaluated benchmarks, there are a few compiled dynamic calls to applicatives due to higher-order function use in the benchmarks, and there are no compiled dynamic calls to operatives, as all operative use has been eliminated.
We also varied the inputs for RB-Tree shown in Table \ref{pe:rb} to give a sense for how the number scale with respect to input size.

The incredible slowdown implied by these tables comes to full fruition in our RB-Tree test in Fig.~\ref{fig:kraken_nqueens_rbtree}.
We kept this run shorter because Kraken's non-partial-evaluating interpreter takes an incredibly long time even for 100 insertions (40 minutes).
The compounding layers of repeated macro-like operative calls in the non-partially-evaluated Kraken version cause a ~70,000x slowdown relative to the partial evaluated, optimized, and compiled version.
For the remaining benchmarks, we remove the naive interpreted \krakenSpace version, as in each case its performance is so bad as to blow out the graph and make it impossible to do any comparison.
In our optimized Kraken, our partial evaluation algorithm is able to fully collapse these levels of inefficiency, evaluate and inline the results, and give the backend more specialized code to optimize, emitting a compiled version that handily beats not only the NewLisp-fexpr implementation but even the NewLisp-macro implementation, as can be seen in Fig.~\ref{fig:kraken_vs_world_fib}.
We kept the benchmark sizes small in this test because the stack limits of NewLisp prevent sizes larger then ~880, while the Tail Call Elimination performed by the \krakenSpace compiler allows us to run much larger benchmarks, including the run of 4,800,000 inserts to the RB-Tree.
This result shows the dramatic effect of partial evaluation and compiler optimizations on runtime for \kraken. Our technique takes the performance of a fully fexpr based language from being completely infeasible to being faster than a macro-based dynamic scripting language currently in use.
% \begin{center}
% \begin{table}[ht]
% \caption{Number of call to the runtime's eval function for Fib. The table shows the non-partial evaluation numbers -> partial evaluation numbers}
% 	\begin{tabular}{||c | c c c c c ||} 
% 		\hline
% 		&Evals & Eval w1 Calls & Eval w0 Calls & Comp Dyn w1 Calls & Comp Dyn w0 Calls\\ [0.5ex] 
% 		\hline\hline
% 		Fib 10 & 8468 -> 0 & 2167 -> 0  & 777 -> 0 & 1 -> 0 & 0 -> 0 \\ 
% 		\hline
% 		  Fib 15  & 87916 -> 0 & 22478 -> 0 & 7961 -> 0 & 1 -> 0 & 0 -> 0 \\ 
%         \hline
% 		  Fib 20 & 969010 -> 0 & 247731 -> 0 & 87633 -> 0 & 1 -> 0 & 0 -> 0 \\ 
%     \hline
% 		  Fib 25 & 10740492 -> 0 & 2745825 -> 0  & 971209 -> 0 & 1 -> 0 & 0 -> 0 \\ 
% 		\hline
% 	\end{tabular}
%     \label{pe:fib}
%  \end{table}
% \end{center}

\begin{figure}[h]
\caption{Constant Fold and Deriv}
\includegraphics[width=0.45\textwidth]{cfold_table.csv_}
\includegraphics[width=0.45\textwidth]{deriv_table.csv_}
\label{fig:kraken_const_deriv}
\vspace{-6mm}
\end{figure}
\subsection{Comparison between Kraken Versions}
Beyond the massive speedup from partial-evaluation, Fig. \ref{fig:kraken_const_deriv} and \ref{fig:kraken_nqueens_rbtree} show the effect of the various compiler optimizations we described by disabling them one by one.
 Our main four optimizations have a strong positive effect on runtime, with the exception of lazy environment instantiation. Lazy environment instantiation helps massively on fib, and some on Deriv, but generally hurts the rest slightly.


\begin{figure}[h]
\caption{N-Queens}
\includegraphics[width=0.45\textwidth]{nqueens_table.csv_}
\includegraphics[width=0.45\textwidth]{slow_rbtree_table.csv_}
\label{fig:kraken_nqueens_rbtree}
\vspace{-4mm}
\end{figure}


\subsection{Comparison against Others}


To give a general idea of our current performance, we also show a Fibonacci benchmark that mostly exercises pure function-call speed and inlining as seen in Fig. ~\ref{fig:kraken_vs_world_fib}.
We include Python and Chez Scheme to give a general idea for where an exemplar slow and an exemplar fast dynamic language would fall.
With the benefit of our partial evaluation, compilation, and leaning upon mature WebAssembly implementations, we beat both, but this should be taken with a grain of salt, as this is a very limited micro-benchmark only meant to give a general sense of the order of magnitude of our performance.



\label{sec:eval1}
\begin{figure}[h]
\caption{Kraken vs. Others. Ordered by fastest to slowest}
\includegraphics[width=0.45\textwidth]{fib_table.csv_}
\includegraphics[width=0.45\textwidth]{rbtree_table.csv_}
\label{fig:kraken_vs_world_fib}
\end{figure}

%\label{sec:eval_nqueens}
%\begin{figure}[h]
%\caption{N-Queens}
%\includegraphics[width=0.45\textwidth]{nqueens_table.csv_}
%\includegraphics[width=0.45\textwidth]{slow_nqueens_table.csv_}
%\label{fig:kraken_nqueens}
%\end{figure}

%\label{sec:eval_nqueens}
%\begin{figure}[h]
%\caption{Kraken, N-Queens, absolute value and log-scale}
%\includegraphics[width=0.45\textwidth]{nqueens_table.csv_}
%\includegraphics[width=0.45\textwidth]{nqueens_table.csv_log}
%\label{fig:kraken_nqueens}
%\end{figure}
%\label{sec:eval_nqueensp}
%\begin{figure}[h]
%\caption{Kraken, N-Queens, absolute value and log-scale}
%\includegraphics[width=0.45\textwidth]{slow_nqueens_table.csv_}
%\includegraphics[width=0.45\textwidth]{slow_nqueens_table.csv_log}
%\label{fig:kraken_nqueensp}
%\end{figure}

%\label{sec:eval_cfold}
%\begin{figure}[h]
%\caption{C-Fold}
%\includegraphics[width=0.45\textwidth]{cfold_table.csv_}
%\includegraphics[width=0.45\textwidth]{slow_cfold_table.csv_}
%\label{fig:kraken_cfold}
%\end{figure}
%\label{sec:eval_cfold}
%\begin{figure}[h]
%\caption{Kraken, C-Fold, absolute value and log-scale}
%\includegraphics[width=0.45\textwidth]{cfold_table.csv_}
%\includegraphics[width=0.45\textwidth]{cfold_table.csv_log}
%\label{fig:kraken_cfold}
%\end{figure}
%\label{sec:eval_cfoldp}
%\begin{figure}[h]
%\caption{Kraken, C-Fold, absolute value and log-scale}
%\includegraphics[width=0.45\textwidth]{slow_cfold_table.csv_}
%\includegraphics[width=0.45\textwidth]{slow_cfold_table.csv_log}
%\label{fig:kraken_cfoldp}
%\end{figure}

%\label{sec:eval_deriv}
%\begin{figure}[h]
%\caption{Deriv}
%\includegraphics[width=0.45\textwidth]{deriv_table.csv_}
%\includegraphics[width=0.45\textwidth]{slow_deriv_table.csv_}
%\label{fig:kraken_deriv}
%\end{figure}
%\label{sec:eval_deriv}
%\begin{figure}[h]
%\caption{Kraken, Deriv, absolute value and log-scale}
%\includegraphics[width=0.45\textwidth]{deriv_table.csv_}
%\includegraphics[width=0.45\textwidth]{deriv_table.csv_log}
%\label{fig:kraken_deriv}
%\end{figure}
%\label{sec:eval_derivp}
%\begin{figure}[h]
%\caption{Kraken, Deriv, absolute value and log-scale}
%\includegraphics[width=0.45\textwidth]{slow_deriv_table.csv_}
%\includegraphics[width=0.45\textwidth]{slow_deriv_table.csv_log}
%\label{fig:kraken_derivp}
%\end{figure}

%\subsection{Comparison against state-of-the-art languages}
%\label{sec:eval3}

%\begin{figure}[h]
%\caption{Kraken vs. S.o.t.A.}
%\includegraphics[width=0.45\textwidth]{cfold_table.csv_}
%\includegraphics[width=0.45\textwidth]{rbtree_table.csv_}
%\label{fig:kraken_vs_world1}
%\end{figure}

%\begin{figure}[h]
%\caption{Kraken vs. S.o.t.A.}
%\includegraphics[width=0.45\textwidth]{deriv_table.csv_}
%\includegraphics[width=0.45\textwidth]{nqueens_table.csv_}
%\label{fig:kraken_vs_world2}
%\end{figure}

% \begin{figure}[h]
% \caption{Kraken vs. S.o.t.A. (Log)}
% \includegraphics[width=0.45\textwidth]{cfold_table.csv_log}
% \includegraphics[width=0.45\textwidth]{rbtree_table.csv_log}
% \label{fig:kraken_vs_world_log_1}
% \end{figure}
% \begin{figure}[h]
% \caption{Kraken vs. S.o.t.A. (Log)}
% \includegraphics[width=0.45\textwidth]{deriv_table.csv_log}
% \includegraphics[width=0.45\textwidth]{nqueens_table.csv_log}
% \label{fig:kraken_vs_world_log_2}
% \end{figure}

%As we noted before with the Fib(30) microbenchmark in Section \ref{sec:eval1}, we remain significantly slower than state-of-the-art compiled languages.
%This is particularly true for memory-intensive benchmarks due to our naive reference-counting and malloc/free implementations.
%However, our results are of a similar order of magnitude to the difference between the state-of-the-art compiled languages and dynamic scripting languages, like Python's results in the Fib(30) microbenchmark.
%We assert that is not a fundamental limitation because the classic f-expr slowness is being eliminated, as shown by Fig. \ref{fig:kraken_vs_newlisp1} and Fig. \ref{fig:kraken_vs_newlisp2}.
%In future work, we plan to expand our compile-time analysis and optimization to implement a modified, dynamic-language version of Perceus reference counting.
%With this change, we belive \krakenSpace can be competitive with these state-of-the-art languages.

%\subsection{Case Study: Red-Black Tree}
%\label{sec:casestudy}

%\begin{figure}[h]
%\caption{Kraken vs. S.o.t.A. - RB-Tree Focus}
%\includegraphics[width=0.4\textwidth]{rbtree_table.csv_}
%\includegraphics[width=0.4\textwidth]{rbtree_table.csv_log}
%\label{fig:kraken_vs_world_rbtree}
%\end{figure}


%To evaluate our partial evaluation algorithm and compiler, we extracted the benchmarks used by the Koka language project from their code repository and added Kraken versions, as well as implementing a naive Fibonacci microbenchmark ourselves to evaluate pure function call speed.\\
%With partial evaluation and the compiler optimizations listed above, we get fairly strong performance on purely numerical computations, such as the naive Fibonacci microbenchmark.
%Unfortunately, the overhead of our unsophisticated reference counting, dynamic type checking, and bounds checking causes poor performance on benchmarks involving data structures relative to mainstream programming language implementations.
%This is not a fundamental limitation, and will be addressed in future work, as recounted in the next section.
%It should be noted, however, that while the performance relative to established language implementations is very poor for the memory-intensive benchmarks (600-900x slower), we still realize a massive speedup compared to an unoptimized and non-partial-evaluated f-expr implementation (100,000x faster)!


We provide some comments on the growth conditions which constituted the majority of our analysis in sections \ref{sec:Hmixing} and \ref{sec:Hsigma}. In the simplest cases of Lemma \ref{lemma:unstableGrowth}, growth was established in an analogous fashion to the old one-step expansion condition (\ref{eq:oldOneStepExpansion}), finding the relevant Jacobians $M_j$ and checking that their expansion factors $K(M_j)$ satisfy
\begin{equation}
    \label{eq:discussionOneStep}
    \sum_j \frac{1}{K(M_j)} <1.
\end{equation}
For the more complicated cases, the inductive method used to establish growth near the accumulation points in Lemma \ref{lemma:unstableGrowth} and the weakened one-step expansion condition (\ref{eq:oneStep}) both address the same fundamental issue: the splitting of unstable curves by singularities into an unbounded number of small components. They circumvent this obstacle in rather different ways, however. While (\ref{eq:oneStep}) generalises (\ref{eq:discussionOneStep}) to ensure an growth of unstable curves `on average' (see \cite{chernov_statistical_2009} for a precise statement), our inductive method is a more direct adaptation of (\ref{eq:discussionOneStep}), using it to generate contradictory geometric conditions which a hypothetical non-growing unstable curve must satisfy. It may be possible to prove Theorem \ref{sec:Hmixing} using (\ref{eq:oneStep}) as the basis for growth. Since we required (\ref{eq:oneStep}) anyway for proving Theorem \ref{thm:HsigmaExp}, this could potentially condense our analysis, but only to a minor extent. A convenience of the method used in section \ref{sec:Hmixing} is that, by way of the `simple intersection' property, it naturally gives geometric information on the images of manifolds, useful for proving the property \textbf{(M)} of Theorem \ref{thm:katok-strelcyn}.

We expect that essentially analogous analysis can be applied to establish mixing properties in a wide class of piecewise linear non-uniformly hyperbolic maps, including those (like the OTM) which sit on the boundary of ergodicity and beyond. While we have relied on the precise partition structure of $H_\sigma$, its fundamental feature (self-similar sequences of elements $A^k$, sharing boundaries with its neighbours $A^{k-1},A^{k+1}$ and accumulating onto some point $p$) is quite typical to return map systems. See, for example, those of various stadium billiards \cite{chernov_chaotic_2006,chernov_improved_2008,chernov_statistical_2009} and LTMs \cite{springham_polynomial_2014}. Indeed, the same method can be used to prove the Bernoulli property for non-monotonic LTMs \cite{myers_hill_mixing_2022}, where monotonicity of the manifold images cannot be assumed and the classical argument \cite{sturman_mathematical_2006} fails. The OTM is the pointwise limit of these maps as the boundary shrinks to null measure. It further has utility in proving growth conditions for maps which are uniformly hyperbolic but possess regions $A_j$ where the hyperbolicity is very weak, signified by $K(M_j) \approx 1$, so that (\ref{eq:discussionOneStep}) fails. Typically this leads to suboptimal bounds on mixing windows, see e.g. \cite{wojtkowski_model_1981,przytycki_ergodicity_1983,myers_hill_family_2022}. The map $H_{(\eta,\eta)}$ for $\eta \approx 1/2$ is another example, possessing weak hyperbolicity over $A_2, A_3$. Letting $\varepsilon = |\eta-1/2|>0$, there is an upper bound $N = N(\varepsilon)$ on escape times from the intersections $A_2\cap \sigma, A_3 \cap \sigma$. The growth lemma then follows by applying the inductive step roughly $N$ times and can be established for arbitrarily small $\varepsilon$, opening the door to establishing optimal mixing windows.

The above gives two examples of piecewise linear perturbations to $H$ where mixing with respect to Lebesgue is preserved and our methods can be applied. Nonlinear perturbations to the shear profiles complicate the analysis in several ways. Firstly as the map's Jacobians takes on a broader range of values, cone invariance becomes an increasingly harder condition to establish. Cones must be widened, giving looser bounds on expansion factors, which may already be weak due to new regions of weaker stretching. This, together with the change from polygonal to curvilinear return time partition elements and nonlinear local manifolds, adds some complexity to showing growth conditions. This does not rule out certain (small) nonlinear perturbations however. There is some leeway in the inequalities which govern cone invariance and growth of local manifolds, the latter of which is not too dissimilar from the piecewise linear setting (see Lemmas \ref{lemma:piecewiseApprox}, \ref{lemma:componentLength}). Certain small perturbations would not alter the \emph{topological} structure of the return time partition, i.e. which elements share boundaries, the key information needed for setting up the induction. Finally while the partition elements would no longer be polygonal, only coarse geometric information is required for verifying each inductive step. Following the above, a potential perturbation could be to replace the linear portions of each shear by a cubic, perturbing the tent profile
\[  f(t) = \begin{cases} 2t & 0 \leq t \leq 1/2, \\ 2(1-t) & 1/2 \leq t \leq 1 ,\end{cases} \]
of the OTM shears to
\[  f_a(t) = \begin{cases} \frac{1}{8} t \left(16 - a + 6at - 8at^{2} \right) & 0 \leq t \leq 1/2, \\ \frac{1}{8}\left(1-t\right)\left( 16 - a + 6a\left(1-t\right) - 8a\left(1-t\right)^{2}\right)  & 1/2 \leq t \leq 1, \end{cases}   \]
for $a>0$. For small enough $a$ the gradient range $f'(t)$ is restricted to small neighbourhoods of $\{ 2, -2\}$ and the escape time partition retains a similar structure. We illustrate this in Figure \ref{fig:perturbations}, showing escapes from the square $S_3$ under the map $G \circ F$, equivalent to escapes from the perturbed $A_3$ under the $G \circ F$, but with a cleaner geometry for comparison. When $a$ is too large the analogy to the OTM breaks down. At $a=16$ the map is twice differentiable everywhere and features a new source of slowed mixing, the Jacobian is the identity at the corner points $x,y \in \{  0, 1/2 \}$ giving locally parabolic behaviour (visible in the escape time partition). 

\begin{figure}
    \centering
    \includegraphics[width=0.24 \linewidth]{0.png}
    \includegraphics[width=0.24 \linewidth]{4.png}
    \includegraphics[width=0.24 \linewidth]{8.png}
    \includegraphics[width=0.24 \linewidth]{16.png}
    \caption{Partition of escape times from $S_3$ under the mapping $F \circ G$ for $a= 0,4,8,16$. }
    \label{fig:perturbations}
\end{figure}

\section{Conclusion}\label{sec:conclusion}
In this work, we focus on addressing the fundamental challenge of OOD detection tasks, which is how to fully understand the semantic discrepancy between the ID/OOD samples. We reveal that the key to success in the realistic SCOOD task is to allocate as many ID samples in the unlabeled set correctly as possible. To this end, we propose a novel uncertainty-aware optimal transport scheme that introduces class-specific energy scores as guidance for effective label assignment. Experimental results show that our method achieves better performance than previous state-of-the-art methods on SCOOD benchmarks.

\textbf{Limitations.} In addition to temperature scaling, other techniques such as feature clipping applied in ReAct~\cite{sun2021react} also enhance the performance of energy score, so how to obtain an OOD score that best fits the SCOOD task can be further explored. Moreover, a setting highly related to SCOOD has been proposed in \cite{katz2022training} and formulated as a constrained optimization problem. We will also theoretically analyze these practical OOD settings in our feature work.

% \section*{Acknowledgments}
\textbf{Acknowledgments.} 
This work is supported by National Key R\&D Program of China under Grant 2020AAA0105701, National Natural Science Foundation of China (NSFC) under Grants 61872327, Major Special Science and Technology Project of Anhui, National Natural Science Foundation of China (62033012) and Ant Group through Ant Research Intern Program.



\bibliographystyle{ACM-Reference-Format}
\bibliography{sample-base}

\end{document}
\endinput
%%
%% End of file `sample-acmlarge.tex'.
