\section{Radial mask optimization}
In this section, we present the radial mask optimization process. We begin in Section~\ref{subsec:RadialMaskParameterization} presenting a parameterization of the radial coded mask, in order to constrain the radial shape of the coded mask throughout the optimization process. Then section~\ref{subsec:LossFunction} presents our objective function and optimization scheme for the search for the best parameters of the radial mask. Finally, section~\ref{subsec:OptimizationResults} presents the results of the optimization process and compares the optimized mask to the original radial mask proposed in the literature. 

\begin{figure}[!t]
    \centering
    \includegraphics[width=\linewidth]{figs/MaskOptimization/RadialMaskParameterizationV2.pdf}
    \caption{Overview of the radial coded mask parameterization process.
    Left: the usual modeling of a coded mask, where the light transmittance on an arbitrary discretized position in the Cartesian coordinates $(x,y)$ on the mask plane is represented as $m(x,y)$.
    Right: our proposed radial-mask parameterization, where each element of a vector $\Theta$ is related to the light transmittance~$m(\theta)$ of one of the radial sections of a coded mask. 
    The radial sections are aligned along the angular axis $\theta$.
    The logistic sigmoid function~$\sigma(\cdot)$ is multiplied on the original vector elements in order to ensure that all light transmittance values are in the range $[0,1]$.}
    \label{fig:MaskParameterization}
\end{figure}

% Explain how to perform shape-aware optimization for radial coded masks
\subsection{Radial Mask Parameterization}
\label{subsec:RadialMaskParameterization}
In past studies related to deep optics, the mask optimization process has been done independently for each element $m(x,y)$, which represents the light transmittance at the discretized Cartesian coordinates $(x,y)$ on the mask-plane.
However, this cannot be done for this work, as it is unlikely for the radial characteristic of the mask to be preserved throughout the optimization process.
In order to address this issue, we parameterize the coded mask by splitting its area into $N_\mathrm{t}$ radial sections $\boldsymbol{\Theta}=[\theta_0, \theta_1, \cdots, \theta_{N_\mathrm{t}-1}]$ along angular coordinate, where each section originates in the center of the mask and expands to its edges.
We then enforce every mask element inside the same radial section to have the same light transmittance value, as follows:
\begin{equation}
    m(x,y) = m(\theta_k)  \  \forall \ (x,y)  \ \in \ \theta_k,
\end{equation}
where $k$ is the index of the radial section, and $(x,y) \in \theta_k$ indicates a set of spatial mask coordinates that belong to the $k$-th radial section. In this scenario, the value of the light transmittance for the $k$-th radial section is:
\begin{equation}
    m(\theta_k) = \sigma( m'(\theta_k)),
\end{equation}
where $\sigma(\cdot)$ represents a logistic sigmoid function, and $m'(\theta_k)$ is a single scalar value related to the light transmittance in the $k$-th radial section. Here, the sigmoid function is employed to ensure that the values of light transmittance of the mask were in the interval $[0,1]$. For simplicity of representation, we will not explicitly write the $\sigma(\cdot)$ for the sigmoid function from this point onwards.

Here we denote a dimensionally-reduced radial-mask vector $\boldsymbol{m}_\mathrm{\theta} \in \mathbb{R}^{N_\mathrm{t} \times 1}$ that represents all the light-transmittance parameters in a radial mask along angular axis $[m(\theta_0), m(\theta_1), \cdots, m(\theta_{N_\mathrm{t}-1})]$,
and $p(\cdot)$ represents the mapping of the light transmittance elements of the radial mask in angular coordinates to a coded mask in Cartesian coordinates.
Assuming that the PSF~$\textbf{h}$ can be approximated as a mask vector itself, in which a diffraction effect is ignored, the PSF vector in Eq.~\eqref{eq:conv} can be related to the dimensionally-reduced radial-mask vector as follows:
\begin{equation}
    \textbf{h} = p(\boldsymbol{m}_{\mathrm{\theta}}).
\end{equation}

%Using a polar-coordinate system, we represent the mask parameters to be optimized as a 1D vector~$\Theta$, whose components represent the angular coordinates or sections of the coded mask, as follows:
%\begin{equation}
%    \Theta = [\theta_0, \theta_1, \cdots, \theta_{N-1}],
%\end{equation}
%where $N \in \mathbb{N}$ is the total count of radial sections in the mask.}

On the left side of Fig.~\ref{fig:MaskParameterization} we demonstrate a common coded mask modeling approach, where the light transmittance on a position $(x,y)$ on the plane of the mask is denoted as $m(x,y)$. On the right side of Fig.~\ref{fig:MaskParameterization}, we show our proposed radial mask parameterization process. The number of radial sections~$N_\mathrm{t}$ is a hyperparameter, and it was determined manually before the optimization experiments. We empirically selected a value of $70$ for this parameter. 
This was decided based on the resolution used for the coded mask on the optimization experiments, which was $140 \times 140$ pixels.
%\red{A higher number of radial structures could render some of them to have many discontinuous areas, which is not desirable.}
% We, therefore, can represent the radial mask as a 1 dimensional vector, where each element is related to the light transmittance of one of the radial sections of the coded mask. 

% Define loss function to be optimized (MTF loss)
\subsection{Loss Function and Optimization Process}
\label{subsec:LossFunction}
For optimization, we aim at improving the frequency response, i.e. MTF, of the lensless optical system. For this paper, we desire to increase MTF values of a coded mask across all frequencies.
The MTF of a PSF~$\textbf{h}$ is defined as follows~\cite{Goodman1996}:
\begin{equation}
    \mathrm{MTF}(\textbf{h}) = \mathrm{normalize}\left(|\mathcal{F}[\textbf{h}]|\right),
\end{equation}
where $\mathrm{normalize}(\cdot)$ is the normalization operator with a value at zero frequency, $\mathcal{F[\cdot]}$ corresponds to a 2D discrete Fourier transform, and $|\cdot|$ computes the absolute values.
Our proposed loss function for mask-parameter optimization is as follows:
\begin{align}
    \mathcal{L}(\boldsymbol{m}_\mathrm{\theta}) 
    &= (-1)\times \mathrm{mean}(\mathrm{MTF}(\textbf{h}))\\
    &= (-1)\times \mathrm{mean}(\mathrm{MTF}(p(\boldsymbol{m}_\mathrm{\theta}))),
\end{align}
where $\mathrm{mean(\cdot)}$ represents the average over all elements.
% As we want to increase the MTF values over frequencies, we multiply by $-1$ in order to turn it into a minimization problem.
Note that as we aim to increase the overall MTF of the optimized coded mask, we multiply the averaged MTF by $-1$ in order to use this loss in a minimization problem. The problem to be solved for obtaining the MTF-targeted optimized design of the radial mask can be simply written as:
\begin{equation}
\hat{\boldsymbol{m}}_\mathrm{\theta}=\argmin_{\boldsymbol{m}_\mathrm{\theta}}\mathcal{L}(\boldsymbol{m}_\mathrm{\theta}),
\label{eq:minimize}
\end{equation}
where the $\hat{\boldsymbol{m}}_\mathrm{\theta} \in \mathbb{R}^{N_\mathrm{t}\times 1}$ is the vector representing the optimized radial coded mask having dimensionally-reduced optimized parameters.
By solving Eq.~\eqref{eq:minimize} and applying the mapping function~$p$ after optimization, the radial-shape-preserved MTF-targeted optimized coded mask with spatial light-transmittance parameters can be obtained.

\begin{figure}[!t]
    \centering
    \includegraphics[width=\linewidth]{figs/MaskOptimization/RadialMasksCompare.pdf}
    \caption{Radial masks used for comparison.
    The baseline radial masks with (a)~$20$, (b)~$40$, and (c)~$60$ radial sections, respectively.
    (d)~Our optimized radial mask.}
    \label{fig:RadialMaskCompare}
\end{figure}

\begin{figure}[!t]
    \centering
    \includegraphics[width=\linewidth]{figs/MaskOptimization/MtfComparison_zerofreq.pdf}
    \caption{Comparison of MTFs of the baseline and optimized radial masks.
    $\omega$ indicates the Nyquist frequency of a mask.}
    \label{fig:MtfComparison}
\end{figure}

% Present results of the radial mask optimization, comparing MTF to the hand-crafted radial masks from previous work.
\subsection{Optimization Experiment}
\label{subsec:OptimizationResults}
For the optimization, we used the Adam optimizer~\cite{Kingma_Ba_Adam_ICLR_2015} with a learning rate of $0.01$ and optimized for $2000$ epochs. The values of the vector that represents the radial mask were initialized randomly with a uniform distribution inside the interval~$[-0.5, 0.5]$. As baselines for this experiment, we used the widely-used star-chart-like radial mask as Ref.~\cite{nakamura_etal_radial_IAOC_2020}.
Such a radial mask has two important properties: (1)~they are cyclical, meaning that all radial features are the same angle apart from their immediate neighbors, and (2)~they are binary in terms of light transmittance.
For the experiments, we compared our optimized mask against three baseline radial masks, where the first of them had 20 radial sections, the second had 40 radial sections, and the final one had 60 radial sections.

Figure~\ref{fig:RadialMaskCompare} presents the baseline radial masks, as well as our optimized radial mask.
Similarly to the baseline masks, our optimized mask also retained a binary pattern. The average light transmittance of the optimized mask is approximately 43~\%, which is not distant to the 50~\% of the baseline hand-crafted radial masks.
Interestingly, neither the binarization nor the average light transmittance were enforced explicitly throughout training and were achieved solely by optimization through our proposed MTF-targeted loss. The main difference between the optimized mask and the baseline radial masks is that ours has an acyclic pattern for its radial sections.

Figure~\ref{fig:MtfComparison} presents a comparison of the MTFs of the radial masks.
For the baseline masks, we observed that the cyclic characteristic of the mask defines a trade-off between low-frequency and high-frequency response. 
That is because an increase in radial sections increases the high-frequency response of the mask, but incurs a decrease of sparsity around the area of the mask that reduces the low-frequency response.
Our optimized mask, on the other hand, leveraged acyclicity to maintain areas with more and less sparsity, which improved the overall MTF values up to its Nyquist limit.