\documentclass[lettersize,journal]{IEEEtran}
% \usepackage{amsmath,amsfonts,amssymb}
% \usepackage{algorithmic}
% \usepackage{algorithm}
% \usepackage{array}
% \usepackage[caption=false,font=normalsize,labelfont=sf,textfont=sf]{subfig}
% \usepackage{textcomp}
% \usepackage{stfloats}
% \usepackage{url}
% \usepackage{verbatim}
% \usepackage{graphicx}
% \usepackage{cite}

\usepackage{xcolor}
\usepackage{cite}
\usepackage{amsmath,amssymb,amsfonts,amsbsy}
\usepackage{algorithmic}
\usepackage{graphicx}
\usepackage{textcomp}
\DeclareMathOperator*{\argmin}{arg\, min\, }
\def\BibTeX{{\rm B\kern-.05em{\sc i\kern-.025em b}\kern-.08em
    T\kern-.1667em\lower.7ex\hbox{E}\kern-.125emX}}
% updated with editorial comments 8/9/2021
\def\BibTeX{{\rm B\kern-.05em{\sc i\kern-.025em b}\kern-.08em
    T\kern-.1667em\lower.7ex\hbox{E}\kern-.125emX}}
\usepackage{lineno}
\usepackage{subcaption}
\usepackage[normalem]{ulem}

\newcommand{\red}[1]{\textcolor{red}{#1}}
\newcommand{\blue}[1]{\textcolor{blue}{#1}}
\newcommand{\green}[1]{\textcolor{green}{#1}}
% \linenumbers

\begin{document}

% Change here to exclude header of IEEE Trans. Comp. Imaging for initial Arxiv submission
\def\ManuscriptType{ARXIV}
\def\IEEEpaper{IEEE}
\def\ARXIVpaper{ARXIV}


\title{Extended Depth-of-Field Lensless Imaging\\using an Optimized Radial Mask}

\author{Jos\'{e} Reinaldo Cunha Santos A. V. Silva Neto, Tomoya Nakamura, Yasushi Makihara,\\and Yasushi Yagi,~\IEEEmembership{Senior Member, IEEE},
\thanks{This work was supported by JST FOREST under Grant JPMJFR206K. (Corresponding author: Tomoya Nakamura.)

José Reinaldo Cunha Santos A. V. Silva Neto, 
Tomoya Nakamura, 
Yasushi Makihara, 
and Yasushi Yagi 
are with the SANKEN, Osaka University, Osaka, 567-0047, Japan (e-mail: vieira@am.sanken.osaka-u.ac.jp; nakamura@am.sanken.osaka-u.ac.jp; makihara@am.sanken.osaka-u.ac.jp; 
yagi@am.sanken.osaka-u.ac.jp).
}}

% The paper headers
\ifx\ManuscriptType\IEEEpaper
\markboth{IEEE transactions on computational imaging, vol. xx, YEAR}{C. S. A. V. Silva Neto \MakeLowercase{\textit{et al.}}: Extended Depth-of-Field Lensless Imaging using an Optimized Radial Mask}
\fi
% \IEEEpubid{0000--0000/00\$00.00~\copyright~2021 IEEE}
% Remember, if you use this you must call \IEEEpubidadjcol in the second
% column for its text to clear the IEEEpubid mark.

\maketitle

\begin{abstract}
The freedom of design of coded masks used by mask-based lensless cameras is an advantage these systems have when compared to lens-based ones. We leverage this freedom of design to propose a shape-preserving optimization scheme for a radial-type amplitude coded mask, used for extending the depth of field (DOF) of a lensless camera. Our goal is to identify the best parameters for the coded mask, while retaining its radial characteristics and therefore extended-DOF capabilities. We show that our optimized radial mask achieved better overall frequency response when compared to a naive implementation of a radial mask. We also quantitatively and qualitatively demonstrated the extended DOF imaging achieved by our optimized radial mask in simulations by comparing it to different non-radial coded masks. Finally, we built a prototype camera to validate the extended DOF capabilities of our coded mask in real scenarios.
\end{abstract}

\begin{IEEEkeywords}
Lensless camera, extended depth-of-field, PSF engineering
\end{IEEEkeywords}

\section{Introduction}


Recent years have witnessed the rise of human digitization~\cite{habermannDeepCapMonocularHuman2020,alexanderCREATINGPHOTOREALDIGITAL,pengNeuralBodyImplicit2021,alldieckDetailedHumanAvatars2018, rajANRArticulatedNeural2020}. This technology greatly impacts the entertainment, education, design, and engineering industry.
There is a well-developed industry solution for this task.
High-fidelity reconstruction of humans can be achieved either with full-body laser scans~\cite{saitoSCANimateWeaklySupervised2021}, dense synchronized multi-view cameras~\cite{xiangModelingClothingSeparate2021a,xiangDressingAvatarsDeep2022a}, or light stages~\cite{alexanderCREATINGPHOTOREALDIGITAL}.
However, these settings are expensive and tedious to deploy and consist of a complex processing pipeline, preventing the technology's democratization.

Another solution is to view the problem as inverse rendering and learn digital humans directly from custom-collected data.
Traditional approaches directly optimize explicit mesh representation~\cite{loperSMPLSkinnedMultiperson2015, fangRMPERegionalMultiperson2018, pavlakosExpressiveBodyCapture2019} which suffers from the problems of smooth geometry and coarse textures~\cite{prokudinSMPLpixNeuralAvatars2020,alldieckVideoBasedReconstruction2018}. Besides, they require professional artists to design human templates, rigging, and unwrapped UV coordinates.
Recently, with the help of volumetric-based implicit representations~\cite{mildenhallNeRFRepresentingScenes2020, parkDeepSDFLearningContinuous2019, meschederOccupancyNetworksLearning2019} and neural rendering~\cite{laineModularPrimitivesHighPerformance2020, liuSoftRasterizerDifferentiable2019, thiesDeferredNeuralRendering2019}, 
one can easily digitize a quality-plausible human avatar from video footage~\cite{jiangNeuManNeuralHuman2022,wengHumanNeRFFreeviewpointRendering}.
Particularly, volumetric-based implicit representations~\cite{mildenhallNeRFRepresentingScenes2020, pengNeuralBodyImplicit2021} can reconstruct scenes or objects with much higher fidelity against previous neural renderer~\cite{thiesDeferredNeuralRendering2019,prokudinSMPLpixNeuralAvatars2020}, and is more user-friendly as it does not need any human templates, pre-set rigging, or UV coordinates.
Captured visual footage and corresponding skeleton tracking are enough for training.
However, better reconstructions and more friendly usability are at the expense of the following factors.
1) \textbf{Inefficiency:}
They require longer optimization times (typically tens of hours or days) and inference slowly.
Volume rendering~\cite{mildenhallNeRFRepresentingScenes2020,lombardiNeuralVolumesLearning2019} formulates images by querying the densities and colors of millions of spatial coordinates. 
In the training stage, due to memory constraints, only a small fraction of points are sampled which leads to slow convergence speed.
2) \textbf{Entangled representations}:
The geometry, materials, and motion dynamics are entangled in the neural networks. 
Due to the implicit nature of neural nets, one can hardly edit one property without touching the others~\cite{yuanNeRFEditingGeometryEditing2022a,liuEditingConditionalRadiance2021}.
3) \textbf{Graphics incompatibility}:
Volume rendering is incompatible with the current popular graphic pipeline,
which renders triangular/quadrilateral meshes efficiently with the rasterization technique.
Many downstream applications require mesh rasterization in their workflow (\eg, editing~\cite{foundationBlenderOrgHome}, simulation~\cite{benderPositionBasedSimulationMethods2015}, real-time rendering~\cite{akenine2019real}, ray-tracing~\cite{waldRTXRayTracing}).
Although there are approaches~\cite{lorensenMarchingCubesHigh,labelleIsosurfaceStuffingFast2007} can convert volumetric fields into meshes, the gaps from discrete sampling degrade the output quality in terms of both meshes and textures.


To address these issues, we present \textbf{EMA}, a method based on \textbf{E}fficient \textbf{M}eshy neural fields to reconstruct animatable human \textbf{A}vatars.
Our method enjoys flexibility from implicit representations and efficiency from explicit meshes, yet still maintains high-fidelity reconstruction quality.
Given video sequences and the corresponding pose tracking, our method digitizes humans in terms of canonical triangular meshes, physically-based rendering (PBR) materials, and skinning weights \textit{w.r.t.} skeletons.
We jointly learn the above components via inverse rendering~\cite{laineModularPrimitivesHighPerformance2020,chenDIBRLearningPredict2021,chenLearningPredict3D2019} in an end-to-end manner.
Each of them is derived from a separate neural field, which relaxes the requirements of a preset human template, rigging, or UV coordinates.
Specifically, we predict a canonical mesh out of a signed distance field (SDF) by differentiable marching tetrahedra~\cite{shenDeepMarchingTetrahedra2021,gaoGET3DGenerativeModel,gaoLearningDeformableTetrahedral2020,munkbergExtractingTriangular3D2022}, then we extend the marching tetrahedra~\cite{shenDeepMarchingTetrahedra2021} for spatial-varying materials by utilizing a neural field to predict PBR materials \textit{on the mesh surfaces} after rasterization~\cite{munkbergExtractingTriangular3D2022,hasselgrenShapeLightMaterial2022,laineModularPrimitivesHighPerformance2020}.
To make the canonical mesh animatable, we take another neural field to model the forward linear blend skinning for the meshes. 
Given a posed skeleton, the canonical mesh is then transformed into the corresponding poses.
Finally, we shade the mesh with a rasterization-based differentiable renderer~\cite{laineModularPrimitivesHighPerformance2020} and train our models with a photo-metric loss.
After training, we export the mesh with materials and discard the neural fields.

\looseness=-1
There are several merits of our method design.
1) \textbf{Efficiency}:
Powered by efficient mesh rendering, our method can render in real-time.
Besides, the training speed is boosted as well, 
since we compute loss holistically on the whole image and the gradients only flow on the mesh surface. In contrast, volume rendering takes limited pixels for loss computation and back-propagates the gradients in the whole space.
Our method only needs about an hour of training and minutes of optimization are enough for plausible avatar reconstruction.
2) \textbf{Disentangled representations}:
Our shape, materials, and motion modules are disentangled naturally by design, which facilitates editing. 
Besides, Canonical meshes with forward skinning modeling handle the out-of-distribution poses better.
3) \textbf{Graphics compatibility}:
Our derived mesh representation is compatible with 
the prominent graphic pipeline, which leads to instant downstream applications (\eg, the shape and materials can be edited directly in design software~\cite{foundationBlenderOrgHome}).
To further improve reconstruction quality, we additionally optimize image-based environment lights and non-rigid motions.


We conduct extensive experiments on standards benchmarks H36M~\cite{ionescuHuman36MLarge2014b} and ZJU-MoCap~\cite{pengNeuralBodyImplicit2021}.
Our method achieves very competitive performance for novel view synthesis, generalizes better for novel poses, 
and significantly improves both training time and inference speed against previous arts.
Our research-oriented code reaches real-time inference speed (100+ FPS for rendering $512\times512$ images).
We in addition showcase applications including novel pose synthesis, material editing, and relighting.
\section{Radial mask optimization}
In this section, we present the radial mask optimization process. We begin in Section~\ref{subsec:RadialMaskParameterization} presenting a parameterization of the radial coded mask, in order to constrain the radial shape of the coded mask throughout the optimization process. Then section~\ref{subsec:LossFunction} presents our objective function and optimization scheme for the search for the best parameters of the radial mask. Finally, section~\ref{subsec:OptimizationResults} presents the results of the optimization process and compares the optimized mask to the original radial mask proposed in the literature. 

\begin{figure}[!t]
    \centering
    \includegraphics[width=\linewidth]{figs/MaskOptimization/RadialMaskParameterizationV2.pdf}
    \caption{Overview of the radial coded mask parameterization process.
    Left: the usual modeling of a coded mask, where the light transmittance on an arbitrary discretized position in the Cartesian coordinates $(x,y)$ on the mask plane is represented as $m(x,y)$.
    Right: our proposed radial-mask parameterization, where each element of a vector $\Theta$ is related to the light transmittance~$m(\theta)$ of one of the radial sections of a coded mask. 
    The radial sections are aligned along the angular axis $\theta$.
    The logistic sigmoid function~$\sigma(\cdot)$ is multiplied on the original vector elements in order to ensure that all light transmittance values are in the range $[0,1]$.}
    \label{fig:MaskParameterization}
\end{figure}

% Explain how to perform shape-aware optimization for radial coded masks
\subsection{Radial Mask Parameterization}
\label{subsec:RadialMaskParameterization}
In past studies related to deep optics, the mask optimization process has been done independently for each element $m(x,y)$, which represents the light transmittance at the discretized Cartesian coordinates $(x,y)$ on the mask-plane.
However, this cannot be done for this work, as it is unlikely for the radial characteristic of the mask to be preserved throughout the optimization process.
In order to address this issue, we parameterize the coded mask by splitting its area into $N_\mathrm{t}$ radial sections $\boldsymbol{\Theta}=[\theta_0, \theta_1, \cdots, \theta_{N_\mathrm{t}-1}]$ along angular coordinate, where each section originates in the center of the mask and expands to its edges.
We then enforce every mask element inside the same radial section to have the same light transmittance value, as follows:
\begin{equation}
    m(x,y) = m(\theta_k)  \  \forall \ (x,y)  \ \in \ \theta_k,
\end{equation}
where $k$ is the index of the radial section, and $(x,y) \in \theta_k$ indicates a set of spatial mask coordinates that belong to the $k$-th radial section. In this scenario, the value of the light transmittance for the $k$-th radial section is:
\begin{equation}
    m(\theta_k) = \sigma( m'(\theta_k)),
\end{equation}
where $\sigma(\cdot)$ represents a logistic sigmoid function, and $m'(\theta_k)$ is a single scalar value related to the light transmittance in the $k$-th radial section. Here, the sigmoid function is employed to ensure that the values of light transmittance of the mask were in the interval $[0,1]$. For simplicity of representation, we will not explicitly write the $\sigma(\cdot)$ for the sigmoid function from this point onwards.

Here we denote a dimensionally-reduced radial-mask vector $\boldsymbol{m}_\mathrm{\theta} \in \mathbb{R}^{N_\mathrm{t} \times 1}$ that represents all the light-transmittance parameters in a radial mask along angular axis $[m(\theta_0), m(\theta_1), \cdots, m(\theta_{N_\mathrm{t}-1})]$,
and $p(\cdot)$ represents the mapping of the light transmittance elements of the radial mask in angular coordinates to a coded mask in Cartesian coordinates.
Assuming that the PSF~$\textbf{h}$ can be approximated as a mask vector itself, in which a diffraction effect is ignored, the PSF vector in Eq.~\eqref{eq:conv} can be related to the dimensionally-reduced radial-mask vector as follows:
\begin{equation}
    \textbf{h} = p(\boldsymbol{m}_{\mathrm{\theta}}).
\end{equation}

%Using a polar-coordinate system, we represent the mask parameters to be optimized as a 1D vector~$\Theta$, whose components represent the angular coordinates or sections of the coded mask, as follows:
%\begin{equation}
%    \Theta = [\theta_0, \theta_1, \cdots, \theta_{N-1}],
%\end{equation}
%where $N \in \mathbb{N}$ is the total count of radial sections in the mask.}

On the left side of Fig.~\ref{fig:MaskParameterization} we demonstrate a common coded mask modeling approach, where the light transmittance on a position $(x,y)$ on the plane of the mask is denoted as $m(x,y)$. On the right side of Fig.~\ref{fig:MaskParameterization}, we show our proposed radial mask parameterization process. The number of radial sections~$N_\mathrm{t}$ is a hyperparameter, and it was determined manually before the optimization experiments. We empirically selected a value of $70$ for this parameter. 
This was decided based on the resolution used for the coded mask on the optimization experiments, which was $140 \times 140$ pixels.
%\red{A higher number of radial structures could render some of them to have many discontinuous areas, which is not desirable.}
% We, therefore, can represent the radial mask as a 1 dimensional vector, where each element is related to the light transmittance of one of the radial sections of the coded mask. 

% Define loss function to be optimized (MTF loss)
\subsection{Loss Function and Optimization Process}
\label{subsec:LossFunction}
For optimization, we aim at improving the frequency response, i.e. MTF, of the lensless optical system. For this paper, we desire to increase MTF values of a coded mask across all frequencies.
The MTF of a PSF~$\textbf{h}$ is defined as follows~\cite{Goodman1996}:
\begin{equation}
    \mathrm{MTF}(\textbf{h}) = \mathrm{normalize}\left(|\mathcal{F}[\textbf{h}]|\right),
\end{equation}
where $\mathrm{normalize}(\cdot)$ is the normalization operator with a value at zero frequency, $\mathcal{F[\cdot]}$ corresponds to a 2D discrete Fourier transform, and $|\cdot|$ computes the absolute values.
Our proposed loss function for mask-parameter optimization is as follows:
\begin{align}
    \mathcal{L}(\boldsymbol{m}_\mathrm{\theta}) 
    &= (-1)\times \mathrm{mean}(\mathrm{MTF}(\textbf{h}))\\
    &= (-1)\times \mathrm{mean}(\mathrm{MTF}(p(\boldsymbol{m}_\mathrm{\theta}))),
\end{align}
where $\mathrm{mean(\cdot)}$ represents the average over all elements.
% As we want to increase the MTF values over frequencies, we multiply by $-1$ in order to turn it into a minimization problem.
Note that as we aim to increase the overall MTF of the optimized coded mask, we multiply the averaged MTF by $-1$ in order to use this loss in a minimization problem. The problem to be solved for obtaining the MTF-targeted optimized design of the radial mask can be simply written as:
\begin{equation}
\hat{\boldsymbol{m}}_\mathrm{\theta}=\argmin_{\boldsymbol{m}_\mathrm{\theta}}\mathcal{L}(\boldsymbol{m}_\mathrm{\theta}),
\label{eq:minimize}
\end{equation}
where the $\hat{\boldsymbol{m}}_\mathrm{\theta} \in \mathbb{R}^{N_\mathrm{t}\times 1}$ is the vector representing the optimized radial coded mask having dimensionally-reduced optimized parameters.
By solving Eq.~\eqref{eq:minimize} and applying the mapping function~$p$ after optimization, the radial-shape-preserved MTF-targeted optimized coded mask with spatial light-transmittance parameters can be obtained.

\begin{figure}[!t]
    \centering
    \includegraphics[width=\linewidth]{figs/MaskOptimization/RadialMasksCompare.pdf}
    \caption{Radial masks used for comparison.
    The baseline radial masks with (a)~$20$, (b)~$40$, and (c)~$60$ radial sections, respectively.
    (d)~Our optimized radial mask.}
    \label{fig:RadialMaskCompare}
\end{figure}

\begin{figure}[!t]
    \centering
    \includegraphics[width=\linewidth]{figs/MaskOptimization/MtfComparison_zerofreq.pdf}
    \caption{Comparison of MTFs of the baseline and optimized radial masks.
    $\omega$ indicates the Nyquist frequency of a mask.}
    \label{fig:MtfComparison}
\end{figure}

% Present results of the radial mask optimization, comparing MTF to the hand-crafted radial masks from previous work.
\subsection{Optimization Experiment}
\label{subsec:OptimizationResults}
For the optimization, we used the Adam optimizer~\cite{Kingma_Ba_Adam_ICLR_2015} with a learning rate of $0.01$ and optimized for $2000$ epochs. The values of the vector that represents the radial mask were initialized randomly with a uniform distribution inside the interval~$[-0.5, 0.5]$. As baselines for this experiment, we used the widely-used star-chart-like radial mask as Ref.~\cite{nakamura_etal_radial_IAOC_2020}.
Such a radial mask has two important properties: (1)~they are cyclical, meaning that all radial features are the same angle apart from their immediate neighbors, and (2)~they are binary in terms of light transmittance.
For the experiments, we compared our optimized mask against three baseline radial masks, where the first of them had 20 radial sections, the second had 40 radial sections, and the final one had 60 radial sections.

Figure~\ref{fig:RadialMaskCompare} presents the baseline radial masks, as well as our optimized radial mask.
Similarly to the baseline masks, our optimized mask also retained a binary pattern. The average light transmittance of the optimized mask is approximately 43~\%, which is not distant to the 50~\% of the baseline hand-crafted radial masks.
Interestingly, neither the binarization nor the average light transmittance were enforced explicitly throughout training and were achieved solely by optimization through our proposed MTF-targeted loss. The main difference between the optimized mask and the baseline radial masks is that ours has an acyclic pattern for its radial sections.

Figure~\ref{fig:MtfComparison} presents a comparison of the MTFs of the radial masks.
For the baseline masks, we observed that the cyclic characteristic of the mask defines a trade-off between low-frequency and high-frequency response. 
That is because an increase in radial sections increases the high-frequency response of the mask, but incurs a decrease of sparsity around the area of the mask that reduces the low-frequency response.
Our optimized mask, on the other hand, leveraged acyclicity to maintain areas with more and less sparsity, which improved the overall MTF values up to its Nyquist limit.
\section{Simulations}
% Present reconstruction results for dual-depth object reconstruction showcasing the extended-depth-of-field capabilities of the radial coded mask when comparing to other coded aperture types
The primary goal of a radial mask is to extend the DOF of a lensless imaging system. So far, we have only searched for the best parameters for such a coded mask, without investigating its extended DOF properties. In this section, we determine the extended DOF of a radial mask through simulations. We also show that the reconstructed image from the lensless camera can be electronically refocused, after the image sensor measurement capture, by changing the scaling of the PSF of the coded mask. 

\begin{figure}[!t]
    \centering
    \includegraphics[width=\linewidth]{figs/DOF_simulation/SankenOU_diagram.pdf}
    \caption{Geometrical setup of simulations.
    The OU pattern object was positioned at a distance of $5.0$~cm away from the coded mask, while the farther object was at $30.0$~cm away.
    The distance between the image sensor and the coded mask was set to $4.0$ mm.}
    \label{fig:SimSetup}
\end{figure}

\begin{figure}[!t]
    \centering
    \includegraphics[width=\linewidth, trim = 0cm 0cm 0cm 0cm]{figs/DOF_simulation/ADMM_UDN_analysis.pdf}
    \caption{Results of the imaging simulation.
    The top two rows present, respectively, the PSFs and sensor measurements.
    The next two rows are the reconstructed images using the ADMM algorithm and their close-ups of the OU pattern object, which cannot be correctly reconstructed by conventional methods.
    The next two rows are the reconstruction by the UDN algorithm and close-ups.
    From the left column, the results correspond to the ground truth, the radial mask, the FZA, and the random mask, respectively.
    }
    \label{fig:SimDOF_recon}
\end{figure}

\subsection{Conditions}
\label{subsec:DOF_sim}
The simulation was performed as a dual-depth object reconstruction, where we had two objects at two different distances from the lensless camera. We geometrically model the PSFs projected from both distances, and use them to generate the simulated sensor measurements for the full scene. The simulation of the sensing process was performed by calculating Eq.~\eqref{eq:3Dforward}. We then reconstructed an image from the full sensor measurements by using a single PSF, that could be from either of the distances. Generally, for any type of coded mask, it is expected that the object placed at the same distance at which the PSF was calibrated should be reconstructed with a higher quality than objects at different distances from the lensless camera.

As baselines for comparison against the radial mask, 
we used a Fresnel zone aperture~(FZA)~\cite{Shimano2018, Wu2020, Nakamura2020OE} and a random mask~\cite{Nakamura2019,Zheng2021,Boominathan2020} as examples.
The random mask is a naive design for 2D lensless imaging that has good MTF up to cutoff frequency.
The FZA is a coded mask with a structure only in the radial direction, opposite to the radial mask, and is suitable for digital refocusing applications.

For the lensless image reconstruction, we used two algorithms, namely the alternating direction method of multipliers (ADMM) method~\cite{boyd_etal_ADMM_NOW_2011}, and the untrained deep network (UDN) method~\cite{monakhova_etal_UDN_OptExp_2021}.
Both methods are based on the iterative error-minimization algorithm involving regularization.
For the regularization, the ADMM uses the minimization of 2D total variation~(TV)~\cite{Rubin1992} of a reconstructed image, while the UDN implements it by an untrained generative deep neural network, i.e., employment of deep image prior~\cite{DIP}.
Compared to learning-based methods~\cite{Sinha2017, Monakhova2019OE, Barba2019, Rego2022}, the results can be explainable and their precision is not restricted to a domain of learning.

Figure~\ref{fig:SimSetup} shows the simulated experimental setup.
It involves a planar plush toy and a planar OU pattern positioned $30.0~\mathrm{cm}$ and $5.0~\mathrm{cm}$ away from the coded mask, respectively.
The axial interval between the mask and an image sensor was set to $4.0~\mathrm{mm}$.
Figure~\ref{fig:SimDOF_recon} presents the mask patterns used for simulations, corresponding captured images, and reconstruction results with the ADMM and UDN algorithms using the PSF calibrated for a distance of $30.0~\mathrm{cm}$.
We set the size of the RGB captured measurements, simulated PSFs, and reconstructed images to $512 \times 612$ pixels.
In simulations, the noise was ignored to analyze the upper limit of the effect of the proposed methodology; however, a noise analysis can be drawn from the prototype camera experiments.
The coded masks used in the simulations are the same ones that were used for the prototype camera experiment.
In the reconstruction process, we used 150,000 iterations for the UDN algorithm and 100 iterations per channel for the ADMM algorithm.
The optimization code was implemented in Pytorch with a computational environment including a GPU (GeForce 3090 by NVIDIA), 32 GiB RAM, and a 10-core CPU (i9-10900K by Intel).


\subsection{Reconstruction Results}
We limited the effective area of the mask to approximately $50~\%$ in the central region for increasing the stability of reconstruction~\cite{Antipa2018}, and the remaining perimeter of the mask was light-shielded.
From the reconstruction results of Fig.~\ref{fig:SimDOF_recon}, we observed that the plush toy was correctly reconstructed by all three types of coded masks. This was expected, as the PSF used for reconstruction was the one calibrated for the same distance as the plush-toy distance. The OU pattern, however, was closer to the lensless system than the calibrated PSF and was only reconstructed properly by the radial mask. Additionally, we note that the peak signal-to-noise ratio (PSNR) of the radial-mask reconstruction was significantly higher than that of the FZA and random coded masks. Table~\ref{tab:SimAnalysis}(a) presents a quantitative analysis of the OU pattern reconstruction between the three types of coded masks shown in Fig.~\ref{fig:SimDOF_recon}. We note that the radial mask achieves the best reconstruction of the OU pattern in all metrics including SSIM~\cite{Wang2004} and LPIPS~\cite{Zhang2018}.

\begin{figure}[!t]
    \centering
    \includegraphics[scale=0.8]{figs/DOF_simulation/electronicRefocus.pdf}
    \caption{Demonstration of the refocusing ability of a lensless camera.
    The lensless sensor measurements were the same ones shown in Fig.~\ref{fig:SimDOF_recon}.
    The far-focus and close-focus PSFs indicate simulated PSFs created by a light source placed $30.0$~cm and $5.0$~cm away from the coded mask, respectively.
    The images were reconstructed by the ADMM algorithm.}
    \label{fig:ElectronicRefocus}
\end{figure}

\begin{table}[]
\centering
\caption{Quantitative analysis of the simulations.}
\label{tab:SimAnalysis}
\subcaption*{(a) Comparison of the quality of the reconstructed OU patterns for different types of coded masks, as presented in Figure~\ref{fig:SimDOF_recon}.}
\label{tab:ReconCompare}
\resizebox{\columnwidth}{!}{%

\begin{tabular}{lcccccc}
\multicolumn{1}{c}{}        &          & ADMM     &                                &          & UDN      &           \\
\multicolumn{1}{c|}{}       & PSNR (↑) & SSIM (↑) & \multicolumn{1}{c|}{LPIPS (↓)} & PSNR (↑) & SSIM (↑) & LPIPS (↓) \\ \hline
\multicolumn{1}{l|}{Radial} & \textbf{20.64} & \textbf{0.6151} & \multicolumn{1}{c|}{\textbf{0.2290}} & \textbf{19.71} & \textbf{0.6700} & \textbf{0.1451} \\
\multicolumn{1}{l|}{FZA}    & 15.38    & 0.4536   & \multicolumn{1}{c|}{0.4285}    & 14.99    & 0.5300   & 0.3876    \\
\multicolumn{1}{l|}{Random} & 14.97    & 0.4862   & \multicolumn{1}{c|}{0.6189}    & 15.04    & 0.5111   & 0.5493   
\end{tabular}%
}
\bigskip
\subcaption*{(b) Correlation between close focus and far focus PSFs presented in Fig.~\ref{fig:ElectronicRefocus}.}
\begin{tabular}{c|ccc}
    & Radial                                          & FZA                                             & Random                                           \\\hline
MAE & $1.37\times10^{-6}$ & $3.44\times10^{-6}$ & $3.54\times10^{-6}$
\end{tabular}%
\end{table}

\begin{figure*}[!t]
    \centering
    \includegraphics[width=\textwidth]{figs/DOF_exp/FullDofExpV2.pdf}
    \caption{(a)~A prototype of a lensless camera, with an SLM creating the radial mask, and an image sensor behind it.
    (b)~Setup for the experiment.
    (c)~Calibrated PSFs for the radial, FZA, and random mask.
    (d)~Experimentally obtained sensor measurements.
    Reconstruction results using the (e)~ADMM and (f)~UDN algorithm.
    (g) Close-ups of the OU pattern reconstructed by the UDN algorithm.}
    \label{fig:RealExp}
\end{figure*}

\subsection{Refocusing Ability}
Additionally, we present in Fig.~\ref{fig:ElectronicRefocus} the refocus of the reconstructed image after sensor measurements capture. We show that by changing the scaling of the PSF, it is possible to focus on either the closer object, or the farther object in the simulated scene. We note, however, that our optimized radial mask is capable of reconstructing both objects independently of the scaling used for the PSF. Table~\ref{tab:SimAnalysis}(b) presents a quantitative comparison between far-focus and close-focus PSFs for the three types of coded masks presented in Fig.~\ref{fig:ElectronicRefocus}. We note that the radial mask presents a lower mean absolute error~(MAE) between far and close-focus PSFs, when compared to FZA and random masks. This justifies why the reconstruction algorithm is capable of reconstructing both objects successfully using either of the calibrated radial PSFs, which is not the case for the other types of coded masks. 


\section{Optical Experiments with a Prototype Camera}

\subsection{Setup}
Finally, we create a prototype lensless camera, to validate the extended DOF of the radial mask in the real world.
Similarly to the simulation, this experiment consists of reconstructing two objects at different distances from the lensless imager, but in a real scenario.
Figure~\ref{fig:RealExp}(a) shows a frontal view of the prototype lensless camera, which consists of an axial stack of a coded mask and an image sensor.
The coded mask was implemented by a transmissive liquid-crystal SLM~(LC2012 by HOLOEYE Photonics) and two polarization plates in the crossed Nicols configuration.
All coded masks to be tested are originally binary, therefore, the light transmittance of the SLM was designed as binary, and central $188 \times 228$ pixels with $36 \times 36\ \mathrm{\mu m}$ pitches were used for implementing the coded masks.
The fill factor of the SLM was $58~\%$.
Approximately $4.0~\mathrm{mm}$ behind the modulation plane of the SLM, we placed a color CMOS image sensor~(BFS-U3-51S5C-BD by Teledyne FLIR) whose pixel count was $2048 \times 2448$ with $3.45~\mathrm{\mu m} \times 3.45~\mathrm{\mu m}$ pitches.
In the experiment, $8$-bit RGB captured images were readout and they were downsampled to $512 \times 612$ pixels for reconstruction.
As well as simulations, the periphery of the mask was shielded for increasing reconstruction stability where the effective area of the mask was approximately $50~\%$.
The center of the effective area of the SLM and the image sensor were aligned by translation stages, and the planes of the two elements were adjusted to be parallel.

Figure~\ref{fig:RealExp}(b) shows the experimental setup including the camera and targets to be imaged.
In front of a lensless camera prototype, we placed two diffuse reflective objects: a stuffed toy known as SANKEN, which is one of the symbols of Osaka University~(SANKEN plush toy), and a black tape with the letters 'OU' printed on it~(OU letters).
The SANKEN toy and the OU letters were placed at approximately $30$ and $9$ centimeters~(cm) away from the mask, respectively.
These objects were illuminated by a white LED light Installed around $12~\mathrm{cm}$ above the camera.

\subsection{PSF Calibration}
The top right in Fig.~\ref{fig:RealExp} shows the calibrated PSFs. The PSFs were calibrated by experimental capture of a spherical wave emitted from a light point source placed $30~\mathrm{cm}$ away from the camera, which was the same distance as the SANKEN toy. The light-point source we used was composed of a 
semiconductor laser whose central wavelength was 532~nm~(Stradus 532 by Vortran Laser Technology), followed by a spatial filter~(SFB-16DMRO-OBL40-25 by SIGMA KOKI) which contained a pinhole whose diameter was $25~\mathrm{\mu m}$. The combination of the laser with the spatial filter generated a spherical wave.

\subsection{Experiments}
%Capture
The second row of the right in Fig.~\ref{fig:RealExp} shows the captured lensless measurements.
Although the captured image cannot be recognized by human vision, the encoded images of objects at multiple distances were multiply recorded based on the physical model in Eq.~\eqref{eq:3Dforward}.
The captured image also contains color information.
Note that each coded image of Fig.~\ref{fig:RealExp} was normalized for visualization.

%Reconstruction
The PSF was calibrated by the point light source positioned $30~\mathrm{cm}$ away from the coded mask. Therefore, it was expected for the object at a $30~\mathrm{cm}$ distance to be correctly reconstructed for all three types of coded masks. The OU pattern object, however, was placed $9~\mathrm{cm}$ away from the coded mask and was expected to be more challenging to be reconstructed properly.
The coded masks and reconstruction algorithms used here were the same as those used in the simulations. The reconstructions for the radial, FZA, and random coded masks using the ADMM and UDN algorithms are shown in the third and fourth rows of the right part of Fig.~\ref{fig:RealExp}, respectively. The bottom row presents the close-up view of the OU letters reconstructed by the UDN algorithm. As expected, the plush toy was correctly reconstructed in all experiments, independently of the type of coded mask or the reconstruction algorithm employed. We note, however, that reconstruction with resolving the two letters on the OU pattern was only successful by the radial coded mask, due to its robustness against the scaling of its PSF, i.e., extended DOF characteristics. The reconstructions using the FZA and random masks, on the other hand, were blurred and the two letters seemed to mix together.



\section{Conclusion}
% Summarize the results, indicate possible future work directions
We proposed a radial-shape-preserving optimization scheme for coded masks, which can be used to systematically create radial masks with better overall frequency response when compared to the conventional radial masks.
We showed through simulations that the optimized radial mask was capable of extending the effective DOF of a lensless camera when compared to other types of coded masks.
We also built a prototype lensless camera and empirically validated the extended DOF capabilities of the radial mask in real scenarios.

\subsection{Limitations and Future Works}
Theoretically, the PSF of a radial-shaped coded mask is depth-independent, meaning that it is not affected by radial scaling as illustrated in Figure~\ref{fig:intro_complete}(c).
That would be the case if the effective area of the coded mask were larger than that of an image sensor.
In practice, however, the coded pattern is often smaller than the effective pixel area of an image sensor, with the edges of the coded pattern being light-shielded.
This shielding is necessary when the forward model is approximately described by a convolution because it suppresses the amount of information interception by the cropping function in sensing.
In this case, even though the reconstruction processing works well and the coded pattern is depth-independent, the complete PSF formed by the coded pattern and its edge is actually depth dependent to some extent.
In future work, we will address this issue by non-convolutional, e.g. matricial, modeling of the forward problem, and/or the application of more robust compressive reconstruction algorithms such as the primal-dual splitting method, and deep-unrolling method.

In addition, this work only addressed the amplitude-modulation-type coded mask.
When compared to phase masks, one limitation of amplitude masks is the lower light-use efficiency and optical cut-off frequency.
In theory, however, the essence of this work is that the PSF is radially-shaped, and the mask implementation method and its design should be free.
Therefore, it is necessary for future works to develop masks for extended-DOF lensless imaging using phase masks with high light-utilization efficiency.

Finally, the acyclic nature of our optimized radial mask, while being leveraged for an improved frequency response, can also be a limiting factor for reconstruction of objects near the edges of the camera's field of view. That is because the low- and high-frequency components are unevenly distributed around the mask's area, and a translation of the PSF to areas near the edges of the image sensor may remove specific frequency-rich areas. Which is not the case for periodic or symmetric radial masks. 

\red{For future research, an interesting direction to be considered is to combine radial and non-radial features on a single optimized mask. In Section A of the supplementary materials, we show that a non-optimized random coded mask achieves higher in-focus PSNR for its reconstruction when compared to our optimized radial mask. An area that may be especially appealing towards combining radial and non-radial features may be polar coordinate masks\cite{Chen2017, Don2017}, that are a more generic representation of our parameterized radial mask. A promising direction would be to use a parameterized polar coordinate mask and optimize it using a loss that somehow defines a tradeoff between extended DOF and in-focus reconstruction quality.}

%\section*{Acknowledgments}
%\red{This should be a simple paragraph before the References to thank those individuals and institutions who have supported your work on this article.}


\begin{thebibliography}{1}
\bibliographystyle{IEEEtran}

\bibitem{Boominathan2022} V. Boominathan, J. T. Robinson, L. Waller, and A. Veeraraghavan, “Recent advances in lensless imaging,” {\it Optica}, vol. 9, no. 1, pp. 1--16, Jan. 2022.

\bibitem{Goodman1996} J. W. Goodman, Introduction to Fourier Optics. McGraw-Hill, 1996.

\bibitem{Antipa2018} N. Antipa, G. Kuo, R. Heckel, B. Mildenhall, E. Bostan, R. Ng, and L. Waller, “DiffuserCam: lensless single-exposure 3D imaging,” Optica, vol. 5, no. 1, pp. 1–9, 2018.

\bibitem{Martz1962} L. Mertz and N. O. Young, “Fresnel transformation of images (Fresnel coding and decoding of images),” in {\it Optical Instruments and Techinques}, 1962, p. 305.

\bibitem{Shimano2018} T. Shimano, Y. Nakamura, K. Tajima, M. Sao, and T. Hoshizawa, “Lensless light-field imaging with Fresnel zone aperture: quasi-coherent coding,” {\it Appl. Opt.}, vol. 57, no. 11, pp. 2841--2850, 2018.

\bibitem{Zheng2020} Y. Zheng and M. Salman Asif, “Joint Image and Depth Estimation With Mask-Based Lensless Cameras,” IEEE Trans. Comput. Imaging, vol. 6, pp. 1167--1178, 2020.

\bibitem{Adams2017} J. K. Adams, V. Boominathan, B. W. Avants, D. G. Vercosa, F. Ye, R. G. Baraniuk, J. T. Robinson, and A. Veeraraghavan, “Single-frame 3D fluorescence microscopy with ultraminiature lensless FlatScope,” {\it Sci. Adv.}, vol. 3, no. 12, pp. 1--10, 2017.

\bibitem{Tian2022} F. Tian and W. Yang, “Learned lensless 3D camera,” {\it Opt. Express}, vol. 30, no. 19, pp. 34479--34496, Sep. 2022.

\bibitem{Tan2017} J. Tan, V. Boominathan, A. Veeraraghavan, and R. Baraniuk, “Flat focus: depth of field analysis for the FlatCam lensless imaging system,” in {\it 2017 IEEE International Conference on Acoustics, Speech and Signal Processing (ICASSP)}, 2017, pp. 6473--6477.

\bibitem{Hua2023} Y. Hua, M. S. Asif, and A. C. Sankaranarayanan, “Spatial and axial resolution limits for mask-based lensless cameras,” {\it Opt. Express}, vol. 31, no. 2, pp. 2538--2550, Jan. 2023.

\bibitem{Kutulakos2009} K. N. Kutulakos and S. W. Hasinoff, “Focal Stack Photography: High-Performance Photography with a Conventional Camera,” in {\it International Conference on Machine Vision and Applications}, 2009, pp. 332--337.

\bibitem{Huang2022} Z. Huang, J. A. Fessler, and T. B. Norris, “Focal stack camera: depth estimation performance comparison and design exploration,” {\it Opt. Contin.}, vol. 1, no. 9, pp. 2030--2042, Sep. 2022.

\bibitem{Brady2009} D. J. Brady, K. Choi, D. L. Marks, R. Horisaki, and S. Lim, “Compressive Holography,” {\it Opt. Express}, vol. 17, no. 15, p. 13040, Jul. 2009.

\bibitem{Candes2008} E. J. Candes and M. Wakin, “An Introduction To Compressive Sampling,” {\it IEEE Signal Process. Mag.}, vol. 25, no. 2, pp. 21--30, Mar. 2008.

\bibitem{Hua2020} Y. Hua, S. Nakamura, M. S. Asif, and A. C. Sankaranarayanan, “SweepCam - Depth-Aware Lensless Imaging Using Programmable Masks,” {\it IEEE Trans. Pattern Anal. Mach. Intell.}, vol. 42, no. 7, pp. 1606--1617, 2020.

\bibitem{Kuthirummal2011} S. Kuthirummal, H. Nagahara, C. Zhou, and S. K. Nayar, “Flexible Depth of Field Photography,” {\it IEEE Trans. Pattern Anal. Mach. Intell.}, vol. 33, no. 1, pp. 58--71, Jan. 2011.

\bibitem{Gill2013} P. R. Gill and D. G. Stork, “Lensless Ultra-Miniature Imagers Using Odd-Symmetry Spiral Phase Gratings,” in {\it Imaging and Applied Optics}, 2013, p. CW4C.3.

\bibitem{Gill2013OL} P. R. Gill, “Odd-symmetry phase gratings produce optical nulls uniquely insensitive to wavelength and depth,” {\it Opt. Lett.}, vol. 38, no. 12, pp. 2074--2076, 2013.

\bibitem{nakamura_etal_radial_IAOC_2020} T. Nakamura, S. Igarashi, S. Torashima, and M. Yamaguchi, "Extended depth-of-field lensless camera using a radial amplitude mask." in \textit{Imaging and Applied Optics Congress}, OSA Technical Digest, Optica Publishing Group, 2020.

\bibitem{Rasouli2018} S. Rasouli, A. M. Khazaei, and D. Hebri, “Talbot carpet at the transverse plane produced in the diffraction of plane wave from amplitude radial gratings,” {\it J. Opt. Soc. Am. A}, vol. 35, no. 1, pp. 55--64, 2018.

\bibitem{Horisaki_etal_DeeplyCoded_OL_2020} R. Horisaki, Y. Okamoto, and J. Tanida, “Deeply coded aperture for lensless imaging,” {\it Opt. Lett.}, vol. 45, no. 11, pp. 3131--3134, 2020.

\bibitem{Sitzmann2018} V. Sitzmann, S. Diamond, Y. Peng, X. Dun, S. Boyd, W. Heidrich, F. Heide, and G. Wetzstein., “End-to-end optimization of optics and image processing for achromatic extended depth of field and super-resolution imaging,” {\it ACM Trans. Graph.}, vol. 37, no. 4, pp. 1--13, 2018.

\bibitem{Zhou2021} H. Zhou, H. Feng, W. Xu, Z. Xu, Q. Li, and Y. Chen, “Deep denoiser prior based deep analytic network for lensless image restoration,” Opt. Express, vol. 29, no. 17, pp. 27237--27253, Aug. 2021.

\bibitem{Bacca_etal_DeepCoded_IEEE_2021} J. Bacca, T. Gelvez-Barrera and H. Arguello, "Deep Coded Aperture Design: An End-to-End Approach for Computational Imaging Tasks," in \textit{IEEE Trans. Comput. Imaging}, vol. 7, pp. 1148--1160, 2021.

\bibitem{Hain2018} H. Haim, S. Elmalem, R. Giryes, A. M. Bronstein, and E. Marom, “Depth Estimation From a Single Image Using Deep Learned Phase Coded Mask,” {\it IEEE Trans. Comput. Imaging}, vol. 4, no. 3, pp. 298--310, 2018.

\bibitem{Chang2018} J. Chang, V. Sitzmann, X. Dun, W. Heidrich, and G. Wetzstein, “Hybrid optical-electronic convolutional neural networks with optimized diffractive optics for image classification,” {\it Sci. Rep.}, vol. 8, no. 1, p. 12324, 2018.

\bibitem{Kingma_Ba_Adam_ICLR_2015} D. P. Kingma, J. Ba, "Adam: A Method for Stochastic Optimization," in \textit{3rd International Conference on Learning Representations (ICLR)}, 2015.

\bibitem{Wu2020} J. Wu, H. Zhang, W. Zhang, G. Jin, L. Cao, and G. Barbastathis, “Single-shot lensless imaging with fresnel zone aperture and incoherent illumination,” Light Sci. Appl., vol. 9, no. 1, p. 53, Apr. 2020.

\bibitem{Nakamura2020OE} T. Nakamura, T. Watanabe, S. Igarashi, X. Chen, K. Tajima, K. Yamaguchi, T. Shimano, and M. Yamaguchi, “Superresolved image reconstruction in FZA lensless camera by color-channel synthesis,” Opt. Express, vol. 28, no. 26, pp. 39137--39155, Dec. 2020.

\bibitem{Nakamura2019} T. Nakamura, K. Kagawa, S. Torashima, and M. Yamaguchi, “Super Field-of-View Lensless Camera by Coded Image Sensors,” Sensors, vol. 19, no. 6, p. 1329, 2019.

\bibitem{Zheng2021} Y. Zheng, Y. Hua, A. C. Sankaranarayanan, and M. S. Asif, “A Simple Framework for 3D Lensless Imaging with Programmable Masks,” in 2021 IEEE/CVF International Conference on Computer Vision (ICCV), Oct. 2021, no. Iccv, pp. 2583--2592.

\bibitem{Boominathan2020} V. Boominathan, J. K. Adams, J. T. Robinson, and A. Veeraraghavan, “PhlatCam: Designed Phase-Mask Based Thin Lensless Camera,” IEEE Trans. Pattern Anal. Mach. Intell., vol. 42, no. 7, pp. 1618--1629, 2020.

\bibitem{boyd_etal_ADMM_NOW_2011} S. Boyd, N. Parikh, E. Chu, B. Peleato, and J. Eckstein, "Distributed optimization and statistical learning via the alternating direction method of multipliers," Foundations and Trends in Machine Learning, Vol 3, pp 1-122, 2011.

\bibitem{monakhova_etal_UDN_OptExp_2021} K. Monakhova, V. Tran, G. Kuo, and L. Waller, "Untrained networks for compressive lensless photography," Opt. Express 29, 20913-20929, 2021.

\bibitem{Rubin1992} L. I. Rubin, S. Osher, and E. Fatemi, “Nonlinear total variation based noise removal algorithms,” Phys. D, vol. 60, pp. 259--268, 1992.

\bibitem{DIP} D. Ulyanov, A. Vedaldi, and V. Lempitsky, “Deep Image Prior,” Int. J. Comput. Vis., vol. 128, no. 7, pp. 1867--1888, 2020.

\bibitem{Sinha2017} A. Sinha, J. Lee, S. Li, and G. Barbastathis, “Lensless computational imaging through deep learning,” Optica, vol. 4, no. 9, pp. 1117--1125, 2017.

\bibitem{Monakhova2019OE} K. Monakhova, J. Yurtsever, G. Kuo, N. Antipa, K. Yanny, and L. Waller, “Learned reconstructions for practical mask-based lensless imaging,” Opt. Express, vol. 27, no. 20, pp. 28075--28090, 2019.

\bibitem{Barba2019} G. Barbastathis, A. Ozcan, and G. Situ, “On the use of deep learning for computational imaging,” Optica, vol. 6, no. 8, pp. 921--943, 2019.

\bibitem{Rego2022} J. Rego, H. Chen, S. Li, J. Gu, and S. Jayasuriya, “Deep Camera Obscura: An Image Restoration Pipeline for Pinhole Photography,” Opt. Express, vol. 30, no. 15, pp. 27214--27235, 2022.

\bibitem{Wang2004} Z. Wang, A. C. Bovik, H. R. Sheikh, and E. P. Simoncelli, “Image Quality Assessment: From Error Visibility to Structural Similarity,” IEEE Trans. Image Process., vol. 13, no. 4, pp. 600--612, Apr. 2004.

\bibitem{Zhang2018} R. Zhang, P. Isola, A. A. Efros, E. Shechtman, and O. Wang, “The Unreasonable Effectiveness of Deep Features as a Perceptual Metric,” in 2018 IEEE/CVF Conference on Computer Vision and Pattern Recognition, Jun. 2018, no. 1, pp. 586--595.



\end{thebibliography}

\vspace{11pt}
 \begin{IEEEbiography}[{\includegraphics[width=1in,height=1.25in,clip,keepaspectratio]{figs/photo_vieira}}]{Jos\'{e} Reinaldo Cunha Santos A. V. Silva Neto}
 received the B.S. in engineering and M.S. in computer science from the University of Brasilia, in 2019 and 2021 respectively. Currently, he is a PhD candidate on the computer science department of Osaka University. His research interests include computational photography, with a preference for lensless imaging, and deep learning techniques applied to computer vision.
 \end{IEEEbiography}
 \begin{IEEEbiography}[{\includegraphics[width=1in,height=1.25in,clip,keepaspectratio]{figs/photo_nakamura}}]{Tomoya Nakamura}
received the Ph.D. degree from Osaka University, in 2015. He is currently an Associate Professor with SANKEN, Osaka University. His research interests include computational imaging, computational photography, and applied optics. He is a member of the Optica and the IPSJ. He received several honors and awards, including the International Display Workshops (IDW 2017), the Best Paper Award, and the 4th International Workshop on Image Sensor and Systems (IWISS 2018), the Open Poster Session Award 1st Place.
 \end{IEEEbiography}
  \begin{IEEEbiography}[{\includegraphics[width=1in,height=1.25in,clip,keepaspectratio]{figs/photo_makihara}}]{Yasushi Makihara}
received the B.S., M.S., and Ph.D. degrees in engineering from Osaka University, in 2001, 2002, and 2005, respectively. He is currently a Professor with SANKEN (The Institute of Scientific and Industrial Research), Osaka University. His research interests include computer vision, pattern recognition, and image processing, including gait recognition, pedestrian detection, morphing, and temporal superresolution. He is a member of the IPSJ, the IEICE, the RSJ, and the JSME. He received several honors and awards, including the 2nd International Workshop on Biometrics and Forensics (IWBF 2014), the IAPR Best Paper Award, the 9th IAPR International Conference on Biometrics (ICB 2016), and the Honorable Mention Paper Award. He was the Program Co-Chair of the 4th Asian Conference on Pattern Recognition (ACPR 2017).
 \end{IEEEbiography}
 \begin{IEEEbiography}[{\includegraphics[width=1in,height=1.25in,clip,keepaspectratio]{figs/photo_yagi}}]{Yasushi Yagi}
(Senior Member, IEEE) received the Ph.D. degree from Osaka University, in 1991. In 1985, he joined the Product Development Laboratory, Mitsubishi Electric Corporation, where he was involved in robotics and inspections. He became a Research Associate, in 1990, a Lecturer, in 1993, an Associate Professor, in 1996, and a Professor, in 2003, with Osaka University, where he was the Director of SANKEN (The Institute of Scientific and Industrial Research), from 2012 to 2015. He was the Executive Vice President of Osaka University, from 2015 to 2019. His research interests include computer vision, pattern recognition, biometrics, human sensing, medical engineering, and robotics. He is a fellow of the IPSJ and a member of the IEICE and the RSJ. He is a member of the Editorial Board of the {\it International Journal of Computer Vision}. He is the Vice President of the Asian Federation of Computer Vision Societies. He was awarded the ACM VRST2003 Honorable Mention Award, the IEEE ROBIO2006 Finalist of the T. J. Tan Best Paper in Robotics, the IEEE ICRA2008 Finalist for the Best Vision Paper, the PSIVT2010 Best Paper Award, the MIRU2008 Nagao Award, the IEEE ICCP2013 Honorable Mention Award, the MVA2013 Best Poster Award, the IWBF2014 IAPR Best Paper Award, and the {\it IPSJ Transactions on Computer Vision and Applications} Outstanding Paper Award (2011 and 2013). International conferences for which he has served as the Chair include ROBIO2006 (PC), ACCV (2007PC and 2009GC), PSVIT2009 (FC), and ACPR (2011PC, 2013GC, 2021GC, and 2023GC). He has al
so served as an Editor for the IEEE ICRA Conference Editorial Board (2008 and 2011). He was the Editor-in-Chief of the {\it IPSJ Transactions on Computer Vision and Applications}.
 \end{IEEEbiography}
\vfill

\end{document}


