\section{Conclusion}
% Summarize the results, indicate possible future work directions
We proposed a radial-shape-preserving optimization scheme for coded masks, which can be used to systematically create radial masks with better overall frequency response when compared to the conventional radial masks.
We showed through simulations that the optimized radial mask was capable of extending the effective DOF of a lensless camera when compared to other types of coded masks.
We also built a prototype lensless camera and empirically validated the extended DOF capabilities of the radial mask in real scenarios.

\subsection{Limitations and Future Works}
Theoretically, the PSF of a radial-shaped coded mask is depth-independent, meaning that it is not affected by radial scaling as illustrated in Figure~\ref{fig:intro_complete}(c).
That would be the case if the effective area of the coded mask were larger than that of an image sensor.
In practice, however, the coded pattern is often smaller than the effective pixel area of an image sensor, with the edges of the coded pattern being light-shielded.
This shielding is necessary when the forward model is approximately described by a convolution because it suppresses the amount of information interception by the cropping function in sensing.
In this case, even though the reconstruction processing works well and the coded pattern is depth-independent, the complete PSF formed by the coded pattern and its edge is actually depth dependent to some extent.
In future work, we will address this issue by non-convolutional, e.g. matricial, modeling of the forward problem, and/or the application of more robust compressive reconstruction algorithms such as the primal-dual splitting method, and deep-unrolling method.

In addition, this work only addressed the amplitude-modulation-type coded mask.
When compared to phase masks, one limitation of amplitude masks is the lower light-use efficiency and optical cut-off frequency.
In theory, however, the essence of this work is that the PSF is radially-shaped, and the mask implementation method and its design should be free.
Therefore, it is necessary for future works to develop masks for extended-DOF lensless imaging using phase masks with high light-utilization efficiency.

Finally, the acyclic nature of our optimized radial mask, while being leveraged for an improved frequency response, can also be a limiting factor for reconstruction of objects near the edges of the camera's field of view. That is because the low- and high-frequency components are unevenly distributed around the mask's area, and a translation of the PSF to areas near the edges of the image sensor may remove specific frequency-rich areas. Which is not the case for periodic or symmetric radial masks. 

\red{For future research, an interesting direction to be considered is to combine radial and non-radial features on a single optimized mask. In Section A of the supplementary materials, we show that a non-optimized random coded mask achieves higher in-focus PSNR for its reconstruction when compared to our optimized radial mask. An area that may be especially appealing towards combining radial and non-radial features may be polar coordinate masks\cite{Chen2017, Don2017}, that are a more generic representation of our parameterized radial mask. A promising direction would be to use a parameterized polar coordinate mask and optimize it using a loss that somehow defines a tradeoff between extended DOF and in-focus reconstruction quality.}