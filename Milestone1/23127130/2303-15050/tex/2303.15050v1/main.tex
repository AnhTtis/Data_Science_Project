%\documentclass[prl,twocolumn,superscriptaddress]{revtex4-1}

\documentclass[ reprint,
superscriptaddress,
 amsmath,amssymb,
 aps, floatfix
]{revtex4-2}
\usepackage{graphicx}
\usepackage{dcolumn}
\usepackage{bm}
\usepackage{color}
\usepackage{amsmath}
\usepackage{amssymb}
\usepackage{braket}
\usepackage{amsfonts}
\usepackage{soul}
\usepackage{url}
\usepackage{array,tabularx}
\usepackage{graphics}
\usepackage{times}
\usepackage{subfigure}
\usepackage[export]{adjustbox}
\usepackage[mathscr]{euscript}
\newcommand\x{4.2}
\newcommand\y{2.8}
\newcommand\z{4.2}

\setlength{\belowcaptionskip}{-15pt}
\setlength{\abovecaptionskip}{8pt}

%\bibliographystyle{apsrev4-1}

\begin{document}

%\title{ Emerging anamolous hybrid skin-topological effect in one dimensional Non Hermitian SSH chain}
\title{Coexisting localized and extended \textit{in-gap} states in non Hermitian system}
%Hawking black hole horizon tunneling using feedback-based mechanical circuits
%Hawking black hole radiation tunneling using feedback-based phononic crystals
\author{Sayan Jana}\email{sayanjana@tauex.tau.ac.il}
\affiliation{School of Mechanical Engineering, Tel Aviv University, Tel Aviv 69978, Israel}
\author{Lea Sirota}\email{leabeilkin@tauex.tau.ac.il}
\affiliation{School of Mechanical Engineering, Tel Aviv University, Tel Aviv 69978, Israel}







\begin{abstract}
We study the interplay of two different non-Hermitian constitutive parameters: directional coupling and onsite gain/loss balancing along with topology in a coupled one dimensional (1D) lattice chain. This work demonstrates two distinct localization behaviors for the bulk and discrete \textit{in-gap} eigen modes in the system. We numerically show that the bulk and half of the \textit{in-gap} modes are strongly confined at the boundaries due to directional hopping and topological localization, respectively. Whereas, correlation between the constitutive non-Hermitian parameters can tune the energy and localization length of the remaining \textit{in-gap} modes and demonstrates two novel phenomena: one is when the directional strength and gain/loss match these states become completely extended. Second, coalescing of these modes into exceptional point (EP). We present explicit analytical solutions for the eigen function and energy of \textit{in-gap} modes, that exactly matches with our numerical results and uncover the origin of delocalization and emergence of EP.
\end{abstract}


\maketitle

The Hermiticity of a Hamiltonian, which ties the system with the fundamental physical reality, ensures energy and particle conservation. Interaction with the environment, on the other hand, can leads to non-Hermitian dynamics in non-equilibrium and open systems \cite{bender2007making,bender1999pt}, which behave substantially differently from Hermitian counterparts. Recently, non-Hermitian systems have received a lot of attention from the scientific community, leading to the discovery of many previously unknown principles and phenomena, such as unidirectional transmission \cite{feng2013experimental,guo2009observation}
 , non-reciprocal wave guiding \cite{turitsyna2017guided}, non Hermitian Hall effect \cite{hirsbrunner2019topology,yoshida2019non,yao2018non}, an exceptional Fermi arc \cite{zhou2018observation}, etc. Numerous systems, including  metamaterials \cite{hou2020topological,ghatak2020observation,schomerus2020nonreciprocal}, photonics \cite{midya2018non,el2019dawn}, electrical \cite{li2020critical,stegmaier2021topological,helbig2020generalized} and mechanical accoustic circuits \cite{gu2021controlling,zhang2021acoustic,wang2022non}, have been extensively employed to investigate these phases.\\
\indent{} Non-Hermiticity is usually engineered in two ways: through onsite gain/loss balancing and implementing asymmetrical directional hopping between two sites $t_{ab}\ne t_{ba}$. With the recent introduction of non-Hermiticity into the topological phases of matter, topological physics has lately advanced beyond the Bloch band theory \cite{bansil2016colloquium}. Particularly, in non-Hermitian lattice systems in presence of non reciprocal coupling, a fascinating phenomenon called the non-Hermitian skin effect (NHSE) \cite{okuma2020topological,lee2019anatomy} has been discovered.  It describes wave localization of all the eigen modes toward the boundaries in open boundary condition (OBC) . Furthermore, this effect also causes the difference in energy spectra between an open and a periodic chain \cite{helbig2020generalized,lee2019anatomy}.
Its emergence substantially disrupts the conventional Hermitian bulk-boundary correspondence (BBC) and opens up new research avenues for non-Hermitian topological phases of matter in the form of generalized BBC \cite{yao2018edge,yang2020non,yokomizo2019non,deng2019non}.\\ 
\indent{} An interaction between topology and NHSE, on the other hand, can produce some nontrivial effects, such as complete delocalization of topological modes \cite{zhu2021delocalization} and hybrid skin-topological effects \cite{zou2021observation,lee2019hybrid,li2022gain}. In the latter , the skin effect only affects topological edge modes, resulting in corner localization, while bulk modes remain unaffected. This type of unusual skin mode has recently been observed in two-dimensional systems in presence of non-reciprocal couplings \cite{zou2021observation,lee2019hybrid}, as well as implementing gain/loss in Haldane model \cite{li2022gain}. The question of whether qualitatively novel phases can be established by the simultaneous interplay of topology, gain/loss, and directional couplings will therefore be interesting to explore. 
\begin{figure}[tb]
\begin{tabular}{c}
\includegraphics[width=8cm]{fig1a2.eps}\\
\end{tabular}
\caption{ An illustration of the interface between two non-Hermitian SSH chains \textit{I} and \textit{II}. Two chains are coupled through inter chain coupling $t_{c}$ in maroon. Two sub lattics A and B are denoted by the black and bars, respectively. Intra-cell directional couplings $t_{L/R}^{I/II}$, are shown in blue and orange line. $t_{2}$ in black denotes the inter-cell coupling. Gain (i$\beta$) in chain \textit{I} and loss (-i$\beta$) in chain \textit{II} are added on both the sub lattices. $t_c$=0 signifies two dis-connected chains.}
 \label{f1}
\end{figure}
\begin{figure*}[tb]
\textbf{(a) Energy spectra} 
\begin{center}
\setlength{\tabcolsep}{-.0002pt}
\begin{tabular}{c c c c c c}
$\beta$=0.3 & $\beta$=0.5 & $\beta$=0.7 & $\beta$=0.97 & $\beta$=1.0 & $\beta$=2.0 
\\
\includegraphics[width=\y cm]{fig2a1.eps}
&  
\includegraphics[width=\y cm]{fig2a2.eps} 
&  
\includegraphics[width=\y cm]{fig2a3.eps} 
&  
\includegraphics[width=\y cm]{fig2a4.eps} 
&  
\includegraphics[width=\y cm]{fig2a5.eps}
&  
\includegraphics[width=\y cm]{fig2a6.eps}

\end{tabular}
\end{center} 
\textbf{(b) Eigen function distribution} 
\begin{center}
\setlength{\tabcolsep}{-.0002pt}
\begin{tabular}{c c c c c c}
$\beta$=0.3 & $\beta$=0.5 & $\beta$=0.7 & $\beta$=0.97 & $\beta$=1.0 & $\beta$=2.0 
\\
\includegraphics[width=\y cm]{fig2b1.eps}
&  
\includegraphics[width=\y cm]{fig2b2.eps} 
&  
\includegraphics[width=\y cm]{fig2b3.eps} 
&  
\includegraphics[width=\y cm]{fig2b4.eps} 
&  
\includegraphics[width=\y cm]{fig2b5.eps}
&  
\includegraphics[width=\y cm]{fig2b6.eps}

\end{tabular}
\end{center} 
\caption{(a) Energy eigen values of the coupled double chain non- Hermitian SSH Hamiltonian in (\ref{eq1}) as a function of gain/loss strength $\beta$. (b) Associated eigen function
distribution. We fixed $t$=1, $t_{2}$=1.3, $\gamma$=0.5 and $N$=80.}
\label{f2}
\end{figure*}

\indent{}In this Letter, we study how two of these factors—non reciprocal couplings and  on-site gain/loss potential—interact with topology. We discover the formation of a novel kind of anomalous skin effect in which the bulk and the discrete \textit{``in-gap''} modes exhibit distinct localization behaviors at the boundaries of an one dimension (1D) coupled chain network in OBC. In this configuration, the NHSE exists through the bulk modes which are still localized at the configuration's endpoints. While it is feasible to achieve two distinct phases of the \textit{in-gap} modes depending on the relative strength of the gain/loss potential and directional coupling. The first is to control the localization of half of the \textit{in-gap} modes, which get extended at critical non-Hermitian parameters. The second describes a transition between two distinct gaps with gap closure coalescing into a single exceptional point (EP). We also obtain analytical formulas for the eigen energy and eigen function of \textit{in-gap} modes and theoretically establish the transition from localize to extended state. We found that the degree of de-localization is independent of system size for large systems. The remaining modes are still restricted to the two open ends.  Moreover, the non-trivial topology of the bulk bands and \textit{in-gap} modes is characterized by the Winding number \textit{``W"}~\cite{yao2018edge} and biorthogonal polarization \textit{``P"}~\cite{kunst2018biorthogonal} .\\
\indent{} In our model, we consider a junction between two 1D non Hermitian Su-Schrieffer-Heeger (SSH)~\cite{guo2021exact,yao2018edge} chains with different non reciprocal coupling. The schematic of our model is depicted in Fig.~\ref{f1}. The chain is made up of two sublattices $A$ and $B$ within a unit cell. In both chains, we considered an equal number of unit cells (N).  The intra cell coupling strength is represented by $t_2$, and the directional non-reciprocal inter cell couplings in both chains \textit{I} and \textit{II}, are represented by $t_{L/R}^{I/II}$. We also added onsite gain and loss strengths $\pm i\beta_{}$ in chain \textit{I} and \textit{II}, respectively. The coupling $t_{c}$ inter connects the two chains at one end. We begin with the 1D tight-binding model Hamiltonian of the coupled chain  in real space, which has the following matrix form:

 \begin{equation} \label{eq1}
H_{s}= 
\begin{bmatrix}
  H_{I} &
  t_{c}\\ 
  t_{c} & H_{II}
\end{bmatrix}
\end{equation} 

where,

\begin{equation} 
H_{I/II}= 
\begin{bmatrix}
 i\beta_{I/II}& t_{L}^{I/II}& 0 &  0 & 0\\ 
t_{R}^{I/II}  &  i\beta_{I/II} & t_{2}  & 0 & 0\\ 
0  & t_{2} & i\beta_{I/II} & t_{L}^{I/II} & 0\\ 
0  &  0 & t_{R}^{I/II}  & i\beta_{I/II} & \dots \\
0  &  0 & 0  & \vdots & \ddots\\
\end{bmatrix} \ ,
\end{equation}


and $t_{L/R}^{I/II}$= $t\pm{\gamma}^{I/II} $, $\beta^{I/II}$ = $\pm{\beta}$. We choose $\gamma^{I/II}$ to be opposite to each other   $\gamma^{I/II}$= $\pm{\gamma}$.

\begin{figure*}[tb]
\textbf{(a) Energy spectra} 
\begin{center}
\setlength{\tabcolsep}{-0.2pt}
\begin{tabular}{c c c c}
$\beta$=0.1 & $\beta$=0.3 & $\beta$=$\gamma$ & $\beta$=0.7 
\\
\includegraphics[width=\x cm]{fig3a1.eps}
&  
\includegraphics[width=\x cm]{fig3a2.eps} 
&  
\includegraphics[width=\x cm]{fig3a3.eps} 
&  
\includegraphics[width=\x cm]{fig3a4.eps} 

\end{tabular}
\end{center} 
\textbf{(b) Eigen function distribution} 
\begin{center}
\setlength{\tabcolsep}{-0.2pt}
\begin{tabular}{c c c c}
$\beta$=0.1 & $\beta$=0.3 & $\beta$=$\gamma$ & $\beta$=0.7
\\
\includegraphics[width=\x cm]{fig3b1.eps}
&  
\includegraphics[width=\x cm]{fig3b2.eps} 
&  
\includegraphics[width=\x cm]{fig3b3.eps} 
&  
\includegraphics[width=\x cm]{fig3b4.eps} 

\end{tabular}
\end{center} 
\caption{(a) Energy eigen values of the coupled double chain non-Hermitian SSH Hamiltonian in (\ref{eq1}) as a function of gain/loss strength $\beta$. (b) Associated eigen function distribution. We fixed $t$=1, $t_{2}$=1, $\gamma$=0.5 and $N$=80.}
\label{f3}
\end{figure*}



\indent{} In the main text we investigate the scenario $t_c\neq0$ i.e., both the chains are coupled in one end through the coupling $t_{c}$. We choose $t_{c}$=$t_{2}$. For a detailed discussion on the characteristics of a single chain with directional couplings in presence of gain/loss, see the Supplemental Material~\cite{SM}(\ref{SA}). We set the strength of the $t_2$ connection such that both chains individually remains in the topological regime. Fig.~\ref{f2} illustrates both the eigen energy spectrum and the associated eigen function distribution obtained from the numerical diagonalization of (\ref{eq1}) are plotted for the range of the strength $\beta$, from small $\beta$ = 0.3 to the large  $\beta$ = 2, through the $\beta$ = $\gamma$. There are four bulk bands in all in this coupled double-chain configuration, as shown by the colors black, green, orange, and blue in Fig.~\ref{f2}(a). In addition to the bulk bands, there are four \textit{in-gap} modes that occur between the bulk bands. These modes are encountered in pairs ($E,-E$) and emerge from topological edge modes that existed in uncoupled chains. In the Supplemental Material~\cite{SM}, we demonstrate how the edge modes that were initially present in the uncoupled chains were transformed into \textit{in-gap} modes by controlling the inter chain coupling strength $t_{c}$. Two of these modes with energies $\pm i \beta$ are represented by magenta triangles. The remaining modes are shown in red circles. We denote them as $\pm \Delta$. As shown by the eigen function plots in Fig.~\ref{f2}(b), all bulk states with energy in the positive imaginary axis in blue are localized at the left end, whereas bulk states with energy in the negative imaginary axis are localized in the right ends, as shown in green. The two \textit{in-gap} modes with energies +$i\beta$, -$i\beta$ in cyan, are localized at the left and right ends, exhibiting typical topological edge state behavior~\cite{yao2018edge}.\\
\indent{} We now examine the evolution of the other two in-gap modes $\pm \Delta$, highlighted in red in Fig.~\ref{f2}(a). These two modes are  initially on the real energy axis (see. $\beta$=0.3). The energy difference $2\Delta$ steadily decreases as we raise the strength of $\beta$, and at the critical point where $\beta_{c}=0.97$, the energy difference between these modes becomes zero. Further away from the critical point ($ \beta > \beta_{c} $) their energy becomes imaginary. So, as a function of $\beta$, there exists a phase shift of the energy from the real to imaginary values, with a gap closing at $\beta_{c}$.  This crossover is particularly distinctive, and the crossing point is known as exceptional point (EP) \cite{keck2003unfolding}.\\
\indent{}  A intriguing signature of EP can be observed in the associated eigen function distribution, which is depicted in the inset of Fig.~\ref{f2}(b). These two modes are initially (see. $\beta$=0.3), confined equally to both open ends of the  chains \textit{I} and \textit{II}. The energy difference $2\Delta$ gradually decreases with the increase in $\beta$, and it correlates to a steady drop in localization length, until it becomes entirely delocalized at the point $\beta$=$\gamma$. Further increment (see. $\beta$=0.7) gradually localizes the eigen functions at the junction and becomes completely localized at $\beta$=$\beta_c$. Away from the criticality $\beta \gtrsim \beta _c$ (see. $\beta$=1.0) results in the eigen functions start decaying on either side of the junction: i.e. modes with energy +$\Delta$ clusters more at the junction's left end and -$\Delta$ clusters more at the junction's right end. This unique crossover in localization length and location across the junction shows a distinctive signature of the presence of an EP. Raising $\beta$ away from the critical point steadily increases the concentration of $\pm \Delta$ modes in chain \textit{I} and \textit{II}, respectively. These modes started localizing at the left and right open ends after reaching the large limit $\beta\gg\beta_c$ (see. $\beta$=2).
%In the large limit  , these modes again started getting localized at left and right open ends. 
%There are not any Hermitian, reciprocal, or topological analogs for all these emerging modes. 
These salient features in non-Hermitian system are very distinctive and the main outcomes of our work.\\
\indent{} In Supplemental Material~\cite{SM}(\ref{SB}), we demonstrate an analytical framework for obtaining the eigen energies and eigen functions, which will provide more insight into the interaction of $\gamma$ and $\beta$. In particular assuming that the system size is very large, i.e., N $\gg 1$ we obtain the expressions for the eigen energies of the \textit{in-gap} modes $\pm \Delta$ as,
\begin{equation} \label{eq3}
\Delta = \pm i \sqrt{t^2-t_{2}^2+\beta^2-\gamma^2}.
\end{equation}
From (\ref{eq3}), we observe that at the specific parameter values $t_2$=t and $\beta$=$\gamma$, the square root disappears which points to the formation of EP. Thus at $t_{2}$=t, two distinct events: delocalization of the eigen function and the formation of EP, should occur at the same point when the directionality and gain/loss matches i.e., $\beta$=$\gamma$. This particular prediction is confirmed numerically and illustrated in Fig.~\ref{f3}. The inset of Fig.~\ref{f3}(b) demonstrates the behavior of the eigen function of the modes $\pm \Delta$. In this particular case $t_{2}$=t all bulk states and the two \textit{in-gap} states with energies +$i\beta$ and -$i\beta$ exhibit similar characteristics as shown in Fig.~\ref{f2}.\\
\indent{} The exact expressions for the components of the right eigen function (see. Supplemental Material ~\cite{SM}, (\ref{sq12})) in chains \textit{I} and \textit{II} of the mode +$\Delta$ are as follows:
    \begin{align}\label{eq4}
    \begin{cases}
\psi_{nA}^{I}(\Delta)= \Bigg[ \frac{\mu_{1}^{}}{\mu_{2}^{I}}\left(\frac{\mu_{2}^{}}{t_{L}^{I}}e^{i \theta}\right)^n+ \left(\frac{\mu_{1}^{}}{t_{L}^{I}}e^{-i\theta}\right)^n\Bigg]\Phi_{A1}^{I},   \\
   \psi_{nB}^{I}(\Delta)=\frac{\mu_{1}^{}}{i t_{L}^{I}}\Bigg[ \left(\frac{\mu_{2}^{}}{t_{L}^{I}}e^{i \theta}\right)^n - \left(\frac{\mu_{1}^{}}{t_{L}^{I}}e^{-i\theta}\right)^n\Bigg]\Phi_{A1}^{I}, \\
    \psi_{nB}^{II}(\Delta)=\Bigg[ \frac{\mu_{1}^{}}{\mu_{2}^{}}\left(\frac{\mu_{1}^{}}{t_{R}^{II}}e^{i \theta}\right)^n+ \left(\frac{\mu_{2}^{}}{t_{R}^{II}}e^{-i\theta}\right)^n\Bigg]\Phi_{B1}^{},\\
    \psi_{nA}^{II}(\Delta)=\frac{\mu_{1}^{}}{i t_{R}^{}}\Bigg[ \left(\frac{\mu_{1}^{}}{t_{R}^{II}}e^{i \theta}\right)^n - \left(\frac{\mu_{2}^{}}{t_{R}^{II}}e^{-i\theta}\right)^n\Bigg]\Phi_{B1}^{} ,
    \end{cases}
    \end{align}
     
 where \textit{n} is the unit cell index,  
 $\tan{\theta}$=$\Delta$/$\sqrt{t^2 + \beta^2 - \gamma^2}$,  $\mu_{1/2}^{}$=$\sqrt{t^2+\beta^2-\gamma^2}\mp\beta$.\\
\indent{} The localization of the \textit{in-gap} modes at open boundaries $\pm i\beta$ and localization-delocalization transition of $\pm\Delta$ as demonstrated in both Fig.\ref{f2} and Fig.\ref{f3} can be explained using the analytical formulae (\ref{eq4}). Here, we concentrate especially on the scenario $t_2$=t illustrated in Fig.~\ref{f3}.
In Fig.~\ref{f4}(a), we illustrate the absolute value of energy difference $\delta E$=$|2\Delta|$ between the modes ($\pm\Delta$) as function of $\beta$ obtained numerically (black circles) and analytically (red triangle). Our analytical formula (\ref{eq3}) is confirmed by the numerical results. Next, we investigate the behavior of $\chi$=$|\phi_{B1}^{II}/\phi_{A1}^{I}|$, which determines the projection of any eigen function on the chain \textit{II}  (details in the Supplemental Material ~\cite{SM}, (\ref{sq13})). We plot $\chi(\Delta)$ as function of $\beta$ in Fig.~\ref{f4}(b) and observe that it has a constant value of `1' throughout the domain $\beta \leq \gamma$, which indicates that this mode is equally concentrated in both the chains.  In contrast, there is a sharp transition to `0' for $\beta>\gamma$ implying that the contribution is only coming from the chain  \textit{I}. Similarly, $\chi(-\Delta)$ as function of $\beta$ will show that the other mode with energy $-\Delta$ remain concentrated the in chain \textit{II} for  $\beta>\gamma$. \\
\indent{} In the region  $\beta \leq \gamma$ , $\theta$ is real and $|e^{\pm i \theta}|$=1,  $|\mu_{1}/t_{L}^{I}| \ll 1$. Hence, the localization length is solely dependent on the factor $|\mu_{2}/t_{L}^{I}|$. As the value of $\beta$ steadily grows, $\mu_{2}$ simultaneously increases and leads to gradual delocalization of the mode $+\Delta$. Now, at $\beta$=$\gamma$, $\mu_{2}$=$t_{L}^{I}$ and  $\mu_{1}$=$t_{R}^{I}$. Therefore far from the left end in chain \textit{I} i.e., $n \gg 1$,  $|\psi_{n(A/B)}^{I}(\Delta)|$ converges to $|t_{R}^{I}/t_{L}^{I}|$. This indicates that the degree of delocalization becomes independent of the system size at large N.  Similarly, we can obtain the convergence of $|\psi_{n(A/B)}^{II}(\Delta)|$ to $t_{L}^{II}/t_{R}^{II}|$ in chain \textit{II}. Finally, in the region  $\beta>\gamma$,  $\theta$ turns imaginary, and $|e^{i \theta}|<$1, $|e^{-i \theta}|>$1 and $|e^{-i \theta}(\mu_{1}/t_{L}^{I})|<1$. Hence, in the limit $n \gg 1$, we obtain $|\psi_{n(A/B)}^{I}(\Delta)|$ $\to$ 0, demonstrating the transition from total de-localization to a localized state in the open left end. The evolution of the mode -$\Delta$ can be verified in similar way.\\
\indent{} The other \textit{in-gap} mode with energy $E_{OBC}$=i$\beta$ satisfies $E_{OBC}^{I}$=0, $E_{OBC}^{II}$=2i$\beta$ and $|\chi(i\beta)|$ $\to$ 0~\cite{SM}. Which conclude that this specific mode is solely belongs to the chain $I$, and its localization at the left boundary can be verified from its eigen function in real space
\begin{align} \label{eq9} 
 \begin{cases}
   \psi_{nA}^{I}(i\beta)= \Bigg[\left( \frac{-t_{R}^{I}}{t_{2}} \right)^n\Bigg]\Phi_{A1}^{I},  \\
    \psi_{nB}^{I}(i\beta)=  0.
    \end{cases}
\end{align}
Similarly the state with energy $E_{OBC}$=-i$\beta$ satisfies $E_{OBC}^{I}$=-2i$\beta$, $E_{OBC}^{II}$=0 and $|1/\chi(-i\beta)|$ $\to$ 0~\cite{SM}. Hence, it is localized on the right end of the chain $II$ with real space eigen function that takes the form,
\begin{align} \label{eq10} %\tag{S12}
 \begin{cases}
   \psi_{nB}^{II}(-i\beta)= \Bigg[\left( \frac{-t_{L}^{II}}{t_{2}} \right)^n\Bigg]\Phi_{B1}^{II},  \\
    \psi_{nA}^{II}(-i\beta)=  0.
    \end{cases}
\end{align}
According to  (\ref{eq9}) and (\ref{eq10}), the localization of the \textit{in-gap} modes with energies +$i\beta$ and -$i\beta$ is independent of the $\beta$.

\begin{figure}[tb]
\begin{tabular}{c c c c}
  \textbf{(a)}  &   \textbf{(b)} \\
  \includegraphics[width=4.0cm, valign=c]{sj2.eps}  &    \includegraphics[width=4cm, valign=c]{sj.eps} \\
  & & \\
  \textbf{(c)}  &   \textbf{(d)} \\
  \includegraphics[width=4.0cm, valign=c]{Fig3d_2.eps}  &    \includegraphics[width=4.0cm, valign=c]{Fig3d.eps} \\
  \end{tabular}
\caption{  (a) The absolute value of energy difference between two \textit{in-gap} states $\pm\Delta$ as a function of $\beta$ from numerical results (black circles) and analytical results (red triangles). (b) Variation of $\chi(\Delta)$ with $\beta$. (c) Winding numbers $W_{I,II}$ for the bulk of the chains \textit{I} and \textit{II} are shown in blue and green. (d) Bi orthogonal polarization $P$ for the \textit{in-gap} modes $\pm i \beta$ and $\pm \Delta$ shown in red and blue. In all the plots we fixed $t$=1, $t_{2}$=1, $\gamma$=0.5.}
 \label{f4}
\end{figure}

\indent{}Now, for bulk states with $\operatorname{Im}(E)$$>0$, $|\chi|$ $\to$ 0 ~\cite{SM} demonstrating that these specific bulk states remain in chain $I$. These specific states also satisfy  $|\lambda_{1}^{I}|\approx|\lambda_{2}^{I}|<1$. Hence, in the limit $n\gg 1$, $|\psi_{n(A/B)}^{I}|$ $\to$ 0 implies localization at the open left end, as seen in Fig.~\ref{f2}(b). Similarly, the localization of the bulk modes $\operatorname{Im}(E)<0$ at the open right end can be stems from the conditions $|1/\chi|$ $\to$ 0 and $|\lambda_{1}^{II}|\approx|\lambda_{2}^{II}|<1$. Our analytical formulation therefore agrees with the numerical results.\\
\indent{} Finally, we discuss the topological properties of our system. The Winding number $W$ calculated using the associated generalized Brillouin zone (GBZ)~\cite{SM} for the bulk states of the chains \textit{I} and \textit{II} are illustrated as a function of $\beta$ in Fig.~\ref{f4}(c). The quantize value $W_{I/II}=\pm 1$ reveals the non trivial topology of the bulk bands. The difference 2$|W_{I}$-$W_{II}|$ correctly predicts the total number of  \textit{in-gap} modes that exist in the system, satisfying the generalized BBC~\cite{yao2018edge}. As a potential marker for topology of the \textit{in-gap} modes $\bigl\{\pm i\beta,\pm\Delta\bigl\}$ we calculate bi orthogonal polarization \textit{P}~\cite{kunst2018biorthogonal} given as,

\begin{equation}\label{eq11}
    P=\lim_{N\to \inf} \biggl \langle \Psi_{L} \bigg| \frac{\sum_{n=1}^N (N-n)\Pi_{n}}{N}  \bigg| \Psi_{R} \biggr \rangle
\end{equation}\\
where $\Pi_{n}$= $\sum_{m}\Bigl(c_{n,m}^{I\dagger}\ket{0}\bra{0}c_{n,m}^{I} + c_{n,m}^{II\dagger}\ket{0}\bra{0}c_{n,m}^{II}\Bigr) $ is the projection operator onto unit cell \textit{n}, and \textit{m} is the sub-lattice index. The right eigen state $\ket{\Psi_{R}}$ and left eigen state $\ket{\Psi_{L}}$ of the \textit{in-gap} modes can be obtained exactly as

\begin{align} \label{eq12} %\tag{S12}
 \begin{cases}
   \ket{\Psi_{R}}= \mathrm{N_{R}}\sum_{n=1}^{N}\sum_{m}^{}\Bigl(\psi_{n,m}^{I,R}{c_{n,m}^{I\dagger}}^{}+\psi_{n,m}^{II,R}{c_{n,m}^{II\dagger}}^{}\Bigr) \ket{0} \\
    \ket{\Psi_{L}}=\mathrm{N_{L}}\sum_{n=1}^{N}\sum_{m}^{}\Bigl(\psi_{n,m}^{I,L}{c_{n,m}^{I\dagger}}^{}+\psi_{n,m}^{II,L}{c_{n,m}^{II\dagger}}^{}\Bigr) \ket{0},
    \end{cases}
\end{align}

where $\mathrm{N_{R,L}}$ is the normalization factor and $\psi^{R}$ and $\psi^{L}$ denotes the left and right eigen vectors respectively, as obtained from (\ref{eq1}). Away from EP the eigen vectors with eigen energies $\bigl\{\epsilon_{i},\epsilon_{j}\bigl\}$  satisfies the following bi orthogonal normalization $\braket{\psi^{L}_{i}|\psi^{R}_{j}}$=$\delta_{ij}$. At the EP, however, a single eigen vector \cite{keck2003unfolding} exists at the crossing ($\epsilon_{i}$=$\epsilon_{j}$) spans the eigen space and satisfies $\braket{\psi^{R}_{ij}|\psi^{R}_{ij}}$=1. At EP we rewrite the form of  biorthogonal polarization as  $P_{EP}$=$\lim_{N \to \inf}$ $\biggl \langle \Psi_{R} \bigg| \frac{\sum_{n=1}^N (N-n)\Pi_{n}}{N}  \bigg| \Psi_{R} \biggr \rangle$. In accordance with the biorthogonal normalization requirement, \textit{P} is quantize for every topological boundary state regardless of the details, although the biorthogonal density $\Pi_{n}$ is typically complex valued~\cite{kunst2018biorthogonal}. Fig.~\ref{f4}(d) demonstrates \textit{P} as function of $\beta$. We observe that, in agreement with the Ref.\cite{kunst2018biorthogonal}, \textit{P} always quantize to 1, for both the localized modes $\pm i \beta$. The polarization \textit{P} for the other modes $\pm \Delta$ shows an unique indication of the gap closing at EP, as illustrated in Fig.~\ref{f3}(a). When the energy gap $2\Delta$ is real, it remains at 0 for both modes. Nevertheless, at the EP, it rises to $\frac{1}{2}$ before jumping back to 0 when the gap becomes imaginary. Here the \textit{half integer} value of the \textit{P} clearly shows the signature of EP.\\
\indent{} To conclude, in this Letter we study the interplay of both directional coupling and onsite gain/loss strength in non- Hermitian system. Our model features a coupled chain network of two non Hermitian SSH chains with opposite couplings in OBC. We systematically show that the bulk and ``\textit{in-gap}" modes at the boundaries exhibit two very different localization behaviors. The gain, loss strength also controls the energy and localization length of half of the \textit{in-gap} modes and causes them to become totally extended when the strength of gain/loss and directionality coincides. Additionally we show that depending on the relative strength of these two constituent non-Hermitian factors the mode can coalesce to each other and forms EP. An analytical frame work is established, revealing the origin of these two distinct phenomena. Analytically, we  show that for a large system, the degree of delocalization is independent of system size and only depends on the relative strength of the  directional couplings. The remaining \textit{in-gap} modes are always localized. We calculate winding number \textit{W} as the bulk topological invariant, confirms the total number of the \textit{in-gap} modes that exist in the system. For these \textit{in-gap} modes, biorthogonal polarization \textit{P} is calculated, which shows clearly the fingerprint of EP.
%Increasing the strength from the critical value causes localization again, but on either side of the boundaries, depending on the energy of the modes.
 In terms of practical implementation, our model can be realized employing topo-electric circuits ~\cite{lee2018topolectrical,helbig2020generalized, hofmann2019chiral}, where non-local voltage response and impedance measurement can be utilized to detect \textit{in-gap} modes. The \textit{in-gap} extended modes have the potential to be highly beneficial in broad-area and efficient laser emission ~\cite{longhi2018non} in mirrored resonators as a possible application.
 


\bibliography{paper}

\clearpage

\begin{widetext}

\section{Supplemental Material}


\subsection{DETAILS FOR WINDING NUMBER SOLUTIONS IN THE CASE OF A NON-HERMITIAN SSH CHAIN WITH ONSITE GAIN/LOSS POTENTIAL.} \label{SA}

The Bloch Hamiltonian for the chain \textit{I} reads,

\begin{equation} \label{sqA1}
H_{I}(k)= 
\begin{bmatrix}
  i\beta &
  t_{L}^{I}+t_{2}e^{ik}\\ 
  t_{R}^{I}+t_{2}e^{-ik} & i\beta
\end{bmatrix}
\tag{SA1}
\end{equation} 
 with the corresponding eigen energies,

 \begin{equation} \label{sqA2}
 \epsilon_{1,2}=i\beta\mp\sqrt{t_{2}^2+t_{L}^{I}t_{R}^{I} + e^{ik}t_{2}t_{R}^{I}+ e^{-ik}t_{2}t_{L}^{I}},
 \tag{SA2}
\end{equation} 

 right eigen vectors,

\begin{equation} \label{sqA3}
 \ket{\eta_{1,2}^{R}}=\bigl\{ \mp\frac{\sqrt{t_{2}^2+t_{L}^{I}t_{R}^{I} + e^{ik}t_{2}t_{R}^{I}+ e^{-ik}t_{2}t_{L}^{I}}}{ e^{-ik}t_{2}+t_{R}^{I}},1\bigl\}
 \tag{SA3}
\end{equation} 

and associated left eigen vectors,

\begin{equation} \label{sqA4}
 \ket{\eta_{1,2}^{L}}=\bigl\{ \mp\frac{\sqrt{t_{2}^2+t_{L}^{I}t_{R}^{I} + e^{ik}t_{2}t_{L}^{I}+ e^{-ik}t_{2}t_{R}^{I}}}{ e^{-ik}t_{2}+t_{L}^{I}},1\bigl\}.
 \tag{SA4}
\end{equation} 

The associated right and left eigen vectors corresponding to the eigen values $\bigl\{\epsilon_{i},\epsilon_{j}\bigl\}$ satisfies the biorthogonal normalization $\braket{\eta_{i}^{L}|\eta_{j}^{R}}$=$\delta_{ij}$ and the normalization constant given by $N_{i}$=$\frac{1}{\sqrt{\braket{\eta_{i}^{L}|\eta_{i}^{R}}}}$.

The winding number $W$ for a given band  $\epsilon_{i}$ can be calculated using the following relationship. 
\begin{equation} \label{sqA5}
W_{i}=\frac{1}{\pi}\int_{}^{} \bra{\eta_{i}^{L}}\partial_{k} \ket{\eta_{i}^{R}} \,dk
\tag{SA5}.
\end{equation} 

Substituting (\ref{sqA3}), (\ref{sqA4}) in (\ref{sqA5}) the Bloch form for the  $W_{i}$ reads,

\begin{equation} \label{sqA6}
W_{i}=\frac{1}{\pi}\int_{k}^{}\frac{t_{2}(t_{2}+t\cos{k}+i\gamma\sin{k})}{t^2+t_{2}^2-\gamma^{2}+2tt_{2}\cos{k}+2it_{2}\sin{k}} \,dk
\tag{SA6}.
\end{equation} 

 However, in the presence of NHSE, the Brillouin zone (BZ) $e^{ik}$ defined in a unit circle by the momentum wave vector $k$ is insufficient to predict the system's bulk topology. This discrepancy is overcome by introducing the concept of generalized Brillouin zone (GBZ)~\cite{yao2018edge}via the following substitution ($\lambda \rightarrow e
^{ik}$). The GBZ is defined by the complex plane trajectory of $\lambda$.

Substituting ($\lambda \rightarrow e^{ik}$) in (\ref{sqA6}), the form of $W$ finally reads,

\begin{equation} \label{sqA7}
W_{i}=\int_{\lambda}^{} \frac{t_{2}}{2\pi i} \Big[ \frac{1}{-t-t_{2}\lambda+\gamma}-\frac{1}{(t+\gamma)\lambda-t_{2}\lambda^2}\Big]\,d\lambda.
\tag{SA7}
\end{equation} 

From (\ref{sqA7}) it is evident that the presence of gain/loss does not contribute to winding number \textit{``W''} as the eigen functions $\ket{\eta_{1,2}^{R}}$, $\ket{\eta_{1,2}^{L}}$ does not depend on the diagonal $i \beta$. So the presence of gain/loss does not alternate the characteristics of a single chain when decoupled i.e. $t_{c}$=0 . It is demonstrated in Fig.~\ref{sf1} considering OBC in the presence of $\beta$. We plot all the eigen functions of chain \textit{I} in Fig.~\ref{sf1}(a), which shows the localization of all the states at the left end. The localization of topological modes is shown in the inset of Fig.~\ref{sf1}(a). Because non-reciprocity is reversed in chain \textit{II}, we discovered that all eigen functions are concentrated in the right open end, as shown in Fig.~\ref{sf1}(b) . This apparent localization of all the eigen modes at one end of the chain is refers the NHSE. We plot the band spectrum as a function of the parameter $t_{2}$ in Fig.~\ref{sf1}(c).The gray and red colors illustrate the bulk and edge states, shifted by $\beta$, respectively. Fig.~\ref{sf1}(d) corresponds to the \textit{W} plot which shows that the value of \textit{W} remains quantized to $1$ as edge modes appear, and that the jump in the value of \textit{W} for certain values of $t_{2}$ corresponds exactly to the merging of the edge modes to the bulk modes, satisfying the generalized (BBC)~\cite{yao2018edge}.

\begin{figure*}[tb]
\setcounter{figure}{0}
\renewcommand{\figurename}{Fig.}
\renewcommand{\thefigure}{S\arabic{figure}}
%\textbf{(a) Eigen energy spectra} 
\begin{center}
\setlength{\tabcolsep}{-.00005pt}
\begin{tabular}{c c c c }
(a) & (b) & (c) & (d) 
\\
\includegraphics[width=\z cm]{fig1b.eps}
&  
\includegraphics[width=\z cm]{fig1c.eps} 
&  
\includegraphics[width=\z cm]{fig1d.eps} 
&  
\includegraphics[width=\z cm]{fig1e.eps} 
\end{tabular}
\end{center} 
\caption{ (a) Eigen function localization in the left end of chain $I$ in OBC (b) Eigen function localization in the right end of chain $II$ in OBC. Inset in (a) and (b) shows the distribution of edge modes in Red. (c) Numerical spectra $E$ of  single chain with varying $t_{2}$ in OBC. (d) Winding number W as function of $t_{2}$ . In all the figures we choose t=1, $\gamma$=0.5, $\beta$=0.5 and N=80.}
\label{sf1}
\end{figure*}

\vspace{1.2 cm}

\subsection{DETAILS FOR ANALYTICAL SOLUTIONS OF EIGEN VALUES AND EIGEN FUNCTIONS OF COUPLED NON HERMITIAN SSH MODEL.} \label{SB}

The details for the analytical solution of the 1D coupled non-Hermitian SSH model, with its Hamiltonian provided by (\ref{eq1}) in the main text, is given here.

 We employ The real-space Schrödinger equation  $H\ket{\Psi}$=$E_{OBC}\ket{\Psi}$  to determine the eigenvalues  $E_{OBC}$ of the open boundary chain. With $\ket{\Psi}$=
 $\left( \psi_{1A}^{I},\psi_{1B}^{I},\dotsc,\psi_{N_{}A}^{I},\psi_{N_{}B}^{I},\psi_{N_{}A}^{II},\psi_{N_{}B}^{II},\dotsc,
 \psi_{1A}^{II},\psi_{1B}^{II} \right)$, we obtain the following recurrance relation for the eigen function within the bulk of chain \textit{I},
 \begin{align} \label{sq1}
 \begin{cases}
    %\begin{align}
     E_{OBC}^{I} \psi_{n+1A}^{I}=t_{L}^{I}\psi_{n+1B}^{I}+t_{2}\psi_{n B}^{I}  \\
     E_{OBC}^{I}\psi_{nB}^{I}=t_{R}^{I}\psi_{nA}^{I}+t_{2}\psi_{n+1A}^{I}.
    %\end{align}
    \end{cases}
    \tag{SB1}
\end{align}
Similarly for the chain \textit{II} we obtain,
\begin{align} \label{sq2}
 \begin{cases}
    %\begin{align}
     E_{OBC}^{II} \psi_{nA}^{II}=t_{L}^{II}\psi_{nB}^{II}+t_{2}\psi_{n+1B}^{II}  \\
     E_{OBC}^{II}\psi_{n+1B}^{II}=t_{R}^{II}\psi_{n+1A}^{I}+t_{2}\psi_{nA}^{II}.
    %\end{align}
    \end{cases}
    \tag{SB2}
\end{align}
where $E_{OBC}^{I}$= $E_{OBC}^{}$-$i\beta$, $E_{OBC}^{II}$= $E_{OBC}^{}$+$i\beta$ and $n$ is the unit cell index . We can consider an ansatz for the eigen functions of both chains as a linear combination~\cite{yao2018edge,guo2021exact} in accordance with the theory of linear difference equations:

\begin{equation}\label{sq3}
    \begin{bmatrix} 
      \psi_{nA}^{I/II} \\ 
      \psi_{nB}^{I/II} 
     \end{bmatrix}= (\lambda_{1}^{I/II})^{n} \begin{bmatrix} 
                \phi_{A1}^{I/II} \\ 
                \phi_{B1}^{I/II} 
               \end{bmatrix} + (\lambda_{2}^{I/II})^{n} \begin{bmatrix}
                \phi_{A2}^{I/II} \\ 
                \phi_{B2}^{I/II} 
               \end{bmatrix} ; 
               \tag{SB3}
\end{equation}

 Substituting (\ref{sq3}) in (\ref{sq1}) and (\ref{sq2}) we got
  \begin{equation}\label{sq4}
 \begin{array}{l}
    \Phi_{Aj}^{I}=\frac{E_{OBC}^{I}\Phi_{Bj}^{I}}{t_{R}^{I}+t_{2}\lambda_{j}^{I}}=\frac{\Big[t_{L}^{I}+\frac{t_{2}}{\lambda_{j}^{I}}\Big]\Phi_{Bj}^{I}}{E_{OBC}^{I}}
     
     \\

      \Phi_{Bj}^{II}=\frac{E_{OBC}^{II}\Phi_{Aj}^{II}}{t_{L}^{II}+t_{2}\lambda_{j}^{II}}=\frac{\Big[t_{R}^{II}+\frac{t_{2}}{\lambda_{j}^{II}}\Big]\Phi_{Aj}^{II}}{E_{OBC}^{II}}

      
     \end{array}
     \tag{SB4}
 \end{equation}

 with j=1,2. From (\ref{sq4})  we got the following expressions for $\lambda_{1/2}^{I/II}$,

 \begin{equation}\label{sq5}
 \begin{array}{l}
     
     \lambda_{1/2}^{I}=\frac{{E_{OBC}^{I}}^2-t_{R}^{I}t_{L}^{I}-t_{2}^2\pm\sqrt{\left( {E_{OBC}^{I}}^2-t_{R}^{I}t_{L}^{I}-t_{2}^2 \right){}^2-4 t_2^2t_{R}^{I}t_{L}^{I}}}{2 t_2 t_{L}^{I}}
      \\
    \lambda_{1/2}^{II}=\frac{{E_{OBC}^{II}}^2-t_{R}^{II}t_{L}^{II}-t_{2}^2\pm\sqrt{\left( {E_{OBC}^{II}}^2-t_{R}^{II}t_{L}^{II}-t_{2}^2 \right){}^2-4 t_2^2t_{R}^{II}t_{L}^{II}}}{2 t_2 t_{R}^{II}} 
\end{array}
\tag{SB5}
 \end{equation}


 $\lambda_{1,2}^{I,II}$ denotes the generalized Brillouin zone (GBZ)~\cite{yao2018edge} of the bulk bands. In Fig.~\ref{sf2}, we plot the GBZs for all bands with different values of $\beta$. The GBZ, exhibited by the closed curves in green and black color, corresponds to the bands with $\operatorname{Im}(E)>0$, while  red and black curves correspond to the bands with $\operatorname{Im}(E)<0$.

     \begin{figure*}[tb]
\setcounter{figure}{0}
\renewcommand{\figurename}{Fig.}
\renewcommand{\thefigure}{S2}
%\textbf{(a) Eigen energy spectra} 
\begin{center}
\setlength{\tabcolsep}{-.0005pt}
\begin{tabular}{c c c}
$\beta$=0.3 & $\beta$=0.5 & $\beta$=0.7 
\\
\includegraphics[width=5.0 cm]{Figsp2b.eps}
&  
\includegraphics[width=5.0 cm]{Figsp2c.eps} 
&  
\includegraphics[width=5.0 cm]{Figsp2d.eps} 
\end{tabular}
\end{center} 
\caption{GBZ for the  bulk states denoted by $\lambda_{1,2}^{I}$, $\lambda_{1,2}^{II}$ . The closed curves in light green, black, blue, and red correspond to the GBZ of the bulk bands, as shown in Fig.\ref{f3}(a) of the Main text.  In all the figures we choose t=1, $t_{2}$=1, $\gamma$=0.5, and N=80.}
\label{sf2}
\end{figure*}

 As illustrated in Fig.~\ref{f1}(a) in the main text, the boundary equations for the eigen functions in the real space  can be expressed as follows:

 \begin{align} \label{sq6}
 \begin{cases}
    %\begin{align}
     E_{OBC}^{I} \psi_{1A}^{I}=t_{L}^{I}\psi_{1B}^{I},  \\
     E_{OBC}^{I}\psi_{NB}^{I}=t_{R}^{I}\psi_{NB}^{I}+t_{2}\psi_{NA}^{II},      \\
     E_{OBC}^{II}\psi_{NA}^{II}=t_{L}^{II}\psi_{NB}^{II}+t_{2}\psi_{NB}^{I},      \\
     E_{OBC}^{II}\psi_{1B}^{II}=t_{R}^{II}\psi_{1A}^{II}.
    %\end{align}
    \end{cases}
    \tag{SB6}
\end{align}


The ansatz in (\ref{sq3}) should satisfy the boundary conditions (\ref{sq6}). Now with the substitution of (\ref{sq4}),  (\ref{sq6}) is written in terms of the coefficients $\bigl\{  \Phi_{(A/B)j}^{I/II} \bigl\}$ with (j=1,2). Additionally, we can obtain the coupled equations in (\ref{sq6}) including only the set $\Theta$ as $H_{B}\Theta$=0, where  $\Theta$= $\left[\phi_{B1}^{I},\phi_{B2}^{I},\phi_{A1}^{II},\phi_{A2}^{II}\right]$. $H_{B}$ has the following matrix form:,

  \begin{equation}\label{sq7}
 H_{B}=\begin{bmatrix}
t_{2} & t_{2} & 0 & 0 \\
t_{2} {\lambda_{1}^{I}}^{N+1}\eta_{1}^{I} & t_{2} {\lambda_{2}^{I}}^{N+1}\eta_{2}^{I} & -t_{2} {\lambda_{1}^{II}}^{N} & -t_{2} {\lambda_{2}^{II}}^{N} \\
 -t_{2} {\lambda_{1}^{I}}^{N} &  -t_{2} {\lambda_{2}^{I}}^{N} & t_{2} {\lambda_{1}^{II}}^{N+1}\eta_{1}^{II} & t_{2} {\lambda_{2}^{II}}^{N+1}\eta_{2}^{II} \\
0 & 0 & t_{2} & t_{2}
\end{bmatrix}
\tag{SB7}
\end{equation}
 where $\eta_{j}^{I}$=$\frac{E_{OBC}^{I}}{t_{R}^{I}+t_{2}\lambda_{j}^{I}}$ and $\eta_{j}^{II}$=$\frac{E_{OBC}^{II}}{t_{L}^{II}+t_{2}\lambda_{j}^{II}}$.
 
There exists a nontrivial solution corresponding to coefficients $\Theta$ having nonzero values, as given by $det{H_{B}}$=0. This results to the following characteristic equation,

\begin{equation} \label{sq8}
\begin{split}
&{\lambda_{1}^{I}}^{N}\left({ {\lambda_{2}^{II}}^{N} -\lambda_{1}^{II}}^{N}\right)+{\lambda_{2}^{I}}^{N}\left({\lambda_{1}^{II}}^{N} - {\lambda_{2}^{II}}^{N} \right)
    \\&+\eta_{1}^{I}{\lambda_{1}^{I}}^{N+1}\left({\eta_{1}^{II}\lambda_{1}^{II}}^{N+1} - \eta_{2}^{II}{\lambda_{2}^{II}}^{N+1} \right)
    \\&+\eta_{2}^{I}{\lambda_{2}^{I}}^{N+1}\left({\eta_{2}^{II}\lambda_{2}^{II}}^{N+1} - \eta_{2}^{II}{\lambda_{1}^{II}}^{N+1} \right)=0. 
    \end{split}
    \tag{SB8}
\end{equation}

 When $\lambda_{1/2}^{I/II}(E_{OBC})$ in (\ref{sq5}) is substituted, (\ref{sq8}) becomes the function of N and $E_{OBC}$. The energy eigenvalues in of the finite coupled chain in OBC are thus provided by the set of solutions of the polynomial equation (\ref{sq8}) for a fixed chain length N on both sides. Now assuming that the system size is in the thermodynamic limit, i.e., N $\to$  $\infty$ and taking into account that eigen values with only real or imaginary value satsify $|\lambda_{1}^{I/II}|<|\lambda_{2}^{I/II}|$, and we can approximate (\ref{sq8}) by ignoring all the other terms and including only  $({\lambda_{2}^{I}\lambda_{2}^{II}})^{N} (\eta_{1}^{I}\eta_{2}^{II}\lambda_{c}^{I}\lambda_{2}^{II}-1)$. The analytical equation for the \textit{in-gap} states with eigenvalues $\pm \Delta$ is obtained by substituting the coefficient  $(\eta_{1}^{I}\eta_{2}^{II}\lambda_{2}^{I}\lambda_{2}^{II}-1)=0$, which results in
 
\begin{equation} \label{sq9}
\Delta= \pm i \sqrt{t^2-t_{2}^2+\beta^2-\gamma^2}
\tag{SB9}
\end{equation}
The other two \textit{in-gap} states of energies +$i\beta$ and -$i\beta$  can be obtained from the constraint  $\eta_{1}^{I}\eta_{2}^{II}=0$ or  $\eta_{2}^{I}\eta_{1}^{II}=0$.

Using (\ref{sq1}) and (\ref{sq6}), we can now derive the following relations for the chain $I$.

\begin{align} \label{sq10}
 \begin{cases}
    %\begin{align}
     \Phi_{B1}^{I}=\frac{\Phi_{A1}^{I}}{\eta_{1}^{I}},  \\
     \Phi_{B2}^{I}= \Phi_{B1}^{I}\frac{ \lambda_{1}^{I}}{ \lambda_{2}^{I}}\frac{E_{OBC}^{I}\eta_{1}^{I}-t_{L}^{I}}{-E_{OBC}^{I}\eta_{2}^{I}+t_{L}^{I}} ,      \\
      \Phi_{A2}^{I}=\Phi_{B2}^{I}\eta_{2}^{I}.
    %\end{align}
    \end{cases}
     \tag{SB10}
\end{align}

Similarly, using (\ref{sq2}) and (\ref{sq6}), we can now derive the following relations for the chain $II$.

\begin{align} \label{sq11}  \tag{SB11}
 \begin{cases}
    %\begin{align}
     \Phi_{A1}^{II}=\frac{\Phi_{B1}^{II}}{\eta_{1}^{II}},  \\
     \Phi_{A2}^{II}= \Phi_{A1}^{II}\frac{ \lambda_{1}^{II}}{ \lambda_{2}^{II}}\frac{E_{OBC}^{II}\eta_{1}^{II}-t_{R}^{II}}{-E_{OBC}^{II}\eta_{2}^{II}+t_{R}^{II}} ,      \\
      \Phi_{B2}^{I}=\Phi_{A2}^{II}\eta_{2}^{II}.
    %\end{align}
    \end{cases}
    \end{align}


    
The following expression for the eigen function with eigen energy $E_{OBC}$ in real space is obtained by substituting (\ref{sq10}) and (\ref{sq11}) in (\ref{sq3}).

\begin{widetext}
\begin{align}\label{sq12} \tag{SB12}
%\begin{array}{l}
\psi_{nA}^{I}&= \Bigg[\frac{\left({E_{OBC}^{I}}^2 +t_{L}^{I}t_{R}^{I} -t_{2}^2 +\delta_{I} \right) \left( \frac{{E_{OBC}^{I}}^2 -t_{L}^{I}t_{R}^{I} -t_{2}^2 -\delta_{I}}{t_2 t_{L}^{I}}\right)^n}{{-E_{OBC}^{I}}^2 -t_{L}^{I}t_{R}^{I} +t_{2}^2 +\delta_{I}}+\left(\frac{{E_{OBC}^{I}}^2 -t_{L}^{I}t_{R}^{I} -t_{2}^2 +\delta_{I}}{t_2 t_{L}^{I}}\right)^n\Bigg]\frac{\Phi_{A1}^{I}}{2^{n}}  \nonumber\\
 \psi_{nB}^{I}&= \frac{\Phi_{A1}^{I}E_{OBC}^{I}t_{R}^{I}}{2^{n-1}} \Bigg[\frac{\left( \frac{{E_{OBC}^{I}}^2 -t_{L}^{I}t_{R}^{I} -t_{2}^2 +\delta_{I}}{t_2 t_{L}^{I}}\right)^n + \left( \frac{{E_{OBC}^{I}}^2 -t_{L}^{I}t_{R}^{I} -t_{2}^2 -\delta_{I}}{t_2 t_{L}^{I}}\right)^n}{{E_{OBC}^{I}}^2 +t_{L}^{I}t_{R}^{I} -t_{2}^2 +\delta_{I}}\Bigg] \nonumber \\
 \psi_{nB}^{II}&= \Bigg[\frac{\left({E_{OBC}^{II}}^2 -t_{L}^{II}t_{R}^{II} -t_{2}^2 +\delta_{II} \right) \left( \frac{{E_{OBC}^{II}}^2 -t_{L}^{II}t_{R}^{II} -t_{2}^2 -\delta_{II}}{t_2 t_{R}^{II}}\right)^n}{{-E_{OBC}^{II}}^2 -t_{L}^{II}t_{R}^{II} +t_{2}^2 +\delta_{II}} + \left(\frac{{E_{OBC}^{II}}^2 -t_{L}^{II}t_{R}^{II} -t_{2}^2 +\delta_{II}}{t_2 t_{R}^{II}}\right)^n\Bigg]\frac{\Phi_{B1}^{II}}{2^{n}}  \nonumber\\
 \psi_{nA}^{II}&= \frac{\Phi_{B1}^{II}E_{OBC}^{I}t_{L}^{II}}{2^{n-1}} \Bigg[\frac{\left( \frac{{E_{OBC}^{II}}^2 -t_{L}^{II}t_{R}^{II} -t_{2}^2 +\delta_{II}}{t_2 t_{R}^{II}}\right)^n + \left( \frac{{E_{OBC}^{II}}^2 -t_{L}^{II}t_{R}^{II} -t_{2}^2 -\delta_{II}}{t_2 t_{R}^{II}}\right)^n}{{E_{OBC}^{II}}^2 +t_{L}^{II}t_{R}^{II} -t_{2}^2 +\delta_{II}}\Bigg] \nonumber
 %\nonumber\\
 %&-\frac{\left( \frac{{E_{OBC}^{I}}^2 -t_{L}^{I}t_{R}^{I} -t_{2}^2 -\sqrt{\left(-{E_{OBC}^{I}}^2+t_{L}^{I}t_{R}^{I}+t_{2}^2\right)^2-4t_{L}^{I}t_{R}^{I} t_{2}^2}}{t_2 t_{L}^{I}}\right)^n}{{E_{OBC}^{I}}^2 +t_{L}^{I}t_{R}^{I} -t_{2}^2 +\sqrt{\left(-{E_{OBC}^{I}}^2+t_{L}^{I}t_{R}^{I}-t_{2}^2\right)^2-4t_{L}^{I}t_{R}^{I} t_{2}^2}} \Bigg]
 %\end{array}
\end{align}
 
\end{widetext}

where $\delta_{I}=\sqrt{\left(t_{L}^{I}t_{R}^{I}+t_{2}^2-{E_{OBC}^{I}}^2\right)^2-4t_{L}^{I}t_{R}^{I} t_{2}^2}$ and 
$\delta_{II}=\sqrt{\left(t_{L}^{II}t_{R}^{II}+t_{2}^2-{E_{OBC}^{II}}^2\right)^2-4t_{L}^{II}t_{R}^{II} t_{2}^2}$.

 Note that from (\ref{sq12})  we found, $\Psi_{n(A/B)}^{I}$ depends on $\phi_{A1}^{I}$ and $\Psi_{n(A/B)}^{II}$ depends on $\phi_{B1}^{II}$. Using (\ref{sq6}) we can write down a connecting equation between $\phi_{A1}^{I}$ and $\phi_{B1}^{II}$ as,

\begin{widetext}
\begin{align}\label{sq13}  \tag{SB13}
\Phi_{B1}^{II}=\Phi_{A1}^{I}\frac{\eta_{1}^{II} \lambda_{2}^{II} (E_{OBC}^{II} \eta_{2}^{II}-t_{R}^{II}) \left(\lambda_{1}^{I} {\lambda_{1}^{I}}^{N} (E_{OBC}^{I} \eta_{1}^{I}-t_{L}^{I}) (E_{OBC}^{I}-\eta_{2}^{I} t_{R}^{I})-\lambda_{2}^{I} {\lambda_{1}^{I}}^{N} (E_{OBC}^{I}-\eta_{1}^{I} t_{R}^{I}) (E_{OBC}^{I} \eta_{2}^{I}-t_{L}^{I})\right)}{\lambda_{2}^{I} \eta_{1}^{I} t_{2} (E_{OBC}^{I} \eta_{2}^{I}-t_{L}^{I}) \left(\lambda_{1}^{II} {\lambda_{2}^{II}}^{N} (E_{OBC}^{II} \eta_{1}^{II}-t_{R}^{II})-\lambda_{2}^{II} {\lambda_{1}^{II}}^{N} (E_{OBC}^{II} \eta_{2}^{II}-t_{R}^{II})\right)}
\end{align}
\end{widetext}

Fig.~\ref{sf3} shows the evolution of energy spectra with inter chain coupling $t_{c}$ obtained from diagonalizing (\ref{eq1}), as given in the main text. When $t_{c}$=0, it represents two decoupled chain. As illustrated in Fig.~\ref{sf3}(a) the energy bands in blue corresponds to the chain \textit{I} and bands in red  corresponds to the chain \textit{II}. Two edge modes, for each chain, are shown by red stars and purple triangles, respectively. Now we gradually increase the strength of $t_{c}$. We find that at $t_{c}$=0.7  modes originally at ($\pm i\beta$) splits. Purple triangles remains at ($\pm i\beta$) but the modes in red stars shifting towards the origin as shown in Fig.~\ref{sf3}(b). These modes are labelled $\pm \Delta$ in the main text. Finally at $t_{c}$=$t_{2}$,  $\pm \Delta$ coleases at the origin and forms the EP.
\begin{figure*}[tb]
\setcounter{figure}{0}
\renewcommand{\figurename}{Fig.}
\renewcommand{\thefigure}{S3}
%\textbf{(a) Eigen energy spectra} 
\begin{center}
\setlength{\tabcolsep}{-.0005pt}
\begin{tabular}{c c c}
$t_{c}$=0.0 & $t_{c}$=0.7 & $t_{c}$=$t_{2}$
\\
\includegraphics[width=5.0 cm]{Figsp3a.eps}
&  
\includegraphics[width=5.0 cm]{Figsp3b.eps} 
&  
\includegraphics[width=5.0 cm]{Figsp3c.eps} 
\end{tabular}
\end{center} 
\caption{ Evolution of energy eigen values of the coupled non Hermitian SSH Hamiltonian in (1) as a function of inter-chain coupling $t_{c}$. In all the figures we choose t=1, $t_{2}=1$, $\gamma$=0.5, and N=80.}
\label{sf3}
\end{figure*}





\end{widetext}

\end{document}






