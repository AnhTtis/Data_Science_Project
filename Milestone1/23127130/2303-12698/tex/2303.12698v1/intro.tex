%background of open set recognition
Open set human action recognition has been studied in recent years due to its great potential in real-world applications, such as security surveillance~\cite{aggarwal2011human}, autonomous driving~\cite{roitberg2020open}, and face recognition~\cite{liu2017sphereface}. 
It differs from closed set problems that aim to classify human actions into a predefined set of known classes, since open set methods can identify samples with unseen classes with high accuracy~\cite{geng2020recent}.

\begin{figure}[t]
    \centering
    \includegraphics[width=\linewidth]{figs/concept.pdf}
    \vspace{-5mm}
    \caption{Novelty detection examples of single/multiple actor(s) with single/multiple action(s) in video~\cite{gu-2018-cvpr-ava,sigurdsson2016hollywood_Charades}, where an actor is identified as novel (\textcolor{Goldenrod!80!black}{yellow}) rather than being from a known category (\textcolor{cyan}{cyan}) in inference. Existing works~\cite{bao-2021-ICCV,bendale2016towards} on open set action recognition focus on single actor associated with single action (bottom-left), while our method can handle different situations.} 
    \vspace{-3mm}
    \label{fig:example}
\end{figure}    

%drawbacks of current open set methods   
To this end, several recent methods~\cite{bao-2021-ICCV,bendale2016towards} are proposed for open set human action recognition. 
As shown in the bottom-left of \Cref{fig:example}, they focus on single-actor, single-action based recognition, assuming that each video contains only one single action. 
Compared with traditional softmax scores~\cite{gal-2016-icml,liang2017enhancing,liu2020energy} for closed set recognition, evidential neural networks (ENNs)~\cite{sensoy-2018-nips,bao-2021-ICCV} can provide a principled way to jointly formulate the multi-class classification and uncertainty modeling to measure novelty of an instance more accurately. It assumes that class probability follows a prior Dirichlet distribution. 
However, in more realistic situation with multiple actions of actor(s) (see the upper part of \Cref{fig:example}), the Dirichlet distribution does not hold because the predicted likelihood of each action follows a binomial distribution (\ie, identifying either known or novel action).

%description of our method
In this paper, we introduce a general but understudied problem, namely \textit{novelty detection of actor(s) with multiple actions}. 
Given real-world use cases~\cite{geng2020recent,sozykin2018multi}, the goal is to accurately detect if actor(s) perform novel/unknown action(s) or not. Following~\cite{DBLP:conf/nips/WangLBL21}, an actor is considered unknown if it does not contain any known action(s).
Inspired by the belief theory~\cite{yager-2008-classic,josang-2016-subjective}, we propose a new framework named \textit{\textbf{MU}lti-\textbf{L}abel \textbf{E}vidential learning} (\sysname{}), which is composed of three modules: Actor-Context-Object Relation modeling (\sysnameacor{}), Beta Evidential Neural Network (\sysnamemenn{}), and Multi-label Evidence Debiasing Constraint (\sysnameedc{}).
First, we build \sysnameacor{} representation to exploit the actors' interactions with the surrounding objects and the context.
Then, we use \sysnamemenn{} to estimate the evidence of known actions, and quantify the predictive uncertainty of actions so that unknown actions would incur high uncertainty, \ie, lack of confidence for known predictions.
Here, the evidence indicates actions closest to the predicted one in the feature space and are used to support the decision-making~\cite{sensoy-2018-nips}. 
Instead of relying on Dirichlet distribution~\cite{bao-2021-ICCV}, the evidence in \sysnamemenn{} is regarded as parameters of a Beta distribution which is a conjugate prior of the Binomial likelihood. 

Additionally, in open set recognition, static bias~\cite{li-2018-eccv} may bring a false correlation between the prediction and static cues, such as scenes, resulting in inferior generalization capability of a model.
Therefore, the \sysnameedc{} is added to the objective function of our framework to reduce the static bias for video actions.
We propose a duality-based learning algorithm to optimize the network. 
Specifically, we apply an averaging scheme to proximate primal optimal solutions. 
The primal and dual parameters are updated interactively, where the primal parameters regard model accuracy and dual parameters adjust model debiasing.
The theoretical analysis shows the convergence of the primal solution sequence and gives bounds for both the loss function and the violation of the debiasing constraint in \sysname{}. 
To adapt to our proposed problem, we re-split two representative action recognition datasets (\ie, AVA~\cite{gu-2018-cvpr-ava} and Charades~\cite{sigurdsson2016hollywood_Charades}) into subsets of known actions and novel actions. 
According to the proposed uncertainty and belief based novelty estimation mechanisms, our model outperforms the state-of-the-art on novelty detection. 
The main contributions of this work are summarized:
\begin{itemize}[leftmargin=*,topsep=0pt,itemsep=1ex,partopsep=1ex,parsep=0ex]
    % 
    \item A new framework \sysname{} is proposed for open set action recognition in videos. To the best of our knowledge, this is the first study to detect actors with multiple unknown actions. Furthermore, our method can generalize to scenarios where a video contains either a single or multiple actors associated with one or more actions.
    % 
    \item To optimize the \sysnamemenn{}, we develop a simple but effective algorithm using a primal-dual average scheme update, with theoretical guarantees on the convergence of the primal solution sequence and bounds for both the loss function and the violation of the debiasing constraint.
    % 
    \item To estimate the novelty score for each actor, we introduce four novelty estimation mechanisms. Extensive experiments demonstrate that our proposed \sysname{} outperforms existing methods on novel action detection. 
\end{itemize}  

\begin{figure*}[t]
    \centering
    \includegraphics[width=0.9\linewidth]{figs/arch.pdf}
    \caption{Our \sysname{} framework is composed of three modules. (1) Actor relational features are extracted through the \sysnameacor{} module, where they encode information based on interactions between actor/object instances with the context. (2) Positive and negative evidence are estimated through \sysnamemenn{} to quantify the predictive uncertainty of various human actions. (3) \sysnameedc{} is a debiasing constraint added to the loss function $\mathcal{L}_{beta}$ in \Cref{eq:opt-problem}, which aims to mitigate static bias.}
    \label{fig:arch}
    \vspace{-5mm}  
\end{figure*}