\textbf{Open set action recognition.} 
Despite a great many explorations on video action recognition~\cite{DBLP:conf/iccv/Feichtenhofer0M19,feichtenhofer2020x3d,zhang2021multi}, the majority of existing methods are developed under the assumption that all actions are known a priori (closed set), and limited attention is given to the open set problems. 
OpenMax~\cite{bendale2016towards} is initially proposed for open set recognition, in which it leverages \textit{Extreme Value Theory}~\cite{DBLP:journals/pami/RuddJSB18} to expand the $K$-class softmax classifier. 
Roitberg~\etal~\cite{DBLP:conf/bmvc/RoitbergAS18} develops the voting-based scheme to leverage the estimated uncertainty of action predictions to measure the novelty of the test sample.
DEAR~\cite{bao-2021-ICCV} formulates the open set action recognition problem by estimating the uncertainty of single labeled actions to distinguish between the known and unknown samples. 
Fang~\etal~\cite{fang2021learning} theoretically prove the learnability and the generalization bound for an open set recognition classifier. 
However, such methods focus on simple scenarios where each actor has only one action in a video.

\textbf{Uncertainty estimation.} 
To distinguish unknown samples from known ones, how deep neural networks identify samples belonging to an unrelated data distribution becomes crucial. To this end, a stream of research on Bayesian Neural Nets (BNNs)~\cite{kendall2017uncertainties} is proposed to estimate prediction uncertainty by approximating the moments of the posterior predictive distribution. However, BNNs face several limitations, including the intractability of directly inferring the posterior distribution of the weights given data, the requirement and computational expense of sampling during inference, and the question of how to choose a weight prior~\cite{NEURIPS2020_aab08546}. 
Evidential Neural Networks (ENNs)~\cite{sensoy-2018-nips} have recently been proposed to estimate evidential uncertainty for multi-class classification problems. But they are designed for single-label multi-classification issues by assuming that class probability follows a prior Dirichlet distribution. 

\textbf{Debiasing.} 
Static bias is another challenging issue limiting the generalization capability of a model in an open-set setting~\cite{bao-2021-ICCV}. 
The manifestation of static bias can often be as fraught as the spurious correlation between the prediction and sensitive features like unrelated objects and background~\cite{li-2018-eccv, zhao2020fair}.
Existing works~\cite{li-2018-eccv,bao-2021-ICCV} empirically show that debiasing the model by input data or learned representation can improve the action recognition accuracy.
RESOUND~\cite{li-2018-eccv} indicates that static bias may help to achieve better results in a closed-set setting if an action can overfit it. 
On the contrary, for an open-set setting, DEAR~\cite{bao-2021-ICCV} states the static bias could result in a vulnerable model and further introduces the contrastive evidence debiasing by temporally shuffled feature input and 2D convolution. 
However, it still pushes the sensitive feature to be independent of the non-sensitive one only, in which the dependency of the sensitive feature on model predictions is ignored.       

To address the limitation of the above methods, we propose a new Beta distribution based network with multi-label evidential learning, where static bias is reduced in terms of both indirect and direct dependencies. To our knowledge, it is the first time to solve open set action recognition based on single or multiple actor(s) and action(s).