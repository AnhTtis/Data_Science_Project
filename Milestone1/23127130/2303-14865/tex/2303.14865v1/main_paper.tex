%CVPR 2023 Paper Template
% based on the CVPR template provided by Ming-Ming Cheng (https://github.com/MCG-NKU/CVPR_Template)
% modified and extended by Stefan Roth (stefan.roth@NOSPAMtu-darmstadt.de)

\documentclass[10pt,twocolumn,letterpaper]{article}

%%%%%%%%% PAPER TYPE  - PLEASE UPDATE FOR FINAL VERSION
%\usepackage[review]{cvpr}      % To produce the REVIEW version
%\usepackage{cvpr}              % To produce the CAMERA-READY version
\usepackage[pagenumbers]{cvpr} % To force page numbers, e.g. for an arXiv version

% Include other packages here, before hyperref.
\usepackage{graphicx}
\usepackage{amsmath}
\usepackage{amssymb}
\usepackage{booktabs}
\usepackage[titletoc]{appendix}
\renewcommand{\appendixname}{Supplementary Material}
\usepackage[accsupp]{axessibility}

\DeclareMathOperator*{\argmin}{arg\,min}


% It is strongly recommended to use hyperref, especially for the review version.
% hyperref with option pagebackref eases the reviewers' job.
% Please disable hyperref *only* if you encounter grave issues, e.g. with the
% file validation for the camera-ready version.
%
% If you comment hyperref and then uncomment it, you should delete
% ReviewTempalte.aux before re-running LaTeX.
% (Or just hit 'q' on the first LaTeX run, let it finish, and you
%  should be clear).
\usepackage[pagebackref,breaklinks,colorlinks]{hyperref}

\newcommand{\xl}[1]{\textcolor[rgb]{0,0,1}{[\textbf{Arthur: #1}]}}
\newcommand{\Yu}[1]{\textcolor[rgb]{0.5,0,0.5}{[\textbf{Yu: #1}]}}

% Support for easy cross-referencing
\usepackage[capitalize]{cleveref}
\crefname{section}{Sec.}{Secs.}
\Crefname{section}{Section}{Sections}
\Crefname{table}{Table}{Tables}
\crefname{table}{Tab.}{Tabs.}


%%%%%%%%% PAPER ID  - PLEASE UPDATE
\def\cvprPaperID{5212} % *** Enter the CVPR Paper ID here
\def\confName{CVPR}
\def\confYear{2023}


\begin{document}

%%%%%%%%% TITLE - PLEASE UPDATE
\title{Revisiting Multimodal Representation in Contrastive Learning: From Patch and Token Embeddings to Finite Discrete Tokens}

\author{
    Yuxiao Chen\textsuperscript{\rm 1\thanks{This work was done during a research internship at ByteDance.}},
    Jianbo Yuan\textsuperscript{\rm 2},
    Yu Tian\textsuperscript{\rm 2},
    Shijie Geng\textsuperscript{\rm 1,2},
    Xinyu Li\textsuperscript{\rm 2}, \\
    Ding Zhou\textsuperscript{\rm 2},
    Dimitris N. Metaxas\textsuperscript{\rm 1 \thanks{Dimitris N. Metaxas has been supported by NSF IUCRC CARTA-1747778, 2235405, 2212301, 1951890, 2003874.}},
    Hongxia Yang\textsuperscript{\rm 3} \\
    \textsuperscript{\rm 1}Rutgers University \quad
    \textsuperscript{\rm 2}ByteDance Inc. \quad
    \textsuperscript{\rm 3}Zhejiang University \\
    \texttt{\{yc984, sg1309, dnm\}@rutgers.edu},\\ \texttt{\{jianbo.yuan, yutian.yt, lixinyu.arthur, ding.zhou\}@bytedance.com} \\
    \texttt{hongxia.yang1@gmail.com} 
}
\maketitle
%%%%%%%%% ABSTRACT
\begin{abstract}
% Contrastive learning-based vision-language pre-training approaches, such as CLIP, have demonstrated great success in many vision-language tasks. These approaches achieve cross-modal alignment by encoding a matched image-text pair with similar feature embeddings that aggregate information from visual patches and language tokens. However, since visual patches and text tokens contain different levels of semantics and granularities, aligning the inputs directly using such embeddings is not trivial. To alleviate this issue, we present a novel multimodal representation that first grounds multimodal inputs to a fixed set of shared finite discrete tokens (FDT) based on the input contents. Each token represents certain high-level semantic concepts ranging from objects to actions and attributes. The embeddings of the activated tokens then form the FDT-based representations used for cross-modal alignment at a semantic level. 
% With sparse activation, we show that FDT are encouraged to denote matched visual and textual semantic concepts. 
% In addition, sharing FDT across all modalities provides a light-weighted way of cross-modal interactions for better cross-modal alignment. 
% Unlike previous discrete representation learning methods, FDT can be trained under the contrastive learning scheme from scratch without cold-start problems.
% Using both quantitative and qualitative analyses, we further demonstrate that simply replacing the image and text embeddings in CLIP-style models by FDT representations achieves better cross-modal alignment and superior performance in visual recognition and vision-language downstream tasks compared to previous approaches.
\vspace{-12pt}
Contrastive learning-based vision-language pre-training approaches, such as CLIP, have demonstrated great success in many vision-language tasks. These methods achieve cross-modal alignment by encoding a matched image-text pair with similar feature embeddings, which are generated by aggregating information from visual patches and language tokens. However, direct aligning cross-modal information using such representations is challenging, as visual patches and text tokens differ in semantic levels and granularities. To alleviate this issue, we propose a Finite Discrete Tokens (FDT) based multimodal representation. FDT is a set of learnable tokens representing certain visual-semantic concepts. Both images and texts are embedded using shared FDT by first grounding multimodal inputs to FDT space and then aggregating the activated FDT representations. The matched visual and semantic concepts are enforced to be represented by the same set of discrete tokens by a sparse activation constraint. As a result, the granularity gap between the two modalities is reduced. Through both quantitative and qualitative analyses, we demonstrate that using FDT representations in CLIP-style models improves cross-modal alignment and performance in visual recognition and vision-language downstream tasks.  Furthermore, we show that our method can learn more comprehensive representations, and the learned FDT capture meaningful cross-modal correspondence, ranging from objects to actions and attributes.\footnote{The source code can be found
at \textcolor{magenta}{\url{https://github.com/yuxiaochen1103/FDT}}.} 

%In addition, sharing FDT across all modalities provides a light-weighted way of cross-modal interactions for better cross-modal alignment


%, each of which is encouraged to selectively denote a matched visual-semantic concept including not only objects, but actions and attributes as well. %by sparse activation? 
%without introducing any fusion operations. In addition, FDT serve as multimodal anchors to capture information from each input with better completeness, which alleviates the encoder degradation problem that is commonly observed in vanilla CLIP models. 

% {\color{red}DNM: Can you say how much better? ie a percent? }%including image classification, cross-modal retrieval, etc.


%Cross-modal alignment is achieved by encoding a matched image-text pair to be close to each other through image and text encoders. In this paper, we present a new multimodal presentation grounded to a fixed set of shared, finite learnable tokens (FLT), which can be viewed as fine-grained abstractions of a unified visual-semantic joint space. Each input image or text is represented by a combination of the token representations based on the attention weights between the input and FLT. Our approach provides a light-weighted way of cross-modal interactions for better cross-modal alignment without introducing any fusion operations. In addition, FLT serve as multimodal anchors to capture information from each input with better completeness, which alleviates the encoder degradation problem that is commonly observed in vanilla CLIP models. We conduct extensive experiments to show that %, unlike other methods such as vector-quantization (VQ) and codebook learning, 
%the FLT can be trained with the same contrastive learning scheme from scratch without cold-start problems. With both quantitative and qualitative analysis, we further show that simply replacing the image and text embeddings in CLIP-style models by FLT-based representations achieves better cross-modal alignment and better performance in downstream tasks including image classification, cross-modal retrieval, and xxx task.

% TODO: emphasize the logic flow diff
\end{abstract}

%%%%%%%%% BODY TEXT
\vspace{-12pt}
\section{Introduction}
\label{sec:intro}
\documentclass[../main.tex]{subfiles}
\begin{document}

Magnetically actuated medical robots (MAMR) have seen significant focus and development in recent decades due to their potential for miniaturization~\cite{hu2018small}, tether-less actuation~\cite{popek2016six} and high number of controllable degrees-of-freedom (DOFs)~\cite{salmanipour2018eight,pittiglio2022collaborative}. In fact, magnetically guided catheters have been used to treat cardiac arrhythmias since 2003~\cite{nelson2022magnetically,carpi2009stereotaxis}.

A key aspect in their actuation is pose estimation~\cite{bianchi2019localization,barducci2019adaptive}, enabling closed loop control and delivery of functionality~\cite{norton2019intelligent}. Imaging techniques have long been used for this purpose but are generally tied to limited resolution, harmful radiation exposure and need for additional hospital equipment~\cite{aziz2020medical,pane2022ultrasound,daguerre2022localization}. As such, methods based on magnetic field measurements have received significant attention, with magnetic tracking systems being widely available on the market. These, however, are not compatible with magnetic actuation systems due to distortions on the localization magnetic fields.

To address this issue, significant research on magnetic localization coupled with magnetic actuation systems has been done~\cite{khalil2019magnetic,popek2016six,son2018simultaneous,shao2019novel,taddese2018enhanced}. Several works have been based on magnetic field sensing arrays external to MAMR~\cite{micheal20222d,son2018simultaneous}. While advantageous from a miniaturization and internal power consumption point of view, these systems require calibration of large sensor arrays and have limited localization workspace dimensions. Internal sensing to the MAMR, on the other hand, does not suffer from workspace dimension restrictions. It requires, however, on-board power and heterogeneous localization magnetic fields, with 6-DOF localization having been shown for systems with a single external permanent magnet (EPM). \textcolor{black}{Internal sensing methods have been shown for endoscopic capsules, as well as magnetically guided catheters}~\cite{popek2016six,sperry2022six,taddese2018enhanced,fischer2022using}.

Over recent years the need for enhanced control and manipulability of MAMRs has led to the advent of actuation platforms based on multiple magnetic field sources (MMFS) such as multiple electromagnetic coils and multiple permanent magnets~\cite{kummer2010octomag,hoang2019independent,hong2020magnetic,pittiglio2022collaborative,ryan2017magnetic,stereotaxis_patent}. Some of these platforms have been cleared for human use such as Stereotaxis Genesis RMN\textsuperscript{\tiny\textregistered} based on two permanent magnets, and Magnetecs and Aeon Scientific based on multiple electromagnetic coils. 

Despite this progress, magnetic localization for such systems is lagging behind, with fluoroscopic imaging being currently used~\cite{nelson2022magnetically}. Unlike single magnetic field source systems where the singularity regions and localization limitations have been thoroughly investigated and solved for~\cite{taddese2018enhanced}, magnetic localization for MMFS systems suffers from additional challenges due to the superposition of the magnetic fields leading to configuration-specific singularity regions. Only recently, a 3D position localization system with internal magnetic field sensing was demonstrated for a multi-coil system, \textcolor{black}{for a 3~mm catheter}~\cite{fischer2022using}.

Furthermore, a common conundrum in 6-DOF magnetic localization with internal sensing is finding the rotation around gravity, due to the absence of the Earth's magnetic field measurement~\cite{mahony2008nonlinear}. This has been solved in the past by accurately initializing this missing rotation angle and tracking it with a gyroscope~\cite{pittiglio2020observability, di2016jacobian}. However, this is prone to errors over time, especially for slow moving systems where gyroscope data is not as sensitive. Additionally, if communication to the MAMR is lost, a new accurate initialization is needed, proving impossible mid medical intervention. More recently, Taddese et al.~\cite{taddese2018enhanced} fitted an auxiliary coil around a single EPM providing a second set of magnetic field measurements. This solves the missing rotation angle and is also able to eliminate the localization singularity plane when it comes to localization with respect to a single EPM. However, when MMFS are present in the workspace, that singularity plane ceases to exist due to the superposition of magnetic fields, and instead singularity regions are present depending on the relative pose of each EPM.

This paper introduces, for the first time, a 6-DOF magnetic localization method for systems with multiple EPMs without the need for any prior pose information. The method relies on measurements from an accelerometer and a single 3D magnetic field Hall effect sensor (HE), both internal to the MAMR. We analyze the effect that the number of EPMs in the workspace has on the full pose estimation; and demonstrate its performance in a two EPM magnetic actuation platform. Since adding an orthogonal coil is not able to solve for the singularity regions, in this work we do not consider it and instead solve for the missing rotation angle by using multiple magnetic field measurements at different EPM configurations. This works for static or quasi-static systems, with maximum MAMR velocity highly dependent on the actuation system and the magnetic field generated. This is the case for non-actuated parts of a larger system, such as the deployment point at the tip of an endoscope, or for MAMRs while the generated magnetic fields are sufficiently weak to induce actuation. Additionally, unlike common works in literature which parameterize the rotation matrix, in this work the full 6-DOF pose is estimated directly in the special euclidean group $SE(3)$. This avoids any singularities or non-unique representations of the orientation when using Euler angles or quaternions~\cite{mathavaraj2021se,mayhew2011quaternion,taddese2018enhanced}.

\end{document}


\begin{figure*}[t]
\begin{center}
   \includegraphics[width=0.9\linewidth]{figures/method_v4.pdf}
\end{center}
\caption{\textbf{Left:} Overview of the proposed method. Both the image and text information is encoded with shared FDT during cross-modal contrastive pre-training. \textbf{Right:} The process of grounding image or text features to FDT. The attention weights between visual patch/language token and FDT are first calculated, and then max-pooled over all visual patches/language tokens. The attention-weighted sum of FDT is calculated as the FDT-based features. }
\vspace{-10pt}
\label{fig:method}
\end{figure*}

\section{Related Work}
\label{sec:rw}
\section{Related work}
% There is extensive recent work on speeding up analytical queries due to the need for consistent execution times in the face of the explosive growth in the volume of available data.
% In this section, we divide existing work into two categories: maintaining data freshness in MVs (\Cref{sec:server_side}) and optimizations for minimizing ad-hoc query latency (\Cref{sec:client_side}).

% \subsection{Maintaining Data Freshness in MVs}
% \label{sec:server_side}
% There exists a variety of data warehousing applications aimed at supporting low-latency analytical queries on fresh data.
% In particular, these applications require efficiency in the propagation of newly ingested data into downstream MVs.
 
\mypara{Efficient MV Refresh}
Incremental view maintenance (IVM) aims to update MVs to reflect newly ingested data, taking advantage of already computed results to perform the update in a manner more efficient than computing from scratch (full refresh)
~\cite{ahmad2012dbtoaster,mcsherry2013differential,armbrust2013generalized,zeng2016iolap, palpanas2002incremental, griffin1995incremental, agiwal2021napa, braun2015analytics}. 
There is an abundance of work in IVM, including incremental updates on duplicate values~\cite{griffin1995incremental}, non-distributive aggregate functions~\cite{palpanas2002incremental}, higher-order views~\cite{ahmad2012dbtoaster}, and sliding windows~\cite{braun2015analytics}. 
More recent works also investigate the scalability aspect of IVM, proposing scale-independent updates~\cite{armbrust2013generalized} and sampled views~\cite{zeng2016iolap}. Since \system is applicable to arbitrary SQL statements, \system is orthogonal to and is fully compatible with existing IVM techniques.

\mypara{MV Refresh Scheduling}
There exist works on scheduling the refresh of a MV set focusing on resolving cyclic dependencies~\cite{folkert2005optimizing}, minimizing weighted average staleness~\cite{golab2009scheduling}, and prioritizing MVs with the highest speedups on predicted future queries~\cite{ahmed2020automated}.
\system's scheduling to speed up the end-to-end refresh of the MV set is not addressed in existing works.

\mypara{DAG Workflow Scheduling}
The execution of workloads consisting of individual jobs with acyclic dependencies is a well-studied topic~\cite{apacheoozie,sparkdag,marchal2018parallel,bathie2020revisiting,baruah2022ilp}; many of these techniques can be applied to MV refresh runs studied in this paper.
Existing workflow scheduling systems such as Apache Oozie~\cite{apacheoozie}, Apache Airflow~\cite{airflow}, and Spark DAG scheduler~\cite{sparkdag} automate the execution of user-defined workflows following a topological order.
There are recent works aimed at finding more optimal execution orders in terms of peak memory usage~\cite{marchal2018parallel, bathie2020revisiting} and execution time on parallel platforms~\cite{baruah2022ilp}.
While \system is designed for use with MV refresh runs/workloads, our technique on joint scheduling and optimization can be reasonably applied to general workloads as a possible future direction.

% \paragraph{Incremental MV indexing}
% Update-optimized indices such as the log-structured merge-trees (LSM)~\cite{o1996log} are used for indexing MVs due to frequent updates induced by data ingestion~\cite{gupta2016mesa,agiwal2021napa}.
% \system is orthogonal to indexing: \system is capable of efficiently performing MV refresh runs regardless of whether the individual MVs are indexed or not.

% \subsection{Ad-hoc Query Latency Reduction}
% \label{sec:client_side}

% The minimization of ad-hoc analytical query response times is a well-studied topic due to latency being negatively correlated with the productivity of a data analyst during a data analysis session~\cite{liu2014effects}.
% Sessions are commonly conducted within visualization systems that contain a variety of optimization techniques to ensure that query response times fall within a certain latency tolerance.

% \mypara{Data prefetching}
% Data is often loaded into memory on a by-need basis in visualization systems to minimize interference with user-issued query computations~\cite{mani2017effective,xin2021enhancing,galakatos2017revisiting, yan2020auto, battle2016dynamic, crotty2016case, jalaparti2018netco}. 
% Query-time data retrieval can be significantly expedited by anticipating the data usage of the user in future queries and pre-loading the data into memory during the downtime between user queries (`think time'). SMART~\cite{mani2017effective} prefetches data for modified versions of current user-issued queries with different filters and dimensions. A-WARE~\cite{crotty2016case} maintains a sample store constantly refined through ingesting data based on speculations of future plots.
% ForeCache~\cite{battle2016dynamic} uses an SVM to predict the user's current analysis phase and accordingly prefetches data tiles partitioned based on different numerical values. NetCo predicts future queries via log analysis, and solves an ILP formulation to prefetch data to maximize the number of SLO-meeting queries~\cite{jalaparti2018netco}.
% In the case of MV refresh workloads, `think time' is nonexistent as individual MVs are refreshed back-to-back, rendering data prefetching techniques non-applicable.

\mypara{Intermediate Data Caching}
Some existing data visualization systems cache user-defined variables to support the typical incremental construction of data visualizations~\cite{zgraggen2016progressive, eichmann2020idebench} during data analysis sessions~\cite{jupyter, rstudio, colab}. 
Recent work proposes a management scheme for these cached variables under a memory constraint that greedily keeps variables with the highest estimated time savings based on predicted future user behavior via neural networks~\cite{xin2021enhancing}.
While useful for data visualization, a greedy approach to memory management fails to achieve satisfactory results compared to \system.

\mypara{Intermediate Result Reuse}

There exist works on storing intermediate results from computations to speedup future computations~\cite{yang2018intermediate, dursun2017revisiting, nagel2013recycling, michiardi2019memory, galakatos2017revisiting}.
Studied topics include the identification of reuse opportunities by finding overlaps in computation graphs of successive jobs~\cite{yang2018intermediate, michiardi2019memory},
selective storage under a space constraint with heuristics such as reuse probability~\cite{dursun2017revisiting}, expected savings~\cite{yang2018intermediate}, and recompute-storage cost difference~\cite{nagel2013recycling},
and rewriting incoming jobs to take advantage of stored intermediates~\cite{galakatos2017revisiting}.
These works share similarity with \system in their selection of items to store under a memory constraint, however, \system's problem setting requires it to uniquely consider the joint (re)ordering of job executions along with the selection of items.

% work that considers both job execution (re)order as well as intermediate result caching with a bounded amount of memory. but notably lack the joint aspect of \system and cannot be used to achieve immediate speedup on an incoming MV refresh run if no intermediates are stored beforehand. 

\mypara{Incremental Query Processing} Incremental processing (IQP) is useful for cases where not all data required for a query is immediately available. Similar to online aggregation~\cite{hellerstein1997online}, initial results of a query are computed on a subset of required data and progressively refined as the rest of the required data arrives in a predictable pattern~\cite{tang2019intermittent,wangtempura}. Tang et al. propose a dynamic programming formulation to pick intermediate states to store in memory given a limited memory budget~\cite{tang2019intermittent}. Tempura rewrites the query plan for more efficient execution based on predicted data arrival patterns~\cite{wangtempura}. While similarities exist between the problem setting of IQP and \system, such as management of bounded memory, \system notably includes additional joint optimization for the order of MV updates.

% \paragraph{Sampling}
% Sampling has seen wide use in visualization systems for reducing the computation time of ad-hoc queries by computing an approximate result over a subset of data as exact results are not always required by the user~\cite{crotty2016case, mani2017effective, zgraggen2014panoramicdata, kraska2021northstar, galakatos2017revisiting, kandula2016quickr}. 
% Commonly studied topics in sampling for ad-hoc queries include complex query sampling~\cite{kandula2016quickr}, rare event aggregation~\cite{kraska2021northstar, galakatos2017revisiting}, and maintaining consistency between related sampled visualizations~\cite{zgraggen2014panoramicdata}.
% Sampling server-side at the MV level compromises the assumptions of downstream applications and is thus not considered in \system.

% \paragraph{Progressive visualization}
% The latency tolerance for time-consuming queries can be circumvented by presenting a partially-computed visualization to the user within the tolerance, which is then incrementally refined until it is fully accurate~\cite{rahman2017ve, zgraggen2016progressive, crotty2015vizdom, kraska2021northstar, kamat2017infiniviz}.
% Example plots which benefit from progressive visualization include bar charts~\cite{kamat2017infiniviz} and heatmaps~\cite{rahman2017ve}.
% Similar to sampling, study on this topic is orthogonal to \system as pushing out partially-updated MVs compromises downstream assumptions.


\section{Method}
\label{sec:method}
\section{Method}
\label{sec:method}

% \ml{``Inconsistent'' to ``large variation''}

% In this section, we propose our methods based on the observations in Section \ref{sec:motivation}.
In this section, we propose two techniques to further enhance the strong baseline to capture the variation of activation distributions better.
We first introduce spatial re-scaling to adapt the network to pixel-to-pixel variation.
We then propose channel-wise shifting and re-scaling to better capture the channel-to-channel variation.
Meanwhile, as both of the two methods are image-dependent, the image-to-image variation can be captured naturally.
By combining the two methods with our strong baseline, we build our enhanced BNN for SR, named EBSR.

% Because the activation distributions among pixels, channels and images have large variations \red{**are highly inconsistent} in SR networks, we introduce spatial re-scaling to adapt to pixel-wise variations and channel shift and re-scaling to adapt to channel-wise variations. And both of them are image-dependent to adapt to image-wise variations, which means during inference our network re-scales and shifts the distributions of activations flexibly for different input images. Based on these methods, we build an enhanced binary neural network for image super-resolution (EBSR).

% According to [3], the difference of activation magnitudes indicates different scaling factors are needed for each pixel.

\subsection{Spatial Re-scaling}
% It is better to use different scaling factors for different pixels to reduce the quantization error and retain more detailed information for image super-resolution. 

% \ml{In the main method, we do not need to introduce the previous works but can focus on introducing our own method. Channel rescaling in Real-to-binary Net is not relevant in this context.}

% Re-scaling the output of binary convolutions was proposed at the birth of BNN in XNOR-Net \cite{rastegari2016xnor} to reduce quantization error and improve accuracy for image classification tasks.
% It is computed as below:
% \begin{equation}
% \mathcal{A} * \mathcal{W} \approx(\operatorname{sign}(\mathcal{A}) \circledast \operatorname{sign}(\mathcal{W})) \odot \mathcal{K} \alpha
% \label{eq:xnor-net rescale}
% \end{equation}
% where $\circledast$ denotes the binary convolution and $\odot$ denotes the element-wise multiplication.
% $\mathcal{A}$, $\mathcal{W}$, $\alpha$, and $\mathcal{K}$ denote the activation, weight, weight scaling factor, and activation scaling factor, respectively.
%  Later in XNOR-Net++ \cite{bulat2019xnor}, Bulat et al. fuse the activation and weight scaling factors into a single one that is learned end-to-end based on gradients and this improves the classification accuracy on ImageNet dataset.

% % It is computed as Eq.~\ref{eq:xnor-net rescale}, where $\circledast$ denotes 
% %  the binary convolution and $\odot$ denotes the element-wise multiplication. The binary convolution of $\mathcal{A}$ and $\mathcal{W}$ is rescaled by the weight scaling factor $\alpha$ and the activation scaling factor $\mathcal{K}$, both of which are calculated analytically.


% \zc{Similarly, you should explain the meaning of A, W and the operators $\circledast$ in the formula}
% Then in Real-to-binary Net \cite{martinez2020training}, Martinez et al. used a data-driven channel re-scaling module that takes the pre-convolution activations as input to predict the activation scaling factor. Unlike that in XNOR-Net++ \cite{bulat2019xnor}, these scaling factors are not fixed during inference but rather inferred from data. By doing this, they further improved the classification accuracy on ImageNet over XNOR-Net++. 
As is shown in Figure \ref{fig:pixel}, activation distributions have large pixel-to-pixel variation in SR networks
and the difference of activation magnitudes indicates different scaling factors are preferred for different pixels.
Inspired by \cite{martinez2020training}, we propose spatial re-scaling to better adapt the network to the spatial variation
of activation distributions in SR networks.
% fit the various pixel-wise distributions in SR networks.
We take the real-valued activations $A$ before convolution as input and predict pixel-wise scaling factors $S(A)$, which re-scale the binary convolution output. Spatial re-scaling process can be formulated as follows:
\begin{equation}
A * W \approx(\operatorname{sign}(A) \circledast \operatorname{sign}(W)) \odot \alpha \odot S(A)
\label{eq:spatial rescale}
\end{equation}
where $\circledast$ denotes 
the binary convolution and $\odot$ denotes the element-wise multiplication. $A$, $W$, $\alpha$, and $S\left(A\right)$ denote real-valued activations, weights, the scaling factor of weights, and the spatial-wise scaling factor of activations respectively. $S\left(A\right) \in \mathbb{R}^{1\times H\times W}$ can be calculated with a convolution and a sigmoid function.
% as $\sigma\left( CONV\left(A\right)\right)$. 
As shown in Figure \ref{fig:method}(a), real-valued activations first go through a convolution layer,
which has an input channel of $C$ and an output channel of 1, 
and then pass through a sigmoid function to produce the scaling factors $S(A)$ along the spatial dimension.
During inference, the scaling factor will change dynamically according to different input feature maps.
By re-scaling binary convolution output using $S(A)$, we can reduce the quantization error and the original pixel-wise information in FP activation
will be preserved much better.
Spatial re-scaling leads to a large PSNR improvement of 0.24 dB (from 30.30 dB to 31.54 dB) on Set5 and 0.22 dB (from 25.09 dB to 25.31 dB)
on Urban100 compared with our strong baseline. 

\subsection{Channel-wise Shifting and Re-scaling}

\begin{table}[!tb]
\centering
\caption{Comparison between whether to fuse channel-wise shifting and re-scaling or not based on our baseline with spatial re-scaling. }
\label{tab:fusing}

\scalebox{0.65}{
\begin{tabular}{c|cc|cc|cc}
\hline
\multirow{2}{*}{Method}     & \multirow{2}{*}{OPs} & \multirow{2}{*}{Params} & \multicolumn{2}{c|}{Set5} & \multicolumn{2}{c}{Urban100} \\ \cline{4-7} 
                            &                      &                         & PSNR        & SSIM        & PSNR          & SSIM         \\ \hline
Baseline + spatial re-scale & 2.16G                & 0.05M                   & 31.54       & 0.883       & 25.31         & 0.759        \\
+ channel-wise shift and re-scale             & 2.34G                & 0.09M                   & 31.61       & 0.885       & 25.35         & 0.761        \\
+ w/ fusing                   & 2.27G                & 0.08M                   & \textbf{31.64}       & \textbf{0.885}       & \textbf{25.36}         & \textbf{0.761}        \\ \hline
\end{tabular}
}
\end{table}

In SR networks, activation distributions exhibit larger channel-to-channel variation (Figure \ref{fig:chl}).
Both the mean and magnitude of the activation distributions vary significantly across channels.
% Thus we use channel-wise shifting and re-scaling to adapt to various channel-wise distributions. 
\cite{martinez2020training} has proposed the data-driven channel re-scaling, 
but our method differs from them in further introducing data-driven thresholds to handle the channel-wise variation of both mean and magnitude.
Since the blocks to generate the scaling factors and thresholds are very similar, we further propose to fuse them into one module.
% and fusing channel-wise shifting and re-scaling into one module.
We evaluate the effect of fusing the two blocks in Table \ref{tab:fusing}.
With channel-wise shifting and re-scaling fused, our models have fewer operations and parameters overhead and slightly higher performance.

For the specific process, we take the real-valued activations as input and predict different thresholds and scaling factors for each channel. They are also image dependent, e.g., $\beta_{i}$ in Eq.\ref{eq:act_binarize} is no longer fixed during inference but generated according to different input feature maps. Channel-wise shifting and re-scaling can be formulated as follows:
\begin{equation}
A * W \approx(\operatorname{sign}(A-C_s(A)) \circledast \operatorname{sign}(W)) \odot \alpha \odot C_r(A)
\label{eq:channel-wise_shift_and_rescale}
\end{equation}
where $\circledast$ denotes 
the binary convolution and $\odot$ denotes the element-wise multiplication. $C_s(A), C_r(A) \in \mathbb{R}^{C\times1\times1}$ denote the channel-wise threshold and scaling factor, respectively. 
We show the block diagram in Figure \ref{fig:method}(b).
The real-valued input feature map is first squeezed to a ${C\times1\times1}$ vector by a global average pooling (GAP) layer.
The subsequent fully connected layers and ReLU learn the channel-wise information and output a ${2C\times1\times1}$ vector.
Then the ${2C\times1\times1}$ vector is split into two ${C\times1\times1}$ vectors.
We use the first $C$ channels as the channel-wise bias and pass the last $C$ channels through a sigmoid layer 
as the channel-wise scaling factor, which are used to shift the real-valued activations and re-scale the binary convolution output, respectively. 


% \ml{We can mention previously, channel-wise re-scale has been proposed. We propose to fuse them. Add the comparison between fuse v.s. no fuse.}

\begin{figure}[!tbp]%
  \centering
    \includegraphics[width=0.4\textwidth]{fig/methods.png}
  
% \subfloat[channel-wise shifting\&re-scale]{
%     \label{subfig:channel-wise shifting and re-scale}
%     \includegraphics[width=0.2\textwidth]{fig/chl shift and rescale.png}
%   }

  \caption{Block diagram for spatial re-scaling, and channel-wise shifting and re-scaling.} 
  % Input A is the real-valued activation tensor and C, H, and W denote its dimension. GAP stands for global average pooling. The reduction ratio r is set to 16 for a better trade-off between the performance and the number of operations and parameters.}
  \label{fig:method}
\end{figure}


\subsection{Network Structure}

Combining the spatial re-scaling and the channel-wise shifting and re-scaling methods, we construct the enhanced convolution layer (E-Conv).
Then we build our EBSR model based on E-Conv.
In Figure \ref{fig:E-conv}, we compare the binary convolution layer used in the baseline network and our proposed E-Conv.
We use spatial and channel-wise scaling factors to re-scale the binary convolution output,
and use channel-wise shifting to learn appropriate thresholds for each channel before binarization.
The scaling factors and threshold used in E-Conv are learnable and depend on the real-valued input activations.
In this way, our proposed EBSR can adapt to pixel-to-pixel, channel-to-channel, and image-to-image variations
to reduce the large binarization error and preserve more details.
% In this way, our proposed E-Conv reduces the large quantization error caused by binarization and keeps the original information of input feature maps to a large extent.


\begin{figure}[!tb]%
  \centering

    \includegraphics[width=0.5\textwidth]{fig/E-conv.png}

  \caption{Comparison of (a) the binary convolution layer with a skip connection used in our baseline network and (b) the proposed E-Conv.}
  \label{fig:E-conv}
\end{figure}


Figure \ref{fig:network} shows the basic block based on the E-Conv and our EBSR composed of the basic blocks. Following existing works, the convolution layers in the head and tail modules are not binarized. We choose the lightweight EDSR which has 16 basic blocks and 64 channels, and EDSR which has 32 basic blocks and 256 channels as our backbones, which correspond to EBSR-light and EBSR, respectively.

\begin{figure}[!tb]%
  \centering
  {
    \includegraphics[width=0.35\textwidth]{fig/network.png}
  }
  
  \caption{The structure of our proposed EBSR.  Convolution layers in purple are real-valued vanilla 3x3 convolutions.}
  \label{fig:network}
\end{figure}

\section{Experiments}
\label{sec:exp}
\subsection{Experimental Settings}
\noindent{\textbf{Pre-training Datasets.~}} We use four publicly available datasets, including YFCC-15M V2 \cite{declip_benchmark}, Conceptual Captions (CC3M) \cite{cc3m}, Conceptual 12M (CC12M) \cite{cc12m} and LAION115M \cite{li2022blip} datasets to pre-train our models. We construct three different pre-training settings, including \textbf{15M}, \textbf{30M}, and \textbf{145M} settings. Each of the settings uses different combinations of pre-training datasets, as shown in Table. The 15M setting is used for the ablation study and to compare our methods with state-of-the-art methods under a fair setup \cite{declip_benchmark}. The 30M and 145M settings are used to evaluate the  scalability of our model.

\begin{table}[h]
\centering
\resizebox{0.8\linewidth}{!}{
\begin{tabular}{@{}ll@{}}
\toprule
Setting & Dataset                             \\ \midrule
15M     & YFCC-15M V2                         \\
30M     & YFCC-15M V2, CC3M, CC12M            \\
145M    & YFCC-15M V2, CC3M, CC12M, LAION115M \\ \bottomrule
\end{tabular}
}
\caption{The used pre-training datasets under different settings.}
\vspace{-10pt}
\label{tbl:setting}
\end{table}


% In the 15M setting, we only use the YFCC-15M V2 dataset to save training cost for ablation study, while in 30M settings, we use the three datasets.


\begin{table*}[t]
\centering
\resizebox{0.7\linewidth}{!}{
\begin{tabular}{@{}lcccccccccc@{}}
\toprule
\multicolumn{1}{l}{} &
  \rotatebox{0}{C10} &
  \rotatebox{0}{C100} &
  \rotatebox{0}{F101} &
  \rotatebox{0}{PETS} &
  \rotatebox{0}{FLOW} &
  \rotatebox{0}{SUN} &
  \rotatebox{0}{DTD} &
  \rotatebox{0}{CAL} &
  \rotatebox{0}{IN} &
  \rotatebox{0}{AVG} \\ \midrule

SLIP~\cite{slip}   & 50.7 & 25.5 & 33.3 & 23.5 & 49.0   & 34.7 & 14.4 & 59.9 & 34.3 & 36.1 \\
MS-CLIP-S~\cite{msclip}  & -    & -    & -    & -    & -    & -    & -    & -    & 36.7 & -    \\
CLIP~\cite{clip}                   & 60.4 & 33.5 & 39.6 & 23.1 & 54.0 & 42.0 & 17.0 & 65.5 & 37.0                 & 41.3                  \\
FILIP~\cite{filip}    & 65.1 & 34.2 & 43.2 & 24.1 & 52.8 & 50.8 & 24   & 68.9 & 39.5 & 44.7 \\
DeCLIP~\cite{declip}    & 72.8 & 40.3 & 49.9 & 36.2 & 60.1 & 48.8 & 26.4 & 72.7 & 43.2 & 50.0 \\
\midrule
CLIP+FDT (Ours)                & 67.7 & 39.9 & 42.9 & 25.8 & 55.5 & 45.5 & 26.5 & 69.6 & 39.3                 & 45.9 \\
% FILIP+FDT (Ours)                &63.8 & 41.0 & 49.8 & 38.5 & 60.9 & 47.2 & 25.5 & 71.0 & 41.9 & 48.8 ($\uparrow$ 4.1)  \\
DeCLIP+FDT (Ours)               & \textbf{75.7} &
  \textbf{45.2} &
    \textbf{52.9} &
  \textbf{40.7} &
  \textbf{64.6} &
  \textbf{52.0} &
  \textbf{30.7} &
  \textbf{76.2} &
  \textbf{45.8} &
  \textbf{53.8}    \\ \bottomrule
\end{tabular}}

\caption{Zero-shot image classification accuracy (\%) under the 15M setting. The dataset names are abbreviated. 
C10/100 is CIFAR10/100. F101 is Food101. FLOW is Flowers. CAL is Caltech. IN
is ImageNet-1K. ``AVG'' is the average accuracy over all datasets.}
\label{tbl:15m_zs}

\end{table*}

\begin{table*}[ht!]
\centering
\resizebox{0.7\linewidth}{!}{
\begin{tabular}{@{}lccccccccccc@{}}
\toprule
\multicolumn{1}{l}{} &
  \rotatebox{0}{C10} &
  \rotatebox{0}{C100} &
  \rotatebox{0}{F101} &
  \rotatebox{0}{PETS} &
  \rotatebox{0}{FLOW} &
  \rotatebox{0}{SUN} &
  \multicolumn{1}{l}{\rotatebox{0}{CARS}} &
  \rotatebox{0}{DTD} &
  \rotatebox{0}{CAL} &
  \rotatebox{0}{AIR} &
  \rotatebox{0}{AVG} \\ \midrule
SLIP~\cite{slip}  & 87.4 & 69.5 & 71.3 & 70.5 & 91.9 & 66.9 & 27.5 & 65.6 & 86.2 & 27.7 & 66.5 \\
MS-CLIP-S~\cite{msclip}  & 87.2 & 66.7 & 76.0   & 62.1 & 93.8 & 71.7 & 27.5 & 69.4 & 81.6 & \textbf{32.9} & 66.9 \\
CLIP~\cite{clip}                  & 88.3 & 68.6 & 72.1 & 72.5 & 92.6 & 69.5 & 29.8                 & 67.8 & 86.2 & 27.7                     & 67.5                  \\
FILIP~\cite{filip}   & 86.5 & 66.6 & 71.7 & 69.2 & 93   & 69.6 & 30.0   & 66.4 & 85.7 & 27.0   & 66.6 \\
DeCLIP~\cite{declip}   & 89.4 & 69.6 & 75.9 & 71.4 & 95.7 & 71.6 & 30.1 & 66.9 & 89.0   & 26.7 & 68.6 \\
\midrule
CLIP+FDT (Ours)               & 89.1 & \textbf{71.2} & 74.4 & 73.0   & 93.4 & 70.8 & 31.4                 & 69.4 & 87.7 & 27.9                     & 68.8 \\

% FILIP+FDT (Ours)                     &72.1 & 49.3 & 54.9 & 43.4 & 78.1 & 60.4 & 0.5 & 58.7 & 37.6 & 6.7 & \textcolor{red}{46.2}                  \\

DeCLIP+FDT (Ours)                & \textbf{89.8}& \textbf{71.2}& \textbf{77.7}& \textbf{73.9}& \textbf{95.7}& \textbf{72.9}& \textbf{33.7}& \textbf{69.6}& \textbf{89.4}& 26.9&  \textbf{70.1}                      \\
\bottomrule
\end{tabular}}
\caption{Linear probing image classification accuracy (\%) under the 15M setting. The dataset names are abbreviated. 
C10/100 is CIFAR10/100. F101 is Food101. FLOW is Flowers. CAL is Caltech. Air is Aircraft. ``AVG'' is the average accuracy over all datasets.}
\label{tbl:15m_lp}
\vspace{-10pt}
\end{table*}

\begin{table*}[ht!]
\centering
\resizebox{0.8\linewidth}{!}{
\begin{tabular}{@{}lcccccccccccc@{}}
\toprule
            & \multicolumn{4}{c}{Flickr30K} & \multicolumn{4}{c}{MSCOCO} & \multicolumn{4}{c}{VQAv2}                       \\
 & \multicolumn{2}{c}{Image Retrieval} & \multicolumn{2}{c}{Text Retrieval} & \multicolumn{2}{c}{Image Retrieval} & \multicolumn{2}{c}{Text Retrieval} &  &  &  &  \\
            & R@1   & R@5   & R@1   & R@5   & R@1   & R@5  & R@1  & R@5  & y/n  & number & other   & overall               \\ \midrule
SLIP~\cite{slip}    & 23.3  & 47.2  & 35.7  & 65.8  & 13.2  & 31.3 & 21.0   & 44.6 & 69.8 & 34.3   & 38.1    & 50.7                  \\
MS-CLIP-S~\cite{msclip}  & -     & -     & -     & -     & 19.4  & 40.8 & 28.5 & 54.1 & -    & -      & -       & -                     \\
CLIP~\cite{clip}    & 27.6  & 53.9  & 42.8  & 71.5  & 15.9  & 36.7 & 24.8 & 49.8 & 67.7 & 31.9   & 33.6    & 47.5                  \\
FILIP~\cite{filip}      & 30.6  & 58.2  & 46.3  & 74.4  & 16.2  & 37.5 & 25.6 & 50.8 & 68.1 & 34.5   & 36.2    & 49.2                  \\
DeCLIP~\cite{declip}      & 35.5  & 63.0  & 51.2  & 80.7  & 19.6  & 41.9 & 30.1 & 55.6 & \textbf{70.3} & 34.9   & 36.9    & 50.4                  \\

\midrule
CLIP+FDT (Ours)  & 32.6  & 58.6  & 51.0  & 78.3  & 19.4  & 40.8 & 29.6 & 55.3 & 67.8 & 34.6   & 39.6    & 50.6 \\
% FILIP+FDT (Ours)   & 34.5  & 61.2  & 55.8  & 81.8  & 18.9  & 40.5 & 33.1 & 58.9 &      &        & to-test &                       \\
DECLIP+FDT (Ours)  & \textbf{39.4}  & \textbf{66.8}  & \textbf{57.0}  & \textbf{82.3}  & \textbf{22.5}  & \textbf{45.5} & \textbf{34.0} & \textbf{59.6} & 67.8 & \textbf{35.8}   & \textbf{41.3}    & \textbf{51.6} \\ \bottomrule
\end{tabular}
}

\caption{Results of the vision-language tasks under the 15M setting, including the zero-shot image-text retrieval on the Flickr30K and MSCOCO (5K) datasets, and the non-linear probing on VQA v2 dataset.}

\label{tbl:15m_vl}
\vspace{-10pt}
\end{table*}

\noindent{\textbf{Evaluation Protocols.~}}
Following previous work \cite{declip, filip, declip_benchmark}, our method is evaluated on three commonly-used downstream tasks, including zero-shot image classification, linear probe image classification, and zero-shot image-text retrieval. Moreover, we propose a non-linear probe task to evaluate the effectiveness of the learned features for VQA~\cite{vqa}. The FDT-based features are used for all the downstream tasks.

\noindent\emph{Zero-shot image classification.~} In this task, image categories are represented by the text descriptions generated from their names. After extracting the embeddings of these text descriptions and input images by pre-trained encoders, the category of an image can be predicted by choosing the one whose text descriptions have the largest cosine similarity score. 
Following the setting of CLIP and DeCLIP, we construct 80 prompts to evaluate the performance of different approaches. 
We use 9 of the 11 commonly used datasets \cite{declip} for evaluation. The StanfordCars and Aircraft datasets are not used, because the pre-training datasets contain few captions about car models or aircraft types.

\noindent\emph{Linear Probe Image Classification.~} A linear classifier is trained to predict the categories of images based on the FDT-based features of the images. We use 10 of the 11 commonly
used datasets for evaluation. We do not report the results on ImageNet-1K, since conducting hyperparameter sweeping on this dataset is computationally expensive.

% % The zero-shot image classification and image-text retrieval tasks. For the zero-shot image classification tasks, We use 9 of the 11 commonly used datasets \cite{declip}. The other two datasets, including Standfordcars and Aircraft datasets are no used, becuase the pre-training datasets contain little captions about the model of cars or aircarfts. 
% The prompt setting is followed \cite{declip,clip}.
\noindent\emph{Image-text retrieval.~}  The image-text retrieval task is evaluated on the Flickr30K \cite{f30k} and MSCOCO \cite{mscoco} dataset. The recalls at different K values (R@K, K = 1, 5, 10) are reported as the evaluation metrics. They are used to measure the percentage of relevant items that match the queries in top-K retrieved items.~We also report rsum, which is obtained by summing all R@K values.

\noindent\emph{Non-linear probe task.~} The task is to evaluate the capability of learned features for vision-language reasoning tasks. The FDT-based embeddings of an image and its questions are concatenated and fed to two fully-connected layers with non-linear activation to predict the answer. More details can be found in the supplementary materials.



\begin{table*}[t]
\centering
\resizebox{0.9\linewidth}{!}{
\centering
\begin{tabular}{@{}lccc|ccc|cccc@{}}
\toprule
 &
   &
  ZS CLS &
  LP CLS &
  \multicolumn{3}{c|}{ZS-Flickr30K} &
   &
  ZS-MSCOCO &
   &
  VQAv2 \\
 &
  Setting &
  AVG Acc &
  AVG Acc &
  IR R@1 &
  TR R@1 &
  rsum &
  IR R@1 &
  TR R@1 &
  rsum &
  overall \\ \midrule
CLIP &
  15M &
  41.3 &
  67.5 &
  27.6 &
  42.8 &
  343.1 &
  15.9 &
  24.8 &
  236.8 &
  47.5 \\
CLIP+FDT &
  15M &
  45.9($\uparrow$4.6) &
  68.8($\uparrow$1.3) &
  32.6($\uparrow$5.0) &
  51.0($\uparrow$8.2) &
  376.5($\uparrow$33.4) &
  19.4($\uparrow$3.5) &
  29.6($\uparrow$4.8) &
  263.1($\uparrow$26.3) &
  50.6($\uparrow$3.1) \\ \midrule
CLIP &
  30M &
  56.8 &
  73.8 &
  43.6 &
  58.8 &
  431.3 &
  23.3 &
  34.8 &
  300.8 &
  50.6 \\
CLIP+FDT &
  30M &
  61.2($\uparrow$ 4.4) &
  75.6 ($\uparrow$ 1.8) &
  52.5($\uparrow$8.9) &
  70.8($\uparrow$12.0) &
  474.2($\uparrow$42.9) &
  28.3($\uparrow$5.0) &
  43($\uparrow$8.2) &
  337.1 ($\uparrow$36.3) &
  53.4($\uparrow$2.8) \\ \midrule
CLIP &
  145M &
  64 &
  82.1 &
  52.6 &
  67.9 &
  469.8 &
  29.3 &
  42.1 &
  335.2 &
  53.1 \\
CLIP+FDT &
  145M &
  69.0($\uparrow$ 5.0) &
  82.3 ($\uparrow$ 0.2) &
  56.3($\uparrow$3.7) &
  75.9($\uparrow$8.0) &
  489.4($\uparrow$19.6) &
  31.0($\uparrow$1.7) &
  46.4($\uparrow$4.3) &
  353.0($\uparrow$17.8) &
  55.2($\uparrow$2.1) \\ \bottomrule
\end{tabular}

%no_vqa
% \begin{tabular}{@{}cccc|ccc|ccc@{}}
% \toprule
%      &            & ZS CLS  & LP CLS  & \multicolumn{3}{c|}{ZS-Flickr30K} &        & ZS-MSCOCO &       \\
%      & data-scale & AVG Acc & AVG Acc & IR R@1     & TR R@1    & rsum     & IR R@1 & TR R@1    & rsum  \\ \midrule
% CLIP & 15M        & 41.3    & 67.5    & 27.6       & 42.8      & 343.1    & 15.9   & 24.8      & 236.8 \\
% CLIP+FDT &
%   15M &
%   45.9($\uparrow$4.6) &
%   68.8($\uparrow$1.3) &
%   32.6($\uparrow$5.0) &
%   51.0($\uparrow$8.2) &
%   376.5($\uparrow$33.4) &
%   19.4($\uparrow$3.5) &
%   29.6($\uparrow$4.8) &
%   263.1($\uparrow$26.3) \\ \midrule
% CLIP & 30M        & 56.8    & 73.8    & 43.6       & 58.8      & 431.3    & 23.3   & 34.8      & 300.8 \\
% CLIP+FDT &
%   30M &
%   61.2($\uparrow$ 4.4) &
%   75.6 ($\uparrow$ 1.8) &
%   52.5($\uparrow$8.9) &
%   70.8($\uparrow$12.0) &
%   474.2($\uparrow$42.9) &
%   28.3($\uparrow$5.0) &
%   43($\uparrow$8.2) &
%   337.1 ($\uparrow$36.3) \\ \midrule
% CLIP & 145M       & 64      & 82.1    & 52.6       & 67.9      & 469.8    & 29.3   & 42.1      & 335.2 \\
% CLIP+FDT &
%   145M &
%   69.0($\uparrow$ 5.0) &
%   82.3 ($\uparrow$ 0.2) &
%   56.3($\uparrow$3.7) &
%   75.9($\uparrow$8.0) &
%   489.4($\uparrow$19.6) &
%   31.0($\uparrow$1.7) &
%   46.4($\uparrow$4.3) &
%   353.0($\uparrow$17.8) \\ \bottomrule
% \end{tabular}

}
\caption{Ablation study results when using different scales of training data. ``ZS'' means zero-shot. ``AVG'' is average.  ``ACC'' is accuracy. ``LP'' stands for linear prob.  ``CLS'' represents classification. ``IR'' and ``TR'' are image retrieval and text retrieval, respectively.}
\label{tbl:data_scale}
\vspace{-5pt}
\end{table*}


\begin{table*}[ht!]
\centering
\resizebox{0.9\linewidth}{!}{
% \begin{tabular}{@{}lcc|ccc|ccc@{}}
% \toprule
%               & ZS CLS  & LP CLS  & \multicolumn{3}{c|}{ZS-Flickr30K} &        & ZS-MSCOCO &       \\
%               & AVG Acc & AVG Acc & IR R@1     & TR R@1     & rsum    & IR R@1 & TR R@1    & rsum  \\ \midrule
% CLIP-ViT-B/32 & 41.3    & 67.5    & 27.6       & 42.8       & 15.9    & 24.8   & 24.8      & 236.8 \\
% CLIP-ViT-B/32+FDT &
%   45.9($\uparrow$4.6) &
%   68.8($\uparrow$1.3) &
%   32.6($\uparrow$5.0) &
%   51.0($\uparrow$8.2) &
%   19.4($\uparrow$3.5) &
%   29.6($\uparrow$4.8) &
%   29.6($\uparrow$4.8) &
%   263.1($\uparrow$26.3) \\ \midrule
% CLIP-ViT-B/16 & 45.2    & 68.8    & 35.3       & 50.5       & 19.3    & 29.7   & 34.8      & 300.8 \\
% CLIP-ViT-B/16+FDT &
%   49.9($\uparrow$4.7) &
%   71.3($\uparrow$2.5) &
%   41.6($\uparrow$6.3) &
%   60.8($\uparrow$10.3) &
%   23.4($\uparrow$4.1) &
%   35.3($\uparrow$5.6) &
%   43($\uparrow$8.2) &
%   337.1 ($\uparrow$36.3) \\ \midrule
% CLIP-Swin-B   & 39.6    & 68.5    & 30.5       & 48.5       & 17.7    & 26     & 42.1      & 335.2 \\
% CLIP-Swin-B+FDT &
%   42.4($\uparrow$2.8) &
%   70.7($\uparrow$2.2) &
%   39.6($\uparrow$9.1) &
%   57.9($\uparrow$9.4) &
%   22.3($\uparrow$4.6) &
%   33.8($\uparrow$7.8) &
%   46.4($\uparrow$4.3) &
%   353.0($\uparrow$17.8) \\ \bottomrule
% \end{tabular}
% \begin{tabular}{@{}llllllll@{}}
% \toprule
%               & ZS CLS AVG & LP CLS AVG & ZS-Flickr30K-IR & ZS-Flickr30K-TR & ZS MSCOCO-IR & ZS MSCOCO-TR & VQAv2 \\ \midrule
% CLIP-ViT-B/32 & 41.3       & 67.5       & 27.6            & 42.8            & 15.9         & 24.8         & 47.5  \\
% CLIP-ViT-B/32+FDT &
%   45.9($\uparrow$4.6) &
%   68.8($\uparrow$1.3) &
%   32.6($\uparrow$5.0) &
%   51($\uparrow$8.2) &
%   19.4($\uparrow$3.5) &
%   29.6($\uparrow$4.8) &
%   50.6($\uparrow$3.1) \\ \midrule
% CLIP-ViT-B/16 & 45.2       & 68.8       & 35.3            & 50.5            & 19.3         & 29.7         & 49.2  \\
% CLIP-ViT-B/16+FDT &
%   49.9($\uparrow$4.7) &
%   71.3($\uparrow$2.5) &
%   41.6($\uparrow$6.3) &
%   60.8($\uparrow$10.3) &
%   23.4($\uparrow$4.1) &
%   35.3$\uparrow$5.6) &
%    \\ \midrule
% CLIP-Swin-B   & 39.6       & 68.5       & 30.5            & 48.5            & 17.7         & 26.0           & 46.5  \\
% CLIP-Swin-B+FDT &
%   42.4($\uparrow$2.8) &
%   70.7($\uparrow$2.2) &
%   39.6($\uparrow$9.1) &
%   57.9($\uparrow$9.4) &
%   22.3($\uparrow$4.6) &
%   33.8($\uparrow$7.8) &
%   51.6($\uparrow$5.1)
%    \\ \bottomrule
% \end{tabular}
% \begin{tabular}{@{}lcc|ccc|ccc@{}}
% \toprule
%  &
%   ZS CLS &
%   LP CLS &
%   \multicolumn{3}{c|}{ZS-Flickr30K} &
%    &
%   ZS-MSCOCO &
%    \\
%  &
%   AVG Acc &
%   AVG Acc &
%   IR R@1 &
%   TR R@1 &
%   rsum &
%   IR R@1 &
%   TR R@1 &
%   rsum \\ \midrule
% CLIP-ViT-B/32 &
%   41.3 &
%   67.5 &
%   27.6 &
%   42.8 &
%   343.1 &
%   15.9 &
%   24.8 &
%   236.8 \\
% CLIP-ViT-B/32+FDT &
%   45.9($\uparrow$4.6) &
%   68.8($\uparrow$1.3) &
%   32.6($\uparrow$5.0) &
%   51.0($\uparrow$8.2) &
%   376.5($\uparrow$33.4) &
%   19.4($\uparrow$3.5) &
%   29.6($\uparrow$4.8) &
%   263.1($\uparrow$26.3) \\ \midrule
% CLIP-ViT-B/16 &
%   45.2 &
%   68.8 &
%   35.3 &
%   50.5 &
%   387.8 &
%   19.3 &
%   29.7 &
%   263.6 \\
% CLIP-ViT-B/16+FDT &
%   49.9($\uparrow$4.7) &
%   71.3($\uparrow$2.5) &
%   41.6($\uparrow$6.3) &
%   60.8($\uparrow$10.3) &
%   425.5($\uparrow$37.7) &
%   23.4($\uparrow$4.1) &
%   35.3($\uparrow$5.6) &
%   295.4 ($\uparrow$31.8) \\ \midrule
% CLIP-Swin-B &
%   39.6 &
%   68.5 &
%   30.5 &
%   48.5 &
%   368.1 &
%   17.7 &
%   26.0 &
%   247.6 \\
% CLIP-Swin-B+FDT &
%   42.4($\uparrow$2.8) &
%   70.7($\uparrow$2.2) &
%   39.6($\uparrow$9.1) &
%   57.9($\uparrow$9.4) &
%   415.5($\uparrow$47.4) &
%   22.3($\uparrow$4.6) &
%   33.8($\uparrow$7.8) &
%   288.3($\uparrow$40.7) \\ \bottomrule
% \end{tabular}

\begin{tabular}{lcc|ccc|cccc}
\toprule
 &
  ZS CLS &
  LP CLS &
  \multicolumn{3}{c|}{ZS-Flickr30K} &
   &
  ZS-MSCOCO &
   &
  VQAv2 \\
 &
  AVG Acc &
  AVG Acc &
  IR R@1 &
  TR R@1 &
  rsum &
  IR R@1 &
  TR R@1 &
  rsum &
  Overall \\ \midrule
CLIP-ViT-B/32 &
  41.3 &
  67.5 &
  27.6 &
  42.8 &
  343.1 &
  15.9 &
  24.8 &
  236.8 &
  47.5 \\
CLIP-ViT-B/32+FDT &
  45.9($\uparrow$4.6) &
  68.8($\uparrow$1.3) &
  32.6($\uparrow$5.0) &
  51.0($\uparrow$8.2) &
  376.5($\uparrow$33.4) &
  19.4($\uparrow$3.5) &
  29.6($\uparrow$4.8) &
  263.1($\uparrow$26.3) &
  50.6($\uparrow$3.1) \\ \midrule
CLIP-ViT-B/16 &
  45.2 &
  68.8 &
  35.3 &
  50.5 &
  387.8 &
  19.3 &
  29.7 &
  263.6 &
  49.2 \\
CLIP-ViT-B/16+FDT &
  49.9($\uparrow$4.7) &
  71.3($\uparrow$2.5) &
  41.6($\uparrow$6.3) &
  60.8($\uparrow$10.3) &
  425.5($\uparrow$37.7) &
  23.4($\uparrow$4.1) &
  35.3($\uparrow$5.6) &
  295.4 ($\uparrow$31.8) &
  54.3($\uparrow$5.1) \\ \midrule
CLIP-Swin-B &
  39.6 &
  68.5 &
  30.5 &
  48.5 &
  368.1 &
  17.7 &
  26.0 &
  247.6 &
  46.5 \\
CLIP-Swin-B+FDT &
  42.4($\uparrow$2.8) &
  70.7($\uparrow$2.2) &
  39.6($\uparrow$9.1) &
  57.9($\uparrow$9.4) &
  415.5($\uparrow$47.4) &
  22.3($\uparrow$4.6) &
  33.8($\uparrow$7.8) &
  288.3($\uparrow$40.7) &
  51.6($\uparrow$5.1) \\ \bottomrule
\end{tabular}

}
\caption{Ablation Study results when using different image encoder architectures. ``ZS'' means zero-shot. ``AVG'' is average.  ``ACC'' is accuracy. ``LP'' stands for linear prob.  ``CLS'' represents classification. ``IR'' and ``TR'' are image retrieval and text retrieval.}
\label{tbl:enc_abl}
\vspace{-10pt}
\end{table*}

\noindent{\textbf{Implementation Details.~}} We evaluate our method by incorporating it into two state-of-the-art contrastive vision-language pre-training approaches, namely CLIP \cite{clip} and DECLIP \cite{declip}. Our implementation is based on the open-source PyTorch implementation\footnote{https://github.com/Sense-GVT/DeCLIP} of the two methods. We use 16384 tokens, each with 512 dimensions. Please refer to the supplementary material for detailed information.


\begin{table*}[t]
\centering
\resizebox{0.75\linewidth}{!}{
\begin{tabular}{@{}lcc|ccc|cccc@{}}
\toprule
\multicolumn{1}{l}{} & ZS CLS  & LP CLS  & \multicolumn{3}{c|}{ZS-Flickr30K} &        & ZS-MSCOCO &       & VQAv2   \\
FDT size           & AVG Acc & AVG Acc & IR R@1     & TR R@1    & rsum     & IR R@1 & TR R@1    & rsum  & overall \\ \midrule
-                    & 41.3    & 67.5    & 27.6       & 42.8      & 343.1    & 15.9   & 24.8      & 236.8 & 47.5    \\
8192                 & 42.8    & 67.9    & 32.7       & 50.6      & 374.6    & 18.5   & 29.1      & 258.1 & 50.1    \\
16384                & \textbf{45.9}    & \textbf{68.8}    & 32.6       & \textbf{51.0}        & 376.5    & \textbf{19.4}   & 29.6      & \textbf{263.1} & 50.6    \\
24576                & 45.2    &68.6         & \textbf{33.3}       & 50.4      & \textbf{378.5}    & 18.6   & \textbf{29.7}      & \textbf{263.1} & \textbf{51.4}    \\ \bottomrule
\end{tabular}}
\caption{Results of the models with different FDT sizes. The row whose FDT value is ``-'' represents the original CLIP model. ``ZS'' means zero-shot. ``AVG'' is average.  ``ACC'' is accuracy. ``LP'' stands for linear prob.  ``CLS'' represents classification. ``IR'' and ``TR'' are image retrieval and text retrieval.}
\label{tbl:fdt_size}
\vspace{-5pt}
\end{table*}


\begin{table*}[ht!]
\centering
\resizebox{0.75\linewidth}{!}{
\begin{tabular}{@{}lcc|ccc|ccc|c@{}}
\toprule
                     & ZS CLS  & LP CLS  & \multicolumn{3}{c|}{ZS-Flickr30K} &        & ZS-MSCOCO &       & VQAv2   \\
                     & AVG Acc & AVG Acc & IR R@1     & TR R@1    & rsum     & IR R@1 & TR R@1    & rsum  & overall \\ \midrule
CLIP                 & 41.3    & 67.5    & 27.6       & 42.8      & 343.1    & 15.9   & 24.8      & 236.8 & 47.5    \\
CLIP+FDT$_{\mathrm{Softmax}}$ *   & 5.2     & -       &5.4            &1.7           &45.5          &2.4        &0.8           &26.2       &-         \\
CLIP+FDT$_{\mathrm{Sparsemax}}$ * & 32.4    & -       &10.5            &32.5           &242.4          & 6.0        & 18.3           & 157.5       &-      \\ \midrule
CLIP+FDT$_{\mathrm{Softmax}}$     & 43.9    &68.7         & 33.3       & 47.9      & 377.6    & 19.2   & 28.3      & 258.8 & 47.9    \\
CLIP+FDT$_{\mathrm{Sparsemax}}$   & 45.9    & 68.8    & 32.6       & 51.0      & 376.5    & 19.4   & 29.6      & 263.1 & 50.6    \\ \bottomrule
\end{tabular}
}
\caption{Results of models trained with (Sparsemax) and without (Softmax) sparse constraints. The rows marked with ``*'' are the results when using FDT weights as features (see Section \ref{sec_exp_sparse}). ``ZS'' means zero-shot. ``AVG'' is average.  ``ACC'' is accuracy. ``LP'' stands for linear prob.  ``CLS'' represents classification. ``IR'' and ``TR'' are image retrieval and text retrieval.} 
\label{tbl:sparse}
\vspace{-5pt}
\end{table*}

\subsection{Comparison with State-of-the-Art Approaches}
We compare our method with the state-of-the-art CLIP family approaches on the benchmark proposed in \cite{declip_benchmark}. In this benchmark, methods are compared fairly by pre-training them using the same training recipe and data (our 15M setting).~Note that the original paper only reports the results for zero-shot classification on the ImageNet dataset, and the results of other tasks are obtained by directly applying the released checkpoints for evaluation.

The results for zero-shot image classification, linear prob image classification, and vision-language reasoning tasks are reported in Table \ref{tbl:15m_zs}, \ref{tbl:15m_lp}, and \ref{tbl:15m_vl}, respectively. 
First, we observe that using the proposed FDT-based representation with CLIP (i.e., CLIP+FDT) can achieve significant performance improvement over CLIP on all the downstream tasks. 
Notably, CLIP+FDT can outperform FILIP \cite{filip}, which aligns image and text information at the fine-grained patch and language token levels. The results suggest that aligning global cross-modal information in a unified space is more effective than directly aligning fine-grained patches and language tokens with different granularities.
Interestingly, the linear probe results show that CLIP+FDT can learn a comparable image encoder with DeCLIP, which applies various self-supervised pretext tasks that have already been proven effective for visual recognition. One possible reason is that aligning the information in a unified space helps our model better leverage semantic supervision signals in the language domain. We can also see that our method can significantly improve DeCLIP for all the tasks and achieve state-of-the-art performance on the benchmark. It shows that our approach is compatible with self-supervised learning tasks to improve CLIP. Moreover, FDT can improve the VQAv2 task, which requires the capability of collaborative multi-modal reasoning and content understanding.


% The results for zero-shot image classification, linear prob, and vision language tasks are reported in Table \ref{tbl:15m_zs}, \ref{tbl:15m_lp}, and \ref{tbl:15m_vl}, respectively.  First, we can observe that using the proposed FDT-based representation with CLIP (CLIP+FDT) can achieve significant performance improvement over CLIP on all the downstream tasks. Notably, CLIP+FDT can outperform FILIP \cite{filip}, which aligns image text information at the patch and language token level. \textcolor{red}{The results show that aligning cross-modal information with a unified space at the global level is more effective than at the finer-grained level with an inconsistent gra level.} Interestingly, the results on the linear probe task show CLIP+FDT can learn a comparable image encoder with DeCLIP, which applies various self-supervised tasks that are effective for visual recognition. One possible reason is that aligning the information in a unified space can better leverage semantic supervision signals existing in the language domain. We can also see that our method can also significantly improves DeCLIP for all the tasks, and achieve state-of-the-art performance on the benchmark. It demonstrates that our method is compatible with self-supervised learning tasks to improve CLIP. Moreover, we find that using FDT can improve the VQAv2 tasks which require collaborative multi-modal reasoning and content understanding.


% Notably, using FDT with CLIP can achieve comparable performance with DeCLIP The results indicate that our method can help the CLIP learn better visual encoders.


% can outperform the FILIP and SLIP, which improve CLIP by incorporating self-supervised learning and token-level alignment. Moreover, incorporating our method into DeCLIP can further improve the performance of DeCLIP, and achieve the state of the art performances. 

% We can see that our method consistently improves CLIP and DeCLIP for the linear prob tasks shown in Table \ref{tbl:15m_lp}. Notably, using FDT with CLIP can achieve comparable performance with DeCLIP The results indicate that our method can help the CLIP learn better visual encoders.

% Our model also achieves significant performance for VL tasks.  

% \subsection{Vision Tasks}

% The zero-shot image classification results are reported in Table \ref{tbl:zs_cls}, \ref{}. We can see that our proposed methods can achieve significant performance improvement over the three baseline methods on both the 15M and 30M setting, demonstrating the generality of our method. We also notice that our methods can achieve noticeable improvement for the datasets that requires the model to recognize the fine-grained distinction among classes, such as PETS and Flowers. One possible reason is that our method can help the model to extract fine-grained information.






% \subsection{Visual-Language Tasks}

% We report the results for the zero-shot image-text retrieval tasks on Table \ref{tbl:zs_cls}. The results show that incorporating our methods significantly outperforms the correspondent baseline approaches. We speculate that this is because our proposed method learned better alignment between text and images.  The consistent improvement on the 15M and 30M datasets demonstrate that our method can scale well as dataset size increases.
%




%FILIP       &      &      &      &      &      &      &       \\
%FILIP+Code  &      &      &      &      &      &      &       \\ \hline




\subsection{Ablation Study}
In this section, we conduct ablation studies to investigate how different factors influence the performance of our approach. These factors include the pre-training data scale, image encoder architecture, and several design choices of our method. Throughout the ablation study, we use the CLIP model as the baseline to save computation costs.


\noindent{\textbf{Pretraining Data Scale.}} We evaluate the performance of our methods on different pre-training data scales by further pre-training the model on 30M and 145M data. According to the results presented in Table \ref{tbl:data_scale}, our method still achieves improved performance for all the downstream tasks when pre-trained on larger datasets. We also note that the improvement for the linear probing setting is minor when pre-trained on 145M data. We assume this is because the performance of the model saturates. To further improve the performance of the image encoder, a more vision-specific training task is needed. Note that using FDT still achieves significant performance improvements on 145M data for other tasks. Interestingly, our model achieves significant improvements on the 30M data. One possible reason is that our FDT can benefit significantly from cleaning supervision information in the CC3M \cite{cc3m} and CC12M \cite{cc12m} datasets. We have similar observations for the VQAv2 task.

\noindent{\textbf{Image Encoder Architecture.~}} We evaluate the influence of different image encoder architectures on our proposed method, and the results are reported in Table \ref{tbl:enc_abl}. We observe that our method still significantly outperforms CLIP when using different types of image encoders. Additionally, FDT slightly adds an average of 6\% more parameters, 13\% more training time, and 12\% less throughput when using different encoder architectures. The detailed results can be found in the supplementary materials.

% Please add the following required packages to your document preamble:
% \usepackage{booktabs}

% Interestingly, we find that the improvement is more significant when using Swin-B \cite{swin}.~One possible reason is that the image embedding extracted from Swin-B is largely inconsistent with the text features, since the Swin architecture is designed with an inductive bias for image data \cite{swin}.~Our FDT features can effectively bridge the gap between these features, and enhance performance.

\noindent{\textbf{FDT Numbers.}} The performance of models trained with different learnable token numbers are shown in Table \ref{tbl:fdt_size}. We can see that using 8192 tokens can already achieve an improvement over CLIP. Increasing the FDT size to 16384 obtains a more significant improvement than 8192, since it can encode more types of information. Furthermore, growing the FDT size to 24576 achieves a slight improvement over 16384 for the zero-shot image-text retrieval task on the Flickr30K dataset and VQA task. We set the FDT size as 16384 in our implementation because it achieves the best performance-efficiency tradeoff.


\noindent{\textbf{Sparse Constraints.}} 
\label{sec_exp_sparse}
% We first demonstrate that applying sparse constraints help the model learn better cross-modal correspondence, where the same cross-modal information is represented by using the same subset of FDT. 
In this section, we aim to demonstrate that applying sparse constraints helps the model learn better cross-modal correspondence, where the same cross-modal information is represented using the same subset of FDT. To this end, we evaluate the performance when using the FDT weights (Equation \ref{eq:img_softmax}, \ref{eq:text_softmax} and \ref{eq:sparsemax}) of each image or sentence as the features for zero-shot image classification and image-text retrieval tasks. The results are reported in Table~\ref{tbl:sparse}. From the table, we can see that using sparse constraints (Sparsemax) achieves significantly better performance for all tasks. The results demonstrate that adding sparse constraints to FDT weights can lead to better cross-modal correspondence. 
Additionally, we can also see that without sparse constraints (Softmax), FDT-based features can also achieve significant performance over CLIP. Adding a sparse constraint (Sparsemax) achieves a larger performance improvement. This is because the granularities are further unified by representing the same cross-modal information with the same token set.


% We first compare the code usage ratio, which is the number of tokens whose weights are large than zeros divided by tokens numbers for a image and sentences. We can see that using sparsemax function significantly reduces code weights, which demonstrates that using sparsemax can obtain more sparse code weights. We find that using sparsemax funtion ac


% between the models trained with softmax and sparsemax, 


% We can see that without sparse constraints, the model use all the code to represent an image or text. Using the constraints significantly reduces the code usage by 73.4\%. More importantly, we compare the performance when using the codebook weight $W$ as features. It shows that using sparse constraints achieves significantly better results. It means that using sparse constraints can better align the  usage based on the visual-semantic similarity of image-text pairs.

\begin{figure}[t!]
\begin{center}
    \includegraphics[width=\linewidth]{figures/t2i_case_study.pdf}
\end{center}
\vspace{-10pt}
\caption{Examples shows the top-5 retrieved images for the given text queries for the text-to-image retrieval task on MSCOCO.}
\label{fig:coco_case}
\vspace{-15pt}
\end{figure}


\begin{figure*}[ht!]
\begin{center}
    \includegraphics[width=0.95\linewidth] {figures/code_vis_v2.pdf}
\end{center}
\vspace{-5pt}
\caption{Example of the top-5 most relevant image patches and text tokens of four FDT tokens. Note that the redundant text tokens in the top-5 are removed. The color of the heatmap from blue to red denotes the relevance between patches and FDT from small to large.}
\label{fig:code_vis}
\vspace{-10pt}
\end{figure*}




\subsection{Analysis of the Completeness of Alignment}

Since the granularities of image and text information are inconsistent, the learned model may fail to capture key semantic concepts \cite{learning_concepts}. In this experiment, we empirically evaluate whether unifying the granularities through the proposed FDT can alleviate the problem.  The model pretrained on the 145M dataset is used for this evaluation.

To this end, we design a probing experiment on the MSCOCO dataset. Using the object detection annotations in the training split of MSCOCO, we construct 305,723 sentence pairs. For each sentence pair, one \emph{matched sentence} describes all objects in an image, while the other \emph{partially matched sentence} only captures part of the objects. Please refer to the supplementary material for more details about how we constructed these sentence pairs. 

We then use pre-trained models to extract the embeddings of images and sentences and compute the similarity scores between the images and these constructed sentences. If the learned model comprehensively captures the semantic concepts, the similarity between an image and its matched sentence should be higher than that between the partially matched sentence. We found that the CLIP+FDT models can meet our expectation in 68.2 \% of all sentence pairs, surpassing the CLIP model by 7.6\%. The results demonstrate that FDT can help the CLIP model more comprehensively capture various semantic concepts. We assume that this is because the FDT serve as the prior knowledge that guides encoders to extract cross-modally shared high-level semantic concepts. This not only facilitates cross-modal interactions but also helps encoders capture semantic information from images and texts more comprehensively.

In addition, we show two cases for the text-to-image retrieval task in Figure \ref{fig:coco_case}. We can see that the images retrieved by CLIP ignore some important concepts described in the text queries. For example, in terms of the text query ``baseball players entertaining a crowd of spectators'', four out of the five images retrieved by the CLIP models contain baseball players only but with no spectators. Moreover, the image containing spectators is ranked lower than the two images without spectators. In contrast, FDT can retrieve images that contain both baseball players and spectators. More results are provided in the supplementary material. 


% for the text query ``Three elephants standing in the water'', the top-2 retrieved images contain no water, while the image contains water only ranked in the third and fourth retrieved samples. ``spectators'' and ``water'' 



% Please add the following required packages to your document preamble:
% \usepackage{booktabs}
% \begin{table}[t]
% \caption{Results of models with and without sparse constraints of code weights. ``code usage ratio'' are the mean average ratio of code whose weights are larger than 0 calculated on the MS-COCO dataset. ZS-CLS is the average zero-shot image classification result; and ZS-retrieval is the results of zero-shot image-text retrieval tasks; ``code weights'' means using the code weights as feature for the downstream tasks.}
% \label{tbl:sparse}
% \centering
% \resizebox{1\linewidth}{!}{
% \begin{tabular}{@{}lccccc@{}}
% \toprule
% \multicolumn{1}{c}{} &
%   code usage ratio &
%   \begin{tabular}[c]{@{}c@{}}ZS-CLS  \\ code weight\end{tabular} &
%   \begin{tabular}[c]{@{}c@{}}ZS-retrieval \\ code weights\end{tabular} &
%   ZS-CLS &
%   ZS-retrieval \\ \midrule
% Softmax &
%   100.0 &
%   5.2 &
%   26.2 &
%   43.8 &
%   258.9 \\
% Sparsemax &
%   26.6 &
%   34.4 &
%   157.5 &
%   45.9 &
%   263.1 \\ \bottomrule
% \end{tabular}}
% \end{table}


\subsection{Visualization of Learned FDT}
To explicitly show the cross-modal correspondence learned by our FDT, we visualize the top-5 most relevant image patches and text tokens (using Equation \ref{eq:img_relevance} and \ref{eq:text_relevance}) of four FDT tokens in Figure~\ref{fig:code_vis}. The MSCOCO dataset and the model pretrained on the 145M dataset are used for visualization. The example cases show that each token captures different types of cross-modal correspondence, including actions (jump/jumping), objects, and attributes (orange color). Moreover, the learned FDT can potentially detect correspondent patches from the images. For example, the second token has high relevance values with patches of cats, while having low relevance with other patches. More results can be found in the supplementary material. 


\section{Conclusions}
In this paper, we introduce a new multimodal presentation using finite discrete tokens (FDT). Specifically, a set of learnable tokens shared by all modalities are used to represent multimodal information conveyed in the image and text modalities. %The multimodal inputs are first grounded to a subset of FDT. The combination of the activated FDT embeddings is then used as the representations of the multimodal inputs.
Our approach is a light-weighted way of fulfilling cross-modal interaction, where FDT serves as multimodal anchors to capture information from each input with better completeness. This help alleviate the model degradation problem commonly observed in vanilla CLIP models.~Our FDT can be trained with the contrastive learning scheme from scratch without cold-start problems.~Both quantitative and qualitative results demonstrate that FDT representations achieve better cross-modal alignment and performance on various downstream tasks, including image classification, cross-modal retrieval, and VQA. Additionally, the learned FDT capture meaningful cross-modal correspondence, ranging
from objects to actions and attributes.




% \section{FDT Analysis}
% \subsection{Cross-modal Interaction and Alignment}
% % sparsemax
% % 
% \subsection{Representation Degradation}
% \subsection{Limitations}
% % data and model scale
% % not adding fusion

% \section{Conclusions}

%%%%%%%%% REFERENCES
{\small
\bibliographystyle{ieee_fullname}
\bibliography{egbib}
}

\clearpage
\appendix

%%%%%%%%% TITLE - PLEASE UPDATE
%%%%%%%%% BODY TEXT
\section{Pre-training Implementation Details}
We implement the projecting function that maps patch or language token features to the FDT space as a fully-connected layer with GELU activation (see Section 3.2). Two different projecting functions are applied for mapping patch and language token features, respectively. We regularize the FDT using weight decay, with a rate of 0.1. We set the batch sizes as 4096, 8192, and 32768 when pretraining the models under the 15M, 30M, and 145M settings, respectively. To ensure a fair comparison with the DECLIP~\cite{declip} and FILIP~\cite{filip} models, we use the same data augmentation as these models when training the CLIP and CLIP+FDT models. Consequently, our reported results of the CLIP model on the 15M setting are better than those reported in the 15M benchmark~\cite{declip_benchmark}. We train ViT-B/32 based \cite{vit} models considering our limited computation resource.  The input image resolution is 224 $\times$ 224, and the maximal input language token number is 77. Following \cite{declip_benchmark}, we apply the AdamW optimizer \cite{loshchilov2018decoupled} with a weight decay rate of 0.1 during pre-training. The learning rate is first linearly increased to 0.001 with one epoch for warmup, and then decayed to 0 following the cosine strategy \cite{cosinedecay}.~We use NVIDIA A100 GPUs for pre-training.

\section{Downstream Implementation Details}
\subsection{Downstream Datasets}
\noindent{\textbf{Image Classification Tasks.~}}Following \cite{declip}, we evaluate our method on 11 datasets, including CIFAR-10~\cite{cifar10}, CIFAR-100~\cite{cifar10}, SUN397~\cite{sun397}, Stanford Cars~\cite{cars}, FGVC Aircraft~\cite{aircraft},
Describable Textures~\cite{dtd}, Oxford-IIIT Pets~\cite{pets}, Caltech-101~\cite{caltech}, Oxford Flowers 102~\cite{flower}, Food-101~\cite{food101}, and ImageNet-1K~\cite{imagenet}. 

\noindent{\textbf{Image-Text Retrieval.~}}Our method is tested on two standard benchmarks: Flickr30K~\cite{f30k} and MSCOCO~\cite{mscoco}. For MSCOCO, we report the results on the 5K setting.

\noindent{\textbf{Non-Linear Probe task.~}}We conduct the experiments on the VQAv2 dataset \cite{vqa}. Following the standard protocol \cite{meter}, we train the models with both training and validation data, and test the models on the test-dev set.

\subsection{Implementation Details}
\noindent{\textbf{Zero-shot Image Classification.~}}For a fair comparison, we use the domain-specific prompts and category names proposed by CLIP~\cite{clip}. Note that we do not report the results on the StanfordCars and Aircraft datasets, because the pertaining datasets contain few captions about the category names of these datasets. For example, only 0.04\% and 0\% of descriptions contain aircraft and car category names on the 15M setting.

\noindent{\textbf{Linear Probe Image Classification.~}}We train a logistic regression classifier using L-BFGS, following CLIP \cite{clip}. We set the maximum iterations number to 1,000, and determine the L2 regularization weights following DECLIP's hyperparameter sweeping strategy \cite{declip}. We do not report the results on the ImageNet-1K dataset, due to the high computational cost of conducting hyperparameter sweeping on the dataset.

\noindent{\textbf{Non-linear Probe Task.~}}The downstream task head consists of a fully-connected layer with GELU activation and a fully-connected layer. The extracted FDT features of images and questions are concatenated and then fed to the downstream task head to predict the answers. The encoders and FDT are frozen during the training. The downstream head is optimized by the AdamW optimizer \cite{loshchilov2018decoupled}. We set the learning rate as 0.005, and decay it to 0 following the cosine strategy \cite{cosinedecay}.


\section{Completeness Probing Experiment Details}
Given an image that contains $N$ objects, its \textit{matched sentence}
is ``An photo contains $o_1$, $o_2$ ..., $o_{N-1}$, and $o_N$'', where $o_i$ is the name of the $i$-th object in the images and all the objects are included. For the \textit{partially matched sentence}, we randomly remove an object and use the remaining $N-1$ objects to construct a caption. For example, if the $N$-th object is removed, the partially matched sentence is ``An photo contains $o_1$, $o_2$ ..., and $o_{N-1}$''. We can construct $N$ partially matched sentences for the image, resulting in $N$ \textit{sentence pairs} for the image. In our experiments, we obtain the object presence information of images based on the object detection annotations of the MSCOCO \cite{mscoco} dataset. We construct 305,723 sentence pairs using all images in the MSCOCO training split.

\section{FDT Visualization Details}

We use the model pre-trained on the 145M setting for visualization because it achieves the best performance. To visualize an FDT token, we first calculate its relevance score between patches/language tokens following Equations 4 and 6 without using max-pooing. We then display the relevance scores between the FDT token and the images corresponding to the top-5 most relevant patches, since we find that the patches alone cannot fully convey the object information. We increase the resolution by reducing the patch stride to 4, following the method proposed in \cite{amir2021deep}. For text modality, we show the top-5 most relevant language tokens of the FDT token. 


\section{Additional Experiment Results}

\subsection{Text-to-Image Retrieval Cases}

We further provide five cases for the text-to-image retrieval task in Figure \ref{fig:coco_case_supp}. We have the same observation that the images retrieved by the CLIP+FDT well match the text queries, while those retrieved by the CLIP models often overlook important concepts mentioned in the text queries.

\begin{figure}[t!]
\begin{center}
    \includegraphics[width=1\linewidth]{figures/t2i_supp.pdf}
\end{center}
\vspace{-5pt}
\caption{Examples show the top-5 retrieved images for the given text queries in the text-to-image retrieval task on MSCOCO.}
\label{fig:coco_case_supp}
\end{figure}
\vspace{-5pt}


\subsection{Visualization of Learned FDT}

We present eight learned FDT in Figure \ref{fig:fdt}. These cases further show that FDT can learn meaningful cross-modal correspondence. 

\begin{figure*}[t]
\begin{center}
    \includegraphics[width=0.8\linewidth]{figures/supp_code_v2.pdf}
\end{center}
\vspace{-5pt}
\caption{The top-5 most relevant image patches and text tokens of eight FDT tokens. Note that the redundant text tokens in the top-5 are removed. The color of the heatmap from blue to red denotes the relevance between patches and FDT from small to large.}
\label{fig:fdt}
\end{figure*}
\vspace{-5pt}



\subsection{Pretraining Data Scale}

The results of the models pretrained with different scales of training data are reported in Table \ref{tbl:scale_zs_cls}, \ref{tbl:scale_lp_cls}, \ref{tbl:scale_zs_itr}, and \ref{tbl:scale_vqa}.


\begin{table*}[t!]

\centering
\resizebox{0.6\linewidth}{!}{
\begin{tabular}{@{}ccccccccccc@{}}
\toprule
\multicolumn{1}{l}{} &
  \rotatebox{0}{C10} &
  \rotatebox{0}{C100} &
  \rotatebox{0}{F101} &
  \rotatebox{0}{PETS} &
  \rotatebox{0}{FLOWE} &
  \rotatebox{0}{SUN} &
  \rotatebox{0}{DTD} &
  \rotatebox{0}{CAL} &
  \rotatebox{0}{IN} &
  \rotatebox{0}{AVG} \\ \midrule
\multicolumn{1}{l}{15M}  &      &      &      &      &      &      &      &      &                      &                       \\ \midrule
CLIP                     & 60.4 & 33.5 & 39.6 & 23.1 & 54.0 & 42.0 & 17.0 & 65.5 & 37.0                 & 41.3                  \\
CLIP+FDT                & 67.7 & 39.9 & 42.9 & 25.8 & 55.5 & 45.5 & 26.5 & 69.6 & 39.3                 & 45.9 ($\uparrow$ 4.6) \\ \midrule
\multicolumn{1}{l}{30M}  &      &      &      &      &      &      &      &      & \multicolumn{1}{l}{} & \multicolumn{1}{l}{}  \\ \midrule
CLIP                     & 77.2 & 48.1 & 59.1 & 58.4 & 58.2 & 52.6 & 28.0 & 80.8 & 48.8                 & 56.8                  \\
CLIP+FDT                & 81.9 & 56.5 & 62.6 & 62.3 & 59.5 & 56.7 & 33.6 & 84.8 & 53.3                 & 61.2($\uparrow$ 4.4)  \\ \midrule
\multicolumn{1}{l}{145M} &      &      &      &      &      &      &      &      & \multicolumn{1}{l}{} & \multicolumn{1}{l}{}  \\ \midrule
CLIP                     & 80.9 & 53.9 & 69.1 & 68.9 & 59.3 & 52.1 & 43.0 & 90.1 & 59.0                 & 64.0                  \\
CLIP+FDT                & 87.1 & 63.7 & 73.5 & 77.0 & 65.0 & 56.2 & 47.7 & 90.5 & 60.4                 & 69.0($\uparrow$ 5.0)    \\ \bottomrule
\end{tabular}}
\vspace{-5pt}
\caption{Zero-shot image classification accuracy (\%) when using different scales of training data. The dataset names are abbreviated. C10/100 is CIFAR10/100. F101 is Food101. FLOW is Flowers. CAL is Caltech. IN is ImageNet-1K. ``AVG'' is the average accuracy over all datasets.}
\label{tbl:scale_zs_cls}
\end{table*}


\begin{table*}[t!]
\centering
\resizebox{0.65\linewidth}{!}{
\begin{tabular}{@{}cccccccccccc@{}}
\toprule
\multicolumn{1}{l}{} &
  \rotatebox{0}{C10} &
  \rotatebox{0}{C100} &
  \rotatebox{0}{F101} &
  \rotatebox{0}{PETS} &
  \rotatebox{0}{FLOW} &
  \rotatebox{0}{SUN} &
  \multicolumn{1}{l}{\rotatebox{0}{CARS}} &
  \rotatebox{0}{DTD} &
  \rotatebox{0}{CAL} &
  \rotatebox{0}{AIR} &
  \rotatebox{0}{AVG} \\ \midrule
\multicolumn{1}{l}{15M}  &      &      &      &      &      &      & \multicolumn{1}{l}{} &      &      &                          &                       \\ \midrule
CLIP                     & 88.3 & 68.6 & 72.1 & 72.5 & 92.6 & 69.5 & 29.8                 & 67.8 & 86.2 & 27.7                     & 67.5                  \\
CLIP+FDT                & 89.1 & 71.2 & 74.4 & 73.0   & 93.4 & 70.8 & 31.4                 & 69.4 & 87.7 & 27.9                     & 68.8 ($\uparrow$ 1.3) \\ \midrule
\multicolumn{1}{l}{30M}  &      &      &      &      &      &      & \multicolumn{1}{l}{} &      &      & \multicolumn{1}{l}{}     & \multicolumn{1}{l}{}  \\ \midrule
CLIP                     & 92.0   & 74.7 & 78.8 & 80.7 & 93.7 & 72.6 & 55.9                 & 71.4 & 88.6 & 29.7                     & 73.8                  \\
CLIP+FDT                & 93.8 & 77.8 & 81.6 & 82.6 & 94.5 & 74.3 & 54.4                 & 73.9 & 92.3 & 30.9                     & 75.6 ($\uparrow$ 1.8) \\ \midrule
\multicolumn{1}{l}{145M} &      &      &      &      &      &      & \multicolumn{1}{l}{} &      &      & \multicolumn{1}{l}{}     & \multicolumn{1}{l}{}  \\ \midrule
CLIP                & 95.2 & 80.6 & 86.1 & 87.5 & 96.5 & 76.3 & 87.6                 & 77.2 & 94.7 & 39.5  & 82.1                      \\
CLIP+FDT                     & 94.8 & 80.8 & 85.5 & 85.8 & 95.7 & 75.9 & 88.1                 & 78.5 & 94.6 & 42.9                     & 82.3 ($\uparrow$ 0.2)                 \\\bottomrule
\end{tabular}}
\vspace{-5pt}
\caption{Linear probing image classification accuracy (\%) when using different scales of training data. The dataset names are abbreviated. C10/100 is CIFAR10/100. F101 is Food101. FLOW is Flowers. CAL is Caltech. Air is Aircraft. ``AVG'' is the average accuracy over all datasets.}
\label{tbl:scale_lp_cls}
\end{table*}



\begin{table*}[t!]
\centering
\resizebox{0.85\linewidth}{!}{
\begin{tabular}{@{}lcccccccccccccc@{}}
\toprule
             & \multicolumn{7}{c}{Flickr30K}                   & \multicolumn{7}{c}{MSCOCO}                       \\ \midrule
 &
  \multicolumn{3}{c}{Image Retrieval} &
  \multicolumn{3}{c}{Text Retrieval} &
   &
  \multicolumn{3}{c}{Image Retrieval} &
  \multicolumn{3}{c}{Text Retrieval} &
   \\
             & R@1  & R@5  & R@10 & R@1  & R@5  & R@10 & rsum  & R@1  & R@5  & R@10 & R@1  & R@5  & R@10  & rsum  \\ \midrule
15M setting  &      &      &      &      &      &      &       &      &      &      &      &      &       &       \\ \midrule
CLIP         & 27.6 & 53.9 & 64.4 & 42.8 & 71.5 & 82.9 & 343.1 & 15.9 & 36.7 & 47.8 & 24.8 & 49.8 & 61.8  & 236.8 \\
CLIP + FDT  & 32.6 & 58.6 & 68.5 & 51.0 & 78.3 & 87.5 & 376.5 ($\uparrow$ 33.4) & 19.4 & 40.8 & 51.9 & 29.6 & 55.3 & 66.1  & 263.1 ($\uparrow$ 26.3) \\ \midrule
30M setting  &      &      &      &      &      &      &       &      &      &      &      &      &       &       \\ \midrule
CLIP         & 43.6 & 72.8 & 81.3 & 58.8 & 84.2 & 90.6 & 431.3 & 23.3 & 46.9 & 58.6 & 34.8 & 63.3 & 73.9  & 300.8 \\
CLIP + FDT  & 52.5 & 78.7 & 86.4 & 70.8 & 90.8 & 95.0 & 474.2 ($\uparrow$ 42.9) & 28.3 & 53.3 & 64.3 & 43.0 & 69.0 & 79.2  & 337.1 ($\uparrow$ 36.3)\\ \midrule
145M setting &      &      &      &      &      &      &       &      &      &      &      &      &       &       \\ \midrule
CLIP         & 52.6 & 78.5 & 86.4 & 67.9 & 89.9 & 94.5 & 469.8 & 29.3 & 54.1 & 65.4 & 42.1 & 67.1 & 77.2 & 335.2 \\
CLIP + FDT &
  56.3 &
  80.7 &
  87.6 &
  75.9 &
  93.6 &
  95.3 &
  489.4 ($\uparrow$ 19.6) &
  31.0 &
  55.7 &
  66.7 &
  46.4 &
  71.9 &
  81.3 &
  353.0 ($\uparrow$ 17.8) \\ \bottomrule
\end{tabular}}
\vspace{-5pt}
\caption{Zero-shot image-text retrieval results on the Flickr30K and MSCOCO (5K) datasets when using different scales of training data.}
\label{tbl:scale_zs_itr}
\end{table*}

\begin{table*}[t]
\centering
\resizebox{0.32\linewidth}{!}{
\begin{tabular}{@{}lcccc@{}}
\toprule
             & y/n  & number & other & overall               \\ \midrule
15M setting  &      &        &       &                       \\ \midrule
CLIP         & 67.7 & 31.9   & 33.6  & 47.5                  \\
CLIP + FDT  & 67.8 & 34.6   & 39.6  & 50.6 ($\uparrow$ 3.1) \\ \midrule
30M setting  &      &        &       &                       \\ \midrule
CLIP         & 69.7 & 34.8   & 37.8  & 50.6                  \\
CLIP + FDT  & 68.8 & 36.4   & 42.0  & 53.4 ($\uparrow$ 2.8) \\ \midrule
145M setting &      &        &       &                       \\ \midrule
CLIP         & 70.9 & 36.5   & 41.7  & 53.1                  \\
CLIP + FDT  & 71.5 & 37.9   & 45.2  & 55.2 ($\uparrow$ 2.1) \\ \bottomrule
\end{tabular}}
\vspace{-5pt}
\caption{Results of non-linear probing on VQA v2 dataset when using different scales of training data.}
\label{tbl:scale_vqa}
\end{table*}



\subsection{Image Encoder Architecture}

To evaluate the influence of encoder architectures on our methods, we pre-trained the models with different image encoder architectures. The results for various downstream tasks are reported in in Table \ref{tbl:enc_zs_cls}, \ref{tbl:enc_lp_cls}, \ref{tbl:enc_zs_itr}, and \ref{tbl:enc_vqa}. We also report the computation costs when using different encoder architectures in Table \ref{tbl:cost}.


\begin{table*}[t!]

\centering
\resizebox{0.7\linewidth}{!}{
\begin{tabular}{@{}ccccccccccc@{}}
\toprule
\multicolumn{1}{l}{} &
  \rotatebox{0}{C10} &
  \rotatebox{0}{C100} &
  \rotatebox{0}{F101} &
  \rotatebox{0}{PETS} &
  \rotatebox{0}{FLOW} &
  \rotatebox{0}{SUN} &
  \rotatebox{0}{DTD} &
  \rotatebox{0}{CAL} &
  \rotatebox{0}{IN} &
  \rotatebox{0}{AVG} \\ \midrule


ViT-B/32                     & 60.4 & 33.5 & 39.6 & 23.1 & 54.0 & 42.0 & 17.0 & 65.5 & 37.0                 & 41.3                  \\
ViT-B/32+FDT                & 67.7 & 39.9 & 42.9 & 25.8 & 55.5 & 45.5 & 26.5 & 69.6 & 39.3                 & 45.9 ($\uparrow$ 4.6) \\ \midrule
ViT-B/16                     & 64.6 & 32.1 & 49.7 & 25.7 & 59.7 & 43.4 & 21.3 & 67.9 &  42.1& 45.2                 \\
ViT-B/16+FDT                & 74.0 & 42.1 & 49.4 & 28.5 & 62.2 & 50.5 & 25.1 & 71.4 &                   45.6& 49.9 ($\uparrow$ 4.7)  \\ \midrule

SwinV2-B                     &  58.3& 23.3& 39.3& 20.0&  55.2& 40.1& 18.9& 62.1&     38.9& 39.6                  \\
SwinV2-B+FDT               & 58.9& 26.0& 44.7& 23.8& 55.4& 43.3& 21.4& 66.2& 42.3&   42.4 ($\uparrow$ 2.8)    \\ \bottomrule

\end{tabular}}
\vspace{-5pt}
\caption{Zero-shot image classification accuracy (\%) when using different image encoder architectures. The dataset names are abbreviated. C10/100 is CIFAR10/100. F101 is Food101. FLOW is Flowers. CAL is Caltech. IN is ImageNet-1K. ``AVG'' is the average accuracy over all datasets.}
\label{tbl:enc_zs_cls}

\end{table*}


\begin{table*}[t!]
\centering
\resizebox{0.7\linewidth}{!}{
\begin{tabular}{@{}cccccccccccc@{}}
\toprule
\multicolumn{1}{l}{} &
  \rotatebox{0}{C10} &
  \rotatebox{0}{C100} &
  \rotatebox{0}{F101} &
  \rotatebox{0}{PETS} &
  \rotatebox{0}{FLOW} &
  \rotatebox{0}{SUN} &
  \multicolumn{1}{l}{\rotatebox{0}{CARS}} &
  \rotatebox{0}{DTD} &
  \rotatebox{0}{CAL} &
  \rotatebox{0}{Air} &
  \rotatebox{0}{AVG} \\ \midrule
ViT-B/32                     & 88.3 & 68.6 & 72.1 & 72.5 & 92.6 & 69.5 & 29.8                 & 67.8 & 86.2 & 27.7                     & 67.5                  \\
ViT-B/32+FDT                & 89.1 & 71.2 & 74.4 & 73.0   & 93.4 & 70.8 & 31.4                 & 69.4 & 87.7 & 27.9                     & 68.8 ($\uparrow$ 1.3) \\ \midrule

ViT-B/16                & 89.2& 69.5& 80.3& 75.1& 95.9& 73.4& 33.4& 71.5& 88.3& 32.0& 68.8 \\ 

ViT-B/16+FDT                     &   89.3&  71.6& 82.3& 75.8& 96.1& 74.2&  34.0& 71.8& 88.6&  29.3& 71.3 ($\uparrow$ 2.5)                \\\midrule

SwinV2-B                     & 85.6&  65.1&  78.5&  71.4&  94.3& 72.3&                  30.8&  69.4&  85.9&  32.1& 68.5                  \\
SwinV2-B+FDT                &86.8 & 67.5 & 80.5 & 75.6 & 94.8& 73.1& 33.4&  72.7& 88.9&  34.0& 70.7   ($\uparrow$ 2.2)                   \\ \bottomrule
\end{tabular}}
\vspace{-5pt}
\caption{Linear probing image classification accuracy (\%) when using different image encoder architectures. The dataset names are abbreviated. C10/100 is CIFAR10/100. F101 is Food101. FLOW is Flowers. CAL is Caltech. Air is Aircraft. ``AVG'' is the average accuracy over all datasets.} 
\label{tbl:enc_lp_cls}

\end{table*}


\begin{table*}[t!]
\centering
\resizebox{0.95\linewidth}{!}{
\begin{tabular}{@{}lcccccccccccccc@{}}
\toprule
                & \multicolumn{7}{c}{Flickr30K}                                     & \multicolumn{7}{c}{MSCOCO}                                        \\ \midrule
 & \multicolumn{3}{c}{Image Retrieval} & \multicolumn{3}{c}{Text Retrieval} &  & \multicolumn{3}{c}{Image Retrieval} & \multicolumn{3}{c}{Text Retrieval} &  \\
                & R@1  & R@5  & R@10 & R@1  & R@5  & R@10 & rsum                    & R@1  & R@5  & R@10 & R@1  & R@5  & R@10 & rsum                    \\ \midrule
ViT-B/32        & 27.6 & 53.9 & 64.4 & 42.8 & 71.5 & 82.9 & 343.1                   & 15.9 & 36.7 & 47.8 & 24.8 & 49.8 & 61.8 & 236.8                   \\
ViT-B/32+FDT   & 32.6 & 58.6 & 68.5 & 51.0   & 78.3 & 87.5 & 376.5 ($\uparrow$ 33.4) & 19.4 & 40.8 & 51.9 & 29.6 & 55.3 & 66.1 & 263.1 ($\uparrow$ 26.3) \\ \midrule
ViT-B/16        & 35.3 & 60.6 & 71.7 & 50.5 & 81.1 & 88.6 & 387.8                   & 19.3 & 41.3 & 52.8 & 29.7 & 54.3 & 66.2 & 263.6                   \\
ViT-B/16+FDT & 41.6 & 67.5 & 76.9 & 60.8 & 86.1 & 92.6 & 425.5($\uparrow$ 37.7)  & 23.4 & 46.7 & 58.0   & 35.3 & 60.4 & 71.6 & 295.4($\uparrow$ 31.8)  \\ \midrule
SwinV2-B        & 30.5 & 56.8 & 67.8 & 48.5 & 77.7 & 86.8 & 368.1                   & 17.7 & 38.4 & 49.7 & 26.0   & 52.1 & 63.7 & 247.6                   \\
SwinV2-B+FDT & 39.6 & 65.2 & 74.9 & 57.9 & 85.7 & 92.2 & 415.5($\uparrow$ 47.4)  & 22.3 & 44.9 & 56.2 & 33.8 & 60.1 & 71.0   & 288.3($\uparrow$ 40.7)  \\ \bottomrule
\end{tabular}}
\vspace{-5pt}
\caption{Zero-shot image-text retrieval results on the Flickr30K and MSCOCO (5K) datasets when using different image encoder architectures.}
\label{tbl:enc_zs_itr}
\end{table*}


\begin{table*}[t!]
\centering
\resizebox{0.4\linewidth}{!}{
\begin{tabular}{@{}lcccc@{}}
\toprule
                & y/n  & number & other & overall               \\ \midrule
ViT-B/32        & 67.7 & 31.9   & 33.6  & 47.5                  \\
ViT-B/32 + FDT   & 67.8 & 34.6   & 39.6  & 50.6 ($\uparrow$ 3.1) \\ \midrule
ViT-B/16        & 69.0 & 33.2   & 36.0  & 49.2                  \\
ViT-B/16 + FDT & 72.0   & 37.6   & 42.9  & 54.3($\uparrow$ 5.1)  \\ \midrule
SwinV2-B        & 67.8 & 29.4   & 32.1  & 46.5                  \\
SwinV2-B + FDT & 68.6 & 34.5   & 41.0    & 51.6($\uparrow$ 5.1)  \\ \bottomrule
\end{tabular}
}
\vspace{-5pt}
\caption{Results of non-linear probing on VQA v2 dataset when using different image encoder architectures.}
\label{tbl:enc_vqa}
\end{table*}

\begin{table*}[t!]
\centering
\resizebox{0.5\linewidth}{!}{
\begin{tabular}{@{}lcccc@{}}
\toprule
 & \#param & FLOPs & \begin{tabular}[c]{@{}c@{}}Training time\\ (s/iter)\end{tabular} & \begin{tabular}[c]{@{}c@{}}Inference throughput\\ (image-text pairs/s)\end{tabular} \\ \midrule
CLIP-ViT-B/32     & 151M & 7.3G  & 0.50 &  808.5\\
CLIP-ViT-B/32+FDT & 161M & 9.4G  & 0.60 & 642.8  \\ \midrule
CLIP-ViT-B/16     & 150M & 20.5G & 1.15 & 315.7  \\
CLIP-ViT-B/16+FDT & 160M & 25.1G & 1.29 & 272.5  \\ \midrule
CLIP-Swin-B       & 151M & 18.4G & 1.41 & 258.3  \\
CLIP-Swin-B+FDT   & 161M & 20.5G & 1.51 & 248.1  \\ \bottomrule
\end{tabular}}
\vspace{-5pt}
\caption{Computation cost when using different image encoder architecture.}
\label{tbl:cost}
\end{table*}

\subsection{FDT Number}
The results of models trained with
different FDT numbers are shown in Table \ref{tbl:fdt_zs_cls}, \ref{tbl:fdt_lp_cls}, \ref{tbl:fdt_zs_itr}, and \ref{tbl:fdt_vqa}.

\begin{table*}[t!]
\centering
\resizebox{0.7\linewidth}{!}{
\begin{tabular}{@{}lcccccccccc@{}}
\toprule
FDT size &
  \rotatebox{0}{C10} &
  \rotatebox{0}{C100} &
  \rotatebox{0}{F101} &
  \rotatebox{0}{PETS} &
  \rotatebox{0}{FLOW} &
  \rotatebox{0}{SUN} &
  \rotatebox{0}{DTD} &
  \rotatebox{0}{CAL} &
  \rotatebox{0}{IN} &
  \rotatebox{0}{AVG} \\ \midrule
-     & 60.4          & 33.5          & 39.6          & 23.1          & 54.0 & 42.0          & 17.0          & 65.5          & 37.0          & 41.3          \\
8192  & \textbf{70.4} & \textbf{40.4} & 38.3          & 19.9          & 51.3 & 42.8          & 16.6          & 68.1          & 37.8          & 42.8          \\
16384 & 67.7          & 39.9          & \textbf{42.9} & \textbf{25.8} & 55.5 & \textbf{45.5} & \textbf{26.5} & 69.6          & 39.3          & \textbf{45.9} \\
24576 & 69.0          & 39.1          & 41.9          & 24.2          & \textbf{55.7} & 44.4          & 21.8          & \textbf{70.5} & \textbf{39.8} & 45.2          \\ \bottomrule
\end{tabular}}
\vspace{-5pt}
\caption{Zero-shot image classification accuracy (\%) of models  with different FDT sizes. The row whose FDT value is ``-'' represents the CLIP model. The dataset names are abbreviated. C10/100 is CIFAR10/100. F101 is Food101. FLOW is Flowers. CAL is Caltech. IN is ImageNet-1K. ``AVG'' is the average accuracy over all datasets.}
\label{tbl:fdt_zs_cls}
\end{table*}



\begin{table*}[t!]
\centering
\resizebox{0.7\linewidth}{!}{
\begin{tabular}{@{}cccccccccccc@{}}
\toprule
\multicolumn{1}{l}{FDT size} &
  \rotatebox{0}{C10} &
  \rotatebox{0}{C100} &
  \rotatebox{0}{F101} &
  \rotatebox{0}{PETS} &
  \rotatebox{0}{FLOW} &
  \rotatebox{0}{SUN} &
  \multicolumn{1}{l}{\rotatebox{0}{CARS}} &
  \rotatebox{0}{DTD} &
  \rotatebox{0}{CAL} &
  \rotatebox{0}{Air} &
  \rotatebox{0}{AVG} \\ \midrule
-                     & 88.3 & 68.6 & 72.1 & 72.5 & 92.6 & 69.5 & 29.8                 & 67.8 & 86.2 & 27.7                     & 67.5                  \\
8192  & 89.1 & 70.3 & 72.8 & 70.7 & \textbf{93.4} & 70.1 & 29.6 & 68.5 & 87.2 & 27.5 & 67.9 \\
16384 & 89.1 & \textbf{71.2} & 74.4 & \textbf{73.0} & \textbf{93.4} & \textbf{70.8} & \textbf{31.4} & 69.4 & \textbf{87.7} & 27.9 & \textbf{68.8} \\
24576 & \textbf{89.3} & 71.0 & \textbf{74.9} & 71.2 & \textbf{93.4} & 70.6 & 30.1 & \textbf{69.8} & 87.2 & \textbf{28.7} & 68.6                 \\ \bottomrule
\end{tabular}}
\vspace{-5pt}
\caption{Linear probing image classification accuracy (\%) of models  with different FDT sizes. The row whose FDT value is ``-'' represents the CLIP model. The dataset names are abbreviated. C10/100 is CIFAR10/100. F101 is Food101. FLOW is Flowers. CAL is Caltech. Air is Aircraft. ``AVG'' is the average accuracy over all datasets.}
\label{tbl:fdt_lp_cls}
\end{table*}

\begin{table*}[t!]
\centering
\resizebox{0.8\linewidth}{!}{
\begin{tabular}{@{}lcccccccccccccc@{}}
\toprule
         & \multicolumn{7}{c}{Flickr30K}                   & \multicolumn{7}{c}{MSCOCO}                      \\ \midrule
 &
  \multicolumn{3}{c}{Image Retrieval} &
  \multicolumn{3}{c}{Text Retrieval} &
   &
  \multicolumn{3}{c}{Image Retrieval} &
  \multicolumn{3}{c}{Text Retrieval} &
   \\
FDT size & R@1  & R@5  & R@10 & R@1  & R@5  & R@10 & rsum  & R@1  & R@5  & R@10 & R@1  & R@5  & R@10 & rsum  \\ \midrule
-        & 27.6 & 53.9 & 64.4 & 42.8 & 71.5 & 82.9 & 343.1 & 15.9 & 36.7 & 47.8 & 24.8 & 49.8 & 61.8 & 236.8 \\
8192     & 32.7 & 58.3 & 68.7 & 50.6 & 77.4 & 86.9 & 374.6 & 18.5 & 40.4 & 51.7 & 29.1 & 53.6 & 64.8 & 258.1 \\
16384 &
  32.6 &
  58.6 &
  68.5 &
  \textbf{51.0} &
  \textbf{78.3} &
  \textbf{87.5} &
  376.5 &
  \textbf{19.4} &
  \textbf{40.8} &
  \textbf{51.9} &
  29.6 &
  55.3 &
  66.1 &
  \textbf{263.1} \\
24576 &
  \textbf{33.3} &
  \textbf{60.3} &
  \textbf{70.4} &
  50.4 &
  78.1 &
  86.0 &
  \textbf{378.5} &
  18.6 &
  40.3 &
  51.8 &
  \textbf{29.7} &
  \textbf{55.8} &
  \textbf{66.9} &
  \textbf{263.1} \\ \bottomrule
\end{tabular}

}
\vspace{-5pt}
\caption{Zero-shot image-text retrieval results on the Flickr30K and MSCOCO (5K) datasets of models with different FDT sizes. The row whose FDT value is ``-'' represents the CLIP model.}
\label{tbl:fdt_zs_itr}
\end{table*}


\begin{table*}[t!]
\centering
\resizebox{0.3\linewidth}{!}{
\begin{tabular}{@{}lcccc@{}}
\toprule
FDT size & y/n           & number        & other         & overall       \\ \midrule
-        & 67.7          & 31.9          & 33.6          & 47.5          \\
8192     & 68.1          & 33.3          & 38.5          & 50.1          \\
16384    & 67.8          & 34.6          & 39.6          & 50.6          \\
24576    & \textbf{68.7} & \textbf{35.2} & \textbf{40.3} & \textbf{51.4} \\ \bottomrule
\end{tabular}}
\vspace{-5pt}
\caption{Results of non-linear probing on VQA v2 dataset of models with different FDT sizes. The row whose FDT value is ``-'' represents the CLIP model.}
\label{tbl:fdt_vqa}
\end{table*}



\subsection{Sparse Constraints}
We report the results of the models  trained with and without sparse constraint in Table \ref{tbl:sparse_zs_cls}, \ref{tbl:sparse_lp_cls}, \ref{tbl:sparse_zs_itr}, and \ref{tbl:sparse_vqa}.


\begin{table*}[t!]
\centering
\resizebox{0.8\linewidth}{!}{
\begin{tabular}{@{}lcccccccccc@{}}
\toprule
 &
  \rotatebox{0}{C10} &
  \rotatebox{0}{C100} &
  \rotatebox{0}{F101} &
  \rotatebox{0}{PETS} &
  \rotatebox{0}{FLOW} &
  \rotatebox{0}{SUN} &
  \rotatebox{0}{DTD} &
  \rotatebox{0}{CAL} &
  \rotatebox{0}{IN} &
  \rotatebox{0}{AVG} \\ \midrule
CLIP                              & 60.4 & 33.5 & 39.6 & 23.1 & 54.0 & 42.0 & 17.0 & 65.5 & 37.0 & 41.3 \\
CLIP+FDT$_{\mathrm{Softmax}}$ *   & 23.7 & 1.2  & 4.6  & 2.7  & 1.8  & 3.5  & 4.2  & 4.1  & 1.2  & 5.2  \\
CLIP+FDT$_{\mathrm{Sparsemax}}$ * & 59.9 & 24.7 & 17.3 & 20.9 & 35.1 & 31.2 & 20.8 & 56.8 & 25.0   & 32.4 \\ \midrule
CLIP+FDT$_{\mathrm{Softmax}}$     & 68.7 & 36.9 & 35.5 & 27.9 & 53.8 & 43.8 & 23.1 & 66.6 & 38.6 & 43.9 \\
CLIP+FDT$_{\mathrm{Sparsemax}}$   & 67.7 & 39.9 & 42.9 & 25.8 & 55.5 & 45.5 & 26.5 & 69.6 & 39.3 & 45.6 \\ \bottomrule
\end{tabular}}
\vspace{-5pt}
\caption{Zero-shot image classification accuracy (\%) of models trained with (Sparsemax) and without (Softmax) sparse constraints. The rows marked with ``*'' are the results when using FDT weights as features. The dataset names are abbreviated. C10/100 is CIFAR10/100. F101 is Food101. FLOW is Flowers. CAL is Caltech. IN is ImageNet-1K. ``AVG'' is the average accuracy over all datasets.}
\label{tbl:sparse_zs_cls}
\end{table*}


\begin{table*}[t!]
\centering
\resizebox{0.8\linewidth}{!}{
\begin{tabular}{@{}cccccccccccc@{}}
\toprule
\multicolumn{1}{l}{} &
  \rotatebox{0}{C10} &
  \rotatebox{0}{C100} &
  \rotatebox{0}{F101} &
  \rotatebox{0}{PETS} &
  \rotatebox{0}{FLOW} &
  \rotatebox{0}{SUN} &
  \multicolumn{1}{l}{\rotatebox{0}{CARS}} &
  \rotatebox{0}{DTD} &
  \rotatebox{0}{CAL} &
  \rotatebox{0}{Air} &
  \rotatebox{0}{AVG} \\ \midrule
CLIP                     & 88.3 & 68.6 & 72.1 & 72.5 & 92.6 & 69.5 & 29.8                 & 67.8 & 86.2 & 27.7                     & 67.5                  \\
CLIP+FDT$_{\mathrm{Softmax}}$  & 88.0 & 71.7 & 74.8 & 71.9 & 93.8 & 70.4 & 30.5 & 69.8 & 87.3 & 28.6 & 68.7 \\
CLIP+FDT$_{\mathrm{Sparsemax}}$ & 89.1 & 71.2 & 74.4 & 73.0   & 93.4 & 70.8 & 31.4 & 69.4 & 87.7 & 27.9 & 68.8 \\\bottomrule
\end{tabular}}
\vspace{-5pt}
\caption{Linear probing image classification accuracy (\%) of models  trained with (Sparsemax) and without (Softmax) sparse constraints. The dataset names are abbreviated. C10/100 is CIFAR10/100. F101 is Food101. FLOW is Flowers. CAL is Caltech. Air is Aircraft. ``AVG'' is the average accuracy over all datasets.}
\label{tbl:sparse_lp_cls}
\end{table*}


\begin{table*}[t!]
\centering
\resizebox{0.95\linewidth}{!}{
\begin{tabular}{@{}lcccccccccccccc@{}}
\toprule
                                  & \multicolumn{7}{c}{Flickr30K}                   & \multicolumn{7}{c}{MSCOCO}                      \\ \midrule
 & \multicolumn{3}{c}{Image Retrieval} & \multicolumn{3}{c}{Text Retrieval} &  & \multicolumn{3}{c}{Image Retrieval} & \multicolumn{3}{c}{Text Retrieval} &  \\
FDT size                          & R@1  & R@5  & R@10 & R@1  & R@5  & R@10 & rsum  & R@1  & R@5  & R@10 & R@1  & R@5  & R@10 & rsum  \\ \midrule
CLIP                              & 27.6 & 53.9 & 64.4 & 42.8 & 71.5 & 82.9 & 343.1 & 15.9 & 36.7 & 47.8 & 24.8 & 49.8 & 61.8 & 236.8 \\
CLIP+FDT$_{\mathrm{Softmax}}$ *   & 5.4  & 12.0 & 16.3 & 1.7  & 3.8  & 6.3  & 45.5  & 2.4  & 6.8  & 9.7  & 0.8  & 2.4  & 4.1  & 26.2  \\
CLIP+FDT$_{\mathrm{Sparsemax}}$ * & 10.5 & 29.8 & 39.2 & 32.5 & 59.8 & 70.6 & 242.4 & 6.0  & 16.5 & 24.1 & 18.3 & 40.5 & 52.1 & 157.5 \\ \midrule
CLIP+FDT$_{\mathrm{Softmax}}$     & 33.3 & 60.7 & 69.5 & 47.9 & 78.0   & 88.2 & 377.6 & 19.2 & 40.3 & 51.7 & 28.3 & 53.8 & 65.5 & 258.8 \\
CLIP+FDT$_{\mathrm{Sparsemax}}$   & 32.6 & 58.6 & 68.5 & 51.0 & 78.3 & 87.5 & 376.5 & 19.4 & 40.8 & 51.9 & 29.6 & 55.3 & 66.1 & 263.1 \\ \bottomrule
\end{tabular}}
\vspace{-5pt}
\caption{Zero-shot image-text retrieval results on the Flickr30K and MSCOCO (5K) datasets of models trained with (Sparsemax) and without (Softmax) sparse constraints. The rows marked with ``*'' are the results when using FDT weights as features.}
\label{tbl:sparse_zs_itr}

\end{table*}


\begin{table*}[t]
\centering
\resizebox{0.45\linewidth}{!}{
\begin{tabular}{@{}lcccc@{}}
\toprule
 & y/n           & number        & other         & overall       \\ \midrule
CLIP        & 67.7          & 31.9          & 33.6          & 47.5          \\
CLIP+FDT$_{\mathrm{Softmax}}$     & 65.7          & 31.9          & 36.2          & 47.9          \\
CLIP+FDT$_{\mathrm{Sparsemax}}$    & 67.8          & 34.6          & 39.6          & 50.6 \\ \bottomrule
\end{tabular}}
\vspace{-5pt}
\caption{Results of non-linear probing on VQAv2 dataset of models trained with (Sparsemax) and without (Softmax) sparse constraints.}
\label{tbl:sparse_vqa}
\end{table*}





%%%%%%%%% REFERENCES
% \clearpage
% {\small
% \bibliographystyle{ieee_fullname}
% \bibliography{egbib}
% }


\end{document}
