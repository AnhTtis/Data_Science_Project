% \color{red}
\section{Background and Related Work}\label{sec:related-work}
\subsection{Population Literacy Assessment}
Recognizing the importance of literacy in reducing poverty and expanding lifelong opportunities, the United Nations has included \textit{literacy} as part of its Sustainable Development Goals (Goal Target 4.6)~\cite{mundial2016education,SDG17}. However, tracking population-level literacy development for different demographics globally remains challenging, with most existing datasets incomplete, dated, or costly to obtain~\cite{50Yearso58:online}.

Worldwide, the United Nations Educational, Scientific and Cultural Organization (UNESCO) has been tracking country-level literacy rates for major demographics such as youth, adults, men, and women~\cite{unesco-literacy-data}. However, their data is based on self-declaration of reading and writing skills, often collected by asking the head of the household to answer questions like: ``\textit{Can you (and others in your home) read and write a simple sentence?}'' As a result, the data may overstate actual skills and not capture any notion of functional literacy~\cite{50Yearso58:online}. Even after adding a simple test of reading skills in the data collection process, the results only group people into three big categories -- illiterate, functional literate, and literate -- and cannot measure literacy on a continuum. Despite issues in the UNESCO data, they are still a major reference point, especially for developing countries and regions where the government infrastructures for census and population surveys are scarce. 

In the developed world, many countries have invested significant efforts to develop and implement modern literacy assessments that capture population literacy skills beyond a simple {\it literacy-illiteracy dichotomy}. In the US, the National Adult Literacy Survey (NALS) has been funded by the federal government in 1992 and 2003~\cite{NCES:2002,NCES:2007}. Internationally, there have been coordinated efforts to assess adult literacy skills through programs such as the Program for International Assessment of Adult Competencies (PIAAC), involving 39 countries and regions since its inception in 2012~\cite{PIAAC}. While these assessments provide more granular and contextualized literacy skill measures, they are expensive to administrate and hard to scale: both NALS and PIAAC were conducted once per decade, requiring 8 to 10 months to conduct the surveys and interviews, and a few years to compile the results~\cite{NCES:2002,NCES:2007,PIAAC}.

As a result, the Global Alliance to Monitor Learning has recently made the call to develop ``efficient'', and ``light''  methodologies to gather nuanced, standardized data that allows for cross-national tracking and comparison~\cite{50Yearso58:online}. This work directly responded to this call, by proposing a data-driven method that leverages social media data to estimate the literacy skills of diverse geographic and demographic populations in a cost-effective way with unprecedented coverage. Although our data were collected from only one platform -- Facebook, its high penetration in many parts of the world allows our method to capture the literacy skills of the entire population, especially populations with high Internet adoption.


\subsection{Digital Literacy}
Although closely related to \textit{digital literacy} -- the ability to operate and communicate through digital technology~\cite{hargittai2009update}, language literacy is composed of fundamental language skills such as reading, writing, and numeracy,  that are often a prerequisite for digital literacy~\cite{bach2018poverty,dimaggio2001digital,mckinsey2014}. In fact, research has shown that the lack of language literacy skills is a top barrier to Internet access and technology adoption~\cite{bach2018poverty,dimaggio2001digital,mckinsey2014}. 

As human society enter an increasingly technological and informational-rich age, modern literacy assessment programs such as PIAAC also include the assessment of ``problem-solving skills in technology-rich environment'', showing several demographic differences and similarities in literacy and digital literacy proficiency in the developed countries~\cite{OECD:2013}. For example, while the gender gap in favor of men was observed with digital literacy skills, there is a very small or non-existent gender difference in literacy skills. Similarly, the age gap in favor of young people was more observed with digital literacy than with literacy~\cite{OECD:2013}. The results from PIAAC also showed a significant interaction effect between gender, age, and socio-economic backgrounds on literacy and digital literacy~\cite{OECD:2013}, inspiring us to explore similar trends for the broader global population covered by this research. 

With over 27,000 new Internet users every hour and many of them from traditionally low-resourced regions~\cite{Theriseo30:online}, this work measures and characterizes the language literacy skills of the population that is already online - as captured on Facebook, laying the foundation for future research on more contextualized literacies such as digital literacy and information literacy.


\subsection{Literacy and Social Media}

Most studies of literacy in the social media context focus on youth and their literacy practice. For example, many studies documented the young social media users' practice of ``remixing'' - creating, uploading, selecting, copy-pasting, combining, and co-producing content in their profiles and timelines, noting a new literacy practice that is more collaborative, dynamic, and multi-model than traditional, print literacy~\cite{Erstad-2007, Alvermann-2008, Greenhow-2009, Perkel-2010, Greenhow-2012,Davies-2012}. As a result, the task of ``reading'' has changed significantly, becoming more technically simple yet socially complex while ~\cite{Kress-2003,Erstad-2007}.

%(Social media \& literacy development, e.g. \cite{Black-2008})
Numerous research examined the relationship between social media and literacy development, especially, for socially disadvantaged groups and English Language Learners (ELLs). Most of these works supported the benefits of social media in literacy development. For example, \cite{Clark-2009} found that reading and writing blogs enhanced the confidence in writing for young people in the UK; \cite{Sabaruddin-2019} documented the use of Facebook is positively correlated with improved English skills for college students in Indonesia; and \cite{Black-2008}) argued that social media technologies can support ELLs to develop valuable print literacy, based on longitudinal ethnographic studies of adolescent ELLs literate and social activities around online fandom communities. However, some research also suggested that social media use can negatively impact the reading culture and academic performance of students~\cite{Kojo-2018}.


As misinformation on social media became a public concern~\cite{Vosoughi-2018-science,Edelson-2021}, some recent work highlighted the importance of literacy skills in the social media age. For example, data from the PISA 2018 reading assessment showed that less than 10\% of the 15-year-old students in OECD countries had the reading proficiency level to distinguish facts from opinions, which could significantly impact their abilities to assess the quality and credibility of information spread on social media~\cite{OECD-2021-pisa}. While an increasing amount of attention has been devoted to developing the digital and media literacy of online populations, this work underlines the prevalence of language literacy challenges and calls for future research in understanding the scale and impact of misinformation on low literacy populations.

\color{black}
