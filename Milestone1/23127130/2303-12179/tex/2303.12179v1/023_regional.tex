\subsection{Within-Country Regional Disparity in Online Language Literacy}\label{sec:regional}

While it is widely acknowledged that the disparity in education resources and technology infrastructure has contributed to the digital divide between developed and developing countries~\cite{warf2020geographies,wdr:2016}, there have been only a few studies that examined the digital disparities within a country -- often limited to studying a single country with the digital divide among gender and ethnicity groups~\cite{pew:2012,Hilbert:2011}. Here we intend to provide insights into the within-country regional digital disparities for a large number of countries across the world. Extending our methodology, we measure the within-country regional disparity in online language literacy by quantifying the variability of regional \olles for a given country. More specifically, the variability of regional \olles is calculated as the standard deviation of \olles aggregated across available regions within a country, thus a larger value indicates a higher variability observed in \olles at a sub-national level. We define a region as a sub-national administrative division (self-government or jurisdiction under a country's national laws), such as a state or a province. Countries with less than two regions having a minimum of a thousand active adult Facebook users in our dataset are excluded, which resulted in 119 countries in our analysis. Fig.~\ref{fig:region_map} in \smapp provides summary statistics and a map of the within-country regional disparity in \olle, as well as the representative countries with a relatively high or low level of regional disparity from each geographical group.  

\subsubsection{Within-country regional disparity in \olle is associated with multiple inequality measures}
We examine the societal backdrop for the observed regional disparities in online language literacy. Our particular interest is in the link between the regional disparity in \olle and countries' resource distribution, such as the inequalities in education and income within a country, as well as its overall education and socio-technical development. Using multiple regression analysis, we find that, after controlling for all other variables, inequality in education and the Internet penetration rate have a strong and positive association with regional disparities in \olle (Fig.~\ref{fig:gender_model}C). Not surprisingly, a higher level of overall educational attainment predicts a smaller regional variation in online language literacy skills, which is converse to the effect of inequality in education. The inequality in income, as captured by a country's Gini index, however, appears to have a negative relationship with the \olle regional disparity, indicating a greater income inequality is associated with smaller regional \olle disparity within the country.  

\subsubsection{Inequality paradox}
\yrrr{The interaction between inequalities in education and income is also observed (Fig.~\ref{fig:gender_model}D), where a greater level of within-country regional disparity in \olle is predicted for countries with one of the two conditions: either the country has a relatively high level of income and education inequalities, or has a relatively low level of both inequalities.} This finding suggests an ``{\it inequality paradox}'' -- a paradoxical pattern we notice that links the offline socioeconomic inequalities to online language skill disparity in surprising ways. \yrrr{For example, in countries with a higher level of either economic or educational inequalities, access to social media is more likely to be reserved for the more socio-economically advantaged groups, and therefore show a less regional disparity in \olle (corresponding to the top-left or bottom-right corner of Fig.~\ref{fig:gender_model}D). In contrast, in countries where education inequality is low but economic inequality is high, a higher level of regional disparity in \olle is observed (corresponding to the bottom-right corner of Fig.~\ref{fig:gender_model}D).} Similar patterns are observed when taking the geographical grouping into account, suggesting that the observed patterns are common across societies (see \smapp Tables \ref{tab:model-region-main}--\ref{tab:model-region-4v} for more details).
% The interaction between inequalities in education and income is also observed (Fig.~\ref{fig:gender_model}D), where a greater level of within-country regional disparity in \olle is predicted when a country has either a relatively high or low level of both inequalities. This finding suggests an ``{\it inequality paradox}'' -- a paradoxical pattern we notice that links the offline socioeconomic inequalities to online language skill disparity in surprising ways. For example, in countries with a higher level of economic inequalities, access to social media is more likely to be reserved for the more economically advantaged groups, and therefore show a less regional disparity in \olle. Similar patterns are observed when taking the geographical grouping into account, suggesting that the observed patterns are common across societies (see \smapp Tables \ref{tab:model-region-main}--\ref{tab:model-region-4v} for more details).


% \begin{figure*}[t]
  \centering
    \subfloat[SAN]
{\includegraphics[width=0.1595\linewidth]{figures/fig4/pc_san.pdf}}\hfill
    \subfloat[\textbf{Ours}]
{\includegraphics[width=0.1595\linewidth]{figures/fig4/pc_ours.pdf}}\hfill
     \subfloat[GT]
 {\includegraphics[width=0.1595\linewidth]{figures/fig4/pc_gt.pdf}}\hfill
    \subfloat[SAN]
{\includegraphics[width=0.166\linewidth]{figures/fig4/mess_san.pdf}}\hfill
    \subfloat[\textbf{Ours}]
{\includegraphics[width=0.166\linewidth]{figures/fig4/mess_ours.pdf}}\hfill
    \subfloat[GT]
{\includegraphics[width=0.166\linewidth]{figures/fig4/mess_gt.pdf}}\hfill
\\
\vspace{-10pt}
\caption{\textbf{Qualitative comparison to SAN~\citep{xu2023side}.} We visualize the results of PC-459 dataset in (a-c). For (d-f), we visualize the results from the MESS benchmark~\citep{blumenstiel2023mess} across three domains: underwater (top), human parts (middle), and agriculture (bottom).} 
\label{fig:qualitative}
\vspace{-10pt}
\end{figure*}


