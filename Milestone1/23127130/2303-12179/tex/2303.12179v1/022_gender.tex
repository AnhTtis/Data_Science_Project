\subsection{Gender Difference in Online Language Literacy}\label{sec:gender}

Although the gender gap in literacy has been shrinking globally in recent decades, women are still facing obstacles when accessing school and the Internet~\cite{unesco2017,owidinternet}. As a result, serious male-favoring gender gaps in literacy skills still persist in the Middle East and North Africa, South Asia, and Sub-Saharan Africa regions \cite{world2020global}. An earlier study showed that, in low-income countries, female Internet penetration is 24\% lower than that for males \cite{fatehkia2018using}. To track the gender literacy gap in the online population, we calculate the standardized difference in \olle between female and male Facebook users in each country. This measure captures both the direction and the size of the gender gap in online language literacy, where a positive value indicates a female-favoring gap, and vice versa. To ensure a sufficient sample size of the male and female subpopulations, we drop countries (17 out of 167) with fewer than a thousand adult users in either gender group in our dataset for this analysis. While the gender digital divide has generally referred to the gaps in access and use of digital technology, our measure calls for attention to the disparity in language skills between women and men who are already online. Understanding that gender is non-binary, we only present the female-male gender gap here for two reasons: (a) it enables us to correlate our estimates with existing gender gap data; (b) in our dataset, we do not have sufficient data from users with self-reported non-binary gender information to deliver a reliable estimation for this sub-population.


\subsubsection{Women collectively scored higher than men on Facebook in most countries, with substantial \yrepj{male-favoring} gaps in two regions}
Fig.~\ref{fig:gender_gap}A shows the gender differences in \olle captured in our data.  Among the 160 countries where the gender gap in \olle is calculated, 69 countries (43.1\%) have significant female-favoring gaps and 54 countries (33.8\%) have significant male-favoring gaps. The remaining 37 countries do not have a significant gender gap in \olle. \yrepj{The significance is determined based on whether a country's male-favoring gap (or female-favoring gap) falls above the 95\% confidence limits of the expected male-favoring gaps (or female-favoring gaps).}
Overall, we observe more countries having female-favoring gaps in our measure, suggesting on average higher language literacy skills for women than men among today's online population. 

Fig.~\ref{fig:gender_gap}A highlights countries with the most and least substantial female-favoring gaps. As observed in Fig.~\ref{fig:gender_gap}A, almost all of the countries with the world's largest advanced economies (the G7) have a significant female-favoring gap, with Italy the only exception. \swepj{This finding is generally consistent with the data collected in recent PISA tests, which showed that girls outperformed boys in reading in all the OCED countries and regions~\cite{pisa-2018-gender}. Despite progress, gender-based inequalities are still pervasive in today's society.} To summarize the global state of online literacy difference between male and female subpopulations, a world map of \olle gender gap is provided in \smapp Fig.~\ref{fig:gender_map}. Notably, while most regions on average appear to have female-favoring gaps, two regions (Sub-Saharan Africa and Northern Africa \& Western Asia) still show substantial gaps favoring men. In Sub-Saharan Africa, there are 22 countries with male-favoring gaps, compared to 8 with female-favoring gaps; in Northern Africa \& Western Asia, there are 15 countries with male-favoring gaps,  compared to \yrepj{only} two countries with female-favoring gaps. \swepj{The regional patterns here are generally consistent with what was reported by UNESCO in the official literacy rate data~\cite{unesco2017}. Our measure, however, characterizes more countries with female-favoring literacy gaps than the official data, indicating a potential populational difference between women on Facebook and the general female population in a country. We will further explore the relationship between populational factors and the gender gap in \olle in the next subsection.}

%\subsubsection{Gender gap in OLLE significantly correlates with existing measures}
\subsubsection{Understand the societal context for gender online literacy gap}
We first compare the observed gender gap in \olle with other country-level measures such as overall \olle, income per capita, Gini index, average education attainment, and Internet penetration rate. As shown in Fig.~\ref{fig:gender_gap} 
\yrepj{(B,C,D) the \olle gender gaps are positively correlated with the countries' \olle (Spearman's rank correlation $\rho=0.59$, 95\% CI [0.47,0.68], $p<0.001$), overall education ($\rho=0.59$, 95\% CI [0.44,0.71], $p<0.001$), and Internet penetration ($\rho=0.30$, 95\% CI [0.14, 0.46], $p<0.001$), suggesting that women are disproportionally disadvantaged in low-resourced, low-literacy countries. Fig.~\ref{fig:gender_gap} (E,F,G) show that countries' \olle gender gaps significantly correlate with other gender parity measures, including a positive association with the female-male difference in offline literacy rates ($\rho=0.43$, 95\% CI [0.26, 0.58], $p<0.001$), women's civic participation ($\rho=0.48$, 95\% CI [0.3, 0.62], $p<0.001$), and a negative association with the countries' Gender Inequality Index or GII ($\rho=-0.4$, 95\% CI [-0.57, -0.23], $p<0.001$).}
The GII reflects how women are disadvantaged in multiple dimensions of human development and thus a negative association is expected~\cite{gender-inequality-index}. Interestingly, among all gender parity or empowerment measurements, women's civic participation -- the extent to which women have the ability to express themselves and to participate in civil society \cite{coppedge2019v} -- appears to have the strongest association with the \olle gender gap. This could suggest that the offline structural barriers to women's civic participation are strongly associated with their literacy relative to men in the online space.

%\subsubsection{Gender gap in \olle is associated with various aspects of countries' socio-technical development}
\swepj{To better understand the societal context for online literacy gender gap globally, we further examine the relationship between country-level variables and the observed gender gap in \olle through multiple regression Ordinary Least Squares (OLS) model.} Given that many of the country-level variables are highly correlated (see \smapp Table~\ref{tab:gender_corr}), we only include the most relevant variables in this analysis. Fig.~\ref{fig:gender_model}A summarizes the estimated effect of these variables, where the effect of geographical grouping is further detailed in \smapp (see Tables \ref{tab:model-gender-main}--\ref{tab:model-gender-3v-geo}). 
Based on the OLS estimation, the \olle gender gap remains significantly and positively associated with overall education status and women's civic participation while controlling for other variables.
\yrepj{The overall Internet penetration rate is negatively associated with the \olle gender gap. This may look counterintuitive. One possible interpretation is that a lower level of Internet penetration rate excludes groups from lower socioeconomic status to participate in the digital world, and women in those groups also tend to lack opportunities in education and many other developmental aspects. Hence, a lower level of Internet penetration rate ironically serves as an equalizer for the \olle gender gap.}
Interestingly, the OLS model also reveals an interaction effect between the overall Internet penetration rate and women's civic participation on the \olle gender gap. When a country's Internet penetration rate is high, the country's \olle gap may be either high or low -- depending on whether the country has a high level of women's civic participation (Fig.~\ref{fig:gender_model}B). This could suggest that technological advancements -- i.e., the adoption of the Internet -- are not necessarily associated with more opportunities or higher skills for women relative to men unless such a relationship appears in a society where women have the chance to actively participate in civic processes.



% \begin{figure}[!t]
\centering
\setlength{\abovecaptionskip}{0pt}
\setlength{\belowcaptionskip}{-2pt}
\subfigure[] {\ \ \includegraphics [width=0.47\linewidth]{fig/stats_2_4DCT.pdf}} \ 
\subfigure[] {\includegraphics [width=0.47\linewidth]{fig/stats_1_COPD.pdf}}
\caption{Statistical Analysis of our technique and existing methods. We performed the Friedman test for multiple comparisons along the Wilcoxon test for pair-wise comparison for (a) 4DCT and (b) COPD datasets. }
\label{FigTest}
\vspace{0pt}
\end{figure}


