% \subsection*{Measurement validation}\label{sec:world-validation}
\subsection{Online Language Literacy Worldwide}\label{sec:world}

\subsubsection{Significant agreement between \olles and official literacy data}
Following the methodology detailed in \method, we generate \olle, the calibrated online language literacy estimates, in 167 countries or regions whose representative languages are among the 12 selected languages and have at least a thousand adult Facebook users who posted publicly in the representative language during our data collection period. To ensure we obtain a sufficient sample size of the population in each country, we leave out five countries---China, Iran, Russia, Kazakhstan, and Turkmenistan---where Facebook use is curbed by the countries' government regulations or other policy challenges. We show the country and world population coverage breaking down by languages in \smapp Fig~\ref{fig:countries_covered}. The raw and calibrated values of online language literacy estimates for all 167 countries can be found in \smapp Table~\ref{tab:all_country}. 

Fig~\ref{fig:eval} illustrates the key steps of our methodology, as well as the correlation between \olles and official literacy rate data in~\ref{fig:eval}G. As visualized in Fig.~\ref{fig:eval}G, we find a strong and positive correlation between \olle with the reported literacy rates (Spearman's rank correlation $\rho=0.78$, \yrepj{95\% CI [0.69, 0.84]}, $p<0.001$ based on out-of-sample evaluation). Similar results are found when validating our estimates with global educational attainment statistics: \olle is highly correlated with a country's average schooling years ($\rho=0.78$, \yrepj{95\% CI [0.65, 0.87]}, $p<0.001$; see details in \smapp Fig.~\ref{fig:corr_lit_edu}). These findings indicate that our estimates do reflect populations' literacy skills and can be used as a reliable proxy for official literacy and educational attainment statistics when such data are unavailable or outdated.



\subsubsection{Understand global online literacy inequalities through \olle}
\swepj{One direct application of \olle is to track the state of literacy for online populations across the globe. Mapping \olles by country in Fig.~\ref{fig:lit_map}B, we see significant inequalities in online language literacy skills across geographical regions, with the ``global south'' countries collectively lagging behind in terms of online population literacy skills.}  Fig.~\ref{fig:lit_map}A summarizes the aggregated statistics for seven geographical groups and benchmarks the online literacy gaps across regions. The bottom 10\% of countries with the lowest \olles are primarily located in Sub-Saharan Africa (13 countries), plus one in Latin America \& the Caribbean (Haiti), and one in Northern Africa \& Western Asia (Algeria). \swepj{While our result is consistent with the geographical patterns observed in official literacy rate data~\cite{unesco2017}, it also highlights the persistence of literacy gaps across offline and online populations, calling out for additional literacy support for a substantial percentage of the online population in today's digital environment.}

\swepj{On the other end of the spectrum, the top-ranked countries in terms of \olle are all located in the Europe \& North American region, as well as Oceania, with the top 3 countries being Belarus, Ukraine, and San Marino. While this result is again largely consistent with the official literacy rate data by UNESCO, it also suggests potential biases introduced by language-based calibration. For example, countries with Russian as the representative language (e.g. Belarus, Ukraine) could get an extra boost during calibration due to the overall high literacy rates in Russian-speaking countries reported in the official data.}

