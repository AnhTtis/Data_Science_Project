
\section{Introduction}\label{sec:intro}

Literacy, the ability to comprehend and produce textual information, is known as the foundation for many important personal and social functions. For individuals, the lack of literacy skills is associated with reduced access to education~\cite{NCES:2002,kutner2007literacy,schutz2008education}, employment~\cite{NCES:2002,NCES:2007,kutner2007literacy,ferrer2006effect,bonikowska2008literacy}, social benefits~\cite{schwerdt2018literacy}, as well as poorer health outcomes~\cite{dewalt2005health,OECD:2013} and lower civic engagement \cite{NCES:2002,NCES:2007,OECD:2013,gerger2008}. Collectively, literacy is considered a prerequisite for democracy and socioeconomic development \cite{bonikowska2008literacy,gerger2008}. 

\swepj{Despite a substantial
increase in global literacy rates over recent decades, there were still 750 million adults – two-thirds of whom were women – remaining illiterate in 2016~\cite{unesco2017}. The rise of digital communication technology has brought new challenges to those with limited literacy skills: as more and more public, professional, and social communications shift to the digital, text-mediated environment, a lack of literacy skills can not only exclude people from the information and resources available online but also expose them to greater (mis)informational vulnerability and harms~\cite{mundial2016education,bach2018poverty}. 
%\yrvv{Today, literacy is more essential than ever; it is now a fundamental requirement of communication in an increasingly digital, text-mediated world. Many social and economic activities, services by governments and businesses, as well as information and resources are increasingly available online, but those with poor literacy skills are unable to take advantage of these opportunities \cite{mundial2016education}. 
With most existing literacy programs and research focusing on school children and educational settings, we see a significant gap in our understanding of \emph{literacy practices and challenges in the digital environment}. 
%As of 2019, social media platforms are used by one-in-three people in the world, and more than two-thirds of all Internet users \cite{Theriseo30:online} -- digitally mediated communications are now an ordinary part of many people's lives.}
%To fill in this gap, we need to first characterize the range of \textit{language literacy skills across online populations}. 
In this study, we take a data-driven approach, leveraging the data available on Facebook -- the most widely adopted social media platform with a third of the world's population using it regularly~\cite{Meta_2022Q3_report} - to obtain a representative and up-to-date sample of literacy activities (e.g. reading and writing textual content) by the global online population.}


\swepj{
This study systemically examines the \textit{language literacy skills of online populations} (henceforth called  \textit{online language literacy}) for more than 160 countries and regions around the globe. We introduce a new population-level measure called {\it online language literacy estimate} ({\it \olle}) that is based on aggregated and de-identified written content posted publicly on Facebook. Thanks to the reach of Facebook to hundreds of millions of active users from low-resourced regions such as Africa, Latin American, and South East Asia~\cite{Meta_2022Q3_earnings}, our measure is able to estimate and track population-level language literacy at an unprecedented level of coverage, resolution, and timeliness comparing to traditional literacy assessment methods~\cite{rammstedt2016introduction}, while achieving an overall strong correlation with available official data. With \olle calculated for different gender, country, and regional groups across the world, we capture the disparities in online literacy across broad geographical areas and explore gender and regional literacy gaps under a diverse set of societal contexts. Our results not only quantify the association between online language literacy gaps and offline inequality metrics, but also uncover the complex interaction between literacy, Internet adoption, and civic participation for women. In summary, the main contributions of our work are:}
\swepj{
\begin{itemize}
    \item We develop a global online language literacy estimate (\olle) using Facebook data from over 160 countries in 12 languages. 
    \item We evaluate our measure with existing offline population literacy benchmarks, showing a strong correlation and broader coverage than current official data.
    \item We demonstrate how the online language literacy measure can be used to track gender and regional literacy gaps and unpack the complex societal context around literacy and literacy development.
\end{itemize}
}
\swepj{
The rest of the paper is structured as follows: Sec.~\ref{sec:related-work} offers a literature review of related work to contextualize our study. Sec.~\ref{sec:method} describes our methodology and the dataset used for developing the online language literacy estimate (\olle). Sec.~\ref{sec:results} validates the resulting \olle's with existing literacy assessment data and presents an overview of online language literacy skills across the world. We also share a few applications of \olle in studying and understanding regional and gender inequalities globally. Sec.~\ref{sec:discussion} discusses the implications and limitations of this study, and concludes our work. 
}


% \section*{OLLE: Online Language Literacy Estimate}\label{sec:olle-intro}
