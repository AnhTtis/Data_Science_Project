\addcontentsline{toc}{subsection}{D.}
\subsection{Case study: India as a multilingual country}\label{sec:india}


\yrv{We choose India as a case study for countries using multiple languages to study the effect of choosing the most used language as a single representative language for literacy estimate. While India uses Hindi and English as official languages nationwide, it has no single national language. It has over 30 states/union territories, each of which has its own official language(s). There are 22 official languages recognized by country officials, in addition to some other languages recognized as additional official languages at the regional level. In this analysis, we include the additional five most used languages in India according to the India census reported in 2011 \cite{Censusof39:online}: Hindi (43.6\%), Bengali (8.3\%), Marathi (7.1\%), Telugu (6.9\%), and Tamil (5.9\%). Languages with less than 5\% speakers among the Indian national population are not considered. On Facebook, the most used language in Indian users' public posts is English (en), which has about three times the users posting in Hindi (hi), and about 20, 44, 90, and 181 times those posting in Bengali (bn), Marathi (mr), Telugu (te), and Tamil (ta), respectively. 
Across regions, the non-English language using populations on Facebook are sparse. Only 14 (48.2\%) regions have more than 10\% of the number of English posters posting in Hindi, and only 4 (13.8\%) and 1 (3.4\%) regions have more than 1\% of the number of English posters posting in Bengali and Marathi.} 

\yrv{We then create a language literacy estimation for each of the six languages, using the same approach but include the additional languages (Hindi, Bengali, Marathi, Telugu, and Tamil) from fastText unigram data \cite{grave2018learning}. For validation, we gather the regional literacy survey reported in the India census 2011 \cite{Censusof39:online}, which is the most recent data available. 
Fig.~\ref{fig:india} shows the comparison of our language estimation with the officially reported literacy data. 
\yrrr{We first estimate the language literacy for every language. Fig.~\ref{fig:india} A and C-E show the estimation based on posts in a single language. Note that, while there are regional differences, the use of Hindi, Bengali, and Marathi is extremely sparse in most regions. Therefore, the non-English language estimates alone cannot be directly used to create a regional measurement. Due to the sparse use of non-English language on the platform, the correlation between the non-English language estimates and the reported literacy is insignificant. We additionally create a multi-language estimation weighted by the popularity of each language within a region, as shown in Fig~\ref{fig:india} B.}
%We first estimate the language literacy for every language, and additionally create a multi-language estimation weighted by the popularity of each language within a region. Fig.~\ref{fig:india} A shows the estimation based on posts in English only, and Fig.~\ref{fig:india} B shows the estimation based on multiple languages. 
We observe that both English-only and multi-language estimates (without any additional calibration) have significant correlations with the reported literacy data (Spearman's rank correlations with positive 95\% CIs and $p<0.005$). However, literacy estimation based on multiple languages has neglectable improvement over English-based estimation in terms of the correlations -- from 0.51 to 0.52. This is likely due to the low rate of users posting in non-English languages in many regions. \yrrr{Here, the comparison relies on the officially reported literacy data, which has a limitation: they do not capture the change in regional literacy levels since 2011, and likely do not properly reflect the diverse language skills used by minority populations.} 
%Another potential reason is that officially reported literacy data do not capture the change of regional literacy levels since 2011, or may not properly reflect the diverse language skills used by minority populations.
This case study illustrates the challenge of obtaining gold-standard literacy measures for multilingual countries. While this does not prevent us to create a multi-language literacy measurement per country, for validation purposes, we simply choose a single representative language for multilingual countries. Thus the correlation should be interpreted with caution -- the officially reported data that guide this choice often has a bias against language minorities.}
