%% BioMed_Central_Tex_Template_v1.06
%%                                      %
%  bmc_article.tex            ver: 1.06 %
%                                       %

%%IMPORTANT: do not delete the first line of this template
%%It must be present to enable the BMC Submission system to
%%recognise this template!!

%%%%%%%%%%%%%%%%%%%%%%%%%%%%%%%%%%%%%%%%%
%%                                     %%
%%  LaTeX template for BioMed Central  %%
%%     journal article submissions     %%
%%                                     %%
%%          <8 June 2012>              %%
%%                                     %%
%%                                     %%
%%%%%%%%%%%%%%%%%%%%%%%%%%%%%%%%%%%%%%%%%


%%%%%%%%%%%%%%%%%%%%%%%%%%%%%%%%%%%%%%%%%%%%%%%%%%%%%%%%%%%%%%%%%%%%%
%%                                                                 %%
%% For instructions on how to fill out this Tex template           %%
%% document please refer to Readme.html and the instructions for   %%
%% authors page on the biomed central website                      %%
%% http://www.biomedcentral.com/info/authors/                      %%
%%                                                                 %%
%% Please do not use \input{...} to include other tex files.       %%
%% Submit your LaTeX manuscript as one .tex document.              %%
%%                                                                 %%
%% All additional figures and files should be attached             %%
%% separately and not embedded in the \TeX\ document itself.       %%
%%                                                                 %%
%% BioMed Central currently use the MikTex distribution of         %%
%% TeX for Windows) of TeX and LaTeX.  This is available from      %%
%% http://www.miktex.org                                           %%
%%                                                                 %%
%%%%%%%%%%%%%%%%%%%%%%%%%%%%%%%%%%%%%%%%%%%%%%%%%%%%%%%%%%%%%%%%%%%%%

%%% additional documentclass options:
%  [doublespacing]
%  [linenumbers]   - put the line numbers on margins

%%% loading packages, author definitions

%\documentclass[twocolumn]{bmcart}% uncomment this for twocolumn layout and comment line below
\documentclass{bmcart}

%%% Load packages
%\usepackage{amsthm,amsmath}
%\RequirePackage{natbib}
%\RequirePackage[authoryear]{natbib}% uncomment this for author-year bibliography
%\RequirePackage{hyperref}
\usepackage[utf8]{inputenc} %unicode support
%\usepackage[applemac]{inputenc} %applemac support if unicode package fails
%\usepackage[latin1]{inputenc} %UNIX support if unicode package fails


%%%%%%%%%%%%%%%%%%%%%%%%%%%%%%%%%%%%%%%%%%%%%%%%%
%%                                             %%
%%  If you wish to display your graphics for   %%
%%  your own use using includegraphic or       %%
%%  includegraphics, then comment out the      %%
%%  following two lines of code.               %%
%%  NB: These line *must* be included when     %%
%%  submitting to BMC.                         %%
%%  All figure files must be submitted as      %%
%%  separate graphics through the BMC          %%
%%  submission process, not included in the    %%
%%  submitted article.                         %%
%%                                             %%
%%%%%%%%%%%%%%%%%%%%%%%%%%%%%%%%%%%%%%%%%%%%%%%%%


\def\includegraphic{}
\def\includegraphics{}



%%% Put your definitions there:
\startlocaldefs
\endlocaldefs

\usepackage{array}
\usepackage{caption}
\usepackage{graphicx}
\usepackage{siunitx}
\usepackage{colortbl}
\usepackage{multirow}
\usepackage{hhline}
\usepackage{calc}
\usepackage{tabularx}
\usepackage{threeparttable}
\usepackage{xspace}
% \usepackage[dvipsnames]{xcolor}
\usepackage{colortbl}
% \usepackage[colorlinks=true,linkcolor=blue,citecolor=black,urlcolor=black]{hyperref}
\PassOptionsToPackage{hyphens}{url}
\usepackage{hyperref}
\usepackage{longtable}
\usepackage{booktabs}
\usepackage{nameref}
% \usepackage{wrapfig}
\usepackage{amsmath}

\hypersetup{
	colorlinks,%
	citecolor=black,%
	filecolor=cyan,%
	linkcolor=black,%
	urlcolor=black
}

\PassOptionsToPackage{numbers/authoryear}{natbib}

\usepackage{soul}
\usepackage[colorinlistoftodos,prependcaption,textsize=tiny]{todonotes}
\setlength{\marginparwidth}{1.5cm}
\usepackage[normalem]{ulem}

\AtBeginDocument{
  \let\namerefOld\nameref
  \renewcommand{\nameref}[1]{{\bf  \color{red}{\namerefOld{#1}}}}
}

% \newcommand{\yrl}[1]{\todo[linecolor=olive,backgroundcolor=white,bordercolor=purple]{YRL:#1}}
\newcommand{\sw}[1]{\todo[linecolor=olive,backgroundcolor=white,bordercolor=purple]{SW:#1}}
\newcommand{\wm}[1]{\todo[linecolor=olive,backgroundcolor=white,bordercolor=purple]{WM:#1}}
\newcommand{\yrl}[1]{{\textcolor{green}{[YRL: #1]}}}

\def \check#1{{}}
% \def \check#1{{\color{orange}{\textbf{[#1]}} }}
\newcommand{\rev}[1]{{\textcolor{blue}{#1}}}
\newcommand{\del}[1]{{\textcolor{lightgray}{#1}}} %% text to be deleted/revised
\newcommand{\toref}{\xspace{\textcolor{red}{REF?}}\xspace} %% text to be deleted/revised
\newcommand{\yrv}[1]{{\textcolor{black}{#1}}}
\newcommand{\yrvv}[1]{{\textcolor{black}{#1}}}
\newcommand{\yrepj}[1]{{\textcolor{black}{#1}}}
\newcommand{\swepj}[1]{{\textcolor{black}{#1}}}
\newcommand{\yrrr}[1]{{\textcolor{black}{#1}}}
\newcommand{\swvv}[1]{{\textcolor{black}{#1}}}
\newcommand{\wam}[1]{{\textcolor{black}{#1}}}

% \newcommand{\yrvv}[1]{\color{blue}{#1}\color{black}}
\newenvironment{yrvp}{\par\color{red}}{\par}

\newcommand{\method}{Section~\ref{sec:method}\xspace}
% \newcommand{\method}{{\textcolor{blue}{\it Methods}}\xspace}
\newcommand{\smapp}{{\textcolor{blue}{\it Supplemental Material}}\xspace}
% \newcommand{\smappsec}[1]{{\textcolor{blue}{{\it Supplemental Material}: Section #1}}\xspace}
\newcommand{\smappsec}[1]{{\textcolor{blue}{{\it Supplemental Material}: \ref{#1}}\xspace}}

% \newcommand{\method}{\nameref{sec:method}\xspace}
\newcommand{\supp}{\nameref{sec:supp}\xspace}
% \newcommand{\method}{{\bf Methods and Materials}\xspace}
% \newcommand{\supp}{{\bf Supplementary Information}\xspace}
\newcommand{\olle}{{\tt\small \color{black} OLLE}\xspace}
\newcommand{\olles}{{\tt\small \color{black} OLLE}'s\xspace}
% \newcommand{\bigword}{{\color{red} lower-frequency words}\xspace}
% \newcommand{\Bigword}{{\color{red} Lower-frequency words}\xspace}
% \newcommand{\bigw}{{\color{red} lower-frequency word}\xspace}

\newcommand{\bigword}{{\color{black} LoFF words}\xspace}
\newcommand{\Bigword}{{\color{black} LoFF words}\xspace}
\newcommand{\bigw}{{\color{black} LoFF word}\xspace}


%%% cross-ref to external file
%% @see [Cross referencing with the xr package in Overleaf - Overleaf, Online LaTeX Editor](https://www.overleaf.com/learn/how-to/Cross_referencing_with_the_xr_package_in_Overleaf) 

\usepackage{xr-hyper}

\makeatletter
\newcommand*{\addFileDependency}[1]{% argument=file name and extension
  \typeout{(#1)}
  \@addtofilelist{#1}
  \IfFileExists{#1}{}{\typeout{No file #1.}}
}
\makeatother

\newcommand*{\myexternaldocument}[1]{%
    \externaldocument{#1}%
    \addFileDependency{#1.tex}%
    \addFileDependency{#1.aux}%
}



% \myexternaldocument{01_supp}

%%% Begin ...
\begin{document}

%%% Start of article front matter
\begin{frontmatter}

\begin{fmbox}
\dochead{Research}

%%%%%%%%%%%%%%%%%%%%%%%%%%%%%%%%%%%%%%%%%%%%%%
%%                                          %%
%% Enter the title of your article here     %%
%%                                          %%
%%%%%%%%%%%%%%%%%%%%%%%%%%%%%%%%%%%%%%%%%%%%%%

\newcommand{\fulltitle}{Mapping Language Literacy At Scale: A Case Study on Facebook}
\title{\fulltitle}

%%%%%%%%%%%%%%%%%%%%%%%%%%%%%%%%%%%%%%%%%%%%%%
%%                                          %%
%% Enter the authors here                   %%
%%                                          %%
%% Specify information, if available,       %%
%% in the form:                             %%
%%   <key>={<id1>,<id2>}                    %%
%%   <key>=                                 %%
%% Comment or delete the keys which are     %%
%% not used. Repeat \author command as much %%
%% as required.                             %%
%%                                          %%
%%%%%%%%%%%%%%%%%%%%%%%%%%%%%%%%%%%%%%%%%%%%%%

\author[
   addressref={aff1},                   % id's of addresses, e.g. {aff1,aff2}
   corref={aff1},                       % id of corresponding address, if any
%   noteref={n1},                        % id's of article notes, if any
   email={yurulin@pitt.edu}   % email address
]{\inits{YRL}\fnm{Yu-Ru} \snm{Lin}}
\author[
   addressref={aff2},
   email={xx@fb.com}
]{\inits{SW}\fnm{Shaomei} \snm{Wu}}
\author[
   addressref={aff3},
   email={xx@fb.com}
]{\inits{WM}\fnm{Winter} \snm{Mason}}

%%%%%%%%%%%%%%%%%%%%%%%%%%%%%%%%%%%%%%%%%%%%%%
%%                                          %%
%% Enter the authors' addresses here        %%
%%                                          %%
%% Repeat \address commands as much as      %%
%% required.                                %%
%%                                          %%
%%%%%%%%%%%%%%%%%%%%%%%%%%%%%%%%%%%%%%%%%%%%%%

\address[id=aff1]{%                           % unique id
  \orgname{School of Computing and Information, University of Pittsburgh}, % university, etc
%   \street{Waterloo Road},                     %
  %\postcode{}                                % post or zip code
  \city{Pittsburgh, PA},                              % city
  \cny{USA}                                    % country
}
\address[id=aff2]{%
  \orgname{AImpower.org},
  \city{Mountain View, CA},
  \cny{USA}
}
\address[id=aff3]{%
  \orgname{Meta},
  \city{Menlo Park, CA},
  \cny{USA}
}

%%%%%%%%%%%%%%%%%%%%%%%%%%%%%%%%%%%%%%%%%%%%%%
%%                                          %%
%% Enter short notes here                   %%
%%                                          %%
%% Short notes will be after addresses      %%
%% on first page.                           %%
%%                                          %%
%%%%%%%%%%%%%%%%%%%%%%%%%%%%%%%%%%%%%%%%%%%%%%

% \begin{artnotes}
% %\note{Sample of title note}     % note to the article
% \note[id=n1]{Equal contributor} % note, connected to author
% \end{artnotes}

\end{fmbox}% comment this for two column layout

%%%%%%%%%%%%%%%%%%%%%%%%%%%%%%%%%%%%%%%%%%%%%%
%%                                          %%
%% The Abstract begins here                 %%
%%                                          %%
%% Please refer to the Instructions for     %%
%% authors on http://www.biomedcentral.com  %%
%% and include the section headings         %%
%% accordingly for your article type.       %%
%%                                          %%
%%%%%%%%%%%%%%%%%%%%%%%%%%%%%%%%%%%%%%%%%%%%%%
\begin{abstractbox}

\begin{abstract} % abstract
% \parttitle{First part title} %if any
% Text for this section.

% \parttitle{Second part title} %if any
\swepj{Literacy is one of the most fundamental skills for people to access and navigate today's digital environment.} 
\swepj{This work systematically studies the language literacy skills of online populations for more than 160 countries and regions across the world, including many low-resourced countries where official literacy data are particularly sparse.} Leveraging public data on Facebook, we develop a population-level literacy estimate for the online population that is based on aggregated and de-identified public posts written by adult Facebook users globally, significantly improving both the coverage and resolution of existing literacy tracking data. 
We found that, on Facebook, women collectively show higher language literacy than men in many countries, but substantial gaps remain in Africa and Asia. Further, our analysis reveals a considerable regional gap within a country that is associated with multiple socio-technical inequalities, suggesting an ``inequality paradox'' -- where the online language skill disparity interacts with offline socioeconomic inequalities in complex ways. These findings have implications for global women's empowerment and socioeconomic inequalities. 
\end{abstract}

%%%%%%%%%%%%%%%%%%%%%%%%%%%%%%%%%%%%%%%%%%%%%%
%%                                          %%
%% The keywords begin here                  %%
%%                                          %%
%% Put each keyword in separate \kwd{}.     %%
%%                                          %%
%%%%%%%%%%%%%%%%%%%%%%%%%%%%%%%%%%%%%%%%%%%%%%

\begin{keyword}
\kwd{global literacy}
\kwd{global inequalities}
\kwd{social media demography}
\kwd{information accessibility}
\kwd{cross-language measurement}
\end{keyword}


\end{abstractbox}


%
%\end{fmbox}% uncomment this for twcolumn layout

\end{frontmatter}

%%%%%%%%%%%%%%%%%%%%%%%%%%%%%%%%%%%%%%%%%%%%%%
%%                                          %%
%% The Main Body begins here                %%
%%                                          %%
%% Please refer to the instructions for     %%
%% authors on:                              %%
%% http://www.biomedcentral.com/info/authors%%
%% and include the section headings         %%
%% accordingly for your article type.       %%
%%                                          %%
%% See the Results and Discussion section   %%
%% for details on how to create sub-sections%%
%%                                          %%
%% use \cite{...} to cite references        %%
%%  \cite{koon} and                         %%
%%  \cite{oreg,khar,zvai,xjon,schn,pond}    %%
%%  \nocite{smith,marg,hunn,advi,koha,mouse}%%
%%                                          %%
%%%%%%%%%%%%%%%%%%%%%%%%%%%%%%%%%%%%%%%%%%%%%%


%%%%%%%%%%%%%%%%%%%%%%%%% start of article main body
% <put your article body there>


\section{Introduction}\label{sec:intro}

Literacy, the ability to comprehend and produce textual information, is known as the foundation for many important personal and social functions. For individuals, the lack of literacy skills is associated with reduced access to education~\cite{NCES:2002,kutner2007literacy,schutz2008education}, employment~\cite{NCES:2002,NCES:2007,kutner2007literacy,ferrer2006effect,bonikowska2008literacy}, social benefits~\cite{schwerdt2018literacy}, as well as poorer health outcomes~\cite{dewalt2005health,OECD:2013} and lower civic engagement \cite{NCES:2002,NCES:2007,OECD:2013,gerger2008}. Collectively, literacy is considered a prerequisite for democracy and socioeconomic development \cite{bonikowska2008literacy,gerger2008}. 

\swepj{Despite a substantial
increase in global literacy rates over recent decades, there were still 750 million adults – two-thirds of whom were women – remaining illiterate in 2016~\cite{unesco2017}. The rise of digital communication technology has brought new challenges to those with limited literacy skills: as more and more public, professional, and social communications shift to the digital, text-mediated environment, a lack of literacy skills can not only exclude people from the information and resources available online but also expose them to greater (mis)informational vulnerability and harms~\cite{mundial2016education,bach2018poverty}. 
%\yrvv{Today, literacy is more essential than ever; it is now a fundamental requirement of communication in an increasingly digital, text-mediated world. Many social and economic activities, services by governments and businesses, as well as information and resources are increasingly available online, but those with poor literacy skills are unable to take advantage of these opportunities \cite{mundial2016education}. 
With most existing literacy programs and research focusing on school children and educational settings, we see a significant gap in our understanding of \emph{literacy practices and challenges in the digital environment}. 
%As of 2019, social media platforms are used by one-in-three people in the world, and more than two-thirds of all Internet users \cite{Theriseo30:online} -- digitally mediated communications are now an ordinary part of many people's lives.}
%To fill in this gap, we need to first characterize the range of \textit{language literacy skills across online populations}. 
In this study, we take a data-driven approach, leveraging the data available on Facebook -- the most widely adopted social media platform with a third of the world's population using it regularly~\cite{Meta_2022Q3_report} - to obtain a representative and up-to-date sample of literacy activities (e.g. reading and writing textual content) by the global online population.}


\swepj{
This study systemically examines the \textit{language literacy skills of online populations} (henceforth called  \textit{online language literacy}) for more than 160 countries and regions around the globe. We introduce a new population-level measure called {\it online language literacy estimate} ({\it \olle}) that is based on aggregated and de-identified written content posted publicly on Facebook. Thanks to the reach of Facebook to hundreds of millions of active users from low-resourced regions such as Africa, Latin American, and South East Asia~\cite{Meta_2022Q3_earnings}, our measure is able to estimate and track population-level language literacy at an unprecedented level of coverage, resolution, and timeliness comparing to traditional literacy assessment methods~\cite{rammstedt2016introduction}, while achieving an overall strong correlation with available official data. With \olle calculated for different gender, country, and regional groups across the world, we capture the disparities in online literacy across broad geographical areas and explore gender and regional literacy gaps under a diverse set of societal contexts. Our results not only quantify the association between online language literacy gaps and offline inequality metrics, but also uncover the complex interaction between literacy, Internet adoption, and civic participation for women. In summary, the main contributions of our work are:}
\swepj{
\begin{itemize}
    \item We develop a global online language literacy estimate (\olle) using Facebook data from over 160 countries in 12 languages. 
    \item We evaluate our measure with existing offline population literacy benchmarks, showing a strong correlation and broader coverage than current official data.
    \item We demonstrate how the online language literacy measure can be used to track gender and regional literacy gaps and unpack the complex societal context around literacy and literacy development.
\end{itemize}
}
\swepj{
The rest of the paper is structured as follows: Sec.~\ref{sec:related-work} offers a literature review of related work to contextualize our study. Sec.~\ref{sec:method} describes our methodology and the dataset used for developing the online language literacy estimate (\olle). Sec.~\ref{sec:results} validates the resulting \olle's with existing literacy assessment data and presents an overview of online language literacy skills across the world. We also share a few applications of \olle in studying and understanding regional and gender inequalities globally. Sec.~\ref{sec:discussion} discusses the implications and limitations of this study, and concludes our work. 
}


% \section*{OLLE: Online Language Literacy Estimate}\label{sec:olle-intro}

% \color{red}
\section{Background and Related Work}\label{sec:related-work}
\subsection{Population Literacy Assessment}
Recognizing the importance of literacy in reducing poverty and expanding lifelong opportunities, the United Nations has included \textit{literacy} as part of its Sustainable Development Goals (Goal Target 4.6)~\cite{mundial2016education,SDG17}. However, tracking population-level literacy development for different demographics globally remains challenging, with most existing datasets incomplete, dated, or costly to obtain~\cite{50Yearso58:online}.

Worldwide, the United Nations Educational, Scientific and Cultural Organization (UNESCO) has been tracking country-level literacy rates for major demographics such as youth, adults, men, and women~\cite{unesco-literacy-data}. However, their data is based on self-declaration of reading and writing skills, often collected by asking the head of the household to answer questions like: ``\textit{Can you (and others in your home) read and write a simple sentence?}'' As a result, the data may overstate actual skills and not capture any notion of functional literacy~\cite{50Yearso58:online}. Even after adding a simple test of reading skills in the data collection process, the results only group people into three big categories -- illiterate, functional literate, and literate -- and cannot measure literacy on a continuum. Despite issues in the UNESCO data, they are still a major reference point, especially for developing countries and regions where the government infrastructures for census and population surveys are scarce. 

In the developed world, many countries have invested significant efforts to develop and implement modern literacy assessments that capture population literacy skills beyond a simple {\it literacy-illiteracy dichotomy}. In the US, the National Adult Literacy Survey (NALS) has been funded by the federal government in 1992 and 2003~\cite{NCES:2002,NCES:2007}. Internationally, there have been coordinated efforts to assess adult literacy skills through programs such as the Program for International Assessment of Adult Competencies (PIAAC), involving 39 countries and regions since its inception in 2012~\cite{PIAAC}. While these assessments provide more granular and contextualized literacy skill measures, they are expensive to administrate and hard to scale: both NALS and PIAAC were conducted once per decade, requiring 8 to 10 months to conduct the surveys and interviews, and a few years to compile the results~\cite{NCES:2002,NCES:2007,PIAAC}.

As a result, the Global Alliance to Monitor Learning has recently made the call to develop ``efficient'', and ``light''  methodologies to gather nuanced, standardized data that allows for cross-national tracking and comparison~\cite{50Yearso58:online}. This work directly responded to this call, by proposing a data-driven method that leverages social media data to estimate the literacy skills of diverse geographic and demographic populations in a cost-effective way with unprecedented coverage. Although our data were collected from only one platform -- Facebook, its high penetration in many parts of the world allows our method to capture the literacy skills of the entire population, especially populations with high Internet adoption.


\subsection{Digital Literacy}
Although closely related to \textit{digital literacy} -- the ability to operate and communicate through digital technology~\cite{hargittai2009update}, language literacy is composed of fundamental language skills such as reading, writing, and numeracy,  that are often a prerequisite for digital literacy~\cite{bach2018poverty,dimaggio2001digital,mckinsey2014}. In fact, research has shown that the lack of language literacy skills is a top barrier to Internet access and technology adoption~\cite{bach2018poverty,dimaggio2001digital,mckinsey2014}. 

As human society enter an increasingly technological and informational-rich age, modern literacy assessment programs such as PIAAC also include the assessment of ``problem-solving skills in technology-rich environment'', showing several demographic differences and similarities in literacy and digital literacy proficiency in the developed countries~\cite{OECD:2013}. For example, while the gender gap in favor of men was observed with digital literacy skills, there is a very small or non-existent gender difference in literacy skills. Similarly, the age gap in favor of young people was more observed with digital literacy than with literacy~\cite{OECD:2013}. The results from PIAAC also showed a significant interaction effect between gender, age, and socio-economic backgrounds on literacy and digital literacy~\cite{OECD:2013}, inspiring us to explore similar trends for the broader global population covered by this research. 

With over 27,000 new Internet users every hour and many of them from traditionally low-resourced regions~\cite{Theriseo30:online}, this work measures and characterizes the language literacy skills of the population that is already online - as captured on Facebook, laying the foundation for future research on more contextualized literacies such as digital literacy and information literacy.


\subsection{Literacy and Social Media}

Most studies of literacy in the social media context focus on youth and their literacy practice. For example, many studies documented the young social media users' practice of ``remixing'' - creating, uploading, selecting, copy-pasting, combining, and co-producing content in their profiles and timelines, noting a new literacy practice that is more collaborative, dynamic, and multi-model than traditional, print literacy~\cite{Erstad-2007, Alvermann-2008, Greenhow-2009, Perkel-2010, Greenhow-2012,Davies-2012}. As a result, the task of ``reading'' has changed significantly, becoming more technically simple yet socially complex while ~\cite{Kress-2003,Erstad-2007}.

%(Social media \& literacy development, e.g. \cite{Black-2008})
Numerous research examined the relationship between social media and literacy development, especially, for socially disadvantaged groups and English Language Learners (ELLs). Most of these works supported the benefits of social media in literacy development. For example, \cite{Clark-2009} found that reading and writing blogs enhanced the confidence in writing for young people in the UK; \cite{Sabaruddin-2019} documented the use of Facebook is positively correlated with improved English skills for college students in Indonesia; and \cite{Black-2008}) argued that social media technologies can support ELLs to develop valuable print literacy, based on longitudinal ethnographic studies of adolescent ELLs literate and social activities around online fandom communities. However, some research also suggested that social media use can negatively impact the reading culture and academic performance of students~\cite{Kojo-2018}.


As misinformation on social media became a public concern~\cite{Vosoughi-2018-science,Edelson-2021}, some recent work highlighted the importance of literacy skills in the social media age. For example, data from the PISA 2018 reading assessment showed that less than 10\% of the 15-year-old students in OECD countries had the reading proficiency level to distinguish facts from opinions, which could significantly impact their abilities to assess the quality and credibility of information spread on social media~\cite{OECD-2021-pisa}. While an increasing amount of attention has been devoted to developing the digital and media literacy of online populations, this work underlines the prevalence of language literacy challenges and calls for future research in understanding the scale and impact of misinformation on low literacy populations.

\color{black}

\section{Methods}\label{sec:method}

% \color{red}
\subsection{Facebook dataset}\label{sec:dataset}
To obtain a written sample of online populations worldwide, we collect public posts written in any of the 12 chosen languages created by Facebook users who are at least 18 years old and active during a 30-day period between April 20 and May 20, 2020. To ensure that the collected posts represent the writings of individual Facebook users, we exclude posts made by pages, organizational accounts, and public profiles. We also exclude posts that did not contain any text or text that was shorter than 2 characters or longer than 1000 characters, as well as posts that contained URLs as these are more likely to be copied and pasted from other sources rather than composed by users. \swvv{After pre-processing the text, the median length of posts in our dataset is 18 characters, the average is 51 characters, and the 95th percentile is 187 characters.}

%\wam{While we did not obtain explicit consent from Facebook users to contribute their data, use of this data for this project is permitted under the terms of service on the platform, and user privacy and anonymity was ensured throughout data collection and processing.}
%None of the personal identifiable information (PII) was collected during data collection and processing. No personal or private posts were collected or included in our dataset. 

All the analysis and statistics were computed over a populational aggregate, based on self-reported attributes such as country and gender. \olle were only calculated for groups with at least 1,000 active users, to minimize the risk of user de-anonymization. The original text, together with the intermediate results, were dropped after populational level \olle and other statics were calculated. 

More details about Facebook public post data collection and use can be found in \smappsec{sec:procedure}.
\color{black}

\subsection{Estimating online language literacy}\label{sec:estimate}
We use the vast amount of text produced by Facebook users to quantify populations' online language literacy skills. This method relies on two assumptions: (i) literacy across populations may be measured collectively by aggregating the observed data within country or regional borders, and (ii) vocabulary usage patterns observed in corpora of written texts can be used as a proxy for language practice in a digital environment. The first assumption comes from prior observations that literacy skill is locally clustered, largely determined by the socioeconomic status of the local community and by the abundance of public resources such as public school systems and libraries \cite{kutner2007literacy}. This also alleviates the need to analyze individual-level data, which are harder to de-identify, and often too sparse for reliable estimates. The second assumption is motivated by the relationship between literacy and vocabulary knowledge found in the education domain \cite{lee2011size,curtis2006role,ouellette2006s,national2012improving,schmitt2013introduction}. In practice, vocabulary size was often employed as a proxy for literacy skill, as reading comprehension cannot be achieved unless the reader knows 95\% of the words in the text \cite{lauren1989special}, and a certain vocabulary size is required for unassisted text comprehension \cite{nation2006large}. Vocabulary knowledge of words at various frequency levels has been used to measure total vocabulary size. For example, the Vocabulary Size Test (VST) tests a learner's knowledge of 140 or 200 words, with 10 words sampled from each 1000 word in frequency levels based on British National Corpus \cite{beglar2007vocabulary}. 


Analogous to the VST method, we \swepj{measure the aggregated use of lower-frequency words (``\bigword'') - secondary vocabulary words outside the high-frequency everyday lexicons - in public Facebook posts as a proxy for online populations' literacy skills in a given language.} \yrepj{Intentionally designed as a population-level, aggregated measure, our approach does not collect any personally identifiable information or any personal/private content (see \smappsec{sec:procedure}).} For maximal geographical coverage, we pick the twelve (12) most widely used languages (in terms of countries) and algorithmically define a set of lower-frequency words for each language. The 12 selected languages are: Arabic (ar), German (de), English (en), Spanish (es), French (fr), Italian (it),  Malay (ms), Dutch (nl), Portuguese (pt), Russian (ru), Turkish (tr), and Chinese (zh). The next section will detail our method to determine the sets of \bigword based on a multi-lingual reference corpus.

% \subsection*{Reference Language Corpus}
\subsection{Detecting \bigword from a reference language corpus}\label{sec:bigword}
\swepj{While the VST method relies on the British National Corpus (BNC) for baseline word-frequencies,} we use the fastText unigram data \cite{grave2018learning} as our reference corpus. \swepj{Comparing to other popular language corpora such as BNC and Google Books Ngram~\cite{google-ngrams}, fastText has two major advantages:} (i) coverage: The distributed fastText data currently supports 157 languages and covers all 12 languages considered in our study. (ii) \swepj{up-to-date representation of vocabularies used by online population: fastText is based on texts collected from Wikipedia and Common Crawl~\footnote{https://commoncrawl.org/}, containing petabytes of web page data collected since 2008.}  
 
Using fastText data, we are able to retrieve up 200,000 most frequent words in each language as the candidates for \bigword in that language. \swepj{Understanding that the exact range of \bigword may vary depending on the language, we first try to understand the use of these top 200,000 words by Facebook users through language-specific ``word popularity curves'':} a scatter plot where the x-axis represents the rank of a word by its frequency (0 for the most frequent), and the y-axis represents ``word popularity'', as measured by the percentage of unique users in the language population who have used that word in our data. Fig.~\ref{fig:word_curve} (\smapp) shows the word popularity curves for the 12 chosen languages in our study. 

 \swepj{As illustrated in Fig.~\ref{fig:word_curve}, the word popularity curves first decline sharply for the most frequent words before settling into a relatively flat region. The transitional area - the ``elbow'' (or ``knee'') region of the smoothed word popularity curve - corresponds to the words that are neither too popular nor too unpopular, thus ideally covering the set of \bigword for that language. Mathematically, the curvature is a mathematical measure of how much a function differs from a straight line \cite{satopaa2011finding,antunes2018knee}, and the elbow areas start at the point of maximum curvature in a popularity curve. Estimating the knee/elbow point for a continuous function is straightforward since the curvature is well-defined for continuous functions; however, it is a challenging task for discrete data. We leverage the ``Kneedle'' detection \cite{satopaa2011finding}, an efficient algorithm that can efficiently detect knee points in discrete data, and the standard maximum curvature approach on a smoothed function learned from the discrete points. We found this hybrid approach is more robust to rescaling and small fluctuations in our data. Details of this elbow detection is given in \smappsec{sec:elbow}.}



Fig.~\ref{fig:eval} A-C highlight the elbow range detected from the word popularity curve for each of the three most used languages (English, Spanish, and Arabic). Words falling into the elbow range are defined as the ``\bigword'' in the language. Fig.~\ref{fig:word_curve} shows the detected elbow ranges for all 12 languages. 


\subsection{Calibration for language use and bias}\label{sec:adjust}
After determining the \bigword per language, we can calculate the relative frequency of \bigword, among all the public text posted by Facebook users from a given country in a given language. We denote the relative frequency of \bigword as $\bar{w}_{c,l}$, with $l$ representing the 12 languages considered in this study and $c$ representing the 167 countries whose official or dominant languages are among these languages. 

While $\bar{w}_{c,l}$ makes it possible to track online language literacy for multiple languages in parallel, we decide to use one representative language per country in all our analyses for simplicity. For most countries, the official language is chosen as the representative language. For countries with no or multiple official languages, we use the language that is used by most users in that country. For example, for India, a country that uses both Hindi and English as official languages, we use English as the representative language for India since English was used by the largest number of Facebook users in India based on our data. Fig.~\ref{fig:countries_covered} (\smapp) shows the number of countries broken down by dominant language.


\swepj{As validation, we group countries by the representative language and compare the value of $\bar{w}_{c}$ with the official literacy rate data for each country. After excluding countries where the official literacy rate is not available or where the Internet penetration is lower than 25\% percent, the three most widely used languages (in terms of the number of countries covered) in our data are English (41 countries), Spanish (19 countries), and Arabic (15 countries). Fig.~\ref{fig:eval} D-F shows the relationship between the officially reported literacy rates and $\bar{w}_{c}$, for English, Spanish, and Arabic, respectively: each dot represents a country, with $x$ value corresponding to the rank-based quantile normalization\cite{beasley2009rank} of $\bar{w}_{c}$, and $y$ value corresponding to the rank-based quantile normalization of its official literacy rate. The shadow areas in Fig.~\ref{fig:eval} D-F visualize Spearman's rank correlation coefficient between the two variables, with 95\% CIs\footnote{\yrrr{Without specification otherwise, the confidence interval for all the correlation coefficients are produced using a nonparametric bootstrap procedure based on the percentile method (with 1000 bootstrap replicates).
}}. As seen in Fig.~\ref{fig:eval} D-F, all three sets of countries exhibit strong positive correlations, though the Spanish-speaking countries show a larger variance than countries mostly using English or Arabic. The positive correlations between $\bar{w}_c$ and the reported literacy rate in the three top languages suggest the efficacy of using the relative \bigw count as a measure of literacy across the different most-used languages.}
The correlations for other languages are not reported because none of the remaining languages cover more than 7 countries.

To generate a global online literacy estimate that is directly comparable across languages, we also calibrate $\bar{w}_{c}$ for all countries using the official literacy data collected by UNESCO~\cite{unesco-literacy-data}. 
Given that the two measures have different distributions, both $\bar{w}_{c}$ and the official literacy data were transformed to better fit a normal distribution. A rank-based ordered quantile normalization transformation \cite{beasley2009rank} was used for both $\bar{w}_c$ and the literacy rates, where the transformation $g(\cdot)$ is given by $g(x_i)=\Phi^{-1}\left(\frac{r_i-0.5}{n}\right)$ where $\Phi$ refers to the standard normal CDF, $x_i$ is a continuous measurement observed for each object $i$, $r_i$ refers to the sample rank of $i$ when the measurements are placed in ascending order, and $n$ refers to the number of observations. In the case of new values that fall outside the observed domain of $x$, we adopt the standard procedure and use generalized linear models to estimate the ranks beyond the bounds of the original domain of $x$. 

A linear fixed effect model is employed to quantify the systematic differences across languages {\it in relation to the offline literacy rates}. Let $l^{(i)}$ be the representative language for a (country) population $i$, the calibrated estimate, denoted as $\hat{w}_i$, is given as 
$\hat{w}_i \propto {y}_{il}$, where
\begin{align*}
{y}_{il} &= \beta x_{il} + \alpha_l + \epsilon_{il},
\end{align*}
where $x_{il} = g(\bar{w}_i)$ is the direct literacy estimate of population $i$, $\alpha_l$ is the language-specific effect, $\epsilon_{il}$ is the idiosyncratic error term with $\epsilon_{il} \sim \mathcal{N}(0,\,\hat{\sigma}_{\epsilon}^{2})$, and $\beta$ is the parameter. \wam{After $\beta$ is learned, \textbf{the global online language literacy estimate (\olle) is the calibrated estimate $\hat{w}_i$}, obtained by rescaling ${y}_{il}$ between 0 and 1.} This rescaling step is to make the \olle value more interpretable and to facilitate the comparison across populations and subpopulations by a single index. 

Since the information about the official literacy rates has been used in the calibration, to get an unbiased evaluation, we use the leave-one-out procedure to obtain an out-of-sample evaluation -- for each $i$, the calibrated estimate was generated by using all countries other than $i$ in the model (the estimated parameters can be found in Model (a) in Table \ref{tab:model-lit-main}). In other words, the language calibration is calculated from the residualized mean aggregated over the rest of the countries using the same language as $i$. 

\swepj{While the official literacy rates data has served an important role in our study to validate and calibrate \olle, they are not suitable as target variables for building a predictive model for online language literacy, for a few reasons: 1) conceptually, we want to distinguish online population literacy and general population literacy in this study; 2) technically, the official literacy rates data were collected for different countries at different time, with methodological variances over the years, introducing extra noise and latent variables for a robust supervised model.}


\subsection{Country-level socioeconomic covariates}\label{sec:socioecon}
To understand the relationship between literacy and other social factors, we collect information about countries' socioeconomic status and technical development from multiple data sources. Tables~\ref{tab:gender_corr} and \ref{tab:region_corr} list all variables used in our study, with definitions and the sources where the variables are gathered. The first-order correlations among these variables are provided in Table~\ref{tab:gender_corr}. Due to the heterogeneous distributions across variables, we report correlations using Spearman's rank correlation coefficient unless otherwise stated.

\swepj{Inspired by previous research showing the effect of a country's income, Internet penetration, and other gender inequality measure on the digital gender gap~\cite{fatehkia2018using}, we study the relationship between social factors and online literacy gaps through regression analysis with the gender or regional differences in \olle as the dependent variable and country-level variables as the independent variables.}
All the regression models presented in this work are based on standard OLS estimates, where all variables are first separately transformed to better fit a normal distribution. For example, the income variable was transformed logarithmically. Considering the geographical clustering of many of the socioeconomic variables, in each of the regression analyses, we provide models with and without controls for geographical groups. The control for geographical groups signifies whether the pattern observed in our study is a global phenomenon or particular to certain areas. For example, when predicting the gender gap in \olle, the coefficient estimates are consistent between the models with and without geographical information. We show in \smapp Tables~\ref{tab:model-gender-main}-\ref{tab:model-gender-3v-geo} the detailed estimates of regression models and their comparisons. The consistent coefficient estimates are also found in predicting regional disparity (\smapp Tables~\ref{tab:model-region-main}-\ref{tab:model-region-4v}).





\section{Results}\label{sec:results}
The results are two-fold. We first validate of our literacy estimates with existing official data and report the global state of online language literacy across 167 countries. Next, we consider within-country demographic segments such as gender and regions to benchmark the differences in online language literacy across subpopulations, and examine the socioeconomic factors that explain those differences. 

% \subsection*{Measurement validation}\label{sec:world-validation}
\subsection{Online Language Literacy Worldwide}\label{sec:world}

\subsubsection{Significant agreement between \olles and official literacy data}
Following the methodology detailed in \method, we generate \olle, the calibrated online language literacy estimates, in 167 countries or regions whose representative languages are among the 12 selected languages and have at least a thousand adult Facebook users who posted publicly in the representative language during our data collection period. To ensure we obtain a sufficient sample size of the population in each country, we leave out five countries---China, Iran, Russia, Kazakhstan, and Turkmenistan---where Facebook use is curbed by the countries' government regulations or other policy challenges. We show the country and world population coverage breaking down by languages in \smapp Fig~\ref{fig:countries_covered}. The raw and calibrated values of online language literacy estimates for all 167 countries can be found in \smapp Table~\ref{tab:all_country}. 

Fig~\ref{fig:eval} illustrates the key steps of our methodology, as well as the correlation between \olles and official literacy rate data in~\ref{fig:eval}G. As visualized in Fig.~\ref{fig:eval}G, we find a strong and positive correlation between \olle with the reported literacy rates (Spearman's rank correlation $\rho=0.78$, \yrepj{95\% CI [0.69, 0.84]}, $p<0.001$ based on out-of-sample evaluation). Similar results are found when validating our estimates with global educational attainment statistics: \olle is highly correlated with a country's average schooling years ($\rho=0.78$, \yrepj{95\% CI [0.65, 0.87]}, $p<0.001$; see details in \smapp Fig.~\ref{fig:corr_lit_edu}). These findings indicate that our estimates do reflect populations' literacy skills and can be used as a reliable proxy for official literacy and educational attainment statistics when such data are unavailable or outdated.



\subsubsection{Understand global online literacy inequalities through \olle}
\swepj{One direct application of \olle is to track the state of literacy for online populations across the globe. Mapping \olles by country in Fig.~\ref{fig:lit_map}B, we see significant inequalities in online language literacy skills across geographical regions, with the ``global south'' countries collectively lagging behind in terms of online population literacy skills.}  Fig.~\ref{fig:lit_map}A summarizes the aggregated statistics for seven geographical groups and benchmarks the online literacy gaps across regions. The bottom 10\% of countries with the lowest \olles are primarily located in Sub-Saharan Africa (13 countries), plus one in Latin America \& the Caribbean (Haiti), and one in Northern Africa \& Western Asia (Algeria). \swepj{While our result is consistent with the geographical patterns observed in official literacy rate data~\cite{unesco2017}, it also highlights the persistence of literacy gaps across offline and online populations, calling out for additional literacy support for a substantial percentage of the online population in today's digital environment.}

\swepj{On the other end of the spectrum, the top-ranked countries in terms of \olle are all located in the Europe \& North American region, as well as Oceania, with the top 3 countries being Belarus, Ukraine, and San Marino. While this result is again largely consistent with the official literacy rate data by UNESCO, it also suggests potential biases introduced by language-based calibration. For example, countries with Russian as the representative language (e.g. Belarus, Ukraine) could get an extra boost during calibration due to the overall high literacy rates in Russian-speaking countries reported in the official data.}


\subsection{Gender Difference in Online Language Literacy}\label{sec:gender}

Although the gender gap in literacy has been shrinking globally in recent decades, women are still facing obstacles when accessing school and the Internet~\cite{unesco2017,owidinternet}. As a result, serious male-favoring gender gaps in literacy skills still persist in the Middle East and North Africa, South Asia, and Sub-Saharan Africa regions \cite{world2020global}. An earlier study showed that, in low-income countries, female Internet penetration is 24\% lower than that for males \cite{fatehkia2018using}. To track the gender literacy gap in the online population, we calculate the standardized difference in \olle between female and male Facebook users in each country. This measure captures both the direction and the size of the gender gap in online language literacy, where a positive value indicates a female-favoring gap, and vice versa. To ensure a sufficient sample size of the male and female subpopulations, we drop countries (17 out of 167) with fewer than a thousand adult users in either gender group in our dataset for this analysis. While the gender digital divide has generally referred to the gaps in access and use of digital technology, our measure calls for attention to the disparity in language skills between women and men who are already online. Understanding that gender is non-binary, we only present the female-male gender gap here for two reasons: (a) it enables us to correlate our estimates with existing gender gap data; (b) in our dataset, we do not have sufficient data from users with self-reported non-binary gender information to deliver a reliable estimation for this sub-population.


\subsubsection{Women collectively scored higher than men on Facebook in most countries, with substantial \yrepj{male-favoring} gaps in two regions}
Fig.~\ref{fig:gender_gap}A shows the gender differences in \olle captured in our data.  Among the 160 countries where the gender gap in \olle is calculated, 69 countries (43.1\%) have significant female-favoring gaps and 54 countries (33.8\%) have significant male-favoring gaps. The remaining 37 countries do not have a significant gender gap in \olle. \yrepj{The significance is determined based on whether a country's male-favoring gap (or female-favoring gap) falls above the 95\% confidence limits of the expected male-favoring gaps (or female-favoring gaps).}
Overall, we observe more countries having female-favoring gaps in our measure, suggesting on average higher language literacy skills for women than men among today's online population. 

Fig.~\ref{fig:gender_gap}A highlights countries with the most and least substantial female-favoring gaps. As observed in Fig.~\ref{fig:gender_gap}A, almost all of the countries with the world's largest advanced economies (the G7) have a significant female-favoring gap, with Italy the only exception. \swepj{This finding is generally consistent with the data collected in recent PISA tests, which showed that girls outperformed boys in reading in all the OCED countries and regions~\cite{pisa-2018-gender}. Despite progress, gender-based inequalities are still pervasive in today's society.} To summarize the global state of online literacy difference between male and female subpopulations, a world map of \olle gender gap is provided in \smapp Fig.~\ref{fig:gender_map}. Notably, while most regions on average appear to have female-favoring gaps, two regions (Sub-Saharan Africa and Northern Africa \& Western Asia) still show substantial gaps favoring men. In Sub-Saharan Africa, there are 22 countries with male-favoring gaps, compared to 8 with female-favoring gaps; in Northern Africa \& Western Asia, there are 15 countries with male-favoring gaps,  compared to \yrepj{only} two countries with female-favoring gaps. \swepj{The regional patterns here are generally consistent with what was reported by UNESCO in the official literacy rate data~\cite{unesco2017}. Our measure, however, characterizes more countries with female-favoring literacy gaps than the official data, indicating a potential populational difference between women on Facebook and the general female population in a country. We will further explore the relationship between populational factors and the gender gap in \olle in the next subsection.}

%\subsubsection{Gender gap in OLLE significantly correlates with existing measures}
\subsubsection{Understand the societal context for gender online literacy gap}
We first compare the observed gender gap in \olle with other country-level measures such as overall \olle, income per capita, Gini index, average education attainment, and Internet penetration rate. As shown in Fig.~\ref{fig:gender_gap} 
\yrepj{(B,C,D) the \olle gender gaps are positively correlated with the countries' \olle (Spearman's rank correlation $\rho=0.59$, 95\% CI [0.47,0.68], $p<0.001$), overall education ($\rho=0.59$, 95\% CI [0.44,0.71], $p<0.001$), and Internet penetration ($\rho=0.30$, 95\% CI [0.14, 0.46], $p<0.001$), suggesting that women are disproportionally disadvantaged in low-resourced, low-literacy countries. Fig.~\ref{fig:gender_gap} (E,F,G) show that countries' \olle gender gaps significantly correlate with other gender parity measures, including a positive association with the female-male difference in offline literacy rates ($\rho=0.43$, 95\% CI [0.26, 0.58], $p<0.001$), women's civic participation ($\rho=0.48$, 95\% CI [0.3, 0.62], $p<0.001$), and a negative association with the countries' Gender Inequality Index or GII ($\rho=-0.4$, 95\% CI [-0.57, -0.23], $p<0.001$).}
The GII reflects how women are disadvantaged in multiple dimensions of human development and thus a negative association is expected~\cite{gender-inequality-index}. Interestingly, among all gender parity or empowerment measurements, women's civic participation -- the extent to which women have the ability to express themselves and to participate in civil society \cite{coppedge2019v} -- appears to have the strongest association with the \olle gender gap. This could suggest that the offline structural barriers to women's civic participation are strongly associated with their literacy relative to men in the online space.

%\subsubsection{Gender gap in \olle is associated with various aspects of countries' socio-technical development}
\swepj{To better understand the societal context for online literacy gender gap globally, we further examine the relationship between country-level variables and the observed gender gap in \olle through multiple regression Ordinary Least Squares (OLS) model.} Given that many of the country-level variables are highly correlated (see \smapp Table~\ref{tab:gender_corr}), we only include the most relevant variables in this analysis. Fig.~\ref{fig:gender_model}A summarizes the estimated effect of these variables, where the effect of geographical grouping is further detailed in \smapp (see Tables \ref{tab:model-gender-main}--\ref{tab:model-gender-3v-geo}). 
Based on the OLS estimation, the \olle gender gap remains significantly and positively associated with overall education status and women's civic participation while controlling for other variables.
\yrepj{The overall Internet penetration rate is negatively associated with the \olle gender gap. This may look counterintuitive. One possible interpretation is that a lower level of Internet penetration rate excludes groups from lower socioeconomic status to participate in the digital world, and women in those groups also tend to lack opportunities in education and many other developmental aspects. Hence, a lower level of Internet penetration rate ironically serves as an equalizer for the \olle gender gap.}
Interestingly, the OLS model also reveals an interaction effect between the overall Internet penetration rate and women's civic participation on the \olle gender gap. When a country's Internet penetration rate is high, the country's \olle gap may be either high or low -- depending on whether the country has a high level of women's civic participation (Fig.~\ref{fig:gender_model}B). This could suggest that technological advancements -- i.e., the adoption of the Internet -- are not necessarily associated with more opportunities or higher skills for women relative to men unless such a relationship appears in a society where women have the chance to actively participate in civic processes.



% \begin{figure}[!t]
\centering
\setlength{\abovecaptionskip}{0pt}
\setlength{\belowcaptionskip}{-2pt}
\subfigure[] {\ \ \includegraphics [width=0.47\linewidth]{fig/stats_2_4DCT.pdf}} \ 
\subfigure[] {\includegraphics [width=0.47\linewidth]{fig/stats_1_COPD.pdf}}
\caption{Statistical Analysis of our technique and existing methods. We performed the Friedman test for multiple comparisons along the Wilcoxon test for pair-wise comparison for (a) 4DCT and (b) COPD datasets. }
\label{FigTest}
\vspace{0pt}
\end{figure}



\subsection{Within-Country Regional Disparity in Online Language Literacy}\label{sec:regional}

While it is widely acknowledged that the disparity in education resources and technology infrastructure has contributed to the digital divide between developed and developing countries~\cite{warf2020geographies,wdr:2016}, there have been only a few studies that examined the digital disparities within a country -- often limited to studying a single country with the digital divide among gender and ethnicity groups~\cite{pew:2012,Hilbert:2011}. Here we intend to provide insights into the within-country regional digital disparities for a large number of countries across the world. Extending our methodology, we measure the within-country regional disparity in online language literacy by quantifying the variability of regional \olles for a given country. More specifically, the variability of regional \olles is calculated as the standard deviation of \olles aggregated across available regions within a country, thus a larger value indicates a higher variability observed in \olles at a sub-national level. We define a region as a sub-national administrative division (self-government or jurisdiction under a country's national laws), such as a state or a province. Countries with less than two regions having a minimum of a thousand active adult Facebook users in our dataset are excluded, which resulted in 119 countries in our analysis. Fig.~\ref{fig:region_map} in \smapp provides summary statistics and a map of the within-country regional disparity in \olle, as well as the representative countries with a relatively high or low level of regional disparity from each geographical group.  

\subsubsection{Within-country regional disparity in \olle is associated with multiple inequality measures}
We examine the societal backdrop for the observed regional disparities in online language literacy. Our particular interest is in the link between the regional disparity in \olle and countries' resource distribution, such as the inequalities in education and income within a country, as well as its overall education and socio-technical development. Using multiple regression analysis, we find that, after controlling for all other variables, inequality in education and the Internet penetration rate have a strong and positive association with regional disparities in \olle (Fig.~\ref{fig:gender_model}C). Not surprisingly, a higher level of overall educational attainment predicts a smaller regional variation in online language literacy skills, which is converse to the effect of inequality in education. The inequality in income, as captured by a country's Gini index, however, appears to have a negative relationship with the \olle regional disparity, indicating a greater income inequality is associated with smaller regional \olle disparity within the country.  

\subsubsection{Inequality paradox}
\yrrr{The interaction between inequalities in education and income is also observed (Fig.~\ref{fig:gender_model}D), where a greater level of within-country regional disparity in \olle is predicted for countries with one of the two conditions: either the country has a relatively high level of income and education inequalities, or has a relatively low level of both inequalities.} This finding suggests an ``{\it inequality paradox}'' -- a paradoxical pattern we notice that links the offline socioeconomic inequalities to online language skill disparity in surprising ways. \yrrr{For example, in countries with a higher level of either economic or educational inequalities, access to social media is more likely to be reserved for the more socio-economically advantaged groups, and therefore show a less regional disparity in \olle (corresponding to the top-left or bottom-right corner of Fig.~\ref{fig:gender_model}D). In contrast, in countries where education inequality is low but economic inequality is high, a higher level of regional disparity in \olle is observed (corresponding to the bottom-right corner of Fig.~\ref{fig:gender_model}D).} Similar patterns are observed when taking the geographical grouping into account, suggesting that the observed patterns are common across societies (see \smapp Tables \ref{tab:model-region-main}--\ref{tab:model-region-4v} for more details).
% The interaction between inequalities in education and income is also observed (Fig.~\ref{fig:gender_model}D), where a greater level of within-country regional disparity in \olle is predicted when a country has either a relatively high or low level of both inequalities. This finding suggests an ``{\it inequality paradox}'' -- a paradoxical pattern we notice that links the offline socioeconomic inequalities to online language skill disparity in surprising ways. For example, in countries with a higher level of economic inequalities, access to social media is more likely to be reserved for the more economically advantaged groups, and therefore show a less regional disparity in \olle. Similar patterns are observed when taking the geographical grouping into account, suggesting that the observed patterns are common across societies (see \smapp Tables \ref{tab:model-region-main}--\ref{tab:model-region-4v} for more details).


% \begin{figure*}[t]
  \centering
    \subfloat[SAN]
{\includegraphics[width=0.1595\linewidth]{figures/fig4/pc_san.pdf}}\hfill
    \subfloat[\textbf{Ours}]
{\includegraphics[width=0.1595\linewidth]{figures/fig4/pc_ours.pdf}}\hfill
     \subfloat[GT]
 {\includegraphics[width=0.1595\linewidth]{figures/fig4/pc_gt.pdf}}\hfill
    \subfloat[SAN]
{\includegraphics[width=0.166\linewidth]{figures/fig4/mess_san.pdf}}\hfill
    \subfloat[\textbf{Ours}]
{\includegraphics[width=0.166\linewidth]{figures/fig4/mess_ours.pdf}}\hfill
    \subfloat[GT]
{\includegraphics[width=0.166\linewidth]{figures/fig4/mess_gt.pdf}}\hfill
\\
\vspace{-10pt}
\caption{\textbf{Qualitative comparison to SAN~\citep{xu2023side}.} We visualize the results of PC-459 dataset in (a-c). For (d-f), we visualize the results from the MESS benchmark~\citep{blumenstiel2023mess} across three domains: underwater (top), human parts (middle), and agriculture (bottom).} 
\label{fig:qualitative}
\vspace{-10pt}
\end{figure*}



% \input{sections/024_covid}

\section{Discussion}\label{sec:discussion}

\paragraph{A data-driven, cross-language, cross-country online literacy estimation}
Taking advantage of the abundance of user-generated text online, our proposed methodology of measuring online language literacy \yrv{can be scaled across languages and subpopulations, as long as population-level text corpora are available. \olle complements the traditional sources and makes it possible to} monitor future progress and answer questions such as: whether the online language skills improve faster (slower), or whether the literacy gap is closing (widening), particularly in low-literacy countries. \swepj{While our current dataset only contains text generated within a 30-day period, further collection of similar data over a longer period of time will offer new insights on the temporal evolution of \olles across the world.}

\paragraph{Tracking global trend in online language literacy}
This study reveals the current state of global online language literacy. Based on our study, the estimated global online language literacy has remarkably high correlation with documented country-level literacy rates and educational attainments data (with $\rho=0.78$ in both correlations; see Fig.~\ref{fig:eval} and \smapp Fig.~\ref{fig:corr_lit_edu}). This finding has two implications. First, given that 86\% of the world population are now reportedly literate (i.e., able to read and write)~\cite{unesco2017}, our study suggests the variation in language skills remains {\it within the literate, online population}. Even though many countries now have more than 95\% literate populations, our online literacy map (Fig.~\ref{fig:lit_map}) has revealed the nuanced differences among the online population's language skills in these countries. Second, beyond a few options in assessing a country's digital advancement, such as the Internet penetration measure, \olle's robust correlation with offline literacy and educational data make it a more relevant alternative for tracking the {\it outcome} of a country's access to education and resources for global literacy development. 

\paragraph{Women's empowerment, social inequalities, and online language literacy disparities}
In our study, we show that the gender difference in \olle is significantly correlated with various offline gender parity metrics, including gender gaps in literacy rate, education, GII (which also considers the economic standing across gender groups), and women's civic participation index. This suggests that a country's offline gender equity progress is crucially relevant to how literate populations across genders participate online. In contrast to existing studies that exposed the well-known correlation between the gender gaps and a country's economic and technical development \cite{world2020global,fatehkia2018using}, we find that the link is not trivial. The relationship between countries' Internet penetration and the gender gap in \olle is {\it not monotonic}, and only when there is a sufficiently high level of women participation in civic society does the \olle gender gap align with countries' Internet access. In countries with a low level of women's civic participation, the \olle gap favoring men persists even with the rise of the overall Internet penetration (Fig.~\ref{fig:gender_model}B). This finding highlights the crucial social condition that allows more literate women to participate online. We also observe non-trivial relationships among multiple inequalities in our analysis of within-country regional disparity. The regional disparity in \olle is positively associated with unequal education, but the relationship is not simple when comparing countries with different levels of income inequality -- for example, a lower disparity \olle may reflect the homogenized language skills from only the economically advantaged subpopulations, or those from only more educated subpopulations. Our study explicates the complex relationship between the multidimensional inequality measurements and their manifestation on digital populations' online literacy skills.


\subsection{Limitations and future research opportunities}\label{sec:discussion}
We discuss the limitations of this study and highlight where study results must be interpreted with caution, as well as future research opportunities.

\paragraph{Self-selection bias in Facebook data}
Traditional literacy surveys are expensive to implement and many areas of the world have limited resources for survey research. The challenge for gathering nationally representative samples is not unique to traditional survey research; more recent assessments -- for example, PIAAC, which was predominantly administered on computers, were subject to selection effects and therefore required additional adjustment \cite{yamamoto2013scaling}. Our analyses are not immune from self-selection bias where the use of Facebook varies in popularity across different demographics and the differences also vary with countries and regions \cite{Facebook2020Q2earning:online}, as well as user subcultures. \yrv{For example, the observed gender gaps may be due to the over-representation of more privileged women online  \cite{magno2014international,kashyap2021analysing}.}
On Facebook, a user's comfort level of posting likely depends on their language skills; those with very limited vocabulary may not be in the data, or may choose to communicate via other modalities, e.g., images or videos. 
This study only considers users' text-based interactions on Facebook and thus the estimates likely miss out on people at the low end of vocabulary skills. To some extent, sampling bias may be mitigated by post-sampling weightings with demographic information, as has been demonstrated in recent data-driven studies \cite{park2019global}. Such an approach nevertheless depends on sufficiently rich demographic information in the data. In our study, only data disaggregated by gender and coarse-grained geographical grouping are available. Future work may consider tackling the selection bias by separately collecting users' information on demographics and their social media interaction practice. 

\paragraph{Representative languages and language-based calibration}
\yrepj{In the current study, a country's online language literacy was measured based on a single representative language (either the official language or the most used language). One potential risk of relying on a single representative language is that the regional disparity measure in a multilingual country may simply capture the distribution of languages, rather than the diversity of language skills. To address this concern, we perform robust checks in the \smappsec{sec:domi} and Fig.~\ref{fig:domi} and do not see systematic biases associated with different penetration rates of the representative language. }

\yrepj{Another potential risk is to underestimate the language literacy of countries that have sizable language minorities (including people who use/speak a dialect), multilingual communities, or multiple monolingual subpopulations, since their data are largely excluded from our methodology.}  For example, an English-majority country with a larger Spanish-speaking population may score lower in a measure of English-language literacy skills. For such countries, focusing on improving a dominant-language literacy measure can be potentially harmful, since more resources may be allocated in favor of the dominant language.  
\swepj{In \smappsec{sec:india}, we present a case study using India as an example of a multilingual country and show that literacy estimation based on multiple languages has neglectable improvement over English-based estimation in its correlation with the official literacy data (see Fig.~\ref{fig:domi}). However, we acknowledge the official literacy data often have a bias against language minorities and recommend future work consider measuring the online language skills separately for all languages used by sizable populations within a country, to better understand the literacy skills and needs across diverse communities.}


\swepj{The use of official literacy data for cross-language \olle calibration also introduces potential biases and noises.} As mentioned in Section \ref{sec:world}, the post-calibration \olles for the Russian-speaking countries are likely to be overestimated due to their historically high literacy rates in the official data. On the other hand, \olles for Arabic-speaking countries be underestimated due to the fact that the alternative learning (e.g., religious education) provided in those countries was not included in the official literacy data. Languages concentrated in only one or a few countries, such as Japanese and Korean, are not considered in our study due to the lack of benchmark data that can be used for validation or calibration. Therefore, to establish an adequate common scale for more languages, future research will benefit from more comprehensive and up-to-date data for literacy skills across languages.

\paragraph{Thresholding vs. continuum measurement}
Our measure relies on thresholding the observed word frequency bands -- i.e., the set of \bigword was identified by the automatically determined word frequency cut-offs -- but one may also consider the continuum of the word frequency range. Our choice of focusing on particular word frequency bands is aligned with the existing literature in language comprehension research. For example, studies from English language comprehension distinguish the utility of high-, mid-, and low-frequency vocabulary: the high-frequency vocabulary (e.g., the most frequent 2000- or 3000-word families from a particular English corpus) provides the largest lexical coverage of any text but is not sufficient for adequate reading comprehension, while the low-frequency vocabulary (including the words over the 9000-word families) is too infrequent and thus has very limited utility; only the mid-frequency vocabulary gives the important range of words required for reading authentic materials \cite{nation2006large,masrai2019vocabulary}. \yrepj{However, different vocabulary sets may serve significant functions for different populations; for example, high-frequency vocabulary has been shown as an important source of knowledge for second-language learners \cite{masrai2019vocabulary}.} Future work may take into account the continuum of the word frequency range and investigate the level of contribution provided by the various word frequency bands to online language skills.


\paragraph{Heterogeneity in social media texts}
A potential concern about using social media text to measure the language skills of a population is how to deal with social media users' heterogeneous behaviors, e.g., some users may post more than others, and some tend to copy content from elsewhere, which could disproportionally impact the population-level measurement.
\swepj{We adopt a few methods to address this concern, including counting each unique unigram once per user, and leaving out posts that are likely to be copy-pasted (see \smappsec{sec:procedure} for more details). However, we did not perform efficacy evaluation for these methods, and would encourage} future work further examine the impact of text recycling and text production disparities for online literacy assessment. 

\paragraph{Aggregate vs. individual measures, and correlations}~\olles are generated based on aggregate data, which inherently poses risks of ecological fallacy compared to other literacy data collected through individual-level tests and surveys. 
In our study, the between-country correlations only involve the between-country differences in aggregate statistics of the within-country distributions, and the unmeasured within-country measures could be uncorrelated, or could even be correlated in the opposite direction. Taking into account individual assessment in a multi-level analysis \yrrr{with a proper privacy protection mechanism} may be a fruitful direction to reduce aggregation bias and the ecological fallacy in future research.
In the case of the observed association between the gender gap in \olle and women's civic engagement, a less ambiguous interpretation -- whether higher literacy empowers women for civic engagement, or civic engagement leads to legal and institutional changes that enhance literacy, or other cultural, religious, political, and socio-economic conditions influence both women's civic engagement and progress in online language literacy -- requires further research to carefully examine the causal pathways.




\subsection{Conclusions}
This work develops a scalable language literacy measurement to monitor the collective language literacy of the online population using social media data from more than 160 countries. The measure then allows for tracking the trends and inequalities in online language literacy and their relationships with various socioeconomic conditions. Our findings identify key regions and populations disproportionally impacted by literacy challenges, and suggest that education or technical infrastructure alone is not sufficient to explain the variance in online population language literacy skills. Our study calls out the need for more attention and resources to be allocated to populations with limited online literacy skills -- especially those who also suffer from poverty, low resource, and other structural discrimination, to empower them through global challenges such as misinformation and social inequality, and to sustain the overall progress in democratic and socioeconomic development.




%%%%%%%%%%%%%%%%%%%%%%%%%%%%%%%%%%%%%%%%%%%%%%
%%                                          %%
%% Backmatter begins here                   %%
%%                                          %%
%%%%%%%%%%%%%%%%%%%%%%%%%%%%%%%%%%%%%%%%%%%%%%

\begin{backmatter}
\section*{Abbreviations}\label{sec:abbr}
OLLE: online language literacy estimate \\
LoFF words: lower-frequency words \\
UNESCO: United Nations Educational, Scientific and Cultural Organization \\
NALS: National Adult Literacy Survey \\
PIAAC: Program for International Assessment of Adult Competencies \\
ELLs: English Language Learners \\
VST: Vocabulary Size Test \\
CDF: cumulative distribution function \\
OLS: ordinary least squares \\
GII: Gender Inequality Index \\


\section*{Availability of data and materials}\label{sec:data_availability}
% Data to reproduce these results will be available in the Open Science Framework (\url{https://osf.io/zcpej/}).
%% \url{https://osf.io/zcpej/
Data aggregated at the country level (country-level literacy estimates and summary statistics) will be made available in the Open Science Framework (OSF, at \url{https://osf.io/zcpej/}) upon publication of this manuscript. Facebook requires that this work was to be done in compliance with Facebook's Data Policy and research ethics review process (\url{www.facebook.com/policy.php}). Restrictions apply to the availability of the disaggregated data (user- or post-level data), so they are not publicly available. Data aggregated at the country level and other datasets that support the findings of this study will be available from the OSF repository with the permission of the authors, upon reasonable request. The analysis code used to derive the main results are available in the Open Science Framework (OSF) upon publication of this manuscript.


\section*{Competing interests}
  The authors declare that they have no competing interests.

\section*{Author's contributions}
YRL and SW conceived and designed the research; YRL and SW developed the analysis tools; YRL performed the experiments and analyzed the data; YRL, SW, and WM wrote the paper.

\section*{Acknowledgements}
% We thank anonymous reviewers for their valuable feedback.
% Omit for blind review.
We thank Lada Adamic, Michael Macy, Mike Bailey, Pablo Barbera, Devra Moehler, Logan Schmid, Alex Pompe, Niki Ramchandani, Alex Leavitt, James Lo, and Edouard Grave, and anonymous reviewers for their valuable feedback.

% \section{Funding} The authors declare no funding.
% The author Yu-Ru Lin would like to acknowledge the support from the NSF grant \#1637067. Any opinions, findings, and conclusions or recommendations expressed in this material do not necessarily reflect the views of the funding sources.




% People to thank: Lada Adamic, Michael Macy, Mike Bailey, Pablo Barberá, Devra Moehler, Logan Schmid, Alex Pompe, Niki Ramchandani, Alex Leavitt, James Lo, and Edouard Grave

%%%%%%%%%%%%%%%%%%%%%%%%%%%%%%%%%%%%%%%%%%%%%%%%%%%%%%%%%%%%%
%%                  The Bibliography                       %%
%%                                                         %%
%%  Bmc_mathpys.bst  will be used to                       %%
%%  create a .BBL file for submission.                     %%
%%  After submission of the .TEX file,                     %%
%%  you will be prompted to submit your .BBL file.         %%
%%                                                         %%
%%                                                         %%
%%  Note that the displayed Bibliography will not          %%
%%  necessarily be rendered by Latex exactly as specified  %%
%%  in the online Instructions for Authors.                %%
%%                                                         %%
%%%%%%%%%%%%%%%%%%%%%%%%%%%%%%%%%%%%%%%%%%%%%%%%%%%%%%%%%%%%%

% if your bibliography is in bibtex format, use those commands:
\bibliographystyle{bmc-mathphys} % Style BST file (bmc-mathphys, vancouver, spbasic).
\bibliography{reference.bib,reference_supp.bib}
% for author-year bibliography (bmc-mathphys or spbasic)
% a) write to bib file (bmc-mathphys only)
% @settings{label, options="nameyear"}
% b) uncomment next line
%\nocite{label}

% or include bibliography directly:
% \begin{thebibliography}
% \bibitem{b1}
% \end{thebibliography}

%%%%%%%%%%%%%%%%%%%%%%%%%%%%%%%%%%%
%%                               %%
%% Figures                       %%
%%                               %%
%% NB: this is for captions and  %%
%% Titles. All graphics must be  %%
%% submitted separately and NOT  %%
%% included in the Tex document  %%
%%                               %%
%%%%%%%%%%%%%%%%%%%%%%%%%%%%%%%%%%%

%%
%% Do not use \listoffigures as most will included as separate files

\section*{Figures}
Figure 1-4.\\
\begin{figure}[!t]
\centering
\setlength{\abovecaptionskip}{0pt}
\setlength{\belowcaptionskip}{0pt}
\includegraphics[width=\linewidth]{fig/teaser.pdf}
\caption{\textbf{Our proposed workflow framework.} We seek to optimise, through a coordinate MLP, the mapping $\mathbf{\Phi}$ to align the coordinates between the source and target images. Our highlight is a new regulariser, whose effect is illustrated in the middle part. Our proposed conformal-invariant hyperelastic regulariser enforces volume presentation, controls changes in length and area, and ensures smoothness of deformation yielding to a better optimisation outcome.}
\label{FigOverviewFigOverview}
\vspace{0pt}
\end{figure}
\begin{figure*}[t]
    \centering
    \includegraphics[width=\linewidth]{images/architecture-v9.pdf}
    \caption{Our proposed method, PASS, consists of three classifiers trained in a round-robin fashion, with two classifiers (\(h_{\gamma_{2},\gamma_{3}}\) in green) being used to select samples for training the other classifier (\(h_{\gamma_{1}}\) in red).
    The training process begins with a warm-up of all classifiers, followed by the sample selection stage. During the selection stage, the peer classifiers calculate the prediction agreement using cosine similarity between their posterior distributions, followed by Otsu's thresholding~\cite{otsu1979threshold} to automatically find the threshold $t$ to select the clean set ${D}_{\text{clean}}$ and noisy set ${D}_{\text{noisy}}$. In the training stage, we follow the robust noisy-label training algorithm.}
    \label{fig:architecture}
\end{figure*}
\begin{figure}[!t]
\centering
\setlength{\abovecaptionskip}{0pt}
\setlength{\belowcaptionskip}{-2pt}
\subfigure[] {\ \ \includegraphics [width=0.47\linewidth]{fig/stats_2_4DCT.pdf}} \ 
\subfigure[] {\includegraphics [width=0.47\linewidth]{fig/stats_1_COPD.pdf}}
\caption{Statistical Analysis of our technique and existing methods. We performed the Friedman test for multiple comparisons along the Wilcoxon test for pair-wise comparison for (a) 4DCT and (b) COPD datasets. }
\label{FigTest}
\vspace{0pt}
\end{figure}


\begin{figure*}[t]
  \centering
    \subfloat[SAN]
{\includegraphics[width=0.1595\linewidth]{figures/fig4/pc_san.pdf}}\hfill
    \subfloat[\textbf{Ours}]
{\includegraphics[width=0.1595\linewidth]{figures/fig4/pc_ours.pdf}}\hfill
     \subfloat[GT]
 {\includegraphics[width=0.1595\linewidth]{figures/fig4/pc_gt.pdf}}\hfill
    \subfloat[SAN]
{\includegraphics[width=0.166\linewidth]{figures/fig4/mess_san.pdf}}\hfill
    \subfloat[\textbf{Ours}]
{\includegraphics[width=0.166\linewidth]{figures/fig4/mess_ours.pdf}}\hfill
    \subfloat[GT]
{\includegraphics[width=0.166\linewidth]{figures/fig4/mess_gt.pdf}}\hfill
\\
\vspace{-10pt}
\caption{\textbf{Qualitative comparison to SAN~\citep{xu2023side}.} We visualize the results of PC-459 dataset in (a-c). For (d-f), we visualize the results from the MESS benchmark~\citep{blumenstiel2023mess} across three domains: underwater (top), human parts (middle), and agriculture (bottom).} 
\label{fig:qualitative}
\vspace{-10pt}
\end{figure*}



%%%%%%%%%%%%%%%%%%%%%%%%%%%%%%%%%%%
%%                               %%
%% Tables                        %%
%%                               %%
%%%%%%%%%%%%%%%%%%%%%%%%%%%%%%%%%%%

%% Use of \listoftables is discouraged.
%%
% \section*{Tables}
% \begin{table}[h!]
% \caption{Sample table title. This is where the description of the table should go.}
%       \begin{tabular}{cccc}
%         \hline
%           & B1  &B2   & B3\\ \hline
%         A1 & 0.1 & 0.2 & 0.3\\
%         A2 & ... & ..  & .\\
%         A3 & ..  & .   & .\\ \hline
%       \end{tabular}
% \end{table}

%%%%%%%%%%%%%%%%%%%%%%%%%%%%%%%%%%%
%%                               %%
%% Additional Files              %%
%%                               %%
%%%%%%%%%%%%%%%%%%%%%%%%%%%%%%%%%%%

\section*{Additional Files}
  \subsection*{Additional file --- Supplemental Information}
  
This document includes the additional information:
(1) Supplementary text;
(2) Figures S1 to S7;
(3) Tables S1 to S12;
(4) SI References. 
In addition, data aggregated at the country level (country-level literacy estimates and summary statistics) and the R analysis code for deriving the main results will be made available in the Open Science Framework (OSF) upon publication of this manuscript. Facebook requires that this work was to be done in compliance with Facebook's Data Policy and research ethics review process (\url{www.facebook.com/policy.php}). Restrictions apply to the availability of the disaggregate data (user- or post-level data), so they are not publicly available. Data aggregated at the country level and other datasets that support the findings of this study will be available from the OSF repository with the permission of the authors, upon reasonable request.


\end{backmatter}


\clearpage
\newcommand{\fulltitle}{Mapping Language Literacy At Scale: A Case Study on Facebook}
\begin{center}
{\Large\bf Supplemental Information -- \fulltitle}
\end{center}
\vspace*{.5in}

% \begin{fmbox}
% \dochead{Research}

% %%%%%%%%%%%%%%%%%%%%%%%%%%%%%%%%%%%%%%%%%%%%%%
% %%                                          %%
% %% Enter the title of your article here     %%
% %%                                          %%
% %%%%%%%%%%%%%%%%%%%%%%%%%%%%%%%%%%%%%%%%%%%%%%

% \newcommand{\fulltitle}{Mapping Language Literacy At Scale: A Case Study on Facebook}
% \title{Supplemental Information -- \fulltitle}


% %%%%%%%%%%%%%%%%%%%%%%%%%%%%%%%%%%%%%%%%%%%%%%
% %%                                          %%
% %% Enter the authors here                   %%
% %%                                          %%
% %% Specify information, if available,       %%
% %% in the form:                             %%
% %%   <key>={<id1>,<id2>}                    %%
% %%   <key>=                                 %%
% %% Comment or delete the keys which are     %%
% %% not used. Repeat \author command as much %%
% %% as required.                             %%
% %%                                          %%
% %%%%%%%%%%%%%%%%%%%%%%%%%%%%%%%%%%%%%%%%%%%%%%

% \author[
%   addressref={aff1},                   % id's of addresses, e.g. {aff1,aff2}
%   corref={aff1},                       % id of corresponding address, if any
% %   noteref={n1},                        % id's of article notes, if any
%   email={yurulin@pitt.edu}   % email address
% ]{\inits{YRL}\fnm{Yu-Ru} \snm{Lin}}
% \author[
%   addressref={aff2},
%   email={xx@gmail.com}
% ]{\inits{SW}\fnm{Shaomei} \snm{Wu}}
% \author[
%   addressref={aff3},
%   email={xx@fb.com}
% ]{\inits{WM}\fnm{Winter} \snm{Mason}}

% %%%%%%%%%%%%%%%%%%%%%%%%%%%%%%%%%%%%%%%%%%%%%%
% %%                                          %%
% %% Enter the authors' addresses here        %%
% %%                                          %%
% %% Repeat \address commands as much as      %%
% %% required.                                %%
% %%                                          %%
% %%%%%%%%%%%%%%%%%%%%%%%%%%%%%%%%%%%%%%%%%%%%%%

% \address[id=aff1]{%                           % unique id
%   \orgname{School of Computing and Information, University of Pittsburgh}, % university, etc
% %   \street{Waterloo Road},                     %
%   %\postcode{}                                % post or zip code
%   \city{Pittsburgh, PA},                              % city
%   \cny{USA}                                    % country
% }

% \address[id=aff2]{%
%   \orgname{AImpower.org},
%   \city{Mountain View, CA},
%   \cny{USA}
% }

% \address[id=aff3]{%
%   \orgname{Meta},
%   \city{Menlo Park, CA},
%   \cny{USA}
% }

% %%%%%%%%%%%%%%%%%%%%%%%%%%%%%%%%%%%%%%%%%%%%%%
% %%                                          %%
% %% Enter short notes here                   %%
% %%                                          %%
% %% Short notes will be after addresses      %%
% %% on first page.                           %%
% %%                                          %%
% %%%%%%%%%%%%%%%%%%%%%%%%%%%%%%%%%%%%%%%%%%%%%%

% % \begin{artnotes}
% % %\note{Sample of title note}     % note to the article
% % \note[id=n1]{Equal contributor} % note, connected to author
% % \end{artnotes}

% \end{fmbox}% comment this for two column layout

% %%%%%%%%%%%%%%%%%%%%%%%%%%%%%%%%%%%%%%%%%%%%%%
% %%                                          %%
% %% The Abstract begins here                 %%
% %%                                          %%
% %% Please refer to the Instructions for     %%
% %% authors on http://www.biomedcentral.com  %%
% %% and include the section headings         %%
% %% accordingly for your article type.       %%
% %%                                          %%
% %%%%%%%%%%%%%%%%%%%%%%%%%%%%%%%%%%%%%%%%%%%%%%
% %
% %\end{fmbox}% uncomment this for twcolumn layout

% \end{frontmatter}

%%%%%%%%%%%%%%%%%%%%%%%%%%%%%%%%%%%%%%%%%%%%%%
%%                                          %%
%% The Main Body begins here                %%
%%                                          %%
%% Please refer to the instructions for     %%
%% authors on:                              %%
%% http://www.biomedcentral.com/info/authors%%
%% and include the section headings         %%
%% accordingly for your article type.       %%
%%                                          %%
%% See the Results and Discussion section   %%
%% for details on how to create sub-sections%%
%%                                          %%
%% use \cite{...} to cite references        %%
%%  \cite{koon} and                         %%
%%  \cite{oreg,khar,zvai,xjon,schn,pond}    %%
%%  \nocite{smith,marg,hunn,advi,koha,mouse}%%
%%                                          %%
%%%%%%%%%%%%%%%%%%%%%%%%%%%%%%%%%%%%%%%%%%%%%%

%%%%%%%%%%%%%%%%%%%%%%%%% start of article main body
% <put your article body there>

\setcounter{table}{0}
\renewcommand{\thetable}{S\arabic{table}}%
\setcounter{figure}{0}
\renewcommand{\thefigure}{S\arabic{figure}}%
\setcounter{section}{0}
\renewcommand{\thesection}{S\arabic{section}}%

\addcontentsline{toc}{section}{Supplementary Text}
\section{Supplementary Text}\label{sec:supp_text}

\addcontentsline{toc}{subsection}{Language literacy and visual information consumption}
\subsection{Language literacy and visual information consumption}\label{sec:vis}

We examine the relationship between populations' language literacy levels and their interest in different types of content. Because our assessment concerns the ability to process textual content, we assume there exists a negative relationship between a population's literacy estimate and their attention toward non-textual (e.g., visual) content. Fig.~\ref{fig:lit_visual} shows the correlations between countries' \olles ($x$-axes) and the relative time spent on visual content by the countries' Facebook population ($y$-axes).  As expected, populations with a lower level of language literacy tend to spend relatively more time on visual content. The global correlation is $-0.38$ (Spearman's rank correlation; $p<0.001$), with correlations over different areas ranging from $-0.29$ (Latin America \& the Caribbean; $p<0.062 $) to $-0.62$ (Europe/Oceania/Northern America; $p<0.001$)\footnote{As two of the seven geographical groups only have few countries, we merge the seven groups into five (based on proximity) to provide an adequate statistical description.}.

\addcontentsline{toc}{subsection}{Elbow range detection in popularity curves}
\subsection{Elbow range detection in popularity curves}\label{sec:elbow}
\Bigword are determined based on ``elbow'' range on the word popularity curve for each language, where the relative word frequencies begin to have a systematic decline. 
\yrepj{Fig.~\ref{fig:word_curve} shows the popularity of words in decreasing order, i.e., from the most to the least popular word, as popularity curves. It can be seen that a systematic decline in the word popularity appears at the point of maximum curvature in a popularity curve. In other words, the interest region associated with \bigword corresponds to the ``elbow'' (or ``knee'') point on the smoothed word popularity curve. Mathematically, the curvature is a mathematical measure of how much a function differs from a straight line \cite{satopaa2011finding,antunes2018knee}. Estimating the knee/elbow point for a continuous function is straightforward since the curvature is well-defined for continuous functions; however, it is a challenging task for discrete data. It is also an inherently heuristic process \cite{antunes2018knee}. To reliably detect the elbow range, we leverage the ``Kneedle'' detection \cite{satopaa2011finding}, an efficient algorithm that can efficiently detect knee points in discrete data, and the standard maximum curvature approach on a smoothed function learned from the discrete points. First, we employ generalized additive models with cross-validation to learn a smooth function for each of the popularity curves \cite{hastie1990generalized}. As shown in Fig.~1 E-G, we define an elbow range as an area between two points $k_0$ and $k_1$ (highlighted in red) that best describe the systematic decline in the curve. The two points were determined by combining two heuristic methods: (i) the standard maximum curvature points that can be calculated from any continuous function, and (ii) the approximate knee points (Kneedle detection method) based on the notion that knee points differ most from the straight line connecting the curve's two endpoints \cite{satopaa2011finding}. Note that while the two notions may be considered to be conceptually similar, the approximate knee points (the second notion) are not necessarily the maximum curvature points especially when the curves are skewed. In a right-skewed curve (as in the case of a word popularity curve), the approximate knee points tend to fall into the right of the maximum curvature points. Thus, we detect an elbow by two points $k_0$ and $k_1$ through maximum curvature measurement and approximate knee point detection method respectively. Unlike other knee/elbow detention methods that are sensitive to noises and rescaling, we found this hybrid approach is more robust to rescaling and small fluctuations in our data.} 
Words with ranks falling into the elbow range are considered to be the ``\bigword'' in the language. Fig.~1 A-C highlights the elbow range detected from the word popularity curve for each of the three most used languages, and Fig.~\ref{fig:word_curve} shows the detected elbow ranges for all 12 languages. 

\addcontentsline{toc}{subsection}{Procedure for estimating online language literacy}
\subsection{Procedure for estimating online language literacy}\label{sec:procedure}

For a given population with a given language, the procedure to measure the collective language literacy involves the following steps:

\begin{enumerate}
\item[(i)] Processing of user-generated texts: We use public posts written in any of the chosen languages created by Facebook users who are at least 18 years old and active during a 30-day period between April 20 and May 20, 2020. We exclude posts that did not contain any text or text that was shorter than 2 characters or longer than 1000 characters, as well as posts that contained URLs as these are more likely to be copied and pasted from other sources rather then composed by users. 

\item[(ii)] Aggregate statistics per user: After tokenizing the public posts, for each user, we quantify the number of unique words (unigrams) that falls in the range of \bigword. We then obtain a relative \bigw count $w_u$ that is normalized by the active level of post creation per user, i.e., $w_u$ is given by (the total number of \bigword observed from $u$'s public posts) / (the total number of $u$'s public posts). We count each unigram once per user, regardless of the frequency used, to avoid overestimating the use of particular words or the inflation from copy-pasted content.

\item[(iii)] Aggregate statistics per population: For each geographically bounded community (e.g. a county or a region) with at least 1000 active users observed in the study period, the population-level estimate is calculated as $\bar{w}$, the average of $w_u$'s over all active users $u$'s in the geographically bounded community. The gender- or region-disaggregate population-level estimates also require a minimum of 1000 active users observed in the study period in any of the disaggregate groups. The threshold of 1000 unique users from any group was chosen to ensure user privacy and the statistical power of our method. We also exclude users who produced a high volume of posts (above 75 percentiles) to avoid a small number of highly productive users dominating the measurement.
\end{enumerate}

Throughout the procedure, none of the personal identifiable information or any personal or private content was used. Only the aggregate statistics $w_u$ and $\bar{w}$ were generated from the process. 

\paragraph{How to retrieve pre-computed \bigword}
\yrrr{\bigword are determined based on the word popularity curves derived from the Facebook users' use of the up 200,000 most frequent words in each language. These pre-computed \bigword can be retrieved through the following steps:}
\begin{itemize}
\item[(i)] Install the fastText\footnote{\url{https://fasttext.cc/docs/en/python-module.html\#installation}}
\item[(ii)] Run \texttt{download\_model.py \$lang} to get the dictionary of a specific language, where \texttt{\$lang} is the language indicator (e.g., {\texttt en} for English, {\texttt es} for Spanish). This script will download the dictionary in a binary file (let \texttt{\$filename} be the filename of the downloaded file).
\item[(ii)] Run \texttt{fasttext dump \$filename dict > \$ofilename} to convert the binary dictionary file to a text file (let \texttt{\$ofilename} be the filename of the output text file). This file contains up to 200,000 lines where each line is a word and its frequency. The frequencies can be used to rank the words from the most to the least frequent. 
\item[(iv)] Extract the \bigword based on the knee points listed in Table~\ref{tab:knees}. For example, the \bigword in English correspond to the words in the fastText dictionary that are ranked between 5,000 to 9,000 in the decreasing order of word frequency.
\end{itemize}


\addcontentsline{toc}{subsection}{D.}
\subsection{Case study: India as a multilingual country}\label{sec:india}


\yrv{We choose India as a case study for countries using multiple languages to study the effect of choosing the most used language as a single representative language for literacy estimate. While India uses Hindi and English as official languages nationwide, it has no single national language. It has over 30 states/union territories, each of which has its own official language(s). There are 22 official languages recognized by country officials, in addition to some other languages recognized as additional official languages at the regional level. In this analysis, we include the additional five most used languages in India according to the India census reported in 2011 \cite{Censusof39:online}: Hindi (43.6\%), Bengali (8.3\%), Marathi (7.1\%), Telugu (6.9\%), and Tamil (5.9\%). Languages with less than 5\% speakers among the Indian national population are not considered. On Facebook, the most used language in Indian users' public posts is English (en), which has about three times the users posting in Hindi (hi), and about 20, 44, 90, and 181 times those posting in Bengali (bn), Marathi (mr), Telugu (te), and Tamil (ta), respectively. 
Across regions, the non-English language using populations on Facebook are sparse. Only 14 (48.2\%) regions have more than 10\% of the number of English posters posting in Hindi, and only 4 (13.8\%) and 1 (3.4\%) regions have more than 1\% of the number of English posters posting in Bengali and Marathi.} 

\yrv{We then create a language literacy estimation for each of the six languages, using the same approach but include the additional languages (Hindi, Bengali, Marathi, Telugu, and Tamil) from fastText unigram data \cite{grave2018learning}. For validation, we gather the regional literacy survey reported in the India census 2011 \cite{Censusof39:online}, which is the most recent data available. 
Fig.~\ref{fig:india} shows the comparison of our language estimation with the officially reported literacy data. 
\yrrr{We first estimate the language literacy for every language. Fig.~\ref{fig:india} A and C-E show the estimation based on posts in a single language. Note that, while there are regional differences, the use of Hindi, Bengali, and Marathi is extremely sparse in most regions. Therefore, the non-English language estimates alone cannot be directly used to create a regional measurement. Due to the sparse use of non-English language on the platform, the correlation between the non-English language estimates and the reported literacy is insignificant. We additionally create a multi-language estimation weighted by the popularity of each language within a region, as shown in Fig~\ref{fig:india} B.}
%We first estimate the language literacy for every language, and additionally create a multi-language estimation weighted by the popularity of each language within a region. Fig.~\ref{fig:india} A shows the estimation based on posts in English only, and Fig.~\ref{fig:india} B shows the estimation based on multiple languages. 
We observe that both English-only and multi-language estimates (without any additional calibration) have significant correlations with the reported literacy data (Spearman's rank correlations with positive 95\% CIs and $p<0.005$). However, literacy estimation based on multiple languages has neglectable improvement over English-based estimation in terms of the correlations -- from 0.51 to 0.52. This is likely due to the low rate of users posting in non-English languages in many regions. \yrrr{Here, the comparison relies on the officially reported literacy data, which has a limitation: they do not capture the change in regional literacy levels since 2011, and likely do not properly reflect the diverse language skills used by minority populations.} 
%Another potential reason is that officially reported literacy data do not capture the change of regional literacy levels since 2011, or may not properly reflect the diverse language skills used by minority populations.
This case study illustrates the challenge of obtaining gold-standard literacy measures for multilingual countries. While this does not prevent us to create a multi-language literacy measurement per country, for validation purposes, we simply choose a single representative language for multilingual countries. Thus the correlation should be interpreted with caution -- the officially reported data that guide this choice often has a bias against language minorities.}

\addcontentsline{toc}{subsection}{E.}
\subsection{Robust check: regional disparity and language dominance}\label{sec:domi}

\yrepj{Our \olle is created based on a country's representative language, i.e., the language used by the most Facebook users in the country. One potential risk of relying on a single representative language is that the regional disparity measure in a multilingual country may simply capture the distribution of languages, rather than the diversity of language skills. To test this, we examine the relationship between the user percentage of the representative language in a multilingual country and the country's regional disparity measure. Among the 167 countries studied, there are 20 multilingual countries, but only 13 meet the criteria to have a regional disparity measure. Recall in Section~\ref{sec:procedure} that each geographically bounded community (i.e., in this case, a region within a country) with at least 1000 active users observed in the study period. For these 13 multilingual countries, we plot the countries' percentage of Facebook users using the representative language on the $y$-axis and on the $x$-axis, either (A) \olle, or (B) regional disparity as shown in Fig.~\ref{fig:domi}. If there is a systematic bias, e.g., countries with low representative language user percentages tend to have high a regional disparity measure, we would see a trend in such a plot. However, we do not observe a clear systematic bias. While our sample size is limited, this analysis is helpful for checking whether there is a potential bias in the small sample of multilingual countries.} 


\addcontentsline{toc}{section}{Supplementary Figures}
% \section*{Supplementary Figures}\label{sec:supp_figs}

\begin{figure*}
    \centering
    \includegraphics[width=.95\linewidth]{figS01_word_curves.pdf}
    \caption{Determining the ``\bigw'' range using curvature and knee points detection. In half of the studied languages, the ranks of the \Bigword range between 5000 and 9000. Others (zh, it, ru, tr, nl, and ms) have wider or narrower ranges.}
    \label{fig:word_curve}
\end{figure*}

\begin{figure*}
    \centering
    \includegraphics[width=.95\linewidth]{figS02_countries_covered_fbUsers.pdf}
    \caption{Countries covered in our estimation. There are 167 countries in 12 different languages, including 147 countries with a single or dominant language and 20 multilingual countries. For a multilingual country (having multiple official languages), we use the most used language of the country to estimate its language literacy.
    (A) The number of countries in each language. (B) The total population in each language. (C) The Facebook user count in each language, according to publicly available information about Facebook penetration statistics in 2022 \cite{Facebook15:online}. The populations from the 20 multilingual countries are excluded in (B) and (C) because we do not have the sub-population estimates of different languages within these multilingual countries. 
    }
    \label{fig:countries_covered}
\end{figure*}

\begin{figure*}
    \centering
    % \includegraphics[width=\linewidth]{sfigs/corr-lit-edu.pdf}
    \includegraphics[width=.95\linewidth]{figS03_eval_olle_3x2_CI.pdf}
    \caption{The literacy estimate ($x$-axis) obtained from Models (a,b,c) listed in Table \ref{tab:model-lit-main}, compared with the reported literacy rate and the education in terms of schooling years ($y$-axis). The reported literacy rate was transformed for normality. 
(A,B) Literacy estimate adjusted by Model (a).
(C,D) Literacy estimate adjusted by Model (b).
(E,F) Literacy estimate adjusted by Model (c).}
    \label{fig:corr_lit_edu}
\end{figure*}

\begin{figure*}
    \centering
    % \includegraphics[width=\linewidth]{sfigs/corr-lit-visual-v2}
    \includegraphics[width=.95\linewidth]{figS04_corr_lit_vs_visual_CI}
    \caption{Relationship between countries' \olles ($x$-axes) and the relative visual time-spent ($y$-axes), where the relative visual time-spent is given as the proportion of time spent on photos and videos relative to the time on news feeds, photos and videos combined. Correlations are reported based on Spearman's rank correlation. }
    \label{fig:lit_visual}
\end{figure*}

\begin{figure*}
    \centering
    \includegraphics[width=\linewidth]{figS05_map-gender}
    \caption{Gender differences in online language literacy across the world. (A) Summary of the standardized female-male differences across the seven geographical groups. Dashed line marks the global average (0.345, population-weighted), and a diamond indicates the population-weighted mean of the group. (B) Map of female-male differences available in our dataset.}
    \label{fig:gender_map}
\end{figure*}

\begin{figure*}
    \centering
    \includegraphics[width=\linewidth]{figS06_map-regional}
    \caption{Within-country regional disparity in online language literacy across the world. (A) Summary of the within-country regional disparity across the seven geographical groups. Dashed line marks the global average (0.032, population-weighted), and a diamond indicates the population-weighted mean of the group. (B) Map of country-level regional disparity available in our dataset. (C) Countries with a higher or lower level of regional disparity from each geographical group.}
    \label{fig:region_map}
\end{figure*}

% \input{sfigs/india/fig_india}

\begin{figure*}
    \centering
    % \includegraphics[width=.9\linewidth]{sfigs/india/IN_region_lit_6lang_eval.png}
    \includegraphics[width=.9\linewidth]{figS07_IN_region_lit_corr}
    \includegraphics[width=.9\linewidth]{figS07b_IN_region_lit_corr}
    \caption{A multilingual country case study with India's user population. We include the six most used languages in India (English, Hindi, Bengali, Marathi, Telugu, and Tamil) to compare the literacy estimation based on the single representative language (English) with the estimation based on multiple languages. The results suggest that literacy estimation based on multiple languages has neglectable improvement over English-based estimation in terms of Spearman's rank correlations. The y-axis indicates the officially reported literacy level, and the x-axes indicate (A) the language literacy estimated using the regions' public posts in English, (B) the estimation using multiple languages combined, and (C-E) the estimation using posts in Hindi, Bengali, and Marathi only, respectively.}
    \label{fig:india}
\end{figure*}

\begin{figure*}
    \centering
    \includegraphics[width=.9\linewidth]{figS08_multi_lang_dominant_perc}
    \caption{The relationship between the user percentage of the dominant language ($y$-axis) and (A) \olle, or (B) regional disparity, in 13 multilingual countries studied. We do not observe a clear systematic bias. This serves as a robust check to see whether there is a potential bias in the set of multilingual countries. }
    \label{fig:domi}
\end{figure*}
\clearpage
\addcontentsline{toc}{section}{Supplementary Tables}
% \section*{Supplementary Tables}\label{sec:supp_tabs}


\begin{table}[]
    \centering
    \caption{Knee points detected based on the Facebook popularity curves shown in Fig.~\ref{fig:word_curve}. The two knee points, measured in 1,000 words, determine the \bigword in each language. For example, the \bigword in English correspond to the words in the fastText `{\texttt en}' dictionary that are ranked between 5,000 to 9,000 in the decreasing order of word frequency.}
    \label{tab:knees}
    \begin{tabular}{c|c|c|c|c |c|c|c|c |c|c|c|c}
    \toprule
        Language & en & es & fr & ar & de & zh & pt & it & ru & tr & ml & ms\\
        \midrule
        $k_0$ & 5 & 5 & 5 & 5 & 5 & 5 & 5 & 5 & 5 & 5 & 5 & 6\\
        $k_1$ & 9 & 9 & 9 & 9 & 9 & 16 & 9 & 11 & 8 & 11 & 10 & 21\\
    \bottomrule
    \end{tabular}
\end{table}
% Table created by stargazer v.5.2.2 by Marek Hlavac, Harvard University. E-mail: hlavac at fas.harvard.edu
% Date and time: Mon, Jul 13, 2020 - 12:30:50 PM
\begin{table*}[!htbp] \centering {\tiny 
  \caption{OLS for predicting the reported literacy rate with online literacy estimates. Model (a) is the fixed-effect model accounting for language-specific bias. The calibrated online literacy estimates (\olles) are produced using model (a). The observations include all countries having online literacy estimates and corresponding predictors. Countries without sufficient Internet penetration ($<25\%$) are excluded to obtain reliable calibrated models. For comparison, models (b,c) include additional predictors, the Internet penetration and income. All variables were transformed to better fit a normal distribution.} 
  \label{tab:model-lit-main} 
\begin{tabular}{@{\extracolsep{-10pt}}lccc} 
\\[-1.8ex]\hline 
\hline \\[-1.8ex] 
\\[-1.8ex] & \multicolumn{3}{c}{DV: reported literacy rate} \\ 
 & (a) & (b) & (c) \\ 
\hline \\[-1.8ex] 
 est. literacy & 0.80$^{***}$ (0.61, 0.99) & 0.52$^{***}$ (0.27, 0.76) & 0.53$^{***}$ (0.27, 0.79) \\ 
  \% Internet &  & 0.32$^{***}$ (0.13, 0.50) & 0.49$^{***}$ (0.19, 0.80) \\ 
  income &  &  & $-$0.19 ($-$0.48, 0.11) \\ 
  language [de] & 1.43$^{***}$ (0.72, 2.14) & 1.22$^{***}$ (0.53, 1.90) & 1.29$^{***}$ (0.59, 1.99) \\ 
  language [en] & 0.94$^{***}$ (0.55, 1.33) & 0.98$^{***}$ (0.61, 1.35) & 0.97$^{***}$ (0.58, 1.37) \\ 
  language [es] & 1.31$^{***}$ (0.82, 1.80) & 1.15$^{***}$ (0.68, 1.62) & 1.25$^{***}$ (0.76, 1.74) \\ 
  language [fr] & 1.43$^{***}$ (0.79, 2.06) & 1.13$^{***}$ (0.51, 1.75) & 1.30$^{***}$ (0.57, 2.03) \\ 
  language [it] & 1.24$^{**}$ (0.26, 2.22) & 1.75$^{***}$ (0.78, 2.73) & 1.57$^{**}$ (0.25, 2.90) \\ 
  language [ms] & $-$0.75 ($-$1.75, 0.25) & $-$0.55 ($-$1.50, 0.40) & $-$0.52 ($-$1.48, 0.44) \\ 
  language [nl] & 0.91$^{**}$ (0.21, 1.62) & 0.78$^{**}$ (0.10, 1.45) & 0.97$^{**}$ (0.21, 1.73) \\ 
  language [pt] & 0.62$^{*}$ ($-$0.06, 1.29) & 0.61$^{*}$ ($-$0.03, 1.26) & 0.66$^{*}$ (0.01, 1.31) \\ 
  language [ru] & 3.41$^{***}$ (2.64, 4.17) & 3.14$^{***}$ (2.41, 3.88) & 3.10$^{***}$ (2.34, 3.86) \\ 
  language [tr] & 0.86$^{*}$ ($-$0.09, 1.81) & 0.96$^{**}$ (0.05, 1.86) & 1.04$^{**}$ (0.12, 1.95) \\ 
  language [zh] & $-$0.62 ($-$1.64, 0.40) & $-$0.35 ($-$1.32, 0.63) &  \\ 
  Constant & $-$0.94$^{***}$ ($-$1.27, $-$0.61) & $-$0.90$^{***}$ ($-$1.21, $-$0.59) & $-$0.94$^{***}$ ($-$1.26, $-$0.61) \\ 
 \hline \\[-1.8ex] 
OOS correlation $\rho$ & 0.78 & 0.8 & 0.79 \\ 
OOS RMSE & 0.7 & 0.66 & 0.68 \\ 
OOS R$^{2}$ & 0.51 & 0.57 & 0.55 \\ 
Observations & 98 & 98 & 86 \\ 
R$^{2}$ & 0.64 & 0.68 & 0.69 \\ 
Adjusted R$^{2}$ & 0.59 & 0.63 & 0.64 \\ 
AIC & 205.52 & 195.32 & 172.75 \\ 
BIC & 241.70 & 234.10 & 209.57 \\ 
Residual Std. Error & 0.64 (df = 85) & 0.61 (df = 84) & 0.61 (df = 72) \\ 
F Statistic & 12.49$^{***}$ (df = 12; 85) & 13.76$^{***}$ (df = 13; 84) & 12.52$^{***}$ (df = 13; 72) \\ 
\hline 
\hline \\[-1.8ex] 
\textit{Note:}  & \multicolumn{3}{r}{$^{*}$p$<$0.1; $^{**}$p$<$0.05; $^{***}$p$<$0.01} \\ 
 & \multicolumn{3}{r}{The out-of-sample (OOS) Spearman correlation $\rho$, RMSE, and R$^{2}$ are obtained using leave-one-out cross-validation.} \\ 
\end{tabular} }
\end{table*} 
% Table created by stargazer v.5.2.2 by Marek Hlavac, Harvard University. E-mail: hlavac at fas.harvard.edu
% Date and time: Wed, Jul 15, 2020 - 02:56:08 PM
\begin{table*}[!htbp] \centering {\tiny 
  \caption{OLS for predicting the reported literacy rate without online literacy estimates. All variables were transformed to better fit a normal distribution.} 
  \label{tab:model-no-lit} 
\begin{tabular}{@{\extracolsep{-10pt}}lccc} 
\\[-1.8ex]\hline 
\hline \\[-1.8ex] 
\\[-1.8ex] & \multicolumn{3}{c}{DV: reported literacy rate} \\ 
 & (a) & (b) & (c) \\ 
\hline \\[-1.8ex] 
 \% Internet & 0.54$^{***}$ (0.37, 0.71) & 0.58$^{***}$ (0.43, 0.73) & 0.74$^{***}$ (0.43, 1.05) \\ 
  income &  &  & $-$0.15 ($-$0.47, 0.18) \\ 
  language [de] &  & 1.14$^{***}$ (0.40, 1.89) & 1.22$^{***}$ (0.45, 1.98) \\ 
  language [en] &  & 0.96$^{***}$ (0.55, 1.36) & 0.95$^{***}$ (0.52, 1.38) \\ 
  language [es] &  & 0.69$^{***}$ (0.23, 1.14) & 0.79$^{***}$ (0.31, 1.27) \\ 
  language [fr] &  & 0.64$^{**}$ (0.02, 1.27) & 0.75$^{*}$ (0.004, 1.50) \\ 
  language [it] &  & 2.55$^{***}$ (1.57, 3.53) & 2.34$^{***}$ (0.95, 3.74) \\ 
  language [ms] &  & 0.08 ($-$0.90, 1.07) & 0.14 ($-$0.86, 1.14) \\ 
  language [nl] &  & 0.70$^{*}$ ($-$0.04, 1.43) & 0.95$^{**}$ (0.11, 1.78) \\ 
  language [pt] &  & 0.33 ($-$0.35, 1.01) & 0.40 ($-$0.30, 1.10) \\ 
  language [ru] &  & 2.51$^{***}$ (1.78, 3.25) & 2.49$^{***}$ (1.72, 3.26) \\ 
  language [tr] &  & 1.17$^{**}$ (0.20, 2.15) & 1.27$^{**}$ (0.27, 2.26) \\ 
  language [zh] &  & 0.42 ($-$0.56, 1.41) &  \\ 
  Constant & 0.00 ($-$0.17, 0.17) & $-$0.78$^{***}$ ($-$1.11, $-$0.44) & $-$0.83$^{***}$ ($-$1.18, $-$0.48) \\ 
 \hline \\[-1.8ex] 
OOS correlation $\rho$ & 0.52 & 0.73 & 0.72 \\ 
OOS RMSE & 0.86 & 0.7 & 0.73 \\ 
OOS R$^{2}$ & 0.26 & 0.51 & 0.48 \\ 
Observations & 98 & 98 & 86 \\ 
R$^{2}$ & 0.29 & 0.62 & 0.62 \\ 
Adjusted R$^{2}$ & 0.28 & 0.56 & 0.56 \\ 
AIC & 249.67 & 211.51 & 188.23 \\ 
BIC & 257.42 & 247.70 & 222.59 \\ 
Residual Std. Error & 0.85 (df = 96) & 0.66 (df = 85) & 0.67 (df = 73) \\ 
F Statistic & 39.04$^{***}$ (df = 1; 96) & 11.32$^{***}$ (df = 12; 85) & 10.10$^{***}$ (df = 12; 73) \\ 
\hline 
\hline \\[-1.8ex] 
\textit{Note:}  & \multicolumn{3}{r}{$^{*}$p$<$0.1; $^{**}$p$<$0.05; $^{***}$p$<$0.01} \\ 
 & \multicolumn{3}{r}{The out-of-sample (OOS) Spearman correlation $\rho$, RMSE, and R$^{2}$ are obtained using leave-one-out cross-validation.} \\ 
\end{tabular} }
\end{table*} 
\clearpage
% Table created by stargazer v.5.2.2 by Marek Hlavac, Harvard University. E-mail: hlavac at fas.harvard.edu
% Date and time: Wed, Jul 01, 2020 - 06:15:42 AM
% \begin{table}[!htbp] \centering
% \begin{tabular}{@{\extracolsep{-10pt}} lcrrrrr} 
{
\begingroup
\onecolumn
\tiny \centering
\begin{longtable}[H]{@{\extracolsep{-8pt}}lcrrrrrc}
  \caption{Online language literacy estimates for all countries included in this study. Measures include: average relative \bigword, \olle ($N=167$), female-male gender gap ($N=160$) and regional disparity ($N=119$).} 
  \label{tab:all_country} 
\\
\hline \\[-1.8ex] 
% \\ \toprule
country & code & big word ($\bar{w}$) & \olle & female-male gap & regional disparity & no. regions & rep. language \\ 
\hline \\[-1.8ex] 
Algeria & DZA & 0.584 & 0.309 & -0.584 & 0.022 & 48 & ar \\ 
American Samoa & ASM & 0.811 & 0.593 & 1.744 & -- & -- & en \\ 
Angola & AGO & 0.634 & 0.419 & -0.281 & 0.008 & 17 & pt \\ 
Anguilla & AIA & 0.521 & 0.416 & 0.372 & -- & -- & en \\ 
Antigua \& Barbuda & ATG & 0.715 & 0.519 & 1.422 & -- & -- & en \\ 
Argentina & ARG & 0.584 & 0.505 & 0.166 & 0.03 & 24 & es \\ 
Armenia & ARM & 0.308 & 0.624 & 1.684 & 0.047 & 10 & ru \\ 
Aruba & ABW & 0.785 & 0.58 & 0.476 & -- & -- & nl \\ 
Australia & AUS & 1.004 & 0.688 & 0.456 & 0.019 & 8 & en \\ 
Austria & AUT & 0.786 & 0.667 & 0.481 & 0.03 & 9 & de \\ 
Bahamas & BHS & 0.687 & 0.501 & 1.056 & 0.026 & 2 & en \\ 
Bahrain & BHR & 0.784 & 0.429 & -2.293 & 0.084 & 2 & ar \\ 
Barbados & BRB & 0.743 & 0.544 & 1.325 & -- & -- & en \\ 
Belarus & BLR & 0.709 & 0.868 & 0.746 & 0.067 & 7 & ru \\ 
Belgium & BEL & 0.876 & 0.621 & -0.411 & 0.117 & 3 & nl \\ 
Belize & BLZ & 0.729 & 0.528 & 1.011 & 0.021 & 4 & en \\ 
Benin & BEN & 0.548 & 0.498 & -1.861 & 0.042 & 12 & fr \\ 
Bermuda & BMU & 0.955 & 0.659 & 0.551 & -- & -- & en \\ 
Bolivia & BOL & 0.437 & 0.439 & -0.262 & 0.033 & 9 & es \\ 
Botswana & BWA & 0.535 & 0.422 & 0.144 & 0.004 & 8 & en \\ 
Brazil & BRA & 0.557 & 0.376 & 0.469 & 0.026 & 27 & pt \\ 
British Virgin Islands & VGB & 0.688 & 0.501 & 1.408 & -- & -- & en \\ 
Brunei & BRN & 1.522 & 0.468 & 0.465 & -- & -- & ms \\ 
Burkina Faso & BFA & 0.303 & 0.327 & -1.921 & 0.074 & 6 & fr \\ 
Burundi & BDI & 0.141 & 0.211 & -0.324 & -- & -- & fr \\ 
Cameroon & CMR & 0.638 & 0.477 & 0.016 & 0.027 & 10 & en \\ 
Canada & CAN & 0.965 & 0.664 & 0.695 & 0.065 & 13 & en \\ 
Cape Verde & CPV & 0.682 & 0.44 & 0.379 & 0.029 & 6 & pt \\ 
Caribbean Netherlands & BES & 0.849 & 0.609 & -- & -- & -- & nl \\ 
Cayman Islands & CYM & 0.872 & 0.618 & 0.949 & -- & -- & en \\ 
Central African Republic & CAF & 0.317 & 0.36 & 0 & -- & -- & fr \\ 
Chad & TCD & 0.418 & 0.202 & -3.344 & 0.056 & 2 & ar \\ 
Chile & CHL & 0.585 & 0.512 & 0.37 & 0.015 & 15 & es \\ 
Colombia & COL & 0.436 & 0.434 & 0.328 & 0.039 & 32 & es \\ 
Comoros & COM & 0.238 & 0.027 & -- & -- & -- & ar \\ 
Congo - Brazzaville & COG & 0.316 & 0.36 & -0.791 & 0.001 & 2 & fr \\ 
Congo - Kinshasa & COD & 0.312 & 0.359 & -1.552 & 0.095 & 11 & fr \\ 
Cook Islands & COK & 0.848 & 0.61 & -- & -- & -- & en \\ 
Costa Rica & CRI & 0.486 & 0.462 & -0.039 & 0.043 & 7 & es \\ 
Côte d’Ivoire & CIV & 0.299 & 0.313 & -1.604 & 0.032 & 14 & fr \\ 
Cuba & CUB & 0.475 & 0.459 & -0.078 & 0.018 & 16 & es \\ 
Curaçao & CUW & 0.795 & 0.587 & 0.022 & -- & -- & nl \\ 
Cyprus & CYP & 0.846 & 0.601 & -0.159 & 0.009 & 3 & tr \\ 
Djibouti & DJI & 0.468 & 0.258 & -2.006 & -- & -- & ar \\ 
Dominica & DMA & 0.621 & 0.467 & 0.227 & -- & -- & en \\ 
Dominican Republic & DOM & 0.386 & 0.352 & -0.189 & 0.029 & 31 & es \\ 
Ecuador & ECU & 0.451 & 0.453 & 0.002 & 0.034 & 24 & es \\ 
Egypt & EGY & 0.68 & 0.347 & -0.145 & 0.028 & 27 & ar \\ 
El Salvador & SLV & 0.435 & 0.429 & -0.937 & 0.033 & 14 & es \\ 
Equatorial Guinea & GNQ & 0.536 & 0.474 & 0.119 & 0.007 & 2 & es \\ 
Fiji & FJI & 0.678 & 0.492 & 0.482 & 0.016 & 3 & en \\ 
France & FRA & 0.767 & 0.632 & 1.094 & 0.025 & 22 & fr \\ 
French Guiana & GUF & 0.473 & 0.479 & 0.105 & -- & -- & fr \\ 
French Polynesia & PYF & 0.494 & 0.484 & 0.941 & -- & -- & fr \\ 
Gabon & GAB & 0.311 & 0.359 & -1.445 & 0.059 & 3 & fr \\ 
Gambia & GMB & 0.522 & 0.416 & -0.048 & 0.089 & 2 & en \\ 
Germany & DEU & 0.854 & 0.695 & 0.113 & 0.05 & 16 & de \\ 
Ghana & GHA & 0.576 & 0.448 & -0.697 & 0.014 & 10 & en \\ 
Gibraltar & GIB & 0.945 & 0.651 & 0.774 & -- & -- & en \\ 
Grenada & GRD & 0.704 & 0.507 & 1.918 & 0.051 & 2 & en \\ 
Guadeloupe & GLP & 0.613 & 0.538 & 0.882 & -- & -- & fr \\ 
Guam & GUM & 0.772 & 0.567 & 1.475 & -- & -- & en \\ 
Guatemala & GTM & 0.403 & 0.383 & -0.526 & 0.039 & 22 & es \\ 
Guinea & GIN & 0.348 & 0.366 & 0.085 & 0.076 & 8 & fr \\ 
Guinea-Bissau & GNB & 0.299 & 0.183 & 0.842 & -- & -- & pt \\ 
Guyana & GUY & 0.65 & 0.48 & 0.871 & 0.035 & 4 & en \\ 
Haiti & HTI & 0.28 & 0.265 & -1.976 & 0.047 & 9 & fr \\ 
Honduras & HND & 0.387 & 0.364 & -0.281 & 0.024 & 18 & es \\ 
Hong Kong SAR China & HKG & 2.021 & 0.57 & 0.02 & -- & -- & zh \\ 
India & IND & 0.423 & 0.361 & 0.887 & 0.044 & 34 & en \\ 
Iraq & IRQ & 0.745 & 0.399 & -1.859 & 0.109 & 19 & ar \\ 
Ireland & IRL & 0.944 & 0.65 & 1.235 & 0.02 & 26 & en \\ 
Isle of Man & IMN & 1.086 & 0.693 & 0.335 & -- & -- & en \\ 
Italy & ITA & 1.159 & 0.757 & -0.109 & 0.006 & 20 & it \\ 
Jamaica & JAM & 0.561 & 0.437 & 0.772 & 0.021 & 11 & en \\ 
Jersey & JEY & 0.949 & 0.655 & 1.234 & -- & -- & en \\ 
Jordan & JOR & 0.918 & 0.496 & 0.01 & 0.02 & 12 & ar \\ 
Kenya & KEN & 0.62 & 0.467 & -0.208 & 0.021 & 8 & en \\ 
Kiribati & KIR & 0.557 & 0.433 & 0.751 & -- & -- & en \\ 
Kuwait & KWT & 0.759 & 0.405 & -2.727 & 0.061 & 6 & ar \\ 
Kyrgyzstan & KGZ & 0.299 & 0.599 & 1.485 & 0.108 & 2 & ru \\ 
Lebanon & LBN & 0.734 & 0.386 & -0.771 & 0.052 & 6 & ar \\ 
Lesotho & LSO & 0.485 & 0.41 & -0.03 & 0.088 & 6 & en \\ 
Liberia & LBR & 0.74 & 0.538 & 0.483 & 0.053 & 2 & en \\ 
Libya & LBY & 0.697 & 0.355 & -0.143 & 0.017 & 19 & ar \\ 
Liechtenstein & LIE & 0.779 & 0.656 & -- & -- & -- & de \\ 
Luxembourg & LUX & 0.775 & 0.643 & 1.075 & 0.059 & 2 & fr \\ 
Macau SAR China & MAC & 1.69 & 0.509 & 0.433 & -- & -- & zh \\ 
Madagascar & MDG & 0.252 & 0.253 & 0.078 & 0.012 & 18 & fr \\ 
Malawi & MWI & 0.522 & 0.416 & -0.09 & 0.036 & 5 & en \\ 
Malaysia & MYS & 1.807 & 0.511 & 0.936 & 0.032 & 15 & ms \\ 
Mali & MLI & 0.244 & 0.25 & -0.537 & 0.051 & 6 & fr \\ 
Malta & MLT & 0.718 & 0.525 & 0.733 & 0.029 & 4 & en \\ 
Marshall Islands & MHL & 0.729 & 0.531 & 1.173 & -- & -- & en \\ 
Mauritania & MRT & 0.8 & 0.442 & -2.055 & 0.063 & 2 & ar \\ 
Mauritius & MUS & 0.568 & 0.44 & 0.59 & 0.018 & 9 & en \\ 
Mayotte & MYT & 0.406 & 0.406 & 0.773 & -- & -- & fr \\ 
Mexico & MEX & 0.428 & 0.418 & 0.318 & 0.037 & 32 & es \\ 
Micronesia (Federated States of) & FSM & 0.532 & 0.418 & 0.139 & 0.007 & 2 & en \\ 
Monaco & MCO & 0.883 & 0.7 & -- & -- & -- & fr \\ 
Morocco & MAR & 0.628 & 0.322 & -1.058 & 0.026 & 16 & ar \\ 
Mozambique & MOZ & 0.288 & 0.148 & -1.678 & 0.051 & 10 & pt \\ 
Nauru & NRU & 0.611 & 0.465 & -- & -- & -- & en \\ 
Netherlands & NLD & 0.839 & 0.603 & -0.215 & 0.025 & 12 & nl \\ 
New Caledonia & NCL & 0.469 & 0.478 & 0.032 & 0.039 & 2 & fr \\ 
New Zealand & NZL & 1.001 & 0.679 & 0.686 & 0.016 & 16 & en \\ 
Nicaragua & NIC & 0.42 & 0.406 & -0.553 & 0.031 & 17 & es \\ 
Niger & NER & 0.294 & 0.299 & -2.103 & 0.061 & 4 & fr \\ 
Nigeria & NGA & 0.814 & 0.597 & 0.002 & 0.064 & 37 & en \\ 
Northern Mariana Islands & MNP & 0.708 & 0.51 & 1.28 & -- & -- & en \\ 
Oman & OMN & 0.796 & 0.441 & -2.218 & 0.088 & 3 & ar \\ 
Pakistan & PAK & 0.455 & 0.403 & 0.262 & 0.037 & 8 & en \\ 
Palestinian Territories & PSE & 0.905 & 0.49 & 0.266 & 0.019 & 2 & ar \\ 
Panama & PAN & 0.446 & 0.449 & 0.005 & 0.043 & 8 & es \\ 
Papua New Guinea & PNG & 0.539 & 0.423 & 0.013 & 0.019 & 17 & en \\ 
Paraguay & PRY & 0.429 & 0.424 & -0.883 & 0.035 & 17 & es \\ 
Peru & PER & 0.535 & 0.47 & 0.001 & 0.009 & 25 & es \\ 
Philippines & PHL & 0.551 & 0.426 & 0.735 & 0.019 & 17 & en \\ 
Portugal & PRT & 0.711 & 0.458 & -0.41 & 0.018 & 20 & pt \\ 
Puerto Rico & PRI & 0.632 & 0.528 & 0.375 & 0.022 & 68 & es \\ 
Qatar & QAT & 0.812 & 0.444 & -2.442 & -- & -- & ar \\ 
Réunion & REU & 0.579 & 0.522 & 0.461 & -- & -- & fr \\ 
Rwanda & RWA & 0.521 & 0.416 & -0.037 & 0.015 & 5 & en \\ 
Saint Martin (French part) & MAF & 0.699 & 0.576 & -- & -- & -- & fr \\ 
Samoa & WSM & 0.655 & 0.483 & 0.725 & -- & -- & en \\ 
San Marino & SMR & 1.208 & 0.768 & -0.191 & -- & -- & it \\ 
São Tomé \& Príncipe & STP & 0.399 & 0.265 & 0.168 & -- & -- & pt \\ 
Saudi Arabia & SAU & 0.764 & 0.409 & -1.981 & 0.026 & 13 & ar \\ 
Senegal & SEN & 0.285 & 0.267 & -1.618 & 0.035 & 14 & fr \\ 
Seychelles & SYC & 0.66 & 0.486 & 0.931 & -- & -- & en \\ 
Sierra Leone & SLE & 0.759 & 0.553 & 0.714 & 0.076 & 4 & en \\ 
Singapore & SGP & 0.742 & 0.541 & 1.602 & -- & -- & en \\ 
Solomon Islands & SLB & 0.545 & 0.424 & 0.081 & -- & -- & en \\ 
Somalia & SOM & 0.608 & 0.317 & -0.959 & 0.04 & 5 & ar \\ 
South Africa & ZAF & 0.514 & 0.414 & 0.052 & 0.018 & 9 & en \\ 
South Sudan & SSD & 0.605 & 0.464 & -0.514 & 0.01 & 2 & en \\ 
Spain & ESP & 0.892 & 0.684 & 0.033 & 0.023 & 17 & es \\ 
St. Kitts \& Nevis & KNA & 0.753 & 0.55 & 1.657 & 0.003 & 2 & en \\ 
St. Lucia & LCA & 0.631 & 0.473 & 0.792 & 0.02 & 3 & en \\ 
St. Vincent \& Grenadines & VCT & 0.641 & 0.478 & 0.183 & 0.043 & 2 & en \\ 
Sudan & SDN & 0.673 & 0.34 & -0.032 & 0.165 & 9 & ar \\ 
Suriname & SUR & 0.572 & 0.442 & -0.185 & 0.031 & 2 & nl \\ 
Swaziland & SWZ & 0.555 & 0.43 & -0.326 & 0.016 & 4 & en \\ 
Switzerland & CHE & 0.812 & 0.678 & -0.113 & 0.033 & 22 & de \\ 
Syria & SYR & 0.7 & 0.356 & -0.107 & 0.105 & 14 & ar \\ 
Taiwan & TWN & 2.101 & 0.583 & 0.565 & 0.033 & 16 & zh \\ 
Tajikistan & TJK & 0.229 & 0.527 & 2.07 & 0.005 & 2 & ru \\ 
Tanzania & TZA & 0.444 & 0.397 & -0.037 & 0.012 & 19 & en \\ 
Timor-Leste & TLS & 0.416 & 0.29 & -0.193 & -- & -- & pt \\ 
Togo & TGO & 0.273 & 0.262 & -0.839 & -- & -- & fr \\ 
Tonga & TON & 0.739 & 0.538 & 1.969 & -- & -- & en \\ 
Trinidad \& Tobago & TTO & 0.677 & 0.489 & 0.948 & 0.028 & 13 & en \\ 
Tunisia & TUN & 0.709 & 0.362 & 0.058 & 0.05 & 24 & ar \\ 
Turkey & TUR & 0.835 & 0.593 & 0.055 & 0.037 & 78 & tr \\ 
Turks \& Caicos Islands & TCA & 0.676 & 0.489 & 0.935 & -- & -- & en \\ 
Uganda & UGA & 0.624 & 0.469 & 0.06 & 0.035 & 22 & en \\ 
Ukraine & UKR & 0.584 & 0.806 & 0.625 & 0.044 & 27 & ru \\ 
United Arab Emirates & ARE & 0.718 & 0.374 & -1.421 & 0.025 & 6 & ar \\ 
United Kingdom & GBR & 0.987 & 0.671 & 0.807 & 0.011 & 4 & en \\ 
United States & USA & 0.951 & 0.657 & 0.755 & 0.017 & 51 & en \\ 
Uruguay & URY & 0.572 & 0.499 & 0.104 & 0.02 & 19 & es \\ 
U.S. Virgin Islands & VIR & 0.842 & 0.607 & 1.41 & -- & -- & en \\ 
Uzbekistan & UZB & 0.223 & 0.524 & 1.558 & 0.018 & 9 & ru \\ 
Venezuela & VEN & 0.437 & 0.444 & -1.165 & 0.028 & 24 & es \\ 
Yemen & YEM & 0.979 & 0.52 & -0.703 & 0.048 & 7 & ar \\ 
Zambia & ZMB & 0.568 & 0.441 & -0.083 & 0.016 & 9 & en \\ 
Zimbabwe & ZWE & 0.602 & 0.464 & -0.295 & 0.024 & 8 & en \\ 
\hline \\[-1.8ex] 
\multicolumn{8}{l}{The representative language for each country is chosen as the most used language by the country's population observed on Facebook.} \\ 
% \bottomrule
% \end{tabular} 
% \end{table} 
\end{longtable} 
\endgroup
}
% Table created by stargazer v.5.2.2 by Marek Hlavac, Harvard University. E-mail: hlavac at fas.harvard.edu
% Date and time: Tue, Jul 14, 2020 - 01:22:52 PM
\begin{table*}[!htbp] \centering {\tiny
  \caption{Variables related to gender gap analysis. Reported $N$ is the number of countries matched with our data.} 
  \label{tab:gender_vars}
\begin{tabular}{@{\extracolsep{-4pt}}lrrrrrrll} 
\\[-1.8ex]\hline 
\hline \\[-1.8ex] 
Statistic & \multicolumn{1}{c}{N} & \multicolumn{1}{c}{Mean} & \multicolumn{1}{c}{St. Dev.} & \multicolumn{1}{c}{Min} & \multicolumn{1}{c}{Max} & \multicolumn{1}{c}{Median} & \multicolumn{1}{l}{Definition} & \multicolumn{1}{l}{Source} \\ 
\hline \\[-1.8ex] 
\olle gap & 160 & 0.053 & 1.002 & $-$3.344 & 2.070 & 0.080 & female-map gap in \olle &\\ 
offline literacy (all) & 143 & 84.181 & 19.263 & 19.100 & 100.000 & 93.464 & literacy rate & UNESCO \cite{unesco2019produce}\\  
offline literacy (gap) & 114 & $-$0.068 & 0.090 & $-$0.301 & 0.182 & $-$0.036 & female-male gap in literacy & UNESCO \cite{unesco2019produce}\\ 
education (all) & 106 & 7.884 & 2.741 & 1.880 & 13.180 & 8.085 & mean schooling years & Barro-Lee Educational Attainment Data \cite{BarroLee73:online}\\ 
education (gap) & 106 & $-$0.517 & 0.968 & $-$3.250 & 1.600 & $-$0.420 & female-map gap in schooling years & Barro-Lee Educational Attainment Data \cite{BarroLee73:online}\\ 
\% Internet (all) & 151 & 0.520 & 0.284 & 0.020 & 0.984 & 0.555 & overall Internet penetration & ITU Internet gender gap \cite{Aboutthe43:online}\\
\% Internet (gap) & 128 & 0.880 & 0.123 & 0.545 & 1.000 & 0.919 & femal-male gap in Internet penetration & Digital gender gap (U. Oxford) \cite{Aboutthe43:online} \\ 
civic (all) & 123 & 0.695 & 0.211 & 0.105 & 0.973 & 0.746 & overall civic society participation & V-Dem Institute \cite{coppedge2019v}\\ 
civic (women) & 123 & 0.720 & 0.173 & 0.234 & 0.937 & 0.775 & women's civic society participation & V-Dem Institute \cite{coppedge2019v}\\ 
GII & 107 & 0.398 & 0.194 & 0.040 & 0.835 & 0.424 & Gender Inequality Index & HDRO \cite{HDROAPII26:online}\\  
\hline \\[-1.8ex] 
\end{tabular} }
\end{table*} 
% Table created by stargazer v.5.2.2 by Marek Hlavac, Harvard University. E-mail: hlavac at fas.harvard.edu
% Date and time: Tue, Jul 14, 2020 - 01:35:00 PM
\begin{table*}[!htbp] \centering {\tiny 
  \caption{Correlations among variables related to gender gap or gender equity. All correlations are reported using Spearman rank correlation coefficients.} 
  \label{tab:gender_corr} 
\begin{tabular}{@{\extracolsep{-4pt}} lllllllllll} 
\\[-1.8ex]\hline 
\hline \\[-1.8ex] 
 & \olle gap & \olle & \shortstack[l]{offline\\[-.5ex]literacy (all)} & \shortstack[l]{offline\\[-.5ex]literacy (gap)} & eduation (all) & education (gap) & \% Internet (all) & \% Internet (gap) & civic (all) & civic (women) \\ 
\hline \\[-1.8ex] 
\olle &  0.585$^*$$^*$$^*$ &  &  &  &  &  &  &  &  &  \\ 
off. literacy (all) &  0.524$^*$$^*$$^*$ &  0.740$^*$$^*$$^*$ &  &  &  &  &  &  &  &  \\ 
off. literacy (gap) &  0.420$^*$$^*$$^*$ &  0.438$^*$$^*$$^*$ &  0.709$^*$$^*$$^*$ &  &  &  &  &  &  &  \\ 
eduation (all) &  0.587$^*$$^*$$^*$ &  0.739$^*$$^*$$^*$ &  0.872$^*$$^*$$^*$ &  0.694$^*$$^*$$^*$ &  &  &  &  &  &  \\ 
education (gap) &  0.225$^*$  &  0.207$^*$  &  0.515$^*$$^*$$^*$ &  0.787$^*$$^*$$^*$ &  0.455$^*$$^*$$^*$ &  &  &  &  &  \\ 
\% Internet (all) &  0.304$^*$$^*$$^*$ &  0.573$^*$$^*$$^*$ &  0.748$^*$$^*$$^*$ &  0.582$^*$$^*$$^*$ &  0.743$^*$$^*$$^*$ &  0.434$^*$$^*$$^*$ &  &  &  &  \\ 
\% Internet (gap) &  0.487$^*$$^*$$^*$ &  0.533$^*$$^*$$^*$ &  0.663$^*$$^*$$^*$ &  0.779$^*$$^*$$^*$ &  0.730$^*$$^*$$^*$ &  0.620$^*$$^*$$^*$ &  0.617$^*$$^*$$^*$ &  &  &  \\ 
civic (all) &  0.287$^*$$^*$  &  0.355$^*$$^*$$^*$ &  0.159  & -0.019  &  0.358$^*$$^*$$^*$ &  0.028  &  0.228$^*$  &  0.328$^*$$^*$$^*$ &  &  \\ 
civic (women) &  0.479$^*$$^*$$^*$ &  0.484$^*$$^*$$^*$ &  0.394$^*$$^*$$^*$ &  0.317$^*$$^*$  &  0.564$^*$$^*$$^*$ &  0.252$^*$  &  0.429$^*$$^*$$^*$ &  0.598$^*$$^*$$^*$ &  0.695$^*$$^*$$^*$ &  \\ 
GII & -0.391$^*$$^*$$^*$ & -0.624$^*$$^*$$^*$ & -0.847$^*$$^*$$^*$ & -0.610$^*$$^*$$^*$ & -0.838$^*$$^*$$^*$ & -0.475$^*$$^*$$^*$ & -0.872$^*$$^*$$^*$ & -0.647$^*$$^*$$^*$ & -0.219$^*$  & -0.414$^*$$^*$$^*$ \\ 
\hline \\[-1.8ex] 
\multicolumn{11}{l}{significance levels: $^{***}p<0.001$; $^{**}p<0.01$; $^*p<0.05$} \\ 
\end{tabular} }
\end{table*} 

% Table created by stargazer v.5.2.2 by Marek Hlavac, Harvard University. E-mail: hlavac at fas.harvard.edu
% Date and time: Tue, Jul 14, 2020 - 01:40:52 PM
\begin{table*}[!htbp] \centering {\tiny
  \caption{Variables related to country resource access and inequality. Reported $N$ is the number of countries matched with our data.} 
  \label{tab:region_vars} 
\begin{tabular}{@{\extracolsep{-4pt}}lrrrrrrll} 
\\[-1.8ex]\hline 
\hline \\[-1.8ex] 
Statistic & \multicolumn{1}{c}{N} & \multicolumn{1}{c}{Mean} & \multicolumn{1}{c}{St. Dev.} & \multicolumn{1}{c}{Min} & \multicolumn{1}{c}{Max} & \multicolumn{1}{c}{Median} & \multicolumn{1}{l}{Definition} & \multicolumn{1}{l}{Source} \\  
\hline \\[-1.8ex] 
regional disparity & 119 & 0.037 & 0.027 & 0.001 & 0.165 & 0.030 & St. Dev. of sub-national \olle \\  
income & 115 & 16.193 & 17.054 & 0.800 & 71.160 & 9.359 & GNI per capita in 1,000 US\$ (2011 PPP) & HDI \cite{GlobalDa32:online}\\ 
Gini index & 94 & 40.004 & 7.973 & 25.000 & 63.000 & 40.200 & Gini coefficient for income & HDR \cite{HumanDev90:online}\\  
education (all) & 95 & 7.864 & 2.753 & 1.880 & 13.180 & 7.970 & mean schooling years & Barro-Lee Educational Attainment Data \cite{BarroLee73:online}\\ 
unequal education & 102 & 21.025 & 14.073 & 0.800 & 49.300 & 17.450 & Inequality in education & HDR \cite{HumanDev90:online}\\  
\% Internet (all) & 119 & 0.507 & 0.275 & 0.020 & 0.980 & 0.508 & overall Internet penetration & ITU Internet gender gap \cite{Aboutthe43:online}\\ 
civic (all) & 110 & 0.704 & 0.213 & 0.105 & 0.973 & 0.764 & overall civic society participation & V-Dem Institute \cite{coppedge2019v}\\ 
\hline \\[-1.8ex] 
\end{tabular} }
\end{table*} 
% Table created by stargazer v.5.2.2 by Marek Hlavac, Harvard University. E-mail: hlavac at fas.harvard.edu
% Date and time: Tue, Jul 14, 2020 - 01:45:05 PM
\begin{table*}[!htbp] \centering {\tiny 
  \caption{Correlations among variables related to country resource access and inequality. All correlations are reported using Spearman rank correlation coefficients.} 
  \label{tab:region_corr} 
\begin{tabular}{@{\extracolsep{-4pt}} llllllll} 
\\[-1.8ex]\hline 
\hline \\[-1.8ex] 
 & regional disparity & \olle & income & Gini index & education (all) & unequal education & \% Internet (all) \\ 
\hline \\[-1.8ex] 
regional disparity &  &  &  &  &  &  &  \\ 
\olle & -0.158  &  &  &  &  &  &  \\ 
income & -0.156  &  0.476$^*$$^*$$^*$ &  &  &  &  &  \\ 
Gini index & -0.271$^*$$^*$  & -0.346$^*$$^*$$^*$ & -0.260$^*$  &  &  &  &  \\ 
eduation (all) & -0.312$^*$$^*$  &  0.706$^*$$^*$$^*$ &  0.745$^*$$^*$$^*$ & -0.303$^*$$^*$  &  &  &  \\ 
unequal education &  0.283$^*$$^*$  & -0.681$^*$$^*$$^*$ & -0.703$^*$$^*$$^*$ &  0.240$^*$  & -0.875$^*$$^*$$^*$ &  &  \\ 
\% Internet (all) & -0.043  &  0.532$^*$$^*$$^*$ &  0.905$^*$$^*$$^*$ & -0.306$^*$$^*$  &  0.751$^*$$^*$$^*$ & -0.735$^*$$^*$$^*$ &  \\ 
civic (all) & -0.012  &  0.344$^*$$^*$$^*$ &  0.192$^*$  & -0.154  &  0.366$^*$$^*$$^*$ & -0.207$^*$  &  0.237$^*$  \\ 
\hline \\[-1.8ex] 
\multicolumn{8}{l}{significance levels: $^{***}p<0.001$; $^{**}p<0.01$; $^*p<0.05$} \\ 
\end{tabular} }
\end{table*} 




% Table created by stargazer v.5.2.2 by Marek Hlavac, Harvard University. E-mail: hlavac at fas.harvard.edu
% Date and time: Tue, Jul 14, 2020 - 03:07:54 PM
\begin{table*}[!htbp] \centering {\tiny 
  \caption{OLS for predicting female-male gender gap in online literacy estimates. Models (d) corresponds to the figure in the main text. Models (b,c,e,f) include alternative interaction terms. Values in parentheses are the lower and upper bounds of the 95\% confidence intervals of the estimated effects.} 
  \label{tab:model-gender-main} 
\begin{tabular}{@{\extracolsep{-10pt}}lcccccc} 
\\[-1.8ex]\hline 
\hline \\[-1.8ex] 
\\[-1.8ex] & \multicolumn{6}{c}{DV: female-male gender gap} \\ 
 & (a) & (b) & (c) & (d) & (e) & (f) \\ 
\hline \\[-1.8ex] 
 women civic & 0.32$^{***}$ & 0.33$^{***}$ & 0.35$^{***}$ & 0.38$^{***}$ & 0.37$^{***}$ & 0.40$^{***}$ \\ 
  & (0.15, 0.48) & (0.14, 0.52) & (0.17, 0.52) & (0.20, 0.56) & (0.18, 0.56) & (0.22, 0.57) \\ 
  & & & & & & \\ 
 (women civic):(\% Internet) & 0.25$^{***}$ &  &  & 0.13$^{*}$ &  &  \\ 
  & (0.10, 0.41) &  &  & ($-$0.02, 0.29) &  &  \\ 
  & & & & & & \\ 
 (women civic):(education) &  & 0.28 &  &  & 0.19 &  \\ 
  &  & ($-$0.10, 0.66) &  &  & ($-$0.14, 0.53) &  \\ 
  & & & & & & \\ 
 education (all) & 0.25$^{***}$ & 0.37$^{***}$ & 0.90$^{**}$ & 0.20$^{**}$ & 0.28$^{**}$ & 0.66$^{**}$ \\ 
  & (0.08, 0.42) & (0.13, 0.61) & (0.19, 1.61) & (0.05, 0.35) & (0.06, 0.49) & (0.03, 1.28) \\ 
  & & & & & & \\ 
 (education):(\% Internet) &  &  & 0.51$^{*}$ &  &  & 0.36 \\ 
  &  &  & ($-$0.02, 1.05) &  &  & ($-$0.11, 0.84) \\ 
  & & & & & & \\ 
 \% Internet (all) & $-$0.02 & $-$0.08 & $-$0.23$^{*}$ & $-$0.22$^{*}$ & $-$0.28$^{**}$ & $-$0.41$^{***}$ \\ 
  & ($-$0.21, 0.16) & ($-$0.27, 0.11) & ($-$0.50, 0.03) & ($-$0.45, 0.01) & ($-$0.50, $-$0.06) & ($-$0.69, $-$0.13) \\ 
  & & & & & & \\ 
 Central/Southern/Eastern Asia &  &  &  & 1.66$^{***}$ & 1.77$^{***}$ & 1.80$^{***}$ \\ 
  &  &  &  & (1.05, 2.27) & (1.17, 2.37) & (1.22, 2.39) \\ 
  & & & & & & \\ 
 Europe/Oceania/Northern America &  &  &  & 0.63$^{**}$ & 0.76$^{***}$ & 0.79$^{***}$ \\ 
  &  &  &  & (0.03, 1.22) & (0.20, 1.33) & (0.25, 1.34) \\ 
  & & & & & & \\ 
 Latin America \& the Caribbean &  &  &  & 0.29 & 0.32 & 0.38$^{*}$ \\ 
  &  &  &  & ($-$0.15, 0.74) & ($-$0.12, 0.77) & ($-$0.06, 0.82) \\ 
  & & & & & & \\ 
 Northern Africa \& Western Asia &  &  &  & 0.29 & 0.30 & 0.41 \\ 
  &  &  &  & ($-$0.26, 0.84) & ($-$0.26, 0.85) & ($-$0.14, 0.96) \\ 
  & & & & & & \\ 
 Constant & $-$0.35$^{***}$ & $-$0.30$^{***}$ & $-$0.47$^{***}$ & $-$0.69$^{***}$ & $-$0.72$^{***}$ & $-$0.89$^{***}$ \\ 
  & ($-$0.51, $-$0.18) & ($-$0.48, $-$0.13) & ($-$0.75, $-$0.19) & ($-$1.02, $-$0.37) & ($-$1.05, $-$0.40) & ($-$1.28, $-$0.50) \\ 
  & & & & & & \\ 
\hline \\[-1.8ex] 
Out-of-sample RMSE & 0.8 & 0.83 & 0.83 & 0.72 & 0.73 & 0.73 \\ 
Out-of-sample R2 & 0.24 & 0.21 & 0.2 & 0.39 & 0.38 & 0.38 \\ 
Observations & 101 & 101 & 101 & 101 & 101 & 101 \\ 
R$^{2}$ & 0.34 & 0.29 & 0.30 & 0.51 & 0.51 & 0.51 \\ 
Adjusted R$^{2}$ & 0.31 & 0.26 & 0.27 & 0.47 & 0.46 & 0.47 \\ 
AIC & 240.92 & 248.67 & 247.25 & 217.60 & 219.21 & 218.11 \\ 
BIC & 256.61 & 264.36 & 262.94 & 243.75 & 245.36 & 244.26 \\ 
Residual Std. Error & 0.77 (df = 96) & 0.80 (df = 96) & 0.80 (df = 96) & 0.67 (df = 92) & 0.68 (df = 92) & 0.68 (df = 92) \\ 
F Statistic & 12.26$^{***}$ (df = 4; 96) & 9.58$^{***}$ (df = 4; 96) & 10.06$^{***}$ (df = 4; 96) & 12.19$^{***}$ (df = 8; 92) & 11.82$^{***}$ (df = 8; 92) & 12.07$^{***}$ (df = 8; 92) \\ 
\hline 
\hline \\[-1.8ex] 
\textit{Note:}  & \multicolumn{6}{r}{$^{*}$p$<$0.1; $^{**}$p$<$0.05; $^{***}$p$<$0.01} \\ 
\end{tabular} }
\end{table*} 



% Table created by stargazer v.5.2.2 by Marek Hlavac, Harvard University. E-mail: hlavac at fas.harvard.edu
% Date and time: Tue, Jul 14, 2020 - 03:07:59 PM
\begin{table*}[!htbp] \centering { \tiny 
  \caption{OLS for predicting female-male gender gap in online literacy estimates. Models include alternative predictors and no interaction term. Values in parentheses are the lower and upper bounds of the 95\% confidence intervals of the estimated effects.} 
  \label{tab:model-gender-3v} 
\begin{tabular}{@{\extracolsep{-10pt}}lcccccc} 
\\[-1.8ex]\hline 
\hline \\[-1.8ex] 
\\[-1.8ex] & \multicolumn{6}{c}{DV: female-male gender gap} \\ 
 & (a) & (b) & (c) & (d) & (e) & (f) \\ 
\hline \\[-1.8ex] 
 women civic & 0.39$^{***}$ & 0.39$^{***}$ & 0.40$^{***}$ & 0.37$^{***}$ &  &  \\ 
  & (0.22, 0.56) & (0.22, 0.57) & (0.20, 0.60) & (0.15, 0.59) &  &  \\ 
  & & & & & & \\ 
 civic (all) &  &  &  &  & 0.21$^{**}$ & 0.21$^{**}$ \\ 
  &  &  &  &  & (0.05, 0.38) & (0.01, 0.41) \\ 
  & & & & & & \\ 
 education (gap) &  & 0.05 &  & 0.02 &  &  \\ 
  &  & ($-$0.12, 0.23) &  & ($-$0.19, 0.24) &  &  \\ 
  & & & & & & \\ 
 education (all) & 0.24$^{***}$ &  & 0.24$^{**}$ &  & 0.27$^{***}$ & 0.24$^{**}$ \\ 
  & (0.07, 0.42) &  & (0.06, 0.43) &  & (0.09, 0.46) & (0.04, 0.43) \\ 
  & & & & & & \\ 
 \% Internet (gap) &  &  & $-$0.03 & 0.10 &  & 0.15 \\ 
  &  &  & ($-$0.25, 0.20) & ($-$0.17, 0.37) &  & ($-$0.06, 0.36) \\ 
  & & & & & & \\ 
 \% Internet (all) & $-$0.05 & 0.04 &  &  & 0.07 &  \\ 
  & ($-$0.24, 0.14) & ($-$0.16, 0.23) &  &  & ($-$0.12, 0.26) &  \\ 
  & & & & & & \\ 
 Constant & $-$0.25$^{***}$ & $-$0.25$^{***}$ & $-$0.26$^{***}$ & $-$0.25$^{***}$ & $-$0.24$^{***}$ & $-$0.25$^{***}$ \\ 
  & ($-$0.40, $-$0.09) & ($-$0.41, $-$0.09) & ($-$0.42, $-$0.09) & ($-$0.42, $-$0.08) & ($-$0.41, $-$0.07) & ($-$0.42, $-$0.07) \\ 
  & & & & & & \\ 
\hline \\[-1.8ex] 
Out-of-sample RMSE & 0.83 & 0.87 & 0.85 & 0.88 & 0.91 & 0.92 \\ 
Out-of-sample R2 & 0.19 & 0.13 & 0.21 & 0.13 & 0.09 & 0.1 \\ 
Observations & 101 & 101 & 98 & 98 & 101 & 98 \\ 
R$^{2}$ & 0.27 & 0.22 & 0.28 & 0.22 & 0.17 & 0.19 \\ 
Adjusted R$^{2}$ & 0.25 & 0.19 & 0.25 & 0.20 & 0.15 & 0.17 \\ 
AIC & 248.88 & 256.07 & 243.76 & 250.52 & 261.40 & 254.21 \\ 
BIC & 261.96 & 269.14 & 256.69 & 263.44 & 274.48 & 267.13 \\ 
Residual Std. Error & 0.81 (df = 97) & 0.83 (df = 97) & 0.81 (df = 94) & 0.84 (df = 94) & 0.86 (df = 97) & 0.86 (df = 94) \\ 
F Statistic & 11.93$^{***}$ (df = 3; 97) & 8.89$^{***}$ (df = 3; 97) & 11.96$^{***}$ (df = 3; 94) & 9.07$^{***}$ (df = 3; 94) & 6.77$^{***}$ (df = 3; 97) & 7.58$^{***}$ (df = 3; 94) \\ 
\hline 
\hline \\[-1.8ex] 
\textit{Note:}  & \multicolumn{6}{r}{$^{*}$p$<$0.1; $^{**}$p$<$0.05; $^{***}$p$<$0.01} \\ 
 & \multicolumn{6}{r}{Models (c,d,f) have fewer observations (and slightly lower prediction error) due to missing data in the new predictor.} \\ 
\end{tabular} }
\end{table*} 


% Table created by stargazer v.5.2.2 by Marek Hlavac, Harvard University. E-mail: hlavac at fas.harvard.edu
% Date and time: Tue, Jul 14, 2020 - 03:08:06 PM
\begin{table*}[!htbp] \centering {\tiny 
  \caption{OLS for predicting female-male gender gap in online literacy estimates. Models include alternative predictors and controls for geographical groups. Values in parentheses are the lower and upper bounds of the 95\% confidence intervals of the estimated effects.} 
  \label{tab:model-gender-3v-geo} 
\begin{tabular}{@{\extracolsep{-10pt}}lcccccc} 
\\[-1.8ex]\hline 
\hline \\[-1.8ex] 
\\[-1.8ex] & \multicolumn{6}{c}{DV: female-male gender gap} \\ 
 & (a) & (b) & (c) & (d) & (e) & (f) \\ 
\hline \\[-1.8ex] 
 women civic & 0.41$^{***}$ & 0.43$^{***}$ & 0.35$^{***}$ & 0.37$^{***}$ &  &  \\ 
  & (0.23, 0.58) & (0.26, 0.61) & (0.16, 0.53) & (0.17, 0.56) &  &  \\ 
  & & & & & & \\ 
 civic (all) &  &  &  &  & 0.16$^{*}$ & 0.16 \\ 
  &  &  &  &  & ($-$0.02, 0.34) & ($-$0.04, 0.37) \\ 
  & & & & & & \\ 
 education (gap) &  & 0.13$^{*}$ &  & 0.10 &  &  \\ 
  &  & ($-$0.02, 0.29) &  & ($-$0.08, 0.29) &  &  \\ 
  & & & & & & \\ 
 education (all) & 0.19$^{**}$ &  & 0.19$^{**}$ &  & 0.21$^{**}$ & 0.18$^{**}$ \\ 
  & (0.04, 0.34) &  & (0.02, 0.36) &  & (0.05, 0.38) & (0.01, 0.36) \\ 
  & & & & & & \\ 
 \% Internet (gap) &  &  & $-$0.07 & $-$0.07 &  & 0.08 \\ 
  &  &  & ($-$0.30, 0.15) & ($-$0.33, 0.20) &  & ($-$0.14, 0.30) \\ 
  & & & & & & \\ 
 \% Internet (all) & $-$0.28$^{**}$ & $-$0.30$^{**}$ &  &  & $-$0.11 &  \\ 
  & ($-$0.50, $-$0.06) & ($-$0.54, $-$0.06) &  &  & ($-$0.34, 0.11) &  \\ 
  & & & & & & \\ 
 Central/Southern/Eastern Asia & 1.81$^{***}$ & 1.99$^{***}$ & 1.48$^{***}$ & 1.64$^{***}$ & 1.67$^{***}$ & 1.50$^{***}$ \\ 
  & (1.22, 2.41) & (1.40, 2.59) & (0.92, 2.03) & (1.09, 2.19) & (1.02, 2.32) & (0.91, 2.10) \\ 
  & & & & & & \\ 
 Europe/Oceania/Northern America & 0.84$^{***}$ & 0.97$^{***}$ & 0.46$^{*}$ & 0.58$^{**}$ & 0.83$^{***}$ & 0.53$^{*}$ \\ 
  & (0.30, 1.39) & (0.42, 1.53) & ($-$0.03, 0.96) & (0.07, 1.08) & (0.24, 1.43) & (0.002, 1.05) \\ 
  & & & & & & \\ 
 Latin America \& the Caribbean & 0.37 & 0.44$^{*}$ & 0.18 & 0.25 & 0.39 & 0.17 \\ 
  & ($-$0.07, 0.81) & ($-$0.004, 0.88) & ($-$0.28, 0.65) & ($-$0.22, 0.72) & ($-$0.09, 0.88) & ($-$0.32, 0.67) \\ 
  & & & & & & \\ 
 Northern Africa \& Western Asia & 0.35 & 0.48$^{*}$ & $-$0.16 & $-$0.04 & 0.03 & $-$0.20 \\ 
  & ($-$0.20, 0.90) & ($-$0.07, 1.04) & ($-$0.63, 0.31) & ($-$0.50, 0.42) & ($-$0.58, 0.65) & ($-$0.74, 0.33) \\ 
  & & & & & & \\ 
 Constant & $-$0.73$^{***}$ & $-$0.81$^{***}$ & $-$0.50$^{***}$ & $-$0.58$^{***}$ & $-$0.65$^{***}$ & $-$0.50$^{***}$ \\ 
  & ($-$1.06, $-$0.40) & ($-$1.14, $-$0.48) & ($-$0.80, $-$0.20) & ($-$0.88, $-$0.28) & ($-$1.02, $-$0.28) & ($-$0.82, $-$0.17) \\ 
  & & & & & & \\ 
\hline \\[-1.8ex] 
Out-of-sample RMSE & 0.73 & 0.75 & 0.76 & 0.8 & 0.8 & 0.82 \\ 
Out-of-sample R$^{2}$ & 0.37 & 0.35 & 0.35 & 0.31 & 0.28 & 0.29 \\ 
Observations & 101 & 101 & 98 & 98 & 101 & 98 \\ 
R$^{2}$ & 0.50 & 0.48 & 0.48 & 0.46 & 0.41 & 0.42 \\ 
Adjusted R$^{2}$ & 0.46 & 0.44 & 0.44 & 0.42 & 0.36 & 0.37 \\ 
AIC & 218.59 & 222.02 & 218.82 & 222.81 & 235.76 & 230.10 \\ 
BIC & 242.12 & 245.56 & 242.09 & 246.07 & 259.30 & 253.36 \\ 
Residual Std. Error & 0.68 (df = 93) & 0.69 (df = 93) & 0.70 (df = 90) & 0.72 (df = 90) & 0.74 (df = 93) & 0.75 (df = 90) \\ 
F Statistic & 13.29$^{***}$ (df = 7; 93) & 12.40$^{***}$ (df = 7; 93) & 12.00$^{***}$ (df = 7; 90) & 11.01$^{***}$ (df = 7; 90) & 9.13$^{***}$ (df = 7; 93) & 9.30$^{***}$ (df = 7; 90) \\ 
\hline 
\hline \\[-1.8ex] 
\textit{Note:}  & \multicolumn{6}{r}{$^{*}$p$<$0.1; $^{**}$p$<$0.05; $^{***}$p$<$0.01} \\ 
 & \multicolumn{6}{r}{Models (c,d,f) have fewer observations (and slightly lower prediction error) due to missing data in the new predictor.} \\ 
\end{tabular} }
\end{table*} 




% Table created by stargazer v.5.2.2 by Marek Hlavac, Harvard University. E-mail: hlavac at fas.harvard.edu
% Date and time: Tue, Jul 14, 2020 - 03:27:10 PM
\begin{table*}[!htbp] \centering {\tiny 
  \caption{OLS for predicting within-country regional disparity in online literacy estimates. Models (d) corresponds to the figure in the main text. Models (b,c,e,f) include alternative interaction terms. Values in parentheses are the lower and upper bounds of the 95\% confidence intervals of the estimated effects.} 
  \label{tab:model-region-main} 
\begin{tabular}{@{\extracolsep{-10pt}}lcccccc} 
\\[-1.8ex]\hline 
\hline \\[-1.8ex] 
\\[-1.8ex] & \multicolumn{6}{c}{DV: within-country regional disparity} \\ 
 & (a) & (b) & (c) & (d) & (e) & (f) \\ 
\hline \\[-1.8ex] 
 unequal edu & 0.47$^{**}$ & 0.34$^{**}$ & 0.32$^{*}$ & 0.41$^{**}$ & 0.25 & 0.25 \\ 
  & (0.10, 0.84) & (0.02, 0.66) & ($-$0.02, 0.66) & (0.03, 0.80) & ($-$0.10, 0.60) & ($-$0.12, 0.63) \\ 
  & & & & & & \\ 
 (unequal edu):(Gini) & 0.18 &  &  & 0.25$^{*}$ &  &  \\ 
  & ($-$0.07, 0.44) &  &  & ($-$0.02, 0.51) &  &  \\ 
  & & & & & & \\ 
 (unequal edu):(\% Internet) &  & $-$0.09 &  &  & $-$0.15 &  \\ 
  &  & ($-$0.36, 0.18) &  &  & ($-$0.46, 0.17) &  \\ 
  & & & & & & \\ 
 (unequal edu):(education) &  &  & 0.003 &  &  & 0.01 \\ 
  &  &  & ($-$0.37, 0.38) &  &  & ($-$0.38, 0.41) \\ 
  & & & & & & \\ 
 Gini & $-$0.22$^{**}$ & $-$0.22$^{**}$ & $-$0.24$^{**}$ & $-$0.35$^{**}$ & $-$0.32$^{**}$ & $-$0.37$^{**}$ \\ 
  & ($-$0.42, $-$0.02) & ($-$0.43, $-$0.02) & ($-$0.46, $-$0.02) & ($-$0.61, $-$0.09) & ($-$0.61, $-$0.04) & ($-$0.67, $-$0.08) \\ 
  & & & & & & \\ 
 \% Internet (all) & 0.40$^{**}$ & 0.33$^{*}$ & 0.35$^{**}$ & 0.46$^{**}$ & 0.34$^{*}$ & 0.41$^{*}$ \\ 
  & (0.08, 0.72) & (0.0000, 0.66) & (0.01, 0.70) & (0.08, 0.83) & ($-$0.05, 0.74) & ($-$0.02, 0.83) \\ 
  & & & & & & \\ 
 education (all) & $-$0.29$^{**}$ & $-$0.27$^{*}$ & $-$0.31 & $-$0.33$^{**}$ & $-$0.29$^{*}$ & $-$0.36 \\ 
  & ($-$0.55, $-$0.02) & ($-$0.56, 0.02) & ($-$0.83, 0.21) & ($-$0.60, $-$0.06) & ($-$0.59, $-$0.003) & ($-$0.90, 0.19) \\ 
  & & & & & & \\ 
 Central/Southern/Eastern Asia &  &  &  & $-$0.17 & $-$0.05 & $-$0.24 \\ 
  &  &  &  & ($-$1.06, 0.72) & ($-$1.04, 0.94) & ($-$1.16, 0.68) \\ 
  & & & & & & \\ 
 Europe/Oceania/Northern America &  &  &  & $-$0.54 & $-$0.42 & $-$0.47 \\ 
  &  &  &  & ($-$1.51, 0.43) & ($-$1.40, 0.56) & ($-$1.47, 0.54) \\ 
  & & & & & & \\ 
 Latin America \& the Caribbean &  &  &  & 0.26 & 0.27 & 0.19 \\ 
  &  &  &  & ($-$0.36, 0.88) & ($-$0.38, 0.92) & ($-$0.46, 0.84) \\ 
  & & & & & & \\ 
 Northern Africa \& Western Asia &  &  &  & $-$0.09 & $-$0.03 & $-$0.18 \\ 
  &  &  &  & ($-$0.89, 0.72) & ($-$0.90, 0.83) & ($-$1.03, 0.67) \\ 
  & & & & & & \\ 
 Constant & $-$0.04 & $-$0.06 & 0.004 & 0.03 & $-$0.06 & 0.11 \\ 
  & ($-$0.24, 0.16) & ($-$0.33, 0.21) & ($-$0.30, 0.31) & ($-$0.47, 0.52) & ($-$0.66, 0.54) & ($-$0.51, 0.73) \\ 
  & & & & & & \\ 
\hline \\[-1.8ex] 
Out-of-sample RMSE & 0.9 & 0.92 & 0.92 & 0.92 & 0.95 & 0.95 \\ 
Out-of-sample R$^{2}$ & 0.14 & 0.1 & 0.11 & 0.09 & 0.08 & 0.06 \\ 
Observations & 79 & 79 & 79 & 79 & 79 & 79 \\ 
R$^{2}$ & 0.25 & 0.23 & 0.23 & 0.29 & 0.26 & 0.25 \\ 
Adjusted R$^{2}$ & 0.19 & 0.18 & 0.17 & 0.19 & 0.17 & 0.16 \\ 
AIC & 207.37 & 209.02 & 209.48 & 210.91 & 213.57 & 214.53 \\ 
BIC & 223.96 & 225.60 & 226.07 & 236.98 & 239.63 & 240.59 \\ 
Residual Std. Error & 0.86 (df = 73) & 0.86 (df = 73) & 0.87 (df = 73) & 0.86 (df = 69) & 0.87 (df = 69) & 0.88 (df = 69) \\ 
F Statistic & 4.75$^{***}$ (df = 5; 73) & 4.36$^{***}$ (df = 5; 73) & 4.24$^{***}$ (df = 5; 73) & 3.09$^{***}$ (df = 9; 69) & 2.73$^{***}$ (df = 9; 69) & 2.61$^{**}$ (df = 9; 69) \\ 
\hline 
\hline \\[-1.8ex] 
\textit{Note:}  & \multicolumn{6}{r}{$^{*}$p$<$0.1; $^{**}$p$<$0.05; $^{***}$p$<$0.01} \\ 
\end{tabular} }
\end{table*} 


% Table created by stargazer v.5.2.2 by Marek Hlavac, Harvard University. E-mail: hlavac at fas.harvard.edu
% Date and time: Tue, Jul 14, 2020 - 03:30:01 PM
\begin{table*}[!htbp] \centering {\tiny 
  \caption{OLS for predicting within-country regional disparity in online literacy estimates. Models include alternative predictors and no interaction term. Values in parentheses are the lower and upper bounds of the 95\% confidence intervals of the estimated effects.} 
  \label{tab:model-region-4v} 
\begin{tabular}{@{\extracolsep{-10pt}}lcccccc} 
\\[-1.8ex]\hline 
\hline \\[-1.8ex] 
\\[-1.8ex] & \multicolumn{6}{c}{DV: within-country regional disparity} \\ 
 & (a) & (b) & (c) & (d) & (e) & (f) \\ 
\hline \\[-1.8ex] 
 unequal edu & 0.33$^{**}$ & 0.15 & 0.18 & 0.26 & 0.13 & 0.15 \\ 
  & (0.01, 0.64) & ($-$0.17, 0.46) & ($-$0.12, 0.47) & ($-$0.10, 0.61) & ($-$0.22, 0.48) & ($-$0.19, 0.50) \\ 
  & & & & & & \\ 
 Gini & $-$0.24$^{**}$ & $-$0.28$^{***}$ & $-$0.28$^{***}$ & $-$0.37$^{***}$ & $-$0.34$^{**}$ & $-$0.33$^{**}$ \\ 
  & ($-$0.44, $-$0.05) & ($-$0.49, $-$0.08) & ($-$0.48, $-$0.08) & ($-$0.63, $-$0.10) & ($-$0.61, $-$0.06) & ($-$0.60, $-$0.06) \\ 
  & & & & & & \\ 
 \% Internet (all) & 0.35$^{**}$ &  &  & 0.40$^{**}$ &  &  \\ 
  & (0.04, 0.67) &  &  & (0.02, 0.77) &  &  \\ 
  & & & & & & \\ 
 education (all) & $-$0.31$^{**}$ & $-$0.20 & $-$0.20 & $-$0.34$^{**}$ & $-$0.24 & $-$0.27$^{*}$ \\ 
  & ($-$0.57, $-$0.05) & ($-$0.49, 0.08) & ($-$0.46, 0.05) & ($-$0.61, $-$0.06) & ($-$0.54, 0.07) & ($-$0.55, $-$0.003) \\ 
  & & & & & & \\ 
 income &  & $-$0.01 &  &  & $-$0.06 &  \\ 
  &  & ($-$0.32, 0.31) &  &  & ($-$0.45, 0.33) &  \\ 
  & & & & & & \\ 
 civic (all) &  &  & 0.08 &  &  & 0.17 \\ 
  &  &  & ($-$0.15, 0.32) &  &  & ($-$0.11, 0.45) \\ 
  & & & & & & \\ 
 Central/Southern/Eastern Asia &  &  &  & $-$0.23 & $-$0.03 & 0.13 \\ 
  &  &  &  & ($-$1.14, 0.67) & ($-$0.96, 0.89) & ($-$0.83, 1.08) \\ 
  & & & & & & \\ 
 Europe/Oceania/Northern America &  &  &  & $-$0.46 & 0.11 & 0.02 \\ 
  &  &  &  & ($-$1.44, 0.52) & ($-$0.94, 1.17) & ($-$0.86, 0.90) \\ 
  & & & & & & \\ 
 Latin America \& the Caribbean &  &  &  & 0.19 & 0.47 & 0.51 \\ 
  &  &  &  & ($-$0.43, 0.82) & ($-$0.16, 1.10) & ($-$0.09, 1.12) \\ 
  & & & & & & \\ 
 Northern Africa \& Western Asia &  &  &  & $-$0.17 & 0.22 & 0.39 \\ 
  &  &  &  & ($-$0.99, 0.64) & ($-$0.60, 1.05) & ($-$0.45, 1.22) \\ 
  & & & & & & \\ 
 Constant & 0.002 & 0.001 & $-$0.02 & 0.10 & $-$0.17 & $-$0.23 \\ 
  & ($-$0.19, 0.19) & ($-$0.20, 0.20) & ($-$0.22, 0.19) & ($-$0.40, 0.59) & ($-$0.68, 0.34) & ($-$0.70, 0.25) \\ 
  & & & & & & \\ 
\hline \\[-1.8ex] 
Out-of-sample RMSE & 0.91 & 0.92 & 0.92 & 0.95 & 0.96 & 0.95 \\ 
Out-of-sample R$^{2}$ & 0.12 & 0.09 & 0.1 & 0.08 & 0.05 & 0.05 \\ 
Observations & 79 & 79 & 79 & 79 & 79 & 79 \\ 
R$^{2}$ & 0.23 & 0.18 & 0.18 & 0.25 & 0.21 & 0.22 \\ 
Adjusted R$^{2}$ & 0.18 & 0.13 & 0.14 & 0.17 & 0.12 & 0.13 \\ 
AIC & 207.48 & 212.39 & 211.87 & 212.54 & 217.18 & 215.68 \\ 
BIC & 221.70 & 226.61 & 226.09 & 236.23 & 240.87 & 239.37 \\ 
Residual Std. Error & 0.86 (df = 74) & 0.89 (df = 74) & 0.89 (df = 74) & 0.87 (df = 70) & 0.90 (df = 70) & 0.89 (df = 70) \\ 
F Statistic & 5.38$^{***}$ (df = 4; 74) & 3.94$^{***}$ (df = 4; 74) & 4.09$^{***}$ (df = 4; 74) & 2.97$^{***}$ (df = 8; 70) & 2.30$^{**}$ (df = 8; 70) & 2.52$^{**}$ (df = 8; 70) \\ 
\hline 
\hline \\[-1.8ex] 
\textit{Note:}  & \multicolumn{6}{r}{$^{*}$p$<$0.1; $^{**}$p$<$0.05; $^{***}$p$<$0.01} \\ 
\end{tabular} }
\end{table*} 

% \input{stabs/tab_covid_regr}
\clearpage



%%%%%%%%%%%%%%%%%%%%%%%%%%%%%%%%%%%%%%%%%%%%%%
%%                                          %%
%% Backmatter begins here                   %%
%%                                          %%
%%%%%%%%%%%%%%%%%%%%%%%%%%%%%%%%%%%%%%%%%%%%%%

\begin{backmatter}

%%%%%%%%%%%%%%%%%%%%%%%%%%%%%%%%%%%%%%%%%%%%%%%%%%%%%%%%%%%%%
%%                  The Bibliography                       %%
%%                                                         %%
%%  Bmc_mathpys.bst  will be used to                       %%
%%  create a .BBL file for submission.                     %%
%%  After submission of the .TEX file,                     %%
%%  you will be prompted to submit your .BBL file.         %%
%%                                                         %%
%%                                                         %%
%%  Note that the displayed Bibliography will not          %%
%%  necessarily be rendered by Latex exactly as specified  %%
%%  in the online Instructions for Authors.                %%
%%                                                         %%
%%%%%%%%%%%%%%%%%%%%%%%%%%%%%%%%%%%%%%%%%%%%%%%%%%%%%%%%%%%%%

% if your bibliography is in bibtex format, use those commands:
% \bibliographystyle{bmc-mathphys} % Style BST file (bmc-mathphys, vancouver, spbasic).
% \bibliography{reference.bib,reference_supp.bib}
% for author-year bibliography (bmc-mathphys or spbasic)
% a) write to bib file (bmc-mathphys only)
% @settings{label, options="nameyear"}
% b) uncomment next line
%\nocite{label}

% or include bibliography directly:
% \begin{thebibliography}
% \bibitem{b1}
% \end{thebibliography}


\end{backmatter}
\end{document}
