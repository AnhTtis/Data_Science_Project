\setcounter{table}{0}
\renewcommand{\thetable}{S\arabic{table}}%
\setcounter{figure}{0}
\renewcommand{\thefigure}{S\arabic{figure}}%
\setcounter{section}{0}
\renewcommand{\thesection}{S\arabic{section}}%

\addcontentsline{toc}{section}{Supplementary Text}
\section{Supplementary Text}\label{sec:supp_text}

\addcontentsline{toc}{subsection}{Language literacy and visual information consumption}
\subsection{Language literacy and visual information consumption}\label{sec:vis}

We examine the relationship between populations' language literacy levels and their interest in different types of content. Because our assessment concerns the ability to process textual content, we assume there exists a negative relationship between a population's literacy estimate and their attention toward non-textual (e.g., visual) content. Fig.~\ref{fig:lit_visual} shows the correlations between countries' \olles ($x$-axes) and the relative time spent on visual content by the countries' Facebook population ($y$-axes).  As expected, populations with a lower level of language literacy tend to spend relatively more time on visual content. The global correlation is $-0.38$ (Spearman's rank correlation; $p<0.001$), with correlations over different areas ranging from $-0.29$ (Latin America \& the Caribbean; $p<0.062 $) to $-0.62$ (Europe/Oceania/Northern America; $p<0.001$)\footnote{As two of the seven geographical groups only have few countries, we merge the seven groups into five (based on proximity) to provide an adequate statistical description.}.

\addcontentsline{toc}{subsection}{Elbow range detection in popularity curves}
\subsection{Elbow range detection in popularity curves}\label{sec:elbow}
\Bigword are determined based on ``elbow'' range on the word popularity curve for each language, where the relative word frequencies begin to have a systematic decline. 
\yrepj{Fig.~\ref{fig:word_curve} shows the popularity of words in decreasing order, i.e., from the most to the least popular word, as popularity curves. It can be seen that a systematic decline in the word popularity appears at the point of maximum curvature in a popularity curve. In other words, the interest region associated with \bigword corresponds to the ``elbow'' (or ``knee'') point on the smoothed word popularity curve. Mathematically, the curvature is a mathematical measure of how much a function differs from a straight line \cite{satopaa2011finding,antunes2018knee}. Estimating the knee/elbow point for a continuous function is straightforward since the curvature is well-defined for continuous functions; however, it is a challenging task for discrete data. It is also an inherently heuristic process \cite{antunes2018knee}. To reliably detect the elbow range, we leverage the ``Kneedle'' detection \cite{satopaa2011finding}, an efficient algorithm that can efficiently detect knee points in discrete data, and the standard maximum curvature approach on a smoothed function learned from the discrete points. First, we employ generalized additive models with cross-validation to learn a smooth function for each of the popularity curves \cite{hastie1990generalized}. As shown in Fig.~1 E-G, we define an elbow range as an area between two points $k_0$ and $k_1$ (highlighted in red) that best describe the systematic decline in the curve. The two points were determined by combining two heuristic methods: (i) the standard maximum curvature points that can be calculated from any continuous function, and (ii) the approximate knee points (Kneedle detection method) based on the notion that knee points differ most from the straight line connecting the curve's two endpoints \cite{satopaa2011finding}. Note that while the two notions may be considered to be conceptually similar, the approximate knee points (the second notion) are not necessarily the maximum curvature points especially when the curves are skewed. In a right-skewed curve (as in the case of a word popularity curve), the approximate knee points tend to fall into the right of the maximum curvature points. Thus, we detect an elbow by two points $k_0$ and $k_1$ through maximum curvature measurement and approximate knee point detection method respectively. Unlike other knee/elbow detention methods that are sensitive to noises and rescaling, we found this hybrid approach is more robust to rescaling and small fluctuations in our data.} 
Words with ranks falling into the elbow range are considered to be the ``\bigword'' in the language. Fig.~1 A-C highlights the elbow range detected from the word popularity curve for each of the three most used languages, and Fig.~\ref{fig:word_curve} shows the detected elbow ranges for all 12 languages. 

\addcontentsline{toc}{subsection}{Procedure for estimating online language literacy}
\subsection{Procedure for estimating online language literacy}\label{sec:procedure}

For a given population with a given language, the procedure to measure the collective language literacy involves the following steps:

\begin{enumerate}
\item[(i)] Processing of user-generated texts: We use public posts written in any of the chosen languages created by Facebook users who are at least 18 years old and active during a 30-day period between April 20 and May 20, 2020. We exclude posts that did not contain any text or text that was shorter than 2 characters or longer than 1000 characters, as well as posts that contained URLs as these are more likely to be copied and pasted from other sources rather then composed by users. 

\item[(ii)] Aggregate statistics per user: After tokenizing the public posts, for each user, we quantify the number of unique words (unigrams) that falls in the range of \bigword. We then obtain a relative \bigw count $w_u$ that is normalized by the active level of post creation per user, i.e., $w_u$ is given by (the total number of \bigword observed from $u$'s public posts) / (the total number of $u$'s public posts). We count each unigram once per user, regardless of the frequency used, to avoid overestimating the use of particular words or the inflation from copy-pasted content.

\item[(iii)] Aggregate statistics per population: For each geographically bounded community (e.g. a county or a region) with at least 1000 active users observed in the study period, the population-level estimate is calculated as $\bar{w}$, the average of $w_u$'s over all active users $u$'s in the geographically bounded community. The gender- or region-disaggregate population-level estimates also require a minimum of 1000 active users observed in the study period in any of the disaggregate groups. The threshold of 1000 unique users from any group was chosen to ensure user privacy and the statistical power of our method. We also exclude users who produced a high volume of posts (above 75 percentiles) to avoid a small number of highly productive users dominating the measurement.
\end{enumerate}

Throughout the procedure, none of the personal identifiable information or any personal or private content was used. Only the aggregate statistics $w_u$ and $\bar{w}$ were generated from the process. 

\paragraph{How to retrieve pre-computed \bigword}
\yrrr{\bigword are determined based on the word popularity curves derived from the Facebook users' use of the up 200,000 most frequent words in each language. These pre-computed \bigword can be retrieved through the following steps:}
\begin{itemize}
\item[(i)] Install the fastText\footnote{\url{https://fasttext.cc/docs/en/python-module.html\#installation}}
\item[(ii)] Run \texttt{download\_model.py \$lang} to get the dictionary of a specific language, where \texttt{\$lang} is the language indicator (e.g., {\texttt en} for English, {\texttt es} for Spanish). This script will download the dictionary in a binary file (let \texttt{\$filename} be the filename of the downloaded file).
\item[(ii)] Run \texttt{fasttext dump \$filename dict > \$ofilename} to convert the binary dictionary file to a text file (let \texttt{\$ofilename} be the filename of the output text file). This file contains up to 200,000 lines where each line is a word and its frequency. The frequencies can be used to rank the words from the most to the least frequent. 
\item[(iv)] Extract the \bigword based on the knee points listed in Table~\ref{tab:knees}. For example, the \bigword in English correspond to the words in the fastText dictionary that are ranked between 5,000 to 9,000 in the decreasing order of word frequency.
\end{itemize}


\addcontentsline{toc}{subsection}{D.}
\subsection{Case study: India as a multilingual country}\label{sec:india}


\yrv{We choose India as a case study for countries using multiple languages to study the effect of choosing the most used language as a single representative language for literacy estimate. While India uses Hindi and English as official languages nationwide, it has no single national language. It has over 30 states/union territories, each of which has its own official language(s). There are 22 official languages recognized by country officials, in addition to some other languages recognized as additional official languages at the regional level. In this analysis, we include the additional five most used languages in India according to the India census reported in 2011 \cite{Censusof39:online}: Hindi (43.6\%), Bengali (8.3\%), Marathi (7.1\%), Telugu (6.9\%), and Tamil (5.9\%). Languages with less than 5\% speakers among the Indian national population are not considered. On Facebook, the most used language in Indian users' public posts is English (en), which has about three times the users posting in Hindi (hi), and about 20, 44, 90, and 181 times those posting in Bengali (bn), Marathi (mr), Telugu (te), and Tamil (ta), respectively. 
Across regions, the non-English language using populations on Facebook are sparse. Only 14 (48.2\%) regions have more than 10\% of the number of English posters posting in Hindi, and only 4 (13.8\%) and 1 (3.4\%) regions have more than 1\% of the number of English posters posting in Bengali and Marathi.} 

\yrv{We then create a language literacy estimation for each of the six languages, using the same approach but include the additional languages (Hindi, Bengali, Marathi, Telugu, and Tamil) from fastText unigram data \cite{grave2018learning}. For validation, we gather the regional literacy survey reported in the India census 2011 \cite{Censusof39:online}, which is the most recent data available. 
Fig.~\ref{fig:india} shows the comparison of our language estimation with the officially reported literacy data. 
\yrrr{We first estimate the language literacy for every language. Fig.~\ref{fig:india} A and C-E show the estimation based on posts in a single language. Note that, while there are regional differences, the use of Hindi, Bengali, and Marathi is extremely sparse in most regions. Therefore, the non-English language estimates alone cannot be directly used to create a regional measurement. Due to the sparse use of non-English language on the platform, the correlation between the non-English language estimates and the reported literacy is insignificant. We additionally create a multi-language estimation weighted by the popularity of each language within a region, as shown in Fig~\ref{fig:india} B.}
%We first estimate the language literacy for every language, and additionally create a multi-language estimation weighted by the popularity of each language within a region. Fig.~\ref{fig:india} A shows the estimation based on posts in English only, and Fig.~\ref{fig:india} B shows the estimation based on multiple languages. 
We observe that both English-only and multi-language estimates (without any additional calibration) have significant correlations with the reported literacy data (Spearman's rank correlations with positive 95\% CIs and $p<0.005$). However, literacy estimation based on multiple languages has neglectable improvement over English-based estimation in terms of the correlations -- from 0.51 to 0.52. This is likely due to the low rate of users posting in non-English languages in many regions. \yrrr{Here, the comparison relies on the officially reported literacy data, which has a limitation: they do not capture the change in regional literacy levels since 2011, and likely do not properly reflect the diverse language skills used by minority populations.} 
%Another potential reason is that officially reported literacy data do not capture the change of regional literacy levels since 2011, or may not properly reflect the diverse language skills used by minority populations.
This case study illustrates the challenge of obtaining gold-standard literacy measures for multilingual countries. While this does not prevent us to create a multi-language literacy measurement per country, for validation purposes, we simply choose a single representative language for multilingual countries. Thus the correlation should be interpreted with caution -- the officially reported data that guide this choice often has a bias against language minorities.}

\addcontentsline{toc}{subsection}{E.}
\subsection{Robust check: regional disparity and language dominance}\label{sec:domi}

\yrepj{Our \olle is created based on a country's representative language, i.e., the language used by the most Facebook users in the country. One potential risk of relying on a single representative language is that the regional disparity measure in a multilingual country may simply capture the distribution of languages, rather than the diversity of language skills. To test this, we examine the relationship between the user percentage of the representative language in a multilingual country and the country's regional disparity measure. Among the 167 countries studied, there are 20 multilingual countries, but only 13 meet the criteria to have a regional disparity measure. Recall in Section~\ref{sec:procedure} that each geographically bounded community (i.e., in this case, a region within a country) with at least 1000 active users observed in the study period. For these 13 multilingual countries, we plot the countries' percentage of Facebook users using the representative language on the $y$-axis and on the $x$-axis, either (A) \olle, or (B) regional disparity as shown in Fig.~\ref{fig:domi}. If there is a systematic bias, e.g., countries with low representative language user percentages tend to have high a regional disparity measure, we would see a trend in such a plot. However, we do not observe a clear systematic bias. While our sample size is limited, this analysis is helpful for checking whether there is a potential bias in the small sample of multilingual countries.} 


\addcontentsline{toc}{section}{Supplementary Figures}
% \section*{Supplementary Figures}\label{sec:supp_figs}

\begin{figure*}
    \centering
    \includegraphics[width=.95\linewidth]{figS01_word_curves.pdf}
    \caption{Determining the ``\bigw'' range using curvature and knee points detection. In half of the studied languages, the ranks of the \Bigword range between 5000 and 9000. Others (zh, it, ru, tr, nl, and ms) have wider or narrower ranges.}
    \label{fig:word_curve}
\end{figure*}

\begin{figure*}
    \centering
    \includegraphics[width=.95\linewidth]{figS02_countries_covered_fbUsers.pdf}
    \caption{Countries covered in our estimation. There are 167 countries in 12 different languages, including 147 countries with a single or dominant language and 20 multilingual countries. For a multilingual country (having multiple official languages), we use the most used language of the country to estimate its language literacy.
    (A) The number of countries in each language. (B) The total population in each language. (C) The Facebook user count in each language, according to publicly available information about Facebook penetration statistics in 2022 \cite{Facebook15:online}. The populations from the 20 multilingual countries are excluded in (B) and (C) because we do not have the sub-population estimates of different languages within these multilingual countries. 
    }
    \label{fig:countries_covered}
\end{figure*}

\begin{figure*}
    \centering
    % \includegraphics[width=\linewidth]{sfigs/corr-lit-edu.pdf}
    \includegraphics[width=.95\linewidth]{figS03_eval_olle_3x2_CI.pdf}
    \caption{The literacy estimate ($x$-axis) obtained from Models (a,b,c) listed in Table \ref{tab:model-lit-main}, compared with the reported literacy rate and the education in terms of schooling years ($y$-axis). The reported literacy rate was transformed for normality. 
(A,B) Literacy estimate adjusted by Model (a).
(C,D) Literacy estimate adjusted by Model (b).
(E,F) Literacy estimate adjusted by Model (c).}
    \label{fig:corr_lit_edu}
\end{figure*}

\begin{figure*}
    \centering
    % \includegraphics[width=\linewidth]{sfigs/corr-lit-visual-v2}
    \includegraphics[width=.95\linewidth]{figS04_corr_lit_vs_visual_CI}
    \caption{Relationship between countries' \olles ($x$-axes) and the relative visual time-spent ($y$-axes), where the relative visual time-spent is given as the proportion of time spent on photos and videos relative to the time on news feeds, photos and videos combined. Correlations are reported based on Spearman's rank correlation. }
    \label{fig:lit_visual}
\end{figure*}

\begin{figure*}
    \centering
    \includegraphics[width=\linewidth]{figS05_map-gender}
    \caption{Gender differences in online language literacy across the world. (A) Summary of the standardized female-male differences across the seven geographical groups. Dashed line marks the global average (0.345, population-weighted), and a diamond indicates the population-weighted mean of the group. (B) Map of female-male differences available in our dataset.}
    \label{fig:gender_map}
\end{figure*}

\begin{figure*}
    \centering
    \includegraphics[width=\linewidth]{figS06_map-regional}
    \caption{Within-country regional disparity in online language literacy across the world. (A) Summary of the within-country regional disparity across the seven geographical groups. Dashed line marks the global average (0.032, population-weighted), and a diamond indicates the population-weighted mean of the group. (B) Map of country-level regional disparity available in our dataset. (C) Countries with a higher or lower level of regional disparity from each geographical group.}
    \label{fig:region_map}
\end{figure*}

% \input{sfigs/india/fig_india}

\begin{figure*}
    \centering
    % \includegraphics[width=.9\linewidth]{sfigs/india/IN_region_lit_6lang_eval.png}
    \includegraphics[width=.9\linewidth]{figS07_IN_region_lit_corr}
    \includegraphics[width=.9\linewidth]{figS07b_IN_region_lit_corr}
    \caption{A multilingual country case study with India's user population. We include the six most used languages in India (English, Hindi, Bengali, Marathi, Telugu, and Tamil) to compare the literacy estimation based on the single representative language (English) with the estimation based on multiple languages. The results suggest that literacy estimation based on multiple languages has neglectable improvement over English-based estimation in terms of Spearman's rank correlations. The y-axis indicates the officially reported literacy level, and the x-axes indicate (A) the language literacy estimated using the regions' public posts in English, (B) the estimation using multiple languages combined, and (C-E) the estimation using posts in Hindi, Bengali, and Marathi only, respectively.}
    \label{fig:india}
\end{figure*}

\begin{figure*}
    \centering
    \includegraphics[width=.9\linewidth]{figS08_multi_lang_dominant_perc}
    \caption{The relationship between the user percentage of the dominant language ($y$-axis) and (A) \olle, or (B) regional disparity, in 13 multilingual countries studied. We do not observe a clear systematic bias. This serves as a robust check to see whether there is a potential bias in the set of multilingual countries. }
    \label{fig:domi}
\end{figure*}
\clearpage
\addcontentsline{toc}{section}{Supplementary Tables}
% \section*{Supplementary Tables}\label{sec:supp_tabs}


\begin{table}[]
    \centering
    \caption{Knee points detected based on the Facebook popularity curves shown in Fig.~\ref{fig:word_curve}. The two knee points, measured in 1,000 words, determine the \bigword in each language. For example, the \bigword in English correspond to the words in the fastText `{\texttt en}' dictionary that are ranked between 5,000 to 9,000 in the decreasing order of word frequency.}
    \label{tab:knees}
    \begin{tabular}{c|c|c|c|c |c|c|c|c |c|c|c|c}
    \toprule
        Language & en & es & fr & ar & de & zh & pt & it & ru & tr & ml & ms\\
        \midrule
        $k_0$ & 5 & 5 & 5 & 5 & 5 & 5 & 5 & 5 & 5 & 5 & 5 & 6\\
        $k_1$ & 9 & 9 & 9 & 9 & 9 & 16 & 9 & 11 & 8 & 11 & 10 & 21\\
    \bottomrule
    \end{tabular}
\end{table}
% Table created by stargazer v.5.2.2 by Marek Hlavac, Harvard University. E-mail: hlavac at fas.harvard.edu
% Date and time: Mon, Jul 13, 2020 - 12:30:50 PM
\begin{table*}[!htbp] \centering {\tiny 
  \caption{OLS for predicting the reported literacy rate with online literacy estimates. Model (a) is the fixed-effect model accounting for language-specific bias. The calibrated online literacy estimates (\olles) are produced using model (a). The observations include all countries having online literacy estimates and corresponding predictors. Countries without sufficient Internet penetration ($<25\%$) are excluded to obtain reliable calibrated models. For comparison, models (b,c) include additional predictors, the Internet penetration and income. All variables were transformed to better fit a normal distribution.} 
  \label{tab:model-lit-main} 
\begin{tabular}{@{\extracolsep{-10pt}}lccc} 
\\[-1.8ex]\hline 
\hline \\[-1.8ex] 
\\[-1.8ex] & \multicolumn{3}{c}{DV: reported literacy rate} \\ 
 & (a) & (b) & (c) \\ 
\hline \\[-1.8ex] 
 est. literacy & 0.80$^{***}$ (0.61, 0.99) & 0.52$^{***}$ (0.27, 0.76) & 0.53$^{***}$ (0.27, 0.79) \\ 
  \% Internet &  & 0.32$^{***}$ (0.13, 0.50) & 0.49$^{***}$ (0.19, 0.80) \\ 
  income &  &  & $-$0.19 ($-$0.48, 0.11) \\ 
  language [de] & 1.43$^{***}$ (0.72, 2.14) & 1.22$^{***}$ (0.53, 1.90) & 1.29$^{***}$ (0.59, 1.99) \\ 
  language [en] & 0.94$^{***}$ (0.55, 1.33) & 0.98$^{***}$ (0.61, 1.35) & 0.97$^{***}$ (0.58, 1.37) \\ 
  language [es] & 1.31$^{***}$ (0.82, 1.80) & 1.15$^{***}$ (0.68, 1.62) & 1.25$^{***}$ (0.76, 1.74) \\ 
  language [fr] & 1.43$^{***}$ (0.79, 2.06) & 1.13$^{***}$ (0.51, 1.75) & 1.30$^{***}$ (0.57, 2.03) \\ 
  language [it] & 1.24$^{**}$ (0.26, 2.22) & 1.75$^{***}$ (0.78, 2.73) & 1.57$^{**}$ (0.25, 2.90) \\ 
  language [ms] & $-$0.75 ($-$1.75, 0.25) & $-$0.55 ($-$1.50, 0.40) & $-$0.52 ($-$1.48, 0.44) \\ 
  language [nl] & 0.91$^{**}$ (0.21, 1.62) & 0.78$^{**}$ (0.10, 1.45) & 0.97$^{**}$ (0.21, 1.73) \\ 
  language [pt] & 0.62$^{*}$ ($-$0.06, 1.29) & 0.61$^{*}$ ($-$0.03, 1.26) & 0.66$^{*}$ (0.01, 1.31) \\ 
  language [ru] & 3.41$^{***}$ (2.64, 4.17) & 3.14$^{***}$ (2.41, 3.88) & 3.10$^{***}$ (2.34, 3.86) \\ 
  language [tr] & 0.86$^{*}$ ($-$0.09, 1.81) & 0.96$^{**}$ (0.05, 1.86) & 1.04$^{**}$ (0.12, 1.95) \\ 
  language [zh] & $-$0.62 ($-$1.64, 0.40) & $-$0.35 ($-$1.32, 0.63) &  \\ 
  Constant & $-$0.94$^{***}$ ($-$1.27, $-$0.61) & $-$0.90$^{***}$ ($-$1.21, $-$0.59) & $-$0.94$^{***}$ ($-$1.26, $-$0.61) \\ 
 \hline \\[-1.8ex] 
OOS correlation $\rho$ & 0.78 & 0.8 & 0.79 \\ 
OOS RMSE & 0.7 & 0.66 & 0.68 \\ 
OOS R$^{2}$ & 0.51 & 0.57 & 0.55 \\ 
Observations & 98 & 98 & 86 \\ 
R$^{2}$ & 0.64 & 0.68 & 0.69 \\ 
Adjusted R$^{2}$ & 0.59 & 0.63 & 0.64 \\ 
AIC & 205.52 & 195.32 & 172.75 \\ 
BIC & 241.70 & 234.10 & 209.57 \\ 
Residual Std. Error & 0.64 (df = 85) & 0.61 (df = 84) & 0.61 (df = 72) \\ 
F Statistic & 12.49$^{***}$ (df = 12; 85) & 13.76$^{***}$ (df = 13; 84) & 12.52$^{***}$ (df = 13; 72) \\ 
\hline 
\hline \\[-1.8ex] 
\textit{Note:}  & \multicolumn{3}{r}{$^{*}$p$<$0.1; $^{**}$p$<$0.05; $^{***}$p$<$0.01} \\ 
 & \multicolumn{3}{r}{The out-of-sample (OOS) Spearman correlation $\rho$, RMSE, and R$^{2}$ are obtained using leave-one-out cross-validation.} \\ 
\end{tabular} }
\end{table*} 
% Table created by stargazer v.5.2.2 by Marek Hlavac, Harvard University. E-mail: hlavac at fas.harvard.edu
% Date and time: Wed, Jul 15, 2020 - 02:56:08 PM
\begin{table*}[!htbp] \centering {\tiny 
  \caption{OLS for predicting the reported literacy rate without online literacy estimates. All variables were transformed to better fit a normal distribution.} 
  \label{tab:model-no-lit} 
\begin{tabular}{@{\extracolsep{-10pt}}lccc} 
\\[-1.8ex]\hline 
\hline \\[-1.8ex] 
\\[-1.8ex] & \multicolumn{3}{c}{DV: reported literacy rate} \\ 
 & (a) & (b) & (c) \\ 
\hline \\[-1.8ex] 
 \% Internet & 0.54$^{***}$ (0.37, 0.71) & 0.58$^{***}$ (0.43, 0.73) & 0.74$^{***}$ (0.43, 1.05) \\ 
  income &  &  & $-$0.15 ($-$0.47, 0.18) \\ 
  language [de] &  & 1.14$^{***}$ (0.40, 1.89) & 1.22$^{***}$ (0.45, 1.98) \\ 
  language [en] &  & 0.96$^{***}$ (0.55, 1.36) & 0.95$^{***}$ (0.52, 1.38) \\ 
  language [es] &  & 0.69$^{***}$ (0.23, 1.14) & 0.79$^{***}$ (0.31, 1.27) \\ 
  language [fr] &  & 0.64$^{**}$ (0.02, 1.27) & 0.75$^{*}$ (0.004, 1.50) \\ 
  language [it] &  & 2.55$^{***}$ (1.57, 3.53) & 2.34$^{***}$ (0.95, 3.74) \\ 
  language [ms] &  & 0.08 ($-$0.90, 1.07) & 0.14 ($-$0.86, 1.14) \\ 
  language [nl] &  & 0.70$^{*}$ ($-$0.04, 1.43) & 0.95$^{**}$ (0.11, 1.78) \\ 
  language [pt] &  & 0.33 ($-$0.35, 1.01) & 0.40 ($-$0.30, 1.10) \\ 
  language [ru] &  & 2.51$^{***}$ (1.78, 3.25) & 2.49$^{***}$ (1.72, 3.26) \\ 
  language [tr] &  & 1.17$^{**}$ (0.20, 2.15) & 1.27$^{**}$ (0.27, 2.26) \\ 
  language [zh] &  & 0.42 ($-$0.56, 1.41) &  \\ 
  Constant & 0.00 ($-$0.17, 0.17) & $-$0.78$^{***}$ ($-$1.11, $-$0.44) & $-$0.83$^{***}$ ($-$1.18, $-$0.48) \\ 
 \hline \\[-1.8ex] 
OOS correlation $\rho$ & 0.52 & 0.73 & 0.72 \\ 
OOS RMSE & 0.86 & 0.7 & 0.73 \\ 
OOS R$^{2}$ & 0.26 & 0.51 & 0.48 \\ 
Observations & 98 & 98 & 86 \\ 
R$^{2}$ & 0.29 & 0.62 & 0.62 \\ 
Adjusted R$^{2}$ & 0.28 & 0.56 & 0.56 \\ 
AIC & 249.67 & 211.51 & 188.23 \\ 
BIC & 257.42 & 247.70 & 222.59 \\ 
Residual Std. Error & 0.85 (df = 96) & 0.66 (df = 85) & 0.67 (df = 73) \\ 
F Statistic & 39.04$^{***}$ (df = 1; 96) & 11.32$^{***}$ (df = 12; 85) & 10.10$^{***}$ (df = 12; 73) \\ 
\hline 
\hline \\[-1.8ex] 
\textit{Note:}  & \multicolumn{3}{r}{$^{*}$p$<$0.1; $^{**}$p$<$0.05; $^{***}$p$<$0.01} \\ 
 & \multicolumn{3}{r}{The out-of-sample (OOS) Spearman correlation $\rho$, RMSE, and R$^{2}$ are obtained using leave-one-out cross-validation.} \\ 
\end{tabular} }
\end{table*} 
\clearpage
% Table created by stargazer v.5.2.2 by Marek Hlavac, Harvard University. E-mail: hlavac at fas.harvard.edu
% Date and time: Wed, Jul 01, 2020 - 06:15:42 AM
% \begin{table}[!htbp] \centering
% \begin{tabular}{@{\extracolsep{-10pt}} lcrrrrr} 
{
\begingroup
\onecolumn
\tiny \centering
\begin{longtable}[H]{@{\extracolsep{-8pt}}lcrrrrrc}
  \caption{Online language literacy estimates for all countries included in this study. Measures include: average relative \bigword, \olle ($N=167$), female-male gender gap ($N=160$) and regional disparity ($N=119$).} 
  \label{tab:all_country} 
\\
\hline \\[-1.8ex] 
% \\ \toprule
country & code & big word ($\bar{w}$) & \olle & female-male gap & regional disparity & no. regions & rep. language \\ 
\hline \\[-1.8ex] 
Algeria & DZA & 0.584 & 0.309 & -0.584 & 0.022 & 48 & ar \\ 
American Samoa & ASM & 0.811 & 0.593 & 1.744 & -- & -- & en \\ 
Angola & AGO & 0.634 & 0.419 & -0.281 & 0.008 & 17 & pt \\ 
Anguilla & AIA & 0.521 & 0.416 & 0.372 & -- & -- & en \\ 
Antigua \& Barbuda & ATG & 0.715 & 0.519 & 1.422 & -- & -- & en \\ 
Argentina & ARG & 0.584 & 0.505 & 0.166 & 0.03 & 24 & es \\ 
Armenia & ARM & 0.308 & 0.624 & 1.684 & 0.047 & 10 & ru \\ 
Aruba & ABW & 0.785 & 0.58 & 0.476 & -- & -- & nl \\ 
Australia & AUS & 1.004 & 0.688 & 0.456 & 0.019 & 8 & en \\ 
Austria & AUT & 0.786 & 0.667 & 0.481 & 0.03 & 9 & de \\ 
Bahamas & BHS & 0.687 & 0.501 & 1.056 & 0.026 & 2 & en \\ 
Bahrain & BHR & 0.784 & 0.429 & -2.293 & 0.084 & 2 & ar \\ 
Barbados & BRB & 0.743 & 0.544 & 1.325 & -- & -- & en \\ 
Belarus & BLR & 0.709 & 0.868 & 0.746 & 0.067 & 7 & ru \\ 
Belgium & BEL & 0.876 & 0.621 & -0.411 & 0.117 & 3 & nl \\ 
Belize & BLZ & 0.729 & 0.528 & 1.011 & 0.021 & 4 & en \\ 
Benin & BEN & 0.548 & 0.498 & -1.861 & 0.042 & 12 & fr \\ 
Bermuda & BMU & 0.955 & 0.659 & 0.551 & -- & -- & en \\ 
Bolivia & BOL & 0.437 & 0.439 & -0.262 & 0.033 & 9 & es \\ 
Botswana & BWA & 0.535 & 0.422 & 0.144 & 0.004 & 8 & en \\ 
Brazil & BRA & 0.557 & 0.376 & 0.469 & 0.026 & 27 & pt \\ 
British Virgin Islands & VGB & 0.688 & 0.501 & 1.408 & -- & -- & en \\ 
Brunei & BRN & 1.522 & 0.468 & 0.465 & -- & -- & ms \\ 
Burkina Faso & BFA & 0.303 & 0.327 & -1.921 & 0.074 & 6 & fr \\ 
Burundi & BDI & 0.141 & 0.211 & -0.324 & -- & -- & fr \\ 
Cameroon & CMR & 0.638 & 0.477 & 0.016 & 0.027 & 10 & en \\ 
Canada & CAN & 0.965 & 0.664 & 0.695 & 0.065 & 13 & en \\ 
Cape Verde & CPV & 0.682 & 0.44 & 0.379 & 0.029 & 6 & pt \\ 
Caribbean Netherlands & BES & 0.849 & 0.609 & -- & -- & -- & nl \\ 
Cayman Islands & CYM & 0.872 & 0.618 & 0.949 & -- & -- & en \\ 
Central African Republic & CAF & 0.317 & 0.36 & 0 & -- & -- & fr \\ 
Chad & TCD & 0.418 & 0.202 & -3.344 & 0.056 & 2 & ar \\ 
Chile & CHL & 0.585 & 0.512 & 0.37 & 0.015 & 15 & es \\ 
Colombia & COL & 0.436 & 0.434 & 0.328 & 0.039 & 32 & es \\ 
Comoros & COM & 0.238 & 0.027 & -- & -- & -- & ar \\ 
Congo - Brazzaville & COG & 0.316 & 0.36 & -0.791 & 0.001 & 2 & fr \\ 
Congo - Kinshasa & COD & 0.312 & 0.359 & -1.552 & 0.095 & 11 & fr \\ 
Cook Islands & COK & 0.848 & 0.61 & -- & -- & -- & en \\ 
Costa Rica & CRI & 0.486 & 0.462 & -0.039 & 0.043 & 7 & es \\ 
Côte d’Ivoire & CIV & 0.299 & 0.313 & -1.604 & 0.032 & 14 & fr \\ 
Cuba & CUB & 0.475 & 0.459 & -0.078 & 0.018 & 16 & es \\ 
Curaçao & CUW & 0.795 & 0.587 & 0.022 & -- & -- & nl \\ 
Cyprus & CYP & 0.846 & 0.601 & -0.159 & 0.009 & 3 & tr \\ 
Djibouti & DJI & 0.468 & 0.258 & -2.006 & -- & -- & ar \\ 
Dominica & DMA & 0.621 & 0.467 & 0.227 & -- & -- & en \\ 
Dominican Republic & DOM & 0.386 & 0.352 & -0.189 & 0.029 & 31 & es \\ 
Ecuador & ECU & 0.451 & 0.453 & 0.002 & 0.034 & 24 & es \\ 
Egypt & EGY & 0.68 & 0.347 & -0.145 & 0.028 & 27 & ar \\ 
El Salvador & SLV & 0.435 & 0.429 & -0.937 & 0.033 & 14 & es \\ 
Equatorial Guinea & GNQ & 0.536 & 0.474 & 0.119 & 0.007 & 2 & es \\ 
Fiji & FJI & 0.678 & 0.492 & 0.482 & 0.016 & 3 & en \\ 
France & FRA & 0.767 & 0.632 & 1.094 & 0.025 & 22 & fr \\ 
French Guiana & GUF & 0.473 & 0.479 & 0.105 & -- & -- & fr \\ 
French Polynesia & PYF & 0.494 & 0.484 & 0.941 & -- & -- & fr \\ 
Gabon & GAB & 0.311 & 0.359 & -1.445 & 0.059 & 3 & fr \\ 
Gambia & GMB & 0.522 & 0.416 & -0.048 & 0.089 & 2 & en \\ 
Germany & DEU & 0.854 & 0.695 & 0.113 & 0.05 & 16 & de \\ 
Ghana & GHA & 0.576 & 0.448 & -0.697 & 0.014 & 10 & en \\ 
Gibraltar & GIB & 0.945 & 0.651 & 0.774 & -- & -- & en \\ 
Grenada & GRD & 0.704 & 0.507 & 1.918 & 0.051 & 2 & en \\ 
Guadeloupe & GLP & 0.613 & 0.538 & 0.882 & -- & -- & fr \\ 
Guam & GUM & 0.772 & 0.567 & 1.475 & -- & -- & en \\ 
Guatemala & GTM & 0.403 & 0.383 & -0.526 & 0.039 & 22 & es \\ 
Guinea & GIN & 0.348 & 0.366 & 0.085 & 0.076 & 8 & fr \\ 
Guinea-Bissau & GNB & 0.299 & 0.183 & 0.842 & -- & -- & pt \\ 
Guyana & GUY & 0.65 & 0.48 & 0.871 & 0.035 & 4 & en \\ 
Haiti & HTI & 0.28 & 0.265 & -1.976 & 0.047 & 9 & fr \\ 
Honduras & HND & 0.387 & 0.364 & -0.281 & 0.024 & 18 & es \\ 
Hong Kong SAR China & HKG & 2.021 & 0.57 & 0.02 & -- & -- & zh \\ 
India & IND & 0.423 & 0.361 & 0.887 & 0.044 & 34 & en \\ 
Iraq & IRQ & 0.745 & 0.399 & -1.859 & 0.109 & 19 & ar \\ 
Ireland & IRL & 0.944 & 0.65 & 1.235 & 0.02 & 26 & en \\ 
Isle of Man & IMN & 1.086 & 0.693 & 0.335 & -- & -- & en \\ 
Italy & ITA & 1.159 & 0.757 & -0.109 & 0.006 & 20 & it \\ 
Jamaica & JAM & 0.561 & 0.437 & 0.772 & 0.021 & 11 & en \\ 
Jersey & JEY & 0.949 & 0.655 & 1.234 & -- & -- & en \\ 
Jordan & JOR & 0.918 & 0.496 & 0.01 & 0.02 & 12 & ar \\ 
Kenya & KEN & 0.62 & 0.467 & -0.208 & 0.021 & 8 & en \\ 
Kiribati & KIR & 0.557 & 0.433 & 0.751 & -- & -- & en \\ 
Kuwait & KWT & 0.759 & 0.405 & -2.727 & 0.061 & 6 & ar \\ 
Kyrgyzstan & KGZ & 0.299 & 0.599 & 1.485 & 0.108 & 2 & ru \\ 
Lebanon & LBN & 0.734 & 0.386 & -0.771 & 0.052 & 6 & ar \\ 
Lesotho & LSO & 0.485 & 0.41 & -0.03 & 0.088 & 6 & en \\ 
Liberia & LBR & 0.74 & 0.538 & 0.483 & 0.053 & 2 & en \\ 
Libya & LBY & 0.697 & 0.355 & -0.143 & 0.017 & 19 & ar \\ 
Liechtenstein & LIE & 0.779 & 0.656 & -- & -- & -- & de \\ 
Luxembourg & LUX & 0.775 & 0.643 & 1.075 & 0.059 & 2 & fr \\ 
Macau SAR China & MAC & 1.69 & 0.509 & 0.433 & -- & -- & zh \\ 
Madagascar & MDG & 0.252 & 0.253 & 0.078 & 0.012 & 18 & fr \\ 
Malawi & MWI & 0.522 & 0.416 & -0.09 & 0.036 & 5 & en \\ 
Malaysia & MYS & 1.807 & 0.511 & 0.936 & 0.032 & 15 & ms \\ 
Mali & MLI & 0.244 & 0.25 & -0.537 & 0.051 & 6 & fr \\ 
Malta & MLT & 0.718 & 0.525 & 0.733 & 0.029 & 4 & en \\ 
Marshall Islands & MHL & 0.729 & 0.531 & 1.173 & -- & -- & en \\ 
Mauritania & MRT & 0.8 & 0.442 & -2.055 & 0.063 & 2 & ar \\ 
Mauritius & MUS & 0.568 & 0.44 & 0.59 & 0.018 & 9 & en \\ 
Mayotte & MYT & 0.406 & 0.406 & 0.773 & -- & -- & fr \\ 
Mexico & MEX & 0.428 & 0.418 & 0.318 & 0.037 & 32 & es \\ 
Micronesia (Federated States of) & FSM & 0.532 & 0.418 & 0.139 & 0.007 & 2 & en \\ 
Monaco & MCO & 0.883 & 0.7 & -- & -- & -- & fr \\ 
Morocco & MAR & 0.628 & 0.322 & -1.058 & 0.026 & 16 & ar \\ 
Mozambique & MOZ & 0.288 & 0.148 & -1.678 & 0.051 & 10 & pt \\ 
Nauru & NRU & 0.611 & 0.465 & -- & -- & -- & en \\ 
Netherlands & NLD & 0.839 & 0.603 & -0.215 & 0.025 & 12 & nl \\ 
New Caledonia & NCL & 0.469 & 0.478 & 0.032 & 0.039 & 2 & fr \\ 
New Zealand & NZL & 1.001 & 0.679 & 0.686 & 0.016 & 16 & en \\ 
Nicaragua & NIC & 0.42 & 0.406 & -0.553 & 0.031 & 17 & es \\ 
Niger & NER & 0.294 & 0.299 & -2.103 & 0.061 & 4 & fr \\ 
Nigeria & NGA & 0.814 & 0.597 & 0.002 & 0.064 & 37 & en \\ 
Northern Mariana Islands & MNP & 0.708 & 0.51 & 1.28 & -- & -- & en \\ 
Oman & OMN & 0.796 & 0.441 & -2.218 & 0.088 & 3 & ar \\ 
Pakistan & PAK & 0.455 & 0.403 & 0.262 & 0.037 & 8 & en \\ 
Palestinian Territories & PSE & 0.905 & 0.49 & 0.266 & 0.019 & 2 & ar \\ 
Panama & PAN & 0.446 & 0.449 & 0.005 & 0.043 & 8 & es \\ 
Papua New Guinea & PNG & 0.539 & 0.423 & 0.013 & 0.019 & 17 & en \\ 
Paraguay & PRY & 0.429 & 0.424 & -0.883 & 0.035 & 17 & es \\ 
Peru & PER & 0.535 & 0.47 & 0.001 & 0.009 & 25 & es \\ 
Philippines & PHL & 0.551 & 0.426 & 0.735 & 0.019 & 17 & en \\ 
Portugal & PRT & 0.711 & 0.458 & -0.41 & 0.018 & 20 & pt \\ 
Puerto Rico & PRI & 0.632 & 0.528 & 0.375 & 0.022 & 68 & es \\ 
Qatar & QAT & 0.812 & 0.444 & -2.442 & -- & -- & ar \\ 
Réunion & REU & 0.579 & 0.522 & 0.461 & -- & -- & fr \\ 
Rwanda & RWA & 0.521 & 0.416 & -0.037 & 0.015 & 5 & en \\ 
Saint Martin (French part) & MAF & 0.699 & 0.576 & -- & -- & -- & fr \\ 
Samoa & WSM & 0.655 & 0.483 & 0.725 & -- & -- & en \\ 
San Marino & SMR & 1.208 & 0.768 & -0.191 & -- & -- & it \\ 
São Tomé \& Príncipe & STP & 0.399 & 0.265 & 0.168 & -- & -- & pt \\ 
Saudi Arabia & SAU & 0.764 & 0.409 & -1.981 & 0.026 & 13 & ar \\ 
Senegal & SEN & 0.285 & 0.267 & -1.618 & 0.035 & 14 & fr \\ 
Seychelles & SYC & 0.66 & 0.486 & 0.931 & -- & -- & en \\ 
Sierra Leone & SLE & 0.759 & 0.553 & 0.714 & 0.076 & 4 & en \\ 
Singapore & SGP & 0.742 & 0.541 & 1.602 & -- & -- & en \\ 
Solomon Islands & SLB & 0.545 & 0.424 & 0.081 & -- & -- & en \\ 
Somalia & SOM & 0.608 & 0.317 & -0.959 & 0.04 & 5 & ar \\ 
South Africa & ZAF & 0.514 & 0.414 & 0.052 & 0.018 & 9 & en \\ 
South Sudan & SSD & 0.605 & 0.464 & -0.514 & 0.01 & 2 & en \\ 
Spain & ESP & 0.892 & 0.684 & 0.033 & 0.023 & 17 & es \\ 
St. Kitts \& Nevis & KNA & 0.753 & 0.55 & 1.657 & 0.003 & 2 & en \\ 
St. Lucia & LCA & 0.631 & 0.473 & 0.792 & 0.02 & 3 & en \\ 
St. Vincent \& Grenadines & VCT & 0.641 & 0.478 & 0.183 & 0.043 & 2 & en \\ 
Sudan & SDN & 0.673 & 0.34 & -0.032 & 0.165 & 9 & ar \\ 
Suriname & SUR & 0.572 & 0.442 & -0.185 & 0.031 & 2 & nl \\ 
Swaziland & SWZ & 0.555 & 0.43 & -0.326 & 0.016 & 4 & en \\ 
Switzerland & CHE & 0.812 & 0.678 & -0.113 & 0.033 & 22 & de \\ 
Syria & SYR & 0.7 & 0.356 & -0.107 & 0.105 & 14 & ar \\ 
Taiwan & TWN & 2.101 & 0.583 & 0.565 & 0.033 & 16 & zh \\ 
Tajikistan & TJK & 0.229 & 0.527 & 2.07 & 0.005 & 2 & ru \\ 
Tanzania & TZA & 0.444 & 0.397 & -0.037 & 0.012 & 19 & en \\ 
Timor-Leste & TLS & 0.416 & 0.29 & -0.193 & -- & -- & pt \\ 
Togo & TGO & 0.273 & 0.262 & -0.839 & -- & -- & fr \\ 
Tonga & TON & 0.739 & 0.538 & 1.969 & -- & -- & en \\ 
Trinidad \& Tobago & TTO & 0.677 & 0.489 & 0.948 & 0.028 & 13 & en \\ 
Tunisia & TUN & 0.709 & 0.362 & 0.058 & 0.05 & 24 & ar \\ 
Turkey & TUR & 0.835 & 0.593 & 0.055 & 0.037 & 78 & tr \\ 
Turks \& Caicos Islands & TCA & 0.676 & 0.489 & 0.935 & -- & -- & en \\ 
Uganda & UGA & 0.624 & 0.469 & 0.06 & 0.035 & 22 & en \\ 
Ukraine & UKR & 0.584 & 0.806 & 0.625 & 0.044 & 27 & ru \\ 
United Arab Emirates & ARE & 0.718 & 0.374 & -1.421 & 0.025 & 6 & ar \\ 
United Kingdom & GBR & 0.987 & 0.671 & 0.807 & 0.011 & 4 & en \\ 
United States & USA & 0.951 & 0.657 & 0.755 & 0.017 & 51 & en \\ 
Uruguay & URY & 0.572 & 0.499 & 0.104 & 0.02 & 19 & es \\ 
U.S. Virgin Islands & VIR & 0.842 & 0.607 & 1.41 & -- & -- & en \\ 
Uzbekistan & UZB & 0.223 & 0.524 & 1.558 & 0.018 & 9 & ru \\ 
Venezuela & VEN & 0.437 & 0.444 & -1.165 & 0.028 & 24 & es \\ 
Yemen & YEM & 0.979 & 0.52 & -0.703 & 0.048 & 7 & ar \\ 
Zambia & ZMB & 0.568 & 0.441 & -0.083 & 0.016 & 9 & en \\ 
Zimbabwe & ZWE & 0.602 & 0.464 & -0.295 & 0.024 & 8 & en \\ 
\hline \\[-1.8ex] 
\multicolumn{8}{l}{The representative language for each country is chosen as the most used language by the country's population observed on Facebook.} \\ 
% \bottomrule
% \end{tabular} 
% \end{table} 
\end{longtable} 
\endgroup
}
% Table created by stargazer v.5.2.2 by Marek Hlavac, Harvard University. E-mail: hlavac at fas.harvard.edu
% Date and time: Tue, Jul 14, 2020 - 01:22:52 PM
\begin{table*}[!htbp] \centering {\tiny
  \caption{Variables related to gender gap analysis. Reported $N$ is the number of countries matched with our data.} 
  \label{tab:gender_vars}
\begin{tabular}{@{\extracolsep{-4pt}}lrrrrrrll} 
\\[-1.8ex]\hline 
\hline \\[-1.8ex] 
Statistic & \multicolumn{1}{c}{N} & \multicolumn{1}{c}{Mean} & \multicolumn{1}{c}{St. Dev.} & \multicolumn{1}{c}{Min} & \multicolumn{1}{c}{Max} & \multicolumn{1}{c}{Median} & \multicolumn{1}{l}{Definition} & \multicolumn{1}{l}{Source} \\ 
\hline \\[-1.8ex] 
\olle gap & 160 & 0.053 & 1.002 & $-$3.344 & 2.070 & 0.080 & female-map gap in \olle &\\ 
offline literacy (all) & 143 & 84.181 & 19.263 & 19.100 & 100.000 & 93.464 & literacy rate & UNESCO \cite{unesco2019produce}\\  
offline literacy (gap) & 114 & $-$0.068 & 0.090 & $-$0.301 & 0.182 & $-$0.036 & female-male gap in literacy & UNESCO \cite{unesco2019produce}\\ 
education (all) & 106 & 7.884 & 2.741 & 1.880 & 13.180 & 8.085 & mean schooling years & Barro-Lee Educational Attainment Data \cite{BarroLee73:online}\\ 
education (gap) & 106 & $-$0.517 & 0.968 & $-$3.250 & 1.600 & $-$0.420 & female-map gap in schooling years & Barro-Lee Educational Attainment Data \cite{BarroLee73:online}\\ 
\% Internet (all) & 151 & 0.520 & 0.284 & 0.020 & 0.984 & 0.555 & overall Internet penetration & ITU Internet gender gap \cite{Aboutthe43:online}\\
\% Internet (gap) & 128 & 0.880 & 0.123 & 0.545 & 1.000 & 0.919 & femal-male gap in Internet penetration & Digital gender gap (U. Oxford) \cite{Aboutthe43:online} \\ 
civic (all) & 123 & 0.695 & 0.211 & 0.105 & 0.973 & 0.746 & overall civic society participation & V-Dem Institute \cite{coppedge2019v}\\ 
civic (women) & 123 & 0.720 & 0.173 & 0.234 & 0.937 & 0.775 & women's civic society participation & V-Dem Institute \cite{coppedge2019v}\\ 
GII & 107 & 0.398 & 0.194 & 0.040 & 0.835 & 0.424 & Gender Inequality Index & HDRO \cite{HDROAPII26:online}\\  
\hline \\[-1.8ex] 
\end{tabular} }
\end{table*} 
% Table created by stargazer v.5.2.2 by Marek Hlavac, Harvard University. E-mail: hlavac at fas.harvard.edu
% Date and time: Tue, Jul 14, 2020 - 01:35:00 PM
\begin{table*}[!htbp] \centering {\tiny 
  \caption{Correlations among variables related to gender gap or gender equity. All correlations are reported using Spearman rank correlation coefficients.} 
  \label{tab:gender_corr} 
\begin{tabular}{@{\extracolsep{-4pt}} lllllllllll} 
\\[-1.8ex]\hline 
\hline \\[-1.8ex] 
 & \olle gap & \olle & \shortstack[l]{offline\\[-.5ex]literacy (all)} & \shortstack[l]{offline\\[-.5ex]literacy (gap)} & eduation (all) & education (gap) & \% Internet (all) & \% Internet (gap) & civic (all) & civic (women) \\ 
\hline \\[-1.8ex] 
\olle &  0.585$^*$$^*$$^*$ &  &  &  &  &  &  &  &  &  \\ 
off. literacy (all) &  0.524$^*$$^*$$^*$ &  0.740$^*$$^*$$^*$ &  &  &  &  &  &  &  &  \\ 
off. literacy (gap) &  0.420$^*$$^*$$^*$ &  0.438$^*$$^*$$^*$ &  0.709$^*$$^*$$^*$ &  &  &  &  &  &  &  \\ 
eduation (all) &  0.587$^*$$^*$$^*$ &  0.739$^*$$^*$$^*$ &  0.872$^*$$^*$$^*$ &  0.694$^*$$^*$$^*$ &  &  &  &  &  &  \\ 
education (gap) &  0.225$^*$  &  0.207$^*$  &  0.515$^*$$^*$$^*$ &  0.787$^*$$^*$$^*$ &  0.455$^*$$^*$$^*$ &  &  &  &  &  \\ 
\% Internet (all) &  0.304$^*$$^*$$^*$ &  0.573$^*$$^*$$^*$ &  0.748$^*$$^*$$^*$ &  0.582$^*$$^*$$^*$ &  0.743$^*$$^*$$^*$ &  0.434$^*$$^*$$^*$ &  &  &  &  \\ 
\% Internet (gap) &  0.487$^*$$^*$$^*$ &  0.533$^*$$^*$$^*$ &  0.663$^*$$^*$$^*$ &  0.779$^*$$^*$$^*$ &  0.730$^*$$^*$$^*$ &  0.620$^*$$^*$$^*$ &  0.617$^*$$^*$$^*$ &  &  &  \\ 
civic (all) &  0.287$^*$$^*$  &  0.355$^*$$^*$$^*$ &  0.159  & -0.019  &  0.358$^*$$^*$$^*$ &  0.028  &  0.228$^*$  &  0.328$^*$$^*$$^*$ &  &  \\ 
civic (women) &  0.479$^*$$^*$$^*$ &  0.484$^*$$^*$$^*$ &  0.394$^*$$^*$$^*$ &  0.317$^*$$^*$  &  0.564$^*$$^*$$^*$ &  0.252$^*$  &  0.429$^*$$^*$$^*$ &  0.598$^*$$^*$$^*$ &  0.695$^*$$^*$$^*$ &  \\ 
GII & -0.391$^*$$^*$$^*$ & -0.624$^*$$^*$$^*$ & -0.847$^*$$^*$$^*$ & -0.610$^*$$^*$$^*$ & -0.838$^*$$^*$$^*$ & -0.475$^*$$^*$$^*$ & -0.872$^*$$^*$$^*$ & -0.647$^*$$^*$$^*$ & -0.219$^*$  & -0.414$^*$$^*$$^*$ \\ 
\hline \\[-1.8ex] 
\multicolumn{11}{l}{significance levels: $^{***}p<0.001$; $^{**}p<0.01$; $^*p<0.05$} \\ 
\end{tabular} }
\end{table*} 

% Table created by stargazer v.5.2.2 by Marek Hlavac, Harvard University. E-mail: hlavac at fas.harvard.edu
% Date and time: Tue, Jul 14, 2020 - 01:40:52 PM
\begin{table*}[!htbp] \centering {\tiny
  \caption{Variables related to country resource access and inequality. Reported $N$ is the number of countries matched with our data.} 
  \label{tab:region_vars} 
\begin{tabular}{@{\extracolsep{-4pt}}lrrrrrrll} 
\\[-1.8ex]\hline 
\hline \\[-1.8ex] 
Statistic & \multicolumn{1}{c}{N} & \multicolumn{1}{c}{Mean} & \multicolumn{1}{c}{St. Dev.} & \multicolumn{1}{c}{Min} & \multicolumn{1}{c}{Max} & \multicolumn{1}{c}{Median} & \multicolumn{1}{l}{Definition} & \multicolumn{1}{l}{Source} \\  
\hline \\[-1.8ex] 
regional disparity & 119 & 0.037 & 0.027 & 0.001 & 0.165 & 0.030 & St. Dev. of sub-national \olle \\  
income & 115 & 16.193 & 17.054 & 0.800 & 71.160 & 9.359 & GNI per capita in 1,000 US\$ (2011 PPP) & HDI \cite{GlobalDa32:online}\\ 
Gini index & 94 & 40.004 & 7.973 & 25.000 & 63.000 & 40.200 & Gini coefficient for income & HDR \cite{HumanDev90:online}\\  
education (all) & 95 & 7.864 & 2.753 & 1.880 & 13.180 & 7.970 & mean schooling years & Barro-Lee Educational Attainment Data \cite{BarroLee73:online}\\ 
unequal education & 102 & 21.025 & 14.073 & 0.800 & 49.300 & 17.450 & Inequality in education & HDR \cite{HumanDev90:online}\\  
\% Internet (all) & 119 & 0.507 & 0.275 & 0.020 & 0.980 & 0.508 & overall Internet penetration & ITU Internet gender gap \cite{Aboutthe43:online}\\ 
civic (all) & 110 & 0.704 & 0.213 & 0.105 & 0.973 & 0.764 & overall civic society participation & V-Dem Institute \cite{coppedge2019v}\\ 
\hline \\[-1.8ex] 
\end{tabular} }
\end{table*} 
% Table created by stargazer v.5.2.2 by Marek Hlavac, Harvard University. E-mail: hlavac at fas.harvard.edu
% Date and time: Tue, Jul 14, 2020 - 01:45:05 PM
\begin{table*}[!htbp] \centering {\tiny 
  \caption{Correlations among variables related to country resource access and inequality. All correlations are reported using Spearman rank correlation coefficients.} 
  \label{tab:region_corr} 
\begin{tabular}{@{\extracolsep{-4pt}} llllllll} 
\\[-1.8ex]\hline 
\hline \\[-1.8ex] 
 & regional disparity & \olle & income & Gini index & education (all) & unequal education & \% Internet (all) \\ 
\hline \\[-1.8ex] 
regional disparity &  &  &  &  &  &  &  \\ 
\olle & -0.158  &  &  &  &  &  &  \\ 
income & -0.156  &  0.476$^*$$^*$$^*$ &  &  &  &  &  \\ 
Gini index & -0.271$^*$$^*$  & -0.346$^*$$^*$$^*$ & -0.260$^*$  &  &  &  &  \\ 
eduation (all) & -0.312$^*$$^*$  &  0.706$^*$$^*$$^*$ &  0.745$^*$$^*$$^*$ & -0.303$^*$$^*$  &  &  &  \\ 
unequal education &  0.283$^*$$^*$  & -0.681$^*$$^*$$^*$ & -0.703$^*$$^*$$^*$ &  0.240$^*$  & -0.875$^*$$^*$$^*$ &  &  \\ 
\% Internet (all) & -0.043  &  0.532$^*$$^*$$^*$ &  0.905$^*$$^*$$^*$ & -0.306$^*$$^*$  &  0.751$^*$$^*$$^*$ & -0.735$^*$$^*$$^*$ &  \\ 
civic (all) & -0.012  &  0.344$^*$$^*$$^*$ &  0.192$^*$  & -0.154  &  0.366$^*$$^*$$^*$ & -0.207$^*$  &  0.237$^*$  \\ 
\hline \\[-1.8ex] 
\multicolumn{8}{l}{significance levels: $^{***}p<0.001$; $^{**}p<0.01$; $^*p<0.05$} \\ 
\end{tabular} }
\end{table*} 




% Table created by stargazer v.5.2.2 by Marek Hlavac, Harvard University. E-mail: hlavac at fas.harvard.edu
% Date and time: Tue, Jul 14, 2020 - 03:07:54 PM
\begin{table*}[!htbp] \centering {\tiny 
  \caption{OLS for predicting female-male gender gap in online literacy estimates. Models (d) corresponds to the figure in the main text. Models (b,c,e,f) include alternative interaction terms. Values in parentheses are the lower and upper bounds of the 95\% confidence intervals of the estimated effects.} 
  \label{tab:model-gender-main} 
\begin{tabular}{@{\extracolsep{-10pt}}lcccccc} 
\\[-1.8ex]\hline 
\hline \\[-1.8ex] 
\\[-1.8ex] & \multicolumn{6}{c}{DV: female-male gender gap} \\ 
 & (a) & (b) & (c) & (d) & (e) & (f) \\ 
\hline \\[-1.8ex] 
 women civic & 0.32$^{***}$ & 0.33$^{***}$ & 0.35$^{***}$ & 0.38$^{***}$ & 0.37$^{***}$ & 0.40$^{***}$ \\ 
  & (0.15, 0.48) & (0.14, 0.52) & (0.17, 0.52) & (0.20, 0.56) & (0.18, 0.56) & (0.22, 0.57) \\ 
  & & & & & & \\ 
 (women civic):(\% Internet) & 0.25$^{***}$ &  &  & 0.13$^{*}$ &  &  \\ 
  & (0.10, 0.41) &  &  & ($-$0.02, 0.29) &  &  \\ 
  & & & & & & \\ 
 (women civic):(education) &  & 0.28 &  &  & 0.19 &  \\ 
  &  & ($-$0.10, 0.66) &  &  & ($-$0.14, 0.53) &  \\ 
  & & & & & & \\ 
 education (all) & 0.25$^{***}$ & 0.37$^{***}$ & 0.90$^{**}$ & 0.20$^{**}$ & 0.28$^{**}$ & 0.66$^{**}$ \\ 
  & (0.08, 0.42) & (0.13, 0.61) & (0.19, 1.61) & (0.05, 0.35) & (0.06, 0.49) & (0.03, 1.28) \\ 
  & & & & & & \\ 
 (education):(\% Internet) &  &  & 0.51$^{*}$ &  &  & 0.36 \\ 
  &  &  & ($-$0.02, 1.05) &  &  & ($-$0.11, 0.84) \\ 
  & & & & & & \\ 
 \% Internet (all) & $-$0.02 & $-$0.08 & $-$0.23$^{*}$ & $-$0.22$^{*}$ & $-$0.28$^{**}$ & $-$0.41$^{***}$ \\ 
  & ($-$0.21, 0.16) & ($-$0.27, 0.11) & ($-$0.50, 0.03) & ($-$0.45, 0.01) & ($-$0.50, $-$0.06) & ($-$0.69, $-$0.13) \\ 
  & & & & & & \\ 
 Central/Southern/Eastern Asia &  &  &  & 1.66$^{***}$ & 1.77$^{***}$ & 1.80$^{***}$ \\ 
  &  &  &  & (1.05, 2.27) & (1.17, 2.37) & (1.22, 2.39) \\ 
  & & & & & & \\ 
 Europe/Oceania/Northern America &  &  &  & 0.63$^{**}$ & 0.76$^{***}$ & 0.79$^{***}$ \\ 
  &  &  &  & (0.03, 1.22) & (0.20, 1.33) & (0.25, 1.34) \\ 
  & & & & & & \\ 
 Latin America \& the Caribbean &  &  &  & 0.29 & 0.32 & 0.38$^{*}$ \\ 
  &  &  &  & ($-$0.15, 0.74) & ($-$0.12, 0.77) & ($-$0.06, 0.82) \\ 
  & & & & & & \\ 
 Northern Africa \& Western Asia &  &  &  & 0.29 & 0.30 & 0.41 \\ 
  &  &  &  & ($-$0.26, 0.84) & ($-$0.26, 0.85) & ($-$0.14, 0.96) \\ 
  & & & & & & \\ 
 Constant & $-$0.35$^{***}$ & $-$0.30$^{***}$ & $-$0.47$^{***}$ & $-$0.69$^{***}$ & $-$0.72$^{***}$ & $-$0.89$^{***}$ \\ 
  & ($-$0.51, $-$0.18) & ($-$0.48, $-$0.13) & ($-$0.75, $-$0.19) & ($-$1.02, $-$0.37) & ($-$1.05, $-$0.40) & ($-$1.28, $-$0.50) \\ 
  & & & & & & \\ 
\hline \\[-1.8ex] 
Out-of-sample RMSE & 0.8 & 0.83 & 0.83 & 0.72 & 0.73 & 0.73 \\ 
Out-of-sample R2 & 0.24 & 0.21 & 0.2 & 0.39 & 0.38 & 0.38 \\ 
Observations & 101 & 101 & 101 & 101 & 101 & 101 \\ 
R$^{2}$ & 0.34 & 0.29 & 0.30 & 0.51 & 0.51 & 0.51 \\ 
Adjusted R$^{2}$ & 0.31 & 0.26 & 0.27 & 0.47 & 0.46 & 0.47 \\ 
AIC & 240.92 & 248.67 & 247.25 & 217.60 & 219.21 & 218.11 \\ 
BIC & 256.61 & 264.36 & 262.94 & 243.75 & 245.36 & 244.26 \\ 
Residual Std. Error & 0.77 (df = 96) & 0.80 (df = 96) & 0.80 (df = 96) & 0.67 (df = 92) & 0.68 (df = 92) & 0.68 (df = 92) \\ 
F Statistic & 12.26$^{***}$ (df = 4; 96) & 9.58$^{***}$ (df = 4; 96) & 10.06$^{***}$ (df = 4; 96) & 12.19$^{***}$ (df = 8; 92) & 11.82$^{***}$ (df = 8; 92) & 12.07$^{***}$ (df = 8; 92) \\ 
\hline 
\hline \\[-1.8ex] 
\textit{Note:}  & \multicolumn{6}{r}{$^{*}$p$<$0.1; $^{**}$p$<$0.05; $^{***}$p$<$0.01} \\ 
\end{tabular} }
\end{table*} 



% Table created by stargazer v.5.2.2 by Marek Hlavac, Harvard University. E-mail: hlavac at fas.harvard.edu
% Date and time: Tue, Jul 14, 2020 - 03:07:59 PM
\begin{table*}[!htbp] \centering { \tiny 
  \caption{OLS for predicting female-male gender gap in online literacy estimates. Models include alternative predictors and no interaction term. Values in parentheses are the lower and upper bounds of the 95\% confidence intervals of the estimated effects.} 
  \label{tab:model-gender-3v} 
\begin{tabular}{@{\extracolsep{-10pt}}lcccccc} 
\\[-1.8ex]\hline 
\hline \\[-1.8ex] 
\\[-1.8ex] & \multicolumn{6}{c}{DV: female-male gender gap} \\ 
 & (a) & (b) & (c) & (d) & (e) & (f) \\ 
\hline \\[-1.8ex] 
 women civic & 0.39$^{***}$ & 0.39$^{***}$ & 0.40$^{***}$ & 0.37$^{***}$ &  &  \\ 
  & (0.22, 0.56) & (0.22, 0.57) & (0.20, 0.60) & (0.15, 0.59) &  &  \\ 
  & & & & & & \\ 
 civic (all) &  &  &  &  & 0.21$^{**}$ & 0.21$^{**}$ \\ 
  &  &  &  &  & (0.05, 0.38) & (0.01, 0.41) \\ 
  & & & & & & \\ 
 education (gap) &  & 0.05 &  & 0.02 &  &  \\ 
  &  & ($-$0.12, 0.23) &  & ($-$0.19, 0.24) &  &  \\ 
  & & & & & & \\ 
 education (all) & 0.24$^{***}$ &  & 0.24$^{**}$ &  & 0.27$^{***}$ & 0.24$^{**}$ \\ 
  & (0.07, 0.42) &  & (0.06, 0.43) &  & (0.09, 0.46) & (0.04, 0.43) \\ 
  & & & & & & \\ 
 \% Internet (gap) &  &  & $-$0.03 & 0.10 &  & 0.15 \\ 
  &  &  & ($-$0.25, 0.20) & ($-$0.17, 0.37) &  & ($-$0.06, 0.36) \\ 
  & & & & & & \\ 
 \% Internet (all) & $-$0.05 & 0.04 &  &  & 0.07 &  \\ 
  & ($-$0.24, 0.14) & ($-$0.16, 0.23) &  &  & ($-$0.12, 0.26) &  \\ 
  & & & & & & \\ 
 Constant & $-$0.25$^{***}$ & $-$0.25$^{***}$ & $-$0.26$^{***}$ & $-$0.25$^{***}$ & $-$0.24$^{***}$ & $-$0.25$^{***}$ \\ 
  & ($-$0.40, $-$0.09) & ($-$0.41, $-$0.09) & ($-$0.42, $-$0.09) & ($-$0.42, $-$0.08) & ($-$0.41, $-$0.07) & ($-$0.42, $-$0.07) \\ 
  & & & & & & \\ 
\hline \\[-1.8ex] 
Out-of-sample RMSE & 0.83 & 0.87 & 0.85 & 0.88 & 0.91 & 0.92 \\ 
Out-of-sample R2 & 0.19 & 0.13 & 0.21 & 0.13 & 0.09 & 0.1 \\ 
Observations & 101 & 101 & 98 & 98 & 101 & 98 \\ 
R$^{2}$ & 0.27 & 0.22 & 0.28 & 0.22 & 0.17 & 0.19 \\ 
Adjusted R$^{2}$ & 0.25 & 0.19 & 0.25 & 0.20 & 0.15 & 0.17 \\ 
AIC & 248.88 & 256.07 & 243.76 & 250.52 & 261.40 & 254.21 \\ 
BIC & 261.96 & 269.14 & 256.69 & 263.44 & 274.48 & 267.13 \\ 
Residual Std. Error & 0.81 (df = 97) & 0.83 (df = 97) & 0.81 (df = 94) & 0.84 (df = 94) & 0.86 (df = 97) & 0.86 (df = 94) \\ 
F Statistic & 11.93$^{***}$ (df = 3; 97) & 8.89$^{***}$ (df = 3; 97) & 11.96$^{***}$ (df = 3; 94) & 9.07$^{***}$ (df = 3; 94) & 6.77$^{***}$ (df = 3; 97) & 7.58$^{***}$ (df = 3; 94) \\ 
\hline 
\hline \\[-1.8ex] 
\textit{Note:}  & \multicolumn{6}{r}{$^{*}$p$<$0.1; $^{**}$p$<$0.05; $^{***}$p$<$0.01} \\ 
 & \multicolumn{6}{r}{Models (c,d,f) have fewer observations (and slightly lower prediction error) due to missing data in the new predictor.} \\ 
\end{tabular} }
\end{table*} 


% Table created by stargazer v.5.2.2 by Marek Hlavac, Harvard University. E-mail: hlavac at fas.harvard.edu
% Date and time: Tue, Jul 14, 2020 - 03:08:06 PM
\begin{table*}[!htbp] \centering {\tiny 
  \caption{OLS for predicting female-male gender gap in online literacy estimates. Models include alternative predictors and controls for geographical groups. Values in parentheses are the lower and upper bounds of the 95\% confidence intervals of the estimated effects.} 
  \label{tab:model-gender-3v-geo} 
\begin{tabular}{@{\extracolsep{-10pt}}lcccccc} 
\\[-1.8ex]\hline 
\hline \\[-1.8ex] 
\\[-1.8ex] & \multicolumn{6}{c}{DV: female-male gender gap} \\ 
 & (a) & (b) & (c) & (d) & (e) & (f) \\ 
\hline \\[-1.8ex] 
 women civic & 0.41$^{***}$ & 0.43$^{***}$ & 0.35$^{***}$ & 0.37$^{***}$ &  &  \\ 
  & (0.23, 0.58) & (0.26, 0.61) & (0.16, 0.53) & (0.17, 0.56) &  &  \\ 
  & & & & & & \\ 
 civic (all) &  &  &  &  & 0.16$^{*}$ & 0.16 \\ 
  &  &  &  &  & ($-$0.02, 0.34) & ($-$0.04, 0.37) \\ 
  & & & & & & \\ 
 education (gap) &  & 0.13$^{*}$ &  & 0.10 &  &  \\ 
  &  & ($-$0.02, 0.29) &  & ($-$0.08, 0.29) &  &  \\ 
  & & & & & & \\ 
 education (all) & 0.19$^{**}$ &  & 0.19$^{**}$ &  & 0.21$^{**}$ & 0.18$^{**}$ \\ 
  & (0.04, 0.34) &  & (0.02, 0.36) &  & (0.05, 0.38) & (0.01, 0.36) \\ 
  & & & & & & \\ 
 \% Internet (gap) &  &  & $-$0.07 & $-$0.07 &  & 0.08 \\ 
  &  &  & ($-$0.30, 0.15) & ($-$0.33, 0.20) &  & ($-$0.14, 0.30) \\ 
  & & & & & & \\ 
 \% Internet (all) & $-$0.28$^{**}$ & $-$0.30$^{**}$ &  &  & $-$0.11 &  \\ 
  & ($-$0.50, $-$0.06) & ($-$0.54, $-$0.06) &  &  & ($-$0.34, 0.11) &  \\ 
  & & & & & & \\ 
 Central/Southern/Eastern Asia & 1.81$^{***}$ & 1.99$^{***}$ & 1.48$^{***}$ & 1.64$^{***}$ & 1.67$^{***}$ & 1.50$^{***}$ \\ 
  & (1.22, 2.41) & (1.40, 2.59) & (0.92, 2.03) & (1.09, 2.19) & (1.02, 2.32) & (0.91, 2.10) \\ 
  & & & & & & \\ 
 Europe/Oceania/Northern America & 0.84$^{***}$ & 0.97$^{***}$ & 0.46$^{*}$ & 0.58$^{**}$ & 0.83$^{***}$ & 0.53$^{*}$ \\ 
  & (0.30, 1.39) & (0.42, 1.53) & ($-$0.03, 0.96) & (0.07, 1.08) & (0.24, 1.43) & (0.002, 1.05) \\ 
  & & & & & & \\ 
 Latin America \& the Caribbean & 0.37 & 0.44$^{*}$ & 0.18 & 0.25 & 0.39 & 0.17 \\ 
  & ($-$0.07, 0.81) & ($-$0.004, 0.88) & ($-$0.28, 0.65) & ($-$0.22, 0.72) & ($-$0.09, 0.88) & ($-$0.32, 0.67) \\ 
  & & & & & & \\ 
 Northern Africa \& Western Asia & 0.35 & 0.48$^{*}$ & $-$0.16 & $-$0.04 & 0.03 & $-$0.20 \\ 
  & ($-$0.20, 0.90) & ($-$0.07, 1.04) & ($-$0.63, 0.31) & ($-$0.50, 0.42) & ($-$0.58, 0.65) & ($-$0.74, 0.33) \\ 
  & & & & & & \\ 
 Constant & $-$0.73$^{***}$ & $-$0.81$^{***}$ & $-$0.50$^{***}$ & $-$0.58$^{***}$ & $-$0.65$^{***}$ & $-$0.50$^{***}$ \\ 
  & ($-$1.06, $-$0.40) & ($-$1.14, $-$0.48) & ($-$0.80, $-$0.20) & ($-$0.88, $-$0.28) & ($-$1.02, $-$0.28) & ($-$0.82, $-$0.17) \\ 
  & & & & & & \\ 
\hline \\[-1.8ex] 
Out-of-sample RMSE & 0.73 & 0.75 & 0.76 & 0.8 & 0.8 & 0.82 \\ 
Out-of-sample R$^{2}$ & 0.37 & 0.35 & 0.35 & 0.31 & 0.28 & 0.29 \\ 
Observations & 101 & 101 & 98 & 98 & 101 & 98 \\ 
R$^{2}$ & 0.50 & 0.48 & 0.48 & 0.46 & 0.41 & 0.42 \\ 
Adjusted R$^{2}$ & 0.46 & 0.44 & 0.44 & 0.42 & 0.36 & 0.37 \\ 
AIC & 218.59 & 222.02 & 218.82 & 222.81 & 235.76 & 230.10 \\ 
BIC & 242.12 & 245.56 & 242.09 & 246.07 & 259.30 & 253.36 \\ 
Residual Std. Error & 0.68 (df = 93) & 0.69 (df = 93) & 0.70 (df = 90) & 0.72 (df = 90) & 0.74 (df = 93) & 0.75 (df = 90) \\ 
F Statistic & 13.29$^{***}$ (df = 7; 93) & 12.40$^{***}$ (df = 7; 93) & 12.00$^{***}$ (df = 7; 90) & 11.01$^{***}$ (df = 7; 90) & 9.13$^{***}$ (df = 7; 93) & 9.30$^{***}$ (df = 7; 90) \\ 
\hline 
\hline \\[-1.8ex] 
\textit{Note:}  & \multicolumn{6}{r}{$^{*}$p$<$0.1; $^{**}$p$<$0.05; $^{***}$p$<$0.01} \\ 
 & \multicolumn{6}{r}{Models (c,d,f) have fewer observations (and slightly lower prediction error) due to missing data in the new predictor.} \\ 
\end{tabular} }
\end{table*} 




% Table created by stargazer v.5.2.2 by Marek Hlavac, Harvard University. E-mail: hlavac at fas.harvard.edu
% Date and time: Tue, Jul 14, 2020 - 03:27:10 PM
\begin{table*}[!htbp] \centering {\tiny 
  \caption{OLS for predicting within-country regional disparity in online literacy estimates. Models (d) corresponds to the figure in the main text. Models (b,c,e,f) include alternative interaction terms. Values in parentheses are the lower and upper bounds of the 95\% confidence intervals of the estimated effects.} 
  \label{tab:model-region-main} 
\begin{tabular}{@{\extracolsep{-10pt}}lcccccc} 
\\[-1.8ex]\hline 
\hline \\[-1.8ex] 
\\[-1.8ex] & \multicolumn{6}{c}{DV: within-country regional disparity} \\ 
 & (a) & (b) & (c) & (d) & (e) & (f) \\ 
\hline \\[-1.8ex] 
 unequal edu & 0.47$^{**}$ & 0.34$^{**}$ & 0.32$^{*}$ & 0.41$^{**}$ & 0.25 & 0.25 \\ 
  & (0.10, 0.84) & (0.02, 0.66) & ($-$0.02, 0.66) & (0.03, 0.80) & ($-$0.10, 0.60) & ($-$0.12, 0.63) \\ 
  & & & & & & \\ 
 (unequal edu):(Gini) & 0.18 &  &  & 0.25$^{*}$ &  &  \\ 
  & ($-$0.07, 0.44) &  &  & ($-$0.02, 0.51) &  &  \\ 
  & & & & & & \\ 
 (unequal edu):(\% Internet) &  & $-$0.09 &  &  & $-$0.15 &  \\ 
  &  & ($-$0.36, 0.18) &  &  & ($-$0.46, 0.17) &  \\ 
  & & & & & & \\ 
 (unequal edu):(education) &  &  & 0.003 &  &  & 0.01 \\ 
  &  &  & ($-$0.37, 0.38) &  &  & ($-$0.38, 0.41) \\ 
  & & & & & & \\ 
 Gini & $-$0.22$^{**}$ & $-$0.22$^{**}$ & $-$0.24$^{**}$ & $-$0.35$^{**}$ & $-$0.32$^{**}$ & $-$0.37$^{**}$ \\ 
  & ($-$0.42, $-$0.02) & ($-$0.43, $-$0.02) & ($-$0.46, $-$0.02) & ($-$0.61, $-$0.09) & ($-$0.61, $-$0.04) & ($-$0.67, $-$0.08) \\ 
  & & & & & & \\ 
 \% Internet (all) & 0.40$^{**}$ & 0.33$^{*}$ & 0.35$^{**}$ & 0.46$^{**}$ & 0.34$^{*}$ & 0.41$^{*}$ \\ 
  & (0.08, 0.72) & (0.0000, 0.66) & (0.01, 0.70) & (0.08, 0.83) & ($-$0.05, 0.74) & ($-$0.02, 0.83) \\ 
  & & & & & & \\ 
 education (all) & $-$0.29$^{**}$ & $-$0.27$^{*}$ & $-$0.31 & $-$0.33$^{**}$ & $-$0.29$^{*}$ & $-$0.36 \\ 
  & ($-$0.55, $-$0.02) & ($-$0.56, 0.02) & ($-$0.83, 0.21) & ($-$0.60, $-$0.06) & ($-$0.59, $-$0.003) & ($-$0.90, 0.19) \\ 
  & & & & & & \\ 
 Central/Southern/Eastern Asia &  &  &  & $-$0.17 & $-$0.05 & $-$0.24 \\ 
  &  &  &  & ($-$1.06, 0.72) & ($-$1.04, 0.94) & ($-$1.16, 0.68) \\ 
  & & & & & & \\ 
 Europe/Oceania/Northern America &  &  &  & $-$0.54 & $-$0.42 & $-$0.47 \\ 
  &  &  &  & ($-$1.51, 0.43) & ($-$1.40, 0.56) & ($-$1.47, 0.54) \\ 
  & & & & & & \\ 
 Latin America \& the Caribbean &  &  &  & 0.26 & 0.27 & 0.19 \\ 
  &  &  &  & ($-$0.36, 0.88) & ($-$0.38, 0.92) & ($-$0.46, 0.84) \\ 
  & & & & & & \\ 
 Northern Africa \& Western Asia &  &  &  & $-$0.09 & $-$0.03 & $-$0.18 \\ 
  &  &  &  & ($-$0.89, 0.72) & ($-$0.90, 0.83) & ($-$1.03, 0.67) \\ 
  & & & & & & \\ 
 Constant & $-$0.04 & $-$0.06 & 0.004 & 0.03 & $-$0.06 & 0.11 \\ 
  & ($-$0.24, 0.16) & ($-$0.33, 0.21) & ($-$0.30, 0.31) & ($-$0.47, 0.52) & ($-$0.66, 0.54) & ($-$0.51, 0.73) \\ 
  & & & & & & \\ 
\hline \\[-1.8ex] 
Out-of-sample RMSE & 0.9 & 0.92 & 0.92 & 0.92 & 0.95 & 0.95 \\ 
Out-of-sample R$^{2}$ & 0.14 & 0.1 & 0.11 & 0.09 & 0.08 & 0.06 \\ 
Observations & 79 & 79 & 79 & 79 & 79 & 79 \\ 
R$^{2}$ & 0.25 & 0.23 & 0.23 & 0.29 & 0.26 & 0.25 \\ 
Adjusted R$^{2}$ & 0.19 & 0.18 & 0.17 & 0.19 & 0.17 & 0.16 \\ 
AIC & 207.37 & 209.02 & 209.48 & 210.91 & 213.57 & 214.53 \\ 
BIC & 223.96 & 225.60 & 226.07 & 236.98 & 239.63 & 240.59 \\ 
Residual Std. Error & 0.86 (df = 73) & 0.86 (df = 73) & 0.87 (df = 73) & 0.86 (df = 69) & 0.87 (df = 69) & 0.88 (df = 69) \\ 
F Statistic & 4.75$^{***}$ (df = 5; 73) & 4.36$^{***}$ (df = 5; 73) & 4.24$^{***}$ (df = 5; 73) & 3.09$^{***}$ (df = 9; 69) & 2.73$^{***}$ (df = 9; 69) & 2.61$^{**}$ (df = 9; 69) \\ 
\hline 
\hline \\[-1.8ex] 
\textit{Note:}  & \multicolumn{6}{r}{$^{*}$p$<$0.1; $^{**}$p$<$0.05; $^{***}$p$<$0.01} \\ 
\end{tabular} }
\end{table*} 


% Table created by stargazer v.5.2.2 by Marek Hlavac, Harvard University. E-mail: hlavac at fas.harvard.edu
% Date and time: Tue, Jul 14, 2020 - 03:30:01 PM
\begin{table*}[!htbp] \centering {\tiny 
  \caption{OLS for predicting within-country regional disparity in online literacy estimates. Models include alternative predictors and no interaction term. Values in parentheses are the lower and upper bounds of the 95\% confidence intervals of the estimated effects.} 
  \label{tab:model-region-4v} 
\begin{tabular}{@{\extracolsep{-10pt}}lcccccc} 
\\[-1.8ex]\hline 
\hline \\[-1.8ex] 
\\[-1.8ex] & \multicolumn{6}{c}{DV: within-country regional disparity} \\ 
 & (a) & (b) & (c) & (d) & (e) & (f) \\ 
\hline \\[-1.8ex] 
 unequal edu & 0.33$^{**}$ & 0.15 & 0.18 & 0.26 & 0.13 & 0.15 \\ 
  & (0.01, 0.64) & ($-$0.17, 0.46) & ($-$0.12, 0.47) & ($-$0.10, 0.61) & ($-$0.22, 0.48) & ($-$0.19, 0.50) \\ 
  & & & & & & \\ 
 Gini & $-$0.24$^{**}$ & $-$0.28$^{***}$ & $-$0.28$^{***}$ & $-$0.37$^{***}$ & $-$0.34$^{**}$ & $-$0.33$^{**}$ \\ 
  & ($-$0.44, $-$0.05) & ($-$0.49, $-$0.08) & ($-$0.48, $-$0.08) & ($-$0.63, $-$0.10) & ($-$0.61, $-$0.06) & ($-$0.60, $-$0.06) \\ 
  & & & & & & \\ 
 \% Internet (all) & 0.35$^{**}$ &  &  & 0.40$^{**}$ &  &  \\ 
  & (0.04, 0.67) &  &  & (0.02, 0.77) &  &  \\ 
  & & & & & & \\ 
 education (all) & $-$0.31$^{**}$ & $-$0.20 & $-$0.20 & $-$0.34$^{**}$ & $-$0.24 & $-$0.27$^{*}$ \\ 
  & ($-$0.57, $-$0.05) & ($-$0.49, 0.08) & ($-$0.46, 0.05) & ($-$0.61, $-$0.06) & ($-$0.54, 0.07) & ($-$0.55, $-$0.003) \\ 
  & & & & & & \\ 
 income &  & $-$0.01 &  &  & $-$0.06 &  \\ 
  &  & ($-$0.32, 0.31) &  &  & ($-$0.45, 0.33) &  \\ 
  & & & & & & \\ 
 civic (all) &  &  & 0.08 &  &  & 0.17 \\ 
  &  &  & ($-$0.15, 0.32) &  &  & ($-$0.11, 0.45) \\ 
  & & & & & & \\ 
 Central/Southern/Eastern Asia &  &  &  & $-$0.23 & $-$0.03 & 0.13 \\ 
  &  &  &  & ($-$1.14, 0.67) & ($-$0.96, 0.89) & ($-$0.83, 1.08) \\ 
  & & & & & & \\ 
 Europe/Oceania/Northern America &  &  &  & $-$0.46 & 0.11 & 0.02 \\ 
  &  &  &  & ($-$1.44, 0.52) & ($-$0.94, 1.17) & ($-$0.86, 0.90) \\ 
  & & & & & & \\ 
 Latin America \& the Caribbean &  &  &  & 0.19 & 0.47 & 0.51 \\ 
  &  &  &  & ($-$0.43, 0.82) & ($-$0.16, 1.10) & ($-$0.09, 1.12) \\ 
  & & & & & & \\ 
 Northern Africa \& Western Asia &  &  &  & $-$0.17 & 0.22 & 0.39 \\ 
  &  &  &  & ($-$0.99, 0.64) & ($-$0.60, 1.05) & ($-$0.45, 1.22) \\ 
  & & & & & & \\ 
 Constant & 0.002 & 0.001 & $-$0.02 & 0.10 & $-$0.17 & $-$0.23 \\ 
  & ($-$0.19, 0.19) & ($-$0.20, 0.20) & ($-$0.22, 0.19) & ($-$0.40, 0.59) & ($-$0.68, 0.34) & ($-$0.70, 0.25) \\ 
  & & & & & & \\ 
\hline \\[-1.8ex] 
Out-of-sample RMSE & 0.91 & 0.92 & 0.92 & 0.95 & 0.96 & 0.95 \\ 
Out-of-sample R$^{2}$ & 0.12 & 0.09 & 0.1 & 0.08 & 0.05 & 0.05 \\ 
Observations & 79 & 79 & 79 & 79 & 79 & 79 \\ 
R$^{2}$ & 0.23 & 0.18 & 0.18 & 0.25 & 0.21 & 0.22 \\ 
Adjusted R$^{2}$ & 0.18 & 0.13 & 0.14 & 0.17 & 0.12 & 0.13 \\ 
AIC & 207.48 & 212.39 & 211.87 & 212.54 & 217.18 & 215.68 \\ 
BIC & 221.70 & 226.61 & 226.09 & 236.23 & 240.87 & 239.37 \\ 
Residual Std. Error & 0.86 (df = 74) & 0.89 (df = 74) & 0.89 (df = 74) & 0.87 (df = 70) & 0.90 (df = 70) & 0.89 (df = 70) \\ 
F Statistic & 5.38$^{***}$ (df = 4; 74) & 3.94$^{***}$ (df = 4; 74) & 4.09$^{***}$ (df = 4; 74) & 2.97$^{***}$ (df = 8; 70) & 2.30$^{**}$ (df = 8; 70) & 2.52$^{**}$ (df = 8; 70) \\ 
\hline 
\hline \\[-1.8ex] 
\textit{Note:}  & \multicolumn{6}{r}{$^{*}$p$<$0.1; $^{**}$p$<$0.05; $^{***}$p$<$0.01} \\ 
\end{tabular} }
\end{table*} 

% \input{stabs/tab_covid_regr}
\clearpage
