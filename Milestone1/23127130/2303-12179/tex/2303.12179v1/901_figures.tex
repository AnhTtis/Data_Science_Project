\addcontentsline{toc}{section}{Supplementary Figures}
% \section*{Supplementary Figures}\label{sec:supp_figs}

\begin{figure*}
    \centering
    \includegraphics[width=.95\linewidth]{figS01_word_curves.pdf}
    \caption{Determining the ``\bigw'' range using curvature and knee points detection. In half of the studied languages, the ranks of the \Bigword range between 5000 and 9000. Others (zh, it, ru, tr, nl, and ms) have wider or narrower ranges.}
    \label{fig:word_curve}
\end{figure*}

\begin{figure*}
    \centering
    \includegraphics[width=.95\linewidth]{figS02_countries_covered_fbUsers.pdf}
    \caption{Countries covered in our estimation. There are 167 countries in 12 different languages, including 147 countries with a single or dominant language and 20 multilingual countries. For a multilingual country (having multiple official languages), we use the most used language of the country to estimate its language literacy.
    (A) The number of countries in each language. (B) The total population in each language. (C) The Facebook user count in each language, according to publicly available information about Facebook penetration statistics in 2022 \cite{Facebook15:online}. The populations from the 20 multilingual countries are excluded in (B) and (C) because we do not have the sub-population estimates of different languages within these multilingual countries. 
    }
    \label{fig:countries_covered}
\end{figure*}

\begin{figure*}
    \centering
    % \includegraphics[width=\linewidth]{sfigs/corr-lit-edu.pdf}
    \includegraphics[width=.95\linewidth]{figS03_eval_olle_3x2_CI.pdf}
    \caption{The literacy estimate ($x$-axis) obtained from Models (a,b,c) listed in Table \ref{tab:model-lit-main}, compared with the reported literacy rate and the education in terms of schooling years ($y$-axis). The reported literacy rate was transformed for normality. 
(A,B) Literacy estimate adjusted by Model (a).
(C,D) Literacy estimate adjusted by Model (b).
(E,F) Literacy estimate adjusted by Model (c).}
    \label{fig:corr_lit_edu}
\end{figure*}

\begin{figure*}
    \centering
    % \includegraphics[width=\linewidth]{sfigs/corr-lit-visual-v2}
    \includegraphics[width=.95\linewidth]{figS04_corr_lit_vs_visual_CI}
    \caption{Relationship between countries' \olles ($x$-axes) and the relative visual time-spent ($y$-axes), where the relative visual time-spent is given as the proportion of time spent on photos and videos relative to the time on news feeds, photos and videos combined. Correlations are reported based on Spearman's rank correlation. }
    \label{fig:lit_visual}
\end{figure*}

\begin{figure*}
    \centering
    \includegraphics[width=\linewidth]{figS05_map-gender}
    \caption{Gender differences in online language literacy across the world. (A) Summary of the standardized female-male differences across the seven geographical groups. Dashed line marks the global average (0.345, population-weighted), and a diamond indicates the population-weighted mean of the group. (B) Map of female-male differences available in our dataset.}
    \label{fig:gender_map}
\end{figure*}

\begin{figure*}
    \centering
    \includegraphics[width=\linewidth]{figS06_map-regional}
    \caption{Within-country regional disparity in online language literacy across the world. (A) Summary of the within-country regional disparity across the seven geographical groups. Dashed line marks the global average (0.032, population-weighted), and a diamond indicates the population-weighted mean of the group. (B) Map of country-level regional disparity available in our dataset. (C) Countries with a higher or lower level of regional disparity from each geographical group.}
    \label{fig:region_map}
\end{figure*}

% \input{sfigs/india/fig_india}

\begin{figure*}
    \centering
    % \includegraphics[width=.9\linewidth]{sfigs/india/IN_region_lit_6lang_eval.png}
    \includegraphics[width=.9\linewidth]{figS07_IN_region_lit_corr}
    \includegraphics[width=.9\linewidth]{figS07b_IN_region_lit_corr}
    \caption{A multilingual country case study with India's user population. We include the six most used languages in India (English, Hindi, Bengali, Marathi, Telugu, and Tamil) to compare the literacy estimation based on the single representative language (English) with the estimation based on multiple languages. The results suggest that literacy estimation based on multiple languages has neglectable improvement over English-based estimation in terms of Spearman's rank correlations. The y-axis indicates the officially reported literacy level, and the x-axes indicate (A) the language literacy estimated using the regions' public posts in English, (B) the estimation using multiple languages combined, and (C-E) the estimation using posts in Hindi, Bengali, and Marathi only, respectively.}
    \label{fig:india}
\end{figure*}

\begin{figure*}
    \centering
    \includegraphics[width=.9\linewidth]{figS08_multi_lang_dominant_perc}
    \caption{The relationship between the user percentage of the dominant language ($y$-axis) and (A) \olle, or (B) regional disparity, in 13 multilingual countries studied. We do not observe a clear systematic bias. This serves as a robust check to see whether there is a potential bias in the set of multilingual countries. }
    \label{fig:domi}
\end{figure*}