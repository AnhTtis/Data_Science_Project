\begin{figure*}[t]
\centering
    \includegraphics[width=.9\linewidth]{fig01_anno_v}
    \caption{
    {\bf Creating online language literacy estimate.}
Our methodology produces online language literacy estimate (\olle) through three major steps: 1. (A-C) The set of \bigword (lower-frequency frequent words) is algorithmically determined based on the vocabulary popularity in the language corpus; the red bands in (A-C) indicate the selected sets of \bigword in the three most widely used languages in our data (English, Spanish, Arabic). 2. (D-F)
Normalized total occurrence of \bigword in Facebook dataset from each country is used as a language-specific online literacy estimate for that country. (D-F) show the strong correlations found between our estimates and countries' officially reported literacy rates in English, Spanish, and Arabic, respectively. 3. (G) The calibrated global estimates, \olle, are generated after addressing language group bias and shown here with a strong correlation with reported literacy rates (Spearman's rank correlation coefficient $\rho=0.78$, \yrepj{95\% CI [0.69, 0.84]}, $p<0.001$.) Error bounds represent the 95\% confidence intervals. 
\yrepj{In (D-G), each dot represents a country, with $x$ value indicating the country's raw (D-F) or calibrated (G) literacy estimate and $y$ value the country's officially reported literacy rate. }
%Language-specific estimate of the relative occurrence of \bigword is calculated from Facebook user population in each country. Strong correlations are found between such a direct estimate and countries' officially reported literacy rates in the three largest language groups. (G) The global estimate, \olle, is generated after calibrating by language-group bias. The result indicates a substantial correlation between \olle and the reported literacy rates in terms of Spearman's rank correlation coefficient ($\rho$).
}
    \label{fig:eval}
\end{figure*}
