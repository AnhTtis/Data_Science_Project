\addcontentsline{toc}{subsection}{E.}
\subsection{Robust check: regional disparity and language dominance}\label{sec:domi}

\yrepj{Our \olle is created based on a country's representative language, i.e., the language used by the most Facebook users in the country. One potential risk of relying on a single representative language is that the regional disparity measure in a multilingual country may simply capture the distribution of languages, rather than the diversity of language skills. To test this, we examine the relationship between the user percentage of the representative language in a multilingual country and the country's regional disparity measure. Among the 167 countries studied, there are 20 multilingual countries, but only 13 meet the criteria to have a regional disparity measure. Recall in Section~\ref{sec:procedure} that each geographically bounded community (i.e., in this case, a region within a country) with at least 1000 active users observed in the study period. For these 13 multilingual countries, we plot the countries' percentage of Facebook users using the representative language on the $y$-axis and on the $x$-axis, either (A) \olle, or (B) regional disparity as shown in Fig.~\ref{fig:domi}. If there is a systematic bias, e.g., countries with low representative language user percentages tend to have high a regional disparity measure, we would see a trend in such a plot. However, we do not observe a clear systematic bias. While our sample size is limited, this analysis is helpful for checking whether there is a potential bias in the small sample of multilingual countries.} 
