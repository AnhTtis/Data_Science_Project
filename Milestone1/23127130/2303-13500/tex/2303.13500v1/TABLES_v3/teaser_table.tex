\begin{table*}[ht]
\setlength\tabcolsep{5pt}
\renewcommand{\arraystretch}{1.2}
\small
\centering
\resizebox{\columnwidth}{!}{%
\begin{tabular}{c | c | ccccccccccc } 

 \multicolumn{1}{c}{} &\multicolumn{1}{c}{} & \multicolumn{2}{c}{\hlb{\normalsize Generalization}} & \multicolumn{3}{c}{\hlb{\normalsize Robustness}} & \multicolumn{4}{c}{\hlb{\normalsize Calibration}} & \multicolumn{1}{c}{\hlb{\normalsize Anomaly Det.}} & \multicolumn{1}{c}{\hlb{\normalsize Rep. Similarity}} \\ \cmidrule(lr){1-1} \cmidrule(lr){2-2} \cmidrule(lr){3-4} \cmidrule(lr){5-7}  \cmidrule(lr){8-11}  \cmidrule(lr){12-12} \cmidrule(lr){13-13} 
 
\multirow{2}*{Dataset}& \multirow{2}*{Protocol} &  ID & OOD\footnote{For CIFAR10, we show the OOD accuracy for (CIFAR10.1/STL10)} & C & $\overline{\text{C}}$\footnote{For Domainnet, we use compute the 15 naturalistic corruptions on the "Real" environment, instead of applying $\bar{C}$}.& Adv. & ID & C & $\overline{\text{C}}$ & OOD. & Out-of-Class & ID \\
& & \textcolor{gray}{Acc.} & \textcolor{gray}{Acc.} & \textcolor{gray}{Acc.} & \textcolor{gray}{Acc.} & \textcolor{gray}{Acc.} & \textcolor{gray}{1-RMS} &  \textcolor{gray}{1-RMS} & \textcolor{gray}{1-RMS} & \textcolor{gray}{1-RMS} & \textcolor{gray}{AUROC}  & \textcolor{gray}{CKA}  \\ 
\hline
CIFAR10&\lp & 0.9138 & 0.8188/0.8192 & {0.6912} & 0.6553 & 0.0003  & \underline{0.95945} & \underline{0.83025} & \underline{0.8142} & 0.8696 & 0.6206 & 1.0000 \\
CIFAR10&\ft & \textbf{0.9539} & \textbf{0.8962}/\underline{0.8545} & \textbf{0.7434} & \textbf{0.7553} & \textbf{0.0231} & \textbf{0.9668} & \textbf{0.83635} & \textbf{0.8453} & \textbf{0.9232} & \textbf{1.0000} & 0.6831\\
CIFAR10&\lpft & \underline{0.9442} & \underline{0.8775}/\textbf{0.8581} & \underline{0.6921} & \underline{0.6790} & \underline{0.0018} & 0.9521 & 0.7849  & 0.7721 & \underline{0.88633} & \underline{0.6511} & 0.7853 \\
\hline
DomainNet&\lp & \underline{0.8913} & \textbf{0.8013} & \underline{0.6019} & \textbf{0.6020} & 0.1768 & \textbf{0.9638} & \textbf{0.9045} & \textbf{0.8571}& \textbf{0.9264} & 0.8679 & 1.0000\\
DomainNet&\ft & 0.7613 & 0.4522  & 0.5186 & 0.2744 & \textbf{0.4164} & 0.8368 & 0.7234 & 0.7234
 & 0.6379 & \underline{0.8841} & 0.6092\\
DomainNet&\lpft & \textbf{0.8985} & \underline{0.7990} & \textbf{0.6343} & \underline{0.5979} & \underline{0.1927} & \underline{0.9566} & \underline{0.8445} & \underline{0.8445} & \underline{0.8899} & \textbf{0.9022} & 0.9222\\
\hline
Living17&\lp & \underline{0.9521} & \underline{0.8124} & \underline{0.7010} & \underline{0.7377} & \textbf{0.2350} & \underline{0.9313} & 0.8693 & \underline{0.8801} & \underline{0.9117} & \underline{0.9907} & 1.0000\\
Living17&\ft & 0.9518 & 0.7168 & \underline{0.7011} & 0.7164 & 0.1563 & 0.8873 & \underline{0.9019} & 0.8604 & \textbf{0.9295} & 0.9794 & 0.7847 \\
Living17&\lpft & \textbf{0.9643} & \textbf{0.8261} & \textbf{0.7426} & \textbf{0.7671} & \underline{0.2135} & \textbf{0.9782} & \textbf{0.9472} & \textbf{0.9451} & 0.8742 & \textbf{0.9924} & 0.9887\\
\hline
\end{tabular}%
}
\caption{\textbf{Safety and Generalization Performance of Adaptation Protocols.} (Best in \textbf{bold}. Second best \underline{underlined}.) While \lpft~indeed achieves strong ID and OOD performance across datasets, we see that different protocols may perform better when safety evaluation is also considered. For CIFAR-10, which requires the most distortion as evidenced by lower CKA scores, we see that \ft~is the most effective; \lpft~and \lp are most effective, respectively, on Living17 and DomainNet, which require significantly less distortion. This suggests that, while mitigating feature distortion is effective for improving generalization, it is not always sufficient for also improving safety. } 
% \vspace{-10pt}
\label{tab:teaser_table}
\end{table*}