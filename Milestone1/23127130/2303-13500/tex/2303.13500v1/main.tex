
\documentclass{article} % For LaTeX2e
\usepackage{iclr2023_conference,times}
  
% Optional math commands from https://github.com/goodfeli/dlbook_notation.
\newcommand{\bbox}{\text{bbox}}
\newcommand{\alphapck}{\alpha_\bbox}
\newcommand{\kcycle}{\text{k-CyPCK}}
\newcommand{\cycle}{\text{-CyPCK}}

\newcommand{\I}{\mathbf{I}}
\newcommand{\Ia}{\I^\text{a}}
\newcommand{\Ib}{\I^\text{b}}
\newcommand{\Iatob}{\I^\text{a $\rightarrow$ b}}
\newcommand{\F}{\mathbf{F}}
\newcommand{\Fa}{\F^\text{a}}
\newcommand{\Fb}{\F^\text{b}}
\newcommand{\f}{\mathbf{f}}
\newcommand{\fa}{\f^\text{a}}
\newcommand{\fb}{\f^\text{b}}
\newcommand{\p}{\mathbf{p}}
\newcommand{\pa}{\p^\text{a}}
\newcommand{\pb}{\p^\text{b}}
\newcommand{\A}{\boldsymbol{\Phi}_\text{align}}
\newcommand{\G}{\mathbf{G}}
\newcommand{\C}{\mathbf{C}}
\newcommand{\Ca}{\C^\text{a}}
\newcommand{\Cb}{\C^\text{b}}
\newcommand{\cc}{\mathbf{c}}
\newcommand{\cca}{\cc^\text{a}}
\newcommand{\ccb}{\cc^\text{b}}
\newcommand{\Irec}{\I_\text{Recon}}
\newcommand{\M}{\mathbf{M}}
\newcommand{\Mrec}{\M_\text{Recon}}
\newcommand{\loss}{\mathcal{L}}
\newcommand{\T}{\mathcal{T}}
\newcommand{\W}{\mathcal{W}}
\newcommand{\Id}{\mathcal{I}}


\usepackage{hyperref}
\hypersetup{
    colorlinks,
    linkcolor={red!50!black},
    citecolor={blue!50!black},
    urlcolor={blue!80!black}
}
\usepackage{url}
\usepackage{amsmath}
\usepackage{amssymb}
\usepackage{mathtools}
\usepackage{amsthm,multirow,xcolor}
\usepackage{wrapfig,soul} %subfig
\usepackage{adjustbox}
\usepackage{microtype}
\usepackage{graphicx}
\usepackage{subcaption}
% \usepackage{colortbl}
% \usepackage[table]{xcolor}
% \usepackage{nicematrix}
\usepackage{booktabs}

\usepackage{enumitem}
\colorlet{sb}{cyan!30}
\DeclareRobustCommand{\hlb}[1]{{\sethlcolor{sb}\hl{#1}}}
\DeclareRobustCommand{\hlb}[1]{{\textit{\textbf{#1}}}}
% \title{Exploring the Design of Adaptation Protocols for Improved Generalization and Machine Learning Safety}

\title{A Closer Look at Model Adaptation using Feature Distortion and Simplicity Bias} 
% Authors must not appear in the submitted version. They should be hidden
% as long as the \iclrfinalcopy macro remains commented out below.
% Non-anonymous submissions will be rejected without review.

\author{Puja Trivedi \\ 
CSE Department \\
University of Michigan\\
\texttt{pujat@umich.edu} \\
\And
Danai Koutra \\ 
CSE Department \\
University of Michigan\\
\texttt{dkoutra@umich.edu} \\
\And
Jayaraman J. Thiagarajan\\
Center of Applied Scientific Computing \\
Lawrence Livermore Natl. Laboratory \\
\texttt{jjayaram@llnl.gov} \\
}

% The \author macro works with any number of authors. There are two commands
% used to separate the names and addresses of multiple authors: \And and \AND.
%
% Using \And between authors leaves it to \LaTeX{} to determine where to break
% the lines. Using \AND forces a linebreak at that point. So, if \LaTeX{}
% puts 3 of 4 authors names on the first line, and the last on the second
% line, try using \AND instead of \And before the third author name.

\newcommand{\fix}{\marginpar{FIX}}
\newcommand{\new}{\marginpar{NEW}}
\newcommand{\ft}{\texttt{FT}}
\newcommand{\lp}{\texttt{LP}}
\newcommand{\udp}{\texttt{LP(UDP)}}
\newcommand{\vat}{\texttt{LP(VAT)}}
\newcommand{\soup}{\texttt{LP(Soup)}}
%\newcommand{\lpft}{\texttt{LP+FT}}
\newcommand{\pt}[1]{\textcolor{blue}{#1}}
% \newcommand{\lpft}{\lp\texttt{+}~\ft}
\newcommand{\lpft}{\texttt{LP+FT}}
\iclrfinalcopy % Uncomment for camera-ready version, but NOT for submission.
\begin{document}


\maketitle
\begin{abstract}
Advances in the expressivity of pretrained models have increased interest in the design of adaptation protocols which enable safe \textit{and} effective transfer learning.
Going beyond conventional linear probing (LP) and fine tuning (FT) strategies, protocols that can effectively control feature distortion, i.e., the failure to update features orthogonal to the in-distribution, have been found to achieve improved out-of-distribution generalization (OOD). 
In order to limit this distortion, the LP+FT protocol, which first learns a linear probe and then uses this initialization for subsequent FT, was proposed.
However, in this paper, we find when adaptation protocols (LP, FT, LP+FT) are also evaluated on a variety of safety objectives (e.g., calibration, robustness, etc.), a complementary perspective to feature distortion is helpful to explain protocol behavior. To this end, we study the susceptibility of protocols to simplicity bias (SB), i.e. the well-known propensity of deep neural networks to rely upon simple features, as SB has recently been shown to underlie several problems in robust generalization. Using a synthetic dataset, we demonstrate the susceptibility of existing protocols to SB. Given the strong effectiveness of LP+FT, we then propose modified linear probes that help mitigate SB, and lead to better initializations for subsequent FT. We verify the effectiveness of the proposed LP+FT variants for decreasing SB in a controlled setting, and their ability to improve OOD generalization and safety on three adaptation datasets.\let\thefootnote\relax\footnote{Correspondence to \texttt{pujat@umich.edu}.} 
\end{abstract}

\section{Introduction}

\section{Introduction}\label{sec:intro}

Literacy, the ability to comprehend and produce textual information, is known as the foundation for many important personal and social functions. For individuals, the lack of literacy skills is associated with reduced access to education~\cite{NCES:2002,kutner2007literacy,schutz2008education}, employment~\cite{NCES:2002,NCES:2007,kutner2007literacy,ferrer2006effect,bonikowska2008literacy}, social benefits~\cite{schwerdt2018literacy}, as well as poorer health outcomes~\cite{dewalt2005health,OECD:2013} and lower civic engagement \cite{NCES:2002,NCES:2007,OECD:2013,gerger2008}. Collectively, literacy is considered a prerequisite for democracy and socioeconomic development \cite{bonikowska2008literacy,gerger2008}. 

\swepj{Despite a substantial
increase in global literacy rates over recent decades, there were still 750 million adults – two-thirds of whom were women – remaining illiterate in 2016~\cite{unesco2017}. The rise of digital communication technology has brought new challenges to those with limited literacy skills: as more and more public, professional, and social communications shift to the digital, text-mediated environment, a lack of literacy skills can not only exclude people from the information and resources available online but also expose them to greater (mis)informational vulnerability and harms~\cite{mundial2016education,bach2018poverty}. 
%\yrvv{Today, literacy is more essential than ever; it is now a fundamental requirement of communication in an increasingly digital, text-mediated world. Many social and economic activities, services by governments and businesses, as well as information and resources are increasingly available online, but those with poor literacy skills are unable to take advantage of these opportunities \cite{mundial2016education}. 
With most existing literacy programs and research focusing on school children and educational settings, we see a significant gap in our understanding of \emph{literacy practices and challenges in the digital environment}. 
%As of 2019, social media platforms are used by one-in-three people in the world, and more than two-thirds of all Internet users \cite{Theriseo30:online} -- digitally mediated communications are now an ordinary part of many people's lives.}
%To fill in this gap, we need to first characterize the range of \textit{language literacy skills across online populations}. 
In this study, we take a data-driven approach, leveraging the data available on Facebook -- the most widely adopted social media platform with a third of the world's population using it regularly~\cite{Meta_2022Q3_report} - to obtain a representative and up-to-date sample of literacy activities (e.g. reading and writing textual content) by the global online population.}


\swepj{
This study systemically examines the \textit{language literacy skills of online populations} (henceforth called  \textit{online language literacy}) for more than 160 countries and regions around the globe. We introduce a new population-level measure called {\it online language literacy estimate} ({\it \olle}) that is based on aggregated and de-identified written content posted publicly on Facebook. Thanks to the reach of Facebook to hundreds of millions of active users from low-resourced regions such as Africa, Latin American, and South East Asia~\cite{Meta_2022Q3_earnings}, our measure is able to estimate and track population-level language literacy at an unprecedented level of coverage, resolution, and timeliness comparing to traditional literacy assessment methods~\cite{rammstedt2016introduction}, while achieving an overall strong correlation with available official data. With \olle calculated for different gender, country, and regional groups across the world, we capture the disparities in online literacy across broad geographical areas and explore gender and regional literacy gaps under a diverse set of societal contexts. Our results not only quantify the association between online language literacy gaps and offline inequality metrics, but also uncover the complex interaction between literacy, Internet adoption, and civic participation for women. In summary, the main contributions of our work are:}
\swepj{
\begin{itemize}
    \item We develop a global online language literacy estimate (\olle) using Facebook data from over 160 countries in 12 languages. 
    \item We evaluate our measure with existing offline population literacy benchmarks, showing a strong correlation and broader coverage than current official data.
    \item We demonstrate how the online language literacy measure can be used to track gender and regional literacy gaps and unpack the complex societal context around literacy and literacy development.
\end{itemize}
}
\swepj{
The rest of the paper is structured as follows: Sec.~\ref{sec:related-work} offers a literature review of related work to contextualize our study. Sec.~\ref{sec:method} describes our methodology and the dataset used for developing the online language literacy estimate (\olle). Sec.~\ref{sec:results} validates the resulting \olle's with existing literacy assessment data and presents an overview of online language literacy skills across the world. We also share a few applications of \olle in studying and understanding regional and gender inequalities globally. Sec.~\ref{sec:discussion} discusses the implications and limitations of this study, and concludes our work. 
}


% \section*{OLLE: Online Language Literacy Estimate}\label{sec:olle-intro}


\section{Related Work and Background}\label{sec:related_work}
Here, we discuss the most relevant work on adaptation protocols and simplicity bias; we discuss additional related work in Sup. \ref{sup:relatedwork}.

\paragraph{Adaptation Protocols.}
For a comprehensive overview of transfer learning, please see the surveys of~\citet{zhuang21_transferlearningsurvey} and~\citet{Pan10_TransferSurvey}. Here, we discuss the works that are most relevant to our own. \citet{Kirichenko22_LastLayerRetrain} recently demonstrated that models are able to learn both core features and spurious features. However, classifiers can rely upon spurious features, harming performance on minority groups. To reduce the reliance on spurious features, they propose to retrain the classifier on a small amount of ``re-weighting" data, which allows the model to leverage the core features instead of the spurious features. Other modifications and heuristics have also been proposed to improve \ft's performance, including side-tuning~\citep{Zhang19_Sidetuning}, which tunes a small secondary network that is then combined with the original model, using larger/smaller learning rates for the classifier, as well as regularization-based methods \citep{Jiang20_SMART}. In this work, we focus on two popular and effective protocols, \lp~and \ft. We additionally study the \lpft~protocol as it is theoretically-grounded, does not require re-weighting data, is designed to exploit high-quality pre-trained representations and achieves SOTA OOD performance during adaptation. 

\paragraph{Simplicity Bias.}
It is well-known that DNNs demonstrate a bias toward simple, potentially less expressive features~\citep{brutzkus18_sgd,Soudry18_ImplicitBias,Gunasekar18,geirhos18_texturebias,Hermann20_OriginsTexture,Lubana23_ModeConnectivity}, such as textures and backgrounds, and that this bias can lead to shortcuts that limit the generalization of DNNs. Indeed, recently \citet{Shah20_SimplicityBias} formalized this intuition by more precisely defining simplicity bias, based on the number of linear components to define a decision boundary, and showed that SB leads to non-robust decision boundaries that affects a model's sensitivity to distribution shifts and adversarial perturbations. In brief, by learning simple features first, models become invariant to complex features, potentially leading to narrow decision boundaries which can fail to generalize under data shifts. Notably, DNNs exhibit this bias even when complex features are more expressive and necessary for fitting the distribution. While various techniques have recently been proposed to mitigate simplicity bias when training from scratch or in the context of pretraining \citep{Teney21_EvadingSB}, we are, to the best of our knowledge, the first to rigorously study the role of simplicity in the context of model adaptation. 


\section{Joint Analysis of Protocol Safety and Generalization}
In this section, we evaluate the performance of adaptation protocols across several additional safety objectives \citep{Hendrycks21_UnsolvedProblems}, as practical transfer learning applications require both strong and safe generalization. Through this expanded evaluation, we find that no single protocol is optimal across all safety objectives. Indeed, the inability of \lpft~to induce safe adaptation indicates that a complementary perspective to feature distortion, namely simplicity bias, is necessary when designing generalizable and safe protocols (see Sec. \ref{sec:simplicity}). We further argue that by constraining models around the \lp~initialization during \ft, \lpft~may inadvertently harm safety performance by hampering models' abilities to learn complex, task-specific features needed for robust generalization. While we expand upon the role of \lp~initialization in Secs. \ref{sec:simplicity} and \ref{sec:hardness}, we begin, here, by introducing the expanded evaluation and experimental setup.

\paragraph{Experimental Setup.} Three downstream adaptation tasks (and their respective OOD distributions) are considered: CIFAR-10 (ID)  $\rightarrow$ \{STL10, CIFAR10.1\} (OOD), Domainnet-Sketch $\rightarrow$ \{Domainnet-ClipArt, Domainnet-Painting, Domainnet-Real\} and Living17 (Source) $\rightarrow$ Living17 (Target). These datasets are selected as they correspond to two different types of distribution shifts (standard domain adaptation and subpopulation) and three levels of distortion (low, medium, high). A MoCo-V2 ResNet-50~\citep{He20_MoCo} pretrained on ImageNet-1K is used as the base-feature extractor for CIFAR10 and Living17 experiments, and the CLIP ResNet-50 image encoder pretrained on 400 million (image,text) pairs is used for Domainnet-Sketch. These models are selected as they provide sufficiently high-quality representations capable of generalizing to both ID and OOD downstream data \citep{Kumar22_FinetuningDistorts}. We perform grid-search to find the best hyper-parameters, and average over $3$ seeds. See Sup. \ref{sup:experimentaldetails} for additional details. 

\paragraph{Expanded Evaluation.} In addition to OOD accuracy on the aforementioned distribution shifts, we report performance on the following metrics in order to evaluate adapted models on key problems in machine learning safety~\citep{Hendrycks21_UnsolvedProblems}. Our evaluation setup is inspired by \citet{Hendrycks21_PixMix}:
\begin{itemize}[leftmargin=*]
    \item\textit{Mean Corruption Accuracy (mCA/m$\Bar{C}$A):} We consider two sets of corruptions: the $15$ naturalistic corruptions $(Corr)$~\citep{Hendrycks19_CIFAR10C}, and 10 perceptually dissimilar corruptions $(\overline{Corr})$~\citep{mintun21_cbar}. Corruptions are applied to the ID test dataset and the average accuracy over each set is reported.
    \item\textit{Calibration Error (RMSE)}: It is important that models are well-calibrated so that practitioners may trust the provided predictions in high-risk applications \cite{Guo17_Calibration}. We measure the root mean square error of calibration as follows: {\tiny$\sqrt{\mathbb{E}_{\mathrm{C}}\left[(\mathbb{P}(Y=\hat{Y} \mid \mathrm{C}=\mathrm{c})-\mathrm{c})^{2}\right]}$}, where {\small$\mathrm{C}$} indicates the confidence scores, while {\small$\hat{Y}$} and {\small$Y$} denote the model's predictions and ground-truth labels, respectively.
    \item\textit{Anomaly Detection Performance (AUROC):} Recognizing when samples are anomalous allows models to abstain from making uninformed and inapplicable predictions. We consider samples from Blobs, Gaussian, LSUN, Places69, Rademacher, Textures, and SVHN datasets as anomalies and report the AUROC (area under the ROC curve) of the binary classification problem of detecting such samples as anomalies. 
    \item\textit{Adversarial Accuracy:} DNNs are well-known to be fooled by imperceptible distortions \citep{ilyas19_bugs}. We use a 2/225, 10-step PGD \citep{Madry18_PGD} attack to measure the robustness of models to such perturbations.
\end{itemize}
 
We make the following observations regarding the behavior of different protocols using this expanded evaluation. In brief, we find that no single protocol is effective across all datasets in jointly obtaining strong and safe adaptation, and that, on low distortion adaptation tasks, the quality of the \lp~initialization is critical as pre-trained feature extractor is not substantially updated during \lpft. 

\subsection*{Observation 1: Mitigating Feature Distortion may not induce safe adaptation.}
\begin{wrapfigure}{rt}{0.34\textwidth}
\vspace{-0.2in}
\begin{center}
\includegraphics[width=0.34\textwidth]{FIGS_v3/sample_heatmap-10-vert.pdf}
\end{center}
\caption{\textbf{Disparate Performance of Protocols.} We plot the average rank of each protocol for different safety and generalization metrics. We see no single protocol achieves top rank across all metrics.}
\vspace{-0.65in}
\label{fig:ranking}
\end{wrapfigure}

Here, we ask how protocols perform when we consider both safety and generalization objectives to better understand the feature distortion perspective. In particular, if \lpft~is able to outperform \lp~and \ft~in this expanded evaluation, then it suggests that solely mitigating feature distortion during \ft~may be sufficient to induce robust adaptation. To test this claim, we rank protocol performance for each safety metric, where ranks are first computed for each dataset separately, and then averaged. Results are shown in Fig. \ref{fig:ranking}. Smaller ranks correspond to better performance.

\textit{Results.} \lpft~obtains the best rank for ID and OOD accuracy as expected, as well as $Corr$ and $\overline{Corr}$ accuracy. However, we also see that \ft~is better ranked for Adversarial Accuracy and OOD calibration, while \lp~is better ranked for ID calibration and $\overline{Corr}$ calibration. However, given that \lpft~trails behind protocols that are not explicitly designed to limit distortion on some safety metrics, it is clear that a complementary perspective is needed to better understand protocol behavior. Indeed, \lpft~has the best \textit{average} rank, indicating that it is a good starting point to improve upon. 

The above results are aggregated across different types of distribution shifts; we extend this analysis next by considering the interplay between individual datasets and protocol performance. These detailed results are presented in Table \ref{tab:teaser_table}.

\subsection*{Observation 2: Linear Probing Solutions Matter.} 

Naturally, the amount of distortion required to effectively adapt a pretrained model to a downstream task will vary in accordance to the similarity of the downstream and pretraining data. Here, we seek to understand how protocols behave under different levels of distortion. In particular, we hypothesize that the \lp~initialization becomes more influential for \lpft~in low distortion settings, as subsequent \ft~remains in the vicinity of initialization. To this end, we compute the batched centered kernel alignment (CKA) score~\citep{nguyen2020wide} with respect to the adapted and pretrained models, and take a closer look at performance across metrics. We note that while CKA is better suited for measuring distortion than the L2 norm as used by \cite{Kumar22_FinetuningDistorts}, other neural representation metrics can also be used \citep{Ding21_RepSimStatTest,Davari23_ReliabilityCKA}. 

\textit{Results.} As shown in Fig. \ref{fig:cka-bars}, we see that minimal distortion (CKA $\ge$ 0.9) is required to obtain competitive \lpft~performance on DomainNet and Living17. However, on CIFAR10, which requires the most distortion as evidenced by lower CKA scores, \ft~is the most effective protocol for safety measures and is very comparable on generalization performance (see Table \ref{tab:teaser_table}).

The effectiveness of \lp~and \lpft~on Living17 in improving OOD generalization over \ft~is hardly surprising, as Living17 is a subset of ImageNet, on which the base feature-encoder was already trained. 
\begin{wrapfigure}{l}{0.30\textwidth}
\begin{center}
% \vspace{-0.2in}
\includegraphics[width=0.30\textwidth]{FIGS_v3/sample_cka-7.pdf}
\end{center}
\caption{\textbf{Dataset Distortion.} We plot the CKA similarity between adapted and pretrained models. DomainNet and Living17 require low distortion, as seen by performance of \lpft~across metrics with high CKA ($>0.9$).} 
\label{fig:cka-bars}
\vspace{-0.2in}
\end{wrapfigure}
In contrast, on DomainNet, the difficulty of \ft~in matching the ID \textit{test} task performance, despite achieving high training accuracy, suggests \ft~may learn a solution that relies upon shortcuts (or simple features) that do not generalize. 
We emphasize that \lpft~greatly benefits from strong \lp~initializations on these low-distortion datasets as corresponding CKA scores show that very limited updates are made during \ft. While \lpft~does induce meaningful improvements over \lp~on Living17 and performs comparably to \lp~on DomainNet, we stress the model must be kept close to the \lp~initialization during \ft. Indeed, to obtain acceptable \lpft~performance, small learning rates (3e-7,1e-5) and frozen batch-norm parameters during \ft~are necessary.

\textit{Summary.} Taken jointly, these results suggest that while solely mitigating feature distortion may not be sufficient to ensure that adapted models perform well on safety metrics across different levels of shift, improving the \lp~initialization may be a viable solution to obtaining strong and safe generalization. Indeed, the effectiveness of \lpft~on low distortion datasets and its high average ranking indicates that it is a promising protocol to build upon. To understand how to build better protocols, we next introduce \textit{simplicity bias} as a complementary perspective to feature distortion. 

\begin{table*}[ht]
\setlength\tabcolsep{5pt}
\renewcommand{\arraystretch}{1.2}
\small
\centering
\resizebox{\columnwidth}{!}{%
\begin{tabular}{c | c | ccccccccccc } 

 \multicolumn{1}{c}{} &\multicolumn{1}{c}{} & \multicolumn{2}{c}{\hlb{\normalsize Generalization}} & \multicolumn{3}{c}{\hlb{\normalsize Robustness}} & \multicolumn{4}{c}{\hlb{\normalsize Calibration}} & \multicolumn{1}{c}{\hlb{\normalsize Anomaly Det.}} & \multicolumn{1}{c}{\hlb{\normalsize Rep. Similarity}} \\ \cmidrule(lr){1-1} \cmidrule(lr){2-2} \cmidrule(lr){3-4} \cmidrule(lr){5-7}  \cmidrule(lr){8-11}  \cmidrule(lr){12-12} \cmidrule(lr){13-13} 
 
\multirow{2}*{Dataset}& \multirow{2}*{Protocol} &  ID & OOD\footnote{For CIFAR10, we show the OOD accuracy for (CIFAR10.1/STL10)} & C & $\overline{\text{C}}$\footnote{For Domainnet, we use compute the 15 naturalistic corruptions on the "Real" environment, instead of applying $\bar{C}$}.& Adv. & ID & C & $\overline{\text{C}}$ & OOD. & Out-of-Class & ID \\
& & \textcolor{gray}{Acc.} & \textcolor{gray}{Acc.} & \textcolor{gray}{Acc.} & \textcolor{gray}{Acc.} & \textcolor{gray}{Acc.} & \textcolor{gray}{1-RMS} &  \textcolor{gray}{1-RMS} & \textcolor{gray}{1-RMS} & \textcolor{gray}{1-RMS} & \textcolor{gray}{AUROC}  & \textcolor{gray}{CKA}  \\ 
\hline
CIFAR10&\lp & 0.9138 & 0.8188/0.8192 & {0.6912} & 0.6553 & 0.0003  & \underline{0.95945} & \underline{0.83025} & \underline{0.8142} & 0.8696 & 0.6206 & 1.0000 \\
CIFAR10&\ft & \textbf{0.9539} & \textbf{0.8962}/\underline{0.8545} & \textbf{0.7434} & \textbf{0.7553} & \textbf{0.0231} & \textbf{0.9668} & \textbf{0.83635} & \textbf{0.8453} & \textbf{0.9232} & \textbf{1.0000} & 0.6831\\
CIFAR10&\lpft & \underline{0.9442} & \underline{0.8775}/\textbf{0.8581} & \underline{0.6921} & \underline{0.6790} & \underline{0.0018} & 0.9521 & 0.7849  & 0.7721 & \underline{0.88633} & \underline{0.6511} & 0.7853 \\
\hline
DomainNet&\lp & \underline{0.8913} & \textbf{0.8013} & \underline{0.6019} & \textbf{0.6020} & 0.1768 & \textbf{0.9638} & \textbf{0.9045} & \textbf{0.8571}& \textbf{0.9264} & 0.8679 & 1.0000\\
DomainNet&\ft & 0.7613 & 0.4522  & 0.5186 & 0.2744 & \textbf{0.4164} & 0.8368 & 0.7234 & 0.7234
 & 0.6379 & \underline{0.8841} & 0.6092\\
DomainNet&\lpft & \textbf{0.8985} & \underline{0.7990} & \textbf{0.6343} & \underline{0.5979} & \underline{0.1927} & \underline{0.9566} & \underline{0.8445} & \underline{0.8445} & \underline{0.8899} & \textbf{0.9022} & 0.9222\\
\hline
Living17&\lp & \underline{0.9521} & \underline{0.8124} & \underline{0.7010} & \underline{0.7377} & \textbf{0.2350} & \underline{0.9313} & 0.8693 & \underline{0.8801} & \underline{0.9117} & \underline{0.9907} & 1.0000\\
Living17&\ft & 0.9518 & 0.7168 & \underline{0.7011} & 0.7164 & 0.1563 & 0.8873 & \underline{0.9019} & 0.8604 & \textbf{0.9295} & 0.9794 & 0.7847 \\
Living17&\lpft & \textbf{0.9643} & \textbf{0.8261} & \textbf{0.7426} & \textbf{0.7671} & \underline{0.2135} & \textbf{0.9782} & \textbf{0.9472} & \textbf{0.9451} & 0.8742 & \textbf{0.9924} & 0.9887\\
\hline
\end{tabular}%
}
\caption{\textbf{Safety and Generalization Performance of Adaptation Protocols.} (Best in \textbf{bold}. Second best \underline{underlined}.) While \lpft~indeed achieves strong ID and OOD performance across datasets, we see that different protocols may perform better when safety evaluation is also considered. For CIFAR-10, which requires the most distortion as evidenced by lower CKA scores, we see that \ft~is the most effective; \lpft~and \lp are most effective, respectively, on Living17 and DomainNet, which require significantly less distortion. This suggests that, while mitigating feature distortion is effective for improving generalization, it is not always sufficient for also improving safety. } 
% \vspace{-10pt}
\label{tab:teaser_table}
\end{table*}\label{sec:tradeoffs}

\section{Mitigating Simplicity Bias \& Feature Distortion for Safe Adaptation}\label{sec:simplicity}
In the preceding section, we found that training on compromised data may in fact decrease the GEP error. We hypothesize that this decrease can be partially attributed to the noisy datasets, mitigating \textit{simplicity bias}, i.e., the well known propensity of deep neural networks to rely upon simple, spurious features in lieu of more complex/expressive ones~\cite{brutzkus18_sgd,Gunasekar18,Shah20_SimplicityBias,geirhos18_texturebias}. Given that simple features are not expected to generalize on \textit{o.o.d} datasets, but DNNs remain susceptible to relying upon such simple features, we posit that GEPs will also see decreased performance on distributions where simple features are no longer indicative of the label. We test this hypothesis using a synthetic setting that controls the discriminability of simple features on target datasets, as discussed below.
\\
\noindent \textit{Experimental Setup.} We use a custom ``dominoes" dataset \cite{Shah20_SimplicityBias} of complex and simple features by pairing each class from CIFAR10 (complex feature) with the corresponding digit class in MNIST (simple feature) \cite{Trivedi22_Adaptation}. (See Fig. \ref{fig:cifar_mnist}). Three levels of correlation (95\%, 99\%, 100\%) between the target and simples features are considered during training.  When predicting generalization, we sample complex features from STL10, as well as create a variant that randomizes the spurious correlation between simple and complex features. We fine-tune a MoCo-V2 pretrained ResNet-50~\cite{He20_MoCo} for 20 epochs with lr=0.001 and average results over $3$ seeds.

\begin{figure}
    \centering
    \includegraphics[width = 0.75\columnwidth]{FIGS/cifar-mnist-3.png}
    \caption{\textbf{Simplicity Bias Dataset, (Fig. 2 \cite{Trivedi22_Adaptation}).} Dominoes comprised of complex (CIFAR10) and simple (MNIST) features are used to control the simplicity bias on target datasets.}
    \label{fig:cifar_mnist}
    \vspace{-0.3cm}
\end{figure}


As shown in Table. \ref{tab:simp_bias}, we see that the prediction error often substantially increases when evaluating on the randomized target dataset, e.g., where the simple feature is no longer predictive. While we would expect that the target accuracy decreases, the decreased GEP performance is particularly troubling as such methods are intended to detect these very failures. Moreover, we note that as the correlation between the simple and complex feature increases (Corr=0.95 vs. Corr=1.0), the gap between GEP's performance on the Corr. and Rand. variants of the target dataset increases. Indeed, the Corr. MAE decreases as the training dataset correlation increases (Corr=0.95 vs. 1.0), but the Rand. MAE increases. This result further highlights the harmful role of simplicity bias on GEP performance. 

\section{Evaluating Generalization and Safety of the \lpft~Family}\label{sec:hardness}
Given the effectiveness of incorporating hardness promoting (hp) augmentations with the family of \lpft~protocols (hp-\lpft) in mitigating simplicity bias in a synthetic setting, we further evaluate the modified protocols on the three real-world datasets (Living17, DomainNet, and CIFAR10) with respect to the generalization and safety metrics introduced in Sec. \ref{sec:tradeoffs}. We present our results in Tables \ref{tab:living17},\ref{tab:domainnet}, and \ref{tab:cifar10}); our observations are summarized below. Any method-specific hyperparameters (e.g., epsilon) are tuned using ID validation data and all results are reported over three seeds. We provide additional results in Supp. \ref{sup:additional_results}.

\textit{Results.} As discussed in Sec. \ref{sec:tradeoffs}, these three datasets represent scenarios where different levels of distortion are necessary when adapting the pretrained model. On Living17, a setting which requires minimal distortion during adaptation, we see that vanilla \lpft~is quite effective with respect to both generalization and safety metrics and is a difficult baseline to surpass. Indeed, while hp-\lpft~variants do not lead to significant benefits, they generally perform comparably to vanilla \lpft. On DomainNet, a setting where fairly low distortion is required for \lpft~but \ft~struggles to find a good solution, we see that hp-\lpft~variants induce some slight benefits with respect to ID/OOD generalization and robustness, though vanilla \lp~and hp-\lp~have better calibration performance. In contrast on CIFAR10, which requires more distortion to obtain an acceptable solution, we see that hp-\lpft~ variants lead to improved generalization and a noticeable boost in corruption robustness. \vat  + \ft~ and \vat~are particularly effective in this regard. Lastly, across all datasets, we observe that hp-\lpft~protocols lead to similar distortion to vanilla \lpft, which suggests that any additional benefits of hp-\lpft~should not be attributed to only reducing feature distortion. 

\begin{table}[!t]
\setlength\tabcolsep{5pt}
\renewcommand{\arraystretch}{1.3}
\small
\centering
\resizebox{\columnwidth}{!}{%
\begin{tabular}{c | ccccccccccc } 
\multicolumn{1}{c}{}& \multicolumn{2}{c}{\hlb{\normalsize Generalization}} & \multicolumn{3}{c}{\hlb{\normalsize Robustness}} & \multicolumn{4}{c}{\hlb{\normalsize Calibration}} & \multicolumn{1}{c}{\hlb{\normalsize Anomaly Det.}} & \multicolumn{1}{c}{\hlb{\normalsize Rep. Similarity}} \\ \cmidrule(lr){1-1} \cmidrule(lr){2-3} \cmidrule(lr){4-6}  \cmidrule(lr){7-10}  \cmidrule(lr){11-11} \cmidrule(lr){12-12} 

\multirow{2}*{Protocol} &  ID & OOD & C & $\overline{\text{C}}$ & Adv. & ID & C & $\overline{\text{C}}$ & OOD. & Out-of-Class & ID \\
& \textcolor{gray}{Acc.} & \textcolor{gray}{Acc.} & \textcolor{gray}{Acc.} & \textcolor{gray}{Acc.} & \textcolor{gray}{Acc.} & \textcolor{gray}{1-RMS} &  \textcolor{gray}{1-RMS} & \textcolor{gray}{1-RMS} & \textcolor{gray}{1-RMS} & \textcolor{gray}{AUROC}  & \textcolor{gray}{CKA}  \\ 
\hline

\lp & 0.9521 & 0.8124 & 0.7010 & 0.7378 & {0.2350} & 0.9313 & 0.8693 & 0.8802 & {0.9117} & 0.9907 & 1.0000 \\
\ft                    & 0.9518 & 0.7168 & 0.7011 & 0.7164 & 0.1563 & 0.8873 & 0.9019 & 0.8604 & \textbf{0.9295} & 0.9794 & 0.7847 \\
\lp+\ft                 & \underline{0.9643} & \ul{0.8261} & 0.7426 & {0.7671} & 0.2135 & \textbf{0.9782} & 0.9472 & 0.9451 & 0.8742 & {0.9924} & 0.9887 \\
\hline
\udp              & 0.9524	& 0.8110 & 	0.7017	& 0.7382	& \ul{0.2353}	& 0.9308& 	0.8691& 	0.8801	& \ul{0.9118}	& 0.9908	& 1.0000 \\
\vat              & 0.9524	& 0.8122	& 0.7010	& 0.7379	& 0.2345	& 0.9299& 	0.8682	& 0.8791	& 0.9103	& 0.9907	& 1.0000 \\
\soup             & 0.9439	& 0.7996	& 0.6874	& 0.7290	& \textbf{0.2451}	& 0.8806	& 0.7868	& 0.8094	& 0.9064	& 0.9897	& 1.0000 \\
\hline
\udp  + \ft       & {0.9637}	& \textbf{0.8265}	& \ul{0.7448}	& \ul{0.7681}	& 0.2157	& \ul{0.9768}	& 0.9464& 	\ul{0.9467}	& 0.8757	& \ul{0.9927}& 	0.98927 \\
\vat  + \ft       & \textbf{0.9647}	& {0.8247}	& 0.7425	& 0.7650	& 0.2224	& 0.9727	& \textbf{0.9521}	& {0.9463}	& 0.8775	& {0.9925}	& 0.9893 \\
\soup + \ft       & 0.9608	& 0.8163	& \textbf{0.7456}	& \textbf{0.7684}	& 0.1855	& 0.9760	& \ul{0.9498}	& \textbf{0.9492}	& 0.8678	& \textbf{0.9936}	& 0.98540 \\
\hline
\end{tabular}%
}
\caption{\textbf{Living17: Hardness Promoting Augmentation and Adaptation.} In this low-distortion adaptation setting, we see that vanilla \lpft~is an effective baseline and that hardness promoting variants of \lpft~tend to perform comparably.}
\vspace{-10pt}
\label{tab:living17}
\end{table}

\begin{table}[!t]
\setlength\tabcolsep{5pt}
\renewcommand{\arraystretch}{1.3}
\small
\centering
\resizebox{\columnwidth}{!}{%
\begin{tabular}{c | ccccccccccc } 
\multicolumn{1}{c}{}& \multicolumn{2}{c}{\hlb{\normalsize Generalization}} & \multicolumn{3}{c}{\hlb{\normalsize Robustness}} & \multicolumn{4}{c}{\hlb{\normalsize Calibration}} & \multicolumn{1}{c}{\hlb{\normalsize Anomaly Det.}} & \multicolumn{1}{c}{\hlb{\normalsize Rep. Similarity}} \\ \cmidrule(lr){1-1} \cmidrule(lr){2-3} \cmidrule(lr){4-6}  \cmidrule(lr){7-10}  \cmidrule(lr){11-11} \cmidrule(lr){12-12} 

\multirow{2}*{Protocol} &  ID & OOD & C & $\overline{\text{C}}$ & Adv. & ID & C & $\overline{\text{C}}$ & OOD. & Out-of-Class & ID \\
& \textcolor{gray}{Acc.} & \textcolor{gray}{Acc.} & \textcolor{gray}{Acc.} & \textcolor{gray}{Acc.} & \textcolor{gray}{Acc.} & \textcolor{gray}{1-RMS} &  \textcolor{gray}{1-RMS} & \textcolor{gray}{1-RMS} & \textcolor{gray}{1-RMS} & \textcolor{gray}{AUROC}  & \textcolor{gray}{CKA}  \\ 
\hline

\lp                    & 0.8913 & \ul{0.8013} & 0.6019 & 0.6020 & 0.1768 & 0.9638 & 0.9264 & \ul{0.9045} & {0.9014} & 0.8679 & 1.0000 \\
\ft                    & 0.7613 & 0.4522  & 0.5186 & 0.2744 & 0.4164 & 0.8368 & 0.7234 & 0.7234 & 0.6379 & 0.8841 & 0.6092\\
\lp + \ft              & 0.8985 & 0.7990 & 0.6343 & 0.5979 & 0.1927 & 0.9566 & 0.8445 & 0.8445 & 0.8899 & 0.9022 & 0.9222\\
\hline
\udp              & 0.8919 & \textbf{0.8021} & 0.6022 & 0.6101 & 0.1345 & 0.9635 & 0.9250 & \textbf{0.9047} & 0.8619 & 0.8714 & 1.0000 \\
\vat              & 0.8836	& 0.7914 & 0.5893&	0.5963&	0.1687&	0.8897&	\textbf{0.9552}&{0.8905}&	\textbf{0.9178}&	0.8735&	1.0000 \\
\soup             & 0.8787 & 0.7977 & 0.5951 &	0.6048 & 0.1731	& 0.8844 & \ul{0.9479} & 0.8861 &\ul{0.9176}	& 0.8661 &	1.0000 \\
\hline
\udp  + \ft       & {0.9033} &0.7965 & \ul{0.6414} & 	\textbf{0.6178} & 	0.1778 & 	0.9436 & 	0.8533 & 0.79415 & 	0.752 & 0.8857 & 	0.9662 \\
\vat  + \ft       & \ul{0.9048} &0.8009 & \textbf{0.6466} & 	\ul{0.6131} & 	\ul{0.1942} & 	\textbf{0.9686} & 	0.8911 & 0.8428	& 0.7985	& \textbf{0.9204} &	0.9370 \\
\soup + \ft       & \textbf{0.9051}	& \ul{0.8013} & 0.6393  & 	0.6091  & 	\textbf{0.1954}  & 	\ul{0.9670}  & 	0.9042 & 0.8692	 & 0.8246 &	\ul{0.9097}  & 0.9281 \\
\hline
\end{tabular}%
}
\caption{\textbf{DomainNet: Hardness Promoting Augmentations and Adaptation.} While relatively low distortion is induced by \lpft~, \ft~struggles to find a viable solution. Here, hardness-promoting \lpft~variants, particularly \vat+\ft~ do slightly improve ID and OOD generalization as well as robustness to corruptions.}
\label{tab:domainnet}
% \vspace{-10pt}
\end{table}
\textit{Discussion.} We find that while vanilla \lpft~is already an effective protocol, especially in settings where low distortion is required, hp-\lpft~can provide some benefits and performs competitively. We suspect that the performance of these modified protocols can further be improved if more sophisticated simplicity bias mitigation strategies are used. Indeed, our central claim, that adaptation protocols should mitigate feature distortion and simplicity, is not dependent on a specific strategy. We additionally note that while such mitigation strategies may optionally \textit{also} be used during \ft, they cannot \textit{solely} be used in \ft. Indeed, in the case of extreme simplicity, if the \lp~classifier relies upon simple features to find a low-loss solution, during the subsequent \ft~step, gradients may not be back propagated in directions that contain complex features. This entails that the decision boundary continues to rely upon simple features and is at risk of reduced safety performance. We provide further discussion in Supp.\ref{sup:ablation}. To this end, we recommend incorporating hardness-promoting augmentations during \lp~as a potential safe-guard to simplicity bias. 

\begin{table}[!t]
\setlength\tabcolsep{5pt}
\renewcommand{\arraystretch}{1.3}
\small
\centering
\resizebox{\columnwidth}{!}{%
\begin{tabular}{c | ccccccccccc } 
\multicolumn{1}{c}{}& \multicolumn{2}{c}{\hlb{\normalsize Generalization}} & \multicolumn{3}{c}{\hlb{\normalsize Robustness}} & \multicolumn{4}{c}{\hlb{\normalsize Calibration}} & \multicolumn{1}{c}{\hlb{\normalsize Anomaly Det.}} & \multicolumn{1}{c}{\hlb{\normalsize Rep. Similarity}} \\ \cmidrule(lr){1-1} \cmidrule(lr){2-3} \cmidrule(lr){4-6}  \cmidrule(lr){7-10}  \cmidrule(lr){11-11} \cmidrule(lr){12-12} 

\multirow{2}*{Protocol} &  ID & OOD & C & $\overline{\text{C}}$ & Adv. & ID & C & $\overline{\text{C}}$ & OOD. & Out-of-Class & ID \\
& \textcolor{gray}{Acc.} & \textcolor{gray}{Acc.} & \textcolor{gray}{Acc.} & \textcolor{gray}{Acc.} & \textcolor{gray}{Acc.} & \textcolor{gray}{1-RMS} &  \textcolor{gray}{1-RMS} & \textcolor{gray}{1-RMS} & \textcolor{gray}{1-RMS} & \textcolor{gray}{AUROC}  & \textcolor{gray}{CKA}  \\ 
\hline

\lp & 0.9138 & 0.8190 & 0.6912 & 0.6553 & 0.0003 & 0.9595 & 0.8303 & 0.8142 & 0.8696 & 0.6206 & 1.0000 \\
\ft & \ul{0.9539} & 0.8754 & \ul{0.7434} & \textbf{0.7553} &\textbf{0.0231} & 0.9668 & 0.8364 & \ul{0.8453} & 0.9232 & \textbf{1.0000} & 0.6831 \\
\lp+\ft & 0.9442 & 0.8678 & 0.6921 & 0.6790 & 0.0018 & 0.9521 & 0.7849 & 0.7721 & 0.8864 & 0.6511 & 0.7853 \\
\hline
\udp              & 0.9033&0.8356&0.6948&0.6643&0.0003&\textbf{0.9689}&0.9111&0.9023&0.9277&0.9033&1.0000 \\
\vat              & 0.8977&0.8251&0.6742&0.6483&0.0002&0.9265&\textbf{0.9255}&\textbf{0.9139}&\textbf{0.9375}&0.7200& 1.0000 \\
\soup             & 0.9052&0.8353&0.6917&0.6588&0.0003&0.9605&\ul{0.9205}&\ul{0.9037}&\ul{0.9364}&0.8859& 1.0000 \\
\hline
\udp  + \ft       & 0.944&0.8848&0.7028&0.6986&0.0004&0.9670&0.8472&0.8476&0.9237&\ul{0.9559}&0.7764 \\
\vat  + \ft       & \textbf{0.9611}&\textbf{0.8900}&\textbf{0.7442}&\ul{0.7321}&\ul{0.0027}&0.9294&0.8355&0.8281&0.9178&0.8276&0.7839 \\
\soup + \ft       & 0.9466&\ul{0.8892}&0.7031&0.6931&0.0001&\ul{0.9678}&0.8390&0.8287&0.9216&0.9265&0.7806 \\
\hline
\end{tabular}%
} 
\caption{\textbf{CIFAR10: Hardness Promoting Augmentations and Adaptation.} In contrast to Living17 and DomainNet, \ft~is more effective than \lpft~in the safety metrics and performs comparably on ID/OOD generalization. However, hardness-promoting variants, particularly \vat, see noticeable improvements with respect to generalization \& corruptions, performing comparably to \ft. }
\label{tab:cifar10}
\vspace{-10pt}
\end{table}


\section{Conclusion}
In this paper, we took a closer look at the behavior of protocols designed for adapting large-scale pretrained models to downstream datasets. While it is argued that adaptation protocols should be designed to mitigate feature distortion (e.g., \lpft) in order to improve ID and OOD generalization, we found that when additional aspects of safe generalization are evaluated (e.g., prediction calibration error, adversarial robustness etc.), mitigating feature distortion alone is not sufficient. We then considered the complementary perspective, that adaptation protocols should also mitigate simplicity bias. Using a synthetic dominoes dataset that allows for control over the correlation between simple and complex features, we found that protocols have varying levels of effectiveness in reducing reliance upon simple features. While, as expected, \ft, is most susceptible to simplicity bias, we see that \lpft~is able to balance both distortion and simplicity bias in settings where the correlation between simple and complex features is not too extreme. Motivated by the benefits of \lpft~and given the known relationship between simplicity bias and sub-optimal generalization, we used ``hardness-promoting'' \lp~initializations (virtual adversarial, uncertainty-driven perturbations, sparse soups) to further improve \lpft's performance. 
These modifications helped reduce \lpft's reliance upon simple features on the synthetic dataset. On three real-world datasets, these modified protocols led to some improvements in safety and generalization performance, further validating the need to consider both distortion and simplicity bias when designing adaptation protocols.

\newpage
\subsubsection*{Acknowledgments}
We thank Ekdeep Singh Lubana for several helpful discussions during the course of this project. This work was performed under the auspices of the U.S. Department of Energy by the Lawrence Livermore National Laboratory under Contract No. DE-AC52-07NA27344, Lawrence Livermore National Security, LLC.and was supported by the LLNL-LDRD Program under Project No. 21-ERD-012. It was also partially supported by the National Science Foundation under CAREER Grant No.~IIS 1845491. PT began this work as an intern at Lawrence Livermore National Laboratory.
\bibliography{main}
\bibliographystyle{iclr2023_conference}
\newpage
\appendix
\section*{Appendix}
\section{Additional Related Work}\label{sup:relatedwork}
For a comprehensive overview of transfer learning, please see the surveys of~\citeauthor{zhuang21_transferlearningsurvey} and~\citeauthor{Pan10_TransferSurvey}. Here, we discuss a few directly works directly relevant to our own.

Recently, \citeauthor{Kumar22_FinetuningDistorts} demonstrated that learning probing prior to fine-tuning (e.g., \lpft) can improve both in-distribution and out-of-distribution performance when transferring to a downstream task given a highly expressive, pretrained model. They demonstrated that \ft~only modifies features in the ID representation subspace and not in other directions, which can lead higher OOD error as direction outside the ID subspace are necessary for OOD generalization. However, by initializing \ft with a trained linear probe, feature distortion can be decreased since this initialization is closer to optimal model, and thus requires less distortion in ID subspace, preserving the expressiveness of the original model. Concurrently, \citeauthor{Kirichenko22_LastLayerRetrain} demonstrated that models are able to learn both core features and spurious features. However, classifiers can rely upon spurious features, harming performance on minority groups. To reduce the reliance on spurious features, they propose to retrain the classifier on a small amount of ``re-weighting" data, that allows the model to leverage the core features instead of the spurious features. 

Other modifications and heuristics have also been proposed to improve fine-tuning, including side-tuning~\citep{Zhang19_Sidetuning}, which tunes a small secondary network that is then combined with the original model, using larger/smaller learning rates for the classifier, as well as regularization-based methods \citep{Jiang20_SMART}. We focus on the \lpft~protocol, as it is principled and achieves strong OOD performance. 

Additionally, several works have studied properties of the model that influence the effectiveness of transfer learning~\citep{Azizpour16_FactorsTransfer,Huh16_ImageNetGoodTransfer,Kornblith19_BetterImageNetTransferBetter,Lee23_surgicalFT,evci22_Head2Toe,Lee23_DivDis,Izmailov22_FeatLearningSpurious,Lubana23_ModeConnectivity, Rame22_DiWA,Trivedi22_GraphSSL}, including the robustness of pretrained features ~\citep{Salman20,Utrera21_AdvTrainedImageNets}. While the connection between adversarial training and improved feature representations~\citep{Zhu21_FeaturePurification,Kaur19_PerceptuallyAligned} has been studied, we use virtual adversarial training during \lp~to learn a better classifier that is less reliant upon simple features, and we do not use an adversarially trained feature extractor. 
Finally, we note that while we are, to the best of our knowledge, the first to consider this holistic evaluation of safety and generalization in the context of transfer learning with highly expressive pretrained models, \citeauthor{Hendrycks21_PixMix} have considered the trade-offs induced by different data augmentation strategies~\citep{Yun19_CutMix,Devries17_cutout,Hendrycks20_AugMix,Cubuk18_AutoAugment,Cubuk20_RandAug} on safety metrics in supervised learning. We emphasize that while our evaluation is similar, that our work focuses on a different context and contains an additional layer of complexity as we consider the interaction between adaptation protocols, generalization behavior and safety performance.
\section{Experimental Details}
Please see the \href{https://github.com/pujacomputes/23-ICLR-Adaptation.git}{https://github.com/pujacomputes/23-ICLR-Adaptation.git} for training details. In brief, we performed grid-search to find the best parameters, which are as follows. 
For CIFAR-10 and CIFAR-100, we train only the classifier for 200 epochs with LR=30 during \lp. For \ft, the entire model is trained for 20 epochs with LR=1e-5. For \lpft, the model's classifier is initialized with the solution found by \lp, and then it is fine-tuned for 20 epochs. A grid-search was conducted to determine the LR for \lp~and \ft. For Domain-Net Experiments, we use 200 epochs with LR=30 during \lp. For \ft, the entire model is trained for 20 epochs with LR=3e-4. For \lpft, the model's classifier is initialized with the solution found by \lp, and then it is fine-tuned for 20 epochs, using LR=3e-7. Furthermore, following \citeauthor{Kumar22_FinetuningDistorts}, we freeze the batchnorm layers during \lpft. A  CLIP~\citep{Radford21_Clip} pretrained ResNet-50 is used for the DomainNet experiments, while a MoCoV2~\citep{He20_MoCo} is used for all CIFAR experiments. We use augmentation functions from timm~\citep{rw2019timm} and compute CKA scores using the packaged provided by  \href{https://github.com/AntixK/PyTorch-Model-Compare}{torch-cka}. When using augmented protocols, the same LRs are used. Note, all results were obtained by averaging over 3 seeds. We consider model soups of sizes 5,10,20, tune $\epsilon$ in 0.005, 0.01, 0.02 and 0.1 for UDP, and $\alpha$ in 0.001, 0.01, 0.1 for VAT. For CIFAR-MNIST results, LP is done for 100 epochs, and FT is done for 20 epochs.

\subsection{Motivation for Hardness-Promoting Variants}\label{sup:motivation} 
We selected UDP~\citep{pagliardini22_udp}, VAT~\citep{Miyato17_Vat}, and model-soups~\citep{wortsman22_modelsoup} as simplicity bias mitigation strategies due to their effectiveness and ease of use. We emphasize, however, that our findings are not specific to the choice of a given mitigation strategy and we expect that advancements in such strategies will further improve the effectiveness of our proposed \lpft variants. At present, the selected strategies are strong, representative mitigations that we have confirmed are effective at mitigating simplicity bias in the adaptation context using the synthetic dominoes dataset in Sec. \ref{sec:simplicity}.

We conceptually justify each strategy here: 
\begin{itemize}
    \item UDP is designed to help mitigate simplicity bias by learning by a large margin classifier, opposed to a narrow margin classifier that relies upon simple features. As noted by \citet{Shah20_SimplicityBias}, such narrow margin classifiers are sensitive to small perturbations and the simple features supporting the decision boundary may not be discriminative under distribution shifts. By maximizing uncertainty (instead of loss) to create adversarial perturbations, UDP is able to learn a maximum-margin classifier that is better able to handle such shifts. Notably, to create such a maximum-margin classifier, the model will necessarily learn more complex features;
    \item We use virtual adversarial training (VAT) to help avoid reliance upon simple features, as VAT enforces distribution smoothness so that classifiers become robust in some epsilon neighborhood around the input. We note that we are performing this training in the hidden representation space, so perturbations correspond may be altering high-level semantics. To maintain strong performance under such high-level perturbations, the model should learn to rely upon more complex features, and learn a better margin classifier;
    \item We use model-soups so that we may learn a set of classifiers that rely upon disjoint sets of features. By learning a set of diverse classifiers, we are able to average classifiers that have learned to rely upon different features, instead of becoming overly reliant upon a single simple feature. In future work, we intend to build a theoretical framework that helps us better justify these interventions and create new ones. 
\end{itemize}

\subsection{Applying Simplicity Bias Mitigation Strategies to Fine-Tuning Step.}\label{sup:ablation}
\begin{wrapfigure}{r}{0.5\textwidth}
\begin{center}
\includegraphics[width=0.5\textwidth]{FIGS_v3/Appendix-Variants.pdf}
\end{center}
\caption{\textbf{Applying Mitigation Strategies to \ft.} We create \ft~variants of our \lp~mitigation strategies and evaluate them on the synthetic dominoes dataset. We see that \ft~variants lose performance with respect to \lp~variants, indicating that interventions must be undertaken during the \lp~step as originally proposed.}
\label{sup:variants}
\vspace{-.2in}
\end{wrapfigure}
To demonstrate that simplicity bias mitigation strategies must be applied during the \lp~step of \ft~for maximum effectiveness, we conduct the following additional experiment.

\textit{Setup.} We evaluate two additional protocols where \texttt{VAT} and \texttt{UDP} are applied only during the \ft~step, ($\lp \texttt{+}\ft(\texttt{VAT})$, and $\lp \texttt{+} \ft(\texttt{UDP})$), on the synthetic dominoes dataset. We plot the results for Randomized OOD Accuracy in Fig. \ref{sup:variants}. 

\textit{Results.} Here, we see that, across three different correlation ratios, \ft~variants lose performance with respect to the \lp~mitigation variants. Notably, \lp \texttt{+} \ft~(\texttt{UDP})~loses up to 4\% performance with respect to \lp (\texttt{UDP})\texttt{+} \ft. While performance drops are not as large for \texttt{VAT}, we nonetheless see that \lp \texttt{+} \ft (\texttt{VAT})~loses performance with respect to \lp (\texttt{VAT})\texttt{+} \ft.

Our results in Fig. \ref{sup:variants} support our conceptual argument that mitigation strategies must be undertaken during the \lp~step to ensure that subsequent \ft~is in a direction that preserves complex features; applying mitigation strategies during \ft~may be too late to avoid simplicity bias. We note that applying mitigation strategies during \ft, in addition to \lp, may further improve performance, and we will add these variants in the final version. We did not include a \ft~soup variant as it would be prohibitively expensive to train and average large soups of entire models (instead of classifiers). This highlights the computational efficiency of implementing mitigation strategies in the \lp~step itself. \label{sup:experimentaldetails}
\section{Additional Results}\label{sup:additional_results}
Below, we include results corresponding to different hyperparameters (number of souped classifiers, $\alpha$ for vat, and $\delta$ for udp). 

\begin{table*}[ht]
\setlength\tabcolsep{5pt}
\small
\centering
\resizebox{\columnwidth}{!}{%
\begin{tabular}{l | ccccccccccc } 
 \multicolumn{1}{c}{} & \multicolumn{2}{c}{\hlb{Generalization}} & \multicolumn{3}{c}{\hlb{Robustness}} & \multicolumn{4}{c}{\hlb{Calibration}} & \multicolumn{1}{c}{\hlb{Anomaly Detection}} & \multicolumn{1}{c}{\hlb{Rep. Similarity}} \\ \cmidrule(lr){2-3} \cmidrule(lr){4-6}  \cmidrule(lr){7-10}  \cmidrule(lr){11-11} \cmidrule{12-12}
 
Protocol & ID & OOD & C & $\overline{\text{C}}$ & Adv. & ID & C & $\overline{\text{C}}$ & OOD. & Out-of-Class & ID  \\
& \textcolor{gray}{Acc.} & \textcolor{gray}{Acc.} & \textcolor{gray}{Acc.} & \textcolor{gray}{Acc.} & \textcolor{gray}{Acc.} & \textcolor{gray}{1-RMS} &  \textcolor{gray}{1-RMS} & \textcolor{gray}{1-RMS} & \textcolor{gray}{1-RMS} & \textcolor{gray}{AUROC} & \textcolor{gray}{CKA} \\
\toprule
\lp              & 0.9138 & 0.8190 & 0.6912 & 0.6553 & 0.0003 & 0.9595 & 0.8303 & 0.8142 & 0.8696 & 0.6206 & 1.0000 \\
\midrule
\lp + soup-5        & 0.9108 & 0.8348 & 0.7007 & 0.6678 & 0.0002 & 0.9748 & 0.8943 & 0.8835 & 0.9108 & 0.8463 & 1.0000 \\
\lp + soup-10       & 0.9129 & 0.8359 & 0.6985 & 0.6652 & 0.0003 & 0.9669 & 0.9104 & 0.8956 & 0.9205 & 0.8713 & 1.0000 \\
\lp + soup-20       & 0.9052 & 0.8353 & 0.6917 & 0.6588 & 0.0003 & 0.9605 & 0.9205 & 0.9037 & 0.9364 & 0.8859 & 1.0000 \\
\midrule
\lp + udp-0.005     & 0.9129 & 0.8332 & 0.7015 & 0.6702 & 0.0003 & 0.9729 & 0.8879 & 0.8817 & 0.9017 & 0.8708 & 1.0000 \\
\lp + udp-0.01      & 0.9033 & 0.8356 & 0.6948 & 0.6643 & 0.0003 & 0.9689 & 0.9111 & 0.9023 & 0.9277 & 0.9033 & 1.0000 \\
\lp + udp-0.02      & 0.8885 & 0.8281 & 0.6796 & 0.6492 & 0.0004 & 0.9655 & 0.9259 & 0.9142 & 0.9473 & 0.9217 & 1.0000 \\
\lp + udp-0.1       & 0.8573 & 0.8005 & 0.6290 & 0.6064 & 0.0007 & 0.9245 & 0.9235 & 0.9143 & 0.9531 & 0.8570 & 1.0000 \\
\lp + vat-0.001     & 0.9189 & 0.8276 & 0.6945 & 0.6606 & 0.0006 & 0.9714 & 0.8564 & 0.8442 & 0.8927 & 0.7159 & 1.0000 \\
\lp + vat-0.01      & 0.8977 & 0.8251 & 0.6742 & 0.6483 & 0.0002 & 0.9265 & 0.9255 & 0.9139 & 0.9375 & 0.7200 & 1.0000 \\

\midrule
\ft              & 0.9539 & 0.8754 & 0.7434 & 0.7553 & 0.0231 & 0.9668 & 0.8364 & 0.8453 & 0.9232 & 1.0000 & 0.6831 \\
\midrule
\lp+\ft           & 0.9442 & 0.8678 & 0.6921 & 0.6790 & 0.0018 & 0.9521 & 0.7849 & 0.7721 & 0.8864 & 0.6511 & 0.7853 \\
\midrule
(\lp+soup-5) +\ft    & 0.9466 & 0.8832 & 0.6997 & 0.6861 & 0.0001 & 0.9639 & 0.8197 & 0.8051 & 0.9155 & 0.9020 & 0.7603 \\
(\lp+soup-10) +\ft   & 0.9467 & 0.8857 & 0.7022 & 0.6907 & 0.0001 & 0.9660 & 0.8307 & 0.8182 & 0.9184 & 0.9161 & 0.7671 \\
(\lp+soup-20) +\ft   & 0.9466 & 0.8892 & 0.7031 & 0.6931 & 0.0001 & 0.9678 & 0.8390 & 0.8287 & 0.9216 & 0.9265 & 0.7806 \\
\midrule
(\lp+udp-0.005) +\ft & 0.9458 & 0.8864 & 0.6962 & 0.6893 & 0.0005 & 0.9643 & 0.8127 & 0.8110 & 0.9119 & 0.9180 & 0.7742 \\
(\lp+udp-0.01) +\ft  & 0.9450 & 0.8869 & 0.7048 & 0.6977 & 0.0004 & 0.9642 & 0.8335 & 0.8311 & 0.9209 & 0.9419 & 0.7746 \\
(\lp+udp-0.02) +\ft  & 0.9440 & 0.8848 & 0.7028 & 0.6986 & 0.0004 & 0.9670 & 0.8472 & 0.8476 & 0.9237 & 0.9559 & 0.7764 \\
(\lp+udp-0.1) + \ft  & 0.9435 & 0.8836 & 0.6959 & 0.6952 & 0.0000 & 0.9676 & 0.8449 & 0.8525 & 0.9355 & 0.9651 & 0.7382 \\
(\lp+vat)+\ft     & 0.9611 & 0.8900 & 0.7442 & 0.7321 & 0.0027 & 0.9294 & 0.8355 & 0.8281 & 0.9178 & 0.8276 & 0.7839 \\
\bottomrule
\end{tabular}%
}
\caption{\textbf{CIFAR10, Hardness-Promoting Augmentations.}} 
\vspace{-10pt}
\end{table*}

\begin{table*}[ht]
\setlength\tabcolsep{5pt}
\small
\centering
\resizebox{\columnwidth}{!}{%
\begin{tabular}{l | ccccccccccc } 
 \multicolumn{1}{c}{} & \multicolumn{2}{c}{\hlb{Generalization}} & \multicolumn{3}{c}{\hlb{Robustness}} & \multicolumn{4}{c}{\hlb{Calibration}} & \multicolumn{1}{c}{\hlb{Anomaly Detection}} & \multicolumn{1}{c}{\hlb{Rep. Similarity}} \\ \cmidrule(lr){2-3} \cmidrule(lr){4-6}  \cmidrule(lr){7-10}  \cmidrule(lr){11-11} \cmidrule{12-12}
 
Protocol & ID & OOD & C & $\overline{\text{C}}$ & Adv. & ID & C & $\overline{\text{C}}$ & OOD. & Out-of-Class & ID  \\
& \textcolor{gray}{Acc.} & \textcolor{gray}{Acc.} & \textcolor{gray}{Acc.} & \textcolor{gray}{Acc.} & \textcolor{gray}{Acc.} & \textcolor{gray}{1-RMS} &  \textcolor{gray}{1-RMS} & \textcolor{gray}{1-RMS} & \textcolor{gray}{1-RMS} & \textcolor{gray}{AUROC} & \textcolor{gray}{CKA} \\
\toprule
\lp             & 0.9521 & 0.8124 & 0.7010 & 0.7378 & 0.2350 & 0.9313 & 0.8693 & 0.8802 & 0.9117 & 0.9907 & 1.0000  \\
\midrule
\lp + udp-0.005      & 0.9524 & 0.8114 & 0.7012 & 0.7379 & 0.2337 & 0.9304 & 0.8699 & 0.8806 & 0.9108 & 0.9907 & 1.000       \\
\lp + udp-0.01       & 0.9524 & 0.8110 & 0.7017 & 0.7382 & 0.2353 & 0.9308 & 0.8691 & 0.8801 & 0.9118 & 0.9908 & 1.000       \\
\lp + udp-0.02       & 0.9500 & 0.8126 & 0.7036 & 0.7387 & 0.2373 & 0.9343 & 0.8621 & 0.8763 & 0.9135 & 0.9913 & 1.000       \\
\lp + udp-0.1        & 0.9459 & 0.8165 & 0.6840 & 0.7220 & 0.2339 & 0.9032 & 0.8243 & 0.8427 & 0.8990 & 0.9882 & 1.000       \\
\midrule
\lp + soup-5         & 0.9439 & 0.7996 & 0.6874 & 0.7290 & 0.2451 & 0.8806 & 0.7868 & 0.8094 & 0.9064 & 0.9897 & 1.0000  \\
\lp + soup-10        & 0.9373 & 0.7904 & 0.6767 & 0.7220 & 0.2547 & 0.8496 & 0.7478 & 0.7709 & 0.8841 & 0.9887 & 1.0000  \\
\lp + soup-20        & 0.9298 & 0.7841 & 0.6601 & 0.7082 & 0.2575 & 0.8056 & 0.7084 & 0.7305 & 0.8274 & 0.9867 & 1.0000  \\
\midrule
\lp + vat-0.001    & 0.9524 & 0.8122 & 0.7010 & 0.7379 & 0.2345 & 0.9299 & 0.8682 & 0.8791 & 0.9103 & 0.9907 & 1.0000  \\
\midrule
\ft             & 0.9518 & 0.7168 & 0.7011 & 0.7164 & 0.1563 & 0.8873 & 0.9019 & 0.8604 & 0.9295 & 0.9794 & 0.7847  \\
\midrule
\lp+\ft          & 0.9643 & 0.8261 & 0.7426 & 0.7671 & 0.2135 & 0.9782 & 0.9472 & 0.9451 & 0.8742 & 0.9924 & 0.9887  \\
\midrule
(\lp+udp-0.005) +\ft  & 0.9627 & 0.8243 & 0.7434 & 0.7666 & 0.2153 & 0.9811 & 0.9456 & 0.9445 & 0.8736 & 0.9922 & 0.98950 \\
(\lp+udp-0.01) +\ft   & 0.9627 & 0.8253 & 0.7436 & 0.7669 & 0.2133 & 0.9812 & 0.9454 & 0.9447 & 0.8737 & 0.9923 & 0.98957 \\
(\lp+udp-0.02) +\ft   & 0.9637 & 0.8265 & 0.7448 & 0.7681 & 0.2157 & 0.9768 & 0.9464 & 0.9467 & 0.8757 & 0.9927 & 0.98927 \\
(\lp+udp-0.1) +\ft    & 0.9614 & 0.8249 & 0.7499 & 0.7689 & 0.2165 & 0.9808 & 0.9441 & 0.9420 & 0.8711 & 0.9912 & 0.9861  \\
\midrule
(\lp+soup-5) + \ft    & 0.9608 & 0.8163 & 0.7456 & 0.7684 & 0.1855 & 0.9760 & 0.9498 & 0.9492 & 0.8678 & 0.9936 & 0.98540 \\
(\lp+soup-10) + \ft   & 0.9580 & 0.8114 & 0.7445 & 0.7678 & 0.1753 & 0.9838 & 0.9503 & 0.9488 & 0.8748 & 0.9938 & 0.98360 \\
(\lp+soup-20) + \ft   & 0.9594 & 0.8165 & 0.7450 & 0.7684 & 0.1782 & 0.9893 & 0.9503 & 0.9490 & 0.8609 & 0.9936 & 0.98190 \\
\midrule
(\lp+vat-0.001) +\ft & 0.9647 & 0.8247 & 0.7425 & 0.7650 & 0.2224 & 0.9727 & 0.9521 & 0.9463 & 0.8775 & 0.9925 & 0.9370 \\
\bottomrule
\end{tabular}%
}
\caption{\textbf{Living17, Hardness-Promoting Augmentations}} 
\vspace{-10pt}
\end{table*}
\begin{table*}[ht]
\setlength\tabcolsep{5pt}
\small
\centering
\resizebox{\columnwidth}{!}{%
\begin{tabular}{l | ccccccccccc } 
 \multicolumn{1}{c}{} & \multicolumn{2}{c}{\hlb{Generalization}} & \multicolumn{3}{c}{\hlb{Robustness}} & \multicolumn{4}{c}{\hlb{Calibration}} & \multicolumn{1}{c}{\hlb{Anomaly Detection}} & \multicolumn{1}{c}{\hlb{Rep. Similarity}} \\ \cmidrule(lr){2-3} \cmidrule(lr){4-6}  \cmidrule(lr){7-10}  \cmidrule(lr){11-11} \cmidrule{12-12}
 
Protocol & ID & OOD & Sketch-C & Real-C & Adv. & ID & Sketch-C & Real-C & OOD. & Out-of-Class & ID  \\
& \textcolor{gray}{Acc.} & \textcolor{gray}{Acc.} & \textcolor{gray}{Acc.} & \textcolor{gray}{Acc.} & \textcolor{gray}{Acc.} & \textcolor{gray}{1-RMS} &  \textcolor{gray}{1-RMS} & \textcolor{gray}{1-RMS} & \textcolor{gray}{1-RMS} & \textcolor{gray}{AUROC} & \textcolor{gray}{CKA} \\

\toprule
\lp                    & 0.8913 & 0.8013 & 0.6019 & 0.6020 & 0.1768 & 0.9638 & 0.9264 & 0.9045 & 0.9014 & 0.8679 & 1.0000 \\
\lp+augmix             & 0.8897 & 0.7998 & 0.6336 & 0.6104 & 0.1872 & 0.9718 & 0.9230 & 0.9263 & 0.9083 & 0.8818 & 1.0000 \\
\lp+autoaug            & 0.8944 & 0.8057 & 0.6419 & 0.6257 & 0.1857 & 0.9614 & 0.9357 & 0.9309 & 0.9022 & 0.8849 & 1.0000 \\
\lp+randaug            & 0.8971 & 0.8090 & 0.6392 & 0.6232 & 0.1877 & 0.9559 & 0.9321 & 0.9312 & 0.9036 & 0.8875 & 1.0000 \\
\lp+vat                & 0.8836 & 0.7914 & 0.5893 & 0.5963 & 0.1687 & 0.8897 & 0.9552 & 0.8905 & 0.9178 & 0.8735 & 1.0000 \\
\midrule
\ft                    & 0.7613 & 0.4522 & 0.5186 & 0.2744 & 0.4164 & 0.8368 & 0.6379 & 0.7234 & 0.5597 & 0.8841 & 0.6092 \\
\ft+augmix             & 0.8246 & 0.5233 & 0.5911 & 0.3408 & 0.4802 & 0.9308 & 0.8042 & 0.8665 & 0.6761 & 0.9255 & 0.5272 \\
\ft+autoaug            & 0.7786 & 0.5161 & 0.5561 & 0.3160 & 0.4313 & 0.9157 & 0.7485 & 0.8246 & 0.7324 & 0.9231 & 0.7025 \\
\ft+randaug            & 0.7823 & 0.5370 & 0.5610 & 0.3298 & 0.4551 & 0.9160 & 0.7970 & 0.8682 & 0.7444 & 0.9318 & 0.6477 \\
\midrule
\lp+\ft                 & 0.8985 & 0.7990 & 0.6343 & 0.5979 & 0.1927 & 0.9566 & 0.8899 & 0.8445 & 0.8024 & 0.9022 & 0.9222 \\
\lp+(\ft+augmix)        & 0.9047 & 0.8081 & 0.6673 & 0.5980 & 0.2597 & 0.9768 & 0.9200 & 0.9067 & 0.8443 & 0.9155 & 0.8811 \\
\lp+(\ft+autoaug)       & 0.9023 & 0.8028 & 0.6571 & 0.5851 & 0.2354 & 0.9830 & 0.9249 & 0.8990 & 0.8484 & 0.9034 & 0.9096 \\
\lp+(\ft+randaug)       & 0.9054 & 0.8099 & 0.6703 & 0.6152 & 0.2489 & 0.9786 & 0.9194 & 0.9044 & 0.8598 & 0.9252 & 0.9000 \\
\midrule
(\lp+vat) +\ft  & 0.9048 & 0.8009 & 0.6466 & 0.6131 & 0.1942 & 0.9686 & 0.8911 & 0.8428 & 0.7985 & 0.9204 & 0.9370 \\
(\lp+vat) +(\ft+augmix) & 0.9032 & 0.8024 & 0.6589 & 0.5896 & 0.2525 & 0.9769 & 0.9169 & 0.8929 & 0.8384 & 0.9212 & 0.8673 \\
(\lp+vat)+(\ft+autoaug) & 0.9003 & 0.8049 & 0.6600 & 0.5862 & 0.2331 & 0.9783 & 0.9178 & 0.9000 & 0.8381 & 0.9149 & 0.9244 \\
(\lp+vat)+(\ft+randaug) & 0.9006 & 0.8060 & 0.6651 & 0.5894 & 0.2622 & 0.9762 & 0.9197 & 0.8993 & 0.8414 & 0.9238 & 0.8956 \\
\bottomrule
\end{tabular}%
}
\caption{\textbf{DomainNet, Diversity Promoting Augmentations and Generalization Trade-offs.}}
\vspace{-10pt}
\end{table*}
\begin{table}[!t]
\setlength\tabcolsep{5pt}
\renewcommand{\arraystretch}{1.3}
\small
\centering
\resizebox{\columnwidth}{!}{%
\begin{tabular}{c | ccccccccccc } 
\multicolumn{1}{c}{}& \multicolumn{2}{c}{\hlb{\normalsize Generalization}} & \multicolumn{3}{c}{\hlb{\normalsize Robustness}} & \multicolumn{4}{c}{\hlb{\normalsize Calibration}} & \multicolumn{1}{c}{\hlb{\normalsize Anomaly Det.}} & \multicolumn{1}{c}{\hlb{\normalsize Rep. Similarity}} \\ \cmidrule(lr){1-1} \cmidrule(lr){2-3} \cmidrule(lr){4-6}  \cmidrule(lr){7-10}  \cmidrule(lr){11-11} \cmidrule(lr){12-12} 

\multirow{2}*{Protocol} &  ID & OOD & C & $\overline{\text{C}}$ & Adv. & ID & C & $\overline{\text{C}}$ & OOD. & Out-of-Class & ID \\
& \textcolor{gray}{Acc.} & \textcolor{gray}{Acc.} & \textcolor{gray}{Acc.} & \textcolor{gray}{Acc.} & \textcolor{gray}{Acc.} & \textcolor{gray}{1-RMS} &  \textcolor{gray}{1-RMS} & \textcolor{gray}{1-RMS} & \textcolor{gray}{1-RMS} & \textcolor{gray}{AUROC}  & \textcolor{gray}{CKA}  \\ 
\hline
\lp              & 0.9297 & 0.9083 & 0.8532 & 0.7491 & 0.7077 & 0.9794 & 0.9006 & 0.9007 & 0.9301 & 0.9623 & 0.0668 \\
\midrule
\lp + soup-5        & 0.9220 & 0.9151 & 0.8315 & 0.7432 & 0.7050 & 0.9598 & 0.9232 & 0.9279 & 0.9623 & 0.9665 & 0.1399 \\
\lp + soup-10      & 0.9156 & 0.9135 & 0.8183 & 0.7344 & 0.6985 & 0.9476 & 0.9221 & 0.9271 & 0.9732 & 0.9602 & 0.1778 \\
\lp + soup-20      & 0.9069 & 0.9064 & 0.8065 & 0.7216 & 0.6885 & 0.9279 & 0.9129 & 0.9191 & 0.9714 & 0.9484 & 0.2617 \\
\midrule
\lp + udp-0.005    & 0.9299 & 0.9092 & 0.8533 & 0.7494 & 0.7079 & 0.9794 & 0.9009 & 0.9003 & 0.9312 & 0.9614 & 0.0822 \\
\lp + udp-0.01    & 0.9298 & 0.9097 & 0.8535 & 0.7495 & 0.7083 & 0.9795 & 0.9007 & 0.9006 & 0.9316 & 0.9616 & 0.0880 \\
\lp + udp-0.02     & 0.9294 & 0.9108 & 0.8538 & 0.7497 & 0.7088 & 0.9789 & 0.9012 & 0.9014 & 0.9335 & 0.9631 & 0.1017 \\
\lp + udp-0.1      & 0.9238 & 0.9218 & 0.8377 & 0.7488 & 0.7111 & 0.9801 & 0.9154 & 0.9216 & 0.9517 & 0.9645 & 0.1478 \\
\midrule
\lp + vat-0.001    & 0.9298 & 0.9091 & 0.8533 & 0.7493 & 0.7078 & 0.9801 & 0.9014 & 0.9012 & 0.9325 & 0.9614 & 0.0784 \\
\lp + vat-0.01     & 0.9295 & 0.9094 & 0.8531 & 0.7494 & 0.7080 & 0.9800 & 0.9039 & 0.9040 & 0.9342 & 0.9632 & 0.0837 \\
\lp + vat-0.1      & 0.9275 & 0.9106 & 0.8493 & 0.7481 & 0.7087 & 0.9581 & 0.9191 & 0.9246 & 0.9589 & 0.9598 & 0.1528 \\
\midrule
\ft             & 0.9724 & 0.8761 & 0.9218 & 0.8131 & 0.8074 & 0.9577 & 0.8429 & 0.8418 & 0.8855 & 0.9138 & 0.9317 \\
\midrule
\lp+\ft         & 0.9692 & 0.9387 & 0.9195 & 0.8106 & 0.7736 & 0.9451 & 0.8034 & 0.7743 & 0.9026 & 0.8949 & 0.5349 \\
\midrule
(\lp+soup-5) +\ft    & 0.9685 & 0.9417 & 0.9210 & 0.8136 & 0.7787 & 0.9385 & 0.8079 & 0.7765 & 0.9102 & 0.8974 & 0.5315 \\
(\lp+soup-10) +\ft   & 0.9681 & 0.9411 & 0.9220 & 0.8178 & 0.7824 & 0.9382 & 0.8119 & 0.7796 & 0.9072 & 0.8933 & 0.5521 \\
(\lp+soup-20) +\ft  & 0.9677 & 0.9395 & 0.9213 & 0.8164 & 0.7837 & 0.9385 & 0.8107 & 0.7817 & 0.9070 & 0.8964 & 0.5411 \\
\midrule
(\lp+udp-0.005) +\ft  & 0.9677 & 0.9297 & 0.9142 & 0.8104 & 0.7710 & 0.9422 & 0.8024 & 0.7718 & 0.8942 & 0.8916 & 0.6428 \\
(\lp+udp-0.01) +\ft   & 0.9677 & 0.9359 & 0.9195 & 0.8098 & 0.7721 & 0.9417 & 0.8029 & 0.7732 & 0.9019 & 0.8999 & 0.4239 \\
(\lp+udp-0.02) +\ft  & 0.9687 & 0.9349 & 0.9195 & 0.8136 & 0.7724 & 0.9437 & 0.8067 & 0.7736 & 0.8994 & 0.8981 & 0.5015 \\
(\lp+udp-0.1) +\ft    & 0.9688 & 0.9423 & 0.9242 & 0.8174 & 0.7811 & 0.9408 & 0.8130 & 0.7815 & 0.9072 & 0.9064 & 0.4496 \\
\midrule
(\lp+vat-0.001)+\ft & 0.9681 & 0.9366 & 0.9180 & 0.8111 & 0.7727 & 0.9422 & 0.8033 & 0.7732 & 0.9013 & 0.8962 & 0.5904 \\
(\lp+vat-0.01)+\ft  & 0.9689 & 0.9366 & 0.9168 & 0.8121 & 0.7766 & 0.9455 & 0.8062 & 0.7791 & 0.9013 & 0.8918 & 0.5687 \\
(\lp+vat-0.1)+\ft  & 0.9692 & 0.9402 & 0.9207 & 0.8127 & 0.7743 & 0.9420 & 0.8068 & 0.7734 & 0.9083 & 0.8978 & 0.4398 \\
\bottomrule
\end{tabular}%
}
\caption{\textbf{CIFAR10 with Resnet101/SimCLR Pretrained Model.} We see that with a larger model, and different pretraining method, our proposed variants still have some benefits. We note that the baseline performance is also improved as a result of a more larger pretrained model.}
% \vspace{-10pt}
\label{tab:rn101-cifar10}
\end{table}

\end{document}
