

\subsection{\centering \centering Motivation}
\subsection*{Why was the dataset created?} 
% For instance, was there a specific task in mind? was there a specific gap that needed to be filled?
\noindent The dataset was created to facilitate future academic Computer Vision research about human aesthetic preference.
\subsection*{Who created this dataset (\eg which team, research group) and on behalf of which entity (\eg company, institution, organization)?}
\noindent The dataset was created by researchers at MMLab, The Chinese University of Hong Kong.
% The Chinese University of Hong Kong and CUHK-SenseTime Joint Laboratory.
% \subsection*{Who funded the creation of the dataset?}
% % If there is an associated grant, please provide the name of the grantor and the grant name and number.
% \noindent 
% The dataset was funded by .


\subsection{\centering Composition}
\subsection*{What do the instances that comprise the dataset represent (\eg
documents, photos, people, countries)? Are there multiple types of instances? (\eg movies, users, ratings; people, interactions between them; nodes, edges)}
\noindent
The instances are prompts and generated images, along with human preference choices among the images generated by the same prompt.


\subsection*{Are relationships between instances made explicit in the data (\eg social network links, user/movie ratings, etc.)?}
\noindent
Yes, instances generated by the same user are identified by the same user id, which is anonymized for privacy.


\subsection*{How many instances are there? (of each type, if appropriate)?}
\noindent 
There are 25,205 instances in the dataset.


\subsection*{What data does each instance consist of? ``Raw'' data (\eg unprocessed text or images) or Features/attributes? Is there a label/target associated with instances? If the instances related to people, are sub-populations identified (\eg by age, gender, etc.) and what is their distribution?}
\noindent 
Each instance consists of $n\in {2,3,4}$ image, one prompt and one human choice. 

\subsection*{Is any information missing from individual instances? If so, please
provide a description, explaining why this information is missing (\eg
because it was unavailable). This does not include intentionally removed
information, but might include, \eg redacted text.}
\noindent 
Yes, we omit the specific parameters for generating the images, such as diffusion steps and guidance scale.
They are omitted because we are more interested in the users' preference about the generated images, rather than how they are created.
Also, since the same batch of images (among which users make comparisons) are always generated with the same set of parameters except the random seed, they are irrelevant variables when studying human preferences.

\subsection*{Is everything included or does the data rely on external resources?}
\noindent The dataset is self-contained.

\subsection*{Are there recommended data splits and evaluation measures? (\eg training, development, testing; accuracy or AUC)}
\noindent 
In our experiments, we use a training set of 20,205 instances and validation set of 5,000 images, which will be made public.
We recommend using accuracy (\%) with one decimal place.

\subsection*{Are there any errors, sources of noise, or redundancies in the dataset?}
\noindent Yes. The users are not prompted to selected images fitting their preference, so there should be noise in the collected data.


\subsection*{Is the dataset self-contained, or does it link to or otherwise rely on external resources (\eg websites, tweets, other datasets)?}
% If it links to or relies on external resources, a) are there guarantees that they will exist, and remain constant, over time; b) are there official archival versions of the complete dataset (\ie including the external resources as they existed at the time the dataset was created); c) are there any restrictions (\eg licenses, fees) associated with any of the external resources that might apply to a future user? Please provide descriptions of all external resources and any restrictions associated with them, as well as links or other access points, as appropriate.
\noindent The dataset is self-contained.

\subsection*{Does the dataset contain data that might be considered confidential (\eg data that is protected by legal privilege or by doctorpatient confidentiality, data that includes the content of individuals non-public communications)?}
% If so, please provide a description. 
\noindent 
No, the dataset is collected from the Stable Foundation Discord server, which is publicly available for any user with an account.
% According to Section II of the \href{https://huggingface.co/stabilityai/stable-diffusion-2/blob/main/LICENSE-MODEL}{license} of Stable Diffusion, the users are granted with non-exclusive copyright and patent licenses.

\subsection*{Does the dataset contain data that, if viewed directly, might be offensive, insulting, threatening, or might otherwise cause anxiety? If so, please describe why.}
\noindent
We collect images and their prompts from the Stable Foundation discord server. 
Even though the discord server has rules against users sharing any NSFW (not suitable for work, such as sexual and violent content) and illegal images, our dataset still contains some NSFW images and prompts that were not removed by the server moderators.

\subsection*{Does the dataset relate to people?}
% \texttt{If not, you may skip the remaining questions in this section.}}\
\noindent Yes, the prompts are written by users and the choices are made by users.



\subsection*{Does the dataset identify any subpopulations (\eg by age, gender)?}
% \texttt{If so, please describe how these subpopulations are identified and provide a description of their respective distributions within the dataset.}}

\noindent No.




\subsection*{Is it possible to identify individuals (\ie one or more natural persons), either directly or indirectly (\ie in combination with other data) from the dataset?}
% If so, please describe how.
\noindent No.

\subsection*{Does the dataset contain data that might be considered sensitive in any way (\eg data that reveals racial or ethnic origins, sexual orientations, religious beliefs, political opinions or union memberships, or locations; financial or health data; biometric or genetic data; forms of government identification, such as social security numbers; criminal history)?}
% \texttt{If so, please provide a description. 

The dataset may contain sensitive data, because the prompts written by users may contain sensitive information, such as public figures and religious beliefs.

\subsection*{What experiments were initially run on this dataset? Have a summary of those results.}
\noindent 
It has been used to validate the correlation between human preference and several popular image quality evaluation metrics, and serve as the training data for a human preference classifier.
The results show that the tested metrics do not correlate well with human preference, and the correlation of the ViT-L/14 version of CLIP can be improved via fine-tuning on the dataset.


% \subsection*{Any other comments?}

\subsection{\centering Data Collection Process}

% \end{center}
\subsection*{How was the data associated with each instance acquired?}
% Was the data directly observable (\eg raw text, movie ratings), reported by subjects ((\eg survey responses), or indirectly inferred/derived from other data (\eg part-of-speech tags, model-based guesses for age or language)? If data was reported by subjects or indirectly inferred/derived from other data, was the data validated/verified? If so, please describe how.

\noindent
The prompts, images and human choices are directly observable from the Stable Foundation Discord server.

\subsection*{What mechanisms or procedures were used to collect the data (\eg hardware apparatus or sensor, manual human curating, software program, software API)?}
% How were these mechanisms or procedures validated?
\noindent Automatic scraping procedures were used to collect the data.


\subsection*{If the dataset is a sample from a larger set, what was the sampling strategy (\eg deterministic, probabilistic with specific sampling probabilities)?}

\noindent  The dataset is not a sample of a larger set.

\subsection*{Who was involved in the data collection process (\eg students, crowd-workers, contractors) and how were they compensated (\eg
how much were crowdworkers paid)?}

\noindent The authors of this paper were solely involved in the data collection process.

\subsection*{Over what time-frame was the data collected?}
% \texttt{Does this timeframe match the creation timeframe of the data associated with the instances (\eg recent crawl of old news articles)? If not, please describe the timeframe in which the data associated with the instances was created
\noindent %The data was collected over the month of June 2018 and 
The dataset covers the chat history of dreambot channels between Dec. $2^{nd}$ 2022 and Jan. $18^{th}$ 2023.

\subsection*{Were any ethical review processes conducted (\eg by an institutional review board)?}
% If so, please provide a description of these review processes, including the outcomes, as well as a link or other access point to any supporting documentation.
\noindent No official processes were conducted, due to the public nature of the data on Discord channel.


\subsection*{Does the dataset relate to people?}
% If not, you may skip the remaining questions in this section.
\noindent No.

\subsection*{Did you collect the data from the individuals in question directly, or obtain it via third parties or other sources (\eg websites)?}
\noindent 
The data was obtained from public messages in the Discord server.

% \subsection*{Were the individuals in question notified about the data collection?}
% % \texttt{If so, please describe (or show with screenshots or other information) how notice was provided, and provide a link or other access point to, or otherwise reproduce, the exact language of the notification itself.}
% \noindent No. 

% \subsection*{Did the individuals in question consent to the collection and use of their data?}
% % If so, please describe (or show with screenshots or other information) how consent was requested and provided, and provide a link or other access point to, or otherwise reproduce, the exact language to which the individuals consented.
% \noindent No.


% \subsection*{If consent was obtained, were the consenting individuals provided with a mechanism to revoke their consent in the future or for certain uses?}
% % If so, please provide a description, as well as a link or other access point to the mechanism (if appropriate).
% \noindent 

% %No. The data was crawled from public web sources, and the individuals appeared in news stories. But there was no explicit informing of these individuals that their images were being assembled into a dataset.

\subsection*{Has an analysis of the potential impact of the dataset and its use on data subjects (\eg a data protection impact analysis)been conducted?}

%  If so, please provide a description of this analysis, including the outcomes, as well as a link or other access point to any supporting documentation.
\noindent No analysis has been conducted.


% \subsection*{Any other comments?}
% % \noindent 


\subsection{\centering Data Preprocessing}
\subsection*{What preprocessing/cleaning was done? (\eg discretization or bucketing, tokenization, part-of-speech tagging, SIFT feature extraction, removal of instancess, processing of missing values)?}
% \texttt{If so, please provide a description. If not, you may skip the remainder of the questions in this section.}}
\noindent No preprocessing is done on the images and prompts. 
% The images were sized to size $256\times 256$.

% \subsection*{Was the ``raw'' data saved in addition to the preprocessed/cleaned data? (\eg to support unanticipated future uses)}
% \noindent Due to the large size of the original images, we only store resized images.

%\subsection*{Is the preprocessing software available?}
%\noindent 

%\subsection*{Any other comments?}
% \subsection*{Does this dataset collection/processing procedure achieve the motivation for creating the dataset stated in the first section of this datasheet? If not, what are the limitations?}

\subsection{\centering Uses}
\subsection*{Has the dataset been used for any tasks already? If so, please provide a description.}
\noindent As described in the paper, this dataset has been used for analysis about several image quality evaluation metrics and training the proposed human preference classifier.

\subsection*{Is there a repository that links to any or all papers or systems that use the dataset?}
%  If so, please provide a link or other access point.
\noindent
% Papers using this dataset will be specified on WikiScenes's website.
No.

\subsection*{What (other) tasks could the dataset be used for?}
\noindent 
It can be used for tasks related to human preference on generated images.

\subsection*{Is there anything about the composition of the dataset or the way it was collected and preprocessed/cleaned/labeled that might impact future uses?}
% For example, is there anything that a future user might need to know to avoid uses that could result in unfair treatment of individuals or groups (\eg stereotyping, quality of service issues) or other undesirable harms (\eg financial harms, legal risks) If so, please provide a description. Is there anything a future user could do to mitigate these undesirable harms?
\noindent Yes. 
As discussed in Sec.~\ref{sec:limitations}, the dataset is biased towards the preference of the certain group of people that are active in the Stable Foundation Discord server.

% We made various compositional decisions considering future uses as we discuss in the main paper. For example, we provide hierarchical WikiCategories to allow for hierarchy-aware solutions.


\subsection*{Are there tasks for which the dataset should not be used?}
% \texttt{If so, please provide a description.The dataset should not be used for tasks that are high stakes (e.g. law enforcement).}
\noindent No.

%\subsection*{Any other comments?}


\subsection{\centering Data Distribution}
\subsection*{Will the dataset be distributed to third parties outside of the entity (\eg company, institution, organization) on behalf of which
the dataset was created? If so, please provide a description}
\noindent Yes. Researchers at academic institutions will be able to request access to the dataset. %Requests must specify intended use and discuss ethical considerations of tasks. We place these restrictions to minimize potential for misuse.

\subsection*{How will the dataset be distributed? (\eg tarball on website, API, GitHub; does the data have a DOI and is it archived redundantly?)}
\noindent We will provide download links for researchers on a GitHub repository. %We will provide researchers whose requests are approved with specific access to download the dataset via email. The dataset will not be publicly available on any website.

\subsection*{When will the dataset be distributed?}
\noindent Before April 15, 2023.

\subsection*{Will the dataset be distributed under a copyright or other intellectual property (IP) license, and/or under applicable terms of use (ToU)?}
\noindent We will provide a terms of use agreement with the dataset. The dataset as a whole will be distributed under a non-commercial license. % and specific images will carry their own licenses (which we also include in the data).

\subsection*{Have any third parties imposed IP-based or other restrictions on
the data associated with the instances? If so, please describe these restrictions, and provide a link or other access point to, or otherwise reproduce, any relevant licensing terms, as well as any fees associated
with these restrictions.}
\noindent No.

\subsection*{Do any export controls or other regulatory restrictions apply to
the dataset or to individual instances? If so, please describe these
restrictions, and provide a link or other access point to, or otherwise
reproduce, any supporting documentation.}
\noindent Unknown.

%\subsection*{Any other comments?}

\subsection{\centering Dataset Maintenance}
\subsection*{Who is supporting/hosting/maintaining the dataset?}
\noindent The authors of this paper are maintainers of this dataset.

\subsection*{How can the owner/curator/manager of the dataset be contacted
(\eg email address)?}
\noindent By email: wuxiaoshi@link.cuhk.edu.hk .

\subsection*{Is there an erratum?}
\noindent At this time, we are not aware of errors in our dataset. However, we will create an erratum as errors are identified.

\subsection*{Will the dataset be updated? If so, how often and by whom? 
How will updates be communicated? (\eg mailing list, GitHub)}
\noindent 
The dataset will be updated by the authors on an at-will basis (but no more than once a month).
% via email to those with access. By terms of use, users will be expected to apply updates before any further use.

\subsection*{If the dataset relates to people, are there applicable limits on the
retention of the data associated with the instances (\eg were individuals in question told that their data would be retained for a
fixed period of time and then deleted)? If so, please describe these
limits and explain how they will be enforced.}
\noindent No such limits are established.

\subsection*{Will older versions of the dataset continue to be supported/hosted/maintained?}
\noindent N/A

\subsection*{If others want to extend/augment/build on this dataset, is there a mechanism for them to do so? If so, is there a process for tracking/assessing the quality of those contributions. What is the process for communicating/distributing these contributions to users?}
\noindent There will not be a mechanism to build on top of the dataset.

%\subsection*{Any other comments?}
% \noindent 