\documentclass{appolb}
\usepackage{graphicx}
\usepackage{epsfig}

\newcommand{\Pomeron}{{I\!\!P}}
\newcommand{\Reggeon}{{I\!\!R}}
% graphicx package included for placing figures in the text
%------------------------------------------------------

%%%%%%%%%%%%%%%%%%%%%%%%%%%%%%%%%%%%%%%%%%%%%%%%%%
%                                                %
%    BEGINNING OF TEXT                           %
%                                                %
%%%%%%%%%%%%%%%%%%%%%%%%%%%%%%%%%%%%%%%%%%%%%%%%%%
\begin{document}
% \eqsec  % uncomment this line to get equations numbered by (sec.num)
\title{Exclusive heavy vector meson photoproduction on nuclei in NLO perturbative QCD%
\thanks{Presented at XXIX Cracow EPIPHANY Conference on Physics at the Electron-Ion Collider and Future Facilities, Cracow, Poland, 
January 16-19, 2023}%
% you can use '\\' to break lines
}
\author{K.J. Eskola, V. Guzey, T. L\"oyt\"ainen, H. Paukkunen
\address{University of Jyvaskyla, Department of Physics, P.O. Box 35, FI-40014 University of Jyvaskyla, Finland and Helsinki Institute of Physics, P.O. Box 64, FI-00014 University of Helsinki, Finland}
\\[3mm]
{C.A. Flett
\address{Universit{\'e} Paris-Saclay, CNRS, IJCLab, 91405 Orsay, France}
}
}
\maketitle

\begin{abstract}

We make predictions for the cross section of coherent $J/\psi$ photoproduction in Pb-Pb and O-O 
ultraperipheral collisions (UPCs) at the LHC as a function of the $J/\psi$ rapidity $y$  in the framework of collinear factorization and next-to-leading order (NLO) perturbative QCD. We quantify the strong scale dependence and
significant uncertainties due to nuclear PDFs and show that our approach provides a reasonable description of the LHC data on coherent $J/\psi$ photoproduction in Pb-Pb UPCs. 
We demonstrate that these uncertainties are reduced by approximately a factor of 10 
in the scaled ratio of the O-O and Pb-Pb UPC cross sections. Our analysis indicates the dominance of the quark contribution to the UPC cross section at central rapidities, which affects the interpretation of the UPC data. 

\end{abstract}
  
\section{Introduction}
\label{sec:intro}

Studies of exclusive photoproduction of light and heavy vector mesons off protons and nuclei have been given a new impetus by
measurements of so-called ultraperipheral collisions (UPCs) at the Large Hadron Collider (LHC). 
In UPCs, ions (protons, nuclei) pass each other at large impact parameters, the short-range strong interaction between the
colliding hadrons is suppressed, and the reaction proceeds via emission of quasi-real photons, which are usually
treated in the Weizs\"acker-Williams equivalent photon approximation~\cite{Budnev:1975poe}.
At the LHC, the maximal energy of these photons reaches the TeV-range, which makes it effectively the highest energy photon collider to date.
Taking advantage of this, in UPCs these photons can be used as a probe to study open questions of the proton and nucleus structure 
and the strong interaction dynamics in 
quantum chromodynamics (QCD) as well as to search for new physics~\cite{Bertulani:2005ru,Baltz:2007kq,Contreras:2015dqa,Klein:2019qfb}. 

One of the most thoroughly studied UPC processes at the LHC is exclusive photoproduction of heavy vector mesons, in particular,
photoproduction of $J/\psi$. The interest in it is driven by the original observation that in the leading double logarithmic approximation
of perturbative QCD (pQCD),
the $J/\psi$ photoproduction cross section is directly proportional to the small-$x$ gluon density of the target squared~\cite{Ryskin:1992ui}. 

The application of this idea to nuclear targets allows for a direct comparison of the nuclear suppression factor extracted from the data on coherent $J/\psi$ photoproduction in lead-lead (Pb-Pb) UPCs at the LHC with small-$x$ nuclear modifications of 
the nuclear gluon density~\cite{Guzey:2013xba,Guzey:2013qza,Guzey:2020ntc}.
The good agreement with the ALICE data at the $J/\psi$ rapidity $y \approx 0$ 
%vg at $\sqrt{s_{NN}}=2.76$ TeV~\cite{ALICE:2013wjo} and $\sqrt{s_{NN}}=5.02$ TeV~\cite{ALICE:2021gpt}
gives direct evidence of the significant $\sim 40$\% gluon nuclear shadowing (suppression) in lead
%vg $R_g(x,\mu^2)=g_A(x,\mu^2)/[A g_p(x,\mu^2)] \approx 0.6$ 
at $x \approx 6 \times 10^{-4}-10^{-3}$, 
in agreement with the predictions of the leading twist approach~\cite{Frankfurt:2011cs} and the EPPS16 nuclear parton distribution function (nPDFs)~\cite{Eskola:2016oht}.
%($g_A(x,\mu^2)$ and $g_p(x,\mu^2)$ refer to the gluon distributions in the nucleus and the proton, respectively,
%$x$ is the gluon momentum fraction and $\mu={\cal O}(m_c)$ is the resolution scale with 
%$m_c$ being the charm quark mass).

Further progress in obtaining new constraints on proton and nucleus 
parton distribution functions (PDFs) using $J/\psi$ photoproducton in UPCs requires both experimental and theoretical efforts.
One perspective direction is studies of UPCs accompanied by forward neutron emission~\cite{Guzey:2013jaa}, which allows one to separate the high-$W$ and low-$W$ contributions to the UPC cross section at a given $y$ ($W$ is the photon-nucleon invariant energy) and, hence, to probe much lower values of $x$ down to $x \approx 6 \times 10^{-5}$~\cite{CMS:2022nnw}.
On the theory side, significant advances have been made to refine calculations of exclusive $J/\psi$ photoproduction. 
These include an analysis of effects associated with the gluon and charm quark transverse momenta in the high-energy factorization
approach~\cite{Ryskin:1995hz}, the calculation of next-to-leading order (NLO) pQCD corrections in the framework of collinear factorization~\cite{Ivanov:2004vd,Jones:2015nna} and their 
taming using the $Q_0$-subtraction method~\cite{Flett:2019pux}, and estimates of relativistic corrections~\cite{Lappi:2020ufv}
and pre-asymptotic effects~\cite{Frankfurt:1997fj} associated with the charmonium light-cone wave function.

In this paper, we summarize our predictions for coherent $J/\psi$ photoproduction in Pb-Pb and oxygen-oxygen (O-O) UPCs at the LHC in the 
framework of collinear factorization and NLO pQCD~\cite{Eskola:2022vpi,Eskola:2022vaf}. We quantify the strong scale dependence and
significant PDF-related uncertainties of our predictions.  We show that NLO pQCD provides a reasonable description of the 
LHC data on coherent $J/\psi$ photoproduction in Pb-Pb UPCs, and also that the uncertainties are reduced by approximately a factor of 10 in the ratio of the O-O and Pb-Pb UPC cross sections.
A surprising consequence of our analysis is the dominance of the quark 
contribution at central rapidities, which challenges the interpretation of the data in terms of the small-$x$ nuclear gluon distribution.


\section{Exclusive $J/\psi$ photoproduction on nuclei in NLO pQCD}
\label{sec:nlo}

The cross section of exclusive coherent $J/\psi$ photoproduction in Pb-Pb UPCs is given by a sum
of two terms since both of the colliding ions can serve as a photon source and a target~\cite{Baltz:2007kq},
\begin{equation}
\frac{d\sigma}{dy}=\left[k \frac{dN_{\gamma/A}}{dk} \sigma^{\gamma A \to J/\psi A}\right]_{k=k^{+}}+
\left[k \frac{dN_{\gamma/A}}{dk} \sigma^{\gamma A \to J/\psi A}\right]_{k=k^{-}} \,,
\label{eq:upc}
\end{equation}
where $y$ is the $J/\psi$ rapidity and $k dN_{\gamma/A}/dk$ is the photon flux calculated in the equivalent-photon
approximation, 
see details in~\cite{Eskola:2022vpi}.
Therefore, for a given value of $y$, 
there is a two-fold ambiguity in the photon energy, $k^{\pm}=(M_{J/\psi}/2)e^{\pm y}$, where $M_{J/\psi}$ is the $J/\psi$
mass.

The information on the nuclear partonic structure and the strong interaction dynamics is contained in the underlying
$\sigma^{\gamma A \to J/\psi A}$ cross section of $J/\psi$ photoproduction.
Assuming that the charm quark mass $m_c$ provides a hard scale so that collinear factorization for hard exclusive processes is applicable~\cite{Collins:1996fb}, the $\gamma+A \to J/\psi+A$
amplitude to NLO accuracy is given by the following convolution~\cite{Ivanov:2004vd,Jones:2015nna},
\begin{equation}
{\cal M} \propto \sqrt{\langle O_1 \rangle_{V}} \int^{1}_{-1}dx \left[T_g(x,\xi)F_A^g(x,\xi,t,\mu)
+T_q(x,\xi)F_A^{q,S}(x,\xi,t,\mu)\right] \,,
\label{eq:amp}
\end{equation}
where $\langle O_1 \rangle_{V}$ is the non-relativistic QCD (NRQCD) matrix element determined from the $J/\psi \to l^{+} l^{-}$
leptonic decay, $T_g(x,\xi)$ and $T_q(x,\xi)$ are the gluon and quark NLO coefficient functions, and 
$F_A^g(x,\xi,t,\mu)$ and 
\allowbreak
$F_A^{q,S}(x,\xi,t,\mu)$ are the gluon and quark singlet nuclear generalized parton distribution functuons (GPDs),
respectively. The GPDs depend on the light-cone momentum fractions $x$ and $\xi$, the momentum transfer to the target squared $t$,
and the scale $\mu$. 
In our analysis, we take the renormalization and factorization scales to be equal and allow them to vary in the 
interval $m_c < \mu < 2 m_c$.
The derivations in~\cite{Ivanov:2004vd,Jones:2015nna} assume the static approximation for the charmonium wave function~\cite{Hoodbhoy:1996zg}, whose consistency requires that $m_c=M_{J/\psi}/2$. 

In the graphical form, the amplitude of Eq.~(\ref{eq:amp}) is presented in Fig.~\ref{fig:amp} showing gluon (left)
and quark (right) contributions. At leading order (LO), there is only the gluon term, while at NLO both the gluons and quarks 
contribute.
\begin{figure}[htb]
\centerline{%
\includegraphics[width=6cm]{Jpsi_gA.pdf}
\hspace{0.75cm}
\includegraphics[width=6cm]{Jpsi_gA2.pdf}}
\caption{Graphical representation of the $\gamma+A \to J/\psi+A$ amplitude in NLO pQCD: gluon (left) and quark (right) contributions.}
\label{fig:amp}
\end{figure}

The parameter $\xi$ in Eq.~(\ref{eq:amp}) quantifies the imbalance (so-called skewness) of the momentum fractions carried by the partons attached to the target (see Fig.~\ref{fig:amp}) and is fixed by the process kinematics. In the limit of high $W$, 
$\xi \approx M_{J/\psi}^2/(2W^2) \ll 1$, which in general simplifies the modeling of GPDs. Indeed, for sufficiently 
large values of $\mu$, the $\mu^2$ evolution of GPDs almost completely washes out the information on the $\xi$-dependence of the GPDs at 
some low starting scale $\mu_0 \sim 1$ GeV~\cite{Dutrieux:2023qnz}, where they can hence be taken in the $\xi \to 0$ limit. 
As a result, one can relate GPDs to PDFs~\cite{Shuvaev:1999ce}.
%vg and further extend these ideas to successful phenomenological applications for the description of small-$x$ data on exclusive processes~\cite{Kumericki:2009uq}.

In our work, we neglect the $\xi$-dependence of the nuclear GPDs because its effects are significantly smaller than
the theoretical uncertainties associated with the choice of the scale of $\mu$ and those due to nuclear PDFs, which we
consider in detail in Sec.~\ref{sec:results}. Employing the symmetries of GPDs, we use
\begin{eqnarray}
F_A^g(\pm x,\xi,t,\mu) &=&xg_A(x,\mu) F_A(t) \,, \nonumber\\
F_A^{q,S}(\pm x,\xi,t,\mu) &= & \sum_{q=u,d,s,c}\left[\theta(x)q_A(x,\mu)-\theta(-x){\bar q}_A(x,\mu)\right] F_A(t) \,, 
%F_A^{q,S}(-x,\xi,t,\mu) &=&-\sum_{q=u,d,s,c}{\bar q}_A(x,\mu) F_A(t) \,,
\end{eqnarray}
where $x \in [0,1]$, $F_A(t)$ is the nuclear form factor, $g_A(x,\mu)$, $q_A(x,\mu)$ and ${\bar q}_A(x,\mu)$ are nuclear gluon, quark and antiquark PDFs.
We use the Fermi model (Woods-Saxon) for the nuclear density~\cite{DeVries:1987atn} in the calculation of
the nuclear form factor and such state-of-the-art nuclear PDFs as EPPS16~\cite{Eskola:2016oht}, 
nNNPDF2.0~\cite{AbdulKhalek:2020yuc}, EPPS21~\cite{Eskola:2021nhw}, nNNPDF3.0~\cite{AbdulKhalek:2022fyi}, and
nCTEQ15WZSIH~\cite{Duwentaster:2021ioo}.


\section{Predictions for coherent $J/\psi$ photoproduction in Pb-Pb and O-O UPCs at the LHC}
\label{sec:results}

The formalism presented in the previous section allows us to make predictions for the cross sections of coherent $J/\psi$ photoproduction in Pb-Pb and O-O UPCs at the LHC as a function of the $J/\psi$ rapidity $y$.

Figure~\ref{fig:EPPS21_run2_scale} quantifies the scale dependence of our NLO pQCD predictions for 
$d\sigma^{{\rm Pb}+{\rm Pb} \to {\rm Pb}+J/\psi+{\rm Pb}}/dy$ at $\sqrt{s_{NN}}=5.02$ TeV and compares them with all available Run 2 data, 
see~\cite{Eskola:2022vpi} for references. The calculations are carried out with the central EPPS21 nPDFs 
and at the values of the scale $\mu$ spanning the $M_{J/\psi}/2 < \mu < M_{J/\psi}$ range.  One can see from the figure that while the scale dependence is very strong, 
 it is possible to find an optimal scale, $\mu=2.39$ GeV in the considered case, which provides a 
 %vg simultaneously good description of all 
 reasonable (good at $y \sim 0$ and worse at large $|y|$) description   of the LHC data on coherent $J/\psi$ photoproduction in Pb-Pb UPCs, so
 that the data at all $y$ lie within our scale uncertainty envelope.
 

 
 
 Note that while the use of other nuclear PDFs leads to similar results at central rapidities, there are certain differences 
 at large $|y|$. In particular, the nCTEQ15WZSIH nPDFs provide a somewhat better agreement with the forward ALICE and 2018 LHCb data, 
 which may hint on the sensitivity to the strange quark distributions in nuclei.
 
\begin{figure}[tb]
\centerline{%
\includegraphics[width=11.cm]{UpdatedScale_Lead2.pdf}}
\caption{NLO pQCD predictions for the cross section of coherent $J/\psi$ photoproduction
in Pb-Pb UPCs at $\sqrt{s_{NN}}=5.02$ TeV as a function of the $J/\psi$ rapidity $y$
 and their comparison with the LHC Run 2 data. 
The calculations correspond to the EPPS21 nPDFs and $\mu=M_{J/\psi}/2=1.55$ GeV (lower boundary), $\mu=M_{J/\psi}=3.1$ GeV (upper boundary), and $\mu=2.39$ GeV (optimal scale). From~\cite{Eskola:2022vaf}.}
\label{fig:EPPS21_run2_scale}
\end{figure}


In Fig.~\ref{fig:EPPS21_run2_error}, we show our predictions for $d\sigma^{{\rm Pb}+{\rm Pb} \to {\rm Pb}+J/\psi+{\rm Pb}}/dy$ at $\sqrt{s_{NN}}=5.02$ TeV using different nPDFs. The curves correspond to
the central EPPS21 (blue solid), nCTEQ15WZSIH (red dashed), 
and nNNPDF3.0 (green dotted) nPDFs along with their propagated uncertainties given by the respective shaded bands.
The calculations are performed at the optimal scales corresponding to each PDF set; the quality of the agreement with the data is illustrated by a comparison with the Run 2 data. One can see from the figure that the uncertainties due to nuclear PDFs are quite significant and exceed the experimental errors. Hence, this can be viewed as an opportunity to improve the determination of nuclear PDFs using the data 
on $J/\psi$ photoproduction in nucleus-nucleus UPCs.

Note that compared to our results in~\cite{Eskola:2022vpi}, the abnormally large uncertainty associated with the EPPS16 
nPDFs (in fact with the CT14 baseline PDFs) disappears, when using the more recent EPPS21 nPDFs. Also, since the nNNPDF3.0 nPDFs correspond to a much less constrained fit, the corresponding nuclear PDFs uncertainties are the largest among those shown in Fig.~\ref{fig:EPPS21_run2_error}.
Note also that the nCTEQ15WZSIH errors do not include the free proton PDF uncertainties.

\begin{figure}[tb]
\centerline{%
\includegraphics[width=11.cm]{UpdatedPDF_Lead2.pdf}}
\caption{NLO pQCD predictions for 
$d\sigma^{{\rm Pb}+{\rm Pb} \to {\rm Pb}+J/\psi+{\rm Pb}}/dy$ at 5.02 TeV
%vg $\sqrt{s_{NN}}=5.02$ TeV 
as a function of the $J/\psi$ rapidity $y$ for EPPS21, nNNPDF3.0, and nCTEQ15WZSIH nPDFs, see text for details. For comparison, the Run 2 LHC data are also shown.  From~\cite{Eskola:2022vaf}.}
%vgThe curves correspond to the central nPDFs at the appriate
%optimal scales and the shaded bands show the propagated nPDF uncertainties. For comparison, the Run 2 LHC data are also shown. }
\label{fig:EPPS21_run2_error}
\end{figure}

Anticipating the oxygen-oxygen (O-O) run at the LHC, we also made predictions for the cross section of coherent $J/\psi$ photoproduction in O-O UPCs for four collision energies, $\sqrt{s_{NN}}=2.76$, 5.02, 6.37 and 7 TeV. Our results for 
$d\sigma^{{\rm O}+{\rm O} \to {\rm O}+J/\psi+{\rm O}}/dy$ as a function of $y$ are shown in Fig.~\ref{fig:oxygen}. The calculations correspond 
to using the central EPPS21 nPDFs as input and a range of scales 
%$\mu$ in the 
$m_c < \mu < 2 m_c$. 
%interval. 
As in the case of Pb-Pb UPCs, we find that the scale dependence and the uncertainty due to nPDFs are large.

\begin{figure}[tb]
\centerline{%
\includegraphics[width=12.5cm]{OO_Scales2.pdf}}
\caption{NLO pQCD predictions for the cross section of coherent $J/\psi$ photoproduction in O-O UPCs at $\sqrt{s_{NN}}=2.76$, 5.02, 6.37 and 7 TeV as a function of the $J/\psi$ rapidity $y$.
 The curves correspond to the predictions based on central EPPS21 nPDFs and several values of 
 %vg the scale 
 $\mu$ in the interval of
$m_c < \mu < 2 m_c$. From~\cite{Eskola:2022vaf}.
}
\label{fig:oxygen}
\end{figure}

One can reduce these uncertainties by considering the following scaled ratio of the oxygen-oxygen to lead-lead UPC cross sections, 
\begin{eqnarray}
\tilde{R}^{{\rm O}/{\rm Pb}} &=& \left(\frac{208 Z_{\rm Pb}}{16 Z_{\rm O}} \right)^2 \frac{d\sigma({\rm O}+{\rm O} \to {\rm O}+J/\psi+{\rm O})/dy}{d\sigma({\rm Pb}+{\rm Pb} \to {\rm Pb}+J/\psi+{\rm Pb})/dy} \nonumber\\
&=& \left(\frac{208 Z_{\rm Pb}}{16 Z_{\rm O}} \right)^2 R^{{\rm O}/{\rm Pb}}\,,
\label{eq:R}
\end{eqnarray}
where $Z_{\rm Pb}=79$ and $Z_{\rm O}=8$ are the nucleus electric charges. The ratios $\tilde{R}^{{\rm O}/{\rm Pb}}$ as a function of $y$ for $\sqrt{s_{NN}}=2.76$, 5.02, 6.37 and 7 TeV are shown in Fig.~\ref{fig:oxygen_ratio}. 
For the labeling of the curves, see Fig.~\ref{fig:oxygen}.
A comparison with the results in 
Figs.~\ref{fig:EPPS21_run2_scale} and \ref{fig:oxygen}
 demonstrates that in the ratio of the cross sections for different nuclei, 
 the scale dependence is reduced by approximately a factor of 10 compared to the individual Pb-Pb and O-O cross sections.
This can be explained by the fact that since the NLO coefficient functions for the oxygen and lead targets are the same, the 
differences come mostly from the oxygen and lead nPDFs. Additionally, there is a  partial cancellation of uncertainties associated with proton PDFs.

\begin{figure}[tb]
\centerline{%
\includegraphics[width=12.5cm]{EPPS21_ScaleUncert.pdf}}
\caption{The scaled ratio of the O-O and Pb-Pb UPC cross sections, see Eq.~(\ref{eq:R}), as a function of the
$J/\psi$ rapidity $y$ for four values of $\sqrt{s_{NN}}$. See Fig.~\ref{fig:oxygen} for the labeling of the curves.
From~\cite{Eskola:2022vaf}.
}
\label{fig:oxygen_ratio}
\end{figure}

\begin{figure}[htb]
\centerline{%
\includegraphics[width=12.5cm]{EPPS_oGoQ.pdf}}
\caption{The separate quark, gluon and their interference contributions to
$d\sigma^{{\rm Pb}+{\rm Pb} \to {\rm Pb}+J/\psi+{\rm Pb}}/dy$ as a function of $y$ at $\sqrt{s_{NN}}=5.02$ TeV.
From~\cite{Eskola:2022vpi}.}
\label{fig:quark}
\end{figure}

One of the most striking conclusions of our analyses is the observation that the quark contribution dominates the 
cross section of $J/\psi$ photoproduction in Pb-Pb and O-O UPCs at non-forward rapidities, which originates from
the strong cancellations between LO and NLO gluon terms in Eq.~(\ref{eq:amp}). This is illustrated in 
Fig.~\ref{fig:quark} showing the separate quark, gluon and their interference contributions to
$d\sigma^{{\rm Pb}+{\rm Pb} \to {\rm Pb}+J/\psi+{\rm Pb}}/dy$ as a function of $y$ at $\sqrt{s_{NN}}=5.02$ TeV.
The curves correspond to the EPPS16 nPDFs and $\mu=2.37$ GeV is the 
%vg corresponding 
optimal scale.
One can see from the figure that the quarks provide the dominant contribution for $|y| < 2$.


At face value, this changes the interpretation of the data on coherent $J/\psi$ photoproduction in heavy-ion UPCs as a probe of small-$x$ nuclear gluons and replaces it with an observation that both the magnitude and shape of the rapidity dependence
of the UPC cross section depend on an interplay 
%vvg and 
of
the quark and gluon contributions. While this is most likely a feature of NLO pQCD, we anticipate that at 
next-to-next-leading order (NNLO), mixing of quark terms at both NLO and NNLO should also play a role.
%interference terms wholly in the quark sector should also play a role.
%now play a role in addition.



\section{Conclusions}

Using the framework of collinear factorization and NLO pQCD, we made predictions for the rapidity dependence of the cross section 
of coherent $J/\psi$ photoproduction in Pb-Pb and O-O UPCs at the LHC. We quantified the strong scale dependence and
significant uncertainty due to nuclear PDFs of our predictions and showed that they provide a reasonable description of the 
LHC data on coherent $J/\psi$ photoproduction in Pb-Pb UPCs. We demonstrated that these uncertainties are dramatically reduced 
in the scaled ratio of the O-O and Pb-Pb UPC cross sections. Also, our analysis indicated the dominance of the quark contribution to the UPC cross section at central rapidities, so the interpretation of the UPC data must be taken with care.
%vgwhich affects the interpretation of the UPC data. 


\section*{Acknowledgements}
We acknowledge the financial support from the Magnus Ehrnrooth foundation (T.L.), the Academy of Finland
projects 308301 (H.P.) and 330448 (K.J.E.). This research was funded as part of the Center of Excellence in
Quark Matter of the Academy of Finland (projects 346325 and 346326)
%This research is part 
and of the European
Research Council project ERC-2018-ADG-835105 YoctoLHC.

\begin{thebibliography}{99}

%\cite{Budnev:1975poe}
\bibitem{Budnev:1975poe}
V.~M.~Budnev, I.~F.~Ginzburg, G.~V.~Meledin and V.~G.~Serbo,
%``The Two photon particle production mechanism. Physical problems. Applications. Equivalent photon approximation,''
Phys. Rept. \textbf{15} (1975), 181-281.
%doi:10.1016/0370-1573(75)90009-5
%1396 citations counted in INSPIRE as of 08 Mar 2023

%\cite{Bertulani:2005ru}
\bibitem{Bertulani:2005ru}
C.~A.~Bertulani, S.~R.~Klein and J.~Nystrand,
%``Physics of ultra-peripheral nuclear collisions,''
Ann. Rev. Nucl. Part. Sci. \textbf{55} (2005), 271-310.
%doi:10.1146/annurev.nucl.55.090704.151526
%[arXiv:nucl-ex/0502005 [nucl-ex]].
%433 citations counted in INSPIRE as of 08 Mar 2023

%\cite{Baltz:2007kq}
\bibitem{Baltz:2007kq}
A.~J.~Baltz \textit{et al.},
%G.~Baur, D.~d'Enterria, L.~Frankfurt, F.~Gelis, V.~Guzey, K.~Hencken, Y.~Kharlov, M.~Klasen and S.~R.~Klein, \textit{et al.}
%``The Physics of Ultraperipheral Collisions at the LHC,''
Phys. Rept. \textbf{458} (2008), 1-171.
%doi:10.1016/j.physrep.2007.12.001
%[arXiv:0706.3356 [nucl-ex]].
%570 citations counted in INSPIRE as of 08 Mar 2023

%\cite{Contreras:2015dqa}
\bibitem{Contreras:2015dqa}
J.~G.~Contreras and J.~D.~Tapia Takaki,
%``Ultra-peripheral heavy-ion collisions at the LHC,''
Int. J. Mod. Phys. A \textbf{30} (2015), 1542012.
%doi:10.1142/S0217751X15420129
%110 citations counted in INSPIRE as of 08 Mar 2023

%\cite{Klein:2019qfb}
\bibitem{Klein:2019qfb}
S.~R.~Klein and H.~M\"antysaari,
%``Imaging the nucleus with high-energy photons,''
Nature Rev. Phys. \textbf{1} (2019) no.11, 662-674.
%doi:10.1038/s42254-019-0107-6
%[arXiv:1910.10858 [hep-ex]].
%61 citations counted in INSPIRE as of 08 Mar 2023

%\cite{Ryskin:1992ui}
\bibitem{Ryskin:1992ui}
M.~G.~Ryskin,
%``Diffractive J / psi electroproduction in LLA QCD,''
Z. Phys. C \textbf{57} (1993), 89-92.
%doi:10.1007/BF01555742
%544 citations counted in INSPIRE as of 08 Mar 2023

%\cite{Guzey:2013xba}
\bibitem{Guzey:2013xba}
V.~Guzey, E.~Kryshen, M.~Strikman and M.~Zhalov,
%``Evidence for nuclear gluon shadowing from the ALICE measurements of PbPb ultraperipheral exclusive $J/{\psi}$ production,''
Phys. Lett. B \textbf{726} (2013), 290-295.
%doi:10.1016/j.physletb.2013.08.043
%[arXiv:1305.1724 [hep-ph]].
%101 citations counted in INSPIRE as of 08 Mar 2023

%\cite{Guzey:2013qza}
\bibitem{Guzey:2013qza}
V.~Guzey and M.~Zhalov,
%``Exclusive $J/{\psi}$ production in ultraperipheral collisions at the LHC: constrains on the gluon distributions in the proton and nuclei,''
JHEP \textbf{10} (2013), 207.
%doi:10.1007/JHEP10(2013)207
%[arXiv:1307.4526 [hep-ph]].
%92 citations counted in INSPIRE as of 08 Mar 2023

%\cite{Guzey:2020ntc}
\bibitem{Guzey:2020ntc}
V.~Guzey, E.~Kryshen, M.~Strikman and M.~Zhalov,
%``Nuclear suppression from coherent $J /\psi$ photoproduction at the Large Hadron Collider,''
Phys. Lett. B \textbf{816} (2021), 136202.
%doi:10.1016/j.physletb.2021.136202
%[arXiv:2008.10891 [hep-ph]].
%30 citations counted in INSPIRE as of 08 Mar 2023

%\cite{ALICE:2013wjo}
%\bibitem{ALICE:2013wjo}
%E.~Abbas \textit{et al.} [ALICE],
%``Charmonium and $e^+e^-$ pair photoproduction at mid-rapidity in ultra-peripheral Pb-Pb collisions at $\sqrt{s_{\rm NN}}$=2.76 TeV,''
%Eur. Phys. J. C \textbf{73} (2013) no.11, 2617.
%doi:10.1140/epjc/s10052-013-2617-1
%[arXiv:1305.1467 [nucl-ex]].
%293 citations counted in INSPIRE as of 08 Mar 2023

%\cite{ALICE:2021gpt}
%\bibitem{ALICE:2021gpt}
%S.~Acharya \textit{et al.} [ALICE],
%``Coherent $J/\psi$ and $\psi'$ photoproduction at midrapidity in ultra-peripheral Pb-Pb collisions at $\sqrt{s_{\mathrm{NN}}}~=~5.02$ TeV,''
%Eur. Phys. J. C \textbf{81} (2021) no.8, 712.
%doi:10.1140/epjc/s10052-021-09437-6
%[arXiv:2101.04577 [nucl-ex]].
%42 citations counted in INSPIRE as of 08 Mar 2023

%\cite{Frankfurt:2011cs}
\bibitem{Frankfurt:2011cs}
L.~Frankfurt, V.~Guzey and M.~Strikman,
%``Leading Twist Nuclear Shadowing Phenomena in Hard Processes with Nuclei,''
Phys. Rept. \textbf{512} (2012), 255-393.
%doi:10.1016/j.physrep.2011.12.002
%[arXiv:1106.2091 [hep-ph]].
%186 citations counted in INSPIRE as of 06 Mar 2023

%\cite{Eskola:2016oht}
\bibitem{Eskola:2016oht}
K.~J.~Eskola, P.~Paakkinen, H.~Paukkunen and C.~A.~Salgado,
%``EPPS16: Nuclear parton distributions with LHC data,''
Eur. Phys. J. C \textbf{77} (2017) no.3, 163.
%doi:10.1140/epjc/s10052-017-4725-9
%[arXiv:1612.05741 [hep-ph]].
%532 citations counted in INSPIRE as of 01 Mar 2023

%\cite{Guzey:2013jaa}
\bibitem{Guzey:2013jaa}
V.~Guzey, M.~Strikman and M.~Zhalov,
%``Disentangling coherent and incoherent quasielastic $J/\psi$ photoproduction on nuclei by neutron tagging in ultraperipheral ion collisions at the LHC,''
Eur. Phys. J. C \textbf{74} (2014) no.7, 2942.
%doi:10.1140/epjc/s10052-014-2942-z
%[arXiv:1312.6486 [hep-ph]].
%50 citations counted in INSPIRE as of 08 Mar 2023

%\cite{CMS:2022nnw}
\bibitem{CMS:2022nnw}
 [CMS],
%``Photon-nucleus energy dependence of coherent J/$\psi$ cross session in ultraperipheral PbPb collisions at $5.02~\mathrm{TeV}$ with CMS,''
CMS-PAS-HIN-22-002.
%1 citations counted in INSPIRE as of 08 Mar 2023

%\cite{Ryskin:1995hz}
\bibitem{Ryskin:1995hz}
M.~G.~Ryskin, R.~G.~Roberts, A.~D.~Martin and E.~M.~Levin,
%``Diffractive J / psi photoproduction as a probe of the gluon density,''
Z. Phys. C \textbf{76} (1997), 231-239.
%doi:10.1007/s002880050547
%[arXiv:hep-ph/9511228 [hep-ph]].
%223 citations counted in INSPIRE as of 08 Mar 2023

%\cite{Ivanov:2004vd}
\bibitem{Ivanov:2004vd}
D.~Y.~Ivanov, A.~Schafer, L.~Szymanowski and G.~Krasnikov,
%``Exclusive photoproduction of a heavy vector meson in QCD,''
Eur. Phys. J. C \textbf{34} (2004) no.3, 297-316
[erratum: Eur. Phys. J. C \textbf{75} (2015) no.2, 75].
%doi:10.1140/epjc/s2004-01712-x
%[arXiv:hep-ph/0401131 [hep-ph]].
%108 citations counted in INSPIRE as of 08 Mar 2023

%\cite{Jones:2015nna}
\bibitem{Jones:2015nna}
S.~P.~Jones, A.~D.~Martin, M.~G.~Ryskin and T.~Teubner,
%``Exclusive $J/\psi$ and $\Upsilon$ photoproduction and the low $x$ gluon,''
J. Phys. G \textbf{43} (2016) no.3, 035002.
%doi:10.1088/0954-3899/43/3/035002
%[arXiv:1507.06942 [hep-ph]].
%57 citations counted in INSPIRE as of 08 Mar 2023

%\cite{Flett:2019pux}
\bibitem{Flett:2019pux}
C.~A.~Flett, S.~P.~Jones, A.~D.~Martin, M.~G.~Ryskin and T.~Teubner,
%``How to include exclusive $J/\psi$ production data in global PDF analyses,''
Phys. Rev. D \textbf{101} (2020) no.9, 094011.
%doi:10.1103/PhysRevD.101.094011
%[arXiv:1908.08398 [hep-ph]].
%29 citations counted in INSPIRE as of 08 Mar 2023

%\cite{Lappi:2020ufv}
\bibitem{Lappi:2020ufv}
T.~Lappi, H.~M\"antysaari and J.~Penttala,
%``Relativistic corrections to the vector meson light front wave function,''
Phys. Rev. D \textbf{102} (2020) no.5, 054020.
%doi:10.1103/PhysRevD.102.054020
%[arXiv:2006.02830 [hep-ph]].
%37 citations counted in INSPIRE as of 08 Mar 2023

%\cite{Frankfurt:1997fj}
\bibitem{Frankfurt:1997fj}
L.~Frankfurt, W.~Koepf and M.~Strikman,
%``Diffractive heavy quarkonium photoproduction and electroproduction in QCD,''
Phys. Rev. D \textbf{57} (1998), 512-526.
%doi:10.1103/PhysRevD.57.512
%[arXiv:hep-ph/9702216 [hep-ph]].
%220 citations counted in INSPIRE as of 08 Mar 2023

%\cite{Eskola:2022vpi}
\bibitem{Eskola:2022vpi}
K.~J.~Eskola, C.~A.~Flett, V.~Guzey, T.~L\"oyt\"ainen and H.~Paukkunen,
%``Exclusive J/\ensuremath{\psi} photoproduction in ultraperipheral Pb+Pb collisions at the CERN Large Hadron Collider calculated at next-to-leading order perturbative QCD,''
Phys. Rev. C \textbf{106} (2022) no.3, 035202.
%doi:10.1103/PhysRevC.106.035202
%[arXiv:2203.11613 [hep-ph]].
%16 citations counted in INSPIRE as of 08 Mar 2023

%\cite{Eskola:2022vaf}
\bibitem{Eskola:2022vaf}
K.~J.~Eskola, C.~A.~Flett, V.~Guzey, T.~L\"oyt\"ainen and H.~Paukkunen,
%``Next-to-leading order perturbative QCD predictions for exclusive $J/\psi$ photoproduction in oxygen-oxygen and lead-lead collisions at the LHC,''
[arXiv:2210.16048 [hep-ph]].
%2 citations counted in INSPIRE as of 08 Mar 2023

%\cite{Collins:1996fb}
\bibitem{Collins:1996fb}
J.~C.~Collins, L.~Frankfurt and M.~Strikman,
%``Factorization for hard exclusive electroproduction of mesons in QCD,''
Phys. Rev. D \textbf{56} (1997), 2982.
%-3006.
%doi:10.1103/PhysRevD.56.2982
%[arXiv:hep-ph/9611433 [hep-ph]].
%979 citations counted in INSPIRE as of 08 Mar 2023

%\cite{Hoodbhoy:1996zg}
\bibitem{Hoodbhoy:1996zg}
P.~Hoodbhoy,
%``Wave function corrections and off forward gluon distributions in diffractive J / psi electroproduction,''
Phys. Rev. D \textbf{56} (1997), 388-393.
%doi:10.1103/PhysRevD.56.388
%[arXiv:hep-ph/9611207 [hep-ph]].
%73 citations counted in INSPIRE as of 20 Mar 2023

%\cite{Dutrieux:2023qnz}
\bibitem{Dutrieux:2023qnz}
H.~Dutrieux, M.~Winn and V.~Bertone,
%``Exclusive meets inclusive at small Bjorken-$x_B$: how to relate exclusive measurements to PDFs based on evolution equations,''
[arXiv:2302.07861 [hep-ph]].
%1 citations counted in INSPIRE as of 11 Mar 2023

%\cite{Shuvaev:1999ce}
\bibitem{Shuvaev:1999ce}
A.~G.~Shuvaev, K.~J.~Golec-Biernat, A.~D.~Martin and M.~G.~Ryskin,
%``Off diagonal distributions fixed by diagonal partons at small x and xi,''
Phys. Rev. D \textbf{60} (1999), 014015.
%doi:10.1103/PhysRevD.60.014015
%[arXiv:hep-ph/9902410 [hep-ph]].
%282 citations counted in INSPIRE as of 26 Jan 2023

%\cite{Kumericki:2009uq}
%\bibitem{Kumericki:2009uq}
%K.~Kumeri\v{c}ki and D.~Mueller,
%``Deeply virtual Compton scattering at small $x_B$ and the access to the GPD H,''
%Nucl. Phys. B \textbf{841}, 1-58 (2010).
%doi:10.1016/j.nuclphysb.2010.07.015
%[arXiv:0904.0458 [hep-ph]].
%208 citations counted in INSPIRE as of 11 Mar 2023

%\cite{DeVries:1987atn}
\bibitem{DeVries:1987atn}
H.~De Vries, C.~W.~De Jager and C.~De Vries,
%``Nuclear charge and magnetization density distribution parameters from elastic electron scattering,''
Atom. Data Nucl. Data Tabl. \textbf{36}, 495-536 (1987).
%doi:10.1016/0092-640X(87)90013-1
%1753 citations counted in INSPIRE as of 11 Mar 2023

%\cite{Eskola:2016oht}
%\bibitem{Eskola:2016oht}
%K.~J.~Eskola, P.~Paakkinen, H.~Paukkunen and C.~A.~Salgado,
%``EPPS16: Nuclear parton distributions with LHC data,''
%Eur. Phys. J. C \textbf{77}, no.3, 163 (2017).
%doi:10.1140/epjc/s10052-017-4725-9
%[arXiv:1612.05741 [hep-ph]].
%533 citations counted in INSPIRE as of 11 Mar 2023

%\cite{AbdulKhalek:2020yuc}
\bibitem{AbdulKhalek:2020yuc}
R.~Abdul Khalek, J.~J.~Ethier, J.~Rojo and G.~van Weelden,
%``nNNPDF2.0: quark flavor separation in nuclei from LHC data,''
JHEP \textbf{09}, 183 (2020).
%doi:10.1007/JHEP09(2020)183
%[arXiv:2006.14629 [hep-ph]].
%83 citations counted in INSPIRE as of 11 Mar 2023

%\cite{Eskola:2021nhw}
\bibitem{Eskola:2021nhw}
K.~J.~Eskola, P.~Paakkinen, H.~Paukkunen and C.~A.~Salgado,
%``EPPS21: a global QCD analysis of nuclear PDFs,''
Eur. Phys. J. C \textbf{82}, no.5, 413 (2022).
%doi:10.1140/epjc/s10052-022-10359-0
%[arXiv:2112.12462 [hep-ph]].
%35 citations counted in INSPIRE as of 11 Mar 2023

%\cite{AbdulKhalek:2022fyi}
\bibitem{AbdulKhalek:2022fyi}
R.~Abdul Khalek, R.~Gauld, T.~Giani, E.~R.~Nocera, T.~R.~Rabemananjara and J.~Rojo,
%``nNNPDF3.0: evidence for a modified partonic structure in heavy nuclei,''
Eur. Phys. J. C \textbf{82}, no.6, 507 (2022).
%doi:10.1140/epjc/s10052-022-10417-7
%[arXiv:2201.12363 [hep-ph]].
%34 citations counted in INSPIRE as of 11 Mar 2023

%\cite{Kovarik:2015cma}
%\bibitem{Kovarik:2015cma}
%K.~Kovarik, A.~Kusina, T.~Jezo, D.~B.~Clark, C.~Keppel, F.~Lyonnet, J.~G.~Morfin, F.~I.~Olness, J.~F.~Owens and I.~Schienbein, \textit{et al.}
%``nCTEQ15 - Global analysis of nuclear parton distributions with uncertainties in the CTEQ framework,''
%Phys. Rev. D \textbf{93}, no.8, 085037 (2016).
%doi:10.1103/PhysRevD.93.085037
%[arXiv:1509.00792 [hep-ph]].
%488 citations counted in INSPIRE as of 11 Mar 2023

%\cite{Duwentaster:2021ioo}
\bibitem{Duwentaster:2021ioo}
P.~Duwent\"aster \textit{et al.},
%, L.~A.~Husov\'a, T.~Je\v{z}o, M.~Klasen, K.~Kova\v{r}\'\i{}k, A.~Kusina, K.~F.~Muzakka, F.~I.~Olness, I.~Schienbein and J.~Y.~Yu,
%``Impact of inclusive hadron production data on nuclear gluon PDFs,''
Phys. Rev. D \textbf{104} (2021), 094005.
%doi:10.1103/PhysRevD.104.094005
%[arXiv:2105.09873 [hep-ph]].
%20 citations counted in INSPIRE as of 20 Mar 2023


\end{thebibliography}


\end{document}

