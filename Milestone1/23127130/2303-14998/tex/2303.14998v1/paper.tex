% This is samplepaper.tex, a sample chapter demonstrating the
% LLNCS macro package for Springer Computer Science proceedings;
% Version 2.20 of 2017/10/04
%
\documentclass[runningheads]{llncs}
%
\usepackage{graphicx}
\usepackage{amsmath,amssymb}
\DeclareRobustCommand{\bbone}{\text{\usefont{U}{bbold}{m}{n}1}}
\DeclareMathOperator{\EX}{\mathbb{E}}
\usepackage{color}

% Used for displaying a sample figure. If possible, figure files should
% be included in EPS format.
%
% If you use the hyperref package, please uncomment the following line
% to display URLs in blue roman font according to Springer's eBook style:
% \renewcommand\UrlFont{\color{blue}\rmfamily}

\newcommand{\repeatthanks}{\textsuperscript{\thefootnote}}

\begin{document}
%
\title{Multi-view Cross-Modality MR Image Translation for Vestibular Schwannoma and Cochlea Segmentation}
\titlerunning{Multi-view Cross-Modality MR Image Translation}
\author{
Bogyeong Kang \and
Hyeonyeong Nam \and
Ji-Wung Han \and
\\Keun-Soo Heo \and
Tae-Eui Kam\thanks{Corresponding author.}
}
%
\authorrunning{B. Kang et al.}
% First names are abbreviated in the running head.
% If there are more than two authors, 'et al.' is used.
%
\institute{
Department of Artificial Intelligence, Korea University, Seoul, Republic of Korea \\ \email{\{kangbk,kamte\}@korea.ac.kr}}
\maketitle              % typeset the header of the contribution
%
\begin{abstract}
In this work, we propose a multi-view image translation framework, which can translate contrast-enhanced $\text{T}_1$ (ce$\text{T}_1$) MR imaging to high-resolution $\text{T}_2$ (hr$\text{T}_2$) MR imaging for unsupervised vestibular schwannoma and cochlea segmentation. We adopt two image translation models in parallel that use a pixel-level consistent constraint and a patch-level contrastive constraint, respectively. Thereby, we can augment pseudo-hr$\text{T}_2$ images reflecting different perspectives, which eventually lead to a high-performing segmentation model. Our experimental results on the CrossMoDA challenge show that the proposed method achieved enhanced performance on the vestibular schwannoma and cochlea segmentation.

%
\keywords{Multi-view image translation \and Cross-modality \and MRI segmentation \and Unsupervised domain adaptation.}

\end{abstract}

% Importance and appeal of children's drawings
Children's depictions of the human figure are highly expressive and varied.
As one of the very first subjects children attempt to draw, the representation begins as an almost unintelligible cloud of scribbles. 
As the child grows, their representation of the human figure becomes more developed and is extended to graphically represent many different types of characters: people, animals, and even personified objects (see Figure 1).

Who among us has not wished, either as a child or as an adult, to see such figures come to life and move around on the page?
Sadly, while it is relatively fast to produce a single drawing, creating the sequence of images necessary for animation is a much more tedious endeavor, requiring discipline, skill, patience, and sometimes complicated software.
As a result, most of these figures remain static upon the page.

% We built a system to animate them.
Inspired by the importance and appeal of the drawn human figure, we design and build a system to automatically animate it given an in-the-wild photograph of a child's drawing. 
Our system is fast, intuitive, and robust to much of the variation present in these types of drawings, making it well-suited to allow our target audience--children--to see their own characters coming to life.
The system is comprised of four stages: figure detection, segmentation masking, pose estimation/rigging, and animation. 
We describe each stage and identify common causes of failure in each. 
For object detection and pose estimation, we make use of existing computer vision models designed to detect human figures and joints in photographs; we fine-tune these models for use with children's drawings.
For segmentation, we present a straightforward, image processing-based method that, for animation purposes, is more useful and accurate than segmentation masks obtained from a fine-tuned object detection model.
During the animation step, we take advantage of the \textit{twisted perspective} commonly seen in children’s drawings to retarget motion capture data onto the character in a novel and appealing way.

% We use existing machine learning models. However, given the wide domain gap it's not clear how much fine-tuning data was needed. So we ran some experiments to find out and report it.
While our system leverages existing models and techniques, most are not directly applicable to the task due to the many differences between photographic images and simple pen and paper representations. 
To this end, we couple the presentation of our system with a set of experiments exploring the relationship between fine-tuning training set size and success rates.
We also include a perceptual study validating viewer preference for incorporating \textit{twisted perspective} into the motion retargeting step.

We validate the desirability and appeal of our system by building and publicly releasing a version of it as the \AD Demo \,\cite{animateddrawings}.
Launched in December 2021, this demo has been used by millions of people around the world to animate their children's drawings.
Inspired by this reception, our second contribution is The Amateur Drawings Dataset: \hjs{180,000 drawings and user-accepted annotations collected, with consent, through the demo. See Section \ref{sec:UI} for a description of how the annotations were generated.}
We believe this dataset will be a resource to researchers from various fields seeking to better understand the space of amateur drawings, evaluate new algorithms in this domain, or develop new drawing-based tools in general.

To summarize, our contributions are as follows:
\begin{enumerate}
    \item 
    We explore the problem of automatic sketch-to-animation for children's drawings of human figures and present a framework that achieves this effect. We also present a set of experiments determining the amount of training data necessary to achieve high levels of success and a perceptual study validating the usefulness of our motion retargeting technique.
    \item To encourage additional research in the domain of amateur drawings, we present a first-of-its-kind dataset of 180,000 user-submitted amateur drawings, along with user-accepted bounding box, segmentation mask, and joint location annotations.
\end{enumerate}

Upon acceptance of this paper, we plan to publicly release the Amateur Drawings Dataset, project code, and fine-tuned model weights.


\section{Related Works}
% \FZ{can rw be shortened a bit?}\WW{removed a few sentences, will come back and remove more if needed}

\paragraph{Neural Radiance Fields} 
NeRF~\cite{nerf} is a 5D plenoptic function that models volume density and view-dependent appearance. Its differentiable volume rendering allows robust image-based 3D reconstruction and motivated an explosion of related works in areas including high-quality novel view synthesis~\cite{mipnerf360, refnerf, hdrnerf, nerf++}, 3D asset synthesis~\cite{gram, graf}, and efficient reconstruction and rendering~\cite{instantNGP, kilonerf, plenoxels, directvoxgo, plenoctrees}. Despite their outstanding novel view synthesis performance, many analyses suggest their geometry tends to produce artifacts such as sparse density floaters and inner volume~\cite{mipnerf360, refnerf, nerf++}. 
% Marching cube~\cite{marching_cube} 
Direct surface extraction on the density field hence suffers from those artifacts, whereas the depth extraction method does not guarantee watertight surfaces and requires additional surface reconstruction \cite{tsdf, alpha_shape, poisson_recon}
% such as TSDF fusion~\cite{tsdf}, poison surface reconstruction~\cite{poisson_recon} or alpha shapes~\cite{alpha_shape}
. UNISURF~\cite{UNISURF} attempts to mitigate this issue by gradually transiting from volume rendering to surface rendering, but its surface is tightly coupled as a fixed level set on the opacity field and it still only works on solid objects. Our approach deviates from this by using a separate implicit surface field with decoupled opacity to model semi-transparent and thin surfaces.

\vspace{-5pt}
\paragraph{SDF For Multi-view 3D Reconstruction} 
% To reconstruct well-defined surfaces from multi-view images, SDF has been widely utilized with various differentiable rendering techniques, which map 3D representations into 2D images in a differentiable way and therefore allow optimization via photometric loss.
SDF has been extensively employed with differentiable rendering methods to reconstruct surfaces from multi-view images. 
SDFDiff~\cite{sdfdiff} uses a voxel grid and trilinear interpolation to represent the SDF and develops differentiable sphere tracing to find an estimated intersection. 
Our approach deviates from this by using a closed-form solution to solve for \textit{exact} intersections with a more generalized implicit surface field, while also finding more than just the nearest intersections.
% supporting differentiable multi-surface rendering through alpha compositing. 
IDR~\cite{idr} proposes a differentiable sphere-tracing algorithm and optimizes the surface together with a volumetric BRDF. 
% However, this rendering procedure only has gradients defined at first intersections, causing the optimization to be difficult at depth discontinuities. 
VolSDF~\cite{volsdf} maps SDF to volume density via Laplacian CDF and optimizes the SDF via volume rendering.
% within a derived error bound on the approximated transparency function. 
NeuS~\cite{neus} similarly maps an SDF to unbiased weights in the volume rendering equation via a logistic sigmoid function. 
% With a trainable parameter that controls the spread of the weights around the zero-level set, it starts as a coarse surface with "spread-out" effects and converges to a concrete surface in the end. 
% This means their optimization still assumes surface solidity, and hence struggles on semi-transparent or thin surfaces. 
% Although VolSDF and NeuS both incorporate volume rendering in the optimization of SDF, they still assume surface solidity and encourage the rendering weights to become $1$ on the surface to fully occlude the ray. They hence still struggle on semi-transparent or thin surfaces. 
NeuS has motivated several further applications in different areas, such as sparse view surface reconstruction~\cite{sparse_neus}, fast reconstruction~\cite{neus2, voxel_surf, hash_sdf}, and finer details modeling such as HFS~\cite{HFS}, which applies a mapping from SDF to transparency and incorporates an additional displacement field to improve the reconstruction of fine details. 
A crucial and common limitation in existing SDF optimization methods is the assumption of surface solidity. Even the methods that utilize volume rendering in surface reconstruction, such as VolSDF and NeuS, still enforce convergence to solid surfaces in the end. Hence, they cannot properly reconstruct semi-transparent surfaces and suffer from thin structures with strong blending effects. 
% In comparison, our approach does not assume solid surfaces and can model both semi-transparent and thin objects.







\section{Proposed Method}
\subsection{Overview}
Fig.~\ref{fig1} shows an overview of our proposed framework, which consists of three parts; (1) multi-view image translation, (2) segmentation model training, and (3) self-training. Specifically, we first generate the pseudo-hr$\text{T}_2$ images with various characteristics through multi-view image translation. After that, we train the segmentation model using the multi-view pseudo-hr$\text{T}_2$ images and the labels of the ce$\text{T}_1$ images. In the self-training, the trained segmentation model first performs pseudo-labeling of real hr$\text{T}_2$  images, and then is further trained by including the pseudo-labeled real hr$\text{T}_2$  images in the next training phase.

\begin{figure}[t]
	\includegraphics[width=\textwidth]{fig/fig_1.jpg}
	\caption{The overview of our proposed framework.}
	\label{fig1}
\end{figure}

\subsection{Multi-view image translation}
We first translate ce$\text{T}_1$ images into multi-view pseudo-hr$\text{T}_2$ images by adopting CycleGAN~\cite{zhu2017unpaired} and QS-Attn~\cite{hu2022qs} in parallel.

\subsubsection{CycleGAN.} CycleGAN uses cycle-consistency loss to translate the source domain ce$\text{T}_1$ images into the target domain hr$\text{T}_2$ images. Cycle-consistency loss described in Eq.~\ref{eq7} encourages $F(G({x}_s ))$ to be equal to ${x}_s$ and $G(F({x}_t ))$ to be equal to ${x}_t$ in pixel-level when given the $G:{X}_s\to{X}_t$  and $F:{X}_t\to{X}_s$ generators~\cite{zhu2017unpaired}.
\begin{equation}
    {L}_{cycle}=\left\|F(G({x}_s)) - {x}_s\right\| + \left\|G(F({x}_t )) - {x}_t\right\|\label{eq7}
\end{equation}

\subsubsection{QS-Attn.} QS-Attn is an unpaired image translation model that is improved from CUT~\cite{park2020contrastive}. CUT preserves the structural information by constraining the patches from the same location on the source and the translated images to be close, compared to the different locations. CUT maximizes the mutual information between the source and translated images through the Eq.~\ref{eq8} ~\cite{park2020contrastive}, 
\begin{equation}
    {L}_{con}=-\log\Biggl[\cfrac{\exp(q\cdot{k}^{+}/\tau)}{\exp(q\cdot{k}^{+}/\tau)+\sum\nolimits_{i=1}^{N-1}\exp(q\cdot{k}^{-}/\tau)}\Biggr]\label{eq8}
\end{equation}

\noindent{where $q$ is the anchor feature from the translated image and ${k}^+$ is a single positive at the same location in the source image and ${k}^-$ are $(N-1)$ negatives at the other locations, and $\tau$ is a temperature~\cite{hu2022qs}.}

However, CUT~\cite{park2020contrastive} calculates the contrastive loss between the randomly selected  patches, which could have less domain-relevant information. QS-Attn addresses this limitation by adopting the QS-Attn module, which can select domain-relevant patches. The QS-Attn module constructs the attention matrix ${A}_g$ using the features in the source images and then obtains the entropy ${H}_g$ by following Eq.~\ref{eq9} ~\cite{hu2022qs}.

\begin{equation}
    {H}_g(i)=-\sum_{j=1}^{HW}{A}_g(i,j)\log{{A}_g(i,j)}\label{eq9}
\end{equation}

\noindent{Of note, the smaller entropy ${H}_g$ means the more important feature. Thus, ${A}_g$ is sorted in ascending order according to entropy ${H}_g$ to select domain-relevant patches~\cite{hu2022qs}. By calculating the contrastive loss using the selected domain-relevant patches, the structures of the source domain better preserve, and more realistic images are generated compared to CUT~\cite{park2020contrastive}.}

We empirically found that CycleGAN with pixel-level cycle-consistency loss allows the model to better reflect the intensity and the texture of the VS and cochleas in the target images, while QS-Attn takes advantage of preserving the structure of them more clearly via patch-level contrastive loss (refer to Section~\ref{discussion} for more details). By using them together, our multi-view image translation can augment pseudo-hr$\text{T}_2$ images from different perspectives, and it can help improve the performance of the following segmentation model.

\subsection{Segmentation and Self-training}
Motivated by the previous works~\cite{shin2022cosmos,dong2021unsupervised,choi2021using}, we also utilize nnUNet~\cite{isensee2021nnu} and self-training procedure~\cite{xie2020self} to construct the segmentation model. nnUNet is a powerful segmentation framework that automatically performs pre-processing, training, and post-processing with heuristic rules~\cite{isensee2021nnu}. Self-training is carried out to reduce the distribution gap between real hr$\text{T}_2$ and translated hr$\text{T}_2$ images and to improve the robustness of the segmentation model for unseen real hr$\text{T}_2$ scans. The segmentation and self-training procedure consists of four steps; (1) training the segmentation model using the translated hr$\text{T}_2$ scans with labels of the ce$\text{T}_1$ scans. (2) Generating pseudo labels of unlabeled real hr$\text{T}_2$ scans by using the trained segmentation model. (3) Retraining the segmentation model using both the translated hr$\text{T}_2$ scans with labels of the ce$\text{T}_1$ scans and the real hr$\text{T}_2$ scans with pseudo labels. 4) Repeating Steps 2-3 to achieve further performance improvement.

\section{Experiments and Results}

\subsection{Dataset and preprocessing}
We used the CrossMoDA dataset~\footnote{https://crossmoda-challenge.ml/}~\cite{dorent2022crossmoda} for training, validation. The CrossMoDA dataset consists of data from two different institutions: London and Tilburg. The London data consists of 105 ce$\text{T}_1$ scans and 105 hr$\text{T}_2$ scans. The ce$\text{T}_1$ scans were acquired with the in-plane resolution of 0.4×0.4mm, in-plane matrix of 512×512, and slice thickness of 1.0 to 1.5 mm with an MPRAGE sequence (TR=1900 ms, TE=2.97 ms, TI=1100 ms). Meanwhile, hr$\text{T}_2$ scans were acquired with the in-plane resolution of 0.5×0.5mm, in-plane matrix of 384×384 or 448×448, and slice thickness of 1.0 to 1.5 mm with a 3D CISS or FIESTA sequence (TR=9.4 ms, TE=4.23ms). For the Tilburg data set, ce$\text{T}_1$ scans and hr$\text{T}_2$ scans consist of 105 subjects each. The ce$\text{T}_1$ scans were acquired with the in-plane resolution of 0.8×0.8mm, in-plane matrix of 256×256, and slice thickness of 1.5 mm with a 3D-FFE sequence (TR=25 ms, TE=1.82 ms). The hr$\text{T}_2$ scans were acquired with the in-plane resolution of 0.4×0.4mm, in-plane matrix of 512×512, and slice thickness of 1.0 mm with a 3D-TSE sequence (TR=2700 ms, TE=160 ms, ETL=50)~\cite{dorent2022crossmoda}. The training dataset of the $\text{CrossMoDA2022 Challenge  }^1$ contains a total of 210 ce$\text{T}_1$ scans with annotation labels and 210 hr$\text{T}_2$ scans without annotation labels. In addition, they provide 64 scans of hr$\text{T}_2$ images for validation.

Since the voxel spaces vary across scans, all the images were resampled to [0.41, 0.41, 1.5] voxel sizes. For image translation, the 3D MRI images were sliced into a series of 2D images along the axial plane and the images were center-cropped and resized to 256 × 256. After performing image translation, the translated hr$\text{T}_2$ images were merged into 3D MR imaging and fed into the segmentation model.

\subsection{Implementation details} 
We implement CycleGAN~\cite{zhu2017unpaired}, QS-Attn~\cite{hu2022qs}, and nnUNet~\cite{isensee2021nnu}, following their default parameter settings. We also apply a global attention in QS-Attn~\cite{hu2022qs}, and ensemble selection in nnUNet~\cite{isensee2021nnu} for the final prediction. All the implementations are powered by RTX 3090 24GB GPUs. The training of CycleGAN, QS-Attn, and nnUNet is performed with PyTorch 1.8.0, 1.7.1, and 1.10.2, respectively.

\subsection{Results}
Table~\ref{tab5} and Fig.~\ref{fig2} show the VS and cochlea segmentation results with different image translation methods. The proposed multi-view image translation framework with CycleGAN~\cite{zhu2017unpaired} and QS-Attn~\cite{hu2022qs} shows better performance compared to other methods using each model alone. Moreover, we greatly improved the performance of the segmentation model with self-training. As a result, our proposed method obtained a great achievement with a mean dice score of 0.8504±0.0466 in the validation period. 






\begin{table}[t]
\centering
\caption{Segmentation results with dice and ASSD scores (ST: self-training).}
\label{tab5}
\resizebox{\textwidth}{!}{%
\begin{tabular}{c|ccc|cc}
\hline
Translation                                             & \multicolumn{3}{c|}{Dice score}                                                                                                                                                                                                   & \multicolumn{2}{c}{ASSD}                                                                                                                   \\ \cline{2-6} 
model                                                   & \multicolumn{1}{c|}{VS}                                                        & \multicolumn{1}{c|}{Cochlea}                                                         & Mean                                                         & \multicolumn{1}{c|}{VS}                                                         & Cochlea                                                         \\ \hline
\begin{tabular}[c]{@{}c@{}}CycleGAN\\ (\emph{w/o.} ST) \end{tabular}                      & \multicolumn{1}{c|}{\begin{tabular}[c]{@{}c@{}}0.7798\\ ($\pm{0.1901}$)\end{tabular}} & \multicolumn{1}{c|}{\begin{tabular}[c]{@{}c@{}}0.8066\\ ($\pm{0.0323}$)\end{tabular}} & \begin{tabular}[c]{@{}c@{}}0.7932\\ ($\pm{0.0972}$)\end{tabular} & \multicolumn{1}{c|}{\begin{tabular}[c]{@{}c@{}}0.8750\\ ($\pm{0.9222}$)\end{tabular}} & \begin{tabular}[c]{@{}c@{}}0.2422\\ ($\pm{0.1608}$)\end{tabular} \\ \hline
\begin{tabular}[c]{@{}c@{}}QS-Attn\\ (\emph{w/o.} ST) \end{tabular}                       & \multicolumn{1}{c|}{\begin{tabular}[c]{@{}c@{}}0.7779\\ ($\pm{0.1825}$)\end{tabular}} & \multicolumn{1}{c|}{\begin{tabular}[c]{@{}c@{}}0.8158\\ ($\pm{0.0287}$)\end{tabular}} & \begin{tabular}[c]{@{}c@{}}0.7968\\ ($\pm{0.0929}$)\end{tabular} & \multicolumn{1}{c|}{\begin{tabular}[c]{@{}c@{}}0.6667\\ ($\pm{0.3891}$)\end{tabular}} & \begin{tabular}[c]{@{}c@{}}0.2365\\ ($\pm{0.1573}$)\end{tabular} \\ \hline
\begin{tabular}[c]{@{}c@{}}Proposed\\ (\emph{w/o.} ST) \end{tabular}                      & \multicolumn{1}{c|}{\begin{tabular}[c]{@{}c@{}}0.8043\\ ($\pm{0.1656}$)\end{tabular}} & \multicolumn{1}{c|}{\begin{tabular}[c]{@{}c@{}}0.8158\\ ($\pm{0.0289}$)\end{tabular}} & \begin{tabular}[c]{@{}c@{}}0.8101\\ ($\pm{0.0863}$)\end{tabular} & \multicolumn{1}{c|}{\begin{tabular}[c]{@{}c@{}}0.5742\\ ($\pm{0.2461}$)\end{tabular}} & \begin{tabular}[c]{@{}c@{}}0.2387\\ ($\pm{0.1581}$)\end{tabular} \\ \hline
\begin{tabular}[c]{@{}c@{}}Proposed\\ (\emph{w.} ST) \end{tabular}                      & \multicolumn{1}{c|}{\begin{tabular}[c]{@{}c@{}}\textbf{0.8520}\\ \textbf{($\pm{0.0889}$)}\end{tabular}} & \multicolumn{1}{c|}{\begin{tabular}[c]{@{}c@{}}\textbf{0.8488}\\ \textbf{($\pm{0.0235}$)}\end{tabular}} & \begin{tabular}[c]{@{}c@{}}\textbf{0.8504}\\ \textbf{($\pm{0.0466}$)} \end{tabular} & \multicolumn{1}{c|}{\begin{tabular}[c]{@{}c@{}}\textbf{0.4748}\\ \textbf{($\pm{0.2072}$)} \end{tabular}} & \begin{tabular}[c]{@{}c@{}}\textbf{0.1992}\\ \textbf{($\pm{0.1524}$)}\end{tabular} \\ \hline
\end{tabular}%
}
\end{table}



\begin{figure}[hbt!]
	\includegraphics[width=\textwidth]{fig/fig_2.jpg}
	\caption{Qualitative comparison of segmentation results for validation set. We visualize the segmentation results of VS (red) and cochlea (green) (ST: Self-training).}
	\label{fig2}
\end{figure}



\begin{figure}[hbt!]
	\includegraphics[width=\textwidth]{fig/fig_3.jpg}
	\caption{Performance comparison of VS and cochlea segmentation models (ST: Self-training).}
	\label{fig3}
\end{figure}

We conducted paired \emph{t}-test among CycleGAN~\cite{zhu2017unpaired}, QS-Attn~\cite{hu2022qs}, and our proposed method (\emph{w/o.} self-training, ST) to compare the segmentation performance, and the results are plotted in Fig.~\ref{fig3}. CycleGAN, QS-Attn, and our proposed method (\emph{w/o.} ST) show statistical significance with \emph{p} $<$ 0.05 for the dice score of VS and mean values. In addition, our proposed method (\emph{w/o.} ST) is statistically better with \emph{p} $<$ 0.0001 than CycleGAN on the dice score of cochleas. Through this statistical comparison, we proved that our proposed framework achieved better performance compared to other methods that use either of the two models alone.









\newpage

\section{Discussion and New Perspectives}\label{sec:discuss}
% and Future Directions

In this section, we first discuss challenges and practical considerations, including non-stationarity, heterogeneity, unobserved confounders, subsampling, and expert knowledge.
Then, two new perspectives of temporal causal discovery are provided, which in our opinion will be a promising avenue for future research.

\subsection{Challenges and Practical Considerations}



\textbf{Non-stationarity of data:} We are often faced with non-stationarity in practical scenarios, where the probability distributions of temporal variables conditional on their causes or even the causal relations may change across time, especially for temporal data.
In this condition, causal discovery approaches presuming a fixed causal model may give misleading results. 
Whereas, several types of research have shown that non-stationarity contains information for causal discovery \cite{CD_from_change/conf/uai/TianP01, CD_from_change/peters2016causal, Discussion/Nonstation_hetero/ijcai_ZhangHZGS17, Discussion/Nonstation/state_space_icml_Huang0GG19}.
Thus, it's important to properly tackle the non-stationarity in applications.
Non-stationarity may result from the change of underlying systems and can be seen as a soft intervention \cite{soft_interv/korb2004varieties} done by nature. 
Following this idea, a line of work \cite{Discussion/Nonstation_hetero/ijcai_ZhangHZGS17, Discussion/Nonstation_hetero2/jmlr/Huang0ZRSGS20} leverages a surrogate such as time and domain index to account for nonstationarity where the causal relations are changed, and the CD-NOD framework is proposed. 
Instead of leveraging informative non-stationarity to causal structure learning, another set of research focuses on modeling time-varying relationships \cite{Discussion/Nonstation/pr_GaoY22}. 
Besides, the approach for slowly varying non-stationary process, such as evolutionary spectral and locally stationary processes, is proposed in \cite{Discussion/Nonstation/slowly_varying/du2020causal}.






\textbf{Heterogeneity of data:} In causal discovery for practical applications, the heterogeneity of data lies in two levels: (1) The interacting temporal processes are heterogeneous (having different distributions), for instance, causally related meteorological observations from different stations are influenced by several major weather systems separately \cite{Discussion/heterogeneous/pakdd_BehzadiHP19}. (2) The underlying generating process changes across data sets or different domains \cite{intro/nonts_surveys/glymour2019review}, for instance stock prices from different markets \cite{Discussion/Nonstation_hetero2/jmlr/Huang0ZRSGS20} or individual behaviours in different paradigms \cite{MTS/Attention/icdm_InGRA_ChuWMJZY20}.
For the first condition where the heterogeneity exists among temporal variables, the inferred relations of the traditional causal discovery approaches, which have been designed for specific homogeneous data types, may be inaccurate. As a remedy, several variants of Granger causality, based on methods such as generalized linear models and minimum message length, are proposed in \cite{Applications/anomaly/work2_icdm_BehzadiHP17, Discussion/heterogeneous/pakdd_BehzadiHP19, Discussion/heterogeneous/entropy/Hlavackova-Schindler20}.
For the second condition, a line of work \cite{Discussion/Nonstation_hetero/ijcai_ZhangHZGS17,  Discussion/Nonstation_hetero2/jmlr/Huang0ZRSGS20} leverages the distribution shift from heterogeneity as a soft intervention to assist causal structure learning, which is similar to that in non-stationary data.  
Whereas, another line of causal discovery approaches \cite{MTS/Attention/icdm_InGRA_ChuWMJZY20, Discussion/NewForm/ACD_LoweMSW22} in the second condition focuses on inductively modeling typical structure in heterogeneous data within an end-to-end framework. 



\textbf{Unobserved confounders:}
In practice, we are often met with cases where causal sufficiency is violated, \ie, there exist unobserved confounders. 
This challenging setting may lead to incorrect causal relations~\cite{MTS/FCM/VAR_LINGAM_extend2_icml_GeigerZSGJ15}.
As summarized in Table~\ref{tab:ts_category_overview}, most temporal causal discovery approaches cannot handle unobserved confounders in a straightforward way.
Several constraint-based approaches are designed without causal sufficiency and approaches
Besides, unobserved confounders are modeled by applying a structural bias in~\cite{Discussion/NewForm/ACD_LoweMSW22}.
Several recent studies termed as causal representation learning take a new perspective on unobserved confounders.
It will be detailed in subsection (\ref{subsection:causal_rep}).

\textbf{Subsampling:} In real-world applications, temporal data, especially time series, may be sampled at a rate lower than the rate of the underlying causal process due to the difficulties in data collection.
An ordinary causal discovery algorithm for sub-sampled time series may lead to spurious causal relations and missed ones. 
Several remarks and approaches~\cite{Discussion/subsample/work1, Discussion/subsample/work2_icml_GongZSTG15, Discussion/subsample/work3_nips_rateagnostic_PlisDFC15, Discussion/subsample/uai_subsample_aggr_GongZSGT17, Discussion/subsample/work5_pgm_constraintOPT_HyttinenPJED16, Discussion/subsample/biometrika/tank2019identifiability} are proposed for this issue.

\textbf{Expert knowledge: }Expert knowledge can help the causal discovery process in practice.  % 要强调practical issues.
The approaches of fusing expert knowledge can be categorized into three types~\cite{intro/nonts_surveys/BN21}: (1) \textit{Soft constraints}: the learning process can be influenced by the knowledge~\cite{Discussion/knowledge/ausai/ODonnellNHKAH06}. % (\ie, conditions given with a probability $0<p<1$).
(2) \textit{Hard constraints}: the learnt structure must conform to the enforced requirements (\ie, conditions given with a probability $p=0$ or $p=1$). 
In~\cite{Discussion/knowledge/artmed/AsvatourianLML20}, hard constraints are leveraged in structure learning with a time dependant exposure.
Studies in~\cite{MTS/SB/NTS_NOTEARS} add prior knowledge forbidding the existence of intra-slice dependencies, which is helpful to recover edges that are not explicitly encoded by the prior knowledge.
(3) \textit{Interactive learning}: the human input is leveraged in the learning process~\cite{Discussion/knowledge/ecsqaru/MessaoudLA09, Discussion/knowledge/kdd/MelkasSCMNMP21,https://doi.org/10.48550/arxiv.2206.05420, 9222294}.








\subsection{New perspectives}


\subsubsection{Extension in amortized and supervised paradigms}


In the traditional paradigms, causal discovery methods mostly either treat observational data separately or train a distinct model for each individual. 
These methods do not make full use of the common structure across different samples or supervised information from the datasets whose causal structures are clearly explored, thus suffering from several issues such as the small sample challenge and lack the inductive capability.
Recently, causal discovery is conducted in new paradigms to solve this problem. We can roughly categorize them into two groups: methods based on \textbf{amortized modeling} \cite{MTS/Attention/icdm_InGRA_ChuWMJZY20, Discussion/NewForm/ACD_LoweMSW22}, and methods based on \textbf{supervised learning} \cite{benozzo2017supervised, wang2022meta}.
We introduce them in this subsection, which we believe are a promising avenue for future research. 


In amortized modeling, a global causal discovery framework is trained for individuals with different causal structures. 
As for scenarios with temporal data, these approaches have been detailed in \ref{subsection:NN_Granger} as the deep learning extension of Granger causality with inductive modeling.
InGRA \cite{MTS/Attention/icdm_InGRA_ChuWMJZY20} leverages prototype learning to extract common causal structure while ACD \cite{Discussion/NewForm/ACD_LoweMSW22} proposes an encoder-decoder framework to conduct amortized causal discovery. These methods make full use of information from massive samples and are able to infer causal relations for newly arrived individuals, which are useful in real-world applications such as e-commerce, social network, and neuroimages.

Another line of work has predominately focused on treating the inference process as a black box and learning the mapping from sample data to causal graph structures via supervised learning. Here the label information is causal structure and can be easily accessed in synthetic datasets. 
Earlier work \cite{Discussion/NewForm/RCC/jmlr/Lopez-PazMR15, DBLP:conf/aaai/TonSF21} on learning causal relations by supervised learning is restricted to learning pairwise causal direction where the problem is cast into a classification task to distinguish between $X \to Y$ and $Y \to X$ by using observed samples.
It's later extended to discovery graph structure in \cite{Discussion/NewForm/DAG_EQ/corr/abs-2006-04697,petersen2022causal}.
As the labeled information for training is often originated from synthetic data or real-world datasets which have been explored, the requirement of a supervised approach, in which the distributions of training and test data match or highly overlap, is not guaranteed. In \cite{Discussion/NewForm/ML4S/kdd/00040DJWH022, Discussion/NewForm/CSIvA_DeepMind}, methods such as vicinal graph and meta-learning are leveraged in supervised causal discovery to tackle this `domain shift' issue.  
For the temporal setting, a supervised estimation of Granger causality between time series is proposed in \cite{benozzo2017supervised}. As a recent advance, a method for learning causal discovery is proposed in \cite{wang2022meta} where the learned from large datasets with known causal relations outperform the algorithm in the traditional paradigm when testing on temporal datasets such as fMRI. 
% It's also noted in \cite{wang2022meta} that the causal discovery algorithms in traditional paradigm depart from strong human assumptions about causality. In these approaches (such as constraint-based, score-based and Granger causality), human intuition is implemented in different form. 




% \subsubsection{Extension causal discovery towards causal representation learning (to edit)}
\subsubsection{Extension in causal representation learning}
\label{subsection:causal_rep}
% \subsection{Nonlinear ICA, causal representation learning...}

Extracting the causes of particular phenomena whether explicitly or implicitly from a deep learning black box can be beneficial to the downstream tasks.
The aforementioned causal discovery methods focus on inferring relations between observed variables, or start from the premise that the causal variables are given before hand.
Although some approaches learn causal relations under unobserved variables.
There exist real-world observations (e.g., sensor measurements, image pixels in video) which are not well structured to causal variables to begin with. 
As a generalization of causal discovery from observed variables, there has recently been a growing interest in \textbf{causal representation learning} \cite{CausalRepresentation/nontemp/icml/LocatelloPRSBT20, CausalRepresentation/nontemp/towardsCRL/ScholkopfLBKKGB21, CausalRepresentation/nontemp/CausalVAE/YangLCSHW21}, which aims at learning representation of causal factors in an underlying system.
It estimates latent causal variable graphs from observations.




A line of works in causal representation learning identifies independent factors of variations based on disentanglement and Independent Component Analysis (ICA).
At the heart of this methodology is the postulation of mutually independent latent factors.
It's hard to identify true latent variables, especially in general nonlinear cases.
As a remedy, recent approaches \cite{CausalRepresentation/nontemp/icml/LocatelloPRSBT20, CausalRepresentation/iVAE_nontemp/aistats/KhemakhemKMH20, DBLP:conf/aistats/HyvarinenM17, DBLP:conf/nips/HyvarinenM16} leverage additional information in multiple views, auxiliary variables, or temporal structure, combined with deep learning methods like VAEs and contrastive learning.
A connection between ICA and causality has been recently drawn in \cite{CausalRepresentation/IMA/nontemp/nips/GreseleKSSB21, DBLP:conf/uai/Monti0H19}.
In the context of temporal data, the identifiability of causal variables from temporal sequences is discussed in latent temporal causal process estimation (\textbf{LEAP}) \cite{Discussion/latent/iclr_LEAP_YaoSHS022}. It first provides causal identifiability conditions in a nonparametric, nonstationary setting, and a parametric setting. Then it proposes a learning framework to extract latent causal relations, which extends VAE with a learned causal process network by enforcing the assumed conditions.
The non-stationary noise, modeled by flow-based estimators, can be viewed as a soft intervention to aid identification.
In line with LEAP, subsequent works \cite{TDRL_DBLP:journals/corr/abs-2210-13647} extend the identification theory to a more general case.   % Change to NIPS form citation




Another line of work leverage intervention and data augmentation to help to identify latent causal relations. Under data augmentation, it's demonstrated in \cite{CausalRepresentation/line2/nips/KugelgenSGBSBL21} that common contrastive learning methods can block-identify causal variables that remain unchanged. 
For the temporal setting, \textbf{CITRIS} \cite{CausalRepresentation/CITRIS/icml/LippeMLACG22} is proposed. It's a VAE framework learning causal representation where latent causal factors have possibly been interved on.
By using intervention target information for identification, CITRIS is devoid of suffering from functional or distributional form constraints.
Besides, causal factors in CITRIS are considered as either scalars or potentially multidimensional vectors, which is more practical in complex scenarios. Along this line of work, instantaneous causal relations are extracted in iCITRIS \cite{CausalRepresentation/interv/iCITRIS/abs-2206-06169}.























%\section{}
%\label{sec:resDir}


\section{Conclusion}
\label{sec:conclusion}
% <>
Since its advent in 1931, Koopman operator theory \cite{koopman:1931} has only recently been actively utilized for solving practical problems, thanks to the introduction of the DMD algorithm in 2008 \cite{schmid:2008}. Since then, a multitude of DMD algorithm variations have risen to prominence and found utility across various fields. A notable feature of our survey paper was reviewing and categorizing the results of over 100 research papers based on both application and algorithm type in smart mobility and vehicle engineering  (see Table~\ref{tab1} and Section~\ref{sec:vehicApp}).  Additionally, this survey paper identified potential research gaps in smart mobility and vehicular engineering applications (Remarks~\ref{remGap1}--\ref{remGap6}). Finally, this review paper discussed theoretical aspects of Koopman operator theory that have been largely neglected by the smart mobility and vehicle engineering community and yet have large potential for contributing to solving open problems in these areas (see Section~\ref{subsec:theorIssue}).

\noindent{\textbf{Future Research Directions.}}	Given the emergence of cyber-threats against connected and autonomous vehicles as well as robotic systems (see, e.g.,~\cite{nekouei2021randomized,mohammadi2022generation}), a future research direction might include utilizing Koopman operator-based algorithms for designing cyber-resilient vehicular and smart mobility applications (see, e.g.,~\cite{taheri2022data} for a related line of research). Another potential research direction is using Koopman operator-based algorithms for predicting the motion of vulnerable road users (VRUs), e.g., pedestrians and cyclists (see, e.g.,~\cite{pool2019context,scholler2020constant}). Finally, rehabilitation robotics and robotic exoskeletons can be the benefactors of the predictive capabilities of Koopman operator-based algorithms for detecting tripping events and/or system  identification in various modes of locomotion (see, e.g.,~\cite{kumar2019extremum,aprigliano2019pre}).



%Fig. 1 depicts the accumulation of such algorithms since 2014, which are particular to vehicle engineering and smart mobility, i.e., the focus of this review. Table 1 summarizes the varieties of relevant algorithms developed in those studies. Furthermore, we have highlighted theoretical issues, whose expansion will have potential applications to the wide research area of smart mobility and vehicle engineering.  

%Although fairly comprehensive, we have found several gaps in this research area. In particular, we could not find any studies related to elevators, robots/vehicles employing crawling, slithering, hopping or peristaltic locomotion, arctic or special-terrain vehicles such as those employing screws or tracks, hovercraft and other amphibious vehicles or subsystems which tolerate flexible environments, classification or guidance systems related to vehicles for drilling or agriculture, or for current-ripple, power-split, battery health monitoring, nuclear propulsion, exoskeletons/prosthetics, personal mobility, motorsports, specialized rovers or similar open problems in emerging areas.  These examples are, of course, not exhaustive.  
%
%The purely data-driven nature of Koopman operators holds the promise of capturing unknown and complex dynamics for reduced-order model generation and system identification, through which the rich machinery of linear control techniques can be utilized. The emergent nature of the smart mobility and vehicular-related applications, where  the Koopman operator  in each particular application needs to be approximated, implies that the development of various Koopman operator approximation  algorithms is expected to grow along with the vehicular problems they aim to solve.  Given the ongoing development of this research area and the many existing open problems in the fields of smart mobility and vehicle engineering, a survey of techniques and open challenges of applying Koopman operator theory to this vibrant area is warranted.  To the best of our knowledge, this survey paper is the \emph{first of its kind} reviewing the applications of Koopman operator theory within a focused research area, namely, smart mobility and vehicle engineering applications. A \emph{notable feature} of our survey paper is reviewing and categorizing the results of over 100 research papers based on both application and algorithm type  (see Tables~\ref{tab1}--~\ref{tab4} and Section~\ref{sec:vehicApp}) that are concerned with the applications of Koopman operator theory to the field of smart mobility and vehicular engineering. Such a \emph{comprehensive and  detailed categorization} will be beneficial to the research practitioners working in the field.  Furthermore, this review paper discusses theoretical aspects of Koopman operator theory that have been largely neglected by the smart mobility and vehicle engineering community and yet have large potential for contributing to solving open problems in these areas. Additionally, our survey paper seeks to \emph{identify gaps} in the smart mobility and vehicle engineering research where new and existing Koopman operator-based methods have the potential to further develop and address unsolved problems  potentially benefiting from the perspectives of nonlinear system identification, control, global linearization, and the predictive powers that Koopman operator theory has to offer (see, e.g., Remarks~\ref{remGap1}--\ref{remGap6}). 


\newpage
\subsubsection{Acknowledgment.}
This work was supported by Institute of Information \& communications Technology Planning \& Evaluation (IITP) grant funded by the Korea government (MSIT) (No. 2019-0-00079, Artificial Intelligence Graduate School Program (Korea University)), and the National Research Foundation of Korea (NRF) grant funded by the Korea government (MSIT) (No. 2020R1C1C1013830, No. 2020R1A4A1018309).

\bibliographystyle{splncs04}
\bibliography{ref}

\end{document}
