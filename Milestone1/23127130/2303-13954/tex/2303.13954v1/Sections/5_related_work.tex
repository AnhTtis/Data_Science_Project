%%%%%%%%%%%%%%%%%%%%%%%%%%%%%%%%%%%%%%%%%%%%%%%%%%%%%%%%%%%%%%%%%%%%%%%%%%%%%%%
%
% File: 5_related_work.tex
% Self-contained tex file for the related work
%
%%%%%%%%%%%%%%%%%%%%%%%%%%%%%%%%%%%%%%%%%%%%%%%%%%%%%%%%%%%%%%%%%%%%%%%%%%%%%%%
\section{Related Work}
\label{sec:5_related}
Techniques to perform runtime optimizations for irregular memory accesses have been worked on for many years, and of particular relevance to our work is the \emph{inspector-executor} technique~\cite{SALTZ91B,KenImprovingCachePerformance,strout2003compile}.
Das et al.~\cite{das1994communication} presented an inspector-executor optimization similar to ours, but that work predates the PGAS model, so is not directly applicable due to fundamental differences in programming model design and implementation.
For PGAS languages, Su and Yelick~\cite{TitaniumOpt} developed an inspector-executor optimization that is similar to ours but for the language Titanium. 
However, Titanium differs from Chapel in its execution model as well as its language constructs for parallel loops, since it is based on Java.
This leads to an overall different approach to static analysis.
Alvanos et al.~\cite{UPCStaticDynamicCoal,ALVANOS20162} described an inspector-executor framework for the PGAS language UPC~\cite{el2005upc}, which also differs from Chapel.
UPC uses a SPMD model and requires explicit constructs to control the processor affinity of parallel loops.
Like Titanium, these differences lead to a significantly different approach to the static analysis used for the optimization.
However, we plan to explore some of their techniques in future work.
For compiler optimizations related specifically to Chapel, Kayraklioglu et al.~\cite{enginLCPC2021} presented an optimization to aggregate remote accesses to distributed arrays.
However, their optimization does not specifically target irregular memory accesses, which leads to significantly different approaches to static analysis and code transformation.
Furthermore, the applications we evaluate in Section~\ref{sec:4_perfEval} are not candidates for their aggregation optimization.