%%%%%%%%%%%%%%%%%%%%%%%%%%%%%%%%%%%%%%%%%%%%%%%%%%%%%%%%%%%%%%%%%%%%%%%%%%%%%%%
%
% File: 6_concl.tex
% Self-contained tex file for the conclusions
%
%%%%%%%%%%%%%%%%%%%%%%%%%%%%%%%%%%%%%%%%%%%%%%%%%%%%%%%%%%%%%%%%%%%%%%%%%%%%%%%
\section{Conclusions and Future Work}
\label{sec:6_concl}
While the PGAS model provides user productivity advantages for writing distributed irregular applications, the resulting code often has poor runtime performance due to fine-grained remote communication.
In this work we have presented a compiler optimization for Chapel programs that specifically targets irregular memory access patterns within parallel loops and automatically applies code transformations to replicate remotely accessed data.
We demonstrated that the optimization provides runtime speed-ups as large as 52x on a Cray XC system and 364x on an Infiniband system.
To this end, we have shown that significant performance gains can be achieved without sacrificing user productivity.

For future work, we plan to improve upon our compiler optimization framework and address some of the limitations, such as optimizing multiple irregular accesses in the same loop.
We also plan to design additional compiler optimizations to serve as alternatives to selective data replication when that cannot be applied, as it currently is only applicable for read-only data.
Future optimizations will specifically target writes.

\section*{Acknowledgements}
We would like to thank Brad Chamberlain, Engin Kayraklioglu, Vass Litvinov, Elliot Ronaghan and Michelle Strout from the Chapel team for their guidance on working with the Chapel compiler, as well as providing access to the Cray XC system that was used in our performance evaluation.