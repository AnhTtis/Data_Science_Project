%%%%%%%%%%%%%%%%%%%%%%%%%%%%%%%%%%%%%%%%%%%%%%%%%%%%%%%%%%%%%%%%%%%%%%%%%%%%%%%
%
% File: 0_abstract.tex
% Self-contained tex file for the abstract
%
%%%%%%%%%%%%%%%%%%%%%%%%%%%%%%%%%%%%%%%%%%%%%%%%%%%%%%%%%%%%%%%%%%%%%%%%%%%%%%%

\begin{abstract}
Irregular memory access patterns pose performance and user productivity challenges on distributed-memory systems. 
They can lead to fine-grained remote communication and the data access patterns are often not known until runtime.
The Partitioned Global Address Space (PGAS) programming model addresses these challenges by providing users with a view of a distributed-memory system that resembles a single shared address space.
However, this view often leads programmers to write code that causes fine-grained remote communication, which can result in poor performance.
Prior work has shown that the performance of irregular applications written in Chapel, a high-level PGAS language, can be improved by manually applying optimizations.
However, applying such optimizations by hand reduces the productivity advantages provided by Chapel and the PGAS model.
We present an inspector-executor based compiler optimization for Chapel programs that automatically performs remote data replication.
While there have been similar compiler optimizations implemented for other PGAS languages, high-level features in Chapel such as implicit processor affinity lead to new challenges for compiler optimization.
We evaluate the performance of our optimization across two irregular applications.
Our results show that the total runtime can be improved by as much as 52x on a Cray XC system with a low-latency interconnect and 364x on a standard Linux cluster with an Infiniband interconnect, demonstrating that significant performance gains can be achieved without sacrificing user productivity.
\end{abstract}