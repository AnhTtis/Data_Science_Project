\documentclass{article}
\usepackage[left=1in, right=1in, top=1in, bottom=1in]{geometry}
\usepackage{enumitem}
\usepackage{authblk}
\usepackage{graphicx}	
\usepackage{amsmath, amsthm, amssymb}
\usepackage{xspace}
\usepackage{comment}
\usepackage{hyperref}
\usepackage{xcolor}
\usepackage{algorithm}
\usepackage{algpseudocode}

\newcommand{\msay}[1]{{\color{blue} Michael says: #1}}
\newcommand{\nsay}[1]{{\color{red} Nicholas says: #1}}
\newcommand{\psay}[1]{{\color{teal} Puck says: #1}}

\newcommand{\arXiv}[2]{\href{https://arxiv.org/abs/#1}{arXiv:#1 #2}}

\newcommand{\cc}[1]{\ensuremath{\mathsf{#1}}}
\newcommand{\algprobm}[1]{\textsc{#1}\xspace}
\newcommand{\GI}{\algprobm{GI}}
\newcommand{\GpI}{\algprobm{GpI}}
\newcommand{\GpIlong}{\algprobm{Group Isomorphism}}
\newcommand{\GIlong}{\algprobm{Graph Isomorphism}}
\newcommand{\GpCodeEqlong}{\algprobm{Group Code Equivalence}}

\newcommand{\Oh}{O}
\newcommand{\Soc}{\text{Soc}}
\newcommand{\Fac}{\text{Fac}}
\newcommand{\F}{\mathbb{F}}
\newcommand{\Z}{\mathbb{Z}}
\newcommand{\N}{\mathbb{N}}

\theoremstyle{plain}
\newtheorem{theorem}{Theorem}[section]
\newtheorem{proposition}[theorem]{Proposition}
\newtheorem{corollary}[theorem]{Corollary}
\newtheorem{lemma}[theorem]{Lemma}
\newtheorem{fact}[theorem]{Fact}

\theoremstyle{definition}
\newtheorem{definition}[theorem]{Definition}
\newtheorem{remark}[theorem]{Remark}
\newtheorem{question}[theorem]{Question}


\newcommand{\Lem}[1]{Lemma~\ref{#1}\xspace}
\newcommand{\Cor}[1]{Corollary~\ref{#1}\xspace}
\newcommand{\Prop}[1]{Proposition~\ref{#1}\xspace}
\newcommand{\Thm}[1]{Theorem~\ref{#1}\xspace}
\newcommand{\Def}[1]{Definition~\ref{#1}\xspace}
\newcommand{\Rmk}[1]{Remark~\ref{#1}\xspace}
\newcommand{\Algorithm}[1]{Algorithm~\ref{#1}\xspace}
\newcommand{\wt}{\text{wt}}

\DeclareMathOperator{\GL}{GL}
\DeclareMathOperator{\PSL}{PSL}
\DeclareMathOperator{\Inn}{Inn}
\DeclareMathOperator{\Aut}{Aut}
\DeclareMathOperator{\ncl}{ncl}
\DeclareMathOperator{\rw}{rw}
\DeclareMathOperator{\Iso}{Iso}
\DeclareMathOperator{\rad}{Rad}
\DeclareMathOperator{\pker}{PKer}
\DeclareMathOperator{\Sub}{Sub}
\DeclareMathOperator{\poly}{poly}
\DeclareMathOperator{\supp}{supp}
\DeclareMathOperator*{\argmin}{argmin}
\DeclareMathOperator{\cw}{cw}
\DeclareMathOperator{\rk}{rk}
\newcommand{\sgn}{sgn}

%additional commands
\DeclareMathOperator{\dist}{dist}


\title{Logarithmic Weisfeiler--Leman and Treewidth\footnote{This work was completed in part at the 2022 Graduate Research Workshop in Combinatorics, which was supported in
part by NSF grant \#1953985 and a generous award from the Combinatorics Foundation. ML was partially supported by J. A. Grochow's NSF award CISE-2047756 and the University of Colorado Boulder, Department of Computer Science Summer Research Fellowship.} }

\author[1]{Michael Levet}
\author[2]{Puck Rombach}
\author[3]{Nicholas Sieger}
\affil[1]{Department of Computer Science, University of Colorado Boulder}
\affil[2]{Department of Mathematics and Statistics, University of Vermont}
\affil[3]{Department of Mathematics, University of California San Diego}


\begin{document}
\maketitle

\begin{abstract}
In this paper, we show that the $(3k+4)$-dimensional Weisfeiler--Leman  algorithm can identify graphs of treewidth $k$ in $O(\log n)$ rounds. This improves the result of Grohe \& Verbitsky (ICALP 2006), who previously established the analogous result for $(4k+3)$-dimensional Weisfeiler--Leman. In light of the equivalence between Weisfeiler--Leman and the logic $\textsf{FO} + \textsf{C}$ (Cai, F\"urer, \& Immerman, \textit{Combinatorica} 1992), we obtain an improvement in the descriptive complexity for graphs of treewidth $k$. Precisely, if $G$ is a graph of treewidth $k$, then there exists a $(3k+5)$-variable formula $\varphi$ in $\textsf{FO} + \textsf{C}$ with quantifier depth $O(\log n)$ that identifies $G$ up to isomorphism.
\end{abstract}


\thispagestyle{empty}

\newpage

\setcounter{page}{1}

\section{Introduction}
\label{sec:introduction}


\noindent The \algprobm{Graph Isomorphism} problem (\algprobm{GI}) takes as input two graphs $G$ and $H$, and asks if there exists an isomorphism $\varphi : V(G) \to V(H)$. It is known that $\algprobm{GI} \in \textsf{NP} \cap \textsf{coAM}$. The $\algprobm{GI}$ problem is in particular conjectured to be $\textsf{NP}$-intermediate. That is, belonging to $\textsf{NP}$ but neither in $\textsf{P}$ nor $\textsf{NP}$-complete~\cite{Ladner}. Algorithmically, the best known upper-bound on its complexity is $n^{\Theta(\log^{2} n)}$, and is due to Babai~\cite{BabaiGraphIso}. It remains open as to whether $\algprobm{GI}$ belongs to $\textsf{P}$. There is considerable evidence suggesting that $\algprobm{GI}$ is not $\textsf{NP}$-complete~\cite{Schoning, BuhrmanHomer, ETH, BabaiGraphIso, GILowPP, ArvindKurur, MATHON1979131}. In contrast, the best known lower bound on the complexity of $\algprobm{GI}$ is $\textsf{DET}$~\cite{Toran}, which contains $\textsf{NL}$ and is a subclass of $\textsf{TC}^{1}$. 

The $k$-dimensional Weisfeiler--Leman algorithm ($k$-WL) serves as a key combinatorial tool in \algprobm{GI}. It works by iteratively coloring $k$-tuples of vertices in an isomorphism-invariant manner. On its own, Weisfeiler--Leman serves as an efficient polynomial-time isomorphism test for several families of graphs, including trees~\cite{edmonds_1965, ImmermanLander1990}, planar graphs~\cite{KieferPonomarenkoSchweitzer, GroheKieferPlanar}, graphs of bounded rank width~\cite{grohe2019canonisation}, graphs of bounded genus~\cite{GroheBoundedGenus, grohe_et_al:LIPIcs:2019:10693}, and graphs for which a specified minor $H$ is forbidden~\cite{GroheForbiddenMinor}. It is also worth noting that $1$-WL identifies almost all graphs~\cite{Rossman2009EhrenfeuchtFrassGO} and $2$-WL identifies almost all regular graphs~\cite{BollobasRegular, KuceraRegular}. In the case of graphs of bounded treewidth~\cite{GroheVerbitsky} and planar graphs~\cite{GroheVerbitsky, GroheKieferPlanar}, Weisfeiler--Leman serves even as an $\textsf{NC}$ isomorphism test. Despite the success of WL as an isomorphism test, it is insufficient to place \algprobm{GI} into $\textsf{P}$~\cite{CFI, NeuenSchweitzerIR}. Nonetheless, WL remains an active area of research. For instance, Babai's quasipolynomial-time algorithm~\cite{BabaiGraphIso} combines $O(\log n)$-WL with group theoretic techniques. 

Graphs of bounded treewidth have received significant interest, both for general isomorphism testing and via WL. Bodlaender~\cite{BodlaenderPolyIso} exhibited the first polynomial-time isomorphism test for this family. In 1999, Grohe \& Mari$\widetilde{\text{n}}$o~\cite{GroheMarino} showed that graphs of treewidth $k$ are identified by $(k+2)$-WL. Subsequently, Grohe \& Verbitsky~\cite{GroheVerbitsky} showed that the $(4k+3)$-WL algorithm identifies graphs of treewidth $k$ in $O(\log n)$ rounds, yielding a $\textsf{TC}^{1}$-- and the first $\textsf{NC}$-- isomorphism test. Wagner~\cite{WagnerBoundedTreewidth} subsequently showed that graphs of bounded treewidth can be canonized in $\textsf{AC}^{1}$. Das, Tor\'an, \& Wagner~\cite{DasToranWagner} further improved the bound to $\textsf{LogCFL} = \textsf{SAC}^{1}$ for isomorphism testing. Elberfeld \& Schweitzer~\cite{ElberfeldSchweitzer} finally exhibited a logspace canonization procedure for this family, which implies that isomorphism testing is $\textsf{L}$-complete under many-to-one $\textsf{AC}^{0} = \textsf{FO}$ reductions for graphs of bounded treewidth.


\noindent \\ \textbf{Main Results.} In this paper, we investigate the power of the Weisfeiler--Leman algorithm in deciding isomorphism for graphs of bounded treewidth. 

\begin{theorem} \label{thm:MainParallel1}
The $(3k+4)$-dimensional Weisfeiler--Leman algorithm identifies graphs of treewidth $k$ in $O(\log n)$ rounds.
\end{theorem}

In order to prove \Thm{thm:MainParallel1}, we utilize a result of~\cite{Bodlaender} that graphs of treewidth $k$ admit a \textit{binary} tree decomposition of width $\leq 3k+2$ and height $O(\log n)$. With this decomposition in hand, we leverage a pebbling strategy that is considerably simpler than that of both Kiefer \& Neuen \cite[Theorem~6.4]{KieferNeuenDecompose} (who follow the strategy of Grohe \& Mari$\widetilde{\text{n}}$o~\cite{GroheMarino}) and Grohe \& Verbitsky~\cite{GroheVerbitsky}.



\begin{remark}
Grohe \& Verbitsky~\cite{GroheVerbitsky} previously showed that the $(4k+3)$-WL identifies graphs of treewidth $k$ in $O(\log n)$ rounds. As a consequence, they obtained the first $\textsf{TC}^{1}$ isomorphism test for this family. In light of the close connections between Weisfeiler--Leman and $\textsf{FO} + \textsf{C}$~\cite{ImmermanLander1990, CFI}, they also obtained that if $G$ has treewidth $k$, then there exists a $(4k+4)$-variable formula $\varphi$ in $\textsf{FO} + \textsf{C}$ with quantifier depth $O(\log n)$ such that whenever $H \not \cong G$, $G \models \varphi$ and $H \not \models \varphi$. In light of \Thm{thm:MainParallel1}, we obtain the following improvement in the descriptive complexity for graphs of bounded treewidth.
\end{remark}

\begin{corollary} \label{cor:CorMain}
Let $G$ be a graph of treewidth $k$ and $H \not \cong G$. Then there exists a formula $\varphi_{G} \in \mathcal{C}_{3k+5, O(\log n)}$ that identifies $G$ up to isomorphism. That is, for any $H \not \cong G$, $G \models \varphi_{G}$ and $H \not \models \varphi_{G}$.
\end{corollary}



\section{Preliminaries}

\subsection{Weisfeiler--Leman}

We begin by recalling the Weisfeiler--Leman algorithm for graphs, which computes an isomorphism-invariant coloring. Let $\Gamma$ be a graph, and let $k \geq 2$ be an integer. The $k$-dimension Weisfeiler--Leman, or $k$-WL, algorithm begins by constructing an initial coloring $\chi_{0} : V(\Gamma)^{k} \to \mathcal{K}$, where $\mathcal{K}$ is our set of colors, by assigning each $k$-tuple a color based on its isomorphism type. That is, two $k$-tuples $(v_{1}, \ldots, v_{k})$ and $(u_{1}, \ldots, u_{k})$ receive the same color under $\chi_{0}$ if and only if the map $v_i \mapsto u_i$ (for all $i \in [k]$) is an isomorphism of the induced subgraphs $\Gamma[\{ v_{1}, \ldots, v_{k}\}]$ and $\Gamma[\{u_{1}, \ldots, u_{k}\}]$ and for all $i, j$, $v_i = v_j \Leftrightarrow u_i = u_j$. 

For $r \geq 0$, the coloring computed at the $r$th iteration of Weisfeiler--Leman is refined as follows. For a $k$-tuple $\overline{v} = (v_{1}, \ldots, v_{k})$ and a vertex $x \in V(\Gamma)$, define
\[
\overline{v}(v_{i}/x) = (v_{1}, \ldots, v_{i-1}, x, v_{i+1}, \ldots, v_{k}).
\]

\noindent The coloring computed at the $(r+1)$st iteration, denoted $\chi_{r+1}$, stores the color of the given $k$-tuple $\overline{v}$ at the $r$th iteration, as well as the colors under $\chi_{r}$ of the $k$-tuples obtained by substituting a single vertex in $\overline{v}$ for another vertex $x$. We examine this multiset of colors over all such vertices $x$. This is formalized as follows:
\begin{align*}
\chi_{r+1}(\overline{v}) = &( \chi_{r}(\overline{v}), \{\!\!\{ ( \chi_{r}(\overline{v}(v_{1}/x)), \ldots, \chi_{r}(\overline{v}(v_{k}/x) ) \bigr| x \in V(\Gamma) \}\!\!\} ),
\end{align*}
where $\{\!\!\{ \cdot \}\!\!\}$ denotes a multiset.

Note that the coloring $\chi_{r}$ computed at iteration $r$ induces a partition of $V(\Gamma)^{k}$ into color classes. The Weisfeiler--Leman algorithm terminates when this partition is not refined, that is, when the partition induced by $\chi_{r+1}$ is identical to that induced by $\chi_{r}$. The final coloring is referred to as the \textit{stable coloring}, which we denote $\chi_{\infty} := \chi_{r}$.


\begin{comment}
The $1$-dimensional Weisfeiler--Leman algorithm, sometimes referred to as \textit{Color Refinement}, works nearly identically. Two vertices of $\Gamma$ receive the same initial color if and only if they have the same degree. For the refinement step, we have that:
\[
\chi_{r+1}(u) = (\chi_{r}(u), \{\!\!\{ \chi_{r}(v) : v \in N(u) \}\!\!\} ).
\]

\noindent We have that $1$-WL terminates when the partition on the vertices is not refined.
\end{comment}

\begin{remark}
Grohe \& Verbitsky~\cite{GroheVerbitsky} previously showed that for fixed $k \geq 2$, the classical $k$-dimensional Weisfeiler--Leman algorithm for graphs can be effectively parallelized. Precisely, each iteration (including the initial coloring) can be implemented using a logspace uniform $\textsf{TC}^{0}$ circuit. 
\end{remark}

As we are interested in both the Weisfeiler--Leman dimension and the number of rounds, we will use the following notation.

\begin{definition}
Let $k \geq 2$ and $r \geq 1$ be integers. The $(k, r)$-WL  algorithm is obtained by running $k$-WL for $r$ rounds. Here, the initial coloring counts as the first round.
\end{definition}




\subsection{Pebbling Game}

We recall the bijective pebble game introduced by~\cite{Hella1989, Hella1993} for WL on graphs. This game is often used to show that two graphs $X$ and $Y$ cannot be distinguished by $k$-WL. The game is an Ehrenfeucht--Fra\"iss\'e game (c.f.,~\cite{Ebbinghaus:1994, Libkin}), with two players: Spoiler and Duplicator. We begin with $k+1$ pairs of pebbles, which are placed beside the graph. Each round proceeds as follows.
\begin{enumerate}
\item Spoiler picks up a pair of pebbles $(p_{i}, p_{i}^{\prime})$. 
\item We check the winning condition, which will be formalized later.
\item Duplicator chooses a bijection $f : V(X) \to V(Y)$.
\item Spoiler places $p_{i}$ on some vertex $v \in V(X)$. Then $p_{i}^{\prime}$ is placed on $f(v)$. 
\end{enumerate} 

Let $v_{1}, \ldots, v_{m}$ be the vertices of $X$ pebbled at the end of step 1, and let $v_{1}^{\prime}, \ldots, v_{m}^{\prime}$ be the corresponding pebbled vertices of $Y$. Spoiler wins precisely if the map $v_{\ell} \mapsto v_{\ell}^{\prime}$ does not extend to an isomorphism of the induced subgraphs $X[\{v_{1}, \ldots, v_{m}\}]$ and $Y[\{v_{1}^{\prime}, \ldots, v_{m}^{\prime}\}]$. Duplicator wins otherwise. Spoiler wins, by definition, at round $0$ if $X$ and $Y$ do not have the same number of vertices. We note that $X$ and $Y$ are not distinguished by the first $r$ rounds of $k$-WL if and only if Duplicator wins the first $r$ rounds of the $(k+1)$-pebble game~\cite{Hella1989, Hella1993, CFI}. 

\begin{remark}
In our work, we explicitly control for both pebbles and rounds. In our theorem statements, we state explicitly the number of pebbles on the board. If Spoiler can win with $k$ pebbles on the board, then we are playing in the $(k+1)$-pebble game. Note that $k$-WL corresponds to $k$-pebbles on the board.
\end{remark}



\subsection{Logics}

We recall key notions of first-order logic. We have a countable set of variables $\{x_{1}, x_{2}, \ldots, \}$. Formulas are defined inductively. For the basis, we have that $x_{i} = x_{j}$ is a formula for all pairs of variables. Now if $\varphi$ is a formula, then so are the following: $\varphi \land \varphi, \varphi \vee \varphi, \neg{\varphi}, \exists{x_{i}} \, \varphi,$ and $\forall{x_{i}} \, \varphi$. In order to define logics on graphs, we add a relation $E(x, y)$, where $E(x,y) = 1$ if and only if $\{x,y\}$ is an edge of our graph. In keeping with the conventions of~\cite{CFI}, we refer to the first-order logic with relation $R$ as $\mathcal{L}$ and its $k$-variable fragment as $\mathcal{L}_{k}$. We refer to the logic $\mathcal{C}$ as the logic obtained by adding counting quantifiers $\exists^{\geq n} x \, \varphi$ (there exist at least $n$ elements $x$ that satisfy $\varphi$) and $\exists{!n} \, x \, \varphi$ (there exist exactly $n$ elements $x$ that satisfy $\varphi$) as $\mathcal{C}$ and its $k$-variable fragment as $\mathcal{C}_{k}$. 

The \textit{quantifier depth} of a formula $\varphi$ (belonging to either $\mathcal{L}$ or $\mathcal{C}$) is the depth of its quantifier nesting. We denote the quantifier depth of $\varphi$ as $\text{qd}(\varphi)$ This is defined inductively as follows.
\begin{itemize}
    \item If $\varphi$ is atomic, then $\text{qd}(\varphi) = 0$.

    \item $\text{qd}(\neg \varphi) = \text{qd}(\varphi)$.

    \item $\text{qd}(\varphi_{1} \vee \varphi_{2}) = \text{qd}(\varphi_{1} \land \varphi_{2}) = \max\{ \text{qd}(\varphi_{1}), \text{qd}(\varphi_{2})\}$.

    \item $\text{qd}(Qx \, \varphi) = \text{qd}(\varphi) + 1$, where $Q$ is a quantifier in the logic.
\end{itemize}

\noindent We denote the fragment of $\mathcal{L}_{k}$ (respectively, $\mathcal{C}_{k}$) where the formulas have quantifier depth at most $r$ as $\mathcal{L}_{k,r}$ (respectively, $\mathcal{C}_{k,r}$). We note that the graphs $X$ and $Y$ are distinguished by $(k,r)$-WL if and only if there exists a formula $\varphi \in \mathcal{C}_{k+1,r}$ such that $X \models \varphi$ and $Y \not \models \varphi$. Furthermore, $X$ is identified by $(k,r)$-WL if and only if there exists a formula $\varphi \in \mathcal{C}_{k+1,r}$ such that (i) $X \models \varphi$, and (ii) for any $Y \not \cong H$, $Y \not \models \varphi$~\cite{ImmermanLander1990, CFI}.


%%treewidth%%
\section{Detecting Separators via Weisfeiler--Leman} \label{sec:separators}

In this section, we extend \cite[Section~4]{KieferNeuenDecompose} to show that Weisfeiler--Leman can detect separators in $O(\log n)$ rounds.

%\nsay{The initial goal of this section is to copy over Kiefer and Neuen's section 4 keeping track of the number of rounds of WL needed, then to copy over section 6. The main goal is to detect graphs of treewidth $k$ using $O(\log(n))$ rounds and $4k + 1$ pebbles? }



\begin{lemma}\label{lem:distances}
Let $G, H$ be graphs. Suppose that $(u, v) \mapsto (x,y)$ have been pebbled. Suppose that $\text{dist}(u, v) \neq \text{dist}(x,y)$. Then Spoiler can win with $1$ additional pebble and $O(\log n)$ rounds.
\end{lemma}

\begin{proof}
Let $f : V(G) \to V(H)$ be the bijection that Duplicator selects. Let $m$ be a midpoint on a shortest $u-v$ path. Spoiler pebbles $m$. Necessarily, we have that either $\text{dist}(u, m) < \text{dist}(x, f(m))$ or $\dist(m, v) < \dist(y, f(v))$. Without loss of generality, suppose that $\dist(u, m) < \dist(x, f(m))$. We iterate on the above argument, starting from $(u, m) \mapsto (x, f(m))$ and reusing the pebble on $v$. As $\dist(u, m) \leq \dist(u, v)/2$, we eventually reach in at most $\lceil \log_{2} n\rceil$ rounds an instance where $um \in E(G)$, but $f(u)f(m) \not \in E(H)$. The result now follows.
\end{proof}


\begin{comment}
\begin{proof}
We use induction on $\dist(u,v)$. Without loss of generality, assume that $\dist(u,v) < \dist(x,y)$ as we can exchange the roles of $G$ and $H$. 

If $\dist(u,v) = 1$, Spoiler has already won as $x$ and $y$ cannot be adjacent. 

Assume the claim holds for any $u', v'\in V(G)$ such that  $\dist(u',v') < k$, and assume that $\dist(u,v) = k$. Let $f:V(G)\to V(H)$ be the bijection selected by Duplicator so that $f(u) = x$ and $f(v) = y$. Let $m$ be the midpoint of a shortest $uv$ path closer to $u$. Then $\dist(u,m) \leq \dist(u,v)/2$. We claim that at least one of the following two conditions holds:
\begin{itemize}
    \item $\dist(f(u),f(m)) > \dist(u,m)$.
    \item $\dist(f(v),f(m)) > \dist(v,m)$.
\end{itemize}
Indeed, if neither condition held, then there is an $xy$ path of length at most $\dist(u,m) + \dist(v,m) = \dist(u,v)$ contradicting our assumption that $\dist(u,v) < \dist(x,y)$. 

Now we can give Spoiler's strategy. Given that $(u,v)\mapsto (x,y)$ have been pebbled, Spoiler places a pebble on $m$. By the above claim, either $\dist(f(u),f(m)) > \dist(u,m)$ or $\dist(f(v),f(m)) > \dist(v,m)$. Assume without loss of generality that $\dist((f(u),f(m)) > \dist(u,m)$. Spoiler plays the next round with the starting pair of $m,v$. As $\dist(u,m) < \dist(u,v)$, Spoiler can win for the pair $u,m$. Hence, Spoiler can win on the pair $u,v$. 

At the next step, Spoiler will take an additional round to remove the pebble on $v$, and then proceed as above.

%If $\dist(u,v) > \dist(x,y)$, Spoiler plays the same strategy as the above but chooses $m$ to be the vertex such that $f(m)$ is the closer midpoint of a shortest path from $x$ to $y$. 


Note that $\dist(u,m) \leq \dist(u,v)$, so there can at most $\log(\dist(u,v)) \leq \log(n)$ applications of the above strategy. As each application takes $2$ rounds of $k$-WL, it follows that we use only $O(\log(n))$ rounds. 
\end{proof}
\end{comment}

\begin{lemma} \label{lem:ColoredPaths}
Let $G(V, E, \chi), H(V, E, \chi)$ be vertex-colored graphs. Let $S$ be a set of colors. Suppose that $(u, v) \mapsto (x,y)$ have been pebbled. If there exists a $u-v$ path of length $d$ using only vertices from $S$, but no such $x-y$ path of length $d$ exists in $H$, then Spoiler can win with $1$ additional pebble and $O(\log n)$ additional rounds.
\end{lemma}

\begin{proof}
The pebble game works identically in the case of colored graphs. The only modification is that we check whether the map on the pebbled vertices satisfies $\chi(g_{i}) = \chi(f(g_{i}))$ for all $i$. In light of \Lem{lem:distances}, the result now follows.
\end{proof}


\begin{lemma}[Compare rounds c.f. {\cite[Lemma~4.2]{KieferNeuenDecompose}}] \label{lem:ModifiedKN4.2}
Let $G$ be a connected graph, and let $\{w_{1}, w_{2}\}$ be a $2$-separator of $G$. Let $C$ be a connected component of $G - \{w_{1}, w_{2}\}$ such that $|C| < n/2$, and let $v \in V(C)$. Let $u \in V(G)$ such that the following conditions hold:
\begin{enumerate}[label=(\alph*)]
\item $\text{dist}(u,v) \leq 2$, and 
\item $\text{dist}(u, w_{i}) \leq \text{dist}(v, w_{i}) - 2$ for both $i = 1, 2$.
\end{enumerate}

\noindent Suppose that $(w_{1}, w_{2}, u) \mapsto (w_{1}, w_{2}, v)$ have been pebbled. We have that Spoiler can win with $2$ additional pebbles and $O(\log n)$ additional rounds.
\end{lemma}

\begin{proof}
We apply Lemma~\ref{lem:distances} to the pairs $(w_1,u)$ and $(w_1,v)$. As $\text{dist}(u, w_{1}) \leq \text{dist}(v, w_{1}) - 2$, Spoiler can distinguish $u$ from $v$ with $2$ additional pebbles and $O(\log(n))$ additional rounds. As $w_2$ must remain pebbled, we have that Spoiler actually requires $2$ additional pebbles rather than the $1$ additional pebble prescribed by Lemma~\ref{lem:distances}.
\end{proof}


\begin{lemma}[Compare rounds c.f. {\cite[Lemma~4.3]{KieferNeuenDecompose}}]\label{lem: distance bound}
Let $G = (U, V, E)$ be a $2$-connected bipartite graph with $n$ vertices. Let $r \geq \Omega(\log n)$, and let $\chi_{r}$ be the coloring computed by $(4, r)$-WL. Suppose that:
\begin{enumerate}[label=(\alph*)]
\item $\chi_{r}(u, u) = \chi_{r}(u',u')$ for all $u, u' \in U$,
\item $\chi_{r}(v, v) = \chi_{r}(v',v')$ for all $v, v' \in V$, and
\item $G$ has a $2$-separator $\{w_{1}, w_{2}\}$, with $w_{1} \in U$ and $w_{2} \in V$.
\end{enumerate}

\noindent Let $d := \text{dist}(w_{1}, w_{2})$, and let $C$ be the vertex set of a connected component of $G - \{ w_{1}, w_{2}\}$ with $|C| \leq (n-2)/2$. Then $\text{dist}(v, w_{i}) \leq d+1$ for all $v \in C$ and all $i \in \{1,2\}$.
\end{lemma}

\begin{proof}
We use \Lem{lem:ModifiedKN4.2} in place of \cite[Lemma~4.2]{KieferNeuenDecompose}. The proof now goes through, \textit{mutatis mutandis}. 
\end{proof}

\begin{lemma}[Compare rounds c.f. {\cite[Lemma~4.4]{KieferNeuenDecompose}}] \label{lem:KNLemma4.4}
    Let $G = (U,V,E)$ be a $2$-connected bipartite graph with $n$ vertices. Let $k \geq 4, r \geq \Omega(\log n)$, and let $\chi_{G}$ be the coloring computed by $(k,r)$-WL. Suppose that the following holds:
    \begin{enumerate}
        \item $\chi_{r}(u, u) = \chi_{r}(u',u')$ for all $u, u' \in U$,
        \item $\chi_{r}(v, v) = \chi_{r}(v',v')$ for all $v, v' \in V$, and
        \item $G$ has a $2$-separator $\{w_{1}, w_{2}\}$.
    \end{enumerate} Then $w_1w_2\notin E(G)$.
\end{lemma}

\begin{proof}
We use \Lem{lem: distance bound} in place of \cite[Lemma~4.3]{KieferNeuenDecompose}. The proof now goes through \textit{mutatis mutandis}.
\end{proof}



Kiefer \& Neuen use \cite[Corollary~7]{KieferPonomarenkoSchweitzer} to distinguish cut from non-cut vertices. We may apply Lemma~\ref{lem:distances} instead, in the following manner.

\begin{lemma} \label{lem:CutVertices}
Let $G, H$ be connected graphs. Suppose $v$ is a cut vertex, and $v'$ is not a cut vertex. We have that $(4, O(\log n))$-WL will distinguish $v$ from $v'$.
\end{lemma}

\begin{proof}
Suppose that $v \mapsto v'$ has been pebbled. We may thus treat $v \mapsto v'$ as having been individualized. Now as $v$ is a cut vertex for $G$, there exist vertices $u, w$ that belong to different components of $G - \{v\}$. At the next two rounds, Spoiler pebbles $(u, w)$. Let $x,y \in V(H)$ be the corresponding pebbled elements. As $H$ is connected and $v'$ is not a cut vertex, $x, y$ belong to the same connected component. We now apply Lemma~\ref{lem:ColoredPaths} to the color class specified by individualizing $v \mapsto v'$. 
\end{proof}

\begin{theorem}[Compare rounds c.f. {\cite[Theorem~4.1]{KieferNeuenDecompose}}] \label{thm:KieferNeuenThm4.1}
    Let $G$ be a $2$-connected graph with a $2$-separator $\{w_1,w_2\}$. Now let $k \geq 4, r \geq \Omega(\log n)$. Let $\chi_{G, r}$ be the coloring computed by $(k,r)$-WL. Suppose that for every $v\in V(G)$, either $\chi_G(v,v) = \chi_G(w_1,w_1)$ or $\chi_G(v,v) = \chi_G(w_2,w_2)$. Then $G$ is a cycle.
\end{theorem}

\begin{proof}
We follow the strategy of \cite[Theorem~4.1]{KieferNeuenDecompose}. By \cite[Lemma~3.8]{KieferNeuenDecompose}, we may assume without loss of generality that $\chi_{G,r}(w_{1}, w_{1}) \neq \chi_{G,r}(w_{2}, w_{2})$. The proof is by induction on $n := |V(G)|$. When $n \leq 4$, a case analysis of the possible graphs $G$ yields the result.

Let $n \geq 5$. It suffices to prove the statement for an $n$-vertex graph $G$ with maximum edge set that satisfies the assumptions of the theorem. Define:
\begin{align*}
U &:= \{ v \in V(G) : \chi_{G,r}(v,v) = \chi_{G,r}(w_{1},w_{1}) \}\\
V &:= \{ v \in V(G) : \chi_{G,r}(v,v) = \chi_{G,r}(w_{2},w_{2}) \}.
\end{align*}

\noindent Let $U_{1}, \ldots, U_{k}$ be the connected components of the induced subgraph $G[U]$, and let $V_{1}, \ldots, V_{\ell}$ be the connected components of the induced subgraph $G[V]$. Without loss of generality, suppose that $w_{1} \in U_{1}$ and $w_{2} \in V_{1}$. Let $C$ be the vertex set of a connected component of $G - \{ w_{1}, w_{2}\}$ of size at most $(n-2)/2$. Let $C' := C - (C \cup \{w_{1},w_{2}\})$.

\begin{quotation}
\noindent \textbf{Claim 1 }[(c.f. \cite[Claim~4.7]{KieferNeuenDecompose}]. Suppose that $C \subseteq U_{1} \cup V_{1}$ or $C' \subseteq U_{1} \cup V_{1}$. Then $G$ is a cycle.

\begin{proof}[Proof (Sketch).]
The proof is virtually identical to \cite[Claim~4.7]{KieferNeuenDecompose}. We outline the changes. First, Kiefer \& Neuen use the fact that $2$-WL can distinguish arcs $(u, v)$ where $u, v$ belong to the same connected component, from arcs $(u', v')$ where $u', v'$ belong to different connected components. By Lemma~\ref{lem:distances}, we have that $(4, O(\log n))$-WL will distinguish $(u, v)$ from $(u', v')$. Furthermore, we make use of Lemma~\ref{lem:CutVertices} instead of \cite[Corollary~7]{KieferPonomarenkoSchweitzer} to obtain that $(4, O(\log n))$-WL detects cut vertices. Thus, we may use the coloring computed by $(4, O(\log n))$-WL in place of the stable coloring computed by $2$-WL. The remainder of the proof from Kiefer \& Neuen~\cite{KieferNeuenDecompose} goes through \textit{mutatis mutandis}.
\end{proof}
\end{quotation}


\noindent So we now assume that $C \not \subseteq U_{1} \cup V_{1}$ and $C' \not \subseteq U_{1} \cup V_{1}$. Now as $w_{1} \in U_{1}, w_{2} \in V_{1}$, and $G[U], G[V]$ do not have any cut vertices, we have that $G[U]$ and $G[V]$ contain at least two connected components. Let $G'$ be the graph with $V(G') = \{ U_{1}, \ldots, U_{k}, V_{1}, \ldots, V_{\ell}\}$. Now the edges of $G'$ are precisely the pairs $U_{i}V_{j}$ where there exist $u \in U_{i}, v \in V_{j}$ such that $uv \in E(G)$. We now claim that $G'$ satisfies the assumptions of this theorem. We first note that $G' - \{ U_{1}, V_{1}\}$ is not connected. Thus, $\{U_{1}, V_{1}\}$ forms a $2$-separator of $G'$. Now the coloring computed by $(k,r)$-WL on $G$ refines the coloring computed by $(k,r)$-WL on $G[U]$. Furthermore, as every vertex of $U$ has the same color as $w_{1}$, we have that every component of $G[U]$ has the same color as $U_{1}$. By a similar argument, every component of $G[V]$ has the same color as $V_{1}$.

Since $G$ is connected, we have that $G'$ is connected. It remains to be shown that $G'$ is $2$-connected. The argument is identical to that provided by Kiefer \& Neuen~\cite{KieferNeuenDecompose}, except that we use Lemma~\ref{lem:CutVertices} instead of \cite[Corollary~7]{KieferPonomarenkoSchweitzer}. We have already established that the $U_{i}, U_{j}$ have the same color for all $i,j$, and similarly $V_{i}, V_{j}$ have the same color for all $i, j$.




Suppose that $|V(G')| < |V(G)|$. Kiefer \& Neuen \cite[Theorem~4.1]{KieferNeuenDecompose} previously established that $G'$ is a cycle. Their claim holds, though we use our Claim 1 in place of their \cite[Claim~4.7]{KieferNeuenDecompose}.

We note that $|V(G')| \leq |V(G)|$. So if $|V(G')| \not < V(G)|$, then we have that $|V(G')| = |V(G)|$. In this case, $G$ is bipartite with bipartition $\{U, V\}$, where all the $U_{i}, V_{j}$ sets are singletons. By \Lem{lem:KNLemma4.4}, we have that $w_{1}w_{2} \not \in E(G)$. Let $d := \text{dist}(w_{1}, w_{2})$. Note that $d$ is odd, and so $d \geq 3$.

\begin{quotation}
\noindent \textbf{Claim 2 }[c.f. \cite[Claim~4.8]{KieferNeuenDecompose}]. Let $u,v \in V(G)$ such that $\text{dist}(u, v) < d$. Then there is a unique shortest path from $u$ to $v$.

\begin{proof}
The proof is similar to \cite[Claim~4.8]{KieferNeuenDecompose}. We outline the changes. Kiefer \& Neuen defined an auxiliary graph $G''$, where $V(G'') = V(G)$ and 
\[
E(G'') = E(G) \cup \{ u''v'' : \chi_{G}(u, v) = \chi_{G}(u'', v'') \}. 
\]

\noindent (Kiefer \& Neuen refer to $G''$ as $G'$. However, to avoid overloading variables, we will refer to the auxiliary graph as $G''$ so as not to avoid confusion with the bipartite graph $G'$ under consideration leading up to Claim 2.) We have previously established that under the coloring computed by $(4, O(\log n))$-WL, we have that $U_{i}, U_{j}$ have the same color for all $i,j$, and similarly $V_{i}, V_{j}$ have the same color for all $i, j$. But the $U_{i}, V_{j}$ sets are all singletons. Furthermore, the vertices in $U$ all have the same color as $w_{1}$, and the vertices in $V$ all have the same color as $w_{2}$. So for every $v \in V(G)$, $v$ receives the same color as either $w_{1}$ or $w_{2}$. The remainder of the proof from Kiefer \& Neuen~\cite{KieferNeuenDecompose} goes through \textit{mutatis mutandis}.
\end{proof}
\end{quotation}


The remainder of the proof is virtually identical to that of Kiefer \& Neuen, with the exception of using some of our results in place of theirs. We briefly describe the changes. In \cite[Claim~4.9]{KieferNeuenDecompose}, we use (i) our Claim 2 instead of their \cite[Claim~4.8]{KieferNeuenDecompose}, and (ii) our \Lem{lem:ModifiedKN4.2} instead of their \cite[Lemma~4.2]{KieferNeuenDecompose}. The modifications to \cite[Claims~4.10 and 4.11]{KieferNeuenDecompose} follow similarly, using as well our modifications to \cite[Claim~4.9]{KieferNeuenDecompose}. We note that the proof of \cite[Claim~4.11]{KieferNeuenDecompose} makes use of $2$-WL (without control on rounds) via  \cite[Theorem~3.15]{KieferNeuenDecompose}. However, since we are considering vertices with distance $\leq 4$ from $\{w_{1}, w_{2}\}$, we may instead use the coloring computed by $(4, O(1))$-WL. The result now follows.
\end{proof}



\section{Detecting Decompositions via Weisfeiler--Leman}

Let $S$ be a set of colors. We say that a path $u_{0}, u_{1}, \ldots, u_{\ell}$ \textit{avoids} $S$ if $\chi_{G}(u_{i}, u_{i}) \not \in S$ for all $i \in [\ell-1]$. Note that no restrictions are imposed on the endpoints. By \Lem{lem:ColoredPaths}, $(4, O(\log n))$-WL detects whether a path avoids $S$. Let $r \geq 0$ and define $G|_{S,r}(V, E)$ to be the graph with vertex set $V = \{ v \in V(G) : \chi_{G,r}(v,v) \in S\}$ and edge set:
\[
E := \{ uv : \text{ there exists an } S\text{-avoiding path from } u \text{ to } v \}.
\]

\noindent Similarly to the work in Section~\ref{sec:separators}, we extend several results of Kiefer \& Neuen~\cite{KieferNeuenDecompose}. The main change is to show that we can use $(4, O(\log n))$-WL, where Kiefer \& Neuen instead use $2$-WL without controlling for rounds. The main differences between their approach and ours are most apparent in Corollary~\ref{cor:5.3}.


\begin{lemma}[Compare c.f. {\cite[Lemma~5.1]{KieferNeuenDecompose}}] \label{lem:KNLem5.1}
Let $G$ be a graph, $r \geq \Omega(\log n)$, and let $S \subseteq \{ \chi_{G,r}(v,v) : v \in V(G) \}.$ Then $\chi_{G|_{S},r}(v,v) = \chi_{G|_{S},r}(u,u)$ for all $u, v \in V(G)$ satisfying $\chi_{G,r}(v,v) = \chi_{G,r}(u,u)$.
\end{lemma}

\begin{proof}
By \Lem{lem:ColoredPaths}, $(4, O(\log n))$-WL detects whether there is a path between two vertices that avoid $S$. So there exists a set of colors $T$ such that $uv \in E(G|_{S})$ if and only if $\chi_{G}(u,v) \in T$. It follows that every refinement step in $G|_{S}$ will also be performed in $G$.
\end{proof}


\begin{theorem}[Compare rounds c.f. {\cite[Theorem~5.2]{KieferNeuenDecompose}}] \label{thm:KNThm5.2}
Let $G, H$ be connected graphs, and suppose $G$ is $2$-connected. Let $r \geq \Omega(\log n)$. Let $w_{1}, w_{2} \in V(G)$ s.t. $\{w_{1}, w_{2}\}$ forms a $2$-separator. Let $v_{1}, v_{2} \in V(H)$ such that $\chi_{G,r}(w_{1}, w_{2}) = \chi_{H, r}(v_{1}, v_{2})$. Then $\{ v_{1}, v_{2}\}$ forms a $2$-separator in $H$.
\end{theorem}

\begin{proof}
The argument is identical as in \cite[Theorem~5.2]{KieferNeuenDecompose}, with the exception that we use the coloring computed by $(4,r)$-WL rather than the stable coloring computed by $2$-WL. The result now follows.
\end{proof}


\begin{corollary}[Compare rounds c.f. {\cite[Corollary~5.3]{KieferNeuenDecompose}}] \label{cor:5.3}
Let $k \geq 2$ and $r \geq \Omega(\log n)$. Let $\chi_{G,r}$ be the coloring computed by $(\max\{k+2,4\},r)$-WL applied to $G$, and define $\chi_{H,r}$ analogously. Let $G, H$ be graphs, and suppose $\{ w_{1}, \ldots, w_{k}\} \subseteq V(G)$ is a $k$-separator in $G$. If $\chi_{G,r}(w_{1}, \ldots, w_{k}) = \chi_{H,r}(v_{1}, \ldots, v_{k})$, then $\{ v_{1}, \ldots, v_{k}\}$ forms a $k$-separator in $H$.
\end{corollary}

\begin{proof}
Our strategy follows that of \cite[Corollary~5.3]{KieferNeuenDecompose}. Suppose first that $k = 2$. As $G$ has a $2$-separator by assumption, we have that the vertex connectivity number $\kappa(G) \leq 2$. We consider several cases.
\begin{itemize}
\item \textbf{Case 1:} Suppose first that $G$ and $H$ are both $2$-connected. The result follows from \Thm{thm:KNThm5.2}.

\item \textbf{Case 2:} Suppose instead that one such graph is connected, but not $2$-connected. Without loss of generality, suppose that $\kappa(G) = 2$ and $\kappa(H) = 1$. Then $H$ contains a cut vertex, while $G$ does not. So regardless of the bijection $f : V(G) \to V(H)$ that Duplicator selects, Spoiler begins by pebbling a cut vertex $w' \in V(H)$. Let $w = f^{-1}(w') \in V(G)$. By \Lem{lem:CutVertices}, Spoiler wins with $3$ additional pebbles and $O(\log n)$ additional rounds.

\item \textbf{Case 3:} Suppose now that $\kappa(G) \leq 2$ and $\kappa(H) > 2$. By Case 2, we may assume that $\kappa(G) = 2$. Now $w_{1}, w_{2}$ must lie on a common $2$-connected component of $G$. Otherwise, they do not form a $2$-separator. Let $C_{G}$ denote the vertex set of this component. Suppose that $v_{1}, v_{2}$ do not lie on the same $2$-connected component of $H$. We will show that Spoiler has a winning strategy in the pebble game, starting from the configuration $((w_{1}, w_{2}), (v_{1}, v_{2}))$. Spoiler pebbles a $2$-separator $(x_{1}, x_{2})$ other than $w_{1}, w_{2}$ of $C_{G}$. Let $(y_{1}, y_{2})$ be the corresponding pebbled vertices in $H$. Regardless of Duplicator's choice of bijections at each round, we will have that either (i) $H - \{ y_{1}, y_{2}\}$ is connected or (ii) $v_{1}, v_{2}$ are on different components of $H - \{y_{1}, y_{2}\}$. In either case, we have by \Lem{lem:distances} that Spoiler wins with $2$ additional pebbles and $O(\log n)$ rounds.

So we may now assume that $v_{1}, v_{2}$ lie on the same $2$-connected component, which we call $C_{H}$. By an argument similar to the one above, we have that if $w, w' \in C_{G}$ and $v \in C_{H}, v' \not \in C_{H}$, then $(4, O(\log n))$-WL will distinguish $(w, w')$ from $(v, v')$. We claim that this implies that $(4, O(\log n))$-WL applied to $(G[C_{G}], H[C_{H}])$ will fail to distinguish $(w_{1}, w_{2})$ from $(v_{1}, v_{2})$. Suppose for a contradiction that this is not the case. In a similar manner as Kiefer \& Neuen~\cite{KieferNeuenDecompose}, we consider the pebble game on $(G[C_{G}], H[C_{H}])$ from the initial configuration $((w_{1}, w_{2}), (v_{1}, v_{2}))$. We first observe that if Duplicator selects a bijection $f : V(G) \to V(H)$ that does not map $f(C_{G}) = C_{H}$ setwise, then Spoiler can (without loss of generality) pebble some vertex $w \in C_{G}$ such that $f(w) \not \in C_{H}$. As $((w_{1}, w_{2}), (v_{1}, v_{2}))$ have been pebbled, we have by \Lem{lem:distances} that Spoiler wins with $2$ additional pebbles and $O(\log n)$ additional rounds. Therefore, a winning strategy in the pebble game on $(G[C_{G}], H[C_{H}])$ from the initial configuration $((w_{1}, w_{2}), (v_{1}, v_{2}))$ can be lifted to a winning strategy in the pebble game on $(G, H)$. As Spoiler wins with a total of $4$ pebbles and $O(\log n)$ rounds, this contradicts the assumption that $(4,r)$-WL applied to $(G, H)$ fails to distinguish $(w_{1}, w_{2})$ from $(v_{1}, v_{2})$.

By \Thm{thm:KNThm5.2}, it follows that $(v_{1}, v_{2})$ is a $2$-separator in $H[C_{H}]$, and hence a $2$-separator in $H$.
\end{itemize} 

\noindent \\ We now consider the case when $k > 2$. Let $G, H$ be graphs, and suppose that $\{ w_{1}, \ldots, w_{k} \}$ is a $k$-separator in $G$. Let $\{v_{1}, \ldots, v_{k}\} \subseteq V(H)$ such that $(\max\{k+2,4\}, r)$-WL fails to distinguish $(w_{1}, \ldots, w_{k})$ from $(v_{1}, \ldots, v_{k})$. Thus, Duplicator has a winning strategy in the $(k+3)$-pebble, $r$-round game, starting from the configuration $((w_{1}, \ldots, w_{k}), (v_{1}, \ldots, v_{k}))$. Let $G' := G - \{ w_{1}, \ldots, w_{k}\}$ and $H' := H - \{ v_{1}, \ldots, v_{k}\}$. We claim that $(4,r)$-WL applied to $(G', H')$ fails to distinguish $(w_{k-1}, w_{k})$ from $(v_{k-1}, v_{k})$. Indeed, if this is not the case, Spoiler can win in the $4$-pebble, $r$-round game starting from $((w_{k-1}, w_{k}), (v_{k-1}, v_{k}))$. Spoiler's strategy can be lifted to a winning strategy in the $(k+3)$-pebble, $r$-round game on $(G, H)$ by never moving the pebbles on $(w_{1}, \ldots, w_{k-2}) \mapsto (v_{1}, \ldots, v_{k-2})$, contradicting the assumption that $(\max\{k+2,4\}, r)$-WL fails to distinguish $(w_{1}, \ldots, w_{k})$ from $(v_{1}, \ldots, v_{k})$.

Now we may assume that $G', H'$ both fail to be connected; otherwise, by \Lem{lem:distances}, Spoiler can win in the pebble game on $G', H'$ using $3$ pebbles and $O(\log n)$ rounds. The remainder of the argument proceeds identically as in \cite[Corollary~5.3]{KieferNeuenDecompose}. We note that in the case when $G', H'$ have at most two vertices, Kiefer \& Neuen appeal to \cite[Corollary~7]{KieferPonomarenkoSchweitzer}, which implicitly uses the full power of $2$-WL. However, as $G', H'$ have at most $2$ vertices, $2$-WL terminates in at most $2$ rounds when applied to $G', H'$. The result now follows.
\end{proof}



\section{Proof of Main Result}

Our goal in this section is to prove the following.

\begin{theorem} \label{thm:LogarithmicTreewidth}
Let $G$ be a graph with treewidth $k$, and let $H \not \cong G$. We have that $(3k+4, O(\log n))$-WL distinguishes $G$ from $H$.
\end{theorem}

One key ingredient is that WL can detect separators with only $O(\log n)$ rounds. To this end, we recall the following notation introduced in~\cite{KieferNeuenDecompose}. Let $(v_{1}, \ldots, v_{k+1})$ be a $(k+1)$-tuple of vertices in a graph $G$. Define $s_{G}(v_{1}, \ldots, v_{k+1}) := |C|$, where $C$ is the vertex set of the unique connected component of $G - \{ v_{1}, \ldots, v_{k}\}$ containing $v_{k+1}$. If $v_{k+1} \in \{ v_{1}, \ldots, v_{k}\}$, then $s_{G}(v_{1}, \ldots, v_{k+1}) = 0$.

\begin{lemma} \label{lem:Lem6.2-Connected}
Let $G, H$ be graphs that are connected, but not $2$-connected. Let $(v_{1}, v_{2}, v_{3}) \in V(G)^{3}, (w_{1}, w_{2}, w_{3}) \in V(H)^{3}$. If $s_{G}(v_{1}, v_{2}, v_{3}) \neq s_{H}(v_{1}, v_{2}, v_{3})$, then Spoiler can win with $3$ additional pebbles and $O(\log n)$ rounds starting from the configuration $((v_{1}, v_{2}, v_{3}), (w_{1}, w_{2}, w_{3}))$.
\end{lemma}

\begin{proof}
Without loss of generality, we may assume that $v_{3} \not \in \{v_{1}, v_{2}\}$ and $w_{3} \not \in \{w_{1}, w_{2}\}$. Let $C$ be the connected component of $G - \{v_{1}\}$ containing $v_{3}$, and let $C'$ be the connected component of $H - \{w_{1}\}$ containing $w_{1}$. If $|C| \neq |C'|$, then by \Lem{lem:distances}, Spoiler can win with $3$ additional pebbles and $O(\log n)$ rounds. By a similar argument, Spoiler can win if $v_{2} \in C$ and $w_{2} \not \in C'$ (or vice-versa).

Now suppose that both $v_{1}, v_{2}$ are cut vertices. We may assume that $w_{1}, w_{2}$ are also both cut vertices. Otherwise, by \Lem{lem:distances}, Spoiler can win with $3$ additional pebbles and $O(\log n)$ rounds. By an argument similar to the one in the preceding paragraph, Spoiler can win with $3$ additional pebbles and $O(\log n)$ rounds if for some $i \in \{1, 2\}$, one of the following conditions fails to hold:
\begin{itemize}
\item $v_{3}$ lies in the same connected component of $v_{2-i}$ in $G  - \{v_{i}\}$ if and only if $w_{3}$ lies in the same connected component of $w_{2-i}$ in $H - \{w_{i}\}$.
\item The size of the connected component in $G - \{v_{i}\}$ containing $v_{3}$ is the same as the size of the connected component in $H - \{ w_{i}\}$ containing $w_{3}$.
\end{itemize}

\noindent The above conditions determine $s_{G}, s_{H}$ in this case. Thus, if $s_{G} \neq s_{H}$, then Spoiler can win with $3$ additional pebbles and $O(\log n)$ additional rounds. 
\end{proof}

\begin{lemma} \label{lem:Lem6.2-2Connected}
Let $G, H$ be $2$-connected graphs on $n$ vertices. Let $(v_{1}, v_{2}, v_{3}) \in V(G)^{3}, (w_{1}, w_{2}, w_{3}) \in V(H)^{3}$. If $s_{G}(v_{1}, v_{2}, v_{3}) \neq s_{H}(v_{1}, v_{2}, v_{3})$, then Spoiler can win with $3$ additional pebbles and $O(\log n)$ rounds starting from the configuration $((v_{1}, v_{2}, v_{3}), (w_{1}, w_{2}, w_{3}))$.
\end{lemma}

\begin{proof}
We follow the strategy of \cite[Theorem~5.8]{KieferNeuenDecompose}. If neither $\{v_{1}, v_{2}\}$ nor $\{w_{1}, w_{2}\}$ are $2$-separators, then $G - \{ v_{1}, v_{2}\}$ and $H - \{ w_{1}, w_{2}\}$ are connected. As $|G| = |H|$ by assumption, the result vacuously holds.

If $\{v_{1}, v_{2}\}$ is a $2$-separator and $\{w_{1}, w_{2}\}$ is not (or vice-versa), then by \Thm{thm:KNThm5.2}, Spoiler can win with $3$ additional pebbles (reusing the pebble pair on $v_{3} \mapsto w_{3}$) and $O(\log n)$ rounds. 

So now suppose that $\{ v_{1}, v_{2}\}$ is a $2$-separator in $G$ and $ \{w_{1}, w_{2}\}$ is a $2$-separator in $H$`. Let $S = \{ \chi_{G,r}(v_{1}, v_{2}), \chi_{G,r}(v_{2}, v_{2})\}$ and $A = \{ \chi_{H,r}(w_{1}, w_{1}), \chi_{H,r}(w_{2}, w_{2})$. Define $G'$ to be the graph with vertex set $V(G') := \{ v \in V(G) : \chi_{G,r}(v,v) \in S\}$ and $E(G')$ to be the pairs $uv$ where there is a $u-v$ path in $G$ that avoids $S$. Define $H'$ analogously with respect to the set $A$. By the proof of \cite[Theorem~5.2]{KieferNeuenDecompose}, $G', H'$ are $2$-connected. We have several cases to consider.
\begin{itemize}
\item \textbf{Case 1:} Suppose that $|V(G')| = 2$. This case is handled identically as in the proof of \cite[Theorem~5.8]{KieferNeuenDecompose}.

\item \textbf{Case 2:} Suppose instead that there is a vertex set $C$ such that $V(G') \subseteq C \cup \{g_{1}, g_{2}\}.$ This case is handled identically as in the proof of \cite[Theorem~5.8]{KieferNeuenDecompose}.

\item \textbf{Case 3:} If neither Case 1 nor Case 2 are satisfied, then $\{v_{1}, v_{2}\}$ is a $2$-separator in $G'$. By \Lem{lem:KNLem5.1} and \Thm{thm:KieferNeuenThm4.1}, we have that $G'$ is a cycle. As $\{v_{1}, v_{2}\}$ is a $2$-separator for $G'$, we have that $|V(G')| \geq 4$ and $v_{1}v_{2} \not \in E(V')$. By similar argument, we have that $H'$ is a cycle, $|V(H')| \geq 4$, and $w_{1}w_{2} \not \in E(H')$. The remainder of the argument follows identically as in the proof of \cite[Theorem~5.8]{KieferNeuenDecompose}.
\end{itemize}
\end{proof}

\begin{lemma}[Compare rounds c.f. {\cite[Lemma~6.2]{KieferNeuenDecompose}}] \label{lem:6.2}
Let $k \geq 2$. Let $G, H$ be graphs. Let $(v_{1}, \ldots, v_{k+1}) \in V(G)^{k+1}, (w_{1}, \ldots, w_{k+1}) \in V(H)^{k+1}$. If $s_{G}(v_{1}, \ldots, v_{k+1}) \neq s_{H}(w_{1}, \ldots, w_{k+1})$, then Spoiler can in the $(\max\{k+3,4\})$-pebble game using at most $O(\log n)$ rounds, starting from the initial configuration \\ $((v_{1}, \ldots, v_{k+1}), (w_{1}, \ldots, w_{k+1})).$
\end{lemma}

\begin{proof}
We follow the strategy of \cite[Lemma~6.2]{KieferNeuenDecompose}. Without loss of generality, we may assume that $v_{k+1} \not \in \{v_{1}, \ldots, v_{k}\}$, $w_{k+1} \not \in \{w_{1}, \ldots, w_{k}\}$, and that $G$ and $H$ are connected graphs of the same order. By \Cor{cor:5.3}, we may assume that $G$ if $2$-connected if and only if $H$ is $2$-connected; otherwise, Spoiler wins with $\max\{k+3,4\}$ pebbles in $O(\log n)$ rounds. Consider first the case when $k = 2$. 
\begin{itemize}
\item \textbf{Case 1:} Assume first that both graphs are connected, but not $2$-connected. This case is handled by \Lem{lem:Lem6.2-Connected}.

\item \textbf{Case 2:} Suppose instead that $G$ and $H$ are both $2$-connected. This case is handled by \Lem{lem:Lem6.2-2Connected}.
\end{itemize}

\noindent Finally, consider the case in which $k > 2$. Let $\hat{G} := G - \{ v_{1}, \ldots, v_{k-2}\}$ and $\hat{H} := H - \{ w_{1}, \ldots, w_{k-2}\}$. Then $s_{\hat{G}}(v_{k-1}, v_{k}, v_{k+1}) \neq s_{\hat{H}}(w_{k-1}, w_{k}, w_{k+1})$. So Spoiler can win with $3$ additional pebbles and $O(\log n)$ rounds in the pebble game on $(\hat{G}, \hat{H})$ starting from the configuration $((v_{k-1}, v_{k}, v_{k+1}), (w_{k-1}, w_{k}, w_{k+1}))$. But then Spoiler can win in the $\max\{k+2, 4\}$-pebble game with $O(\log n)$  rounds starting from the initial configuration $((v_{1}, \ldots, v_{k+1}), (w_{1}, \ldots, w_{k+1}))$ by never moving the first $k-2$ pebbles.
\end{proof}



\noindent \\ We now prove \Thm{thm:LogarithmicTreewidth}.

\begin{proof}[Proof of \Thm{thm:LogarithmicTreewidth}]
Let $G$ be a graph of treewidth $k$, and let $H$ be a graph not isomorphic to $G$. Let $(T, \beta)$ be a tree decomposition for $G$ of width $\leq 3k+2$ and height $O(\log n)$, with $T$ a binary tree, as prescribed by~\cite{Bodlaender}. Let $s$ be the root node of $T$. Spoiler begins by pebbling the vertices of $\beta(s)$, using at most $3k+2$ pebbles. Now as $(T, \beta)$ is a tree decomposition, we have that for any edge $uv \in E(T)$, $\beta(u) \cap \beta(v)$ is a separator of $G$. Let $f : V(G) \to V(H)$ be the bijection that Duplicator selects. By \Cor{cor:5.3}, if there exists a subset $S$ of the pebbled vertices in $G$, such that $f(S)$ is not a separator of the same size, then Spoiler can win with $2$ additional pebbles and $O(\log n)$ additional rounds. This yields a total of $3k+4$ pebbles. So suppose that $f$ preserves separators of pebbled vertices.

Now let $\ell$ be the left child of $s$ in $T$, and let $r$ be the right child of $s$ in $T$. Denote the separators $S_{\ell} := \beta(\ell) \cap \beta(s)$ and $S_{r} := \beta(r) \cap \beta(s)$. By \Lem{lem:6.2}, we may assume that if $v$ belongs to an $m$-vertex component of $G - S_{\ell}$ (respectively, $G - S_{r})$, then $f(v)$ belongs to an $m$-vertex component of $H - f(S_{\ell})$ (respectively, $H - f(S_{r})$). Otherwise, Spoiler may win with $O(\log n)$ additional rounds. While \Lem{lem:6.2} prescribes $3$ additional pebbles, we note that there exists a vertex in $\beta(s) \setminus \beta(\ell)$ (respectively, $\beta(s) \setminus \beta(r)$). So we may reuse one such pebble in $\beta(s)$, resulting in only $2$ additional pebbles (for a total of $3k+4$ pebbles).

Without loss of generality, suppose that $G \setminus S_{\ell} \not \cong H \setminus f(S_{\ell})$. Spoiler moves the pebbles in $\beta(s) \setminus \beta(\ell)$ into $\beta(\ell)$. We now iterate on this argument starting from $\ell$ as the root of our subtree in the tree decomposition $(T, \beta)$. As $G \not \cong H$, we will eventually reach a stage (such as when all of $\beta(v)$ is pebbled for some leaf node $v \in V(T)$) where the map induced by the pebbled vertices does not extend to an isomorphism.

It remains to analyze the number of pebbles and rounds. At each level of the tree, we use $3k+2$ rounds to pebble the vertices in a given bag. We may use $2$ additional pebbles and $O(\log n)$ rounds at a given level as prescribed by \Cor{cor:5.3} or \Lem{lem:6.2}. However, invoking \Cor{cor:5.3} or \Lem{lem:6.2} results in Spoiler winning. As $T$ has height $O(\log n)$ and $k$ is bounded, this results in $O(\log n)$ rounds, as desired. The result now follows.
\end{proof}


\section{Conclusion}

We showed that the $(3k+4)$-WL identifies graphs of treewidth $k$ in $O(\log n)$ rounds, improving upon the work of Grohe \& Verbitsky~\cite{GroheVerbitsky}, who established the analogous result for $(4k+3)$-WL. As a corollary, we obtained that graphs of treewidth $k$ are identified by $\textsf{FO} + \textsf{C}$ formulas with $(3k+5)$-variables and quantifier depth $O(\log n)$. We contrast this with the work of Kiefer \& Neuen \cite[Theorem~6.4]{KieferNeuenDecompose}, who showed that $k$-WL identifies graphs of treewidth $k$, though they did not control for rounds. Naturally, it would be of interest to close the gap between our upper bound of $(3k+4)$ on the Weisfeiler--Leman dimension required to achieve $O(\log n)$ rounds and the best known upper bound of $k$ on the Weisfeiler--Leman dimension (without controlling for rounds) of graphs of treewidth $k$. One approach would be to improve the result on~\cite{Bodlaender}, to provide a tree decomposition of width $\leq 3k+2$ and height $O(\log n)$ for graphs of treewidth $k$. It would also be of interest to examine special families of graphs with bounded treewidth that possess additional structure, which Weisfeiler--Leman can exploit to achieve $O(\log n)$ rounds with only $k+O(1)$ pebbles.

Kiefer \& Neuen \cite[Theorem~6.1]{KieferNeuenDecompose} also established a lower bound of $\lceil k/2 \rceil - 2$ on the Weisfeiler--Leman dimension (again, without controlling for rounds) of graphs of treewidth $k$. It would be of interest to strengthen their lower bound on the Weisfeiler--Leman dimension when restricting to $O(\log n)$ rounds.

\bibliographystyle{alphaurl}
\bibliography{references}




\end{document}
