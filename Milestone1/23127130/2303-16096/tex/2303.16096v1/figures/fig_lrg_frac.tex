\begin{figure*}
\centering
  \setlength{\tabcolsep}{0pt}
    \begin{tabular}{ c c }
      {\large $z$-band model} & {\large $W1$-band model} \\
      \includegraphics[width=0.49\linewidth, trim=0 2.578cm 0 0, clip]{figures/lrgfrac_Mz_r-z_z0p40-0p50_south.png} &
      \includegraphics[width=0.49\linewidth, trim=0 2.578cm 0 0.2mm, clip]{figures/lrgfrac_MW1_r-W1_z0p40-0p50_south.png} \\
      \includegraphics[width=0.49\linewidth, trim=0 2.578cm 0 0, clip]{figures/lrgfrac_Mz_r-z_z0p50-0p60_south.png} &
      \includegraphics[width=0.49\linewidth, trim=0 2.578cm 0 0.2mm, clip]{figures/lrgfrac_MW1_r-W1_z0p50-0p60_south.png} \\
      \includegraphics[width=0.49\linewidth]{figures/lrgfrac_Mz_r-z_z0p60-0p70_south.png} &
      \includegraphics[width=0.49\linewidth]{figures/lrgfrac_MW1_r-W1_z0p60-0p70_south.png}
    \end{tabular}
\caption{
LRG fractions in color--magnitude space used to select optical and IR LRG samples from our magnitude-limited mock galaxy catalogs.
The first and second columns show optical and IR LRG targets, respectively, in ${r-z}$ color versus $M_z$ magnitude space.
The third and fourth columns show optical and IR LRG targets, respectively, in ${r-W1}$ color versus $M_{W1}$ magnitude space.
Each row shows a different redshift bin.
Note that the distributions of optical LRGs in the $z$-band model (first column) and IR LRGs in the $W1$-band model (fourth column) are compact by design, i.e., the fraction of galaxies that are LRGs is either $\sim0$ or $\sim1$ nearly everywhere in color--magnitude space. In contrast, the distributions of IR LRGs in the $z$-band model (second column) and optical LRGs in the $W1$-band model (third column) have a broader gradient between 0 and 1, especially toward fainter magnitudes.
The boxed region in each panel marks the luminous end of the red sequence, where essentially all galaxies are expected to be LRGs. The LRG fraction in this boxed region is at least 98\% for the $W1$-band model (third and fourth columns), but only {92--94\%} for the $z$-band model (first and second columns).}
\label{fig:lrg_frac}
\end{figure*}
