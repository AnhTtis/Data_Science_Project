We have used a SHAM and age distribution matching framework to construct magnitude-limited mock galaxy catalogs at $z\sim0.43$, 0.52, and 0.63 with ${r-z}$ and ${r-W1}$ colors. From these catalogs we select mock LRG samples according to both the optical and IR DESI LRG target selection functions.
This work is the first application SHAM modeling in the infrared, complimenting the few existing studies of the galaxy--halo connection of ``DESI-like" LRGs.

Our models reproduce the number densities, luminosity functions, color distributions, and magnitude-dependent projected clustering (in the luminosity range of DESI LRG targets) of the parent galaxy samples from the Legacy Surveys DR9 photometry that serve as the basis for DESI LRG target selection. With the mock LRG samples selected from our magnitude-limited mock galaxy catalogs, we predict the halo occupation statistics of both optical and IR DESI LRGs at a fixed cosmology. We assess the differences between these two LRG populations, as well as the effect of using the $z$-band versus the 3.4 micron $W1$-band for SHAM and age distribution matching.

The main results of this work are:

\begin{enumerate}
  \item Both the optical and IR DESI LRG target selections exclude some of the most luminous galaxies that would appear to be LRGs based on their position on the red sequence in optical color--magnitude space (Figure~\ref{fig:lrg_frac}). This is a result of the specific DESI LRG target selection cut intended to exclude blue, low-redshift (${z \lesssim 0.4}$) galaxies (Figure~\ref{fig:not-lrg_outliers}, \S\ref{subsec:mock_lrg_select}).

  \item Optical and IR LRGs occupy similar regions of optical color--magnitude space (${r-z}$ versus $M_z$), where the red sequence corresponds to the very reddest ${r-z}$ colors (Figure~\ref{fig:cmd}). In IR color--magnitude space (${r-W1}$ versus $M_{W1}$) there is a sizable galaxy population at significantly redder colors than the red sequence. IR LRGs occupy most of this region of excess ${r-W1}$ color, but optical LRGs are largely excluded. This could be a result of AGN activity artificially inflating the $W1$-band luminosities for some of these objects (\S\ref{subsec:clust_lrg}).
  
  \item There are clear distinctions between the LRG samples obtained from the optical versus IR selections that are apparent from the data alone, namely among their optical ($r-z$ versus $M_z$) and IR ($r-W1$ versus $M_{W1}$) color--magnitude distributions, clustering, and color--color distributions. Our IR-based ($W1$-band) model predicts greater differences between the halo occupation statistics of optical and IR LRGs than the $z$-band model (Figure~\ref{fig:hod_lrg}, Table~\ref{tab:hod_stats}), and is therefore the preferred regime for this comparative study of the two selections.

  \item Age distribution matching, which assumes a monotonic correlation between halo age and galaxy color at fixed luminosity, tends to overpredict the clustering amplitude of DESI LRGs (Figure~\ref{fig:wp_lrg}). Introducing scatter into the assumed age--color relation improves agreement between predictions and the data somewhat, although the greatest improvement comes from increasing this scatter to a level that is equivalent to assigning colors at random. Our models therefore suggest that either galaxy color is uncorrelated with halo age in the LRG regime, or there is some additional model parameter that has been neglected and is possibly related to our use of photometric redshifts.

 \item Both DESI LRG target selections yield populations with a non-trivial LRG--halo connection that does not reach unity ($\langle N_{\rm cen} \rangle \sim 1$) for the most massive halos (Figure~\ref{fig:hod_lrg}, \S\ref{subsec:lrg-halo}). However, the IR selection achieves greater completeness ($\langle N_{\rm cen} \rangle \geq 91\%$) than the optical selection ($\langle N_{\rm cen} \rangle \sim 85$--$92\%$) across all redshift bins studied. Our results of $\langle N_{\rm cen} \rangle < 1$ is qualitatively consistent with the SAM predictions of \citet{hernandez-aguayo_etal21} (Figure~\ref{fig:hod_compare}, \S\ref{subsec:lit_compare}), although they find a lower maximum completeness of $\sim70\%$ at $z\sim0.64$.
\end{enumerate}

A natural extension of this work would be to utilize spectroscopic redshifts from the DESI BGS sample \citep{hahn_etal22} to conduct a similar study at lower redshift. The BGS\footnote{Specifically the BGS Bright sample; BGS also contains BGS Faint and AGN samples with different selection criteria.} is nearing completion and is obtaining spectroscopic redshifts for a magnitude-limited ($r<19.5$) sample of over 800 galaxies per \degsq at $z < 0.4$.
While the LRG target selection algorithms used in this work are designed to identify LRGs at $0.4 \lesssim z \lesssim 1.0$, these cuts could easily be modified to select LRGs in the same redshift range as the BGS. Mock LRGs could then be selected from SHAM-based magnitude-limited mock galaxy catalogs tuned to recreate the magnitude and color distributions of the BGS sample.
