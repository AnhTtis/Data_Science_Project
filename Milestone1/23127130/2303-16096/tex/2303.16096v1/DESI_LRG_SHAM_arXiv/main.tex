\documentclass[twocolumn,apj,iop,tighten]{emulateapj2}

\usepackage{float}
\usepackage{graphicx}
\usepackage{xfrac}
\usepackage{adjustbox}
\usepackage{upgreek}

% \usepackage{lineno}
% \linenumbers

%%%%% GENERAL MATH COMMANDS
% Reals
\newcommand{\R}{{\mathbb R}}
% Integers
\newcommand{\Z}{{\mathbb Z}}
% Naturals
\newcommand{\N}{{\mathbb N}}
% Expectation
\DeclareMathOperator*{\E}{\mathbb{E}}
% ^th notation
\newcommand{\tth}{^{\text{th}}}
% Small dots for integer range [a .. b]
\newcommand{\sdots}{\,..\,}
% Vectorized version of matrix
\newcommand{\matvec}{\mbox{vec}}

% := sign
\newcommand{\defeq}{\vcentcolon=}
% Zero function
\newcommand{\zf}{\mathbf{0}}
% Vector of ones
\newcommand{\ones}{\mathbf{1}}

% Argmin and argmax definitions
\DeclareMathOperator*{\argmax}{arg\,max}
\DeclareMathOperator*{\argmin}{arg\,min}


%%%%% PROBLEM STATEMENT NOTATION 
% \newcommandtwoopt{\St}[2][t][]{{S_{#1}^{#2}}} % State
\newcommand{\task}[1][i]{{\mathcal{T}_{#1}}} % Task, optionally takes index
\newcommand{\tasks}{\{ \task \}_{i=1}^N}
\newcommand{\losst}[1][i]{{l_{#1}}}
\newcommand{\lossv}[1][i]{{l_{#1}^{\textrm{val}}}}
\newcommand{\tasktarget}{{\mathcal{T}_{\textrm{target}}}}
\newcommand{\lossttarget}{l_{\textrm{target}}}
\newcommand{\lossvtarget}{l_{\textrm{target}}^{\textrm{val}}}
\newcommand{\lossttargetit}{l_{\textrm{target}}^{(k)}}
\newcommand{\losstotal}{l^{\textrm{total}}}
\newcommand{\lossopt}{l^*}

\newcommand{\thetait}[2]{\theta_{#1}^{(#2)}}
\newcommand{\phit}[1]{\phi^{(#1)}}
\newcommand{\hist}[2]{S_{#1}^{(#2)}}
\newcommand{\grad}[2]{G_{#1}^{(#2)}}

\newcommand{\Alg}{\textup{\textbf{Opt}}}
\newcommand{\MetaAlg}{\textup{\textbf{MetaOpt}}}

%%%%% Theorems
\newtheoremstyle{mytheoremstyle} % name
    {\topsep}                    % Space above
    {\topsep}                    % Space below
    {\itshape}                   % Body font
    {}                           % Indent amount
    {\scshape}                   % Theorem head font
    {.}                          % Punctuation after theorem head
    {.5em}                       % Space after theorem head
    {}  % Theorem head spec (can be left empty, meaning ‘normal’)
\theoremstyle{mytheoremstyle}
\theoremstyle{plain}
\newtheorem{theorem}{Theorem}
\newtheorem{proposition}{Proposition}
\newtheorem{assumption}{Assumption}
\newtheorem{definition}{Definition}
\newtheorem{lemma}{Lemma}
\theoremstyle{remark}
\newtheorem{remark}{Remark}


\begin{document}
\title{The galaxy--halo connection of DESI luminous red galaxies with subhalo abundance matching}
\shorttitle{DESI LRG--halo connection with SHAM}
\shortauthors{Berti et al.}

\author{
Angela M.\ Berti\altaffilmark{1},
Kyle S.\ Dawson\altaffilmark{1},
Wilber Dominguez\altaffilmark{2}
}
	
\altaffiltext{1}{Department of Physics \& Astronomy, University of Utah, Salt Lake City, UT 84112, USA}
\altaffiltext{2}{Department of Physics \& Astronomy, Swarthmore College, 500 College Ave, Swarthmore, PA 19081, USA}

\begin{abstract}
We use subhalo abundance and age distribution matching to create magnitude-limited mock galaxy catalogs at $z\sim0.43$, 0.52, and 0.63 with $z$-band and 3.4 micron $W1$-band absolute magnitudes and ${r-z}$ and ${r-W1}$ colors. From these magnitude-limited mocks we select mock luminous red galaxy (LRG) samples according to the $(r-z)$-based (optical) and $(r-W1)$-based (infrared) selection criteria for the LRG sample of the Dark Energy Spectroscopic Instrument (DESI) Survey.
Our models reproduce the number densities, luminosity functions, color distributions, and projected clustering of the DESI Legacy Surveys that are the basis for DESI LRG target selection. We predict the halo occupation statistics of both optical and IR DESI LRGs at fixed cosmology, and assess the differences between the two LRG samples. We find that IR-based SHAM modeling represents the differences between the optical and IR LRG populations better than using the $z$-band, and that age distribution matching overpredicts the clustering of LRGs, implying that galaxy color is uncorrelated with halo age in the LRG regime.
Both the optical and IR DESI LRG target selections exclude some of the most luminous galaxies that would appear to be LRGs based on their position on the red sequence in optical color--magnitude space. Both selections also yield populations with a non-trivial LRG--halo connection that does not reach unity for the most massive halos. We find the IR selection achieves greater completeness ($\gtrsim 90\%$) than the optical selection across all redshift bins studied.
\end{abstract}

\section{Introduction}\label{sec:intro}
\section{Introduction}

The increasing complexity of source code poses a key challenge to the reliability of large-scale software systems. Software bugs in these systems can lead to safety issues~\cite{bug_safety} for users around the world as well as cause non-negligible financial losses~\cite{bug_loss}. As such, developers have to spend a large amount of time and effort on bug fixing. Consequently, \aprfull (\apr), designed to automatically generate patches to fix software bugs, has attracted wide attention from both academia and industry~\cite{long2016prophet, legoues2012genprog, long2015spr, lou2020can, tufano2018empstudy}. 


To achieve \apr, one popular approach is known as Generate-and-Validate (G\&V)~\cite{qi2015gv, ghanbari2019prapr, lou2020can, le2016hdrepair, legoues2012genprog, wen2018capgen, hua2018sketchfix, martinez2016astor, koyuncu2020fixminder, liu2019tbar, liu2019avatar}, which is typically based on the following pipeline: First, fault localization techniques~\cite{wong2016fl, abreu2007ochiai, zhang2013injecting, papadakis2015metallaxis, li2019deepfl, li2017transforming} are applied to determine the suspicious locations in programs where bugs are likely to exist. Then, the buggy locations are used by the \apr tools to generate a list of patches that replace buggy lines with correct lines. Afterward, each patch is validated against the original test suite to identify any \emph{plausible patches} (i.e., passing all tests in the test suite). Finally, to determine the \emph{correct patches}, developers examine the list of plausible patches to see if any of them can correctly fix the bug. 

Traditional \apr tools can mainly be categorized into heuristic-based~\cite{legoues2012genprog, le2016hdrepair, wen2018capgen}, constraint-based~\cite{mechtaev2016angelix, le2017s3, demacro2014nopol, long2015spr} and \template~\cite{ghanbari2019prapr, hua2018sketchfix, martinez2016astor, liu2019tbar, liu2019avatar}. Among these traditional tools, \template \apr tools~\cite{ghanbari2019prapr, liu2019tbar, benton2020effectiveness} have been able to achieve state-of-the-art results. \Template \apr tools typically leverage pre-defined templates (e.g., adding a nullness check) for bug fixing. However, since these fix templates are typically handcrafted, the number and types of bugs they are able to fix can be limited. 



To address the limitations of traditional \apr, researchers have proposed various \learning \apr tools~\cite{li2020dlfix, chen2018sequencer, jiang2021cure, lutellier2020coconut, zhu2021recoder, ye2022rewardrepair} based on the \nmtfull (\nmt) architecture~\cite{sutskever2014mt} where the input is the buggy code snippets and the goal is to translate the buggy code snippets into a fixed version. To accomplish this, \learning \apr tools require supervised training datasets with pairs of both buggy and fixed code snippets in order to learn how to perform this translation step. These training data are usually obtained by mining historical bug fixes using heuristics/keywords~\cite{dallmeier2007benchmark}, which can be imprecise for identifying bug-fixing commits; even the actual bug-fixing commits can include irrelevant code changes, leading to further pollution in the dataset~\cite{xia2022alpharepair}.
% 
Moreover, it can be hard for such \apr tools to generalize and fix bug types unseen during training. 



To better leverage recent advances in \plmfull{s} (\plm{s}), researchers~\cite{xia2022alpharepair, xia2023repairstudy, kolak2022patch, prenner2021codexws} have directly applied \plm{s} to generate patches without bug-fixing datasets. These \llm-based \apr tools work by either directly generating a complete code function~\cite{prenner2021codexws, xia2023repairstudy} or predict/infill the correct code snippet given its surrounding context~\cite{xia2022alpharepair, xia2023repairstudy}. By directly using \llm{s} that are pre-trained on billions of open-source code snippets, \llm-based \apr tools can achieve state-of-the-art performance on many repair datasets~\cite{xia2022alpharepair}. 


% 
%
%

Traditional \apr tools have long used the insight of the \emph{plastic surgery hypothesis}~\cite{barr2014plastic} where it states that the code ingredients to fix a bug already exist within the same project. Traditional \apr tools have manually designed pattern-~\cite{ghanbari2019prapr, saha2017elixir} or heuristic-based~\cite{jiang2018simfix, legoues2012genprog} approaches to finding and using such relevant code ingredients to generate fixes for bugs. However, the plastic surgery hypothesis has been largely ignored in \llm-based \apr. In fact, \llm provides a unique opportunity to fully automate the plastic surgery hypothesis idea via fine-tuning (learning project-specific information via model updates from the buggy project) and prompting (directly providing relevant code ingredients to the model), and make it directly applicable to different languages (since the \llm{s} are typically multi-lingual).%
Moreover, despite the intensive manual efforts involved, traditional \apr tools still cannot fully leverage project-specific information due to large search space for leveraging/composing existing code ingredients. In contrast, the project-specific information can effectively leveraged by \llm{s} due to their power in code understanding/vectorization, e.g., even partial/imprecise information may still guide \llm{s} in correct patch generation!
 To this end, we ask the question: \emph{How useful is the plastic surgery hypothesis in the era of \plm{s}}?








\mypara{Our Work.} To answer the question, we present \ourtech{\xspace} -- a \llm-based approach that automatically utilizes the plastic surgery hypothesis by systematically combining multiple fine-tuning and prompting strategies for \apr. \ourtech fine-tunes \plm{s} using two novel domain-specific training strategies: \textbf{\epfinetune} -- we fine-tune using the original buggy project by aggressively masking out a high percentage of tokens, which allows \plm to learn project-specific code tokens and programming styles; and \textbf{\rofinetune} -- which only masks out a single continuous code sequence per training sample, allowing the model to get used to the final \csapr task of predicting a single continuous code sequence. Furthermore, we directly leverage the ability for \plm{s} to understand natural language instructions and introduce a novel prompting strategy, \textbf{\idprompting}, which uses information retrieval and static analysis to obtain a list of relevant identifiers for the buggy lines. While such relevant identifiers are critical for fixing some difficult bugs, they may not be seen by the \llm during inference due to limited context window size. Through the use of prompting, we directly tell the model to use these extracted identifiers (relevant code ingredients) to generate the correct code. Finally, to perform repair, we combine all four model variants (including the base model, both fine-tuned models and the base model with prompting) for the final repair.





While our insight of leveraging the plastic surgery hypothesis for \llm-based \apr is generalizable across different types of \plm{s}, to implement \ourtech, we choose a recent \plm{\xspace}, \ctfive~\cite{wang2021codet5}, which is pre-trained on millions of open-source code snippets. \ctfive is an encoder-decoder model trained using \mspfull (\msp) objective where a percentage of tokens are masked out and each continuous masked token sequence is referred to as a masked span. Also, although we only extract relevant identifiers from the current buggy project (since this paper focuses on the plastic surgery hypothesis), our work can be easily extended to obtain other code information (such as relevant statements or functions) from other sources, such as  the massive pre-training corpora~\cite{husain2020codesearchnet} or historical bug-fixing datasets~\cite{jiang2019infer}, which can provide more coding knowledge for \llm{s}. Besides, although we mainly focus on using traditional string comparison algorithms for information retrieval in this paper, these techniques can be easily replaced by other frequency-based retrieval~\cite{robertson2009probabilistic} and neural search (or embedding-based search)~\cite{reimers2019sentence}.
  In summary, this paper makes the following contributions:


%


\begin{itemize}[noitemsep, leftmargin=*, topsep=0pt]
    \item \textbf{Dimension.} This paper is the first to revisit the important plastic surgery hypothesis in the era of \llm{s}. It opens up a new dimension for \llm-based \apr to incorporate previously neglected information from the buggy project itself to boost \apr performance. Furthermore, it demonstrates the promising future of retrieval-based prompting for modern \llm-based \apr.
    \item \textbf{Implementation.} We implement \ourtech based on the recent \ctfive model. We augment the model using two novel fine-tuning strategies: \epfinetune and \rofinetune, along with a novel prompting strategy based on information retrieval and static analysis: \idprompting. We combine the patches generated by all four models together and perform patch ranking to speed up \apr.% 
    \item \textbf{Evaluation Study.} We conduct an extensive evaluation against state-of-the-art \apr tools. On the widely studied \dfj 1.2 and 2.0 datasets~\cite{just2014dfj}, \ourtech is able to achieve the new state-of-the-art results of 89 and 44 correct bug fixes (15 and 8 more than best baseline) respectively.  Furthermore, we perform a broad ablation study to justify our design. \ourtech demonstrates for the first time that the plastic surgery hypothesis can substantially boost \llm-based \apr and advance state-of-the-art \apr, while being fully automated and general. Moreover, even partial/imprecise code ingredients may still effectively guide \llm{s} for \apr!
\end{itemize}



\section{Simulations and data}\label{sec:data_sims}
In this section we describe the cosmological simulation, halo finder, and associated halo properties used for our models. We also present the data from which we select parent galaxy samples for training our models, as well as the DESI LRG target selection functions.


\subsection{Simulations}\label{subsec:sims}

We use halo catalogs and merger histories obtained with the publicly available \texttt{ROCKSTAR} phase-space temporal halo finder \citep{behroozi_etal13b} for the MultiDark Planck 2 (MDPL2) simulation\footnote{www.cosmosim.org} \citep{klypin_etal16}.
MDPL2 assumes Planck cosmology \citep[$h=0.6777$, $\Omega_{\rm m}=0.307115$;][]{planck_collab16} and evolves $3840^3$ dark matter particles in a 1~\Gpch cubic volume, beginning at $z=120$.
The particle resolution is $1.51\times10^9\ h^{-1}\ \msun$. In total 126 snapshots are available between $z\sim15$ and $z=0$. For this work we use three snapshots at $z=0.425$, 0.523, and 0.628.

The \texttt{ROCKSTAR} halo finder is designed to preserve particle--halo membership and identify accurate halo merger trees across multiple time steps of a simulation.
For MDPL2, \texttt{ROCKSTAR} halo catalogs are mass-complete for halos (including subhalos) above ${\mvir\gtrsim11.4\times10^{11}\ h^{-1}\ \msun}$.

\subsubsection{Subhalo abundance matching}\label{subsubsec:sham}

In SHAM modeling, mock galaxies are assigned to dark matter halos by exploiting the correlation between some galaxy property---usually stellar mass or luminosity---and a halo property such as virial mass or circular velocity. Circular velocity is defined as ${\vcirc(r,z) \equiv \sqrt{GM(<r,z)/r}}$, where $M(<r,z)$ is the enclosed mass within radius $r$ at redshift $z$.
SHAM has been tested with several versions of halo circular velocity. In this work we use \vpeak, the peak value of maximum circular velocity (${\vmax(z) = {\rm max}\{v_{\rm circ}(r,z)\}}$) achieved throughout a halo's entire assembly history.


\subsubsection{Age distribution matching}\label{subsubsec:adm}

\begin{figure}
\centering
\includegraphics[width=\linewidth]{figures/halo_history_zstarve_example_zsnap0p42531.png}
\caption{
Mass accretion histories and \zstarve values of four randomly selected halos from the $\zsim=0.425$ snapshot of the MDPL2 simulation. The most massive halo (purple dash-dotted line) has the earliest starvation redshift ($\zstarve=4.63$), corresponding to when its mass first reaches the characteristic value ${\mchar=10^{12}\ \msunh}$. The other three halos also reach \mchar (at later redshifts), but in accordance with Eq.~\ref{eq:zstarve} the redshift at which this occurs for each halo is not necessarily the same as its \zstarve value, e.g., the least massive halo (orange solid line) has $\zstarve=1.36$ although its mass doesn't exceed \mchar until $z\lesssim1$.
}
\label{fig:zstarve_example}
\end{figure}


Age distribution matching assumes a correlation at fixed luminosity between galaxy color and some proxy (at fixed model luminosity) for the age of the halo in which each mock galaxy resides. Redder colors (i.e., older, quenched galaxies) are generally assigned to older halos.
Model colors are assigned at fixed luminosity (in practice in narrow luminosity bins) because the galaxy color distribution is highly dependent on luminosity.

We equate the cumulative distribution $\mathcal{D}_{\rm gal}$ of galaxies of $K$-corrected color $\mathcal{C}$ at fixed absolute magnitude $M_X$ to the cumulative distribution $\mathcal{D}_{\rm halo}$ of halo age proxy $A$ at fixed model absolute magnitude:
%
\begin{equation}\label{eq:color_assign}
\mathcal{D}_{\rm gal} ( < \mathcal{C}\ |\ M_X )=\mathcal{D}_{\rm halo}( < A\ |\ {\rm model}\ M_X ).
\end{equation}

\noindent In Equation~\ref{eq:color_assign} $(\mathcal{C},M_X) = {(r-z, M_z)}$ OR ${(r-W1, M_{W1})}$ for this work.

Implementations of age distribution matching at $z\sim0$ using spectroscopic galaxy redshifts from SDSS have used halo starvation redshift, \zstarve, for the halo age proxy $A$ in Eq.\ \ref{eq:color_assign} \citep{hearin_watson13, hearin_etal14, safonova_etal21}. In general, \zstarve represents the redshift at which a galaxy loses its supply of cold gas, which leads to the quenching of star formation and the reddening of the galaxy. Multiple physical processes relevant to a halo's assembly history can affect the value of \zstarve for a given halo, which \citet{hearin_watson13} incorporate into the following definition:
%
\begin{equation}\label{eq:zstarve}
\zstarve \equiv \max\{\zchar,\, \zacc,\, \zform\}.
\end{equation}

\noindent In Equation~\ref{eq:zstarve}:
%
\begin{itemize}
\item \zchar is either the redshift at which a halo's mass first exceeds some characteristic value, \mchar, or the redshift of the relevant simulation snapshot (\zsim) for halos that never achieve \mchar;
%
\item \zacc is the redshift at which a subhalo accretes onto a parent halo (for host halos ${\zacc=\zsim}$). \citet{hearin_watson13} follow \citet{behroozi_etal13c}, which defines \zacc as the snapshot after which a subhalo always remains a subhalo. They note that alternative definitions, such as that of \citet{wetzel_etal14}, where \zacc is the snapshot at which a subhalo has been identified as such for two consecutive snapshots, have little impact on their results.
%
\item \zform is the ``formation" redshift at which a halo transitions from the fast to slow accretion regime.
\end{itemize}
\noindent We use same definition of \zform as \citet{hearin_watson13}, motivated by \citet{wechsler_etal02}:
%
\begin{equation}\label{eq:zform}
\zform \equiv \frac{c_{\rm vir}}{4.1 a_0} - 1,
\end{equation}

\noindent where ${c_{\rm vir} = R_{\rm vir}/R_{\rm s}}$ is a halo's concentration at the time of observation, indicated by $a_0$. For host halos $a_0$ is the scale factor of the relevant simulation snapshot, while for subhalos $a_0$ is the scale factor at the time of accretion: ${\zacc = 1/{a_0} - 1}$. $R_{\rm vir}$ is the virial radius of a halo, and $R_{\rm s}$ is the NFW scale radius \citep{navarro_etal97}.

We adopt the value of ${\mchar=10^{12}\ \msunh}$ used in \citet{hearin_watson13}, who note that their results are insensitive to the precise value of \mchar used. The empirical and physical motivation for \mchar are described in detail in \S6.3 of \citet{hearin_watson13}. Briefly, there is empirical support for a characteristic halo mass above which star formation is highly inefficient: ${\sim10^{12}\ \msunh}$ is the halo mass at which the SHMR peaks, falling off rapidly at higher halo masses \citep{behroozi_etal13a, yang_etal12, yang_etal13, moster_etal13, watson_conroy13}, and \citet{behroozi_etal13d} have shown that this mass remains essentially constant throughout much of cosmic history.

We compute \zstarve for all halos in our model from the publicly available \texttt{ROCKSTAR} halo merger trees for MDPL2. Figure~\ref{fig:zstarve_example} shows sample halo mass accretion histories and corresponding \zstarve values for four randomly selected halos from the $\zsim=0.425$ snapshot of MDPL2.


\subsection{Photometry and redshift estimates}\label{subsec:photometry}

We use publicly available catalogs from the ninth data release (DR9) of the DESI Legacy Imaging Surveys\footnote{www.legacysurvey.org/dr9} \citep{dey_etal19}. The Legacy Surveys provide optical imaging in the $g$, $r$, and $z$ bands from a combination of three public surveys:\ the DECam Legacy Survey \citep[DECaLS;][]{flaugher_etal15, blum_etal16},
the Beijing-Arizona Sky Survey \citep[BASS;][]{zou_etal17b},
and the Mayall $z$-band Legacy Survey \citep[MzLS;][]{silva_etal16}.
The Legacy Surveys also include four mid-infrared bands from the Wide-field Infrared Survey Explorer \citep[WISE;][]{wright_etal10},
although only the 3.4 micron $W1$-band is relevant for DESI LRG target selection.

In total the Legacy Surveys cover $14,000$ square degrees visible from the northern hemisphere, comprised of two contiguous regions within the northern and southern galactic caps. To avoid effects from systematic differences among data from the three component optical surveys, we limit our study to the approximately 9,000 square degrees covered by DECaLS.

\citet{zhou_etal20b} compute photometric redshifts for the full catalog of DECaLS DR7 objects using the random forest regression machine learning algorithm in \texttt{Scikit-Learn} \citep{pedregosa_etal11} and a ``truth" dataset of spectroscopic and many-band photometric redshifts for objects within DR7.
They quantify the accuracy of their photometric redshifts with the normalized median absolute deviation \citep[NMAD;][]{dahlen_etal13}:\ $\sigma_{\rm NMAD}=1.48 \times {\rm median}( | \Delta z | /(1+z_{\rm spec}))$, where $\Delta z = \zphot - z_{\rm spec}$ and $z_{\rm spec}$ are the redshift truth values used to train the random forest algorithm, and report $\sigma_{\rm NMAD}=0.021$ for LRGs.
Their outlier rate for LRGs is 1\%, where outliers are objects with $| \Delta z | > 0.1\times(1+z_{\rm spec})$.
Additionally, \citet{zhou_etal20b} estimate their redshifts are accurate for objects with apparent $z$-band magnitude $z < 21$, well beyond the $z<20.7$ cut we use to select our target galaxy samples, described in \S\ref{subsec:parent_samples} below.


\subsection{DESI LRG target selection}\label{subsec:lrg_target_select}

The DESI LRG target sample is intended to serve as a cosmological tracer spanning the redshift range $\sim0.4 < z \lesssim 1.0$. The sample lies between the low-redshift Bright Galaxy Survey \citep[BGS;][]{hahn_etal22} tracer sample at $z\lesssim0.4$, and the Emission Line Galaxy \citep[ELG;][]{raichoor_etal22} sample, optimized to trace the density field over the approximate range ${1.0 < z < 1.6}$.

Two different selection algorithms were considered for the DESI LRG sample:\ an optical selection function based on $z$-band magnitude and ${r-z}$ color, and an infrared selection function based on $W1$-band magnitude and ${r-W1}$ color. Both selections are tuned to yield a constant LRG target density of $\sim600$ objects per square degree and a comoving number density around ${5\times10^{-4}\ h^3\ {\rm Mpc}^{-3}}$ at ${0.4 < z < 0.8}$. Both the optical and IR selections were tested in DESI's Survey Validation (SV; DESI et al., in prep.) observations. Based largely on the calibration of $W1$-band imaging, the DESI Main Survey uses exclusively the IR selection. A complete description of DESI LRG target selection is given in \citet{zhou_etal22}. Here we cover the details most relevant for this work.

Due to slight differences in photometry among BASS, MzLS, and DECaLS, the optical and IR DESI LRG target selections use slightly different cuts for the north (BASS and MzLS) and south (DECaLS) galactic caps. As this work uses $g$, $r$, and $z$ magnitudes from DECaLS, we use the corresponding optical LRG target selection cuts \citep{zhou_etal20a}:
%
\begin{subequations}\label{eq:lrg_opt}
  \begin{align}
   % stellar rejection cut               
   & z - W1 > 0.8\times(r - z) - 0.6\label{eq:lrg_opt_stellar}, \\
   \begin{split}
   % eliminate low-redshift or bluer objects
   & ( (g - W1 > 2.6)\ {\rm AND}\ (g - r > 1.4) )\ {\rm OR} \\
   & \phantom{00} (r - W1 > 1.8)\label{eq:lrg_opt_lowz},
   \end{split} \\
   \begin{split}
   % color-dependent magnitude limit selects only most luminous objects at given redshift
   & ( r - z > 0.45\times(z - 16.83) )\ {\rm AND}\ (r - z > 0.7) \\
   & \phantom{00}{\rm AND}\ (r - z > 0.19\times(z - 13.80) )\label{eq:lrg_opt_color-mag},
   \end{split} \\
   %& r - z > 0.7, \\
   % ensure targets yield secure spec-z measurements
   & z_{\rm fiber} < 21.5\label{eq:lrg_opt_fiber}.
   \end{align}
\end{subequations}
%
\noindent The relevant IR LRG target selection cuts for this work are \citep{zhou_etal22}:
%
\begin{subequations}\label{eq:lrg_ir}
   % stellar rejection cut
  \begin{align}
   & z - W1 > 0.8 \times (r - z) - 0.6\label{eq:lrg_ir_stellar}, \\
   % eliminate low-redshift or bluer objects
   & ( g - W1 > 2.9)\ {\rm OR}\ (r - W1 > 1.8)\label{eq:lrg_ir_lowz}, \\
   \begin{split}
   % color-dependent magnitude limit selects only most luminous objects at given redshift
   & ( ( r - W1 > 1.8 \times (W1 - 17.14) )\ {\rm AND} \\
   & \phantom{00} ( r - W1 > W1 - 16.33 ) )\ {\rm OR}\ ( r - W1 > 3.3 )\label{eq:lrg_ir_color-mag},
  \end{split} \\
   % ensure targets yield secure spec-z measurements
   & z_{\rm fiber} < 21.6\label{eq:lrg_ir_fiber}.
  \end{align}
\end{subequations}

Equations~\ref{eq:lrg_opt_stellar} and \ref{eq:lrg_ir_stellar} are designed to reject stars, Eqs.\ \ref{eq:lrg_opt_lowz} and \ref{eq:lrg_ir_lowz} remove blue and low-redshift objects, Eqs.\ \ref{eq:lrg_opt_color-mag} and \ref{eq:lrg_ir_color-mag} are color-dependent magnitude limits that select only the most luminous objects at a given redshift, and $z_{\rm fiber}$ in Eqs.\ \ref{eq:lrg_opt_fiber} and \ref{eq:lrg_ir_fiber} is the expected $z$-band flux within a DESI fiber. All magnitudes in Eqs.\ \ref{eq:lrg_opt} and \ref{eq:lrg_ir} use the AB system, and are corrected for galactic extinction using the relevant \texttt{MW\_TRANSMISSION} values from the Legacy Surveys DR9.


\subsection{Parent galaxy samples}\label{subsec:parent_samples}

\begin{figure}
\centering
\includegraphics[width=\linewidth]{figures/abs_mag_cuts.png}
\caption{Distribution of $K$-corrected $z$-band (left column) and $W1$-band (right column) absolute magnitudes for the three redshift bins in our study:\ ${0.4 < \zphot < 0.5}$ (top), ${0.5 < \zphot < 0.6}$ (middle), and ${0.6 < \zphot < 0.7}$ (bottom). Each panel shows the distribution of all galaxies with $z<20.7$ (solid gray), as well as the distributions of optical (hatched purple; see Eq.~\ref{eq:lrg_opt}) and IR (solid orange; see Eq.~\ref{eq:lrg_ir}) DESI LRG targets. The leftmost dashed black line in each panel is the absolute magnitude cut that defines each absolute magnitude-limited parent galaxy sample for our models. At fainter magnitudes (to the left of the cut in each panel) the DECaLS sample is incomplete. The three dotted black lines in each panel denote the luminosity bins used to constrain model parameters (see \S\ref{subsec:error} and Tables \ref{tab:parent_samples} and \ref{tab:mag_bins}).
}
\label{fig:abs_mag_cuts}
\end{figure}


\begin{figure}
\centering
  \setlength{\tabcolsep}{0pt}
    \begin{tabular}{ c c }
      \includegraphics[width=0.5\linewidth, trim=0 3cm 0 2mm, clip]{cmd_Mz-rz_z0p40-0p50_optical-lrg.png} &
      \includegraphics[width=0.5\linewidth, trim=0 2.9cm 0 0, clip]{cmd_MW1-rW1_z0p40-0p50_optical-lrg.png} \\
      \includegraphics[width=0.5\linewidth, trim=0 0 0 2mm, clip]{cmd_Mz-rz_z0p40-0p50_IR-lrg.png} &
      \includegraphics[width=0.5\linewidth, trim=0 0 0 0, clip]{cmd_MW1-rW1_z0p40-0p50_IR-lrg_AB.png} \\
    \end{tabular}
  \caption{
  Color--magnitude diagrams of the parent galaxy samples and DESI LRG target samples (overlaid white contours) in the {$0.4 < \zphot < 0.5$} redshift bin. The left column shows optical ($K$-corrected $r-z$ color versus $M_z$ magnitude), while the right column shown IR ($K$-corrected $r-W1$ color versus $M_{W1}$ magnitude).
  The top (bottom) row shows the distribution of optical (IR) LRGs in each color--magnitude space.
  The boxed regions labeled ``A" and ``B" are referenced in \S\ref{subsec:clust_lrg} below.
 }
  \label{fig:cmd}
\end{figure}


\begin{deluxetable}{ r l }
\tablecaption{
Bitmasks applied to photometry for parent galaxy sample selection.
Additional details at legacysurvey.org/dr9/bitmasks.
\label{tab:masks}
}
\tablehead{
\colhead{\texttt{MASKBIT}} & \colhead{Description}
}
\startdata
{5, 6, 7} & {bad pixel in all of a set of overlapping $g$, $r$, or $z$-band } \\
{} & {images} \\
{8, 9} & bad pixel in a WISE $W1$ or $W2$ bright star mask \\
{11} & {pixel within locus of a radius-magnitude relation for} \\
{} & {Gaia\tablenotemark{a} DR2 stars to $G < 16$} \\
{12} & pixel in a Siena Galaxy Atlas\tablenotemark{b} large galaxy \\
{13} & pixel in a globular cluster \\
\hline \\
\vspace{-1ex} \\
\colhead{\texttt{FITBIT}} & \colhead{Description} \\
\vspace{-1ex} \\
\hline \\
\vspace{-1ex} \\
{6} & source is a medium-bright star \\
{7} & Gaia source\tablenotemark{c} \\
{8} & Tycho-2 star\tablenotemark{d} \\
\enddata
\tablenotetext{a}{\citet{gaia_collab18}}
\tablenotetext{b}{\citet{moustakas_etal21}}
\tablenotetext{c}{\citet{gaia_collab16}}
\tablenotetext{d}{\citet{hog_etal00}}
\end{deluxetable}


\begin{deluxetable*}{ r r r r c c c }
\tablecaption{Properties of parent galaxy samples and corresponding simulation snapshot redshifts (\zsim).
\label{tab:parent_samples}
}
\tablehead{
\colhead{\multirow{2}{*}{Redshift bin}} & \colhead{\multirow{2}{*}{\zsim}} & \colhead{\multirow{2}{*}{Luminosity cut\tablenotemark{a}}} & \colhead{\multirow{2}{*}{$N_{\rm gal}$}} & \colhead{\neff} & \multicolumn{2}{c}{Included fraction of DESI LRG targets} \\
\colhead{} & \colhead{} & \colhead{} & \colhead{} & \colhead{$[\times 10^{-3}\ h^3\ {\rm Mpc}^{-3}]$} & \colhead{Optical selection} & \colhead{IR selection}
}
\startdata
\multirow{2}{*}{$0.4 < \zphot < 0.5$} & \multirow{2}{*}{0.425} & $^{0.43}M_z < -21.60$ & 8,314,309 & 5.60 & 0.992 & 0.982 \\
& & $^{0.43}M_{W1} < -22.25$ & 7,565,153 & 4.33 & 0.992 & 0.992 \\
\vspace{-1ex} \\
\hline
\vspace{-1ex} \\
\multirow{2}{*}{$0.5 < \zphot < 0.6$} & \multirow{2}{*}{0.523} & $^{0.52}M_z < -21.60$ & 7,804,346 & 2.86 & 0.992 & 0.988 \\
& & $^{0.52}M_{W1} < -22.85$ & 4,909,857 & 2.01 & 0.992 & 0.992 \\
\vspace{-1ex} \\
\hline
\vspace{-1ex} \\
\multirow{2}{*}{$0.6 < \zphot < 0.7$} & \multirow{2}{*}{0.628} & $^{0.63}M_z < -21.85$ & 6,548,126 & 1.46 & 0.994 & 0.990 \\
& & $^{0.63}M_{W1} < -23.15$ & 4,758,470 & 1.12 & 0.993 & 0.994 \\
\enddata
\tablenotetext{a}{$K$-correction redshifts are rounded to two decimal places for clarity, e.g., $^{0.43}M_z$ indicates absolute $z$-band magnitudes are $K$-corrected to $\zsim=0.425$.}
\end{deluxetable*}


\begin{deluxetable*}{ r r r r r r }
\tablecaption{Luminosity bins used to constrain the magnitude dependence of model parameters.
\label{tab:mag_bins}
}
\tablehead{
\multicolumn{2}{c}{$0.4 < \zphot < 0.5$} & \multicolumn{2}{c}{$0.5 < \zphot < 0.6$} & \multicolumn{2}{c}{$0.6 < \zphot < 0.7$} \\
\vspace{-1ex} \\
\hline
\vspace{-1ex} \\
\colhead{Luminosity bin} & \colhead{$N_{\rm gal}$} & \colhead{Luminosity bin} & \colhead{$N_{\rm gal}$} & \colhead{Luminosity bin} & \colhead{$N_{\rm gal}$} \\
\vspace{-1ex} \\
\hline
\vspace{-1ex} \\
\multicolumn{6}{c}{$z$-band}
}
\startdata
$-21.60 > {^{0.43}M_z} > -21.85$ & 2,458,920 & $-21.60 > {^{0.52}M_z} > -21.85$ & 2,304,853 & $-21.85 > {^{0.63}M_z} > -22.10$ & 2,040,336 \\
$-21.85 > {^{0.43}M_z} > -22.10$ & 1,880,295 & $-21.85 > {^{0.52}M_z} > -22.10$ & 1,732,820 & $-22.10 > {^{0.63}M_z} > -22.35$ & 1,512,940 \\
$-22.10 > {^{0.43}M_z} > -22.35$ & 1,317,269 & $-22.10 > {^{0.52}M_z} > -22.35$ & 1,205,545 & $-22.35 > {^{0.63}M_z} > -22.60$ & 1,039,573 \\
$ {^{0.43}M_z} <-22.35$ & 1,919,594 & $ {^{0.52}M_z} < -22.35$ & 1,855,014 & $ {^{0.63}M_z} < -22.60$ & 1,401,662 \\
\cutinhead{$W1$-band}
$-22.25 > {^{0.43}M_{W1}} > -22.55$ & 2,324,816 & $-22.85 > {^{0.52}M_{W1}} > -23.15$ & 1,771,554 & $-23.15 > {^{0.63}M_{W1}} > -23.45$ & 1,634,860 \\
$-22.55 > {^{0.43}M_{W1}} > -22.85$ & 1,837,859 & $-23.15 > {^{0.52}M_{W1}} > -23.45$ & 1,245,107 & $-23.45 > {^{0.63}M_{W1}} > -23.75$ & 1,273,814 \\
$-22.85 > {^{0.43}M_{W1}} > -23.15$ & 1,277,192 & $-23.45 > {^{0.52}M_{W1}} > -23.75$ & 754,154 & $-23.75 > {^{0.63}M_{W1}} > -24.05$ & 798,012 \\
$ {^{0.43}M_{W1}} <-23.15$ & 1,565,914 & ${^{0.52}M_{W1}} <-23.75$ & 693,491 & $ {^{0.63}M_{W1}} < -24.05$ & 685,799 \\
\enddata 
\end{deluxetable*}


A primary goal of this work is to create mock galaxy catalogs that are both statistically complete and represent a superset of the color--magnitude space occupied by DESI LRG targets. Selection of DESI LRG targets is based entirely on $g$, $r$, $z$, and $W1$ apparent magnitudes (see \S\ref{subsec:lrg_target_select}), so we would ideally create mock catalogs where every mock galaxy has an apparent magnitude in each of these bands. SHAM, however, exploits the correlation between some physical halo property (e.g., circular velocity) and a physical galaxy property independent of redshift (e.g., luminosity). We therefore train our mock catalogs on galaxy samples that are complete to an \emph{absolute} magnitude threshold that includes all DESI LRG targets (in the relevant redshift bin; see below).

To select suitable parent galaxy samples, we first apply an apparent $z$-band magnitude cut of $z < 20.7$, and take an additional step to remove stars by excluding catalog sources with ${\rm TYPE} = \texttt{PSF}$. We also apply the masks described in Table~\ref{tab:masks}, which are provided with DECaLS DR9, to remove sources affected by bad pixels or contamination from bright stars. Finally, a geometric mask is applied to ensure complete angular coverage by the catalogs of random points provided with DR9 (see \S\ref{subsec:wprp}). The resulting sample contains $\mathcal{O}(10^8)$ galaxies, sufficient to divide it into redshift bins and maintain low statistical error.

We initially tested six redshift bins of width $\Delta\zphot=0.1$ between $\zphot=0.4$ and $\zphot=1.0$, but found that DECaLS photometry is only deep enough to apply our model up to $\zphot \sim 0.7$. At $\zphot \gtrsim 0.7$ the data are incomplete above the absolute magnitude threshold that encompasses DESI LRG targets. We therefore limit our study to three redshift bins of $\Delta\zphot=0.1$ within $0.4 < \zphot < 0.7$.

For each redshift bin we compute $z$- and $W1$-band absolute magnitudes using photometric redshifts to obtain distance moduli.
We $K$-correct absolute magnitudes to the redshift of the relevant simulation snapshot, \zsim (see Table~\ref{tab:parent_samples}) with the \idl package \kcorrect \citep{blanton_roweis07}
using DECam $g$, $r$, and $z$ and WISE $W1$ and $W2$ filter responses.
For each redshift bin we use the simulation snapshot closest to the median \zphot of the data, e.g., galaxies in the ${0.4 < \zphot < 0.5}$ bin are $K$-corrected to $\zsim=0.425$.

The final step in selecting parent galaxy samples is to identify $z$- and $W1$-band absolute magnitude cuts in each redshift bin that yield complete samples which also include the full absolute magnitude range of DESI LRG targets in that bin. Figure~\ref{fig:abs_mag_cuts} shows $K$-corrected absolute magnitude distributions for all $z<20.7$ DECaLS galaxies in each redshift bin. For each bandpass and redshift bin combination, we identify an absolute magnitude cut (dashed black lines in Figure~\ref{fig:abs_mag_cuts}) that eliminates fainter galaxies where DECaLS becomes incomplete, while preserving $\gtrsim99\%$ of DESI LRG targets (solid orange and hatched purple histograms).

Table~\ref{tab:parent_samples} lists the details of each parent galaxy sample, including the relevant absolute magnitude cut, sample size, effective number density, and the included fractions of IR and optical DESI LRG targets. Besides enforcing statistically complete parent samples, these magnitude cuts also eliminate galaxies with larger photometric redshift errors, increasing the accuracy of clustering measurements (see \S\ref{subsec:wprp}).

Figure~\ref{fig:cmd} shows example optical ($r-z$ versus $M_z$) and IR ($r-W1$ versus $M_{W1}$) color--magnitude diagrams of the magnitude-limited parent galaxy samples for the {$0.4 < \zphot < 0.5$} redshift bin. Also shown in Figure~\ref{fig:cmd} are the color--magnitude distributions of optical and IR DESI LRG targets.


\section{Modeling}\label{sec:model}
\section{Proposed Framework: {\ourmodel}}
\label{model}


In this section, we introduce a novel self-supervised co-training framework {\ourmodel}.
The proposed framework is illustrated in Figure~\ref{fig:intro_model} and works in three phases.
Phase one automatically generates two sets of pseudo labels.
We use a combination of off-the-shelf pre-trained POS and NER taggers, knowledge graph, and GPT-2 scorer for generating the first set of pseudo labels automatically without any hand-crafted rules for matching the slot values.
The other set of pseudo labels is acquired through a zero-shot slot filling model~\cite{liu2020coach}, trained on the out-of-domain dataset.
It is critical to emphasize that both sets of labels are noisy and incomplete which poses serious challenges to training effective models for the task of open-domain slot filling.
Phase two fine-tunes the pre-trained BERT to the slot filling task that effectively transfers the knowledge from the pre-trained language model~(LM) to overcome the issue of label incompleteness to some extent. 
Further, we employ the early stopping technique to minimize the noise in the labels.
The output of this phase is two BERT models that can generate soft labels for self-supervision during co-training in phase three.
Phase three leverages the fine-tuned models and further trains them in an iterative fashion.
Specifically, the proposed peer training approach facilitates high-confidence soft label selection for the other peer to perform training. This phase progressively reduces the noise in the labels and enables effective model fitting. 



\subsection{Phase One: Automatic Label Generation}
To acquire the first set of labels, we perform the following steps.
First of all, off-the-shelf trained POS and NER taggers are used to predict initial estimates of the slot values irrespective of the slot types. Then, the type information of the slot values is queried from the KG and the slot value is tagged for the most appropriate slot in the target domain.
This approach, however, produces low recall. 
To expand the candidate slot values, we generate n-grams of the natural language text and employ a partial matching scheme to query the KG for type information (e.g., \myspecial{Jason} \myspecial{Aldean} = \myspecial{American} \myspecial{singer}) of the n-grams if the entry exists.
This process generates multiple overlapping hypotheses about the slot values.
We replace a span of text that corresponds to a slot value by its type information and a GPT-2 based scorer (see Section~\ref{sec:nlpmodels}) is used to select the best candidate based on the fluency of the text.
Naturally, if a token (or span of tokens) is replaced by its type, the sentence should score higher as compared to the case where an inappropriate substitution is performed. 
We select the best hypothesis if the score is greater than the threshold.
Intuitively, the candidate selection threshold can automatically be searched based on a small validation set from the target domain, making the label generation process fully automatic. 
The other set of noisy labels is acquired by the zero-shot slot filling model~\cite{liu2020coach} that has been trained using an out-of-domain dataset. It is important to highlight that the zero-shot slot filling model does not require any labeled in-domain training example. 
To summarize the automatic label generation phase, both sets of labels are acquired in a fully automatic fashion without any hand-crafting.


In contrast to previous work in weak supervision~\cite{ren2015clustype,he2017autoentity,fries2017swellshark,giannakopoulos2017unsupervised} that obtains a single set of noisy labels and then propose techniques to overcome the challenge of fitting an effective model to the noisy labels, we acquire two sets of complementary labels.
The choice of these two sets of labels is guided by the intuition that they should be complementary and the models trained on these sets of labels should be able to share complementary information with the other to improve the performance in the later phases of the framework.
Essentially, the first set of labels carries information from external knowledge sources, whereas the labels generated through the pre-trained zero-shot slot filling model capture how the slot values are mentioned in other domains.
%
To further elaborate on the motivation and our process for the first set of labels (i.e., labels using KG and other NLP models), the pre-trained LMs have been shown to have a great deal of knowledge~\cite{petroni2019language}, thus should be capable of generating automatic labels with no need of external KG. 
To the best of our knowledge, there exists no work that shows that accurate token-level automatic labeling (e.g., slot filling task) is possible with pre-trained LMs. 
Moreover, such approaches would require heavy prompting in each new target domain, whereas our label generation process is fully automatic and only relies on the readily-available pre-trained NLP models and external KG.

\subsection{Phase Two: LM-assisted Weak Supervision}
Since we do not have access to dataset $\{(\mathbf{X}_n,\mathbf{Y}_n)\}_{n=1}^N$ with true ground-truth labels.
We use pseudo labels generated in phase one, $\{(\mathbf{X}_n,\mathbf{D}_n)\}_{n=1}^N$, to learn 
$f_{m,c}(\cdot; \cdot)$ that outputs the probability of the $m$-th token to take on class $c$. 
We learn $f_{m,c}(\cdot; \cdot)$ by minimizing the following loss over the noisy dataset $\{(\mathbf{X}_n,\mathbf{D}_n)\}_{n=1}^N$: 
$$
\hat\theta = \argmin_{\theta}\frac{1}{N}\sum_{n=1}^{N} \ell(\mathbf{D}_n, f(\mathbf{X}_{n}; \theta)),
\label{eq:stage1}
$$
where $\ell(\mathbf{D}_n, f(\mathbf{X}_{n}; \theta)) = \frac{1}{M} \sum_{m=1}^{M} -\log{f_{m,d_{n, m}}(\mathbf{X}_{n}; \theta)}$. 
We employ the pre-trained multilingual BERT with token-level classification head that uses Adam optimizer \cite{kingma2014adam,Liu2019} with early stopping and multiple random initializations. 


Since slot filling task is similar to the MLM training objective of the BERT, we employ pre-trained BERT as the backbone model.
That is, MLM's goal is to predict the masked tokens using bidirectional contexts. Similarly, slot filling tries to predict the label for a token leveraging both left and right contexts simultaneously, which makes the pre-trained BERT an ideal model of choice that greatly facilitates minimizing incomplete labels.
It is important to highlight that our automatically generated labels are not only incomplete but also potentially wrong.
The training strategies employed in this phase minimize the noise in the label to some extent. 
Specifically, early stopping can provide a strong regularization and would not let the model overfit to the noisy labels, especially wrong labels. 
Moreover, early stopping does not let the model forget the knowledge in the pre-trained model.
Similarly, multiple random initializations enforce robustness. 
Since the model is fine-tuned on the noisy labels, averaging the predictions of multiple models for each token ensures that wrong labels end up with low probabilities and true labels consistently achieve high probabilities.
Using the above-mentioned strategies, we train two slot filling models, which we call the peers. The peer one is trained on the first set of pseudo labels that were generated using POS and NER taggers, KG, and the GPT-2 scorer in phase one. Similarly, peer two is trained using the predictions of the zero-shot slot filling model~\cite{liu2020coach}.
Both models have the same architecture and follow the same training procedures.

\begin{table*}[t!]
\centering
\caption{Dataset statistics.}
\vspace{-7pt}
\label{tab:dataset}
\begin{tabular}{lccccc}
\toprule
\textbf{Dataset}  & \textbf{Dataset Size} & \textbf{Vocab. Size} & \textbf{Avg. Length} & \textbf{\# of Domains} & \textbf{\# of Slots} \\ \hline
\textbf{SGD}      & 188K                  & 33.6K                & 13.8                 & 20                     & 240                  \\
\textbf{MultiWoZ} & 67.4K                 & 10.5K                & 13.3                 & 8                      & 61 \\
\bottomrule
\end{tabular}
\vspace{-7pt}
\end{table*}

\subsection{Phase Three: Self-supervised Co-training}
We introduce an iterative peer training algorithm where both peers generate high-confidence soft labels for training the other peer in the next iteration. 
Theoretically, these peers can be anything, but in this work, 
we explore two of the most promising directions that have shown the promise to minimize the need for manual labeling for the task: zero-shot learning and distant supervision.
This phase uses a self-supervised co-training scheme to exploit the patterns of slot values from other domains through the labels generated by the zero-shot filling model (i.e., peer two)~\cite{liu2020coach} as well as utilize the knowledge in external KGs and pre-trained models via labels provided by the peer one.
Specifically, we initialize the peers trained in phase two and use their pseudo labels to kick-start training in this phase.
Specifically, peer one $f_{m,c}(\cdot; \theta_{\textrm{p1}})$ would generate labels $\{\tilde{\mathbf{Y}}^{(t)}_n = [\tilde{y}_{n,1}^{(t)}, ..., \tilde{y}_{n,m}^{(t)}]\}_{n=1}^{N}$ for peer two $f_{m,c}(\cdot; \theta_{\textrm{p2}})$ at the $t$-th iteration by:
$$
\tilde{y}_{n,m}^{(t)} = \argmax_{c}{f_{m,c}(\mathbf{X}_n; \theta_{\textrm{p1}}^{(t)})}. 
\label{eq:pseudo}
$$

Based on these labels, the peer two can be fine-tuned by: 
$$
\hat\theta_{\textrm{p2}}^{(t+1)} = \argmin_{\theta}\frac{1}{N}\sum_{n=1}^N \ell(\tilde{\mathbf{Y}}_n^{(t)}, f(\mathbf{X}_{n}; \theta)).
\label{eq:self_train1}
$$

Similarly, peer two $f_{m,c}(\cdot; \theta_{\textrm{p2}})$ would generate pseudo labels for peer one $f_{m,c}(\cdot; \theta_{\textrm{p1}})$ that are used to fine-tune peer one. 
We also notice that it is beneficial to stop early during this phase as well, to improve the model fitting and gradually reduce the noise associated with the automatically generated labels.
Since pseudo labels are refined gradually in an iterative way, both peers can benefit from the knowledge contained within the labels of the other while avoiding overfitting.
Furthermore, as an alternative to pseudo labels, we also generate soft labels that are used for confidence re-weighting. 
The high-confidence soft label selection strategy enables better model fitting and efficient learning via better quality of the automatic labels.
Specifically, for the given $m$-th token in the $n$-th training example, the probability for all classes $C$ is $[f_{m,1}(\mathbf{X}_n;\theta),...,f_{m,C}(\mathbf{X}_n;\theta)]$. 
Following ~\cite{xie2016unsupervised}, at $t$-th iteration, peer one generates soft labels, $\{\mathbf{S}_n^{(t)} = [\mathbf{s}_{n,m}^{(t)}]_{m=1}^M \}_{n=1}^N$, as given below:
$$
\mathbf{s}_{n,m}^{(t)} = [s_{n,m,c}^{(t)}]_{c=1}^{C} = \Bigg[  \frac{f_{m,c}^2(\mathbf{X}_n;\theta_{\textrm{peer1}}^{(t)})/p_{c}}{\sum_{c'=1}^C f_{m,c'}^2(\mathbf{X}_n;\theta_{\textrm{peer1}}^{(t)})/p_{c'}}\Bigg]_{c=1}^{C}
\label{eq:soft}
$$ 
where $p_{c} = \sum_{n=1}^N \sum_{m=1}^M f_{m,c}(\mathbf{X}_n;\theta_{\textrm{p1}}^{(t)})$ computes the frequency of the tokens for the $c$-th class. 
Then, peer two $f(\cdot; \theta_{\textrm{p2}}^{(t+1)})$ is fine-tuned by:
$$
\theta_{\textrm{p2}}^{(t+1)} = \argmin_{\theta} \frac{1}{N} \sum_{n=1}^{N} \ell_{\rm KL}(\mathbf{S}_n^{(t)}, f(\mathbf{X}_{n}; \theta)),
$$
where $\ell_{\rm KL}(\cdot,\cdot)$ is the KL-divergence-based loss:
$$
\ell_{\rm KL}(\mathbf{S}_n^{(t)}, f(\mathbf{X}_{n}; \theta))=\frac{1}{M}\sum_{m=1}^M\sum_{c=1}^C - s_{n,m,c}^{(t)} \log f_{m,c}(\mathbf{X}_{n}; \theta).
\label{eq:klloss}
$$

Moreover, we also investigate selecting tokens that have high confidence. 
For instance, we pick high-confidence tokens from the $m$-th input example at the $t$-th iteration by  
$
H^{(t)}_n = \{m : \max_{c} s_{n,m,c}^{(t)} > \epsilon \},
$
where $\epsilon\in [0,1]$ is a threshold that can be searched based on a small validation set. 
Then, peer two $f(\cdot; \theta_{\textrm{p2}}^{(t+1)})$ is fine-tuned by:
$$
\theta_{\textrm{p2}}^{(t+1)} %&= \argmin_{\theta} \frac{1}{N} \sum_{n=1}^{N} \ell_{\rm S-KL}(\bS_n^{(t)}, f(\bX_{n}; \theta)) \\
= \argmin_{\theta} \frac{1}{N|H^{(t)}_n|}\sum_{n=1}^{N} \sum_{m\in H^{(t)}_n}\sum_{c=1}^C - s_{n,m,c}^{(t)} \log f_{m,c}(\mathbf{X}_{n}; \theta).
$$

This phase improves the robustness to effectively fit the model for tokens with high confidence. 
Both peers keep sharing information and their confidence by producing soft labels for their counterparts until they approximate to the true labels while employing early stopping and scheduled learning rates.
It is important to remind that phase three is the most important phase that progressively reduces noise from the labels to a great extent and enables superior performance for the task of open-domain slot filling.

\section{Predicted LRG properties}\label{sec:predict}
In this section we present the results of our $z$-band and $W1$-band models. We report the HOD parameters of mock LRGs and compare the predicted clustering to the data and to relevant studies in the literature.

\subsection{LRG clustering}\label{subsec:clust_lrg}

\begin{figure*}
  \centering
    \includegraphics[width=\linewidth]{figures/lrg_csrand.png}
    \caption{Predicted clustering of IR (top row) and optical (bottom row) mock LRGs compared to the clustering of the relevant LRG target samples from the data (black points in each panel).
    Dotted purple lines in each panel show mock LRGs from the default $z$-band model (monotonic correspondence between $r-z$ color and \zstarve at fixed luminosity), while solid purple lines show LRGs from the $z$-band model with randomized colors.
    Dashed orange lines in each panel show mock LRGs from the default $W1$-band model, and solid orange lines show LRGs from the $W1$-band model with randomized colors.
    Each column shows a different redshift bin.
    }
\label{fig:wp_lrg}
\end{figure*}


\begin{figure}
\centering
\includegraphics[width=\linewidth]{color-color_IR-opt_compare_z0p40-0p50.png}
\caption{IR color ($r-W1$) versus optical color ($r-z$) for IR (left panel) and optical (right panel) LRGs at $0.4<\zphot<0.5$.
}
\label{fig:color_color}
\end{figure}


\begin{figure}
  \centering
    \includegraphics[width=\linewidth]{figures/data_LRG-SED_redder.png}
    \includegraphics[width=\linewidth]{figures/data_LRG-SED_bluer.png}
  \caption{Distributions of ${r-z}$, ${g-r}$, and ${r-W1}$ colors of optical (shaded histograms) and IR (open histograms) LRGs from above (Box A; top three panels) and along (Box B; bottom three panels) the red sequence in ${r-W1}$ versus $M_{W1}$ color--magnitude space. The regions of color--magnitude space corresponding to Box A and Box B are shown in Figure~\ref{fig:cmd}. The mean, median, and skewness of each distribution, as well as the number of LRGs in each population, are given in Table~\ref{tab:data_sed}. Optical and IR LRGs from along the red sequence have similar distributions of each color, while optical LRGs above the red sequence have redder ${r-z}$ and ${g-r}$ and bluer ${r-W1}$ colors compared to IR LRGs in the same $W1$-band luminosity range.
  }
\label{fig:data_sed}
\end{figure}


\begin{deluxetable*}{ c }
  \tablecaption{Statistics of distributions of ${r-z}$, ${g-r}$, and ${r-W1}$ colors of optical and IR LRGs selected from above (Box A) and along (Box B) the red sequence in IR color--magnitude space (${r-W1}$ versus $M_{W1}$). The boxed regions A and B are shown for the $0.4 < \zphot < 0.5$ redshift bin in Figure~\ref{fig:cmd}, and the color distributions are shown in Figure~\ref{fig:data_sed}. The mean, median, and skewness of each distribution is shown for IR (optical) LRGs in bold (plain) text.
  \label{tab:data_sed}
  }
  \startdata
  \includegraphics[width=\linewidth, trim=0 2mm 0 0, clip]{tables/tab5.png} \\
  \enddata
\end{deluxetable*}

%\begin{deluxetable*}{ r r r r r r r r r r r r r r }
%\tablecaption{Statistics of distributions of ${r-z}$, ${g-r}$, and ${r-W1}$ colors of optical and IR LRGs selected from above (Box A) and along (Box B) the red sequence in IR color--magnitude space (${r-W1}$ versus $M_{W1}$). The boxed regions A and B are shown for the $0.4 < \zphot < 0.5$ redshift bin in Figure~\ref{fig:cmd}, and the color distributions are shown in Figure~\ref{fig:data_sed}. The mean, median, and skewness of each distribution is shown for IR (optical) LRGs in bold (plain) text.
%\label{tab:data_sed}
%}
%\tablehead{
%& \multicolumn{6}{l}{{\large Box A} (LRGs above red sequence)} & & \multicolumn{6}{l}{{\large Box B} (LRGs along red sequence)}
%}
%\startdata
%\multicolumn{14}{c}{ $0.4 < \zphot < 0.5$ } \\
%\vspace{-1ex} \\
%\hline
%\vspace{-1ex} \\
%& \multicolumn{2}{c}{$(r - z)$} & \multicolumn{2}{c}{~~$(g - r)$} & \multicolumn{2}{c}{~~$(r - W1)$} &
%& \multicolumn{2}{c}{$(r - z)$} & \multicolumn{2}{c}{~~$(g - r)$} & \multicolumn{2}{c}{~~$(r - W1)$} \\
%\vspace{-1ex} \\
%\hline
%\vspace{-1ex} \\
%mean & \bf{1.04} & 1.10 & ~~\bf{1.64} & 1.74 & ~~\bf{2.30} & 2.18 & ~~mean & \bf{0.98} & 1.01 & ~~\bf{1.72} & 1.78 & ~~\bf{1.74} & 1.68 \\
%median & \bf{1.06} & 1.10 & ~~\bf{1.65} & 1.74 & ~~\bf{2.25} & 2.14 & ~~median & \bf{0.98} & 1.00 & ~~\bf{1.78} & 1.80 & ~~\bf{1.73} & 1.67 \\
%skewness & \bf{-1.86} & 0.44 & ~~\bf{-0.36} & 1.15 & ~~\bf{1.44} & 1.30 & ~~skewness & \bf{-0.63} & 0.33 & ~~\bf{-1.20} & -0.73 & ~~\bf{0.12} & -0.02 \\
%\vspace{-1ex} \\
%\hline
%\vspace{-1ex} \\
%$\boldsymbol{N}_{\rm \mathbf{IR}}$	& \multicolumn{2}{l}{\bf{170,492}}	& \multicolumn{4}{l}{\multirow{2}{*}{$N_{\rm IR}/N_{\rm opt}=3.44$}} &
%$\boldsymbol{N}_{\rm \mathbf{IR}}$	& \multicolumn{2}{l}{\bf{313,311}}	& \multicolumn{4}{l}{\multirow{2}{*}{$N_{\rm IR}/N_{\rm opt}=0.60$}} \\
%$N_{\rm opt}$	& \multicolumn{2}{l}{\phantom{00}49,517} & & & & & $N_{\rm opt}$ & \multicolumn{2}{l}{\phantom{0}521,970} & & & & \\[-1ex]
%\cutinhead{$0.5 < \zphot < 0.6$}
%\vspace{-1ex} \\
%& \multicolumn{2}{c}{$(r - z)$} & \multicolumn{2}{c}{~~$(g - r)$} & \multicolumn{2}{c}{~~$(r - W1)$} &
%& \multicolumn{2}{c}{$(r - z)$} & \multicolumn{2}{c}{~~$(g - r)$} & \multicolumn{2}{c}{~~$(r - W1)$} \\
%\vspace{-1ex} \\
%\hline
%\vspace{-1ex} \\
%mean & \bf{1.18} & 1.29 & ~~\bf{1.71} & 1.87 & ~~\bf{2.64} & 2.53 & ~~mean & \bf{1.13} & 1.18 & ~~\bf{1.71} & 1.80 & ~~\bf{2.14} & 2.11 \\
%median & \bf{1.20} & 1.30 & ~~\bf{1.73} & 1.88 & ~~\bf{2.61} & 2.49 & ~~median & \bf{1.16} & 1.20 & ~~\bf{1.79} & 1.81 & ~~\bf{2.16} & 2.12 \\
%skewness & \bf{-0.99} & -1.17 & ~~\bf{-0.60} & 0.23 & ~~\bf{1.11} & 1.68 & ~~skewness & \bf{-0.83} & -0.82 & ~~\bf{-1.30} & -1.14 & ~~\bf{-0.46} & -0.41 \\
%\vspace{-1ex} \\
%\hline
%\vspace{-1ex} \\
%$\boldsymbol{N}_{\rm \mathbf{IR}}$	& \multicolumn{2}{l}{\bf{179,348}}	& \multicolumn{4}{l}{\multirow{2}{*}{$N_{\rm IR}/N_{\rm opt}=3.51$}} &
%$\boldsymbol{N}_{\rm \mathbf{IR}}$	& \multicolumn{2}{l}{\bf{441,691}}	& \multicolumn{4}{l}{\multirow{2}{*}{$N_{\rm IR}/N_{\rm opt}=0.67$}} \\
%$N_{\rm opt}$	& \multicolumn{2}{l}{\phantom{00}51,093} & & & & & $N_{\rm opt}$ & \multicolumn{2}{l}{\phantom{0}660,486} & & & & \\[-1ex]
%\cutinhead{$0.6 < \zphot < 0.7$}
%\vspace{-1ex} \\
%& \multicolumn{2}{c}{$(r - z)$} & \multicolumn{2}{c}{~~$(g - r)$} & \multicolumn{2}{c}{~~$(r - W1)$} &
%& \multicolumn{2}{c}{$(r - z)$} & \multicolumn{2}{c}{~~$(g - r)$} & \multicolumn{2}{c}{~~$(r - W1)$} \\
%\vspace{-1ex} \\
%\hline
%\vspace{-1ex} \\
%mean & \bf{1.39} & 1.47 & ~~\bf{1.77} & 1.90 & ~~\bf{2.96} & 2.90 & ~~mean & \bf{1.35} & 1.41 & ~~\bf{1.69} & 1.79 & ~~\bf{2.56} & 2.54 \\
%median & \bf{1.42} & 1.48 & ~~\bf{1.79} & 1.90 & ~~\bf{2.92} & 2.87 & ~~median & \bf{1.40} & 1.43 & ~~\bf{1.76} & 1.81 & ~~\bf{2.59} & 2.56 \\
%skewness & \bf{-1.31} & -0.84 & ~~\bf{-0.39} & 0.04 & ~~\bf{1.36} & 1.90 & ~~skewness & \bf{-1.43} & -1.23 & ~~\bf{-0.88} & -0.61 & ~~\bf{-0.85} & -0.74 \\
%\vspace{-1ex} \\
%\hline
%\vspace{-1ex} \\
%$\boldsymbol{N}_{\rm \mathbf{IR}}$	& \multicolumn{2}{l}{\bf{249,983}}	& \multicolumn{4}{l}{\multirow{2}{*}{$N_{\rm IR}/N_{\rm opt}=1.88$}} &
%$\boldsymbol{N}_{\rm \mathbf{IR}}$	& \multicolumn{2}{l}{\phantom{,}\bf{748,591}}	& \multicolumn{4}{l}{\multirow{2}{*}{$N_{\rm IR}/N_{\rm opt}=0.72$}} \\
%$N_{\rm opt}$	& \multicolumn{2}{l}{\phantom{0}132,656} & & & & & $N_{\rm opt}$ & \multicolumn{2}{l}{1,044,175} & & & & \\
%\enddata 
%\end{deluxetable*}


Our $z$-band and $W1$-band models yield magnitude-limited mock galaxy catalogs from which we select mock LRGs. From these results we predict the clustering of DESI LRG target samples.

Figure~\ref{fig:wp_lrg} shows the predicted clustering of IR and optical mock LRGs compared to the relevant DESI LRG target sample. In addition to the data (black points with error bars), each panel of Figure~\ref{fig:wp_lrg} shows the clustering of mock LRGs according to the $z$-band and $W1$-band models, with and without the addition of scatter to the model color--\zstarve relation (\S\ref{subsec:color_assign}).

As shown in the top row of Figure~\ref{fig:wp_lrg}, the $z$-band and $W1$-band models predict very similar clustering signals for IR LRGs. The predicted amplitude of both the one-halo and two-halo terms exceeds that of the data in the two lowest redshift bins, and is in better agreement in the highest redshift bin ($\zsim=0.628$). Across all redshift bins the addition of scatter to the color--\zstarve relation brings the predicted clustering amplitude of IR LRGs closer to the data, reducing the clustering amplitude of the one-halo term by $\sim15$--$20\%$, and the two-halo term by $\sim10\%$, depending on redshift bin. However, this reduction in the discrepancy between the model and data is a ceiling achieved only by adding such a high degree of scatter that color is uncorrelated with our proxy for halo assembly history\footnote{We also tested using proxies for IR and optical LRGs by selecting the $N$ brightest mock galaxies from the relevant magnitude-limited mock galaxy catalog to recreate the number densities of the IR and optical DESI LRG target samples, instead of applying the DESI LRG target selections to the magnitude-limited mocks as described in \S\ref{subsec:mock_lrg_select}. The result was a slightly larger predicted clustering amplitude compared to the ``random colors" models that do use the DESI LRG target selections, shown in Figure~\ref{fig:wp_lrg}.} (\zstarve).

The bottom row of Figure~\ref{fig:wp_lrg} shows that the $z$-band and $W1$-band models predict different clustering amplitudes for optical LRGs. The $z$-band model overpredicts the clustering amplitude of optical LRGs to a similar degree as the overprediction for IR LRGs. Assigning colors at random such that galaxy color is uncorrelated with halo assembly history again reduces this discrepancy of the one-halo (two-halo) term by $\sim20$--$25\%$ ($\sim10$--$15\%$).

On the other hand, the $W1$-band model prediction for the clustering of optical LRGs is in very good agreement with the data, especially the two-halo term (the one-halo term is actually slightly underpredicted). Additionally, adding \emph{any} amount of scatter to the color--\zstarve relation has no effect on the predicted clustering amplitude in this case.

The distributions of optical and IR LRGs in color--magnitude space are helpful for interpreting Figure~\ref{fig:wp_lrg}. As Figure~\ref{fig:cmd} shows, in optical space (${M_r-M_z}$ versus $M_z$), both optical and IR LRGs are largely confined to the red sequence, with the distribution extending below it toward bluer color, especially at higher redshift.

In contrast, in IR space (${M_r-M_{W1}}$ versus $M_{W1}$) the galaxy distribution does extend significantly above the red sequence. For example, in the ${0.4 < \zphot < 0.5}$ redshift bin, the red sequence is at ${M_r-M_{W1}\sim1.6}$, but the distribution extends to ${M_r-M_{W1} \gtrsim 3.0}$. IR LRGs occupy the entire region of excess ${M_r-M_{W1}}$ color across the full range of $M_{W1}$, while optical LRGs occupy only the part of this region that corresponds to more luminous $M_{W1}$, i.e., the optical LRG selection excludes a population of galaxies with very red ${M_r-M_{W1}}$ colors and moderately luminous $W1$-band luminosities that are included by the IR selection.

The top panels of Figure~\ref{fig:data_sed} explore the differences in the color distributions of optical and IR LRGs located above the red sequence (Box A in Figure~\ref{fig:cmd}) in ${r-W1}$ versus $M_{W1}$ color--magnitude space. The bottom three panels (Box B in Figure~\ref{fig:cmd}) show the color distributions of optical and IR LRGs from the same $M_{W1}$ range \emph{along} the red sequence. 

Along the infrared red sequence, the SEDs of optical and IR LRGs are quite similar. However, the two selections diverge above the red sequence. Table~\ref{tab:data_sed} lists the mean, median, and skewness of each color distribution, as well as the number of LRGs in each population.

Above the red sequence IR LRGs have bluer ${r-z}$ and ${g-r}$ colors than optical LRGs in the same region of color--magnitude space, while IR LRGs have redder ${r-W1}$ colors than their optical counterparts from the same color--magnitude region.
The same general trend is seen along the red sequence, but the color differences between the optical and IR LRG selections are smaller, i.e., along the red sequence IR LRGs also have bluer ${r-z}$ and ${g-r}$ colors than optical LRGs, but by only half as much as above the red sequence. Similarly, IR LRGs along the red sequence have redder ${r-W1}$ colors than their optical counterparts, but the color difference is two to three times smaller than for IR and optical LRGs above the red sequence.

Activity from active galactic nuclei (AGN) may artificially inflate the observed IR ($W1$-band) luminosities and artificially redden the $r-W1$ colors of some galaxies that pass the IR LRG selection \citep[e.g.,][]{webster_etal95, georgakakis_etal09, banerji_etal12, glikman_etal12, kim_im18, klindt_etal19, rivera_etal21}, relative to galaxies with comparable IR luminosities and ${r-W1}$ colors but no AGN activity.
The bluer ${r-z}$ colors of IR LRG targets cause these objects to be excluded by the optical LRG selection (e.g., panel (e) of Figure~\ref{fig:not-lrg_outliers}).
Our IR ($W1$-band) model would assign mock IR LRGs with artificially red $r-W1$ colors and high IR luminosities due to AGN activity to halos with higher bias than if their $r-W1$ colors and $W1$-band magnitudes were not influenced by AGN activity.
This could explain why our $W1$-band model overpredicts the clustering amplitude of IR LRGs relative to optical LRGs, as shown in Figure~\ref{fig:wp_lrg}.

The influence of AGN activity may also explain the lack of correlation between IR and optical color when assigning colors to mock galaxies based solely on a proxy for halo age (here \zstarve). This modeling assumption attributes galaxy color entirely to halo mass accretion history, and does not account for how baryonic effects such as AGN activity may contribute to galaxy color. In other words, optical color is not necessarily a reliable proxy for IR color. This is also illustrated by Figure~\ref{fig:color_color}, which shows optical (${r-z}$) versus IR (${r-W1}$) color for both IR and optical LRGs in the $0.4 < \zphot < 0.5$ redshift bin. The optical and IR colors of optical LRGs are more closely correlated than for IR LRGs, but in both cases there is considerable scatter in $r-W1$ at fixed $r-z$.
\citet{xu_etal22} also find a lack of correlation between halo assembly properties and galaxy color using a semi-analytic model.

\subsection{The LRG--halo connection}\label{subsec:lrg-halo}

\begin{figure*}
\centering
    \includegraphics[width=\linewidth]{hod-lrg_cs-v2.png}
    \caption{Predicted HODs of IR (top row) and optical (bottom row) mock LRGs from both the $z$-band (heavier purple lines) and $W1$-band (heavier orange lines) models. Also shown in fainter purple and orange lines are HODs of the $z$-band and $W1$-band magnitude-limited mock galaxy catalogs from which mock LRGs are selected. Solid (dotted) lines show results for central (satellite) galaxies and LRGs.
    Each column shows a different redshift bin.
    The best-fit values of a standard five-parameter HOD model, as well as the satellite fraction, for each mock LRG sample are given in Table~\ref{tab:hod_params}.
    }
\label{fig:hod_lrg}
\end{figure*}


\begin{deluxetable*}{ c c c c c }
\tablecaption{Predicted halo occupation completeness for mock central LRGs by selection (IR or optical). Results for the $W1$-band ($z$-band) model are in bold (plain) text.
\label{tab:hod_stats}
}
\tablehead{
\colhead{\zsim} & \colhead{LRG selection} &
\colhead{peak $\langle N_{\rm cen} | \mvir \rangle$\tablenotemark{a}} &
\colhead{min \mvir of peak $\langle N_{\rm cen} | \mvir \rangle$\tablenotemark{b}} &
\colhead{$\langle N_{\rm cen} | (\mvir > 10^{14.75}\ \msunh) \rangle$\tablenotemark{c}}
}
\startdata
\multirow{2}{*}{$0.425$} & IR & {\bf 0.92}~~0.91 & {\bf 14.45}~~14.85 & {\bf 0.74}~~0.70 \\
	& optical & {\bf 0.85}~~0.90 & {\bf 14.35}~~14.55 & {\bf 0.62}~~0.71 \\
\vspace{-1ex} \\
\hline
\vspace{-1ex} \\
\multirow{2}{*}{$0.523$} & IR & {\bf 0.91}~~0.87 & {\bf 14.45}~~14.65 & {\bf 0.77}~~0.61 \\
	& optical & {\bf 0.85}~~0.84 & {\bf 14.75}~~14.65 & {\bf 0.69}~~0.64 \\
\vspace{-1ex} \\
\hline
\vspace{-1ex} \\
\multirow{2}{*}{$0.628$} & IR & {\bf 0.96}~~0.82 & {\bf 14.85}~~14.55 & {\bf 0.96}~~0.31 \\
	& optical & {\bf 0.92}~~0.84 & {\bf 14.95}~~14.55 & {\bf 0.92}~~0.38 \\
\enddata
\tablenotetext{a}{Highest central occupation fraction reached across all halo \mvir.}
\tablenotetext{b}{Minimum halo \mvir at which the highest central occupation fraction is achieved.}
\tablenotetext{c}{Mean central occupation fraction at the largest halo masses ($\mvir \gtrsim 10^{14.75}\ \msunh$), which is universally less or equal to than the peak central occupation fraction.}
\end{deluxetable*}


We compute the mean central and satellite halo occupation statistics as a function of halo mass directly from the mock galaxy catalogs and halo abundances from the MDPL2 simulation. The results are shown in Figure~\ref{fig:hod_lrg}.
Each panel of Figure~\ref{fig:hod_lrg} shows ${\langle N | \mvir \rangle}$ for the full magnitude-limited mock galaxy catalog in gray, while optical and IR LRGs are shown in purple and orange, respectively.

The key conclusions of Figure~\ref{fig:hod_lrg} are summarized in Table~\ref{tab:hod_stats}, which gives the peak value of ${\langle N_{\rm cen} \rangle}$ for optical and IR LRGs predicted by both the $z$-band and $W1$-band models, as well as the mean value of ${\langle N_{\rm cen} \rangle}$ for the most massive halos, which is universally less than the corresponding peak value.

The $z$-band model does not predict significant differences between the populations selected by the IR and optical LRG selection functions. According to this model the central LRG halo occupation fraction peaks at $\sim90\%$ at $z\sim0.43$, and drops to $\sim70\%$ at the largest halo masses, while by $z\sim0.63$ $\langle N_{\rm cen}\rangle$ peaks at 82--84\% and falls to $\sim35\%$ for the most massive halos, regardless of LRG selection function.

In contrast, the $W1$-band model \emph{does} predict a different LRG--halo relationship for IR versus optical LRGs. In all redshift bins this model has ${\langle N_{\rm cen} \rangle}$ for IR LRGs peaking at around 92\% and remaining at $\gtrsim75\%$ for the most massive halos, while ${\langle N_{\rm cen} \rangle}$ for optical LRGs peaks at 85--92\%, and falls at the largest halo masses as low as 62\% ($z\sim0.43$) to 69\% ($z\sim0.52$), although it remains at 92\% for the highest redshift bin ($z\sim0.63$).

\begin{deluxetable}{ l r }
\tablecaption{Prior ranges for HOD fit parameters (Eqs.~\ref{eq:hod_cen} and \ref{eq:hod_sat}).
\label{tab:hod_params_priors}
}
\tablehead{
\colhead{Parameter} & \colhead{Prior interval}
}
\startdata
$\log(M_{\rm min})$ & $(11.0,\ 14.0)$ \\
$\sigma_{\log M}$ & $(0.001,\ 1.5)$ \\
$\alpha$ & $(0.0,\ 2.0)$ \\
$\log(M_0)$ & $(11.0,\ 14.0)$ \\
$\log(M_1)$ & $(11.5,\ 15.5)$ \\
\enddata
\end{deluxetable}


\begin{deluxetable*}{ c c c c c c c c }
\tablecaption{Best-fit values of standard five-parameter HOD model (Eqs.~\ref{eq:hod_cen} and \ref{eq:hod_sat}), and the satellite fraction, $f_{\rm sat}$, of IR and optical mock LRG samples. Results for the $W1$-band ($z$-band) model are in bold (plain) text.
\label{tab:hod_params}
}
\tablehead{
\colhead{\zsim} & \colhead{LRG selection} &
\colhead{$\log(M_{\rm min}/\msunh)$} &
\colhead{$\sigma_{\log M}$} & \colhead{$\alpha$} &
\colhead{$\log(M_0/\msunh)$} &
\colhead{$\log(M_1/\msunh)$} &
\colhead{$f_{\rm sat}$}
}
\startdata
\multirow{2}{*}{0.425} & IR & {\bf 13.28}~~13.35 & {\bf 0.93}~~0.95 & {\bf 0.98}~~1.00 & {\bf 12.93}~~12.87 & {\bf 14.02}~~14.07 & {\bf 14.8}~~14.5 \\
 & optical & {\bf 13.32}~~13.26 & {\bf 1.05}~~0.93 & {\bf 0.95}~~0.98 & {\bf 12.91}~~12.89 & {\bf 14.00}~~14.00 & {\bf 14.9}~~14.8 \\
\vspace{-1ex} \\
\hline
\vspace{-1ex} \\
\multirow{2}{*}{0.523} & IR & {\bf 13.37}~~13.52 & {\bf 0.99}~~1.14 & {\bf 0.94}~~0.91 & {\bf 12.94}~~12.98 & {\bf 14.05}~~14.09 & {\bf 14.4}~~14.5 \\
 & optical & {\bf 13.45}~~13.44 & {\bf 1.15}~~1.11 & {\bf 0.95}~~0.96 & {\bf 12.84}~~12.85 & {\bf 14.05}~~14.05 & {\bf 14.7}~~14.6 \\
\vspace{-1ex} \\
\hline
\vspace{-1ex} \\
\multirow{2}{*}{0.628} & IR & {\bf 13.26}~~13.40 & {\bf 1.02}~~1.18 & {\bf 0.95}~~0.95 & {\bf 12.88}~~12.82 & {\bf 13.95}~~13.99 & {\bf 14.1}~~14.4 \\
 & optical & {\bf 13.30}~~13.31 & {\bf 1.19}~~1.16 & {\bf 0.96}~~0.94 & {\bf 12.78}~~12.82 & {\bf 13.93}~~13.93 & {\bf 14.4}~~14.7 \\
\enddata 
\end{deluxetable*}


We fit a standard five-parameter HOD model to each optical and IR mock LRG sample Table~\ref{tab:hod_params}.
For ease of comparison with their results we use the same functional form as \citet{zhou_etal20b}, who derive HOD parameters from clustering measurements of DESI LRG targets using photometric redshifts in bins that approximately match the redshift bins we use. This parameterization is described in detail in \citet{zheng_etal05} and \citet{zheng_etal07}.
Briefly, the probability $\langle N_{\rm cen} \rangle$ that a halo of virial mass \mvir hosts a central galaxy is given by
%
\begin{equation}\label{eq:hod_cen}
  \langle N_{\rm cen} | \mvir \rangle = \frac{1}{2} \! \left( 1 + {\rm erf}{\left[\frac{\log(\mvir)-\log(M_{\rm min})}{\sigma_{\log M}} \right]} \right),
\end{equation}
%
\noindent where the parameter $\log(M_{\rm min})$ is the minimum halo mass for hosting a central galaxy, and the parameter $\sigma_{\log M}$ defines the steepness of the transition of $\langle N_{\rm cen} \rangle$ from $\sim0$ at low \mvir to $\sim1$ at high \mvir.

The mean number of satellite galaxies, $\langle N_{\rm sat} \rangle$, hosted by a halo of virial mass \mvir is approximated by a power law given by 
%
\begin{equation}\label{eq:hod_sat}
  \langle N_{\rm sat}|\mvir\rangle = {\left(\frac{\mvir-M_0}{M_1} \right)}^{\alpha},
\end{equation}
%
\noindent where $M_0$, $M_1$, and $\alpha$ are the remaining free parameters of the HOD fit. The HOD fits of \citet{zhou_etal20b} incorporate a sixth nuisance parameter to account for photometric redshift uncertainty, which in this work is addressed by the magnitude-dependent line-of-sight scatter parameter, \sigmalos (Eq.\ \ref{eq:sigmalos}).
Table~\ref{tab:hod_params_priors} gives the prior interval for each HOD parameter. The best-fit HOD parameters and satellite fraction of each mock LRG sample are given in Table~\ref{tab:hod_params}.


\subsection{Comparison with other ``DESI-like" LRG studies}\label{subsec:lit_compare}

\begin{figure*}
\centering
  \includegraphics[width=0.7\linewidth]{hod-compare-all.png}
  \caption{Comparison of LRG HOD parameters from this work at $z\sim0.63$ (orange lines) with the analytic HOD fits of \citet{zhou_etal20b} (green lines) and semi-analytic model of \citet{hernandez-aguayo_etal21} (gray lines) in comparable redshift bins.
  Solid, dotted, and dash-dotted lines show results respectively for central LRGs, satellite LRGs, and the combination of both.
  }
  \label{fig:hod_compare}
\end{figure*}

There are two main studies of the LRG--halo connection of ``DESI-like" LRGs suitable for comparison with our results. \citet{zhou_etal20b} (who provide the photometric redshifts used here) measure the clustering of LRGs at ${0.41 < \zphot < 0.93}$ selected from DECaLS DR7 photometry using the optical target selection (Eq.\ \ref{eq:lrg_opt}), and fit a standard five-parameter analytic HOD model in five redshift bins. 

\citet{zhou_etal20b} find little to no evolution of HOD parameters for LRGs across ${0.4 \lesssim z \lesssim 0.8}$. Our results are consistent with a lack of HOD parameter evolution across the subset of this redshift range that we model, although some of our predicted parameter values themselves differ from those of \citet{zhou_etal20b}, particularly $\sigma_{\log M}$. \citet{zhou_etal20b} find a very steep transition from a ${\langle N{\rm cen} \rangle \sim0}$ to 1 ($\sigma_{\log M}\sim0$--0.28; see their Figure 13), while our models predict a more gradual transition, with ${\sigma_{\log M}\sim1}$.

Figure~\ref{fig:hod_compare} compares our predicted LRG halo occupation statistics at $z\sim0.63$ with the best-fit HOD parameterization of \citet{zhou_etal20b} in the relevant redshift bin from their study ($0.61 < \zphot < 0.72$). Of the redshift bins \citet{zhou_etal20b} use that overlap with this work, this is the bin in which they find the greatest deviation of $\langle N{\rm cen} \rangle$ from a step function, although we also select this bin for Figure~\ref{fig:hod_compare} to enable comparison with an additional study, discussed later in this section.

Our HOD predictions for satellite LRGs agree with those of \citet{zhou_etal20b} at halo masses below ${\log(\mvir/ \msunh) \sim 14}$, and our overall satellite fractions of $f_{\rm sat}\sim0.14$--0.15 across both model bands and LRG selections is consistent with \citet{zhou_etal20b}, who find $f_{\rm sat}\sim0.13$--0.16, depending on redshift.

\citet{zhou_etal20b} obtain best-fit parameters for analytic forms of central and satellite LRG HODs (Eqs.\ \ref{eq:hod_cen} and \ref{eq:hod_sat}), which will necessarily correspond to 100\% of massive halos containing a central LRG due to the functional form of Eq.\ \ref{eq:hod_cen} (this is also the case for analytic HOD fits to our mock LRG samples, shown in Table~\ref{tab:hod_params}). Enforcing 100\% halo occupation at the high-mass end by LRGs also suppresses accounting for potential contamination by lower-mass halos. 

As Figure~\ref{fig:hod_lrg} shows, our models do \emph{not} predict that central LRG halo occupation reaches unity, instead peaking at 82--96\% and falling in most cases to $\sim60$--77\% for the most massive halos (${\log(\mvir/ \msunh) \gtrsim 14,75}$), although this depends on both model band ($z$ or $W1$) and LRG selection (IR or optical; see Table~\ref{tab:hod_stats}).

The other study to which we can compare our results is \citet{hernandez-aguayo_etal21}, who use the \textsc{galform} semi-analytic galaxy formation model \citep{cole_etal00} and nine snapshots from the Planck-Millennium $N$-body simulation \citep{baugh_etal19} to study the galaxy--halo connection of ``DESI-like" LRGs at redshifts between 0.6 and 1.0. Their LRG HOD results for $z=0.64$ are overlaid with our $\zsim\sim0.63$ results in Figure~\ref{fig:hod_compare}.

\citet{hernandez-aguayo_etal21} predict central and satellite LRG abundances below our model predictions, although they note that \textsc{galform} underpredicts the abundance of LRGs across the entire redshift range they study, with the greatest discrepancy at $z\sim0.6$--0.7.
The shape of their central LRG HOD is notably similar to ours, consistent with a gradual transition to a peak central LRG halo occupation less than unity that decreases at the largest halo masses. Specifically, \citet{hernandez-aguayo_etal21} find that ${\langle N_{\rm cen}|\mvir\rangle}$ for ``DESI-like" LRGs reaches a maximum of $\sim70\%$ at ${\log(\mvir/\msunh) \sim 13.75}$, and is as low as $\sim40\%$ at ${\log(\mvir/\msunh) \sim 14.5}$.


\section{Summary and conclusion}\label{sec:summary}

This paper is organized as follows: Section \ref{sec:preliminaries} formalizes notation and summarizes concepts in measure theory, time-delay, occupation measures, and \ac{ODE} peak estimation. Section \ref{sec:peak_lp} defines an \ac{MV}-solution for free-terminal-time \ac{DDE} solutions to create a measure-\ac{LP} that upper-bounds \eqref{eq:peak_delay_traj}.
Section \ref{sec:moment} reviews the Moment-\ac{SOS} hierarchy and applies it to finding \acp{SDP} to upper-bound the peak-estimation measure \ac{LP}.
Section \ref{sec:examples} provides two examples of \ac{DDE} peak estimation. Section \ref{sec:conclusion} concludes the paper.


\section*{Acknowledgements}
\chapter*{Acknowledgement}
\addcontentsline{toc}{chapter}{Acknowledgement}
The authors thank Andrzej Kupsc, Sergey Barsuk, Olivier Callot and Wolfgang K{\"u}hn for their contribution on the CDR draft.
%The authors thank the international review committee XXX for their great effort in reading the CDR draft and providing valuable suggestions. 
The STCF working group thanks all 
the colleagues in the world-wide community for many profitable discussions
and expresses gratitude to the Hefei Comprehensive National Science Center for their strong support.  This work is supported by: international 
partnership program of the Chinese Academy of Sciences Grant No. 211134KYSB20200057.

\bibliography{main}

\end{document}
