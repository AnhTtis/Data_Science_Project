In this section we present the results of our $z$-band and $W1$-band models. We report the HOD parameters of mock LRGs and compare the predicted clustering to the data and to relevant studies in the literature.

\subsection{LRG clustering}\label{subsec:clust_lrg}

\begin{figure*}
  \centering
    \includegraphics[width=\linewidth]{figures/lrg_csrand.png}
    \caption{Predicted clustering of IR (top row) and optical (bottom row) mock LRGs compared to the clustering of the relevant LRG target samples from the data (black points in each panel).
    Dotted purple lines in each panel show mock LRGs from the default $z$-band model (monotonic correspondence between $r-z$ color and \zstarve at fixed luminosity), while solid purple lines show LRGs from the $z$-band model with randomized colors.
    Dashed orange lines in each panel show mock LRGs from the default $W1$-band model, and solid orange lines show LRGs from the $W1$-band model with randomized colors.
    Each column shows a different redshift bin.
    }
\label{fig:wp_lrg}
\end{figure*}


\begin{figure}
\centering
\includegraphics[width=\linewidth]{color-color_IR-opt_compare_z0p40-0p50.png}
\caption{IR color ($r-W1$) versus optical color ($r-z$) for IR (left panel) and optical (right panel) LRGs at $0.4<\zphot<0.5$.
}
\label{fig:color_color}
\end{figure}


\begin{figure}
  \centering
    \includegraphics[width=\linewidth]{figures/data_LRG-SED_redder.png}
    \includegraphics[width=\linewidth]{figures/data_LRG-SED_bluer.png}
  \caption{Distributions of ${r-z}$, ${g-r}$, and ${r-W1}$ colors of optical (shaded histograms) and IR (open histograms) LRGs from above (Box A; top three panels) and along (Box B; bottom three panels) the red sequence in ${r-W1}$ versus $M_{W1}$ color--magnitude space. The regions of color--magnitude space corresponding to Box A and Box B are shown in Figure~\ref{fig:cmd}. The mean, median, and skewness of each distribution, as well as the number of LRGs in each population, are given in Table~\ref{tab:data_sed}. Optical and IR LRGs from along the red sequence have similar distributions of each color, while optical LRGs above the red sequence have redder ${r-z}$ and ${g-r}$ and bluer ${r-W1}$ colors compared to IR LRGs in the same $W1$-band luminosity range.
  }
\label{fig:data_sed}
\end{figure}


\begin{deluxetable*}{ c }
  \tablecaption{Statistics of distributions of ${r-z}$, ${g-r}$, and ${r-W1}$ colors of optical and IR LRGs selected from above (Box A) and along (Box B) the red sequence in IR color--magnitude space (${r-W1}$ versus $M_{W1}$). The boxed regions A and B are shown for the $0.4 < \zphot < 0.5$ redshift bin in Figure~\ref{fig:cmd}, and the color distributions are shown in Figure~\ref{fig:data_sed}. The mean, median, and skewness of each distribution is shown for IR (optical) LRGs in bold (plain) text.
  \label{tab:data_sed}
  }
  \startdata
  \includegraphics[width=\linewidth, trim=0 2mm 0 0, clip]{tables/tab5.png} \\
  \enddata
\end{deluxetable*}

%\begin{deluxetable*}{ r r r r r r r r r r r r r r }
%\tablecaption{Statistics of distributions of ${r-z}$, ${g-r}$, and ${r-W1}$ colors of optical and IR LRGs selected from above (Box A) and along (Box B) the red sequence in IR color--magnitude space (${r-W1}$ versus $M_{W1}$). The boxed regions A and B are shown for the $0.4 < \zphot < 0.5$ redshift bin in Figure~\ref{fig:cmd}, and the color distributions are shown in Figure~\ref{fig:data_sed}. The mean, median, and skewness of each distribution is shown for IR (optical) LRGs in bold (plain) text.
%\label{tab:data_sed}
%}
%\tablehead{
%& \multicolumn{6}{l}{{\large Box A} (LRGs above red sequence)} & & \multicolumn{6}{l}{{\large Box B} (LRGs along red sequence)}
%}
%\startdata
%\multicolumn{14}{c}{ $0.4 < \zphot < 0.5$ } \\
%\vspace{-1ex} \\
%\hline
%\vspace{-1ex} \\
%& \multicolumn{2}{c}{$(r - z)$} & \multicolumn{2}{c}{~~$(g - r)$} & \multicolumn{2}{c}{~~$(r - W1)$} &
%& \multicolumn{2}{c}{$(r - z)$} & \multicolumn{2}{c}{~~$(g - r)$} & \multicolumn{2}{c}{~~$(r - W1)$} \\
%\vspace{-1ex} \\
%\hline
%\vspace{-1ex} \\
%mean & \bf{1.04} & 1.10 & ~~\bf{1.64} & 1.74 & ~~\bf{2.30} & 2.18 & ~~mean & \bf{0.98} & 1.01 & ~~\bf{1.72} & 1.78 & ~~\bf{1.74} & 1.68 \\
%median & \bf{1.06} & 1.10 & ~~\bf{1.65} & 1.74 & ~~\bf{2.25} & 2.14 & ~~median & \bf{0.98} & 1.00 & ~~\bf{1.78} & 1.80 & ~~\bf{1.73} & 1.67 \\
%skewness & \bf{-1.86} & 0.44 & ~~\bf{-0.36} & 1.15 & ~~\bf{1.44} & 1.30 & ~~skewness & \bf{-0.63} & 0.33 & ~~\bf{-1.20} & -0.73 & ~~\bf{0.12} & -0.02 \\
%\vspace{-1ex} \\
%\hline
%\vspace{-1ex} \\
%$\boldsymbol{N}_{\rm \mathbf{IR}}$	& \multicolumn{2}{l}{\bf{170,492}}	& \multicolumn{4}{l}{\multirow{2}{*}{$N_{\rm IR}/N_{\rm opt}=3.44$}} &
%$\boldsymbol{N}_{\rm \mathbf{IR}}$	& \multicolumn{2}{l}{\bf{313,311}}	& \multicolumn{4}{l}{\multirow{2}{*}{$N_{\rm IR}/N_{\rm opt}=0.60$}} \\
%$N_{\rm opt}$	& \multicolumn{2}{l}{\phantom{00}49,517} & & & & & $N_{\rm opt}$ & \multicolumn{2}{l}{\phantom{0}521,970} & & & & \\[-1ex]
%\cutinhead{$0.5 < \zphot < 0.6$}
%\vspace{-1ex} \\
%& \multicolumn{2}{c}{$(r - z)$} & \multicolumn{2}{c}{~~$(g - r)$} & \multicolumn{2}{c}{~~$(r - W1)$} &
%& \multicolumn{2}{c}{$(r - z)$} & \multicolumn{2}{c}{~~$(g - r)$} & \multicolumn{2}{c}{~~$(r - W1)$} \\
%\vspace{-1ex} \\
%\hline
%\vspace{-1ex} \\
%mean & \bf{1.18} & 1.29 & ~~\bf{1.71} & 1.87 & ~~\bf{2.64} & 2.53 & ~~mean & \bf{1.13} & 1.18 & ~~\bf{1.71} & 1.80 & ~~\bf{2.14} & 2.11 \\
%median & \bf{1.20} & 1.30 & ~~\bf{1.73} & 1.88 & ~~\bf{2.61} & 2.49 & ~~median & \bf{1.16} & 1.20 & ~~\bf{1.79} & 1.81 & ~~\bf{2.16} & 2.12 \\
%skewness & \bf{-0.99} & -1.17 & ~~\bf{-0.60} & 0.23 & ~~\bf{1.11} & 1.68 & ~~skewness & \bf{-0.83} & -0.82 & ~~\bf{-1.30} & -1.14 & ~~\bf{-0.46} & -0.41 \\
%\vspace{-1ex} \\
%\hline
%\vspace{-1ex} \\
%$\boldsymbol{N}_{\rm \mathbf{IR}}$	& \multicolumn{2}{l}{\bf{179,348}}	& \multicolumn{4}{l}{\multirow{2}{*}{$N_{\rm IR}/N_{\rm opt}=3.51$}} &
%$\boldsymbol{N}_{\rm \mathbf{IR}}$	& \multicolumn{2}{l}{\bf{441,691}}	& \multicolumn{4}{l}{\multirow{2}{*}{$N_{\rm IR}/N_{\rm opt}=0.67$}} \\
%$N_{\rm opt}$	& \multicolumn{2}{l}{\phantom{00}51,093} & & & & & $N_{\rm opt}$ & \multicolumn{2}{l}{\phantom{0}660,486} & & & & \\[-1ex]
%\cutinhead{$0.6 < \zphot < 0.7$}
%\vspace{-1ex} \\
%& \multicolumn{2}{c}{$(r - z)$} & \multicolumn{2}{c}{~~$(g - r)$} & \multicolumn{2}{c}{~~$(r - W1)$} &
%& \multicolumn{2}{c}{$(r - z)$} & \multicolumn{2}{c}{~~$(g - r)$} & \multicolumn{2}{c}{~~$(r - W1)$} \\
%\vspace{-1ex} \\
%\hline
%\vspace{-1ex} \\
%mean & \bf{1.39} & 1.47 & ~~\bf{1.77} & 1.90 & ~~\bf{2.96} & 2.90 & ~~mean & \bf{1.35} & 1.41 & ~~\bf{1.69} & 1.79 & ~~\bf{2.56} & 2.54 \\
%median & \bf{1.42} & 1.48 & ~~\bf{1.79} & 1.90 & ~~\bf{2.92} & 2.87 & ~~median & \bf{1.40} & 1.43 & ~~\bf{1.76} & 1.81 & ~~\bf{2.59} & 2.56 \\
%skewness & \bf{-1.31} & -0.84 & ~~\bf{-0.39} & 0.04 & ~~\bf{1.36} & 1.90 & ~~skewness & \bf{-1.43} & -1.23 & ~~\bf{-0.88} & -0.61 & ~~\bf{-0.85} & -0.74 \\
%\vspace{-1ex} \\
%\hline
%\vspace{-1ex} \\
%$\boldsymbol{N}_{\rm \mathbf{IR}}$	& \multicolumn{2}{l}{\bf{249,983}}	& \multicolumn{4}{l}{\multirow{2}{*}{$N_{\rm IR}/N_{\rm opt}=1.88$}} &
%$\boldsymbol{N}_{\rm \mathbf{IR}}$	& \multicolumn{2}{l}{\phantom{,}\bf{748,591}}	& \multicolumn{4}{l}{\multirow{2}{*}{$N_{\rm IR}/N_{\rm opt}=0.72$}} \\
%$N_{\rm opt}$	& \multicolumn{2}{l}{\phantom{0}132,656} & & & & & $N_{\rm opt}$ & \multicolumn{2}{l}{1,044,175} & & & & \\
%\enddata 
%\end{deluxetable*}


Our $z$-band and $W1$-band models yield magnitude-limited mock galaxy catalogs from which we select mock LRGs. From these results we predict the clustering of DESI LRG target samples.

Figure~\ref{fig:wp_lrg} shows the predicted clustering of IR and optical mock LRGs compared to the relevant DESI LRG target sample. In addition to the data (black points with error bars), each panel of Figure~\ref{fig:wp_lrg} shows the clustering of mock LRGs according to the $z$-band and $W1$-band models, with and without the addition of scatter to the model color--\zstarve relation (\S\ref{subsec:color_assign}).

As shown in the top row of Figure~\ref{fig:wp_lrg}, the $z$-band and $W1$-band models predict very similar clustering signals for IR LRGs. The predicted amplitude of both the one-halo and two-halo terms exceeds that of the data in the two lowest redshift bins, and is in better agreement in the highest redshift bin ($\zsim=0.628$). Across all redshift bins the addition of scatter to the color--\zstarve relation brings the predicted clustering amplitude of IR LRGs closer to the data, reducing the clustering amplitude of the one-halo term by $\sim15$--$20\%$, and the two-halo term by $\sim10\%$, depending on redshift bin. However, this reduction in the discrepancy between the model and data is a ceiling achieved only by adding such a high degree of scatter that color is uncorrelated with our proxy for halo assembly history\footnote{We also tested using proxies for IR and optical LRGs by selecting the $N$ brightest mock galaxies from the relevant magnitude-limited mock galaxy catalog to recreate the number densities of the IR and optical DESI LRG target samples, instead of applying the DESI LRG target selections to the magnitude-limited mocks as described in \S\ref{subsec:mock_lrg_select}. The result was a slightly larger predicted clustering amplitude compared to the ``random colors" models that do use the DESI LRG target selections, shown in Figure~\ref{fig:wp_lrg}.} (\zstarve).

The bottom row of Figure~\ref{fig:wp_lrg} shows that the $z$-band and $W1$-band models predict different clustering amplitudes for optical LRGs. The $z$-band model overpredicts the clustering amplitude of optical LRGs to a similar degree as the overprediction for IR LRGs. Assigning colors at random such that galaxy color is uncorrelated with halo assembly history again reduces this discrepancy of the one-halo (two-halo) term by $\sim20$--$25\%$ ($\sim10$--$15\%$).

On the other hand, the $W1$-band model prediction for the clustering of optical LRGs is in very good agreement with the data, especially the two-halo term (the one-halo term is actually slightly underpredicted). Additionally, adding \emph{any} amount of scatter to the color--\zstarve relation has no effect on the predicted clustering amplitude in this case.

The distributions of optical and IR LRGs in color--magnitude space are helpful for interpreting Figure~\ref{fig:wp_lrg}. As Figure~\ref{fig:cmd} shows, in optical space (${M_r-M_z}$ versus $M_z$), both optical and IR LRGs are largely confined to the red sequence, with the distribution extending below it toward bluer color, especially at higher redshift.

In contrast, in IR space (${M_r-M_{W1}}$ versus $M_{W1}$) the galaxy distribution does extend significantly above the red sequence. For example, in the ${0.4 < \zphot < 0.5}$ redshift bin, the red sequence is at ${M_r-M_{W1}\sim1.6}$, but the distribution extends to ${M_r-M_{W1} \gtrsim 3.0}$. IR LRGs occupy the entire region of excess ${M_r-M_{W1}}$ color across the full range of $M_{W1}$, while optical LRGs occupy only the part of this region that corresponds to more luminous $M_{W1}$, i.e., the optical LRG selection excludes a population of galaxies with very red ${M_r-M_{W1}}$ colors and moderately luminous $W1$-band luminosities that are included by the IR selection.

The top panels of Figure~\ref{fig:data_sed} explore the differences in the color distributions of optical and IR LRGs located above the red sequence (Box A in Figure~\ref{fig:cmd}) in ${r-W1}$ versus $M_{W1}$ color--magnitude space. The bottom three panels (Box B in Figure~\ref{fig:cmd}) show the color distributions of optical and IR LRGs from the same $M_{W1}$ range \emph{along} the red sequence. 

Along the infrared red sequence, the SEDs of optical and IR LRGs are quite similar. However, the two selections diverge above the red sequence. Table~\ref{tab:data_sed} lists the mean, median, and skewness of each color distribution, as well as the number of LRGs in each population.

Above the red sequence IR LRGs have bluer ${r-z}$ and ${g-r}$ colors than optical LRGs in the same region of color--magnitude space, while IR LRGs have redder ${r-W1}$ colors than their optical counterparts from the same color--magnitude region.
The same general trend is seen along the red sequence, but the color differences between the optical and IR LRG selections are smaller, i.e., along the red sequence IR LRGs also have bluer ${r-z}$ and ${g-r}$ colors than optical LRGs, but by only half as much as above the red sequence. Similarly, IR LRGs along the red sequence have redder ${r-W1}$ colors than their optical counterparts, but the color difference is two to three times smaller than for IR and optical LRGs above the red sequence.

Activity from active galactic nuclei (AGN) may artificially inflate the observed IR ($W1$-band) luminosities and artificially redden the $r-W1$ colors of some galaxies that pass the IR LRG selection \citep[e.g.,][]{webster_etal95, georgakakis_etal09, banerji_etal12, glikman_etal12, kim_im18, klindt_etal19, rivera_etal21}, relative to galaxies with comparable IR luminosities and ${r-W1}$ colors but no AGN activity.
The bluer ${r-z}$ colors of IR LRG targets cause these objects to be excluded by the optical LRG selection (e.g., panel (e) of Figure~\ref{fig:not-lrg_outliers}).
Our IR ($W1$-band) model would assign mock IR LRGs with artificially red $r-W1$ colors and high IR luminosities due to AGN activity to halos with higher bias than if their $r-W1$ colors and $W1$-band magnitudes were not influenced by AGN activity.
This could explain why our $W1$-band model overpredicts the clustering amplitude of IR LRGs relative to optical LRGs, as shown in Figure~\ref{fig:wp_lrg}.

The influence of AGN activity may also explain the lack of correlation between IR and optical color when assigning colors to mock galaxies based solely on a proxy for halo age (here \zstarve). This modeling assumption attributes galaxy color entirely to halo mass accretion history, and does not account for how baryonic effects such as AGN activity may contribute to galaxy color. In other words, optical color is not necessarily a reliable proxy for IR color. This is also illustrated by Figure~\ref{fig:color_color}, which shows optical (${r-z}$) versus IR (${r-W1}$) color for both IR and optical LRGs in the $0.4 < \zphot < 0.5$ redshift bin. The optical and IR colors of optical LRGs are more closely correlated than for IR LRGs, but in both cases there is considerable scatter in $r-W1$ at fixed $r-z$.
\citet{xu_etal22} also find a lack of correlation between halo assembly properties and galaxy color using a semi-analytic model.

\subsection{The LRG--halo connection}\label{subsec:lrg-halo}

\begin{figure*}
\centering
    \includegraphics[width=\linewidth]{hod-lrg_cs-v2.png}
    \caption{Predicted HODs of IR (top row) and optical (bottom row) mock LRGs from both the $z$-band (heavier purple lines) and $W1$-band (heavier orange lines) models. Also shown in fainter purple and orange lines are HODs of the $z$-band and $W1$-band magnitude-limited mock galaxy catalogs from which mock LRGs are selected. Solid (dotted) lines show results for central (satellite) galaxies and LRGs.
    Each column shows a different redshift bin.
    The best-fit values of a standard five-parameter HOD model, as well as the satellite fraction, for each mock LRG sample are given in Table~\ref{tab:hod_params}.
    }
\label{fig:hod_lrg}
\end{figure*}


\begin{deluxetable*}{ c c c c c }
\tablecaption{Predicted halo occupation completeness for mock central LRGs by selection (IR or optical). Results for the $W1$-band ($z$-band) model are in bold (plain) text.
\label{tab:hod_stats}
}
\tablehead{
\colhead{\zsim} & \colhead{LRG selection} &
\colhead{peak $\langle N_{\rm cen} | \mvir \rangle$\tablenotemark{a}} &
\colhead{min \mvir of peak $\langle N_{\rm cen} | \mvir \rangle$\tablenotemark{b}} &
\colhead{$\langle N_{\rm cen} | (\mvir > 10^{14.75}\ \msunh) \rangle$\tablenotemark{c}}
}
\startdata
\multirow{2}{*}{$0.425$} & IR & {\bf 0.92}~~0.91 & {\bf 14.45}~~14.85 & {\bf 0.74}~~0.70 \\
	& optical & {\bf 0.85}~~0.90 & {\bf 14.35}~~14.55 & {\bf 0.62}~~0.71 \\
\vspace{-1ex} \\
\hline
\vspace{-1ex} \\
\multirow{2}{*}{$0.523$} & IR & {\bf 0.91}~~0.87 & {\bf 14.45}~~14.65 & {\bf 0.77}~~0.61 \\
	& optical & {\bf 0.85}~~0.84 & {\bf 14.75}~~14.65 & {\bf 0.69}~~0.64 \\
\vspace{-1ex} \\
\hline
\vspace{-1ex} \\
\multirow{2}{*}{$0.628$} & IR & {\bf 0.96}~~0.82 & {\bf 14.85}~~14.55 & {\bf 0.96}~~0.31 \\
	& optical & {\bf 0.92}~~0.84 & {\bf 14.95}~~14.55 & {\bf 0.92}~~0.38 \\
\enddata
\tablenotetext{a}{Highest central occupation fraction reached across all halo \mvir.}
\tablenotetext{b}{Minimum halo \mvir at which the highest central occupation fraction is achieved.}
\tablenotetext{c}{Mean central occupation fraction at the largest halo masses ($\mvir \gtrsim 10^{14.75}\ \msunh$), which is universally less or equal to than the peak central occupation fraction.}
\end{deluxetable*}


We compute the mean central and satellite halo occupation statistics as a function of halo mass directly from the mock galaxy catalogs and halo abundances from the MDPL2 simulation. The results are shown in Figure~\ref{fig:hod_lrg}.
Each panel of Figure~\ref{fig:hod_lrg} shows ${\langle N | \mvir \rangle}$ for the full magnitude-limited mock galaxy catalog in gray, while optical and IR LRGs are shown in purple and orange, respectively.

The key conclusions of Figure~\ref{fig:hod_lrg} are summarized in Table~\ref{tab:hod_stats}, which gives the peak value of ${\langle N_{\rm cen} \rangle}$ for optical and IR LRGs predicted by both the $z$-band and $W1$-band models, as well as the mean value of ${\langle N_{\rm cen} \rangle}$ for the most massive halos, which is universally less than the corresponding peak value.

The $z$-band model does not predict significant differences between the populations selected by the IR and optical LRG selection functions. According to this model the central LRG halo occupation fraction peaks at $\sim90\%$ at $z\sim0.43$, and drops to $\sim70\%$ at the largest halo masses, while by $z\sim0.63$ $\langle N_{\rm cen}\rangle$ peaks at 82--84\% and falls to $\sim35\%$ for the most massive halos, regardless of LRG selection function.

In contrast, the $W1$-band model \emph{does} predict a different LRG--halo relationship for IR versus optical LRGs. In all redshift bins this model has ${\langle N_{\rm cen} \rangle}$ for IR LRGs peaking at around 92\% and remaining at $\gtrsim75\%$ for the most massive halos, while ${\langle N_{\rm cen} \rangle}$ for optical LRGs peaks at 85--92\%, and falls at the largest halo masses as low as 62\% ($z\sim0.43$) to 69\% ($z\sim0.52$), although it remains at 92\% for the highest redshift bin ($z\sim0.63$).

\begin{deluxetable}{ l r }
\tablecaption{Prior ranges for HOD fit parameters (Eqs.~\ref{eq:hod_cen} and \ref{eq:hod_sat}).
\label{tab:hod_params_priors}
}
\tablehead{
\colhead{Parameter} & \colhead{Prior interval}
}
\startdata
$\log(M_{\rm min})$ & $(11.0,\ 14.0)$ \\
$\sigma_{\log M}$ & $(0.001,\ 1.5)$ \\
$\alpha$ & $(0.0,\ 2.0)$ \\
$\log(M_0)$ & $(11.0,\ 14.0)$ \\
$\log(M_1)$ & $(11.5,\ 15.5)$ \\
\enddata
\end{deluxetable}


\begin{deluxetable*}{ c c c c c c c c }
\tablecaption{Best-fit values of standard five-parameter HOD model (Eqs.~\ref{eq:hod_cen} and \ref{eq:hod_sat}), and the satellite fraction, $f_{\rm sat}$, of IR and optical mock LRG samples. Results for the $W1$-band ($z$-band) model are in bold (plain) text.
\label{tab:hod_params}
}
\tablehead{
\colhead{\zsim} & \colhead{LRG selection} &
\colhead{$\log(M_{\rm min}/\msunh)$} &
\colhead{$\sigma_{\log M}$} & \colhead{$\alpha$} &
\colhead{$\log(M_0/\msunh)$} &
\colhead{$\log(M_1/\msunh)$} &
\colhead{$f_{\rm sat}$}
}
\startdata
\multirow{2}{*}{0.425} & IR & {\bf 13.28}~~13.35 & {\bf 0.93}~~0.95 & {\bf 0.98}~~1.00 & {\bf 12.93}~~12.87 & {\bf 14.02}~~14.07 & {\bf 14.8}~~14.5 \\
 & optical & {\bf 13.32}~~13.26 & {\bf 1.05}~~0.93 & {\bf 0.95}~~0.98 & {\bf 12.91}~~12.89 & {\bf 14.00}~~14.00 & {\bf 14.9}~~14.8 \\
\vspace{-1ex} \\
\hline
\vspace{-1ex} \\
\multirow{2}{*}{0.523} & IR & {\bf 13.37}~~13.52 & {\bf 0.99}~~1.14 & {\bf 0.94}~~0.91 & {\bf 12.94}~~12.98 & {\bf 14.05}~~14.09 & {\bf 14.4}~~14.5 \\
 & optical & {\bf 13.45}~~13.44 & {\bf 1.15}~~1.11 & {\bf 0.95}~~0.96 & {\bf 12.84}~~12.85 & {\bf 14.05}~~14.05 & {\bf 14.7}~~14.6 \\
\vspace{-1ex} \\
\hline
\vspace{-1ex} \\
\multirow{2}{*}{0.628} & IR & {\bf 13.26}~~13.40 & {\bf 1.02}~~1.18 & {\bf 0.95}~~0.95 & {\bf 12.88}~~12.82 & {\bf 13.95}~~13.99 & {\bf 14.1}~~14.4 \\
 & optical & {\bf 13.30}~~13.31 & {\bf 1.19}~~1.16 & {\bf 0.96}~~0.94 & {\bf 12.78}~~12.82 & {\bf 13.93}~~13.93 & {\bf 14.4}~~14.7 \\
\enddata 
\end{deluxetable*}


We fit a standard five-parameter HOD model to each optical and IR mock LRG sample Table~\ref{tab:hod_params}.
For ease of comparison with their results we use the same functional form as \citet{zhou_etal20b}, who derive HOD parameters from clustering measurements of DESI LRG targets using photometric redshifts in bins that approximately match the redshift bins we use. This parameterization is described in detail in \citet{zheng_etal05} and \citet{zheng_etal07}.
Briefly, the probability $\langle N_{\rm cen} \rangle$ that a halo of virial mass \mvir hosts a central galaxy is given by
%
\begin{equation}\label{eq:hod_cen}
  \langle N_{\rm cen} | \mvir \rangle = \frac{1}{2} \! \left( 1 + {\rm erf}{\left[\frac{\log(\mvir)-\log(M_{\rm min})}{\sigma_{\log M}} \right]} \right),
\end{equation}
%
\noindent where the parameter $\log(M_{\rm min})$ is the minimum halo mass for hosting a central galaxy, and the parameter $\sigma_{\log M}$ defines the steepness of the transition of $\langle N_{\rm cen} \rangle$ from $\sim0$ at low \mvir to $\sim1$ at high \mvir.

The mean number of satellite galaxies, $\langle N_{\rm sat} \rangle$, hosted by a halo of virial mass \mvir is approximated by a power law given by 
%
\begin{equation}\label{eq:hod_sat}
  \langle N_{\rm sat}|\mvir\rangle = {\left(\frac{\mvir-M_0}{M_1} \right)}^{\alpha},
\end{equation}
%
\noindent where $M_0$, $M_1$, and $\alpha$ are the remaining free parameters of the HOD fit. The HOD fits of \citet{zhou_etal20b} incorporate a sixth nuisance parameter to account for photometric redshift uncertainty, which in this work is addressed by the magnitude-dependent line-of-sight scatter parameter, \sigmalos (Eq.\ \ref{eq:sigmalos}).
Table~\ref{tab:hod_params_priors} gives the prior interval for each HOD parameter. The best-fit HOD parameters and satellite fraction of each mock LRG sample are given in Table~\ref{tab:hod_params}.


\subsection{Comparison with other ``DESI-like" LRG studies}\label{subsec:lit_compare}

\begin{figure*}
\centering
  \includegraphics[width=0.7\linewidth]{hod-compare-all.png}
  \caption{Comparison of LRG HOD parameters from this work at $z\sim0.63$ (orange lines) with the analytic HOD fits of \citet{zhou_etal20b} (green lines) and semi-analytic model of \citet{hernandez-aguayo_etal21} (gray lines) in comparable redshift bins.
  Solid, dotted, and dash-dotted lines show results respectively for central LRGs, satellite LRGs, and the combination of both.
  }
  \label{fig:hod_compare}
\end{figure*}

There are two main studies of the LRG--halo connection of ``DESI-like" LRGs suitable for comparison with our results. \citet{zhou_etal20b} (who provide the photometric redshifts used here) measure the clustering of LRGs at ${0.41 < \zphot < 0.93}$ selected from DECaLS DR7 photometry using the optical target selection (Eq.\ \ref{eq:lrg_opt}), and fit a standard five-parameter analytic HOD model in five redshift bins. 

\citet{zhou_etal20b} find little to no evolution of HOD parameters for LRGs across ${0.4 \lesssim z \lesssim 0.8}$. Our results are consistent with a lack of HOD parameter evolution across the subset of this redshift range that we model, although some of our predicted parameter values themselves differ from those of \citet{zhou_etal20b}, particularly $\sigma_{\log M}$. \citet{zhou_etal20b} find a very steep transition from a ${\langle N{\rm cen} \rangle \sim0}$ to 1 ($\sigma_{\log M}\sim0$--0.28; see their Figure 13), while our models predict a more gradual transition, with ${\sigma_{\log M}\sim1}$.

Figure~\ref{fig:hod_compare} compares our predicted LRG halo occupation statistics at $z\sim0.63$ with the best-fit HOD parameterization of \citet{zhou_etal20b} in the relevant redshift bin from their study ($0.61 < \zphot < 0.72$). Of the redshift bins \citet{zhou_etal20b} use that overlap with this work, this is the bin in which they find the greatest deviation of $\langle N{\rm cen} \rangle$ from a step function, although we also select this bin for Figure~\ref{fig:hod_compare} to enable comparison with an additional study, discussed later in this section.

Our HOD predictions for satellite LRGs agree with those of \citet{zhou_etal20b} at halo masses below ${\log(\mvir/ \msunh) \sim 14}$, and our overall satellite fractions of $f_{\rm sat}\sim0.14$--0.15 across both model bands and LRG selections is consistent with \citet{zhou_etal20b}, who find $f_{\rm sat}\sim0.13$--0.16, depending on redshift.

\citet{zhou_etal20b} obtain best-fit parameters for analytic forms of central and satellite LRG HODs (Eqs.\ \ref{eq:hod_cen} and \ref{eq:hod_sat}), which will necessarily correspond to 100\% of massive halos containing a central LRG due to the functional form of Eq.\ \ref{eq:hod_cen} (this is also the case for analytic HOD fits to our mock LRG samples, shown in Table~\ref{tab:hod_params}). Enforcing 100\% halo occupation at the high-mass end by LRGs also suppresses accounting for potential contamination by lower-mass halos. 

As Figure~\ref{fig:hod_lrg} shows, our models do \emph{not} predict that central LRG halo occupation reaches unity, instead peaking at 82--96\% and falling in most cases to $\sim60$--77\% for the most massive halos (${\log(\mvir/ \msunh) \gtrsim 14,75}$), although this depends on both model band ($z$ or $W1$) and LRG selection (IR or optical; see Table~\ref{tab:hod_stats}).

The other study to which we can compare our results is \citet{hernandez-aguayo_etal21}, who use the \textsc{galform} semi-analytic galaxy formation model \citep{cole_etal00} and nine snapshots from the Planck-Millennium $N$-body simulation \citep{baugh_etal19} to study the galaxy--halo connection of ``DESI-like" LRGs at redshifts between 0.6 and 1.0. Their LRG HOD results for $z=0.64$ are overlaid with our $\zsim\sim0.63$ results in Figure~\ref{fig:hod_compare}.

\citet{hernandez-aguayo_etal21} predict central and satellite LRG abundances below our model predictions, although they note that \textsc{galform} underpredicts the abundance of LRGs across the entire redshift range they study, with the greatest discrepancy at $z\sim0.6$--0.7.
The shape of their central LRG HOD is notably similar to ours, consistent with a gradual transition to a peak central LRG halo occupation less than unity that decreases at the largest halo masses. Specifically, \citet{hernandez-aguayo_etal21} find that ${\langle N_{\rm cen}|\mvir\rangle}$ for ``DESI-like" LRGs reaches a maximum of $\sim70\%$ at ${\log(\mvir/\msunh) \sim 13.75}$, and is as low as $\sim40\%$ at ${\log(\mvir/\msunh) \sim 14.5}$.
