The Dark Energy Spectroscopic Instrument \citep[DESI;][]{desi_collab16a, desi_collab16b} survey is a spectroscopic galaxy redshift survey of unprecedented scale that will classify tens of millions of galaxies in four target classes over $\sim14,000$ square degrees at the lowest redshifts of this program. DESI's bright galaxy survey \citep[BGS;][]{hahn_etal22} extends to $z\sim0.4$, while the the luminous red galaxy \citep[LRG;][]{zhou_etal22} sample will reach to $z\sim1$ and cover 20 times the volume of the BGS\footnote{DESI will observe emission line galaxies (ELGs) to $z\sim1.6$, and quasars to $z\sim3.5$.}.

The DESI target samples are optimized for precision measurements of cosmological parameters. However, DESI also offers novel opportunities to study galaxy evolution and the high-mass end of the stellar-to-halo mass relation (SHMR), provided that sample selection effects are well understood.

Unlike the magnitude-limited BGS sample of relatively nearby galaxies, the DESI LRG sample is selected with a comparatively complex set of magnitude and color cuts, creating incompleteness that may depend on any combination of galaxy color, stellar mass, and redshift. However, the LRG sample covers a volume 20 times larger than that of the BGS sample, and will contain a much higher number density of spectroscopic redshifts at ${0.4 < z < 1}$ than any previous spectroscopic galaxy redshift survey, making it a good sample for statistical studies with negligible sample and cosmic variance uncertainties. The rarity and associated low number density of massive galaxies $(\sim2\times10^{-5}\ {\rm Mpc}^{-3}$ for $\logm > 11.5)$ means that such large volumes are essential for obtaining sample sizes large enough for statistically significant measurements of the high-mass end of the galaxy stellar mass function (SMF) and the SHMR of LRGs.

Existing studies of the galaxy--halo relationship for ``DESI-like" LRG samples include halo occupation distribution (HOD) modeling \citep[e.g.,][]{seljak00, berlind_weinberg02, bullock_etal02, berlind_etal03, zheng_etal07, zheng_weinberg07} in \citet{zhou_etal20b}, and semi-analytic modeling in \citet{hernandez-aguayo_etal21} with {\sc Galform} \citep{cole_etal00}, which predicts absolute magnitudes with dust attenuation.

In its simplest form, the HOD model provides a statistical description for how galaxies occupy dark matter halos solely as a function of halo mass.
\citet{zhou_etal20b} fit the projected clustering of ``DESI-like" LRGs measured in five redshift bins from $0.4 < z < 0.9$ with a five-parameter HOD model plus a sixth nuisance parameter to account for photometric redshift uncertainties.
They find similar HOD parameters at $0.4 < z < 0.8$, and statistically significant differences in model parameters for only the highest redshift bin $(0.8 < z < 0.9)$.
While \citet{zhou_etal20b} demonstrate that clustering measurements using photometric redshifts are sufficient to constrain the HOD parameters of ``DESI-like" LRGs, the standard HOD framework they use does not accommodate galaxy populations with halo occupation fractions that do not reach unity for the most massive halos.

\citet{hernandez-aguayo_etal21} apply the {\sc Galform} semi-analytic model (SAM) to the Planck-Millennium cosmological $N$-body simulation \citep{baugh_etal19}.
They then convert predicted absolute magnitudes to apparent magnitudes at various redshift snapshots, enabling the DESI LRG target selection to be applied directly to {\sc Galform} mock galaxy catalogs to select mock LRG samples.
\citet{hernandez-aguayo_etal21} find that the DESI LRG selection criteria exclude a small but important fraction of the most massive galaxies ($\logm > 11.15$).
Consequently, their model predicts that the halo occupation fraction of LRGs does \emph{not} reach unity for the most massive halos, and actually drops with increasing mass, indicative of a non-trivial LRG--halo connection that is not modeled well with a standard HOD.
By comparing the HOD and subhalo mass functions of stellar mass-selected mock galaxies against those of mock LRG samples, \citet{hernandez-aguayo_etal21} show that the DESI LRG selection cuts likely affect the selection of subhalos populated by LRGs, i.e., (sub)halo mass is by itself insufficient to determine whether a subhalo hosts a LRG.

Subhalo abundance matching \citep[SHAM; e.g.,][]{vale_ostriker04, conroy_etal06, behroozi_etal10, reddick_etal13} is an empirical technique for assigning galaxies to dark matter halos in numerical $N$-body simulations by assuming a correlation between a galaxy property---usually luminosity or stellar mass---and halo property such as mass or circular velocity.
In the simplest application of SHAM, mock galaxy catalogs are constructed to reproduce the number density and luminosity or stellar mass function (SMF) of a target dataset with a single free parameter to allow scatter in the galaxy property--halo property correlation.

Extensions to the SHAM framework to incorporate dependencies between additional galaxy and halo properties are broadly referred to as conditional abundance matching \citep[CAM; e.g.,][]{hearin_etal14, zentner_etal14}.
In addition to a primary galaxy property--halo property correlation, CAM models assume a correlation between secondary galaxy and halo properties at a fixed value of the primary property.
Age distribution matching \citep{hearin_watson13} is a form of CAM that equates galaxy color or a similar property with a proxy for the age of dark matter halos.
Unlike the standard HOD framework, in which the statistical relationship between galaxies and dark matter halos is solely a function of stellar and halo mass, CAM can naturally accommodate galaxy assembly bias, the dependence of galaxy properties (besides stellar mass) on the mass accretion history of their host halos \citep[e.g.,][and references therein]{zentner_etal14, wechsler_tinker18}.

The SHAM framework has been used to study the dependence of sample completeness on redshift and stellar mass for LRGs from the Baryon Oscillation Spectroscopic Survey \citep[BOSS;][]{dawson_etal13}, which obtained spectroscopic redshifts of 1.5 million galaxies with $\logm > 11$ to $z\sim0.7$.
BOSS contains two color- and magnitude-selected samples of massive galaxies \citep[$\logm > 11$;][]{reid_etal16}:\ the LOWZ sample of LRGs at $0.15 < z < 0.43$, and the approximately stellar mass-limited constant mass (CMASS) sample, which includes galaxies of all colors at $0.43 < z < 0.8$.

\citet{saito_etal16} use SHAM to construct $z\sim0.5$ mock galaxy catalogs for the BOSS CMASS sample with and without added assembly bias effects.
They use the Stripe 82 Massive Galaxy Catalog \citep[S82-MGC;][]{bundy_etal15} to replicate the total galaxy SMF above $\logm > 10.5$ over $0.43 < z < 0.7$ and assign galaxies to halos in the MultiDark simulation \citep{riebe_etal13}.
\citet{saito_etal16} find that assembly bias does factor into the galaxy--halo connection for high-mass galaxies, i.e., the SHMR for these galaxies has some dependence on galaxy color, and should not be inferred from the clustering signal without any consideration of color.

\citet{yu_etal22} model LRGs from BOSS and eBOSS \citep{dawson_etal16} at ${0.2 < z < 1.0}$ with a SHAM framework that includes two additional free parameters to account for redshift uncertainty and sample incompleteness.

SHAM has also been used to model ``DESI-like" samples at low redshift.
\citet{safonova_etal21} use SHAM and age distribution matching to create $z\sim0.1$ mock galaxy catalogs representative of the DESI BGS sample with $r$-band luminosities and ${g-r}$ colors.
The low redshift range of the BGS sample allows them to utilize spectroscopic redshifts from the Sloan Digital Sky Survey \citep[SDSS;][]{york_etal00} and Galaxy and Mass Assembly \citep[GAMA;][]{loveday_etal12} project.

In this work, we use SHAM and age distribution matching to create magnitude-limited, mock galaxy catalogs at multiple redshifts within the redshift range of the DESI LRG sample.
Two distinct target selection algorithms were considered for the DESI LRG sample:\ an optical selection that uses $r-z$ color, and an infrared selection that uses $r-W1$\footnote{$W1$ is the 3.4 micron band of the Wide-field Infrared Survey Explorer \citep[WISE;][]{wright_etal10}} color.
We select mock LRG samples from our magnitude-limited mocks based on both the optical and IR DESI target selections that match the number density and two-dimensional color--magnitude space distribution of each DESI LRG target sample. We then predict the clustering signal and halo occupation statistics of these samples as a function of redshift.

Our method is novel approach to modeling DESI LRGs that complements existing SAM and HOD models by utilizing the full photometric samples that are the basis of DESI LRG target selection.
We offset the precision of training data lost to photometric redshift errors by driving down cosmic variance uncertainties with complete photometric samples from an unprecedented survey volume.
Finally, this work offers a comparative study of the samples selected by the optical and IR selection algorithms considered for DESI LRGs.

The structure of this paper is as follows. In \S\ref{sec:data_sims} describe the cosmological simulation and photometric galaxy samples used in this work.
\S\ref{sec:model} describes our modeling procedure and the two-point statistics used to constrain our models.
In \S\ref{sec:predict}, we present the predicted properties of LRG samples,
and in \S\ref{sec:summary} we summarize the conclusions of this work.
Where applicable, we assume the cosmological parameter values of \citet{planck_collab16}:\ $h=0.6777$ and $\Omega_{\rm m}=0.307115$, under the assumption of a flat, $\Lambda$CDM cosmological model.