\begin{figure}
\centering
  \setlength{\tabcolsep}{0pt}
    \begin{tabular}{ c c }
      \includegraphics[width=0.5\linewidth, trim=0 3cm 0 2mm, clip]{cmd_Mz-rz_z0p40-0p50_optical-lrg.png} &
      \includegraphics[width=0.5\linewidth, trim=0 2.9cm 0 0, clip]{cmd_MW1-rW1_z0p40-0p50_optical-lrg.png} \\
      \includegraphics[width=0.5\linewidth, trim=0 0 0 2mm, clip]{cmd_Mz-rz_z0p40-0p50_IR-lrg.png} &
      \includegraphics[width=0.5\linewidth, trim=0 0 0 0, clip]{cmd_MW1-rW1_z0p40-0p50_IR-lrg_AB.png} \\
    \end{tabular}
  \caption{
  Color--magnitude diagrams of the parent galaxy samples and DESI LRG target samples (overlaid white contours) in the {$0.4 < \zphot < 0.5$} redshift bin. The left column shows optical ($K$-corrected $r-z$ color versus $M_z$ magnitude), while the right column shown IR ($K$-corrected $r-W1$ color versus $M_{W1}$ magnitude).
  The top (bottom) row shows the distribution of optical (IR) LRGs in each color--magnitude space.
  The boxed regions labeled ``A" and ``B" are referenced in \S\ref{subsec:clust_lrg} below.
 }
  \label{fig:cmd}
\end{figure}
