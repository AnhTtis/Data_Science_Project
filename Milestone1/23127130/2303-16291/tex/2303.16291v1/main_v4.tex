\documentclass[aps,prb,twocolumn,longbibliography,groupedaddress,showpacs,floatfix,superscriptaddress]{revtex4-1}
%%\documentclass[aps,prb,twocolumn,longbibliography,groupedaddress,showpacs,floatfix,superscriptaddress]{revtex4-1}
\usepackage[colorlinks=true,citecolor=blue,linkcolor=blue,breaklinks=true]{hyperref}
\usepackage{graphicx}  
\usepackage{dcolumn}   
\usepackage[english]{babel}
\usepackage{bm}        
%\usepackage{amssymb}  
\usepackage{amsmath, amsthm, amssymb}
\usepackage{bbold} 
%\usepackage[fleqn]{amsmath}
%\usepackage{amsthm}
\usepackage{mathrsfs}
\usepackage{array}
\usepackage[usenames]{color}
%\usepackage{color}
\usepackage{soul} 
\usepackage{ulem} 
\usepackage{float}

%% avoids incorrect hyphenation, added Nov/08 by SSR
%% \hyphenation{ALPGEN}
%% \hyphenation{EVTGEN}
%% \hyphenation{PYTHIA}
%% RTS version:
\tolerance=1
\emergencystretch=\maxdimen
\hyphenpenalty=10000
\hbadness=10000
\newcommand{\calH}{\mathcal{H}}
\newcommand{\iv}{\mathbf{i}}
\newcommand{\jv}{\mathbf{j}}
%\DeclareUnicodeCharacter{2212}{-}

\newcommand{\rcol} {\textcolor{red}}
\newcommand{\bcol} {\textcolor{blue}}

\begin{document}
\title%{Thermopower properties of interacting electrons \\ \textcolor{blue}
{Effects of lattice geometry on thermopower properties of the repulsive Hubbard model}
\author{Willdauany C. de Freitas Silva}
\affiliation{Instituto de F\'isica, Universidade Federal do Rio de Janeiro, Rio de Janeiro, RJ 21941-972, Brazil}
\author{Maykon V. Araujo}
\affiliation{Departamento de F\'isica, Universidade Federal do Piau\'i, 64049-550 Teresina PI, Brazil}
\author{Sayantan Roy}
\affiliation{Department of Physics, The Ohio State University, Columbus OH 43210, USA}
\author{Abhisek Samanta}
\affiliation{Department of Physics, The Ohio State University, Columbus OH 43210, USA}
\author{Natanael de C. Costa}
\affiliation{Instituto de F\'isica, Universidade Federal do Rio de Janeiro, Rio de Janeiro, RJ 21941-972, Brazil}
\author{Nandini Trivedi}
\affiliation{Department of Physics, The Ohio State University, Columbus OH 43210, USA}
\author{Thereza Paiva}
\affiliation{Instituto de F\'isica, Universidade Federal do Rio de Janeiro, Rio de Janeiro, RJ 21941-972, Brazil}

%\date{\today}

\begin{abstract}
%Thermoelectric effects are responsible for the direct conversion of thermal energy into electrical energy (and vice versa) and have a wide range of technological applications. The thermoelectric performance of a material is determined by its Seeback coefficient ($S$), or the themopower, which may be enhanced, among other sources, by strong electronic correlations. 
%Thermoelectric effects are responsible for the direct conversion of thermal energy into electrical energy (and vice versa) and have a wide range of technological applications. The thermoelectric performance of a material is determined by its Seeback coefficient ($S$), or the themopower, which may be enhanced, among other sources, by strong electronic correlations. 
We obtain the Seebeck coefficient or thermopower $S$, which determines the conversion efficiency from thermal to electrical energy, for the two-dimensional Hubbard model on different geometries (square, triangular, and honeycomb lattices) for different electronic densities and interaction strengths. Using Determinantal Quantum Monte Carlo (DQMC) we find the following key results: (a) the bi-partiteness of the lattice affects the doping dependence of $S$; (b) strong electronic correlations can greatly enhance $S$ and produce non-trivial sign changes as a function of doping especially in the vicinity of the Mott insulating phase; (c) $S(T)$ near half filling can show non-monotonic behavior as a function of temperature.  
%\rcol{In particular, for the square lattice, there is a critical density at which $S$ is almost temperature independent and beyond that, the temperature dependence of $S$ changes it's shape.} 
We emphasize the role of strong interaction effects in engineering better devices for energy storage and applications, as captured by our calculations of the power factor $PF=S^2 \sigma$ where $\sigma$ is the dc conductivity. 

%suggesting it is related to the transfer of spectral weight due to the opening of the Mott gap. \bcol{We also find nontrivial sign changes in the Seebeck coefficient near half filling and for strong correlations, which signal a change in the carrier type from that expected in a Fermi liquid description, at temperatures precluding any spontaneous symmetry breaking.} %altogether with an breaking of the Fermi liquid behavior, which suggests that the nature of this phenomena would be, indeed, a reconstruction of the Fermi surface. 
\end{abstract}

\pacs{
71.10.Fd, %Lattice fermion models (Hubbard model, etc.)
71.30.+h, %Metal-insulator transitions and other electronic transitions
02.70.Uu  % Applications of Monte Carlo methods
}

\maketitle


%\noindent
\section{Introduction}
%The efficiency of energy production and conversion is a theme that receive scientific attention due to the necessity of a susteinable world. In such scenario, the thermoelectric materials appears as a possible route, once it could convert into electrical energy, the hitherto unused excess heat that is necessary to sustain many industrial or daily processes. However, to a real possibility of large-scale applications, a figure of merit(which is a dimensionless quantity) \textit{ZT}$\geqslant3$ is needed, what is still a problem nowadays once much of the classical thermoelectrics, the heavily doped semiconductors or semimetals, presents a \textit{ZT} close to 1.\cite{Wiss_2011} Moreover, such materials present some technical problems, such as toxicity, thermal instability, scarcity of elements such as Te\cite{H_bert_2015}. In order to deal with these kind of problems, the development of novel thermoelectric applications follows two routes: The first one is, or the optimization of the already known compounds through tecniques like band-structure engineering \cite{Guangqian_2015,Joseph_2008} and nanostructuration \cite{Hicks_1993}, or the seek for new classes of compounds which exhibit uncommon properties(such as badwidth renormalization or Kondo Ressonances)\cite{Wiss_2010}, normally related to strong electron correlations. The last one is through phonons\cite{Sharma_2019}, decoupling of electronic and thermal transport in order to increase the figure of merit by decreasing the phononic part of thermal conductivity, once these two quantities are inversely proportional.
Over the past decades, a great deal of interest has been given to increasing the efficiency of electrical devices. %and it has become one of the most important concerns for the next decades.
As a possible route to this end, the thermoelectric materials may play a crucial role, once they exhibit induced voltage in presence of a temperature gradient, whose magnitude is quantified by the Seebeck coefficient, the thermoelectric Power Factor, and, eventually, the Figure of Merit\,\cite{H_bert_2015}.
However, there are many technical issues that make the development of efficient thermoelectric materials a challenge, e.g., the toxicity of the compounds, or their thermal instability.
There are few ways to overcome these problems: (i) optimizing the already known compounds through band-structure engineering and nanostructuration techniques\,\cite{Guangqian_2015,Joseph_2008,Hicks_1993}, or (ii) seeking new classes of compounds which exhibit unconventional properties %(such as, Fermi surface reconstruction),
usually related to strong electron-electron interactions\,\cite{Wiss_2010}.
In view of the increasing number of novel correlated compounds, controlling and manipulating %such many-body states 
geometry and correlations to enhance the thermopower properties is an open issue.

%\textcolor{red}{[NCC: Remider to me... describe in further details how electronic correlations may affect the thermoelectric properties.]}

Despite intense experimental efforts, further theoretical investigations are required, in particular to understand interacting electronic compounds, such as Na$_{x}$CoO$_{2}$ or FeSb$_{2}$, which exhibit unusual large thermopower response\,\cite{Wiss_2010,Wiss_2011,Tomczak_2018}.
In the former, the combination of quasi two-dimensional character with band topology and strong electronic correlations make this material an interesting playground to examine thermoelectricity. Once the charge carriers are confined to the hexagonal layers of Co atoms, a disordered distribution of Na ions above and below it can induce a charge imbalance. Using local-density approximation and dynamical mean-field theory, Held et al\,\cite{Wiss_2010} showed that  disorder combined with the pudding-mold band structure and strong correlations enforce the electron-hole imbalance and enhances the thermopower by $200 \% $ with respect to the non-interacting case. 


The thermopower of superconducting cuprates has been experimentally studied\,\cite{high-tc-1,high-tc-2,high-tc-3,high-tc-4}, with different compounds displaying very similar behavior with a sign change of the Seebeck coefficient near the maximum critical temperature\,\cite{high-tc-1}. This nearly universal behavior has been the subject of theoretical interest, being ascribed to a possible underlying critical point\,\cite{high-tc-teo-3}, to the presence of a van Hove singularity\,\cite{high-tc-teo-1}, and has been recently observed for the Hubbard Model with next-nearest neighbor hopping\,\cite{Deveraux}.



%\bcol{The effects of strong electronic correlations has drastic effects in the physical observables of materials that can be described by hamitlonians of kind Eq \eqref{Eq:Hamil}. Transport and spectroscopy are two such candidate observables, that can characterize emergent phases due to interactions. Recent measurement of Hall coefficient in cuprate superconductors \cite{badoux2016change} have already hinted at deviations from Fermi Liquid like behavior, while analytical\cite{scheurer2018topological} and numerical calculations(DMFT)\cite{wu2018pseudogap} have established development of topological order and fractionalization in these kind of systems. Measurement of thermopowers in FQHE states can also be used as a possible probe for non abelian quasiparticle statistics.\cite{yang2009thermopower} }


In view of these stimulating results, we explore how  electron-electron interactions and geometry  affect the Seebeck coefficient, and the thermoelectric Power Factor.
To this end, we use unbiased Quantum Monte Carlo simulations to study the single band repulsive Hubbard model. We analyze the thermoelectric and electrical transport properties in the long wavelength DC limit in two-dimensions, in the square, triangular and honeycomb lattices. Our study finds a strong influence of particle hole symmetry of the many body spectrum and DOS on the behavior of the Seebeck coefficient with respect to doping. We show non trivial sign changes of the Seebeck coefficient that signal a deviation of the Fermi surface from the Luttinger count and a subsequent anomalous change of the type of carriers below and above half filling. The sign change is also accompanied by a significant increase of the Seebeck coefficient near half filling with respect to the non-interacting and weakly interacting case. We show that despite using a simplified thermodynamic formula for the Seebeck coefficient that is independent of dynamical quantities, we are able to capture the effects of strong correlation.


The paper is organized as follows: In Section \ref{model_meth} we discuss the Hubbard model, an introduction to the Seebeck coefficient, Kelvin formula and the auxiliary field QMC method used to solve it. In Section \ref{Entropy} we discuss how to calculate the entropy and present our results for this quantity. Section \ref{DOS_conductivity} shows the Local Density of States and conductivity results. In Section \ref{Sec_Seebeck}  we discuss the Seebeck coefficient and  in Section \ref{Sec_Power_factor} the Power Factor. Finally, in section \ref{conclu} we summarize our findings.
%%%%%%%%%%%%%%%%%%%%%%%%%%%%%%%%%%%%%%%%%%%%%%%%%%%%%%%%%%%%%%%%%%

\section{Model and methods}
\label{model_meth}
The repulsive Fermi Hubbard model describes electrons on a lattice with an onsite repulsive interaction, with the Hamiltonian
%%
\begin{align}\label{Eq:Hamil}
\nonumber \mathcal{H} = & -t\sum_{\substack{\langle \textbf{i},\textbf{j} \rangle},\sigma} \big( c_{\textbf{i}, \sigma}^{\dagger}c_{\textbf{j}, \sigma}+ {\rm H.c.} \big) - \mu \sum_{\substack{\textbf{i}}, \sigma} n_{\textbf{i},\sigma}
\\  & + U   \sum_{\substack{\textbf{i}}} (n_{\textbf{i},\uparrow}-1/2)(n_{\textbf{i},\downarrow}-1/2),
\end{align}
where the sums run over sites of a given two-dimensional lattice, with $\langle \mathbf{i}, \mathbf{j} \rangle$ denoting nearest-neighbor sites.
Here, we use the second quantization formalism, with $c^{\dagger}_{\mathbf{i}, \sigma}$ ($c^{\phantom{\dagger}}_{\mathbf{i}, \sigma}$) being creation (annihilation) operators of electrons on a given site $\mathbf{i}$, and spin $\sigma$, while $n_{\mathbf{i},\sigma} \equiv c^{\dagger}_{\mathbf{i}, \sigma} c_{\mathbf{i}, \sigma}$ are number operators.
The first two terms on the right-hand side of Eq.\,\eqref{Eq:Hamil} correspond to the hopping of electrons, and the chemical potential $\mu$, respectively, with the latter determining the filling of the lattice.
The third term describes the local repulsive interaction between electrons, with interaction strength \textit{U}; the factor of $1/2$  is introduced to ensure invariance of the hamiltonian under particle-hole transformations on bipartite lattices. This implies that for the bipartite lattices we consider here (square and honeycomb lattices), $\mu=0$ sets half-filling for all temperatures.



Our central quantities of interest are the transport coefficients, and their behavior with respect to doping and strength of interactions. The transport coefficients are defined through the following relations,
\begin{align}
    \vec{j} &= \tensor{L}^{11}\vec{E}+\tensor{L}^{12}(-\vec{\nabla} T) \nonumber \\ 
    \vec{j}^q &= \tensor{L}^{21}\vec{E}+\tensor{L}^{22}(-\vec{\nabla}T)
    \label{Eq:conductivity}
\end{align}
where $\vec{j}$ and $\vec{j}^q$ are the electrical and thermal currents, and the $\tensor{L}$s are rank 2 tensors defining conductivities of the system. The tensors $\tensor{L}^{11}$ and $\tensor{L}^{22}$ are the electrical and thermal conductivities, and $\tensor{L}^{12}$($\tensor{L}^{21}$) are the thermoelectrical(electrothermal) conductivities. The thermopower or Seebeck coefficient is defined as
\begin{align}
    S &= \frac{\big(\tensor{L}^{12}\big)_{xx}}{\big(\tensor{L}^{11}\big)_{xx}}= \frac{1}{T}\frac{\big(\tensor{L}^{21}\big)_{xx}}{\big(\tensor{L}^{11}\big)_{xx}},
    \label{Eq:Seebeck}
\end{align}
where the second equality is due to Onsager's reciprocity relations\cite{onsager1931reciprocal}.
Using linear response theory with respect to electrical field and temperature, the Seebeck coefficient in the Kubo formalism can also be written as

\begin{align}
    S(q_x,\omega) =  \frac{1}{T}\frac{\chi_{\hat{\rho}(q_x),\hat{K}(-q_x)}(\omega)}{\chi_{\hat{\rho}(q_x)\hat{\rho}(-q_x)}(\omega)} 
    \label{Eq:Seebeck_Kubo}
\end{align}
where 
\begin{align}
    \chi_{\hat{\rho}(q)\hat{K}(-q)}(\omega) &\!=\! \lim_{\eta \rightarrow 0}\sum_{n,m}(f_n\!-\!f_m)\frac{\langle n|\hat{\rho}(q)|m\rangle \langle m|\hat{K}(-q)|n\rangle}{\omega+i\eta+\epsilon_n-\epsilon_m} \nonumber \\
        \chi_{\hat{\rho}(q)\hat{\rho}(-q)}(\omega) &= \lim_{\eta \rightarrow 0}\sum_{n,m}(f_n\!-\!f_m)\frac{\langle n|\hat{\rho}(q)|m\rangle \langle m|\hat{\rho}(-q)|n\rangle}{\omega+i\eta+\epsilon_n-\epsilon_m} \nonumber \\
        \label{Eq:Susceptibility}
\end{align}
define the electrothermal and electrical conductivities, respectively. Evaluation of a Kubo-like formula (Eq \eqref{Eq:Seebeck_Kubo}) is not easy for interacting systems in the thermodynamic limit, although it is the most insightful. Alternate formulas like the Mott formula\cite{jonson1980mott}, Heikes-Mott\cite{heikes1961}, high frequency Seebeck\cite{shastry2006sum,shastry2008electrothermal,zdroj2007} and Kelvin formula\cite{peterson2010kelvin,shastry2008electrothermal} exist but are limited by their applicability to specific scenarios (weakly correlated metal at low temperatures for the Mott formula, high-temperature limit for the Heikes-Mott formula, and measurement of transport at high frequencies compared to characteristic energy scale for high-frequency Seebeck formula). The Kelvin formula was proposed by Lord Kelvin to provide reciprocity between Seebeck and Peltier coefficients, %although its justification of the reciprocity 
and is calculated in the slow limit ($q_x \rightarrow 0,\omega \rightarrow 0$); it can be derived by taking the slow limit of Eq (\ref{Eq:Seebeck_Kubo}) \cite{peterson2010kelvin}. %survives in the opposite limit in which Onsager proved the reciprocity relations later ($\omega \rightarrow 0,q_x \rightarrow 0$)\cite{onsager1931reciprocal}. 
The Kelvin formula for the Seebeck coefficient is
\begin{align}
\label{eq:seebeck}
\nonumber S_{\rm Kelvin} &= \left. - \frac{1}{e} {\frac{\partial \mu }{\partial T}}\right|_{V,n} = \left.  \frac{1}{e} {\frac{\partial s}{\partial n}}\right|_{T,V}~,
\end{align}
%\begin{align}
%\label{eq:seebeck}
%\nonumber S_{\rm Kelvin} = \lim_{\omega \rightarrow 0, q \rightarrow 0} S(q,\omega) &= \left. - \frac{1}{e} {\frac{\partial \mu }{\partial T}}\right|_{V,n} = \left.  \frac{1}{e} {\frac{\partial s}{\partial n}}\right|_{T,V}~.
%\end{align}
where the second equality follows from Maxwell's relations. Although expressed in terms of thermodynamic quantities, which are sufficient to capture the effects from the many-body density of states, it misses kinematic factors like contributions from velocities at the Fermi surface and relaxation times. Nonetheless, the effects of strong correlations on the low-frequency transport behavior are taken into account, as it retains the $\omega<U$ approximation, which the high frequency formula, $S^{*}$ misses. The justification and benchmark of using the Kelvin formula for strongly interacting systems like the Hubbard model and fractionalized systems like $\nu = 5/2$ FQHE states have already been established\cite{arsenault2013entropy,peterson2010kelvin}.

We investigate the thermodynamics and transport properties of Eq.\,\eqref{Eq:Hamil} on three different lattices: square, triangular and honeycomb. In particular, we examine the behavior of the
entropy, conductivity, Local Density of States (LDOS), Seebeck
coefficient, and Power Factor as functions of lattice filling, for
different values of interaction strengths. To this end, we perform unbiased determinant quantum Monte Carlo (DQMC) simulations~\cite{Blankenbecler81,Hirsch83,Hirsch85,White89},  a state-of-the-art numerical method that maps a many-particle interacting fermionic system into a single-particle (quadratic form) one, with the aid of bosonic auxiliary fields.
More details about the methodology may be found in, e.g., Refs.\,\onlinecite{gubernatis16,becca17,Santos03}, and references therein.
Our DQMC simulations are performed for finite-sized systems (with 100, 144 and 162 sites for the square, triangular, and honeycomb lattices, respectively), and for interaction strengths $U/t = 0$, 2, 4, 6, 8, and 10; i.e., from non-interacting to strong coupling regimes.
Throughout this work, unless otherwise indicated, we consider $T/t=0.5$, which corresponds to an energy scale low enough to observe the crossover towards an insulating phase at half-filling~\cite{Kim20,Simkovic20,Lenihan21}. Hereafter, we define the lattice constant as unity, and the hopping integral $t$ as the energy scale.

\section{Entropy}
\label{Entropy}
\begin{figure}[t]
\includegraphics[scale=0.40]{Lay_m_n_b2.pdf}
\caption{(Color online) Density $n$ as a function of chemical potential $\mu$ for the (a) square, (b) triangular, and (c) honeycomb lattices, for fixed $T/t=0.5$, and different interaction strengths $U/t$. Here, and in all subsequent figures, when not shown, error bars are smaller than symbol size. A Mott plateau is formed above a critical interaction strength which depends on lattice geometry. Note the width of this plateau is a measure of the charge gap or ``Mottness".}
\label{fig:density}
\end{figure}

\begin{figure}[t]
\includegraphics[scale=0.40]{Lay_Entro_n_b2.pdf}
\caption{(Color online) Entropy $s$ as a function of the electronic density $n$ for the (a) square, (b) triangular, and (c) honeycomb lattices, for fixed $T/t=0.5$, and different interaction strengths $U/t$. Note that the entropy is very different among the three lattices in the noninteracting limit. In presence of sufficiently large interactions, a local
minimum appears at the half-filling for all three lattices; However, it becomes qualitatively similar for the square and honeycomb lattices, while it is different for the non-bipartite triangular lattice.}
\label{fig:entropy}
\end{figure}

We start our analysis with the electronic density as a function of chemical potential shown in Figure \ref{fig:density} for $T/t=0.5$ and the different lattices studied: (a) square, (b) triangular, and (c) honeycomb. Particle-hole symmetry is evident for the square and honeycomb lattices, where $n(\mu)=2-n(-\mu)$. In all geometries, correlations lead to the formation of a Mott plateau (insulating phase) around half-filling, although the critical interaction strength for the onset of ``Mottness" %and the filling dependence of the Mott insulating behavior 
is strongly influenced by the lattice geometry. 

As the Seebeck coefficient is obtained from the entropy, we now turn to discuss its behavior for these geometries.
The entropy can be obtained from the electronic density, $n(\mu)$,
%and whose procedure is outlined as follows. We gather data on a fine grid of chemical potentials at two different temperatures, so as to extract the numerical derivative $\partial n / \partial T$. From that we integrate over the chemical potential $\mu $
by integrating it over the chemical potential $\mu $,
%\begin{align}\label{eqn:entropy}
    %s(\mu, T) &= \int_{-\infty}^\mu d\mu \frac{\partial n}{\partial T}_\mu 
%\end{align}
\begin{equation}
    s(\mu, T) = \int_{-\infty}^\mu d\mu \left.  \frac{\partial n}{\partial T}\right|_\mu 
\label{eqn:entropy}
\end{equation}
Figure \ref{fig:entropy}\,(a) displays $s(\mu,T)$ as a function of $n$ for the square lattice. One may notice that, at $T/t=0.5$, the results for $U/t = 2$ (red circles symbols) and $U/t = 4$ (blue diamond symbols) exhibit very similar behavior to the non-interacting one (solid black curve), once  the temperature is high enough to destroy the correlation effects for such small interaction strengths. However as
 $U/t$ increases, e.g.~$U/t=6$, 8, and 10 the entropy presents a local minimum at half-filling, being drastically reduced as $T/t \to 0$, due to the Mott gap formation in the ground state\,\cite{Bonca03,Khatami11,Simkovic20,Mikelsons09,Paiva01}. The increased entropy for the metallic state in the vicinity of half-filling in the presence of interactions has been observed for the cubic lattice and is relevant for cold  fermionic atoms trapped in optical lattices, where the metallic region of the atomic cloud is used to absorb entropy and allow a central Mott region at a higher entropy per particle \cite{Paiva11}. 

Unlike what is observed for the square lattice, the entropy behavior on the triangular geometry only exhibits such a local minimum at $U/t=10$, the largest interaction strength considered, as presented in Fig.\,\ref{fig:entropy}\,(b).
This suggests an absence of a Mott gap for weak and intermediate interaction strengths. In fact, at $U/t\gtrsim 8$, a small dip starts to form around $n=1$, in line with the expectation for a metal-to-insulator transition occurring for $U/t \approx 7-8$ \cite{Yoshioka,Shirikawa,Schauss}. The lack of particle-hole symmetry in the triangular lattice  shown in Fig.\,\ref{fig:density}\,(c) is clearly also present in the entropy.

The entropy for the honeycomb lattice is more subtle.
Similar to the square and triangular lattices, a local minimum appears at half-filling in presence of strong interaction; however, different from what is seen in the two previous cases, here $s(n, T)$ displays a suppression at half-filling even for the non-interacting case.
%In fact, $s(n=1, T) \to 0$ as the temperature is reduced.
This behavior is understood by recalling that the honeycomb lattice has a vanishing  DOS  at half-filling, with van Hove singularities below and above it.
That is, the results of Fig.\,\ref{fig:entropy} confirm our expectations that the entropy provides hints about the DOS, irrespective of the interaction strengths.
These odd features of the honeycomb lattice lead to strong changes for the Seebeck coefficient at $U=0$, as we shall see in Section \ref{Sec_Seebeck}, and may obscure some of the the electronic correlation effects.
%%%%%%%%%%%%%%%%%%%%%%%%%%%%%%%%%%%%%%%%%%%%%%%%%%%%%%%%%%%%%%%%%%%%%%%%%%%%%%%%%%%%%%%%%%%%%%

\section{Local Density of states and conductivity}
\label{DOS_conductivity}

As discussed before, Fig.\,\ref{fig:density} signals the occurrence of a Mott insulating state due to the presence of a plateau in the density as a function of chemical potential, driven by correlations. The subtleties in the entropy require an analysis of  the suppression of spectral weight for different densities. In order to avoid numerical analytical continuations,  we examine the LDOS only at the Fermi level ($\omega=0$), which is obtained through\,\cite{Trivedi95}
\begin{equation}
\label{eq:densos}
N(\omega = 0) \approx \frac{\beta}{\pi} G(|\mathbf{i}-\mathbf{j}| = 0,\; \tau = \beta/2),
\end{equation} 
where $G(\mathbf{r}, \tau)$ is the real space and imaginary time Green's function calculated at $\mathbf{r} \equiv \mathbf{i}-\mathbf{j} = 0$, and imaginary time $\tau = \beta/2$ (hereafter, $\beta \equiv 1/(k_BT)$ is the inverse of temperature, and $k_B$ the Boltzmann constant).
\begin{figure}[t]
\includegraphics[scale=0.40]{DOS_beta2.pdf}
\caption{(Color online) Local density of states (LDOS) at the Fermi level at $T/t=0.5$ as a function of density for the (a) square, (b) triangular, and (c) honeycomb lattices. Note the similarity of the doping dependence of LDOS to that of the entropy (Fig.~\ref{fig:entropy}) in the non-interacting limit. When correlations increase, LDOS develops a dip near half-filling, leading to maxima at intermediate densities which depend on the lattice geometry. The LDOS also becomes qualitatively similar to the square and honeycomb lattices in the strongly interacting limit, but is quite different from the triangular lattice which, in contrast, is not bipartite and has an asymmetric particle-hole many-body spectrum.}
\label{fig:DOS}
\end{figure}

Fig.~\ref{fig:DOS} shows $N(\omega=0)$ for the (a) square, (b) triangular, and (c) honeycomb lattices. 
 A common feature for all geometries is that the non-interacting case is an upper bound for the LDOS, with data close to the empty and completely filled systems showing a negligible  dependence on the interaction strength. %\bcol{At half-filling, correlations lead to the imaginary part of the self-energy at $\omega = 0$ having a maximum, which implies a large scattering rate and hence a short quasiparticle lifetime, which leads to the depletion of quasiparticle weight and LDOS}. I left this out as we are not calculating the self-energy. 
For the temperature shown, $T/t=0.5$, we can see that for the square lattice and $U/t=2$ the dip at half-filling has not developed yet. As correlations increase, the dip starts to form, leading to maxima at $n \approx 0.6$ and  $n \approx 1.4$. The non-interacting ground state for the honeycomb lattice is known to be a semi-metal, with a vanishing LDOS at half-filling at $T/t \rightarrow 0$. Figure \ref{fig:DOS}(c) shows a dip in the DOS at $T/t=0.5$ already for $U/t=0$, in line with entropy data. Similar to what is seen for the square lattice, maxima develop as correlations increase, but the positions are moved to $n = 0.5$ and $n=1.5$. The triangular lattice LDOS shows a distinct behavior, as can be seen in figure \ref{fig:DOS}(b), with a broad peak for $U/t=0$ located at $n \approx 1.6$. The effects of correlations are only relevant for $n \gtrsim 0.6$. For the larger values of $U/t$ considered, a small broad peak is present at $n \simeq 0.75$ and a higher one at $n \simeq 1.6$, with a dip at half-filling signaling the Mott state for large $U$.

To further investigate the transport properties, we now turn to the longitudinal dc conductivity,
\begin{equation}\label{eq:sigma_dc}
\sigma_{dc} = \frac{\beta^2}{\pi} \Lambda_{xx}(\mathbf{q=0}, \tau = \beta/2),
\end{equation}
in which
\begin{equation}
\Lambda_{xx}(\mathbf{q}, \tau ) = 
\langle j_{x}(\mathbf{q}, \tau) j_{x}(-\mathbf{q}, 0)  \rangle
\end{equation}
is the current-current correlation function with $ j_{x}(\mathbf{q}, \tau) $ being the Fourier transform of the unequal-time current operator
\begin{equation}
j_x(\mathbf{i},\tau)=\mathrm{e}^{\tau\mathcal{H}}
  \left[
        it\sum_\sigma
            \left(c_{\mathbf{i}+\mathbf{x},\sigma}^\dagger 
                  c_{\mathbf{i},\sigma}^{\phantom{\dagger}}
                  - 
                  c_{\mathbf{i},\sigma}^\dagger  
                  c_{\mathbf{i}+\mathbf{x},\sigma}^{\phantom{\dagger}}
            \right)
  \right]
\mathrm{e}^{-\tau\mathcal{H}}.
\label{jx}
\end{equation}
Here  we also avoid analytical continuation, see, e.g., Refs.~\onlinecite{Trivedi96,Denteneer99,Mondaini12}.



%\begin{equation}
%j_x( {\bf i}, \tau)= e^{H \tau} \big( it \sum_\sigma  c_{{\textbf i}+\hat{x} \sigma}^{\dagger} c_{{\textbf{i}} \sigma}  - 
% c_{{\textbf{i} \sigma}}^{\dagger}c_{\textbf{i} + \hat{x}}   \big)e^{-H \tau}    
%\end{equation}

%From that one can obtain the current-current operator:
%\begin{equation}
%\Lambda_{xx}({\bf q}=0; i \omega_m) = {1 \over N} \int_0^\beta d\tau j_x( {\bf i},\tau) j_x(0,0) e^{i {\bf q} \cdot {\bf i}} e^{-i \omega_n  \tau}
%\end{equation}
%where $\omega_n = 2n \pi / \beta$.

%\begin{equation}
%    Im \Lambda \simeq \omega \sigma_{dc}
%\end{equation}


%\begin{equation}
%\Lambda_{xx}({\bf q}=0; \tau = \beta/2)= \pi \sigma_{dc}/\beta^2    
%\end{equation}

%\begin{equation}
%    \Lambda_{xx} ({\bf q}, \tau)= \int_{- \infty}^{ \infty} {d \omega \over \pi} exp(-\omega \tau) Im\Lambda_{xx}({\bf q},\omega)
%\end{equation}

%As the interaction is turned on, the spin, charge and momentum of the fermions corresponding to the occupied states remain unchanged, while their dynamical properties, such as their mass, magnetic moment etc. are renormalized to new values


\begin{figure}[t]
\includegraphics[scale=0.40]{Lay_Condut_beta2.pdf}
\caption{(Color online) Longitudinal DC conductivity at $T/t=0.5$ as a function of density for the (a) square, (b) triangular and (c) honeycomb lattices. The dip at half-filling signifies the opening of the Mott gap, and the presence (absence) of bipartiteness is reflected in the particle-hole symmetric (asymmetric) behavior of the conductivity. The locations of the peaks of $N(\omega = 0)$ are very different from that of $\sigma_{DC}$ in the strongly interacting regime. Note that the dip in LDOS in Fig. \ref{fig:DOS} is not accompanied by a dip in $\sigma_{dc}$ in the honeycomb lattice.}
\label{fig:conductivity}
\end{figure}


Similar to what is seen for the LDOS, correlations reduce the conductivity, with the non-interacting conductivity as an upper limit for all the geometries studied as shown in Fig.\,\ref{fig:conductivity}.
Once again, correlations are shown to be irrelevant to transport properties for densities near the completely empty or filled bands, while its effects increase as half-filling is approached, with $\sigma_{dc} \to 0$ as $U/t$ increases.  It is interesting to note that, for the honeycomb lattice, the dip in the non-interacting DOS is not accompanied by a dip in the conductivity [shown in Figs.\,\ref{fig:DOS}(c) and \ref{fig:conductivity}(c), respectively].
%\rcol{Note to Nandini - Should we keep this sentence? Because} \bcol{Although for $U = 0$, the DOS for honeycomb lattice at half filling goes to 0, $\sigma_{dc}$ does not. To see this, we can consider the interband conductivity, given by
%\begin{align}
%    \sigma_{ij}(q,\omega) \!&=\!\! \sum_{k,n\neq m}\!\frac{f[\epsilon_{n}(k+q)]\!-\!f[\epsilon_{m}(k)]}{\epsilon_{n}(k+q)\!-\!\epsilon_{m}(k)}v^{i}(k)_{nm}v^{j}_{mn}(k+q) \\ \nonumber
%    & \times\delta(\omega\!-\!\epsilon_{n}(k+q)\!+\!\epsilon_{m}(k)\!)
%\end{align}
%where the Fermi occupation functions $f[\epsilon_{n}(k)$] are calculated at the energies of the band $n$ at the momenta $k$. It turns out that due to the linear nature of the bands near the Dirac point (for the honeycomb), which is relevant at half-filling, the velocity matrix elements are constants, and the ratio of the numerator and denominator is constant as $\omega \rightarrow 0$. This makes the $\sigma_{dc}$ constant at half-filling, although the DOS vanishes.}
For large values of $U/t$, the maxima for the conductivity are not at the same densities as the ones for the LDOS; for the square lattice, the conductivity has maxima at $n=0.5$ and $n=1.5$. For the honeycomb lattice, the maxima are at $n=0.6$ and $1.4$ and for the triangular lattice, the maxima are at $n=0.45$ and $n=1.45$.  For the triangular lattice, one can see that both the $U/t=0$ peak and the higher intensity peak for the LDOS, which are above half-filling, move to densities below half-filling for the conductivity.  


%%%%%%%%%%%%%%%%%%%%%%%%%%%%%%%%%%%%%%%%%%%%%%%%%%%%%%%%%%%%%%%%%%%%%%%%%%%%%%%%%%%%%%%%%%%%%%%%%%%%
\section{Seebeck coefficient}
\label{Sec_Seebeck}
We now turn to the Seebeck coefficient, which is obtained from the entropy by using the Kelvin formula\,\cite{Peterson10,arsenault2013entropy},
%\begin{align}
%\label{eq:seebeck}
%\nonumber S_{\rm Kelvin} = \lim_{\omega \rightarrow 0 q \to 0} S(q,\omega) &= \left. - \frac{1}{e} {\frac{\partial \mu }{\partial T}}\right|_{V,n} = \left.  \frac{1}{e} {\frac{\partial s}{\partial n}}\right|_{T,V}~.
%\end{align}
\begin{align}
%\label{eq:seebeck}
S_{\rm Kelvin} =  \left.  \frac{1}{e} {\frac{\partial s}{\partial n}}\right|_{T,V}~.
\label{Eq_kelvin_seebeck}
\end{align}
Within this approach, Fig.\,\ref{fig:Seebeck} displays the behavior of $S_{\rm Kelvin}$ in units of $k_B/e^2$ as a function of the electronic density for the (a) square, (b) triangular, and (c) honeycomb lattices. 


At this point, it is worth recalling that the sign of the Seebeck coefficient is directly related to the type of carrier, being negative for holes and positive for electrons.  As a consequence of particle-hole symmetry for the square and honeycomb lattices, one has $S_{\rm Kelvin}(n)=-S_{\rm Kelvin}(2-n)$, leading to $S_{\rm Kelvin}(n=1)=0$. For all geometries examined, the effect of correlations is strongly dependent on density, being negligible for $n \lesssim 0.3$ and $n \gtrsim 1.7$, as previously observed for entropy, DOS and conductivity.

\begin{figure}[t]
\includegraphics[scale=0.40]{Lay_Seebeckxn_b2.pdf}
\caption{(Color online) Seebeck coefficient for different values of the interaction strength at $T/t=0.5$ as a function of density for the (a) square, (b) triangular, and (c) honeycomb lattices. For clarity, we have reduced the set of $U/t$ values compared to previous plots. In the non-interacting limit, the Seebeck coefficient changes sign at half-filling for bipartite lattices (square and honeycomb), but at a finite doping for triangular lattice. With strong interactions, there is an enhancement of the Seebeck coefficient near half-filling, as well as an anomalous sign changes away from half-filling, signaling a change in carrier type. Note that for the honeycomb lattice, there is a sign change in Seebeck coefficient away from half filling even at $U/t = 0$, due to presence of Van Hove singularities from the Bloch bands.}
\label{fig:Seebeck}
\end{figure}


%\textcolor{blue}
%{Observing the behavior of the Seebeck coefficient for the square lattice, 
 As expected, the Seebeck coefficient for the non-interacting square lattice presents only one sign change, at half-filling, as shown in Fig.\,\ref{fig:Seebeck}\,(a). %and reflects the change in carrier type, .
 However, notice that in the presence of strong correlations ($U/t \gtrsim 6$) there is an anomalous behavior, characterized by a change of sign for densities away from half-filling, also displayed in Fig.\,\ref{fig:Seeback-U}\,(a) for $n \approx 0.96$. %\rcol{We need to clarify if this density is for a fixed value of $U/t$, or happens at all $U/t$ above a threshold.} This density is the one shown in the plot, it is an example.
 In addition, there is a notable increase in the absolute value of Seebeck coefficient, compared to the non-interacting case. For instance, at $n=0.99$, $S_{\rm Kelvin} \approx 2.37 ~ k_B/e^2$, at $U/t = 10$, while for $U/t = 0$, $S_{\rm Kelvin} \approx -0.01 ~   k_B/e^2$. This steep increase can be explained by noting that the Seebeck coefficient, as defined in Eq.~\eqref{Eq:Seebeck}, is the ratio between the longitudinal thermoelectric and electrical conductivities. As we approach the Mott insulator at half-filling, electrons form local moments and electrical transport is strongly reduced, as shown by both conductivity and LDOS in the previous section. %Recall that Eq. \ref{Eq:Seebeck} for the Seebeck coefficient has the thermoelectrical current in the numerator and the electrical conductivity in the denominator; therefore peaks in the vicinity of the insulating state can be understood by the fast decreasing electrical conductivity close to half-filling.
 The fast-decreasing electrical conductivity in the vicinity of half-filling must be accompanied by a non-vanishing  thermoelectric current for the peaks to form.
We can understand this as follows: at half-filling and strong correlations, each site is singly occupied and local moments are completely formed, there is no electric or thermoelectric transport and the Seebeck coefficient is zero. As we move slightly away from half-filling, there is a background of local moments over which $p=1-n$ carriers lead to thermoelectric and electric currents. This reduced number of carriers is in line with the breaking of Luttinger count \cite{Osborne,Hanke,Sakai} that has been established for the Fermi Hubbard model  and is in agreement with the change in carrier density in Hall experiments for YBCO\,\cite{badoux2016change}.
 
 %while the DC electrical conductivity tends to 0 on approaching the half-filling point. Hence, there is a divergence of $S_{\text{Kelvin}}$ whenever the Mott insulating regime is approached.%\rcol{Note to Nandini - Do we need to show this rigorously by taking $\omega \rightarrow 0 $ limit of Eq \eqref{Eq:Seebeck_Kubo}?}
 
Due to absence of particle-hole symmetry in the triangular lattice, the sign change of $S_{\text{Kelvin}}$ in the non-interacting limit occurs away from half-filling,  at $n = 1.42$, as shown in Fig.\,\ref{fig:Seebeck}\,(b), as opposed to the square lattice, which has particle-hole symmetry, and exhibits the sign change symmetrically around half filling.
There is a range of densities, around  $0.5 \le n \le 0.9 $ where correlations lead to a small increase in modulus of the Seebeck coefficient.
%; for instance at $n=0.75$, $S_{\rm Kelvin}=- 0.48 ~ k_B/e^2$ for the non-interacting triangular lattice whereas it reaches $S_{\rm Kelvin}=- 0.91 ~ k_B/e^2$ for $U/t=10$. Contrary to what is seen in the square lattice
In contrast to the square lattice, the peak in the Seebeck coefficient for the triangular lattice %\rcol{Which peak are we referring to here?} 
%idue to strong interactions 
at $n=0.99$ is only present for very strong interactions. 
%does not represent an increase in thermopower for all $U/t$. 
For $U/t= 8$, $S_{\rm Kelvin} \approx -0.18 ~ k_B/e^2 $, while for $U/t = 0$, for this electronic density  $S_{\rm Kelvin} \approx -0.55 ~ k_B/e^2 $. %\rcol{Can we argue why the peak for $U/t = 8/0$ is so small? From previous sections, it seems like a Mott gap should open up at $U/t \sim 7-8$.} 
Strong correlations, around $U/t=10$, are needed to change the sign of the Seebeck coefficient at half-filling, as presented in Fig.\,\ref{fig:Seeback-U}\,(b).
Interestingly, above half-filling and below the sign-change point for $U/t=0$, the correlations are detrimental to the thermopower up to $U/t=8$, bringing the Seebeck coefficient closer to zero.% Once again, at $U/t=10$ a sign change is observed. 

\begin{figure}[t]
\includegraphics[scale=0.34]{Lay_b2_SxU_ns.pdf}
\caption{(Color online) Seebeck coefficient as a function of $U/t$, at a fixed $T/t=0.5$ for different densities in (a) square, (b) triangular and (c) honeycomb lattices. For densities close to half-filling, the Seebeck coefficient changes sign and enhances as $U/t$ is increased.}
\label{fig:Seeback-U}
\end{figure} 

Fig.\,\ref{fig:Seebeck}\,(c) shows the Seebeck coefficient for the honeycomb lattice.
For the non-interacting system, $S_{\rm Kelvin}$ changes sign at  $n = 0.6$, 1.0, and 1.4, therefore the charge of the carriers is negative for $n < 0.6$ and $1.0 < n < 1.4$, while it is positive for $0.6 < n < 1.0$ and $n> 1.4$.
Interestingly, these densities correspond to the peaks and dip positions for the conductivity, as seen in Fig.\,\ref{fig:conductivity}\,(c).
We recall that such changes in carriers come from the fact that the honeycomb lattice has two bands for the noninteracting case, so one may expect changes from electron to hole properties of the transport coefficients depending on the filling of each band. % The metal to Mott insulator quantum phase transition takes place $U/t= 3.8$, then below this value the electronic correlation is not significant. Note here: (TP) I have removed this argument as the increase in the square lattice is very similar and U_c=0.
However, correlations push the sign change to values closer to half-filling, going from $n=0.6$ and $1.4$ at $U/t=0$ to $n=0.85$ and $1.15$ at $U/t=8$. One thing to note is that although the behavior of the Seebeck coefficient is very different between the square and the honeycomb lattice in the noninteracting limit, adding interactions and transitioning to the Mott insulating state seems to wash these differences away. This can be understood by noting that in the noninteracting picture, the transport is determined by the Bloch bands which are different for square and honeycomb lattices. However, with strong interactions, Mott physics destroys the Bloch bands, opens up a Mott gap, and forms upper and lower Hubbard bands. While details of the single particle bands are erased, the information about the particle-hole symmetry imprinted in the many body spectrum is still retained. 
%) should we discuss this sentence?}. 
%Since the Kelvin formula captures the behavior of the many-body density of states from the Hubbard bands and reflects the particle-hole symmetric nature of the many-body spectrum of the square and honeycomb lattice, 
Hence the behavior of the Seebeck coefficient is qualitatively similar between the square and honeycomb lattice but different from the triangular lattice.
%, which lacks a particle-hole symmetric many-body spectrum and density of states.

Similar to the square lattice, there is a significant increase in the Seebeck coefficient for the honeycomb lattice close to half-filling at $U/t = 6$, 8 and 10, as shown for $n=0.96$ and $n=0.99$ in Fig.\,\ref{fig:Seeback-U}\,(c). For $U/t = 10$, for example, we have that for $n = 0.99$, $S_ {\rm Kelvin} \approx 2.30 ~ k_B/e^2$, while in the non-interacting case ($U/t = 0$) for this density value, $S_{\rm Kelvin} \approx 0.03 k_B/e^2$, an enhancement of two orders of magnitude.


% \begin{figure}[t]

%Figure \ref{fig:Seeback-U} shows the Seebeck coefficient as a function of interaction strength and selected densities for all three geometries. Particle-hole symmetry drives the Seebeck coefficient to zero for all $U/t$ for the square and honeycomb lattices. For the square lattice, shown in Fig. \ref{fig:Seeback-U}(a) one can clearly see that close to half-filling there is a strong dependence on $U/t$ and an interaction-driven change in carrier type for densities very close to half-filling as seen for $n=0.94$ and 0.99.


We also analyze the effects of temperature on the thermopower for the square lattice at $U/t=10$.
Figure \ref{fig:Seeback-U10}\,(a) displays $S_{\rm Kelvin}$ as a function of $n$ for different $T/t$, while Fig.\,\ref{fig:Seeback-U10}\,(b) shows $S_{\rm Kelvin}$ as a function of $T/t$ for different densities.
Curves for different temperatures nearly cross around $n=0.5$ and $n=1.5$ [Fig.\,\ref{fig:Seeback-U10}(a)], leading to almost horizontal lines for $n=0.53$ and $n=1.47$ in Fig.\,\ref{fig:Seeback-U10}(b).
For $n \lesssim 0.5$ and $n \gtrsim 1.5$ reducing the temperature is detrimental to the thermopower, as seen by a reduction of the modulus of the Seebeck coefficient for $n=0.2$.
For densities in the range $0.5 \lesssim n \lesssim 0.9$ and $1.1 \lesssim n \lesssim 1.5$, the behavior of $S_{\rm Kelvin}$ with temperature is non-monotonic, and can change sign with $T/t$,  as shown for $n=0.75$ in figure \ref{fig:Seeback-U10}(b).
Finally, the anomalous behavior in the vicinity of half-filling has a marked dependence on temperature, increasing the modulus of the Seebeck coefficient as $T/t$ is reduced. This fast increase of the thermopower with decreasing temperature close to half-filling has also been observed in the $t-J$ model\cite{peterson2010kelvin}, Hubbard model on an FCC lattice\cite{arsenault2013entropy} and $t-t^\prime-U$ Hubbard Model \cite{Deveraux}.

\begin{figure}[t]
\includegraphics[scale=0.40]{Lay_b2_U10_Sxt_ns.pdf}
\caption{(Color online) Seebeck coefficient for the square lattice at $U/t=10$ as (a) a function of density for the square lattice at different temperatures (b) a function of temperature for different densities. In panel (a), the Seebeck coefficient shows anomalous behavior at low temperatures, and approaches the free particle limit as $T/t$ is increased. In panel (b), at low densities ($n=0.2$), the Seebeck coefficient has the expected sign, and monotonically decreases with temperature. At a critical density $n \approx 0.5$ (also shown for $n \approx 1.5$ in the electron doped side), the Seebeck coefficient is almost temperature independent. 
%Beyond this density, the Seebeck coefficient starts to have a   (iii) For the all the temperatures considered here, Seebeck coefficient for $n \lessapprox 0.75$ has the sign expected from Fermi liquid theory, whereas the sign is anomalous for $0.75<n<1.0$, and has nonmonotonic behavior. 
In the anomalous region ($0.75 \lessapprox n \lessapprox 1.0$), peaks in the Seebeck coefficient start to move towards the smallest temperature considered here, as one moves towards the half-filling limit, where it becomes 0 for all temperatures.   }
\label{fig:Seeback-U10}
\end{figure}


% %\includegraphics[scale=0.33]{figures/Seebeck/S6_S0.pdf}
%  \caption{(Color online) Seebeck coefficient as a function of density.}
%  \label{fig:S~6}
% \end{figure}

%\begin{figure}[t]
%\includegraphics[scale=0.33]{figures/deltaS/delS lay.pdf}
% \caption{(Color online) Seebeck coefficient as a function of density.}
% \label{fig:S~6}
%\end{figure}


%%%%%%%%%%%%%%%%%%%%%%%%%%%%%%%%%%%%%%%%%%%%%%%%%%%%%%%%%%%%%%%%%%%%%%%%%%%%%%%%%%%%%%%%%%%%%%%%%%%
\section{Thermoelectric Power factor}
\label{Sec_Power_factor}
Proceeding, we now discuss the effects of correlations to the thermoelectric Power Factor (\textit{PF}), defined as 
\begin{equation}
    PF= S^2 \sigma,
\end{equation}
where the dc conductivity ($\sigma$) and the Seebeck coefficient (\textit{S}) were obtained in Sections~\ref{DOS_conductivity} and \,\ref{Sec_Seebeck}, respectively. Simultaneously increasing both the modulus of the Seebeck coefficient and the conductivity maximizes the Power Factor. Strategies to determine the Seebeck coefficient that leads to an optimum thermoelectric Power Factor have been sought theoretically for systems that can be described by the Boltzmann transport equation\,\cite{PF-1} and experimentally for  CZTS thin films\,\cite{PF-2} and La-doped SrTiO$_3$\,\cite{PF-3}
 thin films.

\begin{figure}[t]
\includegraphics[scale=0.35]{Lay_PF_S2sigma_b2.pdf}
\caption{(Color online) Thermoelectric Power factor as a function of density for (a) square, (b) triangular and (c) honeycomb lattices. In the free particle limit (almost empty and almost filled lattice), the Power factor exhibits a peak for all values of $U/t$. Interaction causes additional peaks to develop very close to the Mott insulating limit for large interaction strengths. Additional features for intermediate doping also appear that are strongly influenced by lattice geometry.}
\label{fig:PF}
\end{figure}


Fig.~\ref{fig:PF} shows the thermoelectric Power Factor as a function of density for fixed $T/t=0.5$, and for the three analyzed geometries. A common feature to all data is the presence of peaks close to the empty ($n \lesssim 0.1$) and completely filled lattices ($n \gtrsim 1.9$). Starting with completely empty (filled) lattices, as density increases (decreases) the conductivity increases, and the Seebeck coefficient modulus decreases, leading to peaks that are independent of interaction strength. %\bcol{(In this nearly empty (filled) regime, the competition between the conductivity  Seebeck coefficient near completely empty and filled bands) this statement is not clear}. 
As the conductivity goes to zero for Mott insulators, $PF \to 0$ at half-filling for a geometry-dependent value of $U/t$.  For the square lattice, the hump in $PF$ for $U/t=0$ around $n  \approx 0.6$ (1.4) turns into a peak at $n \approx 0.5$ (1.5) with the increase of correlations, the dominant contribution coming from the conductivity.
%(\bcol{the peek seems to be at 0.45, not 0.5}. The height of the intermediate densities peaks approaches that of the nearly empty (filled) bands ones as correlations are increased. On the other hand, the peaks close to half-filling are absent for the non-interacting system, and come from the sharp increase in the Seebeck coefficient for the interacting case 


The effect of correlations for the intermediate densities peak of the honeycomb lattice is more subtle, as can be seen in figure \ref{fig:PF}(c). The peaks for the non-interacting system at $n=0.85$  ($n=1.15$) decrease in intensity with correlations for $U/t=2$ and 4, and then a shoulder develops at lower (higher) densities for larger $U/t$. Close to half-filling once again correlations drive the Seebeck coefficient peak up which in turn leads to the $PF$ peaks.


For the triangular lattice one can see that the non-interacting system has a peak at $n=0.95$. Correlations move the peak to lower densities and increase its intensity, as clearly seen in figure \ref{fig:PF}(b), where the peak for $U/t=10$ is at $n \simeq 0.7$. As for the other geometries, a peak develops in the vicinity of the Mott insulating state, here only seen for the larger values of $U/t$ studied.

Comparing the overall effects of correlations in the different geometries analyzed, we see that there are clear correlation-induced peaks near half-filling in all cases. For the square and honeycomb lattices, the non-interacting thermoelectric power factor is zero at half-filling and is driven to  $PF  \approx 0.1 \  k_B^2/e^2 h$. Around quarter and three-quarter fillings as well, correlations also play a relevant role in increasing the power factor for the square and honeycomb lattices. Correlations are very effective in increasing the Power Factor in a triangular lattice, where  $PF  \simeq 0.27 \ k_B^2/e^2 h$, the largest value obtained, for $U/t=10$ at $n \sim 0.7$. For this geometry, we observe that density can be used to tune the thermoelectric Power Factor.


%%%%%%%%%%%%%%%%%%%%%%%%%%%%%%%%%%%%%%%%%%%%%%%%%%%%%%%%%%%%%%%%%%%%%%%%%%%%%%%%%%%%%%%%%%%%%%%%%%

\section{Conclusions}
\label{conclu}
We have studied the thermoelectric properties of strongly interacting two dimensional systems with different geometries. Our results clearly show an anomaly in the Seebeck coefficient in the vicinity of half-filling, characterized by an enhanced response depending on both geometry and interaction strength. The anomaly is characterized by a change in the sign of the carriers which is accompanied by an interaction-induced increase. The anomaly is also intensified with the reduction of temperature. 

The thermoelectric Power Factor displays a competition between the Seebeck coefficient and the conductivity. The anomaly in the Seebeck coefficient is reflected in the $PF$, with correlation-driven peaks immediately below and above half-filling at geometry-dependent values of $U/t$.  The decreasing conductivity near half-filling is the limiting factor in the intensity of  $PF$ in this region of densities. Away from half-filling,  at intermediate densities (around $n=0.4-0.6$ and $n=1.4-1.6$) the peaks in $PF$ have a strong contribution from the conductivity with positions  strongly dependent on geometry. For this range of densities, peak position and intensity can be tuned by correlations. Although the Seebeck coefficient is smaller for the triangular lattice, the Power Factor for this geometry shows the higher peak values and the stronger tunability with density and correlations, making it a strong candidate for enhanced thermoelectric properties. 

\section*{ACKNOWLEDGMENTS}
%%%%%%%%%%%%%%%%%%%%%%%%%%%%%%%%%%%%%%%%%%%%%%%%%%%%%%%%%%%%%%%%%%
%%%%%%%%%%%%%%%%%%%%%%%%%%%%%%%%%%%%%%%%%%%%%%%%%%%%%%%%%%%%%%%%%%

Financial support from 
Fundação Carlos Chagas Filho de Amparo à Pesquisa do Estado do Rio de
Janeiro, grant numbers E-26/204.308/2021 (W.C.F.S.),  E-26/200.258/2023 - SEI-260003/000623/2023 (N.C.C.),  E-26/200.959/2022 (T.P.), and E-26/210.100/2023 (T. P.); CNPq grant numbers  313065/2021-7 (N.C.C.), 403130/2021-2 (T.P.), and 308335/2019-8 (T.P.) is gratefully acknowledged.
 We also acknowledge support from INCT-IQ. We acknowledge support from NSF Materials Research Science and Engineering Center (MRSEC) Grant No. DMR-2011876 and NSF-DMR 2138905 (SR,AS,NT). We also acknowledge computational resources from the Unity cluster at the Ohio State University.


% \begin{figure}[t]
% %\includegraphics[scale=0.29]{figures/Triang1/Cond_beta1.pdf}
% \includegraphics[scale=0.29]{figures/Triang1/Cond_beta2.pdf}
% %\includegraphics[scale=0.29]{figures/Triang1/Cond_beta3.pdf}
% \caption{(Color online) Conductivity as a function of density for the triangular lattice. Size effects clear here for $\beta=3$ and small $U$}
% \label{fig:triang-conductivity}
% \end{figure}

% \begin{figure}[t]
% %\includegraphics[scale=0.29]{figures/Honey1/Cond_beta1.pdf}
% \includegraphics[scale=0.29]{figures/Honey1/Cond_beta2.pdf}
% %\includegraphics[scale=0.29]{figures/Honey1/Cond_beta3.pdf}
% \caption{(Color online) Conductivity as a function of density for the honeycomb lattice. Size effects clear here for $\beta=3$ and small $U$}
% \label{fig:honey-conductivity}
% \end{figure}


% \begin{figure}[t]
% %\includegraphics[scale=0.29]{figures/Square1/n^2k_beta _1.pdf}
% \includegraphics[scale=0.29]{figures/Square1/n^2k_beta_2.pdf}
% %\includegraphics[scale=0.29]{figures/Square1/n^2_beta_3.pdf}
% \caption{(Color online) Compressibility as a function of density for the square lattice.}
% \label{fig:square-compress}
% \end{figure}

% \begin{figure}[t]
% %\includegraphics[scale=0.29]{figures/Triang1/n2k_beta1.pdf}
% \includegraphics[scale=0.29]{figures/Triang1/n2k_beta2.pdf}
% %\includegraphics[scale=0.29]{figures/Triang1/n2k_beta3.pdf}
% \caption{(Color online) Compressibility as a function of density for the triangular lattice.}
% \label{fig:trian-compress}
% \end{figure}

% \begin{figure}[t]
% %\includegraphics[scale=0.29]{figures/Honey1/n^2kapa_beta _1.pdf}
% \includegraphics[scale=0.29]{figures/Honey1/n^2kapa_beta_2.pdf}
% %\includegraphics[scale=0.29]{figures/Honey1/k_beta_3.pdf}
% \caption{(Color online) Compressibility as a function of density for the honeycomb lattice.}
% \label{fig:honey-compress}
% \end{figure}


%\begin{figure}[t]
%\includegraphics[scale=0.29]{figures/Square1/DOS_beta1.pdf}
%\includegraphics[scale=0.29]{figures/Square1/DOS_beta2.pdf}
%\includegraphics[scale=0.29]{figures/Square1/DOS_beta3.pdf}
%\caption{(Color online) Density of states at the Fermi level as a function of density for the square lattice. }
%\label{fig:square-DOS}
%\end{figure}


% \begin{figure}[t]
%\includegraphics[scale=0.29]{figures/Triang1/DOS_beta1.pdf}
% \includegraphics[scale=0.29]{figures/Triang1/DOS_beta2.pdf}
% %\includegraphics[scale=0.29]{figures/Triang1/DOS_beta3.pdf}
% \caption{(Color online) Density of states at the Fermi level as a function of density for the triangular lattice. }
% \label{fig:triangular-DOS}
% \end{figure}

%\begin{figure}[t]
%\includegraphics[scale=0.29]{figures/Honey1/DOS_Beta1.pdf}
%\includegraphics[scale=0.29]{figures/Honey1/DOS_Beta2.pdf}
%\includegraphics[scale=0.29]{figures/Honey1/DOS_Beta3.pdf}
%\caption{(Color online) Density of states at the Fermi level as a function of density for the honeycomb lattice. }
% \label{fig:honey-DOS}
%\end{figure}

% \begin{figure}[t]
% \includegraphics[scale=0.29]{figures/Square1/S_spi_b1.pdf}
% \includegraphics[scale=0.29]{figures/Square1/S_spi_b2.pdf}
% %\includegraphics[scale=0.29]{figures/Square1/S_spi_b3.pdf}
% \caption{(Color online) Near-neighbor spin-spin correlation functions as a function of density for the square lattice.}
% \label{fig:square-spin1}
% \end{figure}


% \begin{figure}[t]
% \includegraphics[scale=0.29]{figures/Honey1/SiSj_Beta1.pdf}
% \includegraphics[scale=0.29]{figures/Honey1/SiSj_Beta2.pdf}
% %\includegraphics[scale=0.29]{figures/Honey1/SiSj_Beta3.pdf}
% \caption{(Color online) Near-neighbor spin-spin correlation functions as a function of density for the honeycomb lattice.}
% \label{fig:honey-spin1}
% \end{figure}


% \begin{figure}[t]
% \includegraphics[scale=0.29]{figures/Square1/docc_beta1.pdf}
% \includegraphics[scale=0.29]{figures/Square1/docc_beta2.pdf}
% %\includegraphics[scale=0.29]{figures/Square1/docc_beta3.pdf}
% \caption{(Color online) Double occupancy as a function of density for the square lattice.}
% \label{fig:square-Docc}
% \end{figure}

% \begin{figure}[t]
% \includegraphics[scale=0.29]{figures/Honey1/Dcc_Beta1.pdf}
% \includegraphics[scale=0.29]{figures/Honey1/Dcc_Beta2.pdf}
% %\includegraphics[scale=0.29]{figures/Honey1/Dcc_Beta3.pdf}
% \caption{(Color online) Double occupancy as a function of density for the honeycomb lattice.}
% \label{fig:honey-Docc}
% \end{figure}



% \begin{figure}[t]
% %\includegraphics[scale=0.29]{figures/Square1/S2sigma_b1.pdf}
% \includegraphics[scale=0.29]{figures/Square1/S^2sigma_b2.pdf}
% %\includegraphics[scale=0.29]{figures/Honey1/Dcc_Beta3.pdf}
% \caption{(Color online) Power factor as a function of density for the square lattice.}
% \label{fig:square-Ssigma}
% \end{figure}

% \begin{figure}[t]
% %\includegraphics[scale=0.29]{figures/Triang1/S2sigma_b1.pdf}
% \includegraphics[scale=0.29]{figures/Triang1/S2sigma_b2.pdf}
% %\includegraphics[scale=0.29]{figures/Honey1/Dcc_Beta3.pdf}
% \caption{(Color online) Power factor as a function of density for the triangular lattice.}
% \label{fig:triangular-Ssigma}
% \end{figure}


%\begin{figure}[t]
%\includegraphics[scale=0.29]{figures/Honey1/S2sigmab1.pdf}
%\includegraphics[scale=0.43]{figures/Fermi_Surface/FSv01.pdf}
%\includegraphics[scale=0.29]{figures/Honey1/Dcc_Beta3.pdf}
%\caption{(Color online) Fermi Surfaces ...}
%\label{fig:FS}
%\end{figure}

\color{black}
%\begin{thebibliography}{100}

\bibliography{references.bib}

\end{document}
