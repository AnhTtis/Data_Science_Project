\section{Conclusion and Future Work} \label{sec:conc}
%Machine learning has proved suitable for solving numerous previously impossible tasks. 
This paper explores how supervised classification models can be used to predict the results of table tennis matches.  The original dataset was retrieved from OSAI \citep{OSAI}.

This paper utilises existing models as implemented in Scikit-learn (logistic regression, random forest classification, SVMs, multi-layer perceptrons); our contribution lies in applying these using 5-fold cross-validation and hyperparameter tuning to the problem of table tennis match prediction.
We also propose using a handful of engineered features, from which a non-linear rank difference has been proved to be the most salient in our ablation study. To investigate overfitting, We consider aggregating feature across all matches of a player including or excluding the target match and demonstrate that our model performs comparably in both cases. T

 Our results are comparable to the accuracy of state-of-the-art \textit{tennis} prediction models (approx. 70\% accuracy). Following hyperparameter tuning, the difference between models was often modest. Other considerations when picking a model for similar applications could include training time or model transparency (at both of which random forests excel). 
%Future works could focus on a selection of the most important features established from the random forest model.

Future work could explore combining TTNet with our prediction model to provide live match predictions. It would be also interesting quantifying uncertainty and to test against real betting odds. As automated table tennis analytics are becoming available below professional leagues, the authors are also interested whether the importance of features and the model choice transfers to these matches as well.