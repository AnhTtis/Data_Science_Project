% CVPR 2023 Paper Template
% based on the CVPR template provided by Ming-Ming Cheng (https://github.com/MCG-NKU/CVPR_Template)
% modified and extended by Stefan Roth (stefan.roth@NOSPAMtu-darmstadt.de)

\documentclass[10pt,twocolumn,letterpaper]{article}

%%%%%%%%% PAPER TYPE  - PLEASE UPDATE FOR FINAL VERSION
% \usepackage[review]{cvpr}      % To produce the REVIEW version
%\usepackage{cvpr}              % To produce the CAMERA-READY version
\usepackage[pagenumbers]{cvpr} % To force page numbers, e.g. for an arXiv version

% Include other packages here, before hyperref.
\usepackage{graphicx}
\usepackage{amsmath}
\usepackage{amssymb}
\usepackage{booktabs}
\usepackage{soul}
\usepackage{enumitem}
\usepackage{multirow}                   

% It is strongly recommended to use hyperref, especially for the review version.
% hyperref with option pagebackref eases the reviewers' job.
% Please disable hyperref *only* if you encounter grave issues, e.g. with the
% file validation for the camera-ready version.
%
% If you comment hyperref and then uncomment it, you should delete
% ReviewTempalte.aux before re-running LaTeX.
% (Or just hit 'q' on the first LaTeX run, let it finish, and you
%  should be clear).
\usepackage[pagebackref,breaklinks,colorlinks]{hyperref}


% Support for easy cross-referencing
\usepackage[capitalize]{cleveref}
\crefname{section}{Sec.}{Secs.}
\Crefname{section}{Section}{Sections}
\Crefname{table}{Table}{Tables}
\crefname{table}{Tab.}{Tabs.}

\newcommand{\datasetName}{\textsc{\textit{Fantastic Breaks}}}

%%%%%%%%% PAPER ID  - PLEASE UPDATE
\def\cvprPaperID{9788} % *** Enter the CVPR Paper ID here
\def\confName{CVPR}
\def\confYear{2023}


\begin{document}

%%%%%%%%% TITLE - PLEASE UPDATE
\title{\textit{Fantastic Breaks:} A Dataset of Paired 3D Scans of Real-World Broken Objects and Their Complete Counterparts}

\author{    
Nikolas Lamb, Benjamin Molloy, Cameron Palmer, Sean Banerjee, Natasha Kholgade Banerjee\\
Clarkson University, Potsdam NY, USA\\
{\tt\small \{lambne, molloybr, campalme, sbanerje, nabernje\}@clarkson.edu}
% First Author\\
% Institution1\\
% Institution1 address\\
% {\tt\small firstauthor@i1.org}
% For a paper whose authors are all at the same institution,
% omit the following lines up until the closing ``}''.
% Additional authors and addresses can be added with ``\and'',
% just like the second author.
% To save space, use either the email address or home page, not both
}

\maketitle

%%%%%%%%% ABSTRACT
\begin{abstract}
  Automated shape repair approaches currently lack access to datasets that describe real-world damage geometry. We present \emph{Fantastic Breaks (and Where to Find Them: \url{https://terascale-all-sensing-research-studio.github.io/FantasticBreaks})}, a dataset containing  scanned, waterproofed, and cleaned 3D meshes for 78 broken objects, paired and geometrically aligned with complete counterparts. \emph{Fantastic Breaks} contains class and material labels, synthetic proxies of repair parts that join to broken meshes to generate complete meshes, and manually annotated fracture boundaries. Through a detailed analysis of fracture geometry, we reveal differences between \emph{Fantastic Breaks} and datasets of synthetically fractured objects generated using geometric and physics-based methods. We show experimental results of shape repair with \emph{Fantastic Breaks} using  multiple learning-based approaches pre-trained using a synthetic dataset and re-trained using a subset of \emph{Fantastic Breaks}.
\end{abstract}

\section{Introduction}
\label{sec:intro}

Damage to objects is an expected occurrence of everyday real-world usage. However, when damage occurs, objects that could be repaired are often thrown out. Additive manufacturing techniques are rapidly becoming accessible at the consumer level, with 3D printing technologies available for materials such as plastics, metals, and even ceramics and wood. Though to date, practical approaches for repair have been largely manual and restricted to niche areas such as cultural heritage restoration, a large body of recent research has emerged on the automated reversal of damage, including reassembly using 3D scans of fractured parts~\cite{brown2008system,funkhouser2011learning,papaioannou2001virtual,papaioannou2003automatic,huang2006reassembling,zhang20153d,hong2019potsac,hong2021structure,zhan2020generative,wu2020pq,harish2022rgl,chen2022neural}, or generation of new repair parts when portions of the original object are irretrievably lost due to damage~\cite{li2011symmetry,sipiran2014approximate,gregor2015automatic,mavridis2015object,papaioannou2017reassembly,sipiran2018completion,hermoza20183d,lamb2022mendnet,lamb2022deepmend,lamb2022deepjoin}. Geometry-driven approaches based on shape matching~\cite{li2011symmetry,sipiran2014approximate,gregor2015automatic,mavridis2015object,papaioannou2017reassembly,sipiran2018completion,lamb2019automated,brown2008system,funkhouser2011learning,papaioannou2001virtual,papaioannou2003automatic,huang2006reassembling,hong2019potsac,hong2021structure}, being limited in their ability to be usable for objects of unknown complete geometry, have given way to learning-driven approaches~\cite{zhan2020generative,wu2020pq,harish2022rgl,chen2022neural,hermoza20183d,lamb2022mendnet,lamb2022deepmend,lamb2022deepjoin} aimed at generalization to repair at a large scale. 

\begin{figure}
    \centering
    \includegraphics[width=\linewidth]{figures/teaser5.jpg}
    \caption{We present \textit{Fantastic Breaks}, a dataset of 3D scans of real-world broken objects (top) aligned to 3D scans of complete counterparts (bottom). Objects span classes such as mugs, plates, figurines, jars, and bowls\textemdash{}household items prone to damage.}
    \label{fig:teaser}
\end{figure}

However, a principal challenge limiting  understanding real-world damage and learning how to perform repair is that datasets of real-world damage for consumer space objects are virtually non-existent. Current learning-driven approaches for repair use datasets where fracture-based damage is synthetically generated using geometric approaches such as Boolean operations with primitives~\cite{gregor2015automatic,lamb2021using,lamb2022mendnet,lamb2022deepmend,lamb2022deepjoin,chen2022neural}. As they make assumptions about the fracture process rather than using data-driven fracture generation, such geometric methods are unlikely to be generalizable to real-world damage. Through Breaking Bad, Sell\'{a}n et al.~\cite{sellan2022breaking} have taken a first step toward large-scale fracture dataset generation. The Breaking Bad dataset consists of 3D shapes from Thingi10k~\cite{zhou2016thingi10k} and PartNet~\cite{mo2019partnet} that have been subjected to physics-based damage using fracture modes~\cite{sellan2022breakinggood}.
By removing macro-scale shape assumptions embodied by geometric primitives, Breaking Bad is a promising step for research in shape assembly and repair. However, fractures in Breaking Bad suffer from typical issues of resolution and simulation time step size that underlie physics simulations. The dataset is thus unfortunately hindered in providing a faithful representation of real-world damage. 

In this work, we contribute \textit{Fantastic Breaks}, the first dataset of 3D scans of damaged objects paired with 3D scans of their complete non-damaged counterparts, as shown in Figure~\ref{fig:teaser}. Each damaged object\textemdash{}hereafter referred to as a broken object due to the nature of damage suffered\textemdash{}is 3D scanned and geometrically registered to a 3D scan of the complete object prior to damage, such that intact regions of the complete and broken scans are aligned. Currently, the dataset contains 78 broken/complete pairs. 

Our damage infliction often leaves one broken part intact, destroying or over-fragmenting the remainder of the object. We use an off-the-shelf subtraction-based approach~\cite{lamb2019automated} to synthetically generate repair part proxies from the intact broken part aligned to its complete counterpart. We also provide manually annotated class labels, material labels, and the fractured region on the broken mesh. Our work emulates the endeavor of groups in vision and robotics that contribute datasets of 3D scanned everyday-use objects~\cite{singh2014bigbird,calli2015benchmarking,calli2017yale,kasper2012kit,downs2022google}. Our analysis of real-world fracture properties of objects in \textit{Fantastic Breaks} reveals fine-scale structure that enables \textit{Fantastic Breaks} to overcome the drawbacks of synthetic datasets. We use our dataset to evaluate existing approaches to automatically generate new repair parts for broken shapes.

We summarize our contributions as follows:

\begin{enumerate}[noitemsep,topsep=0pt]
    \item We contribute the first 3D scanned real-world dataset of geometrically aligned broken/complete object pairs.
    \item We provide class, material, and fracture surface annotations, and ground truth repair part proxies.
    \item We contribute a geometric analysis of \textit{Fantastic Breaks} in comparison to existing synthetic fracture datasets.
    \item We provide evaluations of existing shape repair approaches using \textit{Fantastic Breaks}.
\end{enumerate}

\section{Related Work}
\label{sec:relatedwork}

\paragraph{Fracture Datasets.}Existing real-world fracture collections are restricted to scanning of multiple shards corresponding to a small set of originally intact objects, e.g., 7 objects~\cite{huang2006reassembling}, 3 frescoes~\cite{brown2008system}, and 3 large-scale structures (Akrotiri settlement, Tongeren Roman excavation, and one fresco)~\cite{funkhouser2011learning}. Since fracture acquisition is performed post-damage for historical objects where no known counterparts exist, the datasets lack knowledge of the complete proxies. The Hampson Museum cultural heritage dataset~\cite{payne2009designing} contains 3D scans for 138 cultural heritage objects. The dataset lacks paired damaged/complete data, or annotations to reveal what objects are damaged and what are intact, preventing them from being used to train repair or assembly approaches. Hong et al.~\cite{hong2021structure} show assembly results using a dataset of 5 shattered pots, with 3D scans acquired for the shattered fragments and the original pots. The original scans are used as evaluation oracles. Lamb et al.~\cite{lamb2019automated} repair 22 damaged objects by subtracting the damaged objects from \textit{a priori} known complete proxies. Shell objects have their interiors filled prior to scanning. 

Recognizing the need for large-scale datasets for learning-driven repair, a few datasets contain synthetic or scanned models subjected to synthetic fracture using geometric techniques such as subtracting primitives~\cite{gregor2015automatic,lamb2021using,lamb2022mendnet,lamb2022deepmend,lamb2022deepjoin,chen2022neural}, or using physics models of fracture~\cite{sellan2022breaking}. As we demonstrate in Section~\ref{sec:analysis}, real-world physical damage demonstrates geometric characteristics that differ from the break patterns of synthetically generated damage. Geometric fracture models~\cite{gregor2015automatic,lamb2021using,lamb2022mendnet,lamb2022deepmend,lamb2022deepjoin,chen2022neural} are only as precise as the primitive being used for fracture, e.g., Chen et al.~\cite{chen2022neural} use five simplistic cut functions\textemdash{}planar, sine, parabolic, square, and pulse\textemdash{}modeled as oriented height fields. Lamb et al.~\cite{lamb2021using,lamb2022mendnet,lamb2022deepmend,lamb2022deepjoin} subtract randomly rotated and translated geometric primitives such as a cube, icosphere, and sub-divided icosphere, with and without random surface perturbation to simulate micro-scale detail. Gregor et al.~\cite{gregor2015automatic} use spheres whose surfaces are perturbed using small-scale details of a single digitized material. In all cases, mid-scale detail is represented as cutouts via analytical primitives, which is not generalizable to real-world damage. 

The Breaking Bad dataset of Sell\'{a}n et al.~\cite{sellan2022breaking} uses their earlier work~\cite{sellan2022breakinggood} to model and simulate fracture modes of the object, enabling them to circumvent assumptions made by geometric methods. Their approach struggles to generate high-resolution detail, as it requires evolving the simulation over small time steps that can be computationally infeasible at a large scale. Our \textit{Fantastic Breaks} dataset fills the gap in the lack of real-world damaged object datasets by contributing an even-now growing repository of 3D scanned real-world damaged objects rigidly aligned with 3D scans of their real-world counterparts.

\paragraph{Real-World Object Datasets.} The capture of large-scale complete 3D scans of objects is an arduous task, due to the need for a multi-staged approach consisting of multiple presentations of the object to a scanner ensuring full coverage of hidden parts, registration of multiple scans, cleaning to correct imprecise geometry, eliminate holes, and correct deep concavities, and, depending on the application, waterproofing to ensure closed surfaces. Massive 3D datasets typically comprise single-viewpoint RGB-D images~\cite{janoch2013category,lai2013rgb,xiang2016objectnet3d}, room-scale 3D scans~\cite{dai2017scannet}, or scans of large objects such as motorcycles, statues, benches, and tables~\cite{choi2016large} that cannot be readily impacted to capture real-world damage. The Berkeley BigBIRD dataset~\cite{singh2014bigbird} contained 100 3D models at the time of publication and has grown to 125 models. Objects were recorded by placing on a turntable and models were created by stitching multiple images acquired using a Kinect and Canon setup. Using the BigBIRD setup, Calli et al.~\cite{calli2015benchmarking,calli2017yale} collected the YCB dataset for a set of 77 objects as a collaboration between Yale, CMU, and UC Berkeley. Due to placement on a turntable, hidden portions, e.g., the base or concavities not visible to the cameras, are not captured. The Karlsruhe Institute of Technology (KIT) dataset~\cite{kasper2012kit} consists of 145 objects captured using a Konica Minolta Vi-900 digitizer, with multiple scans acquired to capture object bases. 

Google Scanned Objects (GSO)~\cite{downs2022google} is perhaps the largest complete table-top 3D-scan dataset at this time, consisting of, to-date, 1,030 3D scans of household objects. GSO is collected by 3D imaging projector-cast patterns through a collaboration between robotics researchers at Google and Open Robotics since 2019. 41 researchers were involved in creating GSO. \textit{Fantastic Breaks} has been acquired at a small rural university by 1 graduate student and 9 undergraduates, advised by 2 faculty mentors. With 156 3D scans (78 broken, 78 paired complete counterparts, and growing), \textit{Fantastic Breaks} exceeds university tabletop datasets~\cite{singh2014bigbird,calli2015benchmarking,calli2017yale,kasper2012kit} in total items, and is comparable to the YCB dataset in number of object identities. 

Within the larger problem domain of minimizing waste, a few 2D datasets have arisen for object detection and segmentation in waste images~\cite{sousa2019automation,proencca2020taco,bashkirova2022zerowaste}. Object materials span cardboard, plastic, glass, and metal. An opportunity exists to capture RGB-D images of waste, identify objects capable of being repaired, and geometrically couple them with our dataset to conduct in-the-wild repair.

\paragraph{Shape Repair.}The creation of the \textit{Fantastic Breaks} dataset is motivated by applications in object repair. When object parts are available, shape assembly approaches focus on joining 3D shape representations of the object parts. Geometric approaches exist to match fracture boundaries via segmentation~\cite{papaioannou2001virtual,papaioannou2003automatic}, feature description extraction and geometric model refinement~\cite{huang2006reassembling}, align fractured shapes to a proxy template~\cite{zhang20153d}, or conduct iteratively registration similar to structure from motion~\cite{hong2019potsac,hong2021structure}. A number of learning-based approaches exist to provide assembly, most of which assume holistic parts with simple surfaces~\cite{zhan2020generative,wu2020pq,harish2022rgl}, and one assumes arbitrary geometry at shape boundaries~\cite{chen2022neural}.

When parts of an object are irretrievable, shape repair approaches address generation of the lost parts. Early automation approaches to circumvent historically manual repair included finding symmetries and self-similarities in objects~\cite{li2011symmetry,sipiran2014approximate,gregor2015automatic,mavridis2015object,papaioannou2017reassembly,sipiran2018completion}, though these approaches are unsuccessful when non-symmetric object parts are broken off. One approach~\cite{lamb2019automated} circumvents the small-scale artifact issue of Boolean subtraction by automatically extracting and joining exterior and fractured regions for repair parts using real-world damaged and complete scans. The approach requires the scan of a complete 3D proxy to be provided as input, that may not be feasible for one-of-a-kind instances, or even if available, may prove tedious to obtain. 

Recent work such as 3D-ORGAN~\cite{hermoza20183d}, MendNet~\cite{lamb2022mendnet}, DeepMend~\cite{lamb2022deepmend}, and DeepJoin~\cite{lamb2022deepjoin} uses deep learning to conduct shape repair without knowledge of the complete proxy by representing damaged, complete, and repair shapes using voxels~\cite{hermoza20183d} or deep functions~\cite{lamb2022mendnet,lamb2022deepmend,lamb2022deepjoin}. DeepMend and DeepJoin report higher success due to the use of implicit functions which enable representation to arbitrary resolutions, and due to the expression of fractured and restoration shapes in terms of constructive solid geometry operations between the complete object and a break surface whose representation is learnt during training. Though DeepMend and DeepJoin show qualitative results on a few examples of 3D scans of real-world damaged objects, in all cases, training and quantitative evaluation is conducted using datasets with synthetically-generated fractures.

\section{Data Collection and Processing}
\label{sec:datacoll}

\textbf{Object Acquisition.} We conducted a community-wide acquisition of everyday household objects that suffer damage. Our goal was to have a collection consisting not only of objects that can be damaged, but also of objects that have already suffered damage. To perform our acquisition, we made purchases at the local thrift store where we found intact and damaged objects, and requested donations from the local community. Though the \textit{Fantastic Breaks} dataset as presented in this paper contains damaged/complete pairs, we have also collected pre-damaged objects that may lack complete counterparts, as they provide insight into real-world fracture. Our collection consists of commonly damaged household objects such as mugs, plates, and figurines spanning materials such as ceramics, plastics, glass, and wood. In some instances, we were able to pre-acquire pairs, where one object in the pair was damaged and the other was intact. In most cases, we manually damaged a complete object to obtain the broken version. When possible, we attempted to acquire pairs of complete objects, and damage one of the objects, enabling us to store physical damaged/complete pairs. We inflicted damage by dropping the object, striking the object with a rubber mallet or metal hammer, and, in 2 cases, snapping the object after anchoring it against a table. The left of Figure~\ref{fig:fracturescan} shows an example mug with the handle shattered after striking with a mallet. Figure~\ref{fig:collection} shows example broken objects and broken/complete pairs in our dataset.

\begin{figure}[t!]
    \centering
    \includegraphics[width=\linewidth]{figures/fracturescan.jpg}
    \caption{Left: Complete mug on top, and broken mug on the bottom showing main intact part and the shattered handle. Right: Object on scanner with 3D scan shown in the inset.}
    \label{fig:fracturescan}
\end{figure}

\begin{figure}[t!]
    \centering
    \includegraphics[width=\linewidth]{figures/collection.jpg}
    \caption{Example broken objects on top and broken/complete pairs on the bottom acquired to build the \textit{Fantastic Breaks} dataset.}
    \label{fig:collection}
\end{figure}

\textbf{3D Scanning.} We use an Einscan SP 3D turntable-based scanner to acquire 3D scans. We use Einscan's proprietary software, EXScan, to operate the scanner. The scanning operation rotates the scanner's turntable 8 times, acquires 8 2.5D images of the object via an attached RGB-D sensor, and fuses the images into a 3D mesh. Given the diversity of object geometry where objects may contain deep concavities or complex fractures, we had to conduct careful staging of each object and perform multiple presentations of the object to maximize acquisition of the object surface. The presentation count ranged from 2 for flat objects such as plates or objects with higher convexity such as figurines, to 6 for objects with concavities such as mugs. We used the registration tool in EXScan to fuse scans into a 3D mesh.

\begin{figure}[t!]
    \centering
    \includegraphics[width=\linewidth]{figures/cleaning2.pdf}
    \caption{Meshes prior to (left) and after cleaning and waterproofing for (a)~broken and (b)~complete object examples waterproofed using EXScan, and (c)~broken and (d)~complete objects cleaned using Netfabb. For (d), we use a primitive to fix the erroneously angled surfaces created by Netfabb mesh repair tools. Complete objects are shown as cutaways to reveal pre-cleaning artifacts.}
    \label{fig:cleaning}
\end{figure}

\textbf{Mesh Cleaning.} Given the 3D scan of a complete or broken object, we employ a sequence of operations to ensure that the meshes are of high quality. We visually inspect each model, and if the model lacks any large holes in the mesh, we use the waterproofing tool built into EXScan to ensure that each surface is a closed 2D surface. While performing waterproofing in EXScan is preferred, if the model contains large holes or artifacts, we repair the original post-registration mesh by performing manual hole-filling using Autodesk NetFabb. For objects with deep concavities such as cups and mugs, the concavities may not have been well-presented during the scanning process, in which case the interior regions may not be metrically accurate. We repair the mesh by using subtraction with a geometric primitive such as a cylinder. For a final clean, we apply the Extended Repair set of scripts in Netfabb to merge nearby vertices, remove double and flipped triangles, close all holes, wrap the mesh surface to remove interior faces, and remove small connected components. Figure~\ref{fig:cleaning} demonstrates examples of meshes generated using our cleaning process.

\textbf{Mesh Orientation.} Once each mesh is cleaned, we manually orient each mesh such that its principal axes are aligned to the Cartesian axes. We ensure that the base of the object is aligned with the \textit{xz}-plane, and we rotate the object about the \textit{y}-axis to ensure alignment within its category. For example, we align all mugs to ensure that the handles are aligned with the negative \textit{x} direction, and all figurines so that they face in the negative \textit{x} direction.

\textbf{Mesh Alignment.} Given the cleaned meshes for the broken object and its complete counterpart, we transform the broken mesh such that its intact regions are aligned with the corresponding regions of the complete mesh. We  perform an initial manual alignment of the broken mesh to the complete mesh, and refine the alignment using the iterative closest point (ICP)~\cite{besl1992method} algorithm. We conduct a post-alignment normalization of each mesh to ensure similarly scaled data for learning-driven repair. We perform normalization by scaling the broken and complete mesh such that the complete mesh is entirely contained within a unit cube. We provide access to non-normalized and normalized meshes as part of the dataset. Figure~\ref{fig:triplets} shows examples of the broken mesh aligned to the complete mesh.

\textbf{Ground Truth Restoration Estimation.} Our data collection involves scanning of broken objects that have either been acquired as pre-damaged, or have had damage inflicted through a destructive fracture process, such that only one part of the object remains intact. Such an occurrence is not uncommon, e.g., as shown in Figure~\ref{fig:fracturescan}, the fracture process causes the handle to further fragment into a number of small pieces that are infeasible to rescue and/or reassemble. However, shape repair approaches can benefit from knowledge of the geometry of ground truth parts needed to complete the object shape, in order to perform training and evaluation. We contribute proxy ground truth 3D meshes for the repair parts. We synthetically generate the repair part proxies using the approach of Lamb et al.~\cite{lamb2019automated}. Given aligned complete and broken shapes, the approach recovers restoration meshes that smoothly join the broken shape to yield the complete geometry, lack small-scale artifacts prevalent in Boolean subtraction, and lack grooves at the fracture boundary demonstrated by approximate subtraction techniques based on a distance thresholding. Figure~\ref{fig:triplets} shows restoration meshes generated for example cleaned and waterproofed 3D broken and complete scans.

\textbf{Ground Truth Fracture Surface Annotation.} To assist with approaches in  repair that rely on accurate knowledge of the ground truth fracture surface for training or evaluation, we manually annotate triangles corresponding to the fracture surface. Figure~\ref{fig:segments} shows an example of ground truth fracture annotation.
\begin{figure}[t!]
    \centering
    \includegraphics[width=\linewidth]{figures/triplets.jpg}
    \caption{For each triplet of meshes, we show the complete mesh on the left, the broken mesh aligned to the complete mesh on the right, and the restoration in pink detached from the broken.}
    \label{fig:triplets}
\end{figure}

\setlength{\tabcolsep}{1pt}
\begin{table}[t!]
    \centering
    \caption{Object distribution by class (Phys.=Physical).}
    \footnotesize
    \begin{tabular}{@{}c|c|ccccccccc|c@{}}
    \toprule
    & Class & Mug & Plate & Figurine & Bowl & Cup & Jar & Coaster & Box & Misc & Total\\
    \hline
   \multirow{2}{*}{\rotatebox{90}{Phys.}} & Broken & 34 & 28 & 21 & 14 & 8 & 7 & 4 & 3 & 12 & 131\\
    & Complete & 29 & 31 & 29 & 16 & 13 & 11 & 5 & 3 & 20 & 157\\
    \hline
   \multirow{3}{*}{\rotatebox{90}{Scan}} & Broken & 34 & 27 & 19 & 12 & 7 & 6 & 4 & 2 & 9 & 120\\
    & Complete & 29 & 31 & 28 & 14 & 8 & 9 & 5 & 3 & 18 & 145\\
    & Pairs & 18 & 21 & 15 & 8 & 5 & 5 & 2 & 1 & 3 & 78\\
\bottomrule
    \end{tabular}
    \label{tab:class}
\end{table}

\section{Analysis of \textit{Fantastic Breaks}}
\label{sec:analysis}

\paragraph{Summary of Dataset.}

We have annotated each object with a category level, its material, and the approach of damaging the object. At the time of this paper submission, we have acquired 131 physical damaged objects and 157 physical complete objects. Among the 131 broken objects, 14 objects are naturally damaged, and the remaining have damage inflicted manually by dropping (20) striking using a mallet (70), striking using a metal hammer (25), and anchoring and snapping (2). 108 of the broken objects have a counterpart within the 157 complete objects. We have acquired pre-cleaned scanned meshes for 120 of the broken objects and 145 of the complete objects. 78 of the 120 broken scanned objects are paired with their respective 78 complete scanned objects. These 78 pairs are fully cleaned and have restorations extracted. Tables~\ref{tab:class} and \ref{tab:material} summarize physical, scanned, and paired scanned broken and complete objects by class and material respectively. It should be noted that this is a growing dataset, i.e., the physical and scanned sets are continuing to expand. We expect to have all 157 complete objects broken, and we continue to acquire, break, scan, clean, and process complete and broken objects.

\begin{figure}[t!]
    \centering
    \includegraphics[width=.9\linewidth]{figures/segments.jpg}
    \caption{Broken meshes (gray) with  labeled fracture surface (red).}
    \label{fig:segments}
\end{figure}
\setlength{\tabcolsep}{2pt}
\begin{table}[t!]
    \centering
    \caption{Object distribution by material (Mat., Phys.=Physical).}
    \footnotesize
    \begin{tabular}{@{}c|c|cccccc|c@{}}
    \toprule
    & Mat. & Ceramic & Plastic & Glass & Plaster & Wood & Other & Total\\
    \hline
   \multirow{2}{*}{\rotatebox{90}{Phys.}} & Broken & 105 & 14 & 4 & 3 & 3 & 2 & 131\\
    & Complete & 113 & 20 & 9 & 3 & 5 & 7 & 157\\
    \hline
   \multirow{3}{*}{\rotatebox{90}{Scan}} & Broken & 99 & 10 & 3 & 3 & 3 & 2 & 120\\
    & Complete & 111 & 17 & 3 & 3 & 5 & 6 & 145\\
    & Pairs & 64 & 7 & 2 & 3 & 2 & 0 & 78\\
\bottomrule
    \end{tabular}
    \label{tab:material}
\end{table}

\paragraph{Analysis of Geometric Properties.}

\begin{figure*}[t!]
    \centering
    \includegraphics[width=\linewidth]{figures/normohhist.pdf}
    \caption{Broken shapes with inset showing geometric detail, alternate (alt.) top-down views, and normal orientation histograms (NOHs) for (a)~Breaking Bad, (b)~Geometric Breaks, and (c)~\textit{Fantastic Breaks}. NOHs are viewed as images where higher intensity lines represent higher bin counts. (a)~Breaking Bad shapes show sparse structure due to limitations on fracture resolution. (b)~While Geometric Breaks shapes have semi-dense structure, they demonstrate regularity attributed to the breaking primitive. (c)~\textit{Fantastic Breaks} shapes have dense surface structure and irregular break patterns characteristic of arbitrary real-world fracture.}
    \label{fig:norm_hist}
\end{figure*}

We quantitatively compare geometric properties of the \textit{Fantastic Breaks} dataset to the \texttt{everyday} subset of Breaking Bad~\cite{sellan2022breaking}, and the Geometric Breaks dataset provided by Lamb et al.~\cite{lamb2022deepjoin}, both of which contain synthetic fractures. The \texttt{everyday} subset of Breaking Bad contains 542 objects each fractured 100 times, for a total of 54,200 broken objects. Breaks are obtained using a physics driven fracturing technique in which a set of fracture modes are computed and used to simulate object fracture patterns that result from an impact to the object. The Geometric Breaks dataset contains 25,449 objects from ShapeNet~\cite{chang2015shapenet} and 1,042 objects from the GSO dataset. Objects are fractured by subtracting them with a randomized convex geometric primitive. For Breaking Bad we compute quantitative metrics over 78 objects selected randomly from the subset of objects in the \texttt{everyday} set that contain 2 broken parts (13,532 contain 2 broken parts). For Geometric Breaks, we compute metrics over a subset of 78 randomly selected objects from the GSO and ShapeNet mugs classes. We combine the GSO and ShapeNet mugs class as the GSO dataset lacks mugs.

We provide a summary of the mean number of faces and vertices for each broken mesh in each dataset on the left of Table~\ref{tab:quant}. Our broken meshes are, on average, at least ten times more dense than existing fractured object datasets. As observed by Sell\'{a}n et al.~\cite{sellan2022breaking}, a common indicator of fractures generated using pre-computed fracture methods are a large number of convex fractured shapes. In the final three columns of Table~\ref{tab:quant} we provide the 25$^{\textrm{th}}$, 50$^{\textrm{th}}$, and 75$^{\textrm{th}}$ percentiles of broken shape convexity. We compute convexity as the ratio of the volume of the broken mesh to the volume of the mesh's convex hull, as used by Attene et al.~\cite{attene2008hierarchical}. Highly convex shapes show a convexity value near~1. Our dataset shows a consistently lower fractured shape convexity than other datasets, with the 75$^{\textrm{th}}$ percentile of breaks showing a convexity value of 0.486, compared to 0.831 for Breaking Bad and 0.910 for Geometric Breaks.

\begin{table}[t!]
\centering
\small
\caption{Number of vertices, faces, and convexity percentiles for Breaking Bad, Geometric Breaks, and \textit{Fantastic Breaks} datasets. Max Vertices and Faces and min Convexity values are bolded.}
\begin{tabular}{@{}l|cc|ccc@{}}
\toprule
Dataset & \# Vertices & \# Faces & C 25th & C 50th & C 75th \\ \hline
Breaking Bad & 4,664.6 & 18,221.9 & 0.252 & 0.562 & 0.831 \\
Geometric & 48,624.9 & 97,247.7 & 0.568 & 0.751 & 0.910 \\
\textit{Fantastic} & \textbf{571,869.2} & \textbf{1,144,518.2} & \textbf{0.236} & \textbf{0.329} & \textbf{0.486} \\
\bottomrule
\end{tabular}
\label{tab:quant}
\end{table}

Though synthetically generated fractured objects may show coarse geometric variation, they struggle to generate fine detail at the fractured region. As shown in Figures~\ref{fig:norm_hist}(a) and \ref{fig:norm_hist}(b), synthetically generated fractured objects are characterized by piecewise fractured regions that lack high frequency surface variation, i.e. they do not capture the fine scale surface variability of real fractures. Simulation of object fractures with fine-scale surface variability of the same resolution as real object fractures using a physics engine as done in Breaking Bad is intractable with current hardware. Fractures generated by subtracting geometric primitives demonstrate unnatural regularity as shown in Figure~\ref{fig:norm_hist}(b). 

To quantify coarse and fine scale surface variability, we introduce the normal sparseness metric, which measures the degree to which normals on the fractured surface span the space of all possible orientations when grouped into \textit{n} discrete bins. For a given broken object, we extract the fractured region of the broken mesh. We obtain \textit{n} evenly spaced points on the unit sphere, and bin the normal vectors of the fractured region into one of \textit{n} bins based on their closest point on the unit sphere to generate a normal orientation histogram (NOH) for the object. We show NOHs for three example objects in Figure~\ref{fig:norm_hist}. To compute the normal sparseness, we count the number of empty bins in the NOH and divide by \textit{n}. A fractured region with coarse geometry is expected to show high normal sparseness as its normals are expected to span a small portion of the space of all possible orientations. The choice for \textit{n} determines the scale of surface variability measured. A small value for \textit{n} measures coarse surface variability, i.e. highly concave or convex fractures will have a low normal sparseness, regardless of fine scale surface geometry. A large value for \textit{n} measures fine-scale surface variability. If \textit{n} is large, only objects with fine-scale surface variability show low normal sparseness.

We show quantitative results for the normal sparseness for two values of \textit{n} in Figure~\ref{fig:norm_box}(a) and Figure~\ref{fig:norm_box}(b). To measure coarse and fine surface variability, we set \textit{n} to 42 and 642 respectively. As shown in Figure~\ref{fig:norm_box}(a), \textit{Fantastic Breaks} shows a similar sparseness spread to Geometric Breaks, indicating that both datasets contain breaks that have a high degree of coarse surface variability, though \textit{Fantastic Breaks} contains more objects with fine surface variability, as shown the lower spread in Figure~\ref{fig:norm_box}(b). \textit{Fantastic Breaks} shows the lowest mean normal sparseness value of 0.253 when \textit{n} is 42, compared to 0.602 for Breaking Bad and 0.359 for Geometric Breaks. As shown in Figure~\ref{fig:norm_box}(b), \textit{Fantastic Breaks} shows the lowest median normal sparseness when \textit{n} is 642, indicating that it contains a large number of fractures with fine scale surface variability, and includes several objects with extremely low sparseness. In contrast, the piecewise fractures of Breaking Bad, and to a lesser extend Geometric Breaks, demonstrate high normal sparseness, as shown in Figure~\ref{fig:norm_hist}(a), indicating a lack of fine scale surface variability. \textit{Fantastic Breaks} also has the lowest mean mean normal sparseness of 0.475, compared to 0.926 for Breaking Bad and 0.749 for Geometric Breaks.

\section{Experimental Evaluation}
\label{sec:exper}

\begin{figure}[t!]
    \centering
    \includegraphics[width=\linewidth]{figures/fig_ns_box.pdf}
    \caption{Box plots for distribution of normal sparseness over 78 objects from Breaking Bad, Geometric Breaks, and all objects from \textit{Fantastic Breaks} using (a) 42 bins and (b) 642 bins.}
    \label{fig:norm_box}
\end{figure}

As we provide complete, broken, and restoration shapes with ground truth fractured region annotations, our dataset may be used to train approaches that perform automated shape repair. We test three prior shape repair approaches on our dataset: MendNet~\cite{lamb2022mendnet}, DeepMend~\cite{lamb2022deepmend}, and DeepJoin~\cite{lamb2022deepjoin}. These approaches generate repair parts assuming that the missing part has been lost or destroyed during the damage process. MendNet, DeepMend, and DeepJoin represent shapes by learning a function to reconstruct shapes as implicit surfaces, and require watertight 3D meshes as input. DeepJoin also requires meshes to have correct surface normals, which \textit{Fantastic Breaks} provides.

To train shape repair methods, we pre-train a given network on a subset of the Geometric Breaks dataset or Breaking Bad dataset. For Geometric Breaks, we train with objects from the GSO and ShapeNet mugs subsets. For Breaking Bad, we train with \texttt{everyday} objects that have 2 parts. After training for 2,000 epochs on synthetically fractured objects, we train for an additional 1,000 epochs on objects from the \textit{Fantastic Breaks} dataset. To generate training data, for each broken, complete, and restoration mesh tuple in both datasets, we sample points on the surface of each mesh and compute the signed distance function (SDF), occupancy, and normal field value for each sample point. For DeepMend and DeepJoin we compute a break surface that acts as a proxy for the fracturing process by fitting a thin-plate spline to the fractured region vertices, as described by Lamb et al.~\cite{lamb2022deepjoin}. We use 812 train and 25 test objects from Geometric Breaks, and 730 train and 25 test objects from Breaking Bad. For the \textit{Fantastic Breaks} dataset, we add objects that belong to a class with a single sample to the test set. \textit{Fantastic Breaks} has 53 train and 25 test objects.

In Figure~\ref{fig:repair} we show repairs for broken objects before and after training on \textit{Fantastic Breaks}. Before re-training on our real objects, DeepMend and DeepJoin may struggle to generate repairs that fully restore real objects, e.g. the plates on the left of Figure~\ref{fig:repair}(a) and the plate and bowl on the left of Figure~\ref{fig:repair}(b). Re-training on \textit{Fantastic Breaks} visually improves repairs, as shown in the middle columns of Figure~\ref{fig:repair}, producing more holistic repairs. As shown in the right columns of Figure~\ref{fig:repair}, re-training still allows generation of repairs for synthetically fractured objects. 

To measure the accuracy of predicted repairs we use the chamfer distance, as described by Park et al.~\cite{park2019deepsdf}, and normal consistency, as described by Mescheder et al.~\cite{mescheder2019occupancy}. As MendNet, DeepMend, and DeepJoin perform optimization during inference to obtain repairs, inference proceeds non-deterministically. For pre-training on Geometric Breaks we report quantitative metrics over 3 re-trained models and 7 inference runs with re-training, and 21 inference runs without re-training, totaling 21 trials. For pre-training on Breaking Bad we report quantitative metrics over 1 re-trained model and 7 inference runs with re-training, and 7 inference runs without re-training, totaling 7 trials. 

\begin{figure}[t!]
    \centering
    \includegraphics[width=\linewidth]{figures/fig_repair.pdf}
    \caption{Predicted repair shapes in pink trained with (top) Geometric Breaks and (bottom) Breaking Bad using (a) DeepJoin and (b) DeepMend. Objects from \textit{Fantastic Breaks} are shown before and after re-training, which improves qualitative results.}
    \label{fig:repair}
\end{figure}

Table~\ref{tab:repair} shows success rate, chamfer distance and normal consistency for predicted repairs using MendNet, DeepMend, and DeepJoin. As shown in the final supercolumn of Table~\ref{tab:repair} top, re-training on the \textit{Fantastic Breaks} dataset decreases the chamfer of repair parts in all experiments except for MendNet on Geometric Breaks. As acknowledged by the authors, MendNet struggles to restore real objects when trained on synthetic fractures. Thus re-training on 53 real samples may not benefit learning. Table~\ref{tab:repair} top shows that for DeepMend and DeepJoin, re-training does not impact performance on Geometric Breaks in terms of chamfer distance and improves normal consistency. Though Table~\ref{tab:repair} bottom shows that re-training does not decrease the chamfer of Breaking Bad objects, it does increase normal consistency for DeepMend. Breaking Bad objects span a wide range of object types may have more than half of the object removed, making it a challenging dataset for repair. Re-training especially decreases the chamfer for DeepMend and DeepJoin when testing on \textit{Fantastic Breaks} objects, from 0.081 to 0.043, and from 0.031 to 0.025 respectively when pre-trained on Geometric Breaks and from 0.085 to 0.041 and 0.048 to 0.028 when pre-trained on Breaking Bad. 

\setlength{\tabcolsep}{4pt}
\begin{table}[t!]
\caption{Success rate (SR), chamfer distance (CD), and normal consistency (NC), with training on Geometric Breaks (GB, top) and on Breaking Bad (BB, bottom), before (left) and after (right) re-training on \textit{Fantastic Breaks} (FB).}
    \centering
    \small
    \begin{tabular}{@{}cc|ccc|ccc@{}}
    \toprule
    & & \multicolumn{3}{c|}{Train with GB Only} & \multicolumn{3}{c}{Train with GB + FB}\\
    & Test & SR\% & CD & NC & SR\% & CD & NC\\
    \hline
    \multirow{2}{*}{\rotatebox{90}{$\begin{matrix}\textrm{Mend}\\\textrm{Net}\end{matrix}$}}& GB & \textbf{70.9} & \textbf{0.149} & 0.246 & 65.5 & 0.203 & \textbf{0.257}\\
    & FB & 83.6 & 0.092 & \textbf{0.258} & \textbf{85.7} & \textbf{0.091} & 0.239\\
    \hline
    \multirow{2}{*}{\rotatebox{90}{$\begin{matrix}\textrm{Deep}\\\textrm{Mend}\end{matrix}$}}& GB & \textbf{99.4} & 0.200 & 0.359 & 96.4 & \textbf{0.170} & \textbf{0.460}\\
    & FB & 93.0 & 0.081 & \textbf{0.262} &\textbf{99.0} & \textbf{0.043} & 0.247\\
    \hline
    \multirow{2}{*}{\rotatebox{90}{$\begin{matrix}\textrm{Deep}\\\textrm{Join}\end{matrix}$}}& GB & \textbf{98.5} & 0.159 & 0.395 & 97.7 & \textbf{0.153} & \textbf{0.529}\\
    & FB & 99.0 & 0.031 & \textbf{0.270} & \textbf{100.0} & \textbf{0.025} & 0.259\\
    \midrule
    \midrule
    & & \multicolumn{3}{c|}{Train with BB Only} & \multicolumn{3}{c}{Train with BB + FB}\\
    & Test & SR\% & CD & NC & SR\% & CD & NC\\
    \hline
    \multirow{2}{*}{\rotatebox{90}{$\begin{matrix}\textrm{Deep}\\\textrm{Mend}\end{matrix}$}} & BB & \textbf{99.4} & \textbf{0.103} & \textbf{0.385} & 98.9 & 0.172 & 0.254 \\
    & FB & 93.7 & 0.050 & 0.451 & \textbf{98.9} & \textbf{0.041} & \textbf{0.500} \\
    \hline
    \multirow{2}{*}{\rotatebox{90}{$\begin{matrix}\textrm{Deep}\\\textrm{Join}\end{matrix}$}}& BB & 99.4 & \textbf{0.105} & 0.335 & \textbf{100.0} & 0.146 & \textbf{0.357} \\
    & FB & \textbf{100.0} & 0.048 & 0.377 & \textbf{100.0} & \textbf{0.028} & \textbf{0.526} \\ 
    \bottomrule
    \end{tabular}
    \label{tab:repair}
\end{table}
\section{Discussion} 
\label{sec:discussion}

We present \textit{Fantastic Breaks}, a novel dataset that contains full 360$^\circ$ 3D scans of broken objects geometrically aligned with 3D scans of their complete counterparts, with manual annotations of classes, materials, and fracture surfaces, and synthetic proxies for 3D meshes representing repair parts. The dataset continues to grow in number of physical objects and scans. The dataset is one of the first of its kind, enabling learning of the characteristics of fractured objects. \textit{Fantastic Breaks} provides data-driven insight into fracture, overcoming the deficits of geometric approaches that make prior assumptions about the damage process that are not widely applicable, as well as the concerns of datasets based on physics simulations that are limited by current hardware in modeling real-world geometry.

An obvious limitation of our endeavor is that employing destructive processes to damage objects for the purpose of real-world data acquisition is unsustainable at a large scale. We advocate that our dataset be leveraged to learn patterns of break and internal geometric structure that are common across objects of similar materials, such as ceramics or plastics, and classes, such as mugs or cups, and to use generative approaches to conduct data-driven synthesis of breaks and internal structure given 3D models of complete objects. For instance, an interesting observation of our collection is that for reasons of cost, sustainable production, and functionality, nearly all our objects have shell rather than solid structures, whereas scans of whole objects such as bottles or figurines cannot capture the internal shell structure. By exposing the internal structure, the dataset provides the opportunity to learn how to hollow out 3D models of objects, an opportunity absent from prior 3D scan datasets.

\textit{Fantastic Breaks} currently contains tabletop objects that are easier to scan using desktop scanners. Larger objects may necessitate more elaborate setups, e.g., room-scale imaging systems. Future data collection can investigate the minimal number of viewpoints needed to acquire geometrically relevant understanding of internal object structure. For instance, to capture the fracture pattern of a broken chair leg, it may be sufficient to use a depth camera to image the broken region from 1-2 viewpoints, and deform the mesh of a 3D proxy of the chair acquired from a public repository to register the proxy to the viewpoints. 

We have provided evaluations of existing approaches on automatic reconstruction of new repair parts using learning-driven approaches. The dataset is widely applicable to a range of other tasks, e.g., the broken and restoration meshes can be used to perform shape assembly cognizant of precision joins for real-world fracture boundaries. We evaluate shape repair approaches that do not require prior knowledge of the fractured region. However, via our manual fracture surface annotations, we also plan to release the incomplete meshes devoid of the fracture surface. These incomplete meshes will benefit research in partial shape completion~\cite{fei2022comprehensive}, which up until now has largely focused on synthetically generated partial shapes or on depth scans. Datasets of real-world 3D scans have contributed to significant advancements in robotic manipulation~\cite{fang2020graspnet,chao2021dexycb}. \textit{Fantastic Breaks} provides enhanced impacts in robotics research by enabling investigation of robot-driven repair, object grasp while being cognizant of fractured regions to minimize further damage, and damaged object handling for safe human-robot handover. 

%%%%%%%%% REFERENCES
{\small
\bibliographystyle{ieee_fullname}
\bibliography{references}
}

\end{document}
