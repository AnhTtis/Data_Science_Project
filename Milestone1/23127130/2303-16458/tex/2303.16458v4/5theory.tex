\section{Theoretical Analysis}

In this section, we theoretically analyze the rationality of the generator space and possible downstream space in W2PGNN.  Detailed proofs of the following theorems can be found in Appendix~\ref{app:proof}.



\subsection{Theoretical Justification of Generator Space}
\vpara{Our generator preserves the properties of graphons.}
We first theoretically prove that any generator in the generator space still preserve the properties of graphon (\emph{i.e.}, a bounded symmetric function $ [0,1]^{2} \rightarrow [0,1]$, summarized in the following theorem.
\begin{theorem} \label{theor-preserve}
For a set of graphon basis $\{B_i\}$, the corresponding generator space $\Omega=\{ f(\{\alpha_i\},\{B_i\}) \mid \forall \ \{\alpha_i\},\{B_i\} \}$ is the convex hull of $\{B_i\}$.
\end{theorem}

\vpara{Our generator preserves the key transferable patterns in graphon basis.}
As a preliminary, we first introduce the concept of \emph{graph motifs} as a useful description of transferable graph patterns and leverage \emph{homomorphism density} as a measure to quantify the degree to which the patterns inherited in a graphon.
\begin{definition}[Graph motifs~\cite{milo2002network}]
Given a graph $G=(V,E)$ ($V$ and $E$ are node and edge set), graph motifs are substructures  $F=(V^\prime,E^\prime)$ that recur significantly in statistics, where $V^{\prime} \subset V, E^{\prime} \subset E$ and $\left|V^{\prime}\right| \ll|V|$.
\end{definition}
Graph motifs can be roughly taken as the key transferable graph patterns across graphs~\cite{zhang2021motif}. 
For example, the motif  ($\vcenter{\hbox{\includegraphics[width=2.4ex,height=2.4ex]{figure/loop.pdf}}}$) has the same meaning of ``feedforward loop'' across networks of control system, gene systems or organisms.


Then, we introduce the measure of homomorphism density  $t(F,B)$ to quantify the relative frequency of the key transferable pattern, \emph{i.e.}, graph motifs $F$, inherited in graphon $B$.
\begin{definition}[Homomorphism density~\cite{lovasz2012large}]
Consider a graph motif $F=(V^\prime,E^\prime)$, we define a homomorphisms of $F$ into graph $G=({V}, {E})$ as an adjacency-preserving map from $V^\prime$ to $V$, where 
$(i,j) \in {E}^\prime$ implies $(i,j) \in {E}$.  There could be multiple maps from $V^\prime$ to $V$, but only some of them are homomorphisms. Therefore, the definition of homomorphism density $t(F,G)$ is introduced to
quantify the relative frequency with which the graph motif $F$ appears in $G$.


Analogously, the homomorphism density of graphs can be extended into the graphon $B$.
We denote $t(F,B)$ as the homomorphism density of graph motif $F$ into graphon $B$, which represents the relative frequency of $F$ occurring in a collection of graphs  $\{{G}_i\}$ that convergent to graphon $B$, \emph{i.e.}, $ t({F}, B) =
\lim _{i \rightarrow \infty} t\left({F}, \{{G}_i\}\right)$.
\end{definition}

Now, we are ready to quantify how much the transferable patterns in graphon basis can be preserved in our generator by exploring the difference between the homomorphism density of graph motifs into the graphon basis and that into our generator.

\begin{theorem}\label{theor:diff}
Assume a graphon basis $\{B_1, \cdots, B_k\}$ and their convex combination $f( \{\alpha_i\}, \{B_i\}) = \sum _{i=1}^k \alpha_i B_i$. The $a$-th element of graphon basis $B_a$ corresponds to a motif set. For each motif $F_a$ in the motif set, the difference between the homomorphism density of $F_a$ in $f( \{\alpha_i\}, \{B_i\})$ and  that in  basis element $B_a$ is upper bounded by
\begin{equation}
|t(F_a,f(\{ \alpha_i\}, \{B_i\}))-t(F_a,B_a)|\leq \sum _{b=1,b\neq a}^k  |F_a| \alpha_b || B_b -B_a||_\square
\end{equation}
where $|F_a|$ represents the number of nodes in motif $F_a$,  $||\cdot||_\square$ is the cut norm.
\end{theorem}
Theorem~\ref{theor:diff} indicates the graph motifs (\emph{i.e.}, key transferable patterns) inherited in each basis element can be preserved in our generator, which justifies the rationality to take the generator as a representative and comprehensive summary of pre-training data.



\subsection{Theoretical Justification of Possible Downstream Space}


The possible downstream space includes the graphs generated from generator $f( \{\alpha_i\}, \{B_i\})$. We here provide a theoretical justification that the generated graphs in possible downstream space can inherit key transferable graph patterns (\emph{i.e.}, graph motifs) in the generator.



\begin{theorem}\label{thm:down}
Given a graph generator $f(\{ \alpha_i\}, \{B_i\})$, we can obtain sufficient number of random graphs $\mathbb{G} = \mathbb{G}(n, f( \{\alpha_i\}, \{B_i\}))$ with $n$ nodes 
generated from $f( \{\alpha_i\}, \{B_i\})$.
The homomorphism density of graph motif $F$ in $\mathbb{G}$ can be considered approximately equal to that in $f( \{\alpha_i\}, \{B_i\})$ with high probability and can be represented as
\begin{equation}
\mathrm{P}(|t(F, \mathbb{G})-t(F, f( \{\alpha_i\}, \{B_i\}))|>\varepsilon) \leq 2 \exp \left(-\frac{\varepsilon^{2} n}{8 \mathrm{v}(F)^{2}}\right),
\end{equation}
where $\mathrm{v}(F)$ denotes the number of nodes in $F$, and $0 \leq \epsilon \leq 1$. 
\end{theorem}
Theorem~\ref{thm:down} indicates that 
the homomorphism density of graph motifs into the generated graphs in the possible downstream space
can be inherited from our generator to a significant degree.


