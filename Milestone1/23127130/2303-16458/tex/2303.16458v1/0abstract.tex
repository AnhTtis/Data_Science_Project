\begin{abstract}

\renewcommand{\thefootnote}{}\footnote{\textsuperscript{$*$}Both authors contributed equally to this research.}
\renewcommand{\thefootnote}{}\footnote{\textsuperscript{$\dagger$} Corresponding author.}

Recently, graph pre-training has attracted wide research attention, which aims to learn transferable knowledge from unlabeled graph data so as to improve downstream performance. Despite these recent attempts, the negative transfer is a major issue when applying graph pre-trained models to downstream tasks. Existing works made great efforts on {the issue of} \emph{what to pre-train} and \emph{how to pre-train} by designing a number of graph pre-training and fine-tuning strategies. However, there are indeed cases where no matter how advanced the strategy is, the ``pre-train and fine-tune'' paradigm still cannot achieve clear benefits.
This paper introduces a generic framework W2PGNN to answer the crucial question of \emph{when to pre-train} (\emph{i.e.}, in what situations could we take advantage of graph pre-training) before performing effortful pre-training or fine-tuning. 
We start from a new perspective to explore the complex generative mechanisms from the pre-training data to downstream data. 
In particular, W2PGNN first fits the pre-training data into graphon bases, where each element of graphon basis (\emph{i.e.}, a graphon) identifies a fundamental transferable pattern shared by a collection of pre-training graphs. All convex combinations of graphon bases give rise to a generator space, from which graphs generated form the solution space for those downstream data that can benefit from pre-training.
In this manner, the feasibility of pre-training can be quantified as the highest generation probability of the downstream data from any generator in the generator space. 
W2PGNN provides three broad applications, including providing the application scope of graph pre-trained models, quantifying the feasibility of performing pre-training, and helping select pre-training data to enhance downstream performance.
We give a theoretically sound solution for the first application and extensive empirical justifications for the latter two applications.


\end{abstract}