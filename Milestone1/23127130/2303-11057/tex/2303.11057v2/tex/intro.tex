\vspace{-1.5mm}
\section{Introduction}
\label{sec:introduction}
\vspace{-1.5mm}

% trim={left button right up}
\begin{figure}[t]
  \centering
  \includegraphics[width=\linewidth, trim={0cm, 0cm, 0cm, 0cm}, clip]{figs/teaser_modified.pdf}
  \caption{
  \textbf{Deformable Object Manipulation} has many difficulties. 1) It requires \textbf{multiple steps} to complete. 2) Most actions can hardly facilitate tasks, for the \textbf{exceptionally complex states and dynamics}.
  3) Many \textbf{local optimal states} are temporarily closer to the target, but making following actions harder to coordinate for the whole task.
  We propose to learn \textbf{Foresightful Dense Visual Affordance} aware of future actions to avoid local optima for deformable object manipulation, with real-world implementations.
  }
  \label{fig_teaser}
  \vspace{-3mm}
\end{figure}


Many kinds of deformable objects, such as ropes and fabrics, exist everywhere in our daily life.
Perceiving and manipulating deformable objects plays a significant role and paves the way for future home-assistant robots.

Unlike rigid or articulated objects, due to the complex dynamics, high-dimensional and nearly infinite degrees of freedom, large action space, and severe self-occlusion, deformable objects pose much more challenges to manipulate.
Moreover, unlike tasks for rigid objects (like grasping) or articulated objects (like pushing a door) that require one or only a few steps to accomplish, deformable object manipulation tasks usually require many steps to accomplish, laying much focus on relationships and influences between actions in a sequence, as an action leading to local optimal states may not eventually complete the task.

Specifically, as shown in Figure~\ref{fig_teaser}, unfolding crumpled cloth requires a sequence of actions (pick-and-place).
Because of the exceptionally complex states and dynamics, and large action space, most actions fail to facilitate the task.
Moreover, although some cloth in local optimal states temporarily have larger coverage areas than others, following actions face difficulties in smoothly completing the task.

Proposed by Gibson~\cite{gibson1977theory} and aimed at providing indicative information for agents to execute actions (\textit{e.g.}, elementary actions such as picking and pulling) and thus facilitating downstream tasks, visual affordance is arousing much attention in vision and robotics. 
Recent works have demonstrated its efficiency in a large range of tasks like grasping~\cite{mandikal2020graff, montesano2009learning, corona2020ganhand, mia2017affordance, kokic2020learning, zeng2018robotic}, manipulating articulated objects~\cite{mo2021where2act, wu2022vatmart, wang2021adaafford, act_the_part} and assisting robots in a scene~\cite{interaction-exploration, nagarajan2020ego, goff2019building}.
Among them, point-level dense affordance~\cite{wu2022vatmart, mo2021where2act, mo2021o2oafford, wang2021adaafford} learns whether an action on each point of the object could facilitate the task. Compared with Reinforcement Learning (RL) approaches, dense affordance is stably supervised and has better generalization ability towards objects with diverse shapes.%is stably learned with stable ground truth supervision on each point.

The above dense affordance is suitable for representing deformable objects with complex states, capable of indicating whether diverse actions could help complete the task.

While most previous works only study dense affordance for short-term manipulation on rigid~\cite{zhao2022dualafford} or articulated objects~\cite{mo2021where2act, wu2022vatmart},
to tackle the local optima problem in multi-step manipulation,
we move a step towards equipping dense affordance with foresightfulness for future states.

Inspired by Dynamic Programming with Bellman Equation~\cite{bellman1966dynamic} and Q-Learning~\cite{watkins1992q},
estimating a state's \textbf{`value'} (expected return in the long term, instead of only the current performance) for future actions to coordinate and eventually complete the task will help avoid local optimal states and boost the smoothness and quickness of multi-step manipulation.
Dense affordance is suitable for such `value' estimation,
because
such `value' requires understanding and aggregating a large number of diverse following actions on complex states and their corresponding results. 

With such state `value's (instead of only the current performance) in supervisions, dense affordance would gain foresightfulness for the future.

We propose to learn dense visual affordance for manipulating deformable objects,
and further estimate state `value's by aggregating such affordance to avoid local optima and smoothly accomplish multi-step tasks.
As shown in Figure~\ref{fig_teaser} (Down), the task can be accomplished smoothly using our proposed dense affordance in the real world.
To learn such representations, we propose a novel framework generic to diverse tasks with many novel designs, such as the stage-by-stage stable training and \textit{Fold to Unfold} efficient multi-stage data collection.
Thus the proposed affordance could be learned stably, efficiently, and self-supervisedly without hand-crafted policies for different tasks.
Experiments on representative benchmark tasks demonstrate our framework's impressive performance.


\vspace{1mm}
In summary, our contributions are:
\begin{itemize}
    \vspace{-1mm}
    \item We propose to use dense visual affordance for manipulating deformable objects with complex states and dynamics, using such representation to estimate state `value's for future actions to avoid local optima and smoothly accomplish multi-step manipulation tasks;
    \vspace{0mm}
    \item We propose a self-supervised framework with novel designs such as multi-stage training and efficient data collection to learn the proposed affordance stably; 
    \vspace{-3mm}
    \item Qualitative and quantitative results on representative benchmarks and real-world experiments demonstrate the superiority of our proposed dense visual affordance and learning framework for deformable objects.
\end{itemize}
