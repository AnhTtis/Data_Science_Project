\documentclass[10pt,twocolumn,letterpaper]{article}

\usepackage{iccv}
\usepackage{times}
\usepackage{epsfig}
\usepackage{graphicx}
\usepackage{amsmath}
\usepackage{amssymb}
\usepackage{booktabs}
% Include other packages here, before hyperref.

% If you comment hyperref and then uncomment it, you should delete
% egpaper.aux before re-running latex.  (Or just hit 'q' on the first latex
% run, let it finish, and you should be clear).
\usepackage[pagebackref=true,breaklinks=true,letterpaper=true,colorlinks,bookmarks=false]{hyperref}

\iccvfinalcopy % *** Uncomment this line for the final submission

% \def\iccvPaperID{6150} % *** Enter the ICCV Paper ID here
\def\httilde{\mbox{\tt\raisebox{-.5ex}{\symbol{126}}}}

% Pages are numbered in submission mode, and unnumbered in camera-ready
\ificcvfinal\pagestyle{empty}\fi

\begin{document}

%%%%%%%%% TITLE
\title{Learning Foresightful Dense Visual Affordance \\ for Deformable Object Manipulation}


\author{\textbf{Ruihai Wu}$^{1,3,4}$\thanks{Equal contribution, order determined by coin flip.} \quad  \textbf{Chuanruo Ning}$^{2,1}$\footnotemark[1] \quad \textbf{Hao Dong}$ ^{1,3,4}$\thanks{Corresponding author}\\
$^1$CFCS, School of CS, PKU \quad % Peking University \quad
$^2$School of EECS, PKU \quad %Peking University\quad\\
$^3$BAAI \quad
\\$^4$National Key Laboratory for Multimedia Information Processing, School of CS, PKU \quad\\
% {\tt\normalsize \{wuruihai,hao.dong\}@pku.edu.cn},
% {\tt\normalsize chuanruo@stu.pku.edu.cn}
% \\
% \\ \\
% \url{https://hyperplane-lab.github.io/vat-mart}
}

% \author{Ruihai Wu\thanks{Equal contribution, order determined by coin flip.} \qquad \qquad \qquad Chuanruo Ning\footnotemark[1] \qquad \qquad \qquad Hao Dong\thanks{Corresponding author}\\
% CFCS, School of CS, PKU \qquad School of EECS, PKU \qquad CFCS, School of CS, PKU \\
% { BAAI \quad\quad\quad\quad\quad\quad\quad\quad\quad\quad\quad\quad\quad\quad\quad\quad\quad\quad\quad\quad\quad BAAI} \\
% { National Key Laboratory for Multimedia Information Processing \quad National Key Laboratory for Multimedia Information Processing}
%  \\ %Peking University\quad \\
% {\tt\normalsize wuruihai@pku.edu.cn \qquad chuanruo@stu.pku.edu.cn \quad hao.dong@pku.edu.cn}
% % \\
% % \\ \\
% % \url{https://hyperplane-lab.github.io/vat-mart}
% }

% \author{Ruihai Wu\\
% Institution1\\
% Institution1 address\\
% {\tt\small firstauthor@i1.org}
% % For a paper whose authors are all at the same institution,
% % omit the following lines up until the closing ``}''.
% % Additional authors and addresses can be added with ``\and'',
% % just like the second author.
% % To save space, use either the email address or home page, not both
% \and
% Second Author\\
% Institution2\\
% First line of institution2 address\\
% {\tt\small secondauthor@i2.org}
% }

\maketitle
% Remove page # from the first page of camera-ready.
\ificcvfinal\thispagestyle{empty}\fi


%%%%%%%%% ABSTRACT

\begin{abstract}
Contrastive Language-Image Pre-training, benefiting from large-scale unlabeled text-image pairs, has demonstrated great performance in open-world vision understanding tasks. 
However, due to the limited Text-3D data pairs, adapting the success of 2D Vision-Language Models (VLM) to the 3D space remains an open problem. 
Existing works that leverage VLM for 3D understanding generally resort to constructing intermediate 2D representations for the 3D data, but at the cost of losing 3D geometry information. To take a step toward open-world 3D vision understanding, we propose \textbf{C}ontrastive \textbf{L}anguage-\textbf{I}mage-\textbf{P}oint Cloud \textbf{P}retraining (CLIP$^2$) to directly learn the transferable 3D point cloud representation in realistic scenarios with a novel proxy alignment mechanism. 
Specifically, we exploit naturally-existed correspondences in 2D and 3D scenarios, and build well-aligned and instance-based text-image-point proxies from those complex scenarios.
On top of that, we propose a cross-modal contrastive objective to learn semantic and instance-level aligned point cloud representation.
% \xd{too vague description, any more specific name?} objective for 3D point cloud representation learning. 
Experimental results on both indoor and outdoor scenarios show that our learned 3D representation has great transfer ability in downstream tasks, including zero-shot and few-shot 3D recognition, which boosts the state-of-the-art methods by large margins.   %\hjh{give some numbers}. \xh{comparing to XXX}
Furthermore, we provide analyses of the capability of different representations in real scenarios and present the optional ensemble scheme. %\jg{cannot understand the last sentence}. 



\end{abstract}

%We build the proxy alignment based on the observations that the data collections of real-scenarios are usually conducted in a cross-modal way, which naturally align 2D and 3D data by sensor calibration. 
%Specifically, we set a generic caption list as text proxies, then align the relative image proxies by 2D Vision-Language Model and obtain the corresponding 3D proxies, which eventually form the alignment between text-image-3D proxies.
%\jg{We exploit naturally-existed correspondences in paired 2D and 3D scenarios, and build well-aligned and instance-based text-image-point proxies from those complex scenarios.} 

% \vspace{-3mm}

\section{Introduction}

The increasing complexity of source code poses a key challenge to the reliability of large-scale software systems. Software bugs in these systems can lead to safety issues~\cite{bug_safety} for users around the world as well as cause non-negligible financial losses~\cite{bug_loss}. As such, developers have to spend a large amount of time and effort on bug fixing. Consequently, \aprfull (\apr), designed to automatically generate patches to fix software bugs, has attracted wide attention from both academia and industry~\cite{long2016prophet, legoues2012genprog, long2015spr, lou2020can, tufano2018empstudy}. 


To achieve \apr, one popular approach is known as Generate-and-Validate (G\&V)~\cite{qi2015gv, ghanbari2019prapr, lou2020can, le2016hdrepair, legoues2012genprog, wen2018capgen, hua2018sketchfix, martinez2016astor, koyuncu2020fixminder, liu2019tbar, liu2019avatar}, which is typically based on the following pipeline: First, fault localization techniques~\cite{wong2016fl, abreu2007ochiai, zhang2013injecting, papadakis2015metallaxis, li2019deepfl, li2017transforming} are applied to determine the suspicious locations in programs where bugs are likely to exist. Then, the buggy locations are used by the \apr tools to generate a list of patches that replace buggy lines with correct lines. Afterward, each patch is validated against the original test suite to identify any \emph{plausible patches} (i.e., passing all tests in the test suite). Finally, to determine the \emph{correct patches}, developers examine the list of plausible patches to see if any of them can correctly fix the bug. 

Traditional \apr tools can mainly be categorized into heuristic-based~\cite{legoues2012genprog, le2016hdrepair, wen2018capgen}, constraint-based~\cite{mechtaev2016angelix, le2017s3, demacro2014nopol, long2015spr} and \template~\cite{ghanbari2019prapr, hua2018sketchfix, martinez2016astor, liu2019tbar, liu2019avatar}. Among these traditional tools, \template \apr tools~\cite{ghanbari2019prapr, liu2019tbar, benton2020effectiveness} have been able to achieve state-of-the-art results. \Template \apr tools typically leverage pre-defined templates (e.g., adding a nullness check) for bug fixing. However, since these fix templates are typically handcrafted, the number and types of bugs they are able to fix can be limited. 



To address the limitations of traditional \apr, researchers have proposed various \learning \apr tools~\cite{li2020dlfix, chen2018sequencer, jiang2021cure, lutellier2020coconut, zhu2021recoder, ye2022rewardrepair} based on the \nmtfull (\nmt) architecture~\cite{sutskever2014mt} where the input is the buggy code snippets and the goal is to translate the buggy code snippets into a fixed version. To accomplish this, \learning \apr tools require supervised training datasets with pairs of both buggy and fixed code snippets in order to learn how to perform this translation step. These training data are usually obtained by mining historical bug fixes using heuristics/keywords~\cite{dallmeier2007benchmark}, which can be imprecise for identifying bug-fixing commits; even the actual bug-fixing commits can include irrelevant code changes, leading to further pollution in the dataset~\cite{xia2022alpharepair}.
% 
Moreover, it can be hard for such \apr tools to generalize and fix bug types unseen during training. 



To better leverage recent advances in \plmfull{s} (\plm{s}), researchers~\cite{xia2022alpharepair, xia2023repairstudy, kolak2022patch, prenner2021codexws} have directly applied \plm{s} to generate patches without bug-fixing datasets. These \llm-based \apr tools work by either directly generating a complete code function~\cite{prenner2021codexws, xia2023repairstudy} or predict/infill the correct code snippet given its surrounding context~\cite{xia2022alpharepair, xia2023repairstudy}. By directly using \llm{s} that are pre-trained on billions of open-source code snippets, \llm-based \apr tools can achieve state-of-the-art performance on many repair datasets~\cite{xia2022alpharepair}. 


% 
%
%

Traditional \apr tools have long used the insight of the \emph{plastic surgery hypothesis}~\cite{barr2014plastic} where it states that the code ingredients to fix a bug already exist within the same project. Traditional \apr tools have manually designed pattern-~\cite{ghanbari2019prapr, saha2017elixir} or heuristic-based~\cite{jiang2018simfix, legoues2012genprog} approaches to finding and using such relevant code ingredients to generate fixes for bugs. However, the plastic surgery hypothesis has been largely ignored in \llm-based \apr. In fact, \llm provides a unique opportunity to fully automate the plastic surgery hypothesis idea via fine-tuning (learning project-specific information via model updates from the buggy project) and prompting (directly providing relevant code ingredients to the model), and make it directly applicable to different languages (since the \llm{s} are typically multi-lingual).%
Moreover, despite the intensive manual efforts involved, traditional \apr tools still cannot fully leverage project-specific information due to large search space for leveraging/composing existing code ingredients. In contrast, the project-specific information can effectively leveraged by \llm{s} due to their power in code understanding/vectorization, e.g., even partial/imprecise information may still guide \llm{s} in correct patch generation!
 To this end, we ask the question: \emph{How useful is the plastic surgery hypothesis in the era of \plm{s}}?








\mypara{Our Work.} To answer the question, we present \ourtech{\xspace} -- a \llm-based approach that automatically utilizes the plastic surgery hypothesis by systematically combining multiple fine-tuning and prompting strategies for \apr. \ourtech fine-tunes \plm{s} using two novel domain-specific training strategies: \textbf{\epfinetune} -- we fine-tune using the original buggy project by aggressively masking out a high percentage of tokens, which allows \plm to learn project-specific code tokens and programming styles; and \textbf{\rofinetune} -- which only masks out a single continuous code sequence per training sample, allowing the model to get used to the final \csapr task of predicting a single continuous code sequence. Furthermore, we directly leverage the ability for \plm{s} to understand natural language instructions and introduce a novel prompting strategy, \textbf{\idprompting}, which uses information retrieval and static analysis to obtain a list of relevant identifiers for the buggy lines. While such relevant identifiers are critical for fixing some difficult bugs, they may not be seen by the \llm during inference due to limited context window size. Through the use of prompting, we directly tell the model to use these extracted identifiers (relevant code ingredients) to generate the correct code. Finally, to perform repair, we combine all four model variants (including the base model, both fine-tuned models and the base model with prompting) for the final repair.





While our insight of leveraging the plastic surgery hypothesis for \llm-based \apr is generalizable across different types of \plm{s}, to implement \ourtech, we choose a recent \plm{\xspace}, \ctfive~\cite{wang2021codet5}, which is pre-trained on millions of open-source code snippets. \ctfive is an encoder-decoder model trained using \mspfull (\msp) objective where a percentage of tokens are masked out and each continuous masked token sequence is referred to as a masked span. Also, although we only extract relevant identifiers from the current buggy project (since this paper focuses on the plastic surgery hypothesis), our work can be easily extended to obtain other code information (such as relevant statements or functions) from other sources, such as  the massive pre-training corpora~\cite{husain2020codesearchnet} or historical bug-fixing datasets~\cite{jiang2019infer}, which can provide more coding knowledge for \llm{s}. Besides, although we mainly focus on using traditional string comparison algorithms for information retrieval in this paper, these techniques can be easily replaced by other frequency-based retrieval~\cite{robertson2009probabilistic} and neural search (or embedding-based search)~\cite{reimers2019sentence}.
  In summary, this paper makes the following contributions:


%


\begin{itemize}[noitemsep, leftmargin=*, topsep=0pt]
    \item \textbf{Dimension.} This paper is the first to revisit the important plastic surgery hypothesis in the era of \llm{s}. It opens up a new dimension for \llm-based \apr to incorporate previously neglected information from the buggy project itself to boost \apr performance. Furthermore, it demonstrates the promising future of retrieval-based prompting for modern \llm-based \apr.
    \item \textbf{Implementation.} We implement \ourtech based on the recent \ctfive model. We augment the model using two novel fine-tuning strategies: \epfinetune and \rofinetune, along with a novel prompting strategy based on information retrieval and static analysis: \idprompting. We combine the patches generated by all four models together and perform patch ranking to speed up \apr.% 
    \item \textbf{Evaluation Study.} We conduct an extensive evaluation against state-of-the-art \apr tools. On the widely studied \dfj 1.2 and 2.0 datasets~\cite{just2014dfj}, \ourtech is able to achieve the new state-of-the-art results of 89 and 44 correct bug fixes (15 and 8 more than best baseline) respectively.  Furthermore, we perform a broad ablation study to justify our design. \ourtech demonstrates for the first time that the plastic surgery hypothesis can substantially boost \llm-based \apr and advance state-of-the-art \apr, while being fully automated and general. Moreover, even partial/imprecise code ingredients may still effectively guide \llm{s} for \apr!
\end{itemize}


\section{Related work}
\noindent \textbf{Video foundation models.}
With sufficient computational power and an abundant source of data, there have been attempts to build a single large-scale foundation model that can be adapted to diverse downstream tasks.
Along with the success of foundations models in the natural language processing domain~\cite{brown2020language,chen2021evaluating,devlin2019bert} and in computer vision~\cite{bertasius2021space,jia2021scaling,radford2021learning}, video data has become another data type of interest, as it has grown in scale due to numerous internet video-sharing platforms.
Accordingly, several methods to train a video foundation model have been proposed.
Due to the innate multi-modality of video data, \textit{i.e.}, a combination of visual $\cdot$ vocal $\cdot$ textual context, most works have centered around the variations of the cross-modal attention mechanism \cite{akbari2021vatt,bertasius2021space,gabeur2020multi,luo2020univl,neimark2021video,tan2021look,wei2020multi,yang2021taco}.
In addition, as most video data lack proper labels or descriptions, contrastive learning methods were studied to learn meaningful feature representations or enhance video-text alignment in a self-supervised manner \cite{akbari2021vatt,kuang2021video,luo2020univl,yang2021taco}.

More specifically, MERLOT \cite{zellers2021merlot} proposed a multi-modal representation learning method for visual commonsense reasoning, which also performed well in twelve video reasoning tasks.
VATT \cite{akbari2021vatt} introduced a multi-modal learning method via contrastive learning. 
The pre-trained model performed well in a variety of vision tasks from image classification to video action recognition and zero-shot video retrieval.
Another representative work, UniVL \cite{luo2020univl} proposed a straightforward pre-training method with auxiliary loss functions. 
After fine-tuning on a specific task, the pre-trained model performed outstandingly in a wide range of tasks of text-to-video retrieval, action segmentation, action step localization, video sentiment analysis, and video captioning.
Other foundation models for multiple video tasks include \cite{li2020hero,sun2019learning,sun2019videobert,zhu2020actbert,fu2021violet,wang2022all}. 

\noindent \textbf{Auxiliary learning.}
In order to enhance the performance of one or a multitude of primary tasks, auxiliary learning methods can be incorporated.
\cite{ruder2017overview} introduced Multi-task learning (MTL) to the deep neural networks by training a single model with multiple task losses to assist learning on the main task.
Such a method is generally adapted to pre-train the foundation models in the self-supervised manner~\cite{li2020hero,sun2019learning,sun2019videobert,zhu2020actbert,fu2021violet,wang2022all}.
However, these various pretext task losses used in the pre-training phase are ignored in the fine-tuning phase, and only the primary task loss is minimized.

Recently, meta-learning methods have been introduced for auxiliary learning.
\cite{liu2019self,navon2020auxiliary,shu2019meta} proposed a meta-learning method in which the model learns auxiliary tasks to generalize well to unseen data. 
In these settings, a separate subset of data is held out as the primary task, while the others are used as auxiliary tasks that aid the primary task's performance.
Similar methods were adopted for computer vision tasks such as semantic segmentation \cite{xu2021leveraging}.
Other domain applications include navigation tasks with reinforcement learning \cite{ye2021auxiliary}, or self-supervised learning methods on graph data \cite{hwang2020self}.
\section{The Semi-Oblivious Chase Procedure}\label{sec:semi}
%

The semi-oblivious chase (or simply chase) takes as input a database $D$ and a set $\dep$ of TGDs, and constructs an instance that contains $D$ and satisfies $\dep$.
%
A central notion in this context is that of trigger.
%are those of trigger, active trigger, and trigger application.

\begin{definition}%[\textbf{Trigger Application}]
	Given a set $\dep$ of TGDs and an instance $I$, a {\em trigger} for $\dep$  on $I$ is a pair $(\sigma,h)$, where $\sigma \in \dep$ and $h$ is a homomorphism from $\body{\sigma}$ to $I$.
	%
	The {\em result} of $(\sigma,h)$, denoted $\result{\sigma}{h}$, is the set $\mu(\head{\sigma})$, where $\mu : \var{\head{\sigma}} \ra \ins{C} \cup \ins{N}$ is defined as follows:
	%
	%$\mu(x) = h(x)$ if $x \in \fr{\sigma}$, and $\mu(x) = \bot_{\sigma,h_{|\fr{\sigma}}}^{x}$ otherwise,
	\[
	\mu(x)\
	=\ \left\{
	\begin{array}{ll}
	h(x) & \quad \text{if } x \in \fr{\sigma}\\
	&\\
	\bot_{\sigma,h_{|\fr{\sigma}}}^{x} & \quad \text{otherwise}
	\end{array} \right.
	\]
	where $\bot_{\sigma,h_{|\fr{\sigma}}}^{x} \in \ins{N}$.  Let $T(\dep,I)$ be the set of triggers for $\dep$ on $I$.	\hfill\markfull
\end{definition}




Observe that in the definition of $\result{\sigma}{h}$, each existentially quantified variable $x$ of $\head{\sigma}$ is mapped by $\mu$ to a null value of $\ins{N}$ whose name is uniquely determined by the trigger $(\sigma,h)$ and the variable $x$ itself. This means that, given a trigger $(\sigma,h)$, we can unambiguously construct the set of atoms $\result{\sigma}{h}$.
%
The central idea of the chase is, starting from a database $D$, to exhaustively apply triggers for the given set $\dep$ of TGDs on the instance constructed so far.
%
More precisely, given a database $D$ and a set $\dep$ of TGDs, let
\[
\mathsf{chase}^{0}(D,\dep)\ =\ D,
\]
and for each $i>0$, let
\[
\mathsf{chase}^{i}(D,\dep)\ =\ \mathsf{chase}^{i-1}(D,\dep)\ \cup\ \bigcup_{(\sigma,h) \in S} \result{\sigma}{h},
\]
where $S = T(\dep,\mathsf{chase}^{i-1}(D,\dep))$. 
%
We finally define {\em the result of the chase of $D$ w.r.t.~$\dep$} as the (possibly infinite) instance
\[
\chase{D}{\dep}\ =\ \bigcup_{i \geq 0} \mathsf{chase}^{i}(D,\dep).
\]


\ignore{
The semi-oblivious chase procedure (or simply chase) takes as input a database $D$ and a set $\dep$ of TGDs, and constructs an instance that contains $D$ and satisfies $\dep$.
%
Central notions in this context are those of trigger, active trigger, and trigger application.

\begin{definition}%[\textbf{Trigger Application}]
	Given a set $\dep$ of TGDs and an instance $I$, a {\em trigger} for $\dep$  on $I$ is a pair $(\sigma,h)$, where $\sigma \in \dep$ and $h$ is a homomorphism from $\body{\sigma}$ to $I$.
	%
	The {\em result} of $(\sigma,h)$, denoted $\result{\sigma}{h}$, is the set $\mu(\head{\sigma})$, where $\mu : \var{\head{\sigma}} \ra \ins{C} \cup \ins{N}$ is defined as follows:
	%
	%$\mu(x) = h(x)$ if $x \in \fr{\sigma}$, and $\mu(x) = \bot_{\sigma,h_{|\fr{\sigma}}}^{x}$ otherwise,
	\[
	\mu(x)\
	=\ \left\{
	\begin{array}{ll}
	h(x) & \quad \text{if } x \in \fr{\sigma}\\
	&\\
	\bot_{\sigma,h_{|\fr{\sigma}}}^{x} & \quad \text{otherwise}
	\end{array} \right.
	\]
	where $\bot_{\sigma,h_{|\fr{\sigma}}}^{x}$ is a null value from $\ins{N}$.
	%
	The trigger $(\sigma,h)$ is {\em active} if $\result{\sigma}{h} \not\subseteq I$.
	%
	The {\em application} of $(\sigma,h)$ to $I$ returns the instance $J = I \cup \result{\sigma}{h}$ and is denoted as $I \app{\sigma}{h} J$.
	\hfill\markfull
\end{definition}


Observe that in the definition of $\result{\sigma}{h}$ above, each existentially quantified variable $x$ of $\head{\sigma}$ is mapped by $\mu$ to a null value of $\ins{N}$ whose name is uniquely determined by the trigger $(\sigma,h)$ and the variable $x$ itself. This means that, given a trigger $(\sigma,h)$, we can unambiguously extract the set of atoms 
$\result{\sigma}{h}$.



%\medskip

%\noindent
%\textbf{Semi-Oblivious Chase.}
The central idea of the chase is, starting from a database $D$, to exhaustively apply active triggers for the given set $\dep$ of TGDs on the instance constructed so far. This is formalized via the notion of (semi-oblivious) chase derivation, which can be finite or infinite.


\begin{definition}
	Consider a database $D$ and a set $\dep$ of TGDs.
	%We consider the two cases where a derivation is finite or infinite:
	\begin{itemize}
		\item A finite sequence $(I_i)_{0 \leq i \leq n}$ of instances, with $D = I_0$ and $n \geq 0$, is a {\em chase derivation} of $D$ w.r.t.~$\dep$ if, for each $i \in \{0,\ldots,n-1\}$, there is an active trigger $(\sigma,h)$ for $\dep$ on $I_i$ with $I_i \app{\sigma}{h} I_{i+1}$, and there is no active trigger for $\dep$ on $I_n$. The {\em result} of such a chase derivation is the instance $I_n$.
		
		
		\item An infinite sequence $(I_i)_{i \geq 0}$ of instances, with $D = I_0$, is a {\em chase derivation} of $D$ w.r.t.~$\dep$ if, for each $i \geq 0$, there is an active trigger $(\sigma,h)$ for $\dep$ on $I_i$ such that $I_i \app{\sigma}{h} I_{i+1}$. Moreover, $(I_i)_{i \geq 0}$ is {\em fair} if, for each $i \geq 0$, and for every active trigger $(\sigma,h)$ for $\dep$ on $I_i$, there exists $j > i$ such that $(\sigma,h)$ is not an active trigger for $\dep$ on $I_j$. 
		%The latter is known as the {\em fairness condition}, and guarantees that all the active triggers will be deactivated. %
		The {\em result} of such a chase derivation is the instance $\bigcup_{i \geq 0} \, I_i$.
	\end{itemize}
	%
	%The {\em result} of a chase derivation is defined as the union of all the instances occurring in it. 
	A chase derivation is {\em valid} if it is finite or infinite and fair.  \hfill\markfull
\end{definition}


Let us stress that infinite but unfair chase derivations are not considered as valid ones since they do not serve the main purpose of the chase, that is, to build an instance that satisfies the given set of TGDs. Indeed, given the set $\dep$ consisting of the TGDs
\[
\sigma\ =\ R(x,y) \ra \exists z \, R(y,z) \qquad \sigma'\ =\ R(x,y) \ra P(x,y),
\]
the result of the unfair chase derivation of $D = \{R(a,b)\}$ w.r.t.~$\dep$ that involves only triggers of the form $(\sigma,\cdot)$, i.e., only the TGD $\sigma$ is used, does not satisfy $\sigma'$, and thus, it does not satisfy $\dep$.
%
Interestingly, for every database $D$ and set $\dep$ of TGDs, any two valid chase derivations of $D$ w.r.t.~$\dep$ have always the same result, which implies that all valid chase derivations are either finite or infinite~\cite{GrOn18}. Therefore, in the rest of the paper, we can safely refer to {\em the} result of the chase of $D$ w.r.t. $\dep$, which we will denote by $\chase{D}{\dep}$. 
}


%\subsection{Non-Uniform Chase Termination}\label{sec:problem}
%

\medskip

\noindent
\textbf{Chase Termination.}
The result of the chase may be infinite even for very simple settings: it is easy to see that for $D = \{R(a,b)\}$ and $\dep = \{R(x,y) \ra \exists z \, R(y,z)\}$, $\chase{D}{\dep}$ is infinite.
%; in particular, $\chase{D}{\dep} = \{R(a,b),R(b,\bot_1),R(\bot_1,\bot_2),R(\bot_2,\bot_3),\ldots\}$, where $\bot_1,\bot_2,\ldots$ are null values.
%
This leads to the following problem, parameterized by a class $\class{C}$ of TGDs such as $\class{SL}$ (the class of simple-linear TGDs) and $\class{L}$ (the class of linear TGDs):


\medskip

\begin{center}
	\fbox{
		\begin{tabular}{ll}
			%{\small PROBLEM} : & %$\mathsf{ChaseTermination}(\class{C})$
			%\\
			{\small INPUT} : & A database $D$ and a set $\dep$ of TGDs from $\class{C}$.
			\\
			{\small QUESTION} : &  Is the instance $\chase{D}{\dep}$ finite?
	\end{tabular}}
\end{center}

\medskip

\noindent This problem has been recently studied in~\cite{CaGP22} for the classes of simple-linear and linear TGDs. Interestingly, for both classes, the finiteness of the result of the chase has been syntactically characterized by exploiting the notion of non-uniform weak-acyclicity. 
%
We proceed to recall this acyclicity notion, and then present the characterizations established in~\cite{CaGP22}, which in turn lead to simple algorithms for checking the finiteness of the result of the chase.
%
Note that, for the sake of clarity, in the rest of the paper we assume TGDs with a non-empty frontier, i.e., we assume that there is at least one variable in a TGD $\sigma$ that occurs both in $\body{\sigma}$ and $\head{\sigma}$. This assumption can be made without loss of generality since, given a database $D$ and a set $\dep$ of TGDs, we can easily construct a set $\dep'$ of TGDs with a non-empty frontier by slightly modifying $\dep$ such that $\chase{D}{\dep}$ is finite iff $\chase{D}{\dep'}$ is finite.


\medskip

\noindent
\textbf{Non-Uniform Weak-Acyclicity.} Weak-acyclicity was introduced in~\cite{FKMP05} as the main formalism for data exchange purposes, which guarantees the finiteness of the result of the chase for {\em every} input database. Non-uniform weak-acyclicity is the database-dependent variant of weak-acyclicity introduced in~\cite{CaGP22}. We proceed to give the formal definitions.
%
We first need to recall the notion of the {\em dependency graph} of a set $\dep$ of TGDs, 
%which symbolically encodes how terms may propagate during the chase.
%The {\em dependency graph} of set $\dep$ of TGDs 
defined as a directed multigraph $\depg{\dep}=(N,E)$, where $N = \pos{\sch{\dep}}$ and $E$ contains {\em only} the following edges.
%
For each TGD $\sigma \in \dep$ with $\head{\sigma} = \{\alpha_1,\ldots,\alpha_k\}$, for each $x \in \frontier{\sigma}$, and for each position $\pi \in \posvar{\body{\sigma}}{x}$:
\begin{itemize}
	\item For each $i \in [k]$ and for each $\pi' \in \posvar{\alpha_i}{x}$, there exists a \emph{normal} edge $(\pi,\pi') \in E$.
	%
	\item For each existentially quantified variable $z$ in $\sigma$, $i \in [k]$, and $\pi' \in \posvar{\alpha_i}{z}$, there is a \emph{special} edge $(\pi,\pi') \in E$.
\end{itemize}
%
We further need to define when a predicate is reachable from another predicate. 
%
Given predicates $R,P \in \sch{\dep}$, {\em $P$ is reachable from $R$ (w.r.t.~$\dep$)} if $R = P$, or there exists a path in $\depg{\dep}$ from a position of the form $(R,i)$ to a position of the form $(P,j)$.
%
%we write $R \ra_\dep P$  if $R = P$, or there exists a TGD $\sigma \in \dep$ such that $R$ occurs in $\body{\sigma}$ and $P$ occurs in $\head{\sigma}$. We say that {\em $P$ is reachable from $R$ (w.r.t.~$\dep$)}, denoted $R \reach{\dep} P$, if (i) $R \ra_\dep P$, or (ii) there exists $T \in \sch{\dep}$ such that $R \reach{\dep} T$ and $T \ra_\dep P$.
%in $\depg{\dep}$, denoted $R \reach{\dep} P$, if there exists a path in $\depg{\dep}$ from a position $(R,i)$ to a position $(P,j)$, for some $i \in [\arity{R}]$ and $j \in [\arity{P}]$.
Given a database $D$, we say that a (not necessarily simple and possibly cyclic) path $C$ in $\depg{\dep}$ is \emph{$D$-supported} if there exists an atom $R(\bar t) \in D$ and a node of the form $(P,i)$ in $C$ such that $P$ is reachable from $R$.
%
We are now ready to recall (non-uniform) weak-acyclicity.



\begin{definition}\label{def:dwa}
	Consider a database $D$ and a set $\dep$ of TGDs. We say that $\dep$ is {\em weakly-acyclic w.r.t.~$D$}, or {\em $D$-weakly-acyclic}, if there is no $D$-supported cycle in $\depg{\dep}$ with a special edge. 
	%
	We say that $\dep$ is {\em weakly-acyclic} if there is no cycle in $\depg{\dep}$ with a special edge. \hfill\markfull
\end{definition}


\smallskip

\noindent
\textbf{Characterizing the Finiteness of the Chase.}
It is not very difficult to show that whenever a set $\dep$ of TGDs (not necessarily linear) is $D$-weakly-acyclic, then the instance $\chase{D}{\dep}$ is finite. In other words, the $D$-weak-acyclicity of $\dep$ is a sufficient condition for the finiteness of $\chase{D}{\dep}$. What is more interesting is that, assuming that $\dep$ is a set of simple-linear TGDs, the $D$-weak-acyclicity of $\dep$ is also a necessary condition for the finiteness of $\chase{D}{\dep}$. This leads to the following characterization established in~\cite{CaGP22}:

\begin{theorem}\label{the:characterization-simple-linear}
	Consider a database $D$ and a set $\dep \in \class{SL}$ of TGDs. It holds that $\chase{D}{\dep}$ is finite iff $\dep$ is $D$-weakly-acyclic.
\end{theorem}

For linear TGDs, it turned out that non-uniform weak-acyclicity is not powerful enough for characterizing the finiteness of the chase instance. Here is an example given in~\cite{CaGP22} that illustrates this fact:
%This is illustrated by the following example.


\begin{example}
	Consider the database $D = \{R(a,b)\}$ and the singleton set $\dep$ consisting of the (non-simple) linear TGD
	\[
	R(x,x)\ \ra\ \exists z \, R(z,x). 
	\]
	It is easy to see that there is no trigger for $\dep$ on $D$. This means that $\chase{D}{\dep} = D$ is finite, whereas $\dep$ is {\em not} $D$-weakly-acyclic. \hfill\markfull
\end{example}


To obtain a characterization analogous to Theorem~\ref{the:characterization-simple-linear}, the authors of~\cite{CaGP22} used the technique of {\em simplification} to convert linear TGDs into simple-linear TGDs, while preserving the finiteness of the chase instance. We proceed to recall this technique.
%
Let $\bar t = (t_1,\ldots,t_n)$ be a tuple of (not necessarily distinct) terms. We write $\unique{\bar t}$ for the tuple obtained from $\bar t$ by keeping only the first occurrence of each term in $\bar t$.
%
For example, if $\bar t = (x,y,x,z,y)$, then $\unique{\bar t} = (x,y,z)$.
%
For each $i \in [n]$, the \emph{identifier of $t_i$ in $\bar t$}, denoted $\id{\bar t}{t_i}$, is the integer that identifies the position of $\unique{\bar t}$ at which $t_i$ appears. 
%
We write $\id{}{\bar t}$ for the tuple $(\id{\bar t}{t_1},\ldots,\id{\bar t}{t_n})$.
%
For example, if $\bar t = (x,y,x,z,y)$, then $\id{}{\bar t} = (1,2,1,3,2)$.
%
For an atom $\alpha = R(\bar t)$, the {\em simplification of $\alpha$}, denoted $\simple{\alpha}$, is the atom $R_{\id{}{\bar t}}(\unique{\bar t})$, whereas the {\em shape of $\alpha$}, denoted $\shape{\alpha}$, is the predicate $R_{\id{}{\bar t}}$. We can naturally refer to the simplification and the shape of a set of atoms.
%
For a tuple of variables $\bar x = (x_1,\ldots,x_n)$, a \emph{specialization of $\bar x$} is a function $f$ from $\bar x$ to $\bar x$ such that $f(x_1) = x_1$, and $f(x_i) \in \{f(x_1),\ldots,f(x_{i-1}),x_i\}$, for each $i \in \{2,\ldots,n\}$.
We write $f(\bar x)$ for $(f(x_1),\ldots,f(x_n))$. We are now ready to recall how a set of linear TGDs is converted into a set of simple-linear TGDs.

\begin{definition}\label{def:simplification}
	Consider a linear TGD $\sigma$ of the form
	\[
	R(\bar x) \ra \exists \bar z\, \psi(\bar y,\bar z), 
	\]
	where $\bar y \subseteq \bar x$, and a specialization $f$ of $\bar x$. The {\em simplification of $\sigma$ induced by $f$} is the simple-linear TGD
	\[
	\simple{R(f(\bar x))} \rightarrow \exists \bar z\, \simple{\psi(f(\bar y),\bar z)}.
	\]
	We write $\simple{\sigma}$ for the set of all simplifications of $\sigma$ induced by some specialization of $\bar x$.
	%
	For a set $\dep \in \class{L}$ of TGDs, the {\em simplification of $\dep$} is defined as the set
	\[
	\simple{\dep}\ =\ \bigcup_{\sigma \in \dep} \simple{\sigma}
	\]
	consisting only of simple-linear TGDs. \hfill\markfull
\end{definition}

We can now recall the characterization for the finiteness of the chase instance for linear TGDs, established in~\cite{CaGP22}, which is similar to the one for simple-linear TGDs, with the key difference that first we need to simplify both the database and the set of linear TGDs:

\begin{theorem}\label{the:characterization-linear}
	Consider a database $D$ and a set $\dep \in \class{L}$ of TGDs. Then, $\chase{D}{\dep}$ is finite iff $\simple{\dep}$ is $\simple{D}$-weakly-acyclic.
\end{theorem}

It is clear that Theorems~\ref{the:characterization-simple-linear} and~\ref{the:characterization-linear} provide simple algorithms for checking whether the chase instance is finite. In particular, given a database $D$ and a set $\dep$ of simple-linear TGDs, we simply need to check whether $\dep$ is $D$-weakly-acyclic, in which case the algorithm returns \true; otherwise, it returns \false. The same holds when $\dep$ is a set of linear TGDs, with the difference that the algorithm first needs to simplify $D$ and $\dep$, and then perform the acyclicity check.
%
Our goal is to experimentally evaluate the above algorithms with the aim of understanding which input parameters affect their performance, clarifying whether they can be applied in a practical context, and revealing their performance limitations. Of course, a naive implementation of the above algorithms, especially for linear TGDs where the expensive simplification must be applied, will lead to poor performance, and thus, will not be very useful towards our goal. Hence, we need to somehow convert the above theoretical algorithms into practical algorithms that are amenable to efficient implementations. This is the subject of the next section.
\section{Method}
\label{sec:method}

% \ml{``Inconsistent'' to ``large variation''}

% In this section, we propose our methods based on the observations in Section \ref{sec:motivation}.
In this section, we propose two techniques to further enhance the strong baseline to capture the variation of activation distributions better.
We first introduce spatial re-scaling to adapt the network to pixel-to-pixel variation.
We then propose channel-wise shifting and re-scaling to better capture the channel-to-channel variation.
Meanwhile, as both of the two methods are image-dependent, the image-to-image variation can be captured naturally.
By combining the two methods with our strong baseline, we build our enhanced BNN for SR, named EBSR.

% Because the activation distributions among pixels, channels and images have large variations \red{**are highly inconsistent} in SR networks, we introduce spatial re-scaling to adapt to pixel-wise variations and channel shift and re-scaling to adapt to channel-wise variations. And both of them are image-dependent to adapt to image-wise variations, which means during inference our network re-scales and shifts the distributions of activations flexibly for different input images. Based on these methods, we build an enhanced binary neural network for image super-resolution (EBSR).

% According to [3], the difference of activation magnitudes indicates different scaling factors are needed for each pixel.

\subsection{Spatial Re-scaling}
% It is better to use different scaling factors for different pixels to reduce the quantization error and retain more detailed information for image super-resolution. 

% \ml{In the main method, we do not need to introduce the previous works but can focus on introducing our own method. Channel rescaling in Real-to-binary Net is not relevant in this context.}

% Re-scaling the output of binary convolutions was proposed at the birth of BNN in XNOR-Net \cite{rastegari2016xnor} to reduce quantization error and improve accuracy for image classification tasks.
% It is computed as below:
% \begin{equation}
% \mathcal{A} * \mathcal{W} \approx(\operatorname{sign}(\mathcal{A}) \circledast \operatorname{sign}(\mathcal{W})) \odot \mathcal{K} \alpha
% \label{eq:xnor-net rescale}
% \end{equation}
% where $\circledast$ denotes the binary convolution and $\odot$ denotes the element-wise multiplication.
% $\mathcal{A}$, $\mathcal{W}$, $\alpha$, and $\mathcal{K}$ denote the activation, weight, weight scaling factor, and activation scaling factor, respectively.
%  Later in XNOR-Net++ \cite{bulat2019xnor}, Bulat et al. fuse the activation and weight scaling factors into a single one that is learned end-to-end based on gradients and this improves the classification accuracy on ImageNet dataset.

% % It is computed as Eq.~\ref{eq:xnor-net rescale}, where $\circledast$ denotes 
% %  the binary convolution and $\odot$ denotes the element-wise multiplication. The binary convolution of $\mathcal{A}$ and $\mathcal{W}$ is rescaled by the weight scaling factor $\alpha$ and the activation scaling factor $\mathcal{K}$, both of which are calculated analytically.


% \zc{Similarly, you should explain the meaning of A, W and the operators $\circledast$ in the formula}
% Then in Real-to-binary Net \cite{martinez2020training}, Martinez et al. used a data-driven channel re-scaling module that takes the pre-convolution activations as input to predict the activation scaling factor. Unlike that in XNOR-Net++ \cite{bulat2019xnor}, these scaling factors are not fixed during inference but rather inferred from data. By doing this, they further improved the classification accuracy on ImageNet over XNOR-Net++. 
As is shown in Figure \ref{fig:pixel}, activation distributions have large pixel-to-pixel variation in SR networks
and the difference of activation magnitudes indicates different scaling factors are preferred for different pixels.
Inspired by \cite{martinez2020training}, we propose spatial re-scaling to better adapt the network to the spatial variation
of activation distributions in SR networks.
% fit the various pixel-wise distributions in SR networks.
We take the real-valued activations $A$ before convolution as input and predict pixel-wise scaling factors $S(A)$, which re-scale the binary convolution output. Spatial re-scaling process can be formulated as follows:
\begin{equation}
A * W \approx(\operatorname{sign}(A) \circledast \operatorname{sign}(W)) \odot \alpha \odot S(A)
\label{eq:spatial rescale}
\end{equation}
where $\circledast$ denotes 
the binary convolution and $\odot$ denotes the element-wise multiplication. $A$, $W$, $\alpha$, and $S\left(A\right)$ denote real-valued activations, weights, the scaling factor of weights, and the spatial-wise scaling factor of activations respectively. $S\left(A\right) \in \mathbb{R}^{1\times H\times W}$ can be calculated with a convolution and a sigmoid function.
% as $\sigma\left( CONV\left(A\right)\right)$. 
As shown in Figure \ref{fig:method}(a), real-valued activations first go through a convolution layer,
which has an input channel of $C$ and an output channel of 1, 
and then pass through a sigmoid function to produce the scaling factors $S(A)$ along the spatial dimension.
During inference, the scaling factor will change dynamically according to different input feature maps.
By re-scaling binary convolution output using $S(A)$, we can reduce the quantization error and the original pixel-wise information in FP activation
will be preserved much better.
Spatial re-scaling leads to a large PSNR improvement of 0.24 dB (from 30.30 dB to 31.54 dB) on Set5 and 0.22 dB (from 25.09 dB to 25.31 dB)
on Urban100 compared with our strong baseline. 

\subsection{Channel-wise Shifting and Re-scaling}

\begin{table}[!tb]
\centering
\caption{Comparison between whether to fuse channel-wise shifting and re-scaling or not based on our baseline with spatial re-scaling. }
\label{tab:fusing}

\scalebox{0.65}{
\begin{tabular}{c|cc|cc|cc}
\hline
\multirow{2}{*}{Method}     & \multirow{2}{*}{OPs} & \multirow{2}{*}{Params} & \multicolumn{2}{c|}{Set5} & \multicolumn{2}{c}{Urban100} \\ \cline{4-7} 
                            &                      &                         & PSNR        & SSIM        & PSNR          & SSIM         \\ \hline
Baseline + spatial re-scale & 2.16G                & 0.05M                   & 31.54       & 0.883       & 25.31         & 0.759        \\
+ channel-wise shift and re-scale             & 2.34G                & 0.09M                   & 31.61       & 0.885       & 25.35         & 0.761        \\
+ w/ fusing                   & 2.27G                & 0.08M                   & \textbf{31.64}       & \textbf{0.885}       & \textbf{25.36}         & \textbf{0.761}        \\ \hline
\end{tabular}
}
\end{table}

In SR networks, activation distributions exhibit larger channel-to-channel variation (Figure \ref{fig:chl}).
Both the mean and magnitude of the activation distributions vary significantly across channels.
% Thus we use channel-wise shifting and re-scaling to adapt to various channel-wise distributions. 
\cite{martinez2020training} has proposed the data-driven channel re-scaling, 
but our method differs from them in further introducing data-driven thresholds to handle the channel-wise variation of both mean and magnitude.
Since the blocks to generate the scaling factors and thresholds are very similar, we further propose to fuse them into one module.
% and fusing channel-wise shifting and re-scaling into one module.
We evaluate the effect of fusing the two blocks in Table \ref{tab:fusing}.
With channel-wise shifting and re-scaling fused, our models have fewer operations and parameters overhead and slightly higher performance.

For the specific process, we take the real-valued activations as input and predict different thresholds and scaling factors for each channel. They are also image dependent, e.g., $\beta_{i}$ in Eq.\ref{eq:act_binarize} is no longer fixed during inference but generated according to different input feature maps. Channel-wise shifting and re-scaling can be formulated as follows:
\begin{equation}
A * W \approx(\operatorname{sign}(A-C_s(A)) \circledast \operatorname{sign}(W)) \odot \alpha \odot C_r(A)
\label{eq:channel-wise_shift_and_rescale}
\end{equation}
where $\circledast$ denotes 
the binary convolution and $\odot$ denotes the element-wise multiplication. $C_s(A), C_r(A) \in \mathbb{R}^{C\times1\times1}$ denote the channel-wise threshold and scaling factor, respectively. 
We show the block diagram in Figure \ref{fig:method}(b).
The real-valued input feature map is first squeezed to a ${C\times1\times1}$ vector by a global average pooling (GAP) layer.
The subsequent fully connected layers and ReLU learn the channel-wise information and output a ${2C\times1\times1}$ vector.
Then the ${2C\times1\times1}$ vector is split into two ${C\times1\times1}$ vectors.
We use the first $C$ channels as the channel-wise bias and pass the last $C$ channels through a sigmoid layer 
as the channel-wise scaling factor, which are used to shift the real-valued activations and re-scale the binary convolution output, respectively. 


% \ml{We can mention previously, channel-wise re-scale has been proposed. We propose to fuse them. Add the comparison between fuse v.s. no fuse.}

\begin{figure}[!tbp]%
  \centering
    \includegraphics[width=0.4\textwidth]{fig/methods.png}
  
% \subfloat[channel-wise shifting\&re-scale]{
%     \label{subfig:channel-wise shifting and re-scale}
%     \includegraphics[width=0.2\textwidth]{fig/chl shift and rescale.png}
%   }

  \caption{Block diagram for spatial re-scaling, and channel-wise shifting and re-scaling.} 
  % Input A is the real-valued activation tensor and C, H, and W denote its dimension. GAP stands for global average pooling. The reduction ratio r is set to 16 for a better trade-off between the performance and the number of operations and parameters.}
  \label{fig:method}
\end{figure}


\subsection{Network Structure}

Combining the spatial re-scaling and the channel-wise shifting and re-scaling methods, we construct the enhanced convolution layer (E-Conv).
Then we build our EBSR model based on E-Conv.
In Figure \ref{fig:E-conv}, we compare the binary convolution layer used in the baseline network and our proposed E-Conv.
We use spatial and channel-wise scaling factors to re-scale the binary convolution output,
and use channel-wise shifting to learn appropriate thresholds for each channel before binarization.
The scaling factors and threshold used in E-Conv are learnable and depend on the real-valued input activations.
In this way, our proposed EBSR can adapt to pixel-to-pixel, channel-to-channel, and image-to-image variations
to reduce the large binarization error and preserve more details.
% In this way, our proposed E-Conv reduces the large quantization error caused by binarization and keeps the original information of input feature maps to a large extent.


\begin{figure}[!tb]%
  \centering

    \includegraphics[width=0.5\textwidth]{fig/E-conv.png}

  \caption{Comparison of (a) the binary convolution layer with a skip connection used in our baseline network and (b) the proposed E-Conv.}
  \label{fig:E-conv}
\end{figure}


Figure \ref{fig:network} shows the basic block based on the E-Conv and our EBSR composed of the basic blocks. Following existing works, the convolution layers in the head and tail modules are not binarized. We choose the lightweight EDSR which has 16 basic blocks and 64 channels, and EDSR which has 32 basic blocks and 256 channels as our backbones, which correspond to EBSR-light and EBSR, respectively.

\begin{figure}[!tb]%
  \centering
  {
    \includegraphics[width=0.35\textwidth]{fig/network.png}
  }
  
  \caption{The structure of our proposed EBSR.  Convolution layers in purple are real-valued vanilla 3x3 convolutions.}
  \label{fig:network}
\end{figure}

\section{Additional Experiments}
\label{sec:moreexp}

\subsection{Main Experiment Settings}
\label{sec:black_box_models}



\textbf{Target Models:} We evaluate the performance of different attacks on 31 black-box models, including 16 normally trained models with different architectures --- AlexNet \citep{alexnet}, VGG-16 \citep{vgg}, GoogleNet \citep{googlenet}, Inception-V3 \citep{inception}, ResNet-152 \citep{resnet}, DenseNet-121 \citep{densenet}, SqueezeNet \citep{iandola2016squeezenet}, ShuffleNet-V2 \citep{ma2018shufflenet}, MobileNet-V3 \citep{mobilenet}, EfficientNet-B0 \citep{tan2019efficientnet}, MNasNet \citep{tan2019mnasnet}, ResNetX-400MF \citep{regnet}, ConvNeXt-T \citep{liu2022convnext}, ViT-B/16 \citep{vit}, Swin-S \citep{liu2021swin}, MaxViT-T \citep{tu2022maxvit}, and 8 adversarially trained models~\citep{pgd,wei2023cfa} available on RobustBench \citep{croce2020robustbench} --- FGSMAT \citep{adversarialMLAtScale} with Inception-V3, Ensemble AT (EnsAT) \citep{tramer2017ensemble} with Inception-ResNet-V2, FastAT \citep{wong2020fast} with ResNet-50, PGDAT \citep{Engstrom2019Robustness,salman2020adversarially} with ResNet-50, ResNet-18, Wide-ResNet-50-2, a variant of PGDAT tuned by bag-of-tricks (PGDAT$^\dagger$) \citep{debenedetti2022light} with XCiT-M12 and XCiT-L12. Most defense models are state-of-the-art on RobustBench \citep{croce2020robustbench}. Regarding defenses other than adversarial training, we consider 7 defenses (\ie, HGD~\citep{liao2018defense_HGD}, R\&P~\citep{xie2017mitigating_randomization_RandP}, 
Bit (BitDepthReduction in \citet{guo2018countering_JPEG}), JPEG~\citep{guo2018countering_JPEG},
RS~\citep{cohen2019certified_r},
NRP~\citep{naseer2020self_NRP},
DiffPure~\citep{nie2022diffpure}) that are robust against black-box attacks.

\subsection{Comparison with \citet{Naseer_2021_ICCV_on_generating}}

In order to show that our method works well under different surrogate models and target models, we supplement an experiment following the setting in \citet{Naseer_2021_ICCV_on_generating}:



\begin{table}[htbp]
\centering
\caption{\textbf{Attack success rate (\%,$\uparrow$).}}
\footnotesize
\begin{tabu}{l|l|ccccc} 
\hline
           Surrogates               &  Methods  &  VGG19$_{bn}$      & Dense121       & Res50        & Res152       & WRN50-2         \\ 
\hline
\multirow{2}{*}{$V_{ens}$} &   \citet{Naseer_2021_ICCV_on_generating}   & 97.34           & 71.41          & 71.68           & 50.78           & 48.03           \\
                           & MI-CWA & \textbf{100.00} & \textbf{94.50} & \textbf{94.30}  & \textbf{81.50}  & \textbf{89.20}  \\ 
\hline
\multirow{2}{*}{$D_{ens}$} &   \citet{Naseer_2021_ICCV_on_generating}   & 76.96           & 96.25          & 88.81           & 83.48           & 81.85           \\
                           & MI-CWA & \textbf{99.60}  & \textbf{99.90} & \textbf{99.80}  & \textbf{99.50}  & \textbf{99.60}  \\ 
\hline
\multirow{2}{*}{$R_{ens}$} &    \citet{Naseer_2021_ICCV_on_generating}  & 90.43           & 94.39          & 96.67           & 95.48           & 92.63           \\
                           & MI-CWA & \textbf{99.50}  & \textbf{99.70} & \textbf{100.00} & \textbf{100.00} & \textbf{99.90}  \\
\hline
\end{tabu}
\label{table:nasser_ensemble}
\end{table}

Here, $V_{ens}$ represents the ensemble comprising VGG-11, VGG-13, VGG-16, and VGG-19. $R_{ens}$ denotes the ensemble of ResNet-18, ResNet-50, ResNet-101, and ResNet-152. Similarly, $D_{ens}$ corresponds to the ensemble of DenseNet-121, DenseNet-161, DenseNet-169, and DenseNet-201. In this configuration, surrogate models exhibit high similarity. Therefore, it is effective to assess the algorithm's capacity for generalization to unseen target models, particularly those that differ significantly from the surrogate models.

As shown in \cref{table:nasser_ensemble}, our method surpasses previous techniques by an average of approximately 20\%, underscoring the efficacy of our approach in attacking unseen target models.




\subsection{Attacking using less diverse surrogate models}

To further illustrate the efficacy of the CWA algorithm even with a limited diversity of surrogate models, we conducted additional experiments in this section. Specifically, we limit our selection to models of the ResNet family, employing only ResNet-18, ResNet-32, ResNet-50, and ResNet-101~\citep{resnet} as surrogate models. As shown in \cref{table:extraexp}, our SSA-CWA still achieves superior results than other attackers. These results suggest that even when the diversity of surrogate models is limited, adversarial examples generated using the CWA attacker are less prone to overfitting to those surrogate models and continue to generalize effectively to previously unseen target models.



\begin{table}
\centering
\footnotesize
\setlength{\tabcolsep}{2.3pt}
\caption{\textbf{Black-box attack success rate (\%, $\uparrow$) on NIPS2017 dataset.} Surrogates are ResNet18, ResNet32, ResNet50 and ResNet101.}
\label{table:extraexp}
\scalebox{0.8}{
\begin{tabu}{l|l|cccccccccc|cccc} 
\hline
Method                            & Backbone        & FGSM & BIM  & MI   & DI   & TI   & VMI           & SVRE & PI   & SSA           & RAP  & MI-SAM & MI-CSE       & MI-CWA       & SSA-CWA        \\ 
\hline
\multirow{16}{*}{Normal}          & AlexNet         & 77.6 & 69.2 & 74.9 & 79.2 & 78.2 & 85.3          & 80.9 & 82.6 & 89.7          & 81.9 & 79.5   & 83.1         & 82.9         & \textbf{92.7}  \\
                                  & VGG-16          & 71.3 & 91.0 & 90.8 & 95.8 & 84.7 & 97.0          & 97.3 & 91.0 & 99.1          & 94.1 & 96.7   & 99.1         & 99.4         & \textbf{100.0}   \\
                                  & GoogleNet       & 58.8 & 88.4 & 87.7 & 95.4 & 79.6 & 96.7          & 96.5 & 88.3 & 98.8          & 90.5 & 94.2   & 98.3         & 98.4         & \textbf{99.9}  \\
                                  & Inception-V3    & 58.7 & 79.0 & 81.6 & 91.8 & 77.1 & 94.1          & 92.2 & 80.9 & 97.0          & 84.0 & 88.1   & 92.4         & 91.3         & \textbf{98.7}  \\
                                  & ResNet-152      & 60.2 & 96.6 & 96.1 & 96.8 & 89.9 & 98.5          & 99.5 & 96.4 & 99.1          & 96.2 & 99.1   & \textbf{100.0} & \textbf{100.0} & \textbf{100.0}   \\
                                  & DenseNet-121    & 63.1 & 96.2 & 95.2 & 97.2 & 90.3 & 98.2          & 99.3 & 96.0 & 99.4          & 95.0 & 99.0   & 99.8         & 99.8         & \textbf{100.0}   \\
                                  & SqueezeNet      & 86.4 & 89.8 & 90.2 & 95.7 & 86.7 & 96.2          & 96.7 & 93.2 & 98.6          & 93.7 & 94.8   & 98.0         & 98.6         & \textbf{99.8}  \\
                                  & ShuffleNet-V2   & 83.1 & 78.1 & 83.3 & 86.8 & 78.8 & 91.2          & 89.4 & 85.1 & 95.1          & 89.1 & 88.1   & 91.5         & 91.8         & \textbf{97.8}  \\
                                  & MobileNet-V3    & 60.6 & 65.6 & 69.5 & 80.6 & 77.3 & 89.7          & 78.2 & 80.4 & 92.2          & 77.1 & 78.2   & 80.2         & 79.9         & \textbf{95.7}  \\
                                  & EfficientNet-B0 & 55.6 & 89.3 & 87.7 & 94.7 & 77.9 & 96.8          & 96.9 & 86.6 & 98.8          & 92.5 & 95.5   & 98.1         & 97.9         & \textbf{99.9}  \\
                                  & MNasNet         & 66.8 & 87.1 & 84.8 & 93.5 & 75.6 & 96.4          & 95.0 & 84.3 & 98.1          & 92.1 & 94.8   & 97.5         & 97.0         & \textbf{99.9}  \\
                                  & RegNetX-400MF   & 60.3 & 87.9 & 86.8 & 94.9 & 86.0 & 97.2          & 94.9 & 89.3 & 98.7          & 91.3 & 94.4   & 97.6         & 97.5         & \textbf{100.0}   \\
                                  & ConvNeXt-T      & 42.7 & 80.8 & 77.6 & 88.2 & 57.0 & 94.0          & 87.6 & 69.3 & 94.9          & 86.5 & 89.4   & 90.2         & 87.9         & \textbf{97.4}  \\
                                  & ViT-B/16        & 37.0 & 51.7 & 52.9 & 64.3 & 53.8 & \textbf{81.7} & 60.5 & 55.8 & 81.4          & 50.2 & 61.9   & 48.8         & 46.6         & 71.7           \\
                                  & Swin-S          & 36.0 & 62.3 & 60.3 & 72.4 & 39.4 & 83.8          & 70.4 & 48.9 & \textbf{84.2} & 61.5 & 70.8   & 59.5         & 58.2         & 80.7           \\
                                  & MaxViT-T        & 34.1 & 63.0 & 61.6 & 73.1 & 31.9 & 85.1          & 68.6 & 43.9 & \textbf{86.7} & 58.9 & 70.5   & 56.7         & 54.5         & 79.0           \\ 
\hline
FGSMAT~                           & Inception-V3    & 55.3 & 51.2 & 55.2 & 59.2 & 65.6 & 73.9          & 61.1 & 66.2 & \textbf{84.4} & 59.6 & 58.6   & 60.1         & 60.3         & 78.2           \\ 
\cline{1-2}
EnsAT~                            & IncRes-V2       & 35.6 & 37.6 & 38.6 & 51.1 & 56.7 & 66.7          & 41.8 & 52.5 & \textbf{74.7} & 35.4 & 39.9   & 38.0         & 38.2         & 59.9           \\ 
\cline{1-2}
FastAT                            & ResNet-50       & 45.2 & 42.3 & 44.6 & 44.2 & 46.7 & 47.5          & 45.3 & 48.5 & \textbf{50.4} & 46.6 & 45.3   & 46.2         & 46.2         & 49.6           \\ 
\cline{1-2}
PGDAT~                            & ResNet-50       & 35.9 & 31.2 & 34.1 & 35.1 & 38.4 & 41.3          & 35.7 & 42.0 & \textbf{43.7} & 36.5 & 36.1   & 35.7         & 35.7         & 40.3           \\ 
\cline{1-2}
\multirow{2}{*}{PGDAT~}           & ResNet-18       & 47.1 & 42.0 & 45.4 & 45.3 & 47.6 & 47.2          & 45.6 & 50.4 & 50.6          & 47.8 & 45.8   & 46.8         & 46.9         & \textbf{50.7}  \\
                                  & WRN-50-2        & 28.1 & 22.7 & 25.6 & 27.4 & 31.2 & 32.6          & 27.3 & 33.9 & \textbf{35.3} & 27.7 & 28.0   & 27.6         & 26.7         & 31.7           \\ 
\cline{1-2}
\multirow{2}{*}{PGDAT$^\dagger$~} & XCiT-M12        & 21.7 & 16.8 & 18.5 & 19.9 & 22.9 & 25.0          & 20.7 & 25.3 & \textbf{28.1} & 22.0 & 20.9   & 21.3         & 21.2         & 26.5           \\
                                  & XCiT-L12        & 18.7 & 14.9 & 16.3 & 17.3 & 20.9 & 21.6          & 18.6 & 22.3 & \textbf{26.0} & 18.4 & 18.4   & 18.5         & 18.2         & 22.6           \\ 
\hline
HGD                               & IncRes-V2       & 37.8 & 77.4 & 75.4 & 91.1 & 73.0 & 94.2          & 85.5 & 77.9 & 94.7          & 76.9 & 86.7   & 85.8         & 83.6         & \textbf{97.8}  \\
R\&P                & ResNet-50       & 69.7 & 97.5 & 96.8 & 97.9 & 94.3 & 98.8          & 99.7 & 97.7 & 99.6          & 96.9 & 99.3   & \textbf{100.0} & \textbf{100.0} & \textbf{100.0}   \\
Bit                               & ResNet-50       & 73.2 & 98.4 & 98.4 & 97.7 & 96.6 & 99.1          & 99.9 & 99.2 & 99.7          & 98.7 & 99.6   & \textbf{100.0} & \textbf{100.0} & \textbf{100.0}   \\
JPEG                              & ResNet-50       & 70.5 & 98.1 & 97.8 & 97.2 & 96.3 & 98.9          & 98.7 & 98.5 & 99.7          & 98.1 & 99.5   & \textbf{100.0} & \textbf{100.0} & \textbf{100.0}   \\
RS                                & ResNet-50       & 66.7 & 97.2 & 96.7 & 97.3 & 93.1 & 98.7          & 99.5 & 97.0 & 99.5          & 96.6 & 99.4   & \textbf{100.0} & \textbf{100.0} & \textbf{100.0}   \\
NRP                               & ResNet-50       & 41.0 & 90.4 & 77.0 & 66.4 & 76.0 & \textbf{82.6} & 80.2 & 34.1 & 73.2          & 25.1 & 68.8   & 40.6         & 38.6         & 36.0           \\
DiffPure                          & ResNet-50       & 57.7 & 68.6 & 75.0 & 84.1 & 88.6 & 95.9          & 85.8 & 89.4 & 96.0          & 76.4 & 84.9   & 81.2         & 79.5         & \textbf{95.0}  \\
\hline
\end{tabu}
}
\end{table}
















%------------------------------


\subsection{Attacking using the surrogate in \citet{MI}}

We also evaluate our methods using surrogate models from \citet{MI}, which include Inc-v3, Inc-v4, IncRes-v2, and Res-152 from TensorFlow model garden~\citep{tensorflowmodelgarden2020}. Note that these models are different from the corresponding models in TorchVision. This configuration ensures that the majority of the target models are dissimilar to the surrogate models, providing a robust assessment of our algorithm's capability to generate adversarial examples that effectively transfer to diverse and previously unseen models.

As demonstrated in \cref{table:classification_tfsurrogate}, SSA-CWA outperforms previous methods significantly when attacking both normally trained models and defended models. Notably, our SSA-CWA achieves a remarkable 94.6\% attack success rate against challenging DiffPure defenses~\citep{nie2022diffpure}. This underscores the effectiveness of our approach in targeting state-of-the-art defense mechanisms.







\begin{table}[t]
\centering
\footnotesize
\setlength{\tabcolsep}{2.4pt}
\caption{\textbf{Black-box attack success rate (\%,$\uparrow$) on NIPS2017 dataset.} Surrogate models are Inc-v3, Inc-v4, IncRes-v2 and Res-152 from \citet{MI}.}
\label{table:classification_tfsurrogate}
\scalebox{0.8}{
\begin{tabu}{l|l|cccccccccc|ccccc}
\hline
Method                            & Backbone        & FGSM & BIM  & MI   & DI   & TI   & VMI  & SVRE & PI   & SSA           & RAP  & MI-SAM  & MI-CSE           & MI-CWA           & SSA-CWA        \\ 
\hline
\multirow{16}{*}{Normal}          & AlexNet         & 75.8 & 64.2 & 71.3 & 73.3 & 73.5 & 77.6 & 74.8 & 80.2 & 80.1          & 80.4 & 73.7 & 76.8          & 77.8          & \textbf{86.0}  \\
                                  & VGG-16          & 75.3 & 79.8 & 82.6 & 90.8 & 73.9 & 93.0 & 89.5 & 85.2 & 94.9          & 94.1 & 88.9 & 96.0          & 96.3          & \textbf{99.2}  \\
                                  & GoogleNet       & 61.3 & 78.3 & 79.4 & 91.5 & 74.5 & 94.3 & 87.9 & 84.0 & 96.1          & 93.3 & 88.1 & 96.1          & 95.3          & \textbf{99.4}  \\
                                  & Inception-V3    & 70.6 & 91.5 & 92.3 & 97.3 & 90.1 & 97.4 & 94.4 & 94.6 & 97.8          & 96.0 & 95.8 & 97.6          & 97.5          & \textbf{99.7}  \\
                                  & ResNet-152      & 54.9 & 81.4 & 80.3 & 91.9 & 71.1 & 95.6 & 87.0 & 82.0 & 96.3          & 93.1 & 89.8 & 94.7          & 95.4          & \textbf{99.3}  \\
                                  & DenseNet-121    & 64.1 & 85.7 & 85.4 & 94.7 & 80.6 & 95.8 & 91.3 & 89.2 & 97.0          & 94.9 & 92.4 & 97.4          & 97.1          & \textbf{99.7}  \\
                                  & SqueezeNet      & 85.9 & 80.1 & 84.3 & 89.6 & 78.5 & 90.3 & 89.0 & 87.1 & 92.9          & 92.8 & 87.9 & 91.9          & 93.3          & \textbf{97.6}  \\
                                  & ShuffleNet-V2   & 81.1 & 71.3 & 76.5 & 81.3 & 69.8 & 82.8 & 81.9 & 81.2 & 84.4          & 85.7 & 80.2 & 84.9          & 84.7          & \textbf{91.4}  \\
                                  & MobileNet-V3    & 63.6 & 60.1 & 65.0 & 74.0 & 71.1 & 82.4 & 72.2 & 75.5 & 86.5          & 77.9 & 70.1 & 76.3          & 77.2          & \textbf{91.1}  \\
                                  & EfficientNet-B0 & 60.3 & 77.3 & 77.9 & 90.9 & 68.2 & 94.4 & 86.7 & 80.1 & 95.7          & 93.9 & 88.1 & 93.9          & 93.6          & \textbf{99.2}  \\
                                  & MNasNet         & 67.1 & 70.3 & 74.2 & 87.0 & 63.9 & 89.6 & 84.7 & 72.8 & 92.0          & 91.0 & 81.8 & 91.7          & 90.8          & \textbf{98.3}  \\
                                  & RegNetX-400MF   & 63.6 & 72.1 & 75.8 & 86.3 & 70.3 & 90.6 & 85.1 & 79.3 & 94.6          & 91.2 & 83.9 & 91.1          & 92.1          & \textbf{98.7}  \\
                                  & ConvNeXt-T      & 45.2 & 71.9 & 71.8 & 85.3 & 47.5 & 91.6 & 77.0 & 64.8 & 93.0          & 89.2 & 83.2 & 84.4          & 84.3          & \textbf{96.2}  \\
                                  & ViT-B/16        & 37.8 & 47.1 & 51.6 & 60.4 & 49.1 & 79.6 & 57.4 & 55.9 & \textbf{82.3} & 56.8 & 60.3 & 53.4          & 53.0          & 81.3\textbf{}  \\
                                  & Swin-S          & 37.4 & 54.4 & 58.2 & 71.9 & 38.0 & 82.0 & 63.6 & 47.5 & \textbf{85.2} & 65.9 & 65.4 & 60.4\textbf{} & 59.1          & 84.2           \\
                                  & MaxViT-T        & 35.9 & 57.6 & 60.0 & 74.0 & 29.9 & 83.6 & 64.4 & 42.0 & \textbf{87.1} & 66.0 & 68.9 & 63.9          & 61.7          & 86.9\textbf{}  \\ 
\hline
FGSMAT~                           & Inception-V3    & 61.2 & 54.8 & 57.1 & 62.4 & 68.7 & 73.8 & 61.9 & 68.1 & 79.9          & 63.3 & 60.9 & 63.2          & 63.4          & \textbf{80.2}  \\ 
\cline{1-2}
EnsAT~                            & IncRes-V2       & 36.0 & 39.8 & 40.7 & 57.4 & 61.0 & 72.5 & 45.4 & 55.7 & \textbf{79.9} & 39.1 & 44.1 & 43.0          & 44.5          & 74.3\textbf{}  \\ 
\cline{1-2}
FastAT                            & ResNet-50       & 44.8 & 41.4 & 43.2 & 43.6 & 44.5 & 44.6 & 45.0 & 46.9 & 47.6          & 46.2 & 44.9 & 46.7          & 45.1\textbf{} & \textbf{48.2}  \\ 
\cline{1-2}
PGDAT~                            & ResNet-50       & 35.5 & 30.2 & 32.7 & 33.8 & 36.2 & 37.1 & 34.8 & 39.3 & \textbf{41.6} & 36.6 & 34.2 & 32.3          & 35.1          & 39.8           \\ 
\cline{1-2}
\multirow{2}{*}{PGDAT~}           & ResNet-50       & 46.5 & 40.8 & 43.6 & 43.9 & 45.1 & 45.6 & 45.1 & 47.3 & 47.6          & 47.3 & 45.2 & 45.9          & 46.0\textbf{} & \textbf{49.1}  \\
                                  & WRN-50-2        & 27.5 & 22.3 & 25.2 & 25.9 & 29.2 & 29.7 & 26.8 & 31.4 & \textbf{33.1} & 28.8 & 26.8 & 26.7          & 26.8\textbf{} & 31.5           \\ 
\cline{1-2}
\multirow{2}{*}{PGDAT$^\dagger$~} & XCiT-M12        & 21.1 & 17.1 & 18.8 & 19.7 & 22.1 & 24.5 & 20.6 & 24.5 & \textbf{29.0} & 21.7 & 19.8 & 20.4          & 20.1\textbf{} & 26.9           \\
                                  & XCiT-L12        & 18.9 & 15.0 & 18.0 & 19.5 & 19.3 & 21.8 & 18.5 & 21.1 & \textbf{26.3} & 19.3 & 19.1 & 18.2          & 17.0\textbf{} & 22.9           \\ 
\hline
HGD                               & IncRes-V2       & 45.7 & 82.3 & 81.9 & 93.2 & 78.4 & 96.1 & 84.3 & 86.4 & 96.4          & 90.7 & 90.3 & 93.1          & 92.4          & \textbf{99.5}  \\
R\&P                & ResNet-50       & 65.3 & 79.8 & 80.6 & 93.1 & 74.9 & 93.8 & 86.0 & 84.0 & 95.6          & 92.6 & 87.5 & 93.8          & 94.6          & \textbf{99.1}  \\
Bit                               & ResNet-50       & 64.8 & 78.0 & 81.3 & 92.0 & 72.6 & 94.7 & 87.1 & 83.1 & 96.4          & 93.6 & 88.9 & 94.4          & 94.8\textbf{} & \textbf{99.3}  \\
JPEG                              & ResNet-50       & 61.2 & 78.5 & 79.7 & 90.5 & 77.0 & 94.2 & 88.8 & 82.8 & 95.8          & 92.1 & 87.6 & 92.1          & 93.3          & \textbf{99.2}  \\
RS                                & ResNet-50       & 61.3 & 81.1 & 80.8 & 91.8 & 74.3 & 94.2 & 88.6 & 83.1 & 95.8          & 93.5 & 89.8 & 94.6          & 94.9\textbf{} & \textbf{99.0}  \\
NRP                               & ResNet-50       & 10.0 & 36.6 & 23.2 & 29.3 & 37.0 & 37.7 & 20.9 & 11.4 & \textbf{33.0} & 8.9  & 15.6 & 16.1          & 14.1          & 16.3           \\
DiffPure                          & ResNet-50       & 52.3 & 51.9 & 62.0 & 72.9 & 73.5 & 86.2 & 67.5 & 76.8 & 89.9          & 70.2 & 68.8 & 67.1          & 67.9          & \textbf{94.6}  \\
\hline
\end{tabu}
}
\end{table}













%----------------------------------------------
\subsection{Experiments on $\epsilon=4/255$}
Many adversarially trained models and defenses are predominantly examined within the context of the $\epsilon=4/255$ threat model. To remain consistent with these evaluations, we have also carried out an additional experiment under this specific perturbation budget, $\epsilon=4/255$.


As demonstrated in \cref{table:4/255}, our methods still achieve superior results on most target models. Notably, when attacking the adversarially trained models, our methods outperform previous methods by about 5\% on average. 
This demonstrates the strong efficacy of our methods when attacking with a small perturbation budget, especially adversarially trained models.

\begin{table}[t]
\centering
\footnotesize
\setlength{\tabcolsep}{2pt}
\caption{\textbf{Black-box attack success rate (\%, $\uparrow$) on NIPS2017 dataset}. Settings are same as \cref{sec:classification}, except for $\epsilon$, which is set to $4/255$.}
\label{table:4/255}
\scalebox{0.75}{
\begin{tabu}{l|l|cccccccccc|ccccc} 
\hline
Method                            & Backbone        & FGSM & BIM  & MI   & DI            & TI            & VMI           & SVRE & PI   & SSA  & RAP  & MI-SAM & MI-CSE & MI-CWA        & VMI-CWA       & SSA-CWA        \\ 
\hline
\multirow{16}{*}{Normal}          & AlexNet         & 43.4 & 44.2 & 45.8 & 50.3          & 47.6          & 48.5          & 48.9 & 47.8 & 55.3 & 47.7 & 49.5   & 54.9   & 54.8          & 56.6          & \textbf{59.8}  \\
                                  & VGG-16          & 39.0 & 62.2 & 67.0 & 78.7          & 50.5          & 76.4          & 71.5 & 60.4 & 72.1 & 55.6 & 76.8   & 79.3   & 76.5          & \textbf{83.1} & 77.4           \\
                                  & GoogleNet       & 35.9 & 49.3 & 54.8 & 72.8          & 41.8          & 65.4          & 58.8 & 50.8 & 70.1 & 45.4 & 63.2   & 68.4   & 64.0          & 73.4          & \textbf{76.8}  \\
                                  & Inception-V3    & 35.1 & 43.3 & 47.5 & 62.2          & 42.6          & 55.6          & 53.5 & 46.1 & 65.8 & 44.6 & 54.7   & 63.1   & 60.3          & 65.9          & \textbf{71.1}  \\
                                  & ResNet-152      & 41.7 & 79.7 & 83.7 & 83.7          & 57.6          & 90.0          & 84.5 & 75.3 & 75.8 & 59.1 & 90.4   & 90.3   & 87.3          & \textbf{93.7} & 79.8           \\
                                  & DenseNet-121    & 42.2 & 72.1 & 78.7 & 82.4          & 62.1          & 86.5          & 81.2 & 71.6 & 77.2 & 57.4 & 87.7   & 87.5   & 84.5          & \textbf{90.8} & 81.7           \\
                                  & SqueezeNet      & 49.2 & 63.3 & 67.5 & 76.1          & 56.7          & 74.4          & 73.6 & 63.3 & 76.4 & 60.7 & 73.3   & 80.3   & 79.3          & 83.7          & \textbf{84.3}  \\
                                  & ShuffleNet-V2   & 44.9 & 49.6 & 53.8 & 59.7          & 47.2          & 57.3          & 57.2 & 52.4 & 63.3 & 50.6 & 56.7   & 65.9   & 64.8          & 67.8          & \textbf{70.4}  \\
                                  & MobileNet-V3    & 33.3 & 39.9 & 42.7 & 52.2          & 46.4          & 49.5          & 44.5 & 46.8 & 56.6 & 43.3 & 47.9   & 55.2   & 54.9          & 59.7          & \textbf{65.5}  \\
                                  & EfficientNet-B0 & 29.8 & 49.7 & 57.7 & 71.2          & 37.3          & 67.9          & 55.1 & 48.2 & 62.6 & 44.3 & 66.3   & 69.0   & 64.8          & \textbf{76.2} & 69.9           \\
                                  & MNasNet         & 37.6 & 58.2 & 62.4 & 75.9          & 41.7          & 73.3          & 63.5 & 53.4 & 65.0 & 51.9 & 71.9   & 73.3   & 70.5          & \textbf{77.6} & 71.4           \\
                                  & RegNetX-400MF   & 37.0 & 56.6 & 63.4 & 75.3          & 51.7          & 74.3          & 67.1 & 59.0 & 71.7 & 51.7 & 72.5   & 79.9   & 75.9          & \textbf{83.7} & 82.4           \\
                                  & ConvNeXt-T      & 19.2 & 34.6 & 40.4 & 55.3          & 18.8          & \textbf{50.9} & 34.9 & 27.8 & 34.7 & 31.3 & 48.7   & 41.9   & 35.8          & 46.0          & 32.0           \\
                                  & ViT-B/16        & 16.0 & 16.2 & 20.5 & 24.5          & 21.6          & 26.5          & 18.0 & 23.1 & 21.8 & 20.0 & 24.0   & 25.6   & 24.6          & \textbf{26.8} & 23.4           \\
                                  & Swin-S          & 16.5 & 21.2 & 26.8 & 38.1          & 16.2          & \textbf{31.8} & 24.0 & 20.3 & 24.1 & 23.9 & 32.1   & 27.3   & 24.8          & 29.1          & 24.2           \\
                                  & MaxViT-T        & 17.1 & 21.7 & 26.3 & 39.3          & 11.2          & \textbf{32.8} & 22.5 & 18.4 & 23.4 & 23.1 & 32.2   & 25.5   & 22.4          & 28.6          & 23.7           \\ 
\hline
FGSMAT~                           & Inception-V3    & 35.5 & 35.5 & 38.1 & 40.2          & 40.8          & 39.9          & 39.3 & 38.9 & 50.6 & 40.3 & 40.0   & 51.1   & 50.1          & 50.9          & \textbf{56.3}  \\ 
\cline{1-2}
EnsAT~                            & IncRes-V2       & 21.0 & 21.2 & 22.0 & 26.8          & 28.1          & 24.7          & 23.1 & 24.9 & 32.4 & 23.5 & 23.8   & 31.0   & 31.3          & 30.8          & \textbf{36.7}  \\ 
\cline{1-2}
FastAT                            & ResNet-50       & 38.8 & 39.3 & 39.3 & 39.8          & 40.2          & 39.5          & 39.9 & 40.1 & 42.2 & 39.9 & 39.9   & 45.6   & \textbf{46.1} & \textbf{46.1} & 45.4           \\ 
\cline{1-2}
PGDAT~                            & ResNet-50       & 25.7 & 26.7 & 27.4 & 28.0          & 28.3          & 27.9          & 27.9 & 27.9 & 30.2 & 27.9 & 27.9   & 36.7   & \textbf{36.7} & 36.2          & 33.7           \\ 
\cline{1-2}
\multirow{2}{*}{PGDAT~}           & ResNet-18       & 38.3 & 38.6 & 39.0 & 39.5          & 40.8          & 39.1          & 39.6 & 40.0 & 41.7 & 39.6 & 39.5   & 45.9   & \textbf{45.8} & 45.6          & 45.7           \\
                                  & WRN-50-2        & 17.9 & 18.4 & 18.7 & 19.7          & 20.0          & 18.9          & 19.2 & 19.4 & 20.9 & 19.9 & 19.4   & 28.1   & \textbf{28.1} & 27.4          & 25.5           \\ 
\cline{1-2}
\multirow{2}{*}{PGDAT$^\dagger$~} & XCiT-M12        & 13.3 & 13.5 & 14.0 & 15.0          & 15.5          & 14.9          & 15.1 & 15.2 & 15.4 & 15.2 & 14.8   & 25.5   & \textbf{25.6} & 24.9          & 19.9           \\
                                  & XCiT-L12        & 12.5 & 13.4 & 14.0 & 14.0          & 14.7          & 14.4          & 14.8 & 15.0 & 15.2 & 15.2 & 14.6   & 23.3   & \textbf{22.9} & 22.3          & 18.8           \\ 
\hline
HGD                               & IncRes-V2       & 18.7 & 26.8 & 31.9 & \textbf{50.8} & 30.7          & 43.4          & 26.6 & 32.3 & 31.8 & 24.5 & 39.0   & 34.0   & 29.8          & 38.7          & 33.5           \\
R\&P                & ResNet-50       & 51.9 & 86.3 & 89.7 & 91.4          & 76.6          & 93.1          & 93.5 & 85.5 & 85.4 & 67.8 & 93.0   & 94.8   & 93.8          & \textbf{95.9} & 89.0           \\
Bit                               & ResNet-50       & 56.8 & 95.0 & 95.9 & 88.1          & 82.5          & 96.5          & 98.0 & 92.0 & 91.0 & 81.4 & 97.1   & 99.0   & 99.3          & \textbf{99.8} & 95.6           \\
JPEG                              & ResNet-50       & 52.4 & 82.3 & 88.3 & 82.4          & 83.0          & 93.9          & 88.5 & 86.9 & 83.0 & 66.0 & 93.3   & 90.8   & 89.6          & \textbf{97.3} & 87.9           \\
RS                                & ResNet-50       & 47.2 & 86.3 & 89.1 & 86.8          & 67.3          & 92.7          & 91.0 & 82.3 & 83.5 & 68.5 & 93.0   & 94.8   & 93.4          & \textbf{96.2} & 88.5           \\
NRP                               & ResNet-50       & 47.0 & 78.7 & 81.9 & 76.1          & 65.6          & 87.3          & 86.0 & 72.0 & 72.6 & 60.7 & 88.3   & 78.0   & 76.7          & \textbf{83.6} & 69.6           \\
DiffPure                          & ResNet-50       & 23.9 & 22.0 & 25.6 & 33.1          & \textbf{40.4} & 28.8          & 23.6 & 35.2 & 31.5 & 26.4 & 28.2   & 28.7   & 29.0          & 30.8          & 33.2           \\
\hline
\end{tabu}
}
\end{table}


%----------------------------------------











\begin{table}[t]
\centering
\small
\caption{\textbf{Black-box attack success rate (\%, $\uparrow$)} of our methods without incorporating with MI.}
\label{table:withoutMI}
\begin{tabu}{l|l|ccclll} 
\hline
Method                            & Backbone        & FGSM & BIM  & MI   & SAM  & CSE  & CWA   \\ 
\hline
\multirow{16}{*}{Normal}          & AlexNet         & 76.4 & 54.9 & 73.2 & 69.5 & 88.1 & 87.3  \\
                                  & VGG-16          & 68.9 & 86.1 & 91.9 & 94.3 & 92.5 & 87.6  \\
                                  & GoogleNet       & 54.4 & 76.6 & 89.1 & 92.5 & 92.8 & 88.5  \\
                                  & Inception-V3    & 54.5 & 64.9 & 84.6 & 84.4 & 90.4 & 88.4  \\
                                  & ResNet-152      & 54.5 & 96.0 & 96.6 & 98.0 & 95.3 & 89.1  \\
                                  & DenseNet-121    & 57.4 & 93.0 & 95.8 & 97.7 & 95.5 & 88.1  \\
                                  & SqueezeNet      & 85.0 & 80.4 & 89.4 & 91.3 & 94.6 & 91.2  \\
                                  & ShuffleNet-V2   & 81.2 & 65.3 & 79.9 & 82.4 & 91.9 & 90.9  \\
                                  & MobileNet-V3    & 58.9 & 55.6 & 71.8 & 73.1 & 90.6 & 89.7  \\
                                  & EfficientNet-B0 & 50.8 & 80.2 & 90.1 & 93.6 & 93.3 & 87.6  \\
                                  & MNasNet         & 64.1 & 80.8 & 88.8 & 93.0 & 90.9 & 87.2  \\
                                  & RegNetX-400MF   & 57.1 & 81.1 & 89.3 & 92.9 & 93.6 & 90.1  \\
                                  & ConvNeXt-T      & 39.8 & 68.6 & 81.6 & 86.7 & 83.0 & 72.5  \\
                                  & ViT-B/16        & 33.8 & 35.0 & 59.2 & 55.8 & 84.5 & 83.2  \\
                                  & Swin-S          & 34.0 & 48.2 & 66.0 & 69.6 & 80.2 & 72.0  \\
                                  & MaxViT-T        & 31.3 & 49.7 & 66.1 & 71.1 & 78.5 & 69.0  \\ 
\hline
FGSMAT~                           & Inception-V3    & 53.9 & 43.4 & 55.9 & 55.7 & 88.5 & 89.8  \\ 
\cline{1-2}
EnsAT~                            & IncRes-V2       & 32.5 & 28.5 & 42.5 & 40.3 & 82.1 & 83.4  \\ 
\cline{1-2}
FastAT                            & ResNet-50       & 45.6 & 41.6 & 45.7 & 46.0 & 72.2 & 74.7  \\ 
\cline{1-2}
PGDAT~                            & ResNet-50       & 36.3 & 30.9 & 37.4 & 36.5 & 70.2 & 72.9  \\ 
\cline{1-2}
\multirow{2}{*}{PGDAT~}           & ResNet-18       & 46.8 & 41.0 & 45.7 & 43.6 & 70.8 & 73.6  \\
                                  & WRN-50-2        & 27.7 & 20.9 & 27.8 & 26.4 & 64.7 & 68.1  \\ 
\cline{1-2}
\multirow{2}{*}{PGDAT$^\dagger$~} & XCiT-M12        & 23.0 & 16.4 & 22.8 & 22.8 & 73.0 & 77.7  \\
                                  & XCiT-L12        & 19.8 & 15.7 & 19.8 & 20.5 & 67.1 & 72.0  \\ 
\hline
HGD                               & IncRes-V2       & 36.0 & 78.0 & 76.2 & 82.8 & 86.4 & 81.9  \\
R\&P                & ResNet-50       & 67.9 & 95.8 & 96.3 & 98.3 & 95.0 & 88.7  \\
Bit                               & ResNet-50       & 69.1 & 97.0 & 97.3 & 98.8 & 97.5 & 89.3  \\
JPEG                              & ResNet-50       & 68.5 & 96.0 & 96.3 & 98.6 & 96.1 & 89.0  \\
RS                                & ResNet-50       & 60.9 & 96.1 & 95.6 & 98.2 & 95.8 & 89.9  \\
NRP                               & ResNet-50       & 36.6 & 88.7 & 72.4 & 92.4 & 82.9 & 62.7  \\
DiffPure                          & ResNet-50       & 50.9 & 68.5 & 76.0 & 67.9 & 83.3 & 82.8  \\
\hline
\end{tabu}
\end{table}




























%------------------------------------------

\subsection{Visualization of Adversarial Patches}


\begin{figure}[h]
\centering
\subfigure[YOLOv3]{
\includegraphics[width=2.1cm]{./figures/v3.png}
}
\quad
\subfigure[YOLOv5]{
\includegraphics[width=2.1cm]{./figures/v5.png}
}
\quad
\subfigure[Loss Avg]{
\includegraphics[width=2.1cm]{./figures/ensemble.png}
}
\quad
\subfigure[Adam-CWA]{
\includegraphics[width=2.1cm]{./figures/CW.png}
}
\quad
\subfigure[Strongest]{
\includegraphics[width=2.1cm]{figures/strongest_patch.png}
\label{figure:strongestpatch}
}
\caption{\textbf{Visualization of adversarial patches from different methods.} The patch simply trained by loss ensemble looks like the fusion of those trained by YOLOv3 and YOLOv5. Adam-CWA captures the common weakness of YOLOv3 and YOLOv5, and therefore generates an completely different patch.}
\vspace{-2ex}
\label{figure:patch}
\end{figure}


As illustrated in \cref{figure:patch}, we observe that YOLOv5 is more vulnerable to adversarial attacks compared to YOLOv3. Consequently, the patch generated by the Loss Ensemble method resembles the one obtained by YOLOv5, resulting in similar performance for both patches. We hypothesize that the Loss Ensemble method does not attack the common weakness of YOLOv3 and YOLOv5, and instead nearly solely relies on information from YOLOv5. Contrarily, our proposed method, aims to exploit this common vulnerability and generates a patch that differs significantly from both YOLOv3 and YOLOv5. As a result, our patch are more effective in attacking the object detectors.





In order to craft the strongest universal adversarial patch for object detectors, we ensemble all the models in \citet{tsea} and craft an adversarial patch by our CWA algorithm. The patch is visualized in \cref{figure:strongestpatch}. Our patch outperforms previous patches by a large margin. Compared to the previous state-of-the-art methods~(\ie, the loss ensemble by enhanced baseline in \citet{tsea}), our approach improves by 4.26\%, achieving 4.69\% mAP on eight testing models in \cref{table:detection}.































%--------------------------------------------------------


\subsection{Ablation studies}
In this section, we investigate the roles of the two additional hyperparameters: the reverse step size $r$ and the inner step size $\beta$. We use the same experimental settings as \cref{sec:classification}, and average the attack success rates across all target models.

\begin{figure*}[h]
\centering
\subfigure[$\beta$]{
\includegraphics[width=4cm]{./figures/ablateinnerstepsize.pdf}
\label{figure:ablatebeta}
}
\subfigure[$r$]{
\includegraphics[width=4cm]{./figures/ablater.pdf}
\label{figure:ablater}
}
\subfigure[$\alpha$]{
\includegraphics[width=4cm]{./figures/ablate_alpha.pdf}
\label{figure:ablate_alpha}
}
\end{figure*}

\textbf{Ablation study on inner step size $\beta$. }
Our MI-CSE algorithm could be viewed as optimizing the original loss using the learning rate $\beta$ and the cosine similarity regularization term using the learning rate $\frac{\beta^2}{2}$. As shown in \cref{figure:ablatebeta}, as $\beta$ increases, the proportion of the regularization term gradually increases, leading to an increase in the attack success rate. However, when $\beta$ becomes too large, the error of our algorithm also increases, causing the attack success rate to plateau or decrease. Therefore, we need to choose an appropriate value of $\beta$ to balance between the effectiveness of the regularization term and the overall performance of the algorithm.


\textbf{Ablation study on reverse step size $r$. }
\cref{figure:ablater} shows that the attack success rate initially increases and then decreases as $16/255/r$ increases. This is because when the reverse step size is too large, the optimization direction is opposite to the forward step, leading to a decrease in the attack success rate. Thus, decreasing the reverse step size initially increases the attack success rate. As the reverse step size continues to decrease, MI-CWA gradually degrade to MI-CSE, causing the attack success rate to converge to MI-CSE.



\textbf{Ablation study on step size $\alpha$. } 
We evaluate the average attack success rate for various $\alpha$, ranging from 16/255/1 to 16/255/20. As depicted in \cref{figure:ablate_alpha}, When $\alpha < 0.01$, the reverse step size surpass the forward step size. This results in opposing optimization directions and subsequently causes a sharp decline in the efficacy of our method. For $\alpha > 0.01$, the attack success rate remains relatively consistent regardless of specific value of $\alpha$. This showcases our method's resilience to hyper-parameter variations.






\textbf{Ablation study on momentum.} 
We also conduct an experiment where our methods are not combined with MI-FGSM. It is important to note that in other experiments, our methods and the baselines we compare against, except for BIM and FGSM, are combined with MI-FGSM. As shown in \cref{table:withoutMI}, our methods still achieve superior results than FGSM and BIM. However, it is notable that there is a significant decrease in performance compared with the results that incorporate MI-FGSM. This experiment demonstrates that MI-FGSM has become a popular plug-and-play module capable of efficiently enhancing the performance of various attack algorithms, and it has even become a fundamental and indivisible component in the development of advanced attack algorithms.


\section{Conclusion}
In this paper, we proposed a self-adaptive OPC framework for mask optimization for real designs by inspecting the characteristics of a full design. We proposed an extensible OPC solver selector to choose an appropriate solver for patterns with different complexity. Additionally, we also built a dynamic pattern library to reuse optimized masks for repeating patterns with the same geometric shape. We use supervised contrastive learning to embed patterns into vectors and propose a graph-based search strategy for fast pattern matching. At last, we validate the mask reusability by proving pattern shift equivariance property and proposed a practical shift calibration tool. Extensive experiments have shown our frame can achieve OPC speed-robustness co-optimization for real design patterns. 
\section*{Acknowledgments}
This work is supported The Research Grants Council of Hong Kong SAR (Project No.~CUHK14208021).




{\small
\bibliographystyle{ieee_fullname}
\bibliography{reference}
}

% \clearpage

\appendix

\section{Additional Details of Tasks, Settings and Metrics}


\subsection{Tasks}
We select 2 representative tasks from \textit{DeformableRavens} benchmark: \textbf{cable-ring} and \textbf{cable-ring-notarget}, as well as 2 harder tasks from \textit{SoftGym}: \textbf{SpreadCloth} and \textbf{RopeConfiguration} (we use the shape `S' as the target). 
% The aim of the first two tasks is to show our proposed framework can learn meaningful and abundant information without human-designed policy to . We compare our method with imitation learning in these two tasks, which shows our framework 

\begin{itemize}
 \item 

 (1) For \textbf{cable-ring}, it has a ring-shaped cable with 32 beads. The goal of the robot is to manipulate the cable towards a target zone denoted by a green circular ring in the observation. The maximum convex hull area of the cable-ring and the target cable-ring are the same.
% The termination condition is based on whether the area of the convex hull of the cable
% is above a threshold.
 \item 
(2) For \textbf{cable-ring-notarget}, the setting is the same as \textbf{cable-ring}, except that there is no visible target zone in the observation, so that the goal is to manipulate the cable to a circular ring anywhere on the workbench.
 \item 
(3) For \textbf{SpreadCloth}, we use a square cloth. The cloth is randomly perturbed to a crumpled state. The goal of the robot is to manipulate the crumpled cloth into flat state.
 \item 
(4) For \textbf{RopeConfiguration}, the rope is randomly perturbed to a crumpled state. The goal of the robot is to manipulate the rope into the shape of letter 
`S'.
\end{itemize}


\begin{figure}[htbp]
  \centering
  \includegraphics[width=\linewidth, trim={0cm, 12cm, 10cm, 0cm}, clip]{figs/demo.pdf}
  \caption{\textbf{Demonstrations of the selected tasks.} Each image shows an observation and a successful state (the cropped sub-images to the bottom right).}
  \label{fig:tasks}
\end{figure}

\subsection{Simulation}
For \textit{DeformableRavens} benchmark, we use the suite of simulated manipulation tasks using PyBullet~\cite{coumans2016pybullet} physics engine and OpenAI GYM~\cite{1606.01540} interfaces. For \textit{SoftGym} benchmark, we use Nvidia FleX physics simulator and the Python interface for a better simulation of cloth and rope.


\subsection{Metrics}
\label{sec:simu_metric}
For \textbf{cable-ring} and \textbf{cable-ring-notarget}, we follow \textit{DeformableRavens} benchmark, when the convex hull area of the ring beads exceeds a thresh $\beta$, the manipulation is judged as a success. We set $\beta$ to be 0.75 for these two tasks, as established in \textit{DeformableRavens}. We use the success rate as manipulation score, which is number of successful manipulation trajectories divided by total number of manipulation trajectories. 

For the \textbf{SpreadCloth} and \textbf{RopeConfiguration}, following \textit{SoftGym} benchmark, we choose the normalized score as manipulation score, which is $\frac{metric_{final}-metric_{initial}}{metric_{goal}-metric_{initial}}$, where $metric_{final}$ means the score of final state, $metric_{initial}$ means the score of initial state and $metric_{goal}$ means the score of target state. Specifically, for \textbf{SpreadCloth}, we use the coverage area of cloth as the measurement, and $metric_{goal}$ is $1.00$. For \textbf{RopeConfiguration}, we use the negative bipartite graph matching distance as the metric, and $metric_{goal}$ is $-0.04$.


During \textbf{testing}, we randomly select 60 (the same number of trials as in \textit{DeformableRavens}) random seeds representing different initial configurations of the objects, conduct experiments and report manipulation score on these different initial states.


\section{Additional Details of Experiments}

\subsection{Data Collection}
We collect 5,000 interactions in cable-related tasks, and 40000 interactions in \textbf{SpreadCloth} and \textbf{RopeConfiguration} for each step. 
% Specifically, for each start state, we sample 10 random interactions, one of which is the reversed actions, and record the corresponding ending states. 
The training of the \textbf{SpreadCloth} and \textbf{RopeConfiguration} needs more data, for the reason that, the states, kinematics and dynamics of these objects are much more complex.

From a starting state, we collect both successful interaction data using the proposed \textit{Fold to Unfold} data collection method, and failure interaction data using a random policy.
Therefore, the trained dense affordance could represent the distribution of diverse results of diverse actions.
Each interaction data contains the actions (picking point and placing point) and results after the action (\emph{e.g.} cloth coverage area for \textbf{SpreadCloth}).


\subsection{Hyper-parameters}
We set batch size to be 20, and use Adam Optimizer~\cite{kingma2014adam} with 0.0001 as the initial learning rate. 
During \textbf{Integrated Systematic Training (IST)} procedure,
we set learning rate to be 0.00005, as the  affordance modules have been trained before, and are only adapted and integrated into a system in this procedure.

    \subsection{Computing Resources}
We use TensorFlow as our Deep Learning framework. 
Each experiment is conducted on an RTX 3090 GPU, and consumes about 20 GB GPU Memory for training. It takes about 12 hours and 6 hours to respectively train the Placing Module and the Picking Module for one step. Besides, the Integrated Systematic Training procedure consumes 6 hours.

\section{Additional Details of Network Architectures}

The Picking Module and the Placing Module both employ Fully Convolutional Networks (FCNs) with the same structure to extract point-level features. Through the FCNs, the feature of the $W \times H \times C$ dimension input sequentially transforms to $W \times H \times 64$, $W \times H \times 64$, $W/2 \times H/2 \times 128$, $W/4 \times H/4 \times 256$, $W/8 \times H/8 \times 512$, $W/16 \times H/16 \times 512$ (bottleneck of the net work, where global feature is extracted), $W/8 \times H8 \times 512$, $W/8 \times H/8 \times 256$, $W/4 \times H/4 \times 256$, $W/4 \times H/4 \times 128$, $W/2 \times H/2 \times 128$, $W/2 \times H/2 \times 256$, $W \times H \times 256$, $W \times H \times 256$.

Afterwards, the Picking Module uses MLPs with hidden sizes to be (256$\to$256, 256$\to$1) to predict picking affordance, and the Picking Module uses MLPs with hidden sizes to be (1024$\to$256, 256$\to$1) to predict placing affordance. Here, 256 denotes the feature dimension of each (picking or placing) point, and 1024 denotes the dimension of the concatenation of the picking point feature (256), the placing point feature (256), and the global feature (512).



\section{Additional Details of Real-world Experiments}

\subsection{Real-robot Settings}

For real-robot experiments, 
we set up one Franka Panda robot on the workbench,
with a RealSense camera mounted on the robot gripper to take observations.
We use Robot Operating System (ROS)~\cite{quigley2009ros} to control the robot to execute actions.

Additionally, as shown in Figure~\ref{fig:gripper}, as the original fingers of Franka Panda is wide and coarse,
to ensure that the gripper can pick only one layer of cloth instead of two layers at a time,
we design two fine-grained fingers mounted on the fingertips of the original fingers.

% \subsection{Domain Randomization}


\begin{figure}[htbp]
  \centering
  \includegraphics[width=\linewidth, trim={0cm, 0cm, 0cm, 0cm}, clip]{figs/manipulator_merged.png}
  \caption{\textbf{Our Designed Fine-grained Fingers} to better pick deformable objects.}
  \label{fig:gripper}
\end{figure}

\subsection{Real-world Data Collection and Fine-tuning}


As the configurations, kinematics and dynamics of deformable objects (like cloth and ropes) in the real world are different from those in simulation, we fine-tune the trained-in-simulation affordance using real-world collected interactions.


Specifically, following Section 4.5 (\textbf{\textit{Fold to Unfold}: Efficient Multi-stage Data Collection for Learning Foresightful Affordance}) in the main paper,
we collect real-world interactions using the \textit{Fold-to-Unfold} method in different stages (demonstrations shown in Figure~\ref{fig:real-world-data}).
For fine-tuning,
we tune the learned picking and placing affordance stage-by-stage using the above real-world collected data.


\begin{figure}[htbp]
  \centering
  \includegraphics[width=\linewidth, trim={0cm, 0cm, 0cm, 0cm}, clip]{figs/fold_to_unfold_real.pdf}
  \caption{\textbf{Demonstrations of real-world collected data.} State $o_{i+1}$ shows the starting state and state $o_{i}^{\prime}$ show the ending state of interaction data for training.}
  \label{fig:real-world-data}
\end{figure}



\subsection{Metrics}

For \textbf{SpreadCloth}, we use the same metric as in~\ref{sec:simu_metric}, as it is easy to compute the coverage area of the cloth in the real world. 
For \textbf{RopeConfiguration}, we mark a black dot on the rope every 10cm, and compute the distances between these black dots and their ideal locations as the bipartite graph matching distance for further evaluation.


\subsection{Video Records of Real-world Manipulations}

Please see the {\color{red}\textbf{the videos in our project page}} for video records of real-world manipulations for both \textbf{SpreadCloth} and \textbf{RopeConfiguration} tasks.


\section{Assets}
We use the cable assets in \textit{DeformableRavens} benchmark as well as cloth and rope assets in \textit{SoftGym} benchmark, following their licenses. Our proposed assets with novel configurations can be generated using our code.



\end{document}
