\newif\ifLIPICS
\LIPICSfalse
% \LIPICStrue % last one wins

\ifLIPICS
    \documentclass[a4paper,UKenglish,cleveref, autoref, thm-restate,numberwithinsect]{lipics-v2021}
    \usepackage{paper}
\else
    \documentclass[11pt]{article}
    \usepackage{paper}
    \usepackage{cleveref, thm-restate}
\fi



\newif\ifDRAFT 
\DRAFTfalse
% \DRAFTtrue % last one wins


\usepackage{ulem}


\ifDRAFT
    \usepackage{lineno}
    \linenumbers
    % \renewcommand{\baselinestretch}{1.15} % bigger line spacing
\fi


\title{
Lower Bounds for Pseudo-Deterministic Counting in a Stream
}

\ifLIPICS
    % Shay: copied these from the example article. Most of it doesn't matter for the anonymus submission.
    
    \author{Jane {Open Access}}{Dummy University Computing Laboratory, [optional: Address], Country \and My second affiliation, Country \and \url{http://www.myhomepage.edu} }{johnqpublic@dummyuni.org}{https://orcid.org/0000-0002-1825-0097}{(Optional) author-specific funding acknowledgements}%TODO mandatory, please use full name; only 1 author per \author macro; first two parameters are mandatory, other parameters can be empty. Please provide at least the name of the affiliation and the country. The full address is optional. Use additional curly braces to indicate the correct name splitting when the last name consists of multiple name parts.
    
    \author{Joan R. Public\footnote{Optional footnote, e.g. to mark corresponding author}}{Department of Informatics, Dummy College, [optional: Address], Country}{joanrpublic@dummycollege.org}{[orcid]}{[funding]}


    % \author{Shay Sapir}{Weizmann Institute of Science}{shay.sapir@weizmann.ac.il}{}{(Optional) author-specific funding acknowledgements} % TODO

    
    \authorrunning{J. Open Access and J.\,R. Public} %TODO mandatory. First: Use abbreviated first/middle names. Second (only in severe cases): Use first author plus 'et al.'
    
    \Copyright{Jane Open Access and Joan R. Public} %TODO mandatory, please use full first names. LIPIcs license is "CC-BY";  http://creativecommons.org/licenses/by/3.0/
    
    % mandatory: Please choose ACM 2012 classifications from https://dl.acm.org/ccs/ccs_flat.cfm 
    \begin{CCSXML}
<ccs2012>
   <concept>
       <concept_id>10003752.10003809.10010055</concept_id>
       <concept_desc>Theory of computation~Streaming, sublinear and near linear time algorithms</concept_desc>
       <concept_significance>500</concept_significance>
       </concept>
   <concept>
       <concept_id>10003752.10003809.10010055.10010058</concept_id>
       <concept_desc>Theory of computation~Lower bounds and information complexity</concept_desc>
       <concept_significance>300</concept_significance>
       </concept>
   <concept>
       <concept_id>10003752.10010061.10010062</concept_id>
       <concept_desc>Theory of computation~Pseudorandomness and derandomization</concept_desc>
       <concept_significance>100</concept_significance>
       </concept>
 </ccs2012>
\end{CCSXML}

\ccsdesc[500]{Theory of computation~Streaming, sublinear and near linear time algorithms}
\ccsdesc[300]{Theory of computation~Lower bounds and information complexity}
\ccsdesc[100]{Theory of computation~Pseudorandomness and derandomization} % in seperated tex file

    % mandatory; please add comma-separated list of keywords
    \keywords{streaming algorithms, pseudo-deterministic, approximate counting} % Shift Finding?
    
    \category{} %optional, e.g. invited paper
    
    \relatedversion{} %optional, e.g. full version hosted on arXiv, HAL, or other respository/website
    %\relatedversiondetails[linktext={opt. text shown instead of the URL}, cite=DBLP:books/mk/GrayR93]{Classification (e.g. Full Version, Extended Version, Previous Version}{URL to related version} %linktext and cite are optional
    
    %\supplement{}%optional, e.g. related research data, source code, ... hosted on a repository like zenodo, figshare, GitHub, ...
    %\supplementdetails[linktext={opt. text shown instead of the URL}, cite=DBLP:books/mk/GrayR93, subcategory={Description, Subcategory}, swhid={Software Heritage Identifier}]{General Classification (e.g. Software, Dataset, Model, ...)}{URL to related version} %linktext, cite, and subcategory are optional
    
    %\funding{(Optional) general funding statement \dots}%optional, to capture a funding statement, which applies to all authors. Please enter author specific funding statements as fifth argument of the \author macro.
    
    \acknowledgements{I want to thank \dots}%optional
    
    %\nolinenumbers %uncomment to disable line numbering
\else
    \author{
    Vladimir Braverman%
    \thanks{Work partially supported by ONR Award N00014-18-1-2364 and NSF awards 1652257, 1813487 and 2107239}\\ Rice University \\ \texttt{vb21@rice.edu} \and 
    Robert Krauthgamer%
    \thanks{Work partially supported by ONR Award N00014-18-1-2364,
      by a Weizmann-UK Making Connections Grant,
      by a Minerva Foundation grant,
      and the Weizmann Data Science Research Center.
      } \\ Weizmann Institute of Science \\ \texttt{robert.krauthgamer@weizmann.ac.il} \and 
    Aditya Krishnan%
    \thanks{Work partially done while the author was at Johns Hopkins University and supported by the MINDS Data Science Fellowship.} \\Pinecone \\ \texttt{aditya@pinecone.io} \and 
    Shay Sapir%
    \thanks{This research was partially supported by the Israeli Council for Higher Education (CHE) via the Weizmann Data Science Research Center.} \\ Weizmann Institute of Science \\ \texttt{shay.sapir@weizmann.ac.il} }

    \date{}
\fi


\begin{document}
\maketitle

\normalem


\begin{abstract}
Many streaming algorithms provide only a high-probability relative approximation.
These two relaxations,
of allowing approximation and randomization,
seem necessary --
for many streaming problems, both relaxations must be employed simultaneously, 
to avoid an exponentially larger (and often trivial) space complexity.
A common drawback of these randomized approximate algorithms
is that independent executions on the same input have different outputs,
that depend on their random coins.
\emph{Pseudo-deterministic} algorithms combat this issue, and for every input,
they output with high probability the same ``canonical'' solution.


We consider perhaps the most basic problem in data streams,
of counting the number of items in a stream of length at most $n$.
Morris's counter [CACM, 1978] is a randomized approximation algorithm
for this problem that uses $O(\log\log n)$ bits of space,
for every fixed approximation factor (greater than $1$).
Goldwasser, Grossman, Mohanty and Woodruff [ITCS 2020] asked whether
pseudo-deterministic approximation algorithms can match this space complexity.
Our main result answers their question negatively, and shows that such algorithms
must use $\Omega(\sqrt{\log n / \log\log n})$ bits of space.


Our approach is based on a problem that we call \emph{Shift Finding},
and may be of independent interest.
In this problem, one has query access
to a shifted version of a known string $F\in\{0,1\}^{3n}$,
which is guaranteed to start with $n$ zeros and end with $n$ ones, 
and the goal is to find the unknown shift using a small number of queries.
We provide for this problem an algorithm that uses $O(\sqrt{n})$ queries. 
It remains open whether $\poly(\log n)$ queries suffice;
if true, then our techniques immediately imply a nearly-tight
$\Omega(\log n/\log\log n)$ space bound for pseudo-deterministic approximate counting.
\end{abstract}


% Importance and appeal of children's drawings
Children's depictions of the human figure are highly expressive and varied.
As one of the very first subjects children attempt to draw, the representation begins as an almost unintelligible cloud of scribbles. 
As the child grows, their representation of the human figure becomes more developed and is extended to graphically represent many different types of characters: people, animals, and even personified objects (see Figure 1).

Who among us has not wished, either as a child or as an adult, to see such figures come to life and move around on the page?
Sadly, while it is relatively fast to produce a single drawing, creating the sequence of images necessary for animation is a much more tedious endeavor, requiring discipline, skill, patience, and sometimes complicated software.
As a result, most of these figures remain static upon the page.

% We built a system to animate them.
Inspired by the importance and appeal of the drawn human figure, we design and build a system to automatically animate it given an in-the-wild photograph of a child's drawing. 
Our system is fast, intuitive, and robust to much of the variation present in these types of drawings, making it well-suited to allow our target audience--children--to see their own characters coming to life.
The system is comprised of four stages: figure detection, segmentation masking, pose estimation/rigging, and animation. 
We describe each stage and identify common causes of failure in each. 
For object detection and pose estimation, we make use of existing computer vision models designed to detect human figures and joints in photographs; we fine-tune these models for use with children's drawings.
For segmentation, we present a straightforward, image processing-based method that, for animation purposes, is more useful and accurate than segmentation masks obtained from a fine-tuned object detection model.
During the animation step, we take advantage of the \textit{twisted perspective} commonly seen in children’s drawings to retarget motion capture data onto the character in a novel and appealing way.

% We use existing machine learning models. However, given the wide domain gap it's not clear how much fine-tuning data was needed. So we ran some experiments to find out and report it.
While our system leverages existing models and techniques, most are not directly applicable to the task due to the many differences between photographic images and simple pen and paper representations. 
To this end, we couple the presentation of our system with a set of experiments exploring the relationship between fine-tuning training set size and success rates.
We also include a perceptual study validating viewer preference for incorporating \textit{twisted perspective} into the motion retargeting step.

We validate the desirability and appeal of our system by building and publicly releasing a version of it as the \AD Demo \,\cite{animateddrawings}.
Launched in December 2021, this demo has been used by millions of people around the world to animate their children's drawings.
Inspired by this reception, our second contribution is The Amateur Drawings Dataset: \hjs{180,000 drawings and user-accepted annotations collected, with consent, through the demo. See Section \ref{sec:UI} for a description of how the annotations were generated.}
We believe this dataset will be a resource to researchers from various fields seeking to better understand the space of amateur drawings, evaluate new algorithms in this domain, or develop new drawing-based tools in general.

To summarize, our contributions are as follows:
\begin{enumerate}
    \item 
    We explore the problem of automatic sketch-to-animation for children's drawings of human figures and present a framework that achieves this effect. We also present a set of experiments determining the amount of training data necessary to achieve high levels of success and a perceptual study validating the usefulness of our motion retargeting technique.
    \item To encourage additional research in the domain of amateur drawings, we present a first-of-its-kind dataset of 180,000 user-submitted amateur drawings, along with user-accepted bounding box, segmentation mask, and joint location annotations.
\end{enumerate}

Upon acceptance of this paper, we plan to publicly release the Amateur Drawings Dataset, project code, and fine-tuned model weights.

\section{Preliminaries}
\label{sec2}
Deep learning has brought new inspirations to camera calibration, enabling a fully automatic calibration procedure without manual intervention. Here, we first summarize two prevalent paradigms in learning-based camera calibration: regression-based calibration and reconstruction-based calibration. Then, the widely-used learning strategies are reviewed in this research field. The detailed definitions for classical camera models and their corresponding calibration objectives are exhibited in the supplementary material.

\vspace{-0.3cm}

\subsection{Learning Paradigm}
Driven by different architectures of the neural network, the researchers have developed two main paradigms for learning-based camera calibration and its applications.

\noindent \textbf{Regression-based Calibration}
Given an uncalibrated input, the regression-based calibration first extracts the high-level semantic features using stacked convolutional layers. Then, the fully connected layers aggregate the semantic features and form a vector of the estimated calibration objective. The regressed parameters are used to conduct subsequent tasks such as distortion rectification, image warping, camera localization, etc. This paradigm is the earliest and has a dominant role in learning-based camera calibration and its applications. All the first works in various objectives, \textit{e.g.}, intrinsics: Deepfocal \cite{DeepFocal}, extrinsic: PoseNet \cite{PoseNet}, radial distortion: Rong et al. \cite{Rong}, rolling shutter distortion: URS-CNN \cite{URS-CNN}, homography matrix: DHN \cite{DHN}, hybrid parameters: Hold-Geoffroy et al. \cite{Hold-Geoffroy}, camera-LiDAR parameters: RegNet \cite{schneider2017regnet} have been achieved with this paradigm.

\noindent \textbf{Reconstruction-based Calibration}
On the other hand, the reconstruction-based calibration paradigm discards the parameter regression and directly learns the pixel-level mapping function between the uncalibrated input and target, inspired by the conditional image-to-image translation \cite{pix2pix} and dense visual perception\cite{long2015fully, eigen2014depth}. The reconstructed results are then calculated for the pixel-wise loss with the ground truth. In this regard, most reconstruction-based calibration methods \cite{DR-GAN, DDM, DaRecNet, BlindCor} design their network architecture based on the fully convolutional network such as U-Net\cite{ronneberger2015u}. Specifically, an encoder-decoder network, with skip connections between the encoder and decoder features at the same spatial resolution, progressively extracts the features from low-level to high-level and effectively integrates multi-scale features. At the last convolutional layer, the learned features are aggregated into the target channel, reconstructing the calibrated result at the pixel level.

In contrast to the regression-based paradigm, the reconstruction-based paradigm does not require the label of diverse camera parameters. Besides, the imbalance loss problem can be eliminated since it only optimizes the photometric loss of calibrated results. Therefore, the reconstruction-based paradigm enables a blind camera calibration without a strong camera model assumption.

\vspace{-0.3cm}

\subsection{Learning Strategies}
In the following, we review the learning-based camera calibration literature regarding different learning strategies.

\noindent \textbf{Supervised Learning}
Most learning-based camera calibration methods train their networks with the supervised learning strategy, from the classical methods \cite{DeepFocal, PoseNet, DHN, DeepVP, Rong, DeepCalib} to the state-of-the-art methods \cite{DVPD, EvUnroll, FishFormer, DAMG-Homo, SST-Calib}. In terms of the learning paradigm, this strategy supervises the network with the ground truth of the camera parameters (regression-based paradigm) or paired data (reconstruction-based paradigm). In general, they synthesize the training dataset from other large-scale datasets, under the random parameter/transformation sampling and camera model simulation. Some recent works \cite{Zhao, Tan, SPEC, DeepUnrollNet} establish their training dataset using a real-world setup and label the captured images with manual annotations, thereby fostering advancements in this research domain.

\noindent \textbf{Semi-Supervised Learning}
Training the network using an annotated dataset under diverse scenarios is an effective learning strategy. However, human annotation can be prone to errors, leading to inconsistent annotation quality or the inclusion of contaminated data. Consequently, increasing the training dataset to improve performance can be challenging due to the complexity and cost of constructing the dataset. To address this challenge, SS-WPC\cite{SS-WPC} proposes a semi-supervised method for correcting portraits captured by a wide-angle camera. It employs a surrogate task (segmentation) and a semi-supervised method that utilizes direction and range consistency and regression consistency to leverage both labeled and unlabeled data.

\noindent \textbf{Weakly-Supervised Learning}
Although significant progress has been made, data labeling for camera calibration is a notorious costly process, and obtaining perfect ground-truth labels is challenging. As a result, it is often preferable to use weak supervision with machine learning methods. Weakly supervised learning refers to the process of building prediction models through learning with inadequate supervision. Zhu et al. \cite{Zhu} present a weakly supervised camera calibration method for single-view metrology in unconstrained environments, where there is only one accessible image of a scene composed of objects of uncertain sizes. This work leverages 2D object annotations from large-scale datasets, where people and buildings are frequently present and serve as useful ``reference objects'' for determining 3D size.

\begin{figure*}[!t]
  \centering
  \includegraphics[width=1\textwidth]{figures/taxonomy1.pdf}
  %\vspace{-20pt}
  \caption{The structural and hierarchical taxonomy of camera calibration with deep learning. Some classical methods are listed under each category.}
  \label{fig:taxonomy}
  \vspace{-0.2cm}
\end{figure*}

\noindent \textbf{Unsupervised Learning}
Unsupervised learning, commonly referred to as unsupervised machine learning, analyzes and groups unlabeled datasets using machine learning algorithms. UDHN \cite{UDHN} is the first work for a cross-view camera model using unsupervised learning, which estimates the homography matrix of a paired image without the projection labels. By reducing a pixel-wise intensity error that does not require ground truth data, UDHN \cite{UDHN} outperforms previous supervised learning techniques. While preserving superior accuracy and robustness to fluctuation in light, the proposed unsupervised algorithm can also achieve faster inference time. Inspired by this work, increasing more methods leverage the unsupervised learning strategy to estimate the homography such as CA-UDHN \cite{CA-UDHN}, BaseHomo \cite{BasesHomo}, HomoGAN\cite{HomoGAN}, and Liu et al. \cite{Liu}. Besides, UnFishCor \cite{UnFishCor} frees the demands for distortion parameters and designs an unsupervised framework for the wide-angle camera.

\noindent \textbf{Self-supervised Learning}
Robotics is where the phrase ``self-supervised learning'' first appears, as training data is automatically categorized by utilizing relationships between various input sensor signals. Compared to supervised learning, self-supervised learning leverages input data itself as the supervision. Many self-supervised techniques are presented to learn visual characteristics from massive amounts of unlabeled photos or videos without the need for time-consuming and expensive human annotations. SSR-Net \cite{SSR-Net} presents a self-supervised deep homography estimation network, which relaxes the need for ground truth annotations and leverages the invertibility constraints of homography. To be specific, SSR-Net \cite{SSR-Net} utilizes the homography matrix representation in place of other approaches' typically-used 4-point parameterization, to apply the invertibility constraints. SIR \cite{SIR} devises a brand-new self-supervised camera calibration pipeline for wide-angle image rectification, based on the principle that the corrected results of distorted images of the same scene taken with various lenses need to be the same. With self-supervised depth and pose learning as a proxy aim, Fang et al. \cite{Fang} present to self-calibrate a range of generic camera models from raw video, offering for the first time a calibration evaluation of camera model parameters learned solely via self-supervision.

\noindent \textbf{Reinforcement Learning}
Instead of aiming to minimize at each stage, reinforcement learning can maximize the cumulative benefits of a learning process as a whole. To date, DQN-RecNet~\cite{DQN-RecNet} is the first and only work in camera calibration using reinforcement learning. It applies a deep reinforcement learning technique to tackle the fisheye image rectification by a single Markov Decision Process, which is a multi-step gradual calibration scheme. In this situation, the current fisheye image represents the state of the environment. The agent, Deep Q-Network \cite{mnih2015human}, generates an action that should be executed to correct the distorted image.

In the following, we will review the specific methods and literature for learning-based camera calibration. The structural and hierarchical taxonomy is shown in Figure~\ref{fig:taxonomy}. 
	
	\begin{table*}
		\rowcolors{1}{gray!20}{white}
		\centering
		\caption{
			{Details of the learning-based camera calibration and its extended applications from 2015 to 2022, including the method abbreviation, publication, calibration objective, network architecture, loss function, dataset, evaluation metrics, learning strategy, platform, and simulation or not (training data). For the learning strategies, SL, USL, WSL, Semi-SL, SSL, and RL denote supervised learning, unsupervised learning, weakly-supervised learning, semi-supervised learning, self-supervised learning, and reinforcement learning, respectively. }
		}
		\vspace{-6pt}
		\label{table:methods}
		\begin{threeparttable}
			\resizebox{1\textwidth}{!}{
				\setlength\tabcolsep{2pt}
				\renewcommand\arraystretch{0.98}
				% \begin{tabular}{|c|c|r||c|c|c|c|c|c|c|c|c|}  % {lccc}
				\begin{tabular}{c|r||c|c|c|c|c|c|c|c|c}
					\hline
					%\thickhline
					% &\#&
					&\textbf{Method}~~~~~~~~~&\textbf{Publication} &\textbf{Objective} &\textbf{Network}
					&\textbf{Loss Function} & \textbf{Dataset} &\textbf{Evaluation} & \textbf{Learning} &\textbf{Platform} &\textbf{Simulation}\\
					\hline
					\hline
					\multirow{1}{*}{\rotatebox{0}{\textbf{2015}}}
					% &1 &
					&DeepFocal~\cite{DeepFocal} &ICIP &Standard &AlexNet
					&$\mathcal{L}_2$ loss &1DSfM\cite{1DSfM} & Accuracy & SL &Caffe &\\
					&PoseNet~\cite{PoseNet} &ICCV &Standard 
					&GoogLeNet
					&$\mathcal{L}_2$ loss &Cambridge Landmarks\cite{Cambridge_Landmarks} &Accuracy &SL &Caffe& \\
					\hline
					\hline
					\multirow{1}{*}{\rotatebox{0}{\textbf{2016}}}
					% &1&
					&DeepHorizon~\cite{DeepHorizon} &BMVC &Standard &GoogLeNet	&Huber loss &HLW\cite{HLW} & Accuracy & SL &Caffe &\\
					
					&DeepVP~\cite{DeepVP} &CVPR &Standard 
					&AlexNet
					&Logistic loss &YUD\cite{YUD}, ECD\cite{ECD}, HLW\cite{HLW} &Accuracy &SL &Caffe& \\	
					
					&Rong et al.~\cite{Rong} &ACCV &Distortion &AlexNet
					&Softmax loss &ImageNet\cite{ImageNet} &Line length &SL &Caffe&\checkmark\\
					
					&DHN\cite{DHN} &RSSW &Cross-View &VGG
					&$\mathcal{L}_2$ loss &MS-COCO\cite{MS-COCO} &MSE &SL &Caffe&\checkmark\\		\hline
					\hline
					\multirow{1}{*}{\rotatebox{0}{\textbf{2017}}}
					% &1&
					&CLKN~\cite{CLKN} &CVPR &Cross-View  &CNNs	&Hinge loss &MS-COCO\cite{MS-COCO} & MSE & SL &Torch &\checkmark\\
					
		            &HierarchicalNet~\cite{HierarchicalNet} &ICCVW &Cross-View 
					&VGG
					&$\mathcal{L}_2$ loss &MS-COCO\cite{MS-COCO} &MSE &SL &TensorFlow&\checkmark \\
					
					&URS-CNN~\cite{URS-CNN} &CVPR &Distortion 
					&CNNs
					&$\mathcal{L}_2$ loss &Sun\cite{xiao2010sun}, Oxford\cite{philbin2007object}, Zubud\cite{shao2003zubud}, LFW\cite{huang2008labeled} &PSNR, RMSE &SL &Torch&\checkmark\\
					
					&RegNet~\cite{schneider2017regnet} &IV &Cross-Sensor 
					&CNNs
					&$\mathcal{L}_2$ loss &KITTI\cite{KITTI} &MAE &SL &Caffe&\checkmark\\
					
					\hline
					\hline
					\multirow{1}{*}{\rotatebox{0}{\textbf{2018}}}
					% &1&
					&Hold-Geoffroy et al.~\cite{Hold-Geoffroy} &CVPR &Standard &DenseNet	&Entropy loss &SUN360\cite{SUN360} & Human sensitivity & SL &- &\\
					
					&DeepCalib~\cite{DeepCalib} &CVMP &Distortion 
					&Inception-V3
					&Logcosh loss &SUN360\cite{SUN360} &Mean error &SL &TensorFlow&\checkmark \\	
					&FishEyeRecNet~\cite{FishEyeRecNet} &ECCV &Distortion &VGG
					&$\mathcal{L}_2$ loss &ADE20K\cite{ADE20K} &PSNR, SSIM &SL &Caffe&\checkmark\\
					
					&Shi et al.\cite{Shi} &ICPR &Distortion &ResNet
					&$\mathcal{L}_2$ loss &ImageNet\cite{ImageNet} &MSE &SL &PyTorch&\checkmark\\
					
					&DeepFM\cite{DeepFM} &ECCV &Cross-View &ResNet
					&$\mathcal{L}_2$ loss &T\&T\cite{TT}, KITTI\cite{KITTI}, 1DSfM\cite{1DSfM} &F-score, Mean &SL &PyTorch&\checkmark\\
					
					&Poursaeed et al.\cite{Poursaeed} &ECCVW &Cross-View &CNNs
					&$\mathcal{L}_1$, $\mathcal{L}_2$ loss &KITTI\cite{KITTI} &EPI-ABS, EPI-SQR &SL &-& \\
					
					&UDHN\cite{UDHN} &RAL &Cross-View &VGG
					&$\mathcal{L}_1$ loss &MS-COCO\cite{MS-COCO} &RMSE &USL &TensorFlow&\checkmark\\
					
					&PFNet\cite{PFNet} &ACCV &Cross-View &FCN
					&Smooth $\mathcal{L}_1$ loss &MS-COCO\cite{MS-COCO} &MAE &SL &TensorFlow&\checkmark\\
					
					&CalibNet\cite{iyer2018calibnet} &IROS &Cross-Sensor &ResNet
					&Point cloud distance, $\mathcal{L}_2$ loss &KITTI\cite{KITTI} &Geodesic distance, MAE &SL &TensorFlow&\checkmark\\

                        &Chang et al.\cite{chang2018deepvp} &ICRA &Standard &AlexNet
					&Cross-entropy loss &DeepVP-1M~\cite{chang2018deepvp} &MSE, Accuracy &SL &Matconvnet&\\
					
					\hline
					\hline
					\multirow{1}{*}{\rotatebox{0}{\textbf{2019}}}
					% &1&
					&Lopez et al.~\cite{Lopez} &CVPR &Distortion &DenseNet	&Bearing loss &SUN360\cite{SUN360} &MSE & SL &PyTorch &\\
					
					&UprightNet~\cite{UprightNet} &ICCV &Standard &U-Net	&Geometry loss &InteriorNet\cite{InteriorNet}, ScanNet\cite{ScanNet}, SUN360\cite{SUN360} &Mean error & SL &PyTorch &\\
					
					&Zhuang et al.~\cite{Zhuang} &IROS &Distortion &ResNet	&$\mathcal{L}_1$ loss & KITTI\cite{KITTI} &Mean error, RMSE & SL &PyTorch &\checkmark\\
					
					&SSR-Net~\cite{SSR-Net} &PRL &Cross-View &ResNet	&$\mathcal{L}_2$ loss & MS-COCO\cite{MS-COCO} &MAE & SSL &PyTorch &\checkmark\\
					
					&Abbas et al.~\cite{Abbas} &ICCVW &Cross-View &CNNs	&Softmax loss & CARLA\cite{CARLA} &AUC\cite{AUC}, Mean error & SL &TensorFlow &\checkmark\\
					
					&DR-GAN~\cite{DR-GAN} &TCSVT &Distortion &GANs	&Perceptual loss & MS-COCO\cite{MS-COCO} &PSNR, SSIM & SL &TensorFlow &\checkmark\\
					
					&STD~\cite{STD} &TCSVT &Distortion &GANs+CNNs	&Perceptual loss & MS-COCO\cite{MS-COCO} &PSNR, SSIM & SL &TensorFlow &\checkmark\\
					
					&Deep360Up~\cite{Deep360Up} &VR &Standard &DenseNet	&Log-cosh loss\cite{Log-cosh} & SUN360\cite{SUN360} &Mean error & SL &- &\checkmark\\
					
					&UnFishCor~\cite{UnFishCor} &JVCIR &Distortion &VGG	&$\mathcal{L}_1$ loss & Places2\cite{Places2} &PSNR, SSIM & USL &TensorFlow &\checkmark\\
					
					&BlindCor~\cite{BlindCor} &CVPR &Distortion &U-Net	&$\mathcal{L}_2$ loss & Places2\cite{Places2} &MSE & SL &PyTorch &\checkmark\\
					
					&RSC-Net~\cite{RSC-Net} &CVPR &Distortion &ResNet	&$\mathcal{L}_1$ loss & KITTI\cite{KITTI} &Mean error & SL &PyTorch &\checkmark\\
					
					&Xue et al.~\cite{Xue} &CVPR &Distortion &ResNet	&$\mathcal{L}_2$ loss & Wireframes\cite{Wireframes}, SUNCG\cite{SUNCG} &PSNR, SSIM, RPE & SL &PyTorch &\checkmark\\
					
					&Zhao et al.~\cite{Zhao} &ICCV &Distortion &VGG+U-Net	&$\mathcal{L}_1$ loss & Self-constructed+BU-4DFE\cite{BU-4DFE} &Mean error &SL &- &\checkmark\\

					&NeurVPS~\cite{zhou2019neurvps} &NeurIPS &Standard &CNNs	&Binary cross entropy, chamfer-$\mathcal{L}_2$ loss &ScanNet~\cite{ScanNet}, SU3~\cite{SU3} &Angle accuracy &SL &PyTorch &\\

     
					
					\hline
					\hline
					\multirow{1}{*}{\rotatebox{0}{\textbf{2020}}}
					% &1&
					&Sha et al.~\cite{Sha} &CVPR &Cross-View &U-Net	& Cross-entropy loss &World Cup 2014\cite{homayounfar2017sports} &IoU & SL &TensorFlow &\\
				    
				    &Lee et al.~\cite{Lee} &ECCV &Standard &PointNet + CNNs	& Cross-entropy loss &Google Street View\cite{googleStreet}, HLW\cite{HLW} &Mean error, AUC\cite{AUC} & SL &- &\\
				    
				    &MisCaliDet~\cite{MisCaliDet} &ICRA &Distortion &CNNs	& $\mathcal{L}_2$ loss &KITTI\cite{KITTI} &MSE & SL &TensorFlow &\checkmark\\
				    
				    &DeepPTZ~\cite{DeepPTZ} &WACV &Distortion &Inception-V3	& $\mathcal{L}_1$ loss &SUN360\cite{SUN360} &Mean error & SL &PyTorch &\checkmark\\
				    
				    &MHN~\cite{MHN} &CVPR &Cross-View &VGG	&Cross-entropy loss &MS-COCO\cite{MS-COCO}, Self-constructed &MAE & SL &TensorFlow &\checkmark\\
				    
				    &Davidson et al.~\cite{Davidson} &ECCV &Standard &FCN	&Dice loss &SUN360\cite{SUN360} &Accuracy &SL &- &\checkmark\\
				    
				    &CA-UDHN~\cite{CA-UDHN} &ECCV &Cross-View &FCN + ResNet	&Triplet loss &Self-constructed &MSE &USL &PyTorch &\\
				    
				    &DeepFEPE~\cite{DeepFEPE} &IROS &Standard &VGG + PointNet	&$\mathcal{L}_2$ loss &KITTI\cite{KITTI}, ApolloScape\cite{Apolloscape} &Mean error &SL &PyTorch &\\
				    
				    &DDM~\cite{DDM} &TIP &Distortion &GANs	&$\mathcal{L}_1$ loss &MS-COCO\cite{MS-COCO} &PSNR, SSIM &SL &TensorFlow &\checkmark\\
				    
				    &Li et al.~\cite{Li} &TIP &Distortion &CNNs	&Cross-entropy, $\mathcal{L}_1$ loss &CelebA\cite{CelebA} &Cosine distance &SL &- &\checkmark\\
				    
				    &PSE-GAN~\cite{PSE-GAN} &ICPR &Distortion &GANs	&$\mathcal{L}_1$, WGAN loss &Place2\cite{Places2} &MSE &SL &- &\checkmark\\
				    
				    &RDC-Net~\cite{RDC-Net} &ICIP &Distortion &ResNet	&$\mathcal{L}_1$, $\mathcal{L}_2$ loss &ImageNet\cite{ImageNet} &PSNR, SSIM &SL &PyTorch &\checkmark\\
				    
				    &FE-GAN~\cite{FE-GAN} &ICASSP &Distortion &GANs	&$\mathcal{L}_1$, GAN loss &Wireframe\cite{Wireframes}, LSUN\cite{LSUN} &PSNR, SSIM, RMSE &SSL &PyTorch &\checkmark\\
				    
				    &RDCFace~\cite{RDCFace} &CVPR &Distortion &ResNet	&Cross-entropy, $\mathcal{L}_2$ loss &IMDB-Face\cite{IMDB-Face} &Accuracy &SL &- &\checkmark\\
				    
				    &LaRecNet~\cite{LaRecNet} &arXiv &Distortion &ResNet	&$\mathcal{L}_2$ loss &Wireframes\cite{Wireframes}, SUNCG\cite{SUNCG} &PSNR, SSIM, RPE &SL &PyTorch &\checkmark\\
				    
				    &Baradad et al.~\cite{Baradad} &CVPR &Standard &CNNs	&$\mathcal{L}_2$ loss &ScanNet\cite{ScanNet}, NYU\cite{NYU}, SUN360\cite{SUN360} &Mean error, RMS &SL &PyTorch &\\
				    
				    &Zheng et al.~\cite{Zheng} &CVPR &Standard &CNNs	&$\mathcal{L}_1$ loss &FocaLens\cite{FocaLens} &Mean error, PSNR, SSIM &SL &- &\checkmark\\
				    
				    &Zhu et al.~\cite{Zhu} &ECCV &Standard &CNNs + PointNet	&$\mathcal{L}_1$ loss &SUN360\cite{SUN360}, MS-COCO\cite{MS-COCO} &Mean error, Accuracy &WSL &PyTorch &\checkmark\\
				    
				    &DeepUnrollNet~\cite{DeepUnrollNet} &CVPR &Distortion &FCN	&$\mathcal{L}_1$, perceptual, total variation loss &Carla-RS\cite{DeepUnrollNet}, Fastec-RS\cite{DeepUnrollNet}  &PSNR, SSIM &SL &PyTorch &\checkmark\\
				    
				    &RGGNet~\cite{yuan2020rggnet} &RAL &Cross-Sensor &ResNet	&Geodesic distance loss &KITTI\cite{KITTI}  &MSE, MSEE, MRR &SL &TensorFlow &\checkmark\\
				    
				    &CalibRCNN~\cite{shi2020calibrcnn} &IROS &Cross-Sensor &RNNs	&$\mathcal{L}_2$, Epipolar geometry loss &KITTI~\cite{KITTI}  &MAE &SL &TensorFlow &\checkmark\\

				    &SSI-Calib~\cite{zhu2020online} &ICRA &Cross-Sensor &CNNs	&$\mathcal{L}_2$ loss &Pascal VOC 2012~\cite{pascal-voc-2012}  &Mean/standard deviation &SL &TensorFlow &\checkmark\\

				    &SOIC~\cite{wang2020soic} &arXiv &Cross-Sensor &ResNet + PointRCNN	& Cost function &KITTI~\cite{KITTI}  &Mean error &SL &- &\\        

				    &NetCalib~\cite{wu2021netcalib} &ICPR &Cross-Sensor &CNNs	&$\mathcal{L}_1$ loss &KITTI~\cite{KITTI}  &MAE &SL &PyTorch &\checkmark\\

				    &SRHEN~\cite{SRHEN} &ACM-MM &Cross-View &CNNs	&$\mathcal{L}_2$ loss &MS-COCO~\cite{MS-COCO}, SUN397~\cite{SUN360}  &MACE &SL &- &\checkmark\\
                        
				   
					\hline
					\hline
					\multirow{1}{*}{\rotatebox{0}{\textbf{2021}}}
					% &1&
					&StereoCaliNet~\cite{StereoCaliNet} &TCI &Standard &U-Net	&$\mathcal{L}_1$ loss &TAUAgent\cite{TAUAgent}, KITTI\cite{KITTI} &Mean error & SL &PyTorch &\checkmark\\
					
					&CTRL-C~\cite{CTRL-C} &ICCV &Standard &Transformer	&Cross-entropy, $\mathcal{L}_1$ loss &Google Street View\cite{googleStreet}, SUN360\cite{SUN360} &Mean error, AUC\cite{AUC} & SL &PyTorch &\checkmark\\
					
				   &Wakai et al.~\cite{Wakai} &ICCVW &Distortion &DenseNet	&Smooth $\mathcal{L}_1$ loss &StreetLearn\cite{StreetLearn} &Mean error, PSNR, SSIM & SL &- &\checkmark\\
				   &OrdianlDistortion~\cite{OrdianlDistortion} &TIP &Distortion &CNNs	&Smooth $\mathcal{L}_1$ loss & MS-COCO\cite{MS-COCO} &PSNR, SSIM, MDLD & SL &TensorFlow &\checkmark\\
				   
				   &PolarRecNet~\cite{PolarRecNet} &TCSVT &Distortion &VGG + U-Net	&$\mathcal{L}_1$, $\mathcal{L}_2$ loss & MS-COCO\cite{MS-COCO}, LMS\cite{LMS} &PSNR, SSIM, MSE & SL &PyTorch &\checkmark\\
				   
				   &DQN-RecNet~\cite{DQN-RecNet} &PRL &Distortion &VGG	&$\mathcal{L}_2$ loss & Wireframes\cite{Wireframes} &PSNR, SSIM, MSE & RL &PyTorch &\checkmark\\
				   
				   &Tan et al.~\cite{Tan} &CVPR &Distortion &U-Net &$\mathcal{L}_2$ loss & Self-constructed &Accuracy & SL &PyTorch & \\
				   
				   &PCN~\cite{PCN} &CVPR &Distortion &U-Net &$\mathcal{L}_1$, $\mathcal{L}_2$, GAN loss & Place2\cite{Places2} &PSNR, SSIM, FID, CW-SSIM & SL &PyTorch &\checkmark \\
				   
				   &DaRecNet~\cite{DaRecNet} &ICCV &Distortion &U-Net &Smooth $\mathcal{L}_1$, $\mathcal{L}_2$ loss & ADE20K\cite{ADE20K} &PSNR, SSIM & SL &PyTorch &\checkmark \\
				   
				   &DLKFM~\cite{DLKFM} &CVPR &Cross-View &Siamese-Net &$\mathcal{L}_2$ loss & MS-COCO\cite{MS-COCO}, Google Earth, Google Map &MSE & SL &TensorFlow &\checkmark \\
				   
				   &LocalTrans~\cite{LocalTrans} &ICCV &Cross-View &Transformer &$\mathcal{L}_1$ loss & MS-COCO\cite{MS-COCO} &MSE, PSNR, SSIM & SL &PyTorch &\checkmark \\
				   
				   &BasesHomo~\cite{BasesHomo} &ICCV &Cross-View &ResNet &Triplet loss & CA-UDHN\cite{CA-UDHN} &MSE & USL &PyTorch & \\
				   &ShuffleHomoNet~\cite{ShuffleHomoNet} &ICIP &Cross-View &ShuffleNet &$\mathcal{L}_2$ loss & MS-COCO\cite{MS-COCO} &RMSE & SL &TensorFlow &\checkmark \\
				   
				   &DAMG-Homo~\cite{DAMG-Homo} &TCSVT &Cross-View &CNNs &$\mathcal{L}_1$ loss & MS-COCO\cite{MS-COCO}, UDIS\cite{UDIS} &RMSE, PSNR, SSIM & SL &TensorFlow &\checkmark \\
				   
				   &SA-MobileNet~\cite{SA-MobileNet} &BMVC &Standard &MobileNet &Cross-entropy loss& SUN360\cite{SUN360}, ADE20K\cite{ADE20K}, NYU\cite{NYU} &MAE, Accuracy & SL &TensorFlow &\checkmark \\
				   
				   &SPEC~\cite{SPEC} &ICCV &Standard &ResNet &Softargmax-$\mathcal{L}_2$ loss&Self-constructed &W-MPJPE, PA-MPJPE & SL &PyTorch &\checkmark \\
				   
				   &DirectionNet~\cite{DirectionNet} &CVPR &Standard &U-Net &Cosine similarity loss &InteriorNet\cite{InteriorNet}, Matterport3D\cite{Matterport3D}&Mean and median error  & SL &TensorFlow &\checkmark \\
				   
				   &JCD~\cite{JCD} &CVPR &Distortion &FCN &Charbonnier\cite{Charbonnier}, perceptual loss &BS-RSCD \cite{JCD}, Fastec-RS
                   \cite{DeepUnrollNet}&PSNR, SSIM, LPIPS  & SL &PyTorch & \\
                   
                   &LCCNet~\cite{lv2021lccnet} &CVPRW &Cross-Sensor &CNNs &Smooth $\mathcal{L}_1$, $\mathcal{L}_2$ loss &KITTI\cite{KITTI} &MSE  & SL &PyTorch &\checkmark \\
                   
                   &CFNet~\cite{lv2021cfnet} &Sensors &Cross-Sensor &FCN &$\mathcal{L}_1$, Charbonnier\cite{Charbonnier} loss &KITTI\cite{KITTI}, KITTI-360\cite{liao2022kitti} &MAE, MSEE, MRR  & SL &PyTorch &\checkmark \\

                   &Fan\etal~\cite{fan2021inverting} &ICCV &Distortion &U-Net &$\mathcal{L}_1$, perceptual loss &Carla-RS~\cite{DeepUnrollNet}, Fastec-RS~\cite{DeepUnrollNet} &PSNR, SSIM, LPIPS  & SL &PyTorch & \\

                   &SUNet~\cite{SUNet} &ICCV &Distortion &DenseNet + ResNet &$\mathcal{L}_1$, perceptual loss &Carla-RS~\cite{DeepUnrollNet}, Fastec-RS~\cite{DeepUnrollNet} &PSNR, SSIM  & SL &PyTorch & \\

                   &SemAlign~\cite{liu2021semalign} &IROS &Cross-Sensor &CNNs & Semantic alignment loss &KITTI~\cite{KITTI} &Mean/median rotation errors & SL &PyTorch &\checkmark\\
       
				   \hline
				   \hline
				   \multirow{1}{*}{\rotatebox{0}{\textbf{2022}}}
					% &1&
				   &DVPD~\cite{DVPD} &CVPR &Standard &CNNs	&Cross-entropy loss &SU3\cite{SU3}, ScanNet\cite{ScanNet}, YUD\cite{YUD}, NYU\cite{NYU} &Accuracy, AUC\cite{AUC} & SL &PyTorch &\checkmark\\
				   
				   &Fang et al.~\cite{Fang} &ICRA &Standard &CNNs	&$\mathcal{L}_2$ loss &KITTI\cite{KITTI}, EuRoC\cite{EuRoC}, OmniCam\cite{OmniCam} &MRE, RMSE & SSL &PyTorch &\\
				   
				   &CPL~\cite{CPL} &ICASSP &Standard &Inception-V3	&$\mathcal{L}_1$ loss &CARLA\cite{CARLA}, CyclistDetection\cite{CyclistDetection} &MAE & SL &TensorFlow &\checkmark\\
				   
				   &IHN~\cite{IHN} &CVPR &Cross-View  &Siamese-Net	&$\mathcal{L}_1$ loss &MS-COCO\cite{MS-COCO}, Google Earth, Google Map &MACE & SL &PyTorch &\checkmark\\
				   
				   &HomoGAN~\cite{HomoGAN} &CVPR &Cross-View  &GANs	&Cross-entropy, WGAN loss &CA-UDHN\cite{CA-UDHN} &Mean error & USL &PyTorch &\checkmark\\
				   
				   &SS-WPC~\cite{SS-WPC} &CVPR &Distortion  &Transformer	&Cross-entropy, $\mathcal{L}_1$ loss &Tan et al.\cite{Tan} &Accuracy & Semi-SL &PyTorch &\\
				   
				   &AW-RSC~\cite{AW-RSC} &CVPR &Distortion  &CNNs	&Charbonnier\cite{Charbonnier}, perceptual loss &Self-constructed, FastecRS\cite{DeepUnrollNet} &PSNR, SSIM &SL &PyTorch &\\
				   
				   &EvUnroll~\cite{EvUnroll} &CVPR &Distortion  &U-Net	&Charbonnier, perceptual, TV loss &Self-constructed, FastecRS\cite{DeepUnrollNet} &PSNR, SSIM, LPIPS &SL &PyTorch &\\
				   
				   &Do et al.~\cite{Do} &CVPR &Standard  &ResNet&$\mathcal{L}_2$, Robust angular \cite{RobustAngular} loss &Self-constructed, 7-SCENES\cite{7-SCENES} &Median error, Recall &SL &PyTorch &\\
				   
				   &DiffPoseNet~\cite{DiffPoseNet} &CVPR &Standard  &CNNs + LSTM&$\mathcal{L}_2$ loss &TartanAir\cite{TartanAir}, KITTI\cite{KITTI}, TUM-RGBD\cite{TUM-RGBD} &PEE, AEE\cite{AEE} &SSL &PyTorch &\\
				   
				   &SceneSqueezer~\cite{SceneSqueezer} &CVPR &Standard  &Transformer&$\mathcal{L}_1$ loss &RobotCar Seasons\cite{RobotCar}, Cambridge Landmarks\cite{Cambridge_Landmarks}  &Mean error, Recall\cite{AEE} &SL &PyTorch &\\
				   
				   &FocalPose~\cite{FocalPose} &CVPR &Standard  &CNNs&$\mathcal{L}_1$, Huber loss &Pix3D\cite{Pix3D}, CompCars\cite{StanfordCars}, StanfordCars\cite{StanfordCars}  &Median error, Accuracy &SL &PyTorch &\\
				   
				   &DXQ-Net~\cite{jing2022dxq} &arXiv &Cross-Sensor  &CNNs + RNNs&$\mathcal{L}_1$, geodesic loss &KITTI\cite{KITTI}, KITTI-360\cite{liao2022kitti}  &MSE &SL &PyTorch &\checkmark\\
				   
				   &SST-Calib~\cite{SST-Calib} &ITSC &Cross-Sensor  &CNNs &$\mathcal{L}_2$ loss &KITTI\cite{KITTI}  &QAD, AEAD &SL &PyTorch &\checkmark\\
				   &CCS-Net~\cite{zhang2022learning} &IROS &Distortion  &U-Net&$\mathcal{L}_1$ loss &TUM-RGBD\cite{TUM-RGBD} &MAE, RPE &SL &PyTorch &\checkmark\\
				   
				   &FishFormer~\cite{FishFormer} &arXiv &Distortion  &Transformer&$\mathcal{L}_2$ loss &Place2\cite{Places2}, CelebA\cite{CelebA}  &PSNR, SSIM, FID &SL &PyTorch &\checkmark\\
				   
                  &SIR~\cite{SIR} &TIP &Distortion &ResNet &$\mathcal{L}_1$ loss & ADE20K\cite{ADE20K}, WireFrames\cite{Wireframes}, MS-COCO\cite{MS-COCO} &PSNR, SSIM & SSL &PyTorch &\checkmark \\

				   &ATOP~\cite{ATOP} &TIV &Cross-Sensor  &CNNs &Cross entropy loss &Self-constructed + KITTI\cite{KITTI}  &RRE, RTE &SL &- &\\

				   &FusionNet~\cite{wang2022fusionnet} &ICRA &Cross-Sensor  &CNNs+PointNet &$\mathcal{L}_2$ loss &KITTI\cite{KITTI}  &MAE &SL &PyTorch &\checkmark\\

				   &RKGCNet~\cite{RKGCNet} &TIM &Cross-Sensor  &CNNs+PointNet &$\mathcal{L}_1$ loss &KITTI\cite{KITTI}  &MSE &SL &PyTorch &\checkmark\\

                    &GenCaliNet~\cite{GenCaliNet} &ECCV &Distortion &DenseNet	&$\mathcal{L}_2$ loss &StreetLearn\cite{StreetLearn}, SP360\cite{SP360} &MAE, PSNR, SSIM & SL &- &\checkmark\\
       
				   &Liu et al.~\cite{Liu} &TPAMI &Cross-View &ResNet&Triplet loss &Self-constructed  &MSE, Accuracy &USL &PyTorch &\\
				   
				
				\hline
				\end{tabular}
			}
		\end{threeparttable}
	\end{table*}
	
	
	
	
	
	
	
	
	
	
	
	
	

\section{Lower Bounds for PD Approximate Counting via Shift Finding}\label{sec:reductionish_Shift_counting}


In this section, we prove \Cref{thm:connection_Shift_Counting}.
The proof involves three problems from different settings:
(a) PD approximate counting in the streaming model;
(b) Shift Finding in the query-access model; and
(c) MESSAGE in one-way communication with shared randomness.
The proof essentially shows that if there is an algorithm for Shift Finding that makes only $q$ queries and also a streaming algorithm for PD approximate counting that uses $b$ bits of space,
then MESSAGE can be solved using $O(b\log q)$ bits of communication.
Combining this bound with the well-known lower bound for MESSAGE in \Cref{lem:comm_problem_message} yields a lower bound for $b$.

A core idea in the proof is that an execution of a PD streaming algorithm $A$ for the approximate counting problem $\prac$
on a stream with $s^*$ insertions, can be used (even without knowing $s^*$,
by making additional insertions and then querying the streaming algorithm $A$) 
to provide query access to the shifted function $F_{s^*}: x\mapsto F(s^*+x)$.
This query access, along with a query-efficient algorithm for the Shift Finding problem $\prsf$, is then used to solve an instance of the MESSAGE problem $\prmsg$.


In fact, we prove the following theorem,
which holds for each string $F$ separately
(rather than a bound that depends on the worst-case $F$),
and yields \Cref{thm:connection_Shift_Counting} as an immediate corollary.


\begin{theorem}\label{thm:single_F_connection_shiftFind}
Let $A$ be a PD streaming algorithm for problem $\prac$, where $c,n>1$, 
and let $F:[0,(c+1)n]\to\set{0,1}$ be the canonical function of $A$. 
%
Suppose that Shift Finding with respect to this specific $F$ 
(the problem of finding an unknown shift $s^*\in[n]$
with probability at least $9/10$ given query access to $F_{s^*}$)
admits a randomized algorithm that makes
at most $q=q(F)$ (possibly adaptive) queries. 
%
Then the streaming algorithm $A$ must use
$\Omega(\tfrac{\log n}{\log q})$ bits of space.
\end{theorem}



\begin{proof} 
Define algorithm $A'$ to be an amplification of $A$
to success probability $1-1/(10q)$,
by running $O(\log q)$ independent repetitions and reporting their majority. 
Assume there exists an algorithm $Q$ that for every $s^*\in[n]$,
makes at most $q=q(F)$ queries to $F_{s^*}$ (possibly adaptive)
and outputs $s^*$ with probability at least $9/10$.

Consider an instance of problem $\prmsg$ with alphabet $\Sigma = [0,n]$,
and consider the following protocol for it.
Alice starts an execution of the streaming algorithm $A'$ using the shared randomness, 
then takes her input $s^*\in \Sigma$ and makes $s^*$ stream insertions to algorithm $A'$, 
and finally sends the state (memory contents) of $A'$ to Bob.

Bob continues the execution of the streaming algorithm $A'$ (using the shared randomness),
and uses it to provide query access to $F_{s^*}$, as follows.
In order to query $F_{s^*}$ at any index $x$,
Bob makes a fresh copy $A_0$ of the streaming algorithm $A'$,
insert $x$ stream items to algorithm $A_0$ and then reads its output.
With probability at least $1-1/(10q)$, 
the answer that Bob gets is indeed $F_{s^*}(x)$ (because the number of items inserted to this instance of the algorithm is $x+s^*$).
Bob uses this query access and his knowledge of $F$
to simulate algorithm $Q$ 
(with the goal of recovering $s^*$).


Consider Bob's simulation of algorithm $Q$. 
If $Q$ was executed with true query access to $F_{s^*}$,
then it would have had success probability $9/10$,
and would have made a sequence of queries $X_Q$ to $F_{s^*}$. 
This sequence $X_Q$ depends only on $F_{s^*}$ and the coin tosses of algorithm $Q$.
In particular, revealing $X_Q$ (i.e., conditioned on $X_Q$) 
does not affect the coins of the streaming algorithm $A'$,
and it still succeeds with probability at least $1-1/(10q)$. 
We can thus apply a union bound to conclude that
algorithm $A'$ succeeds on all queries $x\in X_Q$
(i.e., outputs the corresponding $F_{s^*}(x)$)
with probability at least $1-q\cdot\tfrac{1}{10q}=9/10$. 
Hence, when Bob simulates algorithm $Q$ using the streaming algorithm $A'$,
with probability $9/10$ (over the coins of $A'$)
the execution is identical to running algorithm $Q$ with true access to $F_{s^*}$,
which itself succeeds with probability $9/10$. 
By a union bound, with probability $8/10$ both algorithm $Q$ and the streaming algorithm $A'$ succeed, in which case Bob recovers $s^*$,
and therefore this communication protocol solves problem $\prmsg$ with alphabet $\Sigma = [0,n]$.


By Lemma~\ref{lem:comm_problem_message}, the message Alice sends must contain $\Omega(\log n)$ bits, and thus the streaming algorithm $A'$ must use $\Omega(\log n)$ bits of space.
Recall that algorithm $A'$ consists of $O(\log q)$ copies of the streaming algorithm $A$ and thus algorithm $A$ must use $\Omega(\tfrac{\log n}{\log q})$ bits of space.
\end{proof}






%%% Local Variables:
%%% mode: latex
%%% TeX-master: "main"
%%% End:

\section{Lower Bound for PD Approximate Counting}\label{sec:LB_PD_counting}

In this section, we prove Theorem~\ref{thm:LB_PD_counting}, i.e., for every $c,n>1$, we prove that every PD streaming algorithm for the approximate counting problem $\prac$ must use $\Omega_c(\sqrt{\tfrac{\log n}{\loglog n}})$ bits of space.

Let $F$ be the canonical function of a PD streaming algorithm for problem $\prac$.
% let $A$ be a PD streaming algorithm for the approximate counting problem $\prac$, and let $F$ be the corresponding canonical function.
Our analysis is split into two cases depending on $F$,
which informally correspond to whether a fixed pattern (like ``01'')
appears in the string $F$ at most $t$ times or not. 
These cases are analyzed using  \Cref{thm:generalized_connection_Shift_Counting,thm:single_F_connection_shiftFind}.
The overall bound will be derived by optimizing the threshold $t$
between the two cases to roughly $t=n/2^{\sqrt{\log n}}$.





\subsection{Scenario One}
In this scenario, there is a specific pattern in $F$
that appears at most $t$ times, where $t=t_c(n)$ will be set at the end of our proof.
We first consider the pattern "01" in $F$,
which corresponds to $x\in [0,(c+1)n-1]$ such that $F(x)=0$ and $F(x+1)=1$, and later generalize this pattern to a broader family.


\begin{lemma}\label{cl:hardness_few_01}
If the pattern "01" appears at most $t$ times in $F$, 
then every PD streaming algorithm for problem $\prac$ whose canonical function is $F$
must use $\Omega(\tfrac{\log (n/t)}{\loglog (cn)})$ bits of space.
\end{lemma}



% The proof of \Cref{cl:hardness_few_01} is by a reduction.. \ssnote{moved to \Cref{sec:reductionish_Shift_counting}.}


\begin{proof}%[Proof of \Cref{cl:hardness_few_01}]
The proof is by a reduction from problem MESSAGE, similarly to the proof of \Cref{thm:single_F_connection_shiftFind}.
Perhaps the most delicate part is the definition of an alphabet $\Sigma$ for the MESSAGE problem $\prmsg$, and it proceeds as follows.


% The construction 
% We shall first define a mapping $M:[n]\to [cn]$. 
Given $s\in[n]$, consider the following execution of Binary Search (B.S.)
on the function $F_s$. 
Initialize $l=0$ and $r=cn+1$,
and at every iteration query $F_{s}(\floor{\tfrac{l+r}{2}})$;
if $F_{s}(\floor{\tfrac{l+r}{2}})=0$,
then $l\gets\floor{\tfrac{l+r}{2}}$, otherwise $r\gets\floor{\tfrac{l+r}{2}}$.
These iterations maintain the invariant that $F_s(l)=0$ and $F_s(r)=1$,
and after at most $\log (cn)$ iterations arrive at $r=l+1$ with the pattern "01".
Define a mapping $M:[n]\to [cn]$ such that $M(s)$ is the location where the binary search finds a "01" in $F_s$, i.e., the final index $l$; thus $F(s+M(s))=0$ and $F(s+M(s)+1)=1$.


In order to define an alphabet $\Sigma$,
consider a partitioning of $[n]$ to buckets, defined
such that items $s,s'$ are from the same bucket $B$ if and only if they are mapped to the same value $M(s)=M(s')$.
For every bucket $B$ and every $s,s'\in B$,
we know from above that $F(s'+M(s))=0$ and $F(s'+M(s)+1)=1$,
so there are at most $t$ possibilities for $s'$ (one of which is $s'=s$), and thus the size of the bucket $|B|\leq t$.
Define $\Sigma\subset [n]$ by taking one representative from each bucket.
Thus, every $s_1\neq s_2\in \Sigma$ satisfy $M(s_1)\neq M(s_2)$ and
 $|\Sigma|\geq n/t$.



Let $A$ be a streaming algorithm whose canonical function is $F$ and let algorithm $A'$ be an amplification of algorithm $A$ that succeeds with probability $1-1/(10\log (cn))$ (by making $O(\loglog (cn))$ repetitions and taking the majority).
Consider an instance of the MESSAGE problem $\prmsg$,
and proceed similarly to the proof of \Cref{thm:single_F_connection_shiftFind}.
We provide a self-contained analysis for completeness.
Alice and Bob perform the following protocol.
Alice starts an execution of algorithm $A'$ using the shared randomness.
For input $s^*\in \Sigma$, she inserts $s^*$ stream items to algorithm $A'$ and sends the state (memory contents) of this algorithm $A'$ to Bob.
In order to get query access to $F_{s^*}$ at index $x$, 
Bob makes a fresh copy $A_0$ of algorithm $A'$, continues the algorithm's execution (using the shared randomness),
inserts $x$ stream items to algorithm $A_0$ and finally reads its output.
Bob uses this query access to simulate the B.S. algorithm on $F_{s^*}$ (with the goal of recovering $M(s^*)$).
He then infers which bucket corresponds to his result, and outputs the representative of that bucket (which is $s^*$ if he recovers $M(s^*)$).



If the B.S. algorithm were executed with true query access to $F_{s^*}$, then it would have output $M(s^*)$ and would have made a sequence of queries $X_{BS}$ to $F_{s^*}$.
This sequence depends only on $F_{s^*}$, and in particular
independent of the random coins of algorithm $A'$.
Thus by a union bound, algorithm $A'$ succeeds on all queries $x\in X_{BS}$ (i.e. outputs the corresponding $F_{s^*}(x)$) with probability at least $1-\log (cn)\cdot 1/(10\log (cn))=9/10$.
Hence, when Bob simulates the B.S. algorithm using the streaming algorithm $A'$, then with probability $9/10$ the execution is identical to running the B.S. algorithm with true query access to $F_{s^*}$.
Thus with this probability $9/10$, Bob recovers $M(s^*)$, and hence outputs $s^*$, which concludes the correctness analysis of the communication protocol.

    
    
     By Lemma~\ref{lem:comm_problem_message}, the message Alice sends must contain $\Omega(\log|\Sigma|)\geq \Omega(\log (n/t))$ bits, and thus algorithm $A'$ must use $\Omega(\log (n/t))$ bits of space.
    Recall that algorithm $A'$ is made of $O(\loglog (cn))$ copies of algorithm $A$ and thus algorithm $A$ must use $\Omega(\tfrac{\log (n/t)}{\loglog (cn)})$ bits of space.
\end{proof}

\begin{remark}
This proof can be easily generalized to prove \Cref{thm:generalized_connection_Shift_Counting}.
The first extension is
by replacing the B.S. algorithm and the corresponding buckets with any deterministic algorithm $Q$ that returns a subset containing $s^*$.
In order to generalize $Q$ to any PD algorithm $Y$,
consider the canonical function of $Y$ instead of the mapping $M$, and apply the same proof.
It holds because the crucial property of the B.S. algorithm was the existence of the mapping $M$.
Then by an additional union bound, both algorithms $Q$ and $A'$ succeed with probability $8/10$ (as in the proof of \Cref{thm:single_F_connection_shiftFind}).
\end{remark}


We now generalize \Cref{cl:hardness_few_01} to a larger family of patterns in $F$, where each pattern
is characterized by a parameter $k\in [n]$, and appears at index $x\in [0,(c+1)n-k]$ such that $F(x)=0$ and $F(x+k)=1$.
Denote such a pattern by "$0?^{k-1}1$", 
where each question mark can represent either $0$ or $1$,
and the number of question marks is $k-1<n$.
A copy of this pattern can be found in $O(\log \tfrac{n}{k})$ queries to $F_{s^*}$ by a binary search on the grid $(0,k,...,\ceil{\tfrac{cn}{k}}k)$, since $F_{s^*}(0)=0$ and $F_{s^*}(\ceil{\tfrac{cn}{k}}k)=1$.
Hence, if there exists $k$ for which this pattern appears at most $t$ times in $F$, then the communication protocol above can be adjusted to imply that algorithm $A$ must use at least $\Omega(\tfrac{\log (n/t)}{\loglog (cn/k)})\geq \Omega(\tfrac{\log (n/t)}{\loglog (cn)})$ bits of space.
The only change in the proof is in the number of queries that Bob makes,
which affects the number of repetitions in algorithm $A'$,
and thus only affects the $\loglog$ term.


\begin{corollary}\label{cor:scenario_1}
    If for some $k\leq n$ the pattern "$0?^{k-1}1$" appears at most $t$ times in $F$, then every PD streaming algorithm for problem $\prac$ whose canonical function is $F$, must use $\Omega(\tfrac{\log (n/t)}{\loglog (cn)})$ bits of space.

\end{corollary}


\subsection{Scenario Two}

In this scenario, for every $k\leq n$ the pattern "$0?^{k-1}1$" appears at least $t$ times in $F$.

\begin{lemma}\label{lem:scenario_2}
    If for all $k\in [n]$, the pattern "$0?^{k-1}1$"  appear at least $t$ times in $F$, then every PD streaming algorithm for problem $\prac$ whose canonical function is $F$, must use $\Omega(\tfrac{\log n}{\log(cn/t)+\loglog n})$ bits of space.
\end{lemma}


\begin{proof}
In this case, there is an algorithm for the Shift Finding problem $\prsf$ using $q=O(\tfrac{cn\log n}{t})$ queries to $F_{s^*}$, as follows.


\begin{enumerate}
\item let $S=[0,n]$
\item repeat the following $\tfrac{10cn\log n}{t}$ times:
  \begin{enumerate}
  \item pick $r\in[cn]$ uniformly at random and query $F_{s^*}(r)$
  \item let $S\gets \{s\in S:\ F(s+r)= F_{s^*}(r) \}$
  \end{enumerate}
\item if $|S|=1$, return $s\in S$; else return FAIL
\end{enumerate}
% \rnote{in pseudocode, do not use capital letters and period. }
% \ssnote{Done. Is the if-else statement ok?}
% \ssnote{Use the same format as in \Cref{alg:sqrt_n_shift_finding}? Not sure. The alg here is not particularly important, only matters as part of the proof.}


The final set $S$ clearly contains the shift $s^*$.
It remains to show that all $s\neq s^*$ are removed from the set $S$ with high probability.

Fix $s\in [n], s\neq s^*$. 
There are $t$ values for $r\in[cn]$ for which $F(s^*+r)\neq F(s+r)$, as follows.
Assume without loss of generality that $s^*<s$ and denote $k = s-s^*\in [n]$.
Let $l$ be a location that corresponds to the pattern "$0?^{k-1}1$" in $F$, i.e. $F(l)=0$ and $F(l+k)=1$.
If $l\in [s^*+1,s^*+cn]$, then there is $r\in[cn]$ such that $s^*+r=l$, for which $F(s^*+r)=0\neq F(l+k) = F(s+r)$.
There are at least $t$ locations for this pattern (i.e. possible values for $l$), thus it remains to show that indeed $l\in [s^*+1,s^*+cn]$.
It must be that $l+k> n$ since $F(x)=0$ for all $x\leq n$, and similarly $l\leq cn$ since $F(x)=1$ for all $x> cn$.
Hence $l\in [n-k+1,cn]\subset [s^*+1,s^*+cn]$,
and thus there are $t$ values for $r\in[cn]$ for which $F(s^*+r)\neq F(s+r)$ (each value for $r$ corresponds to a possible value for $l$).



Thus, in each repetition, $s$ is removed from the set $S$ with probability at least $\tfrac{t}{cn}$.
The probability $s$ is not removed after $\tfrac{10cn\log n}{t}$ repetitions
is $(1-\tfrac{t}{cn})^{(10cn\log n)/t}<\tfrac{1}{n^2}$.
By a union bound, all $s\neq s^*$ are removed with probability $1-\tfrac{1}{n}$,
which concludes the correctness analysis of the algorithm for problem $\prsf$.

By \Cref{thm:single_F_connection_shiftFind}, every PD streaming algorithm for the approximate counting problem $\prac$ with a canonical function $F$ 
must use $\Omega(\tfrac{\log n}{\log((cn\log n)/t)})$ bits of space.

\end{proof}


\subsection{Concluding the Proof of Theorem~\ref{thm:LB_PD_counting}}

Concluding the two scenarios, set $t=n/2^{\sqrt{\log n \cdot \log\log (cn)}}$ and get by Corollary~\ref{cor:scenario_1} and Lemma~\ref{lem:scenario_2} that every PD streaming algorithm for the approximate counting problem $\prac$ must use
\[
\Omega(\min\{\tfrac{\log (n/t)}{\loglog (cn)}, \tfrac{\log n}{\log ((cn/t)\log n)}\})
= \Omega(\tfrac{\log n}{\sqrt{\log n\loglog (cn)} + \log c})
\]
bits of space, which boils down to $\Omega(\sqrt{\tfrac{\log n}{\loglog n}})$ for $c<2^{\sqrt{\log n \loglog n}}$.






%%% Local Variables:
%%% mode: latex
%%% TeX-master: "main"
%%% End:



% Shift finding
\section{Shift Finding Algorithm}\label{sec:shift_finding_alg}

One can hope to prove tighter lower bounds for PD streaming algorithms for the approximate counting problem $\prac$, and a possible approach is by solving the Shift Finding problem $\prsf$ using $\polylog n$ queries.
Recall that in problem $\prsf$, the input is a string $P\in\set{0,1}^{(c-1)n}$, 
which can be represented by a string $F$ which is a concatenation of $n$ zeros, $P$ and then $n$ ones; and query access to a shifted version of $F$ with shift $s^*$, denoted $F_{s^*}$.
As stated in Theorem~\ref{thm:shift_finding_alg},
we show a deterministic algorithm for problem $\prsf$ using $O(\sqrt{cn})$ queries (Algorithm~\ref{alg:sqrt_n_shift_finding}), and we leave open the question whether it is the right bound.
 The proof relies on an efficient verification algorithm that for input $s$, uses $2$ queries and returns 'yes' if and only if $s=s^*$, as stated in Lemma~\ref{lem:shift_verifier_witness_2queries} and described next.
\begin{proof}[Proof of Lemma~\ref{lem:shift_verifier_witness_2queries}]
    Denote by $l\in [n+1,cn+1]$ the smallest number such that $F(l)=1$, and by $r\in[n,cn]$ the largest number such that $F(r)=0$.
    For input $s\in[0,n]$, the verification algorithm
    returns 'no' if $F_{s^*}(l-s)=0$ or $F_{s^*}(r-s)=1$, and
    otherwise returns 'yes'.

    If $s=s^*$, then $F_{s^*}(x-s)=F(x)$ and the verification algorithm outputs 'yes'.
    If $s> s^*$, then $s^*-s+l<l$ and thus $F_{s^*}(l-s) = F(s^*-s+l)=0$ 
    and the verification algorithm outputs 'no'.
    Similarly, if $s< s^*$ then $F_{s^*}(r-s)=1$ and the verification algorithm outputs 'no'.
\end{proof}


\begin{remark}
There is a randomized algorithm for problem $\prsf$ using $\tilde{O}_c(\sqrt{n})$ queries that is similar to
the proof of Theorem~\ref{thm:LB_PD_counting} in \Cref{sec:LB_PD_counting}.
It proceeds by considering those two scenarios.
In scenario one, instead of constructing the set $\Sigma$, query witnesses for all the $t$ possible shifts using $2t$ queries and hence recover the unknown shift $s^*$.
In scenario two, the proof of Theorem~\ref{thm:LB_PD_counting} shows how to find the unknown shift $s^*$ in $O(\tfrac{cn}{t}\log n)$ queries with high probability.
Hence, by setting $t=\sqrt{cn\log n}$, this algorithm finds the unknown shift in $O(\max \{t+\log (cn), \tfrac{cn}{t}\log n\}) \leq O(\sqrt{cn\log n})$ queries with high probability.
\end{remark}

Next is a slight improvement, 
a deterministic algorithm in $O(\sqrt{cn})$ queries, proving Theorem~\ref{thm:shift_finding_alg}.

\begin{algorithm}
    \caption{Deterministic Shift Finding in $O(\sqrt{cn})$ queries}\label{alg:sqrt_n_shift_finding}

    \begin{algorithmic}[1]
        \Require $n,c,F$ and query access to $F_{s^*}$
        \Ensure $s^*$

        \State $Q \gets (F_{s^*}(0),F_{s^*}(\sqrt{cn}),F_{s^*}(2\sqrt{cn}),...,F_{s^*}(cn))$
        \State let $S\gets\set{s\in[0,n]:\forall i\in[0,\sqrt{cn}], F_s(i\sqrt{cn})=Q(i)}$  \Comment{i.e. the set of all shifts that could have produced $Q$}
        \For{$s\in S$}
            \State check the witness of $s$
            \If{$s=s^*$} return $s$
            \EndIf
        \EndFor
    \end{algorithmic}
\end{algorithm}


\begin{lemma}
    The set $S$ in Algorithm~\ref{alg:sqrt_n_shift_finding} is of size $O(\sqrt{cn})$.
\end{lemma}
\begin{proof}
    Assume by contradiction that $|S|\geq \sqrt{cn}+1$.
    Hence by the pigeonhole principle, there exists $s_1<s_2\in S$ such that $s_1=s_2 \mod \sqrt{cn}$.
    Hence for all $i\in[0,\sqrt{cn}-\tfrac{s_2-s_1}{\sqrt{cn}}]$,
    \begin{align*}
        Q(i) = F_{s_2}(i\sqrt{cn})
            = F_{s_1}(s_2-s_1 + i\sqrt{cn})
            = Q(\tfrac{s_2-s_1}{\sqrt{cn}} + i),
    \end{align*}
    where the first and last transitions hold since $s_1,s_2\in S$ and $\tfrac{s_2-s_1}{\sqrt{cn}}$ is an integer number, and the second transition is by definition.
    Thus $Q$ has a period of length $\tfrac{s_2 - s_1}{\sqrt{cn}}\leq \lfloor\tfrac{s_2}{\sqrt{cn}}\rfloor$.
    However, for $i\in [\sqrt{cn}-\lfloor\tfrac{s_2}{\sqrt{cn}}\rfloor +1 , \sqrt{cn}]$ the values that $Q$ get are $Q(i) = F_{s_2}(i\sqrt{cn})=1$ since $s_2 + i\sqrt{cn}\geq cn$;
    % the last $\lfloor\tfrac{s_2}{\sqrt{n}}\rfloor$ entries in $Q$ are equal $1$,
    thus all entries in $Q$ are equal $1$, which contradicts the fact that $Q(0)=0$, and thus completes the proof.
\end{proof}

Algorithm~\ref{alg:sqrt_n_shift_finding} returns the shift $s^*$ since $s^*\in S$ and by the correctness of the verifier in Lemma~\ref{lem:shift_verifier_witness_2queries}.
The number of queries Algorithm~\ref{alg:sqrt_n_shift_finding} makes is $O(|S|+|Q|)=O(\sqrt{cn})$, which proves Theorem~\ref{thm:shift_finding_alg}.

%%% Local Variables:
%%% mode: latex
%%% TeX-master: "main"
%%% End:

\ifLIPICS
    \bibliographystyle{plainurl}% the mandatory bibstyle
\else
    \bibliographystyle{alphaurl}
\fi
\bibliography{references.bib}
\end{document}


