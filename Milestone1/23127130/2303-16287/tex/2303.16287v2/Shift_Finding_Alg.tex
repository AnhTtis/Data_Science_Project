\section{Shift Finding Algorithm}\label{sec:shift_finding_alg}

One can hope to prove tighter lower bounds for PD streaming algorithms for the approximate counting problem $\prac$, and a possible approach is by solving the Shift Finding problem $\prsf$ using $\polylog n$ queries.
Recall that in problem $\prsf$, the input is a string $P\in\set{0,1}^{(c-1)n}$, 
which can be represented by a string $F$ which is a concatenation of $n$ zeros, $P$ and then $n$ ones; and query access to a shifted version of $F$ with shift $s^*$, denoted $F_{s^*}$.
As stated in Theorem~\ref{thm:shift_finding_alg},
we show a deterministic algorithm for problem $\prsf$ using $O(\sqrt{cn})$ queries (Algorithm~\ref{alg:sqrt_n_shift_finding}), and we leave open the question whether it is the right bound.
 The proof relies on an efficient verification algorithm that for input $s$, uses $2$ queries and returns 'yes' if and only if $s=s^*$, as stated in Lemma~\ref{lem:shift_verifier_witness_2queries} and described next.
\begin{proof}[Proof of Lemma~\ref{lem:shift_verifier_witness_2queries}]
    Denote by $l\in [n+1,cn+1]$ the smallest number such that $F(l)=1$, and by $r\in[n,cn]$ the largest number such that $F(r)=0$.
    For input $s\in[0,n]$, the verification algorithm
    returns 'no' if $F_{s^*}(l-s)=0$ or $F_{s^*}(r-s)=1$, and
    otherwise returns 'yes'.

    If $s=s^*$, then $F_{s^*}(x-s)=F(x)$ and the verification algorithm outputs 'yes'.
    If $s> s^*$, then $s^*-s+l<l$ and thus $F_{s^*}(l-s) = F(s^*-s+l)=0$ 
    and the verification algorithm outputs 'no'.
    Similarly, if $s< s^*$ then $F_{s^*}(r-s)=1$ and the verification algorithm outputs 'no'.
\end{proof}


\begin{remark}
There is a randomized algorithm for problem $\prsf$ using $\tilde{O}_c(\sqrt{n})$ queries that is similar to
the proof of Theorem~\ref{thm:LB_PD_counting} in \Cref{sec:LB_PD_counting}.
It proceeds by considering those two scenarios.
In scenario one, instead of constructing the set $\Sigma$, query witnesses for all the $t$ possible shifts using $2t$ queries and hence recover the unknown shift $s^*$.
In scenario two, the proof of Theorem~\ref{thm:LB_PD_counting} shows how to find the unknown shift $s^*$ in $O(\tfrac{cn}{t}\log n)$ queries with high probability.
Hence, by setting $t=\sqrt{cn\log n}$, this algorithm finds the unknown shift in $O(\max \{t+\log (cn), \tfrac{cn}{t}\log n\}) \leq O(\sqrt{cn\log n})$ queries with high probability.
\end{remark}

Next is a slight improvement, 
a deterministic algorithm in $O(\sqrt{cn})$ queries, proving Theorem~\ref{thm:shift_finding_alg}.

\begin{algorithm}
    \caption{Deterministic Shift Finding in $O(\sqrt{cn})$ queries}\label{alg:sqrt_n_shift_finding}

    \begin{algorithmic}[1]
        \Require $n,c,F$ and query access to $F_{s^*}$
        \Ensure $s^*$

        \State $Q \gets (F_{s^*}(0),F_{s^*}(\sqrt{cn}),F_{s^*}(2\sqrt{cn}),...,F_{s^*}(cn))$
        \State let $S\gets\set{s\in[0,n]:\forall i\in[0,\sqrt{cn}], F_s(i\sqrt{cn})=Q(i)}$  \Comment{i.e. the set of all shifts that could have produced $Q$}
        \For{$s\in S$}
            \State check the witness of $s$
            \If{$s=s^*$} return $s$
            \EndIf
        \EndFor
    \end{algorithmic}
\end{algorithm}


\begin{lemma}
    The set $S$ in Algorithm~\ref{alg:sqrt_n_shift_finding} is of size $O(\sqrt{cn})$.
\end{lemma}
\begin{proof}
    Assume by contradiction that $|S|\geq \sqrt{cn}+1$.
    Hence by the pigeonhole principle, there exists $s_1<s_2\in S$ such that $s_1=s_2 \mod \sqrt{cn}$.
    Hence for all $i\in[0,\sqrt{cn}-\tfrac{s_2-s_1}{\sqrt{cn}}]$,
    \begin{align*}
        Q(i) = F_{s_2}(i\sqrt{cn})
            = F_{s_1}(s_2-s_1 + i\sqrt{cn})
            = Q(\tfrac{s_2-s_1}{\sqrt{cn}} + i),
    \end{align*}
    where the first and last transitions hold since $s_1,s_2\in S$ and $\tfrac{s_2-s_1}{\sqrt{cn}}$ is an integer number, and the second transition is by definition.
    Thus $Q$ has a period of length $\tfrac{s_2 - s_1}{\sqrt{cn}}\leq \lfloor\tfrac{s_2}{\sqrt{cn}}\rfloor$.
    However, for $i\in [\sqrt{cn}-\lfloor\tfrac{s_2}{\sqrt{cn}}\rfloor +1 , \sqrt{cn}]$ the values that $Q$ get are $Q(i) = F_{s_2}(i\sqrt{cn})=1$ since $s_2 + i\sqrt{cn}\geq cn$;
    % the last $\lfloor\tfrac{s_2}{\sqrt{n}}\rfloor$ entries in $Q$ are equal $1$,
    thus all entries in $Q$ are equal $1$, which contradicts the fact that $Q(0)=0$, and thus completes the proof.
\end{proof}

Algorithm~\ref{alg:sqrt_n_shift_finding} returns the shift $s^*$ since $s^*\in S$ and by the correctness of the verifier in Lemma~\ref{lem:shift_verifier_witness_2queries}.
The number of queries Algorithm~\ref{alg:sqrt_n_shift_finding} makes is $O(|S|+|Q|)=O(\sqrt{cn})$, which proves Theorem~\ref{thm:shift_finding_alg}.

%%% Local Variables:
%%% mode: latex
%%% TeX-master: "main"
%%% End: