\documentclass{ifacconf}
\usepackage{natbib}        % required for bibliography
\usepackage{myStyle}
\providecommand{\dt}{\ensuremath{k}\xspace}
\renewcommand{\dt}{\ensuremath{k}\xspace}


%% refs
\DeclareDocumentCommand{\theoremRef}{m}{Theorem~\ref{#1}}
\DeclareDocumentCommand{\figRef}{m}{Fig.~\ref{#1}}
\DeclareDocumentCommand{\assRef}{m}{Assumption.~\ref{#1}}
\DeclareDocumentCommand{\tabRef}{m}{Table~\ref{#1}}
\DeclareDocumentCommand{\secRef}{m}{Section~\ref{#1}}
\DeclareDocumentCommand{\lemmaRef}{m}{Lemma~\ref{#1}}
\DeclareDocumentCommand{\defRef}{m}{Definition~\ref{#1}}
\DeclareDocumentCommand{\exampleRef}{m}{Example~\ref{#1}}
\usepackage{url}
\usepackage{booktabs}
\usepackage{cases}
\DeclareDocumentCommand{\alert}{m}{\textcolor{red}{#1}}

\usepackage{eso-pic}
\AddToShipoutPictureBG*{%
	\AtPageUpperLeft{%
		\setlength\unitlength{1in}%
		%% change \dimexpr0.5\paperwidth\relax appropriately
		\hspace*{\dimexpr0.5\paperwidth\relax} 
		\makebox(0,-1.75)[c]{
			\begin{tabular}{c c}
					Max van Haren,
				
				%% change the line break \\ as necessary
				Polynomial Feedforward for Linear Parameter-Varying Systems: a Kernel Regularized Approach, \\
				
				%% try to avoid submitted to, only publish OA when accepted							
				To appear in
				
				%% when the paper is not accepted yet, 
				%% do not mention specific conference or journal
				{\em 22nd IFAC World Congress}, 
				Yokohama, Japan, 2023, 
				
				%% change the line break and/or date \\ as necessary
				uploaded to arXiv \today \\ 
\end{tabular}}}}



\begin{document}
%opening
\begin{frontmatter}
\title{%Polynomial Feedforward for Linear Parameter-Varying Systems: a Kernel Regularized Approach
%1: Identification of Dynamically-Scheduled Inverse Models of LPV Motion Systems: a Kernel-Based Approach \\
%Identification of Inverse Models of LPV Motion Systems with Dynamic Scheduling: a Kernel-Based Approach \\
%Feedforward Control for LPV Motion Systems with Dynamic Scheduling: a Kernel-Based Approach. \\
%2: Kernel-Based Identification of Dynami cally-Scheduled Inverse Models of LPV Motion Systems, with Application to Feedforward Control
%Feedforward for LPV Motion Systems: A Kernel-Based Identification Approach
A Kernel-Based Identification Approach to LPV Feedforward: With Application to Motion Systems
}
\thanks[footnoteinfo]{This work is part of the research programme VIDI with project number 15698, which is (partly) financed by the Netherlands Organisation for Scientific Research (NWO). In addition, this research has received funding from the ECSEL Joint Undertaking under grant agreement 101007311 (IMOCO4.E). The Joint Undertaking receives support from the European Union Horizon 2020 research and innovation programme.}
\author[First]{M. van Haren} 
\author[Second,First]{L. Blanken} 
\author[First,Third]{T. Oomen}

\address[First]{Control Systems Technology Section, Eindhoven University of Technology, The Netherlands (e-mail: m.j.v.haren@tue.nl).}
\address[Second]{Sioux Technologies, Eindhoven, the Netherlands}
\address[Third]{Delft Center for Systems and Control, Delft University of Technology, the Netherlands}

\begin{abstract}
The increasing demands for motion control result in a situation where Linear Parameter-Varying (LPV) dynamics have to be taken into account. Inverse-model feedforward control for LPV motion systems is challenging, since the inverse of an LPV system is often dynamically dependent on the scheduling sequence. The aim of this paper is to develop an identification approach that directly identifies dynamically scheduled feedforward controllers for LPV motion systems from data. In this paper, the feedforward controller is parameterized in basis functions, similar to, e.g., mass-acceleration feedforward, and is identified by a kernel-based approach such that the parameter dependency for LPV motion systems is addressed. The resulting feedforward includes dynamic dependence and is learned accurately. The developed framework is validated on an example.
\end{abstract}
\begin{keyword}
	Mechatronics, Motion control systems, Linear parameter-varying systems, Bayesian methods, data-driven control
	% LPV system identification, Learning for control, grey box modelling, machine learning, data-based control, 
\end{keyword}
\end{frontmatter}

% Importance and appeal of children's drawings
Children's depictions of the human figure are highly expressive and varied.
As one of the very first subjects children attempt to draw, the representation begins as an almost unintelligible cloud of scribbles. 
As the child grows, their representation of the human figure becomes more developed and is extended to graphically represent many different types of characters: people, animals, and even personified objects (see Figure 1).

Who among us has not wished, either as a child or as an adult, to see such figures come to life and move around on the page?
Sadly, while it is relatively fast to produce a single drawing, creating the sequence of images necessary for animation is a much more tedious endeavor, requiring discipline, skill, patience, and sometimes complicated software.
As a result, most of these figures remain static upon the page.

% We built a system to animate them.
Inspired by the importance and appeal of the drawn human figure, we design and build a system to automatically animate it given an in-the-wild photograph of a child's drawing. 
Our system is fast, intuitive, and robust to much of the variation present in these types of drawings, making it well-suited to allow our target audience--children--to see their own characters coming to life.
The system is comprised of four stages: figure detection, segmentation masking, pose estimation/rigging, and animation. 
We describe each stage and identify common causes of failure in each. 
For object detection and pose estimation, we make use of existing computer vision models designed to detect human figures and joints in photographs; we fine-tune these models for use with children's drawings.
For segmentation, we present a straightforward, image processing-based method that, for animation purposes, is more useful and accurate than segmentation masks obtained from a fine-tuned object detection model.
During the animation step, we take advantage of the \textit{twisted perspective} commonly seen in children’s drawings to retarget motion capture data onto the character in a novel and appealing way.

% We use existing machine learning models. However, given the wide domain gap it's not clear how much fine-tuning data was needed. So we ran some experiments to find out and report it.
While our system leverages existing models and techniques, most are not directly applicable to the task due to the many differences between photographic images and simple pen and paper representations. 
To this end, we couple the presentation of our system with a set of experiments exploring the relationship between fine-tuning training set size and success rates.
We also include a perceptual study validating viewer preference for incorporating \textit{twisted perspective} into the motion retargeting step.

We validate the desirability and appeal of our system by building and publicly releasing a version of it as the \AD Demo \,\cite{animateddrawings}.
Launched in December 2021, this demo has been used by millions of people around the world to animate their children's drawings.
Inspired by this reception, our second contribution is The Amateur Drawings Dataset: \hjs{180,000 drawings and user-accepted annotations collected, with consent, through the demo. See Section \ref{sec:UI} for a description of how the annotations were generated.}
We believe this dataset will be a resource to researchers from various fields seeking to better understand the space of amateur drawings, evaluate new algorithms in this domain, or develop new drawing-based tools in general.

To summarize, our contributions are as follows:
\begin{enumerate}
    \item 
    We explore the problem of automatic sketch-to-animation for children's drawings of human figures and present a framework that achieves this effect. We also present a set of experiments determining the amount of training data necessary to achieve high levels of success and a perceptual study validating the usefulness of our motion retargeting technique.
    \item To encourage additional research in the domain of amateur drawings, we present a first-of-its-kind dataset of 180,000 user-submitted amateur drawings, along with user-accepted bounding box, segmentation mask, and joint location annotations.
\end{enumerate}

Upon acceptance of this paper, we plan to publicly release the Amateur Drawings Dataset, project code, and fine-tuned model weights.

%\cleardoublepage
\section{Problem Formulation}
\label{sec:problem}
In this section, the problem related to feedforward control for LPV motion systems is formulated. First, the control setting and feedforward goal for LPV motion systems is described. Second, polynomial feedforward for LTI systems is investigated. Third, challenges in designing feedforward controllers for LPV motion systems are shown, that motivates the problem definition in \secRef{sec:pdef}.
%%%%%%%%%%%%%%%%%%%%%%%%%%%%%%%%%%%%%%%%%%%%%%%%%%%%%%%%%%%%%%%%%%%%%%%%%%%%%%%%%%%%%%%%%%%%%%%%%%%%%%%%%%%%%%%%%%%%%%
\subsection{Control Setting}
The control goal is to develop LPV feedforward controller $F_{LPV}$ to reduce the tracking error $e=r-y$ for single-input single-output LPV system $G_{LPV}$, where perfect tracking is achieved by $F_{LPV}=G_{LPV}^{-1}$. The control structure can be seen in \figRef{fig:controlStructure}, where $C$ is a stabilizing feedback controller.
\begin{figure}
	\centering
	\includegraphics{pdf/controlStructure.pdf}
	\caption{Feedforward structure considered}
	\label{fig:controlStructure}
\end{figure}
The reference trajectory $r$ is a smooth reference, which can be differentiated at least four times, as in \citet{Lambrechts2005}. The considered class of LPV systems $G_{LPV}$ can be represented in Continuous-Time (CT) by input-output representations and are shown in \defRef{def:LPV}.
\begin{defn}[CT-IO-LPV system]
	\label{def:LPV}
	The considered class of LPV motion systems are static dependent on the scheduling and given by
	\begin{equation}
		\label{eq:LPVdesc}
			G_{LPV}:\:\sum_{i=-2}^{n_a}a_i(\rho(t)) y^{(i)}(t) \!=\!\sum_{j=0}^{n_{\mathrm{b}}}b_j(\rho(t)) \frac{d^{j}}{d t^{j}} \iint u(t) \, dt^2 
	\end{equation}
with scheduling sequence $\rho \in \mathbb{R}^{n_\rho}$ and $y^{(i)}(t)$ is the ${i}$th time derivative of $y(t)$ if $i\geq0$, and the $i$th integral if $i<0$. The following is assumed of the considered class of LPV systems.
\begin{assum}
	\label{ass:assLPVSystem}
	The following two assumptions are made for the considered LPV systems.
	\begin{enumerate}
		\item Only the body that is of interest is connected to the fixed world, hence the double integral of the input signal $u$ in \eqref{eq:LPVdesc}.
		\item The contributions of the zero dynamics to the system, i.e., $\sum_{j=1}^{n_b}b_j(\rho(t))\frac{d^j}{dt^j}\iint u(t) \: dt^2$, are negligible.
	\end{enumerate}
\end{assum}
The LPV coefficients $a_i(\rho(t))$ and $b_j(\rho(t))$ have a static dependency on $\rho(t)$, e.g., $a_i(\rho(t))=\rho(t)$ or $a_i(\rho(t)) = \rho^3(t)$. For ease of notation, the dependence of signals on $(t)$ is from now on omitted.
\end{defn}
The considered class of LPV motion systems in \defRef{def:LPV} shows a double integral of the input signal, which is generally the case for motion systems with masses. 
\begin{rem}
	The LPV coefficients $a_i(\rho)$ and $b_j(\rho)$ have a static dependency on the scheduling sequence $\rho$, but can be extended to include dynamic dependency, i.e., $a_i(\rho,\dot{\rho},\ldots)$ and $b_j(\rho,\dot{\rho},\ldots)$, and is part of ongoing research.
\end{rem}


%%%%%%%%%%%%%%%%%%%%%%%%%%%%%%%%%%%%%%%%%%%%%%%%%%%%%%%%%%%%%%%%%%%%%%%%%%%%%%%%%%%%%%%%%%%%%%%%%%%%%%%%%%%%%%%%
\subsection{LTI Feedforward}
LTI polynomial feedforward, see e.g. \Citet{Lambrechts2005}, approximates an inverse system to reduce the tracking error $e$. LTI polynomial feedforward applied to the LPV system in \eqref{eq:LPVdesc} parameterizes the feedforward by evaluating \eqref{eq:LPVdesc} at $\rho=\bar{\rho}$, assumes that $b_j=0 \: \forall j>0$ according to \assRef{ass:assLPVSystem}, and differentiates both sides twice, resulting in
\begin{equation}
	\label{eq:FFPolLTI}
	F_{LTI}:\quad u_{ff}= \sum_{i=-2}^{n_a}\frac{ a_i(\bar{\rho})}{b_0(\bar{\rho})} \frac{d^{i+2}}{dt^{i+2}}r = \sum_{i=1}^{n_\theta} \theta_i \psi_i\left( \frac{d}{dt}\right)r,
\end{equation}
where $\psi_i(\frac{d}{dt})$ contain differentiators, e.g., $\psi_i(\frac{d}{dt})=\frac{d^2}{dt^2}$ for acceleration feedforward. The parameters can be tuned manually \citep{Lambrechts2005} or estimated by using data \citep{Boeren2018a}. The resulting feedforward controller is an interpretable, simple and effective strategy for LTI systems, but does not take LPV dynamics into account. 
%%%%%%%%%%%%%%%%%%%%%%%%%%%%%%%%%%%%%%%%%%%%%%%%%%%%%%%%%%%%%%%%%%%%%%%%%%%%%%%%%%%%%%%%%%%%%%%%%%%%%%%%%%%%%%%%
\subsection{Feedforward Problem for LPV Systems}
Developing an inverse model for polynomial feedforward for LPV systems, similar to polynomial feedforward for LTI systems, is challenging due to the dynamic scheduling dependency introduced by deriving an inverse model. The dynamic dependence is observed when inverting \eqref{eq:LPVdesc}, i.e., by differentiating both sides twice, derivatives of the scheduling sequence $\rho$ directly appear. A fixed structure for identifying the inverse model could be used, e.g., a polynomial of $\rho$, $\dot{\rho}$ and $\ddot{\rho}$, however, it is unclear how to choose the structure and order. \exampleRef{example:LPVinverse} illustrates the complexity of feedforward control for LPV systems.

\begin{exmp}
	\label{example:LPVinverse}
	(Feedforward problem for LPV system) Consider the two-mass-spring-damper system with parameter-dependent spring in \figRef{fig:LPVMSD}, with input $u$, that is the force on the first mass, and output $y$, that is the position of the second mass.
	\begin{figure}
		\centering
		\includegraphics{pdf/TwoMassSystem.pdf}
		\caption{LPV mass-spring-damper with parameter-dependent spring $k(\rho)$ that is considered in this paper.}
		\label{fig:LPVMSD}
	\end{figure}
	The input-output behavior in the form of \eqref{eq:LPVdesc} is given by
	\begin{equation}
		\label{eq:exampleIO}
		\resizebox{\hsize}{!}{%
			$
			\begin{aligned}
				&\left(m_2m_1\frac{d^2}{dt^2} +\left( c\left(m_1+m_2 \right)+c_2m_1\right) \frac{d}{dt}+\left( k(\rho)\left(m_1+m_2 \right)+cc_2 \right)  \right) y \\
				&+k(\rho)c_2 \int y \,dt= \left(c\frac{d}{dt}+k(\rho) \right) \iint u \,dt^2,
			\end{aligned}$}%
	\end{equation}
	which directly shows the double integral of the input signal seen in \eqref{eq:LPVdesc}. LTI polynomial feedforward from \eqref{eq:FFPolLTI} is
	\begin{equation}
		\label{eq:LTIPolFFExample}
		\begin{aligned}
			u_{ff} = &\frac{m_2m_1}{k}\ddddot{r} + \frac{c_2m_1+c\left( m_1+m_2\right)}{k}   \dddot{r}\\
			&+ \left( \left(m_1+m_2 \right)+\frac{cc_2}{k}\right)\ddot{r} + c_2\frac{d}{dt}\dot{r} ,
		\end{aligned}
	\end{equation}
	that consists of the well-known snap, jerk, acceleration and velocity feedforward. However, the true inverse dynamics of \eqref{eq:exampleIO}, when neglecting the zero dynamics of the system, are given by
	\begin{equation}
		\label{eq:exampleIO2}
		\resizebox{\hsize}{!}{
			$
			\begin{aligned}
				u&\approx\frac{m_2m_1}{k(\rho)}\ddddot{y} + \frac{c\left( m_1+m_2\right)+c_2m_1}{k(\rho)} \dddot{y}+ \left(m_1+m_2 \right)\ddot{y} +c_2 \dot{y}\\
				&+ \left( \frac{\frac{2\dot{\rho}^2{k^{\prime}}^2(\rho)}{k(\rho)}-\dot{\rho}^2k^{\prime\prime}(\rho)-\ddot{\rho}k^\prime(\rho)}{k^2(\rho)}-\frac{2 \dot{\rho}k^\prime(\rho)  }{k^2(\rho)}\right)f(y,\dot{y},\ddot{y}),\\
			\end{aligned}$}
	\end{equation}
	with $f(y,\dot{y},\ddot{y}) = m_1m_2  \ddot{y}+\left(c\left( m_1+m_2\right) + c_2m_1\right) \dot{y}\allowbreak+\allowbreak \left(k(\rho)\left( m_1+m_2\right) +cc_2\right)y $, where $k^\prime=\frac{dk(\rho)}{d\rho}$ and $k^{\prime\prime}=\frac{d^2k(\rho)}{d\rho^2}$. When comparing LTI feedforward in \eqref{eq:LTIPolFFExample} with the approximate inverse in \eqref{eq:exampleIO2}, it is observed that LTI feedforward lacks both static and dynamic dependency on $\rho$.
\end{exmp}



%%%%%%%%%%%%%%%%%%%%%%%%%%%%%%%%%%%%%%%%%%%%%%%%%%%%%%%%%%%%%%%%%%%%%%%%%%%%%%%%%%%%%%%%%%%%%%%%%%%%%%%%%%%%%%%%%%%%
\subsection{Problem Definition}
\label{sec:pdef}
A technique for manual tuning or direct data-driven identification of feedforward controllers for LPV motion systems, including dynamic dependency on the scheduling sequence, is currently lacking. The problem addressed in this paper is the direct identification of LPV polynomial feedforward controller of the form
\begin{equation}
	\label{eq:Ftilde}
	\tilde{F}_{LPV}:\quad u_{ff} = \sum_{i=1}^{n_\theta} \theta_i(\rho,\dot{\rho},k^\prime(\rho),\ldots) \psi_i\left( \frac{d}{dt}\right)r,
\end{equation}
based on input-output data $\{u,y\}$, for the class of LPV motion systems in \eqref{eq:LPVdesc}, that includes dynamic dependence on the scheduling sequence $\rho$, where the structure and order of the model is not fixed a priori, and minimizes the tracking error $e$ in the configuration of \figRef{fig:controlStructure}.












%\newpage
\section{Linearly Parameterized Feedforward for LPV motion systems}
\label{sec:BFFF}
In this section, a polynomial feedforward strategy for LPV systems as in \eqref{eq:Ftilde} is developed, by posing an alternative parameterization for the class of LPV motion systems in \eqref{eq:LPVdesc}, that includes dynamic dependency on the scheduling sequence, but simplifies the identification problem significantly.

The key idea is to rewrite the system dynamics in \eqref{eq:LPVdesc} as
\begin{equation}
	\label{eq:IOModel}
	\sum_{i=-n_i}^{n_a}a_i(\rho) y^{(i)}=\sum_{j=0}^{n_{\mathrm{b}}}b_j(\rho) \frac{d^{j}}{d t^{j}} w,
\end{equation}
where a change of variables is used as $w={ \iint} u \, dt^2$.
%The key idea is to define the feedforward controller using the second integral of the input signal $u$, that is equal to $w$, and defining the parameters in this domain. Hence, the dynamics of the LPV system in \eqref{eq:LPVdesc} are written as
%\begin{equation}
%	\label{eq:IOModel}
%	\sum_{i=-n_i}^{n_a}a_i(\rho) y^{(i)}=\sum_{j=0}^{n_{\mathrm{b}}}b_j(\rho) \frac{d^{j}}{d t^{j}} w.
%\end{equation}
Similarly to LTI polynomial feedforward in \eqref{eq:FFPolLTI}, the inverse model for LPV systems is parameterized by setting the values of $b_j \, \forall j\geq 1$ in \eqref{eq:IOModel} to zero, resulting in
\begin{subnumcases}{F_{LPV}:}
	w_{ff}=\sum_{i=1}^{n_\theta}\theta_i(\rho) \psi_i\left(\frac{d}{dt},I \right)r, \label{eq:FFModel}
	\\
	u_{ff} = \frac{d^2}{dt^2}w_{ff}, \label{eq:FFCalc}
\end{subnumcases}
where $\psi$ contains differentiators $\frac{d}{dt}$ or integrals $I$, e.g. $\psi_i\left( \frac{d}{dt},I\right) r = \frac{d^2}{dt^2}r=\ddot{r}$ or $\psi_i\left(\frac{d}{dt},I\right) r=Ir = { \int} r \, dt$. 

Note that from \eqref{eq:FFModel}, the second integral of the input $w_{ff}$ is composed out of basis functions, in contrast to the input $u_{ff}$ for LTI polynomial feedforward. The dynamic dependence on the scheduling sequence seen in \eqref{eq:exampleIO2} is introduced by the second derivative with respect to time in \eqref{eq:FFCalc}, which introduces time derivatives of $\theta(\rho)$. In \exampleRef{example:LPVmsdFF}, an example is shown for the two-mass system.
\begin{exmp}[LPV feedforward]
	\label{example:LPVmsdFF}
	Consider the two mass-spring-damper system from \exampleRef{example:LPVinverse}, with input-output behavior in \eqref{eq:exampleIO}. The polynomial feedforward strategy is then defined, by neglecting the zero in the right-hand side of \eqref{eq:exampleIO} according to \assRef{ass:assLPVSystem}, i.e., $\frac{c}{k(\rho)}\approx 0$, as
	\begin{equation}
		\label{eq:exampleFFPara}
		\resizebox{0.85\hsize}{!}{
			$
		\begin{aligned}
			w_{ff} =  &\underbrace{\vphantom{\frac{m_2}{k(\rho)}}c_2}_{\theta_1(\rho)}\underbrace{\vphantom{\frac{m_2}{k(\rho)}}\int}_{\psi_1} r \, dt+\underbrace{\vphantom{\frac{m_2m_1}{k(\rho)}}\left(m_1+m_2 + \frac{cc_2}{k(\rho)} \right)}_{\theta_2(\rho)}\underbrace{\vphantom{\frac{m_2m_1}{k(\rho)}}1}_{\psi_2}r\\
			&+\underbrace{\frac{c\left( m_1+m_2\right)+c_2m_1}{k(\rho)}}_{\theta_3(\rho)}\underbrace{\vphantom{\frac{m_2m_1}{k(\rho)}}\frac{d}{dt}}_{\psi_3}r+\underbrace{\frac{m_2m_1}{k(\rho)}}_{\theta_4(\rho)}\underbrace{\vphantom{\frac{m_2m_1}{k(\rho)}}\frac{d^2}{dt^2}}_{\psi_4}r,
		\end{aligned}$}
	\end{equation}
	where the applied feedforward force is calculated using \eqref{eq:FFCalc}. The applied feedforward force contains both static and dynamic scheduling dependence when substituting \eqref{eq:exampleFFPara} into \eqref{eq:FFCalc}, i.e.,
	\begin{equation}
		\label{eq:exampleResultingFF}
		\resizebox{\hsize}{!}{
			$
		\begin{aligned}
			u_{ff} &= \frac{m_2m_1}{k(\rho)}\ddddot{r} + \frac{c\left( m_1+m_2\right) +c_2m_1}{k(\rho)} \dddot{r}+ \left(m_1+m_2 \right)\ddot{r} +c_2 \dot{r}\\
			& + \left( \frac{\frac{2\dot{\rho}^2{k^{\prime}}^2(\rho)}{k(\rho)}-\dot{\rho}^2k^{\prime\prime}(\rho)-\ddot{\rho}k^\prime(\rho)}{k^2(\rho)}-\frac{2 \dot{\rho}k^\prime(\rho)  }{k^2(\rho)}\right)f(r,\dot{r},\ddot{r}),\\
		\end{aligned}
	$}
	\end{equation}
which is equal to \eqref{eq:exampleIO2} when substituting $y$ for $r$.
\end{exmp}
The applied feedforward force $u_{ff}$ in \eqref{eq:FFCalc} includes the dynamic dependency on the scheduling signal, e.g. shown in \eqref{eq:exampleIO2}, while the modeled $w_{ff}$ in \eqref{eq:FFModel} is only statically dependent on the scheduling sequence.

%The resulting signal $w_{ff}$ in \eqref{eq:exampleFFPara} has less parameters than the signal $u$ in \eqref{eq:exampleIO2}, hence the parameters are easier to identify, and contains the dynamic dependence due to the calculation of $u_{ff}$ by taking the second derivative.






%\newpage
\section{Kernel Regularized Learning of LPV Feedforward Parameters}
\label{sec:kernel}
In this section, the functions $\theta_i(\rho)$ in \eqref{eq:FFModel} are identified using kernel regularization, which models the functions without a specified structure or order, since the solution is in the infinite-dimensional Reproducing Kernel Hilbert Space (RKHS). Second, kernel design for LPV feedforward parameters is described. Finally, the developed approach is summarized in a procedure.
%%%%%%%%%%%%%%%%%%%%%%%%%%%%%%%%%%%%%%%%%%%%%%%%%%%%%%%%%%%%%%%%%%%%%%%%%%%%%%%%%%%%%%%%%%%%%%%%%%%%%%%%%%%%%%%%%%%%%%%%%%%%%%%
\subsection{Kernel Regularized Identification}
A cost function is defined using input-output data as \citep{Pillonetto2014,Blanken2020}
\begin{equation}
	\label{eq:minimization}
	\hat{\Theta} = \arg \min_\Theta \left\| \overline{w}-\Phi\Theta\right\|^2 + \gamma\|\Theta\|^2_\mathcal{H},
\end{equation}
with Euclidean norm $\|\cdot\|$, $\Phi\Theta$ equal to $w_{ff}$ in \eqref{eq:FFModel}, and measurement data vector $\overline{w}$, that is constructed as
\begin{equation}
	\label{eq:defineWbar}
	\overline{w}= \begin{bmatrix}w(0T_s) & w(1T_s) & \cdots & w((N-1)T_s)\end{bmatrix}^\top.
\end{equation}
The squared induced norm on the RKHS $\mathcal{H}$ is denoted as $\|\Theta\|^2_\mathcal{H}$, that is given by \citep{Pillonetto2014},
\begin{equation}
	\label{eq:RKHS}
	\|\Theta\|^2_\mathcal{H} = \Theta^\top K^{-1} \Theta,
\end{equation}
with kernel $K$. The parameter vector $\Theta$ and basis function matrix $\Phi$ are built up as
\begin{equation}
	\label{eq:ThetaPhi}
	\begin{aligned}
		\Theta = \begin{bmatrix}
			\overline{\theta}_1^\top &
			\overline{\theta}_2^\top &
			\cdots &
			\overline{\theta}_{n_\theta}^\top
		\end{bmatrix}^\top, &&
		\Phi = \begin{bmatrix}
			\overline{\phi}_1 & \overline{\phi}_2 & \cdots & \overline{\phi}_{n_\theta}
		\end{bmatrix},
	\end{aligned}
\end{equation}
where the individual parameter vector $\overline{\theta}_i$ and $\overline{\phi}$ are constructed by gathering the values over a training period as
\begin{equation}
	\label{eq:defineThetaPhibar}
			\resizebox{0.85\hsize}{!}{
		$
	\begin{aligned}
		\overline{\theta}_i &= \begin{bmatrix}
			\left( \theta_i(\rho)\right)\left( 0T_s\right)  \\
			\left(\theta_i(\rho)\right)\left( 1T_s\right)  \\
			\vdots \\
			\left(\theta_i(\rho)\right)\left( \left(N-1 \right) T_s)\right) 	\end{bmatrix} \\
		\overline{\phi}_{i}&=\begin{bmatrix}
			\left(\psi_{i} y\right)\left(0 T_{s}\right) & 0 & \cdots & 0 \\
			0 & \left(\psi_{i} y\right)\left(1 T_{s}\right) & \cdots & \vdots \\
			\vdots & \ddots & \ddots & \vdots \\
			0 & \cdots & \cdots & \left(\psi_{i} y\right)\left((N-1) T_{s}\right)
		\end{bmatrix}
	\end{aligned}
$}
\end{equation}
where $(\frac{d}{dt},I)$ has been left out for brevity. 

The solution to the cost function in \eqref{eq:minimization} is given by \citep{Pillonetto2014,Blanken2020}
\begin{equation}
	\label{eq:solTheta}
	\hat{\Theta} =K \Phi^{\top}\left(\Phi K \Phi^{\top}+\gamma I_{N}\right)^{-1} \overline{w},
\end{equation}
where parameters $\theta$ are estimated at any $\rho^*$ using the representer theorem \cite[Section~9.2]{Pillonetto2014}.
 \begin{rem}
	\label{rem:bias}
Note that \eqref{eq:minimization} is an open-loop solution, while a closed-loop control structure is assumed as shown in \figRef{fig:controlStructure}, hence measurement noise introduces bias \Citep{Blanken2020}. The addition of instrumental variables is capable of removing this bias, and is reported elsewhere.
\end{rem}
The kernel $K$ can be designed to incorporate prior knowledge on the feedforward parameters, such as smoothness or periodicity, and will be discussed in the next section.


%%%%%%%%%%%%%%%%%%%%%%%%%%%%%%%%%%%%%%%%%%%%%%%%%%%%%%%%%%%%%%%%%%%%%%%%%%%%%%%%%%%%%%%%%%%%%%%%%%%%%%%%%%%%%%%%%%%%%%%%%%%%%%%%
\subsection{Kernels for LPV Feedforward Parameters}
The kernel incorporates prior knowledge on the feedforward parameters, hence is important to design carefully. The optimal kernel for solving \eqref{eq:minimization}, when treating feedforward parameters as random variables, is equal to
\begin{equation}
	\label{eq:optKernel}
		\resizebox{0.88\hsize}{!}{
		$
	\Pi = \mathbb{E}\left(\Theta \Theta^\top \right) = \begin{bmatrix}
		\mathbb{E}( \overline{\theta}_1\overline{\theta}_1^\top) & 	\mathbb{E}( \overline{\theta}_1\overline{\theta}_2^\top) & 	\cdots & 	\mathbb{E}( \overline{\theta}_1\overline{\theta}_{n_\theta}^\top) \\
		\mathbb{E}( \overline{\theta}_2\overline{\theta}_1^\top) & \mathbb{E}( \overline{\theta}_2\overline{\theta}_2^\top) & \cdots & \vdots \\
		\vdots & \vdots & \ddots & \vdots \\
		\mathbb{E}( \overline{\theta}_{n_\theta}\overline{\theta}_1^\top) & \cdots & \cdots & \mathbb{E}( \overline{\theta}_{n_\theta}\overline{\theta}_{n_\theta}^\top)
	\end{bmatrix}. $}
\end{equation}
For LPV motion systems, parameters may correlate, i.e., $\mathbb{E}\left( \overline{\theta}_i\overline{\theta}_j\right) \neq 0 \, \forall j\neq i$. For example, when looking at \eqref{eq:exampleFFPara}, parameters $\theta_3$ and $\theta_4$ are scaled versions of each other. Hence, the framework is capable of incorporating correlation between feedforward parameters.


The optimal kernel is approximated by a kernel matrix,
\begin{equation}
	\label{eq:covker}
	\mathbb{E}\left( \overline{\theta}_i\overline{\theta}_j^\top\right) = K_{ij}(\overline{\rho},\overline{\rho}),
\end{equation}
which only has a static dependency on $\rho$, while the framework produces feedforward which is dynamically dependent on the scheduling sequence as shown in \secRef{sec:BFFF}. 

The kernel matrix $K_{ij}$ is determined by evaluating a kernel function, such as the squared exponential kernel function 
\begin{equation}
	\label{eq:SEKernel}
	k_{ij,SE}(\rho,\rho^\prime) = \sigma_{ij}^2\exp\left( -\frac{\left( \rho-\rho^\prime\right) ^2}{2\ell_{ij}^2}\right) .
\end{equation}
The hyperparameters of the kernel, i.e., for the squared exponential kernel in \eqref{eq:SEKernel} the output variances $\sigma_{ij}^2$ and length scales $\ell_{ij}$ can be tuned using marginal-likelihood optimization. The kernel choice provides the user to apply prior knowledge on the feedforward parameters.

%%%%%%%%%%%%%%%%%%%%%%%%%%%%%%%%%%%%%%%%%%%%%%%%%%%%%%%%%%%%%%%%%%%%%%%%%%%%%%%%%%%%%%%%%%%%%%%%%%%%%%%%%%%%%%%%%%%%%%%%%%%%%%%%
\subsection{Developed Procedure}
The developed procedure is summarized in Procedure~\ref{proc:1}.
\vspace{3pt}\hrule\begin{proced} \textit{(Kernel regularized LPV feedforward identification)} \hfill \vspace{0.5mm} \hrule
	\label{proc:1}
	\begin{enumerate}
		\item Apply reference $r$ to closed-loop system in \figRef{fig:controlStructure} and record $y$, $\rho$ and $u$.
		\item Construct kernel matrices by evaluating \eqref{eq:covker}. 
		\item Calculate matrix $\Phi$ using \eqref{eq:ThetaPhi} and \eqref{eq:defineThetaPhibar}.
		\item Compute $\bar{w}$ from \eqref{eq:defineWbar} using the second integral of the input $w=\iint u \: dt^2$.
		\item Estimate the feedforward parameters $\hat{\Theta}$ using \eqref{eq:solTheta}.
	\end{enumerate}
	\vspace{0pt} 	\hrule 	\vspace{-2pt}
\end{proced}
To conclude, kernel regularized identification is capable of identifying LPV feedforward parameters with input-output data of a system, without specifying a structure or order. In the following section, an example is shown that validates the developed framework.
%\newpage
\documentclass[a4paper,twoside]{article}

\usepackage{epsfig}
\usepackage{subcaption}
\usepackage{calc}
\usepackage{amssymb}
\usepackage{amstext}
\usepackage{amsmath}
\usepackage{amsthm}
\usepackage{multicol}
\usepackage{multirow}
\usepackage{pslatex}
\usepackage{makecell}
\usepackage{apalike}
\usepackage{xurl}
\usepackage{tabularx}
\usepackage[bottom]{footmisc}
\usepackage{SCITEPRESS}     % Please add other packages that you may need BEFORE the SCITEPRESS.sty package.

% OUR PACKAGES
\usepackage{natbib} % citing authors by names with \citet{}, can be used as \usepackage[numbers]{natbib} for numbered citations


\begin{document}

\title{Rethinking Certification for Higher Trust and Ethical Safeguarding of~Autonomous Systems}

% \author{\authorname{Dasa Kusnirakova\orcidAuthor{0000-0002-5341-902X} and Barbora Buhnova\orcidAuthor{0000-0003-4205-101X}}
\author{\authorname{Dasa Kusnirakova and Barbora Buhnova}
\affiliation{Faculty of Informatics, Masaryk University, Brno, Czech Republic}
% \affiliation{\sup{2}Department of Computing, Main University, MySecondTown, MyCountry}
\email{\{kusnirakova, buhnova\}@mail.muni.cz}
}

\keywords{Autonomous Systems, Trust, Certification, Regulation, Ethics}

\abstract{With the increasing complexity of software permeating critical domains such as autonomous driving, new challenges are emerging in the ways the engineering of these systems needs to be rethought. Autonomous driving is expected to continue gradually overtaking all critical driving functions, which is adding to the complexity of the certification of autonomous driving systems. 
As a response, certification authorities have already started introducing strategies for the certification of autonomous vehicles and their software.
But even with these new approaches, the certification procedures are not fully catching up with the dynamism and unpredictability of future autonomous systems, and thus may not necessarily guarantee compliance with all requirements imposed on these systems. 
In this paper, we identified a number of issues with the proposed certification strategies, which may impact the systems substantially. For instance, we emphasize the lack of adequate reflection on software changes occurring in constantly changing systems, or low support for systems' cooperation needed for the management of coordinated moves. Other shortcomings concern the narrow focus of the awarded certification by neglecting aspects such as the ethical behaviour of autonomous software systems. 
The contribution of this paper is threefold. 
First, we discuss the motivation for the need to modify the current certification processes for autonomous driving systems. 
Second, we analyze current international standards used in the certification processes towards requirements derived from the requirements laid on dynamic software ecosystems and autonomous systems themselves. 
Third, we outline a concept for incorporating the missing parts into the certification procedure.}

\onecolumn \maketitle \normalsize \setcounter{footnote}{0} \vfill



\section{\uppercase{Introduction}}

According to the most recent estimates, the majority of human driving will be replaced by autonomous vehicle (AV) technology by the year 2050 \citep{litman2022}. Besides the expected benefits of increased road safety, significant cost savings and reduced energy consumption and pollution \citep{dia2020}, the introduction of self-driving vehicles into public places creates new challenges for safeguarding the mobility ecosystem as a whole in order to ensure vehicles' safe operation. Therefore, alongside the quickly emerging autonomous software advancements, it is critical to develop regulatory frameworks that would be designed to adapt to the technological changes, so that the safety risks are minimized.

Regulations inevitably have to evolve to keep up with technological progress, more so in the face of the increasing software intensity of autonomous-driving systems. This has already occurred in the history of vehicle certification methods. Initially, when vehicles were made only from mechanical components (brakes, tyres), the driving function and all decision-making was in hands of the driver. In this case, classic certification approaches were sufficient. However, following the introduction of ABS\footnote{Anti-lock Braking System; a safety system used on land vehicles activated in case of a skid in order to allow the driver to maintain more control over the vehicle.} and other systems with a greater level of complexity, it became clear that the classical approach was insufficient in assessing all safety-relevant aspects because of the extensive number of potential testing scenarios. As a consequence, process- and functional-oriented safety audits were introduced, one of which is the Annex 6 of the UN Regulation no.79 \citep{un-reg-79}.

Future road vehicles are expected to gradually overtake more critical driving functions, until manual driving might be fully replaced. This will shift the focus and the responsibility from the driver towards the system installed in the AV. With that, the importance and complexity of the electronic control systems utilized in vehicles will continue to increase. This will also significantly increase the number of possible scenario variations. However, the testing phase performed in the same way as for conventional vehicles, that is, verifying the system based on a predefined set of tests, will be able to thoroughly examine only a limited subset of all safety areas and scenarios \citep{kalra2016}. Besides that, autonomous systems, i.e. \textit{"systems changing their behaviour in response to unanticipated events during operation"} \citep{watson2005autonomous}, among which AVs clearly belong, rely on automatic software updates at runtime in order to adjust to specific environment or context changes \citep{deco2021}, or to improve AI components of an AV in general \citep{cert-eu-report2020}. Again, the traditional certification procedures are not designed to promptly deal with this situation. They do not assume any, or at most a limited number of changes in already certified systems. But just like software engineering does not end with the deployment of the system, future AVs will require novel technological as well as legal approaches for their thorough quality control even during runtime in order to ensure public road safety. 

The challenges of the AV certification go even beyond the safety of autonomous ecosystems. Studies have shown that trust plays an essential role in the adoption of automated systems \citep{trust-cioroaica2019}, from both the societal (people are willing to accept and use the systems) as well as technological (interactions between communicating systems during runtime are trustworthy) point of view \citep{survey-trust-mng2022}. However, continual technological progress of AV exceeding the boundaries of previously defined safety requirements hinders trust formation in these systems \citep{trust-cioroaica2020}, and certification, generally perceived as a trust-building mechanism, is failing to provide sufficient legal guarantees in its current state. It is, therefore, necessary to adapt the certification methods to the envisioned technological advancements and facilitate the adoption process of AV into society.

In this paper, we first explore the necessity to improve the present certification methods for autonomous driving systems, and second, we evaluate recently published standards against the identified requirements for future autonomous systems. Taking into consideration the expertise from social computing as well as the characteristics of dynamic autonomous ecosystems in which such systems operate, we outline a concept for incorporating the missing parts into the certification procedure. To this end, we formulate the requirements for the future certification procedures of autonomous systems, select relevant standards and evaluate them against the defined requirements.
% Furthermore, we analyze the suitability of current certification approaches to these requirements and we list examples in which the existing certification falls short.


% Therefore, we believe currently established certification procedures for conventional vehicles may no longer be appropriate for the future usage. 
The rest of the paper is organized as follows. Section 2 presents related work in the field of certification of autonomous driving systems. Section 3 lists the characteristics and specifies the certification context for autonomous systems, based on which we define the requirements for standards used for the certification of future autonomous systems. In Section 4, we describe the research methodology and select certification standards for evaluation. Section 5 presents the evaluation results. In Section 6, we outline the suggestions for improvements, needed for the standards to reflect the specifics of future autonomous (eco-)systems.

% regular software updates will not concern only newly manufactured cars for sale, but will be needed also for vehicles that have already been produced.

% Such increased dynamics in the software updating process will also increase the probability for a potentially faulty SW being installed in the vehicle (malfunctioning either intentionally - a virus, hijack, or unintentionally - bug not revealed by tests). 



% * Current certification process is sufficient for static environments.
% * But physical and one-time testing, typically at the time of production, is no longer sufficient to assess an AV's performance in a variety of real-world scenarios, and dynamically changing contexts. 
% * Moreover, despite technical, procedural and environmental aspects, morality and ethical behaviour of the smart agents need to be taken into consideration and somehow reflected in the certificate, too.
% Therefore, there is an emerging need for modification of the current certification concept in order to reflect the actual state in a rapidly changing environment and to serve as a trustworthy source of information for all further decision-making processes within such a dynamic autonomous ecosystem.



\section{\uppercase{Related Work}}
\label{sec:relwork}

The question of the suitability of the current vehicle certification processes for the future has already been to some extent discussed in the literature. However, to the best of our knowledge, no paper has yet highlighted the importance of balancing both trust and ethics in the process of certifying autonomous systems.

% critics on current certification
\citet{bonnin2018} has pointed out that certification changes are needed. But their criticism of the standards was mainly directed towards the insufficient reflection of the technological advancement in the connected software development processes. The article does not cover specific characteristics of autonomous vehicles and the ecosystems in which they operate. Other criticisms of certification procedures were detected, too. In \citep{burzio2018}, which asks for modifications from the standpoint of cyber-security. But the most often addressed certification problems are those related to the safety of AVs, as presented in \citep{zhao2022} or \citep{cummings2019}. The last mentioned paper specifically criticizes possibly lower vehicle safety unless software upgrades are taken into account in the certification process. 


Naturally, the identified shortcomings and criticism of the currently used certification procedures sparked the development of suggestions for improvements. In \citep{unece2022}, GRVA\footnote{Working Party preparing draft regulations, guidance documents and interpretation documents for adoption by the parent body} group working under United Nations Economic Commission for Europe (UNECE) in collaboration with experts of the International Organization of Motor Vehicle Manufacturers (OICA) presented a new way of validating autonomous vehicles for the purpose of certification based on a multi-pillar approach consisting of a scenario catalog. \citet{dynamic2022} proposed dynamic certification built on modelling and testing, which are constantly intertwined during the system's life cycle. Besides that, a verifiably-correct dynamic self-certification framework for autonomous systems is discussed in \citep{fisher2018}, while \citet{hussein2021} introduced the concept of a certification framework for autonomous driving systems based on the Turing test. Digital certificates are proposed to be used in combination with trust and reputation policies for ensuring safety and detection of hijacking vehicles in \citep{garcia2019}. All of the attempts partially cover the shortcomings of static certification, however, the debate regarding certification is grounded solely in the context of safety and security and lacks to consider other aspects, such as ethics and trust of AVs. 

% trust and ethics considerations - standards yes, certification not yet
Even though trust and ethics are in general frequently debated issues in the context of self-driving cars,
% despite pointing out their importance \citep{henschke2020,trust-cioroaica2020}, 
the literature presents these concepts more in connection with privacy preservation \citep{lai2021}, or general calls for adjusting the software development standards and best practices \citep{kwan2021,myklebust2020} for building software with social responsibility. Considering trust and ethics directly within the certification process itself does not seem to be covered yet. 
To this end, if we are to trust the systems that make important decisions for us not only in the area of safety, but also in ethics, we must ensure that these aspects are also considered at the stage of vehicle certification providing legal guarantees.



\section{\uppercase{Specification of the certification context}}
\label{sec:specs}
Responsible system certification can be hardly achieved when performed both in isolation from the environment in which the system operates, and without taking into account the characteristics of the system under consideration.
AVs are considered cyber-physical systems (CPSs), i.e. \textit{”smart networked systems with embedded sensors, processors and actuators that are designed to sense and interact with the physical world”} \citep{cps-def}. Therefore, the characteristics of CPSs can aid with the identification of the characteristics of AVs needed for thorough certification. 

The aim of this section is to cover the key characteristics of autonomous CPSs as well as the whole surrounding ecosystems, and specify the certification context based on which we then define the requirements for certification standards for future autonomous systems.


\subsection{Autonomous cyber-physical systems}
The characteristics of autonomous systems need to be considered when determining requirements for certification frameworks for driving systems.
\citet{weyns2021} defines key principles for future CPS engineering principles, from which the characteristics of future autonomous systems can be derived. The following principles are listed: 
(1)~\textit{crossing boundaries} related to close contact between social-, physical-, and cyber-spaces;
(2)~\textit{leveraging the humans} and their integration in the design and operation processes instead of treating them only as users of a system;
% (3) fluid modelling
(3)~\textit{on-the-fly coalitions} as a way of addressing complex problems through forming multi-agent systems;
(4)~\textit{dynamically assured resilience} to withstand uncertainty, context changes or any other disruptions and continue in service provision;
(5)~\textit{learn novel tasks}, that is, utilizing knowledge from the past effectively to deal with novel situations.


\begin{figure*}[t]
\centering
\includegraphics[width=330px]{img.jpg}
\caption{Key certification standards' aspects on future autonomous systems}
\label{fig:reqs}
\end{figure*}


\subsection{Dynamic software ecosystems}
Besides that, the ecosystem in which the autonomous system operates must also be considered. The ecosystem identifies its surroundings and entities communicating with the system, specifies the relationships between them as well as determines the system context, that is, the system's main meaning and ways of use.

\citet{deco2021} defined ecosystems formed by software autonomous systems as ecosystems supporting dynamic, smart, and autonomous features which are required by modern software systems. In particular, the following features need to be taken into consideration when designing certification suitable for autonomous systems: 
(1) \textit{automation} understood as automated self-adaptation of the ecosystem on context changes during runtime;
(2) \textit{autonomy} meaning there is no longer any human monitoring of the systems inside the ecosystem; 
(3) \textit{dynamic goal evaluation} as intelligent adaptation to dynamic needs  with the intention to achieve the proposed and explicitly defined goals; 
% (4) \textit{goals classification} as support for self-regulating mechanisms; and 
(4) \textit{automated trust management} as a crucial concept driving decision-making\footnote{the overarching ecosystem goal derives from the achievement of tactical goals on lower levels and collaboration, which is needed to fulfill the main goal, relies on trust guarantees in such dynamic and hard-to-predict environments}; and 
(5) \textit{architecture implications} as the ecosystem's dynamic nature, driven by the constant (dis-)connecting nodes, impacts the architecture of the ecosystem.


\subsection{Ethical aspects}
In case of human-operated vehicles, the drivers themselves are responsible for applying the basic rules of safety and morality when driving. However, integrating autonomous systems into ecosystems shared with humans shifts the ethical aspects of driving on the system. And when humans are not in charge anymore, the system must bear a certain moral obligation instead. 

Examples of such circumstances include the well-known trolley problem \citep{trolley}, in which a collision is irreversible and any action that is made results in a tragedy. Other examples where ethical aspects are applied concern an exhibition of altruistic behaviour (e.g., informing other road members about a danger they cannot yet see, such as a person crossing the street on the red light), or a demonstration of solidarity with surrounding entities (e.g., transparency in terms of notifying other drivers of one's change in speed or direction before doing so, in order to maintain smooth traffic and avoid dangerous situations).

Thus besides the primary requirements on autonomous systems, which is assuring safety and security \citep{safety-and-security-first}, ethical considerations might govern the gray space where it becomes clear that some level of harm might happen anyway, or an action can be done to help the overall ecosystem, although such action is not required by law.

% In case of human-operated vehicles, the drivers themselves are responsible for applying the basic rules of safety and morality when driving, such as moving out of the way for a rushing emergency vehicle, not parking in parking slots reserved for people with disabilities if not being eligible for it, or stopping for pedestrians crossing the street in a crosswalk. However, integrating autonomous systems into ecosystems shared with humans shifts the ethical aspects of driving on the system. And when humans are not in charge anymore, the system must bear a certain moral obligation instead. 

% % Thus besides the primary requirements on autonomous systems, which is safety and security, ethical considerations might govern the gray space where we either know that some level of harm might happen anyway, or we might do an action to help the overall ecosystem although we do not have to.

% The aim of this paper is not to go against the primary requirements placed on autonomous systems, which is to assure safety and security in every situation. However, we are convinced that it is essential that ethical considerations become a standard principle of autonomous vehicles, not that it is considered only as a nice-to-have feature.
% As stated above, an autonomous CPS system does not operate independently but shares the ecosystem with other entities, such as technological devices but also people, and impacts them both substantially. Therefore, ethical principles must be considered.

The idea of implementation of ethical principles in autonomous driving elaborates even further on one of the requirements for a CPS, in particular \textit{leveraging the humans}. When ethical standards are applied, people are no longer seen as just users but as equal members of the ecosystem with individual needs and dignity.

The research generally agrees on the fact that applying ethical norms is crucial for trust formation in autonomous systems \citep{towards-kwan-2021,ethics-wang-2022,virtue-gerdes-2020}. In human society, ethical diversity is observed, which necessitates the possibility to adjust the moral setting of a system according to one's individual preferences within a personal ethical framework. However, a personal ethical framework with no regulation is insufficient as it would lead to numerous ethical dilemmas. A preferred ethical model should therefore reflect public rational ethical inclinations while at the same time provide users with limited freedom to adjust the particular ethical setting according to their personal preferences \citep{ethics-wang-2022}. 
The contribution to the discussion regarding the form and way of the implementation of specific ethical principles in autonomous systems is out of the scope of this paper, though. Instead, as this paper deals with the readiness of current certification standards for autonomous vehicles, we focus on the examination of whether individual standards take into account ethical issues regarding the driving task. 


\section{\uppercase{Methodology}}
To identify the gaps within the certification procedures of autonomous systems, the requirements on these systems need to be defined and a set of standards for evaluation needs to be established. We derive the requirements based on the characteristics of autonomous systems as well as based on the characteristics in which such systems operate. We draw the knowledge from literature devoted to the study of autonomous systems in dynamic software ecosystems. As for the standards selection, we have selected standards published or re-confirmed within the last seven years (2016-2022) in various fields of the automotive domain from international organizations for standardization and evaluated their readiness for fully autonomous driving. The steps are more thoroughly described below.

\begin{figure*}[t]
\centering
\includegraphics[width=420px]{img-ISO25119.png}
\caption{Elements of safety of the intended functionality defined by the standard ISO 21448 \citep{iso-figure-source}}
\label{fig:iso}
\centering
\end{figure*}

\subsection{Requirements selection}
The characteristics of the dynamic software ecosystems, in which autonomous systems operate, as well as the systems themselves, presented in Section \ref{sec:specs}, lead to the need for the shift from staticity towards a more dynamic approach.
By staticity, we mean the certification standards that shall no longer be static in terms of:

\bigskip
\begin{enumerate}
    \item \textit{time}, that a vehicle is granted a certificate based on its status at a single time point, typically at the moment of production,
    \item \textit{context}, that the certificate is granted based on pre-defined and finite set of tests,
    \item \textit{collaboration}, that the standard neither supports dynamic creation of coalitions nor considers communication with other entities in the environment,
    \item static \textit{tools} used within the certification process, such as documentation examination, model report assessment or visual inspection.
\end{enumerate}


As we believe future certification standards should be revised to reflect the ecosystems' dynamicity, we suggest the shift the aforementioned static characteristics into their dynamic version, towards which we evaluate the selected standards to check their readiness for autonomous systems. We refer specifically to the dynamicity in the following aspects:

\begin{enumerate}
    \item \textit{time}, that the standard is able to deal with dynamic system changes, e.g. caused by a software update,
    \item \textit{context}, that the standard suggests using tools to thoroughly verify the systems' functionality in dynamic context, such as unforseen and upredictable situations,
    \item \textit{collaboration}, that the standard supports dynamic creation of multi-agent systems to enable optimization and solving of complex problems more efficiently, e.g. through on-the-fly coalitions,
    \item \textit{tools}, to check whether the tools used to verify the systems' compliance with the standard are dynamic.
\end{enumerate}

For the analysis of the selected standards, in addition to the dynamic characteristics, we also consider

\begin{enumerate}
  \setcounter{enumi}{4}
  \item \textit{ethics}, meaning whether ethical issues are explicitly discussed or taken into account within the cases addressed in the particular standard under evaluation.
\end{enumerate}

The selected key certification standards aspects on future autonomous systems are visualized in Figure \ref{fig:reqs}. 



% \begin{table*}[ht]
% \centering
% \begin{tabular}{|l|l|cccc c }
% \hline
% \multicolumn{1}{|c }{\multirow{2}{*}{\textbf{Standard}}} & \multicolumn{1}{c }{\multirow{2}{*}{\textbf{Field}}} & \multicolumn{4}{c }{\textbf{Dynamicity in}}                                                                                                 & \multicolumn{1}{c }{\multirow{2}{*}{\textbf{Ethics}}} \\ \cline{3-6}
% \multicolumn{1}{|c }{}                                   & \multicolumn{1}{c }{}                                & \multicolumn{1}{c }{\textbf{Time}} & \multicolumn{1}{c }{\textbf{Context}} & \multicolumn{1}{c }{\textbf{Collaboration}} & \multicolumn{1}{c }{\textbf{Tools}} & \multicolumn{1}{c }{}                                 \\ \hline

% ISO 11270 &  Safety\&Assurance & \multicolumn{1}{c }{} & \multicolumn{1}{c }{}   & \multicolumn{1}{c }{}  & \checkmark &   \\ \hline
% ISO 15037 series &  Safety\&Assurance & \multicolumn{1}{c }{} & \multicolumn{1}{c }{ \checkmark }   & \multicolumn{1}{c }{}  &  &  \\ \hline
% ISO 21448 &  Safety\&Assurance & \multicolumn{1}{c }{ \checkmark } & \multicolumn{1}{c }{ \checkmark }   & \multicolumn{1}{c }{ \checkmark }  & \checkmark &  \checkmark \\ \hline
% ISO 22735 &  Safety\&Assurance & \multicolumn{1}{c }{} & \multicolumn{1}{c }{ \checkmark }   & \multicolumn{1}{c }{}  & &   \\ \hline
% ISO 22737 & Safety\&Assurance  & \multicolumn{1}{c }{} & \multicolumn{1}{c }{ \checkmark }   & \multicolumn{1}{c }{}  & &   \\ \hline
% ISO/DIS 26262 series &  Safety\&Assurance & \multicolumn{1}{c }{ \checkmark } & \multicolumn{1}{c }{ \checkmark }   & \multicolumn{1}{c }{ \checkmark }  & \checkmark &   \\ \hline
% ISO/SAE 21434 &  Cybersecurity & \multicolumn{1}{c }{ \checkmark } & \multicolumn{1}{c }{ \checkmark }   & \multicolumn{1}{c }{ \checkmark }  & \checkmark &   \\ \hline
% % ISO/PAS 5112 &  Cybersecurity & \multicolumn{1}{c }{} & \multicolumn{1}{c }{}   & \multicolumn{1}{c }{}  & &   \\ \hline
% ISO/TR 21959 series & Human factor & \multicolumn{1}{c }{} & \multicolumn{1}{c }{ \checkmark }   & \multicolumn{1}{c }{}  & & \checkmark  \\ \hline
% ISO/TR 4804 &  Safety\&Assurance, Cybersecurity & \multicolumn{1}{c }{ \checkmark } & \multicolumn{1}{c }{ \checkmark }   & \multicolumn{1}{c }{ $\circ$ }  & \checkmark &  \checkmark \\ \hline
% ISO/TS 5255 series &  Data & \multicolumn{1}{c }{} & \multicolumn{1}{c }{}   & \multicolumn{1}{c }{ $\circ$ }  & &   \\ \hline
% % ITU FGAI4AD series &  Safety\&Assurance & \multicolumn{1}{c }{} & \multicolumn{1}{c }{ \checkmark }   & \multicolumn{1}{c }{}  & \checkmark &  \checkmark  \\ \hline
% SAE J2945  &  Safety\&Assurance & \multicolumn{1}{c }{\checkmark} & \multicolumn{1}{c }{}   & \multicolumn{1}{c }{ $\circ$ }  & \checkmark & \checkmark \\ \hline
% SAE J3048  &  Safety\&Assurance & \multicolumn{1}{c }{} & \multicolumn{1}{c }{}   & \multicolumn{1}{c }{ $\circ$ }  &  &  \\ \hline
% SAE J3061 &  Cybersecurity & \multicolumn{1}{c }{ \checkmark } & \multicolumn{1}{c }{ \checkmark }   & \multicolumn{1}{c }{}  & \checkmark &   \\ \hline
% UL 4600 &  Safety\&Assurance & \multicolumn{1}{c }{ \checkmark } & \multicolumn{1}{c }{ \checkmark }   & \multicolumn{1}{c }{ $\circ$ }  & \checkmark &  \checkmark \\ \hline
% \end{tabular}
% \caption{Evaluation of certification standards. Full support is marked by \checkmark, the sign $\circ$ represents partial support.}
% \label{table:evaluation}
% \end{table*}




\subsection{Standards selection}
\label{sec:standard-selection}
The documents for evaluation were chosen in the following manner. First, we seached for standards issued or re-confirmed in the last seven years (2016-2022) by international organizations for standardization (ASAM, ISO, ITU, SAE, UNECE). We focused on standards published for light-duty vehicles (passenger cars) in the domain of on-road automated driving, regarding software requirements specification, software updates, testing, validation and verification, or guidance for automated driving in general. 

From this search, we identified 27 relevant documents. However, not all of the selected items could be analyzed. In addition to documents publicly unavailable at the time of writing of this paper (e.g. ISO/AWI TS 5083 being under development), we also omitted documents defining only the taxonomy and terminology, exchange format or language specifications (ASAM standards), and documents not directly related to automated driving systems (e.g. audit and other control activities guidelines, such as in ISO/PAS 5112). The final list of standards selected for evaluation is in Table \ref{table:description}.




\section{Standards evaluation}
% static - time (single point of certification), context (predefined tests), goals (safety, security, emissions), tools (documentation examination, specification checks, model report assessment, visual inspection)
After the selection of relevant documents, we performed an evaluation of the standards' suitability for fully autonomous driving against the requirements defined in Section \ref{sec:standard-selection}. Most of the selected standards are concerned with safety, which is understood in terms of a) functional and system specification, b) identification of triggering events, c) reduction of safety risk, and d) validation and verification of the system's functionality (see Figure \ref{fig:iso}).

The evaluation results are presented in Table \ref{table:evaluation}, the standards are sorted according to their identifier in alphabetical order. The evaluation results in each of the reviewed aspects are discussed in a more detailed manner in the following paragraphs, with providing the evaluation summary at the end of this section.




\subsection{Time dynamicity}
Only seven of the reviewed documents show readiness for the dynamicity in the time context. That is, these standards include measures for the case when there are additional updates of an already certified system, and do not require re-execution of the whole certification process conducted by an authority. Typically, these standards provide guidelines for creating the necessary documentation of changes, performing thorough testing, and arranging on-road monitoring for verifying that the update has not disrupted the intended functionality of the vehicle (UL 4600). 

The rest of the standards do not provide any information regarding software updates (e.g. ISO 15037, ISO 22735). This makes it unclear how to proceed in case changes in the software are needed, such as ensuring recognition of a new type of traffic sign, and whether the awarded certificate becomes invalid after an update in the already certified software.

\subsection{Context dynamicity}
Unlike all other categories, dynamicity within the context category seem to be relatively well addressed. In terms of system verification, most of the standards require performing simulation tests with unforseen and upredictable situations to thoroughly verify the systems' safety or security in the development phase of an autonomous vehicle system. Some standards, such as ISO 21448, provide detailed guidance on verifying the system's performance in unknown scenarios. Public road testing is also widely used in the reviewed documents as a testing method in the final stages of the development process. However, techniques and requirements for public road-testing rules differ over countries, and a global regulatory framework for public road AV testing is yet to be developed~\citep{bakar2022}. 

Standards not supporting context dynamicity, which are missing a check mark in the table, usually rely on the verification of systems' functionality based only on predefined set of tests (ISO 11270, SAE J3018), or the information about testing tools is missing due to the standard's focus area (ISO/TS 5255). 




\begin{table*}[htp]
\footnotesize
\centering
\begin{tabular*}{\textwidth}{c l l l}
\hline
\textbf{\hspace{5pt}No.\hspace{5pt}} & \textbf{Standard ID \hspace{50pt}} & \textbf{Name \hspace{180pt}} & \textbf{Field}    \\ 
\hline
\hline

1            & ISO 11270         & \makecell[l]{Intelligent transport systems - \\Lane keeping assistance systems (LKAS) — \\Performance requirements and test procedures} & Safety\&Assurance \\ \hline
2            &    ISO 15037 series    &     \makecell[l]{Road vehicles — Vehicle dynamics test methods}                                                                                                                  &   Safety\&Assurance                \\ \hline
3             &       ISO 21448    &      \makecell[l]{Road vehicles — Safety of the intended functionality}                                                                                                                 &    Safety\&Assurance               \\ \hline
 4            &      ISO 22735            &    \makecell[l]{Road vehicles — Test method to evaluate \\the performance of lane-keeping assistance systems}                                                                                                                   &          Safety\&Assurance         \\ \hline
  5           &      ISO 22737             &      \makecell[l]{Intelligent transport systems — \\Low-speed automated driving (LSAD) systems for \\predefined routes — Performance requirements, \\system requirements and performance test procedures}                                                                                                                 &       Safety\&Assurance            \\ \hline
 6            &      ISO/DIS 26262 series             &  \makecell[l]{Road vehicles — Functional safety}                                                                                                                     &     Safety\&Assurance              \\ \hline
  7           &      ISO/SAE 21434             &         \makecell[l]{Road vehicles — Cybersecurity engineering}                                                                                                              &                  Cybersecurity \\ \hline
  8           &    ISO/TR 21959 series               &        \makecell[l]{Road vehicles — Human performance and state \\in the context of automated driving}  &   Human factor     \\ \hline
  9           &  ISO/TR 4804      &     \makecell[l]{Road vehicles — Safety and cybersecurity for \\automated driving systems — \\Design, verification and validation}    &    \makecell[l]{Safety\&Assurance, \\Cybersecurity}   \\ \hline
  10           &   ISO/TS 5255 series     &     \makecell[l]{Intelligent transport systems — \\Low-speed automated driving system (LSADS) service}    &   Data    \\ \hline
  11           &   SAE J2945     &     \makecell[l]{On-Board System Requirements for \\V2V Safety Communications}    &  Safety\&Assurance     \\ \hline
  12           &   SAE J3048     &     \makecell[l]{Driver-Vehicle Interface Considerations \\for Lane Keeping Assistance Systems}    &   Safety\&Assurance    \\ \hline
  13           &   SAE J3061     &     \makecell[l]{Cybersecurity Guidebook for \\Cyber-Physical Vehicle Systems}    &   Cybersecurity    \\ \hline
  14          &   UL 4600     &     \makecell[l]{Standard for Safety for the \\Evaluation of Autonomous Products}    &    Safety\&Assurance   \\ \hline
\end{tabular*}
\caption{List of standards selected for evaluation.}
\label{table:description}

\bigskip
\bigskip
\bigskip

\centering

\begin{tabular}{c cccc c }
\hline
% \multirow{2}{*}{\hspace{5pt}\textbf{No.}\hspace{5pt}} & \multicolumn{4}{c }{\textbf{Dynamicity in}}                                                                & \multirow{2}{*}{\hspace{12pt}\textbf{Ethics}\hspace{12pt}}  \\ 
% \cline{2-5}
%                           & \multicolumn{1}{c }{\hspace{5pt}\textbf{Time}\hspace{15pt}} & \multicolumn{1}{c }{\hspace{15pt}\textbf{Context}\hspace{15pt}} & \multicolumn{1}{c }{\textbf{Collaboration}} & \multicolumn{1}{c }{\hspace{5pt}\textbf{Tools}\hspace{15pt}} &                          \\ 


\hline
\makecell[c]{ \textbf{\hspace{5pt}No.\hspace{5pt}} } & \makecell[c]{ \textbf{Time dynamicity} } &  \hspace{5pt}  \makecell[c]{ \textbf{Context dynamicity} } & \hspace{5pt} \makecell[c]{ \textbf{Collaboration dynamicity} } & \hspace{5pt} \makecell[c]{ \textbf{Tools dynamicity} } &  \hspace{5pt} \makecell[c]{ \textbf{Ethics} }
\\ 

\hline
\hline

1 &   \multicolumn{1}{c }{-} & \multicolumn{1}{c }{-}   & \multicolumn{1}{c }{-}  & \checkmark &   -  \\ \hline
2 &   \multicolumn{1}{c }{-} & \multicolumn{1}{c }{ \checkmark }   & \multicolumn{1}{c }{-}  & - & - \\ \hline
3 &   \multicolumn{1}{c }{ \checkmark } & \multicolumn{1}{c }{ \checkmark }   & \multicolumn{1}{c }{ \checkmark }  & \checkmark &  \checkmark \\ \hline
4 &   \multicolumn{1}{c }{-} & \multicolumn{1}{c }{ \checkmark }   & \multicolumn{1}{c }{-}  & - & -  \\ \hline
5 &  \multicolumn{1}{c }{-} & \multicolumn{1}{c }{ \checkmark }   & \multicolumn{1}{c }{-}  & - & -  \\ \hline
6 &   \multicolumn{1}{c }{ \checkmark } & \multicolumn{1}{c }{ \checkmark }   & \multicolumn{1}{c }{ \checkmark }  & \checkmark &  - \\ \hline
7 &   \multicolumn{1}{c }{ \checkmark } & \multicolumn{1}{c }{ \checkmark }   & \multicolumn{1}{c }{ \checkmark }  & \checkmark &  - \\ \hline
8 &  \multicolumn{1}{c }{-} & \multicolumn{1}{c }{ \checkmark }   & \multicolumn{1}{c }{-}  & - & \checkmark  \\ \hline
9 &   \multicolumn{1}{c }{ \checkmark } & \multicolumn{1}{c }{ \checkmark }   & \makecell[c]{ unclear collaboration \\form and purpose }  & \checkmark &  \checkmark \\ \hline
10 &   \multicolumn{1}{c }{-} & \multicolumn{1}{c }{-}   & \makecell[c]{ unclear collaboration \\form and purpose }  & - & -  \\ \hline
% ITU FGAI4AD series &  Safety\&Assurance & \multicolumn{1}{c }{} & \multicolumn{1}{c }{ covered }   & \multicolumn{1}{c }{}  & covered &  covered  \\ \hline
11  &   \multicolumn{1}{c }{\checkmark} & \multicolumn{1}{c }{-}   & \makecell[c]{ unclear collaboration \\form and purpose }  & \checkmark & \checkmark \\ \hline
12  &   \multicolumn{1}{c }{-} & \multicolumn{1}{c }{-}   & \makecell[c]{ unclear collaboration \\form and purpose }  & - & - \\ \hline
13 &   \multicolumn{1}{c }{ \checkmark } & \multicolumn{1}{c }{ \checkmark }   & \multicolumn{1}{c }{-}  & \checkmark & -  \\ \hline
14 &   \multicolumn{1}{c }{ \checkmark } & \multicolumn{1}{c }{ \checkmark }   & \makecell[c]{ unclear collaboration \\form and purpose }  & \checkmark &  \checkmark \\ \hline
\end{tabular}
\caption{Evaluation of certification standards. The numbers of standards match with the list of standards defined in Table \ref{table:description}.}
\label{table:evaluation}
\end{table*}




\subsection{Collaboration dynamicity}
As for the collaboration dynamicity, standards ISO/DIS 26262 and ISO 21448 provide a comprehensive specification and design of communication between vehicle and other entities within the surrounding ecosysystem. Besides that, ISO/SAE 21434 incorporates distributed cybersecurity activities, assigning cybersecurity responsibilities between multiple parties. Because these three standards explicitly address the possibility of collaboration of multiple agents within the ecosystem, they were marked as fully supportive towards dynamic collaboration. 

However, even though the dynamic creation of coalitions is covered by these documents, we still see an opportunity for improvement. We notice that none of the reviewed standards considers any form of trust management when collaborating with other entities. Since the entities might have malicious intentions, naively trusting any entity willing to collaborate poses a serious security risk to the operation of the entire ecosystem. 

In the UL 4600, ISO/TR 4804, ISO/TS 5255, SAE J2945 and SAE J3048 standards, terms like \textit{"dependencies between items"}, \textit{"vehicle to vehicle communication"}, or \textit{"arrays of systems implementing other vehicle level functions"} have been found. Thus, the possibility of vehicle communication with other entities within their surrounding environment is at least partially addressed. However, it is unclear (1)~how and in which directions the communication is carried, and (2)~whether joint strategy creation is possible. Because of that, these standards were marked as providing only a partial support. All other standards do not seem to directly support collaboration dynamicity at all, or just to a negligible extent.




\subsection{Tools dynamicity}
More than a half of the evaluated documents enable using of dynamic tools for checking the system's compliance with a given standard. Such a check of compliance typically includes on-site testing or execution of further system's verification and validation measures. Even in the cases where dynamic tools are present, a large part of the compliance checks is still performed by static tools, such as manual inspection of documentation, processes and procedures. Standards which are not identified as supportive towards the use of dynamic tools in the assessment table employ only static tools for compliance checks.




\subsection{Ethics}
The series of ISO/TR 21959 standards is devoted to ethics in a large extent. Except for the ethical considerations, the standards also present the best practices in the field of societal trust formation. In particular, they establish guidelines for a better acceptance of autonomous vehicles based on the analysis of various human factors having an impact on perception and trust formation towards autonomous systems within the automotive domain.

But ethical considerations do not appear to be generally taken into account by the assessed standards, though. Four standards in the Safety\& Assurance area refer to the need of absence of unreasonable risk, which is defined as a risk that is \textit{"unacceptable in a certain context according to valid societal moral concepts"} \citep{iso21448}. Other moral aspects do not seem to be addressed, and we find the absence of consideration of moral aspects particularly problematic. We are convinced that autonomous systems such as autonomous vehicles, which have direct responsibility for human lives, bear moral obligation.

The ethics of AI systems in general is a topic that is currently frequently debated, and the authorities are working on complex market regulation strategies. The activities of the European Commission may serve as an illustration. Starting with the publication \textit{"Policy and Investment Recommendations for Trustworthy Artificial Intelligence"} \citep{trustworthy-ai-2019}, the European Commission laid the foundations of AI regulation in the European Union, covering the fundamental questions on the border of law and ethics. Since then, multiple documents discussing ethics have been published, but many concerns and questions are still unanswered. 

Since ethics is a complex issue, it is possible that the authors of standards in the field of autonomous driving are awaiting the central regulatory bodies' further recommendations and will adjust the documents once the direction in which the regulation will evolve is clear. In any case, autonomous systems cannot be developed without ensuring human and societal needs are taken into consideration during the systems' operation. We are convinced that even on a technical level, it is necessary to establish mechanisms which can control the observance of ethical principles, and which can be followed up after further guidelines are published by the central authorities.


\subsection{Evaluation summary}
To sum up, the evaluation results are diverse. While some of the standards show clear insufficient preparedness to reflect the dynamicity of future autonomous systems and ecosystems in which they operate (e.g. ISO 22735, ISO 22737, or SAE J3018), other standards show signs of readiness for autonomous driving after meeting all (ISO 21448) or a vast majority (ISO/TR 4804, UL 4600) of the specified requirements. Yet, there is room for improvement even in this case.

Namely, in all the analyzed standards, we identified the absence of considering the creation of dynamic coalitions as well as the absence of trust management in standards with support towards coalition dynamicity as one of the most urgent deficiencies from the evaluated aspects. Besides that, we undoubtedly see shortcomings in the field of ethics, too. We observe that most of the standards are narrowly focused, usually towards safety, while neglecting ethical aspects, which are often directly related. 


% Future adjustments of the certification processes are unavoidable. Vehicles are expected to become even more complex, and overtake all critical driving functions. 

% Current certification methods seem not to be designed to promptly deal with the changes, though. Inspired by the analysis performed by \citep{fisher2018,dynamic2022}, we have identified the staticity of the currently used certification approaches as the most critical issue for the future. By staticity, we mean the current approaches are static in terms of (1) \textit{time}, as a vehicle is granted a certificate based on its status at a single time point, typically at the moment of production; (2) \textit{context}, as the certificate is granted based on pre-defined and finite set of tests (3) \textit{goals}, as the certificates are issued towards a single and unchangeable objective (e.g. safety, security, emissions); and (4) \textit{tools} used within the certification process, such as documentation examination, model report assessment or visual inspection. 


\section{\uppercase{Suggested Improvements}}
\label{sec:solution}
Given the presented evaluation and a relatively long process of preparation and approval of standards, which typically takes three years from the initial proposal to its final publication according to ISO\footnote{\url{https://www.iso.org/developing-standards.html}}, it is not surprising that standards do not fully reflect the advancement in dynamic autonomous systems. Several shortcomings in the field of ethics were identified, too. Yet, there are mechanisms that could help the standards better support the dynamicity of the software ecosystems and the future autonomous cyber-physical systems, while reflecting the moral setting of the society as well. 

In this section, we outline five suggested directions that could lead to a better fit of the standards with the requirements of future dynamic autonomous systems and ecosystems, including an advanced assurance of ethical awareness. 

\subsection{Real-time validation of a certificate and its properties}
Traditional certificates, whose purpose is to provide certain guarantees about the quality of the certified system, are granted at a specific point in time, usually at the time of production of the certified system. However, in a dynamic and ever-changing environment, such a certificate may not keep its validity over time; instead, its validity may change depending on the context or deteriorate over time.

The certification schemes for dynamic autonomous systems shall account for some form of identification or quantification of the certificate deterioration, in order to address the issue of time dynamicity. In other words, in a dynamic environment, it is necessary to constantly re-check the certificate's validity and decide to what extent the certificate and its guarantees can be trusted.

For better illustration of the need for the implementation of real-time certificate validation, consider a scenario where an AV has a valid certificate (\textit{valid} in the traditional meaning of the certification concept) but its behavior is suspicious, raising the possibility that the vehicle has malicious intentions. Other vehicles will be able to react to such a situation more effectively if there are mechanisms for evaluating suspicious manners that deviate from the expected, typical behavior. In this case, the real-time validity check would show that the vehicle was not certified to handle this specific situation, or that the certificate is outdated because vehicle's software has not been updated on a version fixing critical bugs that allow attackers to hack the vehicle. 

\subsection{Certificate combined with vehicle's reputation}
The examination of certification standards revealed certain flaws, which might lead to a false sense of security. As discussed above, there may be situations in which the vehicle is certified, yet its safe operation is impaired (e.g. after a faulty software update that is not detected by static tools used to check the system's compliance with a standard). A~possible solution to this issue could be linking the certificate to the vehicle's reputation.

In dynamic ecosystems, deciding who to trust becomes a challenging task. As studied in the disciplines of trust, reputation is used as one of the tools for trust-building. Reputation can be defined as \textit{the overall quality of an entity derived from the judgements by other entities in the underlying network, which is globally visible to all members of the network} \citep{trust-and-reputation}, and when such information is propagated through the network of connected entities, it can have a substantial effect on decision-making. Besides that, by providing information allowing distinction between trustworthy and untrustworthy nodes, reputation can also help in dealing with observable misbehavior \citep{rep-def} and minimizing damage in case of an insider attack \citep{reputation1}. 

This concept may be particularly useful for addressing the time dynamicity problem within certification. Same as people gain reputation by having their actions evaluated by their peers, an entity's reputation in a smart ecosystem is dependent on how it behaves and interacts with others \citep{buhnova2022tutorial}. Moreover, a collection of experience during runtime feeding updates of the score of trustworthiness could also be used to promote or demote the certificate's validity. 

Tying the certificate to the vehicle’s reputation could help other entities respond to changes in a dynamic environment in an even more flexible manner. Demonstrated on a hypothetical scenario, a vehicle would be less trusted if its reputation reported by other vehicles declined despite having a valid certificate, caused by other vehicles reporting its sudden suspicious behavior (indicating a software bug or an attack). In case of a serious reputation drop, the vehicle's certificate could be temporarily or completely revoked to prevent further damage.


\subsection{Extension of the certificate's status to a scale}
Another alternative, which is partly related to the linkage of the certificate to the concept of reputation discussed above, is to redefine the concept of certification in terms of the range of potential values it can acquire. The traditional view on a certificate nowadays treats it as a binary value (\textit{valid/invalid}, or \textit{granted/not granted} certificate). Such a perception might be unnecessarily restricting. 

Instead, we suggest rethinking the concept and considering it rather a scale to better represent the current certification status of the system installed in the vehicle or other autonomous system. Indeed, there are multiple ways to interpret this newly proposed concept. To mention some of the possible meanings, the scale might represent e.g. the number of software updates installed in the system, or the amount of time elapsed since the last official verification of the system's compliance with a particular standard. The exact interpretation of the certification scale is up for further research and discussion.

\subsection{Considering certificate's context-dependant validity}
The implementation of trust management appears to be necessary even in addressing the context and coalition dynamism issue. During runtime, evaluating the acceptable level of trust, or eventually the vulnerability risks, is strongly context-dependent.

Consider two AVs initializing mutual communication in two scenarios. In scenario A, the AVs' intention is to exchange a batch of weather-related data and then stop any further interaction. In scenario B, the entities eshablish a connection with the purpose of creating a vehicle platoon in order to reduce fuel consumption due to lower air resistance. However, close collaboration needed for vehicle platoon formation in Scenario B raises more serious trust concerns about the safety of riding in such close proximity compared to the situation of exchanging data in Scenario A. And even if a vehicle may be certified in correctly handling interaction with other vehicles in some contexts, its behavior may not be verified or guaranteed in the other contexts. Therefore, before engaging in any interaction, it is the system's responsibility to verify whether the awarded certificate, as well as the other party, can be trusted in the given context.

\subsection{Certificates combined with ethical concerns}
Evaluating the safety of products or their environmental footprint before they are allowed to enter the market is nowadays a common practice. But the review of current standards in this paper has shown that ethical aspects are still not frequently considered. Technology development has to be ethically supervised, though. Otherwise, intelligent systems developed with the intention to help can easily turn to harm or disadvantage certain groups of people.

Our idea to address the certification gap regarding the ethics of autonomous systems is to assess the ethics of an autonomous system in the same way as safety or environmental aspects. In particular, we suggest combining certification with Ethical Digital Identities (EDI), a concept introduced by \citet{edi2022}. Derived from the concept of Digital Identities \citep{digital-identities-2005}, EDI serve as the basis for safeguarding the evolution of intelligent safety-critical systems in terms of ethics. 



\section{\uppercase{Conclusion}}
\label{sec:conclusion}
In this paper, we evaluated the readiness of current certification standards for future autonomous driving systems. We analyzed the characteristics of both autonomous systems and the dynamic software ecosystems in which they operate, from which we derived a set of requirements, namely \textit{Time Dynamicity, Context Dynamicity, Collaboration Dynamicity, Tools Dynamicity}, and \textit{Ethics}, which we then used to assess the standards.

The results demonstrate that the present standards are not entirely ready for the expansion of autonomous driving systems, and also assisted us in identifying their primary shortcomings. One of the most serious deficiencies is referring to the \textit{Collaboration Dynamicity} aspect. We criticize mainly the lack of support for the creation of dynamic coalitions among standards, as well as the complete absence of any kind of trust management strategies for establishing communication with other entities. Another shortcoming concerns neglecting ethical aspects in the standards' focus.

In order to address the identified gaps, we outlined a concept for the improvement of certification standards. We present five ideas for rethinking the certification that we believe will help move the discussion towards a complete solution of the identified problems, so that standardization for autonomous systems (not only those in the automotive domain) will better fit the requirements of future dynamic systems and ecosystems with ethical awareness, which can be trusted. The presented ideas are subject for further research and will be elaborated on in our future work.




% \section{\uppercase{Introduction}}
% \label{sec:introduction}

% Your paper will be part of the conference proceedings
% therefore we ask that authors follow the guidelines explained in
% this example in order to achieve the highest quality possible
% \citep{Smith98}.

% Be advised that papers in a technically unsuitable form will be
% returned for retyping. After returned the manuscript must be
% appropriately modified.

% \section{\uppercase{Manuscript Preparation}}

% We strongly encourage authors to use this document for the
% preparation of the camera-ready. Please follow the instructions
% closely in order to make the volume look as uniform as possible
% \citep{Moore99}.

% Please remember that all the papers must be in English and without
% orthographic errors.

% Do not add any text to the headers (do not set running heads) and
% footers, not even page numbers, because text will be added
% electronically.

% For a best viewing experience the used font must be Times New
% Roman, except on special occasions, such as program code
% \ref{subsubsec:program_code}.

% \vfill
% \subsection{Manuscript Setup}

% The template is composed by a set of 7 files, in the
% following 2 groups:\\
% \noindent {\bf Group 1.} To format your paper you will need to copy
% into your working directory, but NOT edit, the following 4 files:
% \begin{verbatim}
%   - apalike.bst
%   - apalike.sty
%   - article.cls
%   - scitepress.sty
% \end{verbatim}

% \noindent {\bf Group 2.} Additionally, you may wish to copy and edit
% the following 3 example files:
% \begin{verbatim}
%   - example.bib
%   - example.tex
%   - scitepress.eps
% \end{verbatim}


% \subsection{Page Setup}

% The paper size must be set to A4 (210x297 mm). The document
% margins must be the following:

% \begin{itemize}
%     \item Top: 3,3 cm;
%     \item Bottom: 4,2 cm;
%     \item Left: 2,6 cm;
%     \item Right: 2,6 cm.
% \end{itemize}

% It is advisable to keep all the given values because any text or
% material outside the aforementioned margins will not be printed.

% \vfill
% \subsection{First Section}

% This section must be in one column.

% \subsubsection{Title and Subtitle}

% Use the command \textit{$\backslash$title} and follow the given structure in "example.tex". The title and subtitle must be with initial letters
% capitalized (titlecased). The separation between the title and subtitle is done by adding a colon ":" just before the subtitle beginning. In the title or subtitle, words like "is", "or", "then", etc. should not be capitalized unless they are the first word of the title or subtitle. No formulas or special characters of any form or language are allowed in the title or subtitle.

% \subsubsection{Authors and Affiliations}

% Use the command \textit{$\backslash$author} and follow the given structure in "example.tex". Please note that the name of each author must start with its first name.

% \subsubsection{Keywords}

% Use the command \textit{$\backslash$keywords} and follow the given structure in "example.tex". Each paper must have at least one keyword. If more than one is specified, please use a comma as a separator. The sentence must end with a period.

% \subsubsection{Abstract}

% Use the command \textit{$\backslash$abstract} and follow the given structure in "example.tex".
% Each paper must have an abstract up to 200 words. The sentence
% must end with a period.

% \subsection{Second Section}

% Files "example.tex" and "example.bib" show how to create a paper
% with a corresponding list of references.

% This section must be in two columns.

% Each column must be 7,5-centimeter wide with a column spacing
% of 0,8-centimeter.

% The section text must be set to 10-point.

% Section, subsection and sub-subsection first paragraph should not
% have the first line indent.

% To remove the paragraph indentation (only necessary for the
% sections), use the command \textit{$\backslash$noindent} before the
% paragraph first word.

% If you use other style files (.sty) you MUST include them in the
% final manuscript zip file.

% \vfill
% \subsubsection{Section Titles}

% The heading of a section title should be in all-capitals.

% Example: \textit{$\backslash$section\{FIRST TITLE\}}

% \subsubsection{Subsection Titles}

% The heading of a subsection title must be with initial letters
% capitalized (titlecased).

% Words like "is", "or", "then", etc. should not be capitalized unless
% they are the first word of the subsection title.

% Example: \textit{$\backslash$subsection\{First Subtitle\}}

% \subsubsection{Sub-Subsection Titles}

% The heading of a sub subsection title should be with initial letters
% capitalized (titlecased).

% Words like "is", "or", "then", etc should not be capitalized unless
% they are the first word of the sub subsection title.

% Example: \textit{$\backslash$subsubsection\{First Subsubtitle\}}

% \subsubsection{Tables}

% Tables must appear inside the designated margins or they may span
% the two columns.

% Tables in two columns must be positioned at the top or bottom of the
% page within the given margins. To span a table in two columns please add an asterisk (*) to the table \textit{begin} and \textit{end} command.

% Example: \textit{$\backslash$begin\{table*\}}

% \hspace*{1.5cm}\textit{$\backslash$end\{table*\}}\\

% Tables should be centered and should always have a caption
% positioned above it. The font size to use is 9-point. No bold or
% italic font style should be used.

% The final sentence of a caption should end with a period.

% \begin{table}[h]
% \caption{This caption has one line so it is
% centered.}\label{tab:example1} \centering
% \begin{tabular}{|c c }
%   \hline
%   Example column 1 & Example column 2 \\
%   \hline
%   Example text 1 & Example text 2 \\
%   \hline
% \end{tabular}
% \end{table}

% \begin{table}[h]
% \vspace{-0.2cm}
% \caption{This caption has more than one line so it has to be
% justified.}\label{tab:example2} \centering
% \begin{tabular}{|c c }
%   \hline
%   Example column 1 & Example column 2 \\
%   \hline
%   Example text 1 & Example text 2 \\
%   \hline
% \end{tabular}
% \end{table}

% Please note that the word "Table" is spelled out.

% \vfill
% \subsubsection{Figures}

% Please produce your figures electronically, and integrate them into
% your document and zip file.

% Check that in line drawings, lines are not interrupted and have a
% constant width. Grids and details within the figures must be clearly
% readable and may not be written one on top of the other.

% Figure resolution should be at least 300 dpi.

% Figures must appear inside the designated margins or they may span
% the two columns.

% Figures in two columns must be positioned at the top or bottom of
% the page within the given margins. To span a figure in two columns please add an asterisk (*) to the figure \textit{begin} and \textit{end} command.

% Example: \textit{$\backslash$begin\{figure*\}}

% \hspace*{1.5cm}\textit{$\backslash$end\{figure*\}}

% Figures should be centered and should always have a caption
% positioned under it. The font size to use is 9-point. No bold or
% italic font style should be used.

% \begin{figure}[!h]
%   \centering
%    {\epsfig{file = SCITEPRESS.eps, width = 5.5cm}}
%   \caption{This caption has one line so it is centered.}
%   \label{fig:example1}
%  \end{figure}

% \begin{figure}[!h]
%   \vspace{-0.2cm}
%   \centering
%    {\epsfig{file = SCITEPRESS.eps, width = 5.5cm}}
%   \caption{This caption has more than one line so it has to be justified.}
%   \label{fig:example2}
%   \vspace{-0.1cm}
% \end{figure}

% The final sentence of a caption should end with a period.



% Please note that the word "Figure" is spelled out.

% \subsubsection{Equations}

% Equations should be placed on a separate line, numbered and
% centered.\\The numbers accorded to equations should appear in
% consecutive order inside each section or within the contribution,
% with the number enclosed in brackets and justified to the right,
% starting with the number 1.

% Example:

% \begin{equation}\label{eq1}
%     a=b+c
% \end{equation}

% \subsubsection{Algorithms and Listings}

% Algorithms and Listings captions should have font size 9-point, no bold or
% italic font style should be used and the final sentence of a caption should end with a period.
% Captions with one line should be centered and if it has more than one line it should be set to justified.

% \vfill
% \subsubsection{Program Code}\label{subsubsec:program_code}

% Program listing or program commands in text should be set in
% typewriter form such as Courier New.

% Example of a Computer Program in Pascal:

% \begin{small}
% \begin{verbatim}
%  Begin
%      Writeln('Hello World!!');
%  End.
% \end{verbatim}
% \end{small}


% The text must be aligned to the left and in 9-point type.

% \subsubsection{Reference Text and Citations}

% References and citations should follow the APA (Author, date)
% System Convention (see the References section in the compiled
% manuscript). As example you may consider the citation
% \citep{Smith98}. Besides that, all references should be cited in the
% text. No numbers with or without brackets should be used to list the
% references.

% References should be set to 9-point. Citations should be 10-point
% font size.

% You may check the structure of "example.bib" before constructing the
% references.

% For more instructions about the references and citations usage
% please see the appropriate link at the conference website.

% \section{\uppercase{Copyright Form}}

% For the mutual benefit and protection of Authors and
% Publishers, it is necessary that Authors provide formal written
% Consent to Publish and Transfer of Copyright before publication of
% the Book. The signed Consent ensures that the publisher has the
% Author's authorization to publish the Contribution.

% The copyright form is located on the authors' reserved area.

% The form should be completed and signed by one author on
% behalf of all the other authors.

% \section{\uppercase{Conclusions}}
% \label{sec:conclusion}

% Please note that ONLY the files required to compile your paper should be submitted. Previous versions or examples MUST be removed from the compilation directory before submission.

% We hope you find the information in this template useful in the preparation of your submission.


\section*{\uppercase{Acknowledgements}}

This research was supported by ERDF "CyberSecurity, CyberCrime and Critical Information Infrastructures Center of Excellence" (No. CZ.02.1.01/0.0/0.0/16\_019/0000822).




%   The authors would also like to thank the anonymous referees for
%   their valuable comments and helpful suggestions. The work is
%   supported by the \grantsponsor{GS501100001809}{National Natural
%     Science Foundation of
%     China}{http://dx.doi.org/10.13039/501100001809} under Grant
%   No.:~\grantnum{GS501100001809}{61273304}
%   and~\grantnum[http://www.nnsf.cn/youngscientsts]{GS501100001809}{Young
%     Scientsts' Support Program}.


% If any, should be placed before the references section
% without numbering. To do so please use the following command:
% \textit{$\backslash$section*\{ACKNOWLEDGEMENTS\}}

\bibliographystyle{apalike}
{\small
\bibliography{example}}


% \section*{\uppercase{Appendix}}

% If any, the appendix should appear directly after the
% references without numbering, and not on a new page. To do so please use the following command:
% \textit{$\backslash$section*\{APPENDIX\}}

\end{document}


\section{Conclusion}\label{sec:conclusion}
In this work, we focus on addressing the fundamental challenge of OOD detection tasks, which is how to fully understand the semantic discrepancy between the ID/OOD samples. We reveal that the key to success in the realistic SCOOD task is to allocate as many ID samples in the unlabeled set correctly as possible. To this end, we propose a novel uncertainty-aware optimal transport scheme that introduces class-specific energy scores as guidance for effective label assignment. Experimental results show that our method achieves better performance than previous state-of-the-art methods on SCOOD benchmarks.

\textbf{Limitations.} In addition to temperature scaling, other techniques such as feature clipping applied in ReAct~\cite{sun2021react} also enhance the performance of energy score, so how to obtain an OOD score that best fits the SCOOD task can be further explored. Moreover, a setting highly related to SCOOD has been proposed in \cite{katz2022training} and formulated as a constrained optimization problem. We will also theoretically analyze these practical OOD settings in our feature work.

% \section*{Acknowledgments}
\textbf{Acknowledgments.} 
This work is supported by National Key R\&D Program of China under Grant 2020AAA0105701, National Natural Science Foundation of China (NSFC) under Grants 61872327, Major Special Science and Technology Project of Anhui, National Natural Science Foundation of China (62033012) and Ant Group through Ant Research Intern Program.

%\begin{table}[hb]
	\begin{center}
		\caption{Average of the NRMSE on the parameters $\theta_1$ and $\theta_2$ for different levels of noise after 100 Monte Carlo Simulations. WITH true IVS}
		\label{tab:RMSEExampleIV}
		\begin{tabular}{llll}
			& \multicolumn{3}{c}{NRMSE}                                                                                                                                                              \\ \cline{2-4} 
			& $\sigma_\nu=0$ 	& $\sigma_\nu=10^-8$ 			& $\sigma_\nu=10^-6$ \\ 
			\toprule
			$\theta_1$ 	&   $4.2\cdot 10 ^{-6}$    	&   $6.0\cdot 10 ^{-4}$     	& $3.6\cdot 10 ^{-3}$      \\
			$\theta_2$ 	&   $2.0\cdot10^{-8}$     &   $5.7\cdot10^{-8}$  &   $5.5\cdot 10 ^{-5}$       \\ 
			$\theta_3$ 	&   $1.4\cdot 10 ^{-6}$     &   $3.3\cdot 10 ^{-4}$ &   $2.5\cdot 10 ^{-3}$       \\ 
			\bottomrule
		\end{tabular}
	\end{center}
\end{table}

%%%%%%%%%%%%%%%%%%%%%%%%%%%%%%%%%%%%%%%%%%%%%%%%%%%%%%%%%%%%%%%%%%%%%%%%%%%%%%%%%%%%%%%%%%%%%%%%%%%%%%%%%%%%%%%%%%%%%%%%%%%%%%%%
\subsection{Addition of Instrumental Variables}
\alert{Since the polynomial feedforward controller is undermodeling the true inverse dynamics of the system, the estimator that minizes \eqref{eq:minimization} leads to a inconsistent estimator, see e.g. \cite{Boeren2015,Kon2022}. Hence, in this section, Instrumental Variables (IVs) are added to create an inconsistent estimator,
	\begin{equation}
		\label{eq:minimizationIV}
		\hat{\Theta}_{IV} = \arg \min_\Theta \left\| Z^\top\left( \overline{w}-\Phi\Theta\right) \right\|^2 + \gamma\|\Theta\|^2_\mathcal{H}.
	\end{equation}
	Similar to \eqref{eq:solTheta}, the solution is derived by taking the derivative with respect to $\Theta$ and setting to zero, i.e.,
	\begin{equation}
		\hat{\Theta}_{IV} =K \Phi^\top Z Z^\top \left(\Phi K\Phi^\top ZZ^\top+\gamma I_{N}\right)^{-1} \overline{w}.
	\end{equation}
	But a better solution is
	\begin{equation}
		\hat{\Theta} =K Z^\top\left(\Phi K Z^\top +\gamma I_{N}\right)^{-1}\overline{w},
	\end{equation}
	that minimizes the cost function
	\begin{equation}
		\arg \min_\Theta \left\| Z^\top\left( \overline{w}-\Phi\Theta\right) \right\|^2 + \gamma \|\Theta\|^2_\mathcal{H_{\Phi^\top Z K}}.
	\end{equation}
	no, this is not true, the resulting kernel is not symmetric so does not hold! maybe approximate....
	The IVs can be chosen freely by the user, where in this paper, the IVs are selected as the selected basis functions, i.e.,
	\begin{equation}
		\begin{aligned}Z_i &= \begin{bmatrix}
				\left(\psi_{i} r\right)\left(0 T_{s}\right) & 0 & \cdots & 0 \\
				0 & \left(\psi_{i} r\right)\left(1 T_{s}\right) & \cdots & \vdots \\
				\vdots & \ddots & \ddots & \vdots \\
				0 & \cdots & \cdots & \left(\psi_{i} r\right)\left((N-1) T_{s}\right)
			\end{bmatrix} \\
			Z &= \begin{bmatrix}
				Z_1 & Z_2 & \cdots & Z_{n_\theta}
			\end{bmatrix}
		\end{aligned}
\end{equation}}


\bibliographystyle{ifacconf}
\bibliography{library}


%%%%%%%%%%%%%%%%%%%%%%%%%%%%%%%%%%%%%%%%%%%%%%%%%%%%%%%%%%%%%%%%%%%%%%%%%%%%%%%%%%%%%%%%%%%%%%%%%%%%%%%%%%%%%%%%%%%%%%%%%%%%%%%%

\end{document}
