\section{Linearly Parameterized Feedforward for LPV motion systems}
\label{sec:BFFF}
In this section, a polynomial feedforward strategy for LPV systems as in \eqref{eq:Ftilde} is developed, by posing an alternative parameterization for the class of LPV motion systems in \eqref{eq:LPVdesc}, that includes dynamic dependency on the scheduling sequence, but simplifies the identification problem significantly.

The key idea is to rewrite the system dynamics in \eqref{eq:LPVdesc} as
\begin{equation}
	\label{eq:IOModel}
	\sum_{i=-n_i}^{n_a}a_i(\rho) y^{(i)}=\sum_{j=0}^{n_{\mathrm{b}}}b_j(\rho) \frac{d^{j}}{d t^{j}} w,
\end{equation}
where a change of variables is used as $w={ \iint} u \, dt^2$.
%The key idea is to define the feedforward controller using the second integral of the input signal $u$, that is equal to $w$, and defining the parameters in this domain. Hence, the dynamics of the LPV system in \eqref{eq:LPVdesc} are written as
%\begin{equation}
%	\label{eq:IOModel}
%	\sum_{i=-n_i}^{n_a}a_i(\rho) y^{(i)}=\sum_{j=0}^{n_{\mathrm{b}}}b_j(\rho) \frac{d^{j}}{d t^{j}} w.
%\end{equation}
Similarly to LTI polynomial feedforward in \eqref{eq:FFPolLTI}, the inverse model for LPV systems is parameterized by setting the values of $b_j \, \forall j\geq 1$ in \eqref{eq:IOModel} to zero, resulting in
\begin{subnumcases}{F_{LPV}:}
	w_{ff}=\sum_{i=1}^{n_\theta}\theta_i(\rho) \psi_i\left(\frac{d}{dt},I \right)r, \label{eq:FFModel}
	\\
	u_{ff} = \frac{d^2}{dt^2}w_{ff}, \label{eq:FFCalc}
\end{subnumcases}
where $\psi$ contains differentiators $\frac{d}{dt}$ or integrals $I$, e.g. $\psi_i\left( \frac{d}{dt},I\right) r = \frac{d^2}{dt^2}r=\ddot{r}$ or $\psi_i\left(\frac{d}{dt},I\right) r=Ir = { \int} r \, dt$. 

Note that from \eqref{eq:FFModel}, the second integral of the input $w_{ff}$ is composed out of basis functions, in contrast to the input $u_{ff}$ for LTI polynomial feedforward. The dynamic dependence on the scheduling sequence seen in \eqref{eq:exampleIO2} is introduced by the second derivative with respect to time in \eqref{eq:FFCalc}, which introduces time derivatives of $\theta(\rho)$. In \exampleRef{example:LPVmsdFF}, an example is shown for the two-mass system.
\begin{exmp}[LPV feedforward]
	\label{example:LPVmsdFF}
	Consider the two mass-spring-damper system from \exampleRef{example:LPVinverse}, with input-output behavior in \eqref{eq:exampleIO}. The polynomial feedforward strategy is then defined, by neglecting the zero in the right-hand side of \eqref{eq:exampleIO} according to \assRef{ass:assLPVSystem}, i.e., $\frac{c}{k(\rho)}\approx 0$, as
	\begin{equation}
		\label{eq:exampleFFPara}
		\resizebox{0.85\hsize}{!}{
			$
		\begin{aligned}
			w_{ff} =  &\underbrace{\vphantom{\frac{m_2}{k(\rho)}}c_2}_{\theta_1(\rho)}\underbrace{\vphantom{\frac{m_2}{k(\rho)}}\int}_{\psi_1} r \, dt+\underbrace{\vphantom{\frac{m_2m_1}{k(\rho)}}\left(m_1+m_2 + \frac{cc_2}{k(\rho)} \right)}_{\theta_2(\rho)}\underbrace{\vphantom{\frac{m_2m_1}{k(\rho)}}1}_{\psi_2}r\\
			&+\underbrace{\frac{c\left( m_1+m_2\right)+c_2m_1}{k(\rho)}}_{\theta_3(\rho)}\underbrace{\vphantom{\frac{m_2m_1}{k(\rho)}}\frac{d}{dt}}_{\psi_3}r+\underbrace{\frac{m_2m_1}{k(\rho)}}_{\theta_4(\rho)}\underbrace{\vphantom{\frac{m_2m_1}{k(\rho)}}\frac{d^2}{dt^2}}_{\psi_4}r,
		\end{aligned}$}
	\end{equation}
	where the applied feedforward force is calculated using \eqref{eq:FFCalc}. The applied feedforward force contains both static and dynamic scheduling dependence when substituting \eqref{eq:exampleFFPara} into \eqref{eq:FFCalc}, i.e.,
	\begin{equation}
		\label{eq:exampleResultingFF}
		\resizebox{\hsize}{!}{
			$
		\begin{aligned}
			u_{ff} &= \frac{m_2m_1}{k(\rho)}\ddddot{r} + \frac{c\left( m_1+m_2\right) +c_2m_1}{k(\rho)} \dddot{r}+ \left(m_1+m_2 \right)\ddot{r} +c_2 \dot{r}\\
			& + \left( \frac{\frac{2\dot{\rho}^2{k^{\prime}}^2(\rho)}{k(\rho)}-\dot{\rho}^2k^{\prime\prime}(\rho)-\ddot{\rho}k^\prime(\rho)}{k^2(\rho)}-\frac{2 \dot{\rho}k^\prime(\rho)  }{k^2(\rho)}\right)f(r,\dot{r},\ddot{r}),\\
		\end{aligned}
	$}
	\end{equation}
which is equal to \eqref{eq:exampleIO2} when substituting $y$ for $r$.
\end{exmp}
The applied feedforward force $u_{ff}$ in \eqref{eq:FFCalc} includes the dynamic dependency on the scheduling signal, e.g. shown in \eqref{eq:exampleIO2}, while the modeled $w_{ff}$ in \eqref{eq:FFModel} is only statically dependent on the scheduling sequence.

%The resulting signal $w_{ff}$ in \eqref{eq:exampleFFPara} has less parameters than the signal $u$ in \eqref{eq:exampleIO2}, hence the parameters are easier to identify, and contains the dynamic dependence due to the calculation of $u_{ff}$ by taking the second derivative.





