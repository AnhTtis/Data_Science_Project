\section{Problem Formulation}
\label{sec:problem}
In this section, the problem related to feedforward control for LPV motion systems is formulated. First, the control setting and feedforward goal for LPV motion systems is described. Second, polynomial feedforward for LTI systems is investigated. Third, challenges in designing feedforward controllers for LPV motion systems are shown, that motivates the problem definition in \secRef{sec:pdef}.
%%%%%%%%%%%%%%%%%%%%%%%%%%%%%%%%%%%%%%%%%%%%%%%%%%%%%%%%%%%%%%%%%%%%%%%%%%%%%%%%%%%%%%%%%%%%%%%%%%%%%%%%%%%%%%%%%%%%%%
\subsection{Control Setting}
The control goal is to develop LPV feedforward controller $F_{LPV}$ to reduce the tracking error $e=r-y$ for single-input single-output LPV system $G_{LPV}$, where perfect tracking is achieved by $F_{LPV}=G_{LPV}^{-1}$. The control structure can be seen in \figRef{fig:controlStructure}, where $C$ is a stabilizing feedback controller.
\begin{figure}
	\centering
	\includegraphics{pdf/controlStructure.pdf}
	\caption{Feedforward structure considered}
	\label{fig:controlStructure}
\end{figure}
The reference trajectory $r$ is a smooth reference, which can be differentiated at least four times, as in \citet{Lambrechts2005}. The considered class of LPV systems $G_{LPV}$ can be represented in Continuous-Time (CT) by input-output representations and are shown in \defRef{def:LPV}.
\begin{defn}[CT-IO-LPV system]
	\label{def:LPV}
	The considered class of LPV motion systems are static dependent on the scheduling and given by
	\begin{equation}
		\label{eq:LPVdesc}
			G_{LPV}:\:\sum_{i=-2}^{n_a}a_i(\rho(t)) y^{(i)}(t) \!=\!\sum_{j=0}^{n_{\mathrm{b}}}b_j(\rho(t)) \frac{d^{j}}{d t^{j}} \iint u(t) \, dt^2 
	\end{equation}
with scheduling sequence $\rho \in \mathbb{R}^{n_\rho}$ and $y^{(i)}(t)$ is the ${i}$th time derivative of $y(t)$ if $i\geq0$, and the $i$th integral if $i<0$. The following is assumed of the considered class of LPV systems.
\begin{assum}
	\label{ass:assLPVSystem}
	The following two assumptions are made for the considered LPV systems.
	\begin{enumerate}
		\item Only the body that is of interest is connected to the fixed world, hence the double integral of the input signal $u$ in \eqref{eq:LPVdesc}.
		\item The contributions of the zero dynamics to the system, i.e., $\sum_{j=1}^{n_b}b_j(\rho(t))\frac{d^j}{dt^j}\iint u(t) \: dt^2$, are negligible.
	\end{enumerate}
\end{assum}
The LPV coefficients $a_i(\rho(t))$ and $b_j(\rho(t))$ have a static dependency on $\rho(t)$, e.g., $a_i(\rho(t))=\rho(t)$ or $a_i(\rho(t)) = \rho^3(t)$. For ease of notation, the dependence of signals on $(t)$ is from now on omitted.
\end{defn}
The considered class of LPV motion systems in \defRef{def:LPV} shows a double integral of the input signal, which is generally the case for motion systems with masses. 
\begin{rem}
	The LPV coefficients $a_i(\rho)$ and $b_j(\rho)$ have a static dependency on the scheduling sequence $\rho$, but can be extended to include dynamic dependency, i.e., $a_i(\rho,\dot{\rho},\ldots)$ and $b_j(\rho,\dot{\rho},\ldots)$, and is part of ongoing research.
\end{rem}


%%%%%%%%%%%%%%%%%%%%%%%%%%%%%%%%%%%%%%%%%%%%%%%%%%%%%%%%%%%%%%%%%%%%%%%%%%%%%%%%%%%%%%%%%%%%%%%%%%%%%%%%%%%%%%%%
\subsection{LTI Feedforward}
LTI polynomial feedforward, see e.g. \Citet{Lambrechts2005}, approximates an inverse system to reduce the tracking error $e$. LTI polynomial feedforward applied to the LPV system in \eqref{eq:LPVdesc} parameterizes the feedforward by evaluating \eqref{eq:LPVdesc} at $\rho=\bar{\rho}$, assumes that $b_j=0 \: \forall j>0$ according to \assRef{ass:assLPVSystem}, and differentiates both sides twice, resulting in
\begin{equation}
	\label{eq:FFPolLTI}
	F_{LTI}:\quad u_{ff}= \sum_{i=-2}^{n_a}\frac{ a_i(\bar{\rho})}{b_0(\bar{\rho})} \frac{d^{i+2}}{dt^{i+2}}r = \sum_{i=1}^{n_\theta} \theta_i \psi_i\left( \frac{d}{dt}\right)r,
\end{equation}
where $\psi_i(\frac{d}{dt})$ contain differentiators, e.g., $\psi_i(\frac{d}{dt})=\frac{d^2}{dt^2}$ for acceleration feedforward. The parameters can be tuned manually \citep{Lambrechts2005} or estimated by using data \citep{Boeren2018a}. The resulting feedforward controller is an interpretable, simple and effective strategy for LTI systems, but does not take LPV dynamics into account. 
%%%%%%%%%%%%%%%%%%%%%%%%%%%%%%%%%%%%%%%%%%%%%%%%%%%%%%%%%%%%%%%%%%%%%%%%%%%%%%%%%%%%%%%%%%%%%%%%%%%%%%%%%%%%%%%%
\subsection{Feedforward Problem for LPV Systems}
Developing an inverse model for polynomial feedforward for LPV systems, similar to polynomial feedforward for LTI systems, is challenging due to the dynamic scheduling dependency introduced by deriving an inverse model. The dynamic dependence is observed when inverting \eqref{eq:LPVdesc}, i.e., by differentiating both sides twice, derivatives of the scheduling sequence $\rho$ directly appear. A fixed structure for identifying the inverse model could be used, e.g., a polynomial of $\rho$, $\dot{\rho}$ and $\ddot{\rho}$, however, it is unclear how to choose the structure and order. \exampleRef{example:LPVinverse} illustrates the complexity of feedforward control for LPV systems.

\begin{exmp}
	\label{example:LPVinverse}
	(Feedforward problem for LPV system) Consider the two-mass-spring-damper system with parameter-dependent spring in \figRef{fig:LPVMSD}, with input $u$, that is the force on the first mass, and output $y$, that is the position of the second mass.
	\begin{figure}
		\centering
		\includegraphics{pdf/TwoMassSystem.pdf}
		\caption{LPV mass-spring-damper with parameter-dependent spring $k(\rho)$ that is considered in this paper.}
		\label{fig:LPVMSD}
	\end{figure}
	The input-output behavior in the form of \eqref{eq:LPVdesc} is given by
	\begin{equation}
		\label{eq:exampleIO}
		\resizebox{\hsize}{!}{%
			$
			\begin{aligned}
				&\left(m_2m_1\frac{d^2}{dt^2} +\left( c\left(m_1+m_2 \right)+c_2m_1\right) \frac{d}{dt}+\left( k(\rho)\left(m_1+m_2 \right)+cc_2 \right)  \right) y \\
				&+k(\rho)c_2 \int y \,dt= \left(c\frac{d}{dt}+k(\rho) \right) \iint u \,dt^2,
			\end{aligned}$}%
	\end{equation}
	which directly shows the double integral of the input signal seen in \eqref{eq:LPVdesc}. LTI polynomial feedforward from \eqref{eq:FFPolLTI} is
	\begin{equation}
		\label{eq:LTIPolFFExample}
		\begin{aligned}
			u_{ff} = &\frac{m_2m_1}{k}\ddddot{r} + \frac{c_2m_1+c\left( m_1+m_2\right)}{k}   \dddot{r}\\
			&+ \left( \left(m_1+m_2 \right)+\frac{cc_2}{k}\right)\ddot{r} + c_2\frac{d}{dt}\dot{r} ,
		\end{aligned}
	\end{equation}
	that consists of the well-known snap, jerk, acceleration and velocity feedforward. However, the true inverse dynamics of \eqref{eq:exampleIO}, when neglecting the zero dynamics of the system, are given by
	\begin{equation}
		\label{eq:exampleIO2}
		\resizebox{\hsize}{!}{
			$
			\begin{aligned}
				u&\approx\frac{m_2m_1}{k(\rho)}\ddddot{y} + \frac{c\left( m_1+m_2\right)+c_2m_1}{k(\rho)} \dddot{y}+ \left(m_1+m_2 \right)\ddot{y} +c_2 \dot{y}\\
				&+ \left( \frac{\frac{2\dot{\rho}^2{k^{\prime}}^2(\rho)}{k(\rho)}-\dot{\rho}^2k^{\prime\prime}(\rho)-\ddot{\rho}k^\prime(\rho)}{k^2(\rho)}-\frac{2 \dot{\rho}k^\prime(\rho)  }{k^2(\rho)}\right)f(y,\dot{y},\ddot{y}),\\
			\end{aligned}$}
	\end{equation}
	with $f(y,\dot{y},\ddot{y}) = m_1m_2  \ddot{y}+\left(c\left( m_1+m_2\right) + c_2m_1\right) \dot{y}\allowbreak+\allowbreak \left(k(\rho)\left( m_1+m_2\right) +cc_2\right)y $, where $k^\prime=\frac{dk(\rho)}{d\rho}$ and $k^{\prime\prime}=\frac{d^2k(\rho)}{d\rho^2}$. When comparing LTI feedforward in \eqref{eq:LTIPolFFExample} with the approximate inverse in \eqref{eq:exampleIO2}, it is observed that LTI feedforward lacks both static and dynamic dependency on $\rho$.
\end{exmp}



%%%%%%%%%%%%%%%%%%%%%%%%%%%%%%%%%%%%%%%%%%%%%%%%%%%%%%%%%%%%%%%%%%%%%%%%%%%%%%%%%%%%%%%%%%%%%%%%%%%%%%%%%%%%%%%%%%%%
\subsection{Problem Definition}
\label{sec:pdef}
A technique for manual tuning or direct data-driven identification of feedforward controllers for LPV motion systems, including dynamic dependency on the scheduling sequence, is currently lacking. The problem addressed in this paper is the direct identification of LPV polynomial feedforward controller of the form
\begin{equation}
	\label{eq:Ftilde}
	\tilde{F}_{LPV}:\quad u_{ff} = \sum_{i=1}^{n_\theta} \theta_i(\rho,\dot{\rho},k^\prime(\rho),\ldots) \psi_i\left( \frac{d}{dt}\right)r,
\end{equation}
based on input-output data $\{u,y\}$, for the class of LPV motion systems in \eqref{eq:LPVdesc}, that includes dynamic dependence on the scheduling sequence $\rho$, where the structure and order of the model is not fixed a priori, and minimizes the tracking error $e$ in the configuration of \figRef{fig:controlStructure}.











