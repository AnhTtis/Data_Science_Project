\section{Introduction}
Feedforward control can compensate for known disturbances in motion control, such as a reference trajectory. Typically, a feedforward controller is based on the inverse of a system \citep{Hunt1996,Butterworth2012}, where the control performance is determined by the accuracy of the inverse model \citep{Devasia2002}. The increasing demands for motion control leads to a situation where Linear Parameter-Varying (LPV) dynamics have to be explicitly taken into account \citep{Wassink2005}.

For Linear Time-Invariant (LTI) systems, polynomial feedforward, where the feedforward signal is a linear combination of basis functions, results in good control performance. Often, the basis functions are chosen such that they relate to physical quantities, such as acceleration feedforward for the inertia \Citep{Lambrechts2005,Oomen2019}, and snap feedforward for the compliance of a system \citep{Boerlage2003}. Several approaches have been developed to tune the feedforward parameters based on data, such as iterative learning control \Citep{VanDeWijdeven2010} and instrumental variable identification \citep{Boeren2015} approaches. However, LTI feedforward leads to suboptimal performance when applied to LPV systems.

A key challenge in feedforward for LPV systems is modeling the dependency on the scheduling sequence. Additionally, the inversion of LPV systems generate terms that are often dynamically dependent on the scheduling sequence, i.e., dependency on the derivatives of the scheduling sequence \citep{Sato2003}. Hence, the dependence on the scheduling, including dynamic dependence, should be taken into account for feedforward of LPV systems and directly determines the achievable performance limit.

Several developments have been made in feedforward for LPV systems, and are directed at 1) identification of static LPV feedforward and 2) feedforward techniques based on forward LPV models. Inverse LPV system design is investigated in \citet{Balas2002, Sato2008}, but rely on the forward LPV model and do not take dynamic dependence into account. In \Citet{VanHaren2022a}, position-dependent snap feedforward is developed, that compensates for the static contribution of the position-dependent compliance. Data-driven feedforward approaches are developed in \citet{Butcher2009,DeRozario2018a}, but do not include dynamic dependence. In \citet{Theis2015,DeRozario2017,Bloemers2018}, state-space models of LPV systems are used to create inverse systems and in \citet{Kontaras2016} a compliance compensation is developed, that do include dynamic dependence, but all heavily rely on the quality of the model, which is not addressed in these papers. Hence, current feedforward approaches for LPV systems either do not take dynamic dependence into account, or heavily depend on models, which directly limits the achievable performance and imposes a large burden on modeling effort.

Although feedforward approaches for LPV systems have been substantially developed, techniques for direct and accurate identification of LPV feedforward controllers that include dynamic scheduling dependence, which is required for high-performance motion control, are currently lacking. In this paper, feedforward parameters for a class of LPV motion systems are directly identified using data with kernel-based approaches, see, e.g., \citep{Pillonetto2014,Blanken2020}, which results in a feedforward strategy that includes dynamic dependence on the scheduling sequence, and retains the polynomial feedforward structure, which is often desirable in motion control \citep{Lambrechts2005, Oomen2019}. This relates to the Bayesian approaches in \citep{Golabi2017,Darwish2018}, yet identifies inverse models for feedforward control and enables dynamic dependence on the scheduling sequence. The contributions include
\begin{itemize}
	\item[(C1) ] Development of a feedforward parameterization for LPV motion systems that includes dynamic dependency on the scheduling sequence.
	\item[(C2) ] Identification of feedforward parameters of the developed parameterization by kernel regularized methods.
	\item[(C3) ] Validation of the framework in a benchmark example.
\end{itemize}
The outline in this paper is as follows. In \secRef{sec:problem}, the feedforward problem for LPV motion systems is shown. In \secRef{sec:BFFF}, the developed feedforward parameterization is introduced. In \secRef{sec:kernel}, the identification of LPV feedforward parameters using input-output data is presented. In \secRef{sec:example}, a benchmark example is shown, validating the framework. Finally, in \secRef{sec:conclusions}, a summary and recommendations are given.


%Multiple approaches have been developed to identify parameteric models of LPV systems, e.g., using least-squares \citep{Bamieh2002}, orthonormal basis functions \citep{Toth2007}, IVs \citep{Toth2012} and Bayesian approaches \citep{Golabi2017,Darwish2018}. 