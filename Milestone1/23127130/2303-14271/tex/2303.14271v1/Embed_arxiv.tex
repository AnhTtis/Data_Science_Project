\UseRawInputEncoding 

\documentclass{article}

\usepackage{proof}
\usepackage{latexsym}
\usepackage{amsfonts}
\usepackage{amssymb}
\usepackage{cite}
%\usepackage{color}

%\newcommand{\bcor}{\begin{cor}}
%\newcommand{\ecor}{\end{cor}}


%\newtheorem{cor}[theorem]{Corollary}

%\usepackage[english,russian]{babel}
\usepackage{hyperref}       % hyperlinks
\usepackage{url}            % simple URL typesetting
\usepackage{booktabs}       % professional-quality tables
\usepackage{amsfonts}       % blackboard math symbols
\usepackage{tablefootnote}
\usepackage{verbatim}
\usepackage{nicefrac}       % compact symbols for 1/2, etc.
\usepackage{microtype}      % microtypography
\usepackage{lipsum}
\usepackage{mathtools}

\usepackage{amsmath,amssymb}
\usepackage{algorithm,algorithmic}
\usepackage{pifont}
\usepackage{cases}
\usepackage{subcaption,graphicx}
\usepackage{stackengine}    % circled symbols
\usepackage{wrapfig}
\usepackage{enumitem}

%\newtheorem{theorem}{Theorem}[section]
%\newtheorem{corollary}[theorem]{Corollary}
%\newtheorem{lemma}[theorem]{Lemma}
\newtheorem{assumption}[theorem]{Assumption}
%\newtheorem{definition}[theorem]{Definition}
%\newtheorem{remark}[theorem]{Remark}
%\newtheorem{proposition}[theorem]{Proposition}

\newcommand*{\LargerCdot}{\raisebox{-0.25ex}{\scalebox{2.4}{$\cdot$}}}
\newcommand{\Sum}{\displaystyle\sum\limits}
\newcommand{\Max}{\max\limits}
\newcommand{\Min}{\min\limits}
\newcommand{\fromto}[3]{{#1}=\overline{{#2},\,{#3}}}
\newcommand{\floor}[1]{\left\lfloor{#1}\right\rfloor}
\newcommand{\ceil}[1]{\left\lceil{#1}\right\rceil}

\newcommand{\tild}{\widetilde}
\newcommand{\eps}{\varepsilon}
\newcommand{\lam}{\lambda}
\newcommand{\ol}{\overline}
\newcommand{\one}{\mathbf{1}}
\newcommand{\cset}{\mathcal{C}}
%\newcommand{\Breg}{\mathcal{D}_{h}}
%\newcommand{\PBreg}{\mathbb{D}_{h}}

%\newcommand{\EndProof}{\begin{flushright}$\square$\end{flushright}}

\newcommand{\circledOne}{\text{\ding{172}}}
\newcommand{\circledTwo}{\text{\ding{173}}}
\newcommand{\circledThree}{\text{\ding{174}}}
\newcommand{\circledFour}{\text{\ding{175}}}
\newcommand{\circledFive}{\text{\ding{176}}}
\newcommand{\circledSix}{\text{\ding{177}}}
\newcommand{\circledSeven}{\text{\ding{178}}}
\newcommand{\circledEight}{\text{\ding{179}}}
\newcommand{\circledNine}{\text{\ding{180}}}
\newcommand{\circledTen}{\text{\ding{181}}}
\newcommand{\balashstar}{\stackMath\mathbin{\stackinset{c}{0ex}{c}{0ex}{\text{\ding{83}}}{\bigcirc}}}
\renewcommand\balashstar{\stackMath\mathbin{\stackinset{c}{0ex}{c}{0ex}{\ast}{\bigcirc}}}


\renewcommand{\le}{\leqslant}
\renewcommand{\ge}{\geqslant}
\renewcommand{\hat}{\widehat}

\newcommand{\numberthis}{\addtocounter{equation}{1}\tag{\theequation}}


\DeclareMathOperator*{\argmin}{arg\,min}
\DeclareMathOperator*{\argmax}{arg\,max}
\DeclareMathOperator*{\Argmin}{Arg\,min}
\DeclareMathOperator*{\Argmax}{Arg\,max}
\DeclareMathOperator{\spn}{span}
\DeclareMathOperator{\kernel}{Ker}
\DeclareMathOperator{\image}{Im}
\DeclareMathOperator{\prox}{prox}
\DeclareMathOperator{\proj}{Proj}
\DeclareMathOperator{\col}{col}
\DeclareMathOperator{\diag}{diag}

\newcommand{\N}{\mathbb{N}}
\newcommand{\R}{\mathbb{R}}
\newcommand{\Z}{\mathbb{Z}}
\newcommand{\V}{\mathbb{V}}
\newcommand{\E}{\mathbb{E}}
%\newcommand{\P}{\mathbb{P}}
\newcommand{\I}{\mathbb{I}}
\newcommand{\F}{\mathbb{F}}

\newcommand{\mA}{{\bf A}}
\newcommand{\mB}{{\bf B}}
\newcommand{\mC}{{\bf C}}
\newcommand{\mD}{{\bf D}}
\newcommand{\mE}{{\bf E}}
\newcommand{\mF}{{\bf F}}
\newcommand{\mG}{{\bf G}}
\newcommand{\mH}{{\bf H}}
\newcommand{\mI}{{\bf I}}
\newcommand{\mJ}{{\bf J}}
\newcommand{\mK}{{\bf K}}
\newcommand{\mL}{{\bf L}}
\newcommand{\mM}{{\bf M}}
\newcommand{\mN}{{\bf N}}
\newcommand{\mO}{{\bf O}}
\newcommand{\mP}{{\bf P}}
\newcommand{\mQ}{{\bf Q}}
\newcommand{\mR}{{\bf R}}
\newcommand{\mS}{{\bf S}}
\newcommand{\mT}{{\bf T}}
\newcommand{\mU}{{\bf U}}
\newcommand{\mV}{{\bf V}}
\newcommand{\mW}{{\bf W}}
\newcommand{\mX}{{\bf X}}
\newcommand{\mY}{{\bf Y}}
\newcommand{\mZ}{{\bf Z}}

\newcommand{\cA}{{\mathcal{A}}}
\newcommand{\cB}{{\mathcal{B}}}
\newcommand{\cC}{{\mathcal{C}}}
\newcommand{\cD}{{\mathcal{D}}}
\newcommand{\cE}{{\mathcal{E}}}
\newcommand{\cF}{{\mathcal{F}}}
\newcommand{\cG}{{\mathcal{G}}}
\newcommand{\cH}{{\mathcal{H}}}
\newcommand{\cI}{{\mathcal{I}}}
\newcommand{\cJ}{{\mathcal{J}}}
\newcommand{\cK}{{\mathcal{K}}}
\newcommand{\cL}{{\mathcal{L}}}
\newcommand{\cM}{{\mathcal{M}}}
\newcommand{\cN}{{\mathcal{N}}}
\newcommand{\cO}{{\mathcal{O}}}
\newcommand{\cP}{{\mathcal{P}}}
\newcommand{\cQ}{{\mathcal{Q}}}
\newcommand{\cR}{{\mathcal{R}}}
\newcommand{\cS}{{\mathcal{S}}}
\newcommand{\cT}{{\mathcal{T}}}
\newcommand{\cU}{{\mathcal{U}}}
\newcommand{\cV}{{\mathcal{V}}}
\newcommand{\cW}{{\mathcal{W}}}
\newcommand{\cX}{{\mathcal{X}}}
\newcommand{\cY}{{\mathcal{Y}}}
\newcommand{\cZ}{{\mathcal{Z}}}

\newcommand{\ba}{{\bf a}}
\newcommand{\bb}{{\bf b}}
\newcommand{\bc}{{\bf c}}
\newcommand{\bd}{{\bf d}}
\newcommand{\be}{{\bf e}}
%\newcommand{\bf}{{\bf f}}
\newcommand{\bg}{{\bf g}}
\newcommand{\bh}{{\bf h}}
\newcommand{\bi}{{\bf i}}
\newcommand{\bj}{{\bf j}}
\newcommand{\bk}{{\bf k}}
\newcommand{\bl}{{\bf l}}
\newcommand{\bm}{{\bf m}}
\newcommand{\bn}{{\bf n}}
\newcommand{\bo}{{\bf o}}
\newcommand{\bp}{{\bf p}}
\newcommand{\bq}{{\bf q}}
\newcommand{\br}{{\bf r}}
\newcommand{\bs}{{\bf s}}
\newcommand{\bt}{{\bf t}}
\newcommand{\bu}{{\bf u}}
\newcommand{\bv}{{\bf v}}
\newcommand{\bw}{{\bf w}}
\newcommand{\bx}{{\bf x}}
\newcommand{\by}{{\bf y}}
\newcommand{\bz}{{\bf z}}

\newcommand{\ds}{\displaystyle}
\newcommand{\norm}[1]{\left\| #1 \right\|}
\newcommand{\normtwo}[1]{\left\| #1 \right\|_2}
\newcommand{\sqn}[1]{\norm{#1}_2^2}
\newcommand{\angles}[1]{\left\langle #1 \right\rangle}
\newcommand{\cbraces}[1]{\left( #1 \right)}
\newcommand{\sbraces}[1]{\left[ #1 \right]}
\newcommand{\braces}[1]{\left\{ #1 \right\}}
\def\<#1,#2>{\langle #1,#2\rangle}

\newcommand{\sigmamax}{\sigma_{\max}(\cA)}
\newcommand{\sigmamaxsqr}{\sigma_{\max}^2(\cA)}
\newcommand{\sigmaminplus}{\sigma_{\min}^+(\cA)}
\newcommand{\sigmaminplussqr}{(\sigma_{\min}^+(\cA))^2}

\usepackage[colorinlistoftodos,bordercolor=blue,backgroundcolor=blue!20,linecolor=blue,textsize=scriptsize]{todonotes}
\newcommand{\arogozin}[1]{\todo[inline]{{\textbf{Alexander R.:} \emph{#1}}}}
\newcommand{\schezhegov}[1]{\todo[inline]{{\textbf{Savelii C.:} \emph{#1}}}}

\title{
Provably well-founded strict partial orders
}
\author{Toshiyasu Arai
\\
Graduate School of Mathematical Sciences
\\
University of Tokyo
\\
3-8-1 Komaba, Meguro-ku,
Tokyo 153-8914, JAPAN
\\
tosarai@ms.u-tokyo.ac.jp
}
%\date{25/3/2023}
\date{}
\begin{document}
\maketitle
\begin{abstract}
In this note we show through infinitary derivations that
each provably well-founded strict partial order in ${\rm ACA}_{0}$ admits an embedding to
an ordinal$<\veps_{0}$.
\end{abstract}

\section{Provably well-founded relations}

A \textit{strict partial order} $\prec$ is an irreflexive  $\fal n(n\not\prec n)$ and
transitive $\fal n,m,k(n\prec m\prec k\to n\prec k)$, relation on $\ome$.
Let `$\prec \mbox{ is a strict partial order}$' denotes the formula
$\fal n(n\not\prec n) \land \fal n,m,k(n\prec m\prec k\to n\prec k)$.
$<_{\veps_{0}}$ denotes a standard $\veps_{0}$-order, while
$<_{\ome}$ the usual order on $\ome$.

\bth\label{th:1}
Assume ${\rm ACA}_{0}\vdash {\rm TI}(\prec)$ for a primitive recursive relation $\prec$.
Then there exist an ordinal $\alp_{1}<\veps_{0}$ and a primitive recursive function $f$ such that
${\rm I}\Sig_{1}$ proves 
\[
\prec \mbox{ is a strict partial order } \to \fal n,m\left(n\prec m \to f(n)<_{\veps_{0}}f(m)<_{\veps_{0}}\alp_{1}\right)
.\]
\end{theorem}

Theorem \ref{th:1} is shown in \cite{Arai98a} by modifying Takeuti's proof in \cite{TRemark,PT2}
in terms of Gentzen's finitary proof\cite{GBew}.
In this note we show Theorem \ref{th:1} through infinitary derivations.

\bcor
Assume ${\rm ACA}_{0}\vdash {\rm TI}(\prec)$ for a primitive recursive relation $\prec$.
Then there exists an extension $\prec^{\prime}$ of $\prec$ such that
$\prec^{\prime}$ is primitive recursive, a well order, and
${\rm ACA}_{0}\vdash {\rm TI}(\prec^{\prime})$.
\ecor
\bprf
Let $n\prec^{\prime}m:\Lrarw f(n)<_{\veps_{0}}f(m) \lor\left(
f(n)=f(m) \land n<_{\ome}m\right)$.
\eprf

\section{Proof}
Assume for a primitive recursive relation $\prec$,
${\rm ACA}_{0}\vdash {\rm TI}(\prec)$.
In what follows argue in ${\rm I}\Sig_{1}$, and assume that $\prec$ is a strict partial order.
There exists an ordinal $\alp_{0}<\veps_{0}$ such that, cf.\cite{A2020}
\beqn\label{eq:E}
\fal n\left[ \vdash^{\alp_{0}}_{0}E(n)\right]
\eeqn
where $\vdash^{\alp}_{c}\Gam$ designates that `there exists a (primitive recursive) infinitary derivation of $\Gam$
with $\ome$-rule and the following inferences $(prg)$ and $(Rep)$
\[
\infer[(prg)]{\vdash^{\alp}_{c}\Gam}
{
\{
\vdash^{\bet}_{c}\Gam,E(m)
\}_{m\prec n}
}
\]
where $\bet<_{\veps_{0}}\alp$, $E$ is a fresh predicate symbol and $(E(n))\in\Gam$.
The subscript $0$ in $\vdash^{\alp_{0}}_{0}\Gam$ indicates that  a witnessed derivation is cut-free.
\[
\infer[(Rep)]{\vdash^{\alp}_{c}\Gam}
{\vdash^{\bet}_{c}\Gam}
\]
where $\bet<_{\veps_{0}}\alp$.

Formally we understand by (\ref{eq:E}) the following fact.
There exist a primitive recursive tree $T\subset{}^{<\ome}\ome$
and a primitive recursive function $H$ such that to each node $\sig\in T$, 
a five data $H(\sig)=(seq(\sig),ord(\sig), rul(\sig), crk(\sig), num(\sig))$ are
assigned by $H$.
Let $\Gam=seq(\sig)$, $\alp=ord(\sig)$, $c=crk(\sig)$ and $n=num(\sig)$.
Then $H(\sig)$ indicates that a sequent $\Gam$ is derived by a derivation in depth at most $\alp$
with cut rank $c$. $J=rul(\sig)$ is the last inference.
\[
\infer[(J)]{\sig \vdash^{\alp}_{c}\Gam}
{
\{
\sig_{i} \vdash^{\bet_{i}}_{c}\Gam_{i}
\}_{i\in I}
}
\]
has to be locally correct with respect to inferences $(\lor), (\land), (\exi), (\fal), (cut), (prg)$
and $(Rep)$,
and $\bet_{i}<_{\veps_{0}}\alp$ for each $i$.
Moreover when $J=rul(\sig)=(prg)$, $(E(n))\in\Gam=seq(\sig)$ with $n=num(\sig)$ is the main
formula of the $(prg)$. 
Then\footnote{$H(\la\, \ra)$ is arbitrary for the root $\la\,\ra$ of the tree.} $H(\la n\ra)=(\{E(n)\}, \alp_{0}, rul(\la n\ra),0)$ for each $n$.
Although $T$ is not assumed to be well-founded,
$rul(\sig)$ is either $(prg)$ or $(Rep)$ for each $\sig\in T$.
Therefore $seq(\sig)\subset\{E(n):n\in\ome\}$.
Let us assume that
\[
\infer[(prg)]{\sig \vdash^{\alp}_{c}\Gam}
{
\{
\sig*\la m\ra \vdash^{\bet}_{c}\Gam,E(m)
\}_{m\prec n}
}
\quad
\infer[(Rep)]{\sig \vdash^{\alp}_{c}\Gam}
{\sig*\la 0\ra \vdash^{\bet}_{c}\Gam}
\]

First we define nodes $\sig_{m}\in T$ by induction on $m$ as follows.
Let $\bet_{m}=ord(\sig_{m})$ and $\Gam_{m}=seq(\sig_{m})$ and $J_{m}=rul(\sig_{m})$.
Namely $\sig_{m} \vdash^{\bet_{m}}_{0}\Gam_{m}$.
It enjoys
\beqn\label{eq:min}
\fal n((E(n))\in\Gam_{m}\Rarw m\preceq n)
\eeqn
\textbf{Case 1}.
$\lnot\exi n<_{\ome}m(m\prec n)$:
Then let $\sig_{m}=\la m\ra$.
This means that $\bet_{m}=\alp_{0}$ and $\Gam_{m}=\{E(m)\}$.
\\
\textbf{Case 2}.
$\exi n<_{\ome}m(m\prec n)$:
Let $n_{0}<_{\ome}m$ be the $<_{\ome}$-least number such that $m\prec n_{0}$ and
$\bet_{n_{0}}=\min_{<_{\veps_{0}}}\{\bet_{n}: n<_{\ome}m, \, m\prec n\}$.
Consider the last inference 
$J_{n_{0}}=rul(\sig_{n_{0}})$ in the derivation of 
$\sig_{n_{0}} \vdash^{\bet_{n_{0}}}_{0}\Gam_{n_{0}}$.
\\
\textbf{Case 2.1}. The last inference $J_{n_{0}}$ is a $(prg)$:
\[
\infer[(prg)]{\sig_{n_{0}} \vdash^{\bet_{n_{0}}}_{0}\Gam_{n_{0}}}
{
\{
\sig_{n_{0}}*\la n\ra \vdash^{\bet}_{0}\Gam_{n_{0}}, E(n)
\}_{n\prec n_{1}}
}
\]
where $\bet<_{\veps_{0}}\bet_{n_{0}}$ and $(E(n_{1}))\in\Gam_{n_{0}}$ with
$n_{1}=num(\sig_{n_{0}})$.
We have $m\prec n_{0}\preceq n_{1}$ by (\ref{eq:min}).
Then let $\sig_{m}=\sig_{n_{0}}*\la m\ra$.
Let $\bet_{m}=\bet$ and $\Gam_{m}=\Gam_{n_{0}}\cup \{E(m)\}$.
If $(E(n))\in\Gam_{n_{0}}$, then $m\prec n_{0}\preceq n$ by (\ref{eq:min}).
Hence (\ref{eq:min}) is enjoyed for $\sig_{m}$ since $\prec$ is assumed to be transitive.
\\
\textbf{Case 2.2}. The last inference $J_{n_{0}}$ is a $(Red)$:
\[
\infer[(Rep)]{\sig_{n_{0}} \vdash^{\bet_{n_{0}}}_{0}\Gam_{n_{0}}
}
{
\sig_{n_{0}}*\la 0\ra \vdash^{\bet}_{0}\Gam_{n_{0}}
}
\]
where $\bet<\bet_{n_{0}}$.
Then let $\sig_{m}=\sig_{n_{0}}*\la 0\ra$.
This means $\bet_{m}=\bet$ and $\Gam_{m}=\Gam_{n_{0}}$.
Again (\ref{eq:min}) is enjoyed for $\sig_{m}$ by the transitivity of $\prec$.

\blem\label{lem:1}
$\fal m\fal n<_{\ome}m\left[
m\prec n \Rarw \bet_{m}<_{\veps_{0}}\bet_{n}
\right]
$.
\elem
\bprf
In \textbf{Case 2}, if $n<_{\ome}m$ and $m\prec n$,
then $\bet_{m}<_{\veps_{0}}\bet_{n_{0}}\leq_{\veps_{0}}\bet_{n}$.
\eprf
\\

Now let us define $\alp_{1}=\ome^{\alp_{0}}$ and $f$ as follows.

\[
f(n)=\max_{<_{\veps_{0}}}\{\ome^{\bet_{n_{0}}}\#\cdots\#\ome^{\bet_{n_{\ell-1}}}\#\ome^{\bet_{n_{\ell}}}:
%n_{0}\prec\cdots\prec n_{\ell-1}\prec n_{\ell}, \,
 \fal i<\ell( n_{i}\prec n_{i+1} \spand n_{i}<_{\ome}n_{\ell}=n)\}
\]
where $\#$ denotes the natural sum.
Note that $n_{i}\neq n_{j}$ for $i<j\leq\ell$ 
%the set is finite 
since $\prec$ is assumed to be a strict partial order.
The following Lemma \ref{lem:2} shows Theorem \ref{th:1}.

\blem\label{lem:2}
$\fal n,m\left[n\prec m \Rarw f(n)<_{\veps_{0}}f(m)<\ome^{\alp_{0}+1}=\alp_{1}\right]$.
\elem
\bprf
Let 
$n_{0},\ldots,n_{\ell-1}<_{\ome}n_{\ell}=n\prec m$ be such that
$n_{0}\prec\cdots\prec n_{\ell-1}\prec n_{\ell}$ and
\[
f(n)=\ome^{\bet_{n_{0}}}\#\cdots\#\ome^{\bet_{n_{\ell-1}}}\#\ome^{\bet_{n_{\ell}}}
.\]
Then $n_{i}\prec m$ and $n_{i}\neq m$.
Let $A=\{i\leq \ell: m<_{\ome}n_{i}\}$ and $B=\{i\leq\ell: n_{i}<_{\ome}m\}$.
Then $A\cup B=\{0,\ldots,\ell\}$ and $A\cap B=\emptyset$.
By Lemma \ref{lem:1} we obtain
$\fal i\in A(\bet_{n_{i}}<_{\veps_{0}}\bet_{m})$, and hence
\beqn\label{eq:A}
\sum\{\ome^{\bet_{n_{i}}} : i\in A\}<_{\veps_{0}}\ome^{\bet_{m}}
\eeqn
where
$\sum\{\alp_{0},\ldots,\alp_{n}\}=\alp_{0}\#\cdots\#\alp_{n}$.
On the other side let
\[
\gam:= \max_{<_{\veps_{0}}}\{\ome^{\bet_{m_{0}}}\#\cdots\#\ome^{\bet_{m_{k-1}}}:
\fal i<k(m_{i}\prec m_{i+1}\spand m_{i}<_{\ome}m_{k}=m)\}
%m_{0}\prec\cdots\prec m_{k-1}\prec m, \, m_{0},\ldots,m_{k-1}<_{\ome}m\}
\]
and $B=\{n_{i_{0}}\prec\cdots\prec n_{i_{\ell-1}}\}$.
Then $n_{i_{0}}\prec\cdots\prec n_{i_{\ell-1}}\prec m$ and $n_{j}<_{\ome}m$
for each $n_{j}\in B$ since $\prec$ is assumed to be transitive.
Therefore
\beqn\label{eq:B}
\sum\{\ome^{\bet_{n_{i}}} : i\in B\} \leq_{\veps_{0}} \gam
\eeqn
By (\ref{eq:B}) and (\ref{eq:A}) we conclude 
\[
f(n)=\sum\{\ome^{\bet_{n_{i}}} : i\in B\} \# \sum\{\ome^{\bet_{n_{i}}} : i\in A\} 
<_{\veps_{0}}\gam\#\ome^{\bet_{m}}=f(m)
.\]
\eprf
\\

When $\prec$ is elementary recursive, then so is $f$.
For almost all theories $T$, Theorem \ref{th:1} holds if the ordinal $\veps_{0}$ is replaced by
the proof-theoretic ordinal of $T$
provided that a reasonable ordinal analysis of $T$ is given.



\begin{thebibliography}{99}
\bibitem{Arai98a}
T. Arai, 
Some results on cut-elimination, provable well-orderings, induction and reflection.
Ann. Pure Appl. Logic 95 (1998) 93-184.

\bibitem{A2020}
T. Arai,
Ordinal analysis with an introduction to proof theory,
Springer, 2020.

\bibitem{GBew}
G. Gentzen, 
Beweibarkeit und Unbeweisbarkeit von Anfangsf\"allen der transfiniten Induktion in der reinen Zahlentheorie, Math. Ann. 119 (1943) 140-161.

\bibitem{TRemark}
G. Takeuti, 
A remark on Gentzen's paper "Beweibarkeit und Unbeweisbarkeit von Anfangsf\"allen der transfiniten Induktion in der reinen Zahlentheorie", Proc. Japan Acad. 39 (1963) 263-269.

\bibitem{PT2}
G. Takeuti, 
Proof Theory, second edition, North-Holland, Amsterdam (1987)
reprinted from Dover Publications (2013)

\end{thebibliography}
\end{document}