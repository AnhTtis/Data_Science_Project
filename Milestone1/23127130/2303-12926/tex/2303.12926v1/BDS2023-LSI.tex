%%%%%%%%%%%%%%%%%%%%%%%%%%%%%%%%%%%%%%%%%%%%%%%%%%%%%%%%%%%%%%%%%%%%%%%%%%
% !TEX encoding = UTF-8 Unicode
%%%%%%%%%%%%%%%%%%%%%%%%%%%%%%%%%%%%%%%%%%%%%%%%%%%%%%%%%%%%%%%%%%%%%%%%%%
\documentclass{elsarticle-ppn}

%--- Packages ---
\usepackage[T1]{fontenc}
\usepackage[utf8]{inputenc}
\usepackage{amsmath,amsfonts,amsthm,amssymb,amsxtra,bbm}
\usepackage{fourier,setspace,graphicx,color,pdflscape}
\usepackage{textcomp}
\usepackage{MnSymbol}
\usepackage{color}
\usepackage[colorlinks=true]{hyperref}
\hypersetup{urlcolor=blue, linkcolor=blue, citecolor=red, anchorcolor=blue]}
\usepackage{mathrsfs}
\usepackage{mathtools}
\usepackage{setspace}
\usepackage{lineno,hyperref}

%--- Equation numbers ---
\usepackage{etoolbox}
\newcounter{taggedeq}
\setcounter{taggedeq}{0}
\pretocmd{\equation}{\stepcounter{taggedeq}}{}{}
\renewcommand*{\theHequation}{\thetaggedeq.\theequation}

%--- Theorem structure ---
\newtheorem{theorem}{Theorem}
\newtheorem{proposition}[theorem]{Proposition}
\newtheorem{lemma}[theorem]{Lemma}
\newtheorem{corollary}[theorem]{Corollary}
\newtheorem{remark}[theorem]{Remark}
\newtheorem{remarks}[theorem]{Remarks}
\newcommand{\subsubsubsection}[1]{\par\noindent$\rhd$~{#1}.~\/}

%--- Commands and math operators ---
\newcommand{\email}[1]{\href{mailto:#1}{\textsf{#1}}}
\newcommand{\R}{{\mathbb R}}
\newcommand{\N}{{\mathbb N}}
\newcommand{\Z}{{\mathbb Z}}
\newcommand{\E}{{\mathscr E}}
\newcommand{\I}{{\mathscr I}}
\renewcommand{\L}{{\mathscr L}}
\renewcommand{\S}{{\mathbb S}}
\newcommand{\be}[1]{\begin{equation}\label{#1}}
\newcommand{\spann}{\mathrm{Span}}
\newcommand{\ee}{\end{equation}}
\renewcommand{\(}{\left(}
\renewcommand{\)}{\right)}
\newcommand{\ird}[1]{\int_{\R^d}{#1}\,dx}
\newcommand{\isd}[1]{\int_{\S^d}{#1}\,d\mu}
\newcommand{\nrm}[2]{\left\|#1\right\|_{\mathrm L^{#2}(\R^d)}}
\newcommand{\nrmS}[2]{\left\|#1\right\|_{\mathrm L^{#2}(\S^d)}}
\newcommand{\irdg}[1]{\int_{\R^d}{#1}\,d\gamma}
\newcommand{\nrmG}[2]{\left\|#1\right\|_{\mathrm L^{#2}(\R^d,d\gamma)}}

\newcommand{\red}{\color{red}}
\newcommand{\blue}{\color{blue}}
\newcommand{\nc}{\normalcolor}
\newcommand{\question}[1]{\red$\rhd$~\textsf{#1}\nc}
\newcommand{\gio}[1]{{\color{blue}{#1}}}
\newcommand{\gioq}[1]{{\color{purple}{#1}}}

%--- 2020 MSC ---
\makeatletter
\@namedef{subjclassname@2020}{\textup{2020} Mathematics Subject Classification}
\makeatother
\newcommand{\msc}[1]{\href{https://mathscinet.ams.org/mathscinet/search/mscdoc.html?code=#1}{#1}}

%--- Bibliography ---
\let\oldthebibliography\thebibliography\let\endoldthebibliography\endthebibliography
\renewenvironment{thebibliography}[1]{\begin{oldthebibliography}{#1}\setlength{\itemsep}{0em}\setlength{\parskip}{0em}}{\end{oldthebibliography}}

%%%%%%%%%%%%%%%%%%%%%%%%%%%%%%%%%%%%%%%%%%%%%%%%%%%%%%%%%%%%%%%%%%%%%%%%%%
\begin{document}
\begin{frontmatter}

\title{Stability for the logarithmic Sobolev inequality}

\author[CEREMADE]{Giovanni Brigati}
\ead{brigati@ceremade.dauphine.fr}
\author[CEREMADE]{Jean Dolbeault\corref{cor}}
\ead{dolbeaul@ceremade.dauphine.fr}
\author[LJLL]{Nikita Simonov}
\ead{nikita.simonov@sorbonne-universite.fr}
\cortext[cor]{Corresponding author}
\address[CEREMADE]{CEREMADE (CNRS UMR n$^\circ$~7534), PSL University, Universit\'e Paris-Dauphine,\newline Place de Lattre de Tassigny, 75775 Paris 16, France}
\address[LJLL]{LJLL (CNRS UMR n$^\circ$~7598), Sorbonne Universit\'e, 4 place Jussieu, 75005 Paris, France}

\begin{abstract} This paper is devoted to stability results for the Gaussian logarithmic Sobolev inequality. Our approach covers several cases involving the strongest possible norm with the optimal exponent, under constraints. Explicit constants are obtained. The strategy of proof relies on entropy methods and the Ornstein-Uhlenbeck flow. \end{abstract}

\begin{keyword}
logarithmic Sobolev inequality, stability, log-concavity, heat flow, entropy, carr\'e du champ.
\MSC[2020] Primary: \msc{39B62}; Secondary: \msc{47J20}, \msc{49J40}, \msc{35A23}, \msc{35K85}.

% {Primary: \msc{26D10}; Secondary: \msc{58J99}, \msc{39B62}, \msc{43A90}, \msc{49J40}, \msc{46E35}}
% 26 (1940-now) Real functions [See also 54C30]

% 26D (1980-now) Inequalities in real analysis {For maximal function inequalities, see 42B25; for functional inequalities, see 39B72; for probabilistic inequalities, see 60E15}
% 26D10 (1980-now) Inequalities involving derivatives and differential and integral operators
% 26D15 (1980-now) Inequalities for sums, series and integrals
%
% 35 (1940-now) Partial differential equations
% 35A (1973-now) General topics in partial differential equations
% 35A23 (2010-now) Inequalities applied to PDEs involving derivatives, differential and integral operators, or integrals
% 35J (1973-now) Elliptic equations and elliptic systems {For global analysis, analysis on manifolds, see 58J10, 58J20}
% 35J20 (1973-now) Variational methods for second-order elliptic equations
% 35J60 (1973-now) Nonlinear elliptic equations
% 35K (1973-now) Parabolic equations and parabolic systems {For global analysis, analysis on manifolds, see 58J35}
% 35K85 (1980-now) Unilateral problems for linear parabolic equations and variational inequalities with linear parabolic operators [See also 35R35, 49J40]
%
% 39 (1940-now) Difference and functional equations
% 39B (1980-now) Functional equations and inequalities [See also 30D05]
% 39B62 (1991-now) Functional inequalities, including subadditivity, convexity, etc. [See also 26A51, 26B25, 26Dxx]
% 39B72 (1991-now) Systems of functional equations and inequalities
%
% 43 (1973-now) Abstract harmonic analysis {For other analysis on topological and Lie groups, see 22Exx}
% 43A (1973-now) Abstract harmonic analysis {For other analysis on topological and Lie groups, see 22Exx}
% 43A85 (1973-now) Harmonic analysis on homogeneous spaces
% 43A90 (1973-now) Harmonic analysis and spherical functions [See also 22E45, 22E46, 33C55]
%
% 46 (1940-now) Functional analysis {For manifolds modeled on topological linear spaces, see 57Nxx, 58Bxx}
% 46E (1973-now) Linear function spaces and their duals [See also 30H05, 32A38, 46F05] {For function algebras, see 46J10}
% 46E35 (1973-now) Sobolev spaces and other spaces of "smooth'' functions, embedding theorems, trace theorems
%
% 47 (1959-now) Operator theory
% 47J (2000-now) Equations and inequalities involving nonlinear operators [See also 46Txx] {For global and geometric aspects, see 58-XX}
% 47J20 (2000-now) Variational and other types of inequalities involving nonlinear operators (general) [See also 49J40]
%
% 49 (1940-now) Calculus of variations and optimal control; optimization [See also 34H05, 34K35, 65Kxx, 90Cxx, 93-XX]
% 49J (1991-now) Existence theories in calculus of variations and optimal control
% 49J40 (1991-now) Variational inequalities [See also 47J20]
%
% 53 (1940-now) Differential geometry {For differential topology, see 57Rxx; for foundational questions of differentiable manifolds, see 58Axx}
% 53C (1973-now) Global differential geometry [See also 51H25, 58-XX] {For related bundle theory, see 55Rxx, 57Rxx}
% 53C21 (1980-now) Methods of global Riemannian geometry, including PDE methods; curvature restrictions [See also 58J60]
% 58 (1973-now) Global analysis, analysis on manifolds [See also 32Cxx, 32Fxx, 32Wxx, 46-XX, 47Hxx, 53Cxx] {For geometric integration theory, see 49Q15}
% 58E (1973-now) Variational problems in infinite-dimensional spaces
% 58E35 (1980-now) Variational inequalities (global problems) in infinite-dimensional spaces
% 58 (1973-now) Global analysis, analysis on manifolds [See also 32Cxx, 32Fxx, 32Wxx, 46-XX, 47Hxx, 53Cxx] {For geometric integration theory, see 49Q15}
% 58J (2000-now) Partial differential equations on manifolds; differential operators [See also\newline 32Wxx, 35-XX, 53Cxx]
% 58J99 (2000-now) None of the above, but in this section
\end{keyword}
\end{frontmatter}


%%%%%%%%%%%%%%%%%%%%%%%%%%%%%%%%%%%%%%%%%%%%%%%%%%%%%%%%%%%%%%%%%%%%%%%%%%
%%%%%%%%%%%%%%%%%%%%%%%%%%%%%%%%%%%%%%%%%%%%%%%%%%%%%%%%%%%%%%%%%%%%%%%%%%
\section{Introduction and main results}\label{Sec:Intro}

Let us consider the \emph{Gaussian logarithmic Sobolev inequality}
\be{LSIg}
\nrmG{\nabla u}2^2\ge\frac12\irdg{|u|^2\,\log\(\frac{|u|^2}{\nrm u2^2}\)}\quad\forall\,u\in\mathrm H^1(\R^d,d\gamma)
\ee
where $d\gamma=\gamma(x)\,dx$ is the normalized Gaussian probability measure with density
\[
\gamma(x)=(2\,\pi)^{-\frac d2}\,e^{-\frac12\,|x|^2}\quad\forall\,x\in\R^d\,.
\]
In this paper we are interested in stability results, that is, in estimating the difference of the two terms in~\eqref{LSIg} from below, by a distance to the set of optimal functions. According to~\cite{MR1132315,bakry1985diffusions}, equality in~\eqref{LSIg} is achieved by functions in the manifold
\[
\mathscr M:=\big\{w_{a,c}\,:\,(a,c)\in\R^d\times\R\big\}
\]
where
\[
w_{a,c}(x)=c\,e^{-a\cdot x}\quad\forall\,x\in\R^d
\]
and only by these functions. The ultimate goal of \emph{stability estimates} is to find a notion of distance $\mathsf d$, an explicit constant $\beta>0$ and an explicit exponent $\alpha>0$, which may depend on $\mathsf d$, such that
\be{SuperStab}\tag{$\mathscr S$}
\nrmG{\nabla u}2^2-\frac12\irdg{|u|^2\,\log\(\frac{|u|^2}{\nrm u2^2}\)}\ge\beta\,\inf_{w\in\mathscr M}\mathsf d(u,w)^\alpha
\ee
for any given $u\in\mathrm H^1(\R^d,d\gamma)$. In this paper we consider the slightly simpler question of finding a specific $w_u\in\mathscr M$ such that
\be{stab}\tag{$\star$}
\nrmG{\nabla u}2^2-\frac12\irdg{|u|^2\,\log\(\frac{|u|^2}{\nrm u2^2}\)}\ge\beta\,\mathsf d(u,w_u)^\alpha\,,
\ee
which provides us with no more than an estimate for~\eqref{SuperStab}: any estimate of $\alpha$ and $\beta$ for~\eqref{stab} is also an estimate for~\eqref{SuperStab}. In order to illustrate the difference between the two questions, let us consider the following elementary example. Assume that $d=1$ and consider the functions $u_\varepsilon(x)=1+\varepsilon\,x$ in the limit as $\varepsilon\to0$. With $\mathsf d(u,w)=\|u'-w'\|_{\mathrm L^2(\R,d\gamma)}$, which is the strongest possible notion of distance that we can expect to control in~\eqref{stab}, elementary computations show that the \emph{deficit} of the logarithmic Sobolev inequality, \emph{i.e.}, the left hand-side in~\eqref{stab}, is
\[
\nrmG{\nabla u_\varepsilon}2^2-\frac12\irdg{|u_\varepsilon|^2\,\log\(\frac{|u_\varepsilon|^2}{\nrm{u_\varepsilon}2^2}\)}=\frac12\,\varepsilon^4+O\big(\varepsilon^6\big)\,,
\]
while, using the test function $w_{a_\varepsilon,c_\varepsilon}\in\mathscr M$ where $a_\varepsilon=2\,\varepsilon$ and $c_\varepsilon=e^{-a_\varepsilon^2/4}$, we obtain
\[
d(u_\varepsilon,1)^2=\|u_\varepsilon'\|_{\mathrm L^2(\R,d\gamma)}^2=\varepsilon^2\quad\mbox{and}\quad\inf_{w\in\mathscr M}\mathsf d(u_\varepsilon,w)^\alpha\le d\(u_\varepsilon,w_{a_\varepsilon,c_\varepsilon}\)^2=\frac12\,\varepsilon^4+O\big(\varepsilon^6\big)\,.
\]
In practice we will consider only the case
\[
w_u=1
\]
in~\eqref{stab} and the above example shows that the best we can hope for without additional restriction is $\alpha\ge4$. Similar examples in higher dimensions can be obtained by considering for an arbitrary given $\nu\in\S^{d-1}$ the functions $u_\varepsilon(x)=1+\varepsilon\,x\cdot\nu$ in the limit as $\varepsilon\to0$. This is not a surprise in view of~\cite{MR3567822,Dolbeault_2015,https://doi.org/10.48550/arxiv.2303.05604}, and also of the detailed Taylor expansions of~\cite{Frank_2022,https://doi.org/10.48550/arxiv.2211.13180,https://doi.org/10.48550/arxiv.2302.03926}. Still with $w_u=1$, we can expect to have $\alpha=2$ in~\eqref{stab} \emph{under  additional conditions}, including for $\mathsf d(u,w)=\nrmG{\nabla u-\nabla w}2$, while it is otherwise banned as shown for instance from~\cite[Theorem~1.2~(2)]{https://doi.org/10.48550/arxiv.2303.05604}, or simply from considering the above example. Before entering the details, let us mention a recent stability result for~\eqref{SuperStab} with $\alpha=2$ involving a constructive although very delicate expression for $\beta>0$ and $\mathsf d(u,w)=\nrmG{u-w}2$ that appeared in~\cite{https://doi.org/10.48550/arxiv.2209.08651}. Here we aim at stronger estimates under additional constraints, with $w_u=1$, which is a different point of view. Let us start by a first stability result.
%-------------------------------------------------------------------------
\begin{proposition}\label{Prop1}
For all $u\in\mathrm H^1(\R^d,d\gamma)$ such that $\nrm u2=1$ and $\nrm{x\,u}2^2\le d$, we have
\be{stabsq1}
\nrmG{\nabla u}2^2-\frac12\irdg{|u|^2\,\log|u|^2}\ge\frac1{2\,d}\,\(\irdg{|u|^2\,\log|u|^2}\)^2
\ee
and, with $\psi(s):=s-\frac d4\,\log\(1+\frac4d\,s\)$, we also have the stronger estimate
\be{stabsq2}
\nrmG{\nabla u}2^2-\frac12\irdg{|u|^2\,\log|u|^2}\ge\psi\(\nrmG{\nabla u}2^2\)\,.
\ee
\end{proposition}
%-------------------------------------------------------------------------
\noindent Similar results are already known in the literature (see for instance~\cite{MR3269872,MR3567822,Dolbeault_2015,https://doi.org/10.48550/arxiv.2303.05604}) and we claim no originality for the the results. Also see references to earlier proofs at the end of the introduction. Our method is based on the \emph{carr\'e du champ} method. Even if some ideas go back to~\cite{MR2152502}, it is elementary, new as far as we know, and of some use for our other results.

Coming back to~\eqref{stab}, we may notice that there is no loss of generality in imposing the condition $\nrm u2=1$, as we can always replace $u$ by $u/\nrm u2$. Because of the Csisz\'ar-Kullback-Pinsker inequality
\be{CKP}
\irdg{|u|^2\,\log|u|^2}\ge\frac14\,\nrmG{|u|^2-1}1^2
\ee
and $\big|\,|u|-1\,\big|=\big|\,|u|^2-1\,\big|/\big|\,|u|+1\,\big|\le\big|\,|u|^2-1\,\big|$, we find that~\eqref{stabsq1} implies~\eqref{stab} type with $\mathsf d(u,w)=\nrmG{u-w}1$ for nonnegative functions $u$, $\alpha=4$, and $\beta=1/(32\,d)$. For functions far away from the optimal functions, say such that $\nrmG{\nabla u}2\ge A$ under the conditions of Proposition~\ref{Prop1}, Inequality~\eqref{stabsq2} provides us with an even stronger stability result of~\eqref{stab} type with $\alpha=2$ and $\mathsf d(u,w)=\nrmG{\nabla u-\nabla w}2$, but with a positive constant $\beta$ which depends on $A>0$. Again, notice that
~\eqref{stab} with such a distance cannot hold without constraints. 

\medskip Next we aim at explicit results with $\alpha=2$, under other constraints. Let
\[
\mathscr C_\star=1+\frac1{1728}\approx1.0005787\,.
\]
%-------------------------------------------------------------------------
\begin{theorem}\label{ThmMain1} For all $u\in\mathrm H^1(\R^d,d\gamma)$ such that $u^2\,\gamma$ is log-concave and such that
\be{constrr}
\irdg{\left(1,x\right)\,|u|^2}=(1,0)\quad\mbox{and}\quad\irdg{|x|^2\,|u|^2}\le d\,,
\ee
we have
\be{stabg}
\nrmG{\nabla u}2^2-\frac{\mathscr C_\star}2\irdg{|u|^2\,\log|u|^2}\ge0\,.
\ee
\end{theorem}
%-------------------------------------------------------------------------
\noindent The condition $\irdg{|x|^2\,|u|^2}\le d$ in~\eqref{constrr} is a simplifying assumption. A result like~\eqref{stabg} also holds if $\irdg{|x|^2\,|u|^2}>d$, but with a constant that differs from $\mathscr C_\star$ and actually depends on $\irdg{|x|^2\,|u|^2}$. We refer to Section~\ref{Sec:Normalization}: see Proposition~\ref{ThmMain1bis} for an extension of Theorem~\ref{ThmMain1}, and also for further comments on the extension of Proposition~\ref{Prop1}. The constant~$\mathscr C_\star$ in~\eqref{LSIalpha} relies on an estimate of~\cite{MR1742893}.


Inequality~\eqref{stabg} with improved constant $\mathscr C_\star>1$ compared to~\eqref{LSIg} can be recast in the form of a stability inequality of type~\eqref{stab} around the normalised Gaussian as
\[
\nrmG{\nabla u}2^2-\frac12\irdg{|u|^2\,\log|u|^2}\ge\frac12\,(\mathscr C_\star-1)\irdg{|u|^2\,\log|u|^2}
\]
for all functions $u\in\mathrm H^1(\R^d,d\gamma)$ such that $\nrm u2=1$, which covers the case $\alpha=2$, $\beta=(\mathscr C_\star-1)/8$ and $\mathsf d(u,w)=\nrmG{u-w}1$ in~\eqref{stab} for nonnegative functions by~\eqref{CKP}, or even in the stronger $\dot{\mathrm H}^1(\R^d,d\gamma)$ semi-norm, as
\[
\nrmG{\nabla u}2^2-\frac12\irdg{|u|^2\,\log|u|^2}\ge\frac{\mathscr C_\star-1}{\mathscr C_\star}\,\nrmG{\nabla u}2^2
\]
for all functions $u\in\mathrm H^1(\R^d,d\gamma)$ such that $\nrm u2=1$, which corresponds to $\alpha=2$, $\beta=(\mathscr C_\star-1)/\mathscr C_\star$ and $\mathsf d(u,w)=\nrmG{\nabla(u-w)}2$ in~\eqref{stab}. By the Gaussian Poincar\'e inequality, notice that the case of~\eqref{stab} with $\alpha=2$, $\beta=(\mathscr C_\star-1)/\mathscr C_\star$ and the standard distance $\mathsf d(u,w)=\nrmG{u-w}2$ is also covered.

\medskip Log-concavity might appear as a rather restrictive assumption, but this is useful because a function which is compactly supported at time $t=0$ evolves through the diffusion flow into a logarithmically concave function after some finite time that can be estimated by the heat flow estimates of~\cite{lee2003geometrical}. This is enough to produce a stability result with an explicit constant. Compact support is in fact a too restrictive condition and we have the following result.
%-------------------------------------------------------------------------
\begin{theorem}\label{ThmMain2} Let $d\ge1$. For any $\varepsilon>0$, there is some explicit $\mathscr C>1$ depending only on $\varepsilon$ such that, for any $u\in\mathrm H^1(\R^d,d\gamma)$ satisfying~\eqref{constrr} and
\be{gautail}
\irdg{|u|^2\,e^{\,\varepsilon\,|x|^2}}<\infty\,,
\ee
then we have
\be{LSIalpha}
\nrmG{\nabla u}2^2\ge\frac{\mathscr C}2\irdg{|u|^2\,\log|u|^2}\,.
\ee
Additionally, if $u$ is compactly supported in a ball of radius $R>0$, then~\eqref{LSIalpha} holds with
\[
\mathscr C=1+\frac{\mathscr C_\star-1}{1+\mathscr C_\star\,R^2}\,.
\]
\end{theorem}
%-------------------------------------------------------------------------
\noindent This expression of the constant~$\mathscr C$ in~\eqref{LSIalpha} is given in the proof, in Section~\ref{Sec:Extension}. The simpler estimate in terms of $R$
relies on Theorem~\ref{ThmMain1}.

\medskip Let us conclude this introduction with a review of the literature. The \emph{logarithmic Sobolev inequality} historically appeared in~\cite{MR0109101,blachman1965convolution}, in a form that was later rediscovered as an equivalent scale-invariant form of the Euclidean version of the inequality in~\cite{MR479373}. We refer to~\cite{Gross75} for the Gaussian version~\eqref{LSIg} of the inequality, and also to~\cite{Federbush} for an equivalent result. The optimality case in the inequality has been characterized in~\cite{MR1132315} but can also be deduced from~\cite{bakry1985diffusions}. Also see~\cite{zbMATH01503413} for a short introductory review with an emphasis on \emph{information theoretical} aspects. The logarithmic Sobolev inequality can be viewed as a limit case of a family of the Gagliardo-Nirenberg-Sobolev (GNS) as observed in~\cite{MR1940370}, in the Euclidean setting, or as a large dimension limit of the Sobolev inequality according to~\cite{MR1164616}. See~\cite{https://doi.org/10.48550/arxiv.2209.08651,https://doi.org/10.48550/arxiv.2302.03926} for recent developments and further references. We refer to~\cite{ane2000inegalites,Guionnet2002,MR2352327,MR3155209} to reference books on the topic. In a classical result on stability in functional inequalities, Bianchi and Egnell proved in~\cite{MR1124290} that the deficit in the Sobolev inequality measures the $\dot{\mathrm H}^1(\R^d,dx)$ distance to the manifold of the Aubin-Talenti functions. The estimate has been made constructive in~\cite{https://doi.org/10.48550/arxiv.2209.08651} where a new $\mathrm L^2(\R^d,dx)$ stability result for the logarithmic Sobolev inequality is also established (also see~\cite{https://doi.org/10.48550/arxiv.2303.05604} for furrther results in strong norms). Still in the Euclidean setting a first stability result in strong norms for the logarithmic Sobolev inequality appears in~\cite{MR3567822}, where the authors give deficit estimates in various distances for functions inducing a Poin\-car\'e inequality. Under the condition $\nrmG{x\,u}2=\sqrt d$, a stability result measured by an entropy is given in~\cite{Dolbeault_2015}. For sequential stability results in strong norms, we refer to~\cite{MR4305006} when assuming a bound on~$u$ in $\mathrm L^4(\R^d,d\gamma)$ and to~\cite{https://doi.org/10.48550/arxiv.2303.05604} when assuming a bound on $|x|^2\,u$ in $\mathrm L^2(\R^d,d\gamma)$. Stability according to other notions of distance has been studied in~\cite{MR3665794,MR3320893,MR3666798}.

To our knowledge, the first result of stability for the logarithmic Sobolev inequality is a reinforcement of the inequality due to E.~Carlen in~\cite{MR1132315} where he introduces an additional term involving the Wiener transform. Stability in logarithmic Sobolev inequality is related to deficit in Gaussian isoperimetry and we refer to~\cite{MR3269872} for an introduction to early results in this direction,~\cite{MR3630285} for a sharp, dimension-free quantitative Gaussian isoperimetric inequality, and~\cite{MR4116725} for recent results and further references. Results of Proposition~\ref{Prop1} are known from~\cite[Theorem~1.1]{MR3269872}
where it is deduced from the HWI inequality due to F.~Otto and C.~Villani~\cite{MR1760620}. Such estimates have even been refined in~\cite{MR4116725}. There are several other proofs. In~\cite{MR3567822}, M.~Fathi, E.~Indrei and M.~Ledoux use a Mehler formula for the Ornstein-Uhlenbeck semigroup and Poincar\'e inequalities. The proof in~\cite{Dolbeault_2015} is based on simple scaling properties of the Euclidean form of the logarithmic Sobolev inequality, which also apply to Gagliardo-Nirenberg inequalities. Various stability results have been proved in Wasserstein's distance: we refer to~\cite{MR3271181,MR3269872,MR3567822,MR4305006,MR4455233,bolley2018dimensional,MR4116725,https://doi.org/10.48550/arxiv.2303.05604}. A key argument for Theorem~\ref{ThmMain1} is the fact that the heat flow preserves log-concavity according, \emph{e.g.},~\cite{saumard2014log}, which is a pretty natural property in this framework: see for instance~\cite{courtade2018quantitative}.

In this paper, we carefully distinguish stability results of type~\eqref{SuperStab} where stability is measured w.r.t.~$\mathscr M$, and of type~\eqref{stab} where the distance to a given function is estimated. Even if this function is normalized and centered, this is not enough as shown in~\cite{https://doi.org/10.48550/arxiv.2303.05604}. Many counter-examples to stability are known, involving Wasserstein's distance for instance in~\cite{MR3271181,MR3269872,MR3567822,MR4305006,MR4455233}, weaker distances like $p$-Wasserstein, or stronger norms like $\mathrm L^p$ or $\mathrm H^1$: see for instance~\cite{MR4305006,https://doi.org/10.48550/arxiv.2303.05604}. The main counter-examples which we might try to apply to our setting are~\cite[Theorem~1.3]{MR4455233} and~\cite[Theorem~4]{MR4116725} but, as already noted in~\cite{MR4305006}, they are based on the fact that the second moment diverges along a sequence of test functions, which is forbidden in our assumptions.

\medskip This paper is organized as follows. Section~\ref{Sec:Entropy} is devoted to the standard \emph{carr\'e du champ} method and a proof of Proposition~\ref{Prop1}. Theorem~\ref{ThmMain1} is proved in Section~\ref{Sec:Stab:log-concave}, under a log-concavity assumption. Using properties of the heat flow, the method is extended to the larger class of functions of Theorem~\ref{ThmMain2} in Section~\ref{Sec:Extension}.


%%%%%%%%%%%%%%%%%%%%%%%%%%%%%%%%%%%%%%%%%%%%%%%%%%%%%%%%%%%%%%%%%%%%%%%%%%
%%%%%%%%%%%%%%%%%%%%%%%%%%%%%%%%%%%%%%%%%%%%%%%%%%%%%%%%%%%%%%%%%%%%%%%%%%
\section{Entropy methods and entropy -- entropy production stability estimates}\label{Sec:Entropy}

This section is devoted to the proof of Proposition~\ref{Prop1}.

%%%%%%%%%%%%%%%%%%%%%%%%%%%%%%%%%%%%%%%%%%%%%%%%%%%%%%%%%%%%%%%%%%%%%%%%%%
\subsection{Definitions and preliminary results}\label{Sec:EntropyPrelim}

Consider the \emph{Ornstein-Uhlenbeck equation} on $\mathbb R^d$
\be{OU}
\frac{\partial h}{\partial t}=\L h\,,\quad h(t=0,\cdot)=h_0\,,\quad(t,x)\in\R^+\times\R^d\,,
\ee
where $\L h:=\Delta h-x\cdot\nabla h$ denotes the \emph{Ornstein-Uhlenbeck} operator.

Let us recall some classical results. If $h_0\in\mathrm{L}^1(\R^d,d\gamma)$ is nonnegative, then there exists a unique nonnegative weak solution to~\eqref{OU} (see for instance~\cite{MR2597943}). The two key properties of the Ornstein-Uhlenbeck operator are
\[
\irdg{v_1\,(\L v_2)}=-\,\irdg{\nabla v_1\cdot\nabla v_2}\quad\mbox{and}\quad[\nabla,\L]\,v=-\,\nabla v\,.
\]
As a consequence, we obtain the two identities
\be{Id1}
\irdg{(\L v)^2}=\irdg{\|\mathrm{Hess}\,v\|^2}+\irdg{|\nabla v|^2}
\ee
and
\be{Id2}
\irdg{\L v\,\frac{|\nabla v|^2}v}=-\,2\irdg{\mathrm{Hess}\,v:\frac{\nabla v\otimes\nabla v}v}+\irdg{\frac{|\nabla v|^4}{v^2}}\,,
\ee
where $\mathrm{Hess}\,v=\nabla\otimes\nabla v$ is the \emph{Hessian matrix} of $v$. Here we use the following notations. If $a$ and $b$ take values in $\R^d$, $a\otimes b$ denotes the matrix $(a_i\,b_j)_{1\le i,j\le d}$. With matrix valued $m$ and $n$, we define $m:n=\sum_{i,j=1}^dm_{i,j}\,n_{i,j}$ and $\|m\|^2=m:m$. If $h$ is a nonnegative solution of~\eqref{OU}, notice that $v=\sqrt h$ solves
\be{OUv}
\frac{\partial v}{\partial t}=\L v+\frac{|\nabla v|^2}v\,.
\ee
Let us fix $\nrm v2=1$, then the \emph{entropy} and the \emph{Fisher information}, respectively defined by
\[
\E[v]:=\irdg{|v|^2\,\log|v|^2}\quad\mbox{and}\quad\I[v]:=\irdg{|\nabla v|^2}\,,
\]
evolve along the flow according to
\[
\frac d{dt}\E[v]=-\,4\,\I[v]\quad\mbox{and}\quad\frac d{dt}\I[v]=-\,2\irdg{\((\L v)^2+\L v\,\frac{|\nabla v|^2}v\)}
\]
if $v$ solves~\eqref{OUv}. Using~\eqref{Id1} and~\eqref{Id2}, we obtain the classical expression of the \emph{carr\'e du champ} method
\be{cdc}
\frac d{dt}\I[v]+2\,\I[v]=-\,2\irdg{\left\|\mathrm{Hess}\,v-\frac{\nabla v\otimes\nabla v}v\right\|^2}
\ee
as for instance in~\cite{MR1842428,MR2435196,MR3155209}. By writing that
\[
\frac d{dt}\(\I[v]-\frac12\,\E[v]\)\le0\quad\mbox{and}\quad\lim_{t\to+\infty}\(\I[v(t,\cdot)]-\frac12\,\E[v(t,\cdot)]\)=0\,,
\]
we recover the standard proof of the \emph{entropy -- entropy production} inequality
\be{LSI-BE}
\I[v]-\frac12\,\E[v]\ge0\,,
\ee
\emph{i.e.}, of~\eqref{LSIg} by the method of~\cite{bakry1985diffusions}.

Several of the above expression can be rephrased in terms of the \emph{pressure variable}
\[
P:=-\,\log h=-\,2\,\log v
\]
using the following elementary identities
\begin{align*}
&\nabla v=-\,\frac12\,e^{-P/2}\,\nabla P\,,\quad\frac{\nabla v\otimes\nabla v}v=\frac14\,e^{-P/2}\,\nabla P\otimes\nabla P\,,\\
&\mathrm{Hess}\,v=-\,\frac12\,e^{-P/2}\,\mathrm{Hess}\,P+\frac14\,e^{-P/2}\,\nabla P\otimes\nabla P\,,
\end{align*}
so that, by taking into account $v\,\nabla P=-\,2\,\nabla v$ and $h=v^2$, we have
\[
\I[v]=\frac14\irdg{|\nabla P|^2\,h}\quad\mbox{and}\;\irdg{\left\|\mathrm{Hess}\,v-\frac{\nabla v\otimes\nabla v}v\right\|^2}=\frac14\irdg{\|\mathrm{Hess}\,P\|^2\,h}\,.
\]

%%%%%%%%%%%%%%%%%%%%%%%%%%%%%%%%%%%%%%%%%%%%%%%%%%%%%%%%%%%%%%%%%%%%%%%%%%
\subsection{Improvements under moment constraints}

In standard computations based on the \emph{carr\'e du champ} method, one usually drops the right-hand side in~\eqref{cdc} which results in the standard exponential decay of $\I[v(t,\cdot)]$ if~$v$ solves~\eqref{OUv} and, after integration on $t\in\R^+$, proves~\eqref{LSIg}. Keeping track of the right-hand side in~\eqref{cdc} provides us with improvements as shown in~\cite{MR2152502,Demange-PhD,MR3103175} in various interpolation inequalities but generically fails in the case of the logarithmic Sobolev inequality. We remedy to this issue by introducing moment constraints. This is not a very difficult result but, as far as we know, it is new in the framework of the \emph{carr\'e du champ} method.
%-------------------------------------------------------------------------
\begin{lemma}\label{lem1} With the notations of Section~\ref{Sec:EntropyPrelim}, if $v\in\mathrm H^2(\R^d,d\gamma)$ is a positive function such that $\irdg{|x|^2\,|v|^2}\le d$, then
\[
4\,\I[v]\le\irdg{(\Delta P)\,h}\le\sqrt{d\irdg{\|\mathrm{Hess}\,P\|^2\,h}}\,.
\]
\end{lemma}
%-------------------------------------------------------------------------
\begin{proof} Using $h\,\nabla P = -\,\nabla h$, we obtain
\[
4\,\I[v]=\irdg{|\nabla P|^2\,h}=-\irdg{\nabla P\cdot\nabla h}=\irdg{h\,(\L P)}\,.
\]
After recalling that $\L P=\Delta P-x\cdot\nabla P$, using an integration by parts we deduce that
\[
-\irdg{h\,x\cdot\nabla P}=\irdg{x\cdot\nabla h}=\irdg{h\(|x|^2-d\)}=\irdg{|v|^2\(|x|^2-d\)}\le0
\]
which proves the first inequality. The second inequality follows from a Cauchy-Schwarz inequality and the arithmetic-geometric inequality
\[
(\Delta P)^2\le d\,\|\mathrm{Hess}\,P\|^2\,.
\]\end{proof}

\begin{proof}[Proof of Proposition~\ref{Prop1}] Let $h=v^2$ be the solution of~\eqref{OU} with initial datum $h_0=u^2$. Since $x\mapsto\big(|x|^2-d\big)$ is an eigenfunction of $\L$ with corresponding eigenvalue $-\,2$ and $\L$ is self-adjoint on $\mathrm L^2(\R^d,d\gamma)$, we have
\begin{multline}\label{Moment}
\frac d{dt}\irdg{\big(|x|^2-d\big)\,h}=\irdg{\big(|x|^2-d\big)\,(\L h)}\\
=\irdg{h\,\L\big(|x|^2-d\big)}=-\,2\irdg{\big(|x|^2-d\big)\,h}\,.
\end{multline}
The sign of $t\mapsto\irdg{\big(|x|^2-d\big)\,h(t,x)}$ is conserved and in particular we have that $\irdg{|x|^2\,|v|^2}\le d$ for any $t\ge0$. For any $i=1$, $2$\ldots $d$, we also notice that $x\mapsto x_i$ is also an eigenfunction of $\L$ with corresponding eigenvalue $-\,1$ so that
\[
\frac d{dt}\irdg{x\,h}=-\irdg{x\,h}
\]
and, as a consequence $\irdg{x\,h(t,\cdot)}=0$ for all $t\ge0$ because $\irdg{x\,h_0}=0$.

For smooth enough solutions, we deduce from Lemma~\ref{lem1},~\eqref{cdc} and~\eqref{LSI-BE} that
\[
\frac d{dt}\I[v]+2\,\I[v]\le-\,\frac8d\,\I^2[v]\le\frac1{2\,d}\,\frac d{dt}\(\E[v]\)^2
\]
and obtain by considering the limit as $t\to+\infty$ that
\[
\I[v]\ge\frac12\,\E[v]+\frac1{2\,d}\(\E[v]\)^2\,.
\]
This provides us with~\eqref{stabsq1}. In the general case, one can get rid of the $\mathrm H^2(\R^d,d\gamma)$ regularity of Lemma~\ref{lem1} by a standard approximation scheme, which is classical and will not be detailed here.

As in~\cite{https://doi.org/10.48550/arxiv.2211.13180}, a better estimate is achieved as follows. Let
\[
\phi(s):=\frac d4\(e^{\frac2d\,s}-1\)\quad\forall\,s\ge0\,.
\]
Using $\frac d{dt}\E[v]=-\,4\,\I[v]$, we notice that
\[
\frac d{dt}\Big(\I[v]-\phi\big(\E[v]\big)\Big)=-\,\frac8d\Big(\I[v]-\phi\big(\E[v]\big)\Big)\,.
\]
Since $\lim_{t\to+\infty}\I[v(t,\cdot)]=0$ as can be deduced from a Gronwall estimate relying on $\frac d{dt}\I[v]\le-\,2\,\I[v]$ and $\lim_{t\to+\infty}\E[v(t,\cdot)]=0$ as a consequence of~\eqref{LSIg}, one knows that
\[
\lim_{t\to+\infty}\Big(\I[v(t,\cdot)]-\phi\big(\E[v(t,\cdot)]\big)\Big)=0\,.
\]
Moreover, Gronwall estimates show that $\I[v(t,\cdot)]-\phi\big(\E[v(t,\cdot)]$ cannot change sign and an asymptotic expansion as $t\to+\infty$ as in~\cite[Appendix~B.4]{https://doi.org/10.48550/arxiv.2211.13180} is enough to obtain that $\I[v(t,\cdot)]-\phi\big(\E[v(t,\cdot)]$ takes nonnegative values for $t>0$ large enough. Altogether, we conclude that
\[
\I[v(t,\cdot)]-\phi\Big(\E[v(t,\cdot)]\Big)\ge0
\]
for any $t\ge0$ and, as a particular case, at $t=0$ for $v(0,\cdot)=u$. The function $\phi$ is convex increasing and, as such, invertible, so that we can also write
\[
\varphi\big(\I[u]\big)^{-1}-\phi\Big(\E[u]\Big)\ge0\,.
\]
his completes the proof of~\eqref{stabsq2} with the convex monotone increasing function
\[
\psi(s):=s-\frac12\,\phi^{-1}(s)\,.
\]
\end{proof}


%%%%%%%%%%%%%%%%%%%%%%%%%%%%%%%%%%%%%%%%%%%%%%%%%%%%%%%%%%%%%%%%%%%%%%%%%%
%%%%%%%%%%%%%%%%%%%%%%%%%%%%%%%%%%%%%%%%%%%%%%%%%%%%%%%%%%%%%%%%%%%%%%%%%%
\section{Stability results}\label{Sec:Stability}

%%%%%%%%%%%%%%%%%%%%%%%%%%%%%%%%%%%%%%%%%%%%%%%%%%%%%%%%%%%%%%%%%%%%%%%%%%
\subsection{Log-concave measures and Poincar\'e inequality}

According to~\cite{MR1374200}, given a Borel probability measure $\mu$ on $\mathbb R^d$, its \emph{isoperimetric constant} is defined as
\[
h(\mu):=\inf_A\frac{\mathrm P_\mu(A)}{\min\big\{\mu(A),1-\mu(A)\big\}}
\]
where the infimum is taken on the set of arbitrary Borel subset $\mathbb R^d$ with $\mu$-perimeter or surface measure $\mathrm P_\mu(A):=\lim_{\varepsilon\to0_+}\big(\mu(A_\varepsilon)-\mu(A)\big)/\varepsilon$ and $A_\varepsilon:=\{x\in\mathbb R^d\,:\,|x-a|<\varepsilon\;\mbox{for some}\;a\in A\}$.  Here and in what follows, we shall say that a measure $\mu$ with density $e^{-\psi}$ with respect to Lebesgue's measure is a \emph{log-concave probability measure} if $\psi$ is a convex function, and denote by $\lambda_1(\mu)$ the first positive eigenvalue of $-\,\L_\psi$ where $\L_\psi$ is the \emph{Ornstein-Uhlenbeck operator} $\L_\psi:=\Delta-\nabla\psi\cdot\nabla$. In that case, we learn from~\cite[Ineq.~(5.8)]{MR2195409} that
\[
\frac14\,h(\mu)^2\le\lambda_1(\mu)\le36\,h(\mu)^2
\]
where the lower bound is J.~Cheeger's inequality that goes back to~\cite{MR0402831} for Riemannian manifolds and also to earlier works by V.G.~Maz'ya~\cite{maz1962negative,maz1962solvability}. This bound was later improved in~\cite{MR683635,MR1186991}. The characterization of $h(\mu)$ has been actively studied, but it is out of the scope of the present paper. We learn from~\cite[Theorem~1.2]{MR1742893} and~\cite[Ineq.~(3.4)]{MR1742893} that
\[
h(\mu)\ge\frac1{6\,\sqrt{3\int_{\mathbb R^d}|x-x_\mu|^2\,d\mu}}\quad\mbox{where}\quad x_\mu=\int_{\mathbb R^d}x\,d\mu
\]
for any log-concave probability measure $\mu$. This estimate is closely related with the results by R.~Kannan, L.~Lov\'asz and M.~Simonovits in~\cite{MR1318794} and their conjecture, which again lies out of the scope of the present paper (see for instance~\cite{MR3500299} for a recent work on the topic).

Altogether, if $\mu$ is a log-concave probability measure with $d\mu=e^{-\psi}\,dx$ such that $\int_{\mathbb R^d}|x|^2\,d\mu\le d$, then we have the \emph{Poincar\'e inequality}
\be{Poincare1}
\int_{\mathbb R^d}|\nabla f|^2\,d\mu\ge\frac1{432}\int_{\mathbb R^d}|f|^2\,d\mu\quad\forall\,f\in\mathrm H^1(\mathbb R^d,d\mu)\;\mbox{such that}\;\int_{\mathbb R^d}f\,d\mu=0\,.
\ee
We refer to~\cite{cattiaux2021functional} and references therein for further estimates on $\lambda_1(\mu)$.

%%%%%%%%%%%%%%%%%%%%%%%%%%%%%%%%%%%%%%%%%%%%%%%%%%%%%%%%%%%%%%%%%%%%%%%%%%
\subsection{Time evolution, log-concave densities and Poincar\'e inequality}

%-------------------------------------------------------------------------
\begin{lemma}\label{lemma5} Let us consider consider a solution $h$ of~\eqref{OU} with initial datum $h_0=v^2$ and assume that $\mu_0:=h_0\,\gamma$ is log-concave. Then $\mu_t:=h(t,\cdot)\,\gamma$ is log-concave for all $t\ge 0$. \end{lemma}
%-------------------------------------------------------------------------
\begin{proof} The function $g:=h\,\gamma$ solves the Fokker-Planck equation
\[
\frac{\partial g}{\partial t}=\Delta g+\nabla\cdot(x\,g)\,.
\]
The function $f$ such that
\[
f(t,x):=g\(\frac12\,\log(1+2\,t),\frac x{\sqrt{1+2\,t}}\)\quad\forall\,(t,x)\in\R^+\times\R^d
\]
solves the heat equation and can be represented using the heat kernel. According for instance to~\cite{saumard2014log,BARDET_2018}, log-concavity is preserved under convolution, which completes the proof.
\end{proof}

The log-concavity property becomes true under the action of the flow of~\eqref{OU} after some delay~$t_\star$ for large classes of initial data. With the notation of Lemma~\ref{lemma5}, for any $R>0$, we read from~\cite[Theorem~5.1]{lee2003geometrical} by K.~Lee and J-L.~V\'azquez that $\mu_t$ is log-concave for any
\be{tstarR}
t\ge t_\star(R):=\log\(\sqrt{R^2+1}\)\,,
\ee
if $v$ is compactly supported in a ball of radius $R>0$, by reducing the problem to the heat flow as in the above proof. As a consequence, we know that~\eqref{Poincare1} holds for any $t\ge t_\star(R)$.

Alternatively, under Assumption~\eqref{gautail}, we learn from a recent paper~\cite[Theorem~2]{chen2021dimension} by H.-B.~Chen, S.~Chewi, and J.~Niles-Weed that the \emph{Poincar\'e inequality}
\be{Poincare2}
\int_{\mathbb R^d}|\nabla f|^2\,d\mu_t\ge\lambda_1(\mu_t)\int_{\mathbb R^d}|f|^2\,d\mu_t\quad\forall\,f\in\mathrm H^1(\mathbb R^d,d\mu_t)\;\mbox{such that}\;\int_{\mathbb R^d}f\,d\mu_t=0
\ee
holds for all $t\ge t_\star^\varepsilon$ with
\[
t_\star^\varepsilon:=\log\big(\sqrt{1+\varepsilon^{-1}}\big)\,,\quad\frac1{\lambda_1(\mu_t)}\le\tau\(\frac{\varepsilon\,\tau}{\varepsilon\,\tau-1}+A^\frac1{\varepsilon\,\tau-1}\)\quad\mbox{and}\quad\tau=\frac12\(e^{2t}-1\)\,.
\]

%%%%%%%%%%%%%%%%%%%%%%%%%%%%%%%%%%%%%%%%%%%%%%%%%%%%%%%%%%%%%%%%%%%%%%%%%%
\subsection{Explicit stability results for log-concave densities}\label{Sec:Stab:log-concave}

Let us start by an elementary observation.
%-------------------------------------------------------------------------
\begin{lemma}\label{Lem:P} If $h\in\mathrm H^1(\R^d,d\gamma)$ is such that $\irdg{x\,h}=0$ and $P=-\,\log h$ is the \emph{pressure variable}, then
\[
\irdg{\nabla P\,h} =0\,.
\]
\end{lemma}
%-------------------------------------------------------------------------
\begin{proof} The result follows from $\irdg{\nabla P\,h}=-\irdg{\nabla h}=\irdg{x\,h}=0$.\end{proof}
\noindent With this result in hand, we can now prove our first main result.
\begin{proof}[Proof of Theorem~\ref{ThmMain1}] The function $h=v^2$ is such that $\irdg{x\,h}=0$ and Lemma~\ref{Lem:P} applies. Since $h\,\gamma$ is log-concave, we can apply~\eqref{Poincare1} with $f=\partial P/\partial x_i$ for any $i=1$, $2$,\ldots $d$ and obtain
\[
\irdg{\|\mathrm{Hess}\,P\|^2\,h}\ge\frac1{432}\irdg{|\nabla P|^2\,h}\,.
\]
It follows from~\eqref{cdc} that
\[
\frac d{dt}\irdg{|\nabla v|^2}+2\irdg{|\nabla v|^2}\le-\,\frac1{864}\irdg{|\nabla v|^2}\,,
\]
and the stability result is obtained as in the proof of Proposition~\ref{Prop1}.\end{proof}

%%%%%%%%%%%%%%%%%%%%%%%%%%%%%%%%%%%%%%%%%%%%%%%%%%%%%%%%%%%%%%%%%%%%%%%%%%
\subsection{Extension by entropy methods and flows}\label{Sec:Extension}

This section is devoted to the proof of Theorem~\ref{ThmMain2}. The key idea is to evolve the function by the Ornstein-Uhlenbeck equation, so that the solution after an \emph{initial time layer} has the log-concavity property of Theorem~\ref{ThmMain1}, at least if the initial datum has compact support. To some extent, the strategy is similar to the one used in~\cite{bonforte2020stability}. During the initial time layer, we use an improved version of the entropy -- entropy production inequality which arises as a consequence of the \emph{carr\'e du champ} method.

\begin{proof}[Proof of Theorem~\ref{ThmMain2}]  Let $h$ be the solution to~\eqref{OU} with initial datum $h_0=u^2$ and define
\[
\mathscr Q(t):=\frac{\I\left[\sqrt{h(t,\cdot)}\right]}{\E\left[\sqrt{h(t,\cdot)}\right]}\quad\forall\,t\ge0\,.
\]
Let us assume first that $v$ has compact support. With no loss of generality, we can assume that $v$ is supported in $B(0,R)$ for some $R>0$. With $t_\star=t_\star(R)$ given by~\eqref{tstarR}, we know from~\cite[Theorem~5.1]{lee2003geometrical} that $\mu_t$ is log-concave at $t=t_\star$ and Theorem~\ref{ThmMain1} applies:
\[
\mathscr Q(t_\star)\ge\frac{\mathscr C_\star}2\,.
\]
With an estimate similar to~\cite[Lemma~2.9]{bonforte2020stability}, we learn from Section~\ref{Sec:Entropy} that
\be{DiffIneq}
\frac{d\mathscr Q}{dt}\le2\,\mathscr Q\,(2\,\mathscr Q-1)\,.
\ee
An integration on $(0,t_\star)$ shows that
\[
\mathscr Q(0)\ge\frac12\(1+\frac{2\,\mathscr Q(t_\star)-1}{1+2\,\mathscr Q(t_\star)\(e^{2\,t_\star}-1\)}\)\ge\frac12\(1+\frac{\mathscr C_\star-1}{1+\mathscr C_\star\,R^2}\)=\frac{\mathscr C}2\,.
\]

Under the more general assumption~\eqref{gautail}, we rely on~\eqref{Poincare2} and obtain with same notations as above and $t_\star=t_\star^\varepsilon$ that
\[
\irdg{\|\mathrm{Hess}\,P\|^2\,h(t,\cdot)}\ge\lambda_1(\mu_t)\irdg{|\nabla P|^2\,h(t,\cdot)}\quad\forall\,t\ge t_\star\,.
\]
Moreover, for some explicit $t_0=t_0(\varepsilon)>t_*$, we notice that $t\mapsto\lambda_1(\mu_t)$ is nonincreasing on $(t_0,+\infty)$. Hence we deduce from
\[
\I[v(t_0,\cdot)]-\frac18\int_{t_0}^{+\infty}\big(4+\lambda_1(\mu_s)\big)\,\E'[v(s,\cdot)]\,ds\ge0
\]
after an integration by parts that
\[
\mathscr Q(t_0)\ge\frac12\(1+\frac14\,\lambda_1(\mu_{t_0})\)=:\mathscr C_0\,.
\]
Using~\eqref{DiffIneq}, we obtain
\[
\mathscr C=1+\frac{\mathscr C_0-1}{1+\mathscr C_0\(e^{2\,t_0}-1\)}\,.
\]
This concludes the proof.\end{proof}

%%%%%%%%%%%%%%%%%%%%%%%%%%%%%%%%%%%%%%%%%%%%%%%%%%%%%%%%%%%%%%%%%%%%%%%%%%
\subsection{Normalization issues}\label{Sec:Normalization}

If we do not assume that $\nrmG u2=1$ and $\nrmG{x\,u}2\le d$, it is still possible to state the analogue of Theorem~\ref{ThmMain1}, but the price to be paid is a dependence on
\[
\kappa[u]:=\frac{\nrmG u2}{\max\left\{\sqrt d,\nrmG{(x-x_0)\,u}2\right\}}\quad\mbox{where}\quad x_0=\irdg{x\,h_0}\,,
\]
which goes as follows.
%-------------------------------------------------------------------------
\begin{proposition}\label{ThmMain1bis} For all $u\in\mathrm H^1\big(\R^d,(1+|x|^2)\,d\gamma\big)$ such that $\irdg{x\,u^2}=0$, and $u^2 \, \gamma$ is log-concave, we have
\[
\nrmG{\nabla u}2^2-\frac12\,\big(1+(\mathscr C_\star-1)\,\kappa[u]\big)\irdg{|u|^2\,\log\(\frac{|u|^2}{\nrmG u2^2}\)}\ge0\,.
\]
\end{proposition}
%-------------------------------------------------------------------------
\begin{proof} We learn from~\eqref{Moment} that
\[
\irdg{|x|^2\,h(t,x)}=d\,\nrmG u2^2+e^{-2t}\irdg{\big(|x|^2-d\big)\,h_0}\quad\forall\,t\ge0\,.
\]
Hence~\cite[Theorem~1.2]{MR1742893} and~\cite[Ineq.~(3.4)]{MR1742893} apply with
\[
h(\mu)\ge\frac{\kappa[u]}{6\,\sqrt3}\,.
\]
and the remainder of the proof of Theorem~\ref{ThmMain1} is unchanged.\end{proof}

A similar extension of Theorem~\ref{ThmMain2} can be done on the same basis. Details are left to the reader. As for Proposition~\ref{Prop1}, we can make the following observations.  The case $\irdg{|x|^2\,|v|^2}\le d$ is already covered in Lemma~\ref{lem1}. If
\[
A:=\irdg{|u|^2\(|x|^2-d\)}
\]
is positive, let us consider the solution $h$ of~\eqref{OU} with initial datum $h_0=u^2$. We know from~\eqref{Moment} that
\[
\irdg{h(t,x)\(|x|^2-d\)}=A\,e^{-2t}
\]
and deduce as in the proof of Lemma~\ref{lem1} that
\[
4\,\I[v]=\irdg{|\nabla P|^2\,h}=\irdg{h\,(\L P)}+A\,e^{-2t}\le\sqrt{d\irdg{\|\mathrm{Hess}\,P\|^2\,h}}+A\,e^{-2t}
\]
with $P=-\log h$. Hence by~\eqref{cdc}, we learn that
\[
\frac d{dt}\I[v]+2\,\I[v]=-\,\frac12\irdg{\|\mathrm{Hess}\,P\|^2\,h}\le-\,\frac1{2\,d}\(4\,\I[v]-A\,e^{-2t}\)^2
\]
and this estimate can be rephrased with $z(t):=e^{2\,t}\,\I[v(t,\cdot)]$ as
\[
z'\le-\,\frac{e^{-\,2\,t}}{2\,d}\,(4\,z-A)^2\,.
\]
Knowing that $z'<0$ is an improvement on the decay rate $\I[v(t,\cdot)]\le\I[u]\,e^{-\,2\,t}$ can be rephrased as an improved entropy -- entropy production inequality for $A>0$.


%%%%%%%%%%%%%%%%%%%%%%%%%%%%%%%%%%%%%%%%%%%%%%%%%%%%%%%%%%%%%%%%%%%%%%%%%%
%%%%%%%%%%%%%%%%%%%%%%%%%%%%%%%%%%%%%%%%%%%%%%%%%%%%%%%%%%%%%%%%%%%%%%%%%%
\section*{Acknowledgements} The authors thank Max Fathi and Pierre Cardaliaguet for fruitful discussions and Emanuel Indrei for stimulating interactions. G.B.~has been funded by the European Union’s Horizon 2020 research and innovation program under the Marie Sklodowska-Curie grant agreement No 754362.\\
\noindent{\scriptsize\copyright\,2023 by the authors. This paper may be reproduced, in its entirety, for non-commercial purposes.}
%%%%%%%%%%%%%%%%%%%%%%%%%%%%%%%%%%%%%%%%%%%%%%%%%%%%%%%%%%%%%%%%%%%%%%%%%%
\small
\begin{thebibliography}{10}

\bibitem{ane2000inegalites}
{\sc C.~An{\'e}, S.~Blach{\`e}re, D.~Chafa{\"\i}, P.~Foug{\`e}res, I.~Gentil,
  F.~Malrieu, C.~Roberto, and G.~Scheffer}, {\em Sur les in{\'e}galit{\'e}s de
  {S}obolev logarithmiques}, vol.~10, Soci{\'e}t{\'e} math{\'e}matique de
  France Paris, 2000.

\bibitem{MR2152502}
{\sc A.~Arnold and J.~Dolbeault}, {\em Refined convex {S}obolev inequalities},
  J. Funct. Anal., 225 (2005), pp.~337--351.

\bibitem{MR1842428}
{\sc A.~Arnold, P.~Markowich, G.~Toscani, and A.~Unterreiter}, {\em On convex
  {S}obolev inequalities and the rate of convergence to equilibrium for
  {F}okker-{P}lanck type equations}, Comm. Partial Differential Equations, 26
  (2001), pp.~43--100.

\bibitem{bakry1985diffusions}
{\sc D.~Bakry and M.~{\'E}mery}, {\em Diffusions hypercontractives}, in
  S{\'e}minaire de Probabilit{\'e}s XIX 1983/84, Springer, 1985, pp.~177--206.

\bibitem{MR3155209}
{\sc D.~Bakry, I.~Gentil, and M.~Ledoux}, {\em Analysis and geometry of
  {M}arkov diffusion operators}, vol.~348 of Grundlehren der Mathematischen
  Wissenschaften [Fundamental Principles of Mathematical Sciences], Springer,
  Cham, 2014.

\bibitem{MR1374200}
{\sc D.~Bakry and M.~Ledoux}, {\em L\'{e}vy-{G}romov's isoperimetric inequality
  for an infinite-dimensional diffusion generator}, Invent. Math., 123 (1996),
  pp.~259--281.

\bibitem{MR3630285}
{\sc M.~Barchiesi, A.~Brancolini, and V.~Julin}, {\em Sharp dimension free
  quantitative estimates for the {G}aussian isoperimetric inequality}, Ann.
  Probab., 45 (2017), pp.~668--697.

\bibitem{BARDET_2018}
{\sc J.-B. Bardet, N.~Gozlan, F.~Malrieu, and P.-A. Zitt}, {\em Functional
  inequalities for {G}aussian convolutions of compactly supported measures:
  Explicit bounds and dimension dependence}, Bernoulli, 24 (2018).

\bibitem{MR1164616}
{\sc W.~Beckner}, {\em {S}obolev inequalities, the {P}oisson semigroup, and
  analysis on the sphere {$\mathbb S^n$}}, Proc. Nat. Acad. Sci. U.S.A., 89
  (1992), pp.~4816--4819.

\bibitem{MR1124290}
{\sc G.~Bianchi and H.~Egnell}, {\em A note on the {S}obolev inequality}, J.
  Funct. Anal., 100 (1991), pp.~18--24.

\bibitem{blachman1965convolution}
{\sc N.~Blachman}, {\em The convolution inequality for entropy powers}, IEEE
  Transactions on Information theory, 11 (1965), pp.~267--271.

\bibitem{MR1742893}
{\sc S.~G. Bobkov}, {\em Isoperimetric and analytic inequalities for
  log-concave probability measures}, Ann. Probab., 27 (1999), pp.~1903--1921.

\bibitem{MR3500299}
{\sc S.~G. Bobkov and D.~Cordero-Erausquin}, {\em K{LS}-type isoperimetric
  bounds for log-concave probability measures}, Ann. Mat. Pura Appl. (4), 195
  (2016), pp.~681--695.

\bibitem{MR3269872}
{\sc S.~G. Bobkov, N.~Gozlan, C.~Roberto, and P.-M. Samson}, {\em Bounds on the
  deficit in the logarithmic {S}obolev inequality}, J. Funct. Anal., 267
  (2014), pp.~4110--4138.

\bibitem{bolley2018dimensional}
{\sc F.~Bolley, I.~Gentil, and A.~Guillin}, {\em Dimensional improvements of
  the logarithmic {S}obolev, {T}alagrand and {B}rascamp--{L}ieb inequalities},
  The Annals of Probability, 46 (2018), pp.~261--301.

\bibitem{bonforte2020stability}
{\sc M.~Bonforte, J.~Dolbeault, B.~Nazaret, and N.~Simonov}, {\em Stability in
  {G}agliardo-{N}irenberg-{S}obolev inequalities: Flows, regularity and the
  entropy method}, Preprint \href{http://arxiv.org/abs/2007.03674}{arXiv:
  2007.03674} and
  \href{https://hal.archives-ouvertes.fr/hal-02887010}{hal-02887010}, to appear
  in \emph{Memoirs of the AMS},  (2023).

\bibitem{https://doi.org/10.48550/arxiv.2211.13180}
{\sc G.~Brigati, J.~Dolbeault, and N.~Simonov}, {\em Logarithmic {S}obolev and
  interpolation inequalities on the sphere: constructive stability results},
  Preprint \href{http://arxiv.org/abs/2211.13180}{arXiv:2211.13180} and
  \href{https://hal.archives-ouvertes.fr/hal-03868496}{hal-03868496},  (2022).

\bibitem{https://doi.org/10.48550/arxiv.2302.03926}
\leavevmode\vrule height 2pt depth -1.6pt width 23pt, {\em On {G}aussian
  interpolation inequalities}, Preprint
  \href{http://arxiv.org/abs/2302.03926}{arXiv: 2302.03926} and
  \href{https://hal.archives-ouvertes.fr/hal-03978122}{hal-03978122}, to appear
  in C.R.~Mathématique,  (2023).

\bibitem{MR683635}
{\sc P.~Buser}, {\em A note on the isoperimetric constant}, Ann. Sci. \'{E}cole
  Norm. Sup. (4), 15 (1982), pp.~213--230.

\bibitem{MR1132315}
{\sc E.~A. Carlen}, {\em Superadditivity of {F}isher's information and
  logarithmic {S}obolev inequalities}, J. Funct. Anal., 101 (1991),
  pp.~194--211.

\bibitem{cattiaux2021functional}
{\sc P.~Cattiaux and A.~Guillin}, {\em Functional inequalities for perturbed
  measures with applications to log-concave measures and to some {B}ayesian
  problems}, Bernoulli, 28 (2022).

\bibitem{MR0402831}
{\sc J.~Cheeger}, {\em A lower bound for the smallest eigenvalue of the
  {L}aplacian}, in Problems in analysis ({S}ympos. in honor of {S}alomon
  {B}ochner, {P}rinceton {U}niv., {P}rinceton, {N}.{J}., 1969), Princeton Univ.
  Press, Princeton, N.J., 1970, pp.~195--199.

\bibitem{chen2021dimension}
{\sc H.-B. Chen, S.~Chewi, and J.~Niles-Weed}, {\em Dimension-free
  log-{S}obolev inequalities for mixture distributions}, Journal of Functional
  Analysis, 281 (2021), p.~109236.

\bibitem{courtade2018quantitative}
{\sc T.~A. Courtade, M.~Fathi, and A.~Pananjady}, {\em Quantitative stability
  of the entropy power inequality}, IEEE Transactions on Information Theory, 64
  (2018), pp.~5691--5703.

\bibitem{MR1940370}
{\sc M.~Del~Pino and J.~Dolbeault}, {\em Best constants for
  {G}agliardo-{N}irenberg inequalities and applications to nonlinear
  diffusions}, J. Math. Pures Appl. (9), 81 (2002), pp.~847--875.

\bibitem{Demange-PhD}
{\sc J.~Demange}, {\em Des {\'e}quations {\`a} diffusion rapide aux
  in{\'e}galit{\'e}s de {S}obolev sur les mod{\`e}les de la g{\'e}om{\'e}trie},
  PhD thesis, Universit{\'e} Paul Sabatier Toulouse 3, 2005.

\bibitem{https://doi.org/10.48550/arxiv.2209.08651}
{\sc J.~Dolbeault, M.~J. Esteban, A.~Figalli, R.~L. Frank, and M.~Loss}, {\em
  Sharp stability for {S}obolev and log-{S}obolev inequalities, with optimal
  dimensional dependence}, Preprint
  \href{http://arxiv.org/abs/2209.08651}{arXiv: 2209.08651} and
  \href{https://hal.archives-ouvertes.fr/hal-03780031}{hal-03780031},  (2022).

\bibitem{MR2435196}
{\sc J.~Dolbeault, B.~Nazaret, and G.~Savar\'{e}}, {\em On the {B}akry-{E}mery
  criterion for linear diffusions and weighted porous media equations}, Commun.
  Math. Sci., 6 (2008), pp.~477--494.

\bibitem{MR3103175}
{\sc J.~Dolbeault and G.~Toscani}, {\em Improved interpolation inequalities,
  relative entropy and fast diffusion equations}, Ann. Inst. H. Poincar\'e
  Anal. Non Lin\'eaire, 30 (2013), pp.~917--934.

\bibitem{Dolbeault_2015}
\leavevmode\vrule height 2pt depth -1.6pt width 23pt, {\em Stability results
  for logarithmic {S}obolev and {G}agliardo--{N}irenberg inequalities},
  International Mathematics Research Notices, 2016 (2016), pp.~473--498.

\bibitem{MR4116725}
{\sc R.~Eldan, J.~Lehec, and Y.~Shenfeld}, {\em Stability of the logarithmic
  {S}obolev inequality via the {F}\"{o}llmer process}, Ann. Inst. Henri
  Poincar\'{e} Probab. Stat., 56 (2020), pp.~2253--2269.

\bibitem{MR2597943}
{\sc L.~C. Evans}, {\em Partial differential equations}, vol.~19 of Graduate
  Studies in Mathematics, American Mathematical Society, Providence, RI,
  second~ed., 2010.

\bibitem{MR3567822}
{\sc M.~Fathi, E.~Indrei, and M.~Ledoux}, {\em Quantitative logarithmic
  {S}obolev inequalities and stability estimates}, Discrete Contin. Dyn. Syst.,
  36 (2016), pp.~6835--6853.

\bibitem{Federbush}
{\sc P.~Federbush}, {\em Partially alternate derivation of a result of
  {N}elson}, J. Mathematical Phys., 10 (1969), pp.~50--52.

\bibitem{MR3666798}
{\sc F.~Feo, E.~Indrei, M.~R. Posteraro, and C.~Roberto}, {\em Some remarks on
  the stability of the log-{S}obolev inequality for the {G}aussian measure},
  Potential Anal., 47 (2017), pp.~37--52.

\bibitem{Frank_2022}
{\sc R.~L. Frank}, {\em Degenerate stability of some {S}obolev inequalities},
  Ann. Inst. H. Poincar\'e Anal. Non Lin\'eaire (Online first). Doi:
  \href{https://doi.org/10.4171/AIHPC/35} {10.4171/AIHPC/35},  (2022).

\bibitem{Gross75}
{\sc L.~Gross}, {\em Logarithmic {S}obolev inequalities}, Amer. J. Math., 97
  (1975), pp.~1061--1083.

\bibitem{Guionnet2002}
{\sc A.~Guionnet and B.~Zegarlinski}, {\em Lectures on logarithmic {S}obolev
  inequalities}, S{\'e}minaire de probabilit{\'e}s de Strasbourg, 36 (2002),
  pp.~1--134.

\bibitem{https://doi.org/10.48550/arxiv.2303.05604}
{\sc E.~Indrei}, {\em Sharp stability for {LSI}}, Preprint
  \href{http://arxiv.org/abs/2303.05604}{arXiv: 2303.05604},  (2023).

\bibitem{MR4305006}
{\sc E.~Indrei and D.~Kim}, {\em Deficit estimates for the logarithmic
  {S}obolev inequality}, Differential Integral Equations, 34 (2021),
  pp.~437--466.

\bibitem{MR3271181}
{\sc E.~Indrei and D.~Marcon}, {\em A quantitative log-{S}obolev inequality for
  a two parameter family of functions}, Int. Math. Res. Not. IMRN, 2014 (2014),
  pp.~5563--5580.

\bibitem{MR1318794}
{\sc R.~Kannan, L.~Lov\'{a}sz, and M.~Simonovits}, {\em Isoperimetric problems
  for convex bodies and a localization lemma}, Discrete Comput. Geom., 13
  (1995), pp.~541--559.

\bibitem{MR4455233}
{\sc D.~Kim}, {\em Instability results for the logarithmic {S}obolev inequality
  and its application to related inequalities}, Discrete Contin. Dyn. Syst., 42
  (2022), pp.~4297--4320.

\bibitem{MR1186991}
{\sc M.~Ledoux}, {\em A simple analytic proof of an inequality by {P}.
  {B}user}, Proc. Amer. Math. Soc., 121 (1994), pp.~951--959.

\bibitem{MR2195409}
{\sc M.~Ledoux}, {\em Spectral gap, logarithmic {S}obolev constant, and
  geometric bounds}, in Surveys in differential geometry. {V}ol. {IX}, vol.~9
  of Surv. Differ. Geom., Int. Press, Somerville, MA, 2004, pp.~219--240.

\bibitem{MR3320893}
{\sc M.~Ledoux, I.~Nourdin, and G.~Peccati}, {\em Stein's method, logarithmic
  {S}obolev and transport inequalities}, Geom. Funct. Anal., 25 (2015),
  pp.~256--306.

\bibitem{MR3665794}
\leavevmode\vrule height 2pt depth -1.6pt width 23pt, {\em A {S}tein deficit
  for the logarithmic {S}obolev inequality}, Sci. China Math., 60 (2017),
  pp.~1163--1180.

\bibitem{lee2003geometrical}
{\sc K.-A. Lee and J.~L. V{\'a}zquez}, {\em Geometrical properties of solutions
  of the porous medium equation for large times}, Indiana University
  Mathematics Journal,  (2003), pp.~991--1016.

\bibitem{maz1962negative}
{\sc V.~G. Maz'ya}, {\em The negative spectrum of the {$n$}-dimensional
  {S}chr{\"o}dinger operator}, in Doklady Akademii Nauk, vol.~144, Russian
  Academy of Sciences, 1962, pp.~721--722.

\bibitem{maz1962solvability}
\leavevmode\vrule height 2pt depth -1.6pt width 23pt, {\em On the solvability
  of the {N}eumann problem}, in Doklady Akademii Nauk, vol.~147, Russian
  Academy of Sciences, 1962, pp.~294--296.

\bibitem{MR1760620}
{\sc F.~Otto and C.~Villani}, {\em Generalization of an inequality by
  {T}alagrand and links with the logarithmic {S}obolev inequality}, J. Funct.
  Anal., 173 (2000), pp.~361--400.

\bibitem{MR2352327}
{\sc G.~Royer}, {\em An initiation to logarithmic {S}obolev inequalities},
  vol.~14 of SMF/AMS Texts and Monographs, American Mathematical Society,
  Providence, RI; Soci\'{e}t\'{e} Math\'{e}matique de France, Paris, 2007.
\newblock Translated from the 1999 French original by Donald Babbitt.

\bibitem{saumard2014log}
{\sc A.~Saumard and J.~A. Wellner}, {\em Log-concavity and strong
  log-concavity: a review}, Statistics surveys, 8 (2014), p.~45.

\bibitem{MR0109101}
{\sc A.~J. Stam}, {\em Some inequalities satisfied by the quantities of
  information of {F}isher and {S}hannon}, Information and Control, 2 (1959),
  pp.~101--112.

\bibitem{zbMATH01503413}
{\sc C.~{Villani}}, {\em {A short proof of the ``concavity of entropy
  power''}}, {IEEE Trans. Inf. Theory}, 46 (2000), pp.~1695--1696.

\bibitem{MR479373}
{\sc F.~B. Weissler}, {\em Logarithmic {S}obolev inequalities for the
  heat-diffusion semigroup}, Trans. Amer. Math. Soc., 237 (1978), pp.~255--269.

\end{thebibliography}

\end{document}
%%%%%%%%%%%%%%%%%%%%%%%%%%%%%%%%%%%%%%%%%%%%%%%%%%%%%%%%%%%%%%%%%%%%%%%%%%
%%%%%%%%%%%%%%%%%%%%%%%%%%%%%%%%%%%%%%%%%%%%%%%%%%%%%%%%%%%%%%%%%%%%%%%%%%