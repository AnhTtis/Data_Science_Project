
\documentclass[twocolumn,showpacs,superscriptaddress,email,floatfix,longbibliography,prl]{revtex4-2}
\usepackage{graphicx}% Include figure files
\usepackage{dcolumn}% Align table columns on decimal point
\usepackage{bm}% bold math
\usepackage{amsmath}
\usepackage{bbold}
\usepackage{xcolor}
\usepackage[hypertexnames=false]{hyperref}% add hypertext capabilities with https://latex.org/forum/viewtopic.php?t=32205
%\usepackage[mathlines]{lineno}% Enable numbering of text and display math
%\linenumbers\relax % Commence numbering lines
\usepackage{physics} % Standard package for physics (bra-ket, del, curl, ...)

%\usepackage[showframe,%Uncomment any one of the following lines to test 
%%scale=0.7, marginratio={1:1, 2:3}, ignoreall,% default settings
%%text={7in,10in},centering,
%%margin=1.5in,
%%total={6.5in,8.75in}, top=1.2in, left=0.9in, includefoot,
%%height=10in,a5paper,hmargin={3cm,0.8in},
%]{geometry}

\usepackage{orcidlink}
\usepackage{subfiles} % Best loaded last in the preamble

\begin{document}

\preprint{APS/123-QED}

\title{Dispersionless subradiant photon storage in one-dimensional emitter chains}% Force line breaks with \\

\author{Marcel Cech\,\orcidlink{0000-0002-9381-6927}}
\affiliation{Institut f\"ur Theoretische Physik, Universit\"at Tübingen, Auf der Morgenstelle 14, 72076 T\"ubingen, Germany}
\author{Igor Lesanovsky\,\orcidlink{0000-0001-9660-9467}}
\affiliation{Institut f\"ur Theoretische Physik, Universit\"at Tübingen, Auf der Morgenstelle 14, 72076 T\"ubingen, Germany}
\affiliation{School of Physics and Astronomy and Centre for the Mathematics and Theoretical Physics of Quantum Non-Equilibrium Systems, The University of Nottingham, Nottingham, NG7 2RD, United Kingdom}
\author{Beatriz Olmos\,\orcidlink{0000-0002-1140-2641}}
\affiliation{Institut f\"ur Theoretische Physik, Universit\"at Tübingen, Auf der Morgenstelle 14, 72076 T\"ubingen, Germany}
\affiliation{School of Physics and Astronomy and Centre for the Mathematics and Theoretical Physics of Quantum Non-Equilibrium Systems, The University of Nottingham, Nottingham, NG7 2RD, United Kingdom}




\begin{abstract}
Atomic emitter ensembles couple collectively to the radiation field. Although an excitation on a single emitter may be short-lived, a collection of them can contain a photon several orders of magnitude longer than the single emitter lifetime. We provide the exact conditions for optimal absorption, long-lived and dispersionless storage, and release, of a single photon in a sub-wavelength one-dimensional lattice of two-level emitters. In particular, we detail two storage schemes. The first is based on the uncovering of approximate flat sections in the single-photon spectrum, such that a single photon can be stored as a wave packet with effective zero group velocity. For the second scheme we exploit the angular dependence of the interactions induced between the emitters and mediated via exchange of virtual photons, which on a ring gives rise to an effective trapping potential for the photon. In both cases, we are able to obtain, within current experimentally accessible parameters, high-fidelity photon storage for times hundreds of times longer than the single emitter lifetime.
\end{abstract}

%\keywords{Suggested keywords}%Use showkeys class option if keyword
                              %display desired
\maketitle

\textit{Introduction.} When an ensemble of emitters is coupled to a common environment, its dynamics is subject to collective phenomena \cite{Dicke1954,Lehmberg1970,James1993}. On one hand, exchange of virtual photons induces dipole-dipole (exchange) interactions among the emitters. On the other hand, spontaneous emission of photons from the ensemble becomes either enhanced (superradiance) or inhibited (subradiance), depending on the collective state in which the photons are stored in the ensemble. These effects are particularly prominent and dominate the dynamics of the system, when the distance between the emitters is smaller than the wavelength of the considered light \cite{Olmos2013,Pellegrino2014,Jenkins2016,Bromley2016,Jenkins2017,Ferioli2021}, in large ensembles with many emitters \cite{Bienaime2012,Ott2013,Araujo2016,Guerin2016}, and in the presence of structured environments, such as nanophotonic waveguides \cite{Ruostekoski2016,Asenjo2017,Lodahl2017,Asenjo2017_1,Ostermann2019,Kornovan2019,Poddubny2022,Pennetta2022,Chu2022}, cavities \cite{Shankar2021,Gegg2018,Pineiro2022,Hotter2023}, and resonators \cite{Kornovan2021}. Owed to the high degree of experimental control available in cold atomic and solid state systems, these collective phenomena are now in the vanguard of research on quantum information processing and metrology \cite{Reitz2022}.

The existence of subradiant, i.e. long-lived, states offers the opportunity of storing light in emitter ensembles for times that exceed the lifetime of a single emitter by several orders of magnitude. Many theoretical proposals and experimental works have been put forward in the last few years with the aim to use this phenomenon for photon storage and release \cite{Needham2019,Asenjo2017,Ferioli2021}, enhanced quantum metrology \cite{Ostermann2013}, atomic optical mirrors \cite{Bettles2016,Rui2020,Ballantine2022,Buckley2022,Srakaew2023}, and entangled state preparation \cite{Guimond2019,Buonaiuto2019}, among others. The main challenge with subradiant states lies in their preparation as, by their very definition, they possess little overlap with typical electric field radiation modes. For example, in a sub-wavelength periodic arrangement of emitters, subradiant states possess intricate phase patterns, which makes their laser excitation highly involved, requiring, for example, phase imprinting protocols or spatially dependent external magnetic fields \cite{Jen2016,Jen2017,Jen2018}.

\begin{figure}[t]
    \centering
    \includegraphics{fig1.png}
    \caption{\textbf{Dispersionless photon trapping.} A single photon is stored as a subradiant wave packet in a 1D lattice of two-level systems. (a-b): In a chain, the nearest neighbor distance $d$ is chosen such that the wave packet is localised in a region where the dispersion relation is flat. (c-d): In a ring, the spatial variation of the nearest neighbour interaction $V_\mathrm{nn}$ creates an effective trapping potential for the wave packet.}
    \label{fig:fig1}
\end{figure}
\begin{figure*}[t]
    \centering
    \includegraphics{fig2.png}
    \caption{\textbf{Flat dispersion relation.} (a): Decay rates and (b): spectrum for an infinite 1D lattice with lattice constant $d/\lambda=0.2$ (blue), $0.2414$ (red) and $0.26$ (orange). Independently of the orientation of the dipoles, completely subradiant modes (with $\Gamma_k^\delta=0$) can be found between the light lines, i.e. for all $2\pi/d-k_a>k>k_a=2\pi/\lambda$. Only when the dipole moments are perpendicular to the lattice ($\delta=\perp$), the spectrum becomes approximately flat around $k=k_0=\pi/d$. (c): Change of the group velocity $v_g^\perp$ at $k=k_0$. This change becomes zero at $d=d_\mathrm{f}=0.2414\lambda$.}
    \label{fig:fig2}
\end{figure*}

In this work, we present two schemes for the subradiant storage of a single photon in a one dimensional lattice that overcome the above challenge and that, moreover, allow for the dispersionless storage of the photon. The first is based on the preparation of a subradiant single-photon wave packet with zero group velocity by analytically finding an optimal ratio between the lattice constant and the photon wavelength such that the system displays an approximately flat dispersion relation. The second scheme exploits the strong angular dependence of the dipole-dipole interactions. By placing the emitters on a ring geometry, we find that the spatial variation of the interactions create an effective trapping potential for the excitation. By analyzing the eigenstates of this potential, we identify again an optimal set of conditions to ensure the long-lived and high-fidelity photon storage.


\textit{System and master equation.} We consider an ensemble of $N$ emitters trapped in a one dimensional configuration --a chain or a ring-- with lattice constant $d$ (see Fig.~\ref{fig:fig1}). We consider the emitters to be two-level systems, where the ground and excited states, $\ket{g}$ and $\ket{e}$, respectively, are separated by an energy $\hbar\omega_a=h c/\lambda$. All emitters are coupled to the free radiation field. Under the Born-Markov and secular approximations, the dynamics of the emitters' degrees of freedom, included in the density matrix $\rho$, can be described by the master equation
\begin{equation}\label{eq:meq}
    \dot{\rho}=-\frac{\mathrm{i}}{\hbar}\left[H,\rho \right]+\sum_{\alpha, \beta}\Gamma_{\alpha \beta}\left(\sigma_\beta\rho\sigma_\alpha^\dagger-\frac{1}{2}\{\sigma_\alpha^\dagger\sigma_\beta,\rho\}\right) \, ,
\end{equation}
with $H=-\hbar\sum_{\alpha\neq\beta}V_{\alpha \beta}\sigma_\alpha^\dagger \sigma_\beta$, and
where $\sigma_\alpha=\ketbra{g_\alpha}{e_\alpha}$ and $\sigma^\dag_\alpha=\ketbra{e_\alpha}{g_\alpha}$ are the lowering and raising operators for the $\alpha$-th emitter, respectively. The first term of Eq.~(\ref{eq:meq}) represents the dipole-dipole (exchange) interactions between emitters which, for emitters $\alpha$ and $\beta$, occur at a rate $V_{\alpha\beta}$. The second term represents the dissipation (spontaneous emission of photons into the radiation field), which possesses in general a collective character: while the diagonal elements of matrix $\Gamma_{\alpha\beta}$ represent the single-emitter spontaneous decay rate $\Gamma_{\alpha\alpha}=\gamma$, collective single-photon decay modes (corresponding to the eigenvectors of $\Gamma_{\alpha\beta}$) arise due to the presence of non-zero off-diagonal elements. When the associated decay rate of a collective mode is larger (smaller) than $\gamma$, the mode is said to be superradiant (subradiant).

For any general environment, both dipole-dipole and dissipation rates can be obtained in terms of the real and imaginary part of the Green's tensor $\bar{G}(\mathbf{r}_\alpha,\mathbf{r}_\beta,\omega_a)$ evaluated at the positions $\mathbf{r}_\alpha$ and $\mathbf{r}_\beta$ of the emitters \cite{Asenjo2017_1,Dzsotjan2010} as $\Gamma_{\alpha \beta} = \frac{2\omega^2_a}{\hbar\epsilon_0c^2}\mathbf{d}^*_\alpha\text{Im}\{\bar{G}(\mathbf{r}_\alpha,\mathbf{r}_\beta,\omega_a)\}\mathbf{d}^T_\beta$ and $V_{\alpha \beta} = \frac{\omega_a^2}{\hbar\epsilon_0c^2}\mathbf{d}^*_\alpha\text{Re}\{\bar{G}(\mathbf{r}_\alpha,\mathbf{r}_\beta,\omega_a)\}\mathbf{d}^T_\beta$, where $\mathbf{d}_\alpha$ is the dipole transition vector corresponding to emitter $\alpha$ (note, that we will consider all dipoles to be aligned throughout the paper, i.e. $\mathbf{d}_\alpha=\mathbf{d}_\beta\equiv\mathbf{d}$). In this work, we will concern ourselves with the situation where all the emitters are in free-space. Here, the Green's tensor is known analytically and is given by
\begin{eqnarray}\label{eq:G0}
    \bar{G}_0(\mathbf{x}_j,k_a)&=&\frac{\mathrm{e}^{\mathrm{i}k_a r_j}}{4\pi k_a^2r_j^3}\left[(k_a^2r_j^2+\mathrm{i}k_ar_j-1)\mathbb{1}\right.\\\nonumber
    && \left.+(-k_a^2r_j^2-3\mathrm{i}k_ar_j+3)\frac{\mathbf{x}_j\otimes\mathbf{x}_j}{r_j^2}\right] \, ,
\end{eqnarray}
where $\mathbf{x}_{j=\alpha-\beta}\equiv\mathbf{r}_\alpha-\mathbf{r}_\beta$ are all the separation vectors between the emitters, $r_j=|\mathbf{x}_j|$, and $k_a=\omega_a/c=2\pi/\lambda$. 



%\section{Subradiant storage conditions}

%The dipole-dipole interactions and collective dissipation (and, hence, the dynamics of the ensemble) depend on the specific distribution of the emitters in space as well as the relative orientation of the emitter dipole moments. Our aim in this paper is to identify strategies for the creation and subradiant storage of a single excitation in a one-dimensional emitter chain with lattice constant $d$ (see Fig.~\ref{fig:fig1}), as well as its release back into the radiation field as a single photon. To do so, we start identifying in this section regimes that allow a single photon to be stored in the lattice as a subradiant exciton wave packet almost completely frozen in space either via the realization of approximate flat bands or exploiting the angular dependence of the interactions.

\textit{Flat dispersion relation.} Let us first focus on the situation where all non-zero components of the Green's tensor are translationally invariant, e.g., on an infinite one-dimensional chain, or on a ring lattice where the dipoles are pointing perpendicularly to the ring's surface. Here, a Fourier transform of the form
\begin{equation}\label{eq:G0tilde}
    \Tilde{G}_0(k,k_a)=\sum_{j} \mathrm{e}^{-\mathrm{i}kx_j}\bar{G}_0(\mathbf{x}_j,k_a) \, ,
\end{equation}
with $x_j=jd$ and $k\in[0,2\pi/d]$ diagonalizes the Green's tensor. We will moreover consider separately the Green's tensor's components $G^\delta_0(\mathbf{x}_j,k_a)$ which are perpendicular ($\delta=\perp$) and parallel ($\delta=\parallel$) to the chain. % and read
%\begin{eqnarray}
%    G^\perp_0(\mathbf{x}_j,k_a)&=&\frac{\mathrm{e}^{\mathrm{i}k_a r_j}}{4\pi k_a^2r_j^3}(k_a^2r_j^2+\mathrm{i}k_ar_j-1)\\
%    G^\parallel_0(\mathbf{x}_j,k_a)&=&\frac{\mathrm{e}^{\mathrm{i}k_a r_j}}{2\pi k_a^2r_j^3}(1-\mathrm{i}k_ar_j),
%\end{eqnarray}
%where $\mathbf{x}_j$ are all the distances between the emitters and $r_j=|\mathbf{x}_j|$. 
The real and imaginary parts of this Fourier transformed components are proportional to the energies and decay rates of the eigenmodes in the system, $V^\delta_k$ and $\Gamma^\delta_k$, respectively.

%For an infinite one-dimensional chain, both components of the Green's tensor are translationally invariant and their Fourier transform yields
%\begin{eqnarray*}
%    G^\perp_0(k,k_a)&=&\frac{1}{4\pi k_a^2}\sum_{j=-\infty}^{\infty} \frac{\mathrm{e}^{-\mathrm{i}k x_j}\mathrm{e}^{\mathrm{i}k_ar_j}}{r_j^3}(k_a^2r_j^2+\mathrm{i}k_ar_j-1)\\
%    G^\parallel_0(k,k_a)&=&\frac{1}{2\pi k_a^2}\sum_{j=-\infty}^{\infty} \frac{\mathrm{e}^{-\mathrm{i}k x_j}\mathrm{e}^{\mathrm{i}k_ar_j}}{r_j^3}(1-\mathrm{i}k_ar_j),
%\end{eqnarray*}
%with $r_j=|j|d$.

In the case of an infinite one-dimensional chain both components of the Green's tensor are translationally invariant. Let us start by analyzing the collective decay rates $\Gamma^\delta_k$ [Fig.~\ref{fig:fig2}(a)]. First, we note that $\Gamma^\delta_k=0$ for all values $2\pi/d-k_a>k>k_a$, i.e., all states that lie between the so-called light lines, are completely subradiant \cite{Needham2019,Asenjo2017}. Outside of these lines, i.e. $k_a\geq k\geq0$ and $2\pi/d\geq k\geq 2\pi/d-k_a$, the states are typically superradiant, i.e. $\Gamma^\delta_k>\gamma=k_a^3|\mathbf{d}|^2/(3\pi\hbar\epsilon_0)$, both for the perpendicular and parallel components. Since we are interested in creating excitations that are long-lived, we will later on focus only on the creation of states that have a large support with the states within the subradiant region.

In the following, we study the real part of the Green's tensor in Fourier space, which yields the spectrum or dispersion relation, %which reads
%\begin{eqnarray*}
%    V_k^\perp&=&3\gamma%\frac{k_a^3|\mathbf{d}|^2}{2\pi\hbar\epsilon_0}
%    \sum_{j=1}^\infty\cos{kr_j} \left[\cos{k_ar_j}\left(\frac{1}{k_ar_j}-\frac{1}{k_a^3r_j^3}\right)\right.\\
%    &&\left.-\frac{\sin{k_ar_j}}{k_a^2r_j^2}\right]\\
%    V_k^\parallel&=&6\gamma%\frac{k_a^3|\mathbf{d}|^2}{\pi\hbar\epsilon_0}
%    \sum_{j=1}^\infty\cos{kr_j}\left(\frac{\cos{k_ar_j}}{k_a^3r_j^3}+\frac{\sin{k_ar_j}}{k_a^2r_j^2}\right),
%\end{eqnarray*}
%with $\gamma=k_a^3|\mathbf{d}|^2/(6\pi\hbar\epsilon_0)$.
shown in Fig.~\ref{fig:fig2}(b). For wave packets which are localized in $k$-space, the derivative of $V_k^\delta$ with respect to $k$ gives us the group velocity $v_g^\delta$ with which the wave packet travels in real space. % which yields
%\begin{eqnarray*}
%    v_g^\perp(k)&=&\frac{3\gamma}{k_a}%\frac{k_a^2|\mathbf{d}|^2}{2\pi\hbar\epsilon_0}
%    \sum_{j=1}^{\infty}\sin{kr_j}\left[\cos{k_ar_j}\left(\frac{1}{k_a^2r_j^2}-1\right)\right.\\
%    &&\left.+\frac{\sin{k_ar_j}}{k_ar_j}\right]\\
%    v_g^\parallel(k)&=&\frac{6\gamma}{k_a}%\frac{k_a^2|\mathbf{d}|^2}{\pi\hbar\epsilon_0}
%    \sum_{j=1}^{\infty}\sin{kr_j}\left(\frac{\cos{k_ar_j}}{k_a^2r_j^2}+\frac{\sin{k_ar_j}}{k_ar_j}\right).
%\end{eqnarray*}
This group velocity is always zero at $k=k_0=\pi/d$, as imposed by the continuity of the dispersion relation at the border of the first Brillouin zone (FBZ). Between the light lines, for $\delta=\parallel$ this is the only zero of the group velocity, the maximum of $V_k^\parallel$. On the other hand, in the case where the dipoles are perpendicular to the chain, $k=k_0$ becomes a saddle point at around $d\approx\lambda/4$, leading to an approximate flat dispersion relation around $k=k_0$. We calculate the value for which the second derivative of $V_k^\perp$ is exactly zero, finding $d_\mathrm{f}=0.2414\lambda$ [see Fig.~\ref{fig:fig2}(c)]. Note, moreover, that this value has also been found to maximize the lifetime of the subradiant states \cite{Kornovan2019,Kornovan2021}.

Similar results can be found for a ring lattice with all dipoles pointing perpendicularly to the ring, which is a translationally symmetric problem for a finite number $N$ of emitters. In this case, %the Fourier transform of the Green's tensor's reads
%\begin{equation}
%    G^R_0(k,k_a)=\frac{1}{4\pi k_a^2}\sum_{j=1}^{N} \frac{\mathrm{e}^{-\mathrm{i}k x_j}\mathrm{e}^{\mathrm{i}k_ar_j}}{r_j^3}(k_a^2r_j^2+\mathrm{i}k_ar_j-1),
%\end{equation}
%where now 
%\begin{equation}
%    r_{j} = d \frac{\sin (\frac{\pi}{N}j)}{\sin \frac{\pi}{N}},
%\end{equation}
$k$ is only defined for a finite set on values $k=2\pi n/(Nd)$ with $n=0,\dots,N-1$. However, both the decay rates and the spectrum have a very similar behavior to the one-dimensional infinite chain, even for small emitter numbers on the order of $N\approx 10$. This means that, on the ring, subradiant states frozen in real space may be already created for very small system sizes. While here the value of the lattice constant $d_\mathrm{f}$ at which the dispersion relation becomes flat cannot be evaluated analytically, numerical results also show that, already at small values of $N$, $d_\mathrm{f}$ tends to the value of the infinite lattice (see Supplemental Material \cite{SM}).

Our aim is to show that the existence of a flat dispersion relation allows for high-fidelity and subradiant storage of a photon. To do so, we investigate the dynamics of a single-photon wave packet. In $k$-space, one can write such a state generally as $\ket{\Psi_0}=\sum_{k=0}^{2\pi/d} f(k,k_s,\sigma) \ket{k},$ where $f(k,k_s,\sigma)$ is a function centered around $k_s$ and width $\sigma$, with $2\pi/d-k_a>k_s>k_a$ and $\sigma\ll2|\pi/d-k_a|$ in order to ensure that the wave packet is subradiant. In particular, if we consider a Gaussian wave packet, i.e. $f(k,k_s,\sigma)=e^{-\frac{(k-k_s)^2}{4\sigma^2}}/\sqrt{\sqrt{2\pi}\sigma}$, the state in real space is also a Gaussian wave packet that reads 
\begin{equation}\label{eq:wavep}
    \ket{\Psi_0}=\sqrt{\frac{\sigma}{\sqrt{2\pi}}}\sum_{\alpha=1}^{N}e^{-ik_sx_\alpha}e^{-x_\alpha^2\sigma^2}\ket{e_\alpha} \, .
\end{equation}
Note, that while this transformation is only exact when the system is translationally invariant, this wave packet is still centered around $k_s$ with a small width in Fourier space and hence is subradiant for all the situations we will consider below.

\begin{figure}[t]
    \centering
    \includegraphics[]{fig3.png}
    \caption{\textbf{Single-photon time evolution.} (a): Survival probability as a function of $d/\lambda$ and time $t$ of a photon stored as a wave packet (\ref{eq:wavep}) with $\sigma = 0.1 \pi / d$ on a lattice with $N=50$ emitters. (b): $P_\mathrm{sur}(t_\mathrm{final})$ for $\gamma t_\mathrm{final}=100$ (dashed line) and $\gamma t_\mathrm{final}=500$ (solid line). (c-d): Same as (a-b) for the fidelity $F(t)$. Inset in (c): Time evolution of $\left<n_\alpha\right>$ at $d = d_\mathrm{f}$.}
    \label{fig:fig3}
\end{figure}

The time evolution from the initial state $\rho_0\equiv\rho(t=0)=\ketbra{\Psi_0}{\Psi_0}$ is obtained by solving the master equation (\ref{eq:meq}) on a finite one dimensional lattice with $N$ emitters in the single excitation regime \cite{Needham2019}. In order to evaluate the subradiant character of the storage, we calculate the survival probability of the photon to remain trapped in the lattice, defined as
\begin{equation}
    P_\mathrm{sur}(t)= \sum_{\alpha=1}^N \operatorname{Tr} \left[\rho_t n_\alpha\right]=\sum_{\alpha=1}^N \expval{n_\alpha}_t \,,
\end{equation}
with $n_\alpha=\ketbra{e_\alpha}{e_\alpha}$ and $\rho_{t}\equiv\rho(t)$. For a single atom in free space, this survival probability is a decaying exponential, $P_\mathrm{sur}(t)=\mathrm{exp}\left(-\gamma t\right)$. We will moreover evaluate the degree of dispersion of the initial wave packet by means of the fidelity \cite{Jozsa1994}, defined here as% (in its simpler form for pure states [Uhlmann1976TransitionProbability, Jozsa1994Fidelity])
\begin{equation}
    F(t)= \left[ \operatorname{Tr} \sqrt{\sqrt{\rho_0} \rho_t \sqrt{\rho_0}}\right]^2 \, .  %= \bra{\Psi_0}\rho(t)\ket{\Psi_0} = \left| \braket{\Psi_t}{\Psi_0}\right|^2
\end{equation}
In Fig.~\ref{fig:fig3} we show the time-evolution of the survival probability and the fidelity for an initial wave packet with $k_s=k_0=\pi/d$ and $\sigma=0.1\pi/d$ created on a finite one-dimensional lattice with $N=50$ emitters, varying the ratio between the lattice spacing and the light's wavelength, $d/\lambda$. In order to approximately maintain the properties found for the infinite lattice, the wave packet is created at the center of the lattice, avoiding thus edge effects. One can observe in Fig.~\ref{fig:fig3}(a) that the survival probability reaches a maximum --$P_\mathrm{sur}(t_\mathrm{final})=0.9997$ for $t_\mathrm{final}=100/\gamma$--, ensuring a subradiant storage, provided that $d/\lambda$ is close to the value $d_\mathrm{f}/\lambda$ predicted above \cite{Kornovan2019}. While the range of $d/\lambda$ values for which $P_\mathrm{sur}(t_\mathrm{final})\approx1$ is quite broad at $t=t_\mathrm{final}=100/\gamma$, this distribution becomes even more peaked around $d_\mathrm{f}/\lambda$ for longer times, $t_\mathrm{final}$ [see Fig.~\ref{fig:fig3}(b)]. On the other hand, in Figs.~\ref{fig:fig3}(c) and (d), we can see that the fidelity already has a maximum $F(t_\mathrm{final})=0.98$ around $d_\mathrm{f}/\lambda$ even at $t_\mathrm{final}=100/\gamma$. Hence, we have shown that a lattice constant $d=d_\mathrm{f}$ allows for an optimal subradiant and dispersionless photon storage.

\begin{figure}[t]
    \centering
    \includegraphics[]{fig4.png}
    \caption{\textbf{Trapped states.} (a): Emitter ring with the dipole moments aligned parallel to its surface. The angle $\theta$ between the dipole moments and the separation vectors on a ring varies with the position of the emitter. (b): Nearest neighbour interactions $V_\mathrm{nn}$ for a ring of $N=50$ emitters and $d=0.234\lambda$ around $\theta=0$. (c): Time evolution of two initial wave packets, centered at $k_s = k_0$ and $k_s = k_0 (1 - 3 / 25)$ (above and below, respectively) and width $\sigma = 0.103 \pi/d$. Note that $d=0.234\lambda$ is not a necessary condition for the trapping, which can also be demonstrated for other values of $d$.}
    \label{fig:fig4}
\end{figure}


\textit{Trapped states.} We will now present a second storage mechanism of the single-photon state (\ref{eq:wavep}), which is not based on the presence of a flat dispersion relation. Given the overall form of the Green's tensor \eqref{eq:G0}, both coherent and incoherent interactions do not only depend on the reduced distance $r/\lambda$ between the emitters, but also on the angle $\theta$ between the two dipoles and the vector $\mathbf{x}$ that separates them. Let us consider a ring where the dipole moments are contained in the plane of the ring [see Fig.~\ref{fig:fig4}(a)]. Here, the angle $\theta$ and hence the nearest neighbor interactions change as a function of the position on the ring. In particular, in the vicinity of $\theta=0$ these interactions can be approximated as $V_\mathrm{nn}(d,\theta)\approx -3\gamma/4 \left\{\left[A(d)-3B(d)\right]\theta^2+2B(d)\right\}$, with $A(d)=\cos{k_a d}/(k_a d)$ and $B(d)=\cos{k_a d}/(k_a^3 d^3)+\sin{k_a d}/(k_a^2 d^2)$ [see Fig.~\ref{fig:fig4}(b)]. Making the simplifying assumption that only nearest neighbors interact, the Hamiltonian of the system can be approximated as a tight binding model
\begin{equation}\label{eq:TB}
    H\approx H_\mathrm{TB}=\sum_j V_\mathrm{nn}(d,\theta_j)\left(\ketbra{j}{j+1}+\mathrm{h.c.}\right) \, .
\end{equation}
We find that the Gaussian wave packet (\ref{eq:wavep}) with $k_s=k_0$ has a large overlap with one of the eigenstates of $H_\mathrm{TB}$. This overlap can be maximized choosing an optimal value of the width $\sigma$. Here, the state (\ref{eq:wavep}) becomes effectively an eigenstate of $H_\mathrm{TB}$, while being almost completely subradiant. Consequently, as we show in Fig.~\ref{fig:fig4}(c) (above), the wave packet remains trapped while keeping its shape orders of magnitude longer than the single emitter lifetime. Similarly, for a value of $k_s$ slightly deviating from $k_0$, the wave packet can be decomposed as a superposition of a few eigenstates of $H_\mathrm{TB}$, which can be identified in the dynamics by the dispersion and eventual revival of the wave packet [Fig.~\ref{fig:fig4}(c) below].

%In particular, the coherent dipole-dipole interactions between two neighboring emitters reads
%\begin{eqnarray}
%    V_\mathrm{nn}(d,\theta)&=&-\frac{3\gamma}{2}%\frac{\omega_a^3|\mathbf{d}|^2}{4\pi\hbar\epsilon_0c^3}
%    \left\{\sin^2{\theta}\frac{\cos{k_a d}}{k_a d}\right.\\\nonumber
%    &&\left.+(3\cos^2{\theta}-1)\left[\frac{\cos{k_a d}}{k_a^3 d^3}+\frac{\sin{k_a d}}{k_a^2 d^2}\right]\right\}.
%\end{eqnarray}



\textit{Subradiant state preparation and release.} Finally, we propose a scheme for the laser excitation of the initial state (\ref{eq:wavep}). We face two challenges: to efficiently store exactly one photon avoiding the absorption of a second one, and to imprint it with a central momentum $k_s$ close to the border of the FBZ, i.e. $k_s\approx k_0$, to ensure its subradiant character. To achieve this, we will consider two additional levels for each atom: a Rydberg state $\ket{r}$, with a very high principal quantum number $n\gg1$, and an intermediate low-lying state $\ket{s}$ [see Fig.~\ref{fig:fig5}(a)]. Two laser fields drive the $\ket{g}\to\ket{s}$ and $\ket{s}\to\ket{r}$ transitions, with Rabi frequencies and momenta $\Omega_{gs}$, $\Omega_{sr}$ and $\mathbf{k}_{gs}$, $\mathbf{k}_{sr}$, respectively. The $\ket{s}$ state is far detuned ($\Delta\gg\Omega_{gs},\Omega_{sr}$), such that the ground state is coupled to the Rydberg state via a two-photon transition. 
%, via a laser field $H^r=\sum_{j}H_j^r$ with $H_j^r=\hbar/2\left[\left(\Omega_1 e^{i\mathbf{k}_1\cdot\mathbf{x}_j}\ket{g}_j\left<s\right|+\Omega_2e^{i\mathbf{k}_2\cdot\mathbf{x}_j}\ket{s}_j\left<r\right|+\mathrm{h.c.}\right)+2\Delta\ket{s}_j\left<s\right|\right]$.

Thanks to the strong long-ranged interactions between atoms in a Rydberg state \cite{Jaksch2000,Lukin2001,Urban2009}, inside an area with $N_b$ atoms determined by the so-called blockade radius $r_b$ (typically much larger than $d$), only one Rydberg excitation can exist. Hence, a $\pi$-pulse on this two-photon transition for a time $\tau_r=\pi/\Omega_\mathrm{eff}$, with $\Omega_\mathrm{eff}=\sqrt{N_b}\Omega_{gs}\Omega_{sr}/{2\Delta}$ will produce the spin wave state $\ket{\Psi_r}>=\sum_{\alpha=1}^{N_b} e^{-x_\alpha^2\sigma^2} e^{-i (\mathbf{k}_{gs}+\mathbf{k}_{sr})\cdot\mathbf{x}_\alpha}\ket{r_\alpha}/\sqrt{N_b}$ (where the width $\sigma$ is set by the waist of the external laser fields), which contains exactly one excitation. Note that, due to the large lifetimes of Rydberg states (compared with the ones of typical low-lying electronic levels), we assume that this Rydberg spin wave is stable. The second step is to map this spin wave into (\ref{eq:wavep}), which is performed by another $\pi$-pulse with a laser that couples resonantly the Rydberg state to the excited state $\ket{e}$ with Rabi frequency $\Omega$ and momentum $\mathbf{k}$ [see Fig.~\ref{fig:fig5}(b)]. %After this mapping, the final state reads
%\begin{equation}
%    \left|\Psi_e\right>=\frac{1}{\sqrt{N_b}}\sum_{j=1}^{N_b} e^{-i %(\mathbf{k}_1+\mathbf{k}_2-\mathbf{k})\cdot\mathbf{x}_j}\left|e_j\righ%t>.
%\end{equation}
For the wave packet (\ref{eq:wavep}) to be subradiant, the momenta must satisfy that $\frac{2\pi}{d}-k_a>k_s=k_{gs}\cos{\theta_{gs}}+k_{sr}\cos{\theta_{sr}}-k\cos{\theta}>k_a$, where $\theta_{gs}$, $\theta_{sr}$ and $\theta$ are the angles formed by the interatomic vector $\mathbf{x}_j$ and the laser momenta. These parameters can be easily adjusted to make $k_s$ lie between the light lines. E.g., for Rb atoms with $\lambda=2\pi/k_{gs}=780$\,nm and $\lambda_r=2\pi/k_{sr}=2\pi/k=480$\,nm, $\theta=\pi/2$ and $\theta_{gs}=\theta_{sr}\approx2\pi/9$ one obtains $k_s\approx\pi/d_\mathrm{f}$.
\begin{figure}
    \centering
    \includegraphics[width=\columnwidth]{fig5.png}
    \caption{\textbf{Subradiant excitation creation and release.} The creation of the subradiant single photon state (\ref{eq:wavep}) with $k_s\approx k_0$ is realized in two steps: (a) Laser excitation of a single-photon spin wave exploiting the strong interactions between atoms on a Rydberg state $\ket{r}$ and (b) mapping of this spin wave to the low-lying excited state $\ket{e}$. (c): The subradiant state (\ref{eq:wavep}) is Raman transferred into a superradiant one with $k_s<k_a$, outside the light lines.}
    \label{fig:fig5}
\end{figure}

Finally, the subradiantly stored photon may be released by similarly transferring its momentum outside the light lines. Here, a Raman transfer between the $\ket{e}$ and the $\ket{s}$ state via the Rydberg state [see Fig.~\ref{fig:fig5}(c)] produces a spin wave with momentum $k_{gs}\cos{\theta_{gs}}<k_a$, hence becoming a superradiant state which will be quickly lost via spontaneous decay.


\textit{Conclusions and outlook.} In this work, we have demonstrated through two different protocols the potential of one-dimensional sub-wavelength emitter chains for high-fidelity single-photon storage. Particularly appealing is the trapping of the single photon exploiting the angular dependence of the dipole-dipole interactions, which provides extremely high fidelities and lifetimes (e.g., for the $N=50$ case explored in Fig.~\ref{fig:fig4}, $F(t_\mathrm{final})=0.998$ and $P_\mathrm{sur}(t_\mathrm{final})=0.999$ for $t_\mathrm{final}=500/\gamma$), while being quite robust against disorder in the emitter positions (see Supplemental Material \cite{SM}). As shown, for example, in \cite{Nikolopoulos2004,Christandl2004,Bose2007}, an appropriate choice of the nearest neighbor hoppings $V_\mathrm{nn}(d,\theta_j)$ gives rise to an approximate tight binding model (\ref{eq:TB}) that may allow for a fully nondispersive transport of the wave packet across the lattice. Since for us $V_\mathrm{nn}(d,\theta_j)$ depends on the external geometry of the system, it will be interesting to investigate spatial arrangements of the emitters that not only optimize the fidelity and lifetime, but that also allow for the transport of the single-photon wave packet for long distances. Extending this scheme to more than one photon, such that, for example, two photons can be stored, transported, interact, and be released, will also be explored.

\acknowledgments
\textit{Acknowledgements.} The research leading to these results has received funding from the Deutsche Forschungsgemeinschaft (DFG, German Research Foundation) under Projects No. 435696605, 449905436 and 452935230, as well as through the Research Unit FOR 5413/1, Grant No. 465199066.

%\bibliography{references.bib}% Produces the bibliography via BibTeX.
%apsrev4-2.bst 2019-01-14 (MD) hand-edited version of apsrev4-1.bst
%Control: key (0)
%Control: author (8) initials jnrlst
%Control: editor formatted (1) identically to author
%Control: production of article title (0) allowed
%Control: page (0) single
%Control: year (1) truncated
%Control: production of eprint (0) enabled
\begin{thebibliography}{49}%
    \makeatletter
    \providecommand \@ifxundefined [1]{%
     \@ifx{#1\undefined}
    }%
    \providecommand \@ifnum [1]{%
     \ifnum #1\expandafter \@firstoftwo
     \else \expandafter \@secondoftwo
     \fi
    }%
    \providecommand \@ifx [1]{%
     \ifx #1\expandafter \@firstoftwo
     \else \expandafter \@secondoftwo
     \fi
    }%
    \providecommand \natexlab [1]{#1}%
    \providecommand \enquote  [1]{``#1''}%
    \providecommand \bibnamefont  [1]{#1}%
    \providecommand \bibfnamefont [1]{#1}%
    \providecommand \citenamefont [1]{#1}%
    \providecommand \href@noop [0]{\@secondoftwo}%
    \providecommand \href [0]{\begingroup \@sanitize@url \@href}%
    \providecommand \@href[1]{\@@startlink{#1}\@@href}%
    \providecommand \@@href[1]{\endgroup#1\@@endlink}%
    \providecommand \@sanitize@url [0]{\catcode `\\12\catcode `\$12\catcode
      `\&12\catcode `\#12\catcode `\^12\catcode `\_12\catcode `\%12\relax}%
    \providecommand \@@startlink[1]{}%
    \providecommand \@@endlink[0]{}%
    \providecommand \url  [0]{\begingroup\@sanitize@url \@url }%
    \providecommand \@url [1]{\endgroup\@href {#1}{\urlprefix }}%
    \providecommand \urlprefix  [0]{URL }%
    \providecommand \Eprint [0]{\href }%
    \providecommand \doibase [0]{https://doi.org/}%
    \providecommand \selectlanguage [0]{\@gobble}%
    \providecommand \bibinfo  [0]{\@secondoftwo}%
    \providecommand \bibfield  [0]{\@secondoftwo}%
    \providecommand \translation [1]{[#1]}%
    \providecommand \BibitemOpen [0]{}%
    \providecommand \bibitemStop [0]{}%
    \providecommand \bibitemNoStop [0]{.\EOS\space}%
    \providecommand \EOS [0]{\spacefactor3000\relax}%
    \providecommand \BibitemShut  [1]{\csname bibitem#1\endcsname}%
    \let\auto@bib@innerbib\@empty
    %</preamble>
    \bibitem [{\citenamefont {Dicke}(1954)}]{Dicke1954}%
      \BibitemOpen
      \bibfield  {author} {\bibinfo {author} {\bibfnamefont {R.~H.}\ \bibnamefont
      {Dicke}},\ }\bibfield  {title} {\bibinfo {title} {Coherence in spontaneous
      radiation processes},\ }\href {https://doi.org/10.1103/PhysRev.93.99}
      {\bibfield  {journal} {\bibinfo  {journal} {Phys. Rev.}\ }\textbf {\bibinfo
      {volume} {93}},\ \bibinfo {pages} {99} (\bibinfo {year} {1954})}\BibitemShut
      {NoStop}%
    \bibitem [{\citenamefont {Lehmberg}(1970)}]{Lehmberg1970}%
      \BibitemOpen
      \bibfield  {author} {\bibinfo {author} {\bibfnamefont {R.~H.}\ \bibnamefont
      {Lehmberg}},\ }\bibfield  {title} {\bibinfo {title} {Radiation from an
      $n$-atom system. i. general formalism},\ }\href
      {https://doi.org/10.1103/PhysRevA.2.883} {\bibfield  {journal} {\bibinfo
      {journal} {Phys. Rev. A}\ }\textbf {\bibinfo {volume} {2}},\ \bibinfo {pages}
      {883} (\bibinfo {year} {1970})}\BibitemShut {NoStop}%
    \bibitem [{\citenamefont {James}(1993)}]{James1993}%
      \BibitemOpen
      \bibfield  {author} {\bibinfo {author} {\bibfnamefont {D.~F.~V.}\
      \bibnamefont {James}},\ }\bibfield  {title} {\bibinfo {title} {Frequency
      shifts in spontaneous emission from two interacting atoms},\ }\href
      {https://doi.org/10.1103/PhysRevA.47.1336} {\bibfield  {journal} {\bibinfo
      {journal} {Phys. Rev. A}\ }\textbf {\bibinfo {volume} {47}},\ \bibinfo
      {pages} {1336} (\bibinfo {year} {1993})}\BibitemShut {NoStop}%
    \bibitem [{\citenamefont {Olmos}\ \emph {et~al.}(2013)\citenamefont {Olmos},
      \citenamefont {Yu}, \citenamefont {Singh}, \citenamefont {Schreck},
      \citenamefont {Bongs},\ and\ \citenamefont {Lesanovsky}}]{Olmos2013}%
      \BibitemOpen
      \bibfield  {author} {\bibinfo {author} {\bibfnamefont {B.}~\bibnamefont
      {Olmos}}, \bibinfo {author} {\bibfnamefont {D.}~\bibnamefont {Yu}}, \bibinfo
      {author} {\bibfnamefont {Y.}~\bibnamefont {Singh}}, \bibinfo {author}
      {\bibfnamefont {F.}~\bibnamefont {Schreck}}, \bibinfo {author} {\bibfnamefont
      {K.}~\bibnamefont {Bongs}},\ and\ \bibinfo {author} {\bibfnamefont
      {I.}~\bibnamefont {Lesanovsky}},\ }\bibfield  {title} {\bibinfo {title}
      {Long-range interacting many-body systems with alkaline-earth-metal atoms},\
      }\href {https://doi.org/10.1103/PhysRevLett.110.143602} {\bibfield  {journal}
      {\bibinfo  {journal} {Phys. Rev. Lett.}\ }\textbf {\bibinfo {volume} {110}},\
      \bibinfo {pages} {143602} (\bibinfo {year} {2013})}\BibitemShut {NoStop}%
    \bibitem [{\citenamefont {Pellegrino}\ \emph {et~al.}(2014)\citenamefont
      {Pellegrino}, \citenamefont {Bourgain}, \citenamefont {Jennewein},
      \citenamefont {Sortais}, \citenamefont {Browaeys}, \citenamefont {Jenkins},\
      and\ \citenamefont {Ruostekoski}}]{Pellegrino2014}%
      \BibitemOpen
      \bibfield  {author} {\bibinfo {author} {\bibfnamefont {J.}~\bibnamefont
      {Pellegrino}}, \bibinfo {author} {\bibfnamefont {R.}~\bibnamefont
      {Bourgain}}, \bibinfo {author} {\bibfnamefont {S.}~\bibnamefont {Jennewein}},
      \bibinfo {author} {\bibfnamefont {Y.~R.~P.}\ \bibnamefont {Sortais}},
      \bibinfo {author} {\bibfnamefont {A.}~\bibnamefont {Browaeys}}, \bibinfo
      {author} {\bibfnamefont {S.~D.}\ \bibnamefont {Jenkins}},\ and\ \bibinfo
      {author} {\bibfnamefont {J.}~\bibnamefont {Ruostekoski}},\ }\bibfield
      {title} {\bibinfo {title} {Observation of suppression of light scattering
      induced by dipole-dipole interactions in a cold-atom ensemble},\ }\href
      {https://doi.org/10.1103/PhysRevLett.113.133602} {\bibfield  {journal}
      {\bibinfo  {journal} {Phys. Rev. Lett.}\ }\textbf {\bibinfo {volume} {113}},\
      \bibinfo {pages} {133602} (\bibinfo {year} {2014})}\BibitemShut {NoStop}%
    \bibitem [{\citenamefont {Jenkins}\ \emph {et~al.}(2016)\citenamefont
      {Jenkins}, \citenamefont {Ruostekoski}, \citenamefont {Javanainen},
      \citenamefont {Jennewein}, \citenamefont {Bourgain}, \citenamefont
      {Pellegrino}, \citenamefont {Sortais},\ and\ \citenamefont
      {Browaeys}}]{Jenkins2016}%
      \BibitemOpen
      \bibfield  {author} {\bibinfo {author} {\bibfnamefont {S.~D.}\ \bibnamefont
      {Jenkins}}, \bibinfo {author} {\bibfnamefont {J.}~\bibnamefont
      {Ruostekoski}}, \bibinfo {author} {\bibfnamefont {J.}~\bibnamefont
      {Javanainen}}, \bibinfo {author} {\bibfnamefont {S.}~\bibnamefont
      {Jennewein}}, \bibinfo {author} {\bibfnamefont {R.}~\bibnamefont {Bourgain}},
      \bibinfo {author} {\bibfnamefont {J.}~\bibnamefont {Pellegrino}}, \bibinfo
      {author} {\bibfnamefont {Y.~R.~P.}\ \bibnamefont {Sortais}},\ and\ \bibinfo
      {author} {\bibfnamefont {A.}~\bibnamefont {Browaeys}},\ }\bibfield  {title}
      {\bibinfo {title} {Collective resonance fluorescence in small and dense atom
      clouds: Comparison between theory and experiment},\ }\href
      {https://doi.org/10.1103/PhysRevA.94.023842} {\bibfield  {journal} {\bibinfo
      {journal} {Phys. Rev. A}\ }\textbf {\bibinfo {volume} {94}},\ \bibinfo
      {pages} {023842} (\bibinfo {year} {2016})}\BibitemShut {NoStop}%
    \bibitem [{\citenamefont {Bromley}\ \emph {et~al.}(2016)\citenamefont
      {Bromley}, \citenamefont {Zhu}, \citenamefont {Bishof}, \citenamefont
      {Zhang}, \citenamefont {Bothwell}, \citenamefont {Schachenmayer},
      \citenamefont {Nicholson}, \citenamefont {Kaiser}, \citenamefont {Yelin},
      \citenamefont {Lukin}, \citenamefont {Rey},\ and\ \citenamefont
      {Ye}}]{Bromley2016}%
      \BibitemOpen
      \bibfield  {author} {\bibinfo {author} {\bibfnamefont {S.~L.}\ \bibnamefont
      {Bromley}}, \bibinfo {author} {\bibfnamefont {B.}~\bibnamefont {Zhu}},
      \bibinfo {author} {\bibfnamefont {M.}~\bibnamefont {Bishof}}, \bibinfo
      {author} {\bibfnamefont {X.}~\bibnamefont {Zhang}}, \bibinfo {author}
      {\bibfnamefont {T.}~\bibnamefont {Bothwell}}, \bibinfo {author}
      {\bibfnamefont {J.}~\bibnamefont {Schachenmayer}}, \bibinfo {author}
      {\bibfnamefont {T.~L.}\ \bibnamefont {Nicholson}}, \bibinfo {author}
      {\bibfnamefont {R.}~\bibnamefont {Kaiser}}, \bibinfo {author} {\bibfnamefont
      {S.~F.}\ \bibnamefont {Yelin}}, \bibinfo {author} {\bibfnamefont {M.~D.}\
      \bibnamefont {Lukin}}, \bibinfo {author} {\bibfnamefont {A.~M.}\ \bibnamefont
      {Rey}},\ and\ \bibinfo {author} {\bibfnamefont {J.}~\bibnamefont {Ye}},\
      }\bibfield  {title} {\bibinfo {title} {{Collective atomic scattering and
      motional effects in a dense coherent medium}},\ }\href
      {https://doi.org/10.1038/ncomms11039} {\bibfield  {journal} {\bibinfo
      {journal} {Nat. Comm.}\ }\textbf {\bibinfo {volume} {7}},\ \bibinfo {pages}
      {11039} (\bibinfo {year} {2016})}\BibitemShut {NoStop}%
    \bibitem [{\citenamefont {Jenkins}\ \emph {et~al.}(2017)\citenamefont
      {Jenkins}, \citenamefont {Ruostekoski}, \citenamefont {Papasimakis},
      \citenamefont {Savo},\ and\ \citenamefont {Zheludev}}]{Jenkins2017}%
      \BibitemOpen
      \bibfield  {author} {\bibinfo {author} {\bibfnamefont {S.~D.}\ \bibnamefont
      {Jenkins}}, \bibinfo {author} {\bibfnamefont {J.}~\bibnamefont
      {Ruostekoski}}, \bibinfo {author} {\bibfnamefont {N.}~\bibnamefont
      {Papasimakis}}, \bibinfo {author} {\bibfnamefont {S.}~\bibnamefont {Savo}},\
      and\ \bibinfo {author} {\bibfnamefont {N.~I.}\ \bibnamefont {Zheludev}},\
      }\bibfield  {title} {\bibinfo {title} {Many-body subradiant excitations in
      metamaterial arrays: Experiment and theory},\ }\href
      {https://doi.org/10.1103/PhysRevLett.119.053901} {\bibfield  {journal}
      {\bibinfo  {journal} {Phys. Rev. Lett.}\ }\textbf {\bibinfo {volume} {119}},\
      \bibinfo {pages} {053901} (\bibinfo {year} {2017})}\BibitemShut {NoStop}%
    \bibitem [{\citenamefont {Ferioli}\ \emph {et~al.}(2021)\citenamefont
      {Ferioli}, \citenamefont {Glicenstein}, \citenamefont {Henriet},
      \citenamefont {Ferrier-Barbut},\ and\ \citenamefont
      {Browaeys}}]{Ferioli2021}%
      \BibitemOpen
      \bibfield  {author} {\bibinfo {author} {\bibfnamefont {G.}~\bibnamefont
      {Ferioli}}, \bibinfo {author} {\bibfnamefont {A.}~\bibnamefont
      {Glicenstein}}, \bibinfo {author} {\bibfnamefont {L.}~\bibnamefont
      {Henriet}}, \bibinfo {author} {\bibfnamefont {I.}~\bibnamefont
      {Ferrier-Barbut}},\ and\ \bibinfo {author} {\bibfnamefont {A.}~\bibnamefont
      {Browaeys}},\ }\bibfield  {title} {\bibinfo {title} {Storage and release of
      subradiant excitations in a dense atomic cloud},\ }\href
      {https://doi.org/10.1103/PhysRevX.11.021031} {\bibfield  {journal} {\bibinfo
      {journal} {Phys. Rev. X}\ }\textbf {\bibinfo {volume} {11}},\ \bibinfo
      {pages} {021031} (\bibinfo {year} {2021})}\BibitemShut {NoStop}%
    \bibitem [{\citenamefont {Bienaim\'e}\ \emph {et~al.}(2012)\citenamefont
      {Bienaim\'e}, \citenamefont {Piovella},\ and\ \citenamefont
      {Kaiser}}]{Bienaime2012}%
      \BibitemOpen
      \bibfield  {author} {\bibinfo {author} {\bibfnamefont {T.}~\bibnamefont
      {Bienaim\'e}}, \bibinfo {author} {\bibfnamefont {N.}~\bibnamefont
      {Piovella}},\ and\ \bibinfo {author} {\bibfnamefont {R.}~\bibnamefont
      {Kaiser}},\ }\bibfield  {title} {\bibinfo {title} {Controlled dicke
      subradiance from a large cloud of two-level systems},\ }\href
      {https://doi.org/10.1103/PhysRevLett.108.123602} {\bibfield  {journal}
      {\bibinfo  {journal} {Phys. Rev. Lett.}\ }\textbf {\bibinfo {volume} {108}},\
      \bibinfo {pages} {123602} (\bibinfo {year} {2012})}\BibitemShut {NoStop}%
    \bibitem [{\citenamefont {Ott}\ \emph {et~al.}(2013)\citenamefont {Ott},
      \citenamefont {Wubs}, \citenamefont {Lodahl}, \citenamefont {Mortensen},\
      and\ \citenamefont {Kaiser}}]{Ott2013}%
      \BibitemOpen
      \bibfield  {author} {\bibinfo {author} {\bibfnamefont {J.~R.}\ \bibnamefont
      {Ott}}, \bibinfo {author} {\bibfnamefont {M.}~\bibnamefont {Wubs}}, \bibinfo
      {author} {\bibfnamefont {P.}~\bibnamefont {Lodahl}}, \bibinfo {author}
      {\bibfnamefont {N.~A.}\ \bibnamefont {Mortensen}},\ and\ \bibinfo {author}
      {\bibfnamefont {R.}~\bibnamefont {Kaiser}},\ }\bibfield  {title} {\bibinfo
      {title} {Cooperative fluorescence from a strongly driven dilute cloud of
      atoms},\ }\href {https://doi.org/10.1103/PhysRevA.87.061801} {\bibfield
      {journal} {\bibinfo  {journal} {Phys. Rev. A}\ }\textbf {\bibinfo {volume}
      {87}},\ \bibinfo {pages} {061801} (\bibinfo {year} {2013})}\BibitemShut
      {NoStop}%
    \bibitem [{\citenamefont {Ara\'ujo}\ \emph {et~al.}(2016)\citenamefont
      {Ara\'ujo}, \citenamefont {Kre\ifmmode \check{s}\else
      \v{s}\fi{}i\ifmmode~\acute{c}\else \'{c}\fi{}}, \citenamefont {Kaiser},\ and\
      \citenamefont {Guerin}}]{Araujo2016}%
      \BibitemOpen
      \bibfield  {author} {\bibinfo {author} {\bibfnamefont {M.~O.}\ \bibnamefont
      {Ara\'ujo}}, \bibinfo {author} {\bibfnamefont {I.}~\bibnamefont {Kre\ifmmode
      \check{s}\else \v{s}\fi{}i\ifmmode~\acute{c}\else \'{c}\fi{}}}, \bibinfo
      {author} {\bibfnamefont {R.}~\bibnamefont {Kaiser}},\ and\ \bibinfo {author}
      {\bibfnamefont {W.}~\bibnamefont {Guerin}},\ }\bibfield  {title} {\bibinfo
      {title} {Superradiance in a large and dilute cloud of cold atoms in the
      linear-optics regime},\ }\href
      {https://doi.org/10.1103/PhysRevLett.117.073002} {\bibfield  {journal}
      {\bibinfo  {journal} {Phys. Rev. Lett.}\ }\textbf {\bibinfo {volume} {117}},\
      \bibinfo {pages} {073002} (\bibinfo {year} {2016})}\BibitemShut {NoStop}%
    \bibitem [{\citenamefont {Guerin}\ \emph {et~al.}(2016)\citenamefont {Guerin},
      \citenamefont {Ara\'ujo},\ and\ \citenamefont {Kaiser}}]{Guerin2016}%
      \BibitemOpen
      \bibfield  {author} {\bibinfo {author} {\bibfnamefont {W.}~\bibnamefont
      {Guerin}}, \bibinfo {author} {\bibfnamefont {M.~O.}\ \bibnamefont
      {Ara\'ujo}},\ and\ \bibinfo {author} {\bibfnamefont {R.}~\bibnamefont
      {Kaiser}},\ }\bibfield  {title} {\bibinfo {title} {Subradiance in a large
      cloud of cold atoms},\ }\href
      {https://doi.org/10.1103/PhysRevLett.116.083601} {\bibfield  {journal}
      {\bibinfo  {journal} {Phys. Rev. Lett.}\ }\textbf {\bibinfo {volume} {116}},\
      \bibinfo {pages} {083601} (\bibinfo {year} {2016})}\BibitemShut {NoStop}%
    \bibitem [{\citenamefont {Ruostekoski}\ and\ \citenamefont
      {Javanainen}(2016)}]{Ruostekoski2016}%
      \BibitemOpen
      \bibfield  {author} {\bibinfo {author} {\bibfnamefont {J.}~\bibnamefont
      {Ruostekoski}}\ and\ \bibinfo {author} {\bibfnamefont {J.}~\bibnamefont
      {Javanainen}},\ }\bibfield  {title} {\bibinfo {title} {Emergence of
      correlated optics in one-dimensional waveguides for classical and quantum
      atomic gases},\ }\href {https://doi.org/10.1103/PhysRevLett.117.143602}
      {\bibfield  {journal} {\bibinfo  {journal} {Phys. Rev. Lett.}\ }\textbf
      {\bibinfo {volume} {117}},\ \bibinfo {pages} {143602} (\bibinfo {year}
      {2016})}\BibitemShut {NoStop}%
    \bibitem [{\citenamefont {Asenjo-Garcia}\ \emph
      {et~al.}(2017{\natexlab{a}})\citenamefont {Asenjo-Garcia}, \citenamefont
      {Moreno-Cardoner}, \citenamefont {Albrecht}, \citenamefont {Kimble},\ and\
      \citenamefont {Chang}}]{Asenjo2017}%
      \BibitemOpen
      \bibfield  {author} {\bibinfo {author} {\bibfnamefont {A.}~\bibnamefont
      {Asenjo-Garcia}}, \bibinfo {author} {\bibfnamefont {M.}~\bibnamefont
      {Moreno-Cardoner}}, \bibinfo {author} {\bibfnamefont {A.}~\bibnamefont
      {Albrecht}}, \bibinfo {author} {\bibfnamefont {H.~J.}\ \bibnamefont
      {Kimble}},\ and\ \bibinfo {author} {\bibfnamefont {D.~E.}\ \bibnamefont
      {Chang}},\ }\bibfield  {title} {\bibinfo {title} {Exponential improvement in
      photon storage fidelities using subradiance and ``selective radiance'' in
      atomic arrays},\ }\href {https://doi.org/10.1103/PhysRevX.7.031024}
      {\bibfield  {journal} {\bibinfo  {journal} {Phys. Rev. X}\ }\textbf {\bibinfo
      {volume} {7}},\ \bibinfo {pages} {031024} (\bibinfo {year}
      {2017}{\natexlab{a}})}\BibitemShut {NoStop}%
    \bibitem [{\citenamefont {Lodahl}\ \emph {et~al.}(2017)\citenamefont {Lodahl},
      \citenamefont {Mahmoodian}, \citenamefont {Stobbe}, \citenamefont
      {Rauschenbeutel}, \citenamefont {Schneeweiss}, \citenamefont {Volz},
      \citenamefont {Pichler},\ and\ \citenamefont {Zoller}}]{Lodahl2017}%
      \BibitemOpen
      \bibfield  {author} {\bibinfo {author} {\bibfnamefont {P.}~\bibnamefont
      {Lodahl}}, \bibinfo {author} {\bibfnamefont {S.}~\bibnamefont {Mahmoodian}},
      \bibinfo {author} {\bibfnamefont {S.}~\bibnamefont {Stobbe}}, \bibinfo
      {author} {\bibfnamefont {A.}~\bibnamefont {Rauschenbeutel}}, \bibinfo
      {author} {\bibfnamefont {P.}~\bibnamefont {Schneeweiss}}, \bibinfo {author}
      {\bibfnamefont {J.}~\bibnamefont {Volz}}, \bibinfo {author} {\bibfnamefont
      {H.}~\bibnamefont {Pichler}},\ and\ \bibinfo {author} {\bibfnamefont
      {P.}~\bibnamefont {Zoller}},\ }\bibfield  {title} {\bibinfo {title} {Chiral
      quantum optics},\ }\href {https://doi.org/10.1038/nature21037} {\bibfield
      {journal} {\bibinfo  {journal} {Nature}\ }\textbf {\bibinfo {volume} {541}},\
      \bibinfo {pages} {473} (\bibinfo {year} {2017})}\BibitemShut {NoStop}%
    \bibitem [{\citenamefont {Asenjo-Garcia}\ \emph
      {et~al.}(2017{\natexlab{b}})\citenamefont {Asenjo-Garcia}, \citenamefont
      {Hood}, \citenamefont {Chang},\ and\ \citenamefont {Kimble}}]{Asenjo2017_1}%
      \BibitemOpen
      \bibfield  {author} {\bibinfo {author} {\bibfnamefont {A.}~\bibnamefont
      {Asenjo-Garcia}}, \bibinfo {author} {\bibfnamefont {J.~D.}\ \bibnamefont
      {Hood}}, \bibinfo {author} {\bibfnamefont {D.~E.}\ \bibnamefont {Chang}},\
      and\ \bibinfo {author} {\bibfnamefont {H.~J.}\ \bibnamefont {Kimble}},\
      }\bibfield  {title} {\bibinfo {title} {Atom-light interactions in
      quasi-one-dimensional nanostructures: A green's-function perspective},\
      }\href {https://doi.org/10.1103/PhysRevA.95.033818} {\bibfield  {journal}
      {\bibinfo  {journal} {Phys. Rev. A}\ }\textbf {\bibinfo {volume} {95}},\
      \bibinfo {pages} {033818} (\bibinfo {year} {2017}{\natexlab{b}})}\BibitemShut
      {NoStop}%
    \bibitem [{\citenamefont {Ostermann}\ \emph {et~al.}(2019)\citenamefont
      {Ostermann}, \citenamefont {Meignant}, \citenamefont {Genes},\ and\
      \citenamefont {Ritsch}}]{Ostermann2019}%
      \BibitemOpen
      \bibfield  {author} {\bibinfo {author} {\bibfnamefont {L.}~\bibnamefont
      {Ostermann}}, \bibinfo {author} {\bibfnamefont {C.}~\bibnamefont {Meignant}},
      \bibinfo {author} {\bibfnamefont {C.}~\bibnamefont {Genes}},\ and\ \bibinfo
      {author} {\bibfnamefont {H.}~\bibnamefont {Ritsch}},\ }\bibfield  {title}
      {\bibinfo {title} {Super- and subradiance of clock atoms in multimode optical
      waveguides},\ }\href {https://doi.org/10.1088/1367-2630/ab05fb} {\bibfield
      {journal} {\bibinfo  {journal} {New J. Phys.}\ }\textbf {\bibinfo {volume}
      {21}},\ \bibinfo {pages} {025004} (\bibinfo {year} {2019})}\BibitemShut
      {NoStop}%
    \bibitem [{\citenamefont {Kornovan}\ \emph {et~al.}(2019)\citenamefont
      {Kornovan}, \citenamefont {Corzo}, \citenamefont {Laurat},\ and\
      \citenamefont {Sheremet}}]{Kornovan2019}%
      \BibitemOpen
      \bibfield  {author} {\bibinfo {author} {\bibfnamefont {D.~F.}\ \bibnamefont
      {Kornovan}}, \bibinfo {author} {\bibfnamefont {N.~V.}\ \bibnamefont {Corzo}},
      \bibinfo {author} {\bibfnamefont {J.}~\bibnamefont {Laurat}},\ and\ \bibinfo
      {author} {\bibfnamefont {A.~S.}\ \bibnamefont {Sheremet}},\ }\bibfield
      {title} {\bibinfo {title} {Extremely subradiant states in a periodic
      one-dimensional atomic array},\ }\href
      {https://doi.org/10.1103/PhysRevA.100.063832} {\bibfield  {journal} {\bibinfo
       {journal} {Phys. Rev. A}\ }\textbf {\bibinfo {volume} {100}},\ \bibinfo
      {pages} {063832} (\bibinfo {year} {2019})}\BibitemShut {NoStop}%
    \bibitem [{\citenamefont {Poddubny}(2022)}]{Poddubny2022}%
      \BibitemOpen
      \bibfield  {author} {\bibinfo {author} {\bibfnamefont {A.~N.}\ \bibnamefont
      {Poddubny}},\ }\bibfield  {title} {\bibinfo {title} {Driven anti-bragg
      subradiant correlations in waveguide quantum electrodynamics},\ }\href
      {https://doi.org/10.1103/PhysRevA.106.L031702} {\bibfield  {journal}
      {\bibinfo  {journal} {Phys. Rev. A}\ }\textbf {\bibinfo {volume} {106}},\
      \bibinfo {pages} {L031702} (\bibinfo {year} {2022})}\BibitemShut {NoStop}%
    \bibitem [{\citenamefont {Pennetta}\ \emph {et~al.}(2022)\citenamefont
      {Pennetta}, \citenamefont {Lechner}, \citenamefont {Blaha}, \citenamefont
      {Rauschenbeutel}, \citenamefont {Schneeweiss},\ and\ \citenamefont
      {Volz}}]{Pennetta2022}%
      \BibitemOpen
      \bibfield  {author} {\bibinfo {author} {\bibfnamefont {R.}~\bibnamefont
      {Pennetta}}, \bibinfo {author} {\bibfnamefont {D.}~\bibnamefont {Lechner}},
      \bibinfo {author} {\bibfnamefont {M.}~\bibnamefont {Blaha}}, \bibinfo
      {author} {\bibfnamefont {A.}~\bibnamefont {Rauschenbeutel}}, \bibinfo
      {author} {\bibfnamefont {P.}~\bibnamefont {Schneeweiss}},\ and\ \bibinfo
      {author} {\bibfnamefont {J.}~\bibnamefont {Volz}},\ }\bibfield  {title}
      {\bibinfo {title} {Observation of coherent coupling between super- and
      subradiant states of an ensemble of cold atoms collectively coupled to a
      single propagating optical mode},\ }\href
      {https://doi.org/10.1103/PhysRevLett.128.203601} {\bibfield  {journal}
      {\bibinfo  {journal} {Phys. Rev. Lett.}\ }\textbf {\bibinfo {volume} {128}},\
      \bibinfo {pages} {203601} (\bibinfo {year} {2022})}\BibitemShut {NoStop}%
    \bibitem [{\citenamefont {Chu}\ \emph {et~al.}(2022)\citenamefont {Chu},
      \citenamefont {Angelopoulou}, \citenamefont {Lodahl},\ and\ \citenamefont
      {Rotenberg}}]{Chu2022}%
      \BibitemOpen
      \bibfield  {author} {\bibinfo {author} {\bibfnamefont {X.-L.}\ \bibnamefont
      {Chu}}, \bibinfo {author} {\bibfnamefont {V.}~\bibnamefont {Angelopoulou}},
      \bibinfo {author} {\bibfnamefont {P.}~\bibnamefont {Lodahl}},\ and\ \bibinfo
      {author} {\bibfnamefont {N.}~\bibnamefont {Rotenberg}},\ }\bibfield  {title}
      {\bibinfo {title} {Subradiant states for two imperfect quantum emitters
      coupled by a nanophotonic waveguide},\ }\href
      {https://doi.org/10.1103/PhysRevA.106.053702} {\bibfield  {journal} {\bibinfo
       {journal} {Phys. Rev. A}\ }\textbf {\bibinfo {volume} {106}},\ \bibinfo
      {pages} {053702} (\bibinfo {year} {2022})}\BibitemShut {NoStop}%
    \bibitem [{\citenamefont {Shankar}\ \emph {et~al.}(2021)\citenamefont
      {Shankar}, \citenamefont {Reilly}, \citenamefont {J\"ager},\ and\
      \citenamefont {Holland}}]{Shankar2021}%
      \BibitemOpen
      \bibfield  {author} {\bibinfo {author} {\bibfnamefont {A.}~\bibnamefont
      {Shankar}}, \bibinfo {author} {\bibfnamefont {J.~T.}\ \bibnamefont {Reilly}},
      \bibinfo {author} {\bibfnamefont {S.~B.}\ \bibnamefont {J\"ager}},\ and\
      \bibinfo {author} {\bibfnamefont {M.~J.}\ \bibnamefont {Holland}},\
      }\bibfield  {title} {\bibinfo {title} {Subradiant-to-subradiant phase
      transition in the bad cavity laser},\ }\href
      {https://doi.org/10.1103/PhysRevLett.127.073603} {\bibfield  {journal}
      {\bibinfo  {journal} {Phys. Rev. Lett.}\ }\textbf {\bibinfo {volume} {127}},\
      \bibinfo {pages} {073603} (\bibinfo {year} {2021})}\BibitemShut {NoStop}%
    \bibitem [{\citenamefont {Gegg}\ \emph {et~al.}(2018)\citenamefont {Gegg},
      \citenamefont {Carmele}, \citenamefont {Knorr},\ and\ \citenamefont
      {Richter}}]{Gegg2018}%
      \BibitemOpen
      \bibfield  {author} {\bibinfo {author} {\bibfnamefont {M.}~\bibnamefont
      {Gegg}}, \bibinfo {author} {\bibfnamefont {A.}~\bibnamefont {Carmele}},
      \bibinfo {author} {\bibfnamefont {A.}~\bibnamefont {Knorr}},\ and\ \bibinfo
      {author} {\bibfnamefont {M.}~\bibnamefont {Richter}},\ }\bibfield  {title}
      {\bibinfo {title} {Superradiant to subradiant phase transition in the open
      system dicke model: dark state cascades},\ }\href
      {https://doi.org/10.1088/1367-2630/aa9cdd} {\bibfield  {journal} {\bibinfo
      {journal} {New J. Phys.}\ }\textbf {\bibinfo {volume} {20}},\ \bibinfo
      {pages} {013006} (\bibinfo {year} {2018})}\BibitemShut {NoStop}%
    \bibitem [{\citenamefont {Pi\~neiro Orioli}\ \emph {et~al.}(2022)\citenamefont
      {Pi\~neiro Orioli}, \citenamefont {Thompson},\ and\ \citenamefont
      {Rey}}]{Pineiro2022}%
      \BibitemOpen
      \bibfield  {author} {\bibinfo {author} {\bibfnamefont {A.}~\bibnamefont
      {Pi\~neiro Orioli}}, \bibinfo {author} {\bibfnamefont {J.~K.}\ \bibnamefont
      {Thompson}},\ and\ \bibinfo {author} {\bibfnamefont {A.~M.}\ \bibnamefont
      {Rey}},\ }\bibfield  {title} {\bibinfo {title} {Emergent dark states from
      superradiant dynamics in multilevel atoms in a cavity},\ }\href
      {https://doi.org/10.1103/PhysRevX.12.011054} {\bibfield  {journal} {\bibinfo
      {journal} {Phys. Rev. X}\ }\textbf {\bibinfo {volume} {12}},\ \bibinfo
      {pages} {011054} (\bibinfo {year} {2022})}\BibitemShut {NoStop}%
    \bibitem [{\citenamefont {Hotter}\ \emph {et~al.}(2023)\citenamefont {Hotter},
      \citenamefont {Ostermann},\ and\ \citenamefont {Ritsch}}]{Hotter2023}%
      \BibitemOpen
      \bibfield  {author} {\bibinfo {author} {\bibfnamefont {C.}~\bibnamefont
      {Hotter}}, \bibinfo {author} {\bibfnamefont {L.}~\bibnamefont {Ostermann}},\
      and\ \bibinfo {author} {\bibfnamefont {H.}~\bibnamefont {Ritsch}},\
      }\bibfield  {title} {\bibinfo {title} {Cavity sub- and superradiance for
      transversely driven atomic ensembles},\ }\href
      {https://doi.org/10.1103/PhysRevResearch.5.013056} {\bibfield  {journal}
      {\bibinfo  {journal} {Phys. Rev. Res.}\ }\textbf {\bibinfo {volume} {5}},\
      \bibinfo {pages} {013056} (\bibinfo {year} {2023})}\BibitemShut {NoStop}%
    \bibitem [{\citenamefont {Kornovan}\ \emph {et~al.}(2021)\citenamefont
      {Kornovan}, \citenamefont {Savelev}, \citenamefont {Kivshar},\ and\
      \citenamefont {Petrov}}]{Kornovan2021}%
      \BibitemOpen
      \bibfield  {author} {\bibinfo {author} {\bibfnamefont {D.~F.}\ \bibnamefont
      {Kornovan}}, \bibinfo {author} {\bibfnamefont {R.~S.}\ \bibnamefont
      {Savelev}}, \bibinfo {author} {\bibfnamefont {Y.}~\bibnamefont {Kivshar}},\
      and\ \bibinfo {author} {\bibfnamefont {M.~I.}\ \bibnamefont {Petrov}},\
      }\bibfield  {title} {\bibinfo {title} {High-q localized states in finite
      arrays of subwavelength resonators},\ }\href
      {https://doi.org/10.1021/acsphotonics.1c01262} {\bibfield  {journal}
      {\bibinfo  {journal} {ACS Photonics}\ }\textbf {\bibinfo {volume} {8}},\
      \bibinfo {pages} {3627} (\bibinfo {year} {2021})}\BibitemShut {NoStop}%
    \bibitem [{\citenamefont {Reitz}\ \emph {et~al.}(2022)\citenamefont {Reitz},
      \citenamefont {Sommer},\ and\ \citenamefont {Genes}}]{Reitz2022}%
      \BibitemOpen
      \bibfield  {author} {\bibinfo {author} {\bibfnamefont {M.}~\bibnamefont
      {Reitz}}, \bibinfo {author} {\bibfnamefont {C.}~\bibnamefont {Sommer}},\ and\
      \bibinfo {author} {\bibfnamefont {C.}~\bibnamefont {Genes}},\ }\bibfield
      {title} {\bibinfo {title} {Cooperative quantum phenomena in light-matter
      platforms},\ }\href {https://doi.org/10.1103/PRXQuantum.3.010201} {\bibfield
      {journal} {\bibinfo  {journal} {PRX Quantum}\ }\textbf {\bibinfo {volume}
      {3}},\ \bibinfo {pages} {010201} (\bibinfo {year} {2022})}\BibitemShut
      {NoStop}%
    \bibitem [{\citenamefont {Needham}\ \emph {et~al.}(2019)\citenamefont
      {Needham}, \citenamefont {Lesanovsky},\ and\ \citenamefont
      {Olmos}}]{Needham2019}%
      \BibitemOpen
      \bibfield  {author} {\bibinfo {author} {\bibfnamefont {J.~A.}\ \bibnamefont
      {Needham}}, \bibinfo {author} {\bibfnamefont {I.}~\bibnamefont
      {Lesanovsky}},\ and\ \bibinfo {author} {\bibfnamefont {B.}~\bibnamefont
      {Olmos}},\ }\bibfield  {title} {\bibinfo {title} {Subradiance-protected
      excitation transport},\ }\href {https://doi.org/10.1088/1367-2630/ab31e8}
      {\bibfield  {journal} {\bibinfo  {journal} {New J. Phys.}\ }\textbf {\bibinfo
      {volume} {21}},\ \bibinfo {pages} {073061} (\bibinfo {year}
      {2019})}\BibitemShut {NoStop}%
    \bibitem [{\citenamefont {Ostermann}\ \emph {et~al.}(2013)\citenamefont
      {Ostermann}, \citenamefont {Ritsch},\ and\ \citenamefont
      {Genes}}]{Ostermann2013}%
      \BibitemOpen
      \bibfield  {author} {\bibinfo {author} {\bibfnamefont {L.}~\bibnamefont
      {Ostermann}}, \bibinfo {author} {\bibfnamefont {H.}~\bibnamefont {Ritsch}},\
      and\ \bibinfo {author} {\bibfnamefont {C.}~\bibnamefont {Genes}},\ }\bibfield
       {title} {\bibinfo {title} {Protected state enhanced quantum metrology with
      interacting two-level ensembles},\ }\href
      {https://doi.org/10.1103/PhysRevLett.111.123601} {\bibfield  {journal}
      {\bibinfo  {journal} {Phys. Rev. Lett.}\ }\textbf {\bibinfo {volume} {111}},\
      \bibinfo {pages} {123601} (\bibinfo {year} {2013})}\BibitemShut {NoStop}%
    \bibitem [{\citenamefont {Bettles}\ \emph {et~al.}(2016)\citenamefont
      {Bettles}, \citenamefont {Gardiner},\ and\ \citenamefont
      {Adams}}]{Bettles2016}%
      \BibitemOpen
      \bibfield  {author} {\bibinfo {author} {\bibfnamefont {R.~J.}\ \bibnamefont
      {Bettles}}, \bibinfo {author} {\bibfnamefont {S.~A.}\ \bibnamefont
      {Gardiner}},\ and\ \bibinfo {author} {\bibfnamefont {C.~S.}\ \bibnamefont
      {Adams}},\ }\bibfield  {title} {\bibinfo {title} {Enhanced optical cross
      section via collective coupling of atomic dipoles in a 2d array},\ }\href
      {https://doi.org/10.1103/PhysRevLett.116.103602} {\bibfield  {journal}
      {\bibinfo  {journal} {Phys. Rev. Lett.}\ }\textbf {\bibinfo {volume} {116}},\
      \bibinfo {pages} {103602} (\bibinfo {year} {2016})}\BibitemShut {NoStop}%
    \bibitem [{\citenamefont {Rui}\ \emph {et~al.}(2020)\citenamefont {Rui},
      \citenamefont {Wei}, \citenamefont {Rubio-Abadal}, \citenamefont {Hollerith},
      \citenamefont {Zeiher}, \citenamefont {Stamper-Kurn}, \citenamefont {Gross},\
      and\ \citenamefont {Bloch}}]{Rui2020}%
      \BibitemOpen
      \bibfield  {author} {\bibinfo {author} {\bibfnamefont {J.}~\bibnamefont
      {Rui}}, \bibinfo {author} {\bibfnamefont {D.}~\bibnamefont {Wei}}, \bibinfo
      {author} {\bibfnamefont {A.}~\bibnamefont {Rubio-Abadal}}, \bibinfo {author}
      {\bibfnamefont {S.}~\bibnamefont {Hollerith}}, \bibinfo {author}
      {\bibfnamefont {J.}~\bibnamefont {Zeiher}}, \bibinfo {author} {\bibfnamefont
      {D.~M.}\ \bibnamefont {Stamper-Kurn}}, \bibinfo {author} {\bibfnamefont
      {C.}~\bibnamefont {Gross}},\ and\ \bibinfo {author} {\bibfnamefont
      {I.}~\bibnamefont {Bloch}},\ }\bibfield  {title} {\bibinfo {title} {A
      subradiant optical mirror formed by a single structured atomic layer},\
      }\href {https://doi.org/10.1038/s41586-020-2463-x} {\bibfield  {journal}
      {\bibinfo  {journal} {Nature}\ }\textbf {\bibinfo {volume} {583}},\ \bibinfo
      {pages} {369–374} (\bibinfo {year} {2020})}\BibitemShut {NoStop}%
    \bibitem [{\citenamefont {Ballantine}\ and\ \citenamefont
      {Ruostekoski}(2022)}]{Ballantine2022}%
      \BibitemOpen
      \bibfield  {author} {\bibinfo {author} {\bibfnamefont {K.~E.}\ \bibnamefont
      {Ballantine}}\ and\ \bibinfo {author} {\bibfnamefont {J.}~\bibnamefont
      {Ruostekoski}},\ }\bibfield  {title} {\bibinfo {title} {Unidirectional
      absorption, storage, and emission of single photons in a collectively
      responding bilayer atomic array},\ }\href
      {https://doi.org/10.1103/PhysRevResearch.4.033200} {\bibfield  {journal}
      {\bibinfo  {journal} {Phys. Rev. Res.}\ }\textbf {\bibinfo {volume} {4}},\
      \bibinfo {pages} {033200} (\bibinfo {year} {2022})}\BibitemShut {NoStop}%
    \bibitem [{\citenamefont {Buckley-Bonanno}\ \emph {et~al.}(2022)\citenamefont
      {Buckley-Bonanno}, \citenamefont {Ostermann}, \citenamefont {Rubies-Bigorda},
      \citenamefont {Patti},\ and\ \citenamefont {Yelin}}]{Buckley2022}%
      \BibitemOpen
      \bibfield  {author} {\bibinfo {author} {\bibfnamefont {S.}~\bibnamefont
      {Buckley-Bonanno}}, \bibinfo {author} {\bibfnamefont {S.}~\bibnamefont
      {Ostermann}}, \bibinfo {author} {\bibfnamefont {O.}~\bibnamefont
      {Rubies-Bigorda}}, \bibinfo {author} {\bibfnamefont {T.~L.}\ \bibnamefont
      {Patti}},\ and\ \bibinfo {author} {\bibfnamefont {S.~F.}\ \bibnamefont
      {Yelin}},\ }\bibfield  {title} {\bibinfo {title} {Optimized geometries for
      cooperative photon storage in an impurity coupled to a two-dimensional atomic
      array},\ }\href {https://doi.org/10.1103/PhysRevA.106.053706} {\bibfield
      {journal} {\bibinfo  {journal} {Phys. Rev. A}\ }\textbf {\bibinfo {volume}
      {106}},\ \bibinfo {pages} {053706} (\bibinfo {year} {2022})}\BibitemShut
      {NoStop}%
    \bibitem [{\citenamefont {Srakaew}\ \emph {et~al.}(2023)\citenamefont
      {Srakaew}, \citenamefont {Weckesser}, \citenamefont {Hollerith},
      \citenamefont {Wei}, \citenamefont {Adler}, \citenamefont {Bloch},\ and\
      \citenamefont {Zeiher}}]{Srakaew2023}%
      \BibitemOpen
      \bibfield  {author} {\bibinfo {author} {\bibfnamefont {K.}~\bibnamefont
      {Srakaew}}, \bibinfo {author} {\bibfnamefont {P.}~\bibnamefont {Weckesser}},
      \bibinfo {author} {\bibfnamefont {S.}~\bibnamefont {Hollerith}}, \bibinfo
      {author} {\bibfnamefont {D.}~\bibnamefont {Wei}}, \bibinfo {author}
      {\bibfnamefont {D.}~\bibnamefont {Adler}}, \bibinfo {author} {\bibfnamefont
      {I.}~\bibnamefont {Bloch}},\ and\ \bibinfo {author} {\bibfnamefont
      {J.}~\bibnamefont {Zeiher}},\ }\bibfield  {title} {\bibinfo {title} {A
      subwavelength atomic array switched by a single rydberg atom},\ }\bibfield
      {journal} {\bibinfo  {journal} {Nat. Phys.}\ }\href
      {https://doi.org/https://doi.org/10.1038/s41567-023-01959-y}
      {https://doi.org/10.1038/s41567-023-01959-y} (\bibinfo {year}
      {2023})\BibitemShut {NoStop}%
    \bibitem [{\citenamefont {Guimond}\ \emph {et~al.}(2019)\citenamefont
      {Guimond}, \citenamefont {Grankin}, \citenamefont {Vasilyev}, \citenamefont
      {Vermersch},\ and\ \citenamefont {Zoller}}]{Guimond2019}%
      \BibitemOpen
      \bibfield  {author} {\bibinfo {author} {\bibfnamefont {P.-O.}\ \bibnamefont
      {Guimond}}, \bibinfo {author} {\bibfnamefont {A.}~\bibnamefont {Grankin}},
      \bibinfo {author} {\bibfnamefont {D.~V.}\ \bibnamefont {Vasilyev}}, \bibinfo
      {author} {\bibfnamefont {B.}~\bibnamefont {Vermersch}},\ and\ \bibinfo
      {author} {\bibfnamefont {P.}~\bibnamefont {Zoller}},\ }\bibfield  {title}
      {\bibinfo {title} {Subradiant bell states in distant atomic arrays},\ }\href
      {https://doi.org/10.1103/PhysRevLett.122.093601} {\bibfield  {journal}
      {\bibinfo  {journal} {Phys. Rev. Lett.}\ }\textbf {\bibinfo {volume} {122}},\
      \bibinfo {pages} {093601} (\bibinfo {year} {2019})}\BibitemShut {NoStop}%
    \bibitem [{\citenamefont {Buonaiuto}\ \emph {et~al.}(2019)\citenamefont
      {Buonaiuto}, \citenamefont {Jones}, \citenamefont {Olmos},\ and\
      \citenamefont {Lesanovsky}}]{Buonaiuto2019}%
      \BibitemOpen
      \bibfield  {author} {\bibinfo {author} {\bibfnamefont {G.}~\bibnamefont
      {Buonaiuto}}, \bibinfo {author} {\bibfnamefont {R.}~\bibnamefont {Jones}},
      \bibinfo {author} {\bibfnamefont {B.}~\bibnamefont {Olmos}},\ and\ \bibinfo
      {author} {\bibfnamefont {I.}~\bibnamefont {Lesanovsky}},\ }\bibfield  {title}
      {\bibinfo {title} {{Dynamical creation and detection of entangled many-body
      states in a chiral atom chain}},\ }\href
      {https://doi.org/10.1088/1367-2630/ab4f50} {\bibfield  {journal} {\bibinfo
      {journal} {New J. Phys.}\ }\textbf {\bibinfo {volume} {21}},\ \bibinfo
      {pages} {113021} (\bibinfo {year} {2019})}\BibitemShut {NoStop}%
    \bibitem [{\citenamefont {Jen}\ \emph {et~al.}(2016)\citenamefont {Jen},
      \citenamefont {Chang},\ and\ \citenamefont {Chen}}]{Jen2016}%
      \BibitemOpen
      \bibfield  {author} {\bibinfo {author} {\bibfnamefont {H.~H.}\ \bibnamefont
      {Jen}}, \bibinfo {author} {\bibfnamefont {M.-S.}\ \bibnamefont {Chang}},\
      and\ \bibinfo {author} {\bibfnamefont {Y.-C.}\ \bibnamefont {Chen}},\
      }\bibfield  {title} {\bibinfo {title} {Cooperative single-photon subradiant
      states},\ }\href {https://doi.org/10.1103/PhysRevA.94.013803} {\bibfield
      {journal} {\bibinfo  {journal} {Phys. Rev. A}\ }\textbf {\bibinfo {volume}
      {94}},\ \bibinfo {pages} {013803} (\bibinfo {year} {2016})}\BibitemShut
      {NoStop}%
    \bibitem [{\citenamefont {Jen}(2017)}]{Jen2017}%
      \BibitemOpen
      \bibfield  {author} {\bibinfo {author} {\bibfnamefont {H.~H.}\ \bibnamefont
      {Jen}},\ }\bibfield  {title} {\bibinfo {title} {Phase-imprinted multiphoton
      subradiant states},\ }\href {https://doi.org/10.1103/PhysRevA.96.023814}
      {\bibfield  {journal} {\bibinfo  {journal} {Phys. Rev. A}\ }\textbf {\bibinfo
      {volume} {96}},\ \bibinfo {pages} {023814} (\bibinfo {year}
      {2017})}\BibitemShut {NoStop}%
    \bibitem [{\citenamefont {Jen}(2018)}]{Jen2018}%
      \BibitemOpen
      \bibfield  {author} {\bibinfo {author} {\bibfnamefont {H.}~\bibnamefont
      {Jen}},\ }\bibfield  {title} {\bibinfo {title} {Directional subradiance from
      helical-phase-imprinted multiphoton states},\ }\href
      {https://doi.org/10.1038/s41598-018-25592-5} {\bibfield  {journal} {\bibinfo
      {journal} {Sci. Rep.}\ }\textbf {\bibinfo {volume} {8}},\ \bibinfo {pages}
      {7163} (\bibinfo {year} {2018})}\BibitemShut {NoStop}%
    \bibitem [{\citenamefont {Dzsotjan}\ \emph {et~al.}(2010)\citenamefont
      {Dzsotjan}, \citenamefont {S\o{}rensen},\ and\ \citenamefont
      {Fleischhauer}}]{Dzsotjan2010}%
      \BibitemOpen
      \bibfield  {author} {\bibinfo {author} {\bibfnamefont {D.}~\bibnamefont
      {Dzsotjan}}, \bibinfo {author} {\bibfnamefont {A.~S.}\ \bibnamefont
      {S\o{}rensen}},\ and\ \bibinfo {author} {\bibfnamefont {M.}~\bibnamefont
      {Fleischhauer}},\ }\bibfield  {title} {\bibinfo {title} {Quantum emitters
      coupled to surface plasmons of a nanowire: A green's function approach},\
      }\href {https://doi.org/10.1103/PhysRevB.82.075427} {\bibfield  {journal}
      {\bibinfo  {journal} {Phys. Rev. B}\ }\textbf {\bibinfo {volume} {82}},\
      \bibinfo {pages} {075427} (\bibinfo {year} {2010})}\BibitemShut {NoStop}%
    \bibitem [{SM()}]{SM}%
      \BibitemOpen
      \href@noop {} {}\bibinfo {note} {See Supplemental Material for
      details.}\BibitemShut {Stop}%
    \bibitem [{\citenamefont {Jozsa}(1994)}]{Jozsa1994}%
      \BibitemOpen
      \bibfield  {author} {\bibinfo {author} {\bibfnamefont {R.}~\bibnamefont
      {Jozsa}},\ }\bibfield  {title} {\bibinfo {title} {Fidelity for mixed quantum
      states},\ }\href {https://doi.org/10.1080/09500349414552171} {\bibfield
      {journal} {\bibinfo  {journal} {J. Mod. Opt.}\ }\textbf {\bibinfo {volume}
      {41}},\ \bibinfo {pages} {2315} (\bibinfo {year} {1994})}\BibitemShut
      {NoStop}%
    \bibitem [{\citenamefont {Jaksch}\ \emph {et~al.}(2000)\citenamefont {Jaksch},
      \citenamefont {Cirac}, \citenamefont {Zoller}, \citenamefont {Rolston},
      \citenamefont {C\^ot\'e},\ and\ \citenamefont {Lukin}}]{Jaksch2000}%
      \BibitemOpen
      \bibfield  {author} {\bibinfo {author} {\bibfnamefont {D.}~\bibnamefont
      {Jaksch}}, \bibinfo {author} {\bibfnamefont {J.~I.}\ \bibnamefont {Cirac}},
      \bibinfo {author} {\bibfnamefont {P.}~\bibnamefont {Zoller}}, \bibinfo
      {author} {\bibfnamefont {S.~L.}\ \bibnamefont {Rolston}}, \bibinfo {author}
      {\bibfnamefont {R.}~\bibnamefont {C\^ot\'e}},\ and\ \bibinfo {author}
      {\bibfnamefont {M.~D.}\ \bibnamefont {Lukin}},\ }\bibfield  {title} {\bibinfo
      {title} {Fast quantum gates for neutral atoms},\ }\href
      {https://doi.org/10.1103/PhysRevLett.85.2208} {\bibfield  {journal} {\bibinfo
       {journal} {Phys. Rev. Lett.}\ }\textbf {\bibinfo {volume} {85}},\ \bibinfo
      {pages} {2208} (\bibinfo {year} {2000})}\BibitemShut {NoStop}%
    \bibitem [{\citenamefont {Lukin}\ \emph {et~al.}(2001)\citenamefont {Lukin},
      \citenamefont {Fleischhauer}, \citenamefont {Cote}, \citenamefont {Duan},
      \citenamefont {Jaksch}, \citenamefont {Cirac},\ and\ \citenamefont
      {Zoller}}]{Lukin2001}%
      \BibitemOpen
      \bibfield  {author} {\bibinfo {author} {\bibfnamefont {M.~D.}\ \bibnamefont
      {Lukin}}, \bibinfo {author} {\bibfnamefont {M.}~\bibnamefont {Fleischhauer}},
      \bibinfo {author} {\bibfnamefont {R.}~\bibnamefont {Cote}}, \bibinfo {author}
      {\bibfnamefont {L.~M.}\ \bibnamefont {Duan}}, \bibinfo {author}
      {\bibfnamefont {D.}~\bibnamefont {Jaksch}}, \bibinfo {author} {\bibfnamefont
      {J.~I.}\ \bibnamefont {Cirac}},\ and\ \bibinfo {author} {\bibfnamefont
      {P.}~\bibnamefont {Zoller}},\ }\bibfield  {title} {\bibinfo {title} {Dipole
      blockade and quantum information processing in mesoscopic atomic ensembles},\
      }\href {https://doi.org/10.1103/PhysRevLett.87.037901} {\bibfield  {journal}
      {\bibinfo  {journal} {Phys. Rev. Lett.}\ }\textbf {\bibinfo {volume} {87}},\
      \bibinfo {pages} {037901} (\bibinfo {year} {2001})}\BibitemShut {NoStop}%
    \bibitem [{\citenamefont {Urban}\ \emph {et~al.}(2009)\citenamefont {Urban},
      \citenamefont {Johnson}, \citenamefont {Henage}, \citenamefont {Isenhower},
      \citenamefont {Yavuz}, \citenamefont {Walker},\ and\ \citenamefont
      {Saffman}}]{Urban2009}%
      \BibitemOpen
      \bibfield  {author} {\bibinfo {author} {\bibfnamefont {E.}~\bibnamefont
      {Urban}}, \bibinfo {author} {\bibfnamefont {T.~A.}\ \bibnamefont {Johnson}},
      \bibinfo {author} {\bibfnamefont {T.}~\bibnamefont {Henage}}, \bibinfo
      {author} {\bibfnamefont {L.}~\bibnamefont {Isenhower}}, \bibinfo {author}
      {\bibfnamefont {D.~D.}\ \bibnamefont {Yavuz}}, \bibinfo {author}
      {\bibfnamefont {T.~G.}\ \bibnamefont {Walker}},\ and\ \bibinfo {author}
      {\bibfnamefont {M.}~\bibnamefont {Saffman}},\ }\bibfield  {title} {\bibinfo
      {title} {Observation of rydberg blockade between two atoms},\ }\href
      {https://doi.org/https://doi.org/10.1038/nphys1178} {\bibfield  {journal}
      {\bibinfo  {journal} {Nat. Phys.}\ }\textbf {\bibinfo {volume} {5}},\
      \bibinfo {pages} {110} (\bibinfo {year} {2009})}\BibitemShut {NoStop}%
    \bibitem [{\citenamefont {Nikolopoulos}\ \emph {et~al.}(2004)\citenamefont
      {Nikolopoulos}, \citenamefont {Petrosyan},\ and\ \citenamefont
      {Lambropoulos}}]{Nikolopoulos2004}%
      \BibitemOpen
      \bibfield  {author} {\bibinfo {author} {\bibfnamefont {G.~M.}\ \bibnamefont
      {Nikolopoulos}}, \bibinfo {author} {\bibfnamefont {D.}~\bibnamefont
      {Petrosyan}},\ and\ \bibinfo {author} {\bibfnamefont {P.}~\bibnamefont
      {Lambropoulos}},\ }\bibfield  {title} {\bibinfo {title} {Electron wavepacket
      propagation in a chain of coupled quantum dots},\ }\href
      {https://doi.org/10.1088/0953-8984/16/28/019} {\bibfield  {journal} {\bibinfo
       {journal} {J. Phys.: Condens. Matter}\ }\textbf {\bibinfo {volume} {16}},\
      \bibinfo {pages} {4991} (\bibinfo {year} {2004})}\BibitemShut {NoStop}%
    \bibitem [{\citenamefont {Christandl}\ \emph {et~al.}(2004)\citenamefont
      {Christandl}, \citenamefont {Datta}, \citenamefont {Ekert},\ and\
      \citenamefont {Landahl}}]{Christandl2004}%
      \BibitemOpen
      \bibfield  {author} {\bibinfo {author} {\bibfnamefont {M.}~\bibnamefont
      {Christandl}}, \bibinfo {author} {\bibfnamefont {N.}~\bibnamefont {Datta}},
      \bibinfo {author} {\bibfnamefont {A.}~\bibnamefont {Ekert}},\ and\ \bibinfo
      {author} {\bibfnamefont {A.~J.}\ \bibnamefont {Landahl}},\ }\bibfield
      {title} {\bibinfo {title} {Perfect state transfer in quantum spin networks},\
      }\href {https://doi.org/10.1103/PhysRevLett.92.187902} {\bibfield  {journal}
      {\bibinfo  {journal} {Phys. Rev. Lett.}\ }\textbf {\bibinfo {volume} {92}},\
      \bibinfo {pages} {187902} (\bibinfo {year} {2004})}\BibitemShut {NoStop}%
    \bibitem [{\citenamefont {Bose}(2007)}]{Bose2007}%
      \BibitemOpen
      \bibfield  {author} {\bibinfo {author} {\bibfnamefont {S.}~\bibnamefont
      {Bose}},\ }\bibfield  {title} {\bibinfo {title} {Quantum communication
      through spin chain dynamics: an introductory overview},\ }\href
      {https://doi.org/10.1080/00107510701342313} {\bibfield  {journal} {\bibinfo
      {journal} {Contemp. Phys.}\ }\textbf {\bibinfo {volume} {48}},\ \bibinfo
      {pages} {13} (\bibinfo {year} {2007})}\BibitemShut {NoStop}%
    \end{thebibliography}%
    

\setcounter{equation}{0}
\setcounter{figure}{0}
\setcounter{table}{0}
\makeatletter
\renewcommand{\theequation}{S\arabic{equation}}
\renewcommand{\thefigure}{S\arabic{figure}}

\makeatletter
\renewcommand{\theequation}{S\arabic{equation}}
\renewcommand{\thefigure}{S\arabic{figure}}


%%%%%%%%%%%%%%%%%%%%%%%%%%%%%%%%%%%%%%%%%%%%%%%%%%%%%%%%%%%%%%%
%% COMMENT THIS OUT FOR PHYS REV VERSION
%%
%% YOU NEED IT IF YOU WOULD LIKE TO COMPILE
%% THE APPENDIX TO THE SAME PDF FOR ARXIV VERSION

\onecolumngrid
\clearpage
\subfile{supplemental_material}
%%%%%%%%%%%%%%%%%%%%%%%%%%%%%%%%%%%%%%%%%%%%%%%%%%%%%%%%%%%%%%%

\end{document}
%
% ****** End of file apssamp.tex ******
