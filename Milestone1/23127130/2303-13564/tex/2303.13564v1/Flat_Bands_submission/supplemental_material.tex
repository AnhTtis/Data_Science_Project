\documentclass[showpacs,superscriptaddress,email,floatfix,longbibliography,prl]{revtex4-2}

\usepackage{graphicx}% Include figure files
\usepackage{dcolumn}% Align table columns on decimal point
\usepackage{bm}% bold math
\usepackage{amsmath}
\usepackage{bbold}
\usepackage{xcolor}
\usepackage{hyperref}% add hypertext capabilities
%\usepackage[mathlines]{lineno}% Enable numbering of text and display math
%\linenumbers\relax % Commence numbering lines
\usepackage{physics} % Standard package for physics (bra-ket, del, curl, ...)

%\usepackage[showframe,%Uncomment any one of the following lines to test 
%%scale=0.7, marginratio={1:1, 2:3}, ignoreall,% default settings
%%text={7in,10in},centering,
%%margin=1.5in,
%%total={6.5in,8.75in}, top=1.2in, left=0.9in, includefoot,
%%height=10in,a5paper,hmargin={3cm,0.8in},
%]{geometry}

\usepackage{orcidlink}


\begin{document}


\onecolumngrid
\setcounter{page}{1}

\begin{center}
{\Large SUPPLEMENTAL MATERIAL}
\end{center}
\begin{center}
\vspace{0.8cm}
{\Large Dispersionless subradiant photon storage in one-dimensional emitter chains}
\end{center}
\begin{center}
Marcel Cech\,\orcidlink{0000-0002-9381-6927},$^{1}$ Igor Lesanovsky\,\orcidlink{0000-0001-9660-9467},$^{1,2}$ and Beatriz Olmos\,\orcidlink{0000-0002-1140-2641}$^{1,2}$
\end{center}
\begin{center}
$^1${\em Institut f\"ur Theoretische Physik, Universit\"at T\"ubingen,}\\
{\em Auf der Morgenstelle 14, 72076 T\"ubingen, Germany}\\
$^2${\em School of Physics and Astronomy and Centre for the Mathematics\\ and Theoretical Physics of Quantum Non-Equilibrium Systems,\\ The University of Nottingham, Nottingham, NG7 2RD, United Kingdom}
\end{center}

% -------------------------------------------------------------------------------------------------------------------------------------------- %
\section{I. Flat dispersion relation in finite systems}

\begin{figure}[ht]
    \centering
    \includegraphics{fig_s1.png}
    \caption{\textbf{Flat dispersion relation.} (a): Decay rates and (b): spectrum for an finite 1D ring lattice of $N=20$ emitters with lattice constant $d/\lambda=0.2$ (blue), $0.2414$ (red) and $0.26$ (orange), where the dipole moments are perpendicular to the surface of the ring. The same features as in Fig.~\ref{fig:fig2} in the main manuscript (infinite 1D lattice) are found, i.e., subradiant modes between the light lines together with approximately flat spectrum around $k=k_0=\pi/d$. (c): Optimal lattice spacing $d_\mathrm{f}(N)$ for flat dispersion relation as a function of $N$. Already for small numbers of emitters $N$, it approaches the value for the infinite chain $d_\mathrm{f}=0.2414\lambda$ (red solid line).}
    \label{fig:fig_s1}
\end{figure}

In this section, we briefly discuss the applicability of arguments we made for the infinite chain to finite but periodic systems. We concentrate our study on $N$ emitters arranged on a ring lattice with dipole moments perpendicular to the ring plane.

Utilizing the same approach as for the infinite one-dimensional chain, we calculate the collective decay rate [Fig.~\ref{fig:fig_s1}(a)] and spectrum or dispersion relation [Fig.~\ref{fig:fig_s1}(b)] for a finite ring lattice of $N=20$ emitters. The direct comparison to Fig. \ref{fig:fig2}(a-b) in the main manuscript shows a strong resemblance between the finite and infinite case. We find that subradiant states with finite lifetimes smaller than the single atom decay rate are found for momenta $2\pi/d-k_a>k>k_a$. Furthermore, the dispersion relation for these values of $k$ is approximately flat for values of $d$ close to the one found for the infinite lattice, $d_\mathrm{f}$. 

We quantitatively search for the optimal lattice spacing $d_\mathrm{f}(N)$ in the finite ring with $N$ emitters by numerically calculating the lattice spacing $d_\mathrm{f}(N)$ with a vanishing curvature of $V_k^\perp$ at $k=k_0$. In Fig.~\ref{fig:fig_s1}(c) we compare these values to $d_\mathrm{f}=0.2414\lambda$ found for the infinite chain. We observe that, as expected, already at small values of $N$, $d_\mathrm{f}(N)$ tends to the value of the infinite lattice.



\section{II. Disorder}
We give here a brief discussion on how robust the two mechanisms for long-lived and dispersionless storage are against positional disorder. Deviations of the emitter positions from a perfect lattice give rise to disorder in both the interaction and decay rates in Eq.~(\ref{eq:meq}). We model the disorder by averaging over many realizations. In each realization we choose the position of the emitters randomly according to a three dimensional Gaussian centered on each lattice site with equal widths on all three directions, $\sigma_{d}$, which is on practice determined by the lattice depth. We investigate the influence of the disorder on the survival probability and the fidelity of both storage mechanisms.

\begin{figure}[ht]
    \centering
    \includegraphics{fig_s2.png}
    \caption{\textbf{Disorder.} Influence of increasing disorder on storage with (a): flat dispersion relation [see Fig.~\ref{fig:fig3}] and (b): trapped states [see Fig.~\ref{fig:fig4}]. Each panel compares the excitation probability $\left< n_\alpha \right>_t$ at site $\alpha$, the survival probability $P_\mathrm{sur}(t)$, the fidelity $F(t)$ and the ratio of latter $F(t) / P_\mathrm{sur}(t)$ two without disorder (leftmost panel, blue lines) to the averages over $100$ realizations of disorders characterized by the widths $\sigma_d = 0.01d,\,...,\,0.05d$. We set $d=0.234\lambda$ and investigate wave packets (\ref{eq:wavep}) with (a): $k_s=k_0$ and $\sigma=0.1\pi/d$ on a lattice with perpendicular dipoles of $N=50$ emitters and (b): $k_s=k_0$ and $\sigma=0.103\pi/d$ on a ring lattice of the same size with dipole moments aligned parallel to its surface.}
    \label{fig:fig_s2}
\end{figure}

Analyzing the results in Fig.~\ref{fig:fig_s2}, we find that both storage mechanisms exhibit a similar robustness to the disorder. With increasing disorder, the survival probability as well as the fidelity decreases in comparison to the regular lattice spacing. However, the storage is still notably enhanced over the single atom case. We also plot the ratio $F(t) / P_\mathrm{sur}(t)$, which represents the dispersion of the wave packet conditioned to the photon not having been emitted. This ratio remains particularly high for the trapped state with $k_s=k_0$ [see Fig.~\ref{fig:fig_s2} (b)], meaning that if the excitation is still in the system after a time $t$, the state in which the photon is stored will still be the wave packet (\ref{eq:wavep}).
% -------------------------------------------------------------------------------------------------------------------------------------------- %


%%%%%%%%%%%%%%%%%%%%%%%%%%%%%%%%%%%%%%%%%%%%%%%%%%%%%%%%%%%%%%%
%% COMMENT THIS OUT FOR ARXIV VERSION
%% YOU NEED IT IF YOU WOULD LIKE TO COMPILE THE APPENDIX ONLY

%\clearpage
%\bibliography{biblio}
%%%%%%%%%%%%%%%%%%%%%%%%%%%%%%%%%%%%%%%%%%%%%%%%%%%%%%%%%%%%%%%

\end{document}