\documentclass[showpacs,superscriptaddress,email,floatfix,longbibliography,prl]{revtex4-2}

\usepackage{graphicx}% Include figure files
\usepackage{dcolumn}% Align table columns on decimal point
\usepackage{bm}% bold math
\usepackage{amsmath}
\usepackage{bbold}
\usepackage{xcolor}
\usepackage{hyperref}% add hypertext capabilities
%\usepackage[mathlines]{lineno}% Enable numbering of text and display math
%\linenumbers\relax % Commence numbering lines
\usepackage{physics} % Standard package for physics (bra-ket, del, curl, ...)
\usepackage{float}

%\usepackage[showframe,%Uncomment any one of the following lines to test 
%%scale=0.7, marginratio={1:1, 2:3}, ignoreall,% default settings
%%text={7in,10in},centering,
%%margin=1.5in,
%%total={6.5in,8.75in}, top=1.2in, left=0.9in, includefoot,
%%height=10in,a5paper,hmargin={3cm,0.8in},
%]{geometry}

\usepackage{orcidlink}


\begin{document}

% style for Supplemental material numbering


\setcounter{equation}{0}
\setcounter{figure}{0}
\setcounter{table}{0}
\setcounter{page}{1}
\makeatletter
\renewcommand{\theequation}{S\arabic{equation}}
\renewcommand{\thefigure}{S\arabic{figure}}
\renewcommand{\bibnumfmt}[1]{[S#1]}
\renewcommand{\citenumfont}[1]{S#1}
\onecolumngrid
\setcounter{page}{1}

\begin{center}
{\Large SUPPLEMENTAL MATERIAL}
\end{center}
\begin{center}
\vspace{0.8cm}
{\Large Dispersionless subradiant photon storage in one-dimensional emitter chains}
\end{center}
\begin{center}
Marcel Cech\,\orcidlink{0000-0002-9381-6927},$^{1}$ Igor Lesanovsky\,\orcidlink{0000-0001-9660-9467},$^{1,2}$ and Beatriz Olmos\,\orcidlink{0000-0002-1140-2641}$^{1,2}$
\end{center}
\begin{center}
$^1${\em Institut f\"ur Theoretische Physik, Universit\"at T\"ubingen,}\\
{\em Auf der Morgenstelle 14, 72076 T\"ubingen, Germany}\\
$^2${\em School of Physics and Astronomy and Centre for the Mathematics\\ and Theoretical Physics of Quantum Non-Equilibrium Systems,\\ The University of Nottingham, Nottingham, NG7 2RD, United Kingdom}
\end{center}

% -------------------------------------------------------------------------------------------------------------------------------------------- %

\section{I. Calculation of $d_\mathrm{f}(\delta)$ for flat dispersion relation}
In this section, we detail the derivation of the lattice spacing $d_\mathrm{f}(\delta)$ to have an approximate flat dispersion relation at different orientation angle $\delta$ of the dipole moments with respect to the chain. We start by considering the dispersion relation
\begin{align}
    \label{eq:V_k_delta}
    \begin{split}
        V_k^\delta = \frac{3 \gamma}{4 k_a^3d^3} \operatorname{Re} \Bigl[& (1-3\cos^2\delta) \left( \operatorname{Li}_3(\mathrm{e}^{\mathrm{i}(k_a + k)d}) + \operatorname{Li}_3(\mathrm{e}^{\mathrm{i}(k_a - k)d}) - \mathrm{i}k_a d \operatorname{Li}_2(\mathrm{e}^{\mathrm{i}(k_a + k)d}) - \mathrm{i}k_a d \operatorname{Li}_2(\mathrm{e}^{\mathrm{i}(k_a - k)d}) \right)\\
        & + \sin^2\delta \left(k_a^2 d^2 \operatorname{Ln} (1-\mathrm{e}^{\mathrm{i}(k_a + k)d}) + k_a^2 d^2 \operatorname{Ln} (1-\mathrm{e}^{\mathrm{i}(k_a - k)d}) \right) \Bigr] \,
    \end{split}
\end{align}
for an emitter chain of lattice constant $d$ and dipoles oriented at an site-independent angle $\delta$ \cite{Asenjo2017}. In this expression, $\operatorname{Li}_n(x)$ denotes the polylogarithm of order $n$. We now calculate the second derivative of Eq.\,(\ref{eq:V_k_delta}) with respect to $k$, which essentially encodes the change of the group velocity $v_k^\delta$, at $k = k_0 = \pi / d$ (the end/beginning of the Brillouin zone) and find \cite{Zhang2020flatband}
\begin{align}
    \label{eq:change_in_v_g_delta}
    \frac{\partial}{\partial k} v_k^\delta \Bigr|_{k_0} = \frac{3\gamma d}{2 k_a} \left[ (1 - 3 \cos^2\delta) \left( \operatorname{Ln}\left(2 \cos \frac{k_a d}{2}\right) + \frac{k_a d}{2} \tan(\frac{k_a d}{2}) \right) - \sin^2\delta \left( \frac{k_a d}{2} \right)^2 \frac{1}{\cos^2(\frac{k_a d}{2})} \right] \, .
    %\propto\operatorname{Re} \left[  -i (1-3\cos^2\delta)\left(k_a d + i \operatorname{Ln}(1+\mathrm{e}^{i k_a d}) - \frac{k_a d}{1 + \mathrm{e}^{ik_a d}} \right) + \sin^2\delta \left(k_a^2 d^2 \left( \frac{1}{(1 + \mathrm{e}^{i k_a d})^2} - \frac{1}{1 + \mathrm{e}^{i k_a d}} \right)\right) \right] \, .
\end{align}
We now set this expression to zero to find the condition for an approximate flat dispersion relation. Despite the analytic form of the expression above, the equation is transcendental and hence we calculate the solutions numerically.

\begin{figure}[ht]
    \centering
    \includegraphics{fig_s1.pdf}
    \caption{\textbf{Lattice spacings with flat dispersion relation.} {(a):} Parameter pairs $\{\delta, d_\mathrm{f}(\delta)\}$ for which Eq.\,(\ref{eq:change_in_v_g_delta}) vanishes. The lattice spacing $d_\mathrm{f} = 0.2414\lambda$ (red dashed line) providing a flat dispersion relation for perpendicular dipoles poses an upper bound for the lattice spacings with this feature. For two chosen lattice spacings $d_\mathrm{f}(\delta) = 0.1 \lambda, 0.2\lambda$, the insets show the dispersion relation with a flat section around $k_0 = \frac{\pi}{d}$ and well outside the radiative regime (gray shading). {(b-e):} Survival probability and fidelity, respectively, as shown in Fig.~3 in the main manuscript but for $\delta = 3\pi/8$ and $d_\mathrm{f}(\delta) = 0.2\lambda$ (orange dashed line).} %As before, $\gamma t_\mathrm{final}=100$ (blue dashed line) and $\gamma t_\mathrm{final}=500$ (blue solid line).}
    \label{fig:fig_s1}
\end{figure}

We visualize the results in Fig.~\ref{fig:fig_s1}(a). Here, we observe that there is a minimum angle $\delta_\mathrm{min}$ below which the equation does not have a solution. Taking the limit $d\to0$ in Eq.\,(\ref{eq:change_in_v_g_delta}), we find $\delta_\mathrm{min} = \arccos(1 / \sqrt{3})$. For $\delta\in\left(\delta_\mathrm{min},\pi/2\right]$, there exists a $d_\mathrm{f}(\delta)$ which leads to an approximate flat band, as one can see from the two insets in Fig.~\ref{fig:fig_s1}(a). The remarkably subradiant and dispersionless dynamics demonstrated in the main manuscript for $\delta=\pi/2$ persists for all these pairs of values, as illustrated in Fig.~\ref{fig:fig_s1}(b-e), where we investigate the survival probability and the fidelity for times up to $\gamma t_\mathrm{final}=100$ and $\gamma t_\mathrm{final}=500$, respectively.  % Minimal theta= 0.3043043043043043 /pi at d= 0.014199999999999975 lambda with flat region in my calculations.)





% -------------------------------------------------------------------------------------------------------------------------------------------- %
\section{II. Flat dispersion relation in finite systems}

\begin{figure}[ht]
    \centering
    \includegraphics{fig_s2.pdf}
    \caption{\textbf{Flat dispersion relation.} (a): Decay rates and (b): spectrum for a finite 1D ring lattice of $N=20$ emitters with lattice constant $d/\lambda=0.2$ (blue), $0.2414$ (red) and $0.26$ (orange), where the dipole moments are perpendicular to the surface of the ring. The same features as in Fig.~2 in the main manuscript (infinite 1D lattice) are found, i.e., subradiant modes between the light lines together with approximately flat spectrum around $k=k_0=\pi/d$. (c): Optimal lattice spacing $d_\mathrm{f}(N)$ for flat dispersion relation as a function of $N$. Already for small numbers of emitters $N$, it approaches the value for the infinite chain $d_\mathrm{f}=0.2414\lambda$ (red solid line).}
    \label{fig:fig_s2}
\end{figure}

In this section, we briefly discuss the applicability of arguments we made for the infinite chain to finite but periodic systems. We concentrate our study on $N$ emitters arranged on a ring lattice with dipole moments perpendicular to the ring plane ($\delta = \pi/2 \equiv \perp$).

Utilizing the same approach as for the infinite one-dimensional chain, we calculate the collective decay rate [Fig.~\ref{fig:fig_s2}(a)] and spectrum or dispersion relation [Fig.~\ref{fig:fig_s2}(b)] for a finite ring lattice of $N=20$ emitters. The direct comparison to Fig. 2(a-b) in the main manuscript shows a strong resemblance between the finite and infinite case. We find that subradiant states with finite lifetimes smaller than the single atom decay rate are found for momenta $2\pi/d-k_a>k>k_a$. Furthermore, the dispersion relation for these values of $k$ is approximately flat for values of $d$ close to the one found for the infinite lattice, $d_\mathrm{f}$. 

We quantitatively search for the optimal lattice spacing $d_\mathrm{f}(N)$ in the finite ring with $N$ emitters by numerically calculating the lattice spacing $d_\mathrm{f}(N)$ with a vanishing curvature of $V_k^\perp$ at $k=k_0$. In Fig.~\ref{fig:fig_s2}(c) we compare these values to $d_\mathrm{f}=0.2414\lambda$ found for the infinite chain. We observe that, as expected, already at small values of $N$, $d_\mathrm{f}(N)$ tends to the value of the infinite lattice.
% -------------------------------------------------------------------------------------------------------------------------------------------- %

\section{III. State preparation using a Gaussian beam}
For illustration purposes, we have primarily focused on the storage of a Gaussian wave packet [see Eq.\,(3)]. However, as we discussed in the main manuscript, for our storage scheme to work we only require that the wave packet stored has support mainly on the approximately flat area of the dispersion relation. Here, in particular, we demonstrate the storage of states that were prepared using a Gaussian laser beam.

The electric field at the emitters position $x_\alpha$ [see Fig.~\ref{fig:fig_s3}(a) for the decomposition into $x_\alpha^\parallel$ and $x_\alpha^\perp$] is given by %Throughout this section, we utilize the parameters presented in the \textit{Subradiant state preparation and release} section of the main manuscript. 
\begin{align}
    \label{eq:gaussian_beam}
    E(x_\alpha) = E_0 \frac{w_0}{w(x_\alpha^\parallel)} \exp\left( \frac{-(x_\alpha^\perp)^2}{w(x_\alpha^\parallel)^2} \right) \exp \left( -i \left( k x_\alpha^\parallel  + k \frac{(x_\alpha^\perp)^2}{2R(x_\alpha^\parallel)} -\varphi(x_\alpha^\parallel) \right) \right)\, ,  % = |E(x_\alpha)| \mathrm{e}^{-i \operatorname{Arg}[E(x_\alpha)]}
\end{align}
with the spot size $w(x_\alpha^\parallel) = w_0\sqrt{1 + \left( x_\alpha^\parallel / x_R \right)^2}$, the Rayleigh range $x_R = \pi w_0^2 n / \lambda$ (we set the refractive index $n=1$), the radius of curvature $R(x_\alpha^\parallel) = x_\alpha^\parallel \left( 1 + (x_R / x_\alpha^\parallel)^2 \right)$ and the Gouy phase $\varphi(x_\alpha^\parallel) = \arctan(x_\alpha^\parallel / x_R)$ \cite{svelto2010}. In this expression, the minimal waist $w_0$ and the wavevector $\mathbf{k}$ (such that $k = |\mathbf{k}| = 2\pi / \lambda$) are the two parameters that we control. 

\begin{figure}[t]
    \centering
    \includegraphics{fig_s3.pdf}
    \caption{\textbf{State preparation using a Gaussian beam.} (a): Geometry of excitation with Gaussian beam. The Gaussian beam with wavevector $\mathbf{k}$ hits the chain of atom at an angle $\theta$ such that the cross-hair from its smallest waist $w_0$ and the beam center lies between the two central emitters. The waist lines illustrates where the electric field decreases to $1/\mathrm{e}$ of the axial value. $x_\alpha^\parallel$ ($x_\alpha^\perp$) represent the axial (perpendicular) decomposition of the atomic position $x_\alpha$. (b): Storage of wave packet prepared by Gaussian beam with waist $w_0 \approx 3.4 \lambda$. The waist is chosen such that the initial state has a width of approximately $\sigma = 0.06 \pi / d$. The one-dimensional chain of emitters is characterized by $d = d_\mathrm{f} = 0.2414\lambda$ and $N = 50$. We investigate the survival probability $P_\mathrm{sur}(t)$ and the fidelity $F(t)$ of the new, non-Gaussian initial state. $\mathrm{e}^{-\gamma t}$ corresponds to the single emitter survival probability. The inset underlines that we still obtain dispersionless subradiant storage.}
    \label{fig:fig_s3}
\end{figure}

In Fig.~\ref{fig:fig_s3}(b), we investigate the dispersionless subradiant storage of the new initial state for the chain of emitters at $d = d_\mathrm{f} = 0.2414 \lambda$ with perpendicular dipole moments. We observe that also this wave packet is stored over a much longer time-span than the single emitter's lifetime $1 / \gamma$. Comparing the survival probability $P_\mathrm{sur}(t)$ and fidelity $F(t)$ at $\gamma t_\mathrm{final} = 100$ for this initial state and a wave packet described by $k_s = k_0$ and $\sigma = 0.06 \pi / d$, we find again impressive storage capacities, with $P_\mathrm{sur}(t_\mathrm{final}) \approx 0.999$ and $F(t_\mathrm{final}) \approx 0.998$.

%of the Gaussian beam and the approximation of a plane wave accompanied by a Gaussian envelope as assumed for Eq.\,(\ref{eq:wavep}). In this way, both approaches create a wave packet with strong support on subradiant collective modes and additionally flat dispersion relation. Note that for smaller beam waists $w_0$ (i.e., larger reciprocal widths), where performance of the initial state prepared the Gaussian beam the storage sooner becomes dispersive before the corresponding wave packet in Eq.\,(\ref{eq:wavep}) does.
%We investigate the effect of the exciation of the new, initial state using this Gaussian beam utilize the parameters presented in the \textit{Subradiant state preparation and release} section of the main manuscript (e.g., $\lambda = 780\,$nm, $\theta = 2\pi / 9$ and minimal waist $w_0$ for the $\ket{g} \to\ket{s}$ transition, while the other transitions are driven by lasers with broader waists such that we can approximate them well by plane waves with the respective wavevector). We observe that also this wave packet is stored over much longer than the single emitter lifetime $1/ \gamma$. For $\gamma t = 100$ ($\gamma t = 500$), both the survival probability $P_\mathrm{sur}(t) = 0.9998$ ($P_\mathrm{sur}(t) = 0.9964$) and fidelity $F(t) = 0.9880$ ($F(t) = 0.9334$) remain close to one. In panels (c,d) in Fig.~\ref{fig:fig_s3}, we compare the electric field of the Gaussian beam with the field assumed to prepare the initial state in Eq.\,(\ref{eq:wavep}). We find that the underlying approximation of a plane wave with Gaussian envelope describes the real electric field quite well in the areas (white background) with non-negligible intensity. NEED POSSIBLY CORRECTION AFTER DISCUSSION!!??!!
% -------------------------------------------------------------------------------------------------------------------------------------------- %


\section{IV. Disorder}
We give here a brief discussion on how robust the two mechanisms for long-lived and dispersionless storage are against positional disorder. Deviations of the emitter positions from a perfect lattice give rise to disorder in both the interaction and decay rates in Eq.~(1). We model the disorder by averaging over many realizations. In each realization we choose the position of the emitters randomly according to a three dimensional Gaussian centered on each lattice site with equal widths on all three directions, $\sigma_{d}$, which is in practice determined by the lattice depth. We investigate the influence of the disorder on the survival probability and the fidelity of both storage mechanisms.

\begin{figure}[ht]
    \centering
    \includegraphics{fig_s4.pdf}
    \caption{\textbf{Disorder.} Influence of increasing disorder on storage with (a): flat dispersion relation [see Fig.~3] and (b): trapped states [see Fig.~4]. Each panel compares the excitation probability $\left< n_\alpha \right>_t$ at site $\alpha$, the survival probability $P_\mathrm{sur}(t)$, the fidelity $F(t)$ and the ratio of latter $F(t) / P_\mathrm{sur}(t)$ two without disorder (leftmost panel, blue lines) to the averages over $100$ realizations of disorders characterized by the widths $\sigma_d = 0.01d,\,...,\,0.05d$. We set $d=0.234\lambda$ and investigate wave packets (3) with (a): $k_s=k_0$ and $\sigma=0.1\pi/d$ on a lattice with perpendicular dipoles of $N=50$ emitters and (b): $k_s=k_0$ and $\sigma=0.103\pi/d$ on a ring lattice of the same size with dipole moments aligned parallel to its surface.}
    \label{fig:fig_s4}
\end{figure}

Analyzing the results in Fig.~\ref{fig:fig_s4}, we find that both storage mechanisms exhibit a similar robustness to the disorder. With increasing disorder, the survival probability as well as the fidelity decreases in comparison to the regular lattice spacing. However, the storage is still notably enhanced over the single atom case. We also plot the ratio $F(t) / P_\mathrm{sur}(t)$, which represents the dispersion of the wave packet conditioned to the photon not having been emitted. This ratio remains particularly high for the trapped state with $k_s=k_0$ [see Fig.~\ref{fig:fig_s4} (b)], meaning that if the excitation is still in the system after a time $t$, the state in which the photon is stored will still be the wave packet (3).

% -------------------------------------------------------------------------------------------------------------------------------------------- %








% -------------------------------------------------------------------------------------------------------------------------------------------- %


%%%%%%%%%%%%%%%%%%%%%%%%%%%%%%%%%%%%%%%%%%%%%%%%%%%%%%%%%%%%%%%
%% COMMENT THIS OUT FOR ARXIV VERSION
%% YOU NEED IT IF YOU WOULD LIKE TO COMPILE THE APPENDIX ONLY

%\clearpage
%\bibliography{references.bib}
%%%%%%%%%%%%%%%%%%%%%%%%%%%%%%%%%%%%%%%%%%%%%%%%%%%%%%%%%%%%%%%

\end{document}