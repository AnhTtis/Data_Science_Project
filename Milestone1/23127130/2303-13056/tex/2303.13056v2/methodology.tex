\section{Background}
Consider an N-body system with particles distributed on a uniform grid with positions $\mathbf{q}$. Let $\mathbf{\Psi}_{ZA}(\mathbf{q})$ be their linear Zel'dovich approximation (ZA) approximation at redshift $z=0$ (current time). Thus, the final positions of the particles when they evolve linearly is
\begin{align}
\mathbf{x}_{lin}(\mathbf{q}) = \mathbf{q} + \mathbf{\Psi}_{ZA}(\mathbf{q}).    
\end{align}
Let the final nonlinear displacement of the particle initially at grid site $\mathbf{q}$ be $\mathbf{\Psi}_{NL}(\mathbf{q})$. Thus, the final positions of the particles at redshift $z=0$ under nonlinear evolution is 
\begin{align}
    \mathbf{x}_{nonlin}(\mathbf{q}) = \mathbf{q} + \mathbf{\Psi}_{NL}(\mathbf{q}).
\end{align}
In this work, we investigate the problem of predicting the linear displacement field at redshift $z=0$, given the non-linear displacement field and the cosmological parameters.

\section{Methodology}

\begin{figure*}[h!]
\begin{center}
\centerline{\includegraphics[width=0.8\textwidth]{images/N128/slices_dis.png}}
\caption{Qualitative comparison of the $x,y, \text{and } z$ displacements for a $128 \times 128$ slice of particles from the original nonlinear field and the nonlinear field generated from the linear field predicted by our inverse model. The nonlinear field is generated from the forward modeling emulator \cite{jamieson}. The slice shown here is the same as the one shown in Figure \ref{slices:lin} for a one-to-one comparison.}
\label{slices:dis}
\end{center}
\end{figure*}

We train a \albert{convolutional neural network (CNN)} that takes the nonlinear displacement field $\mathbf{\Psi}_{NL}(\mathbf{q})$ at redshift $z=0$ and the value of $\Omega_m$ as inputs and predicts the linear displacement field $\mathbf{\Psi}_{ZA}(\mathbf{q})$ at redshift $z=0$.
% \sh{I feel like this is a weird transition, is it possible we make the following couple paragraphs a bit more like "background materials", and then go into discussing the methodology?} 
% We train our CNN to output the linear displacement field $\mathbf{\Psi}_{ZA}(\mathbf{q})$ for a given nonlinear displacement field $\mathbf{\Psi}_{NL}(\mathbf{q})$, and a given value of the $\Omega_m$.
We directly use the \albert{V-Net} architecture used in \cite{jamieson} and train it using 100 pairs of nonlinear and linear displacement fields. In terms of the training procedure, our method is almost identical to \cite{jamieson}, with the \albert{main} difference being that we pass the nonlinear displacement field as the input to our model and ask it to predict the linear displacement field. %This is exactly opposite to what was being done by \cite{jamieson}. %The source code of our implementation and experiments is available at \href{https://github.com/vaibhavjindal/map2map/tree/inverse\_mapping}{github.com/vaibhavjindal/map2map/tree/inverse\_mapping}.
\vj{One of the key advantages of using a simple deterministic V-Net, instead of a generative model such as a diffusion model, is the reduced computational complexity during both training and inference phases. Deterministic networks do not involve complex stochastic sampling processes, eliminating the need for numerous forward passes which makes them significantly faster and more resource-efficient than the generative models.}

For our experiments, we train our model using simulations of $128^3$ uniformly distributed particles in a square box with a side-length of $250 \ \mathrm{Mpc}/h$. This corresponds to a Nyquist wavenumber of $k = 1.608 ~\mathrm{Mpc}^{-1}~h$, the theoretical limit beyond which we can't trust the predictions of any model for this setting. For the training data, we randomly generated 100 linear fields for a fixed set of cosmological parameters ($\Omega_m=0.300, \Omega_b=0.050, h=0.700, n_s =0.965,\sigma_{8} = 0.799$). \albert{Since running N-body simulations on these initial conditions are too computationally expensive, we use the emulator by \cite{jamieson} to generate their corresponding nonlinear displacement fields as our model's training targets, which are already highly accurate for our purpose.}

% For out-of-distribution (OOD) evaluation on unseen cosmological parameters, we also evaluate the performance of our model on simulations from the Quijote suite \cite{villaescusa}.  
% Specifically, we rank the 2000 Quijote simulations on the basis of the sum of the relative differences in $\Omega_m$ and $\sigma_8$ from the cosmological parameters of our training data. We choose $\Omega_m \text{and } \sigma_8$ for this ranking as these parameters have the highest effect on the simulated distributions. Then, we evaluate our model on different percentiles to observe the effect of distribution-shift on our model.  Each linear and nonlinear field from the Quijote suite consists of $512^3$ particles spread uniformly in a cube with side length of $1000$ Mpc/$h$.} \drew{Note that the performance of the model is unaffected by the box size as long as the mean particle density is fixed.}

The architecture of our model involves three downsampling and upsampling components each and provides a field-of-view of $96^3$. That is, a focal particle's initial displacement is predicted based on its environment out to its $48^{\mathrm{th}}$ neighbors on the initial particle grid, corresponding to a distance of $93.75~\ \mathrm{Mpc}/h$. This finite field-of-view has a benefit: the evolution of the system on large scales is accurately described by linear theory, so the V-Net model preserves this linear evolution on scales larger than the field-of-view while allowing for nonlinear evolution at smaller (medium range of) scales.

On very small scales, the particles clustered tightly into dark matter halos where their orbits become complicated and difficult to predict. This small-scale clustering imposes a resolution limit on the inverse mapping, making it a one-to-many mapping as numerical errors and floating point precision make the 
%as the fairly lean 
neural network model unable to fully encode the detailed textures of this virialized motion. This limitation is also present in the forward model \cite{jamieson} \albert{and the diffusion model \cite{legin2023posterior}}, which tends to make halos more diffuse on small scales than in the simulations, blurring the details of these sharp structures. For the inverse mapping, the initial linear displacements are blurred for particles that end up inside of halos, limiting the accuracy of the inverse model predominantly on these very small scales \albert{as shown in Section \ref{sec:two-point-corr}.}

Sections \ref{sec:qualitative-analysis}, \ref{sec:one-point-stat}, and \ref{sec:two-point-corr} show the in-distribution performance of our model on unseen pairs of fields generated from the forward direction emulator \cite{jamieson} with the same cosmological parameters as its training data. We also evaluate the out-of-distribution (OOD) performance of our model on the Quijote simulation suite \cite{villaescusa} in Section \ref{sec:ood}.

