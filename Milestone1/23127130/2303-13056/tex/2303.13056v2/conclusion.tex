\section{Conclusion}

In this work, we pioneer \albert{using a CNN} to predict the initial linear field of an N-body simulation given the final nonlinear field and the cosmological parameters. We show that despite the many-to-one nature of the inverse mapping, our \albert{CNN} can still recover the linear fields \albert{for a wide range of scales, even including the smaller scales where the nonlinear physics of gravitational clustering becomes important. In addition, we empirically demonstrate that our model generalizes reasonably well to OOD cosmological parameters.} This suggests that neural networks can be used as approximate, inverse-mapping black boxes for generating trustworthy initial states for more fine-grained sampling-based inverse modeling methods. \albert{Overall, our work demonstrates that despite being much simpler and computationally cheaper than other approaches such as BORG and diffusion models, a deterministic neural network is already capable of reconstructing surprisingly accurate initial conditions of the universe.}

\albert{In future works, we plan to explore sampling-based methods (e.g. Hamiltonian Monte Carlo \cite{Radford}, Active Learning \cite{activelearning}, etc.) with initial states proposed by our model. We expect that these methods will refine the initial conditions, achieving high accuracy at even smaller scales. Another direction for future work is to enhance our model to additionally output uncertainty estimates. This will involve uncertainty quantification (UQ) methods such as optimizing the Gaussian negative log-likelihood loss \cite{gnll}, re-architecting our model as a Bayesian neural network \cite{bnn}, or augmenting our model with conformal predictions \cite{conformal}.}
