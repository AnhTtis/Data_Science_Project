\documentclass{amsart}

\usepackage{amsfonts,mathrsfs}
\usepackage{amsmath,amssymb,amsthm, amscd, bm}
\usepackage{tikz}
\usetikzlibrary{arrows,calc,matrix,topaths,positioning,scopes,shapes,decorations,decorations.markings} 
\usepackage{dsfont}
\usepackage{epsfig}
\usepackage{tikz-cd}
\usepackage{hyperref}
\usepackage{fullpage}
\usepackage[foot]{amsaddr}
\usepackage{adjustbox}
\usepackage{enumitem}
\newcommand{\subscript}[2]{$#1 _ #2$}
\usepackage[utf8x]{inputenc}


\makeatletter



\renewcommand{\email}[2][]{%
  \ifx\emails\@empty\relax\else{\g@addto@macro\emails{,\space}}\fi%
  \@ifnotempty{#1}{\g@addto@macro\emails{\textrm{(#1)}\space}}%
  \g@addto@macro\emails{#2}%
}
\makeatother

\author{Hiroaki  Karuo$^{(1)}$}
\address{${}^{(1)}$ Gakushuin University, Faculty of Sciences, Department of mathematics, 1-5-1 Mejiro, Toshima-ku, Tokyo 171-8588 Japan}
\email{hiroaki.karuo@gakushuin.ac.jp}
\author{Julien Korinman$^{(2)}$}
\address{${}^{(2)}$ Institut Montpelli\'erain Alexander Grothendieck - UMR 5149 Universit\'e de Montpellier. Place Eug\'ene Bataillon, 34090 Montpellier France}
\email{julien.korinman@gmail.com}
\urladdr{https://sites.google.com/site/homepagejulienkorinman/}

\subjclass{$57$R$56$,  $57$M$25$.}

\keywords{Stated skein algebras, quantum cluster algebras, quantum groups,  TQFTs, lattice gauge field theory}


\def\restriction#1#2{\mathchoice
              {\setbox1\hbox{${\displaystyle #1}_{\scriptstyle #2}$}
              \restrictionaux{#1}{#2}}
              {\setbox1\hbox{${\textstyle #1}_{\scriptstyle #2}$}
              \restrictionaux{#1}{#2}}
              {\setbox1\hbox{${\scriptstyle #1}_{\scriptscriptstyle #2}$}
              \restrictionaux{#1}{#2}}
              {\setbox1\hbox{${\scriptscriptstyle #1}_{\scriptscriptstyle #2}$}
              \restrictionaux{#1}{#2}}}
\def\restrictionaux#1#2{{#1\,\smash{\vrule height .8\ht1 depth .85\dp1}}_{\,#2}} 


\newcommand{\quotient}[2]{{\raisebox{.2em}{$#1$}\left/\raisebox{-.2em}{$#2$}\right.}}

\newcommand{\sslash}{\mathbin{/\mkern-6mu/}}
\newcommand{\Hom}{\operatorname{Hom}}
\newcommand{\tr}{\operatorname{tr}}
\newcommand{\Tr}{\operatorname{Tr}}
\newcommand{\qtr}{\operatorname{qtr}}
\newcommand{\SL}{\operatorname{SL}}
\newcommand{\id}{id}
\newcommand{\Span}{\operatorname{Span}}
\newcommand{\End}{\operatorname{End}}
\newcommand{\GL}{\operatorname{GL}}
\newcommand{\PGL}{\operatorname{PGL}}
\newcommand{\Vect}{\operatorname{Vect}}
\newcommand{\Specm}{\operatorname{MaxSpec}}
\newcommand{\Mod}{\operatorname{Mod}}
\newcommand{\qdim}{\operatorname{qdim}}
\newcommand{\St}{\operatorname{St}}
\newcommand{\Mat}{\operatorname{Mat}}
\newcommand{\skein}{\mathcal{S}_A}
\newcommand{\Aut}{\operatorname{Aut}}
\newcommand{\Rib}{\operatorname{Rib}}
\newcommand{\Rker}{\operatorname{Rker}}
\newcommand{\LKer}{\operatorname{Lker}}
\newcommand{\col}{\operatorname{col}}
\newcommand{\Gp}{\operatorname{Gp}}
\newcommand{\Var}{\operatorname{Var}}
\newcommand{\CAlg}{\operatorname{CAlg}}
\newcommand{\Comod}{\operatorname{Comod}}
\newcommand{\Arc}{\operatorname{Arc}}
\newcommand{\Mut}{\operatorname{Mut}}
\newcommand{\Ob}{\operatorname{Ob}}
\newcommand{\res}{\operatorname{res}}
\newcommand{\Indecomp}{\operatorname{Indecomp}}
\newcommand{\uq}{\mathfrak{u}_q\mathfrak{sl}_2}
\newcommand{\Ext}{\operatorname{Ext}}
\newcommand{\Rep}{\operatorname{Rep}}
\newcommand{\Image}{\operatorname{Im}}
\newcommand{\QMA}{unrestricted quantum moduli algebra}
\newcommand{\QMAs}{unrestricted quantum moduli algebras}


\newcommand{\qbinom}[2]{
\left[ 
\begin{array}{l} #1 \\ #2 \end{array}
\right]
}




\newcommand{\Crosspos}{
\tikz[baseline=-0.4ex,scale=0.5,>=stealth]{	
\draw [fill=gray!45,gray!45] (-.6,-.6)  rectangle (.6,.6);
\draw[line width=1.2,-] (-0.4,-0.52) -- (.4,.53);
\draw[line width=1.2,-] (0.4,-0.52) -- (0.1,-0.12);
\draw[line width=1.2,-] (-0.1,0.12) -- (-.4,.53);
}}

\newcommand{\Crossposor}{
\tikz[baseline=-0.4ex,scale=0.5,>=stealth]{	
\draw [fill=gray!45,gray!45] (-.6,-.6)  rectangle (.6,.6);
\draw[line width=1.2,-stealth] (-0.4,-0.52) -- (.4,.53);
\draw[line width=1.2,-] (0.4,-0.52) -- (0.1,-0.12);
\draw[line width=1.2,-stealth] (-0.1,0.12) -- (-.4,.53);
}}

\newcommand{\Crossneg}{
\tikz[baseline=-0.4ex,scale=0.5,>=stealth]{	
\draw [fill=gray!45,gray!45] (-.6,-.6)  rectangle (.6,.6);
\draw[line width=1.2,-] (-0.4,0.53) -- (.4,-.52);
\draw[line width=1.2,-] (-0.4,-0.52) -- (-0.1,-0.12);
\draw[line width=1.2,-] (0.1,0.12) -- (.4,.53);
}}

\newcommand{\Crossnegor}{
\tikz[baseline=-0.4ex,scale=0.5,>=stealth]{	
\draw [fill=gray!45,gray!45] (-.6,-.6)  rectangle (.6,.6);
\draw[line width=1.2, stealth-] (-0.4,0.53) -- (.4,-.52);
\draw[line width=1.2,-] (-0.4,-0.52) -- (-0.1,-0.12);
\draw[line width=1.2, -stealth] (0.1,0.12) -- (.4,.53);
}}

\newcommand{\Arrowup}{
\tikz[baseline=-0.4ex,scale=0.5,>=stealth]{ 
\draw [fill=gray!45,gray!45] (-.6,-.6)  rectangle (.6,.6)   ;
\draw[line width=1.2, -stealth] (-0.4,-0.52) ..controls +(.3,.5).. (-.4,.53);
\draw[line width=1.2, -stealth] (0.4,-0.52) ..controls +(-.3,.5).. (.4,.53);
}}

\newcommand{\circleoriented}{
\tikz[baseline=-0.4ex,scale=0.5,rotate=90]{ 
\draw [fill=gray!45,gray!45] (-.6,-.6)  rectangle (.6,.6)   ;
\draw[line width=1.2,black, 
decoration={markings, mark=at position 0.1 with {\arrow{<}}},
        postaction={decorate}
] (0,0)  circle (.4)   ;
}}

\newcommand{\heightexch}[3]{
	\begin{tikzpicture}[baseline=-0.4ex,scale=0.5, >=stealth]
	\draw [fill=gray!60,gray!45] (-.7,-.75)  rectangle (.4,.75)   ;
	\draw[#1] (0.4,-0.75) to (.4,.75);
	\draw[line width=1.2] (0.4,-0.3) to (-.7,-.3);
	\draw[line width=1.2] (0.4,0.3) to (-.7,.3);
	\draw (0.65,0.3) node {\scriptsize{$#2$}}; 
	\draw (0.65,-0.3) node {\scriptsize{$#3$}}; 
	\end{tikzpicture}
}
\newcommand{\heightcurve}{
\begin{tikzpicture}[baseline=-0.4ex,scale=0.5]
\draw [fill=gray!20,gray!45] (-.7,-.75)  rectangle (.4,.75)   ;
\draw[-] (0.4,-0.75) to (.4,.75);
\draw[line width=1.2] (-.7,-0.3) to (-.4,-.3);
\draw[line width=1.2] (-.7,0.3) to (-.4,.3);
\draw[line width=1.15] (-.4,0) ++(-90:.3) arc (-90:90:.3);
\end{tikzpicture}
}

\newcommand{\heightcurveright}{
\begin{tikzpicture}[baseline=-0.4ex,scale=0.5]
\draw [fill=gray!20,gray!45] (-.7,-.75)  rectangle (.4,.75)   ;
\draw[-] (-0.7,-0.75) to (-.7,.75);
\draw[line width=1.2] (0.1,-0.3) to (.4,-.3);
\draw[line width=1.2] (0.1,0.3) to (.4,.3);
\draw[line width=1.15] (.1,0) ++(90:.3) arc (90:270:.3);
\end{tikzpicture}
}

\newcommand{\heightexchright}[3]{
	\begin{tikzpicture}[baseline=-0.4ex,scale=0.5, >=stealth]
	\draw [fill=gray!60,gray!45] (-.7,-.75)  rectangle (.4,.75)   ;
	\draw[#1] (-0.7,-0.75) to (-0.7,.75);
	\draw[line width=1.2] (0.4,-0.3) to (-.7,-.3);
	\draw[line width=1.2] (0.4,0.3) to (-.7,.3);
	\draw (-1,0.3) node {\scriptsize{$#2$}}; 
	\draw (-1,-0.3) node {\scriptsize{$#3$}}; 
	\end{tikzpicture}
}

\newcommand{\bigonheightcurve}[2]{
\begin{tikzpicture}[baseline=-0.4ex,scale=0.5,>=stealth]
\draw [fill=gray!45,gray!45] (-.7,-.75)  rectangle (.4,.75)   ;
\draw[->] (0.4,-0.75) to (.4,.75);
\draw[->] (-0.7,-0.75) to (-.7,.75);
\draw[line width=1.2] (0.4,-0.3) to (0,-.3);
\draw[line width=1.2] (0.4,0.3) to (0,.3);
\draw[line width=1.1] (0,0) ++(90:.3) arc (90:270:.3);
\draw (0.65,0.3) node {\scriptsize{$#1$}}; 
\draw (0.65,-0.3) node {\scriptsize{$#2$}}; 
\end{tikzpicture}
}

\newcommand{\bigonheightcurveright}[2]{
\begin{tikzpicture}[baseline=-0.4ex,scale=0.5,>=stealth]
\draw [fill=gray!45,gray!45] (-.7,-.75)  rectangle (.4,.75)   ;
\draw[->] (0.4,-0.75) to (.4,.75);
\draw[->] (-0.7,-0.75) to (-.7,.75);
\draw[line width=1.2] (-.7,-0.3) to (-.4,-.3);
\draw[line width=1.2] (-.7,0.3) to (-.4,.3);
\draw[line width=1.15] (-.4,0) ++(-90:.3) arc (-90:90:.3);
\draw (-1,0.3) node {\scriptsize{$#1$}}; 
\draw (-1,-0.3) node {\scriptsize{$#2$}}; 
\end{tikzpicture}
}

\newcommand{\diamoond}[1]{
\begin{tikzcd}
{} & \scriptsize{#1} \ar[ld, "b"] \ar[rd, "c"] & {} \\
u \ar[rd, "d"] & {} & v \ar[ld, "e"] \\
{} & w & {}
\end{tikzcd}
}


%\pagestyle{plain}

\begin{document}

\theoremstyle{plain}
\newtheorem{theorem}{Theorem}[section]
\newtheorem{proposition}[theorem]{Proposition}
\newtheorem{corollary}[theorem]{Corollary}
\newtheorem{lemma}[theorem]{Lemma}
\theoremstyle{definition}
\newtheorem{notations}[theorem]{Notations}
\newtheorem{convention}[theorem]{Convention}
\newtheorem{problem}[theorem]{Problem}
\newtheorem{definition}[theorem]{Definition}
\theoremstyle{remark}
\newtheorem{remark}[theorem]{Remark}
\newtheorem{conjecture}[theorem]{Conjecture}
\newtheorem{example}[theorem]{Example}
\newtheorem{strategy}[theorem]{Strategy}
\newtheorem{question}[theorem]{Question}

\title[Classification of semi-weight representations of reduced stated skein algebras]{Classification of semi-weight representations of reduced stated skein algebras}
%
%
%\author{Julien Korinman}
%

\date{}
\maketitle

%%%%%%%%%%%%%%%%%%%%%%%%%%%%%%%%%%%%%%%%%%%%%%%%%%%%%%%%%%%%%%%%%%%%%%%%%%%%%%%%



\begin{abstract} 
We classify the finite dimensional semi-weight representations of the  reduced stated skein algebras at odd roots of unity of connected essential marked surfaces which either have a boundary component with at least two boundary arcs or which do not have any unmarked boundary component.  We deduce computations of the PI-degrees and Azumaya loci of unreduced stated skein algebras of essential marked surfaces having at most one boundary puncture per boundary component and of the \QMAs  {} of lattice gauge field theory. 
\end{abstract}


%%%%%%%%%%%%%%%%%%%%%%%%%%%%%%%%%%%%%%%%%%%%%%%%%%%%%%%%%%%%%%%%%%%%%%%%%%%%%%%


\section{Introduction}

\subsection{Background on reduced stated skein algebras and their representations}\label{sec_background}

Let $\Sigma$ be an oriented compact surface and $\mathcal{A}$ a finite set of disjoint arcs embedded in $\partial \Sigma$. The pair $\mathbf{\Sigma}:=(\Sigma, \mathcal{A})$ will be called a \textit{marked surface}.
For $A\in \mathbb{C}^*$ a root of unity of odd order $N$, the \textit{reduced stated skein algebra} $\overline{\mathcal{S}}_A(\mathbf{\Sigma})$ was introduced in \cite{CostantinoLe19} as the quotient of the stated skein algebra by the kernel of Bonahon-Wong quantum trace. In particular, when $\mathcal{A}=\emptyset$, $\overline{\mathcal{S}}_A(\mathbf{\Sigma})$ is the usual Kauffman-bracket skein algebra.
Its representations appear in quantum hyperbolic geometry and are conjectured to form the building blocks of some $\SL_2$-HQFT (see \cite{BaseilhacBenedettiInvariant, BaseilhacBenedetti05, BaseilhacBenedetti15, BaseilhacBenedettiHTQFT, BaseilhacBenedettiLocRep}). Let $Z_{\mathbf{\Sigma}}$ denote the center of  $\overline{\mathcal{S}}_A(\mathbf{\Sigma})$. By \cite{KojuQuesneyClassicalShadows} (after the original work in \cite{BonahonWong1}), there exists a \textit{Frobenius morphism} 
$$ Fr_{\mathbf{\Sigma}} : \overline{\mathcal{S}}_{+1}(\mathbf{\Sigma}) \hookrightarrow Z_{\mathbf{\Sigma}}$$
which embeds the commutative algebra $\overline{\mathcal{S}}_{+1}(\mathbf{\Sigma}) $ at $A=+1$ into $Z_{\mathbf{\Sigma}}$. Let $Z_{\mathbf{\Sigma}}^0 \subset Z_{\mathbf{\Sigma}}$ denote the image of the Frobenius morphism. 
\par  A representation $r: \overline{\mathcal{S}}_A(\mathbf{\Sigma})\to \End(V)$ is a \textit{weight representation} if $V$ is semi-simple as a module over $Z_{\mathbf{\Sigma}}$ and is called a \textit{semi-weight representation} if $V$ is semi-simple as a $Z_{\mathbf{\Sigma}}^0$ module. The purpose of this paper is to make progresses towards the 

\begin{problem}\label{problem_classification}
Classify all finite dimensional  weight and semi-weight representations of  $\overline{\mathcal{S}}_A(\mathbf{\Sigma})$. 
\end{problem}

We will solve Problem \ref{problem_classification} in the case where $\mathbf{\Sigma}=(\Sigma, \mathcal{A})$ is a connected marked surface which either has a boundary component with at least two boundary arcs or which has at least one boundary arc and no unmarked boundary component.

 The center $Z_{\mathbf{\Sigma}}$ was computed in \cite{KojuAzumayaSkein} and is described as follows. 
 Partition the set of boundary components of $\Sigma$ into two subsets $\pi_0(\partial \Sigma)= \mathring{\mathcal{P}} \bigsqcup \Gamma^{\partial}$ where 
  $\mathring{\mathcal{P}}$ is the subset of boundary components  which do not intersect $\mathcal{A}$ and $\Gamma^{\partial}$ the boundary components which contain some boundary arcs. For each $p\in \mathring{\mathcal{P}}$, there is a central element $\gamma_p \in Z_{\mathbf{\Sigma}}$ and for each $\partial \in \Gamma^{\partial}$ there is an invertible central element $\alpha_{\partial} \in Z_{\mathbf{\Sigma}}$ such that $Z_{\mathbf{\Sigma}}$ is generated by the image of the Frobenius morphism together with the elements $\gamma_p$ and $\alpha_{\partial}^{\pm 1}$. Let $\widehat{X}(\mathbf{\Sigma}):=\Specm(Z_{\mathbf{\Sigma}})$, $X(\mathbf{\Sigma}):=\Specm(Z_{\mathbf{\Sigma}}^0)$ and $p: \widehat{X}(\mathbf{\Sigma})\to X(\mathbf{\Sigma})$ the map defined by the inclusion $Z_{\mathbf{\Sigma}}^0 \subset Z_{\mathbf{\Sigma}}$.
  
 For a point $\widehat{x} \in \widehat{X}(\mathbf{\Sigma})$ (i.e. a maximal ideal in $Z_{\mathbf{\Sigma}}$), we denote by $\chi_{\widehat{x}} : Z_{\mathbf{\Sigma}} \to \mathbb{C}$ the corresponding character with kernel $\widehat{x}$ and use similar notations for $X(\mathbf{\Sigma})$. The variety  $\widehat{X}(\mathbf{\Sigma})$ is then described as 
\begin{multline*}  \widehat{X}(\mathbf{\Sigma}) := \{ \widehat{x}= (x, h_p, h_{\partial})_{p,\partial}, x\in X(\mathbf{\Sigma}),
 h_p\in \mathbb{C} \mbox{ is s.t. } T_N(h_p)=\chi_x(\gamma_p) \mbox{ for }p\in \mathring{P}, 
\\ h_{\partial} \in \mathbb{C}^* \mbox{is s.t. } h_{\partial}^N = \chi_x(\alpha_{\partial}) \mbox{ for }\partial \in \Gamma^{\partial} \}.\end{multline*}
 Here $T_N(X)$ denotes the $N$-th Chebyshev polynomial of first kind. The morphism $p$ is just the projection $p(x, h_p, h_{\partial})= x$. It was also proved in \cite{KojuAzumayaSkein} that $\overline{\mathcal{S}}_A(\mathbf{\Sigma})$ is free over $Z_{\mathbf{\Sigma}}$ of rank $R$ which is a perfect square: $R=D^2$ for some integer $D=D_{\mathbf{\Sigma}}$ named the \textit{PI-degree} of $\overline{\mathcal{S}}_A(\mathbf{\Sigma})$.
 \par For $\widehat{x} \in \widehat{X}(\mathbf{\Sigma})$ let $\mathcal{I}_{\widehat{x}} \subset \overline{\mathcal{S}}_A(\mathbf{\Sigma})$ be the ideal generated by elements $z-\chi_{\widehat{x}}(z)$ for $z\in Z_{\mathbf{\Sigma}}$ and write 
 $$ \overline{\mathcal{S}}_A(\mathbf{\Sigma})_{\widehat{x}}:= \quotient{ \overline{\mathcal{S}}_A(\mathbf{\Sigma})}{\mathcal{I}_{\widehat{x}}}.$$
An indecomposable weight representation $r$ sends the central elements to scalars, thus induces a point $\widehat{x_r} \in \widehat{X}(\mathbf{\Sigma})$, named its \textit{ classical shadow}  such that $r$ factorizes to a representation of the finite dimensional algebra $\overline{\mathcal{S}}_A(\mathbf{\Sigma})_{\widehat{x_r}}$. As we shall prove, for a  semi-weight indecomposable module $P$ every simple submodule and every simple subquotient have the same classical shadow which we call the classical shadow of $P$.
\par 
The \textit{Azumaya locus} of $\overline{\mathcal{S}}_A(\mathbf{\Sigma})$ is 
$$ \mathcal{AL}(\mathbf{\Sigma}):= \{ \widehat{x} \in \widehat{X}(\mathbf{\Sigma})\mbox{ such that } \overline{\mathcal{S}}_A(\mathbf{\Sigma})_{\widehat{x}}\cong \Mat_D(\mathbb{C}) \}.$$
Since the matrix algebra $\Mat_D(\mathbb{C}) $ has a unique indecomposable representation (up to isomorphism) which is irreducible, for each $\widehat{x} \in \widehat{X}(\mathbf{\Sigma})$ there exists a unique indecomposable weight representation with classical shadow $\widehat{x}$. 
\par A marked surface $\mathbf{\Sigma}=(\Sigma, \mathcal{A})$ is said \textit{essential} if each of its connected component contains at least one boundary arc in $\mathcal{A}$. 
In this case, it is proved in \cite{KojuQuesneyClassicalShadows} that the variety $X(\mathbf{\Sigma})$ admits a geometric interpretation as follows. For each boundary arc $a\in \mathcal{A}$, fix a point $v_a \in a$ and write $\mathbb{V}:=\{v_a, a\in \mathcal{A}\}$. Let $\Pi_1(\Sigma, \mathbb{V})$ denote the full subcategory of the fundamental groupoid $\Pi_1(\Sigma)$ generated by $\mathbb{V}$; so the set of objects of $\Pi_1(\Sigma, \mathbb{V})$ is $\mathbb{V}$ and a morphism $\alpha : v_a \to v_b$ is an homotopy class of path starting at $v_a$ and ending at $v_b$. Let $\SL_2$ be the category with a single element whose endomorphism set is $\SL_2(\mathbb{C})$ and let $\mathcal{R}_{\SL_2}(\mathbf{\Sigma})$ denote the set of functors $\rho : \Pi_1(\Sigma, \mathbb{V})\to \SL_2$. In  \cite{KojuQuesneyClassicalShadows} it is proved that $X(\mathbf{\Sigma})$ embeds into $\mathcal{R}_{\SL_2}(\mathbf{\Sigma})$ (see Section \ref{sec_moduli_spaces} for details).
For $x\in X(\mathbf{\Sigma})$, let $\rho_x$ be the associated functor. 
 For $p\in \mathring{\mathcal{P}}$, let $a$ be boundary arc in the same connected component than $p$ and let $\alpha_p \in \pi_1(\Sigma, v_a)$ be the homotopy class of a simple closed curve which bounds an annulus whose second boundary component is $p$. Then $x$ is called \textit{central at } $p$ if $\rho_x(\alpha_p)=\pm \mathds{1}_2$. This property is clearly independent on the choice of $a$ and $\alpha_p$. 
\vspace{2mm}
\par It was proved in \cite{FrohmanKaniaLe_UnicityRep} for unmarked surfaces and in \cite{KojuAzumayaSkein} for marked surfaces that $\mathcal{AL}(\mathbf{\Sigma})$ is open dense so a \textit{generic} indecomposable weight representation is irreducible and determined by its classical shadow. However the computation of the Azumaya locus remains a challenging problem. For closed (and thus unmarked) surfaces, the Azumaya loci was computed in \cite{KojuKaruo_Azumaya}. For $\mathbf{\Sigma}=(\Sigma_{g,1}, \{a\})$ a genus $g$ surface with a single boundary component and a single boundary arc, it was proved in \cite{KojuMCGRepQT} that $ \overline{\mathcal{S}}_{A}(\mathbf{\Sigma})$ is Azumaya, i.e. that $\mathcal{AL}(\mathbf{\Sigma})=\widehat{X}(\mathbf{\Sigma})$. For $\mathbf{\Sigma}=\mathbb{T}:=(\mathbb{D}^2, \{a,b,c\})$ a disc with three boundary arcs (usually called the triangle), it was showed in \cite{CostantinoLe19} that $ \overline{\mathcal{S}}_{A}(\mathbb{T})$ is isomorphic to a quantum torus, thus it is Azumaya as well. 


\subsection{Main results}

 In order to state our main theorem, let us introduce some notations. Write $\Delta_+(\widehat{\mathfrak{sl}}_2):= \{ (n,m) \in \mathbb{N}^2 | (n-m)^2 \leq 1\}$ and decompose it as 
   $\Delta_+(\widehat{\mathfrak{sl}}_2)= \{(0,1), (1,0)\} \bigsqcup \Delta^{\Re}_{++}(\widehat{\mathfrak{sl}}_2)\bigsqcup \Delta^{Im}_+(\widehat{\mathfrak{sl}}_2)$ where 
 $$  \Delta^{\Re}_{++}(\widehat{\mathfrak{sl}}_2)= \{ (k, k+1), k\geq1\} \cup \{(k+1, k), k\geq 1\}, \quad \Delta^{Im}_+(\widehat{\mathfrak{sl}}_2)=\{ (k,k), k\geq 1\}.$$
 Further write 
 $$ \Delta:= \{S, P\} \sqcup   \Delta^{\Re}_+(\widehat{\mathfrak{sl}}_2) \sqcup  \Delta^{Im}_+(\widehat{\mathfrak{sl}}_2)\times \mathbb{CP}^1, $$ 
where $S,P$ are some formal parameters and write $\Delta \sqcup \overline{\Delta}:= \Delta \times \{0,1\}$ (two copies of $\Delta$ where we denote an element $(\alpha, 0)$ simply by $\alpha$ and an element $(\alpha, 1)$ by $\overline{\alpha}$). Let $\widehat{x}=(x,h_p,h_{\partial})_{p, \partial} \in \widehat{X}(\mathbf{\Sigma})$ and decompose the set of inner punctures as $\mathring{\mathcal{P}}= \mathring{\mathcal{P}}_0 \sqcup \mathring{\mathcal{P}}_1 \sqcup \mathring{\mathcal{P}}_2$ where 
\begin{align*}
&  \mathring{\mathcal{P}}_2:= \{p\in \mathring{\mathcal{P}}, \mbox{ such that } x \mbox{ is central at }p \mbox{ and }h_p \neq \pm 2\} \\
&  \mathring{\mathcal{P}}_1:= \{p\in \mathring{\mathcal{P}}\setminus \mathring{\mathcal{P}}_2, \mbox{ such that } \chi_x(\gamma_p) = \pm 2 \mbox{ and }h_p \neq \pm 2\}.
\end{align*}
A map $\sigma : \mathring{\mathcal{P}}\to \Delta \bigsqcup \overline{\Delta} $ is called a \textit{coloring compatible with} $\widehat{x}$ if $(1)$ for $p\in \mathring{\mathcal{P}}_0$, then $\sigma(p)=S$ and $(2)$ if $p\in \mathring{\mathcal{P}}_1$, then $\sigma(p)\in \{S,P\}$. By convention, if $\mathring{\mathcal{P}}=\emptyset$, we consider that every $\widehat{x}$ has a unique coloring. We write
$$ m:= | \mathring{\mathcal{P}}_2| \quad \mbox{ and }\quad m':= |\mathring{\mathcal{P}}_1\cup \mathring{\mathcal{P}}_2|.$$
\par 
Let $\widehat{X}^{reg}(\mathbf{\Sigma})$ denote the regular part (smooth points) of $\widehat{X}(\mathbf{\Sigma})$. Let $\mathcal{C}$ be the category of weight $\overline{\mathcal{S}}_A(\mathbf{\Sigma})$ modules and $\overline{\mathcal{C}}$ be the category of semi-weight modules. 


 The main result of this paper is the following

\begin{theorem}\label{main_theorem_intro} Let $\mathbf{\Sigma}=(\Sigma, \mathcal{A})$ be a connected essential marked surface  which either has a boundary component with at least two boundary arcs or which does not have any inner puncture.
 Let  $\widehat{x}=(x, h_p, h_{\partial}) \in \widehat{X}(\mathbf{\Sigma})$.
\begin{enumerate}
\item $\widehat{x}\in \mathcal{AL}(\mathbf{\Sigma})$ if and only if $m=0$; 
\item $\widehat{x} \in \widehat{X}^{reg}(\mathbf{\Sigma})$ if and only if $m'=0$ (so $\widehat{X}^{reg}(\mathbf{\Sigma})\subset \mathcal{AL}(\mathbf{\Sigma})$); 
\item the indecomposable semi-weight representations with classical shadow $\widehat{x}$ are in $1:1$ correspondence with the set of coloring compatible with $\widehat{x}$; 
\item the indecomposable weight representations with classical shadow $\widehat{x}$ correspond to the colorings taking values in $\{S,\overline{S}, ((1,1), 1), \overline{((1,1), 1)}\}$: there are thus $4^m$ such representations; 
\item the irreducible representations with classical shadow $\widehat{x}$ correspond to the colorings taking values in $\{S, \overline{S}\}$: there are thus $2^m$ such representations;
\item the indecomposable representations with classical shadow $\widehat{x}$ which are projective objects in $\mathcal{C}$ correspond to the colorings sending the elements of $\mathring{\mathcal{P}}_1$ to $S$ and the elements of $\mathcal{P}_2$ to an element in $\{((1,1), 1), \overline{((1,1), 1)}\}$: there are thus $2^m$ such representations; 
\item the indecomposable representations with classical shadow $\widehat{x}$ which are projective objects in $\overline{\mathcal{C}}$ correspond to the colorings sending the elements of $\mathring{\mathcal{P}}_1$ to $P$ and the elements of $\mathring{\mathcal{P}}_2$ to an element in $\{P, \overline{P}\}$: there are thus $2^m$ such representations.
\end{enumerate}
\end{theorem}

We will construct explicitly the  indecomposable semi weight representations of Theorem \ref{main_theorem_intro}, so it solves Problem \ref{problem_classification} for these marked surfaces. 
\vspace{2mm}
\par The proof of Theorem \ref{main_theorem_intro} uses a preliminary result which is interesting on its own.  For $\mathbf{\Sigma}_1, \mathbf{\Sigma}_2$ two  marked surfaces and $a_i$ a boundary arc of $\mathbf{\Sigma}_i$ ($i=1,2$), we denote by $\mathbf{\Sigma}_1\cup_{a_1\# a_2} \mathbf{\Sigma}_2$ the marked surface obtained by gluing $a_1$ and $a_2$ together. There exists an injective algebra morphism, named the \textit{splitting morphism}
$$ \theta_{a_1 \# a_2}: \overline{\mathcal{S}}_A(\mathbf{\Sigma}_1\cup_{a_1 \# a_2} \mathbf{\Sigma}_2) \hookrightarrow \overline{\mathcal{S}}_A(\mathbf{\Sigma}_1)\otimes \overline{\mathcal{S}}_A(\mathbf{\Sigma}_2).$$
If $V_1$ and $V_2$ are modules over  $\overline{\mathcal{S}}_A(\mathbf{\Sigma}_1)$ and $ \overline{\mathcal{S}}_A(\mathbf{\Sigma}_2)$, then by precomposing with $ \theta_{a_1 \# a_2}$, 
$V_1\otimes V_2$ becomes a module over $\overline{\mathcal{S}}_A(\mathbf{\Sigma}_1\cup_{a_1 \# a_2} \mathbf{\Sigma}_2)$.


 \begin{theorem}\label{theorem_gluing_intro} Let  $\mathbf{\Sigma}_1$ and $\mathbf{\Sigma}_2$ be marked surfaces and for $i=1,2$ let $a_i$ be a boundary arcs of $\mathbf{\Sigma}_i$ such that the connected component of $\partial \Sigma_i$ which contains $a_i$ also contains at least another boundary arc. 
 Write $\mathbf{\Sigma}:=\mathbf{\Sigma}_1 \cup_{a_1\# a_2} \mathbf{\Sigma}_2$.
 Then  any indecomposable (semi) weight $\overline{\mathcal{S}}_A(\mathbf{\Sigma})$-module is isomorphic to a module $V_1\otimes V_2$ with $V_i$ a (semi) weight indecomposable $ \overline{\mathcal{S}}_A(\mathbf{\Sigma}_i)$ module. Conversely, any such $ \overline{\mathcal{S}}_A(\mathbf{\Sigma})$ module $V_1\otimes V_2$ is (semi) weight indecomposable.
\end{theorem}

The proof of Theorem \ref{main_theorem_intro} goes as follows. A 2P marked surface is a connected marked surface which has at least two boundary arcs and no unmarked boundary component.
When $\mathbf{\Sigma}$ is 2P marked surfaces, it follows from the work of M\"uller \cite{Muller} and L\^e-Yu \cite{LeYu_SSkeinQTraces} that $\overline{\mathcal{S}}_A(\mathbf{\Sigma})$ is isomorphic to a quantum cluster algebra. The Azumaya loci of quantum cluster algebras have been studied in \cite{MNTY_AzumayaClusterAlgebras} and we will deduce from this study that $\overline{\mathcal{S}}_A(\mathbf{\Sigma})$ is Azumaya.
For $\mathbf{\Sigma}=\mathbb{D}_1:=(\Sigma_{0,2}, \{a,b\})$ an annulus with two boundary arcs in the same boundary component, it was proved in \cite{KojuQGroupsBraidings} that $ \overline{\mathcal{S}}_{A}(\mathbb{D}_1) \cong U_q\mathfrak{gl}_2$ is isomorphic to the (simply connected) quantum enveloping algebra of $\mathfrak{gl}_2$ whose representation theory is very similar to the one of $U_q\mathfrak{sl}_2$.
The simple representations of $U_q\mathfrak{sl}_2$ were classified in \cite{DeConciniKacRepQGroups, ArnaudonRoche}, the finite dimensional indecomposable representations of the small quantum group $\mathfrak{u}_q\mathfrak{sl}_2$ were classified in \cite{Suter_Uqsl2Modules},  and many more indecomposable representations were found in \cite{Arnaudon_Uqsl2Rep}. However, at the authors knowledge, the indecomposable $U_q\mathfrak{sl}_2$ modules have never been classified. We will provide a classification of all indecomposable semi-weight $U_q\mathfrak{gl}_2$ modules and prove that 
Theorem \ref{main_theorem_intro}  holds for $\mathbb{D}_1$. If  $\mathbf{\Sigma}$ is a connected marked surface which has a boundary component with at least two boundary arcs, it can be obtained by gluing several copies of $\mathbb{D}_1$ to a 2P marked surface so Theorem \ref{main_theorem_intro} will follow from Theorem \ref{theorem_gluing_intro}. When $\mathbf{\Sigma}$ is an essential marked surface without inner puncture, it is either a 2P marked surface or is has a single boundary arc in which case its reduced stated skein algebra is Azumaya by \cite{KojuMCGRepQT}. 
\vspace{2mm}
\par The techniques of the present paper can also be used to study the representations of the unreduced stated skein algebras. Let $\mathbf{\Sigma}$ be an essential marked surface and let $\mathbf{\Sigma}^*$ be the same marked surface where each boundary arc has been replaced by a pair of boundary arcs inside the same boundary component. L\^e and Yu constructed in \cite{LeYu_SSkeinQTraces} an embedding $j: \mathcal{S}_A(\mathbf{\Sigma}) \hookrightarrow \overline{\mathcal{S}}_A(\mathbf{\Sigma}^*)$ of the unreduced stated skein algebra of $\mathbf{\Sigma}$ into the reduced stated skein algebra of $\mathbf{\Sigma}^*$. Therefore every module over $\overline{\mathcal{S}}_A(\mathbf{\Sigma}^*)$ is also a module over $\mathcal{S}_A(\mathbf{\Sigma})$ so Theorem \ref{main_theorem_intro} provides a large family  of representations for the unreduced stated skein algebras as well. We will deduce the following:

 \begin{theorem}\label{theorem_Skein_intro}
 Let $\mathbf{\Sigma}=(\Sigma_{g,n}, \mathcal{A})$ be a connected essential marked surface of genus $g$ with $n$ boundary components such each boundary component contains at most one boundary arc. So it has $k:= n-|\mathcal{A}|$ inner punctures $p_1, \ldots, p_k$ and $|\mathcal{A}|$ boundary punctures $p_{\partial_1}, \ldots, p_{\partial_{|\mathcal{A}|}}$.
 \begin{enumerate}
 \item The center $\underline{Z}_{\mathbf{\Sigma}}$ of $\mathcal{S}_A(\mathbf{\Sigma})$ is generated by the image $\underline{Z}^0_{\mathbf{\Sigma}}$ of the Frobenius morphism together with the peripheral curves $\gamma_{p_1}, \ldots, \gamma_{p_k}$. So the closed points of $\widehat{\underline{X}}(\mathbf{\Sigma}):= \Specm(\underline{Z}_{\mathbf{\Sigma}})$ are in $1:1$ correspondence with the set of elements $\widehat{\rho}=(\rho, h_{p_1}, \ldots, h_{p_k})$ with $\rho: \Pi_1(\Sigma, \mathbb{V})\to \SL_2$ and $h_{p_i}\in \mathbb{C}$ is such that $T_N(h_{p_i})=-\tr(\rho(\gamma_{p_i}))$. 
 \item The PI-degree $\underline{D}_{\mathbf{\Sigma}}$ of  $\mathcal{S}_A(\mathbf{\Sigma})$ is equal to $N^{3g-3+n+2|\mathcal{A}|}(= D_{\mathbf{\Sigma}^*})$. 
 \item For $\rho: \Pi_1(\Sigma, \mathbb{V})\to \SL_2$, write $\mu(\rho):= (\rho(\alpha(p_{\partial_1})), \ldots, \rho(\alpha(p_{\partial_{|\mathcal{A}|}})) \in (\SL_2)^{|\mathcal{A}|}$.
 Let  $\widehat{\rho}=(\rho, h_{p_i}) \in \widehat{\underline{X}}(\mathbf{\Sigma})$.
 \begin{itemize}
 \item[(i)] If $\mu(\rho)\in (\SL_2^0)^{|\mathcal{A}|}$ and for each $1\leq i \leq k$, either $\tr(\rho(\gamma_{p_i}))\neq \pm 2$ or $\tr(\rho(\gamma_{p_i}))=\pm 2$ and $h_{p_i}=\mp 2$, then $\widehat{\rho}$ belongs to the Azumaya locus of $\mathcal{S}_A(\mathbf{\Sigma})$.
 \item[(ii)] Suppose that either $\mu(\rho)\in (\SL_2^1)^{|\mathcal{A}|}$ or that $\mu(\rho)\in (\SL_2^0)^{|\mathcal{A}|}$ and there exists $i$ such that $\tr(\rho(\gamma_{p_i}))=\pm 2$ and $h_{p_i}\neq \mp 2$, then $\widehat{\rho}$ does not belong to the Azumaya locus of $\mathcal{S}_A(\mathbf{\Sigma})$.
\end{itemize}
 \end{enumerate}
 \end{theorem}
In this theorem, we have considered the simple Bruhat decomposition $\SL_2= \SL_2^0 \sqcup \SL_2^1$ where $\SL_2^0$ is the subset of matrices $\begin{pmatrix} a & b \\ c & d\end{pmatrix}$ such that $a\neq 0$ and $\SL_2^1$ the set of such matrices with $a=0$. 
In the particular case where $|\mathcal{A}|=1$, i.e. when $\mathbf{\Sigma}=(\Sigma_{g,n+1}, \{a\})$ is a genus $g$ surface with $n+1$ boundary components and a single boundary arc, the algebra $\mathcal{L}_{g,n}:= \mathcal{S}_A(\mathbf{\Sigma})$ is called the \textit{\QMA} in lattice gauge field theory and first  appeared in the work of Buffenoir-Roche and Alekseev-Grosse-Schomerus (\cite{AlekseevGrosseSchomerus_LatticeCS1,AlekseevGrosseSchomerus_LatticeCS2, AlekseevSchomerus_RepCS, BuffenoirRoche, BuffenoirRoche2}). For these algebras, Theorem \ref{theorem_Skein} specializes to 

\begin{corollary}\label{coro_QMA_intro}
\begin{enumerate}
\item The center of $\mathcal{L}_{g,n}$ is generated by the image of the Frobenius morphism and the peripheral curves $\gamma_{p_1}, \ldots, \gamma_{p_n}$.
\item The PI-degree of  $\mathcal{L}_{g,n}$ is $N^{3g+n}$.
\item The Azumaya locus of $\mathcal{L}_{g,n}$ is the locus of elements $\widehat{\rho}=(\rho, h_{p_i})$ such that $\mu(\rho)\in \SL_2^0$ and $\tr(\rho(\gamma_{p_i}))=\pm 2 \Rightarrow h_{p_i}=\mp 2$.
\end{enumerate}
\end{corollary}
When $n=0$, Corollary \ref{coro_QMA_intro} was proved in \cite[Theorem $1$]{GanevJordanSafranov_FrobeniusMorphism}. When $g=0$, the first point of Corollary \ref{coro_QMA_intro} was proved in \cite[Theorem $1.1$]{BaseilhacRoche_LGFT1}, while the second point was proved in \cite[Theorem $1.3$]{BaseilhacRoche_LGFT2}. S.Baseilhac have informed the authors that an alternative proof of the whole Corollary \ref{coro_QMA_intro} and a generalization for arbitrary gauge group (the case considered here is $\SL_2$) will appear in \cite{BaseilhacFaitgRoche_LGFT4}. 


\vspace{2mm}
\par Reduced stated skein algebras have been recently generalized for $\SL_n$ gauge group in \cite{LeSikora_SSkein_SLN, LeYu_SLNQTraces} where similar relations with quantum cluster algebras are proved so it seems natural to expect that the techniques of the present paper should extend to higher rank cases. However, it is proved in \cite{FeldvossWitherspoon_RepSmallQG, FeldvossWitherspoon_RepSmallQG2} that the small quantum group $\mathfrak{u}_q\mathfrak{g}$ is wild for semi-simple Lie algebras $\mathfrak{g}$ of rank at least $3$: the problem of classifying its indecomposable representations is unsolvable. Therefore there is little hope that the problem of classifying the semi-weight indecomposable representations of $\SL_n$ reduced stated skein algebras for $n\geq 3$ could be solvable. 
\vspace{2mm}
\par Let us conclude this introduction by explaining why we choose to classify the semi-weight representations. The long-term objective is to construct a $\SL_2$ HQFT whose vector spaces associated to (decorated) surfaces would be indecomposable representations of the reduced stated skein algebras; such HQFT are expected to extend the construction in \cite{BCGPTQFT} and to refine the ones in \cite{BaseilhacBenedettiHTQFT} and  the links invariants of these conjectural HQFT were built in \cite{KojuQGroupsBraidings} using such representations. Since we want to decorate our surfaces by some flat connections thought as points in $X(\mathbf{\Sigma})$, it is natural to consider at least the weight representations. Moreover, in non semi-simple TQFTs \cite{BCGPTQFT} some spaces are semi-weight modules for the skein algebras which are not weight modules; by analyzing the construction carefully we see that this necessity follows from the observation in \cite{Arnaudon_Uqsl2Rep} that the category of semi-weight representations is the smallest  full subcategory of $U_q\mathfrak{sl}_2-\Mod$ which contains the simple modules and is stable by $\otimes$. So it seems natural to consider the class of semi-weight representations. 




\subsection{Plan of the paper}
In Section \ref{sec_sskein}, we review the definition and main properties of the reduced stated skein algebras. Section \ref{sec_moduli_spaces} is devoted to the study of the schemes $X(\mathbf{\Sigma})$ and $\widehat{X}(\mathbf{\Sigma})$.
 We then study the representation theory of $ \overline{\mathcal{S}}_{A}(\mathbb{D}_1) \cong U_q\mathfrak{gl}_2$ in Section \ref{sec_D1} and postpone most of the proofs to Appendix \ref{appendix}. We then study the case of 2P marked surfaces  in Section \ref{sec_cluster_algebras} using quantum cluster algebras. Theorem \ref{theorem_gluing_intro} is proved in Section \ref{sec_gluing} and  Theorem \ref{main_theorem_intro} is proved in Section \ref{sec_classification}. In Section \ref{sec_QMS}, we prove Theorem \ref{theorem_Skein_intro} and Corollary \ref{coro_QMA_intro}.

\vspace{2mm}\par 
\textit{Acknowledgments.} The authors thank S.Baseilhac, D.Calaque,  F.Costantino, M.Faitg,  T.L\^e and P.Roche  for valuable conversations. The skein interpretation of Alekseev's morphism which appears in Section \ref{sec_QMS} was suggested to us by T.L\^e which we warmly thank. J.K. acknowledges support from  the European Research Council (ERC DerSympApp) under the European Union’s Horizon 2020 research and innovation program (Grant Agreement No. 768679).
H.K. was partially supported by JSPS KAKENHI Grant Numbers JP22K20342, JP23K12976. 

 \section{Reduced stated skein algebras} \label{sec_sskein}
 
 \subsection{Definition of reduced stated skein algebras}
 
 \begin{definition}[Marked surfaces]\label{def_marked_surfaces}
 \begin{enumerate}
 \item A \textit{marked surface} is a pair $\mathbf{\Sigma}=(\Sigma, \mathcal{A})$ where $\Sigma$ is a compact oriented surface and $\mathcal{A}$ is a finite set of pairwise disjoint closed intervals embedded in $\partial \Sigma$ named \textit{boundary arcs}. 
 The orientation of $\Sigma$ induces an orientation of $\partial \Sigma$ and thus an orientation of each boundary arc.
 We impose that each boundary arc $a\subset \partial \Sigma$ is equipped with a parametrization $\varphi_a: [0,1] \hookrightarrow \partial \Sigma$ such that $\varphi_a([0,1])=a$ and such that $\varphi_a$ is an oriented embedding (where $[0,1]$ is oriented from $0$ to $1$). A marked surface  $\mathbf{\Sigma}=(\Sigma, \mathcal{A})$ is called \textit{unmarked} if $\mathcal{A}=\emptyset$ and is said \textit{essential} if each connected component of $\Sigma$ contains at least one boundary arc.
  \item A connected component of $\partial \Sigma \setminus \mathcal{A}$ is called a \textit{puncture} and, pictorially, we will represent them as a puncture $\bullet$. There are two kind of punctures: the ones homeomorphic to a circle, which correspond to unmarked boundary components are called \textit{inner punctures} and their set is denoted by $\mathring{\mathcal{P}}$; the ones homeomorphic to an open interval, which lyes between two boundary arcs of the same boundary component, are called \textit{boundary punctures} and their set is denoted by $\mathcal{P}^{\partial}$. We denote by $\Gamma^{\partial}$ the set of boundary components which intersect $\mathcal{A}$ non trivially. 
 \item If $\mathbf{\Sigma}_1=(\Sigma_1, \mathcal{A}_1)$ and $\mathbf{\Sigma}_2=(\Sigma_2, \mathcal{A}_2)$ are two marked surfaces and $a_1, a_2$ are boundary arcs of $\mathbf{\Sigma}_1$ and $\mathbf{\Sigma}_2$ respectively, we denote by $\mathbf{\Sigma}_1\cup_{a_1\# a_2} \mathbf{\Sigma}_2:= (\Sigma_1\cup_{a_1\# a_2}\Sigma_2, (\mathcal{A}_1 \cup \mathcal{A}_2)\setminus \{a_1, a_2\})  $ the marked surface obtained by gluing $a_1$ to $a_2$ where 
 $$ \Sigma_1\cup_{a_1\# a_2}\Sigma_2:= \quotient{ \Sigma_1 \bigsqcup \Sigma_2}{ (\varphi_{a_1}(1-t)=\varphi_{a_2}(t), t\in [0,1])}.$$
\end{enumerate}
 \end{definition}
 
 
 
 \begin{definition}[Tangles and diagrams]\label{def_tangles} Let $\mathbf{\Sigma}=(\Sigma, \mathcal{A})$ a marked surface.
 \begin{enumerate}
 \item  A \textit{tangle} in $ \mathbf{\Sigma} \times (0,1)$   is a  compact framed, properly embedded $1$-dimensional manifold $T\subset \Sigma\times (0,1)$ such that  $\partial T \subset \mathcal{A}\times (0,1)$ and 
 for every point of $\partial T$ the framing is parallel to the $(0,1)$ factor and points to the direction of $1$.  Here, by framing, we refer to a section of the unitary normal bundle of $T$. The \textit{height} of $(v,h)\in \Sigma\times (0,1)$ is $h$.  If $a\in \mathcal{A}$ is a boundary arc, we impose that no two points in $\partial_aT:= \partial T \cap a\times(0,1)$  have the same heights, hence the set $\partial_aT$ is totally ordered by the heights. Two tangles are isotopic if they are isotopic through the class of tangles that preserve the boundary height orders. By convention, the empty set is a tangle only isotopic to itself.
 \item Let $\pi : \Sigma\times (0,1)\to \Sigma$ be the projection with $\pi(v,h)=v$. A tangle $T$ is in \textit{standard position} if for each of its points, the framing is parallel to the $(0,1)$ factor and points in the direction of $1$ and is such that $\pi_{\big| T} : T\rightarrow \Sigma$ is an immersion with at most transversal double points in the interior of $\Sigma$. Every tangle is isotopic to a tangle in standard position. We call \textit{diagram}  the image $D=\pi(T)$ of a tangle in standard position, together with the over/undercrossing information at each double point. An isotopy class of diagram $D$ together with a total order of $\partial_a D:=\partial D\cap a$ for each boundary arc $a$, define uniquely an isotopy class of tangle. When choosing an orientation $\mathfrak{o}(a)$ of a boundary arc $a$ and a diagram $D$, the set $\partial_aD$ receives a natural order by setting that the points are increasing when going in the direction of $\mathfrak{o}(a)$. Pictorially, we will depict tangles by drawing a diagram and an orientation (an arrow) for each boundary arc, as in Figure \ref{fig_statedtangle}. When a boundary arc $a$ is oriented we assume that $\partial_a D$ is ordered according to the orientation.
 \item A connected open diagram without double point is called an \textit{arc}. A closed connected diagram without double point is called a \textit{loop}.
 \item  A \textit{state} of a tangle is a map $s:\partial T \rightarrow \{-, +\}$. A pair $(T,s)$ is called a \textit{stated tangle}. We define a \textit{stated diagram} $(D,s)$ in a similar manner.
 \end{enumerate}
 \end{definition}
 
 \begin{figure}[!h] 
\centerline{\includegraphics[width=6cm]{StatedTangle.eps} }
\caption{On the left: a stated tangle. On the right: its associated diagram. The arrows represent the height orders. } 
\label{fig_statedtangle} 
\end{figure} 
 
 
 \begin{definition}[Stated skein algebras](\cite{BonahonWongqTrace, LeStatedSkein})\label{def_skein}
 Let $k$ be a commutative unital ring and $A^{1/2}\in k^{\times}$ and invertible element.
 The \textit{stated skein algebra}  $\mathcal{S}_A(\mathbf{\Sigma})$ is the  free $k$-module generated by isotopy classes of stated tangles in $\mathbf{\Sigma}\times (0, 1)$ modulo the following relations \eqref{eq: skein 1} and \eqref{eq: skein 2}, 
  	\begin{equation}\label{eq: skein 1} 
\begin{tikzpicture}[baseline=-0.4ex,scale=0.5,>=stealth]	
\draw [fill=gray!45,gray!45] (-.6,-.6)  rectangle (.6,.6)   ;
\draw[line width=1.2,-] (-0.4,-0.52) -- (.4,.53);
\draw[line width=1.2,-] (0.4,-0.52) -- (0.1,-0.12);
\draw[line width=1.2,-] (-0.1,0.12) -- (-.4,.53);
\end{tikzpicture}
=A
\begin{tikzpicture}[baseline=-0.4ex,scale=0.5,>=stealth] 
\draw [fill=gray!45,gray!45] (-.6,-.6)  rectangle (.6,.6)   ;
\draw[line width=1.2] (-0.4,-0.52) ..controls +(.3,.5).. (-.4,.53);
\draw[line width=1.2] (0.4,-0.52) ..controls +(-.3,.5).. (.4,.53);
\end{tikzpicture}
+A^{-1}
\begin{tikzpicture}[baseline=-0.4ex,scale=0.5,rotate=90]	
\draw [fill=gray!45,gray!45] (-.6,-.6)  rectangle (.6,.6)   ;
\draw[line width=1.2] (-0.4,-0.52) ..controls +(.3,.5).. (-.4,.53);
\draw[line width=1.2] (0.4,-0.52) ..controls +(-.3,.5).. (.4,.53);
\end{tikzpicture}
\hspace{.5cm}
\text{ and }\hspace{.5cm}
\begin{tikzpicture}[baseline=-0.4ex,scale=0.5,rotate=90] 
\draw [fill=gray!45,gray!45] (-.6,-.6)  rectangle (.6,.6)   ;
\draw[line width=1.2,black] (0,0)  circle (.4)   ;
\end{tikzpicture}
= -(A^2+A^{-2}) 
\begin{tikzpicture}[baseline=-0.4ex,scale=0.5,rotate=90] 
\draw [fill=gray!45,gray!45] (-.6,-.6)  rectangle (.6,.6)   ;
\end{tikzpicture}
;
\end{equation}

\begin{equation}\label{eq: skein 2} 
\begin{tikzpicture}[baseline=-0.4ex,scale=0.5,>=stealth]
\draw [fill=gray!45,gray!45] (-.7,-.75)  rectangle (.4,.75)   ;
\draw[->] (0.4,-0.75) to (.4,.75);
\draw[line width=1.2] (0.4,-0.3) to (0,-.3);
\draw[line width=1.2] (0.4,0.3) to (0,.3);
\draw[line width=1.1] (0,0) ++(90:.3) arc (90:270:.3);
\draw (0.65,0.3) node {\scriptsize{$+$}}; 
\draw (0.65,-0.3) node {\scriptsize{$+$}}; 
\end{tikzpicture}
=
\begin{tikzpicture}[baseline=-0.4ex,scale=0.5,>=stealth]
\draw [fill=gray!45,gray!45] (-.7,-.75)  rectangle (.4,.75)   ;
\draw[->] (0.4,-0.75) to (.4,.75);
\draw[line width=1.2] (0.4,-0.3) to (0,-.3);
\draw[line width=1.2] (0.4,0.3) to (0,.3);
\draw[line width=1.1] (0,0) ++(90:.3) arc (90:270:.3);
\draw (0.65,0.3) node {\scriptsize{$-$}}; 
\draw (0.65,-0.3) node {\scriptsize{$-$}}; 
\end{tikzpicture}
=0,
\hspace{.2cm}
\begin{tikzpicture}[baseline=-0.4ex,scale=0.5,>=stealth]
\draw [fill=gray!45,gray!45] (-.7,-.75)  rectangle (.4,.75)   ;
\draw[->] (0.4,-0.75) to (.4,.75);
\draw[line width=1.2] (0.4,-0.3) to (0,-.3);
\draw[line width=1.2] (0.4,0.3) to (0,.3);
\draw[line width=1.1] (0,0) ++(90:.3) arc (90:270:.3);
\draw (0.65,0.3) node {\scriptsize{$+$}}; 
\draw (0.65,-0.3) node {\scriptsize{$-$}}; 
\end{tikzpicture}
=A^{-1/2}
\begin{tikzpicture}[baseline=-0.4ex,scale=0.5,>=stealth]
\draw [fill=gray!45,gray!45] (-.7,-.75)  rectangle (.4,.75)   ;
\draw[-] (0.4,-0.75) to (.4,.75);
\end{tikzpicture}
\hspace{.1cm} \text{ and }
\hspace{.1cm}
A^{1/2}
\heightexch{->}{-}{+}
- A^{5/2}
\heightexch{->}{+}{-}
=
\heightcurve.
\end{equation}
The product of two classes of stated tangles $[T_1,s_1]$ and $[T_2,s_2]$ is defined by  isotoping $T_1$ and $T_2$  in $\Sigma \times (1/2, 1) $ and $\Sigma \times (0, 1/2)$ respectively and then setting $[T_1,s_1]\cdot [T_2,s_2]=[T_1\cup T_2, s_1\cup s_2]$. Figure \ref{fig_product} illustrates this product.
\end{definition}
\par For an unmarked surface, $\mathcal{S}_A(\mathbf{\Sigma})$ coincides with the classical  Kauffman-bracket skein algebra.

\begin{figure}[!h] 
\centerline{\includegraphics[width=8cm]{dessin.eps} }
\caption{An illustration of the product in stated skein algebras.} 
\label{fig_product} 
\end{figure} 

 
 \begin{definition}[Reduced stated skein algebras]
 \begin{enumerate}
 \item Let $\mathbf{\Sigma}$ a marked surface and $p\in \mathcal{P}^{\partial}$ a boundary puncture between two consecutive boundary arcs $a$ and $b$ on the same boundary component $\partial\in \Gamma^{\partial}$.
The orientation of $\Sigma$ induces an orientation of $\partial$ so a cyclic ordering of the elements of $\partial \cap \mathcal{A}$ we suppose that $b$ is followed by $a$ in this ordering.  We denote by $\alpha(p)$ the arc with one endpoint $v_a\in a$ and one endpoint $v_b \in b$ such that $\alpha(p)$ can be isotoped inside $\partial$. $\alpha(p)$ is called a \textit{corner arc}.
Let $\alpha(p)_{ij}\in \mathcal{S}_A(\mathbf{\Sigma})$ be the class of the stated arc $(\alpha(p), s)$ where $s(v_a)=i$ and $s(v_b)=j$.
 \item We call \textit{bad arc} associated to $p$ the element $\alpha(p)_{-+}\in \mathcal{S}_A(\mathbf{\Sigma})$ (see Figure \ref{fig_bad_arc}). The \textit{reduced stated skein algebra} $\overline{\mathcal{S}}_A(\mathbf{\Sigma})$ is the quotient of $\mathcal{S}_A(\mathbf{\Sigma})$ by the ideal generated by all bad arcs.
 \end{enumerate}
 \end{definition}
 
 
 \begin{figure}[!h] 
\centerline{\includegraphics[width=4cm]{BadArc.eps} }
\caption{A bad arc.} 
\label{fig_bad_arc} 
\end{figure} 
 
 \begin{remark} The \textit{quantum trace} is an algebra morphism $\Tr_q : \mathcal{S}_A(\mathbf{\Sigma}) \to \mathcal{Z}_A(\mathbf{\Sigma}, \Delta)$ between the stated skein algebra and the balanced Chekhov-Fock algebra (a variation of the quantum Teichm\"uller space) which is a quantum torus associated to an ideal triangulation $\Delta$ of $\mathbf{\Sigma}$ (see \cite{BonahonWongqTrace, LeStatedSkein}). Its kernel is precisely the ideal generated by bad arcs (\cite[Theorem $7.12$]{CostantinoLe19}), hence it defines an embedding $\Tr_q: \overline{\mathcal{S}}_A(\mathbf{\Sigma}) \hookrightarrow  \mathcal{Z}_A(\mathbf{\Sigma}, \Delta)$. The spaces which appear in quantum hyperbolic geometry are representations of $\mathcal{Z}_A(\mathbf{\Sigma}, \Delta)$, therefore of $ \overline{\mathcal{S}}_A(\mathbf{\Sigma}) $ as well. This is the motivation for the introduction of reduced stated skein algebras and for the study of their representations.
 \end{remark}
 
 The main interest in extending skein algebras to marked surfaces is the existence of splitting morphisms which we now describe.
  Let $a$, $b$ be two distinct boundary arcs of $\mathbf{\Sigma}$, denote by $\pi : \Sigma\rightarrow \Sigma_{a\#b}$ the projection and $c:=\pi(a)=\pi(b)$. Let $(T_0, s_0)$ be a stated framed tangle of $\Sigma_{a\#b}\times (0,1)$ transversed to $c\times (0,1)$ and such that the heights of the points of $T_0 \cap c\times (0,1)$ are pairwise distinct and the framing of the points of $T_0 \cap c\times (0,1)$ is vertical towards $1$. Let $T\subset \Sigma \times (0,1)$ be the framed tangle obtained by cutting $T_0$ along $c$. 
Any two states $s_a : \partial_a T \rightarrow \{-,+\}$ and $s_b : \partial_b T \rightarrow \{-,+\}$ give rise to a state $(s_a, s, s_b)$ on $T$. 
Both the sets $\partial_a T$ and $\partial_b T$ are in canonical bijection with the set $T_0\cap c$ by the map $\pi$. Hence the two sets of states $s_a$ and $s_b$ are both in canonical bijection with the set $\mathrm{St}(c):=\{ s: c\cap T_0 \rightarrow \{-,+\} \}$. 

\begin{definition}[Splitting morphism]\label{def_gluing_map}
The splitting morphism $\theta_{a\#b}: \mathcal{S}_{A}(\mathbf{\Sigma}_{a\#b}) \rightarrow \mathcal{S}_{A}(\mathbf{\Sigma})$ is the linear map given, for any $(T_0, s_0)$ as above, by: 
$$ \theta_{a\#b} \left( [T_0,s_0] \right) := \sum_{s \in \mathrm{St}(c)} [T, (s, s_0 , s) ].$$
\end{definition}

\begin{figure}[!h] 
\centerline{\includegraphics[width=8cm]{gluing_map.eps} }
\caption{An illustration of the splitting morphism $\theta_{a\#b}$.} 
\label{fig_gluingmap} 
\end{figure} 

\begin{theorem}\label{theorem_gluing}(\cite[Theorem $3.1$]{LeStatedSkein}, \cite[Theorem $7.6$]{CostantinoLe19})
 The linear map $\theta_{a\#b}: \mathcal{S}_A(\mathbf{\Sigma}_{a\#b}) \hookrightarrow \mathcal{S}_A(\mathbf{\Sigma})$ is an injective morphism of algebras. It passes to the quotient to define an injective morphism (still denoted by the same letter) 
 $\theta_{a\#b}: \overline{\mathcal{S}}_A(\mathbf{\Sigma}_{a\#b}) \hookrightarrow \overline{\mathcal{S}}_A(\mathbf{\Sigma})$. 
   Moreover the gluing operation is coassociative in the sense that if $a,b,c,d$ are four distinct boundary arcs, then we have $\theta_{a\#b} \circ \theta_{c\#d} = \theta_{c\#d} \circ \theta_{a\#b}$.
\end{theorem}


 
 
 \subsection{Bases}
 
 We now define two bases for the reduced stated skein algebras. Let us first make a preliminary remark: if $T$ is a tangle in standard positon and $D$ its planar diagram projection, we cannot recover the isotopy class $T$ from the isotopy class of  $D$ since we lost the information about the height orders of the sets $\partial_aT$, $a\in \mathcal{A}$ in the projection.
 
 \begin{convention} Let $(D,s)$ be a stated diagram in $\mathbf{\Sigma}$. Recall that each boundary arc $a\in \mathcal{A}$ receives an orientation from the orientation of $\Sigma$ and that for $v,w \in a$ we write $v<_{\mathfrak{o}^+} w$ if $a$ is oriented from $v$ to $w$. Let $(T,s)$ be a stated tangle such that $(1)$ $(T,s)$ is in standard position (in the sense of Definition \ref{def_tangles}) and its planar projection is $(D,s)$ and $(2)$ for every $a\in \mathcal{A}$, if $v, w \in \partial_a D$ are such that $v<_{\mathfrak{o}^+}w$ then $h(v)<h(w)$. Conditions $(1)$ and $(2)$ completely determine the isotopy class of $(T,s)$ and we will write $[D,s]:= [T,s]\in \overline{\mathcal{S}}_A(\mathbf{\Sigma})$. 
 \end{convention}
 
 \begin{definition}[L\^e's bases]
 \begin{enumerate}
 \item A closed component of a diagram $D$ is trivial if it bounds an embedded disc in $\Sigma$. An open component of $D$ is trivial if it can be isotoped, relatively to its boundary, inside some boundary arc. A diagram is \textit{simple} if it has neither double point nor trivial component. By convention, the empty set is a simple diagram. Let $\mathfrak{o}^+$ denote the orientation of the boundary arcs of $\mathbf{\Sigma}$ induced by the orientation of $\Sigma$ as in Definition \ref{def_marked_surfaces}. For each boundary arc $a$ we write $<_{\mathfrak{o}^+}$ the induced total order on $\partial_a D$. A state $s: \partial D \rightarrow \{ - , + \}$ is $\mathfrak{o}^+-$\textit{increasing} if for any boundary arc $a$ and any two points $x,y \in \partial_a D$, then $x<_{\mathfrak{o}^+} y$ implies $s(x)< s(y)$, with the convention $- < +$. 
 \item We denote by $\mathcal{B}^{L}\subset \overline{\mathcal{S}}_{A}(\mathbf{\Sigma})$ the set of classes $[D,s]$ of stated diagrams such that $D$ is simple,  $s$ is $\mathfrak{o}^+$-increasing and such that $(D,s)$ does not contain bad arc component.
 \end{enumerate}
 \end{definition}
 
 \begin{theorem}(\cite[Theorem $7.1$]{CostantinoLe19}) $\mathcal{B}^{L}$ is a basis of $\overline{\mathcal{S}}_{A}(\mathbf{\Sigma})$.\end{theorem}
 
 \begin{lemma}(\cite[Lemma $6.2$]{KojuAzumayaSkein}) Let $p \in \mathcal{P}^{\partial}$. Then $\alpha(p)_{--}$ is the inverse of $\alpha(p)_{++}$ in $ \overline{\mathcal{S}}_{A}(\mathbf{\Sigma})$ (i.e. $\alpha(p)_{++}\alpha(p)_{--}=1$). \end{lemma}
 
 \begin{definition}[M\"uller's bases]\label{def_Muller_basis1}
 Let $\mathcal{B}^M \subset  \overline{\mathcal{S}}_{A}(\mathbf{\Sigma})$ be the set of classes of stated diagrams $[D,s]$ such that:
 \begin{enumerate}
 \item $D$ is simple, 
 \item if $\alpha \subset D$ is an open connected component of $D$ which is not a corner arc, then both endpoints of $\alpha$ have state $+$, 
 \item for each $p\in \mathcal{P}^{\partial}$, there exists a sign $\varepsilon_p\in \{-,+\}$ such that every component of $D$ parallel to $\alpha(p)$ has both endpoints with state $\varepsilon_p$.
 \end{enumerate}
 \end{definition}
 
 \begin{proposition}\label{prop_Muller_basis}
 $\mathcal{B}^M$ is a basis of $\overline{\mathcal{S}}_{A}(\mathbf{\Sigma})$.
 \end{proposition}
 
 \begin{proof}
 We claim that there exist a bijection $f: \mathcal{B}^M \to \mathcal{B}^L$ and a map $c: \mathcal{B}^M \to \mathbb{Z}$ such that for all $[D,s]\in \mathcal{B}^M$ we have $f([D,s])=A^{c([D,s])/2} [D,s]$. The fact that $\mathcal{B}^M$ is a basis will then follow from the fact that $\mathcal{B}^L$ is a basis. Let $[D,s] \in \mathcal{B}^M$. If $s$ is $\mathfrak{o}^+$ increasing, we set $f([D,s])=[D,s]$ and $c([D,s])=0$. Else, there exists $a\in \mathcal{A}$ and two consecutive endpoints $v_1 <_{\mathfrak{o}^+} v_2$ in $\partial_aD$ such that $s(v_1)=+$ and $s(v_2)=-$ (i.e. locally $[D,s]$ has the form
  $\heightexch{->}{-}{+}$). By definition of $\mathcal{B}^M$, the arc in $D$ which contains $v_2$ is a corner arc $\alpha(p)_{--}$ with both endpoints with state $-$. Let $[D',s']$ be the class of stated arc obtained from $[D,s]$ by gluing $v_1$ and $v_2$ together and pushing the resulting point in the interior of $\Sigma$, i.e. is obtained by the local replacement $\heightexch{->}{-}{+} \mapsto \heightcurve$.
  Using the skein relation \eqref{eq: skein 2}, one has: 
  $$ 
\heightexch{->}{-}{+}
= A^{2}
\heightexch{->}{+}{-}
+ A^{-1/2}
\heightcurve = A^{-1/2}
\heightcurve, $$
where the second equality comes from the fact that the stated diagram corresponding to $\heightexch{->}{+}{-}$ contains the bad arc $\alpha(p)_{-+}$ and thus vanishes in  $\overline{\mathcal{S}}_{A}(\mathbf{\Sigma})$. Therefore $[D,s]=A^{-1/2} [D',s']$. If $[D',s']$ is $\mathfrak{o}$-increasing, we set $f([D,s]):= [D',s']$ and $c([D,s])=1$. Else, $[D',s']$ contains a pair of consecutive points of the form $\heightexch{->}{-}{+}$ and we can repeat the process of replacing 
 $\heightexch{->}{-}{+}$ by $\heightcurve$. Since this process decreases the number of endpoints in $\partial D$, after a finite number of steps, say $n$, we obtain an elements $f([D,s])\in \mathcal{B}^L$ such that $f([D,s])= A^{n/2} [D,s]$ and we set $c([D,s])=n$. It remains to prove that $f$ is a bijection; let us define an inverse map $g: \mathcal{B}^L \to \mathcal{B}^M$.
  Let $[D,s]\in \mathcal{B}^L$.  If $[D,s]\in \mathcal{B}^M$, we set $g([D,s]):=[D,s]$. Else, 
 since $s$ is $\mathfrak{o}^+$ increasing, $(D,s)$ cannot have two sub stated arcs isotopic to $\alpha(p)_{++}$ and $\alpha{p}_{--}$ respectively for some $p\in \mathcal{P}^{\partial}$. Therefore there exists $a\in \mathcal{A}$, an arc $\alpha \subset D$  
 and $v\in \partial_aD \cap \partial \alpha$ such that $s(v)=-$ and such that $(\alpha, s_{| \partial \alpha})$ is not isotopic to $\alpha(p)_{--}$ for any $p \in \mathcal{P}^{\partial}$. 
 Choose a such $v$ which is minimal for the order $<_{\mathfrak{o}^+}$ in $a$. Let $p, p'\in \mathcal{P}^{\partial}$ be the punctures adjacent to $a$ 
 such that the $\mathfrak{o}^+$ orientation of $a$ points from $p$ to $p'$. Let $b\in \mathcal{A}$ be the boundary arc adjacent to $p$ right before  $a$ (possibly $a=b$ if the connected component of $\partial \Sigma$ containing $a$ does not contain any other boundary arc, i.e. if $p=p'$). 
 If $w\in \partial_a D$ is such that $w<_{\mathfrak{o}^+}v$ and $\beta \subset D$ is the arc with endpoint $w$, then $(\beta, s_{| \partial \beta})$ is isotopic to $\alpha(p)_{--}$. Let $[D',s']$ be the element obtained from $[D,s]$ by slightly pushing $v\in a$ to $v' \in b$ above such arcs $\beta$ and by setting $s'(v')=+$ and by adding a copy of $\alpha(p)_{--}$ as illustrated in Figure \ref{fig_move}. Using the same skein relation \eqref{eq: skein 2} than previously, one finds that $[D',s']=A^{-1/2}[D,s]$. If $[D',s']\in \mathcal{B}^M$ we set $g([D,s]):=[D',s']$. Else, we can find another point $w'\in \partial D'$ and repeat the process. Since the operation $[D,s]\mapsto [D',s']$ decreases the number of endpoints of $D$ with state $-$, after a finite number of steps we obtain an element $g([D,s])\in \mathcal{B}^M$. Clearly $f$ and $g$ are inverse to each other so $f$ is a bijection and the proof is complete. 
 
 
 
 \begin{figure}[!h] 
\centerline{\includegraphics[width=8cm]{Move.eps} }
\caption{An illustration of the operation $[D,s]\mapsto [D',s']$.} 
\label{fig_move} 
\end{figure} 
 
 
 \end{proof}
 
 \begin{lemma}\label{lemma_heightexch} Let  $(T,s)$ and $(T',s')$ be two stated tangles in $\mathbf{\Sigma}\times (0,1)$ in standard position (in the sense of Definition \ref{def_tangles})  whose projection on $\mathbf{\Sigma}\times \{1/2\}$ are both equal to the same element $(D,s)$ such that either $[D,s]\in \mathcal{B}^L$ or $[D,s]\in \mathcal{B}^M$.  Then there exists $n\in \mathbb{Z}$ such that $[T,s]=A^{n/2}[T',s']$.
 \end{lemma}
 
 \begin{proof}
 As we proved during the proof of Proposition \ref{prop_Muller_basis}, for each $[D,s]\in \mathcal{B}^M$ there exists $[D',s']\in \mathcal{B}^L$ such that $[D,s]=A^{n/2}[D',s']$ for some $n\in \mathbb{Z}$. So we can assume that the common projection $(D,s)$ of $(T,s)$ and $(T',s')$ is such that $[D,s] \in \mathcal{B}^L$. 
 We say that $(T,s)$ and $(T',s')$ differ by an height exchange move if there exists $a\in \mathcal{A}$ and $v_1, v_2$ two  points in $\partial_a T$ with $h(v_1)<h(v_2)$ which are consecutive, in the sense that there is no $w\in \partial_aT$ with 
 $h(v_1)<_{\mathfrak{o}^+} h(w)<_{\mathfrak{o}^+} h(v_2)$, such that $(T',s')$ is obtained from $(T,s)$ by exchanging the heights of $v_1$ and $v_2$ (this means that $(T,s)\mapsto (T',s')$ is obtained by a local relation $\heightexch{->}{\varepsilon}{\varepsilon'} \mapsto \heightexch{<-}{\varepsilon}{\varepsilon'}$). Any two stated tangles in standard position with the same projection are related by a finite sequence of height exchanges so it suffices to prove the lemma in this case. Exchanging $(T,s)$ and $(T',s')$ if necessary, we can suppose without loss of generality than $v_1 <_{\mathfrak{o}^+} v_2$. 
  Since $[D,s]\in \mathcal{B}^L$, $s$ is $\mathfrak{o}^+$ increasing so  $( s(v_1), s(v_2)) \in \{ (+, +), (-, -), (-,+)\}$. By \cite[Lemma $2.4$]{LeStatedSkein}, one has the following skein relations
 $$ \heightexch{->}{+}{+}= A \heightexch{<-}{+}{+}, \quad  \heightexch{->}{-}{-}= A \heightexch{<-}{-}{-}, \quad  \heightexch{->}{+}{-}= A^{-1} \heightexch{<-}{+}{-}.$$
 So $[T,s]=A^{\pm 1/2}[T',s']$ and the lemma is proved.
 
 \end{proof}
 
 
 \subsection{Center and Azumaya locus}
 
 \begin{convention}
 From now on and in all the rest of the paper with the exception of Subsections \ref{sec_Muller} and \ref{sec_QCA}, we suppose that $k=\mathbb{C}$ and that $A^{1/2}$ is a root of unity of odd order $N\geq 3$ which is fixed once and for all. We denote by $Z_{\mathbf{\Sigma}}$ the center of $\overline{\mathcal{S}}_A(\mathbf{\Sigma})$.
  \end{convention}
  In this subsection, we describe $Z_{\mathbf{\Sigma}}$ and define the Azumaya locus $\mathcal{AL}(\mathbf{\Sigma})$.
  \par 
  Consider a stated arc $\alpha_{ij}$ in some marked surface $\mathbf{\Sigma}$ and denote by $\alpha_{ij}^{(N)}$ the stated tangle obtained by taking $N$ parallel copies of $\alpha_{ij}$ pushed along the framing direction. In the case where both endpoints of $\alpha$ lye in two distinct boundary arcs, one has the equality
$ \alpha_{ij}^{(N)} = (\alpha_{ij})^N$
in $\overline{\mathcal{S}}_A(\mathbf{\Sigma})$, but when both endpoints lye in the same boundary arc, they are distinct. More precisely, suppose the two endpoints, say $v$ and $w$, of $\alpha$ lye in the same boundary arc with $h(v)>h(w)$. Then  $\alpha_{ij}^{(N)}$ is defined by a stated tangle $(\alpha^{(N)}, s)$ where $\alpha^{(N)}=\alpha_1 \cup \ldots \cup \alpha_N$ represents $N$ copies of $\alpha$ and the endpoints $v_i, w_i$ of the copy $\alpha{(i)}$ are chosen such that $h(v_N) > \ldots > h(v_1) > h(w_N)>\ldots >h(w_1)$.

  
\begin{definition}[Chebyshev polynomials]  The $N$-th Chebyshev polynomial of the first kind is the polynomial  $T_N(X) \in \mathbb{Z}[X]$ defined by the recursive formulas $T_0(X)=2$, $T_1(X)=X$ and $T_{n+2}(X)=XT_{n+1}(X) -T_n(X)$ for $n\geq 0$.
\end{definition}



\begin{theorem}\label{theorem_Frobenius}(\cite{BonahonWongqTrace} for unmarked surfaces,  \cite{KojuQuesneyClassicalShadows} for marked surfaces;  see also \cite{BloomquistLe})
There is an embedding 
$$ Fr_{\mathbf{\Sigma}}: \overline{\mathcal{S}}_{+1}(\mathbf{\Sigma}) \hookrightarrow Z_{\mathbf{\Sigma}}$$
sending the (commutative) algebra at $+1$ into the center of the skein algebra at $A^{1/2}$. Moreover, $Fr_{\mathbf{\Sigma}}$ is characterized by the facts that if $\gamma$ is a loop, then $Fr_{\mathbf{\Sigma}}(\gamma) = T_N(\gamma)$ and if $\alpha_{ij}$ is a stated arc, then $Fr_{\mathbf{\Sigma}}(\alpha_{ij})= \alpha_{ij}^{(N)}$.
\end{theorem}
$Fr_{\mathbf{\Sigma}}$ is called the \textit{Frobenius morphism}.
The above characterization of $Fr_{\mathbf{\Sigma}}$ is meaningful  because, as proved in \cite{KojuQuesneyClassicalShadows}, $ \overline{\mathcal{S}}_A(\mathbf{\Sigma})$ is generated by the classes of loops and stated arcs.

 
 \begin{definition}[Punctures and boundary central elements]
 \begin{enumerate}
 \item   For $p\in \mathring{\mathcal{P}}$ an inner puncture, we denote by $\gamma_p \in \overline{\mathcal{S}}_A(\mathbf{\Sigma})$ the class of a peripheral curve encircling $p$ once.
 \item  For $\partial \in \Gamma^{\partial}$ a boundary component which intersects $\mathcal{A}$ non trivially, denote by $p_1, \ldots, p_n$ the boundary punctures in $\partial$ cyclically ordered by $\mathfrak{o}^+$  and define the elements in $\overline{\mathcal{S}}_A(\mathbf{\Sigma})$:
  $$ \alpha_{\partial} := \alpha(p_1)_{++} \ldots \alpha(p_n)_{++}, \quad \mbox{ and } \quad \alpha_{\partial}^{-1}:= \alpha(p_1)_{--} \ldots \alpha_(p_n)_{--}.$$
 \end{enumerate}
 \end{definition}
 
 The elements $\alpha_{\partial}$ and $\alpha_{\partial}^{-1}$ are inverse to each other (see \cite{KojuAzumayaSkein}), hence the notation. 
 
 \begin{definition}[PI-degree]
 \begin{enumerate}
 \item A connected marked surface is \textit{small} if it is either a disc with $0$ or $1$  boundary arc or an unmarked sphere. 
  \item Let $\mathbf{\Sigma}=(\Sigma_{g,n}, \mathcal{A})$ be a connected marked surface of genus $g$ with $n$ boundary component which is not small. Set $D_{\mathbf{\Sigma}}:= N^{3g-3+n+|\mathcal{A}|}$. If $\mathbf{\Sigma}$ is small, we set $D_{\mathbf{\Sigma}}=1$. We extend it to non connected surfaces by the formula $D_{\mathbf{\Sigma}\bigsqcup \mathbf{\Sigma}'}:= D_{\mathbf{\Sigma}} D_{\mathbf{\Sigma}'}$.
  \end{enumerate}
 \end{definition}
 
 \begin{lemma}\label{lemma_additivityD}
 Let $\mathbf{\Sigma}_1, \mathbf{\Sigma}_2$ be two marked surfaces  and let $a_1, a_2$ be some boundary arcs of $\mathbf{\Sigma}_1, \mathbf{\Sigma}_2$ respectively which do not belong to a small component. Then 
 $$ D_{\mathbf{\Sigma}_1 \cup_{a_1 \# a_2}\mathbf{\Sigma}_2} = D_{\mathbf{\Sigma}_1} D_{\mathbf{\Sigma}_2}.$$
 \end{lemma}
 
 \begin{proof} This follows from the facts that $\mathbf{\Sigma}_1 \cup_{a_1 \# a_2}\mathbf{\Sigma}_2$ has $| \mathcal{A}_1| + |\mathcal{A}_2| -2$ boundary arcs and has $|\pi_0(\partial \Sigma_1)| + |\pi_0(\partial \Sigma_2)| -1$ boundary components.
  \end{proof}
 
 \begin{theorem}\label{theorem_center} (\cite{FrohmanKaniaLe_UnicityRep} for unmarked surfaces, \cite{KojuAzumayaSkein} for marked surfaces)
 \begin{enumerate}
 \item The elements $\gamma_p$ and $\alpha_{\partial}^{\pm 1}$ are central and $Z_{\mathbf{\Sigma}}$ is generated by the image of the Frobenius morphism together with these elements. More precisely, $Z_{\mathbf{\Sigma}}$ is isomorphic to the quotient of $$ \overline{\mathcal{S}}_{+1}(\mathbf{\Sigma})[ \gamma_p, \alpha_{\partial}^{\pm 1}; p \in \mathring{P}, \partial \in \Gamma^{\partial} ]$$ by the relations $T_N(\gamma_p)=Fr_{\mathbf{\Sigma}}(\gamma_p)$ and $\alpha_{\partial}^N = Fr_{\mathbf{\Sigma}}(\alpha_{\partial})$.
 \item $ \overline{\mathcal{S}}_A(\mathbf{\Sigma})$ is freely and finitely generated over both $Z_{\mathbf{\Sigma}}$ and over the image of the Frobenius morphism. Its rank over $Z_{\mathbf{\Sigma}}$ is $(D_{\mathbf{\Sigma}})^2$.
 \end{enumerate}
 \end{theorem}
 
 \begin{definition}[Classical schemes]
 \begin{enumerate}
 \item We write $X(\mathbf{\Sigma}):= \Specm( \overline{\mathcal{S}}_{+1}(\mathbf{\Sigma}) )$ and $\widehat{X}(\mathbf{\Sigma}):= \Specm(Z_{\mathbf{\Sigma}})$ and denote by $p: \widehat{X}(\mathbf{\Sigma}) \to {X}(\mathbf{\Sigma})$ the morphism induced by $Fr_{\mathbf{\Sigma}}$.
 \item For $\mathbf{\Sigma}_1, \mathbf{\Sigma}_2$ with boundary arcs $a_1$ and $a_2$, we denote by $ p_{a_1 \# a_2} : {X}(\mathbf{\Sigma}_1)\times {X}(\mathbf{\Sigma}_2) \to {X}(\mathbf{\Sigma}_1\cup_{a_1 \# a_2} \mathbf{\Sigma}_2)$ and $\widehat{p}_{a_1 \# a_2} : \widehat{X}(\mathbf{\Sigma}_1)\times \widehat{X}(\mathbf{\Sigma}_2) \to \widehat{X}(\mathbf{\Sigma}_1\cup_{a_1 \# a_2} \mathbf{\Sigma}_2)$ the dominant maps induced by $\theta_{a_1 \# a_2}$.
 \end{enumerate}
 \end{definition}
 
 By Theorem \ref{theorem_center}, as a set  $\widehat{X}(\mathbf{\Sigma})$ is described as 
 \begin{multline*}  \widehat{X}(\mathbf{\Sigma}) := \{ \widehat{x}= (x, h_p, h_{\partial}), x\in X(\mathbf{\Sigma}),
 h_p\in \mathbb{C} \mbox{ is s.t. } T_N(h_p)=\chi_x(\gamma_p) \mbox{ for }p\in \mathring{P}, 
\\ h_{\partial} \in \mathbb{C}^* \mbox{is s.t. } h_{\partial}^N = \chi_x(\alpha_{\partial}) \mbox{ for }\partial \in \Gamma^{\partial} \}.\end{multline*}
 
 
 \begin{definition}[Azumaya locus and fully Azumaya locus]\label{def_AL}
 \begin{enumerate}
 \item For $\widehat{x} \in \widehat{X}(\mathbf{\Sigma})$ let $\mathcal{I}_{\widehat{x}} \subset \overline{\mathcal{S}}_A(\mathbf{\Sigma})$ be the ideal generated by elements $z-\chi_{\widehat{x}}(z)$ for $z\in Z_{\mathbf{\Sigma}}$ and write 
 $$ \overline{\mathcal{S}}_A(\mathbf{\Sigma})_{\widehat{x}}:= \quotient{ \overline{\mathcal{S}}_A(\mathbf{\Sigma})}{\mathcal{I}_{\widehat{x}}}.$$
 \item The \textit{Azumaya locus} of $ \overline{\mathcal{S}}_A(\mathbf{\Sigma})$ is the subset of $ \widehat{X}(\mathbf{\Sigma}) $ defined by 
 $$ \mathcal{AL}(\mathbf{\Sigma}):= \{ \widehat{x} \in \widehat{X}(\mathbf{\Sigma})\mbox{ such that } \overline{\mathcal{S}}_A(\mathbf{\Sigma})_{\widehat{x}}\cong \Mat_D(\mathbb{C}) \}.$$
 \item The \textit{fully Azumaya locus} of $ \overline{\mathcal{S}}_A(\mathbf{\Sigma})$ is the subset  $\mathcal{FAL}(\mathbf{\Sigma}) \subset {X}(\mathbf{\Sigma}) $ of elements $x\in {X}(\mathbf{\Sigma})$ such that $p^{-1}(x)\subset \mathcal{AL}(\mathbf{\Sigma})$. 
 \end{enumerate}
 \end{definition}
 
 Let us now state the main theorem about Azumaya loci. Let $\mathcal{A}$ be a $\mathbb{C}$-algebra which is $(i)$ finitely generated, $(ii)$ prime and $(iii)$ has finite rank $R$ over its center $Z$. By a theorem of Posner-Formanek  (see \cite[Theorem $I.13.3$]{BrownGoodearl}), $R$ is a perfect square: $R=D^2$ for $D\in \mathbb{N}$ named its \textit{PI-degree}. For $x\in \Specm(Z)$, set $\mathcal{A}_x:= \quotient{\mathcal{A}}{(\chi_x(z)-z, z\in Z)}$ and define the Azumaya locus as $\mathcal{AL}(\mathcal{A})\subset \Specm(Z)$ to the subset of elements $x$ such that $\mathcal{A}_x\cong \Mat_D(\mathbb{C})$. The following generalizes the original work in \cite{DeConciniKacRepQGroups} on quantum groups.
 
 \begin{theorem}\label{theorem_AL}
 \begin{enumerate}
 \item (\cite[Theorem $1.2$]{Brown_AL_discriminant}) 
 If $\mathcal{A}$ satisfies the above hypotheses $(i),(ii),(iii)$ then $\mathcal{AL}(\mathcal{A})$ is open dense.
 \item (\cite[III.1.2.]{BrownGoodearl})  If ${x}\in \Specm(Z)$ does not belong to the Azumaya locus then the PI-degree of $\mathcal{A}_x$  is strictly smaller than the PI-degree $D$ of $\mathcal{A}$. In particular, for $r: \mathcal{A} \to \End(V)$   an irreducible representation with classical shadow $x\in \Specm(Z)$, then $x \in \mathcal{AL}(\mathcal{A})$ if and only if $\dim(V)=D$. Moreover, if $x \notin \mathcal{AL}(\mathcal{A})$, then $\dim(V)< D$.
 \end{enumerate}
 \end{theorem}
 
 That $ \overline{\mathcal{S}}_A(\mathbf{\Sigma})$ is a domain when $\Sigma$ is not closed follows from the fact that it can be embedded into a quantum torus (\cite{BonahonWongqTrace}) (and is proved in \cite{PrzytyckiSikora_SkeinDomain} for closed surfaces). That $ \overline{\mathcal{S}}_A(\mathbf{\Sigma})$ is finitely generated as an algebra is proved in  \cite{BullockGeneratorsSkein} for unmarked surfaces and in \cite{KojuAzumayaSkein} for marked surfaces (see also \cite{KojuPresentationSSkein} for explicit finite presentations in the case of essential marked surfaces).
 
 \begin{corollary}\label{theorem_AL_Dense}(\cite{FrohmanKaniaLe_UnicityRep} for unmarked surfaces, \cite{KojuAzumayaSkein} for marked surfaces)
 $\mathcal{AL}(\mathbf{\Sigma})$ and $\mathcal{FAL}(\mathbf{\Sigma})$ are open dense.
 \end{corollary}
 
  
  Let $Z_{\mathbf{\Sigma}}^0\subset Z_{\mathbf{\Sigma}}$ denote the image of the Frobenius morphism (so $Z_{\mathbf{\Sigma}}^0\cong \overline{\mathcal{S}}_{+1}(\mathbf{\Sigma})$).
  An indecomposable semi-weight representation $\rho$ induces a character over $Z_{\mathbf{\Sigma}}^0$ so defines a point in $x\in X(\mathbf{\Sigma})$ which we call the \textit{shadow} of $\rho$. Let $\mathcal{I}_x \subset \overline{\mathcal{S}}_{A}(\mathbf{\Sigma})$ be the ideal generated by elements $z-\chi_{x}(z)$ for $z\in Z_{\mathbf{\Sigma}}^0$. Then $\rho$ factorizes through the finite dimensional algebra $\overline{\mathcal{S}}_{A}(\mathbf{\Sigma})_x:= \quotient{\overline{\mathcal{S}}_{A}(\mathbf{\Sigma})}{\mathcal{I}_x}$. Let $\mathfrak{m}_x$ be the maximal ideal of $Z_{\mathbf{\Sigma}}^0$ which corresponds to $x$ and define
  $$ Z(x):= \quotient{Z_{\mathbf{\Sigma}}}{\mathfrak{m}_x Z_{\mathbf{\Sigma}}}.$$
   The following theorem is useful to classify the indecomposable semi-weight representations whose shadows lye in the fully Azumaya locus.
  
  \begin{theorem}( \cite[Corollary $2.7$]{BrownGordon_ramificationcenters})\label{theorem_FAL} If $x\in \mathcal{FAL}(\mathbf{\Sigma})$, then 
  $ \overline{\mathcal{S}}_{A}(\mathbf{\Sigma})_x \cong \Mat_D(Z(x))$.
  \end{theorem}
  
 
  
  \begin{corollary}\label{coro_FAL}
  Let $x \in \mathcal{FAL}(\mathbf{\Sigma})$ and write  $r:= |p^{-1}(x)|$ so that we have $r$ irreducible weight representations with classical shadow $\widehat{x}$ such that $p(\widehat{x})=x$. Suppose that there exists $a,b$ such that $a+b=r$ and an algebra isomorphism
  $$ Z(x) \cong  \mathbb{C}^{\oplus a}\oplus \left(\quotient{\mathbb{C}[X]}{(X-1)^2}\right)^{\oplus b}.$$
  Then $\overline{\mathcal{S}}_{A}(\mathbf{\Sigma})$ has $r+b$ isomorphism classes of indecomposable semi-weight modules with shadow $x$: the $r$ irreducible weight ones plus $b$ which are nor irreducible nor weight modules. 
  \end{corollary}
  
  \begin{proof}
  By Theorem \ref{theorem_FAL}, one has $\overline{\mathcal{S}}_{A}(\mathbf{\Sigma})_x \cong \Mat_N(\mathbb{C})^{\oplus a} \oplus \Mat_N( \quotient{\mathbb{C}[X]}{(X-1)^2})^{\oplus b}$. $\Mat_N(\mathbb{C})$ has a single indecomposable representation $V$: the standard action on $V:= \mathbb{C}^N$,  which is irreducible. Let $W$ be the two dimensional representation of $\quotient{\mathbb{C}[X]}{(X-1)^2}$ with basis $\{x, y\}$ such that $X \cdot x= x$ and $X \cdot y=x+y$. Then
  $\Mat_N\left( \quotient{\mathbb{C}[X]}{(X-1)^2} \right)$ has two representations $V\subset P$. $V$ has dimension $N$ and is obtained by the composition 
  $$\Mat_N\left( \quotient{\mathbb{C}[X]}{(X-1)^2}\right) \xrightarrow{ X=1} \Mat_N(\mathbb{C}) \to \End(V).$$
   $P=V\otimes W$ has dimension $2N$ and is obtained by the composition $$\Mat_N\left( \quotient{\mathbb{C}[X]}{(X-1)^2}\right)\cong \Mat_N(\mathbb{C}) \otimes \left(\quotient{\mathbb{C}[X]}{(X-1)^2}\right) \to \End(V)\otimes \End(W) \cong\End(V\otimes W).$$
 
  \end{proof}
 
 \section{Classical moduli spaces}\label{sec_moduli_spaces}
 
 In this subsection, we give a geometric interpretation of $X(\mathbf{\Sigma})$ following \cite{FockRosly, KojuTriangularCharVar} and study its basic properties.
 
 \subsection{Geometric interpretation of $X(\mathbf{\Sigma})$}
 
 \begin{definition}[Relative representation varieties]\label{def_modulispaces} Let $\mathbf{\Sigma}=(\Sigma, \mathcal{A})$ be an essential  marked surface.
 \begin{enumerate}
 \item The \textit{fundamental groupoid} $\Pi_1(\Sigma)$ is the category whose objects are points in $\Sigma$ and whose morphisms $\beta: v_1 \to v_2$ are homotopy classes of path $c_{\beta}: [0,1] \to \Sigma$ such that $c_{\beta}(0)=v_1$ and $c_{\beta}(1)=v_2$. We write $v_1=s(\beta)$ (the source) and $v_2= t(\beta)$ (the target). The composition is the concatenation of paths and the unit at $v\in \Sigma$ is the class $1_v$ of the constant path. For $\beta: v_1 \to v_2$, we denote by $\beta^{-1}: v_2 \to v_1$ the class of the path $c_{\beta^{-1}}(t) = c_{\beta}(1-t)$ (so $\beta \beta^{-1}=1_{t(\beta)}$). For $a\in \mathcal{A}$ we denote by $v_a \in a$ the middle point $v_a:= \varphi_a(1/2)$ and denote by $\mathbb{V}=\{ v_a, a\in \mathcal{A}\}$ the set of such points. $\Pi_1(\Sigma, \mathbb{V})$ is the full subcategory of $\Pi_1(\Sigma)$ generated by $\mathbb{V}$. By abuse of notation, we also denote by $\Pi_1(\Sigma, \mathbb{V})$ the set of morphisms of the underlying category.
 \item The \textit{relative representation variety} $\mathcal{R}_{\SL_2}(\mathbf{\Sigma})$ is the set of functors $\rho: \Pi_1(\Sigma, \mathbb{V}) \to \SL_2$, where $\SL_2$ is seen as a category with only one object $*$ whose set of  endomorphisms is $\SL_2(\mathbb{C})$. It admits a structure of complex affine variety whose algebra of regular functions is 
 $$ \mathcal{O}[\mathcal{R}_{\SL_2}(\mathbf{\Sigma})]:= \quotient{ \mathbb{C}[X_{ij}^{\beta}, i,j\in \{-,+\}, \beta \in \Pi_1(\Sigma, \mathbb{V})]}{\left( M_{\beta_1}M_{\beta_2}=M_{\beta_1 \beta_2}, \det(M_{\beta})=1 \right)}.$$
 Here, for $\beta \in \Pi_1(\Sigma, \mathbb{V})$, $M_{\beta}$ represents the $2\times 2$ matrix with coefficients in the polynomial ring $ \mathbb{C}[X_{ij}^{\beta}, i,j\in \{-,+\}, \beta \in \Pi_1(\Sigma, \mathbb{V})]$ defined by 
 $M_{\beta}=\begin{pmatrix} X_{++}^{\beta} & X_{+-}^{\beta} \\ X_{-+}^{\beta} & X_{--}^{\beta} \end{pmatrix}$ and we quotient by the relations $\det(M_{\beta}):=X_{++}^{\beta}X_{--}^{\beta}- X_{+-}^{\beta}X_{-+}^{\beta} =1$ for all $\beta \in  \Pi_1(\Sigma, \mathbb{V})$ and by the four matrix coefficients of $ M_{\beta_1}M_{\beta_2}-M_{\beta_1 \beta_2}$ for every pair of composable paths (i.e. such that $t(\beta_2)=s(\beta_1)$).
 Clearly the set of closed points of $\mathcal{R}_{\SL_2}(\mathbf{\Sigma}):= \Specm(\mathcal{O}[\mathcal{R}_{\SL_2}(\mathbf{\Sigma})])$ is in canonical bijection with the set of functors  $\rho: \Pi_1(\Sigma, \mathbb{V}) \to \SL_2$.
 \item A \textit{presenting graph} $\Gamma$ for $\mathbf{\Sigma}$ is an embedded oriented graph $\Gamma \subset \Sigma$ whose set of vertices is $\mathbb{V}$ and such that $\Sigma$ retracts on $\Gamma$. We denote by $\mathcal{E}(\Gamma)$ the set of its oriented edges whose elements are seen as paths in $\Pi_1(\Sigma, \mathbb{V})$. 
 \item An oriented arc $\alpha$  in $\mathbf{\Sigma}$ naturally defines a path in $\Pi_1(\Sigma, \mathbb{V})$ which we abusively also denote by $\alpha$. For $i,j \in \{-, +\}$, we denote by $\alpha_{ij}\in \mathcal{S}_A(\mathbf{\Sigma})$ the class of the arc $\alpha$ with state $i$ at its source point $s(\alpha)$ at state $j$ at its target point $t(\alpha)$. For $p\in {\mathcal{P}}^{\partial}$ we fix the canonical orientation of the corner arc $\alpha(p)$ such that $\alpha(p)_{-+}$ is a bad arc. The \textit{small Bruhat cell} is the subset of $\SL_2(\mathbb{C})$ defined by
 $$ \SL_2^{1}:= \{ M = \begin{pmatrix} a & b \\ c & d \end{pmatrix} \in \SL_2(\mathbb{C}) \mbox{ such that }a=0 \}.$$
 The \textit{reduced relative representation variety} is the subvariety: 
 $$ \overline{\mathcal{R}}_{\SL_2}(\mathbf{\Sigma}) = \{ \rho : \Pi_1(\Sigma, \mathbb{V}) \to \SL_2 \mbox{ such that } \rho(\alpha(p))\in \SL_2^{1} \mbox{ for all }p\in {\mathcal{P}}^{\partial} \}.$$
 Said differently, its algebra of regular functions is 
 $$ \mathcal{O}[\overline{\mathcal{R}}_{\SL_2}(\mathbf{\Sigma})] := \quotient{ \mathcal{O}[\mathcal{R}_{\SL_2}(\mathbf{\Sigma})]}{ ( X_{++}^{\alpha(p)}, p\in {\mathcal{P}}^{\partial})}.$$
 \end{enumerate}
 
 \end{definition}
 
 Let $\Arc(\mathbf{\Sigma})$ be the set of oriented arcs in $\mathbf{\Sigma}$.
 
 \begin{theorem}\label{theorem_classical_limit}(\cite[Theorem $3.18$]{KojuQuesneyClassicalShadows}, see also \cite[Theorem $4.7$]{KojuPresentationSSkein}) Let $\mathbf{\Sigma}$ be an essential marked surface. There exists a map $w: \Arc(\mathbf{\Sigma})\to \{0, 1\}$ and an isomorphism $\Psi_w: \mathcal{S}_{+1}(\mathbf{\Sigma}) \xrightarrow{\cong} \mathcal{O}[\mathcal{R}_{\SL_2}(\mathbf{\Sigma})]$ characterized by the formula:
$$ \Psi_w \begin{pmatrix} \alpha_{++} & \alpha_{+-} \\ \alpha_{-+} & \alpha_{--} \end{pmatrix} = (-1)^{w(\alpha)} \begin{pmatrix} 0 & -1 \\ 1 & 0\end{pmatrix} \begin{pmatrix} X_{++}^{\alpha} & X_{+-}^{\alpha} \\ X_{-+}^{\alpha} & X_{--}^{\alpha} \end{pmatrix} = (-1)^{w(\alpha)} \begin{pmatrix} -X_{-+}^{\alpha} & -X_{--}^{\alpha} \\ X_{++}^{\alpha} & X_{+-}^{\alpha} \end{pmatrix}, \quad \mbox{ for all }\alpha \in \Arc(\mathbf{\Sigma}).$$
 \end{theorem}
 
 \begin{corollary}\label{coro_classical_limit}  Let $\mathbf{\Sigma}$ be an essential marked surface. There exists a map $w: \Arc(\mathbf{\Sigma})\to \{0, 1\}$ and an isomorphism $\overline{\Psi}_w: \overline{\mathcal{S}}_{+1}(\mathbf{\Sigma}) \xrightarrow{\cong} \mathcal{O}[\overline{\mathcal{R}}_{\SL_2}(\mathbf{\Sigma})]$ characterized by the formula:
$$ \Psi_w \begin{pmatrix} \alpha_{++} & \alpha_{+-} \\ \alpha_{-+} & \alpha_{--} \end{pmatrix} = (-1)^{w(\alpha)} \begin{pmatrix} -X_{-+}^{\alpha} & -X_{--}^{\alpha} \\ X_{++}^{\alpha} & X_{+-}^{\alpha} \end{pmatrix}, \quad \mbox{ for all }\alpha \in \Arc(\mathbf{\Sigma}).$$
Therefore $X(\mathbf{\Sigma})\cong \overline{\mathcal{R}}_{\SL_2}(\mathbf{\Sigma})$.
 \end{corollary}

\begin{proof} The isomorphism $\Psi_w$ sends a bad arc $\alpha(p)_{-+}$ to the element $(-1)^{w(\alpha(p))}X_{++}^{\alpha(p)}$ so induces $\overline{\Psi}_w$ by passing to the quotient. \end{proof}
 
 \subsection{Smooth loci of moduli spaces}
 
 Let $\Gamma$ be a presenting graph for $\mathbf{\Sigma}$. Since $\Sigma$ retracts to $\Gamma$, we obtain an isomorphism
 $$ \varphi_{\Gamma}: \mathcal{R}_{\SL_2}(\mathbf{\Sigma}) \xrightarrow{\cong} (\SL_2(\mathbb{C}))^{\mathcal{E}(\Gamma)}, \quad \varphi_{\Gamma}: \rho \mapsto (\rho(\beta))_{\beta \in \mathcal{E}(\Gamma)}.$$
 In particular $\mathcal{R}_{\SL_2}(\mathbf{\Sigma})$ is smooth. 
 
 \begin{example}\label{example_presenting_graph} Let $\mathbf{\Sigma}$ be a connected essential marked surface of genus $g$ and let us define a presenting graph $\Gamma$. 
 Denote by $\mathcal{A}=\{a_0, \ldots, a_n\}$ the boundary arcs, by $\pi_0(\partial \Sigma) = \{\partial_0, \ldots, \partial_r\}$ the boundary components with $a_0\subset \partial_0$ and write  $v_i:=v_{a_i}$. 
Let $\overline{\Sigma}$ be the surface obtained from $\Sigma$ by gluing a disc along each boundary component $\partial_i$ for $1\leq i \leq r$, and choose $\lambda_1, \mu_1, \ldots, \lambda_g, \mu_g$ some paths in $\pi_1(\Sigma, v_0)$, such that their images in $\overline{\Sigma}$ generate the free group $\pi_1(\overline{\Sigma}, v_0)$ (said differently, the $\lambda_i$ and $\mu_i$ are longitudes and meridians of $\Sigma$). For each  $p\in \mathring{\mathcal{P}}$ choose a peripheral curve $\gamma_p \in \pi_1(\Sigma, v_0)$ encircling $p$ once. Eventually, for each boundary component $\partial_j$, with $1\leq j \leq r$, containing a boundary arc $a_{k_j} \subset \partial_j$ chosen arbitrarily,  choose a path $\delta_{\partial_j} : v_0 \rightarrow v_{k_j}$. The set 
$$\mathbb{G}:= \{ \lambda_i, \mu_i, \gamma_p,  \alpha(p_{\partial}), \delta_{\partial_j} | 1\leq i \leq g, p\in \mathring{\mathcal{P}}, p_{\partial} \in \mathcal{P}^{\partial},  1\leq j \leq r\}$$
is a generating set for $\Pi_1(\Sigma, \mathbb{V})$ (see Figure \ref{fig_generators_final} for an illustration).
 Moreover each of its generators which is of the form $\gamma_p$ of $\alpha(p_{\partial})$ can be expressed as a composition of the other ones and their inverse, therefore a set $\mathcal{E}(\Gamma)$ obtained from $\mathbb{G}$ by removing one of the elements of the form $ \alpha(p_{\partial})$ or $\gamma_p$, forms a set of edges of a presenting graph $\Gamma$.
Note that $\mathbb{G}$ has cardinality $2g-1+|\mathcal{A}|+n$, where $\Sigma=\Sigma_{g,n}$ has genus $g$ with $n$  boundary components, so one has an isomorphism $\mathcal{R}_{\SL_2}(\mathbf{\Sigma}) \cong (\SL_2(\mathbb{C}))^{2g-2+|\mathcal{A}|+n}$. 

\begin{figure}[!h] 
\centerline{\includegraphics[width=9cm]{generators_final.eps} }
\caption{The geometric representatives of a set of generators for $\Pi_1(\Sigma, \mathbb{V})$.} 
\label{fig_generators_final} 
\end{figure} 

\end{example}
 
 \begin{theorem}\label{theorem_smooth} Let $\mathbf{\Sigma}$ be a connected marked surface with at least two boundary arcs. Then $X(\mathbf{\Sigma})$ is a smooth variety.
 \end{theorem}
 
 For $g=\begin{pmatrix} g_{++} & g_{+-} \\ g_{-+} & g_{--} \end{pmatrix} \in \SL_2$, the Zariski tangent space $T_g\SL_2$ is: 
 \begin{multline*}
  T_g\SL_2 = \left\{ X= \begin{pmatrix} X_{++} & X_{+-} \\ X_{-+} & X_{--} \end{pmatrix}, \mbox{s.t.} \det(g+\varepsilon X)\equiv 1 \pmod{\varepsilon^2} \right\} \\ =  \left\{ X= \begin{pmatrix} X_{++} & X_{+-} \\ X_{-+} & X_{--} \end{pmatrix}, \mbox{s.t.} g_{--}X_{++} + g_{++}X_{--} = g_{-+}X_{+-} + g_{+-}X_{-+} \right\}. 
  \end{multline*}
 Similarly, if $g\in \SL_2^{1}$ (i.e. if $g_{++}=0$) then 
 $$ T_g\SL_2^{1}= \left\{X= \begin{pmatrix} 0& X_{+-} \\ X_{-+} & X_{--} \end{pmatrix}, \mbox{s.t.} g_{+-}X_{-+} + g_{-+}X_{+-}=0 \right\}.$$
 
 \begin{lemma}\label{lemma_smooth}
 \begin{enumerate}
 \item Let $X=\SL_2^{1} \times \SL_2$ and consider the regular map $F: X\to \mathbb{C}$, $F(h,g)=(hg)_{--}$. Let $x=(h,g)\in X$. Then there exists $v \in T_h \SL_2^{1} \subset T_h \SL_2^{1} \oplus T_g\SL_2=T_x X$ such that $D_xF(v)\neq 0$. 
 \item 
 Let $Y:= \SL_2^{1} \times (\SL_2)^2$ and consider the regular map $F: Y\to \mathbb{C}$, $F(h,g,f):= (ghg^{-1}f)_{--}$. Let $y=(h,g,f)\in Y$. Then there exists $v \in T_h \SL_2^{1}\oplus T_g\SL_2 \subset T_h \SL_2^{1}\oplus T_g\SL_2\oplus T_f \SL_2=T_yY$ such that $D_yF(v)\neq 0$.
 \end{enumerate}
 \end{lemma}
 
 \begin{proof}
 \par $(1)$ Let $X_1^h, X_2^h \in T_h\SL_2^{1}\subset T_xX$ be the vectors $X_1^h= \begin{pmatrix} 0 & 0 \\ 0 & 1 \end{pmatrix}$, $X_2^h= \begin{pmatrix} 0 & h_{+-} \\ -h_{-+} & 0 \end{pmatrix}$. Then $D_xF(X_1^h)= g_{--}$ and $D_xF(X_2^h)= -g_{+-}$. Since $\det(g)=1$, either $g_{--} \neq 0$ or $g_{+-}\neq 0$.
 \par $(2)$ Consider the vectors $X_1^h, X_2^h, X^g \in T_yY=T_h\SL_2^{1} \oplus T_g\SL_2 \oplus T_f\SL_2$ defined by $X_1^h= \begin{pmatrix} 0 & 0 \\ 0 & 1 \end{pmatrix} \in T_h \SL_2^{1} \subset T_yY$, $X_2^h= \begin{pmatrix} 0 & h_{+-} \\ -h_{-+} & 0 \end{pmatrix} \in T_h \SL_2^{1} \subset T_yY$ and $X^g= \begin{pmatrix} 0 & g_{+-} \\ -g_{-+} & 0 \end{pmatrix} \in T_g \SL_2 \subset T_yY$. 
 One has $D_yF(X_1^h)= g_{--} (g^{-1}f)_{--}$ and $D_yF(X_2^h)= h_{+-}g_{-+}(g^{-1}f)_{--} - h_{-+} g_{--}(g^{-1}f)_{+-}$ so $D_yF(X_1^h)=D_yF(X_2^h)=0$ if and only if $g_{--}=(g^{-1}f)_{--}=0$. In this case, one finds that 
  $$ D_yF(X^g) = -g_{+-}(gh)_{-+}f_{--} - g_{-+}(hg^{-1}f)_{+-} + g_{-+}(gh)_{--}f_{--} = (g_{-+})^2 h_{+-}f_{--}.$$
  Since $g\in \SL_2$, $g_{+-}g_{-+}=-1$ and $g_{-+}\neq 0$. Similarly, $h_{+-}\neq 0$. If $f_{--}=0$ then $f_{+-}\neq 0$ and  the equality $(g^{-1}f)_{--}=0$ would imply $g_{-+}f_{+-}=0$ which is a contradiction. Therefore $f_{--}\neq 0$ and $D_yF(X^g)\neq 0$.
 \end{proof}
 
 
 \begin{proof}[Proof of Theorem \ref{theorem_smooth}] Let $g$ be the genus of $\Sigma$ and $n$ the number of its boundary components. Let $\mathbb{G}$ be the generating set of Example \ref{example_presenting_graph}. Let us first suppose  that $\mathring{\mathcal{P}}\neq \emptyset$. In this case, we can remove from $\mathbb{G}$ one generator $\gamma_p$ to obtain a presenting graph $\Gamma$ such that every corner arc is an edge of $\Gamma$. We obtain an isomorphism
 $$ X(\mathbf{\Sigma}) \xrightarrow[\cong]{\Psi_w^*} \overline{\mathcal{R}}_{\SL_2}(\mathbf{\Sigma}) \xrightarrow[\cong]{\varphi_{\Gamma}} (\SL_2(\mathbb{C}))^{2g-2+n} \times (\SL_2^{1})^{|\mathcal{A}|} .$$
 So $ X(\mathbf{\Sigma})$ is smooth and irreducible because both $\SL_2(\mathbb{C})$ and $\SL_2^{1}\cong \mathbb{C}^* \times \mathbb{C}$ are smooth and irreducible. 
 If $\mathring{\mathcal{P}}=\emptyset$, let $p_0 \in \mathcal{P}^{\partial}$, let $\Gamma$ be the presenting graph obtained from $\mathbb{G}$ by removing $\alpha_{p_0}$ and consider the variety
 $$ \mathcal{R}' := \{ \rho : \Pi_1(\Sigma, \mathbb{V}) \to \SL_2, \mbox{ such that } \rho(\alpha(p)) \in \SL_2^{1} \mbox{ for all }p\in \mathcal{P}^{\partial} \setminus \{ p_0\} \}.$$
 The morphism $\varphi_{\Gamma}$ restricts to an isomorphism $\varphi_{\Gamma}: \mathcal{R}' \cong (\SL_2(\mathbb{C}))^{2g+n-1} \times (\SL_2^{1})^{|\mathcal{A}| -1}$, so $\mathcal{R}'$ is a smooth irreducible variety. 
 Let $F: \mathcal{R}' \to \mathbb{C}$ be the regular function $F=X_{++}^{\alpha(p_0)}$, i.e. $F(\rho)=\rho(\alpha(p_0))_{++}$. One has $ \overline{\mathcal{R}}_{\SL_2}(\mathbf{\Sigma})= F^{-1}(0)$ so it suffices to prove that for every $\rho \in \mathcal{R}'$, one has $D_{\rho}F\neq 0$. Using $\varphi_{\Gamma}$ we identify the Zariski tangent space $T_{\rho}\mathcal{R}'$ with 
 $$ T_{\rho}\mathcal{R}' \cong \oplus_{p\in \mathcal{P}^{\partial} \setminus \{p_0\}} T_{\rho(\alpha(p))} \SL_2^{1} \oplus \oplus_{\beta \in \mathbb{G}\setminus\{\alpha(p)\} } T_{\rho(\beta)}\SL_2.$$
 
 
\par  First suppose that $\mathbf{\Sigma}$ contains two boundary arcs in the same boundary component $\partial$. We suppose that $p_0 \in \partial$ so one has two puncture $p_0,p_1\in \partial$ consecutive in the cyclic order of $\partial$.  
 By definition of $\mathbb{G}$, there exists  a path $\beta$ which is a composition of the elements of $\mathbb{G}\setminus \{\alpha(p_0), \alpha(p_1)\}$ such that $\alpha(p_0)\alpha(p_1)\beta= 1_{s(\alpha(p_0))}$. Therefore
 $$ F(\rho) = \rho( \alpha(p_0))_{++}= \rho( \alpha(p_0)^{-1})_{--} = \left(\rho(\alpha(p_1)) \rho(\beta) \right)_{--}.$$
 By the first item of Lemma \ref{lemma_smooth}, for every $\rho \in \mathcal{R}'$, there exists $v\in T_{\rho(\alpha(p_1))}\SL_2^{1} \subset T_{\rho}\mathcal{R}'$ such that $D_{\rho}F(v)\neq 0$.
 \par If $\mathbf{\Sigma}$ only contains boundary components having a single boundary arc, one can find a boundary component $\partial$ whose unique puncture $p_1\in \partial$ is distinct from $p_0$. By definition of $\mathbb{G}$, there exists  a path $\beta$ which is a composition of the elements of $\mathbb{G}\setminus \{\alpha(p_0), \alpha(p_1), \delta_{\partial}\}$ such that $\alpha(p_0)^{-1}=\delta_{\partial} \alpha(p_1) \delta_{\partial}^{-1} \beta$, so 
 $$F(\rho)= \left( \rho(\delta_{\partial}) \rho(\alpha(p_1)) \rho(\delta_{\partial})^{-1} \rho(\beta) \right)_{--}.$$
 By the second item of Lemma \ref{lemma_smooth},  for every $\rho \in \mathcal{R}'$, there exists $v\in T_{\rho(\alpha(p_1))}\SL_2^{1}\oplus T_{\rho(\delta_{\partial})}\SL_2 \subset T_{\rho}\mathcal{R}'$ such that $D_{\rho}F(v)\neq 0$. 
 In every cases, we have thus proved that $D_{\rho}F\neq 0$ for every $\rho \in \mathcal{R}'$, so $X(\mathbf{\Sigma})$ is smooth.
 \end{proof}
 
 \begin{theorem}\label{theorem_smooth2} Let $\mathbf{\Sigma}$ be an essential marked surface with at least two boundary arcs. Then the regular locus of $\widehat{X}(\mathbf{\Sigma})$ is the subset of elements $\widehat{x}=(x, h_p, h_{\partial})$ such that for every $p\in \mathring{\mathcal{P}}$ then $h_p\notin \{\pm (A^n + A^{-n}), n\in \{1, \ldots, (N-1)/2\} \}$. 
 \end{theorem}
 
 \begin{proof}
 For $z\in \mathbb{C}$ and $z^{1/N}$ an $N$-th root of $z$, the equation $T_N(X)=z+z^{-1}$ has solutions $X= z^{n/N} + z^{-n/N}$ for $n\in \{0, \ldots, N-1\}$ counted with multiplicity, i.e. $T_N(X)-(z+z^{-1})= \prod_{n=0}^{N-1} (X- z^{n/N}-z^{-n/N})$. Therefore, for $c\in \mathbb{C}$, the polynomial $P_{N,c}(X):= T_N(X)-c$ has roots of multiplicity $\geq 2$ if and only if $c=\pm 2$ in which case the multiple roots are $X= \pm (A^n+A^{-n})$ with   $n\in \{1, \ldots, (N-1)/2)\}$.
 The projection $p: \widehat{X}(\mathbf{\Sigma}) \to X(\mathbf{\Sigma})$ is a finite covering branched at the points $x$ such that $\chi_x(\gamma_p)=\pm 2$ for some $p\in \mathring{\mathcal{P}}$ and by, the preceding discussion, a point $\widehat{x}$ has ramification index $1$ if and only if for all $p\in \mathring{\mathcal{P}}$ one has $h_p\notin \{\pm (A^n + A^{-n}), n\in \{1, \ldots, (N-3)/2\} \}$. These points are therefore the smooth points of $\widehat{X}(\mathbf{\Sigma})$.
 \end{proof}
 
 \begin{remark} By the above proof, $\widehat{X}(\mathbf{\Sigma})$ is not a variety but rather an affine scheme when $\mathring{\mathcal{P}}\neq \emptyset$ because $Z_{\mathbf{\Sigma}}= \mathcal{O}[\widehat{X}(\mathbf{\Sigma})]$ is not reduced. Note that $\widehat{X}(\mathbf{\Sigma})$ admits a natural structure of smooth differentiable manifold where every point is smooth (so here we consider the singular locus of $\widehat{X}(\mathbf{\Sigma})$ as a scheme and not as a manifold).
 \end{remark}
 
 \subsection{Behavior for the gluing operation}
 
 The goal of this subsection is to prove the 
 
 \begin{theorem}\label{theorem_gluing_surjective}
 Let  $\mathbf{\Sigma}_1$ and $\mathbf{\Sigma}_2$ be connected marked surfaces and for $i=1,2$ let $a_i$ be a boundary arcs of $\mathbf{\Sigma}_i$ such that the connected component of $\partial \Sigma_i$ which contains $a_i$ also contains at least another boundary arc. 
 Then the maps $p_{a_1\# a_2}: {X}(\mathbf{\Sigma}_1)\times  {X}(\mathbf{\Sigma}_2) \to {X}(\mathbf{\Sigma}_1\cup_{a_1\# a_2} \mathbf{\Sigma}_2)$ and $\widehat{p}_{a_1\# a_2}: \widehat{X}(\mathbf{\Sigma}_1)\times  \widehat{X}(\mathbf{\Sigma}_2) \to \widehat{X}(\mathbf{\Sigma}_1\cup_{a_1\# a_2} \mathbf{\Sigma}_2)$ are surjective.
 \end{theorem}
 
 To prove Theorem \ref{theorem_gluing_surjective}, it is useful to introduce an alternative geometric interpretation of $X(\mathbf{\Sigma})$ defined in \cite{CostantinoLe19}. Let $\mathbf{\Sigma}$ be a connected essential surface and fix a Riemannian metric on $\Sigma$. Let $\pi: U\Sigma \to \Sigma$ be the unitary tangent bundle, so a point $\overrightarrow{v}\in U\Sigma$ is given by a pair $\overrightarrow{v}=(p,v)$ where $p\in \Sigma$ and $v\in T_p\Sigma$ has norm one.
 For such a  $\overrightarrow{v}=(p,v)$, write $-\overrightarrow{v}:= (p, -v)$. 
 \par Recall that the orientation of $\Sigma$ induces an orientation of $\partial \Sigma$.
    For $p \in \partial \Sigma$, let $\overrightarrow{p} \in U\Sigma$ be $\overrightarrow{p}:=(p,v)$ where $v\in T_p \partial \Sigma \subset T_p\Sigma$ is the unitary tangent vector oriented in the direction of $\partial \Sigma$. Said differently, if $n\in T_p\Sigma$ is a vector normal to $\partial \Sigma$ which points outside of $\Sigma$, then $(n,v)$ is a positive basis of $T_p\Sigma$. Write $\widehat{\mathbb{V}} := \{ \pm \overrightarrow{v_a}, a\in \mathcal{A} \}$ (recall that $v_a$ is a point in the middle of the boundary arc $a$). 
 \par For $\overrightarrow{v} \in U\Sigma$ with $p=\pi(\overrightarrow{v})$,  let $\theta_{\overrightarrow{v}} \in \pi_1(U\Sigma, \overrightarrow{v})$ be the homotopy class of the path in $U\Sigma$ which lies inside the circle fiber $\pi^{-1}(p)=T_p\Sigma$ and makes a single positive turn. Also denote by $\theta_{\overrightarrow{v}}^{1/2}$ be the homotopy class of the path starting at $\overrightarrow{v}$ and ending at $-\overrightarrow{v}$ which makes a positive half turn in the fiber $T_p\Sigma$. 
 \par Let $\alpha$ be an oriented arc of $\mathbf{\Sigma}$ with endpoints in some boundary arcs $a$ and $b$ such that $\alpha$ is oriented from $a$ to $b$. Let $c_{\alpha}: [0,1]\to \Sigma$ be an oriented embedding such that $(1)$ $c_{\alpha}([0,1])$ is isotopic to $\alpha$ with the same orientation, $(2)$ $c_{\alpha}'(0)= - \overrightarrow{v_a}$ and $c_{\alpha}'(1)=\overrightarrow{v_b}$ and $(3)$ $c_{\alpha}'(t)$ has norm $1$ for all $t\in [0,1]$. We denote by $\widehat{\alpha}$ the homotopy class of the path $[0,1] \to U\Sigma$, $t\mapsto (c_{\alpha}(t), c'_{\alpha}(t))$. 
 
 \begin{definition}[Twisted relative representation varieties]
 \begin{enumerate}
 \item The \textit{twisted relative representation variety} $\mathcal{R}^{tw}_{\SL_2}(\mathbf{\Sigma})$  is the set of functors $\widehat{\rho}: \Pi_1(U\Sigma, \widehat{\mathbb{V}}) \to \SL_2$ such that $\widehat{\rho}(\theta^{1/2}_{\overrightarrow{v}})= \begin{pmatrix} 0 & -1\\ 1 & 0 \end{pmatrix}$ for all $\overrightarrow{v} \in \widehat{\mathbb{V}}$. It admits a natural structure of affine variety as in Definition \ref{def_modulispaces}
 \item The  \textit{reduced twisted relative representation variety} $\overline{\mathcal{R}}^{tw}_{\SL_2}(\mathbf{\Sigma})$ is the subvariety of  $\mathcal{R}^{tw}_{\SL_2}(\mathbf{\Sigma})$  of functors $\widehat{\rho}$ such that $\widehat{\rho}(\widehat{\alpha(p)})$ is upper triangular for all $p \in \mathcal{P}^{\partial}$.
 \end{enumerate}
 \end{definition}
 
 \begin{theorem}(\cite[Theorem $8.12$]{CostantinoLe19}) There exists an isomorphism $\Psi_{\mathbf{\Sigma}} : X(\mathbf{\Sigma}) \xrightarrow{\cong} {\mathcal{R}}^{tw}_{\SL_2}(\mathbf{\Sigma})$ sending a character $\chi : \mathcal{S}_{+1}(\mathbf{\Sigma}) \to \mathbb{C}$ to the functor $\widehat{\rho}: \Pi_1(U\Sigma, \widehat{\mathbb{V}}) \to \SL_2$ characterized by the fact that for every arc $\alpha$, then 
 $$ \widehat{\rho}(\widehat{\alpha}) = \begin{pmatrix}\chi(\alpha_{++}) & \chi(\alpha_{+-}) \\ \chi(\alpha_{-+}) & \chi(\alpha_{--})\end{pmatrix}.$$
 \end{theorem}
 
 In particular, the ring isomorphism $\Psi_{\mathbf{\Sigma}}^{*}$ sends a bad arc $\alpha(p)_{-+}$ to the function $X_{-+}^{\widehat{\alpha(p)}}$ which sends $\widehat{\rho}$ to the lower left matrix coefficient of $\widehat{\rho}(\widehat{\alpha(p)})$, so we get the 
 
 \begin{corollary}
  The isomorphism $\Psi_{\mathbf{\Sigma}}$ restricts to an isomorphism (still denoted by the same letter) $\Psi_{\mathbf{\Sigma}}: \overline{X}(\mathbf{\Sigma}) \xrightarrow{\cong} \overline{\mathcal{R}}^{tw}_{\SL_2}(\mathbf{\Sigma})$.
  \end{corollary}
  
  
  %Let $\Gamma$ be a presenting graph for $\mathbf{\Sigma}$. Then we get an isomorphism 
   %$$ \varphi^{tw}_{\Gamma}: \mathcal{R}^{tw}_{\SL_2}(\mathbf{\Sigma}) \xrightarrow{\cong} (\SL_2(\mathbb{C}))^{\mathcal{E}(\Gamma)}, \quad \varphi^{tw}_{\Gamma}: \widehat{\rho} \mapsto (\widehat{\rho}(\widehat{\beta}))_{\beta \in \mathcal{E}(\Gamma)}.$$

Now consider $\mathbf{\Sigma}_1$ and $\mathbf{\Sigma}_2$ with $a_1$ and $a_2$ as in Theorem \ref{theorem_gluing_surjective} and write $\mathbf{\Sigma}:=  \mathbf{\Sigma}_1\cup_{a_1\# a_2} \mathbf{\Sigma}_2$.
For $i=1,2$, let $\iota_i : \Sigma_i \hookrightarrow \Sigma$ be the inclusion map.
 Fix a Riemannian metric on $\Sigma$ which induces, by restriction, some Riemannian metrics on $\Sigma_1$ and $\Sigma_2$. Let $\overrightarrow{w} \in U\Sigma$ be the image of $\overrightarrow{v_{a_1}}\in U\Sigma_1 \xrightarrow{\subset} U\Sigma$ by $(\iota_1)_*$. Note that the image of $\overrightarrow{v_{a_2}}\in U\Sigma_2$ in $U\Sigma$ by $(\iota_2)_*$ is $-\overrightarrow{v}$. 
 Write $\Pi_1:= \Pi_1(U\Sigma_1, \mathbb{V}_1)$, $\Pi_2:= \Pi_1(U\Sigma_2, \mathbb{V}_2)$ and $\Pi:= \Pi_1(U\Sigma, \mathbb{V} \cup \{\overrightarrow{w}\})$ and consider the variety: 
 $$ \mathcal{R}^{w}:= \{ \widehat{\rho}\in \Hom(\Pi, \SL_2), \widehat{\rho}(\theta^{1/2}_{\overrightarrow{v}})= \begin{pmatrix} 0 & -1\\ 1 & 0 \end{pmatrix} \mbox{, for all }\overrightarrow{v}\in \widehat{\mathbb{V}} \}.$$
 
 For $i=1,2$, the embedding $\iota_i : \Sigma_i \hookrightarrow \Sigma$ induces a fully faithful functor $(\iota_i)_*: \Pi_i \hookrightarrow \Pi$ so we can consider $\Pi_i$ as a full subcategory of $\Pi$. We are in the following situation: 
 
 \begin{lemma}\label{lemma_groupoid} Let $G$ be a groupoid and $G_1, G_2\subset G$ be two full subcategories of $G$ and $v\in G$ an object such that 
 
 \begin{enumerate}
 \item one has $\Ob(G_1)\cup \Ob(G_2)= \Ob(G)$ and $\Ob(G_1)\cap \Ob(G_2)=\{ v\}$; 
 \item for $v_1\in G_1$ and $v_2\in G_2$, the composition map $G_1(v_1, v)\times G_2(v, v_2) \to G(v_1,v_2)$ is a bijection.
 \end{enumerate}
 Then the restriction map $\res: \Hom(G, \SL_2) \to \Hom(G_1, \SL_2) \times \Hom(G_2, \SL_2)$ sending $\rho$ to the pair $(\restriction{\rho}{G_1}, \restriction{\rho}{G_2})$, is a bijection.
 \end{lemma}
 

 \begin{proof}
Consider the map  $$F: \Hom(G_1, \SL_2) \times \Hom(G_2, \SL_2) \xrightarrow{\cong} \Hom(G, \SL_2)$$ which sends a pair $(\rho_1, \rho_2)$ to the functor $\rho$ such that for $v_1, v_2 \in G$, the restriction $\rho: G(v_1,v_2) \to \SL_2(\mathbb{C})$ is equal to $(i)$ the restriction $\rho_1: G_1(v_1,v_2) \to \SL_2(\mathbb{C})$ if $v_1, v_2 \in \Pi_1$, $(ii)$ to the restriction $\rho_2: G_2(v_1,v_2) \to \SL_2(\mathbb{C})$ if $v_1,v_2\in G_2$, $(iii)$ to the composition $G(v_1,v_2) \cong G_1(v_1, v) \times G_2(v, v_2) \xrightarrow{(\rho_1, \rho_2)}\SL_2(\mathbb{C}) \times \SL_2(\mathbb{C}) \xrightarrow{ \times} \SL_2(\mathbb{C})$ if $v_1\in G_1$, $v_2\in G_2$, $(iv)$ to a similar composition map with $1$ and $2$ exchanged if $v_1\in G_2$ and $v_2\in G_1$. $F$ is clearly the inverse of  $\res$.
\end{proof}
The morphism $F$ of the proof, obtained by replacing $(G,G_1,G_2, v) $  by $(\Pi, \Pi_1, \Pi_2, \overrightarrow{w})$,   induces by restriction an isomorphism (still denoted by the same letter): 
 $$ F :\mathcal{R}^{tw}_{\SL_2}(\mathbf{\Sigma}_1) \times \mathcal{R}^{tw}_{\SL_2}(\mathbf{\Sigma}_2) \xrightarrow{\cong} \mathcal{R}^w.$$
The inclusion $\Pi_1(U\Sigma, \widehat{\mathbb{V}}) \subset \Pi_1(U\Sigma, \widehat{\mathbb{V}} \cup \{\overrightarrow{w}\})$ induces a regular (restriction) map $G: \mathcal{R}^w \to \mathcal{R}^{tw}_{\SL_2}(\mathbf{\Sigma})$.

\begin{lemma}\label{lemma_surjectivity_modulispaces}
The composition $G\circ F : \mathcal{R}^{tw}_{\SL_2}(\mathbf{\Sigma}_1) \times \mathcal{R}^{tw}_{\SL_2}(\mathbf{\Sigma}_2) \to  \mathcal{R}^{tw}_{\SL_2}(\mathbf{\Sigma})$ is surjective and induces by restriction a surjective map 
 $\varphi: \overline{\mathcal{R}}^{tw}_{\SL_2}(\mathbf{\Sigma}_1) \times \overline{\mathcal{R}}^{tw}_{\SL_2}(\mathbf{\Sigma}_2) \to  \overline{\mathcal{R}}^{tw}_{\SL_2}(\mathbf{\Sigma})$.
 \end{lemma}
 
 In order to prove Lemma \ref{lemma_surjectivity_modulispaces}, we first prove the following:
 
 \begin{lemma}\label{lemma_prout} Let $B^+\subset \SL_2(\mathbb{C})$ be the subgroup of upper triangular matrices and let $A,B,C,D \in \SL_2(\mathbb{C})$ be such that $AB \in B^+$ and $CD\in B^+$. Then there exists $g \in \SL_2(\mathbb{C})$ such that $Ag$, $g^{-1}B$, $Dg^{-1}$ and $gC$ are in $B^+$.
 \end{lemma}
 
 \begin{proof} Write $A=\begin{pmatrix} a_{++} & a_{+-} \\ a_{-+} & a_{--} \end{pmatrix}$ with similar notations for $B,C,D,g$. Then 
 $$ \left\{ \begin{array}{l} Ag \in B^+ \\ g^{-1}B \in B^+ \end{array} \right. \Leftrightarrow \begin{pmatrix} a_{-+} & a_{--} \\ b_{-+} & -b_{++} \end{pmatrix} \begin{pmatrix} g_{++} \\ g_{-+} \end{pmatrix} = \begin{pmatrix} 0 \\ 0 \end{pmatrix}.$$
 Since $AB \in B^+ \Leftrightarrow \det \begin{pmatrix} a_{-+} & a_{--} \\ b_{-+} & -b_{++} \end{pmatrix} =0$, one can find $(g_{++}, g_{-+})\neq (0,0)$ which satisfies this equation. Similarly, 
 $$ \left\{ \begin{array}{l} Dg^{-1} \in B^+ \\ gC \in B^+ \end{array} \right. \Leftrightarrow \begin{pmatrix} c_{++} & c_{-+} \\ -d_{--} & d_{-+} \end{pmatrix} \begin{pmatrix} g_{-+} \\ g_{--} \end{pmatrix} = \begin{pmatrix} 0 \\ 0 \end{pmatrix}.$$
 Since $CD\in B^+ \Leftrightarrow \det  \begin{pmatrix} c_{++} & c_{-+} \\ -d_{--} & d_{-+} \end{pmatrix} =0$, one can find $(g_{-+}, g_{--})\neq (0,0)$ which satisfies this equation. Up to multiplying $(g_{-+}, g_{--})$ by a non zero scalar, one can further suppose that $g_{++}g_{--}-g_{+-}g_{-+}=1$, so the matrix $g= \begin{pmatrix} g_{++} & g_{+-} \\ g_{-+} & g_{--} \end{pmatrix}$ satisfies the conclusion of the lemma.
 
 
 \end{proof}
 
 \begin{proof}[Proof of Lemma \ref{lemma_surjectivity_modulispaces}] Since $F$ is a bijection, we need to prove that $G$ is surjective. Let us first define an action of $\SL_2(\mathbb{C})$ on $\mathcal{R}^w$ as follows. For $g\in \SL_2(\mathbb{C})$ and $\rho \in \mathcal{R}^w$, we define $g\bullet \rho \in \mathcal{R}^w$ as the functor which sends a path $\beta: v_1 \to v_2$ in  $\Pi_1(U\Sigma, \widehat{\mathbb{V}} \cup \{ \overrightarrow{w}\})$ to 
 $$ (g\bullet \rho) (\beta) := \left\{ \begin{array}{ll}
 \rho(\beta) & \mbox{, if }v_1 \neq \overrightarrow{w}, v_2 \neq \overrightarrow{w}; \\
 \rho(\beta)g & \mbox{, if }v_1 \neq \overrightarrow{w}, v_2 = \overrightarrow{w}; \\
  g^{-1}\rho(\beta) & \mbox{, if }v_1 = \overrightarrow{w}, v_2 \neq \overrightarrow{w}; \\
  g^{-1}\rho(\beta)g & \mbox{, if }v_1 = v_2 = \overrightarrow{w}.
  \end{array} \right. $$
  Clearly $G(g\bullet \rho) = G(\rho)$ so $G$ induces a map $G': \quotient{ \mathcal{R}^w}{\SL_2(\mathbb{C})} \to \mathcal{R}^{tw}_{\SL_2}(\mathbf{\Sigma})$. Let us prove that $G'$ is surjective (it is not difficult to prove that it is an isomorphism).
\par  Let $c$ be a boundary arc of $\mathbf{\Sigma}$ (which exists by the  hypotheses made in Theorem \ref{theorem_gluing_surjective}) and consider an arbitrary path $\beta: \overrightarrow{v_c} \to \overrightarrow{w}$. Let $\rho \in  \mathcal{R}^{tw}_{\SL_2}(\mathbf{\Sigma})$ and define an extension $\widehat{\rho} \in \mathcal{R}^w$ as follows. First, the restriction of $\widehat{\rho}$ to $\Pi_1(U\Sigma, \widehat{\mathbb{V}})$ is $\rho$ and $\widehat{\rho}(\beta)= \mathds{1}_2$. Next for a path $\gamma: \overrightarrow{w} \to \overrightarrow{v}$ with $\overrightarrow{v}\in \widehat{V}$, set $\widehat{\rho}(\gamma):= \rho(\gamma \beta^{-1})$. Similarly for a path $\gamma': \overrightarrow{v} \to \overrightarrow{w}$ set $\widehat{\rho}(\gamma'):= \rho(\beta^{-1}\gamma' )$. Clearly  $\widehat{\rho} \in \mathcal{R}^w$ and $G(\widehat{\rho})=\rho$ so $G$ and $G'$  are surjective. 
 \par Let us prove that the restriction $\varphi$ of $G\circ F$ is surjective as well. Let $p_1, p_1'$ be the two punctures adjacent to $a_1$ and $p_2, p_2'$ the two punctures adjacent to $a_2$ such that while gluing $a_1$ with $a_2$, $p_1$ and $p_2$ have the same image $p\in \partial \Sigma$ and $p_1', p_2'$ have the same image $p'\in \partial \Sigma$ (see Figure \ref{fig_gluing_surjective}). By assumption, $p_1\neq p_1'$ and $p_2\neq p_2'$. 
 Let $\rho \in \overline{\mathcal{R}}^{tw}_{\SL_2}(\mathbf{\Sigma})$. By surjectivity of $G$, there exists $\widehat{\rho} \in \mathcal{R}^w$ such that $G(\widehat{\rho})=\rho$. Then $\widehat{\rho}$ is the image by $F$ of an element of 
$ \overline{\mathcal{R}}^{tw}_{\SL_2}(\mathbf{\Sigma}_1) \times \overline{\mathcal{R}}^{tw}_{\SL_2}(\mathbf{\Sigma}_2) $ if and only if for all $p\in \mathcal{P}^{\partial_1} \cup \mathcal{P}^{\partial_2}$, then $\widehat{\rho}(\widehat{\alpha(p)}) \in B^+$. Since  $G(\widehat{\rho})=\rho \in  \overline{\mathcal{R}}^{tw}_{\SL_2}(\mathbf{\Sigma})$, $\widehat{\rho}(\widehat{\alpha(p)}) \in B^+$ for all $p\notin \{p_1,p_1', p_2, p_2'\}$. Write $A:= \widehat{\rho}(\widehat{\alpha(p_1)})$, $B:=  \widehat{\rho}(\widehat{\alpha(p_2)})$, $C:=  \widehat{\rho}(\widehat{\alpha(p_2')})$, $D:=  \widehat{\rho}(\widehat{\alpha(p_1')})$. Then $AB= \rho (\widehat{\alpha(p)} \in B^+$ and $CD=\rho (\widehat{\alpha(p')})\in B^+$ so by Lemma \ref{lemma_prout}, there exists $g\in \SL_2(\mathbb{C})$ such that  $Ag$, $g^{-1}B$, $Dg^{-1}$ and $gC$ are in $B^+$. Therefore, writing  $\rho':= g\bullet \widehat{\rho}$ one has $\restriction{\rho'}{\Pi_1}\in \overline{\mathcal{R}}^{tw}_{\SL_2}(\mathbf{\Sigma}_1)$, $\restriction{\rho'}{\Pi_2} \in \overline{\mathcal{R}}^{tw}_{\SL_2}(\mathbf{\Sigma}_2)$ and $\varphi(\restriction{\rho'}{\Pi_1}, \restriction{\rho'}{\Pi_2)} = \rho$. So $\varphi$ is surjective.
 
 
 \end{proof}
 
 \begin{proof}[Proof of Theorem \ref{theorem_gluing_surjective}] 
 The surjectivity of $p_{a_1\# a_2}$ follows from the commutativity of the following diagram
 $$ \begin{tikzcd}
 X(\mathbf{\Sigma}_1) \times X(\mathbf{\Sigma}_2) 
 \ar[r, "p_{a_1\# a_2}"] \ar[d, "\Psi_{\mathbf{\Sigma}_1}\times \Psi_{\mathbf{\Sigma}_2}", "\cong"'] &
 X(\mathbf{\Sigma}) \ar[d, "\Psi_{\mathbf{\Sigma}}", "\cong"'] \\
 \overline{\mathcal{R}}^{tw}_{\SL_2}(\mathbf{\Sigma}_1) \times \overline{\mathcal{R}}^{tw}_{\SL_2}(\mathbf{\Sigma}_2) 
 \ar[r, "\varphi"] & 
  \overline{\mathcal{R}}^{tw}_{\SL_2}(\mathbf{\Sigma}) 
 \end{tikzcd} $$
 together with the surjectivity of $\varphi$ proved in Lemma \ref{lemma_surjectivity_modulispaces}. Let us prove that  $\widehat{p}_{a_1\# a_2}$ is surjective as well. Let $\widehat{x}=({x}, h_p, h_{\partial})_{p, \partial} \in \widehat{X}(\mathbf{\Sigma})$ and fix $(x_1,x_2) \in X(\mathbf{\Sigma}_1)\times X(\mathbf{\Sigma}_2)$ such that $p_{a_1\# a_2}(x_1,x_2)=x$. Let $\partial_1$ be the boundary component of $\Sigma_1$ containing $a_1$ and $\partial_2$ the boundary component of $\Sigma_2$ containing $a_2$ and $\partial'$ the boundary component of $\Sigma$ obtained by $1$-surgery on $\partial_1\cup \partial_2$. By definition $h_{\partial'}^N=\chi_x(\alpha_{\partial})$. Since $\theta_{a_1\# a_2}(\alpha_{\partial_1}\otimes \alpha_{\partial_2})=\alpha_{\partial}$, one can find (and do fix) $h_{\partial_1}, h_{\partial_2} \in \mathbb{C}^*$ such that $h_{\partial}=h_{\partial_1}h_{\partial_2}$ and $h_{\partial_1}^N=\chi_{x_1}(\alpha_{\partial_1})$, $h_{\partial_2}^N= \chi_{x_2}(\alpha_{\partial_2})$. Define $(\widehat{x}_1, \widehat{x}_2)\in \widehat{X}(\mathbf{\Sigma}_1)\times \widehat{X}(\mathbf{\Sigma}_2)$ by, for $i=1,2$,  $\widehat{x}_i= (x_i, h_p, h_{\partial})_{p \in \mathring{\mathcal{P}}_i, \partial \in \Gamma_i^{\partial}}$ where for $p\in \mathring{\mathcal{P}}_i  \ \subset \mathring{\mathcal{P}}$, $h_p$ is the puncture invariant of $\widehat{x}$, for $\partial \in \Gamma_i^{\partial} \setminus \{ \partial_i\}$ then $h_{\partial}$ is the boundary invariant of $\widehat{x}$ and $h_{\partial_i}$ is the previously chosen scalar. Then $\widehat{p}_{a_1\# a_2}(\widehat{x}_1, \widehat{x}_2)= \widehat{x}$, thus $\widehat{p}_{a_1\# a_2}$ is surjective.
 
 
 \end{proof}
 
 
 
 \section{Representations of $ \overline{\mathcal{S}}_{A}(\mathbb{D}_1)$}\label{sec_D1}
 
 Let $\mathbb{D}_1=(\Sigma_{0,2}, \{a_L, a_R\})$ be an annulus with two boundary arcs in the same boundary component. It admits a presenting graph with two edges $\alpha, \beta$ represented in Figure \ref{fig_D1} (note that $\alpha$ and $\beta^{-1}$ are corner arcs). The algebra $ \overline{\mathcal{S}}_{A}(\mathbb{D}_1)$ is isomorphic to the simply laced quantum enveloping algebra $U_q\mathfrak{gl}_2$ described as follows.
 
  
 \begin{figure}[!h] 
\centerline{\includegraphics[width=2cm]{arcs_D1.eps} }
\caption{Two oriented arcs $\alpha$ and $\beta$ in $\mathbb{D}_1$.} 
\label{fig_D1} 
\end{figure} 
 
 
 
 \begin{definition}[Quantum enveloping algebra]
 Write $q:= A^2$. The algebra $U_q\mathfrak{gl}_2$ is presented by the generators $K^{\pm 1/2}, L^{\pm 1/2}, E$ and $F$ together with the following relations:

\begin{align*}
&EK^{1/2}= q^{-1}K^{1/2}E; \quad EL^{1/2}= qL^{1/2} E; \quad  FK^{1/2}=qK^{1/2}F; \quad FL^{1/2}= q^{-1}L^{1/2}F; \\
&xy=yx \mbox{, for all }x,y\in \{K^{\pm 1/2}, L^{\pm 1/2} \}; \quad K^{1/2}K^{-1/2}=L^{1/2}L^{-1/2}= 1; \\
&EF-FE= \frac{K-L}{q-q^{-1}}. 
\end{align*}
 \end{definition}
 
 
 \begin{theorem}\label{theorem_skein_QG}(\cite{KojuQGroupsBraidings})
 There is an isomorphism of algebras $\Psi:  U_q\mathfrak{gl}_2 \xrightarrow{\cong}  \overline{\mathcal{S}}_{A}(\mathbb{D}_1)$ defined by 
 \begin{align*}
 {} & \Psi(K^{1/2})= \alpha_{--}, \quad \Psi(K^{-1/2})= \alpha_{++}, \quad \Psi(L^{1/2})= \beta_{--}, \quad \Psi(L^{-1/2})=\beta_{++} \\
 {} & \Psi(E)=\frac{-A}{q-q^{-1}} \alpha_{+-}\alpha_{--}, \quad \Psi(F)= \frac{A^{-1}}{q-q^{-1}}\beta_{--} \beta_{-+}
 \end{align*}
 \end{theorem}
 
 Note that $H_{\partial}:=K^{-1/2}L^{-1/2}$ is a central element such that $\Psi(H_{\partial})=h_{\partial}$ for $\partial$ the connected component containing $a_L$ and $a_R$. Also, if $p$ denotes the only inner puncture of $\mathbb{D}_1$, then 
 \begin{equation}\label{eq_Casimir}
 C:= - \frac{\Psi^{-1}(\gamma_p)H_{\partial}^{-1}}{(q-q^{-1})^2} = EF+\frac{qL +q^{-1}K}{(q-q^{-1})^2} = FE+ \frac{qK+q^{-1}L}{(q-q^{-1})^2}
 \end{equation}
 is the Casimir element of $U_q\mathfrak{gl}_2$. Let $\widetilde{U}_q \mathfrak{sl}_2:= \quotient{U_q\mathfrak{gl}_2}{(H_{\partial} -1)}$ and $U_q\mathfrak{sl}_2 \subset \widetilde{U}_q \mathfrak{sl}_2$ the subalgebra generated by $K^{\pm 1}, E$ and $F$. The simple modules of  $U_q\mathfrak{sl}_2$ have been classified in \cite{DeConciniKacRepQGroups, ArnaudonRoche} (see also  \cite[Chapter $1$ Section $3.3$]{KlimykSchmudgenQGroups}) whereas many indecomposable semi-weight $U_q\mathfrak{sl}_2$ modules were found and studied in \cite{Arnaudon_Uqsl2Rep}. We now provide a classification of the indecomposable (semi) weight $U_q\mathfrak{gl}_2$ modules as follows.
 
 \begin{definition}[Weight representations of $U_q\mathfrak{gl}_2$]\label{def_QGRep}
 We define three families of indecomposable weight $U_q\mathfrak{gl}_2$-modules as follows. For $n\in \mathbb{Z}$, we write $[n]:=\frac{q^n-q^{-n}}{q-q^{-1}}$. 
\begin{enumerate}
\item For $\lambda, \mu\in \mathbb{C}^*$ and $a,b\in \mathbb{C}$, the module $V(\lambda, \mu, a,b)$ has canonical basis $(v_0, \ldots, v_{N-1})$ and module structure given by:
\begin{align*}
& K^{1/2}v_i = \lambda q^{-i}v_i; \quad L^{1/2}v_i = \mu q^i v_i \mbox{, for }i\in \{0, \ldots, N-1\}; \\
& Fv_{N-1}=bv_0; \quad Fv_i=v_{i+1} \mbox{, for }i \in \{0, \ldots, N-2\}; \\
& Ev_0=a v_{N-1}; \quad Ev_{i}= \left( \frac{q^{-i+1}\lambda^2 - q^{i-1}\mu^2}{q-q^{-1}} [i] +ab \right) v_{i-1}\mbox{, for }i\in \{1, \ldots, N-1\}.
\end{align*}

\item For $\lambda, \mu \in \mathbb{C}^*$, $c\in \mathbb{C}$, the module $\widetilde{V}(\lambda, \mu, c)$ has canonical basis $(w_0, \ldots, w_{N-1})$  and module structure given by: 
\begin{align*}
& K^{1/2}w_i= \lambda q^{i}w_i; \quad L^{1/2}w_i= \mu q^{-i}w_i \mbox{, for }i \in \{0, \ldots, N-1\}; \\
& Ew_{N-1}= cw_0; \quad Ew_i= w_{i+1} \mbox{, for }i\in \{0, \ldots, N-2\}; \\
& F w_0= 0; \quad Fw_{i} = \left( \frac{ \mu^2q^{-i+1} - \lambda^2 q^{i-1} }{q-q^{-1}}[i] \right) w_i \mbox{, for }i\in \{1, \ldots, N-1\}.
\end{align*}

\item For $\mu \in \mathbb{C}^*$, $\varepsilon \in \{-1,+1\}$, $0\leq n \leq N-1$, the module $S_{\mu, \varepsilon, n}$ has canonical basis $(e_0, \ldots, e_n)$ and module structure given by: 
\begin{align*}
&K^{1/2}e_i = \varepsilon \mu A^{n-2i} e_i; \quad L^{1/2}e_i= \mu A^{2i-n}e_i \mbox{, for }i\in \{0, \ldots, n\}; \\
&Fe_n=0; \quad Fe_i = e_{i+1} \mbox{, for }i\in \{0,\ldots, n-1\}; \\
&Ee_0=0; \quad Ee_i= \mu^2 [i][n-i+1]e_{i-1}\mbox{, for }i\in \{1,\ldots, n\}.
\end{align*}

\end{enumerate}
 \end{definition}
 
 \begin{definition}[Standard semi-weight representations of $U_q\mathfrak{gl}_2$] \label{def_semiweight_modules} We define three families of indecomposable semi-weight representations as follows.
 \begin{enumerate}
 \item For $\lambda, \mu, b \in \mathbb{C}^*$ and $a\in \mathbb{C}$ such that 
 $$ h_p:= -(q-q^{-1})^2(\lambda\mu)^{-1}ab-\lambda \mu^{-1}q - \mu\lambda^{-1}q^{-1} = \pm (q^n+q^{-n}) \quad \mbox{ for some }n\in \{1, \ldots, (N-1)/2\}$$
 the module $P(\lambda, \mu, a,b)$ has canonical basis $(x_0, \ldots, x_{N-1}, y_0, \ldots, y_{N-1})$ and module structure given by: 
 \begin{align*}
 & K^{1/2}x_i=\lambda q^{-i}x_i, K^{1/2}y_i=\lambda q^{-i}y_i, L^{1/2}x_i= \mu q^i x_i, L^{1/2}y_i=\mu q^i y_i \mbox{ for }i \in \{0, \ldots, N-1\}; \\
 & Fx_{N-1}=b x_0, Fy_{N-1}=by_0, Fx_i= x_{i+1}, Fy_i=y_{i+1} \mbox{ for }i\in \{0, \ldots, N-2\}; \\
 & Ex_0=ax_{N-1}, Ex_i=\left( \frac{q^{-i+1}\lambda^2 - q^{i-1}\mu^2}{q-q^{-1}} [i] +ab \right)x_{i-1} \mbox{ for }i\in \{1, \ldots, N-1\}; \\
 & Ey_0=a y_{N-1}+b^{-1}x_{N-1}, E y_i= \left( \frac{q^{-i+1}\lambda^2 - q^{i-1}\mu^2}{q-q^{-1}} [i] +ab \right)y_{i-1} + x_{i-1}  \mbox{ for }i\in \{1, \ldots, N-1\}.
 \end{align*}
 
 \item For $\lambda, \mu, c \in \mathbb{C}^*$ such that $\lambda \mu^{-1}= \pm q^n$ for some $n\in \{1, \ldots, (N-1)/2\}$, the module $\widetilde{P}(\lambda, \mu, c)$ has canonical basis $(x_0, \ldots, x_{N-1}, y_0, \ldots, y_{N-1})$ and module structure given by: 
 \begin{align*}
 & K^{1/2}x_i= \lambda q^i x_i, K^{1/2}y_i= \lambda q^i y_i, L^{1/2}x_i= \mu q^{-i}x_i, L^{1/2}y_i= \mu q^{-i} y_i  \mbox{ for }i \in \{0, \ldots, N-1\}; \\
 & Ex_{N-1} = cx_0, Ey_{N-1}=cy_0, Ex_i= x_{i+1}, Ey_i= y_{i+1} \mbox{ for }i\in \{0, \ldots, N-2\}; \\
 & Fx_0= 0, Fx_i= \left( \frac{ \mu^2q^{-i+1} - \lambda^2 q^{i-1} }{q-q^{-1}}[i] \right) x_{i-1}  \mbox{ for }i\in \{1, \ldots, N-1\}; \\
 & Fy_0=x_0, Fy_i=  \left( \frac{ \mu^2q^{-i+1} - \lambda^2 q^{i-1} }{q-q^{-1}}[i] \right) y_{i-1} + x_i  \mbox{ for }i\in \{1, \ldots, N-1\}.
 \end{align*}
 
 
 \item For $\mu \in \mathbb{C}^*$, $\varepsilon \in \{-1,+1\}$, $0\leq n \leq N-2$, the module $P_{\mu, \varepsilon, n}$ has canonical basis $(x_0, \ldots, x_{N-1}, y_0, \ldots, y_{N-1})$ and module structure given by: 
\begin{align*}
& K^{1/2} x_i = \varepsilon \mu A^{-2-n-2i}x_i, K^{1/2}y_i= \varepsilon \mu A^{n-2i}y_i, \quad L^{1/2} x_i= \mu A^{n+2i+2} x_i, L^{1/2}y_i=\mu A^{2i-n}y_i \mbox{, for }i\in \{0, \ldots, N-1\}; \\
& F x_i = x_{i+1}, \quad Fy_i= y_{i+1} \quad \mbox{( where }y_N=x_N=0)\mbox{, for }i\in \{0, \ldots, N-1\}; \\
& Ex_0=0, \quad Ex_i= -\mu^2[i][n+i+1] x_{i-1} \mbox{, for }i\in \{1, \ldots, N-1\}; \\
& Ey_i= \mu^2[i][n-i+1]y_{i-1} + x_{N-n+i-2} \mbox{, for }i\in \{0, \ldots, n\}, \quad E y_i = \mu^2[i] [n-i+1]y_{i-1}   \mbox{, for }i\in \{n+1, \ldots, N-1\}. 
\end{align*}
 
 
 \end{enumerate}
 
 \end{definition}
 
 \begin{definition}[Exceptional semi-weight representations]\label{def_exceptional_rep} For $0\leq n \leq N-1$, write $\overline{n}:=N-2-n$. 
 \begin{enumerate}
 \item For $\mu \in \mathbb{C}^*$, $\varepsilon\in \{-1, +1\}$, $0\leq n \leq N-2$, $k\geq 1$, the module $\Omega^{-k}_{\mu, \varepsilon, n}$ has basis $(e_i^j, \overline{e}_{i'}^{j'}, 0\leq i \leq n, 0\leq i' \leq \overline{n}, 1\leq j \leq k, 1\leq j' \leq k+1)$ and module structure given by:
 \begin{align*}
 & K^{1/2}e_i^j = \varepsilon \mu A^{n-2i}e_i^j, K^{1/2}\overline{e}_i^j =\varepsilon \mu A^{\overline{n}-2i}\overline{e}_i^j, L^{1/2}e_i^j = \mu A^{2i-n}e_i^j, L^{1/2}\overline{e}_i^j= \mu A^{2i-\overline{n}}\overline{e}_i^j; \\
 & F e_n^j = \overline{e}_0^{j+1},  F\overline{e}_{\overline{n}}^j = 0, Fe_i^j = e_{i+1}^j \mbox{ for }i\neq n , F \overline{e}_i^j = \overline{e}_{i+1}^j \mbox{ for } i\neq \overline{n}; \\
 & E e_0^j= \overline{e}_{\overline{n}}^j ,  E \overline{e}_0^j = 0,  Ee_i^j = \mu^2 [i][n-i+1] e_{i-1}^j, E \overline{e}_i^j = \mu^2 [i][\overline{n}-i+1]\overline{e}_{i-1}^j \mbox{ for } i\neq 0.
 \end{align*}
 \item  For $\mu \in \mathbb{C}^*$, $\varepsilon\in \{-1, +1\}$, $0\leq n \leq N-2$, $k\geq 1$, the module $\Omega^{k}_{\mu, \varepsilon, n}$ has basis $(e_i^j, \overline{e}_{i'}^{j'}, 0\leq i \leq n, 0\leq i' \leq \overline{n}, 1\leq j \leq k+1, 1\leq j' \leq k)$ and module structure given by:
 \begin{align*}
 & K^{1/2}e_i^j = \varepsilon \mu A^{n-2i}e_i^j, K^{1/2}\overline{e}_i^j =\varepsilon \mu A^{\overline{n}-2i}\overline{e}_i^j, L^{1/2}e_i^j = \mu A^{2i-n}e_i^j, L^{1/2}\overline{e}_i^j= \mu A^{2i-\overline{n}}\overline{e}_i^j; \\
 & Fe_n^{k+1}=0, F e_n^j = \overline{e}_0^{j} \mbox{ for }j\neq k+1,  F\overline{e}_{\overline{n}}^j = 0, Fe_i^j = e_{i+1}^j \mbox{ for }i\neq n , F \overline{e}_i^j = \overline{e}_{i+1}^j \mbox{ for } i\neq \overline{n}; \\
 & Ee_0^1=0, E e_0^j= \overline{e}_{\overline{n}}^{j-1} \mbox{ for }j\neq 1 ,  E \overline{e}_0^j = 0,  Ee_i^j = \mu^2 [i][n-i+1] e_{i-1}^j, E \overline{e}_i^j = \mu^2 [i][\overline{n}-i+1]\overline{e}_{i-1}^j \mbox{ for } i\neq 0.
 \end{align*}
\item   For $\mu \in \mathbb{C}^*$, $\varepsilon\in \{-1, +1\}$, $0\leq n \leq N-2$, $k\geq 1$ and $\lambda= \in \mathbb{CP}^1= \mathbb{C} \cup \{\infty \}$, the module  $M^k_{\mu, \varepsilon, n}(\lambda)$ has basis $(e_i^j, \overline{e}_{i'}^{j'}, 0\leq i \leq n, 0\leq i' \leq \overline{n}, 1\leq j \leq k, 1\leq j' \leq k)$ and module structure given as follows.
\par  If $\lambda \in \mathbb{C}$ then
 \begin{align*}
 & K^{1/2}e_i^j = \varepsilon \mu A^{n-2i}e_i^j, K^{1/2}\overline{e}_i^j =\varepsilon \mu A^{\overline{n}-2i}\overline{e}_i^j, L^{1/2}e_i^j = \mu A^{2i-n}e_i^j, L^{1/2}\overline{e}_i^j= \mu A^{2i-\overline{n}}\overline{e}_i^j; \\
 & Fe_n^k = \lambda \overline{e}_0^k,  F e_n^j = \lambda \overline{e}_0^{j} + \overline{e}_0^{j+1} \mbox{ for }j\neq k,  F\overline{e}_{\overline{n}}^j = 0, Fe_i^j = e_{i+1}^j \mbox{ for }i\neq n , F \overline{e}_i^j = \overline{e}_{i+1}^j \mbox{ for } i\neq \overline{n}; \\
 &  E e_0^j= \overline{e}_{\overline{n}}^j,  E \overline{e}_0^j = 0,  Ee_i^j = \mu^2 [i][n-i+1] e_{i-1}^j, E \overline{e}_i^j = \mu^2 [i][\overline{n}-i+1]\overline{e}_{i-1}^j \mbox{ for } i\neq 0.
 \end{align*}
If $\lambda= \infty \in \mathbb{CP}^1$, then 
\begin{align*}
 & K^{1/2}e_i^j = \varepsilon \mu A^{n-2i}e_i^j, K^{1/2}\overline{e}_i^j =\varepsilon \mu A^{\overline{n}-2i}\overline{e}_i^j, L^{1/2}e_i^j = \mu A^{2i-n}e_i^j, L^{1/2}\overline{e}_i^j= \mu A^{2i-\overline{n}}\overline{e}_i^j; \\
 &  F e_n^j =  \overline{e}_0^{j},  F\overline{e}_{\overline{n}}^j = 0, Fe_i^j = e_{i+1}^j \mbox{ for }i\neq n , F \overline{e}_i^j = \overline{e}_{i+1}^j \mbox{ for } i\neq \overline{n}; \\
 &  E e_0^k= 0, E e_0^j= \overline{e}_{\overline{n}}^{j+1} \mbox{ for }j \neq k,  E \overline{e}_0^j = 0,  Ee_i^j = \mu^2 [i][n-i+1] e_{i-1}^j, E \overline{e}_i^j = \mu^2 [i][\overline{n}-i+1]\overline{e}_{i-1}^j \mbox{ for } i\neq 0.
 \end{align*}

 \end{enumerate}
 \end{definition}
 We postpone the proof of the following theorem to the Appendix \ref{appendix}. Let $\mathcal{C}$ be the category of weight $U_q\mathfrak{gl}_2$ modules and $\overline{\mathcal{C}}$ the category of semi-weight $U_q\mathfrak{gl}_2$ modules.
 
 \begin{theorem}\label{theorem_representations_QG}
 \begin{enumerate}
 \item Every modules $V(\lambda, \mu, a,b), \widetilde{V}(\lambda, \mu, c), S_{\mu, \varepsilon, n}$ are weight indecomposable and every weight indecomposable $U_q\mathfrak{gl}_2$ module is isomorphic to one of them. 
 \item Every modules $P(\lambda, \mu, a,b), \widetilde{P}(\lambda, \mu, c), P_{\mu, \varepsilon, n}, \Omega^k_{\mu, \varepsilon, n}, \Omega^{-k}_{\mu, \varepsilon, n}, M_{\mu, \varepsilon, n}^k(\lambda)$ are semi weight indecomposable and every indecomposable semi-weight module is either isomorphic to one of them or is a weight module.
 \item  The modules $S_{\mu, \varepsilon, n}$ are simple. The module $\widetilde{V}(\lambda, \mu, c)$ is simple if and only if either $c\neq 0$ or $\lambda \mu^{-1} \neq \pm q^{n-1}$ for all $n\in \{1, \ldots, N-1\}$. The module $V(\lambda, \mu, a, b)$ is simple if and only if either $\prod_{i\in \mathbb{Z}/N\mathbb{Z}} (ab+\frac{q^{1-i}\lambda^2 - q^{i-1}\mu^2}{q-q^{-1}}[i])\neq 0$ or $\lambda \mu^{-1} \neq \pm q^{n-1}$ for all $n\in \{1, \ldots, N-1\}$. Any simple $U_q\mathfrak{gl}_2$-module is isomorphic to one of these modules.
 \item The weight representations $S_{\mu, \epsilon, N-1}$, $V(\lambda, \mu, a,b)$ and $\widetilde{V}(\lambda, \mu, c)$ are projective objects in $\mathcal{C}$ and any indecomposable projective object of $\mathcal{C}$ is isomorphic to one of them.
 \item The semi weight representations $S_{\mu, \epsilon, N-1}$, $P_{\mu, \epsilon, n}$, $P(\lambda, \mu, a,b)$ and $\widetilde{P}(\lambda, \mu, c)$ are projective objects in $\overline{\mathcal{C}}$ and any indecomposable projective object of $\overline{\mathcal{C}}$ is isomorphic to one of them.
\end{enumerate}
\end{theorem}
By Theorem \ref{theorem_representations_QG}, given a semi-weight indecomposable module $P$, every simple submodule of $P$ and every simple quotient of $P$ have the same classical shadow; we call \textit{classical shadow} of $P$ this common shadow. 
Let us identify $X(\mathbb{D}_1)$ with the set of pairs $g=(g_+, g_-)$ where $g_+ \in B^+$ and $g_-\in B^-$  are upper and lower triangular matrices of $\SL_2(\mathbb{C})$. More precisely, for $x\in X(\mathbb{D}_1)$, we set 
\begin{align*}
{} & g_+:= \begin{pmatrix} \chi_x(\alpha_{++}^N) &  \chi_x(\alpha_{+-}^N) \\ 0 & \chi_x(\alpha_{--}^N) \end{pmatrix} = \chi_x\begin{pmatrix} K^{-N/2} & -(q-q^{-1})^N E^N K^{-N/2} \\ 0 & K^{N/2} \end{pmatrix}, \quad 
  \mbox{ and } \\
{}&  g_-:= \begin{pmatrix} \chi_x(\beta_{++}^N) &  0\\ \chi_x(\beta_{-+}^N)  & \chi_x(\beta_{--}^N) \end{pmatrix}= \chi_x \begin{pmatrix} L^{-N/2} & 0 \\ (q-q^{-1})^N L^{-N/2}F^N & L^{N/2} \end{pmatrix}
 . 
 \end{align*}
  Write $\varphi(g):=g_{-}^{-1}g_+$. $\widehat{X}(\mathbb{D}_1)$ is identified with the set of elements $(g, h_p, h_{\partial})$ with $g=(g_+, g_-) \in X(\mathbb{D}_1)$, $T_N(\gamma_p)=-\tr (\varphi(g))$ and $h_{\partial}^{-N}$ is the left upper matrix coefficient of $g_-g_+$. An indecomposable representation of weight   $(g, h_p, h_{\partial})$ is called cyclic/ semi-cyclic/ diagonal/ central if $\varphi(g)\in \SL_2(\mathbb{C})$ is not triangular/ triangular not diagonal/ diagonal not central/ central (i.e. $\varphi(g)=\pm \mathds{1}_2$) respectively. 
 \begin{notations} 
  Write $\Delta_+(\widehat{\mathfrak{sl}}_2):= \{ (n,m) \in \mathbb{N}^2 | (n-m)^2 \leq 1\}$ and decompose it as 
  \\ $\Delta_+(\widehat{\mathfrak{sl}}_2)= \{(0,1), (1,0)\} \sqcup \Delta^{\Re}_{++}(\widehat{\mathfrak{sl}}_2)\sqcup \Delta^{Im}_+(\widehat{\mathfrak{sl}}_2)$ where 
 $$  \Delta^{\Re}_{++}(\widehat{\mathfrak{sl}}_2)= \{ (k, k+1), k\geq1\} \cup \{(k+1, k), k\geq 1\}, \quad \Delta^{Im}_+(\widehat{\mathfrak{sl}}_2)=\{ (k,k), k\geq 1\}.$$
 Further write 
 $$ \Delta:= \{S, P\} \sqcup   \Delta^{\Re}_{++}(\widehat{\mathfrak{sl}}_2) \sqcup  \Delta^{Im}_+(\widehat{\mathfrak{sl}}_2)\times \mathbb{CP}^1.$$ 
 Given $\mu \in \mathbb{C}^*$, $\varepsilon \in \{-1, +1\}$, $n\in \{0, \ldots, N-2\}$, define a map $\underline{\sigma}_{\mu, \varepsilon, n}: \Delta \to \Indecomp(U_q\mathfrak{gl}_2)$ by
 $$ \underline{\sigma}_{\mu, \varepsilon, n}: S \mapsto S_{\mu, \varepsilon, n}, P\mapsto P_{\mu, \varepsilon, n}, (k+1, k)\mapsto \Omega^k_{\mu, \varepsilon, n}, (k, k+1) \mapsto \Omega^{-k}_{\mu, \varepsilon, n}, ((k,k), \Lambda) \mapsto M^k_{\mu, \varepsilon, n}(\Lambda).$$
 Consider the set $\Delta \sqcup \overline{\Delta}$ made of two copies of $\Delta$; said differently $\Delta \sqcup \overline{\Delta}:= \Delta \times \{0,1\}$ and for $\alpha \in \Delta$ we write $\alpha:= (\alpha, 0) \in \Delta \bigsqcup \overline{\Delta}$ and $\overline{\alpha}:= (\alpha, 1) \in \Delta \sqcup \overline{\Delta}$. For $\mu \in \mathbb{C}^*$, $\varepsilon \in \{-1, +1\}$, $n\in \{0, \ldots, (N-3)/2\}$, recall that $\overline{n}=N-n-2$ and  consider the map 
 $$\sigma_{\mu, \varepsilon, n}:= \underline{\sigma}_{\mu, \varepsilon, n}\sqcup \underline{\sigma}_{\mu, \varepsilon, \overline{n}} : \Delta \sqcup \overline{\Delta} \to  \Indecomp(U_q\mathfrak{gl}_2).$$
 \end{notations}
 
   
  As a consequence of Theorem \ref{theorem_representations_QG}, we obtain the 

\begin{corollary}\label{coro_QG}
\begin{enumerate}
\item The Azumaya locus of $U_q\mathfrak{gl}_2 \cong \overline{\mathcal{S}}_A(\mathbb{D}_1)$ is the set of elements $(g, h_p, h_{\partial})$ such that either $\varphi(g)$ is not central or $\varphi(g)=\pm \mathds{1}_2$ and $h_p=\mp 2$. The regular locus $\widehat{X}^{reg}(\mathbb{D}_1)$ is the set of elements $(g, h_p, h_{\partial})$ such that either $\tr(\varphi(g))\neq \pm 2$ or such that  $h_p=\pm 2$ (so $\mathcal{X}^{reg}(\mathbb{D}_1)\subset \mathcal{AL}(\mathbb{D}_1)$).
\item If $\widehat{x} \in \widehat{X}^{reg}(\mathbb{D}_1)$ there exists a unique indecomposable semi weight module (up to isomorphism) with shadow $\widehat{x}$; it is a weight module and it is simple.
\item If $\widehat{x}\in \mathcal{AL}(\mathbb{D}_1)\setminus \widehat{X}^{reg}(\mathbb{D}_1)$, there exist (up to isomorphism) two semi-weight modules with shadow $\widehat{x}$; one is a weight module and the other is not. 
\item Let $\widehat{x}=(g,h_p, h_{\partial})\in \widehat{X}(\mathbb{D}_1)$ which is not in the Azumaya locus, so one has $g=\left( \varepsilon \begin{pmatrix} \Lambda & 0 \\ 0 & \Lambda^{-1} \end{pmatrix},  \begin{pmatrix} \Lambda & 0 \\ 0 & \Lambda^{-1} \end{pmatrix} \right)$ for some $\varepsilon = \pm 1$ and $\Lambda \in \mathbb{C}^*$ and $h_p=-\varepsilon (q^{n+1} + q^{-(1+n)})$ for some $n\in \{0, \ldots, (N-3)/2\}$. Let $\mu \in \mathbb{C}^*$ be the unique scalar such that $\mu^{-N} = \Lambda$ and $h_{\partial}= \varepsilon \mu^{-2}$. Then $\sigma_{\mu, \varepsilon, n}$ is a bijection between $\Delta \sqcup \overline{\Delta} $ and the set of isomorphism classes of indecomposable $U_q\mathfrak{gl}_2$ modules with classical shadow $\widehat{x}$.
\end{enumerate}
\end{corollary}

In particular, Theorem \ref{main_theorem_intro} is verified for $\mathbb{D}_1$.

 
 \section{Reduced stated skein algebras of 2P-marked surfaces and quantum cluster algebras} \label{sec_cluster_algebras}

 A \textit{2P-marked surface} is a connected marked surface $\mathbf{\Sigma}$ which has  at least two boundary arcs and no inner puncture (i.e. every boundary component contains at least one boundary arc). As we shall see, the reduced stated skein algebra $\overline{\mathcal{S}}_A(\mathbf{\Sigma})$ of a 2P-marked surface is isomorphic to a quantum cluster algebra $\mathcal{A}_q(\mathbf{\Sigma})$ whose Azumaya loci have been studied in \cite{MNTY_AzumayaClusterAlgebras}. It follows that these algebras are Azumaya. To relate these two algebras, we first introduce the \textit{localized M\"uller algebra} $\mathcal{M}^0_A(\mathbf{\Sigma})$ following \cite{Muller} and define two isomorphisms $\mathcal{M}^0_A(\mathbf{\Sigma}) \cong \overline{\mathcal{S}}_A(\mathbf{\Sigma})$ and $\mathcal{A}_q(\mathbf{\Sigma}) \cong \mathcal{M}^0_A(\mathbf{\Sigma})$.

\subsection{M\"uller algebras and their localizations}\label{sec_Muller}

During this subsection and the next one, we denote by $k$ an arbitrary commutative unital ring and $A^{1/2}\in k^{\times}$ an invertible element (not necessarily a root of unity).

\begin{definition}[M\"uller skein algebras](\cite{Muller})\label{def_Muller} Let $\mathbf{\Sigma}$ be an essential marked surface. 
\begin{enumerate}
\item The \textit{M\"uller algebra} $\mathcal{M}_A(\mathbf{\Sigma})$ is the quotient of the $k$-free module generated by isotopy classes of (non stated) tangles $T\subset \mathbf{\Sigma}\times (0,1)$ by the skein relations 
\begin{equation*} 
\begin{tikzpicture}[baseline=-0.4ex,scale=0.5,>=stealth]	
\draw [fill=gray!45,gray!45] (-.6,-.6)  rectangle (.6,.6)   ;
\draw[line width=1.2,-] (-0.4,-0.52) -- (.4,.53);
\draw[line width=1.2,-] (0.4,-0.52) -- (0.1,-0.12);
\draw[line width=1.2,-] (-0.1,0.12) -- (-.4,.53);
\end{tikzpicture}
=A
\begin{tikzpicture}[baseline=-0.4ex,scale=0.5,>=stealth] 
\draw [fill=gray!45,gray!45] (-.6,-.6)  rectangle (.6,.6)   ;
\draw[line width=1.2] (-0.4,-0.52) ..controls +(.3,.5).. (-.4,.53);
\draw[line width=1.2] (0.4,-0.52) ..controls +(-.3,.5).. (.4,.53);
\end{tikzpicture}
+A^{-1}
\begin{tikzpicture}[baseline=-0.4ex,scale=0.5,rotate=90]	
\draw [fill=gray!45,gray!45] (-.6,-.6)  rectangle (.6,.6)   ;
\draw[line width=1.2] (-0.4,-0.52) ..controls +(.3,.5).. (-.4,.53);
\draw[line width=1.2] (0.4,-0.52) ..controls +(-.3,.5).. (.4,.53);
\end{tikzpicture}
\quad 
\begin{tikzpicture}[baseline=-0.4ex,scale=0.5,rotate=90] 
\draw [fill=gray!45,gray!45] (-.6,-.6)  rectangle (.6,.6)   ;
\draw[line width=1.2,black] (0,0)  circle (.4)   ;
\end{tikzpicture}
= -(A^2+A^{-2}) 
\begin{tikzpicture}[baseline=-0.4ex,scale=0.5,rotate=90] 
\draw [fill=gray!45,gray!45] (-.6,-.6)  rectangle (.6,.6)   ;
\end{tikzpicture}
;
\end{equation*}

\begin{equation*}
\begin{tikzpicture}[baseline=-0.4ex,scale=0.5,>=stealth]
\draw [fill=gray!45,gray!45] (-.7,-.75)  rectangle (.4,.75)   ;
\draw[->] (0.4,-0.75) to (.4,.75);
\draw[line width=1.2] (0.4,-0.3) to (0,-.3);
\draw[line width=1.2] (0.4,0.3) to (0,.3);
\draw[line width=1.1] (0,0) ++(90:.3) arc (90:270:.3);
\end{tikzpicture}
=0; 
\quad 
\heightexch{->}{}{} = A \heightexch{<-}{}{}
\end{equation*}
The multiplication in $\mathcal{M}_A(\mathbf{\Sigma})$ is defined by stacking the tangles as in Definition \ref{def_skein}.
\item For $p\in \mathcal{P}^{\partial}$, let $\alpha(p) \in \mathcal{M}(\mathbf{\Sigma})$ be the class of the corner arc. The \textit{localized M\"uller algebra} $\mathcal{M}^0_A(\mathbf{\Sigma}):= \mathcal{M}_A(\mathbf{\Sigma})[\alpha_p^{-1}, p\in \mathcal{P}^{\partial}]$ is the algebra obtained from $\mathcal{M}_A(\mathbf{\Sigma})$ by formally inverting the elements $\alpha(p)$.
\end{enumerate}
\end{definition}

A diagram $D$ defines an element $[D] \in \mathcal{M}^0_A(\mathbf{\Sigma})$ and $\mathcal{M}^0_A(\mathbf{\Sigma})$ is spanned by elements of the form $[D]\alpha(p_1)^{-k_1}\ldots \alpha(p_n)^{-k_n}$ where $\mathcal{P}^{\partial}=\{p_1, \ldots, p_n\}$ and $k_i \geq 0$. An element $[D]\alpha(p_1)^{-k_1}\ldots \alpha(p_n)^{-k_n}$ is called \textit{reduced} if for each $p_i \in \mathcal{P}^{\partial}$, either $k_i=0$ or $D$ does not contain any component isotopic to $\alpha(p_i)$. 

\begin{definition}\label{def_Muller_basis2}  Let $\underline{\mathcal{B}}^M \subset \mathcal{M}^0_A(\mathbf{\Sigma})$ be the subset of reduced elements $[D]\alpha(p_1)^{-k_1}\ldots \alpha(p_n)^{-k_n}$ such that $D$ is simple.\end{definition}

\begin{proposition}\label{prop_Muller_basis2}(\cite[Proposition $5.3$]{Muller}) $\underline{\mathcal{B}}^M$ is a basis of $\mathcal{M}^0_A(\mathbf{\Sigma})$.
\end{proposition}

As remarked in \cite{LeStatedSkein}, by comparing the skein relations in Definitions \ref{def_skein} and \ref{def_Muller}, we see that  there is an algebra morphism $f: \mathcal{M}_A(\mathbf{\Sigma})\to \overline{\mathcal{S}}_A(\mathbf{\Sigma})$ sending the class $[T]$ of a tangle to $f([T]):= [T, s^+]$ where $s^+$ sends every points of $\partial T$ to $+$. Since $\alpha(p)_{++}$ has inverse $\alpha(p)_{--}$, $f$ extends to a morphism (denoted by the same letter) $f: \mathcal{M}^0_A(\mathbf{\Sigma})\to \overline{\mathcal{S}}_A(\mathbf{\Sigma})$ sending $\alpha(p)^{-1}$ to $\alpha(p)_{--}$. 

\begin{theorem}\label{theorem_SSkein_Muller}(\cite[Theorem $5.2$]{LeYu_SSkeinQTraces}) $f: \mathcal{M}^0_A(\mathbf{\Sigma})\to \overline{\mathcal{S}}_A(\mathbf{\Sigma})$ is an isomorphism.
\end{theorem}

\begin{remark} 
By comparing Definitions \ref{def_Muller_basis2} and \ref{def_Muller_basis1} and using Lemma \ref{lemma_heightexch}, we see that for each $b \in \underline{\mathcal{B}}^M$ there exists $n_b \in \mathbb{Z}$ such that $f$ sends the basis $\{ A^{n_b/2} b , b \in \underline{\mathcal{B}}^M\}$ to the basis $\mathcal{B}^M$
so the theorem follows from Propositions \ref{prop_Muller_basis} and \ref{prop_Muller_basis2}.
\end{remark}

\subsection{Quantum cluster algebras}\label{sec_QCA}

Quantum cluster algebras were introduced in \cite{BerensteinZelevinsky} after the introduction of classical cluster algebras in \cite{FominZelevinsky} and their Poisson structure in \cite{GekhtmanShapiroVainshtein03}. We will closely follow the notations in \cite{Muller}.

\begin{definition}[Quantum cluster algebras and their upper algebras]
Let $\mathcal{F}$ be a field with $k\subset \mathcal{F}$ and such that $\mathcal{F}$ is finitely generated over $k$.
\begin{enumerate}
\item A \textit{quantum seed} in $\mathcal{F}$ is a triple $\mathbb{S}=(B, \Lambda, M)$ where:
\begin{itemize}
\item the \textit{exchange matrix} $B$ is an $N\times ex$ $\mathbb{Z}$-valued matrix where $ex\subset \{1, \ldots, N\}$ such that, if $\pi: \mathbb{Z}^N \to \mathbb{Z}^{ex}$ denotes the projection, then $\pi B$ is skew-symmetric; 
\item the \textit{compatibility matrix} $\Lambda$ is an $N\times N$ skew symmetric matrix valued in $\mathbb{Z}$ such that $\Lambda B= D \iota$ where $D$ is diagonal matrix with positive diagonal elements and $\iota$ is the matrix of the inclusion $\mathbb{Z}^{ex} \hookrightarrow \mathbb{Z}^N$; 
\item $M: \mathbb{Z}^N \to \mathcal{F} \setminus\{ 0 \}$ is a map such that $M(\alpha)M(\beta)= A^{\frac{1}{2} \Lambda(\alpha, \beta)} M(\alpha + \beta)$ and such that $\mathcal{F}$ is the field of fractions of the quantum torus $k M(\mathbb{Z}^N)$. 
\end{itemize}
\item For $\mathbb{S}=(B, \Lambda, M)$ a quantum seed and $i \in ex$, the \textit{mutation of }$\mathbb{S}$ \textit{ in the direction of }$i$ is the quantum seed $\mathbb{S}'=(B', \Lambda', M')$ characterized by the following properties:
\begin{itemize}
\item $B'_{jk} = \left\{
 \begin{array}{ll}
-B_{jk} & \mbox{, if }i=j \mbox{ or }i=k; \\
 B_{jk} + \frac{1}{2}( |B_{ji}| B_{ik} + B_{ji} |B_{ik}|) & \mbox{, else.}
\end{array} \right.$;
\item for $\alpha \in \mathbb{Z}^N$ such that $\alpha_i=0$ then $M(\alpha)=M'(\alpha)$; 
\item one has 
$$ M'(e_i) = M \left( -e_i + \sum_{B_{ji}>0} B_{ji}e_j \right) + M\left( -e_i - \sum_{B_{ji}<0} B_{ji}e_j \right).$$
\end{itemize}
By \cite[Section $4.4$]{BerensteinZelevinsky},  $\mathbb{S}'$ exists and is unique. We denote by $\Mut(\mathbb{S})$ the set of quantum seeds which can be obtained from $\mathbb{S}$ by a finite sequence of mutations. 
\item The \textit{quantum cluster algebra} $\mathcal{A}_q(\mathbb{S})$ is the $k$ subalgebra of $\mathcal{F}$ generated by elements $M'(\alpha)$ for  $\mathbb{S}'=(B',\Lambda',M')\in \Mut(\mathbb{S})$ and $\alpha \in \mathbb{Z}^N$ such that $\alpha_i\geq 0$ for $i\in ex$ . 
\item The \textit{quantum upper cluster algebra} $\mathcal{U}_q(\mathbb{S})$ is the intersection: 
$$ \mathcal{U}_q(\mathbb{S}) := \bigcap_{\mathbb{S}'=(B', \Lambda', M')\in \Mut(\mathbb{S})} k M'(\mathbb{Z}^N).$$
\end{enumerate}
\end{definition}
By \cite[Corollary $5.2$]{BerensteinZelevinsky}, one has $\mathcal{A}_q(\mathbb{S}) \subset \mathcal{U}_q(\mathbb{S})$ (this is the so-called Laurent phenomenon). In many cases, this inclusion is an equality. One can associate to any triangulated marked surface $(\mathbf{\Sigma}, \Delta)$ a quantum seed. This was first remarked at the classical level in \cite{GekhtmanShapiroVainshtein05} and extended to the quantum case in \cite{Muller}.



\begin{definition}[Seeds associated to triangulated marked surfaces]
\begin{enumerate}
\item A connected marked surface $\mathbf{\Sigma}$ is \textit{triangulable} if it is essential, has no inner puncture  and is not a disc with one or two boundary arcs. For such a $\mathbf{\Sigma}$, an $\mathcal{A}$-\textit{triangulation} $\Delta$ is a maximal set of pairwise disjoint and pairwise non isotopic arcs with an indexation $\Delta= \{x_1, \ldots, x_{|\Delta|}\}$ (recall from Definition \ref{def_tangles} than an arc is an open connected diagram without crossing). Clearly every corner arc $\alpha(p)$ belongs to $\Delta$ and we denote by $\mathring{\Delta}\subset \Delta$ the set of arcs which are not corner arcs and by $ex \subset \{1, \ldots, |\Delta|\}$ the subset of indices $i$ such that $x_i \in \mathring{\Delta}$. For $x_i \in \Delta$, we denote by $\partial_1 x_i$ and $\partial_2 x_i$ its two endpoints (indexed arbitrarily). For two endpoints $\partial_i x$ and $\partial_j y$ which belong to the same boundary arc $a$ (where $x,y \in \Delta$, $i,j \in \{1,2\}$) such that $\partial_ix <_{\mathfrak{o}^+} \partial_j y$, we say that $\partial_i x$ and $\partial_j y$ are \textit{consecutive} if there does not exist $z\in \Delta$ and $k\in \{1,2\}$ such that  $\partial_ix <_{\mathfrak{o}^+} \partial_k z <_{\mathfrak{o}^+} \partial_j y$.
\item Let $(\mathbf{\Sigma}, \Delta)$ be a triangulated marked surface and  $\mathcal{F}_{\mathbf{\Sigma}}$ the field of fraction of $\mathcal{M}_A(\mathbf{\Sigma})$ (obtained by inverting every non zero elements). Let $\mathbb{S}_{(\mathbf{\Sigma}, \Delta)}= (B^{\Delta}, \Lambda^{\Delta}, M^{\Delta})$ be the quantum seed over $\mathcal{F}_{\mathbf{\Sigma}}$ defined by: 
\begin{itemize}
\item For $i \in \{1, \ldots, |\Delta|\}$ and $j\in ex$,  
$$ B^{\Delta}_{i,j}:= \sum_{u,v \in \{1,2\}} 
\left\{ \begin{array}{ll}
-1 & \mbox{, if }\partial_u x_i \mbox{ and }\partial_v x_j \mbox{ are in the same boundary arc }  \mbox{ and } \partial_u x_i <_{\mathfrak{o}^+} \partial_v x_j \mbox{ are consecutive}; \\
+1 & \mbox{, if }\partial_u x_i \mbox{ and }\partial_v x_j \mbox{ are in the same boundary arc }  \mbox{ and } \partial_v x_j <_{\mathfrak{o}^+} \partial_u x_i \mbox{ are consecutive};  \\
0 & \mbox{, if } \partial_u x_i \mbox{ and }\partial_v x_j \mbox{ else.} 
\end{array} \right. $$
\item  For $i, j  \in \{1, \ldots, |\Delta|\}$,  
$$ \Lambda^{\Delta}_{i,j}:= \sum_{u,v \in \{1,2\}} 
\left\{ \begin{array}{ll}
0 & \mbox{, if } \partial_u x_i \mbox{ and }\partial_v x_j \mbox{ are in two distinct boundary arcs;} \\
+1 & \mbox{, if }\partial_u x_i \mbox{ and }\partial_v x_j \mbox{ are in the same boundary arc }  \mbox{ and } \partial_u x_i <_{\mathfrak{o}^+} \partial_v x_j; \\
-1 & \mbox{, if }\partial_u x_i \mbox{ and }\partial_v x_j \mbox{ are in the same boundary arc }  \mbox{ and } \partial_v x_j <_{\mathfrak{o}^+} \partial_u x_i. 
\end{array} \right. $$
\item $M^{\Delta}(e_i):= x_i$ for $i\in \{1, \ldots, |\Delta|\}$.
\end{itemize}
By \cite[Proposition $7.8$]{Muller}, $\mathbb{S}_{(\mathbf{\Sigma}, \Delta)}$ is indeed a quantum seed.
\item By \cite[Theorem $7.9$]{Muller}, one has $\Mut(\mathbb{S}_{(\mathbf{\Sigma}, \Delta)}) = \{ \mathbb{S}_{(\mathbf{\Sigma}, \Delta')}, \Delta' \mbox{ an }\mathcal{A}-\mbox{ triangulation of }\mathbf{\Sigma} \}$. Therefore $\mathcal{A}_q(\mathbb{S}_{(\mathbf{\Sigma}, \Delta)})$ and $\mathcal{U}_q(\mathbb{S}_{(\mathbf{\Sigma}, \Delta)})$ only depend on $\mathbf{\Sigma}$ and not on $\Delta$; we denote them by $\mathcal{A}_q(\mathbf{\Sigma})$ and $\mathcal{U}_q(\mathbf{\Sigma})$ respectively.
\end{enumerate}
\end{definition}


\begin{theorem}\label{theorem_QCA_Muller} Let $(\mathbf{\Sigma}, \Delta)$ be a triangulated marked surface. 
\begin{enumerate}
\item \cite[Theorem $7.16$]{Muller} We have the inclusions $\mathcal{A}_q(\mathbf{\Sigma}) \subset \mathcal{M}^0_A (\mathbf{\Sigma}) \subset \mathcal{U}_q(\mathbf{\Sigma})$.
\item \cite[Theorem $9.7$]{Muller} If moreover $\mathbf{\Sigma}$ is a 2P-marked surface, then $\mathcal{A}_q(\mathbf{\Sigma}) = \mathcal{M}^0_A (\mathbf{\Sigma}) =\mathcal{U}_q(\mathbf{\Sigma})$.
\end{enumerate}
\end{theorem}
 
 Therefore, for a 2P-marked surface $\mathbf{\Sigma}$ which is not the bigon,  using Theorems \ref{theorem_SSkein_Muller} and \ref{theorem_QCA_Muller}, we have an isomorphism $f: \overline{\mathcal{S}}_A(\mathbf{\Sigma}) \xrightarrow{\cong} \mathcal{A}_q(\mathbf{\Sigma})$. 
 
 \subsection{Representations of reduced stated skein algebras of 2P-surfaces}
 
 For now on, we suppose that $k=\mathbb{C}$ and that $A^{1/2}\in \mathbb{C}^*$ is a root of unity of odd order $N\geq 3$. So $\mathcal{A}_q(\mathbb{S})$ is now a complex algebra.
 
 \begin{definition}[Frobenius and fully Azumaya loci of quantum cluster algebras] Suppose that the field $\mathcal{F}$ is a finite extension of $\mathbb{C}$ and let $\mathbb{S}= (B, \Lambda, M)$ a quantum seed for $\mathcal{F}$. Let $Z_{\mathbb{S}}$ denote the center of $\mathcal{A}_q(\mathbb{S})$. 
 \begin{enumerate}
 \item The \textit{Frobenius morphism} $Fr_{\mathbb{S}}: \mathcal{A}_{+1}(\mathbb{S}) \hookrightarrow Z_{\mathbb{S}}$ is the embedding defined by $Fr_{\mathbb{S}}(M(\alpha))= M(N \alpha)$. 
 \item We write $\widehat{X}(\mathbb{S}):= \Specm (Z_{\mathbb{S}})$ and $X(\mathbb{S}):= \Specm(\mathcal{A}_{+1}(\mathbb{S}))$ and denote by $p: \widehat{X}(\mathbb{S}) \to X(\mathbb{S})$ the dominant map defined by $Fr_{\mathbb{S}}$. 
 \item The \textit{Azumaya locus} $\mathcal{AL}(\mathbb{S}) \subset \widehat{X}(\mathbb{S})$ and \textit{fully Azumaya locus} $\mathcal{FAL}(\mathbb{S}) \subset X(\mathbb{S})$ of  $\mathcal{A}_q(\mathbb{S})$ are defined as in Definition \ref{def_AL}.
 \end{enumerate}
 \end{definition}
 
 \begin{theorem}\label{theorem_MNKY}(\cite[Theorem $6.1$]{MNTY_AzumayaClusterAlgebras}) Suppose that the field $\mathcal{F}$ is a finite extension of $\mathbb{C}$ and let $\mathbb{S}= (B, \Lambda, M)$ a quantum seed for $\mathcal{F}$. 
 Suppose that 
 \begin{enumerate}
 \item  $N$ is prime to the diagonal entries of the diagonal matrix $D$ defined by $\Lambda B = D \iota$ and 
 \item one has $\mathcal{A}_q(\mathbb{S})=\mathcal{U}_q(\mathbb{S})$.
 \end{enumerate}
 Then the fully Azumaya locus of $\mathcal{A}_q(\mathbb{S})$ contains the regular locus of $X(\mathbb{S})$. In particular, if $X(\mathbb{S})$ is smooth, then  $\mathcal{A}_q(\mathbb{S})$ is Azumaya. 
 \end{theorem}
 
 Theorem \label{theorem_MNKY} permits to prove Theorem \ref{main_theorem_intro} for 2P-marked surfaces. We will also use the following: 
 
 \begin{theorem}( \cite[Theorem $1$]{KojuMCGRepQT})\label{theorem_MCG} If $\mathbf{\Sigma}= (\Sigma_{g,1}, \{a\})$ is a connected marked surface with a single boundary component which has a single boundary arc, then $\overline{\mathcal{S}}_A(\mathbf{\Sigma})$ is Azumaya.
 \end{theorem}
 
  
 \begin{corollary}\label{coro_2Psurfaces}
 For $\mathbf{\Sigma}$ a connected essential marked surface without inner puncture, then $\overline{\mathcal{S}}_A(\mathbf{\Sigma})$ is Azumaya and every semi-weight representation is a weight representation.
 \end{corollary}
 
 \begin{proof} We first prove that  $\overline{\mathcal{S}}_A(\mathbf{\Sigma})$ is Azumaya.
  If $\mathbf{\Sigma}=\mathbb{B}$ is the bigon (a disc with two boundary arcs), then $\overline{\mathcal{S}}_A(\mathbb{B})\cong \mathbb{C}[X^{\pm 1}]$ is commutative, so the result is obvious.
 If $\mathbf{\Sigma}= (\Sigma_{g,1}, \{a\})$ has a single boundary component which has a single boundary arc, it is Theorem \ref{theorem_MCG}.
  Else, $\mathbf{\Sigma}$ is a triangulable 2P marked surface; let $\Delta$ be a triangulation. The isomorphism  $f: \overline{\mathcal{S}}_{+1}(\mathbf{\Sigma}) \xrightarrow{\cong} \mathcal{A}_{+1}(\mathbf{\Sigma})$ induces an isomorphism $X(\mathbf{\Sigma}) \cong X(\mathbb{S}_{(\mathbf{\Sigma}, \Delta)})$ so Theorem \ref{theorem_smooth} implies that $X(\mathbb{S}_{(\mathbf{\Sigma}, \Delta)})$ is smooth and it suffices to prove that the two hypotheses of Theorem \ref{theorem_MNKY} are satisfied for $\mathbb{S}_{(\mathbf{\Sigma}, \Delta)}$. In the proof of \cite[Proposition $7.8$]{Muller}, it is proved that $\Lambda^{\Delta} B^{\Delta}=4 \iota$ and since $N$ is odd, the first hypothesis is satisfied. The second follows from Theorem \ref{theorem_QCA_Muller}. Therefore Theorem \ref{theorem_MNKY} implies that $\mathcal{A}_q(\mathbf{\Sigma})\cong \overline{\mathcal{S}}_A(\mathbf{\Sigma})$ is Azumaya. 
  \par We now prove that any semi weight representation is a weight representation.
 Let $\partial_1, \ldots, \partial_n$ be the boundary components of $\Sigma$. 
 For $x\in X(\mathbf{\Sigma})$, since $\mathbf{\Sigma}$ does not contain inner punctures, then $${Z}_{\mathbf{\Sigma}}\cong \quotient{{Z}^0_{\mathbf{\Sigma}}[\alpha_{\partial_1}^{\pm 1}, \ldots, \alpha_{\partial_n}^{\pm 1}]}{ \left( \alpha_{\partial_1}^N - \chi_x(\alpha_{\partial_1}), \ldots, \alpha_{\partial_n}^N-\chi_x(\alpha_{\partial_n}) \right)}.$$ 
 Since the polynomial $X^N-c$ for $c\neq 0$ has its zeros of multiplicity one, one has
 $$Z(x)\cong \mathbb{C}^{\oplus n}.$$
 We conclude using Corollary \ref{coro_FAL}. 
 \end{proof}
 
 
 \section{Gluing marked surfaces together} \label{sec_gluing}
 
  If $V_1$ and $V_2$ are modules over $ \overline{\mathcal{S}}_A(\mathbf{\Sigma}_1)$ and $ \overline{\mathcal{S}}_A(\mathbf{\Sigma}_2)$ respectively and $a_1, a_2$ some boundary arcs of $\mathbf{\Sigma}_1$ and $\mathbf{\Sigma}_2$ respectively, then $V_1\otimes V_2$ is a  $\overline{\mathcal{S}}_A(\mathbf{\Sigma}_1 \cup_{a_1 \# a_2} \mathbf{\Sigma}_2)$-module by precomposing with $\theta_{a_1 \# a_2}$. 
 The goal of this section is to prove the 
 
 \begin{theorem}\label{theorem_gluing} Let  $\mathbf{\Sigma}_1$ and $\mathbf{\Sigma}_2$ be marked surfaces and for $i=1,2$ let $a_i$ be a boundary arcs of $\mathbf{\Sigma}_i$ such that the connected component of $\partial \Sigma_i$ which contains $a_i$ also contains at least another boundary arc. 
 Write $\mathbf{\Sigma}:=\mathbf{\Sigma}_1 \cup_{a_1\# a_2} \mathbf{\Sigma}_2$.
 Then 
$$ \widehat{p}_{a_1 \# a_2} \left(\mathcal{AL}(\mathbf{\Sigma}_1) \times \mathcal{AL}(\mathbf{\Sigma}_2)\right) = \mathcal{AL} (\mathbf{\Sigma}).$$
Moreover, any indecomposable (semi) weight $\overline{\mathcal{S}}_A(\mathbf{\Sigma})$-module is isomorphic to a module $V_1\otimes V_2$ with $V_i$ a (semi) weight indecomposable $ \overline{\mathcal{S}}_A(\mathbf{\Sigma})$ module. Conversely, any such $ \overline{\mathcal{S}}_A(\mathbf{\Sigma})$ module $V_1\otimes V_2$ is indecomposable of the same kind (weight or semi weight).
\end{theorem}
 
Let $Z^1_{\mathbf{\Sigma}}$ be the subalgebra of $Z_{\mathbf{\Sigma}}$ generated by $Z^0_{\mathbf{\Sigma}}$ and the elements $\alpha_{\partial}^{\pm 1}$ for $\partial \in \Gamma^{\partial}$; thus $Z^0_{\mathbf{\Sigma}}\subset Z^1_{\mathbf{\Sigma}} \subset Z_{\mathbf{\Sigma}}$.
 
 \begin{lemma}\label{lemma_gluing} Under the hypotheses of Theorem \ref{theorem_gluing}, any element of $ \overline{\mathcal{S}}_A(\mathbf{\Sigma}_1)\otimes \overline{\mathcal{S}}_A(\mathbf{\Sigma}_2)$ is the product of an element of the image of  $\theta_{a_1 \# a_2}$ with an element of $Z^1_{\mathbf{\Sigma}_1}\otimes Z^1_{\mathbf{\Sigma}_2}$.
 \end{lemma}
 
 \begin{proof}
 Let us introduce some notations illustrated in Figure \ref{fig_gluing_surjective} $(i)$.
Let $\partial_1$ and $\partial_2$ be the boundary components of $\Sigma_1$ and $\Sigma_2$ containing $a_1$ and $a_2$. The orientations of $\Sigma_1, \Sigma_2$ induces orientations of $\partial_1$ and $\partial_2$ thus a cyclic order on their sets of boundary arcs and an orientation of each boundary arc. Let $p_1, p_1'$ be the two punctures adjacent to $a_1$ such that $a_1$ is oriented from $p_1'$ to $p_1$. Let $p_2, p_2'$ be the two punctures adjacent to $a_2$ such that $a_2$ is oriented from $p_2$ to $p_2'$.  Write $\mathbf{\Sigma}= \mathbf{\Sigma}_1\cup_{a_1 \# a_2} \mathbf{\Sigma}_2$. Let $d\subset \Sigma$ be the common image of $a_1$ and $a_2$ in the quotient map $\pi:\Sigma_1 \sqcup \Sigma_2 \to \Sigma$ and write $p:= \pi(p_1)=\pi(p_2)$ and $p':= \pi(p_1')=\pi(p_2')$. Let $b_1, c_1$ be the boundary arcs in $\partial_1$ such that $p_1$ is adjacent to $a_1$ and $c_1$ and $p_2$ is adjacent to $a_1$ and $b_1$ (if $\partial_1$ only has two boundary arcs, then $c_1=b_1$). Let $b_2,c_2$ be the boundary arcs in $\partial_2$ such that $p_2$ is adjacent to $a_2$ and $c_2$ and $p_2'$ is adjacent to $b_2$ and $a_2$. Let $R \subset \overline{\mathcal{S}}_A(\mathbf{\Sigma}_1)\otimes \overline{\mathcal{S}}_A(\mathbf{\Sigma}_2)$ be the subalgebra of elements which are a product of an element of $\theta_{a_1 \# a_2}\left( \overline{\mathcal{S}}_A(\mathbf{\Sigma}) \right)$ with an element of 
$Z^1_{\mathbf{\Sigma}_1}\otimes Z^1_{\mathbf{\Sigma}_2}$.
 We need to show that for every $[D_1, s_1] \in \mathcal{B}^M(\mathbf{\Sigma}_1)$ and for every $[D_2,s_2]\in \mathcal{B}^M(\mathbf{\Sigma}_2)$ then $[D_1,s_1]\otimes 1 \in R$ and $1\otimes [D_2,s_2] \in R$. 
 Let us first prove the following:
 \par \textbf{Claim:} For $\varepsilon = \pm$,  then the four elements  $\alpha(p_1)_{\varepsilon \varepsilon} \otimes 1$, $\alpha(p'_1)_{\varepsilon \varepsilon} \otimes 1$, $1\otimes \alpha(p_2)_{\varepsilon \varepsilon} $ and $1\otimes \alpha(p'_2)_{\varepsilon \varepsilon} $ are in $R$.  
 \\ We prove the claim for $\alpha(p_1)_{++}\otimes 1$ and leave the other similar cases to the reader. 
 \par $\bullet$ $\underline{ \alpha(p_1)_{--} \alpha(p_1')_{++} \otimes 1  \in R:}$ Choose two points $v\in b_1$, $w\in c_1$ and consider an arc $\beta$ oriented from $v$ to $w$ such that $\beta$ can be isotoped relatively to its boundary to an arc $\beta'\subset \partial_1$ containing $a_1$. As illustrated in Figure \ref{fig_gluing_surjective} $(ii)$, one has $\theta_{a_1 \# a_2} (\beta_{+-})= A^{1/2}\alpha(p_1)_{--}\alpha(p_1')_{++}\otimes 1 \in R$.
 \par $\bullet$ $\underline{ \alpha(p_1)_{--} \alpha(p_1')_{--} \otimes 1  \in R:}$ Let $(p_1', p_1, p_3, p_4, \ldots, p_n)$ be the punctures of $\partial_1$ cyclically ordered. Clearly, for $i=3, \ldots, n$,  $\alpha(p_i)_{--}\otimes 1$ is in the image of $\theta_{a_1 \# a_2}$. Moreover $\alpha_{\partial_1}^{-1}= \alpha(p_1)_{--}\alpha(p_1')_{--} \alpha(p_3)_{--}\ldots \alpha(p_n)_{--} \in Z_{\mathbf{\Sigma}_1}$, therefore $ \alpha(p_1)_{--} \alpha(p_1')_{--} \otimes 1  \in R$.
 \par $\bullet$ $\underline{ \alpha(p_1)_{++}^N \otimes 1  \in R:}$ By Lemma \ref{lemma_heightexch}, there exists $n\in \mathbb{Z}$ such that $(\alpha(p_1)_{++})^N = A^{n/2} \alpha(p_1)_{++}^{(N)}$. Since $\alpha(p_1)_{++}^{(N)}$ is in the image of the Frobenius morphism by Theorem \ref{theorem_Frobenius}, one has $(\alpha(p_1)_{++})^N \otimes 1 \in R$. 
 \par $\bullet$ $\underline{ \alpha(p_1)_{++} \otimes 1  \in R:}$ First one has $(\alpha(p_1)_{--})^2= (\alpha(p_1)_{--} \alpha(p_1')_{++} )(\alpha(p_1)_{--} \alpha(p_1')_{--} ) \in R$. Write $N=2k+1$. Then $\alpha(p_1)_{++} = (\alpha(p_1)_{--})^{2k} (\alpha(p_1)_{++})^N \in R.$
 \par The other cases of the claim are proved similarly. Let $[D,s] \in \mathcal{B}^M(\mathbf{\Sigma}_2)$ and let us prove that $1\otimes [D,s] \in R$. By Lemma \ref{lemma_heightexch} there exist $n\in \mathbb{Z}$ and $D^0\subset D$ a sub diagram such that $[D,s]=A^{n/2} [D^0, s^+] \prod_{p \in \mathcal{P}^{\partial}} \alpha(p)_{--}^{n_p}$, where $s^+$ is totally positive in the sense that it sends every endpoints in $\partial D_0$ to $+$. Since every elements $1 \otimes (\alpha(p)_{--})^{n_p}$ are in $R$, it suffices to prove that $1\otimes [D^0, s^+] \in R$ for a totally positive simple stated diagram. For such a $D^0 \subset \Sigma_2$, let $\widehat{D^0} \subset \Sigma$ be the simple diagram obtained from $D^0$ by pushing slightly each point of $\partial_{a_2} D^0$ to $b_1$ while forming a corner arc as in Figure \ref{fig_gluing_surjective} $(iii)$. If $m:= |\partial_{a_2}D^0|$ then  as illustrated in Figure \ref{fig_gluing_surjective} $(iii)$, one has 
 $$ \theta_{a_1 \# a_2} ([\widehat{D^0}, s^+]) = (\alpha(p'_1)_{++})^m \otimes [D^0, s^+].$$
 Since $\alpha(p'_1)_{++}\otimes 1 \in R$ by the claim, we obtain that $1\otimes [D^0, s^+] \in R$. So every element of the form $1\otimes [D,s]$ is in $R$. We prove that the elements $[D,s]\otimes 1$ belong to $R$ similarly. This concludes the proof.
 
  
 \begin{figure}[!h] 
\centerline{\includegraphics[width=14cm]{SurjectivitySplitting.eps} }
\caption{$(i)$ The marked surfaces $\mathbf{\Sigma}_1, \mathbf{\Sigma}_2, \mathbf{\Sigma}$ and some boundary arcs; $(ii)$ An illustration of the equality  $\theta_{a_1 \# a_2} (\beta_{+-})= A^{1/2}\alpha(p_1)_{--}\alpha(p_1')_{++}\otimes 1$; $(iii)$ A diagram $(D^0, s^+)$, its associated diagram $(\widehat{D^0}, s^+)$ and an illustration of the equality $ \theta_{a_1 \# a_2} ([\widehat{D^0}, s^+]) = (\alpha(p'_1)_{++})^m \otimes [D^0, s^+]$.
} 
\label{fig_gluing_surjective} 
\end{figure} 
 
 
 \end{proof}
 
 \begin{lemma}\label{lemma_partial} Let $\rho:  \overline{\mathcal{S}}_A(\mathbf{\Sigma})\to \End(V)$  be  an indecomposable semi-weight representation and $\partial \in \Gamma^{\partial}$. Then  there exists $h_{\partial}\in \mathbb{C}^*$ such that $\rho(\alpha_{\partial})=h_{\partial}\id_V$.
 \end{lemma}
 
 \begin{proof} The operator $\rho(\alpha_{\partial})$ is annihilated by the polynomial $P(X)=X^{N} - \chi_x(\alpha_{\partial})$ so the minimal polynomial of $\rho(\alpha_{\partial})$ divides $P(X)$. Since $\rho(\alpha_{\partial})$ commutes with the image of $\rho$ and since $\rho$ is indecomposable, its minimal polynomial has the form $(X-c)^n$. 
 Since $P(X)$ has its roots of multiplicity one, the minimal polynomial of  $\rho(\alpha_{\partial})$ has the form $X-h_{\partial}$ for some $h_{\partial}$ such that $h_{\partial}^N=\chi_x(\alpha_{\partial})$.
 \end{proof}
 
  Therefore, for $\rho:  \overline{\mathcal{S}}_A(\mathbf{\Sigma})\to \End(V)$   an indecomposable semi-weight representation, one can associate a tuple $\widetilde{x}=(x; h_{\partial})_{\partial \in \Gamma^{\partial}} \in \Specm(Z^1_{\mathbf{\Sigma}})=: \widetilde{X}(\mathbf{\Sigma})$ which we call the shadow of $\rho$. Let $\mathcal{I}_{\widetilde{x}} \subset \overline{\mathcal{S}}_A(\mathbf{\Sigma})$ be the ideal generated by elements $z-\chi_{x}(x)$ for $z\in Z^0_{\mathbf{\Sigma}}$ and elements $\alpha_{\partial}^{\pm 1}-h_{\partial}^{\pm 1}$ for $\partial \in \Gamma^{\partial}$. Then $\rho$ factorizes through the algebra 
  $$ \overline{\mathcal{S}}_A(\mathbf{\Sigma})_{\widetilde{x}} := \quotient{\overline{\mathcal{S}}_A(\mathbf{\Sigma})}{\mathcal{I}_{\widetilde{x}}}.$$
 
 
 \begin{proof}[Proof of Theorem \ref{theorem_gluing}]
 \par \underline{\textit{The case of weight representations:}}
 Let $\widehat{x} \in \widehat{X}(\mathbf{\Sigma})$. By Theorem \ref{theorem_gluing_surjective}, there exist $\widehat{x}_i \in  \widehat{X}(\mathbf{\Sigma}_i)$, $i=1,2$, such that $\widehat{p}_{a_1\# a_2}(\widehat{x}_1, \widehat{x}_2)=\widehat{x}$. The splitting morphism $\theta_{a_1\# a_2}$ induces an injective morphism 
 $$ \widetilde{\theta}_{a_1 \# a_2} : \overline{\mathcal{S}}_A(\mathbf{\Sigma})_{\widehat{x}}  \to \overline{\mathcal{S}}_A(\mathbf{\Sigma}_1)_{\widehat{x}_1}\otimes \overline{\mathcal{S}}_A(\mathbf{\Sigma}_2)_{\widehat{x}_2}.$$
 By Lemma \ref{lemma_gluing}, $ \widetilde{\theta}_{a_1 \# a_2}$ is surjective. To prove that it is injective, recall that by definition, $ \widetilde{\theta}_{a_1 \# a_2}$ is obtained from the injective morphism $ {\theta}_{a_1 \# a_2}$ by passing to the quotient. So we need to prove the inclusion $ {\theta}_{a_1 \# a_2}^{-1}\left( \mathcal{I}_{\widehat{x}_1}\otimes 1 + 1\otimes \mathcal{I}_{\widehat{x}_2}\right) \subset \mathcal{I}_{\widehat{x}}$; this follows from the inclusion $ {\theta}_{a_1 \# a_2}^{-1}\left(Z_{\mathbf{\Sigma}_1}\otimes 1 + 1\otimes Z_{\mathbf{\Sigma}_2}\right) \subset Z_{\mathbf{\Sigma}}$ which immediately follows from the description of the centers in Theorem \ref{theorem_center}.
 \par So  $\widetilde{\theta}_{a_1 \# a_2}$ is an isomorphism. By definition, an indecomposable weight $ \overline{\mathcal{S}}_A(\mathbf{\Sigma})$ module with classical shadow $\widehat{x}$ is the same thing than an indecomposable representation of the finite dimensional algebra $ \overline{\mathcal{S}}_A(\mathbf{\Sigma})_{\widehat{x}}$ (and similarly for $\mathbf{\Sigma}_i$). Therefore there is a $1:1$ correspondance between isomorphism classes of pairs of indecomposable weight representation $(V_1, V_2)$ where $V_i$ has classical shadow $\widehat{x}_i$ and the set of isomorphism classes of weight $ \overline{\mathcal{S}}_A(\mathbf{\Sigma})$ representations with shadow $\widehat{x}$; the isomorphism sends $(V_1,V_2)$ to $V_1\otimes V_2$. By Theorem \ref{theorem_AL}, $\widehat{x} \in \mathcal{AL}(\mathbf{\Sigma})$ if and only if $\dim \left(\overline{\mathcal{S}}_A(\mathbf{\Sigma})_{\widehat{x}}\right) = (D_{\mathbf{\Sigma}})^2$ and if  $\widehat{x} \notin \mathcal{AL}(\mathbf{\Sigma})$ then $\dim \left(\overline{\mathcal{S}}_A(\mathbf{\Sigma})_{\widehat{x}}\right) < (D_{\mathbf{\Sigma}})^2$ (and similarly for $\mathbf{\Sigma}_i$). By Lemma \ref{lemma_additivityD}, we have $D_{\mathbf{\Sigma}}=D_{\mathbf{\Sigma}_1}D_{\mathbf{\Sigma}_2}$ so $\widehat{x} \in \mathcal{AL}(\mathbf{\Sigma})$ if and only if $x_i \in \mathcal{AL}(\mathbf{\Sigma}_i)$ for $i=1,2$.
 \vspace{2mm}
 \par  \underline{\textit{The case of semi weight representations:}} Let $\widetilde{x}=(x, h_{\partial})_{\partial \in \Gamma^{\partial}}\in \widetilde{X}(\mathbf{\Sigma})$ and denote by $\widetilde{p}_{a_1\# a_2} : \widetilde{X}(\mathbf{\Sigma}_1)\times  \widetilde{X}(\mathbf{\Sigma}_2) \to  \widetilde{X}(\mathbf{\Sigma})$ the morphism induced by $\theta_{a_1\# a_2}$.
 As showed in the proof of Theorem \ref{theorem_gluing_surjective}, there exists $(\widetilde{x}_1, \widetilde{x}_2) \in \widetilde{X}(\mathbf{\Sigma}_1)\times  \widetilde{X}(\mathbf{\Sigma}_2)$ such that  $\widetilde{p}_{a_1\# a_2} (\widetilde{x}_1, \widetilde{x}_2)= \widetilde{x}$. $\theta_{a_1\# a_2}$ induces a morphism 
 $$ \widetilde{\theta}_{a_1 \# a_2} : \overline{\mathcal{S}}_A(\mathbf{\Sigma})_{\widetilde{x}}  \to \overline{\mathcal{S}}_A(\mathbf{\Sigma}_1)_{\widetilde{x}_1}\otimes \overline{\mathcal{S}}_A(\mathbf{\Sigma}_2)_{\widetilde{x}_2}.$$
 Again Lemma \ref{lemma_gluing} proves the surjectivity of $ \widetilde{\theta}_{a_1 \# a_2}$ and the injectivity follows from the inclusion 
 \\ $ {\theta}_{a_1 \# a_2}^{-1}\left( \mathcal{I}_{\widetilde{x}_1}\otimes 1 + 1\otimes \mathcal{I}_{\widetilde{x}_2}\right) \subset \mathcal{I}_{\widetilde{x}}$, so  $ \widetilde{\theta}_{a_1 \# a_2}$ is an isomorphism. Therefore we have a $1:1$ correspondance between indecomposable representations $V$ of  $ \overline{\mathcal{S}}_A(\mathbf{\Sigma})$ with shadow $\widetilde{x}$ and pairs $(V_1, V_2)$ with $V_i$ an indecomposable $ \overline{\mathcal{S}}_A(\mathbf{\Sigma}_i)$ semi weight module with shadow $\widetilde{x}_i$ (the correspondence sends $(V_1, V_2)$ to $V_1\otimes V_2$). This concludes the proof.
 \end{proof}
 

 \section{Classification of weight representations of reduced stated skein algebras}\label{sec_classification}
 
 Let $\mathbf{\Sigma}$  be a connected essential marked surface.
 For $x\in X(\mathbf{\Sigma})$, we denote by $\rho_x: \Pi_1(\Sigma, \mathbb{V}) \to \SL_2$ the functor associated by Corollary \ref{coro_classical_limit}. For $p\in \mathring{\mathcal{P}}$, fix a boundary arc $a$ in the same connected component than $p$ and let $\alpha_p\in \pi_1(\Sigma, v_a)$ be the class of a simple closed curve which bounds an annulus whose other boundary component is $p$. We say that $x$ \textit{is central at} $p$ if $\rho_x(\alpha_p)=\pm \mathds{1}_2$. This property clearly does not depend on the choices of $a$ and $\alpha_p$. 
\par   Let $\widehat{x}=(x,h_p,h_{\partial})_{p, \partial} \in \widehat{X}(\mathbf{\Sigma})$ and decompose the set of inner punctures as $\mathring{\mathcal{P}}= \mathring{\mathcal{P}}_0 \bigsqcup \mathring{\mathcal{P}}_1 \sqcup \mathring{\mathcal{P}}_2$ where 
\begin{align*}
&  \mathring{\mathcal{P}}_2:= \{p\in \mathring{\mathcal{P}}, \mbox{ such that } x \mbox{ is central at }p \mbox{ and }h_p \neq \pm 2\} \\
&  \mathring{\mathcal{P}}_1:= \{p\in \mathring{\mathcal{P}}\setminus \mathring{\mathcal{P}}_2, \mbox{ such that } \chi_x(\gamma_p) = \pm 2 \mbox{ and }h_p \neq \pm 2\}.
\end{align*}
A map $\sigma : \mathring{\mathcal{P}}\to \Delta \bigsqcup \overline{\Delta} $ is called a \textit{coloring compatible with} $\widehat{x}$ if $(1)$ for $p\in \mathring{\mathcal{P}}_0$, then $\sigma(p)=S$ and $(2)$ if $p\in \mathring{\mathcal{P}}_1$, then $\sigma(p)\in \{S,P\}$. By convention, if $\mathring{\mathcal{P}}=\emptyset$, we consider that every $\widehat{x}$ has a unique coloring. We write
$$ m:= | \mathring{\mathcal{P}}_2| \quad \mbox{ and }\quad m':= |\mathring{\mathcal{P}}_1\cup \mathring{\mathcal{P}}_2|.$$
 
 Let $\mathcal{C}$ be the category of weight $\overline{\mathcal{S}}_A(\mathbf{\Sigma})$ modules and $\overline{\mathcal{C}}$ be the category of semi weight $\overline{\mathcal{S}}_A(\mathbf{\Sigma})$ modules. The main theorem of this paper is the following:
 
 \begin{theorem}\label{main_theorem}  Let $\mathbf{\Sigma}=(\Sigma, \mathcal{A})$ be a connected essential marked surface  which either has a boundary component with at least two boundary arcs or which do not have any inner puncture.
 Let  $\widehat{x}=(x, h_p, h_{\partial}) \in \widehat{X}(\mathbf{\Sigma})$.
\begin{enumerate}
\item $\widehat{x}\in \mathcal{AL}(\mathbf{\Sigma})$ if and only if $m=0$; 
\item $\widehat{x} \in \widehat{X}^{reg}(\mathbf{\Sigma})$ if and only if $m'=0$ (so $\widehat{X}^{reg}(\mathbf{\Sigma})\subset \mathcal{AL}(\mathbf{\Sigma})$); 
\item the indecomposable semi-weight representations with classical shadow $\widehat{x}$ are in $1:1$ correspondence with the set of colorings compatible with $\widehat{x}$; 
\item the indecomposable weight representations with classical shadow $\widehat{x}$ correspond to the colorings taking values in $\{S,\overline{S}, ((1,1), 1), \overline{((1,1), 1)}\}$: there are thus $4^m$ such representations; 
\item the irreducible representations with classical shadow $\widehat{x}$ correspond to the colorings taking values in $\{S, \overline{S}\}$: there are thus $2^m$ such representations;
\item the indecomposable representations with classical shadow $\widehat{x}$ which are projective objects in $\mathcal{C}$ correspond to the colorings sending the elements of $\mathring{\mathcal{P}}_1$ to $S$ and the elements of $\mathcal{P}_2$ to an element in $\{((1,1), 1), \overline{((1,1), 1)}\}$: there are thus $2^m$ such representations; 
\item the indecomposable representations with classical shadow $\widehat{x}$ which are projective objects in $\overline{\mathcal{C}}$ correspond to the colorings sending the elements of $\mathring{\mathcal{P}}_1$ to $P$ and the elements of $\mathring{\mathcal{P}}_2$ to an element in $\{P, \overline{P}\}$: there are thus $2^m$ such representations.
\end{enumerate}
\end{theorem}
 
 \begin{proof}
 If $\mathbf{\Sigma}$ does not have inner puncture then the theorem follows from Corollary \ref{coro_2Psurfaces}. Else, it contains a boundary component with at least two boundary arcs.
 Write $\mathring{\mathcal{P}}=\{p_1, \ldots, p_n\}$.
$\mathbf{\Sigma}$ can be obtained from a 2P-marked surface $\mathbf{\Sigma}^0$ and $n$ copies of $\mathbb{D}_1$ together, i.e.
 $$ \mathbf{\Sigma}= \mathbf{\Sigma}^0 \cup_{a_1 \# b_1} \mathbb{D}_1 \cup_{a_2\# b_2} \ldots \cup_{a_n \# b_n} \mathbb{D}_1$$
 for some boundary arcs $a_i,b_i$. 
 Let $\theta: \overline{\mathcal{S}}_A (\mathbf{\Sigma}) \hookrightarrow \overline{\mathcal{S}}_A (\mathbf{\Sigma}^0) \otimes \overline{\mathcal{S}}_A (\mathbb{D}_1)^{\otimes n}$ be the induced splitting morphism and $\pi: \widehat{X}(\mathbf{\Sigma}^0) \times \widehat{X}(\mathbb{D}_1)^{\times n} \to \widehat{X}(\mathbf{\Sigma})$ the morphism defined by the restriction of $\theta$ to the center. By Theorem \ref{theorem_gluing}, one has $\pi( \mathcal{AL}(\mathbf{\Sigma}^0) \times \mathcal{AL}(\mathbb{D}_1)^{\times n})= \mathcal{AL}(\mathbf{\Sigma})$. By Corollary \ref{coro_2Psurfaces}, $\mathcal{AL}(\mathbf{\Sigma}^0)= \widehat{X}(\mathbf{\Sigma}^0)$ and by Corollary \ref{coro_QG}, $\mathcal{AL}(\mathbb{D}_1)$ is the set of elements $(g, h_p, h_{\partial})$ such that either $\varphi(g) \neq \pm \mathds{1}_1$ or $\varphi(g)=\pm 2$ and  $h_p = \mp 2$. Therefore $\mathcal{AL}(\mathbf{\Sigma})$ is the set of elements $\widehat{x}=(x,h_p, h_{\partial})$ such that for every $p\in \mathring{\mathcal{P}}$ then either $\rho_x(\alpha_p)\neq \pm \mathds{1}_1$ or $\rho_x(\alpha_p)=\pm \mathds{1}_1$ and $h_p=\mp2$. This proves $(1)$. 
 Assertion $(2)$ is proved in  Theorem \ref{theorem_smooth2}.
 Let $\widehat{x} \in \widehat{X}(\mathbf{\Sigma})$. By Theorem \ref{theorem_gluing_surjective}, there exists $(\widehat{x}_0, \widehat{x}_1, \ldots, \widehat{x}_n) \in \widehat{X}(\mathbf{\Sigma}^0) \times \widehat{X}(\mathbb{D}_1)^{\times n} $ such that $p(\widehat{x}_0, \widehat{x}_1, \ldots, \widehat{x}_n) =\widehat{x}$. 
  By Theorem \ref{theorem_gluing} the (semi) weight indecomposable $\overline{\mathcal{S}}_A(\mathbf{\Sigma})$ module with classical shadow $\widehat{x}$ have the form 
 $V_0 \otimes V_1 \otimes \ldots \otimes V_n$ where $V_0$ is an indecomposable (semi) weight  $\overline{\mathcal{S}}_A (\mathbf{\Sigma}^0)$ module with shadow $\widehat{x}_0$ and $V_i$ is an indecomposable (semi) weight $\overline{\mathcal{S}}_A (\mathbb{D}_1)$-module with shadow $\widehat{x}_i$. By Corollary \ref{coro_2Psurfaces} $V_0$ is unique up to isomorphism.
For $i\in \{1, \ldots, n\}$, write $\widehat{x}_i= (g_i, h_{p_i}, h_{\partial}^i)$ and let $I^2:= \{i \mbox{ such that }p_i \in \mathring{\mathcal{P}}^2\}$ and $I^1:= \{i \mbox{ such that }p_i \in \mathring{\mathcal{P}}^1\}$.
 By Corollary \ref{coro_QG}, for $i\in \{1, \ldots, n\}$, we have up to isomorphism: $(i)$ a single possible $V_i$ if $i\notin I^1 \cup I^2$, $(ii)$ two possible semi-weight modules $V_i$ if $i\in I^1$ and one of them is a  weight module, $(iii)$ 
 if $i \in I^2$, the indecomposable semi-weight modules $V_i$ are in $1:1$ correspondance with $\Delta\sqcup \overline{\Delta}$. Therefore, the isomorphism classes of semi weight $\overline{\mathcal{S}}_A(\mathbf{\Sigma})$ modules 
with shadow $\widehat{x}$ are in $1:1$ correspondance with the set of colorings admissible at $\widehat{x}$ and we have proved $(3)$. Such an indecomposable representation $V_0 \otimes V_1 \otimes \ldots \otimes V_n$ is weight/ simple/projective in $\mathcal{C}$/ projective in $\overline{\mathcal{C}}$ if and only if each of the $V_i$ has the same property so $(4)$, $(5)$, $(6)$ follow from  Corollary \ref{coro_QG}.
  This  concludes the proof.
  
 \end{proof}
 
 \section{Representations of unreduced stated skein algebras and the \QMAs}\label{sec_QMS}
 
 \subsection{Representations of unreduced stated skein algebras}
 
 Let $\mathbf{\Sigma}$ be an essential  marked surface and denote by $\mathbf{\Sigma}^*$ the marked surface obtained from $\mathbf{\Sigma}$ by gluing a triangle along each boundary arc of $\mathbf{\Sigma}$. So each boundary arc $a$ of $\mathbf{\Sigma}$ corresponds to two boundary arcs, say $a'$ and $a''$ in $\mathbf{\Sigma}^*$. Let $i: \mathbf{\Sigma} \to \mathbf{\Sigma}^*$ be the embedding which is the identity outside a small neighborhood of $\mathcal{A}$ and embedding $a$ into $a'$. It is proved in  \cite{LeYu_SSkeinQTraces} that the morphism $i_* : \mathcal{S}_A(\mathbf{\Sigma}) \to \overline{\mathcal{S}}_A(\mathbf{\Sigma}^*)$  induces an injective morphism 
  $$ j : \mathcal{S}_A(\mathbf{\Sigma}) \hookrightarrow \overline{\mathcal{S}}_A(\mathbf{\Sigma}^*).$$
 Therefore, for each (semi-weight) representation $r: \overline{\mathcal{S}}_A(\mathbf{\Sigma}^*)\to \End(V)$, we can associate a representation $r^*= r\circ j : \mathcal{S}_A(\mathbf{\Sigma})  \to \End(V)$. Such a representation will be called a \textit{reduced representation}. Note that since we assume that $\mathbf{\Sigma}$ is essential, then each connected component of  $\mathbf{\Sigma}^*$ as a boundary component with at least two boundary arcs. Thus Theorem \ref{main_theorem} provides a classification of such reduced representations and thus a lot of examples. The goal of this subsection is to prove the 
 
 \begin{theorem}\label{theorem_Skein}
 Let $\mathbf{\Sigma}=(\Sigma_{g,n}, \mathcal{A})$ be a connected essential marked surface of genus $g$ with $n$ boundary components such that each boundary component contains at most one boundary arc. So it has $k:= n-|\mathcal{A}|$ inner punctures $p_1, \ldots, p_k$ and $|\mathcal{A}|$ boundary punctures $p_{\partial_1}, \ldots, p_{\partial_{|\mathcal{A}|}}$.
 \begin{enumerate}
 \item The center $\underline{Z}_{\mathbf{\Sigma}}$ of $\mathcal{S}_A(\mathbf{\Sigma})$ is generated by the image $\underline{Z}^0_{\mathbf{\Sigma}}$ of the Frobenius morphism together with the peripheral curves $\gamma_{p_1}, \ldots, \gamma_{p_k}$. So the closed points of $\widehat{\underline{X}}(\mathbf{\Sigma}):= \Specm(\underline{Z}_{\mathbf{\Sigma}})$ are in $1:1$ correspondence with the set of elements $\widehat{\rho}=(\rho, h_{p_1}, \ldots, h_{p_k})$ with $\rho: \Pi_1(\Sigma, \mathbb{V})\to \SL_2$ and $h_{p_i}\in \mathbb{C}$ is such that $T_N(h_{p_i})=-\tr(\rho(\gamma_{p_i}))$. 
 \item The PI-degree $\underline{D}_{\mathbf{\Sigma}}$ of  $\mathcal{S}_A(\mathbf{\Sigma})$ is equal to $N^{3g-3+n+2|\mathcal{A}|}(= D_{\mathbf{\Sigma}^*})$. 
 \item For $\rho: \Pi_1(\Sigma, \mathbb{V})\to \SL_2$, write $\mu(\rho):= (\rho(\alpha(p_{\partial_1})), \ldots, \rho(\alpha(p_{\partial_{|\mathcal{A}|}})) \in (\SL_2)^{|\mathcal{A}|}$.
 Let  $\widehat{\rho}=(\rho, h_{p_i}) \in \widehat{\underline{X}}(\mathbf{\Sigma})$.
 \begin{itemize}
 \item[(i)] If $\mu(\rho)\in (\SL_2^0)^{|\mathcal{A}|}$ and for each $1\leq i \leq k$, either $\tr(\rho(\gamma_{p_i}))\neq \pm 2$ or $\tr(\rho(\gamma_{p_i}))=\pm 2$ and $h_{p_i}=\mp 2$, then $\widehat{\rho}$ belongs to the Azumaya locus of $\mathcal{S}_A(\mathbf{\Sigma})$.
 \item[(ii)] Suppose that either $\mu(\rho)\in (\SL_2^1)^{|\mathcal{A}|}$ or that $\mu(\rho)\in (\SL_2^0)^{|\mathcal{A}|}$ and there exists $i$ such that $\tr(\rho(\gamma_{p_i}))=\pm 2$ and $h_{p_i}\neq \mp 2$, then $\widehat{\rho}$ does not belong to the Azumaya locus of $\mathcal{S}_A(\mathbf{\Sigma})$.
\end{itemize}
 \end{enumerate}
 \end{theorem}
 
 \begin{remark} In \cite{LeYu_Survey}, L\^e and Yu have announced without proof a formula for the PI-degree of the stated skein algebra of any marked surface. Under the assumptions of the theorem, their formula coincides with ours. As detailed in the next subsection, in the particular case where $|\mathcal{A}|=1$, our theorem re-proves some results of \cite{BaseilhacRoche_LGFT1,BaseilhacRoche_LGFT2, GanevJordanSafranov_FrobeniusMorphism}.
 \end{remark}
 

 \begin{lemma}\label{lemma_Skein1} Let $\underline{Z}'_{\mathbf{\Sigma}}\subset \underline{Z}_{\mathbf{\Sigma}}$ be the algebra generated by the image of the Frobenius and the peripheral curves $h_{p_i}$. Then the rank $R$ of $\mathcal{S}_A(\mathbf{\Sigma})$ over $\underline{Z}'_{\mathbf{\Sigma}}$ satisfies $R= (D_{\mathbf{\Sigma}^*})^2$. \end{lemma}
 
 \begin{proof}
 Let $Z^0 \subset  \mathcal{O}_q\SL_2$ denote the small center of $\mathcal{O}_q\SL_2$ (i.e. the image of the Frobenius map $Fr:\mathcal{O}[\SL_2]\to \mathcal{O}_q\SL_2$ where $\mathcal{O}[\SL_2]$ is the ring of regular functions of $\SL_2$). It is a classical fact (see e.g. \cite[Proposition $2.2$]{BrownGordon_OqG}) that $\mathcal{O}_q\SL_2$ is free over $Z^0$ with rank $[\mathcal{O}_q\SL_2 : Z^0]= N^3$. By \cite[Section $2.4$]{KojuQuesneyClassicalShadows}, one has an algebra isomorphism $\mathcal{O}_q\SL_2\cong \mathcal{S}_A(\mathbb{B})$, where the bigon $\mathbb{B}$ is a disc with two boundary arcs. Consider the set $\mathbb{G}$ of  Example \ref{example_presenting_graph} from which we have removed a corner arc $\alpha(p_{\partial_0})$ and fix an arbitrary indexing of its elements as $\mathbb{G}=\{\alpha_1, \ldots, \alpha_{|\mathbb{G}|}\}$. Each arc $\alpha \in \mathbb{G}$ defines an embedding $\mathbb{B} \hookrightarrow \mathbf{\Sigma}$ which induces a linear  morphism $\varphi_{\alpha}: \mathcal{O}_q\SL_2 \to \mathcal{S}_A(\mathbf{\Sigma})$. Consider the linear map
 $$\varphi^{\mathbb{G}}:= \otimes_{\alpha \in \mathbb{G}} \varphi_{\alpha}: (\mathcal{O}_q\SL_2)^{\otimes \mathbb{G}} \to \mathcal{S}_A(\mathbf{\Sigma}), \quad \varphi^{\mathbb{G}}(x_1\otimes \ldots \otimes x_{|\mathbb{G}|})= \varphi_{\alpha_1}(x_1)\ldots \varphi_{\alpha_{|\mathbb{G}|}}(x_{|\mathbb{G}|}).$$
 By \cite[Corollary $3.8$]{KojuPresentationSSkein}, $\varphi^{\mathbb{G}}$ is a linear isomorphism which commutes with the Frobenius morphism. Therefore if $\underline{Z}_{\mathbf{\Sigma}}^0\subset \mathcal{S}_A(\mathbf{\Sigma})$ denotes the image of the Frobenius morphism, then $\mathcal{S}_A(\mathbf{\Sigma})$ is free over  $\underline{Z}_{\mathbf{\Sigma}}^0$ of rank
 $$ [\mathcal{S}_A(\mathbf{\Sigma}):  \underline{Z}_{\mathbf{\Sigma}}^0] = [\mathcal{O}_q\SL_2 : Z^0]^{|\mathbb{G}|}= (N^3)^{2g+k + 2(|\mathcal{A}|-1)}= N^{6g-6 +3k +6|\mathcal{A}|}.$$
Now $\underline{Z}'_{\mathbf{\Sigma}}$ is generated over $\underline{Z}_{\mathbf{\Sigma}}^0$ by the elements $\gamma_{p_1}, \ldots, \gamma_{p_k}$ which satisfy $T_N(\gamma_{p_i})=Fr(\gamma_{p_i})\in  \underline{Z}_{\mathbf{\Sigma}}^0$. So $[\underline{Z}'_{\mathbf{\Sigma}}: \underline{Z}^0_{\mathbf{\Sigma}}] = N^k$ and 
$$ [\mathcal{S}_A(\mathbf{\Sigma}): \underline{Z}'_{\mathbf{\Sigma}}]= [\mathcal{S}_A(\mathbf{\Sigma}): \underline{Z}^{0}_{\mathbf{\Sigma}}] [\underline{Z}'_{\mathbf{\Sigma}}: \underline{Z}^0_{\mathbf{\Sigma}}] ^{-1}= N^{6g-6+2k + 6|\mathcal{A}|} = (D_{\mathbf{\Sigma}^*})^2.$$

 \end{proof}
 
 
 \par Let $\mathcal{S}_A(\mathbf{\Sigma})^{loc}=\mathcal{S}_A(\mathbf{\Sigma})[(\alpha(p_{\partial_j})_{-+})^{-1}]$ be the algebra obtained from $\mathcal{S}_A(\mathbf{\Sigma})$ by localizing by all bad arcs $\alpha(p_{{\partial}_j})_{-+}$. 
  
 \begin{lemma}\label{lemma_Skein2} 
  $j(\alpha(p_{\partial_j})_{-+})$ is invertible in $\overline{\mathcal{S}}_A(\mathbf{\Sigma}^*)$ so the morphism $j$ induces a morphism
 $$ j^{loc}: \mathcal{S}_A(\mathbf{\Sigma})^{loc} \hookrightarrow \overline{\mathcal{S}}_A(\mathbf{\Sigma}^*).$$
Moreover
 $\overline{\mathcal{S}}_A(\mathbf{\Sigma}^*)$ is generated, as an algebra, by the image of $j^{loc}$ together with its center $Z_{\mathbf{\Sigma}^*}$.
 \end{lemma}
 
 \begin{proof}
 Let us first assume that $\mathbf{\Sigma}=(\Sigma_{0,1}, \{a\})=: \mathbf{m}_1$ is the once punctured monogon, i.e. an annulus with a single boundary arc in one of its boundary component. We write $B_q\SL_2:= \mathcal{S}_A(\mathbf{m}_1)$; the notation is motivated by the fact, proved in  \cite[Proposition $4.17$]{CostantinoLe19} (see also \cite{LeSikora_SSkein_SLN, KojuMurakami_QCharVar}), that this algebra is  isomorphic to Majid's braided quantum group, i.e. the transmutation of $\mathcal{O}_q\SL_2$. Let $\alpha(p)_{ij} \in B_q\SL_2$ be the class of the stated corner arc around the only boundary puncture of $\mathbf{m}_1$. Then $\mathbf{m}_1^* = \mathbb{D}_1$, so $j_{\mathbf{m}_1}$ embeds $B_q\SL_2$ into $U_q\mathfrak{gl}_2$. This map is a skein analogue of a celebrated Majid's morphism in quantum groups theory (see \cite{Majid_QGroups}). Let $\alpha_{ij}, \beta_{kl} \in U_q\mathfrak{gl}_2$ be the stated arcs defined in Section \ref{sec_D1} (so $\alpha_{-+}=\beta_{+-}=0$). A simple computation, shown in Figure \ref{fig_MajidMorphism}, shows that 
 \begin{align*}
 {}&  j_{\mathbf{m}_1} (\alpha(p)_{++})= -A^{5/2} \alpha_{+-}\beta_{++}, &  j_{\mathbf{m}_1} (\alpha(p)_{--})= -A^{5/2} \alpha_{--}\beta_{-+}, \\
 {} &  j_{\mathbf{m}_1} (\alpha(p)_{-+})= -A^{5/2} \alpha_{--}\beta_{++}, &  j_{\mathbf{m}_1} (\alpha(p)_{+-})= -A^{5/2} \alpha_{+-}\beta_{-+}+ A^{1/2}\alpha_{++}\beta_{--}.
  \end{align*}
    
 \begin{figure}[!h] 
\centerline{\includegraphics[width=9cm]{jm1.eps} }
\caption{An illustration of the equality $j_{\mathbf{m}_1}(\alpha(p)_{ij})= -A^{5/2} \alpha_{i-}\beta_{j+} + A^{1/2}\alpha_{i+} \beta_{j-}$.
} 
\label{fig_MajidMorphism} 
\end{figure} 

  In particular, $ j_{\mathbf{m}_1} (\alpha(p)_{-+})$ is invertible and $ j_{\mathbf{m}_1} (\alpha(p)_{-+})^{-1}= -A^{-5/2} \alpha_{++}\beta_{--}$. From these formulas, together with the fact that $\alpha_{++}\beta_{++}, \alpha_{--}\beta_{--} \in Z_{\mathbb{D}_1}$, we deduce that the algebra $A_{\mathbf{m}_1}\subset U_q\mathfrak{gl}_2$ generated by the image of $j_{\mathbf{m}_1}^{loc}$ together with $Z_{\mathbb{D}_1}$, contains every elements of the form $\alpha_{ij}\beta_{kl}$ except possibly $\alpha_{+-}\beta_{--}$ and $\alpha_{++}\beta_{-+}$.  To prove that these two elements also belong to $A_{\mathbf{m}_1}$, note that $\beta_{--}^2=(\alpha_{++}\beta_{--})(\alpha_{--}\beta_{--}) \in A_{\mathbf{m}_1}$ so $\alpha_{+-}\beta_{--}= (\alpha_{+-}\beta_{++}) \beta_{--}^2 \in A_{\mathbf{m}_1}$. Similarly, $\alpha_{++}\beta_{-+}= \alpha_{++}^2(\alpha_{--}\beta_{-+})= (\alpha_{--}\alpha_{-+}) (\alpha_{++}\beta_{++})(\alpha_{++}\beta_{--}) \in A_{\mathbf{m}_1}$. 
  Therefore for every $i,j,k,l \in \{ \pm \}$, $\alpha_{ij}\beta_{kl}\in A_{\mathbf{m}_1}$. So $\alpha_{ij}^2 = (\alpha_{ij} \beta_{++})(\alpha_{ij} \beta_{--}) \in A_{\mathbf{m}_1}$ and similarly $\beta_{kl}^2 \in A_{\mathbf{m}_1}$.
  Write $N=2n+1$ and recall that $\alpha_{ij}^N, \beta_{kl}^N \in A_{\mathbf{m}_1}$. So $\alpha_{++}= \left( \alpha_{++}^2 \right)^{n+1} \alpha_{--}^N \in A_{\mathbf{m}_1}$ and similarly $\alpha_{--}, \beta_{++}, \beta_{--} \in A_{\mathbf{m}_1}$. 
  This implies that $\alpha_{+-}= (\alpha_{+-}\beta_{--}) \beta_{++}\in A_{\mathbf{m}_1}$ and similarly $\beta_{-+}=\alpha_{--} (\alpha_{++}\beta_{-+})\in A_{\mathbf{m}_1}$. So all elements $\alpha_{ij}, \beta_{kl}$ belong to $A_{\mathbf{m}_1}$. Since these elements generate $U_q\mathfrak{gl}_2$, we have proved that $A_{\mathbf{m}_1}=U_q\mathfrak{gl}_2$.
  
  \vspace{2mm}
  \par When $\mathbf{\Sigma}=(\Sigma, \mathcal{A})$ is a general surface satisfying the hypotheses of Theorem \ref{theorem_Skein}, for each boundary puncture $p_{\partial_j}$ contained in a boundary component $\partial_j$, by embedding $\mathbf{m}_1$ into a neighborhood of $\partial_j$, we obtain an algebra morphism $\mu_q^j: B_q\SL_2 \to \mathcal{S}_A(\mathbf{\Sigma})$ sending $\alpha(p)_{kl}$ to $\alpha(p_{\partial_j})_{kl}$. The morphism $\mu_q:= \otimes_{j=1}^{|\mathcal{A}|} \mu_q^j : (B_q\SL_2)^{|\mathcal{A}|} \to \mathcal{S}_A(\mathbf{\Sigma})$ is called the quantum moment map in literature (see \cite{KojuMCGRepQT, KojuSurvey} and reference therein). Similarly, by embedding a copy of  $\mathbb{D}_1$ in the neighborhood of $\partial_j$ seen as a boundary component of $\mathbf{\Sigma}^*$ this time (recall that $\Sigma=\Sigma^*$ but that each $\partial_j$ contains now two boundary arcs), we obtain an algebra morphism $\mu_q^* : (U_q\mathfrak{gl}_1)^{\otimes |\mathcal{A}|} \to \overline{\mathcal{S}}_A(\mathbf{\Sigma}^*)$ such that the following diagram commutes: 
  $$ \begin{tikzcd}
  (B_q\SL_2)^{\otimes |\mathcal{A}|} 
  \ar[rr, hook, "(j_{\mathbf{m}_1})^{\otimes |\mathcal{A}|}"] 
  \ar[d, "\mu_q"] &{}&
   (U_q\mathfrak{gl}_2)^{\otimes |\mathcal{A}|} 
   \ar[d, "\mu_q^*"] \\
  \mathcal{S}_A(\mathbf{\Sigma}) \ar[rr, hook, "j_{\mathbf{\Sigma}}"]
   &{}& \overline{\mathcal{S}}_A(\mathbf{\Sigma}^*)
  \end{tikzcd}$$
   \begin{figure}[!h] 
\centerline{\includegraphics[width=4cm]{QMomentMaps.eps} }
\caption{An illustration of the quantum moment maps $\mu_q$ and $\mu_q^*$.
} 
\label{fig_qMomentMaps} 
\end{figure} 
  So the fact that $j_{\mathbf{m}_1}(\alpha(p)_{-+})$ is invertible implies that $j_{\mathbf{\Sigma}}(\alpha(p_{\partial_j})_{-+}) $ is invertible as well so $j^{loc}_{\mathbf{\Sigma}}$ is well defined. Let $A_{\mathbf{\Sigma}} \subset \overline{\mathcal{S}}_A(\mathbf{\Sigma}^*)$ be the subalgebra generated by the image of $j_{\mathbf{\Sigma}}^{loc}$ and the center $Z_{\mathbf{\Sigma}^*}$. By the preceding case and the commutativity of the above diagram, we see the image $\Image(\mu_q^*)$ of $\mu_q^*$ is included in $A_{\mathbf{\Sigma}}$. Let $a_1, \ldots, a_{|\mathcal{A}|}$ be the boundary arcs of $\mathbf{\Sigma}$. By construction, each $a_i$ corresponds to two boundary arcs $a_i'$ and $a_i''$ of $\mathbf{\Sigma}^*$ and the image of $j_{\mathbf{\Sigma}}$ is spanned by the classes $[D,s]$ of stated diagrams such that $D\cap a_i'' = \emptyset$ for all $1\leq i \leq |\mathcal{A}|$. Let $[D,s] \in \overline{\mathcal{S}}_A(\mathbf{\Sigma}^*)$ be the class of an arbitrary stated diagram. As illustrated in Figure \ref{fig_slide}, by performing skein relations in small neighborhoods of each boundary component $\partial_j$, we see that $[D,s]$ is equal to a sum $[D,s]= \sum_i x_i [D_i,s_i]$ where $x_i \in \Image(\mu_q^*)$ and $D_i \cap a_j''=\emptyset$ for all $j$. This proves that $[D,s]\in A_{\mathbf{\Sigma}}$ so $A_{\mathbf{\Sigma}}=\overline{\mathcal{S}}_A(\mathbf{\Sigma}^*)$ and the proof is complete.
     \begin{figure}[!h] 
\centerline{\includegraphics[width=6cm]{Slide.eps} }
\caption{Using a skein relation, we can write any class $[D,s]$ as a linear combination of products of elements in the image of $\mu_q^*$ (here corner arcs $\alpha(p)_{j+}$ and $\alpha(p)_{j-}$) and elements $[D_i, s_i]$ where the $D_i$ do not intersect the boundary arcs $a''$ (here the boundary arc on the right).
} 
\label{fig_slide} 
\end{figure} 
 \end{proof}
 
  Let $\underline{\widehat{X}}(\mathbf{\Sigma})=\Specm (\underline{Z}'_{\mathbf{\Sigma}})$ and $j^*: \widehat{X}(\mathbf{\Sigma}^*) \to \underline{\widehat{X}}(\mathbf{\Sigma})$ the dominant map defined by the embedding $j: \underline{Z}_{\mathbf{\Sigma}} \hookrightarrow  Z_{\mathbf{\Sigma}^*}$. 
  
  \begin{lemma}\label{lemma_Skein3} The image of $j^*$ is the set of elements $\widehat{\rho}=(\rho, h_{p_1}, \ldots, h_{p_k}) \in \underline{\widehat{X}}(\mathbf{\Sigma})$ such that $\mu(\rho)\in \SL_2^0 \times \ldots \times \SL_2^0$. 
  \end{lemma}
  
  \begin{proof} For each boundary arc $a \in \mathcal{A}$ of $\mathbf{\Sigma}$ fix a point $v_a \in a$ and let $\mathbb{V}:=\{v_a\}_{a\in \mathcal{A}}$. By definition $\mathbf{\Sigma}^*$ is the marked surface with $\Sigma$ as underlying surface and for which each boundary arc $a$ has been  replaced by two boundary arcs $a'$ and $a''$.
  For each $a\in \mathcal{A}$ fix $v_{a'}:= v_a \in a'$ and $v_{a''}\in a''$ and $\mathbb{V}^*=\{v_{a'}, v_{a''}\}_{a\in \mathcal{A}}$. Using Theorem \ref{theorem_classical_limit}, we identify $\underline{\widehat{X}}(\mathbf{\Sigma})$ with the set of elements $\widehat{\rho}=(\rho, h_{p_i})_i$ where $\rho: \Pi_1(\Sigma, \mathbb{V}) \to \SL_2$ and $h_{p_i}\in \mathbb{C}$ is such that $\tr(\rho(\gamma_{p_i}))=-T_N(h_{p_i})$. Recall that $\widehat{X}(\mathbf{\Sigma}^*)$ is identified as the set of elements $\widehat{\rho}^*=(\rho, h_{p_i}, h_{\partial_j})_{i,j}$ where $\rho: \Pi_1(\Sigma, \mathbb{V}^*) \to \SL_2$ sends every corner path $\alpha(p)$ to an element of $\SL_2^1$. Since $\mathbb{V}\subset \mathbb{V}^*$, we have a full embedding $ \Pi_1(\Sigma, \mathbb{V}) \to \Pi_1(\Sigma^*, \mathbb{V}^*)$ and thus a restriction morphism $\Hom(\Pi_1(\Sigma^*, \mathbb{V}^*), \SL_2) \to \Hom(\Pi_1(\Sigma, \mathbb{V}), \SL_2)$ sending $\rho^*$ to  $\restriction{ \rho^*}{\Pi_1(\Sigma, \mathbb{V})}$. By definition, the map $j^*$ satisfies 
  $$ j^*( (\rho^*, h_{p_i}, h_{\partial_j})) = (\restriction{ \rho^*}{\Pi_1(\Sigma, \mathbb{V})}, h_{p_i}).$$
  So given a functor $\rho: \Pi_1(\Sigma, \mathbb{V})\to \SL_2$ such that $\rho(\alpha(p))\in \SL_2^0$ for all boundary puncture $p\in \mathcal{P}^{\partial}$ of $\mathbf{\Sigma}$, we need to prove that there exists a functor $\rho^*: \Pi_1(\Sigma^*, \mathbb{V}^*)\to \SL_2$ such that $\restriction{ \phi_*}{\Pi_1(\Sigma, \mathbb{V})}= \rho$ and such that $\rho^*(\alpha(p'))\in \SL_2^1$ for all boundary puncture $p'$ of $\mathbf{\Sigma}^*$. For a boundary component $\partial$ of $\Sigma$ containing a boundary puncture $p_{\partial}\in P^{\partial}$, the same boundary component considered in $\Sigma^*$ contains two boundary punctures $p_{\partial}'$ and $p_{\partial}''$ such that $\alpha(p_{\partial})= \alpha(p''_{\partial})\alpha(p'_{\partial})$. So lifting $\rho$ to a functor $\rho^*$ amounts to choose for each $p_{\partial}\in P^{\partial}$ two elements $\rho^*(\alpha(p'_{\partial})), \rho^*(\alpha(p''_{\partial})) \in \SL_2^1$ such that $\rho(\alpha(p_{\partial}))= \rho^*(\alpha(p'_{\partial}))\rho^*(\alpha(p''_{\partial}))$. The existence of such lift thus follows from the surjectivity of the map
  $$ \pi : \SL_2^1 \times \SL_2^1 \to \SL_2^0, \quad \pi(A,B):=AB.$$
  Indeed, denote by $B^+, B^- \subset \SL_2$ the subalgebras of upper and lower triangular matrices and write $w:= \begin{pmatrix} 0 & 1 \\ -1 & 0 \end{pmatrix}$ so that $\SL_2^0= B^-B^+$ and $\SL_2^1=B^- w B^+$. Then given $M\in \SL_2^0$ such that $M=M_- M_+$ with $M_{\pm} \in B^{\pm}$ set $A:= M_-w \in \SL_2^1$ and $B:= - wM_+\in \SL_2^1$ then $\pi(AB)=M$. Thus $\pi$ is surjective. This concludes the proof.
  
  
  \end{proof}
 
 
 \begin{proof}[Proof of Theorem \ref{theorem_Skein}]
Let $\widehat{\rho}=(\rho, h_{p_1}, \ldots, h_{p_k}) \in \underline{\widehat{X}}(\mathbf{\Sigma})$ such that $\mu(\rho)\in \SL_2^0 \times \ldots \times \SL_2^0$ and $\widehat{\rho}^* \in \widehat{X}(\mathbf{\Sigma}^*)$ such that $j^*(\widehat{\rho}^*)=\widehat{\rho}$. 
Let $\mathcal{I}_{\widehat{\rho}}\subset \mathcal{S}_A(\mathbf{\Sigma})$ be the ideal generated by elements $z-\chi_{\widehat{\rho}}(z)$ for all $z\in \underline{Z}'_{\mathbf{\Sigma}}$ and write $\mathcal{S}_A(\mathbf{\Sigma})_{\widehat{\rho}}:= \quotient{\mathcal{S}_A(\mathbf{\Sigma})}{\mathcal{I}_{\widehat{\rho}}}$. By Lemma \ref{lemma_Skein2}, $j$ induces a surjective map 
$$ j_{\widehat{\rho}}: \mathcal{S}_A(\mathbf{\Sigma})_{\widehat{\rho}} \to \overline{\mathcal{S}}_A(\mathbf{\Sigma}^*)_{\widehat{\rho}^*}.$$
Let $r: \overline{\mathcal{S}}_A(\mathbf{\Sigma}^*) \to \End(V)$ be an irreducible representation with classical shadow $\widehat{\rho}^*$ and $r':= r \circ j: \mathcal{S}_A(\mathbf{\Sigma})\to \End(V)$. Then $r'$ has classical shadow $\widehat{\rho}$ and is irreducible by surjectivity of $ j_{\widehat{\rho}}$ and the Schur lemma. First suppose that for all $i$, either  $\tr(\rho(\gamma_{p_i}))\neq \pm 2$ or $\tr(\rho(\gamma_{p_i}))= \pm  2$ and $h_{p_i}=\mp 2$. By Theorem \ref{main_theorem}, $\widehat{\rho}^*$ belongs to the Azumaya locus of $\overline{\mathcal{S}}_A(\mathbf{\Sigma}^*)$ so $\dim(V)=D_{\mathbf{\Sigma}^*}$. Since the locus $\mathcal{O}\subset \widehat{\underline{X}}(\mathbf{\Sigma})$ of such $\widehat{\rho}$ is open dense, some of these irreducible representations $r'$ must have their classical shadows in the Azumaya locus of $\mathcal{S}_A(\mathbf{\Sigma})$. Therefore the PI-degree of  $\mathcal{S}_A(\mathbf{\Sigma})$ is $D_{\mathbf{\Sigma}^*}=N^{3g-3+n + 2|\mathcal{A}|}$. Lemma \ref{lemma_Skein1} implies that $\mathcal{S}_A(\mathbf{\Sigma})$ has the same rank over both
 $\underline{Z}'_{\mathbf{\Sigma}}$ and the center $\underline{Z}_{\mathbf{\Sigma}}$,  therefore $\underline{Z}'_{\mathbf{\Sigma}}=\underline{Z}_{\mathbf{\Sigma}}$.  So $\mathcal{O}$ is included into the Azumaya locus of $\mathcal{S}_A(\mathbf{\Sigma})$. 
\par Next suppose that there exists $i$ such that $\tr(\rho(\gamma_{p_i}))= \pm  2$ and $h_{p_i}\neq \mp 2$. Then $\widehat{\rho}^*$ does not belong to the Azumaya locus of $\overline{\mathcal{S}}_A(\mathbf{\Sigma}^*)$ so $\dim(V)< D_{\mathbf{\Sigma}^*}$. This implies that $\widehat{\rho}$ does not belong to the Azumaya locus of  $\mathcal{S}_A(\mathbf{\Sigma})$. 
\par Eventually consider $\widehat{\rho} \in \underline{\widehat{X}}(\mathbf{\Sigma})$ such that $\mu(\rho)\in \SL_2^{1} \times \ldots \times \SL_2^1$. We can consider an irreducible representation $r: \overline{\mathcal{S}}_A(\mathbf{\Sigma}) \to \End(V)$ such that the composition $r': \mathcal{S}_A(\mathbf{\Sigma})\to \overline{\mathcal{S}}_A(\mathbf{\Sigma}) \xrightarrow{r} \End(V)$ has classical shadow $\widehat{\rho}$. Since $D_{\mathbf{\Sigma}}<D_{\mathbf{\Sigma}^*}$ then $\dim(V)< D_{\mathbf{\Sigma}^*}$ so $\widehat{\rho}$ does not belong to the Azumaya locus of $\mathcal{S}_A(\mathbf{\Sigma})$. This concludes the proof. 
 
 \end{proof}
 
 \subsection{Quantum moduli algebras and lattice gauge field theory}

  In the particular case where $\mathbf{\Sigma}=\mathbf{\Sigma}(g,n):=(\Sigma_{g,n+1}, \{a\})$ is a genus $g\geq 0$ surface with $n+1$ boundary components and exactly one of them has a single boundary arc $a$ whereas the others are inner punctures, the algebra $\mathcal{L}_{g,n}:= \mathcal{S}_A(\mathbf{\Sigma})$ is called the  \textit{\QMA}. The algebra $\mathcal{L}_{g,n}$ appeared in literature in various contexts.
\begin{enumerate}
\item They were first defined independently   in \cite{AlekseevGrosseSchomerus_LatticeCS1,AlekseevGrosseSchomerus_LatticeCS2, AlekseevSchomerus_RepCS} and \cite{BuffenoirRoche, BuffenoirRoche2} where they appeared as deformation quantization of the  Fock-Rosly \cite{FockRosly} and Alekseev-Kosman-Malkin-Meinrenken \cite{AlekseevMalkin_PoissonLie, AlekseevKosmannMeinrenken, AlekseevMalkin_PoissonCharVar} moduli spaces (see also \cite{KojuTriangularCharVar}). Under this approach, they were intensively studied in \cite{BaseilhacRoche_LGFT1, BaseilhacRoche_LGFT2, BaseilhacFaitgRoche_LGFT3}.
\item They were rediscovered independently by Habiro under the name \textit{quantum representation variety} in \cite{Habiro_QCharVar}.
\item As we just defined, they are particular cases of stated skein algebras.
\item They eventually appeared under the name \textit{internal skein algebras} in the work of Ben Zvi-Brochier-Jordan \cite{BenzviBrochierJordan_FactAlg1, BenzviBrochierJordan_FactAlg2} and further studied  in \cite{Cooke_FactorisationHomSkein, GunninghamJordanSafranov_FinitenessConjecture, GanevJordanSafranov_FrobeniusMorphism}.
\end{enumerate}
The equivalence between \QMAs and stated skein algebras is proved in \cite{BullockFrohmanKania_LGFT,Faitg_LGFT_SSkein, KojuPresentationSSkein}. The equivalence between internal skein algebras and \QMAs is proved in \cite{BenzviBrochierJordan_FactAlg1}. The equivalence between internal skein algebras and stated skein algebras is proved in \cite{Haioun_Sskein_FactAlg}. The equivalence between quantum representation varieties and stated skein algebras is proved in \cite{KojuMurakami_QCharVar}.

\vspace{2mm}
\par By definition $\mathbf{\Sigma}(g,n)^*=(\Sigma_{g,n+1}, \{a', a''\})$ is the surface $\Sigma_{g,n+1}$ with two boundary arcs $\{a', a''\}$ in the same boundary component. Note that  $\mathbf{\Sigma}(g,n)^*$ is obtained from $g$ copies of $\mathbf{\Sigma}(1,0)^*$ and $n$ copies of $\mathbf{\Sigma}(0,1)^*$ by gluing some pairs of boundary arcs. Therefore one has a splitting morphism 
$$\theta: \overline{\mathcal{S}}_A(\mathbf{\Sigma}(g,n)^*) \hookrightarrow \overline{\mathcal{S}}_A(\mathbf{\Sigma}(1,0)^*)^{\otimes g} \otimes \overline{\mathcal{S}}_A(\mathbf{\Sigma}(0, 1)^*)^{\otimes n}.$$
 Note that $\mathbf{\Sigma}(0, 1)^*=\mathbb{D}_1$ so $\overline{\mathcal{S}}_A(\mathbf{\Sigma}(0, 1)^*)\cong U_q\mathfrak{gl}_2$. On the other hand, $\mathcal{HH}_q:=\mathcal{S}_A(\mathbf{\Sigma}(1, 0)^*)$ is the so-called \textit{Heisenberg double} algebra. To see this, let $\alpha, \beta$ be the arcs of Figure \ref{fig_D1} and consider the matrices
 $$N(\alpha):= \begin{pmatrix} \alpha_{++} & \alpha_{+-} \\ \alpha_{-+} & \alpha_{--} \end{pmatrix}, N(\beta):=  \begin{pmatrix} \beta_{++} & \beta_{+-} \\ \beta_{-+} & \beta_{--} \end{pmatrix}, \mathscr{R}:=\begin{pmatrix} A & 0 & 0 & 0 \\ 0 & 0 &A^{-1} & 0 \\ 0 & A^{-1} & A-A^{-3} & 0 \\ 0 & 0 & 0 & A \end{pmatrix} .$$
  By \cite[Theorem $1.1$]{KojuPresentationSSkein},  $\mathcal{HH}_q$ is generated by the elements $\alpha_{ij}, \beta_{kl}$ with relations $\det_q(N(\alpha))=\det_q(N(\beta))=1$ (where $\det_q \begin{pmatrix} a & b \\ c& d\end{pmatrix}=ad-q^{-1}bc$) and 
 $$ N(\alpha)\odot N(\alpha) = \mathscr{R}^{-1} (N(\alpha)\odot N(\alpha))\mathscr{R}, N(\beta)\odot N(\beta) = \mathscr{R}^{-1} (N(\beta)\odot N(\beta))\mathscr{R}, N(\alpha)\odot N(\beta) = \mathscr{R} (N(\alpha)\odot N(\beta))\mathscr{R}.$$
 Here $\odot$ denotes the Kronecker product and we recognize here Alekseev's relations defining the Heisenberg double. Therefore $\overline{\mathcal{HH}}_q:= \overline{\mathcal{S}}_A(\mathbf{\Sigma}(1, 0)^*)$ is the quotient of $\mathcal{HH}_q$ by the ideal generated by the two bad arcs.
 
     \begin{figure}[!h] 
\centerline{\includegraphics[width=3cm]{HeisenbergDouble.eps} }
\caption{The marked surface $\mathbf{\Sigma}(1,0)^*=(\Sigma_{1,1}, \{a', a''\})$ defining the Heisenberg double $\mathcal{HH}_q$.} 
\label{fig_D1}
\end{figure} 

\par Consider the composition: 
$$ \Phi: \mathcal{L}_{g,n} \xrightarrow{j} \overline{\mathcal{S}}_A(\mathbf{\Sigma}(g,n)^*) \xrightarrow{\theta} (\overline{\mathcal{HH}}_q)^{\otimes g} \otimes (U_q\mathfrak{gl}_2)^{\otimes n}.$$
An analogue of the map $\Phi$ appeared in Alekseev's work \cite{Alekseev_AlekseevMorphism} and was named \textit{Alekseev's morphism} in \cite{BaseilhacFaitgRoche_LGFT3}. Given $V_1, \ldots, V_n$ and $W_1, \ldots, W_g$ some semi-weight indecomposable representations of $U_q\mathfrak{gl}_2$ and $\overline{\mathcal{HH}}_q$ respectively, using the morphism $\Phi$, we endow $W_1\otimes \ldots \otimes W_g \otimes V_1 \otimes \ldots \otimes V_n$ with a structure of module over $\mathcal{L}_{g,n}$.
Theorem \ref{main_theorem} implies that every reduced $\mathcal{L}_{g,n}$ module is isomorphic to a direct sum of such  modules and proves that the representations $W_i$ are in $1:1$ correspondance with the character over the center of $\overline{\mathcal{HH}}_q$ and gives a complete classification for the modules $V_i$. Therefore we obtain an explicit construction of the reduced representations of the \QMAs. 
The map $\mu: \underline{X}(\mathbf{\Sigma})\to \SL_2$ sending $\rho$ to $\rho(\alpha_{\partial})$ was named the \textit{moment map} in  \cite{AlekseevKosmannMeinrenken}.
Theorem \ref{theorem_Skein} implies

\begin{corollary}\label{coro_QMA}
\begin{enumerate}
\item The center of $\mathcal{L}_{g,n}$ is generated by the image of the Frobenius and the peripheral curves $\gamma_{p_1}, \ldots, \gamma_{p_n}$.
\item The PI-degree of  $\mathcal{L}_{g,n}$ is $N^{3g+n}$.
\item The Azumaya locus of $\mathcal{L}_{g,n}$ is the locus of elements $\widehat{\rho}=(\rho, h_{p_i})$ such that $\mu(\rho)\in \SL_2^0$ and $\tr(\rho(\gamma_{p_i}))=\pm 2 \Rightarrow h_{p_i}=\mp 2$.
\end{enumerate}
\end{corollary}

When $n=0$, Corollary \ref{coro_QMA} was proved in \cite[Theorem $1$]{GanevJordanSafranov_FrobeniusMorphism}. When $g=0$, the first point of Corollary \ref{coro_QMA} was proved in \cite[Theorem $1.1$]{BaseilhacRoche_LGFT1}, while the second point was proved in \cite[Theorem $1.3$]{BaseilhacRoche_LGFT2}. 
\par Let us end by a discussion on the classification of representations of $\mathcal{L}_{g,n}$. Let $r: \mathcal{L}_{g,n}\to \End(V)$ be an indecomposable semi-weight representation and consider the only bad arc $\alpha(p)_{-+}\in \mathcal{L}_{g,n}$.
\begin{enumerate}
\item If $r(\alpha(p)_{-+})$ is invertible, then $r$ is isomorphic to a reduced representation, i.e. it factorizes through $\overline{\mathcal{S}}_A(\mathbf{\Sigma}^*)$. These representations are completely classified by Theorem \ref{main_theorem} and the skein Alekseev morphism $\Phi$ permits an explicit construction of them as explained before. Note that every irreducible representation whose classical shadow belongs to the Azumaya locus belongs to this class.
\item If $r(\alpha(p)_{-+})=0$, $r$ factorizes through the reduced stated skein algebra $\overline{\mathcal{S}}_A(\mathbf{\Sigma})$. When $n=0$, by \cite{KojuMCGRepQT} the latter algebra is Azumaya so these representations are classified. When $n\geq 1$, the techniques of the present paper are insufficient to classify such representations though we conjecture that Theorem \ref{main_theorem} still holds in this case. 
\item The last class of representations, for which $r(\alpha(p)_{-+})^N=0$ but $r(\alpha(p)_{-+})\neq 0$, seems very difficult to classify. Such a classification might even be an unsolvable problem. 
\end{enumerate}
 
 \appendix 
 
 \section{Classification of semi-weight $U_q \mathfrak{gl}_2$ modules}\label{appendix}
 
In this appendix, we prove Theorem  \ref{theorem_representations_QG}. Denote by $\mathcal{C}$ the category of finite dimensional weight $U_q\mathfrak{gl}_2$ modules and by $\overline{\mathcal{C}}$ its category of semi-weight modules. For $0\leq n \leq N-1$, we write $\overline{n}:= N-n-2$.
 
 \subsection{Representations in the fully Azumaya locus}
 
  \begin{lemma}\label{lemm1}
\begin{enumerate}
\item 
  For $0\leq n \leq N-2$, we have a non-split exact sequence 
\begin{equation}\label{suitexacte}
 0 \rightarrow S_{\mu, \varepsilon, \overline{n}} \xrightarrow{i} V(\varepsilon \mu A^n, \mu A^{-n}, 0, 0) \xrightarrow{p} S_{\mu, \varepsilon, n} \rightarrow 0, 
 \end{equation}
where the equivariant maps $i$ and $p$ are defined by $i(e_j):= v_{j+n+1}$ for $0\leq j  \leq N-n-2$ and $p(v_j):= \left\{ \begin{array}{ll} e_j &\mbox{, if }0\leq i \leq n; \\ 0 & \mbox{, else.} \end{array} \right.$ 
 \item One has non-split  exact sequences:
$$ 0 \to V(\varepsilon \mu A^{-2-n}, \mu A^{2+n}, 0, 0) \xrightarrow{\iota} P_{\mu, \varepsilon, n} \xrightarrow{p} V(\varepsilon \mu A^n, \mu A^{-n}, 0, 0) \to 0, $$
$$ 0 \to V(\lambda, \mu, a,b)  \xrightarrow{\iota} P(\lambda, \mu, a,b)  \xrightarrow{p}  V(\lambda, \mu, a,b) \to 0,$$
and 
$$0 \to \widetilde{V}(\lambda, \mu, c)  \xrightarrow{\iota} \widetilde{P}(\lambda, \mu, c)  \xrightarrow{p}  \widetilde{V}(\lambda, \mu, c)\to 0, $$
 where $\iota(v_i):= x_i$ and $p(x_i):=0$, $p(y_i):=v_i$ (resp. $\iota(w_i)=x_i$, $p(x_i)=0$, $p(y_i)=w_i$ in the third case). 
 \end{enumerate}
 \end{lemma}
 
 \begin{proof} The proof is a straightforward verification left to the reader. \end{proof}
 
 For $(\lambda, \mu, b) \in \mathbb{C}^*$ and $a \in \mathbb{C}$, the module $P(\lambda, \mu, a,b)$ of Definition \ref{def_semiweight_modules} is always well defined. Also for $c\in \mathbb{C}^*$, $\widetilde{P}(\lambda, \mu, c)$  is well defined as well. However they are not always semi weight modules.

\begin{lemma}\label{lemm2}
\begin{enumerate}
\item $P(\lambda, \mu, a,b)$ is a semi weight module if and only if 
$$ h_p:= -(q-q^{-1})^2(\lambda\mu)^{-1}ab-\lambda \mu^{-1}q - \mu\lambda^{-1}q^{-1} = \pm (q^n+q^{-n}) \quad \mbox{ for some }n\in \{1, \ldots, (N-1)/2\}$$
\item $\widetilde{P}(\lambda, \mu, c)$  is a semi weight module if and only if  $\lambda \mu^{-1}= \pm q^n$ for some $n\in \{1, \ldots, (N-1)/2\}$.
\item $P_{\lambda, \varepsilon, n}$ is always a semi weight module.
\end{enumerate}
\end{lemma}

\begin{proof}
 Denote by $\rho : U_q\mathfrak{gl}_2\to \End (P(\lambda, \mu, a,b))$ the associated representation. Clearly we have $\rho(K^{N/2})= \lambda^N \id$, $\rho(L^{N/2})=\mu^N \id$, $\rho(F^N)=b \id$. 
Let $V:= \Span(x_i, i=0, \ldots, N-1) \subset P(\lambda, \mu, a,b)$ such that $V\cong V(\lambda, \mu, a, b)$ and denote by $\widehat{x}=(g,h_p,h_{\partial})$ its classical shadow. 
In particular: 
$$ h_p:= -(q-q^{-1})^2(\lambda\mu)^{-1}ab-\lambda \mu^{-1}q - \mu\lambda^{-1}q^{-1}$$
and  $\restriction{\rho}{V}(\gamma_p) = h_p \id_V$. Write
$$ e:= -\frac{(\lambda \mu)^N}{(q-q^{-1})^{2N}} \prod_{i\in \mathbb{Z}/N\mathbb{Z}} (h_p +\lambda \mu^{-1} q^{1-2i} + \mu\lambda^{-1}q^{2i-1}).$$
Clearly $\restriction{\rho(E^N)}{V}= e \id_V$. Let us prove that $\rho(E^N)=e \id$ if and only if $h_p \in \{ \pm( q^n+q^{-n}), n\in \{1, \ldots, N-1\} \}$. Write 
$$ e_i(X):= -\frac{\lambda \mu}{(q-q^{-1})^2} (X+ \lambda \mu^{-1} q^{1-2i} + \mu \lambda^{-1}q^{2i-1})$$
and $P(X):= \prod_{i \in \mathbb{Z}/N\mathbb{Z}} e_i(X) - e$ so that $P(h_p)=0$. Using the equality
$$ \rho(E) y_i= e_i(h_p)y_{i-1} +x_{i-1}, $$
we find that 
$$ \rho(E^N)y_i= e y_i + P'(h_p) x_i.$$
So $\rho(E^N)=e \id $ if and only if $h_p$ is a zero of $P(X)$ of multiplicity $>1$. On the other hand, since the minimal polynomial of $\restriction{\rho}{V}(\gamma_p)$ is $T_N(X)+\tr(\varphi(g))$ and since $P(\restriction{\rho}{V}(\gamma_p))=0$, we have $P(X)=-\frac{(\lambda \mu)^N}{(q-q^{-1})^{2N}} (T_N(X)+ \tr(\varphi(g)))$ so $h_p$ is a root or $P(X)$ of multiplicity $>1$ if and only if $h_p \in \{ \pm( q^n+q^{-n}), n\in \{1, \ldots, N-1\} \}$. 
\par The cases of  $\widetilde{P}(\lambda, \mu, c)$ and $P_{\lambda, \varepsilon, n}$  are done similarly (and much easier) and left to the reader.
\end{proof}

 
 
 \begin{lemma}\label{lemm3} 
 \begin{enumerate}
 \item The modules $S_{\mu, \varepsilon, n}, V(\lambda, \mu, a,b), \widetilde{V}(\lambda, \mu, c), P(\lambda, \mu, a,b), \widetilde{P}(\lambda, \mu, c), P_{\mu, \varepsilon, n}$ are indecomposable.
 \item The modules $S_{\mu, \varepsilon, n}$ are simple. The module $\widetilde{V}(\lambda, \mu, c)$ is simple if and only if either $c\neq 0$ or $\lambda \mu^{-1} \neq \pm q^{n-1}$ for all $n\in \{1, \ldots, N-1\}$. The module $V(\lambda, \mu, a, b)$ is simple if and only if either $\prod_{i\in \mathbb{Z}/N\mathbb{Z}} (ab+\frac{q^{1-i}\lambda^2 - q^{i-1}\mu^2}{q-q^{-1}}[i])\neq 0$ or $\lambda \mu^{-1} \neq \pm q^{n-1}$ for all $n\in \{1, \ldots, N-1\}$. 
  \end{enumerate}
\end{lemma}

\begin{proof} Let $V$ be one module of the form $S_{\mu, \varepsilon, n}, V(\lambda, \mu, a,b), \widetilde{V}(\lambda, \mu, c)$ and denote by $(u_i)_i$ its canonical basis (\textit{i.e.} $u_i$ is either $e_i$, $v_i$ or $w_i$ if $V$ has the form $S_{\mu, \varepsilon, n}, V(\lambda, \mu, a,b)$ or $\widetilde{V}(\lambda, \mu, c)$ respectively). Write $\rho : U_q\mathfrak{gl}_2 \rightarrow \End(V)$ the morphism associated to the module structure. Let $\End_{\mathcal{C}}(V)$ be the space of equivariant endomorphisms of $V$ and fix $0 \neq \theta \in \End_{\mathcal{C}}(V)$. By Definition \ref{def_QGRep}, the operator $\rho(K^{1/2})$ is diagonal in the basis $(u_i)_i$ and each eigenvalue arises with multiplicity one. Since $\theta$ commutes with $\rho(K^{1/2})$, it preserves the eigenspace $\mathbb{C}u_0$, hence there exists $z\in \mathbb{C}$ such that $\theta \cdot u_0= z u_0$. First suppose that $z\neq 0$. By definition, the vectors of the canonical basis are characterized by either $u_i= \rho(F)^i u_0$ or $u_i= \rho(E)^i u_0$. Since $\theta$ commutes with both $\rho(E)$ and $\rho(F)$, one has $\theta u_i= z u_i$ and $\theta$ is scalar. Next suppose that $\theta u_0=0$. Since $\theta\neq 0$, there exists an index $i_0$  and  $z\in \mathbb{C}^*$, such that $\theta u_{i_0}=z u_{i_0}$ and $\theta u_i = 0$ for all $0\leq i \leq i_0-1$. Let $W:= \Span \left( u_j, j\geq i_0 \right)\subset V$ and denote by $P_W$ the projection onto $W$ parallel to $\Span\left( u_j, 0\leq j \leq i_0 \right)$. Since the vectors $u_j$, for $j\geq i_0$ are obtained from $u_{i_0}$ by composition with some power of either $\rho(E)$ or $\rho(F)$, the equivariance of $\theta$ implies that $\theta= z P_W$. 
\vspace{2mm}
\par 
Conversely, if there exists such an equivariant projector $P_W$, then one has either $\rho(E)u_{i_0}=0$, if $V$ has the form $V_{\mu, \varepsilon, n}$ or $V(\lambda, \mu, a,b)$, and $\rho(F) u_{i_0}= 0$, if $V$ has the form $\widetilde{V}(\lambda, \mu, c)$. By Definition \ref{def_QGRep}, such an index $i_0$ does not exist for the modules $S_{\mu, \varepsilon, n}$, neither for the modules  $\widetilde{V}(\lambda, \mu, c)$ such that either $c\neq 0$ or $\lambda \mu^{-1} \neq \pm q^{n-1}$ for all $n\in \{1, \ldots, N-1\}$, nor for the modules $V(\lambda, \mu, a, b)$ such that either  $\prod_{i\in \mathbb{Z}/N\mathbb{Z}} (ab+\frac{q^{1-i}\lambda^2 - q^{i-1}\mu^2}{q-q^{-1}}[i])\neq 0$ or $\lambda \mu^{-1} \neq \pm q^{n-1}$ for all $n\in \{1, \ldots, N-1\}$. Hence in each of these cases, one has $\End_{\mathcal{C}}(V)= \mathbb{C} \id$ and these modules are simple. In the other cases, one finds that such an index $i_0$ is unique and by Lemmas \ref{lemm1},  the corresponding projector $P_W$ is equivariant. Hence  $\End_{\mathcal{C}}(V)=\id \oplus P_W$. Moreover in this case, $V$ is isomorphic to a module $V(\varepsilon \mu A^n, \mu A^{-n}, 0, 0)$
and since the exact sequence \eqref{suitexacte} in Lemma \ref{lemm1} does not split, $V$ is indecomposable. Similarly, the fact that the modules $P(\lambda, \mu, a,b), \widetilde{P}(\lambda, \mu, c)$ are indecomposable follows from the existence of the non-split exact sequences in Lemma \ref{lemm1}.
\par 
Let us prove that  $P_{\mu, \varepsilon, n}$ is indecomposable.  Let $\pi:= \rho(C - \frac{\mu^2}{q-q^{-1}} [n+1]) \in \End_{\mathcal{C}}(P_{\mu, \varepsilon, n})$ be the nilpotent map  such that   $\pi(x_j)=0$,  $\pi (y_i)=x_{N-n-i}$ for $i\leq n-1$, $\pi(y_i)=0$ for $i\geq n$ , and let us prove that $\End_{\mathcal{C}}(P_{\mu, \varepsilon, n})=\id \oplus \pi$. This will imply that $\End_{\mathcal{C}}(P_{\mu, \varepsilon, n})\cong \quotient{\mathbb{C}[\pi]}{(\pi^2)}$ is a local ring, thus that  $P_{\mu, \varepsilon, n}$ is indecomposable.
Let  $\theta \in \End_{\mathcal{C}}(P_{\mu, \varepsilon, n})$. Then $\theta(y_0)$ is an eigenvector of $K^{1/2}$ with eigenvalue $\varepsilon\mu A^n$ and is annihilated by $(FE)^2$, so there exists $c,c'\in \mathbb{C}$ such that $\theta(y_0)=cy_0+c' x_{N-n}$. Let us prove that $\theta= c\id + c' \pi$. First, since $\theta(F^i y_0) = F^i \theta(y_0)$, one finds:
$$ \theta(y_i) = \left\{ \begin{array}{ll}
cy_i + c' x_{N-n+i} & \mbox{, if }i\leq n-1; \\
cy_i & \mbox{ if }i\geq n.
\end{array} \right. 
= (c\id +c'\pi) (y_i).$$
Next, since $\theta(Ey_0)=E\theta(y_0)$ one finds $\theta(x_{N-n-1})=cx_{N-n-1}= (c\id +c' \pi) x_{N-n-1}$. Since $\theta(F^i x_{N-n-1})=F^i \theta(x_{N-n-1})$, one gets $\theta(x_i)=cx_i = (c\id+c' \pi)(x_i)$ for $i\geq N-n-1$ and since $\theta(E^i x_{N-n-1})= E^i \theta(x_{N-n-i})$ one finds that $\theta(x_i)=cx_i= (c\id +c' \pi)x_i$ for $i\leq N-n-1$. Thus $\theta= c\id +c' \pi$. 




\end{proof}

\begin{lemma}\label{lemm4} \begin{enumerate}
\item The Azumaya locus $\mathcal{AL}(\mathbb{D}_1)$ is the set of elements $(g,h_p,h_{\partial})$ such that either $\varphi(g)\neq \pm \mathds{1}_2$ or $\varphi(g)=\pm \mathds{1}_2$ and $h_p=\mp 2$. 
\item The fully Azumaya locus $\mathcal{FAL}(\mathbb{D}_1)$ is the set of elements $g$ such that $\varphi(g)\neq \pm \mathds{1}_2$.
\end{enumerate}
\end{lemma}

\begin{proof} The second assertion is an immediate consequence of the first one. To compute  $\mathcal{AL}(\mathbb{D}_1)$, recall from Theorem \ref{theorem_AL} that an irreducible  representation $\rho : U_q\mathfrak{gl}_2\to \End(V)$ with classical shadow $\widehat{x}$ has dimension $N$ if and only $\widehat{x} \in \mathcal{AL}(\mathbb{D}_1)$. When $\widehat{x}= (g,h_p,h_{\partial})$ is such that  $\varphi(g)\neq \pm \mathds{1}_2$, then it is the classical shadow of an $N$-dimensional representation of the form $V(\lambda,\mu,a,b)$ or $\widetilde{V}(\lambda, \mu, c)$ which is irreducible by Lemma \ref{lemm3}. When $\varphi(g)=\pm \mathds{1}_2$ and $h_p=\mp 2$, then $\widehat{x}$ is the shadow of an $N$ dimensional representation $S_{\mu, \varepsilon, N-1}$ which is irreducible by Lemma \ref{lemm3}. So $\widehat{x} \in \mathcal{AL}(\mathbb{D}_1)$ in these cases. If $\varphi(g)=\pm \mathds{1}_2$ and $h_p\neq \mp 2$, then $\widehat{x}$ is the shadow of a representation $S_{\mu, \varepsilon, n}$ with $0\leq n \leq N-2$ whose dimension if $<N$ and which is irreducible by Lemma \ref{lemm3}, so $\widehat{x} \notin \mathcal{AL}(\mathbb{D}_1)$.

\end{proof}

 \begin{lemma}\label{lemm5} Let $\widehat{x}=(g,h_p,h_{\partial}) \in \widehat{X}(\mathbb{D}_1)$. 
 \begin{enumerate}
 \item If $\tr( \varphi(g))\neq \pm 2$, there exists (up to isomorphism) a unique indecomposable semi-weight representation with classical shadow $\widehat{x}$; it is of the form $V(\lambda, \mu, a,b)$ or $\widetilde{V}(\lambda, \mu, c)$.
 \item If $\tr( \varphi(g))= \pm 2$ and $\varphi(g)\neq \mp \mathds{1}_2$, there exist  (up to isomorphism) two indecomposable semi-weight representations with classical shadow $\widehat{x}$; one is of the form $V(\lambda, \mu, a,b)$ or $\widetilde{V}(\lambda, \mu, c)$ and the other has the form $P(\lambda, \mu, a,b)$ or $\widetilde{P}(\lambda, \mu, c)$.
 \end{enumerate}
 \end{lemma}
 
 \begin{proof} When $\varphi(g)\neq \pm \mathds{1}_2$, then $g\in \mathcal{FAL}(\mathbb{D}_1)$ by Lemma \ref{lemm4}. A simple computation shows that 
 $$ Z(g)\cong \left\{ \begin{array}{ll}
 \mathbb{C}^{\oplus N^2} & \mbox{, if }\tr(\varphi(g))\neq \pm 2; \\
 \mathbb{C}^{\oplus N+1}\oplus \left( \quotient{\mathbb{C}[X]}{(X-1)^2} \right)^{\oplus N(N-1)/2}& \mbox{, if }\tr(\varphi(g))= \pm 2.
 \end{array} \right.$$
  By Corollary \ref{coro_FAL}, if $\tr(\varphi(g))\neq \pm2$, one has $N^2$   semi-weight indecomposable representations with shadow $g$ which are all weight representations: these are the representations  of the form $V(\lambda, \mu, a,b)$ or $\widetilde{V}(\lambda, \mu, c)$ whose shadow is $g$. If $\tr(\varphi(g))= \pm 2$, still by Corollary \ref{coro_FAL}, we have $N^2$ indecomposable weight representations with shadow $g$ (these are the representations  of the form $V(\lambda, \mu, a,b)$ or $\widetilde{V}(\lambda, \mu, c)$ whose shadow is $g$) and  $N(N-1)/2$  semi-weight indecomposable representations with shadow $g$ which are not weight: these are the ones of the form $P(\lambda,\mu ,a ,b)$ or $\widetilde{P}(\lambda, \mu, c)$ with shadow $g$;  so we have found them all.
  \end{proof}
 
 \subsection{Representations outside the fully Azumaya locus}
 
 For $\mathcal{A}$ an algebra, we denote by $\Indecomp(\mathcal{A})$ the set of isomorphism classes of indecomposable $\mathcal{A}$-modules.
 \begin{definition}(Small quantum group) The (simply connected) \textit{small quantum enveloping algebra} $\uq$ has generators $E,F, k^{\pm 1/2}$ and relations: 
 $$
Ek^{1/2}= q^{-1}k^{1/2}E; \quad   Fk^{1/2}=qk^{1/2}F; \quad F^N=E^N=k^{N/2}-1=0, \quad EF-FE= \frac{k-k^{-1}}{q-q^{-1}}. 
$$
 \end{definition}
 
Let $\rho: U_q\mathfrak{gl}_2 \to \End(V)$ be an indecomposable semi weight representation whose shadow $g\in X(\mathbb{D}_1)$ is not in the fully Azumaya locus. By Lemma \ref{lemm4}, $g=\left( \varepsilon \begin{pmatrix} \Lambda & 0 \\ 0 & \Lambda^{-1} \end{pmatrix},  \begin{pmatrix} \Lambda & 0 \\ 0 & \Lambda^{-1} \end{pmatrix} \right)$ for some $\varepsilon \in \{-1, +1\}$  and $\Lambda \in \mathbb{C}^*$. 
Since $\rho$ is indecomposable,  there exists $h_{\partial}\in \mathbb{C}^*$ such that $\rho(H_{\partial})=h_{\partial}\id_V$. Let $\mu \in \mathbb{C}^*$ be the unique scalar such that $\Lambda= \mu^N$ and $h_{\partial}=\varepsilon \mu^{-2}$ and denote by $\Indecomp(U_q\mathfrak{gl}_2; \varepsilon, \mu) \subset \Indecomp(U_q\mathfrak{gl}_2)$ the subset of modules with shadow $g$ and $h_{\partial}$. Define a map 
$$ (\cdot)_{\mu, \varepsilon}: \Indecomp(\uq) \to \Indecomp(U_q\mathfrak{gl}_2; \varepsilon, \mu), \quad V\mapsto V_{\mu, \varepsilon}, $$
 by sending $\rho : \uq \to \End(V)$ to the representation $\rho_{\mu, \varepsilon}: U_q\mathfrak{gl}_2 \to \End(V_{\mu, \varepsilon})$ where $V_{\mu, \varepsilon}=V$ as a vector space and 
 $$ \rho_{\mu, \varepsilon}(K^{1/2})=\mu \varepsilon \rho(k^{1/2}), \rho_{\mu, \varepsilon}(L^{1/2})=\mu  \rho(k^{-1/2}), \rho_{\mu, \varepsilon}(E)=\mu^2 \rho(E), \rho_{\mu, \varepsilon}(F)= \rho(F).$$
 
 \begin{lemma}\label{lemm6} $ (\cdot)_{\mu, \varepsilon}: \Indecomp(\uq) \to \Indecomp(U_q\mathfrak{gl}_2; \varepsilon, \mu)$ is a bijection. \end{lemma}
 
 \begin{proof} 
 The construction of the inverse map is straightforward and left to the reader.
 \end{proof}
 
 \begin{notations} We denote by $S_n, P_n, \Omega^{\pm k}_n, M^k_n(\Lambda)$ the $\uq$ modules which correspond to the modules $S_{\mu, \varepsilon, n}, P_{\mu, \varepsilon, n}, \Omega^{\pm k}_{\mu, \varepsilon, n}, M^k_{\mu, \varepsilon, n}(\Lambda)$ respectively.
 \end{notations}
 
 The classification of the indecomposable $\uq$ modules was accomplished in \cite{Suter_Uqsl2Modules} (see also \cite{GSTF_KLlogCFT}) with two differences: $(1)$ in  \cite{Suter_Uqsl2Modules} the parameter $q$ is a root of unity of even order and $(2)$ in  \cite{Suter_Uqsl2Modules} the algebra considered has generators $E,F, k^{\pm 1}$ instead of $E,F, k^{\pm 1/2}$. In the rest of the appendix, we adapt the arguments in \cite{Suter_Uqsl2Modules} to our setting in order to prove the following 
 
 \begin{proposition}\label{prop_uq}
 \begin{enumerate}
 \item The indecomposable $\uq$ modules are the ones isomorphic to the modules  $S_n, P_n$, $\Omega^{\pm k}_n, M^k_n(\Lambda)$ and these modules are pairwise inequivalent.
 \item The irreducible $\uq$ modules are the ones isomorphic to the modules $S_n$.
 \item The projective $\uq$ modules are the ones isomorphic to the modules $S_{N-1}, P_n$.
 \end{enumerate}
 \end{proposition}
 
 The proof of Proposition \ref{prop_uq} is a straightforward adaptation of the arguments in \cite{Suter_Uqsl2Modules} so we reproduce them briefly and refer the reader to \cite{Suter_Uqsl2Modules} for more details. The first step is to decompose $\uq$ into a direct sum of indecomposable left ideals. Write $P_{N-1}:=S_{N-1}$. For $n\in \mathbb{Z}/N\mathbb{Z}$, set 
 $$\varphi_n:= \sum_{i\in \mathbb{Z}/N\mathbb{Z}}q^{-ni}k^{i/2}$$
 so $k^{1/2}\varphi_n= \varphi_n k^{1/2} = q^n \varphi_n$, $E\varphi_n=\varphi_{n+1}E$ and $F\varphi_n= \varphi_{n-1}F$. An easy application of the diamond Lemma for PBW bases shows that $\uq$ admits the following basis
 $$\{ F^a \varphi_n E^b, 0\leq a,b,n \leq N-1\}.$$
 In particular, if $x\in \uq$ is such that $Fx=0$ then there exists a unique $y\in \Span(F^a \varphi_nE^b, a\leq N-2)$ such that $Fy=x$. For $0\leq n \leq N-2$, applying this remark to $x_n:=E^{\overline{n}}F^{N-1}\varphi_n$ we obtain an element $\gamma_n \in \Span(F^a \varphi_nE^b, a\leq N-2)$ uniquely characterized by the fact that $F\gamma_n=x_n$.  Set $\gamma_{N-1}:= F^{N-1} \varphi_{N-1}$ and for $0\leq n \leq N-1$ set
 $$ P_n(0):= \uq \cdot \gamma_n, \quad P_n(h)= P_n(0)\cdot E^h \quad h\in \{0, \ldots, n\}.$$
 
 
 \begin{lemma}\label{lemm7} The algebra $\uq$ admits the direct sum decomposition into left ideals 
 $$ \uq =  \oplus_{n=0}^{N-1}\oplus_{h=0}^n P_n(h)$$
 where $P_n(h)$ is isomorphic to $P_n$ as a left module. Moreover  the indecomposable projective $\uq$ are the ones isomorphic to $P_n$.
 \end{lemma}
 
 \begin{proof} Clearly the $P_n(h)$ are left ideals by definition and  have trivial pairwise intersection. The fact that their direct sum is the whole $\uq$ follows by dimension counting. The fact that $P_n(h)$ is isomorphic to $P_n$ is a straightforward verification left to the reader. Since the $P_n$ are summands of the free rank $1$ $\uq$ module, they are projective. Conversely, we easily see that the tensor product $P_a\otimes P_b$ decomposes as a direct sum of modules $P_n$, so any indecomposable summand of a free $\uq$ module is isomorphic to a $P_n$. 
 \end{proof} 
 
 Set 
 $$ \mathbb{B}_{N-1}:= \oplus_{h=0}^{N-1}P_{N-1}(h), \quad \mathbb{B}_n:= \left( \oplus_{h=0}^n P_n(h) \right) \oplus \left( \oplus_{h'=0}^{\overline{n}} P_{\overline{n}}(h')\right).$$
 We obtain a decomposition 
 $$\uq = \mathbb{B}_{N-1}\oplus \oplus_{n=0}^{(N-3)/2} \mathbb{B}_n $$
 where the $\mathbb{B}_n$ are bilateral ideals. So classifying the indecomposable representations of $\uq$ amount to classifying the indecomposable representations of each $\mathbb{B}_n$. Since $\mathbb{B}_{N-1}\cong \Mat_N(\mathbb{C})$, it admits a unique indecomposable representation which corresponds to $S_{N-1}=P_{N-1}$. Fix $0\leq n\leq (N-3)/2$ and consider the algebra
 $$\mathcal{B}:= \End_{\uq} (P_n \oplus P_{\overline{n}}).$$
 
 By Morita theory, the functor $(P_n\oplus P_{\overline{n}})\otimes_{\uq}\bullet : \mathcal{B}^{op}-\Mod \to \mathbb{B}_n-\Mod$ is an equivalence with quasi inverse $\Hom_{\mathbb{B}_n}(P_n\otimes P_{\overline{n}}, \mathbb{B}_n)\otimes_{\mathbb{B}_n} \bullet$. Recall from the proof of Lemma \ref{lemm3} that $\End_{\uq}(P_n)=\mathbb{C}id_{P_n} \oplus \mathbb{C} \pi_n$ where $\pi_n^2=0$. We easily see that $\Hom_{\uq}(P_n, P_{\overline{n}})=\mathbb{C}\alpha_n \oplus \mathbb{C} \beta_n$ where $\alpha_n, \beta_n$ are characterized by the fact that they send the cyclic vector $y_0\in P_n$ to $\alpha_n(y_0)=x_0\in P_{\overline{n}}$ and $\beta_n(y_0)=y_{n+1}\in P_{\overline{n}}$. Note that $\alpha_{\overline{n}}\beta_n= \beta_{\overline{n}}\alpha_n= \pi_n$,  
 so $\mathcal{B}$ is generated by the morphisms $\id_{P_n}, \id_{P_{\overline{n}}}, \alpha_n, \alpha_{\overline{n}}, \beta_n$ and $\beta_{\overline{n}}$. Consider the Kronecker algebra $\mathcal{K}:= \quotient{\mathbb{C}[X,Y]}{(X^2, Y^2)}$ and the embedding $j: \mathcal{B} \hookrightarrow \Mat_2(\mathcal{K})$ defined by 
 $$ j(\id_{P_n})= \begin{pmatrix} 1 & 0 \\ 0&0 \end{pmatrix}, j(\id_{P_{\overline{n}}})= \begin{pmatrix} 0& 0 \\ 0 & 1 \end{pmatrix}, j(\alpha_n)=\begin{pmatrix} 0 & 0 \\ X & 0 \end{pmatrix}, j_{\beta_n} = \begin{pmatrix} 0 & 0 \\ Y & 0 \end{pmatrix}, j(\alpha_{\overline{n}})= \begin{pmatrix} 0 & X \\ 0 & 0 \end{pmatrix}, j(\beta_{\overline{n}})= \begin{pmatrix} 0 & Y \\ 0 & 0 \end{pmatrix}.$$
 The transposition ${}^t : \Mat_2(\mathcal{K}) \to \Mat_2(\mathcal{K})^{op}$ restricts to an isomorphism $t: \mathcal{B} \xrightarrow{\cong} \mathcal{B}^{op}$.  Consider the $\Mat_2(\mathcal{K})-\mathcal{K}$ bimodule $\mathcal{K}^{\oplus 2}$. The functor $(\mathcal{K}^{\oplus 2})\otimes_{\mathcal{K}} \bullet: \mathcal{K}-\Mod \to \Mat_2(\mathcal{K})-\Mod$ is clearly an equivalence. 
\par 
 Consider the involutive automorphism $\Theta: \mathcal{B}\to \mathcal{B}$ defined by the conjugacy by $\begin{pmatrix} 0& 1 \\ 1 &0 \end{pmatrix} \in  \Mat_2(\mathcal{K})$, said differently $\Theta (\id_{P_n})=\id_{P_{\overline{n}}}$, $\Theta (\alpha_n)=\alpha_{\overline{n}}$ and $\Theta(\beta_n)= \beta_{\overline{n}}$. For $L$ a $\mathcal{B}$-module with module morphism $\rho: \mathcal{B}\to \End(L)$, let $\overline{L}$ be the $\mathcal{B}$-module with module structure $\overline{\rho} (x):= \rho(\Theta(x))$, $x\in \mathcal{B}$. Consider the two projectors $P:= \frac{1+\Theta}{2}$, $Q:= \frac{1-\Theta}{2}$  in $\Mat_2(\mathcal{K})$ so that $P+Q=\id$ and $PQ=0$. Given $L \in \Indecomp( \Mat_2(\mathcal{K}))$, we decompose it as $L=L^+\oplus L^-$ where $L^+:= PL$ and $L^-:= QL$ are two indecomposable $\mathcal{B}$ modules such that $L^-= \overline{L^+}$. We denote by $(\cdot)_+, (\cdot)_-: \Indecomp(\Mat_2(\mathcal{K})) \to \Indecomp(\mathcal{B})$ sending $L$ to $L_+$ and $L_-$ respectively. We can now define two maps $F_+, F_-: \Indecomp(\mathcal{K})\to \Indecomp(\mathbb{B}_n)$ as the compositions: 
 \begin{multline*}
 F_{+, -}: \Indecomp(\mathcal{K}) \xrightarrow[\cong]{ (\mathcal{K}^{\oplus 2})\otimes_{\mathcal{K}} \bullet} \Indecomp( \Mat_2(\mathcal{K})) \\
  \xrightarrow{ (\cdot)_{+, -}} \Indecomp(\mathcal{B}) \xrightarrow[\cong]{ t^*} \Indecomp(\mathcal{B}^{op}) \xrightarrow[\cong]{(P_n\otimes P_{\overline{n}})\otimes \bullet} \Indecomp(\mathbb{B}_n), 
 \end{multline*}
 and a map 
 $$ F:= F_+ \sqcup F_- : \Indecomp(\mathcal{K})\sqcup \Indecomp(\mathcal{K}) \to \Indecomp(\mathbb{B}_n).$$
 
 \begin{lemma}\label{lemm8} $ F: \Indecomp(\mathcal{K})\sqcup \Indecomp(\mathcal{K}) \to \Indecomp(\mathbb{B}_n)$ is a bijection.
 \end{lemma}
 
 \begin{proof} We need to prove that $(\cdot)_+ \sqcup (\cdot)_-: \Indecomp(\Mat_2(\mathcal{K}))\sqcup \Indecomp(\Mat_2(\mathcal{K})) \to \Indecomp(\mathcal{B})$ is a bijection or, equivalently, that the map $(\cdot)_{+,-} :\Indecomp(\Mat_2(\mathcal{K})) \to \quotient{\Indecomp(\mathcal{B})}{(M\sim \overline{M})}$ induced by both $(\cdot)_+$ and $(\cdot)_-$, is a bijection.  First, note that $\Mat_2(\mathcal{K})=\mathcal{B}\oplus \mathcal{B}\theta$ where $\theta= \begin{pmatrix} 0& 1 \\ 1 &0 \end{pmatrix}$.
 Let $M \in \Indecomp(\mathcal{B})$ with module structure $\rho: \mathcal{B} \to \End(M)$ and consider the $\Mat_2(\mathcal{K})$ representation $\widehat{\rho}: \Mat_2(\mathcal{K})\to \End(\widehat{M})$ where $\widehat{M}=M\oplus \overline{M}$ and 
 $$ \overline{\rho}(x) = \begin{pmatrix} \rho(x) & 0 \\ 0 & \overline{\rho}(x) \end{pmatrix}, \quad \overline{\rho}(x\theta)= \begin{pmatrix} 0 & \rho(x) \\  \overline{\rho}(x) & 0 \end{pmatrix}, \quad \mbox{ for }x\in \mathcal{B}.
 $$
 Then $\widehat{M} \in \Indecomp(\Mat_2(\mathcal{K}))$ and $(\widehat{M})_{+}=M$, $(\widehat{M})_-=\overline{M}$ so the application $M \mapsto \widehat{M}$ is the inverse of $(\cdot)_{+,-}$ which is thus a bijection.
 
 
 \end{proof}

It remains to prove that $\Indecomp(\mathcal{K})$ is in bijection with $\Delta$. This classification was accomplished by Kronecker in \cite{Kronecker_KroneckerAlgebra} and is easier to state using quiver representations in the framework of \cite{Kac_QuiverRep}. First, note that if $\rho: \Indecomp(\mathcal{K})\to \End(V)$ is an indecomposable representation with $\rho(XY)\neq 0$, then there exists $v\in V$ with $\rho(XY)v\neq 0$ and the subspace $\Span(v, \rho(X)v, \rho(Y)v, \rho(XY)v)\subset V$ is a $\mathcal{K}$ submodule isomorphic to $\mathcal{K}$ (seen as left module over itself). Since this module is free, it is projective and thus $V\cong \mathcal{K}$. 
In order to make the two modules $F_+(\mathcal{K})$ and $F_-(\mathcal{K})$ explicit, note that by Lemma \ref{lemm6}, the projection map $P_n\to S_n$ given by Lemma \ref{lemm1} (sending $y_0$ to $e_0$) is a  projective cover, so 
$$ \Ext^1(S_n, S_{\overline{n}})= \Hom_{\uq}(P_n, P_{\overline{n}})= \mathbb{C}\alpha_n \oplus \mathbb{C} \beta_n.$$
The extension of $S_n$ by $S_{\overline{n}}$ corresponding to $z_1 \alpha_n + z_2 \beta_n$ is the module $M_n(z_1,z_2)$ with basis $(v_i, i=0, \ldots N-1)$ and module structure:
\begin{align*}
{}& k^{1/2}v_i= A^{n-2i}v_i, Fv_{N-1}=0, F v_n= z_2 v_{n+1}, Fv_i=v_{i+1} \mbox{ for }i\neq n, N-1, \\ {}& Ev_0=z_1 v_{N-1}, Ev_i= [i][n-i+1]v_{i-1} \mbox{ for }i\neq 0.
\end{align*}
The extension is given by  the exact sequence
$$ 0 \to S_{\overline{n}} \xrightarrow{i} M_n(z_1,z_2) \xrightarrow{p} S_n \to 0$$
where $i(e_j)=v_{n+j+1}$, $p(v_i)=e_i$ if $i\leq n$ and $p(v_i)=0$ else. Note that $M_n(0,0)=S_{\overline{n}}\oplus S_n$ and that, for $(z_1,z_2)\neq (0,0)$, then $M_n(z_1,z_2)\cong M_n^1(\lambda)$ where $\lambda=z_1:z_2 \in \mathbb{CP}^1$.
We denote by $S_n \xrightarrow{ z_1 \alpha_n + z_2\beta_n} S_{\overline{n}}$ the extension $M_n(z_1,z_2)$. Then $\mathcal{K}$, seen as a $\mathcal{K}$ module over itself, decomposes as 

$$
\mathcal{K}=
\begin{tikzcd}
{} & \mathbb{C} \ar[ld, "X"] \ar[rd, "Y"'] & {} \\
\mathbb{C}X \ar[rd, "Y"] & {} & \mathbb{C}Y \ar[ld, "X"'] \\
{} & \mathbb{C} XY & {}
\end{tikzcd},  
$$
$$
\mbox{ so }
F_+(\mathcal{K})= 
\begin{tikzcd}
{} & S_n \ar[ld, "\alpha_n"] \ar[rd, "\beta_n"'] & {} \\
S_{\overline{n}} \ar[rd, "\beta_{\overline{n}}"] & {} & S_{\overline{n}} \ar[ld, "\alpha_{\overline{n}}"'] \\
{} & S_n & {}
\end{tikzcd} 
=P_n 
\mbox{ and }
F_-(\mathcal{K})= 
\begin{tikzcd}
{} & S_{\overline{n}} \ar[ld, "\alpha_{\overline{n}}"] \ar[rd, "\beta_{\overline{n}}"] & {} \\
S_n \ar[rd, "\beta_n"] & {} & S_n \ar[ld, "\alpha_n"] \\
{} & S_n & {}
\end{tikzcd} 
=P_{\overline{n}}.
$$

We now classify the representations of the quotient algebra $\quotient{\mathcal{K}}{(XY)}$. Consider the Kronecker quiver 
$Q:= \begin{tikzcd}
a \ar[r, bend left,  "X"] \ar[r, bend right, "Y"] &  b \end{tikzcd}
$ and let $\mathbb{C}Q$ be its path algebra; it is generated by the constant paths $1_a, 1_b$ and the two paths $X,Y$. Consider the embedding $i: \quotient{\mathcal{K}}{(XY)}\hookrightarrow \mathbb{C}Q$  defined by $i(X)=X$, $i(Y)=Y$ and $i(1)=1_a+1_b$. Recall that a representation $(\rho, V)$ of $Q$ is the data of two vector spaces $V_a$ and $V_b$ and two linear morphisms $\rho(X), \rho(Y): V_a \to V_b$ depicted as $\begin{tikzcd}
V_a \ar[r, bend left,  "\rho(X)"] \ar[r, bend right, "\rho(Y)"] &  V_b \end{tikzcd}
$ and that such a representation induces a $\mathbb{C}Q$-module structure on $V_a\oplus V_b$ in the obvious way such that one obtains an equivalence $\mathbb{C}Q-\Mod \cong \Rep(Q)$ (see \cite{Kac_QuiverRep} for details). The set of positive roots of $Q$ 
 is $\Delta_+(\widehat{\mathfrak{sl}}_2)=\{ (n,m) \in \mathbb{N}^2, (n-m)^2\leq 1\}$, the real positive  roots are the $(n,m)$ for which $(n-m)^2=1$ and the imaginary positive roots are the $(n,n), n\geq 1$. If $(\rho, V)$ is an indecomposable representation of $Q$, then the pair $(\dim(V_a), \dim(V_b))$ is a positive root and every positive roots (distinct from $(0,0)$) arise that way (see \cite{Kac_QuiverRep}). To each real positive root corresponds exactly one indecomposable representation and each imaginary positive root corresponds to several indecomposable representations: for the quiver $Q$, one has a moduli space of such representations parametrized by $\mathbb{CP}^1$. Let us now describe the indecomposable representations of $\mathcal{K}$, and thus of $\mathbb{B}_n$. 
 \par $(1)$ The real roots $(1,0)$ and $(0,1)$ correspond to the representations 
 $\begin{tikzcd} \mathbb{C} \ar[r, bend left,  "0"] \ar[r, bend right, "0"] &  0 \end{tikzcd}$ and  $\begin{tikzcd} 0 \ar[r, bend left,  "0"] \ar[r, bend right, "0"] &  \mathbb{C} \end{tikzcd}$ both induces the trivial representation $\mathbb{C}$ of $\mathcal{K}$ which itself induces the modules $F_+(\mathbb{C})=S_n$ and $F_-(\mathbb{C})=S_{\overline{n}}$.
 \par $(2)$ For $k\geq 1$, the real positive root $(k+1,k)$ corresponds to the representation 
 $$\begin{tikzcd} \mathbb{C}^{k+1} \ar[r, bend left,  "M_1"] \ar[r, bend right, "M_2"] &  \mathbb{C}^k \end{tikzcd}, \quad
  M_1= \begin{pmatrix} 1 & {} & 0 & 0\\ {} & \ddots &{} &  \vdots  \\ 0 & {} & 1 & 0 & \end{pmatrix}
 , \quad M_2=
 \begin{pmatrix} 0 & 1 & {} & 0 \\ \vdots & {} & \ddots & {} \\ 0 & 0 & {} & 1 \end{pmatrix}
 .$$
  The image by $F_+$ of this representation is the extension
  $$ \begin{tikzcd}
  S_n \ar[rd, "\beta_n"] & {} & S_n \ar[ld, "\alpha_n"] \ar[rd, "\beta_n"] & {} &S_n \ar[ld, "\alpha_n"]  & \ldots  \ar[rd, "\beta_n"] & {} &  S_n \ar[ld, "\alpha_n"] \\
   {} & S_{\overline{n}} & {} & S_{\overline{n}} & {} & \ldots & S_{\overline{n}} & {}
   \end{tikzcd}$$
 which is $\Omega^k_n$. Similarly, the image by $F_-$ is $\Omega^k_{\overline{n}}$.
 
 
 \par $(3)$ For $k\geq 1$, the real positive root $(k,k+1)$ corresponds to the representation 
 $$\begin{tikzcd} \mathbb{C}^{k} \ar[r, bend left,  "M_1"] \ar[r, bend right, "M_2"] &  \mathbb{C}^{k+1} \end{tikzcd}, \quad
  M_1= \begin{pmatrix} 1 & {} & 0 \\ {} & \ddots & {} \\ 0 & {} & 1 \\ 0 & \ldots & 0 \end{pmatrix}
 , \quad M_2=
 \begin{pmatrix}
 0 & \ldots & 0 \\ 1 & {} & 0 \\ {} & \ddots & {} \\ 0 & {} & 1 
 \end{pmatrix}
  .$$
   The image by $F_+$ of this representation is the extension
  $$ \begin{tikzcd}
  {} & S_n \ar[rd, "\beta_n"] \ar[ld, "\alpha_n"]& {} & S_n \ar[ld, "\alpha_n"] \ar[rd, "\beta_n"] & \ldots &  S_n \ar[rd, "\alpha_n"] &{} \\
    S_{\overline{n}} & {} & S_{\overline{n}} & {} & S_{\overline{n}} & \ldots & S_{\overline{n}} 
   \end{tikzcd}$$
 which is $\Omega^{-k}_n$. Similarly, the image by $F_-$ is $\Omega^{-k}_{\overline{n}}$.
 
 \par $(4)$ For $k\geq 1$ the imaginary positive root $(k,k)$ is the dimension of a family of indecomposable representations indexed by $\lambda \in \mathbb{CP}^1= \mathbb{C}\cup \{\infty\}$ defined by 
 $$\begin{tikzcd} \mathbb{C}^{k} \ar[r, bend left,  "\mathbf{1}_n"] \ar[r, bend right, "J_n(\lambda)"] &  \mathbb{C}^{k} \end{tikzcd} \mbox{, if }\lambda \in \mathbb{C}; \begin{tikzcd} \mathbb{C}^{k} \ar[r, bend left,  "J_n(0)"] \ar[r, bend right, "\mathbf{1}_n"] &  \mathbb{C}^{k} \end{tikzcd} \mbox{, if }\lambda= \infty\mbox{, where }
 J_n(\lambda)= \begin{pmatrix}
 \lambda & 1  & {} & 0 \\
 {} & \ddots & \ddots & {} \\
 {} & {} & \ddots & 1 \\
 0 & {} & {} & \lambda 
 \end{pmatrix}
  .$$
 The image by $F_+$ of this representation is the extension
 
 $$ \begin{tikzcd}
 S_n \ar[rrd, "\beta_n"] \ar[d, "\alpha_n +\lambda \beta_n"] &{}&  S_n \ar[rrd, "\beta_n"] \ar[d, "\alpha_n +\lambda \beta_n"] &{}& S_n\ar[d, "\beta_n"] \ar[rd, "\beta_n"] & \ldots & S_n \ar[d, "\beta_n"] \\
 S_{\overline{n}} &{}& S_{\overline{n}} &{}& S_{\overline{n}} & \ldots & S_{\overline{n}} 
 \end{tikzcd}$$
 if $\lambda \in \mathbb{C}$ and 
 
 $$ \begin{tikzcd}
 S_n \ar[rd, "\alpha_n"] \ar[d, "\beta_n"] & S_n \ar[rd, "\alpha_n"] \ar[d, "\beta_n"] & S_n\ar[d, "\beta_n"] \ar[rd, "\alpha_n"]& \ldots & S_n \ar[d, "\beta_n"] \\
 S_{\overline{n}} & S_{\overline{n}} & S_{\overline{n}} & \ldots & S_{\overline{n}} 
 \end{tikzcd}$$ 
 if $\lambda = \infty$
 which is $M_n^k(\lambda)$. Similarly, the image by $F_-$ is $M_{\overline{n}}^k(\lambda)$. 
 
 
 \begin{proof}[Proof of Proposition \ref{prop_uq}]
 The classification of indecomposable $\uq$ modules follows from Lemma \ref{lemm8} and the above classification of $\Indecomp(\mathcal{K})$. The fact that the $P_n$ are the unique projective indecomposable was proved in Lemma \ref{lemm6}. For $0\leq n \leq N-2$, then the modules $P_n$, $\Omega^{-k}_n$, $\Omega^{k}_n$ and $M_n^k (\lambda)$ admit the proper submodules $S_{\overline{n}}$, $S_{\overline{n}}^{\oplus k+1}$, $S_{\overline{n}}^{\oplus k}$ and $S_{\overline{n}}^{\oplus k}$ respectively, so they are not simple and the $S_n$ are the unique simple $\uq$ modules.  
 
 \end{proof}
 
 
 
 
 
  
 \begin{proof}[Proof of Theorem \ref{theorem_representations_QG}]
 Theorem \ref{theorem_representations_QG} is a consequence of Lemmas  \ref{lemm5}, \ref{lemm6} and Proposition \ref{prop_uq}.
 \end{proof}
 
 
\bibliographystyle{amsalpha}
\bibliography{biblio}

\end{document}
