%\documentclass{natureprintstyle}
\documentclass{nature2}
%\documentclass{nature}

%\documentclass{preparation}
%\documentclass{naturesubmissionstyle}
%\documentclass{final_submission}


\bibliographystyle{naturemag}
\usepackage{epsfig}
\usepackage{color}
\usepackage{bm}
\usepackage{graphicx}
\usepackage{longtable}
\usepackage{amssymb}
\usepackage{rotating}
\usepackage{epsfig}
\usepackage{lineno}
\usepackage{amsmath}
%\linenumbers
%\def\arcsec{\hbox{$^{\prime\prime}$}}
\newcommand{\sol}{$_{\odot}$}
\newcommand{\lya}{Ly$\alpha$}

\newcommand{\apj}{Astrophys. J.}
\newcommand{\spie}{Proc. SPIE}
\newcommand{\pasp}{Publ. Astron. Soc. Pac.}
\newcommand{\apjs}{Astrophys. J. Supp.}
\newcommand{\araa}{Annu. Rev. Astron. Astrophys.}
\newcommand{\mnras}{Mon. Not. R. Astron. Soc.}
\newcommand{\apjl}{Astrophys. J. Let.}
\newcommand{\aap}{Astron. Astrophys.}
\newcommand{\aj}{Astron. J.}
\newcommand{\nat}{Nature}
\newcommand{\na}{New Astron. Rev.}
\newcommand{\aaps}{A\&AS}
\newcommand{\gt}{$>$}
\newcommand{\aapr}{Astron. Astrophys. Rev.}

\newcommand\arcdeg{\mbox{$^\circ$}}
\newcommand\arcmin{\mbox{$^\prime$}}
\newcommand\arcsec{\mbox{$^{\prime\prime}$}}
\newcommand\farcs{\mbox{$.\!\!^{\prime\prime}$}}
\newcommand\farcm{\mbox{$.\!\!^{\prime}$}}
\newcommand\farcd{\mbox{$.\!\!^{\circ}$}}

\usepackage{orcidlink}
\usepackage{hyperref}

\title{The Emergence of Brightest Cluster Galaxy in the Most Massive Protocluster Core}

\author{Dong~Dong Shi \,\orcidlink{0000-0002-4314-5686} $^{1}$, Xin Wang\,\orcidlink{0000-0002-9373-3865}$^{2,3}$, Xian~Zhong Zheng\,\orcidlink{0000-0003-3728-9912}$^{1,4}$, Zheng Cai\,\orcidlink{0000-0001-8467-6478}$^{5}$, Xiaohui Fan\,\orcidlink{0000-0003-3310-0131}$^{6}$, Fuyan Bian\,\orcidlink{0000-0002-1620-0897}$^{7}$, Harry I. Teplitz\,\orcidlink{0000-0002-7064-5424}$^{8}$
% Please add your names below. sequence to be determined %ORCID 0000-0002-7064-5424
}

\begin{document}

\maketitle

\let\thefootnote\relax\footnote{
\begin{affiliations}
  
\item Purple Mountain Observatory, Chinese Academy of Sciences, 10 Yuan Hua Road, Nanjing, Jiangsu 210023, China; \href{mailto:ddshi@pmo.ac.cn}{ddshi@pmo.ac.cn}; \href{mailto:xzzheng@pmo.ac.cn}{xzzheng@pmo.ac.cn}

\item School of Astronomy and Space Sciences, University of Chinese Academy of Sciences (UCAS), Beijing 100049, China 

\item National Astronomical Observatories, Chinese Academy of Sciences, Beijing 100101, China

\item School of Astronomy and Space Sciences, University of Science and Technology of China, Hefei 230026,China
 
\item Department of Astronomy, Tsinghua University, Beijing 100084, China
  
\item Steward Observatory, University of Arizona, 933 North Cherry Avenue, Tucson, AZ 85721, USA

\item European Southern Observatory, Alonso de C{\'o}rdova 3107, Casilla 19001, Vitacura, Santiago 19, Chile

\item IPAC, Mail Code 314-6, California Institute of Technology, 1200 E. California Blvd., Pasadena CA, 91125, USA
 

\end{affiliations}
}
\vspace{-3.5mm}
\begin{abstract}

Distant protoclusters of galaxies, as the progenitors of  massive galaxy clusters in the present day, are expected to host  brightest cluster galaxies (BCGs) at their early formation stage\cite{Overzier2016}. It remains unclear whether the assembly of (some) BCGs is essential to the formation of a mature cluster core or vice versa.
Here we report the detection of a pair of massive quiescent galaxies likely in the process of merging at the centre of the spectroscopically confirmed, extremely massive protocluster BOSS1244 at  a look-back time of 10.58 billion years\cite{Zheng2021, Shi2021, Cai2016}.  These galaxies, BOSS1244-QG1 ($5.25\times 10^{11}$\,M$_{\odot}$) and BOSS1244-QG2 ($1.32\times 10^{11}$\,M$_{\odot}$), were detected with Hubble Space Telescope (HST) grism slitless spectroscopic observations. These two quiescent galaxies are among the brightest member galaxies in BOSS1244 and reside at redshifts $z=2.244$ and $z=2.242$, with a half-light radius of $6.76\pm0.50$ and $2.72\pm0.16$\,kiloparsec, respectively. BOSS1244-QG1 and BOSS1244-QG2 are separated by a projected distance of about 70 physical kiloparsecs, implying that we are likely to be witnessing them during the formation of a massive BCG, through a dry merger with size and mass similiar to the most massive BCGs in the local Universe. We thus infer that BCG formation have taken place before the virialization of the cluster core, which shatter and broaden our previous understanding the BCGs in the mature galaxy clusters. Moreover, we find for the first time in BOSS1244 a strong density-star formation relation over a scale of 20 co-moving Mpc, implying that the quenching of star formation in BCGs and their progenitors is likely governed by environment-related processes before the virialization. 

\end{abstract}

BCGs are the ultra-luminous and massive elliptical galaxies usually located at the centres of galaxy clusters in the Universe\cite{Jones1984, Postman1995, Lin2004, VonDerLinden2007, Lauer2014, Zhao2017}. They are usually located close to the peaks of the cluster X-ray emission. BCGs are often used to test formation theories of ultra-massive galaxies, galaxy clusters and large-scale structures. Theoretical works suggest that the formation of BCGs is driven by cooling flows in cluster cores and galactic cannibalism through dynamical friction\cite{Ostriker1975, White1976, Richstone1976, Fabian1994}. Cosmological simulations predict that BCGs are hierarchically formed through multiple mergers of galaxies\cite{DeLucia2007} and the early assembly of BCGs occurs at higher redshifts ($z>3$)\cite{Rennehan2020}, while observational studies show that some BCGs were already in place at $z=1.2-1.5$ and experienced an early rapid growth rather than the hierarchical assembly\cite{Collins2009}. BCGs so far are generally selected from virialized galaxy clusters at $z<2$. Theoretical simulations suggest that BCGs are formed through dissipationless mergers of massive quiescent galaxies at $z\sim2$\cite{Laporte2013}. However, this prediction has not been verified by observations. Still, the early formation of the BCGs and their star formation quenching at high redshifts ($z>2$) are little know from the observational side. Therefore, direct observations of BCGs in massive protoclusters at high redshifts are essential to understand the connection between BCGs and their host clusters, as well as their formation and evolution.

 
We report observational evidence for a first detection of BCGs in formation in the core of the extremely overdense protocluster BOSS1244  at $z=2.24$, showing that the emergence of BCG may take place before the virialization of cluster core. BOSS1244 is identified as the most overdense galaxy protocluster at $z>2$ discovered to date\cite{Shi2021}. Its densest region is unvirialized and contains two substructrues in redshift space. One is at $z=2.230\pm0.002$ and the other is at $z=2.245\pm0.001$. The two substructrues have a velocity offset of  $1390\pm207$\,km\,s$^{-1}$, likely merging to form a larger system\cite{Shi2021}. Thanks to the high throughput and sensitivity  of Hubble Space Telescope (HST), we identified two quiescent galaxies BOSS1244-QG1 at $z=2.244\pm0.011$ and  BOSS1244-QG2 at $z=2.242\pm0.005$ using the HST Wide Field Camera 3 (WFC3) G141 grism spectroscopy. The detailed data reductions and analyses are presented in Methods. 



Pairs of massive quiescent galaxies in protocluster cores at $z>2$ have been identified in only a few cases so far.  BOSS1244-QG1 and  BOSS1244-QG2 are the brightest galaxies in our sample of 40 spectroscopically confirmed member galaxies over $2.223<z<2.255$ in the density peak region of BOSS1244 traced by H$\alpha$ emission-line galaxies (HAEs), as shown by red hexagons in Figure~\ref{fig:fig1}. The two brightest galaxies are located at the centre of the density peak. Figure~\ref{fig:fig2} displays the HST observations and one-dimensional (1D) spectra of BOSS1244-QG1 and BOSS1244-QG2.  Absorption lines (e.g., Balmer) are securely detected and no obvious emission lines are observed. The two galaxies are separated by 8$\farcs$84, i.e., a projection distance of 72.8 kiloparsecs (kpc), and their velocity separation is $\sim231\pm1158$\,km\,s$^{-1}$. The relative large error is mainly caused by the  low-resolution of the HST grism slitless spectra. 

 
 The typical S{\'e}rsic index ($n$), effective radius ($r_{\rm e}$) and stellar mass (M$_{*}$) of pure BCGs at $z\sim0$ is $4.45_{-0.11}^{+0.15}$, $11.48_{-0.53}^{+0.79}$\,kpc and $1.95_{-0.09}^{+0.04} \times 10^{11}$\,M$_{\odot}$, while these parameters for their progenitors at $z\sim2$ are predicted  to be $2.32_{-0.34}^{+0.44}$, $3.63_{-0.59}^{+0.25}$\,kpc and $8.13_{-1.12}^{+0.94} \times 10^{10}$\,M$_{\odot}$, respectively\cite{Zhao2017}.  Table~\ref{tab:taba} summarizes the physical properties of BOSS1244-QG1 and BOSS1244-QG2 in the protocluster core. Their comparison with BCGs or quiescent galaxies at $z\leq2$ is presented in Extended Data Figure~\ref{fig:fige4}. We note that the two quiescent galaxies have key characteristics close to those of local BCGs, and are much higher than the parameters expected for progenitors of BCGs at $z\sim2$, especially for BOSS1244-QG1.  BOSS1244-QG1 and BOSS1244-QG2 already have a high S{\'e}rsic index ($n>4$) and their stellar masses are mostly assembled. Their stellar ages are inferred as $2.24_{-0.20}^{+0.16}$ and $0.83_{-0.16}^{+0.12}$ billion years, corresponding to a formation redshift of $7.92_{-1.47}^{+2.83}$ and $3.08_{-0.19}^{+0.18}$, respectively. These indicate that BOSS1244-QG2 was quenched recently whereas BOSS1244-QG1 had been quenched at an earlier time. One can conclude that (some) BCGs (e.g., BOSS1244-QG1) have been formed before the virialization of a cluster core. 


 BOSS1244-QG1 and BOSS1244-QG2 are likely in the process of merging. According to the relation between galaxy pairs and merger rates derived from the $N$-body simulations\cite{Husko2022}, the estimated average merging timescale for these two quiescent galaxies is about $1.06\pm0.90$\,gigayear (Gyr), corresponding to a merger completion epoch at redshifts of $z\sim1.61\pm0.42$, in agreement with the results estimated by other models\cite{Kitzbichler2008,Lotz2008,Williams2011}. For comparison, at $z<2$ less than $10\%$ of massive quiescent galaxies are found to merge with quiescent companion of comparable mass, giving a dry major merger rate of only about $0.5\%–1\%$ per Gyr\cite{Williams2011}.
Extended Data Figure~\ref{fig:fige6} presents the evolution of overdense regions in BOSS1244. The dense structure with galaxy overdensity ($\delta_{\rm g}$) of $22.9-40.4$ in BOSS1244 is virialized at redshift $z=0.92-1.10$, less than the redshift of the merger of BOSS1244-QG1 and BOSS1244-QG2. The details of the calculations are presented in Methods. Our results provide direct observational evidence that (some) BCGs occur in unvirialized core region of a protocluster.  It indicates that some BCGs through dry merger take place before the virialization of a cluster core, against the usual expectation for the formation of BCGs in the cluster environments. 


Although BOSS1244-QG1 and BOSS1244-QG2 follow the stellar mass-size relation of field quiescent galaxies at $z=2 - 3$ within the relatively large scatter\cite{vanderWel2014,Mowla2019}, shown in Extended Data Figure~\ref{fig:fige3}, they could continue to grow and form a more massive BCG  through dry major merger. Meanwhile, theoretical simulations  demonstrate that the  large scatter at the high-mass end of the stellar mass–size relation is driven by dry mergers\cite{Nipoti2003}. Given that dry major merging  is essential to form a mature BCG\cite{Bernardi2007, Liu2009, Liu2015, vanDokkum2015}, our observations demonstrate that BOSS1244-QG1 and BOSS1244-QG2, as the progenitors of a mature BCG, have already quenched their star formation before the virialization of the BOSS1244 cluster core. This reveals that the star formation quenching of BCGs and even their progenitors may not be driven by the virialization process of a cluster core at high redshifts. The possible quenching mechanisms are discussed in Methods. In addition, BCGs usually host active galactic nuclei (AGNs) more frequently than coeval field early-type galaxies\cite{Best2007}. AGN feedback heats up the intracluster medium (ICM) around the protocluster core so that hot gas could emit in X-ray wavelengths\cite{Best2007, VonDerLinden2007}. Such processes could be investigated with X-ray observations in the future. 


We find that there is a clear gradient in galaxy star formation intensity from the centre to the outskirts of BOSS1244.  The right panel of Figure~\ref{fig:fig3} shows the relationship between equivalent width (EW) of  H$\alpha$ emission line and radial distance relative to BOSS1244-QG2 for BOSS1244 member galaxies within a scale of 21.5\,co-moving Mpc. The H$\alpha$ line equivalent width (EW) is a good indicator of galaxy star formation rate (SFR) relative to its stellar mass, i.e., specific SFR. One can see that specific SFR decreases at decreasing radial distance, indicating that the concentration of the cluster core is likely responsible for the suppression of star formation in galaxies. This supports that dense environments still suppress galaxy formation at $z>2$. From the spatial distribution of BOSS1244 member galaxies, we notice that the two quiescent galaxies BOSS1244-QG1 and BOSS1244-QG2 reside at the centre of the densest region, normal star-forming galaxies (blue points in Figure~\ref{fig:fig1}) spread in the intermedium densest region, and submilimetre galaxies (SMGs, brown squares in Figure~\ref{fig:fig1}), which are extreme dusty starbursts, are mostly in the outskirts of BOSS1244\cite{Zhang2022}. The lack of SMGs inside the core region of BOSS1244 hints a transition to the environment disfavouring the extreme starbursts\cite{Zhang2022}. The presence of quiescent galaxies BOSS1244-QG1 and BOSS1244-QG2 at the centre and an increase of specific SFR at increasing radial distance suggest an evolutionary picture for the protocluster BOSS1244:  the entire region of  BOSS1244 at early time ($z>3$) grows rapidly; the protocluster core region starts to quench star formation in galaxies, and extends gradually towards the outskirts; still strong star formation activities take place at the outer periphery of the protocluster. 



Taken together, our results unveil that formation of massive BCGs through dry major mergers may happen earlier than the full assembly of a cluster core, and the star formation quenching in BCGs and their progenitors can be completed before the cluster core virialization at $z=2-3$. It is clear that the quenching in BCGs is not regulated by the processes related to virialization or hot intracluster gas.  Massive dry pairs ($>10^{11}$\,M$_{\odot}$) are thought to be a good tracer of protocluster cores ($\sim36\arcsec=300\,$kpc in radius\cite{Ando2020} at $z=2.24$), especially for the most overdense systems in the early Universe.  
BOSS1244-QG1 and BOSS1244-QG2 provide direct evidence for the emergence of ultra-massive quiescent galaxies through dry merger in  the core region of massive protoclusters. In the future progress, James Webb Space Telescope (JWST) with unprecedented sensitivity and resolution in imaging and spectroscopy promises a  breakthrough in understanding of the formation of BCGs in protoclusters at high redshifts. 



\begin{figure*}[!ht]
\setlength{\abovecaptionskip}{0pt}
\begin{center}
\includegraphics[trim=0mm 0mm 0mm 1mm,clip,height=0.48\textwidth]{qs_zoom_in_v1.png}
\caption{{\bf Spatial distribution of member galaxies in BOSS1244 at $z=2.24$.}  {\bf Left:} Density map of HAEs (grey solid contour) and SMGs (brown dotted contour) in BOSS1244. The blue points are the confirmed member star-forming galaxies at $z=2.223-2.255$ from HST grism and ground-based NIR spectroscopy. The white points are the NB-selected HAE candidates and the brown squares are the submillimeter objects from JCMT/SCUBA-2 observations  at 850 $\mu$m.  The blue stars are the SDSS/BOSS quasars at $z = 2.24 \pm 0.02$ and the green diamonds show groups of Ly$\alpha$ absorption systems. The red hexagons are BOSS1244-QG1 at $z=2.2441$ and BOSS1244-QG2 at $z=2.2416$. Five black solid boxes  in the density peak of BOSS1244 show the coverage of  HST observations. The size of black dashed box is $3\farcm5 \times 2\farcm5$. {\bf Right:}  Zoomed-in on the left black dashed box. The positions of two massive quiescent galaxies are marked by  ``QG1'' and ``QG2''.} 
\label{fig:fig1}
\end{center}
\end{figure*}


\begin{figure*}[!ht]
\setlength{\abovecaptionskip}{0pt}
\begin{center}
\includegraphics[trim=0mm 0mm 0mm 0mm,clip,height=0.4\textwidth]{qs_location_v1.png}
\includegraphics[trim=0mm 0mm 0mm 0mm,clip,height=0.4\textwidth]{qs_grism.png}
\caption{ {\bf HST image (left) and one-dimensional grism spectroscopy (right) of BOSS1244-QG1 and BOSS1244-QG2.}  The size of the left HST image is $2\farcm0 \times 2\farcm0$. The inset box ($12\arcsec \times 12\arcsec$) is a zoom-in to the BOSS1244-QG1 and BOSS1244-QG2.  Black circles mark the HAE candidates, blue circles denote the spectroscopically confirmed galaxies at $z=2.24$.  Over this region the  overdensity factor of HAEs is $\delta_{g}>40$. The red hexagons mark QG1 and QG2. In the right plots, light blue curves with 1$\sigma$ uncertainties refer to the extracted spectra from HST/WFC3 grism observations. Red curves are the best fitting with lines based on the Flexible Stellar Population Synthesis (FSPS) template models.  Black curves represent the best-fitting continuum models. Red curves and Black curves show the same trend. The  expected spectral lines (e.g., Balmer or Ca~II absorption) are marked with the vertical blue dotted lines. Red dotted lines show the location of Balmer and 4000\,\AA \, break.} 
\label{fig:fig2}
\end{center}
\end{figure*}



\begin{figure*}[!ht]
\setlength{\abovecaptionskip}{0pt}
\begin{center}
\includegraphics[trim=0mm 0mm 0mm 0mm,clip,height=0.4\textwidth]{qs_redshift_v1.png}
\includegraphics[trim=0mm 0mm 0mm 0mm,clip,height=0.4\textwidth]{qs_ew_density_v1.png}
\caption{ {\bf Left: Redshift distribution of member galaxies against  Declination in BOSS1244.} The color-coded points are  galaxies confirmed by grism at $z=2.223-2.255$, the black points and blue stars are the confirmed HAEs and quasars at $z=2.228-2.245$. Red hexagons denote our targets BOSS1244-QG1 and BOSS1244-QG2 marked with ``QG1'' and ``QG2''.   {\bf Right: H$\alpha$ equivalent width (EW) versus radial distance relative to the quiescent galaxy BOSS1244-QG2 for member galaxies.}  The light blue points are the spectroscopically confirmed galaxies at $z=2.223-2.255$ in the BOSS1244 SW region and the blue diamonds are the average values in different bins. The red hexagons are BOSS1244-QG1 and BOSS1244-QG2. The red solid line represents the robust fit to the whole of member galaxies with errors. The red dashed lines show the 1$\sigma$ scatter. A strong correlation exists between EW and radial distance, suggesting that the gradual suppression star formation in galaxies coincides with the increase of environment density. } 
\label{fig:fig3}
\end{center}
\end{figure*}


\begin{table}
\begin{center} 
%\centering %\scriptsize \footnotesize

\caption{The physical properties of BOSS1244-QG1 and BOSS1244-QG2. \label{tab:taba}} 
\begin{tabular}{lccc}
\hline 
\hline
 Properties & BOSS1244-QG1 & BOSS1244-QG2  \\
 \hline
R.A.\,(J2000.0)   &  12:43:32.04  & 12:43:32.57 \\ 
Decl.\,(J2000.0)  & +35:54:39.68  & +35:54:33.59 \\
spec-$z$  &   2.244$\pm$0.011 & 2.242$\pm$0.005  \\
F125W\,(mag) &  22.13$\pm$0.07 & 22.38$\pm$0.03  \\
F160W\,(mag) &  21.03$\pm$0.04 & 21.56$\pm$0.03  \\
r$_{\rm e,F160W}$\,($\arcsec$) & 0.82$\pm$0.06  & 0.33$\pm$0.02 \\
r$_{\rm e,F160W}$\,($\rm kpc$)  & 6.76$\pm$0.50 & 2.72$\pm$0.16 \\
$n$ (F160W)  & 4.53$\pm$0.29  & 6.36$\pm$0.59\\
$b/a$ (F160W)   & 0.82$\pm$0.02 & 0.83$\pm$0.03 \\
P.A. (F160W)   & 42.02$\pm$3.53 & 0.53$\pm$5.17 \\
D$_{\rm n}$4000 index   & 1.47$\pm$0.29  &  1.24$\pm$0.15 \\
log(M$_{*}$/M$_{\odot}$)  & $11.72_{-0.02}^{+0.05}$ & $11.12_{-0.03}^{+0.11}$ \\
log(SFR/$M_{\odot}$\,yr$^{-1}$)  & $0.92_{-0.02}^{+0.04}$ & $0.02_{-0.04}^{+0.91}$ \\
log(Age/yr)  & $9.35_{-0.04}^{+0.03}$ &  $8.92_{-0.09}^{+0.06}$ \\
log($\tau$/yr)  & $8.45_{-0.04}^{+0.03}$ &  $7.94_{-0.06}^{+0.05}$ \\
$A_{V}$\,(mag)  & $0.79_{-0.07}^{+0.09}$ & $0.42_{-0.09}^{+0.61}$ \\
\hline
\end{tabular}
\end{center}
\end{table}





\begin{thebibliography}{10}
\expandafter\ifx\csname url\endcsname\relax
  \def\url#1{\texttt{#1}}\fi
\expandafter\ifx\csname urlprefix\endcsname\relax\def\urlprefix{URL }\fi
\providecommand{\bibinfo}[2]{#2}
\providecommand{\eprint}[2][]{\url{#2}}


\bibitem{Overzier2016}
\bibinfo{author}{Overzier R.~A.}
\newblock \bibinfo{title}{{The realm of the galaxy protoclusters. A review}}.
\newblock \emph{\bibinfo{journal}{\aapr}} \textbf{\bibinfo{volume}{24}},
  \bibinfo{pages}{14} (\bibinfo{year}{2016}).


\bibitem{Zheng2021}
\bibinfo{author}{Zheng X.~Z., Cai Z., An F.~X., Fan X., Shi D.~D.}
\newblock \bibinfo{title}{{MAMMOTH: confirmation of two massive galaxy overdensities at z = 2.24 with H$\alpha$ emitters}}.
\newblock \emph{\bibinfo{journal}{\mnras}} \textbf{\bibinfo{volume}{500}},
  \bibinfo{pages}{4354-4364} (\bibinfo{year}{2021}).


\bibitem{Shi2021}
\bibinfo{author}{Shi, D.~D., Cai, Z., Fan, X., Zheng, X.~Z., Huang, Y.-H., Xu, J.}
\newblock \bibinfo{title}{{Spectroscopic Confirmation of Two Extremely Massive Protoclusters, BOSS1244 and BOSS1542, at z = 2.24}}.
\newblock \emph{\bibinfo{journal}{\apj}} \textbf{\bibinfo{volume}{915}},
  \bibinfo{pages}{32} (\bibinfo{year}{2021}).

\bibitem{Cai2016}
\bibinfo{author}{Cai, Z., Fan, X., Peirani, S., et al.}
\newblock \bibinfo{title}{Mapping the Most Massive Overdensity Through Hydrogen (MAMMOTH) I: Methodology.}
\newblock \emph{\bibinfo{journal}{\apj}} \textbf{\bibinfo{volume}{833}},
  \bibinfo{pages}{135} (\bibinfo{year}{2016}).


\bibitem{Jones1984}
\bibinfo{author}{Jones, C. \& Forman, W.}
\newblock \bibinfo{title}{{Brightest Cluster Galaxies at the Present Epoch}}.
\newblock \emph{\bibinfo{journal}{\apj}} \textbf{\bibinfo{volume}{276}},
  \bibinfo{pages}{38-55} (\bibinfo{year}{1984}).

\bibitem{Postman1995}
\bibinfo{author}{Postman, M. \& Lauer, T.~R.}
\newblock \bibinfo{title}{{Brightest Cluster Galaxies as Standard Candles}}.
\newblock \emph{\bibinfo{journal}{\apj}} \textbf{\bibinfo{volume}{440}},
  \bibinfo{pages}{28} (\bibinfo{year}{1995}).

\bibitem{Lin2004}
\bibinfo{author}{Lin, Y.-T. \& Mohr, J.~J.}
\newblock \bibinfo{title}{{K-band Properties of Galaxy Clusters and Groups: Brightest Cluster Galaxies and Intracluster Light}}.
\newblock \emph{\bibinfo{journal}{\apj}} \textbf{\bibinfo{volume}{617}},
  \bibinfo{pages}{879-895} (\bibinfo{year}{2004}).


\bibitem{VonDerLinden2007}
\bibinfo{author}{Von Der Linden, A., Best, P.~N., Kauffmann, G., et al.}
\newblock \bibinfo{title}{{How special are brightest group and cluster galaxies}}?
\newblock \emph{\bibinfo{journal}{\mnras}} \textbf{\bibinfo{volume}{379}},
  \bibinfo{pages}{867-893} (\bibinfo{year}{2007}).


\bibitem{Lauer2014}
\bibinfo{author}{Jones, C. \& Forman, W.}
\newblock \bibinfo{title}{{Lauer, T.~R., Postman, M., Strauss, M.~A}}.
\newblock \emph{\bibinfo{journal}{\apj}} \textbf{\bibinfo{volume}{797}},
  \bibinfo{pages}{82} (\bibinfo{year}{2014}).

\bibitem{Zhao2017}
\bibinfo{author}{Zhao, D., Conselice, C.~J., Arag{\'o}n-Salamanca, A.}
\newblock \bibinfo{title}{{Exploring the progenitors of brightest cluster galaxies at z {\ensuremath{\sim}} 2}}.
\newblock \emph{\bibinfo{journal}{\mnras}} \textbf{\bibinfo{volume}{464}},
  \bibinfo{pages}{1393-1414} (\bibinfo{year}{2017}).


\bibitem{Ostriker1975}
\bibinfo{author}{Ostriker, J.~P. \& Tremaine, S.~D.}
\newblock \bibinfo{title}{{Another evolutionary correction to the luminosity of giant galaxies}}.
\newblock \emph{\bibinfo{journal}{\apjl}} \textbf{\bibinfo{volume}{202}},
  \bibinfo{pages}{L113-L117} (\bibinfo{year}{1975}).
 
 \bibitem{White1976}
\bibinfo{author}{White, S.~D.~M.}
\newblock \bibinfo{title}{{Dynamical friction in spherical clusters}}.
\newblock \emph{\bibinfo{journal}{\mnras}} \textbf{\bibinfo{volume}{174}},
  \bibinfo{pages}{19-28} (\bibinfo{year}{1976}).
 
 
\bibitem{Richstone1976}
\bibinfo{author}{Richstone, D.~O.}
\newblock \bibinfo{title}{{Collisions of galaxies in dense clusters. II. Dynamical evolution of cluster galaxies}}.
\newblock \emph{\bibinfo{journal}{\apjl}} \textbf{\bibinfo{volume}{204}},
  \bibinfo{pages}{642-648} (\bibinfo{year}{1976}).

\bibitem{Fabian1994}
\bibinfo{author}{Fabian, A.~C.}
\newblock \bibinfo{title}{{Cooling Flows in Clusters of Galaxies}}.
\newblock \emph{\bibinfo{journal}{\araa}} \textbf{\bibinfo{volume}{32}},
  \bibinfo{pages}{277-318} (\bibinfo{year}{1994}).


\bibitem{DeLucia2007}
\bibinfo{author}{De Lucia G., Blaizot J.}
\newblock \bibinfo{title}{{The hierarchical formation of the brightest cluster galaxies}}.
\newblock \emph{\bibinfo{journal}{\mnras}} \textbf{\bibinfo{volume}{375}},
  \bibinfo{pages}{2-14} (\bibinfo{year}{2007}).

\bibitem{Rennehan2020}
\bibinfo{author}{Rennehan, D., Babul, A., Hayward, C.~C., et al.}
\newblock \bibinfo{title}{{Rapid early coeval star formation and assembly of the most-massive galaxies in the Universe}}.
\newblock \emph{\bibinfo{journal}{\mnras}} \textbf{\bibinfo{volume}{493}},
  \bibinfo{pages}{4607-4621} (\bibinfo{year}{2020}).


\bibitem{Collins2009}
\bibinfo{author}{Collins C.~A., Stott J.~P., Hilton M., Kay S.~T., Stanford S.~A., Davidson M., Hosmer M., et al.}
\newblock \bibinfo{title}{{Early assembly of the most massive galaxies}}.
\newblock \emph{\bibinfo{journal}{\nat}} \textbf{\bibinfo{volume}{458}},
  \bibinfo{pages}{603-606} (\bibinfo{year}{2009}).

\bibitem{Laporte2013}
\bibinfo{author}{Laporte, C.~F.~P., White, S.~D.~M., Naab, T., et al.}
\newblock \bibinfo{title}{{The growth in size and mass of cluster galaxies since $z = 2$}}.
\newblock \emph{\bibinfo{journal}{\mnras}} \textbf{\bibinfo{volume}{435}},
  \bibinfo{pages}{901-909} (\bibinfo{year}{2013}).




\bibitem{Kitzbichler2008}
\bibinfo{author}{Kitzbichler, M.~G. \& White, S.~D.~M.}
\newblock \bibinfo{title}{{A calibration of the relation between the abundance of close galaxy pairs and the rate of galaxy mergers}}.
\newblock \emph{\bibinfo{journal}{\mnras}} \textbf{\bibinfo{volume}{391}},
  \bibinfo{pages}{1489-1498} (\bibinfo{year}{2008}).


\bibitem{Lotz2008}
\bibinfo{author}{Lotz, J.~M., Jonsson, P., Cox, T.~J., et al.}
\newblock \bibinfo{title}{{Galaxy merger morphologies and time-scales from simulations of equal-mass gas-rich disc mergers}}.
\newblock \emph{\bibinfo{journal}{\mnras}} \textbf{\bibinfo{volume}{391}},
  \bibinfo{pages}{1137-1162} (\bibinfo{year}{2008}).



\bibitem{Williams2011}
\bibinfo{author}{Williams, R.~J., Quadri, R.~F., \& Franx, M.}
\newblock \bibinfo{title}{{The Diminishing Importance of Major Galaxy Mergers at Higher Redshifts}}.
\newblock \emph{\bibinfo{journal}{\apj}} \textbf{\bibinfo{volume}{738}},
  \bibinfo{pages}{L25} (\bibinfo{year}{2011}).

\bibitem{Husko2022}
\bibinfo{author}{Hu{\v{s}}ko, F., Lacey, C.~G., \& Baugh, C.~M.}
\newblock \bibinfo{title}{{Statistics of galaxy mergers: bridging the gap between theory and observation}}.
\newblock \emph{\bibinfo{journal}{\mnras}} \textbf{\bibinfo{volume}{509}},
  \bibinfo{pages}{5918-5937} (\bibinfo{year}{2022}).

\bibitem{Mowla2019}
\bibinfo{author}{Mowla L.~A., van Dokkum P., Brammer G.~B., Momcheva I., van der Wel A., Whitaker K., Nelson E., et al.}
\newblock \bibinfo{title}{{ COSMOS-DASH: The Evolution of the Galaxy Size-Mass Relation since z {\ensuremath{\sim}} 3 from New Wide-field WFC3 Imaging Combined with CANDELS/3D-HST}}.
\newblock \emph{\bibinfo{journal}{\apj}} \textbf{\bibinfo{volume}{880}},
  \bibinfo{pages}{57} (\bibinfo{year}{2019}).

\bibitem{vanderWel2014}
\bibinfo{author}{van der Wel A., Franx M., van Dokkum P.~G., Skelton R.~E., Momcheva I.~G., Whitaker K.~E., Brammer G.~B., et al.}
\newblock \bibinfo{title}{{ 3D-HST+CANDELS: The Evolution of the Galaxy Size-Mass Distribution since $z = 3$}}.
\newblock \emph{\bibinfo{journal}{\apj}} \textbf{\bibinfo{volume}{788}},
  \bibinfo{pages}{28} (\bibinfo{year}{2014}).


\bibitem{Nipoti2003}
\bibinfo{author}{Nipoti, C., Londrillo, P., \& Ciotti, L.}
\newblock \bibinfo{title}{{Galaxy merging, the fundamental plane of elliptical galaxies and the M$_{BH}$-{\ensuremath{\sigma}}$_{0}$ relation}}.
\newblock \emph{\bibinfo{journal}{\mnras}} \textbf{\bibinfo{volume}{342}},
  \bibinfo{pages}{501-512} (\bibinfo{year}{2003}).

\bibitem{Bernardi2007}
\bibinfo{author}{Bernardi, M., Hyde, J.~B., Sheth, R.~K..}
\newblock \bibinfo{title}{{The Luminosities, Sizes, and Velocity Dispersions of Brightest Cluster Galaxies: Implications for Formation History}}.
\newblock \emph{\bibinfo{journal}{\aj}} \textbf{\bibinfo{volume}{133}},
  \bibinfo{pages}{1741-1755} (\bibinfo{year}{2007}).

\bibitem{Liu2009}
\bibinfo{author}{Liu, F.~S., Mao, S., Deng, Z.~G., et al.}
\newblock \bibinfo{title}{{Major dry mergers in early-type brightest cluster galaxies}}.
\newblock \emph{\bibinfo{journal}{\mnras}} \textbf{\bibinfo{volume}{396}},
  \bibinfo{pages}{2003-2010} (\bibinfo{year}{2009}).


\bibitem{Liu2015}
\bibinfo{author}{Liu, F.~S., Lei, F.~J., Meng, X.~M., et al.}
\newblock \bibinfo{title}{{Ongoing growth of the brightest cluster galaxies via major dry mergers in the last {\ensuremath{\sim}}6 Gyr}}.
\newblock \emph{\bibinfo{journal}{\mnras}} \textbf{\bibinfo{volume}{447}},
  \bibinfo{pages}{1491-1497} (\bibinfo{year}{2015}).


\bibitem{vanDokkum2015}
\bibinfo{author}{van Dokkum, P.~G., Nelson, E.~J., Franx, M.}
\newblock \bibinfo{title}{{Forming Compact Massive Galaxies}}.
\newblock \emph{\bibinfo{journal}{\apj}} \textbf{\bibinfo{volume}{813}},
  \bibinfo{pages}{23} (\bibinfo{year}{2015}).


\bibitem{Best2007}
\bibinfo{author}{Best, P.~N., von der Linden, A., Kauffmann, G.}
\newblock \bibinfo{title}{{On the prevalence of radio-loud active galactic nuclei in brightest cluster galaxies: implications for AGN heating of cooling flows}}.
\newblock \emph{\bibinfo{journal}{\mnras}} \textbf{\bibinfo{volume}{379}},
  \bibinfo{pages}{894-908} (\bibinfo{year}{2007}).


\bibitem{Zhang2022}
\bibinfo{author}{Zhang Y., Zheng X.~Z., Shi D.~D., Gao Y., Dannerbauer H., An F.~X., Shu X., et al.}
\newblock \bibinfo{title}{{Submillimetre galaxies in two massive protoclusters at z = 2.24: witnessing the enrichment of extreme starbursts in the outskirts of HAE density peaks}}.
\newblock \emph{\bibinfo{journal}{\mnras}} \textbf{\bibinfo{volume}{512}},
  \bibinfo{pages}{4893-4908} (\bibinfo{year}{2022}).


\bibitem{Ando2020}
\bibinfo{author}{Ando, M., Shimasaku, K., \& Momose, R.}
\newblock \bibinfo{title}{{A systematic search for galaxy proto-cluster cores at z {\ensuremath{\sim}} 2}}.
\newblock \emph{\bibinfo{journal}{\mnras}} \textbf{\bibinfo{volume}{496}},
  \bibinfo{pages}{3169-3181} (\bibinfo{year}{2020}).





\end{thebibliography} 

\begin{addendum}

\item [Data availability]

The HST data are available in the Mikulski Archive for Space Telescopes (MAST; https://archive.stsci.edu), under program ID 16276 and 15266. All other data that support the findings of this study are available from the corresponding author upon reasonable request.

\item [Code availability] 
The standard data analysis tools are used in the Python environment, and publicly available codes are presented in Methods.

\item  [Acknowledgements]  
We thank Yu Luo for his useful discussions in this work. This work is supported by the National Science Foundation of China (12233005, 12073078 and 12173088), the science research grants from the China Manned Space Project with NO. CMS-CSST-2021-A02, CMS-CSST-2021-A04 and CMS-CSST-2021-A07.  D.D.S. acknowledges support from China Postdoctoral Science Foundation (2021M703488) and Jiangsu Funding Program for Excellent Postdoctoral Talent (2022ZB473). 


\item[Author  Contributions] 
D.D Shi and X.Z Zheng led the data analyses including the deep multiwavelength imaging and spectroscopy, and writing of the manuscript. 
X.Wang reduced and analyzed the HST grism data.  Z.Cai,  X.Fan, F. Bian and Harry I. Teplitz improved the manuscript. All authors discussed and commented on the manuscript. 


\item[Author Information]  
The authors declare that they have no competing financial interests. Correspondence and requests for materials should be addressed to D.D Shi (ddshi@pmo.ac.cn) and X.Z Zheng (xzzheng@pmo.ac.cn).



\end{addendum}






%%%%%%%%%%

\setcounter{page}{1}
\setcounter{figure}{0}
\setcounter{table}{0}
\renewcommand{\thefigure}{\arabic{figure}}
\renewcommand{\thetable}{\arabic{table}}
\renewcommand{\figurename}{Extended Data Figure}
\renewcommand{\tablename}{Extended Data Table}


\begin{center}
{\bf \Large \uppercase{Methods} }
\end{center}


Assuming a Lambda cold dark matter ($\Lambda$CDM) cosmological model, we adopt the cosmological parameters of $\Omega_{\rm M} = 0.3$, $\Omega_{\rm \Lambda} =0.7$ and $ H\rm_0 = 70$\,km\,s$^{-1}$\,Mpc$^{-1}$. Magnitudes are presented in the AB system throughout this paper.


\section{Observations and data reduction} 

%{\bf Observations and data reduction.}
The HST observations of five pointings in BOSS1244 were obtained between 17 January 2021 and 21 February 2021. The FOV of each pointing  is $2\farcm3\times2\arcmin$. Each pointing contains three visits with different orientations. 
%Each visit has a long (about 1 orbit) grism exposure, together with short F125W pre-imaging
Each visit takes about one-orbit exposure, together with F125W pre-imaging.  The three grism exposures are combined to remove the blending of spectra in the BOSS1244 field. The total exposure times in F125W and G141 are 1,817\,s and 5,917\,s.  The data reduction is conducted using the {\it grizli} pipeline\cite{Brammer2021}. The detailed information about our WFC3/grism slitless observations and data reduction steps can be found in previous work\cite{Wang2021}.  In addition, we also use the archival WFC3 F160W imaging data (with an effective exposure time per pointing 2,614\,s) and the coverage region is the same as the F125W area. The image depths (5$\sigma$, point sources) in F125W and F160W are estimated from photometry of an aperture of radius of 1$\arcsec$ on random positions in the blank background regions to be 24.95\,mag and 24.97\,mag, respectively. 


Furthermore, the analysis included previously-obtained deep ground-based imaging observations in the BOSS1244 field taken with the Large Binocular Telescope (LBT)/LBC through SDT-$U_{\rm spec}$ and Sloan $z$-band in 2018A, Canada–France–Hawaii Telescope (CFHT)/WIRCam through narrowband H$_{2}$S(1) and broadband $K_{\rm s}$ in 2016A, and James Clerk Maxwell Telescope (JCMT)/SCUBA-2 at 850\,$\mu$m in 2018-2019A. The optical and NIR observations were reduced via the standard data reduction pipeline\cite{Schirmer2013, Zheng2021}, including the basic procedures of bias, dark, flat-fielding, removal of bad/hot pixels and sky subtraction. Source detection and photometry was performed using software SExtractor\cite{Bertin1996}; and SCAMP\cite{Bertin2006} was used to measure the astrometric calibration with bright stars, giving an accuracy of rms$\leq0\farcs1$ in these bands. We selected good-quality science images and stacked those of the same filter together with the help of the software tool SWarp\cite{Bertin2002}. The total exposure times in the SDT-$U_{\rm spec}$, $z$, H$_{2}$S(1) and  $K_{\rm s}$ bands are 4.65, 3.92, 7.50 and 7.50\,hours, and their 5$\sigma$ limiting magnitudes (diameter-2$\arcsec$ aperture, point sources) are 26.67, 25.12, 22.58 and 23.29\,mag, respectively. The description of the JCMT/SCUBA-2 observations with integrated time of 18.67\,hours in BOSS1244 and data reduction were presented in our previous work\cite{Zhang2022}. 


 

\section{Data sample} 

We extract one-dimensional (1D) grism slitless spectra in BOSS1244 and compute the redshifts of galaxies through the best fitting of spectral template synthesis. A total of 284 galaxies are identified with robust grism redshift measurements and the redshift range of $0.15<z<4$\cite{Wang2021}. The redshift histogram shows a spike at $z=2.24\pm0.02$, which is consistent with ground-based NIR spectroscopy of HAEs\cite{Shi2021}. Here we choose galaxies at $z=2.223-2.255$ as the protocluster members, which are consistent with the redshift of the NIR narrowband H$_{2}$S(1) filter used to identify the HAE candidates\cite{Zheng2021}. Combined with the previous ground-based NIR spectroscopy of HAEs, a total of 40 member galaxies are confirmed in the BOSS1244 southwest (SW) region. Their spatial distribution is shown in Figure~\ref{fig:fig1}. Their redshift distribution is shown in the left panel of Figure~\ref{fig:fig3}. BOSS1244 SW region contains two peaks in redshift space, suggesting that there are two components in the SW region\cite{Shi2021}. 

BOSS1244-QG1 and BOSS1244-QG2 were optimally extracted and simultaneously fitted with a suite of galaxy templates. We fitted the templates with redshift over a fine ($\Delta z = 0.0004$) grid from $z = 0.2-4.0$, giving the the probability distribution function of redshifts. Following the previous literature\cite{Willis2020}, we defined a robust redshift using $P(z)>0.5$ where $P(z)$ was the intergral of redshift probability distribution function for BOSS1244-QG1/BOSS1244-QG2 over the interval $2.223<z<2.267$. The estimated redshifts of BOSS1244-QG1 and BOSS1244-QG2 are $z=2.2441\pm0.011$ and $z=2.2416\pm0.005$, respectively. They belong to protocluster BOSS1244. Note that BOSS1244-QG2 exhibits clear features of Balmer (H$\delta$, H$\gamma$ or H$\beta$) and Ca~II absorption, while BOSS1244-QG1 shows strong H$\beta$ absorption line and other weak Balmer and Ca~II absorption features.



\section{Parameter Estimation}

{\bf Equivalent Width (EW):} The rest-frame EWs of 40 member galaxies in BOSS1244 with the following equations:


\begin{equation}
\centering
f_{\rm c}=\frac{f_{K_{\rm s}}-f_{\rm H_{2}S(1)}(\Delta \lambda_{\rm H_{2}S(1)}/\Delta \lambda_{K_{\rm s}})}{1-\Delta \lambda_{\rm H_{2}S(1)}/\Delta \lambda_{K_{\rm s}}},
\end{equation}



\begin{equation}
\centering
EW=(1+z)^{-1}\frac{F_{\rm line}}{f_{\rm c}}.
\end{equation}
where $\Delta \lambda_{\rm H_{2}S(1)}=0.0295\,\mu$m and $\Delta\lambda_{K_{\rm s}}=0.327\,\mu$m are the the full width at half-maximum (FWHM) of the H$_{2}$S(1) and $K_{\rm s}$ filters, and $f_{\rm H_2S(1)}$ and $f_{K_{\rm s}}$ are the flux density in the H$_{2}$S(1) and $K_{\rm s}$-band, respectively. The rest-frame EWs of member galaxies tend to become smaller with decreasing radial distance, as shown in the right panel of Figure~\ref{fig:fig3}. The estimated Spearman/Pearson correlation coefficients $r=0.60/0.61$ and  $p=0.00033/0.00019$ confirm that this trend is very significant. The relation is given over a scale of 21.5\,cMpc, larger than the typical scale ($\sim15\,$cMpc) of massive protoclusters at $z\sim2-3$. This result supports that overdense environments still suppress star formation at $z>2$.

Considering BOSS1244-QG1 and BOSS1244-QG2 are relatively compact and isolated objects, their total intrinsic fluxes in multiwavelength data are measured through deconvolution point spread function (PSF) models. We first select several bright, isolated stars and construct a normalized PSF in each band, then the total magnitudes are derived from the best-fitting S{\'e}rsic profile given by GALFIT\cite{Peng2010}. For signals that are too faint to be securely detected in a certain band (e.g., SDT-$U_{\rm spec}$), we use forced photometry (2-3 times FWHM of seeing aperture is fixed) to estimate the upper limits of their fluxes.  


{\bf D$_{n}$4000 index:} The 4000\AA\,break is produced by a combination of metal absorption on the atmosphere of old and cool stars and the lack of flux from young and hot OB stars. The strength of the 4000\AA\,break is used to trace the age of galaxies. 
We use D$_{\rm n}$4000 index to quantify the strength of the 4000\AA\,break; D$_{\rm n}$4000 index is defined as follows\cite{Balogh1999}:
\begin{equation}
\centering
D_{n}4000=\sum_{\small\lambda=4000\,\rm\AA}^{\small 4100\,\rm\AA}F_{\lambda}/\sum_{\small\lambda=3850\,\rm\AA}^{\small 3950\,\rm\AA}F_{\lambda}, 
\end{equation}
where $F_{\lambda}$ is the flux per unit wavelength.

The measured D$_{\rm n}$4000 index in BOSS1244-QG1 is $1.47\pm0.29$, which is slightly larger than BOSS1244-QG2 with $1.24\pm0.15$, suggesting that BOSS1244-QG1 is older than BOSS1244-QG2. This is consistent with the redder color of BOSS1244-QG1 (F125W-F160W=1.10$\pm$0.08) compared to BOSS1244-QG2 (F125W-F160W=0.82$\pm$0.04). 




{\bf Mean merging timescale:} 
The projected distance between BOSS1244-QG1 and BOSS1244-QG2 is about $\sim50$\,h$^{-1}$\,kpc, and their velocity separation is about $231\,$km\,s$^{-1}$. The stellar mass ratio of BOSS1244-QG2 to BOSS1244-QG1 is $>1/4$. We used the following two methods to estimate the merging timescale of BOSS1244-QG1 and BOSS1244-QG2.

{\bf (a) KW08 model\cite{Kitzbichler2008}:} We estimate the average merging timescale based on the relation between galaxy pairs and merger rates derived from the Millennium Simulation. This relation depends very little on cosmological parameters and galaxy formation assumptions. Instead, the orbital times of pairs is a function of projected separation and galaxy properties\cite{Kitzbichler2008}. The equation is following:

\begin{equation}
\centering
{< T_{{\rm merge}}> }^{-1/2}=T_{0}^{-1/2}+f_{1}z+f_{2}(logM_{*}-10), 
\end{equation}
where the coefficients $T_{0}$, $f_{1}$ and $f_{2}$ and their uncertainties are 3,310\,$h^{-1}$\,Myr, $-1.05\pm$0.03$\times$10$^{-3}$\,$h^{-1}$\,Myr$^{-1/2}$ and 6.68$\pm$0.08$\times$10$^{-3}$\,$h^{-1}$\,Myr$^{-1/2}$ when the radial velocity difference of $\leq300\,$km\,s$^{-1}$ and projection distance of $\leq50$\,h$^{-1}$\,kpc are adopted. The estimated mean merging timescale for BOSS1244-QG1 and BOSS1244-QG2 is $2.26\,$Gyr, yielding an end epoch of the merger at redshift $z\sim1.17$. 


 {\bf (b) HLB2022 model\cite{Husko2022}:} Recently, a galaxy formation model with more accurate tracking of subhalo orbits used to study galaxy merger rates and merger timescale at redshift $z=0-10$. We use the updated average merger time-scale formula which is functions of stellar mass and redshift, as well as close projected distance and velocity separation of pairs\cite{Husko2022} to measure the merger time-scale of BOSS1244-QG1 and BOSS1244-QG2. This relation is as follows:  




\begin{equation}
\begin{split}
\centering
T_{\rm merge}(M_{*},z,r_{max},v_{max})=&T^{500}_{20}(M_{*},z)\times(\frac{r_{max}}{20\,h^{-1}\,kpc})^{\alpha}\\&\times\frac{erf(v_{max}/V_{0})^{\beta}}{erf(500\,km\,s^{-1}/V_{0})^{\beta}}
\end{split}
\end{equation}

\begin{equation}
\centering
T^{500}_{20}(M_{*},z)=T_{0}e^{b_{0}(z-z_{0})^{3}}(\frac{M_{*}}{10^{10}})^{a_{0}+a_{1}(1+z)^{a_{2}}}
\end{equation}
where the parameters $T_{0}$, $b_{0}$, $z_{0}$, $a_{0}$, $a_{1}$, $a_{2}$, $V_{0}$, $\alpha$ and $\beta$ are given in Table~2 in ref\cite{Husko2022}. The merging timescale for BOSS1244-QG1 and BOSS1244-QG2 is $1.06\pm0.90\,$Gyr, corresponding to the end epoch of the merger at redshift $z\sim1.61\pm0.42$. This mergring timescale is shorter than the previous estimation. The differences between the two models are presented in detail in Ref\cite{Husko2022}. We adopt this mergering timescale in this work.



{\bf The evolution of density peak in BOSS1244:}
According to the spherical collapse model, we calculate the evolution of densest region of BOSS1244. There are three specific evolutionary steps based on spherical linear collapse model. The first step is turn-around, when the overdense sphere reaches a maximum radius. The corresponding linear overdensity is $\delta_{\rm ta}\approx1.062$. After the turn-around, the overdense sphere is virialized at half its turn-around radius, and the linear overdensity is $\delta_{\rm vir}\approx1.580$. Then the sphere continues to collapse until the collapse stops, at the linear overdensity of $\delta_{\rm c}\approx1.686$. 

We calculate the evolution of density peak region in BOSS1244 using Equation~7 in ref\cite{Cucciati2018}, as  shown in Extended Data Figure~\ref{fig:fige6}. The blue curve is the evolution of typical protocluster-scale region (galaxy overdensity $\delta_{\rm g}=22.9$ at the scale of 15\,cMpc). The evolution of protocluster core region (galaxy overdensity $\delta_{\rm g}=40.4$ at the scale of 8\,cMpc) is represented in red curve. According to the
spherical collapse model, the protocluster will be a virialized system by $z\sim0.92$ and the protocluster core will be virialized at $z\sim1.10$. Even so, the final merger of BOSS1244-QG1 and BOSS1244-QG2 still occured before the virialization of a cluster core. 




\section{Structural parameters of member galaxies } 

We measure structural parameters from the F160W images for member galaxies in BOSS1244, including our two target quiescent galaxies BOSS1244-QG1 and BOSS1244-QG2. We use SExtractor to detect these galaxies and extract their photometric and astrometric parameters, including coordinates, magnitudes, effective radius, axis ratio and position angle as the initial input parameters for GALFIT. We mask the objects close to the targets taking advantage of a segmentation map from SExtractor. GALFIT is run iteratively to obtain the final structural parameters  of the measured galaxies. The structural parameters and total magnitudes of BOSS1244-QG1 and BOSS1244-QG2 are listed in the Main text.

We show several scaling relations between structural parameters (effective radius $r_{\rm e}$, mean effective surface brightness $<\mu_{eff}>$, S{\'e}rsic index $n$ and absolute magnitude with $K$-correction MAG\_ABS) and spatial distance of member galaxies relative to BOSS1244-QG2 in Extended Data Figure~\ref{fig:fige1}. BOSS1244-QG1 is the brightest galaxy in our sample of 40 member galaxies and BOSS1244-QG2 is the second brightest. There is no obvious relationship between structural parameters and radial distance.  BOSS1244-QG1 and BOSS1244-QG2 have $n=4.53\pm0.29$ and $n=6.36\pm0.59$, similar to early-type galaxies in the local Universe. About 20.5 percent of galaxies (including BOSS1244-QG1 and BOSS1244-QG2) with S{\'e}rsic indices of $n>2$ are analogous to bulge-like galaxies, and $16.2\pm6.6$ percent of member star-forming galaxies are bulge-like, which is consistent with those in the virialized cluster XLSSC122 at $z\sim2$\cite{Noordeh2021}.  Additionally, in Extended Data Figure~\ref{fig:fige1}, we find a trend that the fraction of galaxies with $n>2$ slightly increases with decreasing radial distance. This indicates that the fraction of bulge-like galaxies in BOSS1244 may be related to the environment.



\section{SED fitting of BOSS1244-QG1 and BOSS1244-QG2}

We constrain the star formation histories (SFHs) of BOSS1244-QG1 and BOSS1244-QG2 using the multi-wavelength photometry. We use the software Fitting and Assessment of Synthetic Templates (FAST++)  to obtain the stellar properties of BOSS1244-QG1 and BOSS1244-QG2. This is a full rewriten FAST code (FAST++ v1.3) that can handle much larger parameter grids and generate models  with arbitrary SFHs. We adopt the BC03 stellar population model\cite{Bruzual2003}, the Chabrier IMF\cite{Chabrier2003}, the Calzetti dust model\cite{Calzetti2000} and solar metallicity to model the SEDs of BOSS1244-QG1 and BOSS1244-QG2.  . The redshift is fixed at $z=2.2441/2.2416$ derived from the grism spectra. We fit the models with extinction $A_{\rm V}$ ranging over $0-4$\,mag with a step of 0.1\,mag and stellar age ranging over $6.0-14.0$\,dex  in units of year with a step of 0.1\,dex. 


Stellar mass $M_{\ast}$, age ($t$),  $\tau$,  $A_{V}$ and SFR of BOSS1244-QG1 are  $5.25_{-0.24}^{+0.60}\times10^{11}$\,M$_{\odot}$, $2.24_{-0.21}^{+0.15}$\,Gyr, $281.84_{-25.96}^{+19.47}$\,Myr, $0.79_{-0.07}^{+0.09}$\,mag,  $8.32_{-0.38}^{+0.77}$\,$M_{\odot}$\,yr$^{-1}$, respectively. We also compute the SFH quantities $\left\langle \rm SFR_{\rm main} \right\rangle$, $t_{\rm 50}$ and $t_{\rm q}$ based on FAST++, where $\left\langle \rm SFR_{\rm main} \right\rangle$ is the average SFR during the shortest time interval over which 68\% of SFR took place, $t_{\rm 50}$ is the elapsed time since the galaxy had  formed 50\%  of the total stellar mass and $t_{\rm q}$ is the  elapsed time  since SFR drops below 10\%  of $\left\langle \rm SFR_{\rm main} \right\rangle$.  Here we obtain log($\left\langle \rm SFR_{\rm main} \right\rangle$)=2.95$_{-0.02}^{+0.05}$\,$M_{\odot}$\,yr$^{-1}$,  the log($t_{50}$)= 9.30\,yr and the log($t_{q}$)=8.95\,yr.  For the BOSS1244-QG2, the estimated M$_{*}$, age ($t$),  $\tau$,  $A_{V}$ and SFR refer to $1.32_{-0.09}^{+0.38}\times10^{11}$\,M$_{\odot}$, $0.832_{-0.156}^{+0.123}$\,Gyr, $87.09_{-11.24}^{+10.63}$\,Myr, $0.42_{-0.09}^{+0.61}$\,mag,  $1.05_{-0.09}^{+7.46}$\,$M_{\odot}$\,yr$^{-1}$, respectively. We obtain log($\left\langle \rm SFR_{\rm main} \right\rangle$)=2.82$_{-0.04}^{+0.13}$\,$M_{\odot}$\,yr$^{-1}$,  the log($t_{50}$)= 8.92$_{-0.12}^{+0.00}$\,yr and log($t_{q}$)=8.69$_{-0.24}^{+0.00}$\,yr. In addition, in order to examine the stellar properties of BOSS1244-QG1 and BOSS1244-QG2, we use different stellar population libraries: simple stellar populations\cite{Bruzual2003}, evolutionary population synthesis models\cite{Maraston2005} and Flexible Stellar Population Synthesis\cite{Conroy2010}.  Different SFHs including delayed exponentially declining SFH and exponentially declining SFH, three stellar initial mass functions including Salpeter\cite{Salpeter1955}, Kroupa\cite{Kroupa2001} and Chabrier\cite{Chabrier2003}, Calzetti dust law and different metallicity values (subsolar, solar and  supersolar) are selected. The calculated stellar parameters of BOSS1244-QG1 and BOSS1244-QG2 are comparable to previous estimations. We find that when the metallicity ($Z$) is set as a free parameter, the metallicities ($Z$) of BOSS1244-QG1 and BOSS1244-QG2 are in the range of $0.02-0.04$ (0.02 is the solar metallicity). This indicates that BOSS1244-QG1 and BOSS1244-QG2 are massive ($>10^{11}$\,$M_{\odot}$) and metal-rich (solar or supersolar-type) galaxies.  


In order to double-check the stellar properties of BOSS1244-QG1 and BOSS1244-QG2, we also use the software tool Bayesian Analysis of Galaxies for Physical Inference and Parameter EStimation (BAGPIPES) to estimate the stellar properties of BOSS1244-QG1 and BOSS1244-QG2. BAGPIPES\cite{Carnall2018} enables to fit observed spectroscopic and photometric SEDs with spectral  synthesis  models to derive the probability distribution functions (PDFs)  for the SFH, dust and metallicity content of each galaxy.  We find that the results measured with BAGPIPES are consistent with the results of FAST++ tool, but stellar age  changes  if different SFH models are adopted. In Extended Data Figure~\ref{fig:fige2}, we show the results of the double power-law SFH model for BOSS1244-QG1 and BOSS1244-QG2 and the probability distributions of SFRs, mass-weighted age, stellar mass, specific SFR and dust attenuation with 16th, 50th and 84th percentiles marked by the black dashed lines.  All the results are consistent with the conclusion that BOSS1244-QG1 and BOSS1244-QG2 are massive, quenched galaxies, and  BOSS1244-QG1 is older than BOSS1244-QG2. 
Note that only five photometric data points are available, so there are relatively large uncertainties in some parameters (e.g., age).  


In the left panel of Extended Data Figure~\ref{fig:fige3}, we show the relationship between stellar mass $M_{*}$  and SFR for the quenched galaxies in the blank and protocluster fields at $z=1.5-3.2$, finding no difference between quenched galaxies in the blank field and quiescent galaxies in the protoclusters. Our BOSS1244-QG1 and BOSS1244-QG2 are compared with quiescent galaxies in protoclusters or the fields at $z=2-3$.  The right panel of Extended Data Figure~\ref{fig:fige3} presents the size evolution of galaxies in (proto)clusters and the fields with stellar mass $>10^{11.3}$\,M$_{\odot}$ at $z=2.25$. Only few massive quiescent galaxies are identified in protoclusters at $z>2$. Most are found in general field, but their environments are poor explored. We find that BOSS1244-QG1 and BOSS1244-QG2 follow the size evolution of field early-type galaxies considering the relatively large dispersion, especially the size of BOSS1244-QG1 is comparable to the coeval field star-forming galaxies.  



\section{Comparison with BCGs/QGs at $z=0.1-2.0$}

BOSS1244-QG1 and BOSS1244-QG2 are two brightest galaxies in BOSS1244. In Extended Data Figure~\ref{fig:fige4}, we compare BOSS1244-QG1 and BOSS1244-QG2 with other BCGs at $z=0.1-1.8$ or quiescent galaxies in (proto)cluster at $z\sim2$ from the literature\cite{Chu2021, Noordeh2021, Zirm2012}. The absolute magnitudes of BCGs tend to become brighter with redshift; our two quiescent galaxies follow this relation considering the large dispersion. BOSS1244-QG1 is as bright as the BCG in the mature cluster XLSSC122 at $z\sim2$, and BOSS1244-QG1 and BOSS1244-QG2 are brighter than the quiescent galaxies in the mature cluster XLSSC122 and protocluster PKS1138 at $z=2.16$. The effective radii of BCGs/QGs tend to become larger with decreasing redshift. The sizes of BOSS1244-QG1 and BOSS1244-QG2 are smaller than the BCG in the mature cluster XLSSC122, but larger than most quiescent galaxies in XLSSC122 at $z\sim2$ and protocluster PKS1138 at $z=2.16$.  Their mean effective surface brightness and S{\'e}rsic index are no evolution as a function of redshift. The high S{\'e}rsic indices of BOSS1244-QG1 and BOSS1244-QG2 are consistent with the BCGs at $z<2.0$.  The entire galaxy sample follows the Kormendy relation\cite{Kormendy1977} that links the mean effective surface brightness of elliptical galaxies with their effective radii. 

Although BOSS1244-QG1 and BOSS1244-QG2 have characteristics of BCGs, they are likely to grow in size and mass. In Extended Data Figure~\ref{fig:fige5}, we show the line-of-sight velocity distribution and spatial distances of  the member galaxies relative to BOSS1244-QG2 in BOSS1244.  The line-of-sight separation between BOSS1244-QG1 and BOSS1244-QG2 is 8$\farcs$84, i.e., $\sim$72.8\,kpc at $z=2.24$. Their velocity offset is 231\,km\,s$^{-1}$. Dry mergers between them may promote their increase in size and mass.  Besides, there are two HAEs at  $z=2.250/2.227$ with stellar mass of $10^{10.3}$/$10^{11.1}$\,M$_{\odot}$ around BOSS1244-QG1 and BOSS1244-QG2 within 145\,kpc. These provide indirect evidence of interactions between them, and further support an early BCG assembly scenario via multiple mergers\cite{Kubo2021}. As time goes on, the protocluster will become richer and at the same time, BOSS1244-QG1 and BOSS1244-QG2 are expected to undergo a dry merger and create a mature BCG with a dramatically increased size. 


\section{The possible quenching mechanisms}  

The possible mechanisms for quenching, include internal processes (the feedback from active galactic nuclei (AGN) and stellar feedback, and the build-up of hot intracluster gas\cite{Fabian2012, Man2018}), as well as the external processes (such as ram pressure stripping\cite{Gunn1972, Fumagalli2014, Poggianti2017, Roberts2019}, galaxy-galaxy interactions\cite{Farouki1981}, strangulation\cite{Peng2015} and harassment\cite{Moore1996}). The latter processes mostly happen in the overdense regions. Our NIR spectroscopy of HAEs reveals that the densest region of BOSS1244 contains two substructures\cite{Shi2021}, so the quenching of BOSS1244-QG1 at $z=2.2441$ and BOSS1244-QG2 at $z=2.2416$ in the density peak of BOSS1244 may be affected by cosmological or gravitational heating effects. We note that AGN feedback cannot be ruled out. These two quiescent galaxies quenched star formation before the virialization of the cluster core, revealing that the quenching is not regulated by the processes related to virialization or hot intracluster gas. The quenching in BOSS1244-QG2 is likely a faster process than that in BOSS1244-QG1 based on the estimated stellar age. The rapid quenching may result in compact early-type galaxies with high S{\'e}rsic index and the slow quenching, instead, may be attributed to a smooth decline in the gas reservoir, likely caused by fuel starvation\cite{Belli2019}. 

Our results benefit from the high throughput and sensitivity of HST grism spectroscopy, especially for the identification of two massive quiescent galaxies. Our observations provide first direct observational evidence for the formation of BCGs at high redshifts. Our study pioneers the breakthroughs in understanding of the formation of BCGs in (proto)clusters at high redshifts promised by the James Webb Space Telescope (JWST) with the unprecedented sensitivity and resolution in imaging and spectroscopy.




\begin{thebibliography}{10}
\setcounter{enumiv}{32}
\expandafter\ifx\csname url\endcsname\relax
  \def\url#1{\texttt{#1}}\fi
\expandafter\ifx\csname urlprefix\endcsname\relax\def\urlprefix{URL }\fi
\providecommand{\bibinfo}[2]{#2}
\providecommand{\eprint}[2][]{\url{#2}}

\bibitem{Brammer2021}
\bibinfo{author}{Gabe Brammer \& Jasleen Matharu.}
\newblock \bibinfo{title}{{gbrammer/grizli: Release 2021 (1.3.2)}}.
\newblock \emph{\bibinfo{journal}{Zenodo}} \textbf{\bibinfo{volume}{}},
  \bibinfo{pages}{https://doi.org/10.5281/zenodo.5012699} (\bibinfo{year}{2021}).

\bibitem{Wang2021}
\bibinfo{author}{Wang X., Li Z., Cai Z., Shi D.~D., Fan X., Zheng X.~Z., Bian F., et al.}
\newblock \bibinfo{title}{{The Mass-Metallicity Relation at Cosmic Noon in Overdense Environments: First Results from the MAMMOTH-Grism HST Slitless Spectroscopic Survey}}.
\newblock \emph{\bibinfo{journal}{\apj}} \textbf{\bibinfo{volume}{926}},
  \bibinfo{pages}{70} (\bibinfo{year}{2022}).



\bibitem{Schirmer2013}
\bibinfo{author}{Schirmer M.}
\newblock \bibinfo{title}{{THELI: Convenient Reduction of Optical, Near-infrared, and Mid-infrared Imaging Data}}.
\newblock \emph{\bibinfo{journal}{\apjs}} \textbf{\bibinfo{volume}{209}},
  \bibinfo{pages}{21} (\bibinfo{year}{2013}).

\bibitem{Bertin1996}
\bibinfo{author}{Bertin E., Arnouts S.}
\newblock \bibinfo{title}{{SExtractor: Software for source extraction}}.
\newblock \emph{\bibinfo{journal}{\aaps}} \textbf{\bibinfo{volume}{117}},
  \bibinfo{pages}{393-404} (\bibinfo{year}{1996}).

\bibitem{Bertin2006}
\bibinfo{author}{Bertin E., Arnouts S.}
\newblock \bibinfo{title}{{Automatic Astrometric and Photometric Calibration with SCAMP}}.
\newblock \emph{\bibinfo{journal}{Astronomical Society of the Pacific Conference Series}} \textbf{\bibinfo{volume}{351}},
  \bibinfo{pages}{112} (\bibinfo{year}{2006}).

\bibitem{Bertin2002}
\bibinfo{author}{Bertin E., Mellier Y., Radovich M., Missonnier G., Didelon P., Morin B.}
\newblock \bibinfo{title}{{The TERAPIX Pipeline}}.
\newblock \emph{\bibinfo{journal}{Astronomical Society of the Pacific Conference Series}} \textbf{\bibinfo{volume}{281}},
  \bibinfo{pages}{228} (\bibinfo{year}{2002}).


\bibitem{Willis2020}
\bibinfo{author}{Willis, J.~P., Canning, R.~E.~A., Noordeh, E.~S., et al.}
\newblock \bibinfo{title}{{Spectroscopic confirmation of a mature galaxy cluster at a redshift of 2}}.
\newblock \emph{\bibinfo{journal}{\nat}} \textbf{\bibinfo{volume}{577}},
 \bibinfo{pages}{39-41} (\bibinfo{year}{2020}).


\bibitem{Peng2010}
\bibinfo{author}{Peng C.~Y., Ho L.~C., Impey C.~D., Rix H.-W.}
\newblock \bibinfo{title}{{ Detailed Decomposition of Galaxy Images. II. Beyond Axisymmetric Models}}.
\newblock \emph{\bibinfo{journal}{\apj}} \textbf{\bibinfo{volume}{139}},
  \bibinfo{pages}{2097-2129} (\bibinfo{year}{2010}).



\bibitem{Balogh1999}
\bibinfo{author}{Balogh, M.~L., Morris, S.~L., Yee, H.~K.~C., et al.}
\newblock \bibinfo{title}{{Differential Galaxy Evolution in Cluster and Field Galaxies at $z\sim0.3$}}.
\newblock \emph{\bibinfo{journal}{\apj}} \textbf{\bibinfo{volume}{527}},
  \bibinfo{pages}{54-79} (\bibinfo{year}{1999}).



\bibitem{Cucciati2018}
\bibinfo{author}{Balogh, M.~L., Morris, S.~L., Yee, H.~K.~C., et al.}
\newblock \bibinfo{title}{{The progeny of a cosmic titan: a massive multi-component proto-supercluster in formation at $z = 2.45$ in VUDS}}.
\newblock \emph{\bibinfo{journal}{\aap}} \textbf{\bibinfo{volume}{619}},
  \bibinfo{pages}{A49} (\bibinfo{year}{2018}).




\bibitem{Maraston2005}
\bibinfo{author}{Maraston C.}
\newblock \bibinfo{title}{{Evolutionary population synthesis: models, analysis of the ingredients and application to high-z galaxies}}.
\newblock \emph{\bibinfo{journal}{\mnras}} \textbf{\bibinfo{volume}{362}},
  \bibinfo{pages}{799-825} (\bibinfo{year}{2005}).


\bibitem{Conroy2010}
\bibinfo{author}{Conroy C., Gunn J.~E.}
\newblock \bibinfo{title}{{The Propagation of Uncertainties in Stellar Population Synthesis Modeling. III. Model Calibration, Comparison, and Evaluation}}.
\newblock \emph{\bibinfo{journal}{\apj}} \textbf{\bibinfo{volume}{712}},
  \bibinfo{pages}{833-857} (\bibinfo{year}{2010}).

\bibitem{Bruzual2003}
\bibinfo{author}{Bruzual G., Charlot S.}
\newblock \bibinfo{title}{{Stellar population synthesis at the resolution of 2003}}.
\newblock \emph{\bibinfo{journal}{\mnras}} \textbf{\bibinfo{volume}{344}},
  \bibinfo{pages}{1000-1028} (\bibinfo{year}{2003}).


\bibitem{Chabrier2003}
\bibinfo{author}{Chabrier G.}
\newblock \bibinfo{title}{{Stellar population synthesis at the resolution of 2003}}.
\newblock \emph{\bibinfo{journal}{\pasp}} \textbf{\bibinfo{volume}{115}},
  \bibinfo{pages}{763-795} (\bibinfo{year}{2003}).

\bibitem{Calzetti2000}
\bibinfo{author}{Calzetti D., Armus L., Bohlin R.~C., Kinney A.~L., Koornneef J., Storchi-Bergmann T.}
\newblock \bibinfo{title}{{ The Dust Content and Opacity of Actively Star-forming Galaxies}}.
\newblock \emph{\bibinfo{journal}{\apj}} \textbf{\bibinfo{volume}{533}},
  \bibinfo{pages}{682-695} (\bibinfo{year}{2000}).
  
\bibitem{Salpeter1955}
\bibinfo{author}{Salpeter E.~E.}
\newblock \bibinfo{title}{{The Luminosity Function and Stellar Evolution}}.
\newblock \emph{\bibinfo{journal}{\apj}} \textbf{\bibinfo{volume}{121}},
  \bibinfo{pages}{161} (\bibinfo{year}{1955}).

\bibitem{Kroupa2001}
\bibinfo{author}{Kroupa P.}
\newblock \bibinfo{title}{{On the variation of the initial mass function}}.
\newblock \emph{\bibinfo{journal}{\mnras}} \textbf{\bibinfo{volume}{322}},
  \bibinfo{pages}{231-246} (\bibinfo{year}{2001}).

\bibitem{Carnall2018}
\bibinfo{author}{Carnall A.~C., McLure R.~J., Dunlop J.~S., Dav{\'e} R.}
\newblock \bibinfo{title}{{Inferring the star formation histories of massive quiescent galaxies with BAGPIPES: evidence for multiple quenching mechanisms}}.
\newblock \emph{\bibinfo{journal}{\mnras}} \textbf{\bibinfo{volume}{480}},
  \bibinfo{pages}{4379-4401} (\bibinfo{year}{2001}).

\bibitem{Chu2021}
\bibinfo{author}{Chu A., Durret F., M{\'a}rquez I.}
\newblock \bibinfo{title}{{Physical properties of brightest cluster galaxies up to redshift 1.80 based on HST data}}.
\newblock \emph{\bibinfo{journal}{\aap}} \textbf{\bibinfo{volume}{649}},
  \bibinfo{pages}{A42} (\bibinfo{year}{2021}).

\bibitem{Noordeh2021}
\bibinfo{author}{Noordeh E., Canning R.~E.~A., Willis J.~P., Allen S.~W., Mantz A., Stanford S.~A., Brammer G.}
\newblock \bibinfo{title}{{Quiescent galaxies in a virialized cluster at redshift 2: evidence for accelerated size growth}}.
\newblock \emph{\bibinfo{journal}{\mnras}} \textbf{\bibinfo{volume}{507}},
  \bibinfo{pages}{5272-5280} (\bibinfo{year}{2021}).

\bibitem{Zirm2012}
\bibinfo{author}{Zirm A.~W., Toft S., Tanaka M.}
\newblock \bibinfo{title}{{Internal Structure of Protocluster Galaxies: Accelerated Structural Evolution in Overdense Environments}}?
\newblock \emph{\bibinfo{journal}{\mnras}} \textbf{\bibinfo{volume}{744}},
  \bibinfo{pages}{181} (\bibinfo{year}{2012}).

\bibitem{Kormendy1977}
\bibinfo{author}{Kormendy J.}
\newblock \bibinfo{title}{{Brightness distributions in compact and normal galaxies. II. Structure parameters of the spheroidal component}}.
\newblock \emph{\bibinfo{journal}{\apj}} \textbf{\bibinfo{volume}{218}},
  \bibinfo{pages}{333-346} (\bibinfo{year}{1977}).

\bibitem{Kubo2021}
\bibinfo{author}{Kubo M., Umehata H., Matsuda Y., Kajisawa M., Steidel C.~C., Yamada T., Tanaka I., et al.}
\newblock \bibinfo{title}{{A Massive Quiescent Galaxy Confirmed in a Protocluster at $z = 3.09$}}.
\newblock \emph{\bibinfo{journal}{\apj}} \textbf{\bibinfo{volume}{919}},
  \bibinfo{pages}{6} (\bibinfo{year}{2021}).

\bibitem{Kalita2021}
\bibinfo{author}{Kalita B.~S., Daddi E., D'Eugenio C., Valentino F., Rich R.~M., G{\'o}mez-Guijarro C., Coogan R.~T., et al.}
\newblock \bibinfo{title}{{An Ancient Massive Quiescent Galaxy Found in a Gas-rich $z\sim3$ Group}}.
\newblock \emph{\bibinfo{journal}{\apjl}} \textbf{\bibinfo{volume}{917}},
  \bibinfo{pages}{L17} (\bibinfo{year}{2021}).

\bibitem{Kriek2009}
\bibinfo{author}{Kriek M., van Dokkum P.~G., Labb{\'e} I., Franx M., Illingworth G.~D., Marchesini D., Quadri R.~F.}
\newblock \bibinfo{title}{{An Ultra-Deep Near-Infrared Spectrum of a Compact Quiescent Galaxy at $z = 2.2$}}.
\newblock \emph{\bibinfo{journal}{\apj}} \textbf{\bibinfo{volume}{700}},
  \bibinfo{pages}{221-231} (\bibinfo{year}{2009}).

\bibitem{Schreiber2018}
\bibinfo{author}{Schreiber C., Glazebrook K., Nanayakkara T., Kacprzak G.~G., Labb{\'e} I., Oesch P., Yuan T., et al.}
\newblock \bibinfo{title}{{Near infrared spectroscopy and star-formation histories of $3 {\ensuremath{\leq}} z {\ensuremath{\leq}} 4$ quiescent galaxies}}.
\newblock \emph{\bibinfo{journal}{\aap}} \textbf{\bibinfo{volume}{618}},
  \bibinfo{pages}{A85} (\bibinfo{year}{2018}).


\bibitem{Newman2014}
\bibinfo{author}{Newman, A.~B., Ellis, R.~S., Andreon, S., et al.}
\newblock \bibinfo{title}{{Spectroscopic Confirmation of the Rich $z = 1.80$ Galaxy Cluster JKCS 041 using the WFC3 Grism: Environmental Trends in the Ages and Structure of Quiescent Galaxies}}.
\newblock \emph{\bibinfo{journal}{\apj}} \textbf{\bibinfo{volume}{788}},
  \bibinfo{pages}{51} (\bibinfo{year}{2014}).


\bibitem{Belli2014}
\bibinfo{author}{Belli, S., Newman, A.~B., Ellis, R.~S., et al.}
\newblock \bibinfo{title}{{MOSFIRE Absorption Line Spectroscopy of $z > 2$ Quiescent Galaxies: Probing a Period of Rapid Size Growth}}.
\newblock \emph{\bibinfo{journal}{\apjl}} \textbf{\bibinfo{volume}{788}},
  \bibinfo{pages}{L29} (\bibinfo{year}{2014}).
  
\bibitem{Belli2017}
\bibinfo{author}{Belli, S., Newman, A.~B., \& Ellis, R.~S.}
\newblock \bibinfo{title}{{MOSFIRE Spectroscopy of Quiescent Galaxies at $1.5 < z < 2.5$. I. Evolution of Structural and Dynamical Properties}}.
\newblock \emph{\bibinfo{journal}{\apj}} \textbf{\bibinfo{volume}{834}},
  \bibinfo{pages}{18} (\bibinfo{year}{2017}).
  
\bibitem{Glazebrook2017}
\bibinfo{author}{Glazebrook, K., Schreiber, C., Labb{\'e}, I., et al.}
\newblock \bibinfo{title}{{A massive, quiescent galaxy at a redshift of 3.717}}.
\newblock \emph{\bibinfo{journal}{\nat}} \textbf{\bibinfo{volume}{544}},
  \bibinfo{pages}{71-74} (\bibinfo{year}{2017}).
  
  \bibitem{Stockmann2020}
\bibinfo{author}{Stockmann, M., Toft, S., Gallazzi, A., et al.}
\newblock \bibinfo{title}{{X-shooter Spectroscopy and HST Imaging of 15 Massive Quiescent Galaxies at $z$ {\ensuremath{\gtrsim}} 2}}.
\newblock \emph{\bibinfo{journal}{\apj}} \textbf{\bibinfo{volume}{888}},
  \bibinfo{pages}{4} (\bibinfo{year}{2020}).

  \bibitem{Lustig2021}
\bibinfo{author}{Lustig, P., Strazzullo, V., D'Eugenio, C., et al.}
\newblock \bibinfo{title}{{Compact, bulge-dominated structures of spectroscopically confirmed quiescent galaxies at $z\sim3$}}.
\newblock \emph{\bibinfo{journal}{\mnras}} \textbf{\bibinfo{volume}{501}},
  \bibinfo{pages}{2659-2676} (\bibinfo{year}{2021}).

  \bibitem{Forrest2022}
\bibinfo{author}{Forrest, B., Wilson, G., Muzzin, A., et al.}
\newblock \bibinfo{title}{{MAGAZ3NE: High Stellar Velocity Dispersions for Ultra-Massive Quiescent Galaxies at $z\sim3$}}.
\newblock \emph{\bibinfo{journal}{\apj}} \textbf{\bibinfo{volume}{938}},
  \bibinfo{pages}{109} (\bibinfo{year}{2022}).

\bibitem{Kubo2018}
\bibinfo{author}{Kubo, M., Tanaka, M., Yabe, K., et al.}
\newblock \bibinfo{title}{{The Rest-frame Optical Sizes of Massive Galaxies with Suppressed Star Formation at $z \sim4$}}.
\newblock \emph{\bibinfo{journal}{\apj}} \textbf{\bibinfo{volume}{867}},
  \bibinfo{pages}{1} (\bibinfo{year}{2018}).
  





\bibitem{Fabian2012}
\bibinfo{author}{Fabian A.~C.}
\newblock \bibinfo{title}{{Observational Evidence of Active Galactic Nuclei Feedback}}.
\newblock \emph{\bibinfo{journal}{\araa}} \textbf{\bibinfo{volume}{50}},
  \bibinfo{pages}{455-489} (\bibinfo{year}{2012}).

\bibitem{Man2018}
\bibinfo{author}{Man A., Belli S.}
\newblock \bibinfo{title}{{Star formation quenching in massive galaxies}}.
\newblock \emph{\bibinfo{journal}{Nature Astronomy}} \textbf{\bibinfo{volume}{2}},
  \bibinfo{pages}{695-697} (\bibinfo{year}{2018}).

\bibitem{Gunn1972}
\bibinfo{author}{Gunn J.~E., Gott J.~R.}
\newblock \bibinfo{title}{{On the Infall of Matter Into Clusters of Galaxies and Some Effects on Their Evolution}}.
\newblock \emph{\bibinfo{journal}{\apj}} \textbf{\bibinfo{volume}{176}},
  \bibinfo{pages}{1} (\bibinfo{year}{1972}).

\bibitem{Fumagalli2014}
\bibinfo{author}{Fumagalli M., Fossati M., Hau G.~K.~T., Gavazzi G., Bower R., Sun M., Boselli A.}
\newblock \bibinfo{title}{{MUSE sneaks a peek at extreme ram-pressure stripping events - I. A kinematic study of the archetypal galaxy ESO137-001}}.
\newblock \emph{\bibinfo{journal}{\mnras}} \textbf{\bibinfo{volume}{445}},
  \bibinfo{pages}{4335-4344} (\bibinfo{year}{2014}).

\bibitem{Poggianti2017}
\bibinfo{author}{Poggianti B.~M., Moretti A., Gullieuszik M., Fritz J., Jaff{\'e} Y., Bettoni D., Fasano G., et al.}
\newblock \bibinfo{title}{{GASP. I. Gas Stripping Phenomena in Galaxies with MUSE}}.
\newblock \emph{\bibinfo{journal}{\apj}} \textbf{\bibinfo{volume}{844}},
  \bibinfo{pages}{48} (\bibinfo{year}{2017}).

\bibitem{Roberts2019}
\bibinfo{author}{Roberts I.~D., Parker L.~C., Brown T., Joshi G.~D., Hlavacek-Larrondo J., Wadsley J.}
\newblock \bibinfo{title}{{Quenching Low-mass Satellite Galaxies: Evidence for a Threshold ICM Density}}.
\newblock \emph{\bibinfo{journal}{\apj}} \textbf{\bibinfo{volume}{873}},
  \bibinfo{pages}{42} (\bibinfo{year}{2019}).

\bibitem{Farouki1981}
\bibinfo{author}{Farouki R., Shapiro S.~L.}
\newblock \bibinfo{title}{{Computer simulations of environmental influences on galaxy evolution in dense clusters. II - Rapid tidal encounters}}.
\newblock \emph{\bibinfo{journal}{\apj}} \textbf{\bibinfo{volume}{243}},
  \bibinfo{pages}{32-41} (\bibinfo{year}{1981}).


\bibitem{Peng2015}
\bibinfo{author}{Peng Y., Maiolino R., Cochrane R.}
\newblock \bibinfo{title}{{ Strangulation as the primary mechanism for shutting down star formation in galaxies}}.
\newblock \emph{\bibinfo{journal}{\nat}} \textbf{\bibinfo{volume}{521}},
  \bibinfo{pages}{192-195} (\bibinfo{year}{2015}).


\bibitem{Moore1996}
\bibinfo{author}{Moore B., Katz N., Lake G., Dressler A., Oemler A.}
\newblock \bibinfo{title}{{ Galaxy harassment and the evolution of clusters of galaxies}}.
\newblock \emph{\bibinfo{journal}{\nat}} \textbf{\bibinfo{volume}{379}},
  \bibinfo{pages}{613-616} (\bibinfo{year}{1996}).


\bibitem{Belli2019}
\bibinfo{author}{Belli S., Newman A.~B., Ellis R.~S.}
\newblock \bibinfo{title}{{MOSFIRE Spectroscopy of Quiescent Galaxies at $1.5 < z < 2.5$. II. Star Formation Histories and Galaxy Quenching}}.
\newblock \emph{\bibinfo{journal}{\apj}} \textbf{\bibinfo{volume}{874}},
  \bibinfo{pages}{17} (\bibinfo{year}{2019}).

\end{thebibliography}

\setcounter{figure}{0}

\begin{figure*}[!ht]
\setlength{\abovecaptionskip}{0pt}
\begin{center}
\includegraphics[trim=0mm 0mm 0mm 1mm,clip,height=0.45\textwidth]{BCG_evolution.png}
\caption{{\bf Evolution of overdense regions in BOSS1244.} The evolution is calculated in a linear regime. The blue and red curves are the galaxy overdensity ($\delta_{\rm g}$) of 22.9 and 40.4, respectively. The horizontal dashed lines mark $\delta_{\rm ta}\approx1.062$, $\delta_{\rm vir}\approx1.580$ and $\delta_{\rm c}\approx1.686$. The vertical line is merger redshift of BOSS1244-QG1 and BOSS1244-QG2. A mature BCG though dry major merger could be generated before the virialization of a cluster core.}
\label{fig:fige6}
\end{center}
\end{figure*}



\begin{figure*}[!ht]
\setlength{\abovecaptionskip}{0pt}
\begin{center}
\includegraphics[trim=0mm 0mm 0mm 1mm,clip,height=0.25\textwidth]{1244_mag_density_v1.png}
\includegraphics[trim=0mm 0mm 0mm 1mm,clip,height=0.25\textwidth]{1244_size_density_v1.png}
\includegraphics[trim=0mm 0mm 0mm 1mm,clip,height=0.25\textwidth]{1244_mu_density_v1.png}
\includegraphics[trim=0mm 0mm 0mm 1mm,clip,height=0.25\textwidth]{1244_sersic_density_v1.png}
\includegraphics[trim=0mm 0mm 0mm 1mm,clip,height=0.28\textwidth]{1244_sersic2_density_v1.png}
\caption{{\bf Scaling relations between absolute magnitude with $K$-correction (Upper left), effective radius $r_{\rm e}$ (Upper middle), mean effective surface brightness $<\mu_{eff}>$ (Upper right), S{\'e}rsic index $n$ (Bottom left), the fraction of $n\geq2$ (Bottom right) in F160W and spatial distance of member galaxies relative to BOSS1244-QG2.} The light-blue points are the confirmed galaxies at $z=2.223-2.255$. The red filled hexagons are BOSS1244-QG1 and BOSS1244-QG2, which are marked by QG1 and QG2. The red line in the bottom right is the linear best fitting in the relationship between the fraction of $n\geq2$ and radial distance. } 
\label{fig:fige1}
\end{center}
\end{figure*}


\begin{figure*}[!ht]
\setlength{\abovecaptionskip}{0pt}
\begin{center}
\includegraphics[trim=0mm 0mm 0mm 1mm,clip,height=0.54\textwidth]{1244_qs1_SED_v1.png}
\includegraphics[trim=0mm 0mm 0mm 1mm,clip,height=0.54\textwidth]{1244_qs2_SED_v1.png}
\caption{{\bf SED fitting of BOSS1244-QG1 and BOSS1244-QG2 using double power-law model.}  The blue points are the input photometry of galaxies, the orange points are the posterior fitted photometry and the orange line is posterior SED. The bottom panels show the probability distributions of the star formation rate (SFR), mass-weighted age, stellar mass, specific SFR with the 16th, 50th and 84th percentiles marked by the dashed lines. The inner panel in BOSS1244-QG1/BOSS1244-QG2 shows the GALFIT model to the two quiescent galaxies. The three columns from left to right are the original image in F125W/F160W, the GALFIT model, and the residual image. S{\'e}rsic index, effective radius, and axis ratio are marked in model image. The size of each postage stamp image is $10\arcsec \times 10\arcsec$.} 
\label{fig:fige2}
\end{center}
\end{figure*}

\begin{figure*}[!ht]
\setlength{\abovecaptionskip}{0pt}
\begin{center}
\includegraphics[trim=0mm 0mm 0mm 1mm,clip,height=0.35\textwidth]{1244_main_sequence.png}
\includegraphics[trim=0mm 0mm 0mm 1mm,clip,height=0.35\textwidth]{size_evolution.png}

\caption{{\bf Left: the relationship between stellar mass and SFR.} The red filled hexagons are BOSS1244-QG1 and BOSS1244-QG2. The blue symbols are the quiescent galaxies in protocluster SSA22\cite{Kubo2021} at $z=3.09$ and galaxy group RO-1001\cite{Kalita2021} at $z=2.91$. The black symbols are the quenched galaxies in the random fields\cite{Kriek2009, Schreiber2018, Belli2019}. The dashed lines show the sSFR of 10$^{-10}$\,yr$^{-1}$, 10$^{-11}$\,yr$^{-1}$ and 10$^{-12}$\,yr$^{-1}$.  {\bf Right: the size evolution of massive ($\sim10^{11-11.5}$\,M$_{\odot}$) quiescent or BCG galaxies.}  The red filled hexagons are BOSS1244-QG1 and BOSS1244-QG2. The black  points are the BCGs\cite{Chu2021} at $z=0.1-1.8$. The color-coded symbols are the quiescent galaxies in (proto)clusters\cite{Newman2014, Zirm2012, Kubo2021, Noordeh2021}, and the grey symbols are the quiescent galaxies in the fields\cite{Belli2014, Belli2017, Glazebrook2017, Stockmann2020, Lustig2021, Forrest2022}. The black and red lines are the size evolution of quiescent galaxies\cite{vanderWel2014,Kubo2018}, the blue line marks the size evolution of star-forming galaxies\cite{vanderWel2014}.} 
\label{fig:fige3}
\end{center}
\end{figure*}



\begin{figure*}[!ht]
\setlength{\abovecaptionskip}{0pt}
\begin{center}
\includegraphics[trim=0mm 0mm 0mm 1mm,clip,height=0.35\textwidth]{BCG_magabs_redshift_v1.png}
%\includegraphics[trim=0mm 0mm 0mm 1mm,clip,height=0.25\textwidth]{BCG_size_redshift.png}
\includegraphics[trim=0mm 0mm 0mm 1mm,clip,height=0.35\textwidth]{BCG_mu_redshift_v1.png}
\includegraphics[trim=0mm 0mm 0mm 1mm,clip,height=0.35\textwidth]{BCG_n_redshift_v1.png}
\includegraphics[trim=0mm 0mm 0mm 1mm,clip,height=0.35\textwidth]{BCG_mu_size_v1.png}
\caption{{\bf Absolute magnitude with $K$-correction (top-left), mean effective surface brightness (top-right), S{\'e}rsic index (bottom-left) as a function of redshift, and the Kormendy relation (bottom-right).}  The black points are the BCGs at $z=0.1-1.8$, the blue points are the quiescent galaxies and a marked BCG in the mature cluster XLSSC122 at $z\sim2$\cite{Chu2021}. The green points are the quiescent galaxies selected from protocluster PKS1138 at $z=2.16$\cite{Noordeh2021}. The red filled hexagons are BOSS1244-QG1 and BOSS1244-QG2.} 
\label{fig:fige4}
\end{center}
\end{figure*}


\begin{figure*}[!ht]
\setlength{\abovecaptionskip}{0pt}
\begin{center}
\includegraphics[trim=0mm 0mm 0mm 1mm,clip,height=0.45\textwidth]{1244_velocity_density_v1.png}
\caption{{\bf The line-of-sight relative velocity and projection distance relative to BOSS1244-QG2.}  The red filled hexagons are BOSS1244-QG1 and BOSS1244-QG2 and blue points are the confirmed member galaxies in BOSS1244. The blue solid curve is the escape velocity for an Navarro–Frenk–White (NFW) halo with a halo mass of $10^{13.0}$\,M$_{\odot}$. The blue dashed line is the projection escape velocity (the velocity and distance are divided with $\sqrt{3}$ and $\sqrt{1.5}$ , respectively.). }
\label{fig:fige5}
\end{center}
\end{figure*}





\end{document}

