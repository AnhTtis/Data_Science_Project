\section{Problem Formulation}
We first define the keyphrase extraction (KPE) and keyphrase generation (KPG) task and then formulate the keyphrase evaluation process. 

\subsection{Keyphrase Extraction and Generation}
We represent an example for KPE or KPG as a tuple \textbf{$(\mathcal{X},\mathcal{Y})$}. $\mathcal{X}$ is an input document. $\mathcal{Y}=\{y_1,...,y_n\}$ is a \textit{set} of $n$ reference keyphrases written by human. Following \citet{boudin-gallina-2021-redefining}, we classify $y_i$ is as a \textit{present keyphrase} if it matches contiguous sequences of words of $\mathcal{X}$ after stemming or an \textit{absent keyphrase} otherwise. The keyphrase generation task assumes $\mathcal{Y}$ to be a mixture of present and absent keyphrases. As a special case of keyphrase generation, keyphrase extraction only allows present keyphrases in $\mathcal{Y}$ \citep{turney2000learning}.

\subsection{Keyphrase Evaluation}
\label{problem-formulation}
The \textit{keyphrase evaluation} process can be viewed as mapping a 4-element tuple $(\mathcal{X}, \mathcal{Y}, \mathcal{P}, \mathcal{C})$ to a real number via a function $f$. $\mathcal{P}=\{p_1,...,p_m\}$ is a \textit{set} of $m$ predictions made by a model $\mathcal{M}$ on $\mathcal{X}$. $\mathcal{C}$ is a corpus that represents the domain of interest. We further clarify the key terminologies and assumptions in this formulation:

First, $\mathcal{Y}$ and $\mathcal{P}$ are represented as \textit{sets} for generality: many KPG methods do not rank the predictions \citep{ye-etal-2021-one2set} or do not assume that the output order encodes a phrase's quality \citep{yuan-etal-2020-one}.

Second, diverging from the commonly followed \citet{meng-etal-2017-deep} and \citet{yuan-etal-2020-one}, the formulation does not separately evaluate present and absent keyphrases. This decision is essential to enable semantic-based evaluation.

Third, $\mathcal{C}$ is introduced because the domain is critical to determine the quality of a keyphrase in many cases. For example, "sports games" can be an informative keyphrase in the news domain but not in the sports news domain\footnote{\citet{tomokiyo-hurst-2003-language} discuss more examples.}. In this paper, we evaluate this property via retrieval-based evaluation where keyphrases are used for indexing a document among an in-domain corpus (\cref{utility-def}). 

Finally, we note that $f$ only evaluates a certain \textit{aspect} of $\mathcal{M}$. This paper proposes a suite of 4 distinct evaluation aspects with unique designs of $f$ (\cref{section-framework}). We will call $f$ \textit{reference-free} if its calculation is independent of $\mathcal{Y}$, or \textit{reference-based} otherwise. The final score of evaluating $\mathcal{M}$ with $f$ is averaged over a set of testing documents. 


 