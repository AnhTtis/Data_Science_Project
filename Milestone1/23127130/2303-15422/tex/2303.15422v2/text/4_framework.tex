\section{A Fine-grained Evaluation Framework}
\label{section-framework}
We introduce \textsc{KPEval}, a fine-grained evaluation framework for KPE and KPG systems. \textsc{KPEval} incorporates four crucial aspects:

\begin{compactenum}
    \item \textbf{{Saliency}}: $\mathcal{P}$ covers all the salient keyphrases for $\mathcal{X}$. We use human-curated $\mathcal{Y}$ to represent the set of desired salient keyphrases.
    \item \textbf{{Faithfulness}}: $p_i$ is a concept that is semantically grounded in $\mathcal{X}$.
    \item \textbf{{Diversity}}: $\mathcal{P}$ includes diverse keyphrases with minimal repetitions.
    \item \textbf{{Utility}}: $\mathcal{P}$ is able to facilitate downstream applications, such as indexing documents to improve information retrieval performance.
\end{compactenum}



\setlength{\tabcolsep}{3pt}
\begin{table}
    \centering
    \resizebox{\linewidth}{!}{%
    \begin{tabular}{l | c c c c}
    \hline
     & \texttt{KP-Set} & \texttt{Input} & \texttt{Reference} & \texttt{Corpus}  \\
    \hline
    \textbf{{Reference Agreement}} & \cmark &  & \cmark &  \\
    \textbf{{Faithfulness}} &  & \cmark &  &  \\
    \textbf{{Diversity}} & \cmark &  &  &  \\
    \textbf{{Utility}} & \cmark & \cmark &  & \cmark \\
    \hline
    \end{tabular}
    }
    \vspace{-2mm}
    \caption{Assumptions of \textsc{KPEval}'s aspects: whether they operate on a set of keyphrases (\texttt{KP-Set}) and whether they require input, reference, or a corpus.}
    \label{tab:kp-sys-property-assumptions}
    \vspace{-4mm}
\end{table}


We note that by design, these aspects integrate the ideas developed in previous literature. For example, faithfulness and saliency reflect the commonly discussed "informativeness" \citep{tomokiyo-hurst-2003-language}. In addition, diversity \citep{bahuleyan-el-asri-2020-diverse} and retrieval-based utility \citep{boudin-gallina-2021-redefining} have been advocated by previous works. Building upon these studies, \textsc{KPEval} provides a unified perspective and aims to advance the evaluation methodology for these aspects. Table \ref{tab:kp-sys-property-assumptions} further outlines the assumptions of these aspects: whether they are evaluated on single phrases and whether  $\mathcal{X}$, $\mathcal{Y}$, or $\mathcal{C}$ is needed for evaluation. Figure \ref{main-framework} illustrates the evaluation design for each aspect, which we will introduce next. 


