% \section{A Fine-grained Evaluation Framework}
\section{\textsc{KPEval}: Evaluation Aspects}
\label{section-framework}

We introduce \textsc{KPEval}, a fine-grained framework for keyphrase evaluation. \textsc{KPEval} posits to evaluate $\mathcal{P}$'s quality across four crucial aspects:

\begin{compactenum}
    \item \textbf{{Reference Agreement}}: Evaluates the extent to which $\mathcal{P}$ aligns with human-annotated $\mathcal{Y}$.
    \item \textbf{{Faithfulness}}: Determines whether each $p_i$ in $\mathcal{P}$ is semantically grounded in $\mathcal{X}$.
    \item \textbf{{Diversity}}: Assesses whether $\mathcal{P}$ includes diverse keyphrases with minimal repetitions.
    \item \textbf{{Utility}}:  Measures the potential of $\mathcal{P}$ to enhance downstream applications, such as document indexing for improved IR performance.
\end{compactenum}

Table \ref{tab:kp-sys-property-assumptions} outlines the assumptions of the evaluated aspects: whether they are calculated on a set of phrases and whether $\mathcal{X}$, $\mathcal{Y}$, or $\mathcal{C}$ is needed for evaluation. By design, these aspects have deep groundings in the previous literature. Faithfulness and reference agreement can be seen as different definitions of informativeness: the former enforces the information of $p_i$ to be contained in $\mathcal{X}$, while the latter measures $\mathcal{P}$'s coverage of $\mathcal{X}$'s salient information with respect to a background domain \citep{tomokiyo-hurst-2003-language}. Diversity \citep{bahuleyan-el-asri-2020-diverse} and IR-based utility \citep{boudin-gallina-2021-redefining} reflect major efforts to move beyond reference-based evaluation. Building upon these works, \textsc{KPEval} aims to provide a unified perspective and to advance the evaluation methodology. Figure \ref{main-framework} illustrates the evaluation design for each aspect, which we will introduce next.


\setlength{\tabcolsep}{3pt}
\begin{table}
    \centering
    \resizebox{\linewidth}{!}{%
    \begin{tabular}{l | c c c c}
    \hline
     & \texttt{KP-Set} & \texttt{Input} & \texttt{Reference} & \texttt{Corpus}  \\
    \hline
    \textbf{{Reference Agreement}} & \cmark &  & \cmark &  \\
    \textbf{{Faithfulness}} &  & \cmark &  &  \\
    \textbf{{Diversity}} & \cmark &  &  &  \\
    \textbf{{Utility}} & \cmark & \cmark &  & \cmark \\
    \hline
    \end{tabular}
    }
    \vspace{-2mm}
    \caption{Assumptions of \textsc{KPEval}'s aspects: whether they operate on a set of keyphrases (\texttt{KP-Set}) and whether they require input, reference, or a corpus.}
    \label{tab:kp-sys-property-assumptions}
    \vspace{-4mm}
\end{table}
