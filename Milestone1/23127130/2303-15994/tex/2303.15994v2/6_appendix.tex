\appendix

To further substantiate the effectiveness of our proposed method, we provide additional experimental results in the supplementary material that could not be included in the paper due to space limitations.
The supplementary material includes more experimental results on VG-150.

\section{More results on VG-150}
We present an extension of our previous results on scene graph detection (SGDet), the most challenging task in the VG-150 \cite{krishna2017visual} dataset, by including comparative evaluations on two simplified tasks: predicate (relation) classification (PredCls) and scene graph classification (SGCls).
PredCls involves inferring the relations between objects while assuming prior knowledge of their location and category information.
On the other hand, SGCls entails the prediction of the category of an object and the relations between objects, given only their location information.
We provide the detailed results of these tasks in the Tab. \ref{table:vg_main_results_predcls} and Tab. \ref{table:vg_main_results_sgcls}.

These results once again demonstrate that our method can improve the performance of low-frequency relations while also taking into account the performance of high-frequency relations.
It also shows that the HiLo framework is a general technique that yields systematic improvements in both the panoptic scene graph generation (PSG) and scene graph generation (SGG) tasks.

% \newpage

\begin{table}[h!]\small
    \centering
    \begin{tabular}{l|cc}
    \hline
        ~  & \multicolumn{2}{c}{Predicate Classification} \\
        \cline{2-3}
        Method & R/mR@50 & R/mR@100 \\
        \hline
        MOTIF \cite{zellers2018neural} & \textbf{64.0} / 15.2 & \textbf{66.0} / 16.2 \\
        \quad +IETrans \cite{zhang2022fine} & 54.7 / 30.9 & 56.7 / 33.6 \\
        \quad \textbf{+HiLo (ours)} & 53.6 / \textbf{33.6} & 55.5 / \textbf{36.4}  \\
        \hline
        VCTree \cite{tang2019learning} & \textbf{64.5} / 16.3 & \textbf{66.5} / 17.7 \\
        \quad +IETrans \cite{zhang2022fine} & 53.0 / 30.3 & 55.0 / 33.9 \\
        \quad \textbf{+HiLo (ours)} & 53.4 / \textbf{34.0} & 55.2 / \textbf{37.8} \\
        \hline
        Transformer \cite{tang2020unbiased} & \textbf{63.6} / 17.9 & \textbf{65.7} / 19.6 \\
        \quad +IETrans \cite{zhang2022fine} & 51.8 / 30.8 & 53.8 / 34.7 \\
        \quad \textbf{+HiLo (ours)} & 52.9 / \textbf{32.8} & 55.9 / \textbf{36.1} \\
        \hline
        GPSNet \cite{lin2020gps} & \textbf{65.1} / 15.0 & \textbf{66.9} / 16.0 \\
        \quad +IETrans \cite{zhang2022fine} & 52.3 / 31.0 & 54.3 / 34.5 \\
        \quad \textbf{+HiLo (ours)} & 53.3 / \textbf{33.8} & 55.2 / \textbf{37.4} \\
        \hline
    \end{tabular}
    \vspace{+1mm}
    \caption{Comparison between our HiLo framework and other methods on PredCls on the VG-150 dataset. Similar to \cite{zhang2022fine}, we apply IETrans and our own method on top of four leading baselines.
    }
    \label{table:vg_main_results_predcls}
\end{table}


\begin{table}[h!]\small
    \centering
    \begin{tabular}{l|cc}
    \hline
        ~  & \multicolumn{2}{c}{Scene Graph Classification} \\
        \cline{2-3}
        Method & R/mR@50 & R/mR@100 \\
        \hline
        MOTIF \cite{zellers2018neural} & \textbf{38.0} / 8.7 & \textbf{38.9} / 9.3 \\
        \quad +IETrans \cite{zhang2022fine} & 32.5 / 16.8 & 33.4 / 17.9 \\
        \quad \textbf{+HiLo (ours)} & 32.1 / \textbf{18.9} & 33.1 / \textbf{20.9}  \\
        \hline
        VCTree \cite{tang2019learning} & \textbf{39.3} / 8.9 & \textbf{40.2} / 9.5 \\
        \quad +IETrans \cite{zhang2022fine} & 32.9 / 16.5 & 33.8 / 18.1 \\
        \quad \textbf{+HiLo (ours)} & 35.7 / \textbf{21.0} & 36.8 / \textbf{22.7} \\
        \hline
        Transformer \cite{tang2020unbiased} & \textbf{38.1} / 9.9 & \textbf{39.2} / 10.5 \\
        \quad +IETrans \cite{zhang2022fine} & 32.6 / 17.4 & 33.5 / 19.1 \\
        \quad \textbf{+HiLo (ours)} & 32.3 / \textbf{20.1} & 33.3 / \textbf{22.2} \\
        \hline
        GPSNet \cite{lin2020gps} & \textbf{36.9} / 8.2 & \textbf{38.0} / 8.7 \\
        \quad +IETrans \cite{zhang2022fine} & 31.8 / 17.0 & 32.7 / 18.3 \\
        \quad \textbf{+HiLo (ours)} & 31.7 / \textbf{18.3} & 32.5 / \textbf{20.2} \\
        \hline
    \end{tabular}
    \vspace{+1mm}
    \caption{Comparison between our HiLo framework and other methods on SGCls on the VG-150 dataset. Similar to \cite{zhang2022fine}, we apply IETrans and our own method on top of four leading baselines.}
    \vspace{+4mm}
    \label{table:vg_main_results_sgcls}
\end{table}
