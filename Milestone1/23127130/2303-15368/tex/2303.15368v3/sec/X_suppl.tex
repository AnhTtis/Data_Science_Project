\clearpage
\setcounter{page}{1}
\setcounter{section}{0}
\setcounter{figure}{8}
\maketitlesupplementary

\renewcommand{\thesection}{\Alph{section}}

\section{Proof of Theorem 1}
\emph{Proof:}
We consider the scenario where the ray only intersects with the plane once. The weight $w_2$ is the derivative of the logistic sigmoid function scaled by a constant factor $|\cos(\theta)|$, which is a bell-shaped function centered at $f(t^*)=0$ implying that the point $t^*$ is on the surface. Put it in another way, the color weight $w_2$ attains maximum value when the point is on the surface, therefore $w_2$ is unbiased. With the ray truncation mechanism, if there are multiple ray-plane intersections along a single ray, only the first intersection is in effect. Therefore, it is also occlusion-aware.

\section{Bias and Translucency Analysis of Stage 1}
The density function $\sigma_1$ in our Stage 1 coarse training is neither unbiased nor fully opaque, but we select $c=5$ for a good balance. In fact, we can estimate the bias and translucency. For points in front of the surface, the incident angle $\theta$ between the ray and the surface normal is obtuse, so we restrict $\theta$ to the range of $[91^\circ, 180^\circ]$. Assuming $s=1000$, by setting $c=5$, in theory, the offset width between 0.00161 and 0.00566 is obtained relative to the true zero level set, indicating that the maximum relative bias is below 0.5\%. This error level is acceptable for most application scenarios. Moreover, assuming $s=1000$, the surface transparency in the extreme case mentioned above is less than $0.001$. When a ray has a larger incident angle, its transparency becomes even smaller, resulting in an almost opaque density $\sigma_1$. As a result, the weight function $w_1$ is approximately occlusion-aware. Thus, setting the constant $c=5$ offers a good balance between occlusion-awareness and unbiasedness in the first stage training.

\section{Discussion of MeshUDF}
Unlike signed distance fields (SDFs), from which extracting a mesh is extensively studied, extracting a mesh from unsigned distance fields is still an actively developing research field with several challenges, which can lead to sub-optimal reconstruction results. MeshUDF~\cite{Guillard2022} is a UDF-mesh extraction method that has enjoyed considerable popularity, yet it still contains some limitations. Figure~\ref{fig:meshudf} showcases two common limitations of MeshUDF: the extracted mesh exhibits a visible ``staircase effect'' and hole artifacts resulting in a negative visual impact. ``Staircase effect'' and holes are pervasive across the results of NeuralUDF~\cite{Long2023}, NeUDF~\cite{Liu2023NeUDF} and our method. To eliminate these artifacts, we can use DoubleCoverUDF~\cite{Hou2023} for mesh extraction from UDF in the future, but we use MeshUDF in this work for fair comparisons.

\begin{figure}[ht]
    \centering
    \begin{tabular}{ccc}
        \includegraphics[width=.95in]{meshudf/staircase-neuraludf} & \includegraphics[width=.95in]{meshudf/staircase-neudf} & \includegraphics[width=.95in]{meshudf/staircase-ours} \\
        \includegraphics[width=.95in]{meshudf/holes-neuraludf-125-1} & \includegraphics[width=.65in]{meshudf/holes-neudf-276-1} & \includegraphics[width=.95in]{meshudf/holes-ours-522} \\
        NeuralUDF~\cite{Long2023} & NeUDF~\cite{Liu2023NeUDF} & Ours \\
    \end{tabular}
    \caption{The ``staircase effect'' and hole artifacts found in extracted meshes using MeshUDF~\cite{Guillard2022}. The first row shows raw meshes that have visible ``staircases'' widely found in NeuralUDF~\cite{Long2023}, NeUDF~\cite{Liu2023NeUDF} and our method, all using MeshUDF. The second row shows the hole artifacts found in extracted meshes. These artifacts may negatively impact on visual effects.}
    \label{fig:meshudf}
\end{figure}

\section{Implementation Details}
The UDF network is an MLP, consisting of 8 hidden layers, each with 256 elements. We use skip connections after every 4 hidden layers. The output of the UDF network is a single value representing the predicted UDF and a 256-dimensional feature vector used in the color network.

For the color network, we use another MLP with 4 hidden layers, each having 256 elements.
We use the coarse-to-fine strategy proposed by Park \emph{et al}.~\cite{Park2021} for position encoding, setting the maximum number of frequency bands to 16 for the UDF network and 6 for the color network. For background rendering, we use NeRF++~\cite{Zhang2020} for background prediction.
During training, we use the Adam optimizer~\cite{Kingma2015} with a global learning rate of 5e-4.
We sample 512 rays per batch and train our model for 250,000 iterations for the first stage and another 50,000 iterations for the second stage, making up a total of 300,000 iterations. We leverage MeshUDF~\cite{Guillard2022} to extract meshes from trained UDFs.

For the weights of each loss function term, we empirically set $\lambda_1=0.1$, $\lambda_2=0.01$, and $\lambda_3=0.001$, although $\lambda_2$ is occasionally set to $0.02$, and $\lambda_3$ is optional. The weight $\lambda_m$ for mask loss $\mathcal{L}_{loss}$ is set to 0.1 aligning with other works~\cite{Wang2021,Long2023}, if mask supervision is adopted.

\section{More Ablation Studies}

We conduct additional ablation studies in this section.

\begin{figure}[!ht]
    \centering
    \includegraphics[width=2.5in]{compare/vs_relu}
    \caption{Ablation study on the usage of ReLU (orange)~\cite{Fukushima1975} versus softplus (blue)~\cite{Liu2023NeUDF,Dugas2000} in the MLP output layer. The former is non-differentiable at $0$ and its gradient vanishes for negative input, whereas the latter is differentiable everywhere. 
    Using ReLU after the output layer of the MLP, the network makes progress at the early stage of training, but collapses after 40K iterations, leading to a training loss reduction through the rendering of only backgrounds. In contrast, softplus leads to correct learning of both geometry and color, and consistently decreases the training loss over iterations.}
    \label{fig:vs_relu}
\end{figure}


\emph{Non-negativity.} Ensuring that the computed distances in the proposed method are non-negative is important, and can be achieved by applying either ReLU~\cite{Fukushima1975} or softplus~\cite{Dugas2000,Liu2023NeUDF} to the MLP output. However, ReLU is not differentiable at 0 and has vanishing gradients for negative inputs, which can make the network difficult to train. An ablation study confirms that training with ReLU only results in early progress, but fails to learn a valid UDF later on. See Figure~\ref{fig:vs_relu} for details.

\emph{S-value loss.} Although $\mathcal{L}_s$ is optional, it is still important that the learned $s$ is large enough so that the model has better convergence, and the result is sharper. As shown in Figure~\ref{fig:vs_sval}, there are cases where omitting $\mathcal{L}_s$ results in a worse reconstruction result, as the Chamfer distances are higher. However, the impact is negligible both in quantitative metrics and qualitative comparisons, hinting at the optional nature of $\mathcal{L}_s$.


\begin{figure}[!ht]
    \centering
    \begin{tabular}{ccc}
    \includegraphics[width=.9in]{gt/dtu_scan37} & \includegraphics[width=.9in]{ours/dtu_scan37} & \includegraphics[width=.9in]{compare/dtu_scan37_ablation_no_sval_supress} \\
    & CD=0.891 & CD=0.904 \\
    Reference Image & w/ $\mathcal{L}_s$ & w/o $\mathcal{L}_s$\\
    \end{tabular}
    \caption{Qualitative and quantitative ablation study on the s-value loss $\mathcal{L}_s$. The visual impact and the quantitative impact are both very small.}
    \label{fig:vs_sval}
\end{figure}

\begin{figure}[htp]
\centering
\setlength\tabcolsep{1pt}
\begin{tabular}{ccccc}
    & GT & Ours & NeuralUDF & NeUDF \\
    \raisebox{.24in}{\#1} & \includegraphics[width=.685in]{gt/8-7-cut} & \includegraphics[width=.685in]{ours/8-7-cut} & \includegraphics[width=.685in]{neuraludf/8-7-cut} & \includegraphics[width=.685in]{neudf/8-7-cut} \\
    \raisebox{.24in}{\#5} & \includegraphics[width=.685in]{gt/409-1-cut} & \includegraphics[width=.685in]{ours/409-1-cut} & \includegraphics[width=.685in]{neuraludf/409-1-cut} & \includegraphics[width=.685in]{neudf/409-1-cut} \\
    \raisebox{.24in}{\#6} & \includegraphics[width=.685in]{gt/432-1-cut} & \includegraphics[width=.685in]{ours/432-1-cut} & \includegraphics[width=.685in]{neuraludf/432-1-cut} & \includegraphics[width=.685in]{neudf/432-1-cut} \\
    \raisebox{.24in}{\#7} & \includegraphics[width=.685in]{gt/448-2-cut} & \includegraphics[width=.685in]{ours/448-2-cut} & \includegraphics[width=.685in]{neuraludf/448-2-cut} & \includegraphics[width=.685in]{neudf/448-2-cut} \\
    \raisebox{.24in}{\#8} & \includegraphics[width=.685in]{gt/487-1-cut} & \includegraphics[width=.685in]{ours/487-1-cut} & \includegraphics[width=.685in]{neuraludf/487-1-cut} & \includegraphics[width=.685in]{neudf/487-1-cut} \\
    \raisebox{.24in}{\#9} & \includegraphics[width=.685in]{gt/593-1-cut} & \includegraphics[width=.685in]{ours/593-1-cut} & \includegraphics[width=.685in]{neuraludf/593-1-cut} & \includegraphics[width=.685in]{neudf/593-1-cut} \\
    \raisebox{.24in}{LS-C0} & \includegraphics[width=.685in]{gt/30-cut} & \includegraphics[width=.685in]{ours/30-cut} & \includegraphics[width=.685in]{neuraludf/30-cut} & \includegraphics[width=.685in]{neudf/30-cut} \\
    \raisebox{.24in}{SS-D0} & \includegraphics[width=.685in]{gt/92-cut} & \includegraphics[width=.685in]{ours/92-cut} & \includegraphics[width=.685in]{neuraludf/92-cut} & \includegraphics[width=.685in]{neudf/92-cut} \\
    \raisebox{.24in}{NS-D1} & \includegraphics[width=.685in]{gt/320-cut} & \includegraphics[width=.685in]{ours/320-cut} & \includegraphics[width=.685in]{neuraludf/320-cut} & \includegraphics[width=.685in]{neudf/320-cut} \\
    \raisebox{.24in}{LS-C1} & \includegraphics[width=.685in]{gt/522-cut} & \includegraphics[width=.685in]{ours/522-cut} & \includegraphics[width=.685in]{neuraludf/522-cut} & \includegraphics[width=.685in]{neudf/522-cut} \\
    \raisebox{.24in}{Skirt1} & \includegraphics[width=.685in]{gt/591-cut} & \includegraphics[width=.685in]{ours/591-cut} & \includegraphics[width=.685in]{neuraludf/591-cut} & \includegraphics[width=.685in]{neudf/591-cut} \\
    \raisebox{.24in}{SS-C0} & \includegraphics[width=.685in]{gt/598-cut} & \includegraphics[width=.685in]{ours/598-cut} & \includegraphics[width=.685in]{neuraludf/598-cut} & \includegraphics[width=.685in]{neudf/598-cut} \\
\end{tabular}
\caption{Visualization of the learned UDFs on cross sections for the remaining garments from DeepFashion3D.}
\label{fig:vs_udf_more}
\end{figure}

\section{Complete Results}

We present the remaining results on DeepFashion3D~\cite{Zhu2020} dataset in Figure~\ref{fig:vs_udf_more} for UDFs and Figure~\ref{fig:df3d-more} for reconstructed models.
The UDFs of NeuralUDF exhibit apparent oscillation. The UDFs of NeUDF are nearly closed possibly resulting in watertight models. In contrast, our learned UDFs are closest to the ground truth.

NeuralUDF~\cite{Long2023} performs poorly on some cases in Figure~\ref{fig:df3d-more}, possibly due to its complicated visibility indicator function. SDF-based methods such as VolSDF~\cite{Yariv2021} and NeuS~\cite{Wang2021} produce closed or double-cover models, leading to large reconstruction loss. Note that the UDF-based method NeUDF~\cite{Liu2023NeUDF} also fails to learn open models in case SS-D0. The reason is that the learned UDF of NeUDF is usually nearly closed, so it is liable to generate watertight models.

We also present the results on DTU~\cite{Jensen2014} dataset and BlendedMVS~\cite{Yao_2020_CVPR} in Figure~\ref{fig:dtu-blendedmvs-more}. For DTU dataset where quantitative comparisons are feasible, our Stage 2 optimization generally improves the reconstruction results (measured by Chamfer distances) of NeUDF~\cite{Liu2023NeUDF} by around 10\%. The reason is presented in the main text. For BlendedMVS dataset, we encourage readers to focus on the ``bear'' data. The brochure held by the bear (marked in red box) is an open part of the model. NeuralUDF~\cite{Long2023} and NeAT~\cite{Meng2023}, both of which use SDF implicitly or explicitly, as explained in the main text, fail to reconstruct the open brochure. NeUDF~\cite{Liu2023NeUDF} correctly reconstructs the brochure as a single-layer open surface but with large holes. Our method can generate a visually better open surface for such parts in real-life captured data.

\begin{figure*}[p]
\centering
\setlength\tabcolsep{0pt}
\begin{tabular}{ccccccccc}
    & Ref. Image & GT & Ours & VolSDF & NeuS & NeAT & NeuralUDF & NeUDF\\
    \raisebox{.24in}{\#1} & \includegraphics[width=.53in]{gt/8-7}&
    \includegraphics[width=.47in]{gt/8-7-pc}\includegraphics[width=.3in]{gt/8-7-pc-topdown}&
    \includegraphics[width=.47in]{ours/8-7}\includegraphics[width=.3in]{ours/8-7-topdown}&
    \includegraphics[width=.47in]{volsdf/8-7}\includegraphics[width=.3in]{volsdf/8-7-topdown}&
    \includegraphics[width=.47in]{neus/8-7}\includegraphics[width=.3in]{neus/8-7-topdown}&
    \includegraphics[width=.47in]{neat/8-7}\includegraphics[width=.3in]{neat/8-7-topdown}&
    \includegraphics[width=.47in]{neuraludf/8-7}\includegraphics[width=.3in]{neuraludf/8-7-topdown}&
    \includegraphics[width=.47in]{neudf/8-7}\includegraphics[width=.3in]{neudf/8-7-topdown}\\
    \raisebox{.28in}{\#5} & \includegraphics[width=.386in]{gt/409-1}&
    \includegraphics[width=.47in]{gt/409-1-pc}\includegraphics[width=.3in]{gt/409-1-pc-topdown}&
    \includegraphics[width=.47in]{ours/409-1}\includegraphics[width=.3in]{ours/409-1-topdown}&
    \includegraphics[width=.47in]{volsdf/409-1}\includegraphics[width=.3in]{volsdf/409-1-topdown}&
    \includegraphics[width=.47in]{neus/409-1}\includegraphics[width=.3in]{neus/409-1-topdown}&
    \includegraphics[width=.47in]{neat/409-1}\includegraphics[width=.3in]{neat/409-1-topdown}&
    \includegraphics[width=.47in]{neuraludf/409-1}\includegraphics[width=.3in]{neuraludf/409-1-topdown}&
    \includegraphics[width=.47in]{neudf/409-1}\includegraphics[width=.3in]{neudf/409-1-topdown}\\
    \raisebox{.28in}{\#6} & \includegraphics[width=.47in]{gt/432-1}&
    \includegraphics[width=.44in]{gt/432-1-pc}\includegraphics[width=.34in]{gt/432-1-pc-topdown}&
    \includegraphics[width=.44in]{ours/432-1}\includegraphics[width=.34in]{ours/432-1-topdown}&
    \includegraphics[width=.44in]{volsdf/432-1}\includegraphics[width=.34in]{volsdf/432-1-topdown}&
    \includegraphics[width=.44in]{neus/432-1}\includegraphics[width=.34in]{neus/432-1-topdown}&
    \includegraphics[width=.44in]{neat/432-1}\includegraphics[width=.34in]{neat/432-1-topdown}&
    \includegraphics[width=.44in]{neuraludf/432-1}\includegraphics[width=.34in]{neuraludf/432-1-topdown}&
    \includegraphics[width=.44in]{neudf/432-1}\includegraphics[width=.34in]{neudf/432-1-topdown}\\
    \raisebox{.24in}{\#7} & \includegraphics[width=.47in]{gt/448-2}&
    \includegraphics[width=.47in]{gt/448-2-pc}\includegraphics[width=.343in]{gt/448-2-pc-topdown}&
    \includegraphics[width=.47in]{ours/448-2}\includegraphics[width=.343in]{ours/448-2-topdown}&
    \includegraphics[width=.47in]{volsdf/448-2}\includegraphics[width=.343in]{volsdf/448-2-topdown}&
    \includegraphics[width=.47in]{neus/448-2}\includegraphics[width=.343in]{neus/448-2-topdown}&
    \includegraphics[width=.47in]{neat/448-2}\includegraphics[width=.343in]{neat/448-2-topdown}&
    \includegraphics[width=.47in]{neuraludf/448-2}\includegraphics[width=.343in]{neuraludf/448-2-topdown}&
    \includegraphics[width=.47in]{neudf/448-2}\includegraphics[width=.343in]{neudf/448-2-topdown}\\
    \raisebox{.26in}{\#8} & \includegraphics[width=.423in]{gt/487-1}&
    \includegraphics[width=.423in]{gt/487-1-pc}\includegraphics[width=.386in]{gt/487-1-pc-topdown}&
    \includegraphics[width=.423in]{ours/487-1}\includegraphics[width=.386in]{ours/487-1-topdown}&
    \includegraphics[width=.423in]{volsdf/487-1}\includegraphics[width=.386in]{volsdf/487-1-topdown}&
    \includegraphics[width=.423in]{neus/487-1}\includegraphics[width=.386in]{neus/487-1-topdown}&
    \includegraphics[width=.423in]{neat/487-1}\includegraphics[width=.386in]{neat/487-1-topdown}&
    \includegraphics[width=.423in]{neuraludf/487-1}\includegraphics[width=.386in]{neuraludf/487-1-topdown}&
    \includegraphics[width=.423in]{neudf/487-1}\includegraphics[width=.386in]{neudf/487-1-topdown}\\
    \raisebox{.25in}{\#9} & \includegraphics[width=.497in]{gt/593-1}&
    \includegraphics[width=.47in]{gt/593-1-pc}\includegraphics[width=.257in]{gt/593-1-pc-topdown}&
    \includegraphics[width=.47in]{ours/593-1}\includegraphics[width=.257in]{ours/593-1-topdown}&
    \includegraphics[width=.47in]{volsdf/593-1}\includegraphics[width=.257in]{volsdf/593-1-topdown}&
    \includegraphics[width=.47in]{neus/593-1}\includegraphics[width=.257in]{neus/593-1-topdown}&
    \includegraphics[width=.47in]{neat/593-1}\includegraphics[width=.257in]{neat/593-1-topdown}&
    \includegraphics[width=.47in]{neuraludf/593-1}\includegraphics[width=.257in]{neuraludf/593-1-topdown}&
    \includegraphics[width=.47in]{neudf/593-1}\includegraphics[width=.257in]{neudf/593-1-topdown}\\
    \raisebox{.26in}{LS-C0} & \includegraphics[width=.46in]{gt/30}&
    \includegraphics[width=.425in]{gt/30-pc}\includegraphics[width=.3in]{gt/30-pc-topdown}&
    \includegraphics[width=.425in]{ours/30}\includegraphics[width=.3in]{ours/30-topdown}&
    \includegraphics[width=.425in]{volsdf/30}\includegraphics[width=.3in]{volsdf/30-topdown}&
    \includegraphics[width=.425in]{neus/30}\includegraphics[width=.3in]{neus/30-topdown}&
    \includegraphics[width=.425in]{neat/30}\includegraphics[width=.3in]{neat/30-topdown}&
    \includegraphics[width=.425in]{neuraludf/30}\includegraphics[width=.3in]{neuraludf/30-topdown}&
    \includegraphics[width=.425in]{neudf/30}\includegraphics[width=.3in]{neudf/30-topdown} \\
    \raisebox{.26in}{SS-D0} & \includegraphics[width=.369in]{gt/92}&
    \includegraphics[width=.386in]{gt/92-pc}\includegraphics[width=.3in]{gt/92-pc-topdown}&
    \includegraphics[width=.386in]{ours/92}\includegraphics[width=.3in]{ours/92-topdown}&
    \includegraphics[width=.386in]{volsdf/92}\includegraphics[width=.3in]{volsdf/92-topdown}&
    \includegraphics[width=.386in]{neus/92}\includegraphics[width=.3in]{neus/92-topdown}&
    \includegraphics[width=.386in]{neat/92}\includegraphics[width=.3in]{neat/92-topdown}&
    \includegraphics[width=.386in]{neuraludf/92}\includegraphics[width=.3in]{neuraludf/92-topdown}&
    \includegraphics[width=.386in]{neudf/92}\includegraphics[width=.3in]{neudf/92-topdown-roi} \\
    \raisebox{.28in}{NS-D1} & \includegraphics[width=.386in]{gt/320}&
    \includegraphics[width=.36in]{gt/320-pc}\includegraphics[width=.317in]{gt/320-pc-topdown}&
    \includegraphics[width=.36in]{ours/320}\includegraphics[width=.317in]{ours/320-topdown}&
    \includegraphics[width=.36in]{volsdf/320}\includegraphics[width=.317in]{volsdf/320-topdown}&
    \includegraphics[width=.36in]{neus/320}\includegraphics[width=.317in]{neus/320-topdown}&
    \includegraphics[width=.36in]{neat/320}\includegraphics[width=.317in]{neat/320-topdown}&
    \includegraphics[width=.36in]{neuraludf/320}\includegraphics[width=.317in]{neuraludf/320-topdown}&
    \includegraphics[width=.36in]{neudf/320}\includegraphics[width=.317in]{neudf/320-topdown} \\
    \raisebox{.22in}{LS-C1} & \includegraphics[width=.514in]{gt/522}&
    \includegraphics[width=.514in]{gt/522-pc}\includegraphics[width=.257in]{gt/522-pc-topdown}&
    \includegraphics[width=.514in]{ours/522}\includegraphics[width=.257in]{ours/522-topdown}&
    \includegraphics[width=.514in]{volsdf/522}\includegraphics[width=.257in]{volsdf/522-topdown}&
    \includegraphics[width=.514in]{neus/522}\includegraphics[width=.257in]{neus/522-topdown}&
    \includegraphics[width=.514in]{neat/522}\includegraphics[width=.257in]{neat/522-topdown}&
    \includegraphics[width=.514in]{neuraludf/522}\includegraphics[width=.257in]{neuraludf/522-topdown}&
    \includegraphics[width=.514in]{neudf/522}\includegraphics[width=.257in]{neudf/522-topdown} \\
    \raisebox{.20in}{Skirt1} & \includegraphics[width=.386in]{gt/591}&
    \includegraphics[width=.386in]{gt/591-pc}\includegraphics[width=.343in]{gt/591-pc-topdown}&
    \includegraphics[width=.386in]{ours/591}\includegraphics[width=.343in]{ours/591-topdown}&
    \includegraphics[width=.386in]{volsdf/591}\includegraphics[width=.343in]{volsdf/591-topdown}&
    \includegraphics[width=.386in]{neus/591}\includegraphics[width=.343in]{neus/591-topdown}&
    \includegraphics[width=.386in]{neat/591}\includegraphics[width=.343in]{neat/591-topdown}&
    \includegraphics[width=.386in]{neuraludf/591}\includegraphics[width=.343in]{neuraludf/591-topdown}&
    \includegraphics[width=.386in]{neudf/591}\includegraphics[width=.343in]{neudf/591-topdown} \\
    \raisebox{.24in}{SS-C0} & \includegraphics[width=.42in]{gt/598}&
    \includegraphics[width=.446in]{gt/598-pc}\includegraphics[width=.257in]{gt/598-pc-topdown}&
    \includegraphics[width=.446in]{ours/598}\includegraphics[width=.257in]{ours/598-topdown}&
    \includegraphics[width=.446in]{volsdf/598}\includegraphics[width=.257in]{volsdf/598-topdown}&
    \includegraphics[width=.446in]{neus/598}\includegraphics[width=.257in]{neus/598-topdown}&
    \includegraphics[width=.446in]{neat/598}\includegraphics[width=.257in]{neat/598-topdown}&
    \includegraphics[width=.446in]{neuraludf/598}\includegraphics[width=.257in]{neuraludf/598-topdown}&
    \includegraphics[width=.446in]{neudf/598}\includegraphics[width=.257in]{neudf/598-topdown} \\
\end{tabular}
\caption{The remaining qualitative comparisons with VolSDF~\cite{Yariv2021}, NeuS~\cite{Wang2021}, NeAT~\cite{Meng2023} (with mask supervision), NeuralUDF~\cite{Long2023} and NeUDF~\cite{Liu2023NeUDF} on the DeepFashion3D~\cite{Zhu2020} dataset.}

\label{fig:df3d-more}
\end{figure*}

\begin{figure*}[p]
\centering
\setlength\tabcolsep{3pt}
\begin{tabular}{cccccc}
    & Reference Image & Ours & NeUDF & NeuralUDF & NeAT \\
    \raisebox{.3in}{37} & \includegraphics[width=1in]{gt/dtu_scan37} &
    \includegraphics[width=1in]{ours/dtu_scan37} &
    \includegraphics[width=1in]{neudf/dtu_scan37} &
    \includegraphics[width=1in]{neuraludf/dtu_scan37} &
    \includegraphics[width=1in]{neat/dtu_scan37} \\
    \raisebox{.3in}{69} & \includegraphics[width=1in]{gt/dtu_scan69} &
    \includegraphics[width=.83in]{ours/dtu_scan69} &
    \includegraphics[width=.83in]{neudf/dtu_scan69} &
    \includegraphics[width=.83in]{neuraludf/dtu_scan69} &
    \includegraphics[width=.83in]{neat/dtu_scan69} \\
    \raisebox{.3in}{97} & \includegraphics[width=1in]{gt/dtu_scan97} &
    \includegraphics[width=.6in]{ours/dtu_scan97} &
    \includegraphics[width=.6in]{neudf/dtu_scan97} &
    \includegraphics[width=.6in]{neuraludf/dtu_scan97} &
    \includegraphics[width=.6in]{neat/dtu_scan97} \\
    \raisebox{.3in}{105} & \includegraphics[width=1in]{gt/dtu_scan105} &
    \includegraphics[width=.83in]{ours/dtu_scan105} &
    \includegraphics[width=.83in]{neudf/dtu_scan105} &
    \includegraphics[width=.83in]{neuraludf/dtu_scan105} &
    \includegraphics[width=.83in]{neat/dtu_scan105} \\
    \raisebox{.3in}{106} & \includegraphics[width=1in]{gt/dtu_scan106} &
    \includegraphics[width=1in]{ours/dtu_scan106} &
    \includegraphics[width=1in]{neudf/dtu_scan106} &
    \includegraphics[width=1in]{neuraludf/dtu_scan106} &
    \includegraphics[width=1in]{neat/dtu_scan106} \\
    \raisebox{.3in}{114} & \includegraphics[width=1in]{gt/dtu_scan114} &
    \includegraphics[width=.7in]{ours/dtu_scan114} &
    \includegraphics[width=.7in]{neudf/dtu_scan114} &
    \includegraphics[width=.7in]{neuraludf/dtu_scan114} &
    \includegraphics[width=.7in]{neat/dtu_scan114} \\
    \raisebox{.42in}{bear} & \includegraphics[width=.75in]{gt/bmvs_bear_roi} &
    \includegraphics[width=.52in]{ours/bmvs_bear_book_roi} &
    \includegraphics[width=.52in]{neudf/bmvs_bear_book_roi} &
    \includegraphics[width=.52in]{neuraludf/bmvs_bear_book_roi} &
    \includegraphics[width=.52in]{neat/bmvs_bear_book_roi} \\
    \raisebox{.3in}{man} & \includegraphics[width=1in]{gt/bmvs_man} &
    \includegraphics[width=.45in]{ours/bmvs_man} &
    \includegraphics[width=.45in]{neudf/bmvs_man} &
    \includegraphics[width=.45in]{neuraludf/bmvs_man} &
    \includegraphics[width=.45in]{neat/bmvs_man} \\
    \raisebox{.3in}{sculpture} & \includegraphics[width=1in]{gt/bmvs_sculpture} &
    \includegraphics[width=.35in]{ours/bmvs_sculpture} &
    \includegraphics[width=.35in]{neudf/bmvs_sculpture} &
    \includegraphics[width=.35in]{neuraludf/bmvs_sculpture} &
    \includegraphics[width=.35in]{neat/bmvs_sculpture} \\
    
\end{tabular}
\caption{Qualitative comparisons with NeAT~\cite{Meng2023}, NeuralUDF~\cite{Long2023} and NeUDF~\cite{Liu2023NeUDF} on the DTU~\cite{Jensen2014} dataset and BlendedMVS~\cite{Yao_2020_CVPR} dataset.}

\label{fig:dtu-blendedmvs-more}
\end{figure*}

