\documentclass[11pt, reqno]{amsart}

\usepackage[margin=1.3in]{geometry}

\usepackage{graphicx}
\usepackage{subcaption}
\usepackage{stmaryrd}
\usepackage{amssymb}
\usepackage{amsthm}
\usepackage{dsfont}
\usepackage{hyperref}
\usepackage{mathrsfs}
\usepackage{stmaryrd}
\usepackage[lite,initials,msc-links, bibtex-style]{amsrefs}
\usepackage{amsmath}
\usepackage{centernot}
\usepackage{enumerate}
\usepackage{xcolor}
\usepackage{verbatim}

\newtheorem{theorem}{Theorem}[section]
\newtheorem{lemma}[theorem]{Lemma}
\newtheorem{proposition}[theorem]{Proposition}
\newtheorem{corollary}[theorem]{Corollary}
\numberwithin{equation}{section}

\theoremstyle{definition}
\newtheorem{definition}{Definition}
\newtheorem{setup}{Setup}[section]
\newtheorem*{notation}{Notation}
\newtheorem{example}{Example}
\theoremstyle{remark}
\newtheorem{remark}{Remark} 




\newcommand{\clust}{\mathscr{C}}
\newcommand{\coin}{\xi}
\newcommand{\field}{\mathcal{S}}
\newcommand{\msgn}{\sigma}
\newcommand{\IP}{\mathbf{P}}
\newcommand{\IE}{\mathbf{E}}
\newcommand{\graph}{G}

\newcommand{\set}{\omega}
\newcommand{\setm}{\rho}
\newcommand{\setf}{\rho}
\newcommand{\sets}{\Omega}
\newcommand{\currs}{\Omega}
\newcommand{\flows}{\mathcal{F}}
%\newcommand{\n}{\mathbf{n}}
\newcommand{\rcur}{\textnormal{curr}}
\newcommand{\drcur}{\textnormal{d-curr}}
\newcommand{\aflow}{\textnormal{a-flow}}
\newcommand{\ob}{W}
\newcommand{\ising}{\textnormal{Isg}}
\newcommand{\para}{\mathcal{P}}
\newcommand{\domain}{D}
\DeclareMathOperator{\pf}{\textnormal{pf}}
\newcommand{\ccomp}{\kappa}
\newcommand{\ec}{\tilde F}
\newcommand{\source}{{+}}
\newcommand{\sink}{{-}}
\newcommand{\facs}{F}
\newcommand{\match}{M}
\newcommand{\dimer}{\textnormal{dim}}
\newcommand\iid{i.i.d.}
\newcommand{\cont}{C}
\newcommand{\matchs}{\mathcal{M}}
\newcommand{\ncont}{C}
\newcommand{\nclust}{N}
\newcommand*{\vv}[1]{\vec{\mkern0mu#1}}






%Diederik's shortcuts
\newcommand{\ZZ}{\mathbb{Z}}
\newcommand{\PP}{\nu}
%\newcommand{\PPdouble}{\widehat{\mathbb{P}}}
%\newcommand{\Pdouble}{\widehat{\mathbf{P}}}
%\newcommand{\Edouble}{\widehat{\mathbf{E}}}
\newcommand{\E}{\nu}
\newcommand{\id}{\mathds{1}}
\newcommand{\NN}{\mathbb{N}}
\newcommand{\RR}{\mathbb{R}}
\newcommand{\QQ}{\mathbb{Q}}
\renewcommand{\d}{\mathrm{d}}

\renewcommand{\c}[1]{\ensuremath{\mathcal{#1}}}

%To be changed later
\newcommand{\PClass}{{\color{red} our class }}
\newcommand{\Group}{{\ensuremath{\mathbb{G}}}}
\newcommand{\Green}{\ensuremath{\mathbf{G}}}
\newcommand{\J}{J}
\newcommand{\n}{\mathbf{n}}

\newcommand{\exact}{\star}
\newcommand{\coclosed}{\Diamond}
\newcommand{\Hcycle}{\ensuremath{H_{\star}}}
\newcommand{\Hstar}{\ensuremath{H_{\Diamond}} }

%Reverse vector arrow
%\DeclareRobustCommand{\cev}[1]{%
%	{\mathpalette\do@cev{#1}}%
%}
%\newcommand{\do@cev}[2]{%
%	\vbox{\offinterlineskip
%		\sbox\z@{$\m@th#1 x$}%
%		\ialign{##\cr
%			\hidewidth\reflectbox{$\m@th#1\vec{}\mkern4mu$}\hidewidth\cr
%			\noalign{\kern-\ht\z@}
%			$\m@th#1#2$\cr
%		}%
%	}%
%}
%\makeatother


%Comments
\newcommand{\DE}[1]{{\color{red} #1}}
\newcommand{\ML}[1]{{\color{blue} #1 }}

\title{On the duality between height functions and continuous spin models}

\author{Diederik van Engelenburg \and Marcin Lis}


\setcounter{tocdepth}{4}
\setcounter{secnumdepth}{4}

\date{\today}




\begin{document}
\maketitle 

\begin{abstract}
We revisit the classical phenomenon of duality between random integer-valued height functions with positive definite potentials and abelian spin models with O(2) symmetry. We use it to derive 
new results in quite high generality including: a universal upper bound on the variance of the height function in terms of the Green's function (a GFF bound) which among others implies localisation on transient graphs;
monotonicity of said variance with respect to a natural temperature parameter; the fact that delocalisation of the height function implies a BKT phase transition in planar models; and
also delocalisation itself for height functions on periodic ``almost'' planar graphs. 
\end{abstract}

\section{Introduction}
\label{sec:introduction}
% \begin{itemize}
%     % Diffusion of FL
%     \item {\st{Diffusion of FL}}
%     % Security threats to FL
%     \item {\st{Security threats to FL with particular focus on model poisoning}}
%     % Limitations of existing countermeasures
%     \item {\st{Current countermeasures (e.g., KRUM) and their limitations}}
%     % Proposed method and its advantages
%     \item {\st{Intuitive description of the proposed method and its difference (i.e., advantages) w.r.t. state of the art}}
%     % Main contributions
%     \item {\st{Summary of the main contributions of this work}}
%     % Paper's structure and organization
%     \item {\st{Paper's structure and organization}}
% \end{itemize}

% Diffusion of FL
Recently, {\em federated learning} (FL) has emerged as the leading paradigm for training distributed, large-scale, and privacy-preserving machine learning (ML) systems~\cite{mcmahan2017googleai,mcmahan2017aistats}. 
The core idea of FL is to allow multiple edge clients to collaboratively train a shared, global model without disclosing their local private training data.
%Specifically, an FL system consists of a central server and many edge clients; 
A typical FL round involves the following steps: {\em(i)} the server randomly picks some clients and sends them the current, global model; {\em(ii)} each selected client locally trains its model with its own private data; then, it sends the resulting local model to the server;\footnote{Whenever we refer to global/local model, we mean global/local model {\em parameters}.} {\em(iii)} the server updates the global model by computing an \emph{aggregation function}, usually the average (FedAvg), on the local models received from clients.
% \begin{enumerate}
%     \item[{\em(i)}] the server sends the current, global model to the clients and appoints some of them for training;
%     \item[{\em(ii)}] each selected client locally trains its copy of the global model with its own private data; then, it sends the resulting local model back to the server;\footnote{Whenever we refer to global/local model, we mean global/local model {\em parameters}.}
%     \item[{\em(iii)}] the server updates the global model by computing an \emph{aggregation function} on the local models received from clients (by default, the average, also referred to as FedAvg~\cite{mcmahan2017aistats}).
% \end{enumerate}
This process goes on until the global model converges. %(e.g., after a certain number of rounds or other similar stopping criteria).
%\\
% The advantages of FL over the traditional, centralized learning paradigm are undoubtedly clear in terms of flexibility/scalability (clients can join/disconnect from the FL network dynamically), network communications (only model weights\footnote{We will use \textit{parameters} and \textit{weights} interchangeably.} are exchanged between clients and server), and privacy (each client's private training data is kept local at the client's end and not uploaded to the server).
\\
% Security threats to FL
%However, the growing adoption of FL also raises security concerns~\cite{costa2022covert}, particularly about its confidentiality, integrity, and availability.
Although its advantages over standard ML, FL also raises security concerns~\cite{costa2022covert}. %, particularly about its confidentiality, integrity, and availability~\cite{costa2022covert}.
% OLD, LONG VERSION
% Indeed, some work deals with privacy leakage that may expose the local data of some clients~\cite{melis2019sp}. 
% A large body of work, instead, investigates attacks that usually aim to detriment the predictive accuracy of the learned global model. For instance, \emph{data poisoning} attacks achieve this goal by letting an adversary pollute the training set of some corrupt FL clients with maliciously crafted examples~\cite{jagielski2018sp}.
% Similarly, in \emph{model poisoning} the attacker attempts to tweak the global model weights~\cite{bhagoji2019pmlr} by directly perturbing the local model's weights of some infected FL clients before these are sent to the central server for aggregation, usually via so-called Byzantine attacks. 
% It turns out that Byzantine model poisoning attacks severely impact standard FedAvg; therefore, more robust aggregation functions must be designed to make FL systems secure.
Here, we focus on \emph{untargeted model poisoning} attacks~\cite{bhagoji2019pmlr}, where an adversary attempts to tweak the global model weights %\footnote{We will use the terms \textit{parameters} and \textit{weights} interchangeably.} 
by directly perturbing the local model's parameters of some infected clients before these are sent to the central server for aggregation.
In doing so, the adversary aims to jeopardize the global model \textit{indiscriminately} at inference time.
Such model poisoning attacks severely impact standard FedAvg; therefore, more robust aggregation functions must be designed to secure FL systems.
\\
% In this paper, we focus on designing a novel robust aggregation scheme at the server's end to contrast the effect of Byzantine model poisoning attacks.
%
% Current countermeasures and their limitations
%Several countermeasures have been proposed in the literature to combat model poisoning attacks on FL systems.
% Some methods use simple statistics more robust than plain average to smooth the impact of malicious updates (e.g., Trimmed Mean and FedMedian~\cite{yin2018icml}). 
% Other defenses implement outlier detection techniques to discard malicious updates from the aggregation performed at the server's end. Those are either based on heuristics (e.g., Krum/Multi-Krum~\cite{blanchard2017nips} and Bulyan~\cite{mhamdi2018pmlr}) or data-driven approaches (e.g., K-means clustering~\cite{shen2016acm} or DnC via spectral analysis~\cite{shejwalkar2021ndss}). 
% Finally, some strategies rely on a centralized ``source of trust'' to spot potential malicious updates (e.g., FLTrust~\cite{cao2020fltrust}).
% Several countermeasures have been proposed in the literature to combat model poisoning attacks on FL systems, i.e., to discard possible malicious local updates from the aggregation performed at the server's end. 
% These techniques range from simple statistics more robust than plain average (e.g., Trimmed Mean and FedMedian~\cite{yin2018icml}) to outlier detection heuristics (e.g., Krum/Multi-Krum~\cite{blanchard2017nips} and Bulyan~\cite{mhamdi2018pmlr}) or data-driven approaches (e.g., spectral analysis via K-means clustering~\cite{shen2016acm} or spectral analysis), or methods based on ``source of trust'' (e.g., FLTrust~\cite{cao2020fltrust}).
% OLD, LONG VERSION
%Several countermeasures have been proposed in the literature to combat Byzantine model poisoning attacks on FL systems.
% Descriptive statistics
% For example, Trimmed Mean and FedMedian aggregate local model updates using more robust statistics than standard average~\cite{yin2018icml}.
%
% % Heuristics for outlier detection
% Many existing Byzantine-resilient strategies implement some outlier detection heuristics to discard the model updates sent by potentially malicious clients from the input of the aggregation function.
% One of the most popular heuristics is Krum~\cite{blanchard2017nips}.
% This strategy tries to mitigate the impact of Byzantine attacks by selecting as a global model the local model with the smallest sum of Euclidean distances to {\em all} the other local models.
% Although powerful, Krum requires the server to know (or, at least, estimate) the number of malicious FL clients upfront, which is generally impossible in a realistic attack scenario. %
% Moreover, Krum may become ineffective for complex, high-dimensional model parameter spaces due to the curse of dimensionality.
% Bulyan~\cite{mhamdi2018pmlr} tries to overcome this issue by combining Krum with a variant of Trimmed Mean.
% % Data-driven outlier detection
% Other strategies use data-driven outlier detection techniques -- e.g., via K-means clustering~\cite{shen2016acm} -- to spot potential malicious local model updates. 
% %For instance, Shen et al. propose to cluster local model updates with K-means and thus identify outliers.
%
% % Other techniques
% As far as the server is concerned, any local model received can be from a potential malicious client. 
% FLTrust~\cite{cao2020fltrust} assumes the server acts as a client, i.e., trains a local model on an additional {\em trustworthy} dataset at the server's end and compares it against all the local models from other clients. 
% This way, the server can rely on some ``source of trust'' when discarding potentially malicious clients.
%\\
% Limitations of existing Byzantine-resilient strategies
Unfortunately, existing defense mechanisms either rely on simple heuristics (e.g., Trimmed Mean and FedMedian by~\cite{yin2018icml}) or need strong and unrealistic assumptions to work effectively (e.g., foreknowledge or estimation of the number of malicious clients in the FL system, as for Krum/Multi-Krum~\cite{blanchard2017nips} and Bulyan~\cite{mhamdi2018pmlr}, which, however, cannot exceed a fixed threshold).
Furthermore, outlier detection methods using K-means clustering~\cite{shen2016acm} or spectral analysis like DnC~\cite{shejwalkar2021ndss} do not directly consider the temporal evolution of local model updates received.
Finally, strategies like FLTrust~\cite{cao2020fltrust} require the server to collect its own dataset and act as a proper client, thereby altering the standard FL protocol.
\\
% OLD, LONG VERSION
% Overall, existing Byzantine-resilient strategies are either simple heuristics (e.g., FedMedian) or, if they are more complex, they rely on strong and unrealistic assumptions to work effectively (e.g., knowing the number of malicious clients in the FL system in advance, as for Krum and alike).
% Furthermore, data-driven outlier detection methods do not consider the temporary evolution of local model updates received (e.g., K-means clustering). 
% Finally, strategies like FLTrust requires the server to collect its own dataset and act as a proper client, thereby altering the standard FL protocol.
%
% Description of the proposed method
This work introduces a novel pre-aggregation \textit{filter} robust to untargeted model poisoning attacks. Notably, this filter $(i)$ operates without requiring prior knowledge or constraints on the number of malicious clients and $(ii)$ inherently integrates temporal dependencies. 
The FL server can employ this filter as a preprocessing step before applying \textit{any} aggregation function, be it standard like FedAvg or robust like Krum or Bulyan.
Specifically, we formulate the problem of identifying corrupted updates as a multidimensional (i.e., matrix-valued) time series anomaly detection task. 
The key idea is that legitimate local updates, resulting from well-calibrated iterative procedures like stochastic gradient descent (SGD) with an appropriate learning rate, show \textit{higher predictability} compared to malicious updates. This hypothesis stems from the fact that the sequence of gradients (thus, model parameters) observed during legitimate training exhibit regular patterns, as validated in Section~\ref{subsec:intuition}. %until convergence. 
%This regularity may be more pronounced for smooth convex loss functions, but it can still be captured within an appropriate time window, even for more complex and convoluted loss surfaces. 
%We provide evidence of this claim in Appendix~B, where we show that the average mutual information (i.e., ``predictability''), calculated over pairs of legitimate model updates sent at different FL rounds, is significantly higher than the corresponding computation for a malicious client.
\\
Inspired by the matrix autoregressive (MAR) framework for multidimensional time series forecasting~\cite{chen2021je}, we propose the FLANDERS ({\em \textbf{F}ederated \textbf{L}earning meets \textbf{AN}omaly \textbf{DE}tection for a \textbf{R}obust and \textbf{S}ecure}) filter.
The main advantages of FLANDERS over existing strategies like FLDetector~\cite{zhao2020multivariate} are its resilience to large-scale attacks, where $50\%$ or more FL participants are hostile, and the capability of working under realistic non-iid scenarios.
We attribute such a capability to two key factors: $(i)$ FLANDERS works without knowing a priori the ratio of corrupted clients, and $(ii)$ it embodies temporal dependencies between intra- and inter-client updates, quickly recognizing local model drifts caused by evil players. Below, we summarize our main contributions:

\begin{itemize}
\item[{\em(i)}]
We provide empirical evidence that the sequence of models sent by legitimate clients is more predictable than those of malicious participants performing untargeted model poisoning attacks.
\\
\item[{\em(ii)}] 
We introduce FLANDERS, the first pre-aggregation filter for FL robust to untargeted model poisoning based on multidimensional time series anomaly detection.
\\
\item[{\em(iii)}] 
We integrate FLANDERS into Flower,\footnote{\scriptsize{\url{https://flower.dev/}}} a popular FL simulation framework for reproducibility.
\\
\item[{\em(iv)}] 
We show that FLANDERS improves the robustness of the existing aggregation methods under multiple settings: different datasets, client's data distribution (non-iid), models, and attack scenarios.
\\
\item[{\em(v)}] 
We publicly release all the implementation code of FLANDERS along with our experiments.\footnote{\scriptsize{\url{https://anonymous.4open.science/r/flanders_exp-7EEB}}}
\end{itemize}

% Paper's structure and organization
The remainder of the paper is structured as follows. %some related work and the current state-of-the-art solutions to security issues that FL entails. 
Section~\ref{sec:background} covers background and preliminaries. 
In Section~\ref{sec:related}, we discuss related work.
Section~\ref{sec:problem} and Section~\ref{sec:method} describe the problem formulation and the method proposed. % to tackle it. 
Section~\ref{sec:experiments} gathers experimental results. %, and Section~\ref{sec:limitations} discusses some limitations of this work.
Finally, we conclude in Section~\ref{sec:conclusion}.
 %discusses the limitations of this work and draws future research directions.
%reports conclusions and draws perspectives for future research directions.

%%%%%%% OLD %%%%%%%
%to overcome the resilience of Byzantine failures in distributed Stochastic Gradient Descent computations. 
% The strength of Krum is its time complexity, which is linear in the gradient dimension. 
% However, the robustness of the approach is guaranteed for gradient-based learning applications only when the majority of the clients are not compromised. 
% Besides, the aggregation mechanism of Krum, as well as that of similar methods, is robust from a coarse-grained perspective and does not provide solutions to errors and perturbations that may occur at inference time.
%A related approach to~\cite{blanchard2017nips} is the work of Su et al.~\cite{su2016dc}. Here, the authors propose an iterated approximate agreement to tackle a multi-layer scenario attacked by Byzantine agents. 
%However, the method works efficiently on the sole discrete context and it is inapplicable to continuous state environments.
%\gabri{Maybe, we should just talk about the main limitations of existing countermeasures without digging into their details (or, we can just mention Krum as this is the most popular one). I will move the description of all these methods to the Related Work section.}
% !TEX root = ../main_duality.tex

\section{General duality}\label{sec:gendual}

\subsection{Discrete calculus}
We first give a basic background on discrete calculus on graphs, 
staying close to the language of \cite{LyPer}. 
Let $G = (V, E)$ be a locally finite graph and let $\Group$ be a group (we will consider $\Group= \RR, \ZZ$ with addition and $\Group= \mathbb{S}:=\{z\in \mathbb C: |z|=1\}$ with multiplication). 
To keep the exposition homogenous, we will use the additive notation for all considered groups.

A \emph{1-form} $\omega$ taking values in $\Group$ is an antisymmetric function defined on the directed edges $\vec E$ of $G$, 
i.e., such that $\omega_{vv'}= -\omega_{v'v}$, where $vv'$ denotes the directed edge $(v,v')$. The set of $1$-forms will be denoted by $\Omega^1(\Group) = \Omega^1( G,\Group)$, and the set $\mathbb G^V$ by $\Omega^0(\Group) = \Omega^0(G, \Group)$. 
We will identify the space of 1-forms with $\mathbb \Group^E$ by fixing, once and for all, one of the two orientations for each edge in~$E$. 
Define the \emph{boundary} operator $\d^*: \Omega^1(\Group) \to \Omega^0(\Group)$ by 
\begin{align*}
	\d^*\omega_x = \sum_{y\sim x} \omega_{yx},
\end{align*}
where $y\sim x$ indicates that $y$ and $x$ are adjacent in $G$, and the \emph{co-boundary} operator $\d: \Omega^0(\Group) \to \Omega^1(\Group)$ by
\[	
\d f_{xy} = f_y - f_x.
\]
Note that $\d^*$ and $\d$ are homomorphisms between groups $\mathbb G^E$ and $\mathbb G^V$, and hence we can define groups
\[
\Hcycle(\Group) = \Hcycle(G,\Group):= \mathrm{Im}(\d) \cong \Group^V / \ker(\d) \quad \text{ and } \quad \Hstar(\Group) =  \Hstar(G,\Group) :=\ker(\d^*).
\]
For $\Group=\mathbb S$, we will write $d \J$ to be the Haar probability measure on the induced (compact) groups $\Hcycle(\mathbb S)$ and $\Hstar(\mathbb S)$. 
If $\Group$ is only locally compact, the Haar measure is defined up to a multiplicative constant and we fix some normalisation. 
For a more concrete definition, we refer to Appendix \ref{ap:duality}. 
We make the convention that the space over which we integrate determines the measure. 



\begin{notation}
In what follows we will use the letters $\epsilon, \omega$ (resp.\ $f,g,\tau$) to denote deterministic elements of $\Omega^1(\Group)$ (resp.\ $\Omega^0(\Group)$) when $\Group=\mathbb R$ or when $\mathbb G$ is not specified. 
We will write $\n$ and $h$ for (mostly random) elements of $\Omega^1(\mathbb Z)$ and $ \Omega^0(\mathbb Z)$ respectively, and $J$ and $\theta$ for (mostly random)
elements of $\Omega^1(\mathbb S)$ and $ \Omega^0(\mathbb S)$ respectively.

We will also often abuse notation in the following sense: through the identification $\exp( i\theta)\leftrightarrow\theta$, we have $\mathbb{S} \cong (-\pi, \pi]$, and
we will view $\J \in \Omega^1(\mathbb{S})$ as belonging to $\Omega^{1}(\RR)$. On the other hand, by considering real numbers modulo $2\pi$, 
we will map $\Omega^1(\RR)$ to $\Omega^1(\mathbb{S}^1)$.
One has to be careful when going from one space to the other: the embedding does not map $H_{\#} (\mathbb{S})$ to $H_{\#}(\RR)$, 
because for example a $1$-form $\omega \in \Omega^1(\mathbb{S})$ which satisfies $\d^* \omega= 0$ in $\mathbb{S}$, 
only satisfies $\d^*\omega = 0$ modulo $2\pi$ when viewed as a $1$-form in $\Omega^1(\mathbb{R})$. We will also think of $H_{\#}(\mathbb Z)$
as a subset of  $H_{\#}(\mathbb R)$ in the obvious way.
\end{notation}



\subsection{Spin models and random height functions}
In this section we consider a finite graph $G = (V,E)$. 
We will study random \emph{spin} and \emph{height} $1$-forms 
taking values in the spaces $H_{\#}(\mathbb S)$ and $H_{\#}(\mathbb Z)$ respectively for $\# \in \{\coclosed, \exact\}$. 

\begin{definition}[Height function and spin potentials] \label{def:potential}
Let $\mathcal V: \mathbb Z\to \mathbb R \cup \{ +\infty\}$ be symmetric, i.e., $\mathcal V (n)=\mathcal V(-n)$, such that 
\begin{align} \label{eq:Vsum}
\sum_{n\in \mathbb Z}n^2 \exp(-\mathcal V(n))<\infty,
\end{align}
and moreover such that $\exp(-\mathcal V)$
is \emph{positive definite}: for all $\alpha\in \mathbb R$, 
\begin{align}\label{eq:posdef}
	w(\alpha):=\exp(-\mathcal V(0))+\sum_{n=1}^{\infty}\exp(-\mathcal V(n))2\cos( n\alpha) > 0.
\end{align}
We call $\mathcal V$ the \emph{height function potential}, and $\mathcal U (\alpha) := -\log w(\alpha)$ the \emph{spin potential}.
\end{definition}
We will always assume that the considered potentials satisfy the conditions of Definition~\ref{def:potential}.
Note that condition~\eqref{eq:Vsum} implies that the series in \eqref{eq:posdef} is absolutely summable, and moreover that $\omega$, as well as $\mathcal U$, is twice continuously differentiable in $\alpha$.
In general, we will say that a function is positive definite if its Fourier transform is a nonnegative function.
This is not the classical definition of positive definiteness, but it is equivalent to it by Bochner's theorem.
\begin{example} \label{exple:potentials}The following potentials satisfy the conditions of Definition~\ref{def:potential}:
\begin{itemize}
\item  $\mathcal V(n)= -\log(I_{n}(\beta))$, where $I_n(\beta)$ is the modified Bessel function of the first kind, and $\c{U}(t) = -\beta \cos( t)$ for all $\beta>0$ is the potential of the \emph{classical XY model},
\item $\mathcal V(n)=\beta n^2$ for all $\beta >0$ is the potential of the \emph{integer-valued Gaussian free field} and the corresponding $\mathcal U$ defined through the series in~\eqref{eq:posdef} is the potential of the \emph{Villain spin model},
\item  $\mathcal V(n)=\beta \mathbf 1\{ n=\pm 1\}+\infty  \mathbf 1\{ |n|>1\}$ for $\exp(-\beta)<1/2$ is a model of random (nonuniform) \emph{Lipschitz functions}.
\item
	Any \emph{annealed Gaussian} potential $\c{V}$ meaning that there exists a finite Borel measure $\lambda$ on $[0, \infty)$ such that
\[
	e^{-\c{V}(n)} = \int_{[0, \infty)} e^{-\frac{\gamma}{2} n^2} \lambda(d \gamma)
\]
for all $n$. 
It satisfies Definition \ref{def:potential} because the function $n \mapsto \frac{\gamma}{2} n^2$ does and because by dominated convergence, we can exchange the integral and the summation in \eqref{eq:posdef}.  
This class includes the potentials $\c{V}(n) = \beta |n|^a$ for any $a \in (0, 2]$ (see~\cite{AHPS}). 
\end{itemize}
\end{example}

Let $\omega$ be as in Definition~\ref{def:potential}. Fix $\# \in \{\coclosed, \exact\}$, and consider a probability measure on \emph{spin $1$-forms} $J\in H_{\#}(\mathbb{S})$ defined by
\begin{align} \label{def:xy}
	d\mu_{\#}(\J)=d \mu_{G, \#}(\J) =\frac{1}{Z_{\#}} \Big(\prod_{e \in E}w(\J_e) \Big) d \J,
\end{align}
where $Z_{\#}$ is the \emph{partition function}, and $d \J$ denotes the Haar probability measure on the group $H_{\#}(\mathbb{S})$. 
%We denote by $\mu_{\#}[\cdot]$ the expectation with respect to $\mu_{\#}$.
For a 1-form $\epsilon \in \Omega^1(\RR)$, we define the \emph{twisted partition function}
\begin{align*}
	Z_{\#}(\epsilon) = \int_{H_{\#}(\mathbb{S})} \prod_{e\in  E} w  (\J_{e}+\epsilon_e ) d \J, 
\end{align*}
and note that $Z_{\#}(0)=Z_{\#}$.
We also define a probability measure on \emph{height $1$-forms} $\n \in H_{\#}(\ZZ)$ by
\begin{align} \label{eq:defh}
	\nu_{\#}(\n) =\nu_{G, \#} (\n) \propto \exp\Big(-\sum_{vv'\in E}\mathcal V(\n_{vv'})\Big).
\end{align}
Note that this is well defined as the normalisation constant is finite by assumption~\eqref{eq:Vsum}. 

%where $v\sim v'$ means that $\{v,v'\}\in E$.
%Here $\partial G\subset V$ is a chosen set of \emph{boundary vertices} that are wired together.
% We write $\nu$ for the expectation with respect to $\nu$.

For $f,g\in  \Omega^1(\mathbb R)$ and $\epsilon, \omega\in \Omega^1(\mathbb R)$, we will write 
\[
(f,g)_{\Omega^0}= \sum_{v\in V} f_vg_v, \qquad \text{and} \qquad (\epsilon, \omega)_{\Omega^1} = \frac{1}{2}\sum_{\vec{e} \in \vec{E}} \epsilon_{\vec{e}} \:\omega_{\vec{e}} =\sum_{{e} \in {E}} \epsilon_{{e}} \:\omega_{{e}} 
\]
for the standard inner products. We will usually drop the subscripts and simply write $(\cdot,\cdot)$ in case the space is clear from the context.

The central result that we will use is the following duality formula. Even though it is classical (see e.g.\ Appendix A in~\cite{FroSpe}), we will provide its derivation in Appendix~\ref{ap:duality}.
\begin{lemma}[Fourier--Pontryagin duality] \label{lem:duality}
	Let $\# \in \{\coclosed, \exact\}$ and let $-\#$ denote the other element of $ \{\coclosed, \exact\}$. 
	Then for any $\epsilon \in \Omega^1(\RR)$, we have
	\[
	\nu_{-\#} [\exp({ i (\n, \epsilon)}) ] = \frac{Z_{\#}(\epsilon)}{Z_{\#}}=\mu_{\#}\Big[\prod_{e\in E} \frac{w  (\J_e+\epsilon_e )}{w  (\J_{e})}\Big].
	\]
\end{lemma}



Clearly there are two intertwined random objects in the statement of Lemma~\ref{lem:duality}: the height and spin $1$-forms $\n$ and $J$ respectively. We will mostly apply the duality to analyse one of these two models
whose values are the \emph{exact} $1$-forms $H_\exact(\mathbb G)$, since then for each $\omega \in H_\exact(\mathbb G)$, there exists a unique $\tau \in \mathbb G^V$ such that
\[
\d\tau = \omega \qquad \text{ and } \qquad \tau_\partial =0,
\]
 where $\partial\in V$ is a fixed \emph{boundary} vertex of $G$, and $0$ is the identity element of $\mathbb G$. The random configuration $\tau$ is then distributed as a classical spin system with spins assigned to vertices with $0$ boundary conditions at $\partial$, and that interact through edges.


\begin{remark} 
In two dimensions there is a special form of duality where $\Hstar$ on the planar graph $G$ can be seen as $\Hcycle$ on the \emph{planar dual} graph $G^*$ by simply rotating all directed edges by $\pi/2$ to the left.
	Therefore if $\omega$ is a $1$-form such that $\d^*\omega = 0$, 
	there exists a function $\tau$ on the vertices of the {dual graph} $G^*$ (faces of $G$) which has $\omega$ as its gradient, i.e. 
	\begin{align*}
		\omega_{vv'} = \tau_u-\tau_{u'} = \d \tau_{uu'},
	\end{align*}
	where $u,u'$ are the two faces adjacent to $vv'$ from the right and left respectively. In this case, both models in Lemma~\ref{lem:duality} can be seen as classical spin and height function models.
\end{remark}


\begin{remark}
	As mentioned in the introduction, the Fourier--Pontryagin duality is usually applied in the opposite direction to Lemma~\ref{lem:duality}, i.e., to compute the characteristic function of the spin model rather than the height function. On the height function side this results in expectations of nonlocal observables (disorders) which are in general difficult to analyse.
	In our case however the disorder appears on the spin model side, and can be removed from the picture by taking derivatives at zero of the characteristic function.
	This is the main point of view which allows to obtain most of the results in this article using comparatively elementary arguments.
\end{remark}



One of the main tools in this article is the following identity. Even though it is a rather direct consequence of duality, we were unable to find this formulation in the literature.

\begin{lemma}[Covariance duality]\label{L: covariance} 
	Let $\# \in \{\coclosed, \exact\}$ and let $-\#$ be the other element of $ \{\coclosed, \exact\}$. For any $\epsilon, \omega \in \Omega^1(\RR)$, we have
	\[
 \E_{\#}\big[(\n, \epsilon)(\n, \omega)\big] + \mu_{-\#} \big[(\c{U}'(\J), \epsilon)(\c{U}'(\J), \omega)\big] = \sum_{e \in E} \mu_{-\#}[\c{U}''(\J_e)] \epsilon_e\omega_e.
	\]
\end{lemma}

\begin{proof}
It is enough to compute 
\[
\frac\partial {\partial s} \frac\partial {\partial t}	\Big(\nu_{\#} [\exp({i (h,s \epsilon +t \omega)}) ] \Big) \Big|_{s =t =0}
\]
by differentiating under the sign of integration on the right-hand side of the formula from Lemma~\ref{lem:duality}.
\end{proof}


\begin{remark}
In the case of measures $\nu_{\exact}$ on true height functions $h$, the quantity $\E_{\exact}[(\n, \epsilon)(\n, \omega)]=\E_{\exact}[(h,\d^* \epsilon)(h,\d^* \omega)]$ explicitly depends only on $\d^*\epsilon$ and $\d^*\omega$, and hence the rest of the equation above does so implicitly.
\end{remark}


Choosing $\epsilon = \id_{xy} - \id_{yx}$ for some edge $xy$ and $\tau = \id_{uv} - \id_{vu}$ for another edge $uv$, as a corollary we immediately get the following identities:
\begin{align} \label{eq:pontwiseD1}
\E_{\#}[\n_{xy}^2] = \mu_{-\#}[ \mathcal U''(J_{xy})- \mathcal U'(J_{xy})^2]
\end{align}
and
\begin{align} \label{eq:pontwiseD2}
\E_{\#}[\n_{xy}\n_{uv}] = -\mu_{-\#}[\c{U}'(\J_{xy})\c{U}'(\J_{uv})].
\end{align}



\begin{remark}
	Note that Lemma~\ref{L: covariance} implies that the sum of the covariance matrices of two mutually dual edge fields is diagonal, i.e., equals the covariance matrix of (possibly inhomogeneous) white noise. This was known for the discrete Gaussian Free Field ($\mathbb G=\mathbb R$), see e.g.~\cite{dubedat2011topics, Aru15} and Remark~\ref{rem:gff}, in which case the independent sum of the mutually dual edge fields is a collection of independent normal random variables. 
	The continuum analogue for the GFF can be found in \cite{AKM}. 
\end{remark}

\begin{remark} 
		Arguments similar in spirit to Lemma \ref{L: covariance}, but in the context of the Villain model appear in \cite{GaSe}. 
\end{remark}

Having established the covariance duality formula in Lemma~\ref{L: covariance}, we will now discuss several of its rather direct consequences. 
%Most of the results obtained in this section are relatively simple to derive, 
%and rely on some classical potential theory together with known correlation inequalities. 
Unless stated otherwise, we study the models on a finite graph $G = (V,E)$ with a prescribed boundary vertex $\partial \in V$.
We will write $ \Omega_0^0(\mathbb G)$ for the set of functions $f\in \Omega^0(\mathbb G)$ with $f_\partial =0$.
% !TEX root = ../main_duality.tex


\section{Upper bound on the variance of the height function} \label{sec:upperbound}
In this section we consider random {exact} $1$-forms $\n \in \Hcycle(\mathbb Z) $ distributed according to~$\nu_\exact$. As mentioned before, for each such $1$-form $\n\in \Hcycle(\mathbb Z)$, there exists exactly one \emph{height function} $h\in \Omega_0^0(\mathbb Z)$ such that $ \d h =\n $.
Note that in the case of $\mathbb G=\mathbb R$, $\d$ and $\d^*$ are adjoint as linear operators, i.e., for all 
$f\in \Omega^0(\mathbb R)$ and $\omega\in\Omega^1(\mathbb R)$, we have
\begin{align} \label{eq:adjoint}
(f, \d^* \omega)_{\Omega^0} = (\d f,\omega)_{\Omega^1}.
\end{align}
Also note that the operator 
\[
\Delta:=\d^*\d: \Omega^0(\mathbb R) \to \Omega^0(\mathbb R)
\] 
is the \emph{graph Laplacian} on $G$, and it has a well defined inverse $\Delta^{-1}$ on $\Omega^0_0(\mathbb R)$. Moreover, as matrices,
\[
\Delta^{-1} =  \Green D^{-1},
\]
where $D=\textnormal{Diag}(\textnormal{deg}(v))_{v\in V\setminus \{ \partial\}}$ and $\Green$ is the Green's function of simple random walk on $G$ killed upon hitting $\partial$.

Let $f \in  \Omega^0_0(\mathbb R)$ and $\epsilon :=\d \Delta^{-1}f$ so that $\d^*\epsilon =f$. Discarding the explicitly nonnegative term $\mu_{\coclosed} [(\c{U}'(\J), \epsilon)^2]$ in Lemma~\ref{L: covariance} applied to $\epsilon=\omega$, and using \eqref{eq:adjoint} we get
\begin{align} \label{eq:upperbound}
	\nu_{ \exact}[(h, f)_{\Omega^0}^2] =\nu_{ \exact}[(\n, \epsilon)_{\Omega^1}^2] \leq \sum_{e \in  E} \epsilon_{e}^2 \big|\mu_{\coclosed}[\mathcal U''(\J_e) ]\big|  \leq C (\epsilon, \epsilon)_{\Omega^1},
\end{align}
where 
\[
C=\sup_{e\in E} |\mu_{\coclosed}[\mathcal U''(\J_e)]| \leq \sup_{J\in \mathbb S} |\mathcal U''(J)|<\infty.
\] 
On the other hand, by \eqref{eq:adjoint} again	$(\epsilon, \epsilon)_{\Omega^1}= (\d \Delta^{-1}f,\d \Delta^{-1}f)_{\Omega^1}=  ( \Delta^{-1}f,f)_{\Omega^0}$.

\begin{corollary}[GFF upper bound on variance]\label{cor:upper}
	For any $f\in \Omega^0_0(\mathbb R)$,
	\begin{align*}
		\nu_{\exact}[(h, f)^2] \leq  C  ( \Delta^{-1}f,f)_{\Omega^0}=C\sum_{v,v'\in V} f_vf_{v'}\frac{\Green(v, v')}{\deg(v')},
	\end{align*}
	where $C$ is as above.
\end{corollary}

\begin{remark} \label{rem:gff}
	One can also apply duality to the discrete Gaussian free field (GFF) (in this case both the primal and dual fields are real-valued as $\mathbb R$ is self-dual as a locally compact abelian group).
	The GFF is defined similarly to the integer-valued GFF with potential $\mathcal V(t)=t^2$ with the difference that the reference measure in~\eqref{eq:defh} is the Lebesgue measure on $\mathbb R$ and not the counting measure on $\mathbb Z$. 
	The model is self dual in that $\mathcal U(t)=\mathcal V(t)=t^2$, and in the analog of the corollary above actually get an equality since $( \epsilon, \mathcal U'(\J_e))=0$ since $\mathcal U'(\J_e) =2 \J_e \in H_{\coclosed}$, and $\d^* \epsilon \in H_{\exact}$ by definition. This agrees with the fact that the covariance of the GFF is given \emph{exactly} by the inverse Laplacian.
\end{remark}


\begin{remark} \label{rem:tightness}
	Consider an infinite countable graph $\Gamma=(\mathscr V, \mathscr E)$ and a sequence of increasing finite subgraphs exhausting $\Gamma$, i.e., $G_N \nearrow \Gamma$ as $N\to \infty$.
	If $f: \mathscr V\to \mathbb R$ has bounded support and mean zero, i.e., $\sum_{v\in \mathscr V} f_v=0$, where this sum is actually taken over a finite set of vertices, then we can find a 1-form $\epsilon$ on $\mathscr E$ with \emph{bounded} support such that $\d^* \epsilon=f$, and hence 
	\begin{align}\label{eq:local}
	\prod_{e\in E} \frac{w  (\J_{e}+\epsilon_e )}{w  (\J_{e})}=\prod_{e\in \textnormal{supp} (\epsilon)} \frac{w  (\J_{e}+\epsilon_e )}{w  (\J_{e})}
	\end{align}
	is a local bounded continuous function of $\J$ (in the product topology on $\mathbb S^{\mathscr E}$) whenever $\mathcal V$ and $w$ are as in Definition~\ref{def:potential}. Moreover, since $\mathbb S$ is compact metrizable so is $\mathbb S^{\mathscr E}$ with the product topology by Tychonoff's theorem, 
	and hence the edge spin models $\mu_{G_N,\#}$ always form a tight sequence of measures on $\mathbb S^{\mathscr E}$ as $N\to \infty$. 
	This in particular implies that $\mu_{G_N,\coclosed}$ converges weakly along a subsequence.
	Therefore Lemma~\ref{lem:duality} together with \eqref{eq:local} and the fact that
	\[
		\nu_{G_N,\exact} [\exp({i (\n, \epsilon)}) ]=\nu_{G_N,\exact} [\exp({ i (f, h)}) ]
	\]  
	for $N$ large enough so that $G_N$ contains $\textnormal{supp} (\epsilon)$, imply that the random height $1$-forms $\n$ under $\nu_{G_N,\exact}$, and hence also the \emph{differences} of the associated height function $h$, converge weakly along the same subsequence. 
	
One has to be careful as this is in general no longer true if $f$ does not have zero mean, e.g., $f=\delta_v$. Then $\epsilon$ with $\d^* \epsilon=f$ cannot be taken with bounded support 
(there always has to be an infinite path with nonzero values of $\epsilon$).
In this case tightness may fail when \emph{delocalisation} of the height function arises, i.e., $\nu_{G_N,\exact}[(h, f)^2]= \nu_{G_N,\exact}[h_v^2]\to \infty$ 
as $N\to \infty$ (e.g.\ if $\Gamma$ is planar, see Section~\ref{sec:bkt}).
\end{remark}





We also immediately deduce that delocalisation of the height function does not happen on transient graphs for potentials as in Definition~\ref{def:potential}.
We note that our result, despite its simple proof, seems new in this generality, and that such behaviour is expected for a larger class of potentials. 
We also note that the special case of the integer-valued GFF follows from a stronger estimate proved by Fr\"{o}hlich and Park~\cite{FroPar} (see also~\cite{KP}).
Some results in this direction related to the intever-valued GFF can also be found in \cite{AHPS}.

To state the result, we briefly recall the notion of Gibbs measures and {gradient} Gibbs measures (we do it for height functions only, and the definition for spin models used later in the article is completely analogous).
From now on we assume that $\Gamma = (\mathscr V, \mathscr E)$ is a locally finite, infinite graph. 
For a finite set $\Lambda \subset \mathscr V$ write $E(\Lambda)$ for the set of edges with at least one vertex in $\Lambda$.  
Let $\varphi: \Lambda^c \to \ZZ$ be a function and define the probability measure $\mu_{\Lambda}^\varphi$ 
supported on $h: \mathscr{V} \to \ZZ$ satisfying $h \mid_{\Lambda^c} = \varphi$ by
\[
	\nu_{\Lambda}^{\varphi}(h) \propto \exp \Big(-\sum_{e \in E(\Lambda)} \c{V}(\d h_e)\Big). 
\]
In other words, $\nu_{\Lambda}^\varphi$ is the measure $\PP_{\exact}$ from \eqref{eq:defh} with $\varphi$-boundary conditions outside $\Lambda$. 
A probability measure $\nu$ supported on height functions $h: \mathscr V \to \ZZ$ is called a \textit{Gibbs measure} (on $\Gamma$ with respect to the potential $\c{V}$) if it satisfies the Dobrushin--Lanford--Ruelle (DLR) relations:
for all finite sets $\Lambda \subset \mathscr{V}$, 
\[
	\nu_{\Lambda}(\cdot) = \int_{\mathbb{Z}^{\mathscr V}} \nu_{\Lambda}^{\varphi}(\cdot) d \nu(\varphi), 
\]
where $\nu_{\Lambda}$ denotes the restriction of $\nu$ to $\Lambda$.
If $\Gamma$ is a Cayley graph and the measure $\nu$ is invariant under shifts, it is called translation invariant. 
In terms of Gibbs measures, delocalisation corresponds to non-existence of translation invariant Gibbs measures. 

A \emph{gradient} Gibbs measure is a slight variation of the above, where we consider measures supported only on gradients. 
Fix this time a finite set of edges $\Lambda \subset \mathscr{E}$. Let $\omega$ be an exact $1$-form (thus taking value in $\Hcycle(\mathscr{E}, \ZZ)$). 
Define the probability measure $\mu_{\Lambda}^{\omega}$ supported on $1$-forms $\n \in \Hcycle(\mathscr{E}, \ZZ)$ satisfying $h \mid_{\Lambda^c} = \omega \mid_{\Lambda^c}$ as
\[
	\mu_{\Lambda}^{\omega}(\n) \propto \exp\Big(-\sum_{e \in \Lambda} \c{V}(\n_e)\Big). 
\]
A probability measure supported on $1$-forms $\n \in \Hcycle(\mathscr{E}, \ZZ)$ will be called a gradient Gibbs measure if it satisfies the analog of the DLR equation above
in this setup. 

\begin{theorem} \label{thm:extransinv}
	Let $\Gamma = (\mathscr V,\mathscr E)$ be a transient graph and $\c{V}$ a height function potential as in Definition~\ref{def:potential}. 
	Then, there exists an infinite volume Gibbs measure on $\Gamma$ with respect to $\mathcal V$. If $\Gamma$ is moreover an amenable Cayley graph, 
	there exist translation invariant Gibbs measures. 
\end{theorem}

\begin{proof}
Let now $G_N\nearrow \Gamma$, as $N\to \infty$ be an exhaustion of $\Gamma$ by finite subgraphs $G_N$. 
Define the boundary $\partial_N := \partial G_N$ to be the set of vertices in $ G_N$ adjacent to a vertex from outside of $ G_N$.
Let $\E_{G_N, \exact}[\cdot]$ be the expectation associated with the height function on $ G_N$ with $0$-boundary conditions. 
Fix any $\Lambda \subset \c{V}$ finite and let $f: \Lambda \to \RR$ be any function.
It follows from Corollary~\ref{cor:upper} that 
\[
	\E_{G_N,\exact}[(h, f)^2] \leq C \max_{v \in \Lambda} \frac{\Green_{N} (v,v)}{\deg(v)} (f, f)_{\Lambda}. 
\]
where $C < \infty$, and $\Green_N$ is the Green's function of simple random walk on $G_N$ killed on hitting $\partial_N$. 
Since $\Gamma$ is transient, the right-hand side is uniformly bounded in $N$. 
Therefore, the sequence $\PP_{G_N, \exact}(h |_{\Lambda})$ is tight and subsequential limits exist by Prokhorov's theorem. 
By a diagonal argument, we can extract a further sub-sequence $N_K$ so that $\PP_{G_{N_K}, \exact}(h|_{\Lambda})$ converges for each
$\Lambda$ finite. 
Any such subsequential limit is a Gibbs measure as it satisfies the DLR relations. 
This proves the first part of the theorem. 

For the second part, suppose that $\Gamma$ is an amenable Cayley graph, 
so that 
\(
	\E_{G_N, \exact}[h_v^2] \leq C',
\)
for some $C' < \infty$ which is independent of $v$ and the exhaustion $(G_N)_{N\geq 1}$. 
Let $\mu$ be a subsequential limit (which exists by the argument above, and is a Gibbs measure). 
Let $o \in \mathscr V$. Since $\Gamma$ is amenable, there is some F\o lner sequence (also called Van Hove sequence) $(F_N)_{N \geq 1}$ {of sets of vertices} containing $o$. 
This means $F_N \nearrow \mathscr V$ and $|\partial F_N| / |F_N| \to 0$ as $N \to \infty$. 
Let
\[
	\nu_N := \frac{1}{|F_N|} \sum_{x \in F_N} \mu \circ \theta_x, 
\]
where $\theta_x$ is the shift towards $x$ (since $\Gamma$ is a Cayley graph, this is the same as left multiplication in the group).
This is a Gibbs measure because the set of Gibbs measures is closed under translations and convex combinations. 
Moreover, $\nu_N[h_v^2] \leq C'$ for each $N$ and $v \in \mathscr V$. Therefore, $(\nu_N)_{N\geq 1}$ is tight. 
Let $\nu$ be any subsequential limit, which is again a Gibbs measure. 
By construction and since $|\partial F_N| / |F_N| \to 0$, we have $\nu \circ \theta_x = \nu$ for each $x$, and hence $\nu$ is translation invariant.
\end{proof}



% !TEX root = ../main_duality.tex


\section{Delocalisation implies the BKT phase transition in two dimensions} \label{sec:bkt}
\subsection{Background}
In this section we consider the spin and height function models on the square lattice $\mathbb Z^2$ and we show that delocalisation of the height function (defined below) is equivalent to the divergence of a certain series of two-point functions in the dual spin model. One of the conclusions is that delocalisation implies the Berezinskii--Kosterlitz--Thouless (BKT) phase transition in the dual spin model~\cite{Ber1,KosTho}. 

This implication for the classical XY and the Villain spin models, together with a proof of delocalisation of the associated height functions, was first obtained by Fr\"{o}hlich and Spencer in their seminal work establishing the BKT transition~\cite{FroSpe} (also see~\cite{KP} for an exhaustive account).
Recently alternative proofs were provided by Aizenman et al.~\cite{AHPS} (first for the Villain and later also for the XY model) and by the authors~\cite{vEnLis} for the XY model. Together with the new conceptual approach to delocalisation introduced by Lammers~\cite{Lammers}, these works improve our mathematical understanding of the BKT transition.
These results can be thought of as an inequality between the critical points of the mutually dual spin and height function models.
A natural conjecture is that these critical points always coincide. In the case of the XY and Villain model this was confirmed in a recent work Lammers~\cite{Lam23}.


In this section we provide yet another, and arguably the simplest so far, proof of the fact that delocalisation of the height function implies that correlations functions of certain observables in the spin model do not decay exponentially fast in the distance. 
For reflection positive models (which is the case when $-\mathcal U$ is itself positive definite, i.e., has nonnegative coefficients in the Fourier series), we moreover obtain an equivalence between delocalisation and nonsummability of spin correlations.
Our approach, unlike the previous ones, is based solely on duality, and does not invoke any additional (e.g.~graphical) representations of the models at hand. 



\subsection{Notions of delocalisation}
It is now well established that integer-valued height functions on $\mathbb Z^2$ (or in general on periodic planar lattices) undergo a phase transition between a
\emph{localised} \emph{(smooth)} and a \emph{delocalised} \emph{(rough)} regime~\cite{FroSpe,CPST,DHLR,Lis21,DKMO,Lammers,LamOtt,Lammers22,Lam23}. 
%One can interpret this transition as the breaking and recovery of the translation symmetry of $\mathbb Z$.
We say that a potential $\mathcal V$ is localised (on $\mathbb Z^2$) if it admits a translation-invariant Gibbs measure on height functions $h: \mathbb Z^2\to \mathbb Z$. Otherwise we say that $\mathcal V$ delocalises.
It is known that if $\mathcal V$ is {convex} on the integers, and moreover its second discrete derivative is nonincreasing, i.e., $\mathcal V$ is a so-called \emph{supergaussian} potential~\cite{LamOtt,Lammers22}, then delocalisation in this sense is equivalent to the fact that 
\begin{align} \label{eq:vardeloc}
\sup_{N \geq 1} \nu_{\Lambda_N,\exact}[h_{\mathbf 0}^2] =\infty,
\end{align}
where $\mathbf 0$ is the origin of $\mathbb Z^2$, and $\Lambda_N=[-N,N]^2\cap \mathbb Z^2 $ where we identify all vertices in $\Lambda_N$ that are adjacent to $\Lambda^c_N:=\mathbb Z^2 \setminus \Lambda_N$ as one boundary vertex $\partial$ (wired boundary) and set $h_\partial =0$. Moreover for such potentials, the sequence in~\eqref{eq:vardeloc} is nondecreasing in~$N$~\cite{LamOtt}, and it grows up to a mulitplicative constant at least like $\log N$~\cite{Lammers22} (which is consistent with the general conjecture stating that delocalised height functions should behave like the GFF at large scales). 

Yet another approach to delocalisation is to work with infinite volume \emph{gradient} measures and study the variance of the increment of the height between two distant points. This was e.g.\ studied in~\cite{Lis19,Lis21} in the context of the six-vertex model and it will be convenient for us to follow the same route here, as we already know by Remark~\ref{rem:tightness} that translation invariant gradient Gibbs measures always exist for potentials as in Definition~\ref{def:potential}. 
We say that a potential $\mathcal V$ is $\nabla$-\emph{delocalised} (on $\mathbb Z^2$) if for any translation-invariant gradient Gibbs measure $\nu$ (with expectation $\nu$), we have 
\begin{align} \label{eq:graddeloc}
\sup_{v\in \mathbb Z^2} \nu[(h_v-h_{\mathbf 0})^2]= \infty.
\end{align}

\begin{lemma} \label{lem:nabla}
If a potential $\mathcal V$ is delocalised, then it is also $\nabla$-delocalised.
\end{lemma}
\begin{proof}
Suppose otherwise that there exists a translation-invariant gradient Gibbs measure~$\nu$ for which the supremum in~\eqref{eq:graddeloc} is finite.
Then, as in Theorem~\ref{thm:extransinv}, by considering convex combinations of translations of $\nu$ thought of as a measure on height functions $\tilde h$ given by $\tilde h_v=h_v-h_{\mathbf 0}$ we can construct a translation invariant Gibbs measure on height functions which is a contradiction. We leave the details to the reader.
\end{proof}

We note that the opposite implication is also true e.g.\ for potentials $\mathcal V$ that are convex on the integers. Indeed, in this case it is known from the foundational work of Sheffield~\cite{sheffield} that each Gibbs measure for height functions has a finite second moment (since the height at every point has a log-concave distribution). 


\subsection{Setup}\label{sec:setup}
Let us fix mutually dual potentials $\mathcal V$ and $\mathcal U$ as in Definition~\ref{def:potential}.
It will be convenient to consider the spin and height function models on finite, exponentially growing tori $\mathbb T_N= (\mathbb Z / 2^N\mathbb Z)^2$. This way we achieve three properties by construction:
\begin{itemize}
\item we work with measures that are translation invariant and invariant under $\pi/2$-rotations,
\item we can apply the duality of Lemma~\ref{lem:duality} first in the finite volume $\mathbb T_N$, and then take simultaneous (subsequential) infinite-volume limits, $\mathbb T_N\to \mathbb Z^2$ as $N\to \infty$, on both sides of the duality relation,
\item we get an explicit monotonicity in $N$ for the Green's function of the random walk on $\mathbb T_N$ (see below).
\end{itemize}

Let $\mu=\mu_{\mathbb Z^2,\coclosed}$ be any subsequential limit of $\mu_{\mathbb T_{N}, \coclosed}$, and let $\nu =\nu_{\mathbb Z^2,\exact}$ denote the limit of $\nu_{\mathbb T_{N}, \exact}$ taken along the same subsequence (it exists by Remark~\ref{rem:tightness}).
One can think of $\nu$ as a probability measure on height functions $h:\mathbb Z^2\to \mathbb Z$ satisfying $h(\mathbf 0)=0$. By weak convergence, the duality of Lemma~\ref{lem:duality} holds also for $\mu$ and $\nu$ whenever $\epsilon \in \Omega^1(\mathbb Z^2,\RR)$ is of bounded support.
The same is true for Corollary~\ref{L: covariance} and Corollary~\ref{cor:upper}, where we choose $\partial = \mathbf 0$ and consider the Green's function of a random walk on $\mathbb Z^2$ killed at $\mathbf 0$.

Note that by planar duality, we have $H_\coclosed(\mathbb Z^2, \mathbb S) \cong H_\exact((\mathbb Z^2)^*, \mathbb S) $,
where $(\mathbb Z^2)^* \cong \mathbf 0^*+\mathbb Z^2$ with $\mathbf 0^*:=(1/2,1/2)$, is the \emph{dual} square lattice. 
Since $ H_\exact((\mathbb Z^2)^*, \mathbb S) \cong {\mathbb S}^{(\mathbb Z^2)^*\setminus \{ \mathbf 0^* \}}$, we can think of $\mu$ as a Gibbs measure on spin configurations $\theta$ on $(\mathbb Z^2)^*$ where the spin at $\mathbf 0^*$ is fixed to be the identity element of $\mathbb S$. 
%To make the measure translation invariant, we will rotate all spins by an independent uniform element of $\mathbb S $. We will write $\theta$ for the resulting spin configuration, and with a slight abuse of notation, we will write $\mu$ for the measure that also incorporates the randomness of the spin at $\mathbf 0^*$.

Finally, let $v_n=(n,0)\in \mathbb Z^2$, and
let $p_n=(e_0,e_1,\ldots,e_{n-1})$ be the directed horizontal path from $v_0$ to $v_{n}$. We identify $p_n$ with the 1-form that assigns $1$ to each directed edge in $p_n$, 
and $0$ to the directed edges of $\mathbb Z^2$ that are not in $p_n$. For compactness of notation, we write $\J_i=\J_{e_i}$ and $h_i=h_{u_i}$.

\subsection{The implication}

Applying Lemma~\ref{L: covariance} and Corollary~\ref{cor:upper} in finite volume, and then taking the subsequential limit as in Section~\ref{sec:setup}, we have 
\begin{align} \label{eq:vanish}
	0\leq \sum_{i=0}^{n-1}   \mu[\mathcal U''(\J_i)] - \mu \Big[\Big(\sum_{i=0}^{n-1}\mathcal U'(\J_i)\Big)^2\Big]\leq \limsup_{N\to \infty}  \nu_{\mathbb T_{N}, \exact}[(h_{0} -h_{n})^2]\leq \limsup_{N\to \infty} \mathbf G_{N}(v_n,v_n),
\end{align}
where $\mathbf G_{N}$ is the Green's function of simple random walk on $\mathbb T_N$ killed at $\mathbf 0$. This is equivalent to a random walk on $\mathbb Z^2$ killed at all points in $2^N\mathbb Z^2$. Hence, $\mathbf G_{N}\nearrow \mathbf G$ as $N\to \infty$, where $\mathbf G$ is the Green's function of a random walk on $\mathbb Z^2$ killed at $\mathbf 0$.
Classically we have $\mathbf G(v_n,v_n)\leq \textnormal{const}\times\log n$ (see e.g.~\cite{LyPer}). Plugging this bound into \eqref{eq:vanish}, dividing both sides by $n$,
letting $n\to \infty$, and finally using translation invariance of $\mu$, we get
\begin{align}\label{eq:cesaro}
	\lim_{n\to \infty}\frac{1}{n}\sum_{k=1}^{n-1} u_k=\frac12 \mu[\mathcal U''(\J_0)-\mathcal U'(\J_0)^2] , \quad  \textnormal{where} \quad u_k=\sum_{i=1}^k \mu [\mathcal U'(\J_0)\mathcal U'(\J_i)].
\end{align}
In particular $u_k$ converges in the Ces\`aro sense as $k\to \infty$. %Note that $\mu [\mathcal U'(\J_0)\mathcal U'(\J_i)]$ is not-necessarily of definite sign.

\begin{theorem}[Delocalisation implies the BKT phase transition] \label{thm:BKT}
Consider the setup from Section~\ref{sec:setup}.
	If the height function delocalises in the sense that~\eqref{eq:graddeloc} holds true for~$\nu$, then
	\begin{align*}
		\sum_{i=1}^\infty  i |\mu [\mathcal U'(\J_0)\mathcal U'(\J_i)]|= \infty. %\sum_{i=1}^\infty \sum_{k={i}}^\infty  |\mu [\mathcal U'(\J_0)\mathcal U'(t_k)]|=\infty.
	\end{align*}
	In particular, there is no exponential decay of the two-point function $\mu [\mathcal U'(\J_0)\mathcal U'(\J_i)]$ as $i\to \infty$.%, and a Berezinskii--Kosterlitz--Thouless phase transition occurs in the spin model.
\end{theorem}
\begin{proof}
	We can assume that $\sum_{i=1}^\infty |\mu [\mathcal U'(\J_0)\mathcal U'(\J_i)]|<\infty$ since otherwise we are done. This means that $u_k$ converges in the classical sense to its Ces\`aro limit
	from \eqref{eq:cesaro}. Hence, 
	\begin{align} \label{eq:exid}
		\mu[\mathcal U''(\J_0)-\mathcal U'(\J_0)^2] = \lim_{k\to \infty}2 u_k = 2\sum_{i=1}^\infty  \mu [\mathcal U'(\J_0)\mathcal U'(\J_i)].
	\end{align}
	By Lemma~\ref{L: covariance} applied in the infinite volume ($p_n$ has bounded support) and translation invariance of $\mu$, we have
	\begin{align*}
		\nu[(h_{n}-h_{ 0})^2] & =\sum_{i=0}^{n-1}\mu[\mathcal U''(\J_i)-\mathcal U'(\J_i)^2] -2\sum_{i=1}^{n-1} (n-i) \mu [\mathcal U'(\J_0)\mathcal U'(\J_i)] \\
		&= 2n \sum_{i=1}^\infty\mu [\mathcal U'(\J_0)\mathcal U'(\J_i)] -2\sum_{i=1}^{n-1} (n-i) \mu [\mathcal U'(\J_0)\mathcal U'(\J_i)]\\
		&= 2\sum_{i=1}^{n-1}i\mu [\mathcal U'(\J_0)\mathcal U'(\J_i)] +2n\sum_{i=n}^{\infty}  \mu [\mathcal U'(\J_0)\mathcal U'(\J_i)] \\
		&\leq 2 \sum_{i=1}^\infty  i |\mu [\mathcal U'(\J_0)\mathcal U'(\J_i)]|.
	\end{align*}
	By the assumption, and translation and $\pi/2$-rotation invariance of $\mathcal \nu$, we have
	\[
\infty =\sup_{v\in \mathbb Z^2} \nu[(h_{v}-h_{\mathbf 0})^2]\leq 2 \sup_{n\geq 1} \nu[(h_{n}-h_{\mathbf 0})^2],
	\]
which together with the inequality above finishes the proof.
\end{proof}

It is classical that spin correlation functions decay exponentially fast at high temperatures (here the temperature is incorporated in the definition of $\mathcal U$).
This in particular implies that $\sum_{i=1}^\infty i |\mu [\mathcal U'(\J_0)\mathcal U'(\J_i)]|<\infty$. 
From this point of view Theorem~\ref{thm:BKT} says that if the height function delocalises, then the associated
spin model undergoes a BKT phase transition from a regime with exponential decay to a regime with slow decay of correlations.





\subsection{The case of the $XY$ model}
The change of behaviour of the two-point functions $\mu[\mathcal U'(J_0) \mathcal U'(J_i)]$ as $i\to \infty$ clearly indicates a phase transition in the spin model. However it is more common to look at correlations of the type $\mu[\mathcal F(\theta_u-\theta_{u'})]$ when $u$ and $u'$ are far apart, where~$\theta$ is the underlying spin field on $(\mathbb Z^2)^*$ (the faces of $\mathbb Z^2$), and where $\mathcal F$ is some chosen function, e.g. $\mathcal F=\mathcal U$. 

For general spin models, it is not clear how to compare these two types of correlation functions. Here we present an approach based on correlation inequalities in the case of the classical XY model, i.e., when $\mathcal U(t)=- \beta\cos(t)$, where $\beta>0$ is the inverse temperature in the spin model.

To this end, consider the setup as in Theorem~\ref{thm:BKT}. If $\{u_i,u_i'\}$ is the dual edge of $e_i$, writing $\theta_i=\theta_{u_i}$ and $\theta_i'=\theta_{u_i}'$, we have
\begin{align} \label{eq:trigon}
	\frac2{\beta^2} \mu[\mathcal U'(e_0) \mathcal U'(e_i)] &= 2\mu[\sin(\theta_0-\theta_0') \sin(\theta_i-\theta_i')] 
	\\ &=\mu[\cos(\theta_0-\theta_0'-\theta_i+\theta_i')]-\mu[\cos(\theta_0-\theta_0'+\theta_i-\theta_i')] \nonumber
	%\\ &\leq\mu[\cos((\theta_0-\theta_i)+(\theta_0'-\theta_i'))],
\end{align}
A version of the classical Ginibre inequality for the XY model~\cite{Gin} (see also~\cite{BLU}) says that
\begin{align*}
	\mu[ \sin ( \theta _0)\cos(\theta'_0) \sin(\theta_i) \cos(\theta'_i) ] \leq \mu[ \sin ( \theta _0) \sin(\theta_i)] \mu[ \cos(\theta'_0)\cos(\theta'_i) ],
\end{align*}
which after expanding into cosines of sums of angles and disregarding terms that are not invariant under global rotation (shift of angles mod $2\pi$) whose expectations vanish,
we obtain
\begin{align*}
	\mu[\cos(\theta_0+\theta_0'-\theta_i-\theta'_i)]& + \mu[\cos(\theta_0-\theta'_0-\theta_i+\theta'_i)]  -\mu[\cos(\theta_0-\theta'_0+\theta_i-\theta'_i)]  \\
	&\leq 2 \mu[ \cos ( \theta _0-\theta_i)] \mu[ \cos(\theta'_0-\theta'_i) ]  \\
	&= 2 \mu[ \cos ( \theta _0-\theta_i)]^2,
\end{align*}
where the last identity follows by reflection invariance of $\mu$ across the real line.
Analogous inequality follows by exchanging the roles of $\theta_i$ and $\theta'_i$, which results in swapping the signs of the second and third term in the first line.
Using that the first term is positive by the first Griffiths inequality, and combining with \eqref{eq:trigon}, we get that
\begin{align} \label{eq: inequality spin-spin}
	\tfrac12\mu[\cos(\theta_0+\theta_0'-\theta_i-\theta'_i)]+|\mu[\mathcal U'(e_0) \mathcal U'(e_i)] | & \leq \max\{ \mu[ \cos ( \theta _0-\theta_i)]^2,\mu[ \cos ( \theta _0-\theta'_i)]^2\} \nonumber
	\\ &= \mu[ \cos ( \theta _0-\theta_i)]^2,
\end{align}
where the last identity follows from the Messager--Miracle-Sole inequality~\cite{MMS} by applying the reflection across the real line.

These considerations, together with Lemma~\ref{lem:nabla}, lead us to the following corollary that recovers the main result of~\cite{vEnLis}. 

\begin{corollary}\label{C: deloc BKT}
	If the height function associated with the XY model on $\mathbb Z^2$ delocalises, then  
	\[
	\sum_{i=1}^\infty  i  \mu[ \cos ( \theta _0-\theta_i)]^2 \geq \sum_{i=1}^\infty  i  (\mu[ \cos ( \theta _0-\theta_i)]^2-\tfrac12\mu[\cos(\theta_0+\theta_0'-\theta_i-\theta'_i)]) = \infty.
	\] 
	%In particular, correlations do not decay exponentially and a BKT transition occurs. 
\end{corollary}

\begin{remark}
For the XY model it is known that there exists only one translation-invariant Gibbs measure $\mu$ on $\mathbb Z^2$~\cite{Pfister}, and hence regular, instead of subsequential, limits may be taken Section~\ref{sec:setup}.
\end{remark}

\subsection{An equivalence}
When $-\mathcal U$ is itself positive definite, i.e., all its Fourier coefficients are nonnegative, we can actually conclude more than in the above discussion. Indeed, in this case $\mu$ is reflection positive (see Appendix \ref{ap:RP}). This implies that 
\[
\mu [\mathcal U'(\J_0)\mathcal U'(\J_i)] \geq 0
\] 
as this holds true on $\mathbb T_N$ for every $N\geq i$ by reflection positivity. Therefore the Ces\`aro convergence from \eqref{eq:cesaro} implies classical convergence, and \eqref{eq:exid} always holds true.
The same argument as in the proof of Theorem~\ref{thm:BKT} yields the following corollary.
\begin{corollary} \label{cor:equiv}
Consider the setup from Section~\ref{sec:setup}, and moreover assume that $-\mathcal U$ is positive definite. 
Then~\eqref{eq:graddeloc} holds true for~$\nu$ if and only if
	\begin{align} \label{eq:infinite}
		\sum_{i=1}^\infty  i \mu [\mathcal U'(\J_0)\mathcal U'(\J_i)]= \infty. 
	\end{align}
\end{corollary}



\begin{remark}
The identity from \eqref{eq:exid} can be rewritten in a more symmetric form as
\begin{align} \label{eq:finite}
\sum_{i\in \mathbb Z}  \mu [\mathcal U'(\J_0)\mathcal U'(\J_i)]=\mu[\mathcal U''(\J_0)],
\end{align}
where now the sum is over a bi-infinite path of edges. Curiously, this is an exact (but nonlocal) identity for correlation functions that is independent of the (hidden in the definition of $\mathcal U$) 
temperature parameter. In particular the series in \eqref{eq:finite} is always convergent, independently of the temperature. This is in contrast with the behaviour of the series in~\eqref{eq:infinite}
that does undergo a phase transition. This, together with the relation to the gradient of the heigh function~\eqref{eq:pontwiseD2}, is consistent with the conjecture that in delocalised 
regime the discrete GFF describes the large-scale fluctuations of the height function. Indeed, the two-point function of the gradient of the discrete GFF is known to decay like the inverse square of the distance (see e.g.~\cite{BauWeb}).
\end{remark}

% !TEX root = ../main_duality.tex


\section{Monotonicity of variance of the height function in temperature} \label{sec:monot}
In this section we will show that the variance of the height function is monotone in a natural temperature parameter under some further assumptions on the potential.
To the best of our knowledge the result is new, even in the case of planar graphs.
Together with the dichotomy of Lammers \cite{Lammers22}, this directly implies that the height function of the XY model undergoes a sharp phase transition on the square lattice. 

\subsection{Setup} \label{sec:setup2} 
Let $G = (V, E)$ be a finite graph. 
To each edge $e \in E$, associate
\begin{itemize}%[\hspace{0.5cm}(1)]
	\item a twice continuously differentiable spin potential $\c{U}_e: \mathbb{S} \to \RR$ such that $-\mathcal U_e$ is positive definite, 
	\item a non-negative real $\beta_e$ (thought of as the inverse temperature in the spin model), 
	\item the dual potential $\c{V}_{\beta_e} := \c{V}_{\beta_e, e}$ of $\beta_e \c{U}_e$ as in Definition \ref{def:potential}. 
\end{itemize}
We will consider the family of measures $\nu_{\beta, \exact}$ 
for height functions and their dual measures $\mu_{\beta, \coclosed}$, indexed by $\beta = (\beta_e)_{e \in E}$. 

We wish to point out that the above requirements on the spin potential $\c{U}$ can also be described purely in terms of the height function potential:
if $\c{V}$ satisfies the conditions of Definition \ref{def:potential} and $e^{-\c{V}}$ is infinitely divisible (in the sense that each division satisfies Definition \ref{def:potential}), then the corresponding spin potential satisfies the above conditions, see Appendix \ref{ap:div}. 
This equivalence is not important in the remainder of this section. 

\begin{remark}
		The setup here is less general than in the rest of this text, 
		as we need to make a further assumption on the potentials $\c{U}$ (or equivalently on the potentials $\c{V}$ as explained in Appendix \ref{ap:div}). 
\end{remark}

\subsection{Increasing variance}
The main result of this section is the following fact. 
\begin{theorem} \label{T: var_increasing_height}
	Consider the setup as in Section \ref{sec:setup2}. For each $x, y \in V$ and $e \in E$ the function
	\[
		\beta_e \mapsto \nu_{\beta, \exact}[(h_x - h_y)^2]
	\]
	is non-decreasing. 
\end{theorem}

\begin{example} Let us first give some examples of potentials to which this theorem applies. 
	\begin{itemize}
		\item In case of the classical XY model, $-\c{U}(t) = \cos(2\pi t)$ which is positive definite. 
		\item A generalisation of this is any height function that is dual to a spin system with a positive definite potential.
		\item The Gaussian potential $k \mapsto \frac{1}{2\beta} k^2$ does itself not fall into this class. 
		However, it arises as a (rescaled) limit of the XY height function potentials so the conclusion of the above theorem still holds. 
		We will show the result on the height-function side, but it was already known on the dual spin side, see e.g. \cite{NewWu, AHPS}. 
		Let $G$ be a finite graph and let $e \in E$ be some fixed edge. 
		Replace $e$ by $N$ copies of $e$ and set on each copy of this edge the XY potential $\c{V}_{\beta N}(k) = -\log(I_{k}(\beta N))$. 
		Note that the effective height-function potential on the edge $e$ is equal to
		\[
			e^{-\c{V}^{\mathrm{eff}}_N(k)} = I_k(\beta N)^N. 
		\]
		On the other hand, the modified Bessel function $I_k(x)$ has the expansion as $N\to \infty$:
		\[
			I_k(\beta N) = \frac{e^{\beta N}}{ \sqrt{2\pi \beta N}} \left(1 - \frac{4k^2 - 1}{8\beta N} + O(1 / N^2)\right),
		\]
		which goes back to \cite{Kirch_bes}. In particular, as $N \to \infty$, 
		\[
			I_k(\beta N)^N \sim \left(\frac{e^{\beta N}}{ \sqrt{2\pi \beta N}}\right)^N e^{\frac{1}{8 \beta}} e^{- \frac{1}{2\beta}k^2} =: c_{\beta, N} e^{-\frac{1}{2\beta} k^2}, 
		\]
		where the constant $c_{\beta, N}$ does not depend on $k$. 
		Note that for Gibbs measures, the constant $c_{\beta, N}$ corresponds to a global shift of the potential $\c{V}^{\mathrm{eff}}_N$ which does not change the model. 
		Hence, since for each $N$ we can apply Theorem \ref{T: var_increasing_height}, the result holds in the limit as $N \to \infty$, where the potential on the edge $e$ equals $k \mapsto \frac{1}{2\beta}k^2$ as desired. 
	\end{itemize}
\end{example}

To prove the theorem, let us begin by slightly extending Ginibre's inequalities \cite{Gin} to spin models on $\Hstar(\mathbb{S})$ (the original inequality deals with $H_{\exact}(\mathbb{S})$).
\begin{lemma}[Ginibre] \label{L: Ginibre}
		Consider the setup as in Section \ref{sec:setup2}. 
		For all positive definite functions $F:\mathbb{S} \to \RR$ and all $e, f \in E$, we have
		\[
			\frac{\partial}{\partial \beta_e}\mu_{\beta, \coclosed}[F(\J_f)]\geq 0. 
		\]
\end{lemma}
\begin{proof}
	This is proved in Appendix \ref{ap:Gin}. 
\end{proof}





\begin{proof}[Proof of Theorem \ref{T: var_increasing_height}]
We first add an additional edge $g$ connecting $x$ and $y$ (even if there was already such an edge present).
On this edge, we put the potential $-\c{U}_{g}(t) = \cos(t)$ and parameter $\beta_{g} =\lambda \geq 0$. 
Thus, we remain in the setup of Section~\ref{sec:setup2}. We write $\mu_{\beta,\lambda,\coclosed}$ and $\nu_{\beta,\lambda,\exact}$ for the corresponding spin and height-function measure respectively, and note that $\mu_{\beta,\lambda,\coclosed}\to \mu_{\beta,\coclosed}$ as $\lambda\to \infty$. 

Let $\epsilon$ be any $1$-form vanishing on $g$ and such that $\d^*\epsilon=\delta_x-\delta_y$, and let $\epsilon'$ be the $1$-form vanishing outside of $g$ and such that $\d^*\epsilon'=\delta_x-\delta_y$.
By Lemma \ref{L: covariance} applied first to $\epsilon'$ and then to $\epsilon$, we have that 
\begin{align}\label{eq:incr}
	\mathbb \nu_{\beta,\lambda, \exact}[(h_x-h_{y})^2] &=  \beta_g\mu_{\beta,\lambda, \coclosed}[\cos(\J_g)] +\beta_g^2\mu_{\beta, \lambda,\coclosed}[\cos(\J_g)^2]-\beta_g^2 \nonumber \\
	&= \sum_{e \in E}\big( \mu_{\beta,\lambda, \coclosed}[\c{U}_e''(\J_e)] \epsilon^2_e -  \mu_{\beta,\lambda, \coclosed} \big[(\c{U}_e'(\J), \epsilon)^2\big]\big) .
\end{align}
Since $2\cos^2 t=1+\cos2t$ is positive definite we can apply Lemma \ref{L: Ginibre} to the first line above and conclude that \eqref{eq:incr} is nondecreasing in $\beta_e$ for any $e\neq g$.
By weak convergence, the same holds for 
 \[
\sum_{e \in E}\big( \mu_{\beta, \coclosed}[\c{U}_e''(\J_e)] \epsilon^2_e -  \mu_{\beta, \coclosed} \big[(\c{U}_e'(\J), \epsilon)^2\big]\big) = \mathbb \nu_{\beta, \exact}[(h_x-h_{y})^2],
 \]
where the last equality again follows from Lemma \ref{L: covariance}. This ends the proof. 
\end{proof}



% !TEX root = ../main_duality.tex

\subsection{Delocalisation of roughly planar height function models.}
In this section, we will use Theorem \ref{T: var_increasing_height} to deduce that on many planar graphs, the height function delocalises. 
Consider here an infinite lattice $\Gamma = (\mathscr{V}, \mathscr{E})$ embedded in the plane, 
but not necessarily planar. We will assume throughout that $\Gamma$ (under this embedding) invariant under a bi-periodic lattice action, 
and that it has finite degrees. 
An example of such $\Gamma$ is $\ZZ^2$ where all vertices are connected if they are within distance $M < \infty$ from each other. 
Given $\Gamma$, recall that $(G_N)_{N \geq 1}$ is an exhaustion of $\Gamma$ if $G_N$ is a finite subgraph of $\Gamma$ for each $N$, $G_N \subset G_{N + 1}$ and $G_N \nearrow \Gamma$. 
We will also consider the \emph{wiring} of $G_N$ by identifying $G_N^c$ in $\Gamma$ to a single vertex $\partial$ and removing all the self-loops created in this process. 
The obtained graph will be denoted by $G_N^*$. 
On such graphs, we will take the measures $\nu_{N, \beta, \exact}$ as in Section \ref{sec:setup2}, and we identify the space of $1$-forms $\n$ in $\Hcycle(\mathbb{Z})$ with functions $h$ in $\Omega^0_0(\ZZ)$. 

\begin{theorem}[Delocalisation] \label{T: deloc}
	Let $\Gamma$ be as above and consider the setup as in Section~\ref{sec:setup2} where we assume that $\c{U}_e$ is the same for all edges. 
	There exists a $\beta_c < \infty$ such that for all $\beta \geq \beta_c$ and all wired exhaustions $G_N^* \nearrow \Gamma$, 
	\[
		\nu_{N, \beta, \exact}[h_o^2] \to \infty. 
	\]
\end{theorem}

To prove this theorem, we rely on a beautiful result of Lammers \cite{Lammers}:
\begin{theorem}[Theorem 2.7 \cite{Lammers}] \label{T: Lammers}
	Let $\Gamma' = (V, E)$ be an infinite graph with degree at most three, that is invariant under some lattice action. 
	If $\c{V}$ is a convex potential for the height function with
	\[
		\c{V}(\pm 1) \leq \c{V}(0) + \log(2), 
	\]
	then the height function delocalises in the sense that there are no translation invariant Gibbs measures. 
\end{theorem}

In general, the potentials $\c{V}$ as in the setup of Section \ref{sec:setup2} need \emph{not}  be convex. 
However, in some special cases they are, as we will show next.  
This will be crucial for what follows: in Section \ref{sec: red to convex} it will be shown that we can always reduce to this case. 

\begin{lemma} \label{L: XY convex deloc}
	If $-\c{U}(\J) = \cos(i\J)$ for some $i \in \NN$, then $\c{V}_{\beta}$ is convex over $i\ZZ$ for all $\beta$. 
	Moreover, translation invariant Gibbs measures exist if and only if $\nu_{N, \beta, \exact}[h_o^2]$ is bounded uniformly in $N$. 
\end{lemma}
\begin{proof}
	In the case $-\c{U}(\J) = \cos(\J)$, convexity of $\c{V}_{\beta}$ over the integers is an easy consequence of the Tur\'an inequality, see e.g.~\cite{vEnLis}. 
	The extension to $-\c{U}(\J) = \cos(i\J)$ follows from a change of variables. 
	The second statement of the lemma was proved in the case of the XY model in \cite[Theorem 4]{vEnLis}. 
	It follows from a standard dichotomy (see e.g. \cite{LamOtt}) in the case where the height function satisfies the so called ``absolute value FKG'' property, 
	meaning that $|h|$ is FKG, see also \eqref{eq:vardeloc}. 
\end{proof}

\begin{remark}
	We wish to point out that the result of Lammers does \emph{not} depend on the potential $\c{V}$ being the same on each edge, just that it satisfies the condition of Theorem~\ref{T: Lammers} for all edges, and that the potentials are invariant under some lattice action. 
\end{remark}

We will first modify the potentials $-\c{U}$ so they will fit the framework of Theorem \ref{T: Lammers} and Lemma \ref{L: XY convex deloc}. 
Next, we modify the graph $\Gamma$ to obtain a graph $\Gamma'$ to which we can apply Theorem \ref{T: Lammers}
in such a way that $\Gamma'$ embeds into $\Gamma$ and the variance of the height function in $\Gamma'$ is smaller. 

\subsubsection{Reduction to convex potentials} \label{sec: red to convex}
We will apply here a simplification that allows us to only consider potentials of the form $-\c{U}(\J) = \alpha_i \cos(i\J)$. 
Since $-\c{U}$ is positive definite, it can be written as
\[
	-\c{U}(\J) = \alpha_0 + \sum_{i = 1}^\infty \alpha_i\cos(i \J),
\]
with $\alpha_i \geq 0$. Now let $i \geq 1$ be the first mode where $\alpha_i > 0$. Write $-\c{U}'= \alpha_i \cos(i\J)$ 
and $\nu'_{G, \beta, \exact}$ for the corresponding height function measure. 

\begin{lemma} \label{L: pure-pot reduction}
	For any finite graph $G = (V, E)$ with boundary $\partial$ and any $x \in V \setminus \{\partial\}$, we have
	\[	
		\nu'_{G, \beta, \exact}[h_x^2] \leq \nu_{G, \beta, \exact}[h_x^2].
	\]
\end{lemma}
\begin{proof}
	Take $-\c{U}'' = -\c{U} + \c{U}'$ which is positive definite. 
	Write for any $\alpha \geq 0$ 
	\[
		\c{U}_{\alpha}(\J) = \c{U}'(\J) + \alpha\c{U''}(\J), 
	\]
	so that $\c{U}_1 = \c{U}$, $\c{U}_0$ = $\c{U}'$ and $-\c{U}_{\alpha}$ is positive definite for each $\alpha$. 
	 Let $\nu_{G, \beta, \alpha, \exact}$ be  the corresponding height function measure. 
	 Theorem \ref{T: var_increasing_height} implies that for any $x \in V$
	 \[
	 	\frac{\partial}{\partial \alpha} \nu_{G, \beta, \alpha, \exact}[h_x^2] \geq 0,
	 \]
	 so that the variance is minimized at $\alpha = 0$. This shows the result. 
\end{proof}

\subsubsection{Graph Modifications.} \label{sec: graph mod}
Fix $\Gamma$ an infinite graph and $G_N \nearrow \Gamma$ an exhaustion as above.   
%We need to modify graphs in such a way that we keep track of the variance of the height function. 
We wish to perform two operations:
\begin{enumerate}[\hspace{1cm}(a)]
	\item splitting edges into multiple sub-edges and
	\item gluing vertices together, 
\end{enumerate}
in such a way that the variance of the height function does not increase. 

Operation (a) is the easiest: to add $k - 1$ ``evenly spaced'' vertices to an edge without changing the height function on the original graph, 
we wish to find a potential $\c{V}_{\beta}^{(k)}$ such that 
\[
	e^{-\c{V}_{\beta}} = (e^{-\c{V}_{\beta}^{(k)}})^{*k},
\]
where by $^{*k}$ we mean $k$-fold convolution. 

Using basic properties of the Fourier transform, we can take the potential $\c{V}_{\beta}^{(k)} = \c{V}_{\beta / k}$ which is dual to $-(\beta / k)\c{U}$.  
This offers the following lemma. 

\begin{lemma}[Splitting edges] \label{L: splitting edges}
	Suppose $\c{V}$ corresponds to a spin potential $\c{U}$, such that $-\c{U}$ is positive definite.  
	For each $k \in \NN$, we have
	\[
		e^{-\c{V}_{\beta}} = (e^{-\c{V}_{\beta / k} })^{*k}.
	\]
\end{lemma}

Operation (b) will make use of Theorem \ref{T: var_increasing_height}. 
Let $v_1, v_2$ be two vertices in the graph, 
with or without an edge between them and add to the graph the edge $g = \{ v_1, v_2\}$ with the XY potential $\c{V}_{\lambda}(k) = -\log(I_k(\lambda))$ 
with parameter $\lambda$. 
Write $\nu_{N, \beta, \lambda, \exact}$ for the corresponding height function measure on $G_N^*$. 
We will show now that gluing the vertices $v_1, v_2$ corresponds to sending $\lambda$ to $0$ in this setting. 
Indeed, as $\lambda \to 0$, we have
\[
	e^{-\c{V}_\lambda(k)} = I_k(\lambda) \to \begin{cases}
		1, & \text{if } k = 0\\
		0, &\text{else }
	\end{cases} 
\]
which means that the height function measure $\nu_{N, \beta, 0, \exact}$ is supported on height functions with $h_{v_1} = h_{v_2}$. 
Moreover, Theorem \ref{T: var_increasing_height} implies that for any vertex $x$ of $G_N^*$, 
\[
	\frac{\partial}{\partial \lambda} \nu_{N, \beta, \lambda, \exact}(h_x^2) \geq 0, 
\]
so that we find the following result. 

\begin{lemma}[Gluing vertices] \label{L: Gluing vertices}
	Let $x, v_1, v_2 \in V$ and $H_N^*$ be obtained from $G_N^*$ by gluing together $v_1$ and $v_2$. Then
	\(
		\nu_{H_N, \beta, \exact}[h_x^2] \leq \nu_{G_N, \beta, \exact}[h_x^2]. 
	\)
\end{lemma}

To summarise, we have established that gluing two vertices together reduces the variance of the height function, and splitting edges as in Lemma \ref{L: splitting edges} does not change the model. 
These two facts together imply that we can modify $\Gamma$ to obtain a planar graph $\Gamma'$ of degree at most three as we will explain now. 
We first show how to go from any planar graph to a planar graph of degree at most three. 

\subsubsection*{Star-tree transform}
There are many ways to transform a planar graph into a planar graph with degree at most three.
We follow here the elegant idea presented in \cite{GurNach}, where it was (implicitly) stated for the Gaussian free field. 
 Suppose that $G = (V, E)$ is a planar graph with boundary $\partial \in V$ and take the setup of Section \ref{sec:setup2}. Fix a vertex $v_0 \in V$.
It will be slightly more convenient to make a distinction between the number of neighboring vertices of $v_0$ and its degree in multigraphs.
\\
\\
\noindent \textbf{Degree reduction algorithm at $v_0$.} 
\begin{enumerate}[\hspace{0.4cm}1.]
	\item If the number of neighbours of $v_0$ is strictly less than $4$, do nothing.
	\item Label all neighbors of $v_0$ by $v_1, \ldots, v_{2d}$ by starting somewhere and going clockwise around $v_0$, where we \emph{do not} include the last vertex if the number of neighbours is odd. 
	\item Add to each edge $v_0v_i$ an intermediate vertex $x_i$ (note that if there are multiple edges between $v_0$ and $v_i$, then we have created many new vertices). 
	\item put the potential $\c{V}_{\beta_{v_0v_i} / 2}$ on the edges $v_0x_i$ and $x_iv_i$, for each $i$. 
	\item Glue together each pair $x_{2i - 1}$ and $x_{2i}$ (this includes gluing together multiple vertices $x_i$ if they exist). 
\end{enumerate}

Note that this algorithm reduces the \emph{number of neighbors of} $v_0$ by a factor $2$ if this number is even. 
Also note that it creates a multigraph. 
From the splitting and gluing lemmas, we obtain the next result. 
\begin{lemma}\label{L: star-tree transform}
	Applying the degree reduction algorithm at $v_0$ does not increase the variance of $h_x$ for any $x \in V$. 
\end{lemma}

Thus, to reduce the number of neighbors of $v_0$ to $3$ or less, we are left to apply the reduction algorithm inductively, 
and to get a graph of degree three we apply it to all vertices in $G$ other than the boundary vertex. 

To finalize the star-tree transform, we still need to transform the multi-graph into a simple graph. Of course, 
we need to do so without changing the height function model. 
If $e_1, e_2$ are two edges with the same end-points $x$ and $y$ then
\begin{align} \label{eq:adding potentials}
	\nu_{N, \beta, \exact}(h_x - h_y = k) \propto e^{-\c{V}_\beta(k)} e^{-\c{V}_\beta(k)} = e^{-(\c{V}_{\beta} + \c{V}_{\beta})(k)}. 
\end{align}
This observation implies that applying inductively the reduction algorithm and then applying the above observation does result in a graph where:
\begin{enumerate}[(i)]
	\item the number of neighbors of each vertex is less than or equal to $3$,
	\item the variance of the height function is not increased, 
	\item the potentials are of the form $D\c{V}_{\beta / k}$ for some $D$ and $k$ that can depend on the edges.  
\end{enumerate}

\begin{center}
	\begin{figure}
		\begin{subfigure}{.15\textwidth}
			\centering
			\includegraphics[scale =0.20]{Figures/star-tree_1.png}
		\end{subfigure} \hspace{1.0cm}
		\begin{subfigure}{.15\textwidth}
			\centering
			\includegraphics[scale =0.20]{Figures/star-tree_2.png} 
		\end{subfigure}  \hspace{1.0cm}
		\begin{subfigure}{.15\textwidth}
			\centering
			\includegraphics[scale =0.20]{Figures/star-tree_3.png} 
		\end{subfigure} \hspace{1.0cm}
		\begin{subfigure}{.15\textwidth}
			\centering
			\includegraphics[scale =0.20]{Figures/star-tree_4.png} 
		\end{subfigure} \hspace{1.0cm}
		\caption{Left: original graph with $v_0$ in the center. 
			Middle two: first step of the degree-reduction algorithm, dotted lines correspond to vertices to be glued together. 
			Right: Final graph after ``star--tree'' transform, with vertices glued together and all vertices have three or less neighbors.}
		\label{f:star-tree}
	\end{figure}
\end{center} 

\subsubsection{Proof of Theorem \ref{T: deloc}}
Before we finish the proof of Theorem \ref{T: deloc}, let us briefly mention how to go from a ``roughly planar'' graph to a planar graph, see also Figure \ref{f:long_range}. 
We will do so for $\ZZ^2$ where $x \sim y$ if $|x - y|_2 \leq 2$. 
Add to an edge connecting two vertices $x$ and $y$ that are at distance $2$ from eachother a new vertex. 
Glue it to the unique vertex between $x$ and $y$ that is at distance $1$ of each. 
Apply this algorithm to all edges. The obtained graph is planar and the variance of the height function is not increased by Lemma \ref{L: Gluing vertices}. 

\begin{proof}[Proof of Theorem \ref{T: var_increasing_height}]
	Consider the setup of Section \ref{sec:setup2}. 
	By Lemma \ref{L: pure-pot reduction}, we can assume without loss of generality that $-\c{U}(\J) = \cos(i\J)$.  
	Write $\c{V}_{\beta}$ for the corresponding height function potential. 
	
	Let $\Gamma'$ be the planar graph obtained from $\Gamma$ as in Section \ref{sec: graph mod}, with the corresponding potentials $D_e\c{V}_{\beta / k_e}$, 
	$D_e \in (0, \infty)$ and $k_e \in \NN$. 
	Although $D_e, k_e$ may be different on distinct edges, they are uniformly bounded because $\Gamma$ (and hence $\Gamma'$) is invariant a bi-periodic lattice action. 
	
	Let $(G_N)_{N \geq 1}$ be any exhaustion of $\Gamma$ and let $(G_N')_{N \geq 1}$ be the induced exhaustion of $\Gamma'$, obtained from applying the degree reduction algorithm to all of $G_N$ (but not the boundary vertex). 
	Write $\nu'_{N, \beta, \exact}$ for the corresponding height function measure on $G_N'$. 
	It follows from Section \ref{sec: graph mod} that it suffices to prove that for all $\beta$ large enough, $\nu'_{N, \beta, \exact}[h_o^2] \to \infty$. 
	Indeed, in this case we also have $\nu_{N, \beta, \exact}[h_o^2] \to \infty$. 
	
	Note that since $\c{V}_{\beta / k}$ is convex, so is any multiple. 
	Moreover, for each $D \in (0, \infty)$ and $k \in \NN$ we have that for all $\beta$ large enough, 
	\[
		D\c{V}_{\beta / k}(0) \leq D\c{V}_{\beta / k}(1) + \log(2). 
	\]
	Indeed, this follows from the fact that the modified Bessel function satisfies $I_{m}(\beta) / I_{m'}(\beta) \to 1$ as $\beta \to \infty$ (see e.g.~\cite{vEnLis}). 
	Therefore, we can apply Theorem \ref{T: Lammers} and Lemma \ref{L: XY convex deloc} to deduce that for $\beta$ large enough, 
	\[
		\nu'_{N, \beta, \exact}[h_o^2] \to \infty
	\]
	as $N \to \infty$. 
	This proves the theorem. 
\end{proof}

\begin{center}
	\begin{figure}
		\begin{subfigure}{.20\textwidth}
			\centering
			\includegraphics[scale =0.20]{Figures/long_range_1.png}
		\end{subfigure} \hspace{1.5cm}
		\begin{subfigure}{.20\textwidth}
			\centering
			\includegraphics[scale =0.20]{Figures/long_range_2.png} 
		\end{subfigure}  \hspace{1.5cm}
		\begin{subfigure}{.20\textwidth}
			\centering
			\includegraphics[scale =0.20]{Figures/long_range_3.png} 
		\end{subfigure} \hspace{1.0cm}
		\caption{Left: an example of $\ZZ^2$ with long-range interactions; only the edges of the (red) origin are drawn. Middle: gluing. The square (gray) vertices are added, together with the green edges where the gluing will happen. Right: the final (planar) graph.}
		\label{f:long_range}
	\end{figure}
\end{center} 
% !TEX root = ../main_duality.tex



\section{GFF covariance for a projection of the spin model.} \label{sec:projection}
Let $G = (V, E)$ be a finite graph. Recall that $\Omega^1(\RR)$ equipped with the $l^2$-inner product
\(
	(\epsilon, \omega)_{\Omega^1} 
\)
%:=  \frac{1}{2} \sum_{\vec{e} \in \vec{E}} \epsilon_{\vec{e}} \omega_{\vec{e}} = \sum_{e \in E} \epsilon_e \omega_e
is a Hilbert space. In this setting, the linear operators $\d$ and $\d^*$ are mutually adjoint, and hence
the spaces $\Hcycle(\RR)$ and $\Hstar(\RR)$ are orthogonal in $\Omega^1(\RR)$ and span the whole space, i.e.,
\[
\Omega^1(\RR) = \Hcycle(\RR) \oplus \Hstar (\RR). 
\]
We denote by $P_{\coclosed}$ and $P_{\exact}$ the orthogonal projection onto $\Hstar(\RR)$ and $\Hcycle(\RR)$ respectively. 

We focus on finite graphs $G = (V, E)$ with boundary vertex $\partial \in V$ and 
take mutually dual potentials $\c{V}$ and $\c{U}$ as in Definition \ref{def:potential}. 

Since $\c{U}$ is symmetric around $0$ by assumption, the derivative $\c{U}'$ of $\c{U}$ is odd and hence $\c{U}'(\J)$ is a $1$-form in $\Omega^1(\RR)$. 
It thus makes sense to look at the orthogonal decomposition of $\c{U}'(\J)$ in the space $\Hcycle (\mathbb R) \oplus \Hstar (\mathbb R)$. 
Define $\tau$ to be the unique element of $\Omega^0_0(\mathbb R)$ such that 
\[
	\d \tau = P_{\exact}\c{U}'(\J).
\]
We will next obtain the -- in our eyes somewhat remarkable -- result that $\tau$ has the covariance of a Gaussian free field 
irrespective of $\c{U}$. 

\begin{proposition}[GFF covariance] \label{P: projected-Gaussian}
	Let $\tau$ be as above and $ f,g \in \Omega^0_0(\RR)$. Then
	\[
			\inf_{e \in E} \mu_{G, \exact}[\c{U}''(\J_e)] (f, \Green g) \leq \mu_{G, \exact}[(\tau, f)(\tau, g)] \leq \sup_{e \in E} \mu_{G, \exact}[\c{U}''(\J_e)] (f, \Green g). 
	\]
\end{proposition}

We begin with an easy consequence of the duality lemma for covariance \ref{L: covariance}.

\begin{lemma}\label{L: spin Guassian dom}
	For any $f,g\in \Omega^0_0(\mathbb R)$, we have	
	\[
		\mu_{G, \exact}[(\c{U}'(\J), \d g)(\c{U}'(\J), \d f)] = \sum_{e \in E}\mu_{G, \exact}[\c{U}''(\J_e)]\d f_e \d g_e
	\]
\end{lemma}

\begin{proof}
	Let $f,g$ be as in the statement so that $\d f, \d g\in \Hcycle$. 
	Lemma \ref{L: covariance} implies
	\[
		\mu_{G, \exact}[(\c{U}'(\J), \d f)(\c{U}'(\J), \d g)] = \sum_{e \in E} \mu_{G, \exact}[\c{U}''(\J_e)]  \d f_e \d g_{e} - \nu_{G, \coclosed}[(\n, \d f)(\n, \d g)]
	\]
	Since $\Hcycle$ and $\Hstar$ are orthogonal, and the dual height $1$-form $\n$ takes value in $\Hstar$, the right-most term vanishes and the result follows. 
\end{proof}

\begin{proof}[Proof of Proposition \ref{P: projected-Gaussian}]
	For any $g \in \Omega^0(\mathbb R)$, we have 
	\[
		(\c{U}'(\J), \d g) = (P_{\exact}\c{U}'(\J), \d g) = (\tau, \Delta g)
	\] 
	where the first equality holds because $\d g \in \Hcycle$ and the second by definition of $\tau$, the duality of $\d$ and $\d^*$ and because $\Delta = \d^*\d$. 
	As before, $\Green$ denotes the inverse of $\Delta$ (so defined that functions take value $0$ at the boundary) and take now $g = \Green m$, $h = \Green f$. 
	It follows from this and the previous lemma that 
	\[
		\mu_{G, \exact}[(\tau, m)(\tau, f)] =  \sum_{e \in E}\mu_{G, \exact}[\c{U}''(\J_e)] \d f_e \d g_e, 
	\]
	implying the desired result. 
\end{proof}


\section{A central limit theorem} \label{sec:CLT}
Here we consider the same setup as in Section~\ref{sec:setup}, and we establish a central limit theorem for $\mathcal U'(J)$ summed over the path~$p_n$.
The main conclusion of this section is that even though the decay of the correlations of $\mathcal U'(J)$ changes if the height function delocalises, $\mathcal U'(J)$ always satisfies a central limit theorem as shown below. 

Let $(N_k)_{k\geq1}$ be a sequence along which $\mu_{\mathbb{T}_N, \coclosed}$ converges weakly to a measure $\mu=\mu_{\mathbb{T}_N}$. As usual, by duality, $\mu$ can be thought of as a Gibbs measure on $H_\exact(\mathbb Z^2, \mathbb S)$.
By Remark~\ref{rem:tightness}, for any fixed $n$, the difference $h_{v_0}-h_{v_n}$ converges weakly under $\nu_{\mathbb{T}_N, \exact}$, as $N\to \infty$ (as long as $v_0,v_n\in \mathbb{T}_N$). Moreover by Corollary~\ref{cor:upper},
\[
	\lim_{n\to \infty} \limsup_{k\to \infty} \nu_{\mathbb{T}_{N_k}, \exact} [(h_{v_0}-h_{v_n})^2]/{n} \leq \lim_{n\to \infty}c \frac{ \log n }{n}= 0
\] 
for some $c<\infty$, and hence by Lemma~\ref{lem:duality}, for all $t\in \mathbb R$,
\begin{align}\label{eq:phi1}
	1= \lim_{n\to \infty} \lim_{k\to \infty} \nu_{\mathbb{T}_{N_k}, \exact}\big [\exp \big(i\tfrac t {\sqrt n} ({h_{v_0}-h_{v_n}})\big)\big]= \lim_{n\to \infty}\lim_{k\to \infty}{Z_{\mathbb{T}_{N_k},\coclosed}\big (\tfrac t  {\sqrt n}p_n\big)}/{Z_{\mathbb{T}_{N_k},\coclosed}(0)},
\end{align}
where again we identify the path $p_n$ with the associated 1-form.
Using that 
\[
\mathcal U(J+\varepsilon)-\mathcal U( J)= \mathcal U'(J)\varepsilon +\tfrac 12 \mathcal U''(J)\varepsilon^2 +o(\varepsilon^2)
\]
we can write
\begin{align*}
{Z_{\mathbb{T}_{N_k},\coclosed}\big (\tfrac t  {\sqrt n}p_n\big)}/{Z_{\mathbb{T}_{N_k},\coclosed}(0)} 
&= \mu_{\mathbb{T}_{N_k}, \coclosed} \Big [\exp \Big(-t\frac{1}{\sqrt n}\sum_{i=1}^n  \mathcal U' (J_i)  - {t^2} \frac {1}{2n} \sum_{i=1}^n \mathcal U''(J_i)+o(t^2)\Big)\Big],
\end{align*}
where the error is uniform in $k$.
We first note that by weak convergence, as $k\to \infty$, the right-hand side approaches to the same expectation but with respect to $\mu$ (the infinite volume limit of $\mu_{\mathbb{T}_N, \exact}$).
Moreover, the error therm vanishes in the limit $n\to\infty$. Assuming that the spin measure $\mu$ is ergodic, we also have that
\[
\frac 1n \sum_{i=1}^n \mathcal U'' (J_i)\to \mu[\mathcal U'' (J_0)] \quad \mu\textnormal{-a.s. as } n\to \infty
\] 
by Birkhoff's pointwise ergodic theorem (since $J$ is invariant under the shift along the path, and $\mathcal U''$ is bounded). 
Note that in the case of the XY model, there is only one translation invariant Gibbs measure in two dimensions \cite{MMSP} which must therefore be ergodic. 
By the dominated convergence theorem and \eqref{eq:phi1}, we conclude the following central limit theorem.
\begin{theorem} If $\mu$ is ergodic, then for any $t\in \mathbb R$,
\begin{align*}
\lim_{n\to \infty} \mu \Big [\exp \Big(-\frac{t}{\sqrt n}\sum_{i=1}^n  \mathcal U' (J_i)\Big)\Big] = \exp\Big(\frac{t^2}2   \mu [\mathcal U'' (J_0)] \Big).
\end{align*}
In particular,
\[
\frac1{\sqrt n}\sum_{i=1}^n  \mathcal U' (J_i)\to \mathcal N(0, \mu [\mathcal U'' (J_0)])
\]
in distribution as $n\to \infty$.
\end{theorem}
\newpage
\appendix
\section{Appendix for Proofs}

\paragraph{Proof of Theorem \ref{thm:main}.}

\begin{proof}
\label{proof:main}
Our proof has two steps. In Step 1, we will show that SimCLR is equivalent to minimizing the cross entropy loss defined in Eqn.~(\ref{eqn:cross-entropy}). 
In Step 2, we will show  that minimizing the cross-entropy loss 
is equivalent to spectral clustering on $\bfpi$. 
Combining the two steps together, we have proved our theorem. 

\textbf{Step 1: } SimCLR is equivalent to minimizing the cross entropy loss.

The cross-entropy loss takes expectation over 
$\bfW_\bfX\sim \mathbb{P}(\cdot ; \bfpi)$, 
which means $\bfW_\bfX$ has exactly one non-zero entry in each row $i$. By Lemma~\ref{lem:multinomial}, we know every row $i$ of $\bfW_\bfX$ is independent of other rows. Moreover, 
$\bfW_{\bfX,i}\sim \mathcal{M}(1, \bfpi_i/\sum_j \bfpi_{i,j})=\mathcal{M}(1, \bfpi_i)$, because $\bfpi_i$ itself is a probability distribution.
Similarly, we know $\bfW_\bfZ$ also has the row-independent property by sampling over $\mathbb{P}(\cdot;\bfK_\bfZ)$.
Therefore, by Lemma~\ref{lem:cross_split}, we know Eqn.~(\ref{eqn:cross-entropy}) is equivalent to:
\[
 -\sum_{i=1}^n \mathbb{E}_{\bfW_{\bfX,i}}[\log \mathbb{P}(\bfW_{\bfZ,i}=\bfW_{\bfX,i};\bfK_\bfZ)],
\]

This expression takes expectation over $\bfW_{\bfX,i}$ for the given row $i$. Notice that 
$\bfW_{\bfX,i}$ has exactly one non-zero entry, which equals $1$ (same for $\bfW_{\bfZ,i}$). 
As a result
we expand the above expression to be:
\begin{equation}
 -\sum_{i=1}^n \sum_{j\neq i} \Pr(\bfW_{\bfX,i,j}=1)\log \Pr(\bfW_{\bfZ,i,j}=1).
\label{eqn:detailed-expansion}    
\end{equation}


By Lemma~\ref{lem:multinomial}, $\Pr(\bfW_{\bfZ,i,j}=1)=\bfK_{\bfZ,i,j}/\|\bfK_{\bfZ,i}\|_1$ for $j\neq i$. Recall that $\bfK_\bfZ=(k(\bfZ_i-\bfZ_j))_{(i,j)\in[n]^2}$, which means 
$\bfK_{\bfZ,i,j}/\|\bfK_{\bfZ,i}\|_1=\frac{\exp(-\|\bfZ_i-\bfZ_j\|^2/{2\tau})}{\sum_{k\neq i}
\exp(-\|\bfZ_i-\bfZ_k\|^2/{2\tau})
}$ for $j\neq i$, when $k$ is the Gaussian kernel with variance $\tau$. 

Notice that $\bfZ_i=f(\bfX_i)$, so we know
\begin{equation}
-\log \Pr(\bfW_{\bfZ,i,j}=1)=
-\log \frac{\exp(-\|f(\bfX_i)-f(\bfX_j)\|^2/{2\tau})}{\sum_{k\neq i}
\exp(-\|f(\bfX_i)-f(\bfX_k)\|^2/{2\tau}),
}
\label{eqn:infonce-equivalence}    
\end{equation}


The right hand side is exactly the InfoNCE loss defined in Eqn.~(\ref{eqn:infonce}).
Inserting Eqn.~(\ref{eqn:infonce-equivalence}) into Eqn.~(\ref{eqn:detailed-expansion}), we get the SimCLR algorithm, which first samples augmentation pairs $(i,j)$ with $\Pr(\bfW_{\bfX,i,j}=1)$ for each row $i$, and then optimize the InfoNCE loss. 

\textbf{Step 2: } minimizing the cross entropy loss 
is equivalent to spectral clustering on $\bfpi$.


By Lemma~\ref{lem:convert_to_spectral}, we may further convert the loss to 
\begin{equation}
\label{eqn:main-theorem-repul-attr}
\min_{\bfZ}
-\sum_{(i,j)\in [n]^2} \mathbf{P}_{i,j}
\log k (\bfZ_i-\bfZ_j)+\log \mathbf{R}(\bfZ).
\end{equation}
Since $k$ is the Gaussian kernel, this reduces to \[
\min_\bfZ \mathrm{tr}(\bfZ^\top \mathbf{L}(\bfpi) \bfZ)
+\log \mathbf{R}(\bfZ),
\]

where we use the fact that $\mathbb{E}_{\bfW_\bfX\sim \mathbb{P}(\cdot; \bfpi)}[\mathbf{L}(\bfW_\bfX)]
=\mathbf{L}(\bfpi)
$, because the Laplacian operator is linear and $
\mathbb{E}_{\bfW_\bfX\sim \mathbb{P}(\cdot; \bfpi)}(\bfW_\bfX)=\bfpi
$.
\end{proof}

\paragraph{Proof of Theorem \ref{thm:clip}.}
\begin{proof}
Since $\bfW_\bfX\sim \mathbb{P}(\cdot;\bfpi_{\mathbf{A}, \mathbf{B}})$, we know 
$\bfW_\bfX$ has exactly one non-zero entry in each row, denoting the pair that got sampled. 
A notable difference compared to the previous proof is we now have $n_\mathcal{A}+n_\mathcal{B}$ objects in our graph. CLIP deals with this by taking a mini-batch of size $2N$, 
such that $n_\mathcal{A}=n_\mathcal{B}=N$, and adding the $2N$ InfoNCE losses together. We label the objects in $\mathcal{A}$ as $[n_\mathcal{A}]$, and the objects in $\mathcal{B}$ as $\{n_\mathcal{A}+1, \cdots, n_\mathcal{A}+n_\mathcal{B}\}$. 

Notice that $\bfpi_{\mathbf{A}, \mathbf{B}}$ is a bipartite graph, so the edges of objects in $\mathcal{A}$ will only connect to object in $\mathcal{B}$ and vice versa. We can define the similarity matrix in $\cZ$ as $\bfK_\bfZ$, 
where $\bfK_\bfZ(i, j+n_\mathcal{A})=\bfK_\bfZ(j+n_\mathcal{A},i)= k(\bfZ_i-\bfZ_j)$ for $i\in [n_\mathcal{A}], j\in [n_\mathcal{B}]$, and otherwise we set $\bfK_\bfZ(i,j)=0$. 
The rest is same as the previous proof. 
\end{proof}

\paragraph{Proof of Theorem \ref{thm:exponential}.}

\begin{proof}
\label{proof:exponential}
Since the objective function consists of a linear term combined with an entropy regularization, which is a strongly concave function, the maximization problem is a convex optimization problem. Owing to the implicit constraints provided by the entropy function, the problem is equivalent to having only the equality constraint. We then introduce the Lagrangian multiplier $\lambda$ and obtain the following relaxed problem:

$$
\widetilde{E}(\boldsymbol{\alpha})=\psi_{1}-\sum_{i=1}^n \alpha_{i} \psi_{i}+\tau \sum_{i=1}^n \alpha_{i}\log \alpha_{i}+\lambda\left(\boldsymbol{\alpha}^{\top} \mathbf{1}_n-1\right).
$$

As the relaxed problem is unconstrained, taking the derivative with respect to $\alpha_{i}$ yields

$$
\frac{\partial \widetilde{E}(\boldsymbol{\alpha})}{\partial \alpha_{i}}=-\psi_{i}+\tau\left(\log \alpha_{i}+\alpha_{i} \frac{1}{\alpha_{i}}\right)+\lambda=0.
$$

Solving the above equation implies that $\alpha_{i}$ takes the form
$
\alpha_{i}=\exp \left(\frac{1}{\tau} \psi_{i}\right) \exp \left(\frac{-\lambda}{\tau}-1\right).
$ Since $\alpha_{i}$ lies on the probability simplex, the optimal $\alpha_{i}$ is explicitly given by
$
\alpha^{*}_{i}=\frac{\exp \left(\frac{1}{\tau} \psi_{i}\right)}{\sum_{i^{\prime}=1}^n \exp \left(\frac{1}{\tau} \psi_{i^{\prime}}\right)} .
$ Substituting the optimal point into the objective function, we obtain
$$
\begin{aligned}
E\left(\boldsymbol{\alpha}^*\right)  &=\psi_1-\sum_{i=1}^n \frac{\exp \left(\frac{1}{\tau} \psi_{i}\right)}{\sum_{i^{\prime}=1}^n \exp \left(\frac{1}{\tau} \psi_{i^{\prime}}\right)} \psi_{i}+\tau \sum_{i=1}^n \frac{\exp \left(\frac{1}{\tau} \psi_{i}\right)}{\sum_{i^{\prime}=1}^n \exp \left(\frac{1}{\tau} \psi_{i^{\prime}}\right)}\log \frac{\exp \left(\frac{1}{\tau} \psi_{i}\right)}{\sum_{i^{\prime}=1}^n \exp \left(\frac{1}{\tau} \psi_{i^{\prime}}\right)} \\
& =\psi_1 - \tau \log \left(\sum_{i=1}^n \exp \left(\frac{1}{\tau} \psi_{i}\right)\right).
\end{aligned}
$$
Thus, the Lagrangian dual function is given by
\begin{equation*}
-E\left(\boldsymbol{\alpha}^*\right)= -\tau \log \frac{\exp \left(\frac{1}{\tau} \psi_{1}\right)}{\sum_{i=1}^n \exp \left(\frac{1}{\tau} \psi_{i}\right)}.\qedhere
\end{equation*}
\end{proof}



\section{More on Experiments} \label{section: experiment_details}

\paragraph{CIFAR-10 and CIFAR-100} CIFAR-10 ~\citep{krizhevsky2009learning} and CIFAR-100 ~\citep{krizhevsky2009learning} are well-known classic image classification datasets. Both CIFAR-10 and CIFAR-100 contain a total of 60k $32 \times 32$ labeled images of different classes, with 50k for training and 10k for testing. CIFAR-10 is similar to CIFAR-100, except there are 10 different classes in CIFAR-10 and 100 classes in CIFAR-100.

\paragraph{TinyImageNet} TinyImageNet ~\citep{le2015tiny} is a subset of ImageNet ~\citep{deng2009imagenet}. There are 200 different object classes in TinyImageNet, with 500 training images, 50 validation images, and 50 test images for each class. All the images in TinyImageNet are colored and labeled with a size of $64 \times 64$.

\textbf{Pseudo-code.} Algorithm \ref{alg:Training Procedure} presents the pseudo-code for our empirical training procedure.

\begin{algorithm}[!htbp]
\caption{Training Procedure}
\label{alg:Training Procedure}
\begin{algorithmic}[1]
\REQUIRE trainable encoder network $f$, batch size $N$, augmentation strategy \textit{aug}, loss function $L$ with hyperparameters \textit{args}
\FOR {sampled minibatch ${x_i}_{i=1}^N$}
\FORALL{$i \in { 1, ..., N }$}
\STATE draw two augmentations $t_i = \textit{aug}\left(x_i\right) $, $t_i' = \textit{aug}\left(x_i\right) $
\STATE $z_i = f\left(t_i\right)$, $z_i' = f\left(t_i'\right)$
\ENDFOR
\STATE compute loss $\mathcal{L} = L(N, z, z', \textit{args})$
\STATE update encoder network $f$ to minimize $\mathcal{L}$
\ENDFOR
\STATE \textbf{Return} encoder network $f$
\end{algorithmic}
\end{algorithm}

We also provide the pseudo-code for our core loss function used in the training procedure in Algorithm \ref{alg:Core loss}. The pseudo-code is almost identical to SimCLR's loss function, with the exception of an extra parameter $\gamma$.

\begin{algorithm}[!htbp]
\caption{Core loss function $\mathcal{C}$}
\label{alg:Core loss}
\begin{algorithmic}[1]
\REQUIRE batch size $N$, two encoded minibatches $z_1, z_2$, $\gamma$, temperature $\tau$
\STATE $z = \textit{concat}\left(z_1, z_2\right)$
\FOR {$i \in {1, ..., 2N }, j \in {1, ..., 2N}$ }
\STATE $s_{i,j} = \Vert z_i - z_j \Vert_2^{\gamma}$
\ENDFOR
\STATE \textbf{define} $l(i, j)$ \textbf{as} $l(i, j) = - \log \frac{exp\left(s_{i,j}/\tau \right)}{\sum_{k=1}^{2N} \mathbf{1}{[k \ne i]} exp\left(s{i, j} / \tau \right)} $
\STATE \textbf{Return} $\frac{1}{2N} \sum_{k=1}^N\left[l(i, i+N) + l(i+N, i)\right]$
\end{algorithmic}
\end{algorithm}

Utilizing the core loss function $\mathcal{C}$, we can define all kernel loss functions used in our experiments in Table \ref{table: loss definition}. For all $z_i \in z$ with even dimensions $n$, we define $z_{L_i} = z_i\left[0:n/2\right]$ and $z_{R_i} = z_i\left[n/2:n\right]$.

\begin{table}[ht]
\centering
\begin{tabular}{{@{}l|l@{}}}
Kernel  &  Loss function \\ \midrule
Laplacian & $\mathcal{C}\left(N, z, z', \gamma=1, \tau\right)$\\ \midrule
Sum       & $\lambda * \mathcal{C}\left(N, z, z', \gamma=1, \tau_1\right) + (1-\lambda) * \mathcal{C}\left(N, z, z', \gamma=2, \tau_2\right)$  \\ \midrule
Concatenation Sum&$\lambda * \mathcal{C}\left(N, z_L, z'_L, \gamma=1, \tau_1\right) + (1-\lambda) * \mathcal{C}\left(N, z_R, z'_R, \gamma=2, \tau_2\right)$\\ \midrule
$\gamma = 0.5$ & $\mathcal{C}\left(N, z, z', \gamma=0.5, \tau\right)$          \\ 

\end{tabular}

\caption{Definition of kernel loss functions in our experiments}
\label {table: loss definition}
\end{table}

\textbf{Baselines.} We reproduce the SimCLR algorithm using PyTorch Lightning~\citep{PytorchLightning}.

\textbf{Encoder details.}
The encoder $f$ consists of a backbone network and a projection network. We employ ResNet50~\citep{ResNet} as the backbone and a 2-layer MLP (connected by a batch normalization~\citep{ioffe2015batch} layer and a ReLU \cite{nair2010rectified} layer) with hidden dimensions 2048 and output dimensions 128 (or 256 in the concatenation kernel case).

\textbf{Encoder hyperparameter tuning.}
For each encoder training case, we randomly sample 500 hyperparameter groups (sample details are shown in Table \ref{table: Hyperparameter sample}) and train these samples simultaneously using Ray Tune ~\citep{RayTune}, with the ASHA scheduler~\citep{li2018massively}. Ultimately, the hyperparameter group that maximizes the online validation accuracy (integrated in PyTorch Lightning) within 5000 validation steps is chosen for the given encoder training case.

\begin{table}[ht]
\centering

\begin{tabular}{@{}l|l|l@{}}
\midrule
Hyperparameter  & Sample Range & Sample Strategy \\ \midrule
start learning rate & $\left[10^{-2}, 10\right]$ & log uniform \\ \midrule
$\lambda$       & $\left[0, 1\right]$ & uniform \\ \midrule
$\tau$, $\tau_1$, $\tau_2$ & $\left[0, 1\right]$ & log uniform \\ \midrule
\end{tabular}

\caption{Hyperparameters sample strategy}
\label {table: Hyperparameter sample}
\end{table}

\textbf{Encoder training.} 
We train each encoder using the LARS optimizer~\citep{LARSOptimizer}, LambdaLR Scheduler in PyTorch, momentum 0.9, weight decay $10^{-6}$, batch size 256, and the aforementioned hyperparameters for 400 epochs on a single A-100 GPU.

\textbf{Image transformation.} The image transformation strategy, including augmentation, is identical to the default transformation strategy provided by PyTorch Lightning.

\textbf{Linear evaluation.}
The linear head is trained using the SGD optimizer with a cosine learning rate scheduler, batch size 64, and weight decay $10^{-6}$ for 100 epochs. The learning rate starts at $0.3$ and ends at $0$.

\textbf{Moco Experiments.} We also tested our method based on MoCo~\citep{he2019moco}. The results are summarized in Table \ref{tab:results-moco}. Here we choose ResNet18~\citep{ResNet} as the backbone and set a temperature of $0.1$ as default. For our simple sum kernel, we set $\lambda=0.8$. The results show that our method outperforms the original MoCo method.

\begin{table}[thb]
\centering
\caption{MoCo Experiment Results on CIFAR-10 and CIFAR-100.}
\label{tab:results-moco}
\resizebox{\textwidth}{!}{%
\begin{tabular}{@{}c|ccc|ccc@{}}
\toprule
\multirow{3}{*}{Method} & \multicolumn{3}{c|}{CIFAR-10} & \multicolumn{3}{c}{CIFAR-100} \\ \cmidrule(lr){2-4} \cmidrule(lr){5-7} 
                        & 200 epochs & 400 epochs    & 1000 epochs   & 200 epochs & 400 epochs & 1000 epochs         \\ \midrule
MoCo (repro.)         & $76.41 \pm 0.12$    & $80.01 \pm 0.15$          & $84.45 \pm 0.08$    & $\mathbf{47.02 \pm 0.11}$ & $52.50 \pm 0.07$ & $57.62 \pm 0.15$            \\
\midrule
Laplacian Kernel        & ${78.09 \pm 0.10}$    & $\mathbf{83.85 \pm 0.09}$          & $\mathbf{88.34 \pm 0.16}$    & $46.12 \pm 0.22$   & $53.44 \pm 0.17$ & $59.10 \pm 0.14$        \\
Simple Sum Kernel & $\mathbf{78.12 \pm 0.15}$   & $83.23 \pm 0.18$ & $87.50 \pm 0.20$ & $46.65 \pm 0.06$ & $\mathbf{53.62 \pm 0.19}$ & $\mathbf{59.83 \pm 0.12}$\\
\bottomrule
\end{tabular}
}
\end{table}



\section{More Experiments on Synthetic Data}


Consider a scenario with $n$ clusters, each containing $k$ vertices. Let the probability of vertices $u$ and $v$ from the same cluster belonging to $\bfpi$ be $p$. Conversely, for vertices $u$ and $v$ from different clusters, let the probability of belonging to $\pi$ be $q$. We generate the graph $\bfpi$ randomly, based on $p$ and $q$. We experiment with values of $k=100$ and $n=6$ for ease of visualization, embedding all points in a two-dimensional space. Each vertex's initial position originates from a normal distribution. In each iteration, we sample a subgraph of $\bfpi$ uniformly, ensuring each vertex has an out-degree of $1$. We then optimize the corresponding vectors using InfoNCE loss with an SGD optimizer and iterate until convergence. Our experimental setup consists of an SGD learning rate of $1$, an InfoNCE loss temperature of $0.5$, and a batch size of $50$. We evaluate two scenarios with different $p$ and $q$ values: $p=1$, $q=0$, and $p=0.75$, $q=0.2$. The results of these experiments are visualized in Figure \ref{fig:vis-spectral-cluster}. The obtained embeddings exhibit the hallmark pattern of spectral clustering of graph $\bfpi$.

\begin{figure}[!tb]
\centering
\subfigure{
\includegraphics[width=1\textwidth]{Figures/cluster_pi.png}
\label{fig:vis-cluster}
}
\subfigure{
\includegraphics[width=1\textwidth]{Figures/noised_cluster_pi.png}
\label{fig:vis-noised-cluster}
}
\caption{Visualizations of the optimization process using InfoNCE Loss on the vectors corresponding to $\bfpi$. Points of identical color belong to the same cluster within $\bfpi$. To showcase the internal structure of $\bfpi$, we randomly select 10 vertices from each cluster to display the edge distribution of $\bfpi$.}
\label{fig:vis-spectral-cluster}
\end{figure}





\bibliography{PosDef}

\end{document} 