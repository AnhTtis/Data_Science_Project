% !TEX root = ../main_duality.tex


\section{Monotonicity of variance of the height function in temperature} \label{sec:monot}
In this section we will show that the variance of the height function is monotone in a natural temperature parameter under some further assumptions on the potential.
To the best of our knowledge the result is new, even in the case of planar graphs.
Together with the dichotomy of Lammers \cite{Lammers22}, this directly implies that the height function of the XY model undergoes a sharp phase transition on the square lattice. 

\subsection{Setup} \label{sec:setup2} 
Let $G = (V, E)$ be a finite graph. 
To each edge $e \in E$, associate
\begin{itemize}%[\hspace{0.5cm}(1)]
	\item a twice continuously differentiable spin potential $\c{U}_e: \mathbb{S} \to \RR$ such that $-\mathcal U_e$ is positive definite, 
	\item a non-negative real $\beta_e$ (thought of as the inverse temperature in the spin model), 
	\item the dual potential $\c{V}_{\beta_e} := \c{V}_{\beta_e, e}$ of $\beta_e \c{U}_e$ as in Definition \ref{def:potential}. 
\end{itemize}
We will consider the family of measures $\nu_{\beta, \exact}$ 
for height functions and their dual measures $\mu_{\beta, \coclosed}$, indexed by $\beta = (\beta_e)_{e \in E}$. 

We wish to point out that the above requirements on the spin potential $\c{U}$ can also be described purely in terms of the height function potential:
if $\c{V}$ satisfies the conditions of Definition \ref{def:potential} and $e^{-\c{V}}$ is infinitely divisible (in the sense that each division satisfies Definition \ref{def:potential}), then the corresponding spin potential satisfies the above conditions, see Appendix \ref{ap:div}. 
This equivalence is not important in the remainder of this section. 

\begin{remark}
		The setup here is less general than in the rest of this text, 
		as we need to make a further assumption on the potentials $\c{U}$ (or equivalently on the potentials $\c{V}$ as explained in Appendix \ref{ap:div}). 
\end{remark}

\subsection{Increasing variance}
The main result of this section is the following fact. 
\begin{theorem} \label{T: var_increasing_height}
	Consider the setup as in Section \ref{sec:setup2}. For each $x, y \in V$ and $e \in E$ the function
	\[
		\beta_e \mapsto \nu_{\beta, \exact}[(h_x - h_y)^2]
	\]
	is non-decreasing. 
\end{theorem}

\begin{example} Let us first give some examples of potentials to which this theorem applies. 
	\begin{itemize}
		\item In case of the classical XY model, $-\c{U}(t) = \cos(2\pi t)$ which is positive definite. 
		\item A generalisation of this is any height function that is dual to a spin system with a positive definite potential.
		\item The Gaussian potential $k \mapsto \frac{1}{2\beta} k^2$ does itself not fall into this class. 
		However, it arises as a (rescaled) limit of the XY height function potentials so the conclusion of the above theorem still holds. 
		We will show the result on the height-function side, but it was already known on the dual spin side, see e.g. \cite{NewWu, AHPS}. 
		Let $G$ be a finite graph and let $e \in E$ be some fixed edge. 
		Replace $e$ by $N$ copies of $e$ and set on each copy of this edge the XY potential $\c{V}_{\beta N}(k) = -\log(I_{k}(\beta N))$. 
		Note that the effective height-function potential on the edge $e$ is equal to
		\[
			e^{-\c{V}^{\mathrm{eff}}_N(k)} = I_k(\beta N)^N. 
		\]
		On the other hand, the modified Bessel function $I_k(x)$ has the expansion as $N\to \infty$:
		\[
			I_k(\beta N) = \frac{e^{\beta N}}{ \sqrt{2\pi \beta N}} \left(1 - \frac{4k^2 - 1}{8\beta N} + O(1 / N^2)\right),
		\]
		which goes back to \cite{Kirch_bes}. In particular, as $N \to \infty$, 
		\[
			I_k(\beta N)^N \sim \left(\frac{e^{\beta N}}{ \sqrt{2\pi \beta N}}\right)^N e^{\frac{1}{8 \beta}} e^{- \frac{1}{2\beta}k^2} =: c_{\beta, N} e^{-\frac{1}{2\beta} k^2}, 
		\]
		where the constant $c_{\beta, N}$ does not depend on $k$. 
		Note that for Gibbs measures, the constant $c_{\beta, N}$ corresponds to a global shift of the potential $\c{V}^{\mathrm{eff}}_N$ which does not change the model. 
		Hence, since for each $N$ we can apply Theorem \ref{T: var_increasing_height}, the result holds in the limit as $N \to \infty$, where the potential on the edge $e$ equals $k \mapsto \frac{1}{2\beta}k^2$ as desired. 
	\end{itemize}
\end{example}

To prove the theorem, let us begin by slightly extending Ginibre's inequalities \cite{Gin} to spin models on $\Hstar(\mathbb{S})$ (the original inequality deals with $H_{\exact}(\mathbb{S})$).
\begin{lemma}[Ginibre] \label{L: Ginibre}
		Consider the setup as in Section \ref{sec:setup2}. 
		For all positive definite functions $F:\mathbb{S} \to \RR$ and all $e, f \in E$, we have
		\[
			\frac{\partial}{\partial \beta_e}\mu_{\beta, \coclosed}[F(\J_f)]\geq 0. 
		\]
\end{lemma}
\begin{proof}
	This is proved in Appendix \ref{ap:Gin}. 
\end{proof}





\begin{proof}[Proof of Theorem \ref{T: var_increasing_height}]
We first add an additional edge $g$ connecting $x$ and $y$ (even if there was already such an edge present).
On this edge, we put the potential $-\c{U}_{g}(t) = \cos(t)$ and parameter $\beta_{g} =\lambda \geq 0$. 
Thus, we remain in the setup of Section~\ref{sec:setup2}. We write $\mu_{\beta,\lambda,\coclosed}$ and $\nu_{\beta,\lambda,\exact}$ for the corresponding spin and height-function measure respectively, and note that $\mu_{\beta,\lambda,\coclosed}\to \mu_{\beta,\coclosed}$ as $\lambda\to \infty$. 

Let $\epsilon$ be any $1$-form vanishing on $g$ and such that $\d^*\epsilon=\delta_x-\delta_y$, and let $\epsilon'$ be the $1$-form vanishing outside of $g$ and such that $\d^*\epsilon'=\delta_x-\delta_y$.
By Lemma \ref{L: covariance} applied first to $\epsilon'$ and then to $\epsilon$, we have that 
\begin{align}\label{eq:incr}
	\mathbb \nu_{\beta,\lambda, \exact}[(h_x-h_{y})^2] &=  \beta_g\mu_{\beta,\lambda, \coclosed}[\cos(\J_g)] +\beta_g^2\mu_{\beta, \lambda,\coclosed}[\cos(\J_g)^2]-\beta_g^2 \nonumber \\
	&= \sum_{e \in E}\big( \mu_{\beta,\lambda, \coclosed}[\c{U}_e''(\J_e)] \epsilon^2_e -  \mu_{\beta,\lambda, \coclosed} \big[(\c{U}_e'(\J), \epsilon)^2\big]\big) .
\end{align}
Since $2\cos^2 t=1+\cos2t$ is positive definite we can apply Lemma \ref{L: Ginibre} to the first line above and conclude that \eqref{eq:incr} is nondecreasing in $\beta_e$ for any $e\neq g$.
By weak convergence, the same holds for 
 \[
\sum_{e \in E}\big( \mu_{\beta, \coclosed}[\c{U}_e''(\J_e)] \epsilon^2_e -  \mu_{\beta, \coclosed} \big[(\c{U}_e'(\J), \epsilon)^2\big]\big) = \mathbb \nu_{\beta, \exact}[(h_x-h_{y})^2],
 \]
where the last equality again follows from Lemma \ref{L: covariance}. This ends the proof. 
\end{proof}


