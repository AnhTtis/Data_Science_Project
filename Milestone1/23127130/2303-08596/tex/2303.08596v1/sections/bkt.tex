% !TEX root = ../main_duality.tex


\section{Delocalisation implies the BKT phase transition in two dimensions} \label{sec:bkt}
\subsection{Background}
In this section we consider the spin and height function models on the square lattice $\mathbb Z^2$ and we show that delocalisation of the height function (defined below) is equivalent to the divergence of a certain series of two-point functions in the dual spin model. One of the conclusions is that delocalisation implies the Berezinskii--Kosterlitz--Thouless (BKT) phase transition in the dual spin model~\cite{Ber1,KosTho}. 

This implication for the classical XY and the Villain spin models, together with a proof of delocalisation of the associated height functions, was first obtained by Fr\"{o}hlich and Spencer in their seminal work establishing the BKT transition~\cite{FroSpe} (also see~\cite{KP} for an exhaustive account).
Recently alternative proofs were provided by Aizenman et al.~\cite{AHPS} (first for the Villain and later also for the XY model) and by the authors~\cite{vEnLis} for the XY model. Together with the new conceptual approach to delocalisation introduced by Lammers~\cite{Lammers}, these works improve our mathematical understanding of the BKT transition.
These results can be thought of as an inequality between the critical points of the mutually dual spin and height function models.
A natural conjecture is that these critical points always coincide. In the case of the XY and Villain model this was confirmed in a recent work Lammers~\cite{Lam23}.


In this section we provide yet another, and arguably the simplest so far, proof of the fact that delocalisation of the height function implies that correlations functions of certain observables in the spin model do not decay exponentially fast in the distance. 
For reflection positive models (which is the case when $-\mathcal U$ is itself positive definite, i.e., has nonnegative coefficients in the Fourier series), we moreover obtain an equivalence between delocalisation and nonsummability of spin correlations.
Our approach, unlike the previous ones, is based solely on duality, and does not invoke any additional (e.g.~graphical) representations of the models at hand. 



\subsection{Notions of delocalisation}
It is now well established that integer-valued height functions on $\mathbb Z^2$ (or in general on periodic planar lattices) undergo a phase transition between a
\emph{localised} \emph{(smooth)} and a \emph{delocalised} \emph{(rough)} regime~\cite{FroSpe,CPST,DHLR,Lis21,DKMO,Lammers,LamOtt,Lammers22,Lam23}. 
%One can interpret this transition as the breaking and recovery of the translation symmetry of $\mathbb Z$.
We say that a potential $\mathcal V$ is localised (on $\mathbb Z^2$) if it admits a translation-invariant Gibbs measure on height functions $h: \mathbb Z^2\to \mathbb Z$. Otherwise we say that $\mathcal V$ delocalises.
It is known that if $\mathcal V$ is {convex} on the integers, and moreover its second discrete derivative is nonincreasing, i.e., $\mathcal V$ is a so-called \emph{supergaussian} potential~\cite{LamOtt,Lammers22}, then delocalisation in this sense is equivalent to the fact that 
\begin{align} \label{eq:vardeloc}
\sup_{N \geq 1} \nu_{\Lambda_N,\exact}[h_{\mathbf 0}^2] =\infty,
\end{align}
where $\mathbf 0$ is the origin of $\mathbb Z^2$, and $\Lambda_N=[-N,N]^2\cap \mathbb Z^2 $ where we identify all vertices in $\Lambda_N$ that are adjacent to $\Lambda^c_N:=\mathbb Z^2 \setminus \Lambda_N$ as one boundary vertex $\partial$ (wired boundary) and set $h_\partial =0$. Moreover for such potentials, the sequence in~\eqref{eq:vardeloc} is nondecreasing in~$N$~\cite{LamOtt}, and it grows up to a mulitplicative constant at least like $\log N$~\cite{Lammers22} (which is consistent with the general conjecture stating that delocalised height functions should behave like the GFF at large scales). 

Yet another approach to delocalisation is to work with infinite volume \emph{gradient} measures and study the variance of the increment of the height between two distant points. This was e.g.\ studied in~\cite{Lis19,Lis21} in the context of the six-vertex model and it will be convenient for us to follow the same route here, as we already know by Remark~\ref{rem:tightness} that translation invariant gradient Gibbs measures always exist for potentials as in Definition~\ref{def:potential}. 
We say that a potential $\mathcal V$ is $\nabla$-\emph{delocalised} (on $\mathbb Z^2$) if for any translation-invariant gradient Gibbs measure $\nu$ (with expectation $\nu$), we have 
\begin{align} \label{eq:graddeloc}
\sup_{v\in \mathbb Z^2} \nu[(h_v-h_{\mathbf 0})^2]= \infty.
\end{align}

\begin{lemma} \label{lem:nabla}
If a potential $\mathcal V$ is delocalised, then it is also $\nabla$-delocalised.
\end{lemma}
\begin{proof}
Suppose otherwise that there exists a translation-invariant gradient Gibbs measure~$\nu$ for which the supremum in~\eqref{eq:graddeloc} is finite.
Then, as in Theorem~\ref{thm:extransinv}, by considering convex combinations of translations of $\nu$ thought of as a measure on height functions $\tilde h$ given by $\tilde h_v=h_v-h_{\mathbf 0}$ we can construct a translation invariant Gibbs measure on height functions which is a contradiction. We leave the details to the reader.
\end{proof}

We note that the opposite implication is also true e.g.\ for potentials $\mathcal V$ that are convex on the integers. Indeed, in this case it is known from the foundational work of Sheffield~\cite{sheffield} that each Gibbs measure for height functions has a finite second moment (since the height at every point has a log-concave distribution). 


\subsection{Setup}\label{sec:setup}
Let us fix mutually dual potentials $\mathcal V$ and $\mathcal U$ as in Definition~\ref{def:potential}.
It will be convenient to consider the spin and height function models on finite, exponentially growing tori $\mathbb T_N= (\mathbb Z / 2^N\mathbb Z)^2$. This way we achieve three properties by construction:
\begin{itemize}
\item we work with measures that are translation invariant and invariant under $\pi/2$-rotations,
\item we can apply the duality of Lemma~\ref{lem:duality} first in the finite volume $\mathbb T_N$, and then take simultaneous (subsequential) infinite-volume limits, $\mathbb T_N\to \mathbb Z^2$ as $N\to \infty$, on both sides of the duality relation,
\item we get an explicit monotonicity in $N$ for the Green's function of the random walk on $\mathbb T_N$ (see below).
\end{itemize}

Let $\mu=\mu_{\mathbb Z^2,\coclosed}$ be any subsequential limit of $\mu_{\mathbb T_{N}, \coclosed}$, and let $\nu =\nu_{\mathbb Z^2,\exact}$ denote the limit of $\nu_{\mathbb T_{N}, \exact}$ taken along the same subsequence (it exists by Remark~\ref{rem:tightness}).
One can think of $\nu$ as a probability measure on height functions $h:\mathbb Z^2\to \mathbb Z$ satisfying $h(\mathbf 0)=0$. By weak convergence, the duality of Lemma~\ref{lem:duality} holds also for $\mu$ and $\nu$ whenever $\epsilon \in \Omega^1(\mathbb Z^2,\RR)$ is of bounded support.
The same is true for Corollary~\ref{L: covariance} and Corollary~\ref{cor:upper}, where we choose $\partial = \mathbf 0$ and consider the Green's function of a random walk on $\mathbb Z^2$ killed at $\mathbf 0$.

Note that by planar duality, we have $H_\coclosed(\mathbb Z^2, \mathbb S) \cong H_\exact((\mathbb Z^2)^*, \mathbb S) $,
where $(\mathbb Z^2)^* \cong \mathbf 0^*+\mathbb Z^2$ with $\mathbf 0^*:=(1/2,1/2)$, is the \emph{dual} square lattice. 
Since $ H_\exact((\mathbb Z^2)^*, \mathbb S) \cong {\mathbb S}^{(\mathbb Z^2)^*\setminus \{ \mathbf 0^* \}}$, we can think of $\mu$ as a Gibbs measure on spin configurations $\theta$ on $(\mathbb Z^2)^*$ where the spin at $\mathbf 0^*$ is fixed to be the identity element of $\mathbb S$. 
%To make the measure translation invariant, we will rotate all spins by an independent uniform element of $\mathbb S $. We will write $\theta$ for the resulting spin configuration, and with a slight abuse of notation, we will write $\mu$ for the measure that also incorporates the randomness of the spin at $\mathbf 0^*$.

Finally, let $v_n=(n,0)\in \mathbb Z^2$, and
let $p_n=(e_0,e_1,\ldots,e_{n-1})$ be the directed horizontal path from $v_0$ to $v_{n}$. We identify $p_n$ with the 1-form that assigns $1$ to each directed edge in $p_n$, 
and $0$ to the directed edges of $\mathbb Z^2$ that are not in $p_n$. For compactness of notation, we write $\J_i=\J_{e_i}$ and $h_i=h_{u_i}$.

\subsection{The implication}

Applying Lemma~\ref{L: covariance} and Corollary~\ref{cor:upper} in finite volume, and then taking the subsequential limit as in Section~\ref{sec:setup}, we have 
\begin{align} \label{eq:vanish}
	0\leq \sum_{i=0}^{n-1}   \mu[\mathcal U''(\J_i)] - \mu \Big[\Big(\sum_{i=0}^{n-1}\mathcal U'(\J_i)\Big)^2\Big]\leq \limsup_{N\to \infty}  \nu_{\mathbb T_{N}, \exact}[(h_{0} -h_{n})^2]\leq \limsup_{N\to \infty} \mathbf G_{N}(v_n,v_n),
\end{align}
where $\mathbf G_{N}$ is the Green's function of simple random walk on $\mathbb T_N$ killed at $\mathbf 0$. This is equivalent to a random walk on $\mathbb Z^2$ killed at all points in $2^N\mathbb Z^2$. Hence, $\mathbf G_{N}\nearrow \mathbf G$ as $N\to \infty$, where $\mathbf G$ is the Green's function of a random walk on $\mathbb Z^2$ killed at $\mathbf 0$.
Classically we have $\mathbf G(v_n,v_n)\leq \textnormal{const}\times\log n$ (see e.g.~\cite{LyPer}). Plugging this bound into \eqref{eq:vanish}, dividing both sides by $n$,
letting $n\to \infty$, and finally using translation invariance of $\mu$, we get
\begin{align}\label{eq:cesaro}
	\lim_{n\to \infty}\frac{1}{n}\sum_{k=1}^{n-1} u_k=\frac12 \mu[\mathcal U''(\J_0)-\mathcal U'(\J_0)^2] , \quad  \textnormal{where} \quad u_k=\sum_{i=1}^k \mu [\mathcal U'(\J_0)\mathcal U'(\J_i)].
\end{align}
In particular $u_k$ converges in the Ces\`aro sense as $k\to \infty$. %Note that $\mu [\mathcal U'(\J_0)\mathcal U'(\J_i)]$ is not-necessarily of definite sign.

\begin{theorem}[Delocalisation implies the BKT phase transition] \label{thm:BKT}
Consider the setup from Section~\ref{sec:setup}.
	If the height function delocalises in the sense that~\eqref{eq:graddeloc} holds true for~$\nu$, then
	\begin{align*}
		\sum_{i=1}^\infty  i |\mu [\mathcal U'(\J_0)\mathcal U'(\J_i)]|= \infty. %\sum_{i=1}^\infty \sum_{k={i}}^\infty  |\mu [\mathcal U'(\J_0)\mathcal U'(t_k)]|=\infty.
	\end{align*}
	In particular, there is no exponential decay of the two-point function $\mu [\mathcal U'(\J_0)\mathcal U'(\J_i)]$ as $i\to \infty$.%, and a Berezinskii--Kosterlitz--Thouless phase transition occurs in the spin model.
\end{theorem}
\begin{proof}
	We can assume that $\sum_{i=1}^\infty |\mu [\mathcal U'(\J_0)\mathcal U'(\J_i)]|<\infty$ since otherwise we are done. This means that $u_k$ converges in the classical sense to its Ces\`aro limit
	from \eqref{eq:cesaro}. Hence, 
	\begin{align} \label{eq:exid}
		\mu[\mathcal U''(\J_0)-\mathcal U'(\J_0)^2] = \lim_{k\to \infty}2 u_k = 2\sum_{i=1}^\infty  \mu [\mathcal U'(\J_0)\mathcal U'(\J_i)].
	\end{align}
	By Lemma~\ref{L: covariance} applied in the infinite volume ($p_n$ has bounded support) and translation invariance of $\mu$, we have
	\begin{align*}
		\nu[(h_{n}-h_{ 0})^2] & =\sum_{i=0}^{n-1}\mu[\mathcal U''(\J_i)-\mathcal U'(\J_i)^2] -2\sum_{i=1}^{n-1} (n-i) \mu [\mathcal U'(\J_0)\mathcal U'(\J_i)] \\
		&= 2n \sum_{i=1}^\infty\mu [\mathcal U'(\J_0)\mathcal U'(\J_i)] -2\sum_{i=1}^{n-1} (n-i) \mu [\mathcal U'(\J_0)\mathcal U'(\J_i)]\\
		&= 2\sum_{i=1}^{n-1}i\mu [\mathcal U'(\J_0)\mathcal U'(\J_i)] +2n\sum_{i=n}^{\infty}  \mu [\mathcal U'(\J_0)\mathcal U'(\J_i)] \\
		&\leq 2 \sum_{i=1}^\infty  i |\mu [\mathcal U'(\J_0)\mathcal U'(\J_i)]|.
	\end{align*}
	By the assumption, and translation and $\pi/2$-rotation invariance of $\mathcal \nu$, we have
	\[
\infty =\sup_{v\in \mathbb Z^2} \nu[(h_{v}-h_{\mathbf 0})^2]\leq 2 \sup_{n\geq 1} \nu[(h_{n}-h_{\mathbf 0})^2],
	\]
which together with the inequality above finishes the proof.
\end{proof}

It is classical that spin correlation functions decay exponentially fast at high temperatures (here the temperature is incorporated in the definition of $\mathcal U$).
This in particular implies that $\sum_{i=1}^\infty i |\mu [\mathcal U'(\J_0)\mathcal U'(\J_i)]|<\infty$. 
From this point of view Theorem~\ref{thm:BKT} says that if the height function delocalises, then the associated
spin model undergoes a BKT phase transition from a regime with exponential decay to a regime with slow decay of correlations.





\subsection{The case of the $XY$ model}
The change of behaviour of the two-point functions $\mu[\mathcal U'(J_0) \mathcal U'(J_i)]$ as $i\to \infty$ clearly indicates a phase transition in the spin model. However it is more common to look at correlations of the type $\mu[\mathcal F(\theta_u-\theta_{u'})]$ when $u$ and $u'$ are far apart, where~$\theta$ is the underlying spin field on $(\mathbb Z^2)^*$ (the faces of $\mathbb Z^2$), and where $\mathcal F$ is some chosen function, e.g. $\mathcal F=\mathcal U$. 

For general spin models, it is not clear how to compare these two types of correlation functions. Here we present an approach based on correlation inequalities in the case of the classical XY model, i.e., when $\mathcal U(t)=- \beta\cos(t)$, where $\beta>0$ is the inverse temperature in the spin model.

To this end, consider the setup as in Theorem~\ref{thm:BKT}. If $\{u_i,u_i'\}$ is the dual edge of $e_i$, writing $\theta_i=\theta_{u_i}$ and $\theta_i'=\theta_{u_i}'$, we have
\begin{align} \label{eq:trigon}
	\frac2{\beta^2} \mu[\mathcal U'(e_0) \mathcal U'(e_i)] &= 2\mu[\sin(\theta_0-\theta_0') \sin(\theta_i-\theta_i')] 
	\\ &=\mu[\cos(\theta_0-\theta_0'-\theta_i+\theta_i')]-\mu[\cos(\theta_0-\theta_0'+\theta_i-\theta_i')] \nonumber
	%\\ &\leq\mu[\cos((\theta_0-\theta_i)+(\theta_0'-\theta_i'))],
\end{align}
A version of the classical Ginibre inequality for the XY model~\cite{Gin} (see also~\cite{BLU}) says that
\begin{align*}
	\mu[ \sin ( \theta _0)\cos(\theta'_0) \sin(\theta_i) \cos(\theta'_i) ] \leq \mu[ \sin ( \theta _0) \sin(\theta_i)] \mu[ \cos(\theta'_0)\cos(\theta'_i) ],
\end{align*}
which after expanding into cosines of sums of angles and disregarding terms that are not invariant under global rotation (shift of angles mod $2\pi$) whose expectations vanish,
we obtain
\begin{align*}
	\mu[\cos(\theta_0+\theta_0'-\theta_i-\theta'_i)]& + \mu[\cos(\theta_0-\theta'_0-\theta_i+\theta'_i)]  -\mu[\cos(\theta_0-\theta'_0+\theta_i-\theta'_i)]  \\
	&\leq 2 \mu[ \cos ( \theta _0-\theta_i)] \mu[ \cos(\theta'_0-\theta'_i) ]  \\
	&= 2 \mu[ \cos ( \theta _0-\theta_i)]^2,
\end{align*}
where the last identity follows by reflection invariance of $\mu$ across the real line.
Analogous inequality follows by exchanging the roles of $\theta_i$ and $\theta'_i$, which results in swapping the signs of the second and third term in the first line.
Using that the first term is positive by the first Griffiths inequality, and combining with \eqref{eq:trigon}, we get that
\begin{align} \label{eq: inequality spin-spin}
	\tfrac12\mu[\cos(\theta_0+\theta_0'-\theta_i-\theta'_i)]+|\mu[\mathcal U'(e_0) \mathcal U'(e_i)] | & \leq \max\{ \mu[ \cos ( \theta _0-\theta_i)]^2,\mu[ \cos ( \theta _0-\theta'_i)]^2\} \nonumber
	\\ &= \mu[ \cos ( \theta _0-\theta_i)]^2,
\end{align}
where the last identity follows from the Messager--Miracle-Sole inequality~\cite{MMS} by applying the reflection across the real line.

These considerations, together with Lemma~\ref{lem:nabla}, lead us to the following corollary that recovers the main result of~\cite{vEnLis}. 

\begin{corollary}\label{C: deloc BKT}
	If the height function associated with the XY model on $\mathbb Z^2$ delocalises, then  
	\[
	\sum_{i=1}^\infty  i  \mu[ \cos ( \theta _0-\theta_i)]^2 \geq \sum_{i=1}^\infty  i  (\mu[ \cos ( \theta _0-\theta_i)]^2-\tfrac12\mu[\cos(\theta_0+\theta_0'-\theta_i-\theta'_i)]) = \infty.
	\] 
	%In particular, correlations do not decay exponentially and a BKT transition occurs. 
\end{corollary}

\begin{remark}
For the XY model it is known that there exists only one translation-invariant Gibbs measure $\mu$ on $\mathbb Z^2$~\cite{Pfister}, and hence regular, instead of subsequential, limits may be taken Section~\ref{sec:setup}.
\end{remark}

\subsection{An equivalence}
When $-\mathcal U$ is itself positive definite, i.e., all its Fourier coefficients are nonnegative, we can actually conclude more than in the above discussion. Indeed, in this case $\mu$ is reflection positive (see Appendix \ref{ap:RP}). This implies that 
\[
\mu [\mathcal U'(\J_0)\mathcal U'(\J_i)] \geq 0
\] 
as this holds true on $\mathbb T_N$ for every $N\geq i$ by reflection positivity. Therefore the Ces\`aro convergence from \eqref{eq:cesaro} implies classical convergence, and \eqref{eq:exid} always holds true.
The same argument as in the proof of Theorem~\ref{thm:BKT} yields the following corollary.
\begin{corollary} \label{cor:equiv}
Consider the setup from Section~\ref{sec:setup}, and moreover assume that $-\mathcal U$ is positive definite. 
Then~\eqref{eq:graddeloc} holds true for~$\nu$ if and only if
	\begin{align} \label{eq:infinite}
		\sum_{i=1}^\infty  i \mu [\mathcal U'(\J_0)\mathcal U'(\J_i)]= \infty. 
	\end{align}
\end{corollary}



\begin{remark}
The identity from \eqref{eq:exid} can be rewritten in a more symmetric form as
\begin{align} \label{eq:finite}
\sum_{i\in \mathbb Z}  \mu [\mathcal U'(\J_0)\mathcal U'(\J_i)]=\mu[\mathcal U''(\J_0)],
\end{align}
where now the sum is over a bi-infinite path of edges. Curiously, this is an exact (but nonlocal) identity for correlation functions that is independent of the (hidden in the definition of $\mathcal U$) 
temperature parameter. In particular the series in \eqref{eq:finite} is always convergent, independently of the temperature. This is in contrast with the behaviour of the series in~\eqref{eq:infinite}
that does undergo a phase transition. This, together with the relation to the gradient of the heigh function~\eqref{eq:pontwiseD2}, is consistent with the conjecture that in delocalised 
regime the discrete GFF describes the large-scale fluctuations of the height function. Indeed, the two-point function of the gradient of the discrete GFF is known to decay like the inverse square of the distance (see e.g.~\cite{BauWeb}).
\end{remark}
