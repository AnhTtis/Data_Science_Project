% !TEX root = ../main_duality.tex



\section{GFF covariance for a projection of the spin model.} \label{sec:projection}
Let $G = (V, E)$ be a finite graph. Recall that $\Omega^1(\RR)$ equipped with the $l^2$-inner product
\(
	(\epsilon, \omega)_{\Omega^1} 
\)
%:=  \frac{1}{2} \sum_{\vec{e} \in \vec{E}} \epsilon_{\vec{e}} \omega_{\vec{e}} = \sum_{e \in E} \epsilon_e \omega_e
is a Hilbert space. In this setting, the linear operators $\d$ and $\d^*$ are mutually adjoint, and hence
the spaces $\Hcycle(\RR)$ and $\Hstar(\RR)$ are orthogonal in $\Omega^1(\RR)$ and span the whole space, i.e.,
\[
\Omega^1(\RR) = \Hcycle(\RR) \oplus \Hstar (\RR). 
\]
We denote by $P_{\coclosed}$ and $P_{\exact}$ the orthogonal projection onto $\Hstar(\RR)$ and $\Hcycle(\RR)$ respectively. 

We focus on finite graphs $G = (V, E)$ with boundary vertex $\partial \in V$ and 
take mutually dual potentials $\c{V}$ and $\c{U}$ as in Definition \ref{def:potential}. 

Since $\c{U}$ is symmetric around $0$ by assumption, the derivative $\c{U}'$ of $\c{U}$ is odd and hence $\c{U}'(\J)$ is a $1$-form in $\Omega^1(\RR)$. 
It thus makes sense to look at the orthogonal decomposition of $\c{U}'(\J)$ in the space $\Hcycle (\mathbb R) \oplus \Hstar (\mathbb R)$. 
Define $\tau$ to be the unique element of $\Omega^0_0(\mathbb R)$ such that 
\[
	\d \tau = P_{\exact}\c{U}'(\J).
\]
We will next obtain the -- in our eyes somewhat remarkable -- result that $\tau$ has the covariance of a Gaussian free field 
irrespective of $\c{U}$. 

\begin{proposition}[GFF covariance] \label{P: projected-Gaussian}
	Let $\tau$ be as above and $ f,g \in \Omega^0_0(\RR)$. Then
	\[
			\inf_{e \in E} \mu_{G, \exact}[\c{U}''(\J_e)] (f, \Green g) \leq \mu_{G, \exact}[(\tau, f)(\tau, g)] \leq \sup_{e \in E} \mu_{G, \exact}[\c{U}''(\J_e)] (f, \Green g). 
	\]
\end{proposition}

We begin with an easy consequence of the duality lemma for covariance \ref{L: covariance}.

\begin{lemma}\label{L: spin Guassian dom}
	For any $f,g\in \Omega^0_0(\mathbb R)$, we have	
	\[
		\mu_{G, \exact}[(\c{U}'(\J), \d g)(\c{U}'(\J), \d f)] = \sum_{e \in E}\mu_{G, \exact}[\c{U}''(\J_e)]\d f_e \d g_e
	\]
\end{lemma}

\begin{proof}
	Let $f,g$ be as in the statement so that $\d f, \d g\in \Hcycle$. 
	Lemma \ref{L: covariance} implies
	\[
		\mu_{G, \exact}[(\c{U}'(\J), \d f)(\c{U}'(\J), \d g)] = \sum_{e \in E} \mu_{G, \exact}[\c{U}''(\J_e)]  \d f_e \d g_{e} - \nu_{G, \coclosed}[(\n, \d f)(\n, \d g)]
	\]
	Since $\Hcycle$ and $\Hstar$ are orthogonal, and the dual height $1$-form $\n$ takes value in $\Hstar$, the right-most term vanishes and the result follows. 
\end{proof}

\begin{proof}[Proof of Proposition \ref{P: projected-Gaussian}]
	For any $g \in \Omega^0(\mathbb R)$, we have 
	\[
		(\c{U}'(\J), \d g) = (P_{\exact}\c{U}'(\J), \d g) = (\tau, \Delta g)
	\] 
	where the first equality holds because $\d g \in \Hcycle$ and the second by definition of $\tau$, the duality of $\d$ and $\d^*$ and because $\Delta = \d^*\d$. 
	As before, $\Green$ denotes the inverse of $\Delta$ (so defined that functions take value $0$ at the boundary) and take now $g = \Green m$, $h = \Green f$. 
	It follows from this and the previous lemma that 
	\[
		\mu_{G, \exact}[(\tau, m)(\tau, f)] =  \sum_{e \in E}\mu_{G, \exact}[\c{U}''(\J_e)] \d f_e \d g_e, 
	\]
	implying the desired result. 
\end{proof}


\section{A central limit theorem} \label{sec:CLT}
Here we consider the same setup as in Section~\ref{sec:setup}, and we establish a central limit theorem for $\mathcal U'(J)$ summed over the path~$p_n$.
The main conclusion of this section is that even though the decay of the correlations of $\mathcal U'(J)$ changes if the height function delocalises, $\mathcal U'(J)$ always satisfies a central limit theorem as shown below. 

Let $(N_k)_{k\geq1}$ be a sequence along which $\mu_{\mathbb{T}_N, \coclosed}$ converges weakly to a measure $\mu=\mu_{\mathbb{T}_N}$. As usual, by duality, $\mu$ can be thought of as a Gibbs measure on $H_\exact(\mathbb Z^2, \mathbb S)$.
By Remark~\ref{rem:tightness}, for any fixed $n$, the difference $h_{v_0}-h_{v_n}$ converges weakly under $\nu_{\mathbb{T}_N, \exact}$, as $N\to \infty$ (as long as $v_0,v_n\in \mathbb{T}_N$). Moreover by Corollary~\ref{cor:upper},
\[
	\lim_{n\to \infty} \limsup_{k\to \infty} \nu_{\mathbb{T}_{N_k}, \exact} [(h_{v_0}-h_{v_n})^2]/{n} \leq \lim_{n\to \infty}c \frac{ \log n }{n}= 0
\] 
for some $c<\infty$, and hence by Lemma~\ref{lem:duality}, for all $t\in \mathbb R$,
\begin{align}\label{eq:phi1}
	1= \lim_{n\to \infty} \lim_{k\to \infty} \nu_{\mathbb{T}_{N_k}, \exact}\big [\exp \big(i\tfrac t {\sqrt n} ({h_{v_0}-h_{v_n}})\big)\big]= \lim_{n\to \infty}\lim_{k\to \infty}{Z_{\mathbb{T}_{N_k},\coclosed}\big (\tfrac t  {\sqrt n}p_n\big)}/{Z_{\mathbb{T}_{N_k},\coclosed}(0)},
\end{align}
where again we identify the path $p_n$ with the associated 1-form.
Using that 
\[
\mathcal U(J+\varepsilon)-\mathcal U( J)= \mathcal U'(J)\varepsilon +\tfrac 12 \mathcal U''(J)\varepsilon^2 +o(\varepsilon^2)
\]
we can write
\begin{align*}
{Z_{\mathbb{T}_{N_k},\coclosed}\big (\tfrac t  {\sqrt n}p_n\big)}/{Z_{\mathbb{T}_{N_k},\coclosed}(0)} 
&= \mu_{\mathbb{T}_{N_k}, \coclosed} \Big [\exp \Big(-t\frac{1}{\sqrt n}\sum_{i=1}^n  \mathcal U' (J_i)  - {t^2} \frac {1}{2n} \sum_{i=1}^n \mathcal U''(J_i)+o(t^2)\Big)\Big],
\end{align*}
where the error is uniform in $k$.
We first note that by weak convergence, as $k\to \infty$, the right-hand side approaches to the same expectation but with respect to $\mu$ (the infinite volume limit of $\mu_{\mathbb{T}_N, \exact}$).
Moreover, the error therm vanishes in the limit $n\to\infty$. Assuming that the spin measure $\mu$ is ergodic, we also have that
\[
\frac 1n \sum_{i=1}^n \mathcal U'' (J_i)\to \mu[\mathcal U'' (J_0)] \quad \mu\textnormal{-a.s. as } n\to \infty
\] 
by Birkhoff's pointwise ergodic theorem (since $J$ is invariant under the shift along the path, and $\mathcal U''$ is bounded). 
Note that in the case of the XY model, there is only one translation invariant Gibbs measure in two dimensions \cite{MMSP} which must therefore be ergodic. 
By the dominated convergence theorem and \eqref{eq:phi1}, we conclude the following central limit theorem.
\begin{theorem} If $\mu$ is ergodic, then for any $t\in \mathbb R$,
\begin{align*}
\lim_{n\to \infty} \mu \Big [\exp \Big(-\frac{t}{\sqrt n}\sum_{i=1}^n  \mathcal U' (J_i)\Big)\Big] = \exp\Big(\frac{t^2}2   \mu [\mathcal U'' (J_0)] \Big).
\end{align*}
In particular,
\[
\frac1{\sqrt n}\sum_{i=1}^n  \mathcal U' (J_i)\to \mathcal N(0, \mu [\mathcal U'' (J_0)])
\]
in distribution as $n\to \infty$.
\end{theorem}