% !TEX root = ../main_duality.tex

\subsection{Delocalisation of roughly planar height function models.}
In this section, we will use Theorem \ref{T: var_increasing_height} to deduce that on many planar graphs, the height function delocalises. 
Consider here an infinite lattice $\Gamma = (\mathscr{V}, \mathscr{E})$ embedded in the plane, 
but not necessarily planar. We will assume throughout that $\Gamma$ (under this embedding) invariant under a bi-periodic lattice action, 
and that it has finite degrees. 
An example of such $\Gamma$ is $\ZZ^2$ where all vertices are connected if they are within distance $M < \infty$ from each other. 
Given $\Gamma$, recall that $(G_N)_{N \geq 1}$ is an exhaustion of $\Gamma$ if $G_N$ is a finite subgraph of $\Gamma$ for each $N$, $G_N \subset G_{N + 1}$ and $G_N \nearrow \Gamma$. 
We will also consider the \emph{wiring} of $G_N$ by identifying $G_N^c$ in $\Gamma$ to a single vertex $\partial$ and removing all the self-loops created in this process. 
The obtained graph will be denoted by $G_N^*$. 
On such graphs, we will take the measures $\nu_{N, \beta, \exact}$ as in Section \ref{sec:setup2}, and we identify the space of $1$-forms $\n$ in $\Hcycle(\mathbb{Z})$ with functions $h$ in $\Omega^0_0(\ZZ)$. 

\begin{theorem}[Delocalisation] \label{T: deloc}
	Let $\Gamma$ be as above and consider the setup as in Section~\ref{sec:setup2} where we assume that $\c{U}_e$ is the same for all edges. 
	There exists a $\beta_c < \infty$ such that for all $\beta \geq \beta_c$ and all wired exhaustions $G_N^* \nearrow \Gamma$, 
	\[
		\nu_{N, \beta, \exact}[h_o^2] \to \infty. 
	\]
\end{theorem}

To prove this theorem, we rely on a beautiful result of Lammers \cite{Lammers}:
\begin{theorem}[Theorem 2.7 \cite{Lammers}] \label{T: Lammers}
	Let $\Gamma' = (V, E)$ be an infinite graph with degree at most three, that is invariant under some lattice action. 
	If $\c{V}$ is a convex potential for the height function with
	\[
		\c{V}(\pm 1) \leq \c{V}(0) + \log(2), 
	\]
	then the height function delocalises in the sense that there are no translation invariant Gibbs measures. 
\end{theorem}

In general, the potentials $\c{V}$ as in the setup of Section \ref{sec:setup2} need \emph{not}  be convex. 
However, in some special cases they are, as we will show next.  
This will be crucial for what follows: in Section \ref{sec: red to convex} it will be shown that we can always reduce to this case. 

\begin{lemma} \label{L: XY convex deloc}
	If $-\c{U}(\J) = \cos(i\J)$ for some $i \in \NN$, then $\c{V}_{\beta}$ is convex over $i\ZZ$ for all $\beta$. 
	Moreover, translation invariant Gibbs measures exist if and only if $\nu_{N, \beta, \exact}[h_o^2]$ is bounded uniformly in $N$. 
\end{lemma}
\begin{proof}
	In the case $-\c{U}(\J) = \cos(\J)$, convexity of $\c{V}_{\beta}$ over the integers is an easy consequence of the Tur\'an inequality, see e.g.~\cite{vEnLis}. 
	The extension to $-\c{U}(\J) = \cos(i\J)$ follows from a change of variables. 
	The second statement of the lemma was proved in the case of the XY model in \cite[Theorem 4]{vEnLis}. 
	It follows from a standard dichotomy (see e.g. \cite{LamOtt}) in the case where the height function satisfies the so called ``absolute value FKG'' property, 
	meaning that $|h|$ is FKG, see also \eqref{eq:vardeloc}. 
\end{proof}

\begin{remark}
	We wish to point out that the result of Lammers does \emph{not} depend on the potential $\c{V}$ being the same on each edge, just that it satisfies the condition of Theorem~\ref{T: Lammers} for all edges, and that the potentials are invariant under some lattice action. 
\end{remark}

We will first modify the potentials $-\c{U}$ so they will fit the framework of Theorem \ref{T: Lammers} and Lemma \ref{L: XY convex deloc}. 
Next, we modify the graph $\Gamma$ to obtain a graph $\Gamma'$ to which we can apply Theorem \ref{T: Lammers}
in such a way that $\Gamma'$ embeds into $\Gamma$ and the variance of the height function in $\Gamma'$ is smaller. 

\subsubsection{Reduction to convex potentials} \label{sec: red to convex}
We will apply here a simplification that allows us to only consider potentials of the form $-\c{U}(\J) = \alpha_i \cos(i\J)$. 
Since $-\c{U}$ is positive definite, it can be written as
\[
	-\c{U}(\J) = \alpha_0 + \sum_{i = 1}^\infty \alpha_i\cos(i \J),
\]
with $\alpha_i \geq 0$. Now let $i \geq 1$ be the first mode where $\alpha_i > 0$. Write $-\c{U}'= \alpha_i \cos(i\J)$ 
and $\nu'_{G, \beta, \exact}$ for the corresponding height function measure. 

\begin{lemma} \label{L: pure-pot reduction}
	For any finite graph $G = (V, E)$ with boundary $\partial$ and any $x \in V \setminus \{\partial\}$, we have
	\[	
		\nu'_{G, \beta, \exact}[h_x^2] \leq \nu_{G, \beta, \exact}[h_x^2].
	\]
\end{lemma}
\begin{proof}
	Take $-\c{U}'' = -\c{U} + \c{U}'$ which is positive definite. 
	Write for any $\alpha \geq 0$ 
	\[
		\c{U}_{\alpha}(\J) = \c{U}'(\J) + \alpha\c{U''}(\J), 
	\]
	so that $\c{U}_1 = \c{U}$, $\c{U}_0$ = $\c{U}'$ and $-\c{U}_{\alpha}$ is positive definite for each $\alpha$. 
	 Let $\nu_{G, \beta, \alpha, \exact}$ be  the corresponding height function measure. 
	 Theorem \ref{T: var_increasing_height} implies that for any $x \in V$
	 \[
	 	\frac{\partial}{\partial \alpha} \nu_{G, \beta, \alpha, \exact}[h_x^2] \geq 0,
	 \]
	 so that the variance is minimized at $\alpha = 0$. This shows the result. 
\end{proof}

\subsubsection{Graph Modifications.} \label{sec: graph mod}
Fix $\Gamma$ an infinite graph and $G_N \nearrow \Gamma$ an exhaustion as above.   
%We need to modify graphs in such a way that we keep track of the variance of the height function. 
We wish to perform two operations:
\begin{enumerate}[\hspace{1cm}(a)]
	\item splitting edges into multiple sub-edges and
	\item gluing vertices together, 
\end{enumerate}
in such a way that the variance of the height function does not increase. 

Operation (a) is the easiest: to add $k - 1$ ``evenly spaced'' vertices to an edge without changing the height function on the original graph, 
we wish to find a potential $\c{V}_{\beta}^{(k)}$ such that 
\[
	e^{-\c{V}_{\beta}} = (e^{-\c{V}_{\beta}^{(k)}})^{*k},
\]
where by $^{*k}$ we mean $k$-fold convolution. 

Using basic properties of the Fourier transform, we can take the potential $\c{V}_{\beta}^{(k)} = \c{V}_{\beta / k}$ which is dual to $-(\beta / k)\c{U}$.  
This offers the following lemma. 

\begin{lemma}[Splitting edges] \label{L: splitting edges}
	Suppose $\c{V}$ corresponds to a spin potential $\c{U}$, such that $-\c{U}$ is positive definite.  
	For each $k \in \NN$, we have
	\[
		e^{-\c{V}_{\beta}} = (e^{-\c{V}_{\beta / k} })^{*k}.
	\]
\end{lemma}

Operation (b) will make use of Theorem \ref{T: var_increasing_height}. 
Let $v_1, v_2$ be two vertices in the graph, 
with or without an edge between them and add to the graph the edge $g = \{ v_1, v_2\}$ with the XY potential $\c{V}_{\lambda}(k) = -\log(I_k(\lambda))$ 
with parameter $\lambda$. 
Write $\nu_{N, \beta, \lambda, \exact}$ for the corresponding height function measure on $G_N^*$. 
We will show now that gluing the vertices $v_1, v_2$ corresponds to sending $\lambda$ to $0$ in this setting. 
Indeed, as $\lambda \to 0$, we have
\[
	e^{-\c{V}_\lambda(k)} = I_k(\lambda) \to \begin{cases}
		1, & \text{if } k = 0\\
		0, &\text{else }
	\end{cases} 
\]
which means that the height function measure $\nu_{N, \beta, 0, \exact}$ is supported on height functions with $h_{v_1} = h_{v_2}$. 
Moreover, Theorem \ref{T: var_increasing_height} implies that for any vertex $x$ of $G_N^*$, 
\[
	\frac{\partial}{\partial \lambda} \nu_{N, \beta, \lambda, \exact}(h_x^2) \geq 0, 
\]
so that we find the following result. 

\begin{lemma}[Gluing vertices] \label{L: Gluing vertices}
	Let $x, v_1, v_2 \in V$ and $H_N^*$ be obtained from $G_N^*$ by gluing together $v_1$ and $v_2$. Then
	\(
		\nu_{H_N, \beta, \exact}[h_x^2] \leq \nu_{G_N, \beta, \exact}[h_x^2]. 
	\)
\end{lemma}

To summarise, we have established that gluing two vertices together reduces the variance of the height function, and splitting edges as in Lemma \ref{L: splitting edges} does not change the model. 
These two facts together imply that we can modify $\Gamma$ to obtain a planar graph $\Gamma'$ of degree at most three as we will explain now. 
We first show how to go from any planar graph to a planar graph of degree at most three. 

\subsubsection*{Star-tree transform}
There are many ways to transform a planar graph into a planar graph with degree at most three.
We follow here the elegant idea presented in \cite{GurNach}, where it was (implicitly) stated for the Gaussian free field. 
 Suppose that $G = (V, E)$ is a planar graph with boundary $\partial \in V$ and take the setup of Section \ref{sec:setup2}. Fix a vertex $v_0 \in V$.
It will be slightly more convenient to make a distinction between the number of neighboring vertices of $v_0$ and its degree in multigraphs.
\\
\\
\noindent \textbf{Degree reduction algorithm at $v_0$.} 
\begin{enumerate}[\hspace{0.4cm}1.]
	\item If the number of neighbours of $v_0$ is strictly less than $4$, do nothing.
	\item Label all neighbors of $v_0$ by $v_1, \ldots, v_{2d}$ by starting somewhere and going clockwise around $v_0$, where we \emph{do not} include the last vertex if the number of neighbours is odd. 
	\item Add to each edge $v_0v_i$ an intermediate vertex $x_i$ (note that if there are multiple edges between $v_0$ and $v_i$, then we have created many new vertices). 
	\item put the potential $\c{V}_{\beta_{v_0v_i} / 2}$ on the edges $v_0x_i$ and $x_iv_i$, for each $i$. 
	\item Glue together each pair $x_{2i - 1}$ and $x_{2i}$ (this includes gluing together multiple vertices $x_i$ if they exist). 
\end{enumerate}

Note that this algorithm reduces the \emph{number of neighbors of} $v_0$ by a factor $2$ if this number is even. 
Also note that it creates a multigraph. 
From the splitting and gluing lemmas, we obtain the next result. 
\begin{lemma}\label{L: star-tree transform}
	Applying the degree reduction algorithm at $v_0$ does not increase the variance of $h_x$ for any $x \in V$. 
\end{lemma}

Thus, to reduce the number of neighbors of $v_0$ to $3$ or less, we are left to apply the reduction algorithm inductively, 
and to get a graph of degree three we apply it to all vertices in $G$ other than the boundary vertex. 

To finalize the star-tree transform, we still need to transform the multi-graph into a simple graph. Of course, 
we need to do so without changing the height function model. 
If $e_1, e_2$ are two edges with the same end-points $x$ and $y$ then
\begin{align} \label{eq:adding potentials}
	\nu_{N, \beta, \exact}(h_x - h_y = k) \propto e^{-\c{V}_\beta(k)} e^{-\c{V}_\beta(k)} = e^{-(\c{V}_{\beta} + \c{V}_{\beta})(k)}. 
\end{align}
This observation implies that applying inductively the reduction algorithm and then applying the above observation does result in a graph where:
\begin{enumerate}[(i)]
	\item the number of neighbors of each vertex is less than or equal to $3$,
	\item the variance of the height function is not increased, 
	\item the potentials are of the form $D\c{V}_{\beta / k}$ for some $D$ and $k$ that can depend on the edges.  
\end{enumerate}

\begin{center}
	\begin{figure}
		\begin{subfigure}{.15\textwidth}
			\centering
			\includegraphics[scale =0.20]{Figures/star-tree_1.png}
		\end{subfigure} \hspace{1.0cm}
		\begin{subfigure}{.15\textwidth}
			\centering
			\includegraphics[scale =0.20]{Figures/star-tree_2.png} 
		\end{subfigure}  \hspace{1.0cm}
		\begin{subfigure}{.15\textwidth}
			\centering
			\includegraphics[scale =0.20]{Figures/star-tree_3.png} 
		\end{subfigure} \hspace{1.0cm}
		\begin{subfigure}{.15\textwidth}
			\centering
			\includegraphics[scale =0.20]{Figures/star-tree_4.png} 
		\end{subfigure} \hspace{1.0cm}
		\caption{Left: original graph with $v_0$ in the center. 
			Middle two: first step of the degree-reduction algorithm, dotted lines correspond to vertices to be glued together. 
			Right: Final graph after ``star--tree'' transform, with vertices glued together and all vertices have three or less neighbors.}
		\label{f:star-tree}
	\end{figure}
\end{center} 

\subsubsection{Proof of Theorem \ref{T: deloc}}
Before we finish the proof of Theorem \ref{T: deloc}, let us briefly mention how to go from a ``roughly planar'' graph to a planar graph, see also Figure \ref{f:long_range}. 
We will do so for $\ZZ^2$ where $x \sim y$ if $|x - y|_2 \leq 2$. 
Add to an edge connecting two vertices $x$ and $y$ that are at distance $2$ from eachother a new vertex. 
Glue it to the unique vertex between $x$ and $y$ that is at distance $1$ of each. 
Apply this algorithm to all edges. The obtained graph is planar and the variance of the height function is not increased by Lemma \ref{L: Gluing vertices}. 

\begin{proof}[Proof of Theorem \ref{T: var_increasing_height}]
	Consider the setup of Section \ref{sec:setup2}. 
	By Lemma \ref{L: pure-pot reduction}, we can assume without loss of generality that $-\c{U}(\J) = \cos(i\J)$.  
	Write $\c{V}_{\beta}$ for the corresponding height function potential. 
	
	Let $\Gamma'$ be the planar graph obtained from $\Gamma$ as in Section \ref{sec: graph mod}, with the corresponding potentials $D_e\c{V}_{\beta / k_e}$, 
	$D_e \in (0, \infty)$ and $k_e \in \NN$. 
	Although $D_e, k_e$ may be different on distinct edges, they are uniformly bounded because $\Gamma$ (and hence $\Gamma'$) is invariant a bi-periodic lattice action. 
	
	Let $(G_N)_{N \geq 1}$ be any exhaustion of $\Gamma$ and let $(G_N')_{N \geq 1}$ be the induced exhaustion of $\Gamma'$, obtained from applying the degree reduction algorithm to all of $G_N$ (but not the boundary vertex). 
	Write $\nu'_{N, \beta, \exact}$ for the corresponding height function measure on $G_N'$. 
	It follows from Section \ref{sec: graph mod} that it suffices to prove that for all $\beta$ large enough, $\nu'_{N, \beta, \exact}[h_o^2] \to \infty$. 
	Indeed, in this case we also have $\nu_{N, \beta, \exact}[h_o^2] \to \infty$. 
	
	Note that since $\c{V}_{\beta / k}$ is convex, so is any multiple. 
	Moreover, for each $D \in (0, \infty)$ and $k \in \NN$ we have that for all $\beta$ large enough, 
	\[
		D\c{V}_{\beta / k}(0) \leq D\c{V}_{\beta / k}(1) + \log(2). 
	\]
	Indeed, this follows from the fact that the modified Bessel function satisfies $I_{m}(\beta) / I_{m'}(\beta) \to 1$ as $\beta \to \infty$ (see e.g.~\cite{vEnLis}). 
	Therefore, we can apply Theorem \ref{T: Lammers} and Lemma \ref{L: XY convex deloc} to deduce that for $\beta$ large enough, 
	\[
		\nu'_{N, \beta, \exact}[h_o^2] \to \infty
	\]
	as $N \to \infty$. 
	This proves the theorem. 
\end{proof}

\begin{center}
	\begin{figure}
		\begin{subfigure}{.20\textwidth}
			\centering
			\includegraphics[scale =0.20]{Figures/long_range_1.png}
		\end{subfigure} \hspace{1.5cm}
		\begin{subfigure}{.20\textwidth}
			\centering
			\includegraphics[scale =0.20]{Figures/long_range_2.png} 
		\end{subfigure}  \hspace{1.5cm}
		\begin{subfigure}{.20\textwidth}
			\centering
			\includegraphics[scale =0.20]{Figures/long_range_3.png} 
		\end{subfigure} \hspace{1.0cm}
		\caption{Left: an example of $\ZZ^2$ with long-range interactions; only the edges of the (red) origin are drawn. Middle: gluing. The square (gray) vertices are added, together with the green edges where the gluing will happen. Right: the final (planar) graph.}
		\label{f:long_range}
	\end{figure}
\end{center} 