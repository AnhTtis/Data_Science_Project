% !TEX root = ../main_duality.tex

\section{General duality}\label{sec:gendual}

\subsection{Discrete calculus}
We first give a basic background on discrete calculus on graphs, 
staying close to the language of \cite{LyPer}. 
Let $G = (V, E)$ be a locally finite graph and let $\Group$ be a group (we will consider $\Group= \RR, \ZZ$ with addition and $\Group= \mathbb{S}:=\{z\in \mathbb C: |z|=1\}$ with multiplication). 
To keep the exposition homogenous, we will use the additive notation for all considered groups.

A \emph{1-form} $\omega$ taking values in $\Group$ is an antisymmetric function defined on the directed edges $\vec E$ of $G$, 
i.e., such that $\omega_{vv'}= -\omega_{v'v}$, where $vv'$ denotes the directed edge $(v,v')$. The set of $1$-forms will be denoted by $\Omega^1(\Group) = \Omega^1( G,\Group)$, and the set $\mathbb G^V$ by $\Omega^0(\Group) = \Omega^0(G, \Group)$. 
We will identify the space of 1-forms with $\mathbb \Group^E$ by fixing, once and for all, one of the two orientations for each edge in~$E$. 
Define the \emph{boundary} operator $\d^*: \Omega^1(\Group) \to \Omega^0(\Group)$ by 
\begin{align*}
	\d^*\omega_x = \sum_{y\sim x} \omega_{yx},
\end{align*}
where $y\sim x$ indicates that $y$ and $x$ are adjacent in $G$, and the \emph{co-boundary} operator $\d: \Omega^0(\Group) \to \Omega^1(\Group)$ by
\[	
\d f_{xy} = f_y - f_x.
\]
Note that $\d^*$ and $\d$ are homomorphisms between groups $\mathbb G^E$ and $\mathbb G^V$, and hence we can define groups
\[
\Hcycle(\Group) = \Hcycle(G,\Group):= \mathrm{Im}(\d) \cong \Group^V / \ker(\d) \quad \text{ and } \quad \Hstar(\Group) =  \Hstar(G,\Group) :=\ker(\d^*).
\]
For $\Group=\mathbb S$, we will write $d \J$ to be the Haar probability measure on the induced (compact) groups $\Hcycle(\mathbb S)$ and $\Hstar(\mathbb S)$. 
If $\Group$ is only locally compact, the Haar measure is defined up to a multiplicative constant and we fix some normalisation. 
For a more concrete definition, we refer to Appendix \ref{ap:duality}. 
We make the convention that the space over which we integrate determines the measure. 



\begin{notation}
In what follows we will use the letters $\epsilon, \omega$ (resp.\ $f,g,\tau$) to denote deterministic elements of $\Omega^1(\Group)$ (resp.\ $\Omega^0(\Group)$) when $\Group=\mathbb R$ or when $\mathbb G$ is not specified. 
We will write $\n$ and $h$ for (mostly random) elements of $\Omega^1(\mathbb Z)$ and $ \Omega^0(\mathbb Z)$ respectively, and $J$ and $\theta$ for (mostly random)
elements of $\Omega^1(\mathbb S)$ and $ \Omega^0(\mathbb S)$ respectively.

We will also often abuse notation in the following sense: through the identification $\exp( i\theta)\leftrightarrow\theta$, we have $\mathbb{S} \cong (-\pi, \pi]$, and
we will view $\J \in \Omega^1(\mathbb{S})$ as belonging to $\Omega^{1}(\RR)$. On the other hand, by considering real numbers modulo $2\pi$, 
we will map $\Omega^1(\RR)$ to $\Omega^1(\mathbb{S}^1)$.
One has to be careful when going from one space to the other: the embedding does not map $H_{\#} (\mathbb{S})$ to $H_{\#}(\RR)$, 
because for example a $1$-form $\omega \in \Omega^1(\mathbb{S})$ which satisfies $\d^* \omega= 0$ in $\mathbb{S}$, 
only satisfies $\d^*\omega = 0$ modulo $2\pi$ when viewed as a $1$-form in $\Omega^1(\mathbb{R})$. We will also think of $H_{\#}(\mathbb Z)$
as a subset of  $H_{\#}(\mathbb R)$ in the obvious way.
\end{notation}



\subsection{Spin models and random height functions}
In this section we consider a finite graph $G = (V,E)$. 
We will study random \emph{spin} and \emph{height} $1$-forms 
taking values in the spaces $H_{\#}(\mathbb S)$ and $H_{\#}(\mathbb Z)$ respectively for $\# \in \{\coclosed, \exact\}$. 

\begin{definition}[Height function and spin potentials] \label{def:potential}
Let $\mathcal V: \mathbb Z\to \mathbb R \cup \{ +\infty\}$ be symmetric, i.e., $\mathcal V (n)=\mathcal V(-n)$, such that 
\begin{align} \label{eq:Vsum}
\sum_{n\in \mathbb Z}n^2 \exp(-\mathcal V(n))<\infty,
\end{align}
and moreover such that $\exp(-\mathcal V)$
is \emph{positive definite}: for all $\alpha\in \mathbb R$, 
\begin{align}\label{eq:posdef}
	w(\alpha):=\exp(-\mathcal V(0))+\sum_{n=1}^{\infty}\exp(-\mathcal V(n))2\cos( n\alpha) > 0.
\end{align}
We call $\mathcal V$ the \emph{height function potential}, and $\mathcal U (\alpha) := -\log w(\alpha)$ the \emph{spin potential}.
\end{definition}
We will always assume that the considered potentials satisfy the conditions of Definition~\ref{def:potential}.
Note that condition~\eqref{eq:Vsum} implies that the series in \eqref{eq:posdef} is absolutely summable, and moreover that $\omega$, as well as $\mathcal U$, is twice continuously differentiable in $\alpha$.
In general, we will say that a function is positive definite if its Fourier transform is a nonnegative function.
This is not the classical definition of positive definiteness, but it is equivalent to it by Bochner's theorem.
\begin{example} \label{exple:potentials}The following potentials satisfy the conditions of Definition~\ref{def:potential}:
\begin{itemize}
\item  $\mathcal V(n)= -\log(I_{n}(\beta))$, where $I_n(\beta)$ is the modified Bessel function of the first kind, and $\c{U}(t) = -\beta \cos( t)$ for all $\beta>0$ is the potential of the \emph{classical XY model},
\item $\mathcal V(n)=\beta n^2$ for all $\beta >0$ is the potential of the \emph{integer-valued Gaussian free field} and the corresponding $\mathcal U$ defined through the series in~\eqref{eq:posdef} is the potential of the \emph{Villain spin model},
\item  $\mathcal V(n)=\beta \mathbf 1\{ n=\pm 1\}+\infty  \mathbf 1\{ |n|>1\}$ for $\exp(-\beta)<1/2$ is a model of random (nonuniform) \emph{Lipschitz functions}.
\item
	Any \emph{annealed Gaussian} potential $\c{V}$ meaning that there exists a finite Borel measure $\lambda$ on $[0, \infty)$ such that
\[
	e^{-\c{V}(n)} = \int_{[0, \infty)} e^{-\frac{\gamma}{2} n^2} \lambda(d \gamma)
\]
for all $n$. 
It satisfies Definition \ref{def:potential} because the function $n \mapsto \frac{\gamma}{2} n^2$ does and because by dominated convergence, we can exchange the integral and the summation in \eqref{eq:posdef}.  
This class includes the potentials $\c{V}(n) = \beta |n|^a$ for any $a \in (0, 2]$ (see~\cite{AHPS}). 
\end{itemize}
\end{example}

Let $\omega$ be as in Definition~\ref{def:potential}. Fix $\# \in \{\coclosed, \exact\}$, and consider a probability measure on \emph{spin $1$-forms} $J\in H_{\#}(\mathbb{S})$ defined by
\begin{align} \label{def:xy}
	d\mu_{\#}(\J)=d \mu_{G, \#}(\J) =\frac{1}{Z_{\#}} \Big(\prod_{e \in E}w(\J_e) \Big) d \J,
\end{align}
where $Z_{\#}$ is the \emph{partition function}, and $d \J$ denotes the Haar probability measure on the group $H_{\#}(\mathbb{S})$. 
%We denote by $\mu_{\#}[\cdot]$ the expectation with respect to $\mu_{\#}$.
For a 1-form $\epsilon \in \Omega^1(\RR)$, we define the \emph{twisted partition function}
\begin{align*}
	Z_{\#}(\epsilon) = \int_{H_{\#}(\mathbb{S})} \prod_{e\in  E} w  (\J_{e}+\epsilon_e ) d \J, 
\end{align*}
and note that $Z_{\#}(0)=Z_{\#}$.
We also define a probability measure on \emph{height $1$-forms} $\n \in H_{\#}(\ZZ)$ by
\begin{align} \label{eq:defh}
	\nu_{\#}(\n) =\nu_{G, \#} (\n) \propto \exp\Big(-\sum_{vv'\in E}\mathcal V(\n_{vv'})\Big).
\end{align}
Note that this is well defined as the normalisation constant is finite by assumption~\eqref{eq:Vsum}. 

%where $v\sim v'$ means that $\{v,v'\}\in E$.
%Here $\partial G\subset V$ is a chosen set of \emph{boundary vertices} that are wired together.
% We write $\nu$ for the expectation with respect to $\nu$.

For $f,g\in  \Omega^1(\mathbb R)$ and $\epsilon, \omega\in \Omega^1(\mathbb R)$, we will write 
\[
(f,g)_{\Omega^0}= \sum_{v\in V} f_vg_v, \qquad \text{and} \qquad (\epsilon, \omega)_{\Omega^1} = \frac{1}{2}\sum_{\vec{e} \in \vec{E}} \epsilon_{\vec{e}} \:\omega_{\vec{e}} =\sum_{{e} \in {E}} \epsilon_{{e}} \:\omega_{{e}} 
\]
for the standard inner products. We will usually drop the subscripts and simply write $(\cdot,\cdot)$ in case the space is clear from the context.

The central result that we will use is the following duality formula. Even though it is classical (see e.g.\ Appendix A in~\cite{FroSpe}), we will provide its derivation in Appendix~\ref{ap:duality}.
\begin{lemma}[Fourier--Pontryagin duality] \label{lem:duality}
	Let $\# \in \{\coclosed, \exact\}$ and let $-\#$ denote the other element of $ \{\coclosed, \exact\}$. 
	Then for any $\epsilon \in \Omega^1(\RR)$, we have
	\[
	\nu_{-\#} [\exp({ i (\n, \epsilon)}) ] = \frac{Z_{\#}(\epsilon)}{Z_{\#}}=\mu_{\#}\Big[\prod_{e\in E} \frac{w  (\J_e+\epsilon_e )}{w  (\J_{e})}\Big].
	\]
\end{lemma}



Clearly there are two intertwined random objects in the statement of Lemma~\ref{lem:duality}: the height and spin $1$-forms $\n$ and $J$ respectively. We will mostly apply the duality to analyse one of these two models
whose values are the \emph{exact} $1$-forms $H_\exact(\mathbb G)$, since then for each $\omega \in H_\exact(\mathbb G)$, there exists a unique $\tau \in \mathbb G^V$ such that
\[
\d\tau = \omega \qquad \text{ and } \qquad \tau_\partial =0,
\]
 where $\partial\in V$ is a fixed \emph{boundary} vertex of $G$, and $0$ is the identity element of $\mathbb G$. The random configuration $\tau$ is then distributed as a classical spin system with spins assigned to vertices with $0$ boundary conditions at $\partial$, and that interact through edges.


\begin{remark} 
In two dimensions there is a special form of duality where $\Hstar$ on the planar graph $G$ can be seen as $\Hcycle$ on the \emph{planar dual} graph $G^*$ by simply rotating all directed edges by $\pi/2$ to the left.
	Therefore if $\omega$ is a $1$-form such that $\d^*\omega = 0$, 
	there exists a function $\tau$ on the vertices of the {dual graph} $G^*$ (faces of $G$) which has $\omega$ as its gradient, i.e. 
	\begin{align*}
		\omega_{vv'} = \tau_u-\tau_{u'} = \d \tau_{uu'},
	\end{align*}
	where $u,u'$ are the two faces adjacent to $vv'$ from the right and left respectively. In this case, both models in Lemma~\ref{lem:duality} can be seen as classical spin and height function models.
\end{remark}


\begin{remark}
	As mentioned in the introduction, the Fourier--Pontryagin duality is usually applied in the opposite direction to Lemma~\ref{lem:duality}, i.e., to compute the characteristic function of the spin model rather than the height function. On the height function side this results in expectations of nonlocal observables (disorders) which are in general difficult to analyse.
	In our case however the disorder appears on the spin model side, and can be removed from the picture by taking derivatives at zero of the characteristic function.
	This is the main point of view which allows to obtain most of the results in this article using comparatively elementary arguments.
\end{remark}



One of the main tools in this article is the following identity. Even though it is a rather direct consequence of duality, we were unable to find this formulation in the literature.

\begin{lemma}[Covariance duality]\label{L: covariance} 
	Let $\# \in \{\coclosed, \exact\}$ and let $-\#$ be the other element of $ \{\coclosed, \exact\}$. For any $\epsilon, \omega \in \Omega^1(\RR)$, we have
	\[
 \E_{\#}\big[(\n, \epsilon)(\n, \omega)\big] + \mu_{-\#} \big[(\c{U}'(\J), \epsilon)(\c{U}'(\J), \omega)\big] = \sum_{e \in E} \mu_{-\#}[\c{U}''(\J_e)] \epsilon_e\omega_e.
	\]
\end{lemma}

\begin{proof}
It is enough to compute 
\[
\frac\partial {\partial s} \frac\partial {\partial t}	\Big(\nu_{\#} [\exp({i (h,s \epsilon +t \omega)}) ] \Big) \Big|_{s =t =0}
\]
by differentiating under the sign of integration on the right-hand side of the formula from Lemma~\ref{lem:duality}.
\end{proof}


\begin{remark}
In the case of measures $\nu_{\exact}$ on true height functions $h$, the quantity $\E_{\exact}[(\n, \epsilon)(\n, \omega)]=\E_{\exact}[(h,\d^* \epsilon)(h,\d^* \omega)]$ explicitly depends only on $\d^*\epsilon$ and $\d^*\omega$, and hence the rest of the equation above does so implicitly.
\end{remark}


Choosing $\epsilon = \id_{xy} - \id_{yx}$ for some edge $xy$ and $\tau = \id_{uv} - \id_{vu}$ for another edge $uv$, as a corollary we immediately get the following identities:
\begin{align} \label{eq:pontwiseD1}
\E_{\#}[\n_{xy}^2] = \mu_{-\#}[ \mathcal U''(J_{xy})- \mathcal U'(J_{xy})^2]
\end{align}
and
\begin{align} \label{eq:pontwiseD2}
\E_{\#}[\n_{xy}\n_{uv}] = -\mu_{-\#}[\c{U}'(\J_{xy})\c{U}'(\J_{uv})].
\end{align}



\begin{remark}
	Note that Lemma~\ref{L: covariance} implies that the sum of the covariance matrices of two mutually dual edge fields is diagonal, i.e., equals the covariance matrix of (possibly inhomogeneous) white noise. This was known for the discrete Gaussian Free Field ($\mathbb G=\mathbb R$), see e.g.~\cite{dubedat2011topics, Aru15} and Remark~\ref{rem:gff}, in which case the independent sum of the mutually dual edge fields is a collection of independent normal random variables. 
	The continuum analogue for the GFF can be found in \cite{AKM}. 
\end{remark}

\begin{remark} 
		Arguments similar in spirit to Lemma \ref{L: covariance}, but in the context of the Villain model appear in \cite{GaSe}. 
\end{remark}

Having established the covariance duality formula in Lemma~\ref{L: covariance}, we will now discuss several of its rather direct consequences. 
%Most of the results obtained in this section are relatively simple to derive, 
%and rely on some classical potential theory together with known correlation inequalities. 
Unless stated otherwise, we study the models on a finite graph $G = (V,E)$ with a prescribed boundary vertex $\partial \in V$.
We will write $ \Omega_0^0(\mathbb G)$ for the set of functions $f\in \Omega^0(\mathbb G)$ with $f_\partial =0$.