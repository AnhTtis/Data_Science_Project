% !TEX root = ../main_duality.tex

\section{Introduction}
The phenomenon of duality in statistical mechanics goes back to the famous work of Kramers and Wannier who discovered an exact identity 
between the partition functions of the Ising model on a finite planar graph and an Ising model (at a different temperature) on its dual graph~\cite{KW}.
They used it to identify the {self-dual} temperature (that stays invariant under the duality transfomation) as the point of phase transition in the model on the square lattice (that is itself a self-dual graph). 
In an extended version of this correspondence, spin correlation functions are mapped to correlators of dual disorder variables introduced by Kadanoff and Ceva~\cite{KC}. 
This construction has been very fruitful in the study of the Ising model. A notable example are the works of Smirnov~\cite{smirnov}, and Chelkak and Smirnov~\cite{CheSmi}, who derived scaling limits of certain variants of order-disorder correlations (the so called fermionic observables), providing the first proofs of conformal invariance of the critical Ising model.

It is by now classical that analogous duality transformations exist for more general spin models 
whose state space is a locally compact abelian group \cite{WuWa, FroSpe, Savit} (we refer to~\cite{dubedat2011topics} for an introductory account). 
In such a setting Fourier transforms can be used to map one model with values in a group $\mathbb G$ to another model with values in the Pontryagin-dual group $\widehat {\mathbb G}$. For example, for $\mathbb G= {\mathbb Z/q \mathbb Z}$, $ q\in \mathbb N$,
duality was a crucial tool in the study of the planar $q$-state Potts model, the associated random cluster model, and the Ashkin--Teller model (see e.g.~\cite{BefDum,PfiVel,GlaPel,Lis19,Lis21,ADG}). These groups are self-dual in the sense that $\mathbb G\cong \widehat{\mathbb G}$, 
and moreover the models are dual to the same model on the dual graph, but with possibly different coupling constants. Another famous self-dual example is $\mathbb G=\mathbb R$ together with the discrete Gaussian free field, where duality exchanges electric and magnetic operators of the field (see e.g.~\cite{dubedat2011topics}).

In this article we go beyond the self-dual domain and consider the mutually dual but distinct groups of the integers ${\mathbb Z}$ and the circle~$\mathbb S$. 
This results in two very different objects facing each other on the opposite sides of the duality relation: one is a discrete \emph{random height function} with an unbounded set of values, and the other is a \emph{continuous spin model} with spins in the circle.
Duality can be then used to transfer probabilistic information from one side to the other.
 A landmark application of this relation appeared in the work of Fr\"ohlich and Spencer~\cite{FroSpe} who rigorously established the Berezinskii--Kosterlitz--Thouless (BKT) phase transition in the classical XY model on the square lattice (see Section~\ref{sec:bkt} for more background). They first showed delocalisation of the associated height function, and then used duality to conclude that spin correlations decay at most polynomially fast. New proofs of the latter implication appeared recently in~\cite{vEnLis,AHPS} which together with a novel approach to delocalisation introduced by Lammers~\cite{Lammers} yields alternative proofs of the BKT transition. All three proofs of~\cite{FroSpe,vEnLis,AHPS} use duality ``in the same direction'' in that they study spin correlations via disorder correlations in the dual height function. This leads to technical complications as disorders are nonocal functions of the height field.
 In~\cite{vEnLis,AHPS} these issues were taken care of by considering different graphical representations of the models. Here we argue that following duality ``in the opposite direction''
 leads to an even more concise proof (that only uses duality itself) of the implication that delocalisation of the height function implies the BKT transition in the spin model.  
Indeed, when the disorders appear on the spin model side, one can ``localise'' them by simply using the Taylor expansion to the second order, which is clearly impossible when disorders 
come as discrete excitations of the heights. 

Duality is an exact correspondence, and hence one expects that the critical point of the localisation-delocalisation phase transition is dual to the BKT critical point.
This was recently proved for the XY and the Villain model by Lammers~\cite{Lam23}. Here we also provide a result in the same direction for a larger class of models that includes the XY model. 
To be more precise, we establish an equivalence between the delocalisation of the height function and the divergence of a certain series (a type of susceptibility) of correlation functions in the spin model.

 
 Another contribution of this paper is a universal upper bound on the variance of the height function in terms of the variance of the discrete GFF. This holds true for all height function models with positive definite potentials, and moreover irrespectively of the graph being planar or not. 
There are two main applications. In the planar case, e.g., on $\mathbb Z^2$, this leads to a conjecturally optimal (up to a constant) logarithmic in the size of the system upper bound 
when the height function is delocalised. On the other hand, it shows that on transient graphs, e.g. on $\mathbb Z^d$, $d\geq 3$, the variance is always uniformly bounded and the height function is localised.

For a special class of potentials, we also establish monotonicity of the variance of the height function in a natural temperature parameter. 
As far as we know, the only available result of this type is for the integer-valued GFF \cite{KP}. 
We achieve {this} by transporting, through duality, the (appropriately generalised) Ginibre inequalities. 
One consequence is a direct proof of the fact that for the XY height function 
there is only one point of phase transition from a localised to a delocalised regime. 
A (more involved) proof of this fact was first given in~\cite{Lam23}.
Together with the dichotomy of Lammers \cite{Lammers22}, this also shows that for the XY height function on $\ZZ^2$, the transition is sharp.
Another application is that for two-dimensional Euclidean lattices with non nearest neighbour interactions, 
the height function undergoes a localisation-delocalisatoin phase transition.
  
We note that we only consider height functions with positive definite potentials, i.e., those that have well defined dual spin models, and vice versa.
The article is organised as follows:
 \begin{itemize}
 \item In Section~\ref{sec:gendual} we recall the notion of duality, and state in Lemma~\ref{L: covariance} its consequence for the covariances of the gradient of the height function and 
 and gradient of the spin model.
 This is the stepping stone to the remaining results in this article.
 \item In Section~\ref{sec:upperbound} we establish an upper bound on the variance of the height function in terms of the Green's function of the underlying simple random walk.
 The bound is valid on any, not necessarily planar graph, and in particular implies localisation of the height function on graphs on which simple random walk is transient.
 \item In Section~\ref{sec:bkt} in Theorem~\ref{thm:BKT} we give a direct proof of the fact that in two dimensions, delocalisation of the height function implies a BKT phase transition in the spin model in the sense that certain spin correlation functions are not summable. In Corollary~\ref{C: deloc BKT} using classical correlation inequalities we translate this to
 an analogous statement for the standard two-point functions, recovering the main result of~\cite{vEnLis}. Finally, for a subclass of spin models (that includes the classical XY model)
 we show that the above mentioned implication is actually an equivalence.
 \item In Section~\ref{sec:monot} in Theorem \ref{T: var_increasing_height} we show that for a certain class of height functions, the variance is increasing in the inverse temperature. 
 	We use this to prove that a phase transition occurs for these random height functions, when the underlying graph is planar or ``almost planar''. 
 \item In Section~\ref{sec:projection}, using only duality, we show that a certain (non-local) observable of a classical spin model has, up to multiplicative constants, the covariance of the discrete Gaussian free fiel. 
 	Remarkably, this holds for all graphs and does not depend on the temperature.
\item In Section~\ref{sec:CLT} we show a central limit theorem in the planar spin model that holds irrespectively of the temperature.
  \item In Appendix~\ref{ap:duality} we recall and give concise proofs of the main results needed for duality.
 \item In Appendix~\ref{ap:Gin} we extend the Ginibre inequalities to the setting that we need in Section~\ref{sec:monot}.
 \item In Appendix~\ref{ap:RP} we review the notion of reflection positivity that we use in Section~\ref{sec:bkt}.
  \item In Appendix~\ref{ap:div} we review the certain aspects of positive definite functions.
 \end{itemize}
 
 
 \subsection*{Acknowledgements} We thank Christophe Garban for useful remarks on an earlier version of the manuscript.
 The research of DvE was funded by the FWF (Austrian Science Fund) grant P33083: “Scaling limits in random conformal geometry”, and the research of ML
was supported by the FWF grant P36298-N: ``Spins, loops and fields''.
 