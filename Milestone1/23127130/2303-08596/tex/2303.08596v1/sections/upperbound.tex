% !TEX root = ../main_duality.tex


\section{Upper bound on the variance of the height function} \label{sec:upperbound}
In this section we consider random {exact} $1$-forms $\n \in \Hcycle(\mathbb Z) $ distributed according to~$\nu_\exact$. As mentioned before, for each such $1$-form $\n\in \Hcycle(\mathbb Z)$, there exists exactly one \emph{height function} $h\in \Omega_0^0(\mathbb Z)$ such that $ \d h =\n $.
Note that in the case of $\mathbb G=\mathbb R$, $\d$ and $\d^*$ are adjoint as linear operators, i.e., for all 
$f\in \Omega^0(\mathbb R)$ and $\omega\in\Omega^1(\mathbb R)$, we have
\begin{align} \label{eq:adjoint}
(f, \d^* \omega)_{\Omega^0} = (\d f,\omega)_{\Omega^1}.
\end{align}
Also note that the operator 
\[
\Delta:=\d^*\d: \Omega^0(\mathbb R) \to \Omega^0(\mathbb R)
\] 
is the \emph{graph Laplacian} on $G$, and it has a well defined inverse $\Delta^{-1}$ on $\Omega^0_0(\mathbb R)$. Moreover, as matrices,
\[
\Delta^{-1} =  \Green D^{-1},
\]
where $D=\textnormal{Diag}(\textnormal{deg}(v))_{v\in V\setminus \{ \partial\}}$ and $\Green$ is the Green's function of simple random walk on $G$ killed upon hitting $\partial$.

Let $f \in  \Omega^0_0(\mathbb R)$ and $\epsilon :=\d \Delta^{-1}f$ so that $\d^*\epsilon =f$. Discarding the explicitly nonnegative term $\mu_{\coclosed} [(\c{U}'(\J), \epsilon)^2]$ in Lemma~\ref{L: covariance} applied to $\epsilon=\omega$, and using \eqref{eq:adjoint} we get
\begin{align} \label{eq:upperbound}
	\nu_{ \exact}[(h, f)_{\Omega^0}^2] =\nu_{ \exact}[(\n, \epsilon)_{\Omega^1}^2] \leq \sum_{e \in  E} \epsilon_{e}^2 \big|\mu_{\coclosed}[\mathcal U''(\J_e) ]\big|  \leq C (\epsilon, \epsilon)_{\Omega^1},
\end{align}
where 
\[
C=\sup_{e\in E} |\mu_{\coclosed}[\mathcal U''(\J_e)]| \leq \sup_{J\in \mathbb S} |\mathcal U''(J)|<\infty.
\] 
On the other hand, by \eqref{eq:adjoint} again	$(\epsilon, \epsilon)_{\Omega^1}= (\d \Delta^{-1}f,\d \Delta^{-1}f)_{\Omega^1}=  ( \Delta^{-1}f,f)_{\Omega^0}$.

\begin{corollary}[GFF upper bound on variance]\label{cor:upper}
	For any $f\in \Omega^0_0(\mathbb R)$,
	\begin{align*}
		\nu_{\exact}[(h, f)^2] \leq  C  ( \Delta^{-1}f,f)_{\Omega^0}=C\sum_{v,v'\in V} f_vf_{v'}\frac{\Green(v, v')}{\deg(v')},
	\end{align*}
	where $C$ is as above.
\end{corollary}

\begin{remark} \label{rem:gff}
	One can also apply duality to the discrete Gaussian free field (GFF) (in this case both the primal and dual fields are real-valued as $\mathbb R$ is self-dual as a locally compact abelian group).
	The GFF is defined similarly to the integer-valued GFF with potential $\mathcal V(t)=t^2$ with the difference that the reference measure in~\eqref{eq:defh} is the Lebesgue measure on $\mathbb R$ and not the counting measure on $\mathbb Z$. 
	The model is self dual in that $\mathcal U(t)=\mathcal V(t)=t^2$, and in the analog of the corollary above actually get an equality since $( \epsilon, \mathcal U'(\J_e))=0$ since $\mathcal U'(\J_e) =2 \J_e \in H_{\coclosed}$, and $\d^* \epsilon \in H_{\exact}$ by definition. This agrees with the fact that the covariance of the GFF is given \emph{exactly} by the inverse Laplacian.
\end{remark}


\begin{remark} \label{rem:tightness}
	Consider an infinite countable graph $\Gamma=(\mathscr V, \mathscr E)$ and a sequence of increasing finite subgraphs exhausting $\Gamma$, i.e., $G_N \nearrow \Gamma$ as $N\to \infty$.
	If $f: \mathscr V\to \mathbb R$ has bounded support and mean zero, i.e., $\sum_{v\in \mathscr V} f_v=0$, where this sum is actually taken over a finite set of vertices, then we can find a 1-form $\epsilon$ on $\mathscr E$ with \emph{bounded} support such that $\d^* \epsilon=f$, and hence 
	\begin{align}\label{eq:local}
	\prod_{e\in E} \frac{w  (\J_{e}+\epsilon_e )}{w  (\J_{e})}=\prod_{e\in \textnormal{supp} (\epsilon)} \frac{w  (\J_{e}+\epsilon_e )}{w  (\J_{e})}
	\end{align}
	is a local bounded continuous function of $\J$ (in the product topology on $\mathbb S^{\mathscr E}$) whenever $\mathcal V$ and $w$ are as in Definition~\ref{def:potential}. Moreover, since $\mathbb S$ is compact metrizable so is $\mathbb S^{\mathscr E}$ with the product topology by Tychonoff's theorem, 
	and hence the edge spin models $\mu_{G_N,\#}$ always form a tight sequence of measures on $\mathbb S^{\mathscr E}$ as $N\to \infty$. 
	This in particular implies that $\mu_{G_N,\coclosed}$ converges weakly along a subsequence.
	Therefore Lemma~\ref{lem:duality} together with \eqref{eq:local} and the fact that
	\[
		\nu_{G_N,\exact} [\exp({i (\n, \epsilon)}) ]=\nu_{G_N,\exact} [\exp({ i (f, h)}) ]
	\]  
	for $N$ large enough so that $G_N$ contains $\textnormal{supp} (\epsilon)$, imply that the random height $1$-forms $\n$ under $\nu_{G_N,\exact}$, and hence also the \emph{differences} of the associated height function $h$, converge weakly along the same subsequence. 
	
One has to be careful as this is in general no longer true if $f$ does not have zero mean, e.g., $f=\delta_v$. Then $\epsilon$ with $\d^* \epsilon=f$ cannot be taken with bounded support 
(there always has to be an infinite path with nonzero values of $\epsilon$).
In this case tightness may fail when \emph{delocalisation} of the height function arises, i.e., $\nu_{G_N,\exact}[(h, f)^2]= \nu_{G_N,\exact}[h_v^2]\to \infty$ 
as $N\to \infty$ (e.g.\ if $\Gamma$ is planar, see Section~\ref{sec:bkt}).
\end{remark}





We also immediately deduce that delocalisation of the height function does not happen on transient graphs for potentials as in Definition~\ref{def:potential}.
We note that our result, despite its simple proof, seems new in this generality, and that such behaviour is expected for a larger class of potentials. 
We also note that the special case of the integer-valued GFF follows from a stronger estimate proved by Fr\"{o}hlich and Park~\cite{FroPar} (see also~\cite{KP}).
Some results in this direction related to the intever-valued GFF can also be found in \cite{AHPS}.

To state the result, we briefly recall the notion of Gibbs measures and {gradient} Gibbs measures (we do it for height functions only, and the definition for spin models used later in the article is completely analogous).
From now on we assume that $\Gamma = (\mathscr V, \mathscr E)$ is a locally finite, infinite graph. 
For a finite set $\Lambda \subset \mathscr V$ write $E(\Lambda)$ for the set of edges with at least one vertex in $\Lambda$.  
Let $\varphi: \Lambda^c \to \ZZ$ be a function and define the probability measure $\mu_{\Lambda}^\varphi$ 
supported on $h: \mathscr{V} \to \ZZ$ satisfying $h \mid_{\Lambda^c} = \varphi$ by
\[
	\nu_{\Lambda}^{\varphi}(h) \propto \exp \Big(-\sum_{e \in E(\Lambda)} \c{V}(\d h_e)\Big). 
\]
In other words, $\nu_{\Lambda}^\varphi$ is the measure $\PP_{\exact}$ from \eqref{eq:defh} with $\varphi$-boundary conditions outside $\Lambda$. 
A probability measure $\nu$ supported on height functions $h: \mathscr V \to \ZZ$ is called a \textit{Gibbs measure} (on $\Gamma$ with respect to the potential $\c{V}$) if it satisfies the Dobrushin--Lanford--Ruelle (DLR) relations:
for all finite sets $\Lambda \subset \mathscr{V}$, 
\[
	\nu_{\Lambda}(\cdot) = \int_{\mathbb{Z}^{\mathscr V}} \nu_{\Lambda}^{\varphi}(\cdot) d \nu(\varphi), 
\]
where $\nu_{\Lambda}$ denotes the restriction of $\nu$ to $\Lambda$.
If $\Gamma$ is a Cayley graph and the measure $\nu$ is invariant under shifts, it is called translation invariant. 
In terms of Gibbs measures, delocalisation corresponds to non-existence of translation invariant Gibbs measures. 

A \emph{gradient} Gibbs measure is a slight variation of the above, where we consider measures supported only on gradients. 
Fix this time a finite set of edges $\Lambda \subset \mathscr{E}$. Let $\omega$ be an exact $1$-form (thus taking value in $\Hcycle(\mathscr{E}, \ZZ)$). 
Define the probability measure $\mu_{\Lambda}^{\omega}$ supported on $1$-forms $\n \in \Hcycle(\mathscr{E}, \ZZ)$ satisfying $h \mid_{\Lambda^c} = \omega \mid_{\Lambda^c}$ as
\[
	\mu_{\Lambda}^{\omega}(\n) \propto \exp\Big(-\sum_{e \in \Lambda} \c{V}(\n_e)\Big). 
\]
A probability measure supported on $1$-forms $\n \in \Hcycle(\mathscr{E}, \ZZ)$ will be called a gradient Gibbs measure if it satisfies the analog of the DLR equation above
in this setup. 

\begin{theorem} \label{thm:extransinv}
	Let $\Gamma = (\mathscr V,\mathscr E)$ be a transient graph and $\c{V}$ a height function potential as in Definition~\ref{def:potential}. 
	Then, there exists an infinite volume Gibbs measure on $\Gamma$ with respect to $\mathcal V$. If $\Gamma$ is moreover an amenable Cayley graph, 
	there exist translation invariant Gibbs measures. 
\end{theorem}

\begin{proof}
Let now $G_N\nearrow \Gamma$, as $N\to \infty$ be an exhaustion of $\Gamma$ by finite subgraphs $G_N$. 
Define the boundary $\partial_N := \partial G_N$ to be the set of vertices in $ G_N$ adjacent to a vertex from outside of $ G_N$.
Let $\E_{G_N, \exact}[\cdot]$ be the expectation associated with the height function on $ G_N$ with $0$-boundary conditions. 
Fix any $\Lambda \subset \c{V}$ finite and let $f: \Lambda \to \RR$ be any function.
It follows from Corollary~\ref{cor:upper} that 
\[
	\E_{G_N,\exact}[(h, f)^2] \leq C \max_{v \in \Lambda} \frac{\Green_{N} (v,v)}{\deg(v)} (f, f)_{\Lambda}. 
\]
where $C < \infty$, and $\Green_N$ is the Green's function of simple random walk on $G_N$ killed on hitting $\partial_N$. 
Since $\Gamma$ is transient, the right-hand side is uniformly bounded in $N$. 
Therefore, the sequence $\PP_{G_N, \exact}(h |_{\Lambda})$ is tight and subsequential limits exist by Prokhorov's theorem. 
By a diagonal argument, we can extract a further sub-sequence $N_K$ so that $\PP_{G_{N_K}, \exact}(h|_{\Lambda})$ converges for each
$\Lambda$ finite. 
Any such subsequential limit is a Gibbs measure as it satisfies the DLR relations. 
This proves the first part of the theorem. 

For the second part, suppose that $\Gamma$ is an amenable Cayley graph, 
so that 
\(
	\E_{G_N, \exact}[h_v^2] \leq C',
\)
for some $C' < \infty$ which is independent of $v$ and the exhaustion $(G_N)_{N\geq 1}$. 
Let $\mu$ be a subsequential limit (which exists by the argument above, and is a Gibbs measure). 
Let $o \in \mathscr V$. Since $\Gamma$ is amenable, there is some F\o lner sequence (also called Van Hove sequence) $(F_N)_{N \geq 1}$ {of sets of vertices} containing $o$. 
This means $F_N \nearrow \mathscr V$ and $|\partial F_N| / |F_N| \to 0$ as $N \to \infty$. 
Let
\[
	\nu_N := \frac{1}{|F_N|} \sum_{x \in F_N} \mu \circ \theta_x, 
\]
where $\theta_x$ is the shift towards $x$ (since $\Gamma$ is a Cayley graph, this is the same as left multiplication in the group).
This is a Gibbs measure because the set of Gibbs measures is closed under translations and convex combinations. 
Moreover, $\nu_N[h_v^2] \leq C'$ for each $N$ and $v \in \mathscr V$. Therefore, $(\nu_N)_{N\geq 1}$ is tight. 
Let $\nu$ be any subsequential limit, which is again a Gibbs measure. 
By construction and since $|\partial F_N| / |F_N| \to 0$, we have $\nu \circ \theta_x = \nu$ for each $x$, and hence $\nu$ is translation invariant.
\end{proof}


