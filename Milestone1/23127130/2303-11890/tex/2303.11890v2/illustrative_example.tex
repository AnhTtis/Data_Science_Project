

%This section illustrates the proposed approach for the Van der Pol oscillator system. The dynamics of the system translated to the equilibrium point $\overline{x} = (\overline{x_1}, \overline{x_2}) = (0, 0)$, taking into account an integrator to eliminate the steady-state error, is given by

{Consider the following system which consists of the Van der Pol equation with the addition of an integrator:}
\begin{equation} \label{eqn:vdp-ct}
\left\{
\begin{aligned}
 \dot x_1(t) & \!= x_2(t) \\
 \dot x_2(t) & \!= - x_1(t) \!+\! \theta \big(1 \!-\! x_1(t)^2\big) x_2(t) \!+\! u(t) \!+\! d(t) \\
 \dot x_3(t) & \!= {x_1(t)} \\
 y(t) & \! = x_1(t)
\end{aligned}
\right.
\end{equation}
{where $x := \begin{bmatrix} x_1 & x_2 & x_3 \end{bmatrix}^T \in \mcx$ is the state vector, $\theta \in \Theta \subset \bbr$ is an uncertain parameter, $u \in \bbr$ is the control input, $d \in \mcd \subset \bbr$ is an exogenous disturbance, and $y \in \bbr$ is the controlled output.}  

{The control objective is to regulate $y(t)$ around zero considering the following sampled control law:
\begin{equation} \label{eqn:u-sh}
u(t) = u[kT_s] , \ \forall \ t \in [kT_s,(k\!+\!1)T_s), \ k \geq 0 ,
\end{equation}
where $T_s$ is a sufficiently small sampling period. In addition, the discrete-time control signal $u[k T_s] \!=\! u[k] \!=\! u$ will be determined utilizing the control strategy described in Section~\ref{sec:cd}. To this end, by applying the Euler's forward approximation, the quasi-linear discrete-time representation of \eqref{eqn:vdp-ct} as given in \eqref{eqn:G2} is defined by the following matrices:}
\begin{equation*}
\begin{aligned}
{A(x, \theta)} &= { 
\left[\begin{matrix} 1 & T_s  & 0 \\[1mm]
- T_s & 1 \!-\! T_s \theta \left(x_{1}^{2} \!-\! 1 \right)  & 0 \\[1mm]
T_s & 0 & 1
\end{matrix}\right]},  \\[1mm]
   %%%%
B_u(\theta) \!=\! B_u &= \left[\begin{matrix}0\\T_s\\0\end{matrix}\right] \; \text{ and } \; B_w(\theta) \!=\! B_w = \left[\begin{matrix}0 & 0\\T_s & T_s\\0 & 0\end{matrix}\right] .
\end{aligned}
\end{equation*}

{In this example, it will be assumed that:
\begin{itemize}

\item $\mcx = \{ x \in \bbr^3: |x_1| \leq 2 \}$; 

\item $\Theta = \{ \theta \in \bbr: 0.5 \leq \theta \leq 0.9 \}$;

\item $\mcd = \{ d \in\bbr: |d | \leq 1 / \sqrt{2} \}$; and 
\item $\mcu = \{ u_2 \in \bbr: | u_2 | \leq 1 / \sqrt{2} \}$.

\end{itemize}
Notice, in this particular example, that $\mcx$ is only bounded on the $x_1$ direction, since $A(x,\theta)$ is only a function of $x_1$ and $\theta$.}


{In order to design the robust controller, the system matrix is cast as follows 
\begin{equation*}
\left\{
\begin{aligned}
A(x,\theta) & = A_0(\theta) + \Pi(x)^T A_1(\theta) \\ %\eqref{eqn:Pi} 
0_{6 \times 3} & = \Omega_0(x) + \Omega_1(x) \Pi(x)  %\eqref{eqn:Omega}    
\end{aligned} \right.
\end{equation*}
where}
\begin{equation*}
\begin{aligned}
&A_0(\theta) = \left[\begin{matrix}1 & T_{s} & 0\\- T_{s} & T_{s} \theta + 1 & 0\\T_{s} & 0 & 1\end{matrix}\right], A_1(\theta) = \left[\begin{matrix}0 & 0 & 0\\0 & 0 & 0\\0 & 0 & 0\\0 & 0 & 0\\0 & - T_{s} \theta & 0\\0 & 0 & 0\end{matrix}\right] \\
&\Pi(x) = \left[\begin{matrix}x_{1}\otimes I_3 \\ x_{1}^{2}\otimes I_3 \end{matrix}\right], \;\; \Omega_0(x) = \left[\begin{matrix} x_{1} I_3 \\ 0_3\end{matrix}\right] \;  \text{ and } \\
&\Omega_1(x) = \left[\begin{matrix} -I_3 & 0_3 \\ x_{1} I_3  & -I_3 \\\end{matrix}\right] .
\end{aligned}
\end{equation*}

{Hence, by considering $T_s = 0.1$ s {and $\theta=0.75$}, 
%to minimize the effects of the disturbance signal $d(k)$ on the system output $y(k)$, 
the optimization problem \eqref{eqn:op} is solved by applying a line search over $\mu \in (0,1)$ 
leading to the robust control law 
\begin{equation*}
    u = (K_0 + K_1\Pi(x))x = K(x)x
\end{equation*}
for an optimal $\mu = 0.3$, where}
%$u_1(k) = (K_0 + K_1\Pi(x))x = K(x)x$, where $\mu = 0.3$ and
\begin{equation*}
K(x)^T = \left[\begin{matrix}- 0.0217 x_{1}^{2} - 3.514 \cdot 10^{-14} x_{1} - 68.42\\0.7203 x_{1}^{2} + 1.058 \cdot 10^{-14} x_{1} - 16.73\\- 0.01905 x_{1}^{2} - 1.001 \cdot 10^{-13} x_{1} - 90.99\end{matrix}\right] ,
\end{equation*}
and the following reachable set estimate 
% \begin{multline*}
% \mathcal{R} = \{ x \in \bbr^3 : 2.7 x_{1}^{2} + 0.44 x_{1} x_{2} + 9.7 x_{1} x_{3} \\ + 0.033 x_{2}^{2} + 0.73 x_{2} x_{3} + 12 x_{3}^{2} \leq 10^{-3} \} .   
% \end{multline*}
  %%%%
\begin{multline*}
\mathcal{R} = \{ x \in \bbr^3 : 2.2 x_{1}^{2} + 0.36 x_{1} x_{2} + 7.1 x_{1} x_{3} \\ + 0.03 x_{2}^{2} + 0.54 x_{2} x_{3} + 8.2 x_{3}^{2} \leq 10^{-3} \} .   
\end{multline*}

Then, for training the ESN-based controller, a data-set with 5000 samples was constructed based on simulations of the closed-loop system (i.e., with $u[k] = u_1[k]$) by considering that $w$ and $u_2$ were filtered white noises with cutoff frequencies defined to take into account typical low frequency disturbances. In addition, the following was assumed for the ESN hyper-parameters: $(i)$ a spectral radius $\rho = 0.5$; $(ii)$ reservoir size $N = 200$; $(iii)$ leaking rate $\gamma = 0.6$; and $(iv)$ reservoir density of $0.9$. These parameters were chosen considering a grid search and the neural network training error as the performance index. The number of past outputs ($m=1$) and the delay samples ($\delta=2$) were tuned based on the root mean square (RMS) 
%error metric 
value of the system output $y$ 
%and reference $r$ 
when adding the ESN control action (i.e., $u = u_1 + u_2$) relative to the robust control action only (i.e., $u = u_1$).

\vskip 2mm

The performance of the proposed control strategy is evaluated in the sequel by means of numerical simulations considering the plant continuous-time model and the sampled control law as defined in \eqref{eqn:vdp-ct} and \eqref{eqn:u-sh}, respectively. In particular, Fig.~\ref{fig:example:yw}
shows the disturbance attenuation properties of the proposed controller (i.e., $u = u_1+u_2$) compared to the robust control action only (i.e., $u = u_1$). 
%response presented in  Fig.~\ref{fig:example:yw} was obtained from the continuous-time model and approximated using the ZOH method defined in \eqref{eqn:u-sh}. 
%The results demonstrated that the combination of the robust and ESN control actions outperformed the robust controller by an improvement factor of
%an improvement of 
%$54.36\%$ which was computed based on the root mean square error metric for the system output $y$, when adding the ESN control action relative to the robust control action only (which was the same metric utilized to determine the 
%hyper-parameters $\delta$ and $m$ of the ESN controller).
%In addition,  Fig.~\ref{fig:example:x1x2x3} shows the phase portrait of state trajectories (for both control laws) and the reachable set estimation. It is worth noting that the trajectories remain confined within the set $\mathcal{R}$, which is the estimated reachable set of the system. The trajectories are shown in red when $u = u_1$ and in black when $u = u_1 + u_2$.
%
The results show that the combination of robust control actions and ESN outperformed the performance obtained only with the robust controller by an improvement factor of $54.36\%$, which was calculated based on the RMS value of system output $y$ when adding the ESN control action to the robust control law relative to the robust controller only (which was the same metric utilized to determine $\delta$ and $m$ hyper-parameters for training the ESN-based controller). Furthermore, Fig.~\ref{fig:example:x1x2x3}  shows the phase portrait of the state trajectories (for both control laws) and the estimate $\mcr$ of the closed-loop reachable set. It is worth to mention that, as expected, the state trajectories remain confined to the set $\mathcal{ R}$ for all $t\geq0$.
\begin{figure}[htbp]
   \vspace{3mm}
   \begin{center}   
    \includegraphics[width=\columnwidth]{figures/example/yw_simulacao_sqrt2sin1tsin2t.pdf}
     % \includegraphics[width=\columnwidth]{figures/example/yw_simulacao_sqrt2sin1tsin2t.pdf}
   \end{center}
   \caption{System closed-loop response (at the top of figure) considering $u=u_1$ (red dashed line) and $u=u_1+u_2$ (black solid line) for $x_0^T = [ -0.0225 \;\;0.252 \;\; 0.005]$, {$\theta=0.75$} and the disturbance signal (depicted at the bottom of the figure) defined as $d(t) = 0.25\sqrt{2}(\sin(t) + \sin(2t))$. \label{fig:example:yw}}
\end{figure}
\begin{figure}[htbp]
   \begin{center}
     \includegraphics[]{figures/example/x1_x2_x3_cropped.pdf}
   \end{center}
   \caption{Reachable set estimate and phase portrait of state trajectories for $x_0^T = [ -0.0225 \;\;0.252 \;\; 0.005]$,  {$\theta=0.75$} and $d(t) = 0.25\sqrt{2}(\sin(t) + \sin(2t))$, considering the control laws $u = u_1$ (state trajectory in red solid line) and $u = u_1 + u_2$ (state trajectory in black solid line).
   \label{fig:example:x1x2x3}}
\end{figure}

% \begin{figure}[htbp]
%    \begin{center}
%      \includegraphics[]{figures/example/u_simulacao_sqrt2sin1tsin2t.pdf}
%    \end{center}
%    \caption{a.\label{fig:example:u}}
% \end{figure}
% \begin{figure}[htbp]
%    \begin{center}
%      \includegraphics[]{figures/example/x1_x2.pdf}
%    \end{center}
%    \caption{Illustrative example state trajectories and reachable set estimation projection in the $x_1$-$x_2$ plane for $w(t) = 0.25\sqrt{2}(\sin(t) + \sin(2t))$ and $x_0^T = [ -0.0225 \;\;0.252 \;\; 0.005]$. In red when $u = u_1$ and in black when $u = u_1 + u_2$. \label{fig:example:x1x2}}
% \end{figure}