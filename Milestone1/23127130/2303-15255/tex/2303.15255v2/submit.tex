% !TEX TS-program = pdflatex
\documentclass[11pt,a4paper,oneside,final]{article}

\pdfoutput=1
\usepackage{epsf,amsmath,amssymb,dcolumn}
\usepackage{scalefnt,cite,hyperref}
\usepackage{cleveref,etoolbox,color,soul}
\usepackage{float}
%\usepackage{todonotes,xspace}
\usepackage{listings}
\usepackage[utf8]{inputenc}
\usepackage{indentfirst}

\usepackage{caption}
\usepackage{subcaption,graphicx}

\DeclareGraphicsRule{*}{mps}{*}{}
\setlength{\oddsidemargin}{0pt}
\setlength{\textwidth}{17.0cm}
\setlength{\textheight}{22cm}
\addtolength{\jot}{5pt}
\renewcommand{\textfraction}{0}
\topmargin-0.5cm
\oddsidemargin-0.5cm

\Crefname{figure}{Fig.}{Figs.}
\usepackage{placeins}
\usepackage{soul}

\usepackage[normalem]{ulem}
%\usepackage{parskip}

\usepackage[utf8]{inputenc}
\hypersetup{
	unicode,
	pdfproducer={LaTeX},
	pdfcreator={pdflatex}
}

\newcounter{notecount}
\setcounter{notecount}{1}
\newcommand{\note}[3]{\marginpar{\color{#3}$^{\arabic{notecount})}$}
    {\color{#3}\footnotesize
    [#1$_{\arabic{notecount}}$: #2]}%
  \addtocounter{notecount}{1}}%


\begin{document}
\thispagestyle{empty}

\long\def\symbolfootnote[#1]#2{\begingroup%
\def\thefootnote{\fnsymbol{footnote}}\footnote[#1]{#2}\endgroup}

\vspace{1cm}

\begin{center}
\Large\bf\boldmath
Cosmic muons background signals in PMT and SiPM detectors
\unboldmath
\end{center}
\vspace{0.05cm}

\begin{center}
%% \begin{center}
Gholamreza Fardipour Raki$^a$\footnote{Corresponding Author, fardipour@ipm.ir},
Mohsen Khakzad$^a$\footnote{mohsen@ipm.ir},
Shehu AbdusSalam$^b$\footnote{abdussalam@sbu.ac.ir}\\[0.4em]
\end{center}


\begin{center}
{\small
{{\sl ${}^a$ School of particles and accelerator, Institute for Research in fundamental sciences (IPM),\\ P.O.Box 19395-5531, Tehran, Iran}}\\[0.4em]
{\sl ${}^b$ Department of Physics, Shahid Beheshti University, Tehran, Iran }\\[0.2em]  
}
\end{center}


\vspace*{1mm}
\begin{abstract}
\noindent

PMT and SiPM are often used to detect a small number of photons produced by very weak radiations. A light guide usually connects these photodetectors to the test space. This research investigated background signals caused by cosmic muons due to the scintillation of the materials in the light guide and input window of PMT and SiPM. The amount of this scintillation is relatively determined by simulation and experiments. Cosmic muons emitting photons in these structures can leave a significant background error in low radiation and single photon detection if a standard scintillator is not used. These signals can be much stronger than dark current signals.

\end{abstract}

Keywords: SiPM, PMT, Light guide, Input Window, Scintillation, PMMA, Epoxy-resin, Glass, Background Signal, Cosmic Muon 

\setcounter{footnote}{0}

\newpage

\pagenumbering{arabic}

\section{Introduction}
\label{sec:introduction}

Experiments involving particles, including photons, detections have several sources of background signals. Considering Silicon Photo Multiplier (SiPM) and photomultiplier tube (PMT) as detectors developed in \cite{Raki:2022lwn}, we address backgrounds signals to such settings due to interactions of the cosmic muon with various parts of the detectors. For instance, undesirable signals can be caused by the scintillation of glass or polymethyl methacrylate (PMMA) in the light guide, epoxy resin, or glass materials in the upper window of the SiPM and PMT input window materials. 

SiPM is a pixelated photodetector of avalanche photodiode (APD), while the Single-photon avalanche diode (SPAD) is essentially the single-pixel version. For the SiPM pixels, APDs with integrated resistances in series, are connected in parallel through a metal grid and reverse-biased at a voltage a bit lower than the breakdown voltage. SiPM is made of a two-dimensional array of a large number of sensitive SPADs. The diodes are connected in parallel, and their effect is summed at the output. When particles hit the SiPM, the greater the number of affected diodes at the same time, the higher the peak voltage of the output signal. With forward-biased photodiodes, the greater the number of photons in the diode's frequency sensitivity range, the stronger the signal (due to the reduction in the diode's internal resistance). However, in this way, the effect of a small number of photons is never detectable. For the reversely biased case, the signal strength due to weak radiation depends more on the number of stimulated cells. A cell can be stimulated even by a single photon. Thus by arranging suitable biases, signals due to weak radiations can be amplified to strengths similar to those due to strong radiations\cite{Raki:2022lwn}. PMT detector's signal depend on the number and energy of incident photons reaching its photocathode. This, in turn, is limited by the input window material transparency to the photon wavelengths. Therefore one should expect a stronger effect of cosmic muon on hitting a SiPM than a PMT system. 

\textcolor{black}{ In low number and single photon detection by SPAD, SiPM, and PMT which a common scintillator doesn't use, the effect of cosmic muons trajectory in the detector is significant. Single photon detection by SPAD, SiPM, and PMT can be used in quantum information technology\cite{hadfield2009single} and is used in some other applications such as Dark matter detection \cite{APRILE2012573, ApplicationsSinglePhoton}, Flow cytometry\cite{ApplicationsSinglePhoton}, Bio- and chemiluminescence\cite{ApplicationsSinglePhoton}, Time-of-flight LiDAR and Flash LiDAR\cite{ApplicationsSinglePhoton}.}

In this text, in section~\ref{sec:sipm} we reviewed the optical characteristics of glass, PMMA, and epoxy resin. It was found that there is a significant scintillation effect in small dimensions of the material. In more oversized dimensions, the cumulative effect is not significant due to the absorption of emitted photons at distances less than one centimeter. Section~\ref{sec:sipm} also presents the structure of SiPM and the dimensions of the epoxy resin in SiPM.

To study the effect of cosmic muons as a background signal on the photomultiplier (PMT), first in section~\ref{sec:pmt}, we explained the structure of the PMT. We explained the optical characteristics of the input window and photocathode for different types of PMT. These optical characteristics are related to the photoluminescence spectrum of the window material due to the passage of energetic particles and the lifetime of its glow and light absorption spectrum. In section~\ref{sec:Discussion}, these optical characteristics are discussed and it is concluded that all the widely used PMTs and SiPMs are sensitive to the passage of muons, also the condition of this effect is investigated.

Section~\ref{sec:Simulations} describes the simulation of the light guide scintillation and input window of PMT due to the passage of energetic muons using the Gate, a simulation platform based on the Geant4\cite{Gate8.0} processing core, and the results of the simulations are compared. These simulations show the effect of scintillation of these materials and compare its value with BC408, a common scintillator. For each event of coming energetic muon, these materials luminesce but with less number of photons in comparison by BC408. Here Scintillation of BC408 does not come into discussion and is just compared. 

In section~\ref{sec:Experiments}, a practical experiment of the passage of cosmic muons through two PMTs is presented, one of them without a common scintillator and the other with a plastic scintillator BC408. The result of this experiment showed that by confirming the simultaneity of the signals obtained from two PMTs as the signal caused by the cosmic muon passing through the PMTs, the effect of the cosmic muon passing through the PMT without a common scintillator was compared with the other one. Finally, it was confirmed that the effect of the background signal caused by the passage of cosmic muons is quite significant in the detection of single photons or very weak radiations.

We also performed an experiment to evaluate the amount of the scintillation of the SiPM input window and a light guide plus the SiPM setup compared to when adding the BC408 due to the passage of cosmic muon. The results of this test also confirm the significant background signal because of these parts. Our experiments confirmed that there are stronger background signals by cosmic muons in SiPM than in PMT because of the nature of signal production in SiPM by reverse-biased avalanche photodiodes as we mentioned.

Previously, research has been done on the scintillation of the PMT input window by X-rays and radioactive sources\cite{bayat2014scintillation}, But this has not been done for SiPM. Here we investigate the materials one by one to see the amount of scintillation in PMT and SiPM structures, but our main focus is on the effect of cosmic muons as unwanted signals in PMT and SiPM in very weak optical detections, and we try to investigate the number and strength of these signals.

%%  \newpage
\section{Input Window of SiPM and light guides}
\label{sec:sipm}

\subsection{Scintillation of poly (methyl methacrylate) light guide and Epoxy resin in SiPM Window}
Common scintillator like BC408 has many characteristics. Those characteristics are optimized to create more photons when an energetic particle goes through the scintillator. The most important one is the wavelength spectrum of emitted photons and the spectrum of photon absorption\cite{Zhang:2020mqz, semiconductorc}. In materials such as simple glass, those two spectra overlap significantly. In the structure of scintillators, the absorption and emission spectra need to be separated to produce a maximum number of scintillation photons. \textcolor{black}{Typical quantum efficiency spectra of common types of PMTs show that the detection efficiency for PMT is between 300 to less than 700 nm\cite{Zhang:2020mqz}. Despite the utterly different structure of the PMT, the SiPM extended to the longer range of absorption. The effective wavelength range for SiPM (MicroFC-30035-SMT) is 300 to 950nm\cite{semiconductorc}.}

Fluorescence lifetime or decay time is the duration that the scintillation effect continues producing photons from the time that the energetic particle has gone through the material. The decay time of glass scintillators is around 10 to 15 microseconds\cite{fujimoto2015photoluminescence,kroning2002x}, while the decay time for a plastic scintillator or similar materials is around 2 to 8 nanoseconds\cite{kroning2002x, moser1993principles, chen2022broadband}.

Epoxy resin is a rigid and transparent compound that is used to coat electronic components such as photoreceptors and LEDs, therefore the optical characteristics are crucial. The resin contains three basic compounds: Base resin, Hardener, and Catalyzer. This rigid polymer i.e. the resin which is suitable for the optical application is obtained by combining the three compounds at the proper temperature. Each compound before combining has special optical characteristics, and after combining at proper temperature, polymerization, and hardening occur, and the produced polymer has new optical characteristics. \textcolor{black}{For a specific combination, the resin obtains an emission wavelength longer than 300 nm and absorption spectra between 375 to 650 nm.} The decay time for this resin is significantly high, and it is around 0.9 seconds\cite{gallot2005identification}.

Poly (methyl methacrylate), or PMMA known as acrylic and acrylic glass, is called engineering plastic and is a transparent thermoplastic. This PMMA can be precisely cut by infrared (IR) lasers. This type of plastic is also used as a light guide in the detector study which contains SiPM. Although the glass is widely used as a light guide, the decay time of the scintillation is very long compared to the PMMA plastic, around 5 to 10 nanoseconds \cite{ghazy2020preparation}. \textcolor{black}{Comparing the emission and absorption spectra for PMMA it can be seen that the spectra have a large overlap. So, the produced photons by the passage of an energetic particle, cannot travel more than one centimeter inside it. The peak of the emission spectrum for this kind of PMMA is around 590 nm\cite{ghazy2020preparation}. So it is clear that undoped PMMA, epoxy resin, and glass luminance efficiencies are very low, and are not transparent to the self-emission over any significant distance (around 1 cm)\cite{moser1993principles}. However, their emission spectrum is a good match to the common PMTs and SiPMs. Despite the low scintillation, as they are used in the structure of PMT, SiPM, and light guides, these materials have an effective factor in producing unwanted signals in the weak radiation region by cosmic muons.}

\subsection{The origin of scintillation in the SiPM}
     The window on top of the SiPM sensor is mainly made of epoxy resin instead of glass\cite{MICRORBSERIESD}. For the SiPM MicroFC-60035-SMT, the size of the epoxy resin is about $7\times 7$ mm, and the thickness is 0.21 mm\cite{semiconductorc}. With these dimensions, the emitted photons in the resin have a possibility of reaching the surface of the photodetector cells of the SiPM. It is likely that the mean free path of those photons reaching the photodetector is less than 1 cm, and as a result, they are not absorbed along the path\cite{moser1993principles}. 


\section{Input Window and Photocathode of PMT}
\label{sec:pmt}

\subsection{Photocathode of PMT}

Most side-type photomultipliers use a reflection-mode photocathode and a circular structure electron multiplier with high sensitivity and good amplification at a relatively low supply voltage. The head-on type or end-on type has a semi-transparent photocathode, called transmission photocathode, placed on the inner surface of the input window\cite{photonics1998photomultiplier}. 

Most photocathodes are made of compound semiconductors composed of low-work function alkali metals. The spectral response of the photocathode is expressed according to the type of material. Most photocathodes are highly sensitive to ultraviolet photons. But since UV rays tend to be absorbed by the window material, the short wavelength limit is determined by the UV transmission of the window material\cite{hamamatsu2007photomultiplier}.



 \subsection{Input window materials of PMT}
  \textcolor{black}{The window materials commonly used in photomultiplier tubes are as follows\cite{photonics1998photomultiplier}:}  

 \subsubsection{MgF\textsubscript{2} crystal}
\textcolor{black}{Alkali halide crystals are very good in UV transmission. A magnesium fluoride (MgF\textsubscript{2}) crystal allows the transmission of ultraviolet radiation down to 115 nanometers. The emission spectra of MgF\textsubscript{2} crystal samples have significant values around 300 to 500 nm. Afterglow or fluorescence lifetime or decay time is the duration that the scintillation effect continues producing photons from the time that the energetic particle has gone through the material. The afterglow profile of the MgF\textsubscript{2} crystal sample is less than 40 ms\cite{nakamura2017scintillation}. The wavelength transmittance of MgF\textsubscript{2} crystal is higher than 112 nm\cite{hamamatsu2007photomultiplier}.}


 \subsubsection{Sapphire}
\textcolor{black}{Sapphire is made of Al\textsubscript{2}O\textsubscript{3} crystal and its transmittance in the ultraviolet region is between the UV-transmitting glass and synthetic silica. The emission spectra of Al\textsubscript{2}O\textsubscript{3} crystal samples have significant values around 250 to 450 nm. The emission decay time of the Al\textsubscript{2}O\textsubscript{3} crystal sample is in the range of 100 ns\cite{futami2014optical}. The wavelength transmittance of Sapphire is higher than 144 nm\cite{hamamatsu2007photomultiplier}.}


 \subsubsection{Synthetic silica and Quartz}
\textcolor{black}{Synthetic silica transmits UV radiation up to 160 nm and offers lower absorption levels in the UV region compared to fused silica. Fused Silica is a synthetic material, and is the purest glass. Synthetic amorphous silica is manufactured form of silicon dioxide (SiO\textsubscript{2}). It has been produced for a long time without significant changes\cite{croissant2020synthetic}. Quartz is a composition very close to 100 percent of  SiO\textsubscript{2}. Quartz can record the amount of ionizing radiation it is exposed to as a latent signal in its crystal lattice. Quartz crystal has a good transmission efficiency from 160 to 2500 nm. The emission spectrum of quartz crystal samples can be between 300 to 800 nm and the decay time of photoluminescence of quartz crystal is around 200 ms\cite{preusser2009quartz}.}

\textcolor{black}{The glass with its composition and impurity has particular emission and absorption spectrum. Even the effect of the temperature and duration of heat-treating during production changes the shape of the spectra. Both spectra have an overlap, and the result of the overlap indicates that the photon absorption occurs in a short path. The length of this path depends on the structure of the material such as quantum effects and the geometry, and it is around one centimeter or less\cite{ghazy2020preparation}. The decay time of glass scintillator is around 10 to 15 microseconds\cite{fujimoto2015photoluminescence,kroning2002x,ghazy2020preparation}. The wavelength transmittance of Synthetic silica is higher than 158 nm\cite{hamamatsu2007photomultiplier}.}


 \subsubsection{Borosilicate glass}
\textcolor{black}{This is the most widely used window material. Because borosilicate glass has a coefficient of thermal expansion very close to the Kovar alloy used for photomultiplier tube leads, it is often referred to as "Kovar glass". Borosilicate glass does not transmit UV radiation below 300 nm. It is not suitable for detecting ultraviolet rays shorter than this wavelength. In addition, some types of bi-alkaline photocathode tubes use special borosilicate glass (so-called "K-free glass") containing a very small amount of potassium (K40), which may cause unwanted background signals. K-free glass is used primarily for photomultiplier tubes designed for scintillation counting, where low background counts are desired. Emission spectra of Borosilicate glass are in the range of 300 to 600 nm\cite{ehrt2009photoluminescence}. Borosilicate glass photo-luminance decay time is in the range of tens milliseconds\cite{elkhoshkhany2021investigation}. The wavelength transmittance of Borosilicate glass is higher than 300 nm\cite{hamamatsu2007photomultiplier}.}
 

%%   \newpage
\section{Discussion}
\label{sec:Discussion}

\textcolor{black}{Photon counting is an effective method to use a photomultiplier tube to measure very low light radiation. It is often used in chemiluminescence and astronomical photometry or bioluminescence measurement in case the intensity of the light is as low as the incident photons to be separated. This state is called a single photon or photoelectron event. The number of output pulses is directly proportional to the amount of incident light. Since the output of the photomultiplier tube contains a variety of noise pulses along with the original signal pulses, simply counting the pulses without removing the noise or removing the noise-related counting will not result in accurate measurements. The most effective way to remove noise is to check the height of the output pulses\cite{photonics1998photomultiplier}.}
 
As a result of the passage of high-energy photons and high-energy charged particles such as cosmic muons, they create photons based on the property of photoluminescence (scintillation) in the materials located in the Input Window.  \textcolor{black}{The amount of this property was shown to some extent in the section~\ref{sec:pmt}and~\ref{sec:sipm} for each practical material in PMT and SiPM input window.} Another important property is the decay time for photoluminescence. In some applications, this glow needs to end quickly after the primary particle has passed, but for some other applications, it needs to remain more. The original photon to be detected by the PMT and SiPM must be distinguishable from the photons created in the Input Window. If the studied light beam is irradiated for a time longer than microseconds, it is better to use the one that the effective photoluminescence property ends in a short time of about tens of nanoseconds, and in this way, it is easy to separate the background signal resulting from the photoluminescence property in the input window. But to study rays in a short time or very weak radiations, it is better if the property of photoluminescence has a decay longer than a few hundred nanoseconds. This has a significant effect on the height of the background signal and makes it shorter.


In addition to these, another very basic property is Absorption. If the spectrum of photoluminescence and absorption overlap and the glows caused by photoluminescence are quickly absorbed, of course, in this situation the length of the path these photons travel before being absorbed is also important. For the PMT structure, the mentioned photons may not need to pass a path more than one millimeter to reach the photocathode, and this path is usually not a sufficient path to absorb the photon caused by the photoluminescence property in the materials \textcolor{black}{mentioned in section~\ref{sec:pmt}\cite{ghazy2020preparation}}. Because of the smaller size of SiPM compared to PMT, the photon that creates in the input window may have more chance to reach the photodiode's surface.

Due to the large size of the scintillator compared to the epoxy resin in SiPM, the scintillation of this part may not be vital to most detectors. The size of the light guide is usually comparable to the size of the scintillator even though its scintillating is much less, and the emitted photons from the light guide are absorbed in a short path, while the photons produced in scintillators are not. This value for BC408 is about 3 meters. The emission and absorption spectrum of SiPM input window materials have much more overlap than materials in PMT, but the small size of this input window gives enough chance for photons created in the resin to arrive at the photodiode surface. So, in detecting the small number of photons on the earth's surface by SiPM, the scintillation effect of the light guide and epoxy resin in SiPM due to the passage of energetic cosmic muons is significant.

PMT input window material MgF\textsubscript{2} crystal uses with Cs-I and CS-Te photocathode, and this set is suitable for detection of light in the region of 115 to 200 nm and 320 nm respectively \cite{hamamatsu2007photomultiplier}. MgF\textsubscript{2} crystal has a pick of luminesce around 450 nm, and it is transparent to this wavelength, so this set will be affected by cosmic muon easily. The luminance decay time for MgF\textsubscript{2} crystal is tens of milliseconds, and this is great that the energy of luminance is not going out of MgF\textsubscript{2} crystal in a short time, so the height of the background signal due to the passage of muon from input window will be less.

In the case of quartz, types of glass, and synthetic silica, the decay time is in the range of 100 milliseconds. These materials are generally transparent to wavelengths higher than 160 nanometers to more than 1500 nanometers. What is more decisive is their luminance, which has a different spectrum according to the degree of purity, crystal structure, type, and amount of doped materials. Even in the case of glass and silica, heating has a large effect on the spectrum. Most of these materials have a peak luminance of around 400 to 600 nanometers, and due to their transparency to this region, and long decay time, they are affected by cosmic muons and generate significant background signals due to the passage of muons.

Borosilicate glass is transparent for wavelengths higher than 300 nanometers and the luminance has a peak around 390 nanometers. Therefore, it is expected that Borosilicate glass will create a significant background signal due to the passage of cosmic muons too. This signal has a length of time in the range of the decay time, that is, several tens of microseconds.

As a result, it seems that all types of PMT are sensitive to the passage of cosmic muons and it is necessary to adopt a workaround to separate the background signal caused by the passage of cosmic muons from the main signals. Ref.\cite{photonics1998photomultiplier} shows that the height of the signals caused by the passage of muons is high on average, but definitely, these signals may have a lower height, so they can be easily separated from the dark current background signal, but their separation from the studied signal is only possible if, from the point of view of the rate, the studied signal is much more than the number of resulting signals from cosmic muons. On the surface of the earth and in every square centimeter, one cosmic muon with an average energy of 4 Giga-electron volts passes on average every second, generally vertically and towards the earth\cite{AUTRAN201877}. If the rate of the studied signal is in the range of the background signal, it seems that the basic way to reduce the background signal is to conduct experiments deep in the ground or inside tunnels where the height of the material above it is several tens of meters. An experiment has been conducted to count cosmic muons inside a tunnel, which shows that the number of cosmic muons counted in the tunnel has decreased drastically. \textcolor{black}{Figure~\ref{fig:tunnel} shows the result\cite{Raki:2022lwn}. From the comparison of the graphs at 50 meters inside the tunnel, which is about 5 meters of rock above it, the number of counted muons decreases from about 950 to 50 muons in 10 minutes. Of course, in the middle of the tunnel, where there are about 60 meters of rock above it, the number of muons count reaches less than 5 muons every 10 minutes. According to these results, it seems that the reduction of the destructive effect of cosmic muons in the detection of low radiations and single photon detection can be significant in the basements of tall buildings, and such experiments should be performed at least in such places.}

\begin{figure}[ht!]
\centering
\begin{subfigure}{0.47\textwidth}
\includegraphics[width=1\textwidth]{tunnel_a.jpg} 
\caption{Muon count in a tunnel with 550m long each count has done in 10 minutes separated by 10 meters.}
\label{fig:tunnel_a}
\end{subfigure}
\hspace*{2mm}
\begin{subfigure}{0.47\textwidth}
\includegraphics[width=1\textwidth]{tunnel_b.jpg}
\caption{Curve of the height of matter in the top of the tunnel.}
\label{fig:tunnel_b}
\end{subfigure}
\vspace*{-3mm}
\caption{The experiment of cosmic muon count in a tunnel\cite{Raki:2022lwn}.}
\label{fig:tunnel}
\end{figure}
 
We must emphasize that if PMT is used together with a standard scintillator such as BC408 plastic scintillator, then the number of signals caused by cosmic muons in the scintillator is so much greater and stronger than the background signal created in the input window of PMT and this background can be ignored.

\section{Simulations}
\label{sec:Simulations}

\subsection{The simulation of a plastic scintillation}
The Gate is a Monte Carlo simulator package with a core of Geant4\cite{Gate8.0}. In this package, we chose a geometry similar to a photodetector attached to a PMT, a $40\times 40 \times 10$ mm light guide with different materials: BC408, glass, epoxy resin, and PMMA simulated separately. The PMT is attached to the $40\times40$ mm side of the light guide, and the muon source is to the opposite side. For the radiation source, we used the energy spectrum equivalent to the muon radiation at sea level\cite{AUTRAN201877}. Muon particle with its physics lists and processes pre-identified in the Gate. However, for the scintillation processes, it is only provided for certain materials which have enough large scintillation. Therefore by editing "Materials.xml", containing the characteristic of materials given by the Gate, we changed the characteristics of the scintillation for the BC408 to the values related to glass, epoxy resin, and PMMA\cite{Gate8.0}. The main parameters to be varied based on section~\ref{sec:pmt} and ~\ref{sec:sipm}\cite{agostinelli2003geant4, Gate8.0}:

\begin{itemize}
\item{Light Yield (Scintillation Yield)}: Is the number of photons normalized per 1 MeV energy deposited in the scintillator as photons by the trajectory of the energetic particle\cite{Gate8.0}, in the unit of Photon/MeV. The average light yield has a non-linear dependence on the local energy deposition. Although the light yield of the BC408 is 10000 photon/MeV, in the Gate simulation one could set the value to 1000, because of: (1) to approximate the efficiency of the PMT, and (2) to dramatically increase the speed of the code because there were fewer optical tracks to calculate. In this study, we set the value of the light yield to 1 Photon/MeV for glass, PMMA, and epoxy \textcolor{black}{as it is the least meaningful value\cite{Gate8.0}}.

\item{Fast Time Constant: Scintillators have a time distribution spectrum with one or more exponential decay time constants, with each decay component having its intrinsic photon emission spectrum\cite{Gate8.0}. Based on section~\ref{sec:pmt} and ~\ref{sec:sipm}, the decay time for the BC408 is 2.1 ns, for the Glass is 0.9 s, and for the PMMA and the epoxy is almost the same as BC408.

\item{Fast Component}: This is a histogram that sets the rate of emitting photons at each energy level. This should depend on the optical properties of the material\cite{Gate8.0}. Based on section~\ref{sec:pmt} and ~\ref{sec:sipm}, the histogram shifts to the higher energies for PMMA, Glass, and epoxy compared with the BC408. This means the emission of PMMA, Glass, and epoxy has higher energy than BC408}.

\item{ABSLength}: The path length of the photon travels in the matter before the absorption\cite{Gate8.0}. Based on section~\ref{sec:pmt} and ~\ref{sec:sipm},  in BC408, for emitting photons produced in the scintillation process, the parameter is set to 3.8 m. For PMMA and epoxy, it is mostly set to less than 1 cm, and for some other wavelengths, it is to several centimeters.
\end{itemize}

Some photodetector surfaces such as PMT are already defined in the Gate, however, for the SiPM depending on the wavelength, it can be adjusted based on the specifications datasheet of the SiPM. According to the datasheet of the PMT, the rate of efficiency for any wavelength can be adjusted as a histogram in the "Surface.xml" file of the Gate package. But Gate8 and Geant4 haven't prepared simulations suitable for the output of SiPM and even PMT, because simulation can only show the energy and number of photos and total energy of photon bunch that arrives at the photodetector surface in each event. Because of the nature of signal production in SiPM by reverse-biased avalanche photodiodes, it can be concluded that the height of the signal caused by weak radiation depends more on the number of stimulated cells and a cell can be stimulated even with one photon. In this way, weak radiations can stimulate a significant number of SiPM cells, and finally, by creating a suitable bias, a signal with a height similar to that of much stronger radiations can be obtained\cite{Raki:2022lwn, AND9770/D}. In PMT the number of photoelectrons, which is equivalent to the height of the signal, is proportional to the number and energy of incident photons\cite{hamamatsu2007photomultiplier}. So the result of the simulation here is more likely to be the PMT output signals than SiPM signals.

The results of the simulation for the number of events done by the trajectory of cosmic muons versus scintillation photons bunch energy in the Light guide that is made of PMMA, epoxy resin, and glass, detected by PMT are shown in Fig.~\ref{fig:simulation}. Here, the input window material is designed as a light guide and there is no separated input window provided in front of PMT in this simulation.


\begin{figure}[ht!]
\centering
\begin{subfigure}{0.25\textwidth}
\includegraphics[width=1\textwidth]{simulation_n.jpg} 
\caption{Decimal energy axis.}
\label{fig:tunnel_a}
\end{subfigure}
\hspace*{2mm}
\begin{subfigure}{0.7\textwidth}
\includegraphics[width=1\textwidth]{simulation_l.jpg}
\caption{Logarithmic energy axis.}
\label{fig:tunnel_b}
\end{subfigure}
\caption{The results of the simulation for the number of events done by trajectory of cosmic muons versus scintillation photons bunch energy in the Light guide that is made of PMMA, epoxy resin and glass, detected by PMT.}
\label{fig:simulation}
\end{figure}

  \newpage
\section{Experiments}
\label{sec:Experiments}

\subsection{The experiment of cosmic muon signals in the input window of PMT Hamamatsu R580}
R580 is a PMT for scintillation counting and high energy physics, 38 mm (1-1/2 inch) diameter, 10-stage, bialkali photocathode, head-on type. The input window material is Borosilicate glass. In this experiment, we used two similar PMTs, one in the bottom connected to a BC408 plastic scintillator, and the other PMT is not. Between the two detectors there was an iron plate with a thickness of 10 mm to confirm that coincidence signals that may come from two detectors should be caused by a cosmic muon. Two detectors were separated by around 10 cm in height. Figure~\ref{fig:PMT_ex} shows the experiment setup.

\begin{figure}[ht!]
  \centering
  \includegraphics[width=0.4\textwidth]{PMT_ex.jpg}
\vspace*{1mm}
\caption{Setup of the experiment of cosmic muon signals in input window of PMT Hamamatsu R580.}
\label{fig:PMT_ex}
\end{figure}

This test was done on the ground and in a shed. Coincidence signals caused by cosmic muons are two signals that come from two detectors almost at the same time and with a difference of less than 100ns in this setup.  Coincidence signals confirm that a cosmic muon has passed through both detectors. With such a setup, a set of simultaneous signals could be obtained approximately every minute, but by increasing the trigger level, which was set on the signals coming from the above PMT, the number of simultaneous signals was approximately one every five minutes. In this way, it can be seen that on the surface of the earth, a PMT of this type creates a significant background signal caused by cosmic muons on average every minute. The results obtained from two detectors are shown in Fig.~\ref{fig:PMT_signal}.

\begin{figure}[ht!]
  \centering
  \includegraphics[width=0.9\textwidth]{PMT_signal.jpg}
\vspace*{-1mm}
\caption{A sample set of Simultaneous signals made by cosmic muon in two detectors in Fig.~\ref{fig:PMT_alone} setup.}
\label{fig:PMT_signal}
\end{figure}

    
  
  \newpage
\subsection{Cosmic muon effect on PMMA light guide and SiPM input window}

This test also was done on the ground and in a shed. We performed a practical scintillation test of the light guide and the SiPM upper window made of epoxy resin using cosmic muons. The flux of cosmic muons on the earth's surface is about one muon per square centimeter per second\cite{Raki:2022lwn}. Each simultaneous signal is obtained for the upper and lower detectors approximately every sixty seconds. We also measured the number of simultaneous photons due to the effect of muons by the SiPM with no light guide attached to the lower detector. The result was much less than the possibility of receiving every five minutes. Figure~\ref{fig:SiPME} shows the SiPM optical structure of the detector and a light guide made of PMMA in this test\cite{Raki:2022lwn}. 

Figure~\ref{fig:SiPME}.c shows the test setup with a SiPM detector on the top attached to a light guide, BC408, and another detector on the bottom with and without a light guide. When the trig moves slowly away from zero value in the oscilloscope, the noise containing the thermal signals in the SiPM is less visible. The simultaneity of the signals obtained by the two detectors confirms a coincidence between them when muons pass through. These same signals confirm that detectors with SiPM, with and without a light guide, generate relatively strong signals compared to thermal noises due to the passage of cosmic muons. A sample of synchronous signals for the case where the lower detector has a light guide is shown in Fig.~\ref{fig:SiPM_LGuid} and has no light guide in Fig.~\ref{fig:SiPM_NLG}. The size of the light guide in this detector is equal to the size of the input window of PMT Hamamatsu R580 to make the entrance of these two photodetectors in the same size.

\begin{figure}[ht!]
\centering
\begin{subfigure}{0.3\textwidth}
\includegraphics[width=1\textwidth]{SiPME_a.jpg} 
\caption{SiPM detector.}
\label{fig:SiPME_a}
\end{subfigure}
\hspace*{2mm}
\begin{subfigure}{0.3\textwidth}
\includegraphics[width=1\textwidth]{SiPME_b.jpg}
\caption{SiPM and one layer of light guide made of PMMA\cite{Raki:2022lwn}. }
\label{fig:SiPME_b}
\end{subfigure}
\begin{subfigure}{0.2\textwidth}
\includegraphics[width=1\textwidth]{SiPME_c.jpg}
\caption{The structure of the experiment contains one SiPM detector on top that has a light guide and BC408 scintillator and another detector on the bottom, with and without a light guide.}
\label{fig:SiPME_c}
\end{subfigure}
\caption{}
\label{fig:SiPME}
\end{figure}

\begin{figure}[ht!]
  \centering
  \includegraphics[width=0.6\textwidth]{SiPM_LGuide.jpg}
\vspace*{1mm}
\caption{A sample of synchronous signals for the case where the bottom detector (blue line) has a light guide. The top detector (red line) contains light guide and a BC-408 scintillator.}
\label{fig:SiPM_LGuid}
\end{figure}

\begin{figure}[ht!]
  \centering
  \includegraphics[width=0.6\textwidth]{SiPM_NLG.jpg}
\vspace*{1mm}
\caption{An example of the synchronous signal for the case where the bottom detector (blue line) has no light guide and no scintillation source except SiPM itself. The top detector (red line) contains a light guide and a BC-408 scintillator.}
\label{fig:SiPM_NLG}
\end{figure}


  \newpage
\section{Discussion and Conclusion}
\label{sec:conclusions}

The structure of commonly used PMTs and SiPMs was reviewed and according to the optical characteristics of the materials used in their entry window. As a result, it seems that all types of PMT and SiPM are sensitive to the passage of cosmic muons and it is necessary to adopt a workaround to separate the background signal caused by the passage of cosmic muons from the main signals. This background can be easily separated from the dark current signals, but their separation from the studied signal is only possible if, from the point of view of the rate, the studied signal is much more than the number of resulting signals from cosmic muons. For single photon detection or very low radiation detection by PMT and SiPM, the effect of muon background signals is challenging, and the most useful method to decrease in effect is to conduct experiments deep in the ground or inside tunnels where the height of the material above it is several tens of meters. Background signals caused by cosmic muons in SiPM are usually is more strong than PMT, but the smaller size of the SiPM input window could help us to get less trajectory of cosmic muon through the input window.


%\vspace*{-1mm}
\section*{Acknowledgments}
%\vspace*{-1mm}
%
GFR and MK are especially grateful to Dr. Mohammadi Najafabadi from the School of Particles and Accelerators at IPM for the continual interest shown in the project, and the financial support he provided. 
%
%
%\appendix

%% \vfill


{%% \footnotesize
%\linespread{0}\selectfont                                                                                                                    
\bibliographystyle{utphys}
\bibliography{submit.bib}
}

\end{document}
