\begin{figure*}[tbhp!]
\begin{center} 
\begin{tikzpicture}
  \matrix[anchor=center,row sep=0.3cm,column sep=0.35cm,every node/.style={draw,very thick,circle,minimum width=2.5pt,fill,inner sep=0pt,outer sep=2pt}] at (0,0) {
                 &                  &              & & & & &              & \node(abc){};
    \\
    \\
    \\
                 & \node(acd){};   
    \\
    \\
                 &                  &              & & & & &              &               & \node(a2){};
    \\
                 &                  &              & & & & & \node(a1){};
    \\
    \node(c1){}; &                  & \node(c2){}; & & & & &              &               &               & & & & & & & \node(abcd-2){};
    \\
    \\
    \\
    \\
    \\
                 & \node(abcd-1){}; & 
    \\
                 &                  &              & & & & &              &               &               & & & & & & & \node(e){};
    \\
    \\
    \\
    \\
    \\
                 & \node(f){}; 
    \\
  };
  \draw[<-,very thick,color=chocolate,>=latex](abc) -- ++ (90:0.5cm););
  %
  \path (abc) ++ (0.1cm,0.35cm) node[right]{$\tightfbox{$abc$} = \avert$};
  \draw[->,very thick,royalblue] (abc) to node[right]{$\hspace*{-0.2em}\black{a}$} (a1);
  \draw[->,very thick,royalblue] (abc) to node[right]{$\hspace*{-0.2em}\black{a}$} (a2);
  \draw[->,very thick,royalblue,shorten >=6pt] (abc) to node[left,pos=0.3]{$\hspace*{0.2em}\black{c}$} (c1); 
  \draw[->,very thick,royalblue,shorten >=3pt] (abc) to node[right,pos=0.4,xshift=-0.15em]{$\black{c}$} (c2);
  \draw[->,very thick,royalblue,shorten >=4pt] (abc) to node[right,pos=0.85]{$\black{b}$} (f);
  % 
  \path (abc) ++ (-4.5cm,0.5cm) node{\Large$\aonechart\:/\:\aonecharthat$};
  %
%   % Mark the angle XAY
%   \begin{scope}
%     \path[clip] (A) -- (X) -- (Y);
%     \fill[red, opacity=0.5, draw=black] (A) circle (5mm);
%     \node at ($(A)+(30:7mm)$) {$\theta$};
%   \end{scope}
  %\pic[draw=firebrick, text=blue, "$\loopnsteplab{1}$" ,thick]{angle=a1--abc--c2};
  %\pic [draw=red, text=blue, <->, "$\theta$", angle eccentricity=1.5] {angle = mary--origo--bob};
  %
  \path (acd) ++ (0cm,0.4cm) node{$\tightfbox{$acd$} = \bvertbari{1}$};
  \draw[->] (acd) to node[above,pos=0.225]{$a$} (a2);
  \draw[->] (acd) to node[below,%near start,yshift=0.6mm,xshift=0.3mm
                                pos=0.68]{$a$} (a1);
%   \begin{scope}
%     \path[draw,clip] (acd) -- (c1) -- (c2) -- cycle;
%     \fill[red, opacity=0.5, draw=black] (acd) circle (5mm);
%     %\node at ($(A)+(30:7mm)$) {$\theta$};
%   \end{scope}
  
  
  
  \draw[->,forestgreen,very thick] (acd) to node[left]{$\black{c}\hspace*{-0.23em}$} (c1);
  \draw[->,forestgreen,very thick] (acd) to node[left]{$\black{c}\hspace*{-0.15em}$} (c2);
  \draw[->,bend right,relative=false,out=-45,in=167.5,looseness=1] (acd) to node[above,pos=0.8]{$d$} (e);
  %
  \path (a1) ++ (0.1cm,-0.375cm) node{\tightfbox{$a_1$}};
  \draw[->,shorten >=3pt] (a1) to node[below]{$a_1$} (abcd-2);
  %\draw[->,bend left,distance=0.75cm,out=110] (a1) to (abc);
  \draw[->,bend left,distance=2.9cm,relative=false,out=0,in=-20,shorten >=4pt] (a1) to node[right]{$a_1\hspace*{-0.18em}$} (abc);
  %
  \path (a2) ++ (0.32cm,0.4cm) node{\tightfbox{$a_2$}};
  \draw[->] (a2) to node[above,near end]{$a_2$} (abcd-2);
  \draw[->,bend left,distance=1.65cm,relative=false,out=-5,in=-20,shorten >=4pt] (a2) to node[left,pos=0.65]{$a_2$} (abc);
  %\draw[->,bend left,distance=2cm,out=80,in=120] (a2) to (abc);
  %
  \path (c1) ++ (-0.35cm,0.25cm) node{\tightfbox{$c_1$}};
  \draw[->,bend left,relative=false,out=200,in=180,looseness=1.9] (c1) to node[pos=0.55,right]{$c_1\hspace*{1.5em}$} (acd);
  \draw[->] (c1) to node[right]{$\hspace*{-0.18em}c_1$} (abcd-1);
  %\draw[->,bend right,relative=false,out=-30,in=0,looseness=2] (c1) ..controls ($(c2) + (1cm,-1cm)$) .. (acd);
  %
  \path (c2) ++ (0.35cm,-0.25cm) node{\tightfbox{$c_2$}};
  \draw[->] (c2) to node[right]{$\hspace*{-0.21em}c_2$} (abcd-1);
  \draw[->,out=220,in=180,looseness=3.75] (c2) to node[pos=0.275,below]{$c_2$\hspace*{-0.4em}} (acd);
  %\draw[->] (c2) .. controls ($(c2) + (,)$) and ($(acd) + ()$) .. (acd);
  %
  \path (abcd-2) ++ (0.05cm,0cm) node[right]{$\tightfbox{$abcd_2$} \parbox[t]{\widthof{${} = \bvertbari{2}$}}
                                                                             {${} = \bvertbari{2}$
                                                                              \\
                                                                              ${} = \bverti{2}$}$};
  \draw[->] (abcd-2) to node[right]{$d$} (e);
  \draw[->,densely dotted,thick,bend right,distance=1.25cm,relative=false,out=90,in=0] (abcd-2) to (abc);
  %
  \path (abcd-1) ++ (0cm,-0.3cm) node[left]{$\bverti{1} = \tightfbox{$abcd_1$}$};
  \draw[->] (abcd-1) to node[left]{$b\hspace*{-0.1em}$} (f);
  \draw[->,densely dotted,thick,bend left,relative=false,out=180,in=180,looseness=2] (abcd-1) to (acd);
  %\draw[->,densely dotted] (abcd-2) .. controls ($(abcd-2) + (1.75cm,0.5cm)$) and ($(acd) + (1.75cm,-0.5cm)$) .. (acd);
  %
  \path (e) ++ (0cm,-0.4cm) node{\tightfbox{$e$}};
  % \draw[->] (e) .. controls ($(abcd-2) + (-2cm,0.5cm)$) and ($(abc) + (-1.75cm,-0.25cm)$) .. (abc);
  \draw[->,relative=false,out=12.5,in=0,distance=4.75cm] (e) to node[right]{$e$} (abc);
  %
  \path (f) ++ (0cm,-0.45cm) node{\tightfbox{$f$}};
  %\path (f) ++ (0.35cm,0.4cm) node{\tightfbox{$f$}};
  \draw[->,relative=false,out=180,in=180,distance=3.5cm] (f) to node[left]{$f$} (acd);
  
  
  % bisimulation
  \draw[densely dashed,very thick,magenta,bend right,distance=1.5cm] (abcd-1) to (abcd-2);
  %
\end{tikzpicture}\end{center}
  %\vspace*{-4ex}
  \caption{\label{fig:ex:limit:collapse:1}%
    Example of a \protect\onechart~$\protect\aonechart$ with \protect\LLEEwitness~$\protect\aonecharthat$
    for which LLEE preserving collapse is unclear.
    The \protect\LLEEwitness~$\protect\aonecharthat$ is indicated with colored \protect\loopentrytransitions\
    from the `\protect\perpetualloop' vertex $\protect\avert$ (of level 2, blue)
    and from the vertex $\protect\bvertbari{1}$ (of~level~1,~green).
    The dashed magenta line indicates that $\protect\bverti{1}$ and $\protect\bverti{2}$ are \protect\onebisimilar.
%     The reflexive closure of the dashed magenta link
%     that connects the bisimilar vertices $\protect\bverti{1}$ and $\protect\bverti{2}$ 
%     defines a self-\protect\onebisimulation\ on $\aonechart$. 
%     The bisimilarity redundancy $\pair{\bverti{1}}{\bverti{2}}$ is reduced: it
%     satisfies the position condition \ref{pos:3.5} in Lemma~\ref{lem:reduced:br},
%     a situation for which we have not found a \protect\LLEEwitnessed\ transformation that collapses $\bverti{1}$ and $\bverti{2}$.
    }
\end{figure*}
