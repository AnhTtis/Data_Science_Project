Axiomatization and expressibility problems
for Milner's process semantics (1984) of regular expressions modulo bisimilarity
have turned out to be difficult for the full class of expressions with deadlock~0 and empty step~1.
We report on a phenomenon that arises from the added presence of 1 when 0 is available,
and that brings a crucial reason for this difficulty into focus.
To wit,
  while interpretations of \mbox{1-free} regular expressions are closed under bisimulation collapse,
  this is not the case for the interpretations of~arbitrary~regular~expressions. 

Process graph interpretations of 1-free regular expressions 
satisfy the loop existence and elimination property LEE,
which is preserved under bisimulation collapse. 
These features of LEE were applied for showing that 
an equational proof system for \mbox{1-free} regular expressions modulo bisimilarity is complete, 
and that it is decidable in polynomial time whether a process graph is bisimilar to the interpretation of a 1-free regular expression.
% and that bisimilarity of a process graph to the interpretation of some 1-free regular expression is decidable in polynomial time. 

While interpretations of regular expressions do not satisfy the property LEE in general,
we show that LEE can be recovered by refined interpretations as graphs with 1\nb-tran\-si\-tions %, % purely 
                                  % for refined interpretations with 1\nb-tran\-si\-tions,
(which are similar to silent steps for %\finitestate\ 
                                       automata).
This suggests that LEE can be expedient also for the general axiomatization and expressibility problems. % in general.
But a new phenomenon emerges that needs to be addressed:
the property of a process graph 
  `to can be refined into a process graph with 1\nb-tran\-si\-tions\ and with LEE'
is not preserved under bisimulation collapse. 
We provide a 10\nb-ver\-tex graph with two 1\nb-tran\-si\-tions that satisfies LEE,
and in which a pair of bisimilar vertices cannot be collapsed on to each other while preserving the refinement property. 
This implies that the image of the process interpretation of regular expressions is not closed under bisimulation collapse.
% This example witnesses that the process interpretations are not closed under collapse. 


% But a phenomenon arises that starkly contrasts with the situation for 1-free expressions:
% the property of a process graph 
%   `to can be refined into a process graph with 1-tran\-si\-tions and with LEE'
% is \ul{not} preserved under bisimulation collapse. 
% We give a counterexample with only two bisimilar vertices that cannot be collapsed on to each other while preserving LEE. 
% It also shows that the process interpretations are not closed under bisimulation collapse. 




% For Milner's process semantics (1984) of regular expressions,
% axiomatization and expressibility problems have turned out to be difficult to settle 
% in the presence of the empty-step process~1. 
% We describe a phenomenon that brings a crucial reason for that difficulty to light,
% and suggests a way~to~tackle~it.

% This feature of LEE was used for showing that an equational axiomatization 
% of bisimilarity of process graphs denoted by 1-free regular expressions is complete, 
% and that expressibility of a process graph by a 1-free regular expression is decidable in polynomial time. 
% 




% We give a counterexample that arises from a specific case
% in which bisimilar vertices cannot be collapsed on to each other LEE-preservingly. 
% By eliminating all other bisimilarity redundancies 
% we formulate a procedure for obtaining,
% for every finite process graph with 1-tran\-si\-tions and LEE,
% such an approximation of its bisimulation collapse 
% that satisfies LEE and in which every vertex has at most~one~bisimilar~counterpart.






% Axiomatization and expressibility problems
% for Milner's process semantics (1984) of regular expressions modulo bisimilarity
% have turned out to be difficult for the full class of expressions with deadlock~0 and empty step~1.
% We report on a phenomenon that arises from the added presence of 1 when 0 is available,
% and that brings a crucial reason for this difficulty into focus.
% While interpretations of \mbox{1-free} regular expressions are (essentially) closed under bisimulation collapse,
% this is not the case for the interpretations of~arbitrary~regular expressions. 
% 
% Process graph interpretations of 1-free regular expressions 
% satisfy the loop existence and elimination property LEE,
% which is preserved under bisimulation collapse. 
% These features of LEE were applied for showing that 
% an equational proof system for 1-free regular expressions modulo bisimilarity is complete, 
% and that bisimilarity of a process graph to the interpretation of a 1-free regular expression is decidable in polynomial time. 
% 
% While interpretations of regular expressions do not satisfy LEE in general,
% we show that LEE can be recovered purely by refining the interpretations into process graphs with 1\nb-tran\-si\-tions,
% similar to silent steps for finite-state automata.
% This suggests that LEE can also be applied to the axiomatization and expressibility problems in general.
% But a new phenomenon has to be dealt with:
% the property of a process graph 
%   `to can be refined into a process graph with 1-tran\-si\-tions and with LEE'
% is \ul{not} preserved under bisimulation collapse. 
% We give a graph with only two bisimilar vertices that cannot be collapsed on to each other while preserving LEE. 
% This example also witnesses that the process interpretations are not closed under collapse. 





