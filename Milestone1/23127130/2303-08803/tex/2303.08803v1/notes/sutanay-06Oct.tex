Improve the title: % LW: I took a shot at this
- Do you want to add “heterogeneous computing” into the title (say in place of “multiple computing clusters”)?  The big benefit of the system is that you are enabling different computations/tasks to be run on different systems that are optimized for each variant. 
Make the title more specific/exciting?  Processing across different systems is going on for decades.  What is so special about your work?  Scheduling?  Overlapping data transfer/communication with computation? 
If you were to list one pain point in today’s HPC community that you take head-on, which one would that be?  Consider adding that into the title. 
Abstract:
“The key innovation enabling high performance is a subsystem … used to distribute task instructions.” – there is no free lunch.  What specific technical problems you had to solve to get that to work?
Consider providing a quantitative measure of your framework over a traditional approach in which either we co-locate all applications inside one data center/supercomputer and lose performance by mapping computations to non-optimal software/hardware, or we do distributed computation across sites with less optimal communication/coordination.  Which one you are solving, and there is got to be a metric to report.  “We show a reduction of X% in wall-clock time ….”, or “our optimized framework allows us to train more sophisticated models leading to X% improvement in accuracy”, or “discovering X% more optimized molecules”.   
    - That's something we should do. It's not going to be close, so we decided to use a harder baseline: 
Intro:
Paragraph 1 can be tightened.  While it is a matter of writing style, sentences that are direct and avoid a pause in form of dashes are better.  Be direct and pithy: “Simulation tasks may require systems optimized for tightly coupled computations that span multiple nodes— exactly the use case for supercomputers”.  -> “Supercomputers are optimized for tightly coupled computations spanning multiple nodes”.
Last line, paragraph 1: echoing the same comment from title.  “However, building multi-site workflows is difficult due to challenges with networking, data management, and co- scheduling of workloads.”  Where are you targeting most and making most difference?  Networking?  Data management?  Co-scheduling?
Section II:
You had nice visual diagrams in the slides before that showed tasks running on different systems.  Current figure 1 emphasizes too much about the “thinker-task server-worker”.  But you are doing a lot more.  Some of the workers are running simulations on Theta, and some are training a model on DGX2, and some could be running inference on either a CPU or GPU for active learning.  Bring it in – something like the attached slide.   But I believe you had pictures specific to Colmena to convey such messages. 
You can compress section II.C to make room for a bigger image/architecture diagram.
Section III:
This is a long section, and can be tedious for a reviewer to go through so much application details – especially if they don’t work on it.  Make it easy for the reviewer by adding a table where you summarize both applications.  For example, you can list the computation specifications (resources taken for simulation, learning, inference), average runtime, data volume/transfer.  Such a table would go a long way to show the general appeal and will allow others to think how to map into your framework.
Section IV:
Consider adding a table that summarizes the resource utilization/runtimes taken by a traditional implementation and yours.  Make it easy for the reviewer to click on “accept” by showing traditional implementation spends 380 ms for step A, we take 96 ms (page 5, column 2, paragraph below Fig 2) – yielding a nearly 75% reduction. 
Conclusion:
Repeating suggestions from title, abstract and introduction.  Highlight your specific contributions, and where possible, emphasize on quantitative metrics to showcase your work’s impact.
Again, make it easy for reviewers who are not familiar with your applications.  No smart reviewer likes to hit accept on something they don’t understand, but they also get irritated when they have to read through pages of details.  Shrink-wrap and hand over the reasons to accept your work.