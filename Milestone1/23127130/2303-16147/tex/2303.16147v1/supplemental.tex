\documentclass[preprint,12pt, amsmath,amssymb,]{elsarticle}
\usepackage{amssymb}
\usepackage{lscape}
\usepackage{booktabs}
\usepackage[flushleft]{threeparttable}
\usepackage{lineno}
\linenumbers
\usepackage{amsmath}
\usepackage{xcolor}
\journal{Sensor and Actuators A: Physical}

\begin{document}
\begin{frontmatter}
\title{Supplemental: Hysteresis Compensation in Temperature Response of Fiber Bragg Grating Thermometers Using Dynamic Regression}
\author[inst1]{Zeeshan Ahmed}
\affiliation[inst1]{organization={National Institute of Standards and Technology, Physical Measurement Laboratory, Sensor Science Division},            addressline={100 Bureau Drive}, 
            city={Gaithersburg},
            postcode={20899}, 
            state={MD},
            country={USA}}
           
\section{Methodology}
\subsection{Data Exploration}
% \label{sec:sample1}


The data exploration process began with examination of Pearson coefficient heatmaps (Fig SP1) and pairplots (Fig SP 2) of all continuous variables over the entire dataset and then parsed over each of the non-categorical variable. Furthermore, each feature’s temperature and time dependence was examined  separately to ascertain their behavior over time and temperature history. Data exploration indicates that against temperature the following feature show positive correlations: peak center, kurtosis, (fractional) peak width and grating type. No significant correlations were observed between temperature and the following features: time, amplitude, laser power and fringe visibility. Furthermore, we observe peak detuning (a feature created by subtracting the value of peak center value at 293.15 K at the start of measurements from all subsequent measurements) shows stronger correlation with temperature (0.91) as compared to peak center (0.77). 


The temporal evolution of each feature during thermal cycling was examined for each sensor.  Thermal time history and attendant changes in peak center and fractional peak width for sensors S6 and S7 are shown in Fig SP 3 and SP 4, respectively.   We note that the time histories of the two exemplar sensors show significant differences in the magnitude of changes, suggesting the hysteresis evolution is sensitive to the sensor’s history which may  include manufacturing conditions, thermal annealing and other variables.  Furthermore, as shown in Fig  SP 3 and SP 4, changes in spectra-derived features are slow and cumulative, rising above the prevailing measurement noise only over long time periods.  As shown in Fig SP 5, the Allen deviation (ADEV) plots for sensor S5 and S7 show that the measurement uncertainty at any given temperature is dominated by 1/f  noise as evidenced by a linear decrease in variance with integration time. Based on these results we conclude that peak center drift due to hysteresis either occurs outside the observation times used in this study\footnote{note that we allow an equilibriation time of upto 20 mins once the temperature bath reaches the set temperature to eliminate any thermal gradients in the sensor. Hysteretic changes maybe be occurring during this deadtime or over time scales much longer than the mintues to hours long observation times used here} or that the impact of these changes on measurement variance is smaller than the impact of other processes over the observation times used for each temperature measurement. Based on these results and previous literature\cite{Erdogan_decay,FBG_decay} we conclude that the process responsible for hysteresis is likely similar to slow random diffusion of a polymer on a rough energy landscape marred by shallow traps. During thermal annealing the grating appears to be dephasing- as suggested by changes in peak center and width- which suggests that during annealing some of the shallow traps are wiped out and their population transferred to deeper traps. As the effective bandgap of the material redshifts, the refractive index of the optical fiber increases resulting in an offset error of the temperature calibration. In our subsequent analysis we therefore treat the impact of hysteresis as being a cumulative effect of the temperature ramp. 

%We examined the thermo-physics of changes driving hysteresis in FBG by examining the Allen deviation (ADEV) plots to understand the sources of noise contributing to measurement uncertainty at any given temperature. As shown in Fig SP 3, the ADEV plots of different sensors (b) and at different temperatures (a) show a simple linear decrease in uncertainty indicating the measurement uncertainties are dominated by a 1/f noise. As shown in Fig 4 and 5, changes in spectral features (peak center, kurtosis, amplitude and fractional peak width) show a slow, cumulative effect that only becomes readily distinguishable over the measurement noise over long observation histories. 
%As shown in Fig 6a, we observe a small drift in peak center a 15 mins past the equilibration time. We attempted to gain an understanding of energy landscape by examining the rate kinetics of this small change. We begin by calculating the rate of change in peak center (or amplitude) once the sensor had equilibrated at a set temperature (Fig SP 6a). Using this time dependent measurements, we made Arrhenius plots (Ln(k) vs 1/T) when annealing data was available. As shown in Fig SP 5 (b-d) below, the kinetics profile is noisy and does not follow simple Arrhenius kinetics, although we do consistently observe that any temperature the rate constant is always elevated after a thermal cycle. 
   

%\begin{figure}[htbp!]
%\centering\includegraphics[width=12 cm]{kinetics}
%\caption{a) Thermal cycling history of sensor S7 includes different heating rates and temperature excursions. Time and temperature dependent evolution of amplitude (b), kurtosis (c), fractional width (d) and peak center (e) in sensor S7 shows significant memory effects on the spectra. Note the magnitude of the memory effects between the two sensors is significantly different indicating that temporal evolution of spectral response is sensitive to time history of the sensor.  r }
%\end{figure}

\subsection{Modelling Methodology}
The serial correlations or memory effects introduced by hysteresis (see discussion above), pose significant challenges to model evaluation. In any model training endeavor where the goal is to develop a deployable,  generalized model, it is important to appropriately balance the bias-variance tradeoff\cite{islr}. In sensor literature often the focus is on the training error e.g. the mean square error of fitting a function to the entire dataset is reported. Only optimizing the training error runs the risk of over-fitting the data creating a model that reduces residuals when fitted to the data it was trained on but fairs poorly when new data is introduced. To balance the tendency of models to over-learn, the data is divided between a training and validation set (sometimes alternatively referred to as testing set). The model is initially trained on the training set and subsequently evaluated on the validation set. The goal is to choose a model that minimizes both the training and validation scores i.e. a generalizable model that accounts for significant trends in the data without learning the noise. 
\par The presence of serial correlations in the data however make simple train-test split strategy- e.g. randomly selecting and setting aside $30\%$ of the data points for testing set- inappropriate. As shown in table 1, for 9 out of 14 sensors a train-test split results in out-of-sample errors (validation error) that are either similar or smaller than the training errors suggesting the noise is not appropriately balanced between the two datasets. Similar results are seen for k-fold cross validation (CV) where the data set is randomly broken into k sets. The k-fold CV approach has the benefit of minimizing the impact of outliers on the fit\cite{islr}. 
A common solution to the serial correlation problem is to implement time-series cross validation (TSCV)\cite{time_series_forecasting, islr}. Such an approach, however, makes it difficult to interperate the model in terms that are common in metrology e.g. hysteresis. We, therefore, implement a version of TSCV by dividing the data between the first up ramp (training set) and subsequent data (testing set). This ramp-by-ramp division of the data is appropriate because as shown above the effects of changes driving hysteresis are only significant over long time periods. Our approach divides the dataset in two long time periods, allowing us to isolate the uncertainty introduced by long term drift. The model evaluation on the training set is carried out using k-fold cross validation (k=5)\cite{islr}. Consistent with previous literature\cite{flockhart} we find that a quadratic function vastly outperforms a first order model, reducing MSE by $\approx75\%$. We note that cubic functions show a slight improvement ($13 \%$) over quadratic function. For the purposes of this study, however, we have chosen the simpler quadratic function to model the temperature-wavelength (peak center) relationship staying consistent with previous literature\cite{Ahmed_NCSLI, flockhart}. A detailed evaluation of different models where model complexity (number of free parameters) is weighed against its validation error is the subject of a forthcoming manuscript.        

\begin{threeparttable}\footnotesize
\caption{Training and out-of-sample (validation) error in train-test split evaluation of a first order regression model}\label{tab:tech_sum2}
    \begin{tabular}{p{1.5cm}cp{1.5cm}p{.4cm}}
        \toprule
        \multicolumn{1}{c}{}                   \\
        Sensor   & Training  Error (K) & Out of Sample Error (K) & 
        \\
         \midrule\\
        	S1	& 2.107	& 1.984	&\\
                S2	&  1.78 & 1.67 &\\
			S3  & 1.930	& 1.851 &\\
			S4 & 2.314	& 2.291 &\\
			S5 & 0.357	& 0.341	&\\
                S6	&0.994 &	1.023 	&\\
                S7	& 2.116&	2.099 	&\\
			S8  & 2.196&	2.057	&\\
			S9 & 2.301&	2.275	&\\
			S10 & 2.418&	2.472	&\\
                S11	& 2.330&	2.344	&\\
                S12	& 1.947&	1.937 &\\
			S13\tnote{a} & 2.138	&2.161	&\\
			S13\tnote{b}  & 2.322	&2.571	&\\
			S14 & 2.228 & 2.279 &\\

          \bottomrule
          \end{tabular}
          \begin{tablenotes}
          \small
          \item[a]1$\mu W$ input power
          \item[b] 2.5mW input power
       
                      
        \end{tablenotes}
    \end{threeparttable}    
\clearpage
\subsubsection{Dynamic Regression}
As discussed in the manuscript we employed ARIMA models in a dynamic regression scheme to compensate for the long-term drift in the sensor refractive index due to hysteresis. As noted in the manuscript, the ($p,d,q$) terms for each sensor is determined by examining the ACF and PCF plots of the residuals and first-difference residuals \cite{time_series_forecasting}. The time dependent temperature changes, residuals and ACF and PCF plots for each sensor are shown below. We note that the ARIMA models are evaluated by computing their performance for one-step and dynamic prediction. The one-step prediction refers to outputs of the dynamic regression model where the validation set is introduced one-sequential time step at a time and prediction generated using the last time step i.e. predictions are restricted to short time-horizons. In the case of dynamic prediction, the entire validation set is evaluated at once i.e. the input to ARIMA model (residuals) are restricted to only the calibration/testing set. The results of one-step prediction represents the best-case performance that can be expected from these models. The dynamic prediction results, on the other hand, provides a measure of long-term (few weeks) performance of these models.

\begin{figure}[htbp!]
\centering\includegraphics[width=12 cm]{heatmap}
\caption{Heatmap of Pearson correlation coefficient for non-categorical variables for the entire data of FBG temperature response}
\end{figure}

\begin{figure}[htbp!]
\centering\includegraphics[width=12 cm]{pairplot}
\caption{Pairplots of all non-categorical variables over the entire data set}
\end{figure}

\begin{figure}[htbp!]
\centering\includegraphics[width=12 cm]{ramp1}
\caption{a) Thermal cycling history of sensor S6 includes different heating rates and temperature excursions. At 293 K, time and temperature dependent evolution of fractional width b) and Kurtosis c) in sensor S6 shows significant memory effects }
\end{figure}

\begin{figure}[htbp!]
\centering\includegraphics[width=12 cm]{ramp2}
\caption{a) Thermal cycling history of sensor S7 includes different heating rates and temperature excursions. At 293 K, the cumulative effective of thermal treatment's effect on the fractional width (a) and peak center detunning (b) can be seen.}
\end{figure}


\begin{figure}[htbp!]
\centering\includegraphics[width=12 cm]{adev}
\caption{a) Allen deviation plots for sensor S7 at every temperature between 283 K and 393 K recorded over three thermal cycles. Temperature was incremented in 5 K steps b) Allen deviation plots for sensor S5 at 293 K  (start), 353 K , 393 K  from the ramp-up and 293 K  (ramp-down) are shown. For both sensors, regardless of temperature, we observe a dominant linear trend indicating the measurement noise is dominated by 1/f noise process }
\end{figure}


%\begin{figure}[htbp!]
%\centering\includegraphics[width=12 cm]{dynamic_regression}
%\caption{schema for dynamic regression model incorporating ARIMA models to compensate for hysteresis in FBG temperature sensors}
%\end{figure}

\begin{figure}[htbp!]
\centering\includegraphics[width=12 cm]{temperature_history}
\caption{Time history of temperature changes for each sensor is shown}
\end{figure}

\begin{figure}[htbp!]
\centering\includegraphics[width=12 cm]{residual_history}
\caption{Temporal evolution of residuals calculated from regression model for each sensor are shown}
\end{figure}

\begin{figure}[htbp!]
\centering\includegraphics[width=12 cm]{S1}
\caption{ACF and PCF plots calculated form the residuals (top) and first-difference residuals plot (bottom) for S1 are shown.   }
\end{figure}

\begin{figure}[htbp!]
\centering\includegraphics[width=12 cm]{S2}
\caption{ACF and PCF plots calculated form the residuals (top) and first-difference residuals plot (bottom) for S2 are shown.   }
\end{figure}

\begin{figure}[htbp!]
\centering\includegraphics[width=12 cm]{S3}
\caption{ACF and PCF plots calculated form the residuals (top) and first-difference residuals plot (bottom) for S3 are shown.   }
\end{figure}

\begin{figure}[htbp!]
\centering\includegraphics[width=12 cm]{S4}
\caption{ACF and PCF plots calculated form the residuals (top) and first-difference residuals plot (bottom) for S4 are shown.   }
\end{figure}

\begin{figure}[htbp!]
\centering\includegraphics[width=12 cm]{S5}
\caption{ACF and PCF plots calculated form the residuals (top) and first-difference residuals plot (bottom) for S5 are shown.   }
\end{figure}

\begin{figure}[htbp!]
\centering\includegraphics[width=12 cm]{S6}
\caption{ACF and PCF plots calculated form the residuals (top) and first-difference residuals plot (bottom) for S6 are shown.   }
\end{figure}


\begin{figure}[htbp!]
\centering\includegraphics[width=12 cm]{S7}
\caption{ACF and PCF plots calculated form the residuals (top) and first-difference residuals plot (bottom) for S7 are shown.   }
\end{figure}

\begin{figure}[htbp!]
\centering\includegraphics[width=12 cm]{S8}
\caption{ACF and PCF plots calculated form the residuals (top) and first-difference residuals plot (bottom) for S8 are shown.   }
\end{figure}

\begin{figure}[htbp!]
\centering\includegraphics[width=12 cm]{S9}
\caption{ACF and PCF plots calculated form the residuals (top) and first-difference residuals plot (bottom) for S9 are shown.   }

\end{figure}\begin{figure}[htbp!]
\centering\includegraphics[width=12 cm]{S10}
\caption{ACF and PCF plots calculated form the residuals (top) and first-difference residuals plot (bottom) for S10 are shown.   }
\end{figure}

\begin{figure}[htbp!]
\centering\includegraphics[width=12 cm]{S11}
\caption{ACF and PCF plots calculated form the residuals (top) and first-difference residuals plot (bottom) for S11 are shown.   }

\end{figure}\begin{figure}[htbp!]
\centering\includegraphics[width=12 cm]{S12}
\caption{ACF and PCF plots calculated form the residuals (top) and first-difference residuals plot (bottom) for S12 are shown.   }

\end{figure}\begin{figure}[htbp!]
\centering\includegraphics[width=12 cm]{S13_LOW}
\caption{ACF and PCF plots calculated form the residuals (top) and first-difference residuals plot (bottom) for S13 at 1$\mu W$ of input power are shown.   }

\end{figure}\begin{figure}[htbp!]
\centering\includegraphics[width=12 cm]{S13}
\caption{ACF and PCF plots calculated form the residuals (top) and first-difference residuals plot (bottom) for S13 at 2.5 mW of input power are shown.   }

\end{figure}\begin{figure}[htbp!]
\centering\includegraphics[width=12 cm]{Slide14}
\caption{ACF and PCF plots calculated form the residuals (top) and first-difference residuals plot (bottom) for S14 are shown.   }
\end{figure}


 \bibliographystyle{elsarticle-num} 
 \bibliography{cas-refs}

%% else use the following coding to input the bibitems directly in the
%% TeX file.

% \begin{thebibliography}{00}

% %% \bibitem{label}
% %% Text of bibliographic item

% \bibitem{}

% \end{thebibliography}
\end{frontmatter}
\end{document}