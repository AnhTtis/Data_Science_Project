\subsection{Image Classification}
The most common application of test-time adaptation is multi-class image classification.
Firstly, SFDA methods are commonly evaluated and compared on widely used DA datasets, including Digits, Office, Office-Home, VisDA-C, and DomainNet, as described in previous studies \cite{liang2020we,liang2021source,yan2022dual,zhang2022divide}.
Secondly, TTBA and OTTA methods consider natural distribution shifts in object recognition datasets, \eg, corruptions in CIFAR-10-C, CIFAR-100-C, and ImageNet-C, natural renditions in ImageNet-R, misclassified real-world samples in ImageNet-A, and unknown distribution shifts in CIFAR-10.1, as detailed in previous studies \cite{sun2020test,schneider2020improving,wang2021tent,zhang2022memo}.
In addition, TTBA and OTTA methods are also evaluated in DG datasets such as VLCS, PACS, and Office-Home, as described in previous studies \cite{dinnocente2019learning,pandey2021generalization,iwasawa2021test,gan2023decorate}.
Thirdly, TTPA methods always conduct comparisons on CIFAR-10, CIFAR-100, and MNIST by simulating label shift based on a Dirichlet distribution, as detailed in previous studies \cite{lipton2018detecting,alexandari2020maximum,sipka2022hitchhiker}.
Besides, long-tailed datasets, such as ImageNet-LT, CIFAR-100-LT, and Places-LT, are also utilized to evaluate TTPA methods, as depicted in \cite{sipka2022hitchhiker,zhang2022self}.

\subsection{Semantic Segmentation}
Semantic segmentation aims to categorize each pixel of the image into a set of semantic labels, which is a critical module in autonomous driving.
Many domain adaptive semantic segmentation datasets, such as GTA5-to-Cityscapes, SYNTHIA-to-Cityscapes, and Cityscapes-to-Cross-City, are commonly adopted to evaluate SFDA methods, as depicted in \cite{sivaprasad2021uncertainty,liu2021source,wang2022source}.
In addition to these datasets, BDD100k, Mapillary, and WildDash2, and IDD are also used to conduct comparisons for TTBA and OTTA methods, as shown in \cite{zou2022learning,bahmani2022adaptive}.
OTTA methods further utilize Cityscapes-to-ACDC and Cityscapes-to-Foggy\&Rainy Cityscapes for evaluation and comparison, as described in \cite{wang2022continual,volpi2022on,yang2023exploring}.

\subsection{Object Detection}
Object detection is a fundamental computer vision task that involves locating instances of objects in images.
While early TTA methods \cite{jamal2018deep,roychowdhury2019automatic,tang2022source} focus on binary tasks such as pedestrian and face detection, lots of current efforts are devoted to generic multi-class object detection.
Typically, many domain adaptive object detection tasks including Cityscapes-to-BDD100k, Cityscapes-to-Foggy Cityscapes, KITTI-to-Cityscapes, Sim10k-to-Cityscapes, Pascal-to-Clipart\&Watercolor are commonly used by SFDA methods for evaluation and comparison, as detailed in \cite{li2021free,huang2021model,li2022source_cvpr,sinha2023test}.
Additionally, datasets like VOC-to-Social Bikes and VOC-to-AMD are employed to evaluate TTBA methods, as shown in \cite{dinnocente2020one,borlino2022self}.

\subsection{Beyond Vanilla Object Images}
\method{Medical images}
Medical image analysis is another important downstream field of TTA methods, \eg, medical image classification \cite{ma2022test,wang2022metateacher}, medical image segmentation \cite{he2021autoencoder,karani2021test}, and medical image detection \cite{liu2022source}.
Among them, medical segmentation attracts the most attention in this field.

\method{3D point clouds}
Nowadays, 3D sensors have become a crucial component of perception systems.
Many classic tasks for 2D images have been adapted for LiDAR point clouds, such as 3D object classification \cite{tian2022vdm,mirza2022mate}, 3D semantic segmentation \cite{saltori2022gipso}, and 3D object detection \cite{saltori2020sf,hegde2021uncertainty}.

\method{Videos}
As mentioned above, TTBA and OTTA methods can well address how to efficiently adapt an image model to real-time video data for problems such as segmentation \cite{wang2023test}, depth prediction \cite{liu2023meta}, and frame interpolation \cite{choi2020scene,choi2021test}.
Besides, a few studies  \cite{xu2022learning,huang2022relative,yi2023temporal} investigate the SFDA scheme for video-based tasks like action recognition.

\method{Multi-modal data}
Researchers also develop different TTA methods for various multi-modal data, \eg, RGB and audio \cite{plananamente2022test}, RGB and depth \cite{ahmed2022cross,shin2022mmtta}, RGB and motion \cite{huang2022relative}.

\method{Face and body data}
Facial data is also an important application of TTA methods, such as face recognition \cite{zhang2022free},  face anti-spoofing \cite{wang2021self,liu2022source_eccv,zhou2022generative}, deepfake detection \cite{chen2022ost}, and expression recognition \cite{conti2022cluster}.
For body data, TTA methods also pay attention to tasks such as pose estimation \cite{zhang2020inference,kan2022self,ding2023maps} and mesh reconstruction \cite{guan2021bilevel,li2020online}.

\subsection{Beyond Vanilla Recognition Problems}
\method{Low-level vision}
TTA methods can be also applied to low-level vision problems, \eg, image super-resolution \cite{park2020fast,soh2020meta}, image denoising \cite{vaksman2020lidia,gunawan2022test}, image deblurring \cite{chi2021test} and image dehazing \cite{liu2022towards,yu2022source}.
Besides, the TTA paradigm is introduced to image registration \cite{zhu2021test,hong2021deep}, inverse problems \cite{hussein2020image,gilton2021model,darestani2022test,song2023piner}, quality assessment and enhancement \cite{liu2022source_arxiv,wang2022unsupervised}.

\method{Retrieval}
Besides classification problems, TTA can also be applied to retrieval scenarios, \eg, person re-identification \cite{wu2019distilled,xu2022mimic} and sketch-to-image retrieval \cite{sain2022sketch3t,paul2022ttt}.

\method{Generative modeling}
TTA method can also vary the pre-trained generative model for style transfer and data generation \cite{bau2019semantic,kim2022controllable,subramanyam2022single,nitzan2022mystyle}.

\method{Defense}
Another interesting application of TTA is the test-time adversarial defense for image classification \cite{wang2021fighting,shi2021online,yoon2021adversarial,alfarra2022combating}, which tries to generate robust predictions for possible perturbed test samples.

\subsection{Natural Language Processing (NLP)}
The TTA paradigm is also studied in tasks of the NLP field, such as reading comprehension \cite{banerjee2021self},  question answering \cite{yin2022source,ye2022robust},  sentiment analysis \cite{zhang2021matching,antverg2022idani}, entity recognition \cite{wang2021efficient}, and aspect prediction \cite{ben2022pada,volk2022example}.
In particular, a competition~\footnote{\url{https://competitions.codalab.org/competitions/26152}} has been launched under data sharing restrictions, comprising two NLP semantic tasks \cite{laparra2021semeval}: negation detection and time expression recognition.
Besides, early TTPA methods focus on word sense disambiguation \cite{chan2006estimating}.

\subsection{Beyond CV and NLP}
\method{Graph data}
For graph data (\eg, social networks), TTA methods are evaluated and compared on two tasks: graph classification \cite{chen2022graphtta,wang2022testtime} and node classification \cite{mao2021source,jin2023empowering}.

\method{Speech processing}
As far, there have been three TTA methods, \ie, audio classification \cite{boudiaf2023in}, speaker verification \cite{kim2022variational} and speech recognition \cite{lin2022listen}.

\method{Miscellaneous signals}
In addition, TTA methods are also validated on other types of signals, \eg, radar signals \cite{cao2021towards}, EEG signals \cite{lee2023source}, and vibration signals \cite{jiao2022source}.

\method{Reinforcement learning}
Some TTA methods \cite{hansen2021self,liu2023learning} also address the generalization of reinforcement learning policies across different environments.

\subsection{Evaluation}
As the name suggests, TTA methods should evaluate the performance of test data after test-time optimization immediately.
However, there are different protocols for evaluating TTA methods in the field, making a rigorous evaluation protocol important.
Firstly, some SFDA works, particularly for domain adaptive semantic segmentation \cite{sivaprasad2021uncertainty,wang2022source} and classification on DomainNet, adapt the source model to an unlabeled target set and evaluate the performance on the test set that shares the same distribution as the target set. 
However, this in principle violates the setting of TTA, although the performance on the test set is always consistent with that of the target set. 
\emph{We suggest that such SFDA methods report the performance on the target set at the same time.}
Secondly, some SFDA works such as BAIT \cite{yang2021casting} offer an online variant in their papers, but such online SFDA methods differ from OTTA in that the evaluation is conducted after one full epoch.
\emph{We suggest online SFDA methods change the name to ``one-epoch SFDA" to avoid confusion with OTTA methods.}
Thirdly, for continual TTA methods \cite{wang2022continual,niu2022efficient}, the evaluation of each mini-batch is conducted before optimization on that mini-batch.
This manner differs from the standard evaluation protocol of OTTA \cite{sun2020test} where optimization is conducted ahead of evaluation.
\emph{We suggest that continual TTA methods follow the same protocol as vanilla OTTA methods.}
