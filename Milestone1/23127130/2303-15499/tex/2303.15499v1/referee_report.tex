------------------
Dear Editor,

We would like to thank the referee for their careful reading of the manuscript and for their constructive comments, which have strengthened the paper. We have addressed each of these and included our changes in red font in the updated manuscript. Responses to each of the comments are included below.

With best regards on behalf of all authors, 
Ashley Barnes



MAJOR comments:

- A study of the molecular line content of C2c1a and C2c1b is missing. That is of paramount importance to exclude a potential hot core. Please include Appendix C in the main body.
+ A more expanded version of the chemistry discussion is now moved to the main body of the text. This can now be found is section 7, which also includes the table and figure. 

- I question the equilibrium between the introduction and the discussion. The introduction lasts for 1 page 1/2 when the discussion is less than 1/4 of a page. Some parts of the introduction can be deleted and others moved to the discussion. Moreover, the results shall be discussed further. I recommend comparing C2c1a with the other high-mass prestellar core candidates; discussing the number of known massive prestellar cores versus massive protostellar cores; discussing their lifetime; and discussing this timescale regarding the core collapse models.
+ We have shortened the introduction to keep only the high-level background information on the subject, and the introduction of the ALMA-IRDC study that led to the interest in this source. We have moved the discussion of the other massive cores within the section. 

- Eq. 1: The authors used the integrated flux without background subtraction to derive the mass. Fluxes with background subtraction should be used, or a strong justification given. This, plus the use of \beta=1.75 make the masses upper limits in my opinion.
+ This has now been updated throughout, such that all the properties including the mass are calculated with the background subtraction. Overall, this makes no significant difference to the conclusions of the paper. 

- Since the authors did not probe infall motion, the velocity width they derived from N2D+ could be pure infall, pure turbulence, or a mixture of both. I recommend more caution, reminding the reader of this limitation, when computing the turbulent MJ, the \alpha_vir and estimating.
+ We would like to note that this was explicitly highlighted in the previous manuscript after the alpha_vir. To remind the reader, we now expand this after every stability calculation. 





INTERMEDIATE comments:

- line 165: "The cores are marginally resolved..." I disagree: the beam is sensitive to the diameter of the core, not to its radius. The subsequent deconvolution is, I think, irrelevant.
+ Now removed. 

- Eq. 4: The authors assume a Gaussian response of the individual channel in the ALMA receiver. Is it the case? A reference to the ALMA technical literature is needed.
+  Indeed, this is correct, and we have now explicitly outlined this assumption in the text. We also point to the work of Koch et al. 2018 (https://ui.adsabs.harvard.edu/abs/2018RNAAS...2..220K/abstract) who investigate the effect of varying the spectral response function on the measured line width. This work also references the documentation on the ALMA response function (https://safe.nrao.edu/wiki/pub/Main/ALMAWindowFunctions/Note_on_Spectral_Response.pdf). Overall, however, the line width is well resolved, which should have only a minor impact on our results. 

- Table 1: the EVLA data presented by Wang et al. 2012 has an angular resolution of 2.8" (see their section 4.1, and Fig. 3(a)). Then, I wonder how you could get different temperatures although the sources are separated by ~1.5"?
+ We have double-checked the temperature measurements with Wang et al. and Zhang et al. The way it was done for Zhang et al. (2015) paper by averaging  the NH3 temperatures toward the ALMA continuum peaks. Indeed, however, the NH3 temperature map presented in Wang et al. (2012) has a resolution of ~ 2.8". While most 1.4mm continuum peaks in G38.34-P1 are separated by more than 2.8", the two peaks, C2c1a and C2c1b (corresponding to sources 37 and 38 in Zhang et al.) are not fully resolved. We have inspected the NH3 temperature map again, and there is indeed a temperature difference between the C2c1a and C2c1b peaks. But given its resolution of 2.8", we opt to use the mean temperature of 10.5K for both peaks - and this is now updated thought out the text. 

- see my comments on some references in the PDF
+ Updated

+ Responses to the following three points below. 
- line 224: Is there an optically thick line and its optically thin isotopologue in the spectral setup? This would allow them to address the stability of the cores and solve the degeneracy between turbulence and infall.
- Perhaps I missed it, but I did not see any reference to Appendix A and C in the main body of the text.
- Appendix A: I am not convinced that the HNC emission line shows infall. Moreover, this interpretation contradicts the use of the N2D+ line-width as pure turbulent and thermal motions. Could I see the moment 0, moment 1 and channel maps of HNC, please?
+ Indeed there may be some contribution to the line width from infall motion, and we now note that at several points within the text. However, we believe that a complete analysis of the more extended and optically thick lines to investigate the presence of accretion is beyond the scope of this work. Indeed, this core, along with the others within the ALMA-IRDC dataset, will be checked for infall signature in a future project, but the HCN dataset is currently not imaged to a publication standard. In light of this, we have omitted this portion of text, and spectrum, from the analysis.  





MINOR comments:

- caption, and later as well: replace "example" with the "candidate" when speaking of massive prestellar cores
+ Updated throughout. 

- lines 111, 114 and 122: an angular resolution cannot have a PA. You mean "beam".
+ Indeed updated to beam. 

- line 119: I did not understand the meaning of "which was chosen as this is much weaker than CO"
+ Indeed this is confusing, and refers to the fact that the N2D+ was weaker than the CO, and the shorter baselines were then needed to recover the sensitivity. This sentence has now been removed. 

- line 120: longer -> smaller?+
+ Typo - this is now updated.

- line 105: why are projects 2017.1.00687.S and 2018.1.00850.S cited since the data comes from 2017.1.00793.S?
+ Removed.

- line 150: the leaves are much bigger than the beam. How relevant is the smoothing? Are they identical with and without the smoothing?
+ Indeed the smoothing does not make a significant difference. 

- line 218: if you include m_\mu you compute the sound speed for gas typical of the ISM and not for the molecular Hydrogen, no?
+ Updated to just “gas”

- line 233: the Jeans length is rather a diameter than a radius
+ Numbers updated.

- line 259: I got lost here. Which are the "measured masses"? The 37.7 and 4.9 Msun?
+ This is correct - added pointer to ($M$) here.

- line 280: The parameters of the "fiducial case" should be described (each group has its fiducial case)
+ More information is now given in the text.

- line 283: by \sigma, do you mean \sigma_tot, \sigma_th, or \sigma_turb?
+ This refers to \sigma_tot - updated in text.

- Fig. 1 (caption): please precise the weighting parameter that is chosen for the robust imaging
+ Added into the text, the robust was set to -2, which is close to uniform weighting 