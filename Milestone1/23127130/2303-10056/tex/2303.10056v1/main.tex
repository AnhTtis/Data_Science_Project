\documentclass[10pt,twocolumn,letterpaper]{article}

\usepackage{iccv}
\usepackage{times}
\usepackage{epsfig}
\usepackage{graphicx}
\usepackage{amsmath}
\usepackage{amssymb}

\usepackage[textsize=tiny]{todonotes}

\usepackage{wrapfig,lipsum,booktabs}

\usepackage{bbding}
\usepackage{array}
\usepackage{arydshln}

\usepackage{subfigure}
\usepackage{adjustbox}
\usepackage{xcolor}
\usepackage{threeparttable}
\usepackage{makecell}
\usepackage{listings}
\usepackage{pifont}% 

\usepackage[noblocks]{authblk}

% \def\eg{\emph{e.g}\onedot} 
\def\Eg{\emph{E.g}\onedot}
% \def\ie{\emph{i.e}\onedot} 
\def\Ie{\emph{I.e}\onedot}
% \def\etc{\emph{etc}\onedot}
% \def\aka{\emph{a.k.a}\onedot} 

% \newcommand{\etc}{\emph{etc}}
\newcommand{\aka}{\emph{a.k.a}}
% \newcommand{\ie}{\emph{i.e.}}
% \newcommand{\eg}{\emph{e.g.}}

\newcommand{\se}[1]{\textcolor{orange}{[Stefano: #1]}}
\newcommand{\red}[1]{\textcolor{black}{#1}}
\newcommand{\ning}[1]{\textcolor{blue}{Ning: #1}}

% -- Algorithm pseudo codes:
%\usepackage{algorithm}
\usepackage{algpseudocode}
\usepackage[linesnumbered,ruled]{algorithm2e} % for algorithm with vertical lines
\usepackage{booktabs} 

% Include other packages here, before hyperref.

% If you comment hyperref and then uncomment it, you should delete
% egpaper.aux before re-running latex.  (Or just hit 'q' on the first latex
% run, let it finish, and you should be clear).
% \usepackage[breaklinks=true,bookmarks=false]{hyperref}

% \usepackage[pagebackref=true,letterpaper=true,breaklinks=true,bookmarks=false, colorlinks, linkcolor=blue,  anchorcolor=blue,citecolor=blue,anchorcolor=blue,urlcolor=black  ]{hyperref}

\usepackage{hyperref}
\hypersetup{pagebackref=true,letterpaper=true,breaklinks=true,bookmarks=false, colorlinks, linkcolor=blue,  anchorcolor=blue,citecolor=blue,anchorcolor=blue,urlcolor=black }


\iccvfinalcopy % *** Uncomment this line for the final submission


\def\httilde{\mbox{\tt\raisebox{-.5ex}{\symbol{126}}}}

% Pages are numbered in submission mode, and unnumbered in camera-ready
% \ificcvfinal\pagestyle{empty}\fi

\begin{document}

%%%%%%%%% TITLE
\title{GlueGen: Plug and Play Multi-modal Encoders for X-to-image Generation}


\author{Can Qin$^\star$\thanks{This work was done when Can Qin interned at Salesforce AI Research. Primary contact: qin.ca@northeastern.edu}, 
Ning Yu$^\dagger$,
Chen Xing$^\dagger$,
Shu Zhang$^\dagger$,
Zeyuan Chen$^\dagger$, \linebreak
Stefano Ermon$^\ddagger$,
Yun Fu$^\star$, 
Caiming Xiong$^\dagger$,
Ran Xu$^\dagger$ \linebreak
$^\star$Northeastern University, $^\dagger$Salesforce AI Research, $^\ddagger$Stanford University \linebreak
\small{\texttt{qin.ca@northeastern.edu,  ermon@cs.stanford.edu, yunfu@ece.neu.edu, \linebreak \{ning.yu, cxing, shu.zhang, zeyuan.chen, cxiong, ran.xu\}@salesforce.com }}
}

% \author[$\dagger$]{Ning Yu}
% \author[$\dagger$]{Chen Xing}
% \author[$\dagger$]{Shu Zhang}
% \author[$\dagger$]{Zeyuan Chen}

% \author[$\ddagger$]{Stefano Ermon}

% \author[$\star$]{Yun Fu} 



% \author[$\dagger$]{\\Caiming Xiong}
% \author[$\dagger$]{Ran Xu}

% \affil[$\star$]{Northeastern University} 
% \affil[$\dagger$]{Salesforce Research} 
% \affil[$\ddagger$]{Stanford University}

\maketitle
% Remove page # from the first page of camera-ready.
% \ificcvfinal\thispagestyle{empty}\fi

\begin{abstract}
% 背景
% Pre-trained language models have recently drawn much attention due to their high model capacity and remarkable performance. 
% However, it suffers from huge computational and storage costs when we further scale the model size. 
% To alleviate this problem, we propose a parameter-efficient architecture based on matrix decomposition, which can be easily scaled to deep models without increasing parameters or computational cost by two key techniques.
% (i) We proposed a parameter-efficient architecture based on matrix product operator~(MPO) which can decompose weight matrices into central tensors and auxiliary tensors. More specifically, we share the central tensors across transformers that reduce the total parameters significantly. 
% ~(ii) We theoretically investigate the training instability issue and propose initialization methods for both central and auxiliary tensors.
% We demonstrate these techniques to work well across multiple task settings including standard supervised, few-shot, and multitask learning.
% By using fewer parameters~(\ie 8M fewer than 12-layer BERT) and affordable computational cost~(\ie 3.8 days), we are able to train a 48-layer MPOBERT and outperform several competing models.
% --- v2
\ignore{
Pre-trained language models have recently gained widespread attention due to their high capacity and exceptional performance. However, as the size of these models increases, so do the computational and storage costs associated with them. To address this problem, we propose a parameter-efficient architecture based on matrix decomposition. This architecture is capable of scaling deep models without increasing the number of parameters or computational costs by using two key techniques. 
% First, we introduce a matrix product operator (MPO) that can decompose weight matrices into central tensors and auxiliary tensors, which significantly reduces the total number of parameters. 
Firstly, we introduce a matrix product operator~(MPO) that can decompose weight matrices into central tensors and auxiliary tensors, and we share the central tensors across layers, significantly reducing the total number of parameters.
Secondly, we theoretically investigate the issue of training instability and develop initialization methods for both central and auxiliary tensors. Through extensive experimentation, we demonstrate the effectiveness of our proposed techniques across a range of task settings, including standard supervised, few-shot, and multitask learning. Our model, the 48-layer MPOBERT, requires 8M fewer parameters and 3.8 days of computation time compared to the 12-layer BERT model, yet still outperforms several competing models.}
% 问题与挑战
% 解决方案
% 实验总结
% 另外,额外尝试。

% 基于Transformer堆叠的PLM模型取得了很好的效果但是参数量非常大。已有研究发现这种层重复堆叠的方法非常参数低效

% 最新的方法通过共享所有Transformer层的参数(e.g., ALBERT)实现了comparable results。但这一方法却在深层条件下效果出现了明显的下降。我们发现这种共享策略忽视了Transformer层之间的联系,导致降低模型的表达能力且限制了该方法在更深层PLM的拓展.

% 为了解决以上问题,我们提出了一种基于MPO技术的参数共享方法,Parameter Efficient Transformer,强化了Transformer层之间的语义关系,得到了一个参数更小且更深层的PLM。

% 具体而言,我们的方法主要有2个技术创新点:(1)我们借助MPO分解,共享包含核心信息的CT参数,通过增加AT来提升模型层数,而总参数量又不显著增加,使得构建共享的深层网络成为可能;(2)为了优化PET结构,我们改进了MPOP方法实现,设计了CT-warm策略,仅需要xx% steps即可收敛。

% 我们以1/n的参数构建了12层、24层、48层深度的BERT模型,并且在任务1、任务2上取得了更好的效果。另外,在MPOBERT的框架下构建的深度模型相比较Transformer-based方法收敛更快。




% #################################################
% 预训练语言模型在许多NLP任务上取得了令人瞩目的成就。
% 已有的工作显示在PLMs的模型尺寸以及下游任务微调表现上存在scaling法则。
% 然而,PLM巨量的参数导致很难再资源有限的场景下增大模型参数。
% 在本文中,我们结合矩阵分解与参数共享的方法,提出了一个参数高效的预训练方法来增大PLM的模型深度。具体来说,我们引入了MPO分解对参数矩阵的信息进行重组,并在所有Transformer层中共享使用一个公共的中心张量(包含矩阵的主要信息)。在不同层中通过辅助张量保持层间多样性。
% 另外,我们通过理论推导提出了一个更加高效的初始化方案,可以使得模型训练更加稳定。
% 丰富的实验说明了我们的方法的有效性,
Pre-trained language models have shown remarkable performance across a wide range of NLP applications.
Existing work has demonstrated that there exists a scaling law between the size of PLMs and fine-tuning performance.
% However, the large amount of parameters in PLM makes increasing model parameters in resource-limited circumstances challenging.
However, the large number of parameters in PLMs presents challenges in resource-limited settings.
% In this paper, we present a parameter-efficient pre-training approach for scaling PLM by combining matrix decomposition with the parameter-sharing strategy. 
In this paper, we propose a parameter-efficient pre-training approach that utilizes matrix decomposition and parameter-sharing strategies to scale PLMs.
In particular, we adopt the matrix product operator decomposition to rearrange the information of the parameter matrix and share the central tensor~(containing the major information) across all Transformer layers. The layer-specific auxiliary tensors~(containing the supplementary information) are used to enhance the adaptation flexibility.
Furthermore, we derived a more efficient initialization approach, which may improve model training stability.
Extensive experiments have demonstrated the effectiveness of our proposed model in reducing the model size and achieving highly competitive performance
~(\ie with fewer parameters than BERT$_{\rm{BASE}}$, we successfully scale the model depth by a factor of 4$\times$ and even achieve 0.1 points higher than BERT$_{\rm{LARGE}}$ for GLUE score).

\end{abstract}

\section{Introduction}

Text-to-image (T2I) generative models have made great progress in the last few years thanks to algorithmic advances and the availability of large-scale paired training datasets
~\cite{ramesh2022hierarchical,yu2022scaling,rombach2022high,schuhmann2021laion,schuhmann2022laion}. Diffusion-based T2I generative models in particular 
have achieved remarkable results in terms of image quality
~\cite{ho2022classifier,nichol2021glide,balaji2022ediffi,saharia2022photorealistic,mou2023t2i,zhang2023adding}. 
Despite these strong results, controllable generation for these methods is still challenging: generated images are often not faithful to the captions, compositional capabilities are lacking, and prompt engineering is often required to achieve the desired results~\cite{dall-e-prompt-book}. Moreover, most large-scale models have only been trained on English text captions, greatly limiting their use across the world.

\begin{figure}[t]
\vspace{-2mm}
\includegraphics[width=1\linewidth]{   figs/fig_demo_results.pdf}
% \vspace{-3mm}
\caption{Setting of GlueGen. GlueNet is trying to provide an adaptable portal for the Stable Diffusion model to input multi-modal data, such as text, audio, \ie, (a) and (b), or text-audio hybrid signals, \ie (c), for X-to-image generation.}\label{fig:demo-resuls}
% \vspace{-4mm}
\end{figure}

Recent research has emphasized the crucial role of text encoders in improving Text-to-Image (T2I) models' performance, and their ability to comprehend and represent text is considered a bottleneck for image generation~\cite{saharia2022photorealistic, croitoru2022diffusion}. However, the current T2I models' text encoders are often trained on short image captions, which limits their performance on complex prompts and challenges their quality of feature extraction~\cite{rombach2022high}. Furthermore, T2I models' capacities are limited to generating images from text, and they cannot incorporate multimodal conditions such as sound and audio easily. Nevertheless, replacing the text encoder in existing T2I models is challenging since the text encoder and image generator's representation spaces are tightly coupled~\cite{rombach2022high, ramesh2022hierarchical}. This severe domain gap between the new conditions and the existing model impedes the image generation's final performance, and training the entire T2I model from scratch, with higher quality image-caption pairs, would be prohibitively expensive~\cite{edwards_2022}.\footnote{The cost of training a Stable Diffusion model is around 600K USD.}


As seen in Fig.~\ref{fig:demo-resuls}, we propose GlueNet to address the challenge of efficiently replacing or upgrading the text encoder in existing diffusion-based T2I models. With GlueNet, off-the-shelf pre-trained language models and multimodal encoders can be easily aligned with image encoders of T2I models, greatly enlarging their functionalities at a low cost. Importantly, this can be achieved without requiring retraining from scratch or even finetuning, maintaining the representation alignment between the text and image encoders. The proposed method follows an encoder-decoder structure. The encoder of GlueNet first aligns the representation space of the new condition encoder with that of the T2I model's image generator, minimizing both element-wise and global-wise discrepancy. Then, the decoder of GlueNet maps the aligned condition representations back to the original representation space of the new condition encoder by minimizing the reconstruction loss, preserving rich semantics captured by the pre-trained model during alignment training. Align existing models would inevitably decrease feature discriminability~\cite{pmlr-v97-chen19i,cui2020towards}, which makes the feature decoder necessary.
The entire training of GlueNet requires only a parallel corpus with the same content but different modalities or languages. At inference time, only the encoder of GlueNet is applied on top of the new condition encoder for representation alignment.



To verify the effectiveness of the proposed framework, we conducted three major experiments ranging from single- and multi-modal encoders. Firstly, we upgraded the existing text encoders of the Latent Diffusion Model~\cite{rombach2022high} using a stronger language model, T5-3B~\cite{raffel2020exploring}. Our model showed competitive improvements in FID score and user study ranking compared to the baselines but it still required finetuning for the overall performance boost. Secondly, we aligned a multilingual language model, XLM-Roberta-L~\cite{conneau2019unsupervised}, using our approach, enabling multilingual text-to-image generation. It achieved competitive results of translation-based models under a significantly lower training cost. Finally, we demonstrated GlueNet's capability to bring new functionalities beyond text signals into existing T2I models. 
The alignment of the AudioClip~\cite{guzhov2022audioclip} encoder enables sound-to-image generation without requiring any parameter finetuning of the image generator. This new capability allows the existing Stable Diffusion model to generate high-quality images that correspond to sound signals such as dogs barking and street music. This new capability goes beyond the traditional T2I generation and opens up new possibilities for creating multimedia content towards X-to-Image (X2I) generation.

\begin{figure*}[t]
\begin{center}
\includegraphics[height=0.3\linewidth,width=\linewidth]{   figs/framework_new_iccv.pdf}
\end{center}
\vspace{-4mm}
\caption{
Illustration of our desired GlueGen framework. With the proposed GlueNet model of the GlueGen framework, the pre-trained image generator (\ie UNet) can be bridged to off-the-shelf single- or multi-modal encoders to expand their functionalities, \ie, multilingual/sound-to-image generation, within a limited budget. GlueNet is trained offline and does not require back-propagation of UNet and image-text pairs for training. Therefore, GlueGen is flexible and efficient to achieve.
}
\vspace{-4mm}
\label{fig:setting}
\end{figure*}

Our contributions can be summarized as follows:

\begin{itemize}
\item To the best of our knowledge, this is the first work to consider the problem of efficiently aligning a pre-trained audio model with a pre-trained T2I diffusion model for sound-to-image generation.

\item  Extensive experiments on text-to-image generation benchmarks demonstrate the superiority of our model over the baseline LDM method on both image quality and language controllability.

\item Our framework also enables text-to-image generation beyond English prompts without the need of multilingual image-text pairs for retraining.

\end{itemize}

\section{Related Work of Text-to-Image Models}
Generative Adversarial Nets (GANs) \cite{Goodfellow2014GAN} is one of the major frameworks in text-to-image generation. The method in \cite{reed2016generative} was an early stage approach to use a GAN to train a text-to-image generation network. Since then, other GAN based methods, \eg,   Stack-GAN \cite{han2017stackgan}, Attn-GAN \cite{xu2018Attngan} and SD-GAN \cite{Yin2019SDGAN}, have obtained promising results. In addition, recent works showed more improvements on the generation quality. DM-GAN \cite{zhu2019dmgan} improved text-to-image performance by introducing a dynamic memory component. DF-GAN \cite{tao2022dfgan} designed a fusion module to fuse text and image features. LAFITE \cite{zhou2021lafite} took advantage of CLIP's model to construct pseudo image-text pairs, and proposed a GAN model to learn text-image pairs.

Auto-regressive transformer is another major framework in text-to-image generation. DALL-E \cite{ramesh2021zero} and CogView \cite{ding2021cogview} adopted an autoregressive transformer \cite{Vaswani2017Transformer} to train the correspondences between visual tokens and text tokens. Parti \cite{yu2022scaling} used a powerful visual encoder, ViT-VQGAN \cite{Yu2022VitVQGAN}, to improve results upon DALL-E. The method in \cite{gafni2022make} is similar to CogView and DALL-E, while introducing additional controlling elements to improve the tokenization process. N\"{U}WA \cite{wu2022nuwa} and NUWA-infinity \cite{wu2022nuwainfinity} trained autoregressive visual synthesis model to support both text-to-image and text-to-video generation tasks.

Concurrently, diffusion models \cite{sohl2015deep,song2019generative,song2020score} became a main research focus. In such methods, noise is added to an image, and a score network is trained to denoise and recover the input image. GLIDE \cite{nichol2021glide} performed guided inference with and without a classifier network to generate high-fidelity images. DALL-E 2 \cite{ramesh2022hierarchical} and Imagen \cite{saharia2022photorealistic} set new state-of-the-art results in the text-to-image generation. In DALL-E 2, a prior to produce CLIP image embeddings conditioned on text was learned, and a diffusion model was used to decode the image embeddings to an image. In Imagen, a large-scale frozen text encoder T5-XXL \cite{raffel2020exploring} was adopted to generate embeddings, and a cascade of conditional diffusion models was used to map these embeddings to images of increasing resolutions. Latent diffusion model and stable diffusion model \cite{rombach2022high} are state-of-the-art methods that apply the diffusion model on the latent embedding space as in \cite{sinha2021d2c,vahdat2021score}. These methods are computationally friendly while achieving impressive results. To incorporate more conditions for controllable content generation, \cite{zhang2023adding,mou2023t2i,brooks2022instructpix2pix} introduced additional parameters with new data to inject external knowledge within the current pre-trained model. \cite{chen2022altclip} tried to alter the language model for more capacities. However, all these models are designed to adapt text and image (or similar 2d grid) conditions without the attempt for sequential signals such as audio.


\section{Preliminary}
\vspace{-1mm}
\subsection{Latent Diffusion Model}
The Latent Diffusion Model (LDM)~\cite{rombach2022high} and its extension, \ie, Stable Diffusion\footnote{https://github.com/CompVis/stable-diffusion}, which are variants of Denoising Diffusion Probabilistic Model (DDPM)~\cite{ho2020denoising} family, are selected as our baseline models due to their excellent balance in efficiency and visual quality. In general, LDM is composed of two stages. Firstly, there is an auto-encoder (AE) pre-trained by enormous images with the regularization in either KL-divergence or vector quantization~\cite{van2017neural,agustsson2017soft}. An encoder network, \ie, $E$, is applied for latent feature extraction as $z=E(x)$, which can be mapped back to image space by a decoder network. The input images $x$ and reconstructed images $\hat{x}$ are almost identical $x \approx \hat{x}$. 

Secondly, LDM trains a diffusion model in the latent space~\cite{sinha2021d2c,vahdat2021score}. It follows the standard DDPM~\cite{ho2020denoising} with denoising loss and uses U-net~\cite{ronneberger2015u} as the image decoder as in ~\cite{song2019generative}. To enable generative controllability, LDM has applied multiple conditioning signals ($y$) such as text, mask, or layout, encoded aside and injected into the U-net, with the help of cross-attention layers. This can be formulated as:
\begin{equation}
    \mathcal{L}_{LDM} := \mathbb{E}_{z\sim E(x), \varepsilon \sim N(0,1), t, y} \left [ \left \| \varepsilon - \varepsilon_{\theta}(z_t,t,c_{\phi}(y)) \right \| \right ],
\end{equation}
where $t$ represents the time step, $z_t$ is the noise corrupted latent tensor at time step $t$, and $z_0 = E(x)$. $\varepsilon$ is the unscaled Gaussian noise, $c_{\phi}$ is the conditioning network parameterized by $\phi$ and $\varepsilon_{\theta}$ is the Unet-like denoising network (\aka, image decoder). The parameters of both conditioning and denoising networks, \ie, ${\theta, \phi}$, are trained by the LDM loss. During inference,  clean images can be generated via  classifier-free guidance~\cite{ho2022classifier} as:
\begin{equation}
\hat\varepsilon_{\theta}(z_t|y) = \varepsilon_{\theta}(z_t) + s \cdot (\varepsilon_{\theta}(z_t, c_{\phi}(y)) - \varepsilon_{\theta}(z_t)),
\label{eq:sampling}
\end{equation}
where $s$ is the guidance weight to balance text controllability and image fidelity. 

The basic T2I LDM model is trained on Laion-400M~\cite{schuhmann2021laion} dataset. The Stable Diffusion Model (SDM) was trained with more epochs and training data, \ie, Laion2B-en and Laion-high-resolution~\cite{schuhmann2022laion}. 

\begin{figure*}[t]
\begin{center}
\includegraphics[height=0.28\linewidth,width=\linewidth]{   figs/framework_gluenet.pdf}
\end{center}
\vspace{-6mm}
\caption{(a) Illustration of features transformation throughout the model translation/alignment. (b) The general pipeline and learning objectives of our proposed GlueNet. (c) Detailed architecture of GlueNet Encoder/Decoder.
}
\vspace{-4mm}
\label{fig:method}
\end{figure*}

\subsection{Condition Encoder}
LDM uses  Bert-like~\cite{devlin2018bert} encoder as the text encoder jointly trained with the image decoder using the DDPM loss. SDM uses a pre-trained CLIP~\cite{radford2021learning} text encoder frozen during training. It has been found that increasing the size of the text encoder is very helpful for performance~\cite{saharia2022photorealistic}. In order to upgrade existing models, our goal is to be able to efficienlty plug in more advanced language condition encoders such as T5-3B~\cite{raffel2020exploring}, AudioClip~\cite{guzhov2022audioclip}, or XLM-Roberta~\cite{devlin2018bert}.



\section{GlueNet}
\label{sec:method}
\vspace{-1mm}



The text encoder is one of the key components of T2I models, as generation requires precise and fine-grained text embedding for guidance. The overall performance can be greatly boosted by increasing the size of the text encoder~\cite{saharia2022photorealistic}, however, upgrading an existing text model to a more powerful one is challenging. 
Because existing T2I models are not modular, 
directly replacing (or augmenting with) another condition encoder does not work. The key technical challenge is the mis-alignment between the new condition encoder and the old image decoder. Moreover, joint finetuning also falls short due to catastrophic forgetting occuring when updating well trained parameters. 

Considering these two different condition encoders as source and target domains, we apply a neural network to learn to align. As shown in Fig.~\ref{fig:setting}, this paper has presented a general framework called GlueGen.  Within the framework, GlueNet works as an additional module to bridge the new language model and the old image decoder.


\subsection{Objectives}


\red{Given sets of parallel corpus $\mathcal{X}^s = \{x_{ij}^s|i=1,...,M^s, j=1,...,N^s\}$  and $\mathcal{X}^t = \{x_{ij}^t|i=1,...,M^t, j=1,...,N^t\}$ with $M$ as length of tokens and $N$ as total quantity,  the source model ${F}^{s}$ and a target encoder ${F}^{t}$ have mapped the raw data into embeddings denoted as $\mathcal{S} = \{s_{ij}|i=1,...,M_s, j=1,...,N\}$ where $s_{ij} = F^{s}(x_{ij}^s) \in \mathbb{R}^{d_{s}}$ and  $\mathcal{T} = \{t_{ij}|i=1,..,M_t, j=1,...,N\}$ where $t_{ij} = F^{t}(x_{ij}^t) \in \mathbb{R}^{d_{t}}$. Due to the different models, there is severe distribution mismatch between the source and target features $p(s) \neq p(t)$.}
The  GlueNet is an autoencoder whose encoder $M(\cdot, \Theta_{M})$ learns to map the new features from a source language model, \red{\ie, $s \in \mathcal{S}$,} to align with the current target encoder, \red{\ie, $t \in \mathcal{T}$, where $p(M(s, \Theta_{M})) \approx p(t)$}. \red{Therefore,} this translation (\ie, ${F}^{s} \overset{\tiny{\Theta_{M}} }{\longrightarrow}
 {F}^{t}$ ) enables the image generator to understand the new  features coming from the new condition model without finetuning. To achieve this aim, we consider it as a domain adaptation~\cite{ganin2016domain,qin2019pointdan,long2018conditional} problem and apply both element-wise and distribution-wise alignment losses, which includes the minimization of the mean square error (MSE) loss, \ie, $\mathcal{L}_{mse}$, and the adversarial loss, \ie, $\mathcal{L}_{adv}$, measured over a discriminator network, \ie, $D(\cdot, \Theta_D)$:
\begin{align}
\mathcal{L}_{mse}( \Theta_M) & := \mathbb{E}_{x_t \sim X_t, x_s  \sim X_s}[||{F}^{t}(x_t)- \notag \\  & M(F^{s}(x_s), \Theta_M)||_{2}^{2}], \label{eq:loss-mse} \\
\mathcal{L}_{adv}( \Theta_D, \Theta_M) & :=  \mathbb{E}_{x_t  \sim X_t}[\log{D({F}^{t}(x_t), \Theta_D)}] \notag \\ + & \mathbb{E}_{x_s  \sim X_s}[\log{1-D(M({F}^{s}(x_s), \Theta_M), \Theta_D)}], \label{eq:loss-adv}
\end{align}
where $X_s$ and $X_t$ denote the source and target inputs respectively and they can be the same prompts in English (\eg, LDM $\rightarrow$ T5-3B~\cite{raffel2020exploring}) or in bilingual but parallel content for multilingual T2I or audio-label pairs.

Some insightful investigations in transfer learning reveal that the cross-domain alignment will inevitably decrease feature discrimination~\cite{pmlr-v97-chen19i,cui2020towards}. To project the rich semantics necessary for model upgrading, we further apply a decoder network, \ie,  $N(\cdot, \Theta_N)$, for feature reconstruction:
\begin{align}
    \mathcal{L}_{rec}&(\Theta_M, \Theta_N)  :=  \notag \\ &\mathbb{E}_{x_s  \sim X_s} [||x_s- N(M(F^{s}(x_s), \Theta_M), \Theta_N)||_{2}^{2} ]. \label{eq:loss-rec}
\end{align}

 GlueNet is trained in an end-to-end manner and the details of which  can be referred to in Appendix and Sec.~\ref{sec:experiments}. \cite{ma2022principles} introduced that a universal learning engine should seek a compressive closed-loop transcription with good property in structure-preserving. Our cross-model mapping also requires a similar functionality.  GlueNet desires stability after a loop whose input and output should keep almost equivalent, \ie, $x \approx \hat{x} \approx \hat{\hat{x}} \approx ... $, due to the constraint of reconstruction loss. Therefore, inputting the reconstructed data $\hat{x}$ to  GlueNet is expected to be consistent with the previous loop. This is helpful for accelerating  GlueNet training for T5-to-LDM adaptation.




\subsection{Model Architecture}
\label{sec:arch}
The input data to alignment can be represented as the sequence of tokens: $\mathrm{X}_{i,*} = \{x_{i,1}, x_{i,2}, ...,x_{i,S}\}$ where $\mathrm{X}_{i,*} \in \mathcal{X}$ and ${ x_{i,j} \in \mathbb{R}^{C}}$, with sequence length $S$ and token dim $C$. Inspired by the methods of image super-resolution~\cite{lim2017enhanced,ahn2018fast} to densely regress target data under different input-output dimensions, the translator network stacks three sub-nets in sequence, as seen in Fig.~\ref{fig:method} (c). The head net is only used for simple feature transformation. The body net has many residual modules for fine-grained representation learning. The tail network is appended and is mainly employed for dimension conversion to match the target tensors. Every residual module is implemented based on the MLP-based token mixer~\cite{tolstikhin2021mlp,ma2021rethinking} with high efficiency and strong ability in representation learning. Within the MLP-mixer block, it consists of a Token Mixer and a Sequence Mixer for orthometric purposes. The Token Mixer learns the representation of each token with shared MLP, and the Sequence Mixer is designed for learning the same channel in sequential tokens:
\begin{align}
 \mathrm{U}_{\cdot,i} = & \mathrm{X}_{\cdot,i} + \mathrm{W}_{2}\sigma(\mathrm{W}_{1} * \mathrm{LN}_{\gamma_1, \beta_1}(\mathrm{X})_{\cdot,i}), i=1,...,C \\
  \mathrm{X}_{j,\cdot} = & \mathrm{X}_{j,\cdot} + \mathrm{W}_{4}\sigma(\mathrm{W}_{3} * \mathrm{LN}_{\gamma_2, \beta_2}(\mathrm{U})_{j,\cdot}),  j=1,...,S
\end{align}
where $\mathrm{X}_{\cdot,\cdot}$ is the feature, $\mathrm{LN}_{\gamma, \beta}$ denotes layer normalization parameterized by $\gamma$ and $\beta$, and $\sigma$ indicates the non-linear operator such as GELU~\cite{hendrycks2016gaussian}. The $\mathrm{U}_{\cdot,i}$ is the outputs of Seq MLP and $\mathrm{W}$ represents the weight matrix within the MLP mixer. 


\subsection{Cross-modality Alignment}
\label{sec:cross-modal}
\begin{figure}[t]
\vspace{-2mm}
\includegraphics[width=1\linewidth]{     figs/singal_clip_decompose.pdf}
\vspace{-7mm}
\caption{Signal analysis of CLIPText encoder on 77 tokens (used for Stable Diffusion) averaging over 10,000 randomly sampled prompts. (a) Average dis-similarity map (1-cosine) of CLIPText tokens. (b) The distance between each token and the last token, \ie, $dist(s_{\cdot,j}, s_{\cdot,L})$ with $L$ tokens. We can conclude that only the top K tokens are informative.  }\label{fig:signal-clip}
\vspace{-4mm}
\end{figure}
GlueNet provides a way to align the cross-modal representations between the text encoder, CLIPText~\cite{radford2021learning}, with the AudioClip~\cite{guzhov2022audioclip} model that maps sound signals into embedding spaces. This alignment is useful for extending the functionality of diffusion models like Stable Diffusion, which only takes text signals as input. However, achieving this alignment is challenging because the length of the tokens in each model is different. CLIPText's embeddings require 77 tokens in a sequence, while AudioClip maps sound signals to a single token with a sequence length of 1. Using the standard GlueNet directly fails in this case.

During our exploration, we found that condition information is not uniformly distributed across all 77 tokens of CLIPText outputs. The top K tokens contain the majority of the information. We computed an average dissimilarity map over the CLIPText tokens, visualized in Fig.~\ref{fig:signal-clip} (a), and observed a prominent gap around the eighth or ninth token, with the later ones being highly similar. We also found that corrupting the last token embedding with some random noise would not degrade the generative results. Therefore, it can be concluded that the contribution of different tokens decreases progressively and could be estimated by the feature distance with the last token, $w_{\cdot,j} = dist(s_{\cdot,j}, s_{\cdot,M})$, as shown in Fig.~\ref{fig:signal-clip} (b). We take the l2 distance as $dist(\cdot, \cdot)$ and use this value to reweight the GlueNet objective over different tokens for the sound-to-image generation.

Cross-modal signals fusion is another contribution. To do this, we select the top K tokens of each modality output from GlueNet. Then, the average of the remaining later tokens (excluding last K tokens) of two modality features is concatenated with the top ones. This is a non-parametric operation without any training.  K is empirically chosen from 4 to 8 (or higher). For more details, please refer to Appendix. 

\vspace{-3mm}



\section{Experiments}\label{sec:experiments}
\paragraph{Baseline methods} We compare the performance of models trained using 
\INp{} with the following baseline methods: (1) 
\emph{KD}~\citep{hinton2015distilling,beyer2022knowledge} (Online 
distillation): A standard knowledge distillation method with strong teacher 
models and model-specific augmentations, (2)  
\emph{MEALV2}~\citep{shen2020meal} (Fine-tuning distillation): Distill 
knowledge to student with good initialization from multiple teachers, (3) 
\emph{FunMatch}~\citep{beyer2022knowledge} (Patient online distillation): 
Distill for significantly many epochs with strong augmentations, (4) 
\emph{ReLabel}~\citep{yun2021re} (Offline label-map distillation): Pre-computes 
global label maps from the pre-trained teacher, and (5) 
\emph{FKD}~\citep{shen2021fast} (Offline distillation): Pre-computes soft 
labels using multi-crop knowledge distillation. We consider FKD as the baseline 
approach for dataset reinforcement.
%

\vspace{-4mm}
\paragraph{Longer training}\label{sec:long_train}
Recent works have shown that models trained for few epochs (e.g.,\ 100 epochs) 
are sub-optimal and their performance improves with longer training 
\cite{wightman2021resnet,dosovitskiy2020image,touvron2021training}. Following these works, we train 
different models at three epoch budgets, i.e., 150, 300, and 1000 epochs, using 
both \IN{} and \INp{} datasets.  \Cref{tab:long_train_effect} shows  
models trained with \INp{} dataset consistently deliver better accuracy in 
comparison to the ones trained on \IN{}.
%
An epoch of \INp{} consists of exactly one random reinforcement per sample 
in \IN{}.


%

\begin{table}[t!]
    \centering
    \resizebox{0.95\columnwidth}{!}{
        \begin{tabular}{llccc}
            \toprule[1.5pt]
            \multirow{2}{*}{\textbf{Model}} & \multirow{2}{*}{\textbf{Dataset}} 
            & \multicolumn{3}{c}{\textbf{Training Epochs}}  \\
            \cmidrule[1.25pt]{3-5}
             & & \textbf{150} & \textbf{300} & \textbf{1000} \\
             \midrule[1.25pt]
             \multirow{2}{*}{MobileNetV3-Large} & \IN{} & 74.7 & 74.9 & 75.1 \\
             & \INp{} (Ours) & \textbf{76.2} & \textbf{77.0} &  \textbf{77.9} \\
             \midrule
             \multirow{2}{*}{ResNet-50} & \IN{} & 77.4 &  78.8 & 79.6 \\
             & \INp{} (Ours) & \textbf{79.6} & \textbf{80.6} & \textbf{81.7} \\
             \midrule
             \multirow{2}{*}{SwinTransformer-Tiny} & \IN{} & 79.9 & 80.9 &  80.9 \\
              & \INp{} (Ours) & \textbf{82.0} & \textbf{83.0} & \textbf{83.8} \\
            \bottomrule[1.5pt]
        \end{tabular}
    }
    \vspace{-2mm}
    \caption{\textbf{\INp{} models consistently outperform \IN{} models 
        when trained for longer}. Top-1 accuracy on the \IN{} validation set 
        is shown. An epoch has the same number of iterations for 
        \IN{}/\INp{}.}
    \label{tab:long_train_effect}
    \vspace{-4mm}
\end{table}

\vspace{-4mm}
\paragraph{Training and reinforcement time} \Cref{tab:long_train_effect} shows 
\INp{} improves the performance of various models. A natural question 
that arises is: \emph{Does \INp{} introduce computational overhead when 
training models?}
On average, training MobileNetV3-Large, ResNet-50, and SwinTransformer-Tiny is 
$1.12\times$, $1.01\times$, and $0.99\times$ the total training time on 
\IN{}. The extra time for MobileNetV3 is because there is no data 
augmentations in our baseline.
%
\INp{} took 2205 GPUh to generate using 64xA100 GPUs, which is highly 
parallelizable.
For comparison, training ResNet-50 for 
%
300 epochs on 8xA100 GPUs takes 206 GPUh.
The reinforcement generation is a one-time cost that is amortized over many 
uses.  The time to reinforce other datasets and the storage is discussed in 
\cref{sec:cost}.

\begin{table}[b!]
    \centering
    \resizebox{\columnwidth}{!}{
        \begin{tabular}{llccccc}
            \toprule[1.5pt]
            \multirow{2}{*}{\textbf{Model}} & \multirow{2}{*}{\textbf{Dataset}} & \textbf{Offline} & \textbf{Random } & \multirow{2}{*}{\textbf{Epochs}} & \multirow{2}{*}{\textbf{Accuracy}}   \\
             &  & \textbf{KD?} & \textbf{Init.?} &  &   \\
             \midrule[1.25pt]
              & \IN{}~\cite{howard2019searching} & NA & \cmark & 600 & 75.2 \\
             MobileNetV3& FunMatch~\citep{beyer2022knowledge}* & \xmark & \cmark & 1200 & 76.3\\
             -Large & MEALV2~\citep{shen2020meal} & \xmark & \xmark & 180 & 76.9 \\
             & \INp{} (Ours) & \cmark & \cmark & 300 & \textbf{77.0}\\
             \midrule[1pt]
             \multirow{6}{*}{ResNet-50} & \IN{} \cite{wightman2021resnet} & NA & \cmark & 600 & 80.4 \\
             & ReLabel~\citep{yun2021re} & \cmark & \cmark & 300 & 78.9\\
             %
             %
             & FKD~\citep{shen2021fast} & \cmark & \cmark & 300 & 80.1\\
             %
             & MEALV2~\citep{shen2020meal} & \xmark & \xmark & 180 & 80.6\\
             & \INp{} (Ours) & \cmark & \cmark & 300 & 80.6\\
             & \INp{} (Ours) & \cmark & \cmark & 1000 & \textbf{81.7}\\
             & FunMatch~\citep{beyer2022knowledge}* & \xmark & \cmark & 1200 & \textbf{81.8}\\
             %
             %
             \midrule[0.5pt]
             ResNet-101
             & \IN{} \cite{wightman2021resnet} & NA & \cmark & 1000 & 81.5 \\
%
             \midrule[1pt]
             \multirow{4}{*}{ViT-Tiny} & \IN{} \cite{touvron2021training} & NA & \cmark & 300 & 72.2 \\
              & DeiT \cite{touvron2021training} & \xmark & \cmark & 300 & 74.5 \\
             & FKD~\citep{shen2021fast} & \cmark & \cmark & 300 & 75.2\\
             %
             & \INp{} (Ours) & \cmark & \cmark & 300 & \textbf{75.8}\\
             %
             %
             \midrule[1pt]
             \multirow{3}{*}{ViT-Small}
             & \IN{}~\cite{touvron2021training} & NA & \cmark & 300 & 79.8 \\
             & DeiT~\cite{touvron2021training} & \xmark & \cmark & 300 & 81.2 \\
             & \INp{} (Ours) & \cmark & \cmark & 300 & \textbf{81.4}\\
             \midrule[1pt]
             %
             %
             %
             %
             \multirow{3}{*}{ViT-Base$\uparrow$384}
             & \IN{}~\citep{touvron2021training} & NA & \cmark & 300 & 83.1\\
             & DeiT~\cite{touvron2021training} & \xmark & \cmark & 300 & 83.4 \\
             & \INp{} (Ours) & \cmark & \cmark & 300 & \textbf{84.5}\\
            \bottomrule[1.5pt]
        \end{tabular}
    }
    \vspace{-2mm}
    \caption{\textbf{Comparison with state-of-the-art methods on the \IN{} 
    validation set.} Models trained with \INp{} dataset deliver similar or 
    better performance than existing methods. Importantly, unlike online KD 
    methods (e.g., FunMatch or DeiT), \INp{} does not add computational 
    overhead to standard \IN{} training (\cref{fig:wall_clock}). Here, NA 
    denotes standard supervised \IN{} training with no online/offline KD.
    $\uparrow$384 denotes training at 384 resolution. An epoch has the same 
    number of iterations for \IN{}/\INp{}.}
    \label{tab:comparison_to_literature}
    \vspace{-4mm}
\end{table}

\vspace{-4mm}
\paragraph{Comparison with state-of-the-art methods} 
\Cref{tab:comparison_to_literature} compares the performance of models trained 
with \INp{} and existing methods. We make following observations: (1) Compared 
to the closely related method, i.e., FKD, models trained using \INp{} deliver 
better accuracy. (2) We achieve comparable results to online distillation 
methods (e.g., FunMatch), but with fewer epochs and faster training 
(\cref{fig:wall_clock}). (3) Small variants of the same family trained with 
\INp{} achieve similar performance to larger models  trained with \IN{} 
dataset. For example, ResNet-50 (81.7\%) with \INp{} achieves similar 
performance  as ResNet-101 with \IN{} (81.5\%). We observe similar phenomenon 
across other models, including light-weight CNN models. This enables replacing 
large models with smaller variants in their family for faster inference across 
devices, including edge devices, without sacrificing accuracy.

%
%
%
%
%
%
%
%
%
%
%
%
%
%
%
%
%
%
%
%
%
%
%
%
%
%
%
%
%
%
%
%
%
%
%
%


%


%
%
%
%


%
%
%
%
%
%
%
%
%
%
%
%
%
%
%
%
%
%
%
%
%
%
%
%
%
%
%


\subsection{Transfer Learning}
\label{sec:transfer} 
To evaluate the transferability of models 
pre-trained using \INp{} dataset, we evaluate on following tasks: (1) 
semantic segmentation with DeepLabv3 \cite{chen2017rethinking} on the ADE20K 
dataset~\citep{zhou2019semantic}, (2) object detection with Mask-RCNN 
\cite{he2017mask} on the MS-COCO dataset~\citep{lin2014microsoft}, and (3) 
fine-grained classification on the \CIFAR{}~\citep{krizhevsky2009learning}, 
\Flowers{}~\citep{nilsback2008automated}, and \Food{}~\citep{bossard14} 
datasets.



\Cref{tab:transfer_det_seg,tab:transfer_cls} show models trained on the 
\INp{} dataset have better transferability properties as compared to the 
\IN{} dataset across different tasks (detection, segmentation, and 
fine-grained classification). To analyze the isolated impact of \INp{} in 
this section, the fine-tuning datasets are not reinforced. We present all 
combinations of training with reinforced/non-reinforced pretraining/fine-tuning 
datasets in \cref{sec:transfer_full}.


%



%
%


\begin{table}[t!]
    \centering
    \resizebox{0.8\columnwidth}{!}{
        \begin{tabular}{llcc}
            \toprule[1.5pt]
            \multirow{2}{*}{\textbf{Model}} 
            & \multicolumn{1}{c}{\multirow{2}{*}{\textbf{Pretraining dataset}}}
            & \multicolumn{2}{c}{\textbf{Task}}  \\
            \cmidrule[1.25pt]{3-4}
             & & ObjDet & SemSeg \\
             \midrule[1.25pt]
             \multirow{2}{*}{MobileNetV3-Large} & \IN{} & 35.5 & 37.2 \\
             & \INp{} (Ours) & \textbf{36.1} & \textbf{38.5} \\
             \midrule
             \multirow{2}{*}{ResNet-50} & \IN{} & 42.2 &  42.8 \\
             & \INp{} (Ours) & \textbf{42.5} & \textbf{44.2} \\
             \midrule
             \multirow{2}{*}{SwinTransformer-Tiny} & \IN{} & 45.8 & 41.2
             \\
              & \INp{} (Ours) & \textbf{46.5} & \textbf{42.5} \\
            \bottomrule[1.5pt]
        \end{tabular}
    }
    \vspace{-2mm}
    \caption{\textbf{Transfer learning for object detection and semantic 
    segmentation}. For object detection (ObjDet), we report standard mean 
    average precision on MS-COCO dataset while for sementic segmentation 
    (SemSeg), we report mean intersection accuracy on ADE20K dataset. Task 
    datasets are not reinforced.}
    \label{tab:transfer_det_seg}
    \vspace{-4mm}
\end{table}


\subsection{Robustness analysis} 
%
\label{sec:robustnes}
To evaluate the robustness of different models 
trained using the \INp{} dataset, we evaluate on three subsets of the 
\IN{}V2 dataset \citep{recht2019imagenet}, which is specifically designed to 
study the robustness of models trained on the \IN{} dataset. We also 
evaluate \IN{} models on other distribution shift datasets, 
\IN{}-A~\citep{hendrycks2021nae}, \IN{}-R~\citep{hendrycks2021many}, 
\IN{}-Sketch~\citep{wang2019learning}, ObjectNet~\citep{barbu2019objectnet}, 
and \IN{}-C~\citep{hendrycks2019robustness}.
We measure the top-1 accuracy except for \IN{}-C. On \IN{}-C, we measure 
the mean corruption error (mCE) and report 100 minus {mCE}.

\cref{tab:imagenetv2_accuracy} shows that models trained using \INp{} 
dataset are up to 20\% more robust.  Overall, these robustness results 
in conjunction with results in
\cref{tab:long_train_effect} highlight the effectiveness of the proposed 
dataset.

%
%
%
%
%
%
%
%
%
%
%
%
%
%
%
%
%
%
%
%
%
%
%
%
%
%
%
%
%
%
%
%
%
\begin{table*}[t!]
    \centering
    \resizebox{0.95\textwidth}{!}{
        \begin{tabular}{llcccccccc|c}
            \toprule[1.5pt]
            \multirow{2}{*}{\textbf{Model}} & \multirow{2}{*}{\textbf{Dataset}} 
            & \multicolumn{3}{c}{\textbf{\IN{}-V2}}
            & \multirow{2}{*}{\textbf{\IN{}-A}} 
            & \multirow{2}{*}{\textbf{\IN{}-R}} 
            & \multirow{2}{*}{\textbf{\IN{}-Sketch}} 
            & \multirow{2}{*}{\textbf{ObjectNet}} 
            & \multirow{2}{*}{\textbf{\IN{}-C}}
            & \multirow{2}{*}{\textbf{Avg.}}\\
            %
            %
            %
            \cmidrule[1.25pt]{3-5}
             & & V2-A & V2-B & V2-C\\
             \midrule[1.25pt]
             \multirow{2}{*}{MobileNetV3-Large} & \IN{} & 71.5 & 62.9 & 76.8 
             & 4.5 & 32.4 & 20.6 & 32.8 & 21.8 & 30.4\\
             & \INp{} (Ours) & \textbf{75.1} & \textbf{66.3} 
             &  \textbf{80.5} & \textbf{7.6} & \textbf{42.0} & \textbf{29.0} 
             & \textbf{38.1} & \textbf{32.0} & \textbf{37.1}\\
             \midrule
             \multirow{2}{*}{ResNet-50} & \IN{} & 76.3 &  67.4 & 81.3
             & 11.9 & 38.1 & 27.4 & 41.6 & 33.2 & 37.9\\
             & \INp{} (Ours) & \textbf{79.3} & \textbf{71.3} 
             & \textbf{83.8}
             & \textbf{15.1} & \textbf{48.1} & \textbf{34.9} & \textbf{46.8} 
             & \textbf{39.0} & \textbf{43.7}\\
             \midrule
             \multirow{2}{*}{SwinTransformer-Tiny} & \IN{} & 77.0 & 69.3 
             &  81.6 & 21.0 & 37.7 & 25.4 & 40.5 & 36.9 & 39.6\\
              & \INp{} (Ours) & \textbf{81.5} & \textbf{74.1} 
              & \textbf{85.3} & \textbf{30.2} & \textbf{58.0}$^*$ 
              & \textbf{40.8} & \textbf{50.6} & \textbf{46.6} & \textbf{51.1}\\
            \bottomrule[1.5pt]
        \end{tabular}
    }
    \vspace{-2mm}
    \caption{\textbf{\INp{} models are up to 20\% more robust on \IN{} 
    distribution shifts}. All models are trained for 
    1000 epochs.  We report on \IN{}V2 variations Threshold-0.7 (V2-A), 
         Matched-Frequency (V2-B), and Top-Images (V2-C). We report accuracy on 
         all datasets except for \IN{}-C where we report 100 minus mCE
         metric. $^*$ Largest improvement.}
    \label{tab:imagenetv2_accuracy}
    \vspace{-2mm}
    %
\end{table*}




%

%

%
%

\begin{table}[b!]
    \centering
    \resizebox{\columnwidth}{!}{
        \begin{tabular}{llccc}
            \toprule[1.5pt]
            \multirow{2}{*}{\textbf{Model}} 
            & \multicolumn{1}{c}{\multirow{2}{*}{\textbf{Pretraining dataset}}} 
            & \multicolumn{3}{c}{\textbf{Fine-tuning dataset}}  \\
            \cmidrule[1.25pt]{3-5}
            & & \CIFAR{} & \Flowers{} & \Food{}\\
             \midrule[1.25pt]
             \multirow{2}{*}{MobileNetV3-Large}
             & \IN{} &  84.4 & 92.5 & 86.1  \\
             & \INp{} (Ours) &  \textbf{86.0} & \textbf{93.7}  
             & \textbf{86.6}  \\
             \midrule
             %
             %
             %
             %
             %
             %
             %
             \multirow{2}{*}{ResNet-50}
             & \IN{} & 88.4 &  93.6 & 90.0 \\
             & \INp{} (Ours) & \textbf{88.8} & \textbf{95.0} & \textbf{90.5} 
             \\
             \midrule
             \multirow{2}{*}{SwinTransformer-Tiny}
             & \IN{} & 90.6 & 96.3 &  92.3 \\
             & \INp{} (Ours) & \textbf{90.9} & \textbf{96.6} & \textbf{93.0} 
             \\
            \bottomrule[1.5pt]
        \end{tabular}
    }
    \vspace{-2mm}
    \caption{\textbf{Transfer learning for fine-grained object classification.} 
    Only pretraining dataset is reinforced and fine-tuning datasets are not 
    reinforced.  Reinforced pretraining/fine-tuning results in 
    \cref{tab:teaser_results}.}
    \label{tab:transfer_cls}
    \vspace{-4mm}
\end{table}

\subsection{Calibration: Why are \INp{} models robust and transferable?}
\label{sec:calibration}
To understand why \INp{} models are significantly more robust than \IN{} 
models we evaluate their Expected Calibration Error 
(ECE)~\citep{kumar2019verified} on the validation set.  
\cref{fig:val_calib_error} shows that \INp{} models are well-calibrated and 
significantly better than \IN{} models. This matches recent observations 
about ensembles that out-of-distribution robustness is better for 
well-calibrated models~\citep{kumar2022calibrated}. Full calibration results are presented in 
\cref{sec:val_calib_error_full}.

\begin{figure}[b!]
    \centering
    \includegraphics[width=0.8\linewidth]{figures/figs_calib_plus/val_calib_error.pdf}
       \vspace{-4mm}
    \caption{\textbf{\INp{} models are well-calibrated}.
    We plot the Expected Calibration Error (ECE) on the \IN{} validation set 
    over the validation error (normalized by 100 to range $[0, 1]$) for 
    MobileNetV3/ResNet-50/Swin-Tiny architectures trained for 300 and 1000 
    epochs on \IN{} and \INp{}.  \INp{} models are significantly more 
    calibrated, even matching or better than their teacher (IG-ResNext 
    Ensemble). We also observe that the IG-ResNext model is one of the best 
    calibrated models on the validation set from our pool of teachers.
    }
    \label{fig:val_calib_error}
    %
\end{figure}

\subsection{Comparison with FKD and ReLabel.}

We reproduce FKD and ReLabel with our training recipe as well as regenerate the 
dataset of FKD.
We compare the accuracy on \IN{} validation and its distribution shifts as 
well as the cost of dataset generation/storage.  We train models for 300 
epochs.
%
%

\vspace{-4mm}
\paragraph{Training recipe}
We report results of training with our code on the released datasets of 
ReLabel and {FKD}.
In addition to reproducing FKD results by training on their released dataset of 
500-sample per image, we also reproduce their dataset using our code and their 
teacher.  \cref{tab:fkd_comparison} verifies that our improvements are due to 
the superiority of \INp{}, not any other factors such as the training recipe.
Our \INp{}-RRC is also closely related to FKD as it uses the same set of 
augmentations (random-resized-crop and horizontal flip) but together with our 
optimal teacher (4xIG-ResNext).  We observe that \INp{}-RRC achieves better 
results than FKD but still lower than \INp{} 
(\cref{tab:imagenet_plus_table6_e1000,fig:imagenet_delta_150}).

\vspace{-4mm}
\paragraph{Generation/Storage Cost} We provide comparison of generation/storage 
costs in \cref{tab:fkd_comparison}. In our reproduction, generating FKD's data 
takes 2260 GPUh, slightly more than \INp{} because their teacher processes 
inputs at the larger resolution of $475\times 475$ compared to our resolution 
of $224\times 224$.

\vspace{-4mm}
\paragraph{\INp{}-Small} We subsampled \INp{} into a variant that is {10.6} 
GBs, comparable to prior work. We reduce the number of samples per image to 
{100} and store teacher probabilities with top-5 sparsity. If not subsampled 
from \INp{}, generating \INp{}-Small would take half the time of FKD (200 
samples) while still comparable in accuracy to \INp{}.  Note that \INp{} is 
more general-purpose and preferred, especially for long training.

%
%
%
%




\begin{table*}[tbh!]
    \vspace{-2mm}
    \centering
    \resizebox{0.95\textwidth}{!}{
        \begin{tabular}{ccccc|cc|cc|c|cc|cc}
            \toprule[1.5pt]
\textbf{Dataset} & \textbf{Our} & \textbf{Our} & \multicolumn{2}{c}{\textbf{Optimal}} & \textbf{Top-K} & \textbf{Num.} & \multicolumn{2}{c}{\textbf{Storage (GBs)}} & \textbf{Gen. Time} & \multicolumn{2}{c}{\textbf{ResNet-50}}& \multicolumn{2}{c}{\textbf{Swin-Tiny}} \\
& \textbf{Gen.} &\textbf{Train} & \textbf{Teacher} & \textbf{Aug.} &&\textbf{Samples}& Raw &GZIP& \textbf{(GPUh)}& IN & IN-OOD & IN & IN-OOD\\
ReLabel                    & \xmark & \cmark & \xmark & \xmark & 5 &   1 &          10.7 &  4.8&          10 & 79.5 & 45.7 & 81.2 & 48.2\\
FKD                        & \xmark & \cmark & \xmark & \xmark & 5 & 200 &          13.6 &  8.9&     904$^*$ & 79.8 & 45.0 & 82.0 & 48.7\\
FKD                        & \xmark & \cmark & \xmark & \xmark & 5 & 500 &          34.0 & 22.0&     2260$^*$& 80.1 & 45.0 & 82.2 & 48.9\\
FKD                        & \cmark & \cmark & \xmark & \xmark &10 & 400 &          46.3 & 33.4&        1808 & 79.8 & 45.0 & 82.1 & 49.0\\
\midrule[1.25pt]
\INp{}-RRC                 & \cmark & \cmark & \cmark & \xmark &10 & 400 &          46.3 & 33.4&         1993 & 80.3 & 46.5 & 82.4 & 51.0\\
\INp{}-Small  & \cmark & \cmark & \cmark & \cmark & 5 & 100 
            &  10.6& 5.6 & 551& \textbf{80.6} & \textbf{48.9} 
            & \textbf{82.9} & \textbf{54.6}\\
\INp{}                     & \cmark & \cmark & \cmark & \cmark &10 & 400 &          61.5 & 37.5&         2205 & \textbf{80.6} & \textbf{49.1} & \textbf{83.0} & \textbf{54.7}\\
\bottomrule[1.5pt]
        \end{tabular}
    }
    \vspace{-2mm}
    \caption{\textbf{Comparison with Relabel and FKD. Up to 5.6\% better than 
    FKD on ImageNet-OOD}, the average of \IN{}-V2/A/R/S/O/C accuracies.  
    Highlighted accuracies are within 0.2\% of the best. Compared with prior 
    work, we use an optimal teacher (4xIG-ResNext) and optimal combination of 
    augmentations (RRC+RA/RE). $^*$ Our estimates.}
    \label{tab:fkd_comparison}
    %
\end{table*}

\subsection{CLIP-pretrained Teachers}

In this section, we evaluate the performance of CLIP-pretrained 
models~\citep{radford2021learning} fine-tuned on \IN{} as teachers. This 
study complements our large-scale study of teachers in \cref{sec:good_teacher} 
where we evaluated more than {100} SOTA large models and ensembles.  
\Cref{tab:imagenet_plus_clip_vit_mixed_short} compares an ensemble of {4} 
CLIP-pretrained models to our selected ensemble of 4 IG-ResNext models as well 
as a mixture of ResNext, ConvNext, CLIP-ViT, and ViT (abrv. RCCV) models (See 
\cref{sec:clip_vit_mix} for the model names). We generate new \INp{} 
variants and train various architectures for {1000} epochs on each dataset. We 
observe that \INp{} with our previously selected IG-ResNext ensemble is 
superior to CLIP-pretrained and mixed-architecture teachers across 
architectures.
The CLIP variant provides near the maximum gain on Swin-Tiny and mixing it with 
IG-ResNext reduces the gap on CNNs.
%
%
%
%


\begin{table}[tbh!]
    %
    \centering
    \resizebox{0.98\columnwidth}{!}{
        \begin{tabular}{l c ccc}
\toprule
\multirow{2}{*}{\textbf{Model}} & \textbf{\IN{}} & \multicolumn{3}{c}{\textbf{\INp{}}}\\
\cmidrule[1.25pt]{3-5}
&  & \textbf{IG-ResNext}$^*$ & CLIP & Mixed\\
\midrule[1.25pt]
MobileNetV3-Large & ${75.1}$ & $\bm{77.9}_{\scriptscriptstyle +2.9}$ & ${77.2}_{\scriptscriptstyle +2.1}$  & ${77.4}_{\scriptscriptstyle +2.3}$\\
\midrule
ResNet-50 & ${79.6}$ & $\bm{81.7}_{\scriptscriptstyle +2.1}$ & ${81.1}_{\scriptscriptstyle +1.4}$  & $\bm{81.5}_{\scriptscriptstyle +1.8}$ \\
\midrule
Swin-Tiny & ${80.9}$ & $\bm{83.8}_{\scriptscriptstyle +2.8}$ & $\bm{83.7}_{\scriptscriptstyle +2.7}$  & $\bm{83.8}_{\scriptscriptstyle +2.8}$\\
\bottomrule[1.5pt]
\end{tabular}
%
%


    }
    \caption{\textbf{Our selected IG-ResNext ensemble is superior to 
    CLIP-pretrained ensembles.} We reinforce \IN{} dataset with an ensemble 
    of CLIP-pretrained models as well as a mixture of multiple architectures 
    and train various models for 1000 epochs.
    Subscripts show the improvement on top of the \IN{} accuracy.
    $^*$ Our chosen \INp{} variant.
    }
    \label{tab:imagenet_plus_clip_vit_mixed_short}
    \vspace{-5mm}
\end{table}


\section{Conclusion}
% We develop MPOBERT, a parameter-efficient pre-trained language model that allows for fewer parameters and efficient training.
% MPOBERT develops a novel mechanism with which we can easily scale BERT to deep models without increasing parameters or computational costs.
% During training, we propose initialization methods for both the central tensors and the auxiliary tensors based on our theoretical analysis to alleviate the training instability issue.
% We validate the effectiveness via supervised, few-shot and multitask experiments. 
% With fewer and less training costs, MPOBERT outperforms several competing models.
% --v2
We develop MPOBERT, a parameter-efficient pre-trained language model that allows for the efficient scaling of deep models without the need for additional parameters or computational resources. 
We achieve this by introducing an MPO-based Transformer layer and sharing the central tensors across layers. During training, we propose initialization methods for the central and auxiliary tensors, which are based on theoretical analysis to address training stability issues. 
The effectiveness of MPOBERT is demonstrated through various experiments, such as supervised, multitasking, and few-shot where it consistently outperforms other competing models.


{\small
\bibliographystyle{ieee_fullname}
\bibliography{main}
}

\clearpage

\clearpage
\section{Appendix}
\label{sec:appendix}
\subsection{Proofs}
\label{app:proof}
\paragraph{Notations.} We denote $\mathcal{L}(\cdot)$ as the loss function. $LN(x)$ as the standard layer normalization with scale $\gamma=1$ and bias $\beta =0$. Let $\mathcal{O}(\cdot)$ denote standard Big-O notation that suppresses multiplicative constants. $\overset{\Theta}{=} $ stands for equal bound of magnitude. 
We aim to study the magnitude of the model updates. We define the model update as $\left\| \bigtriangleup F\right\|$.
% \begin{definition}
%     $F(x,\theta)$ is updated by $\Theta(\eta)$ per SGD step after initialization as $\eta\to 0$. That is, $\left\| \bigtriangleup F(x)\right\|=\Theta(\eta)$ where $\bigtriangleup F(x)$ can be calculated through $F(x,\theta-\eta\frac{\partial}{\partial \theta}\mathcal{L}(F(x)-y))-F(x;\theta)$.
% \end{definition}

\paratitle{Definition}
    $F(x,\theta)$ is updated by $\Theta(\eta)$ per SGD step after initialization as $\eta\to 0$. That is, $\left\| \bigtriangleup F(x)\right\|=\Theta(\eta)$ where $\bigtriangleup F(x)$ can be calculated through $F(x,\theta-\eta\frac{\partial}{\partial \theta}\mathcal{L}(F(x)-y))-F(x;\theta)$.

\begin{theorem}
    Given an $N$-layer transformer-based model $F(x,\theta)(\theta=\{\theta_1, \theta_2, ...,\theta_N\})$, where $\theta_l$ denotes the parameters in $l$-th layer and each sub-layer is normalized with Post-LN: $x_{l+1}=LN(x_l+G_l(x_l,\theta_l))$. In MPOBERT, $\theta_l$ is decomposed by MPO to local tensors: $\theta_l=u_l\cdot c_l\cdot v_l$, and we share $\{c_i\}_{i=1}^{N}$ across $N$ layers: $c_l=c_1, l=1,2,\cdots,N$. Then $\left\| \bigtriangleup F\right\|$ satisfies:
    \begin{align}
    \left\| \bigtriangleup F\right\|\leq
    % &\sum_{i=1}^{N}\frac{1-u_ic_1v_i}{(1+u_i^2c_1^2v_i^2)^{\frac{3}{2}}}(c_1v_i\left\|u_i^*-u_i \right\| \nonumber\\
    % &+ c_1u_i\left\|v_i^*-v_i \right\| + u_iv_i\left\|c_1^*-c_1 \right\|)
    &\sum_{i=1}^{N}(c_1v_i \left\|u_{i}^*-u_{i}\right\|+ c_1u_i \left\|v_{i}^*-v_{i}\right\| \nonumber\\
    &+ v_iu_i \left\|c_1^*-c_1\right\|)
    \end{align}
    \label{app-thm1}
\end{theorem}
$Proof.$ 
We follow~\cite{zhang2019fixup} and make the following assumptions to simplify the derivations:
\begin{enumerate}
    \item Hidden dimension $d$ equals to $1$;
    \item $var(x+G_l(x))\overset{\Theta}{=}var(x)+var(G_l(x))$;
    \item All relevant weights $\theta$ are positive with magnitude less than $1$.
\end{enumerate}
Given Assumption 1, if $G_l(x)$ is MLP with the weight $\theta_l$, then $G_l(x)\overset{\Theta}{=}\theta_l x$. With assumption 2, we have:
\begin{align}
    x_{l+1}&=f_l(x_l, \theta_l)=\frac{x+G_l(x)}{\sqrt{Var(x+G_l(x))}}\\
    &\overset{\Theta}{=}\frac{1+\theta_l}{\sqrt{1+\theta_l^2}}x_l,
    \label{eq:base}
\end{align}
Then, with Taylor expansion, the model update $\left\|\bigtriangleup F\right\|$satisfies:
\begin{align}
    \left\|\bigtriangleup F\right\|=&\left \|F(x,\theta^*)-F(x, \theta\right \|\nonumber\\
    =&\left\|x_{N+1}^*-x_{N+1}\right\| \nonumber\\
    =&\left\|f(x_{N}^*,\theta_{N}^*) -f(x_{N},\theta_{N})\right\|\nonumber\\
    =&\left \| f(x_{N}^*, U_{N}^*,C_{N}^*,V_{N}^*)\nonumber\right.\\
    &\left.-f(x_N, U_{N},C_{N},V_{N}) \right \| \nonumber\\
    \approx&\left \|\frac{\partial f }{\partial x}(x_{N}^*-x_{N})\nonumber\right.\\
    &\left.+\frac{\partial f}{\partial \theta }\frac{\partial\theta}{\partial U_{N}}(U_{N}^*-U_{N})^T \nonumber\right.\\
    &\left.+\frac{\partial f}{\partial \theta }\frac{\partial\theta}{\partial C_{N}}(C_{N}^*-C_{N})^T\nonumber\right.\\
    &\left.+\frac{\partial f}{\partial \theta }\frac{\partial\theta}{\partial V_{N}}(V_{N}^*-V_{N})^T  \right \|
    \label{eq:9}
\end{align}
With Eq.~\eqref{eq:base}, the magnitude of $\frac{\partial f_l}{\partial x}$ and $\frac{\partial f_l}{\partial \theta}$ is bounded by:
\begin{align}
    & \frac{\partial f_l}{\partial x}\overset{\Theta}{=}\frac{1+\theta_l}{\sqrt{1+\theta_l^2}} \\
    & \frac{\partial f_l}{\partial \theta_l}\overset{\Theta}{=}\frac{1-\theta_l}{(1+\theta_l^2)^{\frac{3}{2}}}x_l
\end{align}
Since we apply MPO decomposition to $\theta_l$, we get:
\begin{align}
    % &\theta_l=u_lc_lv_l
    \theta_l=U_l\cdot C_l \cdot V_l
\end{align}
For simplicity, we reduce the matrices $U$,$C$,$V$ to the scalars $u$,$c$,$v$. 
% In MPOBERT, we share $\{c_i\}_{i=1}^{N}$ across $N$ layers, so it goes to $c_l=c_1$. Considering that $c_1$ is initialized with well-trained parameters, the magnitude of the term $(c_N^*-c_N)$ is negligible compared with others. 
Thus with Assumption 3, Eq.~\eqref{eq:9} is reformulated as:
Finally, with Assumption 3 we have:
\begin{align}
    \left\|\bigtriangleup F \right\|=
    &\left\| x_{N+1}^*-x_{N+1} \right\| \\
    \leq &\sum_{i=1}^{N}\frac{1-u_ic_1v_i}{({{1+u_{i}^2c_1^2v_{i}^2}})^{\frac{3}{2}}}(c_1v_i \left\|u_{i}^*-u_{i}\right\|\nonumber\\
    &+ c_1u_i \left\|v_{i}^*-v_{i}\right\|) + v_iu_i \left\|c_1^*-c_1\right\|) \nonumber\\
    \approx&\sum_{i=1}^{N}(c_1v_i \left\|u_{i}^*-u_{i}\right\|+ c_1u_i \left\|v_{i}^*-v_{i}\right\| \nonumber\\
    &+ v_iu_i \left\|c_1^*-c_1\right\|)
\end{align}
\rightline{$\Box$}
\begin{corollary}
\label{app-thm2}
    % In $N$-layer MPOBERT, we assume that $\left\| \frac{\partial F}{\partial c_1} \right\|=\mathcal{O}(1)$, \ie the gradient signal of $c_1$ from the layers above is bounded, 
    Given that we initialise $c_1$ in MPOBERT with well-trained weights, it is reasonable to assume that updates of $c_1$ are well-bounded.
    Then $\bigtriangleup F$ satisfies $\left\| \bigtriangleup F\right\|=\mathcal{O}(1)$ when for all $i=1,\cdots,N$:
    \begin{equation}
        (v_i^2+u_i^2)(u_Nv_N)=\mathcal{O}(\frac{1}{N})
    \end{equation}
\end{corollary}

$Proof.$ 
For an $N$-layer MPOBERT, we have:
\begin{align}
    \left \|\bigtriangleup F\right \|
    \leq &\sum_{i=1}^{N}(v_i \left\|u_{i}^*-u_{i}\right\|+{u_i \left\|v_{i}^*-v_{i}\right\|}) \\
    \leq &\eta\sum_{i=1}^{N}(v_i \left\|\frac{\partial \mathcal{L}}{\partial F}\right\|\cdot \left\|\frac{\partial F}{\partial \theta_i}\right\|\cdot \left\|\frac{\partial \theta_i}{\partial u_i}\right\|\nonumber\\
    &+{u_i \left\|\frac{\partial \mathcal{L}}{\partial F}\right\|\cdot \left\|\frac{\partial F}{\partial \theta_i}\right\|\cdot\left\|\frac{\partial \theta_i}{\partial v_i}\right\|})
\end{align}
By assumption $\left\| \frac{\partial \mathcal{L}}{\partial F}\right\|=\mathcal{O}(1)$ and $\left\|\frac{\partial F}{\partial {\theta_i}} \right\|\leq\left\| \frac{\partial F}{\partial \theta_N} \right\|\overset{\Theta}{=}\left\| \theta_{N}\right\|$, we achieve:
\begin{align}
    % &{\eta\sum_{i=1}^{N}\frac{1-u_iv_i}{({{1+u_{i}^2v_{i}^2}})^{\frac{3}{2}}}(v_i \left\|\frac{\partial \mathcal{L}}{\partial F}\right\|\cdot \left\|\frac{\partial F}{\partial u_i}\right\|}+{u_i \left\|\frac{\partial \mathcal{L}}{\partial F}\right\|\cdot \left\|\frac{\partial F}{\partial v_i}\right\|})\\
    &\eta\sum_{i=1}^{N}(v_i \left\|\frac{\partial \mathcal{L}}{\partial F}\right\|\cdot \left\|\frac{\partial F}{\partial \theta_i}\right\|\cdot \left\|\frac{\partial \theta_i}{\partial u_i}\right\|\nonumber\\
    &+{u_i \left\|\frac{\partial \mathcal{L}}{\partial F}\right\|\cdot \left\|\frac{\partial F}{\partial \theta_i}\right\|\cdot\left\|\frac{\partial \theta_i}{\partial v_i}\right\|})\\
    =&\eta\sum_{i=1}^{N}(v_i^2u_Nv_N+u_i^2u_Nv_N) \nonumber\\
    = &\mathcal{O}(\sum_{i=1}^{N}(v_i^2+u_i^2)(u_Nv_N))=\mathcal{O}(1),
\end{align} 
\label{eq:bound}
Finally, we achieve:
\begin{equation}
    (v_i^2+u_i^2)(u_Nv_N)=\mathcal{O}(\frac{1}{N})
\end{equation}

Due to symmetry, we set $u_i=u$, $v_i=v$. Thus, from~\ref{eq:bound}, we set $u=v=(2N)^{-\frac{1}{4}}$ to achieve to bound the magnitudes of each update to be independent of model depth $N$, \ie $\left\| \bigtriangleup F\right\|=\mathcal{O}(1)$.
\rightline{$\Box$}

\begin{algorithm}[htb]
    \caption{The MPOBERT training procedure.}
    \begin{algorithmic}[1] %每行显示行号
    \small
        % \Require  $\Matrix{W}$: initial pre-trained weight 
        \Require $\Matrix{W}^{(l)}$: Weight matrix of $l$-th layer in MPOBERT.
        $\Matrix{W}_{A}^{(0)}$: Pre-trained weight matrix in ALBERT.
        % $\Matrix{W}_{Adapter}^{(l)}$: Low rank adapter containing $\Matrix{U}^{(l)}$ and $\Matrix{D}^{(l)}$.
        $\Matrix{U}^{(l)}$ and $\Matrix{D}^{(l)}$: Matrices in low-rank adapter.
        $\eta$: Learning rate.
        $\mathcal{L}$: Stochastic objection function.
        $L$: Model layers number.
        % \Require time step $t\gets 0$~(Initialize timestep)
        \Statex (MPO decomposition)
        \State
        $\{\Tensor{A}_1^{(l)},\Tensor{A}_2^{(l)},\Tensor{C}^{(l)},\Tensor{A}_3^{(l)},\Tensor{A}_4^{(l)}\}$ $\gets$ MPO ($\Matrix{W}^{(l)}$)
        \State
        $\{\Tensor{A}_1^{(0)},\Tensor{A}_2^{(0)},\Tensor{C}^{(0)},\Tensor{A}_3^{(0)},\Tensor{A}_4^{(0)}\}$ $\gets$ MPO ($\Matrix{W}_{A}^{(0)}$)
        \Statex (Initialization Procedure)
        \For {$0<l\leq 24$}
            \State  $\Tensor{C}^{(l)} \gets \Tensor{C}^{(0)} , 
             \{\Tensor{A}_{j}^{(l)}\}_{j=1}^{4} \gets \{\Tensor{A}_{j}^{(0)}\}_{j=1}^{4}$ 
        \EndFor
        \For {$ 24<l\leq L$}
            \State  $\Tensor{C}^{(l)} \gets \Tensor{C}^{(0)} ,    \{\Tensor{A}_{j}^{(l)}\}_{j=1}^{4} \gets \{(2L)^{-\frac{1}{4}}\Tensor{A}_{j}^{(0)}\}_{j=1}^{4}$ 
        \EndFor
        \State $\Matrix{U}^{(l)} \gets \Matrix{0}$, $\Matrix{D}^{(l)} \gets \mathcal{N}(0, \sigma^2)$
        \State $\Matrix{W}^{(l)}=\Tensor{A}_1^{(l)}\Tensor{A}_2^{(l)}\Tensor{C}^{(l)}\Tensor{A}_3^{(l)}\Tensor{A}_4^{(l)}+\Matrix{W}_{Adapter}^{(l)}$
        \Statex (Training procedure with mixed precision and fused implementation techniques.)
        \While {not converged}
            \State $t \gets t+1$
            \State $g_t \gets \frac{\partial\mathcal{L}(\Matrix{W}^{(l)}_t)}{\partial(\Matrix{W}^{(l)}_t)}$
            \State $\Matrix{W}^{(l)}_t \gets \Matrix{W}^{(l)}_{t-1} - \eta \cdot g_t$
        \EndWhile
        \State \Return Converged model
    \end{algorithmic}
\label{alg-overall-process}
\end{algorithm}
\subsection{Training Details}
\label{add-trianing-detail}

\subsubsection{Details of Training}
Here we describe the details of the pre-training process in Algorithm~\ref{alg-overall-process}. 
For pre-training, we tune the learning rate in the range of [$1.0\times 10^{-5}$, $1.0\times 10^{-6}$] and use the LAMB optimizer~\cite{you2020lamb}. Since fine-tuning is typically fast, we run an exhaustive parameter search~(\ie learning rate in the range of [$2.0\times 10^{-4}$, $2.0\times 10^{-6}$], batch size in \{8,16,32\}) and choose the model that performs best on the development set to make predictions on the test set. 


\subsubsection{Details of Training Configurations}
In this part, we list the training configurations of MPOBERT and other representative PLMs in Table~\ref{tab-strongest_variants}.
\begin{table*}[t]
\centering
\begin{tabular}{lrrrcrrr}                       
\toprule
Models    & \#To~(M) & Depth & Samples  & Training time  & GLUR Dev.  &GLUE Test      \\ \midrule
T5$\rm{_{11B}}$        & 11000  & 24     & -   & -    & - & 89.0       \\  \\
T5$\rm{_{BASE}}$        & 220  & 24     & 128$\times$ 524$k$   & \multirow{2}{*}{\thead{16 TPU v3\\ 1 Day~(t5-base)}}  & 84.1    & 82.5       \\  \\
BERT$\rm{_{LARGE}}$      & 330     & 24     & 256$\times$ 1000$k$   & \multirow{2}{*}{\thead{16 Cloud TPUs\\ 4 Days}}  & 84.1    & 81.6       \\  \\
ALBERT$\rm{_{XXLARGE}}$    & 235     & 1     & 4096$\times$ 1.5$M$   & \multirow{2}{*}{\thead{TPU v3\\ 16 Days}}  & 90.0    & -       \\  \\
BART$\rm{_{LARGE}}$      & 407     & 24     & 8000$\times$ 500$k$   & -  & 88.8    & -       \\  \\
RoBERTa$\rm{_{LARGE}}$   & 355     & 24     & 8000$\times$ 500$k$   & \multirow{2}{*}{\thead{1024 V100 GPUs\\ 1 Day}}  & 88.9    & -       \\  \\
XLNet$\rm{_{LARGE}}$     & 361     & 24     & 8192$\times$ 500$k$   & \multirow{2}{*}{\thead{512 TPU v3\\ 5.5 Days}}  & 87.4    & -       \\  \\
MPOBERT$_{48+}$   & 102     & 48    & 4096$\times$ 10$k$    & \multirow{2}{*}{\thead{8 V100 GPUs\\ 3.8 Days}}    & 85.6  & 81.7  \\ \\ \bottomrule
\end{tabular}
\caption{Comparison with the strongest variants of popular PLMs. Since T5$\rm{_{11B}}$ has far more parameters than other candidates, it's reasonable to use T5$\rm{_{base}}$ for a fair comparison.}
\label{tab-strongest_variants}
\end{table*}


\subsection{Experimental Details}
\subsubsection{Details of Fine-tuning Datasets}
\label{add-detail_dataset}
GLUE benchmark covers multiple datasets~(MNLI, QNLI, QQP, CoLA, RTE, MRPC, SST-2)~\footnote{In line with~\citet{raffel2020exploring}, we do not test WNLI due to its adversarial character with respect to the training set.}. 
The SQuAD is a collection of 100$k$ crowd-sourced question/answer pairs. Given a question and a passage, the task is to predict the answer text span in the passage. 

\subsubsection{Details of Evaluation Metrics}
\label{add-detail_metric}
Following~\citet{gao2022parameter}, we employ Matthew's correlation for CoLA, Spearman for STS-B, F1 for MRPC, and accuracy for the remaining tasks as the metrics for the GLUE benchmark.
We compute and present the average scores across all test samples for each of the aforementioned metrics.

\subsubsection{Details of Baseline Models}
\label{add-detail_baseline}
We compare our proposed MPOBERT to the existing competitive deep PLMs and parameter-efficient models. In order to make fair comparisons, we divide the models into three major categories based on their model sizes: 
$\bullet$~{Tiny Models~(\rm{\#To < 50M}).} ALBERT$_{12}$~\cite{lan2019albert} is the most representative PLM that achieves competitive results with only 11M.

$\bullet$~{Small models~(50M< \#To <100M).} T5$_{12}$ is a small variant of T5~\cite{raffel2020exploring} which has only 6 encoder layers and 6 decoder layers. In addition, there are three parameter-efficient Transformer models that have similar parameters, namely MobileBERT~\citep{sun2020mobilebert}, DistilBERT~\citep{sanh2019distilbert} and TinyBERT~\citep{jiao2019tinybert}. We compare with these compressed models to show the benefit of scaling to deeper models over compressing large models to small variants.

$\bullet$~{Base models~(\#To > 100M).} We compare with BERT$_{12}$, XLNet$_{12}$, RoBERTa$_{12}$ and BART$_{12}$ for this category. Note that we only include the base variants that have similar model sizes in order to make a fair comparison. More details about the comparison with the strongest variants are described in Appendix~\ref{app-exp}.
% $\bullet$~\underline{Parameter-efficient Models.} We compare with parameter-efficient models based on compression techniques including MobileBERT~\citep{sun2020mobilebert}, DistilBERT~\citep{sanh2019distilbert} and TinyBERT~\citep{jiao2019tinybert}.

\label{app-exp}





\end{document}
