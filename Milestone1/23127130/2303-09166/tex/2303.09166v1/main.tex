\documentclass{article}

\usepackage{iclr2023_conference,times}
\iclrfinalcopy%


\usepackage[utf8]{inputenc}
\usepackage[T1]{fontenc}
\usepackage{hyperref}
\usepackage{url}
\usepackage{booktabs}
\usepackage{amsfonts}
\usepackage{nicefrac}
\usepackage{microtype}
\usepackage{xcolor}
\usepackage{amsmath}
\usepackage{amssymb}
\usepackage{amsmath}
\usepackage{amsthm}
\usepackage{mathtools}
\usepackage{bm}
\usepackage{graphicx}
\usepackage{subcaption}
\usepackage{cleveref}
\usepackage{pifont}
\usepackage{tabularx}
\usepackage{wrapfig}
\usepackage{thmtools}
\usepackage{thm-restate}
\usepackage{stmaryrd}
\usepackage{enumerate}
\usepackage{cancel}
\usepackage[normalem]{ulem}
\usepackage{marginnote}
\usepackage{wrapfig}
\usepackage{lipsum}
\usepackage[symbol]{footmisc}
\usepackage{fancyvrb}
\usepackage{dsfont}
\usepackage{tikz}
\usetikzlibrary{arrows}
\usetikzlibrary{bayesnet}


\theoremstyle{plain}
\newtheorem{theorem}{Theorem}
\newtheorem{assumption}{Assumption}
\newtheorem{definition}{Definition}
\DeclareMathOperator{\E}{\mathbb{E}}
\DeclareMathOperator{\InfoNCE}{\mathcal{L}_{\text{InfoNCE}}}
\DeclareMathOperator{\SymInfoNCE}{\mathcal{L}_{\text{SymInfoNCE}}}
\DeclareMathOperator{\AME}{\mathcal{L}_{\text{AlignMaxEnt}}}
\DeclareMathOperator{\SymAME}{\mathcal{L}_{\text{SymAlignMaxEnt}}}
\DeclareMathOperator{\simi}{sim}
\DeclareBoldMathCommand{\f}{\mathrm{f}}
\DeclareBoldMathCommand{\g}{\mathrm{g}}
\DeclareBoldMathCommand{\h}{\mathrm{h}}
\DeclareBoldMathCommand{\x}{\mathrm{x}}
\DeclareBoldMathCommand{\X}{X}
\DeclareBoldMathCommand{\y}{y}
\DeclareBoldMathCommand{\Y}{Y}
\DeclareBoldMathCommand{\z}{\mathrm{z}}
\DeclareBoldMathCommand{\Z}{Z}
\DeclareBoldMathCommand{\s}{\mathrm{s}}
\DeclareBoldMathCommand{\c}{\mathrm{c}}
\DeclareBoldMathCommand{\n}{\mathrm{n}}
\DeclareBoldMathCommand{\m}{\mathrm{m}}
\DeclareBoldMathCommand{\a}{\mathrm{a}}
\DeclareBoldMathCommand{\tc}{\mathrm{\tilde{c}}}
\DeclareBoldMathCommand{\ts}{\mathrm{\tilde{s}}}
\DeclareBoldMathCommand{\tm}{\mathrm{\tilde{m}}}
\DeclareBoldMathCommand{\tz}{\mathrm{\tilde{z}}}
\newcommand\given[1][]{\:#1\vert\:}
\newcommand\divfrom[1][]{\:#1\vert\vert\:}
\newcommand{\indep}{\rotatebox[origin=c]{90}{$\models$}}
\newcommand{\Indep}{\mathop{\perp\!\!\!\perp}\nolimits} 
\newcommand{\cmark}{\ding{51}}%
\newcommand{\xmark}{\ding{55}}%
\newcommand\labelAndRemember[2]
  {\expandafter\gdef\csname labeled:#1\endcsname{#2}\label{#1}#2}
\newcommand\recallLabel[1]
   {\csname labeled:#1\endcsname\tag{\ref{#1}}}
\newcommand\hcancel[2][black]{\setbox0=\hbox{$#2$}%
  \rlap{\raisebox{.45\ht0}{\textcolor{#1}{\rule{\wd0}{0.6pt}}}}#2}

\providecommand{\note}[1]{{{\color{lightgray} #1}}}

\hypersetup{%
    colorlinks,
    linkcolor={red!50!black},
    citecolor={blue!50!black},
    urlcolor={blue!80!black}
}

\interfootnotelinepenalty=10000

\title{Identifiability Results for\\ Multimodal Contrastive Learning}

\author{%
\footnotesize{Imant Daunhawer$^{1,\dagger}$, Alice Bizeul$^{1,2}$, Emanuele Palumbo$^{1,2}$, \textbf{Alexander Marx}$^{1,2,*}$ \& \textbf{Julia E. Vogt}$^{1,*}$} \vspace{0.25em}\\
\ \footnotesize{$^1$\,Department of Computer Science, ETH Zurich} \\
\ \footnotesize{$^2$\,ETH AI Center, ETH Zurich} \\ 
}


\newcommand{\fix}{\marginpar{FIX}}
\newcommand{\new}{\marginpar{NEW}}

\begin{document}

\maketitle

\begin{abstract}
  
Contrastive learning is a cornerstone underlying recent progress in multi-view
and multimodal learning, e.g., in representation learning with image/caption
pairs. While its effectiveness is not yet fully understood, a line of recent
work reveals that contrastive learning can invert the data generating process
and recover ground truth latent factors shared between views. In this work, we
present new identifiability results for multimodal contrastive learning,
showing that it is possible to recover shared factors in a more general setup
than the multi-view setting studied previously. Specifically, we distinguish
between the multi-view setting with one generative mechanism (e.g., multiple
cameras of the same type) and the multimodal setting that is characterized by
\emph{distinct} mechanisms (e.g., cameras and microphones). Our work
generalizes previous identifiability results by redefining the generative
process in terms of distinct mechanisms with modality-specific latent
variables. We prove that contrastive learning can block-identify latent factors
shared between modalities, even when there are nontrivial dependencies between
factors. We empirically verify our identifiability results with numerical
simulations and corroborate our findings on a complex multimodal dataset of
image/text pairs. Zooming out, our work provides a theoretical basis for
multimodal representation learning and explains in which settings multimodal
contrastive learning can be effective in practice.

\end{abstract}


\section{Introduction}

\renewcommand{\thefootnote}{\fnsymbol{footnote}}
\footnotetext[1]{Joint authorship. $^\dagger$Correspondence to: \texttt{dimant@ethz.ch}.}
\renewcommand{\thefootnote}{\arabic{footnote}}

Multimodal representation learning is an emerging field whose growth is fueled
by recent developments in weakly-supervised learning algorithms and by the
collection of suitable multimodal datasets. Multimodal data is characterized by
the \textit{co-occurence} of observations from two or more dependent data
sources, such as paired images and captions
\citep[e.g.,][]{Salakhutdinov2009,Shi2019,Radford2021}, and more generally,
multimodal observations are comprised of aligned measurements from different
types of sensors \citep{Baltrusaitis2019}. Co-occurrence is a form of
\textit{weak supervision} \citep{Shu2020,Locatello2020,Chen2020_weak}, in that
paired observations can be viewed as proxies (i.e., weak labels) for a shared
but unobserved ground truth factor.  Among suitable representation learning
methods for weakly supervised data, \textit{contrastive~learning}
\citep{Gutmann2010,Oord2018} stands out because it is designed to leverage
co-occurring observations from different views.  In practice, contrastive
learning achieves promising results for multi-view and multimodal learning---a
prominent example is the contribution of \texttt{CLIP} \citep{Radford2021} to
the groundbreaking advancements in text-to-image generation
\citep{Ramesh2021,Ramesh2022,Rombach2022,Saharia2022}.

Despite its empirical success, it is not sufficiently well understood what
explains the effectiveness of contrastive learning in practice. Recent works
attribute its effectiveness to the recovery of shared latent factors from the
underlying causal graph \citep{Gresele2019,Zimmermann2021,Kuegelgen2021}. From
the perspective of multi-view independent component analysis, it was shown that
contrastive learning can invert a nonlinear mixing function (i.e., a nonlinear
generative process) that is applied to a latent variable with mutually
independent components \citep{Gresele2019,Zimmermann2021}.  More recently,
\citet{Kuegelgen2021} show that contrastive learning can recover shared factors
up to block-wise indeterminacies, even if there are nontrivial causal and
statistical dependencies between latent components.  Collectively, these
results suggest that contrastive learning can identify parts of an unknown data
generating process from pairs of observations alone---even from
high-dimensional multi-view observations with nontrivial dependencies.  In our
work, we investigate the identifiability of shared latent factors in the
\emph{multimodal} setting.

We consider a generative process with modality-specific mixing functions and
modality-specific latent variables.  Our design is motivated by the inherent
heterogeneity of multimodal data, which follows naturally when observations are
generated by different types of sensors \citep{Baltrusaitis2019}. For example,
an agent can perceive its environment through distinct sensory modalities, such
as cameras sensing light or microphones detecting sound waves.  To model
information that is shared between modalities, we take inspiration from the
multi-view setting~\citep{Kuegelgen2021} and allow for nontrivial dependencies
between latent variables.  However, previous work only considers observations
of the same data type and assumes that the same input leads to the same output
across views.  In this work, we introduce a model with \textit{distinct
generative mechanisms}, each of which can exhibit a significant degree of
modality-specific variation.  This distinction renders the multimodal setting
more general compared to the multi-view setting considered by previous work.

In a nutshell, our work is concerned with \emph{identifiability for multimodal
representation learning} and focuses on \emph{contrastive~learning} as a
particular algorithm for which we derive identifiability results. In
\Cref{sec:preliminaries}, we cover relevant background on both topics,
identifiability and contrastive learning.  We then formalize the multimodal
generative process as a latent variable model (\Cref{sec:generative_process})
and prove that contrastive learning can block-identify latent factors shared
between modalities (\Cref{sec:identifiability_results}). We empirically verify
the identifiability results with fully controlled numerical simulations
(\Cref{subsec:numerical_experiment}) and corroborate our findings on a complex
multimodal dataset of image/text pairs (\Cref{subsec:imagetext_experiment}).
Finally, we contextualize related literature~(\Cref{sec:related_work}) and
discuss potential limitations and opportunities for future work
(\Cref{sec:discussion}).


\section{Preliminaries}
\label{sec:preliminaries}

\subsection{Identifiability}
\label{subsec:preliminaries_identifiability}

Identifiability lies at the heart of many problems in the fields of independent
component analysis~(ICA), causal discovery, and inverse problems, among others
\citep{Lehmann2006}. From the perspective of ICA, we consider the relation $\x
= \f(\z)$, where an observation $\x$ is generated from a mixing function $\f$
that is applied to a latent variable $\z$. The goal of ICA is to invert the
mixing function in order to recover the latent variable \emph{from observations
alone}. In many settings, full identifiability is impossible and certain
ambiguities might be acceptable. For example, identifiability might hold for a
subset of components (i.e., partial identifiability). Typical ambiguities
include permutation and element-wise transformations (i.e., component-wise
indeterminacy), or identifiability up to groups of latent variables (i.e.,
block-wise indeterminacy). In the general case, when $\f$ is a nonlinear
function, a landmark negative result states that the recovery of the latent
variable given i.i.d.~observations is fundamentally impossible
\citep{Hyvaerinen1999}. However, a recent line of pioneering works provides
identifiability results for the difficult nonlinear case under additional
assumptions, such as auxiliary variables
\citep{Hyvaerinen2017,Hyvaerinen2019,Khemakhem2020} or multiple views
\citep{Gresele2019,Locatello2020,Zimmermann2021}. 

Most relevant to our investigation are previous works related to
\emph{multi-view nonlinear ICA}
\citep{Gresele2019,Lyu2020,Locatello2020,Kuegelgen2021,Lyu2022}. Generally,
this line of work considers the following generative process: 
\begin{equation}
\label{eq:multiview_ica} \z \sim p_{\z}, \quad \x_1 = \f_1(\z), \quad \x_2 =
\f_2(\z), \end{equation} 
where a latent variable, or a subset of its components, is shared between
\emph{pairs} of observations $(\x_1, \x_2) \sim p_{\x_1, \x_2}$, where the two
views $\x_1$ and $\x_2$ are generated by two nonlinear mixing functions, $\f_1$
and $\f_2$ respectively.  Intuitively, a second view can resolve ambiguity
introduced by the nonlinear mixing, if both views contain a shared signal but
are otherwise sufficiently distinct \citep{Gresele2019}.  Previous works differ
in their assumptions on the mixing functions and dependence relations between
latent components.  The majority of previous work considers mutually
independent latent components \citep{Song2014,Gresele2019,Locatello2020} or
independent groups of shared and view-specific components
\citep{Lyu2020,Lyu2022}. Moreover, some of these works
\citep{Song2014,Gresele2019,Lyu2020,Lyu2022} consider view-specific%
\footnote{\label{footnote:view_specific}%
  Note that we define \emph{modality-specific} functions similar to
  the way \citet{Gresele2019}, \citet{Lyu2020}, and \citet{Lyu2022} define \emph{view-specific}
  functions. To clarify the distinction, we 
  generally assume that observations from different modalities are generated
  by distinct mechanisms $\f_1 \not = \f_2$ with modality-specific latent variables, and we treat the multi-view setting as a special
  case, where $\f_1 = \f_2$ without view-specific~latents.
}
mixing functions.  Venturing beyond the strict assumption of independent
(groups~of) components, \citet{Kuegelgen2021} consider additional causal and
statistical dependencies between latents and show that the subset of shared
components can be identified up to a block-wise indeterminacy.  Our work
considers heterogeneous modalities with causal and statistical dependencies
between latents. We prove that shared factors can be block-identified in a
novel setting with modality-specific mixing functions and modality-specific
latent variables.


\subsection{Contrastive Learning}

Contrastive learning \citep{Gutmann2010,Oord2018} is a self-supervised
representation learning method that leverages weak supervision in the form of
paired observations. On a high level, the method learns to distinguish
``positive'' pairs of encodings sampled from the joint distribution, against
``negative'' pairs sampled from the product of marginals. The popular InfoNCE
objective \citep{Oord2018} is defined as follows:
\begin{equation}\label{eq:info-nce-def}
  \InfoNCE(\g_1, \g_2) =
  \E_{\{\x_1^{i},\x_2^{i}\}_{i=1}^{K} \sim p_{\x_1, \x_2}} \left[
  - \sum_{i=1}^{K} \log \frac{\exp\{\simi(\g_1(\x_1^{i}), \g_2(\x_2^{i}))/\tau\}}{\sum_{j=1}^{K}\exp\{\simi(\g_1(\x_1^{i}), \g_2(\x_2^{j}))/\tau\}} \right] \;,
\end{equation}
where $\g_1$ and $\g_2$ are encoders for the first and second view, $\x_1$ and
$\x_2$ respectively.  It is common to use a single encoder $\g_1 = \g_2$ when
$\x_2$ is an augmented version of $\x_1$ or when two augmentations are sampled
from the same distribution to transform $\x_1$ and $\x_2$ respectively
\citep[e.g.,][]{Chen2020_simclr}. The set of hyperparameters consists of the
temperature $\tau$, a similarity metric $\text{sim}(\cdot,\cdot)$, and an
integer $K$ that controls the number of negative pairs ($K-1$) used for
contrasting. The objective has an information-theoretic interpretation as a
variational lower bound on the mutual information $I(\g_1(\x_1); \g_2(\x_2))$
\citep{Oord2018,Poole2019} and it can also be interpreted as the alignment of
positive pairs (numerator) with additional entropy regularization
(denominator), where the regularizer disincentivizes a degenerate solution in
which both encoders map to a constant \citep{Wang2020}.  Formally, when
instantiating the $\InfoNCE$ objective with $\tau = 1$ and $\simi(a, b) = -(a - b)^2$,
it asymptotically behaves like the objective
\begin{equation}\label{eq:info-nce-estimand-def}
    \AME(\g) = \mathbb{E}_{(\x_1,\x_2) \sim p_{\x_1,\x_2}} 
    \left[ \lVert \g(\x_1) - \g(\x_2) \rVert_2 \right] - H(\g(\x))  
\end{equation}
for a single encoder $\g$, when $K \to \infty$~\citep{Wang2020,Kuegelgen2021}.
 
In the setting with two heterogeneous modalities, it is natural to employ
separate encoders $\g_1 \not= \g_2$, which can represent different
architectures.  Further, it is common to use a symmetrized version of the
objective \citep[e.g., see][]{Zhang2020,Radford2021}, which can be obtained by
computing the mean of the loss in both directions:
\begin{equation}\label{eq:sym-info-nce-def}
  \SymInfoNCE(\g_1, \g_2) =
    \nicefrac{1}{2}\InfoNCE(\g_1,\g_2) + 
    \nicefrac{1}{2}\InfoNCE(\g_2,\g_1). 
\end{equation}
Akin to \Cref{eq:info-nce-estimand-def}, we can approximate the symmetrized
objective for $\tau = 1$ and $\simi(a, b) = -(a - b)^2$, with a large number of
negative samples ($K \to \infty$), as follows:
\begin{equation}\label{eq:sym-info-nce-estimand-def}
    \SymAME(\g_1,\g_2) = \mathbb{E}_{(\x_1,\x_2) \sim p_{\x_1,\x_2}} \left[ \lVert \g_1(\x_1) - \g_2(\x_2) \rVert_2 \right] - \nicefrac{1}{2} \left(H(\g_1(\x_1)) + H(\g_2(\x_2)) \right).
\end{equation}
Since the similarity measure is symmetric, the approximation of the alignment
term is identical for both $\InfoNCE(\g_1,\g_2)$ and $\InfoNCE(\g_2,\g_1)$.
Each entropy term is approximated via the denominator of the respective loss
term, which can be viewed as a nonparametric entropy estimator
\citep{Wang2020}.  For our experiments, we employ the finite-sample estimators
$\InfoNCE$ and $\SymInfoNCE$, while for our theoretical analysis we use the
estimand $\SymAME$ to derive identifiability results.


\section{Generative process}
\label{sec:generative_process}

\begin{figure}
\centering
\begin{tikzpicture}
  \node[obs] (X1) {$\x_1$};%
  \node[latent,above=of X1,xshift=-1cm] (M1) {$\m_1$};%
  \node[latent,above=of X1,xshift=0cm] (S1) {$\s$}; %
  \node[latent,above=of X1,xshift=1cm] (C) {$\c$}; %
  \node[obs, xshift=2cm] (X2) {$\x_2$};%
  \node[latent,above=of X2,xshift=0cm] (S2) {$\ts$}; %
  \node[latent,above=of X2,xshift=1cm] (M2) {$\m_2$};%
  \edge{M1,C,S1}{X1}%
  \edge{M2,C,S2}{X2}%
  \edge{C}{S1}%
  \edge[bend left=50]{S1}{S2}%
  
  \newlength{\grouppad}
  \setlength{\grouppad}{0.6cm}
  \newlength{\groupshift}
  \setlength{\groupshift}{0.1cm}
  \coordinate[below = \grouppad of M1.center, yshift=-\groupshift]  (a1);
  \coordinate[above = \grouppad of M1.center, yshift=-\groupshift]  (a2);
  \coordinate[above = \grouppad of C.center, yshift=-\groupshift] (a3);
  \draw[dashed, black] (a1) arc (-90:-270:\grouppad)
		-- (a3) arc (90:-90:\grouppad)
		-- (a1);
  \draw[dashed] (a2) -- ++(90:0.2) node[above, outer sep = 0pt, inner sep = 0pt, text = black] {$\z_1$};
  \coordinate[below = \grouppad of C.center, yshift=\groupshift]  (b1);
  \coordinate[above = \grouppad of M2.center, yshift=\groupshift]  (b2);
  \coordinate[above = \grouppad of M2.center, yshift=\groupshift] (b3);
  \draw[dashed, lightgray] (b1) arc (-90:-270:\grouppad)
		-- (b3) arc (90:-90:\grouppad)
		-- (b1);
\draw[lightgray, dashed] (b2) -- ++(90:0.2) node[above, outer sep = 0pt, inner sep = 0pt, text = lightgray] {$\z_2$};
\end{tikzpicture}
\caption{Illustration of the multimodal generative process. Latent variables are denoted by clear nodes and observations by shaded nodes. We partition the latent space into $\z_1 = (\c, \s, \m_1)$ and $\z_2 = (\tc, \ts, \m_2)$, where $\tc = \c$ almost everywhere (\Cref{as:content}) and hence we consider only $\c$. Further, $\ts$ is a perturbed version of $\s$ (\Cref{as:style}) and $\m_1$, $\m_2$ are modality-specific variables. The observations $\x_1$ and $\x_2$ are generated by two distinct mixing functions $\f_1 \not = \f_2$, which are applied to the subsets of latent variables $\z_1$ and $\z_2$ respectively.}
\label{fig:lvm}
\end{figure}

In the following, we formulate the multimodal generative process as a latent
variable model (\Cref{sec:lvm}) and then specify our technical
assumptions on the relation between modalities (\Cref{sec:relation}).

\subsection{Latent variable model}
\label{sec:lvm}

On a high level, we assume that there exists a continuous random variable $\z$
that takes values in the latent space $\mathcal{Z} \subseteq \mathbb{R}^n$,
which contains all information to generate observations of both modalities.%
\footnote{%
  Put differently, we assume that all observations lie on a continuous manifold, which can have
  much smaller dimensionality than the observation space of the respective modality.
} 
Moreover, we assume that $\z = (\c,\s,\m_1,\m_2)$ can be uniquely
partitioned into four disjoint parts:
\begin{enumerate}[\itshape(i)]
  \item an invariant part $\c$ which is always shared across modalities,
    and which we refer to as \emph{content};
  \item a variable part $\s$ which may change across modalities, and which we
    refer to as \emph{style};
  \item two modality-specific parts, $\m_1$ and $\m_2$, each of which is unique
    to the respective modality.
\end{enumerate}

Let $\z_1 = (\c, \s, \m_1)$ and $\z_2 = (\tc, \ts, \m_2)$, where $\tc = \c$
almost everywhere and $\ts$ is generated by perturbations that are specified in
\Cref{sec:relation}.  Akin to multi-view ICA (Equation~\ref{eq:multiview_ica}),
we define the generative process for modalities $\x_1$ and $\x_2$ as follows:
\begin{equation}
\label{eq:generative-process}
  \z \sim p_{\z}, \quad \x_1 = \f_1(\z_1), \quad \x_2 = \f_2(\z_2),
\end{equation}
where $\f_1: \mathcal{Z}_1 \to \mathcal{X}_1$ and $\f_2: \mathcal{Z}_2 \to
\mathcal{X}_2$ are two smooth and invertible mixing functions with smooth
inverse (i.e., diffeomorphisms) that generate observations $\x_1$ and $\x_2$
taking values in $\mathcal{X}_1 \subseteq \mathbb{R}^{d_1}$ and $\mathcal{X}_2
\subseteq \mathbb{R}^{d_2}$ respectively.  Generally, we assume that
observations from different modalities are generated by distinct mechanisms
$\f_1 \not = \f_2$ that take modality-specific latent variables as input. As
for the multi-view setting~\citep{Kuegelgen2021}, the considered generative
process  \emph{goes beyond the classical ICA setting} by allowing for
statistical dependencies within blocks of variables (e.g., between dimensions
of $\c$) and also for causal dependencies from content to style, as illustrated
in \Cref{fig:lvm}. We assume that $p_\z$ is a smooth density that factorizes as
$p_\z = p_\c \, p_{\s|\c} \, p_{\m_1}p_{\m_2}$ in the causal setting, and as
the product of all involved marginals when there is no causal dependence from
$\c$ to $\s$.

The outlined generative process is fairly general and it applies to a wide
variety of practical settings.  The content invariance describes a shared
phenomenon that is not directly observed but manifests in the observations from
both modalities. Style changes describe shared influences that are not robust
across modalities, e.g., non-invertible transformations such as data
augmentations, or non-deterministic effects of an unobserved confounder.
Modality-specific factors can be viewed as variables that describe the inherent
heterogeneity of each modality (e.g., background noise). 


\subsection{Relation between modalities}
\label{sec:relation}

Next, we specify our assumptions on the relation between modalities by defining
the conditional distribution $p_{\z_2|\z_1}$, which describes the relation
between latent variables $\z_1$ and $\z_2$, from which observations $\x_1$ and
$\x_2$ are generated via \Cref{eq:generative-process}.  Similar to previous
work in the multi-view setting~\citep{Kuegelgen2021}, we assume that content is
invariant, i.e., $\tc = \c$ almost everywhere (\Cref{as:content}), and that
$\ts$ is a perturbed version of $\s$ (\Cref{as:style}). To state our
assumptions for the multimodal setting, we also need to consider the
modality-specific latent variables.

\begin{assumption}[Content-invariance]
\label{as:content}
The conditional density $p_{\z_2|\z_1}$ over $\mathcal{Z}_2 \times \mathcal{Z}_1$
takes the form
\begin{equation} \label{eq:z2-given-z1}
	p_{\z_2|\z_1}(\z_2|\z_1) = \delta(\tc - \c) p_{\ts|\s}(\ts|\s) p_{\m_2}(\m_2)
\end{equation}
for some continuous density $p_{\ts|\s}$ on $\mathcal{S} \times \mathcal{S}$,
where $\delta( \cdot )$ is the Dirac delta function, i.e., $\tc = \c$~a.e. 
\end{assumption}

To fully specify $p_{\z_2|\z_1}$, it remains to define the style changes, which
are described by the conditional distribution $p_{\ts|\s}$. There are several
justifications for modeling such a stochastic relation between $\s$ and
$\ts$~\citep{Zimmermann2021,Kuegelgen2021}; one could either consider $\ts$ to
be a noisy version of $\s$, or consider $\ts$ to be the result of an
augmentation that induces a soft intervention on $\s$.%
\footnote{%
  Note that the asymmetry between $\z_1$ and $\z_2$ (or between $\s$ and $\ts)$
  is not strictly required.  We chose to write $\z_2$ as a perturbation of
  $\z_1$ to simplify the notation and for consistency with previous work.
  Instead, we could model \emph{both} $\z_1$ and $\z_2$ via perturbations of
  $\z$, as described in \Cref{app:symmetric_generative_process}.
}

\begin{assumption}[Style changes]
\label{as:style}
  Let $\mathcal{A}$ be the powerset of style variables $\{1, \dots, n_s \}$ and
  let $p_A$ be a distribution on $\mathcal{A}$. Then, the style conditional
  $p_{\ts|\s}$ is obtained by conditioning on a set $A$:
  \begin{equation}
      p_{\ts|\s}(\ts|\s) = \sum_{A \in \mathcal{A}} p_A(A) 
      \left( \delta(\ts_{A^c}-\s_{A^c}) p_{\ts_A|\s_A}(\ts_A|\s_A) \right)
  \end{equation}
  where $p_{\ts_A|\s_A}$ is a continuous density on $\mathcal{S}_A \times
  \mathcal{S}_A$, $\mathcal{S}_A \subseteq \mathcal{S}$ denotes the subspace of
  changing style variables specified by $A$, and $A^c = \{ 1, \dots, n_s \}
  \backslash A$ denotes the complement of $A$.
  Further, for any style variable $l \in \{ 1, \dots, n_s \}$, there exists a set
  $A \subseteq \{ 1, \dots, n_s \}$ with $l \in A$, s.t.
  \begin{enumerate}[\itshape(i)]
      \item $p_{A}(A) > 0$,
      \item $p_{\ts_A|\s_A}$ is smooth w.r.t.~both $\s_A$ and $\ts_A$, and
      \item for any $\s_A$, $p_{\ts_A|\s_A}( \cdot | \s_A) > 0$, in some open
        non-empty subset containing $\s_A$.
  \end{enumerate}
\end{assumption}
Intuitively, to generate a pair of observations $(\x_1, \x_2)$, we
independently flip a biased coin for each style dimension to select a subset of
style features $A \subseteq \{ 1, \dots, n_s \}$, which are jointly perturbed
to obtain $\ts$. Condition \emph{(i)} ensures that every style dimension has a
positive probability to be perturbed,%
\!\footnote{%
  If a style variable would be perturbed with zero probability, it would be a
  content variable.
} 
while \emph{(ii)} and \emph{(iii)} are technical smoothness
conditions that will be used for the proof of \Cref{th:main}.

Summarizing, in this section we have formalized the multimodal generative
process as a latent variable model (\Cref{sec:lvm}) and specified our
assumptions on the relation between modalities via the conditional distribution
$p_{\z_1 | \z_2}$ (\Cref{sec:relation}). Next, we segue into the topic of
representation learning and show that, for the specified generative process,
multimodal contrastive learning can identify the content factors up to a
block-wise indeterminacy.


\section{Identifiability Results}
\label{sec:identifiability_results}

First, we need to define block-identifiability \citep{Kuegelgen2021} for the
multimodal setting in which we consider modality-specific mixing functions and
encoders. In the following, $n_c$ denotes the number of content variables and
the subscript $1{:}n_c$ denotes the subset of content dimensions (indexed from
1 to $n_c$ w.l.o.g.).

\begin{definition}[Block-identifiability]
\label{def:block-identified}
  The true content partition $\c = \f_1^{-1}(\x_1)_{1:n_c} = \f_2^{-1}(\x_2)_{1:n_c}$
  is block-identified by a function $\g_i: \mathcal{X}_i \to \mathcal{Z}_i$, 
  with $i \in \{ 1,2 \}$, if there exists an invertible function
  $\h_i: \mathbb{R}^{n_c} \to \mathbb{R}^{n_c}$, s.t.~for the inferred content
  partition $\hat{\c}_i = \g_i(\x_i)_{1:n_c}$ it holds that $\hat{\c}_i = \h_i(\c)$.
\end{definition}
It is important to note that block-identifiability does not require the
identification of \emph{individual} factors, which is the goal in multi-view
nonlinear ICA \citep{Gresele2019,Locatello2020,Zimmermann2021,Klindt2021} and
the basis for strict definitions of disentanglement
\citep{Bengio2013,Higgins2018,Shu2020}. Instead, our goal is to isolate the
group of invariant factors (i.e., the content partition) from the remaining
factors of variation in the data.

Specifically, our goal is to show that contrastive learning can block-identify
the content variables for the multimodal setting described in
\Cref{sec:generative_process}. We formalize this in \Cref{th:main} and thereby
relax the assumptions from previous work by allowing for distinct generating
mechanisms $\f_1 \neq \f_2$ and additional modality-specific latent variables.
\begin{restatable}{theorem}{thmain}
\label{th:main}
  Assume the data generating process described in Sec.~\ref{sec:lvm}, i.e.~data
  pairs $(\x_1, \x_2)$ generated from \Cref{eq:generative-process} with
  $p_{\z_1} = p_{\z \backslash \{ \m_2 \} }$ and $p_{\z_2|\z_1}$ as defined in
  Assumptions~\ref{as:content} and~\ref{as:style}. Further, assume that
  $p_{\z}$ is a smooth and continuous density on $\mathcal{Z}$ with $p_{\z}(\z) > 0$
  almost everywhere. Let $\g_1: \mathcal{X}_1 \to (0,1)^{n_c}$ and 
  $\g_2: \mathcal{X}_2 \to (0,1)^{n_c}$ be smooth functions that minimize $\SymAME$ as
  defined in Eq.~(\ref{eq:sym-info-nce-estimand-def}). Then, $\g_1$ and
  $\g_2$ block-identify the true content variables in the sense of
  Def.~\ref{def:block-identified}. 
\end{restatable}

A proof of \Cref{th:main} is provided in \Cref{app:proof_of_thmain}.
Intuitively, the result states that contrastive learning can identify the
content variables up to a block-wise indeterminacy. Similar to previous work,
the result is based on the optimization of the asymptotic form of the
contrastive loss (Equation~\ref{eq:sym-info-nce-estimand-def}). Moreover,
\Cref{th:main} assumes that the number of content variables is known or that it
can be estimated (e.g., with a heuristic like the elbow method). We address the
question of selecting the encoding size with dimensionality ablations
throughout our experiments. In \Cref{sec:discussion}, we will return to the
discussion of the assumptions in the context of the experimental results.


\section{Experiments}

The goal of our experiments is to test whether contrastive learning can
block-identify content in the multimodal setting, as described by
\Cref{th:main}. First, we verify identifiability in a fully controlled setting
with numerical simulations (\Cref{subsec:numerical_experiment}).  Second, we
corroborate our findings on a complex multimodal dataset of image/text pairs
(\Cref{subsec:imagetext_experiment}). The code is provided in our github
repository.%
\footnote{\url{https://github.com/imantdaunhawer/multimodal-contrastive-learning}.}

\subsection{Numerical Simulation}
\label{subsec:numerical_experiment}

\begin{table}[t]
  \centering
  \begin{subtable}{.45\textwidth}
    \centering
    \resizebox{!}{1.2cm}{%
      \small
      \begin{tabular}{ccccc}
        \toprule
        \multicolumn{3}{c}{\textbf{Generative process}} & \multicolumn{2}{c}{$\bm{R^2}$ \textbf{(nonlinear)}}  \\
        \cmidrule(r){1-3}\cmidrule(r){4-5}
        \textbf{p(chg.)} & \textbf{Stat.} & \textbf{Cau.} & \textbf{Content $\c$} & \textbf{Style $\s$} \\
        \midrule
       1.0 & \xmark & \xmark  & $\textbf{1.00} \pm 0.00$ & $0.00 \pm 0.00$ \\
       0.75 & \xmark & \xmark & $\textbf{0.99} \pm 0.01$ & $0.00 \pm 0.00$ \\
       0.75 & \cmark & \xmark & $\textbf{0.99} \pm 0.00$ & $0.52 \pm 0.09$ \\
       0.75 & \xmark & \cmark & $\textbf{1.00} \pm 0.00$ & $\textbf{0.79} \pm 0.04$ \\
       0.75 & \cmark & \cmark & $\textbf{0.99} \pm 0.01$ & $\textbf{0.81} \pm 0.04$ \\
        \bottomrule
      \end{tabular}
    }  %
    \caption{Original setting}
  \label{subtab:numerical_original}
  \end{subtable}
  \begin{subtable}{.54\textwidth}
    \centering
    \resizebox{!}{1.2cm}{%
      \small
      \begin{tabular}{cccccc}
        \toprule
        \multicolumn{3}{c}{\textbf{Generative process}} & \multicolumn{3}{c}{$\bm{R^2}$ \textbf{(nonlinear)}}  \\
        \cmidrule(r){1-3}\cmidrule(r){4-6}
        \textbf{p(chg.)} & \textbf{Stat.} & \textbf{Cau.} & \textbf{Content $\c$} & \textbf{Style $\s$} & \textbf{Modality $\m_i$}\\
        \midrule
       1.0 & \xmark & \xmark  & $\textbf{0.99} \pm 0.00$ & $0.00 \pm 0.00$ & $0.00 \pm 0.00$ \\
       0.75 & \xmark & \xmark & $\textbf{1.00} \pm 0.00$ & $0.00 \pm 0.00$ & $0.00 \pm 0.00$ \\
       0.75 & \cmark & \xmark & $\textbf{0.95} \pm 0.01$ & $0.56 \pm 0.23$ & $0.00 \pm 0.00$ \\
       0.75 & \xmark & \cmark & $\textbf{0.98} \pm 0.00$ & $\textbf{0.87} \pm 0.04$ & $0.00 \pm 0.00$ \\
       0.75 & \cmark & \cmark & $\textbf{0.95} \pm 0.03$ & $\textbf{0.89} \pm 0.07$ & $0.00 \pm 0.00$ \\
        \bottomrule
      \end{tabular}
    }  %
    \caption{Multimodal setting}
  \label{subtab:numerical_multimodal}
  \end{subtable}
  \caption{%
    Results of the numerical simulations. We compare the original setting ($\f_1
    = \f_2$, left table) with the multimodal setting ($\f_1 \not= \f_2$, right
    table). Only the multimodal setting includes modality-specific latent
    variables.  Each row presents the results of a different setup with varying
    style-change probability p(chg.) and possible statistical (Stat.) and/or
    causal (Caus.) dependencies. Each value denotes the $R^2$ coefficient of
    determination (averaged across 3 seeds) for a nonlinear regression model that
    predicts the respective ground truth factor ($\c, \s$, or $\m_i$) from the
    learned representation.
  }
\label{tab:numerical_m1m2}
\end{table}

We extend the numerical simulation from \citet{Kuegelgen2021} and implement the
multimodal setting using modality-specific mixing functions ($\f_1 \not= \f_2$)
with modality-specific latent variables. The numerical simulation allows us to
measure identifiability with full control over the generative process.  The
data generation is consistent with the generative process described in
\Cref{sec:generative_process}.  We sample $\c \sim \mathcal{N}(0,
\Sigma_{\c})$, $\m_i \sim \mathcal{N}(0, \Sigma_{\m_i})$, and $\s \sim
\mathcal{N}(\a + B\c, \Sigma_{\s})$.  Statistical dependencies within blocks
(e.g., among components of $\c$) are induced by non-zero off-diagonal entries
in the corresponding covariance matrix (e.g., in $\Sigma_{\c}$).  To induce a
causal dependence from content to style, we set $a_i, B_{ij} \sim
\mathcal{N}(0, 1)$; otherwise, we set $a_i, B_{ij} = 0$.  For style changes,
Gaussian noise is added with probability $\pi$ independently for each style
dimension: $\ts_i = \s_i + \epsilon$, where $\epsilon \sim \mathcal{N}(0,
\Sigma_{\epsilon})$ with probability $\pi$.  We generate the observations $\x_1
= \f_1(\c, \s, \m_1)$ and  $\x_2 = \f_2(\c, \ts, \m_2)$ using two
\emph{distinct} nonlinear mixing functions, i.e, for each $i \in \{1, 2\}$,
$\f_i: \mathbb{R}^d \to \mathbb{R}^d$ is a separate, invertible 3-layer~MLP
with LeakyReLU activations.  We train the encoders for 300,000 iterations using
the symmetrized InfoNCE objective (Equation~\ref{eq:sym-info-nce-def}) and the
hyperparameters listed in \Cref{sec:app-details-to-experimental-setting}.  We
evaluate block-identifiability by predicting the ground truth factors from the
learned representation using kernel ridge regression and report the $R^2$
coefficient of determination on holdout data.

\paragraph{Results}
We compare the original setting ($\f_1 = \f_2$,
\Cref{subtab:numerical_original}) with the multimodal setting ($\f_1 \not=
\f_2$, \Cref{subtab:numerical_multimodal}) and find that content can be
block-identified in \emph{both} settings, as the $R^2$ score is close to one
for the prediction of content, and quasi-random for the prediction of style and
modality-specific information.  Consistent with previous work, we observe that
some style information can be predicted when there are statistical and/or
causal dependencies; this is expected because statistical dependencies decrease
the effective dimensionality of content, while the causal dependence $\c \to
\s$ makes style partially predictable from the encoded content information.
Overall, the results of the numerical simulation are consistent with our
theoretical result from \Cref{th:main}, showing that contrastive learning can
block-identify content in the multimodal setting.


\subsection{Image/text pairs}
\label{subsec:imagetext_experiment}

Next, we test whether block-identifiability holds in a more realistic setting
with image/text pairs---two complex modalities with distinct generating
mechanisms. We extend the \emph{Causal3DIdent} dataset
\citep{Kuegelgen2021,Zimmermann2021}, which allows us to measure and control
the ground truth latent factors used to generate complex observations.  We use
\textit{Blender} \citep{Blender} to render high-dimensional images that depict
a scene with a colored object illuminated by a differently colored spotlight
and positioned in front of a colored background.  The scene is defined by 11
latent factors: the shape of the object (7 classes), position of the object
($x, y, z$ coordinates), orientation of the object ($\alpha, \beta, \gamma$
angles), position of the spotlight ($\theta$ angle), as well as the color of
the object, background, and spotlight respectively (one numerical value for
each).

\begin{wrapfigure}{r}{0.43\textwidth} 
\vspace{-18pt}
  \begin{center}
    \includegraphics[width=0.43\textwidth]{figures/image_main.png}
    \vspace{-15pt}
    \caption{Examples of image/text pairs.}
    \label{fig:imagetext_examples}
  \end{center}
\vspace{-13pt}
\end{wrapfigure} 

\paragraph{\emph{Multimodal3DIdent}}
We extend the \emph{Causal3DIdent} dataset to the multimodal setting as
follows. We generate textual descriptions from the latent factors by adapting
the text rendering from the \emph{CLEVR} dataset \citep{Johnson2017}. Each
image/text pair shares information about the shape of the object (cow,
teapot,~etc.) and its position in the scene (e.g.,~bottom-right). For each
position factor, we use three clearly discernable values (top/center/bottom;
left/center/right), which can be described in text more naturally than
coordinates.  While shape and position are always shared (i.e., content)
between the paired image and text, the color of the object is causally
influenced by position and is stochastically shared (i.e., style). For the
object color, we use a continuous hue value, whereas for the text we match the
RGB value with the nearest value from a given palette (i.e., a list of named
colors, such as brown, beige, olive, etc.). The color palette is randomly
sampled from a set of three palettes to ensure the object color depicted in the
image does not uniquely determine the color described in the text. As
modality-specific factors for the images, we have object rotation, spotlight
position, and background color, while for the textual descriptions, we follow
\citet{Johnson2017} and use 5 different types of phrases to introduce
modality-specific variation. Examples of image/text pairs are shown in
\Cref{fig:imagetext_examples}. Further details about the dataset are provided
in \Cref{sec:app-details-to-experimental-setting}.

We train the encoders for 100,000 iterations using the symmetrized InfoNCE
objective (Equation~\ref{eq:sym-info-nce-def}) and the hyperparameters listed
in \Cref{sec:app-details-to-experimental-setting}. For the image encoder we use
a ResNet-18 architecture \citep{He2016} and for the text we use a convolutional
network. As for the numerical simulation, we evaluate block-identifiability by
predicting the ground truth factors from the learned representation. For
continuous factors, we  use kernel ridge regression and report the $R^2$ score,
whereas for discrete factors we report the classification accuracy of an MLP
with a single~hidden~layer. 

\paragraph{Results}
\Cref{fig:imagetext_results} presents the results on \emph{Multimodal3DIdent}
with a dimensionality ablation, where we vary the size of the encoding of the
model. Content factors (object position and shape) are always encoded well,
unless the encoding size is too small (i.e., smaller than 3-4 dimensions). When
there is sufficient capacity, style information (object color) is also encoded,
partly because there is a causal dependence from content to style and partly
because of the excess capacity, as already observed in previous work.
Image-specific information (object rotation, spotlight position, background
color) is mostly discarded, independent of the encoding size. Text-specific
information (phrasing) is encoded to a moderate degree (48--80\% accuracy),
which we attribute to the fact that phrasing is a discrete factor that violates
the assumption of continuous latents. This hints at possible limitations in the
presence of discrete latent factors, which we further investigate in
\Cref{app:additional_experiments} and discuss in \Cref{sec:discussion}.
Overall, our results suggest that contrastive learning can block-identify
content factors in a complex multimodal setting with image/text pairs.

\begin{figure*}[t]
    \centering
    \begin{subfigure}[t]{0.49\textwidth}
        \includegraphics[width=1.0\textwidth]{figures/mc3di_caus_img_nonlin.pdf}
        \caption{Prediction of image factors}
        \label{subfig:imagetext_results_image}
    \end{subfigure}%
    \hfill
    \begin{subfigure}[t]{0.49\textwidth}
        \centering
        \includegraphics[width=1.0\textwidth]{figures/mc3di_caus_txt_nonlin.pdf}
        \caption{Prediction of text factors}
        \label{subfig:imagetext_results_text}
    \end{subfigure}
    \caption{Results on \emph{Multimodal3DIdent} as a function of the encoding
      size of the model. We assess the nonlinear prediction of ground truth
      image factors (left subplot) and text factors (right subplot) to quantify
      how well the learned representation encodes the respective factors.
      Content factors are denoted in bold and style factors in italic. Along
      the x-axis, we vary the encoding size, i.e., the output dimensionality of
      the model. We measure the prediction performance in terms of the $R^2$
      coefficient of determination for continuous factors and classification
      accuracy for discrete factors respectively. Each point denotes the
      average across three seeds and bands show one standard deviation. 
    }
\label{fig:imagetext_results}
\end{figure*}


\section{Related work}
\label{sec:related_work}

\paragraph{Multi-view nonlinear ICA}
The goal of multi-view nonlinear ICA is to identify latent factors shared
between different views, as described in
\Cref{subsec:preliminaries_identifiability}.  There is a thread of works
\citep{Gresele2019,Locatello2020} that recover the latent variable up to a
component-wise indeterminacy in a setting with mutually independent latent
components, or up to block-wise inderterminacies in the case of independent
groups of shared and view-specific components \citep{Lyu2020,Lyu2022}.  Beyond
the assumption of independent (groups of) components, there is a line of works
\citep{Kuegelgen2021,Kong2022} that partition the latent space into blocks of
invariant and blocks of changing components and show that the invariant
components can be identified up to a block-wise indeterminacy, even when there
are nontrivial dependencies between latent components.  Our work advances in
this direction and considers heterogeneous modalities with nontrivial
statistical and causal dependencies between latents. We prove that shared
factors can be block-identified in a novel setting with modality-specific
mixing functions and modality-specific latent variables.

\paragraph{Multimodal representation learning}
Multimodal representation learning seeks to integrate information from
heterogeneous sources into a joint representation
\citep{Baltrusaitis2019,Guo2019}.  There is a myriad of methods designed to
learn representations of multimodal data either directly or indirectly. Among
methods that learn representations indirectly, there are multimodal
autoencoders \citep{Ngiam2011,Geng2022,Bachmann2022,Aghajanyan2022} and a large
variety of multimodal generative models
\citep[e.g.,][]{Suzuki2016,Wu2018,Shi2019,Huang2018,Tsai2019_transformer,Ramesh2021}
that learn representations by backpropagation of different forms of
reconstruction and/or masked prediction error.  A more direct approach is taken
by decoder-free methods that maximize the similarity between the encodings of
different modalities. This class of methods includes nonlinear canonical
correlation analysis
\citep{Akaho2001,Bach2002,Andrew2013,Wang2016,Tang2017,Karami2021} as well as
multi-view and multimodal contrastive learning
\citep{Tian2019_cmc,Bachmann2019,Federici2020,Tsai2021,Radford2021,Poklukar2022}.
While all of the named methods aim to integrate information across modalities,
they do not answer the underlying question of identifiability, which our work
seeks to address. 



\section{Discussion}
\label{sec:discussion}

\paragraph{Implications and scope}
We have shown that contrastive learning can block-identify shared factors in
the multimodal setting. Numerical simulations
(\Cref{subsec:numerical_experiment}) verify our main theoretical result
(\Cref{th:main}), showing that contrastive learning block-identifies content
information (\Cref{def:block-identified}), when the size of the encoding
matches the number of content factors.  Experiments on a complex dataset of
image/text pairs corroborate that contrastive learning can isolate content in a
more realistic setting and even under some violations of the assumptions
underlying \Cref{th:main}. In \Cref{app:additional_experiments}, we include
further experiments that test violations with discrete factors and
dimensionality ablations that examine the robustness and sample complexity. %
More generally, we observe that contrastive learning encodes invariant
information (i.e., content) very well across all settings.  When there is
sufficient capacity, stochastically shared information (i.e., style) is encoded
to a moderate degree, but without affecting the prediction of invariant
information.  For practice, our results suggest that contrastive learning
without capacity constraints can encode \emph{any} shared factor, irregardless
of whether the factor is truly invariant across modalities or if its effect on
the observations is confounded by noise or other factors.  This is in line with
the information-theoretic view \citep{Oord2018,Poole2019}, i.e., that
contrastive learning maximizes the mutual information between
representations---a measure of mutual dependence that quantifies \emph{any}
information that is shared. Our results demonstrate that the size of the
encoding can be reduced to learn a representation that recovers invariant
information, as captured by the notion of block-identifiability. In practice,
this can be leveraged for representation learning in settings of
content-preserving distribution shifts \citep{Mitrovic2021,Federici2021}, where
information relevant for a downstream task~remains~unchanged.

\paragraph{Limitations and outlook}
First, \Cref{th:main} suggests that only \emph{invariant} factors can be
block-identified. However, in practice, there can be pairs of observations for
which the invariance is inadvertently violated, e.g., due to measurement
errors, occlusions, or other mistakes in the data collection.  On the one hand,
such a violation can be viewed as a mere artifact of the data collection and
could be managed via interventions on the generative process, e.g., actions in
reinforcement learning  \citep{Lippe2022,Brehmer2022,Ahuja2022,Lachapelle2022}.
On the other hand, violations of the content-invariance blur the line between
content and style factors and it would be interesting to study identifiability
in a more general setting with \emph{only} stochastically shared factors.
Second, \Cref{th:main} assumes that the number of content factors is known or
that it can be estimated. In practice, this might not be a significant
limitation, since the number of content factors can be viewed as a single
hyperparameter \citep[e.g.,][]{Locatello2020}, though the design of suitable
heuristics is an interesting research direction. We explore the idea of
estimating the number of content factors in \Cref{app:additional_experiments}
\Cref{fig:model_selection}. Third, \Cref{th:main} assumes that all latent
factors are continuous. While this assumption prevails in related work
\citep{Hyvaerinen1999,Hyvarinen2016,Hyvaerinen2019,Gresele2019,Locatello2019,Locatello2020,Zimmermann2021,Kuegelgen2021,Klindt2021},
our results in \Cref{subfig:imagetext_results_text} indicate that in the
presence of discrete factors, some style or modality-specific information can
be encoded.  In \Cref{app:additional_experiments} \Cref{fig:discrete}, we
provide numerical simulations that support these findings.  Finally, our model
can be extended to more than two modalities---a setting for which there are
intriguing identifiability results \citep{Gresele2019,Schoelkopf2016} as well
as suitable learning objectives \citep{Tian2019_cmc,Lyu2022}. Summarizing, the
described limitations mirror the assumptions on the generative process
(\Cref{sec:generative_process}), which may be relaxed in future work.


\section{Conclusion}

We addressed the problem of identifiability for multimodal representation
learning and showed that contrastive learning can block-identify latent factors
shared between heterogeneous modalities. We formalize the multimodal generative
process as a novel latent variable model with modality-specific generative
mechanisms and nontrivial statistical and causal dependencies between latents.
We prove that contrastive learning can identify shared latent factors up to a
block-wise indeterminacy and therefore isolate invariances between modalities
from other changeable factors.  Our theoretical results are corroborated by
numerical simulations and on a complex multimodal dataset of image/text pairs.
More generally, we believe that our work will help in shaping a theoretical
foundation for multimodal representation learning and that further relaxations
of the presented generative process offer rich opportunities for future work.


\newpage
\section*{Acknowledgements}

ID was supported by the SNSF grant \textit{\#200021-188466}. EP was supported
by the grant \textit{\#2021-911} of the Strategic Focal Area ``Personalized
Health and Related Technologies (PHRT)'' of the ETH Domain (Swiss Federal
Institutes of Technology). Experiments were performed on the ETH~Zurich
Leonhard cluster. Special thanks to Kieran~Chin-Cheong for his support in the
early stages of the project as well as to Luigi~Gresele and Julius~von~Kügelgen
for helpful discussions.

\section*{Reproducibility statement}

For our theoretical statements, we provide detailed derivations and state the
necessary assumptions. The generative process is specified in
\Cref{sec:generative_process} and the assumptions for block-identifiability are
referenced in \Cref{th:main}. We test violations of the key assumptions with
suitable experiments (dimensionality ablations; discrete latent factors) and
discuss the limitations of our work in \Cref{sec:discussion}. Further, we
empirically verify our theoretical results with numerical simulations and on
complex multimodal data. To ensure empirical reproducibility, the results of
every experiment were averaged over multiple seeds and are reported with
standard deviations. Information about implementation details, hyperparameter
settings, and evaluation metrics are included in
\Cref{sec:app-details-to-experimental-setting}. Additionally, we publish the
code to reproduce the experiments.

\bibliographystyle{apalike}
\bibliography{references}

\section{Appendix for Proofs}

\paragraph{Proof of Theorem \ref{thm:main}.}

\begin{proof}
\label{proof:main}
Our proof has two steps. In Step 1, we will show that SimCLR is equivalent to minimizing the cross entropy loss defined in Eqn.~(\ref{eqn:cross-entropy}). 
In Step 2, we will show  that minimizing the cross-entropy loss 
is equivalent to spectral clustering on $\bfpi$. 
Combining the two steps together, we have proved our theorem. 

\textbf{Step 1: } SimCLR is equivalent to minimizing the cross entropy loss.

The cross-entropy loss takes expectation over 
$\bfW_\bfX\sim \mathbb{P}(\cdot ; \bfpi)$, 
which means $\bfW_\bfX$ has exactly one non-zero entry in each row $i$. By Lemma~\ref{lem:multinomial}, we know every row $i$ of $\bfW_\bfX$ is independent of other rows. Moreover, 
$\bfW_{\bfX,i}\sim \mathcal{M}(1, \bfpi_i/\sum_j \bfpi_{i,j})=\mathcal{M}(1, \bfpi_i)$, because $\bfpi_i$ itself is a probability distribution.
Similarly, we know $\bfW_\bfZ$ also has the row-independent property by sampling over $\mathbb{P}(\cdot;\bfK_\bfZ)$.
Therefore, by Lemma~\ref{lem:cross_split}, we know Eqn.~(\ref{eqn:cross-entropy}) is equivalent to:
\[
 -\sum_{i=1}^n \mathbb{E}_{\bfW_{\bfX,i}}[\log \mathbb{P}(\bfW_{\bfZ,i}=\bfW_{\bfX,i};\bfK_\bfZ)],
\]

This expression takes expectation over $\bfW_{\bfX,i}$ for the given row $i$. Notice that 
$\bfW_{\bfX,i}$ has exactly one non-zero entry, which equals $1$ (same for $\bfW_{\bfZ,i}$). 
As a result
we expand the above expression to be:
\begin{equation}
 -\sum_{i=1}^n \sum_{j\neq i} \Pr(\bfW_{\bfX,i,j}=1)\log \Pr(\bfW_{\bfZ,i,j}=1).
\label{eqn:detailed-expansion}    
\end{equation}


By Lemma~\ref{lem:multinomial}, $\Pr(\bfW_{\bfZ,i,j}=1)=\bfK_{\bfZ,i,j}/\|\bfK_{\bfZ,i}\|_1$ for $j\neq i$. Recall that $\bfK_\bfZ=(k(\bfZ_i-\bfZ_j))_{(i,j)\in[n]^2}$, which means 
$\bfK_{\bfZ,i,j}/\|\bfK_{\bfZ,i}\|_1=\frac{\exp(-\|\bfZ_i-\bfZ_j\|^2/{2\tau})}{\sum_{k\neq i}
\exp(-\|\bfZ_i-\bfZ_k\|^2/{2\tau})
}$ for $j\neq i$, when $k$ is the Gaussian kernel with variance $\tau$. 

Notice that $\bfZ_i=f(\bfX_i)$, so we know
\begin{equation}
-\log \Pr(\bfW_{\bfZ,i,j}=1)=
-\log \frac{\exp(-\|f(\bfX_i)-f(\bfX_j)\|^2/{2\tau})}{\sum_{k\neq i}
\exp(-\|f(\bfX_i)-f(\bfX_k)\|^2/{2\tau}),
}
\label{eqn:infonce-equivalence}    
\end{equation}


The right hand side is exactly the InfoNCE loss defined in Eqn.~(\ref{eqn:infonce}).
Inserting Eqn.~(\ref{eqn:infonce-equivalence}) into Eqn.~(\ref{eqn:detailed-expansion}), we get the SimCLR algorithm, which first samples augmentation pairs $(i,j)$ with $\Pr(\bfW_{\bfX,i,j}=1)$ for each row $i$, and then optimize the InfoNCE loss. 

\textbf{Step 2: } minimizing the cross entropy loss 
is equivalent to spectral clustering on $\bfpi$.


By Lemma~\ref{lem:convert_to_spectral}, we may further convert the loss to 
\begin{equation}
\label{eqn:main-theorem-repul-attr}
\min_{\bfZ}
-\sum_{(i,j)\in [n]^2} \mathbf{P}_{i,j}
\log k (\bfZ_i-\bfZ_j)+\log \mathbf{R}(\bfZ).
\end{equation}
Since $k$ is the Gaussian kernel, this reduces to \[
\min_\bfZ \mathrm{tr}(\bfZ^\top \mathbf{L}(\bfpi) \bfZ)
+\log \mathbf{R}(\bfZ),
\]

where we use the fact that $\mathbb{E}_{\bfW_\bfX\sim \mathbb{P}(\cdot; \bfpi)}[\mathbf{L}(\bfW_\bfX)]
=\mathbf{L}(\bfpi)
$, because the Laplacian operator is linear and $
\mathbb{E}_{\bfW_\bfX\sim \mathbb{P}(\cdot; \bfpi)}(\bfW_\bfX)=\bfpi
$.
\end{proof}

\paragraph{Proof of Theorem \ref{thm:clip}.}
\begin{proof}
Since $\bfW_\bfX\sim \mathbb{P}(\cdot;\bfpi_{\mathbf{A}, \mathbf{B}})$, we know 
$\bfW_\bfX$ has exactly one non-zero entry in each row, denoting the pair that got sampled. 
A notable difference compared to the previous proof is we now have $n_\mathcal{A}+n_\mathcal{B}$ objects in our graph. CLIP deals with this by taking a mini-batch of size $2N$, 
such that $n_\mathcal{A}=n_\mathcal{B}=N$, and adding the $2N$ InfoNCE losses together. We label the objects in $\mathcal{A}$ as $[n_\mathcal{A}]$, and the objects in $\mathcal{B}$ as $\{n_\mathcal{A}+1, \cdots, n_\mathcal{A}+n_\mathcal{B}\}$. 

Notice that $\bfpi_{\mathbf{A}, \mathbf{B}}$ is a bipartite graph, so the edges of objects in $\mathcal{A}$ will only connect to object in $\mathcal{B}$ and vice versa. We can define the similarity matrix in $\cZ$ as $\bfK_\bfZ$, 
where $\bfK_\bfZ(i, j+n_\mathcal{A})=\bfK_\bfZ(j+n_\mathcal{A},i)= k(\bfZ_i-\bfZ_j)$ for $i\in [n_\mathcal{A}], j\in [n_\mathcal{B}]$, and otherwise we set $\bfK_\bfZ(i,j)=0$. 
The rest is same as the previous proof. 
\end{proof}

\paragraph{Proof of Theorem \ref{thm:exponential}.}

\begin{proof}
\label{proof:exponential}
Since the objective function consists of a linear term combined with an entropy regularization, which is a strongly concave function, the maximization problem is a convex optimization problem. Owing to the implicit constraints provided by the entropy function, the problem is equivalent to having only the equality constraint. We then introduce the Lagrangian multiplier $\lambda$ and obtain the following relaxed problem:

$$
\widetilde{E}(\boldsymbol{\alpha})=\psi_{1}-\sum_{i=1}^n \alpha_{i} \psi_{i}+\tau \sum_{i=1}^n \alpha_{i}\log \alpha_{i}+\lambda\left(\boldsymbol{\alpha}^{\top} \mathbf{1}_n-1\right).
$$

As the relaxed problem is unconstrained, taking the derivative with respect to $\alpha_{i}$ yields

$$
\frac{\partial \widetilde{E}(\boldsymbol{\alpha})}{\partial \alpha_{i}}=-\psi_{i}+\tau\left(\log \alpha_{i}+\alpha_{i} \frac{1}{\alpha_{i}}\right)+\lambda=0.
$$

Solving the above equation implies that $\alpha_{i}$ takes the form
$
\alpha_{i}=\exp \left(\frac{1}{\tau} \psi_{i}\right) \exp \left(\frac{-\lambda}{\tau}-1\right).
$ Since $\alpha_{i}$ lies on the probability simplex, the optimal $\alpha_{i}$ is explicitly given by
$
\alpha^{*}_{i}=\frac{\exp \left(\frac{1}{\tau} \psi_{i}\right)}{\sum_{i^{\prime}=1}^n \exp \left(\frac{1}{\tau} \psi_{i^{\prime}}\right)} .
$ Substituting the optimal point into the objective function, we obtain
$$
\begin{aligned}
E\left(\boldsymbol{\alpha}^*\right)  &=\psi_1-\sum_{i=1}^n \frac{\exp \left(\frac{1}{\tau} \psi_{i}\right)}{\sum_{i^{\prime}=1}^n \exp \left(\frac{1}{\tau} \psi_{i^{\prime}}\right)} \psi_{i}+\tau \sum_{i=1}^n \frac{\exp \left(\frac{1}{\tau} \psi_{i}\right)}{\sum_{i^{\prime}=1}^n \exp \left(\frac{1}{\tau} \psi_{i^{\prime}}\right)}\log \frac{\exp \left(\frac{1}{\tau} \psi_{i}\right)}{\sum_{i^{\prime}=1}^n \exp \left(\frac{1}{\tau} \psi_{i^{\prime}}\right)} \\
& =\psi_1 - \tau \log \left(\sum_{i=1}^n \exp \left(\frac{1}{\tau} \psi_{i}\right)\right).
\end{aligned}
$$
Thus, the Lagrangian dual function is given by
\begin{equation*}
-E\left(\boldsymbol{\alpha}^*\right)= -\tau \log \frac{\exp \left(\frac{1}{\tau} \psi_{1}\right)}{\sum_{i=1}^n \exp \left(\frac{1}{\tau} \psi_{i}\right)}.\qedhere
\end{equation*}
\end{proof}



\section{More on Experiments} \label{section: experiment_details}

\paragraph{CIFAR-10 and CIFAR-100} CIFAR-10 ~\citep{krizhevsky2009learning} and CIFAR-100 ~\citep{krizhevsky2009learning} are well-known classic image classification datasets. Both CIFAR-10 and CIFAR-100 contain a total of 60k $32 \times 32$ labeled images of different classes, with 50k for training and 10k for testing. CIFAR-10 is similar to CIFAR-100, except there are 10 different classes in CIFAR-10 and 100 classes in CIFAR-100.

\paragraph{TinyImageNet} TinyImageNet ~\citep{le2015tiny} is a subset of ImageNet ~\citep{deng2009imagenet}. There are 200 different object classes in TinyImageNet, with 500 training images, 50 validation images, and 50 test images for each class. All the images in TinyImageNet are colored and labeled with a size of $64 \times 64$.

\textbf{Pseudo-code.} Algorithm \ref{alg:Training Procedure} presents the pseudo-code for our empirical training procedure.

\begin{algorithm}[!htbp]
\caption{Training Procedure}
\label{alg:Training Procedure}
\begin{algorithmic}[1]
\REQUIRE trainable encoder network $f$, batch size $N$, augmentation strategy \textit{aug}, loss function $L$ with hyperparameters \textit{args}
\FOR {sampled minibatch ${x_i}_{i=1}^N$}
\FORALL{$i \in { 1, ..., N }$}
\STATE draw two augmentations $t_i = \textit{aug}\left(x_i\right) $, $t_i' = \textit{aug}\left(x_i\right) $
\STATE $z_i = f\left(t_i\right)$, $z_i' = f\left(t_i'\right)$
\ENDFOR
\STATE compute loss $\mathcal{L} = L(N, z, z', \textit{args})$
\STATE update encoder network $f$ to minimize $\mathcal{L}$
\ENDFOR
\STATE \textbf{Return} encoder network $f$
\end{algorithmic}
\end{algorithm}

We also provide the pseudo-code for our core loss function used in the training procedure in Algorithm \ref{alg:Core loss}. The pseudo-code is almost identical to SimCLR's loss function, with the exception of an extra parameter $\gamma$.

\begin{algorithm}[!htbp]
\caption{Core loss function $\mathcal{C}$}
\label{alg:Core loss}
\begin{algorithmic}[1]
\REQUIRE batch size $N$, two encoded minibatches $z_1, z_2$, $\gamma$, temperature $\tau$
\STATE $z = \textit{concat}\left(z_1, z_2\right)$
\FOR {$i \in {1, ..., 2N }, j \in {1, ..., 2N}$ }
\STATE $s_{i,j} = \Vert z_i - z_j \Vert_2^{\gamma}$
\ENDFOR
\STATE \textbf{define} $l(i, j)$ \textbf{as} $l(i, j) = - \log \frac{exp\left(s_{i,j}/\tau \right)}{\sum_{k=1}^{2N} \mathbf{1}{[k \ne i]} exp\left(s{i, j} / \tau \right)} $
\STATE \textbf{Return} $\frac{1}{2N} \sum_{k=1}^N\left[l(i, i+N) + l(i+N, i)\right]$
\end{algorithmic}
\end{algorithm}

Utilizing the core loss function $\mathcal{C}$, we can define all kernel loss functions used in our experiments in Table \ref{table: loss definition}. For all $z_i \in z$ with even dimensions $n$, we define $z_{L_i} = z_i\left[0:n/2\right]$ and $z_{R_i} = z_i\left[n/2:n\right]$.

\begin{table}[ht]
\centering
\begin{tabular}{{@{}l|l@{}}}
Kernel  &  Loss function \\ \midrule
Laplacian & $\mathcal{C}\left(N, z, z', \gamma=1, \tau\right)$\\ \midrule
Sum       & $\lambda * \mathcal{C}\left(N, z, z', \gamma=1, \tau_1\right) + (1-\lambda) * \mathcal{C}\left(N, z, z', \gamma=2, \tau_2\right)$  \\ \midrule
Concatenation Sum&$\lambda * \mathcal{C}\left(N, z_L, z'_L, \gamma=1, \tau_1\right) + (1-\lambda) * \mathcal{C}\left(N, z_R, z'_R, \gamma=2, \tau_2\right)$\\ \midrule
$\gamma = 0.5$ & $\mathcal{C}\left(N, z, z', \gamma=0.5, \tau\right)$          \\ 

\end{tabular}

\caption{Definition of kernel loss functions in our experiments}
\label {table: loss definition}
\end{table}

\textbf{Baselines.} We reproduce the SimCLR algorithm using PyTorch Lightning~\citep{PytorchLightning}.

\textbf{Encoder details.}
The encoder $f$ consists of a backbone network and a projection network. We employ ResNet50~\citep{ResNet} as the backbone and a 2-layer MLP (connected by a batch normalization~\citep{ioffe2015batch} layer and a ReLU \cite{nair2010rectified} layer) with hidden dimensions 2048 and output dimensions 128 (or 256 in the concatenation kernel case).

\textbf{Encoder hyperparameter tuning.}
For each encoder training case, we randomly sample 500 hyperparameter groups (sample details are shown in Table \ref{table: Hyperparameter sample}) and train these samples simultaneously using Ray Tune ~\citep{RayTune}, with the ASHA scheduler~\citep{li2018massively}. Ultimately, the hyperparameter group that maximizes the online validation accuracy (integrated in PyTorch Lightning) within 5000 validation steps is chosen for the given encoder training case.

\begin{table}[ht]
\centering

\begin{tabular}{@{}l|l|l@{}}
\midrule
Hyperparameter  & Sample Range & Sample Strategy \\ \midrule
start learning rate & $\left[10^{-2}, 10\right]$ & log uniform \\ \midrule
$\lambda$       & $\left[0, 1\right]$ & uniform \\ \midrule
$\tau$, $\tau_1$, $\tau_2$ & $\left[0, 1\right]$ & log uniform \\ \midrule
\end{tabular}

\caption{Hyperparameters sample strategy}
\label {table: Hyperparameter sample}
\end{table}

\textbf{Encoder training.} 
We train each encoder using the LARS optimizer~\citep{LARSOptimizer}, LambdaLR Scheduler in PyTorch, momentum 0.9, weight decay $10^{-6}$, batch size 256, and the aforementioned hyperparameters for 400 epochs on a single A-100 GPU.

\textbf{Image transformation.} The image transformation strategy, including augmentation, is identical to the default transformation strategy provided by PyTorch Lightning.

\textbf{Linear evaluation.}
The linear head is trained using the SGD optimizer with a cosine learning rate scheduler, batch size 64, and weight decay $10^{-6}$ for 100 epochs. The learning rate starts at $0.3$ and ends at $0$.

\textbf{Moco Experiments.} We also tested our method based on MoCo~\citep{he2019moco}. The results are summarized in Table \ref{tab:results-moco}. Here we choose ResNet18~\citep{ResNet} as the backbone and set a temperature of $0.1$ as default. For our simple sum kernel, we set $\lambda=0.8$. The results show that our method outperforms the original MoCo method.

\begin{table}[thb]
\centering
\caption{MoCo Experiment Results on CIFAR-10 and CIFAR-100.}
\label{tab:results-moco}
\resizebox{\textwidth}{!}{%
\begin{tabular}{@{}c|ccc|ccc@{}}
\toprule
\multirow{3}{*}{Method} & \multicolumn{3}{c|}{CIFAR-10} & \multicolumn{3}{c}{CIFAR-100} \\ \cmidrule(lr){2-4} \cmidrule(lr){5-7} 
                        & 200 epochs & 400 epochs    & 1000 epochs   & 200 epochs & 400 epochs & 1000 epochs         \\ \midrule
MoCo (repro.)         & $76.41 \pm 0.12$    & $80.01 \pm 0.15$          & $84.45 \pm 0.08$    & $\mathbf{47.02 \pm 0.11}$ & $52.50 \pm 0.07$ & $57.62 \pm 0.15$            \\
\midrule
Laplacian Kernel        & ${78.09 \pm 0.10}$    & $\mathbf{83.85 \pm 0.09}$          & $\mathbf{88.34 \pm 0.16}$    & $46.12 \pm 0.22$   & $53.44 \pm 0.17$ & $59.10 \pm 0.14$        \\
Simple Sum Kernel & $\mathbf{78.12 \pm 0.15}$   & $83.23 \pm 0.18$ & $87.50 \pm 0.20$ & $46.65 \pm 0.06$ & $\mathbf{53.62 \pm 0.19}$ & $\mathbf{59.83 \pm 0.12}$\\
\bottomrule
\end{tabular}
}
\end{table}



\section{More Experiments on Synthetic Data}


Consider a scenario with $n$ clusters, each containing $k$ vertices. Let the probability of vertices $u$ and $v$ from the same cluster belonging to $\bfpi$ be $p$. Conversely, for vertices $u$ and $v$ from different clusters, let the probability of belonging to $\pi$ be $q$. We generate the graph $\bfpi$ randomly, based on $p$ and $q$. We experiment with values of $k=100$ and $n=6$ for ease of visualization, embedding all points in a two-dimensional space. Each vertex's initial position originates from a normal distribution. In each iteration, we sample a subgraph of $\bfpi$ uniformly, ensuring each vertex has an out-degree of $1$. We then optimize the corresponding vectors using InfoNCE loss with an SGD optimizer and iterate until convergence. Our experimental setup consists of an SGD learning rate of $1$, an InfoNCE loss temperature of $0.5$, and a batch size of $50$. We evaluate two scenarios with different $p$ and $q$ values: $p=1$, $q=0$, and $p=0.75$, $q=0.2$. The results of these experiments are visualized in Figure \ref{fig:vis-spectral-cluster}. The obtained embeddings exhibit the hallmark pattern of spectral clustering of graph $\bfpi$.

\begin{figure}[!tb]
\centering
\subfigure{
\includegraphics[width=1\textwidth]{Figures/cluster_pi.png}
\label{fig:vis-cluster}
}
\subfigure{
\includegraphics[width=1\textwidth]{Figures/noised_cluster_pi.png}
\label{fig:vis-noised-cluster}
}
\caption{Visualizations of the optimization process using InfoNCE Loss on the vectors corresponding to $\bfpi$. Points of identical color belong to the same cluster within $\bfpi$. To showcase the internal structure of $\bfpi$, we randomly select 10 vertices from each cluster to display the edge distribution of $\bfpi$.}
\label{fig:vis-spectral-cluster}
\end{figure}



\end{document}
