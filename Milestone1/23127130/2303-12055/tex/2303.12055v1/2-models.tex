\section{Analysis of CRES data \label{sec:analysisoverview}}

\subsection{Overview}

The primary content of this paper is the analysis strategy we developed for CRES data and the results we obtained using it.  The analysis is complex, a consequence of the exploratory nature of the experiment.   Some aspects of the experiment design and the analysis were not anticipated; for example, the gas composition measurement was more important than we had expected, but it would not be as significant in future designs with higher resolution and purer source gas.  Another unexpected complication was the extent to which the efficiency as a function of frequency and energy was modulated by the presence of parasitic reflections in the waveguide, which demanded a very detailed analysis methodology.  A flow chart for the analysis is given in \autoref{fig:flowchart}.  Notwithstanding the complications both expected and unexpected, we have achieved a complete and successful analysis, as reported below.
\begin{figure*}[tb]
\centering
\includegraphics[width=1\textwidth]{Plots/1-introduction/PRC_flow_chart_v2023-03-20_v1.pdf}
\caption{Flow chart of the analysis procedure. Thin arrows represent relationships between individual boxes, while thick arrows describe shared inputs from and/or outputs to a group of boxes in the same bubble. }
\label{fig:flowchart}
\end{figure*}


\subsection{CRES energy spectrum model}\label{sec:Kr_detector_response}

We model a generic detected CRES signal spectrum $\mathcal{S}$ as
\begin{eqnarray}
    \mathcal{S} &=& \epsilon\left(\mathcal{Y}*\mathcal{R}_{\mathrm{PSF}}\right), 
  \label{eq:FullModel1} \\
    \mathcal{R}_{\mathrm{PSF}} &=& \sum_{j=0}^{j_{\mathrm{max}}}\mathcal{A}_j\left(\mathcal{I}*\mathcal{L}_{\mathrm{tot}}^{*j}\right).
    \label{eq:FullModel2}
\end{eqnarray}
Here, all variables are functions of $E_\mathrm{kin}$, as denoted by script lettering. The symbol $*$ represents convolution and $^{*j}$ represents self-convolution $j$ times. The efficiency function $\epsilon$  encodes the probability to detect electron events and is discussed in-detail in \autoref{sec:efficiency_for_tritium}. The underlying true energy spectrum of the electrons is $\mathcal{Y}$. The underlying spectrum is described in \autoref{subsec:Kr Line model} for $^{\mathrm{83m}}$Kr and in \autoref{sec:tritium_model} for tritium.  $\mathcal{R}_{\mathrm{PSF}}$ is the point-spread function, which represents the energy response for mono-energetic electrons---in other words,  how reconstructed energies are shifted and broadened relative to true energies. 

\begin{figure}[t]
  \centering
  \includegraphics[width=1.0\columnwidth]{Plots/2-models/scattering_prc_illustrative_figure_v2023-03-08v1.pdf}
  \caption{(a) A CRES event with frequency jumps corresponding to inelastic scattering $\mathcal{L}_{\mathrm{tot}} = \sum_i(\gamma_i \mathcal{L}_i)$ and energy loss due to cyclotron radiation $\mathcal{L}_r$ labelled. (b) A modeled spectrum broken down into its constituent scatter peaks and labelled to show the roles of model parameters: scatter peak amplitudes $\mathcal{A}_j$, instrumental resolution $\mathcal{I}$ (with 1.66-eV FWHM), and energy loss spectra $\mathcal{L}_{\mathrm{tot}}^{*j}$. Each scatter peak is displayed as a simple Gaussian for illustration purposes.}% This is a plot of Eq. \label{eq:simplified_scattering}.
  \label{fig:scattering_illustration}
\end{figure}

Equation~\ref{eq:FullModel2} shows that the energy point-spread function $\mathcal{R}_{\mathrm{PSF}}$ is comprised of a sum of scatter peaks $\mathcal{I}*\mathcal{L}_{\mathrm{tot}}^{*j}$ weighted by amplitudes $\mathcal{A}_j$, as illustrated in \autoref{fig:scattering_illustration}.  These amplitudes describe the relative likelihood that an electron will first be detected after $j$ scattering events. We account for the possibility of up to $j_{\mathrm{max}}$ scatters before the first detection. We use $j_{\mathrm{max}}=20$ in both $^\mathrm{83m}$Kr and tritium fits, as increasing $j_{\mathrm{max}}$ further has no observable effect on results, for Phase II conditions.  The instrumental resolution $\mathcal{I}$ accounts for broadening from magnetic field inhomogeneity and noise.   The distribution of electrons' energy losses between scatters $\mathcal{L}_{\mathrm{tot}}$ depends on the gas composition, the differential cross section on each gas component, and the loss to cyclotron radiation.  The elements in $\mathcal{R}_{\mathrm{PSF}}$ are described further in the remainder of this section.

\subsubsection{Instrumental resolution \texorpdfstring{$\mathcal{I}$}{}}
The instrumental resolution $\mathcal{I}$ is the spectrum that a source of mono-energetic electrons would have if they were all detected before scattering. $\mathcal{I}$ is broadened only by noise and by the differences in the range of magnetic fields sampled by electrons with different pitch angles and radial positions. These ranges vary with trapping geometry and are determined by simulation for each experiment configuration, as described in \autoref{subsec:ins_res}. The broadening from frequency reconstruction, \emph{e.g.}, due to frequency binning in the Fourier-transformed data and noise, is only ${\sim}$0.2\,eV. This is small compared to the effect of magnetic field inhomogeneity and is therefore neglected, for $\mathcal{I}$. Reconstruction also shifts frequencies upwards by 14\,kHz (equivalent to 0.3\,eV), on average \cite{TERpaper:2022}. 
%this effect is corrected by decreasing the mean magnetic field value in tritium data (see \autoref{sec:Bfield_errors}).
This effect is present in all data ($^{\mathrm{83m}}$Kr and tritium) and we decrease the reported magnetic field value in \autoref{sec:Bfield_errors} to account for it.

\subsubsection{Energy loss spectra \texorpdfstring{$\mathcal{L}_{\mathrm{tot}}^{*j}$}{}}
\label{L_model}
Electrons lose energy primarily by inelastic scattering with gas molecules, causing the jumps in \autoref{fig:spectrogram}, as discussed in \autoref{sec:gas_composition}.
Cyclotron radiation is a smaller, continuous source of electron energy loss, causing the upward track slopes in \autoref{fig:spectrogram}. 
$\mathcal{L}_{\mathrm{tot}}^{*j}$ comprises the distribution of possible energy losses an electron has experienced \emph{before} the first detected track due to inelastic scattering and cyclotron radiation, with the self-convolution $j$ times accounting for $j$ scatters. The electron energy loss spectrum for a single scatter is
\begin{equation}
\begin{aligned}
    \mathcal{L}_{\mathrm{tot}} = (\gamma_1 \mathcal{L}_1 + \gamma_2 \mathcal{L}_2 + \cdots + \gamma_n \mathcal{L}_n)*\mathcal{L}_r\,,\label{eq:combine_peaks_from_different_gases}
\end{aligned}
\end{equation}
where $\mathcal{L}_i$ is the electron inelastic energy loss spectrum for the $i^{\rm th}$ gas species, each $\gamma_i$ is the fraction of inelastic scatters that are due to the specific gas species $i$, and $\mathcal{L}_r$ is the energy loss spectrum due to cyclotron radiation during the missed track.

We determine bounds on $\gamma_i$ from quadrupole mass analyzer data as described in \autoref{gamma_i_determination}. For the shallow trap data, the high resolution allows for a more precise estimate of gas scattering fractions for H$_2$ and He to be determined from the fit to $^\mathrm{83m}$Kr data.  We neglect the energy dependence of the scatter fractions $\gamma_i$ over the small range of energy change.

Each $\mathcal{L}_i$ is calculated in the Bethe theory of electron inelastic scattering (as in~\cite{Inokuti:InelasticScattering1971}), given by
\begin{eqnarray}
\hspace{-0.1in}\frac{d\sigma}{dE} &=& \frac{4 \pi a_0^2R}{E_{\rm kin}} \left[\frac{R}{E}\frac{df}{dE}\ln\left(\frac{4c_EE_{\rm kin}}{R}\right) + \mathcal{O}\left(\frac{R}{E_{\rm kin}}\right)\right], \quad
\label{eq:inelastic scatter energy loss }
\end{eqnarray}
where $R$ is the Rydberg energy, $a_0$ is the Bohr radius, $E_{\rm kin}$ is the incident energy of the electron, $E$ is the energy loss of the electron, $c_E \approx (R/E)^2$, and $df/dE$ is the optical oscillator strength of the gas molecules.
The optical oscillator strength \cite{Chan:AbsoluteOscillatorStrength1991,chan1992absolute,olney1997absolute} data for the relevant gas species are from the LXCat database~\cite{LXCat_carbone2021data,pitchford2017lxcat,pancheshnyi2012lxcat,LXCat:database}. 

\begin{figure}[tb]
  \centering
  \includegraphics[width=1.0\columnwidth]{Plots/2-models/radiation_loss_spectrum.pdf}
  \caption{Simulated $\mathcal{L}_r$, the energy loss distribution due to cyclotron radiation during a missed track, for conditions (\emph{e.g.}, track duration) corresponding to the $^{\mathrm{83m}}$Kr pre-tritium data set.}
  \label{fig:radiation_loss}
\end{figure}

We determine the loss due to cyclotron radiation $\mathcal{L}_r$ using the simulated data described in \autoref{sec:sim-events}.
In these simulated data, missed tracks associated with detected events are identified.
The distribution of differences between track end frequency and track start frequency among these tracks is converted to energy and taken as the radiative energy loss spectrum $\mathcal{L}_r$ (\autoref{fig:radiation_loss}), a function of track duration $\tau$. With most of its weight in a peak between 0 and 3 eV\textemdash reflecting the low likelihood of missing long tracks\textemdash and a modest tail out to ${\sim}$10 eV, $\mathcal{L}_r$ has a much smaller impact than inelastic scatters. 
$\mathcal{L}_r$ is included in the $^\mathrm{83m}$Kr fit model and in simulated data (see \autoref{sec:sim-events}). However, we omit $\mathcal{L}_r$ from the tritium fit model, since Monte Carlo (MC) pseudo-data studies of the type described in \autoref{sec:MCstudies} showed that omitting $\mathcal{L}_r$ has no measurable effect on the tritium analysis. 
 

\subsubsection{Scatter peak amplitudes\label{sec:scatter-peak-amplitudes} \texorpdfstring{$\mathcal{A}_j$}{}} 
Each amplitude  $\mathcal{A}_j$ is the relative likelihood of missing the first $j-1$ tracks and detecting the $j^{\rm th}$ track in an event. Because scattering changes the distribution of electron pitch angles, $\mathcal{A}_j$ is not expected to follow a strictly exponential dependence on $j$. In addition, $\mathcal{A}_j$ depends both on the details of the event reconstruction algorithm\textemdash its thresholds in first track duration, first track SNR, and number of tracks in the remainder of an event\textemdash and on the distributions of these properties among events in the data. Because the thresholds in the reconstruction algorithm are set empirically to exclude the expected distribution of false events rather than from an \emph{a priori} analytical model, the model for $\mathcal{A}_j$ is also phenomenological. 
The model used is
\begin{eqnarray}\label{eq:scatter_peak_amplitude}
    \mathcal{A}_j &=&\exp\Big[-pj^{(-dp+q)}\Big],
\end{eqnarray} 
with free parameters $p$ and $q$. The constant $d=-0.4955$ is chosen to minimize the correlation between $p$ and $q$ and held fixed. The model is found to provide sufficient flexibility to account for the deviation from exponentiality, as confirmed in simulations shown in \autoref{subsubsec: scatter peak amplitude simulation}.  

The dependence of $\mathcal{A}_j$  on the distribution of tracks per event ($N_{\mathrm{tracks}}^{\mathrm{true}}$) is captured in a dependence of $p$ and $q$ on a data set's average $N_{\mathrm{tracks}}^{\mathrm{true}}$ value. The difference in $N_{\mathrm{tracks}}^{\mathrm{true}}$ among data sets stems mainly from differences in gas composition. Gas species have different relative likelihoods of elastic and inelastic scattering, and elastic scattering is far likelier to cause an electron to undergo a large pitch angle change into a non-detectable pitch angle, ending the visible portion of the event. The $p$ and $q$ values for $^{\mathrm{83m}}$Kr data sets are determined from fits to the CRES spectra. The values used for the tritium analysis are extrapolated from $^{\mathrm{83m}}$Kr $p$ and $q$ values as a function of $N_{\mathrm{tracks}}^{\mathrm{true}}$, as described in \autoref{sec:scatter_peak_errors}.


\section{Simulated CRES data \label{sec:simCRES}}

Simulations are an important tool in the Phase~II analysis to generate inputs to the analysis model and to evaluate the performance of event detection and reconstruction methods. Simulated data are used to find numerical descriptions of the instrumental resolution $\mathcal{I}$ (\autoref{subsec:ins_res}) and the radiative loss spectrum $\mathcal{L}_r$ (\autoref{fig:radiation_loss}). In addition, the single deep trap K-line data of each magnetic field step in the background field scans is reproduced in simulation. This allows us to translate the measured frequency-dependent detection efficiency curve $\epsilon$ to an energy-dependent curve used in the tritium data analysis (\autoref{sec:efficiency_for_tritium}). In \autoref{sec:sim-events}, we discuss the basic framework and inputs to Phase~II event simulations and demonstrate good agreement between simulated and real data. Section~\ref{sec:gen_resolution} describes how a numerical instrumental resolution shape $\mathcal{I}$ is constructed from an ensemble of simulated events. 

\subsection{Simulation basics and validation}\label{sec:sim-events}
The Locust package~\cite{AshtariEsfahani:2019mwv} simulates the detection of RF signals by modeling the response of an antenna and receiver to time-varying electromagnetic fields. Locust can independently generate a custom signal to use as input to its receiver chain algorithm, which processes the signal prior to digitization and recording. We developed a Locust signal generator module that simulates chirped signals with typical electron event properties. 
The Phase~II waveguide and trap geometries were implemented in this generator to create realistic Phase~II-like signals. Event starting conditions are sampled from probability density functions.

\begin{figure}[tb]
    \includegraphics[width=0.95\columnwidth]{Plots/3-systematic-uncertainties/3-2-instrumental-resolution/ntracks_5.0_recon_dist.pdf}
    \includegraphics[width=0.95\columnwidth]{Plots/3-systematic-uncertainties/3-2-instrumental-resolution/ntracks_scan_ntracks_scan_dirac_scatter_all_data_sets.pdf}
    \caption{Procedure for extracting the underlying mean number of tracks per event from simulations: (a) For each configured value of $N_{\mathrm{tracks}}^{\mathrm{true}}$, the distribution of reconstructed number of tracks per event is fitted with a geometric distribution of parameter $p_\textnormal{tracks}$. The fit excludes the first bin since the counts are reduced in this bin by event cuts. (b) The intercepts of a linear fit to the fitted $p_\textnormal{tracks}$ for all tested $N_{\mathrm{tracks}}^{\mathrm{true}}$ allows us to find the optimum configuration for each data set. The uncertainty on $p_\textnormal{tracks}$ (horizontal bands) and the uncertainty of the linear fit parameters are propagated to an uncertainty on the extracted mean $N_{\mathrm{tracks}}^{\mathrm{true}}$ (vertical bands).}
    \label{fig:number_of_tracks_extraction}
\end{figure}

We generate a set of events from which the instrumental resolution $\mathcal{I}$ can be derived. For this purpose, monoenergetic electrons are sampled at different radial and axial positions in the waveguide, and with different pitch angles, generating trajectories in response to the magnetic field map of a single-coil trap in the apparatus.
To reproduce multi-trap effects (e.g. SNR and field variation differences between traps), the events from simulations with different trapping fields are combined. 

The average CRES signal frequency and power are calculated from the magnetic field along the electron's trajectory and the power coupled to the $\mathrm{TE}_{11}$ mode. This calculation accounts for the frequency modulation resulting from an electron's pitch angle \cite{Esfahani:2019mpr}. Smaller pitch angles lead to reduced power in the ``carrier'' and more power in sidebands.  Only the carrier is detected in this experiment. In the track detection process, power thresholds are applied which translate to a maximum axial excursion in the trap and a maximum variation of magnetic field seen by the electron during axial motion (see \autoref{sec:electron_trap}).   The resolution $\mathcal{I}$ thus depends on the range of SNR values accepted in analysis.  Once the maximum SNR for a trap has been measured, the resolution can be predicted for any choice of threshold.

Field-shifting studies measured SNR differences between the single traps (\autoref{subsec:efficiency}). These differences are accounted for by multiplying the signal power in each trap with a relative SNR factor. For generating a simulated data set that can directly be compared to recorded data, the overall SNR scale is determined by iteratively adjusting the maximum coupled power until the first-track SNR distribution after reconstruction matches that of real data. The maximum coupled power corresponds to SNR$_{\rm max}$, the SNR of a $90^{\circ}$ electron at $r=0\,\si{mm}$ in trap 3. This is the trap in which power is coupled most effectively into the transporting waveguide mode at the CRES frequency of the $\rm{^{83m}Kr}$ K-line. Later, this procedure for setting the SNR scale is replaced by the method described in \autoref{sec:max-SNR-optimization}.

To simulate multi-track events, a sequence of chirped signals is generated with start frequency and power calculated as described above and is added to a Gaussian noise floor. 
Track slopes are sampled from a Gaussian with a mean ($352.3\,\si{MHz/s}$) and standard deviation ($54.5\,\si{MHz/s}$) corresponding to the mean and standard deviation observed in the deep quad trap $^{\mathrm{83m}}$Kr data. The Gaussian assumption is only approximately valid, but since reconstruction efficiency is relatively insensitive to track slope as long as they are within several $100 \,\si{MHz/s}$ of the mean, achieving a better agreement of the slope distribution is irrelevant.


For each event, the track duration is drawn from a configurable exponential distribution. To propagate uncertainty on the mean track duration $\tau$ (which defines the exponential) when comparing simulated events to data, $\tau$ is drawn from a normal distribution with mean and width according to the fit results listed in \autoref{tab:mml_track_length_fit_results}.  

\begin{figure}
\centering
\includegraphics[width=0.9\columnwidth]{Plots/3-systematic-uncertainties/3-2-instrumental-resolution/data_vs_simulation_katydid_reconstructed_tracks_per_event.pdf}
\includegraphics[width=0.9\columnwidth]{Plots/3-systematic-uncertainties/3-2-instrumental-resolution/data_vs_simulation_katydid_reconstruction_track_length.pdf}
\includegraphics[width=0.9\columnwidth]{Plots/3-systematic-uncertainties/3-2-instrumental-resolution/data_vs_simulation_katydid_reconstruction_with_trigger.pdf}
\caption{Comparison of simulated data to the post-tritium $\mathrm{^{83m}Kr}$ data set after triggering and event reconstruction: (a) number of tracks per event,  (b) first track duration, and (c) first track SNR. The simulations were optimized to reproduce this data.
The good agreement validates the simulations and allows to utilize them to generate simulated instrumental resolutions $\mathcal{I}$. For this purpose simulated data sets were generated to match each data set listed in \autoref{tab:data_set_table}.
}

\label{fig:simulation_validation_track_length_ntracks_snr}
\end{figure}


The number of tracks per event is drawn from a geometric distribution with a configurable expectation value.
The observed mean number of tracks after event reconstruction does not correspond to the underlying truth, because regularly, tracks are missed or two tracks are combined into one during reconstruction. 
To find the right underlying mean number of tracks per event $N_{\mathrm{tracks}}^{\mathrm{true}}$, events of different $N_{\mathrm{tracks}}^{\mathrm{true}}$ are simulated and reconstructed using the same reconstruction methods applied in real data processing. 
The reconstructed number of tracks is then fitted with a geometric distribution, as shown in \autoref{fig:number_of_tracks_extraction}a. 
The success probability $p_{\rm{tracks}}$ of the fitted geometric distribution corresponds to the probability of a detected track to not be followed by another track. 
The relation between the reconstructed and the true mean number of tracks per event is found to be linear \cite{TERpaper:2022}. 
Figure~\ref{fig:number_of_tracks_extraction}b shows $p_{\mathrm{tracks}}$ vs. inputted $N\mathrm{_{tracks}^{true}}$. 
The underlying $N\mathrm{_{tracks}^{true}}$ for each data set can be read off from the intersection of the linear fit with the data set's $p_\mathrm{tracks}$. For each data set, vertical bands indicate the uncertainty on $N\mathrm{_{tracks}^{true}}$ from two contributions: the linear fit uncertainty and the $p_\mathrm{tracks}$ uncertainty. 

The size of the frequency jumps between tracks is drawn from the hydrogen energy loss function from electron-molecule scattering, converted to frequency. Scattering from other gases can be neglected for obtaining $\mathcal{I}$, since $\mathcal{I}$ includes only the start frequencies of first tracks, and the energy loss from scattering affects only the start frequency of consecutive tracks. Pitch angle changes during inelastic scatters are assumed to be small and are ignored, and the power of consecutive tracks in an event is kept constant.  Despite this approximation, after processing with the Phase~II trigger and reconstruction methods, real data sets are well reproduced by simulated events in the main relevant properties: number of tracks per event, track duration, and SNR  (\autoref{fig:simulation_validation_track_length_ntracks_snr}).

\subsection{Simulated instrumental resolution}\label{sec:gen_resolution}
Input resolution ($\mathcal{I}$) shapes are required to analyze all $^{\mathrm{83m}}$Kr calibration data sets, shallow-trap $^{\mathrm{83m}}$Kr data, and tritium data. To obtain the $\mathcal{I}$ distribution for each data set, events in all constituent single traps are simulated. For each simulated event, start frequency, number of tracks, track duration, and true power are recorded.
To take into account the effect of detection efficiency, the generated events are filtered by an efficiency matrix, which is a binned look-up table of the probability for a track to be accepted as a function of SNR, track duration, and number of tracks in the event. 
The efficiency matrix is produced by simulating 100,000  events covering the full parameter space and analyzing the event detection probability  with respect to these three  decisive event properties.
We use the efficiency matrix to avoid the need to process all simulated data with the trigger and reconstruction methods, thereby greatly reducing the processing time. This allows us to iteratively optimize, for example, the event SNR in a data set, with a quick turn-around time in each iteration. 
Before applying the efficiency filter to the tracks, the track power has to be translated to SNR. Locust adjusts the configured signal power to reflect all physical effects in the waveguide. Hence, only the power corresponding to SNR$_{\rm max}$ needs to be set. The optimization of SNR$_{\rm max}$ is described in \autoref{sec:max-SNR-optimization}.


The histogram of the simulated events surviving the filter constitutes the frequency resolution distribution, which is converted to an energy resolution via \autoref{eq:energytofrequency}. 
The total $\mathcal{I}$ is a sum of the resolutions of individual traps.  For each trap, using the mapping from SNR to counts, the SNR is selected such that the number of events in the resolution distribution is proportional to the fraction of events collected in the trap in real data (see trap weight analysis in \autoref{subsec:trap_weights}). An example of a simulated $\mathcal{I}$ in shown in \autoref{subsec:ins_res}.

\section{Calibration with \texorpdfstring{$^{\mathrm{83m}}\mathrm{Kr}$}{}\label{sec:Kr_model}}

Fits to $^{\mathrm{83m}}$Kr electron energy lines from internal conversion are used to both test features of the apparatus setup and estimate parameters for tritium data analysis. Section~\ref{sec:frequency-energy_relation} describes how we verify the relationship between electron energy and cyclotron frequency using simple Voigt fits to  $^{\mathrm{83m}}$Kr lines. For all other $^{\mathrm{83m}}$Kr fits, instead of a Voigt profile, we employ the CRES spectrum model in \autoref{eq:FullModel1}. Accordingly, \autoref{subsec:Kr Line model},  \autoref{subsec:ins_res} and \autoref{subsec:frequency_dependence}  describe the procedure for fitting $^{\mathrm{83m}}$Kr lines (specifically, the 17.8-keV line) using the CRES spectrum model. Fits with this  model produce estimates for the mean magnetic field $B$ experienced by electrons, as well as the parameters $p$ and $q$ that control amplitudes of scatter peaks caused by missed CRES tracks. For the fits to reliably estimate these parameters, an instrumental resolution distribution $\mathcal{I}$ must be inputted to the fits. Section~\ref{subsec:ins_res} describes the $\mathcal{I}$ distributions used for $^{\mathrm{83m}}$Kr fits, including a procedure for finding the resolution width for each data set via an ``SNR scaling optimization," which requires additional $^{\mathrm{83m}}$Kr fits. Section~\ref{subsec:ins_res} also discusses how uncertainties on $\mathcal{I}$ are propagated to $B$, $p$ and $q$ for a $^{\mathrm{83m}}$Kr data set. 
Section~\ref{subsec:frequency_dependence} describes how the dependence of the detection efficiency $\epsilon$ on frequency is measured by comparing the detected event rate for a range of background magnetic field values. This is the last input to the full CRES spectrum model.
 
Section~\ref{sec:shallow_trap} and \autoref{sec:deep-trap-data-and-fits} discuss how this CRES spectrum model is used to fit specific $^{\mathrm{83m}}$Kr data sets. 
In particular, \autoref{sec:shallow_trap} describes a fit to shallow-trap $^{\mathrm{83m}}$Kr data, which demonstrates the energy resolution capabilities of CRES. Finally, \autoref{sec:deep-trap-data-and-fits} discusses fits to $^{\mathrm{83m}}$Kr data obtained with the same deep quad trap used for tritium data. These fits most directly calibrate the tritium energy point-spread function $\mathcal{R}_{\mathrm{PSF}}$. Specifically, the fit outputs are used to predict $p$, $q$ and $B$ for tritium data.

\subsection{Test of the frequency-energy relation} \label{sec:frequency-energy_relation}
To verify the predicted CRES energy-frequency relationship (\autoref{eq:energytofrequency}) across a 14.3-keV range, the $^{\mathrm{83m}}$Kr shallow trap data included measurements of the K, L2, L3, M2, M3, N2 and N3 internal-conversion lines of the 32-keV transition. For each line, the main peak is well separated from the scattering tail and from a $^{\mathrm{83m}}$Kr shakeup/shakeoff structure~\cite{Robertson:2020dd}, in this high-resolution trap. This makes it possible to extract frequencies by fitting the main peaks with Voigt profiles, which have fixed Lorentzian widths as tabulated in~\cite{V_nos_2018}. A constant background is added as a fit parameter when such a background is visible within the fit range selected around the peak. The frequencies extracted are given in \autoref{table:Kr energy lines Venos} and the fit to the K line is shown in \autoref{fig:linearity_Kr_K_freq_line}. 
\begin{figure}[htb]
    \centering
    \includegraphics[width=1\columnwidth]{Plots/2-models/prc_K_line_gaussian_res.pdf}
    \caption{The center frequency $f_K$ of the $^{\mathrm{83m}}$Kr K conversion electron line in the shallow trap configuration, and the standard deviation $\sigma_K$ of the instrumental resolution extracted from a fit to a Voigt profile. }
    \label{fig:linearity_Kr_K_freq_line}
\end{figure}

The measured frequency of each conversion line is expected to be related to its energy according to \autoref{eq:energytofrequency}. The energy of each conversion line is calculated in \cite{V_nos_2018} using the 32-keV gamma energy, as well as a binding energy and recoil energy specific to that line (shown in \autoref{table:Kr energy lines Venos}).  
\begin{table*}[htb]
\caption{The frequencies of the conversion electron lines recorded in the shallow trap configuration. The N2 and N3 lines are not resolved but their frequencies are fitted separately by fixing the intensity ratio and separation between the two lines. The conversion electron line energies are calculated by fixing the gamma energy at the literature value at 32151.6 eV, and using the binding energies and recoil energies from \cite{V_nos_2018}. The 0.5-eV energy scale uncertainty from the gamma energy is not included.} 
\label{table:Kr energy lines Venos}
  \centering
  \renewcommand{\arraystretch}{1.15}
    \begingroup
    \setlength{\tabcolsep}{6pt} % Default value: 6pt
  \begin{tabular}{ l  c  c  c  c}
 \hline\hline
 Line & Conversion & Binding & Recoil &  \phantom{aa}Shallow trap frequency \\  
  & electron &  energy (eV) &  energy (eV) & (kHz) \\ & energy (eV) \\
 \hline\hline
 K & 17\,824.23(4)\phantom{a} & 14\,327.26(4) & 0.120 & 25\,940\,625.2(8)\phantom{a}\\ 
 \hline
 L2 & 30\,419.49(6)\phantom{a} & 1\,731.91(6) & 0.207 & 25\,337\,157.0(6)\phantom{a} \\
 
 L3 & 30\,472.19(5)\phantom{a} & 1\,679.21(5) & 0.207 & 25\,334\,690.7(8)\phantom{a} \\
 \hline
 M2 & 31\,929.26(17) & 222.12(17)  & 0.218 & 25\,266\,701.5(21) \\
 
 M3 & 31\,936.85(11) & 214.54(11) & 0.218 & 25\,266\,348.0(11) \\  
 \hline
 N2 & 32\,136.72(1)\phantom{a} & 14.67(1) & 0.219 & 25\,257\,051.7(27) \\
 N3 & 32137.39(1)\phantom{a} & 14.00(1) & 0.219 & 25\,257\,019.2(27) \\
 \hline\hline
\end{tabular}
\endgroup
\end{table*}
The mean magnetic field $B$ is a fit parameter. \autoref{fig:linearity} shows the frequency-energy relation, demonstrating good agreement.
\begin{figure}
    \centering
    \includegraphics[width = 1\columnwidth]{Plots/2-models/single_parameter_energy_vs_freq.pdf}
    \caption{Fit via \autoref{eq:energytofrequency} of the measured frequencies of conversion lines to their kinetic energies as given in \autoref{table:Kr energy lines Venos}.  From right to left the lines are K, L2, L3, M2, M3, N2 and N3.  The magnetic field is the fit parameter.  The error bars do not include a 0.5-eV energy scale uncertainty from the gamma energy.}
    \label{fig:linearity}
\end{figure}
The magnetic field found in the fit is $B = 0.959023787(42)$\,T.  The points in the residual plot below the figure illustrate the good internal agreement of the data with the  equation. The conversion line energies are calculated by fixing the gamma energy at the literature value 32151.6 eV provided in \cite{V_nos_2018}. An improvement in the gamma energy measurement is planned by the KATRIN collaboration~\cite{Rodenbeck:2022fxc} and could also be made via CRES with a precise independent determination of the magnetic field by NMR.


\subsection{\texorpdfstring{$^{\mathrm{83m}}$Kr}{} fit procedure with CRES spectrum model}
\label{subsec:Kr Line model}
For the remainder of this paper, all $^{\mathrm{83m}}$Kr fits use the CRES spectrum model in \autoref{eq:FullModel1} to fit the 17.8-keV conversion-electron line. Since the structure of this CRES spectrum model is common between $^{\mathrm{83m}}$Kr and tritium data, we can use $^{\mathrm{83m}}$Kr fits to calibrate the tritium energy point-spread function $\mathcal{R}_{\mathrm{PSF}}$ and detection efficiency curve $\epsilon$. The 17.8-keV $^{\mathrm{83m}}$Kr line is a powerful tool, given its narrow (2.774-eV) natural line width, well-understood shape, and closeness to the tritium endpoint at 18.6 keV. 
The underlying spectrum $\mathcal{Y}_{\mathrm{Kr}}$ includes the 17.8-keV $^{\mathrm{83m}}$Kr  main peak with its natural line width, as well as a lower-energy satellite structure from shakeup and shakeoff \cite{Robertson:2020dd}.

Fits to $^{\mathrm{83m}}$Kr data are performed by minimizing a Poisson likelihood $\chi^2$ \cite{Baker:ChiSquare1984},
\begin{align}
    \chi^2_{P} = 2 \sum \big[y_i - n_i + n_i\ln{(n_i/y_i)}\big]\,,
\end{align}
where $y_i$ is the expected number of events in bin $i$ according to Eqs.~\ref{eq:FullModel1} and \ref{eq:FullModel2}, and $n_i$ is the measured  number of events in that bin. 
In this fit,  the magnetic field $B$ and scattering parameters $p$ and $q$ are left free. For the fit to the deep quad trap data, the scatter fractions $\gamma_i$ are inputted, while for the shallow trap data, the scatter fractions for  H$_2$ and He are extracted from the fit, as motivated in \autoref{L_model}.
In the final fits, the detection efficiency variation with frequency $\epsilon$ (as determined in \autoref{subsec:efficiency}) is included in the model.
No background component is included in the fits due to the short run durations and negligible expected background rate. Section~\ref{subsec:electron_data} explains the reasons for expecting negligible background, and \autoref{subsec:background_limit} confirms this assumption.

Numerical scatter peaks serve as fixed inputs to the fitting function. These scatter peaks are produced by convolving data-set-specific simulated instrumental resolutions $\mathcal{I}$ (see \autoref{subsec:ins_res}) with electron energy loss spectra $\mathcal{L}_{\mathrm{tot}}$. To determine $\mathcal{L}_{\mathrm{tot}}$, loss spectra are combined according to \autoref{eq:combine_peaks_from_different_gases}, considering gases present in $^{\mathrm{83m}}$Kr data: Kr, $^3$He, Ar, and H$_2$ and its isotopologues.
 
Because the spectra contain many bins with zero or few counts,  goodness-of-fit testing is performed by MC sampling of the Poisson $\chi^2$, treating the fitted spectrum as the truth, as suggested in~\cite{Baker:ChiSquare1984}. The Poisson $\chi^2$ for a recorded spectrum is compared to the distribution generated from the MC simulation to check the goodness of fit.

\subsection{Instrumental resolution \texorpdfstring{$\mathcal{I}$}{}}\label{subsec:ins_res}
The instrumental resolution $\mathcal{I}$ is an input to $^{\mathrm{83m}}$Kr fits. This resolution describes the broadening of cyclotron frequency due to the dependence on pitch angle and radius of the average magnetic field seen by a trapped monoenergetic electron. 
$\mathcal{I}$  is determined for each trap configuration by simulation with Project 8's Locust software package \cite{AshtariEsfahani:2019mwv} as described in \autoref{sec:simCRES}.
Equations~\ref{eq:FullModel1} and \ref{eq:FullModel2} show the role of $\mathcal{I}$ in the CRES spectrum model used for  $^{\mathrm{83m}}$Kr fits.


\subsubsection{SNR scaling optimization}\label{sec:max-SNR-optimization}
Each instrumental resolution $\mathcal{I}$ has an associated value of SNR$_{\rm max}$, the SNR of a $90^{\circ}$ electron in trap 3 at $r=0\,\si{mm}$. SNR$_{\rm max}$ mostly affects the width of $\mathcal{I}$ while maintaining the distribution's overall shape. In particular, higher SNR$_{\rm max}$ corresponds to wider $\mathcal{I}$ distributions because the overall SNR in track bins is higher, making electrons with smaller pitch angles more detectable. These small-pitch-angle electrons explore a larger range of magnetic fields, broadening the detected frequency spectrum.

The total system gain and noise temperature are not known well enough for each data set to determine SNR$_{\rm max}$. As a result, the width of $\mathcal{I}$ is not exactly known. To find SNR$_{\rm max}$ for a given $^{\mathrm{83m}}$Kr data set, 60 fits are performed using inputted $\mathcal{I}$ distributions corresponding to SNR$_{\rm max}$ values ranging from 12 to 18.  When simulating these $\mathcal{I}$ distributions, the SNR of all events is scaled relative to SNR$_{\rm max}$ before the efficiency filter is applied.  We add a fit parameter to the $^{\mathrm{83m}}$Kr model: a scale factor $s$, which widens or compresses $\mathcal{I}$ during the fit. When $s=1$, this indicates that $\mathcal{I}$ has the best width to describe the data, and thus the best SNR scale. We fit the 60 (SNR$_{\rm max}$, $s$) points to a quadratic function and predict the SNR$_{\rm max}$ for $s=1$. This procedure anchors SNR$_{\rm max}$ to experimental data. It also produces a best-estimate for the standard deviation of $\mathcal{I}$, for each $^{\mathrm{83m}}$Kr data set.


For pre-tritium $^{\mathrm{83m}}$Kr data, the result of the SNR scaling procedure (SNR$_{\rm max}$ = 14.3) is shown in \autoref{fig:max_snr_optimization_oct_data}. Figure~\ref{fig:kr_ftc_res} displays the resulting $\mathcal{I}$ for this data set. To simulate $\mathcal{I}$ for the tritium analysis, we use the same SNR$_{\rm max}$ value as in the pre-tritium 
$^{\mathrm{83m}}$Kr data set, since its properties most closely resemble those of tritium data (see \autoref{tab:mml_track_length_fit_results}). The two data sets are only distinguishable in track duration, which has a sub-dominant effect on the width of $\mathcal{I}$.

\begin{figure}[htb]
    \centering
    \includegraphics[width=1\columnwidth]{Plots/3-systematic-uncertainties/3-2-instrumental-resolution/october_max_snr_optimization_plot_weights.pdf}
    \caption{SNR$_{\rm max}$ optimization for $\mathrm{^{83m}Kr}$ pre-tritium data. The vertical axis is the scale factor $s$ that adjusts the simulated width to match the experimental one for each choice of SNR$_{\rm max}$ in the simulation.}
    \label{fig:max_snr_optimization_oct_data}
\end{figure}


\begin{figure}
\centering   
\includegraphics[width=\columnwidth]{Plots/2-models/Kr_res_pre-tritium_errorbars.pdf}
\caption{Instrumental resolution $\mathcal{I}$ of the pre-tritium $^{\mathrm{83m}}$Kr calibration data set. The error bars include uncertainties from Poisson counting, the efficiency matrix, and the trap weights.}
\label{fig:kr_ftc_res}
\end{figure}

\subsubsection{Uncertainties on \texorpdfstring{$\mathcal{I}$}{} propagated to \texorpdfstring{$^{\mathrm{83m}}$Kr}{} fit results}\label{sec:res-errors-to-Kr}
\label{sec:instrumental_resolution_uncertainty_propagation}

A simulated, fixed $\mathcal{I}$ distribution is inputted to each $\mathrm{^{83m}Kr}$ K-line fit listed in \autoref{tab:mml_track_length_fit_results}. As a result, uncertainties on $\mathcal{I}$ propagate to the fit parameter results ($p$, $q$ and $B$), which in turn feed into the tritium analysis (see Secs.~\ref{sec:Bfield_errors} and~\ref{sec:scatter_peak_errors}). Thus, we estimate the uncertainties in $p$, $q$ and $B$ due to both $\mathcal{I}$ simulation uncertainties and SNR$_{\rm max}$ uncertainties.

For each data set, $\mathcal{I}$  simulation uncertainties are obtained from 100 bootstrapped resolution shapes, which are produced by repeatedly sampling counts in all bins from Gaussian distributions. Each Gaussian's standard deviation equals the bin simulation uncertainty, which includes uncertainties from Poisson counting, the efficiency matrix, and the trap weights. $\mathrm{^{83m}Kr}$ K-line fits are then repeated 100 times, once with each bootstrapped $\mathcal{I}$ as input, to obtain uncertainty distributions for fit parameters. 
Separately, we estimate the SNR$_{\rm max}$ contribution to $\mathrm{^{83m}Kr}$ fit parameter uncertainties. To do so, we fit the data 100 times, each time using an inputted resolution simulated with a different SNR$_{\rm max}$ value sampled from a normal distribution (with a mean from the procedure in \autoref{sec:max-SNR-optimization} and an uncertainty calculated as described in \autoref{sec:sigma_errors}). 
Simulation and SNR$_{\rm max}$ uncertainties are added in quadrature.


\subsection{Field-shifted \texorpdfstring{$\mathrm{^{83m}Kr}$}{} data analysis}
\label{subsec:frequency_dependence}

\begin{figure}[b]
  \centering
  \includegraphics[width=1.0\columnwidth]{Plots/3-systematic-uncertainties/3-3-efficiency/fss_in_q300.pdf}
  \caption{
    The 17.8-keV $\mathrm{^{83m}Kr}$ conversion electron line recorded in the deep quad trap at different magnetic background fields (red / blue).
  }
  \label{fig:q300_fss}
\end{figure}
 
\subsubsection{Detection efficiency measurement}
\label{subsec:efficiency}
Detection efficiency as a function of frequency is an input to the CRES spectrum model. To study the frequency response, we recorded $\mathrm{^{83m}Kr}$ data at a range of background magnetic field values, as described in \autoref{fss_procedure} and in the ``$\mathrm{^{83m}Kr}$ field-shifted'' row in \autoref{tab:mml_track_length_fit_results}. Data were taken in the full quad trap configuration as well as in each individual trapping coil in isolation.
A subset of the $\mathrm{^{83m}Kr}$ K-line data recorded in the quad trap configuration is shown in \autoref{fig:q300_fss}. 
To measure the detection efficiency vs.~frequency, we extracted $\epsilon(f_c)$ at the frequency center of each recorded peak by fitting the data with a reduced version of the full CRES spectrum model that does not include $\epsilon(f_c)$. The number of reconstructed events within $\pm$1~$\si{MHz}$ of the fitted peak's frequency location is compared to the number of events for the data at the unshifted background field ($B=B_0$).
The motivation for the start-frequency cut of $\pm1\,\si{MHz}$ around the peak center is to not average the detection efficiency over a larger frequency range while maintaining a high statistical precision for the efficiency analysis.
The obtained relative count rate vs.~frequency in a given trap (shown in \autoref{fig:fss_event_rates}) is equivalent to the relative $\epsilon(f_c)$ in this trap for (quasi) mono-energetic data like the $\mathrm{^{83m}Kr}$ K-line (the energy spread of K-line electrons is small compared to the resolution width $\mathcal{I}$). For tritium data analysis in the quad trap, $\epsilon$ is summed from the single-trap count rates after a correction for the dependence of SNR on kinetic energy. We motivate and describe this correction in \autoref{sec:efficiency_for_tritium}. 

\subsubsection{Extraction of statistical trap weights}
\label{subsec:trap_weights}
The statistical trap weights $w_i$ correspond to the relative number of detected events in each trap. These weights are used for two purposes: to sum the simulated instrumental resolutions of the 4 traps that compose the quad trap, and to correct the measured efficiency variation with frequency for the tritium analysis as will be discussed in \autoref{sec:energy_correction}.
We extract the $w_i$ at $B=B_0$ from the field-shifted data by minimizing the summed squared differences between the quad trap count rates vs.~frequency and the weighted sum of the single-trap count rates vs.~frequency, with $w_i$ being free parameters.
The resulting weights are $w_1 = 0.076(3)$, $w_2 = 0.341(13)$, $w_3 = 0.381(14)$, and $w_4 = 0.203(20)$, which are in good agreement with the observed count rate differences at $B=B_0$.

Note that the field step sizes are \SI{0.07}{\milli\tesla} in traps 2, 3, and the quad trap. We chose the step sizes in trap 1 and trap 4 to be \SI{0.7}{\milli\tesla} to reduce the total duration of these field-shifting scans for the traps with the  lowest count rate at the nominal frequency position of the K-line($\approx 25.91\,\si{GHz}$).
For the summation, the count rates from trap 1 and 4 are interpolated linearly.
The uncertainties in the interpolated frequency ranges are taken to be equal to the largest deviation from a linear interpolation over the same range in trap 2 or 3 (shown in grey in \autoref{fig:fss_event_rates}). 

\begin{figure}
%\centering   
\includegraphics[width=1.0\columnwidth]{Plots/3-systematic-uncertainties/3-3-efficiency/all_trap_rates_with_residual_plot.pdf}
\caption{Event detection rates from $^{\mathrm{83m}}$Kr K-line data recorded with different single-coil traps (red and blue) and the quad trap (black) in field-shifted data relative to the respective count rate at $f_c \approx 25.91\,\si{GHz}$ (where $B=B_0$). The uncertainties from interpolation are shown in grey. The relative count rates can be summed with weights (green) to match the quad-trap count rate curve (black). The residuals show the differences of the summed single-trap rates and the quad-trap rates divided by the quad-trap count rate uncertainties. The standard deviation of the residuals is larger than 1 and the uncertainties on the tritium efficiency $\epsilon$ (\autoref{sec:energy_correction}) are inflated to account for this.}
\label{fig:fss_event_rates}
\end{figure}






\subsection{\texorpdfstring{$^{\mathrm{83m}}$Kr}{} shallow-trap data and fits \label{sec:shallow_trap}}

To explore the best resolution achievable in Phase II, and to test the CRES spectrum model (\autoref{eq:FullModel1}), we took $^{\mathrm{83m}}$Kr data with the trap coil currents set to the shallow trap configuration in \autoref{fig:quadtrapcoils}.
\begin{figure}[b]
  \centering
\includegraphics[width=1.0\columnwidth]{Plots/2-models/shallow_trap_data_and_fit_with_residuals_bands_and_extras.pdf} \caption{The 17.8-keV $^{\mathrm{83m}}$Kr K-conversion electron line, as measured with CRES in the shallow (high-resolution) electron trapping configuration, with FWHM of 4.0\,eV. The data are the $^{83\mathrm{m}}$Kr shallow data set (Table \ref{tab:mml_track_length_fit_results}).
  }
  
  \label{fig:krypton shallow trap}
\end{figure}
Figure~\ref{fig:krypton shallow trap} shows the fit to these data.
Also shown is the underlying $^{\mathrm{83m}}$Kr lineshape model $\mathcal{Y}_{\mathrm{Kr}}$, which includes both the main peak and the shakeup/shakeoff satellites. 
The figure displays intermediate lineshapes in which additional model elements are added one by one, to exhibit the effects of magnetic field inhomogeneity (treated as equivalent to instrumental resolution $\mathcal{I}$) and scattering. 
In the shallow trap, there are only small differences between the average magnetic fields experienced by trapped electrons with different pitch angles. Accordingly, the broadening from $\mathcal{I}$ (purple curve) is $1.66(19)$\,eV FWHM.
This combines with the natural linewidth of 2.774\,eV FWHM~\cite{Altenmuller:2019ddl} to produce a main peak with a FWHM of 4.0\,eV.  The Gaussian component of the Voigt profile used in \autoref{fig:linearity_Kr_K_freq_line} has a FWHM of $2.10(6)$\,eV, in reasonable agreement given that $\mathcal{I}$ is not Gaussian. Out of all events, 69\% are detected before scattering. Additional curves in \autoref{fig:krypton shallow trap} show events detected after a single scatter and after up to 20 scatters.  In the low-energy tail (below 17.814 eV), scattering events comprise 61\% of counts.

The summed $\chi^2$ of the binned data falls within 1\,$\sigma$ of the mean of the distribution of summed $\chi^2$ values from MC simulations, verifying goodness of fit. This demonstrates the high-resolution capabilities of CRES and validates the $^{\mathrm{83m}}$Kr model. 


\subsection{\texorpdfstring{$^{\mathrm{83m}}$Kr}{} pre-tritium and post-tritium quad trap data and fits\label{sec:deep_trap}}\label{sec:deep-trap-data-and-fits}

\begin{figure}[b]
  \centering
  \includegraphics[width=1.0\columnwidth]{Plots/2-models/deep_trap_data_and_fit_with_residuals_bands_and_extras.pdf}
  \caption{
    The 17.8-keV $^{\mathrm{83m}}$Kr K-conversion electron line, as measured with CRES in the deep (high-statistics) electron trapping configuration, with FWHM of 54.3\,eV. The data are the $^{83\mathrm{m}}$Kr pre-tritium data set (Table \ref{tab:mml_track_length_fit_results}).
  }
  \label{fig:krypton deep trap}
\end{figure}

For the small apparatus in Project 8's Phase II, low statistics are the limiting factor in neutrino mass sensitivity. Thus, to maximize statistics, we took the final tritium data in a trap configuration, shown in \autoref{fig:quadtrapcoils}, that was deeper than the shallow trap and was composed of four individual traps rather than two. 
The $^{\mathrm{83m}}$Kr pre-tritium and post-tritium data sets (see \autoref{tab:mml_track_length_fit_results}) were taken in the same deep quad trap, to provide calibrations of the effective mean magnetic field $B$ and the scattering parameters $p$ and $q$ as inputs for the tritium analysis. \autoref{fig:krypton deep trap} shows the $^{\mathrm{83m}}$Kr pre-tritium data and fit.

The $^{\mathrm{83m}}$Kr line shape is significantly broadened by the 35.6\,eV FWHM instrumental resolution $\mathcal{I}$, due to the larger range of trapped pitch angles, and therefore larger range of average magnetic fields, experienced by electrons in this deeper trap. 
The larger pitch angle acceptance causes more electrons to remain trapped after scattering, leading to a higher average number of tracks per event. 
The larger acceptance also means that events that begin in non-detectable pitch angles have a larger phase space of detectable pitch angles to scatter into; therefore, detected events are more likely to come from this pool.
Therefore, a smaller proportion (53\%) of events are detected before scattering in the deep quad trap than in the shallow trap. 
This gives rise to the enhanced low-energy tail and brings the FWHM to 54.3\,eV. 
Note that for the shallow trap spectrum, the scattering peaks contribute minimally to the width, while for the deep quad trap data, the scattering accounts for a significant portion of the width.

\subsubsection{Uncertainties on \texorpdfstring{$^{\mathrm{83m}}$Kr}{} quad trap fit results \label{sec:Kr-quad-trap-uncertainties}}

For the $^{\mathrm{83m}}$Kr pre-tritium and post-tritium data sets, the comparison of the summed $\chi^2$ for binned data with that of the MC simulation indicated underfitting.
To account for the uncertainty associated with this tension, the uncertainties for $B$, $p$ and $q$ from the maximum likelihood fit are inflated by 17\% and 5\% for the pre-tritium and post-tritium data sets, respectively.
These fit uncertainties are combined with the larger uncertainty contributions from $\mathcal{I}$ and gas composition to produce the total uncertainties on $B$, $p$ and $q$.\footnote{When fit uncertainties are inflated to account for underfitting, this increases the total uncertainties on $B$, $p$ and $q$ by only 1.6\%, 0.3\% and 0.1\%, respectively.} Uncertainties on $\mathcal{I}$ are propagated using the sampling-and-refitting method described in \autoref{sec:res-errors-to-Kr}. The uncertainty on $\mathcal{I}$ due to SNR$_{\rm max}$ is not propagated to $B$, since those variables are independent. SNR$_{\rm max}$ primarily affects the width of $\mathcal{I}$, while $B$ controls the location of the distribution's center; these are two separate moments of $\mathcal{I}$.
To propagate the uncertainty from gas composition, the $^{\mathrm{83m}}$Kr fits are repeated 300 times while sampling the inputted gas scattering contributions from the distributions defined in \autoref{tab:scattering_fraction_results}. The gas composition uncertainties on $B$, $p$ and $q$ are the standard deviations of results from these 300 fits. 

With fit, $\mathcal{I}$, and gas composition uncertainties included, the best estimates of $B$ from $^{\mathrm{83m}}$Kr pre- and post-tritium data differ by 1.6\,$\sigma$. Estimates for $p$ and $q$ are not expected to be consistent between the two quad trap data sets, due to a difference in the mean number of tracks per event (see \autoref{sec:scatter_peak_errors}).


\subsubsection{Validating \texorpdfstring{$^{\mathrm{83m}}$Kr}{} scatter peak amplitudes with simulation}

The amplitude $\mathcal{A}_j(p, q)$ of scatter peak $j$ is proportional to the probability of missing $j$ tracks before detecting an event. Simulations are used to validate the model for $\mathcal{A}_j$ (\autoref{eq:scatter_peak_amplitude}) as well as the fitted values of $p$ and $q$ for quad trap $^{\mathrm{83m}}$Kr data.

\label{subsubsec: scatter peak amplitude simulation}
In the Locust simulations performed to obtain the instrumental resolution $\mathcal{I}$, pitch angle changes are ignored and assumed to be zero. When one is concerned with the properties of first tracks (e.g.,~the start frequency spread captured by $\mathcal{I}$), that approach is sufficient. However, this simplification does not allow the prediction of accurate $A_j$ values. Therefore, we perform a set of toy model simulations (not using Locust) in which inelastic and elastic scattering are modeled separately, with pitch angle changes included. The simulated events are not processed with the Phase~II event reconstruction algorithms. Instead, the event detection process is approximated by power and track length cuts.

For the simulations, it is assumed that inelastic scattering leads to energy loss and small pitch angle changes, while elastic scattering removes electrons from the trap before the next inelastic scattering event~\cite{DavidJoy:ElectronScattering}. 
The inelastic scattering angle $\theta$ follows the distribution~\cite{Rudd:1991differential}
\begin{align}
    P(\theta) \propto \left(1 + \frac{\cos^2\theta}{\alpha^2}\right)^{-1}\,,
\label{eq:inelastic scatter angle}
\end{align}
where $\alpha$ is a data-set-specific constant. The elastic scattering is modeled by assuming a fixed fraction $\kappa$ of electrons leave the trap between inelastic scatters due to elastic scatters. As before, the track duration follows the exponential distribution in \autoref{eqn:track_length_distribution}. The coupled power from a radiating electron is calculated given its instantaneous pitch angle and axial position. 
The detection status of an electron is determined by whether the electron power and track length are both above the corresponding preset thresholds. The SNR threshold chosen matches the threshold applied to tracks in real data by the Phase~II track detection algorithm, prior to combining single tracks into full events (see \cite{TERpaper:2022}). Similarly to in \autoref{sec:sim-events}, it is assumed that the highest simulated power corresponds to the estimated maximum SNR observed in data. The power of all simulated events is translated to SNR accordingly. The track length detection threshold  is set to the minimum recorded track length in data. Tuning $\alpha$ and $\kappa$,  the $A_j$ curve and the mean event length can be simultaneously matched to values extracted from $\mathrm{^{83m}Kr}$ data (see \autoref{fig:scatter-peak_amplitude_simulation}).

This simulation cannot be directly used to predict $A_j$ in tritium data, since $\alpha$ and $\kappa$ are related to the gas composition and differ among data sets. Instead, the values of  $p$ and $q$ for tritium data are found by extrapolating from the $p$ and $q$ results of the pre-tritium and post-tritium $\mathrm{^{83m}Kr}$ fits, as described in \autoref{subsubsec:pq_extrapolation}.

\begin{figure}
\centering
\includegraphics[width=\linewidth]{Plots/3-systematic-uncertainties/3-5-scatter-peak-model/deep_trap_scatter_amplitudes_updated.pdf}
\caption{Number of simulated events that are detected after $j$ scatters. The relative scatter amplitudes $\mathcal{A}_j$ (modeled by \autoref{eq:scatter_peak_amplitude}) result from the combination of track detection probability and pitch angle changes for electrons after scattering (modeled by \autoref{eq:inelastic scatter angle}). The number of detected electrons in each scatter can be modeled by a modified exponential function (solid lines) parameterized by $p$ and $q$. Fitting $p$ and $q$ for pre-tritium $\mathrm{^{83m}Kr}$ data results in an $\mathcal{A}_j$ curve (green) in very good agreement with simulation (orange).}
\label{fig:scatter-peak_amplitude_simulation}
\end{figure}


\section{Tritium model \label{sec:tritium_model}}

\subsection{An analytic model for tritium data analysis\label{sec:T2-beta-model-context}}

This section describes an analytic model of tritium CRES data. The model is used in Bayesian and frequentist analyses (discussed in \autoref{sec:final-analysis}) to measure the tritium endpoint $E_0$ and to place a limit on the neutrino mass $m_{\mathrm{\beta}}$. The tritium model has a similar structure to the $^{\mathrm{83m}}$Kr model, with differences as described in subsections~B-D, below.

 In the tritium model, the beta spectrum $\mathcal{Y}_{\mathrm{tritium}}$ and response function $\mathcal{R}_{\mathrm{PSF}}$ are approximated to make it possible to analytically convolve the two functions, producing a fully analytic likelihood model. This enables computationally efficient inference. Such efficiency is crucial for the Bayesian analysis (described in \autoref{sec:Bayesian-analysis}), since algorithms that perform Bayesian inference in many dimensions---corresponding to many nuisance parameters---tend to be slow at numerical integration. The analytic model of $\mathcal{R}_{\mathrm{PSF}}$ also substantially speeds up the calculation of the response in the frequentist inference.
 
 The tritium model described below includes several approximations, since a more exact model would be non-analytic. These approximations have a minimal effect on $E_0$ and $m_{\mathrm{\beta}}$ results.  Monte Carlo studies test the impact of approximations, as discussed in-detail in \autoref{sec:MCstudies}.

\subsection{Tritium beta decay model\label{sec:T2-beta-model}}
In the tritium model, the underlying spectrum $\mathcal{Y}_{\mathrm{tritium}}$ is the tritium beta spectrum, given by the product of neutrino and electron phase space density factors $D_{\nu}$ and $D_e$. Equation~\ref{eq:FullModel1} shows how this underlying spectrum is used in the full model for CRES data.  When experimental sensitivity is insufficient to resolve individual mass eigenstates, the beta spectrum is given by~\cite{Formaggio:2021nfz} %neutrino phase space density
\begin{equation}
\begin{split}
    &\mathcal{Y}_{\mathrm{tritium}} = D_{\nu} \cdot D_e, \quad \mathrm{where} \\
    &D_{\nu} \propto  \epsilon\big[\epsilon^2 -  m_\beta^2\big]^{1/2} \Theta(\epsilon-m_\beta) \\
    &D_e \propto F(Z, p_\mathrm{e})p_\mathrm{e} E_\mathrm{e}.
    \label{eq:betaspectrumfull}
\end{split}
\end{equation}
In this equation, $\epsilon = E_0 - E_\mathrm{kin}$, $\Theta$ is the Heaviside step function,  $F(Z, p_\mathrm{e})$ is the relativistic Fermi function for charge $Z=2$ of the daughter nucleus, and $p_\mathrm{e}$ are $E_\mathrm{e}$ are the electron momentum and total energy, respectively.  The frequentist analysis model uses \autoref{eq:betaspectrumfull} for the beta spectrum. 

In the Bayesian analysis, $D_\nu$ is approximated according to the formalism in~\cite{AshtariEsfahani:2021moh}. This involves Taylor expanding in $m_\beta^2$ to produce the expression
\begin{equation}
    D_{\nu} \approx  \big[\epsilon^2 -  m_\beta^2/2\big] \Theta(\epsilon-m_\beta).
    \label{eq:betaspectrum}
\end{equation}
In addition, $D_e$ is Taylor expanded to first order around the energy at the center of the  analysis region of interest (ROI), neglecting atomic physics factors that correct the Fermi function. These are good approximations for the Phase II event rate and ROI. The resulting model for $\mathcal{Y}_{\mathrm{tritium}}$ may be analytically convolved with a normal distribution. Thus, for any model of $\mathcal{R}_{\mathrm{PSF}}$ that is expressed as a weighted sum of Gaussians, the full tritium model is analytic. This enables computationally efficient inference, as was discussed in \autoref{sec:T2-beta-model-context}. Here, unlike in Ref.~\cite{AshtariEsfahani:2021moh}, the low-energy edge of the spectrum is not smeared out by magnetic field broadening, since a hard maximum-frequency cut is performed before analysis.\footnote{A low-energy smearing of 0.001\,eV is included in the model for computational stability (negligibly affecting analysis results). In a Bayesian MCMC analysis, infinitely steep drops in probability density can cause Markov chains to behave pathologically~\cite{Stanual2022, Betancourt2015}.}   

The decay of molecular tritium creates a daughter molecule, $^3$HeT$^+$, that can be in an excited state. Thus, a beta spectrum fit must account for the final state distribution, i.e., the probability distribution for how much energy is supplied to rotational, vibrational, and electronic excitations of $^3$HeT$^+$ during the decay.  We use the distribution calculated by Saenz et al.~for $^3$HeT$^+$~\cite{Saenz:2000dul}. Neglecting this final state distribution leads to a bias of \SI{-8.1 \pm 0.8}{eV} on the endpoint energy, as determined by a Bayesian MC study. 
The analysis model uses a sparse approximation of the calculated distribution~\cite{Saenz:2000dul}, shown in \autoref{fig:MFS}. This saves computational time and, as determined by an MC study, does not introduce bias in the results.  The full tritium signal model is a weighted sum of beta decay signal functions with binding energies at the sparse values in \autoref{fig:MFS}, with the overall rate of each function scaled by the final state probability. This serves to discretely convolve the final state distribution with the beta spectrum.

We neglect contributions from the atomic tail extending to beyond \SI{1}{\kilo eV}
below the endpoint; these contributions are highly suppressed and have no measurable effect on the results.  While some electrons are produced by HT, which decays to $^3$HeH$^+$, this molecule's final state distribution is very similar to that of $^3$HeT$^+$, so the difference can be ignored for the energy point-spread function $\mathcal{R}_{\mathrm{PSF}}$ of our deep quad trap. 

\begin{figure}[!htb]
    \centering
    \includegraphics[width = \columnwidth]{Plots/2-models/Mfs_spectrum_sparse_tail_prc-d.pdf}
    \caption{Molecular final states of $^3$HeT$^+$ as calculated in \cite{Saenz:2000dul} and the sparse approximation used in this analysis.}
    \label{fig:MFS}
\end{figure}

\subsection{Tritium energy response function model}\label{sec:T2-det-response}
The $\mathrm{\beta}$-spectrum is convolved with $\mathcal{R}_{\mathrm{PSF}}$, the energy response point-spread function. $\mathcal{R}_{\mathrm{PSF}}$ is modeled as a sum of Gaussians, as motivated in \autoref{sec:T2-beta-model-context} and \autoref{sec:T2-beta-model}. As in the case of $^{\mathrm{83m}}$Kr data, $\mathcal{R}_{\mathrm{PSF}}$ is composed of scatter peaks $\mathcal{I}*\mathcal{L}_{\mathrm{tot}}^{*j}$ (see~\autoref{eq:FullModel2}). 
The scatter spectrum model includes gases H$_2$ and $^3$He, as these have the largest inelastic scatter fractions $\gamma_i$ in tritium data: \SI{0.911 \pm 0.045} and \SI{0.075 \pm 0.040}, respectively (see \autoref{sec:gas_composition} on scatter fraction analysis). At each successive scattering, the electron may scatter from H$_2$ or $^3$He in proportion to the species' abundance and cross section.  We simplify by modeling each scatter peak as a weighted sum of H$_2$ and $^3$He peaks (akin to assuming that each electron scatters off of the same gas type at the end of all missed tracks). An MC study confirmed that this simplification does not alter $E_0$ and $m_\beta$ results.  
Additional MC studies show that we may neglect a small scatter fraction \SI{0.014 \pm 0.009} of carbon monoxide present in tritium data, as well as energy loss from cyclotron radiation.

In the tritium model, each scatter peak $\mathcal{I}*\mathcal{L}_{\mathrm{tot}}^{*j}$
is expressed as a function of $\sigma$, the calculated standard deviation of the simulated tritium resolution $\mathcal{I}$. This enables us to propagate the uncertainty on the resolution width to the endpoint and neutrino mass, via $\sigma$. The $j=0$ peak is modeled separately from the $j\geq1$ peaks, as described below. The model includes $j_{\mathrm{max}}=20$  peaks, since an MC study shows that higher-order peaks have no noticeable effect on results.

For $j=0$, the ``scatter peak" reduces to $\mathcal{I}$. The procedure for simulating $\mathcal{I}$ is described in \autoref{sec:gen_resolution}. Because the $j=0$ peak contributes the most events to $\mathcal{R}_{\mathrm{PSF}}$, it is important to model the peak with a closely-fitting distribution. A simple Gaussian would underestimate the tails of the simulated peak and fail to account for the distribution's small asymmetry. Instead, $\mathcal{I}$ is fitted with a sum of two Gaussians, each described by a mean $\mu_0^{[i]}$ and standard deviation $\sigma_0^{[i]}$, with the distributions then weighted by a parameter $0\le\eta\le 1$:
 
\begin{equation}
\begin{split}
\mathcal{I}*\mathcal{L}_{\mathrm{tot}}^{*0} \approx \eta\mathcal{N}\Big(\mu_0^{[1]}, \sigma_0^{[1]}(\sigma)\Big) +(1-\eta)\mathcal{N}\Big(\mu_0^{[2]}, \sigma_0^{[2]}(\sigma)\Big).
\end{split}
\label{eq:T2_ins_res_model}
\end{equation}
 
As noted above, each scatter peak should be parameterized in terms of $\sigma$. Accordingly, the Gaussian standard deviations in \autoref{eq:T2_ins_res_model} must be functions of $\sigma$. To find these functions, we perform fits to simulated resolutions with a range of $\sigma$ values and observe how $\sigma_0^{[1]}$ and $\sigma_0^{[2]}$ depend on $\sigma$. To vary $\sigma$ in the simulations, the value of SNR$_{\rm max}$ is varied. As discussed in \autoref{sec:max-SNR-optimization}, $\sigma$ is determined primarily by SNR$_{\rm max}$.  The result of this procedure is $\sigma_0^{[1]}(\sigma) = 1.1\sigma + 1.9\,$eV and $\sigma_0^{[2]}(\sigma) = 0.8\sigma - 3.7\,$eV.\footnote{To determine these expressions for $\sigma_0^{[1]}$ and $\sigma_0^{[2]}$, we use three pieces of information:
\begin{enumerate}
\item Simulations show that $\sigma_0^{[2]} = 0.8\sigma_0^{[1]} - 5.2\,$eV for our instrumental resolution. We observe this linear relation by generating 100 simulated resolutions with different maximum detectable track SNRs (SNR$_{\rm max}$), then fitting each resolution with \autoref{eq:T2_ins_res_model}. The size of SNR$_{\rm max}$ controls the width of $\mathcal{I}$ (see \autoref{subsec:ins_res}). In these fits, the Gaussian means, Gaussian standard deviations and $\eta$ are all fitted.
\item We observe that $\sigma \approx \eta\sigma_0^{[1]} + (1-\eta)\sigma_0^{[2]}$ for the same 100 simulated resolutions. The relation is approximate because an exact expression for $\sigma$ must depend on $\mu_0^{[1]}$ and $\mu_0^{[2]}$. In that relation, the right hand side is computed from fit results, and $\sigma$ is calculated directly from the simulated resolutions. 
\item We observe that $\eta=0.66$ is constant within fit errors, for the SNR$_{\rm max}$ uncertainty range in tritium data (quantified in \autoref{sec:sigma_errors}).
\end{enumerate}
Combining the first and second relations, and fixing $\eta$, we find $\sigma_0^{[1]}(\sigma)$ and $\sigma_0^{[2]}(\sigma)$---expressions that depend only on $\sigma$ and constants.} This procedure also demonstrates that $\eta=0.66$ is constant as $\sigma$ varies. By fixing $\eta$ and plugging the expressions for $\sigma_0^{[1]}(\sigma)$ and $\sigma_0^{[2]}(\sigma)$ into \autoref{eq:T2_ins_res_model}, we obtain a model with only three free parameters: $\sigma$, $\mu_0^{[1]}$ and $\mu_0^{[2]}$.
This ``reduced model" fits the simulated resolution data closely. As a check, we confirm that the fitted value of $\sigma$ matches the standard deviation $\sigma$ calculated from points in a given $\mathcal{I}$ distribution.

\begin{figure}[t]
  %\centering
  \includegraphics[width=1.0\columnwidth]{Plots/2-models/ins_res_sum_of_gaussians_different_means_maxSNR-14-3_official7.pdf}
  \caption{
   Fit of \autoref{eq:T2_ins_res_model} to the simulated instrumental resolution $\mathcal{I}$ for tritium data. Bin errors are approximately Gaussian, from sources described in \autoref{subsec:ins_res}. Fit parameters are means of the two normal distributions ($\mu_0^{[1]}$, $\mu_0^{[2]}$) and the resolution standard deviation~$\sigma$. The fraction of counts $\eta$ in the first normal distribution is inputted; we use the average $\eta$ from fits to 100 resolutions with different $\sigma$ values.}
  \label{fig:T2_res_fit}
\end{figure}


The reduced model for the $j=0$ peak can scale in width based on the $\sigma$ parameter, and the model captures the slight asymmetry in $\mathcal{I}$ via $\mu_0^{[1]}$ and $\mu_0^{[2]}$. For tritium data analysis, to find best estimates of $\sigma$, $\mu_0^{[1]}$ and $\mu_0^{[2]}$, we fit the reduced model to an $\mathcal{I}$ distribution generated with the best-estimate SNR$_{\rm max}$ value.  Figure~\ref{fig:T2_res_fit} displays this distribution and fit curve ($\chi^2/\mathrm{ndf} = 44/58$). The fit energy range is limited to produce a good fit to the central region of the resolution, which has the largest impact on the tritium analysis. Uncertainties on $\sigma$ are propagated to the endpoint and neutrino mass results (see \autoref{sec:sigma_errors}), since these results would shift if we under- or over-estimated the widths of scatter peaks. In contrast, $\mu_0^{[1]}$ and $\mu_0^{[2]}$ are fixed in the tritium model because their uncertainties are found to have a negligible effect and are not propagated.


When $j\geq1$, scatter peaks $\mathcal{I}*\mathcal{L}_{(\mathrm{H}_2, \mathrm{He})}^{*j}$ are again modeled as a function of the resolution standard deviation $\sigma$. Monte Carlo studies show that these higher-order peaks can each be modeled as Gaussian without biasing $E_0$ or $m_\beta$ results, despite the asymmetry in $\mathcal{L}_{\mathrm{tot}}^{*j}$. This is the case because the peaks are broadened by $\mathcal{I}$, they overlap with each other, and they each contribute sub-dominantly to the $\mathcal{R}_{\mathrm{PSF}}$. Employing Gaussian distributions allows analytic convolution of $\mathcal{R}_{\mathrm{PSF}}$ with the $\beta$-spectrum model, simplifying computation, as discussed in \autoref{sec:T2-beta-model-context} and \autoref{sec:T2-beta-model}. For a given gas species, each scatter peak term is characterized by a mean and standard deviation that depend only on $\sigma$. The full scatter peak is a weighted sum of these terms:

\begin{equation}
\begin{split}
&\mathcal{I}*\mathcal{L}_{\mathrm{tot}}^{*j} (j\geq1)  \approx \gamma_{\mathrm{H}_2}\Big[\mathcal{I}*\mathcal{L}_{(\mathrm{H}_2)}^{*j}\Big] + (1-\gamma_{\mathrm{H}_2})\Big[\mathcal{I}*\mathcal{L}_{(\mathrm{He})}^{*j}\Big] \\ &\!\!\!\!\!\rightarrow  \gamma_{\mathrm{H}_2}\mathcal{N}\Big(\mu_j^{\mathrm{H}_2}(\sigma), \sigma_j^{\mathrm{H}_2}(\sigma)\Big)  +(1\!-\!\gamma_{\mathrm{H}_2})\mathcal{N}\Big(\mu_j^{\mathrm{He}}(\sigma), \sigma_j^{\mathrm{He}}(\sigma)\Big),
\end{split}
\label{eq:simplified_scattering}
\end{equation}

\noindent where $\gamma_{\mathrm{H}_2}$ is the hydrogen inelastic scatter fraction and $1-\gamma_{\mathrm{H}_2}$ is the helium inelastic scatter fraction.  The Gaussian means $\mu_j$ depend on $\sigma$ because $\mathcal{L}_{\mathrm{tot}}$ is asymmetric, so the convolution with $\mathcal{I}$ can shift the center of each scatter peak when $\sigma$ is large enough (as is the case for deep quad trap data).

The slopes and $y$-intercepts of $\mu_j(\sigma)$ and $\sigma_j(\sigma)$ are fixed during tritium data analysis, so the scatter tail shape depends only on $\sigma$, $\gamma_{\mathrm{H}_2}$ and scatter peak amplitudes. For each gas, the slopes and intercepts for $\mu_j$ and $\sigma_j$ are determined through the following procedure. For a given $j$, we fit Gaussians to 20 sets of scatter peaks broadened by 20 different resolution widths, ranging from $0.5 \sigma$ to $1.5 \sigma$. 
This procedure produces $\mu_j$ and $\sigma_j$ for a range of $\sigma$ values. We then observe and fit the linear dependence of $\mu_j$ and $\sigma_j$ on $\sigma$. 

Combining the $j=0$ and $j\geq1$ scatter peaks, the full tritium-specific model for $\mathcal{R}_{\mathrm{PSF}}$ includes a limited set of free parameters with propagated uncertainties. Those parameters are $\gamma_{\mathrm{H}_2}$, $\sigma$, and the two variables $p$ and $q$, which control the amplitudes $\mathcal{A}_j$ of all scatter peaks.

Each amplitude $\mathcal{A}_j$ multiplies the $j^{\rm th}$ broadened scatter peak in \autoref{eq:T2_ins_res_model} or \ref{eq:simplified_scattering}. The tritium and $^{\mathrm{83m}}$Kr models for $\mathcal{A}_j (p, q)$ are identical; see \autoref{sec:Kr_detector_response}. Tritium-specific $p$ and $q$ values are estimated by slightly shifting $p$ and $q$ from the fit to $^{\mathrm{83m}}$Kr pre-tritium data (see \autoref{sec:deep-trap-data-and-fits}), to account for a difference in the mean number of tracks per event ($N_{\mathrm{tracks}}^{\mathrm{true}}$) between the pre-tritium and tritium data sets.
The procedure for estimating $p$ and $q$ for tritium data is described in more detail in \autoref{sec:scatter_peak_errors}.


\subsection{Binned event rate model}
Per the full CRES spectrum model (\autoref{eq:FullModel1}), we convolve the above models for the beta spectrum and $\mathcal{R}_{\mathrm{PSF}}$, then multiply the result by $\epsilon_k$, the detection efficiency for frequency bin $k$ (see \autoref{sec:efficinecy_binning}). This produces $\mathcal{S}_k(E_\mathrm{kin})$, the signal probability density function within bin $k$. In addition, a false event probability density function $\mathcal{F}(E_\mathrm{kin})$ is introduced.  $\mathcal{F}(E_\mathrm{kin})$ is assumed to be flat in energy because the probability to measure RF noise (the only potentially significant expected background source; see \autoref{subsec:electron_data}) is uniform as a function of cyclotron frequency, and energy is approximately linearly related to frequency over a limited range. Combining signal and background, the expected tritium event rate within bin $k$ is  
\begin{equation}
  \Big(\frac{dN}{dE_\mathrm{kin}}\Big)_k(E_\mathrm{kin}) = r_s \mathcal{S}_k(E_\mathrm{kin})+r_f\mathcal{F}(E_\mathrm{kin}),
  \label{eq:fullT2model}
\end{equation}
\noindent where $r_s$ is the signal rate and $r_f$ is the false event rate. Binning is handled differently in Bayesian and frequentist analyses, as  discussed further in \autoref{sec:final-analysis}.  During analysis, for all bins, cyclotron frequencies are converted to energies according to \autoref{eq:energytofrequency}, using the magnetic field $B$ from the $^{\mathrm{83m}}$Kr fit procedure in \autoref{sec:deep-trap-data-and-fits}. The parameter $B$ is discussed further in \autoref{sec:Bfield_errors}.
