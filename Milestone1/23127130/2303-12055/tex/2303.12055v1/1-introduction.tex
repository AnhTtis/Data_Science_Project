\section{Introduction}


The neutrino comes in three flavors\textemdash electron, muon, and tau\textemdash associated with the charged leptons.  Super-Kamiokande \cite{Fukuda:1998mi} and the Sudbury Neutrino Observatory \cite{Ahmad:2001an} showed that these flavors mix and oscillate, explaining the anomalies in atmospheric and solar neutrino fluxes. Oscillation between the neutrino flavors requires that  neutrinos have mass, in contradiction with the Standard Model of particle physics wherein neutrinos are massless. It is now clear that neutrino mass eigenstates $m_{i=1,2,3}$ exist, of which at least two have non-zero mass.   The flavor eigenstates are linear combinations of $m_i$,  with amplitudes given by the elements of a $3\times 3$ unitary mixing matrix $U$, the PMNS matrix \cite{Pontecorvo:1957cp,Maki:1962mu}.


 Oscillation experiments can determine only the differences of the squares $m_i^2-m_j^2$, not the mass scale, but they constrain  the sum  of the three neutrino masses, $\Sigma m_i$, to be at least \SI{0.05}{\eV} \cite{PDG:2020} because masses are positive definite. It is known from solar neutrino oscillations~\cite{Ahmad:2001an} that $m_2>m_1$, but whether $m_3$ is the lightest mass eigenstate (``inverted ordering'') or the heaviest (``normal ordering'')  is presently unknown.  The neutrino constitutes the first and, so far, the only identified dark matter in the cosmos.  Limits on $\Sigma m_i$ have been obtained from observations of the cosmic microwave background and large-scale cosmic structure (see, for example, the Planck collaboration \cite{Planck:2018vyg} who reported a limit of \SI{<0.12}{\eV} within the framework of the $\Lambda$CDM model).

The most sensitive technique for a direct and model-independent neutrino mass determination is the tritium beta decay endpoint method~\cite{Formaggio:2021nfz}.
The signal for neutrino mass emerges as a phase-space modification of the spectral shape close to the beta endpoint. This signal is independent of whether neutrinos are Majorana or Dirac particles.  Beta spectroscopy measures the electron-weighted neutrino mass,
\begin{eqnarray}
    m_\beta &=&\sqrt{\sum_{i=1}^3 \left| U_{\rm{e}i} \right|^2 m_i^2}\, . \quad
\end{eqnarray}
Neutrino oscillation experiments impose an ultimate lower bound of $m_\beta$\SI{\geq 0.009}{\eV} ($m_\beta$\SI{\geq 0.05}{\eV}) for the case of normal (inverted) mass ordering~\cite{PDG:2020}. 

Recently, the KATRIN experiment has set a new upper limit of \SI{0.8}{\eV} at \SI{90}{\percent} confidence level~\cite{KATRIN:2021uub}.    KATRIN's design goal is a 0.2-\SI{}{\eV} mass sensitivity limit~\cite{KATRIN:2021dfa}, if the mass is not larger. KATRIN is expected to either exhaust the quasidegenerate range of neutrino masses that are large compared to either of the mass differences measured in oscillation experiments, or to measure the neutrino mass if it lies in that range.

The neutrino mass might turn out to be smaller than the state-of-the-art experiment KATRIN can discover.  While there may be ways to extend the reach of KATRIN somewhat~\cite{KATRINSnowmass:2022}, this type of experiment approaches a fundamental limit as a result of its sheer size and  use of molecular tritium (T$_2$).  With T$_2$, one relies on a theoretical prediction of the spectrum of excited states produced in beta decay, which broadens the spectral response.  The neutrino mass effect is spread over a range of slightly differing endpoints for the excitations~\cite{Doss:2006zz,Saenz:2000dul}, exacting both a statistical and a systematic price. A direct measurement of this spectrum of excited states is not possible, although specific tests can be performed \cite{TRIMS:2020nsv} and are in excellent agreement with theory at the percent level.  To advance significantly beyond KATRIN's design sensitivity\textemdash which is the goal of a next-generation tritium endpoint experiment\textemdash requires a different approach. An experiment that sets a neutrino mass limit of $m_{\beta}<0.04$\,eV reaches the full range of neutrino masses allowed under the assumption of an inverted ordering of neutrino mass eigenvalues and would either measure neutrino mass or exclude the inverted ordering~\cite{Esfahani:2017dmu, AshtariEsfahani:2021moh}.
 
The Project 8 Collaboration has devised a new method called Cyclotron Radiation Emission Spectroscopy (CRES) \cite{Monreal:2009za,Esfahani:2017dmu} to reach this goal.
In  CRES, the emitted beta electron's energy is measured by detecting its cyclotron radiation as it spirals in a magnetic field. In special relativity, the cyclotron frequency $f_c$ in a magnetic field $B$ is related to kinetic energy $E_{\rm{kin}}$ as follows:
\begin{equation}
f_c = \frac{1}{2\pi}\frac{eB}{m_{\rm e}+E_{\rm{kin}}/c^2},
\label{eq:energytofrequency}
\end{equation}
where $e$ is the magnitude of the electron charge, $m_{\rm e}$ is the mass of the electron, and $c$ is the speed of light in vacuum. Project~8 aims to combine CRES with the use of atomic tritium as the beta decay source, to remove uncertainty from molecular states.

CRES was first demonstrated with a precision measurement of the energies of single electrons from $^{83\mathrm{m}}$Kr decay by the Project 8 collaboration in 2014 \cite{Asner:2014cwa} (Phase~I). Subsequently, Project 8 has performed a CRES-based tritium experiment (Phase II), which is the subject of this paper. We report details of the tritium endpoint and neutrino mass results from the small-scale Phase II apparatus. These are the first measurements of the continuous tritium spectrum using CRES and the first quantitative exploration of the systematic effects in CRES. A companion Letter~\cite{tritiumPRL:2022} provides a high-level overview of Project 8's Phase II results.

This paper is organized as follows. 
In \autoref{sec:apparatus}, the apparatus and data-taking conditions are described in sufficient detail to give context to the analysis. A more comprehensive paper on the apparatus is in preparation. Section~\ref{sec:datafeatures} describes the general features of CRES data and the data sets to be analyzed, \autoref{sec:analysisoverview} provides an overview of the analysis approach, including a general CRES signal model.  Section~\ref{sec:simCRES} and \autoref{sec:Kr_model} describe simulations and calibration measurements with $^{83 \rm m}$Kr, both of which provide inputs to tritium data analysis. $^{83 \rm m}$Kr data are also used to validate the frequency-energy relation in \autoref{eq:energytofrequency} and to demonstrate the high-resolution capabilities of CRES.
The tritium model for data analysis is developed in \autoref{sec:tritium_model}. This model is an analytic version of the general CRES signal model.
Parameter inputs and systematic uncertainties for the tritium analysis are described in \autoref{sec:systematic_uncertainties}. 
Section~\ref{sec:final-analysis} describes procedures for analyzing tritium data using both Bayesian and frequentist approaches. In \autoref{sec:conclusions}, we present endpoint and neutrino mass results. In addition, a search for events above the endpoint produces a stringent limit on the background rate. The sensitivity of this Phase II experiment to the neutrino mass is compared to an analytic prediction. 


\section{Apparatus \label{sec:apparatus}}

In the Project 8 Phase II apparatus,  molecular tritium or $^{83\mathrm{m}}$Kr is confined in a cryogenic gas cell (the ``CRES cell'') within the  field of a commercial warm-bore  superconducting magnet.  The cell, mechanically supported on an experimental insert, is positioned in the vertical  magnetic field of \SI{0.959}{\tesla}, which induces cyclotron motion and confines electrons radially.  The CRES cell is shown in \autoref{fig:apparatus}.  
\begin{figure}[tb]
  \centering
  \includegraphics[width=\columnwidth]{Plots/1-introduction/apparatus.pdf}
  \caption{Cutaway of the cryogenic CRES cell, where electrons are produced in radioactive decay and magnetically trapped. Cyclotron radiation travels axially up the waveguide (left in rotated view), toward the amplifiers and readout electronics.}
  \label{fig:apparatus}
\end{figure}
Electrons emitted in radioactive decay are trapped axially in dips in the magnetic field created by coils wound around the gas cell. The trap geometry affects resolution, event rate, and event characteristics. For Phase II, electrons are trapped in multiple short traps where the electrons' axial motion is near-harmonic, rather than in a single longer trap---both because of non-uniformity in the background field, and because of a Doppler shift, as explained below.  The currents can be varied independently in the five coils to create traps of varying geometries.  Cyclotron radiation emitted by the electrons then propagates through a waveguide, is amplified in two low-noise cryogenic amplifiers in series, and passes on to the room-temperature elements of the detection chain.

The 1.6-m insert which supports the CRES cell is cooled by a Cryomech AL-60 Gifford-McMahon cryocooler. 
The cell temperature is controlled by a heater on a PID loop.  The dominant noise source in the experiment is thermal, set by the temperature of a custom-made radio-frequency (RF)-absorbing terminator at the lower end of the cell.
To reduce noise, the cell is kept as cold as possible.  A lower limit on the cell temperature of \SI{85}{\kelvin} is set by the requirement that a sufficient density of $^{83\mathrm{m}}$Kr remains in the gas phase.

The Phase-II apparatus was previously described in~\cite{Esfahani:2017dmu}. Further details on the apparatus and tests thereof will be reported in a paper in preparation~\cite{Hardwarepaper:2022}.

\subsection{Electron trap}\label{sec:electron_trap}\label{fss_procedure}




The criterion for trapping a charged particle in a magnetic trap is 
\begin{eqnarray}
    \theta_0 &\geq&  \sin^{-1}\left( \sqrt{ \frac{ B_0}{B_{\rm max}}}\right),
    \label{eq:pitchanglerange}
\end{eqnarray}
where $\theta_0$ is the pitch angle at the lowest-field point, where the field is $B_0$, and $B_{\rm max}$ is the smaller of the maximum field values on either side of $B_0$.  Pitch angle is defined as the angle between the electron's momentum and the local field direction.

The cyclotron radiation produced by electrons propagates along the waveguide axis toward the detection system. 
Because the trapped electrons undergo axial motion, their cyclotron radiation is Doppler-shifted at twice the axial frequency. The maximum Doppler shift is proportional to the electron's axial velocity as it passes through the trap minimum.  The resulting frequency modulation creates sidebands to the mean cyclotron frequency (termed the `carrier').  The modulation index $h$ is defined as the ratio of the peak shift in carrier frequency to the modulation  (here, axial) frequency. For large $h$, sidebands proliferate and the carrier becomes too weak to be detected~\cite{Esfahani:2019mpr}.   Our noise threshold for detecting the carrier corresponds approximately to $h\le1$. In magnetic traps this condition translates directly to a pitch-angle limitation, and, equivalently, a limitation on the axial amplitude, which dictates short traps. To increase statistics, several short traps are spaced along the cell within the most uniform region of the background field.  

Data were taken in two trap configurations. A deep `quad' trap with the magnetic field shown in \autoref{fig:quadtrapcoils}(a) 
\begin{figure}[htb]
    \centering
    \includegraphics[width=0.48\textwidth]{Plots/1-introduction/quadnew_180_252_275_288_20230207.pdf}
    \includegraphics[width=0.48\textwidth]{Plots/1-introduction/shallownew_12_17p85_20230207.pdf}
    \caption{(a) Calculated quad trap coil field. (b) Calculated shallow trap coil field. Note different scales on $y$-axes. The background field from NMR measurements is shown as a blue dotted line. The solid blue lines are for a radius \SI{1}{\mm} from the axis, and the red lines are for 2, 3, and \SI{4}{\mm} in turn. The coil currents  are shown in \SI{}{\mA} under each trap. Trap depths were equalized for the 1-mm radius because the coupling to the waveguide's TE$_{11}$ mode maximizes on the axis.}
    \label{fig:quadtrapcoils}
\end{figure}
was used to maximize the number of trapped electrons and to increase effective volume in this small apparatus. Four of the trap coils shown in \autoref{fig:apparatus} were used; the fifth was not used because the background field varies too steeply there to form a trap.  The 0.959-\SI{}{\tesla} background field is also shown in \autoref{fig:quadtrapcoils}. 
It deviates from homogeneity at the \SI{5 e-4}{} level because of a non-functional trim coil of the superconducting magnet, which causes the slope and curvature seen in the figure.    A shallow double trap was used to demonstrate the high resolution capability of the CRES technique; this trap is shown in \autoref{fig:quadtrapcoils}(b). Parameters for the  traps are listed in \autoref{tab:quadtrapcoils}. 

Table~\ref{tab:quadtrapcoils} also gives the volumes and minimum detectable pitch angles $\theta_0(h=1)$ at the trap centers. The effective volume of each trap $V_{\rm eff}$   is calculated numerically with the aid of \autoref{eq:pitchanglerange} for a maximum axial amplitude equivalent to a modulation index $h=1$ and a radius of \SI{5.02}{\mm}, the waveguide radius.  Other sources of inefficiency, such as track and event reconstruction, are not included in $V_{\rm eff}$.  Also shown is the minimum pitch angle $\theta_0$(min) in each trap.
\begin{table}[tb]
    \centering
    \caption{Calculated quad and shallow trap configurations.}
    \begin{tabular}{lrrrrl}
    \hline\hline
         & Coil 1 & Coil 2 & Coil 3  & Coil 4  & Unit \\
    \hline
    Turns & 64 & 63 & 64 & 65  \\
    \hline
    \multicolumn{6}{l}{\bf Quad trap coil parameters.}\\
    Current & 180 & 252 & 275 & 288  & mA \\
    $z_{\rm min}$  & 11.9 & 34.4  & 56.7  &  78.6 & mm  \\
    $B_0-959$\phantom{aa}  & -1.21173 & -1.21768  & -1.22141  &  -1.21918 & mT \\
    Trap depth  & 0.45423 & 0.57720 & 0.70887 & 0.72418 & mT \\
    $\theta_0(h=1)$  & 89.47 & 89.37 & 89.33 &  89.30 & deg \\
    $\theta_0$(min)  & 88.75 & 88.59 & 88.44 &  88.42  & deg \\
    $V_{\rm eff}$  & 3.0 & 3.6 & 3.9 &  4.1 & mm$^3$ \\
    \hline
    \multicolumn{6}{l}{\bf Shallow trap coil parameters.} \\
    Current & & & 12 & 17.85 & mA \\
    $z_{\rm min}$ & & & 55.6  &  78.4 & mm \\
    $B_0 - 959$ & &  & 0.084724  &  0.087937 & mT \\
    Trap depth &  &   & 0.009905  &  0.036838 & mT \\
    $\theta_0(h=1) $ &  &  & 89.89 & 89.84 & deg \\
    $\theta_0$(min)  &  &  &  89.82 & 89.64  & deg \\
    $V_{\rm eff}$ &  &  & 0.61 &  0.96 & mm$^3$\\
    \hline
    \hline
    \end{tabular}
    \label{tab:quadtrapcoils}
\end{table}
In the table, $z_{\rm min}$ is the location of the trap minimum relative to the lower CRES cell window.  Estimates of $B_0$ are also included in the table; calibrations with $^{83\mathrm{m}}$Kr produced more precise field estimates, as discussed in \autoref{sec:deep-trap-data-and-fits} and \autoref{sec:Bfield_errors}. 

The total $V_{\rm eff}$ for the quad trap is about $10$ times that of the shallow trap, but experimental measurements with $^{83 \rm m}$Kr give a value closer to 40 for this ratio.  This discrepancy is seen in part because the deep quad trap holds more `invisible' electrons with relatively small pitch angles between $\theta_0(h=1)$ and $\theta_0$(min), which can scatter into the detectable pitch-angle range.  In the quad trap, the first observed electron track is unscattered in approximately 50\% of events (Sec.~\ref{sec:deep_trap}), while in the shallow trap, the fraction is approximately 70\% (Sec.~\ref{sec:shallow_trap} and \autoref{fig:spectrogram}).  There are also small corrections for the difference in radiated electron power in the shallow and quad traps, because the frequencies are not exactly the same and only two trap coils are in common.  The coupling of an electron to the waveguide depends on the trap.  The main source of the discrepancy, however, is attributable to the numerical calculation of $V_{\rm eff}$ for very weak traps.  There are inhomogeneities in the background magnetic field which are not well characterized and not included in these calculations.  It is likely that these features become important when the trap is weak.  Evidence for this emerged when still shallower traps were explored and the coil-3 trap disappeared below 10 mA.  This last effect is not significant for the deep quad trap. 

A long additional copper coil called the ``field-shifting solenoid'' (FSS) was inserted into the superconducting solenoid's warm bore that encompasses the CRES cell (\autoref{fig:apparatus}). The FSS was used to homogeneously shift the magnetic field for studies of detection efficiency as a function of frequency, which are described in \autoref{subsec:efficiency}.
By running current through this additional coil the field was shifted in steps of $\Delta B_{\rm FSS}=0.07$\,mT over a range of \SI{\pm 3}{mT}. 

A single-axis fluxgate magnetometer (Schonstedt Instrument Co.~DM2220) was used during the final $^{\mathrm{83m}}$Kr calibration phase to assess the role of environmental magnetic field changes in the laboratory.  Over a 3-day period, peak-to-peak variations of \SI{6}{\percent} in the 0.11-\SI{}{\milli\tesla} vertical laboratory background field were observed at a distance of \SI{3}{m} from the magnet. These variations correspond to a \SI{0.3}{\eV} difference in reconstructed electron energies. The variation within the magnet bore is expected to have been even lower due to the self-shielding provided by a superconducting magnet in persistent mode.  While the fluxgate magnetometer was not available during tritium running, the influence of environmental field variations is considered negligible at the relevant sensitivity level.

\subsection{Gas system}\label{sec:gas_system}

The radioactive gases were released into the apparatus from a custom gas manifold. This manifold connected to the CRES cell via a delivery line running along the insert and through a grid of  holes with diameter less than 0.1 wavelengths of the microwave radiation. The gas system could be run in two modes.  In neutrino mass data acquisition mode, molecular tritium gas was delivered, while in calibration mode, $^{83\mathrm{m}}$Kr gas was delivered. Tritium was stored in a 0.5-\SI{}{\g} non-evaporable getter (SAES ST 172/HI/7.5-7/150 C), the temperature of which was controlled by an ion gauge to maintain the desired operating pressure, usually \SIrange[range-phrase = --,range-units = single]{1.6}{2.6e-6}{\milli\bar} (calibrated for H$_2$). Pressure could be maintained to \SI{\pm3}{\percent}  run-to-run for the duration of data-taking.  In later data sets, we prevented accumulation of $^3$He (produced in tritium decay) by extracting gas continually through a leak valve to a getter-ion pump, which also lowered the concentration of traces of methane, Ar, CO and CO$_2$ impurities; this is referred to as the ``pumped'' configuration. The initial \SIadj{2}{Ci} inventory of tritium was sufficient for a \SIadj{\sim 100}{\day} data-taking period.  Gas composition could be monitored by two residual gas analyzers, an SRS-100 close to the SAES getter, and an Extorr XT100 at the getter-ion pump manifold.  

For calibration studies, the $^{83\mathrm{m}}$Kr gas emanated from $^{83}$Rb (\SI{6}{mCi} on July 19, 2019), adsorbed on zeolite  \cite{Venos:2005vn}. The Kr was mixed with H$_2$ to tune the mean time between electron-gas collisions to match that in tritium data. The H$_2$ was stored in a separate getter and pressure-controlled in the same way as the tritium.


\subsection{Radio-frequency system}\label{subsec:rf_system}

The CRES cell forms the first section of the waveguide through which cyclotron radiation travels toward the amplifiers (Low Noise Factory LNF-LNC22\_40WA). For the Phase II decay cell used in this work, a circular waveguide was chosen over the rectangular WR-42 used previously~\cite{Asner:2014cwa}, to increase the volume and to accept circular polarization. In the frequency range \SIrange[range-units = single]{25.8}{27.0}{\giga\hertz}, waveguide of radius \SI{5.02}{\milli\meter} supports two propagating modes, TE$_{11}$ and TM$_{01}$. The electron couples well to the TE$_{11}$ mode and only weakly to the TM$_{01}$ mode (\autoref{fig:modecoupling}). 
\begin{figure}[htb]
    \centering
    \includegraphics[width=\columnwidth]{Plots/1-introduction/Mode_coupling.pdf}
    \includegraphics[width=\columnwidth]{Plots/1-introduction/mode_impedance.pdf}
    \caption{(a) Coupling of the electron to the two propagating modes as a function of the distance $\rho$ from the axis. The electron has energy 18.6\,keV ($f_c \approx 25.9\,\si{GHz}$) and  pitch angle $90^\circ$.  (b) Mode impedance for the two propagating modes.}
    \label{fig:modecoupling}
\end{figure}
The coupling is maximal on-axis and falls off to a small value near the wall.  

The cyclotron radiation emitted by a trapped electron propagates both downward and upward, passing through RF-transparent CaF$_2$ windows (United Crystals Inc.) that confine the radioactive gas (Fig.~\ref{fig:apparatus}). This crystalline material has a coefficient of thermal expansion that matches copper at low temperatures, and is known to have low permeability to tritium~\cite{osti_4351657}. To minimize interference from reflections, the downward-propagating radiation is absorbed in a custom-made cryogenic RF terminator.  Beyond the uppermost window, the upward-propagating circularly polarized radiation is converted to linear polarization by a quarter-wave `plate.' The radiation is then
transmitted via a WR-42 single-mode rectangular waveguide, including a gold-coated stainless steel section for thermal insulation, to cryogenic amplifiers held at \SI{30}{\kelvin}. 
Residual reflections from the windows, joints, and transitions 
create weak resonant cavity modes within the gas cell, which enhance spontaneous emission at particular frequencies and locations within the trap.  These resonances significantly modify the response to signals from electrons as a function of trap position and frequency, presenting a difficult analysis challenge. For example, the quarter-wave plate's reflection of the TM$_{01}$ mode creates a sharp cavity-like absorption resonance that, despite the weak coupling to the electron, can clearly be seen in the efficiency function at \SI{25.925}{\giga\hertz}, as described in \autoref{sec:slope_correction}.

After cryogenic and room-temperature amplification, room-temperature RF electronics downmix the signal by \SI{24.5}{\giga\hertz}. The downshifted signal is digitized by a ROACH2 DAQ system~\cite{Hickish2016} at \SI{3.2} gigasamples per second, with an FPGA performing digital downconversion to \SI{200} megasamples per second and Fast Fourier Transforms (FFT)  for three separate frequency windows with independently-set center frequencies.
When two 40.96-\SI{}{\micro\second} bins within any 0.5-\SI{}{\milli\second} window exceed a signal-to-noise ratio (SNR) threshold, a compute node writes time-series data to disk. SNR is used instead of absolute power in all stages of data analysis (triggering, event reconstruction, spectrum analysis) to avoid the effects of gain variation of the amplifiers and filters.  

\begin{figure}[htb]
\centering
\includegraphics[width=0.45\textwidth]{Plots/1-introduction/T2_Event0_v3.pdf}
\caption{Spectrogram of chirped electron signals of a single radiating electron. The bin size is 25 kHz by 40.96 $\mu$s.}
\label{fig:spectrogram}
\end{figure}

\begin{table*}[tbp]
\centering
\footnotesize
\caption{Characteristics of key data sets. $N_{\mathrm{events}}$ is the number of reconstructed events in a given data set. $N_{\mathrm{tracks}}^{\mathrm{true}}$ is the estimated mean number of electron tracks per event (including unreconstructed tracks). $\tau$ is the mean time between electron-gas collisions.}
\begingroup
\setlength{\tabcolsep}{6pt} % Default value: 6pt
%\renewcommand{\arraystretch}{1.1} % Default value: 1
\begin{tabular}{llllrrrr}
\hline\hline
Data Set & Purpose & \begin{tabular}[c]{@{}l@{}}Magnetic field\\ configuration\end{tabular} & \begin{tabular}[c]{@{}l@{}}Gas system\\ configuration\end{tabular} & $N_{\mathrm{events}}$ & $N_{\mathrm{tracks}}^{\mathrm{true}}$ & $\tau$ (ms) & Section(s) \\ \hline 
%\addlinespace
\begin{tabular}[c]{@{}l
@{}}$^{83\mathrm{m}}$Kr\\ shallow\end{tabular} & \begin{tabular}[c]{@{}l@{}}\\Demonstrate best \\ resolution and probe\\ detector response\end{tabular} & \begin{tabular}[c]{@{}l@{}}Shallow \\ double trap\end{tabular} & Not pumped & 6831 & $1.24 \pm 0.12$ & $0.342 \pm 0.005$ & \begin{tabular}[c]{@{}l@{}} \autoref{sec:frequency-energy_relation}, \\ \autoref{sec:shallow_trap}
\end{tabular}\\
%\addlinespace
\begin{tabular}[c]{@{}l@{}}$^{83\mathrm{m}}$Kr \\ field-shifted\end{tabular} & \begin{tabular}[c]{@{}l@{}}\\Measure frequency \\ dependence of \\ efficiency and \\ other characteristics\end{tabular} & \begin{tabular}[c]{@{}l@{}}Deep quad trap \&\\ deep single traps;\\ background B \\ field varied\end{tabular} & Not pumped & 9350 & $2.73 \pm 0.24$ & $0.173 \pm 0.003$ & \autoref{subsec:frequency_dependence} \\ %\addlinespace
\begin{tabular}[c]{@{}l@{}}$^{83\mathrm{m}}$Kr \\ pre-tritium\end{tabular} & \begin{tabular}[c]{@{}l@{}}\\Calibrate magnetic \\ field, scattering \\ environment\end{tabular} & Deep quad trap & Not pumped & 87634 & $2.40 \pm 0.27$ & $0.162 \pm 0.001$ & \autoref{sec:deep_trap} \\ 
%\addlinespace
Tritium & \begin{tabular}[c]{@{}l@{}}\\Spectroscopy of \\ tritium endpoint\end{tabular} & Deep quad trap & Pumped & 3770 & $2.36 \pm 0.34$ & $0.146 \pm 0.003$ &    
\begin{tabular}[c]{@{}l@{}}\autoref{sec:tritium_model}, \\ \autoref{sec:final-analysis} \\
\end{tabular}\\
%\addlinespace
\begin{tabular}[c]{@{}l@{}}$^{83\mathrm{m}}$Kr\\ post-tritium\end{tabular} & \begin{tabular}[c]{@{}l@{}}\\Calibrate magnetic \\ field, scattering \\ environment\end{tabular} & Deep quad trap & Pumped & 47426 & $3.37 \pm 0.21$ & $0.188 \pm 0.001$ &  \autoref{sec:deep_trap} \\ 
\hline
\hline
\end{tabular}
\endgroup
\label{tab:mml_track_length_fit_results}
\label{tab:data_set_table}
\end{table*}

\section{CRES data features \label{sec:datafeatures}}

\subsection{Characteristics of electron events}
\label{subsec:electron_data}

Electron events may be displayed in a spectrogram (or `waterfall plot') of frequency vs.~time, with pixels indicating signal power by color or intensity. Figure~\ref{fig:spectrogram} shows a typical event composed of ``tracks,'' with the jumps in frequency from track to track caused by collisions with gas molecules.  Events are continuous in time but discontinuous in frequency.   
The frequency changes after a collision are due to changes in both energy and pitch angle. Pitch-angle changes cause the amplitude of the electron's axial motion to increase or decrease, so the average magnetic field experienced by the electron may increase or decrease.  Between collisions, signals continuously chirp upward in frequency as the electrons radiate energy \cite{Esfahani:2019mpr}.

We use a point-clustering algorithm to identify high-SNR bins occurring close together in time and frequency as belonging to the same electron track~\cite{furse2015techniques, TERpaper:2022}.  A reconstruction algorithm extracts the starting frequency of the first track of each event, identifying it as the ``start frequency'' of the event.

One of CRES's promising features is its immunity to  background. Few processes can masquerade as valid signals, which have  specific  signatures in frequency, duration, power,  time dependence, and event structure.   The interactions of cosmic rays and energetic beta and gamma backgrounds with the gas can in principle lead to the production and trapping of an electron in the right energy range, but this is a rare occurence because of the low density of the gas. 
The dominant background is expected to be from RF noise fluctuations. 
In this analysis, we distinguish electron events from RF noise by a cut on a function of three characteristics: the number of tracks in a candidate event, the duration of the first track, and the SNR of the first track. These characteristics are distributed differently for electron events and RF noise fluctuations.
We set the threshold for this cut before tritium data acquisition at a level expected to allow less than one RF-noise-induced background event beyond the endpoint per 100 days of run time per RF channel at \SI{90}{\percent} confidence level.

Even with the successful elimination of false tracks, the first track {\em visible} in an event is not necessarily the first track following beta decay or internal conversion.  Tracks can be too short or too low in power to be detected. The response function that has been developed takes account of both of these possibilities, and includes also the effects of cyclotron radiation losses and the broadening associated with pitch-angle variations. 

\subsection{Data set features}

We took data using two kinds of radioactive source, $^{83\mathrm{m}}$Kr and tritium.  Studies of the efficiency, energy response, and magnetic field were carried out with $^{83\mathrm{m}}$Kr during the second half of 2019.  Tritium data taking began in mid-December and extended into March 2020, with several weeks of downtime in February for laboratory maintenance. A final week of $^{\mathrm{83m}}$Kr measurements concluded the data-taking campaign in March as planned, just before a global pandemic precluded further data-taking and laboratory work. 

\begin{table*}[htbp]
  \centering
  \renewcommand{\arraystretch}{1.15}
    \begingroup
    \setlength{\tabcolsep}{6pt} % Default value: 6pt
  \begin{tabular}{ l  c  c  c  c c}
              \hline \hline
Data set        & Tritium        & $^{83\mathrm{m}}$Kr shallow           & $^{83\mathrm{m}}$Kr field-shifted          & $^{83\mathrm{m}}$Kr pre-tritium  & $^{83\mathrm{m}}$Kr post-tritium    \\ \hline \hline
Hydrogen isotopes & 91$\pm$5   & 9-98    & 38-98      & 23-91    & 41-99 \\ 
Helium-3          & 8$\pm$4    & 0-99    & 0-59       & 0-67     & 0-58 \\ 
Argon             &         & ${<}$1  & ${<}$1     & 5-10     & ${<}$1 \\ 
Krypton           &         & 1-3     & 2-3        & 2-5      & ${<}$1 \\ 
Carbon monoxide   & 1$\pm$1    &         &            &          & ${<}$1 \\ \hline \hline
\end{tabular}
\caption{Possible ranges of percentages of inelastic scatters due to each of the main gas species present during the tritium data set and the primary $^{\mathrm{83m}}$Kr data sets. Scattering from all non-listed gases for a given data set is negligible. Values are given as ranges with near-uniform probability (with sigmoid edges) for $^{83\mathrm{m}}$Kr data, where deuterium contamination interfered with distinguishing $^3$He from hydrogen isotopes.}
\label{tab:scattering_fraction_results}
\endgroup
\end{table*}

Properties of key data sets are summarized in \autoref{tab:mml_track_length_fit_results} in chronological order. Details of electron trap configurations are discussed in \autoref{sec:electron_trap} and of gas system configurations in \autoref{sec:gas_system}. The mean number of electron tracks per event, $N_{\mathrm{tracks}}^{\mathrm{true}}$, is determined from the post-reconstruction tracks per event in data combined with simulation studies of the relationship between true and reconstructed tracks per event. Section~\ref{sec:scatter_peak_errors} describes how $N_{\mathrm{tracks}}^{\mathrm{true}}$ is used in the tritium data analysis. Section~\ref{subsec:track_length} describes how $\tau$, the mean time between electron-gas collisions, is extracted from the data and used in the tritium analysis. For all data sets that provide direct calibration input to the tritium analysis\textemdash $^{83\mathrm{m}}$Kr field-shifted, $^{83\mathrm{m}}$Kr pre-tritium, tritium, and $^{83\mathrm{m}}$Kr post-tritium\textemdash the gas composition and scattering rate of electrons from gas molecules were kept as similar and stable as possible, despite the absence of krypton and the presence of helium in the gas system during tritium data-taking.

$^{83\mathrm{m}}$Kr data  were acquired in periods lasting a few hours each for deep quad trap data sets. Shallow-trap data sets took several days to acquire adequate statistics. With maximum event durations of \SI{<10}{\milli\second} and rates of \SI{\sim 1}{cps} (counts per second), pileup effects were negligible. A single \SIadj{100}{\mega\hertz}-bandwidth DAQ channel was used to acquire data on the \SIadj{17.8}{\kilo\eV} K internal-conversion line, and sometimes the second channel or both the second and third channels simultaneously took data on the L lines or the M and N lines.

We took tritium data over 82 days in the deep quad trap configuration, with a mean event rate of \SI{0.5 e-3}{cps}.
The analysis window spanned \SIrange[range-phrase = --,range-units = single]{25.81}{25.99}{\giga\hertz}, or \SIrange[range-phrase = --,range-units = single]{16.2}{19.8}{\kilo\eV} in electron energy.
The three DAQ windows were overlapped to minimize efficiency variation with frequency due to windowing, with only the highest-efficiency channel analyzed in the overlap regions \cite{daq_paper}.

\subsection{Scattering}
\label{sec:gas_composition}
\label{gamma_i_determination}

To account for the effects of missed tracks described in Sec.~\ref{subsec:electron_data}, it is necessary to understand the spectrum of an electron's possible energy losses between the start times of successive tracks. The dominant contributor to energy losses between successive tracks is the electron's inelastic scattering with gas molecules; this is affected by which gas species are present in the gas cell. Here we describe our measurements and analysis of the gas composition; in Sec.~\ref{L_model}, we describe where this information comes into the broader analysis. 

Inelastic scatters produce pitch angle changes of ${\sim}0.1 ^\circ$ \cite{DavidJoy:ElectronScattering}---generally small enough that the electron remains trapped after scattering.
By contrast, elastic scattering tends to produce ${\geq} 5^\circ$ changes \cite{DavidJoy:ElectronScattering}, ejecting the electron from the trap and terminating the event. In addition, elastic scattering cross sections are an order of magnitude smaller than inelastic cross sections for the most relevant gas species in our data.  
We therefore neglect elastic scattering as an energy-loss mechanism between tracks.

One input to the inelastic loss spectrum must be measured in-situ: the relative likelihood $\gamma_i$ of an inelastic collision with gas species $i$.   
The $\gamma_i$ factors for key data sets are derived from quadrupole mass analyzer data as follows, with
\begin{eqnarray}
    \gamma_i &=& \frac{\sigma_{i,E}\, C_{\mathrm{temp},i}\, p_{\mathrm{raw,}i}/s_i}{\sum_n(\sigma_{n,E}\, C_{\mathrm{temp,}n}\, p_{\mathrm{raw,}n}/s_n)},
    \label{eq:gammai}
\end{eqnarray}
where $\sigma_{i,E}$ is the total inelastic scattering cross section of an electron of energy $E$ with gas $i$; $C_{\mathrm{temp},i}$ is a factor accounting for gas freezing to CRES cell walls; $p_{\mathrm{raw,}i}$ is the uncorrected partial-pressure reading of the quadrupole mass analyzer; and $s_i$ is the sensitivity factor of the quadrupole mass analyzer to gas species $i$. The uncertainty on each of these quantities is propagated through to the uncertainty on $\gamma_i$ in the standard way. 

The inelastic scattering cross sections $\sigma_{i,E}$ are derived from literature values; contributions to uncertainties on these account for both uncertainties within and differences between published data sets. Values of $\sigma_{i,E}$ for H$_2$ and T$_2$ come from a measurement at 18.6 keV \cite{Aseev2000} and are scaled according to \cite{Liu:1987ka} for 17.8\,keV. Cross sections on $^3$He, Kr, and Ar are taken from \cite{Cullen:1989ig}, with uncertainties from \cite{nagy1980absolute, Brusa:1996hw, zecca2000electron, Cartwright:1992he}. For CO, the cross section is evaluated using the expression from \cite{Hwang:1996hf}, with uncertainties from \cite{Rapp:1965kf, Kanik:1993ge}. 

Because Kr is the only relevant gas species for which significant adsorption to cold walls is expected at 85 K, we take $C_{\mathrm{temp},i}$=1 for all other species. The C$_{\mathrm{temp},\mathrm{Kr}}$ value of 0.90$\pm$0.05 is measured from temperature-varying $^{\mathrm{83m}}$Kr CRES data, taking advantage of the direct dependence of event rate on Kr density in the cell.

We adopt sensitivity factors $s_i$ from quadrupole mass analyzer manuals, with uncertainties estimated from differences between different manufacturers'  values.

We measured the raw partial pressures $p_{\mathrm{raw,}i}$ using the quadrupole mass analyzer. Since $^{\mathrm{83m}}$Kr data sets were completed in hours or at most a few days, gas conditions were stable. Therefore, each data set's $p_{\mathrm{raw,}i}$ measurements are determined from a single representative quadrupole mass analyzer scan. In contrast, the tritium data were taken over months, with some variation in conditions, especially initially as pumping speed settings were optimized. Gas composition for the tritium data set is therefore an average, the sum of measurements taken in each of the states weighted by the accumulated counts in that state.

One additional complication in interpreting $p_{\mathrm{raw,}i}$ values comes from the inability to distinguish species with identical charge-to-mass ratio with the available quadrupole mass analyzers. This creates a challenge because deuterium gas was used in the initial testing during commissioning of the gas system, and was mistakenly allowed to contaminate the reservoir of H$_2$ used for $^{\mathrm{83m}}$Kr data. For $^{\mathrm{83m}}$Kr data sets, therefore, $^3$He and HD cannot be distinguished, and this uncertainty is reflected in the larger error bars on gas composition in these data sets. In contrast, the tritium gas supply was not deuterium-contaminated, so the tritium data do not suffer from this uncertainty.

Table~\ref{tab:scattering_fraction_results} shows the inelastic scattering fraction results, as derived from the quadrupole mass analyzer data and \autoref{eq:gammai}. 
For fits to the $^{\mathrm{83m}}$Kr pre-tritium and post-tritium data sets, to propagate uncertainties, we sample $\gamma_i$ from near-uniform distributions defined according to the ranges in this table, requiring that the sampled fractions sum to 1. This near-uniform shape, flat with sigmoid ends, reflects the inability to distinguish $^3$He from HD in $^{\mathrm{83m}}$Kr data. Section~\ref{sec:Kr-quad-trap-uncertainties} more fully describes uncertainty propagation in $^{\mathrm{83m}}$Kr quad trap fits.

In tritium and $^{\mathrm{83m}}$Kr data, scattering from molecular hydrogen isotopes is dominant. In both data types, $^3$He produced by decay of residual tritium adsorbed to the walls of the gas system is the next-largest contributor to scattering. Tritium data include a small contribution from a mass-28 gas species, likely CO. In addition to the krypton present in $^{\mathrm{83m}}$Kr data sets, argon is also present in the data sets taken before the pumped gas system configuration (\autoref{sec:gas_system}) enabled lower impurity levels.


\subsection{Mean track duration \texorpdfstring{$\tau$}{}}
\label{subsec:track_length}

Track duration is the time between successive scatters of an electron with gas molecules.
In the track reconstruction process, track duration is a powerful tool for discriminating between noise and signal. The offline event reconstruction \cite{TERpaper:2022} removes events of which the first track's signal-to-noise ratio is below certain thresholds. These thresholds are specific to the combination of first track duration and number of tracks in the event. Hence, the gas pressure in the CRES cell impacts the overall event detection efficiency as well as the first track detection efficiency, thereby affecting the electron energy point-spread function $\mathcal{R}_{\mathrm{PSF}}$ (see \autoref{sec:Kr_detector_response}). Track duration also impacts the reconstructed mean number of tracks per event, which in turn affects the shape of the tail from missed tracks in $\mathcal{R}_{\mathrm{PSF}}$, as discussed in \autoref{sec:scatter_peak_errors}. It was thus necessary to assess the track duration distribution for each data set. 

Since track duration is determined by random scattering, which is a Poisson process, the underlying probability density function (PDF) is exponential. However, short tracks are less efficiently detected, so the number $N$ of tracks detected with a certain duration $t$ is given by
\begin{eqnarray}\label{eqn:track_length_distribution}
 \ N(t) &\propto& P_d(t)e^{-t/\tau},
\end{eqnarray}
where $\tau$ is mean time between collisions with gas molecules and $P_d(t)$ is the relative probability of detection as a function of $\tau$. 

\begin{figure}[tb]
  \centering
  \includegraphics[width=1.0\columnwidth]{Plots/1-introduction/Tritium_run_3_track_length_fit.pdf}
  \caption{    Tritium data track duration fit using \autoref{eqn:track_length_distribution} and \autoref{eqn:detection_probability}. The track reconstruction algorithm \cite{TERpaper:2022} favors track lengths equal to a multiple of 3 times the spectrogram (\autoref{fig:spectrogram}) time bin width ($40.96\,\si{\mu s}$). This leads to larger residuals for some track lengths.}
  \label{fig:mll_track_length_fits}
\end{figure}

To fit the full track duration distribution and extract $\tau$, it is necessary to have a model for $P_d(t)$ that accounts for the roll-off in detection efficiency at short track durations. We used the following empirical model, which provides good fits to the data is

\begin{eqnarray}\label{eqn:detection_probability}
 \ P_d(t) &=& \mathrm{erfc}\left(\psi - \sqrt{\frac{\zeta}{t}}\right),
\end{eqnarray}
where $\mathrm{erfc}$ is the complementary error function and $\psi$ and $\zeta$ are determined solely by conditions held constant for all data sets (except the field-shifted data), such as SNR and the event-reconstruction algorithm's success rate at reconstructing short-duration tracks. 

To extract $\tau$ in the core tritium and $\mathrm{^{83m}}$Kr calibration data sets, we first determined $\psi$ and $\zeta$ independently by analyzing reconstructed first-track-duration distributions of $\mathrm{^{83m}}$Kr data sets at five different pressures, and therefore five values of mean track duration. We used the Stan software package \cite{Carpenter2017} to do a Markov chain Monte Carlo (MCMC) Bayesian analysis in which $\psi$ and $\zeta$ were shared for all data sets and $\tau$ was allowed to vary between data sets, with weakly informative priors. The best-fit values and uncertainties for $\psi$ and $\zeta$ were determined from the resulting posterior distributions. 

The resulting information about $\psi$ and $\zeta$ is used in the track-duration distribution fits, in which we extract $\tau$ for the core tritium and $\mathrm{^{83m}}$Kr calibration data sets using negative log-likelihood minimization.
The fit result for the tritium data set is shown in \autoref{fig:mll_track_length_fits} and the extracted mean track durations for all data sets are listed in \autoref{tab:mml_track_length_fit_results}. These $\tau$ values are inputted to the instrumental resolution simulations (\autoref{subsec:ins_res}). 