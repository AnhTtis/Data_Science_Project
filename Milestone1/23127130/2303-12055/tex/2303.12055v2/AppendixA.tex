\appendix

\begin{figure*}[tb]
\centering
\includegraphics[width=1\textwidth]{Plots/1-introduction/PRC_flow_chart_v2023-10-06_v1.pdf}
\caption{Flow chart of the analysis procedure. Thin arrows represent relationships between individual boxes, while thick arrows describe shared inputs from and/or outputs to a group of boxes in the same bubble. }
\label{fig:flowchart}
\end{figure*}


\section{The effect of track duration distribution}
\label{appendix:track_distribution_response_function_etc}
The dependence of efficiency and response function on the track duration distribution comes about as follows. In the track reconstruction process, track duration is a powerful tool for discriminating between noise and signal. The offline event reconstruction \cite{TERpaper:2022} removes events of which the first track's signal-to-noise ratio is below certain thresholds. These thresholds are specific to the combination of first track duration and number of tracks in the event. Hence, the gas pressure in the CRES cell impacts the overall event detection efficiency as well as the first track detection efficiency, thereby affecting the electron energy point-spread function $\mathcal{R}_{\mathrm{PSF}}$ (see \autoref{sec:Kr_detector_response}). Track duration also impacts the reconstructed mean number of tracks per event, which in turn affects the shape of the tail from missed tracks in $\mathcal{R}_{\mathrm{PSF}}$, as discussed in \autoref{sec:scatter_peak_errors}.


\section{Illustration of full analysis procedure}
\label{sec:flowchart}

\autoref{fig:flowchart} shows the analysis flow in greater detail than \autoref{fig:simple_flowchart}. It also displays interdependencies of $^{\mathrm{83m}}$Kr and tritium analyses.

% The analysis is complex, a consequence of the exploratory nature of the experiment. Some of the complexity comes from aspects of the experiment design and the analysis that were not anticipated; for example, the gas composition stability and measurement was more important than we had expected.
% Scattering of electrons with molecules from each gas species contributes to the total detector response. The gas composition was not stable between calibration data and corrections were needed for the main tritium analysis. The uncertainty of the measured gas composition propagates to the endpoint determination and the mass limit.
% In future designs with higher energy resolution, purer source gas, and better composition monitoring, scattering will be less significant and better controlled. As a result, the complication and uncertainty of the analysis is expected to be much reduced.  
% Another unexpected complication was the extent to which the efficiency as a function of frequency and energy was modulated by the presence of parasitic reflections in the waveguide, which demanded a very detailed analysis methodology. In the future, the energy region of interest will be much smaller than in Phase~II and the mode structure a key part of the experimental design. Consequently, large efficiency variations will be avoided, and the analysis further simplified. 


\section{Validating \texorpdfstring{$^{\mathrm{83m}}$Kr}{} scatter peak amplitudes with simulation}\label{appendix:scatter_peak_amplitude_simulation}

In the energy response function model, the amplitude $\mathcal{A}_j(p, q)$ of scatter peak $j$ is proportional to the probability of missing $j$ tracks before detecting an event. Simulations validate the model for $\mathcal{A}_j$ (\autoref{eq:scatter_peak_amplitude}) as well as the fitted values of $p$ and $q$ for quad trap $^{\mathrm{83m}}$Kr data.

We perform a set of toy model simulations (not using Locust) in which inelastic and elastic scattering are modeled separately, with pitch angle changes included. The event detection process is approximated by power and track length cuts. In the Locust simulations in this paper, pitch angle changes are ignored and assumed to be zero. When one is concerned with the properties of first tracks (e.g.,~the start frequency spread captured by $\mathcal{I}$), that approach is sufficient. However, this simplification does not allow the prediction of accurate $\mathcal{A}_j$ values.
%The simulated events are not processed with the Phase~II event reconstruction algorithms. Instead, t

For the simulations, it is assumed that inelastic scattering leads to energy loss and small pitch angle changes, while elastic scattering removes electrons from the trap before the next inelastic scattering event~\cite{DavidJoy:ElectronScattering}. 
The inelastic scattering angle $\theta_s$ follows the distribution in \autoref{eq:inelastic scatter angle}~\cite{Rudd:1991differential}.
The elastic scattering is modeled by assuming a fixed fraction $\kappa$ of electrons leave the trap between inelastic scatters due to elastic scatters. The track duration  follows the exponential distribution in \autoref{eqn:track_length_distribution}. The coupled power from a radiating electron is calculated given its instantaneous pitch angle and axial position. 


The detection status of an electron is determined by whether the electron power and track length are both above the corresponding preset thresholds. The SNR threshold chosen matches the threshold applied to tracks in real data by the Phase~II track detection algorithm, prior to combining single tracks into full events (see \cite{TERpaper:2022}). Similarly to in \autoref{sec:sim-events}, it is assumed that the highest simulated power corresponds to the estimated maximum SNR observed in data. The power of all simulated events is translated to SNR accordingly. The track length detection threshold  is set to the minimum recorded track length in data. Tuning $\alpha$ and $\kappa$,  the $\mathcal{A}_j$ curve and the mean event length can be simultaneously matched to values extracted from $\mathrm{^{83m}Kr}$ data (see \autoref{fig:scatter-peak_amplitude_simulation}). 


\begin{table}[ht]
\caption{Values of the parameters $\alpha$ and $\kappa$ found by tuning the scatter amplitude ($\mathcal{A}_j$) curves from simulations to match those from fits to $^{\mathrm{83m}}$Kr data.}
    \label{tab:scatter_peak_amplitude_alpha_and_kappa}
    \begin{center}
\begin{tabular}{c|c|c|c}
\hline
Data set & \(\alpha\) & \(\kappa\) & Mean of sampled scattering angles\\
\hline
\hline
Pre-tritium & 0.0018 & 0.185 & 0.48$^\circ$ \\
Post-tritium & 0.0025 & 0.150 & 0.64$^\circ$ \\
Shallow trap & 0.0025 & 0.400 & 0.64$^\circ$ \\
\hline
\end{tabular}
\end{center}
\end{table}


\autoref{tab:scatter_peak_amplitude_alpha_and_kappa} compiles the values of  $\alpha$ and $\kappa$ found by tuning the $\mathcal{A}_j$ curves to match those from fits for the three $^\mathrm{83m}$Kr data sets.  This simulation reveals that the fraction of trapped electrons with larger pitch angles increases from scatter to scatter as shown in \autoref{fig:evolution_pitch_angle_distribution}. This explains why the scatter peak amplitude curve deviates from an exponential function in the direction of more electrons in the higher order scatter peaks. However, this simulation cannot be directly used to predict $\mathcal{A}_j$ in tritium data, since $\alpha$ and $\kappa$ are related to the gas composition and differ among data sets. Instead, the values of  $p$ and $q$ for tritium data are found by extrapolating from the $p$ and $q$ results of the pre-tritium and post-tritium $\mathrm{^{83m}Kr}$ fits, as described in \autoref{subsubsec:pq_extrapolation}.


\begin{figure}[ht]
  %\centering
  \includegraphics[width=1.0\columnwidth]{Plots/2-models/change_in_pitch_angle_distribution_in_trapped_electrons.pdf}
  \caption{
   Evolution of the distribution of pitch angles of trapped electrons at the center of the trap from simulation.
   %to validate the model for scatter peak amplitude $A_j$.
   After more scatters, a larger fraction of electrons assume larger pitch angles, which corresponds to higher SNRs---thus explaining the deviation in the $\mathcal{A}_j$ curve from a pure exponential, with an excess in higher order scatter peaks.
   %as shown in \autoref{fig:scatter-peak_amplitude_simulation}. 
   }
  \label{fig:evolution_pitch_angle_distribution}
\end{figure}

\section{Determining parameters for the reduced resolution model}
\label{sec:simplified_res_sims}
The tritium analysis model (\autoref{subsec:tritium_analysis_approximations}) approximates the instrumental resolution $\mathcal{I}$ with a weighted sum of two Gaussians (\autoref{eq:T2_ins_res_model}). We perform fits to simulated resolutions with different $\sigma$ values to determine how $\sigma_0^{[1]}$ and $\sigma_0^{[2]}$ depend on $\sigma$.
%To find these functions, we perform fits to simulated resolutions with a range of $\sigma$ values and observe how $\sigma_0^{[1]}$ and $\sigma_0^{[2]}$ depend on $\sigma$. 
To vary $\sigma$ in the simulations, the maximum detectable track SNR (SNR$_{\rm max}$) is varied. As discussed in \autoref{sec:max-SNR-optimization}, $\sigma$ is determined primarily by SNR$_{\rm max}$.   
%the result of this procedure is $\sigma_0^{[1]}(\sigma) = 1.1\sigma + 1.9\,$eV and $\sigma_0^{[2]}(\sigma) = 0.8\sigma - 3.7\,$eV.

To determine the expressions for $\sigma_0^{[1]}$ and $\sigma_0^{[2]}$ in \autoref{sec:T2-det-response}, we use three pieces of information:
\begin{enumerate}
\item Simulations show that $\sigma_0^{[2]} = 0.8\sigma_0^{[1]} - 5.2\,$eV for our instrumental resolution. We observe this linear relation by generating 100 simulated resolutions with different SNR$_{\rm max}$ values, then fitting each resolution with \autoref{eq:T2_ins_res_model}. In these fits, the Gaussian means, Gaussian standard deviations and $\eta$ are all fitted.
% The size of SNR$_{\rm max}$ controls the width of $\mathcal{I}$ (see \autoref{subsec:ins_res}).
\item We observe that $\sigma \approx \eta\sigma_0^{[1]} + (1-\eta)\sigma_0^{[2]}$ for the same 100 simulated resolutions. The relation is approximate because an exact expression for $\sigma$ must depend on $\mu_0^{[1]}$ and $\mu_0^{[2]}$. In that relation, the right hand side is computed from fit results, and $\sigma$ is calculated directly from the simulated resolutions. 
\item We observe that $\eta=0.66$ is constant within fit errors, for the SNR$_{\rm max}$ uncertainty range in tritium data (quantified in \autoref{sec:sigma_errors}).
\end{enumerate}
Combining the first and second relations, and fixing $\eta$, we find $\sigma_0^{[1]}(\sigma)$ and $\sigma_0^{[2]}(\sigma)$---expressions that depend only on $\sigma$ and constants.