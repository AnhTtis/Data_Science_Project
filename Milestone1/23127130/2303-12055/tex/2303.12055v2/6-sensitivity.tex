\begin{table}[htbp]
\caption{Best-fit values with 1$\sigma$ uncertainties for the tritium endpoint and neutrino mass or mass squared,  and 90$\%$ credible/confidence limits on the neutrino mass. The literature endpoint value is 18574\,eV~\cite{Bodine:2015sma,Myers:2015}. The reported Bayesian $m_\beta$ result is a posterior median with 1$\sigma$ quantiles.}
\label{tab:results}
%\renewcommand*{\arraystretch}{1.4}
\centering
\renewcommand{\arraystretch}{1.5}
\begin{tabular}{lccl}
\hline\hline
 & Bayesian & \phantom{aa}Frequentist\phantom{aa} & Unit \\
\hline
Endpoint\phantom{aa} & $18553^{+18}_{-19}$ & $18548^{+19}_{-19}$ & eV \\ 
$m_\beta $ & $57_{-39}^{+61}$ &  &  eV \\
$m_\beta^2$ &  & $2440_{-13354}^{+10175}$  & eV$^2$ \\
90\% C.L. & $m_\beta < 155 $ & $m_\beta < 152 $ & eV \\
\hline\hline
\end{tabular}
\end{table}

\begin{table}[htb]
\caption{Endpoint uncertainty $\sigma(E_0)$ in the frequentist analysis resulting from systematic effects and statistical precision. Here, individual uncertainties are propagated by sampling each parameter from a PDF and fitting the data with a maximum likelihood fit. The total systematic uncertainty including correlations is 2\,eV smaller than adding individual systematic contributions in quadrature.} 
\centering
\renewcommand{\arraystretch}{1.1}
\begin{tabular}{p{0.40\columnwidth}>{\centering}p{0.36\columnwidth}>{\centering\arraybackslash}p{0.08\textwidth}}
\hline\hline
Effect & Parameters & $\sigma(E_0)$~(eV) \\
\hline
\multicolumn{2}{l}{Systematic w/ correlations}   & $+9, -9$ \\
Systematic quad. sum & & $+11, -11$  \\
\hspace{3mm}Mean magnetic field & $B$ & $+4, -4$ \\
\hspace{3mm}Instr.~resolution std.~& $\sigma$ & $+4, -4$ \\
\hspace{3mm}Scattering energy loss & $\gamma_{\mathrm{H}_2}$, $\mathcal{A} (p,q)$ & $+6, -6$ \\
\hspace{3mm}Bin signal efficiencies & $\epsilon_k$ & $+4, -4$ \\
\hspace{3mm}Frequency dependence & $s_\sigma, s_p, s_q$ vs.~$f_c$ & $+6, -6$ \\

Statistical &  & $+17, -17$ \\
\hline\hline
\label{tab:E0uncertainties}
\end{tabular}
\end{table}

\section{Results}\label{sec:results}

\subsection{Tritium endpoint and neutrino mass limit}\label{sec:tritiumresults}

Table~\ref{tab:results} displays the tritium analysis results for $E_0$, $m_\beta $ or $m_\beta^2$, and 90\% confidence/credible limits on $m_\beta$. Bayesian and frequentist results are similar to one another. In particular, we report a neutrino mass limit of 155 eV (152 eV) from the Bayesian (frequentist) analysis.  In addition, the Bayesian (frequentist) $E_0$ results differ by $1.2\sigma$ $(1.3\sigma)$ from the literature value. The Bayesian $1\sigma$ quantile bounds on $m_\beta$ do not contain zero because all of the posterior probability mass is above zero, since $m_\beta$ is a physical mass parameter. For $m_\beta^2$ in the frequentist analysis, negative values are permitted, and the $1\sigma$ bounds on $m_\beta^2$ are consistent with zero.

Table~\ref{tab:E0uncertainties} reports frequentist systematic and statistical contributions to the total uncertainty on $E_0$. The systematic contributions result from the uncertainties on tritium model parameters, summarized previously in \autoref{tab:priors} and \autoref{tab:freq_var_priors}. The statistical uncertainty on $E_0$ ($\pm 17\,$eV) dominates over systematics ($\pm 9\,$eV), though the latter does have a small effect on the total $E_0$ uncertainty ($\pm 19\,$eV in the frequentist case). All systematic contributions are of a similar size. 
 
 In \autoref{fig:tritium_spectrum}, the measured tritium spectrum is shown with both  Bayesian and frequentist best-fit curves, which agree well.  The small remaining model differences that can be seen result from the different treatment of detection efficiency in parts of the spectrum where the efficiency is changing rapidly with respect to energy. In the frequentist analysis, the best-fit bin-efficiencies $\epsilon_k$ are equal to the values extracted from the field-shifted $\mathrm{^{83m}Kr}$ data. In the Bayesian analysis, $\epsilon_k$ are fitted from the tritium data while constrained by priors that account for uncertainties (like all other nuisance parameters). Accordingly, the Bayesian best-fit efficiency for each bin is the mean of the $\epsilon_k$ posterior. For most bins, the efficiency posteriors are similar to the priors, due to the limited statistical power of tritium data. The exception is the low-energy bins in \autoref{fig:tritium_spectrum}, where the tritium data pulls the $\epsilon_k$  distributions away from the best estimates from field-shifted  $\mathrm{^{83m}Kr}$ data.

The frequentist confidence intervals for the endpoint $E_0$ and neutrino mass squared $m_\beta^2$ are shown in the inset, with the literature values for the endpoint energy~\cite{Bodine:2015sma,Myers:2015} and neutrino mass~\cite{KATRIN:2021uub} close to the 1$\sigma$ contour. The shape of the contours for $m_\beta^2<0$ and the lower limit  given in the third row of \autoref{tab:results} depend on the choice of function for the non-physical regime, here chosen to be \autoref{eq:mainz_method}.

\begin{figure}[htb]
  \centering
  \includegraphics[width=1.0\columnwidth]{Plots/5-final-analysis/12-03-22A_final_E0_real_data_phase_II_tritium_fit_1d.pdf}
  \caption{
    Results: Measured tritium beta-decay spectrum with Bayesian and frequentist fits, as well as endpoint intervals from each analysis. (Inset) Frequentist neutrino mass and endpoint contours. The literature sources for the endpoint energy are~\cite{Bodine:2015sma,Myers:2015} and for the  mass the source is~\cite{KATRIN:2021uub}.
  }
  \label{fig:tritium_spectrum}
\end{figure}

 
\subsection{Background}
\label{subsec:background_limit}

During the 82-day tritium data taking period, zero counts were observed in the spectrum in the \SI{1.2}{\keV} (23.5 bins) region of the detection window that was above the tritium endpoint energy. From this a background limit of \SI{3e-10}{\per\eV\per\second} at 90\% C.L. is obtained. If the background is energy-independent and Poisson distributed with a mean number $\mu$ of events in the 50-eV wide analysis window $\Delta E$ defined below in \autoref{eq:analysiswindow}, then the probability of there being a single background event in the window and zero events in the window above the endpoint is
\begin{eqnarray}
P&=&\frac{\mu^1 e^{-\mu}}{1!}\frac{\mu^0 e^{-21\mu}}{0!}.
\end{eqnarray}
The probability maximizes at 1.7\% for $\mu=1/22=0.045$.  The bulk of the probability is for zero events in the analysis window;  numbers $>1$ are highly improbable.   

 
\subsection{Neutrino mass sensitivity \label{sec:sensitivity}}

Project 8 has developed and made extensive use of a simple analytic model for predicting the sensitivity  of differential spectrometers to neutrino mass.  In the model, the size of the neutrino mass $m_{\beta}$ is deduced from the count rate in a part of the spectrum of width $\Delta E$ contiguous with the endpoint, and it is assumed that the relative endpoint energy and the background are very well determined from high-statistics data in other parts of the spectrum.  The width of this ``analysis window'' is optimized with respect to the background rate $b$, energy resolution $\Delta E_{\rm res}$, and any other contributions to line broadening such as the final-state distribution in molecular T$_2$ beta decay, gas scattering, field inhomogeneity, etc.  Systematic contributions from imperfect knowledge of the resolution contributions are added in quadrature with the statistical contribution.   A complete description of the model may be found in \cite{Formaggio:2021nfz}.

The analytic sensitivity prediction model can now be confronted with data (for the first time) in Phase II. 
Table~\ref{tab:ph2sens} lists relevant parameters.  
\begin{table*}[htb]
    \centering
    \caption{Calculation of Phase II sensitivity, and comparison to analytic sensitivity.}
    \renewcommand{\arraystretch}{1.15}
    \begingroup
    \setlength{\tabcolsep}{6pt} % Default value: 6pt
    \begin{tabular}{lrrrl}
    \hline\hline
         Parameter & Value & Fractional  & Absolute  &  Unit \\
         && Uncertainty & Uncertainty & \\
    \hline
    Total effective volume & 14.6 &  & 1.2 &  mm$^3$\\
    Trigger efficiency &0.229&& 0.003 &  \\
    T\&ER efficiency &0.101&& 0.002 &  \\
    Combined trigger, T\&ER & 0.082 & 0.024 & 0.002 & \\
    Mean track duration &156 &0.07 & & $\mu$s \\
    Total cross section &$3.9\times 10^{-22}$ & 0.05 &  & m$^2$ \\
    Electron speed &0.2625c&  &  & \\
    Density as if all H, D, and T &$2.09\times 10^{17}$& 0.09  & & m$^{-3}$ \\
    Hydrogen fraction &0.918 & 0.046 & & \\
    Activity fraction T/(H+D+T) &0.389 & 0.1 & & \\
    T$_2$ density &$7.45\times 10^{16}$& 0.14 & & m$^{-3}$ \\
    Background (Poissonian) &$<3\times 10^{-10}$&  &  & eV$^{-1}$ s$^{-1}$ \\
    Detectable source activity & 0.32 & & & Bq \\
    Detectable rate $r$ &$9.5\times 10^{-14}$&  & & eV$^{-3}$ s$^{-1}$ \\
    \hline
    Volume $\times$ Efficiency  & 1.20 & 0.09 & & mm$^3$  \\
    \hline
    Run time & 7185228 &0.008& & s \\
    $m_\beta^2$ ($E_0$ free,  no sys.) &$2473$& & ${}^{+9822}_{-13233}$ & eV$^2$  \\
    $\sigma(m_\beta^2)$  &$9822$ &  & $1520$ & eV$^2$  \\
    \hline\hline
    \end{tabular}
    \endgroup
    \label{tab:ph2sens}
\end{table*}
We first calculate the effective volumes $V_{\rm eff}$ from the magnetic field $B$ and the range of pitch angles that can be accommodated without exceeding a modulation index of 1, or equivalently an axial amplitude of \SI{2.8}{mm} (see \autoref{tab:quadtrapcoils}). 

Only a subset of electrons trapped within this axial range is detectable above the noise threshold, mainly because electrons at larger radii couple less strongly to the propagating TE$_{11}$ waveguide mode. We compute detection efficiencies by generating simulated tracks in all 4 traps, selecting the subset with average minimum pitch angles of 89.37 degrees (see \autoref{tab:quadtrapcoils}), and passing those events through triggering and T\&ER (track and event reconstruction) processors.  These efficiencies are correlated and the net detection efficiency is not simply the product of the two.   The results are in \autoref{fig:faketrackefficiencies_vs_pitchangle}.
\begin{figure}[htb]
    \centering
    \hfill
    \includegraphics[width=\columnwidth]{Plots/6-sensitivity/efficiencies_for_minimum_pitch_angle.pdf}
    \caption{Number of simulated trapped electrons vs. pitch angle (blue) in the quad trap. The grey histogram shows the subset of events that were detected by the DAQ trigger (red) and the reconstruction methods (green). From the fraction of detected events the efficiency vs. pitch angle is determined. 
    }
    \label{fig:faketrackefficiencies_vs_pitchangle}
\end{figure}

The density (hydrogen equivalent) found from track length data is \SI{2.09e17}{\per\cubic\metre}, with an effective T$_2$ density of \SI{7.45e16}{\per\cubic\metre}  from mass spectrometer data.  The balance is inactive hydrogen, with a small amount of helium that is treated as hydrogen for these purposes.  The standard deviation in the neutrino mass is taken from the frequentist analysis of the Phase II data.  It is derived with the endpoint energy $E_0$ and background $b$ floating, which is the assumption made in deriving the sensitivity curves.    We use only the positive side of the 1-standard-deviation uncertainty range of the data point, because the best-fit value is in the positive,  physical regime.  Negative fit values require a functional form to be chosen for the negative regime~\cite{Formaggio:2021nfz}. 


In  \autoref{fig:ph2sensitivitydata}, we compare the measured sensitivity from the Phase II analysis described above to the predicted sensitivity for this apparatus and running time.
\begin{figure}[htb]
    \centering
    \includegraphics[width=0.49\textwidth]{Plots/6-sensitivity/ph2_sens_20230310.pdf}
    \includegraphics[width=0.50\textwidth]{Plots/6-sensitivity/ph2sensitivityblowup20230207.pdf}
    \caption{(a) 
    Statistical uncertainty in neutrino mass from Phase II data compared to the predicted sensitivity for a T$_2$ density of \num{7.5e16}  molecules m$^{-3}$, resolution of 50 eV FWHM,  and a background limit of \SI{3e-10}{\per\eV\per\second}  (90\% C.L.).  The data point is from the frequentist analysis.  
    The evolution with volume and efficiency is shown for experiments run for \SI{3e7}{\second}: with the Phase II gas mixture, resolution and background (red solid line);  with improved resolution (\SI{1.5}{\eV}  FWHM), and pure T$_2$ (red dashed line); and with atomic T  for \SI{0.5}{\eV} FWHM resolution and a density that optimizes the experimental reach in neutrino mass (blue dotted line). The number of readout channels is assumed to be 1 for each line which leads to identical background rates.
    Systematic uncertainties have not been fully quantified, however. (b) Expanded view of the region with the Phase II data point and the actual run time of \SI{7.2e6}{\second}. Uncertainty in the gas density is included with the data point rather than by broadening the theoretical curve.}
    \label{fig:ph2sensitivitydata}
\end{figure}
 For the theoretical curves, the analysis window $\Delta E$  is the quadrature sum of $\sqrt{b/r}\simeq 22$ eV and the  resolution, about \SI{50}{\eV} FWHM, where 
 \begin{eqnarray}
     r&=&\frac{dN}{dt~d(\epsilon^3)}
 \end{eqnarray}
 is the count rate in an energy interval $\epsilon$ contiguous with the endpoint.   The analytic model assumes Gaussian statistics.  A mean Gaussian rate of $b=$\,\SI{4.7e-11}{\per\eV\per\second} for the background reproduces the probability in the analysis window for the sensitivity calculation.  Reduced to the bare essentials, the optimum analysis window and the uncertainty in $m_\beta^2$ are~\cite{Formaggio:2021nfz}
\begin{eqnarray}
\Delta E &\simeq& \sqrt{\frac{b}{r}+(\Delta E_{\rm res})^2}, \label{eq:analysiswindow}\\
\sigma_{m_\beta^2} &\simeq & \frac{2}{3}\sqrt{\frac{1}{rt}\left(\Delta E + \frac{b}{r\Delta E}\right)},  \label{eq:sig} 
\end{eqnarray}
where $t$ is the run time and $\Delta E_{\rm res}$ is the FWHM of the combined resolution contributions.  In regimes where $\Delta E$ is dominated by $\Delta E_{\rm res}$, the curves fall with a logarithmic slope of $-1/2$.  Where $\Delta E$ is dominated by $b/r$, the slope steepens to $-3/4$.  The curves flatten out when systematic uncertainties become important, in this case taken to be a \SI{1}{\percent} uncertainty in the resolution.  The curves shown in Fig.~\ref{fig:ph2sensitivitydata}(a) are calculated for a run time of \SI{3 e7}{\second}  and the effective volume for the Phase II data point is scaled to the actual run time.  Rescaling in that way is only accurate for an experiment for which the sensitivity is dominated by $\Delta E_{\rm res}$ rather than background and signal rate, because the signal scales with volume and time while the background scales only with time. The expanded view in panel (b) is therefore calculated with the actual running time of 82 days. The sensitivity determined for Phase II agrees well with the sensitivity curve.  
 

\section{Conclusions and outlook}\label{sec:outlook}

A major motivation in Phase II of Project 8 was to obtain information of value for the future scaling up of the CRES technology for neutrino mass measurement.  With the waveguide-based CRES apparatus described in this work, we have carried out the first measurement of the tritium beta spectrum by this new method. We report an upper limit on neutrino mass at 90\% C.L.~of 155 (152) eV in a Bayesian (frequentist) analysis.  In both analyses, the extrapolated endpoint energy for molecular T$_2$ decay and the neutrino mass limit are found to be consistent with literature values obtained by traditional methods.  No background events were observed in the 82-day running period.  These results from a small-scale apparatus are the first steps in a phased approach to a CRES-based experiment with high sensitivity to neutrino mass. 

The waveguide CRES cell is quite efficient in coupling the electron to the propagating electromagnetic field.  However, because the electron's guiding center in the trap moves along the direction of propagation, the Doppler effect is maximal.  This limits the pitch angle range that can be used, and hence the efficiency with which useful signals can be obtained from the gas in the trap region.  Two approaches to  circumventing this limitation in a large volume are to use an antenna array that is directed at signals emitted by the electron perpendicular to its axial motion, and to use a resonant cavity wherein the phase velocity is much larger and the Doppler shift correspondingly smaller. The sensitivities of large-volume cavity experiments are marked in \autoref{fig:ph2sensitivitydata}.

The CRES method is inherently capable of high resolution, as may be seen simply from the widths of tracks in \autoref{fig:spectrogram}.  These tracks are roughly a single bin wide in frequency space, about 1 part per million, which corresponds to an energy width at the tritium endpoint of \SI{0.5}{\eV}.  That measure of the intrinsic resolution depends on the signal-to-noise ratio, and the system noise temperature in Phase II was about \SI{132}{\kelvin} when the CRES cell was at \SI{85}{\kelvin}.  Substantially lower physical temperatures can be realized, especially for atomic tritium in a magnetic trap.  The total experimental resolution has other contributions, primarily the varying magnetic field experienced by electrons moving in a trap that may itself not be ideal, with spatial inhomogeneities and temporal instabilities.  
Scattering of electrons from gas molecules can lead to energy loss prior to the detection of an event. 

The analysis is complex, a consequence of the experiment's exploratory nature---and in particular, of certain unanticipated aspects of design and analysis. For example, the gas composition stability and measurement was more important than expected, given the large role of scattering in the data. %Some  complexity stems from aspects of the experiment design and analysis that were not anticipated, but which Project 8 is now better-equipped to address in future experiments.
%The uncertainty of the measured gas composition propagates to the endpoint determination and the mass limit.
In future experiments, scattering effects will be less significant and better controlled, due to higher energy resolution, purer source gas, and improved composition monitoring. In addition, in Phase~IV (Project 8's final planned phase), the energy loss from scattering ($\gtrsim 12\,\si{eV}$) will  fall outside the likely energy region of interest.
%in when the most sensitive region in the spectrum will be within 1\,eV of the endpoint.
%As a result, the complication and uncertainty of the analysis are expected to be much reduced.  
Another unexpected complication was the extent to which the efficiency as a function of frequency and energy was modulated by parasitic reflections in the waveguide, demanding a detailed analysis methodology. In the future, the energy region will be smaller and the mode structure will be a key part of the experimental design, avoiding large efficiency variations and further simplifying the analysis.
%will be avoided, and the analysis further simplified. 

The zero background observed is both encouraging and expected. In the CRES method there are no physical detectors for electrons (for example, silicon detectors) that
can be background sources. Electrons or gammas from external radioactivity and cosmic rays do not produce trapped electrons unless an interaction with a gas atom occurs, because electrons produced at the wall can make at most one cyclotron orbit before striking the wall again, while those produced at the ends can make at most a single axial cycle before leaving. Cosmic ray interactions with the dilute source gas itself are negligible even for very large-scale experiments.   In addition, in a differential spectrometer, an electron must be in the right energy range to create a background event.  Electrons in a CRES system are not slowed nearly to rest and reaccelerated as in a retarding-field analyzer, and the counting of ubiquitous slow electrons is thereby avoided.  The primary background that remains is false tracks from random aggregations of noisy pixels.  This RF noise background scales with the number of readout channels and not with the volume. 
This background was the target of a detailed study before data-taking began in Phase II, so that a power threshold could be set to limit false events to less than 1 in 100 days at \SI{90}{\percent} C.L.  The success of that strategy is one of the most important conclusions of this study.  A good signal-to-noise ratio is key to eliminating the RF background at minimal cost in efficiency, and in future CRES designs it is a basic requirement. 




\section*{Acknowledgments \label{sec:ack}}


This material is based upon work supported by the following sources: the U.S. Department of Energy Office of Science, Office of Nuclear Physics, under Award No.~DE-SC0020433 to Case Western Reserve University (CWRU), under Award No.~DE-SC0011091 to the Massachusetts Institute of Technology (MIT), under Field Work Proposal Number 73006 at the Pacific Northwest National Laboratory (PNNL), a multiprogram national laboratory operated by Battelle for the U.S. Department of Energy under Contract No.~DE-AC05-76RL01830, under Early Career Award No.~DE-SC0019088 to Pennsylvania State University, under Award No.~DE-FG02-97ER41020 to the University of Washington, and under Award No.~DE-SC0012654 to Yale University; the National Science Foundation under Award No.~PHY-2209530 to Indiana University, and under Award No.~PHY-2110569 to MIT; the Cluster of Excellence “Precision Physics, Fundamental Interactions, and Structure of Matter” (PRISMA+ EXC 2118/1) funded by the German Research Foundation (DFG) within the German Excellence Strategy (Project ID 39083149); the Karlsruhe Institute of Technology (KIT) Center Elementary Particle and Astroparticle Physics (KCETA); Laboratory Directed Research and Development (LDRD) 18-ERD-028 and 20-LW-056 at Lawrence Livermore National Laboratory (LLNL), prepared by LLNL under Contract DE-AC52-07NA27344, LLNL-JRNL-845409; the LDRD Program at PNNL; Indiana University; and Yale University.  Portions of the research were performed using the Core Facility for Advanced Research Computing at CWRU, the Engaging cluster at the MGHPCC facility, Research Computing at PNNL, and the HPC cluster at the Yale Center for Research Computing.  The $^{83}$Rb/$^{83{\rm m}}$Kr isotope used in this research was supplied by the United States Department of Energy Office of Science through the Isotope Program in the Office of Nuclear Physics.

