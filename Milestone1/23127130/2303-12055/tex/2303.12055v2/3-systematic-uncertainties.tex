\section{Tritium parameter estimates and uncertainties}
\label{sec:systematic_uncertainties}

We study and quantify the following systematic effects for the tritium data analysis: 
\begin{enumerate}
\item The mean magnetic field $B$ that converts cyclotron frequencies to energies;
\item The tritium-specific simulated instrumental resolution $\mathcal{I}$, which determines $\sigma$;
\item Scatter peak amplitudes $\mathcal{A}_j$ (parameterized by $p$ and $q$);
\item The energy-dependent event detection efficiency $\epsilon$;
\item The frequency dependence of the energy point-spread function $\mathcal{R}_{\mathrm{PSF}}$ (specifically, of $\sigma$, $p$, and $q$); and
\item Gas composition, which determines the hydrogen inelastic scattering fraction $f_{\mathrm{H}_2}$ in tritium data.
\end{enumerate}
This section covers items 1-5. Item 6\textemdash gas composition and associated uncertainties\textemdash was discussed in \autoref{sec:gas_composition} and \autoref{sec:T2-det-response}.

Systematic factors affect the tritium data analysis via three pathways. First, $^{\mathrm{83m}}$Kr-specific estimates of some uncertainties (on simulated resolutions and gas composition) are incorporated into the $^{\mathrm{83m}}$Kr quad trap analysis and propagated to $B$, $p$ and $q$---the three tritium model parameters from $^{\mathrm{83m}}$Kr fits. Second, tritium-specific estimates of some uncertainties (on $\sigma$ and $f_{\mathrm{H}_2}$) are directly employed in tritium data analysis and propagated to the endpoint $E_0$ and neutrino mass $m_\beta$ as described in \autoref{sec:final-analysis}. Third, $^{\mathrm{83m}}$Kr fits are used to estimate the variation of parameters ($\epsilon$, $\sigma$, $p$ and $q$) across the tritium ROI, as well as the uncertainty on this variation.

Below, we describe procedures for estimating  parameters in the tritium model, their systematic uncertainties, and where relevant, their correlations. At the end of this section, \autoref{sec:summaryofpriors} summarizes the probability distributions for all tritium model parameters, which account for uncertainties on these parameters.


\subsection{Mean magnetic field \texorpdfstring{$B$}{} \label{sec:Bfield_errors}}

% During tritium data analysis, the average magnetic field $B$ experienced by electrons in the deep quad trap is used to convert event-start frequencies $f_c$ to energies $E_{\text{kin}}$ via \autoref{eq:energytofrequency}.  The energies then enter into the tritium model in \autoref{eq:fullT2model}. 
A systematic uncertainty in $B$ shifts the overall reconstructed energy scale, so uncertainties in $B$ are expected to propagate to $E_0$. By contrast, the $m_\beta$ determination is unaffected  by the $B$ systematic, since $m_\beta$ is altered by the second moment of the energy PSF (spectral broadening), not the first moment (overall energy scale) \cite{robertson:1988aa}. 

The best estimate for $B$ is determined by fitting $^{\mathrm{83m}}$Kr pre-tritium quad trap data.
%following the procedure described in \autoref{sec:Kr_model}. 
We choose this data set because its event features most closely resemble those of tritium data. In particular, as shown in \autoref{tab:mml_track_length_fit_results}, the pre-tritium mean track duration differs from that of tritium data by 10\% (compared with 29\% for post-tritium data), and the pre-tritium mean number of tracks differs by 2\% (compared with 43\% for post-tritium data). This choice minimizes differences between parameters fitted from $^{\mathrm{83m}}$Kr data and tritium parameters. Pre- and post-tritium estimates of $B$ differ by $1.6\sigma$, %(see \autoref{sec:Kr-quad-trap-uncertainties});
indicating that the impact of discrepancies in data-taking conditions on $B$ is relatively small. The best estimate for $B$ is shifted downward by $5\times10^{-7}\,$T to correct for a 14\,kHz (0.3\,eV) mean error in start frequencies. This error is caused by the reconstruction algorithm identifying electron tracks with a small time delay, on average, during which the electron loses energy to radiation. The uncertainty on this shift contributes negligibly to the $B$ uncertainty.

%There is a small systematic shift of $\mathcal{I}$ caused by the event reconstruction algorithm. 
%The algorithm on average picks up the electron tracks with a small delay. Because of the upward track slopes, this corresponds to a frequency shift of $+14\,\si{kHz}$ (equivalent to $+0.3\,\si{eV}$), relative to the true start frequency \cite{TERpaper:2022}.
%This effect is present in all data ($^{\mathrm{83m}}$Kr and tritium). To account for it, we decrease the reported magnetic field value in \autoref{sec:Bfield_errors}.

The uncertainty on $B$ includes three contributions from the $^{\mathrm{83m}}$Kr quad trap fitting process:
%which were described in \autoref{sec:Kr-quad-trap-uncertainties}. These contributions are
the statistical uncertainty outputted by the $^{\mathrm{83m}}$Kr maximum likelihood fit ($\pm 3\times10^{-7}\,$T), the uncertainty in the gas composition of pre-tritium $^{\mathrm{83m}}$Kr data ($\pm 7\times10^{-7}\,$T), and simulation uncertainties on the instrumental resolution input to the $^{\mathrm{83m}}$Kr fit ($\pm 5\times10^{-7}\,$T). 
Combining the three uncertainties in quadrature, we measure a mean magnetic field of $B=0.9578099(9)\,$T. Separately, there is a magnetic field uncertainty from a 0.5\,eV uncertainty on the $^{\mathrm{83m}}$Kr K-line energy~\cite{V_nos_2018}. Accounting for the external K-line uncertainty, we find $B=0.9578104(13)\,$T for the tritium analysis. 



\subsection{Energy resolution \texorpdfstring{$\sigma$}{} \label{sec:sigma_errors}}

The simulated resolution $\mathcal{I}$ for the tritium data analysis differs from the resolutions used for quad trap $^{\mathrm{83m}}$Kr analyses due to slight differences in mean track duration and mean number of tracks per event.
%(see \autoref{sec:simCRES}).  
The standard deviation $\sigma$ of the tritium resolution is estimated by fitting the tritium-specific simulated resolution with the model in \autoref{eq:T2_ins_res_model}. The best fit result is $\sigma=15.10\,$eV.
In the tritium data fits, we include two types of uncertainties on the resolution parameter $\sigma$: (a) uncertainties from the $\mathcal{I}$ simulation process, and (b) an uncertainty on the optimized SNR$_{\rm max}$ value. The procedures for determining (a) and (b) are described below.

There are three contributions to the simulation uncertainty (a): first, Poisson errors on the number of simulated events; second, uncertainties in the efficiency filter matrix; and third, the uncertainty in the number of events contributed from each magnetic trap in the quad trap. The resulting bin errors are propagated to each bin of the histogrammed start frequencies that comprise the simulated resolution. Combined, these uncertainties are approximately Gaussian and are included in a $\chi^2$ fit of $\mathcal{I}$ using \autoref{eq:T2_ins_res_model}. Accordingly, the $0.22$-eV uncertainty on $\sigma$ that the fit outputs accounts for simulation uncertainties.

The SNR$_{\rm max}$ in the tritium simulation  (b) also affects $\sigma$. To estimate the SNR$_{\rm max}$ uncertainty, we compared the optimal SNR$_{\rm max}$ from \autoref{sec:max-SNR-optimization} (14.3) with the result from an alternate method, the first-track SNR matching described in \autoref{sec:sim-events}. 
This produced an estimate of SNR$_{\rm max}$=18.0. The result of the method in \autoref{sec:max-SNR-optimization} provides the SNR$_{\rm max}$ best estimate, because that method ensures that the distribution's width is consistent with data---our primary concern when estimating $\sigma$. Still, the discrepancy between the two estimates sets the uncertainty scale, as it quantifies the impact of small imperfections in the resolution simulation (for example, mis-modeling of pitch angle changes from scattering, which could not be directly compared to data). Thus, we take the SNR$_{\rm max}$ uncertainty to be half the difference between the two estimates: 1.85. This is larger than the uncertainty from the procedure in Sec.~\ref{sec:max-SNR-optimization} (0.15) and twice the difference between the optimal SNR$_{\rm max}$ values for pre- and post-tritium $^{\mathrm{83m}}$Kr data, suggesting that the uncertainty of 1.85 may be conservative.

To find the corresponding uncertainty on $\sigma$, we fit \autoref{eq:T2_ins_res_model} to 100 tritium-specific resolutions, each simulated with an input SNR$_{\rm max}$ sampled from a normal distribution with a mean of 14.3 and standard deviation of 1.85.
The $\sigma$ uncertainty contribution from SNR$_{\rm max}$ is then $1.73\,$eV, the standard deviation of the 100 fit results. Combining simulation and SNR$_{\rm max}$ uncertainties in quadrature, we find $\sigma=15.1\pm1.7\,$eV for the tritium instrumental resolution. 

% Correlations of $\sigma$ with scatter peak amplitude parameters are discussed at the end of \autoref{sec:scatter_peak_errors}.


\subsection{Scatter peak amplitudes \texorpdfstring{$A_j(p, q)$}{}
\label{sec:scatter_peak_errors}}
\label{subsubsec:extract_scatter_peak_amplitudes}
\label{subsubsec:pq_extrapolation}

% The scatter peak amplitudes $\mathcal{A}_j$ (\autoref{sec:scatter-peak-amplitudes}) are most reliably determined from fitting $^{\mathrm{83m}}$Kr calibration data. Predicting $\mathcal{A}_j$ from simulations would require better knowledge of gas composition and inelastic scattering angle distributions (see \autoref{subsubsec: scatter peak amplitude simulation}).

%As demonstrated by toy simulations, the scatter peak amplitudes follow a modified exponential $\mathcal{A}_j = \exp(-pj^{-dp+q})$ (see \autoref{appendix:scatter_peak_amplitude_simulation}). 
We estimate $p$, $q$ for tritium data using $p$, $q$ results from the two $^{\mathrm{83m}}$Kr quad trap fits, adjusting for the difference in mean number of tracks per event ($N_\text{tracks}^{\text{true}}$) between data sets. More true tracks provide more opportunities to detect an event, even after early tracks are missed---increasing the number of events in higher-$j$ scatter peaks. Accordingly, a higher $N_{\mathrm{tracks}}^{\mathrm{true}}$ produces a slower decline in $\mathcal{A}_j$ as a function of $j$. 
Among data sets obtained with the same magnetic trap configuration, such as the tritium and $^{\mathrm{83m}}$Kr quad trap data, we therefore expect that $N_\text{tracks}^{\text{true}}$ is the dominant factor causing differences in $p$ and $q$.  Besides $N_\text{tracks}^{\text{true}}$, other properties affecting missed track probabilities, such as electron pitch angle distributions, are similar among data sets with the same trap configuration. In addition, the mean track duration $\tau$ is determined to be a sub-dominant factor, since fitted $p$ and $q$ values appear to vary randomly with $\tau$ across data sets, suggesting that they are not strongly correlated with $\tau$.

%For each data set, 
The estimated $N_\text{tracks}^{\text{true}}$ for all data sets are listed in \autoref{tab:data_set_table}. 
%is estimated from the observed number of tracks per event (which is smaller than the true number, due to short and low-power tracks) as described in \autoref{sec:sim-events}.
Results from pre- and post-tritium quad trap $^{\mathrm{83m}}$Kr fits define functions $p(N_\text{tracks}^{\text{true}})$ and $q(N_\text{tracks}^{\text{true}})$, which predict $p$ and $q$ for tritium data. 
%We have two $^{\mathrm{83m}}$Kr data points available with the same trap depth and similar conditions; 
We perform linear extrapolations from these to $p$ and $q$ for the $N_\text{tracks}^{\text{true}}$ in tritium data. \autoref{fig:p-q-vs-ntracks} displays the $p$, $q$ estimates and uncertainties used in this extrapolation. The uncertainty on $N_\text{tracks}^{\text{true}}$ is larger for tritium than for $^{\mathrm{83m}}$Kr data because the tritium data set has fewer events. Uncertainties are propagated via Monte Carlo sampling.  For each of the three quad trap data sets, we sample $N_\text{tracks}^{\text{true}}$ from a normal distribution defined by its mean and uncertainty, given in \autoref{tab:mml_track_length_fit_results}. For each $^{\mathrm{83m}}$Kr data set, we sample $p$ and $q$ from a bivariate normal distribution which accounts for their correlated uncertainty contributions. That set of sampled values predicts one $p$-$q$ pair for tritium. By repeating the sampling process, we construct a bivariate uncertainty distribution of tritium $p$-$q$ values. Means, uncertainties and a correlation are computed from this distribution and reported in the first row of \autoref{table:pq_systematic}.  Note that the mean $p$, $q$ values for tritium are close to the pre-tritium $^{\mathrm{83m}}$Kr values; this is because $N_\text{tracks}^{\text{true}}$  is similar for the two data sets. Best estimates for scatter peak amplitudes as a function of scatter order are shown in \autoref{fig:Aj-results}.

\begin{figure}
\centering
\includegraphics[width=0.45\textwidth]{Plots/3-systematic-uncertainties/3-5-scatter-peak-model/p_vs_ntracks_final.pdf}
\includegraphics[width=0.45\textwidth]{Plots/3-systematic-uncertainties/3-5-scatter-peak-model/q_vs_ntracks_final.pdf}
\caption{Fitted scatter peak amplitude parameters (a) $p$ and (b) $q$ vs.~mean number of tracks per event in each data set. For tritium (black dot), parameters are predicted by extrapolating deep quad trap results to the tritium data's number of tracks per event.}
\label{fig:p-q-vs-ntracks}
\end{figure}

\begin{figure}
\centering   
\includegraphics[width=\linewidth]{Plots/3-systematic-uncertainties/3-5-scatter-peak-model/scatter_peak_curve_points_log2.pdf}
\caption{Best-estimate scatter peak amplitudes $\mathcal{A}_j$ for $j$ missed tracks. $^{\mathrm{83m}}$Kr estimates are fitted from data; tritium estimates are extrapolated from $^{\mathrm{83m}}$Kr results.}
\label{fig:Aj-results}
\end{figure}


We include three systematic effects in the $p$-$q$ bivariate uncertainty distributions for $^{\mathrm{83m}}$Kr quad trap data:
statistical fit uncertainties, gas composition uncertainties, and resolution ($\mathcal{I}$) simulation uncertainties. These effects produce shifts in different directions for distinct data sets, so they may be propagated in the extrapolation using independent distributions for pre- and post-tritium $^{\mathrm{83m}}$Kr data.
For each $^{\mathrm{83m}}$Kr data set, the covariance matrix for sampling $p$, $q$ is the sum of three covariance matrices associated with the three systematic effects. For statistical fit uncertainties, the matrix is output by the $^{\mathrm{83m}}$Kr maximum likelihood fit. Gas composition uncertainties are determined as described in \autoref{sec:gas_composition}. Specifically, $^{\mathrm{83m}}$Kr data are re-fitted for 300 gas compositions, producing a $p$-$q$ distribution from which a covariance matrix is calculated. Simulation uncertainties in $\mathcal{I}$  are determined using the method in \autoref{sec:res-errors-to-Kr}, which again yields a $p$-$q$ distribution and corresponding covariance matrix.

\begin{table}
\centering
\caption{Estimates and uncertainties of scatter peak amplitude parameters $p$ and $q$ for tritium data.}
\renewcommand{\arraystretch}{1.1}
\begin{tabular}{l c c c}
\hline \hline 
& $p$ & $q$ & Correlation\\ \hline
Best estimate & 0.89 & 1.12 & N/A\\  
Extrapolation uncertainty & 0.06 & 0.05 & 0.60\\  
SNR$_{\rm max}$ uncertainty & 0.09 & 0.01 & 0.56\\
Total uncertainty & 0.11 & 0.05 & 0.38\\
 \hline \hline
 \label{table:pq_systematic}
\end{tabular}
\end{table}   

One additional effect, SNR$_{\rm max}$ uncertainty, is accounted for after the $p$-$q$ extrapolation. This uncertainty is not propagated through the extrapolation because doing so would treat the effect as uncorrelated among data sets. In fact, mis-modeling the SNR$_{\rm max}$ systematically shifts $p$ and $q$ across data sets by altering the width of $\mathcal{I}$, causing $p$ and $q$ to compensate similarly during both quad trap $^{\mathrm{83m}}$Kr fits. We estimate the SNR$_{\rm max}$ uncertainties and correlation for tritium $p$, $q$  using the method in \autoref{sec:res-errors-to-Kr}, with pre-tritium $^{\mathrm{83m}}$Kr data. Covariance matrices for the extrapolation and the SNR$_{\rm max}$ effect are then summed.
Table~\ref{table:pq_systematic} summarizes the results. We find $p=0.89\pm0.11$ and $q=1.12\pm0.05$, with a correlation of 0.38.

\begin{figure*}[tb]
\centering  
\includegraphics[width=\linewidth]{Plots/3-systematic-uncertainties/3-3-efficiency/efficiency_analysis_flowchart.pdf}
\caption{Energy correction of the detection efficiency curve: Count rate dependencies on frequency are obtained from field-shifted $\mathrm{^{83m}Kr}$ data in each single-coil trap at fixed electron energy. Matching the data to dedicated simulation sets yields $\beta(f_c)$, the relative change of SNR with frequency. From these, the count rate dependence on SNR is obtained. $\beta(f_c)$ is corrected for energy using the analytic power dependence on energy (\autoref{eq:power_from_pheno}) to obtain $\beta(f_c(E_{\mathrm{kin}}))$. Combining this with the count rate vs.~$\beta$ yields the energy-corrected count rate vs.~frequency. The energy-corrected count rates vs.~frequency for single traps are summed to obtain the quad-trap rates which correspond to the quad-trap  efficiency for tritium: $\epsilon(f_c(E_{\mathrm{kin}}))$.}

\label{fig:efficiency_energy_correction}
\end{figure*}

The SNR$_{\rm max}$ effect also causes $p$ and $q$ to be correlated with $\sigma$, the standard deviation of the tritium $\mathcal{I}$.  We assume that, in the tritium analysis, the correlations of $p$ and $q$ with $\sigma$ match correlations observed in $^{\mathrm{83m}}$Kr fits due to the SNR$_{\rm max}$ effect. Pearson correlations of $p$ and $q$ with $\sigma$ are determined using 100 sets of ($\sigma$, $p$, $q$) values for different SNR$_{\rm max}$ values.
%(see \autoref{sec:res-errors-to-Kr}). 
We find that the $p$-$\sigma$ correlation from the SNR$_{\rm max}$ effect is 1.00, but the overall $p$-$\sigma$ correlation is 0.82---accounting for the fact that the parameters' uncertainties include other, uncorrelated contributions. The $q$-$\sigma$ correlation from SNR$_{\rm max}$ is 0.60 and the overall correlation is 0.06.

\subsection{Energy-dependent efficiency \texorpdfstring{$\epsilon$}{}} 
\label{sec:efficiency_for_tritium}
% The tritium spectrum is continuous, covering a range in frequency. The recorded spectral shape therefore depends on the variation of the event detection efficiency $\epsilon$ and the response function $\mathcal{R}_{\mathrm{PSF}}$ over the ROI. Mode structures and interference in the CRES cell cause a strong dependence of the received power on frequency and trap location in the cell, which affects both $\epsilon$ and $\mathcal{R}_{\mathrm{PSF}}$. 

The measured efficiency variation with frequency in \autoref{subsec:efficiency} requires a correction for energy dependence before it can be used for tritium analysis. 

\subsubsection{Efficiency correction for energy dependence}\label{sec:energy_correction}
Determining the detection efficiency vs.~frequency curve for tritium requires an additional step beyond the procedure %described in \autoref{subsec:frequency_dependence} 
for $\mathrm{^{83m}Kr}$ data. This is because the field-shifted $\mathrm{^{83m}Kr}$ data at different frequencies vary in $B$ but are fixed at an energy of 17.8\,keV, while electrons in the tritium spectrum experience the same $B$ but have varied kinetic energies. This leads to a difference in radiated power and therefore in detection efficiency. The power coupled to the transporting $\mathrm{TE_{11}}$ mode is a function of both frequency and energy, as derived in \cite{Esfahani:2019mpr}:
    \begin{equation}\label{eq:power_from_pheno}
    \begin{split}
        P(f_c, E_{\mathrm{kin}}) \propto  {Z_{11}(f_c) e^2 v_0^2(E_{\mathrm{kin}})}, 
    \end{split}
    \end{equation}
where $Z_{11}$ is the $\mathrm{TE}_{11}$ mode impedance and $v_0$ is the magnitude of the electron's velocity. 

The process of correcting $\epsilon(f_c)$ for the energy-dependence of the transmitted power is depicted in \autoref{fig:efficiency_energy_correction}. 
Since $B$ was constant during tritium data collection and $f_c$ and $E_{\rm{kin}}$ are linked by \autoref{eq:energytofrequency}, the goal of the correction is to obtain the efficiency's simultaneous dependence on $f_c$ and $E_{\rm{kin}}$ which we denote $\epsilon(f_c(E_{\mathrm{kin}}))$. 
The energy dependence of $\epsilon$ is added by combining information from {field-shifted $\mathrm{^{83m}Kr}$ data}, {simulation}, and \autoref{eq:power_from_pheno}.
The energy correction relies on the fact that a relative change of SNR leads to a predictable relative change of count rate. Because of the SNR differences between the individual trap locations in the cell, predicting the relation between SNR and count rates is difficult in the quad-trap but can be found by summing the single-trap count rates.
Hence the energy correction is done for single-trap data and the  statistical weights from \autoref{subsec:trap_weights} are used to sum the results.
 
We measure the SNR of events in each trap at each frequency in the field-shifting scans relative to the SNR in trap~3 at the un-shifted background field strength. We name this relative SNR $\beta(f_c)$.
%\footnote{The dependence of SNR on frequency is mostly responsible for the efficiency variation in \autoref{fig:fss_event_rates}.} 
In single-trap data, $\beta(f_c)$ (\autoref{fig:snr_scaling_vs_frequency}) is extracted by matching simulations to the SNR distribution in field-shifted $\mathrm{^{83m}Kr}$ data at each step in the field scans (see \autoref{sec:snr_analysis} for more). 
Combining $\beta(f_c)$ with the relative changes of count rate vs. frequency yields the {count rate vs.~$\beta$} (\autoref{fig:count_rate_vs_snr}). With the exception of two narrow frequency ranges at around $25.828\,\si{GHz}$ and $25.926\,\si{GHz}$, the relation between count rate and $\beta$ is bijective and identical in each trap up to a relative scaling factor resulting from trap-depth differences. At $\sim 25.828\,\si{GHz}$ and $\sim 25.926\,\si{GHz}$, the reconstructed event count rate is decreased by the presence of very high track slopes, especially in traps 2 and 3 (see \autoref{fig:slope_vs_frequency}(a)). 
% Old: Excluding these frequency ranges and fitting the remaining count rate vs. SNR from all traps with a 4\textsuperscript{th}-order polynomial  provides a function that enables empirical prediction of a relative  change in detectable tracks from a relative SNR change. 
%New:
We exclude these frequency ranges (they are treated separately in \autoref{sec:slope_correction}) and fit the remaining count rate vs. SNR from all traps with a 4\textsuperscript{th}-order polynomial. This fit provides a function that enables empirical prediction of a relative  change in detectable tracks from a relative SNR change.

%The count rate and polynomial parameter uncertainties are propagated to the energy-corrected detection efficiency curve.

With the relation of count rate vs. SNR in hand, we perform the energy correction for each individual trap by:
\begin{itemize}
    \item Translating relative count rates vs.~frequency in each trap to relative SNR vs.~frequency $\beta(f_c)$.
    \item Multiplying the relative SNR vs.~frequency with the relative dependence of the coupled power on energy: $P(f_c, E_{\mathrm{kin}}) / P(f_c , E_\mathrm{kin}=17.83\,\si{keV})$.
    \item Translating the energy-corrected SNR vs.~frequency $\beta(f_c(E_{\mathrm{kin}}))$ back to relative count rates vs.~frequency using the $\mathrm{4^{th}}$-order polynomial representing count rate vs.SNR.
\end{itemize}

After performing the energy correction in the single traps, we sum the corrected count rates using the trap weights $w_i$ to obtain energy-corrected quad-trap count rates. 
The resulting relative changes of the quad-trap rates with frequency directly correspond to the energy-corrected detection efficiency curve $\epsilon(f_c(E_{\mathrm{kin}}))$, which is shown in \autoref{fig:final_efficiency}(b).  

Each step of the energy correction process contributes an uncertainty to $\epsilon$. All uncertainties are propagated and amount to the width of the grey curve in \autoref{fig:final_efficiency}(b). The main contributions are the statistical uncertainties from the count rates in each constituting trap and the uncertainty from interpolating the coarse field scans in traps 1 and 4 to the dense steps recorded in traps 2, 3, and the quad trap.
%(see \autoref{subsec:trap_weights}). 
Other uncertainties originate from the extraction of $\beta(f_c)$ and the polynomial fit parameter uncertainties of count rates vs.~$\beta$.

\begin{figure}
\centering   
\includegraphics[width=\columnwidth]{Plots/3-systematic-uncertainties/3-3-efficiency/beta_vs_frequency_in_single_traps.pdf}
\caption{Extracted relative SNR change $\beta(f_c)$ during the background field scan in each single-coil trap.}
\label{fig:snr_scaling_vs_frequency}
\end{figure}

\begin{figure}
\centering   
\includegraphics[width=\columnwidth]{Plots/3-systematic-uncertainties/3-3-efficiency/all_counts_vs_beta_on_same_curve.pdf}
\caption{Measured relative count rate vs.~extracted relative SNR change during background field scans. The data points originating from ${\sim} 25.828\,\si{GHz}$ and ${\sim} 25.926\,\si{GHz}$ (where slopes in trap 2 and 3 are highest) are excluded from this plot. The count rates in each trap are scaled relative to trap 3 to minimize the orthogonal distance to a common 4\textsuperscript{th}-order polynomial fit. The orthogonal distance regression (ODR) takes the x and y uncertainties into account. 
}
\label{fig:count_rate_vs_snr}
\end{figure}

\subsubsection{Extraction of SNR dependence on frequency}
\label{sec:snr_analysis}
The relative SNR $\beta(f_c)$ is extracted from the field-shifted $\mathrm{^{83m}Kr}$ data by comparing the data to simulation. We simulate $\mathrm{^{83m}Kr}$ K-line data for a single deep trap and collect the coupled power from each simulated event. Instead of processing the simulated data with the trigger and offline event reconstruction, we apply a response matrix. This matrix maps the true SNR of an event to a distribution of likely reconstructed SNR that integrates to the detection probability for this event. The coupled power of each event is scaled to SNR before being forward folded with the response matrix. The resulting SNR distribution is then compared to the distributions recorded in the field-shifted data for all scan steps. Prior to the comparison, the simulated and recorded data are both reduced to a frequency slice around the peak center of $\pm 1\,\si{MHz}$ (the same cut was applied to the field-shifted $^{\mathrm{83m}}$Kr data). The process of scaling the power to SNR, mapping it to a detected SNR distribution, and comparing it to the recorded data is repeated in a $\chi^2$ minimization with the power-to-SNR scaling factor as the only free parameter. $\beta(f_c)$ is calculated by dividing the power-to-SNR scaling factors by the factor for $\Delta B_{\mathrm{FSS}} = 0$ in trap 3. The results are shown in \autoref{fig:snr_scaling_vs_frequency}. The similarity to \autoref{fig:fss_event_rates} (count rate vs.~frequency from data) gives rise to the direct relation between count rates and $\beta$ in \autoref{fig:count_rate_vs_snr}. 
Note that the lowest frequency point at which $\beta(f_c)$ could be extracted from fits in trap~4 is $\approx 25.85\,\si{GHz}$. At the two scan points below this frequency, the limited statistical power of this trap prevented a successful SNR analysis. As a result, the energy-corrected detection efficiency in the quad trap $\epsilon(f_c(E_{\rm{kin}}))$  is limited to $f_c \gtrsim 25.85 \,\si{GHz}$. %(\autoref{fig:final_efficiency}(b)).

\begin{figure}[htb]
\centering
\includegraphics[width=0.5\textwidth]{Plots/3-systematic-uncertainties/3-3-efficiency/all_trap_slopes.pdf}

\includegraphics[width=0.5\textwidth]{Plots/3-systematic-uncertainties/3-3-efficiency/tritium_count_rates_dip_accounted_compare.pdf}
\caption{(a) Mean recorded first track slope in events that start within $\pm 1 \,\si{MHz}$ of  the fitted $^{\mathrm{83m}}$Kr peak position. (b) Relative detection efficiency predicted for tritium $\epsilon(f_c(E_{\rm{kin}}))$. The tritium efficiency is given relative to the efficiency for $\mathrm{^{83m}Kr}$ K-line events.  As an intermediate step in the construction of $\epsilon(f_c(E_{\rm{kin}}))$, the high-slope region at 25.926\,GHz was excluded from the correction for energy dependence and the efficiency was linearly interpolated in this range (black). The correction for slope variation assumes that the relative decrease of detection efficiency is independent of the kinetic energy, which results in the red curve. The slope peak at 25.828\,GHz lies outside the calibrated efficiency range and the range covered by the tritium spectrum.}
\label{fig:efficiency_slope_correction}
\label{fig:slope_vs_frequency}
\label{fig:final_efficiency}

\end{figure}

\subsubsection{Efficiency correction for slope variation}\label{sec:slope_correction}
In \autoref{fig:slope_vs_frequency}(a), it can be seen that the average track slope in the field-shifted data varies strongly across the scanned frequency range. 
The high-slope regions occur at frequencies where there is enhanced coupling of the electrons to trapped resonant modes as described in \autoref{subsec:rf_system}. The largest effects are caused by the TM$_{01}$ mode, which is trapped at the upper end by the quarter-wave plate and at the lower end by the terminator (which were designed to match the mode impedance of TE$_{11}$, not TM$_{01}$).  Other, weaker resonances arise from reflections at the windows, coupling flanges, gas connection, etc.

The detection efficiency mostly depends on frequency and trap location in the waveguide. However, the event reconstruction has a small track-slope dependence, too. Over a large range of slopes ($\Delta s \lesssim 300 \,\si{MHz/s}$), this is simply due to the fact that the electron tracks with larger slopes cross more frequency bins per time and hence the power in a single time-frequency bin is reduced.  The SNR-scaling analysis of the field-shifted data incorporates this effect by returning a decreased optimum $\beta$ in frequency regions of high slopes. 

The largest slopes within the tritium analysis frequency range are found at ${\sim}25.926\,\si{GHz}$ in trap 2. Here the reconstruction algorithm sometimes misses or breaks tracks and the reconstruction efficiency is reduced. 
Therefore, we exclude this frequency region from the count rate vs.~$\beta$ correlation (\autoref{fig:count_rate_vs_snr}), and thus also from the efficiency correction for energy dependence.  Instead, we linearly interpolate $\epsilon$ for this frequency range which results in the black efficiency curve in \autoref{fig:efficiency_slope_correction}(b).
The slope peak at ${\sim}25.828\,\si{MHz}$ is below the efficiency analysis range. The smaller peak at around ${\sim}25.86\,\si{GHz}$ is close to the expected tritium endpoint location but the event reconstruction quality is not further diminished by the increased slopes in this region. 

 To re-introduce the variation in efficiency that is observed in the field-shifted data at ${\sim}25.926\,\si{GHz}$, we multiply the interpolated $\epsilon(f_c(E_{\mathrm{kin}}))$ by the relative efficiency decrease in this region. The uncertainty on $\epsilon(f_c(E_{\mathrm{kin}}))$ increases in the process, since the multiplied ratio comes with an uncertainty that is added in quadrature to the uncertainty from interpolation. The result is shown in red in \autoref{fig:efficiency_slope_correction}(b).

\subsubsection{Efficiency binning}
\label{sec:efficinecy_binning}
Since the tritium analysis is performed with binned data, we obtain an efficiency $\epsilon_k$ for each bin $k$ by integrating the quasi-continuous efficiency $\epsilon(f_c(E_{\mathrm{kin}}))$ over the bin.

The interpolation of the measured count rates in each trap makes it possible to calculate a relative efficiency value at any frequency over the calibrated range covered by the field-shifted $\mathrm{^{83m}Kr}$ data. The correction of $\mathcal{\epsilon}$ for energy dependence produces an efficiency curve that applies for any energy corresponding to a frequency at which the field-shifted $\mathrm{^{83m}Kr}$ data had sufficient statistical power to perform the energy correction. This limits the energy range over which the tritium data can be modeled to 16.2--19.0\,\si{keV}. However, for the background-only energy range above the tritium endpoint, $\epsilon(f_c(E_{\mathrm{kin}}))$ is not used because the background is due only to false tracks from noise and is thus energy-independent and unrelated to signal efficiency.

%We perform a binned tritium analysis with bin widths of approximately 2.5\,\si{MHz} or 50\,\si{eV}. Smaller bin widths did not improve the sensitivity and increased computational overhead. 


\subsection{Frequency dependence of energy response function \texorpdfstring{$\mathcal{R}_{\mathrm{PSF}}$}{}}
\label{subsec:frequency_dependent_detector_response}
In $\mathrm{^{83m}Kr}$ data analysis, the response function $\mathcal{R}_{\rm{PSF}}$ is optimized to fit the K-line at its singular frequency position.  However, we know that the SNR varies with frequency and depends on energy, so we can expect that the instrumental resolution $\mathcal{I}$ changes over the frequency and energy ROI. We also expect the shape of the scattering tail to vary with frequency, since the event reconstruction is impacted, for example, by power-coupling changes in the waveguide that manifest in the track slope changes observed in the field-shifted data.

% The instrumental resolution $\mathcal{I}$ is matched to the SNR of each $\mathrm{^{83m}Kr}$ data set (\autoref{sec:max-SNR-optimization}), and $p$ and $q$, which determine the scatter peak amplitudes $A_j$, are free in the fits (\autoref{sec:deep-trap-data-and-fits}). 
% In the tritium data analysis, we use the extracted $\rm{SNR_{max}}$ from the pre-tritium $\mathrm{^{83m}Kr}$ data, since these two data sets are most similar in average event properties (\autoref{tab:data_set_table}).

\begin{figure}
\centering
\includegraphics[width=\linewidth]{Plots/3-systematic-uncertainties/3-3-efficiency/new_complex_lineshape_spr_p_and_q_vs_frequency_roi.pdf}
\caption{The field-shifted quad trap data is analyzed at each magnetic field step with the $^{\mathrm{83m}}$Kr lineshape model. A separate simulated resolution shape was produced in simulations for each step and the response function parameters $p$ and $q$ were extracted from fits. The $p$ and $q$ fit results over the tritium analysis ROI are shown.
The vertical yellow band indicates the high-slope frequency range at $\sim 26.926\,\si{GHz}$.
}
\label{fig:p_and_q_frequency_variation}
\end{figure}
    
To obtain the variation in the shape of $\mathcal{R}_{\mathrm{PSF}}$, we fit the K-line in the field-shifted $\mathrm{^{83m}Kr}$ data at each magnetic field step.  For each frequency position, a simulated instrumental resolution $\mathcal{I}(f_c)$ is created by scaling the optimum $\mathrm{SNR_{max}}$ for the reference field-shifted data from $\Delta B = 0 \,\si{mT}$ by the relative SNR $\beta(f_c)$. The resulting instrumental resolution serves as input to the lineshape fits in which $p$ and $q$ are left free to obtain their variation with frequency. The result is shown in \autoref{fig:p_and_q_frequency_variation}. We verify that the relative variation with frequency is not significantly affected by the gas composition by repeating all fits for the maximum and minimum allowed helium contribution to inelastic scattering. 

The variation of $p$ with frequency is much larger than that of $q$. Monte Carlo studies of tritium data with varying $p$ and $q$ have shown that the shape of $\mathcal{R}_{\mathrm{PSF}}$ and its impact on the tritium analysis results are more sensitive to changes of $q$ than they are to changes of $p$. It is therefore unsurprising that the $^{\mathrm{83m}}$Kr fits constrain $q$ more tightly than $p$.

Note that the mean $p$ and $q$ seen in \autoref{fig:p_and_q_frequency_variation} are  different from their values under tritium conditions due to the differences in gas composition and system noise temperature (heat transfer from the field-shifting solenoid increased the noise temperature by 10\%). However, we expect the tail shape to vary similarly with frequency in the tritium data. By dividing the $p(f_c)$ and $q(f_c)$ values in \autoref{fig:p_and_q_frequency_variation} by their means, we obtain scale factors $s_p(f_c)$ and $s_q(f_c)$. For the tritium analysis, these scale factors multiply the tritium $p$ and $q$ from \autoref{sec:scatter_peak_errors}. 

The expected variation of $\mathcal{I}$ with frequency in the tritium data is obtained similarly to the calculation of $\mathcal{I}(f_c)$ described above. However, the dependence on energy of the efficiency variation $\epsilon(f_c(E_{\rm{kin}}))$ has to be accounted for. We therefore scale the optimum $\rm{SNR_{max}}$ for tritium (identical to the $\rm{SNR_{max}}$ of the pre-tritium $^{\mathrm{83m}}$Kr data) by $\beta(f_c(E_{\rm{kin}}))$.
%obtained in the energy correction described in \autoref{sec:energy_correction}. 
This allows us to generate the expected $\mathcal{I}(f_c(E_{\rm{kin}}))$ at each frequency in the tritium ROI. From these distributions, we calculate width scale factors $s_\sigma(f_c(E_{\rm{kin}}))$, which multiply $\sigma$, the resolution standard deviation from \autoref{sec:sigma_errors}.

The estimate ranges of $s_p$, $s_q$, and $s_{\sigma}$ are given in \autoref{tab:freq_var_priors}.
The impact of the frequency and energy variation of $\mathcal{R}_{\mathrm{PSF}}$ on the tritium analysis is discussed in \autoref{sec:MCstudies}.

\subsection{Summary of parameter probability distributions\label{sec:summaryofpriors}}

Parameter estimates and uncertainties are summarized in \autoref{tab:priors} and \autoref{tab:freq_var_priors}. Statistical uncertainties contribute more to the endpoint interval than the systematic uncertainties combined. When propagating systematic uncertainties to the endpoint and neutrino mass, we use normal distributions to describe the uncertainties on all parameters in \autoref{tab:priors}. For the energy response ($\mathcal{R}_{\mathrm{PSF}}$) parameters $p$, $q$ and $\sigma$, a multivariate normal distribution is used to account for correlations. Physical parameter bounds (e.g., $\sigma>0$) are enforced during fits to tritium data. While normal distributions do not permit such bounds, the distributions are all localized sufficiently far from bounds, as verified by MC studies (see \autoref{sec:MCstudies}).  

\begin{table}[htbp]
\caption{Estimates with uncertainties for parameters in the model of tritium data, derived from $^{\mathrm{83m}}$Kr calibration data and  simulations.}
\centering
\renewcommand{\arraystretch}{1.15}
\begingroup
\setlength{\tabcolsep}{6pt} % Default value: 6pt
\begin{tabular}{l c }
\hline \hline
Parameter & Estimate \\ \hline
$\mathrm{H}_2$ inelastic scatter fraction $\gamma_{\mathrm{H}_2}$ & $0.91(5)$  \\
Mean magnetic field $B$ & $0.9578099(13)$\,T \\
Instrumental resolution $\sigma$ & $15(2)$\,eV \\
Scatter peak amplitudes:~$p$ & $0.89(11)$ \\
Scatter peak amplitudes:~$q$ & $1.12(5)$ \\
Number of events & 3770 events after cuts \\
\hline \hline
\label{tab:priors}
\end{tabular}
\endgroup
\end{table}

\begin{table}[htbp]
\caption{Size and uncertainty of variation of $\epsilon$, $\sigma$, $p$, and $q$ with $f_c$ in the tritium model, derived from $^{\mathrm{83m}}$Kr calibration data and  simulations. The uncertainty on $s_\sigma$ does not affect tritium analysis results and was therefore omitted. }
\centering
\renewcommand{\arraystretch}{1.15}
\begingroup
\setlength{\tabcolsep}{6pt} % Default value: 6pt
\begin{tabular}{l c c }
\hline \hline
\makecell[l]{$f_c$-dependent \\ parameter scaling}& \makecell[c]{Estimate \\ (min--max) }& \makecell[c]{Uncertainty \\ (min--max)} \\ \hline

$\epsilon$ & 0.31--1.42 & 0.02--0.05 \\
$s_\sigma$  & 0.87--1.07 & not included\\
$s_p$  & 0.59--1.53 & 0.07--0.21  \\
$s_q$  & 0.95--1.10 & 0.01--0.06  \\

\hline \hline
\label{tab:freq_var_priors}
\end{tabular}

\endgroup
\end{table}


