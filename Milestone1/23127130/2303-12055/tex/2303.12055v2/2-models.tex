\section{Simulated CRES data \label{sec:simCRES}}

%Simulated data are used to find numerical descriptions of the instrumental resolution $\mathcal{I}$ (\autoref{subsec:ins_res}) and the radiative loss spectrum $\mathcal{L}_r$ (\autoref{fig:radiation_loss}). In addition, the single deep trap K-line data of each magnetic field step in the background field scans is reproduced in simulation. This allows us to translate the measured frequency-dependent detection efficiency curve $\epsilon$ to an energy-dependent curve used in the tritium data analysis (\autoref{sec:efficiency_for_tritium}). In \autoref{sec:sim-events}, we discuss the basic framework and inputs to Phase~II event simulations and demonstrate good agreement between simulated and real data. Section~\ref{sec:gen_resolution} describes how a numerical instrumental resolution shape $\mathcal{I}$ is constructed from an ensemble of simulated events. 


%\subsection{Simulation validation}
\label{sec:sim-events}

Simulations are used to generate inputs to the analysis model and to evaluate the performance of event detection and reconstruction methods. It is therefore crucial that simulated events accurately reproduce the features of real data, especially in the main properties relevant for reconstruction: number of tracks per event, track duration, and SNR.

\subsection{CRES signal generation with Locust}
The Locust software package~\cite{AshtariEsfahani:2019mwv} simulates the detection of RF signals by modeling the response of an antenna and receiver to time-varying electromagnetic fields. Locust can independently generate a custom signal to use as input to its receiver chain algorithm, which processes the signal prior to digitization and recording. We developed a Locust signal generator module that simulates chirped data with typical electron event properties. 
The Phase~II waveguide and trap geometries are implemented in this generator to create realistic Phase~II-like event signals. 
Event starting conditions are sampled from probability density functions. 
The generated CRES signals are added to Gaussian white noise. The relative amplitudes of event signals and noise are set to reproduce the SNR observed in experimental data.  
Simulated and experimental data are processed with the same event reconstruction methods.


We generate a set of simulated events to compare to experimental data. %and validate the use of simulation to derive inputs to the CRES spectrum analysis. 
For this purpose, electrons are sampled at the $\mathrm{^{83m}Kr}$ K-line energy at different radial and axial positions in the waveguide, and with different pitch angles, generating trajectories in response to the magnetic field map of a single-coil trap in the apparatus.
To reproduce multi-trap effects (e.g. SNR and field variation differences between traps), the events from simulations with different trapping fields are combined. 

\subsection{Signal frequency and power}
The average CRES signal frequency and power are calculated from the magnetic field along the electron's trajectory and the power coupled to the $\mathrm{TE}_{11}$ mode. This calculation accounts for the frequency modulation associated with an electron's pitch angle (\autoref{sec:electron_trap}).
%~\cite{Esfahani:2019mpr} citation removed because it's already been referenced earlier in this paper. 
Smaller pitch angles lead to reduced power in the carrier and more power in sidebands.  Only the carrier is detected in this experiment. 
%In the track detection process, power thresholds are applied which translate to a maximum axial excursion in the trap and a maximum variation of magnetic field seen by the electron during axial motion (see \autoref{sec:electron_trap}).   
%The resolution $\mathcal{I}$ thus depends on the range of SNR values accepted in analysis.  Once the maximum SNR for a trap has been measured, the resolution can be predicted for any choice of threshold.
Field-shifting studies measured SNR differences between the single traps (\autoref{subsec:efficiency}). These differences are accounted for by multiplying the signal power in each trap with a relative SNR factor. For generating a simulated data set that can directly be compared to recorded data, the overall SNR scale is determined by iteratively adjusting the maximum coupled power until the first-track SNR distribution after reconstruction matches that of real data. The maximum coupled power corresponds to SNR$_{\rm max}$, the SNR of a $90^{\circ}$ electron at $r=0\,\si{mm}$ in trap 3. This is the trap in which power is coupled most effectively into the transporting waveguide mode at the CRES frequency of the $\rm{^{83m}Kr}$ K-line. Later, this procedure for setting the SNR scale is replaced by the method described in \autoref{sec:max-SNR-optimization}.

\begin{figure}[tb]
    \includegraphics[width=0.95\columnwidth]{Plots/3-systematic-uncertainties/3-2-instrumental-resolution/ntracks_5.0_recon_dist.pdf}
    \includegraphics[width=0.95\columnwidth]{Plots/3-systematic-uncertainties/3-2-instrumental-resolution/ntracks_scan_ntracks_scan_dirac_scatter_all_data_sets.pdf}
    \caption{Procedure for extracting the underlying mean number of tracks per event from simulations: (a) For each configured value of $N_{\mathrm{tracks}}^{\mathrm{true}}$, the distribution of reconstructed number of tracks per event is fitted with a geometric distribution of parameter $p_\textnormal{tracks}$. The fit excludes the first bin since the counts are reduced in this bin by event cuts. (b) The intercepts of a linear fit to the fitted $p_\textnormal{tracks}$ for all tested $N_{\mathrm{tracks}}^{\mathrm{true}}$ allows us to find the optimum configuration for each data set. The uncertainty on $p_\textnormal{tracks}$ (horizontal bands) and the uncertainty of the linear fit parameters are propagated to an uncertainty on the extracted mean $N_{\mathrm{tracks}}^{\mathrm{true}}$ (vertical bands).}
    \label{fig:number_of_tracks_extraction}
\end{figure}

\subsection{Simulated event properties}
To simulate multi-track events, a sequence of chirped signals is generated with start frequency and power calculated as described above.
%and is added to a Gaussian noise floor. 
Track slopes are sampled from a Gaussian distribution with a mean ($352.3\,\si{MHz/s}$) and standard deviation ($54.5\,\si{MHz/s}$) corresponding to the mean and standard deviation observed in the deep quad trap $^{\mathrm{83m}}$Kr data. The Gaussian assumption is only approximately valid, but since reconstruction efficiency is relatively insensitive to track slopes as long as they are within several $100 \,\si{MHz/s}$ of the mean, achieving a better agreement of the slope distribution is unnecessary.


The track durations are drawn from an exponential distribution. For each event, the mean track duration $\tau$ that defines this distribution is drawn from a Gaussian with mean and width according to the fit results listed in \autoref{tab:mml_track_length_fit_results}. This way, the uncertainty on the mean track duration is propagated to the simulated data.

\begin{figure}[tbp]
\centering
\includegraphics[width=0.9\columnwidth]{Plots/3-systematic-uncertainties/3-2-instrumental-resolution/data_vs_simulation_katydid_reconstructed_tracks_per_event.pdf}
\includegraphics[width=0.9\columnwidth]{Plots/3-systematic-uncertainties/3-2-instrumental-resolution/data_vs_simulation_katydid_reconstruction_track_length.pdf}
\includegraphics[width=0.9\columnwidth]{Plots/3-systematic-uncertainties/3-2-instrumental-resolution/data_vs_simulation_katydid_reconstruction_with_trigger.pdf}
\caption{Comparison of simulated data to the post-tritium $\mathrm{^{83m}Kr}$ data set after triggering and event reconstruction: (a) number of tracks per event,  (b) first track duration, and (c) first track SNR. The simulations were optimized to reproduce these data.
The good agreement validates the simulations and enables their use to generate input to the analysis. For this purpose simulated data sets were generated to match each data set listed in \autoref{tab:data_set_table}.
}
\label{fig:simulation_validation_track_length_ntracks_snr}
\end{figure}


The number of tracks per event is drawn from a geometric distribution with a configurable expectation value.
The observed mean number of tracks after event reconstruction does not correspond to the underlying truth ($N_{\mathrm{tracks}}^{\mathrm{true}}$), because sometimes tracks are missed or two tracks are combined into one during reconstruction. 
To find the right $N_{\mathrm{tracks}}^{\mathrm{true}}$ for all data sets, events of different $N_{\mathrm{tracks}}^{\mathrm{true}}$ are simulated and reconstructed. %using the same reconstruction methods applied in real data processing. 
The reconstructed number of tracks is then fitted with a geometric distribution, as shown in \autoref{fig:number_of_tracks_extraction}(a). 
The distribution is characterized by its success probability $p_{\rm{tracks}}$, which corresponds to the probability of a detected track to not be followed by another track. 
The relation between the reconstructed and the true mean number of tracks per event was found to be linear in a separate study in preparation for publication. 
Figure~\ref{fig:number_of_tracks_extraction}(b) shows $p_{\mathrm{tracks}}$ vs. inputted $N\mathrm{_{tracks}^{true}}$. 
The underlying $N\mathrm{_{tracks}^{true}}$ for each data set can be read off from the intersection of the linear fit with the data set's $p_\mathrm{tracks}$. For each data set, vertical bands indicate the uncertainty on $N\mathrm{_{tracks}^{true}}$ from two contributions: the linear fit uncertainty and the $p_\mathrm{tracks}$ uncertainty. 

The sizes of frequency jumps between tracks are drawn from the energy loss function for electron-hydrogen scattering, converted to frequency via \autoref{eq:energytofrequency}. 
Since the distribution of first tracks is the only information from simulations that we use as analysis input (see \autoref{sec:gen_resolution}), there is little sensitivity to the loss function and hydrogen serves for all gases.  It is only required that the jump size be large enough to prevent the reconstruction algorithm from joining tracks that are in fact separate.
% Scattering from other gases can be neglected since we only use information from simulated first-tracks as analysis input (see \autoref{sec:gen_resolution}). %for obtaining $\mathcal{I}$, since $\mathcal{I}$ includes only the start frequencies of first tracks, and the energy loss from scattering affects only the start frequency of consecutive tracks. 
Pitch angle changes during inelastic scatters are assumed to be small and are ignored, and the power of consecutive tracks in an event is kept constant.  Despite these approximations, after processing with the Phase~II trigger and reconstruction methods, real data sets are well reproduced by simulated events in the main properties relevant for reconstruction  (\autoref{fig:simulation_validation_track_length_ntracks_snr}).


\section{CRES spectrum analysis \label{sec:analysisoverview}}

%\subsection{Overview}

%The primary content of this paper is the analysis strategy we developed for CRES data and the results we obtained using it.  The analysis is complex, a consequence of the exploratory nature of the experiment.   Some aspects of the experiment design and the analysis were not anticipated; for example, the gas composition stability and measurement was more important than we had expected.
%Scattering of electrons with molecules from each gas species contributes to the total detector response. The gas composition was not stable between calibration data and corrections were needed for the main tritium analysis. The uncertainty of the measured gas composition propagates to the endpoint determination and the mass limit.
%In future designs with higher energy resolution, purer source gas, and better composition monitoring, scattering will be less significant and better controlled. As a result, the complication and uncertainty of the analysis is expected to be much reduced. 
%, but it would not be as significant in future designs with higher resolution and purer source gas.  
%Another unexpected complication was the extent to which the efficiency as a function of frequency and energy was modulated by the presence of parasitic reflections in the waveguide, which demanded a very detailed analysis methodology. In the future, the energy region of interest will be much smaller than in Phase~II and the mode structure a key part of the experimental design. Consequently, large efficiency variations will be avoided. 

%Notwithstanding the complications both expected and unexpected, we have achieved a complete and successful analysis, as reported below.
This section describes the signal model of CRES spectra used in the analysis, with an overview flow chart in \autoref{fig:simple_flowchart}. \autoref{fig:flowchart} of Appendix~\ref{sec:flowchart} contains a detailed flowchart that reflects all analysis steps and the interdependence of the $\mathrm{^{83m}Kr}$ and $\mathrm{T_2}$ analyses. 
\begin{figure*}[tb]
\centering
\includegraphics[width=0.9\textwidth]{Plots/1-introduction/prc_analysis_diagram_2.pdf}
\caption{Flow chart of construction of Phase II CRES analysis models (\autoref{eq:FullModel1}). To obtain a full model spectrum $\mathcal{S}$, the underlying electron energy spectrum $\mathcal{Y}$ is convolved with the point spread function $\mathcal{R}_{\mathrm{PSF}}$ before being multiplied with the detection efficiency $\mathcal{\epsilon}$. }
\label{fig:simple_flowchart}
\end{figure*}



\subsection{CRES energy spectrum model}\label{sec:Kr_detector_response}

We model a generic detected CRES signal spectrum $\mathcal{S}$ as
\begin{eqnarray}
    \mathcal{S} &=& \epsilon\left(\mathcal{Y}*\mathcal{R}_{\mathrm{PSF}}\right), 
  \label{eq:FullModel1} \\
    \mathcal{R}_{\mathrm{PSF}} &=& \sum_{j=0}^{j_{\mathrm{max}}}\mathcal{A}_j\left(\mathcal{I}*\mathcal{L}_{\mathrm{tot}}^{*j}\right).
    \label{eq:FullModel2}
\end{eqnarray}
Diagrams of these two equations are shown in \autoref{fig:simple_flowchart} and \autoref{fig:scattering_illustration}, respectively. In both equations, all variables are functions of $E_\mathrm{kin}$, as denoted by script lettering. The symbol $*$ represents convolution and $^{*j}$ represents self-convolution $j$ times. The efficiency function $\epsilon$ encodes the probability of detecting electron events. The underlying true energy spectrum of the electrons is $\mathcal{Y}$.
%The underlying spectrum is described in \autoref{subsec:Kr Line model} for $^{\mathrm{83m}}$Kr and in \autoref{sec:tritium_model} for tritium.  
$\mathcal{R}_{\mathrm{PSF}}$ is the point-spread function, which represents the energy response for mono-energetic electrons---in other words,  how reconstructed energies are shifted and broadened relative to true energies. 
%and is discussed in-detail in \autoref{sec:efficiency_for_tritium}

\begin{figure}[tb]
  \centering
  \includegraphics[width=1.0\columnwidth]{Plots/2-models/scattering_prc_illustrative_figure_v2023-09-28v1.pdf}
  \caption{(a) A CRES event with frequency jumps corresponding to inelastic scattering  $\sum_i(\gamma_i \mathcal{L}_i)$ and energy loss due to cyclotron radiation $\mathcal{L}_r$ labelled. These energy losses only affect the spectrum when early tracks in the event are missed. (b) A modeled spectrum broken down into its constituent scatter peaks and labelled to show the roles of model parameters: scatter peak amplitudes $\mathcal{A}_j$, instrumental resolution $\mathcal{I}$, and energy loss spectra $\mathcal{L}_{\mathrm{tot}}^{*j}$. A Gaussian $\mathcal{I}$ is used here for illustration purposes only; a finer-grained instrumental resolution is used in the analysis, as described later.}% This is a plot of Eq. \label{eq:simplified_scattering}.
  \label{fig:scattering_illustration}
\end{figure}

Equation~\ref{eq:FullModel2} shows that the energy point-spread function $\mathcal{R}_{\mathrm{PSF}}$ is comprised of a sum of scatter peaks $\mathcal{I}*\mathcal{L}_{\mathrm{tot}}^{*j}$ weighted by amplitudes $\mathcal{A}_j$, as illustrated in \autoref{fig:scattering_illustration}.  These amplitudes describe the relative likelihood that an electron will first be detected after $j$ scattering events. As an example, for tritium data, the best estimates for the first few $A_j$ values are $A_0=1$ (by definition), $A_1=0.41$, $A_2=0.24$, and $A_3=0.15$. We account for the possibility of up to $j_{\mathrm{max}}$ scatters before the first detection. We use $j_{\mathrm{max}}=20$ in both $^\mathrm{83m}$Kr and tritium fits, as increasing $j_{\mathrm{max}}$ further has no observable effect on results for Phase II conditions.  The instrumental resolution $\mathcal{I}$  is the spectrum that a source of mono-energetic electrons would have if they were all detected before scattering. The distribution of electrons' energy losses between scatters $\mathcal{L}_{\mathrm{tot}}$ depends on the gas composition, the differential cross section on each gas component, and the loss to cyclotron radiation.  The elements in $\mathcal{R}_{\mathrm{PSF}}$ are described further in the remainder of this section.
%accounts for broadening from magnetic field inhomogeneity and noise.  

\subsection{Instrumental resolution \texorpdfstring{$\mathcal{I}$}{}}\label{sec:gen_resolution}
In Phase~II, the instrumental resolution $\mathcal{I}$ accounts for broadening from the differences in mean magnetic fields sampled by detected electrons with different pitch angles and radial positions. These mean field distributions vary with trapping geometry. 
To obtain $\mathcal{I}$ for each data set, mono-energetic events in all constituent single traps are simulated as described in \autoref{sec:sim-events}.
In future Project 8 phases, $\mathcal{I}$ will also account for uncertainties on the mean cyclotron frequencies (\emph{e.g.}, due to frequency binning and noise). In Phase II, this effect was small (${\sim}$0.2\,eV) compared to magnetic field variation.
%and are determined by simulations that are matched to each experiment configuration. %(\autoref{sec:sim-events}) (\autoref{subsec:ins_res}). 
%The broadening from frequency reconstruction, \emph{e.g.}, due to frequency binning in the Fourier-transformed data and noise, is only ${\sim}$0.2\,eV. This is small compared to the effect of magnetic field inhomogeneity and is therefore neglected, for $\mathcal{I}$. 


%\subsection{Simulated instrumental resolution}
%Input resolution ($\mathcal{I}$) shapes are required to analyze all $^{\mathrm{83m}}$Kr calibration data sets, shallow-trap $^{\mathrm{83m}}$Kr data, and tritium data.

The shape of $\mathcal{I}$ depends on the range of track SNR values accepted in analysis. This is because SNR thresholds limit the range of  axial excursions in the trap, which in turn limits the range of mean fields experienced by detected electrons.  %(see \autoref{sec:electron_trap}) 
Locust computes relative signal powers to reflect all physical effects in the waveguide. Hence, only the absolute power scale must be set by configuring the power corresponding to the measured SNR$_{\rm max}$. The value of SNR$_{\rm max}$ is specific to each data set and its optimization is described in \autoref{sec:max-SNR-optimization}.
%SNR$_{\rm max}$ has been measured
%Before applying the efficiency filter to the tracks, the track power has to be translated to SNR.
%For each simulated event, start frequency, number of tracks, track duration, and true power are recorded.
%Instead of processing the generated events with the event detection algorithm

The simulated events are filtered by an efficiency matrix, which is a binned look-up table of the probability for an event to be accepted as a function of SNR, first-track duration, and number of tracks in an event. 
The efficiency matrix is produced by simulating 100,000  events covering the full parameter space and analyzing the event detection probability  with respect to those three event properties.
We use this matrix to avoid processing all simulated data with the trigger and reconstruction methods, thereby greatly reducing the processing time. This allows us to iteratively optimize, for example, the event SNR in a data set, with a quick turn-around time. %in each iteration. 

The density histogram of simulated events that survive the efficiency filter is a frequency resolution distribution, which is converted to an energy resolution via \autoref{eq:energytofrequency}. 
%In Phase~II it is therefore neglected.
We center the energy resolution distribution from each trap on $0\,\si{eV}$ to align the distributions before combining them (in the experiment the trapping field strengths are aligned to minimize the resolution width of recorded data). The total $\mathcal{I}$ is a weighted average of the resolutions of individual traps.   The relative SNR scales of individual traps are determined using the mapping from SNR to counts, requiring that the fraction of events in each trap's resolution matches the fraction collected in the trap in real data.  An example of a simulated $\mathcal{I}$ is shown in \autoref{fig:kr_ftc_res}. 
Because the resolution is centered on $0\,\si{eV}$, a fit of the full spectrum model to data will find the overall energy scale, set by the mean magnetic field experienced by the detected electrons.
%the SNR is selected such % (see trap weight analysis in \autoref{subsec:trap_weights}) %number of counts in each trap's resolution distribution is proportional to

\begin{figure}[htb]
\centering   
\includegraphics[width=\columnwidth]{Plots/2-models/Kr_res_pre-tritium_errorbars.pdf}
\caption{Instrumental resolution $\mathcal{I}$ of the pre-tritium $^{\mathrm{83m}}$Kr calibration data set. The error bars include uncertainties from Poisson counting, the efficiency matrix, and the trap weights.}
\label{fig:kr_ftc_res}
\end{figure}

%In addition to the broadening,

%$\mathcal{I}$ can also be broadened by frequency reconstruction, \emph{e.g.}, due to frequency binning in the Fourier-transformed data and noise. We found that this broadening is only ${\sim}$0.2\,eV and is small compared to the effect of magnetic field variation. We therefore neglect this effect in Phase~II  but will account for it in more sensitive Project 8 experiments.

\subsection{Energy loss spectra \texorpdfstring{$\mathcal{L}_{\mathrm{tot}}^{*j}$}{}}
\label{L_model}
Electrons lose energy primarily by inelastic scattering with gas molecules, causing the jumps in \autoref{fig:spectrogram}.
Cyclotron radiation is a smaller, continuous source of electron energy loss, causing the upward track slopes in \autoref{fig:spectrogram}. 
$\mathcal{L}_{\mathrm{tot}}^{*j}$ comprises the distribution of possible energy losses an electron has experienced \emph{before} the first detected track due to both of these effects, with the self-convolution $j$ times accounting for $j$ scatters. The electron energy loss spectrum for a single scatter is
\begin{equation}
\begin{aligned}
    \mathcal{L}_{\mathrm{tot}} = (\gamma_1 \mathcal{L}_1 + \gamma_2 \mathcal{L}_2 + \cdots + \gamma_n \mathcal{L}_n)*\mathcal{L}_r\,,\label{eq:combine_peaks_from_different_gases}
\end{aligned}
\end{equation}
where $\mathcal{L}_i$ is the electron inelastic energy loss spectrum for the $i^{\rm th}$ gas species, each $\gamma_i$ is the fraction of inelastic scatters that are due to the specific gas species $i$, and $\mathcal{L}_r$ is the energy loss spectrum due to cyclotron radiation during the missed track.

We determine bounds on $\gamma_i$ from quadrupole mass analyzer data as described in \autoref{gamma_i_determination}. For the shallow trap data, the high resolution allows for a more precise estimate of gas scattering fractions for H$_2$ and He to be determined from the fit to $^\mathrm{83m}$Kr data.  We neglect the energy dependence of the scatter fractions $\gamma_i$ over the small range of energy change.

Each $\mathcal{L}_i$ is calculated in the Bethe theory of electron inelastic scattering (as in~\cite{Inokuti:InelasticScattering1971}), given by
\begin{eqnarray}
\hspace{-0.1in}\frac{d\sigma}{dE} &=& \frac{4 \pi a_0^2R}{E_{\rm kin}} \left[\frac{R}{E}\frac{df}{dE}\ln\left(\frac{4c_EE_{\rm kin}}{R}\right) + \mathcal{O}\left(\frac{R}{E_{\rm kin}}\right)\right], \quad
\label{eq:inelastic scatter energy loss }
\end{eqnarray}
where $R$ is the Rydberg energy, $a_0$ is the Bohr radius, $E_{\rm kin}$ is the incident energy of the electron, $E$ is the energy loss of the electron, $c_E \approx (R/E)^2$, and $df/dE$ is the optical oscillator strength of the gas molecules.
The optical oscillator strength \cite{Chan:AbsoluteOscillatorStrength1991,chan1992absolute,olney1997absolute} data for the relevant gas species are from the LXCat database~\cite{LXCat_carbone2021data,pitchford2017lxcat,pancheshnyi2012lxcat,LXCat:database}. 

\begin{figure}[tb]
  \centering
  \includegraphics[width=1.0\columnwidth]{Plots/2-models/radiation_loss_spectrum.pdf}
  \caption{Simulated $\mathcal{L}_r$, the energy loss distribution due to cyclotron radiation during a missed track, for conditions (\emph{e.g.}, track duration) corresponding to $^{\mathrm{83m}}$Kr pre-tritium data. The width of this distribution scales with the average track durations and slopes (slopes correspond to radiated power).}
  \label{fig:radiation_loss}
\end{figure}

We determine the loss due to cyclotron radiation $\mathcal{L}_r$ using the simulated data described in \autoref{sec:sim-events}.
In these simulated data, missed tracks associated with detected events are identified.
The distribution of differences between track end frequency and track start frequency among these tracks is converted to energy and taken as the radiative energy loss spectrum $\mathcal{L}_r$ (\autoref{fig:radiation_loss}).
%, a function of track duration $\tau$. 
With most of its weight in a peak between 0 and 3 eV\textemdash reflecting the low likelihood of missing long tracks\textemdash and a modest tail out to ${\sim}$10 eV, $\mathcal{L}_r$ has a much smaller impact than inelastic scatters. 
 
\begin{figure}[htb]
\centering
\includegraphics[width=\linewidth]{Plots/3-systematic-uncertainties/3-5-scatter-peak-model/deep_trap_scatter_amplitudes.pdf}
\caption{From simulation, amplitudes $\mathcal{A}_j$ of scatter peaks in the CRES response function (blue points), caused by missing $j$ tracks in an event. $\mathcal{A}_j$ are simulated accounting for electron pitch angle changes from scattering (modeled by \autoref{eq:inelastic scatter angle}). $\mathcal{A}_j$ are modeled by a modified exponential function parameterized by $p$ and $q$ (\autoref{eq:scatter_peak_amplitude}). Fitting $p$ and $q$ for pre-tritium $\mathrm{^{83m}Kr}$ data results in a curve (green) in good agreement with simulation (orange).}
\label{fig:scatter-peak_amplitude_simulation}
\end{figure}
%determined by track detection probability and 

\subsection{Scatter peak amplitudes\label{sec:scatter-peak-amplitudes} \texorpdfstring{$\mathcal{A}_j$}{}} 
Each amplitude  $\mathcal{A}_j$ is the relative likelihood of missing the first $j$ tracks in an event and detecting track $j+1$. The function $\mathcal{A}_j(j)$ is nearly exponential. It deviates from an exponential due to pitch-angle changes from scattering (which alter the probability of an electron being trapped and detectable) and event reconstruction thresholds (which here depend on first track duration, first track SNR, and number of tracks in an event).

To determine the functional form of $\mathcal{A}_j(j)$, including the deviation from exponentiality, we perform a toy model simulation and reconstruction of events, then count events in each scatter peak. It is assumed that inelastic scattering leads to energy loss and small pitch angle changes, while elastic scattering removes electrons from the trap before the next inelastic scatter~\cite{DavidJoy:ElectronScattering}. 
The simulated inelastic scattering angle $\theta_s$ follows the distribution~\cite{Rudd:1991differential}
\begin{align}
    P(\theta_s) \propto \left(1 + \frac{\cos^2\theta_s}{\alpha^2}\right)^{-1}\,,
\label{eq:inelastic scatter angle}
\end{align}
where $\alpha$ is a constant that depends on gas composition. A fraction $\kappa$ of electrons leave the trap between inelastic scatters due to elastic scatters. As an example, we find $\alpha=0.0018$ (corresponding to an average sampled scattering angle of 0.48$^\circ$) and $\kappa=0.19$ for pre-tritium $^{\mathrm{83m}}$Kr data (similar to tritium data). This simulation and reconstruction procedure is described in more detail in Appendix~\ref{appendix:scatter_peak_amplitude_simulation}.

\autoref{fig:scatter-peak_amplitude_simulation} shows the dependence of $\mathcal{A}_j$ on $j$ from the simulations. The curve may be parameterized by
\begin{eqnarray}\label{eq:scatter_peak_amplitude}
    \mathcal{A}_j &=&\exp\Big[-pj^{(-dp+q)}\Big],
\end{eqnarray} 
with free parameters $p$ and $q$. The constant $d=0.4955$ is chosen to minimize the correlation between $p$ and $q$ and held fixed. For a specific $^{\mathrm{83m}}$Kr data set, $\mathcal{A}_j(j)$ is determined by fitting the CRES spectrum while using  a response function model that includes \autoref{eq:scatter_peak_amplitude}. This produces estimates of  $p$ and $q$. These parameters must be fitted because $\alpha$ and $\kappa$ are not known externally; they are instead tuned to match CRES data. In future experiments, $\alpha$ and $\kappa$ could be predicted using more precise calibration of gas composition and scattering effects. 
%The model is found to provide sufficient flexibility to account for the deviation from exponentiality, as confirmed in simulations shown in \autoref{subsubsec: scatter peak amplitude simulation}.  %The $p$ and $q$ values for $^{\mathrm{83m}}$Kr data sets are determined from fits to the CRES spectra. 

The $p$ and $q$ values for tritium analysis are extrapolated from $^{\mathrm{83m}}$Kr $p$ and $q$ values as a function of the average number of tracks per event ($N_{\mathrm{tracks}}^{\mathrm{true}}$), as described in \autoref{sec:scatter_peak_errors}. The result for tritium data is $p=0.89\pm 0.11$, $q=1.12 \pm 0.05$. The variation in $N_{\mathrm{tracks}}^{\mathrm{true}}$ among data sets stems mainly from differences in gas composition, which change $\alpha$ and $\kappa$, thus changing $\mathcal{A}_j(j)$.
%Gas species have different relative likelihoods of elastic and inelastic scattering, and elastic scattering is far likelier to cause an electron to undergo a large pitch angle change into a non-detectable pitch angle, ending the visible portion of the event.

%Number of simulated events that are detected after $j$ scatters.

%Because the thresholds in the reconstruction algorithm are set empirically to exclude the expected distribution of false events rather than from an \emph{a priori} analytical model, the model for $\mathcal{A}_j$ is also phenomenological. 
%Because scattering changes the distribution of electron pitch angles, $\mathcal{A}_j$ is not expected to follow a strictly exponential dependence on $j$. In addition, $\mathcal{A}_j$ depends both on the details of the event reconstruction algorithm\textemdash its thresholds in first track duration, first track SNR, and number of tracks in the remainder of an event\textemdash and on the distributions of these properties among events in the data. 






\section{Calibration with \texorpdfstring{$^{\mathrm{83m}}\mathrm{Kr}$}{}\label{sec:Kr_model}}

Fits to $^{\mathrm{83m}}$Kr electron lines are used to both characterize the apparatus and estimate parameters for tritium data analysis. \autoref{sec:frequency-energy_relation} details how we performed  Voigt fits to  $^{\mathrm{83m}}$Kr lines to verify the relation between energy and cyclotron frequency. 
%For all other $^{\mathrm{83m}}$Kr fits, instead of a Voigt profile, we employ the CRES spectrum model in \autoref{eq:FullModel1}. Accordingly,
%\autoref{subsec:Kr Line model},  %\autoref{subsec:ins_res} and \autoref{subsec:frequency_dependence}
The remaining subsections describe fits to the 17.8-keV $^{\mathrm{83m}}$Kr line using the full CRES model in \autoref{eq:FullModel1}, producing estimates of the mean field $B$, %experienced by electrons
scattering parameters $p$ and $q$, and detection efficiency $\epsilon$ as a function of frequency.
%that control amplitudes of scatter peaks caused by missed CRES tracks.
%For the fits to reliably estimate these parameters, an instrumental resolution distribution $\mathcal{I}$ must be inputted to the fits. 
%Section~\ref{subsec:ins_res} describes the $\mathcal{I}$ distributions used for $^{\mathrm{83m}}$Kr fits
%including a procedure for finding the resolution width for each data set via an ``SNR scaling optimization," which requires additional $^{\mathrm{83m}}$Kr fits. Section~\ref{subsec:ins_res} also discusses
%and how uncertainties on $\mathcal{I}$ are propagated to fit results.
%$B$, $p$ and $q$ for a $^{\mathrm{83m}}$Kr data set. 
%Section~\ref{subsec:frequency_dependence} describes how the dependence of the detection efficiency $\epsilon$ on frequency is measured by comparing the detected event rate for a range of background magnetic field values. This is the last input to the full CRES spectrum model.
 
%Section~\ref{sec:shallow_trap} and \autoref{sec:deep-trap-data-and-fits} discuss how this CRES spectrum model is used to fit specific $^{\mathrm{83m}}$Kr data sets. 
%In particular, \autoref{sec:shallow_trap} describes a fit to shallow-trap $^{\mathrm{83m}}$Kr data, which demonstrates the energy resolution capabilities of CRES. Finally, \autoref{sec:deep-trap-data-and-fits} discusses fits to $^{\mathrm{83m}}$Kr data obtained with the same deep quad trap used for tritium data. These fits most directly calibrate the tritium energy point-spread function $\mathcal{R}_{\mathrm{PSF}}$. Specifically, the fit outputs are used to predict $p$, $q$ and $B$ for tritium data.

\begin{figure}[htb]
    \centering
    \includegraphics[width=1\columnwidth]{Plots/2-models/prc_K_line_gaussian_res_residual.pdf}
    \caption{Top: Fit of a Voigt profile (red) to $^{\mathrm{83m}}$Kr K conversion electron events (blue) recorded with the shallow trap. 
    The center frequency $f_K$ and the standard deviation $\sigma_K$ of the instrumental resolution are extracted from the fit. Bottom: Residuals in the fitted frequency range. }
    \label{fig:linearity_Kr_K_freq_line}
\end{figure}

\begin{figure}[htb]
    \centering
    \includegraphics[width = 1\columnwidth]{Plots/2-models/single_parameter_energy_vs_freq.pdf}
    \caption{Fit via \autoref{eq:energytofrequency} of the measured frequencies of conversion lines to their kinetic energies as given in \autoref{table:Kr energy lines Venos}.  From right to left the lines are K, L2, L3, M2, M3, N2 and N3.  The magnetic field is the fit parameter.  The error bars do not include a 0.5-eV energy scale uncertainty from the gamma energy. In the residuals, the uncertainties in the frequencies are projected and added in quadrature to those in the binding energies. The frequency scale shown for the residual is not exact but shows the overall correspondence between the magnitudes of the energy difference and the frequency difference for convenience.}
    \label{fig:linearity}
\end{figure}

\begin{table*}[htb]
\caption{The frequencies of the conversion electron lines recorded in the shallow trap configuration. The N2 and N3 lines are not resolved but their frequencies are fitted separately by fixing the intensity ratio and separation between the two lines. The conversion electron line energies are calculated by fixing the gamma energy at the literature value of 32151.6 eV, and using the binding energies and recoil energies from \cite{V_nos_2018}. The 0.5-eV energy scale uncertainty from the gamma energy is not included.} 
\label{table:Kr energy lines Venos}
  \centering
  \renewcommand{\arraystretch}{1.15}
    \begingroup
    \setlength{\tabcolsep}{6pt} % Default value: 6pt
  \begin{tabular}{ l  c  c  c  c}
 \hline\hline
 Line & Conversion & Binding & Recoil &  \phantom{aa}Shallow trap frequency \\  
  & electron &  energy (eV) &  energy (eV) & (kHz) \\ & energy (eV) \\
 \hline\hline
 K & 17\,824.23(4)\phantom{a} & 14\,327.26(4) & 0.120 & 25\,940\,625.2(8)\phantom{a}\\ 
 \hline
 L2 & 30\,419.49(6)\phantom{a} & 1\,731.91(6) & 0.207 & 25\,337\,157.0(6)\phantom{a} \\
 
 L3 & 30\,472.19(5)\phantom{a} & 1\,679.21(5) & 0.207 & 25\,334\,690.7(8)\phantom{a} \\
 \hline
 M2 & 31\,929.26(17) & 222.12(17)  & 0.218 & 25\,266\,701.5(21) \\
 
 M3 & 31\,936.85(11) & 214.54(11) & 0.218 & 25\,266\,348.0(11) \\  
 \hline
 N2 & 32\,136.72(1)\phantom{a} & 14.67(1) & 0.219 & 25\,257\,051.7(27) \\
 N3 & 32137.39(1)\phantom{a} & 14.00(1) & 0.219 & 25\,257\,019.2(27) \\
 \hline\hline
\end{tabular}
\endgroup
\end{table*}


\subsection{Test of the frequency-energy relation} \label{sec:frequency-energy_relation}
To verify the predicted CRES energy-frequency relationship (\autoref{eq:energytofrequency}) across a 14.3-keV range, the $^{\mathrm{83m}}$Kr shallow trap data included measurements of the K, L2, L3, M2, M3, N2 and N3 internal-conversion lines of the 32-keV transition. For each line, the main peak is well separated from the scattering tail and from a $^{\mathrm{83m}}$Kr shakeup/shakeoff structure~\cite{Robertson:2020boa} in this high-resolution trap. This makes it possible to extract the central frequency of the main peak in each $^{\mathrm{83m}}$Kr spectrum by fitting it with a Voigt profile, which has a fixed Lorentzian width as tabulated in~\cite{V_nos_2018}. A constant background is added as a fit parameter when events from the tail of a different $^{\mathrm{83m}}$Kr line are present within the fit range. The frequencies extracted are given in \autoref{table:Kr energy lines Venos} and the fit to the K line is shown in \autoref{fig:linearity_Kr_K_freq_line}. 


%The measured frequency of each conversion line is expected to be related to its energy according to \autoref{eq:energytofrequency}.
The energy of each conversion line is calculated in \cite{V_nos_2018} using the 32-keV gamma energy, as well as a binding energy and recoil energy specific to that line (shown in \autoref{table:Kr energy lines Venos}).  
%The mean magnetic field $B$ is a fit parameter. 
\autoref{fig:linearity} shows the fitted frequency-energy relation with the mean magnetic field $B$ as free parameter.
The magnetic field found in the fit is $B = 0.959023787(42)$\,T.  Note that this does not include the uncertainty from the gamma energy scale of 0.5\,eV. The points in the residual plot below the figure illustrate the good internal agreement of the data with the  equation. The conversion line energies are calculated by fixing the gamma energy at the literature value of  32151.6 eV provided in \cite{V_nos_2018}. An improvement in the gamma energy measurement is planned by the KATRIN collaboration~\cite{Rodenbeck:2022fxc} and could also be made via CRES with a precise independent determination of the magnetic field by NMR.


\subsection{\texorpdfstring{$^{\mathrm{83m}}$Kr}{} fit procedure with CRES spectrum model}
\label{subsec:Kr Line model}
For the remainder of this paper, all $^{\mathrm{83m}}$Kr fits use the CRES spectrum model in \autoref{eq:FullModel1} to fit the 17.8-keV conversion-electron line. Since the structure of this CRES spectrum model is common between $^{\mathrm{83m}}$Kr and tritium data, we can use $^{\mathrm{83m}}$Kr fits to calibrate the tritium energy point-spread function $\mathcal{R}_{\mathrm{PSF}}$ and detection efficiency curve $\epsilon$. The 17.8-keV $^{\mathrm{83m}}$Kr line is a powerful tool, given its narrow (2.774-eV) natural line width~\cite{Altenmuller:2019ddl}, well understood shape, and closeness to the tritium endpoint at 18.6 keV. 
The underlying spectrum $\mathcal{Y}_{\mathrm{Kr}}$ includes the 17.8-keV $^{\mathrm{83m}}$Kr  main peak with its natural line width, as well as a lower-energy satellite structure from shakeup and shakeoff \cite{Robertson:2020boa}.

In these fits, the magnetic field $B$ and scattering parameters $p$ and $q$ are left free. For the fit to the deep quad trap data, the scatter fractions $\gamma_i$ are inputted, while for the shallow trap data, the scatter fractions for  H$_2$ and He are extracted from the fit, as motivated in \autoref{L_model}.
In the final fits, the detection efficiency variation with frequency $\epsilon$ (as determined in \autoref{subsec:efficiency}) is included in the model.
No background component is included in the fits due to the short run durations and negligible expected background rate. Section~\ref{subsec:electron_data} explains the reasons for expecting negligible background, and \autoref{subsec:background_limit} confirms this assumption.

Numerical scatter peaks serve as fixed inputs to the fitting function. These scatter peaks are produced by convolving data-set-specific simulated instrumental resolutions $\mathcal{I}$ (see \autoref{subsec:ins_res}) with electron energy loss spectra $\mathcal{L}_{\mathrm{tot}}$. To determine $\mathcal{L}_{\mathrm{tot}}$, loss spectra are combined according to \autoref{eq:combine_peaks_from_different_gases}, accounting for $\mathcal{L}_r$ and scattering from gases present in $^{\mathrm{83m}}$Kr data: Kr, $^3$He, Ar, and H$_2$ and its isotopologues.

%Many bins in the $^{\mathrm{83m}}$Kr spectra have expected count rates of $<5$, violating the conditions needed for the standard chi-square statistic, which assumes Gaussian statistics. 
Fits to $^{\mathrm{83m}}$Kr data are performed by minimizing a Poisson likelihood chi-squared \cite{Baker:ChiSquare1984},
\begin{align}
    \chi^2_{\lambda,p} = 2 \sum \big[y_i - n_i + n_i\ln{(n_i/y_i)}\big]\,,
\end{align}
where $y_i$ is the expected number of events in bin $i$ according to Eqs.~\ref{eq:FullModel1} and \ref{eq:FullModel2}, and $n_i$ is the measured  number of events in that bin. Because the spectra contain many bins with zero or few counts, $\chi^2$/DOF is not an optimal figure of merit. Instead, goodness-of-fit testing is performed  as suggested in~\cite{Baker:ChiSquare1984}. Treating the fitted spectrum as the truth, the Poisson $\chi^2_{\lambda,p}$ is sampled by Monte-Carlo, and the distribution of Poisson $\chi^2_{\lambda,p}$ is compared to the $\chi^2_{\lambda,p}$ for the data.

\begin{figure}[htb]
    \centering
    \includegraphics[width=1\columnwidth]{Plots/3-systematic-uncertainties/3-2-instrumental-resolution/october_max_snr_optimization_plot_weights.pdf}
    \caption{SNR$_{\rm max}$ optimization for $\mathrm{^{83m}Kr}$ pre-tritium data. The vertical axis is the scale factor $s$ that adjusts the simulated width of $\mathcal{I}$ to match the experimental one for each choice of SNR$_{\rm max}$ in the simulation.}
    \label{fig:max_snr_optimization_oct_data}
\end{figure}

\subsection{Instrumental resolution \texorpdfstring{$\mathcal{I}$}{}}\label{subsec:ins_res}
The instrumental resolution $\mathcal{I}$ is an input to $^{\mathrm{83m}}$Kr fits. 
%This resolution describes the broadening of cyclotron frequency due to the dependence on pitch angle and radius of the average magnetic field seen by a trapped monoenergetic electron. 
$\mathcal{I}$  is determined for each trap configuration by simulation 
%with Project 8's Locust software package \cite{AshtariEsfahani:2019mwv} 
as described in \autoref{sec:gen_resolution}.
%Equations~\ref{eq:FullModel1} and \ref{eq:FullModel2} show the role of $\mathcal{I}$ in the CRES spectrum model used for  $^{\mathrm{83m}}$Kr fits.


\subsubsection{SNR scaling optimization}\label{sec:max-SNR-optimization}
Each $\mathcal{I}$ distribution has an associated value of SNR$_{\rm max}$, the SNR of a $90^{\circ}$ electron in trap 3 at $r=0\,\si{mm}$. SNR$_{\rm max}$ mostly affects the width of $\mathcal{I}$ while maintaining the distribution's overall shape. In particular, a higher SNR$_{\rm max}$ corresponds to a wider $\mathcal{I}$ distribution because the overall SNR in track bins is higher, making electrons with smaller pitch angles more detectable. These small-pitch-angle electrons explore a larger range of magnetic fields, broadening the detected frequency spectrum.

The total system gain and noise temperature are not known well enough for each $^{\mathrm{83m}}$Kr data set to determine SNR$_{\rm max}$.
%As a result, the width of $\mathcal{I}$ is not exactly known.
Instead, to estimate SNR$_{\rm max}$, 60 $^{\mathrm{83m}}$Kr fits are performed using inputted $\mathcal{I}$ distributions corresponding to SNR$_{\rm max}$ values ranging from 12 to 18.  
%When simulating these $\mathcal{I}$ distributions, the SNR of all events is scaled relative to SNR$_{\rm max}$ before the efficiency filter is applied.  
We add a fit parameter to the $^{\mathrm{83m}}$Kr model: a scale factor $s$, which widens or compresses $\mathcal{I}$ during the fit. When $s=1$, this indicates that $\mathcal{I}$ has the best width to describe the data, and thus the best SNR scale. We fit the 60 (SNR$_{\rm max}$, $s$) points to a quadratic function and predict the SNR$_{\rm max}$ for $s=1$. This procedure anchors SNR$_{\rm max}$ to experimental data. It also produces a best-estimate for the standard deviation of $\mathcal{I}$ for each $^{\mathrm{83m}}$Kr data set.


For pre-tritium $^{\mathrm{83m}}$Kr data, the outcome of the SNR scaling procedure (SNR$_{\rm max}$ = 14.3) is shown in \autoref{fig:max_snr_optimization_oct_data}, and the resulting  $\mathcal{I}$ is displayed in \autoref{fig:kr_ftc_res}. To simulate $\mathcal{I}$ for the tritium analysis, we use the same SNR$_{\rm max}$ value as in the pre-tritium 
$^{\mathrm{83m}}$Kr data, since its properties most closely resemble those of tritium data (see \autoref{tab:mml_track_length_fit_results}). The two data sets are only distinguishable in track duration, which has a sub-dominant effect on the width of $\mathcal{I}$.


% \begin{figure}
% \centering   
% \includegraphics[width=\columnwidth]{Plots/2-models/Kr_res_pre-tritium_errorbars.pdf}
% \caption{Instrumental resolution $\mathcal{I}$ of the pre-tritium $^{\mathrm{83m}}$Kr calibration data set. The error bars include uncertainties from Poisson counting, the efficiency matrix, and the trap weights.}
% \label{fig:kr_ftc_res}
% \end{figure}

\subsubsection{Uncertainties on \texorpdfstring{$\mathcal{I}$}{} propagated to \texorpdfstring{$^{\mathrm{83m}}$Kr}{} fit results}\label{sec:res-errors-to-Kr}
\label{sec:instrumental_resolution_uncertainty_propagation}

A simulated, fixed $\mathcal{I}$ distribution is inputted to each $\mathrm{^{83m}Kr}$ K-line fit listed in \autoref{tab:mml_track_length_fit_results}. As a result, uncertainties on $\mathcal{I}$ propagate to the fit parameter results ($p$, $q$ and $B$), which in turn feed into the tritium analysis.
%(see Secs.~\ref{sec:Bfield_errors} and~\ref{sec:scatter_peak_errors}). 
Thus, we estimate the uncertainties in $p$, $q$ and $B$ due to both $\mathcal{I}$ simulation uncertainties and SNR$_{\rm max}$ uncertainties.

For each data set, $\mathcal{I}$  simulation uncertainties are obtained from 100 bootstrapped resolution shapes, which are produced by repeatedly sampling counts in all bins from Gaussian distributions. Each Gaussian's standard deviation equals the bin simulation uncertainty, which includes uncertainties from Poisson counting, the efficiency matrix, and the trap weights. $\mathrm{^{83m}Kr}$ K-line fits are then repeated 100 times, once with each bootstrapped $\mathcal{I}$ as input, to obtain uncertainty distributions for fit parameters. 
Separately, we estimate the SNR$_{\rm max}$ contribution to $\mathrm{^{83m}Kr}$ fit parameter uncertainties. To do so, we fit the data 100 times, each time using an inputted resolution simulated with a different SNR$_{\rm max}$ value sampled from a normal distribution (with a mean from the procedure in \autoref{sec:max-SNR-optimization} and an uncertainty calculated as described in \autoref{sec:sigma_errors}). 
Simulation and SNR$_{\rm max}$ uncertainties are added in quadrature.

\begin{figure}[b]
  \centering
  \includegraphics[width=1.0\columnwidth]{Plots/3-systematic-uncertainties/3-3-efficiency/fss_in_q300.pdf}
  \caption{
    The 17.8-keV $\mathrm{^{83m}Kr}$ conversion electron line recorded in the deep quad trap at different magnetic background fields (red / blue).
  }
  \label{fig:q300_fss}
\end{figure}

\begin{figure}
%\centering   
\includegraphics[width=1.0\columnwidth]{Plots/3-systematic-uncertainties/3-3-efficiency/all_trap_rates_with_residual_plot.pdf}
\caption{Event detection rates from $^{\mathrm{83m}}$Kr K-line data recorded with different single-coil traps (red and blue) and the quad trap (black) in field-shifted data relative to the respective count rate at $f_c \approx 25.91\,\si{GHz}$ (where $B=B_0$). The uncertainties from interpolation are shown in grey. The relative count rates can be summed with weights (green) to match the quad-trap count rate curve (black). The residuals show the differences of the summed single-trap rates and the quad-trap rates divided by the quad-trap count rate uncertainties. The standard deviation of the residuals is larger than 1 and the uncertainties on the tritium efficiency $\epsilon$ (\autoref{sec:energy_correction}) are inflated to account for this.}
\label{fig:fss_event_rates}
\end{figure}


\subsection{Field-shifted \texorpdfstring{$\mathrm{^{83m}Kr}$}{} data analysis}
\label{subsec:frequency_dependence}
 
\subsubsection{Measurement of detection efficiency vs. frequency}
\label{subsec:efficiency}
Detection efficiency as a function of frequency is an input to the CRES spectrum model. To study the frequency response, we recorded $\mathrm{^{83m}Kr}$ data at a range of background magnetic field values, as described in \autoref{fss_procedure} and in the ``$\mathrm{^{83m}Kr}$ field-shifted'' row in \autoref{tab:mml_track_length_fit_results}. Data were taken in the full quad trap configuration as well as in each individual trapping coil in isolation.
A subset of the $\mathrm{^{83m}Kr}$ K-line data recorded in the quad trap configuration is shown in \autoref{fig:q300_fss}. 
To measure the detection efficiency vs.~frequency, we extracted $\epsilon(f_c)$ at the frequency center of each recorded peak by fitting the data with a reduced version of the full CRES spectrum model that does not include $\epsilon(f_c)$. The number of reconstructed events within $\pm$1~$\si{MHz}$ of the fitted peak's frequency location is compared to the number of events for the data at the unshifted background field ($B=B_0$).
The motivation for the start-frequency cut of $\pm1\,\si{MHz}$ around the peak center is to not average the detection efficiency over a larger frequency range while maintaining a sufficiently high statistical precision for the efficiency analysis.
The obtained relative count rate vs.~frequency in a given trap (shown in \autoref{fig:fss_event_rates}) is equivalent to the relative $\epsilon(f_c)$ in this trap for (quasi) mono-energetic data like the $\mathrm{^{83m}Kr}$ K-line (the energy spread of K-line electrons is small compared to the resolution width $\mathcal{I}$). For tritium data analysis in the quad trap, $\epsilon$ is summed from the single-trap count rates after a correction for the dependence of SNR on kinetic energy. We motivate and describe this correction in \autoref{sec:efficiency_for_tritium}. 


\subsubsection{Extraction of statistical trap weights}
\label{subsec:trap_weights}
The statistical trap weights $w_i$ correspond to the relative number of detected events in each trap. These weights are used for two purposes: to sum the simulated instrumental resolutions of the 4 traps that compose the quad trap, and to correct the measured efficiency variation with frequency for the tritium analysis as will be discussed in \autoref{sec:energy_correction}.
We extract the $w_i$ at $B=B_0$ from the field-shifted data by minimizing the summed squared differences between the quad trap count rates vs.~frequency and the weighted sum of the single-trap count rates vs.~frequency, with $w_i$ being free parameters.
The resulting weights are $w_1 = 0.076(3)$, $w_2 = 0.341(13)$, $w_3 = 0.381(14)$, and $w_4 = 0.203(20)$, which are in good agreement with the observed count rate differences at $B=B_0$.

Note that the field step sizes are \SI{0.07}{\milli\tesla} in traps 2, 3, and the quad trap. We chose the step sizes in trap 1 and trap 4 to be \SI{0.7}{\milli\tesla} to reduce the total duration of these field-shifting scans for the traps with the  lowest count rate at the nominal frequency position of the K-line ($\approx 25.91\,\si{GHz}$).
For the summation, the count rates from trap 1 and 4 are interpolated linearly.
The uncertainties in the interpolated frequency ranges are taken to be equal to the largest deviation from a linear interpolation over the same range in trap 2 or 3 (shown in grey in \autoref{fig:fss_event_rates}). 



\subsection{\texorpdfstring{$^{\mathrm{83m}}$Kr}{} shallow trap data and fits \label{sec:shallow_trap}}

To explore the best resolution achievable in Phase II, and to test the CRES spectrum model (\autoref{eq:FullModel1}), we took $^{\mathrm{83m}}$Kr data with the trap coil currents set to the shallow trap configuration in \autoref{fig:quadtrapcoils}.
Figure~\ref{fig:krypton shallow trap} shows the fit to these data.
Also shown is the underlying $^{\mathrm{83m}}$Kr lineshape model $\mathcal{Y}_{\mathrm{Kr}}$, which includes both the main peak and the shakeup/shakeoff satellites. 
The figure displays intermediate lineshapes in which contributions to the model are included one by one, to exhibit the effects of magnetic field inhomogeneity (treated as equivalent to instrumental resolution $\mathcal{I}$) and scattering. 
In the shallow trap, there are only small differences between the average magnetic fields experienced by trapped electrons with different pitch angles. Accordingly, the broadening from $\mathcal{I}$ (included in the purple curve) is $1.66(19)$\,eV FWHM.
This combines with the natural linewidth of 2.774\,eV FWHM~\cite{Altenmuller:2019ddl} to produce a main peak with a FWHM of 4.0\,eV.  
%The Gaussian component of the Voigt profile used in \autoref{fig:linearity_Kr_K_freq_line} has a FWHM of $2.10(6)$\,eV, in reasonable agreement given that $\mathcal{I}$ is not Gaussian. 
Out of all events, 69\% are detected before scattering. Additional curves in \autoref{fig:krypton shallow trap} show events detected after a single scatter and after up to 20 scatters.  In the low-energy tail (below 17.814 eV), scattering events comprise 61\% of counts.

The summed $\chi^2_{\lambda,p}$ of the binned data (631) falls within 1$\sigma$ of the mean of the distribution of summed $\chi^2_{\lambda,p}$ values from MC simulations (607$\pm$ 40), verifying goodness of fit. This demonstrates the high-resolution capabilities of CRES and validates the $^{\mathrm{83m}}$Kr model. 


\begin{figure}[htb]
  \centering
\includegraphics[width=1.0\columnwidth]{Plots/2-models/shallow_trap_data_and_fit_with_residuals_bands_and_extras.pdf} \caption{The 17.8-keV $^{\mathrm{83m}}$Kr K-conversion electron line, as measured with CRES in the shallow (high-resolution) electron trapping configuration, with FWHM of 4.0\,eV. The data are the $^{83\mathrm{m}}$Kr shallow data set (Table \ref{tab:mml_track_length_fit_results}).
  }
  
  \label{fig:krypton shallow trap}
\end{figure}

\subsection{\texorpdfstring{$^{\mathrm{83m}}$Kr}{} pre-tritium and post-tritium quad trap data and fits\label{sec:deep_trap}}\label{sec:deep-trap-data-and-fits}

%For the small apparatus in Project 8's Phase II, low statistics are the limiting factor in neutrino mass sensitivity. Thus, to maximize statistics, we took the final tritium data in a trap configuration, shown in \autoref{fig:quadtrapcoils}, that was deeper than the shallow trap and was composed of four individual traps rather than two. 
The $^{\mathrm{83m}}$Kr ``pre-tritium" and ``post-tritium" data sets (see \autoref{tab:mml_track_length_fit_results}) were taken in the same deep quad trap as tritium data, to calibrate the mean  field $B$ and scattering parameters $p$ and $q$ for the tritium analysis. \autoref{fig:krypton deep trap} shows the $^{\mathrm{83m}}$Kr pre-tritium data and fit. The $^{\mathrm{83m}}$Kr line shape is significantly broadened by the 35.6\,eV FWHM instrumental resolution $\mathcal{I}$ (\autoref{fig:kr_ftc_res}), due to the large range of average magnetic fields experienced by electrons. 

\begin{figure}[t]
  \centering
  \includegraphics[width=1.0\columnwidth]{Plots/2-models/deep_trap_data_and_fit_with_residuals_bands_and_extras.pdf}
  \caption{
    The 17.8-keV $^{\mathrm{83m}}$Kr K-conversion electron line, as measured with CRES in the deep (high-statistics) electron trapping configuration, with FWHM of 54.3\,eV. The data are the $^{83\mathrm{m}}$Kr pre-tritium data set (Table \ref{tab:mml_track_length_fit_results}).
  }
  \label{fig:krypton deep trap}
\end{figure}


Compared with the shallow trap, the larger pitch angle acceptance in the deep trap causes more electrons to remain trapped after scattering, leading to a higher average number of tracks per event. This also leads to a smaller proportion (53\%) of events being detected before scattering, since events that begin in non-detectable pitch angles have a larger phase space of detectable pitch angles to scatter into.
%; therefore, detected events are more likely to come from this pool.
%Therefore, a smaller proportion (53\%) of events are detected before scattering in the deep quad trap than in the shallow trap. 
This gives rise to the enhanced low-energy tail and brings the FWHM to 54.3\,eV. 
%Note that in the shallow trap spectrum, the scatter peaks contribute minimally to the width, unlike in the deep trap spectrum.
Note that here the scatter peaks merge with the instrumental resolution into a single broad peak. In the deep trap, the FWHM therefore contains both events detected before scattering and events first detected after scattering. 
%while for the deep quad trap data, the scattering accounts for a significant portion of the width.

For the $^{\mathrm{83m}}$Kr pre-tritium and post-tritium data sets, the comparison of the summed $\chi^2_{\lambda,p}$ for binned data (1211 and 1112, respectively) with the distributions of MC-simulated summed $\chi^2_{\lambda,p}$ values (884$\pm$38, 830$\pm$59) indicated underfitting. This tension likely stems from small imperfections in the simulated instrumental resolution, relative to data. 
To account for the uncertainty associated with this tension, the uncertainties for $B$, $p$ and $q$ from the maximum likelihood fit are inflated by 17\% and 5\% for the pre-tritium and post-tritium data sets, respectively.
These fit uncertainties are combined with the larger uncertainty contributions from $\mathcal{I}$ and gas composition to produce the total uncertainties on $B$, $p$ and $q$.\footnote{When fit uncertainties are inflated to account for underfitting, this increases the total uncertainties on $B$, $p$ and $q$ by only 1.6\%, 0.3\% and 0.1\%, respectively.} Uncertainties on $\mathcal{I}$ are propagated using the sampling-and-refitting method described in \autoref{sec:res-errors-to-Kr}. The uncertainty on $\mathcal{I}$ due to SNR$_{\rm max}$ is not propagated to $B$, since those variables are independent. SNR$_{\rm max}$ primarily affects the width of $\mathcal{I}$, while $B$ controls the location of the distribution's center; these are two separate moments of $\mathcal{I}$.
To propagate the uncertainty from gas composition, the $^{\mathrm{83m}}$Kr fits are repeated 300 times while sampling the inputted gas scattering contributions from the distributions defined in \autoref{tab:scattering_fraction_results}. The gas composition uncertainties on $B$, $p$ and $q$ are the standard deviations of results from these 300 fits. 

With fit, $\mathcal{I}$, and gas composition uncertainties included, the best estimates of $B$ from $^{\mathrm{83m}}$Kr pre- and post-tritium data differ by 1.6\,$\sigma$. Estimates for $p$ and $q$ are not expected to be consistent between the two quad trap data sets, due to a difference in the mean number of tracks per event (see \autoref{sec:scatter_peak_errors}).


\section{Tritium models \label{sec:tritium_model}}

In this paper, we employ two models of tritium CRES data: a highly detailed model for data generation in Monte Carlo (MC) studies, and a simplified, analytic model for analysis.
%The simplified model is used for both Bayesian and frequentist inference.
In the detailed generation model, the beta spectrum function is numerically convolved with the energy point-spread function $\mathcal{R}_{\mathrm{PSF}}$, and no approximations are made to either function. In the analysis model, several approximations are made for computational efficiency. MC studies demonstrate that each of these approximations do not affect endpoint ($E_0$) and neutrino mass ($m_{\mathrm{\beta}}$) results, as discussed in \autoref{sec:MCstudies}.  The remainder of this section describes the tritium data generation and analysis models.



\subsection{Detailed tritium model for MC data generation}\label{sec:detailed_gen_model}
%The generation model is highly detailed, to verify that analysis-model approximations have a negligible effect on $E_0$ and $m_\beta$ results.
For tritium data, the underlying spectrum in \autoref{eq:FullModel1} is the beta spectrum $\mathcal{Y}_{\mathrm{tritium}}$, given by the product of neutrino and electron phase space density factors $D_{\nu}$ and $D_e$.
%Equation~\ref{eq:FullModel1} shows how this underlying spectrum is used in the full model for CRES data. 
When experimental sensitivity is insufficient to resolve individual mass eigenstates, the beta spectrum for molecular tritium is given by~\cite{Formaggio:2021nfz} %neutrino phase space density
\begin{equation}
\begin{split}
    &\mathcal{Y}_{\mathrm{tritium}} = D_{\nu} \cdot D_e, \quad \mathrm{where} \\
    &D_{\nu} \propto  \epsilon_\nu\big[\epsilon_\nu^2 -  m_\beta^2\big]^{1/2} \Theta(\epsilon_\nu-m_\beta) \\
    &D_e \propto F(Z, p_\mathrm{e})p_\mathrm{e} E_\mathrm{e}.
    \label{eq:betaspectrumfull}
\end{split}
\end{equation}
In this equation, $\epsilon_\nu = E_0 - V_k - E_\mathrm{kin}$, where $V_k$ is the energy supplied to rotational, vibrational, and electronic excitations of $^3$HeT$^+$ during the decay \cite{Bodine:2015sma}. $\Theta$ is the Heaviside step function,  $F(Z, p_\mathrm{e})$ is the relativistic Fermi function for charge $Z=2$ of the daughter nucleus, and $p_\mathrm{e}$ and $E_\mathrm{e}$ are the electron momentum and total energy, respectively. The MC data generation model uses \autoref{eq:betaspectrumfull}, including all atomic physics corrections to the Fermi function from~\cite{Kleesiek:2018mel}. We numerically convolve \autoref{eq:betaspectrumfull} with the final state distribution for $^3$HeT$^+$, which is the probability distribution of $V_k$ values. We use the final state distribution calculated by Saenz et al.~\cite{Saenz:2000dul} down to a binding energy of $-2288\,$eV.
%. This distribution describes the probability

%The procedure for simulating $\mathcal{I}$ is described in \autoref{sec:gen_resolution}.
The $\mathcal{R}_{\mathrm{PSF}}$ model includes a simulated, tritium-specific instrumental resolution $\mathcal{I}$,
which is numerically convolved (according to \autoref{eq:FullModel2}-\ref{eq:combine_peaks_from_different_gases}) with inelastic scatter spectra calculated from~\cite{LXCat_carbone2021data,pitchford2017lxcat,pancheshnyi2012lxcat,LXCat:database}.
%to form broadened scatter peaks. 
%The convolutions account for cases when the same electron successively scatters off of different gases (neglected in the analysis model).
 We account for rare scatters with CO
 %which comprise $\approx1$\% of inelastic scatters,
 during generation but not analysis. Twenty scatter peaks are generated ($j_{\mathrm{max}}=20$). An MC study shows that including higher-order peaks in the generation and/or analysis models does not alter results. $\mathcal{R}_{\mathrm{PSF}}$ is numerically convolved with $\mathcal{Y}_{\mathrm{tritium}}$.
 % have no noticeable effect on results confirms that including more peaks in the data generation 

\subsection{Approximate tritium model for analysis}
\label{subsec:tritium_analysis_approximations}
The tritium analysis model is used to fit both the tritium spectrum obtained from the apparatus and Monte Carlo spectra. The analysis model includes approximate, analytic expressions for both  $\mathcal{Y}_{\mathrm{tritium}}$ and $\mathcal{R}_{\mathrm{PSF}}$, enabling computationally efficient inference. This is crucial for the Bayesian analysis, since algorithms that perform Bayesian inference in many dimensions---corresponding to many nuisance parameters---tend to be slow at numerical integration. The analytic model of $\mathcal{R}_{\mathrm{PSF}}$ also substantially speeds up the calculation of the response in the frequentist analysis.

For each model approximation described below, a Monte Carlo test is performed by generating an ensemble of spectra with the detailed tritium model, then analyzing those spectra using a model which includes the approximation. These studies show that $E_0$ and $m_\beta$ fit results are unaffected by each approximation, for Phase II data. Even for the resolution and statistics expected in Project 8's final planned phase, the simplified model of $\mathcal{Y}_{\mathrm{tritium}}$ was shown to yield accurate results~\cite{AshtariEsfahani:2021moh}. Some simplifications of $\mathcal{R}_{\mathrm{PSF}}$  may not hold in future experiments. We will refine the $\mathcal{R}_{\mathrm{PSF}}$ model and incorporate numerical components, as needed (which may be practical for Bayesian inference with state-of-the-art tools). 

\subsubsection{Tritium beta decay analysis model\label{sec:T2-beta-model}}
The frequentist analysis model uses \autoref{eq:betaspectrumfull} for the underlying beta spectrum $\mathcal{Y}_{\mathrm{tritium}}$. In the Bayesian analysis, $D_\nu$ is approximated according to the formalism in~\cite{AshtariEsfahani:2021moh}. This involves Taylor expanding in $m_\beta^2$ to produce the expression
\begin{equation}
    D_{\nu} \approx  \big[\epsilon_\nu^2 -  m_\beta^2/2\big] \Theta(\epsilon_\nu-m_\beta).
    \label{eq:betaspectrum}
\end{equation}
In addition, $D_e$ is Taylor expanded to first order around the energy at the center of the  analysis region of interest (ROI), neglecting atomic physics factors that correct the Fermi function. 
%These are good approximations for the Phase II event rate and ROI.
The resulting model for $\mathcal{Y}_{\mathrm{tritium}}$ may be analytically convolved with a normal distribution. Thus, for any model of $\mathcal{R}_{\mathrm{PSF}}$ that is expressed as a weighted sum of Gaussians, the full tritium model is analytic.
%This enables computationally efficient inference, as was discussed in \autoref{sec:T2-beta-model-context}.
Here, unlike in Ref.~\cite{AshtariEsfahani:2021moh}, the low-energy edge of the spectrum is not smeared out by magnetic field broadening, since a hard maximum-frequency cut is performed before analysis.\footnote{A low-energy smearing of 0.001\,eV is included in the model for computational stability, negligibly affecting results. In Bayesian MCMC inference, infinitely steep drops in probability density can cause Markov chains to behave pathologically~\cite{Stanual2022, Betancourt2015}.}   

If the final state distribution of $^3$HeT$^+$ were neglected in the analysis model, this would bias the $E_0$ result by \SI{-8.1 \pm 0.8}{eV}, as computed from a Bayesian MC study. 
To speed up computation, the frequentist and Bayesian models use a sparse approximation of the  final state distribution down to 240 eV below the endpoint. The sparse distribution uses only every 4th excitation energy and associated probability from~\cite{Saenz:2000dul}.
%(\autoref{fig:MFS}) %This saves computational time and
%, as determined by an MC study,
%does not introduce bias in the results. 
%To discretely convolve this distribution with the beta spectrum, the full signal model is a weighted sum of  signal functions with different $V_k$, and weights given by the final state probabilities.
%given by the energies of the red points
%This serves to discretely convolve the final state distribution with the beta spectrum.
%We neglect highly-suppressed contributions from the atomic tail beyond 240 eV below the endpoint.
%as they are highly suppressed.
% and have no measurable effect on the results
While some electrons are produced by HT, which decays to $^3$HeH$^+$, this molecule's final state distribution is similar to that of $^3$HeT$^+$, compared with our resolution. These simplifications introduce no biases in the results.

%so the difference can be ignored for the energy width of $\mathcal{R}_{\mathrm{PSF}}$ in our deep quad trap. 
%energy point-spread function 

%\begin{figure}[!htb]
%    \centering
%    \includegraphics[width = \columnwidth]{Plots/2-models/Mfs_spectrum_sparse_tail_prc-d.pdf}
%    \caption{Molecular final states of $^3$HeT$^+$ as calculated in \cite{Saenz:2000dul} and the sparse approximation used in this analysis.}
%    \label{fig:MFS}
%\end{figure}

\subsubsection{Tritium energy response function analysis model}\label{sec:T2-det-response}
%The $\mathrm{\beta}$-spectrum is convolved with $\mathcal{R}_{\mathrm{PSF}}$, the energy response point-spread function. 
For tritium analysis, the energy response point-spread function $\mathcal{R}_{\mathrm{PSF}}$ is modeled as a sum of Gaussians.
%, each corresponding to one scatter peak $\mathcal{I}*\mathcal{L}_{\mathrm{tot}}^{*j}$.
This allows $\mathcal{R}_{\mathrm{PSF}}$ to be analytically convolved with the beta spectrum, simplifying computation.
%as motivated in \autoref{sec:T2-beta-model}. %As in the case of $^{\mathrm{83m}}$Kr data, $\mathcal{R}_{\mathrm{PSF}}$ is composed of scatter peaks $\mathcal{I}*\mathcal{L}_{\mathrm{tot}}^{*j}$.

%MC studies (with $\mathcal{L}_r$ included in the generation model) show that neglecting $\mathcal{L}_r$ during analysis has no measurable effect on tritium results. 
Within $\mathcal{R}_{\mathrm{PSF}}$, the energy loss function $\mathcal{L}_{\mathrm{tot}}$ accounts for scattering with H$_2$ and $^3$He, as these have the largest inelastic scatter fractions $\gamma_i$ in tritium data: \SI{0.911 \pm 0.045} and \SI{0.075 \pm 0.040}, respectively. The scatter fraction \SI{0.014 \pm 0.009} of CO is omitted from the tritium fit model. We further simplify by modeling each scatter peak as a weighted sum of H$_2$ and $^3$He peaks. This is akin to assuming that a given electron scatters with the same gas type after all missed tracks. The radiative loss $\mathcal{L}_r$ is omitted from the model, since it is small relative to scattering losses. MC studies validate these simplifications. 
%confirmed that these three simplifications do not alter $E_0$ and $m_\beta$ results.  

Each scatter peak $\mathcal{I}*\mathcal{L}_{\mathrm{tot}}^{*j}$
is expressed as a function of $\sigma$, the calculated standard deviation of the simulated resolution $\mathcal{I}$. This enables us to propagate uncertainty on the resolution width to the endpoint and neutrino mass, via $\sigma$. The $j=0$ peak is modeled separately from $j\geq1$ peaks (up to $j_{\mathrm{max}}=20$), as described below.
%, since an MC study shows that higher-order peaks have no noticeable effect on results.

The $j=0$ term reduces to $\mathcal{I}$.  Because this term is the dominant contribution to $\mathcal{R}_{\mathrm{PSF}}$, it is important to model the peak with a closely-fitting distribution. A simple Gaussian would underestimate the tails of $\mathcal{I}$ and fail to account for its small asymmetry. Instead, the simulated $\mathcal{I}$ is fitted with a sum of two normal distributions $\mathcal{N}$ with means $\mu_0^{[i]}$ and standard deviations $\sigma_0^{[i]}$ ($i=1,2$), weighted by a parameter $0\le\eta\le 1$:
 
\begin{equation}
\begin{split}
\mathcal{I}*\mathcal{L}_{\mathrm{tot}}^{*0} \approx \eta\mathcal{N}\Big(\mu_0^{[1]}, \sigma_0^{[1]}(\sigma)\Big) +(1-\eta)\mathcal{N}\Big(\mu_0^{[2]}, \sigma_0^{[2]}(\sigma)\Big).
\end{split}
\label{eq:T2_ins_res_model}
\end{equation}
 
%Since each scatter peak is parameterized in terms of $\sigma$, the Gaussian standard deviations in \autoref{eq:T2_ins_res_model} must be functions of $\sigma$. 

To find how $\sigma_0^{[1]}$ and $\sigma_0^{[2]}$ depend on $\sigma$, we perform fits to simulated resolutions with a range of $\sigma$ values (see Appendix~\ref{sec:simplified_res_sims}). 
%To vary $\sigma$ in the simulations, the value of SNR$_{\rm max}$ is varied. As discussed in \autoref{sec:max-SNR-optimization}, $\sigma$ is determined primarily by SNR$_{\rm max}$.  
The result of this procedure is $\sigma_0^{[1]}(\sigma) = 1.1\sigma + 1.9\,$eV and $\sigma_0^{[2]}(\sigma) = 0.8\sigma - 3.7\,$eV.
%\footnote{To determine these expressions for $\sigma_0^{[1]}$ and $\sigma_0^{[2]}$, we use three pieces of information:
%\begin{enumerate}
%\item Simulations show that $\sigma_0^{[2]} = 0.8\sigma_0^{[1]} - 5.2\,$eV for our instrumental resolution. We observe this linear relation by generating 100 simulated resolutions with different maximum detectable track SNRs (SNR$_{\rm max}$), then fitting each resolution with \autoref{eq:T2_ins_res_model}. The size of SNR$_{\rm max}$ controls the width of $\mathcal{I}$ (see \autoref{subsec:ins_res}). In these fits, the Gaussian means, Gaussian standard deviations and $\eta$ are all fitted.
%\item We observe that $\sigma \approx \eta\sigma_0^{[1]} + (1-\eta)\sigma_0^{[2]}$ for the same 100 simulated resolutions. The relation is approximate because an exact expression for $\sigma$ must depend on $\mu_0^{[1]}$ and $\mu_0^{[2]}$. In that relation, the right hand side is computed from fit results, and $\sigma$ is calculated directly from the simulated resolutions. 
%\item We observe that $\eta=0.66$ is constant within fit errors, for the SNR$_{\rm max}$ uncertainty range in tritium data (quantified in \autoref{sec:sigma_errors}).
%\end{enumerate}
%Combining the first and second relations, and fixing $\eta$, we find $\sigma_0^{[1]}(\sigma)$ and $\sigma_0^{[2]}(\sigma)$---expressions that depend only on $\sigma$ and constants.} 
This procedure also demonstrates that $\eta=0.66$ is constant as $\sigma$ varies. By fixing $\eta$ and plugging the expressions for $\sigma_0^{[1]}(\sigma)$ and $\sigma_0^{[2]}(\sigma)$ into \autoref{eq:T2_ins_res_model}, we obtain a model for $\mathcal{I}$ with only three free parameters: $\sigma$, $\mu_0^{[1]}$ and $\mu_0^{[2]}$.
This ``reduced model" 
%fits the simulated resolution closely. The reduced model
%for the $j=0$ peak
can scale in width based on $\sigma$ and captures the slight asymmetry in $\mathcal{I}$ via $\mu_0^{[1]}$ and $\mu_0^{[2]}$.
We confirm that the fitted value of $\sigma$ matches the standard deviation $\sigma$ calculated directly from $\mathcal{I}$. 

\begin{figure}[t]
  %\centering
  \includegraphics[width=1.0\columnwidth]{Plots/2-models/ins_res_sum_of_gaussians_different_means_maxSNR-14-3_official7.pdf}
  \caption{
   Fit of \autoref{eq:T2_ins_res_model} to the simulated resolution $\mathcal{I}$ for tritium data. Bin errors are approximately Gaussian, from sources described in \autoref{subsec:ins_res}. Fit parameters are means $\mu_0^{[1]}$, $\mu_0^{[2]}$ of the two normal distributions and standard deviation $\sigma$ of $\mathcal{I}$. The fraction of counts $\eta$ in the first normal distribution is inputted, determined by computing the average $\eta$ from fits to 100 resolutions with different $\sigma$ values.}
  \label{fig:T2_res_fit}
\end{figure}


We find the best estimates of $\sigma$, $\mu_0^{[1]}$ and $\mu_0^{[2]}$ by fitting the reduced model to the simulated $\mathcal{I}$ distribution that was generated with the best-estimate SNR$_{\rm max}$ value, as shown in \autoref{fig:T2_res_fit}. The $\chi^2/\mathrm{ndf}$ is $44/58$, and the fit result for $\sigma$ is consistent with the standard deviation calculated directly from $\mathcal{I}$. %\autoref{fig:T2_res_fit} shows this fit ($\chi^2/\mathrm{ndf} = 44/58$). 
The fit energy range is limited to produce a good fit to the central region of $\mathcal{I}$, which has the largest impact on tritium data fits. During tritium analysis, uncertainties are propagated for $\sigma$ (see \autoref{sec:sigma_errors}) but not for $\mu_0^{[1, 2]}$, since an MC study shows that neglecting uncertainties on $\mu_0^{[1, 2]}$ negligibly affects results.
%, since these results would shift if we under- or over-estimated the widths of scatter peaks. In contrast,


When $j\geq1$, each scattering term $\mathcal{I}*\mathcal{L}_{(\mathrm{H}_2)}^{*j}$ or $\mathcal{I}*\mathcal{L}_{(\mathrm{He})}^{*j}$
%are again modeled as a function of $\sigma$.
%Monte Carlo studies show that these higher-order peaks can each be modeled as Gaussian without biasing $E_0$ or $m_\beta$ results, despite the asymmetry in $\mathcal{L}_{\mathrm{tot}}^{*j}$.
 can be modeled by a normal distribution, with a mean and standard deviation that depend only on $\sigma$. This approximation holds despite the asymmetry in $\mathcal{L}_{(\mathrm{H}_2, \mathrm{He})}^{*j}$ for two reasons:  the $j\geq1$ peaks each contribute sub-dominantly to $\mathcal{R}_{\mathrm{PSF}}$, and they are broadened by $\mathcal{I}$, making them more Gaussian. 
 %, they overlap with each other, and they
 %Employing Gaussian distributions allows analytic convolution of $\mathcal{R}_{\mathrm{PSF}}$ with the $\beta$-spectrum model, simplifying computation.
%as discussed in \autoref{sec:T2-beta-model}.
%For a given gas species and $j$, the scatter tail includes one term characterized by a mean and standard deviation that depend only on $\sigma$.
Scatter peaks are modeled as

\begin{equation}
\begin{split}
&\mathcal{I}*\mathcal{L}_{\mathrm{tot}}^{*j} (j\geq1)  \approx \gamma_{\mathrm{H}_2}\Big[\mathcal{I}*\mathcal{L}_{(\mathrm{H}_2)}^{*j}\Big] + (1-\gamma_{\mathrm{H}_2})\Big[\mathcal{I}*\mathcal{L}_{(\mathrm{He})}^{*j}\Big] \\ &\!\!\!\!\!\rightarrow  \gamma_{\mathrm{H}_2}\mathcal{N}\Big(\mu_j^{\mathrm{H}_2}(\sigma), \sigma_j^{\mathrm{H}_2}(\sigma)\Big)  +(1\!-\!\gamma_{\mathrm{H}_2})\mathcal{N}\Big(\mu_j^{\mathrm{He}}(\sigma), \sigma_j^{\mathrm{He}}(\sigma)\Big),
\end{split}
\label{eq:simplified_scattering}
\end{equation}

\noindent where $\gamma_{\mathrm{H}_2}$ is the hydrogen inelastic scatter fraction and $1-\gamma_{\mathrm{H}_2}$ is the helium inelastic scatter fraction.  The Gaussian means $\mu_j$ depend on $\sigma$ because $\mathcal{L}_{(\mathrm{H}_2, \mathrm{He})}^{*j}$ is asymmetric, so the convolution with $\mathcal{I}$ can shift the center of each scatter peak when $\sigma$ is large enough (as is the case for deep quad trap data).

The slopes and $y$-intercepts of $\mu_j(\sigma)$ and $\sigma_j(\sigma)$ are fixed during tritium data analysis, so the scatter tail shape depends only on $\sigma$, $\gamma_{\mathrm{H}_2}$ and scatter peak amplitudes. For each gas, the slopes and intercepts for $\mu_j$ and $\sigma_j$ are determined through the following procedure. For a given $j$, we fit Gaussians to 20 sets of scatter peaks broadened by 20 different resolution widths, ranging from $0.5 \sigma$ to $1.5 \sigma$. 
This procedure produces $\mu_j$ and $\sigma_j$ for a range of $\sigma$ values. We then observe and fit the linear dependence of $\mu_j$ and $\sigma_j$ on $\sigma$. 

Each scatter peak is multiplied by the corresponding amplitude $\mathcal{A}_j(p, q)$.
%multiplies the $j^{\rm th}$ scatter peak in \autoref{eq:T2_ins_res_model} or \ref{eq:simplified_scattering}.
%The tritium and $^{\mathrm{83m}}$Kr models for $\mathcal{A}_j (p, q)$ are identical.
%; see \autoref{sec:Kr_detector_response}.
Tritium-specific $p$ and $q$ values are estimated in \autoref{sec:scatter_peak_errors} by slightly shifting $p$ and $q$ from the fit to $^{\mathrm{83m}}$Kr pre-tritium data, to account for a difference in the mean number of tracks per event ($N_{\mathrm{tracks}}^{\mathrm{true}}$) between $^{\mathrm{83m}}$Kr and tritium data.
%\autoref{sec:scatter_peak_errors} describes the procedure for estimating $p$ and $q$ for tritium data.
% (see \autoref{sec:deep-trap-data-and-fits})
Combining the $j=0$ and $j\geq1$ peaks, the full $\mathcal{R}_{\mathrm{PSF}}$  model for tritium analysis includes a limited set of free parameters with propagated uncertainties: $\gamma_{\mathrm{H}_2}$, $\sigma$, $p$, and $q$.
%and the two variables $p$ and $q$, which control the amplitudes $\mathcal{A}_j$ of all scatter peaks.


\subsection{Event rate model}\label{sec:event_rate}
%Per the full CRES spectrum model (\autoref{eq:FullModel1}), we convolve the above models for the beta spectrum and $\mathcal{R}_{\mathrm{PSF}}$, then multiply the result by $\epsilon_k$, the detection efficiency for frequency bin $k$ (see \autoref{sec:efficinecy_binning}).
For both tritium data generation and analysis models, the signal probability density function $\mathcal{S}(E_\mathrm{kin})$ is given by \autoref{eq:FullModel1}. A false event probability density function $\mathcal{F}(E_\mathrm{kin})$ is also introduced.  $\mathcal{F}(E_\mathrm{kin})$ is assumed to be flat in energy because the probability to measure RF noise (the only expected significant background source) is uniform as a function of cyclotron frequency, and energy is approximately linearly related to frequency over a limited range. Combining signal and background, the expected tritium event rate is  
\begin{equation}
  \frac{dN}{dE_\mathrm{kin}}(E_\mathrm{kin}) = r_s \mathcal{S}(E_\mathrm{kin})+r_f\mathcal{F}(E_\mathrm{kin}),
  \label{eq:fullT2model}
\end{equation}
\noindent where $r_s$ is the signal rate and $r_f$ is the false event rate. Binning of data is handled differently in Bayesian and frequentist analyses, as  discussed in \autoref{sec:final-analysis}.  
% During analysis, for all bins, cyclotron frequencies are converted to energies according to \autoref{eq:energytofrequency}, using the magnetic field $B$ from the $^{\mathrm{83m}}$Kr fit procedure in \autoref{sec:deep-trap-data-and-fits}. The parameter $B$ is discussed further in \autoref{sec:Bfield_errors}.
% (the only potentially significant expected background source; see \autoref{subsec:electron_data})