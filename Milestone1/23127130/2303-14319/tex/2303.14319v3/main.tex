\documentclass[12pt]{amsart}
\usepackage{amstext,amsfonts,amssymb,amscd,amsbsy,amsmath,verbatim, mathrsfs, fullpage}
\usepackage[alphabetic,abbrev,lite]{amsrefs} % for bibliography 
\usepackage{ifthen,tikz}
\usepackage{color}
\usepackage{amsthm}
\usepackage{latexsym}
\usepackage[all]{xy}
\usepackage{enumerate}
\usepackage{mathtools}
\setcounter{MaxMatrixCols}{15}


\DeclarePairedDelimiter\ceil{\lceil}{\rceil}
\DeclarePairedDelimiter\floor{\lfloor}{\rfloor}

\numberwithin{equation}{section}

\newtheorem{lemma}[equation]{Lemma}
\newtheorem{theorem}[equation]{Theorem}
\newtheorem*{gr}{Green's Linear Syzygy Theorem}
\newtheorem{propo}[equation]{Proposition}
\newtheorem{prop}[equation]{Proposition}
\newtheorem{cor}[equation]{Corollary}
\newtheorem{conj}[equation]{Conjecture}
\newtheorem{claim}[equation]{Claim}
%\newtheorem{claim*}{Claim}
\newtheorem{thm}[equation]{Theorem}
\newtheorem{notation}[equation]{Notation}
\newtheorem{convention}[equation]{Convention}
\newtheorem{question}[equation]{Question}


\theoremstyle{definition}
\newtheorem{defn}[equation]{Definition}
\newtheorem{setup}[equation]{Setup}
\newtheorem{example}[equation]{Example}
\newtheorem{construction}[equation]{Construction}
\newtheorem{algorithm}[equation]{Algorithm}
\newtheorem{warning}[equation]{Warning}
\newtheorem{conventions}[equation]{Conventions}

\theoremstyle{remark}
\newtheorem{remark}[equation]{Remark}
\newtheorem{remarks}[equation]{Remarks}
\newtheorem{rem}[equation]{Remark}
\newtheorem*{claim*}{Claim}
\newtheorem*{case*}{Case}

% Commands

\newcommand{\Kos}{\mathcal{K}}
\newcommand{\cS}{\mathcal{S}}
\newcommand{\rC}{\mathrm{C}}
\newcommand{\Cech}{\check{\mathrm{C}}}
\newcommand{\Tate}{{\mathbf{T}}}
\newcommand{\Tail}{{\mathbf{Tail}}}
\newcommand{\cC}{\mathcal{C}}
\newcommand{\cK}{\mathcal{K}}
\newcommand{\tot}{\operatorname{tot}}

\newcommand{\isom}{\cong}
\newcommand{\m}{\mathfrak m}
\newcommand{\PP}{\mathbb P}
\newcommand{\bD}{\mathbf D}
\newcommand{\df}{\operatorname{diff}}
\renewcommand{\P}{\PP}
\newcommand{\bA}{\mathbb A}
\newcommand{\A}{\bA}
\newcommand{\HH}{\mathrm H}
\newcommand{\GG}{\mathbb G}
\newcommand{\ZZ}{\mathbb Z}
\newcommand{\QQ}{\mathbb Q}
\newcommand{\bH}{\mathbf H}
\newcommand{\bF}{\mathbf F}
\newcommand{\DD}{\mathbf D}
\newcommand{\lideal}{\langle}
\newcommand{\rideal}{\rangle}
\newcommand{\initial}{\operatorname{in}}
\newcommand{\pdim}{\operatorname{pdim}}
\newcommand{\Hilb}{\operatorname{Hilb}}
\newcommand{\Spec}{\operatorname{Spec}}
\newcommand{\NE}{\overline{\operatorname{NE}}}
\newcommand{\Eff}{\operatorname{Eff}}
\newcommand{\im}{\operatorname{im}}
\newcommand{\NS}{\operatorname{NS}}
\newcommand{\Frac}{\operatorname{Frac}}
\newcommand{\ch}{\operatorname{char}}
\newcommand{\Proj}{\operatorname{Proj}}
\newcommand{\id}{\operatorname{id}}
\newcommand{\Div}{\operatorname{Div}}
\newcommand{\tr}{\operatorname{tr}}
\newcommand{\Tr}{\operatorname{Tr}}
\newcommand{\Supp}{\operatorname{Supp}}
\newcommand{\Gal}{\operatorname{Gal}}
\newcommand{\Pic}{\operatorname{Pic}}
\newcommand{\QQbar}{{\overline{\mathbb Q}}}
\newcommand{\Br}{\operatorname{Br}}
\newcommand{\Bl}{\operatorname{Bl}}
\newcommand{\Cox}{\operatorname{Cox}}
\newcommand{\Tor}{\operatorname{Tor}}
\newcommand{\diam}{\operatorname{diam}}
\newcommand{\Hom}{\operatorname{Hom}} %done
\newcommand{\sheafHom}{\mathcal{H}om}
\newcommand{\Gr}{\operatorname{Gr}}
\newcommand{\HF}{\operatorname{HF}}
\newcommand{\HP}{\operatorname{HP}}
\newcommand{\Osh}{{\mathcal O}}
\newcommand{\cO}{{\mathcal O}}
\newcommand{\kk}{{\bf k}}
\newcommand{\rank}{\operatorname{rank}}
\newcommand{\length}{\operatorname{length}}
\newcommand{\codim}{\operatorname{codim}}
\newcommand{\depth}{\operatorname{depth}}
%\newcommand{\FF}{\mathbb{F}}
\newcommand{\F}{\FF}
\newcommand{\Sym}{\operatorname{Sym}} %done
\newcommand{\GL}{{GL}}
\newcommand{\R}{\mathbb{R}}
\newcommand{\CC}{\mathbb{C}}
\newcommand{\Syz}{\operatorname{Syz}}
\newcommand{\Prob}{\operatorname{Prob}}
\newcommand{\defi}[1]{\textsf{#1}} % for defined terms
\newcommand{\Htot}{H_{\tot}}
\newcommand{\Ltot}{L_{\tot}}
\newcommand{\beq}{\begin{displaymath}}
\newcommand{\eeq}{\end{displaymath}}
\newcommand{\bs}{\backslash}
\newcommand{\ff}{\mathbf{f}}
\newcommand{\Gam}{\Gamma}
\newcommand{\Lotimes}{\overset{L}{\otimes}}



\newcommand{\Bmod}{\ensuremath{B_\text{mod}}}
\newcommand{\Bint}{\ensuremath{B_\text{int}}}
\newcommand\commentr[1]{{\color{red} \sf [#1]}}
\newcommand\commentb[1]{{\color{blue} \sf [#1]}}
\newcommand\commentm[1]{{\color{magenta} \sf [#1]}}
\newcommand{\daniel}[1]{{\color{blue} \sf $\clubsuit\clubsuit\clubsuit$ Daniel: [#1]}}
\newcommand{\michael}[1]{{\color{red} \sf $\clubsuit\clubsuit\clubsuit$ Michael: [#1]}}
\newcommand{\maya}[1]{{\color{green} \sf $\clubsuit\clubsuit\clubsuit$ Maya: [#1]}}

\def\edim{\operatorname{edim}}
\def\reg{\operatorname{reg}}

\newcommand{\ve}[1]{\ensuremath{\mathbf{#1}}}
\newcommand{\chr}{\ensuremath{\operatorname{char}}}

%Added by MB:
\def\nc{\newcommand}
\def\on{\operatorname}
\nc{\Q}{\mathbb{Q}}
\nc{\RR}{\mathbf{R}}
\nc{\LL}{\mathbf{L}}
\nc{\xra}{\xrightarrow}
\nc{\xla}{\xleftarrow}
\def\a{\alpha}
\def\om{\omega}
\def\Om{\Omega}
\def\DM{\operatorname{DM}}
\def\Coh{\operatorname{Coh}}
\def\Mod{\operatorname{Mod}}
\def\free{\operatorname{free}}
\def\QCoh{\operatorname{QCoh}}
\def\Cpx{\operatorname{Cpx}}
\def\th{\on{th}}
\def\F{\mathcal{F}}
\def\coker{\on{coker}}
\def\p{\partial}
\def\wt{\widetilde}
\def\i{\iota}
\nc{\into}{\hookrightarrow}
\nc{\onto}{\twoheadrightarrow}
\nc{\OO}{\mathcal{O}}
\nc{\Z}{\mathbb{Z}}
\nc{\cA}{\mathcal{A}}
\nc{\w}{\widehat}
\nc{\End}{\on{End}}
\nc{\res}{\frac{1}{x_0x_1}}
\nc{\tF}{\widetilde{F}}
\nc{\tG}{\widetilde{G}}
\nc{\tf}{\widetilde{f}}
\nc{\Com}{\on{Com}}


\nc{\G}{\mathbb{G}}
\nc{\cG}{\mathcal{G}}
\nc{\cE}{\mathcal E}
\nc{\cF}{\mathcal F}
\nc{\cR}{\mathcal R}
\nc{\cD}{\mathcal D}
\nc{\cB}{\mathcal B}
\nc{\cT}{\mathcal T}
\nc{\cL}{\mathcal L}
\def\M{\mathcal{M}}
\nc{\bM}{\mathbf M}
\nc{\bN}{\mathbf N}
\nc{\U}{\mathbf U}
\nc{\BM}{\mathbf B \mathbf M}
\nc{\Dsg}{\on{D}_{\on{sg}}}
\nc{\fC}{\mathcal{C}}
\nc{\fG}{\mathcal{G}}
\nc{\N}{\mathbb{N}}



%When merging files, add these
\nc{\del}{\partial}
\nc{\cone}{\on{cone}}
\nc{\D}{\on{D}_{\on{diff}}}
\nc{\DMb}{\on{D}^b_{\DM}}
\nc{\Db}{\on{D}^{\on{b}}}
\nc{\Kb}{\on{K}^{\on{b}}}
\nc{\fm}{\mathfrak{m}}
\nc{\Flag}{\on{Flag}}
\nc{\DMmin}{\DM_{\on{min}}}
\nc{\Ddiff}{\on{D}_{\on{diff}}}
\nc{\Dbdiff}{\on{D}^\on{b}_{\on{diff}}}
\nc{\wO}{\widehat{\OO}}
\nc{\wT}{\widehat{T}}
\nc{\from}{\leftarrow}
\nc{\wLL}{\widetilde{\LL}}
\nc{\augCech}{\widetilde{\cC}}
\nc{\Fold}{\on{Fold}}
\nc{\Ext}{\on{Ext}}
\nc{\FF}{\mathbf{F}}
\nc{\Comper}{\Com_{\on{per}}}
\nc{\Unfold}{\on{Unfold}}
\nc{\intHom}{\underline{\Hom}}
\nc{\Ex}{\on{Ex}}
\nc{\tg}{\widetilde{g}}
\def\tP{\widetilde{P}}
\def\b{\beta}
\nc{\B}{\mathcal{B}}
\nc{\K}{\mathcal{K}}
\nc{\kos}{\on{Kos}}
\nc{\Perf}{\on{Perf}}
\nc{\tR}{\widetilde{\cR}}
\nc{\X}{\mathcal{X}}
\nc{\Cl}{\on{Cl}}
\nc{\fU}{\mathcal{U}}
\nc{\bU}{\mathbf U}
\def\c{\colon}
\nc{\st}{\on{st}}
\def\E{\mathcal{E}}
\nc{\coh}{\on{coh}}
\def\ex{\on{ex}}
\def\D{\mathcal{D}}
\def\lin{\on{lin}}
\def\g{\gamma}
\nc{\tU}{\U}
\nc{\bC}{\mathbf{C}}
\nc{\aux}{\on{aux}}
\def\d{\mathbf{d}}
\def\phi{\varphi}
\def\cP{\mathcal{P}}
\def\I{\mathcal{I}}
\def\geo{\on{geo}}
\def\T{\mathbf{T}}
\nc\Dsing{\on{D}^{\on{sing}}}
\def\Y{\mathcal{Y}}
\newcommand{\Manoa}{M\=anoa}

\newcommand{\Hawaii}{Hawai\kern.05em`\kern.05em\relax i}
\newcommand{\UHM}{University of \Hawaii \ at \Manoa}
%This command gets rid of the MR number in the bibliography
\def\MR#1{}

\title{A short proof of the Hanlon-Hicks-Lazarev Theorem}

\author{Michael K. Brown}
\author{Daniel Erman}
\thanks{The second author was supported by NSF grant 
DMS-2200469.}

\newcommand{\Addresses}{{
	\vskip\baselineskip
  	\footnotesize
  	\noindent \textsc{Department of Mathematics and Statistics, Auburn University, Auburn, AL} \par\nopagebreak
	\noindent \textit{E-mail address:} \texttt{mkb0096@auburn.edu}
	\vskip\baselineskip
	\noindent \textsc{Department of Mathematics, \UHM, Honolulu, HI} \par\nopagebreak
	\noindent \textit{E-mail address:} \texttt{erman@hawaii.edu}
  }}


\date{\today}



\begin{document}

\begin{abstract}
We give a short, new proof of a recent result of Hanlon-Hicks-Lazarev about toric varieties.  As in their work, this leads to a proof of a conjecture of Berkesch-Erman-Smith on virtual resolutions and to a resolution of the diagonal in the simplicial case. 
%We also discuss a novel application concerning virtual resolutions that ``see'' the birational geometry. \michael{wordsmithed last line of abstract slightly}
\end{abstract}

\maketitle

\section{Main result}
 We give a short, new proof of a recent result of Hanlon-Hicks-Lazarev about toric varieties and their multigraded Cox rings.  Throughout, we let $X$ be a simplicial, projective toric variety over an  algebraically closed field $k$ with $\Cl(X)$-graded Cox ring $S$. Our main result (Theorem~\ref{thm:virtual hochster}) was first proven in~\cite{HHL}, but our proof is independent from their methods.
Our approach is more algebraic and simpler, while their approach is more explicit and connects to a wider range of topics, including symplectic geometry and homological mirror symmetry.  See also the work of Favero-Huang~\cite{FH}, which was completed simultaneously with \cite{HHL} and whose main results coincide with some of Hanlon-Hicks-Lazarev's. %the results of \cite{HHL}.
%whose main results overlap with some of those in~\cite{HHL}.\daniel{I want to give a bit more credit to Favero-Huang but am not sure how.} \michael{I see what you mean, but I played with this for awhile, and I'm not sure how to improve it either. I think it's okay as is. We also give them more credit in two different spots later on Section 1.}



Our interest in these topics begins with a program to extend results on syzygies to multigraded or toric settings.  The basic perspective, introduced by Berkesch-Erman-Smith in~\cite{BES}, is that many classical results about minimal free resolutions will have strong analogues in the toric setting, as long as one replaces minimal free resolutions with the more flexible notion of a virtual resolution.
\begin{defn}
\label{defn:virtual}
Let $M$ be a finitely generated $\Cl(X)$-graded $S$-module.  A \defi{virtual resolution} of $M$ is a free complex $F_\bullet$ of $S$-modules such that there is a quasi-isomorphism $\widetilde{F_\bullet} \xra{\simeq} \widetilde{M}$ of complexes of $\OO_X$-modules.\footnote{If $X$ is smooth, then $\widetilde{F_\bullet}$ consists of sums of line bundles and is sometimes called a line bundle resolution.  See Remark~\ref{rmk:not line bundles} regarding the simplicial case.}
\end{defn}

\noindent The following is a consequence of Hanlon-Hicks-Lazarev's main result~\cite[Theorem~A]{HHL}.

\begin{thm}\label{thm:virtual hochster}
Let $Y$ be a normal toric variety and $Y \into X$ a closed immersion that is a toric morphism \cite[Definition 3.3.3]{CLS}. Denote by $I$ the defining ideal of $Y \into X$ (Definition~\ref{def:ideal}).  The $S$-module $S/I$ admits a virtual resolution of length $\codim(Y\subseteq X)$.
\end{thm}


And here is our short proof of Theorem~\ref{thm:virtual hochster}. The proof relies on some elementary facts about toric varieties that we recall in Lemma~\ref{lem:technical} below.

\begin{proof}[Proof of Theorem~\ref{thm:virtual hochster}]
The Cox ring $S$ of $X$ is positively $\Cl(X)$-graded~\cite[Definition~A.1, Example~A.2]{BE}, and so we may consider $\Cl(X)$-graded minimal free resolutions of $\Cl(X)$-graded $S$-modules. Let $R$ be the normalization of $S/I$ and $F_\bullet$ the minimal free resolution of $R$ as an $S$-module.  Since $Y$ is normal, $\widetilde{R}=\cO_Y$ as a sheaf on $X$, and so $F_\bullet$ is a virtual resolution of $S/I_Y$. By Lemma~\ref{lem:technical}(1) and~\cite[Theorem 1.1.17 and Proposition 1.3.8]{CLS}, the ring $R$ is a product of affine semigroup rings of the same dimension. Hochster's Theorem therefore implies that each component of $R$ is a Cohen-Macaulay ring~\cite[Theorem 1]{hochster}. It follows that $R$ is also a Cohen-Macaulay $S/I$-module: indeed, we have $\dim(R) = \dim(S/I)$, and since $R$ is a finitely generated $S/I$-module~\cite[Theorem 4.14]{eisenbudbook}, any system of parameters on $S/I$ is a system of parameters on each component of $R$ and hence a regular sequence. The length of $F_\bullet$ is the projective dimension of $R$, which, by the Auslander-Buchsbaum formula~\cite[Theorem 19.9]{eisenbudbook}, is equal to $\depth_S(S) - \depth_S(R) = \dim(S) - \dim(S/I)$ (while the version of the Auslander-Buchsbaum formula we cite pertains to local rings, the desired result for the polynomial ring $S$ follows by~\cite[Proposition 1.5.15]{BH}). Lemma~\ref{lem:technical}(2) therefore implies that the length of $F$ is equal to $\codim(Y\subseteq X)$.
\end{proof}

%Let us briefly 
We now describe
%some 
applications of Theorem~\ref{thm:virtual hochster} and their history.  For a fuller discussion, see~\cite[\S1]{HHL}.  We start with a 
%following 
special case, first proven by Hanlon-Hicks-Lazarev:
\begin{thm}[\cite{HHL} Corollary B]\label{thm:virtual diagonal}
The coordinate ring of the diagonal embedding $X\subseteq X\times X$ admits a virtual resolution of length $\dim X$.
\end{thm}
Special cases of Theorem~\ref{thm:virtual diagonal} were studied in~\cite{BE,brown-sayrafi,canonaco}, and ~\cite{BPS,anderson} study closely related questions.
It was known that this result would immediately yield proofs of two conjectures that also had received independent interest. The first conjecture is due to Berkesch-Erman-Smith~\cite[Question 1.3]{BES} and was proven by Hanlon-Hicks-Lazarev:


\begin{thm}[\cite{HHL} Corollary C]\label{thm:virtual syzygy}
Any module $M$ as in Definition~\ref{defn:virtual} has a virtual resolution of length  $\leq \dim X$.
\end{thm}


Hilbert's Syzygy Theorem gives a bound of $\dim S=\dim X + \rank \Cl(X)$; Theorem~\ref{thm:virtual syzygy} implies that the added flexibility of virtual resolutions allows for significantly shorter resolutions, especially when $\rank \Cl(X)$ is large.  See~\cite{BES,HNV,berkesch-klein-loper-yang} and elsewhere for many examples of this phenomenon.   
%In~\cite{HHL}, Theorem~\ref{thm:virtual diagonal} is stated for smooth varieties, but as we will see, the basic ideas easily extend to the simplicial case.  
Prior to~\cite{HHL}, Theorem~\ref{thm:virtual syzygy} had been proven in several special cases: when $\rank \Pic(X)=1$ it essentially follows from Hilbert's Syzygy Theorem; for products of projective spaces it was shown in~\cite[Theorem~1.2]{BES} (see also \cite[Corollary~1.14]{EES}); Yang proved it for any monomial ideal in the Cox ring of a smooth toric variety~\cite{yang}; and Brown-Sayrafi proved it for smooth projective toric varieties of Picard rank 2~\cite{brown-sayrafi}.

The second conjecture, due to Orlov, is the special case of \cite[Conjecture 10]{orlov} for toric varieties. This was first proven by Favero-Huang in \cite[Theorem~1.2]{FH}, and independently and essentially simultaneously in ~\cite[Corollary E]{HHL}.
\begin{thm}\label{thm:rouquier}
%Let $X$ be a normal toric variety.  
The Rouquier dimension of $D^b(X)$ equals $\dim X$.
\end{thm}

Special cases of Theorem~\ref{thm:rouquier} had been established in~\cite{BC, BF, BDM, BFK} before Favero-Huang and Hanlon-Hicks-Lazarev proved it in general. The full version of Orlov's Conjecture states that Theorem~\ref{thm:rouquier} extends to any smooth quasi-projective variety; see \cite[\S 1.2]{BC} for a list of known cases of this conjecture. 

Theorem~\ref{thm:virtual hochster} easily implies Theorems~\ref{thm:virtual diagonal}, ~\ref{thm:virtual syzygy} and~\ref{thm:rouquier}. To prove Theorem~\ref{thm:virtual diagonal}, observe that the diagonal $X \subseteq X \times X$ satisfies the conditions of Theorem~\ref{thm:virtual hochster}. To prove Theorem~\ref{thm:virtual syzygy}, one can simply follow the method of~\cite[Proof of Theorem~1.2]{BES}.  For Theorem~\ref{thm:rouquier}, one can use standard techniques on derived categories; see, e.g., the proof of ~\cite[Corollary E]{HHL}.


%We also obtain a novel corollary involving the birational geometry of $X$.  Let $X_1, \dots, X_r$ be the simplicial, projective varieties that arise in the GKZ decomposition of the secondary fan of $X$, with $X_1=X$.  For any $i$, let $\Delta_{i} \subseteq X \times X_i$ be the closure of the diagonal copy of the torus $(k^*)^{\dim X}\subseteq X\times X_i$ where $(\lambda) \mapsto (\lambda, \lambda)$.    
%We call $\Delta_{i}$ the \defi{quasi-diagonal} in $X\times X_i$.  
%For any graded $S$-module $M$, there is a corresponding sheaf $\widetilde{M}_{X_i}$ on $X_i$ and we say that $F$ is a virtual resolution of $(M,X_i)$ if $\widetilde{F}_{X_i}$ is a line bundle resolution of $\widetilde{M}_{X_i}$.
%  Let $X$ and $X'$ be two simplicial, projective toric varieties.  Assume that $X$ and $X'$ are birational along the torus-invariant open set $U$, so that $U$ is a dense open subset of both $X$ and $X'$.  Let $\Delta
%\begin{thm}[Quasi-diagonal Theorem]\label{thm:virtual quasi-diagonal}
%With notation as above:  there exists a free complex $F_\bullet$ of $S\otimes_k S$-modules  of length $\dim X$ which is simultaneously a virtual resolution of the coordinate ring of $\Delta_{i,j}\subseteq X_i \times X_j$ for all $1 \leq i , j \leq r$.
%\end{thm}
%This result has some interesting corollaries.  
%Let $\Phi_{i}$  be the Fourier-Mukai transform $D^b(X)\to D^b(X_i)$ with kernel $\cO_{\Delta_{i}}$.  When $i=1$ this is the identity, but for $i\ne 1$ it is a nontrivial functor.
%{\color{red}
%\begin{thm}\label{thm:secondary fan}
%Let $M$ as in Definition~\ref{defn:virtual}. There is a complex $F$ of free $S$-modules whose sheafification over $X_i$ is isomorphic in $\Db(X_i)$ to $\Phi_{i}(\widetilde{M})$ for all $1\leq i \leq r$.
%Let $M$ as in Definition~\ref{defn:virtual}.  There is a free complex $F$ of length $\leq \dim X$ that is simultaneously a virtual resolution of $\Phi_{i}(\widetilde{M})$ on $X_i$ for all $1\leq i \leq r$.
%\end{thm}

%Since $X_1=X$, sheafifying $F$ over $X$ gives a representation of $\widetilde{M} \in \Db(X)$.  But $F$ also extends across the secondary fan of $X$, providing a representation over each chamber as well.  Thus, the multigraded algebraic properties of $S$ ``see'' the birational geometry of $X$.  See~\cite[Remark 3.6]{BE} for a similar observation.
%}    
%{\color{blue}
%\begin{cor}\label{cor:secondary fan}
%Let $M$ as in Definition~\ref{defn:virtual}. For any $a\in \Cl(X)$ such that $\cO_X(a)$ is sufficiently we have the following:  
%There is a free complex $F$ of length $\leq \dim X$ that is simultaneously a virtual resolution of $\Phi_{i}(\widetilde{M}(a))$ on $X_i$ for all $1\leq i \leq %r$.
%There is a complex $F$ of free $S$-modules whose sheafification over $X_i$ is isomorphic in $\Db(X_i)$ to $\Phi_{i}(\widetilde{M})$ for all $1\leq i \leq r$.
%Let $M$ as in Definition~\ref{defn:virtual}.  There is a free complex $F$ of length $\leq \dim X$ that is simultaneously a virtual resolution of $\Phi_{i}(\widetilde{M})$ on $X_i$ for all $1\leq i \leq r$.
%\end{cor}
%The proof is a nearly immediate corollary of a  few observations.  First, each $X_i$ can also be realized as a Cox quotient of $\Spec(S)$, with a different irrelevant ideal $B_i$; see~\cite[\S 14.4]{CLS}.  \daniel{There's another diagram here, but I commented it out.}
%We thus have the following diagram:
%\[
%\xymatrix{
%D^b(X_i\times X)\ar[d]^{\pi_2}\ar[r]^{\iota}&D^b_{\operatorname{gr}}%(\Spec(S)\times X)\ar[d]^{p_1}\ar[rd]_{p_2}& \\
%D^b(X_i)\ar[r]^{\kappa_i}&D^b_{\operatorname{gr}}(\Spec(S))&D^b(X)
%}
%\]
%where ``gr'' denotes $\Cl(X)$-graded modules on $\Spec(S)$. Second, we let $P\subseteq S\otimes_k S$ be the prime ideal defining $\Delta \subseteq X\times X$; that is, the sheaf on $X\times X$ corresponding to $(S\otimes_k S)/P$ is $\cO_{\Delta}$.  
%\michael{I think the actual reference is Prop 14.4.14(b) in CLS, but I think this would be a somewhat weird reference, because it's notationally heavy.} \daniel{The right reference is going to be notationally heavy, unfortunately.  I've read through it all and I think I can fill it in soon.} Thus, $(S\otimes_k S)/P$ also determines the sheaf of an integral variety on $X_i\times X$; since it contains the diagonal copy of the torus as a dense open set, this variety must be $\Delta_i$, and the sheaf is $\cO_{\Delta_i}$.  We write $\Delta_S$ for the corresponding subvariety of $\Spec(X)\times X$.
%It follows that the minimal free resolution $G$ of the normalization 
%\michael{no need to normalize anymore, right? Since we don't care about length} 
%of $(S\otimes_k S)/P$ induces a virtual resolution of $\cO_{\Delta_i}$ on $X_i\times X$, and of a similar object on $\Spec(S)\times X$.   It follows that the complex $G$ induces locally free complexes $G^S$ (resp $G^i$) on $\Spec(S)\times X$ (resp on $X_i\times X$) resolving $\cO_{\Delta_S}$ (resp $\cO_{\Delta_i}$.  If we define a Fourier-Mukai transform $\Phi_S\colon D^b(X) \to D^b_{\operatorname{gr}}(S)$ using $G^S$ as the kernel, we then have
%\[
%\xymatrix{
%D^b(X)\ar[r]^{\Phi_S}\ar@/^2.0pc/[rr]^{\Phi_i}&D^b_{\operatorname{gr}}(\Spec(S))\ar[r]&D^b(X_i)
%}
%\]
%where the second is just replacing a graded $S$-module by the corresponding sheaf on $X_i$.
%}    

%Since $X_1=X$, the complex $F$, in particular, a virtual resolution of $M$ on $X$.  But $F$ also extends across the secondary fan of $X$, providing a virtual resolution on each chamber.  Thus, the multigraded algebraic properties of $S$ ``see'' the birational geometry of $X$.  See~\cite[Remark 3.6]{BE} for a similar observation. 
%\begin{proof}[Proof of Theorem~\ref{thm:secondary fan}]
%Let $P\subseteq S\otimes_k S$ be the prime ideal defining $\Delta \subseteq X\times X$; that is, the sheaf on $X\times X$ corresponding to $(S\otimes_k S)/P$ is $\cO_{\Delta}$.  
%Each $X_i$ can also be realized as a Cox quotient of $\Spec(S)$, with a different irrelevant ideal $B_i$; see~\cite[\S 14.4]{CLS}.
%\michael{I think the actual reference is Prop 14.4.14(b) in CLS, but I think this would be a somewhat weird reference, because it's notationally heavy.} \daniel{The right reference is going to be notationally heavy, unfortunately.  I've read through it all and I think I can fill it in soon.}
%Thus, $(S\otimes_k S)/P$ also determines the sheaf of an integral variety on $X_i\times X$; since it contains the diagonal copy of the torus as a dense open set, this variety must be $\Delta_i$, and the sheaf is $\cO_{\Delta_i}$.
%It follows that the minimal free resolution of the normalization \michael{no need to normalize anymore, right? Since we don't care about length} of $(S\otimes_k S)/P$ induces a virtual resolution of $\cO_{\Delta_i}$ on $X_i\times X$.  
%
%\michael{what follows is the end of the (faulty) proof of the former version of the theorem}
%{\color{blue}As in the case of Theorem~\ref{thm:virtual syzygy}, we apply the method of proof from~\cite[Proof of Theorem~1.2]{BES}.  In this case, it involves a Fourier-Mukai transform with respect to the resolution of $\cO_{\Delta_i}$, and we select $a$ to be sufficiently positive so that the corresponding hypercohomology spectral sequence consists of a single row.  Since the free complex $F$ that corresponds to that row is independent of $i$, and since the spectral sequence abuts to $\Phi_i(\widetilde{M}(a))$, we obtain the result.
%\end{proof}
%Without the positivity assumption, one could obtain a similar result but with free monads~\cite{FIND CITATION} in place of virtual resolutions.}

\medskip


Our proof of Theorem~\ref{thm:virtual hochster} is quite simple, perhaps embarrassingly so given the prior partial results on these questions cited above.
%~\cite[Question 1.3]{BES} and the toric case of~\cite[Conjecture 10]{orlov}. 
It is not yet clear how to compare our resolutions to those obtained in~\cite{HHL}, but we believe that the two constructions agree in the case of Theorem~\ref{thm:virtual diagonal}. Their work gives a 
%marvelously 
creative perspective on building these resolutions, drawing motivation from the symplectic side of the mirror symmetry functor and involving a wide array of ideas.%, including discrete morse theory, quiver diagrams, and more.
\footnote{In a different direction, we refer to Borisov's work \cite{borisov} for an alternative proof of Hochster's Theorem \cite[Theorem 1]{hochster}---the main ingredient of our proof of Theorem~\ref{thm:virtual hochster}---and an explanation of how the techniques used there relate to mirror symmetry.} The resolutions they obtain are quite explicit; indeed, their resolution of the diagonal yields a canonical generating set for the derived category of any normal toric variety, proving a claim of Bondal \cite[Corollary D]{HHL}. However, some algebraic aspects of their constructions are harder to determine.  For instance, if $F_\bullet$ is the free complex of $S$-modules corresponding to one of their resolutions, their work implies that the modules $H_i(F_\bullet)$ correspond to the zero sheaf on $X$ for all $i>0$, but it is not clear whether $H_i(F_\bullet)$ equals the zero module on the nose, i.e. it is not clear if $F_\bullet$ is acyclic as a complex of $S$-modules. The $S$-module that arises as $H_0(F_\bullet)$ is also unclear. By comparison, the complexes that arise in our construction are always acyclic, and they resolve normalizations of coordinate rings. However, we are not able to give as explicit of a description of the terms.  It would be very interesting to better compare these complexes, and to compare them with those in~\cite{BE, brown-sayrafi}.  Favero-Huang's approach~\cite{FH} can almost certainly yield all of the above results as well, and it would be interesting to compare to those resolutions too.




\begin{remark}
As our resolutions from Theorem~\ref{thm:virtual hochster} rely only on standard algebraic constructions, they can be directly computed in {\em Macaulay2} \cite{M2}.  The constructions in~\cite{HHL} are explicit, but due to their novelty, computing them in practice requires more effort.  Of course, if one could show that the two constructions coincide, this would shed more light on both.
\end{remark}
\section{Some elementary facts about toric varieties}
\begin{defn}
\label{def:ideal}
Let $X$, $Y$, and $S$ be as in Theorem~\ref{thm:virtual hochster}, $B \subseteq S$ the irrelevant ideal of $X$, and $Z$ the closure in $\Spec(S)$ of the inverse image of $Y$ under the canonical surjection $\pi \colon \Spec(S) \setminus V(B) \to X$. The \defi{defining ideal of $Y$ in $X$} is the radical ideal $I \subseteq S$ corresponding to the closed subset $Z \subseteq \Spec(S)$. 
\end{defn}

\begin{lemma}
\label{lem:technical}
Let $Z$ and $I$ be as in Definition~\ref{def:ideal}.
\begin{enumerate}
\item The irreducible components of $Z$ are affine toric varieties of the same dimension. Furthermore, if the divisor class group $\Cl(X)$ is torsion-free, then $Z$ is irreducible.
\item We have $\dim(S) - \dim(S/I) = \codim(Y \subseteq X)$.
\end{enumerate}
\end{lemma}

\begin{proof}
Since $Y \into X$ is a toric morphism, it induces an embedding $T_Y \into T_X$ on tori and hence a surjection $p \colon M_X \onto M_Y$ of lattices. Taking the pushout of the surjection $p$ and the canonical map $M_X \to \Z^{\dim{S}}$ yields the morphism
\begin{equation}
\label{eqn:ses}
\xymatrix{
0\ar[r]& M_X \ar[r]\ar[d]^-{p}& \ZZ^{\dim S}\ar[r]\ar[d]^{q} &\Cl(X)\ar[r]\ar[d]^{\cong}&0\\
0\ar[r]& M_Y \ar[r]& M' \ar[r]&\Cl(X)\ar[r]& 0
}
\end{equation}
of exact sequences. The abelian group $M'$ is isomorphic to $\Z^r \oplus A$, where $r$ is defined to be $\dim(S) - \dim(X) + \dim(Y)$, and $A$ is some finite abelian group. We observe that $I$ coincides with the radical of $J \coloneqq \ker(S \into k[\Z^{\dim{S}}] \xra{q} k[M'])$; note that $k[M']$ need not be reduced when $\on{char}(k) \ne 0$, since $M'$ may have torsion, and so $J$ need not be radical. Let us verify that $I = \on{rad}(J)$: since $p$ is surjective, the Snake Lemma implies that $q$ is surjective, and so $J$ is the defining ideal of the closure of $\Spec(k[M'])$ in $\Spec(S)$. Diagram~\eqref{eqn:ses} induces the following morphism of short exact sequences of algebraic groups:
$$
\xymatrix{
0 & \ar[l] T_X & \ar[l]_-{\a} \Spec(k[\Z^{\dim S}]) & \ar[l] \ker(\a) & \ar[l] 0 \\
0 & \ar[l] T_Y \ar[u] & \ar[l]_-{\b} \Spec(k[M']) \ar[u] & \ar[l]   \ker(\b) \ar[u]_{\cong}& \ar[l]  0.
}
$$
It follows that $\a^{-1}(T_Y)$ is equal to the image of $\Spec(k[M'])$ in $\Spec(k[\Z^{\dim S}])$. Since $Z$ is equal to the closure of $\a^{-1}(T_Y)$ in $\Spec(S)$, we conclude that $I = \on{rad}(J)$. 

Writing $R = k[\Z^r]$ and $A = \bigoplus_{i = 1}^t \Z / (n_i)$, we have $$
k[M'] \cong R[z_1, \dots, z_t] /(z_1^{n_1} -1, \dots, z_t^{n_t} - 1).
$$
The quotient of $k[M']$ by its nilradical is therefore a product of copies of $R$, and so $I$ is a finite intersection of prime ideals arising as kernels of ring homomorphisms $S \to R$. It therefore follows from \cite[Proposition 1.1.8]{CLS} that the irreducible components of $Z$ are affine toric varieties of dimension $r$. If $\Cl(X)$ is torsion-free, then the bottom row of Diagram~\eqref{eqn:ses} splits, and so $A = 0$, which means $I$ is prime. This proves (1). As for (2): we have shown that $\dim(Z) = r$, which is precisely $\dim(S) - \codim(Y \subseteq X)$. 
%Since $p$ is surjective, the Snake Lemma implies that $q$ is surjective as well. 
\end{proof}

\section{Examples}
%{\color{red}
%\begin{lemma}
%\label{lem:cox}
%In the setting of Theorem~\ref{thm:virtual hochster}, the quotient %$S/I$ is an affine semigroup ring. 
%\end{lemma}

%\begin{proof}
%Let $B$ denote the irrelevant ideal of $X$ and $Z$ the closure in $\Spec(S)$ of the inverse image of $Y$ under the canonical map $\pi \colon \Spec(S) \setminus V(B) \to X$. The affine variety $Z$ is the counterpart of $S/I$ under the usual ideal-variety correspondence. It thus suffices, by \cite[Proposition 1.1.8, Theorem 1.1.17]{CLS}, to show that $Z$ is the closure of the image of a map $T \to \Spec(S)$, where $T$ is a torus. Letting $S'$ and $B'$ denote the Cox ring and irrelevant ideal of $Y$, the inclusion $Y \into X$ is induced by a map $f \colon \Spec(S') \setminus V(B') \to \Spec(S) \setminus V(B)$; the inverse image of the torus in $Y$ under the composition $\pi f$ is the torus $T = (k^*)^{d}$, where $d = \dim(S')$. The map $f$ therefore induces a map $T \to \Spec(S)$; since $T$ is dense in $\Spec(S')$, the closure of the image of $T$ under $f$ is equal to $Z$.
%\end{proof}
%}
\begin{example}
\label{ex:Pn diagonal}
Let $X=\PP^n$ and $T=k[x_0,\dots, x_n,y_0,\dots,y_n]$, the Cox ring of $X\times X$.  Let $I_\Delta\subseteq T$ be the defining ideal (Definition~\ref{def:ideal}) of the diagonal $X\subseteq X\times X$, i.e. the ideal corresponding to the closure of the set of points in  $\Spec(T)$ of the form $(x_0, \dots, x_n, tx_0, \dots, tx_n)$, where $t\in k^*$. One easily checks that $I_\Delta$ is the kernel of the map $S \to  k[x_0, \dots, x_n, y_0, \dots, y_n, t]$ given by $x_i \mapsto x_i$ and $y_i \mapsto tx_i$, and so $T/I_\Delta$ is isomorphic to the normal semigroup ring $k[x_0, \dots, x_n, tx_0, \dots, tx_n]$. The ideal $I_\Delta$ is generated by the $2\times 2$ minors of the matrix
$
\begin{pmatrix}
x_0&x_1&\cdots &x_n\\
y_0&y_1&\cdots &y_n
\end{pmatrix}.
$
More specifically: these minors vanish on $\Delta$, and since this is a generic matrix, the ideal of $2\times 2$ minors is prime of codimension $n$.  As $T/I_\Delta$ is already normal, the virtual resolution of $T/I_\Delta$ arising from Theorem~\ref{thm:virtual hochster} is just the minimal free resolution of $T/I_\Delta$, which is given by the Eagon-Northcott complex on this matrix.
\end{example}


\begin{example}\label{ex:P112 diagonal}
Let $X$ be the weighted projective space $\PP(1,1,2)$ and $T$ the Cox ring $k[x_0,x_1,x_2,y_0,y_1,y_2]$ of $X\times X$.  
By a calculation similar to Example~\ref{ex:Pn diagonal}, the ring $T/I_\Delta$ is isomorphic to the semigroup ring
$
k[x_0,x_1,x_2,tx_0,tx_1,t^2x_2],
$
which is not normal because $tx_2$ lies in the fraction field and satisfies the integral equation $(tx_2)^2  - x_2 \cdot (t^2x_2)=0$.  Let $R$ be the normalization of $T/I_\Delta$. A presentation matrix for $R$ as a $T$-module is given as follows, where the rows correspond to the generators $1$ and $tx_2$:
\[
\bordermatrix{
&&&&& \cr
1 &x_{1}y_{0}-x_{0}y_{1}&x_{2}y_{0}&x_{2}y_{1}&x_{0}y_{2}&x_{1}y_{2}\cr
tx_2& 0&-x_{0}&-x_{1}&-y_{0}&-y_{1}
}.
\]

The free resolution of $R$ as a $T$-module is given by:
\begin{footnotesize}
\begin{equation}
\label{eq:res}
\begin{matrix} T \\ \oplus\\  T(-1,-1)\end{matrix} \xleftarrow{\left[\begin{smallmatrix} x_{1}y_{0}-x_{0}y_{1}&x_{2}y_{0}&x_{2}y_{1}&x_{0}y_{2}&x_{1}y_{2}\\
0&-x_{0}&-x_{1}&-y_{0}&-y_{1}
\end{smallmatrix}\right]} 
\begin{matrix}
T(-1,-1) \\
\oplus\\ 
T(-2,-1)^2 \\ \oplus \\
T(-1,-2)^2 
\end{matrix}
\xleftarrow{\left[\begin{smallmatrix} 
-x_{2}&0&-y_{2}\\
x_{1}&-y_{1}&0\\
-x_{0}&y_{0}&0\\
0&-x_{1}&-y_{1}\\
0&x_{0}&y_{0}
\end{smallmatrix}\right]}  
\begin{matrix}
T(-3,-1) \\
\oplus\\ 
T(-2,-2)\\ \oplus \\
T(-1,-3) 
\end{matrix}
 \gets 0.
\end{equation}
\end{footnotesize}

\noindent Additionally: we have the short exact sequence
$
0\to T/I_\Delta \to R \to Q \to 0,
$
and $Q = tx_2\cdot k[x_2,y_2]$.  One can directly compute that the sheaf $\widetilde{Q}$ corresponding to $Q$ is the zero sheaf on $X \times X$. In fact, since $Q$ is annihilated by $x_0,x_1,y_0$ and $y_1$, we can reduce to checking that $\widetilde{Q}$ is also zero on the affine patch $D(x_2y_2)$.  The global sections of $\widetilde{Q}$ on this patch are $Q[x_2^{-1},y_2^{-1}]_{(0,0)}=0$, and thus $\widetilde{Q}=0$ as desired.
%Namely, since $tx_2$ has degree $(1,1)$, $Q_{(a,b)}\ne 0$ if and only if $a,b\geq 0$ and $a,b$ are both odd.  But a $T$-module $M$ determines the zero sheaf on $X \times X$  if it is zero in degrees $(a,b)$ for all even integers $a,b \gg 0$; {\color{red} indeed, we need only check that $M[1/x_iy_j]_{(0,0)} = 0$ for $0 \le i, j \le 2$, and this is clear.} \michael{I couldn't find a citation so I just wrote a quick proof (one just needs to check that the module is zero on the canonical affine open cover, and this is clear). This previously said that $M$ determines the zero sheaf on $X \times X$ \emph{if and only if}...etc. That is true, but we only need one direction, so it's quicker to just write this. But we can add a proof of the other direction if you prefer.}
\end{example}

\begin{remark}\label{rmk:not line bundles}
Since $\OO(-1)$ and $\OO(-3)$ are not vector bundles on $\PP(1, 1, 2)$, the resolution~\eqref{eq:res} does not induce a locally free resolution of the diagonal. Indeed, virtual resolutions are not guaranteed to induce locally free resolutions of $\OO_X$-modules unless $X$ is smooth. Alternatively, as in~\cite{HHL}, one could consider the corresponding toric stack.
\end{remark}

\begin{remark}
In many of the prior known cases of Theorem~\ref{thm:virtual syzygy}, a slightly stronger result was proven.  Namely, it was shown that for any such $M$, there exists another module $M'$ satisfying $\widetilde{M}=\widetilde{M'}$ and $\pdim(M')\leq \dim X$; see~\cite{EES,bruce-heller-sayrafi,yang}.  It would be interesting to determine if this was true in general. 
%The fact that the virtual resolution of the diagonal from Theorem~\ref{thm:virtual diagonal} has this property makes it seem reasonable to hope that this might hold.
\end{remark}


\section*{Acknowledgments}  We are very grateful to Andrew Hanlon, Jeff Hicks, and Oleg Lazarev for patiently talking to us about their work and for several inspiring conversations. We only found this approach because of our efforts to understand their beautiful results. We also thank Christine Berkesch, Lauren Cranton Heller, Mahrud Sayrafi, and Jay Yang for helpful comments and discussions. Finally, we thank the anonymous referee for many helpful suggestions.
%
%\section*{Dream Conjecture (for another paper)}
%Let $\Delta_{\text{tot}} \subseteq X_1 \times X_2 \times \cdots \times X_r$ be the closure of the diagonally embedded torus.  Let $\mathcal C$ be the full subcategory of $D^b(\Delta_{\text{tot}})$ generated by the image of $\oplus_{i=1}^r D^b(X_i)\to D^b(\Delta_{\text{tot}})$ given by $\mathcal E \mapsto \pi_i^* \mathcal E$.  Let $\Theta \subseteq \Cl(X)$ be the Thomsen collection.  For each $d\in \Theta$ we have that $-d\in \Eff(X)$ and thus it lies in a chamber of the secondary fan corresponding to some $X_i$.  Let $\mathbf{L}_d := \pi_i^* \mathcal O_{X_i}(d)\in D^b(\Delta_{\text{tot}})$.  (If $-d$ lies on the boundary between two chambers, then either choice is fine.)  We claim that $\pi_{j*} \mathbf{L}_d=\cO_{X_j}(d))$ for all $j$.  (Key mechanism is something like this.  Let $D$ be an effective divisor in $H^0(X_i, \cO_{X_i}(-d))$.  Then on $X_i$ we have:
%\[
%0\to \cO_{X_i}(d) \to \cO_{X_i} \to \cO_D \to 0.
%\]
%Since $D$ is nef we can move it away from the exceptional locus and without loss of generality, get that $\pi_i^*$ is exact and $\pi_{j*}$ are exact, and that $\pi_{j*}(\cO_D) = \cO_{\pi_j(D)}$ or something.
%
%If we do this right, then we will immediately get that the Rouquier dimension of $\mathcal C$ is $\dim X$ and that  $\{\mathbf{L}_d | d\in \Theta\}$ generates $\mathcal C$.  We claim that, in fact, there is a natural sense in which the elements of the Thomsen collection form a full strong exceptional collection for $\mathcal C$.  We'll have to work out the correct ordering.

%5.15: The inverse of any big and nef divisor D with the property that PD does not contain any lattice point in its interior has the property that HiX, O(D)= 0 for all i. This follows directly from the standard fact in toric geometry that the Euler characteristics χ(−D) counts the inner lattice points of the lattice polyt
%Hom(f^*, ) = Hom( , f_*)
%Hom(


%\section*{Notes}
%We first claim that $I$ is the kernel of a map of semigroup rings.
%Since $Y\to X$ is a toric morphism, we get an induced map of tori $T_Y\to T_X$ which in turn induces a map of lattices $M_X\to M_Y$.   We also have the canonical sequence:
%\[
%0\to M_X \to \mathbb Z^{\dim S} \to \Cl(X)\to 0.
%\]
%Composing the corresponding element of $\Ext^1(\Cl(X),M_X)$ with the map $M_X\to M_Y$ we obtain a short exact seuqence of the form
%\[
%0\to M_Y \to M' \to \Cl(X)\to 0.
%\]
%Write $r=\dim Y + \dim S - \dim X$.  Note that, since $\Cl(X)$ has rank $\dim S-\dim X$, $M'\cong \ZZ^{r}\oplus M'_{tor}$ where $M'_{tor}$ is a finite abelian group.
%In fact, the induced map on extensions even yields a commutative diagram:
%\[
%\xymatrix{
%0\ar[r]& M_X \ar[r]\ar[d]& \ZZ^{\dim S}\ar[r]\ar[d] &\Cl(X)\ar[r]\ar[d]_{\cong}&0\\
%0\ar[r]& M_Y \ar[r]& M' = \ZZ^{r}\oplus M'_{tor} \ar[r]&\Cl(X)\ar[r]& 0.
%}
%\]

%Our assumption that $Y\subseteq X$ is closed and toric implies that $M_X\to M_Y$ is surjective.  By a variant of the 5-lemma, we then get that $\ZZ^{\dim S}\to M'$ is surjective.  
%Define $I$ as the kernel of the composition:
%\[
%I = \ker(S\to k[\ZZ^{\dim S}]\to k[M']).
%\]
%Note that $k[M']\cong k[\ZZ^r]\otimes_k k[M'_{tor}]\cong (k[\ZZ^r])^{|M'_{tor}|}$\daniel{Char $0$?  I guess we have $k[\ZZ/n]=k[t]/(t^n-1)$.  But that's nonreduced in positive characteristic I think.  Maybe better to go with ``choose a prime $P$''.} , and thus $\Spec(S/I)$ is the Zariski closure of the image of a finite number of copies of tori.  The image of each torus is an affine toric variety of dimension $r$ (and an irreducible component of $\Spec(S/I)$), and so the normalization of $\Spec(S/I)$ is a finite union of affine, normal toric varieties of dimension $r$.  
%\daniel{Ends here.}


%Define $P$ as the kernel of the composition:
%\[
%P = \ker(S\to k[\ZZ^{\dim S}]\to k[M']\to k[\ZZ^r]).
%\]
%Note that $V(P)$ is an affine toric variety as it is the  closure of the map from an $r$-dimensional torus to $\Spec(S)$.  By construction, applying $\pi$ to the torus of $V(P)$ yields the torus of $Y$ and thus $\pi(V(P))$ equals $Y$.   
%Note that $V(P)$ is an irreducible component of $\pi^{-1}(Y)$.  
%The normalization of $S/P$ is therefore Cohen-Macaulay and so on.
%Then $V(I)$ is closed and radical, because $k[M']$ is reduced, and $V(I)$ is the Zariski closure of $\Spec(k[M'])\to \Spec(S)$ which also equals the Zariski closure of $\pi^{-1}(Y)$.  

%If $M'$ has torsion then it could be the case that $I$ is not prime, in which case we will have $I=P_1\cap \cdots \cap P_r$ an intersection of prime ideals.   
%Note that $V(I_0)$ is the preimage of the closed immersion $T_Y\subseteq T_X$ under the map $T_S\to T_X$ \daniel{need to justify this} and so $V(I)$ is $\pi^{-1}(Y)$.  

%This would show that $V(I)$ is the kernel of a map of semigroup rings. So $S/I$ is isomorphic to $k[M'']$ for some subsemigroup $M''\subseteq M'$.  However, $M''$ might have torsion, in which case it would not be an ``affine semigroup'' and so would not satisfy the hypotheses from CLS.  \daniel{But I think that the normalization of $k[M'']$ might still work, because it'll just turn torision pieces into a direct sum.}

%\daniel{Okay latest edits stop here.  But here is my belief:  any primary component $P$ of $I$ will have $\pi: V(P)\to Y$ is surjective.  So if we want a prime ideal, we can choose any such $P$.  Or we can make no choices at all and work with $I$.  The normalization of $S/I$ will be the direct sum of the normalization of things like $S/P$ and so the projective dimension etc will be the same but the resolution might actually be different!  I think $I$ is the better choice for this article.  I think what we want to claim is that the normalization is an equidimensional direct sum of the coordinate rings of affine normal toric varieties.  Each component is CM of the right dimension and so the resolution has the right codimension.}
%Of course $M'\to \ZZ^r$ is also surjective. 
% Similarly, the map $M_Y\to \ZZ^r$ is injective.  We thus obtain a commutative diagram
%\[
%\xymatrix{
%k[M_X]\ar[r]\ar[d]&k[\ZZ^{\dim S}]\ar[d]\\
%k[M_Y]\ar[r]&k[\ZZ^{r}]
%}
%\]
%where the horizontal arrows are injections and the vertical arrows are surjections.  Let $P$ be the kernel of the righthand vertical the composition $S\to k[\ZZ^{\dim S}]\to k[\ZZ^r]$.  (It is prime because the image is a subring of an integral domain.)   Then $V(P)$ is a subvariety of $\Spec(S)$ which has a nonzero intersection with the torus, and thus it induces a sbuvariety of $\Spec(S)-V(B)$ and therefore also a subvariety of $X$. The dimension of $V(P)$ is $r$ because the righthand vertical arrow above is surjective.  Let's consider the variety $Z'\subseteq X$ corresponding to $V(P)$.  Write $\pi: \Spec(S) - V(B)\to X$. 
 %Since $Z'$ intersects the torus and any fiber over $T_X$ has dimension $\dim S - \dim X$, it follwos $Z'$ has dimension $r-(\dim S - \dim X) = \dim Y$.  In particular, $Z'$ has the same dimension as $Y$ and has a dense birational map to $Y$ (they agree on the torus of $Y$).  So both must be equal, I think?

\bibliographystyle{amsalpha}
\bibliography{Bibliography}




\Addresses









\end{document}