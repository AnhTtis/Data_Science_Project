\documentclass[12pt]{amsart}
\usepackage{amstext,amsfonts,amssymb,amscd,amsbsy,amsmath,verbatim, mathrsfs, fullpage}
\usepackage[alphabetic,abbrev,lite]{amsrefs} % for bibliography 
\usepackage{ifthen,tikz}
\usepackage{color}
\usepackage{amsthm}
\usepackage{latexsym}
\usepackage[all]{xy}
\usepackage{enumerate}
\usepackage{mathtools}
\setcounter{MaxMatrixCols}{15}


\DeclarePairedDelimiter\ceil{\lceil}{\rceil}
\DeclarePairedDelimiter\floor{\lfloor}{\rfloor}

\numberwithin{equation}{section}

\newtheorem{lemma}[equation]{Lemma}
\newtheorem{theorem}[equation]{Theorem}
\newtheorem*{gr}{Green's Linear Syzygy Theorem}
\newtheorem{propo}[equation]{Proposition}
\newtheorem{prop}[equation]{Proposition}
\newtheorem{cor}[equation]{Corollary}
\newtheorem{conj}[equation]{Conjecture}
\newtheorem{claim}[equation]{Claim}
%\newtheorem{claim*}{Claim}
\newtheorem{thm}[equation]{Theorem}
\newtheorem{notation}[equation]{Notation}
\newtheorem{convention}[equation]{Convention}
\newtheorem{question}[equation]{Question}


\theoremstyle{definition}
\newtheorem{defn}[equation]{Definition}
\newtheorem{setup}[equation]{Setup}
\newtheorem{example}[equation]{Example}
\newtheorem{construction}[equation]{Construction}
\newtheorem{algorithm}[equation]{Algorithm}
\newtheorem{warning}[equation]{Warning}
\newtheorem{conventions}[equation]{Conventions}

\theoremstyle{remark}
\newtheorem{remark}[equation]{Remark}
\newtheorem{remarks}[equation]{Remarks}
\newtheorem{rem}[equation]{Remark}
\newtheorem*{claim*}{Claim}
\newtheorem*{case*}{Case}

% Commands

\newcommand{\Kos}{\mathcal{K}}
\newcommand{\cS}{\mathcal{S}}
\newcommand{\rC}{\mathrm{C}}
\newcommand{\Cech}{\check{\mathrm{C}}}
\newcommand{\Tate}{{\mathbf{T}}}
\newcommand{\Tail}{{\mathbf{Tail}}}
\newcommand{\cC}{\mathcal{C}}
\newcommand{\cK}{\mathcal{K}}
\newcommand{\tot}{\operatorname{tot}}

\newcommand{\isom}{\cong}
\newcommand{\m}{\mathfrak m}
\newcommand{\PP}{\mathbb P}
\newcommand{\bD}{\mathbf D}
\newcommand{\df}{\operatorname{diff}}
\renewcommand{\P}{\PP}
\newcommand{\bA}{\mathbb A}
\newcommand{\A}{\bA}
\newcommand{\HH}{\mathrm H}
\newcommand{\GG}{\mathbb G}
\newcommand{\ZZ}{\mathbb Z}
\newcommand{\QQ}{\mathbb Q}
\newcommand{\bH}{\mathbf H}
\newcommand{\bF}{\mathbf F}
\newcommand{\DD}{\mathbf D}
\newcommand{\lideal}{\langle}
\newcommand{\rideal}{\rangle}
\newcommand{\initial}{\operatorname{in}}
\newcommand{\pdim}{\operatorname{pdim}}
\newcommand{\Hilb}{\operatorname{Hilb}}
\newcommand{\Spec}{\operatorname{Spec}}
\newcommand{\NE}{\overline{\operatorname{NE}}}
\newcommand{\Eff}{\operatorname{Eff}}
\newcommand{\im}{\operatorname{im}}
\newcommand{\NS}{\operatorname{NS}}
\newcommand{\Frac}{\operatorname{Frac}}
\newcommand{\ch}{\operatorname{char}}
\newcommand{\Proj}{\operatorname{Proj}}
\newcommand{\id}{\operatorname{id}}
\newcommand{\Div}{\operatorname{Div}}
\newcommand{\tr}{\operatorname{tr}}
\newcommand{\Tr}{\operatorname{Tr}}
\newcommand{\Supp}{\operatorname{Supp}}
\newcommand{\Gal}{\operatorname{Gal}}
\newcommand{\Pic}{\operatorname{Pic}}
\newcommand{\QQbar}{{\overline{\mathbb Q}}}
\newcommand{\Br}{\operatorname{Br}}
\newcommand{\Bl}{\operatorname{Bl}}
\newcommand{\Cox}{\operatorname{Cox}}
\newcommand{\Tor}{\operatorname{Tor}}
\newcommand{\diam}{\operatorname{diam}}
\newcommand{\Hom}{\operatorname{Hom}} %done
\newcommand{\sheafHom}{\mathcal{H}om}
\newcommand{\Gr}{\operatorname{Gr}}
\newcommand{\HF}{\operatorname{HF}}
\newcommand{\HP}{\operatorname{HP}}
\newcommand{\Osh}{{\mathcal O}}
\newcommand{\cO}{{\mathcal O}}
\newcommand{\kk}{{\bf k}}
\newcommand{\rank}{\operatorname{rank}}
\newcommand{\length}{\operatorname{length}}
\newcommand{\codim}{\operatorname{codim}}
\newcommand{\depth}{\operatorname{depth}}
%\newcommand{\FF}{\mathbb{F}}
\newcommand{\F}{\FF}
\newcommand{\Sym}{\operatorname{Sym}} %done
\newcommand{\GL}{{GL}}
\newcommand{\R}{\mathbb{R}}
\newcommand{\CC}{\mathbb{C}}
\newcommand{\Syz}{\operatorname{Syz}}
\newcommand{\Prob}{\operatorname{Prob}}
\newcommand{\defi}[1]{\textsf{#1}} % for defined terms
\newcommand{\Htot}{H_{\tot}}
\newcommand{\Ltot}{L_{\tot}}
\newcommand{\beq}{\begin{displaymath}}
\newcommand{\eeq}{\end{displaymath}}
\newcommand{\bs}{\backslash}
\newcommand{\ff}{\mathbf{f}}
\newcommand{\Gam}{\Gamma}
\newcommand{\Lotimes}{\overset{L}{\otimes}}



\newcommand{\Bmod}{\ensuremath{B_\text{mod}}}
\newcommand{\Bint}{\ensuremath{B_\text{int}}}
\newcommand\commentr[1]{{\color{red} \sf [#1]}}
\newcommand\commentb[1]{{\color{blue} \sf [#1]}}
\newcommand\commentm[1]{{\color{magenta} \sf [#1]}}
\newcommand{\daniel}[1]{{\color{blue} \sf $\clubsuit\clubsuit\clubsuit$ Daniel: [#1]}}
\newcommand{\michael}[1]{{\color{red} \sf $\clubsuit\clubsuit\clubsuit$ Michael: [#1]}}
\newcommand{\maya}[1]{{\color{green} \sf $\clubsuit\clubsuit\clubsuit$ Maya: [#1]}}

\def\edim{\operatorname{edim}}
\def\reg{\operatorname{reg}}

\newcommand{\ve}[1]{\ensuremath{\mathbf{#1}}}
\newcommand{\chr}{\ensuremath{\operatorname{char}}}

%Added by MB:
\def\nc{\newcommand}
\def\on{\operatorname}
\nc{\Q}{\mathbb{Q}}
\nc{\RR}{\mathbf{R}}
\nc{\LL}{\mathbf{L}}
\nc{\xra}{\xrightarrow}
\nc{\xla}{\xleftarrow}
\def\a{\alpha}
\def\om{\omega}
\def\Om{\Omega}
\def\DM{\operatorname{DM}}
\def\Coh{\operatorname{Coh}}
\def\Mod{\operatorname{Mod}}
\def\free{\operatorname{free}}
\def\QCoh{\operatorname{QCoh}}
\def\Cpx{\operatorname{Cpx}}
\def\th{\on{th}}
\def\F{\mathcal{F}}
\def\coker{\on{coker}}
\def\p{\partial}
\def\wt{\widetilde}
\def\i{\iota}
\nc{\into}{\hookrightarrow}
\nc{\onto}{\twoheadrightarrow}
\nc{\OO}{\mathcal{O}}
\nc{\Z}{\mathbb{Z}}
\nc{\cA}{\mathcal{A}}
\nc{\w}{\widehat}
\nc{\End}{\on{End}}
\nc{\res}{\frac{1}{x_0x_1}}
\nc{\tF}{\widetilde{F}}
\nc{\tG}{\widetilde{G}}
\nc{\tf}{\widetilde{f}}
\nc{\Com}{\on{Com}}


\nc{\G}{\mathbb{G}}
\nc{\cG}{\mathcal{G}}
\nc{\cE}{\mathcal E}
\nc{\cF}{\mathcal F}
\nc{\cR}{\mathcal R}
\nc{\cD}{\mathcal D}
\nc{\cB}{\mathcal B}
\nc{\cT}{\mathcal T}
\nc{\cL}{\mathcal L}
\def\M{\mathcal{M}}
\nc{\bM}{\mathbf M}
\nc{\bN}{\mathbf N}
\nc{\U}{\mathbf U}
\nc{\BM}{\mathbf B \mathbf M}
\nc{\Dsg}{\on{D}_{\on{sg}}}
\nc{\fC}{\mathcal{C}}
\nc{\fG}{\mathcal{G}}
\nc{\N}{\mathbb{N}}



%When merging files, add these
\nc{\del}{\partial}
\nc{\cone}{\on{cone}}
\nc{\D}{\on{D}_{\on{diff}}}
\nc{\DMb}{\on{D}^b_{\DM}}
\nc{\Db}{\on{D}^{\on{b}}}
\nc{\Kb}{\on{K}^{\on{b}}}
\nc{\fm}{\mathfrak{m}}
\nc{\Flag}{\on{Flag}}
\nc{\DMmin}{\DM_{\on{min}}}
\nc{\Ddiff}{\on{D}_{\on{diff}}}
\nc{\Dbdiff}{\on{D}^\on{b}_{\on{diff}}}
\nc{\wO}{\widehat{\OO}}
\nc{\wT}{\widehat{T}}
\nc{\from}{\leftarrow}
\nc{\wLL}{\widetilde{\LL}}
\nc{\augCech}{\widetilde{\cC}}
\nc{\Fold}{\on{Fold}}
\nc{\Ext}{\on{Ext}}
\nc{\FF}{\mathbf{F}}
\nc{\Comper}{\Com_{\on{per}}}
\nc{\Unfold}{\on{Unfold}}
\nc{\intHom}{\underline{\Hom}}
\nc{\Ex}{\on{Ex}}
\nc{\tg}{\widetilde{g}}
\def\tP{\widetilde{P}}
\def\b{\beta}
\nc{\B}{\mathcal{B}}
\nc{\K}{\mathcal{K}}
\nc{\kos}{\on{Kos}}
\nc{\Perf}{\on{Perf}}
\nc{\tR}{\widetilde{\cR}}
\nc{\X}{\mathcal{X}}
\nc{\Cl}{\on{Cl}}
\nc{\fU}{\mathcal{U}}
\nc{\bU}{\mathbf U}
\def\c{\colon}
\nc{\st}{\on{st}}
\def\E{\mathcal{E}}
\nc{\coh}{\on{coh}}
\def\ex{\on{ex}}
\def\D{\mathcal{D}}
\def\lin{\on{lin}}
\def\g{\gamma}
\nc{\tU}{\U}
\nc{\bC}{\mathbf{C}}
\nc{\aux}{\on{aux}}
\def\d{\mathbf{d}}
\def\phi{\varphi}
\def\cP{\mathcal{P}}
\def\I{\mathcal{I}}
\def\geo{\on{geo}}
\def\T{\mathbf{T}}
\nc\Dsing{\on{D}^{\on{sing}}}
\def\Y{\mathcal{Y}}

\title{Results on Virtual Resolutions for Toric Varieties}

\author{Michael K. Brown}
\author{Daniel Erman}

\newcommand{\Addresses}{{
	\vskip\baselineskip
  	\footnotesize
  	\noindent \textsc{Department of Mathematics and Statistics, Auburn University, Auburn, AL} \par\nopagebreak
	\noindent \textit{E-mail address:} \texttt{mkb0096@auburn.edu}
	\vskip\baselineskip
	\noindent \textsc{Department of Mathematics, University of Wisconsin-Madison, Madison, WI} \par\nopagebreak
	\noindent \textit{E-mail addresses:} \texttt{derman@math.wisc.edu}
  }}
  \thanks{The second author was supported by NSF grant 
DMS-2200469}

\date{\today}



\begin{document}

\maketitle




\begin{abstract}
We give a short, new proof of a recent result of Hanlon-Hicks-Lazarev about toric varieties.  As in their work, this leads to a proof of a conjecture of Berkesch-Erman-Smith on virtual resolutions and to a resolution of the diagonal in the simplicial case.
\end{abstract}

\section{Main result}
 We give a short, new proof of a recent result of Hanlon-Hicks-Lazarev about toric varieties and their multigraded Cox rings.  Throughout, we let $X$ be a simplicial, projective toric variety over a field $k$ with $\Cl(X)$-graded Cox ring $S$. Our main result (Theorem~\ref{thm:virtual hochster}) was first proven in~\cite{HHL}, but our proof is independent from their methods.
Our approach is more algebraic and simpler, but their approach is far more explicit and connects to a much wider range of topics of interest, e.g. symplectic geometry and homological mirror symmetry; see also the related work of Favero-Huang~\cite{FH}. 



Our interest in these topics begins with a program to extend results on syzygies to multigraded or toric settings.  The basic perspective, introduced by Berkesch-Erman-Smith in~\cite{BES}, is that many classical results about minimal free resolutions will have strong analogues in the toric setting, as long as one replaces minimal free resolutions with the more flexible notion of a virtual resolution.
\begin{defn}
\label{defn:virtual}
Let $M$ be a finitely generated $\Cl(X)$-graded $S$-module.  A \defi{virtual resolution} of $M$ is a free complex $F_\bullet$ of $S$-modules such that there is a quasi-isomorphism $\widetilde{F_\bullet} \xra{\simeq} \widetilde{M}$ of complexes of $\OO_X$-modules.
\end{defn}

The following theorem is a consequence of Hanlon-Hicks-Lazarev's main result~\cite[Theorem~A]{HHL}: 
 \begin{thm}\label{thm:virtual hochster}
Let $Y$ be a normal toric subvariety of $X$ with defining ideal $I\subseteq S$.  The $S$-module $S/I$ admits a virtual resolution of length equal to $\codim(Y\subseteq X)$.
\end{thm}

And here is our short proof of Theorem~\ref{thm:virtual hochster}.

\begin{proof}[Proof of Theorem~\ref{thm:virtual hochster}]
Let $R$ be the normalization of $S/I$ and $F_\bullet$ the minimal free resolution of $R$ as an $S$-module.  Since $Y$ is normal, $\widetilde{R}=\cO_Y$ as a sheaf on $X$, and so $F_\bullet$ is a virtual resolution of $S/I_Y$.  Hochster's Theorem implies that $R$ is a Cohen-Macaulay ring~\cite[Theorem 1]{hochster} and therefore also a Cohen-Macaulay $S$-module.  
The length of $F_\bullet$ is the projective dimension of $R$, which, by the Auslander-Buchsbaum Theorem~\cite[Theorem 19.9]{eisenbudbook}, equals the codimension of $S/I$, which is in turn equal to $\codim(Y\subseteq X)$.
\end{proof}

Let us briefly describe some applications of Theorem~\ref{thm:virtual hochster} and their history.  For a fuller discussion, see~\cite[\S1]{HHL}.  First, we have the following special case, first proven by Hanlon-Hicks-Lazarev:
\begin{thm}[Virtual Diagonal Theorem, \cite{HHL} Corollary B]\label{thm:virtual diagonal}
The coordinate ring of the diagonal embedding $X\subseteq X\times X$ admits a virtual resolution of length $\dim X$.
\end{thm}

It was known that this result would immediately yield proofs of two conjectures that had received independent interest. The first conjecture is due to Berkesch-Erman-Smith~\cite[Question 1.3]{BES} and was proven by Hanlon-Hicks-Lazarev:

\begin{thm}[Virtual Syzygy Theorem, \cite{HHL} Corollary C]\label{thm:virtual syzygy}
Any module $M$ as in Definition~\ref{defn:virtual} has a virtual resolution of length  $\leq \dim X$.
\end{thm}

Hilbert's Syzygy Theorem gives a bound of $\dim S=\dim X + \rank \Cl(X)$; Theorem~\ref{thm:virtual syzygy} implies that the added flexibility of virtual resolutions allows for significantly shorter resolutions, especially when $\rank \Cl(X)$ is large.  See~\cite{BES,HNV,berkesch-klein-loper-yang} and elsewhere for many examples of this phenomenon.   In~\cite{HHL}, Theorem~\ref{thm:virtual diagonal} is stated for smooth varieties, but as we will see, the basic ideas easily extend to the simplicial case.  Prior to their work, Theorem~\ref{thm:virtual syzygy} had been proven in several special cases: when $\rank \Pic(X)=1$ it essentially follows from Hilbert's Syzygy Theorem; for products of projective spaces it was shown in~\cite[Theorem~1.2]{BES} (see also \cite[Corollary~1.14]{EES}); Yang proved it for any monomial ideal in the Cox ring of a smooth toric variety~\cite{yang}; and Brown-Sayrafi proved it for smooth projective toric varieties of Picard rank 2~\cite{brown-sayrafi}.

The second conjecture, due to Orlov, is the special case of \cite[Conjecture 10]{orlov} for toric varieties. This was first proven by Favero-Huang in \cite[Theorem~1.2]{FH}, and independently and essentially simultaneously in ~\cite[Corollary E]{HHL}.
\begin{thm}\label{thm:rouquier}
Let $X$ be a normal toric variety.  The Rouquier dimension of $D^b(X)$ equals $\dim X$.
\end{thm}

Special cases of Theorem~\ref{thm:rouquier} had been established in~\cite{BC, BF, BDM, BFK} before Favero-Huang and Hanlon-Hicks-Lazarev proved it in general. The full version of Orlov's Conjecture states that Theorem~\ref{thm:rouquier} extends to any smooth quasi-projective variety; see \cite[\S 1.2]{BC} for a list of known cases of this conjecture. 

Theorem~\ref{thm:virtual hochster} easily implies Theorems~\ref{thm:virtual diagonal}, ~\ref{thm:virtual syzygy} and~\ref{thm:rouquier}. To prove Theorem~\ref{thm:virtual diagonal}, observe that the diagonal $X \subseteq X \times X$ satisfies the conditions of Theorem~\ref{thm:virtual hochster}. To prove Theorem~\ref{thm:virtual syzygy}, one can simply follow the method~\cite[Proof of Theorem~1.2]{BES}.  For Theorem~\ref{thm:rouquier}, one can use standard techniques on derived categories; see, e.g., the proof of ~\cite[Corollary E]{HHL}.

\medskip


Our proof of Theorem~\ref{thm:virtual hochster} is quite simple, perhaps embarrassingly so given the prior partial results on~\cite[Question 1.3]{BES} and the toric case of~\cite[Conjecture 10]{orlov}. It is not yet clear how to compare our resolutions to those obtained in~\cite{HHL}, but we believe that the two constructions agree. Their work gives a marvelously creative perspective on building these resolutions, making essential use of the symplectic side of the mirror symmetry functor, as well as a wide array of ideas, including discrete morse theory, quiver diagrams, and more. (See also Borisov's work \cite{borisov} for an alternative proof of Hochster's Theorem \cite[Theorem 1]{hochster}---the main ingredient of our proof of Theorem~\ref{thm:virtual hochster}---and an explanation of how the techniques used there relate to mirror symmetry.) The resolutions they obtain are quite explicit; indeed, their resolution of the diagonal yields a canonical generating set for the derived category of any normal toric variety, proving a claim of Bondal \cite[Corollary D]{HHL}. However, some algebraic aspects of their constructions are harder to determine.  For instance, if $F_\bullet$ is the free complex of $S$-modules corresponding to one of their resolutions, their work implies that the modules $H_i(F_\bullet)$ correspond to the zero sheaf on $X$ for all $i>0$, but it is not clear whether $H_i(F_\bullet)$ equals the zero module on the nose, i.e. it is not clear if $F_\bullet$ is acyclic as a complex of $S$-modules. The $S$-module that arises as $H_0(F_\bullet)$ is also unclear. By comparison, the complexes that arise in our construction are always acyclic, and we know which modules they resolve. However, we are not able to give as explicit of a description of the terms.  It would be very interesting to better compare these complexes, and to compare them with those in~\cite{BE, brown-sayrafi}.


\begin{remark}
As our resolutions from Theorem~\ref{thm:virtual hochster} rely only on standard algebraic constructions, they can be directly computed in {\em Macaulay2}.  The constructions in~\cite{HHL} are explicit, but due to their novelty, computing them in practice requires more effort.  Of course, if one could show that the two constructions coincide, this would shed more light on both.
\end{remark}






\section{Examples}
\begin{example}
Let $X=\PP^n$ and $T=k[x_0,\dots, x_n,y_0,\dots,y_n]$, the Cox ring of $X\times X$.  Let $I_\Delta\subseteq T$ be the defining ideal of the diagonal $X\subseteq X\times X$.  Since points in $\PP^n$ are equivalence classes of tuples $(x_0, \dots, x_n)$ up to scalar multiple, the diagonal is the closure of the locus of pairs $(x_0, \dots, x_n, tx_0, \dots, tx_n)$ where $t\in k^*$; this corresponds to the semigroup ring $k[x_0, \dots, x_n, tx_0, \dots, tx_n]$, which is normal.

The ideal $I_\Delta$ is generated by the $2\times 2$ minors of the matrix
\[
\begin{pmatrix}
x_0&x_1&\cdots &x_n\\
y_0&y_1&\cdots &y_n
\end{pmatrix}.
\]
More specifically: these minors vanish on $\Delta$, and since this is a generic matrix, the ideal of $2\times 2$ minors is prime of codimension $n$.  As $T/I_\Delta$ is already normal, the virtual resolution of $T/I_\Delta$ arising from Theorem~\ref{thm:virtual hochster} is just the minimal free resolution of $T/I_\Delta$, which is given by the Eagon-Northcott complex on this matrix.
\end{example}


\begin{example}\label{ex:P112 diagonal}
Let $X$ be the weighted projective space $\PP(1,1,2)$ and $T=k[x_0,x_1,x_2,y_0,y_1,y_2]$, the Cox ring of $X\times X$.  
The ring $T/I_\Delta$ is isomorphic to the semigroup ring
$$
k[x_0,x_1,x_2,tx_0,tx_1,t^2x_2],
$$ 
which is not normal because $tx_2$ lies in the fraction field and satisfies the integral equation $(tx_2)^2  - x_2 \cdot (t^2x_2)=0$.  Let $R$ be the normalization of $T/I_\Delta$. A presentation matrix for $R$ as a $T$-module is given as follows, where the rows correspond to the generators $1$ and $tx_2$:
\[
\bordermatrix{
&&&&& \cr
1 &x_{1}y_{0}-x_{0}y_{1}&x_{2}y_{0}&x_{2}y_{1}&x_{0}y_{2}&x_{1}y_{2}\cr
tx_2& 0&-x_{0}&-x_{1}&-y_{0}&-y_{1}
}.
\]

The free resolution of $R$ as a $T$-module is given by:
\begin{footnotesize}
\begin{equation}
\label{eq:res}
\begin{matrix} T \\ \oplus\\  T(-1,-1)\end{matrix} \xleftarrow{\left[\begin{smallmatrix} x_{1}y_{0}-x_{0}y_{1}&x_{2}y_{0}&x_{2}y_{1}&x_{0}y_{2}&x_{1}y_{2}\\
0&-x_{0}&-x_{1}&-y_{0}&-y_{1}
\end{smallmatrix}\right]} 
\begin{matrix}
T(-1,-1) \\
\oplus\\ 
T(-2,-1)^2 \\ \oplus \\
T(-1,-2)^2 
\end{matrix}
\xleftarrow{\left[\begin{smallmatrix} 
-x_{2}&0&-y_{2}\\
x_{1}&-y_{1}&0\\
-x_{0}&y_{0}&0\\
0&-x_{1}&-y_{1}\\
0&x_{0}&y_{0}
\end{smallmatrix}\right]}  
\begin{matrix}
T(-3,-1) \\
\oplus\\ 
T(-2,-2)\\ \oplus \\
T(-1,-3) 
\end{matrix}
 \gets 0.
\end{equation}
\end{footnotesize}

Additionally: we have the short exact sequence
$
0\to S/I_\Delta \to R \to Q \to 0,
$
and $Q = tx_2\cdot k[x_2,y_2]$.  One can directly compute that $Q$ determines the zero sheaf on $X \times X$. Indeed, since $tx_2$ has degree $(1,1)$, $Q_{(a,b)}\ne 0$ if and only if $a,b\geq 0$ and $a,b$ are both odd.  But a $T$-module determines the zero sheaf on $X \times X$  if and only if it is zero in degrees $(a,b)$ for all even integers $a,b \gg 0$.  
\end{example}



\begin{remark}
Since $\OO(-1)$ and $\OO(-3)$ are not vector bundles on $\PP(1, 1, 2)$, the resolution \eqref{eq:res} does not induce a locally free resolution of the diagonal. Indeed, virtual resolutions are not guaranteed to induce locally free resolutions of $\OO_X$-modules unless $X$ is smooth. Alternatively, as in~\cite{HHL}, one could consider the corresponding toric stack.
\end{remark}

\begin{remark}
In many of the prior known cases of Theorem~\ref{thm:virtual syzygy}, a slightly stronger result was proven.  Namely, it was shown that for any such $M$, there exists another module $M'$ satisfying $\widetilde{M}=\widetilde{M'}$ and $\pdim(M')\leq \dim X$; see~\cite{EES,bruce-heller-sayrafi,yang}.  It would be interesting if this was true in general. The fact that the virtual resolution of the diagonal from Theorem~\ref{thm:virtual diagonal} has this property makes it seem reasonable to hope that this might hold.
\end{remark}


\section*{Acknowledgments}  We are very grateful to Andrew Hanlon, Jeff Hicks, and Oleg Lazarev for patiently talking to us about their work and for several inspiring conversations. We only found this approach because of our efforts to understand their beautiful results.  




\bibliographystyle{amsalpha}
\bibliography{Bibliography}
\Addresses









\end{document}


