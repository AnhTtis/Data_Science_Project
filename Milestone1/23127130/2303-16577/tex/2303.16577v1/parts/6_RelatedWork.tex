We summarize research trends in schema design for query workload on relational databases and NoSQL databases.

Since relational databases and NoSQL databases have different characteristics, we categorize existing methods designed for relational and NoSQL databases.

\paragraph{Schema design on relational databases}
Schema design methods for time-depended workload were 
proposed~\cite{pavlo17, Jindal2018, CloudViews2018, Peregrine2019, Kossmann2019}.
Pavlo et al.~\cite{pavlo17} proposed Peloton, a self-driving database framework for in-memory databases.
Peloton introduces recending-horizon control model (RHCM) to predict workload changes and also proposes a method that adaptively changes schema according to the workload changes. 
Kossmann et al.~\cite{Kossmann2019} proposed a dynamic optimization method for self-managing database systems.
Their method uses linear programming to determine an efficient order to tune multiple dependent features, such as index selection, compression schemes, and data placement.

In addition, there are other works ~\cite{Wiese2008, VanAken2017} that dynamically change various tuning parameters in database systems. 
In Wiese et al.~\cite{Wiese2008}, the database administrator registers a tuning procedure, and the  procedure is automatically triggered when a given condition is met, such as when deadlocks occur more frequently than a given threshold.
Aken et al.~\cite{VanAken2017} proposed OtterTune, which automatically tunes the memory size, cache size, etc. by leveraging past experiences: it combines supervised and unsupervised learning methods to choose the most impactful tuning knobs. 

However, those methods are difficult to be applied to NoSQL databases, because there is no clear distinction between physical schema and logical schema in NoSQL databases.

\paragraph{Schema design on NoSQL databases}
There are also studies on schema migration in NoSQL databases.
NoSE~\cite{Mior2017} was proposed as a NoSQL schema design method for static workloads.
NoSE estimates the execution cost of static workload query, and optimizes the schema design.
However, since it does not support time-depended workloads, even if the schema is optimized once, the performance may deteriorate due to workload changes.

There are studies on schema migration for time-depended workloads~\cite{Hillenbrand2020, boncz2019, CONST2020}.
Hillenbrand et al.~\cite{Hillenbrand2020} proposed a method that enumerates multiple data migration patterns using an input schema migration, and then chooses the optimized data migration method based on a rule-based manner.

Google Napa~\cite{DBLP:journals/pvldb/AgiwalLMRSZZCCD21}
guarantees robust query performance. Clients expect low query latency and low variance in latency regardless of the query/data ingestion load.
It also provides users a flexibility to tune the system and meet their goals based on data freshness, resource costs, and query performance.