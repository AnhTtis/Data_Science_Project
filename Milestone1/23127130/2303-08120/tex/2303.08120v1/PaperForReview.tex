% CVPR 2023 Paper Template
% based on the CVPR template provided by Ming-Ming Cheng (https://github.com/MCG-NKU/CVPR_Template)
% modified and extended by Stefan Roth (stefan.roth@NOSPAMtu-darmstadt.de)

\documentclass[10pt,twocolumn,letterpaper]{article}

%%%%%%%%% PAPER TYPE  - PLEASE UPDATE FOR FINAL VERSION
% \usepackage[review]{cvpr}      % To produce the REVIEW version
% \usepackage{cvpr}              % To produce the CAMERA-READY version
\usepackage[pagenumbers]{cvpr} % To force page numbers, e.g. for an arXiv version

% Include other packages here, before hyperref.
\usepackage{graphicx}
\usepackage{amsmath}
\usepackage{amssymb}
\usepackage{booktabs}

\usepackage{multirow}
\usepackage{diagbox}
\usepackage{pifont}
\newcommand{\cmark}{\ding{51}}%
\newcommand{\xmark}{\ding{55}}%

\usepackage{graphbox} 
\usepackage{colortbl}
\usepackage{xcolor}
\usepackage{collcell}
\definecolor{mygray}{gray}{.9}

\newcolumntype{a}{>{\columncolor{mygray}}l}
\newcolumntype{b}{>{\columncolor{white}}c}

\usepackage{color}
\definecolor{citecolor}{HTML}{0071bc}
\usepackage[pagebackref=true,breaklinks=true,colorlinks,citecolor=citecolor,bookmarks=false]{hyperref}

% It is strongly recommended to use hyperref, especially for the review version.
% hyperref with option pagebackref eases the reviewers' job.
% Please disable hyperref *only* if you encounter grave issues, e.g. with the
% file validation for the camera-ready version.
%
% If you comment hyperref and then uncomment it, you should delete
% ReviewTempalte.aux before re-running LaTeX.
% (Or just hit 'q' on the first LaTeX run, let it finish, and you
%  should be clear).
% \usepackage[pagebackref,breaklinks,colorlinks]{hyperref}

% Support for easy cross-referencing
\usepackage[capitalize]{cleveref}
\crefname{section}{Sec.}{Secs.}
\Crefname{section}{Section}{Sections}
\Crefname{table}{Table}{Tables}
\crefname{table}{Tab.}{Tabs.}

%%%%%%%%% PAPER ID  - PLEASE UPDATE
\def\cvprPaperID{2460} % *** Enter the CVPR Paper ID here
\def\confName{CVPR}
\def\confYear{2023}

\usepackage{color}
\newcommand{\red}[1]{\textcolor{red}{#1}}

\begin{document}

%%%%%%%%% TITLE - PLEASE UPDATE
\title{Blind Video Deflickering by Neural Filtering with a Flawed Atlas}

% \author{First Author\\
% Institution1\\
% Institution1 address\\
% {\tt\small firstauthor@i1.org}
% % For a paper whose authors are all at the same institution,
% % omit the following lines up until the closing ``}''.
% % Additional authors and addresses can be added with ``\and'',
% % just like the second author.
% % To save space, use either the email address or home page, not both
% \and
% Second Author\\
% Institution2\\
% First line of institution2 address\\
% {\tt\small secondauthor@i2.org}
% }
\author{Chenyang Lei$^{1,2}$\thanks{Equal contribution} 
\quad  Xuanchi Ren$^{3,4}$\footnotemark[1]
\quad  Zhaoxiang Zhang$^1$ 
\quad  Qifeng Chen$^5$\\
$^1$CAIR, HKISI-CAS  \quad $^2$Princeton University  \quad $^3$University of Toronto \quad $^4$Vector Institute \quad $^5$HKUST \\
}
% \maketitle

\twocolumn[{%
\renewcommand\twocolumn[1][]{#1}%
\maketitle
\begin{center}
\vspace{-1.5 em}
\renewcommand\arraystretch{0.5} 
\centering
% \begin{tabular}{c@{\hspace{1mm}}c@{\hspace{0.5mm}}c@{\hspace{0.5mm}}c}
% \rotatebox{90}{\small \hspace{10mm} Input } &
% \includegraphics[width=0.32\linewidth]{Figure/teaser/Input/Betty/00000.jpg}&
% \includegraphics[width=0.32\linewidth]{Figure/teaser/Input/Betty/00009.jpg}&
% \includegraphics[width=0.32\linewidth]{Figure/teaser/Input/Betty/00035.jpg}\\
% \rotatebox{90}{\small \hspace{10mm} Output }&
% \includegraphics[width=0.32\linewidth]{Figure/teaser/Ours/Betty/00000.jpg}&
% \includegraphics[width=0.32\linewidth]{Figure/teaser/Ours/Betty/00009.jpg}&
% \includegraphics[width=0.32\linewidth]{Figure/teaser/Ours/Betty/00035.jpg}\\
% \end{tabular}

\begin{tabular}{c@{\hspace{1mm}}c@{\hspace{0.5mm}}c@{\hspace{0.5mm}}c@{\hspace{0.5mm}}c}
\rotatebox{90}{\small \hspace{8mm} Input } &
\includegraphics[width=0.24\linewidth]{Figure/teaser/Input/Betty/00000.jpg}&
\includegraphics[width=0.24\linewidth]{Figure/teaser/Input/Betty/00009.jpg}&
\includegraphics[width=0.24\linewidth]{Figure/teaser/Input/Betty/00035.jpg}&
\includegraphics[width=0.24\linewidth]{Figure/teaser/Input/Betty/00071.jpg}\\

\rotatebox{90}{\small \hspace{6mm} Output }&
\includegraphics[width=0.24\linewidth]{Figure/teaser/Ours/Betty/00000.jpg}&
\includegraphics[width=0.24\linewidth]{Figure/teaser/Ours/Betty/00009.jpg}&
\includegraphics[width=0.24\linewidth]{Figure/teaser/Ours/Betty/00035.jpg}&
\includegraphics[width=0.24\linewidth]{Figure/teaser/Ours/Betty/00071.jpg}\\
\end{tabular}

\vspace{-0.5em}
\captionof{figure}{\textbf{Blind deflickering performance of our approach on \textit{Betty Boop} (1934).} Many videos can have flickering artifacts for various
reasons. Our approach takes only the input video and removes the flicker automatically without any extra guidance.}
% \vspace{-2mm}
\label{fig:teaser}
\end{center}
}]

{
  \renewcommand{\thefootnote}%
    {\fnsymbol{footnote}}
  \footnotetext[1]{Equal contribution.}
}

\begin{abstract}
Many videos contain flickering artifacts; common causes
of flicker include video processing algorithms, video generation
algorithms, and capturing videos under specific situations.
Prior work usually requires specific guidance such
as the flickering frequency, manual annotations, or extra
consistent videos to remove the flicker. In this work, we
propose a general flicker removal framework that only receives
a single flickering video as input without additional
guidance. Since it is blind to a specific flickering type or
guidance, we name this “blind deflickering.” The core of
our approach is utilizing the neural atlas in cooperation
with a neural filtering strategy. The neural atlas is a unified
representation for all frames in a video that provides
temporal consistency guidance but is flawed in many cases.
To this end, a neural network is trained to mimic a filter
to learn the consistent features (e.g., color, brightness) and
avoid introducing the artifacts in the atlas. To validate our
method, we construct a dataset that contains diverse real-world
flickering videos. Extensive experiments show that
our method achieves satisfying deflickering performance
and even outperforms baselines that use extra guidance on
a public benchmark. The source code is publicly available at \url{https://chenyanglei.github.io/deflicker}.
\end{abstract}



%%%%%%%%% BODY TEXT

Stochastic optimal control and games have been extensively studied throughout the twentieth century and have found a wide range of applications in various areas such as finance, social sciences, operations research, and epidemic management problems, among others.
In recent years, computational methods for stochastic control and games have made great progress with the help of machine learning tools. A striking example of recent breakthroughs in applied mathematics using such tools is the numerical resolution of general nonlinear parabolic partial differential equations and backward stochastic differential equations in high dimensions~\cite{hutzenthaler2019multilevel, MR3736669,HaJeE:18,SiSp:18}. In short, stochastic control problems study how an agent optimally controls a stochastic dynamical system. The agent perceives some observations of the system's state and can decide to influence the evolution of the state based on these observations. The goal is to optimize an objective function that typically incorporates the cost of controlling the system and the reward for reaching some state. One of the most popular methods to solve such problems is dynamic programming, developed by Richard Bellman in the 1950s~\cite{bellman1957markovian}. However, this method suffers from what Bellman called the \emph{curse of dimensionality}, meaning that its complexity increases drastically with the number of possible states. This is a significant issue for systems that  evolve in continuous and high-dimensional spaces, since they can not be approximated by a small number of states. In such cases, using exact dynamic programming becomes computationally infeasible. 
Additionally, complexity can also arise from the structure of the system's evolution.
For example, in some cases, the system's evolution or its observation may be subject to delay, which appears in many real-world applications, {\it e.g.}, in economics, mechanics, or biology.  
To model the delay feature, the dynamics of the controlled system will depend not only on the current state but also on the history prior to the current time. This makes the problem path-dependent and, therefore, infinite-dimensional.

On the other hand, stochastic differential game theory, which was initiated by \cite{Isaacs1965}, combines theory and optimal control, and provides a framework for modeling and analyzing the behavior of strategic agents in the context of a dynamical system. The theory has been extensively employed in many disciplines, including management science,  economics, social science, and biology. One of the core objectives in differential games is to compute Nash equilibria, \textit{i.e.}, strategy profiles according to which no player has an incentive to deviate unilaterally~\cite{nash1951noncoop}. However, computing Nash equilibria in $N$-agent games is a notoriously hard problem, and direct computation of Nash equilibria is extremely demanding in terms of time and memory \cite{DaGoPa:2009} even for moderately large $N$. The mean-field game paradigm has been introduced independently by Lasry and Lions in \cite{LaLi:2007} and by Huang, Malham\'{e} and Caines in \cite{HuMaCa:06} to provide a tractable approximation for games with very large populations. A mean-field game is a game with a continuum of infinitesimal agents, where any single agent does not influence the rest of the population, which forms the mean field with which each agent interacts. This framework provides an efficient way to compute approximate Nash equilibria for symmetric $N$-agent games when $N$ is large. However, challenges remain in terms of computational complexity for games with high dimensional or complex environments, or when common noise affects the population dynamics. Furthermore, in the intermediate regime when $N$ is moderately large, the mean-field theory does not provide a good approximation of the $N$-player game. In such cases, one may still face a high-dimensional problem. To make these challenges more concrete, we discuss some illustrative examples.



\subsection{Some high-dimensional examples in applications}
We first discuss an example in financial markets. Many problems in economics or finance involve multiple interacting agents. For instance, we may consider a group of traders who buy and sell stocks in a financial market such as the S\&P 500, a free-float weighted measurement stock market index of 500 of the largest companies listed on stock exchanges in the United States. Each trader's portfolio describes the investment in stocks available on the market. To describe the investments of the whole group of traders, we need to incorporate all the traders' portfolios, and hence this description can be very high dimensional. However, if the traders have similar risk preferences, it is sufficient to study how one representative trader optimizes their payoff to understand the whole group's behavior. Any single agent has only a negligible impact on the stocks' prices. However, the impact of the group might be significant because if everyone wants to buy or sell the same stock, then the price will probably be shifted up or down. The portfolio optimization problem then becomes to find a Nash equilibrium. Each agent can anticipate that every other agent will behave like themselves and can thus predict the impact of the group on the stocks' prices. The mean-field game paradigm provides a rigorous framework in which each agent, taken individually, has no impact at all on the group's dynamics, and the problem is to find a fixed point at the population level. This simplifies the analysis. A first approach to describe the solution is through a forward-backward system of partial differential equations (PDEs) composed of a Fokker-Plank (FP) equation for the population distribution and a Hamilton-Jacobi-Bellman (HJB) equation for the value function of an infinitesimal player. However, several difficulties arise. Firstly, even solving the optimal control problem for the representative agent can be challenging: Given the number of stocks, the problem is in high dimension, and using an exact dynamic programming algorithm can be computationally too expensive. Secondly, even if the agent's portfolio state is in a low dimension, the optimal strategy may depend on the average state or control of the group. If the agents are trading the same stocks, then their trading strategies are all subject to the same source of randomness, which implies that the group's average strategy itself is stochastic. In the context of mean-field games, this is formalized through the notion of common noise. In such cases, the PDEs of the forward-backward system characterizing the solutions are no longer deterministic but stochastic {\cite{peng1992stochastichjb,carmonadelarue2014mastereqlarge}, making the system considerably harder to solve. Another approach \cite[Chapter~2]{CaDe2:17} characterizes the solution by a forward-backward system of stochastic differential equations (FBSDEs) which involves the conditional distribution of the forward and the backward processes, given the common noise. A third approach consists of describing the solution by a PDE on the space of probability distributions \cite{cardaliaguetdelaruelasrylions2019master}, but here again, this PDE is difficult to solve since it is posed on a high or infinite-dimensional space. In any case, in this application to optimal trading and as in many others, realistic models lead to the need to approximate functions of high or infinite-dimensional inputs. Altogether, this makes it crucial to develop efficient and accurate deep learning algorithms and theories for computing optimal controls and Nash equilibria in high dimensions.



A second example arises in infectious disease control of multiple regions. In a classic compartmental epidemiological model, each individual in a geographical region is assigned a label, for instance, \textbf{S}usceptible, \textbf{E}xposed, \textbf{I}nfectious, \textbf{R}emoved, \textbf{V}accinated. The transmission of a virus, being infected or recovered, moves individuals from one compartment to another, and this transition is usually described by stochastic dynamical equations. When a disease outbreak is reported, the region planner needs to take measurements to control its spread. The ongoing COVID-19 includes issuing lockdown or work from home policies, developing vaccines and later expanding equitable vaccine distribution, providing telehealth programs, distributing free testing kits, educating the public on how the virus transits, and focusing on surface disinfection. As a region planner, the decisions are usually made by weighting different costs, including the economic loss due to less productivity during lockdown policy or work from home policy, the economic value of life due to the death of infected individuals, and various social-welfare costs due to measurements mentioned above, and many more. Moreover, as the world is more interconnected than ever before, one region's decision will inevitably influence its neighboring regions. For instance, in the US, the decision made by the New York governor will affect the situation in New Jersey as so many people travel daily between the two states. Imagining that both state governors make decisions representing their own benefits but also take into account others' rational decisions, and they may even compete for scarce resources ({\it e.g.}, frontline workers and personal protective equipment), these are precisely the features of a non-cooperative game. A Nash equilibrium computed from such a game will definitely provide some qualitative guidance and insights for policymakers on the impact of certain policies. However, even with only three states (New York, New Jersey, and Pennsylvania) and a simple stochastic SEIR model as in \cite{xuan2020optimal,xuan2020ams}, this problem's state space is already twelve dimensions. Figure~\ref{fig:pandemic} below showcases the equilibrium lockdown policy corresponding to the multi-region SEIR model solved by a deep learning algorithm proposed in \cite{HaHu:19} (see Section~\ref{sec:MarkovianNE}) between the three states. The model parameters are estimated from real data posted by the Centers for Disease Control and Prevention (CDC). In general, the problem dimension is proportional to the number of compartments in the epidemiological model multiplied by the number of regions considered. For the most basic SIR model, the dimension of the problem for US governors will be $3 \times 50 = 150$. 


\begin{figure}[!htb]
    \centering
    \includegraphics[width = 0.75\textwidth, trim = {2em 2em 5em 6em}, clip, keepaspectratio=True]{figure/thetap99_a100.png}
    \caption{A case study of the COVID-19 pandemic in three states: New York (NY), New Jersey (NJ), and Pennsylvania (PA) in \cite{xuan2020optimal}. Plots of optimal policies (top-left), Susceptibles (top-right), Exposed (bottom-left), and Infectious (bottom-right) for three states: New York (blue), New Jersey (orange), and Pennsylvania (green). Large $\ell$ indicates high intensity of lockdown policy. Choices of parameters are referred to \cite[Section 4.2]{xuan2020optimal}. %
    }
    \label{fig:pandemic}
\end{figure}



\subsection{An illustrative linear quadratic model}
\label{sec:intro-LQsysrisk}
To illustrate numerical methods and show that they can correctly compute the problem's solution, it is convenient to have examples with analytical or semi-explicit solutions. We present here an example introduced in \cite{CaFoSu:15} to model the interactions in a system of banks. This model and other similar models with linear-quadratic structures admit a closed-form solution and have found applications in various fields.

We consider a stochastic differential game with $N$ players, and we denote by $\mathcal{I} =  \{1, 2, \ldots, N\}$ the set of players. Each player is interpreted as a bank and the state is its log-reserve.  Let $T$ be a finite time horizon. At each time $t \in [0,T]$,  player $i \in \mathcal{I}$ has a state $X_t^i \in \RR$ and takes an action $\alpha_t^i \in \RR$. The information structure will be discussed later and for now, we proceed informally but we can think of $\alpha^i_t$ as a stochastic process adapted to a filtration that represents the information available to player $i$. The dynamics of the controlled state process on $[0,T]$ are given by
$$
    \ud X_t^i = [a(\overline{X}_t - X_t^i)   + \alpha_t^i ] \ud t  + \sigma \left(\rho \ud W_t^0 + \sqrt{1-\rho^2} \ud W_t^i\right), \quad X_0^i \sim \mu_0, \quad i \in \mc{I}, \quad \overline{X}_t = \frac{1}{N}\sum_{i=1}^N X_t^i,
$$
where $\mu_0$ is a given initial distribution and $\bm{W} =[W^0, W^1,\ldots, W^N]$ are $(N+1)$ $m$-dimensional independent Brownian motions. %
We shall call $W^i$ the idiosyncratic (i.e., individual) noises and $W^0$ the common noise. The parameter $\rho \in[0,1]$ characterizes the noise correlation between agents. Here $a(\overline{X}_t - X_t^i)$ represents the rate at which bank $i$ borrows from or lends to other banks in the lending market, while $\alpha_t^i$  denotes its control rate of cash flows to a central bank. %
Furthermore, $\overline{X}_t = \frac{1}{N} \sum_{i=1}^N X^i_t$ denotes the average state. The $N$ dynamics are thus coupled since all the states $\bm{X}_t = [X_t^1, \ldots, X_t^N]$ affect the drift of every agent. Given a set of strategies $(\bm{\alpha}_t)_{t \in [0,T]} = ([\alpha_t^1, \ldots, \alpha_t^N])_{t \in [0,T]}$, the cost incurred to player $i$ is
\begin{equation}%
    J^i(\bm{\alpha}) = \EE\left[\int_0^T f^i(t, \bm X_t, \bm\alpha_t) \ud t + g^i(\bm X_T)\right],
\end{equation}
where the running cost $f^i: [0,T] \times \RR^{N}\times \RR^N \to \RR$ and the terminal cost $g^i: \RR^{N} \to \RR$ are given by
\begin{equation}
    f^i(t, \bm{x},\balpha) = \half (\alpha^i)^2 - q \alpha^i(\overline{x} - x^i) + \frac{\eps}{2}(\overline{x} - x^i)^2,  \quad g^i(\bm{x}) = \frac{c}{2}(\overline{x} - x^i)^2, \quad \overline{x} = \frac{1}{N} \sum_{i=1}^N x^i. 
\end{equation}
where $\bm{x} = [x^1, \ldots, x^N]$ and $\bm{\alpha} = [\alpha^1, \ldots, \alpha^N]$. 
All the parameters are non-negative. Here $\half(\ctrl^i)^2$ denotes the quadratic cost of the control, and $-q\ctrl^i(\overline{x} - x^i)$ models the incentive to borrowing or lending: Bank $i$ will want to borrow if $X_t^i$ is smaller than $\overline{X}_t$ and lend if $X_t^i$ is larger than $\overline{X}_t$. The quadratic term $(\overline{x} - x^i)^2$ in $f^i$ and $g^i$ penalizes the deviation from the average, given the other players' states. Player $i$ chooses $(\alpha_t^i)_{t \in [0,T]}$ to minimize her cost $J^i(\balpha)$ within some set of admissible strategies. We assume $q\le \epsilon^2$ so that the Hamiltonian is jointly convex in state and control variables, ensuring that there is at most one best response and then at most one Nash equilibrium. In the original work \cite[Section~3.1]{CaFoSu:15}, open-loop and closed-loop equilibria are characterized by semi-explicit formulas using ordinary differential equations.

As the number of agents $N$ grows to infinity, the idiosyncratic noises have a smaller and smaller influence on $\overline{X}$, which, in the limit, depends only on the common noise $W^0$. This is formalized in the following mean-field game (MFG). Let $(W_t)_{0 \leq t \leq T}$ and $(W_t^0)_{0 \leq t \leq T}$  be independent $m$-dimensional Brownian motions. We shall refer to $W$ as the {idiosyncratic noise} of the representative player and to $W^0$ as the {common noise} of the system. We consider the stochastic control problem
\begin{equation*} \inf_{\alpha} \EE\biggl\{\int_0^T \left[
    \frac{\alpha_t^2}{2}-q\alpha_t(m_t-X_t)+\frac{\epsilon}{2}(m_t-X_t)^2
    \right]\ud t +\frac{c}{2}(m_T-X_T)^2 \biggr\}, \end{equation*}
 \begin{equation}\label{def:LQ_SDE} \text{where }\quad\displaystyle \ud X_t = [a(m_t-X_t)+\alpha_t]\ud t  + \sigma(\rho \ud W_t^0 + \sqrt{1-\rho^2}\ud W_t), \quad X_0 \sim \mu_0,\end{equation} and the representative agent controls her state $X$ through a control process $\alpha$. Here $m_t = \EE[X_t|\mcF^{W^0}_t]$ is the conditional population mean given the common noise. As in the $N$-player case, one advantage of LQ models lies in the existence of an analytical solution for the mean-field equilibrium, which can provide a benchmark to test numerical algorithms. In this model, at equilibrium, we have
\begin{align}
    & m_t = \EE[X_0] + \rho\sigma W_t^0, \quad t\in[0,T], \label{def:LQ_m} \\
    & \alpha_t = (q+\eta_t) (m_t -X_t), 
    \quad t\in[0,T], \label{def:LQ_alpha}
\end{align}
where $\eta$ is a deterministic function of time solving the Riccati equation,
\begin{equation*}
    \dot \eta_t = 2(a+q)\eta_t + \eta_t^2 - (\eps - q^2), \quad \eta_T = c.
\end{equation*}
The solution is given by
\begin{equation*}
    \eta_t = \frac{-(\epsilon-q^2)(e^{(\delta^+-\delta^-)(T-t)}-1) -c(\delta^+e^{(\delta^+-\delta^-)(T-t)}-\delta^-)}{(\delta^-e^{(\delta^+-\delta^-)(T-t)} - \delta^+) -c(e^{(\delta^+-\delta^-)(T-t)} -1)},
\end{equation*}
where $\delta^\pm = -(a+q)\pm \sqrt{R}$, $R = (a+q)^2 + (\epsilon-q^2)>0$. At equilibrium, \textit{i.e.}, when all the payers use the equilibrium control, and the minimal expected cost for a representative player is 
\begin{equation*}
u(0, x_0 - \EE[x_0]),\quad\text{with}\quad    u(t,x) = \frac{\eta_t}{2}x^2 + \half \sigma^2(1-\rho^2)\int_t^T \eta_s \ud s.
\end{equation*}


Even though the state is in dimension one only, the presence of the common noise means that the optimal control is a function of the common noise. In this example, the equilibrium control actually depends on the common noise only through the first conditional moment of the distribution. In this case, the equilibrium can be found using neural networks which take as inputs not only the individual player's state but also an estimate of this first conditional moment. This idea can be extended to scenarios where the dependence on the common noise occurs only through a finite dimensional vector of information; see~\cite[Test cases 5 and 6]{carmona2019convergence2} for more details. However, this approach requires estimating aggregate quantities, for example by simulating a finite but large population of particles for many realizations of the common noise. When the interactions are through moments, another approach is to use only one realization of the idiosyncratic noise for each realization of the common noise: 
In \cite{MinHu:21}, based on the rough path theory, a single-loop algorithm called signatured deep fictitious play has been proposed. The proposed algorithm can accurately capture the effect of common uncertainty changes on mean-field equilibria without further training of neural networks, as previously needed in the existing machine learning algorithms. We will provide more details on this method in Section~\ref{sec4_MFG_with_CN} below. Figure~\ref{fig:LQ} showcases the performance for this LQ MFG with common noise, where the benchmark trajectories are simulated according to \eqref{def:LQ_SDE}  with $m_t$ and $\alpha_t$ in \eqref{def:LQ_m} and \eqref{def:LQ_alpha}. 

\begin{figure}[!htb]
    \centering
    \subfloat[$X_t$]{
         \includegraphics[width=0.3\columnwidth]{figure/SDE.pdf}}
    \subfloat[$m_t = \EE(X_t \vert \mcF_t^{W^0})$]{
         \includegraphics[width=0.3\columnwidth]{figure/mt.pdf}}
    \subfloat[Minimized Cost]{
         \includegraphics[width=0.3\columnwidth]{figure/valid_cost.pdf}}
    \caption{The illustrative linear quadratic model in Section~\ref{sec:intro-LQsysrisk}. Panels (a) and (b) give three trajectories of $X_t$, $m_t = \EE[X_t \vert \mcF_t^{W^0}]$ (solid lines) and their approximations $\widehat X_t$ (dashed lines) using different realizations of $(X_0, W, W^0)$ from validation data. Panel (c) shows the minimized cost computed using validation data over fictitious play iterations. 
    Parameter choices are given in \cite[Section 5]{MinHu:21}. %
    }
    \label{fig:LQ}
\end{figure}


\subsection{Organization of the survey}
In the rest of this survey, we review recent developments in machine learning methods and theory for stochastic control and differential games. We shall also identify unsolved challenges, and make connections to real applications. The topics are organized based on the number of players involved in the problem, first in the model-based setting, and then in the model-free setting.
We first review deep learning algorithms for stochastic control problems in Section~\ref{sec:SCP}, including mean field control problems, viewed as a type of control problem where the individual state’s dynamics are influenced by its own law. In Section~\ref{sec:SDG}, we focus on deep learning for stochastic differential games, including (moderately large) $N$-player games and mean-field games. In Section~\ref{sec:RL}, we review the basic principles underpinning model-free reinforcement learning methods for stochastic control and games. We make conclusive remarks and discuss unsolved challenges in Section~\ref{sec:conclusion}. In the appendix, we provide more background on two main tools of modern machine learning, namely, neural network architectures and stochastic gradient descent. We also summarize all the acronyms and the notations frequently used in this paper.

\medskip
\noindent{\bf The scope of the survey. }
This paper focuses on the recently developed neural network-based algorithms aiming at high-dimensional problems. The main reason why we focus on the stochastic setting is that in the deterministic setting, the problems can usually be tackled using open loop controls by reducing the problem to a two-point boundary value problem \cite{kierzenka2001bvp,zang2022machine}, which can be solved in high dimension without machine learning. 

Besides the methods reviewed in this paper, there is a rich literature on methods, some of them related to machine learning but without neural networks. For example, to cite just a few examples, Markov chain based methods \cite{budhiraja2007convergent}, regression based methods \cite{barrera2006numerical,bouchard2004discrete}, and approaches based on PDEs and BSDEs \cite{chen2008semi,forsyth2007numerical}; see, {\it e.g.}, %
the survey papers \cite{kushner1990numerical,jin2022survey} and the references therein. Furthermore, we focus mostly on standard classes of stochastic control problems and games but many other problems are considered in the literature. For instance, we do not discuss in this paper optimal switching and optimal stopping problems, for which numerical algorithms have been extensively developed, \emph{e.g.}, in \cite{kohler2010pricing,HuLudkovski17,becker2019deep,becker2020pricing,hu2020deep,hurephamwarin2020deep,lapeyre2021neural,reppen2022deep,reppen2022neural,gao2022convergence,bayraktar2022deep}. 
\section{Related Work}

\noindent \textbf{Task-specific deflikcering.} Different strategies are designed for specific flickering types. Kanj et al.~\cite{kanj2017flicker} propose a strategy for high-speed cameras. Delon et al.~\cite{delon2010stabilization} present a method for local contrast correction, which can be utilized for old movies and biological film sequences. Xu et al.~\cite{DBLP:conf/eccv/XuAH22} focus on temporal flickering artifacts from GAN-based editing. Videos processed by a specific image-to-image translation algorithm~\cite{isola2017image,CycleGAN2017,he2010single,zhang2016colorful,lei2020polarized,bell2014intrinsic,li2017universal,ouyang2021neural} can suffer from flickering artifacts, and blind video temporal consistency~\cite{bonneel2015blind,lang2012practical,lai2018learning,DBLP:conf/nips/dvp,yao2017occlusion,lei2022deep} is designed to remove the flicker for these processed videos. These approaches are blind to a specific image processing algorithm. However, the temporal consistency of generated frames is guided by a temporal consistent unprocessed video. Bonneel et al.\cite{bonneel2015blind} compute the gradient of input frames as guidance. Lai et al.\cite{lai2018learning} input two consecutive input frames as guidance. Lei et al.~\cite{DBLP:conf/nips/dvp} directly learn the mapping function between input and processed frames. While these approaches achieve satisfying performance on many tasks, a temporally consistent video is not always available. For example, for many flickering videos such as old movies and synthesized videos from video generation methods~\cite{ho2022imagen,singer2022make,zhou2022magicvideo}, the original videos are temporal inconsistent. A concurrent preprint~\cite{abs-2206-03753} attempts to eliminate the need for unprocessed videos during inference. Nonetheless, it still relies on optical flow for local temporal consistency and does not study other types of flickering videos. Our approach has a wider application compared with blind temporal consistency. 

Also, some commercial software~\cite{revision,flicker_free} can be used for deflickering by integrating various task-specific deflickering approaches. However, these approaches require users to have a knowledge background of the flickering types. Our approach aims to remove this requirement so that more videos can be processed efficiently for most users.





\noindent \textbf{Video mosaics and neural atlas.}
Inheriting from panoramic image stitching~\cite{BrownL07}, video mosaicing is a technique that organizes video data into a compact mosaic representation, especially for dynamic scenes.
It supports various applications, including video compression~\cite{IraniAH95}, video indexing~\cite{IraniA98}, video texture~\cite{AgarwalaZPACCSS05}, 2D to 3D conversion~\cite{RiberaCKLN12, SchnyderWS11}, and video editing~\cite{Rav-AchaKRF08}. 
Building video mosaics based on homography warping often fails to depict motions. To handle the dynamic contents, researchers compose foreground and background regions by spatio-temporal masks~\cite{CorreaM10}, or blend the video into a multi-scale tapestry in a patch-based manner~\cite{BarnesGSF10}. 
However, these approaches heavily rely on image appearance information and thus are sensitive to lighting changes and flicker. 

Recently, aiming for consistent video editing, Kasten et al.~\cite{kasten2021layered} propose Neural Layered Atlas (NLA), which decomposes a video into a set of atlases by learning mapping networks between atlases and video frames.
Editing on the atlas and then reconstructing frames from the atlas can achieve consistent video editing.
Follow-up work validates its power on text-driven video stylization~\cite{abs-2206-12396,text2live} and face video editing~\cite{LiuCLLJFG22}. 
Directly adopting NLA to blind deflickering tasks is not trivial and is mainly limited by two-fold: 
\textit{(i)} its performance on the complex scene is still not satisfying with notable artifacts.
\textit{(ii)} it requires segmentation masks as guidance for decomposing dynamic objects, and each dynamic object requires an additional mapping network. 
To facilitate automatic deflickering, we use a single-layered atlas without the need for segmentation masks by designing an effective neural filtering strategy for the flawed atlas. Our proposed strategy is also compatible with other atlas generation techniques. 


\begin{figure*}[t]
\centering
\begin{tabular}{@{}c@{}}
\includegraphics[width=1.0\linewidth]{Figure/method/merge_pipeline_5_crop.pdf}
\end{tabular}
\vspace{-0.5em}
\caption{\textbf{The framework of our approach}. We first generate an atlas as a unified representation of the whole video, providing consistent guidance for deflickering. Since the atlas is flawed, we then propose a neural filtering strategy to filter the flaws. }
\label{fig:framework}
\vspace{-1.2em}
\end{figure*}

\noindent \textbf{Implicit image/video representations.}
With the success of using the multi-layer perceptron (MLP) as a continuous implicit representation for 3D geometry~\cite{MeschederONNG19,nerf,ParkFSNL19}, such representation has gained popularity for representing images and videos~\cite{DBLP:conf/nips/SitzmannMBLW20, NERV, LiNSW21, TancikSMFRSRBN20}. Follow-up work extends these models to various tasks, such as image super-resolution~\cite{ChenL021}, video decomposition~\cite{DBLP:conf/cvpr/YeL0KS22}, and semantic segmentation~\cite{DBLP:conf/eccv/HuCXBCPW22}. 
In our work, we follow \cite{kasten2021layered} to employ a coordinate-based MLP to represent the neural atlases. 

\section{Method}
\label{sec:background}
\subsection{Overview}
\label{subsec:overview}

Let $\{I_t\}_{t=1}^T$ be the input video sequences with flickering artifacts where $T$ is the number of video frames. Our approach aims to remove the flickering artifacts to generate a temporally consistent video $\{O_t\}_{t=1}^T$. Flicker denotes a type of temporal inconsistency that correspondences in different frames share different features (e.g., color, brightness). The flicker can be either global or local in the spatial dimension, either long-term or short-term in the temporal dimension.



Figure~\ref{fig:framework} shows the framework of our approach. We tackle the problem of blind deflickering through the following key designs: 
$(i)$ We first propose to use a single \textit{neural atlas (Section~\ref{subsec:atlas})} for the deflickering task. 
$(ii)$ We design a \textit{neural filtering (Section~\ref{subsec:filter})} strategy for the neural atlas as the single atlas is inevitably flawed. 


% \textcolor{gray}{Enforcing long-term consistency across the whole video is challenging in blind deflickering. First, most architecture in video processing can only take a small number of frames as input, which results in a limited receptive field and is insufficient for long-term consistency. Secondly, optical flow estimation is more challenging for videos with flickering artifacts. As discussed in Bonneel et al.~\cite{bonneel2015blind}, simply warping the first frame with the optical flow to the last frame in a recurrent way is infeasible due to the accumulated errors of estimated flow. At last, unlike Lei et al.~\cite{DBLP:conf/nips/dvp}, blind deflickering methods do not have a temporally consistent video as guidance. 
% }




\subsection{Flawed Atlas Generation}
\label{subsec:atlas}

% What is the neural atlas?

% Why single layer?


\noindent\textbf{Motivation.}
A good blind deflickering model should have the capacity to track correspondences across all the video frames. Most architectures in video processing can only take a small number of frames as input, resulting in a limited receptive field that is insufficient for long-term consistency. To this end, we introduce neural atlases~\cite{kasten2021layered} to the deflickering task as we observe it perfectly fits this task. A neural atlas is a unified and concise representation of all the pixels in a video. As shown in Figure~\ref{fig:framework}(a), let $p =(x,y,t) \in \mathbb{R}^3 $ be a pixel that locates at $(x,y)$ in frame $I_t$. Each pixel $p$ is fed into a mapping network $\mathbb M$ that predicts a 2D coordinate $(u^p,v^p)$ representing the corresponding position of the pixel in the atlas. Ideally, the correspondences in different frames should share a pixel in the atlas, even if the color of the input pixel is different. That is to say, temporal consistency is ensured. 


% as correspondences in different frames share the pixel in the atlas.


% For a perfect atlas, every pixel in video frames can find a position in the atlas. 
% With this key design, our approach can solve a major challenge in blind deflickering: long-term temporal consistency without the guidance of consistent video. 
% In this part, we provide the details for the generation of the atlas. 
\noindent\textbf{Training.} Figure~\ref{fig:framework}(a) shows the pipeline to generate the atlas. For each pixel $p$, we have:
% \red{(why?)
\begin{align}
(u^p,v^p)&=\mathbb M(p), \\
c^p & = \mathbb{A}(\phi(u^p),\phi(v^p)).
\end{align}
The 2D coordinate $(u^p,v^p)$ is fed into an atlas network $\mathbb A$ with positional encoding function $\phi(\cdot)$ to estimate the RGB color $c^p$ for the pixel. The mapping network $\mathbb M$ and the atlas network $\mathbb A$ are trained jointly by optimizing the loss between the original RGB color and the predicted color $c^p$. 
Besides the reconstruction term, a consistency loss is also employed to encourage the corresponding pixels in different frames to be mapped to the same atlas position. We follow the implementation of loss functions in \cite{kasten2021layered}. 
% Note that the use of continuous implicit functions in extracting the atlas enables us to use noisy optical flow and deal with occlusion. 
% The predicted 2D coordinate $(u^p,v^p)$ from correspondences in different frames is trained to be consistent with a consistency loss. 
% This help us that we can use noise optical flow with continuous implicit function.
% actually this is a key difference with Task-agonistic temporal learning. 
After training the networks $\mathbb M$ and $\mathbb A$, we can reconstruct the videos $\{A_t\}_{t=1}^T$ by providing all the coordinates of pixels in the whole video. The reconstructed atlas-based video $\{A_t\}_{t=1}^T$ is temporally consistent as all the pixels are mapped from a single atlas. In this work, we utilize the temporal consistency property of this video for blind deflickering. 



Considering the trade-off between performance and efficiency, we only use a single-layer atlas to represent the whole video, although using two layers (background layer and foreground layer)  or multiple layers of atlases might slightly improve the performance. First, in practice, we notice the number of layers is quite different, which varies from a single layer to multiple layers (more than two), making it challenging to apply them to diverse videos automatically. Besides, we notice that artifacts and distortion are inevitable for many scenes, as discussed in \cite{kasten2021layered}. We present how to handle the artifacts in the flawed atlas with the following network designs in Secion~\ref{subsec:filter}.




\subsection{Neural Filtering and Refinement}
\label{subsec:filter}

\noindent \textbf{Motivation.} 
The neural atlas contains not only the treasure but also the trash. In Section~\ref{subsec:atlas}, we argue that an atlas is a strong cue for blind deflickering since it can provide consistent guidance across the whole video. However, the reconstructed frames from the atlas are flawed. First, as analyzed in NLA~\cite{kasten2021layered}, it cannot perform well when the object is moving rapidly, or multiple layers might be required for multiple objects. For blind deflickering, we need to remove the flicker and avoid introducing new artifacts. Secondly, the optical flow obtained by the flickering video is not accurate, which leads to more flaws in the atlas. At last, there are still some natural changes, such as shadow changes, and we hope to preserve the patterns. 

Hence, we design a \textit{neural filtering} strategy that utilizes the promising temporal consistency property of the atlas-based video and prevents the artifacts from destroying our output videos.




\begin{figure}[t]
\centering
\begin{tabular}{@{}c@{}}
% \includegraphics[width=0.9\linewidth]{LaTeX/Figures/sec4/framework.jpg}
\includegraphics[width=1.0\linewidth]{Figure/method/augumentation_pipeline_2_crop.pdf}
\end{tabular}
\vspace{-0.5em}
\caption{\textbf{Training pipeline of our neural filtering strategy.} We apply two transformations to a clean image $X$ for mimicking the flickering input frame and the flawed atlas frame.}
\label{fig:NeuralFilter}
\vspace{-1em}
\end{figure}

% why we use single-layer atlas instead of multi-layer atlas
% 1) multi need annotation
% 2) multi still have artifact (optinal) -> with an ablation study is good
% \paragraph{Discussion.} While our training scheme is inspired by NeuralAtlas, we choose different designs to achieve our goal. 
\noindent \textbf{Training Strategy.} 
Figure~\ref{fig:framework}(b) shows the framework to use the atlas. Given an input video $\{I_t\}_{t=1}^T$ and an atlas $A$, in every iteration, we get one frame $I_t$ from the video and input it to the filter network $\mathbb F$:
\begin{align}
    O_t^f = {\mathbb F}(I_t, A_t),
\end{align}
where $A_t$ is obtained by fetching pixels from the shared atlas $A$ with the coordinates of $I_t$. 

We design a dedicated training strategy for the flawed atlas, as shown in Figure~\ref{fig:NeuralFilter}. We train the network using only a single image $X$ instead of consecutive frames. In the training time, we apply a transformation $\tau_a(\cdot)$ to distort the appearance, including color, saturation, and brightness of the image, which mimics the flickering pattern in $I_t$. We apply another transformation $\tau_s(\cdot)$ to distort the structures of the image, mimicking the distortion of a flawed atlas-based frame in $A_t$.  
At last, the network is trained by minimizing the L2 loss function $\mathcal{L}$ between prediction ${\mathbb F}(\tau_a(X), \tau_s(X))$ and the clean ground truth $X$ (i.e., the image before augmentation):
\begin{align}
    \mathcal{L} = ||{\mathbb F}(\tau_a(X), \tau_s(X);\theta_{\mathbb F}) - X||_2^2,
\end{align}
where $\theta_{\mathbb F}$ is the parameters of filtering network $\mathbb F$. 

The network tends to learn the invariant part from two distorted views respectively. Specifically, ${\mathbb F}$ learns the structures from the input frame $I_t$ and the appearance (e.g., brightness, color) from the atlas frame $A_t$ as they are invariant to the structure distortion $\tau_s$. At the same time, the distortion of $\tau_s(X)$ would not be passed through the network $\mathbb F$. With this strategy, we achieve the goal of neural filtering with the flawed atlas.

Note that while this network ${\mathbb F}$ only receives one frame, long-term consistency can be enforced since the temporal information is encoded in the atlas-based frame $A_t$.

\noindent \textbf{Local refinement.}
\label{subsec:local}
The video frames $\{O_t^f\}_{t=1}^T$ are consistent with each other globally in both the short term and the long term. However, it might contain local flicker due to the misalignment between input and atlas-based frames. Hence, we use an extra local deflickering network to refine the results further. Prior work has shown that local flicker can be well addressed by a flow-based regularization. We thus choose a lightweight pipeline~\cite{lai2018learning} with modification. As shown in Figure~\ref{fig:framework}(b), we predict the output frame $O_t$ by providing two consecutive frames $O_t^f, O_{t-1}^f$ and previous output $O_{t-1}$ to our local refinement network ${\mathbb L}$. Two consecutive frames are firstly followed by a few convolution layers and then fused with the $O_{t-1}$. The local flickering network is trained with a simple temporal consistency loss $\mathcal L_{local} $ to remove local flickering artifacts:
\begin{align}
    \mathcal{L}_{local}(O_t,O_{t-1}) &=  ||M_{t,t-1} \odot (O_t - \hat{O}_{t-1})||_1 ,
\end{align}
where $\hat{O}_{t-1}$ is obtained by warping the $O_{t-1}$ with the optical flow from frame $t$ to frame $t-1$. $M_{t,t-1}$ is the corresponding occlusion mask. For the frames without local artifacts, the output should $O_t$ be the same as $O_t^f$. Hence, we also provide a reconstruction loss by minimizing the distance between $O_t$ and $O_t^f$ to regularize the quality.

% According to our experiments, while using this local deflickering network individually achieves poor performance, combining it with our atlas-guided network improves the performance effectively. 

\noindent \textbf{Implementation details.}
The network ${\mathbb F}$ is trained on the MS-COCO dataset~\cite{coco} as we only need images for training. We train it for $20$ epochs with a batch size of $8$. For the network ${\mathbb L}$, we train it on the processed DAVIS dataset~\cite{davis,lai2018learning} for $50$ epochs with a batch size of $8$. We adopt the Adam optimizer and set the learning rate to 0.0001.
% We notice train
% For more details, please refer to supplementary materials.
% We train the local network $l$ for 100 epochs with a batch size 8. We adopt the Adam optimizer and set the learning rate to 0.0001. According to our experiments, while using this local deflickering network individually achieve poor performance, combining it with our atlas-guided network improves the performance effectively. 




\begin{table*}[t]
\centering
\begin{tabular}{cc}%
\hfill
\begin{minipage}{0.62\textwidth}
\centering
\renewcommand{\arraystretch}{1.1}
% \resizebox{0.9\linewidth}{!}{ 
\begin{tabular}{lccccc}
\toprule
% \hline
% \rowcolor{white}
% \multirow{2}{*}{Task} 
 \cellcolor{white}Video type    & \multicolumn{3}{c}{$E_{warp} \downarrow$} & \multicolumn{2}{c}{ PSNR $\uparrow$}\\
% \rowcolor{white}
 \cellcolor{white} & {Raw video} & ConvLSTM &  {Ours} & ConvLSTM &  {Ours} \\
 \hline
 % \midrule
 Synthetic & 0.163 & 0.148 & \textbf{0.088} & 21.84 & \textbf{26.46} \\
  \qquad \footnotesize{-- $W = 1$}& 0.199 & 0.168 & \textbf{0.091} & 19.84 & \textbf{27.75} \\
 \qquad \footnotesize{-- $W = 3$}& 0.158 & 0.151 & \textbf{0.086} & 21.71 & \textbf{26.40} \\
 \qquad \footnotesize{-- $W = 10$} & 0.132 & 0.124 & \textbf{0.086} &  23.98 & \textbf{25.21} \\
 % \hline
 \hline
 Processed & 0.128 & 0.118 & \textbf{0.094} & -- & --\\

 \bottomrule
\end{tabular}
% }
\caption*{(a) Quantitative comparison} 

\label{fig:plogp}
\end{minipage}
&
\begin{minipage}{0.33\textwidth}
\centering
\renewcommand{\arraystretch}{1.1}
% \vspace{6mm}
% \resizebox{\linewidth}{!}{ 
% \begin{tabular}{a@{\hspace{3mm}}b@{\hspace{5mm}}b}
\begin{tabular}{lbb}
\toprule
% \hline
% \rowcolor{white}
\cellcolor{white}Video type & \multicolumn{2}{c}{Preference rate} \\
\cellcolor{white} & ConvLSTM & Ours  \\
% \midrule
\hline
{Old movies} & 38.0\% & \textbf{62.0\%} \\
{Old cartoons} & 33.6\% & \textbf{66.4\%} \\
Time-lapse & 31.9\% & \textbf{68.1\%} \\
Slow-motion & 29.0\% & \textbf{71.0\%} \\
\hline
Average & 33.7\% & \textbf{66.3\%} \\
\bottomrule
% \hline
\end{tabular}
% }
% \vspace{0mm}
\caption*{(b) User study} 
\end{minipage}%
\end{tabular}
% \vspace{-0.6em}
\caption{\textbf{Comparison to baselines.} We provide the (a) quantitative comparison for processed and synthetic videos since we have the high-quality optical flow for computing the evaluation metric. The warping errors of our approach are much smaller than the baseline, and our results are more similar to the ground truth on synthetic data, according to the PSNR. For the other real-world videos that cannot provide high-quality optical flow, we provide the (b) user study results for comparison. Our results are preferred by most users. } 
\label{table:comparison_ourdata}
% \vspace{-2mm}
\end{table*}


\begin{figure*}[t]
\centering
\begin{tabular}{@{}c@{\hspace{1mm}}c@{\hspace{1mm}}c@{\hspace{1mm}}c@{\hspace{1mm}}c@{\hspace{1mm}}c@{}}


\rotatebox{90}{\small \hspace{2mm}  }&
\includegraphics[width=0.230\linewidth]{Figure/results/Lily/input/00174.jpg}&
\includegraphics[width=0.230\linewidth]{Figure/results/Lily/convlstm/00174.jpg}&
\includegraphics[width=0.230\linewidth]{Figure/results/Lily/atlas/00175_resize.jpg}&
\includegraphics[width=0.230\linewidth]{Figure/results/Lily/ours/00174.jpg}\\
\rotatebox{90}{\small \hspace{2mm}  }&
\includegraphics[width=0.230\linewidth]{Figure/results/Lily/input/00192.jpg}&
\includegraphics[width=0.230\linewidth]{Figure/results/Lily/convlstm/00192.jpg}&
\includegraphics[width=0.230\linewidth]{Figure/results/Lily/atlas/00193_resize.jpg}&
\includegraphics[width=0.230\linewidth]{Figure/results/Lily/ours/00192.jpg}\\

\rotatebox{90}{\small \hspace{2mm}  }&
\includegraphics[width=0.230\linewidth]{Figure/results/synthetic_camel/input/00039.jpg}&
\includegraphics[width=0.230\linewidth]
{Figure/results/synthetic_camel/convlstm/00039.jpg}&
\includegraphics[width=0.230\linewidth]{Figure/results/synthetic_camel/atlas/00039.jpg}&
\includegraphics[width=0.230\linewidth,height=0.130\linewidth]{Figure/results/synthetic_camel/ours/00039.jpg}\\

\rotatebox{90}{\small \hspace{2mm}  }&
\includegraphics[width=0.230\linewidth]{Figure/results/synthetic_camel/input/00089.jpg}&
\includegraphics[width=0.230\linewidth]
{Figure/results/synthetic_camel/convlstm/00089.jpg}&
\includegraphics[width=0.230\linewidth]{Figure/results/synthetic_camel/atlas/00089.jpg}&
\includegraphics[width=0.230\linewidth,height=0.130\linewidth]{Figure/results/synthetic_camel/ours/00089.jpg}\\

\rotatebox{90}{\small \hspace{2mm}  }&
\includegraphics[width=0.230\linewidth]{Figure/results/old_movie_machester/input/00009.jpg}&
\includegraphics[width=0.230\linewidth]{Figure/results/old_movie_machester/convlstm/00009.jpg}&
\includegraphics[width=0.230\linewidth,height=0.126\linewidth]{Figure/results/old_movie_machester/atlas/00009.jpg}&
\includegraphics[width=0.230\linewidth]{Figure/results/old_movie_machester/ours/00009.jpg}\\

\rotatebox{90}{\small \hspace{2mm}  }&
\includegraphics[width=0.230\linewidth]{Figure/results/old_movie_machester/input/00057.jpg}&
\includegraphics[width=0.230\linewidth]{Figure/results/old_movie_machester/convlstm/00057.jpg}&
\includegraphics[width=0.230\linewidth,height=0.126\linewidth]
{Figure/results/old_movie_machester/atlas/00057.jpg}&
\includegraphics[width=0.230\linewidth]{Figure/results/old_movie_machester/ours/00057.jpg}\\

&Input & \small{ConvLSTM}  & \small{Ours atlas only} & \small{Ours} \\
\end{tabular}
\\
% \vspace{-0.5em}
\caption{\textbf{Qualitative comparisons to baselines.} Our results outperforms the baseline ConvLSTM significantly on various flickering videos. We highly encourage readers to see videos in our project website.}

\label{fig:baselines}
\vspace{-1em}
\end{figure*}


\section{Blind Deflickering Dataset}
\label{sec:dataset}
% What?
%
\begin{table*}[t]
\small
\centering
\renewcommand{\arraystretch}{1.1}
% \resizebox{\linewidth}{!}{
\begin{tabular}{lccccc}
% \hline
\toprule
% \multicolumn{1}{c}{Task}
\multirow{2}{*}{Task} & \multicolumn{5}{c}{$E_{warp} \downarrow$} \\
 & {Processed} & {Bonneel et al.}~\cite{bonneel2015blind} & Lai et al.~\cite{lai2018learning} & DVP~\cite{DBLP:conf/nips/dvp} &  {Ours} \\ 
\hline
\rowcolor{mygray}
w/o Extra Consistent Guidance & -- & {\color{red} \xmark} & {\color{red} \xmark} & {\color{red} \xmark} &  {\color{green} \cmark} \\
\hline
Dehazing~\cite{he2010single} & 0.120 & 0.128 & 0.136 & \underline{0.109} & \textbf{0.106} \\
Spatial White Balancing~\cite{hsu2008light}  & 0.087 & 0.081 & 0.098 & \underline{0.073} & \textbf{0.062}\\
Colorization~\cite{zhang2016colorful} & 0.109 & \underline{0.096} & {0.100} & {0.097} & \textbf{0.084} \\
Enhancement~\cite{gharbi2017deep} & 0.125 & {0.105} & 0.115 & \underline{0.102} & \textbf{0.093} \\
CycleGAN~\cite{CycleGAN2017} & 0.124 & 0.113 & 0.117 & \underline{0.103} & \textbf{0.099} \\
Intrinsic Decomposition & 0.131 & {0.085} & 0.108 & \underline{0.071} & \textbf{0.069} \\
Style Transfer & 0.202 & {0.161} & 0.177 &  \underline{0.143} & \textbf{0.142} \\
\hline
Average Score & 0.128  & {0.110} & 0.122 & \underline{0.100} & \textbf{0.094} \\
% \hline
\bottomrule
\end{tabular}
\vspace{-1mm}
\caption{\textbf{Qualitative comparison with blind video temporal consistency methods that use input videos as extra guidance.} While our approach does not use input videos as guidance, our method achieves better numerical performance compared with the baselines. }
\label{table:MainComparison}\
\vspace{-1.6em}
\end{table*}


We construct the first publicly available dataset for blind deflickering. 

\noindent \textbf{Real-world data.}
We first collect real-world videos that contain various types of flickering artifacts. Specifically, we collect five types of real-world flickering videos: 
\begin{itemize}
\setlength{\itemsep}{0pt}
\setlength{\parsep}{0pt}
\setlength{\parskip}{0pt}
    \item \textit{Old movies} contain complicated flickering patterns. The flicker is caused by multiple reasons, such as unstable exposure time and aging of film materials. Hence, the flickering can be high-frequency or low-frequency, globally and locally. 
    \item \textit{Old cartoons} are similar to old movies, but the structures differ from natural videos.
    \item\textit{Time-lapse} videos capture a scene for a long time, and the environment illumination usually changes a lot. 
    \item \textit{Slow-motion} videos can capture high-frequency changes in lighting. 
    \item \textit{Processed} videos denote the videos processed by various processing algorithms. The patterns of flicker are decided by the specific algorithm. We follow the setting in \cite{lai2018learning}. 
\end{itemize}


\begin{figure}[t]
\centering
\begin{tabular}{@{}c@{\hspace{1mm}}c@{\hspace{1mm}}c@{\hspace{1mm}}c@{\hspace{1mm}}c@{\hspace{1mm}}c@{}}


\rotatebox{90}{\small \hspace{0mm} w/o local  refinement}&
\includegraphics[height=0.34\linewidth]{wo_0_rec.png}&
\includegraphics[height=0.34\linewidth]{Figure/ready/wo_ready.png}\\
\rotatebox{90}{\small \hspace{7mm} Full model }&
\includegraphics[height=0.34\linewidth]{full_0_rec.png}&
\includegraphics[height=0.34\linewidth]{Figure/ready/full_ready.png}\\
\end{tabular}

\vspace{-0.5em}
\caption{\textbf{Ablation study for local refinement module.} The local refinement network is vital to remove the local flickering. Cropped images and their difference maps are placed at the third column.}
\vspace{-1.1em}
\label{fig:refine}
\end{figure}

\noindent \textbf{Synthetic data.}
While real-world videos are good for evaluating perceptual performance, they do not have ground truth for quantitative evaluation. Hence, we create a synthetic dataset that provides ground truth for quantitative analysis. Let $\{G_t\}_{t=1}^T$ be the clean video frames, the flickered video $\{G_t\}_{t=1}^T$ can be obtained by adding the flickering artifacts $F_t$ for each frame at time $t$:
\begin{align}
    I_t = G_t + F_t, 
\end{align}
where $\{F_t\}_{t=1}^T$ is the synthesized flickering artifacts. For the temporal dimension, we synthesize both short-term and long-term flicker. Specifically, we set a window size $W$, which denotes the number of frames that share the same flickering artifacts. We set the window size $W$ as 1, 3, 10 respectively.

\noindent\textbf{Summary.} We provide $20$, $10$, $10$, $10$, $157$, and $90$ for old movies, old cartoons, slow-motion videos, time-lapse videos, processed videos, and synthetic videos.






\section{Experiments}

\subsection{Evaluation Setup} 

\noindent \textbf{Dataset.} We mainly use our constructed Blind Deflickering Dataset for evaluation. Details are presented in Section~\ref{sec:dataset}.


\noindent \textbf{Evaluation metric.} We measure the temporal inconsistency based on the warping error used in DVP~\cite{DBLP:conf/nips/dvp} that considers both short-term and long-term warping errors for quantitative evaluation. Given a pair of frames $O_t$ and $O_s$, the warping error $E_{pair}$ can be calculated by:
\begin{align}
&E_{pair}(O_t,O_{s}) =  ||M_{t,s} \odot (O_t - \hat{O}_{s})||_1,    \\
&E_{warp}^t=E_{pair}(O_t,O_{t-1}) + E_{pair}(O_t, O_1),
\end{align}
where $\hat{O}_{s}$ is obtained by warping the $O_{s}$ with the optical flow from frame $t$ to frame $s$. $M_{t,s}$ is the occlusion mask from frame $t$ and frame $s$. For each frame $t$, the warping error $E_{warp}^t$ is computed with the previous and first frames in the video. We compute the warping error for all the frames in a video. 


\subsection{Comparisons to Baselines}
\noindent \textbf{Baseline.} As our approach is the first method for blind deflickering, no existing public method can be used for comparison. Thus, we design a baseline inspired by blind video temporal consistency approaches: \textit{ConvLSTM}, modified from Lai et al.~\cite{lai2018learning}. Specifically, we replace the consistent input pair of frames with flickered pair frames, and we retrain the ConvLSTM on their training dataset~\cite{lai2018learning}. 




\noindent \textbf{Results.} Table~\ref{table:comparison_ourdata}(a) provides quantitative results on two types of videos where we can estimate high-quality optical flow from consistent videos for computing the warping errors. Our results are consistently better than the baseline. In Table~\ref{table:comparison_ourdata}(b), we conduct a user study on Amazon Mechanical Turk following an A/B test protocol~\cite{ChenK17} to evaluate the perceptual preference between the main baseline ConvLSTM and our method. Each user needs to choose a video with better perceptual quality from videos processed by our method and the baseline. We use all real-world videos that cannot obtain high-quality optical flow for quantitative evaluation.
In total, we have $20$ users and $50$ pairs of comparisons.
Our method outperforms the baseline significantly in all tasks. 
Figure~\ref{fig:baselines} shows the qualitative comparisons between our approach and the baseline. Our approach removes various types of flicker, and our results are more temporal consistent than baselines.



\noindent \textbf{Comparisons to blind temporal consistency {methods}.} 
The comparison between our approach and blind video temporal consistency methods is unfair since our approach \textit{does not} require extra videos and baselines \textit{use} extra input videos as guidance. However, we still provide the comparison results for reference. Specifically, we use three state-of-the-art baselines, including Bonneel et al.~\cite{bonneel2015blind}, Lai et al.~\cite{lai2018learning} and DVP~\cite{DBLP:conf/nips/dvp}. Table~\ref{table:MainComparison} shows the quantitative results between our approach and baselines. The warping error of our approach on various processed videos is lower than all the baselines, which indicates that our approach is more temporal consistent even without the input video guidance. 
\begin{table}[t]
\begin{center}
\resizebox{0.8\linewidth}{!}{
\begin{tabular}{l@{\hspace{3em}}c}
\toprule
\multicolumn{1}{l}{Method} & $E_{warp} \downarrow$  \\
\midrule
Ours without atlas \& neural filtering & 0.131  \\
Ours without local refinement & 0.090  \\
Ours full model & \textbf{{0.088}} \\
\bottomrule
\end{tabular}
}
\end{center}
\vspace{-1.2em}
\caption{ \textbf{Quantitative results of ablation study}. The atlas and neural filtering strategy reduce temporal inconsistency significantly. The local refinement makes a slight difference in warping error but does improve perceptual performance.}
\label{table:ablation}
\vspace{-1em}
\end{table}

\subsection{Ablation Study}

We present the flawed atlas in the third column of  {Figure~\ref{fig:baselines}}. While the temporal consistency between frames is perfect, the atlas-based frames contain many artifacts and distortions. Hence, designing dedicated strategies is essential for blind deflickering. In this part, we analyze the importance of our two modules, respectively: (1) \textit{neural filtering} denotes the atlas generation and neural filtering since they cannot be split. (2) \textit{local refinement} denotes the local refinement module.


\noindent \textbf{Neural filtering.} We first analyze the importance of neural filtering by removing this part. We direct apply our local refinement module on the flickering input videos. As shown in Table~\ref{table:ablation}, removing the neural filtering module significantly increases temporal inconsistency. 

\noindent \textbf{Local refinement.} As shown in Table~\ref{table:ablation}, removing the local refinement network degraded the quantitative performance slightly. In Figure~\ref{fig:refine}, we show some local regions in the frames are temporal inconsistent. While these regions are small and the warping errors are only slightly reduced, this local flickering does hurt the perceptual performance, as shown in our supplementary materials.




\begin{figure}[t]
\centering
\begin{tabular}{@{}c@{\hspace{1mm}}c@{\hspace{1mm}}c@{}}

\includegraphics[width=0.320\linewidth]{Figure/results/experts/expert3/input/00020_rec.png}&
\includegraphics[width=0.320\linewidth]{Figure/results/experts/expert3/deflicker/00020.jpg}&
\includegraphics[width=0.320\linewidth]{Figure/results/experts/expert3/ours/00020.jpg} \\

\includegraphics[width=0.320\linewidth]{Figure/results/experts/expert3/input/00021_rec.png}&
\includegraphics[width=0.320\linewidth]{Figure/results/experts/expert3/deflicker/00021.jpg}&
\includegraphics[width=0.320\linewidth]{Figure/results/experts/expert3/ours/00021.jpg} \\
Input & Human experts & Ours \\

\end{tabular}
\vspace{-0.5em}
\caption{\textbf{Comparison to human experts}.
Input: the lower frame is redder than the upper one. Our method achieves comparable performance with human experts in this case. }%More comparisons are provided in the supplementary materials.
\label{fig:exper}
\vspace{-1.2em}
\end{figure}

\subsection{Comparisons to Human Experts}


We compare our approach to human experts that use commercial software for deflickering. We adopt the RE:Vision DE:Flicker commercial software~\cite{revision} for comparison, following Bonneel et al.~\cite{bonneel2015blind}. Specifically, we obtain the official demos processed by experts and compare them. Figure~\ref{fig:exper} shows the qualitative comparison results. We can see that our approach can obtain competitive results in a fully-automatic manner. Besides, as discussed in Bonneel et al.~\cite{bonneel2015blind}, the videos processed by new users are usually low-quality. 

\subsection{Discussion and Future Work}
\noindent\textbf{Potential applications.} Our model can be applied to all evaluated types of flickering videos. Besides, while our approach is designed for videos, it is possible to apply \textit{Blind Deflickering} for other tasks (e.g., novel view synthesis~\cite{nerf,xie2022high}) where flickering artifacts exist. 

\noindent\textbf{Temporal consistency beyond our scope.} Solving the temporal inconsistency of video content is beyond the scope of deflickering. For example, the contents obtained by video generation algorithms can be very different. Large scratches in old films can destroy the contents and result in unstable videos, which require extra restoration technique~\cite{wan2022bringing}. We leave the study for removing these temporally inconsistent artifacts to the future work.







\section{Conclusion}
In this paper, we define a problem named \textit{blind deflickering} that can remove diverse flickering artifacts without knowing the specific flickering type and extra guidance. We propose the first dedicated approach for this task. The core of our approach is to adopt a neural atlas with a neural filtering strategy. The neural atlas concisely extracts all pixels in the videos and provides strong guidance to enforce long-term consistency, but it is flawed in many regions. We then use a neural network to filter the flaws of the atlas for satisfying performance. We conduct extensive experiments to evaluate the deflickering performance. Results show that our approach outperforms baselines significantly on different datasets, and controlled experiments validate the effectiveness of our key designs.


\section*{Acknowledgement}
\noindent This work was supported by the InnoHK program.

% \clearpage

%%%%%%%%% REFERENCES
{\small
\bibliographystyle{ieee_fullname}
\bibliography{egbib}
}

\clearpage
% supplement
% detailed introduction of atlas paper

\end{document}