\subsection{Behavior When Design is Bug-Free} \label{subsec:no_bug_handling}

As mentioned in previous sections, the evaluation of this work assumes
that the presence of a performance bug has been detected.  However,
detection methodologies may still produce false-positives \emph{i.e.}
cases where a bug-free design is classified as buggy. In this section,
an analysis of the behavior of both methodologies when such case
occurs is conducted.

\subsubsection{False-Positives in CBC}

Each of the four ``test'' architectures without performance bugs
produce a different ranking of units. However, 
the confidence assigned to the highest ranked unit is
much smaller than the confidence obtained by architectures with bugs.

Although low confidence across all the units may signal a
false positive, it could also indicate a gap of coverage, as there
might be a bug in a unit that is not considered in the evaluated
classes.  To prevent that, a model to detect ``Bug-Free''
architectures is trained in the same way as the models for each possible
bug location unit (\emph{i.e.} a ``Bug-Free'' class is
included to $U$, the list of possible bug locations). Results show that
such model performs very well, it provides the highest ranked
confidence score to the bug-free class in all four bug-free
``test'' architectures. This addition does not degrade
results for buggy samples, as the confidence score for the
``Bug-free'' class is more than $200\times$ smaller than any 1st
choice of all the buggy samples and it is not included as a top-5
option in any sample with a performance bug.

\subsubsection{False-Positives P2BC}

For P2BC, we found that there were multiple units with high
classification confidence. In those cases there was nothing in
particular that allowed to differentiate these bug-free samples from
the average behavior observed in samples with bugs.

Similarly to the evaluated for CBC, a ``Bug-Free'' class addition
was tested. In this case however, the method does not provide a
satisfactory accuracy, making CBC more robust to handle false positive
cases. One possible explanation for this is that P2BC is more sensible to
bugs with small average IPC impact, which makes it more vulnerable
to false positives.

