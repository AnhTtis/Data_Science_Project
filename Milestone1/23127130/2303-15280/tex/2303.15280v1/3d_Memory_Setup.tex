\subsection{Bug Localization in Memory Systems} \label{subsec:setup_memory}

Although the focus of this work is the localization of performance
bugs in microprocessor cores, we evaluate the methodologies on
the cache memory system, in order to determine how the methodologies
perform in different setups. Both methodologies are
applied in the exact same manner, but there are minor differences in
the setup used for this evaluation.

Instead of gem5, the ChampSim~\cite{champsim} simulator is used, as it
provides a detailed memory system simulation with a much shorter
simulation time.  Further, by using ChampSim, we highlight the
robustness of our proposed approach.  A total of 96 SimPoints
extracted from 20 applications from the SPEC CPU2017~\cite{spec2017}
benchmark suite are used.  These traces were obtained from the Third
Data Prefetching Championship~\cite{dpc3}. Each of these SimPoints is
2B instructions long, but simulations are stopped after 1B
instructions have been executed. The performance counters are sampled
every 500k cycles due to the long traces being used.

\begin{scriptsize}
\begin{table}[!htb]
  \footnotesize
\centering
\caption{Performance bugs injected to ChampSim and their corresponding locations.}
\label{tab:bugs_champsim}
\resizebox{0.48\textwidth}{!}{%
\begin{tabular}{p{0.12\textwidth}|p{0.35\textwidth}}
  \hline
  \textbf{Bug location} & \textbf{Bug description} \\ \hline
  \multirow{2}{*}{Replacement Policy} & During a cache eviction, the policy evicts the most recently used block, instead of the least recently used. \\ \cline{2-2} 
                        & When a cache block is accessed, the age counter for the replacement policy is not updated. \\ \hline
  \multirow{2}{*}{Prefetcher} & On lookahead prefetching, the path with the least confidence is selected. \\ \cline{2-2} 
                        & Signature Path Prefetcher (SPP)~\cite{kim2016spp} signatures are reset, making the prefetcher use the wrong address. \\ \hline
  \multirow{3}{*}{Other  Operations} & After \textit{N} load misses on L1 data cache, the following L1 data read operation takes  \textit{T} additional clock cycles. \\ \cline{2-2} 
                        & After \textit{N} load misses on L2 cache, the following L2 read operation takes  \textit{T} additional clock cycles. \\ \cline{2-2} 
                        & If there are more than  \textit{Y} misses in less than  \textit{X} cycles, every read operation is delayed by  \textit{T} clock cycles. \\  \hline

\end{tabular}%
}
\end{table}
\end{scriptsize}

The emulated architectures are Intel Nehalem, Sandy Bridge, Ivy
Bridge, Haswell and Skylake, as well as AMD K10 and Ryzen7, and four
artificially generated configurations.  Ryzen7, Haswell and Skylake
are used as testing architectures, while the rest are used for
training the models.

The description of the performance bugs injected to ChampSim, along
with their corresponding locations on the design can be found in
Table~\ref{tab:bugs_champsim}.  Due to the limited number of bugs
available all the bugs are included in the ``seen'' bugs set.

