\begin{figure*}[!ht]
    \centering
     \begin{subfigure}[b]{0.285\textwidth}
     \includegraphics[width=0.98\textwidth]{figures/unseen_var_gt1.pdf}
       \caption{Unseen variations of seen bug types.}
        \label{fig:greater_1_var}
    \end{subfigure}
    ~ 
    \begin{subfigure}[b]{0.285\textwidth}
     \includegraphics[width=0.98\textwidth]{figures/unseen_type_gt1.pdf}
       \caption{Unseen bug types.}
        \label{fig:greater_1_type}
    \end{subfigure}
    ~
    \begin{subfigure}[b]{0.38\textwidth}
     \includegraphics[width=0.98\textwidth]{figures/seen_gt1.pdf}
        \caption{Seen bug variations.}
        \label{fig:greater_1_trained}
    \end{subfigure}
    \vskip 1ex
    \caption{Top-\emph{k} accuracy for bugs with average IPC impact $>$1\%.}
    \vskip 1ex
    \label{fig:per_ipc_1}
\end{figure*}

\section{Evaluation} \label{sec:evaluation}

This section presents the details of the experiments conducted to
evaluate our methodologies.  Sections~\ref{subsec:per_impact_accuracy}
through~\ref{subsec:no_bug_handling} show the results of the
methodologies when applied to the microprocessor core setup.
Section~\ref{subsec:memory_accuracy} presents the results of the
examination on memory systems.  Since we are not aware of prior work
in automatic performance bug localization, no comparison to prior work is
conducted.

Since the localization problem is formulated as a multi-class
classification task, both methodologies produce a sorted list from
highest to lowest probability of the performance bug being located in
each unit. The top-\emph{k} accuracy metric is used to measure results
quality. Here, a result is considered correct if the actual bug
location is found among the first \emph{k} choices suggested by our
techniques. The reason why top-\emph{k} accuracy is more relevant in
this work than other metrics (\emph{e.g.} a confusion matrix) is
because, rather than measuring when a sample is incorrectly classified
into a different class, it is more important to determine how high in
the predicted ranking the actual bug location is, as this represents
how many units the designer or validation engineer would have to search to actually find the
bug. Ideally, we would like the methodology to pinpoint to a single
unit (top-1 accuracy), but due to the challenging nature of the problem, we find 
top-k accuracy a good compromise.

Note that the accuracy of a random guess in a multi-class task is not
50\%, as in the more frequently studied binary classification, instead,
it is actually $1/|U|$ where $U$ is the set of all classes
(units). Therefore, the top-\emph{k} accuracy of a random guess is
given by $k/|U|$. The reported accuracy is obtained as the average
across the four architectures used as test cases.

It is important to note that bugs with small
average IPC impact are hard to catch, and do not represent a priority
for the designers as the gain by fixing them is not high.
Therefore, our main results, presented in Section~\ref{subsec:per_impact_accuracy}, focus in the 
analysis of the results obtained
on bugs with average IPC impact  $>$1\%. Section~\ref{subsec:overall_accuracy} shows the performance
of our methodology when smaller bugs (average IPC impact $>$0.1\%.) are considered.

All the methodologies were implemented in Python. For neural networks
Keras~\cite{chollet2015keras} is used, and for the gradient boosted decision
trees, XGBoost~\cite{chen2016xgboost} is used.



\subsection{Accuracy on High Impact Bugs} \label{subsec:per_impact_accuracy}

In this section, we show the top-\emph{k} accuracy across the implemented bugs
with an average IPC impact $>$1\% in our test architectures. Figure~\ref{fig:per_ipc_1} shows
the top-\emph{k} accuracy obtained by our proposed methodologies as the value of \emph{k}
increases. These results are separated using the bug partitioning scheme explained in Section~\ref{subsec:impl_bugs}.
Figure~\ref{fig:greater_1_var} shows the behavior across the 
``unseen variations of seen bug types'',
Figure~\ref{fig:greater_1_type} shows the results on the 
``unseen bug types'' and Figure~\ref{fig:greater_1_trained} shows the same, but
for the ``seen bug variations'' exclusively.  The figures show values of
\emph{k}$<$5, as longer lists of possible locations do not
provide significant help to designers to conduct their debug.

In the figures, ``CBC (GBDT)'' refers to the case of CBC with
per-time-step classification and gradient boosted tree models 
(100 trees per model, other parameters left at default).
The ``CBC (CNN)'' performs a per-trace classification using
convolutional neural networks, as described in
Section~\ref{subsec:one_stage_methodology}
(2 1D-CONV layers with 100 filters each, followed by
3 FC layers with 300, 100, 50 neurons with ReLU activation functions 
and a final output layer of a single neuron with sigmoid activation).
The ``P2BC'' results
correspond to an implementation with CNNs for the classifiers on the
second stage (the first stage uses GBDT with 250 trees, while the second
stage uses the same architecture as ``CBC (CNN)''). 
The results labeled as ``Random'' represent what would
be obtained with a random guess. The baseline for comparison should be the 
state of the art, which is manual debug. However, quantifying manual debug is difficult, as it varies significantly from person to person. We consider a random result a practically quantifiable baseline for comparison. Even if a manual debug is more accurate than random, given that our methodologies are automated, we are able to speed up the process significantly vs manual debug. 


In this case, CBC (GBDT) is the best performing technique, even better
than the ensemble. The accuracy obtained by the ensemble is
impacted due to P2BC having inferior accuracy for higher impact bugs
when compared to CBC (GBDT). 

Considering all the bugs (weighted average across the three bug partitions), our CBC (GBDT) methodology is able to achieve 76.8\% top-1 accuracy and over 90\% top-3 accuracy. As shown in  
Figure~\ref{fig:per_ipc_1}, results for ``unseen variations of seen bug types''
and ``seen bug variations'' are above that, both achieving a top-3 accuracy of about 95\%. As expected, the
``unseen bug types'' are the hardest to predict accurately, however, even
for these difficult cases of bugs that do not look similar to the ones
used for training, our CBC (GBDT) methodology achieves 76.9\%
top-3 accuracy.

The performance impact produced by a 
performance bug is not necessarily observed on the unit producing the bug. 
Our methodologies make no assumptions regarding this, in fact, the bug which incorrectly marks instructions as “serialized” is an example of how our methodologies are able to handle this. Our methods identify this type of bugs as occurring in the “Rename” stage, however serializing instructions will have an impact on the performance starting at the “issue” stage.

In our evaluation we found the units ``Branch'' and ``Memory'' 
to be the only ones with a localization rate much lower than the average. However, this is also
correlated with the fact that the implemented bugs for those two units had a much lower IPC
impact than the bugs on the other units. 



\begin{figure*}[h!]
  \centering
  \begin{subfigure}[b]{0.285\textwidth}
    \includegraphics[width=0.98\textwidth]{figures/unseen_var_gt01.pdf}
    \caption{Unseen variation of seen bug types.}
    \label{fig:unseen_var_acc}
  \end{subfigure}
  ~
  \begin{subfigure}[b]{0.285\textwidth}
    \includegraphics[width=0.98\textwidth]{figures/unseen_type_gt01.pdf}
    \caption{Unseen bug types.}
    \label{fig:unseen_type_acc}
  \end{subfigure}
  ~
  \begin{subfigure}[b]{0.38\textwidth}
    \includegraphics[width=0.98\textwidth]{figures/seen_gt01.pdf}
    \caption{Seen bug variations.}
    \label{fig:seen_acc}
  \end{subfigure}
  \vskip 1ex
  \caption{Top-\emph{k} accuracy for bugs with average IPC impact $>$0.1\%.}
  \vskip 1ex
  \label{fig:top_k_gt01}
\end{figure*}

\subsection{Accuracy on Low Impact Bugs} \label{subsec:overall_accuracy}

In this section, we analyze the accuracy of our methodologies
for bugs with smaller average IPC impact ($>0.1\%$).
The results for this case are shown in Figure~\ref{fig:top_k_gt01}.

Overall, across all the evaluated bugs, the Ensemble method provides
the highest top-3 accuracy with a 72.5\%. The advantages of the Ensemble
are prominent on the cases of ``unseen bug types'' (Figure~\ref{fig:unseen_type_acc}) and ``seen bug variations'' 
(Figure~\ref{fig:seen_acc}). In this case, just as observed in
Section~\ref{subsec:per_impact_accuracy}, the ``unseen bug types''
are the hardest to localize accurately. Although CBC (GBDT) performs
slightly better than the Ensemble for ``unseen
variations of seen bug types'' (Figure~\ref{fig:unseen_var_acc}), when
all the cases are considered, the Ensemble methodology outperforms it in
this average IPC impact.

As it can be observed by contrasting Figures~\ref{fig:per_ipc_1} and~\ref{fig:top_k_gt01}, the results degrade when bugs with
smaller impact are considered. However, we believe that the our 
results 
are satisfactory given that a 0.1\% IPC impact can be negligible in most
situations.

When each technique is used individually, CBC~(GBDT) is the
best performing, being able to identify the correct location of the
bug in the top-3 ranked options in around 80\% of the cases when a 
similar bug (but not the exact same) was used for training the models.
Although the accuracy achieved by P2BC is not as high as the obtained
by CBC, it provides satisfactory results while using $100\times$ less
storage.
\subsection{Number of Workloads} \label{subsec:nb_probes}

In this section, the impact of the number of workloads on the 
method's accuracy across bugs with average IPC impact greater than 0.1\%
is evaluated.  This evaluation was conducted for the ``CBC
(GBDT)'' implementation, as it was found to be the best performing
stand-alone method.  The experiment starts by using the 190 workloads available
for this work and measuring the top-1 accuracy obtained. The number of 
workloads is iteratively reduced by randomly choosing five to be
discarded in each iteration.  The iterations continue until only five
workloads remain.  This experiment is repeated 100 times to reduce the
impact of the random choices of workload deletion and obtain a
reliable trend.  Figure~\ref{fig:probe_coarse} shows the obtained
results.

\begin{figure}[htb!]
  \centering
  \includegraphics[width = 0.37\textwidth]{figures/ISCA_probe_overall.pdf}
   \caption{Impact of the number of workloads in the top-1 accuracy across all bugs.}
  \label{fig:probe_coarse}
\end{figure}

In the figure, the black line represents the average top-1 accuracy
across the 100 repetitions, while the maximum and minimum accuracy of
any individual iteration are represented by the shaded area.  The
figure shows that, as the number of workloads used to test the
methodology decreases, so does the average quality of results, making
a point that as more workloads are available, higher accuracy will be
achieved. However, we note the degradation in the overall accuracy is
slow and proves that the methodology can achieve satisfactory results
even with a reduced number of workloads.
\subsection{Behavior When Design is Bug-Free} \label{subsec:no_bug_handling}

As mentioned in previous sections, the evaluation of this work assumes
that the presence of a performance bug has been detected.  However,
detection methodologies may still produce false-positives \emph{i.e.}
cases where a bug-free design is classified as buggy. In this section,
an analysis of the behavior of both methodologies when such case
occurs is conducted.

\subsubsection{False-Positives in CBC}

Each of the four ``test'' architectures without performance bugs
produce a different ranking of units. However, 
the confidence assigned to the highest ranked unit is
much smaller than the confidence obtained by architectures with bugs.

Although low confidence across all the units may signal a
false positive, it could also indicate a gap of coverage, as there
might be a bug in a unit that is not considered in the evaluated
classes.  To prevent that, a model to detect ``Bug-Free''
architectures is trained in the same way as the models for each possible
bug location unit (\emph{i.e.} a ``Bug-Free'' class is
included to $U$, the list of possible bug locations). Results show that
such model performs very well, it provides the highest ranked
confidence score to the bug-free class in all four bug-free
``test'' architectures. This addition does not degrade
results for buggy samples, as the confidence score for the
``Bug-free'' class is more than $200\times$ smaller than any 1st
choice of all the buggy samples and it is not included as a top-5
option in any sample with a performance bug.

\subsubsection{False-Positives P2BC}

For P2BC, we found that there were multiple units with high
classification confidence. In those cases there was nothing in
particular that allowed to differentiate these bug-free samples from
the average behavior observed in samples with bugs.

Similarly to the evaluated for CBC, a ``Bug-Free'' class addition
was tested. In this case however, the method does not provide a
satisfactory accuracy, making CBC more robust to handle false positive
cases. One possible explanation for this is that P2BC is more sensible to
bugs with small average IPC impact, which makes it more vulnerable
to false positives.


\subsection{Bug localization in memory systems} \label{subsec:memory_accuracy}

For this setup, top-1 accuracy in both methodologies was
of 100\%. Cases of bug-free architecture obtain low confidence scores
in all possible locations, if a model to classify as
``Bug-Free'' using the same methodologies discussed in
Section~\ref{subsec:no_bug_handling} is included, it achieves
100\% detection rate without false negatives. Although this is a small setup for localization of bugs in memory
systems, its results show that the methodologies presented here are
robust for usage in different system components, as well as different
simulation frameworks.









