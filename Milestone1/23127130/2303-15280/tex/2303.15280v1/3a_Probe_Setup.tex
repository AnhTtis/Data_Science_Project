\subsection{Workload Setup} \label{subsec:probe_setup}

We extracted a total of 190 SimPoints~\cite{sherwood2002simpoints}  from
10 applications in the SPEC CPU2006
benchmark~\cite{spec2006} in order to use them as the set of workloads to
evaluate the proposed methodologies. Our methods consider each of these
SimPoints to be individual workloads. Although SPEC CPU2006 is used, nothing
prevents the usage of the more recent SPEC CPU2017\cite{spec2017},
other benchmark suites, or any custom-made workload the designers or
verification engineers consider relevant for performance validation of
the design under test.  We used SPEC CPU2006 for its shorter
runtimes, smaller memory footprint and greater compatibility with the
gem5~\cite{gem5} execution environment. Unlike cases
where performance improvement techniques are evaluated, here, the
benchmarks are used to extract orthogonal workloads, so complete suite execution is not required.

Each SimPoint contains $\sim$10M instructions, and the applications
used for SimPoint extraction are an arbitrary set of 10 that
were able to compile and run successfully in gem5 across the evaluated
microarchitectures. These applications are: perlbench, bzip2, gcc,
mcf, milc, cactusADM, namd, soplex, sjeng, and libquantum.

In our setup, performance counters are sampled and reset
every 100k cycles. The thresholds for counter selection discussed in
Section~\ref{subsec:counter_selection} are $\alpha = 0.7$ and
$\beta = 0.95$, and were empirically determined.
