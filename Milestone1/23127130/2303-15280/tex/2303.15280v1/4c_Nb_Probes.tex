\subsection{Number of Workloads} \label{subsec:nb_probes}

In this section, the impact of the number of workloads on the 
method's accuracy across bugs with average IPC impact greater than 0.1\%
is evaluated.  This evaluation was conducted for the ``CBC
(GBDT)'' implementation, as it was found to be the best performing
stand-alone method.  The experiment starts by using the 190 workloads available
for this work and measuring the top-1 accuracy obtained. The number of 
workloads is iteratively reduced by randomly choosing five to be
discarded in each iteration.  The iterations continue until only five
workloads remain.  This experiment is repeated 100 times to reduce the
impact of the random choices of workload deletion and obtain a
reliable trend.  Figure~\ref{fig:probe_coarse} shows the obtained
results.

\begin{figure}[htb!]
  \centering
  \includegraphics[width = 0.37\textwidth]{figures/ISCA_probe_overall.pdf}
   \caption{Impact of the number of workloads in the top-1 accuracy across all bugs.}
  \label{fig:probe_coarse}
\end{figure}

In the figure, the black line represents the average top-1 accuracy
across the 100 repetitions, while the maximum and minimum accuracy of
any individual iteration are represented by the shaded area.  The
figure shows that, as the number of workloads used to test the
methodology decreases, so does the average quality of results, making
a point that as more workloads are available, higher accuracy will be
achieved. However, we note the degradation in the overall accuracy is
slow and proves that the methodology can achieve satisfactory results
even with a reduced number of workloads.