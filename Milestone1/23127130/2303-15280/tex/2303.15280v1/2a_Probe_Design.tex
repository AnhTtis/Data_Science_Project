Similar to prior work in bug
detection~\cite{carvajal2021detection}, we leverage the use of
SimPoints~\cite{sherwood2002simpoints} in our performance bug
localization techniques, to identify orthogonal workloads from long
running applications, such as those on the SPEC CPU
suites~\cite{spec2006,spec2017}. With this, short and performance
orthogonal traces, that are relevant for microarchitecture performance
verification, can be automatically extracted. However, the
methodologies proposed here are not restricted to the usage of
SimPoints, and any workload that validation or design engineers
consider appropriate to verify the design can be incorporated and
should only improve the results, as discussed in Section~\ref{subsec:nb_probes}.

\subsection{Performance Counter Selection} \label{subsec:counter_selection}
A microprocessor
typically has hundreds or thousands of performance counters. Since using all of them makes the models unnecessarily large, a small subset is
obtained for each workload using an automated methodology that follows the two-step algorithm described below.
\begin{compactenum}
\itemsep0em 
\item The average Pearson correlation coefficient between each counter
  and the corresponding microbenchmark's IPC across multiple legacy
  architectures is calculated. Counters that are not highly correlated
  with IPC (magnitude lower than a threshold $\alpha$) are removed.
\item Correlation between each pair of the remaining counters is
  calculated. Two highly correlated
  counters (magnitude greater than $\beta$)
  will provide the model with redundant data, in that case,
  one of them is pruned from the list.
\end{compactenum}

The counter selection is completely orthogonal to the bugs that might be present on the system, the procedure is based entirely on the correlation between performance counters and 
IPC in legacy architectures. Although more sophisticated techniques  for automatic extraction of relevant performance events have been recently proposed~\cite{lv2018counterminer}, we find that Pearson correlation factor works sufficiently well in our setup.

We note that different counters are selected for each workload. Among the performance counters that are most frequently selected by our automated methodology we have the following: the number of fetched instructions, percentage of branch instructions, number of writes to registers, percentage of correctly predicted indirect branches, etc.
