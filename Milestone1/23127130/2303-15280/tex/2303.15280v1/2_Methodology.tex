
\section{Problem Definition and Scope}
\label{sec:scope}

%FORMULATION, OBJECTIVE, INPUT, ASSUMPTIONS

The \emph{objective} of this work is to identify the
microarchitectural units where a detected performance bug is located.
The \emph{scope} of this work, as presented here, is limited to
microarchitectural-level performance bugs in the processor core that
affect the IPC of the system\footnote{We note that, while we do not
test on components other than the processor core, there is no
reason to think that this methodology could not work there.}.
Bugs that might affect circuit-level timing (\emph{i.e.} clock
period) are not considered. For testing and insertion convenience, we cover cases where the hardware does not achieve the expected performance due to a microarchitecture bug, that is, due to waste or under-utilization of resources by implementation faults. The methodologies, however, can be used for performance bottlenecks arising from suboptimal hardware, algorithms, or other settings.

%Both of our proposed approaches require legacy
%designs with identified performance bugs in order to train the ML
%models.  Bug-free legacy designs are required only in one of the
%proposed methods, yet, if available, the other can take advantage of
%the additional data.  Due to the thorough pre- and post-silicon
%debug to which the designs are submitted, these legacy designs are
%usually available.
%We have limited the scope of our evaluation to
%buggy designs with a single bug at a time, in parallel to the
%single-fault model which is common practice in VLSI testing
%works. As early work on performance bug localization, we feel this
%represents a good first step towards solving this problem.
  
%The techniques presented here are evaluated using synthetic bugs,
%many of which are based on published errata, as such, they do not
%provide quantitative coverage guarantees. In general performance bug
%coverage is extremely difficult to define and is a potential
%research topic on its own. We know of no prior work which presents a
%definition of such coverage. Nonetheless, evaluated bugs cover
%a large amount of microarchitectural units and affect the system in
%a variety of ways. Thus, we feel these bugs represent a reasonable
%start for early work in this area.

%We also assume that there are no dramatic, structural
%microarchitectural changes between the legacy designs and the
%designs under debug. That said, even when larger shifts occur, the
%methodologies can be partially reused.  For example, consider the
%introduction of the AVX instructions with Intel's Sandy Bridge
%architecture in 2011.  Initially there would be no available data to
%test these instructions using our methodologies, however the rest of
%the Sandy Bridge design could be debugged with our methodology,
%leveraging workloads that do not exercise the new instructions.  In
%future implementations, data from Sandy Bridge can be used to build
%the models required to use our methods for debugging AVX.



\section{Methodology} \label{sec:methodology}

\subsection{Overview}

In this work, we propose two machine learning-based methodologies aiming to
identify the microarchitectural unit where a detected performance
bug exists. Here, we use the term microarchitectural
unit to refer us to a segment of a microprocessor that performs a
specific task. Examples of microarchitectural units in this context
would be Fetch, Decode, Branch Predictors, Load/Store Queue, etc.
  
Both of the proposed approaches leverage performance counter data as
inputs, since they are  available in almost any
microprocessor design.  This prevents the overhead of adding dedicated
data acquisition infrastructure.  In addition, microarchitecture
designs usually have hundreds or thousands of Performance
Counters~\cite{perfmon-intel,perf_counters_amd}, which are generally
sufficient to extract necessary information for performance
estimation. 
%Where prior work has only attempted to detect the existence of such bugs~\cite{carvajal2021detection}, this work aims to determine which functional unit contains that bug.

Here, we use ML due to its capability of knowledge extraction from
data, and its strength to handle complicated nonlinear behaviors.
Existing bug localization, in practice, is by and large manual, while
ML is perhaps the best approach to mimic a human manual process among
mathematical or algorithmic options. However, our goal is not to
remove the human from the debugging task, but merely speedup the
process by providing useful guidance extracted from the data.
Performance counters are used as input features to our ML-based
methodologies.  They are extracted from the results of simulating
(or executing) a diverse set of workloads.
%, which are used as stimuli applied to the microprocessor, in order to measure how it behaves under different circumstances.

The first proposed methodology uses multiple ML models,
each of which classifies if a given unit contains the bug. Then,
results of these models are aggregated across different workloads to
obtain an ordered list of units according to their confidence levels
for bug existence. This methodology is referred to as
\textbf{``Counter-Based Classification''} or \textbf{``CBC''}.

The second methodology is a two stage approach.  Its first stage
includes machine learning models for predicting bug-free performance
in terms of IPC. In the second stage, prediction errors of these
models are utilized for estimating the likelihood of bug existence for
each unit.  The name \textbf{``Performance Prediction error-Based
  Classification''} or \textbf{``P2BC''} is used to refer to this
method.

The output from either of the methodologies provides a priority list
of the most likely microarchitecture units that might contain the
performance bug, this list can be used for further analysis by
validation or design engineers.

Similar to prior work in bug
detection~\cite{carvajal2021detection}, we leverage the use of
SimPoints~\cite{sherwood2002simpoints} in our performance bug
localization techniques, to identify orthogonal workloads from long
running applications, such as those on the SPEC CPU
suites~\cite{spec2006,spec2017}. With this, short and performance
orthogonal traces, that are relevant for microarchitecture performance
verification, can be automatically extracted. However, the
methodologies proposed here are not restricted to the usage of
SimPoints, and any workload that validation or design engineers
consider appropriate to verify the design can be incorporated and
should only improve the results, as discussed in Section~\ref{subsec:nb_probes}.

\subsection{Performance Counter Selection} \label{subsec:counter_selection}
A microprocessor
typically has hundreds or thousands of performance counters. Since using all of them makes the models unnecessarily large, a small subset is
obtained for each workload using an automated methodology that follows the two-step algorithm described below.
\begin{compactenum}
\itemsep0em 
\item The average Pearson correlation coefficient between each counter
  and the corresponding microbenchmark's IPC across multiple legacy
  architectures is calculated. Counters that are not highly correlated
  with IPC (magnitude lower than a threshold $\alpha$) are removed.
\item Correlation between each pair of the remaining counters is
  calculated. Two highly correlated
  counters (magnitude greater than $\beta$)
  will provide the model with redundant data, in that case,
  one of them is pruned from the list.
\end{compactenum}

The counter selection is completely orthogonal to the bugs that might be present on the system, the procedure is based entirely on the correlation between performance counters and 
IPC in legacy architectures. Although more sophisticated techniques  for automatic extraction of relevant performance events have been recently proposed~\cite{lv2018counterminer}, we find that Pearson correlation factor works sufficiently well in our setup.

We note that different counters are selected for each workload. Among the performance counters that are most frequently selected by our automated methodology we have the following: the number of fetched instructions, percentage of branch instructions, number of writes to registers, percentage of correctly predicted indirect branches, etc.

\subsection{Methodology \#1: Counter Based Classification (CBC)} \label{subsec:one_stage_methodology}
\begin{figure*}[tb!]
  \centering
  \includegraphics[width = 0.65\textwidth]{figures/CBC_Micro2022.pdf}
  \caption{Overview of the CBC methodology.}
  \label{fig:one_stage_methodology}
\end{figure*}

The methodology proposed in this section consists of a single machine
learning stage followed by an aggregation procedure. An overview of
this methodology is shown in Figure~\ref{fig:one_stage_methodology}.

If the set of all the evaluated workloads is denoted by $W$ then,
for any given workload $w_i \in W$, we extract data from a set of
performance counters.  The counter data may be taken either from a
simulation or from the actual silicon chip.  The counters used are
those previously selected, using the methodology described in
Section~\ref{subsec:counter_selection}.

Although the automated counter selection can lead to a different
set of performance counters for each workload, CBC achieves better accuracy when every
model uses the super-set composed by the union of the counters
selected for each of the workloads.  This is because, for some
workloads, the selected counters do not contain any information
regarding specific units that might be affected by the performance
bug. Providing every workload with this larger counter set increases
the visibility the models have in all the units of the
system. Although using the union super-set significantly increases
the number of features per model by a factor of $\sim15\times$, this
merged set is still about $10\times$ smaller than the complete list
of available counters.

The data extracted from the performance counters is sampled and reset
every time a certain number of cycles have passed (\emph{e.g.}
every 100k cycles). Resetting the counters ensures 
that their value reflects only their
behavior in the current time-step, without keeping
track of its history. Therefore, the counter data for workload $w_i$ is
arranged as a time-series with $T_i$ elements. This time-series data is then feed into a 
ML classifier. 

In this work, we formulated the bug localization problem
as a multi-class classification task. Here, every 
microarchitectural unit where the bug
might be located corresponds to a class $u_j \in U$, where $U$ is the
set of all units.  To solve the multi-class classification task, some
common strategies in the ML community are one-vs-one (OvO)~\cite{ovo},
one-vs-all (OvA)~\cite{ova} or using a single multi-output model~\cite{bishop2006pattern}.  Although the usage of the latter
was evaluated, it did not achieve the desired
accuracy. A OvA strategy is followed, as it provides
good accuracy with a small fraction of the base classifiers that would be 
required
for a OvO strategy.  Here, a base classifier is a model to
classify if the bug exists in a specific unit for a workload.
With OvA, a total of $|W|\cdot|U|$ base models
(classifiers) are needed.

The base classifier for workload $w_i$ that flags the bug
in unit $u_j$ is denoted as $m_{i,j}$. Each $m_{i,j}$
produces a confidence score,
which is a soft classification in $[0,1]$ for workload $w_i$ to tell
if the bug is at unit $u_j$. Since the confidence scores from all
classifiers of a workload do not add up to one, they do not represent
probability. However, they serve as probability proxies, as higher
scores mean higher probability of the bug being in that unit. The
confidence scores for every unit across all the workloads are summed to
create a final score for each unit. The units with higher scores
have higher probability of the bug being present in
that unit. If sorted, the output provided by CBC represents a ranking of the
most likely units where the bug might be located.

Each base classifier $m_{i,j}$ is trained with data of workload $w_i$ from
one or more
legacy architectures. Samples from legacy architectures with bugs 
occurring at unit $u_j$ are considered ``positive'' cases for the model,
while bugs in any other unit are considered ``negative''. In this way,
the base classifier will learn to identify bugs exclusively on its 
corresponding unit. If available, samples from bug-free legacy architectures
can be used for training, and will be considered ``negative'' samples.

Because of the time-series format used for the performance counters, two different
methods to calculate the confidence score are evaluated and
contrasted:
\begin{compactenum}
\itemsep0em 
\item \emph{Per-trace classification}: Using neural networks that are
  able to take advantage of temporal locality, such as Convolutional
  Neural Networks (CNN)~\cite{lecun1995convolutional} or Long Short-Term
  Neural Networks (LSTM)~\cite{hochreiter1997lstm}, the complete time trace
  can be processed and a single score generated in the end. The
  proposed methodology uses CNNs, as LSTMs did not produce
  satisfactory results.

\item \emph{Per-time-step classification}: At every time-step $t_{i}$,
  a bug location prediction is performed by using the input features
  related to that specific time-step.  With this, a classification
  result per time-step is obtained, ultimately creating a
  ``classification trace''.  Previous time-steps could be added as
  input features in order to provide the models with information
  regarding counter history. However, we found that for the evaluated
  time-step size (100k cycles), adding these had no significant
  benefit.
  This method allows for other ML methods to be used, such as
  Multi-Layer Perceptrons~\cite{hornik1989multilayer}, Random
  Forest~\cite{breiman2001random} or Gradient Boosted Decision
  Trees (GBDT)~\cite{friedman2001greedy}.  Although all these methods were
  evaluated, the proposed methodology uses GBDT, as
  it was found the best performing.  Ultimately, to transform the
  prediction trace into a single value, the mean value
  of prediction scores
  across the
  whole time trace is used as the final score.
\end{compactenum}

\subsection{Methodology \#2: Performance Prediction error-Based Classification (P2BC)} \label{subsec:two_stage_methodology}

\begin{figure*}[htb!]
  \centering
   \includegraphics[width = 0.9\textwidth]{figures/P2BC_Micro2022.pdf}
  \caption{Overview of the P2BC methodology.}
  \label{fig:two_stage_methodology_overview}
\end{figure*}

The Performance Prediction error-Based Classification (P2BC) method is
composed of two different ML stages. An overview of this
is shown in Figure~\ref{fig:two_stage_methodology_overview}. Its first
stage is a set of performance (IPC) prediction models trained
exclusively with bug-free design data, so as to capture the
relationship between the counters and the performance under healthy
conditions.  When such models are applied on buggy designs,
significant prediction errors show up as the healthy conditions no
longer hold.  The second stage uses those prediction errors as
features for its ML classifiers. The intuition behind this approach is
that the specific workloads where the model and the ground truth diverge,
as well as the characteristics of that divergence, can provide
key information on which particular functional unit is the source of
the bug. Hence, P2BC localizes bugs according to symptoms of
performance model failures.

The first stage of the P2BC the methodology is heavily inspired by
prior work on performance bug detection~\cite{carvajal2021detection}, although the second stage of
P2BC differs significantly, as it leverages the full error trace to feed the ML models for
performance bug localization, while prior work used a single error
metric as input to a heuristic rule-based classifier merely for
determining bug existence.


\subsubsection{P2BC Stage 1: Performance Modeling}

Just as described in Section~\ref{subsec:one_stage_methodology}, we extract data from a set of performance counters, either via simulation
or from the actual silicon chip.
The goal of this stage is to model the bug-free processor
performance. To do this, we developed ML models 
that use the performance counter data as input features in order to infer
bug-free behavior of an specific target metric (IPC in this case).
There is one ML regression model for each workload, with counters selected according to
Section~\ref{subsec:counter_selection}. 

Unlike CBC where, in order to increase visibility, the super-set of
the counters selected across all the workloads are used as features
for the ML models, in P2BC each workload uses only
the counters selected specifically for it, as those are sufficient to infer the 
performance obtained by the system. Increasing the set of counters might provide 
undesired redundancy that will reduce the model sensitivity to performance bugs,
degrading the methodology's capacity to localize the bugs.

We use a per-workload model strategy since, due to their
specialization to the behavior of
each microbenchmark, it
achieves lower error (7.7\% RRMSE, in average for our setup) than workload agnostic models (27\%, in average~\cite{ardalani2015cross}). Further, our methodology does not support the usage of static analysis models~\cite{ardalani2019static}, as they are unlikely to trigger bugs due to not being executed on the actual design.

In order to train the bug-free IPC models, we use performance counter data coming exclusively
from bug-free legacy designs in order to establish a bug-free performance baseline. 
In discussions with industry partners, we
find that their availability can be safely assumed given the extensive post-silicon
debug that these designs have gone through by their end-of-life. However, even if it is not
possible to have designs that are completely bug-free, with P2BC we would be able to localize
new bugs that are not part of the baselines learned by the ML models.
That said, prior work~\cite{carvajal2021detection} demonstrated that even
when small performance bugs exist, performance prediction can perform
well.  Using different legacy designs ensures the model learns to
differentiate abnormal behavior due to performance bugs vs. due to
different microarchitectures.

Here, we use GBDT~\cite{friedman2001greedy} models
for performance prediction, which were shown to be the best 
technique for this task~\cite{carvajal2021detection}. These
performance models are not meant to be golden references, in fact,
they perform poorly for buggy designs. However, the prediction errors
contain useful information and serve as features for the 2nd stage
classifiers. Since the performance counters are processed in a time-series
manner, these models produce an inferred (or ML estimated) IPC time trace.

Once the inferred IPC trace is obtained, an error
trace is calculated by computing the difference between
the IPC obtained via ML performance models and the one obtained from
the simulations (or in-silicon execution). These error traces are used
as inputs for the second stage of this methodology.

\subsubsection{P2BC Stage 2: Error-based Bug Localization}

Since the error traces from stage one come from different workloads,
which might have different execution times,
the number of time-steps on each trace might differ between workloads. Therefore,
after error traces are obtained for all the
workloads, they go through a resampling procedure.
The goal of this procedure is to transform all the traces so that they all 
have a uniform number of time-steps $T_R$,
so that ML strategies like CNNs
can be used. This is only required due to the mixing of data from different
workloads, since that is not needed by CBC, resampling is not required in that 
method.

The resampling we use is the Fourier methodology implemented on the SciPy library~\cite{scipy}. This procedure is based on Nyquist-Shannon sampling theorem
\cite{shannon1949communication}.  The basic idea is that, by modifying
the frequency domain spectrum of a signal, the number of samples
needed to capture all the trace information can be changed as well.
Depending on the number of samples in the trace, there can be two
possible resampling mechanisms:

\begin{compactitem}
    \itemsep0em
  \item \textbf{Down-sampling}: This is when the number of 
    time-steps in $w_i$, denoted $T_i$, 
    is greater than $T_R$. In this case, after an FFT is
    applied, the frequency spectrum is truncated at the required
    maximum frequency. With this, the signal can be rebuilt with less
    samples via FFT\textsuperscript{-1}.
  \item \textbf{Up-sampling}: This is when $T_i < T_R$. Here, after an FFT is applied,
    the frequency spectrum is zero-padded.  Then, the signal can be
    rebuilt with more samples via  FFT\textsuperscript{-1}.
\end{compactitem}

Empirical results demonstrated that the best results can be achieved
when the target number of samples is equal to the average number of
time-steps across all the workloads. 

The resampled traces are used as inputs to a multi-class 
classifier which is trained to identify the
microarchitectural unit where a performance bug is located. A
``one-vs-all'' methodology is followed, same as in
Section~\ref{subsec:one_stage_methodology}.

The IPC inference errors obtained by using stage one in legacy architectures
are used to train this stage classifier. As opposed to CBC, where one classifier
per workload is needed, here we use a single OvA classifier. Each base classifier $m_{j}$
is trained with data from one or more legacy architectures, samples from legacy
architectures with bugs occurring at unit $u_j$ are considered
``positive'' cases for the model, while bugs in any other unit
are considered ``negative''. Note that the training of this stage, bug-free architectures
are not mandatory, as they are merely used as ``negative'' samples.

The classifiers are implemented using a 1D-CNN.  The convolution operations are performed exclusively
along different time-steps of the same workload, and every workload is
used as a different channel~\cite{lecun1995convolutional}. The reason
to do this is because there is no relevant information to be learned
across same time-steps of different workload given that the order in which
they appear is
completely arbitrary.

When an inference is performed, the confidence score of every model 
is proportional to the probability assigned by the model for a bug being present at each 
unit. Higher confidence scores are obtained for the most likely locations of the bug.

\subsection{Trade-offs} \label{subsec:tradeoffs}

These two methodologies have their advantages and drawbacks, these
are discussed as follows:
\begin{compactenum}
\itemsep0em 
\item \textbf{Accuracy vs data storage and runtime}: 
  As we show in Section~\ref{sec:evaluation}, CBC generally performs better than P2BC.  However, CBC
  requires a higher number of models than  P2BC. For the former, one model per unit per workload is needed
  ($|W| \cdot |U|$ in total), while the latter requires one model per
  workload for IPC estimation (Stage 1) and one model per unit for bug
  localization (Stage 2), for a total of $|W| + |U|$.  The increased number of
  models in CBC creates a larger data storage requirement
  ($100\times$ in our evaluation), as well as a longer runtime for
  training and inference ($6-10\times$).
  Nevertheless, although CBC is slower than P2BC, for both
  methodologies, full training can be achieved within
  a day, while the inference time is in the order of a just few seconds,
  without paralellization.

\item \textbf{Incremental workload addition}: CBC is more friendly to
  incrementally adding new workloads.  To add a new workload, $|U|$
  new models must be trained for this method. Although for P2BC the
  number of trained models is not significantly larger, only $|U|+1$ (one
  new IPC model, and $|U|$ re-trained models in stage two), adding a new workload
  requires to re-train the entire stage two, which could impact the
  quality of results and requires a longer training period.
    
\item \textbf{Incremental bug unit location addition}: Both approaches
  allow for incremental addition of microarchitectural units where
  bugs might be located.  This is useful for cases when a bug location
  that was not part of the initial list of considered classes needs to be included, or when a new structure is added
  to the system.  In this case, only one new model would need to be
  trained for P2BC, while CBC would need $|W|$. Although
  re-training the models for other units using designs with bugs in
  the new one is not required, doing so might improve the accuracy of
  the overall debugging approach.

\end{compactenum}

\subsection{Ensemble of both methods} \label{subsec:ensemble}

Although both methodologies provide satisfactory results, as shown in
Section~\ref{sec:evaluation}, they don't excel in the same cases.  In
order to take advantage of the strengths of both methods, a simple
ensemble scheme is presented in this section. The procedure is as
follows.

\begin{compactenum}
\itemsep0em 
\item The final confidence scores of each methodology are normalized,
  so that for every design, the sum of the confidences across all the
  units $u_j$ equals one.  Since a ``one-vs-all'' methodology is used,
  this cannot be guaranteed beforehand without this step.
\item For each unit, the average score obtained across both
  methodologies represents its final score.
\item This newly calculated scores are ranked, and their order is used
  to determine the unit with the highest probability to have a
  performance bug.
\end{compactenum}

Overall, the overhead of the proposed methodologies is very low, as ML model training takes about 30-60 minutes per model, and inference is in the order of seconds (this could be further accelerated with GPUs). Further, the simulation of each SimPoint takes 10-20 minutes to complete. Since all our models and simulations are independent, with enough computational resources, they can all be launched in parallel, producing a negligible overhead on the overall debugging process. 
