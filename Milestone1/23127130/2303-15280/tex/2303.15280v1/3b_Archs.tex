\subsection{Simulated Architectures} \label{subsec:sim_archs}

The experiments shown in Section~\ref{sec:evaluation} are based on
simulations performed in gem5~\cite{gem5} using the out-of-order core
model (O3CPU) and x86 ISA in system emulation mode.
%Although all the
%evaluations are based on simulations, the methodologies can also be
%applied in post-silicon debugging with minor modifications, as discussed in Section~\ref{sec:scope}.

We configured gem5 to emulate a diverse set of microarchitectures that
are used to train and test our scheme. The modified core related settings
are: clock period, pipeline width, branch
predictor, and the sizes of re-order buffer, load/store queue, and
instruction queue.  Cache related configurations (size, associativity,
latency, and number of levels), as well as functional unit
characteristics (count, latency, and port organization) are also
modified to map the simulator behavior to the emulated core.  Other
microarchitectural differences besides these are not considered.

In total, 23 different microarchitecture configurations were
implemented, 15 of them are based on multiple Stock-Keeping Unit (SKU)
variants of the Intel Core microarchitectures: Sandy Bridge, Ivy
Bridge, Haswell, Broadwell, Ice Lake and Skylake. The remaining
were artificially created, but use realistic settings.

For \textbf{training} purposes, we use the data from simulations of the eight artificial
microarchitectures, and the SKU variants of Sandy Bridge and Skylake.
To \textbf{test} the techniques we use the remaining four microarchitectures. As such, the test
microarchitectures are not used to train the ML models.

