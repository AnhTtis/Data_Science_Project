\section{Experimental Setup} \label{sec:experimental_setup}

This section elaborates on the evaluation setup of the
proposed methodologies.  Since performance bug localization in cores
is the focus of this work, sections~\ref{subsec:probe_setup}
to~\ref{subsec:impl_bugs} cover in detail its experimental setup
characteristics, while section~\ref{subsec:setup_memory} covers the
changes implemented to apply the proposed methodologies to a memory
system performance bug localization setup.

\subsection{Workload Setup} \label{subsec:probe_setup}

We extracted a total of 190 SimPoints~\cite{sherwood2002simpoints}  from
10 applications in the SPEC CPU2006
benchmark~\cite{spec2006} in order to use them as the set of workloads to
evaluate the proposed methodologies. Our methods consider each of these
SimPoints to be individual workloads. Although SPEC CPU2006 is used, nothing
prevents the usage of the more recent SPEC CPU2017\cite{spec2017},
other benchmark suites, or any custom-made workload the designers or
verification engineers consider relevant for performance validation of
the design under test.  We used SPEC CPU2006 for its shorter
runtimes, smaller memory footprint and greater compatibility with the
gem5~\cite{gem5} execution environment. Unlike cases
where performance improvement techniques are evaluated, here, the
benchmarks are used to extract orthogonal workloads, so complete suite execution is not required.

Each SimPoint contains $\sim$10M instructions, and the applications
used for SimPoint extraction are an arbitrary set of 10 that
were able to compile and run successfully in gem5 across the evaluated
microarchitectures. These applications are: perlbench, bzip2, gcc,
mcf, milc, cactusADM, namd, soplex, sjeng, and libquantum.

In our setup, performance counters are sampled and reset
every 100k cycles. The thresholds for counter selection discussed in
Section~\ref{subsec:counter_selection} are $\alpha = 0.7$ and
$\beta = 0.95$, and were empirically determined.


\begin{scriptsize}
  \begin{table*}[!htb]
  \footnotesize
    \centering
    \caption{Performance bug types injected in gem5 and their
      corresponding locations. Multiple variations of each type were
      implemented for this evaluation.}
    \label{tab:bugs_gem5}
    \resizebox{0.99\textwidth}{!}{
\begin{tabular}{p{0.1\textwidth}|p{0.9\textwidth}}
  \hline
  \textbf{Bug location} & \textbf{Bug type description} \\ \hline
  \multicolumn{1}{l|}{\multirow{2}{*}{Fetch}} & Fetching instructions from the instruction cache takes \textit{T} cycles longer than expected. \\ \cline{2-2} 
  \multicolumn{1}{l|}{} &  Every \textit{T} cycles, the maximum number of instructions that the processor is able to fetch is reduced by \textit{N} entries during one cycle. \\ \hline
  \multicolumn{1}{l|}{Decode} & Instructions that require no source operands are delayed by \textit{T} cycles on the decode stage. \\ \hline
  \multicolumn{1}{l|}{\multirow{5}{*}{Issue}} & If an instruction with opcode \textit{X} reaches the front of the instruction queue, meaning that it has become the oldest instruction there, then the issue process is stalled until the instruction can be issued (all the dependencies have been met, and computational resources are available). Once this occurs, only that instruction leaves the queue during that cycle. Normal behavior is resumed afterward. \\ \cline{2-2} 
  \multicolumn{1}{l|}{} & Every instruction whose opcode is \textit{X} can be retired from the instruction queue only when it becomes the oldest instruction there. A similar bug was found in the Intel Xeon Processors errata~\cite{intel_xeon_errata}. Its description can be found under ``POPCNT instruction may take longer to execute than expected''. \\ \cline{2-2} 
  \multicolumn{1}{l|}{} & If the operands of instructions with opcode \textit{X} depend on the result of an instruction with opcode \textit{Y}, the issuing of the former is stalled by \textit{T} cycles after its operands are ready. \\ \cline{2-2} 
  \multicolumn{1}{l|}{} & If less than \textit{N} slots are available in the instruction queue, delay the next instruction by \textit{T} cycles. \\ \cline{2-2} 
  \multicolumn{1}{l|}{} & The pointer signaling the front of the instruction queue is randomly shifted by \textit{N} positions. This event occurs with a frequency of \textit{T} times per 1000 cycles. \\ \hline
  \multicolumn{1}{l|}{Rename} & All instructions whose opcode is \textit{X} are marked as serializing instructions. This causes all subsequent instructions to be stalled until that instruction has been issued. \\ \hline
  \multicolumn{1}{l|}{\multirow{3}{*}{Execute}} & The latency of functional units handling integer operation \textit{X} is increased by \textit{T} cycles. \\ \cline{2-2} 
  \multicolumn{1}{l|}{} & The latency of functional units handling floating-point operation \textit{X} is increased by  \textit{T} cycles. \\ \cline{2-2} 
  \multicolumn{1}{l|}{} & The latency of functional units handling  ``Single-Instruction Multiple-Data'' operations  \textit{X} is increased by \textit{T} cycles. \\ \hline
  \multicolumn{1}{l|}{Branch} & Branch prediction index table malfunction, effectively reducing the size of the tables by \textit{N} entries. \\ \hline
  \multicolumn{1}{l|}{\multirow{3}{*}{Registers}} & If an instruction with opcode \textit{X} uses physical register \textit{R}, then this instruction is delayed by \textit{T} cycles.  A bug similar to this can be found on Intel 386 DX errata~\cite{intel_386_errata} labeled as ``POPA/POPAD instruction malfunction''. \\ \cline{2-2} 
  \multicolumn{1}{l|}{} & After every \textit{N} times a register has been written, delay the following write by \textit{T} cycles. The inspiration for this bug is the one labeled as ``GPMC may stall after 256 write accesses in NAND\_DATA, NAND\_COMMAND, or NAND\_ADDRESS'' found on the TI AM3517 and TI AM3505 ARM processors errata~\cite{TI_am3517_errate}. \\ \cline{2-2} 
  \multicolumn{1}{l|}{} & The number of physical registers is reduced by \textit{N}. \\ \hline
  \multicolumn{1}{l|}{\multirow{2}{*}{Load/Store Queue}} & For every \textit{N} requests, the load-queue incorrectly rejects entries stating that it is full. \\ \cline{2-2} 
  \multicolumn{1}{l|}{} & For every \textit{N} requests, the store-queue incorrectly rejects entries stating that it is full. \\ \hline
  \multicolumn{1}{l|}{\multirow{2}{*}{Memory}} & After every \textit{N} stores to the same cache line, delay the following write by \textit{T} cycles. \\ \cline{2-2} 
  \multicolumn{1}{l|}{} & The latency of L2 cache is \textit{T} cycles higher than expected. This issue is similar to the documented for NXP MPC7448 RISC processor in its errata~\cite{nxp_7448_errata} labeled as ``L2 latency perfomance issue''. \\ \hline
  \multicolumn{1}{l|}{Re-Order Buffer} & If less than \textit{N} slots are available in the re-order buffer, delay the next instruction by \textit{T} cycles. \\ \hline
  \multicolumn{1}{l|}{Commit} & Every \textit{T} cycles, the maximum number of instructions that the processor is able to commit is reduced by \textit{N} entries during one cycle. \\ \hline

\end{tabular}
}
\end{table*}
\end{scriptsize}

\subsection{Simulated Architectures} \label{subsec:sim_archs}

The experiments shown in Section~\ref{sec:evaluation} are based on
simulations performed in gem5~\cite{gem5} using the out-of-order core
model (O3CPU) and x86 ISA in system emulation mode.
%Although all the
%evaluations are based on simulations, the methodologies can also be
%applied in post-silicon debugging with minor modifications, as discussed in Section~\ref{sec:scope}.

We configured gem5 to emulate a diverse set of microarchitectures that
are used to train and test our scheme. The modified core related settings
are: clock period, pipeline width, branch
predictor, and the sizes of re-order buffer, load/store queue, and
instruction queue.  Cache related configurations (size, associativity,
latency, and number of levels), as well as functional unit
characteristics (count, latency, and port organization) are also
modified to map the simulator behavior to the emulated core.  Other
microarchitectural differences besides these are not considered.

In total, 23 different microarchitecture configurations were
implemented, 15 of them are based on multiple Stock-Keeping Unit (SKU)
variants of the Intel Core microarchitectures: Sandy Bridge, Ivy
Bridge, Haswell, Broadwell, Ice Lake and Skylake. The remaining
were artificially created, but use realistic settings.

For \textbf{training} purposes, we use the data from simulations of the eight artificial
microarchitectures, and the SKU variants of Sandy Bridge and Skylake.
To \textbf{test} the techniques we use the remaining four microarchitectures. As such, the test
microarchitectures are not used to train the ML models.


\subsection{Implemented Bugs} \label{subsec:impl_bugs}

Given its wide acceptance on the computer architecture community, the
gem5~\cite{gem5} simulator is treated as a performance bug-free
simulator. Therefore, in order to evaluate the methodologies,
performance bugs are artificially injected into it. 
Although we use gem5 as a ``bug-free'' baseline, just as any big
software (or hardware) project, it is likely to have bugs,
we believe this should not deter us from implementing debugging 
mechanisms. 

To obtain a list
of bugs that cover most microarchitectural units and are considered
realistic, multiple errata of commercial microprocessors were
reviewed~\cite{intel_xeon_errata,nxp_7448_errata,TI_am3517_errate,intel_386_errata}
and industry experts were consulted. Ultimately, a list of 22
different bug types were implemented in gem5 and are used to evaluate
our scheme. To implement these bugs, we manually edit gem5 code
such that the simulator would behave as the bug describes,
rather than follow its normal behavior.

Although relatively few in the scope of all possible bugs,
these bugs represent a reasonable start for early work in the area.
We examine our localization technique by localizing the
bug to one of 11 possible units. The assignment of bug locations
to each unit, and the description of each performance bug type are
found in Table~\ref{tab:bugs_gem5}. Since the evaluation of the
techniques is based on simulations, no further breakdown is
possible in most units, given the abstraction of unit implementation
in gem5.

For each of these bug types, multiple variations were implemented by
modifying the values of \textit{X}, \textit{Y}, \textit{N}, \textit{R}, and \textit{T}.  The
average IPC impact of these bugs is measured across the used SPEC
CPU2006 applications, and its distribution is shown in
Figure~\ref{fig:bug_hist}.  Bugs with large IPC impact ($>5\%$) are
usually easier to debug, therefore, we include a smaller fraction of
these bugs.  Bugs that produce a performance degradation between
1\%$-$5\%, can be considered to produce a moderate impact, while
degradation $<$1\% is considered small.  Only bugs with IPC degradation $>0.1\%$ were considered for our
evaluations.

\begin{figure}[htb!]
  \centering
  \includegraphics[width = 0.32\textwidth]{figures/bug_distr_ISCA.pdf}
  \caption{Average IPC impact distribution for evaluated bugs.}
  \label{fig:bug_hist}
\end{figure}

For some bug types, none of their variations are used to train the ML models,
so the methodology can be evaluated on bugs that it has
never encountered before, we refer to these as the ``unseen bug
types''.  For the other bug types, some  variations are used to
train the ML models, while others are left exclusively for testing. This partitioning
evaluates how the model performs on bugs that are similar, but not equal, to
those that it has encountered before
(with the variations left exclusively for testing), as well as in cases where
the same bug seen in legacy designs is encountered (with the variations used for training). We refer to these
as ``unseen variations of seen bug types'' and ''seen bug
variations'', respectively. 
For example, consider a particular bug type X, for which we have three variations: X.A, X.B and X.C. If we use data from legacy architectures
with X.A and X.B to train our models, a sample from a test architecture with X.C would
be an ``unseen bug variation of a seen bug type'' while samples from test architectures
with X.A and X.B would be from a ``seen bug variation''. On the other hand, if none of the training samples
have bug X, the samples from test architectures with any variation of this bug would be considered to be from an
``unseen bug type''.

We collected data for three different variations of each of the 22 implemented bug types. For six of those,
neither variation was used during training (``unseen bug types''). For the remaining 16 bug types,
two variations are used for training (``seen bug variations''), while the other one is left exclusively 
for testing (``unseen bug variation of a seen bug type''). In summary, 27.3\% of the samples correspond to
``unseen bug types'', 24.2\% correspond to ``unseen bug variation of a seen bug type'' and 48.5\% correspond to ``seen bug types''.

The data organization is as follows:

\begin{compactitem}
  \itemsep0em 
\item Training data:
  \begin{compactenum}
    \itemsep0em 
  \item Data with positive labels: For each model trained to identify
    performance bugs in unit $u_j$, the samples with positive labels
    are those from bugs in unit $u_j$ which are placed in
    the ``seen bug variations'' category. These are
    exclusively from the ``train'' microarchitectures.
  \item Data with negative labels: For each model corresponding to
    unit $u_j$, the samples with negative labels are those from
    bugs that do not occur in unit $u_j$ 
    (or bug-free cases, if available) and 
    which are
    placed in the ``seen bug variations'' category. These include data
    exclusively from the ``train'' microarchitectures.
  \end{compactenum}
\item Testing data:
  \begin{compactenum}
    \itemsep0em 
  \item Designs with bugs: Samples from ``seen'' and the two
    ``unseen'' categories of bugs coming exclusively from ``test''
    architectures are evaluated.  In any given sample, only one bug is
    inserted.
  \item Bug-free designs: As mentioned earlier, it is assumed that a
    performance bug has already been detected on the design under
    test. However, an analysis of bug-free
    architectures using the methodologies is shown in
    Section~\ref{subsec:no_bug_handling} to evaluate
    false-positives at detection.
  \end{compactenum}
\end{compactitem}

Note that all the samples used for training are
taken from ``seen bug variations'' in ``train'' architectures. On the
other hand, every sample used for testing comes from ``test''
architectures but can either be a ``seen bug variation'', an
``unseen variation of a seen bug type''  or a
completely ``unseen bug type''.

For our evaluation, at most one bug is injected in every design. Although this is a gap to be addressed in future work, the assumption of a single dominant bug is a first step towards solving the problem. We expect that our methodology would work with multiple bugs because, given that the classifiers for each unit are independent (due to OvA), our methodology should produce high confidence of bugs in all (or most) of the units where a bug is detected. 
\subsection{Bug Localization in Memory Systems} \label{subsec:setup_memory}

Although the focus of this work is the localization of performance
bugs in microprocessor cores, we evaluate the methodologies on
the cache memory system, in order to determine how the methodologies
perform in different setups. Both methodologies are
applied in the exact same manner, but there are minor differences in
the setup used for this evaluation.

Instead of gem5, the ChampSim~\cite{champsim} simulator is used, as it
provides a detailed memory system simulation with a much shorter
simulation time.  Further, by using ChampSim, we highlight the
robustness of our proposed approach.  A total of 96 SimPoints
extracted from 20 applications from the SPEC CPU2017~\cite{spec2017}
benchmark suite are used.  These traces were obtained from the Third
Data Prefetching Championship~\cite{dpc3}. Each of these SimPoints is
2B instructions long, but simulations are stopped after 1B
instructions have been executed. The performance counters are sampled
every 500k cycles due to the long traces being used.

\begin{scriptsize}
\begin{table}[!htb]
  \footnotesize
\centering
\caption{Performance bugs injected to ChampSim and their corresponding locations.}
\label{tab:bugs_champsim}
\resizebox{0.48\textwidth}{!}{%
\begin{tabular}{p{0.12\textwidth}|p{0.35\textwidth}}
  \hline
  \textbf{Bug location} & \textbf{Bug description} \\ \hline
  \multirow{2}{*}{Replacement Policy} & During a cache eviction, the policy evicts the most recently used block, instead of the least recently used. \\ \cline{2-2} 
                        & When a cache block is accessed, the age counter for the replacement policy is not updated. \\ \hline
  \multirow{2}{*}{Prefetcher} & On lookahead prefetching, the path with the least confidence is selected. \\ \cline{2-2} 
                        & Signature Path Prefetcher (SPP)~\cite{kim2016spp} signatures are reset, making the prefetcher use the wrong address. \\ \hline
  \multirow{3}{*}{Other  Operations} & After \textit{N} load misses on L1 data cache, the following L1 data read operation takes  \textit{T} additional clock cycles. \\ \cline{2-2} 
                        & After \textit{N} load misses on L2 cache, the following L2 read operation takes  \textit{T} additional clock cycles. \\ \cline{2-2} 
                        & If there are more than  \textit{Y} misses in less than  \textit{X} cycles, every read operation is delayed by  \textit{T} clock cycles. \\  \hline

\end{tabular}%
}
\end{table}
\end{scriptsize}

The emulated architectures are Intel Nehalem, Sandy Bridge, Ivy
Bridge, Haswell and Skylake, as well as AMD K10 and Ryzen7, and four
artificially generated configurations.  Ryzen7, Haswell and Skylake
are used as testing architectures, while the rest are used for
training the models.

The description of the performance bugs injected to ChampSim, along
with their corresponding locations on the design can be found in
Table~\ref{tab:bugs_champsim}.  Due to the limited number of bugs
available all the bugs are included in the ``seen'' bugs set.








