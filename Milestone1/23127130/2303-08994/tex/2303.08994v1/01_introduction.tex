Time-domain simulations form the backbone in many power system analyses such as transient or voltage stability analyses. However, even the simplest set of governing \glspl{DAE} which can describe the system dynamics sufficiently accurate, can impose a significant computational burden during the analysis. Ways to reduce this computational cost while maintaining a sufficiently high level of accuracy is of paramount importance across all applications in the power systems industry. 

Since, generally speaking, there is no closed form analytical solution for \glspl{DAE} \cite{brenan_numerical_1995}, we revert to numerical methods to approximate the dynamic response. Refs.~\cite{stott_power_1979,sauer_power_1998} provide a good overview on general solution approaches and the modelling in the power system context, and \cite{gurrala_parareal_2016,liu_solving_2020,aristidou_time-domain_2015} summarise important developments, mostly relying on model simplification, decompositions, pre-computing partial solutions, and parallelisations. 

A new avenue to solve ordinary and partial differential equations emerged recently through so-called \Gls{SciML} -- a field, which combines scientific computing with \gls{ML}. \Gls{SciML} has been receiving a lot of attention due to the significant potential speed-ups it can achieve for computationally expensive problems, such as the solution of differential equations. More specifically, the authors in \cite{lagaris_artificial_1998}, already 25 years ago, introduced the idea of using artificial \glspl{NN} to approximate such solutions. The idea is that \glspl{NN} learn from a set of training data to interpolate the solution for data points that lie between the training data with high accuracy. Ref.~\cite{raissi_physics-informed_2018} has revived this effort, now named \glspl{PINN}, which has developed into a growing field within \gls{SciML} as \cite{karniadakis_physics-informed_2021} reviews. The key idea of \glspl{PINN} is to directly incorporate the domain knowledge into the learning process. We do so by evaluating if the \gls{NN} output satisfies the set of \glspl{DAE} during training. If it does not, the parameters of the \gls{NN} are adjusted in the next training iteration until the \gls{NN} output satisfies the DAEs. This approach reduces the need for large training datasets and hence the associated costs for simulating them. Ref.~\cite{misyris_physics-informed_2020} introduced \glspl{PINN} in the field of power systems.

Our ultimate goal is to develop \glspl{PINN} as a solution tool for time-domain simulations in power systems. This paper takes a first step, and identifies the strengths and weaknesses of such a method in comparison with existing solution methods with respect to the application specific requirements on the solution method. Stott elaborated nearly half a century ago that, among others, sufficient accuracy, numerical stability, and flexibility were important characteristics that need to be weighed against the solution speed \cite{stott_power_1979}. In an ideal world, we are looking for tools that are highly accurate, numerically stable, and flexible, and at the same time very fast. Several approaches have been proposed to deal with this trade-off, aiming at being faster (at least during run-time) while maintaining accuracy, numerical stability, and flexibility to the extent possible. Some of the promising ones are based on pre-computing parts of the solution of \glspl{DAE}. For example, \gls{SAS}-methods adopt this approach \cite{gurrala_large_2017,duan_power_2017,wang_timepower_2019}. We can push this idea of pre-computing the solution even further: \glspl{PINN}, and \glspl{NN} in general, pre-compute -- learn -- the entire solution, hence, the computation at run-time is extremely fast. Related works in \cite{moya_dae-pinn_2023,li_machine-learning-based_2020,cui_predicting_2021} introduce alternative \gls{NN} architectures and problem setups, primarily driven by considerations on the achieved accuracy. In contrast, our focus lies on assessing \glspl{PINN} from a perspective of a numerical solution method in which accuracy has to be weighed against other numerical characteristics namely speed, numerical stability and flexibility.  The contributions of this work are the following:

\setlist[enumerate,1]{leftmargin=0.5cm}
\begin{enumerate}
    \item We apply \acrfullpl{PINN} to multi-machine systems and show that \glspl{PINN} can be 10 to 1'000 times faster than conventional methods for time-domain simulations, while achieving sufficient accuracy.
    \item We demonstrate that the trade-off between speed and accuracy for \glspl{PINN}, and \glspl{NN} in general, does not directly relate to power system size but rather to the complexity of the dynamics. Hence, \glspl{NN} can solve larger systems equally fast as small ones, if the complexity of the dynamics is comparable. This is contrary to conventional methods, where the solution time is closely linked to the system size.
    \item We examine further numerical properties of \glspl{NN} for solving \glspl{DAE}. Besides speed, one of their key benefits is that \glspl{NN} do not suffer from numerical instability as they solve without any iterative procedure. We also discuss the challenges of flexibility in different parameter settings and we outline concrete directions for future work to resolve them.
    \item Having shown that \glspl{NN} do have significant benefits and desirable properties, we carry out a comprehensive analysis on the performance and training of \glspl{NN} and \glspl{PINN} that can be helpful for future applications. In this context, we introduce \textit{dtNNs}, a regularised form of \glspl{NN}. dtNNs are an intermediate methodological step between NNs and PINNs as they are regularised by the time derivatives at the training data points. 
\end{enumerate}
\Cref{sec:methodology} describes the construction of a \gls{NN}-based approximation for \glspl{DAE} and how to incorporate physical knowledge in dtNNs and \glspl{PINN}. \Cref{sec:case_study} presents the case study and the training setup. \cref{sec:results} shows the results, on which basis we discuss the route forward in \cref{sec:discussion}. \Cref{sec:conclusion} concludes.