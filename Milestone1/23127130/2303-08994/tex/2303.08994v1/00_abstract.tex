The simulation of power system dynamics poses a computationally expensive task. Considering the growing uncertainty of generation and demand patterns, thousands of scenarios need to be continuously assessed to ensure the safety of power systems. \glspl{PINN} have recently emerged as a promising solution for drastically accelerating computations of non-linear dynamical systems. This work investigates the applicability of these methods for power system dynamics, focusing on the dynamic response to load disturbances. Comparing the prediction of \glspl{PINN} to the solution of conventional solvers, we find that \glspl{PINN} can be 10 to 1'000 times faster than conventional solvers. At the same time, we find them to be sufficiently accurate and numerically stable even for large time steps. To facilitate a deeper understanding, this paper also presents a new regularisation of \gls{NN} training by introducing a gradient-based term in the loss function. The resulting \glspl{NN}, which we call dtNNs, help us deliver a comprehensive analysis about the strengths and weaknesses of the \gls{NN} based approaches, how incorporating knowledge of the underlying physics affects \gls{NN} performance, and how this compares with conventional solvers for power system dynamics. 