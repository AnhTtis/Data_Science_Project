%% 
%% Copyright 2007-2020 Elsevier Ltd
%% 
%% This file is part of the 'Elsarticle Bundle'.
%% ---------------------------------------------
%% 
%% It may be distributed under the conditions of the LaTeX Project Public
%% License, either version 1.2 of this license or (at your option) any
%% later version.  The latest version of this license is in
%%    http://www.latex-project.org/lppl.txt
%% and version 1.2 or later is part of all distributions of LaTeX
%% version 1999/12/01 or later.
%% 
%% The list of all files belonging to the 'Elsarticle Bundle' is
%% given in the file `manifest.txt'.
%% 

%% Template article for Elsevier's document class `elsarticle'
%% with numbered style bibliographic references
%% SP 2008/03/01
%%
%% 
%%
%% $Id: elsarticle-template-num.tex 190 2020-11-23 11:12:32Z rishi $
%%
%%
\documentclass[preprint,12pt]{elsarticle}

%% Use the option review to obtain double line spacing
% \documentclass[authoryear,preprint,review,12pt]{elsarticle}

%% Use the options 1p,twocolumn; 3p; 3p,twocolumn; 5p; or 5p,twocolumn
%% for a journal layout:
%% \documentclass[final,1p,times]{elsarticle}
%% \documentclass[final,1p,times,twocolumn]{elsarticle}
%% \documentclass[final,3p,times]{elsarticle}
%% \documentclass[final,3p,times,twocolumn]{elsarticle}
%% \documentclass[final,5p,times]{elsarticle}
%% \documentclass[final,5p,times,twocolumn]{elsarticle}

%% For including figures, graphicx.sty has been loaded in
%% elsarticle.cls. If you prefer to use the old commands
%% please give \usepackage{epsfig}

%% The amssymb package provides various useful mathematical symbols

% \usepackage{natbib}
\usepackage{xcolor}
\usepackage{amsmath}
\usepackage{amssymb}
\usepackage{bm}
\usepackage[shortlabels]{enumitem}

 \usepackage[]{subfig}


\usepackage{tikz,pgf} %and any other packages or tikzlibraries your picture needs
\usepackage{lipsum}
\usetikzlibrary{fit}
\usepackage{pgfplotstable}
\pgfplotsset{compat = 1.15}
\usepgfplotslibrary{statistics}
\usetikzlibrary{intersections}
\usepgfplotslibrary{fillbetween}
\usepgfplotslibrary{colorbrewer}
\pgfplotsset{cycle list/OrRd-3}
\usepgfplotslibrary{groupplots}
\usetikzlibrary{plotmarks}
\usetikzlibrary{decorations.pathreplacing}
\usetikzlibrary{spy}

\usepackage{tikz}
\usetikzlibrary{shapes,arrows}
\usetikzlibrary{calc}
\usepackage[nooldvoltagedirection]{circuitikz}
\usepackage{mathdots}
\usepackage{tabularx}
\usepackage{booktabs}       % professional-quality tables
\usepackage{siunitx}
\usepackage{makecell}
\DeclareSIUnit\pu{p.u.}
% \usepackage{todonotes}
\usepackage{comment}
\usepackage{multirow}
\usepackage{footnote}
\usepackage{pdflscape}
\makesavenoteenv{tabular}
\makesavenoteenv{table}
% \usepackage[hidelinks]{hyperref}
\usepackage{dirtytalk}
\usepackage[capitalise]{cleveref}
\usepackage[acronym,toc,nonumberlist]{glossaries} % load after hyperref
\hyphenation{ana-lysis ana-lyses con-ven-tio-nal}
%% The amsthm package provides extended theorem environments
%% \usepackage{amsthm}

%% The lineno packages adds line numbers. Start line numbering with
%% \begin{linenumbers}, end it with \end{linenumbers}. Or switch it on
%% for the whole article with \linenumbers.
%% \usepackage{lineno}

\newcommand\norm[2]{\ensuremath{\left\lVert#1\right\rVert_{#2}}}
\newcommand{\missingreference}[1][...]{\textcolor{red}{[#1]}}
\DeclareMathOperator{\NN}{NN}

\newcommand{\dataset}[1]{\ensuremath{\mathcal{D}_{\text{#1}}}}
\newcommand{\NNinput}{\ensuremath{\bm{z}_0}}
\newcommand{\NNoutput}{\ensuremath{\hat{\bm{x}}}}
\newcommand{\Foutput}{\ensuremath{\bm{x}}}
\newcommand{\Ffunction}{\ensuremath{\bm{f}}}

\newcommand{\xinitial}{\ensuremath{\bm{x}(t_0)}}

\newcommand{\lossx}{\ensuremath{\mathcal{L}_{x}}}
\newcommand{\lossdt}{\ensuremath{\mathcal{L}_{dt}}}
\newcommand{\lossf}{\ensuremath{\mathcal{L}_{f}}}

\newcommand{\scalelossx}{\ensuremath{\xi_{x, i}}}
\newcommand{\scalelossdt}{\ensuremath{\xi_{dt, i}}}
\newcommand{\scalelossf}{\ensuremath{\xi_{f, i}}}

\newcommand{\weightlossdt}{\ensuremath{\lambda_{dt}}}
\newcommand{\weightlossf}{\ensuremath{\lambda_{f}}}
\newcommand{\weightlossfmax}{\ensuremath{\lambda_{f, \max}}}
\newcommand{\weightlossfzero}{\ensuremath{\lambda_{f,0}}}

\newacronym{NN}{NN}{Neural Network}
\newacronym{PINN}{PINN}{Physics-Informed Neural Network}
\newacronym{BDF}{BDF}{Backward Differentiation Formula}
\newacronym{LBFGS}{L-BFGS}{limited-memory Broyden-Fletcher-Goldfarb-Shanno}
\newacronym{HPC}{HPC}{High Performance Computing}
\newacronym{DAE}{DAE}{Differential-Algebraic Equation}
\newacronym{DTU}{DTU}{Technical University of Denmark}
\newacronym{SciML}{SciML}{Scientific Machine Learning}
\newacronym{ML}{ML}{Machine Learning}
\newacronym{DE}{DE}{Differential Equation}
\newacronym{SAS}{SAS}{Semi-Analytical Solution}
\newacronym{RT}{RT}{Run-Time}
\newacronym{AD}{AD}{Automatic Differentiation}

\definecolor{color_pureNN}{rgb}{0.992, 0.906, 0.78125}
\definecolor{color_dtNN}{rgb}{0.988, 0.730, 0.516}
\definecolor{color_PINN}{rgb}{0.887, 0.289, 0.199}

% plot parameters
\def\yminErrorChar{1.0e-7}
\def\ymaxErrorChar{1.0e-0}

\def\subplotWidthErrorChar{5cm}
\def\horizontalDistanceErrorChar{3.8cm}
\def\subplotHeightErrorChar{3.5cm}
\def\verticallDistanceErrorChar{2.3cm}

\def\pathcolor{blue!50}

\newcommand{\errorDistribution}[1]{
    \addplot[color=\pathcolor, name path = q000] table[x index=0, y index=1] {data/#1};
    \addplot[color=\pathcolor, name path = q100] table[x index=0, y index=7] {data/#1};
    \addplot[color=\pathcolor, name path = q005] table[x index=0, y index=2] {data/#1};
    \addplot[color=\pathcolor, name path = q095] table[x index=0, y index=6] {data/#1};
    \addplot[color=\pathcolor, name path = q025] table[x index=0, y index=3] {data/#1};
    \addplot[color=\pathcolor, name path = q075] table[x index=0, y index=5] {data/#1};
    \addplot[gray!20] fill between [of = q000 and q100];
    \addplot[gray!40] fill between [of = q005 and q095];
    \addplot[gray!60] fill between [of = q025 and q075];
    \addplot[color=black] table[x index=0, y index=4] {data/#1};
}

\newcommand{\validationLossDistribution}[2]{
    \addplot[color=gray!50, name path = q000] table[x index=0, y index=1] {data/#1};
    \addplot[color=gray!50, name path = q100] table[x index=0, y index=2] {data/#1};
    \addplot[color=#2!60, name path = q025] table[x index=0, y index=4] {data/#1};
    \addplot[color=#2!60, name path = q075] table[x index=0, y index=5] {data/#1};
    \addplot[#2!50, opacity=0.5] fill between [of = q000 and q100];
    \addplot[#2!100, opacity=0.5] fill between [of = q025 and q075];
    \addplot[color=#2] table[x index=0, y index=3] {data/#1};
}

\newcommand{\validationLossDistributionOuter}[2]{
    \addplot[draw=none, name path = q000, very thin] table[x index=0, y index=1] {data/#1};
    \addplot[draw=none, name path = q100, very thin] table[x index=0, y index=2] {data/#1};
    \addplot[#2!80, opacity=0.5] fill between [of = q000 and q100]; 
}
\newcommand{\validationLossDistributionInner}[2]{
    \addplot[draw=none, name path = q025] table[x index=0, y index=4] {data/#1};
    \addplot[draw=none, name path = q075] table[x index=0, y index=5] {data/#1};
    \addplot[#2!100, opacity=0.5] fill between [of = q025 and q075]; 
}

\journal{Electric Power Systems Research}
% \renewcommand{\baselinestretch}{0.99}
\begin{document}

\begin{frontmatter}

%% Title, authors and addresses

%% use the tnoteref command within \title for footnotes;
%% use the tnotetext command for theassociated footnote;
%% use the fnref command within \author or \address for footnotes;
%% use the fntext command for theassociated footnote;
%% use the corref command within \author for corresponding author footnotes;
%% use the cortext command for theassociated footnote;
%% use the ead command for the email address,
%% and the form \ead[url] for the home page:
%% \title{Title\tnoteref{label1}}
%% \tnotetext[label1]{}
%% \author{Name\corref{cor1}\fnref{label2}}
%% \ead{email address}
%% \ead[url]{home page}
%% \fntext[label2]{}
%% \cortext[cor1]{}
%% \affiliation{organization={},
%%             addressline={},
%%             city={},
%%             postcode={},
%%             state={},
%%             country={}}
%% \fntext[label3]{}

\title{Physics-Informed Neural Networks\\for Time-Domain Simulations:\\Accuracy, Computational Cost, and Flexibility}

\author[inst1]{Jochen Stiasny\corref{cor1}}
\ead{jbest@dtu.dk}
\cortext[cor1]{Corresponding author}
%% \ead[url]{home page}
%% \fntext[label2]{}
%% \cortext[cor1]{}

\affiliation[inst1]{organization={Department for Wind and Energy Systems, Technical University of Denmark},%Department and Organization
            addressline={Elektrovej}, 
            city={Kgs. Lyngby},
            postcode={2800}, 
            country={Denmark}}

\author[inst1]{Spyros Chatzivasileiadis}
\ead{spchatz@dtu.dk}

\begin{abstract}
%% Text of abstract


\begin{abstract}
% \vspace{-1em}
The diffusion-based generative models have achieved remarkable success in text-based image generation. However, since it contains enormous randomness in generation progress, it is still challenging to apply such models for real-world visual content editing, especially in videos. 
In this paper, we propose \texttt{FateZero}, a zero-shot text-based editing method on real-world videos without per-prompt training or use-specific mask. 
\RM{Specifically, different from a pipeline of two independent inversion and then generation stages, we find the intermediate attention maps during inversions store better structure and motion information. We thus reform them to temporally casual attention and replace them in the generation progress. To further reduce the unnecessary semantic leakage of source video and enhance the editing quality, we then remix the temporally casual attentions via the cross-attention features of the source prompt as the mask.}
To edit videos consistently, we propose several techniques based on the pre-trained models. Firstly, in contrast to the straightforward DDIM inversion technique, our approach captures intermediate attention maps during inversion, which effectively retain both structural and motion information. These maps are directly fused in the editing process rather than generated during denoising. To further minimize semantic leakage of the source video, we then fuse self-attentions with a blending mask obtained by cross-attention features from the source prompt. Furthermore, we have implemented a reform of the self-attention mechanism in denoising UNet by introducing spatial-temporal attention to ensure frame consistency.
Yet succinct, our method is the first one to show the ability of zero-shot text-driven video style and local attribute editing from the trained text-to-image model. We also have a better zero-shot shape-aware editing ability based on the text-to-video model~\cite{tuneavideo}. \RM{Besides video, our unified method also achieves state-of-the-art performance in zero-shot image editing.\chenyang{Need exp or remove the zero-shot image}} Extensive experiments demonstrate our superior temporal consistency and editing capability than previous works.
% The code will be released.
% \chenyang{emphasize: our observation at inversion time} \xiaodong{replacing the bold part to the actual pipeline: \textbf{Specifically, we work on replacing and mixing the attention maps between the inversion and generation since the self-attention map keeps the structure of the original natural image and the cross-attention is semantic-related, after remixing, we replace them in the corresponding generation steps for denoising.}}
% \footnote{Since there is no general video diffusion model is publicly available, we use one-shot video generation method~(Tune-A-Video~\cite{tuneavideo}) as the pretrained video diffusion model for zero-shot video editing\xiaodong{can be removed if we actually zero-shot on video}.}.
\end{abstract}
\end{abstract}

% %%Graphical abstract
% \begin{graphicalabstract}
% \includegraphics{grabs}
% \end{graphicalabstract}

%%Research highlights
% \begin{highlights}
% \item Neural networks are an extremely fast and sufficiently accurate solution method
% \item Research highlight 2
% \end{highlights}

\begin{keyword}
%% keywords here, in the form: keyword \sep keyword
dynamical systems \sep neural networks \sep scientific machine learning \sep time-domain simulation
%% PACS codes here, in the form: \PACS code \sep code
% \PACS 0000 \sep 1111
%% MSC codes here, in the form: \MSC code \sep code
%% or \MSC[2008] code \sep code (2000 is the default)
% \MSC 0000 \sep 1111
\end{keyword}

\end{frontmatter}

%% \linenumbers

%% main text
\glsresetall
\section{Introduction}\label{sec:introduction}
\section{Introduction}

The ability to reason about plans is critical for performing long-horizon tasks \citep{erol1996hierarchical, sohn2018hierarchical, sharma-etal-2022-skill}, compositional generalization \citep{corona-etal-2021-modular} and generalization to unseen tasks and environments \citep{shridhar2020alfred}.
Consider a simple long-horizon planning scenario where a robot is tasked with preparing a meal and serving it on the table. 
This presents a non-trivial planning problem since the agent needs to understand the sequence of operations required to perform the task and search for the relevant objects in the unfamiliar environment by interacting with various objects. %



Large language models have been recently shown to possess commonsense knowledge about the world such as object affordances and physical dynamics \citep{ouyang2022training,chowdhery2022palm}.
Early approaches considered text based environments and fine-tuned PLMs to predict actions given the history of past observations and actions \citep{jansen-2020-visually,micheli-fleuret-2021-language,yao-etal-2020-keep}.
Recent work has used this ability to reason about plans from text instructions in simulated household environments with simplifying assumptions such as text-only environment observations or feedback \citep{huang2022language,ahn2022can,li2022pre,logeswaran-etal-2022-shot}.


We focus on \emph{visually grounded planning} with PLMs --- the ability to adapt plans based on interaction and visual feedback from the environment.
While PLMs have strong planning commonsense priors, predictions from a PLM may not be directly realizable in the environment since the observation and action spaces are unknown.
This requires \emph{grounding} the PLM in the environment and adapting it to observe visual feedback, which is highly non-trivial.
Some prior works assume the availability of a pre-trained affordance function \citep{ahn2022can} or a success detector \citep{mirchandani2021ella}.
Notably, SayCan \citep{ahn2022can} completely decouples the PLM from observation information by selecting actions that have both high affordability (through a pre-trained affordance model) and high PLM likelihood.
Although this partially addresses the grounding problem, the use of visual feedback for action affordance alone is limited.
Often an agent must choose one of many affordable actions using information from observations.
For example, a driving agent should re-navigate and possibly turn around when encountering a ``road closed'' sign, but both turning around and driving forward are indistinguishable to SayCan because they are both affordable and the PLM is blind to observations.

Another workaround explored in prior work is translating the information in the visual observations to text using a pre-trained captioning system \citep{shridhar2021alfworld,huang2022language}.
However, it can be difficult to faithfully describe an image in words and information is lost in this inherently noisy process, which limits the information available to the planner.



Recent work shows that PLMs can be adapted for various natural language tasks by inserting tunable embeddings or soft prompts at the input of the PLM (also called prompt tuning or prefix tuning)~\citep{li-liang-2021-prefix,lester-etal-2021-power}.
This approach also extends to multi-modal understanding tasks such as image captioning \citep{mokady2021clipcap} and VQA \citep{tsimpoukelli2021multimodal} where images are encoded as soft prompts and finetuned for the target task.
Transformer based architectures have also been successfully applied to offline Reinforcement Learning in recent work \citep{chen2021decision,janner2021offline,li2022pre,reid2022can}.

Taking inspiration from these works, we propose the simple approach of embedding visual observations (`visual prompts') and \textit{directly inserting them as PLM input embeddings}.
The visual encoder and PLM are jointly trained for the target task, an approach we call \textbf{\oursfull}~(\ours).
By teaching the PLM to use observations for planning in an end to end manner, we remove the dependency on external data such as captions and affordability information that was used in prior work.
We show that this simple approach performs better than prior PLM-based planning approaches on two embodied planning benchmarks based on ALFWorld~\citep{shridhar2021alfworld} and Virtualhome~\cite{puig2018virtualhome}.




\section{Methodology}\label{sec:methodology}
This section lays out how we train a \gls{NN} that shall be used in time-domain simulations, how the physical equations can be incorporated transforming the \gls{NN} to a dtNN and a \gls{PINN}, and how the resulting approximation is assessed.

\subsection{Approximating the solution to a dynamical system}
A dynamical system is characterised by its temporal evolution being dependent on the system's state variables $\bm{x}$, the algebraic variables $\bm{y}$ and the control inputs $\bm{u}$:
\begin{subequations}
\begin{align}
    \frac{d}{dt}\bm{x} &= \bm{f}_{\text{DAE}}\left(\bm{x}(t), \bm{y}(t), \bm{u}\right)\label{eq:f_DAE}\\
     \bm{0} &= \bm{g}_{\text{DAE}}\left(\bm{x}(t), \bm{y}(t), \bm{u}\right)\label{eq:g_DAE}.
\end{align}
\end{subequations}
For clarity and ease of implementation, we express \labelcref{eq:f_DAE,eq:g_DAE} as
\begin{align}
        \bm{M}\frac{d}{dt}\bm{x} &= \bm{f}(\bm{x}(t), \bm{u}).\label{eq:dynamical_system}
\end{align}
by incorporating $\bm{y}$ into $\bm{x}$ and adding $\bm{M}$, which is a diagonal matrix to distinguish if a state $x_i$ is differential $(M_{ii} \neq 0)$ or algebraic $(M_{ii} = 0)$. We will use a \gls{NN} to define an explicit function $\hat{\bm{x}}(t)$ that shall approximate the solution $\bm{x}(t)$ for all $t \in [t_0, t^{\max}]$, i.e., for the entire \textit{trajectory}, starting from the initial condition $\bm{x}(t_0) = \bm{x}_0$.

\subsection{Neural network as function approximator}
We use a standard feed-forward \gls{NN} with $K$ hidden layers that implements a sequence of linear combinations and non-linear activation functions $\sigma(\cdot)$. In theory, a \gls{NN} with a single hidden layer already constitutes a universal function approximator \cite{cybenko_approximation_1989} if it is wide enough, i.e., the hidden layer consists of enough neurons $N_K$. In practice, restrictions on the width and the process of determining the \gls{NN}'s parameters might limit this universality as \cite{goodfellow_deep_2016} elaborates. Still, a multi-layer \gls{NN} in the form of \eqref{eq:NN_equations} provides us with a powerful function approximator:
\begin{subequations}\label{eq:NN_equations}
\begin{alignat}{2}
    [t, \bm{x}_0^\top, \bm{u}^\top]^\top &= \bm{z}_0 &&\label{eq:NN_input}\\
    \bm{z}_{k+1} &= \sigma{(\bm{W}_{k+1} \bm{z}_k + \bm{b}_{k+1})} && \quad \forall k = 0, 1, ..., K-1\label{eq:NN_hidden_layers}\\
    \hat{\bm{x}} &= \bm{W}_{K+1} \bm{z}_K + \bm{b}_{K+1}.&&\label{eq:NN_output}
\end{alignat}
\end{subequations}
The NN output \NNoutput{} is the system state at the prediction time $t$. The input \NNinput{} is composed of the prediction time $t$, the initial condition $\bm{x}_0$ and the control input $\bm{u}$. The weight matrices $\bm{W}_k$ and bias vectors $\bm{b}_i$ form the adjustable parameters $\bm{\theta}$ of the \gls{NN}.

For the training process, we compile a training dataset \dataset{train}, that maps $\NNinput \mapsto \Foutput$ for a chosen input domain $\mathcal{Z}$ and contains $N = |\dataset{train}|$ points. For our purposes, the input domain is a discrete set of the prediction time, e.g. from \SI{0}{\second} until \SI{10}{\second} with a step size of \SI{0.2}{\second}, and a set of different initial conditions and control inputs, e.g. different power disturbances. The output domain is the rotor angle and frequency at each of the prediction time steps and for each of the studied disturbances.
\begin{align}
    \dataset{train}: \NNinput \mapsto \Foutput \qquad \NNinput \in \mathcal{Z}. \label{eq:dataset_definition}
\end{align}
During training we adjust the \gls{NN}'s parameters $\bm{\theta}$ with an iterative gradient-based optimisation algorithm to minimise the so-called \textit{loss} $\mathcal{L}$ for $\mathcal{D}_{train}$
\begin{subequations}\label{eq:NN_optimisation}
\begin{align}
    \min_{\bm{\theta}} \quad &\mathcal{L}(\dataset{train})\\
    \text{s.t.} \quad & \eqref{eq:NN_input} - \eqref{eq:NN_output}.
\end{align}
\end{subequations}
We do not aim for optimality of \labelcref{eq:NN_optimisation} -- this would lead to over-fitting -- but rather search for parameters $\bm{\theta}$ (i.e., values of the weights and biases) and hyper-parameters (e.g., number of layers $K$ and neurons per layer $N_K$) that yield a low generalisation error $\mathcal{L}(\dataset{test})$ which we assess on a separate test dataset \dataset{test}. During training we use a validation dataset \dataset{validation} to obtain an estimate of the generalisation error, so that the final evaluation with \dataset{test} remains unbiased. It is important, that all three datasets stem from the same sampling procedure and input domain $\mathcal{Z}$.

\subsection{Loss function and regularisation: \glspl{NN}, dtNNs, and \glspl{PINN}}

\subsubsection{Loss Function for Neural Networks}
The simplest loss function for such a problem is to define the loss as the mismatch between the \gls{NN} prediction \NNoutput{} and the ground truth or target \Foutput{}, and measure it using the L2-norm. To account for different orders of magnitude (for example, the voltage angles in radians are often much larger than frequency deviations expressed in \si{\pu}) and levels of variations of the individual states $\bm{x}$, we first apply a scaling factor \scalelossx{} to the error computed per state $i$. A physics-agnostic choice of \scalelossx{} could be to use the state's standard deviation in the training dataset; for more details please see \cref{subsec:NN_setup}. We then apply the squared L2-norm for each data point $j$ and take the average across the dataset \dataset{} to obtain the loss \lossx{}
\begin{align}
    \lossx(\dataset{}) &= \frac{1}{|\dataset{}|}\sum_{j=1}^{|\dataset{}|} \norm{\left(\frac{\hat{x}_i^j- x_i^j}{\scalelossx{}}\right)}{2}^2\label{eq:loss_data}.
\end{align}

\subsubsection{dtNNs}
As an intermediate step between standard \glspl{NN} and \glspl{PINN}, in this subsection we introduce a new regularisation term to loss function \eqref{eq:loss_data}. We do so to avoid the previously mentioned over-fitting and improve the generalisation performance of the \glspl{NN}. To the best of our knowledge, this paper is the first to introduce a regularisation term based on the update function \Ffunction{}(\Foutput{}) from \eqref{eq:dynamical_system}. Using the tool of \gls{AD} \cite{baydin_automatic_2018}, we can compute the derivative of the \gls{NN}, i.e., the time derivative of the approximated trajectory, $\frac{d}{dt}\NNoutput{}$ and compute a loss analogous to \eqref{eq:loss_data} (with a scaling factor \scalelossdt{}):
\begin{align}
    \lossdt(\dataset{}) &= \frac{1}{|\dataset{}|}\sum_{j=1}^{|\dataset{}|} \norm{\left(\frac{\frac{d}{dt}\hat{x}_i^j- \frac{d}{dt} x_i^j}{\scalelossdt{}}\right)}{2}^2\label{eq:loss_dtNN}
\end{align}

\subsubsection{PINNs}
As \cite{lagaris_artificial_1998,raissi_physics-informed_2018} introduced generally, and \cite{misyris_physics-informed_2020} for power systems, we can also regularise such a \gls{NN} by comparing the derivative of the \gls{NN} $\frac{d}{dt}\NNoutput{}$ with the update function evaluated based on the estimated state \Ffunction{}(\NNoutput{}):
\begin{align}
    \lossf(\mathcal{D}_f) &= \frac{1}{|\mathcal{D}_f|}\sum_{j=1}^{|\mathcal{D}_f|} \norm{\left(\frac{M_{ii}\frac{d}{dt}\hat{x}_i^j - f_i(\hat{\bm{x}}^j)}{\scalelossf{}}\right)}{2}^2\label{eq:loss_PINN}
\end{align}
This physics-loss does not require the ground truth state \Foutput{} or its derivative. Quite the contrary, this loss can be queried for any desired point without requiring any form of simulation. We therefore can evaluate a dataset $\mathcal{D}_f$ of randomly sampled or ordered \textit{collocation points} that map to 0
\begin{align}
    \mathcal{D}_f: \NNinput \mapsto \bm{0} \qquad \NNinput \in \mathcal{Z}. \label{eq:collocation_definition}
\end{align}
to essentially assess how well the \gls{NN} approximation follows the physics - any point where this physics loss equals zero is in line with the governing physics of \eqref{eq:dynamical_system}. However, \eqref{eq:collocation_definition} defines a mapping that is not bijective, hence, $\mathcal{L}_{f}(\mathcal{D}_f) = 0$ does not imply that the desired trajectory is perfectly matched, only that a trajectory complying with \eqref{eq:dynamical_system} is matched. As an example, an exact prediction of the steady state of the system will yield $\mathcal{L}_{f}(\mathcal{D}_f) = 0$ even though the target trajectory in \dataset{train} is different.

\subsubsection{Combined loss function during training}
To obtain a single objective or loss value for the training problem \eqref{eq:NN_optimisation}, we weigh the three terms as follows:
\begin{align}
    \mathcal{L} &= \lossx + \weightlossdt \lossdt + \weightlossf \lossf,
\end{align}
where \weightlossdt{} and \weightlossf{} are hyper-parameters of the problem. Subsequently, we refer to a \gls{NN} trained with $\weightlossdt{} = 0, \weightlossf{} = 0$ as \say{vanilla NN}\footnote{In \cite{hastie_elements_2009} \say{vanilla \gls{NN}} refers to a feed-forward \gls{NN} with a single layer, we adopt the term nonetheless for clarity as it expresses the idea of a \gls{NN} without any regularisation well.}, with $\weightlossdt{} \neq 0$, $\weightlossf{} = 0$ as \say{dtNN}, and with $\weightlossdt{} \neq 0, \weightlossf{} \neq 0$ as \say{\gls{PINN}}.

\subsection{Accuracy metrics}

To compare across the different methods and setups, we monitor the loss $\mathcal{L}_x$ in \eqref{eq:loss_data} as the comparison metric throughout the training and evaluation process and as an accuracy metric for the performance assessment. To get a more detailed picture, we also consider the loss value of single points, i.e., before calculating the mean in \labelcref{eq:loss_data}.
However, the loss is dependent on the chosen values for $\scalelossx{}$ and does not provide an easily interpretable meaning. Therefore, we use the maximum absolute error 
\begin{align}
    \max AE_{\mathcal{S}} &= \max_{i \in \mathcal{S}, j \in \dataset{test}}\left(\left|\hat{x}_i^j- x_i^j\right|\right)
\end{align}
as an additional metric for assessment purposes, i.e., based on \dataset{test}, but not during training. Whereas a state-by-state metric would capture most details, we opt to compute the maximum absolute error across meaningful groups of states $i \in \mathcal{S}$ that are of the same units and magnitudes. This aligns with the engineering perspective on the desired accuracy of a method.


\section{Case study}\label{sec:case_study}
% This section briefly introduces the test cases (\cref{subsec:test_cases}) and the implementation of the NN training (\cref{subsec:NN_setup}).
This section introduces the test cases and the details of the NN training.

\subsection{Power system - Kundur 11-bus and IEEE 39-bus system}\label{subsec:test_cases}

As a study setup, we investigate the dynamic response of a power system to a load disturbance. We use a second order model to represent each of the generators in the system. The update equation \labelcref{eq:dynamical_system} formulates for generator buses as
\begin{align}
    \begin{bmatrix} 1 & 0 \\ 0 & 2 H_i \omega_0 \end{bmatrix} \frac{d}{dt} \begin{bmatrix} \delta_{i} \\ \Delta \omega_i \end{bmatrix} &= \begin{bmatrix}
    \Delta \omega_i \\
    P_{mech,i} - D_i\Delta \omega_i + P_{e,i}
    \end{bmatrix}
    \intertext{and for load buses as}
    \begin{bmatrix}d_i \omega_0\end{bmatrix} \frac{d}{dt} \begin{bmatrix}
    \delta_{i}
    \end{bmatrix} &= \begin{bmatrix}
    P_{mech,i} + P_{e,i}
    \end{bmatrix}
\end{align}
where $P_{mech,i} = P_{set, i} + P_{dist, i}$ at bus $i$, with $P_{set, i}$ representing the power setpoint and $P_{dist, i}$ the disturbance . The states $\bm{x}$ are the bus voltage angle $\delta_i$ and the frequency deviation $\Delta \omega_i$ for generator buses, and the bus voltage angle $\delta_i$ for the load buses. The buses are linked through the active power flows in the network defined by the admittance matrix $\Bar{\bm{Y}}_{bus}$ and the vector of complex voltages $\Bar{\bm{V}} = \bm{V}_{m} e^{j\bm{\delta}}$, where the vector $\bm{V}_m$ collects the voltage magnitudes and $\bm{\delta}$ the bus voltage angles:
\begin{align}
    \bm{P}_{e} &= \Re \left(\Bar{\bm{V}} (\Bar{\bm{Y}}_{bus} \Bar{\bm{V}})^*\right).
\end{align}
The $*$ indicates the complex conjugate and $P_{e,i}$ corresponds to the $i$-th entry of vector $\bm{P}_{e}$, i.e., the active power balance at bus $i$. 
In \cref{sec:results}, we demonstrate the methodology on the Kundur 2-area system (11 buses, 4 generators) and the IEEE 39-bus test system (39 buses, 10 generators). For both systems we are using the base power of $\SI{100}{MVA}$ and $\omega_0 = \SI{60}{\hertz}$. The network parameters and set-points stem from the case description of Kundur \cite[p.~813]{kundur_power_1994} and the IEEE 39-bus test case in Matpower \cite{zimmerman_matpower_2011}. The values for the inertia of the generators $H_i$ are [6.5, 6.5, 6.175, 6.175] \si{\pu} for the 11-bus case and [500.0, 30.3, 35.8, 38.6, 26.0, 34.8, 26.4, 24.3, 34.5, 42.0] \si{\pu} for the 39-bus case. The damping factor was set to $D_i = 0.05 \frac{\omega_0}{P_{set,i}}$ in both cases and for the loads to $d_i = 1.0 \frac{P_{set,i}}{\omega_0}$ and $d_i = 0.2 \frac{P_{set,i}}{\omega_0}$ respectively.

\subsection{NN training implementation}\label{subsec:NN_setup}

The entire workflow is implemented in Python 3.8 and available under \cite{stiasny_publicly_2022}. When we use the conventional numerical approaches to carry out the time-domain simulations for this system, the dynamical system is simulated using the Assimulo package \cite{andersson_assimulo_2015} which implements various solution methods for systems of \glspl{DAE}. The training process utilises PyTorch \cite{paszke_pytorch_2019} for the learning process and WandB \cite{biewald_experiment_2020} for monitoring and processing the workflow. The implementation builds on \cite{stiasny_closing_2022} for the steps of the workflow.

All datasets comprise the simulated response of the system over a period of \SI{20}{\second} to a disturbance. The tested disturbance is the step response to an instantaneous loss of load $|P_{dist,i}|$ at bus $i$ with a magnitude between \SI{0}{\pu} and \SI{10}{\pu}, where $i=7$ for the 11-bus system and $i=20$ for the 39-bus system. We record these data in increments of $\Delta t$ and $\Delta P$. The test dataset \dataset{test} which shall serve as a ground truth uses $\Delta t = \SI{0.05}{\second}$ and $\Delta P = \SI{0.05}{\pu}$, resulting in $|\dataset{test}| = 401 \times 201 = 80601$ points. This large size of \dataset{test} is chosen as it densely covers the input domain, hence, we obtain a reliable estimate of the maximum absolute error \labelcref{eq:max_abs_error}. For the training datasets \dataset{train} used in \cref{subsec:NN_training} we create datasets with $\Delta t \in [\num{0.2}, \num{1.0}, \num{2.0}]\si{\second}$ and $\Delta P \in [\num{0.2}, \num{1.0}, \num{2.0}]\si{\pu}$. The validation datasets \dataset{validation} for those scenarios are offset by $\frac{\Delta t}{2}$ and $\frac{\Delta P}{2}$.

For the scalings \scalelossx{} in \eqref{eq:loss_data}, we calculate the average standard deviation $\sigma$ across all voltage angle differences $\delta_{ij}$\footnote{The training process benefits from using the voltage angle difference $\delta_{ij} = \delta_i - \delta_j$, where $j$ indicates a reference bus, as the output of the \gls{NN}. The prediction becomes easier as the occurring drift in the dataset with respect to the variable $t$ is significantly reduced.} and all frequency deviations $\Delta \omega_i$, here the relevant groups of states $\mathcal{S}$:
\begin{align}
    \scalelossx{} &= \frac{1}{|\mathcal{S}|}\sum_{i \in \mathcal{S}} \sigma (x_i(\dataset{}))
\end{align}
Thereby, we aim for equal levels of error within all $\delta_{ij}$ and $\Delta \omega_i$ states and account for the difference in magnitude between them.  \scalelossdt{} and \scalelossf{} are all set to 1.0 to avoid adding further hyperparameters, more elaborate choices based on system analysis or the database are conceivable. During training and testing \scalelossx{} is based on \dataset{train} and \dataset{test} respectively.

The regularisation weights \weightlossdt{} and \weightlossf{} are hyperparameters. For the latter, we incorporate a fade-in dependent on the current epoch $E$ :
\begin{align}
    \weightlossf{}(E) = \min{\left(\weightlossfmax{}; \weightlossfzero{} \; 10^{E/E'}\right)},
\end{align}
where \weightlossfmax{} is the maximum and \weightlossfzero{} the initial regularisation weight and $E'$ determines the ``speed'' of the fade-in. The fade-in causes that \lossx{} and \lossdt{} are first minimised and then \lossf{} helps for ``fine-tuning'' and better generalisation. We apply the \gls{LBFGS} algorithm implemented in PyTorch in the training process, a standard optimiser for PINNs as \cite{cuomo_scientific_2022} reviews. The set of hyperparameters comprises $K$, $N_K$, \weightlossdt{}, \weightlossfmax{}, \weightlossfzero{} $E'$, and additional \gls{LBFGS} parameters. \Cref{tbl:hyperparameters_accuracy} reports the used hyperparameters for the scenarios (letters A-E) in \cref{subsec:NN_training}, \cref{subsec:NN_run_time} is based on scenario E for vanilla NNs.
\begin{table}[!ht]
  \caption{Overview of the tuning range and the selected values of the involved hyperparameters.}
  \label{tbl:hyperparameters_accuracy}
  \centering
  \resizebox{\columnwidth}{!}{%
  \renewcommand{\arraystretch}{1.2}
  \begin{tabular}{lccccccc}
    \toprule
     &  & \multicolumn{2}{c}{vanilla NN} &\multicolumn{2}{c}{dtNN} & \multicolumn{2}{c}{PINN}\\
    Hyper-parameter & Tuning range & A-D & E & A-D & E & A-D & E\\
    \midrule
    Number of layers $K$ & [2, 3, 4, 5] & 5 & 5 & 5 & 5 & 5 & 5 \\
    Nodes per layer $N_K$ & [16, 32, 64, 128] & 32 & 32 & 32 & 32 & 32 & 32 \\
    \midrule
    dt regularisation \weightlossdt & [0.01 - 2.0] & - & - & 0.3 & 1.0 & 0.01 & 0.01\\
    Physics regularisation \weightlossfmax & [0.005 - 10] & - & - & - & - & 0.5 & 0.01\\
    Fade-in speed $E'$ & [10 - 50] & - & - & - & - & 15 & 15\\
    \midrule
    Initial learning rate & [0.1 - 2.0] & 1.0 & 1.6 & 0.5 & 2.0 & 1.2 & 1.0\\
    History size & [100 - 150] & 140 & 140 & 120 & 120 & 120 & 120 \\
    Maximum iterations & [18 - 28] & 22 & 22 & 23 & 20 & 20 & 19\\
    \bottomrule
  \end{tabular}
  }
\end{table}

The hyperbolic tangent ($\tanh{}$) is selected as the activation function $\sigma$ as it is continuously differentiable. We initialise the \gls{NN} weights and biases with samples from the distribution described in \cite{glorot_understanding_2010} and achieve different initial values by altering the seed of the random number generator. All training and timing was performed on the \gls{HPC} cluster at the \gls{DTU} with nodes of 2xIntel Xeon Processor 2650v4 (12 core, 2.20GHz) and 256 GB memory of which we used 4 cores per training run.


\section{Results}\label{sec:results}
\section{Results}\label{sec:Background}

And, so, since 2016, researchers have been probing the submitted methods, and in 2022 NIST published the final 10: ASCON, Elephant, GIFT-COFB, Grain128-AEAD, ISAP, Photon-Beetle, Romulus, Sparkle, TinyJambu, and Xoodyak. A particular focus is on the security of the methods, along with their performance on low-cost FPGAs/embedded processes and their robustness against side-channel attacks.

The current set of benchmarks includes running on an Arduino Uno R3 (AVR ARmega 328P), Arduino Nano Every (AVR ARmega 4809), Arduino MKR Zero (ARM Cortex M10+) and Arduino Nano 33 BLE (ARM Cortex M4F). These are just 8-bit processors and fit into an Arduino board. Along with their processing limitations, they are also limited in their memory footprint (to run code and also to store it). The lightweight cryptography method must thus overcome these limitations, and still, be secure and provide a good performance level. Running AES in block modes on these devices is often not possible, as there is not enough resources. Overall we use a benchmark for encryption — with AEAD (Authenticated Encryption with Additional Data) and for hashing. With AEAD we add extra information — such as the session ID — into the encryption process. This type of method can bind the encryption to a specific stream.



\subsection{ARM Cortex M3}

In Table \ref{tab:table01} [1], we see a sample run using an Arduino Due with an ARM Cortex M3 running at 84MHz. The tests are taken in comparison with the ChaCha20 stream cipher and defined for AEAD, and where the higher the value the better the performance. We can see that Sparkle, Xoodyak and ASCON are the fastest of all. Sparkle has a 100\% improvement, and Xoodyak gives a 60\% increase in speed over ChaCha20. Elephant, ISAP and PHOTON-Beetle have the worst performance for encryption (with around 1/20th of the speed of ChaCha20).

\begin{table*}
\caption{\label{tab:table01} Arduino Due with an ARM Cortex M3 running at 84MHz for encryption against ChaCha20 \cite{light01}}
\centering
\begin{tabular}{|l|l|l|l|l|l|l|l|l|}
\hline
Algorithm&Key Bits&Nonce Bits&Tag Bits&Encrypt 128~B&Decrypt 128~B&Encrypt 16~B &Decrypt 16~B&Aver
\\ \hline \hline
Schwaemm128-128 (SPARKLE)	&128	&128&	128	&1.6	&1.58	&2.84	&2.39	&2.01\\
Xoodyak 	&128	&128	&128	&1.66	&1.51	&1.73	&1.6	&1.62\\
ASCON-128	&128	&128&	128	&1.54	&1.44	&1.78	&1.68	&1.61\\
TinyJAMBU-128 	&128	&96	&64	&0.93	&0.95	&1.63	&1.61	&1.21\\
GIFT-COFB	&128	&128	&128	&1.01	&1.01	&1.16	&1.15	&1.08\\
Grain-128AEAD	&128	&96	&64	&0.26	&0.26	&0.56	&0.56	&0.37\\
Romulus-M1	&128	&128	&128	&0.1	&0.11	&0.15	&0.16	&0.13\\
PHOTON-Beetle-AEAD-ENC-128	&128	&128	&128	&0.06	&0.07	&0.11	&0.12	&0.08\\
ISAP-A-128	&128	&128	&128	&0.08	&0.08	&0.03	&0.04	&0.05\\
Delirium (Elephant)	&128	&96	&128	&0.04	&0.05	&0.06	&0.07	&0.05\\
\hline
\end{tabular}
\end{table*}

Not all of the finalists can do hash functions. Table \ref{tab:table02} outlines these.

\begin{table*}
\caption{\label{tab:table02} Arduino Due with an ARM Cortex M3 running at 84MHz for hashing against BLAKE2s \cite{NISTgov}}
\centering
\begin{tabular}{|l|l|l|l|l|l|}
\hline
Algorithm	& Hash Bits	& 1024 bytes	& 128 bytes	& 16 bytes	& Average\\
\hline\hline
Esch256 (SPARKLE) 	&256	&0.89	&0.78	&1.5	&1.06\\
Xoodyak 	&256	&0.71	&0.65	&1.43	&0.93\\
GIMLI-24-HASH	&256	&0.54	&0.47	&0.86	&0.62\\
ASCON-HASH 	&256	&0.51	&0.41	&0.63	&0.52\\
PHOTON-Beetle-HASH	&256	&0.01	&0.01	&0.05	&0.02\\
\hline
\end{tabular}
\end{table*}


Again, we see Sparkle and Xoodyak in the lead, with Sparkle actually faster in the test than BLAKE2s, and Xoodyak just a little bit slower. ASCON has a weaker performance, and PHOTON-Beetle is relatively slow. For all the tests, the ranking for authenticated encryption is (and where the higher the rank, the better):

14 SPARKLE
12 Xoodyak
12 ASCON
10 TinyJAMBU
9 GIFT-COFB, Gimli
4 Grain-128AEAD,KNOT
0 Elephant, ISAP, PHOTON-Beetle

and for hashing SPARKLE and Xoodyak are ranked the same:

7 SPARKLE, Xoodyak 5 Gimli 3 ASCON 0 PHOTON-Beetle

\subsection{Uno Nano performance}

For AEAD on Uno Nano Every [2], the benchmark is against AES GCM. We can see in \ref{tab:table03} , that SPARKLE is 4.7 times faster than AES GCM for 128-bit data sizes, and Xoodyak comes in second with a 3.3 times improvement over AES GCM. When it comes to 8-bit data sizes TinyJambu actually is the fastest, but where Sparkle and Xoodyak still perform well. PHOTON-Beetle, Grain128 and ISAP do not do well, and only slightly improve on AES GCM. In fact, Grain128 and ISAP are actually slower than AES GCM.




\begin{table*}
\caption{\label{tab:table03} Uno Nano for AEAD against AES GCM and showing cycles (showing fastest of the method)}
\centering
\begin{tabular}{|l|l|l|l|l|l|l|l|l|l|l|}
\hline
Algorithm&Impl.&Primary&Flag&Size&Enc(0:8)&Dec(0:8)&Enc(128:129)&Dec(128:128)&Bench.(128)&Bench.(8)
\\ \hline \hline
sparkle       &rhys	          &yes&	   O3	&12290	&1276	&1316	&4648    &5072  &4.7  &3.3\\
Xoodyak       &XKCP-AVR8	  &yes&    O3	&4560	&2596	&2608	&7184    &7128  &3.3  &1.6\\
knot	      &$avr8_speed$   &no&	   Os	&1664	&2124	&2140	&8144    &8160  &2.9  &2\\
ascon 	      &rhys	          &no&     O3	&5180	&1240	&1284	&8056    &8488  &2.8  &3.3\\
GIFT-COFB     &rhys	          &yes&    O1	&23312	&1852	&1892	&8220    &8776  &2.7  &2.2\\
saeaes	      &ref	          &no&     O3	&17062	&1208	&1212	&8992    &9004  &2.6  &3.4\\
hyena	      &rhys           &yes&    O3	&293860	&1912	&1964	&8960    &9396  &2.5  &2.2\\
elephant      &rhys           &no&     O3	&13106	&1924	&1948	&9260    &9796  &2.4  &2.2\\
estate	      &ref            &yes&    O3	&9434 	&1424	&1448	&10276   &10292 &2.3  &2.9\\
romulus	      &rhys           &no&     O3	&19346 	&1632	&1676	&10152   &10568 &2.2  &2.5\\
spook	      &rhys           &no&     O3	&12942 	&2984	&2968	&10272   &10708 &2.2  &1.4\\
tinyjambu     &rhys           &yes&    O3	&9174 	&1232	&1288	&10364   &10888 &2.2  &3.4\\
subterranean  &rhys           &yes&    Os	&6042 	&3372	&3460	&10288   &10944 &2.2  &1.2\\
orange        &rhys           &yes&    O3	&12140 	&2500	&2536	&11200   &11620 &2    &1.7\\
gimli         &rhys           &yes&    O3	&21272 	&1920	&1956	&11944   &12360 &1.9  &2.2\\
skinny        &rhys           &no&     O1	&12452 	&1604	&1644	&12960   &14372 &1.7  &2.6\\
photon-beetle & $avr8_speed$  &yes&    Os	&3536 	&2444	&2472	&20076   &20092 &1.2  &1.7\\
{\bf reference}&rhys          &yes&    O2	&7874 	&4152	&4156	&23812   &23764 &1    &1\\
grain128aead  &rhys           &yes&    O2	&9532 	&3992	&3980	&30396   &30124 &0.8  &1\\
isap          &rhys           &no&     O2	&3824 	&20212	&20256	&42936   &43372 &0.5  &0.2\\
\hline
\end{tabular}
\end{table*}

And so for AEAD  (performance) the ordering is

1. Sparkle
2. Xoodyak
3. Ascon
4. GIFT-COFB.
5. Elephant.
6. Romulus.
7. Tiny Jambu.
8. PHOTON-Beetle.
9. Grain128
10. ISAP.

For hashing on an Uno Nano Every, Table \ref{tab:table04} shows a similar performance level as to the ARM Cortex M3 assessment. In this case, the benchmark hash is SHA-256, and we can see that it takes Sparkle twice as many cycles for a 128-bit hash, and 2.9 times for Xoodyak. PHOTON-Beetle is way behind with a 128-bit hash and which is 17.4 times slower than SHA-256. That said, though, PHOTON-Beetle could be more focused on reducing power consumption rather than speed. GIMLI and SKINNY are included to show a comparison with well-designed methods in lightweight hashing. It can be seen that every method beats SKINNY, but only SPARKLE and Xoodyak beat GIMLI.


\begin{table*}
\caption{\label{tab:table04}  Uno Nano for hashing against SHA-256 and showing cycles (showing fastest of the method for hashing)}
\centering
\begin{tabular}{|l|l|l|l|l|l|l|l|l|l|l|}
\hline
Algorithm&Impl.&Primary&Flag&Size&h(8)&h(16)&h(32)&h(64)&h(128)&Benchmark
\\ \hline \hline
{\bf reference}&$nacl_ref$    &yes&    O3	&18774 	&768	&768	&772     &1364  &1968  &1\\
sparkle       &rhys	          &yes&	   O1	&7912	&1036	&1036	&1468    &2272  &3884  &2\\
Xoodyak       &XKCP-AVR8	  &yes&    O3	&2604	&1284	&1288	&1924    &3192  &5732  &2.9\\
gimli         &rhys           &yes&    O3	&19554 	&1284	&1920	&2544    &3804  &6312  &3.2\\
ascon 	      &rhys	          &yes&    O3	&2178	&2972	&3552	&4736    &7088  &11784 &6\\
drygascon     &rhys           &no&	   O3	&15500	&4604	&4600	&6540    &10360 &17912 &9.1\\
photon-beetle & $avr8_speed$  &yes&    O3	&2948 	&2372	&2364	&6940    &16084 &34172 &17.4\\
skinny        &rhys           &yes&    O2	&9784 	&7048	&10556	&13976   &20952 &34896 &17.7\\
\hline
\end{tabular}
\end{table*}


And so for hashing (performance) the ordering is:
\begin{enumerate}
    \item Sparkle.
    \item Xoodyak.
    \item Ascon
    \item PHOTON-Beetle. 
\end{enumerate}
    

\section{Discussion}\label{sec:discussion}
\section{Discussion and Conclusion}
To conclude, we propose \textbf{\nickname{}}, the first generalizable human NeRF model that recovers animatable 3D humans from single human image inputs.
To render high-fidelity 3D humans, \nickname{} proposes to learn both global and local details from the bank of 3D-aware hierarchical features comprising global features, point-level features, and pixel-aligned features. 
By using a feature fusion transformer, \nickname{} successfully enhances the information from the 2D observation and complements the information missing from the input image. 
% In addition, by modelling the neural radiance field in the canonical space, \nickname{} can animate 3D humans with free poses.
On four large-scale human datasets, \nickname{} achieves state-of-the-art performance and renders high-fidelity images in both novel views and poses.

\vspace{1.75mm}
\noindent \textbf{Limitations:}
1) There still exists visible artifacts in target renderings when some body parts are occluded in the observation space. 
A better feature presentation like occlusion-aware features may be explored to solve this issue. 
2) How to complement the information missing from single image input remains a challenging problem.
\nickname{} starts from the reconstruction view and can only render deterministic results when predicting novel views.
One potential direction is to investigate the use of conditional generative models to diversely generate higher quality novel views.

\vspace{1.75mm}
\noindent \textbf{Potential Negative Societal Impacts:} 
\nickname{} can be misused to create fake images or videos of real humans and cause negative social impacts.



\section{Conclusion}\label{sec:conclusion}
This paper presented a comprehensive analysis of the use of \acrfull{PINN} for power system dynamic simulations. We show that \glspl{PINN} (i) are 10 to 1'000 times faster than conventional solvers, (ii) do not face issues of numerical instability unlike conventional solvers, and, (iii) achieve a decoupling between the power system size and the required solution time. However, \glspl{PINN} are less flexible (i.e. they do not easily handle parameter changes), and require an up-front training cost. Overall, this makes \gls{PINN}-based solutions well-suited for repetitive tasks as well as task where run-time speed is crucial, such as for screening.

Besides the comparison between conventional and \gls{NN}-based methods, this paper conducts a deeper analysis on the parameters that affect the performance of the \gls{NN} solutions. In that respect, we introduce a new \gls{NN} regularisation, called dtNN, as a intermediate step between \glspl{NN} and \glspl{PINN}. We show that \glspl{PINN} achieve overall higher levels of accuracy, and more balanced error distributions thanks to the evaluation of the collocation points.

\vspace{0.5cm}
\noindent\textbf{Funding:} This work was supported by the European Research Council [Grant Agreement No: 949899].
%% The Appendices part is started with the command \appendix;
%% appendix sections are then done as normal sections

% \newpage
% \appendix

% % \begin{landscape}
%     % \section{Accuracy table}
%     % 
\begin{table}
	\centering
	
	\begin{tabularx}{\linewidth}{ccYYYY}
		& &   &   & \multicolumn{2}{c}{\textbf{95\% CI\textsuperscript{1}}}\\
		\cmidrule(lr){5-6}
		\textbf{Rows} & \textbf{Columns} & $\pmb{\mu}$ & $\pmb{\sigma}$ & \textbf{Lower} & \textbf{Upper}\\
		\midrule
		1 & 2 & 0.990 & 0.021 &  0.938 & 1.043  \\
		
		& 4 & 0.982 & 0.035 & 			0.930 & 1.035  \\
		
		& 6 & 0.925 & 0.112 & 			0.872 & 0.978  \\
		
		2 & 2 & 0.911 & 0.129 & 			0.858 & 0.963  \\
		
		& 4 & 0.878 & 0.161 & 			0.825 & 0.930  \\
		
		& 6 & 0.841 & 0.193 & 			0.789 & 0.894  \\
		
		3 & 2 & 0.834 & 0.193 & 			0.782 & 0.887  \\
		
		& 4 & 0.816 & 0.189 & 			0.763 & 0.868  \\
		
		& 6 & 0.783 & 0.217 & 			0.730 & 0.836  \\
		\bottomrule
	\end{tabularx}
	
	\caption{The accuracy rates of direct interfaces as measured in the first experiment. The table reports the recorded mean values $\mu$ together with the standard deviation $\sigma$. \textsuperscript{1} The confidence interval CI is based on the fitted \ac{EMM} model.}
	\label{tab:cheesyfoot/tables/accuracy}
\end{table}


%     \section{Hyper-parameters}
%     \begin{table}[!ht]
  \caption{Overview of the tuning range and the selected values of the involved hyperparameters.}
  \label{tbl:hyperparameters_accuracy}
  \centering
  \resizebox{\columnwidth}{!}{%
  \renewcommand{\arraystretch}{1.2}
  \begin{tabular}{lccccccc}
    \toprule
     &  & \multicolumn{2}{c}{vanilla NN} &\multicolumn{2}{c}{dtNN} & \multicolumn{2}{c}{PINN}\\
    Hyper-parameter & Tuning range & A-D & E & A-D & E & A-D & E\\
    \midrule
    Number of layers $K$ & [2, 3, 4, 5] & 5 & 5 & 5 & 5 & 5 & 5 \\
    Nodes per layer $N_K$ & [16, 32, 64, 128] & 32 & 32 & 32 & 32 & 32 & 32 \\
    \midrule
    dt regularisation \weightlossdt & [0.01 - 2.0] & - & - & 0.3 & 1.0 & 0.01 & 0.01\\
    Physics regularisation \weightlossfmax & [0.005 - 10] & - & - & - & - & 0.5 & 0.01\\
    Fade-in speed $E'$ & [10 - 50] & - & - & - & - & 15 & 15\\
    \midrule
    Initial learning rate & [0.1 - 2.0] & 1.0 & 1.6 & 0.5 & 2.0 & 1.2 & 1.0\\
    History size & [100 - 150] & 140 & 140 & 120 & 120 & 120 & 120 \\
    Maximum iterations & [18 - 28] & 22 & 22 & 23 & 20 & 20 & 19\\
    \bottomrule
  \end{tabular}
  }
\end{table}
% % \end{landscape}

% \newpage
%% If you have bibdatabase file and want bibtex to generate the
%% bibitems, please use
%%
 \bibliographystyle{elsarticle-num}
 \bibliography{references.bib}

%% else use the following coding to input the bibitems directly in the
%% TeX file.

% \begin{thebibliography}{00}

% %% \bibitem{label}
% %% Text of bibliographic item

% \bibitem{}

% \end{thebibliography}
\end{document}
\endinput
%%
%% End of file `elsarticle-template-num.tex'.
