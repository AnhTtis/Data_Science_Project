\documentclass{article}
\def\uu{\bigsqcup}
\def\eps{\varepsilon}
\def \Z {{\mathbf {Z}}}
\def \R {{\mathbf {R}}}
\def \N {{\mathbf {N}}}
\def \J {{\mathbf {J}}}
\def \A {{\mathcal {A}}}
\def \B {{\mathcal {B}}}
\textwidth=170mm
\oddsidemargin=1mm
%\def\exp{\mathop Exp}
\usepackage[T2A]{fontenc}
\usepackage[cp1251]{inputenc}
%\usepackage[english,russian]{babel}
\usepackage[tbtags]{amsmath}
\usepackage{amsfonts,amssymb}

%\usepackage{mathrsfs}

\usepackage{graphicx}
\graphicspath{{pictures/}}
\DeclareGraphicsExtensions{.pdf,.png,.jpg}


\hoffset 0mm
\voffset -7mm




\begin{document}

\title{  Quasi-similarity and joining-stable invariants of ergodic actions}
\author{ Valery V. Ryzhikov, Jean-Paul Thouvenot}
\date{}


\maketitle

\begin{abstract}   
Answering Vershik's  question we show that   quasi-similarity does not conserve the entropy, proving quasi-similarity of  all  Bernoulli actions of the groups with an  element of infinite order.
We produce joining-stable invariants considering zero  $P$-entropy systems and prove an analog of Pinsker's theorem establishing the  disjointness of  such systems
  with the  actions of comletely positive $P$-entropy. Then we apply Poisson suspensions
to get a class of the corresponding examples.
\end{abstract}
 
\Large
\section{Introduction}

In ergodic theory, there are several types of  equivalences
for measure-preserving actions, let us  mention some of them
in descending order of their strength:
\it isomorphism, weak isomorphism, quasi-similarity, spectral isomorphism. \rm
 

The isomorphism means that actions are conjugate by an ivertible
 measure-preserving transformation. The weak isomorphism of the 
two systems implies that each of them is isomorphic in the above sense to a factor of the other. Recall that the factors are the  restrictions of the action to some invariant sigma-algebras. Quasi-similarity of two actions means
  the existence of an injective  Markov operator with a dense image that intertwines these  
actions. Spectral isomorphism implies the presence of a unitary intertwining operator.



For example, all Bernoulli automorphisms are spectrally isomorphic 
(have countably multiple Lebesgue spectrum), but, as Kolmogorov showed, 
they can be distinguished   by 
entropy. Sinai proved that Bernoulli automorphisms with the same entropy
  are weakly isomorphic, and Ornstein proved their 
isomorphism (for these classical results, see, for example, \cite{KSF}). 

Vershik proposed the concept  of 
quasi-similarity \cite{V}, 
in connection with this several  problems arosed. One of them
   is solved by Fra\k czek and Lemanczyk \cite{FL}: they provided  two automorphisms that are quasi-similar but not weakly isomorphic. 
The question of the invariance  of entropy under the quasi-similarity remained open. In this note  we  
answer this question in  negative showing that all Bernoulli systems are  quasi-similar via Bernoulli joinings from \cite{T}. 



Let an action $T$ possess an inariant sigma-algebra, the restriction of $T$ to this algebra is called factor. We say $S$-factor, if this factor is isomorphic to
an action $S$.   

%If $S$-factors are independent from all in term of joining they are called  %disjoint.  Pinsker proved that  a zero entropy factor and a factor with
%$K$-property are disjoint

The theory of joinings  can be considered as the theory of the representations in a measure-preserving action a system of   $S_i$-factors for a given collection of actions $S_i$. 
An invariant of a dynamical system is called \it joining-stable \rm if any
 action generated by an arbitrary collection of factors with this invariant  possesses it too.  
The property \it to have zero entropy \rm is joining-stable.

Developing ideas from \cite{K} and \cite{R21} we  consider actions with 
zero $P$-entropy and comletely positive $P$-entropy,  give examples of the latter is the class of deterministic actions.  Using  joining stability for zero $P$-entropy, we prove an analog of Pinsker's theorem on disjointness of  an
action with zero  $P$-entropy   from any action with comletely positive $P$-entropy. 


\section{Quasi-similarity of Bernoulli actions}
\vspace{3mm}

\bf Theorem 2.1. \it All Bernoulli actions of a given  countable group with an element of infinite order are quasi-similar. \rm

\vspace{3mm}

Proof.  Let $S$   be the Bernoulli scheme of type $$\left(a, \frac {1} 2 - a, \frac 1 2\right), \ \ 0<a<1/2.$$
It has the following evident Bernoulli factors $S_P$, $S_Q$, where $S_P$ is of types $\left(a,  1  -a\right)$ and  $Q$ is of types  $\left(\frac 1 2 , \frac 1 2\right)$.
These factors  generate our  system $S$.

Let $E^Q$ be the orthogonal projection onto the space of the factor $S_Q$
(this space is denoted by $L_2(Q)$).
We define an  operator $\J$ setting  $\J=E^Q\,|_{L_2(P)}$.   Then $\J$ is a Markov injective operator that intertwines factors $S_P$ and $S_Q$, moreover  $\J L_2(P)$ is dense in  $ L_2(Q)$. So the factors $S_P$, $S_Q$ are quasi-similar.
The idea of constructing such $\J$   was taken from Lemma 4 \cite{T}.






\vspace{4mm}
Below  we present the above for the reader who is not familiar with Bernoulli joinings.
   Let us consider the partitions $$\xi=\{A, X\setminus A\}, 
\ \ 0<\mu(A)=a<1/2,$$ and $$\beta=\{ B, X\setminus B\},
\ \ \mu(B)=1/2, $$ where   $$X=[0,1], \ \ A=[0,a], \ \ B=[0,1/2]. $$

We define a Markov operator $J:L_2(X)\to L_2(X)$, which is an integral operator with the following $\xi\times\beta$-measurable  kernel $K(x,y)$:

\vspace{3mm}
\begin{center}
\includegraphics{1}
\end{center}
 

$$ K(x,y)=0, \ \ (x,y)\in A\times B; \ \ \ \ \ �K(x,y)=\frac 1{1-a},
�\ \ (x,y)\in (X\setminus A)\times B;$$

$$ K(x,y)=2,
�\ \ (x,y)\in A\times (X\setminus B); \ \ \ \ \ �K(x,y)=\frac {1-2a}{1-a},
�\ \ (x,y)\in (X\setminus A)\times (X\setminus B).$$





Recall that, by defenition,  Markov operators $J$  preserve the positivity of functions, moreover, under the action of  $J$ and $J^\ast$ the constants remain fixed.

Let   $S$ be the standard shift in the space $X^\Z$, let $\A$ denote  the sigma-algebra generated by the sets
$$ S^n(\dots\times X \times X \times A \times X \times X  \dots),$$
and $\B$ --  the  sigma-algebra generated by 
$$ S^n(\dots\times X \times X \times B \times X \times X  \dots).$$

The  restriction of  $S$  to
 $\A$ is the Bernoulli automorphism  $T_a$ with entropy equal to $H(\xi)=-a\log_2 a -(1-a)\log_2(1-a)$,  the  restriction of  $S$  to
 $\B$ is the Bernoulli automorphism  $T$ with entropy equal to $H(\beta)=1$. 

Now we consider a Markov operator $\J_a: L_2(X^\Z,\mu^Z)\to L_2(X^\Z,\mu^Z)$
setting $$ \J_a= \bigotimes_{z\in \Z} J =
 \dots\otimes J \otimes J \otimes J \otimes \dots.$$
The operator $\J_a$ intertwines $T_a$ with $T$:

$$ T\J_a=\J_a T_a.$$

The image of  $\J_a$  is dense:

$$ \overline{\J_a L_2(\A)}= L_2(\B),$$

and

$$ Ker\J_a={0},$$
since  tensor powers of injective $J$ are injective as well.
Thus, for all $a$, $0<a<1/2,$   we get that $T_a$ is quasi-similar to $T$.
 
Since $T_a^n$ is quasi-similar to $T^n_{a'}$, $0<a'<1/2,$ 
 for any $c, c'>0$ we find  Bernoulli automorphisms $T$,$T'$ such that
$h(T)=c$, $h(T')=c'$ and  $T$,$T'$ are quasi-similar.

Note also that the operators $$ \J_a\otimes \J_{a^2}\otimes \J_{a^3}\otimes\dots$$
intertwine finite entropy products of Bernoulli automorphisms with  the
infinite entropy automorphism $T\otimes T\otimes T\otimes\dots$. Due to Ornstein's isomorphism  theorem we complete  the proof for $\Z$-action.

For the group actions we literally repeat the above reasoning (considering $(X^G,\mu^G$) instead of $(X^\Z, \mu^\Z)$) and use Stepin's generalization 
 \cite{St} (Theorem 1)  of the mentioned result of Ornstein.

\vspace{3mm}

The following problem becomes relevant: \it  are there
  $K$-automorphisms that are not quasi-similar?\rm



\section{Joining-stable invariants and  $P$-entropy}
Recall that an invariant  of a dynamical system is joining-stable if any
 dynamical system generated by a collection  of factors with this invariant
inherits  it. \rm Examples of such invariants are zero entropy and  zero entropy along a sequence
(called the Kushnirenko entropy, see \cite{K}). Note that the $K$-property and even the continuity of the spectrum are not hereditarily stable invariants.  Two Bernoulli factors can generate a system with
a factor isomorphic to a  given ergodic system with  zero entropy \cite{ST}. 


\vspace{3mm}
\bf Theorem 3.1. \it Let the system $T$ have a joining-stable property,
and the system $S$ is such that it and any of its non-trivial factors do not have this
 property. Then $S$ and $T$ are disjoint. \rm

\vspace{3mm}
Proof.   If the  factors $S$ and  $T$ are not disjoint,
there is a system generated by a countable family $T$-factors that has a non-trivial factor isomorphic to a factor of $S$.  This assertion is term of joining can be found in \cite{T2}, lemme 3.2. From the conditions of theorem
we see that such non-trivial factor cannot existe. Thus, $S$ and $T$ must be  disjoint.

\vspace{3mm}
\bf $P$-entropy. \rm  We consider  the following modification \cite{R21}  of the Kushnirenko entropy.
  Let $P=\{P_j\}$ be a sequence of finite subsets in a countably infinite group $G$.
  For a measure-preserving action $T=\{T_g\}$ of the group $G$, we define 
$$h_j(T,\xi)=\frac 1 {|P_j|} H\left(\bigvee_{p\in P_j}T_p\xi\right),$$
$$h_{P}(T,\xi)={\limsup_j} \ h_j(T,\xi),$$
$$h_{P}(T)=\sup_\xi h_{P}(T,\xi),$$
%$$h^{inf}_{P}(T)=\sup_\xi\liminf_j \ h_j(T,\xi),$$
where $\xi$ denotes a finite measurable partition of the space $X$, and $H(\xi)$ is the entropy of the partition $\xi$:
$$ H(\{C_1,C_2,\dots, C_n\})=-\sum_{i=1}^n \mu( C_i)\ln \mu( C_i).$$

We consider below the following particular case $G=Z$, when $P_j$ are
growing progressions
   $$P_j=\{j,2j,\dots, L(j)j\}, \ \ \ L(j)\to\infty.$$

\vspace{3mm}
\bf Examples of zero $P$-entropy   automorphisms. \rm 
Let $T$  be of zero entropy. The powers $T^n$ also have zero entropy.
So for any $\xi$ and $j$ there is $L(j)$ such that $$h_j(T^j,\xi)<\frac 1 j.$$
We can choose such $j_k\to \infty$ that for $P=\{P_{j_k}:k\in \N\}$ and all finite partitions $\xi$ we get   ${\limsup_k} \ h_{j_k}(T,\xi) =0.$ , $h_P(T)=0$. 

\vspace{3mm}
\bf Lemma  3.2. \it   Zero    $P$-entropy is joinig-stable. \rm

\vspace{3mm}
Lemma follows easily from the fact that $\xi_1\vee\dots\vee \xi_n$ is
generating partition for $F_1\vee\dots\vee F_n$ , if $\xi_m$, $1\leq m\leq n$,  is a generating partition for the factor $F_m$.

\section{ Poisson suspensions with completely positive $P$-entropy} 
We say that  an action has completely positive $P$-entropy if any non-trivial factor
of the action has positive $P$-entropy. 
From Theorem 3.1 and Lemma 3.2 we get the  following assertion.

\vspace{3mm}
\bf Theorem 4.1.  \it Let $T$ have completely positive $P$-entropy  and let $S$ have zero
   $P$-entropy. Then $T$ and $S$ are disjoint ($S$-factor and $T$-factor are always independent). \rm

\vspace{3mm}

\rm 

To have   examples of deterministic completely positive $P$-entropy systems  we 
apply infinite transformations $T$ of  rank one  and their  Poisson suspension
$T_\circ$.   

\bf Rank one transformation. \rm Rank one constructions are determined by the parameters
  $h_1=1$, $r_j\geq 2$ and sets of integers
$$ \bar s_j=(s_j(1), s_j(2),\dots,s_j(r_j)), \ s_j(i)\geq 0, \ j\in\N. $$
The phase space $X$ for such constructions is the union of all 
towers $X_j$, where
$$X_j=\uu_{i=0}^{h_j-1} T^iB_j,$$
$T^iB_j$ are  intervals called floors. At stage $j$, the tower $X_j$ is cut into $r_j$  identical narrow subtowers $X_{j,i}$ (they are called columns),
  and over each column $X_{j,i}$ (of height 
$h_j$) we add  $s_j(i)$  new
  narrow floors. Then the column number $i+1$ stack 
over the column number $i$ to get   one column of height $h_{j+1}$, 
$$h_{j+1}=r_jh_j+\sum_{i=1}{r_j}s_j(i).$$

This column is considered as the tower $X_{j+1}$. This process  of  
cutting, adding  and stacking continues to infinity and we obtain  an invertible  transformation of $X$ that preserves the Lebesgue measure of intervals.
See, for example,\cite{R20},\cite{R23}, where numerous applications  including ones  for Gaussian and Poisson suspensions  have been indicated.


\vspace{3mm}
\bf The Poisson measure. \rm Consider the configuration space $X_\circ$, which consists of all infinite countable sets $x_\circ\subset X$ such that each above interval from  the spaces $X$ contains only a finite number of elements of the set $x_\circ$. 

The space $X_\circ$ is equipped with the Poisson  measure. We call its definition.  
For a  subset $A\subset X$ of a finite $\mu$-measure, we define configuration subsets
  $C(A,k)$, $k=0,1,2,\dots$,
  to $X_\circ$ by the formula
$$C(A,k)=\{x_\circ\in X_\circ \ : \ |x_\circ\cap A|=k\}.$$

  
 All possible finite intersections of the form $\cap_{i=1}^N C(A_i,k_i)$
    form a semiring.   
A Poisson measure $\mu_\circ$ is given on this semiring.
  Provided that the measurable sets $A_1, A_2,\dots, A_N$ do not intersect
and have a finite measure, we set
$$\mu_\circ(\bigcap_{i=1}^N C(A_i,k_i))=\prod_{i=1}^N \frac {\mu(A_i)^{k_i}}{k_i!} e^{-\mu(A_i)}.\eqno (\circ)$$


The meaning of this formula is as follows: if the sets $A$, $B$ do not intersect, then
   probability $\mu_\circ(C(A,k))\cap C(B,m))$ of $k$ points of configuration $x_\circ$ in $A$ simultaneously
    and $m$ points of the configuration $x_\circ$ in $B$ is equal to the product of the probabilities $\mu_\circ(C(A,k))$ and 
$\mu_\circ( C(B,m))$.
   In other words, the events $C(A,k)$ and $C(B,m)$ are independent. Since the sets $A_1, A_2,\dots,A_N$ do not intersect,
 so the product appears in the formula $(\circ)$. Any element of the semiring is a finite union
semiring elements for which the Poisson measure is defined by $(\circ)$. The measure extends from the semiring to the Poisson configuration space 
$(X_\circ,\mu_\circ)$,
  isomorphic to the standard Lebesgue probability space.

An automorphism $T$ of the space $(X,\mu)$  naturally induces an automorphism 
$T_\circ$ of the space $(X_\circ,\mu_\circ)$,
  this $T_\circ$   is called
Poisson suspension.
\newpage
%\vspace{3mm}
\bf  Examples of $T_\circ$  with completely positive $P$-entropy. \rm
Back to rank one transformations, let $s_j(i)>L(j)h_j$,  $L(i)\to\infty$. 
Then $\mu (X_j)\to\infty$  and moreover for the correponding rank one construction $T$ the sets 
$$X_j, \ T^{h_j}X_j,\ T^{2h_j}X_j,\ \dots,\  T^{L(j)h_j}X_j$$ do not intersect.  The same  is automatically true for all  $A\subset X_j$.  


Let  $C=C(A,k)$, where $A\subset X_{j_0}$, for the Poisson suspension $T_\circ$
we see that the sets
$$C, \ T^{h_j}_\circ C,\ T^{2h_j}_\circ C\  \dots, \ T^{L(j)h_j}_\circ C$$
are independent with respect to the Poisson measure.

Standard reasoning shows that $T_\circ$  has a completely positive $P$-entropy,
where $$P=\{ P_j \},  \ \ P_j=\{h_j, 2h_j,\dots, L(j)h_j\}.$$ 
Note also that the Poisson suspesions over the rank one transformations have zero entropy \cite{J}.  

Our theorems and examples have analogues for group actions, but more on that later.



\large


\begin{thebibliography}{99}


\bibitem{KSF} I.P. Kornfeld, Ya.G. Sinai, S.V. Fomin,  Ergodic theory, Moscow, 1980.

\bibitem{V} A. M. Vershik, Polymorphisms, Markov processes, and quasi-similarity, Discrete Contin. Dyn. Syst., 13:5 (2005), 1305-1324

\bibitem{FL} K. Fra\k czek, M. Lemanczyk, 
A note on quasi-similarity of Koopman operators,
J. Lond. Math. Soc., II. Ser. 82, No. 2 (2010), 361-375 

\bibitem{T}J.-P. Thouvenot, Remarques sur les systemes dynamiques donnes avec plusieurs facteurs, Israel Journal of Mathematics, 21 (1975), 215-232 

%\bibitem{LPT}Lemanczyk, M.; Parreau, F.; Thouvenot, J.-P.
%Gaussian automorphisms whose ergodic self-joinings are Gaussian. 
%Fundam. Math. 164, No. 3 (2000), 253-293 


\bibitem{St} A. M. Stepin. Bernoulli shifts on groups. (Russian) Dokl. Akad. Nauk SSSR 223 (1975),  no. 2, 300-302

\bibitem{ST} M. Smorodinsky,  J.-P. Thouvenot, Bernoulli factors that span a transformation,  Isr. J. Math., 32(1979), 39-43.

\bibitem{K}  A. G. Kushnirenko, On metric invariants of entropy type, Russian Math. Surveys, 22:5 (1967), 53-61 

\bibitem{R21} V. V. Ryzhikov, Compact families and typical entropy invariants of measure-preserving actions, Trans. Moscow Math. Soc., 82 (2021), 117-123 


\bibitem{T2}J.-P. Thouvenot,  Entropy, isomorphism and equivalence
in ergodic theory.  Handbook of dynamical systems, Vol. I A
Edited by B. Hasselblatt and A. Katok,
 Elsevier Science B.V, 2002. 

\bibitem{R20}  V. V. Ryzhikov, Measure-preserving rank one transformations, Trans. Moscow Math. Soc., 81:2 (2020), 229-259

\bibitem{R23} V. V. Ryzhikov, Spectra of self-similar ergodic actions, Math. Notes, 113:2 (2023), 274-281
\bibitem{J} E. Janvresse, T. Meyerovitch, T. de la Rue,  E. Roy. Poisson suspensions and entropy for infinite transformations. 
Trans. Amer. Math. Soc. 362 (2010), 3069-3094

\end{thebibliography}
\end{document}

