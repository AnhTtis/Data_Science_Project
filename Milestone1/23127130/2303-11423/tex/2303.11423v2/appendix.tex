\section{Appendix A: computation of MFCC features}

The process of computation of MFCC consists of a number of steps, as follows.
The first step is to apply a pre-emphasis filter to the signal \( x[n] \):
\begin{equation}
    y[n] = x[n] - \alpha x[n-1],
\end{equation}
where \( \alpha \) is typically between 0.95 and 0.97. Next, the signal is divided into overlapping frames. If the frame size is \( N \) and the hop size is \( M \), then each frame can be represented as:
\begin{equation}
    x_m[n] = x[n + mM], \quad m = 0, 1, 2, \ldots
\end{equation}
Each frame is then windowed using a window function \( w[n] \), such as the Hamming window:
\begin{equation}
    x_w[n] = x[n] \cdot w[n], \quad w[n] = 0.54 - 0.46 \cos\left(\frac{2\pi n}{N-1}\right)
\end{equation}
We then Compute the fast Fourier transform (FFT) of each windowed frame to obtain the magnitude spectrum:
\begin{equation}
    X[k] = \sum_{n=0}^{N-1} x_w[n] e^{-j2\pi k n / N}, \quad k = 0, 1, \ldots, N-1
\end{equation}
Next, we apply the mel filter bank to the magnitude spectrum to get the filter bank energies:
\begin{equation}
    E_m = \sum_{k=0}^{N-1} |X[k]|^2 H_m[k], \quad m = 1, 2, \ldots, M
\end{equation}
where \( H_m[k] \) is the mel filter bank.
We then proceed to take the logarithm of the filter bank energies:
\begin{equation}
    F_m = \log(E_m), \quad m = 1, 2, \ldots, M
\end{equation}
Finally, we apply the discrete Cosine transform to the logarithm of the filter bank energies to obtain the MFCCs:
\begin{equation}
    C_n = \sum_{m=0}^{M-1} F_m \cos\left[\frac{\pi n}{M}\left(m + \frac{1}{2}\right)\right], \quad n = 0, 1, \ldots, L-1
\end{equation}
where \( L \) is the number of desired MFCC features.
