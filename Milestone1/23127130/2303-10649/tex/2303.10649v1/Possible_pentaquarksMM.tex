\documentclass[amsmath,amssymb,showkeys,superscriptaddress,floatfix,prd,10pt,aps]{revtex4-2}
%\usepackage{graphicx}
\usepackage{graphicx,epstopdf}
\usepackage{textcomp}
%\usepackage{pdfpages}
%\pdfminorversion=7
\setlength{\paperheight}{11in}
\usepackage[caption=false]{subfig}
\usepackage{multirow}
\usepackage{appendix}
%\bibliographystyle{apsrev4-1}
\bibliographystyle{elsarticle-num}
\def\pslash{p\!\!\!\slash }
\def\qslash{q\!\!\!\slash }
\def\xslash{x\!\!\!\slash }
\def\eslash{\varepsilon\!\!\!\slash }
\def\ve{\vert}
\def\vel{\left|}
\def\ver{\right|}
\usepackage{colortbl}
\usepackage{xcolor}
\usepackage[colorlinks=true, urlcolor=blue, linkcolor=green, citecolor=green]{hyperref}



\begin{document}

\title{Electromagnetic properties of $\bar D^{(*)}\Xi^{\prime}_c$, $\bar D^{(*)}\Lambda_c$,  $\bar D_s^{(*)}\Lambda_c$ and  $\bar D_s^{(*)}\Xi_c$  pentaquarks}

\author{Ula\c{s} \"{O}zdem}%
\email[]{ulasozdem@aydin.edu.tr}
\affiliation{Health Services Vocational School of Higher Education, Istanbul Aydin University, Sefakoy-Kucukcekmece, 34295 Istanbul, T\"{u}rkiye}


\date{\today}
 
\begin{abstract}
To elucidate the internal structure of exotic states is
one of the central purposes of hadron physics. 
Motivated by this, we study the electromagnetic properties of $\bar D^{(*)}\Xi^{\prime}_c$, $\bar D^{(*)}\Lambda_c$,  $\bar D_s^{(*)}\Lambda_c$ and  $\bar D_s^{(*)}\Xi_c$  pentaquarks without strange, with strange and with double strange through QCD light-cone sum rules. We have also evaluated electric quadrupole and magnetic octupole moments of the $\bar D^{*}\Xi^{\prime}_c$, $\bar D^{*}\Lambda_c$,  $\bar D_s^{*}\Lambda_c$ and  $\bar D_s^{*}\Xi_c$ pentaquark states.   The magnetic dipole moment is the leading-order response of a bound system to a soft external magnetic field.  
Thus, it ensures a prominent platform for examination of the internal organizations of hadrons governed by the quark-gluon dynamics of QCD. We look forward to that the present study can stimulate the interest of experimentalist in investigating the electromagnetic properties of the hidden-charmed pentaquark states.
\end{abstract}
\keywords{Magnetic dipole  moment, electromagnetic properties,  molecular picture, hidden-charm pentaquarks, QCD light-cone sum rules}

\maketitle



 


\section{Introduction}\label{motivation}

Although it was suggested long ago that non-conventional states other than standard hadrons might exist, the experimental evidence of these states  was taken to a new level in 2003 with the Belle Collaboration observation of the X(3872) particle \cite{Belle:2003nnu}.
Since then a series of states beyond the non-conventional states have been discovered. Most of those are potentially exotic states such as tetraquarks, pentaquarks, hybrid mesons, glueballs and so on. The exploration of exotic states, particularly how the quarks are grouped inside (i.e., molecular or compact configuration) plays a important role for comprehension of behavior of the low energy QCD. Therefore, investigating the properties of these states and shedding light on their nature is one of the most active areas of research in both experimental and theoretical hadron physics \cite{Esposito:2014rxa,Esposito:2016noz,Olsen:2017bmm,Lebed:2016hpi,Nielsen:2009uh,Brambilla:2019esw,Agaev:2020zad,Chen:2016qju,Ali:2017jda,Guo:2017jvc,Liu:2019zoy,Yang:2020atz,Dong:2021juy,Dong:2021bvy,Meng:2022ozq, Chen:2022asf}.



In 2015, the LHCb Collaboration reported two reasonance states, $P_{c }(4380)$ and $P_{c }(4450)$, in the  $J/\psi\,p$ invariant mass distribution of the $\Lambda_b \rightarrow J/\psi\,p\,K$ decay~\cite{Aaij:2015tga}. In 2019, the LHCb Collaboration updated the observations in the  $\Lambda_b \rightarrow J/\psi\,p\,K$  process by using more data \cite{Aaij:2019vzc}, they not only discovered a new narrow pentaquark state, $P_{c}(4312)$, but also found that the $P_{c }(4450)$  consists of two narrow overlapping peaks, $P_{c}(4440)$, and $P_{c}(4457)$.  In 2020, the LHCb collaboration reported an evidence of a hidden-charm pentaquark candidate, $P_{cs}(4459)$, in the invariant mass spectrum of $J/\psi\Lambda$ in the $\Xi_b^0 \rightarrow J/\psi\,\Lambda\,K^-$ decay~\cite{Aaij:2020gdg}.  In 2021, the LHCb Collaboration reported an evidence for a new pentaquark candidate, $P_c(4337)$,  in $B_s \rightarrow J/\psi\, p \bar p$ decays~\cite{LHCb:2021chn}. Recently, the LHCb Collaboration announced the
observed a new pentaquark state, $P_{cs}(4338)$, in the $J/\psi\Lambda$ mass distribution in the $B^- \rightarrow J/\psi\Lambda^- p $ decays~\cite{Collaboration:2022boa}. The masses, widths and minimal valence quark contents of hidden-charmed pentaquark states are presented in Table \ref{pentaquarks}. Systematical research on their nature and inner structure can progress our comprehension of the non-perturbative behaviors of the strong interaction. 
  Therefore, a lot of theoretical explanations have been dedicated to figure out the  natures of  hidden-charmed pentaquark states~\cite{Shen:2020gpw, Wang:2019nvm, Wu:2010jy, Chen:2016ryt, Shen:2019evi, Xiao:2019gjd, Anisovich:2015zqa, Feijoo:2015kts, Lu:2016roh, Liu:2020hcv, Zou:2021sha, Karliner:2021xnq, Peng:2020hql, Zhu:2021lhd, Hu:2021nvs, Du:2021bgb,  Xiao:2021rgp, Chen:2020uif, Chen:2015moa, Chen:2016otp, Xiang:2017byz, Chen:2019bip, Chen:2020pac, Chen:2020opr, Chen:2020kco, Chen:2021tip, Chen:2022onm, Chen:2019asm, Wu:2021caw, Lu:2021irg, Yang:2021pio, Cheng:2021gca, Clymton:2021thh, Liu:2020ajv, Deng:2022vkv, Shi:2021wyt, Wang:2022gfb, Wang:2021itn, Wang:2020eep, Azizi:2021utt, Wang:2022neq,Karliner:2022erb, Wang:2022mxy,Yan:2022wuz,Meng:2022wgl,Burns:2022uha, Feijoo:2022rxf, Zhu:2022wpi, Azizi:2022qll}.




\begin{table}[htp]
\caption{The masses, widths and minimal valence quark contents of hidden-charmed pentaquark states.}\label{pentaquarks}
\begin{tabular}{l|c|c|cc}
\toprule%[0.8pt] 
%\noalign{\smallskip}
State  & Mass (MeV) & Width (MeV) & Content \\
\toprule%[0.8pt] 
%\noalign{\smallskip}
$P_c(4380)^+$  &            $4380\pm8\pm29$          &       $215\pm18\pm86$           & $uudc\bar{c}$ \\
%\noalign{\smallskip}
$P_c(4312)^+$       & ~~$4311.9\pm0.7^{~+6.8}_{~-0.6}$~~  &  $9.8\pm2.7^{~+3.7}_{~-4.5}$    & $uudc\bar{c}$ \\
%\noalign{\smallskip}
$P_c(4440)^+$      &    $4440.3\pm1.3^{~+4.1}_{~-4.7}$   &  $20.6\pm4.9^{~+8.7}_{~-10.1}$  & $uudc\bar{c}$ \\
%\noalign{\smallskip}
$P_c(4457)^+$      &    $4457.3\pm0.6^{~+4.1}_{~-1.7}$   &  $6.4\pm2.0^{~+5.7}_{~-1.9}$    & $uudc\bar{c}$ \\
%\noalign{\smallskip}
$P_{cs}(4459)^0$   &    $4458.8\pm2.9^{~+4.7}_{~-1.1}$   &  $17.3\pm6.5^{~+8.0}_{~-5.7}$   & $udsc\bar{c}$ \\
%\noalign{\smallskip}
$P_c(4337)^+$       &       $4337^{~+7~+2}_{~-4~-2}$      &  $29^{~+26~+14}_{~-12~-14}$     & $uudc\bar{c}$ \\
%\noalign{\smallskip}
$P_{cs}(4338)^0$   &    $4338.2 \pm 0.7 \pm 0.4$   &  $7.0 \pm 1.2 \pm 1.3$    & $udsc\bar{c}$ \\
\toprule%[0.8pt]
\end{tabular}
\end{table}

The discovery of pentaquark states in the $J/\psi\,p$ system at LHCb opened a new era of examination in hadron spectroscopy. Although mass spectra, decay behavior, and production mechanisms of pentaquark states have attracted much attention on both the theoretical and experimental sides, examination of the electromagnetic features of the these states have not received plenty of attention ~\cite{Wang:2016dzu, Ozdem:2018qeh, Ortiz-Pacheco:2018ccl,Xu:2020flp,Ozdem:2021btf, Li:2021ryu,Ozdem:2021ugy,Gao:2021hmv, Ozdem:2022iqk, Ozdem:2022vip, Ozdem:2022kei, Wang:2022tib}. The magnetic dipole moment is another intrinsic observable of hadrons which may contain prominent data of its quark gluon structure and underlying dynamics.  Different magnetic dipole moments will affect both the total and differential cross sections in the photo- or electro-production of pentaquarks. Therefore, such an investigation will deepen our knowledge of pentaquark states and help us understanding of the underlying dynamics governing their formation.  
%  
In this study, we investigate the magnetic dipole moment of $\bar D^{(*)}\Xi^{\prime}_c$, $\bar D^{(*)}\Lambda_c$,  $\bar D_s^{(*)}\Lambda_c$ and  $\bar D_s^{(*)}\Xi_c$  pentaquarks without strange, with strange and with double strange. % through QCD light-cone sum rules.  
Since the magnetic dipole moments belong to the non-perturbative domain of the QCD, we need non-perturbative methods to be able to perform the calculations of these parameters. The QCD light-cone sum rule is one of the powerful methods developed to calculate non-perturbative effects. In this technique a proper correlation function is computed concerning hadronic parameters and their magnetic dipole moment on one side and quark-gluon degrees of freedom and distribution amplitudes of the on-shell photon on the other side. The distribution amplitudes of photon are expressed concerning functions of different twists. Then, continuum subtraction and Borel transform procedures provided by the quark-hadron duality ansatz are carried out to remove the contributions coming from the higher states and continuum. By equating the results acquired in two different regions, the related sum rules for the magnetic dipole moment of the hadrons are obtained~\cite{Chernyak:1990ag, Braun:1988qv, Balitsky:1989ry}.  


This paper is organized as follows.  We introduce our theoretical framework in Sec. \ref{formalism} and the corresponding numerical results and discussions in Sec. \ref{numerical}.  Finally, a short summary is given in Sec. \ref{summary}.


\begin{widetext}
 
\section{The QCD light-cone sum rules for pentaquark states}\label{formalism}



\subsection{Formalism of the  spin-$\frac{1}{2}$ pentaquarks} %$P_{c\bar c}^{1/2} $ state
 

The correlation function required to perform magnetic dipole moment analysis of the spin-1/2 pentaquark states ( here after $P_{c \bar c}^{1/2} $ ) is given in the following form:
\begin{eqnarray} \label{edmn01}
\Pi(p,q)&=&i\int d^4x e^{ip \cdot x} \langle0|T\left\{J^{P_{c\bar c}^{1/2}}(x)\bar{J}^{P_{c\bar c}^{1/2}}(0)\right\}|0\rangle _\gamma \, , 
\end{eqnarray}
where  $q$ is the momentum of the photon, sub-indice $\gamma$ is the external electromagnetic field,   and $J(x)$ stand for the interpolating currents of  the $P_{c\bar c}^{1/2}$ states.  The corresponding interpolating currents are given by
%
\begin{align}\label{curpcs1}
J^{1}(x)&=\frac{1}{\sqrt{2}}\Big \{ \mid \bar D^0 \Xi^{\prime 0}_c \rangle \, - \mid \bar D^- \Xi^{\prime +}_c \rangle  \Big \}=\frac{1}{\sqrt{2}} \Big \{ \big[\bar c^d(x)i \gamma_5 u^d(x)\big]\big[\varepsilon^{abc} d^{a^T}(x)C\gamma_\mu s^b(x) \gamma^\mu \gamma_5 c^c(x)\big]  \nonumber\\
&
- \big[\bar c^d(x)i \gamma_5 d^d(x)\big]\big[\varepsilon^{abc} u^{a^T}(x)C\gamma_\mu s^b(x)\gamma^\mu \gamma_5
 c^c(x)\big] \Big\} \, ,\nonumber\\
 %%%%%%%%%%%%%%%%%%%%%%%%%%%%%%%%%%%%%%%%%%%%%%%%%%%%%%%%%
 J^{2}(x)&=\mid \bar D^0 \Lambda^{+}_c \rangle   = \Big \{ \big[\bar c^d(x) i\gamma_5 u^d(x)\big]
 \big[\varepsilon^{abc}u^{a^T}(x)C\gamma_5 d^b(x) c^c(x)\big]\Big \}\, ,  \nonumber\\
%%%%%%%%%%%%%%%%%%%%%%%%%%%%%%%%%%%%%%%%%%%%%%%%%%%%%%%%%
J^{3}(x)&=\mid \bar D^- \Lambda^{+}_c \rangle   = \Big \{ \big[\bar c^d(x) i\gamma_5 d^d(x)\big]
 \big[\varepsilon^{abc}u^{a^T}(x)C\gamma_5 d^b(x) c^c(x)\big]\Big \}\, ,  \nonumber\\
%%%%%%%%%%%%%%%%%%%%%%%%%%%%%%%%%%%%%%%%%%%%%%%%%%%%%%%%%
J^{4}(x)&=\mid  \bar D_s^- \Xi^{0}_c \rangle   = \Big \{ \big[\bar c^d(x) i\gamma_5 s^d(x)\big]
 \big[\varepsilon^{abc}d^{a^T}(x)C\gamma_5 s^b(x) c^c(x)\big]\Big \}\, ,  \nonumber\\
%%%%%%%%%%%%%%%%%%%%%%%%%%%%%%%%%%%%%%%%%%%%%%%%%%%%%%%%%
J^{5}(x)&=\mid \bar D_s^- \Lambda^{+}_c \rangle   = \Big \{ \big[\bar c^d(x) i\gamma_5 s^d(x)\big]
 \big[\varepsilon^{abc}u^{a^T}(x)C\gamma_5 d^b(x) c^c(x)\big]\Big \}\, ,  
%%%%%%%%%%%%%%%%%%%%%%%%%%%%%%%%%%%%%%%%%%%%%%%%%%%%%%%%%
 \end{align}
where $u(x)$, $d(x)$, $s(x)$ and $c(x)$ being quark fields, $a$, $b$, $c$ and  $d$ stand for color indices; and the $C$ denotes the charge conjugation operator. 

%We would like to point out that numerous possible interpolating currents can be written for pentaquark states, however, the number of possible interpolating currents that can be written can be slightly reduced when the QCD sum rules and the states to be investigated are considered.  Additionally, we should define the isospins of the interpolating currents to make reliable estimations, it is the key issue to solving the puzzle of those $P_{cs}$ states. Therefore, we construct the local color singlet-singlet type five-quark currents with the definite isospins, which couple potentially to the color singlet-singlet type $P_{cs}$ states rather than to the meson-baryon scattering states or thresholds.






At QCD side, we contract the relevant quark operators in the correlation function at the quark-gluon level with the help of the Wick's theorem. After these manipulations the correlation function of the QCD side are obtained in terms of the light and heavy-quark propagators, and distribution amplitudes (DAs) of the photon:
%
\begin{align}
\label{QCD1}
\Pi_1^{QCD}(p,q)&=- \frac{i}{2}\varepsilon^{abc} \varepsilon^{a^{\prime}b^{\prime}c^{\prime}}\, \int d^4x \, e^{ip\cdot x} \langle 0\mid \Big\{ 
% 
\, \mbox{Tr}\Big[\gamma_5 S_{u}^{dd^\prime}(x) \gamma_5  S_{c}^{d^\prime d}(-x)\Big]  
\mbox{Tr}\Big[\gamma_\mu S_s^{bb^\prime}(x) \gamma_\nu \widetilde S_{d}^{aa^\prime}(x)\Big]
\nonumber\\
&
%     
- \mbox{Tr}\Big[\gamma_5 S_{u}^{da^\prime}(x) \gamma_\nu  \widetilde S_s^{bb^\prime}(x) \gamma_\mu S_{d}^{ad^\prime}(x)   \gamma_5 S_{c}^{d^\prime d}(-x)\Big] 
\nonumber\\ 
&
 - \mbox{Tr}\Big[\gamma_5 S_{d}^{da^\prime}(x) \gamma_\nu  \widetilde S_s^{bb^\prime}(x) \gamma_\mu  S_{u}^{ad^\prime}(x) \gamma_5 S_{c}^{d^\prime d}(-x)\Big]
     \nonumber\\ 
&   +
 \mbox{Tr}\Big[\gamma_5     S_{d}^{dd^\prime}(x) \gamma_5 S_{c}^{d^\prime d}(-x)\Big]  
\mbox{Tr}\Big[\gamma_\mu S_s^{bb^\prime}(x)
  \gamma_\nu 
    \widetilde S_{u}^{aa^\prime}(x)\Big] \Big\} \Big( \gamma_\mu \gamma_5 S_c^{cc^{\prime}}(x)\gamma_5 \gamma_\nu \Big)
\mid 0 \rangle _\gamma \, , 
\end{align}
\begin{align}
\label{QCD2}
\Pi_2^{QCD}(p,q)&=i\varepsilon^{abc} \varepsilon^{a^{\prime}b^{\prime}c^{\prime}}\, \int d^4x \, e^{ip\cdot x} \langle 0\mid \Big\{ 
% 
\, \mbox{Tr}\Big[\gamma_5 S_{u}^{dd^\prime}(x) \gamma_5  S_{c}^{d^\prime d}(-x)\Big]  
\mbox{Tr}\Big[\gamma_5 S_d^{bb^\prime}(x) \gamma_5 \widetilde S_{u}^{aa^\prime}(x)\Big]
\nonumber\\
&
- \mbox{Tr}\Big[\gamma_5 S_{u}^{da^\prime}(x) \gamma_5  \widetilde S_d^{bb^\prime}(x) \gamma_5 S_{u}^{ad^\prime}(x)   \gamma_5 S_{c}^{d^\prime d}(-x)\Big] 
 \Big\} \Big( S_c^{cc^{\prime}}(x) \Big)
\mid 0 \rangle _\gamma \, , 
\end{align}
\begin{align}
\label{QCD3}
\Pi_3^{QCD}(p,q)&=i\varepsilon^{abc} \varepsilon^{a^{\prime}b^{\prime}c^{\prime}}\, \int d^4x \, e^{ip\cdot x} \langle 0\mid \Big\{ 
% 
\, \mbox{Tr}\Big[\gamma_5 S_{d}^{dd^\prime}(x) \gamma_5  S_{c}^{d^\prime d}(-x)\Big]  
\mbox{Tr}\Big[\gamma_5 S_d^{bb^\prime}(x) \gamma_5 \widetilde S_{u}^{aa^\prime}(x)\Big]
\nonumber\\
&
- \mbox{Tr}\Big[\gamma_5 S_{d}^{db^\prime}(x) \gamma_5  \widetilde S_u^{aa^\prime}(x) \gamma_5 S_{d}^{bd^\prime}(x)   \gamma_5 S_{c}^{d^\prime d}(-x)\Big] 
 \Big\} \Big( S_c^{cc^{\prime}}(x) \Big)
\mid 0 \rangle _\gamma \, , 
\end{align}
\begin{align}
\label{QCD4}
\Pi_4^{QCD}(p,q)&=i\varepsilon^{abc} \varepsilon^{a^{\prime}b^{\prime}c^{\prime}}\, \int d^4x \, e^{ip\cdot x} \langle 0\mid \Big\{ 
% 
\, \mbox{Tr}\Big[\gamma_5 S_{s}^{dd^\prime}(x) \gamma_5  S_{c}^{d^\prime d}(-x)\Big]  
\mbox{Tr}\Big[\gamma_5 S_s^{bb^\prime}(x) \gamma_5 \widetilde S_{d}^{aa^\prime}(x)\Big]
\nonumber\\
&
- \mbox{Tr}\Big[\gamma_5 S_{s}^{db^\prime}(x) \gamma_5  \widetilde S_d^{aa^\prime}(x) \gamma_5 S_{s}^{bd^\prime}(x)   \gamma_5 S_{c}^{d^\prime d}(-x)\Big] 
 \Big\} \Big( S_c^{cc^{\prime}}(x) \Big)
\mid 0 \rangle _\gamma \, , 
\end{align}
\begin{align}
\label{QCD5}
\Pi_5^{QCD}(p,q)&=i\varepsilon^{abc} \varepsilon^{a^{\prime}b^{\prime}c^{\prime}}\, \int d^4x \, e^{ip\cdot x} \langle 0\mid \Big\{ 
% 
\, \mbox{Tr}\Big[\gamma_5 S_{s}^{dd^\prime}(x) \gamma_5  S_{c}^{d^\prime d}(-x)\Big]  
\mbox{Tr}\Big[\gamma_5 S_d^{bb^\prime}(x) \gamma_5 \widetilde S_{u}^{aa^\prime}(x)\Big]
 \Big\} \Big( S_c^{cc^{\prime}}(x) \Big)
\mid 0 \rangle _\gamma \, , 
\end{align}
where   
%\begin{equation*}
$\widetilde{S}_{c(q)}^{ij}(x)=CS_{c(q)}^{ij\mathrm{T}}(x)C$ and,
%\end{equation*}%
 $S_{c}(x)$ and $S_{q}(x)$ denote the charm and light-quark propagators. The explicit expressions of
these terms can be written as~\cite{Yang:1993bp, Belyaev:1985wza}
%
\begin{align}
\label{edmn13}
S_{q}(x)&= \frac{1}{2 \pi x^2}\Big(i \frac{\xslash}{x^2}- \frac{m_q}{2}\Big) 
- \frac{\langle \bar qq \rangle }{12} \Big(1-i\frac{m_{q} \xslash}{4}   \Big)
%
- \frac{ \langle \bar qq \rangle }{192}
m_0^2 x^2  \Big(1-i\frac{m_{q} \xslash}{6}   \Big)
-\frac {i g_s }{32 \pi^2 x^2} ~G^{\mu \nu} (x) 
%
\Big[\rlap/{x} 
\sigma_{\mu \nu} +  \sigma_{\mu \nu} \rlap/{x}
 \Big],%\\
%\nonumber\\
\end{align}%
%and
%%
\begin{align}
\label{edmn14}
S_{c}(x)&=\frac{m_{c}^{2}}{4 \pi^{2}} \Bigg[ \frac{K_{1}\Big(m_{c}\sqrt{-x^{2}}\Big) }{\sqrt{-x^{2}}}
+i\frac{{\xslash}~K_{2}\Big( m_{c}\sqrt{-x^{2}}\Big)}
{(\sqrt{-x^{2}})^{2}}\Bigg]
%
-\frac{g_{s}m_{c}}{16\pi ^{2}} \int_0^1 dv\, G^{\mu \nu }(vx)\Bigg[ (\sigma _{\mu \nu }{\xslash}
  +{\xslash}\sigma _{\mu \nu }) \frac{K_{1}\Big( m_{c}\sqrt{-x^{2}}\Big) }{\sqrt{-x^{2}}}
   \nonumber\\
  &
+2\sigma_{\mu \nu }K_{0}\Big( m_{c}\sqrt{-x^{2}}\Big)\Bigg],
\end{align}%
%
where $\langle \bar qq \rangle$ stands for light-quark condensate, $G^{\mu\nu}$ denotes the gluon field strength tensor, $v$ is line variable, and  $K_i$'s are modified Bessel functions of the second kind.   
%

The correlation functions in Eqs. (\ref{QCD1})-(\ref{QCD5})  receive different contributions: perturbative, i.e., when a photon interacts perturbatively with quark propagators, and nonperturbative, i.e., photon interacts with light quarks at a large distance, contributions. Details of this procedure applied to obtain the expression of perturbative and non-perturbative contributions are given in Refs.~\cite{Ozdem:2022vip,Ozdem:2022eds}. %The relevant contributions have been calculated by applying this procedure. 
With this procedure, the calculation of the QCD side of the correlation function is completed.





To obtain second representation of the correlation function, the hadronic side, we isolate the ground state contributions from pentaquark states with spin-parity $J^P =\frac{1}{2}^-$, and acquire the hadronic side,
%
 \begin{align}\label{edmn02}
\Pi^{Had}(p,q)&=\frac{\langle0\mid J^{P_{c \bar c}^{1/2}}(x) \mid
{P_{c \bar c}^{1/2}}(p, s) \rangle}{[p^{2}-m_{P_{c \bar c}^{1/2}}^{2}]} 
\langle {P_{c \bar c}^{1/2}}(p, s)\mid
{P_{c \bar c}^{1/2}}(p+q, s)\rangle_\gamma 
\frac{\langle {P_{c \bar c}^{1/2}}(p+q, s)\mid
\bar J^{P_{c \bar c}^{1/2}}(0) \mid 0\rangle}{[(p+q)^{2}-m_{P_{c \bar c}^{1/2}}^{2}]}+ \cdots 
\end{align}
%where  
where dots represent the contributions of excited states and continuum.


The matrix elements in the above equation are given in terms of hadronic parameters and form factors as: 
%
%\begin{widetext}
\begin{align} 
\langle0\mid J^{P_{c \bar c}^{1/2}}(x)\mid {P_{c \bar c}^{1/2}}(p, s)\rangle=&\lambda_{P_{c \bar c}^{1/2}} \gamma_5 \, u(p,s),\label{edmn04}\\
\nonumber\\
\langle {P_{c \bar c}^{1/2}}(p+q, s)\mid\bar J^{P_{c \bar c}^{1/2}}(0)\mid 0\rangle=&\lambda_{P_{c \bar c}^{1/2}} \gamma_5 \, \bar u(p+q,s)\label{edmn004}
,\\
\nonumber\\
%\end{align}
%\begin{align}
\langle {P_{c \bar c}^{1/2}}(p, s)\mid {P_{c \bar c}^{1/2}}(p+q, s)\rangle_\gamma &=\varepsilon^\mu\,\bar u(p, s)\Big[\big[F_1(q^2)
+F_2(q^2)\big] \gamma_\mu +F_2(q^2)
\frac{(2p+q)_\mu}{2 m_{P_{c \bar c}^{1/2}}}\Big]\,u(p+q, s). \label{edmn005}
\end{align}
%


Then Eqs. (\ref{edmn04})-(\ref{edmn005}) are inserted in the Eq. (\ref{edmn02}), we get the following result for the hadronic side of the correlation function,
%
\begin{align}
\label{edmn05}
\Pi^{Had}(p,q)=&\lambda^2_{P_{c \bar c}^{1/2}}\gamma_5 \frac{\Big(\pslash+m_{P_{c \bar c}^{1/2}} \Big)}{[p^{2}-m_{{P_{c \bar c}^{1/2}}}^{2}]}\varepsilon^\mu \Bigg[\big[F_1(q^2) %
%\nonumber\\
%&
+F_2(q^2)\big] \gamma_\mu
+F_2(q^2)\, \frac{(2p+q)_\mu}{2 m_{P_{c \bar c}^{1/2}}}\Bigg]  \gamma_5 
\frac{\Big(\pslash+\qslash+m_{P_{c \bar c}^{1/2}}\Big)}{[(p+q)^{2}-m_{{P_{c \bar c}^{1/2}}}^{2}]}. 
\end{align}
%
In obtaining the above analytical expression, summation over spins of $P_{c\bar c}^{1/2}$
\begin{align}
\label{edmn0004}
 \sum_s u(p,s)\bar u(p,s)&=\pslash+m_{P_{c\bar c}^{1/2}},\\
  \sum_s u(p+q,s)\bar u(p+q,s)&=(\pslash+\qslash)+m_{P_{c\bar c}^{1/2}},
\end{align}
were also carried out. 

%
The  magnetic dipole form factor, $F_M(q^2)$, is written with respect to the $F_1(q^2)$ and $F_2(q^2)$ form factor at different $q^2$ :
\begin{align}
\label{edmn07}
&F_M(q^2) = F_1(q^2) + F_2(q^2).
\end{align}
 In case of on-shell photon, i.e. $q^2 = 0 $,  the %$F_M (q^2 = 0)$ 
 magnetic dipole form factor is proportional to the magnetic dipole moment $\mu_{P_{c \bar c}^{1/2}}$:
\begin{align}
\label{edmn08}
&\mu_{P_{c \bar c}^{1/2}} = \frac{ e}{2\, m_{P_{c \bar c}^{1/2}}} \,F_M(q^2 = 0).
\end{align}



%
The QCD light-cone sum rules for the magnetic dipole moments of the spin-1/2 pentaquarks are extracted by equating the coefficients of
the structure $\eslash \qslash$ from hadronic and QCD sides of the correlation function. To remove the effects coming from the higher states and continuum, Borel transformation as well as continuum subtraction provided by the quark-hadron duality approximation are used.  When all the above-mentioned procedures are performed, the following results are obtained for the magnetic dipole moments:
%
\begin{align}
\label{edmn15}
\mu_1 \,\lambda^2_1\, m_1 &=e^{\frac{m^2_1}{M^2}}\, \Delta_1^{QCD} (M^2,s_0),~~~~~~~~
\mu_2 \,\lambda^2_2\, m_2 =e^{\frac{m^2_2}{M^2}}\, \Delta_2^{QCD} (M^2,s_0),\nonumber\\
\mu_3 \,\lambda^2_3\, m_3 &=e^{\frac{m^2_3}{M^2}}\, \Delta_3^{QCD} (M^2,s_0),~~~~~~~~
\mu_4 \,\lambda^2_4\, m_4 =e^{\frac{m^2_4}{M^2}}\, \Delta_4^{QCD} (M^2,s_0),\nonumber\\
\mu_5 \,\lambda^2_5\, m_5 &=e^{\frac{m^2_5}{M^2}}\, \Delta_5^{QCD} (M^2,s_0),
\end{align}
%
where $\mu_i$, $m_i$ and $\lambda_i$ are magnetic dipole moment, mass and residue of the related pentaquark states. 
As an example, only the explicit expressions of the $\Delta_1^{QCD} (M^2,s_0)$ function are given in appendix, the rest of the  $\Delta_i^{QCD} (M^2,s_0)$ expressions are in similar forms.



\subsection{Formalism of the  spin-$\frac{3}{2}$ pentaquarks} 

For the magnetic dipole moment of spin-3/2 pentaquark states ( here after $P_{c \bar c}^{3/2} $ ) needed correlation function is introduced as 
 %
\begin{eqnarray} \label{Pc101}
\Pi_{\mu\nu}(p,q)&=&i\int d^4x e^{ip \cdot x} \langle0|T\left\{J_\mu^{P_{c \bar c}^{3/2}}(x)\bar{J}_\nu^{P_{c \bar c}^{3/2}}(0)\right\}|0\rangle _\gamma \, .
\end{eqnarray}
%where $J_\mu^{P_{c \bar c}^{3/2}}(x)$ is interpolating current of  $P_{c \bar c}^{3/2} $ pentaquark state. 
The interpolating currents used in the magnetic dipole moment analyzes of $P_{c \bar c}^{3/2} $ pentaquark states are written as follows~\cite{Wang:2022neq}:
%
\begin{align}\label{curpcs2}
J_\mu^{1}(x)&=\frac{1}{\sqrt{2}}\Big \{ \mid \bar D^{*0} \Xi^{\prime 0}_c \rangle \, - \mid \bar D^{*-} \Xi^{\prime +}_c \rangle  \Big \}=\frac{1}{\sqrt{2}} \Big \{ \big[\bar c^d(x)\gamma_\mu u^d(x)\big]\big[\varepsilon^{abc} d^{a^T}(x)C\gamma_\alpha s^b(x) \gamma^\alpha \gamma_5 c^c(x)\big]  \nonumber\\
&
- \big[\bar c^d(x)\gamma_\mu d^d(x)\big]\big[\varepsilon^{abc} u^{a^T}(x)C\gamma_\alpha s^b(x)\gamma^\alpha \gamma_5
 c^c(x)\big] \Big\} \, ,\nonumber\\
 %%%%%%%%%%%%%%%%%%%%%%%%%%%%%%%%%%%%%%%%%%%%%%%%%%%%%%%%%
 J_\mu^{2}(x)&=\mid \bar D^{*0} \Lambda^{+}_c \rangle   = \Big \{ \big[\bar c^d(x) \gamma_\mu u^d(x)\big]
 \big[\varepsilon^{abc}u^{a^T}(x)C\gamma_5 d^b(x) c^c(x)\big]\Big \}\, ,  \nonumber\\
%%%%%%%%%%%%%%%%%%%%%%%%%%%%%%%%%%%%%%%%%%%%%%%%%%%%%%%%%
J_\mu^{3}(x)&=\mid \bar D^{*-} \Lambda^{+}_c \rangle   = \Big \{ \big[\bar c^d(x) \gamma_\mu d^d(x)\big]
 \big[\varepsilon^{abc}u^{a^T}(x)C\gamma_5 d^b(x) c^c(x)\big]\Big \}\, ,  \nonumber\\
%%%%%%%%%%%%%%%%%%%%%%%%%%%%%%%%%%%%%%%%%%%%%%%%%%%%%%%%%
J_\mu^{4}(x)&=\mid  \bar D_s^{*-} \Xi^{0}_c \rangle   = \Big \{ \big[\bar c^d(x) \gamma_\mu s^d(x)\big]
 \big[\varepsilon^{abc}d^{a^T}(x)C\gamma_5 s^b(x) c^c(x)\big]\Big \}\, ,  \nonumber\\
%%%%%%%%%%%%%%%%%%%%%%%%%%%%%%%%%%%%%%%%%%%%%%%%%%%%%%%%%
J_\mu^{5}(x)&=\mid \bar D_s^{*-} \Lambda^{+}_c \rangle   = \Big \{ \big[\bar c^d(x) \gamma_\mu s^d(x)\big]
 \big[\varepsilon^{abc}u^{a^T}(x)C\gamma_5 d^b(x) c^c(x)\big]\Big \}\, ,  
%%%%%%%%%%%%%%%%%%%%%%%%%%%%%%%%%%%%%%%%%%%%%%%%%%%%%%%%%
 \end{align}

In this part of our analysis, our goal is to get the correlation function with the quark-gluon parameters and photon DAs. Repeating the steps we used in the analysis of spin-1/2 pentaquarks, we get the following results:
%
\begin{align}
\label{QCD6}
\Pi_{\mu\nu-1}^{QCD}(p,q)&= \frac{i}{2}\varepsilon^{abc} \varepsilon^{a^{\prime}b^{\prime}c^{\prime}}\, \int d^4x \, e^{ip\cdot x} \langle 0\mid \Big\{ 
% 
\, \mbox{Tr}\Big[\gamma_\mu S_{u}^{dd^\prime}(x) \gamma_\nu  S_{c}^{d^\prime d}(-x)\Big]  
\mbox{Tr}\Big[\gamma_\alpha S_s^{bb^\prime}(x) \gamma_\beta \widetilde S_{d}^{aa^\prime}(x)\Big]
\nonumber\\
&
%     
- \mbox{Tr}\Big[\gamma_\mu S_{u}^{da^\prime}(x) \gamma_\beta \widetilde S_s^{bb^\prime}(x) \gamma_\alpha S_{d}^{ad^\prime}(x)   \gamma_\nu S_{c}^{d^\prime d}(-x)\Big] 
\nonumber\\ 
&
 - \mbox{Tr}\Big[\gamma_\mu S_{d}^{da^\prime}(x) \gamma_\beta  \widetilde S_s^{bb^\prime}(x) \gamma_\alpha  S_{u}^{ad^\prime}(x) \gamma_\nu S_{c}^{d^\prime d}(-x)\Big]
     \nonumber\\ 
&   +
 \mbox{Tr}\Big[\gamma_\mu     S_{d}^{dd^\prime}(x) \gamma_\nu S_{c}^{d^\prime d}(-x)\Big]  
\mbox{Tr}\Big[\gamma_\alpha S_s^{bb^\prime}(x) \gamma_\beta    \widetilde S_{u}^{aa^\prime}(x)\Big] \Big\} 
    \Big( \gamma^\alpha \gamma_5 S_c^{cc^{\prime}}(x)\gamma_5 \gamma^\beta \Big)
\mid 0 \rangle _\gamma \, , 
\end{align}

\begin{align}
\label{QCD7}
\Pi_{\mu\nu-2}^{QCD}(p,q)&=-i\varepsilon^{abc} \varepsilon^{a^{\prime}b^{\prime}c^{\prime}}\, \int d^4x \, e^{ip\cdot x} \langle 0\mid \Big\{ 
% 
\, \mbox{Tr}\Big[\gamma_\mu S_{u}^{dd^\prime}(x) \gamma_\nu  S_{c}^{d^\prime d}(-x)\Big]  
\mbox{Tr}\Big[\gamma_5 S_d^{bb^\prime}(x) \gamma_5 \widetilde S_{u}^{aa^\prime}(x)\Big]
\nonumber\\
&
- \mbox{Tr}\Big[\gamma_\mu S_{u}^{da^\prime}(x) \gamma_5  \widetilde S_d^{bb^\prime}(x) \gamma_5 S_{u}^{ad^\prime}(x)   \gamma_\nu S_{c}^{d^\prime d}(-x)\Big] 
 \Big\} \Big( S_c^{cc^{\prime}}(x) \Big)
\mid 0 \rangle _\gamma \, , 
\end{align}
\begin{align}
\label{QCD8}
\Pi_{\mu\nu-3}^{QCD}(p,q)&=-i \varepsilon^{abc} \varepsilon^{a^{\prime}b^{\prime}c^{\prime}}\, \int d^4x \, e^{ip\cdot x} \langle 0\mid \Big\{ 
% 
\, \mbox{Tr}\Big[\gamma_\mu S_{d}^{dd^\prime}(x) \gamma_\nu  S_{c}^{d^\prime d}(-x)\Big]  
\mbox{Tr}\Big[\gamma_5 S_d^{bb^\prime}(x) \gamma_5 \widetilde S_{u}^{aa^\prime}(x)\Big]
\nonumber\\
&
- \mbox{Tr}\Big[\gamma_\mu S_{d}^{db^\prime}(x) \gamma_5  \widetilde S_u^{aa^\prime}(x) \gamma_5 S_{d}^{bd^\prime}(x)   \gamma_\nu S_{c}^{d^\prime d}(-x)\Big] 
 \Big\} \Big( S_c^{cc^{\prime}}(x) \Big)
\mid 0 \rangle _\gamma \, , 
\end{align}
\begin{align}
\label{QCD9}
\Pi_{\mu\nu-4}^{QCD}(p,q)&= -i\varepsilon^{abc} \varepsilon^{a^{\prime}b^{\prime}c^{\prime}}\, \int d^4x \, e^{ip\cdot x} \langle 0\mid \Big\{ 
% 
\, \mbox{Tr}\Big[\gamma_\mu S_{s}^{dd^\prime}(x) \gamma_\nu  S_{c}^{d^\prime d}(-x)\Big]  
\mbox{Tr}\Big[\gamma_5 S_s^{bb^\prime}(x) \gamma_5 \widetilde S_{d}^{aa^\prime}(x)\Big]
\nonumber\\
&
- \mbox{Tr}\Big[\gamma_\mu S_{s}^{db^\prime}(x) \gamma_5  \widetilde S_d^{aa^\prime}(x) \gamma_5 S_{s}^{bd^\prime}(x)   \gamma_\nu S_{c}^{d^\prime d}(-x)\Big] 
 \Big\} \Big( S_c^{cc^{\prime}}(x) \Big)
\mid 0 \rangle _\gamma \, , 
\end{align}
\begin{align}
\label{QCD10}
\Pi_{\mu\nu-5}^{QCD}(p,q)&=-i \varepsilon^{abc} \varepsilon^{a^{\prime}b^{\prime}c^{\prime}}\, \int d^4x \, e^{ip\cdot x} \langle 0\mid \Big\{ 
% 
\, \mbox{Tr}\Big[\gamma_\mu S_{s}^{dd^\prime}(x) \gamma_\nu  S_{c}^{d^\prime d}(-x)\Big]  
\mbox{Tr}\Big[\gamma_5 S_d^{bb^\prime}(x) \gamma_5 \widetilde S_{u}^{aa^\prime}(x)\Big]
 \Big\} \Big( S_c^{cc^{\prime}}(x) \Big)
\mid 0 \rangle _\gamma \,.
\end{align}

As a result of these procedures, the analysis of the QCD side of the correlation function for spin-3/2 pentaquarks is completed.


Our next task is to obtain the relevant correlation function based on hadronic parameters.  It is written as follows,
\begin{align}\label{Pc103}
\Pi^{Had}_{\mu\nu}(p,q)&=\frac{\langle0\mid  J_{\mu}^{P_{c \bar c}^{3/2}}(x)\mid
{P_{c \bar c}^{3/2}}(p,s)\rangle}{[p^{2}-m_{{P_{c \bar c}^{3/2}}}^{2}]}
 \langle {P_{c \bar c}^{3/2}}(p,s)\mid
{P_{c \bar c}^{3/2}}(p+q,s)\rangle_\gamma 
\frac{\langle {P_{c \bar c}^{3/2}}(p+q,s)\mid
\bar{J}_{\nu}^{P_{c \bar c}^{3/2}}(0)\mid 0\rangle}{[(p+q)^{2}-m_{{P_{c \bar c}^{3/2}}}^{2}]}+...
\end{align}
In the above equation, the matrix elements of the interpolating current between  the $P_{c \bar c}^{3/2}$ pentaquark state and the vacuum  are given as:
\begin{align}\label{lambdabey}
\langle0\mid J_{\mu}^{P_{c \bar c}^{3/2}}(x)\mid {P_{c \bar c}^{3/2}}(p,s)\rangle&=\lambda_{{P_{c \bar c}^{3/2}}}u_{\mu}(p,s),\nonumber\\
%%%%%%%%%%%%%%%%%%%%%%%%%%%%%%%%%%
\langle {P_{c \bar c}^{3/2}}(p+q,s)\mid
\bar{J}_{\nu}^{P_{c \bar c}^{3/2}}(0)\mid 0\rangle &= \lambda_{{P_{c \bar c}^{3/2}}}\bar u_{\nu}(p+q,s), 
\end{align}
where the $\lambda_{{P_{c \bar c}^{3/2}}}$ and $u_{\mu}(p,s)$ ($u_{\nu}(p+q,s)$)  are the residue and   spinors of the $P_{c \bar c}^{3/2}$ pentaquark states, respectively. To further simplify, the summation over the spin of the the Rarita-Schwinger spinor for the spin-3/2 pentaquarks are given:
%
\begin{align}\label{raritabela}
\sum_{s}u_{\mu}(p,s)\bar u_{\nu}(p,s)&=-\Big(\pslash+m_{P_{c \bar c}^{3/2}}\Big)\Big[g_{\mu\nu}
-\frac{1}{3}\gamma_{\mu}\gamma_{\nu}
 -\frac{2\,p_{\mu}p_{\nu}}
{3\,m^{2}_{{P_{c \bar c}^{3/2}}}}+\frac{p_{\mu}\gamma_{\nu}-p_{\nu}\gamma_{\mu}}{3\,m_{{P_{c \bar c}^{3/2}}}}\Big].
\end{align} 

The transition matrix element $\langle
{P_{c \bar c}^{3/2}}(p)\mid {P_{c \bar c}^{3/2}}(p+q)\rangle_\gamma$ entering Eq.
(\ref{Pc103}) can be introduced as 
\cite{Weber:1978dh,Nozawa:1990gt,Pascalutsa:2006up,Ramalho:2009vc}:
\begin{align}\label{matelpar}
\langle {P_{c \bar c}^{3/2}}(p,s)\mid {P_{c \bar c}^{3/2}}(p+q,s)\rangle_\gamma &=-e\bar
u_{\mu}(p,s)\Bigg[F_{1}(q^2)g_{\mu\nu}\eslash 
-
\frac{1}{2m_{{P_{c \bar c}^{3/2}}}} 
\Big[F_{2}(q^2)g_{\mu\nu} 
+F_{4}(q^2)\frac{q_{\mu}q_{\nu}}{(2m_{{P_{c \bar c}^{3/2}}})^2}\Big]\eslash\qslash
\nonumber\\
&+
F_{3}(q^2)\frac{1}{(2m_{{P_{c \bar c}^{3/2}}})^2}q_{\mu}q_{\nu}\eslash \Bigg]
 u_{\nu}(p+q,s).%\nonumber\\
\end{align}
where $F_i$'s are the Lorentz invariant form factors.  Inserting Eqs. (\ref{lambdabey})-(\ref{matelpar}) into Eq. (\ref{Pc103}) for hadronic side we get 
%\begin{widetext}
 %
\begin{align}\label{fizson}
 \Pi^{Had}_{\mu\nu}(p,q)&=-\frac{\lambda_{_{P_{c \bar c}^{3/2}}}^{2}\,\Big(\pslash+m_{P_{c \bar c}^{3/2}}\Big)}{[(p+q)^{2}-m_{_{P_{c \bar c}^{3/2}}}^{2}][p^{2}-m_{_{P_{c \bar c}^{3/2}}}^{2}]}
 \Bigg[g_{\mu\nu}
-\frac{1}{3}\gamma_{\mu}\gamma_{\nu}-\frac{2\,p_{\mu}p_{\nu}}
{3\,m^{2}_{P_{c \bar c}^{3/2}}}+\frac{p_{\mu}\gamma_{\nu}-p_{\nu}\gamma_{\mu}}{3\,m_{P_{c \bar c}^{3/2}}}\Bigg]\nonumber\\
&\times \Bigg\{F_{1}(q^2)g_{\mu\nu}\eslash -
\frac{1}{2m_{P_{c \bar c}^{3/2}}}
\Big[F_{2}(q^2)g_{\mu\nu}+F_{4}(q^2) \frac{q_{\mu}q_{\nu}}{(2m_{P_{c \bar c}^{3/2}})^2}\Big]\eslash\qslash+\frac{F_{3}(q^2)}{(2m_{P_{c \bar c}^{3/2}})^2}
 q_{\mu}q_{\nu}\eslash\Bigg\}.
\end{align}

The above correlation function contains numerous Lorentz structures, not all of which are independent and this correlation function also includes spin-1/2 contributions. In order for our calculations to be more reliable, we need to get rid of these two problems. We can do this by choosing a specific  ordering for gamma matrices such as $\gamma_{\mu}\pslash\eslash\qslash\gamma_{\nu}$ 
 and abolish terms  with $\gamma_\mu$ at the beginning, $\gamma_\nu$ at the end and those proportional to $p_\mu$ and  $p_\nu$~\cite{Belyaev:1982cd}. 
After these algebraic manipulations, both spin-1/2 contributions are eliminated and all Lorentz structures become independent and the final form of the hadronic side of the correlation function becomes the following form:
%
\begin{align}\label{final phenpart}
\Pi^{Had}_{\mu\nu}(p,q)&=\frac{\lambda_{_{{P_{c \bar c}^{3/2}}}}^{2}}{[(p+q)^{2}-m_{_{{P_{c \bar c}^{3/2}}}}^{2}][p^{2}-m_{_{{P_{c \bar c}^{3/2}}}}^{2}]} 
\Bigg[  g_{\mu\nu}\pslash\eslash\qslash \,F_{1}(q^2) 
%\nonumber\\
%&
-m_{{P_{c \bar c}^{3/2}}}g_{\mu\nu}\eslash\qslash\,F_{2}(q^2)
-
\frac{F_{3}(q^2)}{4m_{{P_{c \bar c}^{3/2}}}}q_{\mu}q_{\nu}\eslash\qslash%\,\nonumber\\
%&
\nonumber\\
&
-
\frac{F_{4}(q^2)}{4m_{{P_{c \bar c}^{3/2}}}^3}(\varepsilon.p)q_{\mu}q_{\nu}\pslash\qslash 
+
\cdots %\mbox{other independent structures}
\Bigg].
\end{align}



The magnetic dipole form factor, $G_{M}(q^2)$, of spin-3/2 pentaquark states can be written in terms of $F_{i}(q^2)$ form factors  as follows:~\cite{Weber:1978dh,Nozawa:1990gt,Pascalutsa:2006up,Ramalho:2009vc}:
\begin{align}
G_{M}(q^2) &= [ F_1(q^2) + F_2(q^2)] ( 1+ \frac{4}{5}
\tau ) -\frac{2}{5} [ F_3(q^2)  
+ 
F_4(q^2)] \tau ( 1 + \tau ), %\nonumber\\
\end{align}
  where $\tau
= -\frac{q^2}{4m^2_{{P_{c \bar c}^{3/2}}}}$. At the static limit, i.e. $q^2=0$, the magnetic dipole moment is achieved in connection with the form factors as:
\begin{eqnarray}\label{mqo1}
G_{M}(q^2=0)&=&F_{1}(q^2=0)+F_{2}(q^2=0).
\end{eqnarray}
The  magnetic dipole moment of the spin-3/2 pentaquark state, ($\mu_{{P_{c \bar c}^{3/2}}}$), is defined  in the following way:
 \begin{eqnarray}\label{mqo2}
\mu_{{P_{c \bar c}^{3/2}}}&=&\frac{e}{2m_{{P_{c \bar c}^{3/2}}}}\big[F_{1}(q^2=0)+F_{2}(q^2=0)\big].
\end{eqnarray}





We have obtained the analytical expressions for both $P_{c\bar c}^{1/2}$ and $P_{c \bar c}^{3/2}$ pentaquark states. The next step in the calculations will be to perform numerical calculations of the analytical expressions obtained for the pentaquark states under investigation.
%


\end{widetext}





\section{Numerical Results}\label{numerical}


The  QCD light-cone sum rules for electromagnetic properties of pentaquark states without strange, with strange and with double strange contains many input parameters that we need their numerical values.  Their numerical values are given as: $m_u=m_d=0$, $m_s =93.4^{+8.6}_{-3.4}\,\mbox{MeV}$, $m_c = 1.27 \pm 0.02\,\mbox{GeV}$~\cite{Workman:2022ynf},    $\langle \bar ss\rangle = 0.8\, \langle \bar uu\rangle$ $\,\mbox{GeV}^3$ with $\langle \bar uu\rangle =  \langle \bar dd\rangle=(-0.24 \pm 0.01)^3\,\mbox{GeV}^3$ \cite{Ioffe:2005ym}, $m_0^{2} = 0.8 \pm 0.1 \,\mbox{GeV}^2$ \cite{Ioffe:2005ym},  
%
$\langle g_s^2G^2\rangle = 0.88~ \mbox{GeV}^4$~\cite{Matheus:2006xi}, $f_{3\gamma}=-0.0039~\mbox{GeV}^2$~\cite{Ball:2002ps} and $\chi=-2.85 \pm 0.5~\mbox{GeV}^{-2}$~\cite{Rohrwild:2007yt}. 
%
To get a numerical values for the electromagnetic multipole moments, we need to define the numerical values of the mass and residue of the pentaquarks. The masses and residues of the  these pentaquarks are borrowed from in Refs.~\cite{Wang:2022neq, Wang:2022gfb}.
Photon DAs and their input parameters required for further analysis are taken from Ref.~\cite{Ball:2002ps}.
%

Predictions for electromagnetic properties extracted from the sum rules depend also on the Borel and continuum subtraction parameters $M^2$ and $s_0$. There should be a working interval where the results obtained will not vary much according to these extra parameters. The choice of working intervals for these extra parameters have to fulfill standard  restrictions imposed on the pole contribution (PC) and convergence of the operator product expansion (OPE). It is convenient to use the following equations to describe these restrictions:
%
\begin{align}
 \mbox{PC} =\frac{\Delta (M^2,s_0)}{\Delta (M^2,\infty)},
 \end{align}
and
\begin{align}
 \mbox{OPE Convergence} =\frac{\Delta^{Dim N} (M^2,s_0)}{\Delta (M^2,s_0)},
 \end{align}
 where $\Delta^{Dim N} (M^2,s_0)=\Delta^{Dim (8+9+10)} (M^2,s_0)$.   
 In the standard analysis of QCD light cone sum rules, pole contribution is expected to be over the $50\%$ for traditional hadrons. However, in multiquark states, this contribution is around $PC\geq 20\%$, which is enough for the reliable analysis.  To be convinced in convergence of the OPE, we expect that these contributions should be less then $5\%$ of total calculations. The working regions obtained for $M^2$ and $s_0$ as a result of these restrictions are given in Table \ref{parameter} together with PC and convergence of OPE values for each state.  It follows from these values that the determined working regions for $M^2$ and $s_0$ meet the constraints coming from the dominance of PC and convergence of the OPE.  
% 
 In Figs. \ref{Msqfig1} and \ref{Msqfig2}, we also illustrate the dependence of the magnetic dipole  moment of  pentaquarks, on the Borel mass parameter, $M^2$ at various values of $s_0$. It follows from these figures, the variation of the respective magnetic dipole moments with respect to $M^2$ is observed to be quite stable. When the variation of $s_0$ is examined, it is observed that the variation of the results according to this parameter is high, however this variation remains within the error limits of the of the method used.
 
\begin{widetext}
 
 \begin{table}[htp]
	\addtolength{\tabcolsep}{10pt}
	\caption{Working intervals of  $s_0$ and  $M^2$ as well as  the PC  and OPE convergence for the magnetic dipole moments of pentaquark states without strange, with strange and with double strange.}
	\label{parameter}
		\begin{center}
		%\scalebox{1.0}{
\begin{tabular}{l|ccccc}
                \hline\hline
                \\
State & $s_0$ (GeV$^2$)& 
$M^2$ (GeV$^2$) & ~~  PC ($\%$) ~~ & ~~  OPE  
 ($\%$) \\
\\
                                        \hline\hline
 %                                       \\
$\bar D \Xi_c^{\prime}$  & $23.5-25.5$ & $4.5-6.5$ & $30-55$ &  $2.8$  
 %                       \\
                        \\
$\bar D \Lambda_c$ & $23.6-25.6$ & $4.5-6.5$ & $31-56$ &  $2.9$  \\
  %                      \\
$\bar D_s \Xi_c$  & $24.4-26.4$ & $4.6-6.6$ & $30-57$ & $ 2.8$   \\
%\\
$\bar D_s \Lambda_c$  & $23.8-25.8$ & $4.6-6.6$ & $32-57$ &  $2.5$   \\
%                         \\
                                        \hline\hline
%                                        \\
$\bar D^* \Xi_c^{\prime}$ & $24.7-26.7$ & $4.6-6.6$ & $31-54$ &  $2.8$  \\
%                        \\
$\bar D^* \Lambda_c$ & $24.9-26.9$ & $4.6-6.6$ & $31-58$ &  $2.9$   \\
 %                       \\
$\bar D_s^* \Xi_c$   & $25.5-27.5$ &$4.7-6.7$ & $30-56$ & $2.7$   \\
%\\
$\bar D_s^* \Lambda_c$ & $24.0-26.0$ & $4.5-6.5$ & $31-57$ &  $2.7$   \\
 %                       \\
                                       \hline\hline
 \end{tabular}
%}
\end{center}
\end{table}

\end{widetext}

Employing all the input parameters we give the final results for the magnetic and higher multipole moments of pentaquark states in Table \ref{sonuc}. The presented uncertainties  in the results are  originated from the errors in the values of all  the input parameters as well as those errors coming from the calculations of the working regions for the extra parameters $M^2$ and $s_0$.

 In Ref.~\cite{Wang:2022tib}, the authors have studied the magnetic dipole moments of the $\bar D \Xi^{\prime}_c$ and $\bar D^{*}\Xi^{\prime}_c$ states in the framework of the constituent quark model and they predicted magnetic dipole moments as $-0.277$ and $-0.184$ for the $\bar D \Xi^{\prime}_c$ and $\bar D^{*}\Xi^{\prime}_c$ states, respectively. 
 When the quark model results are compared with the values obtained in this study, it is seen that the results are not compatible with each other.
 %
In Ref.~\cite{Ozdem:2022iqk}, we obtained magnetic dipole moments of the hidden-charm pentaquark states with quantum number $J^P =1/2^-$ within the QCD light-cone sum rules. In the calculations, we employed the axialvector-diquark-axialvector-diquark-antiquark ($P_{c \bar c}^{11}$) and axialvector-diquark-scalar-diquark-antiquark ($P_{c \bar c}^{10}$) forms of interpolating currents to obtain the magnetic dipole moments of hidden-charm  pentaquarks without strange, with strange and with double strange. The results obtained are given as follows: $\mu_{P_{c \bar cuds}^{11}}= 0.43^{+0.14}_{-0.16}~\mu_N$, $\mu_{P_{c \bar cuud}^{11}}= 0.60^{+0.25}_{-0.22}~\mu_N$, $\mu_{P_{c \bar cudd}^{11}}= 0.62^{+0.26}_{-0.22}~\mu_N$, $\mu_{P_{c \bar cdss}^{11}}= 0.70^{+0.25}_{-0.21}~\mu_N$,  
$\mu_{P_{c \bar cuds}^{10}}= 1.39^{+0.47}_{-0.44}~\mu_N$, $\mu_{P_{c \bar cuud}^{10}}= 1.42^{+0.57}_{-0.50}~\mu_N$, $\mu_{P_{c \bar cudd}^{10}}= 1.51^{+0.54}_{-0.49}~\mu_N$, $\mu_{P_{c \bar cdss}^{10}}= -1.92^{+0.65}_{-0.58}~\mu_N$.
%
It is seen that the results obtained for $\bar D^{0}\Lambda_c^+$, $\bar D^{-}\Lambda_c^+$, $\bar D_s^{-}\Xi_c^0$ and $\bar D_s^{-}\Lambda_c^+$ states are consistent with the axialvector-diquark-axialvector-diquark-antiquark configurations within errors. However, it can easily be seen that there is a large discrepancy with the results obtained using the axialvector-diquark-scalar-diquark-antiquark configuration. Using different models and configurations leads to different predictions. Therefore,  more theoretical investigation and experimental measurements are required to define the properties of these hidden-charmed pentaquark states.
 %
 Comparing the magnetic dipole moment results obtained using different theoretical models with the values obtained in this study may give an idea about the consistency of our estimations.
 
 \begin{widetext}
  
%
  \begin{table}[htp]
	\addtolength{\tabcolsep}{10pt}
	\caption{The magnetic dipole moments of hidden-charm pentaquarks without strange, with strange and with double strange obtained by performing the QCD light-cone sum rules. For completeness, we have also presented higher multipole moments, electric quadrupole ($\mathcal{Q}$) and magnetic octupole ($\mathcal{O}$), of the $\bar D^{*}\Xi^{\prime}_c$, $\bar D^{*}\Lambda_c$,  $\bar D_s^{*}\Lambda_c$ and  $\bar D_s^{*}\Xi_c$ pentaquark states.}
	\label{sonuc}
		\begin{center}
		%\scalebox{1.0}{
\begin{tabular}{l|ccccc}
                \hline\hline
   %             \\
 Parameters& $\bar D \Xi_c^{\prime}$& 
$\bar D^0 \Lambda_c^+$ &$\bar D^- \Lambda_c^+$  & $\bar D_s^- \Xi_c^0$ & $\bar D_s^- \Lambda_c$ \\
    %\\
                                        \hline\hline
                                      \\

 $\mu (\mu_N)$& $-0.10^{+0.03}_{-0.03}$ & $0.44^{+0.17}_{-0.14}$ & $0.45^{+0.17}_{-0.15}$ &  $0.50^{+0.18}_{-0.16}$  & $0.43^{+0.17}_{-0.14}$                        \\
                         \\
                                        \hline\hline
 %     \\  
       Parameters & $\bar D^* \Xi_c^{\prime}$&$\bar D^{*0} \Lambda_c^+$ &$\bar D^{*-} \Lambda_c^+$  & $\bar D_s^{*-} \Xi_c^0$ & $\bar D_s^{*-} \Lambda_c$ \\
%      \\
 \hline \hline
 \\
$\mu (\mu_N)$ & $2.60^{+0.88}_{-0.77}$ & $2.24^{+0.77}_{-0.64}$ & $2.22^{+0.76}_{-0.64}$ &  $2.73^{+0.90}_{-0.81}$  & $1.87^{+0.70}_{-0.62}$  \\
                        \\
$\mathcal{Q}(\times 10^{-1}) (fm^2)$ & $1.60^{+0.40}_{-0.40}$ & $2.61^{+0.60}_{-0.60}$ & $2.50^{+0.60}_{-0.60}$ &  $3.11^{+0.70}_{-0.70}$  & $0.24^{+0.06}_{-0.06}$   \\
                        \\
$\mathcal{O}(\times 10^{-3})(fm^3)$   & $0.32^{+0.05}_{-0.05}$ & $0.11^{+0.04}_{-0.04}$ & $0.10^{+0.04}_{-0.04}$ &  $0.14^{+0.04}_{-0.04}$  & $0.09^{+0.03}_{-0.03}$   \\
                        \\
                                       \hline\hline
 \end{tabular}
%}
\end{center}
\end{table}

 \end{widetext}
 
\section{summary and concluding remarks}\label{summary}
  
Since the discovery of the hidden-charmed pentaquark states $P_c(4380)$ and $P_c(4450)$ by the LHCb Collaboration in 2015, the study and elucidation of pentaquarks, along with other pentaquark candidates discovered, has become an attractive subject in hadron physics. Inspired by this, we have studied the electromagnetic properties of $\bar D^{(*)}\Xi^{\prime}_c$, $\bar D^{(*)}\Lambda_c$,  $\bar D_s^{(*)}\Lambda_c$ and  $\bar D_s^{(*)}\Xi_c$  pentaquarks without strange, with strange and with double strange through QCD light-cone sum rules.  We have also evaluated electric quadrupole and magnetic octupole moments of the $\bar D^{*}\Xi^{\prime}_c$, $\bar D^{*}\Lambda_c$,  $\bar D_s^{*}\Lambda_c$ and  $\bar D_s^{*}\Xi_c$ pentaquark states. 
%
A comparison of our results on magnetic dipole moments of the hidden-charmed pentaquark states with the other models existing in literature is presented.
%

The magnetic dipole moments of the hidden-charm pentaquark states reveal valuable knowledge about the size and shape of the hadrons. Obtaining these parameters can be an prominent step in our understanding of hadron properties according to quark-gluon degrees of freedom. Moreover, the magnetic dipole moment is a key ingredient in the calculation of $J/\psi$ photo-production procedure, which may provide an independent examination of the pentaquark states. It will also be crucial to identify the branching ratios of the different decay modes and decay channels of the hidden-charmed pentaquark states.
With the accumulation of events, magnetic dipole moments of pentaquarks  may be extracted from the comparison of theoretical and experimental cross sections eventually in the near future. %, which may help us understanding of the underlying dynamics governing their formation.
 If the inner structure of exotic states figure out, our comprehension  on the construction of the subatomic world be crucially improved, and our comprehension on the non-perturbative behaviors of the strong interaction at the low energy region would also be crucially improved.

\begin{widetext} 

\begin{figure}[t]
%\centering
\subfloat[]{\includegraphics[width=0.45\textwidth]{J1Msq.pdf}}~~~~
\subfloat[]{\includegraphics[width=0.45\textwidth]{J2Msq.pdf}}\\
\subfloat[]{\includegraphics[width=0.45\textwidth]{J3Msq.pdf}}~~~~
\subfloat[]{\includegraphics[width=0.45\textwidth]{J4Msq.pdf}}\\
\subfloat[]{\includegraphics[width=0.45\textwidth]{J5Msq.pdf}}
 \caption{The magnetic dipole moments of spin-$\frac{1}{2}$ pentaquark states versus $M^2$ at three different values of $s_0$; (a), (b), (c), (d) and (e) denote the $\bar D \Xi^{\prime}_c$, $\bar D^{0}\Lambda_c^+$, $\bar D^{-}\Lambda_c^+$, $\bar D_s^{-}\Xi_c^0$ and $\bar D_s^{-}\Lambda_c^+$    pentaquark states, respectively.}
 \label{Msqfig1}
  \end{figure}
  
%  \end{widetext}
% 
% \begin{widetext}
 
 \begin{figure}[t]
%\centering
\subfloat[]{\includegraphics[width=0.45\textwidth]{J1muMsq.pdf}}~~~~
\subfloat[]{\includegraphics[width=0.45\textwidth]{J2muMsq.pdf}}\\
\subfloat[]{\includegraphics[width=0.45\textwidth]{J3muMsq.pdf}}~~~~
\subfloat[]{\includegraphics[width=0.45\textwidth]{J4muMsq.pdf}}\\
\subfloat[]{\includegraphics[width=0.45\textwidth]{J5muMsq.pdf}}
 \caption{The magnetic dipole moments of spin-$\frac{3}{2}$ pentaquark states versus $M^2$ at three different values of $s_0$; (a), (b), (c), (d) and (e) denote the $\bar D^{*}\Xi^{\prime}_c$, $\bar D^{*0}\Lambda_c^+$, $\bar D^{*-}\Lambda_c^+$, $\bar D_s^{*-}\Xi_c^0$ and $\bar D_s^{*-}\Lambda_c^+$    pentaquark states, respectively.}
 \label{Msqfig2}
  \end{figure}
 
   \end{widetext}









%\section{Acknowledgments}

\begin{widetext}
 


\appendix
  %
  \section*{Appendix: Explicit expression for \texorpdfstring{$\Delta_1^{QCD} (M^2,s_0)$}{}}\label{appenda}
 In this appendix, we present the explicit expressions of the function $\Delta_1^{QCD} (M^2,s_0)$ for the magnetic dipole moment of the $\bar D \Xi^{\prime}_c$ pentaquark state entering into the sum rule.
%
\begin{align}
 \Delta_1^{QCD}(M^2,s_0) &=\frac{m_c P_1 P_2 P_3  }{4076863488 \pi^3} (e_u + e_d)\Bigg\{
-3 (14 m_c m_s I[0, 1, 2, 0] + 19 I[0, 2, 3, 0]) I_ 2[\mathcal S] - 
 27 m_c m_s I_ 2[\mathcal {\tilde S}] I[0, 1, 2, 0] \nonumber\\
 &+ 
 96 m_c m_s I_ 6[h_ {\gamma}] I[0, 1, 2, 0] + 
 24 \Big ((I[0, 2, 1, 0] - 2 I[0, 2, 1, 1] + I[0, 2, 1, 2] - 
       2 I[0, 2, 2, 0] + 2 I[0, 2, 2, 1])
       \nonumber\\
 &\times A[u_ 0] + (A[u_ 0] + 
       I_ 5[A] + \chi m_ 0^2 I_ 5[\varphi_ {\gamma}]) I[0, 2, 3, 
       0]\Big) - 
 8 \chi \big (3 m_c m_s I[0, 2, 3, 0] + 4 I[0, 3, 3, 0] 
 \nonumber\\
 &+ 
    3 I[0, 3, 4, 0]\big) I_ 5[\varphi_ {\gamma}] - 
 8 \chi \Big (3 m_c m_s \big (5 I[0, 2, 1, 0] - 6 I[0, 2, 1, 1] + 
        I[0, 2, 1, 2] - 6 I[0, 2, 2, 0] \nonumber\\
 &+ 2 I[0, 2, 2, 1] + 
        I[0, 2, 3, 0]\big) - 
     3 m_ 0^2 \big (I[0, 2, 1, 0] - 2 I[0, 2, 1, 1] + I[0, 2, 1, 2] - 
        2 I[0, 2, 2, 0] + 2 I[0, 2, 2, 1] \nonumber\\
 &+ I[0, 2, 3, 0]\big) - 
     7 I[0, 3, 1, 0] + 21 I[0, 3, 1, 1] - 21 I[0, 3, 1, 2] + 
     7 I[0, 3, 1, 3] + 17 I[0, 3, 2, 0] \nonumber\\
 &- 34 I[0, 3, 2, 1] + 
     17 I[0, 3, 2, 2] - 13 I[0, 3, 3, 0] + 13 I[0, 3, 3, 1] + 
     3 I[0, 3, 4, 0]\Big) \varphi_ {\gamma}[u_ 0]
 \Bigg\} \nonumber\\
 %%%%%%%%%%%%%%%%%%%%%%%%%%%%%%%%%%%%%%%%%%%%%%%%%%%%%%%%%%%%%%%%%%%%%%
 %%%%%%%%%%%%%%%%%%%%%%%%%%%%%%%%%%%%%%%%%%%%%%%%%%%%%%%%%%%%%%%%%%
 &-\frac { P_ 1 P_ 2^2 } {1019215872\pi^3}\Bigg\{48 m_c e_c \Big (I[0, 2, 1,
         0] - 2 I[0, 2, 1, 1] + I[0, 2, 1, 2] - 2 I[0, 2, 2, 0] + 
       2 I[0, 2, 2, 1]  \nonumber\\
 &- 
       10 \big (I[1, 1, 1, 0] - 2 I[1, 1, 1, 1] + I[1, 1, 1, 2] - 
           2 I[1, 1, 2, 0] + 2 I[1, 1, 2, 1] + I[1, 1, 3, 0]\big)\Big)
           \nonumber\\
 &
    + 96 m_c e_s \Big ( 
       3 I[0, 2, 1, 0] - 6 I[0, 2, 1, 1] + 3 I[0, 2, 1, 2] - 
        6 I[0, 2, 2, 0] + 6 I[0, 2, 2, 1] + 3 I[0, 2, 3, 0] 
        \nonumber\\
 &- 
        10 \big (I[1, 1, 1, 0] - 2 I[1, 1, 1, 1] + I[1, 1, 1, 2] - 
            2 I[1, 1, 2, 0] + 2 I[1, 1, 2, 1] + I[1, 1, 3, 0]\big)\Big)\nonumber\\
            & + (e_d + 
    e_u) m_c  \Bigg (-96 m_c m_s I_ 6[h_ {\gamma}] I[0, 1, 2, 0] + 
    3 I_ 2[\mathcal S] I[0, 2, 3, 0] + 
    12 A[u_ 0] \Big (I[0, 2, 1, 0]\nonumber\\
            & - 2 I[0, 2, 1, 1] + 
       I[0, 2, 1, 2] - 2 I[0, 2, 2, 0] + 2 I[0, 2, 2, 1] + 
       I[0, 2, 3, 0]\Big) + 
    4 \chi \Big (24  m_c m_s (I[0, 2, 1, 0]\nonumber\\
            & - I[0, 2, 1, 1] - 
           I[0, 2, 2, 0]) + 
        3 m_ 0^2 (-2 I[0, 2, 1, 1] + I[0, 2, 1, 2] - 
           2 I[0, 2, 2, 0] + 2 I[0, 2, 2, 1] \nonumber\\
            &+ I[0, 2, 3, 0]) + 
        I[0, 3, 1, 0] - 3 I[0, 3, 1, 1] + 3 I[0, 3, 1, 2] - 
        I[0, 3, 1, 3] + I[0, 3, 2, 0] - 2 I[0, 3, 2, 1]\nonumber\\
            & + 
        I[0, 3, 2, 2] - 5 I[0, 3, 3, 0] + 5 I[0, 3, 3, 1] + 
        3 I[0, 3, 4, 0]\Big) \varphi_ {\gamma}[u_ 0] + 
    4 \Big (3 I_ 5[A] I[0, 2, 3, 
          0] + \chi I_ 5[\varphi_ {\gamma}] \nonumber\\
            & \times (3 m_ 0^2 I[0, 2, 3, 0] - 
           4 I[0, 3, 3, 0] + 3 I[0, 3, 4, 0]) + 
        3 \chi m_ 0^2 I[0, 2, 1, 0] \varphi_ {\gamma}[u_ 0]\Big)\Bigg)
      \Bigg\}\nonumber\\
      %%%%%%%%%%%%%%%%%%%%%%%%%%%%%%%%%%%%%%%%%%%%%%%%%%%%%%%%%%%%%%%%%%%%%%%%%%%%%%%%%%%%%%%%%%%%%%%%%%%%%%%%%%%%%%%%%%%%%%%%%%%%%%%%%%%%%%%%%%%%%%%%%%%%%%%%%%%%%%%%%%%%%%%%%%%%%%%%%%%%%%%%%%%%%%%%
 &     -\frac { P_ 1 P_ 2 } {5435817984 \pi^5}\Bigg\{48 m_c^2 m_0^2 \Bigg (8 \
e_s \Big (-3 I[0, 2, 2, 0] + 3 I[0, 2, 2, 1] + 3 I[0, 2, 3, 0] - 
          8 I[1, 1, 1, 0] 
          \nonumber\\
 &+ 2 I[1, 1, 1, 2] + 
          16 I[1, 1, 2, 0] - 6 I[1, 1, 2, 1] - 8 I[1, 1, 3, 0]\Big) + 
       e_c \Big (9 I[0, 2, 1, 0] - 54 I[0, 2, 1, 1] \nonumber\\
 &+ 
          45 I[0, 2, 1, 2] - 6 I[0, 2, 2, 0] + 42 I[0, 2, 2, 1] - 
          3 I[0, 2, 3, 0] - 
          2 (I[1, 1, 1, 0] + 6 I[1, 1, 1, 1] - 7 I[1, 1, 1, 2]\nonumber\\
 & - 
              2 I[1, 1, 2, 0] - 6 I[1, 1, 2, 1] + 
              I[1, 1, 3, 0])\Big) + 
       2 e_d \Big (I[0, 2, 1, 0] - 2 I[0, 2, 1, 1] + I[0, 2, 1, 2] - 
          8 I[0, 2, 2, 0] \nonumber\\
 &+ 8 I[0, 2, 2, 1] + 7 I[0, 2, 3, 0] - 
          4 (6 I[1, 1, 1, 0] - 11 I[1, 1, 1, 1] + 5 I[1, 1, 1, 2] - 
              12 I[1, 1, 2, 0] \nonumber\\
 &+ 11 I[1, 1, 2, 1] + 
              6 I[1, 1, 3, 0])\Big) + 
       2 e_u (I[0, 2, 1, 0] - 2 I[0, 2, 1, 1] + I[0, 2, 1, 2] - 
           8 I[0, 2, 2, 0] \nonumber\\
 &+ 8 I[0, 2, 2, 1] + 7 I[0, 2, 3, 0] - 
           4 (6 I[1, 1, 1, 0] - 11 I[1, 1, 1, 1] + 5 I[1, 1, 1, 2] - 
               12 I[1, 1, 2, 0] \nonumber\\
 &+ 11 I[1, 1, 2, 1] + 
               6 I[1, 1, 3, 0])\Big)\Bigg) - 
    192 e_c m_c^2 \Big (I[0, 3, 1, 0] + I[0, 3, 1, 1] - 9 I[0, 3, 1, 2] + 
       7 I[0, 3, 1, 3] \nonumber\\
 &- 2 I[0, 3, 2, 0] + 6 I[0, 3, 2, 2] + 
       I[0, 3, 3, 0] - I[0, 3, 3, 1] - I[1, 2, 1, 1] - 
       2 I[1, 2, 1, 2] + 3 I[1, 2, 1, 3]\nonumber\\
 & + 
       2 (I[1, 2, 2, 1] + I[1, 2, 2, 2]) - I[1, 2, 3, 1]\Big) + 
    576 e_s m_c^2 \Big (2 I[0, 3, 1, 0] - 4 I[0, 3, 1, 1] + 3 I[0, 3, 1, 2] \nonumber%\\
  \end{align}
 
 \begin{align}
 &-
        I[0, 3, 1, 3] - 4 I[0, 3, 2, 0] + 6 I[0, 3, 2, 1] - 
       3 I[0, 3, 2, 2] + 2 I[0, 3, 3, 0] - 2 I[0, 3, 3, 1] + 
       8 I[1, 2, 1, 1] \nonumber\\
 &- 11 I[1, 2, 1, 2] + 3 I[1, 2, 1, 3] - 
       16 I[1, 2, 2, 1] + 11 I[1, 2, 2, 2] + 8 I[1, 2, 3, 1]\Big) + 
    144 e_d m_c^2 \Big (7 I[0, 3, 1, 0] \nonumber\\
 &- 15 I[0, 3, 1, 1] + 
       12 I[0, 3, 1, 2] - 4 I[0, 3, 1, 3] - 14 I[0, 3, 2, 0] + 
       22 I[0, 3, 2, 1] - 11 I[0, 3, 2, 2] \nonumber\\
  &+ 7 I[0, 3, 3, 0] - 
       7 I[0, 3, 3, 1] + 36 I[1, 2, 1, 1] - 69 I[1, 2, 1, 2] + 
       33 I[1, 2, 1, 3] - 72 I[1, 2, 2, 1]\nonumber\\
 & + 69 I[1, 2, 2, 2] + 
       36 I[1, 2, 3, 1]\Big) + 
    144 e_u m_c^2 \Big (7 I[0, 3, 1, 0] - 15 I[0, 3, 1, 1] + 
        12 I[0, 3, 1, 2] - 4 I[0, 3, 1, 3] \nonumber\\
 & - 14 I[0, 3, 2, 0] + 
        22 I[0, 3, 2, 1] - 11 I[0, 3, 2, 2] + 7 I[0, 3, 3, 0] - 
        7 I[0, 3, 3, 1] + 36 I[1, 2, 1, 1] \nonumber\\
 &- 69 I[1, 2, 1, 2] + 
        33 I[1, 2, 1, 3] - 72 I[1, 2, 2, 1] + 69 I[1, 2, 2, 2] + 
        36 I[1, 2, 3, 1]\Big)
        \nonumber\\
        &
        +4 m_c \Bigg (3 \chi m_s (-e_u I[0, 4, 4, 0] + (e_d + e_u) I[0, 4, 4, 
         1]) \varphi_ {\gamma}[u_ 0] - 
   16 f_ {3\gamma} \pi^2\Bigg (2 e_s m_c \Big (2 m_ 0^2 (I[0, 1, 1, 
             0] - I[0, 1, 1, 1] \nonumber\\
        &- I[0, 1, 2, 0]) + 11 I[0, 2, 1, 0] - 
         22 I[0, 2, 1, 1] + 11 I[0, 2, 1, 2] - 18 I[0, 2, 2, 0] + 
         18 I[0, 2, 2, 1] + 7 I[0, 2, 3, 0]\Big) 
         \nonumber\\
        &+ 
      e_d \Big (2 m_ 0^2 m_c (I[0, 1, 1, 0] - I[0, 1, 1, 1] - 
            I[0, 1, 2, 0]) + 
         2  m_s \big (I[0, 2, 1, 0] - 2 I[0, 2, 1, 1] + 
            I[0, 2, 1, 2] - 2 I[0, 2, 2, 0] 
            \nonumber\\
        &+ 2 I[0, 2, 2, 1] + 
            I[0, 2, 3, 0]\big) - 
         m_c\big (3 I[0, 2, 1, 0] - 6 I[0, 2, 1, 1] + 
             3 I[0, 2, 1, 2] - 10 I[0, 2, 2, 0] + 10 I[0, 2, 2, 1] 
             \nonumber\\
        &+ 
             7 I[0, 2, 3, 0]\big)\Big) + 
      e_u \Big (2 m_ 0^2 m_c (I[0, 1, 1, 0] - I[0, 1, 1, 1] - 
             I[0, 1, 2, 0]) + 
          2  m_s \big (I[0, 2, 1, 0] - 2 I[0, 2, 1, 1] + 
             I[0, 2, 1, 2] 
             \nonumber\\
        &- 2 I[0, 2, 2, 0] + 2 I[0, 2, 2, 1] + 
             I[0, 2, 3, 0]\big) - 
          m_c\big (3 I[0, 2, 1, 0] - 6 I[0, 2, 1, 1] + 
              3 I[0, 2, 1, 2] - 10 I[0, 2, 2, 0]  
              \nonumber\\
        &+ 10 I[0, 2, 2, 1]+ 
              7 I[0, 2, 3, 0]\big)\Big)\Bigg) \psi^a[u_ 0]\Bigg)
              +
              4 \chi m_c (e_d + 
    e_u) \Bigg (4 m_c \Big (-I[0, 4, 1, 2] + I[0, 4, 1, 3] + 
       I[0, 4, 2, 2]\Big) \nonumber\\
        &+ 
    m_s \Big (3 I[0, 4, 1, 0] - 8 I[0, 4, 1, 1] + 8 I[0, 4, 1, 2] - 
        4 I[0, 4, 1, 3] + I[0, 4, 1, 4] - 9 I[0, 4, 2, 0] + 
        19 I[0, 4, 2, 1] 
        \nonumber\\
        &- 15 I[0, 4, 2, 2] + 5 I[0, 4, 2, 3] + 
        9 I[0, 4, 3, 0] - 14 I[0, 4, 3, 1] + 7 I[0, 4, 3, 2] - 
        3 I[0, 4, 4, 0] + 
        3 I[0, 4, 4, 1]\Big)\Bigg) \varphi_ {\gamma}[u_ 0]\nonumber\\
        &
        +64 f_ {3\gamma} m_c \Bigg ((-5 e_s - 2 e_u) m_c I_ 2[\mathcal V] I[0,
      2, 2, 0] + \Big (2 (e_d + 2 e_s + 
         e_u) m_c (m_ 0^2 I[0, 1, 2, 0] - 
         2 I[0, 2, 2, 0]) \nonumber\\
        &+ (7 (e_d - 2 e_s + e_u) m_c + 
         2 (e_d + e_u) m_s) I[0, 2, 3, 0]\Big) I_5[\psi^a] \Bigg)
      \Bigg\}\nonumber\\
              %%%%%%%%%%%%%%%%%%%%%%%%%%%%%%%%%%%%%%%%%%%%%%%%%%%%%%%%%%%%%%%%%%%%%%%%%%%%%%%%%%%%%%%%%%%%%%%%%%%%%%%%%%%%%%%%%%%%%%%%%%%%%%%%%%%%%%%%%%%%%%%%%%%
              &
              + \frac {P_ 1 P_ 3} {16307453952  \pi^5}\Bigg\{-96 m_0^2\Bigg(3  (3 e_d - e_c + 3 e_u) m_c^2 (I[0, 2, 1, 0] - 2 I[0, 2, 1, 1] + 
          I[0, 2, 1, 2] - 2 I[0, 2, 2, 0] 
          \nonumber\\
        &+ 2 I[0, 2, 2, 1] + 
          I[0, 2, 3, 0] - 2 I[1, 1, 1, 0] + 4 I[1, 1, 1, 1] - 
          2 I[1, 1, 1, 2] + 4 I[1, 1, 2, 0] - 4 I[1, 1, 2, 1] - 
          2 I[1, 1, 3, 0])
          \nonumber\\
        &
       + m_c m_s \Big (e_u (I[0, 2, 1, 0] - 2 I[0, 2, 1, 1] + 
              I[0, 2, 1, 2] - 2 I[0, 2, 2, 0] + 2 I[0, 2, 2, 1] + 
              2 I[1, 1, 1, 0] - 4 I[1, 1, 1, 1] 
              \nonumber\\
        &+ 2 I[1, 1, 1, 2] - 
              4 I[1, 1, 2, 0] + 4 I[1, 1, 2, 1] + 2 I[1, 1, 3, 0]) + 
           e_d (I[0, 2, 1, 0] - 2 I[0, 2, 1, 1] + I[0, 2, 1, 2] - 
              2 I[0, 2, 2, 0]
              \nonumber\\
        &+ 2 I[0, 2, 2, 1] + I[0, 2, 3, 0] + 
              2 I[1, 1, 1, 0] - 4 I[1, 1, 1, 1] + 2 I[1, 1, 1, 2] - 
              4 I[1, 1, 2, 0] + 4 I[1, 1, 2, 1] + 2 I[1, 1, 3, 0]) 
              \nonumber\\
        &+ 
           e_c (25 I[0, 2, 1, 0] - 50 I[0, 2, 1, 1] + 
               25 I[0, 2, 1, 2] - 50 I[0, 2, 2, 0] + 
               50 I[0, 2, 2, 1] + 25 I[0, 2, 3, 0] + 
               2 I[1, 1, 1, 0] \nonumber\\
        &- 4 I[1, 1, 1, 1] + 2 I[1, 1, 1, 2] - 
               4 I[1, 1, 2, 0] + 4 I[1, 1, 2, 1] + 
               2 I[1, 1, 3, 0])\Big)\Bigg) + 
    96 m_s (e_d + e_u + 3 e_c)  \Big (I[0, 3, 1, 0] \nonumber\\
        &- 
       3 I[0, 3, 1, 1] + 3 I[0, 3, 1, 2] - I[0, 3, 1, 3] - 
       41 I[0, 3, 2, 0] + 82 I[0, 3, 2, 1] - 41 I[0, 3, 2, 2] + 
       79 I[0, 3, 3, 0] \nonumber\\
        &- 79 I[0, 3, 3, 1] - 
       3 (13 I[0, 3, 4, 0] - 5 I[1, 2, 1, 1] + I[1, 2, 1, 2] + 
           I[1, 2, 1, 3] + 94 I[1, 2, 2, 1] - 43 I[1, 2, 2, 2]\nonumber\\
        & - 
           89 I[1, 2, 3, 1])\Big) - 
    576 m_c (5 e_d + 7 e_u + 3 e_c) \Big (I[0, 3, 1, 0] - 
        3 I[0, 3, 1, 1] + 3 I[0, 3, 1, 2] - I[0, 3, 1, 3] - 
        41 I[0, 3, 2, 0] \nonumber\\
        %\end{align}
        %\begin{align}
        &+ 82 I[0, 3, 2, 1] - 41 I[0, 3, 2, 2] + 
        79 I[0, 3, 3, 0] - 79 I[0, 3, 3, 1] - 
        3 (13 I[0, 3, 4, 0] - 5 I[1, 2, 1, 1] + I[1, 2, 1, 2] 
        \nonumber\\
        &+ 
            I[1, 2, 1, 3] + 94 I[1, 2, 2, 1] - 43 I[1, 2, 2, 2] - 
            89 I[1, 2, 3, 1])\Big)
            -3 f_{3\gamma} m_c \pi^2  \Bigg (256 e_d m_c  (m_ 0^2 I[0, 1, 2, 
         0] - 2 I[0, 2, 2, 
         0])\nonumber\\
        & \times I_6[\psi_{\gamma}^{\nu}] + (184 e_u m_c I[0, 2, 2, 
         0] - 5 e_d m_s I[0, 2, 3, 0] - 
       5  e_u m_s I[0, 2, 3, 0]) I_ 2[\mathcal V] + 
    64 (e_d + 
        e_u) \Big (- 
          m_s (I[0, 2, 1, 0] 
          \nonumber
       \end{align}
\begin{align}
  &- 2 I[0, 2, 1, 1] + I[0, 2, 1, 2] - 
           2 I[0, 2, 2, 0] + 2 I[0, 2, 2, 1] + I[0, 2, 3, 0]) + 
        m_c (5 I[0, 2, 1, 0] \nonumber\\
        &- 10 I[0, 2, 1, 1] + 
            5 I[0, 2, 1, 2] - 6 I[0, 2, 2, 0] + 6 I[0, 2, 2, 1] + 
            I[0, 2, 3, 0])\Big) \psi^a[u_ 0]\Bigg) + 
 12 e_s m_c^2 \Bigg (-(4 I[0, 3, 2, 0] \nonumber\\
        &+ 
        I[0, 3, 3, 0]) I_ 2[\mathcal S] - 
    4 \Bigg (2 I_ 4[\mathcal {\tilde S}] I[0, 3, 2, 0] + 
       2 I_ 5[\mathcal A] I[0, 3, 2, 0] - 
       2 I_ 6[h_{\gamma}] I[0, 3, 2, 0] + 
       I_ 5[\mathcal A] I[0, 3, 3, 0] \nonumber\\
        &+ 
       I_ 4[\mathcal S] (8 I[0, 3, 2, 0] + I[0, 3, 3, 0])\Big) + 
    4  \Big (I[0, 3, 1, 0] - 5 I[0, 3, 1, 1] + 7 I[0, 3, 1, 2] - 
       3 I[0, 3, 1, 3] - 2 I[0, 3, 2, 0] \nonumber\\
        &+ 6 I[0, 3, 2, 1] - 
       4 I[0, 3, 2, 2] + I[0, 3, 3, 0] - I[0, 3, 3, 1]\Big) A[
       u_ 0]\Bigg) - 
 8 \chi e_s m_c^2 \big(6 I[0, 4, 1, 1] - 21 I[0, 4, 1, 2] \nonumber\\
        &+ 
     22 I[0, 4, 1, 3] - 7 I[0, 4, 1, 4] - 12 I[0, 4, 2, 1] + 
     24 I[0, 4, 2, 2] - 10 I[0, 4, 2, 3] + 6 I[0, 4, 3, 1] - 
     3 I[0, 4, 3, 2]\Big) \varphi_{\gamma}[u_ 0]
      \Bigg\}\nonumber\\
   %%%%%%%%%%%%%%%%%%%%%%%%%%%%%%%%%%%%%%%%%%%%%%%%%%%%%%%%%%%%%%%%%%%%%%%%%%%%%%%%%%%%%%%%%%%%%%%%%%%%%%%%%%%%%%%%%%%%%%%%%%%%%%%%%%%%%%%%%%%%%%%%%%%%%%%%%%%%%%%%%%%%%%%%%%%%%%%%%%%%%%%%%%%%%%%%%%%
   &-\frac{m_c P_2 P_3}{18874368 \pi^3}\Bigg\{ 
   \big(4 (5 e_d - e_u) m_c m_s I[0, 3, 3, 0] - 
    21 (e_d - 2 e_s + e_u) I[0, 4, 4, 0]\big) I_ 4[\mathcal S] + 
 4 (e_d + e_u) \Bigg(-3 I[0, 4, 2, 0] 
 \nonumber\\
        &+ 9 I[0, 4, 2, 1] - 
    9 I[0, 4, 2, 2] + 3 I[0, 4, 2, 3] + 9 I[0, 4, 3, 0] - 
    18 I[0, 4, 3, 1] + 9 I[0, 4, 3, 2] - 9 I[0, 4, 4, 0] + 
    9 I[0, 4, 4, 1] 
    \nonumber\\
        &
        + 3 I[0, 4, 5, 0] + 
    28 m_c m_s \Big (I[0, 3, 2, 0] - 2 I[0, 3, 2, 1] + 
       I[0, 3, 2, 2] - 2 I[0, 3, 3, 0] + 2 I[0, 3, 3, 1] + 
       I[0, 3, 4, 0] 
       \nonumber\\
        &+ 6 I[1, 2, 2, 1] - 3 I[1, 2, 2, 2] - 
       6 I[1, 2, 3, 1]\Big) - 
    12 \Big (3 I[1, 3, 2, 1] - 3 I[1, 3, 2, 2] + I[1, 3, 2, 3] + 
        -6 I[1, 3, 3, 1]
        \nonumber\\
        &+ 3I[1, 3, 3, 2] + 
            3I[1, 3, 4, 1]\Big)\Bigg) - 
 16 e_c \Bigg(m_0^2 \Big (5 m_c m_s \big(3 I[0, 2, 1, 0] - 
          6 I[0, 2, 1, 1] + 3 I[0, 2, 1, 2] - 6 I[0, 2, 2, 0] 
          \nonumber\\
        &+ 
          6 I[0, 2, 2, 1] + 3 I[0, 2, 3, 0] + 
          2I[1, 1, 1, 0] - 4 I[1, 1, 1, 1] + 2I[1, 1, 1, 2] - 
              4 I[1, 1, 2, 0] + 4 I[1, 1, 2, 1] + 
             2 I[1, 1, 3, 0]\big) 
             \nonumber\\
        &- 
       2  \big (I[0, 3, 2, 0] - 2 I[0, 3, 2, 1] + I[0, 3, 2, 2] - 
           2 I[0, 3, 3, 0] + 2 I[0, 3, 3, 1] + I[0, 3, 4, 0] + 
           6 I[1, 2, 2, 1] - 3 I[1, 2, 2, 2] 
           \nonumber\\
        &- 
           6 I[1, 2, 3, 1]\big)\Big) + 
    4 m_c m_s \Big(I[0, 3, 1, 0] - 5 I[0, 3, 1, 1] + 
       7 I[0, 3, 1, 2] - 3 I[0, 3, 1, 3] - 2 I[0, 3, 2, 0] + 
       8 I[0, 3, 2, 1] 
       \nonumber\\
        &- 6 I[0, 3, 2, 2] + I[0, 3, 3, 0] - 
       3 \big (I[0, 3, 3, 1] + I[1, 2, 1, 1] - 2 I[1, 2, 1, 2] + 
           I[1, 2, 1, 3] - 2 I[1, 2, 2, 1] + 2 I[1, 2, 2, 2] 
           \nonumber\\
        &+ 
           I[1, 2, 3, 1]\big)\Big) - I[0, 4, 2, 0] - 
    3 I[0, 4, 2, 1] + 3 I[0, 4, 2, 2] - I[0, 4, 2, 3] - 
    2 I[0, 4, 3, 0] + 4 I[0, 4, 3, 1] - 2 I[0, 4, 3, 2] \nonumber\\
        &+ 
    I[0, 4, 4, 0] - I[0, 4, 4, 1] + 4 I[1, 3, 2, 1] - 
    8 I[1, 3, 2, 2] + 4 I[1, 3, 2, 3] - 8 I[1, 3, 3, 1] + 
    8 I[1, 3, 3, 2] + 4 I[1, 3, 4, 1]\Bigg)   \Bigg\}\nonumber\\
    %%%%%%%%%%%%%%%%%%%%%%%%%%%%%%%%%%%%%%%%%%%%%%%%%%%%%%%%%%%%%%%%%%%%%%%%%%%%%%%%%%%%%%%%%%%%%%%%%%%%%%%%%%%%%%%%%%%%%%%%%%%%%%%%%%%%%%%%%%%%%%%%%%%%%%%%%%%%%%%%%%%%%%%%%%%%%%%%%%%%%%%%%%%%%%%%%%%%
    &
    - \frac {m_c P_ 2^2} {18874368 \pi^3}\Bigg\{\Big (8 e_d m_c m_s I[0, 
         3, 3, 0] + 56 e_u m_c m_s I[0, 3, 3, 0] - 
       3 e_d I[0, 4, 4, 0] - 
       3 e_u I[0, 4, 4, 0]\Big) I_ 4[\mathcal S] \nonumber\\
        &- 
    8 e_c \Bigg (8 m_c m_s \Big (I[0, 3, 1, 0] - 2 I[0, 3, 1, 1] + 
          I[0, 3, 1, 2] - 2 I[0, 3, 2, 0] + 2 I[0, 3, 2, 1] + 
          I[0, 3, 3, 0]\Big) + 3 I[0, 4, 1, 0] 
          \nonumber\\
        &- 15 I[0, 4, 1, 1] + 
       27 I[0, 4, 1, 2] - 21 I[0, 4, 1, 3] + 6 I[0, 4, 1, 4] - 
       10 I[0, 4, 2, 0] + 39 I[0, 4, 2, 1] - 48 I[0, 4, 2, 2]
       \nonumber\\
        &+ 
       19 I[0, 4, 2, 3] + 11 I[0, 4, 3, 0] - 31 I[0, 4, 3, 1] + 
       20 I[0, 4, 3, 2] - 4 I[0, 4, 4, 0] + 7 I[0, 4, 4, 1] + 
       2 m_ 0^2 \Big (6 I[0, 3, 1, 0] \nonumber\\
        &- 18 I[0, 3, 1, 1] + 
          18 I[0, 3, 1, 2] - 6 I[0, 3, 1, 3] - 19 I[0, 3, 2, 0] + 
          38 I[0, 3, 2, 1] - 19 I[0, 3, 2, 2] + 20 I[0, 3, 3, 0] 
          \nonumber\\
        &- 
          20 I[0, 3, 3, 1] - 7 I[0, 3, 4, 0] - 6 I[1, 2, 2, 1] + 
          3 I[1, 2, 2, 2] + 6 I[1, 2, 3, 1]\Big) - 4 I[1, 3, 2, 1] + 
       8 I[1, 3, 2, 2] 
       \nonumber\\
        &- 4 I[1, 3, 2, 3] + 8 I[1, 3, 3, 1] - 
       8 I[1, 3, 3, 2] - 4 I[1, 3, 4, 1]\Bigg) - 
    12 e_s \Big (I[0, 4, 2, 0] - 3 I[0, 4, 2, 1] + 3 I[0, 4, 2, 2] 
    \nonumber\\
        &- 
        I[0, 4, 2, 3] - 3 I[0, 4, 3, 0] + 6 I[0, 4, 3, 1] - 
        3 I[0, 4, 3, 2] + 3 I[0, 4, 4, 0] - 3 I[0, 4, 4, 1] - 
        I[0, 4, 5, 0] 
        \nonumber\\
        &+ 12 I[1, 3, 2, 1] - 12 I[1, 3, 2, 2] + 
        4 I[1, 3, 2, 3] - 24 I[1, 3, 3, 1] + 12 I[1, 3, 3, 2] + 
        12 I[1, 3, 4, 1]\Big)\Bigg\}\nonumber
 \end{align}
 
 \begin{align}
 &+\frac{ P_1}{434865438720 \pi^7}\Bigg\{8 m_c\Bigg(-12 e_c \Big (-3 I[0, 5, 1, 1] + 40 I[0, 5, 1, 2] - 
    91 I[0, 5, 1, 3] + 74 I[0, 5, 1, 4] - 20 I[0, 5, 1, 5] 
     \nonumber\\
        &- 
    11 I[0, 5, 2, 1] - 24 I[0, 5, 2, 2] + 59 I[0, 5, 2, 3] - 
    24 I[0, 5, 2, 4] + 13 I[0, 5, 3, 1] + 4 I[0, 5, 3, 2] - 
    6 I[0, 5, 3, 3] 
     \nonumber\\
        &- 5 I[0, 5, 4, 1] + 4 I[0, 5, 4, 2] + 
    10 m_c m_s \big (I[0, 4, 1, 1] - 3 I[0, 4, 1, 2] + 
       3 I[0, 4, 1, 3] - I[0, 4, 1, 4] - 2 I[0, 4, 2, 1]
        \nonumber\\
        &+ 
       4 I[0, 4, 2, 2] - 2 I[0, 4, 2, 3] + I[0, 4, 3, 1] - 
       I[0, 4, 3, 2] + 
       4 (I[1, 3, 1, 2] - 2 I[1, 3, 1, 3] + I[1, 3, 1, 4] - 
           2 I[1, 3, 2, 2]
            \nonumber\\
        &+ 2 I[1, 3, 2, 3] + 
           I[1, 3, 3, 2])\big) + 
    5 (I[1, 4, 1, 3] - 2 I[1, 4, 1, 4] + I[1, 4, 1, 5] - 
        I[1, 4, 2, 2] + I[1, 4, 2, 4] + 2 I[1, 4, 3, 2]
         \nonumber\\
        &- 
        I[1, 4, 3, 3] - I[1, 4, 4, 2])\Big) + 
 2 e_s \Big (216 I[0, 5, 1, 1] - 792 I[0, 5, 1, 2] + 
    1056 I[0, 5, 1, 3] - 600 I[0, 5, 1, 4] 
     \nonumber\\
        & + 120 I[0, 5, 1, 5] - 
    27 I[0, 5, 2, 0] - 828 I[0, 5, 2, 1] + 2250 I[0, 5, 2, 2] - 
    1860 I[0, 5, 2, 3] + 465 I[0, 5, 2, 4] 
     \nonumber\\
        &+ 81 I[0, 5, 3, 0] + 
    981 I[0, 5, 3, 1] - 1629 I[0, 5, 3, 2] + 543 I[0, 5, 3, 3] - 
    81 I[0, 5, 4, 0] - 234  I[0, 5, 4, 1] + 117 I[0, 5, 4, 2]
     \nonumber\\
        &+ 
    27 I[0, 5, 5, 0] - 27 I[0, 5, 5, 1] + 
    5 (180 I[1, 4, 1, 2] + 520 I[1, 4, 1, 3] - 500 I[1, 4, 1, 4] + 
        160 I[1, 4, 1, 5] + 27 I[1, 4, 2, 1]  
        \nonumber\\
        &+ 315 I[1, 4, 2, 2] - 
        671 I[1, 4, 2, 3] + 329 I[1, 4, 2, 4] - 
        81  I[1, 4, 3, 1] - 90 I[1, 4, 3, 2] + 151 I[1, 4, 3, 3] + 
        81 I[1, 4, 4, 1] 
         \nonumber\\
        &- 45 I[1, 4, 4, 2] - 
        27 I[1, 4, 5, 1])\Big) + 
 e_u \Big (504 I[0, 5, 1, 1] - 1788 I[0, 5, 1, 2] + 
    2324 I[0, 5, 1, 3] - 1300 I[0, 5, 1, 4] 
     \nonumber\\
        &+ 260 I[0, 5, 1, 5] + 
    9 I[0, 5, 2, 0] - 2040 I[0, 5, 2, 1] + 5112 I[0, 5, 2, 2] - 
    4108 I[0, 5, 2, 3] + 1027 I[0, 5, 2, 4] 
     \nonumber\\
        &- 27 I[0, 5, 3, 0] + 
    2577 I[0, 5, 3, 1] - 3849 I[0, 5, 3, 2] + 1283 I[0, 5, 3, 3] + 
    27 I[0, 5, 4, 0] + 66 I[0, 5, 4, 1] - 33 I[0, 5, 4, 2] 
     \nonumber\\
        &- 
    9 I[0, 5, 5, 0] + 9 I[0, 5, 5, 1] + 
    360 m_c m_s \big (2 I[0, 4, 1, 1] - 5 I[0, 4, 1, 2] + 
       4 I[0, 4, 1, 3] - I[0, 4, 1, 4] - 4 I[0, 4, 2, 1] 
        \nonumber\\
        &+ 
       6 I[0, 4, 2, 2] - 2 I[0, 4, 2, 3] + 2 I[0, 4, 3, 1] - 
       I[0, 4, 3, 2] + 
       4 (I[1, 3, 1, 2] - 2 I[1, 3, 1, 3] + I[1, 3, 1, 4] - 
           2 I[1, 3, 2, 2] 
            \nonumber\\
        &+ 2 I[1, 3, 2, 3] + 
           I[1, 3, 3, 2])\big) - 
    5 (468 I[1, 4, 1, 2] - 1412 I[1, 4, 1, 3] + 
        1420 I[1, 4, 1, 4] - 476 I[1, 4, 1, 5] + 9 I[1, 4, 2, 1] 
         \nonumber\\
        &- 
        1185 I[1, 4, 2, 2] + 2359 I[1, 4, 2, 3] - 
        1183 I[1, 4, 2, 4] - 27 I[1, 4, 3, 1] + 
        966 I[1, 4, 3, 2] - 947 I[1, 4, 3, 3] 
         \nonumber\\
        &+ 27 I[1, 4, 4, 1] - 
        249 I[1, 4, 4, 2] - 9  I[1, 4, 5, 1])\Big) + 
 e_d \Big (504 I[0, 5, 1, 1] - 1788 I[0, 5, 1, 2] + 
    2324 I[0, 5, 1, 3] 
     \nonumber\\
        &- 1300 I[0, 5, 1, 4] + 260 I[0, 5, 1, 5] + 
    9 I[0, 5, 2, 0] - 2040 I[0, 5, 2, 1] + 5112 I[0, 5, 2, 2] - 
    4108 I[0, 5, 2, 3] 
     \nonumber\\
        &+ 1027 I[0, 5, 2, 4] - 27 I[0, 5, 3, 0] + 
    2577 I[0, 5, 3, 1] - 3849 I[0, 5, 3, 2] + 1283 I[0, 5, 3, 3] + 
    27 I[0, 5, 4, 0]  \nonumber\\
        &
        + 66 I[0, 5, 4, 1] 
    - 33 I[0, 5, 4, 2] - 
    9 I[0, 5, 5, 0] + 9 I[0, 5, 5, 1] + 
    360 m_c m_s \big (2 I[0, 4, 1, 1] - 5 I[0, 4, 1, 2] + 
       4 I[0, 4, 1, 3] 
        \nonumber\\
        &- I[0, 4, 1, 4] - 4 I[0, 4, 2, 1] + 
       6 I[0, 4, 2, 2] - 2 I[0, 4, 2, 3] + 2 I[0, 4, 3, 1] - 
       I[0, 4, 3, 2] + 
       4 (I[1, 3, 1, 2] - 2 I[1, 3, 1, 3]
        \nonumber\\
        &+ I[1, 3, 1, 4] - 
           2 I[1, 3, 2, 2] + 2 I[1, 3, 2, 3] + 
           I[1, 3, 3, 2])\big) + 
    5 (468 I[1, 4, 1, 2] + 1412 I[1, 4, 1, 3] - 
        1420 I[1, 4, 1, 4] 
         \nonumber\\
        &+ 476 I[1, 4, 1, 5] - 9 I[1, 4, 2, 1] + 
        1185 I[1, 4, 2, 2] - 2359 I[1, 4, 2, 3] + 
        1183 I[1, 4, 2, 4] + 27 I[1, 4, 3, 1] 
         \nonumber\\
        &- 
        966 I[1, 4, 3, 2] + 947 I[1, 4, 3, 3] - 27 I[1, 4, 4, 1] + 
        249 I[1, 4, 4, 2] + 9 I[1, 4, 5, 1])\Big)  \Bigg)\nonumber\\
       & +
        f_{3 \gamma}m_c \Bigg(
         \Big (-45  (11 e_d + 24 e_s) I[0, 4, 3, 0] + 
   e_u (736 m_c m_s I[0, 3, 2, 0] - 459  I[0, 4, 3, 0] - 
      30 I[0, 4, 4, 0]) \nonumber\\
        &- 60 e_s I[0, 4, 4, 0]\Big) I_2[\mathcal V]
        +64 \Big ( \big (8 e_d m_c m_s I[0, 3, 2, 0] + 
      8 e_u m_c m_s I[0, 3, 2, 0] + 4 e_d m_c m_s I[0, 3, 3, 0] + 
      4 e_u m_c m_s \nonumber\\
       & \times I[0, 3, 3, 0] + e_d I[0, 4, 3, 0] + 
      e_s I[0, 4, 3, 0] + e_u I[0, 4, 3, 0] + 
      3 (3 e_d - 7 e_s + 3 e_u) I[0, 4, 4, 0]\big) I_ 5[\psi^a] 
      \nonumber\\
       &+ 
   4 m_c m_s (-4 e_d I_ 6[\psi_ {\gamma} {\nu}] I[0, 3, 2, 
         0] + (e_d + e_u) \big (-I[0, 3, 1, 0] + 7 I[0, 3, 1, 1] - 
           9 I[0, 3, 1, 2] + 3 I[0, 3, 1, 3]
           \nonumber\\
       &+ 2 I[0, 3, 2, 0] - 
           8 I[0, 3, 2, 1] + 4 I[0, 3, 2, 2] - I[0, 3, 3, 0] + 
           I[0, 3, 3, 1]\big) \psi^a[u_ 0])\Big)
           +\Big (e_s \big (-21 I[0, 4, 1, 0]
           \nonumber\\
       &+ 82 I[0, 4, 1, 1] - 
        121 I[0, 4, 1, 2] + 80 I[0, 4, 1, 3] - 20 I[0, 4, 1, 4] + 
        63 I[0, 4, 2, 0] - 185 I[0, 4, 2, 1] + 183 I[0, 4, 2, 2]
        \nonumber\\
       &- 
        61 I[0, 4, 2, 3] - 63 I[0, 4, 3, 0] + 124 I[0, 4, 3, 1] - 
        62 I[0, 4, 3, 2] + 21 I[0, 4, 4, 0] - 21 I[0, 4, 4, 1]\big) + 
     e_d \big (9 I[0, 4, 1, 0] 
     \nonumber\\
       &- 38 I[0, 4, 1, 1] + 
        59 I[0, 4, 1, 2] - 40 I[0, 4, 1, 3] + 10 I[0, 4, 1, 4] - 
        27 I[0, 4, 2, 0] + 85 I[0, 4, 2, 1] - 87 I[0, 4, 2, 2]
        \nonumber\\
       &+ 
        29 I[0, 4, 2, 3] + 27 I[0, 4, 3, 0] - 56 I[0, 4, 3, 1] + 
        28 I[0, 4, 3, 2] - 9 I[0, 4, 4, 0] + 9 I[0, 4, 4, 1]\big) + 
     e_u \big (9 I[0, 4, 1, 0]
     \nonumber\\
       &- 38 I[0, 4, 1, 1] + 
         59 I[0, 4, 1, 2] - 40 I[0, 4, 1, 3] + 10 I[0, 4, 1, 4] - 
         27 I[0, 4, 2, 0] + 85 I[0, 4, 2, 1] - 87 I[0, 4, 2, 2] 
         \nonumber\\
       &+ 
         29 I[0, 4, 2, 3] + 27 I[0, 4, 3, 0] - 56 I[0, 4, 3, 1] + 
         28 I[0, 4, 3, 2] - 9 I[0, 4, 4, 0] + 
         9 I[0, 4, 4, 1]\big)\Big) \psi^a[u_ 0] 
         \nonumber\\
       &+ 
 8 e_d m_c m_s \big (-I[0, 3, 1, 2] + I[0, 3, 1, 3] + 
     I[0, 3, 2, 2]\big) \psi_ {\gamma}^{\nu}[u_ 0]
\Bigg)
        \Bigg\}\nonumber
 \end{align}
 \begin{align}
&-\frac{m_c (3P_2-P_3)}{377487360 \pi^5} \Bigg\{ 
 (e_u+2e_d -5e_s)\Bigg (10 f_ {3\gamma} \pi^2 I_2[\mathcal V] (10 m_ 0^2 m_c I[0, 
        3, 3, 0] - 3 m_s I[0, 4, 4, 0]) + 
   3 \Big ( (-m_c I[0, 5, 3, 0] 
   \nonumber\\
       &+ 
          21 m_s I[0, 5, 4, 0]) I_ 4[\mathcal S] + 
       5 m_ 0^2 \big (3 m_s \big (-I[0, 4, 2, 0] + 
             3 I[0, 4, 2, 1] - 3 I[0, 4, 2, 2] + I[0, 4, 2, 3] + 
             3 I[0, 4, 3, 0] 
             \nonumber\\
       &- 6 I[0, 4, 3, 1] + 3 I[0, 4, 3, 2] - 
             3 I[0, 4, 4, 0] + 3 I[0, 4, 4, 1] + I[0, 4, 5, 0] - 
             12 I[1, 3, 2, 1] + 12 I[1, 3, 2, 2] - 
             4 I[1, 3, 2, 3] 
             \nonumber\\
       &+ 24 I[1, 3, 3, 1] - 
             12 I[1, 3, 3, 2] - 12 I[1, 3, 4, 1]\big) + 
          14 m_c \big (I[0, 4, 2, 0] - 3 I[0, 4, 2, 1] + 
              3 I[0, 4, 2, 2] - I[0, 4, 2, 3] 
              \nonumber\\
       &- 2 I[0, 4, 3, 0] + 
              4 I[0, 4, 3, 1] - 2 I[0, 4, 3, 2] + I[0, 4, 4, 0] - 
              I[0, 4, 4, 1] + 4 I[1, 3, 2, 1] - 8 I[1, 3, 2, 2] + 
              4 I[1, 3, 2, 3] 
              \nonumber\\
       &- 8 I[1, 3, 3, 1] + 
              8 I[1, 3, 3, 2] + 4 I[1, 3, 4, 1]\big)\big) - 
       4 \big (7 m_c \big (2 I[0, 5, 2, 1] - 5 I[0, 5, 2, 2] + 
              4 I[0, 5, 2, 3] - I[0, 5, 2, 4] 
              \nonumber\\
       &- 4 I[0, 5, 3, 1] + 
              6 I[0, 5, 3, 2] - 2 I[0, 5, 3, 3] + 2 I[0, 5, 4, 1] -
               I[0, 5, 4, 2] + 5 I[1, 4, 2, 2] - 
              10 I[1, 4, 2, 3] 
              \nonumber\\
       &+ 5 I[1, 4, 2, 4] - 
              10 I[1, 4, 3, 2] + 10 I[1, 4, 3, 3] + 
              5 I[1, 4, 4, 2]\big) + 
           3 m_s \big (I[0, 5, 2, 0] - 4 I[0, 5, 2, 1] + 
               6 I[0, 5, 2, 2] 
               \nonumber\\
       &- 4 I[0, 5, 2, 3] + I[0, 5, 2, 4] - 
               3 I[0, 5, 3, 0] + 9 I[0, 5, 3, 1] - 
               9 I[0, 5, 3, 2] + 3 I[0, 5, 3, 3] + 
               3 I[0, 5, 4, 0] - 6 I[0, 5, 4, 1]
               \nonumber\\
       &+ 
               3 I[0, 5, 4, 2] - I[0, 5, 5, 0] + I[0, 5, 5, 1] + 
               5 I[1, 4, 2, 1] - 15 I[1, 4, 2, 2] + 
               15 I[1, 4, 2, 3] - 5 I[1, 4, 2, 4] - 
               15 I[1, 4, 3, 1] 
               \nonumber\\
       &+ 30 I[1, 4, 3, 2] - 
               15 I[1, 4, 3, 3] + 15 I[1, 4, 4, 1] - 
               15 I[1, 4, 4, 2] - 
               5 I[1, 4, 5, 1]\big)\big)\Big)\Bigg)
\Bigg\}\nonumber\\
%%%%%%%%%%%%%%%%%%%%%%%%%%%%%%%%%%%%%%%%%%%%%%%%%%%%%%%%%%%%%%%%%%%%%%%%%%%%%%%%%%%%%%%%%%%%%%%%%%%%%%%%%%%%%%%%%%%%%%%%%%%%%%%%%%%%%%%%%%%%%%%%%%%%%%%%%%%%%%%%%%%%%%%%%%%%%%%%%%%%%%%%%%%%%%%%%%%%%%%%%%%%%%%%%%%%%%%%%%%
&+\frac{m_c^2 m_s}{754974720 \pi^7} \Bigg\{e_c \Big (2 I[0, 6, 1, 3] - 5 I[0, 6, 1, 4] + 4 I[0, 6, 1, 5] - 
    I[0, 6, 1, 6] - 4 I[0, 6, 2, 3] + 6 I[0, 6, 2, 4] - 
    2 I[0, 6, 2, 5] 
    \nonumber\\
       &+ 2 I[0, 6, 3, 3] - I[0, 6, 3, 4] + 
    6 I[1, 5, 1, 4] - 12 I[1, 5, 1, 5] + 6 I[1, 5, 1, 6] - 
    12 I[1, 5, 2, 4] + 12 I[1, 5, 2, 5] + 
    6 I[1, 5, 3, 4]\Big) 
    \nonumber\\
       &- (e_d + e_u) \Big (3 I[0, 6, 2, 2] - 
    7 I[0, 6, 2, 3] + 5 I[0, 6, 2, 4] - I[0, 6, 2, 5] - 
    6 I[0, 6, 3, 2] + 8 I[0, 6, 3, 3] - 2 I[0, 6, 3, 4]
    \nonumber\\
       &+ 
    3 I[0, 6, 4, 2] - I[0, 6, 4, 3] + 6 I[1, 5, 2, 3] - 
    12 I[1, 5, 2, 4] + 6 I[1, 5, 2, 5] - 12 I[1, 5, 3, 3] + 
    12 I[1, 5, 3, 4] 
    \nonumber\\
       &+ 6 I[1, 5, 4, 3]\Big) \Bigg\}\nonumber\\
       %%%%%%%%%%%%%%%%%%%%%%%%%%%%%%%%%%%%%%%%%%%%%%%%%%%%%%%%%%%%%%%%%%%%%%%%%%%%%%%%%%%%%%%%%%%%%%%%%%%%%%%%%%%%%%%%%%%%%%%%%%%%%%%%%%%%%%%%%%%%%%%%%%%%%%%%%%%%%%%%%%%%%%%%%%%%%%%%%%%%%%%%%%
       & -\frac{m_c}{21139292160 \pi^7} \Bigg\{
       e_c \Big (60 I[0, 7, 1, 3] - 210 I[0, 7, 1, 4] + 270 I[0, 7, 1, 5] - 
    150 I[0, 7, 1, 6] + 30 I[0, 7, 1, 7] 
    \nonumber\\
       &- 176 I[0, 7, 2, 3] + 
    441 I[0, 7, 2, 4] - 354 I[0, 7, 2, 5] + 89 I[0, 7, 2, 6] + 
    172 I[0, 7, 3, 3] - 260 I[0, 7, 3, 4] + 88 I[0, 7, 3, 5]
    \nonumber\\
       &- 
    56 I[0, 7, 4, 3] + 29 I[0, 7, 4, 4] + 7 I[1, 6, 2, 4] - 
    14 I[1, 6, 2, 5] + 7 I[1, 6, 2, 6] - 14 I[1, 6, 3, 4] + 
    14 I[1, 6, 3, 5]
    \nonumber\\
       &+ 7  I[1, 6, 4, 4]\Big) + 
 42 (e_d + e_s + e_u) \Big (3 I[0, 7, 2, 2] - 10 I[0, 7, 2, 3] + 
    12 I[0, 7, 2, 4] - 6 I[0, 7, 2, 5] + I[0, 7, 2, 6] 
    \nonumber\\
       &- 
    9 I[0, 7, 3, 2] + 21 I[0, 7, 3, 3] - 15 I[0, 7, 3, 4] + 
    3 I[0, 7, 3, 5] + 9 I[0, 7, 4, 2] - 12 I[0, 7, 4, 3] + 
    3 I[0, 7, 4, 4] 
    \nonumber\\
       &- 3 I[0, 7, 5, 2] + I[0, 7, 5, 3] + 
    7 I[1, 6, 2, 3] - 21 I[1, 6, 2, 4] + 21 I[1, 6, 2, 5] - 
    7 I[1, 6, 2, 6] - 21 I[1, 6, 3, 3] 
    \nonumber\\
       &+ 42 I[1, 6, 3, 4] - 
    21 I[1, 6, 3, 5] + 21 I[1, 6, 4, 3] - 21 I[1, 6, 4, 4] - 
    7 I[1, 6, 5, 3]\Big)
       \Bigg\},
 \end{align}
%
where 
\begin{align*}
 {M^2}= \frac{M_1^2 M_2^2}{M_1^2+M_2^2}, ~~~
 u_0= \frac{M_1^2}{M_1^2+M_2^2},
\end{align*}
%
%$ {M^2}= \frac{M_1^2 M_2^2}{M_1^2+M_2^2}$ , $u_0= \frac{M_1^2}{M_1^2+M_2^2} $ 
with $ M_1^2 $ and $ M_2^2 $ being the Borel parameters in the initial and final states, respectively.
Since the same pentaquarks are existed in our the initial and final states, hence we can put, M$_1^2$ = M$_2^2$= 2 M$^2$, which gives rise to $u_0 = 1/2 $.
We can interpret this result as that each quark and antiquark carries half the momentum of the photon.  Here $P_1 =\langle g_s^2 G^2\rangle$ is gluon condensate, $P_2 =\langle \bar q q \rangle$ stands for u/d-quark condensate and $P_3 =\langle \bar s s  \rangle$ denotes s-quark condensate. 
%
  The~$I[n,m,l,k]$ and $I_i[\mathcal{F}]$ functions are
defined as:
\begin{align}
 I[n,m,l,k]&= \int_{4 m_c^2}^{s_0} ds \int_{0}^1 dt \int_{0}^1 dw~ e^{-s/M^2}~
 s^n\,(s-4\,m_c^2)^m\,t^l\,w^k,\nonumber%\\
   \end{align}
 \begin{align}
 I_1[\mathcal{F}]&=\int D_{\alpha_i} \int_0^1 dv~ \mathcal{F}(\alpha_{\bar q},\alpha_q,\alpha_g)
 \delta'(\alpha_ q +\bar v \alpha_g-u_0),\nonumber\\
  I_2[\mathcal{F}]&=\int D_{\alpha_i} \int_0^1 dv~ \mathcal{F}(\alpha_{\bar q},\alpha_q,\alpha_g)
 \delta'(\alpha_{\bar q}+ v \alpha_g-u_0),\nonumber\\
    I_3[\mathcal{F}]&=\int D_{\alpha_i} \int_0^1 dv~ \mathcal{F}(\alpha_{\bar q},\alpha_q,\alpha_g)
 \delta(\alpha_ q +\bar v \alpha_g-u_0),\nonumber\\
   I_4[\mathcal{F}]&=\int D_{\alpha_i} \int_0^1 dv~ \mathcal{F}(\alpha_{\bar q},\alpha_q,\alpha_g)
 \delta(\alpha_{\bar q}+ v \alpha_g-u_0),\nonumber\\
 %\end{align}
 %\begin{align}
   I_5[\mathcal{F}]&=\int_0^1 du~ \mathcal{F}(u)\delta'(u-u_0),\nonumber\\
 I_6[\mathcal{F}]&=\int_0^1 du~ \mathcal{F}(u),\nonumber
 \end{align}
 where $\mathcal{F}$ denotes the corresponding photon DAs.


 \end{widetext}

\bibliography{Possible_pentaquarksMM.bib}


\end{document}
