\section{Conclusions, Limitations and Future Work}

We present a novel algorithm to navigate a legged robot in densely vegetated environments with various traversability. We utilize a few-shot learning classifier trained on a few hundred quadrants of RGB images to distinguish vegetation with different pliability with minimal human annotation. This classification model is combined with estimated vegetation height (from multiple local cost maps), and classification confidence using a novel cost map clearing scheme. Using the resulting vegetation-aware cost map, we deploy a local planner with recovery behaviors to save the robot if it freezes or gets entrapped in dense vegetation.  

Our algorithm has a few limitations. Our primary navigation assumes non-holonomic robot dynamics to utilize DWA's dynamic feasibility guarantees. However, the legged robot dynamics are holonomic and further investigation is required to extend our method to relax its action constraints. We assume that the different kinds of vegetation are not intertwined with one another. This may not be the case in highly forested environments. VERN could lead to collisions in the presence of thin obstacles such as branches that are not detected in the cost maps. In the future, we would like to augment our current method with proprioceptive sensing for navigating more complex terrains with occluded surfaces.