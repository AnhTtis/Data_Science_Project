%%%%%%%%%%%%%%%%%%%%%%%%%%%%%%%%%%%%%%%%%%%%%%%%%%%%%%%%%%%%%%%%%%%%%%%%%%%%%%%%
%2345678901234567890123456789012345678901234567890123456789012345678901234567890
%        1         2         3         4         5         6         7         8

\documentclass[letterpaper, 10 pt, conference]{ieeeconf}  % Comment this line out if you need a4paper
\linespread{0.92}
%\documentclass[a4paper, 10pt, conference]{ieeeconf}      % Use this line for a4 paper

\IEEEoverridecommandlockouts                              % This command is only needed if 
                                                          % you want to use the \thanks command

\overrideIEEEmargins                                      % Needed to meet printer requirements.

%In case you encounter the following error:
%Error 1010 The PDF file may be corrupt (unable to open PDF file) OR
%Error 1000 An error occurred while parsing a contents stream. Unable to analyze the PDF file.
%This is a known problem with pdfLaTeX conversion filter. The file cannot be opened with acrobat reader
%Please use one of the alternatives below to circumvent this error by uncommenting one or the other
%\pdfobjcompresslevel=0
%\pdfminorversion=4

% See the \addtolength command later in the file to balance the column lengths
% on the last page of the document

% The following packages can be found on http:\\www.ctan.org
%\usepackage{graphics} % for pdf, bitmapped graphics files
%\usepackage{epsfig} % for postscript graphics files
%\usepackage{mathptmx} % assumes new font selection scheme installed
%\usepackage{times} % assumes new font selection scheme installed
%\usepackage{amsmath} % assumes amsmath package installed
%\usepackage{amssymb}  % assumes amsmath package installed
\definecolor{purple}{rgb}{1, 0, 1}

\newcommand{\ie}{\emph{i.e.,}\xspace}
\newcommand{\eg}{\emph{e.g.,}\xspace}
\newcommand{\abr}{\emph{abbr.}\xspace}
\newcommand{\ea}{\emph{et al.}\xspace}
\newcommand{\gensync}{\emph{GenSync}\xspace}
\newcommand{\colosseum}{\emph{Colosseum}\xspace}
\newcommand{\srep}{\emph{SREP}\xspace} % Set Reconciliation Enhances
\newcommand{\srepsim}{\emph{SREPSim}\xspace}
% Propagation
\newcommand{\esrep}{\emph{E-SREP}\xspace}
\newcommand{\epsrep}{\emph{EP-SREP}\xspace}
\newcommand{\mesrep}{\emph{ME-SREP}\xspace}
\newcommand{\mempoolsync}{\emph{MempoolSync}}

\newcommand{\fref}[1]{Fig.~\ref{#1}}
\newcommand{\tref}[1]{Table~\ref{#1}}
\newcommand{\aref}[1]{Algorithm~\ref{#1}}
\newcommand{\procref}[1]{Procedure~\ref{#1}}
\newcommand{\sref}[1]{Section~\ref{#1}}
\newcommand{\lineref}[1]{line~\ref{#1}}
\newcommand{\appref}[1]{Appendix~\ref{#1}}

% Change \eqref
\LetLtxMacro{\originaleqref}{\eqref}
\renewcommand{\eqref}{Eq.~\originaleqref}

% Theorems and corollaries
\newcounter{theoremcount}
\setcounter{theoremcount}{0}
\DeclareRobustCommand{\theorem}[1]{%
  \refstepcounter{theoremcount}%
  \noindent\textit{\textbf{Theorem \thetheoremcount\label{theorem:#1}: }}%
}
\DeclareRobustCommand{\theoremref}[1]{Theorem~\ref{theorem:#1}}

\DeclareRobustCommand{\proof}{\emph{Proof:}\xspace}
\DeclareRobustCommand{\qqed}{\hfill$\blacksquare$}

\newcounter{corollcount}
\setcounter{corollcount}{0}
\DeclareRobustCommand{\coroll}[1]{%
  \refstepcounter{corollcount}%
  \noindent\textit{\textbf{Corollary \thecorollcount\label{coroll:#1}: }}%
}
\DeclareRobustCommand{\corollref}[1]{Corollary~\ref{coroll:#1}}

\newcounter{lemmacount}
\setcounter{lemmacount}{0}
\DeclareRobustCommand{\lemma}[1]{%
  \refstepcounter{lemmacount}%
  \noindent\textit{\textbf{Lemma \thelemmacount\label{lemma:#1}: }}%
}
\DeclareRobustCommand{\lemmaref}[1]{Lemma~\ref{lemma:#1}}

\newcounter{definitioncount}
\setcounter{definitioncount}{0}
\DeclareRobustCommand{\definition}[1]{%
  \refstepcounter{definitioncount}%
  \noindent\textit{\textbf{Definition \thedefinitioncount\label{definition:#1}: }}%
}
\DeclareRobustCommand{\defref}[1]{Definition~\ref{definition:#1}}

%notes of different authors
\newif\ifnotes
\notestrue
\notesfalse

\newif\ifdiff
\difftrue
\difffalse

\newcommand{\anote}[1]{\ifnotes $\ll$\textsf{\textcolor{purple}{Ari: {#1}}}$\gg$ \fi}
\newcommand{\nnote}[1]{\ifnotes $\ll$\textsf{\textcolor{orange}{Novak: {#1}}}$\gg$ \fi}
\newcommand{\diff}[1]{\ifdiff\textcolor{orange}{#1}\else#1\fi}

%%% Local Variables:
%%% mode: latex
%%% TeX-master: "main"
%%% End:

\title{\LARGE \bf VERN: Vegetation-aware Robot Navigation in Dense Unstructured Outdoor Environments
}


% \author{Albert Author$^{1}$ and Bernard D. Researcher$^{2}$% <-this % stops a space
% \thanks{*This work was not supported by any organization}% <-this % stops a space
% \thanks{$^{1}$Albert Author is with Faculty of Electrical Engineering, Mathematics and Computer Science,
%         University of Twente, 7500 AE Enschede, The Netherlands
%         {\tt\small albert.author@papercept.net}}%
% \thanks{$^{2}$Bernard D. Researcheris with the Department of Electrical Engineering, Wright State University,
%         Dayton, OH 45435, USA
%         {\tt\small b.d.researcher@ieee.org}}%
% }

\author{Adarsh Jagan Sathyamoorthy$^1$, Kasun Weerakoon$^1$, Tianrui Guan,$^1$ Mason Russell$^2$, \\ Damon Conover$^2$, Jason Pusey$^2$ and Dinesh Manocha$^1$ % <-this % stops a space
\thanks{This work has been conducted as part of the ArtIAMAS cooperative agreement (https://artiamas.umd.edu) and is being partially funded by Army Research Laboratory Grant No. W911NF2120076.}% <-this % stops a space
}


\begin{document}



\maketitle
\thispagestyle{empty}
\pagestyle{empty}

\footnotetext[1]{Authors are with the University of Maryland, College Park.}

\footnotetext[2]{Authors are with the DEVCOM Army Research Laboratory, Aberdeen Proving Ground, Maryland, USA.}

%%%%%%%%%%%%%%%%%%%%%%%%%%%%%%%%%%%%%%%%%%%%%%%%%%%%%%%%%%%%%%%%%%%%%%%%%%%%%%%%
% TODO: Add percentages
\begin{abstract}
We propose a novel method for autonomous legged robot navigation in densely vegetated environments with a variety of pliable/traversable and non-pliable/untraversable vegetation. We present a novel few-shot learning classifier that can be trained on a few hundred RGB images to differentiate flora that can be navigated through, from the ones that must be circumvented. Using the vegetation classification and 2D lidar scans, our method constructs a vegetation-aware traversability cost map that accurately represents the pliable and non-pliable obstacles with lower, and higher traversability costs, respectively. Our cost map construction accounts for misclassifications of the vegetation and further lowers the risk of collisions, freezing and entrapment in vegetation during navigation. Furthermore, we propose holonomic recovery behaviors for the robot for scenarios where it freezes, or gets physically entrapped in dense, pliable vegetation. We demonstrate our method on a Boston Dynamics Spot robot in real-world unstructured environments with sparse and dense tall grass, bushes, trees, etc. We observe an increase of 25-90\% in success rates, 10-90\% decrease in freezing rate, and up to 65\% decrease in the false positive rate compared to existing methods.
\end{abstract}

\vspace{-10pt}
%%%%%%%%% BODY TEXT

\section{Introduction}
\label{section:introduction}
%% 1. why should someone care?

%The advent of advanced interactive computer vision systems~\cite{hololens} and recent progress in vision-language and multi-modal models~\cite{} opens doors for such next generation of assistive agents. 
% We envision that the future assistive agents would build up on these visual and language reasoning capabilities of today and empower users to achieve goals in their everyday lives. In particular, such agents would be able to reason about \emph{unseen} human goals... 
% We posit that such agents would require the ability to understand user goals described in natural language at high-level i.e., without complete details about as well as unseen user goals. 

%Recent progress in augmented reality systems~\cite{hololens, magicleap}, as well as vision-language and multi-modal models~\cite{}, opens doors for the next generation of assistive agents. 
Inspired by recent progress in visual systems~\cite{MagicLeap, ungureanu2020hololens}, we consider an assistive egocentric agent capable of reasoning about daily activities. When invoked via natural language commands, for e.g., while baking a cake, the agent understands the steps involved in baking, tracks progress through the various stages of the task, detects and proactively prevents mistakes by making suggestions. Such an agent would empower users to learn new skills and accomplish tasks efficiently.
% One could envision invoking such an agent merely through natural language descriptions of tasks similar to how present day assistants such as Alexa, Siri etc.~\cite{voice_assistants} are invoked. 
%We envision such agents to empower users in daily life by  invoking them naturally through 

%% 2. Why is it challenging? 
%While recent progress in vision-language and multi-modal models~\cite{} opens doors for such next generation of assistive agents, various challenges remain in making such agents a reality. 
%To make such agents a reality, 

Developing such an egocentric agent capable of tracking and verifying everyday tasks based on their natural language specification is challenging for multiple reasons. First, such an agent must reason about various ways of doing a \emph{multi-step} task specified in natural language. This entails decomposing the task into relevant actions, state changes, object interactions as well as any necessary causal and temporal relationships between these entities. Secondly, the agent must ground these entities in egocentric observations to track progress and detect mistakes. Lastly, to truly be useful, such an agent must support tracking and verification for a combination of tasks and, ideally, even unseen tasks. These three challenges -- causal and temporal reasoning about task structure from natural language, visual grounding of sub-tasks, and compositional generalization -- form the core goals of our work.

% %% 3. What are we doing? What is our approach?
% \aks{I think this is a matter of preference, but I personally don't like related work in intro. I would make this paragraph be about EgoTV and NSG. Starting with something like - "To this end, we propose...", ie, your next paragraph.}
% \nk{+1, we should move parts of this para to lit review and delete the rest.}
% Recent research on language modeling enables decomposing tasks into multiple steps from natural language descriptions~\cite{llm_zero_shot_planning,proscript}. However, such \emph{task decompositions} cannot directly be leveraged for task tracking in egocentric agents because of lack of grounding into the visual observations or context. In parallel, the computer vision community has advanced action recognition~\cite{}, object detection and tracking~\cite{}, hand object interaction and object state change detection~\cite{ego_4d,change_it,}, step classification in procedural tasks~\cite{}, and even vision language reasoning~\cite{nsvqa,nscl,star_situated_reasoning,clevrer}, which may help with the grounding challenge. However, majority of current research on identifying actions, objects, steps, or state changes does not account for the overall task structure. Likewise, predominant research on vision language understanding~\cite{} and multi-modal grounding~\cite{} does not consider the temporal and causal constraints that emerge in task tracking and verification. We therefore focus on the order-aware visual grounding problem in our work, with an eye towards compositional generalization to scale usability of these agents. In particular, we aim to achieve visual grounding of the actions and objects corresponding to each step or sub-task obtained from the task description decomposition in an order-aware manner.

%% 4. What are our results/contributions?
As our first contribution, we propose a benchmark -- \emph{\textbf{Ego}centric \textbf{T}ask \textbf{V}erification} (\etv \inlineimg{figures/TV}) -- and a corresponding dataset in the AI2-THOR~\cite{ai2thor} simulator. % \emoji{tv}
Given a natural language (NL) task description and a corresponding egocentric video of an agent, the goal of \etv is to verify whether the task was successfully completed in the video or not.
\etv contains multi-step tasks with \emph{ordering} constraints on the steps and \emph{abstracted} NL task descriptions with omitted low-level task details inspired by the needs of real-world assistants. We also provide splits of the dataset focused on different generalization aspects, e.g., unseen visual contexts, compositions of steps, and tasks (see Figure~\ref{figure:dataset}).
% Next, we create splits of the dataset focused on different aspects of generalization, ranging from generalization to unseen visual context to unseen compositions of steps and tasks. Figure~\ref{figure:dataset} shows an example task and overview of generalization splits from \etv. Succeeding at \etv tasks requires decomposing tasks into partially-ordered steps from the NL description and order-aware visual grounding of these steps into the video. 

Our second contribution is a novel approach for order-aware visual grounding~--~\emph{\textbf{N}euro-\textbf{S}ymbolic \textbf{G}rounding} (NSG), capable of compositional reasoning and generalizing to unseen tasks owing to its ability to leverage abstract NL descriptions and compositional structure of tasks (task decomposition, ordering).~In contrast, state-of-the-art vision-language models~\cite{coca,clip,videoclip,clip_hitchiker} struggle to ground NL descriptions in egocentric videos, and do not generalize to unseen tasks.~NSG outperforms these models by~$\mathbf{33.8}\%$~on compositional generalization and~$\mathbf{32.8}\%$~on abstractly described task verification. Finally, to evaluate \nsg on real-world data, we instantiate \etv on the CrossTask~\cite{cross_task} instructional video dataset. %Specifically, we synthetically create videos with mistakes in CrossTask. 
We find that it also outperforms state-of-the-art models at task verification on CrossTask. We hope that the \etv~benchmark and dataset will enable future research on egocentric agents capable of aiding in everyday tasks.

% We experiment with many for the \etv tasks. We find that while these models generalize well to unseen visual context, they struggle to perform grounding from abstracted task descriptions and to generalize to new compositions of tasks. To deal with these challenges, we take inspiration from recent research on and develop . ~\rd{unclear why neurosymbolic models would do well on abstraction.} 

% To summarize, our main contributions are:~1)~\etv: a benchmark and synthetic dataset to systematically study egocentric task verification.
% 2)~\nsg: a novel neuro-symbolic approach to enable the core reasoning capability for \etv -- order-aware visual grounding. We demonstrate \nsg's capability on our synthetic \etv dataset as well as a real-world dataset derived from CrossTask. We will release both of these datasets and our models for future research on egocentric task tracking and verification. 


% Assistive agents require the ability to track actions and state changes from an egocentric perspective for effective assistance in day-to-day tasks. For example, an agent helping a user prepare a recipe would need to both generate the steps of the recipe (\textit{plan generation}) and track the user's actions to ensure the plan is executed correctly (\textit{plan verification}). We formulate this as a Video Entailment task~\cite{violin_dataset,9710490} \rd{should we call our task video-based goal entailment?}, wherein, given an egocentric video of an agent (or human) performing a task (\textit{premise}) and a NL task description (\textit{hypothesis}), the objective is to learn a model to track whether the given task was successfully executed in the video. 
% An ideal model should also be able to seamlessly generalize to novel compositions (of actions and objects) unseen during training. \rd{add a line about what we mean by abstraction and why is it important.} To this end, we generate a novel Vision-Language dataset on the AI2-THOR simulator~\cite{ai2thor} to study compositional and abstraction-based generalization. Our dataset provides effective evaluation measures in a controlled setting, while closely reflecting the diversity of real-world events. We implement and train a variety of end-to-end models based on existing state-of-the-art approaches. We empirically demonstrate that neural models suffer from overfitting and cannot effectively generalize to novel compositions of actions, objects, and scenes. 
% To address this problem, we propose an end-to-end Neuro-Symbolic (NeSy) framework that performs plan generation and verification. At the heart of our approach is the hypothesis that symbolic reasoning models are good at generalization and capturing compositional substructure, while neural models are good at learning representations from sensory data~\cite{10.5555/3326943.3327039,nscl,clevrer}. \rd{summarize contributions in a bulleted list.} \rd{also add a line about the main result e.g., x\% improvement as compared to end-to-end models}. 

% \rd{we also evaluate NeSy with real-world data: add briefly about CrossTask experiments.}

% % \fbox{\begin{minipage}{\linewidth}
% % \textbf{Problem Statement}

% % Given:
% % (i) Premise: Egocentric video of an agent performing a task.
% % (ii) Hypothesis: NL description of the task.

% % Learn: A model to track whether the premise entails the hypothesis. The output of the model is True if the given task is executed successfully in the video.
% % \end{minipage}}

% \textbf{Contributions:} 
% \begin{itemize}
%     \item We generate a benchmark video-language dataset to study compositional and abstraction-based generalization.
%     \item We evaluate the performance of a variety of state-of-the-art models and show that these (baseline) models cannot effectively generalize to novel compositions of actions.
%     \item We propose a novel end-to-end NeSy approach that significantly outperforms the baselines on some compositional generalization splits while performing on par with them on the rest.
%     \item We also evaluate our NeSy approach with real-world data showing similar performance improvements.
% \end{itemize}

\section{Related Work}
% TODO: 
%   1. Add content for some DRL works, Graspe
%   2. Organize it as two subsecs vegetation detection and navigation
%   3. Add sentences for VERN overcoming limitations (tall vegetation, pliable+non-pliable together)

In this section, we discuss previous works in outdoor navigation and perception of unstructured vegetation.

\subsection{Perception in Dense Vegetation}
Early works on vegetation detection were typically chlorophyll detectors \cite{Bradley-2004-8857,nguyen-1,double-check-passable}, or basic classification models \cite{auto-terrain-characterization-modeling}.  Nguyen et al. \cite{nguyen-1} proposed a sensor setup to detect near-IR reflectance and used it to define a novel vegetation index. \cite{double-check-passable} extended this detection by using an air compressor device to create strong winds and estimating the levels of resistance to robot motion using the movement of vegetation. However, it needed the robot to be static. Multiple sensor configurations have also been explored for obstacle detection within vegetation \cite{sensing-tech-obstacle-detect} of which thermal camera and RADAR stand out as effective modalities.

Modeling the frictional/lumped-drag characteristics of vegetation \cite{momentum-traversal-mobility-challenges,char-traversal-pliable-veg,thesis} as a measure of the resistance offered to motion, or modeling plant stems using their rotational stiffness and damping characteristics \cite{modeling-traversal-pliable-materials} has also been proposed. However, the method does not use visual feedback, requiring the robot to drive through vegetation first to gauge its pliability.

Wurm et al. \cite{veg-laser-data-structured-outdoor,wurm2009improving} presented methods to detect short grass from lidar scans based on its reflective properties on near IR light. Astolfi et al. \cite{vineyard} demonstrated accurate SLAM and navigation through sensor fusion in a vineyard. However, such methods operated in highly structured environments. 

There are several works that utilize semantic segmentation to understand terrain traversability \cite{ganav,semantic-mapping-auto-off-road-nav,terrain-semantics-multi-legged}. \cite{terrain-semantics-multi-legged} applied semantic segmentation classifier trained on RGB images to oct-tree maps obtained from RGB-D point cloud. Maturana et al. \cite{semantic-mapping-auto-off-road-nav} took a similar approach by training a segmenter and augmenting a 2.5D grid map with it to distinguish tall grass from regular obstacles. However, these existing methods are not suitable for perception in dense, unstructured vegetation.

% TODO: Add RL-based works. Mention lack of simulators for vegetation we wish to operate on
% Cite Graspe
\subsection{Navigation in Unstructured Vegetation}
% Generic outdoor nav works
Although there are many works on outdoor, off-road navigation to handle slopes \cite{terp} and different terrain types \cite{terrapn}, there have been only a few methods for detecting and navigating through vegetation. \cite{Overbye-1} proposed using a terrain gradient map along with the A* algorithm to navigate a large wheeled robot and demonstrated moving over tall grass and short bushes. \cite{Overbye-2} extended this by adding a segmentation layer to the map to detect soft obstacles. However, these methods mostly operate in structured, isolated vegetation and may not work well in unstructured scenarios.

% and used path optimization in the actuator space to obtain fast, optimized plans that obey the robot's kinematic constraints. However, these methods mostly operate in structured, isolated vegetation and may not work well in unstructured scenarios.

% Tall grass nav works
There are several works that have addressed navigating through pliable vegetation such as tall grass \cite{badgr,model-error-katyal}. Kahn et al. \cite{badgr} demonstrated a model that learns from a robot's real-world experiences such as collisions, bumpiness, and its position to navigate outdoor environments. The robot learned from RGB images and associated experiences (labels) to consider tall grass as traversable. However, it does not account for the presence of non-traversable bushes or trees alongside traversable vegetation. Polevoy et al. \cite{model-error-katyal} proposed a model that regresses the difference between the robot's dynamics model and its actual realized trajectory in unstructured vegetation. This acts as a measure of the surrounding vegetation and terrain's traversability. However, both these methods require ``negative" examples such as collisions during training, which may be impractical or dangerous to collect in the real world. 

% The lack of photo-realistic simulators has given rise to offline reinforcement learning methods \cite{offline-rl}. They have demonstrated promising behaviors in grassland-like environments by training with RGB images. However, it is typically difficult to converge to a good navigation policy with such methods.

With the advent of legged robots, several recent works have been focused on developing robust controllers for locomotion purely using proprioception \cite{quadruped-locomotion-challenging-terrain} or fusing it with exteroceptive perception such as elevation maps \cite{robust-perceptive-locomotion-quad}.  A few navigation works focus on estimating the underlying support surface that is hidden by vegetation \cite{support-surface-legged-robot} by fusing haptic feedback from the robot, depth images, and detecting the height of the vegetation. Our method is complementary to these existing works.  

     


\vspace{-5pt}
\section{Diffusion Models and Part-Level Shape Representation}
\label{sec:background}

\subsection{Background on Diffusion Models}
\label{sec:background_ddpm}
We first briefly overview the technical background of diffusion models.
Diffusion models~\cite{Ho:2019DDPM} are latent variable models that approximate a data distribution $q (\mathbf{x}^{(0)})$ with a Markov chain, which is also called a \emph{reverse process}:
\vspace{-2pt}
\begin{align}
    p_{\theta} (\mathbf{x}^{(0)}) \coloneqq \int p_{\theta} (\mathbf{x}^{(0:T)}) d \mathbf{x}^{(1:T)},
\end{align}
where $p_{\theta} (\mathbf{x}^{(0:T)}) = p (\mathbf{x}^{(T)}) \Pi_{t=1}^{T} p_{\theta} (\mathbf{x}^{(t-1)} \vert \mathbf{x}^{(t)})$.
Here, $p (\mathbf{x}^{(T)}) = \mathcal{N} (\mathbf{x}^{(T)}; \mathbf{0}, \mathbf{I})$ is the standard normal prior
enabling tractable sampling.

The conditional probabilities $\{ p_{\theta} (\mathbf{x}^{(t-1)} \vert \mathbf{x}^{(t)}) \}_{t=1}^{T}$ are parameterized by a neural network whose weights are denoted by $\theta$.
The weights are optimized through the \emph{forward} diffusion process $q (\mathbf{x}^{(1:t)} \vert \mathbf{x}^{(0)})$ that sequentially adds Gaussian noises to the data $\mathbf{x}^{(0)} \sim q (\mathbf{x}^{(0)})$:
\begin{align}
\begin{gathered}
    q (\mathbf{x}^{(1:t)} \vert \mathbf{x}^{(0)}) \coloneqq \Pi_{s=1}^{t} q (\mathbf{x}^{(s)} \vert \mathbf{x}^{(s-1)}), \\
    \text{where} \,\, q (\mathbf{x}^{(s)} \vert \mathbf{x}^{(s-1)}) \coloneqq \mathcal{N} \left(\mathbf{x}^{(s)}; \sqrt{1 - \beta^{(s)}} \mathbf{x}^{(s-1)}, \beta^{(s)} \mathbf{I} \right),
    \raisetag{35pt}
\end{gathered}
\end{align}
and $\beta^{(s)}$ is an element of a monotonically increasing sequence $\beta^{(1:T)} \in (0, 1]^{T}$.
By choosing Gaussians as forward diffusion kernels, the conditional densities $q (\mathbf{x}^{(t)} \vert \mathbf{x}^{(0)})$ at $t=1,\dots,T$ can be expressed in the closed form:
\begin{align}
    q (\mathbf{x}^{(t)} \vert \mathbf{x}^{(0)}) = \mathcal{N} (\mathbf{x}^{(t)}; \sqrt{\bar{\alpha}^{(t)}} \mathbf{x}^{(0)}, (1 - \bar{\alpha}^{(t)}) \mathbf{I}),
\end{align}
where $\alpha^{(t)} \coloneqq 1 - \beta^{(t)}$ and $\bar{\alpha}^{(t)} \coloneqq \Pi_{s=1}^{t} \alpha^{(s)}$.
Over the forward process dissipating a sample $\mathbf{x}^{(0)} \sim q (\mathbf{x}^{(0)})$ toward $q(\mathbf{x}^{(T)}) = \mathcal{N} (\mathbf{0}, \mathbf{I})$, the weights $\theta$ parameterizing the reverse process $p_{\theta} (\mathbf{x}^{(0)})$ are learned by optimizing the following variational bound on negative log likelihood:
\begin{equation}
\begin{aligned}
    \mathbb{E}_{q (\mathbf{x}^{(0)})} & [-\log p_{\theta} (\mathbf{x}^{(0)})] \leq \\
    &\mathbb{E}_{q (\mathbf{x}^{(0)}, \dots, \mathbf{x}^{(T)})} \left[-\log \frac{p_{\theta} (\mathbf{x}^{(0:T)})}{q (\mathbf{x}^{(1:T)} \vert \mathbf{x}^{(0)})}\right].
\end{aligned}
\end{equation}
Following Ho \etal~\cite{Ho:2019DDPM}, we parameterize our reverse process $p_{\theta} (\mathbf{x}^{(t-1)} \vert \mathbf{x}^{(t)})$ as:
\begin{equation}
\begin{aligned}
    p_{\theta} (\mathbf{x}^{(t-1)} \vert \mathbf{x}^{(t)}) \coloneqq \mathcal{N} (\mathbf{x}^{(t-1)}; \boldsymbol{\mu}_{\theta} (\mathbf{x}^{(t)}, t), \beta^{(t)} \mathbf{I}).
\end{aligned}
\end{equation}
In particular, we use the parameterization $\boldsymbol{\mu}_{\theta} (\mathbf{x}^{(t)}, t) = \sfrac{1}{\sqrt{\alpha^{(t)}}} (\mathbf{x}^{(t)} - \sfrac{\beta^{(t)}}{\sqrt{1 - \bar{\alpha}^{(t)}}} \boldsymbol{\epsilon}_{\theta} (\mathbf{x}^{(t)}, t))$ and optimize its parameters $\theta$ with a training objective that encourages a network $\boldsymbol{\epsilon}_{\theta}$ to predict the noise $\boldsymbol{\epsilon} \sim \mathcal{N}(\mathbf{0}, \mathbf{I})$ present in the given data:
\begin{align}
    \mathcal{L}(\theta) \coloneqq \mathbb{E}_{t, \mathbf{x}^{(0)}, \boldsymbol{\epsilon}} \left[\left\lVert \boldsymbol{\epsilon} - \boldsymbol{\epsilon}_{\theta} \left(\sqrt{\bar{\alpha}^{(t)}} \mathbf{x}^{(0)} + \sqrt{1 - \bar{\alpha}^{(t)}}\boldsymbol{\epsilon}, t\right) \right\rVert^{2}\right].
    \label{eq:ddpm_loss}
\end{align}







\begin{figure}[t!]
\label{fig:spaghetti_overview}
\includegraphics[width=\linewidth]{figures/pipeline2_draft.pdf}
\caption{\textbf{Part-Level implicit representation by Hertz~\etal~\cite{Hertz:2022Spaghetti}.} A latent vector $\mathbf{z}$ encoding global geometry is first mapped to a set of part latents $\{\mathbf{p}_i\}_{i=1}^N$, each of which is decomposed into extrinsic parameters $\{\mathbf{e}_i\}_{i=1}^N$ and intrinsic latents $\{\mathbf{s}_i\}_{i=1}^N$. The decoder, conditioned on $\{(\mathbf{e}_i, \mathbf{s}_i)\}_{i=1}$, outputs an occupancy value given a query point $\mathbf{x}$.}
\vspace{-10pt}
\end{figure}

\vspace{-15pt}
\subsection{Part-Level Shape Representation}
\label{sec:background_part_representation}
Neural implicit representations~\cite{Chen:2019ImNet, Park:2019Deepsdf, Mescheder:2019OccNet} have been widely exploited in 3D shape generation and reconstruction due to their advantages in capturing fine details without limitation in resolutions even with a small memory footprint. However, their disadvantage of not supporting intuitive editing and manipulation has been a hindrance to increasing their utilization. To remedy the drawback, recent works~\cite{Genova:2019LearningShapeTemplates,Genova:2020LDIF,Hao:2020Dualsdf,Hui:2022NeuralTemplate,Hertz:2022Spaghetti} introduced \emph{dual} representations combining explicit and implicit representations, taking advantage of both of them. Among them, Hertz~\etal~\cite{Hertz:2022Spaghetti}, which our work is based on, was the first introducing a hybrid representation integrating two types of disentanglements simultaneously into an implicit representation: 1) part-level disentanglement, representing each local region separately, and 2) extrinsic-intrinsic disentanglement, describing extrinsic properties (\ie~the approximate shape and transformations) with parameters in the 3D space while encoding intrinsic properties (\ie~geometric details) using a latent code. This novel representation, called SPAGHETTI~\cite{Hertz:2022Spaghetti}, is learned in an auto-decoding setup without any supervision of the part decomposition.


In SPAGHETTI, a 3D shape is first mapped to a global latent $\mathbf{z}$ and then further encoded into a set of part embedding vectors $\{\Vp_i\}_{i=1}^{N}$, where $N$ denotes the number of parts. Each part embedding vector $\Vp_i$ is again mapped into both a set of extrinsic parameters $\Ve_i$ and an intrinsic latent $\Vs_i$ through an MLP. 
The set of extrinsic parameters $\Ve_i = \{\mathbf{c}_i, \mathbf{\Sigma}_i, \mathbf{\pi}_i \}$ of each part represents a Gaussian in the 3D space with mean $\mathbf{c}_i\in\mathbb{R}^3$ and covariance $\mathbf{\Sigma}_i \in \mathbb{R}^{3 \times 3}$, depicting an approximate shape of a part.
$\pi_i \in \mathbb{R}$ is the blending weight for the Gaussian mixture representation of the entire shape: $\sum_{i}\pi_i \mathcal{N}(\B{x} | \B{c}_i, \mathbf{\Sigma}_i)$, describing the volume of the shape as a probability distribution. Since $\{\Ve_i\}_{i=1}^{N}$ can only encode the part-level structural information, the intrinsic latents $\{\Vs_i\}_{i=1}^{N}$ supplement the detailed geometry information so that the pairs of the extrinsic parameters and intrinsic latents can be decoded back to the original shape in an implicit form. Specifically, an implicit decoder $\mathcal{D}$ is trained to predict an occupancy value at point $\mathbf{x}$:
\vspace{-0.5\baselineskip}
\begin{equation}
\begin{aligned}
\label{eq:decoder}
    o = \mathcal{D}\left(\mathbf{x}\, \Big\vert \, \{\Ve_i\}_{i=1}^{N},\,\{\Vs_i\}_{i=1}^{N}\right),
\end{aligned}
\end{equation}
where occupancy value $o \in [0,1]$ is 1 when the query point is inside the shape, and 0 otherwise. 
The keys to achieving both the part-level and extrinsic-intrinsic disentanglements in the training of decoder $\mathcal{D}$ are the regularizations forcing a single pair $(\Ve_i, \Vs_i)$ of a part to determine the occupancy of each point, and the Gaussian parameters in $\Ve_i$ to transform the corresponding local region. See the original paper~\cite{Hertz:2022Spaghetti} for the details of the decoder training.

The extrinsic vector $\mathbf{e}_i$ is precisely represented as a $16$-dimensional vector $\{\mathbf{c}_i, \lambda_i^1, \lambda_i^2, \lambda_i^3, \mathbf{u}_i^1, \mathbf{u}_i^2, \mathbf{u}_i^3, \pi_i\}$, where $\lambda_i^j \in \mathbb{R}$ and $\mathbf{u}_i^j \in \mathbb{R}^3$ are eigenvalues and eigenvectors of the covariance matrix $\mathbf{\Sigma}_i$, while the intrinsic vector $\mathbf{s}_i$ is a 512-dimensional vector. Note that the much smaller extrinsic vector contains the approximate shape information of the part; we leverage this fact in our effective cascaded diffusion model.

Also, note that SPAGHETTI is trained in an auto-decoding setup while regularizing the global latent code $\mathbf{z} \in \mathbb{R}^{512}$ to follow the unit Gaussian. Thus, the shapes can be simply generated by sampling a latent code $\mathbf{z}$ from the unit Gaussian in the $\mathbf{z}$ space, although we demonstrate that diffusion in the extrinsic and intrinsic embedding spaces can produce much more plausible shapes (Section~\ref{sec:shape_generation}).


 
\section{VERN: Vegetation Classification}
Our vegetation classifier uses an RGB image ($I^{RGB}_t$) obtained from a camera on the robot at a time $t$ as input. Although a plant's structure is well preserved in an image, using the entire image for classification is infeasible because the scene in $I^{RGB}_t$ typically contains two or more types of vegetation together. Therefore, we split $I^{RGB}_t$ into four quadrants $Q_1$ (top-left), $Q_2$ (top-right), $Q_3$ (bottom-left), $Q_4$ (bottom-right) as shown in Figs. \ref{fig:network-arch}, \ref{fig:costmap_comparisons}. A quadrant predominantly contains a single type of vegetation typically (see Fig. \ref{fig:costmap_comparisons}). 

We classify vegetation in the quadrants into four classes: 1. sparse grass, 2. dense grass, 3. bush, and 4. tree. We separate sparse and dense grass due to their visual dissimilarity. 

% Mention how mobilenetv3 was made lighter. 
% Explain our network's outputs (distance used as confidence)

\subsection{Data Preparation}
To create the training dataset, images collected from different environments are split into quadrants and grouped together manually based on the vegetation type predominant in a quadrant. Next, pairs of images are created either from the same group or from different groups. The pair with similar images (same group) is automatically labeled 1, or 0 otherwise. The entire data preparation process takes about 30-45 minutes of manual effort. The obtained image pairs and the corresponding labels are passed into the classification network for training.

\subsection{Network Architecture}
% Architecture Explanation
Our vegetation classifier network consists of two identical feature extraction branches to identify the similarity between input image pairs. Our feature extraction branches are based on the MobileNetv3 \cite{mobilenetv3} backbone. We choose MobileNetv3 because it incorporates depth-wise separable convolutions (i.e., fewer parameters), leading to a comparatively lightweight and fast neural network. The outputs of the MobileNetv3 branches ($h_1$ and $h_2$) are one-dimensional latent feature vectors of the corresponding input images. The euclidean distance between the two feature vectors is calculated and passed through a $sigmoid$ activation layer to obtain the predictions.

% Network Archictecture figure
\begin{figure}[t]
      \centering
      \includegraphics[width=7.5cm,height=4.5cm]{Images/network_ref_images.png}
      \caption {\small{VERN's classification network architecture. Quadrants of the real-time camera image are paired with several reference images for each class and fed into the two identical branches of our network. Example reference images for each class are shown at the bottom.}}
      \label{fig:network-arch}
      \vspace{-10pt}
\end{figure}

We utilize the \textit{contrastive loss} function, which is capable of learning discriminative features to evaluate our model during training. Let $\hat{y}$ be the prediction output from the model. The contrastive loss function $J$ is:
\vspace{-5pt}
\begin{equation}
    J = \hat{y} \cdot d^2 + (1-\hat{y}) \cdot max(margin-d,0)^2,
\end{equation}

where $d$ is the euclidean distance between the feature vectors $h_1$ and $h_2$. $Margin$ is used to ensure that dissimilar image pairs are at least $margin$ distance apart.

% Outputs
\subsection{Network Outputs} \label{sec:net-outputs}
During run-time, the quadrants $Q_1, Q_2, Q_3,$ and $Q_4$ are each paired with several reference images of the four classes. The several reference images per class have different viewpoints and lighting conditions. They are fed into the classifier model $\mathcal{F}$ as a batch to obtain predictions as, 
% \vspace{-5pt}
\begin{equation}
    \mathcal{F}(Q_1, Q_2, Q_3, Q_4) = \Tilde{V}_{4 \times 4} |\,\, (\Tilde{V}_{ij} \in [0, 1]) \,\, i, j \in \{1, 2, 3, 4\} .
\end{equation}

% \no Here, $\Tilde{V}_{4 \times 4}$ is the output prediction matrix whose $i^{th}$ row corresponds to quadrant $Q_i$ ($i = \{1, 2, 3, 4\}$) and each column corresponds to one of the vegetation categories.
\no $\Tilde{V}_{4 \times 4}$ is the output prediction matrix whose values $\Tilde{V}_{i,j}$ correspond to the \textit{least} euclidean distance of the quadrant $Q_i$ from any of the reference images for the $j^{th}$ class. The closer $\Tilde{V}_{i,j}$ is to 0, the higher the similarity between $Q_i$ and class $j$. 

We extract two outputs from $\Tilde{V}_{4 \times 4}$. Namely, for each quadrant $Q_i$: 1. the vegetation class that is most similar to it ($\Tilde{v}_i$), and 2. the corresponding similarity score ($d_{i}$):
\vspace{-5pt}
\begin{equation}
    \Tilde{v}_i = \underset{j}{\operatorname{argmin}} \Tilde{V}_{i,j}, \,\, \text{and} \,\, d_{i} = \Tilde{V}_{i,j}. 
\end{equation}

\no For readability, hereafter we denote $\Tilde{v}_i$ as belonging to the Pliable Vegetation (PV) set if $\Tilde{v}_i = 1$ or $2$ (sparse/dense grass), and to the Non-Pliable Vegetation (NPV) set if $\Tilde{v}_i = 3$ or $4$ (bush/tree). We define the confidence of classification $\kappa_i$ as,
\vspace{-7pt}
\begin{equation}
    \kappa_i = e^{-\alpha \cdot d_i}, 
\end{equation}
% \vspace{-5pt}
\no where $\alpha$ is a tunable parameter. We observe that $d_i \to 0 \implies \kappa_i \to 1$, and vice versa.

% Hence the output prediction is a vector of length four, with elements indicating whether each quadrant of the image belongs to the grass category or not.

% \begin{figure}
%     \centering
%     \includegraphics[width=\columnwidth,height=5.3cm]{paper-template-latex/Images/vegetation_classes}
%     \caption{Vegetation categories used in our classifier network. (a) Sparse grass; (b) Dense grass; (c) Bushes; (d) Trees. We consider the sparse and dense grass regions as pliable, while bushes and trees are non-pliable.}
%     \label{fig:vegetation_classes}
%     \vspace{-5pt}
% \end{figure}
\section{VERN: Navigation in Dense Vegetation}
We use occupancy grids/cost maps generated using 2D lidar scans to detect obstacles around the robot. Our cost map can be defined as,

\vspace{-10pt}
\begin{equation}
    C_z(row, col) = \{p \,\, | \,\, p \in \{0, 100\}\}.
\end{equation}

\no Here, z is the height (along the robot's Z-axis) at which the 2D scan is recorded on a plane parallel to the ground. $p$ is a binary variable representing if an obstacle is present at $(row, col)$ of the cost map. In a densely vegetated environment, $C_{z}$ contains obstacles in all the locations where the lidar's scans have a finite proximity value. However, it does not account for the pliability of certain types of vegetation, which makes them passable and therefore not true obstacles. Therefore, we use the classification results, its confidence (Section \ref{sec:net-outputs}), and estimated vegetation height to augment our cost map prior to navigation.

% However, $C_z$ does not account for the pliability of the entities viewed as obstacles. Therefore, we use the classification results in Section \ref{sec:net-outputs} to augment our cost map prior to navigation.

% In the following sections, we explain how multiple cost maps are used to identify \textit{critical regions} with non-pliable obstacles, how our novel cost map clearing algorithm works, where we handle misclassifications in our model $\mathcal{F}$. Additionally, we explain our recovery mechanism to unstick the robot when it freezes or gets stuck physically. VERN's overall system architecture is shown in Fig. \ref{fig:sys-arch}. 

\subsection{Multi-view Cost Maps}
To estimate the height of the environmental obstacles, we use three cost maps of equal dimensions $C_{low}, C_{mid},$ and $C_{high}$ corresponding to 2D scans from three different heights in 3D lidar's point clouds. The cost maps correspond to a height lower, equal to, and higher than the height at which the robot's 3D lidar is mounted, respectively. $C_{low}$ contains obstacles of all heights around the robot. $C_{mid}$ and $C_{high}$ contain taller obstacles such as tall grass, trees, walls, humans, etc. Using only three views instead of projecting the entire point cloud onto a 2D plane reduces the computation burden. It also helps identify truly tall obstacles and avoids misclassifying overhanging leaves from trees as tall obstacles.

We consider tall obstacles as \textit{critical obstacles} due to the high probability of them being solid obstacles such as trees and walls. To identify critical obstacles, we perform the element-wise sum operation as follows,
\vspace{-7pt}
\begin{gather}
    % crit = \{ (row, col) \,\, | \,\, C_{crit}(row, col) > c_{thresh} \} \\
    C_{crit} = C_{high} \bigoplus C_{mid} \bigoplus C_{low}.
\end{gather}

\no $C_{crit}(row, col) \in \{0, 100, 200, 300\}$, and regions with higher costs contain critical obstacles.

% We also use crit as an indicator to cross-verify the classification results $\Tilde{v}_i$ to make our perception robust to errors (see Section \ref{sec:costmap-clearing}).


\subsection{Vegetation-aware Cost Map} \label{sec:costmap-clearing}
To combine $\Tilde{v}_i$, $\kappa_i$ and $C_{crit}$, we must correlate the regions viewed by quadrants $Q_{1,2,3,4}$ in $I^{RGB}_t$ with the corresponding regions in $C_{crit}$. 

\subsubsection{Homography}
To this end, we apply a homography transformation $H$ to project $I^{RGB}_t$ onto the cost map. Let $reg()$ denote a function that returns the real-world $(x, y)$ coordinates corresponding to a region of a cost map or image. We obtain the image quadrant relative to the cost map (see green/red rectangles in Fig. \ref{fig:costmap_comparisons} bottom) as,  
\vspace{-3pt}
\begin{equation}
    Q^C_i = \{ (row, col) | reg(C(row, col)) = reg(H(Q_i)) \}.
\end{equation}

\subsubsection{Cost Map Clearing}
% We utilize the classification results, and the confidence corresponding to the four quadrants of our RGB image along with vegetation height to clear/modify the costs of the grids in $C_{low}$. 

We now use $\Tilde{v}_i$, $\kappa_i$, $C_{crit}$, and $Q^C_i$ to clear/modify the costs of the grids in $C_{low}$. We choose to modify and plan over $C_{low}$ since it detects obstacles of all heights. First, we calculate the normalized height measure (between 0 to $\pi/2$) of the obstacles in each quadrant $i$ using $C_{crit}$ as,
\vspace{-10pt}
\begin{equation}
    h_i = mean(C_{crit}(Q^C_i))/c_{max} \cdot \frac{\pi}{2}, 
\end{equation}

\no where, $c_{max} = 300$ the maximum value in $C_{crit}$. Next, in each quadrant $Q^C_i$ of $C_{low}$ we modify the cost as,

\vspace{-10pt}
\begin{gather}
    C_{VA}(Q^C_i) = C_{low}(Q^C_i) \cdot \frac{clear(\kappa_i, h_i)}{max(clear(\kappa_i, h_i))}\\
    clear() = 
    \begin{cases}
    w_{s \,\text{or} \,d} \cdot (1 - \kappa_i) + \frac{2 \cdot h_i}{\pi} \,\, & \text{if} \, \Tilde{v}_i \, \text{is PV} \\
    (w_{NPV} \cdot \kappa_i + b_{NPV}) + \sin(h_i) \,\, & \text{if} \, \Tilde{v}_i \, \text{is NPV}. \label{eqn:clearing}
    \end{cases}
\end{gather}
\vspace{-10pt}

\no Here, $C_{VA}$ is a vegetation-aware cost map, $w_s, w_d, w_{NPV}$, and $b_{NPV}$ are positive constants satisfying the condition $w_d > w_s$, and $b_{NPV} > w_d + 1$. The weights $w_s$ and $w_d$ are used when $\Tilde{v}_i$ is sparse and dense grass respectively. We incorporate $sin(.)$ for NPV to ensure that the significantly tall obstacles have higher costs and differentiable cost values from short obstacles. In contrast, we consider a linearly varying cost function w.r.t. the vegetation height for PV since the resistance to the robot from pliable tall objects is correlated with their height. We note that $\sin(h_i) \ge \frac{2 \cdot h_i}{\pi}$ for $h_i \in [0, \pi/2]$.

For PV classifications, low confidence and tall vegetation lead to higher navigation costs. Intuitively, this leads to the robot preferring to navigate through high-confidence, and short pliable vegetation whenever possible. Conversely, for NPV classifications, high confidence, and tall vegetation lead to higher costs since such regions should definitely be avoided. Accounting for low-confidence classifications in the formulation helps handle misclassifications and assign costs accordingly.

\begin{prop}
The clear() function modifies $C_{low}$ such that traversability costs in PV regions are always lower than in NPV regions.  
\end{prop}

\begin{proof}
The maximum cost for PV is $w_d + 1$ (when $\kappa_i \to 0$ and $h_i \to \pi/2$), and the minimum cost for NPV is $b_{NPV}$ (when $\kappa_i \to 0$ and $h_i \to 0$). Conditions $w_s < w_d$, and $b_{NPV} > w_d + 1 \implies$ max cost of PV $<$ min cost of NPV. Therefore, in the absence of free space, the robot always navigates through PV and avoids NPV regions. Therefore, regions with PV will always be preferred for navigation.
\end{proof}

%  TODO: Explain two modes of navigation: standard and cautious
\subsection{Cautious Navigation}
We adapt DWA (section \ref{sec:dwa}) to use our vegetation-aware cost map for robot navigation. We calculate the obstacle cost (obs(.)) associated with every $(v, \omega)$ pair by projecting its predicted trajectory $traj^C(v, \omega)$ relative to the cost map over $C_{VA}$ and summing as,
\vspace{-3pt}
\begin{equation}
    obs(v, \omega) = \sum_{(row, col) \in traj^C(v, \omega)} C_{VA}(row, col).
\end{equation}
% \vspace{-5pt}
\no Next, we compute the total cost $Q(v, \omega)$ (equation \ref{eq:dwa_obj_func}). The $(v, \omega)$ that minimizes this cost is used for navigation. In some cases, the robot might have to navigate a region with high traversability cost (represented say in $Q^C_i$) which typically occurs with NPV. To imbibe cautious navigation behaviors for such scenarios, we stunt the robot's complete velocity space as $V_s = [[0, \kappa_i \cdot v_{max}], \kappa_i \cdot [-\omega_{max}, \omega_{max}]]$. 

\subsection{Recovery Behaviors}
In highly dense vegetation or in regions with low confidence classifications, the robot could still freeze or get physically entrapped. We observe that Spot's high degrees of freedom and superior dynamics allows it to recover itself from such situations if it avoids rotational motions. Therefore, we propose holonomic recovery behaviors to recover Spot from such situations. To this end, the robot periodically stores $\textit{safe}^O$ locations relative to the odom frame whenever the conditions for freezing or entrapment (section \ref{sec:adverse-phen}) are not satisfied. If the conditions are satisfied, the robot stores its current location into an $\textit{unsafe}^O$ location list and chooses a safe location such that,

% \begin{gather}
%     \textbf{p}^O_{rec} = \underset{\textbf{p}^O \in safe}{\operatorname{argmin}} \left(dist(\textbf{g}^O, \textbf{p}^O) \right) \,\, s.t \,\,  (\textbf{P}^O - \textbf{P}^O_{rob}) \in Free^O \cap Grass^O, \\
%     [v_x, v_y] = k_p \cdot [\textbf{p}_{rec} - \textbf{p}_{rob}].
% \end{gather}
\vspace{-15pt}
\begin{gather}
    unsafe^C = T^C_O \cdot unsafe^O, \\
    C_{VA}(unsafe^C) = \infty,  
\end{gather}
\vspace{-15pt}
\begin{multline}
    \textbf{p}^O_{rec} = \underset{\textbf{p}^O \in safe}{\operatorname{argmin}} \left(dist(\textbf{g}^O, \textbf{p}^O) \right), \\
    s.t \,\,  (\textbf{P}^O - \textbf{P}^O_{rob}) \in Free^O \cap Grass^O,
    \label{eqn:recover-pt}
\end{multline}
\vspace{-15pt}
\begin{equation}
    [v_x, v_y] = k_p \cdot [\textbf{p}_{rec} - \textbf{p}_{rob}].
\end{equation}

\no Here, $\textbf{p}^O_{rec}$, $\textbf{g}^O$, and $\textbf{p}^O_{rob}$ are the safe location chosen to recover to, the robot's goal and current location. The condition in equation \ref{eqn:recover-pt} ensures that the line connecting the robot and the recovery point only lies in traversable regions. Once the robot recovers, it proceeds to its goal after marking the \textit{unsafe} location with high costs in $C_{VA}$ to circumvent it. 

% Function $LOS^C(\textbf{p}, \textbf{p}_{rob})$ checks if the line connecting the points $\textbf{p}$ and $\textbf{p}_{rob}$ (transformed w.r.t cost map) intersects with any obstacle in $C_{VA}$.   


% \begin{prop}
% At any time instant, navigation using $C_{VA}$ leads to equal or lower deviations from the goal compared to using $C_{low}$.  
% \end{prop}

% \begin{proof}
% Due to our cost map clearing, the quadrants $Q^C_{1,2,3,4}$ in $C_{VA}$ contain costs $\le$ the costs in the same quadrants in $C_{low}$ $\implies$ the obs() costs for all the trajectories calculated using $C_{VA} \le$ obs() costs for the trajectories using $C_{low}$ $\implies$ $\exists $
% \end{proof}
\section{Results and Evaluations}
We detail our method's implementation and experiments conducted on a real robot. Then, we perform ablation studies and comparisons to highlight VERN's benefits.

\begin{figure*}[t]
    \centering
    \includegraphics[width=14.75cm,height=4.75cm]{Images/costmap_comparison_5.png}
    \caption{\small{Snapshots of the vegetation classification results on RGB images (top), and the corresponding cost maps $C_{VA}$ marked with critical obstacles (bottom) from trials in scenarios 2 (left), 3 (center), and 4 (right). A green/red rectangle in a quadrant $Q_{1,2,3,4}$ in the RGB image represents a PV/NPV classification respectively. The same quadrants are projected onto the cost maps ($Q^C_{1,2,3,4}$) along with their classifications for visualization. The cost for the vegetation within the green regions is reduced significantly using equation \ref{eqn:clearing}. The encircled regions in each cost map correspond to an obstacle in the RGB image. [\textbf{Left}]: White - Tree, Blue - Tall grass, [\textbf{Center}]: White - Tree, Blue - Bush, [\textbf{Right}]: White - Tree, Blue - Human. Our classifier accurately detects trees, and bushes as non-pliable/untraversable, and tall grass as traversable. In scenario 4, it accurately detects the human as untraversable due to its dissimilarity with all training classes and tall height.}}
    \label{fig:costmap_comparisons}
    \vspace{-10pt}
\end{figure*}

\subsection{Implementation and Dataset}
VERN's classifier is implemented using Tensorflow 2.0. It is trained in a workstation with an Intel Xeon 3.6 GHz processor and an Nvidia Titan GPU using real-world data collected from a manually teleoperated Spot robot. The robot is equipped with an Intel NUC 11 (a mini-PC with Intel i7 CPU and NVIDIA RTX 2060 GPU), a Velodyne VLP16 LiDAR, and an Intel RealSense L515 camera. The training images ($\sim600$ per class) were collected from various environments in the University of Maryland, College Park campus. Our model takes about 6 hours to train for 55 epochs. 

% \begin{figure}[t]
%       \centering
%       \includegraphics[width=\columnwidth,height=5.0cm]{paper-template-latex/Images/costmap_va_4.png}
%       \caption {\small{$C_{VA}$ corresponding to scenario 2 (\ref{fig:navigation_comparisons} [center]) where the robot is surrounded by dense, tall grass and trees. The regions with critical obstacles are in red, the homography corresponding to the four image quadrants (green indicates PV classification in the rectangular region, blue indicates NPV classification). The robot is marked as a yellow circle. [\textbf{Top right}]:} The view from the robot's camera, and the corresponding quadrants, and classifications. We observe that the left quadrants predominantly contain a tree, and the right quadrants contain tall grass. Our planner's candidate trajectories are shown in orange, and the optimal velocity's trajectory (in green) navigates the robot towards the pliable tall grass on the right.}
%       \label{fig:costmap}
%       \vspace{-15pt}
% \end{figure}


\begin{figure*}[t]
    \centering
    \includegraphics[width=17cm,height=4cm]{Images/sceanrios_w_trajectories_v3.png}
    \caption{\small{Spot robot navigating in: \textbf{[Left]} Scenario 1, \textbf{[Center]} Scenario 2, and \textbf{[Right]} Scenario 4. We observe that VERN navigates through pliable/traversable vegetation in the absence of free space. This leads to significantly higher success rates, low freezing rates, and trajectory lengths. Other methods either freeze or take long, meandering trajectories to the goal.}}
    \label{fig:navigation_comparisons}
    \vspace{-10pt}
\end{figure*}

% \begin{figure}[t]
%       \centering
%       \includegraphics[width=\columnwidth,height=3.5cm]{paper-template-latex/Images/classifier_results_qualitative.png}
%       \caption {\small{ Example vegetation classifier predictions for two camera images from the robot and their corresponding quadrants. Blue color indicates  non-pliable vegetation regions and green color indicates pliable vegetation. We observe that our classifier is able to identify trees, bushes, and solid unseen objects (see the partially visible black color trash can in the top right quadrant of image (b)) as non-pliable while sparse and no-grass regions are detected as pliable. }}
%       \label{fig:classifier_results}
%       % \vspace{-15pt}
% \end{figure}

\subsection{Evaluations} 
We use the following evaluation metrics to compare VERN's performance against several navigation methods: 1. Boston Dynamics' in-built autonomy on Spot, 2. DWA \cite{fox1997dwa}, 3. GA-Nav \cite{ganav}, and 4. GrASPE \cite{graspe}. Spot's in-built autonomy uses its stereo cameras in four directions around the robot, estimates obstacles, and navigates to a goal. DWA is a local planner that utilizes a 2D LiDAR scan for obstacle avoidance. GA-Nav combines semantic segmentation for terrain traversability estimation with elevation maps for outdoor navigation. It is trained on publically available image datasets (RUGD \cite{RUGD2019IROS} and RELLIS-3D \cite{jiang2020rellis3d}). GrASPE is a multi-modal fusion framework that estimates perception sensor reliability to plan paths through unstructured vegetation. We further compare VERN without height estimation and recovery behaviors. The metrics we use for evaluations are,

\begin{itemize}
    \item \textbf{Success Rate} - The number of successful goal-reaching attempts (while avoiding non-pliable vegetation and collisions) over the total number of trials.

    \item \textbf{Freezing Rate} - The number of times the robot got stuck or started oscillating for more than 5 seconds while avoiding obstacles over the total number of attempts. Lower values are better.

    \item \textbf{Normalized Trajectory Length} - The ratio between the robot's trajectory length and the straight-line distance to the goal in all the successful trajectories.

    \item \textbf{False Positive Rate (FPR)} - The ratio between the number of false positive predictions (i.e., actually untraversable/non-pliable obstacles predicted as traversable) and the total number of actual negative (untraversable) obstacles encountered during a trial. We report the average over all the trials.
\end{itemize}

\no If the sum of the success and freezing rates do not equal 100, it indicates that the robot has collided in those cases. We also compare these methods qualitatively using the trajectories pursued while navigating. For perception comparisons, we quantitatively evaluate the accuracy and F-score of MobileNetv3 \cite{mobilenetv3}, EfficientNet \cite{tan2019efficientnet}, and Vision Transformer \cite{vision_transformer} when trained and evaluated on our dataset. We also show VERN's classification results and cost map clearing in Fig. \ref{fig:costmap_comparisons}.

% \no \textbf{Success Rate} - The number of successful goal-reaching attempts (while avoiding non-pliable vegetation and collisions) over the total number of trials.

% \no \textbf{Freezing Rate} - The number of times the robot got stuck or started oscillating for more than 10 seconds, while avoiding obstacles over the total number of attempts.

% \no \textbf{Normalized Trajectory Length} - The ratio between the robot's trajectory length and the straight-line distance to the goal in both successful and unsuccessful trajectories. 

% \no \textbf{False Positive Rate (FPR)} - The ratio between the number of false positive predictions (i.e., actually untraversable trajectories predicted as traversable) and the total number of actual negative (untraversable) samples encountered during a trial. We report the average over all the trials. 

% \no \textbf{False Negative Rate  (FNR)} - The ratio between the number of false negative predictions (i.e., actually traversable trajectories as untraversable) and the total number of actual positive samples encountered during a trial. We report the average over all the trials. 

\vspace{-2pt}
\subsection{Navigation Test Scenarios}
We compare the performance of all the navigation methods in the following real-world outdoor scenarios (see Figs. \ref{fig:cover_image} and \ref{fig:navigation_comparisons}) that differ from the training environments. The scenarios are described in increasing order of difficulty. Ten trials are conducted in each scenario for each method.

\begin{itemize}
\item \textbf{Scenario 1} - Contains sparse tall grass, and trees.

\item \textbf{Scenario 2} - Contains trees and dense tall grass in close proximity. The robot must identify grass and pass through it to reach its goal successfully.

\item \textbf{Scenario 3} - Contains trees, bushes, and dense grass in close proximity. The robot must identify grass and pass through it. See Fig. \ref{fig:cover_image} 

\item \textbf{Scenario 4} - Contains dense grass, trees, and an obstacle (human) unseen during training. 
\end{itemize}

\begin{table}[t]
\resizebox{\columnwidth}{!}{%
\begin{tabular}{ |c |c |c |c |c |c |} 
\hline
\textbf{Metrics} & \textbf{Method} & \multicolumn{1}{|p{1cm}|}{\centering \textbf{Scenario} \\ \textbf{1}} & \multicolumn{1}{|p{1cm}|}{\centering \textbf{Scenario} \\ \textbf{2}} & \multicolumn{1}{|p{1cm}|}{\centering \textbf{Scenario} \\ \textbf{3}} & \multicolumn{1}{|p{1cm}|}{\centering \textbf{Scenario} \\ \textbf{4}}\\ [0.5ex] 
\hline
\multirow{6}{*}{\rotatebox[origin=c]{0}{\makecell{\textbf{Success}\\\textbf{Rate (\%)} \\ (Higher is \\ better)}}} 
& Spot's Inbuilt Planner & 50 & 0 & 0 & 0   \\
 & DWA \cite{fox1997dwa} & 80 & 0 & 0 & 0   \\
 & GA-Nav\cite{ganav}    & 60 & 20 & 30 & 30 \\
 & GrASPE\cite{graspe}   & 80 & 60 & 50 & 40 \\
 & VERN w/o height estimation & 80 & 60 & 40 & 30 \\
 & VERN w/o recovery behavior & 100 & 60 & 40 & 40 \\
 & VERN(ours) & \textbf{100} & \textbf{90} & \textbf{70} & \textbf{70} \\
\hline

\multirow{6}{*}{\rotatebox[origin=c]{0}{\makecell{\textbf{Freezing}\\\textbf{Rate (\%)} \\ (Lower is \\ better)}}} 
& Spot's Inbuilt Planner & 30 & 100 & 100 & 90   \\
 & DWA \cite{fox1997dwa} & 20 & 100 & 100 & 100   \\
 & GA-Nav\cite{ganav}    & 40 & 80 & 70 & 70 \\
 & GrASPE\cite{graspe}   & 10 & 20 & 20 & 50 \\
 & VERN w/o height estimation & 20 & 40 & 40 & 70 \\
 & VERN w/o recovery behavior & 0 & 40 & 60 & 60 \\
 & VERN(ours) & \textbf{0} & \textbf{10} & \textbf{20} & \textbf{30} \\
\hline


\multirow{6}{*}{\rotatebox[origin=c]{0}{\makecell{\textbf{Norm.}\\\textbf{Traj.}\\\textbf{Length}\\ (Closer to 1 \\ is better)}}} 
& Spot's Inbuilt Planner & 1.23 & 0.37 & 0.24 & 0.22   \\
 & DWA \cite{fox1997dwa}  & 1.52 & 0.46 & 0.34 & 0.62   \\
 & GA-Nav\cite{ganav} & 1.31 & 1.39 & 1.32 & 1.49 \\
 & GrASPE\cite{graspe} & 1.18 & 1.09 & 1.46 & 1.37 \\
 & VERN w/o height estimation & 1.39 & 1.41 & 1.35 & 1.30 \\
 & VERN w/o recovery behavior & 1.42 & 1.48 & 1.46 & 1.39 \\
 & VERN(ours) & 1.11 & 1.19 & 1.28 & 1.23 \\
\hline

\multirow{4}{*}{\rotatebox[origin=c]{0}{\makecell{\textbf{ False}\\\textbf{Positive }\\\textbf{Rate}}}} 
% & Spot's Inbuilt Planner & - & - & - & -   \\
% & DWA \cite{fox1997dwa} & - & - & - & -   \\
 & GA-Nav\cite{ganav} & 0.33 & 0.39 & 0.30 & 0.61 \\
 & GrASPE\cite{graspe} & 0.25 & 0.31 & 0.28 & 0.44 \\
  & EfficientNet\cite{tan2019efficientnet}& 0.28 & 0.23 & 0.32& 0.35 \\
  % & VERN w/o height estimation & 0.21 & 0.29 & 0.32 & 0.42 \\
 & VERN(ours) &  \textbf{0.15} &  \textbf{0.18} &  \textbf{0.12} &  \textbf{0.21} \\
\hline


\end{tabular}
}
\caption{\small{Navigation performance of our method compared to other methods on various metrics. VERN outperforms other methods consistently in terms of success rate, freezing rate, false positive rate, and normalized trajectory length in different unstructured outdoor vegetation scenarios.}
}
\label{tab:comparison_table}
\vspace{-10pt}
\end{table}

\begin{table}[t]
\centering
\resizebox{0.75\columnwidth}{!}{
\begin{tabular}{|c|c|c|} 
\hline
\textbf{Methods}\Tstrut \Tstrut & \textbf{Accuracy} & \textbf{F-score} \\ [0.5ex] 
\hline
MobileNetv3 \cite{mobilenetv3} & \textbf{0.957} & \textbf{0.833} \\
\hline
EfficientNet \cite{tan2019efficientnet} & 0.758 & 0.632 \\
\hline
Vision Transformer (ViT) \cite{vision_transformer} & 0.689 & 0.546 \\
\hline
\end{tabular}}

\caption{ \small{\label{Tab:Results2} The accuracy and F-scores (higher values are better) of three feature-extracting backbones used to train our dataset. We observe that our MobileNetv3 has the best accuracy and F-score because of its depth-wise separable convolutions. This leads to faster and better learning.}}
\vspace{-15pt}
\end{table}

\subsection{Analysis and Discussion}
\textbf{Classification Accuracy:} We observe from Table \ref{Tab:Results2} that MobileNetv3 has the best accuracy and F-score compared to other methods. EfficientNet is difficult to train and tune especially due to longer training time requirements. Similarly, ViT requires a lot more data for training and is not compatible with the siamese network framework. 

\textbf{Navigation Comparison:} The quantitative navigation results are presented in Table \ref{tab:comparison_table}. We observe that VERN's (with height estimation used in equation \ref{eqn:clearing} and recovery behaviors) success rates are significantly higher compared to the other methods. Spot's in-built planner and DWA consider all vegetation as obstacles. This leads to Spot becoming unstable or crashing when using its in-built planner and leads to excessive freezing and oscillations when using DWA. 

GA-Nav considers all vegetation (except trees) as partially traversable because it segments most flora as grass. It also cannot differentiate vegetation due to the lack of such precisely human-annotated datasets. Additionally, in cases where vegetation like tall grass partially occludes the RGB image (see scenario 2 in Fig. \ref{fig:navigation_comparisons} [center]), GA-Nav struggles to produce segmentation results leading to the robot freezing or oscillating. In regions with tall grass, GA-Nav's elevation maps also become erroneous.

GrASPE performs the best out of all the existing methods. It is capable of passing through sparse grass and avoiding untraversable obstacles. However, in highly dense grass that is in close proximity to trees, GrASPE is unable to accurately detect and react quickly to avoid collisions with trees.

Since some existing methods do not reach the goal even once, their trajectory lengths are less than 1. The values are still reported to give a measure of their progress toward the goal. Spot's in-built planner progress the least before the robot froze or crashed. In some cases, the methods take meandering trajectories (e.g. DWA) in scenario 1 before reaching the goal. Notably, VERN deviates the least from the robot's goal and that reflects in the low trajectory length. 

\textbf{FPR}: We compare VERN's vegetation classifier's false positive rate (FPR) with GA-Nav and the EfficientNet-based classifier using manual ground truth labeling of the vegetation in the trials. We observe that GA-Nav leads to a significantly high FPR in all four scenarios primarily because GA-Nav's terrain segmentation predictions are trained on the RUGD dataset. Moreover, GA-Nav's incorrect segmentation under varying lighting conditions and occlusions increases the false positive predictions. GrASPE and the EfficientNet-based classifier are trained using the same images we used to train VERN. However, we observe that their accuracy is comparatively lower than our VERN model. This is primarily due to the fine-grained feature learning capabilities of VERN's MobileNetv3-based backbone. Additionally, GrASPE's predictions lead to erroneous results when its 3D point cloud cannot identify the geometry of the vegetation such as trees and bushes under visually cluttered instances.  

\textbf{Ablation study}: We compared VERN and its variants without using height estimation, and recovery behaviors. We observe that when the height estimation is not used for cost map clearing, the robot's success rate drops. This is mainly because height helps differentiate short and tall pliable vegetation (where the robot could freeze). Additionally, estimating the critical regions based on the height helps avoid obstacles such as humans (Scenario 4) which are not part of our classifier. 

Using recovery behaviors helps reduce freezing in the presence of dense obstacles (especially in scenarios 2, 3, and 4). Additionally, moving the robot to a safe location, and marking and remembering the unsafe region allows the robot to preemptively avoid the region in subsequent trials. 



\section{Conclusions, Limitations and Future Work}

We present a novel algorithm to navigate a legged robot in densely vegetated environments with various traversability. We utilize a few-shot learning classifier trained on a few hundred quadrants of RGB images to distinguish vegetation with different pliability with minimal human annotation. This classification model is combined with estimated vegetation height (from multiple local cost maps), and classification confidence using a novel cost map clearing scheme. Using the resulting vegetation-aware cost map, we deploy a local planner with recovery behaviors to save the robot if it freezes or gets entrapped in dense vegetation.  

Our algorithm has a few limitations. Our primary navigation assumes non-holonomic robot dynamics to utilize DWA's dynamic feasibility guarantees. However, the legged robot dynamics are holonomic and further investigation is required to extend our method to relax its action constraints. We assume that the different kinds of vegetation are not intertwined with one another. This may not be the case in highly forested environments. VERN could lead to collisions in the presence of thin obstacles such as branches that are not detected in the cost maps. In the future, we would like to augment our current method with proprioceptive sensing for navigating more complex terrains with occluded surfaces.
% \section{Acknowledgement}

This work was supported by the U.S. Army Combat Capabilities Development Command (DEVCOM) Army Research Laboratory (ARL) Artificial Intelligence and Autonomy for Multi-Agent Systems (ArtIAMAS) program through a cooperative agreement (https://artiamas.umd.edu) and is being partially funded by DEVCOM ARL Grant No. W911NF2120076.  The US Government is authorized to reproduce and distribute reprints for Government purposes notwithstanding any copyright notation herein.  The views and conclusions contained in this document are those of the authors and should not be interpreted as representing the official policies, either expressed or implied, of the DEVCOM ARL or US Government.

\bibliographystyle{IEEEtran}
\bibliography{References}

\end{document}
