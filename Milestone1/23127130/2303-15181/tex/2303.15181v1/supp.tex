
%% bare_jrnl_compsoc.tex
%% V1.4b
%% 2015/08/26
%% by Michael Shell
%% See:
%% http://www.michaelshell.org/
%% for current contact information.
%%
%% This is a skeleton file demonstrating the use of IEEEtran.cls
%% (requires IEEEtran.cls version 1.8b or later) with an IEEE
%% Computer Society journal paper.
%%
%% Support sites:
%% http://www.michaelshell.org/tex/ieeetran/
%% http://www.ctan.org/pkg/ieeetran
%% and
%% http://www.ieee.org/

%%*************************************************************************
%% Legal Notice:
%% This code is offered as-is without any warranty either expressed or
%% implied; without even the implied warranty of MERCHANTABILITY or
%% FITNESS FOR A PARTICULAR PURPOSE! 
%% User assumes all risk.
%% In no event shall the IEEE or any contributor to this code be liable for
%% any damages or losses, including, but not limited to, incidental,
%% consequential, or any other damages, resulting from the use or misuse
%% of any information contained here.
%%
%% All comments are the opinions of their respective authors and are not
%% necessarily endorsed by the IEEE.
%%
%% This work is distributed under the LaTeX Project Public License (LPPL)
%% ( http://www.latex-project.org/ ) version 1.3, and may be freely used,
%% distributed and modified. A copy of the LPPL, version 1.3, is included
%% in the base LaTeX documentation of all distributions of LaTeX released
%% 2003/12/01 or later.
%% Retain all contribution notices and credits.
%% ** Modified files should be clearly indicated as such, including  **
%% ** renaming them and changing author support contact information. **
%%*************************************************************************


% *** Authors should verify (and, if needed, correct) their LaTeX system  ***
% *** with the testflow diagnostic prior to trusting their LaTeX platform ***
% *** with production work. The IEEE's font choices and paper sizes can   ***
% *** trigger bugs that do not appear when using other class files.       ***                          ***
% The testflow support page is at:
% http://www.michaelshell.org/tex/testflow/


\documentclass[10pt,journal,compsoc]{IEEEtran}
%
% If IEEEtran.cls has not been installed into the LaTeX system files,
% manually specify the path to it like:
% \documentclass[10pt,journal,compsoc]{../sty/IEEEtran}





% Some very useful LaTeX packages include:
% (uncomment the ones you want to load)


% *** MISC UTILITY PACKAGES ***
%
%\usepackage{ifpdf}
% Heiko Oberdiek's ifpdf.sty is very useful if you need conditional
% compilation based on whether the output is pdf or dvi.
% usage:
% \ifpdf
%   % pdf code
% \else
%   % dvi code
% \fi
% The latest version of ifpdf.sty can be obtained from:
% http://www.ctan.org/pkg/ifpdf
% Also, note that IEEEtran.cls V1.7 and later provides a builtin
% \ifCLASSINFOpdf conditional that works the same way.
% When switching from latex to pdflatex and vice-versa, the compiler may
% have to be run twice to clear warning/error messages.



% to avoid loading the natbib package, add option nonatbib:
%    \usepackage[nonatbib]{neurips_2022}
\newcommand{\xjqi}[1]{{\color{orange}{\bf\sf [xjqi: #1]}}}
\newcommand{\lzz}[1]{{\color{cyan}{\bf\sf [lzz: #1]}}}
\newcommand{\phil}[1]{{\color{red}{[ph: #1]}}}
%\newcommand{\phil}[1]{{\color{blue}{\bf\sf [Ph: #1]}}}
\newcommand{\pdai}[1]{{\color{green}{\bf\sf [dp: #1]}}}
\newcommand{\rh}[1]{{\color{cyan}{\bf\sf [RH: #1]}}}
\newcommand{\etal}{{\textit{et al.}}}
\newcommand{\ie}{\textit{i.e.}}
\newcommand{\eg}{\textit{e.g.}}
\newcommand{\vs}{\textit{vs.}}

\usepackage[utf8]{inputenc} % allow utf-8 input
\usepackage[T1]{fontenc}    % use 8-bit T1 fonts
\usepackage{hyperref}       % hyperlinks
\usepackage{url}            % simple URL typesetting
\usepackage{booktabs}       % professional-quality tables
\usepackage{amsfonts}       % blackboard math symbols
\usepackage{nicefrac}       % compact symbols for 1/2, etc.
\usepackage{microtype}      % microtypography
\usepackage{xcolor}         % colors
\usepackage[T1]{fontenc}
\usepackage[linesnumbered,boxed,ruled,commentsnumbered]{algorithm2e}
\usepackage{times}
\usepackage{stfloats}
\usepackage{epsfig}
\usepackage{graphicx}
\usepackage{amsmath}
\usepackage{amssymb}
\usepackage[utf8]{inputenc} % allow utf-8 input
\usepackage[T1]{fontenc}    % use 8-bit T1 fonts
%\usepackage{hyperref}       % hyperlinks
\usepackage{url}            % simple URL typesetting
\usepackage{booktabs}       % professional-quality tables
\usepackage{amsfonts}       % blackboard math symbols
\usepackage{nicefrac}       % compact symbols for 1/2, etc.
\usepackage{microtype}      % microtypography
\usepackage{graphicx}
\usepackage{multirow}
\usepackage{wrapfig}
\usepackage{xcolor}
\usepackage{bbm}
\usepackage{amsmath}
\usepackage{comment}
\usepackage{dsfont}
\usepackage{bbm}
\usepackage{wrapfig}
\usepackage{float}
\usepackage{enumitem}
\usepackage{amssymb}


% *** CITATION PACKAGES ***
%
\ifCLASSOPTIONcompsoc
  % IEEE Computer Society needs nocompress option
  % requires cite.sty v4.0 or later (November 2003)
  \usepackage[nocompress]{cite}
\else
  % normal IEEE
  \usepackage{cite}
\fi
% cite.sty was written by Donald Arseneau
% V1.6 and later of IEEEtran pre-defines the format of the cite.sty package
% \cite{} output to follow that of the IEEE. Loading the cite package will
% result in citation numbers being automatically sorted and properly
% "compressed/ranged". e.g., [1], [9], [2], [7], [5], [6] without using
% cite.sty will become [1], [2], [5]--[7], [9] using cite.sty. cite.sty's
% \cite will automatically add leading space, if needed. Use cite.sty's
% noadjust option (cite.sty V3.8 and later) if you want to turn this off
% such as if a citation ever needs to be enclosed in parenthesis.
% cite.sty is already installed on most LaTeX systems. Be sure and use
% version 5.0 (2009-03-20) and later if using hyperref.sty.
% The latest version can be obtained at:
% http://www.ctan.org/pkg/cite
% The documentation is contained in the cite.sty file itself.
%
% Note that some packages require special options to format as the Computer
% Society requires. In particular, Computer Society  papers do not use
% compressed citation ranges as is done in typical IEEE papers
% (e.g., [1]-[4]). Instead, they list every citation separately in order
% (e.g., [1], [2], [3], [4]). To get the latter we need to load the cite
% package with the nocompress option which is supported by cite.sty v4.0
% and later. Note also the use of a CLASSOPTION conditional provided by
% IEEEtran.cls V1.7 and later.





% *** GRAPHICS RELATED PACKAGES ***
%
\ifCLASSINFOpdf
  % \usepackage[pdftex]{graphicx}
  % declare the path(s) where your graphic files are
  % \graphicspath{{../pdf/}{../jpeg/}}
  % and their extensions so you won't have to specify these with
  % every instance of \includegraphics
  % \DeclareGraphicsExtensions{.pdf,.jpeg,.png}
\else
  % or other class option (dvipsone, dvipdf, if not using dvips). graphicx
  % will default to the driver specified in the system graphics.cfg if no
  % driver is specified.
  % \usepackage[dvips]{graphicx}
  % declare the path(s) where your graphic files are
  % \graphicspath{{../eps/}}
  % and their extensions so you won't have to specify these with
  % every instance of \includegraphics
  % \DeclareGraphicsExtensions{.eps}
\fi
% graphicx was written by David Carlisle and Sebastian Rahtz. It is
% required if you want graphics, photos, etc. graphicx.sty is already
% installed on most LaTeX systems. The latest version and documentation
% can be obtained at: 
% http://www.ctan.org/pkg/graphicx
% Another good source of documentation is "Using Imported Graphics in
% LaTeX2e" by Keith Reckdahl which can be found at:
% http://www.ctan.org/pkg/epslatex
%
% latex, and pdflatex in dvi mode, support graphics in encapsulated
% postscript (.eps) format. pdflatex in pdf mode supports graphics
% in .pdf, .jpeg, .png and .mps (metapost) formats. Users should ensure
% that all non-photo figures use a vector format (.eps, .pdf, .mps) and
% not a bitmapped formats (.jpeg, .png). The IEEE frowns on bitmapped formats
% which can result in "jaggedy"/blurry rendering of lines and letters as
% well as large increases in file sizes.
%
% You can find documentation about the pdfTeX application at:
% http://www.tug.org/applications/pdftex






% *** MATH PACKAGES ***
%
%\usepackage{amsmath}
% A popular package from the American Mathematical Society that provides
% many useful and powerful commands for dealing with mathematics.
%
% Note that the amsmath package sets \interdisplaylinepenalty to 10000
% thus preventing page breaks from occurring within multiline equations. Use:
%\interdisplaylinepenalty=2500
% after loading amsmath to restore such page breaks as IEEEtran.cls normally
% does. amsmath.sty is already installed on most LaTeX systems. The latest
% version and documentation can be obtained at:
% http://www.ctan.org/pkg/amsmath





% *** SPECIALIZED LIST PACKAGES ***
%
%\usepackage{algorithmic}
% algorithmic.sty was written by Peter Williams and Rogerio Brito.
% This package provides an algorithmic environment fo describing algorithms.
% You can use the algorithmic environment in-text or within a figure
% environment to provide for a floating algorithm. Do NOT use the algorithm
% floating environment provided by algorithm.sty (by the same authors) or
% algorithm2e.sty (by Christophe Fiorio) as the IEEE does not use dedicated
% algorithm float types and packages that provide these will not provide
% correct IEEE style captions. The latest version and documentation of
% algorithmic.sty can be obtained at:
% http://www.ctan.org/pkg/algorithms
% Also of interest may be the (relatively newer and more customizable)
% algorithmicx.sty package by Szasz Janos:
% http://www.ctan.org/pkg/algorithmicx




% *** ALIGNMENT PACKAGES ***
%
%\usepackage{array}
% Frank Mittelbach's and David Carlisle's array.sty patches and improves
% the standard LaTeX2e array and tabular environments to provide better
% appearance and additional user controls. As the default LaTeX2e table
% generation code is lacking to the point of almost being broken with
% respect to the quality of the end results, all users are strongly
% advised to use an enhanced (at the very least that provided by array.sty)
% set of table tools. array.sty is already installed on most systems. The
% latest version and documentation can be obtained at:
% http://www.ctan.org/pkg/array


% IEEEtran contains the IEEEeqnarray family of commands that can be used to
% generate multiline equations as well as matrices, tables, etc., of high
% quality.




% *** SUBFIGURE PACKAGES ***
%\ifCLASSOPTIONcompsoc
%  \usepackage[caption=false,font=footnotesize,labelfont=sf,textfont=sf]{subfig}
%\else
%  \usepackage[caption=false,font=footnotesize]{subfig}
%\fi
% subfig.sty, written by Steven Douglas Cochran, is the modern replacement
% for subfigure.sty, the latter of which is no longer maintained and is
% incompatible with some LaTeX packages including fixltx2e. However,
% subfig.sty requires and automatically loads Axel Sommerfeldt's caption.sty
% which will override IEEEtran.cls' handling of captions and this will result
% in non-IEEE style figure/table captions. To prevent this problem, be sure
% and invoke subfig.sty's "caption=false" package option (available since
% subfig.sty version 1.3, 2005/06/28) as this is will preserve IEEEtran.cls
% handling of captions.
% Note that the Computer Society format requires a sans serif font rather
% than the serif font used in traditional IEEE formatting and thus the need
% to invoke different subfig.sty package options depending on whether
% compsoc mode has been enabled.
%
% The latest version and documentation of subfig.sty can be obtained at:
% http://www.ctan.org/pkg/subfig




% *** FLOAT PACKAGES ***
%
%\usepackage{fixltx2e}
% fixltx2e, the successor to the earlier fix2col.sty, was written by
% Frank Mittelbach and David Carlisle. This package corrects a few problems
% in the LaTeX2e kernel, the most notable of which is that in current
% LaTeX2e releases, the ordering of single and double column floats is not
% guaranteed to be preserved. Thus, an unpatched LaTeX2e can allow a
% single column figure to be placed prior to an earlier double column
% figure.
% Be aware that LaTeX2e kernels dated 2015 and later have fixltx2e.sty's
% corrections already built into the system in which case a warning will
% be issued if an attempt is made to load fixltx2e.sty as it is no longer
% needed.
% The latest version and documentation can be found at:
% http://www.ctan.org/pkg/fixltx2e


%\usepackage{stfloats}
% stfloats.sty was written by Sigitas Tolusis. This package gives LaTeX2e
% the ability to do double column floats at the bottom of the page as well
% as the top. (e.g., "\begin{figure*}[!b]" is not normally possible in
% LaTeX2e). It also provides a command:
%\fnbelowfloat
% to enable the placement of footnotes below bottom floats (the standard
% LaTeX2e kernel puts them above bottom floats). This is an invasive package
% which rewrites many portions of the LaTeX2e float routines. It may not work
% with other packages that modify the LaTeX2e float routines. The latest
% version and documentation can be obtained at:
% http://www.ctan.org/pkg/stfloats
% Do not use the stfloats baselinefloat ability as the IEEE does not allow
% \baselineskip to stretch. Authors submitting work to the IEEE should note
% that the IEEE rarely uses double column equations and that authors should try
% to avoid such use. Do not be tempted to use the cuted.sty or midfloat.sty
% packages (also by Sigitas Tolusis) as the IEEE does not format its papers in
% such ways.
% Do not attempt to use stfloats with fixltx2e as they are incompatible.
% Instead, use Morten Hogholm'a dblfloatfix which combines the features
% of both fixltx2e and stfloats:
%
% \usepackage{dblfloatfix}
% The latest version can be found at:
% http://www.ctan.org/pkg/dblfloatfix




%\ifCLASSOPTIONcaptionsoff
%  \usepackage[nomarkers]{endfloat}
% \let\MYoriglatexcaption\caption
% \renewcommand{\caption}[2][\relax]{\MYoriglatexcaption[#2]{#2}}
%\fi
% endfloat.sty was written by James Darrell McCauley, Jeff Goldberg and 
% Axel Sommerfeldt. This package may be useful when used in conjunction with 
% IEEEtran.cls'  captionsoff option. Some IEEE journals/societies require that
% submissions have lists of figures/tables at the end of the paper and that
% figures/tables without any captions are placed on a page by themselves at
% the end of the document. If needed, the draftcls IEEEtran class option or
% \CLASSINPUTbaselinestretch interface can be used to increase the line
% spacing as well. Be sure and use the nomarkers option of endfloat to
% prevent endfloat from "marking" where the figures would have been placed
% in the text. The two hack lines of code above are a slight modification of
% that suggested by in the endfloat docs (section 8.4.1) to ensure that
% the full captions always appear in the list of figures/tables - even if
% the user used the short optional argument of \caption[]{}.
% IEEE papers do not typically make use of \caption[]'s optional argument,
% so this should not be an issue. A similar trick can be used to disable
% captions of packages such as subfig.sty that lack options to turn off
% the subcaptions:
% For subfig.sty:
% \let\MYorigsubfloat\subfloat
% \renewcommand{\subfloat}[2][\relax]{\MYorigsubfloat[]{#2}}
% However, the above trick will not work if both optional arguments of
% the \subfloat command are used. Furthermore, there needs to be a
% description of each subfigure *somewhere* and endfloat does not add
% subfigure captions to its list of figures. Thus, the best approach is to
% avoid the use of subfigure captions (many IEEE journals avoid them anyway)
% and instead reference/explain all the subfigures within the main caption.
% The latest version of endfloat.sty and its documentation can obtained at:
% http://www.ctan.org/pkg/endfloat
%
% The IEEEtran \ifCLASSOPTIONcaptionsoff conditional can also be used
% later in the document, say, to conditionally put the References on a 
% page by themselves.




% *** PDF, URL AND HYPERLINK PACKAGES ***
%
%\usepackage{url}
% url.sty was written by Donald Arseneau. It provides better support for
% handling and breaking URLs. url.sty is already installed on most LaTeX
% systems. The latest version and documentation can be obtained at:
% http://www.ctan.org/pkg/url
% Basically, \url{my_url_here}.





% *** Do not adjust lengths that control margins, column widths, etc. ***
% *** Do not use packages that alter fonts (such as pslatex).         ***
% There should be no need to do such things with IEEEtran.cls V1.6 and later.
% (Unless specifically asked to do so by the journal or conference you plan
% to submit to, of course. )


% correct bad hyphenation here
\hyphenation{op-tical net-works semi-conduc-tor}


\begin{document}
%
% paper title
% Titles are generally capitalized except for words such as a, an, and, as,
% at, but, by, for, in, nor, of, on, or, the, to and up, which are usually
% not capitalized unless they are the first or last word of the title.
% Linebreaks \\ can be used within to get better formatting as desired.
% Do not put math or special symbols in the title.
\title{ISS++: Image as Stepping Stone for\\ Text-Guided 3D Shape Generation: Supplementary Material}
%
%
% author names and IEEE memberships
% note positions of commas and nonbreaking spaces ( ~ ) LaTeX will not break
% a structure at a ~ so this keeps an author's name from being broken across
% two lines.
% use \thanks{} to gain access to the first footnote area
% a separate \thanks must be used for each subsection as LaTeX2e's \thanks
% was not built to handle multiple subsections
%
%
%\IEEEcompsocitemizethanks is a special \thanks that produces the bulleted
% lists the Computer Society journals use for "first footnote" author
% affiliations. Use \IEEEcompsocthanksitem which works much like \item
% for each affiliation group. When not in compsoc mode,
% \IEEEcompsocitemizethanks becomes like \thanks and
% \IEEEcompsocthanksitem becomes a line break with idention. This
% facilitates dual compilation, although admittedly the differences in the
% desired content of \author between the different types of papers makes a
% one-size-fits-all approach a daunting prospect. For instance, compsoc 
% journal papers have the author affiliations above the "Manuscript
% received ..."  text while in non-compsoc journals this is reversed. Sigh.


\author{Zhengzhe Liu,
        Peng Dai,
        Ruihui Li, Xiaojuan Qi, Chi-Wing Fu% <-this % stops a space
\IEEEcompsocitemizethanks{\IEEEcompsocthanksitem Z. Liu and C. Fu are with the Department
of Computer Science and Engineering, The Chinese University of Hong Kong, Hong Kong.\protect\\
% note need leading \protect in front of \\ to get a newline within \thanks as
% \\ is fragile and will error, could use \hfil\break instead.
E-mail: zzliu@cse.cuhk.edu.hk; cwfu@cse.cuhk.edu.hk
\IEEEcompsocthanksitem P. Dai and X. Qi are with the Department of Electrical and Electronic Engineering, The University of Hong Kong, Hong Kong.% <-this % stops an unwanted space
\IEEEcompsocthanksitem R. Li is with College of Computer Science and Electronic Engineering, The Hunan University, China.}
%\thanks{Manuscript received April 19, 2005; revised August 26, 2015.}
}

% note the % following the last \IEEEmembership and also \thanks - 
% these prevent an unwanted space from occurring between the last author name
% and the end of the author line. i.e., if you had this:
% 
% \author{....lastname \thanks{...} \thanks{...} }
%                     ^------------^------------^----Do not want these spaces!
%
% a space would be appended to the last name and could cause every name on that
% line to be shifted left slightly. This is one of those "LaTeX things". For
% instance, "\textbf{A} \textbf{B}" will typeset as "A B" not "AB". To get
% "AB" then you have to do: "\textbf{A}\textbf{B}"
% \thanks is no different in this regard, so shield the last } of each \thanks
% that ends a line with a % and do not let a space in before the next \thanks.
% Spaces after \IEEEmembership other than the last one are OK (and needed) as
% you are supposed to have spaces between the names. For what it is worth,
% this is a minor point as most people would not even notice if the said evil
% space somehow managed to creep in.



% The paper headers
\markboth{IEEE TRANSACTIONS ON PATTERN ANALYSIS AND MACHINE INTELLIGENCE}%
{Shell \MakeLowercase{\textit{et al.}}: Bare Demo of IEEEtran.cls for Computer Society Journals}
% The only time the second header will appear is for the odd numbered pages
% after the title page when using the twoside option.
% 
% *** Note that you probably will NOT want to include the author's ***
% *** name in the headers of peer review papers.                   ***
% You can use \ifCLASSOPTIONpeerreview for conditional compilation here if
% you desire.



% The publisher's ID mark at the bottom of the page is less important with
% Computer Society journal papers as those publications place the marks
% outside of the main text columns and, therefore, unlike regular IEEE
% journals, the available text space is not reduced by their presence.
% If you want to put a publisher's ID mark on the page you can do it like
% this:
%\IEEEpubid{0000--0000/00\$00.00~\copyright~2015 IEEE}
% or like this to get the Computer Society new two part style.
%\IEEEpubid{\makebox[\columnwidth]{\hfill 0000--0000/00/\$00.00~\copyright~2015 IEEE}%
%\hspace{\columnsep}\makebox[\columnwidth]{Published by the IEEE Computer Society\hfill}}
% Remember, if you use this you must call \IEEEpubidadjcol in the second
% column for its text to clear the IEEEpubid mark (Computer Society jorunal
% papers don't need this extra clearance.)



% use for special paper notices
%\IEEEspecialpapernotice{(Invited Paper)}



% for Computer Society papers, we must declare the abstract and index terms
% PRIOR to the title within the \IEEEtitleabstractindextext IEEEtran
% command as these need to go into the title area created by \maketitle.
% As a general rule, do not put math, special symbols or citations
% in the abstract or keywords.



% make the title area
\maketitle


% To allow for easy dual compilation without having to reenter the
% abstract/keywords data, the \IEEEtitleabstractindextext text will
% not be used in maketitle, but will appear (i.e., to be "transported")
% here as \IEEEdisplaynontitleabstractindextext when the compsoc 
% or transmag modes are not selected <OR> if conference mode is selected 
% - because all conference papers position the abstract like regular
% papers do.
\IEEEdisplaynontitleabstractindextext
% \IEEEdisplaynontitleabstractindextext has no effect when using
% compsoc or transmag under a non-conference mode.



% For peer review papers, you can put extra information on the cover
% page as needed:
% \ifCLASSOPTIONpeerreview
% \begin{center} \bfseries EDICS Category: 3-BBND \end{center}
% \fi
%
% For peerreview papers, this IEEEtran command inserts a page break and
% creates the second title. It will be ignored for other modes.
\IEEEpeerreviewmaketitle


In this supplementary material, we first provide more detailed results of human perceptual evaluation results (Section~\ref{sec:human}). Then we provide our text sets (Section~\ref{sec:textset}) and more detailed results on the feature space alignment (Section~\ref{sec:alignment}). Finally, we summarize the notations in the paper (Section~\ref{sec:notations}). 

\section{Detailed results on the human perceptual evaluation}\label{sec:human}

For each method, we gather the evaulation scores from all participants and obtain the ``Consistency Score'' as $s/n$, where $s$ is the total score and $n$ is the number of samples.  
\begin{table*}
\centering
\caption{Total scores $s$ from the ten volunteers out of 52 generated shapes.}
\label{tab:consistent}
\scalebox{1}{
  \begin{tabular}{ccccccccccccc}
    \toprule
     Method &  1 & 2 & 3 & 4 &5 & 6 & 7& 8& 9 & 10 & mean $\pm$ std & Consistency Score (\%) $\uparrow$ \\
    \midrule
 CLIP-Forge  & 21 &
35 &
34&
16.5&
18.5&
29&
21.5&
4&
21&
17 & 21.75 $\pm$ 9.16  & 41.83 $\pm$ 17.62 \\
     Dearm Fields & 13&
17.5&
7&
19&
6.5&
22&
22&
9&
10&
6 & 13.2 $\pm$ 6.41 & 25.38 $\pm$ 12.33 \\
    \midrule
 $E_{\text{I}}$+$D$ & 14&
9.5&
5&
18&
4.5&
25&
15.5&
4.05&
4.5&
9 &10.91 $\pm$ 7.06 & 20.97 $\pm$ 13.59\\
     w/o stage-1 & 1.5&
1&
0&
1.5&
0&
3.5&
2&
0&
0&
0.5 & 1.00 $\pm$ 1.15 & 1.92 $\pm$ 2.22\\
     w/o stage-2 & 20&
14.5&
8.5&
23.5&
7.5&
27&
19&
8.5&
4.5&
20.5 & 15.35 $\pm$ 7.73 & 29.52 $\pm$ 14.86\\
     w/o $L_{\text{bg}}$ & 17.5&
15.5&
8.5&
21&
8&
27&
26.5&
9.5&
5&
22.5 & 16.1 $\pm$ 8.06 & 30.96 $\pm$ 15.49\\
    \midrule
 GLIDE+DVR & 5&
3.5&
1.5&
10&
2&
13&
6.5&
0.5&
1&
3 &4.60 $\pm$ 4.13 & 8.85 $\pm$ 7.94\\
     LAFITE+DVR & 23&
33.5&
31&
23&
27&
31.5&
27.5&
14.5&
32.5&
27.5 & 27.10 $\pm$ 5.75 & 52.12 $\pm$ 11.05\\
    \midrule
   ISS (ours) & \textbf{32}&
\textbf{37}&
\textbf{35.5}&
\textbf{29.5}&
\textbf{29}&
\textbf{37}&
\textbf{29}&
\textbf{17.5}&
\textbf{32.5}&
\textbf{33} & \textbf{31.20} $\pm$ \textbf{5.69}  & \textbf{60.00} $\pm$ \textbf{10.94}\\
    \bottomrule
  \end{tabular}
  }
\end{table*}  












\begin{table*}
\centering
\caption{A/B/C Test results of the ten volunteers. The numbers in the table indicate the number of shapes from the corresponding method he/she likes most out of the three candidates. Volunteers can optionally select ``pass'' instead of ``A/B/C'' if he cannot decide which one is the best.}
\label{tab:ABC}
\scalebox{1}{
  \begin{tabular}{ccccccccccccc}
    \toprule
    Category & Method & 1 & 2 & 3 & 4 & 5 & 6 & 7 & 8 & 9 & 10 & mean $\pm$ std  $\uparrow$ \\
    \midrule
Existing works &  CLIP-Forge  & 9& 
17& 
12& 
13& 
6& 
9& 
3& 
6& 
6& 
8& 8.9 $\pm$ 4.12 \\
    \midrule
  SOTA Text2Image+SVR  & LAFITE+DVR & 9& 
16& 
9& 
12& 
7& 
13& 
9& 
20& 
8& 
14& 
11.7 $\pm$ 4.11
    \\
    \midrule
    Ours & ISS & \textbf{27}&
\textbf{19}&
\textbf{21}&
\textbf{20}&
\textbf{17}&
\textbf{25}&
\textbf{19}&
\textbf{26}&
\textbf{13}&
\textbf{30}&
\textbf{21.7} $\pm$ \textbf{5.19} \\
    \bottomrule
  \end{tabular}
  }
\end{table*}    
 




\section{Our text sets used in the experiments}\label{sec:textset}


Recent works~\cite{chen2018text2shape,sanghi2021clip,jain2021zero} proposed their own text sets. However, their datasets have some limitations and are not suitable to evaluate our approach. The dataset of Text2shape~\cite{chen2018text2shape} contains text descriptions in only two categories, \ie, Table and Chair; CLIP-Forge~\cite{sanghi2021clip} lacks of descriptions on the color and texture; while Dream Fields~\cite{jain2021zero} utilizes text descriptions containing complex scenes and actions. To fairly evaluate our approach, we propose two text datasets on the ShapeNet~\cite{shapenet2015} and CO3D~\cite{reizenstein2021common} categories, respectively, shown in Tables~\ref{tab:shapenet} and~\ref{tab:co3d}. 



\begin{table*}
\centering
\caption{Texts on ShapeNet~\cite{shapenet2015}. They are utilized to measure FID (Table 1 of the main paper), and employed in Human Perceptual Evaluation (Table~\ref{tab:consistent}) and A/B/C Test(Table~\ref{tab:ABC}). 
}
\label{tab:shapenet}
\scalebox{1}{
  \begin{tabular}{cc}
    \toprule
      a glass single leg circular table &
a wooden double layers table \\
a square metal table&
a round shaped single legged wooden table \\
this is a bar stool with metal arches as a design feature&
a children chair with little legs\\
a swivel chair with wheels&
a red recliner seems confortable\\
a red car&
a green SUV\\
a large black truck&
a long luxury black car\\
army fighter jet&
a black airplane with long white wings\\
a blue airplane with short wings&
boeing 747\\
a big ship for transportation&
a boat with sail\\
a watercraft&
a wooden boat\\
a blue sofa&
sofa with legs\\
a sofa with black backrest&
a small sofa\\
a long brown bench&
a marble bench\\
a metal bench&
concrete bench\\
a military sniper rifle with long barrel&
a rifle with magazines\\
a short rifle&
rifle shotgun\\
a computer monitor&
a display\\
a monitor with square base&
a TV monitor\\
a cabinet with cylindrical legs&
a cupboard\\
a long brown cabinet&
a wardrobe\\
a desk lamp&
bedside lamp\\
lamp supported by a long pillar&
mushroom-like lamp\\
a mobile phone&
a small cell phone\\
a mobile phone with black screen&
an iphone\\
a columnar loudspeaker&
a loudspeaker with metal surface\\
a wooden loudspeaker&
a cylindrical loudspeaker\\
    \bottomrule
  \end{tabular}
  }
\end{table*}  







\begin{table*}
\centering
\caption{Texts on CO3D~\cite{reizenstein2021common}.}
\label{tab:co3d}
\scalebox{1}{
  \begin{tabular}{cccc}
    \toprule
    A big apple & A red apple &
    A green bottle  & A tall cylindrical bottle \\
    A white cup & A wooden cup &
    A large black microwave & A white cuboid microwave \\
    A black skateboard & A green long skateboard &
    A cute toytruck & A large toy truck \\
    A blue backpack & A red big backpack &
    A white bowl& A big wooden bowl \\
    A red round frisbee & A blue large frisbee&
    A big blue motorcycle & A black large wheels motorcycle\\
    A circular stop sign & A triangle stop sign &
    Tv screen & A grey big tv screen \\
    A basketball & A tennis ball &
    A large broccoli  & A green broccoli \\
    A hairdryer & A yellow hairdryer&
    A black mouse & A white mouse\\
    A cuboid big suitcase & A large size tall suitcase &
    A round umbrella  & A big black umbrella\\
    A big banana & A long banana&
    A cream round cake & A chocolate mooncake \\
    A blue handbag & A red big handbag&
    An orange  & A large round orange \\
    A teddybear & A cute teddybear &
    A blue fat vase & A blue tall vase \\
    A black baseball bat & A long wooden baseball bat &
    A blue car & A red car\\
    An egg hotdog & A sausage hotdog&
    A black parkingmeter & A white tall parkingmeter \\
    A black toaster & A round toaster&
    Tall wineglass & Single leg big wineglass\\
    A brown baseball glove & A black big baseball glove &
    A big carrot & A long carrot\\
    A red hydrant & A yellow hydrant&
    A large round pizza & A tomato meat pizza\\
    A white toilet & A fat white toilet&
    A stone bench & A wooden long bench \\
    A gray iphone & A black phone &
    A long black keyboard & A short white keyboard\\
    A short tree & A tall green tree &
    A toy bus & One decker toy bus \\
    A blue bicycle & A black bicycle &
    A blue chair & A wooden chair \\
    A red kite & A long blue kite &
    A TV remote & A long white remote \\
    A book with blue cover & A black book&
    Brown couch & A long brown couch\\
    A open laptop & A black laptop&
    An egg sandwich & A meat sandwich \\
    A cute toy train & A short blue toy train &
    Chocolate donut & Big circular donut \\
    \bottomrule
  \end{tabular}
  }
\end{table*}  


\section{Detailed results on feature space alignment}\label{sec:alignment}


The detailed results of Table 3 in the main paper is presented in Table~\ref{tab:mapping} of this supplementary file. Please refer to Section 4.7 in the main paper for more analysis. 


\begin{table*}
\centering
\caption{Distance changes in the feature space mapping of all the 52 samples in the test set. $d$ means cosine distance. Almost all distances are consistently reduced after our stage-2 alignment.  }
\label{tab:mapping}
\scalebox{0.75}{
  \begin{tabular}{cccccc}
    \toprule
    Text & $d(M(f_I),M(f_T))$&$d(M(f_I), f_S))$&$d(M(f_T), f_S))$&$d(M’(f_T), f_S))$&$d(M(f_T), M’(f_T))$  \\
    \midrule
a glass single leg circular table &0.63&0.31&0.58&0.14&0.34\\
a wooden double layers table & 0.72&0.10&0.64&0.14&0.65 \\
a square metal table&0.76&0.32&0.44&0.21&0.30\\
a round shaped single legged wooden table&0.47&0.24&0.21&0.21&0.30\\
this is a bar stool with metal arches as a design feature&0.43&0.33&0.35&0.18&0.17\\
a children chair with little legs&0.79&0.20&0.43&0.12&0.35\\
a swivel chair with wheels&0.61&0.30&0.38&0.22&0.11\\
a red recliner seems confortable&0.33&0.20&0.23&0.24&0.05\\
a red car&1.02&0.19&0.78&0.10&0.69\\
a green SUV&0.49&0.12&0.24&0.19&0.25\\
a large black truck&0.74&0.19&0.50&0.15&0.53\\
a long luxury black car&0.67&0.22&0.54&0.09&0.54\\
army fighter jet&0.43&0.54&0.43&0.25&0.28\\
a black airplane with long white wings&0.62&0.12&0.32&0.16&0.10\\
a blue airplane with short wings&0.51&0.14&0.55&0.33&0.40\\
boeing 747& 0.35&0.13&0.23&0.06&0.20\\
a big ship for transportation&0.95&0.17&0.72&0.35&0.21\\
a boat with sail&0.39&0.19&0.46&0.12&0.45\\
a watercraft&0.12&0.14&0.27&0.22&0.06\\
a wooden boat&0.76&0.29&0.52&0.12&0.49\\
a blue sofa&0.46&0.20&0.34&0.12&0.27\\
sofa with legs&0.67&0.13&0.55&0.10&0.35\\
a sofa with black backrest&0.59&0.14&0.33&0.08&0.16\\
a small sofa&0.49&0.20&0.33&0.10&0.28\\
a long brown bench&0.57&0.29&0.34&0.18&0.27\\
a marble bench&0.61&0.16&0.46&0.12&0.29\\
a metal bench&0.18&0.29&0.20&0.20&0.09\\
concrete bench&0.48&0.07&0.46&0.11&0.36\\
a military sniper rifle with long barrel&0.47&0.14&0.29&0.07&0.21\\
a rifle with magazines&0.71&0.48&0.51&0.17&0.43\\
a short rifle&0.75&0.15&0.81&0.35&0.52\\
rifle shotgun&0.38&0.13&0.25&0.10&0.09\\
a computer monitor&0.28&0.07&0.22&0.07&0.20\\
a display&0.22&0.18&0.15&0.13&0.02\\
a monitor with square base&0.56&0.17&0.75&0.33&0.34\\
a TV monitor&0.70&0.16&0.41&0.11&0.27\\
a cabinet with cylindrical legs&0.53&0.24&0.27&0.18&0.11\\
a cupboard&0.54&0.37&0.42&0.19&0.45\\
a long brown cabinet&0.57&0.29&0.34&0.18&0.27\\
a wardrobe&0.41&0.13&0.32&0.16&0.26\\
a desk lamp&0.92&0.35&0.51&0.16&0.45\\
bedside lamp&0.71&0.48&0.55&0.17&0.43\\
lamp supported by a long pillar&0.19&0.12&0.18&0.05&0.14\\
mushroom-like lamp&1.20&0.17&1.06&0.32&0.46\\
a mobile phone&0.75&0.09&0.85&0.28&0.55\\
a small cell phone&0.49&0.20&0.33&0.10&0.28\\
a mobile phone with black screen&0.94&0.39&0.61&0.14&0.68\\
an iphone&0.75&0.25&0.54&0.22&0.57\\
a columnar loudspeaker&0.66&0.25&0.54&0.15&0.40\\
a loudspeaker with metal surface&0.34&0.17&0.44&0.16&0.30\\
a wooden loudspeaker&0.72&0.10&0.64&0.14&0.65\\
a cylindrical loudspeaker&1.06&0.19&0.82&0.31&0.27\\
    \midrule
\textbf{mean $\pm$ std}&\textbf{0.58 $\pm$ 0.234}&
\textbf{0.21 $\pm$  0.10}&
 \textbf{0.45  $\pm$  0.20} &
 \textbf{0.17 $\pm$ 0.08}&
 \textbf{0.32 $\pm$ 0.17}\\
    \bottomrule
  \end{tabular}
  }
\end{table*}    


\section{Notations}\label{sec:notations}

In this section, we summarize symbols and notations used in the paper to facilitate readers to follow up. Please refer to Table~\ref{tab:notation} for more details. 

\begin{table}
\centering
\caption{Summary of notations used in paper. 
}

\label{tab:notation}
\scalebox{0.9}{
  \begin{tabular}{cc|cc}
    \toprule
      Notation & Description & Notation & Description  \\
      $E_S$ & encoder of SVR model & $E_I$ &  CLIP image encoder \\
      $E_T$ &CLIP text encoder   & $D$ & Decoder of SVR model \\
      $M$ & Mapper &  $M'$ & Mapper after stage-2 alignment \\
      $D_o$ &occupancy decoder& $D_c$ &color decoder \\
        $R$ & rendered images & $f_I$ & CLIP image feature \\
        $f_T$ & CLIP text feature & $f_S$ & shape feature in SVR model \\
      $o$ & camera center & $p$ & query point \\
      $d$ & cosine distance & $L_M$ &regression loss in stage 1\\
      $L_D$ & original loss in SVR model & $L_{bg}$& background loss\\
      $L_{bg\_1}$& background loss in stage 1 & $L_{bg\_2}$& background loss in stage 2\\
      $L_C$ & CLIP consistency loss & $L_{P}$& 3D prior loss\\
        $\Omega_T$ & CLIP text feature space & $\Omega_S$ & shape feature space from the SVR model \\
        $\Omega_I$ & CLIP image feature space\\
    \bottomrule
  \end{tabular}
  }
\end{table}  


\bibliographystyle{ieee}
\bibliography{egbib}

\end{document}