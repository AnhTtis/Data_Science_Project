\appendix
\onecolumn

\section{Experimental setup}
\label{app:exp_setup}

In this part, we provide some details about our methods. We are performing AutoAttack \cite{croce2020reliable} on the base classifier, then performing our defense method. Similar to previous works such as \cite{schwinn2022improving}. We use $N=30$ steps of TETRA and perform hyperparameter tuning, finding the best step size $\alpha$ and $\gamma$ for every classifier. We use these parameters for all the evaluations, clean images and all of the attacks. We state both $\alpha$ and $\gamma$ for every dataset in \cref{table:cifar10_param,table:cifar100_param,table:imagenet_param}. 

We evaluated the other test-time defense, DRQ \cite{schwinn2022improving}, with the official code using the reported parameters.

\setlength{\tabcolsep}{4pt}
\begin{table}[ht!]
\begin{center}

\begin{tabular}{lccccc}
\hline\noalign{\smallskip}\hline

Method & Architecture & TTM & $\alpha$ & $\gamma$ \\

\hline\noalign{\smallskip}

Madry et al. \cite{madry2017towards} + TETRA & RN50 & $L_{2}, \epsilon=0.5$ & 1.5 & 200\\

\hline\noalign{\smallskip}

Rebuffi et al. \cite{rebuffi2021fixing} + TETRA & WRN28-10 & $L_{2}, \epsilon=0.5$ & 0.5 & 400\\

\hline\noalign{\smallskip}

Rebuffi et al. \cite{rebuffi2021fixing} + TETRA & WRN28-10 & $L_{\infty}, \epsilon=8/255$ & 0.1 & 300\\

\hline\noalign{\smallskip}

Gowal et al. \cite{gowal2020uncovering} + TETRA & WRN70-16 & $L_{\infty}, \epsilon=8/255$ & 0.3 & 300\\

\hline\noalign{\smallskip}

Vanila + TETRA & WRN28-10 & - & 0.05 & 300\\


\hline\noalign{\smallskip}\hline\noalign{\smallskip}
\end{tabular}
\caption{CIFAR10 params. In the first column, we state the method. In the next columns, we state the architecture, the trained threat model (TTM), $\alpha$ which is the step size and $\gamma$ which is the regularization weight.}
\label{table:cifar10_param}
\end{center}
\end{table}


\setlength{\tabcolsep}{4pt}
\begin{table}[ht!]
\begin{center}

\begin{tabular}{lccccc}
\hline\noalign{\smallskip}\hline

Method & Architecture & TTM & $\alpha$ & $\gamma$ \\

\hline\noalign{\smallskip}

Rebuffi et al. \cite{rebuffi2021fixing} + TETRA & WRN28-10 & $L_{\infty}, \epsilon=8/255$ & 0.1 & 300\\

Rebuffi et al. \cite{rebuffi2021fixing} + FETRA & WRN28-10 & $L_{\infty}, \epsilon=8/255$ & 0.1 & 300\\


\hline\noalign{\smallskip}

Gowal et al. \cite{gowal2020uncovering} + TETRA & WRN70-16 & $L_{\infty}, \epsilon=8/255$ & 0.1 & 100\\

Gowal et al. \cite{gowal2020uncovering} + FETRA & WRN70-16 & $L_{\infty}, \epsilon=8/255$ & 0.1 & 100\\


\hline\noalign{\smallskip}\hline\noalign{\smallskip}
\end{tabular}
\caption{CIFAR100 params. In the first column, we state the method. In the next columns, we state the architecture, the trained threat model (TTM), $\alpha$ which is the step size and $\gamma$ which is the regularization weight.}
\label{table:cifar100_param}
\end{center}
\end{table}


\setlength{\tabcolsep}{4pt}
\begin{table}[ht!]
\begin{center}

\begin{tabular}{lccccc}
\hline\noalign{\smallskip}\hline

Method & Architecture & TTM & $\alpha$ & $\gamma$ \\

\hline\noalign{\smallskip}

Madry et al. \cite{madry2017towards} + FETRA & RN50 & $L_{2}, \epsilon=3.0$ & 6.0 & 5500\\

\hline\noalign{\smallskip}

Salman et al. \cite{salman2020adversarially} + FETRA & WRN50-2 & $L_{2}, \epsilon=3.0$ & 6.0 & 3000\\

\hline\noalign{\smallskip}

Madry et al. \cite{madry2017towards} + FETRA & RN50 & $L_{\infty}, \epsilon=4/255$ & 1.0 & 6000\\

\hline\noalign{\smallskip}

Salman et al. \cite{salman2020adversarially} + FETRA & WRN50-2 & $L_{\infty}, \epsilon=4/255$ & 1.0 & 3000\\


\hline\noalign{\smallskip}\hline\noalign{\smallskip}
\end{tabular}
\caption{ImageNet params. In the first column, we state the method. In the next columns, we state the architecture, the trained threat model (TTM), $\alpha$ which is the step size and $\gamma$ which is the regularization weight.}
\label{table:imagenet_param}
\end{center}
\end{table}




\clearpage
\section{RPGD analysis}
\label{app:RPGD}

\begin{figure}[h!]
    \centering
    \includegraphics[width=\textwidth]{images/RPGD_cifar10.pdf}
    \caption{CIFAR10 top $k$ accuracy comparison PGD vs RPGD. the x-axis of the left figure represents the top $k$ group size that we select, using Madry et al. \cite{madry2017towards} $\ell_{2}, \epsilon=0.5$. The y-axis represents the top $k$ accuracy, the probability that the true label is contained in the top $k$ group. On the right figure, we present the difference between the two graphs of the left figure, $PGD - RPGD$.}
    \label{fig:RPGD_cifar10}
\end{figure}

\begin{figure}[h!]
    \centering
    \includegraphics[width=\textwidth]{images/RPGD_cifar100.pdf}
    \caption{CIFAR100 top $k$ accuracy comparison PGD vs RPGD. the x-axis of the left figure represents the top $k$ group size that we select, using Madry et al. \cite{rebuffi2021fixing} $\ell_{\infty}, \epsilon=8/255$. The y-axis represents the top $k$ accuracy, the probability that the true label is contained in the top $k$ group. On the right figure, we present the difference between the two graphs of the left figure, $PGD - RPGD$.}
    \label{fig:RPGD_cifar100}
\end{figure}





\clearpage
\section{Ablation study}
\label{app:ablation}


In this part, we discuss the ablations that we performed in order to better understand the contribution of different parts of our method. In \cref{app:vanila} we discuss the necessity of the PAG property in the TETRA algorithm, next in \cref{app:distances} we discuss different options for the distance metric used for classification.








\subsection{Vanila classifier}
\label{app:vanila}

TETRA can be applied to any differentiable classifier. We claim, however, that it enhances the classifier robustness only over classifiers that possess PAG. In this part, we empirically support this claim. 

In \cref{table:cifar10_vanila}, we present TETRA accuracy on CIFAR10 dataset, where the classifier is vanilla trained. As we can see, TETRA achieves an accuracy of around $1\%$ for all of the attacks. When applying TETRA to PAG classifiers, it achieves much better results, as presented in \cref{table:cifar10}. Meaning that TETRA performs well only when applied to classifiers that possess the PAG property. The reason is that our method heavily relies on the generative power of PAG, which does not exist in vanilla-trained classifiers.

\setlength{\tabcolsep}{4pt}
\begin{table}[ht!]
\begin{center}

\begin{tabular}{lcccccccc}
\hline\noalign{\smallskip}\hline

\multirow{3}{*}{Method} & \multirow{3}{*}{Architecture} & \multirow{3}{*}{TTM} & \multirow{3}{*}{Standard} & \multicolumn{4}{c}{Attack}\\

 & & & & \multicolumn{2}{c}{$L_{\infty}$} & \multicolumn{2}{c}{$L_{2}$}\\
 & & & & $8/255$ & $16/255$ & $0.5$ & $1.0$\\

\hline\noalign{\smallskip}\hline\noalign{\smallskip}

Vanila & \multirow{3}{*}{WRN28-10} & \multirow{3}{*}{None} & $\textbf{95.26\%}$ & $00.00\%$ & $00.00\%$ & $00.00\%$ & $00.00\%$\\

DRQ \cite{schwinn2022improving} & & & $10.95\%$ & $\textbf{11.31\%}$ & $\textbf{11.04\%}$ & $\textbf{11.75\%}$ & $\textbf{11.38\%}$\\

TETRA & & & $93.04\%$ & $01.11\%$ & $01.12\%$ & $01.24\%$ & $01.10\%$\\


\hline\noalign{\smallskip}\hline\noalign{\smallskip}
\end{tabular}
\caption{CIFAR10 vanilla classifier results. In the first column, we state the method. We report three consecutive lines of results. One for the base method and then two test time boosting methods: DRQ \cite{schwinn2022improving} and TETRA. In the next columns, we state the architecture, the trained threat model (TTM), and four attacks with different threat models.}
\label{table:cifar10_vanila}
\end{center}
\end{table}





\clearpage
\subsection{Distance metrics}
\label{app:distances}

In TETRA's second phase, we calculate the distance between the input image and the transformed images, and we classify based on the shortest one. Hence, the distance metric that we use for the classification is important. Different metrics have different properties, and we aim at a distance metric that is able to measure the semantic distance between images. 

We compare $\ell_{2}$, $\ell_{1}$ and LPIPS \cite{zhang2018unreasonable} distances over CIFAR10 dataset, and presente the results in \cref{table:cifar10_distances}. We compare the results using the following defense methods \cite{madry2017towards,rebuffi2021fixing,gowal2020uncovering}. As demonstrated, $\ell_{2}$ distance metric performs better, therefore is a favorable choice.


\setlength{\tabcolsep}{4pt}
\begin{table}[ht!]
\begin{center}

\begin{tabular}{lcccccccc}
\hline\noalign{\smallskip}\hline

\multirow{3}{*}{Method} & \multirow{3}{*}{Architecture} & \multirow{3}{*}{TTM} & \multirow{3}{*}{Standard} & \multicolumn{4}{c}{Attack}\\

 & & & & \multicolumn{2}{c}{$L_{\infty}$} & \multicolumn{2}{c}{$L_{2}$}\\
 & & & & $8/255$ & $16/255$ & $0.5$ & $1.0$\\

\hline\noalign{\smallskip}\hline\noalign{\smallskip}

AT \cite{madry2017towards} & \multirow{4}{*}{RN50} & \multirow{4}{*}{$L_{2}, \epsilon=0.5$} & $\textbf{90.83\%}$ & $29.04\%$ & $00.93\%$ & $69.24\%$ & $36.21\%$\\

TETRA LPIPS & & & $85.91\%$ & $54.49\%$ & $17.46\%$ & $78.70\%$ & $61.68\%$  \\

TETRA $L_1$ & & & $85.91\%$ & $\textbf{54.55\%}$ & $\textbf{17.64\%}$ & $78.70\%$ & $61.68\%$\\

TETRA $L_2$ & & & $85.76\%$ & $\textbf{54.55\%}$ & $\textbf{17.64\%}$ & $\textbf{78.74\%}$ & $\textbf{61.87\%}$\\

\hline\noalign{\smallskip}

Rebuffi et al. \cite{rebuffi2021fixing} & \multirow{4}{*}{WRN28-10} & \multirow{4}{*}{$L_{2}, \epsilon=0.5$} & $\textbf{91.79\%}$ & $47.85\%$ & $05.00\%$ & $78.80\%$ & $54.73\%$\\

TETRA LPIPS  & & & $87.31\%$ & $58.57\%$ & $09.97\%$ & $85.30\%$ & $\textbf{67.03\%}$\\

TETRA $L_1$ & & & $87.31\%$ & $58.57\%$ & $09.97\%$ & $85.30\%$ & $\textbf{67.03\%}$\\

TETRA $L_2$ & & & $88.33\%$ & $\textbf{59.74\%}$ & $\textbf{11.06\%}$ & $\textbf{85.57\%}$ & $66.01\%$\\


\hline\noalign{\smallskip}

Rebuffi et al. \cite{rebuffi2021fixing} & \multirow{4}{*}{WRN28-10} & \multirow{4}{*}{$L_{\infty}, \epsilon=8/255$} & $\textbf{87.33\%}$ & $60.77\%$ & $25.44\%$ & $66.72\%$ & $35.01\%$\\

TETRA LPIPS & & & $80.97\%$ & $66.49\%$ & $33.54\%$ & $74.73\%$ & $\textbf{59.25\%}$\\

TETRA $L_1$ & & & $80.97\%$ & $66.49\%$ & $33.54\%$ & $74.73\%$ & $\textbf{59.25\%}$\\

TETRA $L_2$ & & & $84.86\%$ & $\textbf{66.96\%}$ & $\textbf{35.15\%}$ & $\textbf{74.84\%}$ & $53.32\%$\\


\hline\noalign{\smallskip}

Gowal et al. \cite{gowal2020uncovering} & \multirow{4}{*}{WRN70-16} & \multirow{4}{*}{$L_{\infty}, \epsilon=8/255$} & $\textbf{91.10\%}$ & $65.88\%$ & $25.95\%$ & $66.44\%$ & $27.22\%$\\

TETRA LPIPS & & & $83.41\%$ & $71.51\%$ & $38.49\%$ & $76.53\%$ & $\textbf{58.60\%}$\\

TETRA $L_1$ & & & $83.41\%$ & $71.51\%$ & $38.49\%$ & $\textbf{76.56\%}$ & $\textbf{58.60\%}$\\

TETRA $L_2$ & & & $87.58\%$ & $\textbf{72.00\%}$ & $\textbf{40.44\%}$ & $75.65\%$ & $49.22\%$\\


\hline\noalign{\smallskip}\hline\noalign{\smallskip}
\end{tabular}
\caption{CIFAR10 results. In the first column, we state the method. For every base method, we report three consecutive lines of results. One for the base method and then two TETRA distance metric variations used for classification: $L_2$ and LPIPS \cite{zhang2018unreasonable}. In the next columns, we state the architecture, the trained threat model (TTM), and four attacks with different threat models.}
\label{table:cifar10_distances}
\end{center}
\end{table}


% \setlength{\tabcolsep}{4pt}
\begin{table}[ht!]
\begin{center}

\begin{tabular}{lcccccccc}
\hline\noalign{\smallskip}\hline

\multirow{3}{*}{Method} & \multirow{3}{*}{Architecture} & \multirow{3}{*}{TTM} & \multirow{3}{*}{Standard} & \multicolumn{4}{c}{Attack}\\

 & & & & \multicolumn{2}{c}{$L_{\infty}$} & \multicolumn{2}{c}{$L_{2}$}\\
 & & & & $8/255$ & $16/255$ & $0.5$ & $1.0$\\

\hline\noalign{\smallskip}\hline\noalign{\smallskip}

AT \cite{madry2017towards} & \multirow{3}{*}{RN50} & \multirow{3}{*}{$L_{2}, \epsilon=0.5$} & $90.83\%$ & $29.04\%$ & $00.93\%$ & $69.24\%$ & $36.21\%$\\

TETRA $L_2$ & & & $87.40\%$ & $51.66\%$ & $14.96\%$ & $78.66\%$ & $59.82\%$\\

TETRA LPIPS & & & $87.36\%$ & $51.95\%$ & $15.22\%$ & $78.70\%$ & $59.87\%$\\

\hline\noalign{\smallskip}

Rebuffi et al. \cite{rebuffi2021fixing} & \multirow{3}{*}{WRN28-10} & \multirow{3}{*}{$L_{2}, \epsilon=0.5$} & $91.79\%$ & $47.85\%$ & $05.00\%$ & $78.80\%$ & $54.73\%$\\

TETRA $L_2$ & & & $88.23\%$ & $59.99\%$ & $11.45\%$ & $85.56\%$ & $66.15\%$\\

TETRA LPIPS  & & & $84.57\%$ & $59.95\%$ & $11.57\%$ & $83.28\%$ & $66.09\%$\\

\hline\noalign{\smallskip}

Rebuffi et al. \cite{rebuffi2021fixing} & \multirow{3}{*}{WRN28-10} & \multirow{3}{*}{$L_{\infty}, \epsilon=8/255$} & $87.33\%$ & $60.77\%$ & $25.44\%$ & $66.72\%$ & $35.01\%$\\

TETRA $L_2$ & & & $85.00\%$ & $66.86\%$ & $34.88\%$ & $74.24\%$ & $53.02\%$\\

TETRA LPIPS & & & $80.34\%$ & $67.54\%$ & $35.63\%$ & $74.89\%$ & $56.39\%$\\


\hline\noalign{\smallskip}\hline\noalign{\smallskip}
\end{tabular}
\caption{CIFAR10 results. In the first column, we state the method. For every base method, we report three consecutive lines of results. One for the base method and then two TETRA distance metric variations used for classification: $L_2$ and LPIPS \cite{zhang2018unreasonable}. In the next columns, we state the architecture, the trained threat model (TTM), and four attacks with different threat models.}
\label{table:cifar10_LPIPS}
\end{center}
\end{table}
 - transformation using LPIPS


% \subsection{Transformation process}
% \label{app:transformation_process}



\clearpage
\section{Runtime analysis}
In this part, we compare the inference time of the test-time methods that we used, over CIFAR10 and CIFAR100. For CIFAR10 we compare TETRA to DRQ \cite{schwinn2022improving}, and for CIFAR100 we compare FETRA to DRQ \cite{schwinn2022improving}. As can be seen, for both of the datasets, our method is slower than the baseline. Our method, however, is faster than DRQ \cite{schwinn2022improving}. These experiments were performed using one GeForce RTX 3080 with batch size $=1$.

\label{app:runtime}
\setlength{\tabcolsep}{4pt}
\begin{table}[ht!]
\begin{center}

\begin{tabular}{lcccc}
\hline\noalign{\smallskip}\hline

\multirow{1}{*}{Method} & \multirow{1}{*}{Architecture} & \multirow{1}{*}{TTM} & \multirow{1}{*}{Inference time}\\


\hline\noalign{\smallskip}\hline\noalign{\smallskip}

AT \cite{madry2017towards} & \multirow{3}{*}{RN50} & \multirow{3}{*}{$L_{2}, \epsilon=0.5$} & $\times 1$\\

DRQ \cite{schwinn2022improving} & & & $\times 160$\\

TETRA & & & $\times 23$\\

\hline\noalign{\smallskip}

Rebuffi et al. \cite{rebuffi2021fixing} & \multirow{3}{*}{WRN28-10} & \multirow{3}{*}{$L_{2}, \epsilon=0.5$} & $\times 1$\\

DRQ \cite{schwinn2022improving} & & & $\times 117$\\

TETRA & & & $\times 26$\\

\hline\noalign{\smallskip}

Rebuffi et al. \cite{rebuffi2021fixing} & \multirow{3}{*}{WRN28-10} & \multirow{3}{*}{$L_{\infty}, \epsilon=8/255$} & $\times 1$\\

DRQ \cite{schwinn2022improving} & & & $\times 119$\\

TETRA & & & $\times 26$\\

\hline\noalign{\smallskip}

Gowal et al. \cite{gowal2020uncovering} & \multirow{3}{*}{WRN70-16} & \multirow{3}{*}{$L_{\infty}, \epsilon=8/255$} & $\times 1$\\

DRQ \cite{schwinn2022improving} & & & $\times 290$\\

TETRA & & & $\times 127$\\


\hline\noalign{\smallskip}\hline\noalign{\smallskip}
\end{tabular}
\caption{Inference time comparison over CIFAR10. In this table we perform an inference time comparison between 3 defense methods. For every base classifier, we report three consecutive lines of inference time. One for the base method, next we present DRQ, and finally TETRA. In the first column we present the method name. Next we present the architecture, and the trained threat model (TTM), and finally we present the inference time. This value stands for how much time it takes for every method to perform.}
\label{table:cifar10_runtime}
\end{center}
\end{table}

\setlength{\tabcolsep}{4pt}
\begin{table}[ht!]
\begin{center}

\begin{tabular}{lcccc}
\hline\noalign{\smallskip}\hline

\multirow{1}{*}{Method} & \multirow{1}{*}{Architecture} & \multirow{1}{*}{TTM} & \multirow{1}{*}{Inference time}\\


\hline\noalign{\smallskip}\hline\noalign{\smallskip}

Rebuffi et al. \cite{rebuffi2021fixing} & \multirow{3}{*}{WRN28-10} & \multirow{3}{*}{$L_{inf}, \epsilon=8/255$} & $\times 1$\\

DRQ \cite{schwinn2022improving} & & & $\times 686 $\\

FETRA & & & $\times 27$\\

\hline\noalign{\smallskip}

Gowal et al. \cite{gowal2020uncovering} & \multirow{3}{*}{WRN70-16} & \multirow{3}{*}{$L_{\infty}, \epsilon=8/255$} & $\times 1$\\

DRQ \cite{schwinn2022improving} & & & $\times 1380$\\

FETRA & & & $\times 121$\\


\hline\noalign{\smallskip}\hline\noalign{\smallskip}
\end{tabular}
\caption{Inference time comparison over CIFAR100. In this table we perform an inference time comparison between 3 defense methods. For every base classifier, we report three consecutive lines of inference time. One for the base method, next we present DRQ, and finally TETRA. In the first column we present the method name. Next we present the architecture, and the trained threat model (TTM), and finally we present the inference time. This value stands for how much time it takes for every method to perform.}
\label{table:cifar100_runtime}
\end{center}
\end{table}





