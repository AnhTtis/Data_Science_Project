\section{Discussion and conclusion}
\label{sec:discussion_conclusion}

This work presents a novel test-time adversarially robust classification method called TETRA, which requires neither training nor access to training data. TETRA utilizes the PAG property in order to boost the performance of any differentiable AT classifier. To the best of our knowledge, this is the first time that an AT classifier has exploited the PAG property. TETRA does not use the standard evaluation flow as it operates in two steps. First, it transforms the input image toward all of the classes of the dataset. Next, it classifies based on the shortest transformation. Our method was validated through an extensive evaluation, using AutoAttack on different architectures and AT training methods, and three standard datasets: CIFAR10, CIFAR100 and ImageNet. Although our method's improvement comes at the cost of a slight clean accuracy degradation and longer inference time, it significantly improves the performance of adversarially perturbed data compared to previous methods. We not only enhance the robustness of the same attack on which the model was trained, but we significantly enhance the robustness of unseen attacks as well. Future work will mainly focus on improving clean accuracy and on accelerating inference time.