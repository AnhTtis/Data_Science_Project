\setlength{\tabcolsep}{4pt}
\begin{table}[ht!]
\begin{center}

\begin{tabular}{lcccccccc}
\hline\noalign{\smallskip}\hline

\multirow{3}{*}{Method} & \multirow{3}{*}{Architecture} & \multirow{3}{*}{TTM} & \multirow{3}{*}{Standard} & \multicolumn{4}{c}{Attack}\\

 & & & & \multicolumn{2}{c}{$L_{\infty}$} & \multicolumn{2}{c}{$L_{2}$}\\
 & & & & $8/255$ & $16/255$ & $0.5$ & $1.0$\\

\hline\noalign{\smallskip}\hline\noalign{\smallskip}

Vanila & \multirow{3}{*}{WRN28-10} & \multirow{3}{*}{None} & $\textbf{95.26\%}$ & $00.00\%$ & $00.00\%$ & $00.00\%$ & $00.00\%$\\

DRQ \cite{schwinn2022improving} & & & $10.95\%$ & $\textbf{11.31\%}$ & $\textbf{11.04\%}$ & $\textbf{11.75\%}$ & $\textbf{11.38\%}$\\

TETRA & & & $93.04\%$ & $01.11\%$ & $01.12\%$ & $01.24\%$ & $01.10\%$\\


\hline\noalign{\smallskip}\hline\noalign{\smallskip}
\end{tabular}
\caption{CIFAR10 vanilla classifier results. In the first column, we state the method. We report three consecutive lines of results. One for the base method and then two test time boosting methods: DRQ \cite{schwinn2022improving} and TETRA. In the next columns, we state the architecture, the trained threat model (TTM), and four attacks with different threat models.}
\label{table:cifar10_vanila}
\end{center}
\end{table}
