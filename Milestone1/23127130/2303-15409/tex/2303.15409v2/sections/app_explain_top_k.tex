\section{Explain Top-$k$}
\label{app:explain_top_k}


The Top-$k$ version of CODIP sometimes outperforms the full CODIP by focusing on the most likely classes, thereby reducing the influence of less relevant or noisy class transformations. By concentrating on a smaller set of high-confidence predictions, the Top-$k$ version enhances accuracy by avoiding potential noise from less likely classes. For instance, in \cref{fig:explain_top_k_example}, we show an image of a tulip (from CIFAR-100). CODIP initially predicts Oak tree followed by Tulip. However, by applying the Top-$k$ filter, Oak tree is excluded, allowing Tulip to be correctly selected. Similarly, for a television image, CODIP might predict Palm tree followed by Television. The Top-$k$ filter excludes Palm tree, leading to the accurate prediction of Television. These examples demonstrate how the Top-$k$ version of CODIP can improve performance by concentrating on the most likely classes and filtering out irrelevant ones.



\begin{figure}[h!]
  \centering
  \begin{subfigure}[b]{0.2\linewidth}
    \centering
    \includegraphics[width=\linewidth]{images/top_k_70.jpeg}
  \end{subfigure}
  \hspace{0.02\linewidth} % Small space between images
  \begin{subfigure}[b]{0.2\linewidth}
    \centering
    \includegraphics[width=\linewidth]{images/top_k_163.jpeg}
  \end{subfigure}
  \caption{\textbf{\AlgoNameTop Success vs. CODIP Failure.} Two examples from the CIFAR-100 dataset where the Top-$k$ approach succeeds while CODIP fails. The left image shows a correctly classified tulip, while the right image demonstrates another example of correct classification by the Top-$k$ method.}
  \label{fig:explain_top_k_example}
\end{figure}

% \begin{figure}[h!]
%   \centering
%   \begin{subfigure}[b]{0.45\linewidth}
%     \centering
%     \includegraphics[width=0.3\linewidth]{images/top_k_70.jpeg}
%     % \caption{
%     %   \textbf{CIFAR-100 example 1} An image of a tulip that the Top-$k$ predicts correctly while CODIP misclassified.
%     % }
%   \end{subfigure}
%   % \hspace{0.2\linewidth} % Adds horizontal space between the subfigures
%   \begin{subfigure}[b]{0.45\linewidth}
%     \centering
%     \includegraphics[width=0.3\linewidth]{images/top_k_163.jpeg}  % Replace with your second image path
%     % \caption{
%     %   \textbf{CIFAR-100 example 2} Another example where the Top-$k$ predicts correctly while CODIP misclassified.
%     % }
%   \end{subfigure}
%   \caption{\textbf{\AlgoNameTop Success vs. CODIP Failure} Two examples from the CIFAR-100 dataset where the Top-$k$ approach succeeds while CODIP fails.}
%   \label{fig:explain_top_k_example}
% \end{figure}



% \begin{figure*}[h!]
    
%   \begin{center}
%     % \includegraphics[width=0.2\textwidth]{images/top_k_70.jpeg}
%   \end{center}
%   \caption{
%     \textbf{CIFAR-100 example} An image of a tulip that the Top-$k$ predicts correctly while CODIP misclassified.
% }
%   \label{fig:explain_top_k_example}
% \end{figure*}






% \begin{figure}[h!]
%   \centering
%   \begin{subfigure}[b]{0.45\textwidth}
%     \centering
%     \includegraphics[width=\textwidth]{images/top_k_70.jpeg}
%     \caption{
%       \textbf{CIFAR-100 example} An image of a tulip that the Top-$k$ predicts correctly while CODIP misclassified.
%     }
%   \end{subfigure}
%   \hfill
%   \begin{subfigure}[b]{0.45\textwidth}
%     \centering
%     \includegraphics[width=\textwidth]{images/top_k_163.jpeg}  % Replace with your second image path
%     \caption{
%       \textbf{CIFAR-100 example 2} Another example where the Top-$k$ predicts correctly while CODIP misclassified.
%     }
%   \end{subfigure}
%   \caption{Two examples from the CIFAR-100 dataset where the Top-$k$ approach succeeds while CODIP fails.}
% \end{figure}
