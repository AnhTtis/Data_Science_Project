\section{Preliminary}
\label{sec:notation}


\begin{table}[!t]
    \small
    \centering
    \resizebox{\linewidth}{!}{
    \begin{tabular}{ll}
        \toprule
        {\textbf{Indices}: }  \\
        $c$ & Index for a class ($c \in \{ 1, \dots, C \}$) \\
        $r$ & Index for AL round ($r \in \{ 1, \dots, R\} = [R]$) \\
        $k$ & Index for a client ($k \in \{ 1, \dots, K\} = [K]$) \\
        \midrule
        {\textbf{Parameters}:} \\
        $B$ & Labeling budget for each AL round $r$ \\
        $\alpha$ & Local heterogeneity level\\
        $\rho$  & Global imbalance ratio\\ 
        \midrule
        {\textbf{Data}:} \\
        $U_k^r$ & Pool of unlabeled instances for a client $k$ at round $r$ \\
        $L_k^r$ & A queried instance set from $U_k^r$ at round $r$ \\
        $D_k^r$ & An available labeled set at round $r$ \\ 
        \midrule
        {\textbf{Weights}:} \\
        $\Theta^{r*}$ & Aggregated weights via FL phases on $D^r$  (\textit{global} model)  \\
        $\Theta_{k*}^{r}$ & Separately optimized weights on $D_k^r$ (\textit{local-only} model)  \\
        \bottomrule
    \end{tabular}}
    \vspace*{-0.2cm}
    \caption{Summary of notations throughout the paper.}
    \label{tab:notation}
    \vspace*{-5pt}
\end{table}

\smallskip
\noindent\textbf{AL Procedure.}
For the ease of understanding, we summarize notations in Table\,\ref{tab:notation}.
At the first AL round\,(\ie, $r=1$), each client $k$ randomly selects $B$ instances, ${L}_k^1 = \{x_1, \dots, x_B\}$, from ${U}_k^1$, and oracles annotate them to obtain the initial labeled set ${D}_k^1 = \{(x_1, y_1), \dots, (x_B, y_B)\}$. 
For the next round ($r\geq 2$), based on the given querying strategy $\mathcal{A}(\cdot)$ and the model parameters $\Theta$, the query set of the $k$-th client at round $r$ is sampled by
\begin{equation}
{L}_k^{r} = \mathcal{A}({U}_k^{r}, \Theta, B), ~~{\rm where}~~ {U}_k^{r} = {U}_k^{r-1} \setminus {L}_k^{r-1}.
\label{eq:al_general_form}
\end{equation}

The querying function $\mathcal{A}(\cdot)$ in Eq.\,\eqref{eq:al_general_form} depends on which AL algorithm is used. 
For example, Entropy sampling\cite{confidence_sampling} queries the instances with the highest uncertainty like:
\begin{equation} 
\mathcal{A}({U}, \Theta, B) = \underset{ x_i \in {L},\,|{L}\vert=B,\,{L} \subseteq {U}}{\arg \max} H(p(y | x_i; \Theta))\\
\label{eq:querying}
\end{equation}
where $H(p)\!=\!-\!\sum_{c=1}^C p_c \, \ln p_c$, and $p$ is the predictive probability.
The query set is annotated by the oracle and assembled to expand the available labeled set, \ie, ${D}_k^{r} = {D}_k^{r-1} \cup \{ (x_i, y_i)\,|\,x_i\in{L}_k^{r} \}$. 


\noindent\\
\textbf{FL Procedure.}
After each AL round, we perform the FL procedure of which the objective is to obtain the optimal parameter $\Theta^{r*}$ such that it minimizes the target loss on the given labeled set for all clients, $D^{r} = \cup_{k=1}^{K} D_{k}^{r}$,
\begin{equation}
    \Theta^{r*} = \underset{\Theta}{\arg\min} \, f(\Theta^r) = \underset{\Theta}{\arg\min}\,\frac{1}{|D^{r}|} \sum_{i=1}^{|D^{r}|} f_i(\Theta^r)
    \label{eq:fl_global_loss}
\end{equation}
where $f_i(\Theta) = \ell(x_i, y_i; \Theta)$ and $\ell(\cdot)$ is the loss function determined by the network parameter $\Theta$.
However, due to data privacy, the global model is optimized based on the reformulated update rule on the partitioned data over clients:
\begin{gather}
    f(\Theta^r) = \sum_{k=1}^K \frac{|D_k^{r}|}{|D^{r}|} \, F(\Theta_k^r), \nonumber \\
    \text{ where } F(\Theta_k^r) = \frac{1}{|D_k^{r}|} \sum_{(x_i, y_i) \in D_k^{r}} \ell(x_i, y_i; \Theta_k^r).
    \label{eq:reform_fl_loss}
\end{gather}
The model $\Theta_k^r$ is updated locally on the client side for its local data $D_k^{r}$, and then it is aggregated globally to generate a global model $\Theta^r$. The local update and model aggregation steps are alternated until the global model converges; this corresponds to the most popular FL training pipeline, FedAvg proposed by \cite{fedavg}.


The previous studies\cite{fal_waste_disaster, f_al} have typically used the converged global model $\Theta^{r*}$ in the next AL round as the query selector of Eq.\,\eqref{eq:querying}.
However, considering the hierarchy structure in the FAL framework, it is also possible to use a separately optimized model on local partitioned data; replacing $D^r$ with $D_k^r$ in Eq.\,\eqref{eq:fl_global_loss}. It is often referred to as the local-only model\cite{fedbabu, fedrep}, and we denote it as $\Theta_{k}^{r*}$. 
In the following section, we investigate what these models are specialized in and when they are beneficial to use.
