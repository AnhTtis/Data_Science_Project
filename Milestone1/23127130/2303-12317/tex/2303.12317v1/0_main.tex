% CVPR 2023 Paper Template
% based on the CVPR template provided by Ming-Ming Cheng (https://github.com/MCG-NKU/CVPR_Template)
% modified and extended by Stefan Roth (stefan.roth@NOSPAMtu-darmstadt.de)

\documentclass[10pt,twocolumn,letterpaper]{article}

%%%%%%%%% PAPER TYPE  - PLEASE UPDATE FOR FINAL VERSION
% \usepackage[review]{cvpr}      % To produce the REVIEW version
\usepackage{cvpr}              % To produce the CAMERA-READY version
%\usepackage[pagenumbers]{cvpr} % To force page numbers, e.g. for an arXiv version

% Include other packages here, before hyperref.
\usepackage{graphicx}
\usepackage{amsmath}
\usepackage{amssymb}
\usepackage{booktabs}
\usepackage{tabularx}
\usepackage{makecell}
\renewcommand\theadfont{}
\usepackage{algorithm}
\usepackage{algorithmic}
\usepackage{scalerel}

\DeclareMathOperator*{\argmin}{arg\,min}
\usepackage{amsfonts}
\usepackage{mathtools}
\usepackage{bbm}
\usepackage{multirow}
\usepackage{dsfont}
\usepackage{wrapfig}
\usepackage{enumitem}
\usepackage{comment}
\usepackage{subcaption}
\usepackage{array}
\usepackage{amsthm}
\usepackage{xcolor} % http://www.ctan.org/tex-archive/macros/latex/contrib/xcolor
\makeatletter
\@namedef{ver@everyshi.sty}{}
\makeatother
\usepackage{tikz}
\usetikzlibrary{fit,calc}
\usepackage{minitoc}
% \usepackage{mathptmx}
% \newcommand{\sm}[1]{{\scriptscriptstyle#1}}


% It is strongly recommended to use hyperref, especially for the review version.
% hyperref with option pagebackref eases the reviewers' job.
% Please disable hyperref *only* if you encounter grave issues, e.g. with the
% file validation for the camera-ready version.
%
% If you comment hyperref and then uncomment it, you should delete
% ReviewTempalte.aux before re-running LaTeX.
% (Or just hit 'q' on the first LaTeX run, let it finish, and you
%  should be clear).
\usepackage[pagebackref,breaklinks,colorlinks]{hyperref}


% Support for easy cross-referencing
\usepackage[capitalize]{cleveref}
\crefname{section}{Sec.}{Secs.}
\Crefname{section}{Section}{Sections}
\Crefname{table}{Table}{Tables}
\crefname{table}{Tab.}{Tabs.}


%define a marking command
\newcommand*{\tikzmk}[1]{\tikz[remember picture,overlay,] \node (#1) {};\ignorespaces}
%define a boxing command, argument = colour of box
\newcommand{\boxit}[1]{\tikz[remember picture,overlay]{\node[yshift=3pt,fill=#1,opacity=.25,fit={(A)($(B)+(\linewidth,.8\baselineskip)$)}] {};}\ignorespaces}
\newcommand{\boxitt}[1]{\tikz[remember picture,overlay]{\node[yshift=3pt,fill=#1,opacity=.25,fit={(A)($(B)+(0.86\linewidth,.8\baselineskip)$)}] {};}\ignorespaces}
%define some colours according to algorithm parts (or any other method you like)
\colorlet{orange}{orange!80}
\colorlet{blue}{cyan!80}

%%%%%%%%% PAPER ID  - PLEASE UPDATE
\def\cvprPaperID{7645}% *** Enter the CVPR Paper ID here
\def\confName{CVPR}
\def\confYear{2023}


\renewcommand \thepart{}
\renewcommand \partname{}


\begin{document}
\doparttoc % Tell to minitoc to generate a toc for the parts
\faketableofcontents % Run a fake tableofcontents command for the partocs


\newcommand{\algname}{{LoGo}}  
\newtheorem{definition}{Definition}
\newtheorem{observation}{Observation}
\newtheorem{remark}{Remark}
\doparttoc % Tell to minitoc to generate a toc for the parts
\faketableofcontents % Run a fake tableofcontents command for the partocs


%%%%%%%%% TITLE - PLEASE UPDATE
\title{Re-thinking Federated Active Learning based on Inter-class Diversity}


\author{
SangMook Kim${}^{1}$\thanks{equal contribution} \qquad Sangmin Bae${}^{1}$\footnotemark[1] \qquad Hwanjun Song${}^{2}$\thanks{corresponding authors} \qquad Se-Young Yun${}^{1}$\footnotemark[2] \vspace{2.5pt}\\
${}^{1}$KAIST AI \qquad ${}^{2}$NAVER AI LAB\\
{\tt\small \{sangmook.kim, bsmn0223, yunseyoung\}@kaist.ac.kr \qquad ghkswns91@gmail.com}
}

\maketitle

%%%%%%%%% ABSTRACT
\begin{abstract}
\vspace{-2pt}
Although federated learning has made awe-inspiring advances, most studies have assumed that the client's data are fully labeled.
However, in a real-world scenario, every client may have a significant amount of unlabeled instances.
Among the various approaches to utilizing unlabeled data, a federated active learning framework has emerged as a promising solution. 
In the decentralized setting, there are two types of available query selector models, namely `global' and `local-only' models, but little literature discusses their performance dominance and its causes.
In this work, we first demonstrate that the superiority of two selector models depends on the global and local inter-class diversity.
Furthermore, we observe that the global and local-only models are the keys to resolving the imbalance of each side.
Based on our findings, we propose \algname{}, a FAL sampling strategy robust to varying local heterogeneity levels and global imbalance ratio, that integrates both models by two steps of active selection scheme.
\algname{} consistently outperforms six active learning strategies in the total number of 38 experimental settings. The code is available at: \url{https://github.com/raymin0223/LoGo}.
\end{abstract}
\vspace{-2pt}

%%%%%%%%% BODY TEXT

\section{Introduction}

The ability to reason about plans is critical for performing long-horizon tasks \citep{erol1996hierarchical, sohn2018hierarchical, sharma-etal-2022-skill}, compositional generalization \citep{corona-etal-2021-modular} and generalization to unseen tasks and environments \citep{shridhar2020alfred}.
Consider a simple long-horizon planning scenario where a robot is tasked with preparing a meal and serving it on the table. 
This presents a non-trivial planning problem since the agent needs to understand the sequence of operations required to perform the task and search for the relevant objects in the unfamiliar environment by interacting with various objects. %



Large language models have been recently shown to possess commonsense knowledge about the world such as object affordances and physical dynamics \citep{ouyang2022training,chowdhery2022palm}.
Early approaches considered text based environments and fine-tuned PLMs to predict actions given the history of past observations and actions \citep{jansen-2020-visually,micheli-fleuret-2021-language,yao-etal-2020-keep}.
Recent work has used this ability to reason about plans from text instructions in simulated household environments with simplifying assumptions such as text-only environment observations or feedback \citep{huang2022language,ahn2022can,li2022pre,logeswaran-etal-2022-shot}.


We focus on \emph{visually grounded planning} with PLMs --- the ability to adapt plans based on interaction and visual feedback from the environment.
While PLMs have strong planning commonsense priors, predictions from a PLM may not be directly realizable in the environment since the observation and action spaces are unknown.
This requires \emph{grounding} the PLM in the environment and adapting it to observe visual feedback, which is highly non-trivial.
Some prior works assume the availability of a pre-trained affordance function \citep{ahn2022can} or a success detector \citep{mirchandani2021ella}.
Notably, SayCan \citep{ahn2022can} completely decouples the PLM from observation information by selecting actions that have both high affordability (through a pre-trained affordance model) and high PLM likelihood.
Although this partially addresses the grounding problem, the use of visual feedback for action affordance alone is limited.
Often an agent must choose one of many affordable actions using information from observations.
For example, a driving agent should re-navigate and possibly turn around when encountering a ``road closed'' sign, but both turning around and driving forward are indistinguishable to SayCan because they are both affordable and the PLM is blind to observations.

Another workaround explored in prior work is translating the information in the visual observations to text using a pre-trained captioning system \citep{shridhar2021alfworld,huang2022language}.
However, it can be difficult to faithfully describe an image in words and information is lost in this inherently noisy process, which limits the information available to the planner.



Recent work shows that PLMs can be adapted for various natural language tasks by inserting tunable embeddings or soft prompts at the input of the PLM (also called prompt tuning or prefix tuning)~\citep{li-liang-2021-prefix,lester-etal-2021-power}.
This approach also extends to multi-modal understanding tasks such as image captioning \citep{mokady2021clipcap} and VQA \citep{tsimpoukelli2021multimodal} where images are encoded as soft prompts and finetuned for the target task.
Transformer based architectures have also been successfully applied to offline Reinforcement Learning in recent work \citep{chen2021decision,janner2021offline,li2022pre,reid2022can}.

Taking inspiration from these works, we propose the simple approach of embedding visual observations (`visual prompts') and \textit{directly inserting them as PLM input embeddings}.
The visual encoder and PLM are jointly trained for the target task, an approach we call \textbf{\oursfull}~(\ours).
By teaching the PLM to use observations for planning in an end to end manner, we remove the dependency on external data such as captions and affordability information that was used in prior work.
We show that this simple approach performs better than prior PLM-based planning approaches on two embodied planning benchmarks based on ALFWorld~\citep{shridhar2021alfworld} and Virtualhome~\cite{puig2018virtualhome}.



\section{Related work}
\noindent \textbf{Implict Neural Representation}.
Implicit neural representations (also known as coordinate-based representations) are a popular way to parameterize content of all kinds, such as audio, images, video, or 3D scenes~\cite{FFL, siren, srn, NeRF}.
Recent works \cite{NeRF, DeepSDF, occnet, srn} build neural implicit fields for geometric reconstruction and novel view synthesis achieving outstanding performance.
The implicit neural representation is continuous, resolution-independent, and expressive, and is capable of reconstructing geometric surface details and rendering photo-realistic images. 
%
While explicit representations like point clouds\cite{points1, NHR}, meshes\cite{NT}, and voxel grids\cite{deepvoxels, occnet, NeuralVolume, voxel1} are usually limited in resolution due to memory and topology restrictions.
%
One of the most popular implicit representations - Neural Radiance Field (NeRF) \cite{NeRF} -  proposes to combine the neural radiance field with differentiable volume for photo-realistic novel views rendering of static scenes. However, NeRF requires optimizing the 5D neural radiance field for each scene individually, which usually takes hours to converge. Recent works\cite{PixelNeRF, ibrnet, MVSNeRF} try to extend NeRF to generalization with sparse input views.
%
In this work, we extend the neural radiance field to a general human reconstruction scenario by introducing conditional geometric code and appearance code. 


\noindent \textbf{3D Model-based Human Reconstruction}
With the emergence of human parametric models like SMPL\cite{SMPL,SMPLX} and SCAPE\cite{SCAPE}, many model-based 3D human reconstruction works have attracted wide attention from academics. Benefiting from the statistical human prior, some works\cite{tex2shape, Multi-Garment, expose, VIBE} can reconstruct the rough geometry from a single image or video. 
However, limited by the low resolution and fixed topology of statistical models, these methods cannot represent arbitrary body geometry, such as clothing, hair, and other details well. 
To address this problem, some works\cite{PIFu, pifuhd} propose to use pixel-aligned features together with neural implicit fields to represent the 3D human body, but still have poor generalization for unseen poses. To alleviate such generalization issues, \cite{pamir, arch, doublefield} incorporate the human statistical model SMPL\cite{SMPL, SMPLX} into the implicit neural field as a geometric prior, which improves the performance on unseen poses. 
Although these methods have achieved stunning performance on human reconstruction, high-quality 3D scanned meshes are required as supervision, which is expensive to acquire in real scenarios. Therefore, prior works\cite{PIFu, pifuhd, pamir, arch} are usually trained on synthetic datasets and have poor generalizability to real scenarios due to domain gaps. To alleviate this limitation, 
some works\cite{neuralbody, Anim-NeRF, animnerf_zju, humannerf, arah, a-nerf}  combine neural radiance fields\cite{NeRF} with SMPL\cite{SMPL} to represent the human body, which can be rendered to 2D images by differentiable rendering. 
Currently, some works\cite{gpnerf, genebody, NHP, keypointNeRF, doublediffuse, doublefield} can quickly create neural human radiance fields from sparse multi-view images without optimization from scratch.
While these methods usually rely on accurate SMPL estimation which is not always applicable in practical applications. 
% We introduce a xxx

% identity-specific models, like NeuralBody\cite{neuralbody}
% generalizable models, 
% SMPL\cite{SMPL}, SMPLX\cite{SMPLX}, SCAPE\cite{SCAPE}, Tex2Shape\cite{tex2shape}, Multi-Garment Net\cite{Multi-Garment}, VIBE\cite{VIBE}, Expose\cite{expose}, NeuralBody\cite{neuralbody}, Anim-NeRF\cite{animnerf_zju, animnerf}, Neural Actor\cite{neuralactor}, SelfRecon\cite{selfrecon}, HumanNeRF\cite{humannerf}, PIFu\cite{PIFu}, PIFuHD\cite{pifuhd}, Pamir\cite{pamir}, Arch\cite{arch}, Double Field\cite{doublefield}, GNR\cite{genebody}, NHP\cite{NHP}, GPNeRF\cite{gpnerf}, KeypointNeRF\cite{keypointNeRF}, DoubleDiffuse\cite{doublediffuse}











\begin{figure*}[ht]
    \centering
    \vspace{-1em}
    \includegraphics[width=1.0\linewidth]{figures/method/pipeline.png}
    \vspace{-1.5em}
    \caption{\textbf{The architecture of our method}. Given $m$ calibrated multi-view images and registered SMPL, we build the generalizable model-based neural human radiance field. First, we utilize the image encoder to extract multi-view image features, which are used to provide geometric and appearance information, respectively. In order to adequately exploit the geometric prior, we propose the visibility-based attention mechanism to construct a structured geometric body embedding, which is further diffused to form a geometric feature volume. For any spatial point $\mathbf{x}$, we trilinearly interpolate the feature volume $\mathcal{G}$ to obtain the geometric code $\mathbf{g}(\mathbf{x})$. In addition, we also propose geometry-guided attention to obtain the appearance code $\mathbf{a}(\mathbf{x}, \mathbf{d})$ directly from the multi-view image features. We then feed the geometric code $\mathbf{g}(\mathbf{x})$ and appearance code $\mathbf{a}(\mathbf{x}, \mathbf{d})$ into the MLP network to build the neural feature field $(\mathbf{f}, \sigma) = F(\mathbf{g}(\mathbf{x}), \mathbf{a}(\mathbf{x}, \mathbf{d}))$. Finally, we employ volume rendering and neural rendering to generate the novel view image.
    % \Liqian{1) Add section ref. 2) add detailed caption. 3) Modulate the fig, each module corresponds to a sub-section. 4) keep fig text  consistent with method text}
    }
    \vspace{-1em}
    \label{fig:architecture}
\end{figure*}

\section{Preliminary}
\label{sec:notation}


\begin{table}[!t]
    \small
    \centering
    \resizebox{\linewidth}{!}{
    \begin{tabular}{ll}
        \toprule
        {\textbf{Indices}: }  \\
        $c$ & Index for a class ($c \in \{ 1, \dots, C \}$) \\
        $r$ & Index for AL round ($r \in \{ 1, \dots, R\} = [R]$) \\
        $k$ & Index for a client ($k \in \{ 1, \dots, K\} = [K]$) \\
        \midrule
        {\textbf{Parameters}:} \\
        $B$ & Labeling budget for each AL round $r$ \\
        $\alpha$ & Local heterogeneity level\\
        $\rho$  & Global imbalance ratio\\ 
        \midrule
        {\textbf{Data}:} \\
        $U_k^r$ & Pool of unlabeled instances for a client $k$ at round $r$ \\
        $L_k^r$ & A queried instance set from $U_k^r$ at round $r$ \\
        $D_k^r$ & An available labeled set at round $r$ \\ 
        \midrule
        {\textbf{Weights}:} \\
        $\Theta^{r*}$ & Aggregated weights via FL phases on $D^r$  (\textit{global} model)  \\
        $\Theta_{k*}^{r}$ & Separately optimized weights on $D_k^r$ (\textit{local-only} model)  \\
        \bottomrule
    \end{tabular}}
    \vspace*{-0.2cm}
    \caption{Summary of notations throughout the paper.}
    \label{tab:notation}
    \vspace*{-5pt}
\end{table}

\smallskip
\noindent\textbf{AL Procedure.}
For the ease of understanding, we summarize notations in Table\,\ref{tab:notation}.
At the first AL round\,(\ie, $r=1$), each client $k$ randomly selects $B$ instances, ${L}_k^1 = \{x_1, \dots, x_B\}$, from ${U}_k^1$, and oracles annotate them to obtain the initial labeled set ${D}_k^1 = \{(x_1, y_1), \dots, (x_B, y_B)\}$. 
For the next round ($r\geq 2$), based on the given querying strategy $\mathcal{A}(\cdot)$ and the model parameters $\Theta$, the query set of the $k$-th client at round $r$ is sampled by
\begin{equation}
{L}_k^{r} = \mathcal{A}({U}_k^{r}, \Theta, B), ~~{\rm where}~~ {U}_k^{r} = {U}_k^{r-1} \setminus {L}_k^{r-1}.
\label{eq:al_general_form}
\end{equation}

The querying function $\mathcal{A}(\cdot)$ in Eq.\,\eqref{eq:al_general_form} depends on which AL algorithm is used. 
For example, Entropy sampling\cite{confidence_sampling} queries the instances with the highest uncertainty like:
\begin{equation} 
\mathcal{A}({U}, \Theta, B) = \underset{ x_i \in {L},\,|{L}\vert=B,\,{L} \subseteq {U}}{\arg \max} H(p(y | x_i; \Theta))\\
\label{eq:querying}
\end{equation}
where $H(p)\!=\!-\!\sum_{c=1}^C p_c \, \ln p_c$, and $p$ is the predictive probability.
The query set is annotated by the oracle and assembled to expand the available labeled set, \ie, ${D}_k^{r} = {D}_k^{r-1} \cup \{ (x_i, y_i)\,|\,x_i\in{L}_k^{r} \}$. 


\noindent\\
\textbf{FL Procedure.}
After each AL round, we perform the FL procedure of which the objective is to obtain the optimal parameter $\Theta^{r*}$ such that it minimizes the target loss on the given labeled set for all clients, $D^{r} = \cup_{k=1}^{K} D_{k}^{r}$,
\begin{equation}
    \Theta^{r*} = \underset{\Theta}{\arg\min} \, f(\Theta^r) = \underset{\Theta}{\arg\min}\,\frac{1}{|D^{r}|} \sum_{i=1}^{|D^{r}|} f_i(\Theta^r)
    \label{eq:fl_global_loss}
\end{equation}
where $f_i(\Theta) = \ell(x_i, y_i; \Theta)$ and $\ell(\cdot)$ is the loss function determined by the network parameter $\Theta$.
However, due to data privacy, the global model is optimized based on the reformulated update rule on the partitioned data over clients:
\begin{gather}
    f(\Theta^r) = \sum_{k=1}^K \frac{|D_k^{r}|}{|D^{r}|} \, F(\Theta_k^r), \nonumber \\
    \text{ where } F(\Theta_k^r) = \frac{1}{|D_k^{r}|} \sum_{(x_i, y_i) \in D_k^{r}} \ell(x_i, y_i; \Theta_k^r).
    \label{eq:reform_fl_loss}
\end{gather}
The model $\Theta_k^r$ is updated locally on the client side for its local data $D_k^{r}$, and then it is aggregated globally to generate a global model $\Theta^r$. The local update and model aggregation steps are alternated until the global model converges; this corresponds to the most popular FL training pipeline, FedAvg proposed by \cite{fedavg}.


The previous studies\cite{fal_waste_disaster, f_al} have typically used the converged global model $\Theta^{r*}$ in the next AL round as the query selector of Eq.\,\eqref{eq:querying}.
However, considering the hierarchy structure in the FAL framework, it is also possible to use a separately optimized model on local partitioned data; replacing $D^r$ with $D_k^r$ in Eq.\,\eqref{eq:fl_global_loss}. It is often referred to as the local-only model\cite{fedbabu, fedrep}, and we denote it as $\Theta_{k}^{r*}$. 
In the following section, we investigate what these models are specialized in and when they are beneficial to use.

\section{Observation and Analysis}
\label{sec:analysis}

In this section, we analyze the performance trend between global and local-only models as the query selector, with respect to the degree of class imbalance in local and global data distribution.
We synthetically adjust two indicators of inter-class diversity, $\alpha \in \{0.1, 1.0, \infty\}$ and $\rho \in \{1, 5, 10, 20\}$, on CIFAR-10 benchmark.
As $\alpha$ and $\rho$ get lower and higher, the levels of local heterogeneity and global imbalance increase, respectively (refer to Appendix\,\ref{sec:dataset_summary} for the detailed data distribution).
For both query selectors, we use Entropy sampling\cite{confidence_sampling} as an active learning algorithm, and the training set is progressively labeled with the query ratio of 10\% in each AL round.

\noindent\\
{\textbf{Comparison Metric.}}
We evaluate the superiority over AL rounds through pairwise comparison\cite{badge, alfa_mix}, widely used in conventional AL literature.
We repeat each experimental setup, a pair of $\alpha$ and $\rho$, with four different seeds and obtain a set of four accuracy results $a_{r}=\{a_{r,1},..., a_{r,4}\}$ at each round $r$. Then, we conduct a two-sided t-test, where $t$-score is defined by Definition\,\ref{def:t_score} for two given strategies $i$ and $j$. Note that the strategy denotes the combination of the sampling strategy and the type of query selector. 

\begin{definition}\!\!\!{\normalfont\cite{semenick1990tests}}\, Let $a_{r}^{i}$ and $a_{r}^{j}$ be the set of accuracies for two different FAL strategies $i$ and $j$. Then, $t$-score at AL round $r$ is formulated as:
\begin{equation}
\begin{gathered}
t_{r}^{ij} = \frac{\sqrt{4} \mu^{ij}_{r}}{\sigma^{ij}_{r}},\,\,\, \text{\normalfont where } \mu^{ij}_{r} \!= \!\frac{1}{4}\sum_{l=1}^{4} \big(a_{r,l}^{i} - a_{r,l}^{j}\big) \!\!\! \\ \text{\normalfont and}\,\,\, \sigma^{ij}_{r}=\sqrt{\frac{1}{3}\sum_{l=1}^{4} \Big(\big(a_{r,l}^{i} - a_{r,l}^{j}\big) - \mu^{ij}_{r}\Big)}.
\label{eq:t_score}
\end{gathered}
\end{equation}
\label{def:t_score}
\end{definition}
\vspace*{-0.4cm}
\noindent Here, the strategy $i$ is considered to beat the strategy $j$ if $t_r^{ij} >$ 2.776. Therefore, the \textit{winning rate} for all AL rounds is formulated as follows:
\begin{equation}
{\sf win}^{ij} =  \sum_{r=1}^{R} \frac{1}{R} \mathds{1}_{t_{r}^{ij} > \text{2.776}}.
\label{eq:cell_value}
\end{equation}
The value of winning rate becomes 1 if the strategy $i$ beats the strategy $j$ over all AL rounds.

\begin{figure}[t!]
\centering
\includegraphics[width=0.92\linewidth]{figure/analysis/cifar10_lt.pdf}
 \vspace*{-7pt}
\caption{Gap between the winning rate of global and local-only models by varying global imbalance ratio ($\rho$) and local heterogeneity ($\alpha$) on CIFAR-10 benchmark. The experimental setups for (a)-(d) are also compatible with Figure\,\ref{fig:obs2_cnt_acc} and Table\,\ref{tab:emd}.}
\vspace{-16pt}
\label{fig:cifar_lt}
\end{figure}


\begin{observation} 
The superiority of local-only and global query-selecting models varies according to the degree of local heterogeneity and global imbalance ratio.
\label{obs:obs1}
\end{observation}

In Figure\,\ref{fig:cifar_lt}, we summarize the performance gap between two query models depending on the local heterogeneity level\,(indicated by different shapes) and the global imbalance ratio\,(increased along x-axis).
The y-axis represents the gap of the winning rate in Eq.\,\eqref{eq:cell_value} between global and local-only models; thus, the value becomes positive up to +1 if the global model beats the local-only model, otherwise negative up to -1. 
At a glance, there is a clear and consistent superiority of two query models according to $\alpha$ and $\rho$, where the dominance has intensified toward both extremes (\eg, upper right and lower left).
This observation contradicts the previous findings that the global model has always outperformed the local-only model as the query selector in a FAL framework\cite{f_al}. 
We provide more in-depth analysis in following Obs.\,\ref{obs:obs2} and Obs.\,\ref{obs:obs3}. \qed

\begin{figure*}[t!]
    \centering
    \begin{subfigure}[b]{0.47\linewidth}
    \centering
\includegraphics[width=\linewidth]{figure/analysis/obs2/rho1_alpha01_total.pdf}
    \caption{Low global imbalance ($\rho=1$) and high heterogeneity ($\alpha=0.1$).}     
    \end{subfigure}
    \hspace{15pt}
    \centering
    \begin{subfigure}[b]{0.47\linewidth}
    \centering
    \includegraphics[width=\linewidth]{figure/analysis/obs2/rho1_alpha_inf_total.pdf}
    \caption{Low global imbalance ($\rho=1$) and low heterogeneity ($\alpha=\infty$).}       
    \end{subfigure}
    \begin{subfigure}[b]{0.47\linewidth}
    \centering
    \includegraphics[width=\linewidth]{figure/analysis/obs2/rho20_alpha_01_total.pdf}
    \caption{High global imbalance ($\rho=20$) and high heterogeneity ($\alpha=0.1$).}
    \end{subfigure}
    \hspace{15pt}
    \begin{subfigure}[b]{0.47\linewidth}
    \centering
    \includegraphics[width=\linewidth]{figure/analysis/obs2/rho20_alpha_inf_total.pdf}
    \caption{High global imbalance ($\rho=20$) and low heterogeneity ($\alpha=\infty$).}  
    \end{subfigure}
    \vspace*{-0.2cm}
    \caption{Matrices of data count\,(left) and class-wise accuracy\,(right) for CIFAR-10, with four combinations of $\rho=\{1, 20\}$ and $\alpha=\{0.1, \infty\}$. k1--10 denote ten clients, c1--10 are ten classes of CIFAR-10. 
    There are two types of data counts, \# of per-client local instances (1-10th rows) and aggregated instances over all clients (the last row). Similarly, the test accuracy is measured with local-only models (1-10th rows) and a global model (the last row). 
    (a)--(d) setups correspond to those of Figure\,\ref{fig:cifar_lt}. See Appendix\,\ref{sec:detail_analysis_matrices} for the more cases.
    }
    \vspace*{-0.2cm}
    \label{fig:obs2_cnt_acc}
\end{figure*}


\noindent \\
\vspace{-15pt}
\begin{observation} 
As local heterogeneity increases\,($\alpha \downarrow$), a local-only query selector is preferred due to the increased significance of local inter-class diversity. 
\label{obs:obs2}
\end{observation}


As the collapse of the local inter-class balance (\ie, lower $\alpha$) incurs severe performance degradation due to a weight divergence\cite{astraea, fedavg}, addressing local imbalance can improve the learning stability and performance.
Since the local-only models are separately trained on each client, in general, the local-only model has shown higher confidence for its own data distribution than the global model\cite{fedbabu, fedrep}.
In Figure\,\ref{fig:obs2_cnt_acc}, we visualized the number of class instances and the class-wise test accuracy in the first AL round (see the caption for details).
Specifically, we confirmed a high correlation between counts and accuracy in Figure\,\ref{fig:obs2_cnt_acc}-(a), such that the local-only models have higher accuracy than the global model for major classes of their data distribution.
Thus, by the nature of favoring low-confident instances in AL, the local-only model tends to select the instances with local minority classes as the query.




More precisely, we verify that the local-only model indeed queries the locally balanced set using earth mover's distance\,(EMD)\cite{emd_measure}. 
In Table\,\ref{tab:emd}, local EMD measures the mean of distance between class distribution of local query sets and a uniform distribution. The lower the value, the more balanced the locally queried instances.
As shown in Table\,\ref{tab:emd}-(a) with high local heterogeneity, the local EMD of the local-only model\,(L) is lower than that of the global model\,(G). 
That is, the local-only model queries more diverse instances than the global model with respect to the local inter-class diversity. 

% 
Meanwhile, in the case of (b), the global model, trained with more samples, has higher accuracy over the classes due to little distribution discrepancy.
Although the more accurate model is likely to have the higher prediction confidence, it does not mean that it is better at identifying the required instances based on the current local dataset, which the global model had not directly learned. 
In practice, the local-only model still chose the more locally balanced query set (Table\,\ref{tab:emd}-(b)), and we supposed this contradiction makes no sizeable winning gap of the case (b) in Figure\,\ref{fig:cifar_lt}. \qed


\begin{table}[t!]
    \small
    \renewcommand*{\arraystretch}{1}
    \addtolength{\tabcolsep}{-1pt}
    \resizebox{\linewidth}{!}{
    \begin{tabular}{c|c|cccc|cccc}
         \toprule
        & & \multicolumn{4}{c|}{\bf Obs.\,\ref{obs:obs2}: Local EMD $(\downarrow)$} & \multicolumn{4}{c}{\bf Obs.\,\ref{obs:obs3}: Global EMD $(\downarrow)$} \\
        \cmidrule(l{2pt}r{2pt}){3-6} \cmidrule(l{2pt}r{2pt}){7-10}
        \multirow{-2.5}{*}{\!\!Case} & \multirow{-2.5}{*}{\!Model\,\!\!} & 10\% &  20\% & 30\% & 40\% &  10\% & 20\% & 30\% & 40\% \\
        \midrule
        \multirow{2}{*}{\!\!(a)} & G & 0.632 & 0.638 & 0.641 & 0.643 & 0.019 & 0.064 & 0.086 & 0.095  \\
        & L & 0.632 &  0.597 &  0.592 &  0.595 & 0.019 &  0.050 &  0.050 &  0.046  \\
        \hline
        \multirow{2}{*}{\!\!(b)} & G & 0.049 & 0.077 & 0.070 & 0.084 & 0.014 & 0.070 & 0.066 & 0.063  \\
        & L & 0.049 &  0.042 &  0.054 &  0.059 & 0.014 &  0.025 &  0.044 &  0.053  \\
        \hline
       \multirow{2}{*}{\!\!(c)}  & G & 0.692 & 0.680 & 0.676 & 0.674 & 0.377 &  0.300 &  0.294 &  0.294  \\
        & L & 0.692 &  0.641 &  0.633 &  0.636 & 0.377 & 0.334 & 0.326 & 0.321 \\\hline
        \multirow{2}{*}{\!\!(d)}  & G & 0.371 &  0.298 &  0.284 &  0.274 & 0.368 &  0.294 &  0.282 &  0.272 \\
        & L & 0.371 & 0.313 & 0.293 & 0.290 & 0.368 & 0.309 & 0.287 & 0.288  \\ \bottomrule
    \end{tabular}}
    \vspace*{-0.2cm}
    \caption{Local EMD and global EMD on CIFAR-10. We summarize the results of four AL rounds with the labeling budget of 10\% per round. (a)--(d) setups correspond to those of Figure \ref{fig:cifar_lt}. See Appendix\,\ref{sec:detail_analysis_emd} for EMDs of more cases.}
    \label{tab:emd}
    \vspace{-12pt}
\end{table}


\begin{observation}
As the degree of global class imbalance increases ($\rho \uparrow$), it is more advantageous to exploit a global model that alleviates the global class imbalance.
\label{obs:obs3}
\end{observation}

Based on Obs.\,\ref{obs:obs2}, the local-only model should outperform the global model in the case (c), but there is no clear superiority between them with respect to the winning rate in the case (c) of Figure \ref{fig:cifar_lt}.
The only answer for this conundrum is the presence of global minority classes due to the high global imbalance ratio.
The local heterogeneity is obviously a crucial factor by Obs.\,\ref{obs:obs2}, but global class imbalance is also another factor that significantly degrades the classification performance in the FAL framework.
Here, the major challenge is that neither the central server nor local clients cannot access any information of aggregated data due to privacy preservation.
The only way to address this problem, we should utilize the global model that implicitly learns the knowledge of the entire data distribution through the aggregation phase in Eq.\,\eqref{eq:reform_fl_loss}.

We introduce an additional global EMD, the indicator of measuring the inter-class diversity of the aggregated queried set over all clients. 
As can be seen in Table \ref{tab:emd}-(c), where the global imbalance ratio is high, we confirm that the global query selector\,(G) favors to query global minority classes. 
The global EMD of the global model is lower than that of the local-only model, \ie the more globally balanced query set, but the local EMD is the opposite. 

Meanwhile, in the case of (d), the minority classes are always the same from the global and local perspectives.
It is different from the case (b), where the minority classes with respect to the accuracy differ in each client depending on the informativeness of instances despite the same instance number.
Therefore, in this scenario, the global model has high confidence even in local datasets, leading to significantly overwhelming the local-only model in the case (d) of Figure \ref{fig:cifar_lt}.
In conclusion, the global inter-class diversity is an essential factor when global imbalance exists. \qed 


\section{Measuring and mitigating spurious concept association via concept space}
\label{sec:method}

In this section, we present statistical approaches (1) to measure two aspects of systematic error: the associations between a concept and instances (especially misclassifications) (Section \ref{sec:method-concept-association}) and the attribution of predictions to certain concepts (Section \ref{sec:method-concept-influence}) for Validation phase (\textbf{T3}), and (2) to mitigate spurious associations (Section \ref{sec:method-debiasing}) in Mitigation phase (\textbf{T4}).

\subsection{Measuring concept associations}
\label{sec:method-concept-association}
First, we present a method of measuring the association between a concept and instances. We start by defining two different levels of associations, (1) Instance-level association: It quantifies how a concept is closely aligned with an individual instance at a fine-grained level. (2) Class-level association: By aggregating instance-level associations into classes, we compute how a concept is associated with a group of instances in each target class. This measure can be extended to quantify the between-class disparity over concept associations.

To compute these measures, we take the well-known perspective of concept space where the association between two subspaces, concepts, and instances,  can be measured. Suppose that $N$ images are given in a machine learning task using a deep classifier $f$ consisting of a group of layers. Given a specific layer $l$ of interest, a subset of images $X$ and segments $S$ are vectors $\{\vec{v}_x\}_{x \in X}$ and $\{\vec{v}_s\}_{s \in S}$ in the space of activations of layer $l$. An interpretable concept $c$ is represented as a group of segment images $S_c$ that share the semantics of the target concept. We define a concept vector $\vec{v_{c}}$ as a mean vector of a group of segment vectors (e.g., two segment sets colored red and yellow in Fig. \ref{fig:method-subspace-alignment}a).

\subsubsection{Instance-level concept association} First, we define the instance-level association between a concept $c$ and images $\vec{X}$ in the space of layer $l$. Given an image vector $\vec{v}_x$ and a concept vector $\vec{v}_c$, the association can be computed via cosine similarity:

\begin{equation}
    \label{equation:concept-association}
    CA_{(x, c)} = \frac{\vec{v}_x \times \vec{v}_c}{\|\vec{v}_x\| \|\vec{v}_c\| }
\end{equation}

\begin{figure}[!ht]
    \centering
    \includegraphics[width=0.95\columnwidth]{figures/subspace-alignment.pdf}
    \vspace{-12.5pt}
    \caption{\label{fig:method-subspace-alignment}
    \textbf{The preprocessing steps for aligning instances and segments.} In the original space (a), two different concepts are not well-distinguishable. For better alignment, (b) the image subspace is normalized as shown, and (c) the segment subspace is then projected into the normalized image space. After preprocessing, two concept vectors are better discriminated in their directions.
    }
    \vspace{-2.5pt}
\end{figure}

\textbf{Subspace alignment.} These concept associations are measured via pairwise cosine similarity as angles between images $\{\vec{v_x}\}_{x \in X}$ and a concept $\vec{c}$ that seemingly work well; however, they are not distinguishable (Fig. \ref{fig:method-subspace-alignment}a), as pointed out in \cite{ConceptWhiteningInterpretable2020}. In our case, two subspaces of instances and segments are misaligned and placed far away from each other, so that two distinct concepts in Fig. \ref{fig:method-subspace-alignment}a result in indistinguishable cosine similarities, thus misleading associations between instances and concepts.

To address the misalignment problem, we perform the preprocessing to make two spaces better align with each other, consisting of two steps as illustrated in Fig. \ref{fig:method-subspace-alignment}: First, we normalize all instances to project them into the standardized space. Second, to align the segment space with the instance space, we project segment vectors into the normalized instance space using mean and standard deviation of instance vectors. As a result of the preprocessing steps, two concept vectors are well-distinguishable in their directions as much as they are semantically different (Fig. \ref{fig:method-subspace-alignment}c).

\textbf{Exclusive concept association.} After the preprocessing for subspace alignment, we compute exclusive concept association. This measure identifies which concepts within each image are relatively more influential than others as the top-associated concept.


\begin{figure}[!ht]
    \centering
    \includegraphics[width=\columnwidth]{figures/exclusive-concept-association.pdf}
    \caption{\label{fig:method-ex}
    \textbf{Three detailed steps for measuring exclusive concept association.}
    }
\end{figure}

Specifically, we take the following steps to compute exclusive association scores: (1) Measuring absolute concept associations: for a group of instances $X$ and concepts $C$, we calculate the pairwise cosine similarity (Equation \ref{equation:concept-association}) between $\{\vec{v_x}\}_{x \in X}$ and $\{\vec{v_c}\}_{c \in C}$ to measure how a concept and instance is associated in an absolute manner (Fig. \ref{fig:method-ex}a). From this matrix, by ranking instances in order of absolute scores for each row, we can obtain absolute concept associations as a set of per-concept image rankings $R_{raw(C)}=\{r_{raw(c)}\}_{c \in C}$. (2) Normalizing concept associations per image: for each image, association scores are standardized to convert them to relative association scores (i.e., column-wise standardization of the matrix in Fig. \ref{fig:method-ex}a). From the normalized matrix, each row as relative concept associations for each image is converted to a ranking, resulting in a set of per-image concept rankings (Fig. \ref{fig:method-ex}b). (3) Extracting the most associated concepts: From the per-image rankings, we extract the most-associated concept in each image, resulting in a obtain a list of top-associated concepts for each instance $\{(x, c_{top(x)})\}$. (4) Ranking the most-associated images per concept: These pairs are further grouped by each concept whose images are in the order of associations, resulting in a set of per-concept image rankings (Fig. \ref{fig:method-ex}d). These rankings $R_{ex(C)}=\{r_{ex(c)}\}_{c \in C}$ indicate how a concept is exclusively associated with a set of instances. 

\textbf{Combined association score.} Given two types of ranked associations, $R_{raw(C)}$ and $R_{ex(C)}$, we aim to combine them to find out instances that are the most associated with a concept absolutely and exclusively than other concepts. We use FREX score \cite{bischof2012summarizingfrex}, which was originally introduced in topic modeling, to combine raw and exclusive associations $R_{raw(C)}$ and $R_{ex(C)}$ across topic words:

\begin{equation}
    CA_{comb} = (\frac{w}{ECDF_{R_{ex}}} + \frac{1-w} {ECDF_{R_{raw}}})^{-1}
    \label{equation:combined-association-score}
\end{equation}

where $w$ is the weight between 0 and 1 given to exclusivity and ECDF is the empirical CDF function. We set $w=0.2$ to combine two rankings in our setting. Based on FREX score, we derive the combined ranking $R_{comb}$ to identify the list of concept-associated instances for the subsequent tasks.

\subsubsection{Class-level concept association and Between-class disparity}
\label{sec:method-between-class-disparity}

Based on the underlying measure of combined concept associations in Equation \ref{equation:combined-association-score}, we compute the between-class disparity to measure how a concept is more biased towards a certain class to compare the class-wise magnitude of concept associations. In the system, we use this metric over training examples to identify concept-wise biases towards classes in the training phase. It allows us to trace back to which concepts were learned to be patterns of one class relative to another. The between-class disparity is computed via taking the difference between the sum of combined concept associations based on the following equation:

\begin{equation}
    \label{equation:between-disparity}
    Disparity_{(c,X)} = \sum_{x \in X_{pos}}{CA_{comb(c,x)}} - \sum_{x \in X_{neg}}{CA_{comb(c,x)}}
\end{equation}

where $X_{neg}$ and $X_{pos}$ are two sets of instances belonging to either positive and negative class. When the difference is greater than zero, it indicates the concept is biased towards the positive class, otherwise negative class.

\subsection{Quantifying concept influences}
\label{sec:method-concept-influence}
Next, we quantify the influence of concepts on certain predictions, especially to measure how false predictions are attributed to a specific concept. We employ the approach of TCAV \cite{InterpretabilityFeatureAttributionQuantitative}, which uses directional derivatives to quantify how predictions over a group of instance vectors $\{\vec{v_x}\}_{x \in X}$ towards a certain class are sensitive to the direction of a concept vector $\vec{v_c}$. Compared to TCAV leveraging a concept activation vector (CAV) as an orthogonal direction of a hyperplane of trained linear classifier, we identify a concept vector $\vec{v_c}$ from a centroid of segment vectors $\vec{v_S}/\|S\|$, which is cost-effective in computations without having to train linear classifiers multiple times.

\subsection{Debiasing spurious concept associations}
\label{sec:method-debiasing}
In the workflow, the Mitigation phase aims to reduce the degree of spurious associations between a certain concept and target class. This requires a strategy to correct the spurious concept-class associations. While there are various approaches, such as active learning (i.e., adding training examples with desirable patterns), our goal is to prevent any undesirable concepts from being associated with a certain class. In this context, we propose a method to debias the proportion of concepts from instances that are highly associated with those concepts. We take the approach of debiasing from the line of work in fair representation \cite{sutton2018biased, bolukbasi2016man, zhao2018learning} (e.g., removing the gender bias from gender-specific words). For internal representations of instances containing undesirable attributes, it projects the target instances into the orthogonal direction of the unwanted attribute such as job or gender to neutralize them. This approach is applicable to our context to remove the property of a concept (as a vector $\vec{v_c}$) from each concept-associated instance (as a vector $\vec{v_i}, i \in I$). We apply this step to a group of training samples in $r_{comb(c)}$ that are identified as the most associated with a concept $c$:

\begin{equation}
    Debias_{(i,c)}: {\vec{v_i}} - \frac{{\vec{v_i}}*{\vec{v_c}}}{\|{\vec{v_c}} \times {\vec{v_c}}\| } \times {\vec{v_c}}
    \label{equation:debias}
\end{equation}

After this step, we compute the remaining bias ratio, the degree of the between-class disparity in Equation \ref{equation:between-disparity} being mitigated before and after debiasing:

\begin{equation}
    \label{equation:remaining-bias-ratio}
    Remaining \ Bias \ Ratio_{(c, X)} = 1 - \frac{(Disparity_{AF(c, X)} - Disparity_{BF(c, X)})}{Disparity_{BF(c, X)}}
\end{equation}



  We evaluate Alg.~\ref{alg: SCG}%the Stochastic Continuous Greedy (SCG) algorithm
  , %described in \ref{alg: SCG}, 
  with sampling and polynomial estimators over two well-known problem instances (influence maximization and facility location%, and data summarization
  ) with real and synthetic %different graph settings
  datasets. We summarize these setups in Tab.~\ref{tab:datasets}. For a more detailed overview of the datasets and experiment parameters, please refer to App.~\ref{app:exps}\deleted{of the supplement}. Our code \replaced{is publicly accessible}{will be public once the submission is reviewed}.\footnote{\url{https://github.com/neu-spiral/StochSubMax}}
  
\begin{wraptable}{r}{6cm}
% \begin{table}[t]
\vspace*{-25pt}
\begin{center}
    \begin{tabular}{|c|c|ccc|cc|}
    \hline
    \thead{instance} & \thead{dataset} & \thead{$|z|$} & \thead{$|S|$} & \thead{$|E|$} & \thead{m} & \thead{k} \\%& \thead{$f^*$}\\
    \hline
    % \thead{IM} & \texttt{GreedyTricker} & 1 & 12 & 13 & 2 & 1 \\%& 0.6\\
    \thead{IM} & \texttt{SBPL} & 20 & 400 & 914 & 4 & 1 \\%& 0.06\\
    % \thead{IM} & \texttt{SyntheticBipartiteUniform} & 100 & 200 & 400 & 4 & 2 & 0.35\\
    \thead{IM} & \texttt{ZKC} & 20 & 34 & 78 & 2 & 3 \\%& -\\
    \thead{FL} & \texttt{MovieLens} & 4000 & 6041 & 256 & 10 & 2 \\%& -\\
    % \thead{SM} & \texttt{MovieLens} & - & - & - & - & - & -\\
    % \thead{SM} & \texttt{Twitter} & - & 42104 & - & 30 & 2 & -\\
    \hline
    \end{tabular}
    %\vspace*{-10pt}
\caption{{Datasets and Experiment Parameters.}}\label{tab:datasets}\end{center}
\vspace*{-25pt}
% \end{table}
\end{wraptable}
 
\noindent\textbf{Algorithms.} We compare the performance of different estimators. These estimators are: (a) sampling estimator (SAMP) with $N = 1, 10, 20, 100$ %, 1000$
and (b) polynomial estimator (POLY) with $L = 1, 2%, 3
$. %We also vary the number of iterations $T$ of the SCG algorithm where $T = 100, 200, 500, 1000, 2000$.

% \begin{table*}[t] \label{tab:final_estimates}
% \caption{Obtained utilities under different estimators}
% \resizebox{\textwidth}{!}{
%     \centering
%     \begin{tabular}{l r r r r r r} 
%     \hline
%     $\texttt{dataset}$ & SAMP1 & SAMP10 & SAMP20 & SAMP100 & POLY1 & POLY2\\% & \multicolumn{2}{c}{POLY3} \\
%     \hline
%     % \texttt{GreedyTricker} & 0.538 & 0.557 & 0.553 & 0.546 & \textbf{0.571} & 0.546 \\% & - & - \\
%     \hline
%     \texttt{SyntheticBipartitePowerLaw} & 0.049 & 0.049 & 0.049 & - & 0.061 & \textbf{0.062} \\
%     \hline
%     \texttt{ZKC} & 0.324 & 0.326 & 0.324 & 0.327 & 0.318 & \textbf{0.332} \\
%     \hline
%     \texttt{MovieLens} & 0.031 & 0.031 & 0.031 & 0.031 & \textbf{0.051} & - \\
%     \hline
% \end{tabular}}
% \end{table*}

\noindent\textbf{Metrics.} We evaluate the performance of the estimators with their clock running time and via %$f^*$, where $f^* = \max f(\mathbf{y})$ is 
the maximum result ($\max f(\mathbf{y})$) obtained using the best available estimator for a given setting.

\noindent\textbf{Results.} The trajectory of the utility obtained at each iteration of the stochastic continuous greey algorithm $f(\mathbf{y})$ is plotted as a function of time in Fig.~\ref{fig:CGiters}. %In Fig.~\ref{fig:GreedyTricker_loglog} POLY1 outperforms other estimators including the sampling estimators in terms of utility. Moreover, POLY1 is more than $10$ times faster than SAMP20 while it runs in comparable time to SAMP1. 
In Fig.~\ref{fig:SBPL_loglog}, we observe that polynomial estimators outperforms sampling estimators in terms of utility. Moreover, POLY1 runs $10$ times faster than SAMP20 and runs in comparable time to SAMP1. In Fig.~\ref{fig:ZKC_loglog}, POLY2 outperforms all estimators whereas POLY1 slightly underperforms. Finally, in Fig.~\ref{fig:MovieLens_loglog} we observe that POLY1 consistently outperforms sampling estimators.

The final outcomes of the objective functions of the estimators are reported as a function of time in Fig.~\ref{fig:final_estimates}. In Fig.~\ref{fig:SBPL_paretolog} and~\ref{fig:ZKC_paretolog}, POLY2 outperforms other estimators in terms of utility. Again in Fig.~\ref{fig:SBPL_paretolog}, POLY1 outperforms sampling estimators in terms of utility and runs in comparable time to SAMP1 while in Fig.~\ref{fig:MovieLens_paretolog}, POLY1 outperforms sampling estimators both in terms of time and utility. %Highest objective value is highlighted for each example. Based on this table, we can conclude that the polynomial estimators are better choices than the sampling estimators.
Ideally, we would expect the performance of the estimators to improve as the degree of the polynomial or the number of samples increase. The examples where this is not always the case can be explained by the stochastic nature of the problem.


\begin{figure}[t]
\centering
% \subfigure[\texttt{GreedyTricker}]{
% \begin{minipage}{0.46\linewidth}
% \centering
% \includegraphics[width=1\linewidth]{images/GreedyTricker_logtime.eps}
% \centering
% \label{fig:GreedyTricker_loglog}\vspace*{-10pt}
% \end{minipage}
% }
\subfigure[\texttt{SyntheticBipartitePowerLaw}]{
\begin{minipage}{0.31\linewidth}
\centering
\includegraphics[width=1\linewidth]{images/SyntheticBipartitePowerLaw_logtime.eps}
\centering
\label{fig:SBPL_loglog}\vspace*{-10pt}
\end{minipage}
}
% \subfigure[\texttt{SyntheticBipartiteUniform}]{
% \begin{minipage}{0.45\linewidth}
% \centering
% \includegraphics[width=1\linewidth]{images/RB100_uniform_100_100_400_k_2_100_FW_logtime.eps}
% \label{fig:FLsynth1_loglog}\vspace*{-10pt}
% \end{minipage}
% }
\subfigure[\texttt{ZKC}]{
\begin{minipage}{0.31\linewidth}
\centering
\includegraphics[width=1\linewidth]{images/zkc_logtime.eps}
\label{fig:ZKC_loglog}\vspace*{-10pt}
\end{minipage}
}
\subfigure[\texttt{MovieLens}]{
\begin{minipage}{0.31\linewidth}
\centering
\includegraphics[width=1\linewidth]{images/MovieLens_logtime.eps}
\centering
\label{fig:MovieLens_loglog}
\end{minipage}
}
% \subfigure[\texttt{Twitter}]{
% \begin{minipage}{0.45\linewidth}
% \centering
% \includegraphics[width=1\linewidth]{images/emptyplot.png}
% \label{fig:6}
% \end{minipage}
% }
\vspace*{-10pt}
\caption{Trajectory of the FW algorithm. Utility of the function at the current $\vc{y}$ as a function of time is marked for every %$10$th 
iteration.} 
 \vspace*{-13pt}
\label{fig:CGiters}
\end{figure}

\begin{figure}[t]
\centering
% \subfigure[\texttt{GreedyTricker}]{
% \begin{minipage}{0.45\linewidth}
% \centering
% \includegraphics[width=1\linewidth]{images/IM_fooler_bipartite_N_5_k_1_100_FW_paretolog.eps}
% \label{fig:IMsynth1_paretolog}\vspace*{-12pt}
% \end{minipage}
% }
\subfigure[\texttt{SBPL}]{
\begin{minipage}{0.30\linewidth}
\centering
\includegraphics[width=1\linewidth]{images/IM_RB20powerlaw_200_200_914_k_1_100_FW_paretolog.eps}
\label{fig:SBPL_paretolog}\vspace*{-12pt}
\end{minipage}
}
% \subfigure[\texttt{SyntheticBipartiteUniform}]{
% \begin{minipage}{0.45\linewidth}
% \centering
% \includegraphics[width=1\linewidth]{images/RB100uniform_100_100_400_k_2_100_FW_paretolog.eps}
% \label{fig:FLsynth1_paretolog}\vspace*{-12pt}
% \end{minipage}
% }
\subfigure[\texttt{ZKC}]{
\begin{minipage}{0.30\linewidth}
\centering
\includegraphics[width=1\linewidth]{images/ZKC_paretolog.eps}
\label{fig:ZKC_paretolog}\vspace*{-12pt}
\end{minipage}
}
\subfigure[\texttt{MovieLens}]{
\begin{minipage}{0.30\linewidth}
\centering
\includegraphics[width=1\linewidth]{images/MovieLens_paretolog.eps}
\label{fig:MovieLens_paretolog}\vspace*{-12pt}
\end{minipage}
}
% \subfigure[\texttt{Twitter}]{
% \begin{minipage}{0.45\linewidth}
% \centering
% \includegraphics[width=1\linewidth]{images/emptyplot.png}
% \label{fig:SMsynth1_paretolog}\vspace*{-12pt}
% \end{minipage}
% }
\vspace*{-10pt}
\caption{Comparison of different estimators on different problems. Blue lines represent the performance of the POLY estimators and the marked points correspond to POLY1 and POLY2 %, POLY3 
respectively. Orange lines represent the performance of the SAMP estimators and the marked points correspond to SAMP1, SAMP10, SAMP20, SAMP100 respectively.}
%\vspace*{-15pt}
\label{fig:final_estimates}
\end{figure}
This work presented a simulation approach that centers around finding iteratively an approximation of the evolution of the algebraic variables in the power system \glspl{DAE}. The approximation of the dynamic state evolutions by NNs, instead of classical explicit numerical integration schemes, allows larger time-steps to be realized while being fast to execute. This work aimed at providing a proof of concept, it is foreseeable that future work on this method shares many typical questions with established \gls{DAE} solvers, hence, by applying various existing techniques the computational performance and scalability of the approach should improve significantly.



%%%%%%%%% REFERENCES
{\small
\bibliographystyle{ieee_fullname}
\bibliography{egbib}
}

\clearpage
\appendix
\onecolumn
\addcontentsline{toc}{section}{Appendices}

\clearpage

\section{Detailed Local Data Distribution}
\label{sec:dataset_summary}

We adopt a Latent Dirichlet Allocation (LDA) strategy for Non-IID setting \cite{fedma, moon}, where each client $k$ is assigned the partition of classes by sampling $\mathbf{p}_k \sim Dir(\alpha \cdot \mathds{1})$, where $\mathds{1} \in \, \mathbb{R}^C$.
$\alpha$ is a concentration parameter that controls the local heterogeneity level. The smaller $\alpha$, the more heterogeneous data distribution.
Since we consider a fairness issue in the FAL framework, the total number of samples should be equally partitioned for all clients.
Therefore, we made a doubly stochastic matrix $P = [\tilde{\mathbf{p}}_1, \dots, \tilde{\mathbf{p}}_K]^\top$ by scaling $\mathbf{p}_k$ to $\tilde{ \mathbf{p}}_k$, when the number of client and class are same (i.e., $P$ is a square matrix).
Note that we set the sum of columns and rows to the proper values for a non-square matrix.
We visualized the examples of CIFAR-10 when the clients $K=10$ in Figure \ref{fig:data_dist}.

\begin{figure}[h!]
\centering
\begin{minipage}{0.94\linewidth}
    \begin{subfigure}[b]{0.32\linewidth}
    \includegraphics[width=\linewidth]{figure/data_distribution/rho1/dir0.1_rho1.pdf}
    \caption{\small {$\rho$ = 1 and $\alpha$ = 0.1}}
    \end{subfigure}
    \hfill
    \begin{subfigure}[b]{0.32\linewidth}
    \includegraphics[width=\linewidth]{figure/data_distribution/rho1/dir1_rho1.pdf}
    \caption{\small {$\rho$ = 1 and $\alpha$ = 1.0}}
    \end{subfigure}
    \hfill
    \begin{subfigure}[b]{0.32\linewidth}
    \includegraphics[width=\linewidth]{figure/data_distribution/rho1/dir10000_rho1.pdf}
    \caption{\small {$\rho$ = 1 and $\alpha$ = $\infty$}}
    \end{subfigure}
    \begin{subfigure}[b]{0.32\linewidth}
    \centering
    \includegraphics[width=\linewidth]{figure/data_distribution/rho5/dir0.1_rho5.pdf}
    \caption{\small {$\rho$ = 5 and $\alpha$ = 0.1}}
    \end{subfigure}
    \hfill
    \begin{subfigure}[b]{0.32\linewidth}
    \centering
    \includegraphics[width=\linewidth]{figure/data_distribution/rho5/dir1_rho5.pdf}
    \caption{\small {$\rho$ = 5 and $\alpha$ = 1.0}}
    \end{subfigure}
    \hfill
    \begin{subfigure}[b]{0.32\linewidth}
    \centering
    \includegraphics[width=\linewidth]{figure/data_distribution/rho5/dir10000_rho5.pdf}
    \caption{\small {$\rho$ = 5 and $\alpha$ = $\infty$}}
    \end{subfigure}
    \begin{subfigure}[b]{0.32\linewidth}
    \centering
    \includegraphics[width=\linewidth]{figure/data_distribution/rho10/dir0.1_rho10.pdf}
    \caption{\small {$\rho$ = 10 and $\alpha$ = 0.1}}
    \end{subfigure}
    \hfill
    \begin{subfigure}[b]{0.32\linewidth}
    \centering
    \includegraphics[width=\linewidth]{figure/data_distribution/rho10/dir1_rho10.pdf}
    \caption{\small {$\rho$ = 10 and $\alpha$ = 1.0}}
    \end{subfigure}
    \hfill
    \begin{subfigure}[b]{0.32\linewidth}
    \centering
    \includegraphics[width=\linewidth]{figure/data_distribution/rho10/dir10000_rho10.pdf}
    \caption{\small {$\rho$ = 10 and $\alpha$ = $\infty$}}
    \end{subfigure}
    \begin{subfigure}[b]{0.32\linewidth}
    \centering
    \includegraphics[width=\linewidth]{figure/data_distribution/rho20/dir0.1_rho20.pdf}
    \caption{\small {$\rho$ = 20 and $\alpha$ = 0.1}}
    \end{subfigure}
    \hfill
    \begin{subfigure}[b]{0.32\linewidth}
    \centering
    \includegraphics[width=\linewidth]{figure/data_distribution/rho20/dir1_rho20.pdf}
    \caption{\small {$\rho$ = 20 and $\alpha$ = 1.0}}
    \end{subfigure}
    \hfill
    \begin{subfigure}[b]{0.32\linewidth}
    \centering
    \includegraphics[width=\linewidth]{figure/data_distribution/rho20/dir10000_rho20.pdf}
    \caption{\small {$\rho$ = 20 and $\alpha$ = $\infty$}}
    \end{subfigure}
\end{minipage}
\caption{Visualization of the client data distribution on CIFAR-10. Each color represents a different class. The higher $\rho$ denotes the more global imbalanced distribution. The higher $\alpha$ denotes the more locally balanced data.}
\label{fig:data_dist}
\end{figure}

\clearpage

\section{Detailed Analysis Results}
\label{sec:detail_analysis}


\subsection{Detailed Matrices for Data Counts and Accuracy}
\label{sec:detail_analysis_matrices}

We summarized the detailed matrices for the combinations of $\rho$ = $\{$1, 5, 10, 20$\}$ and $\alpha$ = $\{$0.1, 1.0, $\infty \}$.

\begin{figure*}[!h]
\centering
\begin{minipage}{0.8\linewidth}
    \centering
    \begin{subfigure}[b]{0.32\linewidth}
    \centering
    \includegraphics[width=\linewidth]{figure/analysis/obs2/appendix/rho1_alpha01_total.pdf}
    \caption{$\rho$ = 1 and $\alpha$ = 0.1}         
    \end{subfigure}
    \hfill
    \centering
    \begin{subfigure}[b]{0.32\linewidth}
    \centering
    \includegraphics[width=\linewidth]{figure/analysis/obs2/appendix/rho1_alpha1_total_v.pdf}
    \caption{$\alpha$ = 1.0}       
    \end{subfigure}
    \hfill
    \begin{subfigure}[b]{0.32\linewidth}
    \centering
    \includegraphics[width=\linewidth]{figure/analysis/obs2/appendix/rho1_alpha_inf_total.pdf}
    \caption{$\alpha$ = $\infty$}     
    \end{subfigure}
    \vspace*{-0.2cm}
\end{minipage}
\caption{Matrices of data count (top) and class-wise accuracy (down) when $\rho$ = 1.}
\vspace*{-0.2cm}
\label{fig:cnt_acc_rho1}
\end{figure*}

\vspace{-10pt}
\begin{figure*}[!h]
\centering
\begin{minipage}{0.8\linewidth}
    \centering
    \begin{subfigure}[b]{0.32\linewidth}
    \centering
    \includegraphics[width=\linewidth]{figure/analysis/obs2/appendix/rho5_alpha01_total.pdf}
    \caption{$\alpha$ = 0.1}         
    \end{subfigure}
    \hfill
    \centering
    \begin{subfigure}[b]{0.32\linewidth}
    \centering
    \includegraphics[width=\linewidth]{figure/analysis/obs2/appendix/rho5_alpha1_total.pdf}
    \caption{$\alpha$ = 1.0}       
    \end{subfigure}
    \hfill
    \begin{subfigure}[b]{0.32\linewidth}
    \centering
    \includegraphics[width=\linewidth]{figure/analysis/obs2/appendix/rho5_alpha_inf_total.pdf}
    \caption{$\alpha$ = $\infty$}     
    \end{subfigure}
    \vspace*{-0.2cm}
\end{minipage}
\caption{Matrices of data count (top) and class-wise accuracy (down) when $\rho$ = 5.}
\vspace*{-0.2cm}
\label{fig:cnt_acc_rho5}
\end{figure*}

\newpage

\begin{figure*}[!h]
\centering
\begin{minipage}{0.8\linewidth}
    \centering
    \begin{subfigure}[b]{0.32\linewidth}
    \centering
    \includegraphics[width=\linewidth]{figure/analysis/obs2/appendix/rho10_alpha01_total.pdf}
    \caption{$\alpha$ = 0.1}         
    \end{subfigure}
    \hfill
    \centering
    \begin{subfigure}[b]{0.32\linewidth}
    \centering
    \includegraphics[width=\linewidth]{figure/analysis/obs2/appendix/rho10_alpha1_total.pdf}
    \caption{$\alpha$ = 1.0}       
    \end{subfigure}
    \hfill
    \begin{subfigure}[b]{0.32\linewidth}
    \centering
    \includegraphics[width=\linewidth]{figure/analysis/obs2/appendix/rho10_alpha_inf_total.pdf}
    \caption{$\alpha$ = $\infty$}     
    \end{subfigure}
    \vspace*{-0.2cm}
\end{minipage}
\caption{Matrices of data count (top) and class-wise accuracy (down) when $\rho$ = 10.}
\vspace*{-0.2cm}
\label{fig:cnt_acc_rho10}
\end{figure*}


\begin{figure*}[!h]
\centering
\begin{minipage}{0.8\linewidth}
    \centering
    \begin{subfigure}[b]{0.32\linewidth}
    \centering
    \includegraphics[width=\linewidth]{figure/analysis/obs2/appendix/rho20_alpha_01_total.pdf}
    \caption{$\alpha$ = 0.1}         
    \end{subfigure}
    \hfill
    \centering
    \begin{subfigure}[b]{0.32\linewidth}
    \centering
    \includegraphics[width=\linewidth]{figure/analysis/obs2/appendix/rho20_alpha_1_total.pdf}
    \caption{$\alpha$ = 1.0}       
    \end{subfigure}
    \hfill
    \begin{subfigure}[b]{0.32\linewidth}
    \centering
    \includegraphics[width=\linewidth]{figure/analysis/obs2/appendix/rho20_alpha_inf_total.pdf}
    \caption{$\alpha$ = $\infty$}     
    \end{subfigure}
    \vspace*{-0.2cm}
\end{minipage}
\caption{Matrices of data count (top) and class-wise accuracy (down) when $\rho$ = 20.}
\vspace*{-0.2cm}
\label{fig:cnt_acc_rho20}
\end{figure*}

\newpage
\subsection{Detailed Earth Mover Distance}
\label{sec:detail_analysis_emd}
Table \ref{tab:detail_emd} \ summarizes the detailed local and global EMD for the combinations of $\rho$ = $\{$1, 5, 10, 20$\}$ and $\alpha$ = $\{$0.1, 1.0, $\infty \}$.

\begin{table}[H]
\centering
\small
\renewcommand*{\arraystretch}{0.85}
\addtolength{\tabcolsep}{1pt}
\resizebox{0.8\linewidth}{!}{
\begin{tabular}{cc|c|ccccc|ccccc}
\toprule
\multirow{2}{*}{$\rho$} &
  \multirow{2}{*}{$\alpha$} &
  \multirow{2}{*}{model} &
  \multicolumn{5}{c|}{Local EMD $(\downarrow)$} &
  \multicolumn{5}{c}{Global EMD $(\downarrow)$} \\
\cmidrule(l{2pt}r{2pt}){4-8} \cmidrule(l{2pt}r{2pt}){8-13}
                   &                           &   & 10\%  & 20\%  & 30\%  & 40\%  & 50\%  & 10\%  & 20\%  & 30\%  & 40\%  & 50\%  \\
                   \midrule
\multirow{2}{*}{1} & \multirow{2}{*}{0.1}      & G & 0.632 & 0.638 & 0.641 & 0.643 & 0.646 & 0.019 & 0.064 & 0.086 & 0.095 & 0.091 \\
                   &                           & L & 0.632 & 0.597 & 0.592 & 0.595 & 0.601 & 0.019 & 0.050 & 0.050 & 0.046 & 0.055 \\ \midrule
\multirow{2}{*}{1} & \multirow{2}{*}{1.0}      & G & 0.297 & 0.297 & 0.300 & 0.300 & 0.300 & 0.017 & 0.066 & 0.079 & 0.084 & 0.083 \\
                   &                           & L & 0.297 & 0.248 & 0.232 & 0.235 & 0.241 & 0.017 & 0.053 & 0.065 & 0.068 & 0.074 \\ \midrule
\multirow{2}{*}{1} & \multirow{2}{*}{$\infty$} & G & 0.049 & 0.077 & 0.070 & 0.065 & 0.061 & 0.014 & 0.070 & 0.066 & 0.063 & 0.060 \\
                   &                           & L & 0.049 & 0.042 & 0.054 & 0.059 & 0.066 & 0.014 & 0.025 & 0.044 & 0.053 & 0.062 \\  \midrule
% Rho = 5

\multirow{2}{*}{5} & \multirow{2}{*}{0.1}      & G & 0.662 & 0.663 & 0.666 & 0.666 & 0.669 & 0.211 & 0.201 & 0.196 & 0.194 & 0.195 \\ 
                   &                           & L & 0.662 & 0.628 & 0.627 & 0.628 & 0.634 & 0.211 & 0.232 & 0.232 & 0.236 & 0.228 \\ \midrule
\multirow{2}{*}{5} & \multirow{2}{*}{1.0}      & G & 0.402 & 0.391 & 0.387 & 0.388 & 0.389 & 0.206 & 0.188 & 0.180 & 0.173 & 0.169 \\
                   &                           & L & 0.402 & 0.309 & 0.306 & 0.306 & 0.341 & 0.206 & 0.200 & 0.201 & 0.196 & 0.196 \\ \midrule
\multirow{2}{*}{5} & \multirow{2}{*}{$\infty$} & G & 0.213 & 0.190 & 0.178 & 0.168 & 0.165 & 0.206 & 0.185 & 0.174 & 0.162 & 0.163 \\
                   &                           & L & 0.213 & 0.179 & 0.176 & 0.180 & 0.180 & 0.206 & 0.176 & 0.173 & 0.178 & 0.180 \\ \midrule

\multirow{2}{*}{10} & \multirow{2}{*}{0.1}      & G & 0.692 & 0.685 & 0.687 & 0.685 & 0.685 & 0.280 & 0.268 & 0.267 & 0.265 & 0.267 \\
                    &                           & L & 0.692 & 0.652 & 0.650 & 0.654 & 0.660 & 0.280 & 0.270 & 0.277 & 0.282 & 0.281 \\ \midrule
\multirow{2}{*}{10} & \multirow{2}{*}{1.0}      & G & 0.491 & 0.463 & 0.459 & 0.456 & 0.455 & 0.297 & 0.263 & 0.247 & 0.244 & 0.242 \\
                    &                           & L & 0.491 & 0.408 & 0.402 & 0.405 & 0.415 & 0.297 & 0.256 & 0.257 & 0.255 & 0.255 \\ \midrule
\multirow{2}{*}{10} & \multirow{2}{*}{$\infty$} & G & 0.315 & 0.240 & 0.229 & 0.223 & 0.222 & 0.303 & 0.237 & 0.226 & 0.222 & 0.221 \\
                    &                           & L & 0.315 & 0.238 & 0.237 & 0.239 & 0.240 & 0.303 & 0.237 & 0.234 & 0.237 & 0.239 \\ \midrule

\multirow{2}{*}{20} & \multirow{2}{*}{0.1}      & G & 0.692 & 0.680 & 0.676 & 0.674 & 0.677 & 0.377 & 0.300 & 0.294 & 0.294 & 0.298 \\
                    &                           & L & 0.692 & 0.641 & 0.633 & 0.636 & 0.644 & 0.377 & 0.304 & 0.326 & 0.321 & 0.323 \\ \midrule
\multirow{2}{*}{20} & \multirow{2}{*}{1.0}      & G & 0.481 & 0.455 & 0.450 & 0.448 & 0.448 & 0.374 & 0.311 & 0.300 & 0.295 & 0.292 \\
                    &                           & L & 0.481 & 0.448 & 0.437 & 0.431 & 0.437 & 0.374 & 0.354 & 0.342 & 0.303 & 0.304 \\ \midrule
\multirow{2}{*}{20} & \multirow{2}{*}{$\infty$} & G & 0.371 & 0.298 & 0.284 & 0.274 & 0.276 & 0.368 & 0.294 & 0.282 & 0.271 & 0.272 \\
                    &                           & L & 0.371 & 0.313 & 0.293 & 0.290 & 0.289 & 0.368 & 0.309 & 0.287 & 0.288 & 0.289 \\
                   
                   
                   
                   \bottomrule
\end{tabular}}
    \caption{Local and global EMD on CIFAR-10 for 12 combinations of $\rho$ = $\{$1, 5, 10, 20$\}$ and $\alpha$ = $\{$0.1, 1.0, $\infty \}$.}
    \label{tab:detail_emd}
\end{table}

\clearpage
\section{Pseudo Algorithm of LoGo}
\label{sec:pseudo_algorithm}


Algorithm\,\ref{alg:logo} is the overall pipeline of the FAL framework.
Specifically, we summarize the detailed pseudocode of our \algname{} algorithm.

\begin{algorithm}[h]
\small
\caption{FAL framework with \algname{} algorithm}
\label{alg:logo}
\textbf{Input}: initialized parameter $\Theta$; unlabeled data $U^{\scaleto{1}{4pt}}$; sampling strategy $\mathcal{A}$; labeling budget $B$; clients number $K$; AL round $R$; \\
\textbf{Output}: trained parameter $\Theta^{\scaleto{R*}{4pt}}$ \\
\\
\textbf{\# Alternating AL and FL Procedure}
\begin{algorithmic}[1]
\FOR{$k=1, \dots, K$}
\STATE Randomly sample $L_{\scaleto{k}{4pt}}^{\scaleto{1}{4pt}}= \{ x_{\scaleto{1}{4pt}}, \dots, x_{\scaleto{B}{4pt}} \}$ from $U_{\scaleto{k}{4pt}}^{\scaleto{1}{4pt}}$, and $U_k^{2} = U_k^1 \setminus L_k^1$
\STATE Get the labeled set $D_{\scaleto{k}{4pt}}^{\scaleto{1}{4pt}}$\, from the oracles
\ENDFOR
\STATE $\Theta^{\scaleto{1*}{4pt}}=$\,\texttt{FedAvg}\,($\Theta$, $D^{\scaleto{1}{4pt}}, K$) \\
\, \\
\FOR{$r=2, \dots, R$}
\FOR{$k=1, \dots, K$}
\STATE $D^{r}_k, \,U_k^{r+1}=$\,\,\texttt{LoGo}\,($\Theta^{(r-1)*}$, $D^{r-1}_{\scaleto{k}{4pt}}, U_k^r$)
\ENDFOR
\STATE $\Theta^{r*}=$\,\texttt{FedAvg}\,($\Theta$, $D^{r}, K$)
\ENDFOR
\end{algorithmic}
\, \\
\textbf{Function}\,\,\texttt{LoGo}:
\begin{algorithmic}[1] 
\STATE \textbf{\# Macro Step}
\STATE 
Train a local-only model $\Theta^{(r-1)}_{k*}$ from the scratch only using $D_k^{r-1}$
\STATE  For each $x\in U_k^r$, calculate the gradient embedding $g_{\hat{y}}^x$  by Eq.\,\eqref{eq:gradient}
\STATE Cluster $U_k^r$ into $B$ clusters($\mathcal{C}_1,...,\mathcal{C}_B$) by Eq.\,\eqref{eq:kmeans} \\
\, \\
\STATE \textbf{\# Micro Step}
\STATE 
$L_k^r= \emptyset $
\FOR{$\mathcal{C}_{\scaleto{i}{4pt}}=\mathcal{C}_{\scaleto{1}{4pt}}, \dots, \mathcal{C}_{\scaleto{B}{4pt}}$}
\STATE $L_k^r = L_k^r \cup \{ \mathcal{A}(\mathcal{C}_{\scaleto{i}{4pt}}, \Theta^{(r-1)*}, 1) \}$
\STATE $D_k^r = D_k^{r-1} \cup D_k^r$\, and\, $U_{\scaleto{k}{4pt}}^{\scaleto{r+1}{4pt}} = U_{\scaleto{k}{4pt}}^{\scaleto{r}{3pt}} \setminus L_k^r$
\ENDFOR 
\STATE \textbf{return}  $D_k^r$, \,$U_k^{r+1}$
\end{algorithmic}
\, \\
\textbf{Function}\,\,\texttt{FedAvg}:
\begin{algorithmic}[1]
\FOR{$\,FL\,\,round$\,}
\STATE Distribute $\Theta$ to the all client
\FOR{$k = 1, \dots, K$}
\STATE Train $\Theta_k$ on $D_k^r$ by minimizing $\mathbb{E}_{D_k^r}[\ell(x,y; \Theta_k)]$
\ENDFOR
\STATE $\Theta = (\sum_k \Theta_k) / K$
\ENDFOR
\STATE \textbf{return} $\Theta$
\end{algorithmic}

\end{algorithm}

\clearpage
\section{Experimental Settings}
\label{sec:exp_settings}

\subsection{Datasets}
We mainly experimented on two natural image datasets (CIFAR-10\footnote{https://www.cs.toronto.edu/~kriz/cifar.html}, SVHN\footnote{http://ufldl.stanford.edu/housenumbers}) and three medical image datasets\footnote{https://medmnist.com/} (PathMNIST, DermaMNIST, OrganAMNIST). 
Table \ref{tab:dataset_summray} provides a summary of the five datasets.
For the details of partitioning data to each client, please refer to Appendix\,\ref{sec:dataset_summary}.

\begin{table*}[t]
\caption{Details of the evaluated datasets.}
\label{tab:dataset-summary}
\vskip 0.15in
\centering
\begin{tabular}{@{}lllllll@{}}
\toprule
             & \begin{tabular}[c]{@{}l@{}}Number of\\ Series\end{tabular} & \begin{tabular}[c]{@{}l@{}}Time Steps\\ per Series\end{tabular} & Period  & Sampling & \begin{tabular}[c]{@{}l@{}}Number of\\ Features\end{tabular} & Data Split [\%] \\ \midrule
\solarSmall  & 8                                                          & 2,000                                                            & 01--03 2018    & 60m      & 8                                                            & 60/15/25   \\ 
\solarOneY   & 50                                                         & 8,760                                                            & 2019    & 60m      & 8                                                            & 60/15/25   \\ 
\solarThreeY & 50                                                         & 26,304                                                           & 2018--20 & 60m      & 8                                                            & 34/33/33   \\ \midrule
\airTen      & 12                                                         & 35,064                                                           & 2013--17 & 60m      & 11                                                           & 34/33/33   \\ 
\airTwenty   & 12                                                         & 35,064                                                           & 2013--17 & 60m      & 11                                                           & 34/33/33   \\ \midrule
\sapflux     & 24                                                         & \begin{tabular}[c]{@{}l@{}}15,000--\\ 20,000\end{tabular}          & 2008--16 & varying  & 10                                                           & 34/33/33   \\ \midrule
\hydro       & 531                                                        & 9,862                                                          & 1981--2008 & 24h      & 41                                                           & 34/33/33   \\ \bottomrule
\end{tabular}
\end{table*}


\subsection{Implementation Details}
For the FL training pipeline, we set the number of FL rounds to 100 and local update epochs to 5.
We used a SGD optimizer with the initial learning rate of 0.01 and the momentum of 0.9.
The learning rate was decayed by 0.1 at half and three-quarters of federated learning rounds to ensure convergence, and we used a random horizontal flipping as data augmentation.
For training local-only models, we trained the model using the aforementioned settings for 50 epochs. However, the training was terminated if the training accuracy reached 99\%.
It should be noted that we averaged the classification accuracy of the last 5 epochs in each round and repeated all experiments with four different seeds.
All algorithms were implemented using PyTorch 1.11.0 and executed using NVIDIA RTX 3080 GPUs.




\subsection{Experimental Categories}
A total of six categories were considered in the evaluation:
\begin{enumerate}
\item {`Query selector'} of whether to use a local-only or global model with the six compared strategies. 
\item  {`Heterogeneity level'} of varying degree of class imbalance. We adopt a Latent Dirichlet Allocation (LDA) \cite{moon} strategy. 
For example, the smaller $\alpha$, the more heterogeneous the data distribution. 
\item  {`Imbalance ratio'} of used datasets. 
We classified five datasets for evaluation based on the imbalance ratio $\rho$.
CIFAR-10 and PathMNIST belong to a low imbalance ratio ($\rho<2$), and SVHN, DermaMNIST, and OrganAMNIST belong to a high imbalance ratio ($\rho\geq 2$).
\item  {`Model architecture.'} We employed four layers of convolution neural network for a base architecture and also experimented with ResNet-18 \cite{resnet} and MobileNet \cite{mobilenet}. 
\item  `{Budget size}' for labeling. We tested small (1\%), medium (5\%), and large (20\%) budget sizes for each round. 
\item  {`Model initialization'} of either learning from scratch (random) or from the checkpoint of the previous AL round (continue). \looseness=-1
\end{enumerate}

\newpage
\subsection{Combination of experimental settings}
We compared our algorithms and baselines in 38 comprehensive experimental settings, which are the combinations of the aforementioned six categories.
All the experimental combinations we performed are summarized in Table \ref{tab:setting_summary}.

\begin{table*}[h!]
\centering
\small
\renewcommand*{\arraystretch}{1}
\addtolength{\tabcolsep}{1pt}
\resizebox{0.75\linewidth}{!}{
\begin{tabular}{cccccc}
\toprule
Query Selelctor   & Dir($\alpha$) & Data Type     & Model Arch.   & Budget Size & Model Init. \\ 
\midrule
Global          & 0.1 & CIFAR-10     & 4CNN   & 5\%         & Random     \\ 
Global          & 0.1 & SVHN        & 4CNN   & 5\%         & Random     \\ 
Global          & 0.1 & PathMNIST   & 4CNN   & 5\%         & Random     \\ 
Global        & 0.1  & OrganAMNIST & 4CNN & 5\%         & Random     \\ 
Global        & 0.1  & DermaMNIST & 4CNN   & 5\%         & Random     \\ 
Global        & 1   & CIFAR-10      & 4CNN   & 5\%         & Random     \\ 
Global        & 1   & SVHN         & 4CNN    & 5\%         & Random     \\ 
Global        & $\infty$   & CIFAR-10     & 4CNN   & 5\%         & Random     \\ 
Global        & $\infty$   & SVHN        & 4CNN   & 5\%         & Random     \\ 
Global        & 0.1  & CIFAR-10      & 4CNN  & 5\%         & Continue   \\ 
Global        & 0.1  & SVHN         & 4CNN   & 5\%         & Continue   \\ 
Global        & 0.1  & CIFAR-10      & ResNet-18 & 5\%         & Random     \\ 
Global      & 0.1    & SVHN       & ResNet-18  & 5\%         & Random     \\ 
Global        & 0.1  & CIFAR-10      & MobileNet  & 5\%         & Random     \\ 
Global        & 0.1  & SVHN         & MobileNet  & 5\%         & Random     \\ 
Global        & 0.1  & CIFAR-10      & 4CNN   & 1\%         & Random     \\ 
Global        & 0.1  & SVHN         & 4CNN   & 1\%         & Random     \\ 
Global        & 0.1  & CIFAR-10      & 4CNN   & 20\%        & Random     \\ 
Global        & 0.1  & SVHN         & 4CNN   & 20\%        & Random     \\ 
Local-only    & 0.1  & CIFAR-10      & 4CNN  & 5\%         & Random     \\ 
Local-only    & 0.1  & SVHN         & 4CNN   & 5\%         & Random     \\ 
Local-only    & 0.1  & PathMNIST    & 4CNN   & 5\%         & Random     \\ 
Local-only    & 0.1  & OrganAMNIST  & 4CNN   & 5\%         & Random     \\ 
Local-only   & 0.1   & DermaMNIST   & 4CNN & 5\%         & Random     \\ 
Local-only      & 1  & CIFAR-10    & 4CNN    & 5\%         & Random     \\ 
Local-only     & 1   & SVHN       & 4CNN   & 5\%         & Random     \\ 
Local-only     & $\infty$  & CIFAR-10     & 4CNN   & 5\%         & Random     \\ 
Local-only     & $\infty$  & SVHN       & 4CNN  & 5\%         & Random     \\ 
Local-only      & 0.1 & CIFAR-10    & 4CNN   & 5\%         & Continue   \\ 
Local-only     & 0.1  & SVHN        & 4CNN  & 5\%         & Continue   \\ 
Local-only      & 0.1 & CIFAR-10    & ResNet-18    & 5\%         & Random     \\ 
Local-only     & 0.1  & SVHN        & ResNet-18 & 5\%         & Random     \\ 
Local-only     & 0.1  & CIFAR-10     & MobileNet  & 5\%         & Random     \\ 
Local-only     & 0.1  & SVHN         & MobileNet  & 5\%         & Random     \\ 
Local-only     & 0.1  & CIFAR-10      & 4CNN  & 1\%         & Random     \\ 
Local-only     & 0.1  & SVHN         & 4CNN   & 1\%         & Random     \\ 
Local-only     & 0.1  & CIFAR-10      & 4CNN   & 20\%        & Random     \\ 
Local-only     & 0.1  & SVHN        & 4CNN   & 20\%        & Random     \\ 
\bottomrule
\end{tabular}}
\caption{Summary of the entire experimental combinations.}
\label{tab:setting_summary}
\end{table*}



\clearpage
\section{Computational Cost of Query Selection}
\label{sec:computational_cost}
In Table\,\ref{tab:time_cost}, we measured the wallclock time for various combinations of the algorithm, query selector, and labeling ratio.
We confirmed that as the percentage of labeled data increases, the time required to measure the importance score with the global model decreases due to the reduced amount of unlabeled data.
Conversly, the local-only model takes more time as it requires training on a larger number of labeled samples.
Our LoGo algorithm shows a comparable computational cost to the baselines that use the local-only model\,(L) for query selection.
Note that we used a simple Entropy sampling within LoGo algorithm to measure the uncertainty, and the only possible bottleneck is \textit{k}-means clustering in the Macro step.


\begin{table}[h!]
\centering
\addtolength{\tabcolsep}{-1pt}
\renewcommand*{\arraystretch}{1.05}
\resizebox{0.7\linewidth}{!}{
\begin{tabular}{l|cc|cc|cc|cc|cc|c}
\toprule
                   & \multicolumn{2}{c|}{\!\!\!Entropy\!\!\!} & \multicolumn{2}{c|}{Coreset} & \multicolumn{2}{c|}{BADGE} & \multicolumn{2}{c|}{GCNAL} & \multicolumn{2}{c|}{ALFA-Mix} & LoGo \\
 \cmidrule(l{2pt}r{2pt}){2-3} \cmidrule(l{2pt}r{2pt}){4-5} \cmidrule(l{2pt}r{2pt}){6-7} \cmidrule(l{2pt}r{2pt}){8-9} \cmidrule(l{2pt}r{2pt}){10-11} \cmidrule(l{2pt}r{2pt}){12-12}
\multirow{-2.5}{*}{Query ratio} & G & L & G & L & G & L & G & L & G & L & G,\,L \\ \midrule
5\%\,$\rightarrow$\,10\% & 5.99 \!  & \! 8.85 & 7.32 & 10.24 & 14.43 \!  & \! 17.36 & 8.20 & 11.13 & 13.88 \! & 20.87 & 17.10 \\ \hline
40\%\,$\rightarrow$\,45\% & 4.17 \!  & \! 33.59 & 7.02 & 33.99 & 10.01 \!  & \! 39.11 & 8.11 & 35.46 & 11.94 \! & 41.99 & 37.42 \\ \hline
75\%\,$\rightarrow$\,80\% & 3.95 \!  & \! 59.57 & 6.72 & 58.98 & 3.95 \!  & \! 62.62 & 7.71 & 60.26 & 10.46 \! & 65.16 & 56.81 \\
\bottomrule
\end{tabular}
}
\caption{Computational cost on CIFAR-10 with 4 layers of CNN. We averaged the query selection time\,(sec.) of all 10 clients, measured on a RTX 3090 GPU.}
\label{tab:time_cost}
\end{table}

\section{LoGo with Various FL Methods}
\label{sec:various_fl_algo}
We have further experimented with two federated learning algorithms, FedProx\cite{fedprox} and SCAFFOLD\cite{scaffold}, in conjunction with AL strategies.
Specifically, we compared our LoGo with baselines that demonstrated Top-1 or Top-2 performance more than once in Table\,\ref{tab:acc_comparision}. The experimental configurations are same to those used in Table\,\ref{tab:acc_comparision}.
As summarized in Table\,\ref{tab:compare_fedprox_scaffold}, LoGo consistently outperforms the baselines for both federated learning algorithms.
This observation suggests that LoGo is an orthogonal selection algorithm that can be integrated with any federated learning algorithm, having potential to improve the performance in various applications.

\begin{table}[h]
    \small
    \centering
    \renewcommand*{\arraystretch}{0.97}
    \resizebox{0.7\linewidth}{!}{
    \begin{tabular}{l|l|c|ccc|ccc}
        \toprule
         & &  & \multicolumn{3}{c|}{CIFAR-10} & \multicolumn{3}{c}{SVHN} \\
         \cmidrule(l{2pt}r{2pt}){4-6} \cmidrule(l{2pt}r{2pt}){7-9}
        \multirow{-2.5}{*}{FL\,algo.} & \multirow{-2.5}{*}{Method} & \multirow{-2.5}{*}{Model} & 20\% & 40\% & 60\% & 20\% & 30\% & 40\% \\
        \midrule
         \multirow{7}{*}{FedProx} & & G & 62.89 & 67.52 & 70.38 & 82.22 & 84.34 & 85.42  \\
         & \multirow{-2}{*}{Entropy} & L  & \underline{65.72} & \underline{70.57} & \underline{72.42} &82.08 & 83.73 & 85.30  \\
         \cline{2-9}
         & & G & 64.16 & 68.62 & 70.82 & \underline{83.09} & \textbf{84.65} & 85.84  \\
         & \multirow{-2}{*}{BADGE} & L  & 65.54 & 70.56 & 72.30 &81.99 & 84.17 & 85.17  \\
         \cline{2-9}
         & & G & 63.77 & 68.34 & 70.78 & 82.63 & 84.48 & \underline{85.94}  \\
         & \multirow{-2}{*}{ALFA-Mix} & L  & 63.44 & 67.83 & 70.31 &80.71 & 82.81 & 84.22  \\
         \cline{2-9}
          & \textbf{LoGo} & G, L & \textbf{65.79} & \textbf{70.61} & \textbf{72.61} & \textbf{83.12} & \underline{84.61} & \textbf{86.09}  \\
         \midrule
         \multirow{7}{*}{SCAFFOLD} & & G & 65.58 & 70.37 & 72.52 & 82.75 & 85.69 & 86.48  \\
         & \multirow{-2}{*}{Entropy} & L  & 67.96 & \underline{72.67} & \underline{74.06} & 83.24 & 84.30 & 85.82  \\
         \cline{2-9}
         & & G & 66.33 & 70.68 & 72.79 & 83.80 & 84.72 & \textbf{86.93}  \\
         & \multirow{-2}{*}{BADGE} & L  & \underline{68.27} & 72.52 & 73.79 & 83.40 & 84.61 & 86.16  \\
         \cline{2-9}
         & & G & 66.11 & 70.50 & 72.55 & \underline{84.11} & \textbf{85.72} & 86.14  \\
         & \multirow{-2}{*}{ALFA-Mix} & L  & 66.11 & 70.00 & 71.91 & 82.15 & 82.89 & 84.74  \\
         \cline{2-9}
          & \textbf{LoGo} & G, L & \textbf{68.33} & \textbf{72.77} & \textbf{74.48} & \textbf{84.29} & \underline{85.70} & \underline{86.73}  \\
         \bottomrule
    \end{tabular}}
    \caption{Classification accuracy on two benchmarks with FedProx\,($\mu$\,=\,0.01) and SCAFFOLD. We compared to three overwhelming baselines and averaged three random seeds. \textbf{Bold} and \underline{underline} mean Top-1 and Top-2, respectively.}
    \label{tab:compare_fedprox_scaffold}
    \vspace{-0.37cm}
\end{table}



\clearpage
\section{Detailed Experimental Results}
\label{sec:exp_detail_results}

In this Section, \ref{sec:detail_comp_matrix} summarizes all the comparison matrices results based on six categories: query selector, heterogeneity level, imbalance ratio, model architecture, budget size, and model initialization in Figure\,\ref{fig:bar_graph}. Figure\,\ref{fig:comp_selector}--\ref{fig:comp_model_init} are breakdowns of the matrix in Figure\,\ref{fig:comparision_matrix} into six categories.
\ref{sec:detail_comp_performance} provides comprehensive line plots for 38 experimental settings.
It can be seen that \algname{} overwhelms the baselines in most cases at both each category and detailed experimental setting level.


\subsection{Detailed Penalty Comparision Matrix}
\label{sec:detail_comp_matrix}

A maximum value of each matrix corresponds to Table\,\ref{tab:setting_summary}, and the bar plots in Figure\,\ref{fig:bar_graph} are calculated from these matrices.

\begin{figure*}[htb]
\centering
    \begin{subfigure}[b]{0.3\linewidth}
    \raggedleft
    \includegraphics[width=\linewidth]{figure/experiments/comparison/query_selector/global_comp.pdf}
    \caption{Global}
    \end{subfigure}
    \hspace{10pt}
    \begin{subfigure}[b]{0.3\linewidth}
    \raggedright
    \includegraphics[width=\linewidth]{figure/experiments/comparison/query_selector/local.pdf}
    \caption{Local-only}
    \end{subfigure}
    \caption{Pairwise penalty matrix for a query selector category. The maximum value of both matrices is 19.}
    \label{fig:comp_selector}
\end{figure*}

\vspace{-15pt}
\begin{figure*}[htb]
    \centering
    \begin{subfigure}[b]{0.3\linewidth}
    \includegraphics[width=\linewidth]{figure/experiments/comparison/dir/dir0_1.pdf}
    \caption{$\alpha = 0.1$}
    \end{subfigure}
    \hfill
    \begin{subfigure}[b]{0.3\linewidth}
    \includegraphics[width=\linewidth]{figure/experiments/comparison/dir/dir1_0.pdf}
    \caption{$\alpha = 1.0$}
    \end{subfigure}
    \hfill
    \begin{subfigure}[b]{0.3\linewidth}
    \includegraphics[width=\linewidth]{figure/experiments/comparison/dir/dir10_0.pdf}
    \caption{$\alpha = \infty$}
    \end{subfigure}
    \caption{Pairwise penalty matrix for a heterogeneity level category. The maximum value of three matrices is 30, 4, and 4.}
    \label{fig:comp_hetero}
\end{figure*}


\vspace{-15pt}
\begin{figure*}[htb]
    \centering
    \begin{subfigure}[b]{0.3\linewidth}
    \raggedleft
    \includegraphics[width=\linewidth]{figure/experiments/comparison/rho/rho_under2.pdf}
    \caption{$\rho <$ 2}
    \end{subfigure}
    \hspace{5pt}
    \begin{subfigure}[b]{0.3\linewidth}
    \raggedright
    \includegraphics[width=\linewidth]{figure/experiments/comparison/rho/rho_over2.pdf}
    \caption{$\rho \ge$ 2}
    \end{subfigure}
    \caption{Pairwise penalty matrix for imbalance ratio category. The maximum value of two matrices is 18 and 20, respectively.}
    \label{fig:comp_data_type}
\end{figure*}


\vspace{-15pt}
\begin{figure*}[!t]
    \centering
    \begin{subfigure}[b]{0.3\linewidth}
    \includegraphics[width=\linewidth]{figure/experiments/comparison/architecture/4cnn.pdf}
    \caption{Four Convolutional Neural Network}
    \end{subfigure}
    \hfill
    \begin{subfigure}[b]{0.3\linewidth}
    \includegraphics[width=\linewidth]{figure/experiments/comparison/architecture/resnet.pdf}
    \caption{ResNet-18}
    \end{subfigure}
    \hfill
    \begin{subfigure}[b]{0.3\linewidth}
    \includegraphics[width=\linewidth]{figure/experiments/comparison/architecture/mobilenet.pdf}
    \caption{MobileNet}
    \end{subfigure}
    \caption{Pairwise penalty matrix for a model architecture category. The maximum value of three matrices is 30, 4, and 4.}
    \label{fig:comp_model_arch}
\end{figure*}

\vspace{-15pt}
\begin{figure*}[!t]
    \centering
    \begin{subfigure}[b]{0.3\linewidth}
    \includegraphics[width=\linewidth]{figure/experiments/comparison/budget_size/budget_1.pdf}
    \caption{Budget 1\%}
    \end{subfigure}
    \hfill
    \begin{subfigure}[b]{0.3\linewidth}
    \includegraphics[width=\linewidth]{figure/experiments/comparison/budget_size/budget_5.pdf}
    \caption{Budget 5\%}
    \end{subfigure}
    \hfill
    \begin{subfigure}[b]{0.3\linewidth}
    \includegraphics[width=\linewidth]{figure/experiments/comparison/budget_size/budget_20.pdf}
    \caption{Budget 20\%}
    \end{subfigure}
    \caption{Pairwise penalty matrix for a budget size category. The maximum value of three matrices is 4, 30, and 4, respectively.}
    \label{fig:comp_budget_size}
\end{figure*}

% \vspace{-60pt}
\vspace{-15pt}
\begin{figure*}[!t]
    \centering
    \begin{subfigure}[b]{0.3\linewidth}
    \raggedleft
    \includegraphics[width=\linewidth]{figure/experiments/comparison/model_init/random.pdf}
    \caption{Random initialization}
    \end{subfigure}
    \hspace{10pt}
    \begin{subfigure}[b]{0.3\linewidth}
    \raggedright
    \includegraphics[width=\linewidth]{figure/experiments/comparison/model_init/continue.pdf}
    \caption{Continue initialization}
    \end{subfigure}
    \caption{Pairwise penalty matrix for a model initialization category. The maximum value of two matrices is 34 and 4.}
    \label{fig:comp_model_init}
\end{figure*}



\clearpage

\subsection{Detailed Performance Comparision}
\label{sec:detail_comp_performance}

For the line plots, we note that `Random' and `Ours' are independent of the query selector type.

\begin{figure*}[h!]
    \centering
    \includegraphics[width=0.8\linewidth]{figure/experiments/appendix/cifar10_total.pdf}
    \caption{Test accuracy on CIFAR-10, four layers of CNN, $\alpha=0.1$,  medium budget size\,(5\%), and random initialization. }
    \label{fig:app_cifar10}
\end{figure*}

\vspace{-15pt}
\begin{figure*}[h!]
    \centering
    \includegraphics[width=0.8\linewidth]{figure/experiments/appendix/svhn_total.pdf}
    \caption{Test accuracy on SVHN, four layers of CNN, $\alpha=0.1$,  medium budget size\,(5\%), and random initialization.}
    \label{fig:app_svhn}
\end{figure*}

\vspace{-15pt}
\begin{figure*}[!h]
    \centering
    \includegraphics[width=0.8\linewidth]{figure/experiments/appendix/path_total.pdf}
    \caption{Test accuracy on PathMNIST, four layers of CNN, $\alpha=0.1$,  medium budget size\,(5\%), and random initialization.}
    \label{fig:app_pathmnist}
\end{figure*}

\vspace{-15pt}
\begin{figure*}[!h]
    \centering
    \includegraphics[width=0.8\linewidth]{figure/experiments/appendix/derma_total.pdf}
    \caption{Test accuracy on DermaMNIST, four layers of CNN, $\alpha=0.1$,  medium budget size\,(5\%), and random initialization.}
    \label{fig:app_dermamnist}
\end{figure*}


\vspace{-15pt}
\begin{figure*}[!h]
    \centering
    \includegraphics[width=0.8\linewidth]{figure/experiments/appendix/organ_total.pdf}
    \caption{Test accuracy on OrganAMNIST, four layers of CNN, $\alpha=0.1$,  medium budget size\,(5\%), and random initialization.}
    \label{fig:app_organmnist}
\end{figure*}


\vspace{-15pt}
\begin{figure*}[!h]
    \centering
    \includegraphics[width=0.8\linewidth]{figure/experiments/appendix/cifar10_total_dir1.pdf}
    \caption{Test accuracy on CIFAR-10, four layers of CNN, \textbf{$\alpha=1.0$},  medium budget size\,(5\%), and random initialization.}
    \label{fig:app_cifar_dir1}
\end{figure*}

\vspace{-15pt}
\begin{figure*}[!h]
    \centering
    \includegraphics[width=0.8\linewidth]{figure/experiments/appendix/svhn_total_dir1.pdf}
    \caption{Test accuracy on SVHN, four layers of CNN, \textbf{$\alpha=1.0$},  medium budget size\,(5\%), and random initialization.}
    \label{fig:app_svhn_dir1}
\end{figure*}

\vspace{-15pt}
\begin{figure*}[!h]
    \centering
    \includegraphics[width=0.8\linewidth]{figure/experiments/appendix/cifar10_total_dir10.pdf}
    \caption{Test accuracy on CIFAR-10, four layers of CNN, \textbf{$\alpha=\infty$},  medium budget size\,(5\%), and random initialization.}
    \label{fig:app_cifar_dir10}
\end{figure*}



\vspace{-15pt}
\begin{figure*}[!h]
    \centering
    \includegraphics[width=0.8\linewidth]{figure/experiments/appendix/svhn_total_dir10.pdf}
    \caption{Test accuracy on SVHN, four layers of CNN, \textbf{$\alpha=\infty$},  medium budget size\,(5\%), and random initialization.}
    \label{fig:app_svhn_dir10}
\end{figure*}



\vspace{-15pt}
\begin{figure*}[!h]
    \centering
    \includegraphics[width=0.8\linewidth]{figure/experiments/appendix/cifar10_total_mobilenet.pdf}
    \caption{Test accuracy on CIFAR-10, MobileNet, $\alpha=0.1$,  medium budget size\,(5\%), and random initialization.}
    \label{fig:app_cifar10_mobile}
\end{figure*}



\vspace{-15pt}
\begin{figure*}[!h]
    \centering
    \includegraphics[width=0.8\linewidth]{figure/experiments/appendix/svhn_total_mobilenet.pdf}
    \caption{Test accuracy on SVHN, MobileNet, $\alpha=0.1$,  medium budget size\,(5\%), and random initialization.}
    \label{fig:app_svhn_mobile}
\end{figure*}



\vspace{-15pt}
\begin{figure*}[!h]
    \centering
    \includegraphics[width=0.8\linewidth]{figure/experiments/appendix/cifar10_resnet_total.pdf}
    \caption{Test accuracy on CIFAR-10, ResNet-18, $\alpha=0.1$,  medium budget size\,(5\%), and random initialization.}
    \label{fig:app_cifar10_resnet}
\end{figure*}



\vspace{-15pt}
\begin{figure*}[!h]
    \centering
    \includegraphics[width=0.8\linewidth]{figure/experiments/appendix/svhn_resnet_total.pdf}
    \caption{Test accuracy on SVHN, ResNet-18, $\alpha=0.1$,  medium budget size\,(5\%), and random initialization.}
    \label{fig:app_svhn_resnet}
\end{figure*}


\vspace{-15pt}
\begin{figure*}[!h]
    \centering
    \includegraphics[width=0.8\linewidth]{figure/experiments/appendix/cifar10_budget1_total.pdf}
    \caption{Test accuracy on CIFAR-10, four layers of CNN, $\alpha=0.1$,  small budget size\,(1\%), and random initialization.}
    \label{fig:app_cifar10_budget1}
\end{figure*}


\vspace{-15pt}
\begin{figure*}[!h]
    \centering
    \includegraphics[width=0.8\linewidth]{figure/experiments/appendix/svhn_budget1_total.pdf}
    \caption{Test accuracy on SVHN, four layers of CNN, $\alpha=0.1$,  small budget size\,(1\%), and random initialization.}
    \label{fig:app_svhn_budget1}
\end{figure*}


\vspace{-15pt}
\begin{figure*}[!h]
    \centering
    \includegraphics[width=0.8\linewidth]{figure/experiments/appendix/cifar10_total_budget20.pdf}
    \caption{Test accuracy on CIFAR-10, four layers of CNN, $\alpha=0.1$,  large budget size\,(20\%), and random initialization.}
    \label{fig:app_cifar10_budget20}
\end{figure*}


\vspace{-15pt}
\begin{figure*}[!h]
    \centering
    \includegraphics[width=0.8\linewidth]{figure/experiments/appendix/svhn_total_budget20.pdf}
    \caption{Test accuracy on SVHN, four layers of CNN, $\alpha=0.1$,  large budget size\,(20\%), and random initialization.}
    \label{fig:app_svhn_budget20}
\end{figure*}

\newpage
\vspace{-15pt}
\begin{figure*}[!h]
    \centering
    \includegraphics[width=0.8\linewidth]{figure/experiments/appendix/cifar10_total_continue.pdf}
    \caption{Test accuracy on CIFAR-10, four layers of CNN, $\alpha=0.1$,  medium budget size\,(5\%), and continue initialization.}
    \label{fig:app_cifar10_cont}
\end{figure*}



\vspace{-15pt}
\begin{figure*}[t]
    \centering
    \includegraphics[width=0.8\linewidth]{figure/experiments/appendix/svhn_total_continue.pdf}
    \caption{Test accuracy on SVHN, four layers of CNN, $\alpha=0.1$,  medium budget size\,(5\%), and continue initialization.}
    \label{fig:app_svhn_cont}
\end{figure*}



\end{document}
