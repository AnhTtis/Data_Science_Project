
%97\% of companies use open source (GitHub talk)
%Linux foundation surveys

\section{Concluding Remarks}
\label{sec:discussion}
\boldification{reiterate the importance of this work--similar to part of intro}
Understanding the \textit{why} and \textit{how} companies contribute to OSS helps create a more comprehensive picture of the different players in the OSS ecosystem. 
Our results reveal that participating in OSS helps companies gain a business advantage and also grow their reputation. Reciprocating by giving back to the community is another key motivator. Smaller and larger companies often differ in why they participate in OSS. For example, when considering \textit{reputation} as a motivation, smaller companies participated as it allowed them to gain reputation and to participate in the same space as big-name companies, whereas larger companies reported that OSS engagement signals good citizenship helping cement their reputation in the community. 


%\textbf{Lessons learned on OSS sportsmanship}
%Recurring sentiments about participation in OSS appeared throughout our interviews that reflected virtuous ``OSS sportsmanship'' (see Table \ref{table:sportsmanshipTable}). Here, we highlight a few lessons in prose from fellow companies on the OSS cause: 
%On companies' sportsmanship, \textsc{don't ``bulldoze'' your way into getting what you want} and \inlinequote{contribute in the way that makes sense to the project, which may or may not always be, again, what the company would want (P20).} \textsc{But take the less shiny tasks and help prevent maintainers' burnout}-- \inlinequote{tak[ing] that technical depth task that clean out things, that restructure things (P7).} 
%When it comes to open sourcing a project,  \textsc{prevent abandonware} by open sourcing projects your company uses internally and cares about. \textsc{Though if it happens} and the project goes out of sync \textsc{have a strategic call} by  \inlinequote{transfer[ing] it to people who care about them [abandoned project] (P4).} 
%\textsc{And finally, be a polyglot}: \inlinequote{be able to speak multiple languages (P10)}--fear, risk management, community, and money when advocating OSS to \inlinequote{different levels of leadership. And showing how open source helps the business (P18).} 


\textbf{The OSS renaissance is well underway.}
Open Source has its roots in the Free Software movement, which emerged as the antithesis of the commercial software movement. Without going into the discussion comparing these movements, it is possible to notice that OSS grew and expanded far beyond the weekend warriors. %and the fight against corporations. 
OSS is no longer produced and maintained by a dedicated few, but rather now is a space for different players of different capacities to thrive together--triggering the positive feedback loop of technology and business innovation. 

While tech companies are invested in OSS now more than ever, non-tech and low-tech industries are also seeing the value of OSS (e.g., financial companies) and are establishing Open Source Program Offices (OSPOs) to systematize their contributions and policy implementations. Universities (e.g., John Hopkins, and Rochester Institute of Technology) are also participating by investing in OSPOs to centralize their OSS activities and simplify inter-university and company-university collaboration. %away from the contracts, licenses and grant hustles. 
Organizations such as UNICEF and the United Nations are looking to work with the OSS community to help reach their sustainable development goals  \cite{opensourceUN, sdgsUN}. With a major part of national infrastructures being dependent on OSS (e.g., toll booths, hospitals), municipalities such as the City of Paris have their own OSPO and are contributing city-level digital solutions (e.g., Lutece~\cite{luteceParis, luteceGitHub}) and it is only a matter of time before we start seeing nationwide efforts to invest in OSS and government OSPOs. 

\textbf{Implications for companies.}
Our findings can help nudge companies to contribute to OSS by understanding the benefits of OSS for their business, their branding, and the overall OSS ecosystem. Companies can then identify the motivations that are important to them and prioritize their contributions accordingly. Our mapping between the motivations, ways to contribute, and the benefiting entities, can help companies be more systematic about their OSS presence. Not only that, but the lessons learned from our company participants can serve as best practices for engaging with the OSS community; practices that help foster a high level of sportsmanship and a symbiotic OSS-company relationship. It is important for companies to recognize that they may be a crucial part of the sustainability of the project, as many interviewees mentioned reciprocity as a motivation to join. Reciprocity not only works as a ``social responsibility'' mechanism but as a way to guarantee that the project will be healthy, which in turn benefits the company. We expect that our results enlighten companies, motivating more organizations to see the benefits of engaging in OSS.
 

\textbf{Implications for OSS projects.}
When company involvement in OSS is done right (Table \ref{table:sportsmanshipTable}), it creates a symbiotic ecosystem where both parties mutually benefit. OSS projects can use our findings to help attract company participation and solicit contributions for their project needs.

By understanding companies' motivations, projects can provide an environment that is attractive to companies. For instance, reputation is a driver for companies to contribute, so if the project's community is \inlinequote{an abusive community, even companies don’t want to get involved (P18).} This is an additional impetus for projects to foster a healthy, respectful community to attract company participation. To retain these companies, the project can provide open leadership and governance, one where everyone is \inlinequote{welcome to provide input on the governance and directions of the project (P9).} This can nudge companies to depend on and contribute to a project in a more systematic way, \inlinequote{know[ing] that they're contributing on a level playing field (P11).}


\textbf{Implications for researchers.}
Interventions designed to help sustain OSS need to take into account the different OSS players, their goals, and how they contribute. Our study complements the expansive literature on individuals' experiences in OSS by investigating companies as players. Researchers can investigate other OSS players (e.g., Universities, Government bodies), the interplay of these players, or the role OS    S can play in facilitating collaboration between these parties. Looking at the other side, it is also important to understand the perspective of individual projects and contributors when they are a part of a company-owned or company-sponsored ecosystem, and how the motivations from the different sides align. 

In conclusion, we hope our results encourage and provide guidance for current and new participants in the OSS ecosystem--specifically companies--to engage, collaborate in the open, create better software, and broaden the economic pie. 
%We believe that  lessons from this study will help companies and OSS projects continue to foster this symbiotic relationship, where each sustains the other effectively.

%the company goals and ways to contribute as one of the players. 

