\section{Threats to validity}
\label{sec:threat}
 %it's a threat but this is what we did to minimize it...
 %-saturation?

\textbf{Construct validity} in qualitative research is related to the definition of constructs. One issue could arise from asking incorrect questions which we mitigated by piloting the interview script with the research team. Another issue pertains to the qualitative coding process. To mitigate this and avoid misinterpretation, we compared new emerging codes with the existing code set \cite{barney2017discovery} and met frequently with the research team to discuss and clarify the codes. Additionally, the final code set was reviewed and finalized by the research team. To maximize credibility, we depict examples of evidence from observations to findings (see Table \ref{tab:motivationTable}). 



%During the coding phase, we used the
%constant comparison technique \cite{barney2017discovery} whereby each interpretation and finding is compared with
%existing findings as it emerges from the data analysis to increase the construct validity of our findings.

\textbf{Internal validity} is related to capturing reality as closely as possible, which in our case is company motivations and contributions to OSS. The characteristics of our sample may have influenced our results. Half of our interviewees were part of the OSPO team within their company even though we did not push for such a split. The sizes of companies from which we interviewed participants were balanced (7 small, 1 medium, 9 large). We reached saturation after the $13^{th}$ interview. We also performed member checking to validate our constructs and received validation from all eight of our respondents.  



% The interviewed participants were evenly distributed across different-sized companies  (7 small, 1 medium, 9 large)
% 12/ 17
% balanced company size 




%To accurately capture these motivations and contributions, our sample includes participants from different-sized companies who have different roles. Our participants' roles are related to some OSS part of their company, which validates the knowledge they contributed to our study. Additionally, we validated our constructs via member checking where we received validation from all XX of our respondents.  


\textbf{Reliability} refers to the extent that our results can be replicated. In short, it is difficult to replicate qualitative research since human behaviors, feelings, and perceptions change over time. However, we maintained consistency by constantly comparing the analysis with already existing codes and having weekly meetings to discuss and adjust codes and categories until
we reached an agreement. We also performed member
checking with eight participants who confirmed our interpretations. Finally, the results presented in this paper are related to Open Source participation, thus, we do
not expect that all our findings will apply to other contexts. 


\textbf{Theoretical saturation.} A potential threat to validity regards reaching theoretical saturation. The quality, rather than the size, of the sample of participants, is essential to increase our confidence in the findings. We interviewed 20 participants with different OSS-related roles. %and the other half were part of non-OSPO related roles that are knowledgeable about the companies' participation in OSS (e.g., CEO, Ecosystem Strategist). 
The 20 participants represented 17 companies of different sizes, from big technology companies, such as Microsoft and Google, to growing startups. We kept interviewing participants until no new codes emerged (13 interviews). Moreover, we conducted seven additional interviews and we did not find any new constructs. %As mentioned previously, the number of interviewed participants was adequate to uncover and understand the core categories in any all-defined cultural domain or study of
%lived experience \cite{}. 
Although we cannot claim theoretical saturation, our sample helped us uncover a consistent and comprehensive account of the nature of company motivations to contribute to OSS and uncover the different forms of contribution. 

