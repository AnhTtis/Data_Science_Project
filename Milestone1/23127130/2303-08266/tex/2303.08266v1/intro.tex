\section{Introduction}
\label{sec:intro}

%\boldification{The OSS has shifted where more and more companies are involved in it. Companies stopped viewing OSS as an ecosystem that was a threat to their existence and started seeing it an ecosystem of opportunity.}
%and Google's first release of Android in 2008, the OSS community

%\boldification{Companies' view of OSS has transitioned from a risky endeavor to an ecosystem of opportunity}

\boldification{***Open source transitioning from individual, decentralized participation to more and more companies participating. This has shaped the new OSS ecosystem: entanglement of individuals, and companies, big and small.***}
Open Source Software (OSS) is no longer a ``weekend warrior's endeavor"~\cite{robles2019twenty}. Over the last 20 years, the OSS ecosystem composition has changed drastically. OSS is now fundamental to company operations--not only for the code that they depend on, but also for their role in an ecosystem to which they actively contribute~\cite{robles2019twenty, fitzgerald2006transformation}. This is a paradigm shift from the early days when OSS was viewed as a threat that commoditized software to today where individuals and companies work symbiotically. About 77\% of respondents to the 2022 State of Open Source survey \cite{2022OSSSurvey} said they increased the use of OSS in their organizations over the last 12 months. In fact, major companies are now increasingly spearheading the development of state-of-the-art OSS technology (e.g., Kubernetes) \cite{casalicchio2020state}. 

%\boldification{Research has mainly focused on studying OSS from the perspective of the individual contributors, to understand their motivation, roles, and challenges}

\boldification{**Prior work has identified the hows and whys of individual participants.}
Research thus far, however, has largely investigated the OSS ecosystem from individual contributors' perspective: their experiences in OSS, their motivations to contribute~\cite{gerosa2021motivation, lee2017understanding, von2012carrots, hannebauer2016motivation, oreg2008exploring , roberts2006understanding, huang2021leaving}, the different ways to participate~\cite{jergensen2011onion, nakakoji2002evolution ,trinkenreich2020hidden,rebouccas2017does, schilling2012will, zhou2014will, steinmacher2021being}, and the challenges they face~\cite{steinmacher2015systematic, steinmacher2015social,guizani2021long ,lee2017understanding, huang2021leaving}. 

\boldification{***–which entails the old OSS ecosystem, our knowledge now has gaps in understanding the ecosystem, since we don’t know what role companies play.***}
Our understanding of how an OSS ecosystem works and how to sustain it is, therefore, modeled on the ``old world'' OSS ecosystem, and not the new hybrid reality, where ``every software company is an open source company'' \cite{eclipseOpenCollab}. 
We have little understanding of \textit{why} and \textit{how} companies get involved in OSS. 
It is important to understand what motivates companies to engage in OSS for multiple reasons. First, not doing so paints an incomplete picture of the OSS ecosystem, which is arguably one of the pillars of software engineering today~\cite{openSourceUseLinux, openSourceUse}. Second, without a clear picture of why companies participate in OSS, efforts to sustain OSS rely only on a subsection of contributors and are less likely to succeed. 
Third, highlighting the motivations and different mechanisms of engaging with OSS can help other companies recognize the benefits of being part of this hybrid OSS landscape, not only benefiting the company but also making the OSS ecosystem stronger. 

%A lack of understanding of companies' involvement with OSS paints an incomplete picture of the OSS ecosystem.  This, in turn, makes our research endeavors to sustain and maintain OSS incomplete and highly reliant on some OSS players and not others. As a consequence, researchers and OSS communities may end up putting efforts on solving a problem that leaves something out of the equation, which would require rework later. In order to put companies' participation in the loop, it is paramount to understand their motivations to be part of and the multi-faceted ways that they engage in OSS. This understanding paints a more complete OSS landscape that helps understand the relationship between different players and the OSS ecosystem.

\boldification{Prior work has mainly looked at licensing or how companies are doing inner sourcing.}
Past work that has investigated company involvement with OSS focused on
the legal perspective of companies' usage of OSS components~\cite{harutyunyan2019getting, harutyunyan2019industry, harutyunyan2018understanding}, their communication practices when interacting with OSS projects~\cite{butler2018investigation, butler2019company}, 
how adoption of OSS matches different company business models~\cite{hecker1999setting, munga2009adoption, spijkerman2018open}, 
or how companies are using the open source model internally (i.e., innersource) \cite{stol2014key, capraro2016inner, stol2014inner, edison2020inner, morgan2011exploring, capraro2018patch, carroll2017value, carroll2018examining}. These studies do not investigate the motivations of why companies contribute to OSS.

To the best of our knowledge, Linaker and Regnell's case study is the only work that has investigated company motivations---why and when companies decide to open source their products \cite{linaaker2020share}. They identified the objectives that led three organizations (a US media and technology organization, a European-based hardware electronics manufacturer, and the Swedish Public Employment Service) to open source their software. This work serves as a starting point for building a picture of why companies participate in OSS. 

In this paper, we aim to create a more comprehensive understanding of company motivations and mechanisms for engaging with the OSS ecosystem by asking:
\begin{enumerate}[leftmargin=0pt]
    \item[] \textbf{RQ1. }\textit{What motivates companies to contribute to OSS?}
    \item[] \textbf{RQ2. }\textit{How do companies contribute to OSS?}
\end{enumerate}

By creating an understanding of company motivations (\textbf{RQ1}) we aim to help projects provide an environment that is not only attractive to passionate individuals, but also to companies that can invest in and sustain the project in the long run.
Further, articulating companies' motivations to contribute to OSS can also nudge other companies to get more involved. By mapping the different ways companies contribute to OSS~(\textbf{RQ2}) we provide guidance to companies that are looking to engage in OSS. OSS projects can also reflect on the wide range of support companies provide and solicit support that aligns with their project needs. 
%Therefore, to guide these companies figure out how to contribute to OSS, we map the different ways companies engage with OSS by asking:

\boldification{*** [we did this] RQ’s explored through 20 interviews–with big and small companies, participants with different roles ***}
We answer our research questions through interviews with 20 participants from different companies, playing different roles related to OSS (e.g. Open Source Program Office (OSPO) Lead, OSPO Manager, CEO, Ecosystem Strategist) and working at companies of different sizes, including Microsoft, Google, RedHat, and Spotify. We then qualitatively analyzed the data through inductive coding to create a conceptual model of company motivations to engage with OSS and the different ways to do so.

%"you have these competing companies who collaborate everyday together and they have not signed anything."

%\boldification{Our paper's contributions include understanding why companies contribute to OSS, the different ways companies contribute, and discussion of lessons learned for a healthy/symbiotic relationship between companies and OSS}








\boldification{*** Our results can help:
Understand the impact of companies' participation on OSS 
Effectively mitigate companies' bus factors/ turnover
Provide projects with the knowledge to help them attract/ solicit the support they need from companies 
Effectively attract and retain companies to OSS 
Nudge companies to participate ***}

The main contributions of this paper are (1) a comprehensive model of company
motivations to contribute to OSS, (2) a conceptual model showing the multi-faceted ways that companies engage in OSS, linked to their motivations, and the benefiting entities, and (3) lessons learned from companies to foster a healthy OSS-company relationship. 
%\boldification{We hope our contributions help guide more companies to get involved in OSS in a ``symbiotic'' way-- as learned through our findings}
We hope that our insights on company motivations and contributions to OSS would encourage and provide guidance for more companies to engage in OSS, collaborate in the open, create better software and broaden the economic pie. The lessons we learned from this study will help both companies and OSS projects continue to foster a symbiotic relationship. %in which they sustain each other. % thus activating the ``virtuous cycle.''

%Our work complements this body of research by focusing on companies as OSS contributors instead of looking only on individuals' levels. We also identify the ways to engage and the reasons that lead companies to engage with OSS. We contributed a comprehensive model of companies' motivations to participate in OSS beyond open sourcing products. Indeed, we show that the ways to contribute are multi-faceted, and are tied to the motivation of the companies, and may influence different entities. 

%- enriching the ecosystem with more players, better software quality because of more competition and coopetition, bigger market share... 


%Ideas: can talk about expanding the economic pie... more companies in OSS provide more opportunities to individuals to get noticed and get hired and companies to want to get noticed and get hired 
%- enriching the ecosystem with more players, better software quality because of more competition and coopetition, bigger market share... 

%nudges more companies to explore the benefits of contributing to OSS and the different ways they can do so. Getting more companies involved would help build a diverse OSS ecosystem an 

%to get involved in Open Source and give back to an ecosystem that has given so much our research community create more nuanced approaches to attract and support a diverse set of contributors to OSS.

%%giving software away isn't commoditizing a product, it's inviting your customer right? So suddenly, suddenly, what was viewed as as sort of a threat to money became an opportunity to expand your ecosystem play. And in fact, large tech companies compete with each other as to how much they can give away, and how much adoption they can get on what they give away. So there's quite a competition between, oh, I'll open source something, you'll open source something they both do about the same thing.

%P10 ideas of starting the paper: 
%"A significant amount of the focus and attention on open source is on individuals who contribute to open source projects, largely viewed as like that passion, like the weekend warrior, or the graduate student who wants to improve their skills. And, and the early days of open source had quite a bit of that. There has been a transition over the 20 plus years of Open Source and the 38 years of or so of the free software"

%"So you know, the history right, free software in the early 80s, Richard Stallman and sort of a movement of changing copyright laws, and then the Open Source Initiative, in like, was a 1999, sort of refining that and saying, well, there is a way to do open source that is not necessarily the Free Software way of doing it, where everything you distribute has to be licensed in a hereditary way. But there's a sort of a corporate friendly, or more frankly, libertarian viewpoint

%"if you think about the 20 years, there was this transition, I think, around 2008 ish, I would call that as being maybe this, this period where companies, large tech companies, IBM, and IBM was fairly early, but IBM, maybe more so HP, Intel, Facebook, which which had, which was very new in 2008. Google, which was get, you know, had certainly a lot of legs under it and getting popularity, Yahoo, which had already sort of maybe was peeking, AOL, had also been involved.
%So a lot of it's sort of the tech companies had recognized that open source is a 

%fundamental part of technology. And as companies that base their existence on understanding and manipulating and leveraging technology, open source has a play, and what Microsoft did in 1999, and that era, sort of viewing open source as a fundamental threat to its existence."

%So in the earlier stages of tech companies as being I published software, and anyone who gives software away commoditizes my product and is therefore a problem. That was how Microsoft viewed it, and Oracle and some of the something like Oracle Knutsen, some of the earlier companies said, Wait, we hire engineers, we pay for them. We write software, we license that software for people to use, If anyone gives us away for free, that's a problem, because we need a captive market. group of other tech companies said, licensing software isn't necessarily the only way to make money off software. In fact, you can give a lot of software away and get people into your ecosystem into your network. So whether we're, you know, collecting data in order to advertise to them, or getting them to use our search engine or email system, or our news feed in order to understand their intent and advertise to them, or upsell them to a pro verse, whatever, there's ways to make money, that that don't require you to go and get a shrink wrap box, a shrink wrap box of like, word 2003 and install it, you know, for 190 like that, that sort of model. It's like, here's, I give you 109 I get a CD and I get software. Now, I'll give you a Google Doc for free. But you need a Google account. And if you have a Google account, you get email. And if you have email, then you can get a little pencil like you know, sort of get you in. And in that model. 
%giving software away isn't commoditizing a product, it's inviting your customer, right? So suddenly, what was viewed as as sort of a threat to money became an opportunity to expand your ecosystem play. And in fact, large tech companies compete with each other as to how much they can give away, and how much adoption they can get on what they give away. So there's quite a competition between, oh, I'll open source something, you'll open source something they both do about the same thing.



%[P15] Well, because Because ultimately, 80 to 90% of you know work, you know, measured in lines of code lines of documentation, hours, events, whatever, is done by somebody who is paid to work on, right. Um, people like to talk about folks working in their spare time and working in their garages and that sort of thing. And there was a time, you know, and by there was a time I mean, like, the 90s, when that was like two thirds of open source, right. When I started working on Linux and Postgres, it was on the side, and what I mainly got paid for was to write Microsoft code. Hmm. You know, I was a Microsoft registered consultant. The but but that changed and is no longer the case. Right? And if you look at you know, who writes Kubernetes who writes Linux or writes any of the big projects or any of the heavily adopted projects, it is, you know, 80 to 90% people who are paid to work on it, either as their main job or as an accessory to their job. The and So, so that's the main thing, right? The fact that we'll have all this open source software, the fact that most of software in the world is open source, is because all of these hours paid hours got put into it. So that's what open source gets out of it is the main, that's the main thing, right? Um, you know, plus companies pay for some other things, too, right? They pay for infrastructure and stuff, right? The Linux Foundation is very, definitely not operating for free. Um, you know, neither is Apache or or anybody else, the open source foundations are all funded. Generally by companies that have products that involve open source in some way.


%ce communities, like I mean, think of other big open source communities React, Facebook, TypeScript, I forgot who that was, uh, what else? TensorFlow, Google. Yeah. dotnet, Microsoft. Right? like these are all of the largest open source projects in existence. And they're all explicitly owned by corporations. [P3]