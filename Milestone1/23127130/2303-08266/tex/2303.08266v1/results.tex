\section{Findings}
\label{sec:findings}

%paper about donating projects to OSS 
%\subsection{What does OSS mean for companies?}
%"So corporations have always been pretty much command and control with a little bit of marketplace dynamic. But OSS is about this network, the network effect. Why are corporations doing that? Like, that's this third modality of sharing information that's incredibly effective. Corporations are using this modality of hierarchical command control, which is incredibly ineffective. So maybe corporations need to learn and participate in OSS. And that's that sort of got me from anti-OSS to learning about network production models, to running an OSPO. And now being very involved in lots of things related to OSS. And then this conversation" [P10]
In this section, we describe our findings on what motivates companies to contribute to OSS and the different ways companies contribute to OSS.%, and some lessons learned. 
\subsection{RQ1: What motivates companies to contribute to OSS?}

According to our interviewees, company motivations to contribute to OSS fall under four top-level categories: \textit{Founder(s)' Vision, Reputation, Business Advantage, and Reciprocity} (see Table \ref{tab:motivationTable}). We will discuss what we found for each of these categories in the following sections.

%yes, we have some. So of course, you can claim to be an expert, if you contribute regularly to a project, yes, kind of probably. So that's one way of also getting this kind of trust, building a reputation in that sense. 

% 3:17 ¶ 73 in [P2] Daniel-Bitergia interview-D.docx
%we need to use other marketing efforts to be known and to be to be seen out there being being an open source company, I think, helps build like trustable branding.
% Please add the following required packages to your document preamble:
% \usepackage{multirow}
\begin{table*}[htb]
%\centering
%\scriptsize
\caption{Company motivations to contribute to OSS}
\label{tab:motivationTable}
% Please add the following required packages to your document preamble:
% \usepackage{multirow}
\resizebox{\textwidth}{!}{
\begin{tabular}{llll}
\hline
  \textbf{\begin{tabular}[c]{@{}c@{}} Category \end{tabular}} &
  \textbf{Subcategory} &
  \textbf{Participants} &
  \textbf{Exemplary Quotes} 
  \\\hline
  
  
  
   %____________________Founder's Ideology_________________________
 
  \multicolumn{2}{l}{\multirow{1}{*}{\begin{tabular}[l]{@{}l@{}} Founder(s)' Vision  \end{tabular}}}

%---------------------Founder(s)' ideology-----------------------  
   
  \begin{tabular}[l]{@{}l@{}} \end{tabular} 
  
  &

   \begin{tabular}[l]{@{}l@{}} P1, P2, P5, P7,\\ P9, P12 \end{tabular} 

&
 
 \begin{tabular}[l]{@{}l@{}} ``Why open source? One: it's ideological. The founders just believed that was the right thing to do'' (P1)
  \\
 % ``The four founders basically were avid contributors of open source, and they decided that's, that should be our DNA'' (P7)
    ``The reason why we started [company name] was to create an open source alternatives to large enterprise'' (P9)
  
  \\
  
%``[founder's name]'s vision and intent for the project in the first place ''[P5]

   \end{tabular} \\ 
  
%----------- Problem is open source in its essence----------
  % &
  %\begin{tabular}[l]{@{}l@{}} Problem is open source in it's essence \end{tabular} 
 % &
  
   %\begin{tabular}[l]{@{}l@{}}\\ P12, P7, P9 \end{tabular} 
  % & 
    
  %\begin{tabular}[l]{@{}l@{}}

   %\\ 
   %``so <company name> was built on top of open source. So without open source <company name> wouldn't exist ''[P7]
  % \end{tabular} \\ 
  \hline
%____________________Reputation_________________________
  \multirow{12}{*}{Reputation} &
  
%-----------VISIBILITY-------------
  \cellcolor{gray!15}\begin{tabular}[l]{@{}l@{}} Visibility \end{tabular} &
  
  \cellcolor{gray!15}\begin{tabular}[l]{@{}l@{}} P1-P9, P11, \\ P13-P18 \end{tabular} &
  
  \cellcolor{gray!15}\begin{tabular}[l]{@{}l@{}} ``Where open source is superior is it gives us superior visibility'' (P15)\\ 
  
  %``So it's clear that brings visibility and bragging rights'' (P9)
  
 % ``We decided to do is write a blog post... Well, a bunch of people started commenting on that. And it raised awareness of [company \\name] because we wrote it, but we didn't mention [company name], really, in that entire article... so it's actually a great marketing \\tool to talk about technology to use, even if it doesn't promote your, your own product'' [P8] \\
%  ``Having companies involved can also help you with... marketing, right?... that helps give visibility to some of these these projects as\\ well'' [P11] \\
  %``open sourcing these bits and pieces of whenever you can make it like tangible and like very transparent about \\ what was actually going on inside of the company. So that can that can help the deal of branding [P4]''
  ``If you actually came up with this standard, then you also get a lot of exposure'' (P4) %... And is again it kind of is like our desire to be able \\ to be a standardized piece of technology'' (P4)
 \end{tabular} \\
 
 %It could also be a business that decides to release a project because they want to create a standard in the world around that particular software, like, you know, Google did with Kubernetes, it became the de facto way to do container orchestration, as and so many other projects have become kind of the de facto standard for doing something. And the company that created it and released it has an advantage, because they are seen as thought leaders, people want to join them from a recruitment perspective.[P18]
 
%And, and where open source is superior is it gives us superior visibility into our own supply chain. Right as in, we don't have to-- with proprietary software, you know, if we have a proprietary component, like we were proprietary, authentication component, and we're like, hey, is this being maintained in the light of new security exploits? Right? In that case, you have to trust the vendor to know whether or not that's being maintained. [P15]

%So it's clear that brings visibility and bragging rights. So for instance, I can I usually use when I make presentation at conferences as a joke as an intro, the fact that I've, I've had my signup, ID to one of the largest deletion of code in the Linux kernel. And that's true. So identify the line of code in there. But ideally, you'd like 1000s and 1000s of lines of code, which were actually lines of Commons license. You know, doing doing this on a massive scale on 10s of 10s. Of 1000s. of files. That's a skill.[P9]

%[showcasing technological achievements] 
  %4:37 ¶ 115 in [P1] Georg-Bitergia interiew.docx
%And we are active in these communities showing, showing off our skills and experience? They're saying, Oh, yes, we want more of that. Can you dig in more? Show us more what, and we basically use these communities to showcase what we can do. And whether our potential customers are themselves active or just observing doesn't matter. It creates that contact to then have a conversation.

%4:11 ¶ 27 in [P1] Georg-Bitergia interiew.docx
%hey hire us for the expertise. Because we have worked with a lot of open source projects, and foundations and companies that we are really good at knowing what the data is, what is available, what kind of questions we can answer.

%3:18 ¶ 73 in [P2] Daniel-Bitergia interview.docx
%contributing to open source is about demonstrating that you have the expertise and the knowledge contributing, and then you can serve specific pieces of knowledge.

%3:21 ¶ 73 in [P2] Daniel-Bitergia interview.docx
 %So then we can mess with your inner source effectiveness, let's say. But then at the same time, we are serving with the community. So we are defining inner source. So it's not that we know about inner source is that we are defining inner source and for this you can see that we have this pattern here about the maturity model, this pattern there about this or that and then we have these presentations at the inner source commons.

%2:101 ¶ 23 in [P4] Per-Spotify interview.docx
%So like open sourcing these bits and pieces of whenever you can make it like tangible and like very transparent about what was actually going on inside of the company. So that can that can help the deal of branding of the technical ideas of an app like Spotify, I think. And Spotify saw a lot of good press with like this, Spotify wrapped the day over the Christmas period where everyone got like the year overview of music. And that kind of thing took like a ton of machine learning a ton of data and a lot of engineering tricks, to do this little little feature. And so this is like this is one of the ways they want to be more transparent about like, how do we actually use these, like very sophisticated systems of understanding music and user behavior and, and music history and so on. So we can kind of create this little feature called wrapped, which is just one month a year. 

%8:72 ¶ 71 in [P7] Josep-Aiven interview.docx
%Because they were asking if we have some experts on X, Y, Z? And then it's like, yes, we have some. So of course, you can claim to be an expert, if you contribute regularly to a project, yes, kind of probably.

%8:125 ¶ 135 in [P7] Josep-Aiven interview.docx
%write articles on presence while trying to showcase what we do share the Word of either what we do at Aiven, what we do in the open source space, how we shape things. 

%9:9 ¶ 16 in [P11] Dawn-VM Interview.docx
%so people tend to think of VMware as like the people who did virtual machines. And we've really moved beyond that. And so we-- well our focus right now is really on building platforms and application developer tool sets that people can use on top of those platforms to build their applications. So we do that across all the clouds

%[be seen as the founder of a technology that is now industry standard]
%2:51 ¶ 37 in [P4] Per-Spotify interview.docx
%And if you actually came up with this standard, then you also get a lot of exposure. Like Google got a lot of exposure from like, creating Kubernetes, which is like the industry standard for like management services. Now again, and Spotify wants to kind of create the same kind of narrative for like internal



%``if you actually came up with this standard, then you also get a lot of exposure... And is again it kind of is like our desire to be able to be a standardized piece of technology''[P4]

%2:90 ¶ 97 in [P4] Per-Spotify interview.docx
%h, yeah. And is again it kind of is like our desire to be able to be a standardized piece of technology instead of just an esoteric Spotify technology

%17:11 ¶ 88 in [P14] Harish RedHat Interview.docx
%And then the Linux kernel community, eventually settled on the model that we had also adopted. We already had adopted. And so guess what happens? The SUSE guys had to scramble, because now they have a kernel that is only using that type of threading which the upstream is not accepting. They say, yeah, it doesn't make sense. And so they had to scramble and I know of customers who were abandoned by SUSE because they were already going down that path, which was not the right path to go and SUSE couldn't quickly, you know, re engineer the stuff so that they can so they have to sunset that portion of the tech so that they can go into the newer threading model in the kernel. So. So that's the kind of benefit that we get as an example. So benefit, right?

%18:37 ¶ 298 – 302 in [P13] Tobie OSS consultant Interview.docx
%Yeah, it's a lot of money. Yeah. Um, and so that was just essentially by commoditizing. The layer under what it is that they were serving
%[R] So this is Facebook being able to save that amount by by standardizing
%[P13] by essentially open sourcing, sort of like the structure and diagrams of how they were making. 

%18:38 ¶ 296 in [P13] Tobie OSS consultant Interview.docx
%But let's say that they realize they could make a lot more money selling hardware than, than selling their software. Right. They could make the whole zoom thing, completely open source, for example. And make it work extremely well with their hardware. So that Sure, they would lose money and have like, fewer people actually paying for it, but they would become the de facto solution for all hardware, built on for, you know, whether like, regardless of who was using that system or not. 

%1:13 ¶ 187 in [P3] Omer-VM interiew.docx
 %the sort of street cred of the founders of React,

%1:19 ¶ 187 in [P3] Omer-VM interiew.docx
%get an entire class of engineers, who are now familiar with the tech stack that Facebook uses, right? Or at the very least, like one of the main web frameworks that they're using, that they have sort of home grown, right. And they also get viewed as, like a cooler place to work. 

%2:58 ¶ 37 in [P4] Per-Spotify interview.docx
%They, they want to leverage open source as much as possible. And the best way to do this is to make sure that what you are working on inside of your company, and the thing you depend on actually becomes the industry standard, not just something you maintain. Because if it becomes an industry standard, then you gain all the benefits from being part of the standards. And if you actually came up with this standard, then you also get a lot of exposure. Like Google got a lot of exposure from like, creating Kubernetes, which is like the industry standard for like management services. Now again, and Spotify wants to kind of create the same kind of narrative for like internal 

%12:16 ¶ 130 in [P15] JoshB RedHat Interview.docx
%But you know, have partners and customers that are using it. Yeah. And you don't want to piss them off. By terminating the tool? Well, if it's open source, you can just kind of gradually pull all your engineers off of it, and let them have it. Um, you know, so this is what's called the abandonware strategy. Now, that that's generally used as a derogatory term, but sometimes abandonware is good, right? Because it's something we can do with open source that you could never do with proprietary software, where you say, Hey, I know you guys are still using this. I know you still like it, but it doesn't make any sense for us as a business anymore. So here, it's yours. Now you can have it.


%[visibility/positioning in community]

  %\makecell{Some really \\ longer text}
  %the CEO of [large OSS foundation] announced in a keynote [city and name of conference]. We are now Running [OSS community]. And all of these companies and universities are part of this [company name] was there. So that was like great visibility, a really good milestone for us[P2]
  
  %You know, the other benefit that sometimes people see is you know, having having companies involved can also help you with I don't know, like, I guess marketing, right? So so we have, you know, we have relatively popular blogs and social media channels, and we can promote things about some of the open source projects that we're working on. And that that helps give visibility to some of these these projects as well. So I guess visibility is what I was talking about sovereignly I guess I guess it's marketing.[P11]
  
  %7:22 ¶ 47 in [P8] Johnathan-Goliath interview.docx
%And so we decided to do is write a blog post saying, here's all the things here's how you might want to use JSON, the JSON library in Zephyr, and oh, here's some gotchas we found. Well, a bunch of people started commenting on that. And it raised awareness of Goliath because we wrote it, but we didn't mention Goliath, really, in that entire article. And the maintainers of the Zephyr project, want to see how we can start to contribute back. So that material, so it's actually a great marketing tool to talk about technology to use, even if it doesn't promote your, your own product. And so we're seeing that by by proxy of doing it. 

  
  %So we have a branding, we have collateral point of bringing more people hire more people keep more people here. Get more engaged with more companies that might be the future partners. For example, last year, we had several conversation with Red Hat, who is a huge company about open source. And yes, he has a lot of money. And because of this conference discussion, right now, we had strategic discussion to do more business together. And everybody was everything was based in the first open source commit and open source community discussion. So yes, there is no money directly on those open source products, but it's your work to keep it that's why Zup is zookeeping it so with branding, it's kind of like that.[P6]
  
  %So two reasons. One is personal. And one is business, the personal reason is that we whoever is engaged in in these communities may have a personal interest wants to be wants to learn something new wants to be recognized as, as a leader in that space. So it's a little bit like personal branding. So one, for example, I'm involved in the IEEE sa open community advisory group. And part of that is to build up my credentials as an open source strategist and being an expert on open source communities. For Bitergia, as a company to encourage this and do this, is because being active in open source communities, gives us access to decision makers, at companies that could potentially be customers. And so we use this also as a way for finding prospects and generating business for the company. And most of the business that we do have came through contacts that we fostered in being active in these communities.[P1]
  
  %Definitely, I suppose in the marketing and branding level, that the marketing and branding folks always have, like, how is this contributing to our-- I don't work necessarily at that level. But I think that the level we do work at is we want to show up as in some of the ways I was talking about like humble, sincere, dedicated, more than a lot of time brand is about being like when I hear marketing and brand I hear, "we're the best at this look at us" you know, and I don't think that's what we're going for at all. I mean, occasionally, occasionally, somebody might be proud of something, but we're really looking to be part of something. Our brand is being part of something and not the best at something, [P17]
  
  %So we need to use other marketing efforts to be known and to be to be seen out there being being an open source company, I think, helps build like trustable branding.[P2]
  
 %12:3 ¶ 26 in [P15] JoshB RedHat Interview.docx
%The the second way that we're involved with with open source is community. As in open source projects, collect people around them, which are known as open source communities. And open source communities are a good way for us to reach people. It's a major portion of our marketing picture, when I'm talking about marketing with a big M. The and and they supply a lot of sort of key inputs to marketing, you know, both not just, you know, people to basically advertise to, but also as a way of learning what is it that people want, so that we can actually build those products? Right. Um, so marketing in terms of market research, and in terms of getting direct feedback? You know, like, if we're thinking about something new in Linux, we can put it in fedora and see how it does in fedora. Um, you know, before we have to offer five years of support for it.

  
  %7:13 ¶ 31 in [P8] Johnathan-Goliath interview.docx
%It's more about being able to have our voice heard, emphasis on priorities for our company, but also to drive the general shift. Like, because I participate in the marketing side, I'm able to make the marketing efforts better, like, "oh, maybe we should go talk to these people. Or maybe we should have this change the conference upcoming." And so I wouldn't have that opportunity if I wasn't a project member.

  
  %6:47 ¶ 159 in [P6] Otavio-Zup interview.docx
%We have several views in different countries for up to the company. So right now we have I don't know the number but 20,000 views around the globe. We have several countries don't have the right number right now.

%6:73 ¶ 159 in [P6] Otavio-Zup interview.docx
%And how can we engage money from the open source products basically first with brand thanks to the four open source products.

%2:35 ¶ 19 in [P4] Per-Spotify interview.docx
%And there is like, one aspect, just like branding, like that a company, what they do, again, what they do inside is not very visible on the tech side,

%2:56 ¶ 19 in [P4] Per-Spotify interview.docx
%just like branding, like that a company, what they do, again, what they do inside is not very visible on the tech side, like a like an app like Spotify, it's actually quite hard to understand, like, how it actually works, technically. And why should that even be interesting? For many doing point of view, if you're like a top notch engineer at Google, or Facebook or Apple, why should you work at Spotify? It's just like an app that plays music. 

%2:51 ¶ 37 in [P4] Per-Spotify interview.docx
%And if you actually came up with this standard, then you also get a lot of exposure. Like Google got a lot of exposure from like, creating Kubernetes, which is like the industry standard for like management services. Now again, and Spotify wants to kind of create the same kind of narrative for like internal

 %3:17 ¶ 73 in [P2] Daniel-Bitergia interview.docx
%we need to use other marketing efforts to be known and to be to be seen out there being being an open source company, I think, helps build like trustable branding.

 %5:74 ¶ 213 in [P5] Jill-SlimAI interview.docx
%Could branding make it easier? Sure. I mean, it might make it more easily recognizable. Yeah. Are there open source projects on market today that don't look anything like their corporate sponsors? Yes. Are there some that do? Yes.

  
 %3:19 ¶ 73 in [P2] Daniel-Bitergia interview.docx
%Yeah, so So the, the open source communities we are part of, we can see them as a way to position branding, to partially use their marketing efforts into reaching out other people that they need, for instance, metrics, or open source metrics, or they need to measure PlayerHealth. And then they go to, to grimoire lab-- to CHAOSS, and they say, well, there is there is a university running or so we I can I can hire them or I can hire, Bitergia for more, you know, professional services or industrial services. Good. So this is like a good place where you can go and fish for potential customers. It

%17:25 ¶ 16 in [P14] Harish RedHat Interview.docx
%So to answer the question is, what is in it for Red Hat? The brand recognition, oh, the guys in Red Hat, they advised us to try to do this. Yeah, they did help us out, we did not pay them all, we paid them something. Oh, do we didn't buy the subscription from them. All, we bought some subscription for them, because we were glad that they helped us, you know, so the leaves hopefully leaves a positive, a positive impression with the organization. Because they have a lot of them, they immediately start comparing us against, for example, I'm picking the name because a very common name, Oracle, or Microsoft, or they just want to, they just come here to make money. They want my money, they want to do this, No, they won't give me anything else. But when they come here, they also want my money, but they also give me other things. And it's okay, even if I don't, even if I don't subscribe to their services, even if I don't buy their service, they're still friendly to me. Whereas the other guys, Oh, you didn't buy from me, I'm walking away. So it's a very different, it's a very human thing. I mean, it's not a it's not a win, win lose environment, it's a win win environment. So you know, in order for the other side to say, oh, I need to win this customer, that means the customers lose the money to me.

%19:19 ¶ 65 in [P16] Stephen Transcribed Interview.docx
%doing good, looks good. And looking good, has, you know, has a potential positive effect on your bottom line? Right. So I think, I think that's definitely one component of it for for enterprises. Participating in open source, like, it looks good to be doing good. 

%19:20 ¶ 65 in [P16] Stephen Transcribed Interview.docx
%making sure that we're, we try to be equitable with the way that we've maintained some of these things within, within smaller parts of various foundations. But also realizing that there's a component of it, that you have to show people why they should care about you.

%-----------Verifiable Trust-------------  
  &
  \begin{tabular}[l]{@{}l@{}} Building\\verifiable\\trust \end{tabular} &
  
  \begin{tabular}[l]{@{}l@{}} P1, P2, P5-P7,\\ P15, P17, P18 \end{tabular} &
  
  \begin{tabular}[l]{@{}l@{}}
  ``If we have a proprietary component...you have to trust the vendor to know whether or not that's being maintained...\\Whereas if it's an open source project,  I can watch the contribution history and I can know whether or not it's being maintained'' (P15)\\
  
  ``The biggest benefit from contributing is building trust'' (P18) %, and from trust comes influence. So you can actually influence the direction of a\\ project, especially if you're very dependent upon the project
   
  
  %``Because they were asking if we have some experts on X, Y, Z? \\ And then it's like, yes, we have some. So of course, you can claim to be an expert, if you contribute regularly to a project'' (P7)\\
  \end{tabular} \\
  
   %And it's the basis of your service or product. If you are not able to influence the direction, then you're left constantly reacting to the project's changed direction, so you need to be at the table.[P18]
  
  %And there's such a wide variety of components. And these components have become standardized, because everybody's using it. So developers start trusting if if, you know, all of these companies are using it, it must be a good component, instead of building it themselves. [P18]
  
  %4:11 ¶ 27 in [P1] Georg-Bitergia interiew-D.docx
%hire us for the expertise. Because we have worked with a lot of open source projects, and foundations and companies that we are really good at knowing what the data is, what is available, what kind of questions we can answer.

%And right now, you can think the open sourcing as a marketing budget to the company. So for example, it's [recording?] but then our proposal, but I can say that we have Vivo company, who decided to join us because they used an open source product. So the open source product was the first step to hire our service. It's almost the same that Red Hat does. But instead of be around these products, we are offering people outsourcing slash consulting service.[P6]

%3:18 ¶ 73 in [P2] Daniel-Bitergia interview-D.docx
%contributing to open source is about demonstrating that you have the expertise and the knowledge contributing, and then you can serve specific pieces of knowledge. 

%3:21 ¶ 73 in [P2] Daniel-Bitergia interview-D.docx
%  we are serving with the community. So we are defining [a concept]. So it's not that we know about [name of the concept] is that we are defining [name of concept] and for this you can see that we have this pattern here about the maturity model, this pattern there about this or that and then we have these presentations at [conference name]
  
 % 8:72 ¶ 71 in [P7] Josep-Aiven interview - D.docx
%Because they were asking if we have some experts on X, Y, Z? And then it's like, yes, we have some. So of course, you can claim to be an expert, if you contribute regularly to a project



%4:37 ¶ 115 in [P1] Georg-Bitergia interiew-D.docx
%And we are active in these communities showing, showing off our skills and experience? And whether our potential customers are themselves active or just observing doesn't matter. It creates that contact to then have a conversation.

%Right. And, and where open source is superior is it gives us superior visibility into our own supply chain. Right as in, we don't have to-- with proprietary software, you know, if we have a proprietary component, like we were proprietary, authentication component, and we're like, hey, is this being maintained in the light of new security exploits? Right? In that case, you have to trust the vendor to know whether or not that's being maintained. You can't tell except your faith in that vendor. Whereas if it's an open source project, I can watch the contribution history and I can know whether or not it's being maintained. Now, on an aggregate basis, it's hard to keep track of all that information, but it's at least possible with proprietary software. You just have to guess.[P15]


%So our users trust us because we're giving them open source value... Also, because of that trust, and the way that you treat them, they are more likely to contribute feedback that is timely and relevant, which will further enhance your ability to produce good products. [P5]

%5:10 ¶ 137 in [P5] Jill-SlimAI interview.docx
%And they are more likely to stay with you because they are happier because of that trust. 

%5:49 ¶ 65 in [P5] Jill-SlimAI interview.docx
%capture the attention of developers who are curious, developers who like to solve problems, developers who like to engage in communities to work with other developers who have similar problems. Those are the types of people who will continue to benefit from the additional features that we build into our SAS.

%I mentioned community. I think that accountability to communities is also really important. It helps us make better decisions... There's like one example I think of, Microsoft wanting to put telemetry in the repo and the community was like, "No way!" And so they're like, "Okay." So you know, like that. But, you know, the act of that decision really helped build trust. So I think that the thing that Microsoft benefits from in working with open source--I'm trying to--Sorry-- I'm trying to mold this or nudge this out of my brain-- Is that, you know, that that trust piece, I mean, if you act with you know, if you show that you're, you know, as a company that you're willing to, to sidestep what you think is best and make sure that decision makings include the community, then you might have opportunity to go places and build trust you might not otherwise have. And that builds trust in the product, it builds trust in the company and Microsoft has a slogan of being built on trust. So I think that directly aligns with just generally, as a company, trust is important for us, in our customer, our customers feel about the work that we do. So the community is a great extension or example of that with communities. [P17]


%-----------Networking-------------  
  &
  \cellcolor{gray!15}\begin{tabular}[l]{@{}l@{}} Networking \end{tabular} &
  
  \cellcolor{gray!15}\begin{tabular}[l]{@{}l@{}} P1, P2, P5-P8 \\P13 \end{tabular} &
  
  \cellcolor{gray!15}\begin{tabular}[l]{@{}l@{}}
  %``We've had two three customers... the biggest ones we've  had... And they decided to go with us, not that much because of the open\\ source product, but... the open source knowledge that we are there we are positioned there. So then we should hire these people\\ because they know'' [P2]\\
  ``Puts us in a position where we are we are surrounded by other principle aligned practitioners'' (P5)\\ 
  
  %``we try to contact as many people as we knew, and then asking them, \\you know to introduce to ask new people and so on. 
  ``So we were kind of trying to nurture our network of people and learning from them'' (P2)

   \end{tabular} \\
%[access to big name companies]
  %And because of that, we have more contacts with the companies inside CNCF such us Apple, Microsoft, Red Hat again, and more companies so we engage our brand to be related with huge companies around the globe. [P6]
  
  %So this is like a good place where you can go and fish for potential customers. It is it is much more clear. In the case of the inner source commons, where there are corporations that are joining the foundation. And they say, well, I need someone, I need help with inner source I need you know, and because of this, then this is another good place where we can say hey, I don't know we we can help you understanding the maturity of your inner source journey. [P2]
  
%[surround with peer companies]
%5:66 ¶ 81 in [P5] Jill-SlimAI interview.docx
%events, puts us in a position where we are we are surrounded by other principle aligned practitioners. They could be other open source projects, they could be students, they could be other companies that have the same type of philosophies that we do about open source and cloud native. And so there are there are programs like, you know, being part of the Linux Foundation. You know, when when our when DockerSlim is running, it observes a container at the kernel. Right? Well, that kernel wouldn't exist without Linux.

%[opportunity to get hired]
%And basically, we try to contact as many people as we knew, and then asking them, you know, to, to introduce to ask new people and so on. So we were kind of trying to nurture our network of people and learning from them. And that point in time, we had, we started having some some others, because red hat was one of the very first customers as well, just because we had the OpenStack foundation as a customer.[P2]

%[networking]
%gives us access to decision makers, at companies that could potentially be customers. And so we use this also as a way for finding prospects and generating business for the company. And most of the business that we do have came through contacts that we fostered in being active in these communities.[P1]

%so we focus a lot of our, we'll call it marketing dollars, we focus a lot of our growth dollars on participating in activities and events, that put us in that same ecosystem as these other like minded organizations and practitioners. [P5]

%And if you think about a Brazilian, there's a third world third world country have our brand related to this company is a nice good step to us especially because okay, you will become more related with these companies that mean more clients coming[P6]

%One is the network that you've built. Um, and that, you know, we've also talked about in terms of like, how actually, this, you know, concretely helps you, in situation like, Hey, you wanna solve a problem? Will you actually know who to talk to?[P13]

 %So this is like a good place where you can go and fish for potential customers. It is it is much more clear. In the case of the inner source commons, where there are corporations that are joining the foundation. And they say, well, I need someone, I need help with inner source I need you know, and because of this, then this is another good place where we can say hey, I don't know we we can help you understanding the maturity of your inner source journey.[P2]
 
 %We've had two three customers large, the biggest ones we've had, that are proprietary ones. And they decided to go with us, not that much because of the open source product, but because we were part of the inner source commons. Yeah, the open source knowledge that we are there we are positioned there. So then we should hire these people because they know about the inner source in this case.[P2]
 
 %it's nice to be close and around Google, Red Hat, Apple, Microsoft, Spotify, and so on, and be related with those brands. It is nice to client as well. Right? So okay, you want to hire our service? Take a look at where we are. We are. So that is Google. That is apple. There's Microsoft, that's us. So did you see any competitors of us over there? No. So yes, it's off money if you hire us instead of any competitors. [P6]

%increase your brands, and be around people, because also have our business products to bring more people to pay attention us. it's nice to be close and around Google, Red Hat, Apple, Microsoft, Spotify, and so on, and be related with those brands. It is nice to client as well. Right? So okay, you want to hire our service? Take a look at where we are. We are. So that is Google. That is apple. There's Microsoft, that's us. So did you see any competitors of us over there? No. So yes, it's off money if you hire us instead of any competitors. [P6]



%------------Attract Talent-----------------
   &
  \begin{tabular}[l]{@{}l@{}} Attracting\\talent \end{tabular} &
  
  \begin{tabular}[l]{@{}l@{}} P3-P6, P8-P10, \\P13, P15 \end{tabular} &
  
  \begin{tabular}[l]{@{}l@{}}
  
  ``Open sourcing is one of the ways that that can make it much more interesting for outside candidates'' (P4) \\
  
  %``A lot of people decided to come here, be hired here, because the open source products'' (P6)\\ 
  
  ``It's good for recruitment...if my engineering tool is open source, and I need to hire new people, for the engineering team, \\the first group of people I look at is people who've made random contributions to that tool'' (P15)\end{tabular}\\
  
  
 % ``It's not about market share. It's about knowledge share, right? Like how quickly do developers want to reach for your tooling. Right?\\ Because that's a hiring tool. That's a hiring tool, right?'' [P3]\\
  %So this mindset also, cause many companies that didn't go open source to lose talent and don't retain talent. So this works also as a way to retain talent, like, Hey, we are not just a boring company in the financial systems, I don't know, for instance, like, we are a tech company, and we are promoting open source and we really want you to create things and linear space to do so. Right? And also to like your Yeah, like your personal brand gets in the pilot for for your future career.[P19]
  
  % Because you're like, you get to like use this open source library, which helps you which means you can get jobs at other places, right? Like you can like you can leave Facebook with React experience and go work again, or go work at google or Go work, you know, wherever you want to go. So that's part of it, too.[P3]
  
  %``because a lot of people decided to come here be hired here, because the open source products, ... a lot of people who is coming only because [company name] has open source products [P6]''

  
  %capture the attention of developers who are curious, developers who like to solve problems, developers who like to engage in communities to work with other developers who have similar problems. Those are the types of people who will continue to benefit from the additional features that we build into our SAS. [P5]
  
  % And so so open sourcing is one of the ways that that can make it much more interesting for for outside candidates that they, they want in work on something so interesting. And the other part is like if the system is interesting, they might be able to work on anything in the open on like an open source project that has a lot of traction in the world.[P4]
  
  %7:23 ¶ 59 in [P8] Johnathan-Goliath interview.docx
    %There's also another thing which we haven't been successful yet. But I've seen this is a different form of marketing, which is recruiting. I know a lot of developer product companies that use open source, they talk about all the stuff they use and how they use it. And when other people see that article or that blog post or whatever, like oh, that's that's cool technology I'm interested in, they seem like nice people. Maybe that'd be a cool place for me to work too and so I think that's an effective way for open source, commercial source companies to to create that kind of content for recruiting.
    
%10:45 ¶ 196 in [P9] Phillipe NexBe Interview.docx
%This is also a way for us to hire some of the very best in the field. It's difficult because we we're not Google or Facebook, and we don't have the ability to pay the same command station that launched and many, many tech companies would be able to pay and offer the set of benefits not even about pay, but just your whole set of benefits, without fight against. Knowing that we are bootstrapped and autonomous and self funded. But on the other hand, we have, we have a lot of other interesting 

%10:46 ¶ 200 in [P9] Phillipe NexBe Interview.docx
%but it happens. And that means that it's a very efficient way for us to attract and retain talent that couldn't be even able to do given our limited resources and size otherwise.

%11:25 ¶ 8 in [P10] Gil Interview.docx
%do I attract engineers? How do I retain engineers? Because they are working on open source projects they love to work in. You know, how do I appeal to their best nature to make them better engineers? 

%11:27 ¶ 11 in [P10] Gil Interview.docx
%I can hire somebody who already knows it. I don't have to pay for them

%[fostering developer love]
%1:26 ¶ 187 in [P3] Omer-VM interiew.docx
%You just you need-- developer Love is the currency of open source. 

%5:86 ¶ 99 in [P5] Jill-SlimAI interview.docx
%l moments where a developer finds our tools, loves them, tweets about them. We say thank you for sharing

%[bypassing recruitment/training steps]
%2:33 ¶ 89 in [P4] Per-Spotify interview.docx
%other thing is if if the thing that you're used to inside of the company actually become common knowledge of [non DDS?], then also you'd see your onboarding time of engineers go down. So you can see like, again, go back to the classic example, then Kubernetes, is now an industry standard. And so inside of Google, for instance, as I go, why using Kubernetes, and for them, it's like a big advantage that the industry is now just knows it as a standard thing. It would be much harder, if people had to go to Google and then learn Kubernetes from the bottom up, it would take so long to actually onboard people because it's but because it's actually a standard for how you build micro services, then any engineer that Google hire now will know Kubernetes. So for us it is the same is that we cannot do the same scale, because we know it's big but we do want to expose these things, and rather focus on on promoting them as a as one of the best ways of doing a thing instead of keeping it internal.

%11:29 ¶ 11 in [P10] Gil Interview.docx
%We, it's our own secret way of moving messages in a message bus. Like, why would I want I want to create a secret way of leaving messages in the message bus and then spend six weeks every time I hire somebody to teach them our secret way? And why would they want to learn our secret ways? It's completely useless when they leave.


%[hiring tool]
%It's not about market share. It's about knowledge share, right? Like how quickly do developers want to reach for your tooling. Right? Because that's a hiring tool. That's a hiring tool, right? You--React is a Facebook technology. We're using React to GitHub. Now, if they want to hire one of our engineers, they already know their tech stack. Right? Or they know at least like a familiarity enough with it, right? Like,[P3]

%2:28 ¶ 15 in [P4] Per-Spotify interview.docx
%Like a lot of internal teams inside of a big company do fairly interesting things, but no one knows about. And so open source is a good way of kind of getting the teammate name out in the world, and thereby also become a bit easier to actually hire interesting talent.

%2:30 ¶ 23 in [P4] Per-Spotify interview.docx
%so so open sourcing is one of the ways that that can make it much more interesting for for outside candidates that they, they want in work on something so interesting. And the other part is like if the system is interesting, they might be able to work on anything in the open on like an open source project that has a lot of traction in the world.

%2:56 ¶ 19 in [P4] Per-Spotify interview.docx
 %just like branding, like that a company, what they do, again, what they do inside is not very visible on the tech side, like a like an app like Spotify, it's actually quite hard to understand, like, how it actually works, technically. And why should that even be interesting? For many doing point of view, if you're like a top notch engineer at Google, or Facebook or Apple, why should you work at Spotify? It's just like an app that plays music. 

%6:12 ¶ 209 in [P6] Otavio-Zup interview.docx
 %the hiring process is a good way to use the open source products as well, because I can see the contribution the PR, if without do the [whiteboards boring?] that thing that everybody hates. So it's a crazy process to, to the interviewer to the interviewer. So it's bad. It's a win win situation with the open source.

%6:74 ¶ 159 in [P6] Otavio-Zup interview.docx
%The second one is to receive more goods, and high arts outstanding people, especially because a lot of people decided to come here be hired here, because the open source products, not everybody, we work in open source products, but a lot of people who is coming only because Zup has open source products and that's nice to us

%10:47 ¶ 196 – 200 in [P9] Phillipe NexBe Interview.docx
%Knowing that we are bootstrapped and autonomous and self funded. But on the other hand, we have, we have a lot of other interesting things in the work we do. The involvement we have with the open source community at large. All that makes us fairly niche. The fact that we involve mentoring else, committees of students, not by aspiring contributors to know about us, even though we're very tiny. 

%12:15 ¶ 118 in [P15] JoshB RedHat Interview.docx
%Number two, it's good PR, right. The number three, it's good for recruitment. And that's something I haven't talked about, you know, also within the Red Hat stuff, right, which is, hey, if my cast engineering tool is open source, and I need to hire new people, for the cast engineering team, the first group of people I look at is people who've made random contributions to that tool, because I already know they can write code to work on the tool, right? I already know that they are at least semi qualified. The plus it works the other way. Right? It works as a, hey, if I get a job at this company, I can work on this cool cast engineering tool.

%18:8 ¶ 310 – 312 in [P13] Tobie OSS consultant Interview.docx
%I mean, I think it does, or, frankly, companies that are good at open source, attract by like, attract, but I mean, it's, I mean, all of this works together, right? It's kind of like if you have a I mean, yeah, if you have a more more attractive set of people, and you're working in your organization working on exciting stuff. And that stuff isn't in the open, we can show show it off to like, potential other employees. I mean, that's attractive.


%-----------Adoption-------------
   &
  \cellcolor{gray!15}\begin{tabular}[l]{@{}l@{}} Fostering\\adoption \end{tabular} &

  \cellcolor{gray!15}\begin{tabular}[l]{@{}l@{}} P2, P4, P5, P8, \\P10, P11, P15,\\ P18 \end{tabular} &
  
  \cellcolor{gray!15}\begin{tabular}[l]{@{}l@{}}``It [open source] has been fundamental instrumental to our go-to market and our adoption'' (P8) % and just how we build our company'' [P8] \\ 
  \\
  
  ``They ended up eventually open sourcing theirs as well, because there wasn't any real adoption of it as a proprietary software'' (P15)\\
  
  %``And in fact, large tech companies compete with each other as to how much they can give away, and how much adoption they can get on\\ what they give away'' (P10)%So there's quite a competition between, oh, I'll open source something, you'll open source something they\\ both do about the same thing''[P10] \\
  
  \end{tabular} \\ 
 \hline
 
 %And there's such a wide variety of components. And these components have become standardized, because everybody's using it. So developers start trusting if if, you know, all of these companies are using it, it must be a good component, instead of building it themselves. [P18]
 
 %2:7 ¶ 11 in [P4] Per-Spotify interview.docx
%with the stuff they do. So they actually have open source projects, they want to like commercialize and make into a separate business. So besides just having an app, they also want to start selling software that actually started as open source software

%And there was an inflection point in the open source project in 2019, where we-- where Kyle saw massive adoption. There was like this overnight hockey stick in in in the amount of users the amount of stars on GitHub, like the project was just skyrocketing. And John and Kyle came together to have this conversation around: Could we-- could we take the value that we're seeing in this open source project and create a company that takes it to the next level. [P5]

%So this model is a product lead growth model. The product, the usage, adoption of the product. is the go to market strategy. We don't have a sales team. [P5]

%3:4 ¶ 17 in [P2] Daniel-Bitergia interview.docx
%One of them is pick up because you want to have a better positioning, for instance, there is we can think of software monopolies. So the best way to get your technology adopted is by open sourcing this, because then everyone will go to the open source option, in some cases, just because this is free, as in free beer, in some other cases,

% But we specifically chose to build our solution on top of an open source project Zephyr. And, you know, it has been fundamental instrumental to our go to market and our adoption, and just how we build our company. [P8]

%And in fact, large tech companies compete with each other as to how much they can give away, and how much adoption they can get on what they give away. So there's quite a competition between, oh, I'll open source something, you'll open source something they both do about the same thing. [P10]

% I mean, in some cases, it is but for a lot of cases, it's not a weekend warrior, that's going to create Kubernetes it's Google, who going to pay engineers full time all day every day and feed them to make a technology and then strategically give that to a foundation to grow in public, so that more people use a technology and oh, by the way, they have a cloud providing service that you pay for that, you know, has some sort of consistency with with that technology.[P10]

%9:46 ¶ 44 in [P11] Dawn-VM Interview.docx
%Because over time, we don't, we don't make money on that initial sale, right, we make money on them continuing to use this over a long period of time. So what we don't want is to, for salespeople to spend a whole bunch of time chasing someone has been not going to use it and not going to renew after the first year, what we want are people that are invested in the project, they know what it's going to do, they know what they need out of it. And, and they're going to commit to it for a long period of time. 

%9:65 ¶ 94 in [P11] Dawn-VM Interview.docx
%And you know, and another thing that especially enterprise customers really look for is, you know, they don't want to be locked into one vendor, right. And so, with platforms that are based on something like Kubernetes, they can use, they can use our, you know, our Kubernetes, but they can also move to somebody else's. So let's say it's, you know, they decide they don't like working with us anymore. They can move to, you know, Google's version, or Amazon's or Microsoft's, or, you know, anybody, anybody else's version of, you know, a Kubernetes type platform. So, so, you know, by not locking them in to a proprietary technology, it makes our customers I think, a lot happier and a lot more likely to work with us. And, and then our assumption is that we are so awesome that they won't want to switch to someone else. But they could. And so this gives them kind of a peace of mind when when choosing a solution that they're not going to be locked into it forever, even when it's not working for them.

%So you know, they did market their application system, they ended up eventually open sourcing theirs as well, because there wasn't any real adoption of it as a proprietary software. [P15]
 
 
%________________Business Advantage_____________________
 \multirow{8}{*}{\begin{tabular}[l]{@{}l@{}} Business \\Advantage \end{tabular}} 
 
%-----------Engineering need-------------
%   &
%  \begin{tabular}[l]{@{}l@{}}Engineering need \end{tabular} &
  
%  \begin{tabular}[l]{@{}l@{}}P2, P4, P7, P8,\\ P12, P13, P15,\\ P17, P18 \end{tabular} &
    
%  \begin{tabular}[l]{@{}l@{}}``it comes from a need from an engineering team that says we might want to use that thing. [P7]
%``Allows [companies] to build software \\ far more rapidly than we might otherwise''[P12]
%``Excellence in engineering, I think, you know high quality software, I think we want, \\ again, to be a humble participant in the ecosystem of which we benefit.'' [P17]

%  \end{tabular} \\
  
  %For us, I think, you know, if you, you know, especially with the the discussions around software supply chain security happening right now, and the, you know, kind of the executive order from the government, from the US government, specifically, around software supply chain security, like the, I think we as an industry are coming to this realization, that open source is an existential need, and the unwillingness to, to kind of get on the train, and participate in a very big way, is an existential threat to your, your business. [P16]
  
  %3:2 ¶ 17 in [P2] Daniel-Bitergia interview.docx
%Yeah and, another way of answering this question might be, because the some of the, I would say, some of the greatest technologies, pieces of technology, around the world, for specific contexts, are open source example, we can go for machine learning or big data. So if you are willing to start playing with the technology, or having that technology in place, that's definitely where you have to go.

  %2:51 ¶ 37 in [P4] Per-Spotify interview.docx
%And if you actually came up with this standard, then you also get a lot of exposure. Like Google got a lot of exposure from like, creating Kubernetes, which is like the industry standard for like management services. Now again, and Spotify wants to kind of create the same kind of narrative for like internal

%2:58 ¶ 37 in [P4] Per-Spotify interview.docx
%They, they want to leverage open source as much as possible. And the best way to do this is to make sure that what you are working on inside of your company, and the thing you depend on actually becomes the industry standard, not just something you maintain. Because if it becomes an industry standard, then you gain all the benefits from being part of the standards. And if you actually came up with this standard, then you also get a lot of exposure. Like Google got a lot of exposure from like, creating Kubernetes, which is like the industry standard for like management services. Now again, and Spotify wants to kind of create the same kind of narrative for like internal 

%2:65 ¶ 89 in [P4] Per-Spotify interview.docx
 %They do not want to change their infrastructure, that often they don't want to migrate into something new, the reason why they migrate is because a better thing has become the industry standard, basically. Yeah, so. So a benefit for us is that if we can take a component and open source and actually make that into an industry standard, or something that's like widely used and understood, that's better for us, because then first of all, you get external contributors chance to work on your project and use yourself
 
 %2:67 ¶ 37 in [P4] Per-Spotify interview.docx
 %But the the reason why they did this is because They, they want to leverage open source as much as possible. And the best way to do this is to make sure that what you are working on inside of your company, and the thing you depend on actually becomes the industry standard, not just something you maintain.
 
 %8:28 ¶ 23 in [P7] Josep-Aiven interview.docx
%and understanding Postgres, it's better than I operate. Oracle and then on top of that, there is the licensing costs and who knows Oracle only if you use your Oracle, you need to do certifications. You can not that It's easier when it's everything is open source.

%8:70 ¶ 75 in [P7] Josep-Aiven interview.docx
%one is the consumption part. So one thing is that we get to do things that other ones already did, right. So sometimes they already guides there, some, some people already did some thinking about some processes, and we can just benefit off that one. So we can just use things directly from from people that put some brains and time and thought on that. So that's that's one part. 

%8:102 ¶ 119 in [P7] Josep-Aiven interview.docx
%it comes from a need from an engineering team that says we might want to use that thing. Y


%7:4 ¶ 19 in [P8] Johnathan-Goliath interview.docx
%we look to either use open source whenever we can, for our own stuff, as well as open standards. And the thing with open standards is they're only open in open source, even if there are few open standards that exist, they're proprietary, but fundamentally, are not adopting the industry. So therefore, everything we do in the world of IoT is some sort of standard, and therefore, we can only use open source. 

%16:47 ¶ 85 in [P12] Joshua tidelift Interview.docx
%Open source allows us to build software far more rapidly than we might otherwise.

%16:56 ¶ 109 in [P12] Joshua tidelift Interview.docx
%invest in the legal, you know, making sure that we're we're good with our licenses.


  %What they did instead was to offer that for free. Right, and those you know, and those same players haven't bought those licenses, would have been able to sell those servers to competitors of Facebook. Right. And so what Facebook did instead was to say, well, we're just going to give those for free. Everyone can use them. Right. And that enabled competition, because folks can just build those. Right? And obviously, the more players have an somewhere like the the cheaper things, because there's more competition, right? And people are ready to take smaller margins because they're invested. And they don't have to pay for the licensing fees.[P13] 
  
  %12:13 ¶ 96 in [P15] JoshB RedHat Interview.docx
%um mostly because the lawyers don't have to meet. Right, there's an established set of practices of oh, hey, you have this, you know, you have this CLI statement, and you have this CLA, and you have this open source license, and you're good to go.

%Well, I mean, excellence in engineering, I think, you know high quality software, I think we want, again, to be a humble participant in the ecosystem of which we benefit. And excellence in engineering, especially around security, you know, we're part of the OpenSSF, we find a lot of security initiatives and in projects.[P17]

%And one of the first ways companies get involved is they start consuming open source, because developers today, modern software development really includes building with open source components, because they're so available. And there's such a wide variety of components. And these components have become standardized, because everybody's using it. So developers start trusting if if, you know, all of these companies are using it, it must be a good component, instead of building it themselves. [P18]

%-----------Business depends on OSS-------------
   &
   
  \begin{tabular}[l]{@{}l@{}} Business\\dependency\\on OSS \end{tabular} &
  
  \begin{tabular}[l]{@{}l@{}} P2-P5, P7, \\ P9-P12, P14- \\P17 \end{tabular} &
  
  \begin{tabular}[l]{@{}l@{}}
  %``We want these projects to be healthy, alive and well maintained, and sustaining. Because we're depending on them'' [P9]\\
  ``Most systems as [product name] is built on open source software, that is like the the basis of everything is open source dependencies. \\ Then our engineers also contribute to open source projects  that they depend on'' (P4)
\\
``So we have a product line called [product name], which is kind of our flagship like looking forward most strategic project \\that we have at [company name], and that's built entirely on top of [OSS dependency]'' (P11)
  \\
  %``There's a journey that companies seem to go on from like... thinking they don't use it, to realizing that they do... to a point where \\companies start realizing they need to give back ... Ultimately, they know at this point, that a healthy ecosystem means good\\ business for them, too'' [P12]
  \end{tabular} \\ 
  
  %So producing open source is might be a business advantage, depending depending on where you are in the market or what your purposes are. [P2]
  
  % the same for the production, so then you won't be part of the of the communities to have influence in the roadmap to define where the project goes or, it's so there is let's say, like, CTO study level decision there. Or it's because the people within the company they are using, and perhaps producing open source on their own, but not because there is a general policy.[P2]
  
  %1:4 ¶ 39 in [P3] Omer-VM interiew.docx
%So the name of the game for open source contribution is basically like, commits, and then governance and leadership, right? So getting elected to leadership positions in these communities is really important. 

%1:8 ¶ 67 in [P3] Omer-VM interiew.docx
%VMware is Ford, Ford really gives a shit what happens to that engine, right? Because if something happens to that engine that they have to that creates work for Ford, right? Or even worse, makes it so that Ford can't use the engine the way they were using it before. Now you've got a problem. Right?

%2:68 ¶ 45 in [P4] Per-Spotify interview.docx
%Well, it's sometimes, it's just because they they have an issue with the product. And they just fixed it up. That's it. So that you're scratching your own itch kind of thing.

%So first of all, it consumes a lot of open source. And most systems as Spotify is built on open source software, that is like the the basis of everything is open source dependencies. Then our engineers also contribute to open source projects that they depend on. And this is this is on various formal, informal ways. But they contribute where they see a need, basically.[P4]

%2:104 ¶ 117 in [P4] Per-Spotify interview.docx
 %if you just depend on open source code, and then when you start depending on open source code, there's a very natural register as contributing to open source code

%He, he believes in open source, he's a consumer of open source projects, we have-- we as a business consume other open source projects in our product delivery process, right. Like, we have other projects that our developers use that are open source.[P5]

%5:48 ¶ 77 in [P5] Jill-SlimAI interview.docx
%And so for us to patron Beautify or Nuxt, which are some of the things that we use at Slim, that really matters to us. We actually want to see those projects continue, because they're not only doing good for other developers, but they're doing good for us.

%So everything is a journey, in that sense. So the goal is obviously to try to impact as many on the governance of the projects as possible to be as again, for the same thing as before, to guarantee that it's not just one company behind and to guarantee different opinions, voices and points of interests. It's a journey depends on the project, and depends on how easy is to get into those circles on the projects. Some projects are more closed, and some projects are more welcoming. [P7]

%And we, it's, we're very selfish, we want these projects to be healthy, alive and well maintained, and sustaining. Because we're depending on them. So we ensure that we we help them we contribute within our meetings, which are small, whatever we can to ensure their well being that means reporting bugs, fixing bugs when we can, promoting them when we can, and so on and so on.[P9]

%11:37 ¶ 41 in [P10] Gil Interview.docx
%I might want to fix it, I might want to contribute that fix. And because they might fix it wrong. Right? I might want to make sure that our engineer fixes it to ensure that our fix is the one that gets put in the upstream project. Right. So on that case, of course, I want to contribute back.

%11:47 ¶ 47 in [P10] Gil Interview.docx
 %definitely want to fix a bug in Kafka because that thing is so slow. And if we can make it faster,


%Yeah, we think that being in leadership roles, and being part of the governance is really important. And so for a lot of the projects that we work on, we do expect our contributors to eventually try to move into maintainer positions approver positions, leadership.[P11]

%9:29 ¶ 24 in [P11] Dawn-VM Interview.docx
%Because the reality is, you know, once you once you get involved in these communities and are involved at a deep enough level, that you're moving into leadership positions, that's where you really get the benefit from it

%9:32 ¶ 28 in [P11] Dawn-VM Interview.docx
%I wish we had more contributions to projects that we just use, as opposed to the projects that we build our business on.

%9:34 ¶ 28 in [P11] Dawn-VM Interview.docx
%in those leadership positions, gives you gives you a couple of different things it gives you, like I said, insight into the project, 

%9:35 ¶ 28 in [P11] Dawn-VM Interview.docx
%also gives you and also gives you influence. 

%9:36 ¶ 28 in [P11] Dawn-VM Interview.docx
%you know, you can better better know what types of contributions they're likely to accept, and which ones they aren't. 

%9:37 ¶ 28 in [P11] Dawn-VM Interview.docx
%what we what we do really encourage our individual teams is, you know, not to fork the project and carry all of the technical debt internally, where they're trying to patch it when a new release comes out. And it's it's just an awful lot of work. So we're trying to convince more and more business units that they need to contribute these fixes back upstream

%9:38 ¶ 28 in [P11] Dawn-VM Interview.docx
%it improves security, there are just so many benefits to that. 

%9:66 ¶ 94 in [P11] Dawn-VM Interview.docx
%without something like Kubernetes, we, we wouldn't have been able to spin up this Tanzu product line in a relatively short period of time, considering, you know, how long it normally takes to build a product at that sophistication, but, but by having something like a Kubernetes, we can just innovate on top of that, because we have, we have the platform. And you know, 

%16:26 ¶ 49 in [P12] Joshua tidelift Interview.docx
%Right. So that's probably more probably more opportunistic, like, hey, we ran into a sharp corner, we found a fix for it. Here's the fix. Yeah, more of that style.

%16:37 ¶ 69 in [P12] Joshua tidelift Interview.docx
 %because the business model only makes sense because we have the open source ecosystem that we have, and what I mean by that is, you know, 

%And it's only as companies grow and start applying tooling to understand how extensive their use of open source is, when they start realizing like, oh, we need to invest in the security of this, we need to invest in the legal, you know, making sure that we're we're good with our licenses. [P12]

%17:21 ¶ 6 in [P14] Harish RedHat Interview.docx
%And if you've never done that before, that's why we try and help them. So that's the other part of the story. We from an OSPO perspective, there are organizations that are struggling trying to figure out how to do this. And they reach out to us for help. And so OSPO trying to fill that role and from Asia Pacific perspective, that ends up becoming me doing it with my colleagues, my colleagues in whichever city that they may be in it could be in Japan, it could be in Tokyo or in Beijing or you know, wherever, whichever goes if if there is if there is a requirement, let's see what we can do.

%12:2 ¶ 26 in [P15] JoshB RedHat Interview.docx
%So I mean, obviously, from a technical basis, how we're involved in open sources, that's how we build all of our stuff, right? Every Red Hat product starts with an open source project, where we do engineering, either by ourselves or together with contributors who work for other companies or no company

%12:11 ¶ 92 in [P15] JoshB RedHat Interview.docx
%And I would say, right, and I would say, it depends, right. And that usually depends on what our level of commitment is to the project, right? Because if it's a public project, in order to get any influence over things like roadmap, you have to have somebody who is a major contributor to that project. Or you have to be financially sponsoring that project at a substantial level. Right? Because otherwise, why would they listen to you? Yeah. The, I mean, obviously, if it's a project that we started, we have, you know, control over all of these things. Or if it's a project that we co founded, with other companies, which happens a fair amount, then we have whatever our sort of equal say, is over those things.

%19:14 ¶ 50 in [P16] Stephen Transcribed Interview.docx
%Right, I think that, you know, because often you write, you write a check, and it's for a foundation, and it's large enough, then there may be a, you know, say a board seat associated with it. Right. And right there, you you get an immediate way to, to kind of influence that project or set of projects. And that's great. 

%Well, I mean, the goal is that all upstream projects are being contributed to so if we're using an open source project, and we find a bug, or there's something that we can be, you know, that we will always contribute to upstream projects. There are lots of folks at Microsoft who contribute to open source on their personal [tell end?]. There are folks that we-- belong to [use?] and we call our open source stats program inside of Microsoft, who mentor and help others learn how to do that. So to answer your question, lots of different ways, but it's very much part of the culture.[P17]

%Well, I think security both ways, right, like making sure that any security issues that we find, or you know, that we're pushing fixes to those up and that we also benefit from those, I think security is one of those, like probably the most important example.[P17]

%-----------Coopetition-------------  
  &
 \cellcolor{gray!15}\begin{tabular}[l]{@{}l@{}}Coopetition  \end{tabular} &
 
  \cellcolor{gray!15}\begin{tabular}[l]{@{}l@{}} P3, P7, P8, \\P11, P13-P15, \\P17-P19 \end{tabular} &
  
  \cellcolor{gray!15}\begin{tabular}[l]{@{}l@{}}%``it's an implicit resource sharing agreement between companies that are working on the same problem'' [P3]
  ``The benefit is open collaboration, you know, I don't have to be the only person trying to figure out how to solve a problem'' (P14)
  \\
  ``You need to get...multiple vendors involved so that you can a) like improve your product because you have to like be accepting \\ opinions from lots of different people and then b) you get more market share that way'' (P3)
  
  \end{tabular} \\
  
  % So taking this back to like the technology space, right, like a serverless offering is great on its own. But when you're a major corporation, you want to buy, like these open source projects, you want to buy services around them. That means that the expertise about how to implement that technology, right, the support when that technology goes down, some kind of guarantee on security vulnerabilities, right. And these are the sort of things like if you can imagine the car example these are the cupholders. This is the radio. This is the shape of it, right. Yeah. So so that's sort of how These these companies are, are collaborating with one another, and they're doing it in an open way, because it doesn't actually matter if Joe Schmo gets an engine, right? Like, it doesn't matter if their competitors get an engine, that's not the competition, the competition is everything that only these larger companies can do, like provide you with like services and provide you with security. And they sort of give you this, this full package that is surrounding that, that core technology that is much more valuable than technology itself. [P3]
  
  % If five cloud companies have the same challenge, it makes sense to work together to solve that problem, and not solve it through partnerships and agreements, because that's a very heavy legal lift. But to come together in a foundation and to say, let's create a charter, that's, you know, maybe one company has started solving that problem, and then release that code to that foundation, when everybody else starts working on that, and can continue to improve that. And I think that's how they're, you know, networking, open source networking at the Linux Foundation, it started with at&t contributing code, and saying, Hey, we did this to solve our own problem. But it feels like, you know, everybody's struggling with the same thing. And maybe if we can all work together, we can improve this, so it reduces the burden for AT&T to do have to develop this all by themselves. And it also helps others who have the same problem. [P18]
  
   %And I think that's how they're, you know, networking, open source networking at the Linux Foundation, it started with at&t contributing code, and saying, Hey, we did this to solve our own problem. But it feels like, you know, everybody's struggling with the same thing. And maybe if we can all work together, we can improve this, so it reduces the burden for AT&T to do have to develop this all by themselves. And it also helps others who have the same problem. And so and, and these structures, like foundations were created to have this neutral place, where companies can come together and collaborate.[P18]
   
    %So open source is not separate or optional, but it's central to excellence in open source engineering, and culturally speaking, very much in line with our values of building on the work of others and finding others to build with us.[P17]
 
  %1:3 ¶ 243 in [P3] Omer-VM interiew.docx
%ou need to get you need to get you need to get multi vendors like multiple vendors involved so that you can a) like improve your product because you have to like be accepting opinions from lots of different people and then b) you get more market share that way.

%1:12 ¶ 247 in [P3] Omer-VM interiew.docx
%Yeah, think about it. Like Think about it like this, right? You have. So this is where the this is where the car analogy starts to fall apart. Okay, if I have a bunch of work work, like, if I have a bunch of services on Red Hat's OpenShift? Which is you--like their Red Hat serverless is Knative under the hood. If I want to move that to VMware. I have a lot easier of a time moving that over than if I was doing it in Google with their own proprietary stuff. Because the basis is the same. And so now these companies can steal customers from each other.

%1:14 ¶ 31 – 32 in [P3] Omer-VM interiew.docx
%You can imagine auto manufacturers, like, all need an engine for their for their cheap car, they all need a thing that gets in gasoline. And like gets out. Like horsepower, right? Like, like, and the KPI is miles per gallon, right. So the the further you can go on a gallon of gas, the better the engine is, the cheaper it is to make, the better the engine is, there's all these things that are beneficial for all of these companies who make this same car. Now, this would never happen in real life, because this is called collusion. Or, you know, there's like there's like anti monopoly rules against this, right, because of the way that like certain large organizations work. But there is still some open source work that's being done in these like manufacturing sectors. So you can imagine, like, if I'm Red Hat, or if I'm VMware, or if I'm Google, and I want to offer this, like this car, so to speak, and I know that like, I need to build an engine for it. And I'm looking around and realizing that like, all of my competitors need to build this engine for it. And people don't really buy a car for the engine, they buy a car for how it looks, they buy a car for the features, they buy a car for safety, they, you know, they're not really buying the car for the engine, there's all these little things around the engine that make a car worth buying. Right? 


%1:15 ¶ 21 in [P3] Omer-VM interiew.docx
%view open source work, especially when it's like a couple of large vendors who are working together as staff augmentation, right? They have, they have 20 engineers, those 20 engineers will go further working on a project with 80 other engineers from four different companies, and they will if you try to solve that same problem by yourself, so functionally, it's a it's an implicit, not explicit, right. So it's not written down. But it's an implicit resource sharing agreement between companies that are working on the same problem as an example. And I'm just gonna keep talking until you stop me by the way, I have a lot of fun.

%1:27 ¶ 21 in [P3] Omer-VM interiew.docx
%that same problem by yourself, so functionally, it's a it's an implicit, not explicit, right. So it's not written down. But it's an implicit resource sharing agreement between companies that are working on the same problem as an example. 

%1:28 ¶ 21 in [P3] Omer-VM interiew.docx
%is a way for them to do the hard work only once, right, and to do it in a way that is like,

%8:74 ¶ 75 in [P7] Josep-Aiven interview.docx
%then there is the other one, which is we can also share these things back so we can also share, what do we do that is different that maybe some other ones will try to do or show another way, show how we can how we did it and what we're doing and what works and what doesn't. And then of course, it's sharing experiences between different groups and people learning from each other.

%8:75 ¶ 75 – 76 in [P7] Josep-Aiven interview.docx
%creating as a group also like some kind of nice ideas, nice ways to go nice, nice roads to point to other companies to say these other ways. These are the different ways to do it. These are the different roads you can take. Pick the one that fits best to you. So that's one of the things that we buy being present in these groups. Of course, it's like the getting some information, obviously, shaping different opinions, different ideas, different ways of doing things.

%7:27 ¶ 85 in [P8] Johnathan-Goliath interview.docx
%There's that but also just like, collaboration and partnerships become six orders of magnitude easier, because all that is open source. That's probably the two biggest reasons.

%um, yeah, so I would say that the biggest benefit that we see is innovation, because we can only do so much on our own right you know, we we hire whole teams of people who are working on similar things. tend to think of things in similar ways. And you get a real innovation boost by pulling in people who work from other companies who will use things differently than you do a lot of different different use cases, they'll have different skills to contribute, they'll have different ways of thinking about things. [P11]

%17:10 ¶ 88 in [P14] Harish RedHat Interview.docx
%The benefit is open collaboration, you know, I don't have to be the only person trying to figure out how to solve a problem. Let's go work with open source community and see, you know, other ideas, other people, you know, doing different things, oh, they fix that problem over there. Okay, let's fix this problem here. Oh, this connects, it gives you enough some examples, right? There was this, okay?

%17:14 ¶ 102 in [P14] Harish RedHat Interview.docx
%I feel that you tend to be a little bit more accountable. It's no longer a fire and forget kind of thing. Well, I write the code, I, you know, put it into the repository, and it gets compiled, and then I don't know what happened to you. I don't know who's using it. Not okay. But here, yeah, you know, because my name is going to be there. I may want to care that, you know, someone says, Hey, hurry, share the code that you wrote was horrible. Or what did I do? Okay, let me go and see what what did I do, right? That I think is the power of the model of this, this open collaboration, and this is the benefit that we get, as well as our customers or customers also, then the customer can turn around and say, you know, we got some engineers who like to work on this, this aspect of this product? Can you can we collaborate with you? Yeah, sure. You want to fix some bugs? Let's start with fixing some bugs first.

%18:4 ¶ 234 in [P13] Tobie OSS consultant Interview.docx
%One is the network that you've built. Um, and that, you know, we've also talked about in terms of like, how actually, this, you know, concretely helps you, in situation like, Hey, you wanna solve a problem? Will you actually know who to talk to?

%18:20 ¶ 60 in [P13] Tobie OSS consultant Interview.docx
%it's fairly easy to waste a week on an incredibly dumb bug, that if you actually know everyone in the community, you know, the key people in the community building that software, you're going to solve in like, 15 minutes, because you're going to email someone that you know, and say, hey, I'll just chat them and say, hey, you know, John, whatever, I'm bumping into this weird issue, what's going on? And you know, they will get back to you in a second. You know, and like, within minutes saying, oh, yeah, this is a known problem. There was a patch, look at that pull request here, blah, blah. All right. So these aspects are, are usually more valuable.

%12:25 ¶ 26 in [P15] JoshB RedHat Interview.docx
%together with contributors who work for other companies or no company. The second way is considered the ideal. Right? If we can make that happen, we do make that happen. Because, you know, why would you put in 100% of the effort if you didn't have to, plus, getting contributions from outside Red Hat often bring with them perspectives from outside Red Hat, which allow us to make products that fit a broader clientele.

%-----------Closer Channels-------------
   &
  \begin{tabular}[l]{@{}l@{}} Closer\\channels  \end{tabular} &
  
    \begin{tabular}[l]{@{}l@{}} P1, P5, P7, \\P13-P15 \end{tabular} &
    
  \begin{tabular}[l]{@{}l@{}}  
  %``One is the network that you've built. Um, and that, you know, we've also talked about in terms of like, how actually, this, you know, \\concretely helps you, in situation like, Hey, you wanna solve a problem? Will you actually know who to talk to?'' [P13]
  
  %``direct feedback...That is a huge benefit... you can look into the contributors\\ file and see who contributed... want to talk to them directly? Go ahead and do it'' [P14]
 
 ``I want to be able to, you know, how are my customers using it? What are the challenges they are facing?...I don't want it to be, \\you know, only coming by feedback and then somebody collects all the feedback and then massages the feedback'' (P14)
% \\
%``If you actually know everyone in the community...the key people in the community building that software, you're going to solve in like 15\\ minutes'' (P13)
  
  \end{tabular} \\
  
  %Yes, or because they, the companies come with an interest. And we are active in these communities showing, showing off our skills and experience? They're saying, Oh, yes, we want more of that. Can you dig in more? Show us more what, and we basically use these communities to showcase what we can do. And whether our potential customers are themselves active or just observing doesn't matter. It creates that contact to then have a conversation.[P1]
  
  %5:33 ¶ 137 in [P5] Jill-SlimAI interview.docx
%So our users trust us because we're giving them open source value. Our early adopter SAS users trust us because they're using the platform, and they're getting value from it. And we're listening to their feedback in both the open source and the SAS

%5:57 ¶ 31 in [P5] Jill-SlimAI interview.docx
%You know, one of the things that's really cool about having an open source community is is you learn from them just as much as they learn from you. And, and so that that's woven into how we operate the business, is taking the user feedback, taking the user learning, taking the user, the new problems that users want to solve, like taking all of that and ingesting that back into our product delivery process. 

%And that's what it does. And of course, Aiven believes in open source because by using open source, you have big, bigger reach. [P7]

%18:4 ¶ 234 in [P13] Tobie OSS consultant Interview.docx
%One is the network that you've built. Um, and that, you know, we've also talked about in terms of like, how actually, this, you know, concretely helps you, in situation like, Hey, you wanna solve a problem? Will you actually know who to talk to?

%18:20 ¶ 60 in [P13] Tobie OSS consultant Interview.docx
%it's fairly easy to waste a week on an incredibly dumb bug, that if you actually know everyone in the community, you know, the key people in the community building that software, you're going to solve in like, 15 minutes, because you're going to email someone that you know, and say, hey, I'll just chat them and say, hey, you know, John, whatever, I'm bumping into this weird issue, what's going on? And you know, they will get back to you in a second. You know, and like, within minutes saying, oh, yeah, this is a known problem. There was a patch, look at that pull request here, blah, blah. All right. So these aspects are, are usually more valuable.


%17:12 ¶ 88 in [P14] Harish RedHat Interview.docx
%Another example is this will have to do with real time, fix real time, Linux kernel, article, real time kernel. So what is real time kernel, the request came from a customer. That was interesting, this actually came from a customer which is which happens to be the US Navy, the US Navy, because all the naval ships are running Red Hat Linux inside, which is, you know, all this, you know, systems

%17:13 ¶ 96 in [P14] Harish RedHat Interview.docx
%That's right, direct feedback. Yes, exactly. Exactly. That is a huge benefit. That's a big benefit. Like, you can look in the when you install [rel-- Raheny]? precedents, for example, Fedora or CentOS, or whatever, you can look into the contributors file and see who contributed you have all the email name, names and email addresses that you want to talk to them directly? Go ahead and do it

%17:14 ¶ 102 in [P14] Harish RedHat Interview.docx
%I feel that you tend to be a little bit more accountable. It's no longer a fire and forget kind of thing. Well, I write the code, I, you know, put it into the repository, and it gets compiled, and then I don't know what happened to you. I don't know who's using it. Not okay. But here, yeah, you know, because my name is going to be there. I may want to care that, you know, someone says, Hey, hurry, share the code that you wrote was horrible. Or what did I do? Okay, let me go and see what what did I do, right? That I think is the power of the model of this, this open collaboration, and this is the benefit that we get, as well as our customers or customers also, then the customer can turn around and say, you know, we got some engineers who like to work on this, this aspect of this product? Can you can we collaborate with you? Yeah, sure. You want to fix some bugs? Let's start with fixing some bugs first.

%17:35 ¶ 90 in [P14] Harish RedHat Interview.docx
%So, three of us, we came together to say, Okay, what do we need to do to make the Linux kernel real time? Okay, sorted out the thing and he made it made it work a real time. Then, of course, how does react do what we do we upstream the whole thing, we send the improvements and fixes upstream to the Linux kernel itself. And where did that then show up? That's real time component arrived at the Linux kernel that was shipped by Red Hat to our customers landed in a customer in Chicago Mercantile Exchange. 

%17:36 ¶ 90 in [P14] Harish RedHat Interview.docx
%These are the kind of innovations that you know, could we have done it on our own? Yeah, but maybe we could, but you know, it's gonna take a long time. And we have a real customer who has a real requirement to test something, we can test it make it happen, and then we make it available to everybody else.

%17:37 ¶ 96 in [P14] Harish RedHat Interview.docx
%Here you can, go ahead and reach out directly. We are not we are not putting a firewall. They say no, you cannot talk to my developers. Because the developers must also understand what the customer's requirements are. Because if I'm building something I want to be able to, you know, know, how are my customers using it? What are the challenges they are facing? It I don't want it to be, you know, only coming by feedback and then somebody collects all the feedback and then massages the feedback.

%12:25 ¶ 26 in [P15] JoshB RedHat Interview.docx
%together with contributors who work for other companies or no company. The second way is considered the ideal. Right? If we can make that happen, we do make that happen. Because, you know, why would you put in 100% of the effort if you didn't have to, plus, getting contributions from outside Red Hat often bring with them perspectives from outside Red Hat, which allow us to make products that fit a broader clientele.

%12:26 ¶ 26 in [P15] JoshB RedHat Interview.docx
%also as a way of learning what is it that people want, so that we can actually build those products? Right. Um, so marketing in terms of market research, and in terms of getting direct feedback? You know, like, if we're thinking about something new in Linux, we can put it in fedora and see how it does in fedora. Um, you know, before we have to offer five years of support for it.

  
%-----------Innovation------------- 
   &
  \cellcolor{gray!15}\begin{tabular}[l]{@{}l@{}}Innovation \end{tabular} &
  
    \cellcolor{gray!15}\begin{tabular}[l]{@{}l@{}}P11, P13, P16, \\ P17, P19 \end{tabular} &
 
  \cellcolor{gray!15}\begin{tabular}[l]{@{}l@{}}%``Innovation is a big part of our of our open source story... because we can... innovate with the rest of the ecosystem around these \\ core, open source technologies'' [P11]
  
  ``I would say that the biggest benefit that we see is innovation, because we can only do so much on our own'' (P11)
  \\
  ``Innovation is the other one...So the more that we're able to collaborate with external communities...The better the product'' (P17)\end{tabular} \\\hline
  
  %um, yeah, so I would say that the biggest benefit that we see is innovation, because we can only do so much on our own right you know, we we hire whole teams of people who are working on similar things. tend to think of things in similar ways. And you get a real innovation boost by pulling in people who work from other companies who will use things differently than you do a lot of different different use cases, they'll have different skills to contribute, they'll have different ways of thinking about things. [P11]
  
  %So we contribute to some machine learning projects, because we think that that is going to be a way that we can, you know, evolve VMware projects and products in some way. We don't, we don't quite know how yet, but we're, we're contributing and getting more familiar with it. We're also doing that in an area of [refrigerants?] observability. So it's things like, it's more like logging and tracing and trying to better understand what particular platforms are doing so that you can make improvements to them. [P11]
  
  %And open source is exactly the same thing. It's like, if you actually really do it that way, working in the open and being confronted with other neat use cases was how other people use stuff. It's kind of when you pick up a book, right? It's like, oh, it opens your mind to all of these things you had never thought about. But so this works in this generative of innovation. [P13]
  
  %18:3 ¶ 160 – 162 in [P13] Tobie OSS consultant Interview.docx
%And so open source, like is going, like completely changes that by essentially giving you just open forum, where, like minded people from completely different places in the world, and different verticals, and different, you know, profits and nonprofits and ad and like business and you can come together. Um, it was like, a shared understanding of what they're coming together for and a shared understanding of like, speaking about it and dealing with it. Right. And that is extremely unique. And that is actually extremely hugely effective for Lots of different things. I mean, this is how you innovate, right? You like confront completely different perspectives in the place where there was a mutual understanding of the underlying language. So, so that's the kind of things that I find that are, you know, it's hard to explain that about open source. 

  %18:19 ¶ 52 in [P13] Tobie OSS consultant Interview.docx
%large companies do invest into, you know, do leverage that information to make decisions about, you know, unrelated, well, decisions that are not pertaining to that particular project. But, you know, for example, if you see, like, lots of activity around a whole bunch of, you know, types of open source project that, you know, clearly shows that there was interest for that kind of software. And so it might be valuable to, you know, build solutions around this, for example, or even clear, even proprietary solutions.

%19:9 ¶ 44 in [P16] Stephen Transcribed Interview.docx
%So in emerging, you know, so being in emerging tech incubation, there's this this requirements, to, to be a bit ahead of the curve, right, and try out things that may not be may not be the, you know, directly, you know, immediately core to the business. But will be useful in you know, in the six months to 12 months to two years, so on and so forth. 

%19:12 ¶ 44 in [P16] Stephen Transcribed Interview.docx
%So that's our business level group. And, yeah, we, you know, there's, there's that need to be always looking ahead, like, you know, lizard steering to where the puck is going, kind of thing. So,

  %I think innovation is the other one, especially when you-- with the lens of diversity and inclusion. You know, we could have some of the smartest engineers, you know, [Speaker retina?], or 12 of the smartest engineers in the world, building something in a room together. But at the end of the day, that's still just represents those 12 engineers', you know, worldviews and experiences. And you know, if they've, you know, felt exclude, like, there's all different layers to it. So the more that we're able to collaborate with external communities, the more perspective, the more, you know, kind of like user stories or use cases or whatever word you want to use there. The better the product you're building, the more likely it is that it will solve problems for more people and not just those 12 extremely bright people in the room. Right.[P17]
  
%____________________Reciprocity__________________________
 
  \multirow{1}{*}{\begin{tabular}[l]{@{}l@{}} Reciprocity  \end{tabular}}  

%-----------Sustain ecosystem-------------  
   &
  \begin{tabular}[l]{@{}l@{}}  \end{tabular} &
  
    \begin{tabular}[l]{@{}l@{}}P4, P6, P7, \\P12-P14, P16, \\P17 \end{tabular} &
 
  \begin{tabular}[l]{@{}l@{}}
  %``As much as... every company can, in its own capacity, everyone should be contributing back, everyone should be making sure that \\they are not just the receiving end'' [P7]
 
  ``From the perspective of being a...trustworthy, honest and sincere participant in open source that we want to be part of the ecosystem,\\ you know, giving back as much as possible'' (P17) \\
  
  ``We can also share, what do we do...it's sharing experiences between different groups and people learning from each other'' (P7)
  \\
  %``I think tech plays a huge role in society. And there are not a lot of people that understand the technology and how the technology\\ is really made... so I've essentially tried to move increasingly in the space where I could be an advocate of this and have been trying\\ to find my way through that'' [P13]
  \end{tabular} \\ \hline
  
  %2:69 ¶ 121 in [P4] Per-Spotify interview.docx
%ou already mentioned that we are very bad at actually releasing something and then continue caring about it the next few weeks, the actual benefits of open sourcing it.

%6:8 ¶ 179 in [P6] Otavio-Zup interview.docx
%eah, we are in the events, talking about it, spreading the word. And letting everybody know that contributing to open source is good. It's good for you and your company, because you're using them. So I think that's it. I can't remember anything else.

%6:33 ¶ 117 in [P6] Otavio-Zup interview.docx
%And we say it a lot, we try to spread the word about open source, because we all use it, right? We, we use it all the time when we're delivering when we are creating software.

%8:33 ¶ 39 in [P7] Josep-Aiven interview.docx
%the second one is the giving back. So we took lots of things from open source. And we need to give back it's not, it doesn't work, it doesn't scale. And it's not a sustainable thing to do. So as much as every one can, every company can, in its own capacity, everyone should be contributing back, everyone should be making sure that they are not just the receiving end, 

%So for us, we think that this is a systemic problem in the open source ecosystem. And it's not just us, who think that you know, with the Heartbleed, bug in 2014, really kicked off the huge discourse about open source sustainability.[P12]

%17:15 ¶ 110 in [P14] Harish RedHat Interview.docx
%That means we have expanded the ecosystem. We have created more people with skills in this space. So we have broaden the economic pie across industries across economies around the world. To me that is extremely powerful. Extremely powerful. You don't have to always come back to me. Yeah, I enable you, you fly, you stop lying. And if you do well, good for you.

%19:5 ¶ 17 in [P16] Stephen Transcribed Interview.docx
 %they're more about sustainability, right? And there are so many, there are so many different angles that you can you can consider when you're thinking about the sustainability of an open source project, but sustainability of like, open source foundations, that's, that's not just limited to writing the check.

%That's really important to Microsoft, from the from the perspective of being a--what's the word I'm looking for--trustworthy, honest and sincere participant in open source that we want to be part of the ecosystem, you know, giving back as much as possible, but also, some tactical standpoints. [P17]

%-----------Sharing experiences------------- 
   %&
  %\begin{tabular}[l]{@{}l@{}}Sharing experiences \end{tabular} &
  
    %\begin{tabular}[l]{@{}l@{}}P6, P7, P13, P14\\  \end{tabular} & 
    
  %\begin{tabular}[l]{@{}l@{}}``We can also share, what do we do... it's sharing experiences between different groups and people learning from each other'' [P7]
  %\\
 % ``I think tech plays a huge role in society. And there are not a lot of people that understand the technology and how the technology\\ is really made... so I've essentially tried to move increasingly in the space where I could be an advocate of this and have been trying\\ to find my way through that'' [P13] \end{tabular} \\ \hline
  
  %6:8 ¶ 179 in [P6] Otavio-Zup interview.docx
%eah, we are in the events, talking about it, spreading the word. And letting everybody know that contributing to open source is good. It's good for you and your company, because you're using them. So I think that's it. I can't remember anything else.

%8:74 ¶ 75 in [P7] Josep-Aiven interview.docx
%then there is the other one, which is we can also share these things back so we can also share, what do we do that is different that maybe some other ones will try to do or show another way, show how we can how we did it and what we're doing and what works and what doesn't. And then of course, it's sharing experiences between different groups and people learning from each other.

%8:75 ¶ 75 – 76 in [P7] Josep-Aiven interview.docx
%creating as a group also like some kind of nice ideas, nice ways to go nice, nice roads to point to other companies to say these other ways. These are the different ways to do it. These are the different roads you can take. Pick the one that fits best to you. So that's one of the things that we buy being present in these groups. Of course, it's like the getting some information, obviously, shaping different opinions, different ideas, different ways of doing things.

%18:27 ¶ 94 in [P13] Tobie OSS consultant Interview.docx
%I think tech plays a huge role in society. And there are not a lot of people that understand the technology and how the technology is really made. And or mindful of tax impact on society. And so I've essentially tried to move increasingly in the space where I could be an advocate of this and have been trying to find my way through that. Um, so you know, I guess that's the motivation.

%17:22 ¶ 16 in [P14] Harish RedHat Interview.docx
%In this case, I don't care whether you give me the money or not, I used to win something I still win because the goodwill is established, the positive feedback is there. And I think that is you can put $1 value to that it's very hard to put $1 value to something like goodwill. So that's that's really how the long term benefit for Red Hat is because one of the things you may have come across in your research is people will tend to use the phrase in any technology, whatever that technology is, are you going to be the Red Hat of technology? Are you going to be the Red Hat of that? Are you going to be Red Hat for automotive? Are you going to be the Red Hat for medical services. So when people use the word Red Hat that way, they are using some of the norms and how we do what we do, and have been doing for the last 25..28 years in the open source community and with our customers. So they say that makes sense. Who's going to be doing the same thing in the industries? So so in a way the brand recognition is very powerful, you know that the idea is very powerful.



 



 %4:29 ¶ 71 in [P1] Georg-Bitergia interiew.docx
%Why open source, um, one: it's ideological. The founders just believed that was the right thing to do
 
  %3:7 ¶ 17 in [P2] Daniel-Bitergia interview.docx
%depending depending on where you are in the market or what your purposes are. But if you're getting the case, of Bitergia We decided to go with open source, because we were, well it was a mix of things. Um, mainly because part of the solver that we had was already open source because we were coming from a research group. So it was that say kind of philosophical point of view at the very beginning

 %I mean, we are there to contribute on that community of these projects to make the project better, more sustainable, to bring more different ideas on the table as well. So not just one single companies dominating one open source project, but we try to bring we are another companies that we try to also bring opinions there. To not just depend everything on one single company and to create a diversity of opinions, ideas, companies behind and all this stuff. So also to support and just stay true to the open source philosophy.[P7]
 
 %5:1 ¶ 59 in [P5] Jill-SlimAI interview.docx
%we see the intrinsic value of open source as something that is good for the world. And is something that is should always be.

%5:39 ¶ 59 in [P5] Jill-SlimAI interview.docx
%Kyle's vision and intent for the project in the first place. He, he believes in open source, he's a consumer of open source projects

%5:42 ¶ 241 in [P5] Jill-SlimAI interview.docx
%contributing into an environment that matters to me by giving. Right So this concept of like, encourage contribution is important. And that goes back to the culture of like the company culture


 %5:89 ¶ 59 in [P5] Jill-SlimAI interview.docx
%we as a business consume other open source projects in our product delivery process, right. Like, we have other projects that our developers use that are open source. And so not only does Kyle believe in it, and and has the vision for it continuing to be open source. But we as an organization, and people within our company, and within our community, we see the intrinsic value of open source as something that is good for the world. And is something that is should always be.

%8:15 ¶ 59 in [P7] Josep-Aiven interview.docx
%it's a vision of the four founders.

%8:53 ¶ 59 in [P7] Josep-Aiven interview.docx
%And the four founders basically were avid contributors of open source, and they decided that's, that should be our DNA, we want to create a company that way. And that's how we would go.

%10:54 ¶ 208 in [P9] Phillipe NexBe Interview.docx
%the reason why we started nexB was to create an open source alternatives to large enterprise resource planning software, which was mostly proprietary and still mostly proprietary so things like Oracle Applications, or ACP, you know, this.. you've heard about these things, which are large enterprise systems, to manage accounting, finance, sales, marketing, all this kind of things production

%16:23 ¶ 45 in [P12] Joshua tidelift Interview.docx
%he whole philosophy at the core of the company is really about supporting upstream

%16:42 ¶ 73 in [P12] Joshua tidelift Interview.docx
%y. All of the co founders have deep history in open source. Like that most of them been there since the beginning of open source.

%16:45 ¶ 81 in [P12] Joshua tidelift Interview.docx
%I mean, the, you know, the the co founders come from open source and care about it. You know, there's a, there's a real strong belief, and the value of a commons of reusable components that we share of collaborating across boundaries that might otherwise divide us open sources. Open source is amazing. And they saw a problem and that nobody else was fixing in this manner and, and went for it.

%17:34 ¶ 82 in [P14] Harish RedHat Interview.docx
%So we started off open source company. So we know the benefit is there. We know what it is. And we want to not only us benefiting from it, we want to help our customers benefit from it as well.


\hline 
\end{tabular}}

\end{table*}




%---------Start of FOUNDERS VISION________________
\subsubsection{Founder(s)' Vision}
\label{sec:founders}
%The founder(s) vision is usually from the origin of smaller companies' participation in OSS (P1, P2, P7, P5, P9, P12).
Founder(s)' vision can stem from the founder(s)' ideology (P1, P2, P5, P7, P12) attributed to a \inlinequote{philosophical point of view at the very beginning (P2),} \inlinequote{the company culture (P5)} and to the fact that \inlinequote{all of the co-founders have a deep history in open source (P12).} 
%Founder(s)' vision stems from the founder(s)' ideology (P1, P2, P5), the nature of the problem the company is solving (P9), or a mix of both (P7, P12).

Still, the founder(s)' vision can also be attributed to the nature of the problem that the company is solving and its ties to OSS (P9, P7, P12). In some cases, the reason for founding the company is \inlinequote{to create an open source alternative to large enterprise resource planning software (P9)} which makes the ideology become part of the company's core. 

%the fact that the nature of the problem that the company is solving is tied to open source in its very essence. And the reason for founding the company in the first place is
%\inlinequote{to create an open source alternative to large enterprise resource planning software [P9]}. 

\MyBox{Companies contribute to OSS because of their \textbf{founder(s)' vision} and/or because the problem they are solving is tied to OSS in its very essence.}

%--------------------end of founders vision---------------

\subsubsection{Reputation} 
\label{sec:reputation}

Our findings show that investing in OSS can be compared to public relations. For example, P4 compares the motivation to participate in OSS to that of \inlinequote{tak[ing] bits and pieces of the engine room into the open... by like normal PR [Public Relations] conference talks or they could do it like by open sourcing some of these things.}

%`` to take bits and pieces of the engine room into the open and they can do this by like normal PR conference talks or they could do it like by open sourcing some of these things''. 

Our analysis revealed that companies focus on building \textit{reputation} in different ways, which we categorized as follows: \textsc{Visibility}, \textsc{Building verifiable trust}, \textsc{Networking}, \textsc{Attracting talent}, and \textsc{Fostering Adoption}. 

%%%%%%%%%%%%%%%%%%%%%%%%%%%%%%%%%%%%%%%%%%%%%%%%%%%%%%%%%%%%%%%
\boldification{no matter the size of the company, visibility is a driver to being in OSS.}
\textsc{Visibility.}
Our results show that participating in OSS brings visibility to companies of all sizes (P1, P2, P4-P9, P11, P13-P18). For instance, P2 explains \inlinequote{we wanted to be there [OSS project] in our case because this brings a lot of visibility.} Companies join OSS to cement their name to a domain, expand their brand, or receive credit as the creator of a technology that is an industry standard.

\boldification{This visibility helps companies build an understanding of what they do inside, which helps attach the company brand to the domain of expertise that it is known for.}

For example, visibility helps companies signal an understanding of  \inlinequote{what [is] actually going on inside (P4)} by showcasing their technical achievements (P1, P2, P4, P7), which helps cement the name of the company in their domain of expertise.

\boldification{While this is the case for some companies, the visibility capabilities of OSS can also help a companies transition to a different domain of expertise}

While this is the case for a number of companies, it does not end here. Companies can also leverage OSS to expand their brand beyond that for which they are known for (P4, P11). In particular, OSS can help companies expand their domain of expertise as explained by P11: \inlinequote{so people tend to think of [company name] as like the people who did [specific product]. And we've really moved beyond that.}

\boldification{Now visibility doesn't have to mean the same thing to all companies, it can mean becoming more international for growing companies and making an internal technology the \textit{de facto} technology and being seen as the founder}

{Visibility can also hold different meanings to different-sized companies. For small companies, visibility can mean \inlinequote{becom[ing] more international (P6).} Whereas, for large technology companies, standardizing a piece of technology is the ultimate visibility goal (P3, P4, P18) because \inlinequote{if you actually came up with this standard, then you also get a lot of exposure (P4).} A good example of this is how \inlinequote{[for example] open-sourcing React has provided more value for Facebook than leaving it to themselves ever could have. Because (a) they get contributions now on that library; and (b) they get the sort of street credit of the founders of React (P3--who does not work at Facebook).}}

 
%_________________________End visibility__________________

\textsc{Building verifiable trust.}
In our context, building verifiable trust was described as gaining users' trust and/ or gaining the communities' trust. OSS by virtue of its open character, enables companies to openly contribute to a project. Therefore, these companies--especially startups--see OSS as a place where they can showcase their expertise to their users to gain trust (P1, P2, P5, P7, P15, P18). In fact, openly contributing to a project allows users to see the companies’ contributions, their quality, and the rate at which they maintain the project. This differs from the proprietary software world, where users have to trust the company as to whether or not a product is being maintained. P15 explains these differences: \inlinequote{If we have a proprietary component..., you have to trust the vendor to know whether or not that's being maintained... Whereas if it's an open source project,  I can watch the contribution history and I can know whether or not it's being maintained (P15).} 

Gaining users' trust is particularly relevant when a company is at the forefront of a new technology or concept as explained by P2: \inlinequote{And they [users] say, well, I need someone, I need help with [name of concept]... So it's not that we know about [name of concept] it's that we are defining [name of concept] and for this, you can see that we have this pattern here about the maturity model.} This suggests that companies that are working on, or defining a new concept in the open, are more likely to be trusted and potentially hired by users, as users can directly see their contributions and decide on its quality.


While giving your users \inlinequote{open source value (P5)} builds verifiable trust, making it long-lasting requires going the extra mile--listening to and embracing users' feedback: \inlinequote{and what happens when you say, I'm giving you this, and then you deliver this, your users and your customers trust you. And they are more likely to stay with you (P5).}

It is not only startups who seek to build trust, we also found that for large technology companies (P14, P15, P17, P18) \inlinequote{the biggest benefit from contributing is building trust (P18).} However, this trust is specifically directed towards signaling a sense of \inlinequote{accountability to communities (P17)} and showing that \inlinequote{you are willing to sidestep what you think is best and make sure that decision makings include the community (P17).} That built trust is what then enables companies \inlinequote{to be at the table (P18)} and participate in shaping the direction of the projects that they depend on. 

In summary, different types of companies have different goals in creating trust; startups (mainly) contribute to OSS to gain users' trust and a user community, while larger technology companies contribute to OSS to gain the communities' trust.


\textsc{Networking.} Participants described networking as socializing with the OSS community, other companies, and/or being in the same space as big-name companies. This allows companies to open the lines of communication for collaborations and partnerships. Our analysis shows that networking can look different depending on the company's goal. Most small companies found networking to be a driver for OSS participation. 
Some of these companies saw networking as an opportunity to socialize with peer companies and organizations and get to know people in the same space. For instance, P5 depicted networking as being in \inlinequote{that same ecosystem as these other like-minded organizations and practitioners (P5)} and P7 described networking as \inlinequote{socializing on that level of getting to know people, talk to people (P7).}

Other companies saw networking as a way to be in the same space as big-name companies (P1, P2, P6). For example, P6 describes networking for their company as a way to \inlinequote{bring more people to pay attention...and engage our brand to be related with huge companies (P6).} Other companies emphasized networking as a way to ease collaborations. For example, P8 mentioned how networking made
\inlinequote{collaboration and partnerships become six orders of magnitude easier (P8).}

%as a motivation to contribute also looks different depending on the company's goal
%, with growing companies participating in OSS to be in the same space as big-name companies.


%which leads us to the next section, stay tuned ...

%\MyBox{\textbf{Networking:} Growing companies participate in OSS to be in the same space as big name companies, opening the lines of communications for collaboration and partnerships.}

%_________________ATTRACTING TALENT___________
\textsc{Attracting talent.}
Our results show that companies participate in OSS to be seen as a ``cool company'' to work at, hire talents that are familiar with their tech stack, and bypass the interview and onboarding process. Participant P3 shared that companies that contribute to OSS \inlinequote{get viewed as, like a cooler place to work (P3).} P4 explains that \inlinequote{open sourcing is one of the ways that can make it [company] much more interesting for outside candidates (P4)}
and signals to developers that, by joining such a company, they too can work on ``cool technology'' that has a lot of traction in the world (P3, P4, P8, P10, P13). 

Our findings suggest that such reputation is highly beneficial for companies, which can then hire some great people since, as P10 points out, \inlinequote{people who code in public are better coders (P10).} Not only that, but companies also \inlinequote{get an entire class of engineers, who are now familiar with the tech stack that [company] uses (P4)} thus bypassing the interview and training process (P3, P4, P6, P10). 

This is particularly relevant for large technology companies that have open-sourced a technology that has now become the industry standard. For instance, P4 (who does not work at Google) mentioned that \inlinequote{any engineer that Google hires now will know Kubernetes.} 

%So this mindset also, cause many companies that didn't go open source to lose talent and don't retain talent. So this works also as a way to retain talent, like, Hey, we are not just a boring company in the financial systems, I don't know, for instance, like, we are a tech company, and we are promoting open source and we really want you to create things and linear space to do so. Right? And also to like your Yeah, like your personal brand gets in the pilot for for your future career.[P19]

%Small companies also reap the benefits of open source contributions \inlinequote{to hire some of the very best in the field [P9]} even though they \inlinequote{don't have the ability to pay the same [P9]} as larger technology companies do. 


%\MyBox{\textbf{Attracting talent:} companies join OSS to be seen as a ``cool company'' to work, to hire talents familiar with their tech stack, and bypass the interview and onboarding process.}


\textsc{Fostering adoption} in our context refers to a company’s technology being widely adopted. Fostering adoption can be an integral part of building a company's reputation, so much so that the company \inlinequote{would rather focus on broader adoption than just you could say having our name on it (P4).} % its virtue of being open to anyone, OSS helps foster adoption. 

For smaller companies, and as described by P2, \inlinequote{the best way to get your technology adopted is by open-sourcing this (P2).} Our results suggest that open-sourcing technology not only provides a readily available alternative to proprietary software but levels the playing field for startups to compete with larger companies.

For larger companies, fostering adoption can look like, as described by P11, \inlinequote{not locking them [enterprise customers] into a proprietary technology} and giving them the option to deploy the same solution using a different vendor. Our findings show, that while
the open character of OSS helps foster adoption, some companies do not stop at the mere act of open-sourcing a product. Such companies \inlinequote{strategically give that [project] to a foundation to grow in public, so that more people use a technology (P10)} (P4, P11, P18). Our findings highlight that allowing a project to grow under neutral governing bodies gives additional impetus to wider adoption.

\MyBox{Contributing to OSS helps cement \textbf{reputation}. It gives visibility to smaller companies, allowing them to play in the same space as big-name companies. Larger companies benefit by developing  deeper trust with the community signaling good citizenship, which in turn helps in recruiting.}

%%%%%%%%%%%%%%%%%%%%%%%%%%%%%%%%%%%%%%%%%%%%%%%%%%%%%%%%%%%%%%%

%______________________START OF BUSINESS ADVANTAGE_______



%\textsc{Engineering need.} The business advantage journey starts with the engineering need: \inlinequote{thinking they don't use it, to realizing that they do... (P12).}

%OSS is home of \inlinequote{some of the greatest technologies, pieces of technology, around the world (P2).} With such realization companies start their OSS journey by \inlinequote{consum[ing] open source, that's the, the most obvious one, and like 99\% of the companies in the world do that (P7).} By consuming quality and readily available solutions, companies are able to \inlinequote{spin up this [product name] product line in a relatively short period of time (P11).} This--in addition to OSS's independence from licensing and certification needs (P7, P13, P15)--makes OSS solutions more attractive where companies \inlinequote{look to use open source whenever we can, for our own stuff, as well as open standards (P8).}

%\MyBox{\textbf{Engineering need}: companies consume OSS because it hosts high-quality software, it is independent of licensing/ certification needs and helps build quality software rapidly.} 

%---------------business depends on oss-------------------
\subsubsection{Business Advantage}
\label{sec:business}
Over the years, companies' mindset has shifted in a way that now sees OSS as an ecosystem of business opportunities rather than a risky endeavor that commoditizes their products \cite{robles2019twenty}: % This is clearly reported by one of our participants: 

\inlinequote{There's a journey that companies seem to go on from like...thinking they don't use it, to realizing that they do...to a point where companies start realizing they need to give back...Ultimately, they know at this point, that a healthy ecosystem means good business for them, too (P12).}

The Business Advantage motivation is divided into four different subcategories. While companies are driven by a direct business need to contribute to OSS, other motivations (Closer channels, Coopetition, and Innovation) also play a role when it comes to companies' involvement in OSS. We detail these motivations in the next subsections.

\textsc{Business dependency on OSS,} refers to cases where
throughout the companies' journey in OSS, their business becomes dependent on OSS \inlinequote{...to a point where companies start realizing they need to give back because they see that it actually helps them strategically (P12).}

Because the business depends on OSS projects, companies are motivated to contribute to keeping their dependencies healthy (P2-P5, P7, P9-P12, P14-P17). And just like an engine is critical to a car manufacturer: \inlinequote{if something happens to that engine...that works for Ford, right? Or, even worse, it makes it so that Ford can't use the engine the way they were using it before (P3).} Our results show that business dependency on OSS relates to ensuring that the OSS projects a company depends on remain healthy and that the leadership of these projects is not monopolized.

\boldification{Keep dependency healthy}
In fact, as described by P9, companies \inlinequote{want these projects [dependencies] to be healthy, alive and well maintained, and sustaining (P9).} To help with sustainability, companies contribute financially (P4) as well as upstream fixes (P17), which in return prevents carrying any technical debt internally (P11).
 
\boldification{software monopoly} 
The technical soundness of a project dependency is not the only aspect of its health. In fact, P7 explains that \inlinequote{part of the healthiness of the project is also making sure that there are different ideas, opinions, and points of view. It's really easy to fall to a kind of monopolistic way on the open source projects (P7).} Our findings suggest that participating in governance and leadership (P3, P7, P15, P11) helps ensure a shared roadmap. Our findings also highlight that, a shared leadership flows both ways. It cannot just thrive on companies' readiness to get involved in governance roles, it also relies on the project having an open leadership and governance that welcomes participation. When the project in question is led by a company and this company realizes that
 \inlinequote{we cannot be the only ones backing this thing up (P7),} they start seeking \inlinequote{a backing up of different companies (P7)} which in turn becomes a stepping stone for coopetition, which we detail in the next section.

%\MyBox{\textbf{Business dependency on OSS:} companies want to ensure that the OSS projects they depend on are healthy, and the leadership of these projects is not monopolized.} 


%___-----------------Coopetition_____________________
\textsc{Coopetition.}``Coopetition is the act of cooperation between competing companies''\cite{coopetitionDef}. Our findings suggest that in OSS companies can work together on the hard technical problem they all face and compete on other features around the problem. %\inlinequote{all these little things around the [problem] that make a [product] worth buying (P3).}

Our results suggest that coopetition yields more productivity where (1) workload is reduced and (2) products are of higher quality. The workload is reduced (P3, P8, P13-P15) as collaborating helps \inlinequote{do the hard work only once (P3)} and provides a form of resource sharing where \inlinequote{20 engineers will go further working on a project with 80 other engineers from four different companies (P3).}

We also found that the benefits of coopetition are not solely related to productivity. Such practice enables a bigger market share (P3, P15), which in return provides a much larger user base, one that is now empowered to migrate to different service providers that use the same core technology. 

Still, according to P3 and P11, the products end up having higher quality 
because \inlinequote{pulling in people who work from other companies who will use things differently...they'll have different skills...they'll have different ways of thinking (P11).}


%\MyBox{\textbf{Coopetition:} companies participate in open source to collaborate with other companies on the harder problems, share resources, build a better product, and have a bigger market share.} 


\textsc{Closer channels}
in our context, relates to more direct communications, such as direct user feedback and, engineer-to-engineer communication that promotes timely fixes.
OSS is, in fact, a no-middleman land that provides closer channels and engineer-to-engineer communication (P5, P7, P13-P15, P17). The closer channels can take the form of raw users' feedback or customers' collaborations (P5, P14) that is then, as described by P5, \inlinequote{ingest[ed] back into our product delivery process.} 

Closer channels also mean bypassing the company-to-company communication protocols %where \inlinequote{you cannot reach out to the developer directly (P14),} 
and making engineer-to-engineer communication the new norm (P13, P14). As described by P13, \inlinequote{it's fairly easy to waste a week on an incredibly dumb bug, that if you actually know everyone in the community... you're going to solve in like, 15 minutes (P13).}

%\MyBox{\textbf{Closer channels:} companies participate in OSS because it brings the channels of communication closer, such as readily user feedof mentori engineer-to-engineer communication that promotes timely fixes.} 


\textsc{Innovation.} OSS is, by virtue of its character, a space for innovation that provides a business advantage and drives companies' participation (P11, P13, P16, P17, P19). Our findings suggest that innovation takes the form of more perspectives, more user stories, and diverse ideas that shape the company's products (P11, P17, P19). This innovation boost then makes it more likely \inlinequote{that it [product] will solve problems for more people and not just those 12 extremely bright people in the room (P17).}

\boldification{Getting familiar with projects interesting in the future}
For \textit{avant-garde} companies, innovation can also take the form of getting familiar with projects that can be strategic in the future. For instance, P11 explains: \inlinequote{one of the things that we try to do is contribute to projects that we think are going to be strategic for [company name] in the future, but that aren't necessarily strategic for us right now (P11).}


%\MyBox{\textbf{Innovation:} companies participate in OSS because it fosters collaboration, yields new perspectives, and gives companies access to potential strategic projects in the future.} 

\MyBox{Companies start to use OSS as a \textbf{business advantage}, but then their journey evolves to contributing to maintain the tech they are dependent on, and they stay since participating in OSS allows them to cooperate with their competition and innovate quicker.} 


%-----------Start of RECIPROCITY____________________

\subsubsection{Reciprocity} Companies sometimes want to give back to the OSS ecosystem by contributing to the projects they use or other projects to share knowledge and experiences for the greater good of OSS (P2-P5, P7, P9-P12, P14-P17).
%
%\textcolor{red}{
%Reciprocity can take the form of simply giving back to the OSS ecosystem that has given so much}. 
Our findings suggest that reciprocity stems from the realization that companies have extensively used open source and need to give back. For instance, P7 shared \inlinequote{we took lots of things from open source.} P9 explains that if \inlinequote{there's no reciprocity in some ways, even small, then you're missing the point entirely.}

In contrast with the contribution driven by the business depending on OSS (see section \ref{sec:business}), reciprocal contributions are ones that do not necessarily have a direct impact on the company but are rather, as P9 puts it, \inlinequote{the right thing to do.}

%\inlinequote{benefit for us there. I mean, there's no direct benefits in having the [project name] cleaned up, it was more we thought it was the right thing to do [P9]}.

Reciprocity can take the form of mentoring (P1, P2, P9, P12, P14, P17, P20),  \inlinequote{sharing experiences between different groups (P7)} (P6, P13, P14) and advocating OSS by \inlinequote{spreading the word. And letting everybody know that contributing to open source is good. It's good for you and your company (P6).}

\MyBox{Companies want to \textbf{reciprocate} by contributing back to the projects they use and to other projects for the greater good of OSS.}



%so there is no direct benefit for us there. I mean, there's no direct benefits in having the Linux kernel cleaned up, it was more we thought it was the right thing to do. It was pure pro bono investments, syncing the work we do with these tools. In the end, we believe that if more people use open source, more people will notice the service and products we provide. And, and and that will be beneficial to everyone in the long run. [P9]


The four motivation themes (RQ1) are what drive company contributions. To further understand the companies' engagement, in the following section, we link these motivations to the multi-faceted ways companies contribute to OSS, highlighting the entities that benefit from such involvement (see Figure~\ref{fig:contribute}).

\subsection{RQ2: How do companies contribute to OSS?}

%\subsubsection{Contribute to open source}
When it comes to contributing to OSS, there is a number of ways to participate. We identified seven categories of contributions across companies: open sourcing \textsc{Tools \& Products}, \textsc{Developers' Time}, \textsc{Community 
Relations' Time},  \textsc{Advocacy \& Promotion Time}, \textsc{Time Addressing D\&I}, \textsc{Mentoring Time} and \textsc{Financial Support} (see Figure \ref{fig:contribute}).

\begin{figure}[!ht]
\centering
     \includegraphics[width=0.52\textwidth]{figures/contribute.png}
    \caption{Company contributions and their beneficiaries}
    \label{fig:contribute}
\end{figure}

%-------------SOFTWARE/tools and products----------------
\subsubsection{Tools \& Products} %Three of the four motivations that we identified as drivers for companies' participation in OSS (Founder(s)' vision, Business advantage, Reputation--Table \ref{tab:motivationTable}) point to open sourcing internal tools and products as a way to contribute to OSS (see Figure \ref{fig:motivation}).

In our interviews, we could observe that companies contribute by open sourcing \inlinequote{platforms and frameworks (P11),} \inlinequote{tools (P1),} \inlinequote{internal code (P4),} \inlinequote{[product name] community editions (P11),} \inlinequote{SDKs (P12)} and \inlinequote{those things in between, those connectors (P7).} Contributing internal products to OSS is a win-win for both the companies and the OSS ecosystem. 

Our results suggest that the internal tool, when open-sourced, benefits the whole OSS ecosystem (see Figure \ref{fig:contribute}) by enriching it with active projects that are backed up by one or more companies. For the company, the open-sourced tools provide a business advantage. For example, P7 describes such business advantage as \inlinequote{mak[ing] things easier for everyone using [company product]} and  \inlinequote{connect[ing] services that [company] does.}

Companies also reap the reputation benefit of open sourcing a novel solution, \inlinequote{a thing that does not exist} as P7 explains. 
The reputation benefit is magnified when such open-sourced tool is successful and widely adopted. For example, P4 explains \inlinequote{So [product name] is the thing that has the most success, and it's been adopted by a lot of companies... it's very impressive.}

Our results also suggest that when the question becomes \inlinequote{how much adoption they can get on what they give away (P10),} companies strategically donate their projects to an OSS foundation, signaling neutral governance. This, in return, helps sustain OSS foundations and the OSS ecosystem as a whole (see Figure \ref{fig:contribute}). 

Still, when the donated project becomes an industry standard, the company starts to be seen as the leader in that space which increases the company's visibility and reputation. For instance, P4 explains \inlinequote{because if it [OSS project] becomes an industry standard, then you gain all of the benefits from being part of the standards. And if you actually came up with this standard, then you also get a lot of exposure.}



%-----------------end of SOFTWARE------------------------------

%------------TIME and LEADERSHIP/GOVERNANCE---------------------


%\textsc{Time:} 90\% of companies depend on open source code \cite{openSourceUse} and the companies represented by our interviewees were no exception. 


When \inlinequote{the basis of everything is open source dependencies (P4)} contributing time is companies' first response. This takes the form of developers' time (P4, P6, P7-P20), community relations' time (P1, P5-P10, P11, P14, P15, P20), [time] on advocacy \& promotion (P1, P5-P9, P11-P16), [time] addressing D\&I (P5, P6, P13, P17, P19, P20) and [time] on mentoring (P1, P2, P9, P12, P14, P17, P20).

\subsubsection{Developers' Time}
\label{sec:developertime}
Our results show that contributing time often starts in an informal manner, led by developers. For instance, P8 explains:  \inlinequote {In the process of using, we tend to contribute to things we do use.} More specifically, developers' time can take the form of \textit{ad hoc} contributions such as fixing bugs (P9, P10, P12, P13) or adding a feature (P4, P8, P10, P11), where engineers \inlinequote{contribute where they see a need...[as a] scratching your own itch kind of thing (P4)}.% as a \inlinequote{(P4).}

When there are \inlinequote{300 things that depend on this specific dependency (P4),} contributing time in an \textit{ad hoc} way is no longer a viable option. Companies step to the next level, investing developers' time in a more systematic way. This can take the form of assigning employees to work on an OSS dependency. For example, P11 shares how in their company there are \inlinequote{dozens of people who contribute to [dependency name] on a pretty regular basis.} Similarly, P10 explains \inlinequote{I had about between 35 and 50, people whose full-time job  was to work on open source projects that the company used.}
%and engineers whose \inlinequote{full-time job was to work on open source projects (P10).}%that the company used [P10]}.

%With the time investment that companies put into their dependencies

With the time invested in OSS dependencies, companies aspire their \inlinequote{contributors to eventually try to move into maintainer positions, approver positions, leadership (P11)} (P3, P8, P13, P15) (see Figure \ref{fig:contribute}). This is particularly relevant for critical dependencies where \inlinequote{getting elected to leadership positions in these communities is really important (P3)} and allows companies to \inlinequote{help shepherd these projects forward (P18).}
Our results show that by investing developers' time in OSS dependencies, companies are sustaining the very projects that they depend on. This in turn (1) provides a business advantage and increases the company's reputation as one that \inlinequote{believe[s] there's a contribution wheel (P8)} and (2) benefits the OSS dependencies by maintaining a consistent contribution flow (see Figure \ref{fig:contribute}). 


\subsubsection{Community Relations' Time}

Participants revealed that companies realize that it is necessary to build a relationship with the community to work in line with OSS, as mentioned by P9: \inlinequote{there are no special benefits to things in the open...if you're not able to build in the community.} This realization is at the core of companies investing time in community relations in addition to investing developers' time.

In this context, our results show that companies connect with OSS communities by holding open discussions (P5-P8, P15) which benefit both parties. P5 reported that, by doing this, \inlinequote{[the communities] learn more about what we're doing} and P15 complements it by saying that \inlinequote{[the company learns] what is it that people want.} These discussions include \inlinequote{talking about problems (P5),} \inlinequote{answer[ing] questions (P6),} and %when conflicts arise,
being open to compromise by \inlinequote{explaining the benefits, try to reducing the risks or problems of that solution, and trying to reach a better one (P7).} 

Some companies (P1, P11, P20) go the extra mile to help
foster community dynamics by organizing \inlinequote{contributor summits (P11)} and \inlinequote{events, where people [contributors] can come together (P20).} According to P20, this helps contributors to \inlinequote{meet each other.}% In addition, P11 mentioned that these events help the community attend \inlinequote{various community facing functions (P11).} 

Our findings suggest that contributing time to OSS is not only about contributing developers' time, but also spending time to nurture the community. P20 summarizes this as: \inlinequote{not just people spending time on code, but also people spending time on people}  which further magnifies the benefit to the project community (see Figure \ref{fig:contribute}).


%---------------ADVOCACY AND PROMOTION-----------------------
%Motivation: Reputation + recipro
\subsubsection{Advocacy \& Promotion Time} Our participants indicated that several OSS projects are tied to the software itself, making a lot of companies' contributions code related. But company participation does not stop at that. As our participants mentioned, they also \inlinequote{contribute to the overall ecosystem by doing advocacy work (P13)} and promote the prevalence of OSS (P5-P8, P11-P16).

%[general advocacy]
According to our respondents, companies advocate for OSS by encouraging participation internally and externally. Internally, companies encourage their employees to contribute to a project of their choice in their free time and compensate such contribution. P5 explains: \inlinequote{we encouraged our staff to contribute to an open source project that mattered to them, and then submit it for reimbursement.}
Externally, not only do companies \inlinequote{bring to the community, the awareness of how much this [open source] is important (P6)} but they also promote OSS to other companies. This takes the form of leading by example and as P6 describes:  \inlinequote{show to other companies in the ecosystem, that even if you're not a product company, you still may be able to invest money, time, people to contribute.}

%and explaining \inlinequote{how can they [companies] be part of the open source community (P14).} 

According to our participants, companies also encourage people to contribute by creating and sharing the resources to help them do so. During the interviews, they mentioned that the creation of resources takes the form of \inlinequote{blog posts (P6)} (P5, P7, P8, P16), \inlinequote{podcasts (P7)} (P5), \inlinequote{documentation (P15)} (P6-P9, P11), creating content for \inlinequote{open source month (P16),} creating and sharing survey results (P12), and participating in OSS working groups to codify best practices (P1, P18). 

%While capturing the attention of potential new contributors is about \inlinequote{letting everybody know that contributing to open source is good (P6),} \inlinequote{try[ing] to bring people to contribute (P6)} is all about creating and sharing the resources to help them do so. Creating resources takes the form of \inlinequote{blog posts (P6)} (P5, P7, P8, P16), \inlinequote{podcasts (P7)} (P5), \inlinequote{documentation (P15)} (P6-P9, P11), creating content for \inlinequote{open source month (P16),} creating and sharing survey results (P12), and participating in OSS working groups to \inlinequote{capture and codify best practices (P18)} (P1). 

%The natural next step to creating content is \inlinequote{sharing learning and pain points [P7]}. This could be as general as sharing \inlinequote{best practice that we are implementing that many others are not [P6]} or as specific as describing \inlinequote{the different type of OSPO [Open Source Project Office] because we are not the most common OSPO [P7]}.

%While sharing learning points enriches the overall OSS ecosystem and the different OSS foundations and events where they take place, some sharing practices directly benefit the OSS dependencies (see Figure \ref{fig:benefit}). 
%This happens when companies, for instance, use their \inlinequote{popular blogs and social media channels [P11]} to share dependency-related content such as \inlinequote{general [dependency] concepts... how you use it... pitfalls we found [P8]} thus \inlinequote{giv[ing] visibility to some of these projects [P11]} (see Figure \ref{fig:benefit}). 

%[creating event, hackathons and conferences]
We found that, while companies advocate OSS through attending existing conferences and events (P5, P7-P9, P11-P14), some take advocacy a step further by launching their own conferences, events, or hackathons (P1, P6, P7, P9, P11, P12, P20). These events help attract new OSS players by, for example, launching hackathons to help initiate newcomers to \inlinequote{solving some social issues (P6)} or promoting an OSS technology by \inlinequote{starting a conference... talking about open sourcing cloud (P7).}

%Companies also help nurture their existent contributor base by organizing \inlinequote{contributor summits [P11]} to help with contributors' experience and \inlinequote{various community facing functions [P11]} which further magnifies the benefit to the project community (see Figure \ref{fig:benefit}). 


%Developer/leadership time ---Management 
%Relation time (dev + community)
%mentoring time
%advocacy/ promotion time
%Addressing D\&I issues

%-----------------END OF ADVOCACY AND PROMOTION-----------------

%---------------------ADDRESSING D\&I------------------------
\subsubsection{Time Addressing D\&I} 
The participants (P5, P6, P13, P17, P19, P20) mentioned that their companies invest time and effort in addressing OSS Diversity and Inclusion (D\&I) issues. 

We could observe that the companies the interviewees work for contribute to the overall welfare of the OSS ecosystem (see Figure \ref{fig:contribute}) by investing effort in  \inlinequote{understanding how to be inclusive (P17),} understanding \inlinequote{diversity, equity inclusion initiatives (P20)} and understanding the ins and outs of the \inlinequote{code of conduct (P17).}

Once companies are well aware of the meaning and stakes of D\&I, they start thinking about different perspectives: \inlinequote{how to have these sorts of conversations (P20),} \inlinequote{how to support someone who is in, you know, feeling threatened, or unwelcome (P17)} and \inlinequote{how to have recognition and allyship (P20).}

%Who's missing? Why are they missing? Who's not talking? Why aren't they talking?[P17]

Only after building an understanding, do the companies start addressing D\&I issues. For some companies, this takes the form of \inlinequote{investing in more programs and initiatives that we hope will help just to improve that aspect of things [D\&I] (P20).} For other companies, it is more specific, focusing on geolocation diversity in OSS by contributing to OSS in their country of origin (P6). P5 shared that their company contributes to D\&I in OSS by partnering \inlinequote{with an organization that helps people learn how to develop. They come from an underserved population} and 
hiring \inlinequote{developers in countries that are less developed.} %and %, maybe they've never gone to college, or maybe they're doing a career transition [P5]}. 

%Interventions like these can help enrich the OSS ecosystem by bringing new faces in, and with that comes \inlinequote{more ideas from different cultures' perspectives and that also enrich[es] this innovation [P19]} thus closing the circle on the \inlinequote{virtuous cycle [P5]}.

%----------------END OF ADDRESSING D\&I------------------------



%--------------------MENTORING---------------------------
\subsubsection{Mentoring Time} Companies contribute to OSS by participating in mentoring initiatives, whether it be in a formal or informal capacity. By doing so, companies help attract and retain different OSS players, from individuals to other companies. This, in return, benefits the whole OSS ecosystem and ensures the continuity of different mentoring programs (see Figure \ref{fig:contribute}).

When it comes to informal mentoring, our findings show that companies give back to OSS by \inlinequote{mentoring people and always inviting others to contribute (P1)} (P14, P17). This is not only limited to individual contributors (e.g., newcomers) but also extends to helping companies that are not necessarily in the technical industry (e.g., ``manufacturing organizations,'' ``hospitals'' (P14)). According to P14, this is important since these companies \inlinequote{are struggling trying to figure out how to do this [open source].}

Whereas in formal mentoring, some companies run internship programs both internally and externally such as Google Summer of Code or Google Season of Docs (a mentorship program that nurtures technical writers). Other companies mentor aspiring contributors through these very same mentorship programs (P2, P9, P12). For instance, P9 mentioned
 \inlinequote{working with specific programs, like Google Summer of Code programs.} 


At times, these programs can attract contributors and potential hires. For instance, P9's experience with mentorship programs resulted in \inlinequote{candidacies of folks that are willing to work with us.} Or, in the case of P2, \inlinequote{some contributors coming from Google Summer of Code (P2).} 

%and providing \inlinequote{overall positive force for the projects (P20).}
%-------------------END that donG---------------------

%---------------FINANCIAL SUPPORT-----------------------
\subsubsection{Financial Support} According to our interviewees, the companies participate in OSS by supporting one or multiple entities in the open source ecosystem. This can take the form of money set aside to support maintainers, foundations, patronage programs, and sponsoring events.

%Not only do companies \inlinequote{set aside time (P18)} but also \inlinequote{money to work with open source (P18)} (P16, P20).
Our participants mentioned that companies participate in OSS financially \inlinequote{maintaining the health of overall open source ecosystems (P15)} by: contributing money to dependencies (P4, P8, P5), hosting patronage programs (P4, P5), supporting maintainers (P5, P12), sponsoring events (P5, P11, P12, P15), foundations (P11, P12, P15, P16) and mentorship programs (P12) (see Figure \ref{fig:contribute}). 

Similar to investing time on projects that companies depend on, companies are also \inlinequote{more likely to patron that project if we're using it (P5).} This takes the form of \inlinequote{set[ting] aside a certain amount (P5)} (P8) to help these projects \inlinequote{pay the bills, and just like dedicate time for coding (P4).} A more specific way that companies invest money in dependencies is by directly supporting the maintainers (P5, P12, P15) through, for instance, \inlinequote{hiring open source maintainers full time (P12).} 

While providing direct financial support to OSS dependencies helps ensure the health and dependability of specific projects, companies also believe in supporting OSS foundations. For instance, P12 explained that \inlinequote{if we don't also think about supporting the foundations that provide a home for those projects, then we're not really going to solve the problem in its entirety.} Our results show that companies \inlinequote{give a lot of money to foundations (P11)} (P12, P15) to \inlinequote{pay for infrastructure (P15)} and \inlinequote{to pay people to do all the things that they do...to support the open source projects (P11).} This opens the door for companies to be involved in \inlinequote{safety and security (P8)} conversations that are usually \inlinequote{exclusive to sponsored members (P8).} 
%%%%%%
\subsection{Tying it together: A summary}
Having described the motivations that lead companies to contribute to OSS and the different ways companies contribute to OSS, we now tie everything together, summarize our findings, and discuss lessons learned from our interviewees on OSS sportsmanship and being a good OSS citizen.

With 97\% of companies using and relying on OSS \cite{ossCompanyPercent}, profit aside, companies' well-being is now tied to OSS' well-being where the investment in OSS is not just linked to a short-term profit, but rather an investment in the future. Companies invest in reciprocal contributions such as mentoring and advocacy and promotion, as well as addressing D\&I issues, and go as far as archiving OSS code in the Arctic for future generations~\cite{githubArcticOSS}.

Companies are realizing that OSS is also about innovation, being part of the collective brain power, being a humble participant, and doing good for the community.
Company participation in the OSS ecosystem is multi-faceted, beneficial to different entities (see Figure \ref{fig:contribute}), and intertwined with different motivations (see Figure \ref{fig:motivation}). Figure \ref{fig:motivation} depicts how the different types of company contributions (right-hand side) flow from the different motivations (left-hand side).

Increasing reputation by contributing to OSS was one of the most prevalent motivations. Companies increase their reputation not only by open-sourcing tools and products and investing their developers' time but also gain visibility by investing in non-code related activities (e.g., advocacy \& promotion, mentoring).

While companies also contribute time (developers' and community relations), financial support, and software to maintain their business advantage, being a good OSS citizen is a pertinent driver for a variety of contributions. In fact, companies see the benefit in being a humble OSS participant where they do not just take from, but also give back to the community (see reciprocity, Figure \ref{fig:motivation}). One key aspect was taking the time to address issues of Diversity and Inclusion (D\&I). D\&I is an important topic in open source yet a challenging one to address. When companies, that have the tools, resources, and experience, take the initiative to champion D\&I in OSS, it is a significant step forward. 

\begin{figure}[hbtp]

    \includegraphics[width=.49\textwidth]{figures/FromMotivationToContribution.png}
    %\vspace{-3em}
    \caption{Relationship between motivations and the type of contributions}
    \label{fig:motivation}
    %\vspace{-0.8em}
\end{figure}

%While companies recognize the impact of OSS on their business and profit, 
Being a good OSS citizen (see Table \ref{table:sportsmanshipTable}) was seen as key to healthy OSS participation. One way to do so is by \inlinequote{contribut[ing] in the way that makes sense to the project, which may or may not always be, again, what the company would want (P20).} This entails placing the community's needs first and helping out with maintenance tasks.
%, and to parts of the project that they are benefiting from. 

In order to participate in an OSS ecosystem, an open source program in a company needs to be \inlinequote{able to speak multiple languages (P10)}: risk management, engineering benefits, legal \& compliance, and sponsorship, to be able to advocate OSS to \inlinequote{different levels of leadership. And showing how open source helps the business (P18).} 

%Another, is by \inlinequote{tak[ing] that technical depth task that clean out things, that restructure things (P7)} and lessening maintainers burnout.
Finally, it is good to have an abandonware strategy. To create a sustainable project, companies need to open-source projects they use and care about internally. It is also important to have a contingency plan if the project does go out of sync internally by \inlinequote{transfer[ing] it to people who care about them [abandoned project] (P4).} Ensuring that the project does not become a company monopoly and has affiliation diversity is a good step in avoiding abandonment crisis.

\begin{table*}[bht]
%\footnotesize
%\scriptsize
\caption{OSS Sportsmanship: lessons learned}
\vspace{-1.5mm}
\begin{tabular}{p{0.2\textwidth}p{0.1\textwidth}p{0.6\textwidth}}
\toprule
{\textbf{Takeaway}} & {\textbf{Participants}} & {\textbf{Exemplary Quotes}} \\ \hline

%\textbf{Play by the community rules:} The community needs take precedence over the company's needs 
%\newline
\textbf{Don't bulldoze your way into getting what you want:} The community needs take precedence over the company's needs 
&

 P5-P8, P11, P14-P20
&
``You really have to think of the community first...we can't just bulldoze our way into getting what we want. We have to think about how what we want benefits the community'' (P11)
\newline
``We try to pick the things that would be having a bigger impact, bigger benefit for the whole of the community'' (P7) %... The main chunk of the work we do is what can we do for the community that will help them'' 
%\newline
%[This is mentioned in the companies code of business conduct] 
%``So if (a) is right for the project, because that's what you need to do, even though it's negative to [company name], yeah, it doesn't matter. It's important that the project succeeds'' [P14]
\\


%And that could be bias in the project, also about accepting contributions from certain companies or, or to protect, you know, their own direction, or they may see companies as pushing the company agenda and not thinking about the project in general. [P18]

%[boarder security questions]
%Motivation? Well, I mean, excellence in engineering, I think, you know high quality software, I think we want, again, to be a humble participant in the ecosystem of which we benefit. And excellence in engineering, especially around security, you know, we're part of the OpenSSF, we find a lot of security initiatives and in projects. Yeah, those are the two main things anyways.[P17]

% And I think that is what the first thing that organizations need to do in order to be a good open source citizens is to understand open source, and how to work in open source because it's completely different from the product, the proletariat somewhere, like, they don't have a community at all, that didn't open source all the project had here, here, it works with all the community in consensus and, like more democratic.[P19]

%And then coming back inside the company, you're talking about: How do we work successfully with open source, right, and what the norms are in the community. And from an executive leadership perspective, you're talking about how it aligns with the company's objective, and business goals. [P18]

%And, and so I think there is definitely an aspect for being heard the right word is like not letting go of your hubris. And just focusing on what the project needs and being sensitive to the fact that there's a group of people already in place that are doing this work. And understanding how you can support it, if this is something that you you are you find valuable. So I think it's definitely, again, a lot of individuals making their own way. And if companies want to get involved, it's really about empowering the individual. And trusting that their individual will choose the supportive way in the project versus dictating their role in the project. [P20]

%Because I don't necessarily think that the companies always have the best view of that if their employees are the ones that are in this space in the community in the project, then trust them to to contribute in the way that makes sense to the project, which may or may not always be, again, what the company would want for beta. [P20]

% 8:20 ¶ 43 in [P7] Josep-Aiven interview.docx
% So we don't enter that magic conflicts, of course, when sometimes we want to have some feature. But we do usually we discuss and we showcase that the benefits of this feature. That's why we think we should add this one. And then it's a discussion back and forth, explaining the benefits, try to reducing the risks or problems of that solution and trying to reach a better one.

%5:37 ¶ 223 in [P5] Jill-SlimAI interview.docx
%these habits of saying, like, are we really doing right by our users regularly

%And so when you think about, like how that is delivered to developers, today, we have communities, we have a DockerSlim community, we have a Slim SAS community, and the internal team at Slim, is responsible for cultivating both with equal level of we'll call it responsiveness care. And, and, and not just responsiveness and care. But like also like, curiosity. You know, one of the things that's really cool about having an open source community is is you learn from them just as much as they learn from you. And, and so that that's woven into how we operate the business, is taking the user feedback, taking the user learning, taking the user, the new problems that users want to solve, like taking all of that and ingesting that back into our product delivery process. And like, really understanding how we're refining what we're delivering both on the open source and in the SAS, those things are cohesive.[P5]

%5:80 ¶ 227 in [P5] Jill-SlimAI interview.docx
%You do need to communicate to them regularly, in a way that brings them value. Right, and you do need to communicate them to them regularly. So that you can hear what's bringing them value. Like there's got to be that that that two way line of communication

%5:19 ¶ 119 in [P5] Jill-SlimAI interview.docx
 %What we're doing is we're taking the time to be insanely focused on the value that we're delivering to users so that when we do turn that on, we have extreme confidence that we're turning it on at the right time for the right things. Right. So it's essentially doing user research, kind of like you're doing your interviewing right now to validate your hypothesis. Right, and also to be subjected to things you're not considering. 


%6:25 ¶ 151 in [P6] Otavio-Zup interview.docx
%We did not offer. Like I said, our we don't have a business model for those products. So we don't use them to offer services, we only we only invest on their development and so improvement, so we were not making service for money, we are not relating the investment we do to those corporates to achieving some results for the company, we are not related to for money results are not related to how we can have clients to them. So we are strictly related to community. So we asked for it for companies that are wanting to use it, or people that wanted to use it, to use our forum use GitHub to share what they need or features they need, but we we don't have a roadmap and see how we make we may make money how we make make, make service or offer service using so we don't make it we No don't offer Consultancy Services today, using our open source products. And we are not considering to offer service to other companies by using our open source products. So right now, they're not considered for us for anything else than to contribute to the communities.

%And that's something that we should, companies that have the means to provide people to work on open source should be doing those things. So that's one of the things that we tried to do. So we don't enter that magic conflicts, of course, when sometimes we want to have some feature. But we do usually we discuss and we showcase that the benefits of this feature. That's why we think we should add this one. And then it's a discussion back and forth, explaining the benefits, try to reducing the risks or problems of that solution and trying to reach a better one. [P7]

%7:9 ¶ 23 in [P8] Johnathan-Goliath interview.docx
%And we're also contributors, so we joined the board. So we joined the project, we contribute money, but we also contribute time and have individuals on our team who contribute to Zephyr even if it's not directly relevant to take a life as a company. We just want to see the project expand and grow. And so therefore it's strategic to us to see that to invest in the community.


% So it's really challenging, because in all of your work in open source projects, you really have to think of the community first and the project Second, and this is or sorry, and the company second. So project first company second. Because we can't just bulldoze our way into getting getting what we want, we have to, we have to think about how what we want benefits the community as well, and work on things that are going to be of mutual benefit to both. [P11]



% And so Tidelift's view is that for open the open source ecosystem to be healthy, both for the maintainers and the communities as well as the downstream users, we need to be creating predictable income flowing to maintainers. in those communities so they can plan around it and have the time that they need to do the work that is being asked of them.[P12]

%12:3 ¶ 26 in [P15] JoshB RedHat Interview.docx
%The the second way that we're involved with with open source is community. As in open source projects, collect people around them, which are known as open source communities. And open source communities are a good way for us to reach people. It's a major portion of our marketing picture, when I'm talking about marketing with a big M. The and and they supply a lot of sort of key inputs to marketing, you know, both not just, you know, people to basically advertise to, but also as a way of learning what is it that people want, so that we can actually build those products? Right. Um, so marketing in terms of market research, and in terms of getting direct feedback? You know, like, if we're thinking about something new in Linux, we can put it in fedora and see how it does in fedora. Um, you know, before we have to offer five years of support for it.

%19:26 ¶ 101 in [P16] Stephen Transcribed Interview.docx
%And I think they're the the component of again, like being that good open source citizen. I've seen so many instances, across companies, where people just, they'll come stomping in, you know, to an open source project and want a thing or expect a thing at a deadline, you're like, cool, come do it yourself, then you can have it whenever you want it. Right. So like figuring out how to know like, and, and open source timelines, deadlines, if there are any, are never going to be that closely aligned to your internal deadlines, right. So like, being patient, and knowing that, you know, if you are here to consume, you should, you should probably step back and think about how you're going to do that. Right.

%And the community is like, equally consulted on decisions, which has taken them in different directions and they already-- than they originally might have gone. There's like one example I think of, Microsoft wanting to put telemetry in the repo and the community was like, "No way!" And so they're like, "Okay." So you know, like that. But, you know, the act of that decision really helped build trust. [P17]

%-------------------------------------------------------


\cellcolor{gray!15}\textbf{Prevent maintainer burnout:} Participate in maintenance tasks 
& 
\cellcolor{gray!15} P7, P17, P18, P20
& 


\cellcolor{gray!15} ``Something needs to be shiny, for you to do it on your free time...companies that have the means to provide people to work on open source should be doing those things'' (P7) 
%\newline
%``learning how you can help and we often talk about that chopping wood, and carrying water thing... sometimes the most useful things you can do are the non-glamorous things... humble participation is most important'' [P17]
\\
  
  %And I think, if you've been following any of the software, supply, chain reporting, and security, just all these sorts of quotes that like 95 to 97% of companies use open source, but how many of those companies are actively involved in maintaining the things that they use and have a vested interest in it. [P20]
  
  
  %And I think it's often that sort of central support is the least glamorous work. In terms of say, working on just maintenance and operations and working on community management tasks and logistics, connecting people and doing the sort of, again, more community focused efforts that are not our, I'm gonna say it's harder for some companies get involved, depending on their incentives. [P20]
  
%"we hear this thing ... like some maintainer, it's burnout ... leaves the whole thing behind just quits everything in the world and says, I'm, I'm done. So we want to, we want to try to avoid this thing. We want to support them to give people time. So they don't need to carry that burden alone, that will be more people to carry part of it. And if you spread it, then probably it's bearable." [P7]

%8:16 ¶ 43 in [P7] Josep-Aiven interview.docx
%So we want that all the projects that do we do operate and we do manage, they do support this thing natively. And it is something we would like to contribute to many projects. Also we are in the cloud, meaning that everything you transfer costs money and we want to reduce the bill of people. I mean, we don't want people to pay just because because so again transferring a lot of mega bytes on the wire costs money on the on the cloud. If we compress this thing massively on already on Origin, suddenly the bill gets lower just because we did once tiny, small thing on the software, the these are the features that sometimes we want to bring. And most of the times they are sensible in that sense. So there is not much of a discussion.


% In fact, I'd say that we encourage humble participation, learning how you can help and we often talk about that chopping wood, and carrying water thing that you'll you know, you hear in open source, you know, to-- I've run workshops around that, that exact thing. It's like, sometimes the most useful things you can do are the non glamorous things. So like, well, yes, like nails in the leadership position was very visible, some of the best contributed, we emphasize that and like the CNCF has a chopping wood and carrying water award around that. So we also emphasize that so. So both, but I think more so, you know, being humble, it's not like I'm from Microsoft, I can do this. I can take over this. And it's like, no, like, you know, humble participation is most important.[P17]

%And, I mean, companies can do a lot more here as well, if it's really a valuable project, and it's struggling for maintenance. Companies should help, you know, with the maintenance and not just the feature functions, but the heavy lifting of code reviews and maintaining the project and doing the security and the plumbing.[P18]
  
%------------------------------------------------------- 
%\textbf{Put your money where your mouth is:} Where you consume, contribute 

%\newline


\textbf{Keep the contribution wheel spinning:} Where you consume, contribute 
& 
P4-P9, P12, P20
& 
``Everyone should be contributing back, everyone should be making sure that they are not just the receiving end, they are receiving and producing end of the open source'' (P7)


``We generally believe there's a contribution wheel, so we owe, if we can upstream something or contribute back to a project we use, then we will'' (P8) 
\\
  
  %But there is definitely a mentality shift of a technology consumer versus a technology creator, in the sense that if this is an older company, they're used to just paying for money for a thing, there's no expectation that they need to go back and improve the thing. And when they were consuming open source, either directly from the community or via a product version there, we're still this sort of like, well, I'm just going to use it, I'm just a user and like, and submit feedback or feature requests, but like, there is no expectation of dedicating time to improve it. And I think, in general, we haven't really seen the consumer become the contributor is much like, I think we've seen individuals that are working in it, and the engineering side that find their way in and coming more so from technology vendors, or new technology companies. And it's, it's, it's, I don't know, I haven't really seen the end of the funnel coming back. And in terms of actively recruiting that. [P20]
  
  %And but yeah, what I'm supposed to do here at Spotify is to basically streamline how we open source internal code and how we we drive these projects internally, but also how we work with dependencies, open source dependencies, and we support these projects, and so on. [P4]
  
  %5:26 ¶ 69 in [P5] Jill-SlimAI interview.docx
 %And I think that's another thing about like, we call it our virtuous cycle, is our virtuous cycle is giving back into the ecosystem of developers by always providing them value, and then encouraging them to do the same.

  %We like we like to use virtuous, like, it's virtuous for us to do these things. When you put good out, you get good back.Right? It's, it's this kind of like infinity loop of, you know, you, you attract what you put out. And it continues. So if we put out things that are false, we're going to get back things that are false. Doesn't mean that we can't make mistakes. That's not being false. Right? We can, we can hit we can hit bugs, right? [P5]
  
%``We try to spread the word about open source, because we all use it, right? We, we use it all the time when we're are creating software. So we try to help bring this awareness about open source to everybody.'' [P6]

%8:16 ¶ 43 in [P7] Josep-Aiven interview.docx
%So we want that all the projects that do we do operate and we do manage, they do support this thing natively. And it is something we would like to contribute to many projects. Also we are in the cloud, meaning that everything you transfer costs money and we want to reduce the bill of people. I mean, we don't want people to pay just because because so again transferring a lot of mega bytes on the wire costs money on the on the cloud. If we compress this thing massively on already on Origin, suddenly the bill gets lower just because we did once tiny, small thing on the software, the these are the features that sometimes we want to bring. And most of the times they are sensible in that sense. So there is not much of a discussion.

%7:15 ¶ 39 in [P8] Johnathan-Goliath interview.docx
%We use a lot of open source. And in the process of using that we tend to contribute to things we do use.

%7:6 ¶ 19 in [P8] Johnathan-Goliath interview.docx
%And for our own purposes, we will implement a feature, or a let's see a library around the open source project. And then we invariably try to contribute that back whenever we can. Whether it's because it's useful, we think we want to contribute that back to the project. And sometimes it's pragmatic, like, well, it's a small enough feature enough, people will use it, but we don't maintain a fork. So let's just get it upstream. So you don't have to maintain the fork. But also, we just think it's useful, and we want to contribute it back to the community. So again, that's the, for the stuff we run in production, we use open source whenever we can, sometimes we contribute back to improve the product, sometimes we do to make it better.

%So again, you're not able to build in community. And if you're not able to build in the community, then the benefits of open software development and open source are completely zero. There's there's no special benefits to things in the open. It's nice. But it's... there's no there's no reciprocity in some ways, even small, then you're missing the point entirely. [P9]

%So there's there's one possible challenge, but it's more of a problem. That's the only challenge maybe free riding. Which is if you do free riding. Which means when you have contributors, well, users that that eventually abuse and free ride without ever contributing back.[P9]

% but we still participate in events and community forums as much as possible. So conferences, books, I don't know, I mean, I've got open source books on my shelf, but I don't know if that's me consuming it or type that consuming it? You know, we, as I mentioned before, in terms of, more or less on the consumption side, more on the contribution side, we sponsor events, we sponsored programs like Outreachy, which are, you know, paying, paying for internships to work on OpenStack. So we, it's a fairly standard for any company that's pretty invested in open source. It's a fairly standard set of of interactions between us and open source communities. [P12]

%companies start realizing they need to give back for reasons of de-risking their supply chain, to a point where companies starts giving back because they see that it actually helps them strategically, technically, because suddenly they've got influence in projects that they rely on. Ultimately, and like open source enlightenment, from, from a corporate perspective is basically companies that are really making investments in open source that benefit the whole ecosystem. Because they know at this point, that a healthy ecosystem means good business for them, too.[P12]

%-------------------------------------------------------
%\textbf{If the community is happy everyone is happy:} Build a good relationship with the community 
%& 
%P5, P11, P13, P15, P16 
%& 
%``What kind of relationships with community you have is also important'' [P13]\newline
%``We're inviting experts from other ecosystems like <company name> and <company name>, like we are coming forward with, we don't just want to solve the problem for the developers that we solve. We want to understand how developers and other ecosystems are solving similar problems in different environments'' [P5]\newline
%``The conversations are so detached, sometimes from the community, that's that you you should be you should be, you know, kind of, you know, peeking through the forest and seeing what's actually happening on the ground'' [P16] \\

%And, and, and not just responsiveness and care. But like also like, curiosity. You know, one of the things that's really cool about having an open source community is is you learn from them just as much as they learn from you. And, and so that that's woven into how we operate the business, is taking the user feedback, taking the user learning, taking the user, the new problems that users want to solve, like taking all of that and ingesting that back into our product delivery process. [P5]

%You know, I mean, unless you're an expert in that particular field, or you're spending money on that, not much. However, if you have a sense of the community that is around it, then suddenly you can gather a whole bunch of other information, well, oh, look at that. There's like eight other universities in the US, similar as mine, using that software and contributing to it. That's extremely useful information.[P13]

% ``We do engineering, either by ourselves or together with contributors who work for other companies or no company. The second way is considered the ideal'' [P15],

%-------------------------------------------------------  
\cellcolor{gray!15}\textbf{Prevent corporate abandonment:} Open source projects that your team cares about
& 
\cellcolor{gray!15} P4, P17
& 
\cellcolor{gray!15} ``We do want to open source things that play into our own brand of the company...If it's something that is a core part of our teams' function, that's also great because it's something that they [we] will continue working on'' (P4)
%\newline
%``it looks terrible when there's like abandoned repos, if a company just puts something... looking at the upstream, making sure that your engineers are not just, you know, we're very mindful'' [P17]
\\

%2:26 ¶ 151 in [P4] Per-Spotify interview.docx
%e is some some sort of reason why you actually open sourced it, beyond just like personal growth, and actually becomes an integrated part into, like, why is the team considered successful? If it's not, then it's going to be left by the wayside at some point, because they'll be more important things. But if it's actually an integral part of the, the objectives of the team, then they will also continue to be maintained.

%And then you could say on the other end of the scale is like things we do want to open source is like, especially things that play into our own brand of company. Like if it's something about music and technology that would be amazing. If it's something that's a core part of our teams function, that's also great because then it's something that they will continue working on[P4]

%just know what you why you're doing it, you know, just coming out and be like, Oh, we open source something. So that's really important. And then like, it looks terrible when there's like abandoned repos, if a company just puts something else that's around release, around use, again, is like, yeah, looking at the upstream, making sure that your engineers are not just, you know, we're very mindful, a lot of times we have compliance software to help them be mindful.[P17]

%-------------------------------------------------------
\textbf{Have an abandonware strategy:} Transfer ownership of OSS projects that you internally moved away from but the community cares about
& 
P4, P6, P10, P15, P17
& 
``If we stopped caring about them [open sourced projects that the company moved away from], then we should transfer it to people who care about them'' (P4)
\newline
%``We have a project that is actually a big success, like open source wise, but we don't use it internally anymore, and we don't really care about anymore... But outside of [company name], it is, it's a perfectly fine living project with a lot of different contributors'' [P4]
%\newline
``Eventually, you stop using the tool internally. But you know, have partners and customers that are using it...you can just kind of gradually pull all your engineers off of it, and let them have it. Um, you know, so this is what's called the abandonware strategy'' (P15)
%... where you say, ``Hey, I know you guys are still using this. I know you still like it, but it doesn't make any sense for us as a business anymore. So here, it's yours. Now you can have it'' [P15] 
\\

%2:83 ¶ 61 in [P4] Per-Spotify interview.docx
%But outside of Spotify, it is, it's a perfectly fine living project with a lot of different contributors. It's kind of a good tooling for like, kinda like data pipeline thing. And we still own it, we still has our name on it, but we don't use it. And we don't contribute to it at all. It's like done by all these other engineers who kind of just like driving it now. But we have no idea what's going on inside of it.

%So not just related to the communities, not just improving how we know as a technology creator and maintainer, but also investing on the concept of open source ecosystem, and how we could work with them to improve this product instead of just freezing it or just archiving it.[P6]

%had about between 35 and 50, people whose full time job was to work on open source projects, their full time job was to work on open source projects that the company used. So they were committers to a bunch of Apache, basically, Apache projects, for the most part. And their job was to continue to grow and evolve those projects, because the company used those projects. And if some one of them left, we would hire another person to fill the role. Because Because, right. But as it turns out, that isn't as common. More common, is an individual or a small team has a passion about a project convinces their company to let them and if they leave, the company won't replace them. Or if they replace them, they will replace them with somebody, but that person might not contribute. So corporate abandonment of open source is a good signal between is a good way to differentiate between companies that are truly recognize their dependency on open source, versus companies that merely allow a passionate employee to do something that's safe enough, but when they leave, they're not gonna backfill. [P10]

%And it's okay, often tell, tell projects, like, it's okay to re-evaluate [in 2 months?], that's not working or priorities change, we had a re-org, you know, if we had a re-org, and we don't have resources for this, then have a plan for sunsetting that project and talk to the community and, you know, like, offer forks and those kinds of things. So it's, there's a lot to it, but just baby steps and think about where you are now, where do you want to go and revisit?[P17]

%-------------------------------------------------------
\cellcolor{gray!15}\textbf{Avoid company monopoly:} Invite other companies in 
& 
\cellcolor{gray!15} P3, P5, P7, P11, P15, P17
& 
\cellcolor{gray!15} ``Part of the healthiness of the project is also making sure that there are different ideas, opinions, and points of view, it's really easy to fall to a kind of monopolistic way'' (P7) % the best things of the open source is that it's developed in the open with everyone'' 
%\newline
%``For a very successful public open source project, that is one where you have a public community of people contributing and involved, who work for all kinds of different employers, or no employer'' [P15]% or, you know, all over the place'' 
\\
  
  %1:16 ¶ 227 in [P3] Omer-VM interiew.docx
%Yeah, sometimes it happens like that. Sometimes it sometimes it happens where one company is building an engine, and they're like, come help us build an engine. And sometimes, that works. Kubernetes is a great example of that. Google open sourced it, and then invited a bunch of friends along Did you know there's a Kubernetes documentary by the way?

%you need to get you need to get you need to get multi vendors like multiple vendors involved so that you can a) like improve your product because you have to like be accepting opinions from lots of different people and then b) you get more market share that way.[P3]

%1:32 ¶ 203 in [P3] Omer-VM interiew.docx
%Google reached out to a couple of vendors. They got together on an off site. They made a bunch of decisions. They started building stuff. That's how they got involved. 

%We're inviting experts from other ecosystems like Shiny and R, like we are coming forward with, we don't just want to solve the problem for the developers that we solve. We want to understand how developers and other ecosystems are solving similar problems in different environments.[P5]


%8:48 ¶ 51 in [P7] Josep-Aiven interview.docx
%So they wanted to have a backing up of different companies as among them, and we tried to cooperate, like in a, like regular basis, and we discuss and we tried to agree on what to bring what, what makes sense to drive it to

  %"But again, the best things of the open source is that it's developed in the open with everyone. So everyone, and we want to be part of the everyone, we want to make sure that it's everyone. And we don't want to just to be said that. Okay, yeah, it's only one company behind because nobody else is interested. No, we will be interested, we want to be behind those things want to have an opinion, we want to try and help and shape those project, the best way give our opinions and, and do our best to bring it to the greater success" [P7]
  
  %"I mean, we are there to contribute on that community of these projects to make the project better, more sustainable, to bring more different ideas on the table as well. So not just one single companies dominating one open source project, but we try to bring we are another companies that we try to also bring opinions there. To not just depend everything on one single company and to create a diversity of opinions, ideas, companies behind and all this stuff. So also to support and just stay true to the open source philosophy." [P7]
  
  %And you get a real innovation boost by pulling in people who work from other companies who will use things differently than you do a lot of different different use cases, they'll have different skills to contribute, they'll have different ways of thinking about things. And so we see that we see that innovation as being something that is really absolutely critical to us being able to do what we do you know, you without something like Kubernetes, we, we wouldn't have been able to spin up this Tanzu product line in a relatively short period of time, considering, you know, how long it normally takes to build a product at that sophistication, but, but by having something like a Kubernetes, we can just innovate on top of that, because we have, we have the platform.[P11]
  
  %where we do engineering, either by ourselves or together with contributors who work for other companies or no company. The second way is considered the ideal. Right? If we can make that happen, we do make that happen. Because, you know, why would you put in 100% of the effort if you didn't have to, plus, getting contributions from outside Red Hat often bring with them perspectives from outside Red Hat, which allow us to make products that fit a broader clientele. [P15]
  
  %But I mean, I think, also going back to like, Microsoft, and collaboration, like, you know, when we have our performance reviews, one way that we're at one thing that we're asked about is like, how did you build [on with others?] And it's like open sources--or inner source-- Is a great answer to that question. So it's, it's, you know, the fact that it's inner diversity, inclusion is another core pillar for, you know, our reviews and performance reviews, I think that collaboration is at the highest level of your review, I think says something around how important that is. [P17]
  
  %So the more that we're able to collaborate with external communities, the more perspective, the more, you know, kind of like user stories or use cases or whatever word you want to use there. The better the product you're building,[P17]
  
  %I'd say on the releasing open source, like: always have a plan for-- or goals for external collaboration, like releasing open-- [P17]
  
%-------------------------------------------------------
%\textbf{Don't pull the rug out from under your users:} Go from close to open never the other way around 
%& 
%P1, P9 
%P2, P3, P7-- talk about elasticsearch example
%& 
%``We also have customers that require it to be open source. And so even if you wanted to, well, if you wanted to make it proprietary and not open source anymore, then we would lose some of our customers'' [P1]
%\newline
%``That's really the problem... you see, as a prediction of many companies involved in open source, they may start with some kind of project, sometimes with a permissive license. And then they go through copyleft license with contributor license agreements, and then they move to the proprietary'' [P9]
%\\
  %4:30 ¶ 71 in [P1] Georg-Bitergia interiew.docx
%have customers that require it to be open source. And so even if you wanted to, well, if you wanted to make it proprietary and not open source anymore, then we would lose some of our customers.

  %"It's been always brought forward as one of the primary reason why they were doing this switch to proprietary licensing is that there was free riding predatory practices...When you do a lot of the code, or when you produce open source tools, ...You are making gift. And it's a bit difficult to say, Yeah, well, making a gift that I really don't want everyone to benefit from the gift."[P9]
  
  %10:69 ¶ 224 in [P9] Phillipe NexBe Interview.docx
%were very clear about what were the benefits of being open source on a budget. So now, the ratcheting effect to go from permissive to copyleft to proprietary, I think is a terrible thing from a [integral?] and community perspective, because it's, it feels like you're pulling the rug under the feet of your contributors. as you're making each time you're adding extra extra requirements. But going the other way around. Nobody ever complained.

  
%-------------------------------------------------------
 %\textbf{Understand it's a marathon not a sprint:} Understand it's a long term investment 
 %& 
 %P3, P7, P8, P11, P13-P15 
 %& 
%``And sometimes the value is really thin or vague. Or it's something that it will come on long term, not as a short thing. It's not 100 meter sprint, it's a marathon'' [P7]\newline
%``It's seeing the long tail of their investment... We spend a lot of money to make sure that developers like us. And like that is a decade's long investment. Right? We're not reaping benefits on that in six months, then that is that is like a long tail investment for <company name>. And it's a great investment'' [P3]
%\\
  
%-------------------------------------------------------
%\textbf{Go the extra mile} in being present and involved 
%& 
%P7 
%& 
%``We obviously try to be present everywhere we can...We try to go to conferences, we try to write articles on presence while trying to showcase what we do share the Word of either what we do at <company name>, what we do in the open source space, how we shape things'' [P7] 
%\\
  
%-------------------------------------------------------
%\textbf{See beyond the short term competitive advantage:} Open Source projects you care about 
%& 
%P10, P13
%& 
%``You seen that kind of use cases, when folks are thinking about open sourcing things that would otherwise give them a competitive advantage? Right. So the idea is, is this giving me a competitive advantage? Can I get more value? Out of that, ...by using it in a way that makes it more open?'' [P13]
%\\
 
%-------------------------------------------------------
%\textbf{Be there for the right reasons:} Contribute for the right reasons and everything will follow 
%& 
%P7, P14 
%& 
%``So but again, it's not for the sake of that reputation. If you do it only for that it doesn't hold, it will not last long. Things will be seen. And that's not also a sustainable way of doing it. You need to do it for the for the real reason to basically support the project. And if that happens, then then then everything comes up'' [P7]
%\\
 
%-------------------------------------------------------
 \textbf{Be a polyglot in promoting OSS:} 
Speak risk management, engineering benefits, legal \& compliance, and sponsorship 
& 
 P7, P10, P14, P16-P19
& 
 ``An interesting challenge for a successful open source program in a company is to be able to speak multiple languages, to speak to the risk legal people about risk strategy and, you know, compliance, to speak to the engineers about the benefit of being in the goodwill...and then to speak to the CFO, about, here's how we save money, here's how we make money'' (P10) %we save money on these proprietary licenses, and on the tech debt... and all those are the three conversations, the fear, risk management, the love, community, goodwill, and the money are all part of the conversation. So an OSPO has to negotiate between'' [P10]
%\newline
%``The OSPO will be successful, if they manage to convince or to make sure that the philosophy of open source is spread across the company'' [P7]
%\newline
%``there are people who've never worked in open source before and who...are learning or might be fearful. So, I think that's the big picture, but that there are lots of little individual feelings, stories, people have different levels of learning'' [P17] 
\\
%\hline

%ou first need to convince the legal team that you can do open source and the legal team needs to make sure that all the security risk are covered. And all the open source compliance is in place. And all the software inventory is also managed and automated. So that is the first thing. Because if you don't assess security, and you cannot move forward, and once you have this big block, that might take more, or might take less, depending on the hospital under specific issues and, and barriers, then we can move to the community driven states that is more like, Okay, I have everything I know, my company has like, for instance, these as both of these are different inventory, the opens of praise, where I can contribute, great, how can I contribute is when the community location states happens and building internal ambassadors to try to educate on the different teams the importance of open source and how to collaborate in open source, then we'll come more like the engagement phase, like once they know how to do it, it's time to do it as part to contribute to the open source projects, or maybe building stuff and thinking, Oh, we should be open source this project, I think this is really important. And to the point to then be this leaders are advisors of open source of the organization's technology stack. So there are different phases, and it takes time, depending on the organization and the factors. But I think like this chronologic word Yeah, like this order. Makes sense. Right to, to advance smart, easier in this journey.[P19]

%Absolutely, absolutely an OSPO leader needs to be very skilled at talking to leadership, to middle management, to developers, and to the outside world. You're really creating a messaging to the outside world on on companies involvement of open source. And then coming back inside the company, you're talking about: How do we work successfully with open source, right, and what the norms are in the community. And from an executive leadership perspective, you're talking about how it aligns with the company's objective, and business goals. And with middle management, you are hoping that by talking to the executive leadership that we've created enough of air cover for managers, and incentives for managers to make time for their people to do open source. And they're rewarded for doing open source and they're budgeted for doing open source, right. And even when you release a project, we want to make sure that the developer has some time set aside to maintain that project to review issues, accept PRs, review code. Otherwise, it's it's just dumping the project and running and the reputation that the company gets is no one pays attention to the project. Don't get involved in that project. Well so. So it's a lot of hard work. That's one of the hardest tasks, I would say the OSPO has: is working with different levels of leadership. And showing how open source helps the business. And why to set aside time and money to work with open source.[P18]

%That's interesting as well, I think I think for these like different product problems, you already mentioned that we are very bad at actually releasing something and then continue caring about it the next few weeks, the actual benefits of open sourcing it. Because I think we do actually invest a substantial amount of time and actually producing open source code, but we are terrible, actually getting any tangible benefits out of it, is one of them. The other one is like, Spotify is very happy with the success of Backstage and they do want to create more of these kind of successes and, and having more like find go down through like having some sort of formula for how to do this instead of like, just by accident. And, and then I think the last time this was all that's according to tech recruiting in the US, especially, Spotify is seen as a more boring company, than compared to like Apple or Microsoft or Facebook or Netflix, for instance. It's not as it's not.. it's seen as like, boring, basically. It's just a music app. Yeah.[P4]

%8:101 ¶ 111 in [P7] Josep-Aiven interview.docx
%Yeah, I think they they help on deciding those things on buffering those ones as well. So because I guess one of the worst things that can happen to a team is to know that they are in trouble in that sense, or that they will be scrutinized this way. So having those OSPO teams that they try to define those frames, they do all the evangelism at company level, on why, what, what to expect, what's the rhythm, what to not expect, and how these things work. So that's the, that's why you should have probably an OSPO, because then it's concentrated there, it makes sense. But then again, the OSPO should not monopolize all the open source contributions on the OSPO team. They should also try to make sure that all the company understands and does believe in that sense, the open source way, and everyone should be able to contribute. And everyone understands what it means. And it's not something strange or alien to the company, they should make sense should make sure that it belongs to the company. And that is when the OSPO will be successful, if they manage to convince or to make sure that the philosophy of open source is spread across the company.

%And I think that a fair analysis of contribution takes into account that the OSPO or somebody in the company needs to be able to make the call and say, Yes, please do that. We encourage you because it helps or no, you can't do that as an employee of the company that that violates a law or that violates company interest. And why would we pay you to do that[P10]

%And an interesting challenge for a successful open source program in a company is to be able to speak multiple languages, to speak to the risk illegal people about risk strategy abatement and, you know, compliance, to speak to the engineers about the benefit of being in the goodwill that one engenders in this, and then to speak to the CFO, about, here's how we save money, here's how we make money, we save money on these proprietary licenses, and on the tech debt, and on these, whatever, we make money through an ecosystem play, or through getting people to be dependent upon our format, which will then upsell them to tools that they might need that are more resilient, or support or training or, or whatever. And, and all those are the three conversations, the fear, risk management, the love, community, goodwill, and the money are all part of the conversation. So an OSPO has to negotiate between. And if you use the wrong conversation, if you go to the CFO with a love conversation, she'll look at you and say, I don't care, right? In fact, it's my job not to care about that. And if you go to the engineers and say, We're gonna make money, and they're gonna say, Yeah, fine, but you're using us to make money and you know, does it fall through? Right? So you have to, you have to recognize that there's a multifaceted game going on here. It's not just about people who love each other and want to give each other code to help it at. That's true, but it's not the full story at all.[P10]

%You know, it's, it's a very, very different conversation... because we have this and so OSPO tries to take the essence of all of these ideas, and we try to encourage other organizations to do the same. And that's, that's the other part of the work I also do is talking to other other companies like it could be a manufacturing organization, it could be a hospital for that matter. I you know, it doesn't matter who they are. They are not in this industry, right. But they use technology, they use technology in every every part of their business. How can they be part of the open source community? What would take them to become part of that? What would it be? Sometimes they don't care, which is true, which is I don't care. I mean, I just want to get my job done. And once once I'm done with a job with this technology you provide, I just walk away, I'm not okay, that's fine. But there may be others who may say, hey, yeah, I think this is interesting, I want to improve some part of my organization, my whatever. And then once I improve it, make it better. And, you know, whatever. I want to contribute this back to the open source community. Sure, how can I do that? Oh, okay, we can help you to think through the issues, think through the the needs and responsibilities and expectations so that you don't go down the path, which is a problem, because we do have organization say, oh, you know, I didn't realize that this is going to be a lot more work. I thought, I just have to open source and stuff and everything is done. No, that is step zero, that is step zero, then comes the real work, the real work starts after that. And you have to be committed to do that kind of additional work until you reach a point where, you know, yeah, there is enough of a community around it already. You cannot just say I created this, you know, fantastic solution to do accounting packages, great. Open it up, thank you very much. Don't just walk away, that isn't stopped, you now need to encourage others to contribute, create a community around it, it takes time. And if you've never done that before, that's why we try and help them. So that's the other part of the story. We from an OSPO perspective, there are organizations that are struggling trying to figure out how to do this. And they reach out to us for help. And so OSPO trying to fill that role [P14]

%So if you, you know, kind of the approach that we're taking internally, is we've kind of got three, at least three lenses and you know, each of these lenses can can be subdivided as he wants, where there's, there's the kind of the lens of the contributor, the lens of the maintainer and then and then kind of the lens of a sponsor, right. So I'd say... So Oh, yeah, the nurturing each of those, those those qualities, right, we're not just within, you know, not just as Cisco, but you know, within each of each of our employees, right, the hope is that you should feel comfortable to come play right in kind of the open source sandbox. [P16]

%There's different ways I can answer that. I think, from the perspective working in the OSPO, short for open source programs office I know, you know that but maybe for your notes, it is central to excellence in engineering. So open source is not separate or optional, but it's central to excellence in open source engineering, and culturally speaking, very much in line with our values of building on the work of others and finding others to build with us. There's quotes from Sasha, our CEO, or on the importance of collaboration. I mean, the other ones, so there's different ways I could ask that. So that's like how an answered from the OSPO perspective. But you know, there are-- Microsoft is giant, there are people who've never worked in open source before and who, you know, might, you know, are learning or might be fearful, or, you know, so I think that's the big picture, but that there's lots of little individual feelings, stories, people have different levels of learning. [P17]

%-------------------------------------------------------
%\textbf{Sell what's on the truck:} Take the time to fully develop the OSS value before providing service on top
%& 
%P5 
%& 
%``I have seen organizations that are selling things that I described as they are not on the truck, right? They're selling future roadmap to customers... they're not selling what's on the truck, meaning it doesn't exist today. I can't ship it to you. Right. So we're taking a different approach. We're not selling roadmap, we're gonna sell what's on the truck'' [P5]
%\\

%And I have seen organizations that are selling things that I described as they are not on the truck, right? They're selling future roadmap to customers. They're saying, Oh, if you if you renew your contract for this year, we're going to have this, this, this, this and this, and the next 12 months re up now so that you can get that at a discount. But they're selling future roadmap, they're not selling what's on the truck, meaning it doesn't exist today. I can't ship it to you. Right. So we're taking a different approach. We're not selling roadmap, we're gonna sell what's on the truck.[P5]

%-------------------------------------------------------
%\textbf{Leverage the no middleman land:} You can interact with the developers directly 
%& 
%P13, P14 
%& 
%``Stop putting intermediaries all over the place, and suddenly, it's easier to communicate'' [P13]\newline
%``That's right, direct feedback. Yes, exactly. Exactly. That is a huge benefit. That's a big benefit... want to talk to them directly? Go ahead and do it'' [P14]
%\\\hline 
  %"Here you can, go ahead and reach out directly. We are not we are not putting a firewall. They say no, you cannot talk to my developers. Because the developers must also understand what the customer's requirements are. Because if I'm building something I want to be able to, you know, know, how are my customers using it? What are the challenges they are facing? It I don't want it to be, you know, only coming by feedback and then somebody collects all the feedback and then massages the feedback" [P14]
  
  %"Can I speak to them directly? That is the power that we bring. And so majority of our customers are surprised that that can be the case. Because they don't expect it. They don't expect to be able to speak to the developer that develop the stuff that you are running on your enterprise system. They don't expect that you can expect Yeah, you can't be expected to find who wrote Oracle Database. Who are the author's? I mean, they may not be around, they may have left but you know, is there a name that maybe is there? But yeah, I can see it as a customer. I can see I don't know there. So it is giving that human face to, to the whole thing, which I think changes the dynamics" [P14]
  
%--------------------Projects Takeaway-----------------
%\cellcolor{gray!15}\textbf{Avoid toxicity and foster the project's positive environment}
%lower the toxicity of your project have a friendly open project where all contributors are respected and welcomed and interactions are friendly and not toxic 
%&
%\cellcolor{gray!15} P5, P7, P10, P15, P18
% &

% ``If it's an abusive community, even companies don't want to get involved. And the other is, is just, you know, companies look at is this a healthy community?'' [P18]
% \newline
% \cellcolor{gray!15} ``They [companies] run far away because the moment you see an internet spat or an open source spat, the corporation sees up in like, oh my god, like there is no way we're gonna come out of this good, right because all internet fights... they're like, oh my gosh, step away'' [P10]



% %``We are not going to allow our community to create fear, or lack of safety. If you're afraid to ask a question in our community, we're failing. If you're fearful that you're going to look or sound stupid in our community by asking a question, we have not set up our community the right way'' [P5]

% \\

% %And when you have a set of people that are from different backgrounds, and different incentive structures, and companies and locations, there's a lot of nuance that goes into ensuring that you're fostering a healthy and respectful community. And I think more so I think a lot of what we hope to bring to the table beyond just the funding and the people is support in in these sort of softer dynamics that are vital to the success of any project. So that's more of a lofty goal. But we've been investing in more programs and initiatives that we hope will help to just improve that aspect of things and improve the way that we have we create spaces for people to connect, we have been trying to mentor more people on understanding inclusivity diversity, equity inclusion initiatives, how to have these sorts of conversations, how to have recognition and Ally ship and all these sorts of harder, more difficult personal based dynamics. [P20]

% %So projects also need to make it easy for not just company but everyone to contribute, make it a welcoming community, have contribution guidelines, review issues, give proper feedback, of code of conduct, all of those kinds of things. If it's an abusive community, even companies don't want to get involved. And the other is, is just, you know, companies look at is this a healthy community? Is there a proper, you know, cadence to the innovation? Is it maintained by one person, you know, just one person somewhere who randomly will make things happen? [P18]

% %You know, something as simple as having a conversation with an organization that helps people learn how to develop. They come from an underserved population, maybe they've never gone to college, or maybe they're doing a career transition, where they've been doing something for 20 or 30 years. And now they want to transition into development. Partnering with those companies to say, who's graduating out of your programs, do they fit our tech stack? We want to talk to them. Right? So creating opportunities for all sorts of developers. Can we ingest a ton of newbies and make them successful? No, not at our scale. Could we take a few? Yeah. We have developers who work in areas of the world that are you know, not quite like the United States. We are making opportunities for developers in countries that are less developed. And we're giving them significant financial freedom. Because we pay them for their contribution, not for what they should be paid in that country. [P5]

% %5:11 ¶ 189 in [P5] Jill-SlimAI interview.docx
% %And you have to have clearly defined guardrails on what behavior is unacceptable. So when someone is behaving outside of what's acceptable, you also need to take the time to pull them aside and say, This isn't aligned with our community code of conduct or this isn't aligned with our culture and how we communicate? What can I do to help you to get to where you need to be, so that you can contribute? And after you've done that, if they can't opt in, you then very clearly need to say we're not going to allow you to contribute at this time. When you're ready to come back you let us know.

% %We have developers who work in areas of the world that are you know, not quite like the United States. We are making opportunities for developers in countries that are less developed. And we're giving them significant financial freedom. Because we pay them for their contribution, not for what they should be paid in that country. [P5]

% %Again, it's the idea is not to say like we do this thing, just because people come to us, it's we want to make those ecosystems healthy, welcoming, more attractive for everyone to use and contribute. And if some of these people then come to Aiven, because they heard Aiven is actively doing those things, way better.[P7]


% %But for a very successful public open source project, that is one where you have a public community of people, you know, contributing and involved, who work for all kinds of different employers, or no employer or, you know, all over the place. [P15]




% %-------------------------------------------------------
% \cellcolor{gray!15} \textbf{Have open leadership and governance}
% &
% \cellcolor{gray!15} P1, P3, P4, P9, P10, P11, P15, P18 
% & 
% \cellcolor{gray!15}``Companies can be a little bit reluctant to contribute to open source projects that are owned by companies, because, you know, companies are fickle, we change our strategies, and maybe we stop working on the project and goes away. And or maybe we decide to exert too much control and people get uncomfortable'' [P11]
% \newline
% ``If you do the corporate coalition stuff, you generally need the project to be in a foundation from day one, right?... if you don't have a foundation, then one of the companies needs to own things. And then everyone else needs to trust that one company, which is hard'' [P15]
% \\ 

%And this is where I think, you know, foundations like the Apache Foundation, the Linux Foundation, become really good neutral homes, for companies to come together to collaborate. If five cloud companies have the same challenge, it makes sense to work together to solve that problem, and not solve it through partnerships and agreements, because that's a very heavy legal lift. But to come together in a foundation and to say, let's create a charter, that's, you know, maybe one company has started solving that problem, and then release that code to that foundation, when everybody else starts working on that, and can continue to improve that.[P18]

%4:31 ¶ 87 in [P1] Georg-Bitergia interiew.docx
%So Bitergia started the grimorelab project and built a community around the metrics. So Grimoire Lab is the metrics tool that we have. And we had Grimore con, which is event series of bringing people together who wanted to talk about metrics and using Grimoire Lab to get metrics. And and then Bitergia was looking okay, how do we, how do we grow that community more? What What can we do? And so we approached the Linux Foundation to say, hey, Linux Foundation, you're hiring us for getting the metrics, you're using the tools. You also host projects, can you host Grimore lab, as a project as a Linux Foundation project? And the Linux Foundation said, yeah, we can do that. And we also have these researchers at the University of Nebraska, who are doing metrics work, who want to understand Project Health, you have the tools Bitergia let's bring this together and create one community. And so Bitergia Research Group Linux Foundation, we all sat together decided to name this one group chaos. And so chaos is the next step from the Bitergia perspective of the community that we have built. But now, chaos has a bigger scope, than just metrics around the Grimoire Lab tool because now there's also research interest they have the research project. And it's bigger. It's bigger pie with what we are doing as part of that.

%1:17 ¶ 105 in [P3] Omer-VM interiew.docx
%Exactly, exactly. And then when you are collaborating, right, so Knative is about to be donated to the CNCS. That took a lot of work to get to happen years of lobbying and politicking and striking and it's so much stuff, because the vendors that are involved in that project already have representation in these foundations. So in their mind, they would like to put these things where they already have influence, and in a place that they can trust that other vendors won't screw them over. And the CNCF and Linux Foundations have done a pretty good job of being like, Okay, we are a neutral home for these things.

%2:40 ¶ 109 in [P4] Per-Spotify interview.docx
%So we have no involvement other than like, some, I think some people do get involved in like, [inaudible], it's very individual decision based on their needs. So Spotify do not have a strategy of influencing the projects that depend on. It happens on a per dependency basis, if, if an engineer sees like a need to raise a discussion with our team, they will do so. But no, we don't we don't have we don't have a plan to leverage ourselves as a company to to influence our dependencies. A

% it benefits you to have advisors who are also experts in that ecosystem. So we have folks like Kitt Marker and Brendan O'Leary and Kelsey Hightower and Alan Chisa, we have people in the extended ecosystem, who are advisors who spend regular time with us to help us guide the strategy and execution of the business in a way that is appropriate for the ecosystem. Right? When you behave that way, both internally and externally, you can produce more faster, because it's not just us internally doing it. The amplification is way bigger. Mm hmm. Right. And so and so if we, it's not just us talking about us, its advisors, its investors, its users, right. So it amplifies out bigger and faster.[P5]

%10:34 ¶ 160 in [P9] Phillipe NexBe Interview.docx
%And now, in selecting code, a library that we use, that will be critical dependencies for us, that will be typically be a serious criteria. If we think it's important for us to participate in the governance, and we're not allowed to, we would probably just pass and move on to something else and not use the code at all. There's example like that I wouldn't give specific name. But there's a project with which we were involved fairly significantly. And it was very clear that nobody was welcome to provide input on on the governance and directions of the project. So we passed, we disengage, and we disengage completely, basically... So there's there's one possible challenge, but it's more of a problem. That's the only challenge maybe free riding. Which is if you do free riding. Which means when you have contributors, well, users that that eventually abuse and free ride without ever contributing back.[P9]

%And that becomes a very important part of how open source modern open source grows. It's not the weekend warrior. I mean, in some cases, it is but for a lot of cases, it's not a weekend warrior, that's going to create Kubernetes it's Google, who going to pay engineers full time all day every day and feed them to make a technology and then strategically give that to a foundation to grow in public, so that more people use a technology and oh, by the way, they have a cloud providing service that you pay for that, you know, has some sort of consistency with with that technology. [P10]

% And the question is, who owns the intellectual property of the joint venture? So if I have a joint venture, some some of my people, some of you people are, are coming together to create something? Well, somebody has to own the output, it's either one entity owns it, the other entity owns it, it's owned in a joint way. Or it's given to some sort of neutral sort of neutralized status. Those are sort of your basic four outcomes. Well, you want to decide that before you put people in. But last thing you want to do is put people into the fight, and then have a company say we own it. [P10]

%But for us, one of the things that we look at as the primary benefit is really being able to build an ecosystem around the project and build more of a community around the project. Because people can be companies can be a little bit reluctant to contribute to open source projects that are owned by companies, because, you know, companies are fickle, we change our strategies, and maybe we stop working on the project and goes away. And or maybe we decide to exert too much control and people get uncomfortable.[P11]

%I mean, a lot of times that they use, so for example, if you do the corporate coalition stuff, you generally need the project to be in a foundation from day one, right? Because, because, like for a single company, project, that single company can keep paying for the project's resources, you know, and hosting things, etc. And gradually, you know, gradually alienate those items. Right, so gradually, you know, give them up to some sort of public resource or, or you know, come up with a way to have them sponsored or put it in a foundation or whatever, right? There's a flexible timeline around that with one of these corporate coalition's you pretty much need to do it immediately. Because if the CO if you don't have a foundation, then one of the companies needs to own things. And then everyone else needs to trust that one company, which is hard.[P15]
  
\bottomrule
\end{tabular}
\label{table:sportsmanshipTable}
\vspace{-2.5mm}
\end{table*}


%[other quotes]
%o I think like having, using the policies, like having the right policies and guidelines in their project description, like, you know, like the contributing Markdown file, like how to contribute to the project, also all the policies in place in terms of so they like the sign, like, like the license, for instance, like having a document of which license it is because companies won't be contributing to your project, if there is no license sewing, even though you have it on a GitHub or an anti lab doesn't look doesn't mean that it's open source. So yeah, in overall, it's more like having all these documentation documented in the right place. So I make it easier for companies or individuals as well, to be able to take a look at that. Because they might be like also having like automation, automating tools to review all this. So used to be making sure that you're using the right syntax, and you're putting it in the right place. Because if not, it might be like your open source project. Maybe that just because it's not in the right place. And they're not going probably in my project review. And it's they have their own automation tools. [P19]

%[P18] That's a really, really good questions. One of the reasons why sometimes companies don't contribute is because the project makes it so difficult to contribute either. The, there's no proper guide to contribution, there's no coding guide, there's no readme. A there's no proper license. And so the company has to work extremely hard to find out, you know, how do I come to contribute to this. And that could be bias in the project, also about accepting contributions from certain companies or, or to protect, you know, their own direction, or they may see companies as pushing the company agenda and not thinking about the project in general. So projects also need to make it easy for not just company but everyone to contribute, make it a welcoming community, have contribution guidelines, review issues, give proper feedback, of code of conduct, all of those kinds of things. If it's an abusive community, even companies don't want to get involved. And the other is, is just, you know, companies look at is this a healthy community? Is there a proper, you know, cadence to the innovation? Is it maintained by one person, you know, just one person somewhere who randomly will make things happen? And, I mean, companies can do a lot more here as well, if it's really a valuable project, and it's struggling for maintenance. Companies should help, you know, with the maintenance and not just the feature functions, but the heavy lifting of code reviews and maintaining the project and doing the security and the plumbing. And I would say those are some of the things that projects could do. [P18]
