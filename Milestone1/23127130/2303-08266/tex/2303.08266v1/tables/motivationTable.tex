%yes, we have some. So of course, you can claim to be an expert, if you contribute regularly to a project, yes, kind of probably. So that's one way of also getting this kind of trust, building a reputation in that sense. 

% 3:17 ¶ 73 in [P2] Daniel-Bitergia interview-D.docx
%we need to use other marketing efforts to be known and to be to be seen out there being being an open source company, I think, helps build like trustable branding.
% Please add the following required packages to your document preamble:
% \usepackage{multirow}
\begin{table*}[htb]
%\centering
%\scriptsize
\caption{Company motivations to contribute to OSS}
\label{tab:motivationTable}
% Please add the following required packages to your document preamble:
% \usepackage{multirow}
\resizebox{\textwidth}{!}{
\begin{tabular}{llll}
\hline
  \textbf{\begin{tabular}[c]{@{}c@{}} Category \end{tabular}} &
  \textbf{Subcategory} &
  \textbf{Participants} &
  \textbf{Exemplary Quotes} 
  \\\hline
  
  
  
   %____________________Founder's Ideology_________________________
 
  \multicolumn{2}{l}{\multirow{1}{*}{\begin{tabular}[l]{@{}l@{}} Founder(s)' Vision  \end{tabular}}}

%---------------------Founder(s)' ideology-----------------------  
   
  \begin{tabular}[l]{@{}l@{}} \end{tabular} 
  
  &

   \begin{tabular}[l]{@{}l@{}} P1, P2, P5, P7,\\ P9, P12 \end{tabular} 

&
 
 \begin{tabular}[l]{@{}l@{}} ``Why open source? One: it's ideological. The founders just believed that was the right thing to do'' (P1)
  \\
 % ``The four founders basically were avid contributors of open source, and they decided that's, that should be our DNA'' (P7)
    ``The reason why we started [company name] was to create an open source alternatives to large enterprise'' (P9)
  
  \\
  
%``[founder's name]'s vision and intent for the project in the first place ''[P5]

   \end{tabular} \\ 
  
%----------- Problem is open source in its essence----------
  % &
  %\begin{tabular}[l]{@{}l@{}} Problem is open source in it's essence \end{tabular} 
 % &
  
   %\begin{tabular}[l]{@{}l@{}}\\ P12, P7, P9 \end{tabular} 
  % & 
    
  %\begin{tabular}[l]{@{}l@{}}

   %\\ 
   %``so <company name> was built on top of open source. So without open source <company name> wouldn't exist ''[P7]
  % \end{tabular} \\ 
  \hline
%____________________Reputation_________________________
  \multirow{12}{*}{Reputation} &
  
%-----------VISIBILITY-------------
  \cellcolor{gray!15}\begin{tabular}[l]{@{}l@{}} Visibility \end{tabular} &
  
  \cellcolor{gray!15}\begin{tabular}[l]{@{}l@{}} P1-P9, P11, \\ P13-P18 \end{tabular} &
  
  \cellcolor{gray!15}\begin{tabular}[l]{@{}l@{}} ``Where open source is superior is it gives us superior visibility'' (P15)\\ 
  
  %``So it's clear that brings visibility and bragging rights'' (P9)
  
 % ``We decided to do is write a blog post... Well, a bunch of people started commenting on that. And it raised awareness of [company \\name] because we wrote it, but we didn't mention [company name], really, in that entire article... so it's actually a great marketing \\tool to talk about technology to use, even if it doesn't promote your, your own product'' [P8] \\
%  ``Having companies involved can also help you with... marketing, right?... that helps give visibility to some of these these projects as\\ well'' [P11] \\
  %``open sourcing these bits and pieces of whenever you can make it like tangible and like very transparent about \\ what was actually going on inside of the company. So that can that can help the deal of branding [P4]''
  ``If you actually came up with this standard, then you also get a lot of exposure'' (P4) %... And is again it kind of is like our desire to be able \\ to be a standardized piece of technology'' (P4)
 \end{tabular} \\
 
 %It could also be a business that decides to release a project because they want to create a standard in the world around that particular software, like, you know, Google did with Kubernetes, it became the de facto way to do container orchestration, as and so many other projects have become kind of the de facto standard for doing something. And the company that created it and released it has an advantage, because they are seen as thought leaders, people want to join them from a recruitment perspective.[P18]
 
%And, and where open source is superior is it gives us superior visibility into our own supply chain. Right as in, we don't have to-- with proprietary software, you know, if we have a proprietary component, like we were proprietary, authentication component, and we're like, hey, is this being maintained in the light of new security exploits? Right? In that case, you have to trust the vendor to know whether or not that's being maintained. [P15]

%So it's clear that brings visibility and bragging rights. So for instance, I can I usually use when I make presentation at conferences as a joke as an intro, the fact that I've, I've had my signup, ID to one of the largest deletion of code in the Linux kernel. And that's true. So identify the line of code in there. But ideally, you'd like 1000s and 1000s of lines of code, which were actually lines of Commons license. You know, doing doing this on a massive scale on 10s of 10s. Of 1000s. of files. That's a skill.[P9]

%[showcasing technological achievements] 
  %4:37 ¶ 115 in [P1] Georg-Bitergia interiew.docx
%And we are active in these communities showing, showing off our skills and experience? They're saying, Oh, yes, we want more of that. Can you dig in more? Show us more what, and we basically use these communities to showcase what we can do. And whether our potential customers are themselves active or just observing doesn't matter. It creates that contact to then have a conversation.

%4:11 ¶ 27 in [P1] Georg-Bitergia interiew.docx
%hey hire us for the expertise. Because we have worked with a lot of open source projects, and foundations and companies that we are really good at knowing what the data is, what is available, what kind of questions we can answer.

%3:18 ¶ 73 in [P2] Daniel-Bitergia interview.docx
%contributing to open source is about demonstrating that you have the expertise and the knowledge contributing, and then you can serve specific pieces of knowledge.

%3:21 ¶ 73 in [P2] Daniel-Bitergia interview.docx
 %So then we can mess with your inner source effectiveness, let's say. But then at the same time, we are serving with the community. So we are defining inner source. So it's not that we know about inner source is that we are defining inner source and for this you can see that we have this pattern here about the maturity model, this pattern there about this or that and then we have these presentations at the inner source commons.

%2:101 ¶ 23 in [P4] Per-Spotify interview.docx
%So like open sourcing these bits and pieces of whenever you can make it like tangible and like very transparent about what was actually going on inside of the company. So that can that can help the deal of branding of the technical ideas of an app like Spotify, I think. And Spotify saw a lot of good press with like this, Spotify wrapped the day over the Christmas period where everyone got like the year overview of music. And that kind of thing took like a ton of machine learning a ton of data and a lot of engineering tricks, to do this little little feature. And so this is like this is one of the ways they want to be more transparent about like, how do we actually use these, like very sophisticated systems of understanding music and user behavior and, and music history and so on. So we can kind of create this little feature called wrapped, which is just one month a year. 

%8:72 ¶ 71 in [P7] Josep-Aiven interview.docx
%Because they were asking if we have some experts on X, Y, Z? And then it's like, yes, we have some. So of course, you can claim to be an expert, if you contribute regularly to a project, yes, kind of probably.

%8:125 ¶ 135 in [P7] Josep-Aiven interview.docx
%write articles on presence while trying to showcase what we do share the Word of either what we do at Aiven, what we do in the open source space, how we shape things. 

%9:9 ¶ 16 in [P11] Dawn-VM Interview.docx
%so people tend to think of VMware as like the people who did virtual machines. And we've really moved beyond that. And so we-- well our focus right now is really on building platforms and application developer tool sets that people can use on top of those platforms to build their applications. So we do that across all the clouds

%[be seen as the founder of a technology that is now industry standard]
%2:51 ¶ 37 in [P4] Per-Spotify interview.docx
%And if you actually came up with this standard, then you also get a lot of exposure. Like Google got a lot of exposure from like, creating Kubernetes, which is like the industry standard for like management services. Now again, and Spotify wants to kind of create the same kind of narrative for like internal



%``if you actually came up with this standard, then you also get a lot of exposure... And is again it kind of is like our desire to be able to be a standardized piece of technology''[P4]

%2:90 ¶ 97 in [P4] Per-Spotify interview.docx
%h, yeah. And is again it kind of is like our desire to be able to be a standardized piece of technology instead of just an esoteric Spotify technology

%17:11 ¶ 88 in [P14] Harish RedHat Interview.docx
%And then the Linux kernel community, eventually settled on the model that we had also adopted. We already had adopted. And so guess what happens? The SUSE guys had to scramble, because now they have a kernel that is only using that type of threading which the upstream is not accepting. They say, yeah, it doesn't make sense. And so they had to scramble and I know of customers who were abandoned by SUSE because they were already going down that path, which was not the right path to go and SUSE couldn't quickly, you know, re engineer the stuff so that they can so they have to sunset that portion of the tech so that they can go into the newer threading model in the kernel. So. So that's the kind of benefit that we get as an example. So benefit, right?

%18:37 ¶ 298 – 302 in [P13] Tobie OSS consultant Interview.docx
%Yeah, it's a lot of money. Yeah. Um, and so that was just essentially by commoditizing. The layer under what it is that they were serving
%[R] So this is Facebook being able to save that amount by by standardizing
%[P13] by essentially open sourcing, sort of like the structure and diagrams of how they were making. 

%18:38 ¶ 296 in [P13] Tobie OSS consultant Interview.docx
%But let's say that they realize they could make a lot more money selling hardware than, than selling their software. Right. They could make the whole zoom thing, completely open source, for example. And make it work extremely well with their hardware. So that Sure, they would lose money and have like, fewer people actually paying for it, but they would become the de facto solution for all hardware, built on for, you know, whether like, regardless of who was using that system or not. 

%1:13 ¶ 187 in [P3] Omer-VM interiew.docx
 %the sort of street cred of the founders of React,

%1:19 ¶ 187 in [P3] Omer-VM interiew.docx
%get an entire class of engineers, who are now familiar with the tech stack that Facebook uses, right? Or at the very least, like one of the main web frameworks that they're using, that they have sort of home grown, right. And they also get viewed as, like a cooler place to work. 

%2:58 ¶ 37 in [P4] Per-Spotify interview.docx
%They, they want to leverage open source as much as possible. And the best way to do this is to make sure that what you are working on inside of your company, and the thing you depend on actually becomes the industry standard, not just something you maintain. Because if it becomes an industry standard, then you gain all the benefits from being part of the standards. And if you actually came up with this standard, then you also get a lot of exposure. Like Google got a lot of exposure from like, creating Kubernetes, which is like the industry standard for like management services. Now again, and Spotify wants to kind of create the same kind of narrative for like internal 

%12:16 ¶ 130 in [P15] JoshB RedHat Interview.docx
%But you know, have partners and customers that are using it. Yeah. And you don't want to piss them off. By terminating the tool? Well, if it's open source, you can just kind of gradually pull all your engineers off of it, and let them have it. Um, you know, so this is what's called the abandonware strategy. Now, that that's generally used as a derogatory term, but sometimes abandonware is good, right? Because it's something we can do with open source that you could never do with proprietary software, where you say, Hey, I know you guys are still using this. I know you still like it, but it doesn't make any sense for us as a business anymore. So here, it's yours. Now you can have it.


%[visibility/positioning in community]

  %\makecell{Some really \\ longer text}
  %the CEO of [large OSS foundation] announced in a keynote [city and name of conference]. We are now Running [OSS community]. And all of these companies and universities are part of this [company name] was there. So that was like great visibility, a really good milestone for us[P2]
  
  %You know, the other benefit that sometimes people see is you know, having having companies involved can also help you with I don't know, like, I guess marketing, right? So so we have, you know, we have relatively popular blogs and social media channels, and we can promote things about some of the open source projects that we're working on. And that that helps give visibility to some of these these projects as well. So I guess visibility is what I was talking about sovereignly I guess I guess it's marketing.[P11]
  
  %7:22 ¶ 47 in [P8] Johnathan-Goliath interview.docx
%And so we decided to do is write a blog post saying, here's all the things here's how you might want to use JSON, the JSON library in Zephyr, and oh, here's some gotchas we found. Well, a bunch of people started commenting on that. And it raised awareness of Goliath because we wrote it, but we didn't mention Goliath, really, in that entire article. And the maintainers of the Zephyr project, want to see how we can start to contribute back. So that material, so it's actually a great marketing tool to talk about technology to use, even if it doesn't promote your, your own product. And so we're seeing that by by proxy of doing it. 

  
  %So we have a branding, we have collateral point of bringing more people hire more people keep more people here. Get more engaged with more companies that might be the future partners. For example, last year, we had several conversation with Red Hat, who is a huge company about open source. And yes, he has a lot of money. And because of this conference discussion, right now, we had strategic discussion to do more business together. And everybody was everything was based in the first open source commit and open source community discussion. So yes, there is no money directly on those open source products, but it's your work to keep it that's why Zup is zookeeping it so with branding, it's kind of like that.[P6]
  
  %So two reasons. One is personal. And one is business, the personal reason is that we whoever is engaged in in these communities may have a personal interest wants to be wants to learn something new wants to be recognized as, as a leader in that space. So it's a little bit like personal branding. So one, for example, I'm involved in the IEEE sa open community advisory group. And part of that is to build up my credentials as an open source strategist and being an expert on open source communities. For Bitergia, as a company to encourage this and do this, is because being active in open source communities, gives us access to decision makers, at companies that could potentially be customers. And so we use this also as a way for finding prospects and generating business for the company. And most of the business that we do have came through contacts that we fostered in being active in these communities.[P1]
  
  %Definitely, I suppose in the marketing and branding level, that the marketing and branding folks always have, like, how is this contributing to our-- I don't work necessarily at that level. But I think that the level we do work at is we want to show up as in some of the ways I was talking about like humble, sincere, dedicated, more than a lot of time brand is about being like when I hear marketing and brand I hear, "we're the best at this look at us" you know, and I don't think that's what we're going for at all. I mean, occasionally, occasionally, somebody might be proud of something, but we're really looking to be part of something. Our brand is being part of something and not the best at something, [P17]
  
  %So we need to use other marketing efforts to be known and to be to be seen out there being being an open source company, I think, helps build like trustable branding.[P2]
  
 %12:3 ¶ 26 in [P15] JoshB RedHat Interview.docx
%The the second way that we're involved with with open source is community. As in open source projects, collect people around them, which are known as open source communities. And open source communities are a good way for us to reach people. It's a major portion of our marketing picture, when I'm talking about marketing with a big M. The and and they supply a lot of sort of key inputs to marketing, you know, both not just, you know, people to basically advertise to, but also as a way of learning what is it that people want, so that we can actually build those products? Right. Um, so marketing in terms of market research, and in terms of getting direct feedback? You know, like, if we're thinking about something new in Linux, we can put it in fedora and see how it does in fedora. Um, you know, before we have to offer five years of support for it.

  
  %7:13 ¶ 31 in [P8] Johnathan-Goliath interview.docx
%It's more about being able to have our voice heard, emphasis on priorities for our company, but also to drive the general shift. Like, because I participate in the marketing side, I'm able to make the marketing efforts better, like, "oh, maybe we should go talk to these people. Or maybe we should have this change the conference upcoming." And so I wouldn't have that opportunity if I wasn't a project member.

  
  %6:47 ¶ 159 in [P6] Otavio-Zup interview.docx
%We have several views in different countries for up to the company. So right now we have I don't know the number but 20,000 views around the globe. We have several countries don't have the right number right now.

%6:73 ¶ 159 in [P6] Otavio-Zup interview.docx
%And how can we engage money from the open source products basically first with brand thanks to the four open source products.

%2:35 ¶ 19 in [P4] Per-Spotify interview.docx
%And there is like, one aspect, just like branding, like that a company, what they do, again, what they do inside is not very visible on the tech side,

%2:56 ¶ 19 in [P4] Per-Spotify interview.docx
%just like branding, like that a company, what they do, again, what they do inside is not very visible on the tech side, like a like an app like Spotify, it's actually quite hard to understand, like, how it actually works, technically. And why should that even be interesting? For many doing point of view, if you're like a top notch engineer at Google, or Facebook or Apple, why should you work at Spotify? It's just like an app that plays music. 

%2:51 ¶ 37 in [P4] Per-Spotify interview.docx
%And if you actually came up with this standard, then you also get a lot of exposure. Like Google got a lot of exposure from like, creating Kubernetes, which is like the industry standard for like management services. Now again, and Spotify wants to kind of create the same kind of narrative for like internal

 %3:17 ¶ 73 in [P2] Daniel-Bitergia interview.docx
%we need to use other marketing efforts to be known and to be to be seen out there being being an open source company, I think, helps build like trustable branding.

 %5:74 ¶ 213 in [P5] Jill-SlimAI interview.docx
%Could branding make it easier? Sure. I mean, it might make it more easily recognizable. Yeah. Are there open source projects on market today that don't look anything like their corporate sponsors? Yes. Are there some that do? Yes.

  
 %3:19 ¶ 73 in [P2] Daniel-Bitergia interview.docx
%Yeah, so So the, the open source communities we are part of, we can see them as a way to position branding, to partially use their marketing efforts into reaching out other people that they need, for instance, metrics, or open source metrics, or they need to measure PlayerHealth. And then they go to, to grimoire lab-- to CHAOSS, and they say, well, there is there is a university running or so we I can I can hire them or I can hire, Bitergia for more, you know, professional services or industrial services. Good. So this is like a good place where you can go and fish for potential customers. It

%17:25 ¶ 16 in [P14] Harish RedHat Interview.docx
%So to answer the question is, what is in it for Red Hat? The brand recognition, oh, the guys in Red Hat, they advised us to try to do this. Yeah, they did help us out, we did not pay them all, we paid them something. Oh, do we didn't buy the subscription from them. All, we bought some subscription for them, because we were glad that they helped us, you know, so the leaves hopefully leaves a positive, a positive impression with the organization. Because they have a lot of them, they immediately start comparing us against, for example, I'm picking the name because a very common name, Oracle, or Microsoft, or they just want to, they just come here to make money. They want my money, they want to do this, No, they won't give me anything else. But when they come here, they also want my money, but they also give me other things. And it's okay, even if I don't, even if I don't subscribe to their services, even if I don't buy their service, they're still friendly to me. Whereas the other guys, Oh, you didn't buy from me, I'm walking away. So it's a very different, it's a very human thing. I mean, it's not a it's not a win, win lose environment, it's a win win environment. So you know, in order for the other side to say, oh, I need to win this customer, that means the customers lose the money to me.

%19:19 ¶ 65 in [P16] Stephen Transcribed Interview.docx
%doing good, looks good. And looking good, has, you know, has a potential positive effect on your bottom line? Right. So I think, I think that's definitely one component of it for for enterprises. Participating in open source, like, it looks good to be doing good. 

%19:20 ¶ 65 in [P16] Stephen Transcribed Interview.docx
%making sure that we're, we try to be equitable with the way that we've maintained some of these things within, within smaller parts of various foundations. But also realizing that there's a component of it, that you have to show people why they should care about you.

%-----------Verifiable Trust-------------  
  &
  \begin{tabular}[l]{@{}l@{}} Building\\verifiable\\trust \end{tabular} &
  
  \begin{tabular}[l]{@{}l@{}} P1, P2, P5-P7,\\ P15, P17, P18 \end{tabular} &
  
  \begin{tabular}[l]{@{}l@{}}
  ``If we have a proprietary component...you have to trust the vendor to know whether or not that's being maintained...\\Whereas if it's an open source project,  I can watch the contribution history and I can know whether or not it's being maintained'' (P15)\\
  
  ``The biggest benefit from contributing is building trust'' (P18) %, and from trust comes influence. So you can actually influence the direction of a\\ project, especially if you're very dependent upon the project
   
  
  %``Because they were asking if we have some experts on X, Y, Z? \\ And then it's like, yes, we have some. So of course, you can claim to be an expert, if you contribute regularly to a project'' (P7)\\
  \end{tabular} \\
  
   %And it's the basis of your service or product. If you are not able to influence the direction, then you're left constantly reacting to the project's changed direction, so you need to be at the table.[P18]
  
  %And there's such a wide variety of components. And these components have become standardized, because everybody's using it. So developers start trusting if if, you know, all of these companies are using it, it must be a good component, instead of building it themselves. [P18]
  
  %4:11 ¶ 27 in [P1] Georg-Bitergia interiew-D.docx
%hire us for the expertise. Because we have worked with a lot of open source projects, and foundations and companies that we are really good at knowing what the data is, what is available, what kind of questions we can answer.

%And right now, you can think the open sourcing as a marketing budget to the company. So for example, it's [recording?] but then our proposal, but I can say that we have Vivo company, who decided to join us because they used an open source product. So the open source product was the first step to hire our service. It's almost the same that Red Hat does. But instead of be around these products, we are offering people outsourcing slash consulting service.[P6]

%3:18 ¶ 73 in [P2] Daniel-Bitergia interview-D.docx
%contributing to open source is about demonstrating that you have the expertise and the knowledge contributing, and then you can serve specific pieces of knowledge. 

%3:21 ¶ 73 in [P2] Daniel-Bitergia interview-D.docx
%  we are serving with the community. So we are defining [a concept]. So it's not that we know about [name of the concept] is that we are defining [name of concept] and for this you can see that we have this pattern here about the maturity model, this pattern there about this or that and then we have these presentations at [conference name]
  
 % 8:72 ¶ 71 in [P7] Josep-Aiven interview - D.docx
%Because they were asking if we have some experts on X, Y, Z? And then it's like, yes, we have some. So of course, you can claim to be an expert, if you contribute regularly to a project



%4:37 ¶ 115 in [P1] Georg-Bitergia interiew-D.docx
%And we are active in these communities showing, showing off our skills and experience? And whether our potential customers are themselves active or just observing doesn't matter. It creates that contact to then have a conversation.

%Right. And, and where open source is superior is it gives us superior visibility into our own supply chain. Right as in, we don't have to-- with proprietary software, you know, if we have a proprietary component, like we were proprietary, authentication component, and we're like, hey, is this being maintained in the light of new security exploits? Right? In that case, you have to trust the vendor to know whether or not that's being maintained. You can't tell except your faith in that vendor. Whereas if it's an open source project, I can watch the contribution history and I can know whether or not it's being maintained. Now, on an aggregate basis, it's hard to keep track of all that information, but it's at least possible with proprietary software. You just have to guess.[P15]


%So our users trust us because we're giving them open source value... Also, because of that trust, and the way that you treat them, they are more likely to contribute feedback that is timely and relevant, which will further enhance your ability to produce good products. [P5]

%5:10 ¶ 137 in [P5] Jill-SlimAI interview.docx
%And they are more likely to stay with you because they are happier because of that trust. 

%5:49 ¶ 65 in [P5] Jill-SlimAI interview.docx
%capture the attention of developers who are curious, developers who like to solve problems, developers who like to engage in communities to work with other developers who have similar problems. Those are the types of people who will continue to benefit from the additional features that we build into our SAS.

%I mentioned community. I think that accountability to communities is also really important. It helps us make better decisions... There's like one example I think of, Microsoft wanting to put telemetry in the repo and the community was like, "No way!" And so they're like, "Okay." So you know, like that. But, you know, the act of that decision really helped build trust. So I think that the thing that Microsoft benefits from in working with open source--I'm trying to--Sorry-- I'm trying to mold this or nudge this out of my brain-- Is that, you know, that that trust piece, I mean, if you act with you know, if you show that you're, you know, as a company that you're willing to, to sidestep what you think is best and make sure that decision makings include the community, then you might have opportunity to go places and build trust you might not otherwise have. And that builds trust in the product, it builds trust in the company and Microsoft has a slogan of being built on trust. So I think that directly aligns with just generally, as a company, trust is important for us, in our customer, our customers feel about the work that we do. So the community is a great extension or example of that with communities. [P17]


%-----------Networking-------------  
  &
  \cellcolor{gray!15}\begin{tabular}[l]{@{}l@{}} Networking \end{tabular} &
  
  \cellcolor{gray!15}\begin{tabular}[l]{@{}l@{}} P1, P2, P5-P8 \\P13 \end{tabular} &
  
  \cellcolor{gray!15}\begin{tabular}[l]{@{}l@{}}
  %``We've had two three customers... the biggest ones we've  had... And they decided to go with us, not that much because of the open\\ source product, but... the open source knowledge that we are there we are positioned there. So then we should hire these people\\ because they know'' [P2]\\
  ``Puts us in a position where we are we are surrounded by other principle aligned practitioners'' (P5)\\ 
  
  %``we try to contact as many people as we knew, and then asking them, \\you know to introduce to ask new people and so on. 
  ``So we were kind of trying to nurture our network of people and learning from them'' (P2)

   \end{tabular} \\
%[access to big name companies]
  %And because of that, we have more contacts with the companies inside CNCF such us Apple, Microsoft, Red Hat again, and more companies so we engage our brand to be related with huge companies around the globe. [P6]
  
  %So this is like a good place where you can go and fish for potential customers. It is it is much more clear. In the case of the inner source commons, where there are corporations that are joining the foundation. And they say, well, I need someone, I need help with inner source I need you know, and because of this, then this is another good place where we can say hey, I don't know we we can help you understanding the maturity of your inner source journey. [P2]
  
%[surround with peer companies]
%5:66 ¶ 81 in [P5] Jill-SlimAI interview.docx
%events, puts us in a position where we are we are surrounded by other principle aligned practitioners. They could be other open source projects, they could be students, they could be other companies that have the same type of philosophies that we do about open source and cloud native. And so there are there are programs like, you know, being part of the Linux Foundation. You know, when when our when DockerSlim is running, it observes a container at the kernel. Right? Well, that kernel wouldn't exist without Linux.

%[opportunity to get hired]
%And basically, we try to contact as many people as we knew, and then asking them, you know, to, to introduce to ask new people and so on. So we were kind of trying to nurture our network of people and learning from them. And that point in time, we had, we started having some some others, because red hat was one of the very first customers as well, just because we had the OpenStack foundation as a customer.[P2]

%[networking]
%gives us access to decision makers, at companies that could potentially be customers. And so we use this also as a way for finding prospects and generating business for the company. And most of the business that we do have came through contacts that we fostered in being active in these communities.[P1]

%so we focus a lot of our, we'll call it marketing dollars, we focus a lot of our growth dollars on participating in activities and events, that put us in that same ecosystem as these other like minded organizations and practitioners. [P5]

%And if you think about a Brazilian, there's a third world third world country have our brand related to this company is a nice good step to us especially because okay, you will become more related with these companies that mean more clients coming[P6]

%One is the network that you've built. Um, and that, you know, we've also talked about in terms of like, how actually, this, you know, concretely helps you, in situation like, Hey, you wanna solve a problem? Will you actually know who to talk to?[P13]

 %So this is like a good place where you can go and fish for potential customers. It is it is much more clear. In the case of the inner source commons, where there are corporations that are joining the foundation. And they say, well, I need someone, I need help with inner source I need you know, and because of this, then this is another good place where we can say hey, I don't know we we can help you understanding the maturity of your inner source journey.[P2]
 
 %We've had two three customers large, the biggest ones we've had, that are proprietary ones. And they decided to go with us, not that much because of the open source product, but because we were part of the inner source commons. Yeah, the open source knowledge that we are there we are positioned there. So then we should hire these people because they know about the inner source in this case.[P2]
 
 %it's nice to be close and around Google, Red Hat, Apple, Microsoft, Spotify, and so on, and be related with those brands. It is nice to client as well. Right? So okay, you want to hire our service? Take a look at where we are. We are. So that is Google. That is apple. There's Microsoft, that's us. So did you see any competitors of us over there? No. So yes, it's off money if you hire us instead of any competitors. [P6]

%increase your brands, and be around people, because also have our business products to bring more people to pay attention us. it's nice to be close and around Google, Red Hat, Apple, Microsoft, Spotify, and so on, and be related with those brands. It is nice to client as well. Right? So okay, you want to hire our service? Take a look at where we are. We are. So that is Google. That is apple. There's Microsoft, that's us. So did you see any competitors of us over there? No. So yes, it's off money if you hire us instead of any competitors. [P6]



%------------Attract Talent-----------------
   &
  \begin{tabular}[l]{@{}l@{}} Attracting\\talent \end{tabular} &
  
  \begin{tabular}[l]{@{}l@{}} P3-P6, P8-P10, \\P13, P15 \end{tabular} &
  
  \begin{tabular}[l]{@{}l@{}}
  
  ``Open sourcing is one of the ways that that can make it much more interesting for outside candidates'' (P4) \\
  
  %``A lot of people decided to come here, be hired here, because the open source products'' (P6)\\ 
  
  ``It's good for recruitment...if my engineering tool is open source, and I need to hire new people, for the engineering team, \\the first group of people I look at is people who've made random contributions to that tool'' (P15)\end{tabular}\\
  
  
 % ``It's not about market share. It's about knowledge share, right? Like how quickly do developers want to reach for your tooling. Right?\\ Because that's a hiring tool. That's a hiring tool, right?'' [P3]\\
  %So this mindset also, cause many companies that didn't go open source to lose talent and don't retain talent. So this works also as a way to retain talent, like, Hey, we are not just a boring company in the financial systems, I don't know, for instance, like, we are a tech company, and we are promoting open source and we really want you to create things and linear space to do so. Right? And also to like your Yeah, like your personal brand gets in the pilot for for your future career.[P19]
  
  % Because you're like, you get to like use this open source library, which helps you which means you can get jobs at other places, right? Like you can like you can leave Facebook with React experience and go work again, or go work at google or Go work, you know, wherever you want to go. So that's part of it, too.[P3]
  
  %``because a lot of people decided to come here be hired here, because the open source products, ... a lot of people who is coming only because [company name] has open source products [P6]''

  
  %capture the attention of developers who are curious, developers who like to solve problems, developers who like to engage in communities to work with other developers who have similar problems. Those are the types of people who will continue to benefit from the additional features that we build into our SAS. [P5]
  
  % And so so open sourcing is one of the ways that that can make it much more interesting for for outside candidates that they, they want in work on something so interesting. And the other part is like if the system is interesting, they might be able to work on anything in the open on like an open source project that has a lot of traction in the world.[P4]
  
  %7:23 ¶ 59 in [P8] Johnathan-Goliath interview.docx
    %There's also another thing which we haven't been successful yet. But I've seen this is a different form of marketing, which is recruiting. I know a lot of developer product companies that use open source, they talk about all the stuff they use and how they use it. And when other people see that article or that blog post or whatever, like oh, that's that's cool technology I'm interested in, they seem like nice people. Maybe that'd be a cool place for me to work too and so I think that's an effective way for open source, commercial source companies to to create that kind of content for recruiting.
    
%10:45 ¶ 196 in [P9] Phillipe NexBe Interview.docx
%This is also a way for us to hire some of the very best in the field. It's difficult because we we're not Google or Facebook, and we don't have the ability to pay the same command station that launched and many, many tech companies would be able to pay and offer the set of benefits not even about pay, but just your whole set of benefits, without fight against. Knowing that we are bootstrapped and autonomous and self funded. But on the other hand, we have, we have a lot of other interesting 

%10:46 ¶ 200 in [P9] Phillipe NexBe Interview.docx
%but it happens. And that means that it's a very efficient way for us to attract and retain talent that couldn't be even able to do given our limited resources and size otherwise.

%11:25 ¶ 8 in [P10] Gil Interview.docx
%do I attract engineers? How do I retain engineers? Because they are working on open source projects they love to work in. You know, how do I appeal to their best nature to make them better engineers? 

%11:27 ¶ 11 in [P10] Gil Interview.docx
%I can hire somebody who already knows it. I don't have to pay for them

%[fostering developer love]
%1:26 ¶ 187 in [P3] Omer-VM interiew.docx
%You just you need-- developer Love is the currency of open source. 

%5:86 ¶ 99 in [P5] Jill-SlimAI interview.docx
%l moments where a developer finds our tools, loves them, tweets about them. We say thank you for sharing

%[bypassing recruitment/training steps]
%2:33 ¶ 89 in [P4] Per-Spotify interview.docx
%other thing is if if the thing that you're used to inside of the company actually become common knowledge of [non DDS?], then also you'd see your onboarding time of engineers go down. So you can see like, again, go back to the classic example, then Kubernetes, is now an industry standard. And so inside of Google, for instance, as I go, why using Kubernetes, and for them, it's like a big advantage that the industry is now just knows it as a standard thing. It would be much harder, if people had to go to Google and then learn Kubernetes from the bottom up, it would take so long to actually onboard people because it's but because it's actually a standard for how you build micro services, then any engineer that Google hire now will know Kubernetes. So for us it is the same is that we cannot do the same scale, because we know it's big but we do want to expose these things, and rather focus on on promoting them as a as one of the best ways of doing a thing instead of keeping it internal.

%11:29 ¶ 11 in [P10] Gil Interview.docx
%We, it's our own secret way of moving messages in a message bus. Like, why would I want I want to create a secret way of leaving messages in the message bus and then spend six weeks every time I hire somebody to teach them our secret way? And why would they want to learn our secret ways? It's completely useless when they leave.


%[hiring tool]
%It's not about market share. It's about knowledge share, right? Like how quickly do developers want to reach for your tooling. Right? Because that's a hiring tool. That's a hiring tool, right? You--React is a Facebook technology. We're using React to GitHub. Now, if they want to hire one of our engineers, they already know their tech stack. Right? Or they know at least like a familiarity enough with it, right? Like,[P3]

%2:28 ¶ 15 in [P4] Per-Spotify interview.docx
%Like a lot of internal teams inside of a big company do fairly interesting things, but no one knows about. And so open source is a good way of kind of getting the teammate name out in the world, and thereby also become a bit easier to actually hire interesting talent.

%2:30 ¶ 23 in [P4] Per-Spotify interview.docx
%so so open sourcing is one of the ways that that can make it much more interesting for for outside candidates that they, they want in work on something so interesting. And the other part is like if the system is interesting, they might be able to work on anything in the open on like an open source project that has a lot of traction in the world.

%2:56 ¶ 19 in [P4] Per-Spotify interview.docx
 %just like branding, like that a company, what they do, again, what they do inside is not very visible on the tech side, like a like an app like Spotify, it's actually quite hard to understand, like, how it actually works, technically. And why should that even be interesting? For many doing point of view, if you're like a top notch engineer at Google, or Facebook or Apple, why should you work at Spotify? It's just like an app that plays music. 

%6:12 ¶ 209 in [P6] Otavio-Zup interview.docx
 %the hiring process is a good way to use the open source products as well, because I can see the contribution the PR, if without do the [whiteboards boring?] that thing that everybody hates. So it's a crazy process to, to the interviewer to the interviewer. So it's bad. It's a win win situation with the open source.

%6:74 ¶ 159 in [P6] Otavio-Zup interview.docx
%The second one is to receive more goods, and high arts outstanding people, especially because a lot of people decided to come here be hired here, because the open source products, not everybody, we work in open source products, but a lot of people who is coming only because Zup has open source products and that's nice to us

%10:47 ¶ 196 – 200 in [P9] Phillipe NexBe Interview.docx
%Knowing that we are bootstrapped and autonomous and self funded. But on the other hand, we have, we have a lot of other interesting things in the work we do. The involvement we have with the open source community at large. All that makes us fairly niche. The fact that we involve mentoring else, committees of students, not by aspiring contributors to know about us, even though we're very tiny. 

%12:15 ¶ 118 in [P15] JoshB RedHat Interview.docx
%Number two, it's good PR, right. The number three, it's good for recruitment. And that's something I haven't talked about, you know, also within the Red Hat stuff, right, which is, hey, if my cast engineering tool is open source, and I need to hire new people, for the cast engineering team, the first group of people I look at is people who've made random contributions to that tool, because I already know they can write code to work on the tool, right? I already know that they are at least semi qualified. The plus it works the other way. Right? It works as a, hey, if I get a job at this company, I can work on this cool cast engineering tool.

%18:8 ¶ 310 – 312 in [P13] Tobie OSS consultant Interview.docx
%I mean, I think it does, or, frankly, companies that are good at open source, attract by like, attract, but I mean, it's, I mean, all of this works together, right? It's kind of like if you have a I mean, yeah, if you have a more more attractive set of people, and you're working in your organization working on exciting stuff. And that stuff isn't in the open, we can show show it off to like, potential other employees. I mean, that's attractive.


%-----------Adoption-------------
   &
  \cellcolor{gray!15}\begin{tabular}[l]{@{}l@{}} Fostering\\adoption \end{tabular} &

  \cellcolor{gray!15}\begin{tabular}[l]{@{}l@{}} P2, P4, P5, P8, \\P10, P11, P15,\\ P18 \end{tabular} &
  
  \cellcolor{gray!15}\begin{tabular}[l]{@{}l@{}}``It [open source] has been fundamental instrumental to our go-to market and our adoption'' (P8) % and just how we build our company'' [P8] \\ 
  \\
  
  ``They ended up eventually open sourcing theirs as well, because there wasn't any real adoption of it as a proprietary software'' (P15)\\
  
  %``And in fact, large tech companies compete with each other as to how much they can give away, and how much adoption they can get on\\ what they give away'' (P10)%So there's quite a competition between, oh, I'll open source something, you'll open source something they\\ both do about the same thing''[P10] \\
  
  \end{tabular} \\ 
 \hline
 
 %And there's such a wide variety of components. And these components have become standardized, because everybody's using it. So developers start trusting if if, you know, all of these companies are using it, it must be a good component, instead of building it themselves. [P18]
 
 %2:7 ¶ 11 in [P4] Per-Spotify interview.docx
%with the stuff they do. So they actually have open source projects, they want to like commercialize and make into a separate business. So besides just having an app, they also want to start selling software that actually started as open source software

%And there was an inflection point in the open source project in 2019, where we-- where Kyle saw massive adoption. There was like this overnight hockey stick in in in the amount of users the amount of stars on GitHub, like the project was just skyrocketing. And John and Kyle came together to have this conversation around: Could we-- could we take the value that we're seeing in this open source project and create a company that takes it to the next level. [P5]

%So this model is a product lead growth model. The product, the usage, adoption of the product. is the go to market strategy. We don't have a sales team. [P5]

%3:4 ¶ 17 in [P2] Daniel-Bitergia interview.docx
%One of them is pick up because you want to have a better positioning, for instance, there is we can think of software monopolies. So the best way to get your technology adopted is by open sourcing this, because then everyone will go to the open source option, in some cases, just because this is free, as in free beer, in some other cases,

% But we specifically chose to build our solution on top of an open source project Zephyr. And, you know, it has been fundamental instrumental to our go to market and our adoption, and just how we build our company. [P8]

%And in fact, large tech companies compete with each other as to how much they can give away, and how much adoption they can get on what they give away. So there's quite a competition between, oh, I'll open source something, you'll open source something they both do about the same thing. [P10]

% I mean, in some cases, it is but for a lot of cases, it's not a weekend warrior, that's going to create Kubernetes it's Google, who going to pay engineers full time all day every day and feed them to make a technology and then strategically give that to a foundation to grow in public, so that more people use a technology and oh, by the way, they have a cloud providing service that you pay for that, you know, has some sort of consistency with with that technology.[P10]

%9:46 ¶ 44 in [P11] Dawn-VM Interview.docx
%Because over time, we don't, we don't make money on that initial sale, right, we make money on them continuing to use this over a long period of time. So what we don't want is to, for salespeople to spend a whole bunch of time chasing someone has been not going to use it and not going to renew after the first year, what we want are people that are invested in the project, they know what it's going to do, they know what they need out of it. And, and they're going to commit to it for a long period of time. 

%9:65 ¶ 94 in [P11] Dawn-VM Interview.docx
%And you know, and another thing that especially enterprise customers really look for is, you know, they don't want to be locked into one vendor, right. And so, with platforms that are based on something like Kubernetes, they can use, they can use our, you know, our Kubernetes, but they can also move to somebody else's. So let's say it's, you know, they decide they don't like working with us anymore. They can move to, you know, Google's version, or Amazon's or Microsoft's, or, you know, anybody, anybody else's version of, you know, a Kubernetes type platform. So, so, you know, by not locking them in to a proprietary technology, it makes our customers I think, a lot happier and a lot more likely to work with us. And, and then our assumption is that we are so awesome that they won't want to switch to someone else. But they could. And so this gives them kind of a peace of mind when when choosing a solution that they're not going to be locked into it forever, even when it's not working for them.

%So you know, they did market their application system, they ended up eventually open sourcing theirs as well, because there wasn't any real adoption of it as a proprietary software. [P15]
 
 
%________________Business Advantage_____________________
 \multirow{8}{*}{\begin{tabular}[l]{@{}l@{}} Business \\Advantage \end{tabular}} 
 
%-----------Engineering need-------------
%   &
%  \begin{tabular}[l]{@{}l@{}}Engineering need \end{tabular} &
  
%  \begin{tabular}[l]{@{}l@{}}P2, P4, P7, P8,\\ P12, P13, P15,\\ P17, P18 \end{tabular} &
    
%  \begin{tabular}[l]{@{}l@{}}``it comes from a need from an engineering team that says we might want to use that thing. [P7]
%``Allows [companies] to build software \\ far more rapidly than we might otherwise''[P12]
%``Excellence in engineering, I think, you know high quality software, I think we want, \\ again, to be a humble participant in the ecosystem of which we benefit.'' [P17]

%  \end{tabular} \\
  
  %For us, I think, you know, if you, you know, especially with the the discussions around software supply chain security happening right now, and the, you know, kind of the executive order from the government, from the US government, specifically, around software supply chain security, like the, I think we as an industry are coming to this realization, that open source is an existential need, and the unwillingness to, to kind of get on the train, and participate in a very big way, is an existential threat to your, your business. [P16]
  
  %3:2 ¶ 17 in [P2] Daniel-Bitergia interview.docx
%Yeah and, another way of answering this question might be, because the some of the, I would say, some of the greatest technologies, pieces of technology, around the world, for specific contexts, are open source example, we can go for machine learning or big data. So if you are willing to start playing with the technology, or having that technology in place, that's definitely where you have to go.

  %2:51 ¶ 37 in [P4] Per-Spotify interview.docx
%And if you actually came up with this standard, then you also get a lot of exposure. Like Google got a lot of exposure from like, creating Kubernetes, which is like the industry standard for like management services. Now again, and Spotify wants to kind of create the same kind of narrative for like internal

%2:58 ¶ 37 in [P4] Per-Spotify interview.docx
%They, they want to leverage open source as much as possible. And the best way to do this is to make sure that what you are working on inside of your company, and the thing you depend on actually becomes the industry standard, not just something you maintain. Because if it becomes an industry standard, then you gain all the benefits from being part of the standards. And if you actually came up with this standard, then you also get a lot of exposure. Like Google got a lot of exposure from like, creating Kubernetes, which is like the industry standard for like management services. Now again, and Spotify wants to kind of create the same kind of narrative for like internal 

%2:65 ¶ 89 in [P4] Per-Spotify interview.docx
 %They do not want to change their infrastructure, that often they don't want to migrate into something new, the reason why they migrate is because a better thing has become the industry standard, basically. Yeah, so. So a benefit for us is that if we can take a component and open source and actually make that into an industry standard, or something that's like widely used and understood, that's better for us, because then first of all, you get external contributors chance to work on your project and use yourself
 
 %2:67 ¶ 37 in [P4] Per-Spotify interview.docx
 %But the the reason why they did this is because They, they want to leverage open source as much as possible. And the best way to do this is to make sure that what you are working on inside of your company, and the thing you depend on actually becomes the industry standard, not just something you maintain.
 
 %8:28 ¶ 23 in [P7] Josep-Aiven interview.docx
%and understanding Postgres, it's better than I operate. Oracle and then on top of that, there is the licensing costs and who knows Oracle only if you use your Oracle, you need to do certifications. You can not that It's easier when it's everything is open source.

%8:70 ¶ 75 in [P7] Josep-Aiven interview.docx
%one is the consumption part. So one thing is that we get to do things that other ones already did, right. So sometimes they already guides there, some, some people already did some thinking about some processes, and we can just benefit off that one. So we can just use things directly from from people that put some brains and time and thought on that. So that's that's one part. 

%8:102 ¶ 119 in [P7] Josep-Aiven interview.docx
%it comes from a need from an engineering team that says we might want to use that thing. Y


%7:4 ¶ 19 in [P8] Johnathan-Goliath interview.docx
%we look to either use open source whenever we can, for our own stuff, as well as open standards. And the thing with open standards is they're only open in open source, even if there are few open standards that exist, they're proprietary, but fundamentally, are not adopting the industry. So therefore, everything we do in the world of IoT is some sort of standard, and therefore, we can only use open source. 

%16:47 ¶ 85 in [P12] Joshua tidelift Interview.docx
%Open source allows us to build software far more rapidly than we might otherwise.

%16:56 ¶ 109 in [P12] Joshua tidelift Interview.docx
%invest in the legal, you know, making sure that we're we're good with our licenses.


  %What they did instead was to offer that for free. Right, and those you know, and those same players haven't bought those licenses, would have been able to sell those servers to competitors of Facebook. Right. And so what Facebook did instead was to say, well, we're just going to give those for free. Everyone can use them. Right. And that enabled competition, because folks can just build those. Right? And obviously, the more players have an somewhere like the the cheaper things, because there's more competition, right? And people are ready to take smaller margins because they're invested. And they don't have to pay for the licensing fees.[P13] 
  
  %12:13 ¶ 96 in [P15] JoshB RedHat Interview.docx
%um mostly because the lawyers don't have to meet. Right, there's an established set of practices of oh, hey, you have this, you know, you have this CLI statement, and you have this CLA, and you have this open source license, and you're good to go.

%Well, I mean, excellence in engineering, I think, you know high quality software, I think we want, again, to be a humble participant in the ecosystem of which we benefit. And excellence in engineering, especially around security, you know, we're part of the OpenSSF, we find a lot of security initiatives and in projects.[P17]

%And one of the first ways companies get involved is they start consuming open source, because developers today, modern software development really includes building with open source components, because they're so available. And there's such a wide variety of components. And these components have become standardized, because everybody's using it. So developers start trusting if if, you know, all of these companies are using it, it must be a good component, instead of building it themselves. [P18]

%-----------Business depends on OSS-------------
   &
   
  \begin{tabular}[l]{@{}l@{}} Business\\dependency\\on OSS \end{tabular} &
  
  \begin{tabular}[l]{@{}l@{}} P2-P5, P7, \\ P9-P12, P14- \\P17 \end{tabular} &
  
  \begin{tabular}[l]{@{}l@{}}
  %``We want these projects to be healthy, alive and well maintained, and sustaining. Because we're depending on them'' [P9]\\
  ``Most systems as [product name] is built on open source software, that is like the the basis of everything is open source dependencies. \\ Then our engineers also contribute to open source projects  that they depend on'' (P4)
\\
``So we have a product line called [product name], which is kind of our flagship like looking forward most strategic project \\that we have at [company name], and that's built entirely on top of [OSS dependency]'' (P11)
  \\
  %``There's a journey that companies seem to go on from like... thinking they don't use it, to realizing that they do... to a point where \\companies start realizing they need to give back ... Ultimately, they know at this point, that a healthy ecosystem means good\\ business for them, too'' [P12]
  \end{tabular} \\ 
  
  %So producing open source is might be a business advantage, depending depending on where you are in the market or what your purposes are. [P2]
  
  % the same for the production, so then you won't be part of the of the communities to have influence in the roadmap to define where the project goes or, it's so there is let's say, like, CTO study level decision there. Or it's because the people within the company they are using, and perhaps producing open source on their own, but not because there is a general policy.[P2]
  
  %1:4 ¶ 39 in [P3] Omer-VM interiew.docx
%So the name of the game for open source contribution is basically like, commits, and then governance and leadership, right? So getting elected to leadership positions in these communities is really important. 

%1:8 ¶ 67 in [P3] Omer-VM interiew.docx
%VMware is Ford, Ford really gives a shit what happens to that engine, right? Because if something happens to that engine that they have to that creates work for Ford, right? Or even worse, makes it so that Ford can't use the engine the way they were using it before. Now you've got a problem. Right?

%2:68 ¶ 45 in [P4] Per-Spotify interview.docx
%Well, it's sometimes, it's just because they they have an issue with the product. And they just fixed it up. That's it. So that you're scratching your own itch kind of thing.

%So first of all, it consumes a lot of open source. And most systems as Spotify is built on open source software, that is like the the basis of everything is open source dependencies. Then our engineers also contribute to open source projects that they depend on. And this is this is on various formal, informal ways. But they contribute where they see a need, basically.[P4]

%2:104 ¶ 117 in [P4] Per-Spotify interview.docx
 %if you just depend on open source code, and then when you start depending on open source code, there's a very natural register as contributing to open source code

%He, he believes in open source, he's a consumer of open source projects, we have-- we as a business consume other open source projects in our product delivery process, right. Like, we have other projects that our developers use that are open source.[P5]

%5:48 ¶ 77 in [P5] Jill-SlimAI interview.docx
%And so for us to patron Beautify or Nuxt, which are some of the things that we use at Slim, that really matters to us. We actually want to see those projects continue, because they're not only doing good for other developers, but they're doing good for us.

%So everything is a journey, in that sense. So the goal is obviously to try to impact as many on the governance of the projects as possible to be as again, for the same thing as before, to guarantee that it's not just one company behind and to guarantee different opinions, voices and points of interests. It's a journey depends on the project, and depends on how easy is to get into those circles on the projects. Some projects are more closed, and some projects are more welcoming. [P7]

%And we, it's, we're very selfish, we want these projects to be healthy, alive and well maintained, and sustaining. Because we're depending on them. So we ensure that we we help them we contribute within our meetings, which are small, whatever we can to ensure their well being that means reporting bugs, fixing bugs when we can, promoting them when we can, and so on and so on.[P9]

%11:37 ¶ 41 in [P10] Gil Interview.docx
%I might want to fix it, I might want to contribute that fix. And because they might fix it wrong. Right? I might want to make sure that our engineer fixes it to ensure that our fix is the one that gets put in the upstream project. Right. So on that case, of course, I want to contribute back.

%11:47 ¶ 47 in [P10] Gil Interview.docx
 %definitely want to fix a bug in Kafka because that thing is so slow. And if we can make it faster,


%Yeah, we think that being in leadership roles, and being part of the governance is really important. And so for a lot of the projects that we work on, we do expect our contributors to eventually try to move into maintainer positions approver positions, leadership.[P11]

%9:29 ¶ 24 in [P11] Dawn-VM Interview.docx
%Because the reality is, you know, once you once you get involved in these communities and are involved at a deep enough level, that you're moving into leadership positions, that's where you really get the benefit from it

%9:32 ¶ 28 in [P11] Dawn-VM Interview.docx
%I wish we had more contributions to projects that we just use, as opposed to the projects that we build our business on.

%9:34 ¶ 28 in [P11] Dawn-VM Interview.docx
%in those leadership positions, gives you gives you a couple of different things it gives you, like I said, insight into the project, 

%9:35 ¶ 28 in [P11] Dawn-VM Interview.docx
%also gives you and also gives you influence. 

%9:36 ¶ 28 in [P11] Dawn-VM Interview.docx
%you know, you can better better know what types of contributions they're likely to accept, and which ones they aren't. 

%9:37 ¶ 28 in [P11] Dawn-VM Interview.docx
%what we what we do really encourage our individual teams is, you know, not to fork the project and carry all of the technical debt internally, where they're trying to patch it when a new release comes out. And it's it's just an awful lot of work. So we're trying to convince more and more business units that they need to contribute these fixes back upstream

%9:38 ¶ 28 in [P11] Dawn-VM Interview.docx
%it improves security, there are just so many benefits to that. 

%9:66 ¶ 94 in [P11] Dawn-VM Interview.docx
%without something like Kubernetes, we, we wouldn't have been able to spin up this Tanzu product line in a relatively short period of time, considering, you know, how long it normally takes to build a product at that sophistication, but, but by having something like a Kubernetes, we can just innovate on top of that, because we have, we have the platform. And you know, 

%16:26 ¶ 49 in [P12] Joshua tidelift Interview.docx
%Right. So that's probably more probably more opportunistic, like, hey, we ran into a sharp corner, we found a fix for it. Here's the fix. Yeah, more of that style.

%16:37 ¶ 69 in [P12] Joshua tidelift Interview.docx
 %because the business model only makes sense because we have the open source ecosystem that we have, and what I mean by that is, you know, 

%And it's only as companies grow and start applying tooling to understand how extensive their use of open source is, when they start realizing like, oh, we need to invest in the security of this, we need to invest in the legal, you know, making sure that we're we're good with our licenses. [P12]

%17:21 ¶ 6 in [P14] Harish RedHat Interview.docx
%And if you've never done that before, that's why we try and help them. So that's the other part of the story. We from an OSPO perspective, there are organizations that are struggling trying to figure out how to do this. And they reach out to us for help. And so OSPO trying to fill that role and from Asia Pacific perspective, that ends up becoming me doing it with my colleagues, my colleagues in whichever city that they may be in it could be in Japan, it could be in Tokyo or in Beijing or you know, wherever, whichever goes if if there is if there is a requirement, let's see what we can do.

%12:2 ¶ 26 in [P15] JoshB RedHat Interview.docx
%So I mean, obviously, from a technical basis, how we're involved in open sources, that's how we build all of our stuff, right? Every Red Hat product starts with an open source project, where we do engineering, either by ourselves or together with contributors who work for other companies or no company

%12:11 ¶ 92 in [P15] JoshB RedHat Interview.docx
%And I would say, right, and I would say, it depends, right. And that usually depends on what our level of commitment is to the project, right? Because if it's a public project, in order to get any influence over things like roadmap, you have to have somebody who is a major contributor to that project. Or you have to be financially sponsoring that project at a substantial level. Right? Because otherwise, why would they listen to you? Yeah. The, I mean, obviously, if it's a project that we started, we have, you know, control over all of these things. Or if it's a project that we co founded, with other companies, which happens a fair amount, then we have whatever our sort of equal say, is over those things.

%19:14 ¶ 50 in [P16] Stephen Transcribed Interview.docx
%Right, I think that, you know, because often you write, you write a check, and it's for a foundation, and it's large enough, then there may be a, you know, say a board seat associated with it. Right. And right there, you you get an immediate way to, to kind of influence that project or set of projects. And that's great. 

%Well, I mean, the goal is that all upstream projects are being contributed to so if we're using an open source project, and we find a bug, or there's something that we can be, you know, that we will always contribute to upstream projects. There are lots of folks at Microsoft who contribute to open source on their personal [tell end?]. There are folks that we-- belong to [use?] and we call our open source stats program inside of Microsoft, who mentor and help others learn how to do that. So to answer your question, lots of different ways, but it's very much part of the culture.[P17]

%Well, I think security both ways, right, like making sure that any security issues that we find, or you know, that we're pushing fixes to those up and that we also benefit from those, I think security is one of those, like probably the most important example.[P17]

%-----------Coopetition-------------  
  &
 \cellcolor{gray!15}\begin{tabular}[l]{@{}l@{}}Coopetition  \end{tabular} &
 
  \cellcolor{gray!15}\begin{tabular}[l]{@{}l@{}} P3, P7, P8, \\P11, P13-P15, \\P17-P19 \end{tabular} &
  
  \cellcolor{gray!15}\begin{tabular}[l]{@{}l@{}}%``it's an implicit resource sharing agreement between companies that are working on the same problem'' [P3]
  ``The benefit is open collaboration, you know, I don't have to be the only person trying to figure out how to solve a problem'' (P14)
  \\
  ``You need to get...multiple vendors involved so that you can a) like improve your product because you have to like be accepting \\ opinions from lots of different people and then b) you get more market share that way'' (P3)
  
  \end{tabular} \\
  
  % So taking this back to like the technology space, right, like a serverless offering is great on its own. But when you're a major corporation, you want to buy, like these open source projects, you want to buy services around them. That means that the expertise about how to implement that technology, right, the support when that technology goes down, some kind of guarantee on security vulnerabilities, right. And these are the sort of things like if you can imagine the car example these are the cupholders. This is the radio. This is the shape of it, right. Yeah. So so that's sort of how These these companies are, are collaborating with one another, and they're doing it in an open way, because it doesn't actually matter if Joe Schmo gets an engine, right? Like, it doesn't matter if their competitors get an engine, that's not the competition, the competition is everything that only these larger companies can do, like provide you with like services and provide you with security. And they sort of give you this, this full package that is surrounding that, that core technology that is much more valuable than technology itself. [P3]
  
  % If five cloud companies have the same challenge, it makes sense to work together to solve that problem, and not solve it through partnerships and agreements, because that's a very heavy legal lift. But to come together in a foundation and to say, let's create a charter, that's, you know, maybe one company has started solving that problem, and then release that code to that foundation, when everybody else starts working on that, and can continue to improve that. And I think that's how they're, you know, networking, open source networking at the Linux Foundation, it started with at&t contributing code, and saying, Hey, we did this to solve our own problem. But it feels like, you know, everybody's struggling with the same thing. And maybe if we can all work together, we can improve this, so it reduces the burden for AT&T to do have to develop this all by themselves. And it also helps others who have the same problem. [P18]
  
   %And I think that's how they're, you know, networking, open source networking at the Linux Foundation, it started with at&t contributing code, and saying, Hey, we did this to solve our own problem. But it feels like, you know, everybody's struggling with the same thing. And maybe if we can all work together, we can improve this, so it reduces the burden for AT&T to do have to develop this all by themselves. And it also helps others who have the same problem. And so and, and these structures, like foundations were created to have this neutral place, where companies can come together and collaborate.[P18]
   
    %So open source is not separate or optional, but it's central to excellence in open source engineering, and culturally speaking, very much in line with our values of building on the work of others and finding others to build with us.[P17]
 
  %1:3 ¶ 243 in [P3] Omer-VM interiew.docx
%ou need to get you need to get you need to get multi vendors like multiple vendors involved so that you can a) like improve your product because you have to like be accepting opinions from lots of different people and then b) you get more market share that way.

%1:12 ¶ 247 in [P3] Omer-VM interiew.docx
%Yeah, think about it. Like Think about it like this, right? You have. So this is where the this is where the car analogy starts to fall apart. Okay, if I have a bunch of work work, like, if I have a bunch of services on Red Hat's OpenShift? Which is you--like their Red Hat serverless is Knative under the hood. If I want to move that to VMware. I have a lot easier of a time moving that over than if I was doing it in Google with their own proprietary stuff. Because the basis is the same. And so now these companies can steal customers from each other.

%1:14 ¶ 31 – 32 in [P3] Omer-VM interiew.docx
%You can imagine auto manufacturers, like, all need an engine for their for their cheap car, they all need a thing that gets in gasoline. And like gets out. Like horsepower, right? Like, like, and the KPI is miles per gallon, right. So the the further you can go on a gallon of gas, the better the engine is, the cheaper it is to make, the better the engine is, there's all these things that are beneficial for all of these companies who make this same car. Now, this would never happen in real life, because this is called collusion. Or, you know, there's like there's like anti monopoly rules against this, right, because of the way that like certain large organizations work. But there is still some open source work that's being done in these like manufacturing sectors. So you can imagine, like, if I'm Red Hat, or if I'm VMware, or if I'm Google, and I want to offer this, like this car, so to speak, and I know that like, I need to build an engine for it. And I'm looking around and realizing that like, all of my competitors need to build this engine for it. And people don't really buy a car for the engine, they buy a car for how it looks, they buy a car for the features, they buy a car for safety, they, you know, they're not really buying the car for the engine, there's all these little things around the engine that make a car worth buying. Right? 


%1:15 ¶ 21 in [P3] Omer-VM interiew.docx
%view open source work, especially when it's like a couple of large vendors who are working together as staff augmentation, right? They have, they have 20 engineers, those 20 engineers will go further working on a project with 80 other engineers from four different companies, and they will if you try to solve that same problem by yourself, so functionally, it's a it's an implicit, not explicit, right. So it's not written down. But it's an implicit resource sharing agreement between companies that are working on the same problem as an example. And I'm just gonna keep talking until you stop me by the way, I have a lot of fun.

%1:27 ¶ 21 in [P3] Omer-VM interiew.docx
%that same problem by yourself, so functionally, it's a it's an implicit, not explicit, right. So it's not written down. But it's an implicit resource sharing agreement between companies that are working on the same problem as an example. 

%1:28 ¶ 21 in [P3] Omer-VM interiew.docx
%is a way for them to do the hard work only once, right, and to do it in a way that is like,

%8:74 ¶ 75 in [P7] Josep-Aiven interview.docx
%then there is the other one, which is we can also share these things back so we can also share, what do we do that is different that maybe some other ones will try to do or show another way, show how we can how we did it and what we're doing and what works and what doesn't. And then of course, it's sharing experiences between different groups and people learning from each other.

%8:75 ¶ 75 – 76 in [P7] Josep-Aiven interview.docx
%creating as a group also like some kind of nice ideas, nice ways to go nice, nice roads to point to other companies to say these other ways. These are the different ways to do it. These are the different roads you can take. Pick the one that fits best to you. So that's one of the things that we buy being present in these groups. Of course, it's like the getting some information, obviously, shaping different opinions, different ideas, different ways of doing things.

%7:27 ¶ 85 in [P8] Johnathan-Goliath interview.docx
%There's that but also just like, collaboration and partnerships become six orders of magnitude easier, because all that is open source. That's probably the two biggest reasons.

%um, yeah, so I would say that the biggest benefit that we see is innovation, because we can only do so much on our own right you know, we we hire whole teams of people who are working on similar things. tend to think of things in similar ways. And you get a real innovation boost by pulling in people who work from other companies who will use things differently than you do a lot of different different use cases, they'll have different skills to contribute, they'll have different ways of thinking about things. [P11]

%17:10 ¶ 88 in [P14] Harish RedHat Interview.docx
%The benefit is open collaboration, you know, I don't have to be the only person trying to figure out how to solve a problem. Let's go work with open source community and see, you know, other ideas, other people, you know, doing different things, oh, they fix that problem over there. Okay, let's fix this problem here. Oh, this connects, it gives you enough some examples, right? There was this, okay?

%17:14 ¶ 102 in [P14] Harish RedHat Interview.docx
%I feel that you tend to be a little bit more accountable. It's no longer a fire and forget kind of thing. Well, I write the code, I, you know, put it into the repository, and it gets compiled, and then I don't know what happened to you. I don't know who's using it. Not okay. But here, yeah, you know, because my name is going to be there. I may want to care that, you know, someone says, Hey, hurry, share the code that you wrote was horrible. Or what did I do? Okay, let me go and see what what did I do, right? That I think is the power of the model of this, this open collaboration, and this is the benefit that we get, as well as our customers or customers also, then the customer can turn around and say, you know, we got some engineers who like to work on this, this aspect of this product? Can you can we collaborate with you? Yeah, sure. You want to fix some bugs? Let's start with fixing some bugs first.

%18:4 ¶ 234 in [P13] Tobie OSS consultant Interview.docx
%One is the network that you've built. Um, and that, you know, we've also talked about in terms of like, how actually, this, you know, concretely helps you, in situation like, Hey, you wanna solve a problem? Will you actually know who to talk to?

%18:20 ¶ 60 in [P13] Tobie OSS consultant Interview.docx
%it's fairly easy to waste a week on an incredibly dumb bug, that if you actually know everyone in the community, you know, the key people in the community building that software, you're going to solve in like, 15 minutes, because you're going to email someone that you know, and say, hey, I'll just chat them and say, hey, you know, John, whatever, I'm bumping into this weird issue, what's going on? And you know, they will get back to you in a second. You know, and like, within minutes saying, oh, yeah, this is a known problem. There was a patch, look at that pull request here, blah, blah. All right. So these aspects are, are usually more valuable.

%12:25 ¶ 26 in [P15] JoshB RedHat Interview.docx
%together with contributors who work for other companies or no company. The second way is considered the ideal. Right? If we can make that happen, we do make that happen. Because, you know, why would you put in 100% of the effort if you didn't have to, plus, getting contributions from outside Red Hat often bring with them perspectives from outside Red Hat, which allow us to make products that fit a broader clientele.

%-----------Closer Channels-------------
   &
  \begin{tabular}[l]{@{}l@{}} Closer\\channels  \end{tabular} &
  
    \begin{tabular}[l]{@{}l@{}} P1, P5, P7, \\P13-P15 \end{tabular} &
    
  \begin{tabular}[l]{@{}l@{}}  
  %``One is the network that you've built. Um, and that, you know, we've also talked about in terms of like, how actually, this, you know, \\concretely helps you, in situation like, Hey, you wanna solve a problem? Will you actually know who to talk to?'' [P13]
  
  %``direct feedback...That is a huge benefit... you can look into the contributors\\ file and see who contributed... want to talk to them directly? Go ahead and do it'' [P14]
 
 ``I want to be able to, you know, how are my customers using it? What are the challenges they are facing?...I don't want it to be, \\you know, only coming by feedback and then somebody collects all the feedback and then massages the feedback'' (P14)
% \\
%``If you actually know everyone in the community...the key people in the community building that software, you're going to solve in like 15\\ minutes'' (P13)
  
  \end{tabular} \\
  
  %Yes, or because they, the companies come with an interest. And we are active in these communities showing, showing off our skills and experience? They're saying, Oh, yes, we want more of that. Can you dig in more? Show us more what, and we basically use these communities to showcase what we can do. And whether our potential customers are themselves active or just observing doesn't matter. It creates that contact to then have a conversation.[P1]
  
  %5:33 ¶ 137 in [P5] Jill-SlimAI interview.docx
%So our users trust us because we're giving them open source value. Our early adopter SAS users trust us because they're using the platform, and they're getting value from it. And we're listening to their feedback in both the open source and the SAS

%5:57 ¶ 31 in [P5] Jill-SlimAI interview.docx
%You know, one of the things that's really cool about having an open source community is is you learn from them just as much as they learn from you. And, and so that that's woven into how we operate the business, is taking the user feedback, taking the user learning, taking the user, the new problems that users want to solve, like taking all of that and ingesting that back into our product delivery process. 

%And that's what it does. And of course, Aiven believes in open source because by using open source, you have big, bigger reach. [P7]

%18:4 ¶ 234 in [P13] Tobie OSS consultant Interview.docx
%One is the network that you've built. Um, and that, you know, we've also talked about in terms of like, how actually, this, you know, concretely helps you, in situation like, Hey, you wanna solve a problem? Will you actually know who to talk to?

%18:20 ¶ 60 in [P13] Tobie OSS consultant Interview.docx
%it's fairly easy to waste a week on an incredibly dumb bug, that if you actually know everyone in the community, you know, the key people in the community building that software, you're going to solve in like, 15 minutes, because you're going to email someone that you know, and say, hey, I'll just chat them and say, hey, you know, John, whatever, I'm bumping into this weird issue, what's going on? And you know, they will get back to you in a second. You know, and like, within minutes saying, oh, yeah, this is a known problem. There was a patch, look at that pull request here, blah, blah. All right. So these aspects are, are usually more valuable.


%17:12 ¶ 88 in [P14] Harish RedHat Interview.docx
%Another example is this will have to do with real time, fix real time, Linux kernel, article, real time kernel. So what is real time kernel, the request came from a customer. That was interesting, this actually came from a customer which is which happens to be the US Navy, the US Navy, because all the naval ships are running Red Hat Linux inside, which is, you know, all this, you know, systems

%17:13 ¶ 96 in [P14] Harish RedHat Interview.docx
%That's right, direct feedback. Yes, exactly. Exactly. That is a huge benefit. That's a big benefit. Like, you can look in the when you install [rel-- Raheny]? precedents, for example, Fedora or CentOS, or whatever, you can look into the contributors file and see who contributed you have all the email name, names and email addresses that you want to talk to them directly? Go ahead and do it

%17:14 ¶ 102 in [P14] Harish RedHat Interview.docx
%I feel that you tend to be a little bit more accountable. It's no longer a fire and forget kind of thing. Well, I write the code, I, you know, put it into the repository, and it gets compiled, and then I don't know what happened to you. I don't know who's using it. Not okay. But here, yeah, you know, because my name is going to be there. I may want to care that, you know, someone says, Hey, hurry, share the code that you wrote was horrible. Or what did I do? Okay, let me go and see what what did I do, right? That I think is the power of the model of this, this open collaboration, and this is the benefit that we get, as well as our customers or customers also, then the customer can turn around and say, you know, we got some engineers who like to work on this, this aspect of this product? Can you can we collaborate with you? Yeah, sure. You want to fix some bugs? Let's start with fixing some bugs first.

%17:35 ¶ 90 in [P14] Harish RedHat Interview.docx
%So, three of us, we came together to say, Okay, what do we need to do to make the Linux kernel real time? Okay, sorted out the thing and he made it made it work a real time. Then, of course, how does react do what we do we upstream the whole thing, we send the improvements and fixes upstream to the Linux kernel itself. And where did that then show up? That's real time component arrived at the Linux kernel that was shipped by Red Hat to our customers landed in a customer in Chicago Mercantile Exchange. 

%17:36 ¶ 90 in [P14] Harish RedHat Interview.docx
%These are the kind of innovations that you know, could we have done it on our own? Yeah, but maybe we could, but you know, it's gonna take a long time. And we have a real customer who has a real requirement to test something, we can test it make it happen, and then we make it available to everybody else.

%17:37 ¶ 96 in [P14] Harish RedHat Interview.docx
%Here you can, go ahead and reach out directly. We are not we are not putting a firewall. They say no, you cannot talk to my developers. Because the developers must also understand what the customer's requirements are. Because if I'm building something I want to be able to, you know, know, how are my customers using it? What are the challenges they are facing? It I don't want it to be, you know, only coming by feedback and then somebody collects all the feedback and then massages the feedback.

%12:25 ¶ 26 in [P15] JoshB RedHat Interview.docx
%together with contributors who work for other companies or no company. The second way is considered the ideal. Right? If we can make that happen, we do make that happen. Because, you know, why would you put in 100% of the effort if you didn't have to, plus, getting contributions from outside Red Hat often bring with them perspectives from outside Red Hat, which allow us to make products that fit a broader clientele.

%12:26 ¶ 26 in [P15] JoshB RedHat Interview.docx
%also as a way of learning what is it that people want, so that we can actually build those products? Right. Um, so marketing in terms of market research, and in terms of getting direct feedback? You know, like, if we're thinking about something new in Linux, we can put it in fedora and see how it does in fedora. Um, you know, before we have to offer five years of support for it.

  
%-----------Innovation------------- 
   &
  \cellcolor{gray!15}\begin{tabular}[l]{@{}l@{}}Innovation \end{tabular} &
  
    \cellcolor{gray!15}\begin{tabular}[l]{@{}l@{}}P11, P13, P16, \\ P17, P19 \end{tabular} &
 
  \cellcolor{gray!15}\begin{tabular}[l]{@{}l@{}}%``Innovation is a big part of our of our open source story... because we can... innovate with the rest of the ecosystem around these \\ core, open source technologies'' [P11]
  
  ``I would say that the biggest benefit that we see is innovation, because we can only do so much on our own'' (P11)
  \\
  ``Innovation is the other one...So the more that we're able to collaborate with external communities...The better the product'' (P17)\end{tabular} \\\hline
  
  %um, yeah, so I would say that the biggest benefit that we see is innovation, because we can only do so much on our own right you know, we we hire whole teams of people who are working on similar things. tend to think of things in similar ways. And you get a real innovation boost by pulling in people who work from other companies who will use things differently than you do a lot of different different use cases, they'll have different skills to contribute, they'll have different ways of thinking about things. [P11]
  
  %So we contribute to some machine learning projects, because we think that that is going to be a way that we can, you know, evolve VMware projects and products in some way. We don't, we don't quite know how yet, but we're, we're contributing and getting more familiar with it. We're also doing that in an area of [refrigerants?] observability. So it's things like, it's more like logging and tracing and trying to better understand what particular platforms are doing so that you can make improvements to them. [P11]
  
  %And open source is exactly the same thing. It's like, if you actually really do it that way, working in the open and being confronted with other neat use cases was how other people use stuff. It's kind of when you pick up a book, right? It's like, oh, it opens your mind to all of these things you had never thought about. But so this works in this generative of innovation. [P13]
  
  %18:3 ¶ 160 – 162 in [P13] Tobie OSS consultant Interview.docx
%And so open source, like is going, like completely changes that by essentially giving you just open forum, where, like minded people from completely different places in the world, and different verticals, and different, you know, profits and nonprofits and ad and like business and you can come together. Um, it was like, a shared understanding of what they're coming together for and a shared understanding of like, speaking about it and dealing with it. Right. And that is extremely unique. And that is actually extremely hugely effective for Lots of different things. I mean, this is how you innovate, right? You like confront completely different perspectives in the place where there was a mutual understanding of the underlying language. So, so that's the kind of things that I find that are, you know, it's hard to explain that about open source. 

  %18:19 ¶ 52 in [P13] Tobie OSS consultant Interview.docx
%large companies do invest into, you know, do leverage that information to make decisions about, you know, unrelated, well, decisions that are not pertaining to that particular project. But, you know, for example, if you see, like, lots of activity around a whole bunch of, you know, types of open source project that, you know, clearly shows that there was interest for that kind of software. And so it might be valuable to, you know, build solutions around this, for example, or even clear, even proprietary solutions.

%19:9 ¶ 44 in [P16] Stephen Transcribed Interview.docx
%So in emerging, you know, so being in emerging tech incubation, there's this this requirements, to, to be a bit ahead of the curve, right, and try out things that may not be may not be the, you know, directly, you know, immediately core to the business. But will be useful in you know, in the six months to 12 months to two years, so on and so forth. 

%19:12 ¶ 44 in [P16] Stephen Transcribed Interview.docx
%So that's our business level group. And, yeah, we, you know, there's, there's that need to be always looking ahead, like, you know, lizard steering to where the puck is going, kind of thing. So,

  %I think innovation is the other one, especially when you-- with the lens of diversity and inclusion. You know, we could have some of the smartest engineers, you know, [Speaker retina?], or 12 of the smartest engineers in the world, building something in a room together. But at the end of the day, that's still just represents those 12 engineers', you know, worldviews and experiences. And you know, if they've, you know, felt exclude, like, there's all different layers to it. So the more that we're able to collaborate with external communities, the more perspective, the more, you know, kind of like user stories or use cases or whatever word you want to use there. The better the product you're building, the more likely it is that it will solve problems for more people and not just those 12 extremely bright people in the room. Right.[P17]
  
%____________________Reciprocity__________________________
 
  \multirow{1}{*}{\begin{tabular}[l]{@{}l@{}} Reciprocity  \end{tabular}}  

%-----------Sustain ecosystem-------------  
   &
  \begin{tabular}[l]{@{}l@{}}  \end{tabular} &
  
    \begin{tabular}[l]{@{}l@{}}P4, P6, P7, \\P12-P14, P16, \\P17 \end{tabular} &
 
  \begin{tabular}[l]{@{}l@{}}
  %``As much as... every company can, in its own capacity, everyone should be contributing back, everyone should be making sure that \\they are not just the receiving end'' [P7]
 
  ``From the perspective of being a...trustworthy, honest and sincere participant in open source that we want to be part of the ecosystem,\\ you know, giving back as much as possible'' (P17) \\
  
  ``We can also share, what do we do...it's sharing experiences between different groups and people learning from each other'' (P7)
  \\
  %``I think tech plays a huge role in society. And there are not a lot of people that understand the technology and how the technology\\ is really made... so I've essentially tried to move increasingly in the space where I could be an advocate of this and have been trying\\ to find my way through that'' [P13]
  \end{tabular} \\ \hline
  
  %2:69 ¶ 121 in [P4] Per-Spotify interview.docx
%ou already mentioned that we are very bad at actually releasing something and then continue caring about it the next few weeks, the actual benefits of open sourcing it.

%6:8 ¶ 179 in [P6] Otavio-Zup interview.docx
%eah, we are in the events, talking about it, spreading the word. And letting everybody know that contributing to open source is good. It's good for you and your company, because you're using them. So I think that's it. I can't remember anything else.

%6:33 ¶ 117 in [P6] Otavio-Zup interview.docx
%And we say it a lot, we try to spread the word about open source, because we all use it, right? We, we use it all the time when we're delivering when we are creating software.

%8:33 ¶ 39 in [P7] Josep-Aiven interview.docx
%the second one is the giving back. So we took lots of things from open source. And we need to give back it's not, it doesn't work, it doesn't scale. And it's not a sustainable thing to do. So as much as every one can, every company can, in its own capacity, everyone should be contributing back, everyone should be making sure that they are not just the receiving end, 

%So for us, we think that this is a systemic problem in the open source ecosystem. And it's not just us, who think that you know, with the Heartbleed, bug in 2014, really kicked off the huge discourse about open source sustainability.[P12]

%17:15 ¶ 110 in [P14] Harish RedHat Interview.docx
%That means we have expanded the ecosystem. We have created more people with skills in this space. So we have broaden the economic pie across industries across economies around the world. To me that is extremely powerful. Extremely powerful. You don't have to always come back to me. Yeah, I enable you, you fly, you stop lying. And if you do well, good for you.

%19:5 ¶ 17 in [P16] Stephen Transcribed Interview.docx
 %they're more about sustainability, right? And there are so many, there are so many different angles that you can you can consider when you're thinking about the sustainability of an open source project, but sustainability of like, open source foundations, that's, that's not just limited to writing the check.

%That's really important to Microsoft, from the from the perspective of being a--what's the word I'm looking for--trustworthy, honest and sincere participant in open source that we want to be part of the ecosystem, you know, giving back as much as possible, but also, some tactical standpoints. [P17]

%-----------Sharing experiences------------- 
   %&
  %\begin{tabular}[l]{@{}l@{}}Sharing experiences \end{tabular} &
  
    %\begin{tabular}[l]{@{}l@{}}P6, P7, P13, P14\\  \end{tabular} & 
    
  %\begin{tabular}[l]{@{}l@{}}``We can also share, what do we do... it's sharing experiences between different groups and people learning from each other'' [P7]
  %\\
 % ``I think tech plays a huge role in society. And there are not a lot of people that understand the technology and how the technology\\ is really made... so I've essentially tried to move increasingly in the space where I could be an advocate of this and have been trying\\ to find my way through that'' [P13] \end{tabular} \\ \hline
  
  %6:8 ¶ 179 in [P6] Otavio-Zup interview.docx
%eah, we are in the events, talking about it, spreading the word. And letting everybody know that contributing to open source is good. It's good for you and your company, because you're using them. So I think that's it. I can't remember anything else.

%8:74 ¶ 75 in [P7] Josep-Aiven interview.docx
%then there is the other one, which is we can also share these things back so we can also share, what do we do that is different that maybe some other ones will try to do or show another way, show how we can how we did it and what we're doing and what works and what doesn't. And then of course, it's sharing experiences between different groups and people learning from each other.

%8:75 ¶ 75 – 76 in [P7] Josep-Aiven interview.docx
%creating as a group also like some kind of nice ideas, nice ways to go nice, nice roads to point to other companies to say these other ways. These are the different ways to do it. These are the different roads you can take. Pick the one that fits best to you. So that's one of the things that we buy being present in these groups. Of course, it's like the getting some information, obviously, shaping different opinions, different ideas, different ways of doing things.

%18:27 ¶ 94 in [P13] Tobie OSS consultant Interview.docx
%I think tech plays a huge role in society. And there are not a lot of people that understand the technology and how the technology is really made. And or mindful of tax impact on society. And so I've essentially tried to move increasingly in the space where I could be an advocate of this and have been trying to find my way through that. Um, so you know, I guess that's the motivation.

%17:22 ¶ 16 in [P14] Harish RedHat Interview.docx
%In this case, I don't care whether you give me the money or not, I used to win something I still win because the goodwill is established, the positive feedback is there. And I think that is you can put $1 value to that it's very hard to put $1 value to something like goodwill. So that's that's really how the long term benefit for Red Hat is because one of the things you may have come across in your research is people will tend to use the phrase in any technology, whatever that technology is, are you going to be the Red Hat of technology? Are you going to be the Red Hat of that? Are you going to be Red Hat for automotive? Are you going to be the Red Hat for medical services. So when people use the word Red Hat that way, they are using some of the norms and how we do what we do, and have been doing for the last 25..28 years in the open source community and with our customers. So they say that makes sense. Who's going to be doing the same thing in the industries? So so in a way the brand recognition is very powerful, you know that the idea is very powerful.



 



 %4:29 ¶ 71 in [P1] Georg-Bitergia interiew.docx
%Why open source, um, one: it's ideological. The founders just believed that was the right thing to do
 
  %3:7 ¶ 17 in [P2] Daniel-Bitergia interview.docx
%depending depending on where you are in the market or what your purposes are. But if you're getting the case, of Bitergia We decided to go with open source, because we were, well it was a mix of things. Um, mainly because part of the solver that we had was already open source because we were coming from a research group. So it was that say kind of philosophical point of view at the very beginning

 %I mean, we are there to contribute on that community of these projects to make the project better, more sustainable, to bring more different ideas on the table as well. So not just one single companies dominating one open source project, but we try to bring we are another companies that we try to also bring opinions there. To not just depend everything on one single company and to create a diversity of opinions, ideas, companies behind and all this stuff. So also to support and just stay true to the open source philosophy.[P7]
 
 %5:1 ¶ 59 in [P5] Jill-SlimAI interview.docx
%we see the intrinsic value of open source as something that is good for the world. And is something that is should always be.

%5:39 ¶ 59 in [P5] Jill-SlimAI interview.docx
%Kyle's vision and intent for the project in the first place. He, he believes in open source, he's a consumer of open source projects

%5:42 ¶ 241 in [P5] Jill-SlimAI interview.docx
%contributing into an environment that matters to me by giving. Right So this concept of like, encourage contribution is important. And that goes back to the culture of like the company culture


 %5:89 ¶ 59 in [P5] Jill-SlimAI interview.docx
%we as a business consume other open source projects in our product delivery process, right. Like, we have other projects that our developers use that are open source. And so not only does Kyle believe in it, and and has the vision for it continuing to be open source. But we as an organization, and people within our company, and within our community, we see the intrinsic value of open source as something that is good for the world. And is something that is should always be.

%8:15 ¶ 59 in [P7] Josep-Aiven interview.docx
%it's a vision of the four founders.

%8:53 ¶ 59 in [P7] Josep-Aiven interview.docx
%And the four founders basically were avid contributors of open source, and they decided that's, that should be our DNA, we want to create a company that way. And that's how we would go.

%10:54 ¶ 208 in [P9] Phillipe NexBe Interview.docx
%the reason why we started nexB was to create an open source alternatives to large enterprise resource planning software, which was mostly proprietary and still mostly proprietary so things like Oracle Applications, or ACP, you know, this.. you've heard about these things, which are large enterprise systems, to manage accounting, finance, sales, marketing, all this kind of things production

%16:23 ¶ 45 in [P12] Joshua tidelift Interview.docx
%he whole philosophy at the core of the company is really about supporting upstream

%16:42 ¶ 73 in [P12] Joshua tidelift Interview.docx
%y. All of the co founders have deep history in open source. Like that most of them been there since the beginning of open source.

%16:45 ¶ 81 in [P12] Joshua tidelift Interview.docx
%I mean, the, you know, the the co founders come from open source and care about it. You know, there's a, there's a real strong belief, and the value of a commons of reusable components that we share of collaborating across boundaries that might otherwise divide us open sources. Open source is amazing. And they saw a problem and that nobody else was fixing in this manner and, and went for it.

%17:34 ¶ 82 in [P14] Harish RedHat Interview.docx
%So we started off open source company. So we know the benefit is there. We know what it is. And we want to not only us benefiting from it, we want to help our customers benefit from it as well.


\hline 
\end{tabular}}

\end{table*}


