\begin{table}[htb]
%\footnotesize
%\scriptsize
\caption{Reputation: Visibility}
\resizebox{0.48\textwidth}{!}{
\begin{tabular}{p{0.2\textwidth}p{0.1\textwidth}p{0.2\textwidth}}
\toprule
{\textbf{Visibility}} & {\textbf{Participants}} & {\textbf{Example}} \\ \hline
  
%____________________Visibility_________________________

Cement their name to their domain of expertise 


&


P4, P7

 
&
  


``Open sourcing these bits and pieces of whenever you can make it like tangible and like very transparent about what was actually going on inside of the company. So that can that can help the deal of branding... And [company] saw a lot of good press'' [P4]


\\
%And, and where open source is superior is it gives us superior visibility into our own supply chain. Right as in, we don't have to-- with proprietary software, you know, if we have a proprietary component, like we were proprietary, authentication component, and we're like, hey, is this being maintained in the light of new security exploits? Right? In that case, you have to trust the vendor to know whether or not that's being maintained. [P15]

%So two reasons. One is personal. And one is business, the personal reason is that we whoever is engaged in in these communities may have a personal interest wants to be wants to learn something new wants to be recognized as, as a leader in that space. So it's a little bit like personal branding. So one, for example, I'm involved in the IEEE sa open community advisory group. And part of that is to build up my credentials as an open source strategist and being an expert on open source communities. For Bitergia, as a company to encourage this and do this, is because being active in open source communities, gives us access to decision makers, at companies that could potentially be customers. And so we use this also as a way for finding prospects and generating business for the company. And most of the business that we do have came through contacts that we fostered in being active in these communities.[P1]

%%2:101 ¶ 23 in [P4] Per-Spotify interview.docx
%So like open sourcing these bits and pieces of whenever you can make it like tangible and like very transparent about what was actually going on inside of the company. So that can that can help the deal of branding of the technical ideas of an app like Spotify, I think. And Spotify saw a lot of good press with like this, Spotify wrapped the day over the Christmas period where everyone got like the year overview of music. And that kind of thing took like a ton of machine learning a ton of data and a lot of engineering tricks, to do this little little feature. And so this is like this is one of the ways they want to be more transparent about like, how do we actually use these, like very sophisticated systems of understanding music and user behavior and, and music history and so on. So we can kind of create this little feature called wrapped, which is just one month a year. 

%8:125 ¶ 135 in [P7] Josep-Aiven interview.docx
%write articles on presence while trying to showcase what we do share the Word of either what we do at Aiven, what we do in the open source space, how we shape things. 

%8:72 ¶ 71 in [P7] Josep-Aiven interview.docx
%Because they were asking if we have some experts on X, Y, Z? And then it's like, yes, we have some. So of course, you can claim to be an expert, if you contribute regularly to a project, yes, kind of probably.

%-----------Expand brand beyond domain-------------  

Expand their brand beyond the domain that the company is known for 

&

P4, P11

&

``So people tend to think of [company name] as like the people who did [specific product]. And we've really moved beyond that'' [P11]

\\ 

% ``open sourcing these bits and pieces of whenever you can make it like tangible and like very transparent about \\ what was actually going on inside of the company. So that can that can help the deal of branding [P4]''

%2:56 ¶ 19 in [P4] Per-Spotify interview.docx
%just like branding, like that a company, what they do, again, what they do inside is not very visible on the tech side, like a like an app like Spotify, it's actually quite hard to understand, like, how it actually works, technically. And why should that even be interesting? For many doing point of view, if you're like a top notch engineer at Google, or Facebook or Apple, why should you work at Spotify? It's just like an app that plays music.

%9:9 ¶ 16 in [P11] Dawn-VM Interview.docx
%so people tend to think of VMware as like the people who did virtual machines. And we've really moved beyond that. And so we-- well our focus right now is really on building platforms and application developer tool sets that people can use on top of those platforms to build their applications. So we do that across all the clouds

%---------Be credited as founder of technology-------------

Be credited as the founder of an industry standard technology
 
&

P3, P4, P13-P15, P18

&

 ``Business that decides to release a project because they want to create a standard in the world around that particular software... %, like, you know, Google did with Kubernetes, it became the de facto \\ way to do container orchestration... 
 And the company that created it and released it has an advantage, because they are seen as thought leaders'' [P18]

\newline

``Google got a lot of exposure from like, creating Kubernetes, which is like the industry standard for like management services'' [P4] 
\\

%It could also be a business that decides to release a project because they want to create a standard in the world around that particular software, like, you know, Google did with Kubernetes, it became the de facto way to do container orchestration, as and so many other projects have become kind of the de facto standard for doing something. And the company that created it and released it has an advantage, because they are seen as thought leaders, people want to join them from a recruitment perspective.[P18]

%``the best way to do this is to make sure that what you are working on inside of your company... \\becomes an industry standard, then you gain all the benefits from being part of the standards. And if you\\ actually came up with this standard, then you \\ also get a lot of exposure. [P4]%Like Google got \\a lot of exposure from like, creating Kubernetes, which is like\\ the industry standard for like management services'' 

  %``If you actually came up with this\\ standard, then you also get a lot of\\ exposure... And is again it kind of\\ is like our desire to be able to be\\ a standardized piece of \\technology''[P4]
 

%  ``if you actually came up with this standard, then you also get a lot of exposure... And is again it kind of is like our desire to be able to be a standardized piece of technology''[P4]

%2:51 ¶ 37 in [P4] Per-Spotify interview.docx
%And if you actually came up with this standard, then you also get a lot of exposure. Like Google got a lot of exposure from like, creating Kubernetes, which is like the industry standard for like management services. Now again, and Spotify wants to kind of create the same kind of narrative for like internal

%2:90 ¶ 97 in [P4] Per-Spotify interview.docx
%h, yeah. And is again it kind of is like our desire to be able to be a standardized piece of technology instead of just an esoteric Spotify technology

%17:11 ¶ 88 in [P14] Harish RedHat Interview.docx
%And then the Linux kernel community, eventually settled on the model that we had also adopted. We already had adopted. And so guess what happens? The SUSE guys had to scramble, because now they have a kernel that is only using that type of threading which the upstream is not accepting. They say, yeah, it doesn't make sense. And so they had to scramble and I know of customers who were abandoned by SUSE because they were already going down that path, which was not the right path to go and SUSE couldn't quickly, you know, re engineer the stuff so that they can so they have to sunset that portion of the tech so that they can go into the newer threading model in the kernel. So. So that's the kind of benefit that we get as an example. So benefit, right?

%18:37 ¶ 298 – 302 in [P13] Tobie OSS consultant Interview.docx
%Yeah, it's a lot of money. Yeah. Um, and so that was just essentially by commoditizing. The layer under what it is that they were serving
%[R] So this is Facebook being able to save that amount by by standardizing
%[P13] by essentially open sourcing, sort of like the structure and diagrams of how they were making. 

%18:38 ¶ 296 in [P13] Tobie OSS consultant Interview.docx
%But let's say that they realize they could make a lot more money selling hardware than, than selling their software. Right. They could make the whole zoom thing, completely open source, for example. And make it work extremely well with their hardware. So that Sure, they would lose money and have like, fewer people actually paying for it, but they would become the de facto solution for all hardware, built on for, you know, whether like, regardless of who was using that system or not. 

%1:13 ¶ 187 in [P3] Omer-VM interiew.docx
 %the sort of street cred of the founders of React,

%1:19 ¶ 187 in [P3] Omer-VM interiew.docx
%get an entire class of engineers, who are now familiar with the tech stack that Facebook uses, right? Or at the very least, like one of the main web frameworks that they're using, that they have sort of home grown, right. And they also get viewed as, like a cooler place to work. 

%2:58 ¶ 37 in [P4] Per-Spotify interview.docx
%They, they want to leverage open source as much as possible. And the best way to do this is to make sure that what you are working on inside of your company, and the thing you depend on actually becomes the industry standard, not just something you maintain. Because if it becomes an industry standard, then you gain all the benefits from being part of the standards. And if you actually came up with this standard, then you also get a lot of exposure. Like Google got a lot of exposure from like, creating Kubernetes, which is like the industry standard for like management services. Now again, and Spotify wants to kind of create the same kind of narrative for like internal 

%12:16 ¶ 130 in [P15] JoshB RedHat Interview.docx
%But you know, have partners and customers that are using it. Yeah. And you don't want to piss them off. By terminating the tool? Well, if it's open source, you can just kind of gradually pull all your engineers off of it, and let them have it. Um, you know, so this is what's called the abandonware strategy. Now, that that's generally used as a derogatory term, but sometimes abandonware is good, right? Because it's something we can do with open source that you could never do with proprietary software, where you say, Hey, I know you guys are still using this. I know you still like it, but it doesn't make any sense for us as a business anymore. So here, it's yours. Now you can have it.

  
\bottomrule
\end{tabular}}
\label{table:visibilityTable}
\end{table}

%-------------General Visibility Quotes-----------------

%``Where open source is superior is it gives us superior visibility'' [P15] 

%``We decided to do is write a blog post... Well, a bunch of people started commenting on that. And it raised awareness of [company name] because we wrote it, but we didn't mention [company name], really, in that entire article... so it's actually a great marketing tool to talk about technology to use, even if it doesn't promote your, your own product'' [P8] 

%  ``Having companies involved can also help you with... marketing, right?... that helps give visibility to some of these these projects as well'' [P11]

%4:11 ¶ 27 in [P1] Georg-Bitergia interiew.docx
%hey hire us for the expertise. Because we have worked with a lot of open source projects, and foundations and companies that we are really good at knowing what the data is, what is available, what kind of questions we can answer.

%3:18 ¶ 73 in [P2] Daniel-Bitergia interview.docx
%contributing to open source is about demonstrating that you have the expertise and the knowledge contributing, and then you can serve specific pieces of knowledge.

%3:21 ¶ 73 in [P2] Daniel-Bitergia interview.docx
 %So then we can mess with your inner source effectiveness, let's say. But then at the same time, we are serving with the community. So we are defining inner source. So it's not that we know about inner source is that we are defining inner source and for this you can see that we have this pattern here about the maturity model, this pattern there about this or that and then we have these presentations at the inner source commons.


%[visibility/positioning in community]

  %\makecell{Some really \\ longer text}
  %the CEO of [large OSS foundation] announced in a keynote [city and name of conference]. We are now Running [OSS community]. And all of these companies and universities are part of this [company name] was there. So that was like great visibility, a really good milestone for us[P2]
  
  %You know, the other benefit that sometimes people see is you know, having having companies involved can also help you with I don't know, like, I guess marketing, right? So so we have, you know, we have relatively popular blogs and social media channels, and we can promote things about some of the open source projects that we're working on. And that that helps give visibility to some of these these projects as well. So I guess visibility is what I was talking about sovereignly I guess I guess it's marketing.[P11]
  
  %7:22 ¶ 47 in [P8] Johnathan-Goliath interview.docx
%And so we decided to do is write a blog post saying, here's all the things here's how you might want to use JSON, the JSON library in Zephyr, and oh, here's some gotchas we found. Well, a bunch of people started commenting on that. And it raised awareness of Goliath because we wrote it, but we didn't mention Goliath, really, in that entire article. And the maintainers of the Zephyr project, want to see how we can start to contribute back. So that material, so it's actually a great marketing tool to talk about technology to use, even if it doesn't promote your, your own product. And so we're seeing that by by proxy of doing it. 

  
  %So we have a branding, we have collateral point of bringing more people hire more people keep more people here. Get more engaged with more companies that might be the future partners. For example, last year, we had several conversation with Red Hat, who is a huge company about open source. And yes, he has a lot of money. And because of this conference discussion, right now, we had strategic discussion to do more business together. And everybody was everything was based in the first open source commit and open source community discussion. So yes, there is no money directly on those open source products, but it's your work to keep it that's why Zup is zookeeping it so with branding, it's kind of like that.[P6]
  
  %Definitely, I suppose in the marketing and branding level, that the marketing and branding folks always have, like, how is this contributing to our-- I don't work necessarily at that level. But I think that the level we do work at is we want to show up as in some of the ways I was talking about like humble, sincere, dedicated, more than a lot of time brand is about being like when I hear marketing and brand I hear, "we're the best at this look at us" you know, and I don't think that's what we're going for at all. I mean, occasionally, occasionally, somebody might be proud of something, but we're really looking to be part of something. Our brand is being part of something and not the best at something, [P17]
  
  %So we need to use other marketing efforts to be known and to be to be seen out there being being an open source company, I think, helps build like trustable branding.[P2]
  
 %12:3 ¶ 26 in [P15] JoshB RedHat Interview.docx
%The the second way that we're involved with with open source is community. As in open source projects, collect people around them, which are known as open source communities. And open source communities are a good way for us to reach people. It's a major portion of our marketing picture, when I'm talking about marketing with a big M. The and and they supply a lot of sort of key inputs to marketing, you know, both not just, you know, people to basically advertise to, but also as a way of learning what is it that people want, so that we can actually build those products? Right. Um, so marketing in terms of market research, and in terms of getting direct feedback? You know, like, if we're thinking about something new in Linux, we can put it in fedora and see how it does in fedora. Um, you know, before we have to offer five years of support for it.

  
  %7:13 ¶ 31 in [P8] Johnathan-Goliath interview.docx
%It's more about being able to have our voice heard, emphasis on priorities for our company, but also to drive the general shift. Like, because I participate in the marketing side, I'm able to make the marketing efforts better, like, "oh, maybe we should go talk to these people. Or maybe we should have this change the conference upcoming." And so I wouldn't have that opportunity if I wasn't a project member.

  
  %6:47 ¶ 159 in [P6] Otavio-Zup interview.docx
%We have several views in different countries for up to the company. So right now we have I don't know the number but 20,000 views around the globe. We have several countries don't have the right number right now.

%6:73 ¶ 159 in [P6] Otavio-Zup interview.docx
%And how can we engage money from the open source products basically first with brand thanks to the four open source products.

 %3:17 ¶ 73 in [P2] Daniel-Bitergia interview.docx
%we need to use other marketing efforts to be known and to be to be seen out there being being an open source company, I think, helps build like trustable branding.

 %5:74 ¶ 213 in [P5] Jill-SlimAI interview.docx
%Could branding make it easier? Sure. I mean, it might make it more easily recognizable. Yeah. Are there open source projects on market today that don't look anything like their corporate sponsors? Yes. Are there some that do? Yes.

  
 %3:19 ¶ 73 in [P2] Daniel-Bitergia interview.docx
%Yeah, so So the, the open source communities we are part of, we can see them as a way to position branding, to partially use their marketing efforts into reaching out other people that they need, for instance, metrics, or open source metrics, or they need to measure PlayerHealth. And then they go to, to grimoire lab-- to CHAOSS, and they say, well, there is there is a university running or so we I can I can hire them or I can hire, Bitergia for more, you know, professional services or industrial services. Good. So this is like a good place where you can go and fish for potential customers. It

%17:25 ¶ 16 in [P14] Harish RedHat Interview.docx
%So to answer the question is, what is in it for Red Hat? The brand recognition, oh, the guys in Red Hat, they advised us to try to do this. Yeah, they did help us out, we did not pay them all, we paid them something. Oh, do we didn't buy the subscription from them. All, we bought some subscription for them, because we were glad that they helped us, you know, so the leaves hopefully leaves a positive, a positive impression with the organization. Because they have a lot of them, they immediately start comparing us against, for example, I'm picking the name because a very common name, Oracle, or Microsoft, or they just want to, they just come here to make money. They want my money, they want to do this, No, they won't give me anything else. But when they come here, they also want my money, but they also give me other things. And it's okay, even if I don't, even if I don't subscribe to their services, even if I don't buy their service, they're still friendly to me. Whereas the other guys, Oh, you didn't buy from me, I'm walking away. So it's a very different, it's a very human thing. I mean, it's not a it's not a win, win lose environment, it's a win win environment. So you know, in order for the other side to say, oh, I need to win this customer, that means the customers lose the money to me.

%19:19 ¶ 65 in [P16] Stephen Transcribed Interview.docx
%doing good, looks good. And looking good, has, you know, has a potential positive effect on your bottom line? Right. So I think, I think that's definitely one component of it for for enterprises. Participating in open source, like, it looks good to be doing good. 

%19:20 ¶ 65 in [P16] Stephen Transcribed Interview.docx
%making sure that we're, we try to be equitable with the way that we've maintained some of these things within, within smaller parts of various foundations. But also realizing that there's a component of it, that you have to show people why they should care about you.