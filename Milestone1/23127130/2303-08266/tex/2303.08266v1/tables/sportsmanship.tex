\begin{table*}[bht]
%\footnotesize
%\scriptsize
\caption{OSS Sportsmanship: lessons learned}
\vspace{-1.5mm}
\begin{tabular}{p{0.2\textwidth}p{0.1\textwidth}p{0.6\textwidth}}
\toprule
{\textbf{Takeaway}} & {\textbf{Participants}} & {\textbf{Exemplary Quotes}} \\ \hline

%\textbf{Play by the community rules:} The community needs take precedence over the company's needs 
%\newline
\textbf{Don't bulldoze your way into getting what you want:} The community needs take precedence over the company's needs 
&

 P5-P8, P11, P14-P20
&
``You really have to think of the community first...we can't just bulldoze our way into getting what we want. We have to think about how what we want benefits the community'' (P11)
\newline
``We try to pick the things that would be having a bigger impact, bigger benefit for the whole of the community'' (P7) %... The main chunk of the work we do is what can we do for the community that will help them'' 
%\newline
%[This is mentioned in the companies code of business conduct] 
%``So if (a) is right for the project, because that's what you need to do, even though it's negative to [company name], yeah, it doesn't matter. It's important that the project succeeds'' [P14]
\\


%And that could be bias in the project, also about accepting contributions from certain companies or, or to protect, you know, their own direction, or they may see companies as pushing the company agenda and not thinking about the project in general. [P18]

%[boarder security questions]
%Motivation? Well, I mean, excellence in engineering, I think, you know high quality software, I think we want, again, to be a humble participant in the ecosystem of which we benefit. And excellence in engineering, especially around security, you know, we're part of the OpenSSF, we find a lot of security initiatives and in projects. Yeah, those are the two main things anyways.[P17]

% And I think that is what the first thing that organizations need to do in order to be a good open source citizens is to understand open source, and how to work in open source because it's completely different from the product, the proletariat somewhere, like, they don't have a community at all, that didn't open source all the project had here, here, it works with all the community in consensus and, like more democratic.[P19]

%And then coming back inside the company, you're talking about: How do we work successfully with open source, right, and what the norms are in the community. And from an executive leadership perspective, you're talking about how it aligns with the company's objective, and business goals. [P18]

%And, and so I think there is definitely an aspect for being heard the right word is like not letting go of your hubris. And just focusing on what the project needs and being sensitive to the fact that there's a group of people already in place that are doing this work. And understanding how you can support it, if this is something that you you are you find valuable. So I think it's definitely, again, a lot of individuals making their own way. And if companies want to get involved, it's really about empowering the individual. And trusting that their individual will choose the supportive way in the project versus dictating their role in the project. [P20]

%Because I don't necessarily think that the companies always have the best view of that if their employees are the ones that are in this space in the community in the project, then trust them to to contribute in the way that makes sense to the project, which may or may not always be, again, what the company would want for beta. [P20]

% 8:20 ¶ 43 in [P7] Josep-Aiven interview.docx
% So we don't enter that magic conflicts, of course, when sometimes we want to have some feature. But we do usually we discuss and we showcase that the benefits of this feature. That's why we think we should add this one. And then it's a discussion back and forth, explaining the benefits, try to reducing the risks or problems of that solution and trying to reach a better one.

%5:37 ¶ 223 in [P5] Jill-SlimAI interview.docx
%these habits of saying, like, are we really doing right by our users regularly

%And so when you think about, like how that is delivered to developers, today, we have communities, we have a DockerSlim community, we have a Slim SAS community, and the internal team at Slim, is responsible for cultivating both with equal level of we'll call it responsiveness care. And, and, and not just responsiveness and care. But like also like, curiosity. You know, one of the things that's really cool about having an open source community is is you learn from them just as much as they learn from you. And, and so that that's woven into how we operate the business, is taking the user feedback, taking the user learning, taking the user, the new problems that users want to solve, like taking all of that and ingesting that back into our product delivery process. And like, really understanding how we're refining what we're delivering both on the open source and in the SAS, those things are cohesive.[P5]

%5:80 ¶ 227 in [P5] Jill-SlimAI interview.docx
%You do need to communicate to them regularly, in a way that brings them value. Right, and you do need to communicate them to them regularly. So that you can hear what's bringing them value. Like there's got to be that that that two way line of communication

%5:19 ¶ 119 in [P5] Jill-SlimAI interview.docx
 %What we're doing is we're taking the time to be insanely focused on the value that we're delivering to users so that when we do turn that on, we have extreme confidence that we're turning it on at the right time for the right things. Right. So it's essentially doing user research, kind of like you're doing your interviewing right now to validate your hypothesis. Right, and also to be subjected to things you're not considering. 


%6:25 ¶ 151 in [P6] Otavio-Zup interview.docx
%We did not offer. Like I said, our we don't have a business model for those products. So we don't use them to offer services, we only we only invest on their development and so improvement, so we were not making service for money, we are not relating the investment we do to those corporates to achieving some results for the company, we are not related to for money results are not related to how we can have clients to them. So we are strictly related to community. So we asked for it for companies that are wanting to use it, or people that wanted to use it, to use our forum use GitHub to share what they need or features they need, but we we don't have a roadmap and see how we make we may make money how we make make, make service or offer service using so we don't make it we No don't offer Consultancy Services today, using our open source products. And we are not considering to offer service to other companies by using our open source products. So right now, they're not considered for us for anything else than to contribute to the communities.

%And that's something that we should, companies that have the means to provide people to work on open source should be doing those things. So that's one of the things that we tried to do. So we don't enter that magic conflicts, of course, when sometimes we want to have some feature. But we do usually we discuss and we showcase that the benefits of this feature. That's why we think we should add this one. And then it's a discussion back and forth, explaining the benefits, try to reducing the risks or problems of that solution and trying to reach a better one. [P7]

%7:9 ¶ 23 in [P8] Johnathan-Goliath interview.docx
%And we're also contributors, so we joined the board. So we joined the project, we contribute money, but we also contribute time and have individuals on our team who contribute to Zephyr even if it's not directly relevant to take a life as a company. We just want to see the project expand and grow. And so therefore it's strategic to us to see that to invest in the community.


% So it's really challenging, because in all of your work in open source projects, you really have to think of the community first and the project Second, and this is or sorry, and the company second. So project first company second. Because we can't just bulldoze our way into getting getting what we want, we have to, we have to think about how what we want benefits the community as well, and work on things that are going to be of mutual benefit to both. [P11]



% And so Tidelift's view is that for open the open source ecosystem to be healthy, both for the maintainers and the communities as well as the downstream users, we need to be creating predictable income flowing to maintainers. in those communities so they can plan around it and have the time that they need to do the work that is being asked of them.[P12]

%12:3 ¶ 26 in [P15] JoshB RedHat Interview.docx
%The the second way that we're involved with with open source is community. As in open source projects, collect people around them, which are known as open source communities. And open source communities are a good way for us to reach people. It's a major portion of our marketing picture, when I'm talking about marketing with a big M. The and and they supply a lot of sort of key inputs to marketing, you know, both not just, you know, people to basically advertise to, but also as a way of learning what is it that people want, so that we can actually build those products? Right. Um, so marketing in terms of market research, and in terms of getting direct feedback? You know, like, if we're thinking about something new in Linux, we can put it in fedora and see how it does in fedora. Um, you know, before we have to offer five years of support for it.

%19:26 ¶ 101 in [P16] Stephen Transcribed Interview.docx
%And I think they're the the component of again, like being that good open source citizen. I've seen so many instances, across companies, where people just, they'll come stomping in, you know, to an open source project and want a thing or expect a thing at a deadline, you're like, cool, come do it yourself, then you can have it whenever you want it. Right. So like figuring out how to know like, and, and open source timelines, deadlines, if there are any, are never going to be that closely aligned to your internal deadlines, right. So like, being patient, and knowing that, you know, if you are here to consume, you should, you should probably step back and think about how you're going to do that. Right.

%And the community is like, equally consulted on decisions, which has taken them in different directions and they already-- than they originally might have gone. There's like one example I think of, Microsoft wanting to put telemetry in the repo and the community was like, "No way!" And so they're like, "Okay." So you know, like that. But, you know, the act of that decision really helped build trust. [P17]

%-------------------------------------------------------


\cellcolor{gray!15}\textbf{Prevent maintainer burnout:} Participate in maintenance tasks 
& 
\cellcolor{gray!15} P7, P17, P18, P20
& 


\cellcolor{gray!15} ``Something needs to be shiny, for you to do it on your free time...companies that have the means to provide people to work on open source should be doing those things'' (P7) 
%\newline
%``learning how you can help and we often talk about that chopping wood, and carrying water thing... sometimes the most useful things you can do are the non-glamorous things... humble participation is most important'' [P17]
\\
  
  %And I think, if you've been following any of the software, supply, chain reporting, and security, just all these sorts of quotes that like 95 to 97% of companies use open source, but how many of those companies are actively involved in maintaining the things that they use and have a vested interest in it. [P20]
  
  
  %And I think it's often that sort of central support is the least glamorous work. In terms of say, working on just maintenance and operations and working on community management tasks and logistics, connecting people and doing the sort of, again, more community focused efforts that are not our, I'm gonna say it's harder for some companies get involved, depending on their incentives. [P20]
  
%"we hear this thing ... like some maintainer, it's burnout ... leaves the whole thing behind just quits everything in the world and says, I'm, I'm done. So we want to, we want to try to avoid this thing. We want to support them to give people time. So they don't need to carry that burden alone, that will be more people to carry part of it. And if you spread it, then probably it's bearable." [P7]

%8:16 ¶ 43 in [P7] Josep-Aiven interview.docx
%So we want that all the projects that do we do operate and we do manage, they do support this thing natively. And it is something we would like to contribute to many projects. Also we are in the cloud, meaning that everything you transfer costs money and we want to reduce the bill of people. I mean, we don't want people to pay just because because so again transferring a lot of mega bytes on the wire costs money on the on the cloud. If we compress this thing massively on already on Origin, suddenly the bill gets lower just because we did once tiny, small thing on the software, the these are the features that sometimes we want to bring. And most of the times they are sensible in that sense. So there is not much of a discussion.


% In fact, I'd say that we encourage humble participation, learning how you can help and we often talk about that chopping wood, and carrying water thing that you'll you know, you hear in open source, you know, to-- I've run workshops around that, that exact thing. It's like, sometimes the most useful things you can do are the non glamorous things. So like, well, yes, like nails in the leadership position was very visible, some of the best contributed, we emphasize that and like the CNCF has a chopping wood and carrying water award around that. So we also emphasize that so. So both, but I think more so, you know, being humble, it's not like I'm from Microsoft, I can do this. I can take over this. And it's like, no, like, you know, humble participation is most important.[P17]

%And, I mean, companies can do a lot more here as well, if it's really a valuable project, and it's struggling for maintenance. Companies should help, you know, with the maintenance and not just the feature functions, but the heavy lifting of code reviews and maintaining the project and doing the security and the plumbing.[P18]
  
%------------------------------------------------------- 
%\textbf{Put your money where your mouth is:} Where you consume, contribute 

%\newline


\textbf{Keep the contribution wheel spinning:} Where you consume, contribute 
& 
P4-P9, P12, P20
& 
``Everyone should be contributing back, everyone should be making sure that they are not just the receiving end, they are receiving and producing end of the open source'' (P7)


``We generally believe there's a contribution wheel, so we owe, if we can upstream something or contribute back to a project we use, then we will'' (P8) 
\\
  
  %But there is definitely a mentality shift of a technology consumer versus a technology creator, in the sense that if this is an older company, they're used to just paying for money for a thing, there's no expectation that they need to go back and improve the thing. And when they were consuming open source, either directly from the community or via a product version there, we're still this sort of like, well, I'm just going to use it, I'm just a user and like, and submit feedback or feature requests, but like, there is no expectation of dedicating time to improve it. And I think, in general, we haven't really seen the consumer become the contributor is much like, I think we've seen individuals that are working in it, and the engineering side that find their way in and coming more so from technology vendors, or new technology companies. And it's, it's, it's, I don't know, I haven't really seen the end of the funnel coming back. And in terms of actively recruiting that. [P20]
  
  %And but yeah, what I'm supposed to do here at Spotify is to basically streamline how we open source internal code and how we we drive these projects internally, but also how we work with dependencies, open source dependencies, and we support these projects, and so on. [P4]
  
  %5:26 ¶ 69 in [P5] Jill-SlimAI interview.docx
 %And I think that's another thing about like, we call it our virtuous cycle, is our virtuous cycle is giving back into the ecosystem of developers by always providing them value, and then encouraging them to do the same.

  %We like we like to use virtuous, like, it's virtuous for us to do these things. When you put good out, you get good back.Right? It's, it's this kind of like infinity loop of, you know, you, you attract what you put out. And it continues. So if we put out things that are false, we're going to get back things that are false. Doesn't mean that we can't make mistakes. That's not being false. Right? We can, we can hit we can hit bugs, right? [P5]
  
%``We try to spread the word about open source, because we all use it, right? We, we use it all the time when we're are creating software. So we try to help bring this awareness about open source to everybody.'' [P6]

%8:16 ¶ 43 in [P7] Josep-Aiven interview.docx
%So we want that all the projects that do we do operate and we do manage, they do support this thing natively. And it is something we would like to contribute to many projects. Also we are in the cloud, meaning that everything you transfer costs money and we want to reduce the bill of people. I mean, we don't want people to pay just because because so again transferring a lot of mega bytes on the wire costs money on the on the cloud. If we compress this thing massively on already on Origin, suddenly the bill gets lower just because we did once tiny, small thing on the software, the these are the features that sometimes we want to bring. And most of the times they are sensible in that sense. So there is not much of a discussion.

%7:15 ¶ 39 in [P8] Johnathan-Goliath interview.docx
%We use a lot of open source. And in the process of using that we tend to contribute to things we do use.

%7:6 ¶ 19 in [P8] Johnathan-Goliath interview.docx
%And for our own purposes, we will implement a feature, or a let's see a library around the open source project. And then we invariably try to contribute that back whenever we can. Whether it's because it's useful, we think we want to contribute that back to the project. And sometimes it's pragmatic, like, well, it's a small enough feature enough, people will use it, but we don't maintain a fork. So let's just get it upstream. So you don't have to maintain the fork. But also, we just think it's useful, and we want to contribute it back to the community. So again, that's the, for the stuff we run in production, we use open source whenever we can, sometimes we contribute back to improve the product, sometimes we do to make it better.

%So again, you're not able to build in community. And if you're not able to build in the community, then the benefits of open software development and open source are completely zero. There's there's no special benefits to things in the open. It's nice. But it's... there's no there's no reciprocity in some ways, even small, then you're missing the point entirely. [P9]

%So there's there's one possible challenge, but it's more of a problem. That's the only challenge maybe free riding. Which is if you do free riding. Which means when you have contributors, well, users that that eventually abuse and free ride without ever contributing back.[P9]

% but we still participate in events and community forums as much as possible. So conferences, books, I don't know, I mean, I've got open source books on my shelf, but I don't know if that's me consuming it or type that consuming it? You know, we, as I mentioned before, in terms of, more or less on the consumption side, more on the contribution side, we sponsor events, we sponsored programs like Outreachy, which are, you know, paying, paying for internships to work on OpenStack. So we, it's a fairly standard for any company that's pretty invested in open source. It's a fairly standard set of of interactions between us and open source communities. [P12]

%companies start realizing they need to give back for reasons of de-risking their supply chain, to a point where companies starts giving back because they see that it actually helps them strategically, technically, because suddenly they've got influence in projects that they rely on. Ultimately, and like open source enlightenment, from, from a corporate perspective is basically companies that are really making investments in open source that benefit the whole ecosystem. Because they know at this point, that a healthy ecosystem means good business for them, too.[P12]

%-------------------------------------------------------
%\textbf{If the community is happy everyone is happy:} Build a good relationship with the community 
%& 
%P5, P11, P13, P15, P16 
%& 
%``What kind of relationships with community you have is also important'' [P13]\newline
%``We're inviting experts from other ecosystems like <company name> and <company name>, like we are coming forward with, we don't just want to solve the problem for the developers that we solve. We want to understand how developers and other ecosystems are solving similar problems in different environments'' [P5]\newline
%``The conversations are so detached, sometimes from the community, that's that you you should be you should be, you know, kind of, you know, peeking through the forest and seeing what's actually happening on the ground'' [P16] \\

%And, and, and not just responsiveness and care. But like also like, curiosity. You know, one of the things that's really cool about having an open source community is is you learn from them just as much as they learn from you. And, and so that that's woven into how we operate the business, is taking the user feedback, taking the user learning, taking the user, the new problems that users want to solve, like taking all of that and ingesting that back into our product delivery process. [P5]

%You know, I mean, unless you're an expert in that particular field, or you're spending money on that, not much. However, if you have a sense of the community that is around it, then suddenly you can gather a whole bunch of other information, well, oh, look at that. There's like eight other universities in the US, similar as mine, using that software and contributing to it. That's extremely useful information.[P13]

% ``We do engineering, either by ourselves or together with contributors who work for other companies or no company. The second way is considered the ideal'' [P15],

%-------------------------------------------------------  
\cellcolor{gray!15}\textbf{Prevent corporate abandonment:} Open source projects that your team cares about
& 
\cellcolor{gray!15} P4, P17
& 
\cellcolor{gray!15} ``We do want to open source things that play into our own brand of the company...If it's something that is a core part of our teams' function, that's also great because it's something that they [we] will continue working on'' (P4)
%\newline
%``it looks terrible when there's like abandoned repos, if a company just puts something... looking at the upstream, making sure that your engineers are not just, you know, we're very mindful'' [P17]
\\

%2:26 ¶ 151 in [P4] Per-Spotify interview.docx
%e is some some sort of reason why you actually open sourced it, beyond just like personal growth, and actually becomes an integrated part into, like, why is the team considered successful? If it's not, then it's going to be left by the wayside at some point, because they'll be more important things. But if it's actually an integral part of the, the objectives of the team, then they will also continue to be maintained.

%And then you could say on the other end of the scale is like things we do want to open source is like, especially things that play into our own brand of company. Like if it's something about music and technology that would be amazing. If it's something that's a core part of our teams function, that's also great because then it's something that they will continue working on[P4]

%just know what you why you're doing it, you know, just coming out and be like, Oh, we open source something. So that's really important. And then like, it looks terrible when there's like abandoned repos, if a company just puts something else that's around release, around use, again, is like, yeah, looking at the upstream, making sure that your engineers are not just, you know, we're very mindful, a lot of times we have compliance software to help them be mindful.[P17]

%-------------------------------------------------------
\textbf{Have an abandonware strategy:} Transfer ownership of OSS projects that you internally moved away from but the community cares about
& 
P4, P6, P10, P15, P17
& 
``If we stopped caring about them [open sourced projects that the company moved away from], then we should transfer it to people who care about them'' (P4)
\newline
%``We have a project that is actually a big success, like open source wise, but we don't use it internally anymore, and we don't really care about anymore... But outside of [company name], it is, it's a perfectly fine living project with a lot of different contributors'' [P4]
%\newline
``Eventually, you stop using the tool internally. But you know, have partners and customers that are using it...you can just kind of gradually pull all your engineers off of it, and let them have it. Um, you know, so this is what's called the abandonware strategy'' (P15)
%... where you say, ``Hey, I know you guys are still using this. I know you still like it, but it doesn't make any sense for us as a business anymore. So here, it's yours. Now you can have it'' [P15] 
\\

%2:83 ¶ 61 in [P4] Per-Spotify interview.docx
%But outside of Spotify, it is, it's a perfectly fine living project with a lot of different contributors. It's kind of a good tooling for like, kinda like data pipeline thing. And we still own it, we still has our name on it, but we don't use it. And we don't contribute to it at all. It's like done by all these other engineers who kind of just like driving it now. But we have no idea what's going on inside of it.

%So not just related to the communities, not just improving how we know as a technology creator and maintainer, but also investing on the concept of open source ecosystem, and how we could work with them to improve this product instead of just freezing it or just archiving it.[P6]

%had about between 35 and 50, people whose full time job was to work on open source projects, their full time job was to work on open source projects that the company used. So they were committers to a bunch of Apache, basically, Apache projects, for the most part. And their job was to continue to grow and evolve those projects, because the company used those projects. And if some one of them left, we would hire another person to fill the role. Because Because, right. But as it turns out, that isn't as common. More common, is an individual or a small team has a passion about a project convinces their company to let them and if they leave, the company won't replace them. Or if they replace them, they will replace them with somebody, but that person might not contribute. So corporate abandonment of open source is a good signal between is a good way to differentiate between companies that are truly recognize their dependency on open source, versus companies that merely allow a passionate employee to do something that's safe enough, but when they leave, they're not gonna backfill. [P10]

%And it's okay, often tell, tell projects, like, it's okay to re-evaluate [in 2 months?], that's not working or priorities change, we had a re-org, you know, if we had a re-org, and we don't have resources for this, then have a plan for sunsetting that project and talk to the community and, you know, like, offer forks and those kinds of things. So it's, there's a lot to it, but just baby steps and think about where you are now, where do you want to go and revisit?[P17]

%-------------------------------------------------------
\cellcolor{gray!15}\textbf{Avoid company monopoly:} Invite other companies in 
& 
\cellcolor{gray!15} P3, P5, P7, P11, P15, P17
& 
\cellcolor{gray!15} ``Part of the healthiness of the project is also making sure that there are different ideas, opinions, and points of view, it's really easy to fall to a kind of monopolistic way'' (P7) % the best things of the open source is that it's developed in the open with everyone'' 
%\newline
%``For a very successful public open source project, that is one where you have a public community of people contributing and involved, who work for all kinds of different employers, or no employer'' [P15]% or, you know, all over the place'' 
\\
  
  %1:16 ¶ 227 in [P3] Omer-VM interiew.docx
%Yeah, sometimes it happens like that. Sometimes it sometimes it happens where one company is building an engine, and they're like, come help us build an engine. And sometimes, that works. Kubernetes is a great example of that. Google open sourced it, and then invited a bunch of friends along Did you know there's a Kubernetes documentary by the way?

%you need to get you need to get you need to get multi vendors like multiple vendors involved so that you can a) like improve your product because you have to like be accepting opinions from lots of different people and then b) you get more market share that way.[P3]

%1:32 ¶ 203 in [P3] Omer-VM interiew.docx
%Google reached out to a couple of vendors. They got together on an off site. They made a bunch of decisions. They started building stuff. That's how they got involved. 

%We're inviting experts from other ecosystems like Shiny and R, like we are coming forward with, we don't just want to solve the problem for the developers that we solve. We want to understand how developers and other ecosystems are solving similar problems in different environments.[P5]


%8:48 ¶ 51 in [P7] Josep-Aiven interview.docx
%So they wanted to have a backing up of different companies as among them, and we tried to cooperate, like in a, like regular basis, and we discuss and we tried to agree on what to bring what, what makes sense to drive it to

  %"But again, the best things of the open source is that it's developed in the open with everyone. So everyone, and we want to be part of the everyone, we want to make sure that it's everyone. And we don't want to just to be said that. Okay, yeah, it's only one company behind because nobody else is interested. No, we will be interested, we want to be behind those things want to have an opinion, we want to try and help and shape those project, the best way give our opinions and, and do our best to bring it to the greater success" [P7]
  
  %"I mean, we are there to contribute on that community of these projects to make the project better, more sustainable, to bring more different ideas on the table as well. So not just one single companies dominating one open source project, but we try to bring we are another companies that we try to also bring opinions there. To not just depend everything on one single company and to create a diversity of opinions, ideas, companies behind and all this stuff. So also to support and just stay true to the open source philosophy." [P7]
  
  %And you get a real innovation boost by pulling in people who work from other companies who will use things differently than you do a lot of different different use cases, they'll have different skills to contribute, they'll have different ways of thinking about things. And so we see that we see that innovation as being something that is really absolutely critical to us being able to do what we do you know, you without something like Kubernetes, we, we wouldn't have been able to spin up this Tanzu product line in a relatively short period of time, considering, you know, how long it normally takes to build a product at that sophistication, but, but by having something like a Kubernetes, we can just innovate on top of that, because we have, we have the platform.[P11]
  
  %where we do engineering, either by ourselves or together with contributors who work for other companies or no company. The second way is considered the ideal. Right? If we can make that happen, we do make that happen. Because, you know, why would you put in 100% of the effort if you didn't have to, plus, getting contributions from outside Red Hat often bring with them perspectives from outside Red Hat, which allow us to make products that fit a broader clientele. [P15]
  
  %But I mean, I think, also going back to like, Microsoft, and collaboration, like, you know, when we have our performance reviews, one way that we're at one thing that we're asked about is like, how did you build [on with others?] And it's like open sources--or inner source-- Is a great answer to that question. So it's, it's, you know, the fact that it's inner diversity, inclusion is another core pillar for, you know, our reviews and performance reviews, I think that collaboration is at the highest level of your review, I think says something around how important that is. [P17]
  
  %So the more that we're able to collaborate with external communities, the more perspective, the more, you know, kind of like user stories or use cases or whatever word you want to use there. The better the product you're building,[P17]
  
  %I'd say on the releasing open source, like: always have a plan for-- or goals for external collaboration, like releasing open-- [P17]
  
%-------------------------------------------------------
%\textbf{Don't pull the rug out from under your users:} Go from close to open never the other way around 
%& 
%P1, P9 
%P2, P3, P7-- talk about elasticsearch example
%& 
%``We also have customers that require it to be open source. And so even if you wanted to, well, if you wanted to make it proprietary and not open source anymore, then we would lose some of our customers'' [P1]
%\newline
%``That's really the problem... you see, as a prediction of many companies involved in open source, they may start with some kind of project, sometimes with a permissive license. And then they go through copyleft license with contributor license agreements, and then they move to the proprietary'' [P9]
%\\
  %4:30 ¶ 71 in [P1] Georg-Bitergia interiew.docx
%have customers that require it to be open source. And so even if you wanted to, well, if you wanted to make it proprietary and not open source anymore, then we would lose some of our customers.

  %"It's been always brought forward as one of the primary reason why they were doing this switch to proprietary licensing is that there was free riding predatory practices...When you do a lot of the code, or when you produce open source tools, ...You are making gift. And it's a bit difficult to say, Yeah, well, making a gift that I really don't want everyone to benefit from the gift."[P9]
  
  %10:69 ¶ 224 in [P9] Phillipe NexBe Interview.docx
%were very clear about what were the benefits of being open source on a budget. So now, the ratcheting effect to go from permissive to copyleft to proprietary, I think is a terrible thing from a [integral?] and community perspective, because it's, it feels like you're pulling the rug under the feet of your contributors. as you're making each time you're adding extra extra requirements. But going the other way around. Nobody ever complained.

  
%-------------------------------------------------------
 %\textbf{Understand it's a marathon not a sprint:} Understand it's a long term investment 
 %& 
 %P3, P7, P8, P11, P13-P15 
 %& 
%``And sometimes the value is really thin or vague. Or it's something that it will come on long term, not as a short thing. It's not 100 meter sprint, it's a marathon'' [P7]\newline
%``It's seeing the long tail of their investment... We spend a lot of money to make sure that developers like us. And like that is a decade's long investment. Right? We're not reaping benefits on that in six months, then that is that is like a long tail investment for <company name>. And it's a great investment'' [P3]
%\\
  
%-------------------------------------------------------
%\textbf{Go the extra mile} in being present and involved 
%& 
%P7 
%& 
%``We obviously try to be present everywhere we can...We try to go to conferences, we try to write articles on presence while trying to showcase what we do share the Word of either what we do at <company name>, what we do in the open source space, how we shape things'' [P7] 
%\\
  
%-------------------------------------------------------
%\textbf{See beyond the short term competitive advantage:} Open Source projects you care about 
%& 
%P10, P13
%& 
%``You seen that kind of use cases, when folks are thinking about open sourcing things that would otherwise give them a competitive advantage? Right. So the idea is, is this giving me a competitive advantage? Can I get more value? Out of that, ...by using it in a way that makes it more open?'' [P13]
%\\
 
%-------------------------------------------------------
%\textbf{Be there for the right reasons:} Contribute for the right reasons and everything will follow 
%& 
%P7, P14 
%& 
%``So but again, it's not for the sake of that reputation. If you do it only for that it doesn't hold, it will not last long. Things will be seen. And that's not also a sustainable way of doing it. You need to do it for the for the real reason to basically support the project. And if that happens, then then then everything comes up'' [P7]
%\\
 
%-------------------------------------------------------
 \textbf{Be a polyglot in promoting OSS:} 
Speak risk management, engineering benefits, legal \& compliance, and sponsorship 
& 
 P7, P10, P14, P16-P19
& 
 ``An interesting challenge for a successful open source program in a company is to be able to speak multiple languages, to speak to the risk legal people about risk strategy and, you know, compliance, to speak to the engineers about the benefit of being in the goodwill...and then to speak to the CFO, about, here's how we save money, here's how we make money'' (P10) %we save money on these proprietary licenses, and on the tech debt... and all those are the three conversations, the fear, risk management, the love, community, goodwill, and the money are all part of the conversation. So an OSPO has to negotiate between'' [P10]
%\newline
%``The OSPO will be successful, if they manage to convince or to make sure that the philosophy of open source is spread across the company'' [P7]
%\newline
%``there are people who've never worked in open source before and who...are learning or might be fearful. So, I think that's the big picture, but that there are lots of little individual feelings, stories, people have different levels of learning'' [P17] 
\\
%\hline

%ou first need to convince the legal team that you can do open source and the legal team needs to make sure that all the security risk are covered. And all the open source compliance is in place. And all the software inventory is also managed and automated. So that is the first thing. Because if you don't assess security, and you cannot move forward, and once you have this big block, that might take more, or might take less, depending on the hospital under specific issues and, and barriers, then we can move to the community driven states that is more like, Okay, I have everything I know, my company has like, for instance, these as both of these are different inventory, the opens of praise, where I can contribute, great, how can I contribute is when the community location states happens and building internal ambassadors to try to educate on the different teams the importance of open source and how to collaborate in open source, then we'll come more like the engagement phase, like once they know how to do it, it's time to do it as part to contribute to the open source projects, or maybe building stuff and thinking, Oh, we should be open source this project, I think this is really important. And to the point to then be this leaders are advisors of open source of the organization's technology stack. So there are different phases, and it takes time, depending on the organization and the factors. But I think like this chronologic word Yeah, like this order. Makes sense. Right to, to advance smart, easier in this journey.[P19]

%Absolutely, absolutely an OSPO leader needs to be very skilled at talking to leadership, to middle management, to developers, and to the outside world. You're really creating a messaging to the outside world on on companies involvement of open source. And then coming back inside the company, you're talking about: How do we work successfully with open source, right, and what the norms are in the community. And from an executive leadership perspective, you're talking about how it aligns with the company's objective, and business goals. And with middle management, you are hoping that by talking to the executive leadership that we've created enough of air cover for managers, and incentives for managers to make time for their people to do open source. And they're rewarded for doing open source and they're budgeted for doing open source, right. And even when you release a project, we want to make sure that the developer has some time set aside to maintain that project to review issues, accept PRs, review code. Otherwise, it's it's just dumping the project and running and the reputation that the company gets is no one pays attention to the project. Don't get involved in that project. Well so. So it's a lot of hard work. That's one of the hardest tasks, I would say the OSPO has: is working with different levels of leadership. And showing how open source helps the business. And why to set aside time and money to work with open source.[P18]

%That's interesting as well, I think I think for these like different product problems, you already mentioned that we are very bad at actually releasing something and then continue caring about it the next few weeks, the actual benefits of open sourcing it. Because I think we do actually invest a substantial amount of time and actually producing open source code, but we are terrible, actually getting any tangible benefits out of it, is one of them. The other one is like, Spotify is very happy with the success of Backstage and they do want to create more of these kind of successes and, and having more like find go down through like having some sort of formula for how to do this instead of like, just by accident. And, and then I think the last time this was all that's according to tech recruiting in the US, especially, Spotify is seen as a more boring company, than compared to like Apple or Microsoft or Facebook or Netflix, for instance. It's not as it's not.. it's seen as like, boring, basically. It's just a music app. Yeah.[P4]

%8:101 ¶ 111 in [P7] Josep-Aiven interview.docx
%Yeah, I think they they help on deciding those things on buffering those ones as well. So because I guess one of the worst things that can happen to a team is to know that they are in trouble in that sense, or that they will be scrutinized this way. So having those OSPO teams that they try to define those frames, they do all the evangelism at company level, on why, what, what to expect, what's the rhythm, what to not expect, and how these things work. So that's the, that's why you should have probably an OSPO, because then it's concentrated there, it makes sense. But then again, the OSPO should not monopolize all the open source contributions on the OSPO team. They should also try to make sure that all the company understands and does believe in that sense, the open source way, and everyone should be able to contribute. And everyone understands what it means. And it's not something strange or alien to the company, they should make sense should make sure that it belongs to the company. And that is when the OSPO will be successful, if they manage to convince or to make sure that the philosophy of open source is spread across the company.

%And I think that a fair analysis of contribution takes into account that the OSPO or somebody in the company needs to be able to make the call and say, Yes, please do that. We encourage you because it helps or no, you can't do that as an employee of the company that that violates a law or that violates company interest. And why would we pay you to do that[P10]

%And an interesting challenge for a successful open source program in a company is to be able to speak multiple languages, to speak to the risk illegal people about risk strategy abatement and, you know, compliance, to speak to the engineers about the benefit of being in the goodwill that one engenders in this, and then to speak to the CFO, about, here's how we save money, here's how we make money, we save money on these proprietary licenses, and on the tech debt, and on these, whatever, we make money through an ecosystem play, or through getting people to be dependent upon our format, which will then upsell them to tools that they might need that are more resilient, or support or training or, or whatever. And, and all those are the three conversations, the fear, risk management, the love, community, goodwill, and the money are all part of the conversation. So an OSPO has to negotiate between. And if you use the wrong conversation, if you go to the CFO with a love conversation, she'll look at you and say, I don't care, right? In fact, it's my job not to care about that. And if you go to the engineers and say, We're gonna make money, and they're gonna say, Yeah, fine, but you're using us to make money and you know, does it fall through? Right? So you have to, you have to recognize that there's a multifaceted game going on here. It's not just about people who love each other and want to give each other code to help it at. That's true, but it's not the full story at all.[P10]

%You know, it's, it's a very, very different conversation... because we have this and so OSPO tries to take the essence of all of these ideas, and we try to encourage other organizations to do the same. And that's, that's the other part of the work I also do is talking to other other companies like it could be a manufacturing organization, it could be a hospital for that matter. I you know, it doesn't matter who they are. They are not in this industry, right. But they use technology, they use technology in every every part of their business. How can they be part of the open source community? What would take them to become part of that? What would it be? Sometimes they don't care, which is true, which is I don't care. I mean, I just want to get my job done. And once once I'm done with a job with this technology you provide, I just walk away, I'm not okay, that's fine. But there may be others who may say, hey, yeah, I think this is interesting, I want to improve some part of my organization, my whatever. And then once I improve it, make it better. And, you know, whatever. I want to contribute this back to the open source community. Sure, how can I do that? Oh, okay, we can help you to think through the issues, think through the the needs and responsibilities and expectations so that you don't go down the path, which is a problem, because we do have organization say, oh, you know, I didn't realize that this is going to be a lot more work. I thought, I just have to open source and stuff and everything is done. No, that is step zero, that is step zero, then comes the real work, the real work starts after that. And you have to be committed to do that kind of additional work until you reach a point where, you know, yeah, there is enough of a community around it already. You cannot just say I created this, you know, fantastic solution to do accounting packages, great. Open it up, thank you very much. Don't just walk away, that isn't stopped, you now need to encourage others to contribute, create a community around it, it takes time. And if you've never done that before, that's why we try and help them. So that's the other part of the story. We from an OSPO perspective, there are organizations that are struggling trying to figure out how to do this. And they reach out to us for help. And so OSPO trying to fill that role [P14]

%So if you, you know, kind of the approach that we're taking internally, is we've kind of got three, at least three lenses and you know, each of these lenses can can be subdivided as he wants, where there's, there's the kind of the lens of the contributor, the lens of the maintainer and then and then kind of the lens of a sponsor, right. So I'd say... So Oh, yeah, the nurturing each of those, those those qualities, right, we're not just within, you know, not just as Cisco, but you know, within each of each of our employees, right, the hope is that you should feel comfortable to come play right in kind of the open source sandbox. [P16]

%There's different ways I can answer that. I think, from the perspective working in the OSPO, short for open source programs office I know, you know that but maybe for your notes, it is central to excellence in engineering. So open source is not separate or optional, but it's central to excellence in open source engineering, and culturally speaking, very much in line with our values of building on the work of others and finding others to build with us. There's quotes from Sasha, our CEO, or on the importance of collaboration. I mean, the other ones, so there's different ways I could ask that. So that's like how an answered from the OSPO perspective. But you know, there are-- Microsoft is giant, there are people who've never worked in open source before and who, you know, might, you know, are learning or might be fearful, or, you know, so I think that's the big picture, but that there's lots of little individual feelings, stories, people have different levels of learning. [P17]

%-------------------------------------------------------
%\textbf{Sell what's on the truck:} Take the time to fully develop the OSS value before providing service on top
%& 
%P5 
%& 
%``I have seen organizations that are selling things that I described as they are not on the truck, right? They're selling future roadmap to customers... they're not selling what's on the truck, meaning it doesn't exist today. I can't ship it to you. Right. So we're taking a different approach. We're not selling roadmap, we're gonna sell what's on the truck'' [P5]
%\\

%And I have seen organizations that are selling things that I described as they are not on the truck, right? They're selling future roadmap to customers. They're saying, Oh, if you if you renew your contract for this year, we're going to have this, this, this, this and this, and the next 12 months re up now so that you can get that at a discount. But they're selling future roadmap, they're not selling what's on the truck, meaning it doesn't exist today. I can't ship it to you. Right. So we're taking a different approach. We're not selling roadmap, we're gonna sell what's on the truck.[P5]

%-------------------------------------------------------
%\textbf{Leverage the no middleman land:} You can interact with the developers directly 
%& 
%P13, P14 
%& 
%``Stop putting intermediaries all over the place, and suddenly, it's easier to communicate'' [P13]\newline
%``That's right, direct feedback. Yes, exactly. Exactly. That is a huge benefit. That's a big benefit... want to talk to them directly? Go ahead and do it'' [P14]
%\\\hline 
  %"Here you can, go ahead and reach out directly. We are not we are not putting a firewall. They say no, you cannot talk to my developers. Because the developers must also understand what the customer's requirements are. Because if I'm building something I want to be able to, you know, know, how are my customers using it? What are the challenges they are facing? It I don't want it to be, you know, only coming by feedback and then somebody collects all the feedback and then massages the feedback" [P14]
  
  %"Can I speak to them directly? That is the power that we bring. And so majority of our customers are surprised that that can be the case. Because they don't expect it. They don't expect to be able to speak to the developer that develop the stuff that you are running on your enterprise system. They don't expect that you can expect Yeah, you can't be expected to find who wrote Oracle Database. Who are the author's? I mean, they may not be around, they may have left but you know, is there a name that maybe is there? But yeah, I can see it as a customer. I can see I don't know there. So it is giving that human face to, to the whole thing, which I think changes the dynamics" [P14]
  
%--------------------Projects Takeaway-----------------
%\cellcolor{gray!15}\textbf{Avoid toxicity and foster the project's positive environment}
%lower the toxicity of your project have a friendly open project where all contributors are respected and welcomed and interactions are friendly and not toxic 
%&
%\cellcolor{gray!15} P5, P7, P10, P15, P18
% &

% ``If it's an abusive community, even companies don't want to get involved. And the other is, is just, you know, companies look at is this a healthy community?'' [P18]
% \newline
% \cellcolor{gray!15} ``They [companies] run far away because the moment you see an internet spat or an open source spat, the corporation sees up in like, oh my god, like there is no way we're gonna come out of this good, right because all internet fights... they're like, oh my gosh, step away'' [P10]



% %``We are not going to allow our community to create fear, or lack of safety. If you're afraid to ask a question in our community, we're failing. If you're fearful that you're going to look or sound stupid in our community by asking a question, we have not set up our community the right way'' [P5]

% \\

% %And when you have a set of people that are from different backgrounds, and different incentive structures, and companies and locations, there's a lot of nuance that goes into ensuring that you're fostering a healthy and respectful community. And I think more so I think a lot of what we hope to bring to the table beyond just the funding and the people is support in in these sort of softer dynamics that are vital to the success of any project. So that's more of a lofty goal. But we've been investing in more programs and initiatives that we hope will help to just improve that aspect of things and improve the way that we have we create spaces for people to connect, we have been trying to mentor more people on understanding inclusivity diversity, equity inclusion initiatives, how to have these sorts of conversations, how to have recognition and Ally ship and all these sorts of harder, more difficult personal based dynamics. [P20]

% %So projects also need to make it easy for not just company but everyone to contribute, make it a welcoming community, have contribution guidelines, review issues, give proper feedback, of code of conduct, all of those kinds of things. If it's an abusive community, even companies don't want to get involved. And the other is, is just, you know, companies look at is this a healthy community? Is there a proper, you know, cadence to the innovation? Is it maintained by one person, you know, just one person somewhere who randomly will make things happen? [P18]

% %You know, something as simple as having a conversation with an organization that helps people learn how to develop. They come from an underserved population, maybe they've never gone to college, or maybe they're doing a career transition, where they've been doing something for 20 or 30 years. And now they want to transition into development. Partnering with those companies to say, who's graduating out of your programs, do they fit our tech stack? We want to talk to them. Right? So creating opportunities for all sorts of developers. Can we ingest a ton of newbies and make them successful? No, not at our scale. Could we take a few? Yeah. We have developers who work in areas of the world that are you know, not quite like the United States. We are making opportunities for developers in countries that are less developed. And we're giving them significant financial freedom. Because we pay them for their contribution, not for what they should be paid in that country. [P5]

% %5:11 ¶ 189 in [P5] Jill-SlimAI interview.docx
% %And you have to have clearly defined guardrails on what behavior is unacceptable. So when someone is behaving outside of what's acceptable, you also need to take the time to pull them aside and say, This isn't aligned with our community code of conduct or this isn't aligned with our culture and how we communicate? What can I do to help you to get to where you need to be, so that you can contribute? And after you've done that, if they can't opt in, you then very clearly need to say we're not going to allow you to contribute at this time. When you're ready to come back you let us know.

% %We have developers who work in areas of the world that are you know, not quite like the United States. We are making opportunities for developers in countries that are less developed. And we're giving them significant financial freedom. Because we pay them for their contribution, not for what they should be paid in that country. [P5]

% %Again, it's the idea is not to say like we do this thing, just because people come to us, it's we want to make those ecosystems healthy, welcoming, more attractive for everyone to use and contribute. And if some of these people then come to Aiven, because they heard Aiven is actively doing those things, way better.[P7]


% %But for a very successful public open source project, that is one where you have a public community of people, you know, contributing and involved, who work for all kinds of different employers, or no employer or, you know, all over the place. [P15]




% %-------------------------------------------------------
% \cellcolor{gray!15} \textbf{Have open leadership and governance}
% &
% \cellcolor{gray!15} P1, P3, P4, P9, P10, P11, P15, P18 
% & 
% \cellcolor{gray!15}``Companies can be a little bit reluctant to contribute to open source projects that are owned by companies, because, you know, companies are fickle, we change our strategies, and maybe we stop working on the project and goes away. And or maybe we decide to exert too much control and people get uncomfortable'' [P11]
% \newline
% ``If you do the corporate coalition stuff, you generally need the project to be in a foundation from day one, right?... if you don't have a foundation, then one of the companies needs to own things. And then everyone else needs to trust that one company, which is hard'' [P15]
% \\ 

%And this is where I think, you know, foundations like the Apache Foundation, the Linux Foundation, become really good neutral homes, for companies to come together to collaborate. If five cloud companies have the same challenge, it makes sense to work together to solve that problem, and not solve it through partnerships and agreements, because that's a very heavy legal lift. But to come together in a foundation and to say, let's create a charter, that's, you know, maybe one company has started solving that problem, and then release that code to that foundation, when everybody else starts working on that, and can continue to improve that.[P18]

%4:31 ¶ 87 in [P1] Georg-Bitergia interiew.docx
%So Bitergia started the grimorelab project and built a community around the metrics. So Grimoire Lab is the metrics tool that we have. And we had Grimore con, which is event series of bringing people together who wanted to talk about metrics and using Grimoire Lab to get metrics. And and then Bitergia was looking okay, how do we, how do we grow that community more? What What can we do? And so we approached the Linux Foundation to say, hey, Linux Foundation, you're hiring us for getting the metrics, you're using the tools. You also host projects, can you host Grimore lab, as a project as a Linux Foundation project? And the Linux Foundation said, yeah, we can do that. And we also have these researchers at the University of Nebraska, who are doing metrics work, who want to understand Project Health, you have the tools Bitergia let's bring this together and create one community. And so Bitergia Research Group Linux Foundation, we all sat together decided to name this one group chaos. And so chaos is the next step from the Bitergia perspective of the community that we have built. But now, chaos has a bigger scope, than just metrics around the Grimoire Lab tool because now there's also research interest they have the research project. And it's bigger. It's bigger pie with what we are doing as part of that.

%1:17 ¶ 105 in [P3] Omer-VM interiew.docx
%Exactly, exactly. And then when you are collaborating, right, so Knative is about to be donated to the CNCS. That took a lot of work to get to happen years of lobbying and politicking and striking and it's so much stuff, because the vendors that are involved in that project already have representation in these foundations. So in their mind, they would like to put these things where they already have influence, and in a place that they can trust that other vendors won't screw them over. And the CNCF and Linux Foundations have done a pretty good job of being like, Okay, we are a neutral home for these things.

%2:40 ¶ 109 in [P4] Per-Spotify interview.docx
%So we have no involvement other than like, some, I think some people do get involved in like, [inaudible], it's very individual decision based on their needs. So Spotify do not have a strategy of influencing the projects that depend on. It happens on a per dependency basis, if, if an engineer sees like a need to raise a discussion with our team, they will do so. But no, we don't we don't have we don't have a plan to leverage ourselves as a company to to influence our dependencies. A

% it benefits you to have advisors who are also experts in that ecosystem. So we have folks like Kitt Marker and Brendan O'Leary and Kelsey Hightower and Alan Chisa, we have people in the extended ecosystem, who are advisors who spend regular time with us to help us guide the strategy and execution of the business in a way that is appropriate for the ecosystem. Right? When you behave that way, both internally and externally, you can produce more faster, because it's not just us internally doing it. The amplification is way bigger. Mm hmm. Right. And so and so if we, it's not just us talking about us, its advisors, its investors, its users, right. So it amplifies out bigger and faster.[P5]

%10:34 ¶ 160 in [P9] Phillipe NexBe Interview.docx
%And now, in selecting code, a library that we use, that will be critical dependencies for us, that will be typically be a serious criteria. If we think it's important for us to participate in the governance, and we're not allowed to, we would probably just pass and move on to something else and not use the code at all. There's example like that I wouldn't give specific name. But there's a project with which we were involved fairly significantly. And it was very clear that nobody was welcome to provide input on on the governance and directions of the project. So we passed, we disengage, and we disengage completely, basically... So there's there's one possible challenge, but it's more of a problem. That's the only challenge maybe free riding. Which is if you do free riding. Which means when you have contributors, well, users that that eventually abuse and free ride without ever contributing back.[P9]

%And that becomes a very important part of how open source modern open source grows. It's not the weekend warrior. I mean, in some cases, it is but for a lot of cases, it's not a weekend warrior, that's going to create Kubernetes it's Google, who going to pay engineers full time all day every day and feed them to make a technology and then strategically give that to a foundation to grow in public, so that more people use a technology and oh, by the way, they have a cloud providing service that you pay for that, you know, has some sort of consistency with with that technology. [P10]

% And the question is, who owns the intellectual property of the joint venture? So if I have a joint venture, some some of my people, some of you people are, are coming together to create something? Well, somebody has to own the output, it's either one entity owns it, the other entity owns it, it's owned in a joint way. Or it's given to some sort of neutral sort of neutralized status. Those are sort of your basic four outcomes. Well, you want to decide that before you put people in. But last thing you want to do is put people into the fight, and then have a company say we own it. [P10]

%But for us, one of the things that we look at as the primary benefit is really being able to build an ecosystem around the project and build more of a community around the project. Because people can be companies can be a little bit reluctant to contribute to open source projects that are owned by companies, because, you know, companies are fickle, we change our strategies, and maybe we stop working on the project and goes away. And or maybe we decide to exert too much control and people get uncomfortable.[P11]

%I mean, a lot of times that they use, so for example, if you do the corporate coalition stuff, you generally need the project to be in a foundation from day one, right? Because, because, like for a single company, project, that single company can keep paying for the project's resources, you know, and hosting things, etc. And gradually, you know, gradually alienate those items. Right, so gradually, you know, give them up to some sort of public resource or, or you know, come up with a way to have them sponsored or put it in a foundation or whatever, right? There's a flexible timeline around that with one of these corporate coalition's you pretty much need to do it immediately. Because if the CO if you don't have a foundation, then one of the companies needs to own things. And then everyone else needs to trust that one company, which is hard.[P15]
  
\bottomrule
\end{tabular}
\label{table:sportsmanshipTable}
\vspace{-2.5mm}
\end{table*}


%[other quotes]
%o I think like having, using the policies, like having the right policies and guidelines in their project description, like, you know, like the contributing Markdown file, like how to contribute to the project, also all the policies in place in terms of so they like the sign, like, like the license, for instance, like having a document of which license it is because companies won't be contributing to your project, if there is no license sewing, even though you have it on a GitHub or an anti lab doesn't look doesn't mean that it's open source. So yeah, in overall, it's more like having all these documentation documented in the right place. So I make it easier for companies or individuals as well, to be able to take a look at that. Because they might be like also having like automation, automating tools to review all this. So used to be making sure that you're using the right syntax, and you're putting it in the right place. Because if not, it might be like your open source project. Maybe that just because it's not in the right place. And they're not going probably in my project review. And it's they have their own automation tools. [P19]

%[P18] That's a really, really good questions. One of the reasons why sometimes companies don't contribute is because the project makes it so difficult to contribute either. The, there's no proper guide to contribution, there's no coding guide, there's no readme. A there's no proper license. And so the company has to work extremely hard to find out, you know, how do I come to contribute to this. And that could be bias in the project, also about accepting contributions from certain companies or, or to protect, you know, their own direction, or they may see companies as pushing the company agenda and not thinking about the project in general. So projects also need to make it easy for not just company but everyone to contribute, make it a welcoming community, have contribution guidelines, review issues, give proper feedback, of code of conduct, all of those kinds of things. If it's an abusive community, even companies don't want to get involved. And the other is, is just, you know, companies look at is this a healthy community? Is there a proper, you know, cadence to the innovation? Is it maintained by one person, you know, just one person somewhere who randomly will make things happen? And, I mean, companies can do a lot more here as well, if it's really a valuable project, and it's struggling for maintenance. Companies should help, you know, with the maintenance and not just the feature functions, but the heavy lifting of code reviews and maintaining the project and doing the security and the plumbing. And I would say those are some of the things that projects could do. [P18]