\begin{table*}[bht]
%\footnotesize
%\scriptsize
\caption{OSS Sportsmanship: lessons learned for projects}
\begin{tabular}{p{0.2\textwidth}p{0.1\textwidth}p{0.6\textwidth}}
\toprule
{\textbf{Takeaway}} & {\textbf{Participants}} & {\textbf{Quotes}} \\ \hline

  
%--------------------Projects Takeaway-----------------
\textbf{Avoid toxicity and foster the project's positive environment}
%lower the toxicity of your project have a friendly open project where all contributors are respected and welcomed and interactions are friendly and not toxic 
&
 P5, P7, P10, P15, P18
&

``If it's an abusive community, even companies don't want to get involved. And the other is, is just, you know, companies look at is this a healthy community?'' [P18]
\newline
 ``They [companies] run far away because the moment you see an internet spat or an open source spat, the corporation sees up in like, oh my god, like there is no way we're gonna come out of this good, right because all internet fights... they're like, oh my gosh, step away'' [P10]



%``We are not going to allow our community to create fear, or lack of safety. If you're afraid to ask a question in our community, we're failing. If you're fearful that you're going to look or sound stupid in our community by asking a question, we have not set up our community the right way'' [P5]

\\

%And when you have a set of people that are from different backgrounds, and different incentive structures, and companies and locations, there's a lot of nuance that goes into ensuring that you're fostering a healthy and respectful community. And I think more so I think a lot of what we hope to bring to the table beyond just the funding and the people is support in in these sort of softer dynamics that are vital to the success of any project. So that's more of a lofty goal. But we've been investing in more programs and initiatives that we hope will help to just improve that aspect of things and improve the way that we have we create spaces for people to connect, we have been trying to mentor more people on understanding inclusivity diversity, equity inclusion initiatives, how to have these sorts of conversations, how to have recognition and Ally ship and all these sorts of harder, more difficult personal based dynamics. [P20]

%So projects also need to make it easy for not just company but everyone to contribute, make it a welcoming community, have contribution guidelines, review issues, give proper feedback, of code of conduct, all of those kinds of things. If it's an abusive community, even companies don't want to get involved. And the other is, is just, you know, companies look at is this a healthy community? Is there a proper, you know, cadence to the innovation? Is it maintained by one person, you know, just one person somewhere who randomly will make things happen? [P18]

%You know, something as simple as having a conversation with an organization that helps people learn how to develop. They come from an underserved population, maybe they've never gone to college, or maybe they're doing a career transition, where they've been doing something for 20 or 30 years. And now they want to transition into development. Partnering with those companies to say, who's graduating out of your programs, do they fit our tech stack? We want to talk to them. Right? So creating opportunities for all sorts of developers. Can we ingest a ton of newbies and make them successful? No, not at our scale. Could we take a few? Yeah. We have developers who work in areas of the world that are you know, not quite like the United States. We are making opportunities for developers in countries that are less developed. And we're giving them significant financial freedom. Because we pay them for their contribution, not for what they should be paid in that country. [P5]

%5:11 ¶ 189 in [P5] Jill-SlimAI interview.docx
%And you have to have clearly defined guardrails on what behavior is unacceptable. So when someone is behaving outside of what's acceptable, you also need to take the time to pull them aside and say, This isn't aligned with our community code of conduct or this isn't aligned with our culture and how we communicate? What can I do to help you to get to where you need to be, so that you can contribute? And after you've done that, if they can't opt in, you then very clearly need to say we're not going to allow you to contribute at this time. When you're ready to come back you let us know.

%We have developers who work in areas of the world that are you know, not quite like the United States. We are making opportunities for developers in countries that are less developed. And we're giving them significant financial freedom. Because we pay them for their contribution, not for what they should be paid in that country. [P5]

%Again, it's the idea is not to say like we do this thing, just because people come to us, it's we want to make those ecosystems healthy, welcoming, more attractive for everyone to use and contribute. And if some of these people then come to Aiven, because they heard Aiven is actively doing those things, way better.[P7]


%But for a very successful public open source project, that is one where you have a public community of people, you know, contributing and involved, who work for all kinds of different employers, or no employer or, you know, all over the place. [P15]




%-------------------------------------------------------
\textbf{Have open leadership and governance}
&
 P1, P3, P4, P9, P10, P11, P15, P18 
& 
``Companies can be a little bit reluctant to contribute to open source projects that are owned by companies, because, you know, companies are fickle, we change our strategies, and maybe we stop working on the project and goes away. And or maybe we decide to exert too much control and people get uncomfortable'' [P11]
\newline
``If you do the corporate coalition stuff, you generally need the project to be in a foundation from day one, right?... if you don't have a foundation, then one of the companies needs to own things. And then everyone else needs to trust that one company, which is hard'' [P15]
\\ 

%And this is where I think, you know, foundations like the Apache Foundation, the Linux Foundation, become really good neutral homes, for companies to come together to collaborate. If five cloud companies have the same challenge, it makes sense to work together to solve that problem, and not solve it through partnerships and agreements, because that's a very heavy legal lift. But to come together in a foundation and to say, let's create a charter, that's, you know, maybe one company has started solving that problem, and then release that code to that foundation, when everybody else starts working on that, and can continue to improve that.[P18]

%4:31 ¶ 87 in [P1] Georg-Bitergia interiew.docx
%So Bitergia started the grimorelab project and built a community around the metrics. So Grimoire Lab is the metrics tool that we have. And we had Grimore con, which is event series of bringing people together who wanted to talk about metrics and using Grimoire Lab to get metrics. And and then Bitergia was looking okay, how do we, how do we grow that community more? What What can we do? And so we approached the Linux Foundation to say, hey, Linux Foundation, you're hiring us for getting the metrics, you're using the tools. You also host projects, can you host Grimore lab, as a project as a Linux Foundation project? And the Linux Foundation said, yeah, we can do that. And we also have these researchers at the University of Nebraska, who are doing metrics work, who want to understand Project Health, you have the tools Bitergia let's bring this together and create one community. And so Bitergia Research Group Linux Foundation, we all sat together decided to name this one group chaos. And so chaos is the next step from the Bitergia perspective of the community that we have built. But now, chaos has a bigger scope, than just metrics around the Grimoire Lab tool because now there's also research interest they have the research project. And it's bigger. It's bigger pie with what we are doing as part of that.

%1:17 ¶ 105 in [P3] Omer-VM interiew.docx
%Exactly, exactly. And then when you are collaborating, right, so Knative is about to be donated to the CNCS. That took a lot of work to get to happen years of lobbying and politicking and striking and it's so much stuff, because the vendors that are involved in that project already have representation in these foundations. So in their mind, they would like to put these things where they already have influence, and in a place that they can trust that other vendors won't screw them over. And the CNCF and Linux Foundations have done a pretty good job of being like, Okay, we are a neutral home for these things.

%2:40 ¶ 109 in [P4] Per-Spotify interview.docx
%So we have no involvement other than like, some, I think some people do get involved in like, [inaudible], it's very individual decision based on their needs. So Spotify do not have a strategy of influencing the projects that depend on. It happens on a per dependency basis, if, if an engineer sees like a need to raise a discussion with our team, they will do so. But no, we don't we don't have we don't have a plan to leverage ourselves as a company to to influence our dependencies. A

% it benefits you to have advisors who are also experts in that ecosystem. So we have folks like Kitt Marker and Brendan O'Leary and Kelsey Hightower and Alan Chisa, we have people in the extended ecosystem, who are advisors who spend regular time with us to help us guide the strategy and execution of the business in a way that is appropriate for the ecosystem. Right? When you behave that way, both internally and externally, you can produce more faster, because it's not just us internally doing it. The amplification is way bigger. Mm hmm. Right. And so and so if we, it's not just us talking about us, its advisors, its investors, its users, right. So it amplifies out bigger and faster.[P5]

%10:34 ¶ 160 in [P9] Phillipe NexBe Interview.docx
%And now, in selecting code, a library that we use, that will be critical dependencies for us, that will be typically be a serious criteria. If we think it's important for us to participate in the governance, and we're not allowed to, we would probably just pass and move on to something else and not use the code at all. There's example like that I wouldn't give specific name. But there's a project with which we were involved fairly significantly. And it was very clear that nobody was welcome to provide input on on the governance and directions of the project. So we passed, we disengage, and we disengage completely, basically... So there's there's one possible challenge, but it's more of a problem. That's the only challenge maybe free riding. Which is if you do free riding. Which means when you have contributors, well, users that that eventually abuse and free ride without ever contributing back.[P9]

%And that becomes a very important part of how open source modern open source grows. It's not the weekend warrior. I mean, in some cases, it is but for a lot of cases, it's not a weekend warrior, that's going to create Kubernetes it's Google, who going to pay engineers full time all day every day and feed them to make a technology and then strategically give that to a foundation to grow in public, so that more people use a technology and oh, by the way, they have a cloud providing service that you pay for that, you know, has some sort of consistency with with that technology. [P10]

% And the question is, who owns the intellectual property of the joint venture? So if I have a joint venture, some some of my people, some of you people are, are coming together to create something? Well, somebody has to own the output, it's either one entity owns it, the other entity owns it, it's owned in a joint way. Or it's given to some sort of neutral sort of neutralized status. Those are sort of your basic four outcomes. Well, you want to decide that before you put people in. But last thing you want to do is put people into the fight, and then have a company say we own it. [P10]

%But for us, one of the things that we look at as the primary benefit is really being able to build an ecosystem around the project and build more of a community around the project. Because people can be companies can be a little bit reluctant to contribute to open source projects that are owned by companies, because, you know, companies are fickle, we change our strategies, and maybe we stop working on the project and goes away. And or maybe we decide to exert too much control and people get uncomfortable.[P11]

%I mean, a lot of times that they use, so for example, if you do the corporate coalition stuff, you generally need the project to be in a foundation from day one, right? Because, because, like for a single company, project, that single company can keep paying for the project's resources, you know, and hosting things, etc. And gradually, you know, gradually alienate those items. Right, so gradually, you know, give them up to some sort of public resource or, or you know, come up with a way to have them sponsored or put it in a foundation or whatever, right? There's a flexible timeline around that with one of these corporate coalition's you pretty much need to do it immediately. Because if the CO if you don't have a foundation, then one of the companies needs to own things. And then everyone else needs to trust that one company, which is hard.[P15]
  
\bottomrule
\end{tabular}
\label{table:projectSportsmanshipTable}
\end{table*}


%[other quotes]
%o I think like having, using the policies, like having the right policies and guidelines in their project description, like, you know, like the contributing Markdown file, like how to contribute to the project, also all the policies in place in terms of so they like the sign, like, like the license, for instance, like having a document of which license it is because companies won't be contributing to your project, if there is no license sewing, even though you have it on a GitHub or an anti lab doesn't look doesn't mean that it's open source. So yeah, in overall, it's more like having all these documentation documented in the right place. So I make it easier for companies or individuals as well, to be able to take a look at that. Because they might be like also having like automation, automating tools to review all this. So used to be making sure that you're using the right syntax, and you're putting it in the right place. Because if not, it might be like your open source project. Maybe that just because it's not in the right place. And they're not going probably in my project review. And it's they have their own automation tools. [P19]

%[P18] That's a really, really good questions. One of the reasons why sometimes companies don't contribute is because the project makes it so difficult to contribute either. The, there's no proper guide to contribution, there's no coding guide, there's no readme. A there's no proper license. And so the company has to work extremely hard to find out, you know, how do I come to contribute to this. And that could be bias in the project, also about accepting contributions from certain companies or, or to protect, you know, their own direction, or they may see companies as pushing the company agenda and not thinking about the project in general. So projects also need to make it easy for not just company but everyone to contribute, make it a welcoming community, have contribution guidelines, review issues, give proper feedback, of code of conduct, all of those kinds of things. If it's an abusive community, even companies don't want to get involved. And the other is, is just, you know, companies look at is this a healthy community? Is there a proper, you know, cadence to the innovation? Is it maintained by one person, you know, just one person somewhere who randomly will make things happen? And, I mean, companies can do a lot more here as well, if it's really a valuable project, and it's struggling for maintenance. Companies should help, you know, with the maintenance and not just the feature functions, but the heavy lifting of code reviews and maintaining the project and doing the security and the plumbing. And I would say those are some of the things that projects could do. [P18]