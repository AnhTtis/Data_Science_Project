\section{Related Work}

%%%%_____________Mariam Comments_____________________________
%Most oss work focused on individual/project experiences in oss from their motivation to challenges 

%When it comes to companies in OSS, research focuses mainly on innersource, ways to apply OSS processes to a company [] benefits of innersource [] and business models where in that case they .....
 

%However, there is little to no understanding of why and how companies get involved in open source.... 

%%%%_____________Mariam Comments______________________________


\boldification{Handful of literature on individual's participation in open source}
%Motivation, challenges, barriers, newcomers, experienced contributors...
\boldification{innersource--just mentioning that there is innersource out there, as well as OSPOs, but we're not focusing on that}
%some people explored hybrid setting ... 
%the ones that explore how companies get involved we go more in-depth 
%Existing literature on companies' involvement in open source spans a range of focus, from innersource \cite{} to companies' OSPOs \cite{}. Additionally, 

A vast majority of prior literature in OSS has focused on individuals' experiences, from their motivations to participate \cite{gerosa2021motivation, lee2017understanding, von2012carrots, hannebauer2016motivation, oreg2008exploring , roberts2006understanding, huang2021leaving, yang2022projects} to challenges faced when participating \cite{steinmacher2015systematic, steinmacher2015social, guizani2021long ,lee2017understanding, huang2021leaving} and their perception of the OSS ecosystem \cite{lee2019floss, guizani2022perceptions, vasilescu2015perceptions}.


\textbf{Differences between individual and company motivations.}
Individuals' motivations to contribute to OSS has been extensively studied \cite{gerosa2021motivation, lee2017understanding, von2012carrots, hannebauer2016motivation, oreg2008exploring , roberts2006understanding, huang2021leaving}, with the literature review by Von Krogh et al.  \cite{von2012carrots} being the most comprehensive and Gerosa et al. \cite{gerosa2021motivation} work, which builds on \cite{von2012carrots} being the most recent. %In this section we draw a parallel between our findings on companies motivations and prior literature's findings on individuals' motivations. 

From all the individual ``intrinsic'' motivations (Ideology, Altruism, Kindship, Fun), only \textit{Ideology} has a company equivalent. \textit{Founder(s)' vision} is an individual's \textit{Ideology} turned corporate. \textit{Founder(s)' Vision} as a motivator for participation was identified for small companies where the founder(s)' beliefs carried onto their company, defining its culture and DNA.

All individuals' ``internalized extrinsic'' motivations (Reciprocity, Reputation, Own use), except Learning, have their parallel in companies. \textit{Reciprocity} is a common motivation for both individuals and companies, where both actors aim to give back to the community to reciprocate the benefits they got from OSS.
Individuals participate in OSS to improve their own \textit{Reputation} \cite{gerosa2021motivation} and companies their brands. For an individual, gaining reputation can be a segue to getting hired by companies who themselves participate in OSS to attract talent. Companies benefit by hiring individuals who have been working on the project allowing companies to bypass recruitment and onboarding costs. Companies' \textit{business advantage} maps to individual's \textit{own use} but adds a more nuanced definition, since %While \textit{own use} by definition maps to companies' \textit{engineering need}, 
\textit{business advantage} also includes drivers such as coopetition, having closer channels, and improved innovation.

Individuals' ``extrinsic motivation'' (getting paid, career) do not directly map to companies. Companies do not get paid (or have contracts) to contribute to OSS, instead, they are the ones who may sponsor OSS projects. However, companies do indirectly profit from OSS participation as a result of an improved reputation and recruiting talented individuals.

\textbf{Company participation in OSS.}
%innersource + legal compliance
When it comes to companies adopting OSS processes, prior literature investigated innersource adoption~\cite{stol2014key, stol2014inner, edison2020inner, morgan2011exploring, capraro2018patch, carroll2017value}, and the motivations and challenges of applying it~\cite{capraro2016inner, carroll2018examining, gaughan2009examination, stol2011comparative}. Researchers also identified compliance best practices to help companies avoid legal/IP risks when using OSS~\cite{harutyunyan2019getting, harutyunyan2019industry, harutyunyan2018understanding}. OSS-related business models used by companies have been analyzed and compared as a way to understand the benefits and risks of each model~\cite{munga2009adoption, spijkerman2018open, li2019does, mouakhar2017open, deodhar2012strategies}. Similarly, the literature showed how companies' usage of OSS relates to their business models~\cite{dahlander2008firms, riehle2011controlling, morgan2014beyond}. Companies use OSS to increase their productivity and product quality~\cite{ajila2007empirical}. This, in return, encourages companies that started with closed in-house software \cite{andersen2012commercial} or hardware~\cite{li2021understanding} to open source them by following a specific pathway \cite{kochhar2019moving, mortara2011large, zynga2018making, pruett2013comparison}. This results in more hybrid communities, where the intensity of companies' involvement helps understand their impact on volunteer communities \cite{zhang2019companies, zhou2016inflow, dahlander2005relationships, aagerfalk2008outsourcing, maenpaa2018organizing}. 


% ways to contribute 
%Furthermore, researchers have identified some work practices companies use when contributing to a selection of five established OSS projects, some of which  include ``employing core project developers, and joining project steering committees in order to advance strategic interests'' \cite{butler2018investigation, butler2019company}. 



%the communication practices between companies and foundation-owned OSS projects when making a contribution (e.g., an explanatory question on a mailing list when reporting a bug) and the reason behind using public channels
%(e.g., mailing lists, changelogs) \cite{butler2018investigation, butler2019company}. While Butler et al.'s looked at how companies communicate with OSS projects when they make a contribute, we looked at why companies contribute in general and what types of contribution they engage in. 

%_________compare with results___________
Butler et al.~\cite{butler2019company} investigated the communication practices between companies and foundation-owned OSS projects (e.g., questions on a mailing list when reporting a bug) and the reason behind using public channels (e.g., mailing lists, changelogs). They identified the ways for companies to communicate with OSS projects and reasons for making it public. While Butler et al. focused on companies' communication practices when making a contribution, we focused on the multi-faceted ways companies contribute to OSS and identified all the kinds of ties that companies create with OSS and the motivations that lead the company to join OSS. Still, in their case study, Linaker and Regnell found 12 objectives for deciding what software to open source and when. While Linaker and Regnell \cite{linaaker2020share} work focused on a specific type of contribution (open-sourcing software) we focused on a broader perspective, aiming to build a comprehensive motivation model that drives all kinds of contributions, their ties to the motivations and the entities that they benefit. 

Our work complements this body of research by focusing on companies as OSS contributors instead of looking at individuals. We also identify the ways to engage and the reasons that lead companies to engage with OSS. We contributed a comprehensive model of company motivations to contribute to OSS beyond open-sourcing products. Indeed, we show that the ways to contribute are multi-faceted, are tied to the motivation of the companies, and may influence different entities. 


