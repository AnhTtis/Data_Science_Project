\section{Conclusion}
\label{sec:conclusion}

\boldification{reiterate the importance of this work--similar to part of intro}
Understanding \textit{why} and \textit{how} companies  contribute to OSS provides a complete picture of all OSS players sustaining and maintaining the OSS ecosystem---making space for current and new participating companies to contribute effectively. 

\boldification{interesting examples from results}
Interestingly, our results revealed that though growing and large companies share some motivations, their reasoning may differ. For example, smaller companies reported being motivated by gaining a reputation, since OSS allows them to be in the same space as big-name companies; larger companies reported the same motivation, in this case, because their contributions to OSS signal good citizenship. %Furthermore, Reputation motivation was the most common driver for companies' contributions, with six out of seven of the contribution categories stemming from this motivation.

%\boldification{motivation and contribution categories we identified}
%In trying to understand companies' motivations and contributions to OSS, we empirically identified twelve subcategories of company motivations to contribute to OSS and seven high-level categories of OSS contributions across companies. We analyzed these twelve motivations within the four high-level categories we identified: (1) Founder(s)' vision, (2) Business advantage, (3) Reputation, and (4) Reciprocity, where the Reputation motivation led to six out of seven of the companies' contributions. 

\boldification{what our results provide/contribute summary}
In summary, the main contributions of this paper are (1) a comprehensive model of companies’
motivations to contribute to OSS, (2) a conceptual model showing the multi-faceted ways that companies engage in OSS, linked to their motivations, and the benefiting entities, and (3) lessons learned from companies to foster a healthy OSS-company relationship. 

\boldification{conclude and future}
We hope our results encourage and provide guidance for current and new participants in the OSS ecosystem--specifically companies--to engage in OSS, collaborate in the open, create better software and broaden the economic pie. The lessons we learned from this study will help both companies and OSS projects continue to foster this symbiotic relationship in which they sustain each other effectively.
