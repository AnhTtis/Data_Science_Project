\textbf{Visibility -D1.5}:

 \MyBox{\textbf{Visibility:} companies participate in OSS because they want to:  
 \begin{itemize}
     \item Increase their visibility
      \item Be credited as the founder of a technology that is now the industry standard (P18)
     \item Positioning their name to the domain they are known for
     Be recognized in their domain of expertise 
     \item Expand or reposition their brand name beyond the domain that the company is known for
    
     
 \end{itemize}} 
 

In open source \inlinequote {the brand recognition is very powerful [P14]} and companies from different sizes participate in open source to increase their visibility (P1, P2, P4-P9, P11, P13-P16)

\boldification{general motivation visibility}
Our interviewees considered Branding/ Marketing as a driver of open source participation (P1, P2, P4-P9, P11, P13-P16). For instance, P14 stressed that in open source \inlinequote {the brand recognition is very powerful}. One interviewee, the OSPO lead at a large tech company, associated open source branding with the branding of the technical ideas behind an app as opposed to the branding of a product as a black box: 

\longquote{P4}{open sourcing these bits and pieces... can help the deal of branding of the technical ideas of an app... and [company name] saw a lot of good press}


\boldification{Visibility, positioning in the domain the company is already known for}
One forms of Branding/Marketing that open source helps with is  visibility (P4, P2, P1, P9, P7, P6, P5, P15, P16) as explained by P2, CEO of a startup whose company is present and active in multiple OSS projects, \inlinequote{and we wanted to be there in our case, because this brings a lot of visibility}.





%their overall visibility, interviewees from large technology companies considered being recognized for standardizing a piece of technology as the ultimate visibility goal (P3, P4, P18):  
%While visibility is a main part of branding
%Branding and marketing has always been at the core of companies' reputation [XX], and open source provides an open transparent platform that helps spread the company's brand. 





\longquote{P4}{ And the best way to do this is to make sure that what you are working on inside of your company, and the thing you depend on actually becomes the industry standard, not just something you maintain... And if you actually came up with this standard, then you also get a lot of exposure.}
 
Our interviewee highlighted as an example the amount of Branding exposure that both Google and Facebook had gained from open sourcing technologies that are now industry standard: 
 
\longquote{P3}{open sourcing React has provided more value for Facebook than leaving it to themselves ever could have. Because A, they get contributions now on that library. B, they get the sort of street credit of the founders of React}


%"where open source is superior is it gives us superior visibility" [P15]

%P2 also emphasize the importance of visibility for a growing startup \inlinequote{the CEO of [large OSS foundation] announced in a keynote [city and name of conference]. We are now Running [OSS community]. And all of these companies and universities are part of this [company name] was there. So that was like great visibility, a really good milestone for us}

%\boldification{showcasing technical achievement and build understanding of what the company already does}

With the visibility that comes with open source, small companies are able to build a brand name  beyond the borders of their country as explained by P6: \inlinequote{We have several views in different countries...started to become more international... a nice way to increase our brand in our company.}

Once a company is relatively visible in the open source space, it can start to build an understanding of what it actually does internally. In fact, interviewees from companies of varying sizes reported leveraging open source to build an understanding of what the company is doing by showcasing their technical achievements (P7, P4, P2, P1) as explained by P4 \inlinequote{so having that open source would be a good contribution to further increase that understanding of what's actually going on inside.} This helps attach the name of the company to their domain of expertise.

\boldification{Vs. repositioning expends the brand name of the company to more than what it is known for}
However, this positioning and understanding is not static as 
interviewees also saw open source as an ecosystem that can expand their brand name beyond what the company has been known for (P11, P2, P1, P4). In particular, open source can help companies transition to a different domain of expertise beyond the one that they have been known for:  
\longquote{P11}{so people tend to think of [company name] as like the people who did [specific product]. And we've really moved beyond that. And so our focus right now is really on building platforms and application developer tool sets that people can use on top of those platforms} 
