% CVPR 2022 Paper Template
% based on the CVPR template provided by Ming-Ming Cheng (https://github.com/MCG-NKU/CVPR_Template)
% modified and extended by Stefan Roth (stefan.roth@NOSPAMtu-darmstadt.de)

\documentclass[10pt,twocolumn,letterpaper]{article}
\pdfoutput=1
%%%%%%%%% PAPER TYPE  - PLEASE UPDATE FOR FINAL VERSION
% \usepackage[review]{cvpr}      % To produce the REVIEW version
\usepackage{cvpr}              % To produce the CAMERA-READY version
% \usepackage[pagenumbers]{cvpr} % To force page numbers, e.g. for an arXiv version

\makeatletter
\@namedef{ver@everyshi.sty}{}
\makeatother
\usepackage{tikz}
% Include other packages here, before hyperref.
\usepackage{graphicx}
\usepackage{amsmath}
\usepackage{amssymb}
\usepackage{pifont}% http://ctan.org/pkg/pifont
\newcommand{\cmark}{\ding{51}}%
\newcommand{\xmark}{\ding{55}}%
\usepackage{diagbox}
\usepackage[linesnumbered,ruled,vlined]{algorithm2e}
\usepackage{bbding}
\usepackage{booktabs}
\usepackage{utfsym}
\usepackage{makecell}
\usepackage{bm}
\usepackage[accsupp]{axessibility} 
% \usepackage{epsfig,epstopdf,subfigure}
% \usepackage{pifont}
% % \usepackage{multirow}
% \usepackage{tabularx,arydshln}

% It is strongly recommended to use hyperref, especially for the review version.
% hyperref with option pagebackref eases the reviewers' job.
% Please disable hyperref *only* if you encounter grave issues, e.g. with the
% file validation for the camera-ready version.
%
% If you comment hyperref and then uncomment it, you should delete
% ReviewTempalte.aux before re-running LaTeX.
% (Or just hit 'q' on the first LaTeX run, let it finish, and you
%  should be clear).
\usepackage[pagebackref,breaklinks,colorlinks]{hyperref}

\usepackage{multirow}
\newcommand{\tabincell}[2]{\begin{tabular}{@{}#1@{}}#2\end{tabular}}



% Support for easy cross-referencing
\usepackage[capitalize]{cleveref}

\usepackage{amsthm}
\newtheorem{thm}{Theorem}[section]
\newtheorem{prop}[thm]{Proposition}
\newtheorem{lem}[thm]{Lemma}
\newtheorem{cor}[thm]{Corollary}


\crefname{section}{Sec.}{Secs.}
\Crefname{section}{Section}{Sections}
\Crefname{table}{Table}{Tables}
\crefname{table}{Tab.}{Tabs.}


%%%%%%%%% PAPER ID  - PLEASE UPDATE
\def\cvprPaperID{3847} % *** Enter the CVPR Paper ID here
\def\confName{CVPR}
\def\confYear{2023}
\begin{document}

%%%%%%%%% TITLE - PLEASE UPDATE
\title{Bi-directional Distribution Alignment for Transductive Zero-Shot Learning}

\author{Zhicai Wang$^1$, Yanbin Hao$^{1*}$, Tingting Mu$^2$, Ouxiang Li$^1$, Shuo Wang$^1$, Xiangnan He$^{1}$\thanks{Yanbin Hao and Xiangnan He are both the corresponding authors.}\\
$^1$University of Science and Technology of China, $^2$The University of Manchester\\
{\tt\small wangzhic@mail.ustc.edu.cn, haoyanbin@hotmail.com, tingting.mu@manchester.ac.uk,}\\
{\tt\small  lioox@mail.ustc.edu.cn, \{shuowang.hfut,xiangnanhe\}@gmail.com}
}
% \author{Zhicai Wang\\
% University of Science and Technology of China\\
% {\tt\small wangzhic@mail.ustc.edu.cn}
% % For a paper whose authors are all at the same institution,
% % omit the following lines up until the closing ``}''.
% % Additional authors and addresses can be added with ``\and'',
% % just like the second author.
% % To save space, use either the email address or home page, not both
% \and
% Yanbin Hao\\
% University of Science and Technology of China\\
% \and
% Tingting Mu\\
% University of Manchester\\
% \and
% Ouxiang Li\\
% University of Science and Technology of China\\
% \and 
% Shuo Wang\\
% University of Science and Technology of China\\
% \and
% Xiangnan He\\
% University of Science and Technology of China\\
% }
\maketitle
%%%%%%%%% ABSTRACT
\begin{abstract}

Zero-shot learning (ZSL) suffers intensely from the domain shift issue, i.e., the mismatch (or misalignment) between the true and learned data distributions for classes without training data (unseen classes).
%
By learning additionally from unlabelled data collected for the unseen classes, transductive ZSL (TZSL) could reduce the shift but only to a certain extent.
%
To improve TZSL, we propose a novel  approach Bi-VAEGAN which strengthens the distribution alignment between the visual space and an auxiliary space. 
%
As a result, it can reduce largely the domain shift. 
%
The proposed key designs include (1) a bi-directional distribution alignment, (2) a simple but effective $L_2$-norm based feature normalization approach, and (3) a more sophisticated unseen class prior estimation. 
%
%Zhicai: I removed /href as it does not compile in my laptop. You can put it back.
Evaluated by four benchmark datasets,  Bi-VAEGAN\footnote{Code is available at \href{https://github.com/Zhicaiwww/Bi-VAEGAN}{https://github.com/Zhicaiwww/Bi-VAEGAN}} achieves the new state of the art under both the standard and generalized TZSL settings. 
%

 
  % Zero-shot learning (ZSL) is a cross-modal knowledge transfer task, which aims to classify the unseen classes with training on seen classes and some auxiliary information. It is challenging since the domain shift generally exists because of the untouchable unseen data assumption.  To alleviate the domain shift problem introduced by untouchable unseen data, transductive ZSL (TZSL) concedes that unseen data is accessible in the training. Generative models have shown their superiority in the TZSL since their distribution modeling nature.  However, how to conditionally align the cross-modal distribution using the unlabeled target data is still an open question.

    % In this work, we propose a competitive bi-directional cross-modal generative framework, Bi-VAEGAN, for the challenging TZSL. We find a transductive regressor, the cycle-consistent module that maps visual feature back to auxiliary space, is playing an important role in the visual feature alignment.
    % % and help with a more compact and realistic generation. 
    % We also introduce the free-lunch feature $L_2$ normalization and attribute perturbation for generative distribution alignment. 
    % The experiments show that our Bi-VAEGAN opens a significant performance gap with other generative approaches under the given class frequency prior assumption.  Meanwhile, we put a discussion on how the class prior affects the conditional alignment for the unseen data and propose to utilize the cluster property to estimate the class frequency given the unknown prior assumptions. 

\vspace{-0.2cm}
\end{abstract}
\vspace{-0.3cm}

%%%%%%%%% BODY TEXT

% \section{Proposed Bi-VAEGAN}
% \begin{figure*}[htp]
% \centering
% \begin{subfigure}{0.26\linewidth}
% \centering
% \includegraphics[width=1\linewidth]{figures/visualization/map_att_2_0.718.pdf}
% \caption{AWA2} 
% \label{fig.2.1}
% \end{subfigure}
% \begin{subfigure}{0.73\linewidth}
% \centering
% \includegraphics[width=1\linewidth,height=4.2cm]{figures/model/motivation.pdf}
% \caption{Motivation} 
% \label{fig.2.2}

% \end{subfigure}
% \caption{Illustration of the gap in the re-mapped attribute. (a) $'\times'$ with a bigger font size indicates the real attribute, and $'+'$ indicates the re-mapped attribute of the synthesized visual feature. (b) Illustration of our motivation \label{fig:2}}
% \end{figure*}

% \subsection{Notation}
% We use $V^s=\{\bm v^s_i\}_{i = 1}^{n_s}$ and $V^u=\{\bm v^u_i\}_{i = 1}^{n_u}$ to denote the collections of examples from the seen and unseen classes, where each example is characterized by visual features extracted by a pre-trained network.  
% %
% For examples from the seen classes, their class labels  are provided and denoted by  $Y^s = \{y_i \}_{i = 1}^{n_s}$. 
% %
% Attribute (we set as default choice of auxiliary information) is provided to describe both the seen and unseen classes, represented by vector sets $A^s=\{\bm a^s_i\}_{i = 1}^{N_s}$ and $A^u=\{\bm a^u_i\}_{i = 1}^{N_u}$ where $N_s$ and $N_u$ are seen and unseen class numbers. 
% %
% Under the setting of transductive ZSL, a model is trained using the information offered by $D^{tr}=\{V^s,Y^s,V^u\}$ so that it can predict class labels for examples from the unseen classes. This results in the  classifier $f(\bm v): \mathcal{V}^u \rightarrow \mathcal{Y}^u$, where we use $\mathcal{V}$ to denote the encoded visual space.


% \subsection{Data Preprocessing}
% % Zhicai: We indeed need a more formal reason for why L_2 normalization works here. I'd like to give some more details here. Firstly, the feature we are using is extracted from ResNet which ended with a ReLU layer, and thus the raw visual feature is typically in [0, +inf). Second, raw attribute information is also in [0, +inf) and then L_2 normalized as I mentioned below. And what if we use visual features from other backbones?
        
% \noindent\textbf{$L_2$-Feature Normalization:} 
% %
% Feature normalization is an important data preprocessing method by which to guarantee the model training and accelerate the model convergence. 
% % However, in specific scenarios, the normalization method needs to be carefully selected, which has not been studied much in transductive ZSL. 
% A common practice in TZSL is to normalize the visual features by the Min-Max approach i.e.,$\bm v^\prime = \frac{\bm v -\text{min}(\bm v)}{\text{max}(\bm v)-\text{min}(\bm v)}$\cite{narayan2020latent,xian2019f,wu2020self}. However, we find it suffers from the distribution skew of synthesized feature value and it is more beneficial to normalize the visual features by their $L_2$-norm. For a visual feature vector $\bm v_i \in {V}^s \bigcup {V}^u $, it has
% \begin{equation}
% \label{V_norm}
% \bm v ^\prime = L_2(\bm v, r) = \frac{r\bm v}{||\bm v||_2},
% \end{equation}
% where the hyperparameter $r>0$ controls the length of a normalized visual feature vector.  
% The key point in this modification is that we can replace the last $Sigmoid$ function layer (usually accompanied by Min-Max normalized feature) with an $L_2$ normalization layer in the generator. 
% % By doing that, the generative model is easier to mimic the distribution of real data and leads to a faster convergence for the model training. 
% We put more discussion in Section \ref{sec:norm}.
% % This proposed operation is consistent with the attribute normalization and is expected to provide stronger support for the alignment control between the two spaces.



% \noindent\textbf{Attribute Perturbation:}
% %Zhicai (TM): I don't think the following statement is the right justification because your perturbed attribute vector is still a single point...
% %
% %Naive attribute information represents a class as a single point in the attribute space. 
% %We argue that a single point is often biased and a domain representation is more robust and more in line with the real-world context. 
% %
% When representing a class by its attribute information, a common practice is to locate each class by a fixed point in the attribute space. We introduce a location shift for each class to allow a more flexible alignment from the visual space to the attribute space.  The proposal is to augment the attribute vector by adding Gaussian noise, i.e., for each class $i = 1,...,N_s$ (or $N_u$), it has
% %Zhicai (TM): I introduced the function \pi to be used in later equations, check.
% \begin{equation}
% {\bm a}_i =  \bm\pi (\bm a_i,\lambda)=L_2(\bm a_i + \lambda \bm\epsilon,1),
% \end{equation}
% %Zhicai (TM): Is epsilon a scalar or vector? notation \bm a^2_i does not make sense. use \bm \epsilon if it is a vector. Also, add some justification to your \epsilon design.
% where $\bm\epsilon \sim \mathcal{N}(\mathbf{0},\bm a_i \odot \bm a_i)$ 
% %
% and $0<\lambda \leq1$ controls the allowed shifting strength and is tuned as a hyperparameter. The perturbed attribute treats the original attribute as the prototype and introduces a Gaussian shift in which variance is parameterized by the value in each dim. We only implement the perturbation in the training of the regressor (see Section \ref{sec:3.3.1} ), since the generator already introduce randomness by sampling in the latent space ($\bm z \sim \mathcal{N}(\bm 0, \bm 1)$).  
% %
% The perturbed attribute vector is further normalized to a unit vector by its $L_2$-norm, just as in Eq. (\ref{V_norm}). 
% % But the length adjustment by $r$ is not applied as we only need to adjust one space (visual) to meet the other.
 

% % We note the generalized attribute domain as $\mathcal{{A}}$.
% \begin{figure*}[thp]
%     \small
%     \centering
%         \includegraphics[width=0.9\linewidth,height = 7.6cm]{figures/model/BiT-VAEGAN.pdf} 
%         \caption{Our proposed Bi-VAEGAN in TZSL. The network is built upon VAE-GAN architecture and introduces an additional transductive regressor that maps visual features to the auxiliary (e.g. attribute) space. The training strategy consists of iterative training of two levels: \textbf{Level 1}, The training of regressor $\bm R$ is based on simultaneous supervised training on seen data, and adversarial training ($D^a$) on unseen data. \textbf{Level 2},  generator $\bm G$ is adversarially trained by the conditional ($D$) and unconditional ($D^u$) WGAN loss, where regressor $R$ is freeze and provides supervised constraint for unseen generation. The Cluster Prior Estimation (CPE) is introduced for estimating the class prior under the case that the class distribution of unseen classes is not given.}
%         \vspace{-0.5cm}
%         \label{fig.3}
% \end{figure*}


% \subsection{Bidirectional Alignment Model}
% \label{sec:BA}

% We propose a modular model architecture composed of six modules: (1) a conditional VAE encoder $\bm E: \mathcal{V} \times 
% \mathcal{A} \rightarrow \mathbb{R}^k$ mapping the visual features  to a $k$-dimensional hidden representation vector conditioned on the class attributes,  (2) a  conditional visual generator $\bm G: \mathcal{A} \times \mathbb{R}^k  \rightarrow \mathcal{V}$ synthesizing visual features from  a $k$-dimensional random noise vector that is usually sampled from a normal distribution $\mathcal{N}(\bm 0, \bm 1)$, conditioned on the class attributes, 
% (3) a conditional  visual Wasserstein GAN (WGAN)  critic $D: \mathcal{V}^s\times \mathcal{A } \rightarrow \mathbb{R}$ for the seen classes,  (4) a visual WGAN critic $ D^{u}: \mathcal{V}^u\rightarrow \mathbb{R}$ for the unseen classes, (5) a regressor mapping the visual space to the attribute space  $\bm R: \mathcal{V}\rightarrow \mathcal{A }$, and (6) an attribute WGAN critic $D^{a}: \mathcal{A}\rightarrow \mathbb{R}$. 
% %
% Later on, we slight abuse the notation ${\mathcal{A}}$ to denote the pre-processed attribute space (attribute perturbation) .
% % Later on, instead of the output of vectors, we sometimes slightly abuse the notations  $\bm E(\cdot,\cdot)$  and $\bm G(\cdot, \cdot)$ to denote the probability distributions of the hidden representation vector and the generated feature vector, e.g. $\bm z \sim \bm E(\bm v,\bm a)$, and also allow their input to be a distribution (or a space) i.e., where the examples are sampled from. Similarly, we also abuse the notation $ \bm\pi (\cdot,\lambda)$ to denote the distribution of the randomly perturbed attribute vector and allow a distribution or space input. We allow using $L_2(\cdot, r)$  to denote the probability distribution of the normalized visual features only when the left input is a distribution (or a space). 

% The proposed workflow consists of two levels. In \textbf{Level 1}, the regressor $\bm R$ is adversarially trained using the critic $D^{a}$ so that the pseudo attribute features converted from the visual features are aligned with the true attribute features. 
% In \textbf{Level 2}, the visual generator $\bm G$ is adversarially trained using the two critics $D$ and $D^{u}$ so that the generated visual features are aligned with the true visual features. 
% %
% Additionally, the training of $\bm G$ depends on the regressor $\bm R$. 
% %
% This encourages the pseudo attribute features converted from the synthesized visual features to be aligned with the true attribute features.
% %
% To highlight our core innovation of aligning true and synthesized features in both visual and attribute spaces, we name the proposal as bidirectional alignment.

% \subsubsection{Level 1: Regressor Training}
% \label{sec:3.3.1}
% Our regressor $\bm R$ is trained transductive and adversarially.  It is constructed by performing supervised learning using the labeled examples from the seen classes, additionally enhanced by unsupervised learning from the visual features and class attributes of the unseen classes. By  ``unsupervised", we mean that the features and classes are unpaired for examples from the unseen classes.
% Specifically, $\bm R$ learns to map the visual features of each example from the seen classes to be close to its corresponding class attributes via 
% \begin{align}
%     \begin{split}
%         \small
%          L_R^s({\mathcal{A}}^s, \mathcal{V}^s ) =&  \mathbb{E}[\| \bm R (\bm v^s) - {\bm a}^s \|_1],\\  
%     \end{split}
% \end{align}
% and simultaneously distinguish the true unseen attribute features from the pseudo ones computed from the visual features via the adversary objective
% \begin{align}
%     \begin{split}
%     \small
%          L^{u}_{D^a\text{-WGAN}} ({\mathcal{A}}^u, \mathcal{V}^u ) =&  \mathbb{E} \left [D^{a}({\bm a}^u) \right ]  - \mathbb{E} \left [D^{a}(\hat{\bm a}^u) \right ]+\\  
%          & \mathbb{E}_{}  [(\|\nabla_{\bar{\bm a}^u} D^{a}(\bar{\bm a}^u)\|_2-1)^2 ],
%     \end{split}
% \end{align}
% % \begin{align}
% %     \small
% %     \begin{split}
% %      & L^{D^a}_{\text{WGAN}} \left(\mathcal{A}^u, \mathcal{V}^u \right)  \\
% %      =&\;  \mathbb{E}_{{\bm a}^u\sim  \bm\pi (\mathcal{A}^u, \lambda)}  \left [D^{a}({\bm a}^u) \right ]  -\\ 
% %      &  \; \mathbb{E}_{{\bm v}^u\sim L_2(\mathcal{V}^u, r), \hat{\bm a}^u \sim  \bm R({\bm v}^u)}  \left[D^{a}(\hat{\bm a}^u ) \right] +\\
% %      & \; \mathbb{E}_{\bar{\bm a}^u\sim \mathcal {P}_{t}({\bm a}^u,\hat{\bm a}^u)}  \left[(\|\nabla_{\bar{\bm a}^u} D^{a}(\bar{\bm a}^u)\|_2-1)^2 \right].\\
% %     \end{split}
% % \end{align}
% where $\hat{\bm a}^u =\bm R (\bm v^u)$ and $\bar{\bm a}^u\sim \mathcal {P}_{t}({\bm a}^u,\hat{\bm a}^u)$\footnote{$\mathcal {P}_{t}(\bm a,\bm b)$ is an interpolated distribution in the $L_2$ hypersphere. An example sampled from this distribution is computed by $c = L_2(t\bm a+(1-t)\bm b,r)$ with $t\sim \mathcal{U}(0,1)$ where $\|\bm a\|_2=\|\bm b\|_2=r$.}. Note that before add noise perturbation, we sample the real attribute based on the unseen class distribution prior $p^u_{\bm G}( y)$, i.e., $\bm a^u \sim p^u_{\bm G}( y)$ where $\bm a^u \in A^u$. We call this the \textbf{prior sample} process.
% %
% The third term is known as the gradient penalty \cite{gulrajani2017improved},
% which is a more elegant way to meet Lipschitz restriction in the original WGAN\cite{arjovsky2017wasserstein}.
% %

% The regressor $\bm R$ aims at learning a mapping between the visual and attribute features for the seen classes in a supervised manner, and meanwhile learning the distribution of the overall feature domain for the unseen classes in an unsupervised manner. The overall training objective in level-1 is,
% \begin{align}
%     \begin{split}
%         \operatorname*{min}_{\bm R}\operatorname*{max}_{D^a}\; L_{R}^{s}+L_{D^a\text{-WGAN}}^{u}. 
%     \end{split}
% \end{align}
% % \begin{align}
% %     \small
% %     \begin{split}
% %      &L_{R}(\mathcal{A}^s, \mathcal{V}^s, \mathcal{V}^u) \\
% %      = &\;\mathbb{E}_{({\bm v}^s, {\bm a}^s)\sim (L_2(\mathcal{V}^s, r), \bm\pi({\mathcal{A}^s}, \lambda))}[\| \bm R ({\bm v}^s) - {\bm a}^s \|_1]  + \\
% %      &\; \mathbb{E}_{{\bm v}^u\sim L_2(\mathcal{V}^u, r)}  \left[D^{a}(\bm R({{\bm v}}^u )) \right].
% %     \end{split}
% % \end{align}
% The of $L_{R}$ and $L_{D^a\text{-WGAN}}^{u}$ realize the knowledge transfer from seen classes to the unseen in the attribute space. However, the hubness problem \cite{zhang2017learning} will limit the feature discriminability in the attribute space which typically has a smaller dimension. To have the feature aligned in the visual space, this regressor can be used as a good auxiliary module to provide supervised guidance for the training of synthesis in the visual space.


% \subsubsection{Level 2: Generator and Encoder Training}  

% Our generator $\bm G$ is also trained transductive and adversarially. It aims at aligning the synthesized features for the visual space using the visual critics $D$ and $D^u$, while for the attribute space using the frozen regressor $\bm R$.


% The two visual critics are trained to get better at distinguishing the true visual features from the synthesized ones computed by the conditional generator, i.e.d $ \hat{\bm v} \sim  \bm G(\bm a,\bm z) $ where $\bm z \sim \mathcal{N}(\bm 0,\bm 1)$. Note that the synthesized feature $\hat{\bm v}$ is already $L_2$ normalized. For examples from the seen classes, the critic is conditioned on their class attributes, i.e., $D(\hat{\bm v}^s, {\bm a}^s)$, while it is unconditional for examples from the unseen classes i.e., $D^u(\hat{\bm v}^u)$. The adversary training of two critics $D$ and $D^u$ are
% \begin{align}
%     \begin{split}
%     \small
%      L^s_{D\text{-WGAN}}(\mathcal{A}^s, \mathcal{V}^s) =& \mathbb{E}[D( {\bm v}^s,  {\bm a}^s)]- \mathbb{E}[D(\hat{\bm v}^s, {\bm a}^s)] +\\ 
%      & \mathbb{E}[(\|\nabla_{\bar{\bm v}^s} D(\bar{\bm v}^s, {\bm a}^s)\|_2-1)^2],
%     \end{split}
% \end{align}
% % \begin{align}
% %     \small
% %     \begin{split}
% %      &L^s_{D\text{-WGAN}}(\mathcal{A}^s, \mathcal{V}^s)  \\
% %   =& \;   \mathbb{E}_{ ({\bm v}^s, {\bm a}^s)\sim (L_2(\mathcal{V}^s, r), \mathcal{A}^s) }[D( {\bm v}^s,  {\bm a}^s)]-\\
% %      & \; \mathbb{E}_{ (\hat{\bm v}^s,{\bm a}^s) \sim (\bm G({\bm a}^s, \mathcal{N}(\bm 0,\bm 1) ),\mathcal{A}^s)}\left[D(\hat{\bm v}^s, {\bm a}^s)\right] +\\ 
% %      & \; \mathbb{E}_{\bar{\bm v}^s\sim \mathcal {P}_{t}( {\bm v}^s ,\hat{\bm v}^s)}\left[(\|\nabla_{\bar{\bm v}^s} D(\bar{\bm v}^s, {\bm a}^s)\|_2-1)^2\right],
% %     \end{split}
% % \end{align}
% and 
% \begin{align}
%     \label{Du_training}
%         \begin{split}
%         \small
%             L^{u}_{D^u\text{-WGAN}}(\mathcal{A}^u, \mathcal{V}^u)  = &\mathbb{E}[D^{u}({\bm v}^u )]- \mathbb{E}[D^{u}( \hat{\bm v}^u )]+ \\ 
%          & \mathbb{E}[(\|\nabla_{\bar{\bm v}^u} D^u(\bar{\bm v}^u)\|_2-1)^2].
%         \end{split}
%     \end{align}

% % \begin{align}
% % \label{Du_training}
% %     \small
% %     \begin{split}
% %         & L^{u}_{D^u\text{-WGAN}}(\mathcal{A}^u, \mathcal{V}^u) \\
% %       = &\; \mathbb{E}_{{\bm v}^u \sim L_2(\mathcal{V}^u,r)}[D^{u}({\bm v}^u )]- \\
% %      &\; \mathbb{E}_{ (\hat{\bm v}^u,\bm a^u) \sim (\bm G({\bm a}^u, \mathcal{N}(\bm 0,\bm 1)),\mathcal{A}^u)}[D^{u}( \hat{\bm v}^u )]+ \\ 
% %      &\; \mathbb{E}_{\bar{\bm v}^u\sim \mathcal {P}_{t}\left( {\bm v}^u ,\hat{\bm v}^u\right)}\left[(\|\nabla_{\bar{\bm v}^u} D^u(\bar{\bm v}^u)\|_2-1)^2\right],
% %     \end{split}
% % \end{align}
% %


% It is worth commenting on the training of the unconditional visual critic $D^{u}$ in Eq. (\ref{Du_training}) that the critic  captures the $Earth$-$Mover$ distance of overall unseen data distribution, i.e., $W(p^u( \bm v),\int p_{\bm G}^u(\hat{\bm v}|\bm a)p^u_{\bm G}(y)dy)$ (The prior sample is important and we put the discussion in the Section \ref{sec:prior}). This genetic approach can roughly align the conditional distribution of unseen classes but suffers from the absence of any supervised constraints. In addition to the weak alignment achieved by $D^{u}$, we thus use the regressor $\bm R$ further strengthen the constraint,
% \begin{align}
%     \begin{split}
%     \small
%          L_R^u({\mathcal{A}}^u ) =&  \mathbb{E}[\| \bm R (\bm G({\bm a}^u,\bm z)) - {\bm a}^u \|_1].\\  
%     \end{split}
% \end{align}
% % Fixing the visual critics $D$ and $D^u$ and the regressor $\bm R$,  the  encoder $\bm E$ and the generator $\bm G$ are jointly trained by minimizing the following loss function:
% % \begin{align}
% % \small
% %     \begin{split}
% %         & \;L_{G}(\mathcal{A}, \mathcal{V}) \\
% %       =  &\;  L_{\text{VAE}}(\mathcal{A}^s, \mathcal{V}^s) + \\
% %         &\; \mathbb{E}_{(\hat{\bm v}^s,\bm a^s) \sim (\bm G({\bm a}^s, \mathcal{N}(\bm 0,\bm 1)),\mathcal{A}^s)}[D(\hat{\bm v}^s,  {\bm a}^s )]  + \\
% %         &\;\mathbb{E}_{ (\hat{\bm v}^u,\bm a^u) \sim (\bm G({\bm a}^u,\mathcal{N}(\bm 0,\bm 1)),\mathcal{A}^u)} [D^{u}( \hat{\bm v}^u )  + \\
% %         &\; \gamma \| \bm R(\hat{\bm v}^u) -{\bm a}^u\|_1], \label{loss:G}
% %     \end{split}
% % \end{align}

% % \begin{align}
% %     \small
% %     \begin{split}
% %         &L_{\text{VAE}}(\mathcal{A}^s, \mathcal{V}^s)  \\
% %     =    &\;\mathbb{E}_{({\bm v}^s, {\bm a}^s)\sim (L_2(\mathcal{V}^s, r), \mathcal{A}^s)} \left[\text{KL}\left( \bm E({\bm v}^s,{\bm a}^s)  \|\mathcal{N}(\bm 0, \bm 1)\right)\right] \\
% %         &\; +\mathbb{E}_{\bm z^s \sim \bm E({\bm v}^s,{\bm a}^s)} [ \left(\|\bm G({\bm a}^s,\bm z^s)-{\bm v}^s\|^2_2 \right)].  \label{loss:VAE}
% %     \end{split}
% % \end{align}
% Feature-VAE has the merit of preventing model collapse and could be served as a suitable complement to GAN training. Following F-VAEGAN, we introduce the VAE objective to enhance the training of seen data.
% \begin{align}
%     \begin{split}
%     \small
%         L^s_{\text{VAE}}(\mathcal{A}^s,& \mathcal{V}^s)
%     =   \mathbb{E} [\text{KL}( \bm E({\bm v}^s,{\bm a}^s)  \|\mathcal{N}(\bm 0, \bm 1))] + \\
%         &\mathbb{E}_{\bm z^s \sim \bm E({\bm v}^s,{\bm a}^s)} [ \left(\|\bm G({\bm a}^s,\bm z^s)-{\bm v}^s\|^2_2 \right)].  \label{loss:VAE}
%     \end{split}
% \end{align}
% KL is the Kullback-Leibler divergence and the second term of Eq. (\ref{loss:VAE}) is the $MSE$ reconstruction loss in cooperation with the $L_2$ normalized feature. 
% Thus the overall training objective in level-2 is,
% \begin{align}
%     \begin{split}
%         \operatorname*{min}_{\bm E,\bm G}\operatorname*{max}_{D,D^u}\; L^s_{\text{VAE}} + L_{D\text{-WGAN}}^{s} + L_{R}^{u}+L_{D^u\text{-WGAN}}^{u}       
%     \end{split}
% \end{align}
% The above training mainly builds on transferring the knowledge of \textit{ paired visual features} and \textit{class prior} of the unseen classes, and it is enhanced by using the attribute regressor $\bm R$ to constrain further the visual feature generation for unseen classes. The motivation of such design is to utilize better the information from the unseen classes.

% \subsubsection{Predicative Model and Inference}
% \label{sec:3.6}
% After completing the training of the six modules, a predicative model $f$ for classifying examples from the unseen classes is trained. Specifically, it uses training examples each containing a concatenated vector as the input features, which combines the visual features $\bm v$, the hidden representation vector $\bm h$, and the pseudo attribute vector $\bm a$ returned by the regressor $\bm R$.  Given a real unseen instance $\bm v^u_i$, the enhanced representation is,
% \begin{equation}
% \bm x^u_i = [\bm v^u_i, \bm h^u_i, \bm a^u_i],
% \end{equation}
% Thus the classifier $f: \mathcal{V}^u \times \mathcal{H}^u \times \mathcal{A}^u \rightarrow \mathcal{Y}^u$ is working on a multi-modality space. 



% \subsection{The Role of Class Prior }
% \label{sec:clsprior}
% We can think of TZSL as a variant of the domain adaption with a conditional distribution matching problem. In this scenario, we regard the synthesized unseen distribution $p_{\bm G}^u(\hat{\bm v})$  that already knows the unseen label as the source domain $\mathcal{D}_S(\hat{V}^u)$, i.e., we already know the joint distribution $\mathcal{D}_S(\hat{V}^u,Y^u=y)$, and expect it to achieve conditional alignment with the real distribution $p^u({\bm v})$ in the target domain $\mathcal{D}_T({V^u})$.  It is different from some previous practices to find an invariant representation to solve the problem of domain shift, we can directly train the generator to adjust the distribution of the source domain to achieve that. For the case where the real unseen class prior $p^u(y)$ is not known, i.e., the label distribution on the target domain is unknown, we first show that this is very important for achieving conditional alignment. Given the ratio vector of real unseen class prior and sampling prior 
% ${\bf w}_y:=p^u(y)/p^u_{\bm G}(y)$  and the re-weighted source marginal distribution $\mathcal{D}_S^{{\bf w}}(\hat{V}^u):= \sum_{ y \in Y^u } {\bf{w}}_y \cdot \mathcal{D}_S(\hat{V}^u,Y^u=y)$. we could rewrite  the Theorem 3.4 from the generalized label shift theory \cite{tachet2020domain} in a TZSL setting,
%     \begin{align}
%         \footnotesize
%         \begin{split}
%             &\operatorname*{max}_{y \in Y^u}d_{\mathrm{TV}}(\mathcal{D}_{S}(\hat{V}^u\mid Y^u=y),\mathcal{D}_{T}(V ^u\mid Y^u=y)\leq\\
%             &\;\;\;\frac{{\bf w}_{M}\varepsilon_{S}(\hat{Y}^u)+\varepsilon_{T}(\hat{Y}^u)+\sqrt{8{D}_{JS}(\mathcal{D}_{S}^{\bf w}({\hat{V}^u})\vert \mathcal{D}_{T}(V^u))}}{\gamma}.
%             \label{gls}
%     \end{split}
%     \end{align}
% Above shows the upper bound of the total variation (TV) distance between the synthesized and real conditional distribution, where $ {\bf w}_M := \operatorname*{max} {\bf w}_y$ ,$ \gamma := \operatorname*{max} p^u(y)$  and $\varepsilon(\hat{Y}^u)$ is the error of hypothesis $f$ in the corresponding domain ($\varepsilon_{S}(\hat{Y}^u)$ is negligible since we train the classifier with the synthesized feature). The minimization of ${\scriptstyle \sqrt{8{D}_{JS}(\mathcal{D}_{S}^{\bf w}({\hat{V}^u})\vert \mathcal{D}_{T}(V^u))}}$ meets the needs of conditional distribution alignment of unseen data.
% In the case that we could access the real unseen class prior in the training, i.e. ${\bf w} = \bm 1$, the Jensen–Shannon divergence term collapse to ${\displaystyle D_{JS}(\mathcal{D}_{S}({\hat{V}^u})\vert \mathcal{D}_{T}(V^u))}$ and this is sharing the same idea with optimizing the Wasserstein distance of what the critic $D^u$ is doing. Thus a conditional alignment in visual space could be achieved easily. 


% \subsubsection{Unknown Class Frequency Prior}
% \label{sec:prior}

% As far as we are concerned, previous generative TZSL approaches are mostly conducted under the assumption that the unseen class frequency prior is known, i.e., $p^u(y)$ is easy to obtain \cite{narayan2020latent,xian2019f,wu2020self}. However, as the Eq. (\ref{gls}) shows, if class prior is not given and we sample from a random $p^u_{\bm G}(y)$ the minimization of ${\scriptstyle \sqrt{8{D}_{JS}(\mathcal{D}_{S}^{\bf w}({\hat{V}^u})\vert \mathcal{D}_{T}(V^u))}}$ will be hard to realize due to ${\bf w} \neq \bm 1$ . The generator thus fails to give realistic conditional generation. We need to alleviate the problem by estimating the class prior. 

% A key observation is that the unseen data usually have a good clustering property which is usually benefited from our knowledgeable backbone network. Thus we propose to apply a simple $Kmeans$ algorithm in the target domain to provide a stable and mild approximation. However, to give a reasonable estimation of each class, the initialized center of each class needs to be carefully designed. We name our proposed strategy as Cluster Prior Estimation (CPE) and the details could be found in Algorithm \ref{cpe}. The key idea is that we use an auxiliary classifier $f$ which is trained on the source domain to calculate the pseudo prototypes in the target domain. We treat the pseudo prototypes as the initialized center for $kmeans$ such that obtain the class prior estimation. Note that under transductive training it is easy to fall into a trivial solution by directly using uniform assumptions for extremely unbalanced data. Therefore, we recommend performing inductive training on the model first and then using the inductive generator for the first prior update.






\section{INTRODUCTION}


Zero-shot learning (ZSL) was originally known  as zero-data learning in  computer vision  \cite{larochelle2008zero}. 
%
The goal is to tackle the challenging training setup of largely restricted  example and (or) label availability \cite{chen2022msdn}. 
%
For instance, in conventional ZSL,  no training example is provided for the targeted classes, and they are therefore referred to as the unseen classes.   
%
What is provided instead is a large amount of  training examples paired with their class labels but for a different set of  classes  referred to as the seen classes. 
%
This setup is also known as the inductive ZSL. 
%
Its core challenge is to enable the classifier to extract knowledge from the seen classes and transfer it to the unseen classes, assuming the existence of such relevant knowledge \cite{norouzi2013zero,zhang2016zero,xian2018zero}. 
%
For instance,  a ZSL classifier can be constructed to recognize \textit{leopard} images after feeding it the Felinae images like \textit{wildcat}, knowing  \textit{leopard} is relevant to Felinae.  
%
Information on class relevance is typically provided as auxiliary data,  bridging knowledge transfer from the seen to unseen classes.
% 
The auxiliary data can be human-annotated attribute information  \cite{wah2011caltech}, text description \cite{reed2016learning}, knowledge graph \cite{lee2018multi} or a formal description of knowledge (e.g., ontology) \cite{geng2021ontozsl}, etc., which are encoded as (set of) embedding vectors. 
%
Learning solely from auxiliary data to capture class relevance is challenging, resulting in a discrepancy between the true and modeled distributions for the unseen classes, known as the domain shift problem.
%
To ease the learning, another ZSL setup called transductive ZSL (TZSL) is proposed. 
%
It allows to additionally include in the training unlabelled examples collected for the targeted classes.
%
Since it does not require any extra annotation effort to pair the examples from the unseen classes with their class labels, this setup is still practical in real-world applications.

\begin{figure}[t]
\begin{center}
    \includegraphics[width = 1\linewidth,height = 5.3cm]{figures/visualization/abstract.pdf}
\end{center}    
\caption[]{The top figure illustrates the proposed bi-directional generation between the visual and auxiliary spaces. The bottom figure compares the aligned visual space obtained by our method with the unaligned one obtained by inductive ZSL for the unseen classes using the AWA2 data. The bottom right figure shows our approximately aligned attributes in auxiliary space.   \label{fig.1}}
\vspace*{-0.4cm}
\end{figure}

Supported by sufficient data examples, generative models have become popular, for instance, used to synthesize examples to enhance classifier training \cite{xian2018feature,elhoseiny2019creativity,li2019leveraging} and to learn the unseen data distribution \cite{wu2020self,paul2019semantically,xian2019f} under the TZSL setup. 
%
Depending on the label availability, they can be formulated as unconditional generation i.e., $p(\bm v)$, or conditional generation i.e., $p(\bm v|\bm y)$.
%
When being conditioned on the auxiliary information, which is a more informative form of class labels, a data-auxiliary (data-label) joint distribution can be learned.
%
Such distributions  bridge the knowledge between the visual and auxiliary spaces and enables the generator to be a powerful tool for knowledge transfer.
% Zhicai: Added following
By including  an appropriate supervision,  e.g, by using a conditional discriminator to discriminate whether the generation is a realistic \textit{wildcat} image, intra-class data distributions can be aligned with the real ones.
%
However, the challenge of TZSL is  to transfer the knowledge contained by the joint data-auxiliary distribution for the seen classes to improve the distribution modelling  for the unseen classes, and achieve a realistic generation for the unseen classes.
%
A representative generative approach for achieving this is f-VAEGAN \cite{xian2019f}. 
%
It enhances the unseen generation using an unconditional discriminator and learns the overall unseen data distribution.
%
This simple strategy turns out to be effective in approximating the conditional distribution of unseen classes.
%
Most existing works use auxiliary data in the forward generation process  \cite{wu2020self,bo2021hardness}, i.e.,  to generate images from the auxiliary data as by $p(\bm v|\bm y)$. 
%
This can result in weakly guided conditional generation for the unseen classes and the alignment is extremely sensitive to the quality of the auxiliary information.
%
To bridged better between the visual and auxiliary data, particularly for the unseen classes, which is equivalent to enhancing the alignment with the true unseen conditional distribution, we propose a novel bi-directional generation process. 
%
It couples the forward generation process with a backward one, i.e., to generate auxiliary data from the images  as by $p(\bm y|\bm x)$. 


Figure \ref{fig.1} illustrates our proposed bi-directional generation based on a feature-generating framework. It builds upon the baseline design  f-VAEGAN, and is named as Bi-VAEGAN.
%
Overall, the proposed design covers three aspects.  (1) A transductive regressor is added to form the backward generation, synthesizing pseudo auxiliary features conditioned on visual features of an image. 
%
This, together with the forward generation as used in  f-VAEGAN, provides more constraints to learn the unseen conditional distribution, expecting to achieve better alignments between the visual and auxiliary spaces.
%
(2) We introduce  $L_2$-feature normalization, a free-lunch data pre-processing method,  to further support the conditional alignment. 
%
(3) Besides, we note that the (unseen) class prior plays a crucial role in distribution alignment, particularly for those datasets that have extremely unbalanced label distribution. 
%
A poor choice of the class prior can easily lead to a poor alignment. 
%
To address this issue, we propose a simple but effective class prior estimation approach based on a cluster structure contained by the examples from the unseen classes.  
%
The proposed Bi-VAEGAN is compared against various advanced ZSL techniques using four benchmark datasets and achieves a significant TZSL classification accuracy boost compared with the other generative benchmark models.
% \section{INTRODUCTION}

% \begin{figure}[thp]
% \begin{center}
%     \includegraphics[width = 1\linewidth,height = 5.3cm]{figures/visualization/abstract.pdf}
% \end{center}    
% \caption[]{Our approach aims to improve knowledge transfer using auxiliary information from open-world seen data to unseen data.  We utilize transductive generator to synthesize auxiliary information into visual space, and introduce cycle-consistent transductive regressor to further restrict the generation. Our model successfully aligned the real 
% feature with the synthesized in visual space and aligned the pseudo attribute that converted from synthesized feature with the real attribute in the auxiliary space.
% The bottom middle figure shows our aligned unseen visual space of AWA2 and its top right depicts the aligned auxiliary space. \label{fig.1}}
% \end{figure}


% Zero-shot learning (ZSL) was first named zero-data learning in the computer vision community \cite{larochelle2008zero}, which aims to classify those unseen classes that have no labeled data provided in training. For the specific domain of interests to conduct ZSL, there is usually a large amount of labeled data provided which we call those of seen classes.  
% The challenge of ZSL is to transfer knowledge from readily accessible seen classes to those novel unseen classes \cite{norouzi2013zero,zhang2016zero,xian2018zero}. 
% % For example, we have seen amounts of animals from Felinae like wildcat, and we hope the machine to classify the in-the-wild collected leopard figures.
% Auxiliary data, such as attribute information that is human-annotated discrete vectors\cite{wah2011caltech}, text description embedded vectors\cite{reed2016learning} and et al, is typically provided to bridge the knowledge transfer from seen classes to the unseen. 
% % So apart from the similar characters between wildcat and leopards, the text description like 'short legs', 'long body', and 'rosettes marked fur' could help with the classification of unseen leopards.
% Conventional ZSL, also referred to as inductive ZSL, classifies new categories solely based on auxiliary data with no unseen information provided. Due to the \textbf{domain shift} problem, which means there is a discrepancy between the algorithms' modeled unseen distribution and the real one, it typically has poor knowledge transferability.
% Thus, the quality of auxiliary information has a huge impact on transferability.
% In reality, gathering enormous amounts of unlabeled data in the real world isn't difficult at all, which could greatly reduce the domain shift issue. Transductive ZSL (TZSL), a solution to inductive ZSL, thus is proposed to get around the problem of the unconstrained unseen visual domain by assuming the availability of unlabeled test data.

% Generative models in ZSL is to use the generated fake data to train the classifier \cite{xian2018feature,elhoseiny2019creativity,li2019leveraging}, and it has gained great attention in TZSL recently since we can access the real unseen data distribution\cite{wu2020self,paul2019semantically,xian2019f}. 
% Given the data distribution or data-label joint distribution, generative models could be trained to fit the distribution and its conditional generation mechanism could also realize the intra-class distribution alignment (i.e., $p(\bm x|\bm y)$). Thus conditional generative models could be treated as a fairly effective tool to transfer knowledge in the ZSL scenarios by setting the auxiliary information as the condition. 
% In order to capture the overall unseen distribution, generative approaches like f-VAEGAN \cite{xian2019f} typically use an unconditional unseen discriminator, which could achieve a fairly good conditional distribution alignment of the most unseen classes. However, the majority of them only focus on one-directional visual space alignment, which results in a weak, unguided conditional alignment for those unlabeled unseen classes. Because no adaptive information is extracted from the auxiliary and only the forward generation process uses the auxiliary information (i.e., missing from the gradient backward-propagation). The alignment continues to be sensitive to the quality of the auxiliary information.
% % Consequently, F-VAEGAN fails to generate correct feature for the unseen 'walrus' and 'seal' on AWA2 dataset due to their resemble appearrance. 


% To address the constraint generation and enhance the conditional alignment in the unseen domain, we propose our generative approach, Bi-VAEGAN.
% The central concept of our work, where we attempt to align the features in both visual and auxiliary spaces, is shown in Figure \ref{fig.1}.
% % Figure \ref{fig.1} shows the core idea of our work where we try to align the features in both visual and auxiliary spaces.
% % Specifically, unlike f-VAEGAN,  we use an additional transductive regressor to realize the bidirectionally conditional distribution alignment between visual space and auxiliary space.
% In particular, unlike F-VAEGAN, we employ an additional transductive regressor
% to achieve the alignment of the bidirectional conditional distribution between the auxiliary space and the visual space.
% For example, for the appearance-resemble 'horse' and 'sheep' pair on AWA2,
% our approach successfully aligns the synthesized visual feature with the real one while also aligning the pseudo attribute (regressed from the synthesized visual feature) with the real attribute. Benefiting from the bidirectional alignment constraint, even for the visually perplexing pair of 'walrus' and 'seal', our model can decouple the mixed visual cluster and produce a proper synthesis. 

% In this work, we follow previous works that condition knowledge transfer on auxiliary information using a generative adversary network. 
% The difference is that we investigate the regressor module more thoroughly and discover that it is useful for providing supervised information, which enables the model to be more roubust to auxiliary information quality. To achieve a stronger constraint (bidirectional alignment) for the visual generation, we suggest the transductive regressor, which reveals the regressor with the information of unseen features. A better conditional alignment is made possible by a number of free-lunch data augmentation techniques that we also introduce, such as $L_2$ feature normalization and attribute perturbation. 
% % Lastly, we find the (unseen) class frequency prior is important for the conditional distribution alignment, especially for those extremely unbalanced datasets. An unknown class prior could easily damage the convergence guarantee and lead to a terrific alignment. To alleviate the above problem, we propose to use a simple class prior estimation strategy to approximate the class prior. Experimental result shows our simple strategy guarantee our model the competitive result even under the unknown prior case.
% Besides, we find that the (unseen) class frequency prior is crucial for the conditional distribution alignment, particularly for those datasets that have extremely unbalanced label distribution. Through the fusion of knowledge of class prior and auxiliary information model can achieve a better conditional alignment,  whereas an unknown class prior will easily lead to a poor alignment. To address TZSL under the prior unknown scenarios, we propose to approximate the class prior using a simple class prior estimation strategy, which is based on the observation of the cluster property of unseen classes.  Experimental result shows our simple strategy guarantee our model the competitive result even under the unknown prior case.



%Zhicai (TM): Check very carefully whether it makes sense.
\section{RELATED WORKS}
\noindent{\textbf{Inductive Zero-Shot Learning }}
%
Previous works on inductive ZSL learn simple projections from the auxiliary (e.g. semantic)  to visual spaces to enable knowledge transfer from the seen to unseen classes \cite{changpinyo2017predicting,zhang2017learning,DBLP:conf/nips/YuJFGPZ18,chou2020adaptive}.
%
They suffer from the domain shift problem due to the distribution gap between the seen and unseen data. 
%
Relation-Net \cite{sung2018learning} utilizes two embedding modules to align the visual and semantic information, modelling relations  in the embedded space.
%
Generative approaches, e.g., variational auto-encoder (VAE) \cite{doersch2016tutorial} and generative adversarial network (GAN) \cite{creswell2018generative},  synthesize unseen examples and train an additional classifier using the generated examples. Although this improves alignment between the synthesized and true  distributions for the unseen classes, it still suffers from domain shift. 
%
Some works, on the other hand, improve by introducing auxiliary modules. For instance, f-CLSWGAN \cite{xian2018feature}  uses a Wasserstein GAN (WGAN) to model the conditional distribution of the seen data, and introduces a classification loss to improve the generation. 
%
Cycle-WGAN \cite{felix2018multi} employs a semantic regressor with a cycle-consistency loss \cite{zhu2017unpaired} to reconstruct the original semantic features, which provides stronger generation constraints and shares some similar spirit to our work. 
%
However, because there is no knowledge of the unseen data, the inductive ZSL heavily relies on the quality of the auxiliary information, making it challenging to overcome performance bottleneck.

\noindent{\textbf{Transductive Zero-Shot Learning }}
As a concession of inductive ZSL, TZSL uses test-time unseen data to improve  training \cite{fu2015transductive,wu2020self,yang2022iterative}. 
%
A representative approach is visual structure constraint (VSC) \cite{wan2019transductive}. 
%
It exploits the cluster structure of the unseen data and proposes to align the projection centers with the cluster centers. 
%
Recently, generative models have been actively explored and shown superiority in TZSL. 
%
For instance, f-VAEGAN combines  VAE and GAN, and includes an additional unconditional discriminator to capture the unseen distribution. 
%
SDGN \cite{wu2020self} introduces a self-supervised objective to mine discriminability  between the seen and unseen domains. 
%
STHS-WGAN \cite{bo2021hardness} iteratively adds easily distinguishable unseen classes to the training  examples of the seen classes to improve the unseen generation. 
%
However, these previous approaches mostly work with a uni-directional generation from the auxiliary to visual spaces. 
%
This could potentially result in limited constraints when learning unseen distributions. 
%
TF-VAEGAN \cite{narayan2020latent} enhances the generated visual features by utilizing an inductive regressor trained with seen data and a feedback module. Expanding this idea, our work explore additionally the unseen data information in the regressor.



\subsection{Notations}
We use $V^s=\{\bm v^s_i\}_{i = 1}^{n_s}$ and $V^u=\{\bm v^u_i\}_{i = 1}^{n_u}$ to denote the collections of examples from the seen and unseen classes, where each example is characterized by its visual features extracted by a pre-trained network.  
%
For examples from the seen classes, their class labels are provided and denoted by  $Y^s = \{y_i \}_{i = 1}^{n_s}$. 
%
Attributes (we set as the default choice of auxiliary information) are provided to describe both the seen and unseen classes, represented by vector sets $A^s=\{\bm a^s_i\}_{i = 1}^{N_s}$ and $A^u=\{\bm a^u_i\}_{i = 1}^{N_u}$ where $N_s$ and $N_u$ are the numbers of the seen and unseen classes. 
%
Under the TZSL setting, a classifier $f(\bm v): \mathcal{V}^u \rightarrow \mathcal{Y}^u$ is trained to conduct inference on unseen data, where we use $\mathcal{V}$ to denote the  visual representation (feature) space, and  $\mathcal{Y}$ the label space.
%
The  training pipeline learns from information provided by $D^{tr}=\{\langle V^s,Y^s\rangle,V^u,\{A^s, A^u\}\}$, where we use $\langle \cdot, \cdot \rangle$ to highlight the 
paired data.

\subsection{$L_2$-Feature Normalization}

%
Feature normalization is an important data preprocessing method, which can improve the model training and convergence\cite{li2021feature}. 
%
A common practice in TZSL is to normalize the visual features by the Min-Max approach i.e.,$\bm v^\prime = \frac{\bm v -\text{min}(\bm v)}{\text{max}(\bm v)-\text{min}(\bm v)}$\cite{narayan2020latent,xian2019f,wu2020self}. 
%
However, we find that it suffers from the distribution skew when processing the synthesized features and it is more beneficial to normalize the visual features by their $L_2$-norm. 
%
For a visual feature vector $\bm v  \in {V}^s \bigcup {V}^u $, it has
\vspace{-0.3cm}
\begin{equation}
\label{V_norm}
\bm v ^\prime = L_2(\bm v, r) = \frac{r\bm v}{||\bm v||_2},
\end{equation}
where the hyperparameter $r>0$ controls the norm of the normalized feature vector.  
%
As a result, we replace in the generator  its last $sigmoid$  layer  that accompanys the Min-Max approach with an $L_2$-normalization layer. 
%
We discuss further its effect in Section \ref{sec:norm}.

 
\begin{figure*}[thp]
    \small
    \centering
        \includegraphics[width=0.9\linewidth,height = 7.6cm]{figures/model/BiT-VAEGAN.pdf} 
        \caption{The proposed Bi-VAEGAN model architecture under the TZSL setup.}
        \vspace{-0.5cm}
        \label{fig.3}
\end{figure*}


\subsection{Bi-directional Alignment Model}
\label{sec:BA}

We propose a modular model architecture composed of six modules: (1) a conditional VAE encoder $\bm E: \mathcal{V} \times 
\mathcal{A} \rightarrow \mathbb{R}^k$ mapping the visual features  to a $k$-dimensional hidden representation vector conditioned on the class attributes,  (2) a  conditional visual generator $\bm G: \mathcal{A} \times \mathbb{R}^k  \rightarrow \mathcal{V}$ synthesizing visual features from  a $k$-dimensional random  vector that is usually sampled from a normal distribution $\mathcal{N}(\bm 0, \bm 1)$, conditioned on the class attributes, 
(3) a conditional  visual Wasserstein GAN (WGAN)  critic $D: \mathcal{V}^s\times \mathcal{A } \rightarrow \mathbb{R}$ for the seen classes,  (4) a visual WGAN critic $ D^{u}: \mathcal{V}^u\rightarrow \mathbb{R}$ for the unseen classes, (5) a regressor mapping the visual space to the attribute space  $\bm R: \mathcal{V}\rightarrow \mathcal{A }$, and (6) an attribute WGAN critic $D^{a}: \mathcal{A}\rightarrow \mathbb{R}$. 
%
% Later on, we slightly abuse the notation ${\mathcal{A}}$ to denote the pre-processed attribute space (attribute perturbation) .


The proposed workflow consists of two levels. In \textbf{Level-1}, the regressor $\bm R$ is adversarially trained using the critic $D^{a}$ so that the pseudo attributes converted from the visual features align with the true attributes. 
In \textbf{Level-2}, the visual generator $\bm G$ is adversarially trained using the two critics $D$ and $D^{u}$ so that the generated visual features align with the true visual features. 
%
Additionally, the training of $\bm G$ depends on the regressor $\bm R$. 
%
This encourages the pseudo attributes converted from the synthesized visual features align better with the true attributes.
%
To highlight the core innovation of aligning true and fake data in both the visual and attribute spaces, we name the proposal as bi-directional alignment.

\subsubsection{Level 1: Regressor Training}
\label{sec:3.3.1}
The regressor $\bm R$ is trained transductively and adversarially.  It is constructed by performing supervised learning using the labeled examples from the seen classes, additionally enhanced by unsupervised learning from the visual features and class attributes of the unseen classes. By  ``unsupervised", we mean that the features and classes are unpaired for examples from the unseen classes.
For each example from the seen classes, $\bm R$ learns to map its visual features close to its corresponding class attributes via minimizing
\vspace{-0.3cm}
\begin{align}
    \begin{split}
        \small
         L_R^s({\mathcal{A}}^s, \mathcal{V}^s ) =&  \mathbb{E}[\| \bm R (\bm v^s) - {\bm a}^s \|_1].\\  
    \end{split}
\end{align}
Simultaneously, for examples from the unseen classes, it learns to distinguish their true attributes from the pseudo ones computed from the real unseen visual features by maximizing the adversary objective
\begin{align}
\label{adv_obj}
    \begin{split}
    \small
         L^{u}_{D^a\text{-WGAN}} ({\mathcal{A}}^u, \mathcal{V}^u ) =&  \mathbb{E} \left [D^{a}({\bm a}^u) \right ]  - \mathbb{E} \left [D^{a}(\hat{\bm a}^u) \right ]+\\  
         & \mathbb{E}_{}  [(\|\nabla_{\bar{\bm a}^u} D^{a}(\bar{\bm a}^u)\|_2-1)^2 ],
    \end{split}
\end{align}
where $\hat{\bm a}^u =\bm R (\bm v^u)$,  $\bm a^u \sim p^u_{\bm G}( y)$ and $\bar{\bm a}^u\sim \mathcal {P}_{t}({\bm a}^u,\hat{\bm a}^u)$\footnote{$\mathcal {P}_{t}(\bm a,\bm b)$ is an interpolated distribution in the $L_2$ hypersphere. An example sampled from this distribution is computed by $c = L_2(t\bm a+(1-t)\bm b,r)$ with $t\sim \mathcal{U}(0,1)$ where $\|\bm a\|_2=\|\bm b\|_2=r$.}. 
%
Note that the original  attribute vectors are sampled from  the unseen class prior $p^u_{\bm G}( y)$ explained in Section \ref{sec:prior}, and we refer to this as the \textbf{prior sample} process.
%
The third term in Eq. (\ref{adv_obj}) is known as the gradient penalty \cite{gulrajani2017improved},
which enables the Lipschitz restriction in the original WGAN \cite{arjovsky2017wasserstein}.
%

The regressor $\bm R$ aims at learning a mapping from the visual to the attribute features for the seen classes in a supervised manner, and meanwhile learning the distribution of the overall feature domain for the unseen classes in an unsupervised manner. The  level-1 training is formulated by
\begin{align}
\label{level1_training}
    \begin{split}
        \operatorname*{min}_{\bm R}\operatorname*{max}_{D^a}\; L_{R}^{s}+\lambda L_{D^a\text{-WGAN}}^{u},
    \end{split}
\end{align}
%
where $\lambda$ is a hyper-parameter. It enables knowledge transfer from the seen to  unseen classes in the attribute space, where the feature discriminability  is however limited by the hubness problem \cite{zhang2017learning}. 
% 
But it serves as a good auxiliary module to provide ``approximate supervision'' for the unseen distribution alignment later in the visual space.

\subsubsection{Level 2: Generator and Encoder Training}  

The generator $\bm G$ is also trained transductively and adversarially. It aims at aligning the synthesized and true features,   using the visual critics $D$ and $D^u$ in the visual space, while  using the frozen regressor $\bm R$ in the attribute space.

The two visual critics are trained to get better at distinguishing the true visual features from the synthesized ones computed by the conditional generator, i.e. $ \hat{\bm v} \sim  \bm G(\bm a,\bm z) $ where $\bm z \sim \mathcal{N}(\bm 0,\bm 1)$ and $\bm a \sim p_{\bm G}(y)$. 
%
For the seen classes, their class prior, denoted by $p_{\bm G}^{s}(y)$\footnote{For the intra-class alignment, i.e, we know the paired seen labels, the choice of $p_{\bm G}^{s}(y)$ will not affect much of the training.}, is simply estimated from the number of examples collected for each class. 
%
For the unseen classes, the estimated class prior $p_{\bm G}^u(y)$ is used.
%
The synthesized visual feature $\hat{\bm v}$ is already $L_2$-normalized. 
%
For the seen classes, the critic is conditioned on their class attributes, i.e., $D(\hat{\bm v}^s, {\bm a}^s)$, while  for the unseen classes, the critic is unconditional, i.e., $D^u(\hat{\bm v}^u)$. 
%
The two critics $D$ and $D^u$ are trained based on the adversarial objectives:
\begin{align}
\label{Du_training1}
    \begin{split}
    \small
     L^s_{D\text{-WGAN}}(A^s, V^s) =& \mathbb{E}[D( {\bm v}^s,  {\bm a}^s)]- \mathbb{E}[D(\hat{\bm v}^s, {\bm a}^s)] +\\ 
     & \mathbb{E}[(\|\nabla_{\bar{\bm v}^s} D(\bar{\bm v}^s, {\bm a}^s)\|_2-1)^2],
    \end{split}
\end{align}
and 
\begin{align}
    \label{Du_training}
        \begin{split}
        \small
            L^{u}_{D^u\text{-WGAN}}(A^u, V^u)  = &\mathbb{E}[D^{u}({\bm v}^u )]- \mathbb{E}[D^{u}( \hat{\bm v}^u )]+ \\ 
         & \mathbb{E}[(\|\nabla_{\bar{\bm v}^u} D^u(\bar{\bm v}^u)\|_2-1)^2],
        \end{split}
    \end{align}
where $\bar{\bm v}^s$ and $\bar{\bm v}^u$ are sampled from the interpolated distribution as explained in footnote 1. Here, $\hat{\bm v}^u$ is computed from the unseen attributes sampled by $\bm a^{u} \sim p_{\bm G}^{u}(y)$.
%
The critic  $D^{u}$ in Eq. (\ref{Du_training}) captures the Earth-Mover distance over the unseen data distribution.
% , as  $W(p^u( \bm v),\int p_{\bm G}^u(\hat{\bm v}|\bm a)p^u_{\bm G}(y)dy)$. 


Eqs (\ref{Du_training1}) and (\ref{Du_training}) weakly align the conditional distribution of the unseen classes. It suffers from the absence of any supervised constraints. 
%
To further  strengthen the alignment, we introduce another training loss, as
\begin{align}
\label{vae_loss}
    \begin{split}
         L_R^u(A^u ) =&  \mathbb{E}[\| \bm R (\bm G({\bm a}^u,\bm z)) - {\bm a}^u \|_1].\\  
    \end{split}
\end{align}
It employs the regressor $\bm R$ trained in level-1 to  enforce supervised constraints.
%
As shown in f-VAEGAN, feature-VAE has the potential of preventing model collapse, and it could serve as a suitable complement to GAN training. Similarly, we adopt the VAE objective function to enhance the adversarial training over the seen classes:
\begin{align}
    \begin{split}
    \small
        L^s_{\text{VAE}}(A^s,& V^s)
    =   \mathbb{E} [\text{KL}( \bm E({\bm v}^s,{\bm a}^s)  \|\mathcal{N}(\bm 0, \bm 1))] + \\
        &\mathbb{E}_{\bm z^s \sim \bm E({\bm v}^s,{\bm a}^s)} [ \left(\|\bm G({\bm a}^s,\bm z^s)-{\bm v}^s\|^2_2 \right)].  \label{loss:VAE}
    \end{split}
\end{align}
The first term is the Kullback-Leibler divergence and the second is the mean-squared-error (MSE) reconstruction loss using the $L_2$-normalized feature. 
%
Finally,  the  level-2 training is formulated by,
\begin{align}
\label{level2_training}
        \operatorname*{min}_{\bm E,\bm G}\operatorname*{max}_{D,D^u}\; L^s_{\text{VAE}} + \alpha L_{D\text{-WGAN}}^{s} + \beta L_{R}^{u}+ \gamma L_{D^u\text{-WGAN}}^{u},
\end{align}
%
where $\alpha$, $\beta$ and $\gamma$ are hyper-parameters. It  transfers the knowledge of \textit{ paired visual features and attributes} of the seen classes and the estimated \textit{class prior} of the unseen classes, 
%
and   is enhanced by  the attribute regressor $\bm R$ to constrain further the visual feature generation for unseen classes. 
%
The proposed Bi-VAEGAN architecture can be easily modified to accommodate the inductive ZSL setup, by removing all the loss functions using the unseen data $V^u$. This results in the following:
\begin{align}
\label{inductive_training}
    \begin{split}
        \textmd{For level-1: } &\min_{\bm R} L_{R}^{s} \\
         \textmd{For level-2: }&\min_{\bm E,\bm G}\operatorname*{max}_{D} L^s_{\text{VAE}} + \alpha L_{D\text{-WGAN}}^{s} + \beta L_{R}^{u}.
    \end{split}
\end{align}

\subsubsection{Unseen Class Prior Estimation}
\label{sec:prior}

When training based on the objective functions in Eqs. (\ref{adv_obj}) and (\ref{Du_training}), the attributes for the unseen classes are sampled from the class prior:  $\bm a^u \sim p^u_{\bm{G}}( y)$. Since there is no label information provided for the unseen classes, it is not possible to sample from the real class prior $p^u(y)$. An alternative way to estimate $p^u_{\bm{G}}( y)$ is needed.
%
We have observed that examples from the unseen classes possess fairly separable cluster structures in the visual space thanks to the strong backbone network. Therefore, we propose to estimate the unseen class prior based on such cluster structure.
%
We employ the K-means clustering algorithm and carefully design the initialization of its cluster centers since the prior estimation is   sensitive to the initialization.
%
The estimated prior $p_{\bm G}^u(y)$ is iteratively updated and in each epoch cluster centers are re-initialized by the pseudo class centers calculated from an extra classifier $f$. 
%
For the very first estimation of $p_{\bm G}^u(y)$, rather than using the naive but sometimes (if it differs greatly from the real prior) harmful uniform class prior,  we use the inductively trained generator to transfer the seen paired knowledge to have a better informed estimation for the unseen classes.
%
We refer to this estimation approach as the cluster prior estimation (CPE), and its implementation is shown in Algorithm \ref{tioncpe} (lines 1-12).


 
 
 	

\IncMargin{1em}
\begin{algorithm}[tp]\SetAlgoLined
     \SetKwData{Left}{left}\SetKwData{This}{this}\SetKwData{Up}{up} \SetKwFunction{Union}{Union}\SetKwFunction{FindCompress}{FindCompress} \SetKwInOut{Data}{Data}\SetKwInOut{Input}{Input}\SetKwInOut{Output}{Output}
    \small
    \Input{$\langle V^s,Y^s \rangle$,  $V^u$,  $\{A^s, A^u\}$, unseen class number $N_u$, epoch numbers $T_1$ and $T_2$.}
    \Output{$\bm E$, $\bm G$, $\bm R$, $D$, $D^u$, $D^a$.}
    \BlankLine
    \For{ $i = 1$  to $T_1$}{
         Inductive training with $\langle V^s,Y^s \rangle,\{A^s, A^u\}$ by  \\Eq. (\ref{inductive_training})\;
}
      
    \For{ $i = 1$  to $T_2$}{
        Define uniform distribution label set  $Y^u_{\bm G}$\;
        Synthesize paired unseen set  $\langle \hat V^u_{\bm G}, Y^u_{\bm G}\rangle$ using $\bm G$\;
        Train a classifier $f$ using $\langle \hat V^u_{\bm G}, Y^u_{\bm G}\rangle$\; 
        Assign pseudo class labels by $\hat Y^u = f(V^u)$\;
        Compute pseudo class centers $C^u \leftarrow  \langle V^u, \hat Y^u\rangle$\;
    %    Obtain cluster labels $\hat Y_{C}^u$ by   kmeans  for  $V^u$  with cluster centers initialized by $C^u$.
     
    %   Estimate $p_G^u(y)$ from $\hat Y_{C}^u$ by class size.
      \emph{$\hat Y_{kmeans}^u$ = Kmeans($V^u$,$N_u$, InitCenter = $C^u$)}\; 
      $p_{\bm G}^u(y) \leftarrow  \hat Y_{kmeans}^u$\tcc*{\footnotesize{Update prior}}
      Transductive training with $\langle V^s,Y^s \rangle$, $V^u$, $\{A^s, A^u\}$ and $p_{\bm G}^u(y)$ by Eqs (\ref{level1_training}) and (\ref{level2_training})\;
      }
    \caption{Bi-VAEGAN (CPE) }
           \label{tioncpe} 
      \end{algorithm}
 \DecMargin{1em} 
 

\noindent \textbf{Discussion:} We attempt to explain the importance of estimating $p^u_{\bm{G}}( y)$ based on the following corollary, which is a natural result of Theorem 3.4 of \cite{tachet2020domain}.
\begin{cor}
Under the generative TZSL setup, for the unseen classes, the total variation distance between the true conditional visual feature distribution $p^u( \bm{v}|y)$ and the estimated one by the generator $p_{\bm{G}}^u(\hat{\bm{v}}|y)$ is upper bounded by
    \begin{align}
        \begin{split}
            &\operatorname*{max}_{y \in Y^u}d_{\mathrm{TV}}\left(p^u_{\bm G}\left(\hat{\bm v}\mid y\right),p^u\left(\bm v\mid y\right) \right)\\
            \leq& \frac{1}{\min_{y \in Y^u} p^u(y)}\left( \max_{y\in Y^u}\left(\frac{p^u(y)}{p^u_{\bm{G}}(y)}\right) e^u_f(\hat{\bm v})+e^u_f(\bm v)\right. \\
            & \left.+\sqrt{8{d}_{\textmd{JS}}\left( \sum_{ y \in Y^u } p^u(y)p^u_{\bm G}(\hat{\bm v}|y), p^u(\bm v)\right)} \right),
    \end{split}
    \label{gls}
    \end{align}
    where ${d}_{\textmd{JS}}(\cdot, \cdot)$ is the Jensen-Shanon divergence between two distributions, and $e^u_f(\bm x)$ denotes the error probability that the classification  of the input feature vector disagrees with its ground truth using hypothesis $f$.

    \end{cor} 




\noindent    
The above result is a straightforward application of the domain adaptation result in Theorem 3.4 of \cite{tachet2020domain}, obtained by treating $p^u_{\bm G}\left(\hat{\bm v}\mid y\right)$ as the source domain distribution while $p^u\left(\bm v\mid y\right)$ as the target domain distribution. When the estimated  and ground truth class priors are equal, i.e., $p^u(y) = p^u_{\bm{G}}(y)$, Eq. (\ref{gls}) reduces to 
\vspace*{-0.08cm}
  \begin{align}
        \begin{split}
            &\operatorname*{max}_{y \in Y^u}d_{\mathrm{TV}}\left(p^u_{\bm G}\left(\hat{\bm v}\mid y\right),p^u\left(\bm v\mid y\right) \right)\\
            \leq& \frac{1}{\gamma}\left( e^u_f(\hat{\bm v})+e^u_f(\bm v) 
            +\sqrt{8{d}_{\textmd{JS}}\left(   p^u_{\bm G}(\hat{\bm v}), p^u(\bm v)\right)} \right),
    \end{split}
    \end{align}
where $\gamma=\min_{y \in Y^u} p^u(y)$. The effect of the class information is completely removed from the bound. As a result, the success of the conditional distribution alignment is dominated by matching the unconditional distribution in $D^u$. This is important  for our model because of the unsupervised learning nature for the unseen classes. 
\vspace*{-0.2cm}




\subsubsection{Predictive Model and Feature Augmentation}
\label{sec:3.3.4}

After completing the training of the six modules, a predictive model for classifying the unseen examples is trained. This results in a classifier $f: \mathcal{V}^u (\textmd{ or } \mathcal{\hat{V}}^u) \times \mathcal{H}^u \times \hat{\mathcal{A}}^u \rightarrow \mathcal{Y}^u$ working in an augmented multi-modal  feature space \cite{narayan2020latent}. Specifically, the used feature vector $\bm x^u $  concatenates the visual features $\bm v^u$ (or $\hat{\bm v}^u $),  the pseudo attribute vector computed by the  regressor $\hat{\bm a}^u=\bm R\left(\bm v^u\right)$ and the hidden representation vector $\bm h^u$ returned by the first fully-connected layer of regressor, which gives  $\bm x^u =\left[\bm v^u, \bm h^u,  \hat{\bm a}^u \right]$.  
%
It integrates knowledge of the generator and regressor that is  transductive, and presents stronger discriminability.




\begin{table*}[thp]
  \centering
  \footnotesize
   \renewcommand{\multirowsetup}{\centering}

   \resizebox{\textwidth}{!}{\begin{tabular}{l|l|cccc|ccc|ccc|ccc|ccc}
      \hline
      \multicolumn{2}{c|}{\multirow{3}{*}{\textbf{Method}}}& \multicolumn{4}{c}{Zero-Shot Learning}& \multicolumn{12}{|c}{Generalized Zero-Shot Learning}\\\cline{3-18}
      \multicolumn{2}{l|}{}&AWA1&AWA2&CUB&SUN&\multicolumn{3}{c}{AWA1}&\multicolumn{3}{|c}{AWA2}&\multicolumn{3}{|c|}{CUB}&\multicolumn{3}{c}{SUN}\\\cline{3-18}
      \multicolumn{2}{l|}{}&T1&T1&T1&T1&S&U&H&S&U&H&S&U&H&S&U&H\\\hline
      \multirow{5}{*}{I}&F-CLSWGAN\cite{xian2018feature}&59.9&62.5&58.1 &54.9& 76.1&16.8&27.5&81.8&14.0&23.9&33.1&21.8&26.3&63.8&23.7&34.4\\
      &SP-AEN\cite{chen2018zero}&-&58.5&59.2&55.4&-&-&-&\underline{90.9}&23.3&37.1&38.6&24.9&30.3&\textbf{70.6}&34.7&46.6\\
      &DEM\cite{zhang2017learning}&68.4& 67.2& 61.9 &51.7& 32.8&84.7&47.3&86.4&30.5&45.1&25.6&34.3&20.5&54.0&19.6&13.6\\
      &ALE\cite{akata2013label}&68.2& -& 60.8 &57.3& 61.4&57.9&59.6&68.9&52.1&59.4&36.6&42.6&39.4&57.7&43.7&49.7\\
      &LisGAN\cite{li2019leveraging}&70.6& -& 61.7 &58.8& 76.3&52.6&62.3&-&-&-&37.8&42.9&40.2&57.9&46.5&51.6 \\\hline
      \multirow{13}{*}{T}&GMN \cite{sariyildiz2019gradient}&82.5&-&64.6&64.3&79.2&70.8&74.8&-&-&-&70.6&60.2&65.0&40.7&57.1&47.5\\
      &DSRL\cite{ye2017zero}&74.7&72.8&56.8&48.7&74.7&20.8&32.6&-&-&-&25.0&17.7&20.7&39.0&17.3&24.0\\
      % &DecGAN&-&-&-&-&-&-&-&-&-&-&68.4&59.1&63.4&44.3&57.2&49.9\\
      &GFZSL\cite{verma2017simple}&48.1&78.6&50.0&64.0&67.2&31.7&43.1&-&-&-&45.8&24.9&32.2&-&-&-\\
      &ALE\_trans\cite{akata2013label}&-&70.7&54.5&55.7&-&-&-&73.0&12.6&21.5&45.1&23.5&30.9&22.6&19.9&21.2\\
      &PREN\cite{ye2019progressive}&-&78.6&66.4&62.8&-&-&-&88.6&32.4&47.4&55.8&35.2&43.1&27.2&35.4&30.8\\
      &f-VAEGAN$^\dagger$ \cite{xian2019f}&-&89.8&71.1/74.2$^*$&70.1&-&-&-&88.6&84.8&86.7&65.1&61.4&63.2&41.9&60.6&49.6\\
      &SABR-T$^\dagger$ \cite{paul2019semantically}&-&88.9&74.0&67.5&-&-&-&\textbf{91.0}&79.7&85.0&\textbf{73.7}&67.2&70.3&41.5&58.8&48.6\\
      &TF-VAEGAN$^\dagger$ \cite{narayan2020latent}&-&92.6&74.7/77.2$^*$&\underline{70.9}&-&-&-&89.6&87.3&88.4&\underline{72.1}&\underline{69.9}&\underline{71.0}&47.1&\underline{62.4}&\underline{53.7}\\
      &GXE\cite{li2019rethinking}&89.8&83.2&61.3&63.5&\textbf{89.0}&\underline{87.7}&\underline{88.4}&{90.0}&80.2&84.8&68.7&57.0&62.3&58.1&45.4&51.0\\
      &LSA$^\dagger$\cite{hanouti2022learning}&-&92.8&\;\;-\;\;/\underline{80.6}$^*$&71.7&-&-&-&86.7&88.5&87.6&-&-&-&\underline{59.5}&46.0&51.8\\
      &SDGN$^\dagger$ \cite{wu2020self}&\underline{92.3}&93.4&74.9&68.4&88.1&87.3&87.7&89.3&\underline{88.8}&\underline{89.1}&70.2&\underline{69.9}&70.1&46.0&62.0&52.8\\
      &STHS-WGAN$^\dagger$\cite{bo2021hardness}&-&\underline{94.9}&\textbf{77.4}&67.5&-&-&-&-&-&-&-&-&-&-&-&-\\\hline
      % TF-VAEGAN+FN+PreTune&-&77.0&71.7&-\\\hline
     T &Bi-VAEGAN$^\dagger$(ours) &\textbf{93.9}&\textbf{95.8}&\underline{76.8}/\textbf{82.8}$^*$&\textbf{74.2}&\underline{88.3}&\textbf{89.8}&\textbf{89.1}&\textbf{91.0}&\textbf{90.0}&\textbf{90.4}&71.7&\textbf{71.2}&\textbf{71.5}&45.4&\textbf{66.8}&\textbf{54.1}\\\hline
          \end{tabular}}\\
  \caption{TZSL performance comparison where the unseen class prior is provided when needed.``$\dagger$" denotes the method that adopts the known unseen class prior assumption. `*' denotes the result is obtained using fine-grained  visual descriptions (AK2 in Section \ref{sec:attribute}) for CUB and the competing results marked by `*' are cited from \cite{hanouti2022learning}. \label{tab:2}} 
  \vspace*{-0.3cm}

\end{table*}

\section{EXPERIMENT}

We conduct experiments using four datasets, including  AWA1 \cite{lampert2013attribute}, AWA2 \cite{xian2018zero}, CUB \cite{welinder2010caltech} and  SUN \cite{patterson2012sun}.  Visual features are extracted by the pretrained ResNet-101 \cite{he2016deep}. 
% For the fine-grained CUB dataset, there is a noticeable gap between the pretrained and the desired visual features. To improve this, we pre-tune the pretrained ResNet-101 features using a simple neural network for the CUB dataset, which gives a more distinct cluster structure. 
%
Details on the datasets and implementation details are provided in the supplementary material.
%
We conduct experiments at the feature level following  \cite{xian2018zero,narayan2020latent}. Under the TZSL setup, testing performance of the unseen classes is of interest, for which the average per class top-1 (T1)  accuracy is used, denoted by $\textmd{ACC}^u$ (U).   The ZSL community is also interested in the generalized TZSL performance, i.e., testing performance for both the seen and unseen classes \cite{kong2022compactness,pourpanah2022review,li2022siamese,feng2022non}.  We use the harmonic mean of the average per class top-1 accuracies of the seen and unseen classes to assess it, as  $H = \frac{2 \textmd{ACC}^u\times \textmd{ACC}^s}{\textmd{ACC}^u+ \textmd{ACC}^s}$.
%
All existing results reported in the tables come from their published papers. When  ``-" appears, such result is missing from the literature. 





% \begin{table*}[thp]
%     \centering
%     \footnotesize
%      \renewcommand{\multirowsetup}{\centering}

%      \resizebox{\textwidth}{!}{\begin{tabular}{l|lccccc}
%         \hline
%         \multicolumn{1}{c|}{\multirow{3}{*}{\textbf{Method}}}& \multicolumn{5}{c}{Transductive Zero-Shot Learning}\\\cline{2-6}
%         \multicolumn{1}{l|}{}&AWA1&AWA2&CUB(Att.)&CUB(Stc.)&SUN\\\cline{2-6}
%         \multicolumn{1}{l|}{}&T1&T1&T1&T1&T1\\\hline
%         f-VAEGAN$^\dagger$&-&89.8&71.1&74.2$^*$&70.1\\
%         SABR-T$^\dagger$ &-&88.9&74.0&-&67.5\\
%         TF-VAEGAN$^\dagger$ &-&92.6&74.7&77.2$^*$&\underline{70.9}\\
%         LSA$^\dagger$&-&92.8&-&\underline{80.6}$^*$&71.7\\
%         SDGN$^\dagger$ &\underline{92.3}&93.4&74.9&-&68.4\\
%         STHS-WGAN$^\dagger$&-&\underline{94.9}&\textbf{77.4}&-&67.5\\\hline
%         % TF-VAEGAN+FN+PreTune&-&77.0&71.7&-\\\hline
%         Bi-VAEGAN$^\dagger$(ours) &\textbf{93.9}&\textbf{95.8}&\underline{76.8}&\textbf{82.8}$^*$&\textbf{74.2}\\\hline
%             \end{tabular}}\\
% \end{table*}


  % \begin{figure}[htp]
  %     \centering
  %     \begin{subfigure}{0.24\linewidth}
  %     \centering
  %     \includegraphics[width=1.15\linewidth]{figures/visualization/AWA1.pdf}
  %     \caption{AWA1} 
  %     \label{fig.3.4}
  %     \end{subfigure}
  %     \begin{subfigure}{0.24\linewidth}
  %     \centering
  %     \includegraphics[width=1.15\linewidth]{figures/visualization/AWA2.pdf}
  %     \caption{AWA2} 
  %     \label{fig.3.1}
  %     \end{subfigure}
  %     \begin{subfigure}{0.24\linewidth}
  %     \centering
  %     \includegraphics[width=1.15\linewidth]{figures/visualization/cub.pdf}
  %     \caption{CUB} 
  %     \label{fig.3.2}
  %     \end{subfigure}
  %     \begin{subfigure}{0.24\linewidth}
  %     \centering
  %     \includegraphics[width=1.15\linewidth]{figures/visualization/sun.pdf}
  %     \caption{SUN} 
  %     \label{fig.3.3}
  %     \end{subfigure}
  %     % \begin{subfigure}{0.24\linewidth}
  %     % \centering
  %     % \includegraphics[width=1.15\linewidth]{figures/visualization/flo.pdf}
  %     % \caption{FLO} 
  %     % \label{fig.3.4}
  %     % \end{subfigure}
  %     \caption{The unseen classes frequency.\label{fig:att}}
  %     \end{figure}
   
\subsection{ Result Comparison and  Analysis }

\subsubsection{Known Unseen Class Prior}
We compare performance with both inductive  (I) and transductive (T) state of the arts under the same setting for fair comparison. For TZSL approaches, when class prior of the unseen classes is required, the compared existing techniques assume such information is provided. Therefore,  we first apply the same setting for the proposed Bi-VAEGAN. Table \ref{tab:2} reports the results.  To distinguish from our later results obtained by the proposed prior estimation approach, methods using the provided unseen prior are marked by ``$\dagger$".  Note that we do not report the generalized TZSL performance for STHS-WGAN here. This is because it uses a  harder evaluation setting different from the other approaches by assuming that the unseen and seen data are indistinguishable during training. 
%

It can be observed from Table \ref{tab:2} that in general, the transductive approaches outperform the inductive approaches with a large gap. The proposed Bi-VAEGAN outperforms the transductive state of the arts, particularly the two baseline frameworks F-VAEGAN and TF-VAEGAN that  Bi-VAEGAN is built on, on almost all the datasets. The new state-of-the-art performance that we have achieved is 93.9\% (AWA1), 95.8\% (AWA2),  and 74.2\% (SUN) for TZSL, while 89.1\% (AWA1), 90.4\% (AWA2), and 54.1\% (SUN) for generalized TZSL. Note that for the CUB dataset that has less intra-class clustering property, we find a simple feature pre-tuning network will further boost the performance from 76.8\% to 78.0\% and we include the discussion in the supplementary material.
%
It is worth mentioning that  Bi-VAEGAN achieves satisfactory SUN performance where the data is intra-class sample scarce. 
%
It is challenging to learn from the SUN dataset due to its low sample number of each class that inherently makes the conditional generation less discriminative. 
%
Bi-VAEGAN provides more discriminative features benefitting from its bi-directional alignment generation 
%
and the feature augmentation in Section \ref{sec:3.3.4}.
%
% We observe that the unseen accuracy (U) dominates more of the performance gain among methods, as compared to the seen accuracy (S).
\vspace*{-0.2cm}

\begin{table}[th]
  \centering
  \small
   \renewcommand{\multirowsetup}{\centering}
   \begin{tabular}{lcccc}
      \hline
      \multirow{1}{*}{\textbf{Method}}&AWA1&AWA2&CUB&SUN\\\hline
      \multicolumn{5}{l}{\textit{{Non-generative}}} \\

      
      DSRL\cite{ye2017zero}&74.7&72.8&56.8&48.7\\
      GXE\cite{li2019rethinking}&89.8&\underline{83.2}&61.3&63.5\\
      VSC \cite{wan2019transductive}&- &81.7&71.0&62.2\\\hline
      
      \multicolumn{5}{l}{\textit{{Generative with uniform prior}}} \\
      f-VAEGAN$^\ddagger $ \cite{xian2019f}&62.1&56.5&72.1&69.8\\
      TF-VAEGAN$^\ddagger$\cite{narayan2020latent}&63.0&58.6&\underline{74.5}&71.1\\
      % TF-VAEGAN+FN+PreTune&-&77.0&71.7&-\\\hline
      Bi-VAEGAN &66.3&60.3&\textbf{76.8}&\textbf{74.2}\\\hline
      \multicolumn{5}{l}{\textit{Generative with prior estimation}} \\
      Bi-VAEGAN (BBSE) &\underline{90.9}&83.1&72.5&68.4\\
      Bi-VAEGAN (CPE) &\textbf{91.5}&\textbf{85.6}&{74.0}&\underline{71.3}\\\hline
      %   T-VAEGAN+FN+PreTune&88.9 &77.3&70.3&-\\
      \end{tabular}\\
      \caption{Performance comparison in $\textmd{ACC}^u$ for both generative and non-generative techniques using different  unseen class priors. `$^\ddagger$' means our reproduced result.\label{tab:3}}
      \vspace*{-0.6cm}
  \end{table}    
  
  
\subsubsection{Unknown Unseen Class Prior}

In this experiment, the unseen class prior is not provided. For our method, we compare the proposed prior estimation  with a naive assumption of uniform class prior and a different approach that treats the prior estimation as a label shift problem and solves it by the black box shift estimation (BBSE) method \cite{lipton2018detecting}. BBSE builds upon the strong assumption that $p_G^u(\hat{\bm v}\mid y) = p^u({\bm v}\mid y)$, while our CPE assumes the cluster structure plays an important role in class prior estimation. 
%
Details on BBSE estimation are provided in the supplementary material.
%
In Table \ref{tab:3}, we compare our results with the existing ones, where, for methods that need unseen class prior, a uniform prior is used. 
%
By comparing Table \ref{tab:3} with Table \ref{tab:2} for the generative methods, it can be seen that, when the used unseen class prior differs significantly from the one computed from the real class sizes, there are significant performance drops, e.g.,  over $30\%$ for the extremely unbalanced AWA2 dataset.  
%
Both BBSE and CPE could provide a satisfactory prior estimation, while our CPE demonstrates consistently better performance.
%
It can be seen from Figure \ref{fig:att_est} that the CPE prior and the one computed from the real class sizes match pretty well for most classes, and for both the unbalanced and balanced datasets. 


\begin{figure}[th]
  \centering
  \begin{subfigure}{0.32\linewidth}
  \centering
  \includegraphics[width=1.1\linewidth]{figures/visualization/AWA1_estimate.pdf}
  \caption{AWA1} 
  \label{fig.5.4}
  \end{subfigure}
  \begin{subfigure}{0.32\linewidth}
  \centering
  \includegraphics[width=1.1\linewidth]{figures/visualization/AWA2_estimate.pdf}
  \caption{AWA2} 
  \label{fig.5.1}
  \end{subfigure}
  \begin{subfigure}{0.32\linewidth}
      \centering
      \includegraphics[width=1.1\linewidth]{figures/visualization/SUN_estimate.pdf}
      \caption{SUN} 
      \label{fig.5.3}
      \end{subfigure}
  \caption{Comparison between the estimated unseen class distribution prior by CPE (orange) and the provided prior computed from the class sizes (gray). \label{fig:att_est}}
  \vspace*{-0.55cm}
\end{figure}



\subsection{Ablation Study}



We perform an ablation study to examine the effect of the proposed  $L_2$-feature normalization (FN), 
%
the  vanilla inductive regressor (R) trained as the level-1 in Eq. (\ref{inductive_training}), 
%
and transductive regressor (TR) trained adversarially by adding $D^{a}$ as in Eq. (\ref{level1_training}). 
%
The Min-Max normalized f-VAEGAN is used as an alternative to FN, and  
%
the inductive regressor that is trained only with the paired seen data is used as an alternative to TR. 
%
One observation is that FN and TR consistently improve performance over four datasets, respectively. 
%
We conclude the following from  Table \ref{tab:4}: 
%
(1) The $L_2$-feature normalization is a free-lunch setting, requiring minimal effort but resulting in a satisfactory performance gain.
%
(2) A naive implementation of the inductive regressor is beneficial but somehow limited. The regressor trained solely with the seen attributes provides weak constraints to the unseen generation.
%
(3) The adversarially trained transductive regressor integrates the unseen attribute information and exhibits superiority in the bi-directional synthesis. 


  \begin{table}[t]
      \vspace*{-0.1cm}
      \centering
       \renewcommand{\multirowsetup}{\centering}
  
       \begin{tabular}{ccc|ccc}
          \hline
          \multicolumn{3}{c|}{\textbf{Method}}&\multirow{2}{*}{AWA2}&\multirow{2}{*}{CUB}&\multirow{2}{*}{SUN}\\\cline{1-3}
          %Zhicai: Does it make sense to change it to TR?
          FN& R&TR&&&\\\hline
          \xmark&\xmark&\xmark&91.6&72.1&69.8\\
          \xmark&\cmark&\xmark&92.3(+0.7)&74.3(+2.1)&70.8(+1.0)\\
          \xmark&\xmark&\cmark&95.5(+3.2)&75.8(+1.5)&72.2(+1.4)\\
          \cmark&\xmark&\cmark&\textbf{95.8}(+0.3)&\textbf{76.8}(+1.0)&\textbf{74.2}(+2.0)\\\hline
          \end{tabular}\\
  
      \caption{Ablation study of transductive ZSL results.}
  \label{tab:4}.
  \vspace*{-0.9cm}
  \end{table}
  
  \begin{figure}[th]
      \vspace*{-0.4cm}
      \centering
      \begin{subfigure}{0.48\linewidth}
          \centering
          \includegraphics[width=1.1\linewidth]{figures/visualization/norm.pdf}
          \caption{$L_2$ normalization} 
          \label{fig.4.1}
      \end{subfigure}
      \begin{subfigure}{0.48\linewidth}
          \centering
          \includegraphics[width=1.1\linewidth]{figures/visualization/sigmoid.pdf}
          \caption{Min-Max normalization} 
          \label{fig.4.2}
      \end{subfigure}
  
      \begin{subfigure}{0.48\linewidth}
          \centering
          \includegraphics[width=1.1\linewidth,height=3cm]{figures/visualization/acc.pdf}
          \caption{Training accuracy} 
          \label{fig.4.3}
      \end{subfigure} 
      \begin{subfigure}{0.48\linewidth}
          \centering
          \includegraphics[width=1.1\linewidth,height=3cm]{figures/visualization/f.pdf}
          \caption{Effect of $r$} 
          \label{fig.4.4}
      \end{subfigure}
      \caption{(a) and (b) compare the real and synthesized feature value distributions of AWA2 after $L_2$ and Min-Max normalizations, respectively. (c) compares the TZSL performance observed during the training for the two normalization approaches using the simplified model on AWA2 and CUB. (d) displays the TZSL performance for different radius values used in $L_2$-normalization. \label{fig:norm}}
      \vspace*{-0.5cm}
      \end{figure}    

\subsection{Further Examinations and Discussion}

\paragraph{On $L_2$-Feature Normalization}
\label{sec:norm}
The difference between using the proposed $L_2$-normalization  and the standard Min-Max normalization is the use of Eq. (\ref{V_norm}) or the sigmoid activation in the last normalization layer of the generator $\bm G$.
To demonstrate the difference between the two approaches, we perform a simple experiment, where the network structure is compressed to contain only three core modules $\bm G$, $D$, and $D^u$. The distributions of the real and synthesized visual features  after two normalizations are compared in Figures \ref{fig.4.1} and \ref{fig.4.2}. 
%
It can be seen that the $L_2$-normalization results in a better alignment between the two distributions, while Min-Max results in a quite significant gap between the two.
%
We investigate further the two approaches by looking into their partial derivatives with respect to each dimension, i.e., for the $L_2$ norm, $\frac{d v_i^\prime}{d v_i} =  \frac{r}{\|v_i\|_2}$, and for the Min-Max, $\frac{d \sigma(v_i)}{d v_i}= \sigma(v_i)(1-\sigma(v_i))$.
  % \vspace*{-0.2cm}
% \begin{align}
%    L_2: &\;  ,\\
%     \textmd{Min-Max:} &\; {\frac{d \sigma(v_i)}{d v_i}}= \sigma(v_i)(1-\sigma(v_i)).
% \vspace{-0.4cm}
% \end{align}
%
The sigmoid feature has a smaller derivative with a larger input magnitude and vice versa. 
%
This causes the sigmoid to skew the activated output towards the middle value e.g., 0.5, and this is not suitable when the feature distribution is skewed to one side, especially for those features last activated by ReLU.
%
It can be seen in Figure \ref{fig.4.3} that   the $L_2$-normalization performs better than Min-Max in terms of a higher accuracy and faster convergence in early training. 
%
We examine further the effect of the radius parameter $r$ on different datasets in Figure \ref{fig.4.4}. It is observed that a smaller  $r$ could lead to a more stable performance, while a larger $r$ results in an increased gradient that could potentially cause instability in the training. 

\vspace{-0.1cm}
  
  
  \begin{figure}[tp]
      \centering  
  
      \begin{subfigure}{0.49\linewidth}
          \centering
          \includegraphics[width=1\linewidth]{figures/visualization/acc_with_epochs.pdf}
      \end{subfigure}
      \begin{subfigure}{0.49\linewidth}
          \centering
          \includegraphics[width=1\linewidth]{figures/visualization/acc_with_sup_num.pdf}
          \end{subfigure}
      \caption{(a) compares the training accuracies for different auxiliary knowledge. (b) compares TZSL performance with and without the transductive regressor when knowledge AK3 is used.\label{fig:att}}
      \vspace*{-0.58cm}
      \end{figure}


\vspace*{-0.28cm}
     
\paragraph{On Auxiliary Knowledge of CUB} 
\label{sec:attribute}

Auxiliary knowledge plays an overly important role in the success of ZSL, and the motivation of TZSL is to reduce such dependency by learning from unlabelled examples from unseen classes. To examine the effectiveness of our regressor proposed to improve transductive learning and reduce dependency, we conduct experiments using different types of auxiliary knowledge on the CUB dataset.
These include the original 312-dim attribute vectors (AK1), the 1024-dim CNN-RNN embedding from fine-grained visual description \cite{reed2016learning} (AK2), and the 2048-dim averaged visual prototype features of n-shot support (unseen) dataset with label information as assumed in few-shot learning (AK3\#$n$) \cite{li2020adversarial}. 
%
A ground-truth prototype (AK3\#all) is the strongest auxiliary information where the arrival of a conditional alignment is easier. The effectiveness of the prototypes becomes weaker as the number of support examples decreases, where the outliers tend to dominate the generation. AK2 performs similarly well to the ground-truth prototypes, and this indicates that the embeddings learned from a large-scale language model can serve as a good approximation to the ground-truth visual prototypes. 
%
The training accuracies are compared in Figure \ref{fig:att}.(a). 
%Zhicai: Used TR earlier, so did not repeat transductive regressor here
Figure \ref{fig:att}.(b) is produced by removing our  regressor TR (using inductive R) when the model is conditioned on AK3. It is observed that when TR is not employed, a considerable gap opens up for the less informative prototypes. This demonstrates that TR can improve the model's robustness to auxiliary quality. It is important to note that when the auxiliary quality is extremely low (as in the AK3\#1 case), alignment in the unseen domain is hard to realize no matter whether the TR module is presented or not.



\begin{figure}[tp]
  \centering
  \begin{subfigure}{0.55\linewidth}
      \centering
      \includegraphics[width=1.04\linewidth]{figures/visualization/column.pdf}
  \end{subfigure}
  \begin{subfigure}{0.43\linewidth}
      \centering
      \includegraphics[width=1.05\linewidth]{figures/visualization/cluster_FA.pdf}
  \end{subfigure}
  \caption{(a) The classification accuracy of different classes in AWA2. `FA' denotes feature augmentation. (b) T-SNE visualization of augmented real/synthesized feature. \label{fig.fa}}
      \vspace*{-0.45cm}
  \end{figure}


\vspace*{-0.3cm}
%Zhicai: Check. I changed the augmented features to concatenated.  If not, you have adjust Sec 3.3.4 to mention augmentation.
\paragraph{On Feature Augmentation} 

The transductive regressor supports distinguishing difficult examples, while the concatenated features based on cross-modal knowledge (see Section \ref{sec:3.3.4}) improve the feature discriminability. Figure \ref{fig.fa}.(a) shows that the regressor leads to a better alignment for the less discriminative classes, such as `bat' and `walrus', and the concatenated features contribute significantly to hardness-aware alignment. Figure \ref{fig.fa}.(b) shows that for the resemble (hard) pairs, such as `walrus' and `seal', the concatenated features are better decoupled in this multi-modal space and the synthesis becomes more discriminative (compare with the visualization in Figure \ref{fig.1}).
  
  




%------------------------------------------------------------------------
\vspace*{-0.17cm}
\section{CONCLUSION}
We have presented a novel bi-directional cross-modal generative model for TZSL. By generating domain-aligned features for the unseen classes from both the forward and backward directions,  the distribution alignment between the visual and auxiliary spaces has been significantly enhanced. By conducting extensive experiments, we have discovered that $L_2$ normalization can result in a more stable training than the commonly used Min-Max normalization. 
%
We have also conducted a thorough examination of how the unseen class prior can affect the model performance and proposed a more effective prior estimation approach. 
This enables the generative approach to still be robust under the challenging scenario with unknown class priors. 

\vspace*{-0.1cm}
\section{ACKNOWLEDGEMENT}
This work is mainly supported by the National Key Research and Development Program of China (2021ZD0111802). It was also supported in part by the National Key Research and Development Program of China under Grant
2020YFB1406703, and by the National Natural Science Foundation of China (Grants No.62101524 and No.U21B2026).
% In this paper, we propose Bi-VAEGAN, which incorporates bidirectional alignment knowledge for the difficult TZSL problem. We propose a transductive regressor that maps visual features back to auxiliary space and improves classification for unpaired unseen data. Furthermore, we discuss the significance of feature preprocessing and class distribution prior for TZSL, and we propose $L_2$ normalization and CPE to address the issues. However, it is still unclear how to understand the underlying mechanism of this cross-modal domain adaptation problem and solve it in the hard prior-unknown case. We hope that this work will inspire other researchers working in this field. 
% \vspace{0.2cm}
%  We achieved competitive results of the image classification task in ImageNet1K compared with other state-of-the-art models. The key element of PosMLP is the relative positional encoding based token-mixing layer which achieves a high parameter efficiency and sample efficiency by explicitly modeling token relations, and we hope this could inspire more works of positional encoding for MLPs.  
% \newpage
% \newpage
%%%%%%%%% REFERENCES
{\small
\bibliographystyle{ieee_fullname}
\bibliography{egbib}
}
\newpage




\setcounter{section}{0}
\section*{Supplementary Materials}
We present additionally (1) the dataset and implementation details, (2) the explanation of the feature pre-tuning network, (3) an examination of different feature spaces, and (4) the comparison of BBSE and CPE for class prior estimation.

\section{Dataset and Implementation}
\subsection{Dataset} 
%
We conduct  experiments using four benchmark datasets.
%
The Animals with Attributes 1\&2 (AWA1 \cite{lampert2013attribute} \& AWA2 \cite{xian2018zero}) contain 30,475\&37,322 samples from a total of 50 classes, and the dimension of the attribute vector is 85.
%
The Caltech UCSD Bird 200 (CUB) \cite{welinder2010caltech} consists of 11,788 fine-grained images of 200 bird species with an attribute size of 312.
%
The SUN Scene classification (SUN) \cite{patterson2012sun} dataset has 14,340 samples selectecd  from 717 scenes with an attribute size of 102. More details are shown in Table \ref{tab:S1}.

\begin{table}[h] 
    \centering
            \setlength{\tabcolsep}{1.8mm}{
            \begin{tabular}{cccccc}\hline
    \textbf{Dataset}&$N$&att.&stc.& $\|\mathcal{Y}^s\|$& $\|\mathcal{Y}^u\|$\\\hline
    AWA1&30,475&85&-&40&10\\
    AWA2&37,322&85&-&40&10\\
    CUB&11,788&312&1,024&150&50\\
    SUN&14,340&102&-&645&72\\
    % SUN&102&N&645&72\\
        \hline
            \end{tabular}}
              \caption{Statistics of the four datasets. `att.' denotes the attribute size,  `stc.' is the dimension of semantic information extracted from descriptive sentences \cite{reed2016learning}, $\|\mathcal{Y}^s\|$ and $\|\mathcal{Y}^s\|$ correspond to the numbers of the seen and unseen classes, respectively.}
            \label{tab:S1}
    \end{table}
%
Figure \ref{fig:S1}  displays the class distribution prior  estimated from the  class information of the testing samples from the unseen classes, i.e., the percentage of the samples  contained by each class, for the four datasets. 
%
AWA1 and AWA2 have unbalanced class priors, while CUB and SUN have class priors close to a uniform distribution.
%
AWA2 has  more samples from those popular classes like `horse' and `dolphin'. 
  
    \begin{figure}[h]
        \centering
        \begin{subfigure}{0.45\linewidth}
        \centering
        \includegraphics[width=1.12\linewidth]{figures/visualization/AWA1.pdf}
        \caption{AWA1} 
        \label{fig.3.4}
        \end{subfigure}
        \begin{subfigure}{0.45\linewidth}
        \centering
        \includegraphics[width=1.12\linewidth]{figures/visualization/AWA2.pdf}
        \caption{AWA2} 
        \label{fig.3.1}
        \end{subfigure}

        \begin{subfigure}{0.45\linewidth}
        \centering
        \includegraphics[width=1.12\linewidth]{figures/visualization/cub.pdf}
        \caption{CUB} 
        \label{fig.3.2}
        \end{subfigure}
        \begin{subfigure}{0.45\linewidth}
        \centering
        \includegraphics[width=1.12\linewidth]{figures/visualization/sun.pdf}
        \caption{SUN} 
        \label{fig.3.3}
        \end{subfigure}
        \caption{The unseen class prior computed from test data.\label{fig:S1}}
        \end{figure}
        
\subsection{Implementation} 
%
In the training of  all our modules, we use AdamW optimizer \cite{DBLP:conf/iclr/LoshchilovH19} with a learning rate of 0.001 and ($\beta_1$, $\beta_2)$ is set as (0.5, 0.999). The encoder $\bm E$, decoder $\bm G$ and regressor $\bm R$ in Bi-VAEGAN are all two-layer MLPs, in which the hidden layer output has 4,096 dimensions and the inner activation layer is LeakyReLU. 
%
The conditional visual critic $D$, unconditional visual critic $D^u$, and attribute critic $D^a$ are two-layer MLPs where the output of the last layer is a scalar, and the WGAN gradient penalty coefficient is set as 10. 
%
The  level-1 and level-2 trainings proceed alternatively. We conduct one-step level-1  training  for every five steps of level-2 training to accelerate the training speed.
%
The training epochs for AWA1, AWA2, CUB, and SUN are set to be 300, 300, 600, and 400, respectively. In the inference stage, the synthesized feature number of each class  are set to be 3000, 3000, 400, and 400 for for AWA1, AWA2, CUB, and SUN, respectively. The classifier $f$ is a single fully connected (FC) layer and its output dimension is equal to the number of unseen classes for TZSL or the number of both seen and unseen classes for generalized TZSL. 
%

The used hyper-parameters for reporting results are $r$=1 for $L_2$ normalization, $\lambda$=1, $\alpha$=1, $\beta$=10 and $\gamma$=10, where the setting of $\alpha$, $\gamma$ and the WGAN critic training are the same as TF-VAEGAN \cite{narayan2020latent}. 
%
In level-1 training, $\lambda$ is less sensitive and thus set to 1. 
%
Values of $r$ and $\beta$ are searched within $\{1,2,5,\cdots,100\}$ and $\{0.01,0.1,1,10,100\}$, respectively.  Due to the unavailability of a test split in the datasets, we report our results on the validation split, consistent with previous works \cite{narayan2020latent,wu2020self}. For conventional Zero-Shot Learning (ZSL) and Generalized Zero-Shot Learning (GZSL), a more rigorous setting is desired, especially under the impact of current large models.

%Zhicai: Do you need to talk about how to determine hyperparameter? Worth to move any information from rebuttal relevant to this here, if there is any?

\section{Feature Pre-tuning Network}

\label{sec:pre-tune}
\begin{figure}[t]
    \centering
    \includegraphics[width=1\linewidth]{figures/model/pretune.pdf}
    \caption{The used feature pre-tuning network.}
    \label{fig:pre-tune}
\end{figure}


\subsection{Used Approach}

For the CUB dataset, we pre-tune the pre-trained features using a supervised neural network  of which the architecture is shown in Figure \ref{fig:pre-tune}.
%
It builds on an auto-encoder network ($\bm E^\prime$ and $\bm G^\prime$) and consists of two supervised modules that work in the latent space, acting as a regressor ($\bm R^\prime$) and a classifier (CLS). `$^\prime$' denotes it is a different module from the one in the main text.
%
The input and latent  features share the same feature dimension, i.e., 2,048 for the pre-trained ResNet-101.
%
Only the seen classes receive supervision from the two supervised modules. The  training  objective for feature pre-tuning is,
%Zhicai: This is not clear. E, G, R have \prime? What is the difference with and without prime? What is cls? Need to keep consistency (exactly the same style) as in the training objective functions in the main file/
\begin{align}
    \begin{split}
        \operatorname*{min}_{\bm E^\prime, \bm G^\prime, \bm R^\prime, CLS}  L_{MSE} + L_{\bm R^\prime}^s + L_{CLS}^s,
    \end{split}
\end{align}
where 
\begin{align}
L_{CLS}^s (\mathcal{V}^s) = \mathbb{E} [ \operatorname{log}(P(y|v^s)) ].
\end{align}
%
The latent features are extracted by the encoder $\bm E^\prime$ after training for 15 epochs for both the seen and unseen classes. These replace the original visual features to be used as the input of Bi-VAEGAN.
%,i.e, $\bm v_i \leftarrow \bm h_i := E^\prime(\bm v_i)$,  

\subsection{Result Comparison}

Figures \ref{tune.cub} and \ref{tune.awa2} visualize the tuned and untuned features for the CUB and AWA2 datasets, 
%Zhicai: What did you use for getting 2D visulisation? add
using the visulization tool t-SNE? .
% 
The tuned features exhibit  more clear cluster structure for the cross-domain dataset CUB. 
% Zhicai: I don't find this sentence useful, apart from the network names. Check whether it is really needed.
%
%Compared to the untuned features, there exist a gap with the pretrained domain (ImageNet1K \cite{deng2009imagenet} for pretrained ResNet-101).
%
%Zhicai: Check the revised English
It should be noted that our feature pre-tuning network will not be beneficial for datasets that already have a satisfactory cluster structure, and somehow the cluster property could be damaged.
%Zhicai: the English below is very poor. Change it.
%have a smaller domain gap with ImageNet1K

%Zhicai: I change vanilla to untuned, to match your text description.
\begin{figure}[t]
    \centering
    \begin{subfigure}{0.48\linewidth}
        \centering
        \includegraphics[width=1.12\linewidth]{figures/visualization/CUB_10_0.7358276844024658_untuned.pdf}
        \caption{untuned (CUB)} 
    \end{subfigure}
    \begin{subfigure}{0.48\linewidth}
    \centering
    \includegraphics[width=1.12\linewidth]{figures/visualization/CUB_E15_10_0.7358276844024658_tuned_.pdf}
    \caption{pre-tuned (CUB)} 
    \end{subfigure}
    \caption{Visualization of vanilla and pre-tuned CUB features.\label{tune.cub}}
    \end{figure}

\begin{figure}[t]
    \centering
    \begin{subfigure}{0.48\linewidth}
        \centering
        \includegraphics[width=1.12\linewidth]{figures/visualization/AwA2_50_0.9408364295959473_untuned.pdf}
        \caption{untuned (AWA2)} 
        \label{fig.10.1}
    \end{subfigure}
    \begin{subfigure}{0.48\linewidth}
    \centering
    \includegraphics[width=1.12\linewidth]{figures/visualization/AwA2_E15_50_0.9408364295959473_tuned_.pdf}
    \caption{pre-tuned (AWA2)} 
    \label{fig.10.2}
    \end{subfigure}
    \caption{Visualization of vanilla and pre-tuned AWA2 features.\label{tune.awa2}}
    \end{figure}


    \begin{table}[b]
        \centering 
        \small
        \begin{tabular}{cccc}
            \textbf{Model} & AWA2 & CUB$^{\text{AK1}}$& CUB$^{\text{AK2}}$ \\\hline
            \multicolumn{4}{l}{\textit{w/o pre-tuning}}\\
    
            Simple-GAN & 92.7   &68.1 & 79.8\\ 
            Bi-VAEGAN &  \textbf{95.8}   &76.8 &\textbf{82.8}\\\hline
            \multicolumn{4}{l}{\textit{w/ pre-tuning}}\\
            Simple-GAN &  88.9 ($-$3.8)  &76.9 ($+$8.8) & 80.3 ($+$0.5)\\ 
            Bi-VAEGAN & 90.0 ($-$5.8)   &\textbf{78.0} ($+$1.2) &82.0 ($-$0.8)\\\hline
        \end{tabular}
        \caption[]{The effect of feature pre-tuning on AWA2 and CUB. Simple-GAN is a simplified version of  Bi-VAEGAN. 
      %
      All performances are shown in percentage ($\%$).
      %  
      CUB$^{\text{AK1}}$ conditions on the original attribute information (AK1) while CUB$^\text{AK2}$ conditions on the semantic embedding (AK2) extracted from the fine-grained visual description. \label{tab:S2}}
    \end{table}





Table \ref{tab:S2} demonstrates the effect of feature pre-tuning on AWA2 and CUB datasets. We name the simplified model that only contains $\bm G$, $\bm D$, and $\bm D^u$ as a Simple-GAN. 
%
Both Simple-GAN and Bi-VAEGAN use $L_2$ feature normalization. A key observation is that for CUB, feature pre-tuning introduces a noticeable improvement for both models, i.e., $+$8.8 and $+$1.2 respectively, when using the less informative AK1 knowledge.
%
Notably, Simple-GAN significantly benefits from this straightforward strategy and performs comparably  to the untuned Bi-VAEGAN, e.g., 76.9\% vs. 76.8\%.
%
This shows that despite the fact that no additional supervision (regressor) is applied, the visual feature alignment for the tuned features is substantially simpler.
%
We could conclude that the tuned features can lead to a better inter-class discriminability, which enables an easier alignment between the auxiliary and visual spaces  when the class distribution prior is known.
%

Another observation is that  Simple-GAN  benefits less from the feature pre-tuneing ($+0.5$) when it conditions on the more informative AK2.
%
Bi-VAEGAN also shows a small performance drop ($-$0.8) with the feature pre-tuning. 
%
We could conclude that the pre-tuned features are less effective when the auxiliary information is already strong enough.
%
Besides, for the AWA2 dataset, pre-tuning decreases the inter-class discriminability as shown in Figure \ref{tune.awa2}, and a significant performance drop ($-$3.8, $-$5.8) is observed.
%
These indicate that feature pre-tuning is not a completely free-lunch approach and that cross-domain datasets may benefit more from it.
%
Transductive regressor could also achieve a competitive knowledge transfer for the cross-domain dataset.
%
It is easier to provide a better alignment since it doest not change the original features extracted from the powerful backbone. 
%
Overall, both the transductive regressor method and the feature pre-tuning offer advantages of their own and may complement one another in complex real-world  circumstances.


% \section{$L_2$ Normalization}
% \label{app:norm}

% \begin{figure}[htp]
  
%     \begin{subfigure}{0.48\linewidth}
%     \centering
%     \includegraphics[width=1.1\linewidth]{figures/visualization/norm_AwA2.pdf}
%     \caption{AWA2} 
%     \label{fig.11.2}
%     \end{subfigure}
%     \begin{subfigure}{0.48\linewidth}
%     \centering
%     \includegraphics[width=1.1\linewidth]{figures/visualization/norm_SUN.pdf}
%     \caption{SUN} 
%     \label{fig.11.3}
%     \end{subfigure}
%     % \begin{subfigure}{0.24\linewidth}
%     % \centering
%     % \includegraphics[width=1.15\linewidth]{figures/visualization/flo.pdf}
%     % \caption{FLO} 
%     % \label{fig.3.4}
%     % \end{subfigure}
%     \caption{ The value distribution of averaged 2048-d  real/synthesized visual features of our Bi-VAEGAN}
%     \end{figure}

\section{Feature Augmentation}
As a bi-directional distribution alignment technique for TZSL, our Bi-VAEGAN allows the regressor and generator to independently solve the TZSL problem.
% 
In the inference phase, we compare the performance of using four different feature spaces, i.e., attribute space $\mathcal{A}$, hidden space $\mathcal{H} \in \mathbb{R}^{4096}$  corresponding to the hidden representation of the regressor, visual space $\mathcal{V}$ and the augmented multi-modal space $\mathcal{A\times H\times V}$.
%
To conduct inference on $\mathcal{A}$, we have two straightforward choices:
%
(1) Use only the transductively trained $\bm R$ and infer for the test unseen data $\bm{R}(V^u)$ using a 1-nearest neighbor  (1-NN) classifier.
%Zhicai: check whether I misunderstood things
(2) Use both $\bm {G}$ and $\bm {R}$, synthesize the labeled unseen set $\langle \bm{R}(\hat{V}^u_{\bm G}),\hat{Y}^u_{\bm G} \rangle$ in attribute space,  train a neural network classifier using the  labeled set that includes the synthesized examples and infer for $\bm{R}(V^u)$ using this classifier. A similar method of inference can also be applied to the hidden space when this option is chosen.
%Zhicai: It is confusing here. You talked about three types on spaces to infer from, and two approaches for infer from A. what methods can be used to infer from other spaces? Also, how do you compare things in the discussion below. Which choices did you use?

{\bf Discussion.} 
Table \ref{tab:S3} shows the TZSL top-1 accuracy on three datasets using different spaces to conduct inference.
%
The observation could be summarized as, 
%
(1) $\bm{R}$ could be served as an individual module to conduct TZSL inference, but it is much less discriminative that $\bm{G}$.
%
(2) When using $\bm{G}$ to conduct inference,  a multi-modal space is preferred and the rank of spaces' discriminability is  $\mathcal{H}>\mathcal{V}>\mathcal{A}$. We attribute the hidden space absorbing the knowledge of both transductive generator and regressor and the larger dimensionality is also preferred to alleviate the hubness problem.

\begin{table}[h]   

\setlength\tabcolsep{5pt}
    \centering 
    \small
    \begin{tabular}{cccccc}
       \textbf{Module} &\textbf{Space} & AWA2 & CUB$^{\text{AK1}}$  & CUB$^{\text{AK2}}$ & SUN \\\hline
$\bm{R}$ & $\mathcal{A}$ & 73.2  &64.5  &45.0 &52.6\\\hline 
$\bm{G}$ & $\mathcal{V}$ &  94.2  &75.0 &81.8&71.8\\\hline
\multirow{3}{*}{$\bm{R, G}$}       &$\mathcal{A}$ & 89.8  & 65.6 &67.3&53.2\\ 
       & $\mathcal{H}$ &  \textbf{95.8}   &\textbf{77.2} &82.7&73.8\\
       & $\mathcal{A\times H\times V}$ &  \textbf{95.8}  &76.8 &\textbf{82.8}&\textbf{74.2}\\\hline

    \end{tabular}
    \caption[]{TZSL results of Bi-VAEGAN using different feature spaces.\label{tab:S3}}
\end{table}

\section{BBSE vs. CPE for Class Prior Estimation}

\begin{figure}[thp]
    \centering
    \begin{subfigure}{0.45\linewidth}
        \centering
        \includegraphics[width=1.12\linewidth]{figures/visualization/cpe_awa1_cls.pdf}
        \caption{AWA1 (BBSE)} 
        \label{fig.S3.1}
    \end{subfigure}
    \begin{subfigure}{0.45\linewidth}
    \centering
    \includegraphics[width=1.12\linewidth]{figures/visualization/cpe_awa1.pdf}
    \caption{AWA1 (CPE)} 
    \label{fig.S3.2}
    \end{subfigure}%

    \begin{subfigure}{0.45\linewidth}
        \centering
        \includegraphics[width=1.12\linewidth]{figures/visualization/cpe_awa2.pdf}
        \caption{AWA2 (CPE)} 
        \label{fig.S3.3}
        \end{subfigure}%
    \begin{subfigure}{0.45\linewidth}
        \centering
        \includegraphics[width=1.12\linewidth]{figures/visualization/cpe_cub.pdf}
        \caption{CUB (CPE)} 
        \label{fig.S3.3}
        \end{subfigure}%
        \caption{Training accuracy of Bi-VAEGAN using different random seeds when class prior is unknown.}
\end{figure}
Here we explain the  black box shift estimation  (BBSE) \cite{lipton2018detecting} approach for class prior estimation.
%
It attempts to solve the problem in label shift setting \cite{DBLP:conf/nips/GargWBL20} and we consider the TZSL problem in a discrete form i.e., $y \in Y^u = \{0,1,2,..., N_u - 1\}$.
%
We view our synthesized joint distribution $p^u_G(\hat{\bm v},y)$ as the source domain and the unknown joint distribution $p^u(\bm v,y)$ as the target domain.
% 
Under the label shift assumption, i.e., $p_G^{u}(\hat{\bm v}|y) = p^u(\bm v|y)$, we can approximate the unseen prior via the normalized confusion matrix $\bm{C}_{\hat{y},y} := p^u_{\bm G}(\hat{y}|y)$ of synthesized features, where $\hat{y} = f(\hat{\bm v})$ is the predicted label using hypothesis $f$.
%
Following \cite{lipton2018detecting}, when the label shift condition is held and the confusion matrix is invertible, the following equation holds,
\begin{align}
\nonumber
     p^u(\hat{y}) =& \sum\limits_{y\in Y^u} p^u(\hat{y}|y)p^u(y) =\sum\limits_{y\in Y^u} p^u_{\bm G}(\hat{y}|y)p^u(y),\\
      =&\sum\limits_{y\in Y^u} \bm{C}_{\hat{y},y}{p^u(y)},
\end{align} 
thus $p^u(y)$ is computed as,
\begin{align}
    p^u(y) = &\sum\limits_{\hat{y}\in Y^u} \bm{C}^{-1}_{y,\hat{y}}{p^u(\hat{y})}.
\end{align}
%
To compute the confusion matrix $\bm{C}$ we synthesize two labeled unseen set $\langle \hat{V}^u_{\bm G},\hat{Y}^u_{\bm G} \rangle_1$ and $\langle \hat{V}^u_{\bm G},\hat{Y}^u_{\bm G} \rangle_2$.
%
We train the hypothesis on one labeled set and compute the confusion matrix on the other set. Note that as the training process goes, the confusion matrix tends to be an identity matrix and the BBSE estimation collapse to $p^u(y) \leftarrow p^u(\hat{y})$.



\begin{figure}[tp]
    \centering
    \begin{subfigure}{0.48\linewidth}
        \centering
        \includegraphics[width=1.12\linewidth]{figures/visualization/E_3Norm_r1_beta1_S2000WithSyn_acc_0.91796875.pth.pdf}
        \caption{AWA1 + CPE (T1-91.5\%)} 
        \label{fig.S4.1}
    \end{subfigure}
    \begin{subfigure}{0.48\linewidth}
        \centering
        \includegraphics[width=1.12\linewidth]{figures/visualization/E_3Norm_r1_beta1_S2000WithSyn_acc_0.9394033551216125.pth.pdf}
        \caption{AWA1 (T1-93.9\%)} 

        \label{fig.S4.2}
    \end{subfigure}

    \begin{subfigure}{0.48\linewidth}
        \centering
        \includegraphics[width=1.12\linewidth]{figures/visualization/E_3Norm_r1_beta1_S2000WithSyn_acc_0.8595021963119507.pth}
        \caption{AWA2 + CPE (T1-85.6\%)} 
        \label{fig.S4.3}
    \end{subfigure}
    \begin{subfigure}{0.48\linewidth}
    \centering
    \includegraphics[width=1.12\linewidth]{figures/visualization/E_3Norm_r1_beta1_S2000WithSyn_acc_0.9580051302909851.pth.pdf}
    \caption{AWA2 (T1-95.8\%)} 
    \label{fig.S4.4}
    \end{subfigure}%
    \caption{Visualization of real/synthesized unseen visual feature using Bi-VAEGAN. Left column uses CPE strategy and class prior is unknown. The right column is trained with given real class prior. `$\circ$' means the real feature and `+' means the syntehsized feature.\label{fig.S4}}
\end{figure}

{\bf{Discussion.}} 
We display the BBSE and our CPE's training accuracy curves on AWA1 in Figure \ref{fig.S3.1} and Figure \ref{fig.S3.2}. It might be observed that BBSE is more vulnerable to seed selection and that it more readily results in a poor convergence. 
%
This observation can be explained as that the label shift assumption is too strong for prior estimation, so that the neural network classifier performs more unstably. 
%
The non-parametric K-means technique tends to provide a more moderate and reliable estimation since CPE avoids directly employing the black-box neural network classifier and utilizes it as an initialization approximation of the class center instead.

Figure \ref{fig.S4} shows the t-SNE visualizations using CPE when the class prior is unknown. For the more evenly balanced AWA1, our CPE provides a satisfactory alignment between the real and the syntehsized features, and there is only a minor accuracy gap with the known prior scenario (91.5\% vs. 93.9\%).
%
For the more unbalanced AWA2 dataset, the domain between the synthesized and real features shifts noticeably, and the performance disparity with the know-prior scenario increases to (85.6\% vs. 95.8\%). 
% 
This supports the argument  of Corollary 3.1 that the class prior is crucial to the alignment of the conditional distribution for the TZSL.
% 
However, it is still unclear how to proceed with a more accurate class prior estimation when the real prior is highly unbalanced. Different from the widely studied problems of  \textit{covariate shift} and \textit{label shift} in domain adaptation \cite{DBLP:conf/icml/0002CZG19, lipton2018detecting}, the unknown prior TZSL is less well-organized and is more similar to a cross-modal \textit{generalized label shift} problem. 
% \section{Enhanced visual feature with regressor}


    % \IncMargin{1em}
    % \begin{algorithm*} \SetKwData{Left}{left}\SetKwData{This}{this}\SetKwData{Up}{up} \SetKwFunction{Union}{Union}\SetKwFunction{FindCompress}{FindCompress} \SetKwInOut{Input}{input}\SetKwInOut{Output}{output}\SetKwInOut{Parameter}{Parameters}\SetKwInOut{Data}{Data}
    %     \Data{ Paired seen data $\langle V^s,Y^s \rangle$, unseen data $V^u$, attribute set \{$A^s$,$A^u$\}.}
    %     \Input{Uniform initialized sample prior $p_{\bm G}^u(y)$, unseen classes number $N_u$, epochs $T_i$ and $T_j$.}
    %     \Parameter{$\bm E$, $\bm G$, $\bm R$, $D$, $D^u$, $D^a$, classifier $f$.}
    %      \For{ $i$ \KwTo $T_i$}{
    %          \tcp{Inductive training}
    %          \While{not done}{
    %                 $\operatorname*{min}_{\bm R} L_{R}^{s}$ \;
    %              $\operatorname*{min}_{\bm E,\bm G}\operatorname*{max}_{D} L^s_{\text{VAE}} + L_{D\text{-WGAN}}^{s} + L_{R}^{u}.$\;
    %             }
    %       }
    %      \For{ $j$ \KwTo $T_j$}{
    %         Synthesize $\hat V^u$ with $\bm G$\ using $Y^u_{\bm G}$\tcc*{$Y^u_{\bm G}$ is pre-defined label set}
    %         Initialize $f$ and train via $\langle \hat V^u, Y^u_{\bm G}\rangle$\;
    %         $\hat Y^u = f(V^u)$ \tcc*{Calculate pseudo label}
    %         $\{P^u\}\leftarrow\langle V^u, \hat Y^u\rangle$ \tcc*{Calculate Pseudo Prototype}
    %         InitCenter$\leftarrow \{P^u\}$ \;
    %         \emph{$\hat Y_{kmeans}^u$ = Kmeans($V^u$,$N_u$, InitCenter)} \tcc*{Calculate pseudo label via kmeans}
    %         $p_G^u(y) \leftarrow  \hat Y_{kmeans}^u$  \tcc*{Update sample prior}
    %         \tcp{Transductive training}
    %         \While{not done}{
    %             $\operatorname*{min}_{\bm R}\operatorname*{max}_{D^a} L_{R}^{s}+L_{D^a\text{-WGAN}}^{u}.$ \tcc*{Level 1}
    %              $\operatorname*{min}_{\bm E,\bm G}\operatorname*{max}_{D,D^u} L^s_{\text{VAE}} + L_{D\text{-WGAN}}^{s} + L_{R}^{u}+L_{D^u\text{-WGAN}}^{u}.$ \tcc*{Level 2}
    %             }
             
    %       }
    %     %  \emph{}\; 
    %     \caption{Cluster prior estimation (CPE)}
    %            \label{cpe} 
    %       \end{algorithm*}
    %  \DecMargin{1em} 

\end{document}
