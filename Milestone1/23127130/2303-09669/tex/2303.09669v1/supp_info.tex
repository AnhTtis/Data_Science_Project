\documentclass{article}

\usepackage[english]{babel}
\usepackage{graphicx}% Include figure files
\usepackage{bm}% bold math
\usepackage{siunitx}
\usepackage{xfrac}
\usepackage{authblk}
\usepackage[letterpaper,top=2cm,bottom=2cm,left=3cm,right=3cm,marginparwidth=1.75cm]{geometry}

\renewcommand{\thefigure}{S\arabic{figure}}

\title{Predicting discrete-time bifurcations with deep learning\\ ---Supplemental Material---}

\author[1]{Thomas M. Bury}
\author[2]{Daniel Dylewsky}
\author[2]{Chris T. Bauch}
\author[3]{Madhur Anand}
\author[1]{Leon Glass}
\author[1*]{Alvin Shrier}
\author[1*]{Gil Bub}

\affil[1]{Department of Physiology, McGill University, 3655 Promenade Sir William Osler, Montreal, Quebec H3G 1Y6, Canada}
\affil[2]{Department of Applied Mathematics, University of Waterloo, Waterloo, Canada}
\affil[3]{School of Environmental Sciences, University of Guelph, Guelph, Canada}
\affil[*]{Joint last authors}

\date{}
\setcounter{Maxaffil}{0}
\renewcommand\Affilfont{\itshape\small}


\begin{document}

\maketitle

\begin{figure}[ht]
    \centering
    \includegraphics[width=0.7\textwidth]{figures/figure_s1.png}
    \caption{Confusion matrices summarising the performance of Classifier 1 and Classifier 2 on their respective test sets for the multi-class and binary classification problem. Cell values show (row-)normalised classification rates for each class. (A) Classifier 1 on the multi-class classification problem obtains an F1 score of $0.66$. (B) Classifier 2 on the multi-class classification problem obtains an F1 score of $0.85$. (C) Classifier 1 on the binary classification problem obtains an F1 score of $0.82$. (D) Classifier 2 on the binary classification problem obtains an F1 score of $0.98$. PD: period-doubling. NS: Neimark-Sacker. TC: transcritical. PF: pitchfork.}
    \label{fig:confusion_matrices}
\end{figure}


\begin{figure}[ht]
    \centering
    \includegraphics[width=0.5\textwidth]{figures/figure_s2.png}
    \caption{Sample simulations and corresponding bifurcation diagrams for each class in the training data. Panels show the trajectory of $x_n$ (blue), the corresponding bifurcation diagram (black) and the section used for training in each case (between the green vertical lines). The six possible classes are (a) null, (b) period-doubling, (c) Neimark-Sacker, (d) fold, (e) transcritical, and (f) pitchfork. These samples show supercritical versions of the period-doubling, Neimark-Sacker and pitchfork bifurcations, though subcritical examples are also present in the training data. The end time for the extracted data is taken as the time the bifurcation is crossed ($t=600$) or the first time the deviation from equilibrium reaches ten times the noise amplitude, whichever comes first.}
    \label{fig:training_samples}
\end{figure}


\begin{figure}[ht]
    \centering
    \includegraphics[width=0.9\textwidth]{figures/figure_s3.png}
    \caption{Sample simulations of each theoretical model approaching a bifurcation under a range of noise amplitudes (sigma) and rates of forcing (colour). The rates of forcing are chosen such that the bifurcation is crossed after 100 (green), 300 (red) and 500 (blue) units of time. The Fox model goes through a period-doubling bifurcation, the Westerhoff model a Neimark-Sacker bifurcation, the Ricker model a fold bifurcation, the Lotka-Volterra model a transcritical bifurcation, and the Lorenz model a pitchfork bifurcation. Model parameters are provided in Methods.}
    \label{fig:model_samples}
\end{figure}


\begin{figure}[ht]
    \centering
    \includegraphics[width=0.9\textwidth]{figures/figure_s4.png}
    \caption{AUC score (area under the ROC curve) at different values of noise amplitude (sigma) and rate of forcing (RoF) for each indicator (columns) and theoretical model (rows). At each combination of RoF and sigma, we run 100 forced and null simulations, resulting in a total of 5000 simulations for each model. ROC curves are computed using predictions at $80\%$ of the way through the pretransition time series. The bifurcations associated with each model are period-doubling (Fox), Neimark-Sacker (Westerhoff), fold (Ricker), transcritical (Lotka-Volterra), and pitchfork (Lorenz).}
    \label{fig:auc_rof_sigma}
\end{figure}

\begin{figure}[ht]
    \centering
    \includegraphics[width=0.4\textwidth]{figures/figure_s5.png}
    \caption{Proportion of predictions from the classifier that favoured the correct bifurcation for forced trajectories of different noise amplitude (sigma) and rate of forcing (RoF). For each combination of RoF and sigma, 100 forced trajectories are simulated, and predictions are made 80\% of the way through the pretransition time series. The bifurcations associated with each model are period-doubling (Fox), Neimark-Sacker (Westerhoff), fold (Ricker), transcritical (Lotka-Volterra), and pitchfork (Lorenz).}
    \label{fig:dl_rof_sigma}
\end{figure}


\begin{figure}[ht!]
    \centering
    \includegraphics[width=\textwidth]{figures/figure_s6.png}
    \caption{Trends in indicators prior to a period doubling bifurcation in chick heart aggregates (a-l). See Fig.~3 caption for details.}
    \label{fig:chick_pd_1}
\end{figure}

\begin{figure}[ht!]
    \centering
    \includegraphics[width=\textwidth]{figures/figure_s7.png}
    \caption{Trends in indicators prior to a period doubling bifurcation in chick heart aggregates (m-w). See Fig.~3 caption for details.}
    \label{fig:chick_pd_2}
\end{figure}

\begin{figure}[ht!]
    \centering
    \includegraphics[width=\textwidth]{figures/figure_s8.png}
    \caption{Trends in indicators for chick heart aggregates that do not undergo a period-doubling bifurcation (a-l). See Fig.~3 caption for details.}
    \label{fig:chick_null_1}
\end{figure}

\begin{figure}[ht!]
    \centering
    \includegraphics[width=\textwidth]{figures/figure_s9.png}
    \caption{Trends in indicators for chick heart aggregates that do not undergo a period-doubling bifurcation (m-w). See Fig.~3 caption for details.}
    \label{fig:chick_null_2}
\end{figure}


\end{document}