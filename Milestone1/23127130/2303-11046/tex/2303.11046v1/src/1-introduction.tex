\section{Introduction}
This study concerns improving the smoothing method for solving extensive-form games with imperfect information.
An \textit{extensive-form game} is a description of a game played by multiple players, where the game state corresponds to a node in a rooted tree.
The state of the game moves down the tree according to the actions of each player, and each player receives a gain when a leaf of the tree is reached.
This game is a good model for games with information gaps between players, such as poker, because it can represent the incomplete information.
We only deal with games in which there are two players, and the sum of their payoff is zero. 
It is known that in such two-person zero-sum games, the problem of finding the optimal strategy can be formulated as a simple minimax problem since the players do not cooperate with each other.

There are two well-known methods for solving extensive-form games: \textit{counterfactual regret minimization} (CFR)~\cite{zinkevich2007regret} and \textit{excessive gap technique} (EGT)~\cite{nesterov2005excessive}.
CFR is an application of regret minimization, a framework used in online learning, to extensive-form games.
Its variant, CFR+~\cite{tammelin2014solving}, was developed to solve a variant of a two-player poker game called Texas Hold'em, which was a challenging problem in artificial intelligence~\cite{bowling2015heads, brown2018superhuman}.
CFR+ is suited for analyzing large games because the solution error is bounded linearly with the game size.
However, it is only guaranteed to converge at $\order{1/\sqrt{k}}$ rate, where $k$ is the number of iterations.
EGT is a smoothing method for the bilinear saddle-point problem, and its application to extensive-form games has been studied~\cite{hoda2010smoothing}.
EGT is theoretically guaranteed to converge with rate $\order{1/k}$, which is faster than CFR+, but it is not suitable for solving large games.
This is because the term $D/\sigma$, which appears in the error bound, depends poorly on the game size.
Here $\sigma$ is the strong convexity parameter of a function $d$ called \textit{prox-function} used in EGT and $D:=\max_{\bm x} d(\bm x)$.
Various prox-functions have been proposed in the literature~\cite{hoda2010smoothing,kroer2020faster,farina2021better}, with $D/\sigma$ being bounded by a cubic order of the game size.

Our goal is to improve the prox-function so that EGT can be applied to large extensive-form games.
To this purpose, we make two contributions.
First, we improve the bound of $D/\sigma$ from $M_Q^3\ln\max_{I\in\mathcal{I}}\abs{\A(I)}$ to $M_Q^2\ln\abs{\Sigma}$ by eliminating the first-order term of the prox-function proposed in~\cite{farina2021better}.
Here $M_Q$ is the maximum value of the $L_1$-norm among the feasible points of the bilinear saddle-point problem, and $\abs{\Sigma}$ is the dimension of the feasible set, both of which are bounded by the game size.
Also, $\max_{I\in\mathcal{I}}\abs{\A(I)}$ is the maximum number of legal actions at each decision point.
Second, we propose a heuristic in EGT that we call the centering trick.
This trick modifies the prox-function in a way that it takes the minimum value in a temporary solution. 
It is expected to improve the accuracy of the smooth approximation of the bilinear saddle-point problem inside the EGT.
This heuristic accelerates convergence in practice and can be combined with other methods by using their solutions.
Numerical experiments show that EGT with the heuristic combined with CFR+ performs best among several methods, including CFR+, for games of a scale where conventional EGT does not perform well.
This suggests that the proposed smoothing method is effective even for large games.
%This suggests that smoothing methods may be helpful even for large games.

The structure of this paper is given as follows.
Section \ref{sec-pre} introduces the basics of convexity analysis.
Section \ref{sec-smoothing} presents the bilinear saddle-point problem, the prox-function, and the smoothing method (EGT).
Section \ref{sec-efg} introduces the extensive-form games and explains how to transform the problem of finding the optimal strategy of the game into a problem covered by EGT.
In Section \ref{sec-prox}, we propose a prox-function, and we will show that it improves theoretical convergence.
Section \ref{sec-experiments} provides the centering heuristic that accelerates the convergence of EGT in practice and confirms its performance through numerical experiments.
Section \ref{sec-conclusions} summarizes our contributions.