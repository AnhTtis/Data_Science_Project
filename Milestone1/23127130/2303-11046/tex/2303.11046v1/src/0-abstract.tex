The extensive-form game has been studied considerably in recent years.
It can represent games with multiple decision points and incomplete information, and hence it is helpful in formulating games with uncertain inputs, such as poker.
We consider an extended-form game with two players and zero-sum, i.e., the sum of their payoffs is always zero.
In such games, the problem of finding the optimal strategy can be formulated as a bilinear saddle-point problem.
This formulation grows huge depending on the size of the game, since it has variables representing the strategies at all decision points for each player.
To solve such large-scale bilinear saddle-point problems, the excessive gap technique (EGT), a smoothing method, has been studied.
This method generates a sequence of approximate solutions whose error is guaranteed to converge at $\order{1/k}$, where $k$ is the number of iterations.
However, it has the disadvantage of having poor theoretical bounds on the error related to the game size.
This makes it inapplicable to large games.

Our goal is to improve the smoothing method for solving extensive-form games so that it can be applied to large-scale games.
To this end, we make two contributions in this work.
First, we slightly modify the strongly convex function used in the smoothing method in order to improve the theoretical bounds related to the game size.
Second, we propose a heuristic called centering trick, which allows the smoothing method to be combined with other methods and consequently accelerates the convergence in practice.
As a result, we combine EGT with CFR+, a state-of-the-art method for extensive-form games, to achieve good performance in games where conventional smoothing methods do not perform well.
The proposed smoothing method is shown to have the potential to solve large games in practice.