\section{Numerical experiments}
\label{sec-experiments}
In this section, we report the results of solving extensive-form games by using the prox-function \eqref{eqn-prox} with EGT.
As toy instances, we used three games, Kuhn poker, Leduc Hold'em (3 ranks), and Leduc Hold'em (13 ranks), which we explain in detail in Appendix \ref{sec-app-games}.
All implementations used in the experiments are available on~\url{https://github.com/habara-k/egt-on-efg}.

\subsection{Heuristics to accelerate the convergence of EGT}
The parameters and choices between $\mu_1$ and $\mu_2$ to shrink proposed in~\cite{nesterov2005excessive} guarantee the convergence of \eqref{eqn-conv} but are very conservative, then heuristic-based parameter selection will perform better in most cases.
First, we always use the following heuristics in~\cite{farina2021better}:
\begin{itemize}
    \item To start with small $\mu_1$ and $\mu_2$, we call the initializing algorithm \ref{alg-init} with $\mu_1=\mu_2=10^{-6}$.
    Increase $\mu_1$ and $\mu_2$ by 20\% until the output satisfies the excessive gap condition.
    \item In each step, shrink the larger between $\mu_1$ and $\mu_2$.
    \item To obtain a large step size, we call the shrinking algorithm \ref{alg-shrink} with the global $\tau$, which is initially set to 0.5 and is decreased by 50\% while the output does not satisfy the excessive gap condition.
\end{itemize}
See Algorithm \ref{alg-egt} for details.
In addition to these heuristics, this paper proposes a \textit{centering trick}, which is still not considered in the related literature to the best of our knowledge. 
For $\bm x^\prime\in \ri Q$, we can define the following prox-function:
\begin{align}
    \tilde{d}(\bm x; \bm x^\prime) :=
    d(\bm x) - d(\bm x^\prime) - \nabla d(\bm x^\prime)^\T \qty(
    \bm x - \bm x^\prime
    ),
    \label{eqn-prox-z}
\end{align}
which we call $\bm x^\prime$\textit{-centered} prox-function because $\argmin_{\bm x\in Q} \tilde{d}(\bm x; \bm x^\prime) = \bm x^\prime$ holds.
The centering trick uses $\tilde{d}(\bm x;\bm x^\prime)$ as a prox-function for EGT, where $\bm x^\prime$ is a solution obtained by some other method.
The reason for using this centered prox-function is to improve the accuracy of the smoothing approximation $f_{\mu_2}(\phi_{\mu_1})$ in EGT.
In fact, when the prox-function $d_2(d_1)$ takes the minimum value 0 in the optimal solution $y^*(x^*)$ of BSPP, the minimization of $f$ and $f_{\mu_2}$ (the maximization of $\phi$ and $\phi_{\mu_1}$) are equivalent, for any $\mu_2(\mu_1) > 0$.
In addition, we show that
\begin{align}
     \nabla \tilde{d}^*(\bm\xi; \bm x^\prime) 
     := \argmax_{\bm x\in Q} \qty{\bm\xi^\T \bm x - \tilde{d}(\bm x; \bm x^\prime)} 
     = \nabla d^*(\bm\xi + \nabla d(\bm x^\prime)),
\end{align}
so the cost required for the calculation is also $\order{\abs{\Sigma}}$.
Numerical experiments evaluate the performance of the centering trick.

\subsection{Results of the experiments}
Five methods are used: CFR~\cite{zinkevich2007regret}, CFR+~\cite{tammelin2014solving}, EGT, EGT-centering, and EGT-centering with CFR+.
EGT uses the prox-function defined in \eqref{eqn-prox}.
In EGT-centering, the first EGT is performed in 10\% of the total steps, and the remaining 90\% of the steps are performed in the second EGT using the prox-function centered on the solution obtained from the first EGT. 
EGT-centering with CFR+ is a variant of EGT-centering, using CFR+ instead of the first EGT.

From Figure \ref{fig-experiments},
note that EGT-centering and EGT-centering with CFR+ both use EGT and CFR+ for the first 10\% of iterations, respectively, so only the last 90\% of iterations are drawn.
First, since Kuhn poker is a very small game, EGT, which is unsuitable for larger games, performs similarly to CFR+. EGT with centering tricks using the solutions of these two methods, namely EGT-centering and EGT-centering with CFR+, converge faster than the other methods.
Second, since Leduc Hold'em is a larger game than Kuhn poker, EGT converges worse than CFR+. However, we see that EGT-centering performs as well as CFR+ and that EGT-centering with CFR+ converges better than pure CFR+.

These results show that EGT combined with CFR+ by the centering trick converges faster than pure CFR+ in large games where EGT alone performs worse than CFR+.
In addition, although we could not experiment in this paper, the results suggest that EGT with the centering trick has the potential to further exploit the performance of CFR+ in very large games such as those used in~\cite{bowling2015heads, brown2018superhuman}.
Note that the implementations of all five methods are optimized in the same way, resulting in EGT (including the centering trick) taking at most 2 to 3 times longer per iteration than CFR (CFR+).
Thus, although the horizontal axis in Figure~\ref{fig-experiments} represents the number of iterations, changing this to the running time (computational cost) yields almost the same result.

\begin{figure}[htbp]
    \begin{minipage}[b]{\columnwidth}
        \centering
        \includegraphics[width=120mm]{images/20230202-kuhn.png}
    \end{minipage}
    \begin{minipage}[b]{\columnwidth}
        \centering
        \includegraphics[width=120mm]{images/20230202-leduc.png}
    \end{minipage}
    \begin{minipage}[b]{\columnwidth}
        \centering
        \includegraphics[width=120mm]{images/20230202-leduc13.png}
    \end{minipage}
    \caption{Performance of the five methods for solving three games.}
    \label{fig-experiments}
\end{figure}

\begin{algorithm}[htbp]
\caption{Excessive gap technique (with heuristics in~\cite{farina2021better})}
\label{alg-egt}
\begin{algorithmic}[1]
\Function{init}{\null}
\State $\mu\gets 10^{-6}$
\While{true}
    \State $\bm x, \bm y \gets$ Call algorithm \ref{alg-init} with $\mu_1=\mu_2=\mu$.
    \If{$f_{\mu_2}(\bm x) \le \phi_{\mu_1}(\bm y)$}
        \State \Return $\bm x, \bm y, \mu, \mu$
    \EndIf
    \State $\mu \gets 1.2\mu$
\EndWhile
\EndFunction
\Statex
\Function{decrease$\mu_1$}{$\bm x, \bm y, \mu_1, \mu_2, \tau$}
    \While{true}
        \State $\bar{\bm x},\bar{\bm y} \gets $ Call algorithm \ref{alg-shrink} with $\bm x, \bm y, \mu_1, \mu_2,$ and $\tau$.
        \If{$f_{\mu_2}(\bar{\bm x}) \le \phi_{(1-\tau)\mu_1}(\bar{\bm y})$}
            \State \Return $\bar{\bm x}, \bar{\bm y}, (1-\tau)\mu_1, \tau$
        \EndIf
        \State $\tau \gets 0.5\tau$
    \EndWhile
\EndFunction
\Statex
\Ensure{$\bm x\in Q_1, \bm y\in Q_2$: solutions of EGT}
\State $\bm x, \bm y, \mu_1, \mu_2 \gets $\Call{init}{\null}
\State $\tau \gets 0.5$
\For{$k=1,\dots,T-1$}
    \If{$\mu_1 > \mu_2$}
        \State $\bm x, \bm y, \mu_1, \tau \gets $ \Call{decrease$\mu_1$}{$\bm x, \bm y, \mu_1, \mu_2, \tau$}
    \Else
        \State decrease $\mu_2$ similarly.
    \EndIf
\EndFor
\end{algorithmic}
\end{algorithm}

