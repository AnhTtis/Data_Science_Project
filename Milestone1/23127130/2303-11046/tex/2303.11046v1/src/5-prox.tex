\section{Prox-function over the strategy set}
\label{sec-prox}
For solving extensive-form games with EGT, we need a prox-function defined on the convex compact set $Q_i$, defined by~\eqref{eqn-strategy}, the strategy set of each player $i=1,2$.
To guarantee good convergence, it is necessary to define a prox-function $d_i\colon Q_i\to\R$ with large \textit{substantial strong convexity} $\sigma_i/D_i$.

For simplicity, we omit $i$ denoting the player in this section.
Let $M_{Q} := \max_{\bm x\in Q} \norm{\bm x}_1$, which represents the scale of the game.
The prox-function proposed in a previous study~\cite{farina2021better} is shown to be $1/(M_{Q}^3\ln\max_{I\in\mathcal{I}}\abs{\A(I)})$-strongly convex substantially with respect to the $L_1$-norm.
We propose a slightly modified version of this prox-function and show that it is $1/(M_{Q}^2\ln\abs{\Sigma})$-strongly convex substantially with respect to the $L_1$-norm, which is a better guarantee for most games.

Now we propose the prox-function $d\colon Q\to\R$ defined by
\begin{align}
    d(\bm x) := x_\emp\ln x_\emp 
    + \sum_{I\in\mathcal{I}}\sum_{a\in\A(I)} 
    \qty(w_I - \sum_{p(I^\prime)=(I,a)} w_{I^\prime})
    \ x_{I,a}\ln x_{I,a},
    \label{eqn-prox}
\end{align}
where $w_I\in\R$ is defined recursively:
\begin{align}
    w_I := 1 + \max_{a\in\A(I)} 
    \sum_{p(I^\prime)=(I,a)} w_{I^\prime}
    \quad \forall I\in\mathcal{I},
    \label{eqn-w}
\end{align}
and its base case is given by $I\in\mathcal{I}$ with $\qty{I^\prime\in\mathcal{I}\mid p(I^\prime) = (I, a)} = \emptyset$ for all $a\in\A(I)$.
We will denote $\ln x$ as the natural logarithm of $x\ge0$ and assume $0\ln0=0$.
Note that \eqref{eqn-prox} does not satisfy $\min_{\bm x\in Q} d(\bm x)=0$, the third condition for prox-function (see Definition~\ref{def-prox}), so $d-\min_{\bm x\in Q}d(\bm x)$ must be used instead.
For simplicity, however, we will treat $d$ as a prox-function in the following.
This is because the additional constant term is not essential; it does not affect the strong convexity or $\nabla d^*$ and only shifts $d^*$ by a constant.

Note that the prox-function proposed in~\cite{farina2021better} is given by
\begin{align}
    d(\bm x)
    + \sum_{I\in\mathcal{I}} w_I x_{p(I)} \ln\abs{\A(I)},
    \label{eqn-farina}
\end{align}
which is shown to have a minimum value zero.
We have eliminated the first-order term in~\eqref{eqn-farina} by neglecting the adjustment of the minimum to zero. 
As a result, we succeeded in giving a better theoretical guarantee of substantial strong convexity.

\begin{theorem}
\label{thm-sigma}
The prox-function~\eqref{eqn-prox} is $1/M_Q$-strongly convex with respect to the $L_1$-norm.
\end{theorem}
\begin{proof}
This proof is the same as the proof of~\cite[Theorem 5]{farina2021better}.
For $\bm\xi\in\R^\abs{\Sigma}$ and $\bm x\in\ri Q$, we have
\begin{align}
    \bm\xi^\T\nabla^2 d(\bm x)\bm\xi
    &= \frac{\qty(\xi_\emp)^2}{x_\emp} +
    \sum_{I\in\mathcal{I}}\sum_{a\in\A(I)}\qty(
    w_I - \sum_{p(I^\prime)=(I,a)} w_{I^\prime}
    ) \frac{\qty(\xi_{I,a})^2}{x_{I,a}} \\
    &\ge \frac{\qty(\xi_\emp)^2}{x_\emp} +
    \sum_{I\in\mathcal{I}}\sum_{a\in\A(I)} \frac{\qty(\xi_{I,a})^2}{x_{I,a}} \\
    &= \sum_{j\in\Sigma} \frac{\qty(\xi_j)^2}{x_j} \\
    &\ge \frac{
    \qty(\sum_{j\in\Sigma}\abs{\xi_j})^2
    }{\sum_{j\in\Sigma} x_j} \\
    &= \frac{\norm{\bm\xi}_1^2}{\norm{\bm x}_1} 
    \ge \frac{\norm{\bm\xi}_1^2}{M_Q},
\end{align}
where the first equality follows from~\eqref{eqn-prox}, the second inequality comes from~\eqref{eqn-w}, and the fourth inequality is true from the Cauchy-Schwarz inequality:
\begin{align}
    \sum_{j\in\Sigma} \qty(\sqrt{x_j})^2
    \cdot
    \sum_{j\in\Sigma} \qty(\frac{\abs{\xi_j}}{\sqrt{x_j}})^2
    \ge
    \qty(\sum_{j\in\Sigma} \sqrt{x_j}\cdot\frac{\abs{\xi_j}}{\sqrt{x_j}})^2.
\end{align}
\end{proof}

To consider the properties of the conjugate function of the prox-function $d$, we present the following corollary.
\begin{lemma}
\label{lem-prox-another}
The prox-function~\eqref{eqn-prox} satisfies the following equation for $\bm x \in \ri Q$:
\begin{align}
    d(\bm x) 
    &= \sum_{I\in\mathcal{I}}
    w_I x_{p(I)}
    \sum_{a\in\A(I)} 
    \frac{x_{I,a}}{x_{p(I)}}
    \ln\frac{x_{I,a}}{x_{p(I)}}.
    \label{eqn-prox-another}
\end{align}
\end{lemma}
\begin{proof}
First, note that $x_j \ne 0$ for $j\in\Sigma$ for $\bm x\in\ri Q$.
Then for $\bm x\in\ri Q$, we have
%For $\bm x\in Q$, we have $x_\emp\ln x_\emp = 1\ln 1=0$ and $x_{p(I)}=\sum_{a\in\A(I)} x_{I,a}$.
%Note also that $x_j \ne 0$ for $j\in\Sigma$ for $\bm x\in\ri Q$, and we have
\begin{align}
d(\bm x)
&= \sum_{I\in\mathcal{I}}\sum_{a\in\A(I)} 
   \qty(w_I - \sum_{p(I^\prime)=(I,a)} w_{I^\prime})
   \ x_{I,a}\ln x_{I,a} \\
&= 
\qty(
\sum_{I\in\mathcal{I}}\sum_{a\in\A(I)} w_I x_{I,a}\ln x_{I,a}
) - 
\sum_{p(I^\prime)\ne\emp} w_{I^\prime} x_{p(I^\prime)}\ln x_{p(I^\prime)} 
\\
&= 
\sum_{I\in\mathcal{I}}\sum_{a\in\A(I)} w_I x_{I,a}\ln x_{I,a}
- 
\sum_{I\in\mathcal{I}} w_{I} x_{p(I)}\ln x_{p(I)} 
\\
&= \sum_{I\in\mathcal{I}} w_I \qty{
\qty( \sum_{a\in\A(I)} x_{I,a}\ln x_{I,a} )
 - x_{p(I)}\ln x_{p(I)}
} \\
&= \sum_{I\in\mathcal{I}} w_I \qty{
\sum_{a\in\A(I)} x_{I,a}\ln x_{I,a}
- \sum_{a\in\A(I)} x_{I,a}\ln x_{p(I)}
} \\
&= \sum_{I\in\mathcal{I}} w_I \sum_{a\in\A(I)}
x_{I,a}\ln\frac{x_{I,a}}{x_{p(I)}} \\
&= \sum_{I\in\mathcal{I}}
w_I x_{p(I)}
\sum_{a\in\A(I)} 
\frac{x_{I,a}}{x_{p(I)}}
\ln\frac{x_{I,a}}{x_{p(I)}},
\end{align}
where the first and third equalities follow from $x_\emp\ln x_\emp = 1\ln 1 = 0$ for $\bm x\in Q$, and the fifth equality comes from $x_{p(I)}=\sum_{a\in\A(I)} x_{I,a}$ for $\bm x\in Q$.
\end{proof}
From Lemma~\ref{lem-prox-another} and the fact that~\eqref{eqn-prox} is continuous in $Q$, we have
\begin{align}
    d^*(\bm\xi) 
    &= \max_{\bm x\in Q} \qty{
    \bm\xi^\T \bm x - d(\bm x)
    } \\
    &= \sup_{\bm x\in \ri Q} \qty{
    \bm\xi^\T \bm x - 
    \sum_{I\in\mathcal{I}}
    w_I x_{p(I)}
    \sum_{a\in\A(I)} 
    \frac{x_{I,a}}{x_{p(I)}}
    \ln\frac{x_{I,a}}{x_{p(I)}} 
    }.
    \label{eqn-sup-conj}
\end{align}
%Now, choose any $I\in\mathcal{I}$ such that there is no $I^\prime\in\mathcal{I}$ with $p(I^\prime)=(I,a)$ for all $a\in\A(I)$, then the terms on $\qty(x_{I,a})_{a\in\A(I)}$ of the supreme \eqref{eqn-sup-conj} are given by
Now, choose any $I\in\mathcal{I}$ satisfying $\qty{I^\prime\in\mathcal{I}\mid p(I^\prime)=(I,a)}=\emptyset$ for all $a\in\A(I)$, then the terms on $\qty(x_{I,a})_{a\in\A(I)}$ of the supreme~\eqref{eqn-sup-conj} are given by
\begin{align}
    x_{p(I)} 
    \sup_{\bm z\in\ri\Delta_{\abs{\A(I)}}} \qty{
    \sum_{a\in\A(I)} \xi_{I,a} z_a
    - w_I \sum_{a\in\A(I)} z_a\ln z_a
    },
\end{align}
where $z_a := x_{I,a}/x_{p(I)}$ and $\Delta_n$ is the $n$-dimentional simplex.
This maximization subproblem can be solved analytically (see Appendix~\ref{sec-app-subproblem}).
That is, the maximizer is given by
\begin{align}
z_a^* 
:=
\frac{\exp(\xi_{I,a}/w_I)}{\sum_{a^\prime\in\A(I)}\exp(\xi_{I,a^\prime}/w_I)},
\end{align}
which achieves the following supreme:
\begin{align}
    \mathrm{opt}_I
    &:= \sum_{a\in\A(I)}\xi_{I,a} z_a^* - w_I\sum_{a\in\A(I)} z_a^*\ln z_a^* \\
    &= w_I\ln\qty{
    \sum_{a\in\A(I)}\exp(\xi_{I,a}/w_I)
    }.
\end{align}
Substituting this result to~\eqref{eqn-sup-conj}, the terms on $\qty(x_{I,a})_{a\in\A(I)}$ disappear and $\mathrm{opt}_I$ is added to $\xi_{p(I)}$, which is the coefficient of $x_{p(I)}$ in~\eqref{eqn-sup-conj}.

By repeating the above operations in bottom-up order,~\eqref{eqn-sup-conj} can be solved, and the total calculation can be performed in $\order{\abs{\Sigma}}$.
We can also obtain $\nabla d^*(\bm\xi) = \argmax_{\bm x\in Q}\qty{\bm\xi^\T\bm x - d(\bm x)}$, however, only the ratio $\bm z$ is obtained in the above operations.
Then, after solving~\eqref{eqn-sup-conj}, we need to calculate $\nabla d^*(\bm\xi)$ by multiplying $\bm z$ in top-down order.
See Algorithm~\ref{alg-conj-grad} for details.

\begin{algorithm}[H]
\caption{Calculating $d^*(\bm\xi)$ and $\nabla d^*(\bm\xi)$}
\label{alg-conj-grad}
\begin{algorithmic}[1]
\Require $\bm\xi\in\R^\abs{\Sigma}$
\Ensure $y = d^*(\bm\xi),\ \bm z = \nabla d^*(\bm\xi)$
\State $\bm z \gets \bm 0 \in \R^\abs{\Sigma}$
\For{$I\in\mathcal{I}$ in bottom-up order}
    \For{$a\in\A(I)$}
        \State $\xi_{I,a} \gets \exp(\xi_{I,a} / w_I)$
    \EndFor
    \State $\xi_{p(I)} \gets \xi_{p(I)} + w_I\ln\sum_{a\in\A(I)} \xi_{I,a}$
    \For{$a\in\A(I)$}
        \State $z_{I,a} \gets \xi_{I,a} / \sum_{a\in\A(I)} \xi_{I,a}$
    \EndFor
\EndFor
\State $y \gets \xi_\emp$
\State $z_\emp \gets 1$
\For{$I\in\mathcal{I}$ in top-down order}
    \For{$a\in\A(I)$}
        \State $z_{I,a} \gets z_{p(I)} z_{I,a}$
    \EndFor
\EndFor
\end{algorithmic}
\end{algorithm}

\begin{theorem}
\label{thm-d}
The prox-function~\eqref{eqn-prox} satisfies
\begin{align}
    \max_{\bm x\in Q} d(\bm x) 
    - \min_{\bm x\in Q} d(\bm x)
    \le M_Q \ln\abs{\Sigma}.
\end{align}
\end{theorem}
\begin{proof}
Since $Q\subset[0,1]^\abs{\Sigma}$ and $x\ln x \le 0$ for $x\in[0,1]$, we have $\max_{\bm x\in Q} d(\bm x) \le 0$.
Then it is sufficient to show $-\min_{\bm x\in Q} d(\bm x) = d^*(\bm 0) \le M_Q\ln\abs{\Sigma}$.
From the procedure for computing $d^*(\bm\xi)$ presented above, we see that $d^*(\bm 0)$ satisfies the following recursive equation:
\begin{align}
    d^*(\bm 0) 
    &= \sum_{p(I)=\emp} \mathrm{opt}_I \\
    \mathrm{opt}_I
    &= w_I\ln \qty{\sum_{a\in\A(I)} \exp\qty(
    \frac{\sum_{p(I^\prime)=(I,a)} 
    \mathrm{opt}_{I^\prime}}{w_I}
    )
    } \quad \forall I\in\mathcal{I} \label{eqn-opt-i}
\end{align}
Now define $\gamma_I\in\R$ recursively:
\begin{align}
    \gamma_I := \abs{\A(I)} 
    + \sum_{a\in\A(I)}\sum_{p(I^\prime)=(I,a)}
    \gamma_{I^\prime} \quad \forall I\in\mathcal{I},
    \label{eqn-gamma}
\end{align}
then let us show
\begin{align}
    \mathrm{opt}_I \le w_I\ln\gamma_I
    \quad \forall I\in\mathcal{I}
    \label{eqn-opt-ineq}
\end{align}
recursively.
Assume that~\eqref{eqn-opt-ineq} holds for $I^\prime$ which is below $I$ in the sense of the rooted tree, discussed in Section~\ref{sec-efg}.
Then, we have
\begin{align}
    \sum_{p(I^\prime)=(I,a)}
    \mathrm{opt}_{I^\prime}
    &\le \sum_{p(I^\prime)=(I,a)}
    w_{I^\prime}\ln\gamma_{I^\prime} \\
    &\le \qty(
    1 + \sum_{p(I^\prime)=(I,a)} w_{I^\prime}
    )\ln\qty(
    1 + \sum_{p(I^\prime)=(I,a)}\gamma_{I^\prime}
    ) \\
    &\le w_I\ln\qty(
    1 + \sum_{p(I^\prime)=(I,a)}\gamma_{I^\prime}
    ),
\end{align}
where the first inequality follows from the assumption, and the third inequality comes from~\eqref{eqn-w}.
By substituting this to~\eqref{eqn-opt-i}, we have
\begin{align}
    \mathrm{opt}_I
    &\le w_I\ln\qty{
    \sum_{a\in\A(I)} \qty(
    1 + \sum_{p(I^\prime)=(I,a)}
    \gamma_{I^\prime}
    )
    } \\
    &= w_I\ln\gamma_I.
\end{align}
Therefore~\eqref{eqn-opt-ineq} is shown for all $I\in\mathcal{I}$.
By the definition~\eqref{eqn-gamma}, we have
\begin{align}
    1 + \sum_{p(I)=\emp}\gamma_I
    = 1 + \sum_{I\in\mathcal{I}} \abs{\A(I)} = \abs{\Sigma}.
\end{align}
By the definition of $w_I$, we also have
\begin{align}
    1 + \sum_{p(I)=\emp} w_I
    = \max_{\bm x\in Q} \norm{\bm x}_1 = M_Q.
\end{align}
Finally, we obtained the following inequality that we wanted to show.
\begin{align}
    d^*(\bm 0) 
    &= \sum_{p(I)=\emp}\mathrm{opt}(I) \\
    &\le \sum_{p(I)=\emp}w_I\ln\gamma_I \\
    &\le \qty(
    1 + \sum_{p(I)=\emp}w_I
    )
    \ln\qty(
    1 + \sum_{p(I)=\emp}\gamma_I
    ) \\
    &= M_Q\ln\abs{\Sigma}.
\end{align}
\end{proof}

We conclude this section by proving the following theorem.
\begin{theorem}
    The prox-function $d$ defined by~\eqref{eqn-prox} is $1/(M_Q^2\ln\abs{\Sigma})$-strongly convex substantially, with respect to the $L_1$-norm.
    In other words, assume that $d$ is $\sigma$-strongly convex with respect to the $L_1$-norm, and let $D := \max_{\bm x\in Q} d(\bm x) - \min_{\bm x\in Q} d(\bm x)$, then the following inequality holds:
    \begin{align}
        \frac{D}{\sigma} \le M_Q^2\ln\abs{\Sigma}.
    \end{align}
\end{theorem}
\begin{proof}
    It follows from Theorem~\ref{thm-sigma} and Theorem~\ref{thm-d}.
\end{proof}