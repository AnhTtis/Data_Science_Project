\section{Toy instances}
\label{sec-app-games}

In this section, we present some well-known extensive-form games.
These games are used in the numerical experiments in Section~\ref{sec-experiments}.

\subsection{Kuhn poker}
Kuhn poker is a simple variant of Texas Hold'em proposed in~\cite{kuhn1950simplified}, played with three cards, J, Q, and K.
Each player is dealt a card privately, and the game begins with each player betting 1 chip.
Player 1 is the first to act.
Player 1 can either \textit{check} (do nothing and pass the turn to Player 2) or \textit{raise} (bet an additional 1 chip).
%\begin{enumerate}[label=(\alph*)]
\begin{enumerate}[(a)]
    \item If player 1 raises, player 2 chooses to \textit{call} (bet an additional 1 chip) or \textit{fold} (surrender and lose the 1 chip already bet);
    if player 2 calls, the player with the higher-ranked card wins all chips (showdown).
    \label{itm-kuhn}
    \item If player 1 checks, player 2 chooses to check (and showdown) or raise;
    if player 2 raises, player 1 chooses to call or fold as in \ref{itm-kuhn}.
\end{enumerate}
The game tree of Kuhn poker is shown in Figure~\ref{fig-kuhn}, and its information tree of each player is shown in Figures~\ref{fig-kuhnq1} and~\ref{fig-kuhnq2}.

\begin{figure}[tbp]
    \centering
    \includegraphics[width=120mm]{images/kuhn.drawio.png}
    \caption{
    Kuhn poker game tree.
    The nodes marked with ``C'' on a black background are played by the chance player, and the square and circle nodes are played by player 1 and player 2.
    The terminal nodes are represented by a black square with corresponding $u(z)$.
    The arrows correspond to actions, and the state moves down to the pointed node.
    Chance player action probabilities are given and are written on the arrows.
    The nodes played by player 1 and player 2 are partitioned according to the information partition $\mathcal{I}_1=\qty{I_1, \dots, I_6}$ and $\mathcal{I}_2=\qty{J_1, \dots, J_6}$. 
    Each player does not know which card is dealt to their opponent, so they cannot distinguish between nodes belonging to the same information partition.
    }
    \label{fig-kuhn}
\end{figure}

\begin{figure}[htbp]
    \centering
    \includegraphics[width=100mm]{images/kuhnpokerq1.drawio.png}
    \caption{The \textit{information tree} of player 1 in Kuhn poker}
    \label{fig-kuhnq1}
\end{figure}

\begin{figure}[htbp]
    \centering
    \includegraphics[width=120mm]{images/kuhnpokerq2.drawio.png}
    \caption{The \textit{information tree} of player 2 in Kuhn poker}
    \label{fig-kuhnq2}
\end{figure}



\subsection{Leduc Hold'em}
Leduc Hold'em is a simple variant of Texas Hold'em proposed in~\cite{southey2012bayes}, played with pairs of cards J, Q, and K, totaling six cards.
The general rules are the same as in Kuhn poker, but some differences exist.
\begin{itemize}
    \item Each player can raise against the opponent's raise, which is called \textit{re-raise}, but cannot raise against a re-raise.
    \item If both players check or one player calls, instead of an immediate showdown, a \textit{community card} is revealed randomly from the remaining cards, and the game is resumed only once more.
    \item The winner of the showdown is the player with the same card as the community card.
    If there is no such player, the player with the higher-ranked card wins.
    If both players have the same cards, the game is tied (chips are divided equally).
    \item The betting amount of raise is 2 and 4 chips in each phase.
    This means that the maximum move of chips is 1+2+2+4+4=13.
\end{itemize}
The game tree of Leduc hold'em is shown in Figure \ref{fig-leduc}, and the information tree of player 1 is shown in Figure \ref{fig-leducq1}.
Experiments are also performed for Leduc Hold'em (13 ranks), a game in which the number of card ranks is changed from 3 to 13.


\begin{figure}[htbp]
    \centering
    \includegraphics[width=125mm]{images/leducholdem.drawio.png}
    \caption{Leduc Hold'em game tree}
    \label{fig-leduc}
\end{figure}

\begin{figure}[htbp]
    \centering
    \includegraphics[width=120mm]{images/leducholdemq1.drawio.png}
    \caption{The \textit{information tree} of player 1 in Leduc Hold'em}
    \label{fig-leducq1}
\end{figure}