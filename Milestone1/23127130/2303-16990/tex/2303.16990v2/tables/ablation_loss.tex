\begin{table}[t]
%\begin{wraptable}{r}{6.0cm}
\vspace{-0.3cm}
    \centering \scriptsize
    \setlength{\tabcolsep}{4pt}
% \vspace{-.5cm}

    % \begin{subfigure}[t]{\linewidth}
    %     \newcommand{\apbbox}[1]{AP$^\text{bbox}_\text{#1}$}
    %     \newcommand{\apmask}[1]{AP$^\text{mask}_\text{#1}$}
    %     \newcolumntype{Y}{>{\raggedright\arraybackslash}X}
    %     \newcolumntype{Z}{>{\centering\arraybackslash}X}
    
    %     \centering
    %     \footnotesize
    %     \setlength\tabcolsep{1pt}
    %     \renewcommand{\arraystretch}{1.1}
    % \begin{tabularx}{\linewidth}{l@{\hskip 0.3in}c@{\hskip 0.3in}c@{\hskip 0.3in}c@{\hskip 0.3in}c}
    
    %     \toprule
    %     \textbf{Frame Sampling} & None & Global sim & Local sim & Sinkhorn\\
    %     \midrule
    %     \textbf{GYT (Spatio-temporal)}& 15.1&  15.7 & 15.6  & 17.1  \\
    %     \textbf{MYT (Temporal)} &  17.8   & 18.5  &  18.1 & 19.9 \\
    %     \textbf{YC-Inter (Spatial)} & 55.5 &  55.4 & 56.3  &  57.1  \\
    %     \bottomrule
    %     \end{tabularx}
        
    %     %\vspace{-1pt}
    %     \caption{Effect of different frame selection strategy.}
    %     \label{subtab:sampling}
    %     \vspace{5pt}
    % \end{subfigure}
    %\begin{subfigure}[t]{\linewidth}
        \begin{tabularx}{\linewidth}{l@{\hskip 0.3in}c@{\hskip 0.3in}c@{\hskip 0.3in}c}
        \toprule
          &  GroundingYT   &  MiningYT & YC-Inter.   \\
                                        &  Spatio-temp  &  Temporal & Spatial  \\
        \midrule
        only Local loss &   7.29  & 5.23 & 55.29   \\
        only Global loss &  9.28   & 19.12  & 36.23 \\
        w/ Both loss & 19.45  & 20.33  &  58.35   \\
        \bottomrule
        \end{tabularx}
        \vspace{-5pt}
        %\caption{Effect of different frame selection strategy.}
        \label{subtab:sampling}
        %\vspace{5pt}
    %\end{subfigure}
    
    % \begin{subfigure}[t]{\linewidth}
    %     \begin{tabularx}{\linewidth}{l@{\hskip 0.3in}c@{\hskip 0.3in}c@{\hskip 0.3in}c@{\hskip 0.3in}c@{\hskip 0.3in}c}
    %     \toprule
    %     \textbf{Frame length} & 1 & 4 & 8 & 16 & 24 \\
    %     \midrule
    %     \textbf{GYT (Spatio-temporal)} & 5.2  & 9.5  & 16.1  &  17.1  & 16.5  \\
    %     \textbf{MYT (Temporal)} &  8.5   & 13.1  &  18.4 & 19.9  & 19.0 \\
    %     \textbf{YC-Inter (Spatial)} & 31.1  & 48.2 & 55.5  &   57.1  & 56.1  \\
    %     \bottomrule
    %     \end{tabularx}
    %     %\vspace{-2pt}
    %     \caption{Effect of \# video frames used for training}
    %     \label{subtab:ablations2}
    %     \vspace{5pt}
    % \end{subfigure}

    % \begin{subfigure}[t]{\linewidth}
    %     \begin{tabularx}{\linewidth}{l@{\hskip 0.6in}c@{\hskip 0.3in}c@{\hskip 0.3in}c}
    %     \toprule
    %     \textbf{Frame} &  GroundingYT   &  MiningYT & YouCook-Inter.   \\
    %     \textbf{Length}    &  Spatio-temporal  &  Temporal & Spatial  \\
    %     \midrule
    %     1 &   5.2 & 8.5 & 31.1   \\
    %     8 &   16.1 & 18.4 & 55.5   \\
    %     16 &  17.1  & 19.9  &  57.1 \\
    %     24 & 16.5  & 19.0  &  56.1   \\
    %     \bottomrule
    %     \end{tabularx}
    %     %\vspace{-2pt}
    %     \caption{Effect of \# video frames used for training}
    %     \label{subtab:ablations2}
    %     \vspace{5pt}
    % \end{subfigure}

    % \begin{subfigure}[t]{\linewidth}
    %     \begin{tabularx}{\linewidth}{l@{\hskip 0.3in}c@{\hskip 0.3in}c@{\hskip 0.3in}c}
    %     \toprule
    %     \multirow{2}{*}{\textbf{Loss}} &  GroundingYT   &  MiningYT & YouCook-Inter.   \\
    %                                    &  Spatio-temporal  &  Temporal & Spatial  \\
    %     \midrule
    %     only Local loss &   5.7  & 4.5 & 54.3   \\
    %     only Global loss &  7.6   & 18.8  & 32.5 \\
    %     w/ Both loss & 17.1  & 19.9  &  57.1   \\
    %     \bottomrule
    %     \end{tabularx}
    %     %\vspace{-2pt}
    %     \caption{Effect of global and local loss functions}
    %     \label{subtab:ablations3}
    %     \vspace{5pt}
    % \end{subfigure}

    % \begin{subfigure}[t]{\linewidth}
    %     \begin{tabularx}{\linewidth}{l@{\hskip 0.35in}c@{\hskip 0.35in}c@{\hskip 0.35in}c}
    %     \toprule
    %     \textbf{Attention} &  GroundingYT   &  MiningYT & YouCook-Inter.   \\
    %     \textbf{Architecture}    &  Spatio-temporal  &  Temporal & Spatial  \\
    %     \midrule
    %     Self+Cross &  15.4  &  18.7  &  54.1    \\
    %     Cross+Self &  15.9  &  18.9  &    54.5  \\
    %     Cross+Cross &  16.5  &  19.3  &    56.2  \\
    %     Cross+Self+Cross & 17.1  & 19.9  &  57.1   \\
    %     \bottomrule
    %     \end{tabularx}
    %     %\vspace{-2pt}
    %     \caption{Ablation on different attention architecture}
    %     \label{subtab:architecture}
    %     \vspace{5pt}
    % \end{subfigure}

    % \begin{subfigure}[t]{\linewidth}
    %     \begin{tabularx}{\linewidth}{l@{\hskip 0.6in}c@{\hskip 0.3in}c@{\hskip 0.3in}c}
    %     \toprule
    %     \textbf{Dataset} & HT200k & HT370k &  HT100M  \\
    %     \midrule
    %     \textbf{YouCook-Inter} & 57.1  & 56.8  &  57.0   \\
    %     \textbf{GroundingYoutube} & 17.1   & 17.3 &  17.4  \\
    %     \bottomrule
    %     \end{tabularx}
    %     %\vspace{-2pt}
    %     \caption{Effect of different training dataset }
    %     \label{subtab:ablations4}
    %     \vspace{5pt}
    % \end{subfigure}


    % \begin{subfigure}[t]{\linewidth}
    %     \begin{tabularx}{\linewidth}{l@{\hskip 0.6in}c@{\hskip 0.3in}c@{\hskip 0.3in}c}
    %     \toprule
    %     \textbf{Train/test supervision} &VT/VT & VAT/VT & VAT/VAT \\
    %     \midrule
    %     \textbf{YouCook-Inter} & 53.9 & 53.6  & 53.8   \\
    %     \textbf{GroundingYoutube} & 16.2   & 16.8 & 17.0  \\
    %     \bottomrule
    %     \end{tabularx}
    %     %\vspace{-2pt}
    %     \caption{Effect of audio supervision in training and testing}
    %     \label{subtab:ablations5}
    %     %\vspace{5pt}
    % \end{subfigure}
    %\vspace{-10pt}
    
    % \begin{subfigure}[t]{\linewidth}
    %     \begin{tabularx}{\linewidth}{lXccc}
    %     \toprule
    %     \textbf{Noisy supervision} &&No Tracking& BBox Track & Seg Track \\
    %     \midrule
    %     \textbf{YouCook} && 68.3\Drop{4.7} & 71.9 \Drop{1.1} &  73.0  \\
    %     \bottomrule
    %     \end{tabularx}
    %     %\vspace{-2pt}
    %     \caption{Effect of noisy tracking supervision}
    %     \label{subtab:ablations4}
    %     \vspace{-5pt}
    % \end{subfigure}
    
    \caption{
        \textbf{Loss ablations:}
        %We isolate the effects of our training components. 
        %We find that
        % \textbf{(a)} starting with a shorter temporal distance between query-key clips and relaxing the constraint as training progresses improves performance. 
        both losses contribute to the final loss, and the existence of global loss helps the localization task.
        % \textbf{(d)} training with more data improves slightly or no improve.
        % \textbf{(e)} training with audio help us learn temporal information. 
        %\vspace{-0.5cm}
        }
        \vspace{-0.45cm}
    \label{tab:train_ablations}
 %\vspace{-0.5cm}
\end{table}
%\end{wraptable} 



% \begin{table}[t]
%     \scriptsize
%         \centering%\vspace*{1mm}
%        \begin{tabular}{lp{1em}l}
%     \begin{subfigure}[t]{0.355\linewidth}
% 		 \begin{tabularx}{\linewidth}{lp{0em}c}
% 			\toprule
% 				\multirow{2}{*}{\textbf{Localization Loss}} & & mAP  \\
% 					  & & IOU@0.35 \\
% 				\hline
% 				w/o $L_{va}$, $L_{at}$ loss                          &  &   16.11 \\
% 				w/o $L_{at}$ loss                      &  &   19.02 \\
% 				Ours                         &  &   20.39 \\ 
% 			%\midrule
% 			\bottomrule
% 		\end{tabularx}
% 		\caption{Ablation on spatial contrastive loss \label{tab:loss}}
% 		 \end{subfigure} &&

%     \begin{subfigure}[t]{0.355\linewidth}
%     \begin{tabularx}{\linewidth}{lp{1em}c}
% 			\toprule
% 				\multirow{2}{*}{\textbf{Audio Input}} & & mAP  \\
% 					  & & IOU@0.35 \\
% 				\hline
% 				No Audio                    &        &   18.92 \\
% 				Random Audio                    &        &   18.88 \\
% 				Ours                      &     &   20.39 \\ 
% 			%\midrule
% 			\bottomrule
% 		\end{tabularx}
% 		\caption{Ablation on audio \label{tab:no_audio}}
%     \end{subfigure} 
% 		%\vspace{-0.355cm}
% 		\end{tabular}
% 		\begin{subfigure}[t]{0.6\linewidth}
%     \begin{tabularx}{\linewidth}{lp{2em}c}
% 			\toprule
% 				\multirow{2}{*}{\textbf{Attention Architecture}} & & mAP  \\
% 					  & & IOU@0.35 \\
% 				\hline
% 				Self + Cross                        &    &  18.55 \\
% 				Cross + Self                        &    &   18.92 \\
% 				Cross + Cross                       &     &  19.59 \\
% 				Cross + Self + Cross                   &     &   20.39 \\ 
% 			%\midrule
% 			\bottomrule
% 		\end{tabularx}
% 		%\vspace{+0.3cm}
% 		\caption{Ablation on different attention architecture \label{tab:attention}}
%     \end{subfigure} 
% 		%\vspace{-0.35cm}
% 		 \caption{
%         \textbf{Ablations study.} %We find that
%         % \textbf{(a)} starting with a shorter temporal distance between query-key clips and relaxing the constraint as training progresses improves performance. 
%         \textbf{(a)} Learning multimodal interaction across spatial representations from the three modalities leads to the best performance.
%         \textbf{(b)} The model captured useful information from the audio but will not break when the audio is not presented.
%         \textbf{(c)} Cross-attention layer is crucial for learning multimodal interaction, and self-attention boosts the performance slightly.
%         % \textbf{(d)} \bccomment{replacing unsupervised video tracking supervision with a noisy bounding box tracking tube  achieved a significant gain over the baseline.}
%         % \textbf{(e)} adding some amount of spatial constraints based on IoU with tracking mask ensures that different clips contain common salient regions and this improves performance.
%     }
%     \vspace{-0.9cm}
% \end{table}
