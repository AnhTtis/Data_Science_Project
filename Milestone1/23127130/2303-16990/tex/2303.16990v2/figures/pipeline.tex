\begin{figure*}[t]
    \centering
    \includegraphics[width=\textwidth]{figures/Overview_STG_new3.pdf}
    %\includegraphics[width=\textwidth]{figures/pipeline2.pdf}
    %\vspace{-20pt}
    \caption{\textbf{Spatio-temporal grounding approach.} 
    % We incorporate both spatial and temporal information in the training process including three modalities. 
    (a)~We aim to select frames with groundable objects and actions. To this end, projected text features are matched with respective frame features. (b)~Sinkhorn optimal transport is then leveraged to optimize the selected frames wrt. the text input. (c)~Based on the selected frames, a global representation is learned to allow for temporal localization, as well as (d)~a local representation to ground the action description in the spatial region. 
    %Local contrastive loss on video spatio-temporal and text features to learn multimodal interactions between finer-grained features. Global pairwise contrastive loss on video and text features to pull the features close across modalities in a high-level semantic space. 
    }
    \label{fig:pipeline}
    \vspace{-15pt}
\end{figure*}