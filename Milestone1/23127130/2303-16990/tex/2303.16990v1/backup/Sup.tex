% CVPR 2023 Paper Template
% based on the CVPR template provided by Ming-Ming Cheng (https://github.com/MCG-NKU/CVPR_Template)
% modified and extended by Stefan Roth (stefan.roth@NOSPAMtu-darmstadt.de)

\documentclass[10pt,twocolumn,letterpaper]{article}
\makeatletter
\@namedef{ver@everyshi.sty}{}
\makeatother
%%%%%%%%% PAPER TYPE  - PLEASE UPDATE FOR FINAL VERSION
%\usepackage[review]{cvpr}      % To produce the REVIEW version
%\usepackage{cvpr}              % To produce the CAMERA-READY version
%\usepackage[pagenumbers]{cvpr} % To force page numbers, e.g. for an arXiv version
\usepackage{iccv}
\usepackage{times}
% Include other packages here, before hyperref.

\usepackage{tikz}
\usepackage{comment}
\usepackage{amsmath,amssymb} % define this before the line numbering.
\usepackage{color}
\usepackage{tabularx}
\usepackage{multirow}

\usepackage{subcaption}
%\usepackage{subfigure}
\usepackage{booktabs}

\newcommand{\nina}[1]{{\color{cyan}\textbf{Nina:} #1}}
\newcommand{\asli}[1]{{\color{green}\textbf{Asli:} #1}}
\newcommand{\hkc}[1]{\textcolor{blue}{(Hilde: #1)}} %Hilde

%\newcommand{\etal}{\textit{et al}. }
%\newcommand{\ie}{\textit{i}.\textit{e}. }
%\newcommand{\eg}{\textit{e}.\textit{g}. }

\newcommand{\tablestyle}[2]{\setlength{\tabcolsep}{#1}\renewcommand{\arraystretch}{#2}\centering\footnotesize}
\newcommand{\vect}[1]{\mathbf{#1}}
\newcommand{\va}[0]{\vect{a}}
\newcommand{\bc}[1]{\textcolor{blue}{\bf \small [ #1 --Brian]}} 


\newcommand{\vz}[0]{\vect{z}}
% The "axessiblity" package can be found at: https://ctan.org/pkg/axessibility?lang=en
\usepackage[accsupp]{axessibility}  % Improves PDF readability for those with disabilities.

% It is strongly recommended to use hyperref, especially for the review version.
% hyperref with option pagebackref eases the reviewers' job.
% Please disable hyperref *only* if you encounter grave issues, e.g. with the
% file validation for the camera-ready version.
%
% If you comment hyperref and then uncomment it, you should delete
% ReviewTempalte.aux before re-running LaTeX.
% (Or just hit 'q' on the first LaTeX run, let it finish, and you
%  should be clear).
\usepackage[pagebackref,breaklinks,colorlinks]{hyperref}

% Support for easy cross-referencing
\usepackage[capitalize]{cleveref}
\crefname{section}{Sec.}{Secs.}
\Crefname{section}{Section}{Sections}
\Crefname{table}{Table}{Tables}
\crefname{table}{Tab.}{Tabs.}
\def\red{\textcolor{red}}
\def\blue{\textcolor{blue}}


\numberwithin{equation}{section}
\setcounter{table}{4}
\setcounter{figure}{4}
%\setcounter{equation}{0}
\setcounter{footnote}{0}

\usepackage{flushend}
% \renewcommand{\thefigure}{\Alph{figure}}
% \renewcommand{\thetable}{\Alph{table}}
% \renewcommand{\theequation}{\Alph{equation}}
\renewcommand{\thesection}{\Alph{section}}
%%%%%%%%% PAPER ID  - PLEASE UPDATE
% \def\cvprPaperID{4376} % *** Enter the CVPR Paper ID here
% \def\confName{CVPR}
% \def\confYear{2023}

\def\iccvPaperID{2173} % *** Enter the ICCV Paper ID here

\begin{document}

%%%%%%%%% TITLE - PLEASE UPDATE
\title{Supplementary for ``What, when, and where? - Self-Supervised Spatio-Temporal Grounding in Untrimmed Multi-Action Videos from Narrated Instructions''}
%\title{Towards Multimodal Self-supervised Spatio-Temporal  Grounding in \\Instructional Videos}

\author{%
    Brian Chen $^1$  \quad
    Nina Shvetsova$^2$  \quad
    Andrew Rouditchenko$^3$  \quad 
    Daniel Kondermann $^4$  \\
    Samuel Thomas$^{5,6}$  \quad 
    Shih-Fu Chang$^{1}$ \quad 
    Rogerio Feris$^{5,6}$  \quad 
    James Glass$^3$ \quad 
    Hilde Kuehne$^{2,6}$ \quad
    \vspace{1mm} \\
    \small{$^1$Columbia University, $^2$Goethe University Frankfurt, $^3$MIT CSAIL, $^4$Quality Match GmbH},  \\
    \small{$^5$IBM Research AI, $^6$MIT-IBM Watson AI Lab} \\
    \small{
    \texttt{\{bc2754,sc250\}@columbia.edu},\texttt{\{shvetsov,kuehne\}@uni-frankfurt.de},\texttt{\{roudi,glass\}@mit.edu},}   \\ \small{\texttt{dk@quality-match.com}, \texttt{\{sthomas,rsferis\}@us.ibm.com} }
}
%******************
\maketitle
%\noindent\textbf{Supplementary material}

\noindent \textbf{This appendix is organized as follows:} \\
\noindent A. Details on the experimental setup. 
    %a. Implementation details
    
\noindent B. Additional experiments. 
    %Contrastive loss ablation
    %Wide spread v.s. saturated
    %    How much acc goes down in saturated
    
   % Will send Asli frame/GT bbox/mask
   
%\noindent D. Visualization of annotation using videos. 
    %Interpolation the annotation points
    %2 videos

\noindent C. Details on the annotation process.

%\noindent D. Qualitative results. \blue{delete that?}



% Others
% Include code (Done)
% Annotation files (Done)

%Sunday 11am EST (free to join)

\section{Experimental setup}
%\subsection{Implementation details}
\subsection{Baseline details}
\label{sup:baseline}
MIL-NCE~\citep{miech2020end}, which utilizes S3D~\citep{xie2018rethinking} and word2vec~\citep{mikolov2013efficient} to project two modalities into a common space, is chosen as the standard baseline for this task; 
CoMMA~\citep{tan2021look}, the best-performing model for spatial representations in self-supervised learning (we denote CoMMA$\dagger$ to represent the model that uses weights shared by the author\footnote{We thank the authors for providing code and weights.}); CLIP~\citep{radford2021learning}, an image-text model trained with transformer architecture, is further applied as the backbone and trained with~\citep{tan2021look} to construct CoMMA$\ddagger$; GLIP~\citep{li2022grounded} and RegionCLIP~\citep{zhong2022regionclip}, state-of-the-art image-text grounding models that combine large-scale image caption pretraining and object detection fine-tuning, which we consider weakly supervised as the bounding box proposal network was trained on other human-annotated data.
We further construct a strong baseline out of the best methods for temporal and spatial localization, MIL-NCE+RegionCLIP, where we use MIL-NCE for temporal localization and RegionCLIP for spatial grounding following the inference pipeline of Figure \ref{fig:inference} without additional training. 

\subsection{Backbones and Training}
\label{backbone_and_training}
We evaluate the proposed method on backbones, CLIP \citep{radford2021learning} and S3D-word2vec \citep{miech2020end}.
We described the detailed setup as well as the training in the following.

\noindent \textbf{CLIP models.} For both the visual and text backbone, we use the pretrained weights from CLIP \citep{radford2021learning} with transformer ViT-B/32 and fix the encoder. Both the visual and text encoder has a final embedding size of 512. We apply them to video segments with 12-28 seconds, processing 1 frame per second. An evaluation of how many frames to process (identical to the number of seconds) is shown in Table \ref{tab:frames}. We sampled the video with 5 fps. It shows the best results when we start with 80 possible frames $U$ (as described in Section \ref{frame_sampling}), from which $T$ = 16 frames are selected for training. Ablation of the number of frames $T$ used for training is shown in Table \ref{subtab:ablations2}.
We used a batch size of $B$ = 64 video clips.

\noindent \textbf{S3D-word2vec models.}
For the video backbone, we follow~\citep{tan2021look} and use S3D initialized by MIL-NCE on HowTo100M~\citep{miech2020end} at the rate of 5 frames per second and fix the video encoder. 
The global video clip features were max-pooled over time and projected into embeddings of dimension 512. We used the mean-pooled S3D spatio-temporal features to represent the global representation of the video following the S3D architecture \citep{xie2018rethinking}.
For the text feature, we follow ~\citep{miech2019howto100m} using a GoogleNews pre-trained word2vec model~\citep{mikolov2013efficient} and max-pooling over words in a given sentence to acquire the text global feature.  
We follow \citep{miech2020end} to use the max-pooled word embedding to represent the sentence (global representation) since there is no [CLS] token. Also, the sentence feature is used for the query word selection instead of the [CLS] token. 
We use a batch size of $B$ = 96 video clips.


\noindent \textbf{Training.} For the training of both backbone settings, we use an Adam optimizer~\citep{kingma2015adam} with a learning rate of $1\mathrm{e}{-4}$. 
In the setting of fintining CLIP, we set a learning rate of $1\mathrm{e}{-7}$ for the CLIP backbone. 
The model is trained for 10 epochs on 4 V100 GPUs, which takes about two days. 


\subsection{Inference}
\label{inference_sup}
\noindent \textbf{Inference for the proposed model and CoMMA.} For inference in the case of temporal grounding, as shown in Figure \ref{fig:inference}(a), we first normalize the global feature for video and text. We used a (temporal) threshold $\theta$ = 0.5 to separate detections from the background. In spatial grounding, we acquire an attention heatmap using the attention rollout \citep{abnar2020quantifying} described in Section \ref{inference_section}. We set a spatial threshold $\tau$ = 0.01 to create the mask, as shown in Figure \ref{fig:inference}(b). The choice of this spatial threshold is evaluated in Table \ref{tab:thre}. 



\noindent \textbf{GLIP, RegionCLIP baseline inference.} In spatial grounding, we are given a text query and need to localize it in the frame. GLIP and RegionCLIP predict multiple bounding boxes corresponding to the text query. We select the predicted bounding box with the highest confidence score as the prediction result. We use the center point of the predicted bounding box for the pointing game evaluation as the model prediction. For \textit{mAP} evaluation, we use the predicted bounding box to compute IoU with the ground truth bounding box. In spatio-temporal grounding, we input all possible action description labels as candidates similar to Figure \ref{fig:inference}(a). We pick the class with the highest confidence score as the predicted label. If the model made no prediction, we would predict it as ``background''. The spatial inference is the same as the spatial grounding setting.

\noindent \textbf{TubeDETR, STCAT baseline inference.} TubeDETR and STCAT are spatio-temporal grounding models trained to predict a single spatio-temporal tube per video. 
In both cases, TubeDETR and STCAT, we use models trained on the Vid-STG dataset with 448x448 resolution and evaluate them for the task of spatial grounding. 
Since this dataset contains mostly short videos ($<$30sec), we observed that both methods will also only predict a trajectory tube in this temporal range ($<$30sec), no matter how long the input video is. To allow us to apply them to longer videos ($>$30sec), we split the longer videos based on sliding windows of 5-sec for better performance.




\noindent \textbf{MIL-NCE, CLIP baseline inference.} Both models are trained based on global representations for both input modalities, videos/images and text. We can, therefore, directly compute a sentence-to-video-frame similarity to perform the temporal grounding for Figure \ref{fig:inference}(a), following the same process as the proposed method for temporal grounding. For spatial grounding, we compute sentence-to-region feature similarity. Both visual backbones produce a 7x7 grid feature. We normalize the sentence and region features, then select a spatial threshold $\tau$ = 0.5 to create the mask for the \textit{mAP} evaluation.

\subsection{Evaluation metrics}
\label{eval_metric}


\noindent (i) \textbf{Spatio-temporal grounding in untrimmed video} is evaluated on our annotated GroundingYoutube dataset. We combined the spatial and temporal grounding evaluation as before \citep{kuehne2019mining,akbari2019multi} to form the spatio-temporal evaluation. The entire video and the respective pool of action instructions were provided. The model needs to localize each action step in temporal (start-time/end-time) and spatial (location in the video) as described in Figure \ref{fig:inference}. 
% \noindent\textbf{Inferencing.} The model will need to predict the action label per frame by feature similarity between the video and action classes similar to \citep{Zhukov2019CrossTask}. Later, the model will use its predicted action label as the query to perform spatial grounding to localize the action in the video frame.
We evaluate in two metrics: \textbf{IoU+Pointing game} combines the evaluation setting from the spatial grounding \citep{akbari2019multi} and temporal grounding \citep{kuehne2019mining} metrics. For each video frame, the prediction is correct when the model predicts the correct action for the frame. Also, given the predicted action as a query, the maximum point of the heatmap aims to lie within the desired bounding box. We then compute the Intersection over Union (IoU) over all the predictions with the GT to acquire the final score. 
We also compute \textbf{video mAP} following previous evaluation \citep{gu2018ava}, where we set IoU threshold between GT and predicted spatio-temporal tubes. A prediction is correct when it surpasses the IoU threshold. We then compute the mAP over all classes. We form a 3D prediction mask following Figure \ref{fig:inference} and compute IoU between our 3D heatmap and 3D tube.

\noindent (ii) \textbf{Spatial grounding} is given a text query description to localize the corresponding region in the trimmed video. We use GroundingYoutube, Youcook-Interaction, V-HICO, and Daly for evaluation. %Note that the evaluation is spatial only. It evaluates the results for each frame separately without considering the temporal information. 
This task is evaluated using the \textbf{pointing game accuracy}. Given the query text and video, we compute the attention heatmap on the video as described in Figure \ref{fig:inference}(b). If the highest attention similarity score lies in the ground truth bounding box, the result counts as a ``hit" and counts as ``miss" otherwise. The final accuracy is calculated as a ratio between hits to the total number of predictions $\frac{\text{\# hits}}{\text{\# hits} + \text{\# misses}}$. 
We report the mean average precision \textbf{(mAP)} following the settings from V-HICO \citep{li2021weakly}. Given a human-object category as the text query, we aim to localize the spatial location in the video frame.
The predicted location is correct if their Intersection over-Union (IoU) with ground truth bounding boxes is larger than 0.3. 
Since we do not use any bounding box proposal tools or supervision, we create an attention heatmap as described in Figure \ref{fig:inference}(b) to create a mask for IoU computation. 
We follow \citep{li2021weakly} and compute the mAP over all verb-object classes.


\noindent (iii) \textbf{Temporal grounding} \label{temporal_grounding}
provides videos with the respective actions and their ordering, including the background. The goal is to find the correct frame-wise segmentation of the video. We follow the inference procedure in \citep{kuehne2019mining} to compute the alignment given our similarity input matrix. The task is evaluated by intersection over detection (IoD), defined as $\frac{G \cap D}{D}$ the ratio between the intersection of ground-truth action $G$ and prediction $D$ to prediction $D$, and the Jaccard index, which is an (IoU) given as $\frac{G \cap D}{G \cup D}$.



\section{Additional Experiments}

\noindent \textbf{Single-action spatio-temporal grounding.}
Current spatio-temporal detection and grounding datasets \cite{jiang2014thumos,gu2018ava} usually aim to discriminate a single given action class from the background class in a short clip. This differs from our setup of spatio-temporal grounding in untrimmed videos, which usually comprises a set of phrases that need to be detected in a 3-5 min long video. To allow an evaluation of spatio-temporal grounding approaches based on single phrase grounding, we construct a clip-level evaluation where the clip varies from 9 sec to 60 sec. Given an action step, we append the video segments before and after the steps with the same time length of the action step to form the final video clip. This results in 2,895 clips for the spatio-temporal clip grounding evaluation.
For each clip, the temporal action intervals occupy 33\% of corresponding videos, which demonstrates the difficulty of the setting. 
%\bc{talk about GLIP performance increase, how its temporal prediction}
In this setting, instead of selecting the possible action step from a pool, the ground truth action step was given as the text query for spatio-temporal grounding. This allows us to directly compare with supervised spatio-temporal grounding methods \cite{yang2022tubedetr,jin2022embracing} as described in Section \red{5.4}.
As shown in Table \ref{tab:st_clip}, we observe that the baseline GLIP models achieve a much better performance compared to Table \red{1}. This is due to the fact that this setting does not require the model to select the text query from the pool, which the GLIP model was not trained to do. Moreover, we find that weakly supervised methods, GLIP and RegionCLIP, show only limited ability to differentiate the queried action from the background, which leads the model to ground the text query in most of the frames. However, both demonstrate powerful localization ability in foreground action segments, which results in a decent performance. The fully-supervised trained models (TubeDETR, STCAT) achieved a balance in localizing temporally and spatially, resulting in the best performance on this task.





\noindent \textbf{Frames used for selection.}
As shown in Table \ref{tab:frames}, we perform an ablation study on the number of candidates frames $U$ used for training. We found that selecting 16 frames achieves the best performance, comprising the useful video information in training while not including too many irrelevant concepts that diverge from the action/object in the ASR sentence.



\noindent\textbf{Number of frames for training.} We further evaluated the impact of different numbers of frames $T$ used for training. As shown in Table \ref{subtab:ablations2}, selecting fewer frames for training significantly causes the performance to drop. We hypothesize that the model not only fail to capture the temporal dynamics with fewer frames but also loses some frames with groundable objects in the sentence while training. We also hypothesize that with a too large number of frames, more irrelevant frames might be selected during training, which decreases the performance.


\noindent\textbf{Effect of audio in training and testing.} Unlike text which describes a discrete concept as a target to ground, audio serves as a continuous representation that is highly relevant to the temporal information. For example, we can determine an action started when we hear a ``cracking'' sound. In Table \ref{subtab:ablations5}, we tested our model using the additional audio modality. For the audio branch, we compute log-mel spectrograms and use a DAVEnet model~\cite{harwath2018jointly} initialized by MCN on HowTo100M~\cite{chen2021multimodal} to extract audio features. We extend the global and local loss pairs from VT to VT, VA, and AT following \cite{ShvetsovaCVPR22Everything}.
We found when training and testing with audio, the spatio-temporal result increases the temporal performance while the spatial-only result remains the same. This validates our assumption that audio contributes more to temporal understanding. When we trained on audio and tested without audio, the performance increased over the VT model, showing that the audio serves as useful supervision for better video/text representations. 



\noindent \textbf{Threshold for attention mask.}
As shown in Figure \red{3b}, we apply a threshold to create a mask from the result of attention rollout. Note that this threshold $\tau$ is not a hyperparameter that affects the training or the model but simply serves as a means to an end to compute the \textit{mAP} scores. We did not systematically optimize this threshold, but instead, %chose it as giving the most plausible qualitative results. 
Test different thresholds for attention scores for all relevant models (COMMA, ours) using the spatio-temporal grounding \textit{mAP} IoU@0.4 on our GroundingYoutube dataset as shown in Table \ref{tab:thre}. We find 0.01 to be a reasonable threshold among all models, performing best on COMMA and giving at least the second best results for the proposed model. 

%Spatio-temporal grounding performance in different settings
%Saturated v.s. widespread


%\section{Visualization On Demand} %Visualization Elements
\label{sec:visrisk}
Based on environment data and trajectory evaluation, we now present ways of communicating the situation and risks on a visual display to achieve an ADAS.
In this context, we employ a renderer that visualizes all the information in a joint Cartesian coordinate system (see section \ref{subsec:sim}). 
Once driving risks are detected, design elements are overlayed on the display with section \ref{subsec:active} and section \ref{subsec:warning}. 

\subsection{Simulator Environment}
\label{subsec:sim}
Nodes of the R-LDM have a range of potential attributes, such as the 3D position or geometrical shape of objects. 
% For instance, the road centerline is a polyline with bounderies to the left and right. Crosswalks have a defined width and buildings a polygonal outline description. 
In the renderer, we always visualize static and quasi-static data that lie in the field of view from the ego vehicle. 
For this, a local 3D model is generated by converting geographic points with (lat, lon, alt) into Cartesian coordinates of (x, y, z). 
% and project the positonal relations from a view perspective with a transformation matrix. 
Fig. \ref{fig:3Dsimulator} depicts an exemplary map section having several intersections in bird's-eye view.
% with several intersections, stop lines and crosswalks. 
On the top right, the first person view of a vehicle approaching a crosswalk is shown. 

The dynamic data is then added to this static view. A zoomed-in excerpt from the map is given at the bottom of Fig. \ref{fig:3Dsimulator} that includes a recorded GNSS trace (red).
We project the trace onto the connected lane center, which is pictured in green. 
% Because we project the ego position on the closest lane segment, on the bottom right the measured trace is changed in red and the aligned trace is marked in green.
Consequently, the virtual horizon and its possible paths are retrieved as described in section \ref{subsec:ldm}. 
We can lastly update and move the excerpt with the current position from the GNSS to obtain a live simulation.

\subsection{Proactive Support}
\label{subsec:active}
Communication of spatial as well as spatio-temporal relations is crucial for risk-averse driver support. 
% This has the reason that humans can estimate the time better than positions (especially for risks). 
% Velocity contains implicitly the time as well. 
Further sources of information are cause, likelihood and severity of a potential risks.  
% if a collision happens. 
The next step for RNS is the choice of suitable design elements. 
In this process, we suppose that we know where the ego vehicle is driving (i.e., the ego path) from its navigation route. 
Yet, for surrounding vehicles, all paths are considered.

\subsubsection{Hazard Route Element}
The so-called hazard route in Fig. \ref{fig:charts} is a concept that consists of a scale portraying distances to an upcoming risk element.
Furthermore, the geometrical area or length of risks is considered.
Risk is thus measured with respect to the ego path, ranging from the current position  $\Delta l \hspace{-0.03cm}=\hspace{-0.03cm} \unit[0]{m}$ to the end of the path $\Delta l_{h}$.
Here, the length $\Delta l_{h}$ can be chosen according to own preferences. 

At an upcoming intersection, risk is defined by the section of the path that lies within the junction.
Since risk corresponds to exposition time, we encode the path part from the intersection $I_z$ with a color, ranging from green for short intersections to red for long ones. 
%allgemein risiko entlang des pfades zu intersection zone
%share of junction segment to navigation route + 
%one case with large intersection far and one case with small intersection close
Fig. \ref{fig:charts}~a) gives two examples of the hazard route.
The left bar shows a large intersection (e.g. multi-lane four-way stop) in vicinity and the right bar has a small and consecutive medium junction. 
% In the case of collision risk, the intersection zone $I_z$ can be used.
% Depending on the value of $I_z$ (low, medium and large), the area is marked from green, to yellow until red for conveying the criticality. 
This emphasizes that we may include more than one intersection in our warnings.

\begin{figure}[t]
  \centering
  \includegraphics[width=0.95\linewidth]{./img/simulator.png}
  \caption{Rendered road network from two perspectives with the ego position being projected on the navigation route. \vspace{0.45cm}}
  \label{fig:3Dsimulator}
\end{figure}

\begin{figure}[t]
  \centering
  \resizebox{\linewidth}{!}{
  \import{img/}{velocity_scale_new.pdf_tex}}  
  \caption{Chart elements for proactive support. Hazard route (left) and velocity scale (right).} %\vspace{-0.3cm}}
  \label{fig:charts} 
\end{figure} 

\subsubsection{Velocity Scale Element}
The velocity scale, Fig. \ref{fig:charts}~b), is a second chart element which qualifies the difference between the current velocity of the vehicle $v_0$ and the target velocity $v_{\text{tar}}$ from the trajectory evaluation of section \ref{subsec:trajeval}. 
The scale shows possible velocity values, from standstill $v\hspace{-0.05cm}=\hspace{-0.05cm}\unit[0]{m/s}$ to a maximal velocity $v_{\text{max}}$. Depending on the difference $|v_0 \hspace{0.05cm} - \hspace{0.05cm} v_{\text{tar}}|$, the situation is rated as safe with $v_0 \hspace{-0.042cm} \approx \hspace{-0.042cm} v_{\text{tar}}$ (green, left), as dangerous with e.g. $v_0 \hspace{-0.05cm} < \hspace{-0.05cm} v_{\text{tar}}$ (yellow, middle) to critical with $v_0 \hspace{-0.07cm} \ll \hspace{-0.07cm} v_{\text{tar}}$ (red, right). The same cases hold true for the opposite circumstances, i.e., $v_0 \hspace{-0.032cm} > \hspace{-0.032cm} v_{\text{tar}}$. 
This velocity scale can be employed for curve or regulatory risks. 
Moreover, we may set an enforced speed limit as the target velocity $v_{\text{tar}}$ for proactive behavior, once there is no risk ahead. 
%\noindent -Warning vs behavior support \\
%-Ghost vehicle as in game \\

\subsection{Short-Term Warning Elements}
\label{subsec:warning}
In order to emphasize the criticality of the situation, we propose to add further intuitive warning elements as e.g. pop-up signs and lane colorings. 
The following elements augment the proactive elements.

\subsubsection{Pop-up Signs}
Explicit symbols indicate the risk cause accompanied with the event time for collisions ($s_E$), distances to the risk spot for turns (i.e., right curve with $d_r$ and left curve with $d_l$) or stopping distance for crosswalks ($d_c$). In Fig. \ref{fig:popups}~a), the pop-up signs are pictured. 
% Besides the velocity difference, the risk type is an indication for the severity of the situation.
%Examples for collision risk are car-to-car crash., curve risk can be  as a single-car accident and regulatory risks will be a car-to-object collision. 
We want to stress that this is just a selection and more risk causes can be added. 
The purpose is also to clarify the reason for the warning and give more human-understandable information.

\subsubsection{Colored Events}
Finally, we highlight lane parts or positions according to the corresponding risks.  
% the determined color rating from the hazard route and velocity scale and relate the risks to the simulator environment. 
In the instance of curve and regulatory risk, the lane is colored from the ego position up to the point of maximal risk. 
For collision risk, we mark the point of the closest encounter as a red cube.
An illustration for regulatory risk induced from a stop line is depicted in Fig. \ref{fig:popups}~b). Again, the color is defined by the deviation $|v_0-v_{\text{tar}}|$. It also shows the therein considered navigation route with length $\Delta l_h$ and another unlikely path. 

It should be noted that the visualization of warnings only occurs if the risks are actually present. 
%\textcolor{red}{improve language, repeat intersection zone and navigation route}
%eingrauen unlikely paths and navigation path and describe in text, maybe delete Iz -> put line from unlikely path to green arrow
Altogether, the RNS provides a variety of tools to analyze and circumvent critical situations in intersection scenarios, while not overloading the driver's awareness.

\begin{figure}[t]
  \centering
  \resizebox{\linewidth}{!}{
  \import{img/}{colored_lane_new.pdf_tex}}  
  \vspace{-0.53cm}
  \caption{Short-term warning elements. Selected pop-up warnings (left) and colored lane (right).}
  \label{fig:popups} 
\end{figure} 



The ARMBench dataset presents: 1) a collection of sensor data acquired by a robotic manipulation workcell performing pick-and-place operation, 2) metadata and reference images for objects in containers, 3) a set of annotations acquired either automatically, by virtue of the system design, or via manual labeling, and 4) tasks and metrics to benchmark perception algorithms for robotic manipulation. Fig.\ \ref{fig:contributions} illustrates the benchmark tasks and variety of objects captured in the dataset. The dataset captures diversity in objects with respect to Amazon product categories as well as physical characteristics such as size, shape, material, deformability, appearance, fragility, etc. 

The data collection platform is a robotic manipulation workcell performing pick-and-place operation in a warehouse \cite{Sparrow2022}. The workcell contains a robotic arm mounted with a vacuum-based end-effector. It is presented with a heterogeneous collection of objects placed in unstructured configurations within a container (storage tote). The robotic arm is tasked with picking one object at a time (singulation) and place it on moving trays until the container is empty. The empty container ejects the workcell and is replaced by a new container. While the operation is completely autonomous, it includes a human-in-the-loop to monitor the status of each pick-and-place activity, annotate, and resolve any defects during manipulation. Multiple imaging sensors are placed in the workcell to facilitate and validate the pick-and-place operation. Following is a list of sensor data (Fig.\ \ref{fig:intro}) associated with each pick activity:
\begin{itemize}
\item Pick-image: A 5\,MP camera is used to capture a top-down image of the container.
% \item Pick-3D: Two Ensenso sensors capture the 3D point cloud of the source container.
\item Transfer-images: Multiple 5\,MP cameras are placed on different sides in the workcell to capture the moving object from different viewpoints.
% \item Transfer-Barcode: Multiple Cognex barcode sensors are used to scan the barcode of the object during transfer.
\item Place-image: A top-down view of the object is captured once it is placed on the tray.
\item Video: A camera is mounted to capture 720p videos of pick-and-place manipulation processes at 30\,FPS
\end{itemize}
Additionally, the following metadata (Fig.\ \ref{fig:contributions} (b)) is available by virtue of a warehouse tracking system:
\begin{itemize}
\item Container-manifest: A list of objects present in the container along with data such as product description, coarse dimensions, and weight.
\item Reference images: One or more images of objects from previous operations within the warehouse.
\end{itemize}
The sensor data and metadata were consumed by perception algorithms required to autonomously operate the robotic workcell. Benchmarking against these algorithms would not only optimize a manipulation task such as the one used for data collection but also enable more complex and intentional manipulation. This work considers a subset of such perception tasks namely object segmentation, object identification, and defect detection. These are critical not only to make informed grasping and motion decisions but also to track the state of the objects and containers within the warehouse. The following sections will describe these tasks and present the challenges using annotations, baseline algorithms, and evaluation metrics.

\begin{figure*}[!h]
    \centering
    \includegraphics[width=0.99\linewidth]{assets/qualitative.jpg}
    \caption{
    Qualitative comparisons on Objaverse.
    We compare our textured mesh against CLIPMesh~\cite{mohammad2022clip}, Text2Mesh~\cite{michel2022text2mesh}, Latent-Paint~\cite{metzer2022latent}, and the original textures from Objaverse. In comparison with the baselines, our method produces more consistent and detailed 3D textures with respect to the input geometries. Image best viewed in color.
    }
    \label{fig:qualitative}
\end{figure*}




\par\vfill\par


\clearpage
% % ---- Bibliography ----



%%%%%%%%% REFERENCES
{\small
\bibliographystyle{ieee_fullname}
\bibliography{egbib}
}

\end{document}
