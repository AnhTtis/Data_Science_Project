% ****** Start of file apssamp.tex ******
%
%   This file is part of the APS files in the REVTeX 4.2 distribution.
%   Version 4.2a of REVTeX, December 2014
%
%   Copyright (c) 2014 The American Physical Society.
%
%   See the REVTeX 4 README file for restrictions and more information.
%
% TeX'ing this file requires that you have AMS-LaTeX 2.0 installed
% as well as the rest of the prerequisites for REVTeX 4.2
%
% See the REVTeX 4 README file
% It also requires running BibTeX. The commands are as follows:
%
%  1)  latex apssamp.tex
%  2)  bibtex apssamp
%  3)  latex apssamp.tex
%  4)  latex apssamp.tex
%
\documentclass[%
 reprint,
%superscriptaddress,
%groupedaddress,
%unsortedaddress,
%runinaddress,
%frontmatterverbose, 
%preprint,
%preprintnumbers,
%nofootinbib,
%nobibnotes,
%bibnotes,
 amsmath,amssymb,
 aps,
%pra,
%prb,
prl,
%rmp,
%prstab,
%prstper,
%floatfix,
]{revtex4-2}

\usepackage{graphicx}% Include figure files
\usepackage{dcolumn}% Align table columns on decimal point
\usepackage{bm}% bold math
\usepackage{hyperref}% add hypertext capabilities
%\usepackage[mathlines]{lineno}% Enable numbering of text and display math
%\linenumbers\relax % Commence numbering lines
\usepackage{xcolor}

\newcommand{\BC}[1]{{\bf\color{red}#1}}


%\usepackage[showframe,%Uncomment any one of the following lines to test 
%%scale=0.7, marginratio={1:1, 2:3}, ignoreall,% default settings
%%text={7in,10in},centering,
%%margin=1.5in,
%%total={6.5in,8.75in}, top=1.2in, left=0.9in, includefoot,
%%height=10in,a5paper,hmargin={3cm,0.8in},
%]{geometry}

\begin{document}

\preprint{  https://arxiv.org/abs/2303.09341 %APS/123-QED
}

\title{Accreting Primordial Black Holes:\\%as 
Dark Matter Constituents}% Force line breaks with \\
%\thanks{A footnote to the article title}%

\author{Brandon Curd}
 \altaffiliation[%Also at 
 ]{ brandon.curd@utsa.edu%Physics Department, XYZ University.
 }%Lines break automatically or can be forced with \\
\author{Richard Anantua}%
 %\email{Second.Author@institution.edu}
\affiliation{%
Department of Physics $\&$ Astronomy, The University of Texas at San Antonio, One UTSA Circle, San Antonio, TX 78249, USA\\
% This line break forced with \textbackslash\textbackslash
}%

\collaboration{Event Horizon Telescope Collaboration}%\noaffiliation

\author{T. Kenneth Fowler}
% \homepage{http://www.Second.institution.edu/~Charlie.Author}
\email{TKFowler5@aol.com}
\affiliation{
Department of Nuclear Engineering\\
University of California at Berkeley\\
4153 Etcheverry Hall, 
Berkeley, CA, 94720 USA\\
% This line break forced% with \\ 
}%
% \affiliation{
%  Third institution, the second for Charlie Author
% }%

% \author{Delta Author}
% \affiliation{%
%  Authors' institution and/or address\\
%  This line break forced with \textbackslash\textbackslash
% }%

% \collaboration{CLEO Collaboration}%\noaffiliation

\date{\today}% It is always \today, today,
             %  but any date may be explicitly specified

\begin{abstract}
	This paper shows that accretion of positronium plasma between 0.01s-14s after the Big Bang could have created small black holes contributing at least 1\% of present-day dark matter, with uncertainties ranging from 10\% or more. General relativistic magnetohydrodynamic (GRMHD) simulations newly adapted to the early Universe confirm that accretion is due to magneto-rotational instability (MRI) in a rotating plasma. By contrast with Bondi accretion producing primordial masses above solar, MRI could produce masses 10$^{15-18}$g observable by their Hawking radiation contributing to background gamma rays.
 
 % This paper shows %\luke{}
 % that \textcolor{black} accretion of positronium plasma between 0.01 to 14s %into 
 % after the Big Bang could have created small black holes contributing at least 1$\%$ of the dark matter present today, with uncertainties ranging from 10$\%$ or more. % GRMHD simulations confirm that accretion is due to magneto-rotational instability (MRI) in a rotating plasma. 
 % \textcolor{black}{General relativistic magnetohydrodynamic (}GRMHD\textcolor{black}{)} simulations \textcolor{black}{newly adapted to the early Universe} confirm that accretion is due to magneto-rotational instability (MRI) in a rotating plasma. By contrast with Bondi accretion producing primordial masses bigger than the Sun, MRI could produce masses 10$^{15-18}$ g observable by their Hawking radiation contributing to background gamma rays. 
 %
% \textcolor{black}{($10^{15-18}$g)} created by accretion of positronium plasma %in 
% \textcolor{black}{within 15s of} the Big Bang could have contributed significantly to dark matter abundance today. The main features of accretion leading to this result are the primordial magnetic field and plasma rotation due to the Peebles mechanism acting on seed masses. With rotation, the plasma accretion rate is determined by %magneto%-rotational instability (MRI). Predictions based on MRI are verified by GRMHD simulations discussed in the paper. While our calculation of dark matter abundance is independent of black hole masses, results are consistent with a distribution of masses fitting background gamma radiation to Hawking radiation from black holes.
%magnetohydrodynamic (MRI) turbulence. The role of MRI is confirmed by GRMHD simulations producing Bondi accretion in the absence of rotation. Like previous authors, we suggest that rotation occurred naturally in the primordial environment, by the Peebles mechanism causing non-symmetric seed masses to set each other in counter-rotation. Whereas non-rotating Bondi accretion would have produced masses larger than the Sun, we find masses due to MRI in the range 10$^{15-18}$ g observable as background gammas due to their Hawking radiation.
 
% This paper shows that accretion of positronium plasma in the Big Bang
% could have created small  \textcolor{black}{($10^{15-17}$g) }  black holes 
% %accounting for % \textcolor{red}{[an appreciable fraction of the] [most of the] }
% contributing significantly to the 
% dark matter abundance 
% today \textcolor{red}{as well as the gamma ray background}.  
% %Essential features are the primordial magnetic field and some means of setting the plasma in rotation. 
% \textcolor{black}{The main features of accretion leading to this result are 
% the primordial magnetic field and some means of setting the plasma in rotation.}
% %, 
% %which we have verified using  General Relativistic MHD (GRMHD) simulations %\textcolor{red}{
% With rotation, the accretion rate is determined by maneto-hydrodynamic (MRI) turbulence. 
% % \textcolor{red}{in the novel environment of the primordial Universe}. \textcolor{red}{To accomplish this, this work builds upon magnetized Bondi  accretion in GRMHD simulations with MRI through the unprecedented addition of rotation %via the 
% % as a possible consequence of the 
% % Peebles mechanism. }  
% \textcolor{black}{Predictions based on MRI are confirmed by GRMHD simulations building upon the Bondi accretion case with MRI through the %unprecedented 
% addition of rotation %via the 
% as a possible consequence of the 
% Peebles mechanism in the novel environment of the Primordial Universe.}
% %\textcolor{red}{Expand: What are accretion flows resultung from Bondi and MRI?  }

\begin{description}
\item[%Usage
Keywords]
%Secondary publications and information retrieval purposes.
Dark Matter, Positronium, MRI %Primordial Black Hole, General Relativity, Magnetohydrodynamics, GRMHD
% \item[Structure]
% You may use the \texttt{description} environment to structure your abstract;
% use the optional argument of the \verb+\item+ command to give the category of each item. 
\end{description}

\end{abstract}

%\keywords{Suggested keywords}%Use showkeys class option if keyword
                              %display desired
\maketitle

%\tableofcontents

\section{\label{sec:level1} 1. Introduction 
%First-level heading:\protect\\ The line
%break was forced \lowercase{via} \textbackslash\textbackslash
}

This paper reports \textcolor{black}{general relativistic magnetohydrodynamic } (GRMHD) simulations testing 
our conclusion, in \cite{Fowler2023},
\textcolor{black} that accretion by primordial black holes might have contributed significantly to dark matter present today. As reviewed in  %[2]
\cite{Carr2020}, many authors have pursued Hawking’s early suggestion that dark matter might be primordial black holes. Here we focus on the period 0.01 s to 14 s into the Big Bang, when relativistic positronium plasma was the main constituent of mass, with 10$^{9}$ electrons and positrons per proton and neutron  %[3]
\cite{Weinberg1993}.  
We find that this primordial regime probably included plasma rotating in a magnetic field, giving rise to magneto-rotational instability (MRI) \cite{Balbus1998}. We will show that accretion by MRI could produce black hole masses much smaller than those produced by non-rotating Bondi accretion, hence visible as Hawking radiation, as predicted for low mass primordial black holes with correspondingly high Hawking temperature ($T_\mathrm{HAWKING}\propto 1/M$)  \cite{Carr2020,Anantua2009}. 


% This sample document demonstrates proper use of REV\TeX~4.2 (and
% \LaTeXe) in mansucripts prepared for submission to APS
% journals. Further information can be found in the REV\TeX~4.2
% documentation included in the distribution or available at
% \url{http://journals.aps.org/revtex/}.

% When commands are referred to in this example file, they are always
% shown with their required arguments, using normal \TeX{} format. In
% this format, \verb+#1+, \verb+#2+, etc. stand for required
% author-supplied arguments to commands. For example, in
% \verb+\section{#1}+ the \verb+#1+ stands for the title text of the
% author's section heading, and in \verb+\title{#1}+ the \verb+#1+
% stands for the title text of the paper.

% Line breaks in section headings at all levels can be introduced using
% \textbackslash\textbackslash. A blank input line tells \TeX\ that the
% paragraph has ended. Note that top-level section headings are
% automatically uppercased. If a specific letter or word should appear in
% lowercase instead, you must escape it using \verb+\lowercase{#1}+ as
% in the word ``via'' above.

% \subsection{\label{sec:level2}Second-level heading: Formatting}

% This file may be formatted in either the \texttt{preprint} or
% \texttt{reprint} style. \texttt{reprint} format mimics final journal output. 
% Either format may be used for submission purposes. \texttt{letter} sized paper should
% be used when submitting to APS journals.

% \subsubsection{Wide text (A level-3 head)}
% The \texttt{widetext} environment will make the text the width of the
% full page, as on page~\pageref{eq:wideeq}. (Note the use the
% \verb+\pageref{#1}+ command to refer to the page number.) 
% \paragraph{Note (Fourth-level head is run in)}
% The width-changing commands only take effect in two-column formatting. 
% There is no effect if text is in a single column.

% \subsection{\label{sec:citeref}Citations and References}
% A citation in text uses the command \verb+\cite{#1}+ or
% \verb+\onlinecite{#1}+ and refers to an entry in the bibliography. 
% An entry in the bibliography is a reference to another document.

% \subsubsection{Citations}
% Because REV\TeX\ uses the \verb+natbib+ package of Patrick Daly, 
% the entire repertoire of commands in that package are available for your document;
% see the \verb+natbib+ documentation for further details. Please note that
% REV\TeX\ requires version 8.31a or later of \verb+natbib+.

% \paragraph{Syntax}
% The argument of \verb+\cite+ may be a single \emph{key}, 
% or may consist of a comma-separated list of keys.
% The citation \emph{key} may contain 
% letters, numbers, the dash (-) character, or the period (.) character. 
% New with natbib 8.3 is an extension to the syntax that allows for 
% a star (*) form and two optional arguments on the citation key itself.
% The syntax of the \verb+\cite+ command is thus (informally stated)
% \begin{quotation}\flushleft\leftskip1em
% \verb+\cite+ \verb+{+ \emph{key} \verb+}+, or\\
% \verb+\cite+ \verb+{+ \emph{optarg+key} \verb+}+, or\\
% \verb+\cite+ \verb+{+ \emph{optarg+key} \verb+,+ \emph{optarg+key}\ldots \verb+}+,
% \end{quotation}\noindent
% where \emph{optarg+key} signifies 
% \begin{quotation}\flushleft\leftskip1em
% \emph{key}, or\\
% \texttt{*}\emph{key}, or\\
% \texttt{[}\emph{pre}\texttt{]}\emph{key}, or\\
% \texttt{[}\emph{pre}\texttt{]}\texttt{[}\emph{post}\texttt{]}\emph{key}, or even\\
% \texttt{*}\texttt{[}\emph{pre}\texttt{]}\texttt{[}\emph{post}\texttt{]}\emph{key}.
% \end{quotation}\noindent
% where \emph{pre} and \emph{post} is whatever text you wish to place 
% at the beginning and end, respectively, of the bibliographic reference
% (see Ref.~[\onlinecite{witten2001}] and the two under Ref.~[\onlinecite{feyn54}]).
% (Keep in mind that no automatic space or punctuation is applied.)
% It is highly recommended that you put the entire \emph{pre} or \emph{post} portion 
% within its own set of braces, for example: 
% \verb+\cite+ \verb+{+ \texttt{[} \verb+{+\emph{text}\verb+}+\texttt{]}\emph{key}\verb+}+.
% The extra set of braces will keep \LaTeX\ out of trouble if your \emph{text} contains the comma (,) character.

% The star (*) modifier to the \emph{key} signifies that the reference is to be 
% merged with the previous reference into a single bibliographic entry, 
% a common idiom in APS and AIP articles (see below, Ref.~[\onlinecite{epr}]). 
% When references are merged in this way, they are separated by a semicolon instead of 
% the period (full stop) that would otherwise appear.

% \paragraph{Eliding repeated information}
% When a reference is merged, some of its fields may be elided: for example, 
% when the author matches that of the previous reference, it is omitted. 
% If both author and journal match, both are omitted.
% If the journal matches, but the author does not, the journal is replaced by \emph{ibid.},
% as exemplified by Ref.~[\onlinecite{epr}]. 
% These rules embody common editorial practice in APS and AIP journals and will only
% be in effect if the markup features of the APS and AIP Bib\TeX\ styles is employed.

% \paragraph{The options of the cite command itself}
% Please note that optional arguments to the \emph{key} change the reference in the bibliography, 
% not the citation in the body of the document. 
% For the latter, use the optional arguments of the \verb+\cite+ command itself:
% \verb+\cite+ \texttt{*}\allowbreak
% \texttt{[}\emph{pre-cite}\texttt{]}\allowbreak
% \texttt{[}\emph{post-cite}\texttt{]}\allowbreak
% \verb+{+\emph{key-list}\verb+}+.

% \subsubsection{Example citations}
% By default, citations are numerical\cite{Beutler1994}.
% Author-year citations are used when the journal is RMP. 
% To give a textual citation, use \verb+\onlinecite{#1}+: 
% Refs.~\onlinecite{[][{, and references therein}]witten2001,Bire82}. 
% By default, the \texttt{natbib} package automatically sorts your citations into numerical order and ``compresses'' runs of three or more consecutive numerical citations.
% REV\TeX\ provides the ability to automatically change the punctuation when switching between journal styles that provide citations in square brackets and those that use a superscript style instead. This is done through the \texttt{citeautoscript} option. For instance, the journal style \texttt{prb} automatically invokes this option because \textit{Physical 
% Review B} uses superscript-style citations. The effect is to move the punctuation, which normally comes after a citation in square brackets, to its proper position before the superscript. 
% To illustrate, we cite several together 
% \cite{[See the explanation of time travel in ]feyn54,*[The classical relativistic treatment of ][ is a relative classic]epr,witten2001,Berman1983,Davies1998,Bire82}, 
% and once again in different order (Refs.~\cite{epr,feyn54,Bire82,Berman1983,witten2001,Davies1998}). 
% Note that the citations were both compressed and sorted. Futhermore, running this sample file under the \texttt{prb} option will move the punctuation to the correct place.

% When the \verb+prb+ class option is used, the \verb+\cite{#1}+ command
% displays the reference's number as a superscript rather than in
% square brackets. Note that the location of the \verb+\cite{#1}+
% command should be adjusted for the reference style: the superscript
% references in \verb+prb+ style must appear after punctuation;
% otherwise the reference must appear before any punctuation. This
% sample was written for the regular (non-\texttt{prb}) citation style.
% The command \verb+\onlinecite{#1}+ in the \texttt{prb} style also
% displays the reference on the baseline.

% \subsubsection{References}
% A reference in the bibliography is specified by a \verb+\bibitem{#1}+ command
% with the same argument as the \verb+\cite{#1}+ command.
% \verb+\bibitem{#1}+ commands may be crafted by hand or, preferably,
% generated by Bib\TeX. 
% REV\TeX~4.2 includes Bib\TeX\ style files
% \verb+apsrev4-2.bst+, \verb+apsrmp4-2.bst+ appropriate for
% \textit{Physical Review} and \textit{Reviews of Modern Physics},
% respectively.

% \subsubsection{Example references}
% This sample file employs the \verb+\bibliography+ command, 
% which formats the \texttt{\jobname .bbl} file
% and specifies which bibliographic databases are to be used by Bib\TeX\ 
% (one of these should be by arXiv convention \texttt{\jobname .bib}).
% Running Bib\TeX\ (via \texttt{bibtex \jobname}) 
% after the first pass of \LaTeX\ produces the file
% \texttt{\jobname .bbl} which contains the automatically formatted
% \verb+\bibitem+ commands (including extra markup information via
% \verb+\bibinfo+ and \verb+\bibfield+ commands). 
% If not using Bib\TeX, you will have to create the \verb+thebibiliography+ environment 
% and its \verb+\bibitem+ commands by hand.

% Numerous examples of the use of the APS bibliographic entry types appear in the bibliography of this sample document.
% You can refer to the \texttt{\jobname .bib} file, 
% and compare its information to the formatted bibliography itself.

% \subsection{Footnotes}%
% Footnotes, produced using the \verb+\footnote{#1}+ command, 
% usually integrated into the bibliography alongside the other entries.
% Numerical citation styles do this%
% \footnote{Automatically placing footnotes into the bibliography requires using BibTeX to compile the bibliography.};
% author-year citation styles place the footnote at the bottom of the text column.
% Note: due to the method used to place footnotes in the bibliography, 
% \emph{you must re-run Bib\TeX\ every time you change any of your document's footnotes}. 

\section{2. Dark Matter%Math and Equations
}

In %[1]
\cite{Fowler2023}, we treated accretion by an analytical model that accounted for black holes generated by MRI, contributing to dark matter abundance. MRI has already been shown to be the likely mechanism yielding the ultra-high-energy (UHE) cosmic rays produced by Poynting flux %MRI-driven 
jets from gigantic black holes at the centers of galaxies \cite{Colgate2014},\cite{Colgate2015},\cite{Fowler2019}. Up to the entire present day dark matter abundance was predicted if MRI also determined the accretion rate in primordial positronium.The relevant equations are (from \cite{Fowler2023}, Eqs. (1a) - (1e)):

% Inline math may be typeset using the \verb+$+ delimiters. Bold math
% symbols may be achieved using the \verb+bm+ package and the
% \verb+\bm{#1}+ command it supplies. For instance, a bold $\alpha$ can
% be typeset as \verb+$\bm{\alpha}$+ giving $\bm{\alpha}$. Fraktur and
% Blackboard (or open face or double struck) characters should be
% typeset using the \verb+\mathfrak{#1}+ and \verb+\mathbb{#1}+ commands
% respectively. Both are supplied by the \texttt{amssymb} package. For
% example, \verb+$\mathbb{R}$+ gives $\mathbb{R}$ and
% \verb+$\mathfrak{G}$+ gives $\mathfrak{G}$

% In \LaTeX\ there are many different ways to display equations, and a
% few preferred ways are noted below. Displayed math will center by
% default. Use the class option \verb+fleqn+ to flush equations left.

% Below we have numbered single-line equations; this is the most common
% type of equation in \textit{Physical Review}:
% \begin{eqnarray}
% \chi_+(p)\alt{\bf [}2|{\bf p}|(|{\bf p}|+p_z){\bf ]}^{-1/2}
% \left(
% \begin{array}{c}
% |{\bf p}|+p_z\\
% px+ip_y
% \end{array}\right)\;,
% \\
% \left\{%
%  \openone234567890abc123\alpha\beta\gamma\delta1234556\alpha\beta
%  \frac{1\sum^{a}_{b}}{A^2}%
% \right\}%
% \label{eq:one}.
% \end{eqnarray}
% Note the open one in Eq.~(\ref{eq:one}).

% Not all numbered equations will fit within a narrow column this
% way. The equation number will move down automatically if it cannot fit
% on the same line with a one-line equation:
% \begin{equation}
% \left\{
%  ab12345678abc123456abcdef\alpha\beta\gamma\delta1234556\alpha\beta
%  \frac{1\sum^{a}_{b}}{A^2}%
% \right\}.
% \end{equation}

% When the \verb+\label{#1}+ command is used [cf. input for
% Eq.~(\ref{eq:one})], the equation can be referred to in text without
% knowing the equation number that \TeX\ will assign to it. Just
% use \verb+\ref{#1}+, where \verb+#1+ is the same name that used in
% the \verb+\label{#1}+ command.

% Unnumbered single-line equations can be typeset
% using the \verb+\[+, \verb+\]+ format:
% \[g^+g^+ \rightarrow g^+g^+g^+g^+ \dots ~,~~q^+q^+\rightarrow
% q^+g^+g^+ \dots ~. \]


% \subsection{Multiline equations}

% Multiline equations are obtained by using the \verb+eqnarray+
% environment.  Use the \verb+\nonumber+ command at the end of each line
% to avoid assigning a number:
% \begin{eqnarray}
% {\cal M}=&&ig_Z^2(4E_1E_2)^{1/2}(l_i^2)^{-1}
% \delta_{\sigma_1,-\sigma_2}
% (g_{\sigma_2}^e)^2\chi_{-\sigma_2}(p_2)\nonumber\\
% &&\times
% [\epsilon_jl_i\epsilon_i]_{\sigma_1}\chi_{\sigma_1}(p_1),
% \end{eqnarray}
% \begin{eqnarray}
% \sum \vert M^{\text{viol}}_g \vert ^2&=&g^{2n-4}_S(Q^2)~N^{n-2}
%         (N^2-1)\nonumber \\
%  & &\times \left( \sum_{i<j}\right)
%   \sum_{\text{perm}}
%  \frac{1}{S_{12}}
%  \frac{1}{S_{12}}
%  \sum_\tau c^f_\tau~.
% \end{eqnarray}
% \textbf{Note:} Do not use \verb+\label{#1}+ on a line of a multiline
% equation if \verb+\nonumber+ is also used on that line. Incorrect
% cross-referencing will result. Notice the use \verb+\text{#1}+ for
% using a Roman font within a math environment.

% To set a multiline equation without \emph{any} equation
% numbers, use the \verb+\begin{eqnarray*}+,
% \verb+\end{eqnarray*}+ format:
% \begin{eqnarray*}
% \sum \vert M^{\text{viol}}_g \vert ^2&=&g^{2n-4}_S(Q^2)~N^{n-2}
%         (N^2-1)\\
%  & &\times \left( \sum_{i<j}\right)
%  \left(
%   \sum_{\text{perm}}\frac{1}{S_{12}S_{23}S_{n1}}
%  \right)
%  \frac{1}{S_{12}}~.
% \end{eqnarray*}

% To obtain numbers not normally produced by the automatic numbering,
% use the \verb+\tag{#1}+ command, where \verb+#1+ is the desired
% equation number. For example, to get an equation number of
% (\ref{eq:mynum}),
% \begin{equation}
% g^+g^+ \rightarrow g^+g^+g^+g^+ \dots ~,~~q^+q^+\rightarrow
% q^+g^+g^+ \dots ~. \tag{2.6$'$}\label{eq:mynum}
% \end{equation}

% \paragraph{A few notes on \texttt{tag}s} 
% \verb+\tag{#1}+ requires the \texttt{amsmath} package. 
% Place the \verb+\tag{#1}+ command before the \verb+\label{#1}+, if any. 
% The numbering produced by \verb+\tag{#1}+ \textit{does not affect} 
% the automatic numbering in REV\TeX; 
% therefore, the number must be known ahead of time, 
% and it must be manually adjusted if other equations are added. 
% \verb+\tag{#1}+ works with both single-line and multiline equations. 
% \verb+\tag{#1}+ should only be used in exceptional cases---%
% do not use it to number many equations in your paper. 
% Please note that this feature of the \texttt{amsmath} package
% is \emph{not} compatible with the \texttt{hyperref} (6.77u) package.

% Enclosing display math within
% \verb+\begin{subequations}+ and \verb+\end{subequations}+ will produce
% a set of equations that are labeled with letters, as shown in
% Eqs.~(\ref{subeq:1}) and (\ref{subeq:2}) below.
% You may include any number of single-line and multiline equations,
% although it is probably not a good idea to follow one display math
% directly after another.

 

\begin{subequations}
\label{eq:whole}
\begin{eqnarray}
dM/dt \equiv \dot{M} \approx M/t \approx -4\pi \rho_\mathrm{AMB}R_0^2v(R_0)
% {\cal M}=&&ig_Z^2(4E_1E_2)^{1/2}(l_i^2)^{-1}
% (g_{\sigma_2}^e)^2\chi_{-\sigma_2}(p_2)\nonumber\\
% &&\times
% [\epsilon_i]_{\sigma_1}\chi_{\sigma_1}(p_1).
\label{subeq:0}
\end{eqnarray}
\begin{equation}
f^*\approx -[4\pi\rho_\mathrm{AMB}R_0^2v(R_0)t/(4\pi/3)\rho_\mathrm{AMB}R_0^3]=\frac{3|v(R_0)|t}{R_0},\label{subeq:1}\end{equation}
\begin{equation}
v(R_0) = -(\xi/R)^2v_K; \ \ v_K = \sqrt{MG/R_0}
,\label{subeq:2}
\end{equation}
% \begin{equation}
% \frac{LUKE\ I \ AM\ }{Your\ Father}.
% \end{equation}
\begin{equation}
R_0 = [0.02/(\xi/R)^{4/3}]M(t)^{1/3}t^{2/3}
,\label{subeq:3}
\end{equation}
\begin{equation}
M_\mathrm{DM}/M_0 = \frac{10^9}{\frac{m_p+m_n}{m_e}}f^* \approx 6,%\label{subeq:4}
\end{equation}
\begin{equation}
\dot{M}_\mathrm{MRI}/\dot{M}_\mathrm{BONDI} = \frac{4\pi\rho_\mathrm{AMB}R_0^2v_\mathrm{MRI}}{4\pi\rho_\mathrm{AMB} R_g^2c} \approx 11\frac{t}{t_g} \label{subeq:4}\end{equation}
\end{subequations}
% Giving a \verb+\label{#1}+ command directly after the \verb+\begin{subequations}+, 
% allows you to reference all the equations in the \texttt{subequations} environment. 
% For example, the equations in the preceding subequations environment were
% Eqs.~(\ref{eq:whole}).
Here $M_{DM}$ is dark matter; $M_0$ is ordinary matter (protons, neutrons); $M$ is the black hole mass; $\rho_\mathrm{AMB}\approx\frac{10^5}{t^2}$ is the evolving primordial density \cite{Weinberg1993}; and $f^*$ is the fraction accreted over a sphere of radius $R_0$. 
% Big \textcolor{black}{B}ang expansion ($\rho_\mathrm{AMB}\approx 10^5/t^2$ %[3] 
% \cite{Weinberg1993}) determines the timescale $M/t$. The accretion radius $R_0$ is given by combining Equations (1c) and (1a). The magnitude of $v(R_0)$ is a fraction of “free fall” $v_K$ determined by magnetic field line fluctuations $(\xi/R)$. Typically $(\xi/R) = 0.1$ for MRI [5, Sect. 6.1], giving dark matter abundance $M_\mathrm{DM}/M_0\equiv 6$  in agreement with the current best estimate.
% Note that substituting Equation (1d) into Equation (1b) shows that $f^*$ is independent of $M$ and $t$. 
% Thus our analytical model does not determine the masses involved (nor do simulations below scaled to $M$). For guidance, we turn to cosmological constraints defining “mass windows” in Carr [1, Figure 1]: in particular, the lowest window, due to background gamma rays of uncertain origin.  Interpreting these gamma rays as Hawking radiation from black holes determines the mass range required to accommodate any fraction of dark matter. If all dark matter is included, taking $f(M)$ to be the black hole mass distribution function requires % We identify the relevant mass range as that fitting background gamma radiation to Hawking radiation from a mass distribution f(M), giving 
% $\int_{M_1}^{M_2}dM M f = M_\mathrm{DM}=6M_0$ %M1M2 dM M f = MDM = 6M0 
% for $M_1 = 10^{15} < M < M_2 = 10^{18}$ \textcolor{black}{g} (see Section 5). Interpreting $R_0$ as the radius where MRI first overcomes collisional viscous transport yields $M > 10^{14}$ g consistent with this mass range [%1, 
% \cite{Fowler2023}, end of Section 3]. 
Combining Equations (1a,1c) gives the accretion radius $R_0$ in Equation (1d), with dimensionless $(\xi/R)$ representing MRI turbulence as physical excursions of magnetic field lines (distance $\xi$) at position $R$. Extrapolating from saturated MRI in AGN’s gives $(\xi/R) \approx 0.1$ [%5
\cite{Colgate2015}, Sect. 6.1], yielding dark matter abundance $M_\mathrm{DM}/M_0 \approx 6$ that happens to agree with the current best estimate. 


To see the importance of MRI, Equation (1f) compares MRI accretion due to rotation with non-rotating Bondi accretion [\cite{Frank2002}, Sect. 2.5]. %Richard; this [6] is FKR, now Eq (15). Renumber all references [6] to [15] accordingly. 
Bondi accretion (for the same $M$) is concentrated at the black hole, with accretion range $\approx R_g = (MG/c^2)$ [\cite{Frank2002}, Eq. (2.39)]. The main effect of MRI is to increase the accretion range to $R_0$, essential to achieve  $\dot{M}_\mathrm{MRI}/\dot{M}_\mathrm{BONDI} \propto \frac{R_0^2v_\mathrm{MRI}}{R_g^2c} >>1. $
The MRI accretion time $t = (M/\dot{M})
$ is not determined; rather, $t$ determines  $R_0$. Because $ \rho_\mathrm{AMB}  \propto 1/t^2 $ changes rapidly as the Big Bang expands, $t$ in $ \rho_\mathrm{AMB} $ is the maximum time to accrete at this density. For this reason we equate accretion time $t$ on the left side of Equation (1a) to $t$ in $ \rho_\mathrm{AMB} $ on the right hand side. Thus all accretion times in our model fall in the range $ 0.01 s < t < 14s $ for which $ \rho_\mathrm{AMB} \approx 10^{5}/t^{2} $ is valid \cite{Weinberg1993}.

We note that substituting Equation (1d) into Equation (1b) shows that $f^*$ is independent of $M$ and $t$. Thus our analytical model does not determine the masses involved (nor do simulations below scaled to $M$). For guidance we turn to cosmological constraints defining “mass windows” in Carr and K\"uhnel [\cite{Carr2020}, Fig. 1]: in particular, their lowest window due to background gamma rays of uncertain origin. Interpreting these gamma rays as Hawking radiation from black holes will help determine the range of masses contributing to dark matter (see Section 5). In the  language of Carr and K\"uhnel, we propose two “accretion windows,” theirs due to Bondi accretion (large masses) and our MRI mass window coinciding with their low-mass window associated with background gamma rays. 

%(M*MRI/M*BONDI)  (R02vMRI/Rg2 c) >> 1. The MRI accretion time t = (M/M*) is not determined; rather, t determines R0. Because AMB 1/t2 changes rapidly as the Big Bang expands, t in AMB is the maximum time to accrete at this density. For this reason we equate accretion time t on the left side of Equation (1a) to t in AMB on the right hand side. Thus all accretion times in our model fall in the range 0.01 s < t < 14s for which AMB  105/t2 is valid [3].
%essential for an accretion fraction $f^*_\mathrm{MRI} >> f^*_\mathrm{BONDI}$. The MRI accretion time $t = (M/\dot{M}$) is not determined; rather, $t$ determines $R_0$. Because $\rho_\mathrm{AMB} \propto 1/t^2$ changes rapidly as the Big Bang expands, $t$ in $\rho_{AMB}$ is the maximum time to accrete at this density. For this reason we equate accretion time $t$ on the left side of Equation (1a) to $t$ in $\rho_\mathrm{AMB}$ on the right hand side. Thus all accretion times in our model fall in the range 0.01 s $< t <$ 14s for which $\rho_\mathrm{AMB}$ is valid \cite{Weinberg1993}. The test of the model will be how well simulations in Section 4 reproduce vMRI determining MMRI* and with what effective value of $\xi/R$ due to actual turbulence in the code. 

	%\textcolor{red}{KEN: WHAT IS IN THE ELIPSIS ...? 
 %Before discussing primordial accretion dynamics, we note that substituting Equation (1d) … For guidance to we turn to … gamma rays of uncertain origin. Interpreting these gamma rays as Hawking radiation from black holes will help
%determine that range of masses contributing to dark matter (see Section 5). In the  language of Carr et al. %[2]
%\cite{Carr2020} , we propose two “accretion windows,” theirs due to Bondi accretion (large masses) and our MRI mass window coinciding with their low-mass window associated with background gamma rays.}



% \subsubsection{Wide equations}
% The equation that follows is set in a wide format, i.e., it spans the full page. 
% The wide format is reserved for long equations
% that cannot easily be set in a single column:
% \begin{widetext}
% \begin{equation}
% {\cal R}^{(\text{d})}=
%  g_{\sigma_2}^e
%  \left(
%    \frac{[\Gamma^Z(3,21)]_{\sigma_1}}{Q_{12}^2-M_W^2}
%   +\frac{[\Gamma^Z(13,2)]_{\sigma_1}}{Q_{13}^2-M_W^2}
%  \right)
%  + x_WQ_e
%  \left(
%    \frac{[\Gamma^\gamma(3,21)]_{\sigma_1}}{Q_{12}^2-M_W^2}
%   +\frac{[\Gamma^\gamma(13,2)]_{\sigma_1}}{Q_{13}^2-M_W^2}
%  \right)\;. 
%  \label{eq:wideeq}
% \end{equation}
% \end{widetext}
% This is typed to show how the output appears in wide format.
% (Incidentally, since there is no blank line between the \texttt{equation} environment above 
% and the start of this paragraph, this paragraph is not indented.)

%\section{3. Accretion Scenarios %Cross-referencing}
% REV\TeX{} will automatically number such things as
% sections, footnotes, equations, figure captions, and table captions. 
% In order to reference them in text, use the
% \verb+\label{#1}+ and \verb+\ref{#1}+ commands. 
% To reference a particular page, use the \verb+\pageref{#1}+ command.

% The \verb+\label{#1}+ should appear 
% within the section heading, 
% within the footnote text, 
% within the equation, or 
% within the table or figure caption. 
% The \verb+\ref{#1}+ command
% is used in text at the point where the reference is to be displayed.  
% Some examples: Section~\ref{sec:level1} on page~\pageref{sec:level1},
% Table~\ref{tab:table1},%


%Validity of our calculation of dark matter abundance in Equation (1e) requires
%demonstrating that a complete accretion solution exists. The main effects of MRI are included in the following fluid model %[7]
%requires that a solution exists all the way to the black hole. For this, we rely on the following analytical fluid model, discussed in %[1]
%\cite{Fowler2023}. This fluid model includes the effects of MRI 
%\cite{Balbus1998}
%using the following transport equations for mass and momentum:
%\begin{subequations}
%\label{eq:whole}
%\begin{eqnarray}
%\partial \rho /\partial t = - \mathbf{\triangledown} \cdot (\rho \mathbf{v})
% {\cal M}=&&ig_Z^2(4E_1E_2)^{1/2}(l_i^2)^{-1}
% (g_{\sigma_2}^e)^2\chi_{-\sigma_2}(p_2)\nonumber\\
% &&\times
% [\epsilon_i]_{\sigma_1}\chi_{\sigma_1}(p_1).
%\label{Accretionsubeq:0}
%\end{eqnarray}
%\begin{equation}
%\partial (\rho \mathbf{v}) /\partial t = \mathbf{F}(\Omega) -\rho \mathbf{\triangledown}(\Phi +v^2) -  \mathbf{\triangledown}p + [\frac{1}{4\pi}(\mathbf{\triangledown}\times \mathbf{B})\times \mathbf{B}],\label{Accretionsubeq:1}\end{equation}
%\begin{equation}
%-\frac{\partial^2 \rho}{\partial t^2} = \mathbf{\triangledown}\cdot \lbrace \mathbf{F}(\Omega) -\rho \mathbf{\triangledown}(\Phi +v^2) -  \mathbf{\triangledown}p + [\frac{1}{4\pi}(\mathbf{\triangledown}\times \mathbf{B})\times \mathbf{B}] \rbrace
%,\label{Accretionsubeq:2}
%\end{equation}
%\begin{equation}
%\frac{\partial \mathbf{B}}{\partial t} + \mathbf{\triangledown}\times \left(\mathbf{v} \times \mathbf{B} \right)= \mathbf{\triangledown}\times (c \mathbf{D});  \mathbf{D} = -\frac{1}{c}<\mathbf{v}_1\times\mathbf{B}_1>
%,\label{Accretionsubeq:3}
%\end{equation}
%\begin{equation}
%v_R \approx \lbrace (k_\mathrm{R}\xi_\mathrm{R})^2_\mathrm{MRI}v_\mathrm{K} + [1-(R/R_0)] (k_\mathrm{R}\xi_\mathrm{R})^2_\mathrm{Jeans} c_s\rbrace,
%\label{Accretionsubeq:4}\end{equation}
%\begin{equation}
%R_1/R_0 = 0.4(\xi/R)\left(\frac{R_g}{ct}\right)^\frac{1}{3}
%\label{Accretionsubeq:4}\end{equation}
%\end{subequations}
%with mass density $\rho$, pressure $p$, flow velocity $\mathbf{v}$, magnetic field $\mathbf{B}$ and velocity and magnetic field perturbations $\mathbf{v}_1$ and $\mathbf{B}_1$,  in non-relativistic form with the understanding that high temperature replaces rest mass $m_0$ by $m_0(1 + T/m_0c^2)$. Here $\mathbf{F}(\Omega)$ represents rotational Coriolis and tidal forces driving MRI \cite{Balbus1998} %[7]
%that contributes to hyper-resistivity \textbf{D} driving accretion %[6] 
%\cite{Fowler2019}. 
%The gravitational potential $\Phi$ directs flow toward the black hole. A  transition would occur if the flow velocity $v_R \to c$ short of the black hole, occurring at R1 in Equation (2f) given by $R_1^2c = R_0^2v_\mathrm{MRI}(R_0)$. 

%The wave Equation (2c) for waves $ \propto \exp i[k_zz + k_R\xi_R - \omega t] $ combines MRI and Jeans gravitational instability \cite{Toomre1964} %[8] 
%giving $vR \approx <cD/B_z>$ yielding Equation (2e) with sound speed $c_S$. The wave equation is obtained by taking the divergence of Equation (2b) using Equation (2a). MRI instability sets in when $3\Omega^2$ (Keplerian) $> k_z^2v_\mathrm{Alfven}^2$ 
%[\cite{Balbus1998}, Eq. (111)]; Jeans instability when $k_R^3(MG) > k_R^2c_S^2 $ %[8]
%\cite{Toomre1964}.
%Ohm’s Law in Equation (2d) propagates the poloidal magnetic field (cylindrical $B_z,B_r$) to cylindrical $R = R_0, z = 0$ serving both as the accretion radius and as the O-point around which poloidal magnetic  flux circulates. Ohm’s Law includes magnetic flux compression $\mathbf{v} \times \mathbf{B}$ discussed in %[9,10] 
%\cite{Igumenshchev2002,McKinney2012}. 
% In \cite{Fowler2023}, %[1], 
% we showed how this combination of MRI, Ohm’s Law and gravitational flow could maintain accretion around primordial black holes.  

.

% \begin{table}[b]%The best place to locate the table environment is directly after its first reference in text
% \caption{\label{tab:table1}%
% A table that fits into a single column of a two-column layout. 
% Note that REV\TeX~4 adjusts the intercolumn spacing so that the table fills the
% entire width of the column. Table captions are numbered
% automatically. 
% This table illustrates left-, center-, decimal- and right-aligned columns,
% along with the use of the \texttt{ruledtabular} environment which sets the 
% Scotch (double) rules above and below the alignment, per APS style.
% }

% \begin{ruledtabular}
% \begin{tabular}{lcdr}
% \textrm{Left\footnote{Note a.}}&
% \textrm{Centered\footnote{Note b.}}&
% \multicolumn{1}{c}{\textrm{Decimal}}&
% \textrm{Right}\\
% \colrule
% 1 & 2 & 3.001 & 4\\
% 10 & 20 & 30 & 40\\
% 100 & 200 & 300.0 & 400\\
% \end{tabular}
% \end{ruledtabular}
% \end{table}
% %and Fig.~\ref{fig:epsart}.%

%\begin{figure}[b]
%\hspace{-1cm}\includegraphics[width=10cm]{vrsm%InflowVelocity
%.png}% Here is how to import EPS art
%\caption{\label{fig:epsart}  Showing the accretion velocity $v_R$ versus spherical radius $R$, at 
%various times (units $R_g = (MG/c^2); t_g = (c/R_g))$. Note oscillations in $v_R$ due to MRI
%at $R > 100 R_g$. 
%A figure caption. The figure captions are
%automatically numbered.
%}
%\end{figure}

\section{3. GRMHD Simulations%Floats: Figures, Tables, Videos, etc.
}

%The purpose of this paper is to use computer simulations to verify the MRI-driven accretion scenario in Section 3 that explains dark matter abundance.

% This paper uses computer simulations to explore the MRI-driven accretion scenario in Section 3 that could make a significant contribution to dark matter
% \textcolor{black}{in the present Universe.}
The main purpose of this paper is to verify the role of MRI in dark matter abundance to the extent possible with existing  simulation codes. Magnetized Bondi accretion without rotation has been studied extensively \cite{Cunningham2012}, still giving results related to the Bondi rate. Rotation in a magnetic field drives MRI, giving entirely different results. To add MRI we employ the KORAL General Relativistic MHD (GRMHD) code previously used to model MRI in other contexts \cite{Sadowski2013}.
%
%\textcolor{red}{We employ the KORAL General Relativistic MHD code %already used 
%which has been extensively used to model MRI in other contexts %[11]
%\cite{Sadowski2013}. }
%
% \subsection{4a Code %Requirements and 
% Limitations}

%The focus of this work is the role of plasma rotation in accretion in a 
%magnetic field. Magnetized Bondi accretion without rotation has been studied extensively [\cite{Cunningham2012}%11
%, refs. therein], still giving results related to the Bondi rate. Rotation in a magnetic field drives MRI, giving entirely different results. 
%We employ the KORAL GRMHD code previously used to model MRI in other contexts \cite{Sadowski2013}. %[12]. 
%While the primordial field was pervasive \cite{Enqvist1998}%[13]
%, in [1] we suggest that rotation giving MRI required the Peebles’ mechanism causing non-symmetric seed masses to set each other in counter-rotation %[
%\cite{Peebles1969}%14
%Peebles ref.]
. %In KORAL, we create MRI by imposing a weak initial Bz field from $R_1 < R < R_2$ and Keplerian rotation from $\_\_\_\_\_\_$. 

%The parameter range is challenging (see Table 1, applying AGN formulas in \cite{Colgate2014,Colgate2015,Fowler2019} %[4, 5, 6] 
%to the primordial regime for typical $M \approx 10^{16}g, t \approx 1s$). In natural units ($R_g = MG/c^2;$ time $t_g = R_g/c$), one second of accretion time is $10^{23} t_g$ whereas KORAL runs are typically limited to $10^6 t_g$.  
%Yet we find that the accretion velocity required to determine dark matter abundance in Equation (1e) already manifests itself in KORAL simulations near the black hole.


\subsection{3a. Code %Requirements and 
Limitations} 

The KORAL code is fully described in \cite{Sadowski2013}. Phenomena of interest are covered including flux compression in \cite{Igumenshchev2002},\cite{McKinney2012}. Here we use KORAL to calculate accretion as the MRI accretion radius $R_0$ grows in time, starting near the black hole. We evolve the fluid in two dimensions $(r, \theta)$ in modified Kerr-Schild coordinates. The radial grid cells are spaced logarithmically, and we choose inner and outer radial bounds $R_{\rm min}=10^2 GM/c^2$ and $R_{\rm max}=10^6 GM/c^2$. We also use a full $\pi$ in polar angle $\theta$. We choose outflow boundary conditions at both the inner and outer radial bounds, reflective boundary conditions at the top and bottom polar boundaries, and periodic boundary conditions in azimuth. In each simulation, we employ a resolution $N_r\times N_\theta= 640\times 160$. Only the gravity of a Schwarzschild black hole of fixed mass $M$ is included. We take the seed magnetic field generating MRI to be poloidal (cylindrical $B_r,B_z$). While pure $B_z$ creates MRI in the analysis in [\cite{Balbus1998}, Sect. IV.B], we found that an initial $B_r/B_z  \approx$ 0.1 expedites MRI development in KORAL. The parameter range is challenging. See Table 1 applying AGN formulas in \cite{Colgate2014},\cite{Colgate2015},\cite{Fowler2019} to the primordial regime  for typical $M \approx 10^{16}$g, $t \approx$ 1s.  All quantities in the table employ cgs units except temperature in keV. In “natural” units ($R_g = MG/c^2$ and $t_g = R_g/c$), one second of accretion time is $10^{23} t_g$. Our simulation runs up to  $t$  = 5 x $10^7t_g$ are limited to the formation of an MRI jet current near the black hole where most of the accretion power is deposited \cite{Colgate2014}. 
\begin{table*}
\caption{\label{tab:table3}
% This is a wide table that spans the full page
% width in a two-column layout. It is formatted using the
% \texttt{table*} environment. It also demonstates the use of
% \textbackslash\texttt{multicolumn} in rows with entries that span
% more than one column.
Primordial MRI jet parameters, for typical $M=10^{16}$g, density $\rho=10^5$ g/cm$^3$ and temperature $T=10^3$ keV at $t=1$s. All  quantities are in cgs units, except $T$ in keV.
}
\begin{ruledtabular}
\begin{tabular}{ccccc}
 &\multicolumn{2}{c}{%$D_{4h}^1$
 }&\multicolumn{2}{c}{%$D_{4h}^5$
 }\\
 Quantity& %Equation 
 &  Predicted Value&
&Notes\\ \hline
 %Primordial Temperature& %$(3)$
% &$T=10^3/\sqrt{t}$ keV ; $t=0.01-14$ s&& \\
% Primordial Density&
 %$(3)$\footnote{The $z$ parameter of these positions is $z\sim\frac{1}{4}$.}
% &$\rho=10^5/t^2$ g/cm$^3$ & &\\
MRI magnetic field, poloiodal  & & $B_z \approx 10^{10} (a/R)^{3/2} $ gauss%\footnotemark[1]
 %&$(4e)^{\text{a}}$
 \\
MRI magnetic field, toroidal &  & $B_\phi \approx 10^{10} (a/R) $ gauss \\
 Gravitational radius $r_g$, gravitational time $t_g$	& &$r_g\approx10^{-12}$cm, $t_g=10^{-23}$s& &\\
MRI jet Radius $a$& & $a\approx$ 1cm & & (resistive)\\
MRI jet Length $L$	& &$L=0.01ct\approx10^8$cm& &\\
 Rotation $\Omega$ & &$a\Omega/c\approx\sqrt{r_g/5a}\approx10^{-6}$ & &\\
\end{tabular}\label{Table1}
\end{ruledtabular}
\end{table*}


%KORAL calculates accretion as the MRI accretion radius $R_0$ grows in time, starting near the black hole. Only the gravity of a fixed black hole mass M is included. Our analysis in \cite{Fowler2023}%[1]
%suggests that the omitted self-gravity between accreting masses is not important in establishing the onset of MRI, our main objective. 

%\textcolor{red}{KEN: WHAT IS IN THE ELIPSIS?
%But the parameter range is challenging. See Table 1, …  for typical $M \approx  10^16$ g, $t \approx 1$s.  Quantities in the table employ cgs units except temperature in keV. In “natural” units (Rg = MG/c2 and $_g = R/c$), one second of accretion time …


% Figures and tables are usually allowed to ``float'', which means that their
% placement is determined by \LaTeX, while the document is being typeset. 

% Use the \texttt{figure} environment for a figure, the \texttt{table} environment for a table.
% In each case, use the \verb+\caption+ command within to give the text of the
% figure or table caption along with the \verb+\label+ command to provide
% a key for referring to this figure or table.
% The typical content of a figure is an image of some kind; 
% that of a table is an alignment.%

% \begin{figure*}
% \includegraphics{CurdFowlerAnantuaFig2.png
% %fig_2
% }% Here is how to import EPS art
% \caption{\label{fig:wide}Use the figure* environment to get a wide
% figure that spans the page in \texttt{twocolumn} formatting.}
% \end{figure*}

\begin{figure}[b]
\hspace{-0.2cm}
\includegraphics[width=\columnwidth]{Figure_1%final_figures/Figure_1%vrsmv3%Bv3%Bv2%B
.png}% Here is how to import EPS art
\caption{\label{vrsmv3%PMagOnPTot
} %A figure caption. The figure captions are automatically numbered.
Showing the accretion velocity $v_R$ versus spherical radius $R$ at various times (units: $R_g=GM/c^2; t_g=R_g/c$). Note peaks due to MRI. % Showing the magnetic field $B$ versus $R$ at various times. Note 
% correlation of oscillations in $B$ with oscillations in $v_r$ in Figure 1. 
}
\end{figure}

\begin{figure}[b]\hspace{-2cm}
\includegraphics[width=\columnwidth]{Figure_2%final_figures/Figure_5%Bv3%omegav3%Omegav2(1)%Fig3Rot%Fig3Bondiconverged%Deltab
.png
}% Here is how to import EPS art
\caption{\label{B%fig:epsart
} %A figure caption. The figure captions are
%automatically numbered.
Showing the magnetic field $B$ versus $R$ at various times. Note 
correlation of oscillations in $B$ with oscillations in $v_r$ in Figure 1. Curves are scaled by the maximum field strength at the initial state, $B_{max,0}$. %Showing the rotation frequency $\Omega$ versus $R$ at various times.
%Note flattening of $\Omega(R)$ in response to MRI.
}
\end{figure}

\begin{figure}[b]
\hspace{-1.4cm}\includegraphics[width=\columnwidth]{Figure_3%final_figures/Figure_2%omegav3%mdot3%mdot.png
}% Here is how to import EPS art
\caption{\label{Omega%MDot
} 
%A figure caption. The figure captions are
%automatically numbered.
Showing the rotation frequency $\Omega$ versus $R$ at various times.
Note fall in $\Omega(R)$ behind the MRI accretion front.
%Showing the mass accretion rate $M* = dM/dt$ at various times. Fully 
%Developed MRI should further flatten $M^*$ toward the black hole.
}
\end{figure}

\begin{figure}[b]
\hspace{-1.4cm}\includegraphics[width=\columnwidth]{Figure_4%final_figures/Figure_4%mdot3%vrsmv3.png
}% Here is how to import EPS art
\caption{\label{MDot%vrsmv3
} 
%A figure caption. The figure captions are
%automatically numbered.
Showing the mass accretion rate $\dot{M} = dM/dt$ at various times. 
%Fully Developed MRI should further flatten $M^*$ toward the black hole.%CAPTION NEEDED
}
\end{figure}

\begin{figure}[b]
\hspace{-1.4cm}\includegraphics[width=\columnwidth]{Figure_5%final_figures/Figure_3%betav2%mdot3%vrsmv3.png
}% Here is how to import EPS art
\caption{\label{betav2%MDot%vrsmv3
} 
%A figure caption. The figure captions are
%automatically numbered.
Showing $\beta$  the ratio of plasma pressure to magnetic presssure.
%Fully Developed MRI should further flatten $M^*$ toward the black hole.%CAPTION NEEDED
}
\end{figure}


% \begin{figure}[b]
% \hspace{-1.4cm}\includegraphics[width=10cm]{mdot3%vrsmv3.png
% }% Here is how to import EPS art
% \caption{\label{betav2%MDot%vrsmv3
% Showing $\beta$  the ratio of plasma pressure to magnetic presssure.} 
% %A figure caption. The figure captions are
% %automatically numbered.
% %Showing the mass accretion rate $\dot{M} = dM/dt$ at various times.
% %Fully Developed MRI should further flatten $M^*$ toward the black hole.%CAPTION NEEDED
% }
% \end{figure}

%%%%%%%%%%%%%%%%%%%%%%%%%%%%%%%%%%%%%%%%%%%%%%%%%%%%%%%%
%TABLE PLACEHOLDER
%%%%%%%%%%%%%%%%%%%%%%%%%%%%%%%%%%%%%%%%%%%%%%%%%%%%%%%%

% \subsection{4a Code Requirements and Limitations}

% The focus of this work is the role of plasma rotation in accretion in a 
% magnetic field. Magnetized Bondi accretion without rotation has been studied extensively [\cite{Cunningham2012}%11
% , refs. therein], still giving results related to the Bondi rate. Rotation in a magnetic field drives MRI, giving entirely different results. We employ the KORAL GRMHD code previously used to model MRI in other contexts \cite{Sadowski2013}. %[12]. 
% While the primordial field was pervasive \cite{Enqvist1998}%[13]
% , in [1] we suggest that rotation giving MRI required the Peebles’ mechanism causing non-symmetric seed masses to set each other in counter-rotation %[
% \cite{Peebles1969}%14
% %Peebles ref.]
% . %In KORAL, we create MRI by imposing a weak initial Bz field from $R_1 < R < R_2$ and Keplerian rotation from $\_\_\_\_\_\_$. 

% The parameter range is challenging (see Table 1, applying AGN formulas in \cite{Colgate2014,Colgate2015,Fowler2019} %[4, 5, 6] 
% to the primordial regime for typical $M \approx 10^{16}g, t \approx 1s$). In natural units ($R_g = MG/c^2;$ time $t_g = R_g/c$), one second of accretion time is $10^{23} t_g$. Yet we find that the accretion velocity required to determine dark matter abundance in Equation (1e) already manifests itself in KORAL simulations near the black hole.


\subsection{3b. Qualitative Results}

The importance of the magnetic field is verified in Figures \ref{vrsmv3}, \ref{B}, \ref{Omega}; 
%1, 2, 3; 
with corresponding $\dot{M}$ in Figure \ref{MDot} %4 
(logarithmic scales). Spherically-averaged magnitudes are plotted $(\ln|X(R)|)$. Though this obscures any changes in sign, we find this to be representative of the code data.  

Figure 1 compares 
% The importance of the magnetic field is verified in Figures 1, 2 and 3 (logarithmic scales). Figure 1
% compares 
accretion velocity $v_R (R)$ with quasi-static Bondi accretion $\propto 1/R^2$ (\cite{Frank2002}%/[15] 
%FKR
, Sect. 2.5). Figure \ref{B} %2 
plots magnetic field $B(R)$ and Figure \ref{Omega} %3
plots $\Omega(R)$ as rotation adjusts to MRI dynamics. 
Figure 4 gives the corresponding accretion rate $\dot{M}$. To excite MRI, we initiate code runs with Keplerian rotation. In order to adequately resolve MRI in KORAL, we ensure that the fastest growing mode of the MRI is resolved via the MRI quality factor \citep{Hawley2011} $Q_\theta = \frac{2\pi}{\Omega\, dx^\theta}\frac{|b^\theta|}{\sqrt{4\pi\rho}}$, where $dx^\theta$ is the grid cell length in $\theta$ and $b^\theta$ is the magnetic field component in $\theta$. $Q_\theta$ must exceed 5 at a minimum to resolve the MRI. We choose an initial gas to magnetic pressure ratio $\beta$ such that $Q_\theta \approx 196$ initially and grows as the magnetic field amplifies. Runs are in 2D, confirmed by tests in 3D. We also verified that the effects of MRI are independent of $\beta$ (as long as the plasma is gas pressure dominated and MRI is resolved) by rerunning the simulation with $\beta$ increased by 10 and 100 times, respectively. We found nearly identical evolution in each case. Tests of KORAL's validity are discussed in \cite{Sadowski2013}.

In Figure \ref{vrsmv3}, %1, 
we draw attention to peaks in $v_R(R)$ advancing along the curve labeled $0.001v_K$, giving $v_R >> v_{BONDI}$ late in time. Peaks in $v_R(R)$ correlate with peaks in $B(R)$ in Figure \ref{B} %2
and peaks in $\Omega(R)$ in Figure \ref{Omega}, %3
indicating MRI at these locations (if $3\Omega^2 > k^2{v_\mathrm{ALFVEN}}^2$) giving $R/R_g < 3(4\pi\rho{c_s}^2/B^2)\approx 3\beta$ \cite{Balbus1998}; see Figure \ref{betav2}. %5. 
These peaks will be found to advance as $(t/t_{g})^{2/3}$, leading us to identify moving peaks in $v_R$ with the accretion radius $R_0(t) \propto t^{2/3}$ from Ohm's law [\cite{Colgate2015}, Eq. (49)]. Thus these correlated peaks in $v(t)$ and $B(t)$ confirm our predicted propagation of the accretion radius $R_0(t)$ by MRI \cite{Fowler2023}. The dominance of MRI at $R_0(t)$ in Equation (1d) is all that is required to account for dark matter abundance in Equation (1e). 

Accretion at $R_0$ is not much affected by accretion before and after $R_0$. In Figure \ref{vrsmv3} %1,
accretion at $R < R_0$ is due to ejection of angular momentum by the MRI magnetic jet (cylindrical $\partial(\rho r^2\Omega)/\partial t = (1/4\pi)\partial(rB_{\Phi}B_z)/\partial z$ + … [\cite{Colgate2014}, %5,
Eq. (5)]). This leaves weakly-rotating plasma behind the MRI accretion wave, as seen in Figure \ref{Omega}. %3.
Weak rotation kills MRI, so that in KORAL MRI is localized around the advancing $R_0(t)$, while in general $V_R$ at $R_0$ may be due to propagation from MRI in the interior \cite{Fowler2023}. Weak rotation inside $R<R_0$ gives Bondi accretion in a magnetic field discussed in \cite{Cunningham2012}. This scenario is unlikely in space where rotation can be sustained by recycling angular momentum via large MRI jet current loops closing on themselves \cite{Colgate2014}, \cite{Colgate2015}. The more likely scenario in space would be Jeans gravitational instability, %(omitted in KORAL), 
discussed in Section 5.

%The $v_R$ soon pulls away from Bondi accretion, instead following the dashed line proportional $\propto v_K$ $\propto 1/R^{1/2}$. That this is likely due to MRI follows from the correlation of $v_R$ fluctuations in Figure 1 with magnetic fluctuations in Figure 2. Both $v_R$ and the magnetic field propagate outward as predicted. Runs are in 2D (axisymmetric), confirmed by test runs in 3D. Many tests of KORAL validity (conservation laws etc.) are discussed in \cite{McKinney2012}. That our test runs with zero rotation recover Bondi accretion verifies our principal finding that it is rotation that greatly extends the accretion range. 

%We initiate code runs with Keplerian rotation in Figure 3.As noted above, our simulations are mostly limited to $R < a$, the radius of jet current ejected from the accretion disk, this being the region where accretion power driving MRI is concentrated \cite{Colgate2014},\cite{Colgate2015}.That $\Omega(t) \rightarrow 0$ at inner radii concerns the evolution of jet current in time. MRI ejects accretion disk angular momentum as magnetic jets:$½ \dot{M}\Omega = RB_{\Phi}B_z$  with  jet current $j_z \approx 
%[(B_{\Phi} /R) + (\partial B_{\Phi}/\partial R)] $. Taking $B_z$ and $B_{\Phi}$ constant at R <  a gives $j_z \approx \Omega \approx R \rightarrow 0$ near the black hole [\cite{Colgate2015}, App. B].Runs had to be  terminated when R = a approaches the radial boundary of the simulation, to avoid spurious effects at the boundary.  
%However, note that Figures 1 and 4 indicate a halt in advancing $v_R$ at $t > 10_6 t_g$ while $\lvert B \rvert$ continues to advance in Figure 2. How Jeans instability not included in KORAL could sustain mass flow at large R is discussed in [\cite{Fowler2023}] and Section 5; and hyper-resistive mass flow is discussed in [\cite{Fowler2019}, App. A].  

%confirms the tendency of $\dot{M}$ to flatten toward
%the black hole as accretion proceeds. In Figure 1, note that $v_R$ first follows Bondi 

%The
%$v_R$ first follows Bondi accretion, then pulls away to exceed Bondi at $R > 1000 R_g$. That this is likely due to MRI follows from the correlation of $v_R$ fluctuations in Figure 1 with magnetic fluctuations in Figure 2. That this magnetic turbulence propagates the accretion radius outward, as predicted, is shown by the advancing $v_R$ wave-front in Figure 1, reaching $R \approx  5000R_g $ by $t = 10^6t_g$. 
%To conserve computer time, these calculations are 2D (axi-symmetric), sufficient to capture MRI onset \cite{Balbus1998}. % onset [7].
%Results in Figure 1 have been corroborated in 3D \textcolor{red}{(????)}. 
%Results in 3D are essentially the same as 2D for a test case $10^2 < R/R_g < 10^6, t = 10^5 t_g$, except that $v_R$ at the wave front grows somewhat faster in 3D. 
%That the dashed line $v_R \propto 1/R^2$ in Figure 1 is representative of Bondi accretion 
%for our problem was confirmed by a simulation of pure Bondi accretion ($\Omega= B = 0$) in Figure 3, at a temperature slightly below that in Table 1 (to avoid an initial thermal shock not important in understanding MRI accretion).
%has been confirmed in simulations setting $\Omega=0$. 
%A different technique extending Figure 1  to $10^7 t_g$ did not change our fit $v = 0.001 v_K$. 
 
\subsection{3c.  Numbers}
The main number of interest is $A$ in $v_R$ = $Av_K$ in Equation (1c) giving the “quasi-linear” formula $A=(\xi/R)^2$ \cite{Fowler2023}; while KORAL determines A from actual turbulence in the code. By equations (1b), (1c) and (1e), dark matter abundance $M_{DM}/M_0$ varies as:
\begin{equation} 
    M_{DM}/M_{0}= 6 \times 10^{4}A^2
\end{equation}
%As noted above, accounting for $100\%$ of dark matter by our model would require $v_R = C^*v_K$  (Keplerian) with $C^* = (\xi/R)^2 = 0.01$ in Equation (1c). The accretion fraction giving this result is very sensitive to $C^*$, with $f* = (3v_R(R_0)t/R_0) \propto C*^2$. The $v_R$ in Figure 1 breaks away from Bondi rate, with $C^* = 0.001$ at the wave front. This value $C^* = 0.001$ only accounts for $\gtrsim 1\%$ of dark matter abundance.
%, while associating background gamma radiation with Hawking Radiation 
%% As noted above, fully accounting for dark matter abundance by our model 
%% would require $v_R = C*vK$ (Keplerian) with $C* = (\xi/R)^2 = 0.01$ in Equation (1c). The vR in Figure %1 breaks away from the Bondi rate, with C* = 0.001 at the wave front at $t = 10^6 t_g$. 
%% %and evidence for a higher value at longer run times (\propto???). 
%% The value $C^* = 0.001$ already accounts for 10$\%$ of dark matter abundance. Moreover, associating %background gamma radiation with 
%%Hawking radiation 
%from dark matter can fully account for dark matter abundance with a black hole mass distribution accessible by MRI accretion ($< 10^{17} g$, see Section 5). Bondi accretion alone produces weakly-radiating masses greater than the Sun [2], while the existence of an accretion “window” at lower mass $M < 10^{17}$ g is verified in [2, Figure 1]. We conclude that accretion initiated by MRI can account for at least %10$\%$ 
%1$\%$ of dark matter, and possibly all of dark matter, as black holes with $M = 10^{15-17}$ g.

%\subsection{4d. Accretion Rate}

%The accretion rate $\dot{M}$ is shown in Figure 4. The dynamics of $\dot{M}$ is found by integrating mass flow Equation (2a) over the accretion disk volume. We obtain:  

%\begin{subequations}
%\label{eq:whole}
%\begin{eqnarray}
%\int_{R_g}^{R_0} dR 4\pi R^2 \partial \rho  /\partial t = \int_{R_g}^{R_0} 4\pi R^2 \left[ -R^{-2} \frac{\partial(R^2\rho v_R)}{\partial R} \right]
% {\cal M}=&&ig_Z^2(4E_1E_2)^{1/2}(l_i^2)^{-1}
% (g_{\sigma_2}^e)^2\chi_{-\sigma_2}(p_2)\nonumber\\
% &&\times
% [\epsilon_i]_{\sigma_1}\chi_{\sigma_1}(p_1).
%\label{Accretionsubeq:0}
%\end{eqnarray}
%\begin{equation}
%= 4\pi R_0^2v_R(R_0)\rho_\mathrm{AMB} -  \frac{dM_\mathrm{BH}}%{dt},\label{Accretionsubeq:1}\end{equation}
%\begin{equation}
%\frac{dM_\mathrm{BH}}{dt} = \dot{M} =  4\pi R_0^2v_R(R_0)\rho_\mathrm{AMB} -  \frac{dM_\mathrm{Disk}}{dt}
%,\label{Accretionsubeq:2}
%\end{equation}
% \begin{equation}
% \frac{\partial \mathbf{B}}{\partial t} + \mathbf{\triangledown}\times \left(\frac{1}{c} \mathbf{v} \times \mathbf{B} \right)= \mathbf{\triangledown}\times \mathbf{D};  \mathbf{D} = -\frac{1}{c}<\mathbf{v}_1\times\mathbf{B}_1>
% ,\label{Accretionsubeq:3}
% \end{equation}
% \begin{equation}
% v_B \approx \lbrace (k_\mathrm{R}\xi_\mathrm{R})^2_\mathrm{MRI}v_\mathrm{K} + [1-(R/R_0)] (k_\mathrm{R}\xi_\mathrm{R})^2_\mathrm{Jeans} c_s\rbrace,
% \label{Accretionsubeq:4}\end{equation}
%\end{subequations}
%Here $dM_\mathrm{BH}/dt$ is $\dot{M}$ in Equation (1a) and $dM_\mathrm{Disk}/dt$ is the left hand side of	 Equation (3a) representing any accumulation or depletion of mass inside the accretion disk. 

% Ultimately, $dM_\mathrm{Disk}/dt\to 0$ giving the same $\dot{M}$  at all $R$ as time evolves. To check this, in Figure 5 we avoid the long buildup of $\dot{M}$  from zero in Figure 4 by plotting $\dot{M}$  already assuming the MRI velocity as a boundary condition $v_R = C^*v_\mathrm{MRI}(R_0)$ for $R_0$ much larger than that in Figure 4. Varying C* %in Figure 5 
% to flatten $\dot{M}$  gives a different bound of order C* = $\_\_\_\_\_$, corresponding to $\_\_\_\_\_$ $\%$ of dark matter abundance. 

%\subsection{4e. Near the Black Hole}
% While the main feature of our model –- outward propagation of the accretion radius by MRI -- has been verified, the fact that KORAL only includes gravity by a fixed mass M suppresses features of accretion near the black hole predicted in %[1]
% \cite{Fowler2023}. Fixing $M$ in KORAL (and in \cite{Cunningham2012}%[11]
% ) is necessary for extended run times. That fixed $M$ alters accretion near the black hole is evident in Figures 1 and 2. The absence of magnetic fluctuations at $R < R*$ in Figure 2 leaves only Bondi accretion, where R* is both the radius where $v_\mathrm{BONDI} = v_\mathrm{MRI}$  and  also the radius where $v_\mathrm{BONDI}(R*)t = R* = (t/t_g)^{1/3}R_g$, the condition to drop $\partial\rho/\partial t$ giving quasi-static Bondi accretion by Equation (2a). The absence of MRI at $R < R*$ explains the peaking of $B$ at $R = R_g$ in Figure 2, due to flux advection $\mathbf{v} \times \mathbf{B}$ when $\mathbf{D}_\mathrm{MRI} = 0$ in Equation (2d) (giving $B \approx 1/R$ in the figure). Our model in \cite{Fowler2019} %[1] 
% gives a different solution still dominated by MRI at R0 but by Jeans gravitational collapse at $R < R_0$, not accessible by KORAL, occurring at wavelengths avoiding pair annihilation in the hot primordial environment. See [\cite{Fowler2023}%1
% , Section 4] for details.

%While the main feature of our model –- outward propagation of the accretion radius by MRI -- has been verified (in Figure 1), the fact that KORAL only includes gravity by a fixed mass M neglects local Jeans instability [\cite{Fowler2023}, Sect. 4]. The MRI mechanism is little affected, the gravitational potential $\propto MG$ being only additive to rotation $\Omega^2 \approx (MG/R^3)$ in the MRI dispersion relation (%[7]
%\cite{Balbus1998}, Eq. (111)).  This explains 
%stability below $R \approx 100R_g$  in Figure 1, due to fall off of rotation $\Omega$ in Figure 3 [5, App. B]. It also explains the peak in $B$ at $R \to R_g$ in Figure 2, the main effects in the stable region $R < 100 R_g$ being quasi-cylindrical flux compression giving $B \propto 1/R$ in Figure 2 and stable Bondi accretion $v \propto 1/R^2$ in Figure 1.  . Jeans instability should merely break up the stable Bondi flow into  self-confined shells with little effect on the accretion rate determined  by MRI at $R = R_0$. See [1, Sect. 4] for details.  


%%%%%%%%%%%%%%%%%%%%%%%%%%%%%%%%%%%%%%%%%%%%%%%%%%%%%%%

% \begin{table*}[b]
% \caption{\label{tab:table4}%
% %Numbers in columns Three--Five are aligned with the ``d'' column specifier 
% %(requires the \texttt{dcolumn} package). 
% %Non-numeric entries (those entries without a ``.'') in a ``d'' column are aligned on the decimal point. 
% %Use the ``D'' specifier for more complex layouts. 
% Primordial MRI jet parameters, for typical $M=10^{16}$g, $t=1$s}
% \begin{ruledtabular*}
% \begin{tabular}{ccddd}
% Quantity& &
% \multicolumn{1}{c}{\textrm{Value}}&
% \multicolumn{1}{c}{\textrm{.}}&
% \multicolumn{1}{c}{\textrm{.}}\\
% %\mbox{Three}&\mbox{Four}&\mbox{Five}\\
% \hline
% Primordial Temperature&two&\mbox{three}&\mbox{four}&\mbox{five}\\
% Primordial Density&2& 2.77234 & 45672. & 0.69 \\
% MRI magnetic field, poloiodal&two&\mbox{three}&\mbox{four}&\mbox{five}\\
% MRI magnetic field, toroidal&2& 2.77234 & 45672. & 0.69 \\
% Black hole radius $R_g$, transit time $t_g$	
% MRI jet Radius a &2& 2.77234 & 45672. & 0.69 \\
% 	MRI jet Length L $R_g$, transit time $t_g$	 \footnote{Some tables require footnotes.}
%   &C\footnote{Some tables need more than one footnote.}
%   & 12537.64 & 37.66345 & 86.37 \\
%   Rotation $\Omega$
% \end{tabular}
% \end{ruledtabular*}
% \end{table*}


% Insert an image using either the \texttt{graphics} or
% \texttt{graphix} packages, which define the \verb+\includegraphics{#1}+ command.
% (The two packages differ in respect of the optional arguments 
% used to specify the orientation, scaling, and translation of the image.) 
% To create an alignment, use the \texttt{tabular} environment. 

% The best place to locate the \texttt{figure} or \texttt{table} environment
% is immediately following its first reference in text; this sample document
% illustrates this practice for Fig.~\ref{fig:epsart}, which
% shows a figure that is small enough to fit in a single column. 

% In exceptional cases, you will need to move the float earlier in the document, as was done
% with Table~\ref{tab:table3}: \LaTeX's float placement algorithms need to know
% about a full-page-width float earlier. 

% Fig.~\ref{fig:wide}
% has content that is too wide for a single column,
% so the \texttt{figure*} environment has been used.%






% \begin{table*}[b]
% \caption{\label{tab:table4}%
% %Numbers in columns Three--Five are aligned with the ``d'' column specifier 
% %(requires the \texttt{dcolumn} package). 
% %Non-numeric entries (those entries without a ``.'') in a ``d'' column are aligned on the decimal point. 
% %Use the ``D'' specifier for more complex layouts. 
% Primordial MRI jet parameters, for typical $M=10^{16}$g, $t=1$s}
% \begin{ruledtabular*}
% \begin{tabular}{ccddd}
% Quantity& &
% \multicolumn{1}{c}{\textrm{Value}}&
% \multicolumn{1}{c}{\textrm{.}}&
% \multicolumn{1}{c}{\textrm{.}}\\
% %\mbox{Three}&\mbox{Four}&\mbox{Five}\\
% \hline
% Primordial Temperature&two&\mbox{three}&\mbox{four}&\mbox{five}\\
% Primordial Density&2& 2.77234 & 45672. & 0.69 \\
% MRI magnetic field, poloiodal&two&\mbox{three}&\mbox{four}&\mbox{five}\\
% MRI magnetic field, toroidal&2& 2.77234 & 45672. & 0.69 \\
% Black hole radius $R_g$, transit time $t_g$	
% MRI jet Radius a &2& 2.77234 & 45672. & 0.69 \\
% 	MRI jet Length L $R_g$, transit time $t_g$	 \footnote{Some tables require footnotes.}
%   &C\footnote{Some tables need more than one footnote.}
%   & 12537.64 & 37.66345 & 86.37 \\
%   Rotation $\Omega$
% \end{tabular}
% \end{ruledtabular*}
% \end{table*}






%%%%%%%%%%%%%%%%%%%%%%%%%%%%%%%%%%%%%%%%%%%%%%
% \subsection{4a. Main Result}


% % The content of a table is typically a \texttt{tabular} environment, 
% % giving rows of type in aligned columns. 
% % Column entries separated by \verb+&+'s, and 
% % each row ends with \textbackslash\textbackslash. 
% % The required argument for the \texttt{tabular} environment
% % specifies how data are aligned in the columns. 
% % For instance, entries may be centered, left-justified, right-justified, aligned on a decimal
% % point. 
% % Extra column-spacing may be be specified as well, 
% % although REV\TeX~4 sets this spacing so that the columns fill the width of the
% % table. Horizontal rules are typeset using the \verb+\hline+
% % command. The doubled (or Scotch) rules that appear at the top and
% % bottom of a table can be achieved enclosing the \texttt{tabular}
% % environment within a \texttt{ruledtabular} environment. Rows whose
% % columns span multiple columns can be typeset using the
% % \verb+\multicolumn{#1}{#2}{#3}+ command (for example, see the first
% % row of Table~\ref{tab:table3}).%

% %%%%%%%%%%%%%%%%%%%%%%%%%%%%%%%%%%%
% %8/17/23 Main Result
% %%%%%%%%%%%%%%%%%%%%%%%%%%%%%%%%%%%
% % The main result is shown in Figure 1, where spherically-averaged $v_R$ calculated by KORAL is compared with quasi-static Bondi accretion, $<v_\mathrm{R-BONDI}> = c_S (R_g/R)^2 $ in [12, Sect. 2.5]\cite{Frank2002}, with  Schwarzschild radius $R_g = (MG/c_S^2)$. The accretion velocity before and after the MRI wave front is residual Bondi accretion extending to all R < ct. 
% % Note that finally $v_\mathrm{R-KORAL}$ becomes larger than $<v_\mathrm{R-BONDI}>$, beginning around $R = 100R_g$ , yielding a velocity wave front around 2300 $R_g$ at a time $t = 106 t_g$ with $t_g = R_g/c_S$ . This shows that the accretion radius is ultimately driven by MRI -- the main feature of our model.                   	 
% % The MRI wave front corresponds to $R_0 \propto t^{2/3}$ in Equation (1d). The magnitude is less because $R_0$ is derived from the actual accretion rate $dM_\mathrm{accr}/dt$ for accreted mass $M_\mathrm{accr}$, while $M$ is held constant in KORAL. The corrected formula fits the wave front at $R_0 = 2000 R_g$, $t =106 t_g$ in Figure 1 by replacing M by the accreted mass $M_\mathrm{accr}$, giving $R0 \propto (M_\mathrm{accr}/M)^{2/3}$ for $M_\mathrm{accr}/M \approx 10^{-4}$ with $\xi_R/R = 0.1$ fitting dark matter abundance in Equation (1e). As noted in [1], Jeans-unstable oscillations set in at 
% % $\frac{\partial \rho}{\partial t} > - \mathbf{\triangledown}\cdot(\rho\mathbf{v})$, or $R > v_\mathrm{MRI}t$, giving the observed onset $R/R_g > [(\xi_R/R)^2(t/t_g)]^{2/3} \approx 400$.
% % That it is MRI magnetic turbulence that accounts for propagation of R0 is confirmed by magnetic fluctuations in Figure 2, giving the expected $\xi/R \propto B_1/B \approx 0.1$ correlated with MRI velocity fluctuations in Figure 1. That R0 represents MRI propagation to the magnetic O-point is confirmed by the magnetic field plot in 
% % Figure 3, showing the point $R = R_0, z = 0$ where $B_z = 0$.

% %%%%%%%%%%%%%%%%%%%%%%%%%%%%
% % B-FIELD FIGURE PLACEMENT 8-17-23
% % \begin{figure}[b]
% % \includegraphics[width=10cm]{B.png}% Here is how to import EPS art
% % \caption{\label{fig:BField} B-Field}
% % \end{figure}


% The main result is shown in Figure 1, where spherically-averaged $v_R$ calculated by KORAL is compared with quasi-static Bondi accretion,  $<v_\mathrm{R-BONDI}> = c_S (R_g/R)^2 $ in [\cite{Frank2002}, %12,
% Sect. 2.5], with  Schwarzschild radius $R_g = (MG/c_S^2)$. Note that  
% $v_\mathrm{R-KORAL}$  becomes larger than $<v_\mathrm{R-BONDI}>$, beginning around $R = 100R_g$  where flow becomes correlated with magnetic fluctuations. We identify [magnetic fluctuations in Figure 2]
% \textcolor{red}{fluctuations in the magnetic field of Fig. 2a}
% and growth [of the magnetic field] \textcolor{red}{these magnetic field fluctuations }in Figure 3 \textcolor{red}{2b} with an MRI dynamo propagating the magnetic field \cite{Fowler2019,Balbus1998}. %[6, 7]. 
% The MRI wave front corresponds to  $R_0 \propto t^{2/3}$  in Equation (1d).  Another benchmark is the onset of Jeans-unstable oscillations at $\frac{\partial \rho}{\partial t} > - \mathbf{\triangledown}\cdot(\rho\mathbf{v})$, or $R > v_\mathrm{MRI}t$, giving the observed onset $R/R_g > [5(t/t_g)]^{2/3} \approx 200-400$ at $t = 10^6t_g$ . This gives the same $(\xi_R/R) = 0.05 – 0.1$ that explains dark matter abundance. That this $$(\xi/R)$$ is due to magnetic fluctuations is shown by $v_R = <v_{1R} \times B_{1z} /B_z > \approx (v_K/k_RR)(B_{1z}/B_z)^2$ \cite{Fowler2019}, %[6], 
% giving $v_R$ in Equation (1c) for magnetic fluctuations in Figure\textcolor{red}{s} 2 \textcolor{red}{and 3 [OR 2a and 2b]}. That this $(\xi_R/R)$ gives $R_0$ smaller than that \textcolor{red}{of} Equation (1d) is because that formula is derived from the actual accretion rate $dM_\mathrm{accr}/dt$ for accreted mass $M_\mathrm{accr}$, while $M$ is held constant in KORAL. The corrected formula fits the observed wave front at $R_0 = 2000 R_g, t =10^6 t_g$ in Figure 1 by replacing $M$ by the accreted mass $M_\mathrm{accr}$, giving $R_0 \approx (M_\mathrm{accr}/M)^{2/3}$ for $M_\mathrm{accr}/M \approx 10^{-4}$ with $\xi_R/R = 0.1$. 

% \textcolor{red}{Self gravity not included in Koral. Is this a problem for Ken's model?}

% 	\subsection{ 4b. Accretion Flow}
  
% 	As noted in \cite{Fowler2023}, %[1],
%  while MRI initiates accretion, it is gravitational collapse by Jeans instability in our wave equation that sustains accretion all the way to the black hole. That accretion is sustained is shown in Figure 4, showing $dM/dt$ in Equation (1a) obtained by integrating Equation (2a) over the accretion disk volume:


% \begin{subequations}
% \label{eq:whole}
% \begin{eqnarray}
% \int_{R_g}^{R_0} 4\pi R^2 \partial \rho  /\partial t = \int_{R_g}^{R_0} 4\pi R^2 \left[ -R^{-2} \frac{\partial(R^2\rho v_R)}{\partial R} \right]
% % {\cal M}=&&ig_Z^2(4E_1E_2)^{1/2}(l_i^2)^{-1}
% % (g_{\sigma_2}^e)^2\chi_{-\sigma_2}(p_2)\nonumber\\
% % &&\times
% % [\epsilon_i]_{\sigma_1}\chi_{\sigma_1}(p_1).
% \label{Accretionsubeq:0}
% \end{eqnarray}
% \begin{equation}
% = 4\pi R_0^2v_R(R_0)\rho_\mathrm{AMB} -  \frac{dM_\mathrm{BH}}{dt},\label{Accretionsubeq:1}\end{equation}
% \begin{equation}
% \frac{dM_\mathrm{BH}}{dt} = \frac{dM}{dt} =  4\pi R_0^2v_R(R_0)\rho_\mathrm{AMB} -  \frac{dM_\mathrm{Disk}}{dt}
% ,\label{Accretionsubeq:2}
% \end{equation}
% % \begin{equation}
% % \frac{\partial \mathbf{B}}{\partial t} + \mathbf{\triangledown}\times \left(\frac{1}{c} \mathbf{v} \times \mathbf{B} \right)= \mathbf{\triangledown}\times \mathbf{D};  \mathbf{D} = -\frac{1}{c}<\mathbf{v}_1\times\mathbf{B}_1>
% % ,\label{Accretionsubeq:3}
% % \end{equation}
% % \begin{equation}
% % v_B \approx \lbrace (k_\mathrm{R}\xi_\mathrm{R})^2_\mathrm{MRI}v_\mathrm{K} + [1-(R/R_0)] (k_\mathrm{R}\xi_\mathrm{R})^2_\mathrm{Jeans} c_s\rbrace,
% % \label{Accretionsubeq:4}\end{equation}
% \end{subequations}


% Here $\frac{dM_\mathrm{BH}}{dt}$ is $dM/dt$ in Equation (1a) and $\frac{dM_\mathrm{Disk}}{dt}$  is the left hand side of Equation (3a) representing any accumulation or depletion of mass inside the accretion disk, however distributed. Oscillations in $dM/dt$ are mainly due to Jeans instability.   



% \subsection{4c. Simulating MRI in KORAL}

% To model MRI in the primordial atmosphere, we note that unstable exponential growth from the primordial magnetic field order 1\textcolor{red}{$\times10^{10}$} gauss is sufficient to initiate MRI %[13]
% \cite{Enqvist1998}, while there is no source of angular momentum in Equation (2b) (as noted in [\cite{Balbus1998}, Eq. (29)]).  In \cite{Fowler2023}, %[1], 
% we suggest that rotation is due to the Peebles mechanism whereby non-symmetric seed masses can set each other in counter-rotation %[14].
% \cite{Peebles1969}.
% In KORAL, we approximate these conditions by initiating MRI by introducing Keplerian $\Omega$ everywhere outside of $R = 10R_g$; and a poloidal ($B_z$) magnetic field nearly uniform \textcolor{red}{inside} $R < \_\_R_g$ (the red curve in Figure 2).  

% \subsection{4d. The KORAL Code}
% Astrophysical applications of accretion by MRI are challenging, spanning many time and space scales [6]. As already noted, the KORAL code sacrifices detail near the black hole by setting the black hole mass M constant; and our longest runs were two-dimensional, sufficient to capture MRI onset (only coupling $v_R$ and cylindrical $B_z$ \cite{Balbus1998}%[7]
% ) but not the full development in time involving $B_\phi$ (\cite{Fowl2019}, %[15],
% Appen. A.3). 
%%%%%%%%%%%%%%%%%%%%%%%%%%%%%%%%%%%%%%%%%%%%%%%


%\textcolor{red}{Code description: Brandon} 

\section{4. Dark Matter Radiation}


	Fitting Hawking radiation by dark matter to background gamma radiation is discussed in [\cite{Fowler2023}, Sect. 6] and [\cite{Carr2020}, Figure 1]. The measured gamma ray energy spectrum is $g(E) = [0.1(10\mathrm{keV}/4\pi E)^2]$ %[16] 
%\cite{Fow2019},  
\cite{Trombka1983}, 
fitted by $E_g(E) = \int_{M_1}^{M_2} dM f(M)(R_g/<R>)^2 P_\mathrm{HAWKING}$ with black  hole density $f(M)$ and typical black hole radius $R_g$ at a distance $<R>$. In \cite{Fowler2023} %[1] 
we note that $(M_\mathrm{DM}/<R>^2) = 0.03-0.05$ whether the background gamma signal is dominated by the Milky Way or by the  Universe as a whole (with $<R>$ = 3 x $10^{23}$ cm and $M_\mathrm{DM}$ = 3 x $10^{45}$g for the Milky Way and $<R> = 4.4 \times 10^{28}$ cm  and $M_\mathrm{DM} = 10^{56}$g for the visible Universe). Using this, radiation $\propto T^3$, $T_\mathrm{HAWKING} \propto M^{-1}$ and Planck weighting gives an integral equation for $f(M)$ with solution $f = 
1.3 \times 10^{-53} M_\mathrm{DM}M$. This gives [1,Eq.(4c)]:
\begin{equation}
     \int_{M_1}^{M_2} dM M f(M) = (M_2/4 \times 10^{17})^3 M_{DM}
\end{equation}
Thus all dark matter would be included in $f(M)$ if $M_2$ = 4 x $10^{17}$\textcolor{black}{g}.

%the result in Section 1, showing that $\int_{M_1}^{M_2} dM  f(M) = (M_2 /0.4 \times 10^{18})^3  M_\mathrm{DM} = M_\mathrm{DM}$ if $M_2 \approx 10^{18}$ g.%, the upper end of our mass range. 
%Black holes created by Bondi accretion would have masses greater than the Sun \cite{Carr2020} %[2].
%That MRI accretion is faster causes density to 
%As  noted in Section 3, the fact that MRI flow is faster than Bondi causes density to pile up 
%pile up 
%at the black hole, giving $M \approx M_\mathrm{BONDI} (\rho_\mathrm{AMB}/\rho) \approx 10^{15-18}$ g.  %fitting background gamma radiation.


% \section{6. Magnetic Accretion Physics}

% Accretion in a strong magnetic field requires some mechanism of mass flow across magnetic flux surfaces \cite{Frank2002}. %[12].
% A breakthrough occurred when Balbus and Hawley concluded that MRI turbulence produces magnetic jets with currents transporting mass across field lines \cite{Balbus1998}. %[7]. 

% Magnetic jets are reviewed in %[17] 
% \cite{Blandford2019}, interpreted as hyper-resistive flow in \cite{Colgate2014,Colgate2015,Fowler2019}.
% %[4, 5, 6]. 
% Additional magnetic accretion physics is discussed in [\cite{Fowler2023}, Section 6]. Highlights include gravitational collapse and a special feature of primordial jets whereby a short-circuit current in the accretion disk dominates accretion early in time [\cite{Fowler2019}, %Appendix
% App. A]. That accretion should ultimately evolve to a state of gravitational collapse (an extension of seed mass formation) is discussed in [\cite{Fowler2023}, Section 4]. All such dynamical features are contained in our wave Equation (2c). 

% The primordial magnetic field needed to produce MRI could have been created by Biermann battery action due to pressure fluctuations in the ordinary matter content of primordial plasma (though positronium contributions
% tend to cancel) %[18]
% \cite{Kulsrud1997}. Initiating MRI in KORAL by adding rotation correctly represents Peebles seed mass rotation as $\int dR \frac{ \partial}{\partial t} (\rho R(MG/R)^{1/2}) = \alpha (M^2G/R)$ for elliptic distortion $\alpha$ \cite{Peebles1969}. Putting in numbers shows that a seed mass $M >10^6$ g suffices.


%\section{%7. Summary
%$C*$ vs. $\xi/R$}


%GRMHD simulations on KORAL confirm our conclusion in \cite{Fowler2023} %[1] 
%that accretion of rotating primordial positronium in a magnetic field would have greatly extended the accretion 
% radius beyond the %Schwarzschild
% \textcolor{red}{gravitational} radius $R_g$ characterizing Bondi accretion.  Extending the radius yields 
%an accretion %rate that explains dark matter as small 
%rate accounting for dark matter as 
%black holes of mass $10^{15-17}$g identified by their Hawking radiation as observed background gamma radiation. In natural units of $R_g=MG/c^2$  for length, 
%thermal speed $c_S$ for velocity and 
%$t_g = (R_g/c)$ for time, primordial accretion radii are of order $10^{12}R_g$  (meters) for times %of order $10^{22}$ $t_g$ (1 s). But the expected phenomena are already evident within KORAL capability %($10^5$ $R_g$, $10^6$ $t_g$, 1 day run time).  
%(Fig. 1.)

% Accretion flow is initiated at a rate about 100 times slower than free-fall, finally growing to speed $c_S$ near the black hole. This initial accretion velocity, driven by MRI, is all that is required to calculate dark matter abundance, yielding the observed value order 
% 6 times ordinary matter in Equation (1e).  

%Our analytical model allows us to characterize $\%$ dark matter by $C^*$ in the black hole accretion velocity $vR(R0) = C^*(GM/ R_0)^{1/2}$, at the MRI accretion range $R_0$ (Equations (1b, 1e)). Dark matter abundance is sensitive to numbers, changes in $C^*$ giving an accretion fraction $f^* \propto (C^*)^2$. Extrapolating from fully-developed MRI in AGN’s gives $C* = 0.01 \to 100\%$. A value $C^* = 0.03 \to 10\%$ characterizes the longest run times for KORAL GMHD simulations ????.  Identifying these black holes with the gamma ray background gives a mass range $M = 10^{15-17}$ g, also sensitive to details at X-ray frequencies (see Section 1 citing Carr %[2]
%\cite{Carr2020}, Figure 1). We conclude that Big Bang positronium probably did produce black holes by MRI, contributing  as much as 10$\%$ of dark matter today.

 % KORAL simulations account for at least $1\%$ of the dark matter observed today. The key parameter is the accretion velocity $v_R(R_0)$ at a radius $R_0$ serving as the useful boundary of accretion (Equations (1c) and (1d)). Up to $100\%$  was estimated in Section 2 for $v_R(R_0)$ only 10 times faster, based on AGN plasma turbulence levels applied to Equation (2e). KORAL evaluates turbulent transport directly as a measured velocity $v_R$ in the code. Whether longer runs would give higher $v_R(R_0)$ is unknown. ????  An upper bound is obtained by calculating M* for the KORAL velocity $v_R(R_0) = C^*v_K(R_0)$ as a boundary condition, then  adjusting $C^*$ to flatten $M^*$ as it likely would at infinite $t$.  This gives an upper bound C* $\_\_\_\_\_$ yielding  $\_\_\_\_\_\%$ of the dark matter ?????.

\section{5. Jeans Instability} 

%KORAL with constant $M$ does not include Jeans instability, nor pair creation and annihilation. For the short times of KORAL runs, 
KORAL with constant $M$ includes a modified form of Jeans gravitational instability, manifest as oscillations of $\dot{M}$ in Figure \ref{MDot}. %4.
KORAL omits pair creation and annihilation. For the short times of KORAL runs, 
primordial conditions balancing creation and annihilation would prevail \cite{Fowler2023},\cite{Weinberg1993}, while neglecting Jeans instability has little effect on MRI onset and development of principal interest in this paper. That Jeans instability ultimately must play a major role in MRI accretion follows from $v_{MRI}(R_0)t << R_0$, showing that, though MRI is required to extend the accretion range $R_0$, something else has to accelerate accretion $>v_{MRI} $ at $R<R_0$. In \cite{Fowler2023}, we add gravity and pressure to the MRI dispersion relation in (\cite{Balbus1998} Eq. (111)) giving an accretion velocity given by:
\begin{subequations}
\begin{align}  
 v_R(R) = \Sigma_k{<[(\omega/k)(k\xi)^2]_{MRI}+[(\omega/k)(k\xi)^2]_{\textcolor{black}{\mathrm{JEANS}}}>} \notag
\end{align}
 \begin{align}
\approx Av_K(R) + [1-(R/R_0)][\Sigma_k <(k\xi)^2_\mathrm{JEANS}>c]    
 \end{align}
\begin{align}
(R_1/R_0) = 0.7 A^{2/3}(R_g/ct)^{1/6} 
\end{align}
\end{subequations}
The radius $R_1$ in Equation (4b) anticipates a transition in dynamics where $v_R$ first equals speed $c$, given by $cR_1^2$  = $v_{MRI}(R_0)R_0^2$. We assume Jeans frequencies $\omega$ and wave numbers k adjust to sustain accretion. 

Eventually, accretion by Equation (4a) would lead to $v_R \rightarrow c$ at $R_1$. In \cite{Fowler2023}, we predict that mass density all flowing at speed $c$ at $R < R_1$ would pile up, giving rise to Jeans gravitational instability \cite{Toomre1964}. And instability would cause mass flow to break up into self-confined shells in local pressure balance, too thin to annihilate, 
requiring wavenumbers $kR_g > 10^{-4}$ [\cite{Fowler2023}, Eq. (3c)]. Accretion at the black hole gives $M/t\propto \rho(R_g)R_g^2\propto\rho(R_g)M^2$, hence $M \propto 1/\rho$ with $\rho \approx \rho_{AMB}$ for Bondi accretion.
Then pileup of average density $<\rho(R_g)>$ at the black hole increases the accretion rate above the Bondi rate, yielding smaller black hole masses  $M\approx(\rho_{AMB}/<\rho(R_g)>)M_{BONDI}.
$ This result determines the mass range $10^{15}g<M<10^{18}g$ in Equation (3).
\section{6. Summary}

Accretion in a strong magnetic field requires some mechanism of mass flow across magnetic flux surfaces \cite{Frank2002}. A breakthrough occurred when Balbus and Hawley concluded that MRI turbulence produces magnetic jets with currents transporting mass across field lines \cite{Balbus1998}. This paper has presented GRMHD simulations using the KORAL code showing that accretion by MRI could have produced black holes contributing to dark matter. Evidence for the required primordial magnetic field $\approx$ 1 gauss is discussed in \cite{Enqvist1998}, probably created by Biermann Battery action due to primordial pressure fluctuations \cite{Kulsrud1997}. The Peebles mechanism could have produced the required rotation as asymmetric counter-rotating seed masses \cite{Peebles1969}. Carr and K\"uhnel \cite{Carr2020} focus on limitations on allowed masses due to constraints from astrophysical phenomena. Their constraint due to accretion refers to Bondi accretion with zero rotation. Our paper finds a second accretion window, accessed by MRI. Qualitative evidence for this from KORAL simulations is presented in Section 3b of this paper. 

While KORAL simulations exhibit MRI accretion that could produce primordial black holes, how much these black holes contribute to dark matter abundance remains uncertain, ranging 1$\%$ to 100$\%$ .
The evidence for 1$\%$  comes from an approximate fitting $v_R$ in Figure \ref{vrsmv3}  %1
to $v_R$ = A$v_K$ giving A $\approx$ 0.001 in Equation (2); also from the lowest Carr-K\"uhnel mass window in (\cite{Carr2020}, Fig. 1), giving 1$\%$  from Equation (3).The evidence for 100$\%$  is inferred from our analysis of fully-developed AGN structures in (\cite{Colgate2015}, App.B), versus early development in our KORAL simulations that (scaling with $M$) also apply to AGNs.  
%The evidence for 100$\%$  is inferred from fully-developed MRI turbulence levels explaining jet dimensions in AGNs ([\cite{Colgate2015}], Sect. 6.1). 

Results are sensitive to numbers: A = 0.001 $\rightarrow$ 0.003 in Equation (2) would give 10$\%$  of dark matter. And $M_2$ = $10^{17}$ g $\rightarrow 2 \times 10^{17}$ g in Equation (3) would also give 10$\%$ , while 4 x $10^{17}$ g gives 100$\%$. 
\newline

%Accretion in a magnetic field requires some means of transporting mass flow across magnetic flux surfaces \cite{Frank2002}.  A breakthrough occurred 
%with the …
%when Balbus and Hawley concluded that MRI turbulence produces magnetic jets with currents transporting mass across field lines
%across field lines 
%[7, 17 old numbering]
%\cite{Balbus1998,Blandford2019}. This paper has presented GRMHD simulations using the KORAL code showing that accretion by MRI could have produced black holes contributing to dark matter. Evidence for the required primordial magnetic field $\approx$ 1 gauss is discussed in \cite{Enqvist1998}, %[13 old numbering], 
 %The Peebles mechanism could have produced the required rotation as asymmetric counter-rotating seed masses  %[14 old numbering]. 
%\cite{Peebles1969}.

% Carr and Kuhnel %[2]
%\cite{Carr2020}, focus on
%limitations on allowed masses due to constraints from astrophysical data. Their constraint due to accretion refers to Bondi accretion with zero rotation. Our paper finds a second accretion window, accessed by MRI. Qualitative evidence for this from KORAL simulations is presented in Section 3b of this paper. 


%Evidence ranging from 1$\%$ to 100$\%$ suggests the MRI contribution is 10$\%$ or more. As noted in Section 2, 100$\%$ if inferred from MRI jet dimensions for fully-developed MRI in AGN’s ($\xi/R \approx 0.1$). By Equation (1e), the percentage of dark matter is 100$\% \times (3 \times 10^5 f^*)$ with $f^* \propto  v_\mathrm{MRI} = C^*v_K$ where $C^* = (\xi/R)^2$ corresponds to magnetic fluctuations in the theoretical “quasi-linear” formula in Equation (1c),  while $C^*$ fitted to code results reflects actual turbulence levels in the code. In the time available in KORAL runs, fitting $C^*$ to $v_R$ in Figure 1 gives $C^* = 0.001$, while $C^* = 0.01$ would give $100\% (f^* \propto C^{*2})$.

%The estimate of  1$\%$ from Figure 1 is also the smallest value if our black holes correspond to Carr-Kuhnel’s lowest mass window interpreting  background gamma radiation as Hawkng radiation (\cite{Carr2020}, Fig. 1, $10^{15}$ g $< M <$  $10^{17}$ g). However this case, too, is very sensitive to numbers, with $\% \propto  M^3$.


%We conclude that MRI should be taken into account in any future estimates of dark matter abundance. The main physics issue remains how accretion persists all the way to the black hole. In \cite{Fowler2019}, we found that the problem arises at $R = R_1$ in Equation (2f), where $v_R$ first equals $c$. At $R > R_1$, the mass continuity Equation (1a) can reach steady state with density $\rho$ and temperature $T$ at their ambient values, giving $v_RR^2 =$ constant. But if $v = c$, at $R < R_1$, $\partial \rho /\partial t + \triangledown\dot\rho\mathbf{v} = 0$ predicts growth of $\rho$ in $R$ or $t$ or both. If $\rho$ grows, how does the black hole confine the pressure? In \cite{Fowler2023}, we suggest  that the answer is, first, Jeans instability in Equation (2c), due to self-gravity omitted in KORAL; and, second, self-adjustment of wavenumbers $k$ to avoid pair annihilation. We found that self-gravity should create self-confined mass shells too thin to annihilate, with $kR_g  > 10^{-4}$. See \cite{Fowler2023} for details. 


% Tables~\ref{tab:table1}, \ref{tab:table3}, \ref{tab:table4}, and \ref{tab:table2}%
% \begin{table}[b]
% \caption{\label{tab:table2}
% A table with numerous columns that still fits into a single column. 
% Here, several entries share the same footnote. 
% Inspect the \LaTeX\ input for this table to see exactly how it is done.}
% \begin{ruledtabular}
% \begin{tabular}{cccccccc}
%  &$r_c$ (\AA)&$r_0$ (\AA)&$\kappa r_0$&
%  &$r_c$ (\AA) &$r_0$ (\AA)&$\kappa r_0$\\
% \hline
% Cu& 0.800 & 14.10 & 2.550 &Sn\footnotemark[1]
% & 0.680 & 1.870 & 3.700 \\
% Ag& 0.990 & 15.90 & 2.710 &Pb\footnotemark[2]
% & 0.450 & 1.930 & 3.760 \\
% Au& 1.150 & 15.90 & 2.710 &Ca\footnotemark[3]
% & 0.750 & 2.170 & 3.560 \\
% Mg& 0.490 & 17.60 & 3.200 &Sr\footnotemark[4]
% & 0.900 & 2.370 & 3.720 \\
% Zn& 0.300 & 15.20 & 2.970 &Li\footnotemark[2]
% & 0.380 & 1.730 & 2.830 \\
% Cd& 0.530 & 17.10 & 3.160 &Na\footnotemark[5]
% & 0.760 & 2.110 & 3.120 \\
% Hg& 0.550 & 17.80 & 3.220 &K\footnotemark[5]
% &  1.120 & 2.620 & 3.480 \\
% Al& 0.230 & 15.80 & 3.240 &Rb\footnotemark[3]
% & 1.330 & 2.800 & 3.590 \\
% Ga& 0.310 & 16.70 & 3.330 &Cs\footnotemark[4]
% & 1.420 & 3.030 & 3.740 \\
% In& 0.460 & 18.40 & 3.500 &Ba\footnotemark[5]
% & 0.960 & 2.460 & 3.780 \\
% Tl& 0.480 & 18.90 & 3.550 & & & & \\
% \end{tabular}
% \end{ruledtabular}
% \footnotetext[1]{Here's the first, from Ref.~\onlinecite{feyn54}.}
% \footnotetext[2]{Here's the second.}
% \footnotetext[3]{Here's the third.}
% \footnotetext[4]{Here's the fourth.}
% \footnotetext[5]{And etc.}
% \end{table}
% show various effects.
% A table that fits in a single column employs the \texttt{table}
% environment. 
% Table~\ref{tab:table3} is a wide table, set with the \texttt{table*} environment. 
% Long tables may need to break across pages. 
% The most straightforward way to accomplish this is to specify
% the \verb+[H]+ float placement on the \texttt{table} or
% \texttt{table*} environment. 
% However, the \LaTeXe\ package \texttt{longtable} allows headers and footers to be specified for each page of the table. 
% A simple example of the use of \texttt{longtable} can be found
% in the file \texttt{summary.tex} that is included with the REV\TeX~4
% distribution.

% There are two methods for setting footnotes within a table (these
% footnotes will be displayed directly below the table rather than at
% the bottom of the page or in the bibliography). The easiest
% and preferred method is just to use the \verb+\footnote{#1}+
% command. This will automatically enumerate the footnotes with
% lowercase roman letters. However, it is sometimes necessary to have
% multiple entries in the table share the same footnote. In this case,
% there is no choice but to manually create the footnotes using
% \verb+\footnotemark[#1]+ and \verb+\footnotetext[#1]{#2}+.
% \texttt{\#1} is a numeric value. Each time the same value for
% \texttt{\#1} is used, the same mark is produced in the table. The
% \verb+\footnotetext[#1]{#2}+ commands are placed after the \texttt{tabular}
% environment. Examine the \LaTeX\ source and output for
% Tables~\ref{tab:table1} and \ref{tab:table2}
% for examples.

% Video~\ref{vid:PRSTPER.4.010101} 
% illustrates several features new with REV\TeX4.2,
% starting with the \texttt{video} environment, which is in the same category with
% \texttt{figure} and \texttt{table}.%
% \begin{video}
% \href{http://prst-per.aps.org/multimedia/PRSTPER/v4/i1/e010101/e010101_vid1a.mpg}{\includegraphics{vid_1a}}%
%  \quad
% \href{http://prst-per.aps.org/multimedia/PRSTPER/v4/i1/e010101/e010101_vid1b.mpg}{\includegraphics{vid_1b}}
%  \setfloatlink{http://link.aps.org/multimedia/PRSTPER/v4/i1/e010101}%
%  \caption{\label{vid:PRSTPER.4.010101}%
%   Students explain their initial idea about Newton's third law to a teaching assistant. 
%   Clip (a): same force.
%   Clip (b): move backwards.
%  }%
% \end{video}
% The \verb+\setfloatlink+ command causes the title of the video to be a hyperlink to the
% indicated URL; it may be used with any environment that takes the \verb+\caption+
% command.
% The \verb+\href+ command has the same significance as it does in the context of
% the \texttt{hyperref} package: the second argument is a piece of text to be 
% typeset in your document; the first is its hyperlink, a URL.

% \textit{Physical Review} style requires that the initial citation of
% figures or tables be in numerical order in text, so don't cite
% Fig.~\ref{fig:wide} until Fig.~\ref{fig:epsart} has been cited.


\section{Acknowledgements}
\begin{acknowledgments}
\textcolor{black}{We wish to acknowledge  Christopher McKee and Hui Li for useful discussions. %; and we thank Luke Fehlis for editing the manuscript. 
This work was supported by a grant from the Simons Foundation (00001470, BC, RA). 
%a
%nd Simons Collaboration for Extreme Electrodynamics of Compact Sources (SCCEECS) 
Richard Anantua is supported by the Oak Ridge Associated Universities Powe Award.
}
%\dots.
\end{acknowledgments}

\appendix
%%%%%%%%%%%%%%%%%%%%%%%%%%%%%%%%%%%%%%%
% \section{Appendixes}

% To start the appendixes, use the \verb+\appendix+ command.
% This signals that all following section commands refer to appendixes
% instead of regular sections. Therefore, the \verb+\appendix+ command
% should be used only once---to setup the section commands to act as
% appendixes. Thereafter normal section commands are used. The heading
% for a section can be left empty. For example,
% \begin{verbatim}
% \appendix
% \section{}
% \end{verbatim}
% will produce an appendix heading that says ``APPENDIX A'' and
% \begin{verbatim}
% \appendix
% \section{Background}
% \end{verbatim}
% will produce an appendix heading that says ``APPENDIX A: BACKGROUND''
% (note that the colon is set automatically).

% If there is only one appendix, then the letter ``A'' should not
% appear. This is suppressed by using the star version of the appendix
% command (\verb+\appendix*+ in the place of \verb+\appendix+).
%%%%%%%%%%%%%%%%%%%%%%%%%%%%%%%%%%%%%%%


% \section{A little more on appendixes}

% Observe that this appendix was started by using
% \begin{verbatim}
% \section{A little more on appendixes}
% \end{verbatim}

% Note the equation number in an appendix:
% \begin{equation}
% E=mc^2.
% \end{equation}

% \subsection{\label{app:subsec}A subsection in an appendix}

% You can use a subsection or subsubsection in an appendix. Note the
% numbering: we are now in Appendix~\ref{app:subsec}.

% Note the equation numbers in this appendix, produced with the
% subequations environment:
% \begin{subequations}
% \begin{eqnarray}
% E&=&mc, \label{appa}
% \\
% E&=&mc^2, \label{appb}
% \\
% E&\agt& mc^3. \label{appc}
% \end{eqnarray}
% \end{subequations}
% They turn out to be Eqs.~(\ref{appa}), (\ref{appb}), and (\ref{appc}).


%%%%%%%%%%%%%%%%%%%%%%%%%%%%%%%%%%%%%%%%%%%%%%%%%%

% The \nocite command causes all entries in a bibliography to be printed out
% whether or not they are actually referenced in the text. This is appropriate
% for the sample file to show the different styles of references, but authors
% most likely will not want to use it.
% \nocite{*}

% \bibliography{apssamp}% Produces the bibliography via BibTeX.

% $\qquad$
% \newline
% $\qquad$
%%%%%%%%%%%%%%%%%%%%%%%%%%%%%%%%%%%%%%%%%%%%
\bibliographystyle{apsrev4-2} % Tell bibtex which bibliography style to use
%%%%%%%%%%%%%%%%%%%%%%%%%%%%%%%%%%%%%%%%%%%% OLD  BIBLIOGRAPHY
% \bibliography{main%apssamp
% } % Te 
%%%%%%%%%%%%%%%%%%%%%%%%%%%%%%%%%%%%%%%%%%%%

\begin{thebibliography}{99}

\bibitem{Fowler2023} 
  %Accreting Primordial Black Holes as Dark Matter Constituents \\
   T.-K.~Fowler and R.~J.~Anantua, %(2023) 
   \url{https://arxiv.org/abs/2303.09341}.

         \bibitem{Carr2020} 
   B.~Carr and F.~K\"uhnel,  
   Annual Review of Nuclear and Particle Science  {\bf 70}, 355-394 (2020).


         \bibitem{Weinberg1993} 
   S.~Weinberg,  {\sl The First Three Minutes: A Modern View of the Origin of the Universe,}
   Basic Books (1993).

         \bibitem{Balbus1998} 
   S.~A.~Balbus and J.~F~Hawley,  
   Reviews of Modern Physics  {\bf 70}, 1 (1998).

         \bibitem{Anantua2009} 
   R.~Anantua, R.~Easther and J.~T.~Giblin,  
   Physical Review Letters  {\bf 103}, 111303 (2009).


            \bibitem{Colgate2014} 
   S.~A.~Colgate, T.-K.~Fowler, H.~Li and J.~Pino,  
   The Astrophysical Journal   {\bf 789}, 144 (2014).

               \bibitem{Colgate2015} 
   S.~A.~Colgate, T.-K.~Fowler, H.~Li, E.~Bickford-Hooper, J.~McClenaghan, and Z.~Lin,  
   The Astrophysical Journal   {\bf 813}, 136 (2015).

   
               \bibitem{Fowler2019} 
    T.-K.~Fowler, H.~Li, and R.~Anantua,  
   The Astrophysical Journal   {\bf 885}, 4 (2019).

         \bibitem{Frank2002} 
   J.~Frank, A.~King and D.~J.~Raine  {\sl Accretion Power in Astrophysics: Third Edition,}
   Cambridge University Press (2002).

                  \bibitem{Cunningham2012} 
   A.~J.~Cunningham, C.-F.~McKee, R.~I.~Klein, M.~R.~Krumholz,  and R.~Teyssier,  
   The Astrophysical Journal {\bf 744}, 185 (2012).

                     \bibitem{Sadowski2013} 
   A.~Sadowski,  R.~Narayan, A.~Tchekhovskoy, and Y.~Zhu,  
   Monthly Notices of the Royal Astronomical Society  {\bf 429}, 3533-3550 (2013).

                        \bibitem{Igumenshchev2002} 
   I.~V.~Igumenshchev and R.~Narayan,  
   Phys. Reports   {\bf 566}, 137-147 (2002). 

                        \bibitem{McKinney2012} 
   J.~C.~McKinney, A.~Tchekhovskoy, and R.~Blandford,  
   Monthly Notices of the Royal Astronomical Society  {\bf 423}, 3083-3117 (2012). 

                \bibitem{Hawley2011} 
    J.~F.~Hawley, X.~Guan, and J.~H.~Krolik,  
   The Astrophysical Journal   {\bf 738}, 84 (2011).

                        \bibitem{Trombka1983} 
   J.~I.~Trombka and C.~E.~Fichtel,  
   Phys. Reports   {\bf 97}, 173-218 (1983).
   

                  \bibitem{Toomre1964} 
   A.~Toomre,  
   The Astrophysical Journal   {\bf 139}, 1217-1238 (1964).

                              \bibitem{Enqvist1998} 
   K.~Enqvist,  
   International Journal of Modern Physics D  {\bf 7}, 331-349 (1998).

                     \bibitem{Kulsrud1997} 
   R.~M.~Kulsrud, R.~Cen, J.~P.~Ostriker,  and D.~Ryu,  
   The Astrophysical Journal {\bf 480}, 481-491 (1997).

                        \bibitem{Peebles1969} 
   P.~J.~E.~Peebles,  
   The Astrophysical Journal  {\bf 155}, 393 (1969).

   %                      \bibitem{Blandford2019} 
   % R.~D.~Blandford, D.~Meier and A.~Readhead,  
   % Annual Reviews of Astronomy and Astrophysics  {\bf 57}, 467-509 (2019).



   
  % \bibitem{pdg}
  %  Particle Data Group reference: \\
  %  W.-M.~Yao {\sl et al.}, Journal of Physics G {\bf 33}, 1 (2006).


% \bibitem{Fowler2023} 
%   %Accreting Primordial Black Holes as Dark Matter Constituents \\
%    T.-K.~Fowler and R.~J.~Anantua, %(2023) 
%    \url{https://arxiv.org/abs/2303.09341}.

%    \bibitem{Carr2021} 
%    B.~Carr, K.~Kohri, Y.~Sendouda and J.~Yokoyama, %(2023) 
%    Reports on Progress in Physics,  {\bf 84}, 116902 (2021).

%       \bibitem{Carr2020} 
%    B.~Carr and F.~K\"uhnel,  
%    Annual Review of Nuclear and Particle Science  {\bf 70}, 355-394 (2020).

%          \bibitem{Weinberg1993} 
%    S.~Weinberg,  {\sl The First Three Minutes: A Modern View of the Origin of the Universe,}
%    Basic Books (1993).

%       \bibitem{Anantua2008} 
%    R.~Anantua, R.~Easther and J.~T~Giblin,  
%    Physical Review Letters  {\bf 103}, 111303 (2009).

%          \bibitem{Balbus1998} 
%    S.~A.~Balbus and J.~F~Hawley,  
%    Reviews of Modern Physics  {\bf 70}, 1 (1998).

%             \bibitem{Colgate2014} 
%    S.~A.~Colgate, T.-K.~Fowler, H.~Li and J.~Pino,  
%    The Astrophysical Journal   {\bf 789}, 144 (2014).

%                \bibitem{Colgate2015} 
%    S.~A.~Colgate, T.-K.~Fowler, H.~Li, E.~Bickford-Hooper, J.~McClenaghan, and Z.~Lin,  
%    The Astrophysical Journal   {\bf 813}, 136 (2015).

   
%                \bibitem{Fowler2019} 
%     T.-K.~Fowler, H.~Li, and R.~Anantua,  
%    The Astrophysical Journal   {\bf 885}, 4 (2019).

%          \bibitem{Frank2002} 
%    J.~Frank, A.~King and D.~J.~Raine  {\sl Accretion Power in Astrophysics: Third Edition,}
%    Cambridge University Press (2002).

%                   \bibitem{Cunningham2012} 
%    A.~J.~Cunningham, C.-F.~McKee, R.~I.~Klein, M.~R.~Krumholz,  and R.~Teyssier,  
%    The Astrophysical Journal {\bf 744}, 185 (2012).

%                      \bibitem{Sadowski2013} 
%    A.~Sadowski,  R.~Narayan, A.~Tchekhovskoy, and Y.~Zhu,  
%    Monthly Notices of the Royal Astronomical Society  {\bf 429}, 3533-3550 (2013).

%                   \bibitem{Toomre1964} 
%    A.~Toomre,  
%    The Astrophysical Journal   {\bf 139}, 1217-1238 (1964).

%                      \bibitem{Trombka1983} 
%    J.~I.~Trombka and C.~E.~Fichtel,  
%    Phys. Reports   {\bf 97}, 173-218 (1983).

%                         \bibitem{Peebles1969} 
%    P.~J.~E.~Peebles,  
%    The Astrophysical Journal  {\bf 155}, 393 (1969).

%                         \bibitem{Blandford2019} 
%    R.~D.~Blandford, D.~Meier and A.~Readhead,  
%    Annual Reviews of Astronomy and Astrophysics  {\bf 57}, 467-509 (2019).

%                            \bibitem{Enqvist1998} 
%    K.~Enqvist,  
%    International Journal of Modern Physics D  {\bf 7}, 331-349 (1998).

%                      \bibitem{Kulsrud1997} 
%    R.~M.~Kulsrud, R.~Cen, J.~P.~Ostriker,  and D.~Ryu,  
%    The Astrophysical Journal {\bf 480}, 481-491 (1997).

%                      \bibitem{Igumenshchev2002} 
%    I.~V.~Igumenshchev and R.~Narayan,  
%    Phys. Reports   {\bf 566}, 137-147 (2002). 
   
%   % \bibitem{pdg}
%   %  Particle Data Group reference: \\
%   %  W.-M.~Yao {\sl et al.}, Journal of Physics G {\bf 33}, 1 (2006).

\end{thebibliography}


\end{document}
%
% ****** End of file apssamp.tex ******