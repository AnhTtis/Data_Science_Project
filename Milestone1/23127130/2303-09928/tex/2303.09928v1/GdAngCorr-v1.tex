\documentclass{ptephy_v1}%%%%%% to generate preprint number with ptep logo
%\documentclass{ptephy_v1}%%%%where ptephy_v1 is the template name
%\documentclass[preprint]{ptephy_v1}%%%%%% to generate preprint number
%\preprintnumber{XXXX-XXXX} %%% %%% Insert preprint number here

%The authors can define any packages after the \documentclass{ptephy_v1} command.

\usepackage{amsmath}
\usepackage{geometry}
\usepackage{graphics}
\usepackage{multirow}
\usepackage{url}
\usepackage{float}
%\usepackage{amsthm} for dealing with theorem environments,
%\usepackage{hyperref} for linking the cross references
%\usepackage{graphics} for dealing with figures.
%\usepackage{algorithmic} for describing algorithms
%\usepackage{subfig} for getting the subfigures e.g., "Figure 1a and 1b" etc.

% JUST TEMPORARY
\usepackage{color}
\newcommand{\comm}[1]{{\color{red}{\textit{#1}}}} % Comment

%The author can find the documentation of additional supporting files from "http://www.ctan.org"

% *** Do not adjust lengths that control margins, column widths, etc. ***

\newtheorem{theorem}{Theorem}
\newtheorem{condition}{Condition}

%\usepackage[showframe,%Uncomment any one of the following lines to test 
%%scale=0.7, marginratio={1:1, 2:3}, ignoreall,% default settings
%%text={7in,10in},centering,
%%margin=1.5in,
%%total={6.5in,8.75in}, top=1.2in, left=0.9in, includefoot,
%%height=10in,a5paper,hmargin={3cm,0.8in},
%]{geometry}

\begin{document}


\title{Angular correlation of the two gamma rays produced in the thermal neutron capture on gadolinium-155 and gadolinium-157}

\author[1,2]{Pierre Goux}
\author[1,2]{Franz Glessgen}
\author[1,3,8]{Enrico Gazzola}
\author[1,9]{Mandeep Singh Reen}
\author[2]{William Focillon}
\author[2,5,*]{Michel Gonin}
\author[1]{Tomoyuki Tanaka}
\author[1]{Kaito Hagiwara}
\author[1,10]{Ajmi Ali}
%\author[1]{Pretam Kumar Das}
%\author[1]{Iwa Ou}
\author[1,6]{Takashi Sudo}
%\author[1]{Yoshiyuki Yamada}
%\author[1]{Takaaki Mori}
%\author[1]{Tsubasa Kayano}
%\author[1,5]{Rohit Dhir}
\author[1]{Yusuke Koshio}
\author[1,*]{Makoto Sakuda}
%\author[3]{Enrico Gazzola}
\author[3]{Gianmaria Collazuol}
\author[4]{Atsushi Kimura}
\author[4]{Shoji Nakamura}
\author[4]{Nobuyuki Iwamoto}
\author[4,*]{Hideo Harada}
\author[7]{Michael Wurm}
\affil[1]{Department of Physics, Okayama University, Okayama 700-8530, Japan }
\affil[2]{D\'epartement de Physique, \'Ecole Polytechnique, IN2P3/CNRS, 91128 Palaiseau Cedex, France}
%\affil[2]{Department of Physics, ICRR, Japan}
\affil[3]{University of Padova and INFN, Italy}
\affil[4]{Japan Atomic Energy Agency, 2-4 Shirakata, Tokai, Naka, Ibaraki 319-1195, Japan}
\affil[5]{ILANCE, CNRS – University of Tokyo International Research Laboratory, Kashiwa, Chiba 277-8582, Japan}
\affil[6]{Research Center for Nuclear Physics (RCNP), Osaka University, 567-0047 Osaka, Japan}
\affil[7]{Institut f\"ur Physik, Johannes Gutenberg-Universit\"at Mainz, 55128 Mainz, Germany}
\affil[8]{Present address: Finapp srl, 35036 Montegrotto Terme, Padua, Italy}
\affil[9]{Present address: Department of Physics, Akal University, Punjab 151302, India}
\affil[10]{Present address: Department of Physics, University of Winnipeg, Manitoba, Canada}
%\affil[5]{Present address: Research Institute $\&$ Department of Physics and Nano Technology, SRM 
%University, Kattankulathur-603203, Tamil Nadu, India}
\affil[ ]{\email{michel.gonin@polytechnique.edu, harada.hideo@jaea.go.jp, sakuda-m@okayama-u.ac.jp}}

\begin{abstract}
The ANNRI-Gd collaboration studied  in detail the single  $\gamma$-ray spectrum produced from the thermal neutron capture on $^{155}$Gd and $^{157}$Gd in the previous publications~\cite{Hagiwara, Tanaka}. Gadolinium targets were exposed to a neutron beam provided by the Japan Spallation Neutron Source (JSNS) in J-PARC, Japan. 
In the present analysis, one new additional coaxial germanium crystal was used in the analysis in combination with the fourteen  germanium crystals in the cluster detectors to study the angular correlation of the two $\gamma$ rays emitted in the same neutron capture. We present for the first time angular correlation functions for two $\gamma$ rays produced during the electromagnetic cascade transitions in the (n, $\gamma$) reactions on $^{\rm 155}$Gd and $^{\rm 157}$Gd.  As expected, we observe the mild angular correlations for the strong, but rare  transitions from the resonance state to the two energy levels of known spin-parities.  Contrariwise,  we observe negligibly small angular correlations for arbitrary pairs of two  $\gamma$ rays produced in the majority of cascade transitions from the resonance state to the dense continuum states. 
   
\end{abstract}

\subjectindex{C30, C43 Underground experiments, D03, D21 Models of nuclear reactions, D29, H16, H17, H20 Instrumentation for underground experiments
}

\maketitle

%==================================================================================================%
\section{Introduction}
%==================================================================================================%

 The gadolinium (Gd) nucleus is one of the few stable nuclei (Cd, Sm, Gd) featuring unusually large cross sections and resonance enhancements for the thermal neutron capture~\cite{Mughabghab2006, Leinweber, Choi, nTOF}. The two gadolinium isotopes $^{157}$Gd and $^{155}$Gd possess the largest neutron capture cross sections among the stable nuclei~\cite{Mughabghab2006}.
 The element has been used as a neutron absorber in liquid-scintillator-based detectors for neutrino oscillation experiments~\cite{Dchooz, RENO, DayaBay, NEOS, STEREO, Neutrino4, DANSS, SterileCombine, JSNS2},  a neutrino-flux monitor experiment~\cite{PANDA}, and even medical science~\cite{GdNCT}. 
 The application of  Gd-loaded detectors for tagging neutrons has been recently extended to direct dark matter search experiments~\cite{LZ, Xenon}.  The identification of neutrons produced from the inverse beta decay with large efficiency is crucial for the detection of  Supernova Relic Neutrinos (SRN) in a Gd-loaded water Cherenkov detector like Super-Kamiokande~\cite{Vagins, EGADS, SkGd}. Upon neutron capture, the  Gd isotopes $^{157}$Gd and $^{155}$Gd release cascade of $\gamma$ rays with a total energy of 7937 keV for $^{158}$Gd and 8536 keV for $^{156}$Gd.  Due to the Cherenkov threshold applying for the detection of the multiple Compton electrons produced by these $\gamma$ rays in the Super-Kamiokande detector, a precise knowledge and understanding of the cascade $\gamma$-ray energies is absolutely necessary in order to model  the neutron capture efficiency with Monte Carlo simulations. 

 In two previous publications~\cite{Hagiwara, Tanaka}, we reported  measurements of the single $\gamma$-ray spectra  produced from the thermal neutron capture on targets comprising a natural Gd film and gadolinium oxide powders enriched with  $^{\rm 155}$Gd and $^{\rm 157}$Gd. We used the ANNRI spectrometer at J-PARC to record the $\gamma$ rays in the energy range from 0.11 MeV to 8.0 MeV. Moreover, we showed that our Monte Carlo simulation (ANNRI-Gd Model) agreed with our measured spectra reasonably well. 
%The Gd(n,$\gamma$) spectroscopy allows to understand the properties of the compound nucleus and its decays by cascade of $\gamma$ ray emissions. 
%In this paper, we study the capture of thermal neutrons by the isotopes $^{155}$Gd and $^{157}$Gd followed by emission of $\gamma$ rays. 
 We first identified the prominent photopeaks above 5 MeV in the single spectrum and found the secondary transitions associated with each primary photopeak. We listed those  12 and 15 primary $\gamma$ rays for $^{\rm 155}$Gd and $^{\rm 157}$Gd, respectively,  and also identified the secondary $\gamma$ rays.  Those 'discrete' $\gamma$ rays constitute 3-7$\%$ of the total  $\gamma$ rays. However, most of the $\gamma$ rays result from the dense 'continuum' states. 
 
 In the present paper,  we report on  the angular correlations between the two $\gamma$ rays  for some selected discrete and continuum transitions in the $^{\rm 155}$Gd and $^{\rm 157}$Gd(n, $\gamma$) reactions. 
The theoretical calculations for the electromagnetic cascade transitions and the angular correlation function can be found elsewhere~\cite{Frauenfelder, Biedenharn, Rose}. We present a comparison between our data and the expected angular correlations. Recently, a Monte Carlo simulation called the FIFRELIN code~\cite{FIFRELIN} has been developed for the STEREO experiment~\cite{STEREO} which takes into account the angular correlations in cascade transitions. Thus, it is timely for us to present the result of the angular correlations measured in the $^{\rm 155}$Gd and $^{\rm 157}$Gd(n, $\gamma$) reactions. 
\newline 


\begin{figure}[!htbp]
  \begin{center}
	\includegraphics[width=0.8\linewidth]{target-190614.pdf}
  \caption{\label{fig::mafigure}ANNRI Spectrometer.}
	\end{center}
\end{figure}

%==================================================================================================%
\section{Experiment}
\subsection{Experiment And Data Collection}
%==================================================================================================%

The data presented in this analysis were recorded  in December, 2014, with the ANNRI germanium (Ge) spectrometer at the the Materials and
Life Science Experimental Facility (MLF) of J-PARC. The MLF provides a pulsed neutron beam with energies from a few meV up to 100 keV.  Gadolinium oxide powder targets (Gd$_2$O$_3$)  enriched with ${}^{155}$Gd (91.85\%) and ${}^{157}$Gd (88.4\%) were placed inside the ANNRI Ge spectrometer,  which consists of two basic parts: two cluster detectors placed perpendicular to the beam pipe, and eight coaxial detectors placed in a horizontal plane containing the beam pipe and the gadolinium target~\cite{Kimura2012, Kin2011, Kino2011, Kino2014}. The two cluster detectors and the only one of the coaxial detectors were operational during the experiment. A complete description of our experiment and the analysis method, using the two cluster detectors, which consisted of a total of 14 Ge crystals,  can be found in our previous publications~\cite{Hagiwara, Tanaka}  

In the present analysis, we analysed the data recorded by the single coaxial detector in the horizontal plane in addition to the two cluster detectors. The overall view of the ANNRI spectrometer including both the cluster detectors and the coaxial detectors is shown in Fig.1(a); the detailed geometry of the single coaxial detector and its lead shield is shown in Fig.1(b) and Fig.1(c). The energy threshold for $\gamma$ detection of the coaxial detector was about 300 keV. The other seven coaxial crystal detectors were in repair and were not operational during our experiment.  These 15 detectors cover about 16$\%$ of the 4$\pi$ solid angle.  The measurement of background events from neutron beam interactions with the materials surrounding the gadolinium target was performed with an empty target setup.  The background level was extremely low, at a level of 0.1$\%$~\cite{Hagiwara, Tanaka}. 

%%We have used for this analysis both cluster detectors with its anti-coincidence shields made of Bismuth Germanium Oxide and, for the first time, one of eight coaxial crystal detectors. The details concerning the detector configuration, data acquisition, event classification, energy calibration and detector simulations have been already given in our previous publications~\cite{Hagiwara, Tanaka}. We will focus here on measurements of uniformity and efficiency between the different detectors, and the corrections to be applied in order to compare our new data with the theoretical predictions. \newline

\subsection{Calibration}
The coaxial detector is self-contained with an individual aluminum protective case.  In addition to the protective layer, it is also protected by a lead collimator to reduce the solid angle of $\gamma$ rays produced outside the gadolinium target. LiH was filled inside the lead collimator to reduce the neutron background. With the use of cluster detectors alone, the angular correlation measurements would have been quite limited. The addition of the one  coaxial detector allows for a much greater angular coverage, and substantially larger statistics for the angular correlation analysis. The efficiency of the coaxial detector was estimated with the same method as used for the cluster detectors. This method is described in detail in our previous publications~\cite{Hagiwara, Tanaka}.  In brief, we used radioactive calibration sources, e.g. $^{60}$Co, $^{137}$Cs and $^{152}$Eu, as well as the prompt $\gamma$ rays produced by neutron capture reaction $^{35}$Cl($n, \gamma)^{36}$Cl in the energy range between 0.1 MeV and 9 MeV. To calculate the $\gamma$-ray detection efficiency of the coaxial detector, we  divide the number of $\gamma$ rays detected within the photopeaks by the number of $\gamma$ rays expected due to the solid angle of our detector, corrected by the lifetime of our data acquisition.  \newline

The efficiency values obtained for the coaxial detector are shown in Fig.2  as a function of  $\gamma$-ray energy and are compared to our Geant4 Monte Carlo simulation (dashed-dotted curve) that includes the full geometry and materials of the ANNRI detector. The absolute normalization of our data to the simulation was obtained using  the 7414 keV line of the capture reaction $^{35}$Cl($n, \gamma)^{36}$Cl.  The size of the error bars is determined by the statistics of the data and the Monte Carlo simulation. The agreement between our calibration data and the detector simulations (dashed-dot curve) is satisfactory.  
%Sakuda added the following paragraph to stress the importance of one co-axial counter. 2021.03.02. 
Although  the solid angle covered by the single coaxial detector for a $\gamma$ ray  emitted in the target is  only 1\%,  the detector plays an essential role in the analysis of the two $\gamma$ rays. 
%Using only the two cluster detectors, the range of  $cos\theta_{12}$ in which the angular correlation between the two $\gamma$ rays can be measured can be limited to -1$<cos\theta_{12}<$-0.6 and 0.6$<cos\theta_{12}<$1.0, 
%we can add the range  -0.4$<cos\theta_{12}<$0.4 by the angular correlation between the one coaxial detector and the another crystal in the cluster detectors.
While the range of  $cos\theta_{12}$ for the angular correlation between the two $\gamma$ rays is limited to -1$<cos\theta_{12}<$-0.6 and 0.6$<cos\theta_{12}<$1.0 using the two cluster detectors,  the range,  -0.4$<cos\theta_{12}<$0.4,  is added  by the correlation between the one coaxial detector and the another crystal in the cluster detectors.  An exact definition of the  angular correlation  will be given in the next section.

\begin{figure}[ht!]
  \begin{center}
	\includegraphics[width=10.0cm,scale=1.0]{Fig2v2.pdf}
  \caption{\label{fig1} Efficiency measurements for the coaxial detector compared with simulations(dashed-dot curve).}
	\end{center}
\end{figure}

In the previous publications~\cite{Hagiwara, Tanaka}, we studied the uniformity of the counting rate measured by each crystal of the cluster detectors using the data of radioactive calibration sources. Here, we present a new analysis of the uniformity of both the two clusters and the coaxial detector using the prominent photopeaks produced by exposing the gadolinium targets. For a given photopeak, we calculated for each crystal the ratio of the number of raw data events divided by the expected numbers after taking into account the efficiency, the solid angle and the relative intensity of each photopeak. Fig.3 shows the results for $^{158}$Gd (top) and $^{156}$Gd (bottom). In the histograms, the detector number 0 represents the coaxial detector while the numbers 1 to 14 correspond to the crystals of the two cluster detectors.  The figures  show a very good uniformity between the 15 detectors over an energy range of 1 MeV up to 7 MeV. The variation of the ratios by about 10$\%$ is taken as a measure of  the systematic uncertainties of the counting efficiencies. This uniformity of the measured rates over the two cluster detectors and the coaxial detector is an essential prerequisite for the present analysis of the angular correlations.

%Since our targets had a finite size, we also checked the position dependence of the efficiency by taking data with a $^{60}$Co source placed at various positions. The agreement of the data and our simulation was consistent within the uncertainties for each measurement.\newline

\begin{figure}[!htbp]
  \begin{center}
	\includegraphics[width=12.0cm,scale=1.0]{Fig2_v4.pdf}
  \caption{\label{fig2}Normalized counts of various photopeaks for different energy range and for each crystal of the 15  detectors. The top and bottom plots are for the counting rate of various $\gamma$ rays from $^{158}$Gd and $^{156}$Gd, respectively.}
	\end{center}
\end{figure}

\subsection{Data selection and the definition of the angular correlation function  $W(\theta)$}

%Given the geometry of the ANNRI detector, there are many different ways in which two $\gamma$ rays are detected in the detector.  
%Like our previous analysis, we classified our gamma-ray events with two letters M and H. The multiplicity M is the number of hits in a single crystal or in a subgroup of nearby crystals and the H value describes the total number of crystal hits for a gadolinium decay event. 
  As in the previous publications~\cite{Hagiwara, Tanaka},  we classify events by assigning a multiplicity value M and a hit value H to each event.  We defined the multiplicity M  as the combined number of isolated sub-clusters of hit Ge crystals at the upper and the lower  cluster detectors. If the coaxial crystal is hit,  the values of M and  H are both increased by 1, since the hit  is always isolated. The multiplicity M represents the number of observed $\gamma$ rays and the hit value H represents  the total number of Ge crystals hit in the event. We select events categorized as M2H2 and M2H3  to  study the angular correlations of two $\gamma$ rays. \\
%For example, M2H3 means the detection of 2 emitted gamma rays with a spread of hits in 3 crystals.  

\begin{figure}[!htbp]
  \begin{center}
	\includegraphics[width=12.0cm,scale=1.0]{Fig3v1.pdf}
  \caption{\label{fig3} Energy spectra of $\gamma$ ray measured by the ANNRI detector (left) and illustration of the two- and three-step $\gamma$-ray cascades (right) of $^{158}$Gd after thermal neutron capture.}
	\end{center}
\end{figure}

%2022.0209 Sakuda added
We define the angular correlation function  $W(\theta_{12})$ using a sample of 
two $\gamma$ rays detected by the two crystals $(i,j)$,  where $\theta_{12}$ is the angle between the two hit crystals $(i,j)$~\cite{Frauenfelder, Biedenharn, Rose}. 
%This was already described in the Appendix of the our previous publication~\cite{Hagiwara}.  
For thermal neutron capture on $^{158}$Gd,  a typical process producing two $\gamma$ rays  is a two-step or three-step cascade transition in the deexcitation of the initial state of 7937 keV ($2^-$). As illustrated in Fig.\ref{fig3} (right), two $\gamma$ rays of 6750 keV and 1187 keV, or three $\gamma$ rays of 6750 keV, 1107 keV and 80 keV, are produced in these transitions. Fig.\ref{fig3}(left) shows the energy of the second  $\gamma$ ray in case the first $\gamma$ ray of 6750 keV is tagged, using the M2H2 (multiplicity 2) sample.  Two peaks corresponding to the $\gamma$ rays of 1187 keV and 1107 keV are clearly seen, while 80 keV below the energy threshold is not measured.  

The observed number $N_{ij}(\theta_{12})$ of two $\gamma$-ray events with energies   $E_1$  and $E_2$  detected in crystals $i$  and $j$  can be denoted  as  
\begin{eqnarray}
N_{ij}(\theta_{12})= N_0 r_{L,ij} \epsilon_{i}(E_1)  \epsilon_{j}(E_2)  W(\theta_{12}),
\end{eqnarray}
where $N_0$ is the number of two $\gamma$-ray events produced at the target, 
$r_{L,ij}$ is the dead time correction factor for the crystal pair $(i,j)$, which typically is on the order of 90\%,  and  $\epsilon_{i}(E_1)$ and $\epsilon_{j}(E_2)$) are the single photopeak efficiency of  the crystals $i$  and $j$ for $\gamma$-ray  energies $E_1$  and $E_2$, respectively, and  $W(\theta_{12}) $ is the angular correlation function between the two $\gamma$ rays.  
If there is no angular correlation, then the angular correlation function is uniform, $W(\theta_{12})=1.0 $, with respect to $cos \theta_{12}$. The angular correlation function $W(\theta_{12})$ can  be evaluated  in an experiment using Eq.(1) as, 
\begin{eqnarray}
W(\theta_{12}) = C
\sum_{i\neq j=0}^{14}  \frac{N_{ij}(\theta_{12})} {\epsilon_{i}(E_1)  \epsilon_{j}(E_2) }, 
\end{eqnarray}
where $C$ is a constant  and the sum is taken over all possible combinations of $(i,j)$ pairs having the angle $\theta_{12}$.  
%and %The normalization of the angular correlation function is $\int_{-1}^{1}W(\theta)d(cos\theta) =1$. 
 In the analysis, for every pair $(i,j)$ of observed $\gamma$ rays, we calculate  $z=cos\theta_{12}$ and  fill the histogram at a position $z$ with a weight $\frac{N_{ij}(\theta_{12})} {\epsilon_{i}(E_1)  \epsilon_{j}(E_2) }$ given by the right-hand side of Eq.(2).  An overall constant $C$ in Eq.(2)  is arbitrary in the present analysis, but if we evaluate the sum on the right hand side of Eq.(2), it should be roughly equal to the number of two  $\gamma$-ray events produced in the target.  
Any deviation from a uniform and  constant distribution of $W(\theta_{12})$ with respect to $cos \theta_{12}$  suggests the existence of an angular correlation between the two  $\gamma$ rays. 

The calculation of the angular correlation function for the two  $\gamma$ rays from  cascade transitions is based  on the electromagnetic theory and quantum numbers conservation as given in Ref.~\cite{Frauenfelder, Biedenharn, Rose}. 
The angular correlation function $W(\theta)$ is conveniently written in terms of  Legendre polynomials as, 
\begin{eqnarray}
W(\theta) = \sum_{\ell =0}^{\ell_{max}}A_{\ell}P_{\ell}(cos\theta), 
\end{eqnarray}
where $P_{\ell}(z)$ is a Legendre polynomial of a degree $\ell$ and $A_{\ell}$ is the coefficient.  
When the detectors for the two $\gamma$  rays are placed (roughly) at cylindrically symmetrical positions from a given source point, this form is simplified  to contain only leading order terms as $\ell$=0, 2 and 4, limiting transitions to dipole and quadrupole types, as
\begin{eqnarray}
W(\theta) = C[1+A_2P_2(cos\theta)+A_4P_4(cos\theta)],  
\label{eq:CorrFun}
\end{eqnarray}
where $C$ is an overall constant. In any experiment, each $\gamma$-ray detector has a finite size and the 
angular correlation function is subject to the correction for the finite size effect or the angular resolution effect~\cite{Rose1, Camp}. If this effect is taken into  account, the coefficients in Eq.(4) are written as, 
\begin{eqnarray}
A_2=A^\prime _2 Q_2\ \ {\rm and}\ \ A_4=A^\prime _4 Q_4, 
\label{eq:FiniteCorr}
\end{eqnarray}
where $Q_2$ and $Q_4$ are the correction factors, and  $A^\prime _2$ and $A^\prime _4$ are the 
coefficients when each detector has a perfect angular resolution, namely $Q_2$=1.0 and $Q_4$=1.0. For the finite angular resolutions, $Q_2$ and $Q_4$ are less than 1.0 and the measured values for $A_2$ and $A_4$ become smaller than the theoretical values for $A^\prime _2$ and $A^\prime _4$. The formula and tabulated values for the coefficients, $A^\prime _2$ and $A^\prime _4$, are given in Ref.~\cite{Biedenharn}. The formula for the correction factors $Q_2$ and $Q_4$ are also given in Ref.~\cite{Rose1, Camp}. \\
\indent
In our experiment, the angular correlation function $W(\theta)$ is analysed using Eqs.(4) and (5) to determine the coefficients $A^\prime _2$ and $A^\prime _4$. Then, the measured values, $A^\prime _2$ and $A^\prime _4$, can be compared with the theoretical values, $A^\prime _2$ and $A^\prime _4$ ~\cite{Biedenharn}. In our ANNRI geometry, the correction factors are calculated to be $Q_2$=0.93$\pm$0.01 ($Q_2$=0.94$\pm$0.01) and $Q_4$=0.77$\pm$0.01 ($Q_4$=0.80$\pm$0.01) for $|cos\theta|>$0.6 ($|cos\theta|<$0.4), respectively. The uncertainty  in the correction factors comes from the uncertainty in the dead layer thickness (1mm) of the Ge crystal~\cite{Utsunomiya, Terada}.


\subsection{Angular correlation of the two $\gamma$ rays from the cascade transition in $^{60}$Co $\beta ^-$ decay}

Fig.\ref{figCo} shows the angular correlation of the two $\gamma$ rays of 1173 keV and 1332 keV from the cascade transition  (2505 keV, $4^+ \rightarrow$ 1332 keV, $2^+  \rightarrow$  0 keV, $0^+$)  of  $^{60}$Ni from $^{60}$Co $\beta ^-$ decay. We used only the data set of the M2H2 sample. In this analysis, the $^{60}$Co source was set in the target position of the ANNRI detector. 
The predicted values for the coefficients are $A^\prime _2$=0.1020  and $A^\prime _4$=0.0091, respectively. The expected angular correlation is shown in the dashed black curve in Fig.\ref{figCo} 
and it agrees well with data, with $\chi ^2 /dof$=10.5/13. 
If we fit the data using Eqs.(4) and (5) with $A^\prime _2$ being a free parameter and with the fixed value of $A^\prime _4$=0.0091, we obtain $A^\prime _2$=0.15$\pm$0.06 with $\chi ^2 /dof$=9.3/12, which is consistent with the expected value 0.091 within the given uncertainty.  The predicted curve is shown as a red solid curve in Fig.\ref{figCo}. 

\begin{figure}[!htbp]
  \begin{center}
	\includegraphics[width=0.8\linewidth]{Co60AngCorr.pdf}
  \caption{\label{figCo} Measurement of the angular correlation of the two $\gamma$ rays of 1173 keV and 1332 keV from the cascade transition  (2505 keV, $4^+ \rightarrow$ 1332 keV, $2^+ \rightarrow$ 0  keV, $0^+$)  of  $^{60}$Ni in $^{60}$Co $\beta ^-$ decay, The black dashed curve is calculated with the theoretical values for $A^\prime_2$=0.102 and $A^\prime_4$=0.0091 in Eqs.(4) and (5). The red curve is the prediction with $A^\prime_2$=0.15 and $A^\prime_4$=0.0091.  An overall normalization is arbitrary.}
	\end{center}
\end{figure}


\section{Analysis and result}
\subsection{Angular correlation of the two $\gamma$ rays for prominent discrete cascade transitions}

We now study the angular correlation of the two $\gamma$ rays resulting from the prominent discrete cascade transitions of $^{158}$Gd and $^{156}$Gd nuclei. 
% In order to check the consistency of our measurements and possibly the contamination by pileup events, we have verified the energy correlation %of the two detected $\gamma$ rays as will be shown later for the continuum case. The energy correlation diagram clearly shows the events with %the two discrete $\gamma$ rays 675keV and 118keV corresponding to our selection of events. 

The process of producing two  $\gamma$ rays  of 6750 keV and 1187 keV, or,  6750 keV and 1107 keV in the cascade transition  of  $^{158}$Gd was already shown in Fig.\ref{fig3}. Figs.\ref{figM2H2}(a) and \ref{figM2H2}(b) exemplify the selection of the two $\gamma$ rays in the M2H2 sample. We show in Fig.\ref{figM2H2}(a) the energy  of the two $\gamma$-rays ($E_1$ and $E_2$), in which the sum $E_1 + E_2$ is equal to 7937 keV within $\pm$25 keV in the M2H2 sample. We select the strongest two  peaks due to 6750 keV and 1187 keV where we observe  almost no random background. The background rate estimation will be described later. 

 For the angular correlation function for the two  $\gamma$ rays  of 6750 keV and 1187 keV in the  two-step cascade transition (7937 keV, $2^- \rightarrow$ 1187 keV, $2^+ \rightarrow$ 0 keV, $0^+$)  of  $^{158}$Gd, the expected coefficients are  $A^\prime _2$=0.25 and $A^\prime_4$=0.
%%\begin{equation}
%%A^\prime _2=0.25,\ \ A_2=A^\prime _2 Q_2=0.24 \ {\rm and} \ \ A_4=0.
%%\end{equation}
In this cascade transition, the first transition is $E1$ and the second is $E2$. For the cascade transition including an $E1$ transition, $A^\prime _4$ is expected to be 0.0~\cite{Biedenharn, Rose}.  

%\begin{equation}
%W(\theta) \; \propto \;\; 1 + \frac{3}{7} \cos^2\theta.
%\end{equation}
                                        
The angular correlation function measured for these two $\gamma$ rays is shown in Fig.\ref{fig6}. We used two sets of events, namely the  M2H2 sample (black closed circles) and the M2H3 sample (red closed squares).  The data points have been corrected for efficiencies and acceptances according to Eq.(2).  The error bars for all data points are calculated by adding the statistical and systematic uncertainties  in quadrature. The data show a strong angular correlation between the two $\gamma$ rays. 
 If we fit the data using Eqs.(4) and (5) with $A^\prime_2$ being a free parameter and a fixed value of $A^\prime_4$=0.0, we obtain $A^\prime_2$=0.31$\pm$0.03. 
The best fit result with $A^\prime_2$=0.31 and $A^\prime_4$=0.0 is shown in the red solid curve. The agreement between the fit and the data is good ($\chi ^2 /dof  $=31/35). The  best fit values are consistent with the prediction of the expected value 0.25 (shown in black dashed curve) within about 2$\sigma$ level. 

At first glance, the energy distributions shown in  Figs.\ref{fig3} and \ref{fig6}(a) indicate that the background to this sample is negligible. However, we note that there is a chance that the cascade transition of 1107 keV and 80 keV can enter the same crystal, which results in a peak at 1187 keV. Its strength cannot be estimated by the extrapolation of the side-band background rates to the 1187-keV peak. In the following, we  call this probability the coincidence summing probability. We estimated the coincidence summing probability to be about 5$\times 10^{-3}$ by counting the number of events in the 7937-keV peak in  $^{158}$Gd data caused by the coincidence sum of the two $\gamma$ rays of 6750 keV and 1187 keV (7937  keV, $2^- \rightarrow$ 1187 keV, $2^+ \rightarrow $ 0 keV, $0^+$).  Similarly, the 8536-keV peak in  $^{156}$Gd data is caused by  the coincidence sum of the two $\gamma$ rays of 7382 keV and 1154 keV (8536  keV, $2^- \rightarrow$ 1154 keV, $2^+ \rightarrow $ 0 keV, $0^+$). The coincidence summing probability of the 8536-keV peak was found to agree with that of the 7937-keV peak in  $^{158}$Gd data  within 20\%.  For both photopeaks, the direct $M2$ transition of the resonance state  
($2^-$) to the ground state ($0^+$) is strongly suppressed, compared to the $E1$ transition from the resonance state ($2^-$)  to the 1187-keV state ($2^+$, $^{158}$Gd) or to the 1154-keV state ($2^+$, $^{156}$Gd)~\footnote{We mistakenly listed the intensity of the  7937-keV peak as 0.55$\pm$0.03($\times10^{-2}$\%) in the Table 1 of our previous publication~\cite{Hagiwara}. We used this intensity of the 7937-keV peak to estimate the coincidence summing probability in this paper.}. 
The coincidence summing probability to the 1187-keV peak is less than 1\%. In addition, we checked all possible pairs of two $\gamma$ rays in the M2H2 sample with a coincidence sum that evaluates to 6750 keV. Such pairs of the two $\gamma$ rays include 5903 keV and 847 keV (7937  keV, $2^- \rightarrow$ 2034 keV, $3^+ \rightarrow $ 1187 keV, $2^+$) and 5784 keV and 966 keV (7937  keV, $2^- \rightarrow$  2153 keV, $2, 3^+ \rightarrow $ 1187 keV, $2^+$). We estimated the coincidence summing probability to be about 1.5\% of the total  number of counts in the single photopeak of 6750 keV. The coincidence summing effect to the angular correlation function is negligible. 

\begin{figure}[!t]
  \begin{center}
	\includegraphics[width=0.6\linewidth]{M2H2-2gam.pdf}
%	\includegraphics[width=0.48\linewidth]{M2H2-8536keV.pdf}
  \caption{\label{figM2H2} (a) Selection of the two $\gamma$ rays of 6750 keV and 1187 keV in the two-step cascade transition (7937  keV, $2^- \rightarrow$ 1187 keV, $2^+ \rightarrow$ 0 keV, $0^+$)  of $^{158}$Gd. (b) Selection of the two $\gamma$ rays of 7382 keV and 1154 keV in the two-step cascade transition (8536  keV, $2^- \rightarrow$ 1154 keV, $2^+ \rightarrow$ 0 keV, $0^+$)  of $^{156}$Gd.}
	\end{center}
\end{figure}

\begin{figure}[!htbp]
  \begin{center}
	\includegraphics[width=0.8\linewidth]{Fig6new2.pdf}
  \caption{\label{fig6} Measurement of the angular correlation function for the two $\gamma$ rays  of 6750 keV and 1187 keV in the two-step cascade transition (7937  keV, $2^- \rightarrow$ 1187 keV, $2^+ \rightarrow$ 0 keV, $0^+$)  of $^{158}$Gd. The data points  of the  M2H2 sample and the M2H3 sample are plotted in black closed circles and  red closed squares, respectively. The prediction with $A^\prime_2$=0.31 and that with the nominal value $A^\prime_2$=0.25 are shown in the red solid curve and the black dashed curve. Both curves are consistent with the data. }
	\end{center}
\end{figure}


\begin{figure}[!htbp]
  \begin{center}
	\includegraphics[width=10.0cm,scale=1.0]{Fig7new2.pdf}
  \caption{\label{fig7}Measurement of the angular correlation between two $\gamma$ rays of 6750 keV and 1107 keV in the two-step cascade
  ($7937 \  keV, 2^- \rightarrow 1187 \ keV, 2^+ \rightarrow 80 \ keV, 2^+$)  of  $^{158}$Gd. The data points  of the  M2H2 sample and the M2H3 sample are plotted in black closed circles and  red closed squares, respectively.
  The predictions with $A^\prime_2$=-0.11 ($\delta$=-9.0),  $A^\prime_2$=-0.22 ($\delta$=-1.5) and $A^\prime_2$=-0.37  are drawn in black solid curve, black dashed curve and red solid curve, respectively.}
	\end{center}
\end{figure}

\begin{figure}[!htbp]
  \begin{center}
	\includegraphics[width=0.8\linewidth]{Fig8new2.pdf}
  \caption{\label{fig8}Measurement of the angular correlation between two $\gamma$ rays of 7382 keV and 1154 keV  in the prominent two-step cascade transition (8536  keV, $2^- \rightarrow$ 1154 keV, $2^+ \rightarrow$ 0 keV, $0^+$) for $^{156}$Gd. The data points  of the  M2H2 sample and the M2H3 sample are plotted in black closed circles and  red closed squares, respectively. The predictions with $A^\prime_2$=0.10 (best fit, red solid curve) and that with the nominal value $A^\prime_2$=0.25 (black dashed curve) are shown, }
	\end{center}
\end{figure}

Next, the result for the angular correlation between the  two $\gamma$ rays of 6750 keV and 1107 keV in the cascade transition from $^{158}$Gd (7937 keV, $2^-$) are shown in Fig.\ref{fig7}. Note that  the 80 keV $\gamma$-ray in Fig.\ref{fig3}(right) cannot be detected by ANNRI since it is below our experimental threshold. 
The theoretical prediction for the angular correlation function is estimated for  $E1$-$E2$ transition and for $E1$-$M1$ transition as, 
\begin{eqnarray}
A^\prime _2&=&-0.054, \ A^\prime _4=0\  {\rm for\ a\ pure}\  E1-E2  {\rm\ transition \ and,} \nonumber \\
A^\prime _2&=&0.175, \ A^\prime _4=0\  {\rm for\ a\ pure}\  E1-M1  {\rm \ transition.}
%W(\theta)  &\; \propto \;\;& 1 +0.29 \cos^2\theta, {\rm  \  for \   a\  pure\  E1-M1 \   transition, \   and} \  \\
%W(\theta)  &\; \propto \;\;& 1 - 0.079 \cos^2\theta, {\rm  \  for \  a\  pure\  E1-E2 \   transition.} 
\end{eqnarray}
Fig.\ref{fig7} clearly shows a negative value for $A^\prime_2$. If we fit the data using Eqs.(4) and (5) with $A^\prime_2$ being a free parameter and with a fixed value of $A_4$=0.0, we obtain $A^\prime_2$=-0.37$\pm$0.04.  The agreement between the fit and the data is good  with $\chi ^2 /dof  $=38/35. This fit value $A^\prime_2$=-0.37$\pm$0.04 is not consistent with the prediction of either 
 a pure $E1$-$E2$ transition, or a  pure $E1$-$M1$ transition. The best fit is shown as the red solid curve in Fig.\ref{fig7}.  
The previous measurement of the transition (1187 keV, $2^+ \rightarrow$ 80 keV, $2^+$) was performed in a Coulomb excitation experiment and it reported a mixture of $E2$ and $M1$ transitions with the mixture parameter $\delta$=-9.0$\pm$1.5~\cite{McGowanGd, NDSGd158}, where $\delta$ is defined as the ratio of $E2$ to $M1$ transition~\cite{Frauenfelder, Biedenharn, Arns}. It is noted that the angular distribution is expected to show an interference effect caused by the mixture of two multipoles in a single transition. The coefficient $A^\prime_2$ of the angular correlation function can be calculated~\cite{Frauenfelder, Biedenharn, Arns} when the transition is mixed with a mixture parameter $\delta$ and it is given as
\begin{eqnarray}
A^\prime _2=\frac{0.175+0.510\delta-0.0536\delta^2}{1+\delta ^2},  
\end{eqnarray}
where $A^\prime _4$=0, since the first transition is $E1$. The value of $A^\prime_2$ is estimated by Eq.(7) to be -0.11$\pm$0.01 for  the previously reported value $\delta$=-9.0$\pm$1.5 and its prediction is drawn in the black  solid curve in Fig.\ref{fig7}. Our data are inconsistent with this value. 

If we fit the data with $\delta$ as a free parameter, we obtain $\delta=-1.5^{+1.5}_{-0.5}$, which gives $A^\prime_2$=-0.22 from Eq.(7). The prediction is barely consistent with the data. We note that the previous measurement of the transition was measured by comparing the ratio of the 1107-keV rate at two different angles (0$^\circ$  and  90$^\circ$) with respect to the beam~\cite{McGowanGd}. Systematic effects in the previous and the present experiment which measured  the angular correlation function at all angles are rather different. 
The background to our angular correlation analysis due to the coincidence summing effect is at the same level as that of Fig.\ref{fig6} and is estimated to be negligible. 
%Here again, we observe a strong correlation between the two $\gamma$ rays and a very good agreement between Eq.(4) and our data. \newline

Fig.\ref{fig8} shows the angular  correlation  function for the prominent two $\gamma$ rays of the 7382 keV and 1154 keV  in the two-step cascade transitions (8536 keV, $2^- \rightarrow$ 1154 keV, $2^+ \rightarrow$ 0 keV, $0^+$) for $^{156}$Gd. We show in Fig.\ref{figM2H2}(b) the energy  of the two $\gamma$-rays, in which the sum $E_1 + E_2$ is equal to 8536 keV within $\pm$25 keV in the M2H2 sample. We select the two  peaks due to 7382 keV and 1154 keV unambiguously.
For this case, the theoretical prediction for the angular correlation function 
is the same as for the  two-step cascade transition (7937  keV, $2^- \rightarrow$ 1187 keV, $2^+ \rightarrow$ 0 keV, $0^+$)  of  $^{158}$Gd, but the result shown in Fig.\ref{fig8} looks rather different from that of Fig.\ref{fig6}. 
 If we fit the data using Eqs.(4) and (5) with $A^\prime_2$ being a free parameter and with the fixed value of $A^\prime_4$=0.0, we obtain $A^\prime_2$=0.10$\pm$0.04 and the quality of the fit is relatively poor with $\chi ^2 /dof  $=58/35. The prediction with theoretical value $A^\prime_2$=0.25 is also shown as the black dashed curve. 

 We now consider the background levels to each peak of 1154 keV and 7382 keV. The background for the 1154-keV peak caused by the coincidence sum of 1075 keV and 79 keV is estimated to be 0.5\%. 
 We also checked all possible pairs of two $\gamma$ rays in the M2H2 sample whose coincidence sum results in a peak at 7382 keV. We found that  the number of pairs is  more by about a factor of 5 than that for 6750 keV. The pairs of the two $\gamma$ rays are  6345 keV and 1037 keV, which are produced in a cascade transition (8536 keV, $2^- \rightarrow$ 2191 keV, $2^+ \rightarrow$ 1154 keV, $2^+$),  6745 keV and  637 keV, 6427 keV and 955  keV, 6381 keV and 901 keV,  and  6319 keV and  1063  keV. We estimate the coincidence summing probability of all pairs to be about 7.7\% of the total  number of a single photopeak of 7382 keV. Thus, the background to the pairs of the two $\gamma$ rays of 1154 keV and 7382 keV is estimated to be 8.2$\pm$2.0\%.
 Those backgrounds may have smeared the  angular correlation function in addition to the poorer statistics of this sample than that of  the $^{158}$Gd data, as seen in Fig.\ref{figM2H2}.

%the the two-step cascade presented in this figure can be written as 
%\begin{equation}
%W(\theta) \; \propto \;\;  1 + \frac{9}{47} \cos^2(\theta). 
%\end{equation}

%%The agreement between our data and theoretical predictions is satisfactory within error bars.   \\


\subsection{Angular correlation of the two $\gamma$ rays for the continuum}

We also studied the angular correlation of the two $\gamma$ rays emitted in the continuum transitions. In this analysis, we used  only the two $\gamma$ rays from the M2H2 sample for simplicity. 

%%If the $\gamma$ rays selected in our analysis are not part of the discrete decay cascades, by far the most common case, we expect to measure a uniform distribution of their angular correlation. Indeed, the properties of continuous $\gamma$ rays do not allow to identify the cascade (energy, spin, parity) from which the angular correlation can be derived. By taking a large amount of randomly selected $\gamma$ rays, the angular correlation must be lost.\newline

%As in the case of discrete decays, we studied the angular correlation of the two $\gamma$ rays  in the continuum  for both $^{156}$Gd and $^{158}$Gd. 
In addition, we required that the energies of the two $\gamma$ rays are within nine predefined energy ranges that avoid the energies of the strong discrete photopeaks that we have investigated above. 
The nine energy ranges ($a<E_1, E_2<b)$ are chosen as follows: $a-b$ MeV=(1) 1.5-3.5 MeV, (2) 1.5-3.5 MeV,  (3) 1.5-4.5 MeV, (4) 1.5-6.5 MeV, (5) 2.5-4.5 MeV, (6) 2.5-5.5 MeV, (7) 3.5-5.5 MeV, (8) 3.5-6.5 MeV, and (9) 4.5-5.5 MeV. They have been superimposed on the $\gamma$-ray spectra of $^{157}$Gd($n, \gamma$)  and $^{155}$Gd($n, \gamma$) reactions in Fig.\ref{fig9}. Fig.\ref{fig9}(a) and Fig.\ref{fig9}(b) were taken from Fig.4 (Ref.~\cite{Hagiwara}) and Fig.\ref{fig12}(left) (Ref.~\cite{Tanaka}) of our previous publications, respectively.  

Figs.\ref{fig10} and \ref{fig11} present exemplarily the results of angular correlation functions for the energy ranges (2), (4) and (7) for the  $^{158}$Gd and $^{156}$Gd data. The error bars displayed in the figures include  both  statistical and systematic uncertainties. 
We analysed the angular correlation functions for all energy ranges, assuming a form $W(\theta) =C[1+A_2P_2(cos\theta)]$, where a constant $C$ and  the coefficient $A_2$ are  the free  parameters. The results for the  coefficient $A_2$ for all the energy ranges are shown in Fig.\ref{fig12}, where the uncertainty of the coefficient $A_2$ is determined by $\chi ^2$= $\chi ^2_{minimum}+1.0$.    
The values of the coefficient $A_2$ in any energy range are consistent with 0 within a few \%. Hence, we observe no significant angular correlations for any combinations of two  $\gamma$ rays from the continuum. 
%We observe for the entire energy range no angular correlations between the two $\gamma$ rays. All angular correlations between the two $\gamma$  rays for both $^{156}$Gd and $^{158}$Gd are consistent with null angular correlation, namely $|A^\prime_2|$ less than a few \%.  
%%
%%\begin{table}[ht]
%%\centering
%%\caption{Summary of the coefficient $A_2$ of the angular corrrelation function for various energy ranges of the two $\gamma$ rays %%$E_1$ and  $E_2$ in the continuum.  The angular corrrelation function is assumed to be  $W(z)=C(1+A_2P_2(z))$.    
%%}\label{tab:Coefficient}
%%\begin{tabular}{ccc} 
%% \hline
%% \hline
%%&  $^{158}$Gd &  $^{156}$Gd \\
%%Energy interval of $E_1$ and  $E_2$ & Coefficient $A_2$ & Coefficient $A_2$ \\
%% (MeV) &  & \\
%% \hline
%%(1) (1.5, 2.5) & 0.062$\pm$0.097 & 0.041$\pm$0.073 \\
%%(2) (1.5, 3.5) & 0.029$\pm$0.057 & 0.017$\pm$0.048 \\
%%(3) (1.5, 4.5) & -0.004$\pm$0.030 & -0.010$\pm$0.42 \\
%%(4) (1.5, 6.5) & 0.006$\pm$0.038 & 0.032$\pm$0.034 \\
%%(5) (2.5, 4.5) & -0.021$\pm$0.057 & -0.034$\pm$0.050  \\
%%(6) (2.5, 5.5) &  -0.021$\pm$0.045 & -0.001$\pm$0.043 \\
%%(7) (3.5, 5.5) & -0.030$\pm$0.054  & -0.001$\pm$0.055 \\
%%(8) (5.0, 6.5) & -0.005$\pm$0.051 & 0.042$\pm$0.045 \\
%%(9) (4.5, 5.5) &  0.011$\pm$0.073 & 0.068$\pm$0.072 \\
%%\hline
%%\hline
%%\end{tabular}
%%\end{table}

% FIGURE BEGINF


\begin{figure}[!htbp]
  \begin{center}
	\includegraphics[width=15.0cm,scale=1.0]{EnergyRange.pdf}
  \caption{\label{fig9}Energy ranges (1)-(9)($E_1$-$E_2$) of the two $\gamma$ rays are shown in
  arrows over the single energy spectrum of (a) $^{158}$Gd and (b) $^{156}$Gd. Fig.\ref{fig9}(a) and Fig.\ref{fig9}(b) were taken from Fig.4 (Ref.~\cite{Hagiwara}) and Fig.9(left) (Ref.~\cite{Tanaka}) of our previous publications, respectively. }
	\end{center}
\end{figure}


\begin{figure}[!htbp]
  \begin{center}
 \includegraphics[width=4.5cm,scale=1.0]{Fig10-1.pdf}
 \includegraphics[width=4.5cm,scale=1.0]{Fig10-2.pdf}
 \includegraphics[width=4.5cm,scale=1.0]{Fig10-3.pdf}
  \caption{\label{fig10}Measurement of the angular correlation between two $\gamma$ rays ($E_1$ and $E_2$) from the continuum for $^{158}$Gd decays. The energy range for $E_1$ and $E_2$ is (a)(1.5 MeV, 3.5 MeV), (b) (1.5 MeV, 6.5 MeV), and (3.5 MeV, 6.5 MeV).  The red lines show the best fit, which is consistent with a flat distribution.}
	\end{center}
\end{figure}

\begin{figure}[h]
  \begin{center}
%  	\includegraphics[width=14.0cm,scale=1.0]{Fig10-123.pdf}
 \includegraphics[width=4.5cm,scale=1.0]{Fig11-1.pdf}
 \includegraphics[width=4.5cm,scale=1.0]{Fig11-2.pdf}
 \includegraphics[width=4.5cm,scale=1.0]{Fig11-3.pdf}
  \caption{\label{fig11}Measurement of the angular correlation between two $\gamma$ rays ($E_1$ and $E_2$) from the continuum for $^{156}$Gd decays. The energy range for $E_1$ and $E_2$ is (a)(1.5 MeV, 3.5 MeV), (b) (1.5 MeV, 6.5 MeV), and (3.5 MeV, 6.5 MeV).  The red lines show the best fit, which is consistent with a flat distribution.}
	\end{center}
\end{figure}

\begin{figure}[h]
  \begin{center}
    \includegraphics[width=7.0cm,scale=1.0]{Fig12v1-1.pdf}
    \includegraphics[width=7.0cm,scale=1.0]{Fig12v1-2.pdf}
    \caption{Coefficient $A_2$ ((a) $^{158}$Gd and (b) $^{158}$Gd) of the angular correlation function plotted for various energy ranges ($E_1$-$E_2$) of the two $\gamma$ rays in the continuum. The angular corrrelation function is assumed to be in the form of $W(\theta) =C[1+A_2P_2(cos\theta)]$. $A_2$=0 means no correlation.}
    \label{fig12}
  \end{center}
\end{figure}
% FIGURE END
%\begin{figure}[!htbp]
%  \begin{center}
%	\includegraphics[width=0.9\linewidth]{Fig8.png}
%  \caption{\label{fig8}Measurement of the angular correlation between two $\gamma$ rays from the continuum for %$^{156}$Gd decays (number of counts). The red lines show the best fit for flat distributions.}
%	\end{center}
%\end{figure}


\section{Summary and discussion}

Using the ANNRI Ge spectrometer setup at J-PARC, we have studied  for the first time the angular correlations between the two $\gamma$ rays emitted from $^{155}$Gd and $^{157}$Gd targets after capture of thermal neutrons. 
 %Due to our threshold conditions, the energy range for $\gamma$ rays has been limited from 1 to 8 MeV. \newline

We have shown that the angular correlation functions between the two prominent $\gamma$ rays  produced in the strong two-step cascade transitions from the resonance state can be described with the functional form of Eqs.(4) and (5) predicted by electromagnetic theory~\cite{Frauenfelder, Biedenharn, Rose}. 
For the angular correlation function for the two  $\gamma$ rays  of 6750 keV and 1187 keV in the  two-step cascade transition (7937 keV, $2^- \rightarrow$ 1187 keV, $2^+ \rightarrow$ 0 keV, $0^+$)  of  $^{158}$Gd, our data shown in Fig.\ref{fig6} are consistent with the prediction within 2$\sigma$ level. The background to the angular correlation function is  negligible. 

Next,  we showed in Fig.\ref{fig7}  the angular correlation function for the two $\gamma$ rays of the 6750 keV and 1107 keV in the cascade  (7937  keV, $2^- \rightarrow$ 1187 keV, $2^+ \rightarrow$ 80 keV, $2^+$) and  compared it with the prediction of the electromagnetic theory.  The best fit value to our data is not consistent with the prediction of either a pure $E1-E2$ transition, or a  pure $E1-E1$ transition. The previous measurement of the transition (1187 keV, $2^+ \rightarrow$ 80 keV, $2^+$) reported a mixture of $E2$ and $M1$ transitions with a mixing parameter $\delta$=-9.0$\pm$1.5~\cite{McGowanGd}. Instead, our fit to the angular correlation function  results   $\delta=-1.5^{+1.5}_{-0.5}$, corresponding to $A^\prime_2$=-0.22 from Eq.(7). Our result is not consistent with the previous  measurement.  
We note that the previous measurement and the present experiment which measured  the angular correlation function at all angles use   different experimental methods. Further measurements will be necessary.  
The background to the angular correlation analysis in Fig.\ref{fig7} due to coincidence summing effect is again estimated to be negligible. 

We also studied the angular  correlation  function for the prominent two $\gamma$ rays of the 7382 keV and 1154 keV  in the  two-step cascade transitions (8536 keV, $2^- \rightarrow$ 1154 keV, $2^+ \rightarrow$ 0 keV, $0^+$) for $^{156}$Gd in Fig.\ref{fig8}. For this case, the theoretical angular correlation function 
should be the same as the  two-step cascade transition (7937  keV, $2^- \rightarrow$ 1187 keV, $2^+ \rightarrow$ 0 keV, $0^+$)  of  $^{158}$Gd, but the result shown in Fig.\ref{fig8} are different from Fig.\ref{fig6}. 
 We also checked all possible pairs of two $\gamma$ rays in the M2H2 sample whose coincidence sum results in a peak at 7382 keV and found that  the number of pairs is  more by a factor of 5 than for 6750 keV. We estimated the coincidence summing probability of all pairs to be about 7.7\% of the total  number of a single photopeak of 7382 keV. Thus, the background to the pairs of the two $\gamma$ rays of 1154 keV and 7382 keV is estimated to be 8.2$\pm$2.0\%. 
 Those background may have smeared the  angular correlation function in addition to the poorer statistics of this sample than that of  the $^{158}$Gd data (Fig.\ref{figM2H2}). 
 %%We note again that the prompt $\gamma$ rays have sometimes a chance for the background  due to the coincidence summing probability. This background makes a peak at the same energy and cannot be removed from the sample as long as a detector has a finite size.  

%This analysis is a confirmation of the high quality of the ANNRI detector and confirmation of theoretical analysis of the electromagnetic transitions in the heavy nuclei. 
Next, we have studied the angular correlations between two  $\gamma$ rays produced from the continuum transitions, assuming that the angular correlation can be written in a form  $W(\theta) =C[1+A_2P_2(cos\theta)]$. We found that the value of the coefficient $A_2$ is consistent with 0 within uncertainties, typically 0.05 and less than 0.1, as shown in Fig.\ref{fig12}. Hence, we found no angular correlations, for any two  $\gamma$ rays in the continuum for energies below 6.5 MeV. 
%The uncertainty of the coefficient $A_2$ is determined by $\chi ^2$= $\chi ^2_{minimum}+1.0$, where both the statistical and systematic errors are added in quadrature.  We excluded the energy region dominated by the first and the most energetic $\gamma$ ray. We showed in  section 3.1 that the two $\gamma$ rays from the strong 2-step cascade transition  show a characteristic angular correlation. 

This result agrees with our expectations since we picked  random pairs of two $\gamma$ rays in the cascade transition and excluded the prominent strong photopeaks from the pairs. 
%%We showed that our Monte Carlo model (ANNRI-Gd Model) generates the gross 
%%$\gamma$ ray spectrum and agrees with our spectrum measured in the thermal $^{\rm 155}$Gd,  $^{\rm 157}$Gd and $^{\rm nat}$Gd(n, $\gamma$) reactions for the observed multiplicity from 1 to 4 in Fig.12(left) of Ref.~\cite{Hagiwara} and Fig.7(left) of Ref.~\cite{Tanaka}. %We also showed in the Monte Carlo study a feature of the electromagnetic transitions from the resonance state and presented the energy spectra for the first, the second, the third, the fourth and the fifth  $\gamma$ rays.  
%%This suggested that the primary $\gamma$ rays above 5 MeV are generated from the first $E1$ %%transition and that the $\gamma$ rays in the continuum below 6 MeV are produced mainly via the second,  third and fourth transitions.
The mean multiplicity of  $\gamma$ rays produced in the neutron capture reaction is about 5  for $E_{\gamma}>$0.2 MeV. Since we pick a random pair of two $\gamma$ rays in the continuum, the probability that the same pair is selected from the definite spin-parity states must be very small and, as a  result, we expect that they show no angular correlations. 
We note that the $\gamma$ rays from the continuum represent approximately 93{\%} (97\%) of $\gamma$ rays produced in the thermal neutron capture of  $^{\rm 157}$Gd(n, $\gamma$)  reaction  ($^{\rm 155}$Gd(n, $\gamma$) reaction) for $E_{\gamma}>$0.11 MeV~\cite{Hagiwara, Tanaka}.
%%Regardless of such naive predictions, we have confirmed negligibly small angular correlations for the  two $\gamma$ rays within a few \% in the data, though we excluded the strong primary $\gamma$ rays from the pairs. 

In summary, our study of the angular correlation for both  the  two $\gamma$ rays from the strong two-step cascade transition and for the randomly chosen  two $\gamma$ rays in continuum is an important information for the ongoing and future experiments using gadolinium for neutrons tagging in a liquid-scintilator detector or in a water-Cherenkov detector. 

%%%%Sakuda added 20230202 


%%%%Sakuda added 20230202 end  

\section*{Acknowledgement}
\label{sec:Acknowledgements}
%==================================================================================================%

This work is supported by the JSPS Grant-in-Aid for Scientific Research on Innovative Areas 
(Research in a proposed research area) No. 26104006 and also by the JSPS Grant-in-Aid for Scientific Research (C) No. 20K03989. It benefited from the use of the 
neutron beam of the JSNS and the ANNRI detector at the Materials and Life Science Experimental 
Facility of the Japan Proton Accelerator Research Complex.


\newpage


\begin{thebibliography}{9}

\bibitem{Hagiwara}
K. Hagiwara \textit{et al.} [ANNRI-Gd Collaboration], Prog. Theor. Exp. Phys.  \textbf{2019}, 023D01 (2019).
\bibitem{Tanaka} 
T. Tanaka \textit{et al.} [ANNRI-Gd Collaboration], Prog. Theor. Exp. Phys. \textbf{2020},, 043D02 (2020). 
\bibitem{Mughabghab2006}
S. F. Mughabghab, \textit{Atlas of Neutron Resonances, Fifth Edition: Resonance Parameters and Thermal Cross Sections}, Z = 1-100 (Elsevier, Amsterdam, 2006).
\bibitem{Leinweber}
G. Leinweber, D. P. Barry, M. J. Trbovich, J. A. Burke, N. J. Drindak, H. D. Knox, R. V. Ballad, R. C. Block, Y. Danon, and L. I. Severnyak, Nucl. Sci. Eng. \textbf{154}, 261 (2006).
\bibitem{Choi}
H. D. Choi, R. B. Firestone, M. S. Basunia, A. Hurst, B. Sleaford, N. Summers, J. E. Escher, Zs. R\'{e}vay,
L. Szentmikl\'{o}si, T. Belgya and M. Krti\'{c}ka, Nucl. Sci. Eng. \textbf{177}, 219 (2014).
\bibitem{nTOF}
M. Mastromarco \textit{et al.} [n\_TOF Collaboration], Eur. Phys. J. A\textbf{55}, 9 (2019). 
\bibitem{Dchooz}
Y. Abe \textit{et al.} [Double Chooz Collaboration], Phys. Rev. Lett. \textbf{108}, 131801 (2012). 
\bibitem{RENO}
J. K. Ahn \textit{et al.} [RENO Collaboration], Phys. Rev. Lett. \textbf{108}, 191802 (2012). 
\bibitem{DayaBay}
F.P. An \textit{et al.} [Daya Bay Collaboration], Phys. Rev. Lett. \textbf{108}, 171803 (2012);D Adey \textit{et al.} [Daya Bay Collaboration], Phys. Rev. Lett. \textbf{121}, 241805 (2018).
\bibitem{SterileCombine}
P. Adamson \textit{et al.} [MINOS, MINOS+, Daya Bay and Bugey-3], Phys. Rev. Lett. \textbf{125}, 071801 (2020).
\bibitem{NEOS}
Y. J. Ko \textit{et al.} [NEOS Collaboration], Phys. Rev. Lett., \textbf{118}, 121802 (2017).
\bibitem{STEREO}
H. Almaz\'{a}n \textit{et al.}[STEREO Collaboration], Phys. Rev. Lett. \textbf{121}, 161801 (2018);\textit{ibid.}, Eur. Phys. J. \textbf{55}, 183 (2019).
\bibitem{DANSS}
H. Alekseev \textit{et al.}[DANSS Collaboration], Phys.Lett. \textbf{B787}, 56 (2018).
%\bibitem{PROSPECT}
%H. Alekseev \textit{et al.}[DANSS Collaboration], Phys.Lett. \textbf{B787}, 56 (2018).
\bibitem{Neutrino4}
A. P. Serebrov \textit{et al.}[Neutrino-4 Collaboration], JETP Lett. \textbf{109}, 213 (2019).
\bibitem{JSNS2}
S. Ajimura \textit{et al.}[JSNS$^2$-II Collaboration],  arXiv: 2012.10807 [hep-ex] (2020); F. Suekane, PoS (NuFact2019) 129.
\bibitem{PANDA}
S. Oguri, Y. Kuroda, Y. Kato, R. Nakata, Y. Inoue, C. Ito, and M. Minowa, Nucl. Instrum. Meth. A\textbf{757}, 33 (2014).
%\bibitem{ISMRAN}
%D. Mulmule, P.K. Netrakanti, L.M. Panta  and B.K. Nayak,  JINST \textbf{15},  P04021 (2020).
%%\bibitem{Sakurai} 
%%Sakurai, Y. and Kobayashi, T., J. Nucl. Sci. Technol.39, 1294-1297 (2002).
\bibitem{GdNCT} 
S.L. Ho, H.Yue, T. Tegafaw, M. Y. Ahmad, S. Liu, S-W. Nam, Y. Chang, and G. H. Lee, ACS Omega \textbf{2022}, 7, 2533.  
\bibitem{LZ}
K. Pushkin \textit{et al.}[LZ Collaboration], Nucl. Instrum. Meth. A\textbf{936}, 162 (2019). 
\bibitem{Xenon}
E. Aprile \textit{et al.}  (XENON Collaboration), JCAP \textbf{11}, 031 (2020).
\bibitem{Vagins}
 J. F. Beacom and M. R. Vagins, Phys. Rev. Lett. \textbf{93}, 171101 (2004).
\bibitem{SkGd}
H. Sekiya (for Super-Kamiokande Collaboration), PoS \textbf{ICHEP2016}, 982 (2016).
\bibitem{EGADS}
L. Marti \textit{et al.} [Super-Kamiokande Collaboration],  Nucl.Instrum.Meth.A \textbf{959}, 163549 (2020).
\bibitem{Frauenfelder}
H. Frauenfelder,  Annu. Rev. Nucl. Sci. \textbf{2}, 129 (1953).
\bibitem{Biedenharn}
I. C. Biedenharn and M. E. Rose, Rev. Mod. Phys. \textbf{25}, 729 (1953).
\bibitem{Rose}
 M. E. Rose, Elementary Theory of Angular Momentum, John Wiley \& Sons, Inc., New York, 1957. 
\bibitem{FIFRELIN}
A. Chalil, T. Materna1, O. Litaize, A. Chebboubi, F. Gunsing, Eur. Phys. J. A 58, 30(2022).
\bibitem{Kimura2012}
A. Kimura \textit{et al.}, J. Nucl. Sci. Technol.  \textbf{49}, 708 (2012).
\bibitem{Kin2011}
T. Kin \textit{et al.}, J. Korean Phys. Soc. \textbf{59}, 1769 (2011).
\bibitem{Kino2011}
K. Kino \textit{et al.}, Nucl. Instrum. Meth. A\textbf{626}, 58 (2011).
\bibitem{Kino2014}
K. Kino \textit{et al.}, Nucl. Instrum. Meth. A\textbf{736}, 66 (2014).
\bibitem{Rose1}
 M. E. Rose, Phys.Rev. \textbf{91}, 610 (1953).
\bibitem{Camp}
 D. C. Camp and A. L. Van Lehn, Nucl. Instrum. Meth. \textbf{76}, 192 (1969).
\bibitem{Utsunomiya}
H. Utsunomiya,H. Akimune, K. Osaka, T. Kaihori, K. Kurutaka and H. Harada, Nucl. Instrum. Meth. A\textbf{548}, 455 (2005).
\bibitem{Terada}
K. Terada \textit{et al.}, Journal of Nuclear Science and Technology \textbf{53}, 1881 (2005).
%\bibitem{HamiltonGd156}
%J.H. Hamilton, P.E. Little, A.V. Ramayya, E. Collins, N.R. Johnson, J.J. Pinajian and A.F. Kiuk,  Phys. Rev. C\textbf{5},  (2017).
\bibitem{McGowanGd}
F.K. McGowan and W.T. Milner, Phys. Rev. C\textbf{23}, 1926 (1981).
\bibitem{NDSGd158}
N. Nica, Nuclear Data Sheets \textbf{141}, 326 (2017).
\bibitem{Arns}
R.G. Arns and M.L. Wiedenbeck, Phys. Rev. \textbf{111}, 1631 (1958).

%\bibitem{Nagamiya2012:JPARC}
%S. Nagamiya \textit{et al.}, Prog. Theor. Exp. Phys. \textbf{02B001} (2012).
%\bibitem{Baramsai2013:DANCE}
%B. Baramsai \textit{et al.}[DANCE Collaboration], Phys. Rev. C \textbf{87}, 044609 (2013).
%\bibitem{Chyzh:DANCE}
%A. Chyzh \textit{et al.}[DANCE Collaboration],, Phys. Rev. C \textbf{84}, 014306 (2011).
%\bibitem{Kroll2013:DANCE}
%J. Kroll \textit{et al.}, Phys. Rev. C \textbf{88}, 034317 (2013).
%\bibitem{Becvar1998:DICEBOX}
%F. Be\v{c}v\'{a}\v{r} et al., Nucl. Instrum. Meth.A\textbf{417}, 434(1998).
%\bibitem{Groshev1959:GdGam}
%L. V. Groshev, A. M. Demidov, V. N. Lutsenko, and V. I. Pelekhov, J. Nucl. Energy 9, 50 (1959); L. V. Groshev, A. M. Demidov, V. A. Ivanov, V. N. Lutsenko, and V. I. Pelekhov, Bull. Acad. Sci. USSR
%26, 1127 (1963);L. V. Groshev, A. M. Demidov, V. I. Pelekhov, L.L. Sokolovskii, G.A. Bartholomew, A. Doveika, K.M. Eastwood, and S. Monaro, Nucl. Data Tables, A5, 1(1968).
\bibitem{Valenta2015:GdTwoStepGamCasc}
S. Valenta, F. Be\v{c}v\'{a}\v{r}, J. Kroll, M. Krti\v{c}ka, and I. Tomandl, Phys. Rev. C \textbf{92}, 064321 (2015).
%\bibitem{Agostinelli2003:Geant4}
%S. Agoztenelli \textit{et al.} [Geant4 Collaboration], Nucl. Instrum. Meth., \textbf{A506}, 250 (2003).
%\bibitem{Allison2006:Geant4}
%J. Allison \textit{et al.} [Geant4 Collaboration], IEEE Trans. Nucl. Science \textbf{53}, 270 (2006).
\bibitem{DayaBay-sim}
D. Adey \textit{et al.} [Daya Bay Collaboration], Phys. Rev. D\textbf{100}, 052004 (2019). 

%\bibitem{Voignier}
%J. Voignier, S. Joly, and G. Grenier, Nucl. Sci. Eng. 93, 43 (1986).
%\bibitem{Kopecky93}
%J. Kopecky, M. Uhl nd R.E. Chrien, Phys. Rev. C \textbf{47}, 312 (1993).

\end{thebibliography}

\end{document}
\vspace{0.3cm}