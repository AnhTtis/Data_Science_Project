%
\documentclass[final,5p,times,twocolumn]{elsarticle}

%
%

%
%
%
%
%
%
%
%

\journal{Scripta Materialia}

\usepackage{graphicx}%
\usepackage{dcolumn}%
\usepackage{bm}%
\usepackage{miller} %
\usepackage{lineno} %
%
%
%
%
%
%
%
\usepackage{siunitx}
%

%
%
%
%
%
%
%

\newcommand{\MAX}{Ti$_3$AlC$_2$}
\newcommand{\ie}{\emph{i.e.}}
\newcommand{\etal}{\emph{~et al.}}
\renewcommand\thefigure{SM\arabic{figure}} 

\begin{document}

%

\begin{frontmatter}

\title{Features of a nano-twist phase in the nanolayered \MAX
\\
--
\\
Supplementary Material}


\author[lem3,labex]{Julien Guénolé}
\author[lem3,labex]{Vincent Taupin}

\affiliation[lem3]{organization={Université de Lorraine, CNRS, Arts et Métiers, LEM3},
            city={Metz},
            postcode={57070},
            country={France}}

\affiliation[labex]{organization={Labex DAMAS, Université de Lorraine},
            city={Metz},
            postcode={57070},
            country={France}}


\author[lms,ls]{Maxime Vallet}
\affiliation[lms]{organization={Laboratoire Mécanique des Sols, Structures et Matériaux, CentraleSupélec, CNRS UMR 8579, Université Paris-Saclay},
            city={Gif-sur-Yvette},
            postcode={91190},
            country={France}}
\affiliation[ls]{organization={Laboratoire Structures, Propriétés et Modélisation des Solides, CentraleSupélec, CNRS UMR 8580, Université Paris-Saclay},
            city={Gif-sur-Yvette},
            postcode={91190},
            country={France}}         

\author[cmse]{Wenbo Yu}
\affiliation[cmse]{organization={Center of Materials Science and Engineering, School of Mechanical and Electronic Control Engineering, Beijing Jiaotong University},
            city={Beijing},
            postcode={100044},
            country={China}}  

\author[lem3,labex]{Antoine Guitton\corref{cor1}}
 \cortext[cor1]{antoine.guitton@univ-lorraine.fr}

%
             %
             
%
%
%
%

%
%
%
%

\end{frontmatter}

%

\section{Experimental sample}

Fig.~\ref{fig_SM_TEM} shows a low magnification micrography of the experimental sample presented in the manuscript as Figure~1. This bright-field STEM micrography evidences an extended Ti$_3$AlC$_2$ grain embedded in TiC matrix. The presence of Al$_2$O oxide and of a large void at the vicinity of the Ti$_3$AlC$_2$ grain is also visible. The area indicated by a white empty square corresponds to the filtered HR-STEM micrography in HAADF mode, namely Figure~1, in the main manuscript.

\begin{figure}[hb]
\includegraphics[width=0.99\columnwidth]{figure_SM_TEM.png}
\caption{\label{fig_SM_TEM} Low magnification micrography of the experimental sample shown in Figure~1 (white empty square). Bright-field STEM micrography of Ti$_3$AlC$_2$ embedded in TiC matrix. Note the presence of Al$_2$O oxide and of a large void at the vicinity of the Ti$_3$AlC$_2$ grain. }
\end{figure}

\section{Disclination model of the nano-twist phase}

\noindent  The main aspects of the disclination model is presented here. For more details, the reader is referred to papers mentioned in the main text and references therein. A small deformation framework is chosen. The material displacement vector $\bf u$ is a single-valued continuous field. The
distortion tensor is the gradient of the displacement
${\bf U} = {\bf grad \hspace{1mm} u}$. It is a curl-free field:

\begin{equation} \label{curlU}
{\bf curl \hspace{1mm} U} = 0.
\end{equation}

\noindent The strain field $\boldsymbol \epsilon$ is the symmetric part of $\bf U$, while the
rotation field $\boldsymbol \omega$ is the skew-symmetric part. The rotation vector $\vec{\boldsymbol \omega}$ reads:

\begin{equation}\label{xomega}
\vec{{\boldsymbol \omega}} = \frac{1}{2} {\bf curl\, u}.
\end{equation}

\noindent This rotation field is also a single-valued and continuous quantity. The gradient of the rotation vector is the curvature tensor ${\boldsymbol \kappa}$:

\begin{equation}\label{curv}
{\boldsymbol \kappa} = {\bf grad} \vec{{\boldsymbol \omega}}. 
\end{equation}

\noindent It is also called the bend-twist tensor. Like the material distortion ${\bf U}$, the material curvature tensor is a curl-free tensor. To introduce disclinations in the body, we use the deWit's disclination density tensor ${\boldsymbol \theta}$. It has components $\theta_{ij}=\Omega_i t_j$ in the cartesian frame $({\bf e}_x, {\bf e}_y, {\bf e}_z)$. The components correspond to a Frank vector angle $\Omega_i$ about ${\bf e}_i$ per unit resolution surface $\Delta S$, for a line vector component along direction ${\bf e}_j$, $t_j$. Disclinations are indeed line defects, like dislocations, but they introduce a discontinuity of the elastic (or plastic, up to a sign) rotation that corresponds to the Frank vector. Considering a finite surface $S$ of unit normal 
${\bf n}$, which is thread by an ensemble of disclination lines corresponding to a non-zero disclination density, the Frank vector reads:

\begin{equation}\label{Frankvector1}
{\boldsymbol \Omega} =  \int_S {\boldsymbol \theta} \cdot \bf{n} {\it dS}.
\end{equation}

\noindent Because of the elastic/plastic rotation discontinuity the plastic ${\boldsymbol \kappa}_p$ and elastic ${\boldsymbol \kappa}_e$ parts building the total material curvature contain incompatible
components that are not curl-free. The incompatibility of elastic and plastic curvatures writes:

\begin{equation}\label{incompk}
{\bf curl} \hspace{1mm} {\boldsymbol \kappa}_e = - {\bf curl } \hspace{1mm}
{\boldsymbol \kappa}_p = {\boldsymbol \theta}.
\end{equation}

\noindent Although it is not the case in the present simulations, dislocations can also be introduced in addition to disclinations. Such is done by using the Nye's dislocation density tensor ${\boldsymbol \alpha}$. It has components $\alpha_{ij}=B_i t_j$, corresponding to a Burgers vector length $B_i$ along direction ${\bf e}_i$ per unit resolution surface $\Delta S$, and to a dislocation line vector component $t_j$ along direction ${\bf e}_j$. Due to the elastic/plastic displacement discontinuity, ${\it i.e.}$ the Burgers vector, the plastic ${\boldsymbol \epsilon}_p$ and elastic ${\boldsymbol \epsilon}_e$ strains building the total material strain contain incompatible, non curl-free components, such that:

\begin{eqnarray} \label{alphamod}
{\bf curl} \hspace {1mm} {\boldsymbol \epsilon}_e &=& +
{\boldsymbol \alpha} + {\boldsymbol \kappa}_e^t - tr({\boldsymbol
\kappa}_e){\bf I} \label{alphamod1} \\
{\bf curl} \hspace {1mm} {\boldsymbol \epsilon}_p &=& -
{\boldsymbol \alpha} + {\boldsymbol \kappa}_p^t - tr({\boldsymbol
\kappa}_p){\bf I}. \label{alphamod2}
\end{eqnarray}

\noindent The above incompatibility equations show how elastic strains are related to the presence of dislocations and disclination-induced elastic curvatures. The elastic strains and curvatures must satisfy the higher-order quasi-static balance equation:

\begin{equation} \label{Mindlin}
{\bf div} \hspace{1mm} {\boldsymbol \sigma}+ \frac{1}{2} {\bf curl}
\hspace{1mm} {\bf div} \hspace{1mm} {\bf M}= 0.
\end{equation}

In the above equation, ${\boldsymbol \sigma}$ is the Cauchy stress tensor. The tensor ${\bf M}$ is called couple stress tensor and is the work-conjugate of curvatures, just like the stress tensor is that of strains. Usually, the elastic constitutive relations are taken in the form:

\begin{eqnarray}
{\boldsymbol \sigma} &=& {\bf C} : {\boldsymbol \epsilon}_e + {\bf D} :
{\boldsymbol
\kappa}_e  \label{T} \\
{\bf M} &=& {\bf A} : {\boldsymbol \kappa}_e + {\bf B} :
{\boldsymbol \epsilon}_e. \label{M}
\end{eqnarray}

\noindent In the above ${\bf A}$, ${\bf B}$, ${\bf C}$ and ${\bf D}$ are elasticity tensors. In the present simulations, only the classical tensor ${\bf C}$ is considered such that couple stresses are neglected. This choice does not alter the predicted internal stresses due to disclinations. The elasticity moduli are in Voigt notation in the hexagonal crystal frame $C_{11} = C_{22} = \SI{353}{\giga\pascal}$, $C_{12} = \SI{75}{\giga\pascal}$, $C_{13} = \SI{69}{\giga\pascal}$, $C_{33} = \SI{296}{\giga\pascal}$ and $C_{44} = C_{55} = C_{66} = \SI{119}{\giga\pascal}$. All equations are numerically solved with a spectral code taking advantage of Fast Fourier Transform (FFT) algorithms. The incompatibility equations are solved in the Fourier space with the use of centered difference schemes for the evaluation of spatial derivatives. The stress equilibrium equation is solved by using the accelerated scheme together with the use of the rotated scheme for the evaluation of the modified Green operator. In the simulation shown in the main text, the FFT grid size is $256 \times 256 \times 32$ voxels in the X, Y and Z directions. The voxel size is $\SI{0.4}{\nano\meter}$ in all directions. Periodic boundary conditions are used in all directions, like in molecular dynamics simulations. \\

%
%

\begin{figure}[t]
\includegraphics[width=0.9\columnwidth]{Disclination-Schematics.png}
\caption{\label{DisclinLoops} Illustration of the use of the disclination model to build a nano-twist phase. See the text for details.}
\end{figure}

The figure (\ref{DisclinLoops}) explains how the disclination concept is used to build a nano-twist, penny-shaped, phase. The big cube is the matrix material. In this matrix the crystal is unrotated, as sketched by the two small cube orientations near the bottom and the top surfaces. To start building the nano-twist phase, we first introduce a positive twist disclination loop near the bottom surface. The Frank vector is shown in the figure, it is perpendicular to the surface delimited by the disclination loop. By crossing upwards the surface delimited by the disclination loop, the elastic rotation encounters a jump equal to the Frank vector of the disclination loop. The jump of crystal rotation is shown by the rotating arrow inside the disclination loop, and the new orientation of the crystal is shown by the small cube in the middle of the material. The crystal is twisted with respect to the matrix phase. Finally, to finish building the nano-twist phase, we must add a negative disclination loop near the top surface, with the same Frank vector magnitude, such that when crossing the surface delimited by this disclination loop, the crystal rotation encounters an opposite jump and recovers the initial value in the matrix. The two disclination loops are built with appropriate disclination density components ${\theta}_{zx}$ and ${\theta}_{zy}$ spread on a few voxels and such that the Frank vector magnitude according to equation (\ref{Frankvector1}) is 30 degrees. Note that elastic stresses and strains are continuous and not singular close to disclination lines because we use continuous disclination densities.  \\


\end{document}
