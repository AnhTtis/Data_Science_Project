\documentclass[a4paper, oneside, reqno, 12pt]{amsart}
\usepackage[utf8]{inputenc}
\usepackage{amsmath, amssymb, amsfonts}
\usepackage{fullpage}
\usepackage{textcomp, cmap, comment}
\usepackage{mathtools}
\usepackage[nobysame, initials]{amsrefs}
\usepackage[english]{babel}
%\usepackage{cite}

\usepackage{comment}
\usepackage{color, xcolor}
\usepackage{graphicx}
\usepackage{pgf}
\usepackage{pgfplots}
\usepackage{relsize}

\usepackage[justification=centering]{caption}
\usepackage{subcaption}
\captionsetup[subfigure]{labelfont=rm}
\usepackage{wrapfig}

\theoremstyle{plain}
\newtheorem{theorem}{Theorem}
\newtheorem{conjecture}[theorem]{Conjecture}
\newtheorem{statement}[theorem]{Statement}
\newtheorem{proposition}[theorem]{Proposition}
\newtheorem{example}[theorem]{Example}
\newtheorem{lemma}[theorem]{Lemma}
\newtheorem{corollary}[theorem]{Corollary}
\newtheorem{observation}[theorem]{Observation}
\newtheorem{keyobservation}[theorem]{Observation}
\theoremstyle{definition}
\newtheorem{problem}[theorem]{Problem}
\newtheorem{definition}[theorem]{Definition}
\newtheorem{note}[theorem]{Notation}
\newtheorem{fact}[theorem]{Fact}
\theoremstyle{remark}
\newtheorem{remark}[theorem]{Remark}

\DeclareMathOperator{\cost}{cost}
\DeclareMathOperator{\pr}{pr}
\DeclareMathOperator{\dist}{dist}
\DeclareMathOperator{\conv}{conv}
\DeclareMathOperator{\aff}{aff}
\DeclareMathOperator{\vol}{vol}
\DeclareMathOperator{\tr}{tr}
\DeclareMathOperator{\rank}{rank}
\DeclareMathOperator{\cone}{cone}

\newcommand{\Ball}{B}
\newcommand{\BallK}{K} 

\usepackage{mathrsfs}

\usepackage{hyperref}
\hypersetup{
    colorlinks   = true, 
    urlcolor     = blue, 
    linkcolor    = red, 
    citecolor   = green
}

\title{Intersecting balls induced by a geometric graph II}
\author[P.~Barabanshchikova, A.~Polyanskii]{{Polina Barabanshchikova and Alexander~Polyanskii}}

\address{Polina Barabanshchikova,
\newline\hphantom{iii} Moscow Institute of Physics and Technology, Institutskiy per. 9, Dolgoprudny, Russia 141700
}
\email{\href{mailto:barabanshchikova.piu@phystech.edu}{barabanshchikova.piu@phystech.edu}}

\address{Alexander Polyanskii,
\newline\hphantom{iii} Institute of Mathematics and Informatics, Bulgarian Academy of Sciences Bulgaria, Sofia 1113, Acad. G. Bonchev Str., Bl. 8
\newline\hphantom{iii} Moscow Institute of Physics and Technology, Institutskiy per. 9, Dolgoprudny, Russia 141700
}
\email{\href{mailto:alexander.polyanskii@yandex.ru}{alexander.polyanskii@gmail.com}}
\urladdr{\url{http://polyanskii.com}}

\keywords{Infinite descent, convex optimization, Tverberg theorem, max-sum tree, max-sum matching, alternating cycle}
\subjclass[2010]{51K99, 05C50, 51F99, 52C99, 05A99}



\begin{document}

\thispagestyle{empty}

\begin{abstract}
For a graph whose vertices are points in $\mathbb R^d$, consider the closed balls with diameters induced by its edges. The graph is called a \textit{Tverberg graph} if these closed balls intersect. 

A \textit{max-sum tree} of a finite point set $X \subset \mathbb R^d$ is a tree with vertex set $X$ that maximizes the sum of Euclidean distances of its edges among all trees with vertex set $X$. Similarly, a \textit{max-sum matching} of an even set $X \subset \mathbb R^d$ is a perfect matching of $X$ maximizing the sum of Euclidean distances between the matched points among all perfect matchings of~$X$.

We prove that a max-sum tree of any finite point set in $\mathbb R^d$ is a Tverberg graph, which generalizes a recent result of Abu-Affash et al., who established this claim in the plane. Additionally, we provide a new proof of a theorem by Bereg et al., which states that a max-sum matching of any even point set in the plane is a Tverberg graph. Moreover, we proved a slightly stronger version of this theorem.
\end{abstract}

\maketitle

\section{Introduction}

In 1966, Helge Tverberg~\cite{tverberg1966generalization} proved that for any $(r-1)(d+1)+1$ points in $\mathbb{R}^d$, there exists a partition of them into $r$ parts whose convex hulls intersect. This paper studies a variation of Tverberg's theorem recently introduced in~\cite{huemer2019matching, soberon2021tverberg}, and further develops ideas presented in~\cite{pirahmad2022intersecting, barabanshchikova2022intersecting}.

For any graph in this paper, we assume that its vertex set is a finite subset of $\mathbb{R}^d$. The \textit{cost} of a graph $G$ is the sum of Euclidean distances between the pairs of vertices connected by an edge in $G$. We define a \textit{max-sum tree} of a finite point set $X\subset \mathbb{R}^d$ as a tree with vertex set $X$ that maximizes the cost among all trees with vertex set $X$. Similarly, we define a \textit{max-sum matching} of an even point set $X\subset \mathbb{R}^d$ as a perfect matching with vertex set $X$ that maximizes the cost among all perfect matchings with vertex set $X$.

For two points $a,b\in \mathbb R^d$, we denote by $B(ab)$ the closed Euclidean ball with diameter $ab$. A graph $G$ is called a \textit{Tverberg graph} if
\[
    \bigcap_{xy\in E(G)} B(xy) \neq \emptyset.
\]
Similarly, a graph is an \textit{open Tverberg
graph} if the open balls with diameters induced by its edges intersect.

Very recently, Abu-Affash, Carmi, and Maman~\cite{abuaffash2022} proved that a max-sum tree of any finite point set in the plane is a Tverberg graph. In this paper, we generalize their result to higher dimensions.
\begin{theorem}
\label{theorem tree}
A max-sum tree of any finite point set in $\mathbb R^d$ is a Tverberg graph.
\end{theorem}

In 2021, Bereg, Chacón-Rivera, Flores-Peñaloza, Huemer, Pérez-Lantero, and Seara~\cite{bereg2023maximum}*{Theorem~3.14} showed that a max-sum matching of any even set in the plane is a Tverberg graph. Our second result is a new proof of a slightly stronger version of their theorem.
\begin{theorem}
\label{theorem matching}
A max-sum matching of any even set of distinct points in the plane is an open Tverberg graph.
\end{theorem}

In \cite{pirahmad2022intersecting}, the authors established that for any even set of distinct points in $\mathbb{R}^d$, there exists a perfect matching that is an open Tverberg graph. They also questioned whether a max-sum matching satisfies this property, as discussed in \cite{pirahmad2022intersecting}*{Problem 9.3}. While Theorem~\ref{theorem matching} answers this question affirmatively for $d=2$, our approach relies heavily on the arrangement of points in the plane and cannot be extended to higher dimensions. Therefore, the question remains open in general and is an intriguing area for further research; see also Section~\ref{section open problems}.

The proofs of Theorems~\ref{theorem tree} and~\ref{theorem matching} share a common idea: Both rely on the following simple observation about convex functions.

\begin{proposition}[\cite{barabanshchikova2022intersecting}*{Proposition~10}]
\label{proposition: main tool}
For convex functions $f_1, \dots , f_m : \mathbb R^n \to \mathbb R$, let the function $f : \mathbb R^n \to \mathbb R$ be defined by $f(x) :=
\max \{f_1(x), \dots , f_m(x)\}$. For any point $x \in \mathbb R$, put $I(x) := \{1 \leq i \leq m : f_i(x) =
f(x)\}$.

    If the function $f$ attains its global minimum at a point $x_0$ and, for each $i\in I(x_0)$, the function $f_i$ is differentiable at $x_0$, then we have 
    \[
        o\in \conv \big\{ \nabla f_i (x_0): i\in I(x_0)\big\}.
    \]
\end{proposition}

The paper is organized as follows. In Sections~\ref{section proof of theorem tree} and~\ref{section proof of theorem matching}, we prove Theorems~\ref{theorem tree} and~\ref{theorem matching}, respectively. We conclude with a discussion of open problems in Section~\ref{section open problems}.

\section{Intersecting balls induced by a max-sum tree}
\label{section proof of theorem tree}
\begin{proof}[Proof of Theorem~\ref{theorem tree}] 
Let $T$ be a max-sum tree of a finite point set $X\subset \mathbb R^d$.
Consider the function $H: \mathbb{R}^d \xrightarrow{}  \mathbb{R}$ defined by
\[
    H(x) = \max\limits_{a \in X} \| a - x \|.
\]

This function attains its global minimum at a unique point, which is the center of the smallest radius sphere $S$ enclosing $X$. Without loss of generality, we assume that this point coincides with the origin $o$. Put $r:=H(o)$, that is, $S$ is a sphere of radius $r$. If $r=0$, then all points of $X$ coincide with $o$; hence, the theorem is trivial. Thus, we may assume that $r>0$. To complete the proof, we show that for each $pq \in E(T)$, we have
\begin{equation}
    \label{equation: origin lies in the ball B(pq)} o\in B(pq) \text{ or, equivalently, } \langle p, q\rangle\leq 0.
\end{equation}

Let $X_r:= X\cap S$. So, every point $a \in X \setminus X_r$ lies strictly inside the sphere $S$. By Proposition~\ref{proposition: main tool} applied to the function $H$, we have $o\in \conv X_r$. 
Let $X_r'$ be a non-empty subset of $X_r$ such that there exist positive coefficients $\lambda_a$ for $a\in X_r'$ satisfying
\begin{align*}
    \sum \limits_{a \in X_r'} \lambda_a a = 0.
\end{align*}

Let $G$ be the graph with the vertex set $X_r'$ and its two vertices $a,b\in X_r'$ are adjacent if and only if $\langle a, b \rangle \leq 0$. We claim that the graph $G$ is connected. Indeed, for any proper subset $U \subsetneq V(G)$ and its complementary set $W=V(G)\setminus U$, we have
\[
\sum \limits_{u \in U} \lambda_u u = - \sum \limits_{w \in W} \lambda_w w, \text{\ \ and thus,\ \ }
\Big\langle \sum \limits_{u \in U} \lambda_u u,  \sum \limits_{w \in W} \lambda_w w \Big\rangle  \leq 0.
\]
Since $\lambda_a>0$ for each $a\in V(G)=X_r'$, there are $u\in U$ and $w\in W$ with $\langle u, w\rangle \leq 0$, that is, the edge $uw$ connects $U$ and $W$. Therefore, the graph $G$ is connected. 

Finally, we are ready to prove~(\ref{equation: origin lies in the ball B(pq)}). Suppose to the contrary that there is an edge $p q \in E(T)$ with $\langle p, q\rangle > 0$. 
Since
\[
\sum \limits_{a \in V(G)} \lambda_a \langle a,  p \rangle = 0,
\]
there exists a vertex $x \in V(G)$ with $\langle x,  p \rangle \leq 0$. Similarly, there is a vertex $y \in V(G)$ with $\langle y,  q \rangle \leq 0$. The points $x$ and $y$ can coincide.

Since $G$ is a connected graph, consider a path $z_1 \dots z_k$ in $G$ connecting $z_1:=x$ and $z_k:=y$. We claim that $p q$ is the shortest line segment among $qp$, $qz_1, \dots, z_kq$. Indeed, we have
\begin{align*}
\| p - z_1 \|^2 = \|p-x\|^2= \| p \|^2 + \| x \|^2 - 2 \langle p,  x \rangle &\geq \| p \|^2 + r^2 > \\
 \|p\|^2+r^2- 2 \langle p,  q \rangle &\geq  \| p \|^2 +  \| q \|^2  - 2 \langle p,  q \rangle =  \| p - q \|^2.
\end{align*}
Analogously, one can prove that $\|p-q\|$ is strictly less than $\|z_k  - q \|$ or $\|z_i  - z_{i+1} \|$ for each $i\in \{1,\dots, k-1\}$. 

Next, we will find a tree with a cost larger than that of $T$, and this contradiction will complete the proof. By deleting the edge $pq$ from $T$, we have the forest consisting of two trees with vertex sets $U$ and $W$ such that $U\sqcup W= X$, $p\in U$, and $q\in W$. Let $ab$ be an edge of the path $pz_1\dots z_k q$ that connects $U$ and $W$. By replacing the edge $pq$ by $ab$ in the tree $T$, we obtain the desired tree.
\end{proof}

Define the \textit{cost} of a graph $G$ with respect to a function $f:\mathbb{R}_+\to \mathbb R$ as the sum
\[
    \sum_{ab\in E(G)} f(\|a-b\|),
\]
where $\mathbb R_+$ is the set of non-negative real numbers.

We say that a tree with vertex set $X$ is an \textit{$f$-max-sum tree} if it has the maximum cost among all trees with vertex set $X$.
Repeating the argument of Theorem~\ref{theorem tree}, one can easily prove the following statement. 
\begin{theorem}
Given an increasing function $f:\mathbb R_+\to \mathbb R$ and a finite point set $X$ in $\mathbb R^d$, an $f$-max-sum tree of $X$ is a Tverberg graph.
\end{theorem}



\section{Intersecting open disks induced by a max-sum matching}
\label{section proof of theorem matching}
\begin{proof}[Proof of Theorem~\ref{theorem matching}]
\label{subsection: proof of the theorem on open disks}
Let $\mathcal M$ be a max-sum matching of an even set $S$ of pairwise distinct points in the plane. Suppose, to the contrary, that the intersection of the open discs induced by $\mathcal M$ is an empty set. To prove the theorem, it is sufficient to find a matching of $S$ with the larger cost.

Consider the function $G: \mathbb{R}^2 \xrightarrow{}  \mathbb{R}$ defined by
\[
    G(x) = \max\limits_{a b \in \mathcal{M}} \Big\{ \Big\| \frac{a+b}{2} - x \Big\| - \Big\| \frac{a-b}{2} \Big\|\Big\}.
\]
This convex function attains its global minimum at a unique point. Without loss of generality, we may assume that this point coincides with the origin $o$. For the sake of brevity, put $r \coloneqq G(o)$. As the open discs induced by $\mathcal M$ have no common point, the origin $o$ lies out of at least one of them, and hence, we conclude $r \geq 0$. So, the closed disc of radius $r$ centered at~$o$ is the smallest radius disc intersecting the closed disc $B(ab)$ for each $ab\in \mathcal M$. Denote the circle bounding this disc of radius $r$ by~$\Omega$.

Consider the submatching
\begin{equation}
    \label{equation: M_1 in the disc problem}
    \mathcal M_r =\Big\{ab\in \mathcal M:\ \Big\| \frac{a+b}{2} \Big\| - \Big\| \frac{a-b}{2} \Big\| = r \Big\}.
\end{equation}
For each $ab\in \mathcal M$, define the function $g_{ab}:\mathbb R^2 \to \mathbb R$ by
\[
g_{ab}(x):=\Big\| \frac{a+b}{2}-x \Big\| - \Big\| \frac{a-b}{2} \Big\|.
\]
Clearly, it is differentiable everywhere except the point $(a+b)/2$. Since all points of $S$ are pairwise distinct, we have that for $ab\in \mathcal M_r$, the function $g_{ab}$ is differentiable at the origin $o$. Thus, by Proposition~\ref{proposition: main tool} applied to the function $G$, we obtain
$$o\in \conv \Big\{\nabla g_{ab}(o): ab\in \mathcal M_r \Big\} \text{,\ \ and thus,\ \ }  o \in \conv \Big\{\frac{a+b}{2} : ab \in \mathcal M_r\Big\}.$$
Denote by $\mathcal M_r'$ any minimal submatching of $\mathcal M_r$ that inherits this property, that is,
\begin{equation}
    \label{equation: origin lies in the convex hull of centers}
    o\in\conv \Big\{ \frac{a+b}{2}:ab\in \mathcal M_r'\Big\}.
\end{equation}
Since for any edge $ab\in \mathcal M_r'\subseteq \mathcal M_r$ the point $\frac{a+b}{2}$ is distinct from the origin, we have $|\mathcal M_r'|>1$. By Carath\'eodory's theorem and the minimality of $\mathcal M_r'$, we get that the size of $\mathcal M_r'$ equals $2$ or $3$.

\begin{figure}[h!]
	\centering
	\includegraphics{images/circles.pdf}
	\caption{Notation used in the proof of Theorem~\ref{theorem matching}}
	\label{fig:im1}
\end{figure}
Recall that $\Omega$ is the circle centered at the origin of radius $r$. By \eqref{equation: M_1 in the disc problem}, we have that $\Omega$ externally touches the discs $B(a b)$ for $a b \in \mathcal M_r' \subseteq \mathcal M_r$. Let $o_{ab}$ be the only common point of~$\Omega$ and $B(ab)$. Let $\ell_{ab}$ be the internal common tangent line to $\Omega$ and $B(a b)$ passing through $o_{ab}$; see Figure~\ref{fig:im1}. Let $a'$ and $b'$ be the orthogonal projections of the points $a$ and $b$ onto the line~$\ell_{ab}$. Put $r_{a}:=\|a-a'\|$ and $r_{b}:=\|b-b'\|$. Denote by $B_a$ and $B_b$ the closed discs of radius $r_a$ and $r_b$ centered at $a$ and $b$, respectively. It is possible that one of them is of radius 0. Hence the line $\ell_{ab}$ touches the discs $B_a$, $B(ab)$, and $B_b$; see Figure~\ref{fig:im2}. 

Since the point $o_{ab}$ is the projection of $\frac{a+b}{2}$ onto the line $\ell_{ab}$ and the point $\frac{a+b}{2}$ is the midpoint of the line segment $ab$, the point $o_{ab}$ is the midpoint of the segment $a'b'$. Moreover,
\begin{equation}
    \label{equation: touching circles}
    r_{a} + r_{b} = 2\Big\| \frac{a+b}{2}-o_{ab}\Big\|=2\Big(\Big\| \frac{a + b}{2} \Big\|-r\Big)=2\|a-b\|.
\end{equation}
Here the second equality follows from the fact that the points $o$, $o_{ab}$, and $\frac{a+b}{2}$ are collinear.
This fact combined with \eqref{equation: origin lies in the convex hull of centers} also implies $o\in \conv \{o_{ab}:ab\in \mathcal M_r'\}.$ Thus, 
\begin{equation}
    \label{equation: convex hull of o_i}
\text{if $r>0$, then any closed semicircle of $\Omega$ contains a point $o_{ab}$, where $ab\in \mathcal M_r'$.}
\end{equation}


Denote by $S_r\subseteq S$ the set of all endpoints of segments of $\mathcal M_r'$. Let $G$ be the graph on the vertex set $S_r$ with the edge set colored in blue and red as follows
\[
    E_b(G) \coloneqq \mathcal M_r' \;\text{ and }\; E_r(G) \coloneqq \{ab\in E(G) : \| a - b \| > r_a + r_b\}.
\]
By the definition, if points $a$ and $b$ are connected by a blue edge, that is, $ab\in \mathcal M_r'$, then the discs $B_a$ and $B_b$ touch each other externally. Also, $cd$ is a red edge if and only if the discs $B_c$ and $B_d$ are disjoint. 

Let us show that if $G$ contains an alternating cycle, then there is a matching with the larger cost than that of $\mathcal M$. Here, an \textit{alternating cycle} is a simple cycle of even length whose edges are taken alternately from $E_b(G)$ and $E_r(G)$. Indeed, assume that there is a cycle $x_1 y_1 \dots x_m y_m x_{m+1}$, where $x_{m+1}:=x_1$, such that $x_i y_i \in E_b(G)$ and $y_ix_{i+1}\in E_r(G)$. Hence we have
$$\sum \limits_{i=1}^{m} \lVert y_i - x_{i+1} \rVert > \sum \limits_{i=1}^{2m} (r_{y_i} + r_{x_{i+1}}) = \sum \limits_{i=1}^{2m} (r_{y_i} + r_{x_i}) = \sum \limits_{i=1}^{m} \lVert x_i - y_i \rVert.$$
Replacing in \(\mathcal M\) the blue edges from the alternating cycle with the red ones, we obtain the desired perfect matching
\[
\mathcal M\setminus \{x_1y_1,\dots,x_my_m\}\cup \{ y_1x_2,\dots, y_mx_{m+1}\}.
\]
Thus, finding an alternating cycle in $G$ is sufficient to finish the proof. For that, we use the following two lemmas shown in the next subsection.
\begin{lemma}\label{circle_lemma_1}
If $r > 0$, then for any blue edge $a b$, there is another blue edge $cd$ such that either $ac, ad \in E_r(G)$ or $bc, bd\in E_r(G)$.
\end{lemma}
\begin{lemma}\label{circle_lemma_2}
If $r = 0$, then either the graph $G$ contains an alternating cycle of length 4 or for any blue edge $ab$, there is another blue edge $cd$ such that either $ac, bd\in E_r(G)$ or $bc, bd\in E_r(G)$.
\end{lemma}

Suppose to the contrary that $G$ contains no alternating cycle. By Lemmas~\ref{circle_lemma_1} and~\ref{circle_lemma_2} applied to the blue edge $a_1b_1$, without loss of generality, we may assume that there is another blue edge $a_2b_2$ such that $a_1a_2$ and $a_1b_2$ are red edges. Since there is no alternating cycle in $G$, there are no more red edges among vertices $a_1, b_1, a_2, b_2$. Thus, by Lemmas~\ref{circle_lemma_1} and~\ref{circle_lemma_2} applied to the blue edge $a_2b_2$, we may assume that there is a third blue edge $a_3b_3$ such that $a_2a_3$ and $a_2 b_3$ are red edges. (In particular, the size of $E_b(G)=\mathcal M_r'$ is 3.) Applying a similar argument, without loss of generality, we may assume that the edges $a_3 a_1$ and $a_3 b_1 $ are red. Hence the graph $G$ contains the alternating cycle $a_1 b_2 a_2 b_3 a_3 b_1$, a contradiction. This finishes the proof of Theorem~\ref{theorem matching}.
\end{proof}

\subsection{Proof of auxiliary lemmas}

For a blue edge $ab\in \mathcal M_r'$, denote by $s_{ab}$ the common interior tangent line of the discs $B_a$ and $B_b$; see Figure~\ref{fig:im2}.

To prove Lemmas~\ref{circle_lemma_1} and~\ref{circle_lemma_2}, we will use the following simple observation from elementary geometry: \textit{The line $s_{ab}$ passes through the point $o_{ab}$}. (Recall that in the case $r=0$, the point $o_{ab}$ and the origin $o$ coincide.) Indeed, the line $s_{ab}$ is the radical axe of the circles bounding the discs $B_a$ and $B_b$. As $o_{ab}$ is the midpoint of the segment $a'b'$,  the powers of $o_{ab}$ with respect to those circles are equal. Hence $o_{ab}$ lies on $s_{ab}$.

\begin{proof}[Proof of Lemma \ref{circle_lemma_1}]
First, we show that if $a$ coincides with $o_{ab}$, then it is connected with all other vertices but $b$ by red edges. For any edge $cd \in \mathcal M_r'$ distinct from $ab$, the tangent line $\ell_{cd}$ partitions the plane into the open half-plane containing $\Omega \setminus \{o_{c d}\}$ and the closed half-plane containing $B_c$ and $B_d$. Hence $B_a = \{o_{a b}\} \subset \Omega \setminus \{o_{cd}\}$ lies in the open half-plane. (Here we use that if  $cd\ne a b$, then the points $o_{ab}\ne o_{cd}$, which follows from the minimality of $\mathcal M_r'$.) Therefore, the discs $B_c$ and $B_a$ are strictly separated, and thus, $a c$ is a red edge. Similarly, the edge $a d$ is also red.

From now on, we may assume that $a$ and $b$ are distinct from $o_{ab}$. Denote by $\omega(a)$ the subset of $\Omega$ satisfying the following property. A point $t \in \Omega$ belongs to $\omega(a)$ if and only if $B_a$ lies in the same open half-plane as the origin $o$ with respect to the tangent line to $\Omega$ passing through~$t$. The set $\omega(a)$ is an open arc whose endpoints correspond to the external tangent lines of $\Omega$ and $B_a$. Analogously, we define $\omega(b)$. Clearly, $o_{ab}\not \in \omega(a)$ and $o_{ab}\not\in \omega(b)$.
Since the tangent line to $\Omega$ at $-o_{a b}$ is parallel to $\ell_{a b}$, we conclude that $-o_{a b}$ belongs to the sets $\omega(a)$ and $\omega(b)$, simultaneously. Therefore, the union $\omega(a) \cup \omega(b)$ is also an open arc.

%\begin{figure}[h!]
%	\centering
%   \includegraphics{images/rpositive.pdf}
%	\caption{Case $r > 0$}
%	\label{fig:im21}
%\end{figure}


Since the points $a$ and $b$ are distinct from $o_{ab}$, the line $s_{a b}$ meets the open line segment $ab$. Thus, the lines $s_{ab}$ and $\ell_{a b}$ are distinct. Hence $s_{a b}$ intersects the circle $\Omega$ at two points, one of them is $o_{ab}$. Consider two distinct tangent lines $s_a$ and $s_b$ to $\Omega$ that are parallel to $s_{a b}$; see Figure~\ref{fig:im21}. Clearly, the line $s_{ab}$ lies between $s_a$ and $s_b$. Without loss of generality, we may assume that for $x\in \{a,b\}$, the origin $o$ and the disc $B_x$ lie on the same side with respect to $s_x$. Since the lines $s_a$ and $s_b$ touch $\Omega$ at antipodal points, the open arc $\omega(a) \cup \omega(b)$ contains these points. Thus, it also contains a closed semicircle of $\Omega$. By (\ref{equation: convex hull of o_i}), the open arc $\omega(a) \cup \omega(b)$ contains some point $o_{cd}$ for $cd\in \mathcal M_r'$.

Without loss of generality, we may assume $o_{cd} \in \omega(a)$. Hence, the line $\ell_{cd}$ partitions the plane into two half-planes: One of them is open and contains the origin and the disc $B_a$ and another is closed and contains the discs $B_c$ and $B_d$. Therefore, we conclude that $ac$ and $ad$ are red edges.
\end{proof}
\begin{figure}[htp]
    \centering
    \begin{subfigure}[b]{0.5\textwidth}
        \centering
        \includegraphics{images/rpositive.pdf}
    \caption{$r > 0$}
    \label{fig:im21}
    \end{subfigure}
    \begin{subfigure}[b]{0.45\textwidth}
        \centering
        \includegraphics{images/rzero.pdf}
        \caption{$r = 0$}
    \label{fig:im22}
    \end{subfigure}
	\caption{Proofs of Lemmas~\ref{circle_lemma_1} and \ref{circle_lemma_2}}
    \label{fig:im2}
\end{figure}
\begin{proof}[Proof of Lemma \ref{circle_lemma_2}]

The proof of this lemma is similar to the argument of Lemma~\ref{circle_lemma_1}.

For each edge $a b \in \mathcal M_r'$, the point $o_{a b}$ coincides with the origin $o$. Thus, $o$ lies on the lines $\ell_{a b}$ and $s_{a b}$ and belongs to the boundary of $B(ab)$; see Figure \ref{fig:im22}. Since all points of $S$ are distinct, at most one vertex of $G$ coincides with the origin.

First, we prove that if some two discs $B(ab)$ and $B(cd)$ touch externally, then there is an alternating cycle of length 4. Indeed, then the lines $\ell_{ab}$ and $\ell_{cd}$ coincide and the sets $\{B_a, B_b\}$ and $\{B_c,B_d\}$ lie in the opposite closed half-planes induced by the line~$\ell_{ab}=\ell_{cd}$. Suppose that at least one of the pairs $ac, ad, bc, bd$ is not a red edge. (Otherwise, we easily find an alternating 4-cycle.)
Without loss of generality, assume that the pair $ac$ is \textit{not} a red edge. Then the discs $B_a$ and $B_c$ touch each other at the point $a'=c'$ lying on the line $\ell_{ab}=\ell_{cd}$. We know that the point $b'$ is symmetric to $a'$ with respect to $o_{ab}=o$. Similarly, $d'=-c'$ and therefore $b'=d'$, which implies that the discs $B_b$ and $B_d$ touch each other externally. Note that the point $a'=c'$ coincides with the point $b'=d'$ if and only if the points $a',b',c',d'$ coincide with the origin $o$, that is, two of the points $a,b,c,$ and $d$ coincide with the origin, which is impossible. So, the points $a'=c'$ and $b'=d'$ are distinct. Hence the discs $D_a$ and $D_c$ are disjoint, that is, $ac$ is a red edge. Analogously, $bd$ is a red edge, and so, the 4-cycle $abdc$ is alternating. 

From now on, we assume that among the discs induced by $\mathcal M_r'$ there are no two touching externally. Thus, by~\eqref{equation: origin lies in the convex hull of centers} and the minimality of $\mathcal M_r'$, no two vectors of ${X:=\{\frac{c+d}{2}:cd\in \mathcal M_r'\}}$ are collinear.

%\textcolor{red}{Next, we show that for any blue edge $a b$, there is a blue edge $cd$ such that either $ac, ad\in E_r(G)$ or $b c, bd\in E_r(G)$. Assume that the point $a$ coincides with the origin. Then we claim that the vertex $a$ is connected with all other vertices but $b$ by red edges. Indeed, for any edge $cd\in\mathcal M_r'$ distinct from $ab$, the line $\ell_{cd}$ passes through $o$, and the discs $B_c$ and $B_d$ lie in one of the half-planes induced by $\ell_{cd}$. Thus, any of them can intersect $B_a=\{o\}$ if and only if one of the points $c$ or $d$ coincides with the origin. This is impossible because all points of $S$ are distinct and only one of them can coincide with the origin. Hence the edges $ac$ and $ad$ are red. From now on, we may assume that the vertices $a$ and $b$ are distinct from the origin. Let $s'_{a b}$ be the ray emanating from $o$ and passing through the only common point of $B_a$ and $B_b$. Notice that the line $s_{ab}$ contains the ray $s'_{a b}$, and thus, $s'_{ab}$ touches the discs $B_a$ and $B_b$.
%Recall that among the discs induced by $\mathcal M_r'$ there are no two touching each other externally, and so, the vector $a+b$ is not collinear to $-(c+d)$ for any $cd\in \mathcal M_r'$. By \eqref{equation: origin lies in the convex hull of centers}, there is a blue edge $cd \in \mathcal M_r'$ such that the vector $c + d$ forms an obtuse angle with the ray $s_{ab}$. The line $\ell_{cd}$ separates one of the discs $B_{a}$ or $B_{b}$ from the discs $B_c$ and $B_b$. Therefore, either $ax, ay$ are red edges or $bx, by$ are red edges, which finishes the proof of Lemma~\ref{circle_lemma_2}.}

Next, we show that for any blue edge $a b$, there is a blue edge $cd$ such that either $ac, ad\in E_r(G)$ or $b c, bd\in E_r(G)$. First, we consider the case when the point $a$ coincides with the origin $o$. We claim that the vertex $a$ is connected with all other vertices of $G$ but $b$ by red edges. Indeed, for any edge $cd\in\mathcal M_r'$ distinct from $ab$, the disc $B_c$ touches the line $\ell_{cd}$ passing through $o_{cd}=o=a$ at a unique point $c'$. If the point $c'$ coincides with the origin, so does $c$ or $d$. It is impossible because all points of $S$ are distinct. Hence the disc $B_c$ does not intersect $B_a = \{o\}$, and so, the edge $ac$ is red. From now on, we may assume that the vertices $a$ and $b$ are distinct from the origin.

Let $s'_{a b}\subset s_{ab}$ be the ray emanating from $o$ and passing through the only common point of $B_a$ and $B_b$. For any $cd\in \mathcal M_r'$, denote by $H_{cd}^+$ the closed half-plane bounded by the line $\ell_{cd}$ and containing $B_c$ and $B_d$. Put $H_{cd}^-:=\mathbb R^2\setminus H_{cd}^+$. Suppose that for any $cd \in \mathcal M_r'$ the ray $s'_{ab}$ lies in $H_{cd}^+$. Then the angle between the ray $s'_{ab}$ and each vector of $X$ is either right or acute. By \eqref{equation: origin lies in the convex hull of centers}, we have $o\in \conv X$, which is possible only if the set $X$ has a pair of collinear vectors in the opposite directions, a contradiction. Hence, there exists an edge $cd \in \mathcal M_r'$ such that $s'_{ab}\setminus \{o\} \subset H_{cd}^-$. As the lines $\ell_{ab}$ and $\ell_{cd}$ are distinct, the intersection of the open half-plane $H_{cd}^-$ and the line $\ell_{ab}$ is an open ray. Therefore, $H_{cd}^-$ contains one of the discs $B_a$ or $B_b$ touching this open ray and $s'_{ab}\setminus\{o\}\subset H_{cd}^-$. Hence, the line $\ell_{cd}$ separates the discs $B_c, B_d\subset H_{cd}^+$ from one of the discs $B_a$ or $B_b$ lying in $H_{cd}^-$. So, either $ac, ad\in E_r(G)$ or $b c, bd\in E_r(G)$, which finishes the proof of Lemma~\ref{circle_lemma_2}.
\end{proof}

\section{Discussion}
\label{section open problems}

While a max-sum tree of a finite set of distinct points in $\mathbb R^d$ is a Tverberg graph, it may not necessarily be an open Tverberg graph. For instance, consider a rhombus where one diagonal is shorter than its side. In this case, any max-sum tree of the points in the rhombus consists of the largest diagonal and two opposite sides, as illustrated in Figure~\ref{figure rhombus}. It is easy to verify that the intersection of open discs induced by this max-sum tree is empty. Nevertheless, the following problem on open Tverberg graphs remains open.

\begin{figure}[h!]
	\centering
  \includegraphics{images/maxtree.pdf}
	\caption{Max-sum tree that is not an open Tverberg graph}
	\label{figure rhombus}
\end{figure}

\begin{problem}[\cite{pirahmad2022intersecting}*{Problem~9.3}]
Is it true that for any even set of distinct points in $\mathbb R^d$, a max-sum matching is an open Tverberg graph?
\end{problem}

Although we believe that the answer to this problem is positive, we are not able to confirm it in its generality. Moreover, we conjecture a quantitative variation of this problem, which resembles the quantitative results and problems of B\'ar\'any, Katchalski, and Pach; see~\cite{barany1982quantitative}.

\begin{conjecture}\label{conjecture quantitative}
Given a positive integer $d$, there exists a constant $\varepsilon_d>0$ such that any even set $X$ of (distinct) points in $\mathbb R^d$ with minimum distance $r>0$ between points of $X$ satisfies the following property. The intersection of the closed balls induced by a max matching $\mathcal M$ of $X$
contains a ball of radius $\varepsilon_d r$.
\end{conjecture}

It seems that our approach developed to prove Theorem~\ref{theorem matching} does not allow us to confirm Conjecture~\ref{conjecture quantitative} even in the plane. So, it remains interesting to verify it in this special case.

\bibliographystyle{siam}
\bibliography{biblio}

\end{document}
