\documentclass[reqno]{amsart}
\usepackage[scale=0.75, centering, headheight=14pt]{geometry}
\usepackage[latin1]{inputenc}
\usepackage[T1]{fontenc}
\usepackage{lmodern}
\usepackage[english]{babel}
\usepackage{stmaryrd}
\usepackage{tikz}
\usepackage{bbm}
%\usepackage[pagewise]{lineno}\linenumbers
%\usepackage{draftwatermark}
%\SetWatermarkText{not for publication}
%\SetWatermarkScale{0.6}
\usepackage{amsmath,amssymb,amsfonts,amsthm}
\usepackage{mathtools,accents}
\usepackage{mathrsfs}
\usepackage{xfrac}
\usepackage{array} 
%\usepackage{showkeys}
\usepackage{aliascnt}
%%% PACKAGES
\usepackage{booktabs} % for much better looking tables
\usepackage{array} % for better arrays (eg matrices) in maths

\usepackage{verbatim} % adds environment for commenting out blocks of text & for better verbatim
\usepackage{subfig} % make it possible to include more than one captioned figure/table in a single float
\usepackage{mathrsfs}
\usepackage{mathrsfs, dsfont}
\usepackage{amssymb}
\usepackage{amsthm}
\usepackage{amsmath,amsfonts,amssymb,esint}
\usepackage{graphics,color}
\usepackage{enumerate}
\usepackage{mathtools,centernot}

%DA DECOMMENTARE
\usepackage{microtype}
\usepackage{paralist} % very flexible & customisable lists (eg. enumerate/itemize, etc.)
\usepackage{cases}
\usepackage[initials,alphabetic]{amsrefs}
\allowdisplaybreaks

\usepackage{braket}

\usepackage{bm}

\usepackage[citecolor=blue,colorlinks]{hyperref}
\addto\extrasenglish{\renewcommand{\sectionautorefname}{Section}}

\usepackage{enumerate}
\usepackage{xcolor}

\usepackage{aliascnt}

\makeatletter
\def\newaliasedtheorem#1[#2]#3{
  \newaliascnt{#1@alt}{#2}
  \newtheorem{#1}[#1@alt]{#3}
  \expandafter\newcommand\csname #1@altname\endcsname{#3}
}
\makeatother

\makeatletter
\newcommand{\myitem}[1]{%
\item[(#1)]\protected@edef\@currentlabel{#1}%
}
\makeatother

\theoremstyle{plain}
\newtheorem{theorem}{Theorem}[section]
\newaliasedtheorem{lemma}[theorem]{Lemma}
\newaliasedtheorem{proposition}[theorem]{Proposition}
\newaliasedtheorem{claim}[theorem]{Claim}
\newaliasedtheorem{corollary}[theorem]{Corollary}
\newtheorem*{NTEO}{Theorem}
\newtheorem*{NPROP}{Proposition}
\newtheorem*{NCOR}{Corollary}
%\theoremstyle{remark}

\newtheorem{maintheorem}{Theorem}
\renewcommand*{\themaintheorem}{\Alph{maintheorem}}
%\theoremstyle{remark}



\DeclareMathOperator{\diver}{div}


\theoremstyle{definition}
\newaliasedtheorem{definition}[theorem]{Definition}
\newaliasedtheorem{example}[theorem]{Example}
\newtheorem{Cond}{Condition}
\newtheorem{OQ}[]{Open Question}
\newtheorem{QQ}[theorem]{Questions}
\newtheorem{problem}[theorem]{Problem}

\theoremstyle{remark}
\newaliasedtheorem{remark}[theorem]{Remark}
\newaliasedtheorem{ex}[theorem]{Example}
\newtheorem*{notation}{Notation}


\numberwithin{equation}{section}

\newcommand{\e}{{\rm e}}

\def\eps{\varepsilon}
\def\tr{\text{tr}}
\def\R{\mathbb R}
\def\C{{\mathbb C}}% complex numbers
\def\N{{\mathbb N}}% nonnegative integers
\def\Z{{\mathbb Z}}% integers
\def\T{{\mathbb T}}% torus
\def\Q{{\mathbf Q}}% rational numbers
\def\P{\mathbf P}
\def\Sch{{\mathcal S}}% Schwartz space
\def\FT{\mathcal F}% Fourier transform
\def\d{{\partial}}
\def\ra{\rightarrow}
\def\a{{\alpha}}
\def\i{{\infty}}
\DeclareMathOperator{\RE}{Re}
\DeclareMathOperator{\IM}{Im}
\DeclareMathOperator*{\esssup}{ess\,sup}
\DeclareMathOperator{\dv}{div}
\DeclareMathOperator{\Lip}{Lip}

\DeclareMathOperator{\stat}{stat}
\DeclareMathOperator{\err}{err}
\DeclareMathOperator{\spn}{span}
\DeclareMathOperator{\curl}{curl}
\DeclareMathOperator{\dt}{dt}
\DeclareMathOperator{\sign}{sign}
\DeclareMathOperator{\Ker}{Ker}
\DeclareMathOperator{\supp}{supp}
\DeclareMathOperator{\chess}{chess}
\DeclareMathOperator{\rank}{rank}
\DeclareMathOperator{\loc}{loc}
\DeclareMathOperator{\swap}{swap}
\DeclareMathOperator{\initial}{in}
\DeclareMathOperator{\symmD}{D}
\DeclareMathOperator{\GL}{GL}
\DeclareMathOperator{\AC}{AC}
\DeclareMathOperator{\Sym}{Sym}
\DeclareMathOperator{\cof}{cof}
\DeclareMathOperator{\op}{op_{\mathbb{H}}}
\DeclareMathOperator{\opr}{opr_{\mathbb{H}}}
\DeclareMathOperator{\id}{id}
\DeclareMathOperator{\Det}{det}
\newcommand{\M}{\mathbb{M}}
\newcommand{\PO}{\mathbb{P}_\mathbb{H}}
\newcommand{\POL}{\mathbb{P}}
\newcommand{\G}{\mathcal{G}}
\DeclareMathOperator{\co}{co}
\newcommand{\GG}{\mathcal{G}^*}
\newcommand{\DAQ}{\textcolor{red}{\textbf{DA QUI}}}
\newcommand{\W}{W_{\dv}^{1,p}\cap L^\infty(\R^n,\R^n)}
\DeclareMathOperator{\im}{Im}
\newcommand*{\RR}{\ensuremath{\mathcal{R}}}
\newcommand*{\RRc}{\ensuremath{\mathcal{B}}}
\newcommand{\E}{\mathds{E}}
\newcommand{\Sp}{\mathbb{S}}
\newcommand{\weak}{\overset{*}{\rightharpoonup}}
\DeclareMathOperator{\spt}{spt}
\DeclareMathOperator{\Id}{Id}
\DeclareMathOperator{\diam}{diam}
\DeclareMathOperator{\dist}{dist}
\newcommand{\der}{\frac{d}{dt}|_{t = 0}}
\newcommand{\p}{\partial}
\newcommand{\bp}{\bar\partial}
\newcommand{\A}{\mathcal{A}}
\newcommand{\D}{\mathcal{D}}
\newcommand{\OP}{\op (k,m,n,N)}
\newcommand{\OPR}{\opr (k,m,n,N)}
\newcommand{\LA}{\Lambda_{\mathcal{A}}}
\newcommand{\PH}{\mathbb{P}_H}
\newcommand{\XX}{{\mbox{\boldmath$X$}}}
\newcommand{\WW}{{\mbox{\boldmath$W$}}\mkern-4mu}
\newcommand{\YY}{{\mbox{\boldmath$Y$}}}
\newcommand{\uu}{\tilde{u}}
\newcommand{\vt}{{\vartheta}}
\newcommand{\cc}{\mbox{cs}}
\newcommand{\ds}{\mbox{ds}}
\newcommand{\lesssimlarge}{\lesssim_{q \gg 1}}


\title[Nontrivial absolutely continuous part of anomalous dissipation measures in time]
 {Nontrivial absolutely continuous part   of anomalous dissipation measures in time}
 
\author[C. J. P. Johansson and M. Sorella]{Carl Johan Peter Johansson and Massimo Sorella}

\address{Massimo Sorella
\hfill\break EPFL SB, Station 8, CH-1015 Lausanne, Switzerland}
\email{massimo.sorella@epfl.ch}

\address{Carl Johan Peter Johansson
\hfill\break EPFL SB, Station 8, CH-1015 Lausanne, Switzerland}
\email{carl.johansson@epfl.ch}

\begin{document}


\begin{abstract}
%Nontrivial absolutely continuous part   of anomalous dissipation measures in time of weak physical 4D forced Euler solutions
 
%On the absolute continuity of anomalous dissipation measures of weak physical  solutions for 4d forced Euler equations

%Non-trivial absolutely continuous part of time anomalous dissipation measures  for 4d forced Navier--Stokes equations


%Continuous in time anomalous dissipation  for 4d forced Navier--Stokes equations and for advection--diffusion with 3d autonomous velocity field

%Continuous in time anomalous dissipation for 4d forced Navier--Stokes


We positively answer  \cite[Question 2.2 and Question 2.3]{BDL22} in dimension $4$ by building new examples of solutions to the forced $4d$ incompressible Navier-Stokes equations, which exhibit anomalous dissipation, related to the zeroth law of turbulence \cite{kolmo}. 
We also prove that the unique smooth solution $v_\nu$ of the $4d$ Navier--Stokes equations with time-independent body forces is $L^\infty$-weakly* converging to a solution of the forced Euler equations $v_0$ as the viscosity parameter $\nu \to 0$. Furthermore, the sequence
 $\nu |\nabla v_\nu|^2$ is weakly* converging (up to subsequences), in the sense of measure, to $\mu \in \mathcal{M} ((0,1) \times \T^4)$ and $\mu_T = \pi_{\#} \mu $ has a non-trivial absolutely continuous part where $\pi$ is the projection into the time variable. Moreover,  we also show that $\mu$ is close, up to an error measured in $H^{-1}_{t,x}$, to the Duchon--Robert distribution  $ \mathcal{D}[v_0] $ of the solution to the $4d$ forced Euler equations.  Finally, the kinetic energy profile of $v_0$ is smooth in time.
Our result relies on a new anomalous dissipation result for  the advection--diffusion equation with a divergence free $3d$ autonomous velocity field and the study of the $3+\frac{1}{2} $ dimensional  incompressible Navier--Stokes equations. This study motivates some  open problems.

%which is a solution to the 4d Euler equations with body force such that its  Duchon-Robert measure \cite{DR00}, integrated in space, is an absolutely continuous  measure and its density is smooth.

 

%It remains open to construct  sequence $v_\nu$ the last measure coincides with the Duchon-Robert measure of the Euler equations limiting solution. In this direction, we are able to show that this is true up to a small error fixed at the beginning, depending on the construction of the force.


\end{abstract}

\maketitle

\section{Introduction}

We 
 study 
the Navier--Stokes equations with a body force  on the $4$-dimensional torus $\T^4 \simeq \R^4 / \Z^4$, namely 
\begin{align} 
\tag{NS}
\label{e:NSE}
\partial_t v_{\nu} + v_\nu \cdot \nabla v_\nu + \nabla p_\nu = \nu \Delta v_\nu + F_\nu;
\\
\diver v_\nu =0; \notag
\end{align}
where $v_\nu : [0, + \infty) \times \T^4 \to \R^4$ is the velocity field, $p_\nu : [0, \infty) \times \T^4 \to \R$ is the pressure, $\nu \geq 0$ is the viscosity parameter and $F_\nu : [0, + \infty) \times \T^4 \to \R^4$ is a force that may depend on $\nu$. In the case $\nu = 0$ equations~\eqref{e:NSE} reduce to the Euler equations with body force $F_0$.  We consider the Navier--Stokes equations \eqref{e:NSE} with a prescribed initial datum $v_{\initial, \nu } \in L^\infty$. The goal of this paper is to study the possible behaviour of the dissipation sequence $\nu |\nabla v_\nu|^2$ of turbulent Leray-Hopf solutions to the Navier--Stokes equations with body forces.
%and subsequently we will study the possible behaviour of the dissipation measure $\nu |\nabla \theta_\nu|^2$ of solutions to the advection-diffusion equations with autonomous 3d turbulent velocity fields.
Based on data,  in the infinite Reynolds number limit, it is expected to have a continuous in time dissipation behaviour of the solutions. We clarify this physical idea giving precise mathematical definitions and statements and arising some open problems in this direction.
  %We give positive answers to two open problems in fluid dynamics \cite[Question 2.2 and Question 2.3]{BDL22} related to the Navier-Stokes equations on the $4$-dimensional torus.
%\textcolor{green}{We now explain the contribution of our result, the motivation of the  questions \cite[Question 2.2 and Question 2.3]{BDL22} and push forward the theory in this direction leaving interesting open problems.}
%We clarify the motivation for questions \cite[Question 2.2 and Question 2.3]{BDL22}, spell out how our results contribute to and fit into the existing literature and state open problems related to the theory.
We firstly introduce the concept of {\em physical  solutions} to the Euler equations with a body force.  This definition is motivated by the still unsolved physical Onsager conjecture (see for instance in \cite{D17Thesis} for a mathematical definition).

\begin{definition}[Physical solutions] \label{d:physical}
    Let $d\geq 2$ and $v \in L^3 ((0,1) \times \T^d)$ be a distributional solution to the incompressible Euler equations with force $F_0 \in L^{3/2}((0,1) \times \T^d)$ in dimension $d$ with a divergence free initial datum $v_{\initial} \in L^2$. We say that $v $ is a {\em physical  solution} if:
    \begin{enumerate}
        \item there exists a sequence of initial data $\{ v_{\initial, \nu } \}_{\nu > 0}$ such that  $v_{\initial, \nu} \to v_{\initial}$ in $L^2$,
        \item \label{d:property-2} there exists a sequence of Leray solutions $\{ v_\nu \}_{\nu >0}$ of \eqref{e:NSE} with  initial data $v_{\initial, \nu}$ and forces $\{ F_\nu \}_{\nu >0}$
        such that  there exists a subsequence $\{ \nu_q \}_{q \in \N}$ for which we have $v_{\nu_q} \rightharpoonup v$ as $\nu_q \to 0$ in the sense of distribution,
        \item \label{d:property-3} the forces are such that $F_\nu \equiv F_0$ for any $\nu >0$.
    \end{enumerate}
    %there exists a sequence of Leray solutions $\{ v_\nu \}_{\nu >0}$ of \eqref{e:NSE} with  initial data $v_{\initial, \nu}$ such that $v_{\initial, \nu} \to v_{\initial}$ in $L^2$ as $\nu \to 0$,  body forces $F_\nu \equiv F_0$ for any $\nu \geq 0$ and initial datum $v_{\initial}$ such that up to subsequences $\{ \nu_q \}_{q \in \N}$ we have $v_{\nu_q} \rightharpoonup v$ as $\nu_q \to 0$ in the sense of distribution.
\end{definition}

Property \eqref{d:property-2} in the previous definition is motivated by the fact that the dissipation $\nu |\nabla v_\nu|^2$ is assumed to be  bounded in $L^1_{t,x}$ independently on $\nu$ in the theory of turbulence (see for instance \cite{kolmo,frisch}) and the regularity of $v$ is required to be $L^3$ because of the results by Duchon--Robert \cite[Proposition 1 and Proposition 2]{DR00}. More precisely, in \cite{DR00}, the authors proved that for any $d \geq 2$ and $(v_\nu, p_\nu)  \in L^3 \times L^{3/2}$ Leray solution  to the incompressible Navier--Stokes equations \eqref{e:NSE} (or weak solution $(v_0, p_0) \in L^3 \times L^{3/2}$ to the incompressible Euler equations, namely \eqref{e:NSE} with $\nu=0$) with force $F_\nu \in L^{3/2} ((0,1) \times \T^d)$  there exists a distribution $\mathcal{D} [v_\nu] $, such that 
\begin{align}\label{duchon-robert}
    \partial_t \frac{| v_\nu |^2}{2} + \diver \left (v_\nu \left (\frac{|v_\nu|^2}{2} + p_\nu \right ) \right ) + \mathcal{D}[v_\nu] +  \nu |\nabla v_\nu|^2 = F_\nu \cdot v_\nu +  \frac{\nu}{2} \Delta |v_\nu|^2  \qquad \text{in } (0,1) \times \T^d
\end{align} 
holds for any $\nu \in [0, \infty) $
 in the sense of distribution.  
 In this setting we are interested in the anomalous dissipation, namely
 \begin{align} \label{e:zeroth-law}
     \limsup_{\nu \downarrow 0} \nu \int_0^1 \int_{\T^4} | \nabla v_{\nu} |^2 >0 \,.
 \end{align}
 which is related to the zeroth-law of turbulence \cite{kolmo}. 
 
In this context, we remark that  there are several examples (also with force $F_0 \equiv 0$) in which the Duchon-Robert distribution $\mathcal{D}[v_0]$ is not identically zero for 3d Euler equations (see for instance \cite{DLL09,DLL13,Is18,BDLIS,BDLLOnsager,BV18,DLK20,S93,Sh97,Sh00,BMNV21,NV22}), most of them relying on the so called convex integration technique, however none of these examples, to the authors knowledge, are proved to be {\em physical solutions}.
 Since 
 the existence of {\em physical  solutions} which exhibits anomalous dissipation in the general setting for $d \geq 3$ is an hard problem, the authors in \cite{BDL22} suggested to allow the body forces of the forced Navier--Stokes equations 
 $\{ F_{\nu} \}_{\nu >0}$ to depend on $\nu >0$. Nevertheless, we want to rule out trivial examples of \eqref{e:zeroth-law}  given by solutions to the heat equation or the linear Stokes equations with forces depending on $\nu \geq 0$
 (see for instance \cite[Remark 1.2]{BCCDLS22}) in order to get the non linearity in \eqref{e:NSE} involved in the zeroth law of turbulence. 
 Hence a uniform in $\nu$ regularity must be imposed on the sequence of body forces $\{ F_\nu \}_{\nu > 0}$, following the proposal by \cite{BDL22}
 \begin{align} \label{e:regularity-forces-nu}
     \sup_{\nu > 0} \| F_{\nu } \|_{L^{1 + \sigma} ((0,1); C^{\sigma} (\T^d))} < \infty \,.
 \end{align}

 \begin{remark}
    As already noticed in \cite{BDL22}, if $v_{\nu} : [0,1] \times \T^d \to \R^d $ solves the linear Stokes equations  $\partial_t v_\nu - \nu \Delta v_{\nu} + \nabla p_\nu= F_{\nu}$ with divergence free initial datum $v_{\initial} \in L^2$, pressure $p_\nu : [0,1] \times \T^d \to \R$ solving $\Delta p_\nu = \diver F_\nu$, $\diver v_\nu =0$ and forces $\{F_\nu\}_{\nu \geq 0}$ which satisfies \eqref{e:regularity-forces-nu}, then the limsup in \eqref{e:zeroth-law} is 0. 
    Using energy estimates, we also notice that to rule out the
    anomalous dissipation 
    \eqref{e:zeroth-law} for the linear
    Stokes equations it is also sufficient that $F_\nu \to F_0$ in $L^1((0,1); L^2(\T^d))$ 
    as $\nu \to 0$.
\end{remark}

    We now coherently define {\em weak physical  solutions} and {\em anomalous dissipation measure}.


\begin{definition}[Weak physical solutions + anomalous dissipation measure] \label{d:weak-physical}
   The definition of {\em weak physical solutions} is the same as {\em physical solutions} as in Definition \ref{d:physical} replacing property \eqref{d:property-3} with: the forces $\{ F_\nu \}_{\nu >0}$ satisfies \eqref{e:regularity-forces-nu} or $F_\nu \to F_0$ in $L^1 ((0,1) \times \T^d)$.
     %We call $\{ v_{\nu_q} \}_{q \in \N}$ approximating sequence of the {\em weak physical solution} $v$.
     
     We say that $\mu \in \mathcal{M} ((0,1) \times \T^d)$ is an {\em anomalous dissipation measure } associated to $v$ if (up to not relabelled subsequences) we have
     $$ \mu= \lim_{\nu \to 0} (\nu |\nabla v_\nu|^2 + \mathcal{D}[v_\nu]) \,.$$
     We finally say that the push-forward $\mu_T = \pi_{\#} \mu \in \mathcal{M}((0,1))$ with respect to the projection in time map $\pi : (0,1) \times \T^d \to (0,1)$ is  the  corresponding {\em anomalous dissipation measure in time}.
\end{definition}

\begin{remark}
Notice that if $v$ is a {\em weak physical solution} then there exists an anomalous dissipation measure $\mu$ associated to $v$, but uniqueness may fail.
\end{remark}
 
 
 In the very recent paper \cite{BDL22} the authors proved the existence of a {\em weak physical solution} $v \in L^\infty$ to the Euler equations with body force $F_0 \in L^\infty ((0,1); C^\alpha (\T^3))$ for any $\alpha <1$ such that up to subsequences  $\{ \nu_q \}_{q}$
 \begin{enumerate}
     \item \label{intro-1} $\| v_{\nu_q} - v \|_{L^3 (0,1) \times \T^3)} \to 0 $ as $\nu_q \to 0$,
     \item \label{intro-3}
     $\sup_{\nu_q} \| v_{\nu_q} \|_{L^\infty } < \infty \,,$
     \item \label{intro-2} $\| F_{\nu_q} - F_0 \|_{L^\infty ((0,1) ; C^\alpha (\T^3))}  \to 0$ as $\nu_q \to 0$.
 \end{enumerate}
 In \cite{BCCDLS22} the authors improved the result in the following way approaching the Onsager criticality: for any $\alpha <1/3$, there exists a {\em weak physical solution} such that \eqref{intro-1} holds, \eqref{intro-2}  holds with norm $\| \cdot \|_{L^{1+ \sigma} ((0,1); C^{\sigma} (\T^3))}$ for some $\sigma$ and \eqref{intro-3} holds with norm $\| \cdot \|_{L^3 ((0,1); C^\alpha (\T^3))}$.

 In both the results of \cite{BDL22,BCCDLS22} $  \nu |\nabla v_{\nu}|^2$ are weakly* converging in measure (up to not relabelled subsequences) to a measure which is singular in time concentrated at $t=1$. In the following theorem we provide an example of a {\em weak physical solution} such that 
  \eqref{e:zeroth-law} holds and (up to not relabelled subsequences)
  $$\nu |\nabla v_{\nu}|^2 \rightharpoonup^* \mu $$
  weakly* in the sense of measures and 
  $\mu_T = \pi_{\#} \mu $ has a non-trivial absolutely continuous part with respect the Lebesgue measure $\mathcal{L}^1$ on $(0,1)$. 
   We remark that the absolute  continuity of the measure $ \pi_{\#} \mu $ is observed  both in experiments \cite{sree84,PKV02} and simulations \cite{KI03,sree98} of turbulent flows. More precisely at a mathematical level, assuming that the sequence $\{v_\nu \}_{\nu >0}$ has the following uniform in viscosity bound 
  $$ \sup_{\nu \in (0,1)} \| v_\nu \|_{L^3_t B^{1/3}_{3, \infty}} < \infty \,,
  $$
 which is in accordance with the Kolmogorov four-fifth law, we rigorously justify the absolute continuity of $\pi_\# \mu$ thanks to \cite[Theorem 1.6]{Isett13}. 
 Furthermore, we give positive answers to two open problems in fluid dynamics \cite[Question 2.2 and Question 2.3]{BDL22} related to the Navier-Stokes equations on the $4$-dimensional torus.
 
 

 
  %This discussion also motivates to extend the result \cite[Proposition 2]{DR00} to have the existence of a distribution $\mathcal{D}[v]$ such that the identity \eqref{duchon-robert} holds distributionally in $[0,1] \times \T^d$ under suitable assumptions, which to the authors seems technical and not obvious.
 
  
 
 \begin{maintheorem}  \label{t_Onsager}
Let $\beta \in (0, 1/4)$. For any $\alpha \in [0, 1)$ there exist a divergence-free initial datum $v_{in} \in L^\infty(\T^4; \R^4) $ with $\| v_{\initial} \|_{L^2 (\T^4)} =1$
%a sequence of initial data $v_{\initial, \nu} \in C^\infty(\T^4; \R^4) $ $\| v_{\initial} \|_{L^2 (\T^4)} \geq 1/2$ for any $\nu \in (0,1)$ such that $v_{\nu, \initial } \to v_{\initial}$ in $L^\infty$ as $\nu \to 0$
and a time-independent force $F_0 \in C^{\alpha } (\T^4; \R^4)$  such that there exists a weak physical  solution $v_0 \in L^\infty$ of the $4d$ Euler equations with force $F_0$ such that 
\begin{equation}\label{eq:EnergyInEuler}
e(t) = \frac{1}{2} \int_{\T^4} | v_0 (t, x )|^2 dx
\end{equation}
is smooth in $[0,1]$, non-increasing, $e(1) < e(0)$ and  $\int_{\T^4} F_0 (x) \cdot v_0 (t,x) dx =0 $ for any $t \in (0,1)$.

More precisely there exists a sequence of forces $\{ F_\nu \}_{\nu \geq 0} \subset C^{\alpha } (  \T^4; \R^4)$ and initial data $\{ v_{\initial, \nu} \}_{\nu > 0} \subset C^\infty(\T^4; \R^4) $ such that $v_{\nu, \initial} \to v_{\initial}$ in $L^2(\T^4)$, $F_{\nu} \to F_0$ in $C^{\alpha} (\T^4) $ as $\nu \to 0$ and for any $\nu > 0$ there exists a unique smooth solution $(v_{\nu}, p_\nu)$ to \eqref{e:NSE} with $v_\nu(0,\cdot) = v_{\initial}(\cdot)$ satisfying $\sup_{\nu \in [0,1]} \| v_\nu \|_{L^{\infty } ([0,1] \times \T^4)} < \infty$, 
the anomalous dissipation \eqref{e:zeroth-law} and 
 $(v_\nu , p_\nu) \weak (v_0, p_0)$ weakly$\ast$ in $L^\infty ((0,1) \times \T^3 )$.
 
 Furthermore, there exists an anomalous dissipation measure $\mu \in \mathcal{M}((0,1) \times \T^4)$  such that $\| \mu \|_{TV} \geq 1/4$ and  up to not relabelled subsequences we have
 \begin{align} \label{e:convergence-measure}
     \nu |\nabla v_\nu|^2 \rightharpoonup^* \mu
 \end{align} 
 in the sense of measures in $(0,1) \times \T^4$ and $\mu$ is close to $\mathcal{D}[v_0]$, given by formula \eqref{duchon-robert}, in $H^{-1}$
 \begin{align} \label{e:mu-duchon-robert}
     \| \mu - \mathcal{D}[v_0] \|_{H^{-1} ((0,1) \times \T^4)} \leq \beta \,.
 \end{align} 
The corresponding anomalous dissipation measure in time $\mu_T = \pi_{\#} \mu  \in \mathcal{M} (0,1)$ has a non-trivial absolutely continuous part w.r.t. the Lebesgue measure $\mathcal{L}^1$ and its singular part $\mu_{T, \text{sing}}$ is such that $\| \mu_{T, \text{sing}} \|_{TV} \leq \beta$.
 %, where to be precise  $\mu_T = \pi_{\# } \mu  \in \mathcal{M} (0,1)$  is the push-forward of $\mu$ through the projection in time map $\pi : (0,1) \times \T^4 \to (0,1) $ defined as $\pi(t,x) = t$.
\end{maintheorem}

\begin{remark}
    We highlight the following facts:
    \begin{itemize}
        \item We prove \eqref{e:mu-duchon-robert} by showing that up to not relabelled subsequences $v_\nu$ is close to $v_0$ in $L^2((0,1) \times \T^4)$, namely $\| v_{\nu} - v_0 \|_{L^2}^2 \leq \beta$ and the uniform in $\nu>0$ $L^\infty$ bound on $v_\nu$. However,  the authors do not know if it is true that $v_\nu \to v_0$ in $L^2$ up to not relabelled subsequences.
        \item If one could show the $L^2 ((0,1) \times \T^4)$ strong convergence of $v_\nu$ to $v_0$ (up to not relabelled subsequences) in the previous theorem, then one would be able to show in the statement that  (up to not relabelled subsequences) the following distributional limit holds
        \begin{align} \label{e:kolmo-duchon-robert}
        \mathcal{D}[v_0]= \lim_{\nu \to 0} (\nu |\nabla v_\nu|^2 + \mathcal{D}[v_\nu])
        \end{align}
        thanks to the uniform in $\nu>0$ $L^\infty$ bound on  $v_\nu$, which implies the $L^3$ strong convergence of $v_\nu$ to $v_0$, because all the terms pass into the distributional limit in \eqref{duchon-robert}. Unfortunately, the $L^2$ strong convergence is a non-trivial property. Observe also that for any $\nu >0$ in our setting $\mathcal{D}[v_\nu] \equiv 0$ because the solution $v_\nu$ is smooth for any $\nu >0$.
        \item The examples in \cite{BDL22,BCCDLS22} have the strong convergence property of $v_\nu$ to $v_0$ in $L^3$, however their measures $\mu$ defined by \eqref{e:convergence-measure} are concentrated at time $t=1$, therefore the distributional limit in $(0,1) \times \T^3$ \eqref{e:kolmo-duchon-robert} holds, but in this case since we test with compactly supported functions in the time interval $(0,1)$, the limiting Duchon-Robert distribution is such that $\mathcal{D}[v_0] \equiv 0$ in  $(0,1) \times \T^3$. Notice that we can extend those examples ``in a natural way'' to $(0,2) \times \T^3$, but then the proof in \cite{BDL22,BCCDLS22} of the strong convergence of $v_\nu$ to $v_0$ as $\nu \to 0$ in $L^3$ fails (which does not allow us to conclude the distributional limit \eqref{e:convergence-measure} in a straightforward way). 
        %\item We notice that our measure $\mu$ defined in \eqref{e:convergence-measure} is such that $\mu_T (t) = \mu (t, \T^4) $ has a non-trivial absolutely continuous measure in $(0,1)$, but we believe that $\mu$ is supported in $z=1/2$.
    \end{itemize}
\end{remark}


We remark that the distributional identity \eqref{e:kolmo-duchon-robert} implies that the Duchon-Robert distribution is a measure (under the condition that the vanishing viscosity sequence is a sequence of Leray solutions), but this identity is not always satisfied (see Proposition \ref{prop:duchon-anomalous} for an example), proving that in some cases the vanishing viscosity sequence of the 3D forced  Navier--Stokes equations \eqref{e:NSE} is not a good approximation in $L^p_{t,x}$ (for any $p \in [1, \infty]$) of the 3D forced Euler equations (see Corollary \ref{corollary:duchon-anomalous}).
This discussion motivates the following 
 open problem.
 
 %Kolmogorov theory of turbulence \cite{kolmo} is the following {\color{red} cite also Eyink Sreenivasan 2006- Onsager... (equation 50)}.
\begin{OQ} \label{OQ-physical}
Let $d \geq 3$. 
     Is there a (weak) {\em physical solution} $v_0$ to the $d$-dimensional Euler equations with force $F_0$ such that the distributional limit
    \begin{align*} 
        \mathcal{D}[v_0]= \lim_{\nu \to 0} (\nu |\nabla v_\nu|^2 + \mathcal{D}[v_\nu])
    \end{align*}
    holds and the Duchon--Robert distribution  $\mathcal{D}[v_0]$ (see \eqref{duchon-robert}) is not identically zero, proving in particular that in this case the Duchon--Robert distribution is a nontrivial {\em anomalous dissipation measure}. Furthermore, is it possible to have one of the following?
    \begin{itemize}
        \item $\mathcal{D}[v_0]$ is an absolutely continuous measure with respect to the Lebesgue measure $\mathcal{L}^1 \otimes \mathcal{L}^d$.
        \item The spatial dimension of $\supp  ( \mathcal{D}[v_0]) $ is $\gamma \in (d-1, d)$.
        We refer to  \cite[Definition 2.5 and Definition 2.8]{DRI22} for a mathematical definition of the spatial dimension of $\supp  \mathcal{D}[v_0]$.\footnote{The definition of the spatial dimension of the set $\supp ( \mathcal{D}[v_0])$ should be a mathematical definition coherent to the ``fractal dimension'' used in \cite{M74,M75} and described in \cite{FSN78} as  ``a measure of the extent to which the regions in which dissipation is concentrated fill space'', which can possibly differ from those proposed in \cite{DRI22} for the purpose of the open question.}
        %something , but should be something rigorous which characterize the imprecise quantity $\sup_{t \in (0,1)} \dim (\supp (\mathcal{D}[v_0] (t, \cdot ))) \in [0, d]$.} 
    \end{itemize} 
\end{OQ}




To the authors knowledge, also finding a single example of the identity \eqref{e:kolmo-duchon-robert} (where the Duchon--Robert distribution is non-trivial) is an open problem, which is indeed one part of Open Question \ref{OQ-physical} (see \cite[Section B]{eyinksurvey} for a discussion about that identity).
%The possibility that $\mathcal{D}[v_0]$ is absolutely continuous with respect to $\mathcal{L}^1 \otimes \mathcal{L}^d$ is motivated by the homogeneity assumption in \cite{kolmo}. 
The reason for which it is relevant to ask for $\mathcal{D}[v_0]$ to be absolutely continuous with respect to $\mathcal{L}^1 \otimes \mathcal{L}^d$ is the homogeneity assumption in \cite{kolmo}. The second point of Open Question \ref{OQ-physical} is motivated by the $\beta$-model for intermittent turbulent flows proposed by Frisch-Sulem-Nelkin in \cite{FSN78}, which predicts a correction of the so called structure function  in terms of the spatial dimension of $\supp  \mathcal{D}[v_0]$. In \cite{AGH84} the authors fit experimental data to predict that the spatial dimension of $\mathcal{D}[v_0]$ for turbulent flows is  close to $2.8 \in (2,3)$ in dimension $d=3$.

\begin{remark}
    Theorem \ref{t_Onsager}, in this theory, provides an example which is close to get the identity \eqref{e:kolmo-duchon-robert} thanks to \eqref{e:mu-duchon-robert} and  we believe (without a proof of it) that there exists $0< t_1 < t_2 < 1$ such that  $\supp (\mathcal{D}[v_0]) = [t_1, t_2] \times A \times \{ 1/2\} \times \T  \subset (0,1) \times \T^2 \times \T \times \T = (0,1) \times \T^4$, where $\mathcal{L}^2(A) >0$ which in particular implies that the dimension of $\supp (\mathcal{D}[v_0])$ is $4$ in the $5$-dimensional set $(0,1) \times \T^4$.  Indeed, notice that if we were able to prove the identity \eqref{e:kolmo-duchon-robert} proving in particular that the Duchon--Robert distribution is a measure, we could have applied \cite[Theorem 1.2]{DRDI22} to conclude that in our case the Hausdorff dimension of the Duchon--Robert distribution  is at least $4$  in $(0,1) \times \T^4$ (see also \cite{LS18,LS18-SIAM,S05} for related results). 
\end{remark}





Another interesting open problem is asking if one can get the absolutely continuous in time anomalous dissipation measure with the  optimal regularity extending Theorem \ref{t_Onsager}. 
%Optimal both in the Onsager critical space and absolutely continuous in time anomalous dissipation measure.
 Notice that in \cite{BCCDLS22} the authors get the Onsager critical regularity $L^3_t C^{1/3 - \varepsilon}_x$ but the anomalous dissipation measure in that case is expected to be a Dirac delta at time $t=1$ (without a proof of it). 

\begin{OQ}
 Let $\varepsilon>0$, $d \geq 3$. Is there a (weak) {\em physical  solution} to the $d$-dimensional Euler equations with force $F_0$ such that the following hold: \eqref{e:kolmo-duchon-robert}, \eqref{e:zeroth-law}, the corresponding anomalous dissipation measure in time (see Definition \ref{d:weak-physical})
 is absolutely continuous w.r.t. the Lebesgue measure $\mathcal{L}^1$ and
    $$ \sup_{\nu \in (0,1)} \| v_\nu \|_{L^3_t C^{1/3- \varepsilon}} < \infty \,?$$
\end{OQ}






 



 
  The proof of Theorem \ref{t_Onsager} relies on the study of the advection--diffusion equation 
\begin{align} \label{e:ADV-DIFF}
\tag{ADV-DIFF}
\partial_t  \theta_\kappa +  u  \cdot \nabla  \theta_\kappa = \kappa \Delta  \theta_\kappa,
\\
 \theta_{\kappa} (0, \cdot ) = \theta_{\initial} (\cdot )\, . \notag
\end{align}
 where $u$ is a given divergence free autonomous velocity field $u : \T^3 \to \R^3$, $\theta_{\initial } : \T^3 \to \R$ is a given initial datum, $\kappa \geq 0$ is the diffusivity parameter and the unknown is $\theta_{\kappa} : [0,1] \times \T^3 \to \R$. In the case of $\kappa=0$ this equation is known as the advection equation. The basic physical example is the advection of the temperature through diffusive means. 
Our study  is motivated by the Obukhov and Corrsin \cite{Obukhov,Corrsin} prediction of the same scaling as the Navier--Stokes equations in the K41 theory \cite{kolmo} for solutions of the advection-diffusion equation with a turbulent advecting flow in the idealized regime of small diffusivity $\kappa >0$.
However, the only known examples of anomalous dissipation
\begin{equation}\label{diss_main_OC} 
\limsup_{\kappa \to 0}  \, \kappa \int_0^1 \int_{\T^2} | \nabla \theta_\kappa|^2 \, dx\,dt >0 
\end{equation}
for the advection-diffusion equation are \cite{Gautam,CCS22} where in both examples the  dissipation is happening in a single time $t=1$. To clarify our result we recall the definition of the mean energy dissipation given in \cite[p. 20]{frisch}.
\begin{definition}
Let $d \in \N$ and $f \in L^2 ((0,T) \times \T^d)$ we define the mean energy dissipation as 
$$\mathcal{D}_T [f] = -  \frac{1}{2} \frac{d}{dt} \int_{\T^d} |f(x,t)|^2 dx  \,.$$
\end{definition}



We remark that there are several examples in the literature for the advection equation in which the mean energy dissipation is not identically zero and $L^1 (0,1)$ (see for instance \cite{Modena1,Modena2,Modena3,BDLC20,SG21,PS21,chesluo1,chesluo2,MB22}), relying on the convex integration technique, however none of these examples are proved to be vanishing diffusivity limit of a sequence $\{ \theta_\kappa \}_{\kappa >0}$ of (unique) solutions to \eqref{e:ADV-DIFF} which satisfies \eqref{diss_main_OC}.
 Inspired by the very recent construction of the velocity field as in \cite{CCS22} we prove the following result.

 
 \begin{maintheorem}\label{t_anomalous_autonomous}
 Let  $\alpha\in [0,1)$ and $\beta >0$, then there exists an autonomous divergence-free velocity field $u \in C^\alpha (\T^3)$ and an initial datum $\theta_{\initial} \in C^\infty(\T^3)$ with $ \int_{\T^3} \theta_{\initial}=0$, $\| \theta_{\initial} \|_{L^2} =1$ such that the unique solutions $\theta_\kappa$ of the advection-diffusion equation \eqref{e:ADV-DIFF} with initial datum $\theta_{\initial}$  exhibit  anomalous  dissipation \eqref{diss_main_OC}
and there exists a measure $\mu_T \in \mathcal{M}(0,1)$ so that up to non-relabelled subsequences
\begin{equation} \label{eq:absolutely_continuous}
    \mathcal{D}_T [\theta_\kappa] \rightharpoonup \mu_T 
\end{equation}
weakly* in the sense of measure. In addition $ \| \mu_{T}\|_{TV} \geq 1/4$,  the absolutely continuous part of the measure $\mu_T$ w.r.t. the Lebsegue measure $\mathcal{L}^1$ is non-trivial, its singular part $\mu_{T, \text{sing}}$ is such that $\| \mu_{T, \text{sing}} \|_{TV} \leq \beta$ and
\begin{align}\label{e:closeness-H-1}
    \| \mathcal{D}_T [\theta_0] - \mu_T \|_{H^{-1} (0,1)} \leq \beta \,,
\end{align} 
where $\theta_0$ is the unique solution of \eqref{e:ADV-DIFF} with $\kappa =0$, velocity field $u$ and initial datum $\theta_{\initial}$.

Furthermore, $\theta_\kappa \overset{*}{\rightharpoonup} \theta_0 $ weakly* in $L^\infty ((0,1) \times \T^3)$ and (up to not relabelled subsequences) we have the closeness $\| \theta_{\kappa} - \theta_0 \|_{L^\infty ((0,1); L^2(\T^3))} < \beta$,  and 
$$e(t) = \int_{\T^3} | \theta_0 (t, x )|^2 dx  $$ 
is smooth in  $[0,1]$  and it is such that $e(1) < e(0)$.
 \end{maintheorem}

In this context we raise the following open problem.

\begin{OQ}
    Is it possible to construct a divergence--free autonomous velocity field $u \in L^2(\T^2)$ and an initial datum $\theta_{\initial} \in L^\infty$ such that the unique solutions $\theta_\kappa$ of \eqref{e:ADV-DIFF} satisfy \eqref{diss_main_OC}?
\end{OQ}

\subsection*{Plan of the paper} In Section \ref{section:notation} we introduce the notations. Then, in Section \ref{sec:Construction} we construct the initial datum and an autonomous velocity field $u : \T^3 \to \R^3$ that will be used for the proofs of Theorem \ref{t_Onsager} and Theorem \ref{t_anomalous_autonomous}.  Subsequently, in Section \ref{sec:PrelimForThmA} we add some well known preliminaries that will be used in the proofs. In Section \ref{sec:proof-thB} we prove Theorem \ref{t_anomalous_autonomous} and in Section \ref{sec:4D-NS-Proof} we prove Theorem \ref{t_Onsager}. Finally, in Section \ref{sec:duchon-robert-anomalous} we point out some differences between the Duchon--Robert distribution and the anomalous dissipation measure (see Proposition \ref{prop:duchon-anomalous}) and an application (see Corollary \ref{corollary:duchon-anomalous}).

%{\color{blue} notation: the sequence of ``dissipation'' we consider in the theorem are $\kappa_q | \theta_{\kappa_q}|^2$ and $\nu_q | \theta_{\nu_q}|^2$ which must be coherent with the proof. Check  the sign and the constant 2 in front of them in the proofs}

\subsection*{Acknowledgments} MS and CJ are supported by the SNSF Grant 182565 and by the Swiss State Secretariat for Education, Research and Innovation (SERI) under contract number MB22.00034.
The authors are grateful to Elia Bru\`e, Gianluca Crippa, Maria Colombo, Camillo De Lellis and Luigi De Rosa for fruitful discussions and useful remarks on the problem.


%\begin{comment}

\section{Notation} \label{section:notation}
In this section, we explain the notation throughout the paper. First we want to make it clear that we use $x$, $y$, $z$, $w$ as spatial coordinates (up to 4 dimensions). Sometimes, however, we use $x$ as a point in $\T^3$ or $\T^4$, when there is no risk of confusion.
For vector-valued functions $v \colon \T^d \to \R^m$ ($m \geq 2$) we denote the the $j$-th component as $v^{(j)}$. When we are interested in the function built by considering just a selection of components, we adopt the following convention: Let $\{ j_1, \ldots, j_k \} \subset \{ 1, \ldots, m\}$ be distinct integers. We define $v^{(j_1, \ldots, j_k)} \colon \T^d \to \R^m$ as
\[
 v^{(j_1, \ldots, j_k)}(x) \coloneqq
 \begin{pmatrix}
 v^{(j_1)}(x) \\
 \vdots \\
 v^{(j_k)}(x) \\ 
 \end{pmatrix}.
\]
%At several instances in the paper, we wish to compute the gradient with respect to the first two coordinates. This will be denoted as $\nabla_{x,y}$.

In a general dimension $d \in \N$, we denote the restriction of the gradient to the first two components as $\nabla_{x,y} = (\partial_x \,, \partial_y)$ and the restriction of the gradient to the first three components as $\nabla_{x,y,z} = (\partial_x \,, \partial_y \,, \partial_z)$.
Let $\{ A_q \}_{q \geq 1}$ and $\{ B_q \}_{q \geq 1}$ be two sequences. We write $A_q \lesssim B_q$ if there is a universal constant $C$ such that for all $q \geq 1$, we have $A_q \leq C B_q$. We write $A_q \lesssimlarge B_q$ if there are universal constants $C$ and $Q$ such that for all $q \geq Q$, $A_q \leq C B_q$. 
We use the notation $\mathcal{M} (X)$ for the space of Radon measure on a locally compact metric space $X$ and $\| \mu  \|_{TV}$ as the total variation of the measure $\mu \in \mathcal{M}(X)$ (see for instance \cite{rudin}).
Let $\rho \colon B_1 \to \R$ be an arbitrary non-negative function satisfying $\rho \in C^\infty_c (B_1) $, $\int_{\T^2} \rho \, dx = 1$ and $\| \nabla \rho \|_{L^{\infty}(\T^2)} \leq 2$. For any $0<r <1$, we define $\rho_r \colon B_r \to \R$ by $\rho_r(x) = r^{-2} \rho(\sfrac{x}{r})$. We extend it to $\T^2$ by letting it take the value $0$ outside $B_r$. For any function $f \colon \T^2 \to \R$, we define $(f)_{r} = f \star \rho_{r}$ i.e.
\[
 (f)_{r}(x) = (f \star \rho_{r})(x) = \int_{\T^2} f(y) \rho_{r}(x-y) \, dy.
\]

%\end{comment}

%\section{Strategy of the proof and heuristics} \label{section:heuristics}



\section{Construction of the vector field and initial datum}\label{sec:Construction}
%
%{\color{red} COMMENTS: 
%
%- The final stationary velocity field will be called $u$ and not $w$. 
%
%- I prefer ``velocity field'' instead of ``vector field''
%
%- We will use ``advection equation'' instead of ``transport equation''
%
%-  We use $\theta$
%
%- the components of the velocity field will be denoted by $u= (u^{(1)} , u^{(2)} , ..)$}


The goal of this section is to construct the vector field $u$ and the initial datum ${\theta}_{in}$ in Theorem~\ref{t_anomalous_autonomous} as well as the initial datum $v_{\initial}$ and the forces $\{ F_{\nu} \}_{\nu \geq 0}$ in Theorem~\ref{t_Onsager}. The construction relies on a result from \cite{CCS22}. 
 In Subsection~\ref{subsec:ParmeterDefinitions}, we introduce all the parameters we will need. In Subsection~\ref{subsec:ConstructionVectorField}, we construct $u$ and  in Subsection~\ref{subsec:ConstructionInitialDatum}, we build ${\theta}_{in}$. Then, in Subsection~\ref{subsec:ConstructionsForNavierStokes4D}, we build $v_{\initial}$ and $\{ F_{\nu} \}_{\nu \geq 0}$ for Theorem~\ref{t_Onsager}.
\subsection{Choice of parameters}\label{subsec:ParmeterDefinitions}
%In this section, we introduce some parameters that are necessary for the construction of the velocity field.
We consider  $\alpha  < 1$ as in the statement of Theorem \ref{t_anomalous_autonomous}  and 
\begin{enumerate}
 \myitem{P1} $1 - \alpha(1 + \eps \delta)(1 + \delta) - \frac{\delta}{4} > 0$, \label{item:EpsDeltaOne}
 \smallskip
 \myitem{P2} $\eps \leq \frac{\delta^3}{200}$, \label{item:EpsDeltaThree}
\end{enumerate}
We notice  that the existence of $\eps$ and $\delta$  sufficiently small satisfying \eqref{item:EpsDeltaOne} follows from the condition $\alpha <1$ and up to consider $\eps = \eps (\delta) >0$ smaller we can $\eps $ and $\delta$ such that \eqref{item:EpsDeltaOne}, \eqref{item:EpsDeltaThree} are satisfied simultaneously.

Given $a \in (0,1)$ such that $a^{\sfrac{\eps \delta}{8}} +  a^{{\eps \delta^2}}  \leq \frac{1}{20}$, we select $a_0 \leq a$ (to be determined depending on $\beta >0$ and universal constants) and define
\begin{align}\label{d:a_q-sequence}
a_{q + 1}  =  a_q^{1 + \delta}
\end{align}
 for all $q \geq 0$. As pointed out in \cite{CCS22},  the sequence  $\{ a_{q } \}_{q \in \N}$ should be selected in such a way that we can guarantee that $\sfrac{a_{q}}{a_{q+1}}$ is a multiple of 4. This affects the proof only with fixed numerical constants in the estimates and would make the reading more technical, therefore we avoid it.
We fix the parameter
\begin{align} \label{d:gamma}
\gamma =  \frac{\delta}{8}.
\end{align}
 
We fix an integer $m \geq 1$ such that
\begin{equation}\label{eq:InequalityForTheConstantM}
 m - 1 \geq \frac{16}{\delta^2}
\end{equation}
and define the sequence of times
\[
 \left\{
 \begin{array}{ll}
  t_q  =  \frac{1}{4} a_q^{\gamma} & \text{for all $q \in \N$}; \\
  \smallskip
  \overline{t}_q  = \frac{1}{4} a_q^{\gamma - \gamma \delta} & \text{for all $q \in m\N \setminus \{ 0 \}$}; \\
  \smallskip
  \overline{t}_q  =  0 & \text{for all $q \not\in m\N$}. \\
 \end{array}
 \right.
\]
Define the sequence $\{ T_q \}_{q = 0}^{\infty}$ as 
\[
 T_q  =  \sum_{j \geq q} \overline{t}_j + 3 \sum_{j \geq q} t_j < \frac{1}{4} \quad \forall q \geq 0.
\]
Define the intervals
\begin{align*}
 &\mathcal{I}_{-1}  =  (\sfrac{1}{4}, \sfrac{1}{2} - T_0]; \\
 &\mathcal{I}_{q,0}  =  (\sfrac{1}{2} - T_q, \sfrac{1}{2} - T_q + \overline{t}_q] \quad \forall q \geq 0;\\
 &\mathcal{I}_{q,i}  =  (\sfrac{1}{2} - T_q + \overline{t}_q + (i-1)t_q, \sfrac{1}{2} - T_q + \overline{t}_q + i t_q] \quad \forall q \geq 0, \, \forall i = 1,2,3;
\end{align*}
and 
$$ \mathcal{I}_q = \bigcup_{i=0}^3 \mathcal{I}_{q,i} \,, \qquad \mathcal{I} = \bigcup_{q =0}^{\infty}  \bigcup_{i=1}^3 \mathcal{I}_{q,i}$$
Finally, we define the sequences of diffusivity parameters $\{ \kappa_q \}_{q=0}^{\infty}$ and the sequence of viscosity parameters $\{ \nu_q \}_{q=0}^{\infty}$ as
\begin{equation} \label{d:k_q}
 \kappa_q = \nu_q  =  a_q^{2 - \frac{\gamma}{1 + \delta} + 10 \eps}\,.
\end{equation}
%The sequence $\{ \kappa_q \}_{q=0}^{\infty}$ will be used whenever the main object of study is the advection-diffusion / advection equation (i.e. Theorem~\ref{t_anomalous_autonomous}) and $\{ \nu_q \}_{q=0}^{\infty}$ will be used whenever the main object of study is the Navier-Stokes / Euler equation (i.e. Theorem~\ref{t_Onsager}). }
 
  \subsection{Construction of the vector field}\label{subsec:ConstructionVectorField}
In this section we prove the existence of a velocity field, relying on \cite{CCS22}, with all the properties needed to prove Theorem \ref{t_anomalous_autonomous} and Theorem \ref{t_Onsager}. We recall that $\T^3 \cong \R^3 / \Z^3$ and we use the convenient notation of identifying the subsets $A \subset \T^3$ as $A \subset [0,1)^3$.

\begin{definition}
    We  say that $\theta: \T^2 \times [0,\infty) \to \R$ is a bounded stationary solution of the advection equation with a divergence free velocity field $u \in L^\infty ( \T^2 \times (0, \infty) )$ if 
    $$ \int_{\T^2 \times [0, \infty )} (u \theta) \cdot \nabla \varphi =0 \qquad \text{for any } \varphi \in C^\infty_c (  \T^2 \times (0,\infty))\,.$$
\end{definition}
  

 \begin{proposition}\label{prop:ResultFromPreviousPaper}
 \emph{(Existence of a vector field)} There exists an autonomous divergence-free velocity field ${u} \in C^{\infty}_{\text{loc}}( \T^2 \times (0,1) \setminus \{ \sfrac{1}{2} \} ) \cap C^\alpha (\T^3)$ with the following estimates:
 for any $k \in \N$ and $j \in \N$ there exists a constant $C>0$ such that
 \begin{align} \label{prop:estimate_u}
 \| \partial^k_z \nabla_{x,y}^j  u \|_{L^\infty (\T^2 \times \mathcal{I}_q)} \leq C a_q^{1- \gamma} a_q^{- k \gamma} a_{q+1}^{- j (1+ \varepsilon \delta ) } \,, \quad \supp \{ u^{(1,2)} \} \subset  \mathcal{I} \times \T^3
 \end{align}

 
 %\begin{enumerate}
% \item ${u} = 0$ on $\mathcal{I}_{-1}$, $\mathcal{I}_{q,0}$ and $\mathcal{I}_{q,1}$ for all $q$; \label{item:VectorFieldFromPaperProp1}
% \item $\| \overline{u} \|_{L^{\infty}(\mathcal{I}_{q,2} \times \T^2)} \leq 4 a_q^{1 - \gamma}$, $\| \nabla \overline{u} \|_{L^{\infty}(\mathcal{I}_{q,2} \times \T^2)} \leq 4 a_q^{- \gamma}  (2 a_q)^{-1} ^{-1} \lambda_{q+1}^{1 + \eps \delta}$; \label{item:VectorFieldFromPaperProp2}
% \item $\| \overline{u} \|_{L^{\infty}(\mathcal{I}_{q,3} \times \T^2)} \leq 8 a_{q+1} a_q^{- \gamma}$, $\| \nabla \overline{u} \|_{L^{\infty}(\mathcal{I}_{q,3} \times \T^2)} \leq 4 a_q^{- \gamma} \lambda_{q+1}^{\eps \delta}$; \label{item:VectorFieldFromPaperProp3}
% \item $\| \overline{u} \|_{L^{\infty}(\T^2; C^k(1,3))} \lesssim \sup_{q} a_q^{1 - \gamma - 2 k \gamma} < \infty$. \label{item:VectorFieldFromPaperProp4}
% \end{enumerate}
% \emph{(Properties of a solution to the transport equation $\partial_t \overline{\theta} + \overline{u} \cdot \nabla \overline{\theta}  = 0$)} There exists a $\overline{\theta} \colon [\sfrac{1}{4},\sfrac{3}{4}] \times \T^2 \to \R$ solving the transport equation $\partial_t \overline{\theta}  + \overline{u} \cdot \nabla \overline{\theta} = 0$ such that the following properties are valid:
Furthermore, extending $u$ to $\T^2 \times [0, \infty)$ 1-periodically with respect to the third variable, there exists  a bounded stationary solution $ {\theta}_{\stat}:  \T^2 \times [0,\infty) \to \R$ to the advection equation  with the following properties:
 \begin{enumerate}
 \myitem{S1} there exists a bounded function with zero-average  $\theta_{\chess} \colon \T^2 \to \R$ such that \\ $\| {\theta}_{\stat} (\cdot, \cdot , z)  - \theta_{\chess}((2a_q)^{-1} \cdot, (2a_q)^{-1} \cdot ) \|_{L^2(\T^2)}^2 < 40 a_0^{\eps \delta}$ 
  for all $z \in \mathcal{I}_{q,0} \cup \mathcal{I}_{q,1}$ for all $q \in \N$; \label{eq:SolutionTransportEquationProp1}
 \myitem{S2} $\| {\theta}_{\stat} (\cdot, \cdot , z) \|_{L^2(\T^2)}^2 = \| {\theta}_{\stat} (\cdot, \cdot , \sfrac{1}{4}) \|_{L^2(\T^2)}^2 > (1 - 40 a_0^{\eps \delta})$ for all $z \in [0, \sfrac{1}{2} - T_0)$; \label{eq:SolutionTransportEquationProp2} 
 \myitem{S3} there exists a constant $C>0$ such that for any $k, j \in \{0,1 \}$ we have 
 \begin{align} \label{eq:vartheta_stationary}
 \| \partial_z^{k} \nabla_{x,y}^{j} \theta_{\stat} \|_{ L^\infty (\T^2 \times \mathcal{I}_q)} \leq  C a_q^{- k \gamma}  a_{q+1}^{j(-1 - 3 \eps (1 + \delta))} \qquad \text{ for all } q \in \N\,.
 \end{align}
 \label{eq:SolutionTransportEquationProp3}
 \myitem{S4} \label{item:dissipation-theta_0}$\| \theta_{\stat}(\cdot, \cdot, z) \|_{L^2(\T^2)}^2 \leq a_0^{\sfrac{\eps \delta}{2}}$ for all $z > \sfrac{1}{2}$.
 \end{enumerate}
 \end{proposition} 

 

 \begin{proof}
 Let $\tilde u \in C_{loc}^{\infty}((0,1)  \times \T^2)$ be the vector field constructed in \cite[Section 4.2]{CCS22} with the choice of parameters $\alpha \in (0,1)$ as in \eqref{item:EpsDeltaOne}, $\varepsilon , \delta $ as in \eqref{item:EpsDeltaThree} and $\beta =0$. We extend this velocity field $\tilde u $ in $[-1,3] \times \T^2$ defining $\tilde u (t, \cdot) \equiv 0$ for any $t \in (1, 3) \cup (-1, 0)$. By construction, we have that $\tilde u \in C^\alpha ([-1,3] \times \T^2 ;\R^2) \cap C^\infty_{\text{loc}} (([-1,3] \setminus \{ 1\}) \times \T^2 ; \R^2)$. Extend this vector field to $[-1, \infty) \times \T^2$ so that it is 4-periodic in time. We now rescale and translate  the velocity field $\tilde u$ in time defining 
 $$  \overline{u} (t,x) = 4 \tilde u (4t -1, x) \qquad \text{ for all } t \in (0, \infty ) \, , x \in \T^2 \,.$$

Finally, we define
 the autonomous velocity field $u \in  C^{\infty}_{\text{loc}}((0,1) \setminus \{ \sfrac{1}{2} \} \times \T^2; \R^3)  \cap C^\alpha (\T^3 ;  \R^3)   $ as
 $$u (x,y,z) = \left( \begin{array}{c}
\overline{u}^{(1)} (z, x, y) \\
 \overline{u}^{(2)} (z, x, y) \\
 1
 \end{array} \right)  \qquad \text{ for all } (x,y,z ) \in \T^3  $$
 which verifies the estimate \eqref{prop:estimate_u} thanks to  \cite[Equations (4.20a), (4.20b), (4.20c), (4.22)]{CCS22}.
 We observe that $\overline{u}$ is bounded and 1-periodic in time, $\overline{u} \in C^\infty ([0,1/2) \times \T^2)$ and $\overline u (t, \cdot) \equiv 0$ for $t \in (1/2,1]$, therefore the forward integral curves of $\overline{u}$ are unique. Using the superposition principle (see \cite{A08}) we have that the bounded solutions of the advection equation are unique.

 Now, let $\overline{\theta} : [0, \infty) \times \T^2$ be the solution to the advection equation with velocity field $\overline{u}$ and initial datum $\overline{ \theta}_{\initial} (\cdot)= \theta_{\text{chess}, 0} \star \psi_0 (\cdot)$, where $\psi_0 (\cdot) = \psi ((2a_0)^{-1 - \epsilon \delta} \cdot)$ for a smooth and periodic function $\psi \in C^\infty (\T^2)$ and $\theta_{\text{chess}, 0}  (\cdot) = \theta_{\text{chess}} ((2a_0)^{-1} \cdot) $ and the chessboard function $\theta_{\text{chess}} \in L^\infty (\T^2) $ is defined as follows
 $$ \theta_{\text{chess}} (x,y) =\begin{cases}
1 \qquad \ \  x,y \in [0,1/2) \text{ or } x,y \in [1/2, 1);
\\
-1 \qquad \text{otherwise.}
\end{cases}$$ 
 
 
 
 We define $\theta_{\stat} \in L^\infty ((0, \infty) \times (\T^2 \times [0,\infty)) )$ as $\theta_{\stat} (t, x, y , z) = \overline{\theta} (z, x, y)$. It can be verified that $\theta_{\stat}$  is a (time-independent) distributional solution of the advection equation with velocity field $u : \T^2 \times [0, \infty ) \to \R^3$ and initial datum $\theta_{\stat}$.
 


 
 \begin{comment}
 $$\begin{cases} 
 \partial_t \theta  + u \cdot \nabla  \theta =0 \qquad \text{ in } (0,\infty) \times (\T^2 \times [0, \infty));
 \\
 \theta (0, \cdot )  = \theta_{\stat} (\cdot ) \,.
 \end{cases}$$
 \end{comment}

 We observe that \eqref{eq:SolutionTransportEquationProp1} follows from \cite[Equations (4.18), (4.19), (4.25a) and (4.25b)]{CCS22} and the estimate
$$ \sum_{q=0}^\infty 20 a_q^{\epsilon \delta} \leq  20 \sum_{q=0}^\infty a_0^{\epsilon \delta + \epsilon \delta^2 q} \leq 40 a_0^{\epsilon \delta},$$
where the last inequality follows from the choice of $a_0$.


 
  %Observe that \eqref{eq:SolutionTransportEquationProp1} follows from \cite[Equations (4.18), (4.19), (4.25a) and (4.25b)]{CCS22}. 
 Property \eqref{eq:SolutionTransportEquationProp2} follows from the choice of initial datum and 
 property \eqref{eq:SolutionTransportEquationProp3} is a consequence of \cite[Section 8]{CCS22}. Property \eqref{item:dissipation-theta_0} directly follows from Step 2 in \cite[Section 7]{CCS22}.
 \end{proof}



\subsection{Mollification of $\theta_{\chess}$} \label{rmk:AboutAMollifiedVersionOfTheChessFunction}
Since the function $\theta_{\chess}$ defined in Proposition \ref{prop:ResultFromPreviousPaper} is not smooth, we introduce a mollification of it, for future use.
We will consider the mollified function $(\theta_{\text{chess}})_{a_0^{\eps \delta}} = \theta_{\text{chess}} \star \rho_{a_0^{\eps \delta}} \colon \T^2 \to \R$.
 Notice that 
 \begin{equation*}
 \|   \theta_{\text{chess}} - (\theta_{\text{chess}})_{a_0^{\eps \delta}} \|_{L^2(\T^2)} \lesssim  a_0^{\eps \delta},
 \end{equation*}
 In virtue of the previous proposition, this means that 
 \begin{equation}\label{eq:L2DifferenceBetweenChessboardAndItsMollification}
 \| {\theta}_{\stat} (\cdot, \cdot , z)  - (\theta_{\text{chess}})_{a_0^{\eps \delta}}(\cdot, \cdot) \|_{L^2(\T^2)}^2 \lesssim  a_0^{\eps \delta}
  \text{ for all $z \in \mathcal{I}_{q,0} \cup \mathcal{I}_{q,1}$ for all $q \in \N$. }
 \end{equation}
 We also point out that
 \begin{equation}\label{eq:LInftyNormOfTheGradientOfTheMollifiedChessFunction}
  \| \nabla (\theta_{\text{chess}})_{a_0^{\eps \delta}} \|_{L^{\infty}(\T^2)} \leq 2 a_0^{- 3 \eps \delta}.
 \end{equation}
 
 \subsection{Construction of the initial datum ${\theta}_{in}$ and $\theta_0$ solution to \eqref{e:ADV-DIFF} with $\kappa =0$}\label{subsec:ConstructionInitialDatum}
 %In this subsection, we build the initial datum ${\theta}_{\initial}$ in Theorem~\ref{t_anomalous_autonomous}. 
 %Recall $w$ and $\tilde{\theta}$ constructed in the previous subsection. Define $\theta \colon [0,1] \times \T^3 \to \R$ as $\theta(t,x,y,z)  =  \tilde{\theta}(z,x,y)$. Then, $w \cdot \nabla \theta = 0$, meaning that $\theta$ is a stationary solution to the transport equation with vector field $w$. 
 Let $\varphi_{\initial} \in C^{\infty}(\R)$ be such that
 \[
  \supp(\varphi_{\initial}) \subset (0,\sfrac{1}{4}), \quad 0 \leq \varphi_{in} \leq 5, \quad \| \varphi_{in} \|_{L^2(\R)} \in (1,2)
 \]
 in such a way that 
 the initial datum defined as
\begin{align}\label{eq:initialdatum_theta}
\theta_{\initial} (x,y,z) = \theta_{\stat} (x,y,z) \varphi_{\initial} (z),
\end{align}
where $\theta_{\stat}$ is the function given by Proposition \ref{prop:ResultFromPreviousPaper}, is such that  
 \begin{equation}\label{eq:L2NormOfTheInitialDatum}
  \| {\theta}_{\initial} \|_{L^2(\T^3)} =1 \,,
 \end{equation}
where we used \eqref{eq:SolutionTransportEquationProp2} of Proposition \ref{prop:ResultFromPreviousPaper}.

We notice that for all $t \in (0, 1/2)$ we have that ${\theta}_0(t,x,y,z) = \theta_{\stat}(x,y,z) \varphi_{\initial}(z-t)$ is a solution to the advection equation with initial datum $\theta_{\stat}(x,y,z) \varphi_{\initial}(z)$ and velocity field $u$. This will be used in the proof of Theorem~\ref{t_Onsager} and \ref{t_anomalous_autonomous}.

\begin{comment}

 and define $\varphi (z,t ) =\varphi_{\initial}(z - t) $. We observe that $\varphi$ is the unique smooth solution of
 \begin{equation}
   \left\{
  \begin{array}{l}
   \partial_t \varphi + \partial_z \varphi = 0; \\
   \varphi(0,z) = \varphi_{\initial}(z). \\
  \end{array}
  \right.
 \end{equation}
 
By standard observations, $\varphi(t,z)  =  \varphi_{in}(z - t)$. Define $\overline{\theta} \colon [0,1] \times \T^3 \to \R$ as $\overline{\theta}(t,x,y,z)  =  \theta(t,x,y,z) \varphi(t,z)$. We define $\overline{\theta}_{in} \colon \T^3 \to \R$ as $\overline{\theta}_{in}(x,y,z)  =  \overline{\theta}(0,x,y,z)$. Standard computations yield that $\overline{\theta}$ solves
  \begin{equation}
   \left\{
  \begin{array}{l}
   \partial_t \overline{\theta} + w \cdot \nabla \overline{\theta} = 0; \\
   \overline{\theta}(0,x,y,z) = \overline{\theta}_{in}(x,y,z). \\
  \end{array}
  \right.
 \end{equation}
 In addition, we observe that due to \ref{eq:SolutionTransportEquationProp2}
 \begin{equation}\label{eq:L2NormOfTheInitialDatum}
  \| \overline{\theta}_{in} \|_{L^2(\T^3)} > \frac{9}{10}.
 \end{equation}
 \end{comment}
  

\subsection{Construction of the initial datum and the forces for Theorem~\ref{t_Onsager}}\label{subsec:ConstructionsForNavierStokes4D}
In this subsection, we define the objects relevant for the proof of Theorem~\ref{t_Onsager}. We begin by defining $u_0 = u$. 
For any $\nu \in ( \nu_{q+1},   \nu_q]$, we define 
%For each $\nu \in [0, a_0^2)$, we define a velocity field $u_{\nu} \colon \T^3 \to \R^3$ as follows. We begin by defining $u_0 = u$. 
%For each $\nu \in [0, a_0^2)$ define $q_{\nu}  =  \max \{ q \in \N : \nu_q \geq \nu \}$. Notice that with $\nu = \nu_{\overline{q}}$ for some $\overline{q}$, we get $q_{\nu} = \overline{q}$. Then for all $\nu \in (0, a_0^2)$, we define the velocity field $u_{\nu} \colon \T^3 \to \R^3$ as
\begin{align} \label{u_nu}
 u_{\nu} =
 \begin{pmatrix}
  u^{(1)} \mathbbm{1}_{\{ z < \sfrac{1}{2} - T_{q} \}} \\
  u^{(2)} \mathbbm{1}_{\{ z < \sfrac{1}{2} - T_{q} \}} \\
  u^{(3)} \\
 \end{pmatrix}
 =
 \begin{pmatrix}
  u^{(1)} \mathbbm{1}_{\{ z < \sfrac{1}{2} - T_{q} + \sfrac{a_q^{\gamma}}{4} \}} \\
  u^{(2)} \mathbbm{1}_{\{ z < \sfrac{1}{2} - T_{q} + \sfrac{a_q^{\gamma}}{4} \}} \\
  1 \\
 \end{pmatrix}.
\end{align}
and  $\theta_{\nu} \colon [0,1] \times \T^3 \to \R$ be the solution to the advection-diffusion equation
 \begin{equation}\label{eq:Special3DAdvDiff}
 \left\{
 \begin{array}{ll}
 \partial_t {\theta}_{\nu}+ u_{\nu} \cdot \nabla {\theta}_{\nu} = \nu \Delta {\theta}_{\nu}; \\
 {\theta}_{\nu} (0,x,y,z) = {\theta}_{\initial}(x,y,z). \\
 \end{array}
 \right.
\end{equation}
This is a slight abuse of notation since we have already defined $\theta_{\kappa}$ as solution to the advection-diffusion equation \eqref{e:ADV-DIFF} at the beginning of the paper. For this reason, we introduce the following convention: Whenever $\theta$ has a subscript with letter $\kappa$, it is a solution to \eqref{e:ADV-DIFF}, whereas if it has a subscript with letter $\nu$ it is a solution to \eqref{eq:Special3DAdvDiff}.
Now define $v_{\initial} \colon \T^4 \to \R^4$ for Theorem~\ref{t_Onsager} as
\begin{equation} \label{v-initial-nu}
 v_{\initial, \nu}(x,y,z,w)  = 
 \begin{pmatrix}
  u_{\nu}(x,y,z) \\
  \theta_{\initial}(x,y,z)
 \end{pmatrix}
 =
  \begin{pmatrix}
  u_{\nu}^{(1)}(x,y,z) \\
  u_{\nu}^{(2)}(x,y,z) \\
  u_{\nu}^{(3)}(x,y,z) \\
  \theta_{\initial}(x,y,z)
 \end{pmatrix}.
\end{equation}
Finally, we define the forces $\{ F_{\nu} \}_{\nu \geq 0}$ in Theorem~\ref{t_Onsager} as
\begin{equation}
 F_{\nu}(x,y,z,w)  = 
  \begin{pmatrix}
  \partial_z u_{\nu}^{(1)} - \nu \Delta u_{\nu}^{(1)} \\
  \partial_z u_{\nu}^{(2)} - \nu \Delta u_{\nu}^{(2)} \\
  \partial_z u_{\nu}^{(3)} - \nu \Delta u_{\nu}^{(3)} \\
  0
 \end{pmatrix}
 =
   \begin{pmatrix}
  \partial_z u_{\nu}^{(1)} - \nu \Delta u_{\nu}^{(1)} \\
  \partial_z u_{\nu}^{(2)} - \nu \Delta u_{\nu}^{(2)} \\
  0 \\
  0
 \end{pmatrix}.
\end{equation}

\begin{lemma}\label{lemma:AboutTheCollectionOfBodyForces}
 The collection $\{ F_\nu \}_{\nu \in (0,1)}$ is uniformly bounded in $C^{\alpha}(\T^4;\R^4)$ and $F_{\nu} \to F_0$ in $C^{\alpha}(\T^4;\R^4)$, i.e.
 \[
  \sup_{\nu \in (0,1)} \| F_{\nu} \|_{C^{\alpha}(\T^4; \R^4)} < \infty \quad \text{and} \quad \lim_{\nu \to 0} \| F_{\nu} - F_0 \|_{C^{\alpha}(\T^4; \R^4)}.
 \]
\end{lemma}

\begin{proof}

It suffices to prove that there exists $C>0$ such that for any $ \nu \in ( \nu_{q+1},   \nu_q]$ we have
\begin{equation}\label{e:regforce2}
\| \partial_z u^{(1,2)} \|_{C^{\alpha }(\{z \geq 1/2 - T_q + \sfrac{a_q^\gamma}{4} \} )} \to 0 
\qquad\text{ and } \qquad
\| \nu \Delta u_{\nu} \|_{C^\alpha(\T^4  )} \to 0  \,,
\end{equation} 
as $q \to \infty$.
We estimate the first term thanks to \eqref{prop:estimate_u}  and the interpolation inequality 
 \begin{align*}
 \| \partial_z u \|_{C^{\alpha } (\{z \geq 1/2 - T_q + \sfrac{a_q^\gamma}{4} \} )} & \lesssim  a_q^{1- 2 \gamma} a_{q+1}^{- \alpha (1+ \varepsilon \delta) } = a_q^{1- 2 \gamma - \alpha  (1 + \delta )(1+ \varepsilon \delta)} \to 0
 \end{align*} 
 as $q \to \infty$, where in the last limit we used $\gamma = \delta /8$ and \eqref{item:EpsDeltaOne}. We now show the second property in~\eqref{e:regforce2} using $\nu \in (  \nu_{q+1},  \nu_q]$ and \eqref{prop:estimate_u}
 \begin{align*}
 \| \nu \Delta u_\nu \|_{C^{\alpha}(\T^4)} & \leq   \nu_q  \| \Delta u  \|_{C^{\alpha } ( \{ z < 1/2 - T_q + \sfrac{a_q^\gamma}{4} )} 
 \lesssim  a_q^{2 - \frac{\gamma}{1 + \delta} + 10 \varepsilon}  a_{q-1}^{1- \gamma} a_{q}^{-2 -2 \varepsilon \delta} a_{q}^{- \alpha  (1+ \epsilon \delta)}
 \to 0
 \end{align*}
 as $q \to \infty$, in the last we used that 
 $a_q^{2 \epsilon} a_q^{- 2 \epsilon \delta } \leq 1$ and \eqref{item:EpsDeltaOne}.

\end{proof}
 
 \section{Preliminaries }\label{sec:PrelimForThmA}
 In this section, we present some preliminary lemmas that will be used in the proofs of Theorem~\ref{t_anomalous_autonomous} and \ref{t_Onsager}.
 
\subsection{Existence and uniqueness for advection-diffusion}
\begin{lemma} \label{lemma:local-energy}
    Let $d \geq 2$, $u : [0,1] \times \T^d \to \R^d$ be a divergence free velocity field such that $u \in L^\infty((0,1); L^2 (\T^3))$ and a bounded initial datum $\theta_{\initial} \in L^\infty (\T^d)$ then for any $\kappa >0$ there exists a unique solution $\theta_\kappa : [0,1] \times \T^d \to \R$ to the advection-diffusion \eqref{e:ADV-DIFF} such that
    \begin{itemize}
        \item $\theta_\kappa \in L^\infty((0,1) \times \T^3) \cap L^2 ((0,1) ; H^1(\T^d))$,
        \item the local energy equality holds in the sense of distribution 
        $$ \frac{1}{2} \partial_t |\theta_\kappa |^2 +  u \cdot \nabla \theta_\kappa \theta_{\kappa} + \kappa | \nabla \theta_\kappa |^2 = \frac{\kappa}{2} \Delta |\theta_\kappa|^2 \,. $$
        \item the global energy equality holds for all $s \in [0,1]$ for all $t >s $
        \begin{equation} \label{e:energy-equality-global}
           \frac{1}{2} \int_{\T^d} |\theta_\kappa |^2 (t,x ) dx + \kappa  \int_s^t \int_{\T^d}  | \nabla \theta_\kappa |^2(u,x) \, dx du = \frac{1}{2} \int_{\T^d} |\theta_\kappa |^2 (s,x ) dx   \,. 
        \end{equation} 
    \end{itemize}
\end{lemma}
\begin{proof}[Sketch of the proof]
    We provide here a sketch of the proof since it is a well known result. Consider a mollifier $\rho_\varepsilon \in C^\infty (\T^d)$ and $\theta_{\initial , \varepsilon } = \theta_{\initial} \star \rho_\varepsilon$ and $u_\varepsilon = u \star \rho_\varepsilon$. Then there exists a unique smooth solution to 
    \begin{align*}
        \partial_t \theta_{\kappa, \varepsilon} + u_\varepsilon \cdot \nabla \theta_{\kappa, \varepsilon} = \kappa \Delta \theta_{\kappa, \varepsilon}
        \\
        \theta_{\kappa, \varepsilon } (0, \cdot) = \theta_{\initial , \varepsilon} (\cdot )
    \end{align*}
    Multiplying the previous identity by $\theta_{\kappa, \varepsilon}$ and integrating in space-time we get a uniform in $\varepsilon$ bound 
    $$ \int_{\T^3} | \theta_{\kappa, \varepsilon} (t,x)|^2 dx + 2 \kappa \int_{0}^1 \int_{\T^3} |\nabla \theta_{\kappa, \varepsilon} |^2 = \int_{\T^3} |\theta_{\initial, \varepsilon }|^2 \leq \int_{\T^3} |\theta_{\initial }|^2 \,.$$

    Therefore,  up to subsequences, there exists $\theta_{\kappa} \in L^\infty((0,1) \times \T^3) \cap L^2((0,1); H^1(\T^3))$ such that $\theta_{\kappa, \varepsilon} \rightharpoonup^\star \theta_{\kappa}$ as $\varepsilon \to 0$. By the linearity of the equation it is straightforward to check that $\theta_{\kappa} $ is a solution to the advection-diffusion equation with velocity field $u$ and initial datum $\theta_{\initial}$. Finally mollifying the equation with a space-time mollifier $\psi_{\varepsilon} \in C^\infty ((0,1) \times \T^3)$ and multiplying the equation by $\theta_{\kappa, \varepsilon} = \theta_{\kappa} \star \psi_{\varepsilon} $ we have
    $$ \partial_t |\theta_{\kappa, \varepsilon}|^2 + 2 u_\varepsilon \cdot \nabla \theta_{\kappa, \varepsilon} \theta_{\kappa, \varepsilon}  = - 2 \kappa  | \nabla \theta_{\kappa, \varepsilon}|^2 + \kappa \Delta |\theta_{\kappa, \varepsilon}|^2 + R_{\varepsilon} \theta_{\kappa, \varepsilon} \,,$$
    where $R_{\varepsilon} =  2 u_\varepsilon \cdot \nabla \theta_{\kappa, \varepsilon} - 2 ( u \cdot \nabla \theta_{\kappa} )_\varepsilon$ where $( u \cdot \nabla \theta_{\kappa} )_\varepsilon = ( u \cdot \nabla \theta_{\kappa} ) \star \psi_{\varepsilon}$. It is straightforward to check that $\| R_{\varepsilon} \|_{L^1((0,1) \times \T^d)} \to 0$ and that all the other terms pass into the limit as $\varepsilon \to 0$.  Finally, suppose that there are two solutions $\theta_{\kappa,1}, \theta_{\kappa, 2} \in L^\infty((0,1) \times \T^3) \cap L^2 ((0,1); H^1(\T^d))$ then using the equation we observe that $\partial_t \theta_{\kappa, 1}, \partial_t \theta_{\kappa,2} \in  L^2 ((0,1); H^{-1} (\T^d))$, therefore by subtracting the two equations, multiplying the equation $\theta_{\kappa, 1} - \theta_{\kappa, 2}$ and integrating in space-time we get
    $$ \frac{d}{dt} \int_{\T^d} | \theta_{\kappa, 1}(t,x) - \theta_{\kappa,2}(t,x)|^2  dx \leq 0  \,,$$
    in the sense of distributions.
    Finally, observing by the heat equation regularity that $\partial_t \theta_\kappa \in L^2((0,1) \times \T^d)$ which implies that $\theta_\kappa \in C((0,1); L^2 (\T^d))$ we have the global energy equality integrating in space-time the local energy equality.
\end{proof}

 \subsection{Uniqueness result for the advection equation}
\begin{lemma} \label{lemma:uniqueness}
Let $u : \T^3 \to \R^3$ be a bounded, divergence-free velocity field such that $u \in C^{\infty} (\T^2 \times [0,1/2); \R^3)$, $u^{(3)} \equiv 1$ and  $u^{(1)} (\cdot, \cdot, z) = u^{(2)} (\cdot , \cdot ,z) \equiv 0$ for $z > \sfrac{1}{2}$. Then the bounded solutions of the advection equation are unique.
\end{lemma} 
\begin{proof}
We start by proving that integral curves of $u$ are unique. Let $(x,y,z) \in \T^3$ be arbitrary. We will show that solutions $\gamma \colon [0, \infty) \to \T^3$ of
 \begin{equation}\label{eq:IntegralCurvesUniqueness}
  \left\{
  \begin{array}{l}
   \dot{\gamma}(t) = u(\gamma(t)); \\
   \gamma(0) = (x,y,z);
  \end{array}
  \right.
\end{equation}
are unique for all $(x,y,z) \in \T^3$. 
It is enough to prove that such solutions are unique in the time interval $[0, \sfrac{1}{2}]$. Indeed, if this is known, iterating this fact allows us to conclude that we have uniqueness in the time interval $[0, \infty)$. We divide the proof in two distinct cases. \\
\textbf{Case 1: Assume that \pmb{$0 \leq z < \sfrac{1}{2}$}.}
Since $u \in W^{1, \infty}_{loc}(\T^2 \times [0, \sfrac{1}{2}); \R^3)$, solutions to \eqref{eq:IntegralCurvesUniqueness} are unique for $t \in [0, \sfrac{1}{2} - z)$. We prove that the limit of $\gamma(t)$ as $t \to \sfrac{1}{2} - z$ exists. Since $u \in L^{\infty}(\T^3)$, $\| \dot{\gamma} \|_{L^{\infty}((0, \sfrac{1}{2} - z))} \leq \| u \|_{L^{\infty}(\T^3)} < \infty$ and therefore $\lim_{t \to \sfrac{1}{2} - z} \gamma(t)$ exists. Denote this limit as $\gamma(\sfrac{1}{2} - z)$. 
Since $u(x,y,z) = (0,0,1)$ for all $z \in (\sfrac{1}{2},1)$, we must have $\gamma(t) = \gamma(\sfrac{1}{2} - z) + (0,0,t)$ for all $t \in (\sfrac{1}{2} - z, 1 - z)$. This proves uniqueness for $t \in [0, 1 - z) \supset [0, \sfrac{1}{2}]$ as wished. \\
\textbf{Case 2: Assume that \pmb{$\sfrac{1}{2} \leq z < 1$}.}
Since $u(x,y,z) = (0,0,1)$, it is clear that the solution is unique for $t \in [0,1-z)$ and $\gamma(t) = (x,y,z + t)$ for all $t \in [0, 1 - z]$. In the time interval $[1 - z, \sfrac{3}{2} - z]$, Case 1 above applies and yields the desired uniqueness for this second case.

Finally, the integral curves of $u$ being unique combined with the superposition principle (see \cite{A08}), implies that non-negative solutions of the advection equation are unique. Since the equation is linear, also bounded solutions are unique.
\end{proof}
 
 
 \subsection{Stability between advection  equations and advection-diffusion equations}
 We present a result which gives an upper bound on the $L^2$-distance between a solution to the advection-diffusion equation and the advection equation.
 \begin{lemma}\label{lemma:AdvectionDiffusionAndTransport}
 Let $d \in \N$, $0 < T < \infty$ and $\kappa > 0$. Let  $u: [0,T] \times \T^d \to \R^d$ be a bounded velocity field such that there exists an initial datum $\theta_{\initial} \in L^2 (\T^d)$ for which there is a unique solution  $\theta \in L^2 ((0,T); H^1 ( \T^d)  ) \cap C((0,T); L^2 (\T^d))$ of the advection equation.  Let also ${\theta}_{\kappa} $  be the unique solution in the class $ L^2 ((0,T); H^1 (\T^d)) \cap C((0,T); L^2 (\T^d))$ of the advection diffusion equation with velocity field $u$, diffusion parameter $\kappa $ and initial datum $\theta_{\initial}$. Namely, we have
 \[
  \left\{
  \begin{array}{l}
   \partial_t {\theta}_{\kappa} + u \cdot \nabla {\theta}_{\kappa} = \kappa \Delta {\theta}_{\kappa}; \\
   \partial_t {\theta} + u \cdot \nabla {\theta} = 0; \\
   {\theta}_{\kappa}(0, \cdot ) = {\theta} (0, \cdot ) = {\theta}_{\initial} (\cdot ).
  \end{array}
  \right.
 \]
 Then 
 \begin{equation*}
 \| {\theta}_{\kappa}(t, \cdot ) - {\theta} (t, \cdot) \|_{L^2(\T^d)}^2 \leq \kappa \int_0^t \| \nabla {\theta} (s, \cdot ) \|_{L^2(\T^d)}^2 \, ds \quad \text{for all} \quad t \in [0,T].
 \end{equation*}
 \end{lemma}
 \begin{proof}
The existence and uniqueness of solutions of the advection diffusion equation with $L^\infty$ velocity field and $L^2$ initial datum is classical (see for instance \cite{LSU68}). From the regularity of $\theta_\kappa$ and $\theta$ we notice that the energy equality holds
\begin{align*}
 \int_{\T^d} & | \theta_\kappa (t, x) - \theta (t, x) |^2 dx 
 \\
&  = - 2 \kappa \int_0^t \int_{\T^d} | \nabla (\theta_\kappa (s,x) - \theta (s,x) )|^2 dx ds  -2 \kappa  \int_0^t \int_{\T^d} \nabla \theta (s,x) \cdot \nabla (\theta_\kappa (s,x) - \theta (s,x))  dx ds 
\\
& \leq - 2 \kappa \int_0^t \int_{\T^d} | \nabla (\theta_\kappa (s,x) - \theta (s,x) )|^2 dx ds
\\
& \quad + \kappa  \int_0^t \int_{\T^d} | \nabla \theta (s,x)|^2 dx ds + \kappa \int_0^t \int_{\T^d} | \nabla (\theta_\kappa (s,x) - \theta (s,x) ) |^2 dx ds   
\\
& \leq \kappa  \int_0^t \int_{\T^d} | \nabla \theta (s,x)|^2 dx ds  \,.
\end{align*}
 \end{proof} 
 
 \begin{remark}
There are examples for which there are no solutions $\theta \in L^2 ((0,T); H^1 (\T^d))$ to the advection equation with a divergence free velocity field in $L^\infty ((0,T) ;W^{1,p} (\T^d))$ for any $p <\infty$ (see \cite[Theorem 1]{ACM-loss}). Therefore, the existence assumption is non trivial in the previous statement. 
 %On the other hand the advection diffusion equation is well posed with a divergence free velocity field $u \in L^2$ (see ).
 \end{remark}
 \subsection{Enhanced diffusion}
We present a result which gives a quick $L^2$-norm decaying of the solution of the heat equation  thanks to the high frequency of the initial datum.
 \begin{lemma}\label{lemma:EnhancedDiffusion}
 Let $v_{in} \in L^2(\T^d)$ with zero-average, let $\lambda \in \N$ and define $v_{\lambda, in}$ as $v_{\lambda , \initial}(x)  =  v_{\initial}(\lambda x)$ for all $x \in \T^2$. Then let $v_{\lambda} \in L^{\infty}((0, \infty); L^2(\T^d))$ be the solution to the heat equation
 \begin{align} \label{rescaled_heat_equation}
 \left\{
  \begin{array}{l}
   \partial_t v_{\lambda} = \kappa \Delta v_{\lambda}; \\
   v_{\lambda}(0,x) = v_{\lambda, \initial}(x). \\
  \end{array}
  \right.
 \end{align}
 Then the following holds
 \begin{equation*}
  \| v_{\lambda}(t, \cdot) \|_{L^2(\T^d)}^2 \leq \e^{-  \kappa \lambda^2 t} \| v_{\lambda, \initial} \|_{L^2(\T^d)}^2 \qquad \text{ for all } t \in [0, \infty) \,.
 \end{equation*}
  \end{lemma}
  \begin{proof}
  From the heat equation we have that the solution $v_{1} \in L^\infty( (0, \infty); L^2 (\T^d))$ of 
  \[
  \left\{
  \begin{array}{l}
   \partial_t v_{1} = \kappa \Delta v_{1}; \\
   v_{1}(0,x) = v_{1, \initial}(x). \\
  \end{array}
  \right.
 \]
  is such that 
  $$ \| v_{1} (t, \cdot) \|_{L^2}^2 \leq \e^{-  \kappa t} \| v_{1, \initial} (\cdot) \|_{L^2(\T^d)}^2 \qquad \text{ for all } t \in [0, \infty) \,.$$
  Observing that the unique solution in $L^\infty ((0,\infty); L^2(\T^d))$ of \eqref{rescaled_heat_equation} is $v_{\lambda} (t, x) = v_1 (\lambda^2 t, \lambda x)$, we conclude
  $$ \| v_{\lambda}(t, \cdot) \|_{L^2(\T^d)}^2 = \| v_{1}(\lambda^2 t,  \cdot) \|_{L^2(\T^d)}^2  \leq \e^{-  \kappa \lambda^2 t} \| v_{1, \initial} \|_{L^2(\T^d)}^2 = \e^{-  \kappa \lambda^2 t} \| v_{\lambda, \initial} \|_{L^2(\T^d)}^2  \qquad \text{ for all } t \in [0, \infty) \,. $$
  \end{proof}
 
 \subsection{Tail estimates for the advection-diffusion equation}
% In the proof of the theorem, we will work with solutions to the advection-diffusion equation with initial datum very locally supported along one spatial coordinate. Due to diffusion, the solution will be supported on the whole $\T^3$ for any positive time. However, since the diffusion is very small, sufficiently far from the support of the initial datum, the value of the solution will be tiny. Being able to quantify this smallness will be useful in the proof of Theorem~\ref{t_anomalous_autonomous}. The following result does this for solutions to the heat equation.
We present a quantified estimate on the tail of the solutions to the advection diffusion equation with constant coefficient in 1D starting with a compactly supported initial datum.
\begin{lemma}\label{lemma:TailEstimates1DAdvectionDiffusion}
Let $T>0$ and  ${\psi} \colon [0,T] \times \T \to \R$ be the unique bounded solution to
  \begin{align} \label{eqn:lemma_tail}
  \begin{cases}
  \partial_t {\psi} + \partial_x {\psi} = \kappa \partial_{xx} {\psi}; \\
   \psi(0,x) = \psi_{\initial}(x). \\
  \end{cases}
  \end{align}
 where $\psi_{in} \in L^\infty ( \T)$. Let us assume that  $\supp \psi_{in} \subset (a,b)$.  Then, defining 
 $A_c(t) =  ( a-c +t, b+c+ t)^c \subset \T   $,
   we have 
 \begin{equation}\label{eq:TailEstimate1DAdvcetionDiffusion}
 \| {\psi}(t, \cdot) \|_{L^{\infty}(A_c(t))} \leq \| \psi_{in} \|_{L^{\infty}(\T)} \e^{- \frac{c^2}{8 \kappa t} }.
 \end{equation}
\end{lemma}
\begin{proof}
 Let $\tilde \psi \colon [0, T] \times \T \to \R$ be the unique bounded solution to
  \[
  \left\{
  \begin{array}{l}
   \partial_t \tilde \psi = \kappa \partial_{xx} \tilde  \psi; \\
   \tilde \psi|_{t = 0} = \psi_{\initial}. \\
  \end{array}
  \right.
 \]
  We slightly abuse the notation extending $\psi_{\initial}$ to $\R$, namely $\psi_{\initial} (x) = \psi_{\initial}(x - \lfloor x \rfloor) $ for all $x \in \R$. Then the solution to the heat equation on $\R$  is given by
 \[
 \tilde  \psi(t,x) = \int_{- \infty}^{+ \infty} \frac{1}{\sqrt{4 \pi \kappa t}} \e^{- \frac{|x - \xi|^2}{4 \kappa t}} \psi_{\initial}(\xi) \, d \xi.
 \]
 We observe that this solution is $1$-periodic and therefore is also the solution on $\T$. Furthermore,
 for any $x \in (a - c, b + c)^c$
 \begin{align*}
 \tilde \psi(t,x) &= \int_{\{ \xi :  |x - \xi| > c \}} \frac{1}{\sqrt{4 \pi \kappa t}} \e^{- \frac{|x - \xi|^2}{4 \kappa t}} \psi_{in}(\xi) \, d \xi \\
 &\leq \| \psi_{\initial} \|_{L^{\infty}(\T)} \int_{\{ \xi :  |\xi| > c \}} \frac{1}{\sqrt{4 \pi \kappa t}} \e^{- \frac{|\xi|^2}{4 \kappa t}} \, d \xi \\
 &\leq \| \psi_{\initial} \|_{L^{\infty}(\T)} \e^{- \frac{c^2}{8 \kappa t} }.\\
 \end{align*}
 
 Let us now define ${\psi}(t,x) = \tilde \psi(t, x - t)$ for all $x \in \T$ and $t \in [0,T]$,  then ${\psi}$ is the unique bounded solution of \eqref{eqn:lemma_tail} and 
 by the previous computation we have
  $$\| {\psi}(t, \cdot) \|_{L^{\infty}(A_c(t))} = \| \tilde{\psi}(t, \cdot ) \|_{L^{\infty}((a-c, b+c)^c)} \leq \| \psi_{in} \|_{L^{\infty}(\T)} \e^{- \frac{c^2}{8 \kappa t} } \,.$$
 \end{proof}
 Finally, we deduce the following result which applies to the special advection-diffusion equation that we are considering in this paper.
 \begin{corollary}\label{lemma:TailEstimates3DAdvectionDiffusion}
Let $T>0$ and $u : [0,T] \times \T^3 \to \R^3$ be a  bounded velocity field such that $u^{(3)} \equiv 1$. 
 Let $\theta_\kappa \colon [0,T] \times \T^3 \to \R$ be the unique bounded solution to the advection-diffusion equation with velocity field  $u$  and diffusion parameter $\kappa > 0$ and initial datum $\theta_{\initial} \in L^\infty(\T^3)$.
 If we have that $\supp(\theta_{\initial}) \subset \T^2 \times (a,b) \subset \T^3$, then
 \begin{equation*}
 \| \theta_\kappa (t, \cdot ) \|_{L^{\infty}(\T^2 \times A_c(t) )} \leq 2 \| \theta_{\initial} \|_{L^{\infty}(\T^3)} \e^{- \frac{c^2}{8 \kappa t}}, \qquad \text{for all } t \in [0,T]\,,
 \end{equation*}
 where $A_c(t) = (a + t - c, b + t + c)^c \subset \T$.
 \end{corollary}
 \begin{proof}
 Let $M \colon \T \to \R$ be defined as $M(z) = \| \theta_{\initial} \|_{L^{\infty}(\T)} \mathbbm{1}_{(a,b)} (z)$. Let ${\psi} \colon [0,T] \times \T \to \R$ be the unique bounded solution of
  \begin{equation}
  \left\{
  \begin{array}{l}
   \partial_t {\psi} + \partial_z {\psi} = \kappa \partial_{zz} {\psi}; \\
   \psi(0,z) = M(z). \\
  \end{array}
  \right.
  \end{equation}
  By defining $\phi \colon [0,T] \times \T^3 \to \R$ as $\phi(t,x,y,z) = {\psi}(t,z)$ we have a solution to
  \begin{equation}
  \left\{
  \begin{array}{l}
   \partial_t \phi + u \cdot \nabla \phi = \kappa \Delta \phi; \\
   \phi(0,x,y,z) = M(z). \\
  \end{array}
  \right.
  \end{equation}
  By the comparison principle for the advection-diffusion equation (see for instance \cite[Proposition 4.4]{F08}) $|\theta(t,x,y,z)| \leq \phi(t,x,y,z)$ for all $t \in [0,T]$, $(x,y,z) \in \T^3$. Using Lemma~\ref{lemma:TailEstimates1DAdvectionDiffusion} we get
  \begin{align*}
   \| \theta(t) \|_{L^{\infty}(\T^2 \times A_c(t))} & \leq \| \phi(t) \|_{L^{\infty}(\T^2 \times A_c (t) )} = \| {\psi}(t) \|_{L^{\infty}(A_c(t))} \leq \| \theta_{\initial} \|_{L^{\infty}(\T)} \e^{- \frac{c^2}{8 \kappa t}}.
  \end{align*}
 \end{proof}
 

% \subsection{About weak $L^1$-convergence of non-increasing functions}
% The following result improves weakly* $L^\infty$-convergence of functions to $L^\infty_{\text{loc}}$ convergence under suitable assumptions.
% % that the functions are non-increasing and non-negative.
% \begin{lemma}\label{lemma:WeakStarConvergenceOnDecreasingFunctions}
% Let $\{ f_{\kappa} \}_{\kappa > 0}$ be a collection of non-negative bounded functions $f_\kappa : [0,1] \to \R^+$  and let  $f \colon [0,1] \to \R^+$ be a continuous function such that
% \begin{enumerate}
% \item for all $\kappa > 0$, $f_{\kappa}$ and $f$ are non-increasing functions i.e.
% \[
%  s > t \Rightarrow f_{\kappa}(s) \leq f_{\kappa}(t) \quad \forall \kappa > 0 \text{ and } f(s) \leq f(t) \,,
% \]
% \item $f_{\kappa} \weak f$ in the $L^{\infty} ((0,1))$-weakly* sense as $\kappa \to 0$.
% \end{enumerate}
% Then $f_{\kappa} \to f$ as $\kappa \to 0$ in $L^{\infty}_{\text{loc}}((0,1))$.
%\end{lemma}
%\begin{proof}
%Without loss of generality we assume that all the functions are bounded by  $ 1$.
%Let $N \geq 1$ be a large integer and define $x_j = \sfrac{j}{N}$ for all $j = 0, \ldots, N$.
%For any $\eps_N >0$ there exists $\delta_N > 0$ such that for all $\kappa < \delta_N$ and all $j = 0, \ldots, N-1$.
%\[
% \left| \int_0^1 (f_{\kappa} - f)(x) \mathbbm{1}_{[x_j, x_{j+1}]}(x) \, dx \right| < \eps_N.
%\]
%Thus, due to the previous inequality and the fact that all the functions are non-increasing
%\begin{align} \label{weak_1}
% \frac{1}{N}f(x_{j+1}) - \eps_N \leq \int_0^1 f(x) \mathbbm{1}_{[x_j, x_{j+1}]}(x) \, dx - \eps_N \leq \int_0^1 f_{\kappa}(x) \mathbbm{1}_{[x_j, x_{j+1}]}(x) \, dx \leq \frac{1}{N}f_{\kappa}(x_j) 
%\end{align} 
%and 
%\begin{align} \label{weak_2}
%\frac{1}{N}f_{\kappa}(x_{j+1}) \leq \int_0^1 f_{\kappa}(x) \mathbbm{1}_{[x_j, x_{j+1}]}(x) \, dx \leq \int_0^1 f(x) \mathbbm{1}_{[x_j, x_{j+1}]}(x) \, dx + \eps_N \leq \frac{1}{N}f (x_j) + \eps_N\,
%\end{align}
%for all $j=0,..,N-1$.
%%Then, by using once again that the functions are non-increasing and choosing $\eps_N = \sfrac{1}{N^2}$, we find
%%\[ f(x_{j+2}) - \frac{1}{N} \leq f_{\kappa}(x_{j+1}) \leq f_{\kappa}(x) \leq f_{\kappa}(x_{j}) \leq f(x_{j-1}) + \frac{1}{N} \, \forall x \in [x_j, x_{j+1}), \, j = 1, \ldots, N-2. \]
%%We define $x_{N+1} = 1+ \sfrac{1}{N} $, $x_{-1} =- \sfrac{1}{N}$ and $f(x_{N+1}) = f(1)$ and $f(x_{-1}) = f(0)$.
% We define $\underline{g}_N, \overline{g}_N \colon [0,1] \to \R$ as
%$$
% \underline{g}_N(x) = \sum_{j=0}^{N-1} (f(x_{j+2}) - \sfrac{1}{N} ) \mathbbm{1}_{[x_j, x_{j+1}]} (x) $$
% where we define $x_{N+1} = 1 + \sfrac{1}{N}$ and  $f(x_{N+1}) =0$ and
%$$
% \overline{g}_N(x) = \sum_{j=0}^{N-1} (f(x_{j-1}) + \sfrac{1}{N} ) \mathbbm{1}_{[x_j, x_{j+1}]} (x) \,,
% $$
% where we define $x_{-1} = - \sfrac{1}{N}$ and $f(x_{-1}) =1$.
% Since $f$ is uniformly continuous on $[0,1]$ it follows from the definition of $\underline{g}_N$ and $\overline{g}_N$ that $\underline{g}_N, \overline{g}_N \to f$ in $L^\infty_{\text{loc}} ((0,1))$.
% Using that the functions are non-increasing, \eqref{weak_1}, \eqref{weak_2} with $\eps_N = \sfrac{1}{N^2}$ and $0 \leq f, f_\kappa \leq 1$, we find
%\[ f(x_{j+2}) - \frac{1}{N} \leq f_{\kappa}(x_{j+1}) \leq f_{\kappa}(x) \leq f_{\kappa}(x_{j}) \leq f(x_{j-1}) + \frac{1}{N} \, \forall x \in [x_j, x_{j+1}], \, j = 0, \ldots, N-1 \,, \]
%from which it follows that
%%Furthermore, choosing $\varepsilon_N = \sfrac{1}{N^2}$, thanks to \eqref{weak_1} and \eqref{weak_2} we have
%\begin{equation}
%\underline{g}_N  \leq f_{\kappa} \leq \overline{g}_N \text{ for all } \kappa < \delta_N \,.
%\end{equation}
%%Since $f$ is uniformly continuous on $[0,1]$ it follows from the definition of $\underline{g}_N$ and $\overline{g}_N$ that $\underline{g}_N, \overline{g}_N \to f$ uniformly in any compact subset $K \subset (0,1)$ as $N \to \infty$.
%%In other words, $\underline{g}_N, \overline{g}_N \to f$ in $L_{loc}^{\infty}((0,1))$ as $n \to \infty$.
%In virtue of this we deduce that for any $K \subset [0,1]$ we have
%$$\sup_{x \in K} | f_\kappa (x) - f(x)| \leq \sup_{x \in K} | \overline{g}_N (x) - f(x)| + \sup_{x \in K} | \underline{g}_N (x) - f(x)|$$
%which finishes the proof.
%\end{proof}

 
 \section{Proof of Theorem~\ref{t_anomalous_autonomous}} \label{sec:proof-thB}
 
 
 The objective of this section is to prove Theorem~\ref{t_anomalous_autonomous}. In Subsection~\ref{subsec:PlanOfProofA}, we describe the plan of the proof. In the subsequent subsection we describe a partition of unity that is central in our proof. In Subsection~\ref{subsec:ProofOfTheoremAWithoutProofOfLemmas}, we give a proof of Theorem~\ref{t_anomalous_autonomous}. In this proof, we mention several lemmas which we choose not to prove immediately. Instead the proof of these lemmas are postponed to the last subsection, Subsection~\ref{subsec:ProofOfTheLemmasOfProofA}.
 Throughout this section, let $\alpha \in [0,1)$  as in Theorem~\ref{t_anomalous_autonomous} and $u : \T^3 \to \R^3$ be the autonomous velocity field constructed as in Subsection~\ref{subsec:ConstructionVectorField}. Let us fix the smooth initial datum $\theta_{\initial}$ as in \eqref{eq:initialdatum_theta}. 


 
 \subsection{Plan of the proof}\label{subsec:PlanOfProofA}
We begin by fixing some arbitrary $q \in m \N$. The main objective is to prove that 
\begin{equation}\label{eq:LTwoNormAlmostZero}
\| {\theta}_{\kappa_q}(1, \cdot ) \|_{L^{2}(\T^3)}^2 \lesssim a_0^{\eps \delta}.
\end{equation}
By selecting $a_0$ sufficiently small combined with the energy balance
$$\int_{\T^3} | \theta (t, x)|^2 \, dx + 2 \kappa_q \int_0^t \int_{\T^3} | \nabla \theta_{\kappa_q} (t, x) |^2 \, dx \, dt =  \int_{\T^3} | \theta_{\initial} (x)|^2 \, dx $$
 we obtain that 
\begin{equation} \label{eq:total-diss}
 \kappa_q \int_0^1 \| \nabla \theta_{\kappa_q}(t) \|_{L^2(\T^3)}^2 \, dt > \frac{1}{4},
\end{equation}
for any $q \in m \N$,
which proves \eqref{diss_main_OC}.  All the other properties of Theorem \ref{t_anomalous_autonomous} (and similarly for Theorem \ref{t_Onsager}) will follow from a stability estimate of the type $\| \theta_{\kappa_q} - \theta_0 \|_{L^\infty_t L^2_x}^2 \leq \beta  $ (where $\beta >0 $ is a fixed constant defined in Theorem \ref{t_Onsager}) and proving that $\kappa_q |\nabla \theta_{\kappa_q}|^2$ weakly* converges in the sense of measures to a measure $\mu \in \mathcal{M}((0,1) \times \T^4)$ such that $\mu_T  = \pi_{\#} \mu $ has a non trivial absolutely continuous part. The last property is given by trying to characterize as much as possible the sequence $\kappa_q |\nabla \theta_{\kappa_q}|^2$. 

More precisely, the proof is divided into 6 steps.
In the first step, we decompose the initial datum  on the z-axis as ${\theta}_{\initial} = \sum_{j=1}^{N_q} \theta_{\initial, j}$ such that $\supp (\theta_{\initial , j} ) \cap \supp (\theta_{\initial , k})  = \emptyset $ for any $j, k \in \{ 1,..., N_q \}$ with $|j-k| > 1$.
\begin{figure}[htp]
  \includegraphics[clip,width=11cm]{ThetaKappaPicture11.pdf}%
  \caption{The figure shows the profile $\theta_{\initial}$ averaged in $x,y$ depending only on $z$, more precisely on the vertical axis we represent the value $\langle | \theta_{\initial} | \rangle_{x,y} (z) = \int_{\T^2 } |\theta_{\initial}(x,y,z)| dx dy$ and on the horizontal axis  we have the $z$-variable.
}
\end{figure}

In Step 2, we show that the solution to the advection-diffusion equation 
$\theta_{\kappa_q, j}$ with initial datum $\theta_{\initial , j}$ and velocity field $u$
 is close to the solution to the advection equation $\theta_{0,j}$ with initial datum $\theta_{\initial , j}$ and velocity field $u$ up to a certain time. 
 
 \begin{figure}[htp]
\includegraphics[clip,width=11cm]{ThetaKappaPicture21.pdf}
\caption{We represent the functions 
$\langle |\theta_{0,j} | \rangle_{x,y}  $ and $\langle |\theta_{\kappa_q ,j}\rangle_{x,y}   $ at a precise time ($t= t_{q,j}$ that will be defined in \eqref{d:time_t_j}) depending only on $z$ and representing the stability between the two functions at these times $t_{q,j}$, more precisely $\langle |\theta_{0,j} | \rangle_{x,y}  (z) = \int_{\T^2} | \theta_{0,j} (t_{q,i} , x,y,z)|  dx dy$ and similarly $\langle |\theta_{\kappa_q,j} | \rangle_{x,y}  (z) = \int_{\T^2} | \theta_{\kappa ,j} (t_{q,i} , x,y,z)|  dx dy$.
}
\end{figure}

 In Step 3, we prove that  the $L^2$ norm of $\theta_{\kappa_q, j}$ decays rapidly in a certain time interval. 
 \begin{figure}[htp]
\includegraphics[clip,width=11cm]{ThetaKappaPicture31.pdf}
\caption{The figure represents the phenomena of ``anomalous dissipation'' (loss of most of the $L^2$ norm) of the function $\theta_{\kappa_q, i}$. In the figure we draw the function $\langle |\theta_{\kappa_q, i}| \rangle_{x,y} (z)= \int_{\T^2} | \theta_{0,i} (t_{q,i} + \Tilde{t}_q , x,y,z)|  dx dy $  depending only on $z$, the horizontal axis, at a certain time ($t= t_{q,i} + \tilde{t}_q$ we will define later in \eqref{d:time_t_j}).}
\end{figure}

In Step 4, we use all the informations gathered until now, to prove that the solution to the advection-diffusion equation starting from ${\theta}_{\initial}$ decays significantly from time $t=0$ to time $t=1$, as in \eqref{eq:LTwoNormAlmostZero}. In Step 5, we prove that $\theta_\kappa$ weak{$^{\ast}$} converges to $\theta_0$ in $L^{\infty}((0,1) \times \T^3)$ as $\kappa \to 0$ and  $e(t) = \int_{\T^3} | \theta_0 (t, x) |^2 dx$ is a smooth, non-increasing function such that $e(1) < e(0)$. 
 \begin{figure}[htp]
\includegraphics[clip,width=11cm]{Energy.pdf}
\caption{The smooth energy profile of the limiting solution: $e(t) = \int_{\T^3} |\theta_{0} (t, x,y,z)|^2 dx dy dz$.}
\end{figure}

 Finally, in the sixth and final step, we prove that $\mu_T$ has a non-trivial absolutely continuous part.
 
\subsection{Partition of unity in the $z$ variable and corresponding time sequence}\label{subsec:PartitionOfUnityInZ}
In this section we introduce a partition of unity (used later for the $z$ variable) and a corresponding time sequence that will be useful to  describe locally in $z$ the solution $\theta_\kappa$.
 
% In this section we define a partition of unity and the corresponding time  sequence that will be useful in the proof of Theorem~\ref{t_anomalous_autonomous}.
There exists a universal constant $C > 0$ such that for every $q \in \N$ there are a sequences of smooth functions $\{ \chi_j \}_{j=1}^{N_q} \subset C^\infty (\T )$ and a sequence of  points $\{ z_j\}_{j=1}^{N_q} \subset \T$ with the following properties:
 \begin{enumerate}
 \myitem{D1} $0 \leq \chi_j \leq 1$ for all $j = 1,2, \ldots, N_q$;
 \myitem{D2} \label{support:chi_j} $\supp(\chi_j) \subset (z_j, z_j + \sfrac{a_q^{\gamma}}{8})$ for all $j = 1, 2, \ldots, N_q$;
 \myitem{D3} \label{chi:estimate-C-1} $\| \chi_j \|_{C^1(\T)} \leq C a_0^{-1} a_q^{- \gamma}$ for all $j = 1, \ldots, N_q$;
 \myitem{D4} $\sum_{i = 1}^{N_q} \chi_j \equiv 1$ on $\T$;
 \myitem{D5} \label{item:BoundOnTheNumberOfCutoffs} $N_q \leq C a_{q}^{- \gamma}$;
 \myitem{D6} we have
 \begin{equation}\label{eq:AboutAdjacentSupports}
  |\supp(\chi_i) \cap \supp(\chi_j)| \leq a_0 a_q^{\gamma}
 \end{equation} 
 and
 \begin{equation}\label{eq:DisjointnessOfTheSupportsOfCutoffs}
  \inf_{|i-j| >1} \dist \left ( \supp(\chi_{j}) \,, \supp(\chi_i) \right )  \geq \frac{ a_q^{\gamma}}{16} \,.
 \end{equation} 
 \end{enumerate}
 
For future reference, we also define 
\begin{equation} \label{d:time_t_j}
t_{q,j}= \frac{1}{2} - T_q - z_j + \frac{a_q^{\gamma}}{8} \quad \text{and} \quad \tilde{t}_q = 4a_0^{- \eps \delta} a_q^{\frac{\gamma}{1 + \delta} - 10 \eps}.
\end{equation}
Heuristically, $t_{q,j}$ represents the time up to which $\theta_{\kappa_q, j}$ remains close to $\theta_{0, j}$ solution to the advection equation starting from the same initial datum. 
After $t_{q,j}$, $\| \theta_{\kappa_q, j}(t, \cdot) \|_{L^2(\T^3)}$ starts decaying rapidly. In fact, heuristically $\tilde{t}_q$ represents the time needed from $t_{q,j}$ for $\| \theta_{\kappa_q, j}(t, \cdot) \|_{L^2(\T^3)}$ to become small.
\subsection{Proof of Theorem~\ref{t_anomalous_autonomous}}\label{subsec:ProofOfTheoremAWithoutProofOfLemmas}
We point out that the lemmas appearing in this proof are proved in Subsection~\ref{subsec:ProofOfTheLemmasOfProofA}.
Throughout this entire proof , we assume wothout loss of generality that $q$ is sufficiently large in order to have $\tilde{t}_q \leq \overline{t}_q$ thanks to \eqref{item:EpsDeltaThree}.

\textbf{Step 1: Decomposition.}

%{\color{red} - I modified $i=1,..,N_q$ with $j=1,..,N_q$.
%
%- $\chi_i^q $ is now $\chi_j$
%
%- $\overline{\theta}^{q,i}$ is now ${\theta}_{0,j}$
%
%- the velocity field $w$ is now $u$
%
%- $\varphi (t,x)$ is now $\varphi_{\initial}(z-t)$
%
%- $t_{q,i}$ is now $t_{q, j}$ (to be changed above)
%}

The purpose of this step is to decompose the solution ${\theta}_{\kappa_q}$ in a suitable way. Using the partition of unity introduced above in the $z$-variable we cut the support of the initial datum ${\theta}_{in}$ into smaller parts. For each $j = 1, 2,  \ldots, N_q$, we define ${\theta}_{\initial, j} \colon \T^3 \to \R$ as ${\theta}_{\initial , j}(x,y,z)  =  {\theta}_{\initial}(x,y,z) \chi_j(z)$.
Then we define ${\theta}_{\kappa_q , j}$ as the solution to the advection-diffusion equation
\begin{equation}
 \left\{
 \begin{array}{ll}
 \partial_t {\theta}_{\kappa_q, j} + u \cdot \nabla {\theta}_{\kappa_q, j} = \kappa_q \Delta {\theta}_{\kappa_q , j}; \\
 {\theta}_{\kappa_q , j} (0,x,y,z) = {\theta}_{\initial , j}(x,y,z). \\
 \end{array}
 \right.
\end{equation}
Thanks to the linearity of the equation we have ${\theta}_{\kappa_q} = \sum_{j = 1}^{N_q} {\theta}_{\kappa_q , j}$, because $\theta_{\initial } = \sum_{j = 1}^{N_q} {\theta}_{\initial , j}$. Accordingly, we also specify that ${\theta}_{0,j } \colon [0,1/2] \times \T^3 \to \R$ defined as ${\theta}_{0,j}(t,x,y,z) = \theta_{\stat }(x,y,z) \varphi_{\initial} (z-t) \chi_j(z - t)$  is the unique solution  to the advection equation (see Lemma \ref{lemma:uniqueness})
\begin{equation} \label{d:theta_0j}
 \left\{
 \begin{array}{ll}
 \partial_t {\theta}_{0,j} + u \cdot \nabla {\theta}_{0,j} = 0; \\
 {\theta}_{0,j} (0,x,y,z) = {\theta}_{\initial, j}(x,y,z). \\
 \end{array}
 \right.
\end{equation}
\\
\textbf{Step 2: $L^2$-stability between ${\theta}_{\kappa_q, j}$ and ${\theta}_{0,j}$ until time $t_{q,j} < 1/2$.}
The goal of this step is to prove that 
\begin{equation}\label{stability:theta_kq-theta}
\| {\theta}_{\kappa_q, j}(t , \cdot ) - {\theta}_{0,j}(t, \cdot ) \|_{L^{2}(\T^3)}^2 \lesssim a_q^{\gamma + \eps} \qquad \text{for all }  t \in [0,t_{q,j}] \quad \text{for all } j =1, 2, \ldots , N_q \,.
\end{equation}
By Lemma~\ref{lemma:AdvectionDiffusionAndTransport}, we get
\begin{equation}\label{eq:StabilityEstimateApplicationInThmA}
 \| {\theta}_{\kappa_q, j}(t, \cdot ) - {\theta}_{0,j}(t, \cdot ) \|_{L^{2}(\T^3)}^2 \leq \kappa_q \int_0^t \| \nabla {\theta}_{0,j} (s, \cdot ) \|_{L^2(\T^3)}^2 \, ds \,,
\end{equation}
for all $t \in [0, t_{q,j}]$.
We will prove the following estimate:
\begin{lemma}\label{lemma:EstimateOnTheGradientOfTheSolutionToTheTransportEquation}
 There exists a universal constant $C > 0$  such that for all $ q \in m \N$ and $j = 1, \ldots, N_q$ we have
 \begin{equation}\label{eq:EstimateOnTheGradientOfTheSolutionToTheTransportEquation}
  \kappa_q \int_0^{t_{q,j}} \| \nabla {\theta}_{0,j} (s, \cdot ) \|_{L^2(\T^3)}^2 \, ds \leq C a_q^{\gamma + \eps}.
 \end{equation}
\end{lemma}
Hence, we conclude that \eqref{stability:theta_kq-theta} holds.
\begin{comment}
In order to estimate the right-hand side, we estimate the following three quantities:
\begin{enumerate}
 \item $\displaystyle \int_{0}^{t_{q,j}} \int_{\T^3} |\partial_x {\theta}_{0,j}(t,x,y,z)|^2 \, dx dy dz \, dt$ \label{item:QuantityToEstimate1} \\
 \item $\displaystyle \int_{0}^{t_{q,j}} \int_{\T^3} |\partial_y {\theta}_{0,j}(t,x,y,z)|^2 \, dx dy dz \, dt$ \label{item:QuantityToEstimate2}\\
 \item $\displaystyle \int_{0}^{t_{q,j}} \int_{\T^3} |\partial_z {\theta}_{0,j}(t,x,y,z)|^2 \, dx dy dz \, dt$ 
\label{item:QuantityToEstimate3}\\
\end{enumerate}
\fbox{Estimate of \eqref{item:QuantityToEstimate1}:}
\begin{align*}
 &\int_{0}^{t_{q,j}} \int_{\T^3} |\partial_x {\theta}_{0,j}(t,x,y,z)|^2 \, dx dy dz \, dt = \int_{0}^{t_{q,j}} \int_{\T} \underbrace{|\varphi_{\initial}(z-t)|^2 |\chi_j(z-t)|^2}_{= 0 \text{ if } z \in (z_j + t, z_j + t + \sfrac{a_q^{\gamma}}{8})^c} \int_{\T^2} |\partial_x \theta_{\stat}(x,y,z)|^2 \, dx dy \, dz \, dt \\ 
 &\leq \| \varphi_{\initial} \|_{L^{\infty}(\T)}^2 \int_{0}^{t_{q,j}} \int_{z_j + t}^{z_j + t + \sfrac{a_q^\gamma}{8}} \int_{\T^2} |\partial_x \theta_{\stat}(x,y,z)|^2 \, dx dy \, dz \, dt \\
  &= \| \varphi_{\initial} \|_{L^{\infty}(\T)}^2 \int_{z_j}^{\sfrac{1}{2} - T_q + \sfrac{a_q^{\gamma}}{4}} \int_{\T^2} \int_{z - z_j - \sfrac{a_q^{\gamma}}{8}}^{z - z_j}  |\partial_x \theta_{\stat}(x,y,z)|^2  \, dt \, dx dy  \, dz \\
  &\lesssim a_q^{\gamma} \int_{0}^{\sfrac{1}{2} - T_q + \sfrac{a_q^{\gamma}}{4}} \int_{\T^2} |\partial_x \theta_{\stat}(x,y,z)|^2 \, dx dy \, dz 
  \end{align*}
  Using \eqref{eq:vartheta_stationary}  we have 
  \begin{align*}
  &= a_q^{\gamma} \Bigg[ \int_{0}^{\sfrac{1}{2} - T_0} \underbrace{\int_{\T^2} |\partial_x \theta_{\stat}(x,y,z)|^2 \, dx dy}_{\lesssim a_0^{- 2(1 + 3 \eps (1 + \delta))}} \, dz + \sum_{j = 0}^{q-1} \int_{\sfrac{1}{2} - T_j}^{\sfrac{1}{2} - T_{j+1}} \underbrace{\int_{\T^2} |\partial_x \theta_{\stat}(x,y,z)|^2 \, dx dy}_{\lesssim a_{j+1}^{-2(1 + 3 \eps (1 + \delta))}} \, dz \\
  &\qquad + \int_{\sfrac{1}{2} - T_q}^{\sfrac{1}{2} - T_{q} + a_q^{\gamma }} \underbrace{\int_{\T^2} |\partial_x \theta_{\stat}(x,y,z)|^2 \, dx dy}_{\lesssim a_q^{-2(1 + 3 \eps(1 + \delta))}} \, dz \Bigg] \\
  &\lesssim a_q^{\gamma} \Bigg[ a_0^{- 2(1 + 3 \eps (1 + \delta))} + \sum_{j = 0, j \in m \N }^{q-m}  a_j^{\frac{\gamma}{1 + \delta} - 10 \eps} a_{j+1}^{-2(1 + 3 \eps (1 + \delta))} + \sum_{j = 0, j \notin m \N }^{q-1}  a_j^\gamma a_{j+1}^{-2(1 + 3 \eps (1 + \delta))}  + a_q^{\gamma} a_q^{-2(1 + 3 \eps (1 + \delta))} \Bigg]
  \\
  & \lesssim a_q^{\gamma} \left[ a_0^{- 2(1 + 3 \eps (1 + \delta))} + q a_{q-m}^{\frac{\gamma}{1 + \delta} - 10 \eps - 2(1 + 3 \eps (1 + \delta))(1 + \delta)} +  q a_{q-1}^{\gamma - 2(1 + 3 \eps (1 + \delta))(1 + \delta)} +  a_q^{\gamma -2(1 + 3 \eps (1 + \delta))} \right]\,
  \end{align*}
  where we have used that $q \in m \N$ and $a_{j+1} = a_j^{1+ \delta}$ for any $j$.
  Using  also \eqref{eq:InequalityForTheConstantM}  we get
  $$ a_{q-m +1}^{\frac{\gamma (1- \delta)}{(1+\delta)^2} - \frac{10 \eps}{1 + \delta} - 2 (1+ 3 \varepsilon (1+ \delta))} \leq a_q^{\frac{\gamma}{1+\delta} - 2 (1+ 3 \varepsilon (1+ \delta))},$$
concluding that 
$$ \int_{0}^{t_{q,j}} \int_{\T^3} |\partial_x {\theta}_{0,j}(t,x,y,z)|^2 \, dx dy dz \, dt \lesssim q a_{q}^{\gamma + \frac{\gamma}{1 + \delta} - 2(1 + 3 \eps (1 + \delta))}\,.$$
\fbox{Estimate of \eqref{item:QuantityToEstimate2}:} In the same manner, 
\[
 \int_{0}^{t_{q,j}} \int_{\T^3} |\partial_x {\theta}_{0,j}|^2 \, dx dy dz \, dt \lesssim q a_{q}^{\gamma + \frac{\gamma}{1 + \delta} - 2(1 + 3 \eps (1 + \delta))}.
\]
\fbox{Estimate of \eqref{item:QuantityToEstimate3}:} We note that 
\[
 \partial_z {\theta}_{0,j} = \underbrace{\partial_z (\varphi_{\initial}(z-t) \chi_j(z - t)) \theta_{\stat}(x,y,z)}_{= g_1} + \underbrace{\varphi_{\initial}(z-t) \chi_j(z - t) \partial_z \theta_{\stat}(x,y,z)}_{= g_2}.
\]
Using again \eqref{eq:vartheta_stationary} and observing that $\gamma = \sfrac{\delta}{8} <1$
we can use the same arguments which lead to the estimate of \eqref{item:QuantityToEstimate1} and we find
\[
 \int_0^{t_{q,j}} \int_{\T^3} |g_2|^2 \, dx dy dz \, dt \lesssim q a_{q}^{\gamma + \frac{\gamma}{1 + \delta} - 2(1 + 3 \eps (1 + \delta))}.
\]
Using $\| \chi_j \|_{C^1(\T)} \lesssim a_q^{- \gamma}$, we get $\| \varphi_{\initial}( \cdot -t) \chi_j(\cdot - t) \|_{C^{1}(\T)} \lesssim a_q^{- \gamma}$. Thus
\begin{align*}
 \int_0^{t_{q,j}} \int_{\T^3} |g_1|^2 \, dx dy dz \, dt &= \int_0^{t_{q,j}} \int_{\T} \underbrace{|\partial_z (\varphi_{\initial}(z-t) \chi_j(z - t))|^2}_{= 0 \text{ if } z \in (z_j + t, z_j + t + \sfrac{a_q^{\gamma}}{8})^c} \int_{\T^2} |\theta_{\stat}(x,y,z)|^2 \, dx dy \, dz \, dt \\
 &\leq \int_0^{t_{q,j}} \int_{z_j + t}^{ z_j + t + \sfrac{a_q^{\gamma}}{8} } \underbrace{|\partial_z (\varphi_{\initial}(z-t) \chi_j(z - t))|^2}_{\lesssim a_q^{- 2 \gamma}} \, dz \, dt \lesssim a_q^{- \gamma}.
\end{align*}
Thus, using that $\gamma <1/2$, we conclude that 
\[
 \int_0^{t_{q,j}} \int_{\T^3} |\partial_z {\theta}_{0,j} |^2 \, dx dy dz \, dt \lesssim  q a_{q}^{\gamma + \frac{\gamma}{1 + \delta} - 2(1 + 3 \eps (1 + \delta))} \,.
\]
Finally, from the estimates of \eqref{item:QuantityToEstimate1}, \eqref{item:QuantityToEstimate2} and \eqref{item:QuantityToEstimate3} and \eqref{d:k_q}, we find that
\[
\| {\theta}_{\kappa_q, j}(t) - {\theta}_{0,j}(t) \|_{L^2(\T^3)}^2 \lesssim  q a_q^{2 - \frac{\gamma}{1 + \delta} + 10 \eps + \frac{\gamma}{1 + \delta} + \gamma - 2(1 + 3 \eps (1 + \delta))}  \lesssim a_q^{\gamma + \eps} \quad \text{for all} \quad  t \in [0, t_{q,j}],
\]
where we also used that $q a_q^{\varepsilon} \leq 1$ for any $q \in \N$. 
\\
\end{comment}
\\
\textbf{Step 3: Decay of $\| {\theta}_{\kappa_q, j}(t, \cdot) \|_{L^2 (\T^3)}$ in $[t_{q,j}, t_{q,j} + \sfrac{\tilde{t}_q}{2}]$.}
The goal of this step is to prove that
\begin{equation}\label{eq:AutonomousNormalGoalOfStep3}
\| {\theta}_{\kappa_q,j}(t_{q, j} + \sfrac{\tilde{t}_q}{2}, \cdot ) \|_{L^2(\T^3)}^2 \lesssim a_0^{\eps \delta} a_q^{\gamma} \,.
\end{equation}
%such that $q \geq \overline{q}$, where $\overline{q}$ is such that $(a_{\overline q})^{\eps} C \leq 1$ for a universal constant $C >0$ (to be determined) independent on $q$. 
This step is divided into 4 substeps. In Substep 3.1. we prove that ${\theta}_{\kappa_q, j} (t_{q,j} , \cdot)$ is $a_q$-periodic in the $x,y$-variables up to small $L^2$ errors. Thanks to this high periodicity structure, in the remainder of this step, via an approximation of $\theta_{\kappa_q, j}$, we prove that $\| {\theta}_{\kappa_q , j}(t, \cdot ) \|_{L^2(\T^3)}$ decays in the time interval $[t_{q,j}, t_{q,j} + \sfrac{\tilde{t}_q}{2}]$. In Substep 3.2 we introduce an approximation of $\theta_{\kappa_q, j}$ denoted by $\tilde{\theta}_{\kappa_q, j}$. Subsequently, in Substep 3.3, we prove that $\| \tilde{\theta}_{\kappa_q , j}(t, \cdot ) \|_{L^2(\T^3)}$ decays in $[t_{q,j}, t_{q,j} + \sfrac{\tilde{t}_q}{2}]$. Finally, in Substep 3.4, we use this to show that $\| {\theta}_{\kappa_q , j}(t, \cdot ) \|_{L^2(\T^3)}$ decays in $[t_{q,j}, t_{q,j} + \sfrac{\tilde{t}_q}{2}]$ too.
%This step is divided into 5 substeps. In Substep 3.1, we prove that ${\theta}_{\kappa_q, j} (t_{q,j} , \cdot)$ is $a_q$-periodic up to small $L^2$ errors. Thanks to this high periodicity structure, in the remainder of this step, via a series of approximations of $\theta_{\kappa_q, j}$, we prove that $\| {\theta}_{\kappa_q , j}(t, \cdot ) \|_{L^2(\T^3)}$ decays rapidly in the time interval $[t_{q,j}, t_{q,j} + \sfrac{\tilde{t}_q}{2}]$. In Substep 3.2, we introduce an approximation of ${\theta}_{\kappa_q, j}$ denoted by $\hat{\theta}_{\kappa_q, j}$. In Substep 3.3, we introduce an approximation of $\hat{\theta}_{\kappa_q, j}$ denoted by $\tilde{\theta}_{\kappa_q, j}$. Subsequently, in Substep 3.4, we prove that the $L^2$ norm of $\tilde{\theta}_{\kappa_q, j}$ decays rapidly. Finally, in Substep 3.5, we use this to show that the $L^2$ norm of ${\theta}_{\kappa_q, j}$ decays quickly too.
\\
\textbf{Substep 3.1: Almost periodicity of ${\theta}_{\kappa_q, j}(t_{q,j} , \cdot, \cdot )$.}
By Section~\ref{rmk:AboutAMollifiedVersionOfTheChessFunction} there exists $(\theta_{\text{chess}})_{a_0^{\eps \delta}} \in L^\infty (\T^2)$ such that 
\begin{align*}
 &\int_{\T^2} |{\theta}_{0,j}(t_{q,j},x,y,z) - \varphi_{\initial}( z- t_{q,j}) \chi_j(z - t_{q,j}) (\theta_{\text{chess}})_{a_0^{\eps \delta}}((2a_q)^{-1} x, (2a_q)^{-1} y)|^2 \, dx dy \\
 &= |\varphi_{\initial}( z- t_{q,j}) \chi_j (z - t_{q,j})|^2 \int_{\T^2} |\theta_{\stat}(x,y,z) - (\theta_{\text{chess}})_{a_0^{\eps \delta}}((2a_q)^{-1} x, (2a_q)^{-1} y)|^2 \, dx dy \\
 &\overset{\eqref{eq:L2DifferenceBetweenChessboardAndItsMollification}}{\lesssim }  a_0^{\eps \delta} |\varphi_{\initial}( z- t_{q,j}) \chi_j(z - t_{q,j})|^2
\end{align*}
which implies
\begin{align*}
 \| {\theta}_{0,j}(t_{q,j}, \cdot, \cdot, \cdot) & - \varphi_{\initial}( \cdot - t_{q,j})  \chi_j(\cdot - t_{q, j}) (\theta_{\text{chess}})_{a_0^{\eps \delta}}((2a_q)^{-1} \cdot, (2a_{q})^{-1} \cdot) \|_{L^2(\T^3)}^2 \lesssim  a_0^{\eps \delta} a_q^{\gamma} \| \varphi_{\initial} \|_{L^{\infty}(\T)}^2 \lesssim a_0^{\eps \delta} a_q^{\gamma}\,.
\end{align*}
Thus, adding and subtracting $\theta_{0,j}$ and using \eqref{stability:theta_kq-theta}, we find
\begin{equation}\label{eq:SymmetriesL2Distance}
 \| {\theta}_{\kappa_q ,j}(t_{q,j}, \cdot, \cdot, \cdot) - \varphi_{\initial}( \cdot - t_{q,j})\chi_j(\cdot - t_{q, j}) (\theta_{\text{chess}})_{a_0^{\eps \delta}}((2a_q)^{-1} \cdot, (2a_{q})^{-1} \cdot) \|_{L^2(\T^3)}^2 \lesssim a_{q}^{\gamma + \eps} + a_0^{\eps \delta} a_q^{\gamma} \lesssim a_0^{\eps \delta} a_q^{\gamma} \,.
\end{equation}
 \\
\textbf{Substep 3.2: Approximation of ${\theta}_{\kappa_q, j}$ on $[t_{q,j}, t_{q,j} + \sfrac{\tilde{t}_q}{2}]$.}
In this substep, we study the solution $\tilde{\theta}_{\kappa_q, j} \colon [t_{q,j}, t_{q,j} + \sfrac{\tilde{t}_q}{2}] \times \T^3 \to \R$ to
\begin{equation}\label{eq:FirstApproxPDEFirstNotNS}
\left\{
\begin{array}{l}
\partial_t \tilde{\theta}_{\kappa_q, j} + \partial_z \tilde{\theta}_{\kappa_q ,j} = \kappa_q \Delta \tilde{\theta}_{\kappa_q ,j} \qquad \text{ on } [t_{q,j}, t_{q,j} + \sfrac{\tilde{t}_q}{2}] \times \T^3;
\\
\tilde{\theta}_{\kappa_q , j}(t_{q, j}, x, y, z) = \varphi_{\initial}( z - t_{q,j})\chi_j (z - t_{q,j}) (\theta_{\text{chess}})_{a_0^{\eps \delta}}( (2 a_q)^{-1}  x,  (2 a_q)^{-1}  y);
\end{array}
\right.
\end{equation}
We can characterize the solution to \eqref{eq:FirstApproxPDEFirstNotNS}. Let $f_{\kappa_q,j} \colon [t_{q, j}, t_{q, j} + \sfrac{\tilde{t}_q}{2}] \times \T^2 \to \R$ be the solution to the heat equation
%
%{\color{red} NOTATION:
%
%- I changed $f_{q,i}$ with $f_{\kappa_q,j}$
%
%- I changed $\psi_{q,i}$ with $\psi_{\kappa_q,j}$}
\begin{equation}\label{eq:2DHeatEquation}
\left\{
 \begin{array}{l}
  \partial_t f_{\kappa_q, j} = \kappa_q \Delta f_{\kappa_q, j}  \qquad \text{ on } [t_{q,j}, t_{q,j} + \sfrac{\tilde{t}_{q}}{2}]  \times \T^2;
   \\
  f_{\kappa_q, j}(t_{q,j},x,y) = (\theta_{\text{chess}})_{a_0^{\eps \delta}}( (2 a_q)^{-1}  x,  (2 a_q)^{-1}  y).
 \end{array}
\right.
\end{equation}
and $\psi_{\kappa_q,j} \colon [t_{q,j}, t_{q,j} + \sfrac{\tilde{t}_q}{2}] \times \T \to \R$ be the solution to
\begin{equation}\label{eq:1DAdvectionDiffusionEquationInStep3}
\left\{
 \begin{array}{l}
  \partial_t \psi_{\kappa_q,j} + \partial_z \psi_{\kappa_q,j} = \kappa_q \partial_{zz} \psi_{\kappa_q,j} \qquad \text{on }  [t_{q,j}, t_{q,j} + \sfrac{\tilde{t}_{q}}{2}]  \times \T; \\
  \psi_{\kappa_q,j}(t_{q,j},z) = \varphi_{\initial}(z- t_{q,j}) \chi_j (z-t_{q,j}).
 \end{array}
\right.
\end{equation}
Define $\tilde{\theta}_{\kappa_q, j} \colon [t_{q,j}, t_{q,j} + \sfrac{\tilde{t}_q}{2}] \times \T^3 \to \R$ as $\tilde{\theta}_{\kappa_q, j}(t,x,y,z) = \psi_{\kappa_q,j}(t,z)f_{\kappa_q, j}(t,x,y)$. By standard computations, $\tilde{\theta}_{\kappa_q, j}$ solves \eqref{eq:FirstApproxPDEFirstNotNS}.
We notice that
\[
 \partial_t ({\theta}_{\kappa_q, j} - \tilde{\theta}_{\kappa_q,j}) + u^{(1,2)} \cdot \nabla_{x,y} {\theta}_{\kappa_q, j} + \partial_z
({\theta}_{\kappa_q, j} - \tilde{\theta}_{\kappa_q,j}) = \kappa_q \Delta ({\theta}_{\kappa_q, j} - \tilde{\theta}_{\kappa_q,j}).
\]
Therefore, for any $t_{q,j} \leq t \leq t_{q,j} + \sfrac{\tilde{t}_q}{2}$, multiplying by $(\theta_{\kappa_q, j} - \tilde{\theta}_{\kappa_q , j})$ and integrating in space
\begin{align*}
 \frac{1}{2} \frac{d}{dt} \| {\theta}_{\kappa_q, j} &- \tilde{\theta}_{\kappa_q,j}\|_{L^2(\T^3)}^2 + \frac{\kappa_q}{2} \| \nabla ({\theta}_{\kappa_q, j} - \tilde{\theta}_{\kappa_q,j}) \|_{L^2(\T^2)}^2 
 \\
 &= - \int_{\T^3} (u^{(1,2)} \cdot \nabla_{x,y} {\theta}_{\kappa_q, j}) ({\theta}_{\kappa_q, j} - \tilde{\theta}_{\kappa_q,j}) \, dx dy dz - \frac{\kappa_q}{2} \| \nabla ({\theta}_{\kappa_q, j} - \tilde{\theta}_{\kappa_q,j})  \|_{L^2(\T^3)}^2 \\
 &\leq \| u^{(1,2)} \cdot \nabla_{x,y} \tilde{\theta}_{\kappa_q,j} \|_{L^2(\T^3)} \| {\theta}_{\kappa_q, j} - \tilde{\theta}_{\kappa_q,j} \|_{L^2(\T^3)} - \frac{\kappa_q}{2} \| \nabla ({\theta}_{\kappa_q, j} - \tilde{\theta}_{\kappa_q,j})  \|_{L^2(\T^3)}^2 \\
 &\leq \frac{C_P}{2 \kappa_q} \| u^{(1,2)} \cdot \nabla_{x,y} \tilde{\theta}_{\kappa_q,j} \|_{L^2(\T^3)}^2 + \frac{\kappa_q}{2 C_P} \| {\theta}_{\kappa_q, j} - \tilde{\theta}_{\kappa_q,j} \|_{L^2(\T^3)}^2 - \frac{\kappa_q}{2 C_P} \| {\theta}_{\kappa_q, j} - \tilde{\theta}_{\kappa_q,j} \|_{L^2(\T^3)}^2 \\
 &\lesssim \frac{1}{\kappa_q} \| u^{(1,2)} \cdot \nabla_{x,y} \tilde{\theta}_{\kappa_q,j}\|_{L^2(\T^3)}^2 \,,
\end{align*}
where $C_P$ is the constant in the Poincar\'{e} inequality.
Thus, integrating in time we get
\begin{align}\label{eq:DifferenceTildeHat}
 \| {\theta}_{\kappa_q, j}(t, \cdot ) - \tilde{\theta}_{\kappa_q,j}(t, \cdot ) & \|_{L^2(\T^3)}^2 + 2 \kappa_q \int_{t_{q,j}}^{t} \| \nabla ({\theta}_{\kappa_q, j} - \tilde{\theta}_{\kappa_q, j})(s, \cdot)\|_{L^2(\T^3)}^2 \, ds \notag 
 \\
 & \lesssim \frac{1}{\kappa_q} \int_{t_{q,j}}^{t} \| u^{(1,2)} \cdot \nabla_{x,y} \tilde{\theta}_{\kappa_q,j}(s, \cdot) \|_{L^2(\T^3)}^2 \, ds + \| {\theta}_{\kappa_q, j}(t_{q,j}, \cdot ) - \tilde{\theta}_{\kappa_q,j}(t_{q,j}, \cdot ) \|_{L^2(\T^3)}^2 \,
\end{align}
for any $t_{q,j} \leq t \leq t_{q,j} + \sfrac{\tilde{t}_q}{2}$.
We now estimate for any $s \in [t_{q,j}, t_{q,j} + \sfrac{\tilde{t}_q}{2}]$
\begin{align*}
 \| u^{(1,2)} \cdot \nabla_{x,y} & \tilde{\theta}_{\kappa_q,j}(s, \cdot ) \|_{L^2(\T^3)}^2 
 \\
 &= \int_{\T^3} |u^{(1,2)}(x,y,z) \cdot \nabla_{x,y} \tilde{\theta}_{\kappa_q,j}(s,x,y,z)|^2 \, dx dy dz \\
 &\overset{\eqref{prop:estimate_u}}{=} \int_{\T^3 \cap \{ z \in (1/2 - T_q, 1/2 - T_q + {\overline{t}_q} )^c \} } |u^{(1,2)}(x,y,z) \cdot \psi_{\kappa_q,j}(s,z) \nabla_{x,y} f_{\kappa_q, j} (s,x,y)|^2 \, dx dy dz \\
 &\leq \| u^{(1,2)} \|_{L^{\infty}(\T^3)}^2 \| \psi_{\kappa_q,j}(s,\cdot) \|_{L^{\infty}( (1/2 - T_q, 1/2 - T_q + {\overline{t}_q} )^c )}^2 \| \nabla_{x,y} f_{\kappa_q, j} (s) \|_{L^{2}(\T^2)}^2 
\end{align*}
First we notice that by \eqref{support:chi_j} and \eqref{d:time_t_j} we have that $\supp (\chi_j  (\cdot - t_{q,j}) ) \subset (1/2 - T_q + \sfrac{a_q^\gamma}{8} , 1/2 - T_q + \sfrac{a_q^\gamma}{4} )$.
We now apply Lemma~\ref{lemma:TailEstimates1DAdvectionDiffusion} with $a= 1/2 -T_q + \sfrac{a_q^\gamma}{8}$, $b = 1/2 -T_q + \sfrac{a_q^\gamma}{4}$ and $c = a_q^\gamma /8$ to estimate 
$$\| \psi_{\kappa_q,j}(s,\cdot) \|_{L^{\infty}( (1/2 - T_q, 1/2 - T_q + {\overline{t}_q} )^c )}^2\leq \| \varphi_{\initial } \|_{L^\infty(\T)} \| \chi_j \|_{L^\infty(\T)} \exp \left( - \frac{a_q^{2 \gamma}}{ 512 \kappa_q (s- t_{q,j})} \right)\,$$
for any  $s \in [t_{q,j}, t_{q,j}  + \tilde{t}_q/2]$,
where we used that  
$$ (1/2 - T_q, 1/2 - T_q + {\overline{t}_q} )^c  \subset (1/2 - T_q +s, 1/2 - T_q +\sfrac{a_q^\gamma}{8} + \sfrac{a_q^\gamma}{4} + s )^c \qquad \text{for any } s \in [0, {\sfrac{\overline{t}_q}{2}}] \,.$$
Therefore, by standard estimates of the exponential function, we get
\begin{align*}
\| u^{(1,2)} \|_{L^{\infty}(\T^3)}^2 \| \psi_{\kappa_q,j}(s,\cdot) \|_{L^{\infty}( (1/2 - T_q, 1/2 - T_q + {\overline{t}_q} )^c )}^2 \lesssim \kappa_q^3 \tilde{t}_q^3 a_q^{- 6 \gamma}
\end{align*}
In virtue of the previous computation, \eqref{eq:SymmetriesL2Distance} and \eqref{eq:DifferenceTildeHat}, we get
\begin{align} 
 &\| {\theta}_{\kappa_q, j}(t, \cdot ) - \tilde{\theta}_{\kappa_q,j}(t, \cdot ) \|_{L^2(\T^3)}^2 + 2 \kappa_q \int_{t_{q,j}}^{t} \| \nabla ({\theta}_{\kappa_q, j} - \tilde{\theta}_{\kappa_q, j})(s, \cdot)\|_{L^2(\T^3)}^2 \, ds \notag 
 \\
 &\lesssim a_0^{\varepsilon \delta} a_q^\gamma + \frac{1}{\kappa_q} \kappa_q^3 \tilde{t}_q^3 a_q^{- 6 \gamma} \int_{t_{q,j}}^t \| \nabla_{x,y} f_{\kappa_q, j} (s, \cdot) \|_{L^{2}(\T^2)}^2 \, ds  \label{eq:TildeApproxOfHatL2}
 \\
 & \overset{\eqref{e:energy-equality-global}}{\leq} a_0^{\varepsilon \delta} a_q^\gamma + \frac{1}{2} \kappa_q \tilde{t}_q^3 a_q^{- 6 \gamma} \| f_{\kappa_q, j}(t_{q,j}, \cdot) \|_{L^2(\T^2)}^2 \lesssim a_0^{\varepsilon \delta} a_q^\gamma \,,  \notag
\end{align}
for any
$t \in [t_{q, j}, t_{q, j} + \sfrac{{\overline{t}_q}}{2}]$, where we used \eqref{d:gamma}.
\\
\textbf{Substep 3.3: Dissipation of $\tilde{\theta}_{\kappa_q,j}(t, \cdot )$ on $[t_{q, j}, t_{q, j} + \sfrac{\tilde{t}_q}{2}]$.} Recall $f_{\kappa_q, j}$ the solution to \eqref{eq:2DHeatEquation} and $\psi_{\kappa_q,j}$ the solution to \eqref{eq:1DAdvectionDiffusionEquationInStep3}.
%
%{\color{red} NOTATION:
%
%- I changed $f_{q,i}$ with $f_{\kappa_q,j}$
%
%- I changed $\psi_{q,i}$ with $\psi_{\kappa_q,j}$}
By Lemma~\ref{lemma:EnhancedDiffusion}, 
\begin{equation}
\| f_{\kappa_q, j}(t_{q,j} + \sfrac{\tilde{t}_q}{2}) \|_{L^2(\T^2)}^2 \leq \e^{-  \frac{\kappa_q  \tilde{t}_q}{4 a_q^2}} \leq \e^{- a_0^{- \eps \delta}} \leq a_0^{\eps \delta},
\end{equation}
where we used that  $\frac{\kappa_q  \tilde{t}_q}{4 a_q^2} = a_0^{- \eps \delta}$ (see \eqref{d:time_t_j} and \eqref{d:k_q}).
Then
\begin{equation}\label{eq:DecayForThetaTilde}
\begin{split}
 \| \tilde{\theta}_{\kappa_q, j}(t_{q,j} + \sfrac{\tilde{t}_q}{2}, \cdot ) \|_{L^2(\T^3)}^2 &= \int_{\T^3} |\psi_{\kappa_q,j}(t_{q,j} + \sfrac{\tilde{t}_q}{2},z)|^2 |f_{\kappa_q, j}(t_{q,j} + \sfrac{\tilde{t}_q}{2},x,y)|^2 \, dx dy dz \\
 & \int_{\T} |\psi_{\kappa_q,j}(t_{q,j} + \sfrac{\tilde{t}_q}{2},z)|^2\, dz \int_{\T^2} |f_{\kappa_q, j}(t_{q,j} + \sfrac{\tilde{t}_q}{2},x,y)|^2 \, dx dy 
  \\
 &= \| \psi_{\kappa_q,j}(t_{q,j} + \sfrac{\tilde{t}_q}{2}, \cdot ) \|_{L^2(\T)}^2 \| f_{\kappa_q, j}(t_{q,j} + \sfrac{\tilde{t}_q}{2}) \|_{L^2(\T^2)}^2
  \\
 &\leq \| \psi_{\kappa_q,j}(t_{q,j} + \sfrac{\tilde{t}_q}{2}, \cdot ) \|_{L^2(\T)}^2 a_0^{\eps \delta} \leq a_0^{\eps \delta} a_q^{\gamma}.
\end{split}
\end{equation}
\\
\textbf{Substep 3.4: Dissipation of ${\theta}_{\kappa_q,j}(t, \cdot )$  on $[t_{q, j}, t_{q, j} + \sfrac{\tilde{t}_q}{2}]$.}
 By  \eqref{eq:TildeApproxOfHatL2} and \eqref{eq:DecayForThetaTilde} we have
\begin{align} \label{eq:dissipation-theta_kappa-j}
 \| {\theta}_{\kappa_q,j}(t_{q, j} + \sfrac{\tilde{t}_q}{2}) \|_{L^2(\T^3)}^2 &\leq 2 \| {\theta}_{\kappa_q,j}(t_{q, j} + \sfrac{\tilde{t}_q}{2}) - \tilde{\theta}_{\kappa_q, j}(t_{q, j} + \sfrac{\tilde{t}_q}{2}) \|_{L^2(\T^3)}^2  +
  2 \| \tilde{\theta}_{\kappa_q, j}(t_{q, j} + \sfrac{\tilde{t}_q}{2}) \|_{L^2(\T^3)}^2 
 \notag
 \\
 &\lesssim a_0^{\eps \delta} a_q^{\gamma}  + a_0^{\eps \delta} a_q^{\gamma} \lesssim a_0^{\eps \delta} a_q^{\gamma}
\end{align}
for $q \in m \N$.
%\textcolor{purple}{By choosing the universal constant $C$ mentioned at the beginning of this step to be
%\begin{equation}
% C = \dfrac{4 C_1 + 4 C_2 + 2}{6 \cdot 10^{-6} \| \varphi_{\initial} \|_{L^{\infty}(\T)}^2}
%\end{equation}
%we get \eqref{eq:AutonomousNormalGoalOfStep3}.}
\\
%\textbf{Substep 3.2 : Approximation of ${\theta}_{\kappa_q, j}$ on $[t_{q,j}, t_{q,j} + \sfrac{\overline{t}_q}{2}]$.}
%In this substep, we solve
%\begin{equation}\label{eq:FirstApproxPDEFirst}
%\left\{
%\begin{array}{l}
%\partial_t \tilde{\theta}_{\kappa_q, j} + \partial_z \tilde{\theta}_{\kappa_q ,j} = \kappa_q \Delta \tilde{\theta}_{\kappa_q ,j} \qquad \text{ on } [t_{q,j}, t_{q,j} \sfrac{\overline{t}_q}{2}] \times \T^3;
%\\
%\tilde{\theta}_{\kappa_q , j}(t_{q, j}, x, y, z) = \varphi_{\initial}( z - t_{q,j})\chi_j (z - t) \Psi( (2 a_q)^{-1}  x,  (2 a_q)^{-1}  y);
%\end{array}
%\right.
%\end{equation}
%and prove that $\| \tilde{\theta}_{\kappa_q, j}(t_{q,j} + \sfrac{\overline{t}_q}{2}) \|_{L^2(\T^3)}^2 \lesssim a_q^{\eps + \gamma}$. We now characterize the solution to \eqref{eq:FirstApproxPDEFirst}. Let $f_{\kappa_q,j} \colon [t_{q, j}, t_{q, j} + \sfrac{\overline{t}_q}{2}] \times \T^2 \to \R$ be the solution to the heat equation
%%
%%{\color{red} NOTATION:
%%
%%- I changed $f_{q,i}$ with $f_{\kappa_q,j}$
%%
%%- I changed $\psi_{q,i}$ with $\psi_{\kappa_q,j}$}
%\begin{equation}
%\left\{
% \begin{array}{l}
%  \partial_t f_{\kappa_q, j} = \kappa_q \Delta f_{\kappa_q, j}  \qquad \text{ on } [t_{q,j}, t_{q,j} + \sfrac{\overline{t}_{q}}{2}]  \times \T^2;
%   \\
%  f_{\kappa_q, j}(t_{q,j},x,y) = \Psi( (2 a_q)^{-1}  x,  (2 a_q)^{-1}  y).
% \end{array}
%\right.
%\end{equation}
%By Lemma~\ref{lemma:EnhancedDiffusion}, $\| f_{\kappa_q, j}(t_{q,j} + \sfrac{\overline{t}_q}{2}) \|_{L^2(\T^2)}^2 \leq \e^{-  \frac{\kappa_q  \overline{t}_q}{4 a_q^2}} \leq a_q^{\frac{\delta^2 \gamma}{1 + \delta} - 10 \eps} \stackrel{
%\eqref{item:EpsDeltaThree}}{\leq} a_q^{\eps}$, where we used that  $\frac{\kappa_q  \overline{t}_q}{4 a_q^2} = a_q^{- \frac{\delta^2 \gamma}{1 + \delta} + 10 \eps}$.
%Let $\psi_{\kappa_q,j} \colon [t_{q,j}, t_{q,j} + \sfrac{\overline{t}_q}{2}] \times \T \to \R$ be the solution to
%\begin{equation}
%\left\{
% \begin{array}{l}
%  \partial_t \psi_{\kappa_q,j} + \partial_z \psi_{\kappa_q,j} = \kappa_q \partial_{zz} \psi_{\kappa_q,j} \qquad \text{on }  [t_{q,j}, t_{q,j} + \sfrac{\overline{t}_{q}}{2}]  \times \T; \\
%  \psi_{\kappa_q,j}(t_{q,j},z) = \varphi_{\initial}(z- t_{q,j}) \chi_j (z-t_{q,j}).
% \end{array}
%\right.
%\end{equation}
%Define $\tilde{\theta}_{\kappa_q, j} \colon [t_{q,j}, t_{q,j} + \sfrac{\overline{t}_q}{2}] \times \T^3 \to \R$ as $\tilde{\theta}_{\kappa_q, j}(t,x,y,z) = \psi_{\kappa_q,j}(t,z)f_{\kappa_q, j}(t,x,y)$. By standard computations, $\tilde{\theta}_{\kappa_q, j}$ solves \eqref{eq:FirstApproxPDE}. Then
%\begin{equation}\label{eq:DecayForThetaTilde}
%\begin{split}
% \| \tilde{\theta}_{\kappa_q, j}(t_{q,j} + \sfrac{\overline{t}_q}{2}, \cdot ) \|_{L^2(\T^3)}^2 &= \int_{\T^3} |\psi_{\kappa_q,j}(t_{q,j} + \sfrac{\overline{t}_q}{2},z)|^2 |f_{\kappa_q, j}(t_{q,j} + \sfrac{\overline{t}_q}{2},x,y)|^2 \, dx dy dz \\
% & \int_{\T} |\psi_{\kappa_q,j}(t_{q,j} + \sfrac{\overline{t}_q}{2},z)|^2\, dz \int_{\T^2} |f_{\kappa_q, j}(t_{q,j} + \sfrac{\overline{t}_q}{2},x,y)|^2 \, dx dy 
%  \\
% &= \| \psi_{\kappa_q,j}(t_{q,j} + \sfrac{\overline{t}_q}{2}, \cdot ) \|_{L^2(\T)}^2 \| f_{\kappa_q, j}(t_{q,j} + \sfrac{\overline{t}_q}{2}) \|_{L^2(\T^2)}^2
%  \\
% &\leq \| \psi_{\kappa_q,j}(t_{q,j} + \sfrac{\overline{t}_q}{2}, \cdot ) \|_{L^2(\T)}^2 a_q^{\eps} \leq a_q^{\gamma + \eps}.
%\end{split}
%\end{equation}
%\\
%\textbf{Substep 3.3 : \textcolor{purple}{Improved approximation of ${\theta}_{\kappa_q,j}$ on $[t_{q,j}, t_{q,j} + \sfrac{\overline{t}_q}{2}]$.}}
%In this substep, we prove that $\hat{\theta}_{\kappa_q, j} \colon [t_{q,j}, t_{q,j} + \sfrac{\overline{t}_q}{2}] \times \T^3 \to \R$ solution to
%\begin{equation}\label{eq:SecondApproxPDE}
%\left\{
%\begin{array}{l}
%\partial_t \hat{\theta}_{\kappa_q, j} + u \cdot \nabla \hat{\theta}_{\kappa_q, j} = \kappa_q \Delta \hat{\theta}_{\kappa_q, j} \qquad 
%\text{ on }
%[t_{q,j}, t_{q,j} + \sfrac{\overline{t}_q}{2}] \times \T^3;
%\\
%\hat{\theta}_{\kappa_q, j}(t_{q,j}, x, y, z) = \varphi_{\initial }( z- t_{q,j})\chi_j (z - t_{q,j}) \Psi((2a_q)^{-1} x, (2 a_q)^{-1} y);
%\end{array}
%\right.
%\end{equation}
%satisfies $\| \hat{\theta}_{\kappa_q, j}(t_{q, j} + \sfrac{\overline{t}_q}{2}, \cdot ) \|_{L^2(\T^3)}^2 \lesssim a_{q}^{\gamma + \eps}$.
%
%We notice that
%\[
% \partial_t (\hat{\theta}_{\kappa_q, j} - \tilde{\theta}_{\kappa_q,j}) + u^{(1,2)} \cdot \nabla_{x,y} \hat{\theta}_{\kappa_q, j} + \partial_z
%(\hat{\theta}_{\kappa_q, j} - \tilde{\theta}_{\kappa_q,j}) = \kappa_q \Delta (\hat{\theta}_{\kappa_q, j} - \tilde{\theta}_{\kappa_q,j}).
%\]
%Therefore, for any $t_{q,j} \leq t \leq t_{q,j} + \sfrac{\overline{t}_q}{2}$
%\begin{align*}
% \frac{1}{2} \frac{d}{dt} & \| \hat{\theta}_{\kappa_q, j} (t, \cdot) - \tilde{\theta}_{\kappa_q,j} (t, \cdot )\|_{L^2(\T^3)}^2  
% \\
% &= - \int_{\T^3} (u^{(1,2)} \cdot \nabla_{x,y} \hat{\theta}_{\kappa_q, j}) (\hat{\theta}_{\kappa_q, j} - \tilde{\theta}_{\kappa_q,j}) \, dx dy dz - \kappa_q \| \nabla (\hat{\theta}_{\kappa_q, j} - \tilde{\theta}_{\kappa_q,j})  \|_{L^2(\T^3)}^2 \\
% &\leq \| u^{(1,2)} \cdot \nabla_{x,y} \tilde{\theta}_{\kappa_q,j} \|_{L^2(\T^3)} \| \hat{\theta}_{\kappa_q, j} - \tilde{\theta}_{\kappa_q,j} \|_{L^2(\T^3)} - \kappa_q \| \nabla (\hat{\theta}_{\kappa_q, j} - \tilde{\theta}_{\kappa_q,j})  \|_{L^2(\T^3)}^2 \\
% &\leq \frac{C_P}{4 \kappa_q} \| u^{(1,2)} \cdot \nabla_{x,y} \tilde{\theta}_{\kappa_q,j} \|_{L^2(\T^3)}^2 + \frac{\kappa_q}{C_P} \| \hat{\theta}_{\kappa_q, j} - \tilde{\theta}_{\kappa_q,j} \|_{L^2(\T^3)}^2 - \frac{\kappa_q}{C_P} \| \hat{\theta}_{\kappa_q, j} - \tilde{\theta}_{\kappa_q,j} \|_{L^2(\T^3)}^2 \\
% &\lesssim \frac{1}{\kappa_q} \| u^{(1,2)} \cdot \nabla_{x,y} \tilde{\theta}_{\kappa_q,j}  (t, \cdot )\|_{L^2(\T^3)}^2 \,,
%\end{align*}
%where $C_P$ is the constant in the Poincar\'{e} inequality.
%Thus, integrating in time we get
%\begin{equation}\label{eq:DifferenceTildeHat}
% \| \hat{\theta}_{\kappa_q, j}(t, \cdot ) - \tilde{\theta}_{\kappa_q,j}(t, \cdot ) \|_{L^2(\T^3)}^2 \lesssim \frac{1}{\kappa_q} \int_{t_{q,j}}^{t} \| u^{(1,2)} \cdot \nabla_{x,y} \tilde{\theta}_{\kappa_q,j}(s) \|_{L^2(\T^3)}^2 \, ds \,,
%\end{equation}
%for any $t_{q,j} \leq t \leq t_{q,j} + \sfrac{\overline{t}_q}{2}$.
%We now estimate for any $s \in [t_{q,j}, t_{q,j} + \overline{t}_q/2]$
%\begin{align*}
% \| u^{(1,2)} \cdot \nabla_{x,y} & \tilde{\theta}_{\kappa_q,j}(s, \cdot ) \|_{L^2(\T^3)}^2 
% \\
% &= \int_{\T^3} |u^{(1,2)}(x,y,z) \cdot \nabla_{x,y} \tilde{\theta}_{\kappa_q,j}(s,x,y,z)|^2 \, dx dy dz \\
% &= \int_{\T^3 \cap \{ z \in (1/2 - T_q, 1/2 - T_q + \overline{t}_q )^c \} } |u^{(1,2)}(x,y,z) \cdot \psi_{\kappa_q,j}(s,z) \nabla_{x,y} f_{\kappa_q, j} (s,x,y)|^2 \, dx dy dz \\
% &\leq \| u^{(1,2)} \|_{L^{\infty}(\T^3)}^2 \| \psi_{\kappa_q,j}(s,\cdot) \|_{L^{\infty}( (1/2 - T_q, 1/2 - T_q + \overline{t}_q )^c )}^2 \| \nabla_{x,y} f_{\kappa_q, j} (s) \|_{L^{2}(\T^2)}^2 
% \\
% &\lesssim \| \psi_{\kappa_q,j}(s,\cdot) \|_{L^{\infty}( (1/2 - T_q, 1/2 - T_q + \overline{t}_q )^c )}^2 \| \nabla ( \Psi ((2a_q)^{-1} \cdot, (2a_q)^{-1} \cdot) ) \|_{L^{2}(\T^2)}^2  \,.
%\end{align*}
%First we notice that by \eqref{support:chi_j} and \eqref{d:time_t_j} we have that $\supp (\chi_j  (\cdot - t_{q,j}) ) \subset (1/2 - T_q + \sfrac{a_q^\gamma}{8} , 1/2 - T_q + \sfrac{a_q^\gamma}{4} )$.
%\textcolor{purple}{We now apply Lemma~\ref{lemma:TailEstimates1DAdvectionDiffusion} with $a= 1/2 -T_q + \sfrac{a_q^\gamma}{8}$, $b = 1/2 -T_q + \sfrac{a_q^\gamma}{4}$ and $c = a_q^\gamma /8$ to estimate 
%$$\| \psi_{\kappa_q,j}(s,\cdot) \|_{L^{\infty}( (1/2 - T_q, 1/2 - T_q + \overline{t}_q )^c )}^2\leq \| \varphi_{\initial } \|_{L^\infty(\T)} \| \chi_j \|_{L^\infty(\T)} \exp \left( - \frac{a_q^{2 \gamma}}{ 512 \kappa_q (s- t_{q,j})} \right)\,$$
%for any  $s \in [t_{q,j}, t_{q,j}  + \overline{t}_q/2]$,
%where we used that  
%$$ (1/2 - T_q, 1/2 - T_q + \overline{t}_q )^c  \subset (1/2 - T_q +s, 1/2 - T_q +\sfrac{a_q^\gamma}{8} + \sfrac{a_q^\gamma}{4} + s )^c \qquad \text{for any } s \in [0, \overline{t}_q/2] \,.$$
%Therefore we get
%\begin{align*}
%\| \psi_{\kappa_q,j}(s,\cdot) & \|_{L^{\infty}  ( (1/2 - T_q, 1/2 - T_q + \overline{t}_q )^c )}^2 \| \nabla ( \Psi ((2a_q)^{-1} \cdot, (2a_q)^{-1} \cdot) ) \|_{L^{2}(\T^2)}^2  \lesssim a_q^{-1} \exp \left( - \frac{a_q^{2 \gamma}}{ 512 \kappa_q (s - t_{q,j} )} \right) \,.
%\end{align*}
%In virtue of the previous computation  and \eqref{eq:DifferenceTildeHat}, we get
%\begin{align}\label{stability:theta_hat-theta_tilde}
% \| \hat{\theta}_{\kappa_q, j}(t, \cdot ) - \tilde{\theta}_{\kappa_q,j}(t, \cdot ) \|_{L^2(\T^3)}^2 &\lesssim \frac{a_q^{-1}}{\kappa_q}  \int_{t_{q,j}}^t \underbrace{\exp \left( - \frac{a_q^{2 \gamma}}{512 \kappa_q (s-t_{q,j}) } \right)}_{\leq \left( \frac{512 \kappa_q (s - t_{q, j})}{a_q^{2 \gamma}} \right)^{2} } \, ds \lesssim a_q^{-1 - 4 \gamma} \kappa_q  {\overline{t}_q}^{3} \lesssim a_q^{\gamma + \varepsilon}
%\end{align}
%for any
%$t \in [t_{q, j}, t_{q, j} + \sfrac{\overline{t}_q}{2}]$, where we used \eqref{d:gamma}.}
%Thus, thanks to \eqref{eq:DecayForThetaTilde} and \eqref{stability:theta_hat-theta_tilde}
%\[
% \| \hat{\theta}_{\kappa_q, j}(t_{q, j} + \sfrac{\overline{t}_q}{2}, \cdot ) \|_{L^2(\T^3)}^2 \lesssim  \| \hat{\theta}_{\kappa_q, j}(t_{q,j} + \sfrac{\overline{t}_q}{2}, \cdot ) - \tilde{\theta}_{\kappa_q,j}(t_{q,j} + \sfrac{\overline{t}_q}{2}, \cdot ) \|_{L^2(\T^3)}^2 + \| \tilde{\theta}_{\kappa_q,j}(t_{q, j} + \sfrac{\overline{t}_q}{2}) \|_{L^2(\T^3)}^2  \lesssim a_q^{\gamma + \eps}.
%\]
%We call $C_2>0$ the constant for which we have $\| \hat{\theta}_{\kappa_q, j}(t_{q, j} + \sfrac{\overline{t}_q}{2}, \cdot ) \|_{L^2(\T^3)}^2 \leq C_2 a_{q}^{\gamma + \varepsilon}$.
%\\
%\textbf{Substep 3.4 : Decay of $\| {\theta}_{\kappa_q,j} (t, \cdot) \|_{L^2(\T^3)}$ in $[t_{q,j}, t_{q,j} + \sfrac{\overline{t}_{q}}{2}]$.}
%Using standard energy estimates and \eqref{eq:SymmetriesL2Distance} we have 
%% \eqref{eq:SolutionTransportEquationProp1},  \eqref{stability:theta_kq-theta}  we have
%\begin{align*}
% \| {\theta}_{\kappa_q,j}(t, \cdot ) &  - \hat{\theta}_{\kappa_q, j}(t, \cdot) \|_{L^2(\T^3)}^2  \leq \| {\theta}_{\kappa_q,j}(t_{q, j}, \cdot ) - \hat{\theta}_{\kappa_q, j}(t_{q, j} , \cdot ) \|_{L^2(\T^3)}^2 
% \\
% & \leq 2 \| {\theta}_{\kappa_q,j}(t_{q, j}, \cdot ) - \varphi_{\initial} (\cdot - t_{q,j})  \chi_j (\cdot - t_{q,j})   \Psi((2a_q)^{-1} \cdot, (2a_q)^{-1} \cdot )  \|_{L^2(\T^3)}^2 
% \\
% & \quad + 2 \| \varphi_{\initial} (\cdot - t_{q,j})  \chi_j (\cdot - t_{q,j})   \Psi((2a_q)^{-1} \cdot, (2a_q)^{-1} \cdot ) - \hat{\theta}_{\kappa_q, j}(t_{q, j} , \cdot ) \|_{L^2(\T^3)}^2 
%\\
% & \leq 2 C_1 a_q^{\gamma + \eps} + 2 \cdot 10^{-6} \| \varphi_{\initial} \|_{L^{\infty}(\T)}^2 a_q^{\gamma}.
%\end{align*}
%for all $t \in [t_{q, j}, t_{q, j} + \sfrac{\overline{t}_q}{2}]$.
%Thus, 
%\begin{align*}
% \| {\theta}_{\kappa_q,j}(t_{q, j} + \sfrac{\overline{t}_q}{2}) \|_{L^2(\T^3)}^2 &\leq 2 \| {\theta}_{\kappa_q,j}(t_{q, j} + \sfrac{\overline{t}_q}{2}) - \hat{\theta}_{\kappa_q, j}(t_{q, j} + \sfrac{\overline{t}_q}{2}) \|_{L^2(\T^3)}^2 + 2 \| \hat{\theta}_{\kappa_q, j}(t_{q, j} + \sfrac{\overline{t}_q}{2}) \|_{L^2(\T^3)}^2 \\
% &\leq 4 C_1 a_q^{\gamma + \eps} + 4 \cdot 10^{-6} \| \varphi_{\initial} \|_{L^{\infty}(\T)}^2 + 2 C_2 a_q^{\gamma + \eps} \,,
%\end{align*}
%where the last follows from the previous step.
%Therefore, with $q$ large enough depending on the constants $C_1$ and $ C_2$ we have
%\begin{equation*}
% \| {\theta}_{\kappa_q,j}(t_{q, j} + \sfrac{\overline{t}_q}{2}) \|_{L^2(\T^3)}^2 \leq 10^{-5} \| \varphi_{\initial} \|_{L^{\infty}(\T)}^2 a_q^{\gamma}.
%\end{equation*}
\textbf{Step 4: Decay of  $\| {\theta}_{\kappa_q} (t, \cdot) \|_{L^2 (\T^3)}$ for $ t \in (0,1)$.}
In this step, we prove \eqref{eq:LTwoNormAlmostZero}.
For each $j = 0, \ldots, N_q$, we define the sets $\tilde A_{q,j} \subset \T$ as
\[ 
\tilde A_{q,j} = \supp (\chi_j (1- \cdot )) \subset [1+ z_j , 1+ z_j + a_q^\gamma] \simeq_{\T }  [ z_j ,  z_j + \sfrac{a_q^\gamma}{8}]
\]
and the $\frac{a_q^\gamma}{32}$-neighbourhood 
$$A_{q,j} = \left \{ z \in \T : \dist (z, \tilde A_{q,j}) < \frac{a_q^\gamma}{32} \right \}\,. $$

By Corollary~\ref{lemma:TailEstimates3DAdvectionDiffusion}
\begin{equation}\label{eq:ApplicationOfTailEstimatesInThmA}
 \| {\theta}_{\kappa_q,j}(1 , \cdot ) \|_{L^{\infty}(\T^2 \times A_{q,j}^c )} \leq 2 \exp \left( - \frac{(\sfrac{ a_q^{\gamma}}{32})^2}{8 \kappa_q } \right) = 2 \exp \left( - \frac{ a_q^{2 \gamma}}{8 \cdot 32^2 \kappa_q} \right). \\
\end{equation}
This will allow us to conclude the argument later. Now, we observe that
\begin{equation}\label{eq:L2NormOfTheFullSolutionFirstIneq}
 \| {\theta}_{\kappa_q}(1, \cdot ) \|_{L^2(\T^3)}^2 = \left\| \sum_{j = 0}^{N_q} {\theta}_{\kappa_q,j}(1, \cdot ) \right\|_{L^2(\T^3)}^2 \leq 2 \left\| \sum_{j \text{ odd}} {\theta}_{\kappa_q,j}(1, \cdot ) \right\|_{L^2(\T^3)}^2 + 2 \left\| \sum_{j \text{ even}} {\theta}_{\kappa_q,j}(1, \cdot ) \right\|_{L^2(\T^3)}^2
\end{equation}
and
\begin{align*}
 \left\| \sum_{j \text{ odd}} {\theta}_{\kappa_q,j}(1, \cdot) \right\|_{L^2(\T^3)}^2 &= \left\| \sum_{j \text{ odd}} {\theta}_{\kappa_q,j}(1, \cdot ) \mathbbm 1_{\T^2 \times A_{q,j}} (\cdot ) + \sum_{j \text{ odd}} {\theta}_{\kappa_q,j}(1, \cdot ) \mathbbm 1_{\T^2 \times A_{q,j}^c} (\cdot ) \right\|_{L^2(\T^3)}^2 \\
 &\leq 2 \left\| \sum_{j \text{ odd}} {\theta}_{\kappa_q,j}(1, \cdot ) \mathbbm 1_{\T^2 \times A_{q,j}} (\cdot ) \right\|_{L^2(\T^3)}^2 + 2 \left\| \sum_{j \text{ odd}} {\theta}_{\kappa_q,j}(1, \cdot) \mathbbm 1_{\T^2 \times A_{q,j}^c} (\cdot) \right\|_{L^2(\T^3)}^2.
\end{align*}
The collection of open sets $\{ A_{q,j} \}_{ \{ j  \ \text{odd}  \} }$ is a collection of mutually disjoint sets due to \eqref{eq:DisjointnessOfTheSupportsOfCutoffs}. This follows from the definition of the quantities $z_j$. Therefore
\begin{align*}
\left\| \sum_{j \text{ odd}} {\theta}_{\kappa_q,j}(1, \cdot ) 1_{\T^2 \times A_{q,j}} (\cdot ) \right\|_{L^2(\T^3)}^2 &= \sum_{j \text{ odd}} \left\| {\theta}_{\kappa_q,j}(1, \cdot ) 1_{\T^2 \times A_{q,j}} \right\|_{L^2(\T^3)}^2 \leq \sum_{j \text{ odd}} \left\| {\theta}_{\kappa_q,j}(1, \cdot ) \right\|_{L^2(\T^3)}^2 \\
&\leq \sum_{j \text{ odd}} \left\| {\theta}_{\kappa_q,j}(t_{q, j} + \sfrac{\overline{t}_q}{2}, \cdot ) \right\|_{L^2(\T^3)}^2 \lesssim N_q a_0^{\eps \delta} a_q^{\gamma} \lesssim a_0^{\eps \delta}.\\
\end{align*}
By \eqref{eq:ApplicationOfTailEstimatesInThmA},
\begin{align*}
\left\| \sum_{j \text{ odd}} {\theta}_{\kappa_q,j}(1, \cdot ) \mathbbm 1_{\T^2 \times A_{q,j}^c} (\cdot ) \right\|_{L^2(\T^3)} &\leq \left\| \sum_{j \text{ odd}} {\theta}_{\kappa_q,j}(1, \cdot ) 1_{\T^2 \times A_{q,j}^c} \right\|_{L^{\infty}(\T^3)} \leq \sum_{j \text{ odd}} \left\| {\theta}_{\kappa_q,j}(1, \cdot ) \right\|_{L^{\infty}(\T^2 \times A_{q,i}^c)} \\
&\leq \sum_{j \text{ odd}} 2 \exp \left( - \frac{ a_q^{2 \gamma}}{8 \cdot 32^2 \kappa_q} \right)
 \lesssim \sum_{j \text{ odd}}  \frac{8 \cdot 32^2 \kappa_q}{ a_q^{2 \gamma}} \lesssim a_q^{2 - 3\gamma - \frac{\gamma}{1 + \delta} + 10 \eps}  \leq a_q \,.
\end{align*}
Thus, $\left\| \sum_{j \text{ odd}} {\theta}_{\kappa_q,j}(1, \cdot ) \right\|_{L^2(\T^3)}^2 \lesssim a_0^{\eps \delta} + a_q \lesssim a_0^{\eps \delta}$.
The same bound can be proved for the last term in \eqref{eq:L2NormOfTheFullSolutionFirstIneq}. Therefore, we conclude \eqref{eq:LTwoNormAlmostZero}. 

\begin{comment}
    \begin{align*}
 \| {\theta}_{\kappa_q}(1, \cdot ) \|_{L^2(\T^3)}^2 &\leq 2 \left\| \sum_{i \text{ odd}} {\theta}_{\kappa_q,j}(1, \cdot ) \right\|_{L^2(\T^3)}^2 + 2 \left\| \sum_{i \text{ even}} {\theta}_{\kappa_q,j}(1, \cdot ) \right\|_{L^2(\T^3)}^2 \lesssim a_0^{\eps \delta}.
\end{align*}
\end{comment}

\noindent \textbf{Step 5: Smoothness of the energy $e$ and $L^\infty_{t} L^2_x$ closeness.}
We observe that, by Lemma~\ref{lemma:uniqueness}, there exists a unique solution $\theta_0 \in L^\infty ((0,1) \times \T^3)$ of the advection equation, with velocity field $u: \T^3 \to \R^3$ and initial datum $\theta_{\initial} \in L^\infty (\T^3)$ defined in \eqref{eq:initialdatum_theta}. We observe also that any $L^\infty ((0,1) \times \T^3)$-weakly* converging subsequence of $ \{ \theta_{\kappa} \}_{\kappa >0}$ (solutions of the advection diffusion equation with initial datum $\theta_{\initial}$) converges to a solution of the advection equation with initial datum $\theta_{\initial}$. Therefore we conclude that $\theta_\kappa \overset{*}{\rightharpoonup} \theta_0$, as $\kappa \to 0$.
By definition of $\theta_0$ in Subsection~\ref{subsec:ConstructionInitialDatum}, it is clear that 
$$e(t) =\int_{\T^3} | \theta_{0} (t, x) |^2 dx $$
is smooth in $(0,1/2)$.   Thanks to Lemma \ref{lemma:uniqueness} we can uniquely find $\theta_0 : [0,1 ] \times \T^3 \to \R$ a solution to the advection equation with velocity field $u$ and initial datum $\theta_{\initial}$. We also notice that, thanks to $u^{(3)} \equiv 1$ and $u (x,y,z) \equiv (0,0,1)$ for $z \geq 1/2$ and $\supp (\theta_{\initial}) \subset  \T^2 \times (0, 1/4) $ (see \eqref{eq:initialdatum_theta}), there exists $\tau >0$ such that $\int_{\T^3} | \theta_0 (t,x) |^2 dx$ is constant in $(1/2 - \tau , 1)$, therefore $e(t) = \int_{\T^3} | \theta_{0} (t, x) |^2 dx$   is smooth in $(0,1)$.

%Let $\tilde \chi_1, \tilde \chi_2 \in C^\infty (0,1)$ be a partition of unity, i.e. $\tilde \chi_1 (t) + \chi_2 (t) =1$  for all $t \in [0,1]$ and $\supp ( \chi_1 ) \subset [0, 1/2 +)$, $\supp ( \chi_2 ) \subset [1/2 - , 1]$ such that $\tilde \theta_0 (t,x)$  

Finally, we need to prove that for any $\kappa_q $ such that $q \in m \N$ we have the $L^\infty_{t} L^2_x$ closeness of $\theta_{\kappa_q}$ to $\theta_0$, more precisely we claim that there exists a universal constant $C>0$ such that 
\begin{equation}\label{eq:L2ClosenessInTermsOfLimsup}
\limsup_{q \in m\N, \, q \to \infty} \| \theta_{\kappa_q} - \theta_0 \|_{L^\infty ((0,1); L^2(\T^3))}^2 \leq C a_0^{\epsilon \delta}.
\end{equation}
To prove the inequality above we estimate

\begin{align}
\begin{split}\label{eq:ComputationInOrderToComputeL-Inf-L-2}
    \int_{\T^3} |\theta_{\kappa_q} (x ,t) & -\theta_0 (x ,t)|^2 dx   
     =  \int_{\T^3} \left| \sum_{j=0}^{N_q} (\theta_{k_q, j} - \theta_{0,j}) \right|^2
    \\
    & \leq   \underbrace{\sum_{j=0 }^{N_q} \int_{\T^3} |\theta_{\kappa_q, j} (x ,t) -\theta_{0,j} (x ,t)|^2 }_{= I (t)} + \underbrace{ \sum_{i \neq j}^{N_q} \int_{\T^3} |\theta_{\kappa_q, j} (x ,t) -\theta_{0,j} (x ,t)| |\theta_{\kappa_q, i} (x ,t) -\theta_{0,i} (x ,t)|}_{= II(t)}
\end{split}
\end{align}
We start by estimating $\| II \|_{L^{\infty}((0,1))}$.
Using that $| \theta_{\kappa_q, j} (x,y,z, t)| \leq \psi_{\kappa_q, j} (z,t ) $ and $|\theta_{0,j} (x,y,z,t)| \leq \chi_{j} (z-t) $ for any $(x,y, z) \in \T^3$ and $t \in (0,1)$,
we estimate $II(t)$ 
\begin{align*}
    II(t) & \leq \sum_{i \neq j}^{N_q} \int_{\T^3} (\psi_{\kappa_q , j} (z,t) + \chi_{j} (z- t)) (\psi_{\kappa_q , i} (z,t) + \chi_{i} (z- t)).
\end{align*}
Let $S_{q,i}(t)$ be the time-dependent interval defined by $S_{q,i}(t) = (z_i - a_0 a_q^{\gamma} + t, z_i + \sfrac{a_q^{\gamma}}{8} + a_0 a_q^{\gamma} + t)$. By Lemma~\ref{lemma:TailEstimates1DAdvectionDiffusion}, we find that for all $t \in [0,1]$
\begin{equation}\label{eq:PsiOutsideTheSetS-q-i}
 \| \psi_{\kappa_q, i}(t, \cdot) \|_{L^{\infty}(S_{q,i}(t)^c)} \lesssim \exp\left( \frac{(a_0 a_q^{\gamma})^2}{8 \kappa_q t} \right) \leq \frac{8 \kappa_q t}{a_0^2 a_q^{2 \gamma}} \lesssim \frac{a_q}{a_0^2} \lesssimlarge a_0 a_q^{1/2}.
\end{equation}
Note that due to \eqref{eq:AboutAdjacentSupports} and \eqref{eq:DisjointnessOfTheSupportsOfCutoffs}, we have
\begin{equation}\label{eq:AboutHowTheSetsS-q-i-Overlap}
S_{q,i}(t) \cap S_{q,j}(t) = \emptyset \text{ whenever $|i-j| > 1$ and } |S_{q,i}(t) \cap S_{q,i + 1}(t)| \leq 3 a_0 a_q^{\gamma} \text{ for all $t \in [0,1]$.}
\end{equation}
Therefore, for all $t \in [0,1]$, we have
\begin{align*}\label{eq:AboutTwoAdjacentPsis}
\begin{split}
 \int_{\T} |\psi_{\kappa_q, i}| |\psi_{\kappa_q, i+1}|(t,z) \, dz &\leq \int_{S_{q,i}(t) \cap S_{q,i+1}(t)} |\psi_{\kappa_q, i}| |\psi_{\kappa_q, i+1}|(t,z) \, dz 
 \\
 & \quad + \int_{\T \setminus (S_{q,i}(t) \cap S_{q,i+1}(t))} |\psi_{\kappa_q, i}| |\psi_{\kappa_q, i+1}|(t,z) \, dz 
 \lesssimlarge a_0 a_q^{\gamma} + a_0 a_q^{\gamma} \lesssimlarge a_0 a_q^{\gamma}.
\end{split}
\end{align*} 
Also notice that 
%$\supp (\chi_j) \cap \supp (\chi_{i}) = \empty $ for any $|i-j|>1$, 
$\supp (\chi_{j} (\cdot - t)) \subset S_{q,j} (t)$ for any $t \in (0,1)$ and $j =0,\ldots,N_q$ from which we get (using the convention that $\psi_{\kappa_q, N_q + 1} = \psi_{\kappa_q, 1}$ and $\chi_{N_q + 1} = \chi_1$)
\begin{align*}
    II(t) & \leq    \sum_{|i- j| > 1 }^{N_q} \int_{\T^3} (\psi_{\kappa_q , j} (z,t) + \chi_{j} (z- t)) (\psi_{\kappa_q , i} (z,t) + \chi_{i} (z- t))
    \\
    & \quad + \sum_{j=0}^{N_q} \int_{\T^3} (\psi_{\kappa_q , j} (z,t) + \chi_{j} (z- t)) (\psi_{\kappa_q , j+1} (z,t) + \chi_{j+1} (z- t))
    \\
    & \lesssimlarge  N_q^2 a_0^2 a_q + N_q a_0 a_q^{\gamma} \lesssimlarge a_0
\end{align*}
where we used \eqref{item:BoundOnTheNumberOfCutoffs}, \eqref{eq:PsiOutsideTheSetS-q-i} and \eqref{eq:AboutHowTheSetsS-q-i-Overlap}. Hence $\| II \|_{L^{\infty}((0,1))} \lesssimlarge a_0$.
\\
Now, we estimate $\| I \|_{L^{\infty}((0,1))}$.
Splitting $I$ into 3 parts and using the $L^\infty$ bound on $\theta_{\kappa_q}$ yields
\begin{align*}
    I(t)= \sum_{j=0}^{N_q} \int_{\T^3} |\theta_{\kappa_q,j} (x ,t) -\theta_{0,j} (x ,t)|^2 dx  
    & \leq  10 T_q + \underbrace{ \sum_{j=0}^{N_q} \int_{\T^2 \times \{ z < 1/2 - T_q \}} |\theta_{\kappa_q,j} (x,t) - \theta_{0,j} (x,t)|^2 dx }_{= I_1(t)} 
    \\
    & \quad  + \underbrace{ \sum_{j=0}^{N_q}  \int_{\T^2 \times \{ z > 1/2\}} |\theta_{\kappa_q,j} (x,t) - \theta_{0,j} (x,t)|^2 dx  }_{= I_2 (t)}.
\end{align*}
Observing that $T_q \to 0$ as $q \to \infty$, we only have to estimate $\| I_1 \|_{L^{\infty}((0,1))}$ and $\| I_2 \|_{L^{\infty}((0,1))}$.
We now estimate each one of the terms in $I_1$ given by
\[
 I_{1, j}(t) = \int_{\T^2 \times \{ z < 1/2 - T_q \}} |\theta_{\kappa_q,j} (x,t) - \theta_{0,j} (x,t)|^2 dx \text{ for $j = 1, \ldots, N_q$.}
\] 
If $t < t_{q,j}$ (defined in \eqref{d:time_t_j}), we use \eqref{stability:theta_kq-theta} and \eqref{item:BoundOnTheNumberOfCutoffs} to get that $|I_{1,j}(t)| \lesssim a_q^{\gamma + \eps}$.
If $t \geq t_{q,j}$, then by definition of $t_{q,j}$ and $\theta_{0,j}$ (as in \eqref{d:theta_0j}) and the fact that the third component of $u$ is constantly $1$ we have that $\supp (\theta_{0,j} (t, \cdot )) \cap \{ (x,y,z ) \in \T^3 : z < 1/2 - T_q \} = \emptyset$. Furthermore, by defining $\tilde{\psi}_{\kappa, j} : [0,1] \times \T \to \R$
\begin{equation}
\left\{
 \begin{array}{l}
  \partial_t \tilde{\psi}_{\kappa_q,j} + \partial_z \tilde{\psi}_{\kappa_q,j} = \kappa_q \partial_{zz} \tilde{\psi}_{\kappa_q,j} \text{ on } [0,1 ] \times \T
  \\
  \tilde{\psi}_{\kappa_q,j}(0,z) = \varphi_{\initial}(z) \chi_j (z).
 \end{array}
\right.
\end{equation}
we have by comparison principle that  $\theta_{\kappa, j} (t,x,y,z) \leq  \tilde{\psi}_{\kappa, j} (t,z)$ for any $(t,x,y,z ) \in [0,1] \times \T^3$ and using Lemma \ref{lemma:TailEstimates1DAdvectionDiffusion} with the set $A(t) = (z_j + t - \sfrac{a_q^{\gamma}}{8} , z_j +t + 2 \sfrac{a_q^{\gamma}}{8})$ we have
\[
\| \theta_{\kappa_q, j} (t, \cdot ) \|_{L^\infty (\T^2 \times \{ z < 1/2 - T_q\})} \leq  \| \tilde{\psi}_{\kappa_q, j} (t, \cdot ) \|_{L^\infty (A(t)^c)} \lesssim a_q^{\gamma + \epsilon}
\]
for any $t \geq t_{q,j}$, where we used 
the definition of $\kappa_q$ \eqref{d:k_q}, $t_{q,j}$ \eqref{d:time_t_j}, $z_j$ for the supports of $\chi_j$ \eqref{support:chi_j} and standard estimates on the exponential function. Therefore, we have $\| I_{1,j} \|_{L^{\infty}((0,1))} \lesssim a_q^{\gamma + \eps}$. Hence
$$ \| I_1 \|_{L^{\infty}((0,1))}  \leq \sum_{j = 1}^{N_q} \| I_{1,j} \|_{L^{\infty}((0,1))} \leq N_q a_q^{\gamma + \eps} \lesssim a_q^{\eps}.$$
We now estimate each one of the terms of $I_2$ given by
\[
 I_{2, j}(t) = \int_{\T^2 \times \{ z > 1/2\}} |\theta_{\kappa_q,j} (x,t) - \theta_{0,j} (x,t)|^2 dx.
\]
It is straightforward to see that
$$ |I_{2, j}(t)| \leq 
\underbrace{
\int_{\T^2 \times \{ z > 1/2\}} 2 |\theta_{\kappa_q,j} (x,t)|^2  dx }_{= I_{2,j}^{\prime}(t)} +
\underbrace{\int_{\T^2 \times \{ z > 1/2\}} 2   
| \theta_{0,j} (x,t)|^2 dx}_{= I_{2,j}^{\prime \prime}(t)} \,.
$$
If $t < t_{q,j} + \overline{t}_q$, then by Lemma~\ref{lemma:TailEstimates1DAdvectionDiffusion} with $\tilde{\psi}_{\kappa, j}$ and $A(t)$ as before we have
$$ I_{2,j}^{\prime}(t) \leq 2 \| \theta_{\kappa , j} (t, \cdot ) \|_{L^\infty (\T^2 \times \{ z >1/2 \})}^2 \lesssim \| \tilde{\psi}_{\kappa, j} (t, \cdot ) \|_{L^\infty (A(t)^c)}^2 \lesssim a_q^{\gamma + \epsilon}.$$
If $t \geq t_{q,j} + \overline{t}_q$, using \eqref{eq:dissipation-theta_kappa-j} we have 
$$ I_{2,j}^{\prime}(t) \leq \| \theta_{\kappa, j} (t, \cdot) \|_{L^2}^2 \leq \| \theta_{\kappa, j} (t_{q,j} + \overline{t}_q, \cdot) \|_{L^2}^2 \leq a_0^{\epsilon \delta } a_q^{\gamma}.$$
for any $q \in m \N$. Hence $\| I_{2,j}^{\prime} \|_{L^{\infty}((0,1))} \lesssim a_0^{\eps \delta} a_q^{\gamma}$.
Using \eqref{item:dissipation-theta_0} we have that 
$$ \| I_{2,j}^{\prime \prime} \|_{L^{\infty}((0,1))} \lesssim a_0^{\epsilon \delta} a_q^{\gamma}.$$
As a consequence 
$$\| I_2 \|_{L^{\infty}((0,1))} \leq \sum_{j = 1}^{N_q} \| I_{2,j}^{\prime} \|_{L^{\infty}((0,1))} + \sum_{j = 1}^{N_q} \| I_{2,j}^{\prime \prime} \|_{L^{\infty}((0,1))} \lesssim a_0^{\eps \delta}.$$
Thus, we conclude $\| I \|_{L^{\infty}((0,1))} \leq \| I_1 \|_{L^{\infty}((0,1))} + \| I_2 \|_{L^{\infty}((0,1))} \lesssim a_0^{\eps \delta}$. Due to \eqref{eq:ComputationInOrderToComputeL-Inf-L-2}, we conclude that \eqref{eq:L2ClosenessInTermsOfLimsup} holds, as wished.
%In the following we will use the following notation: for any $\mu \in \mathcal{M} (0,1)$ we split it as $\mu = \mu_{a} + \mu_{\perp}$ where $\mu_{a} $ is the absolutely continuous part of the measure with respect to $\mathcal{L}$ and $\mu_{\perp}$ the singular part of $\mu$.
 Thanks to \eqref{eq:L2ClosenessInTermsOfLimsup}, we proved that there exists $f_{\err} \in L^\infty ((0,1) \times \T^3)$ such that $\| f_{\err} \|_{L^\infty ((0,1); L^2 (\T^3))}^2 \leq C a_0^{\epsilon \delta }$ and
 $$  |\theta_{\kappa_q}|^2 \rightharpoonup^*   |\theta_0|^2 +  f_{\err}$$
 weakly* in $L^\infty$, up to subsequences. Indeed 
 \begin{align*}
     \int^{1}_0   \int_{ \T^3 } \left | | \theta_{\kappa_q}|^2 - |\theta_0|^2 \right |^2 \leq ( \| \theta_{\kappa_q} \|_{L^\infty}
     + \| \theta_0 \|_{L^\infty} )^2 \int_0^1 \int_{\T^3} |\theta_{\kappa_q} - \theta_0 |^2 \leq C \| \theta_{\kappa_q} - \theta_0 \|_{L^\infty((0,1); L^2(\T^3))}^2 \,.
 \end{align*}
   We notice that, thanks to the energy balance we have that $\{ \kappa_{q} | \nabla \theta_{\kappa_{q}} |^2\}_{q \in m \N}$ is uniformly bounded in $L^1((0,1) \times \T^3)$, and hence there exists $\mu \in \mathcal{M}([0,1] \times \T^3)$ such that, up to non relabelled subsequences we have 
$$ \kappa_{q} | \nabla \theta_{\kappa_{q}} |^2 \weak \mu \geq 0 \,,$$
in the sense of measure as $q \to \infty$ and we define $\mu_T \in \mathcal{M}(0,1)$ as $\mu_T  = \pi_{\#} \mu $. 
 Therefore, passing into the limit in the sense of distributions as $\kappa_q \to 0$  (along the sequence $\{ \kappa_{q} \}_{q \in m \N} $)  the distributional identity of the local energy equality of Lemma \ref{lemma:local-energy}  we have the following identity in the sense of distribution
 $$ \frac{1}{2} \partial_t |\theta_0|^2 + \frac{1}{2} \partial_t f_{\err} + \frac{1}{2} \diver (u |\theta_0|^2) + \frac{1}{2} \diver (u f_{\err}) + \mu =0 \,,$$
from which, integrating in space, we get  the $H^{-1}_{t}$ closeness 
 $$ \left \| \partial_t \frac{1}{2} \int_{\T^3} |\theta_0|^2  + \mu_T \right \|_{H^{-1}_{t}} = \left \| \frac{1}{2} \partial_t \int_{\T^3} f_{\err}  \right \|_{H^{-1}_{t}} \leq C a_0^{\epsilon \delta } < \beta \,,$$
where
 $\mu_T  = \pi_{\#} \mu $,
  from which we get  \eqref{e:closeness-H-1} up to choose $a_0$ sufficiently small in terms of $\beta$ in  Subsection \ref{subsec:ParmeterDefinitions}.

 


\noindent \textbf{Step 6: Convergence behaviour of $-\frac{1}{2}  e_{\kappa_q}^{\prime}$ for $q \in m \N$.}
All lemmas stated in the proof will be proved at the end of Step 6.
In order to finish the proof of the theorem, we examine the behaviour of 
\[
-\frac{1}{2} e_{\kappa_q}^{\prime}(t) =  \kappa_q \int_{\T^3} |\nabla \theta_{\kappa_q}(t,x)|^2 \, dx.
\]
As observed above, by the energy balance it is clear that up to subsequences $\{ -\frac{1}{2} e_{\kappa_q}^{\prime} \}_{q  \in m \N}$ weak-$\ast$ converges to a non-negative measure $\mu_T \geq 0$. This measure can be decomposed into an absolutely continuous part and a singular part. The goal of this step is to show that the total variation of the singular part is small compared to the one of the absolutely continuous part. We define $A_{\kappa_q} \colon [0,1] \to \R$ as
\[
 A_{\kappa_q}(t) \coloneqq  \kappa_q \sum_{j = 1}^{N_q} \int_{\T^3} |\nabla \tilde{\theta}_{\kappa_q, j}(t,x)|^2 \mathbbm{1}_{[t_{q,j}, t_{q,j} + \tilde{t}_q]} (t) \, dx
\]
and $S_{\kappa_q} \colon [0,1] \to \R$ as $S_{\kappa_q}(t) = -\frac{1}{2} e_{\kappa_q}^{\prime}(t) - A_{\kappa_q}(t)$. We will show that 
\begin{equation}\label{eq:TheTwoPropertiesThatWeWant}
\limsup_{q \to \infty} \| S_{\kappa_q} \|_{L^1((0,1))} \lesssim a_0^{\frac{\eps \delta}{2}}\,, \quad  \sup_{q} \| A_{\kappa_q} \|_{L^{\infty}((0,1))} < \infty \,, \quad \| \mu_T \|_{TV} \geq 1/4 \,.
\end{equation}
This proves that up to subsequences $A_{\kappa_q}$ weak-$\ast$ converges in $L^{\infty}$ to some $\mathcal{A} \in L^{\infty}((0,1))$ and $S_{\kappa_q}$ weak-$\ast$ converges in the sense of measures to $\mathcal{S} \in \mathcal{M}((0,1))$ (which potentially has a singular part) such that $\| \mathcal{S} \|_{TV} \lesssim a_0^{\varepsilon \delta /2} < \beta < 1/4$, by choosing $a_0$ sufficiently small depending on $\beta$ and universal constants which proves that the absolutely continuous part of the measure $\mu_T = \mathcal{A} + \mathcal{S}$ is non-trivial. The proof is divided into 4 substeps. The first 3 substeps are dedicated to prove, via a sequence of approximations that $S_{\kappa_q} = -\frac{1}{2} e_{\kappa_q}^{\prime} - A_{\kappa_q}$ is uniformly small in $L^1$. In the last step, we prove that $\sup_{q} \| A_{\kappa_q} \|_{L^{\infty}((0,1))} < \infty$ and $\| \mu_T \|_{TV} \geq 1/4$. \\
\textbf{Substep 6.1:}
We have
\[
 -\frac{1}{2} e_{\kappa_q}^{\prime}(t) = \underbrace{ \kappa_q \sum_{i = 1}^{N_q} \int_{\T^3} |\nabla \theta_{\kappa_q, i}(t,x)|^2 \, dx}_{= A_{\kappa_q}^{(1)}(t)} + \underbrace{ \kappa_q \sum_{i,j = 1, i \neq j}^{N_q} \int_{\T^3} \nabla \theta_{\kappa_q, i}(t,x) \cdot \nabla \theta_{\kappa_q, j}(t,x) \, dx}_{= S_{\kappa_q}^{(1)}(t)}.
\]
The goal of this substep is to show that $\| S_{\kappa_q}^{(1)} \|_{L^1((0,1))} \lesssimlarge a_0^{\eps \delta}$. Decompose $S_{\kappa_q}^{(1)}$ into two parts
\[
 S_{\kappa_q}^{(1)}(t) =  \underbrace{\kappa_q \sum_{i = 1}^{N_q} \int_{\T^3} \nabla \theta_{\kappa_q, i}(t,x) \cdot \nabla \theta_{\kappa_q, i+1}(t,x) \, dx}_{= I(t)} + \underbrace{ \kappa_q \sum_{|i - j| > 1} \int_{\T^3} \nabla \theta_{\kappa_q, i}(t,x) \cdot \nabla \theta_{\kappa_q, j}(t,x) \, dx}_{= II(t)}
\]
with the convention that $\theta_{\kappa_q, N_q+1}(t,x) = \theta_{\kappa_q, 1}(t,x)$. 
We will prove the two following results at the end of this section:
\begin{lemma}\label{lemma:EstimateOfQuantitiy-II}
There exists a constant $Q \in \N$ and a  universal constant $C>0$ such that for all $q \geq Q$ we have
\begin{equation}\label{eq:EstimateOfQuantitiy-II}
 \| II \|_{L^1((0,1))} \leq C a_q^{\sfrac{1}{8}}.
\end{equation}
\end{lemma}
\begin{lemma}\label{lemma:EstimateOfQuantitiy-I}
There exists a constant $Q \in \N$ and a universal constants $C>0$ such that for all $q \in m \N$ such that $q \geq Q$ we have
\begin{equation}\label{eq:EstimateOfQuantitiy-I}
 \| I \|_{L^1((0,1))} \leq C a_0^{\sfrac{\eps \delta}{2}}.
\end{equation}
\end{lemma}
\begin{comment}
The remainder of this substep consists in estimating $\| I \|_{L^1((0,1))}$ and $\| II \|_{L^1((0,1))}$. \\
\fbox{Estimate of $\| II \|_{L^1((0,1))}$: }. Recall the sequence of cutoff functions $\{ \overline{\chi}_{j} \}_{j = 1}^{N_q}$ introduced in Subsection ?????. For each $j$, let $\psi_{\kappa_q, j} \colon [0,1] \times \T \to \R$ be the solution to the advection-diffusion equation
 \begin{equation}\label{eq:1D-Adv-Diff-Cutoffs}
 \left\{
 \begin{array}{ll}
 \partial_t \overline{\psi}_{\kappa_q, j} + \partial_z \overline{\psi}_{\kappa_q, j} = \partial_{zz} \overline{\psi}_{\kappa_q, j}; \\
 \overline{\psi}_{\kappa_q, j}(0, \cdot) = \overline{\chi}_j(\cdot). \\
 \end{array}
 \right.
\end{equation}
For each $j$, define $\overline{\theta}_{\kappa_q, j}$ as $\overline{\theta}_{\kappa_q, j}(t,x,y,z) = {\theta}_{\kappa_q, j}(t,x,y,z) \overline{\psi}_{\kappa_q, j}(t,z)$. By standard computations, we note that 
 \begin{equation}\label{eq:1D-Adv-Diff-Cutoffs-Prod-With-Solution}
 \left\{
 \begin{array}{ll}
 \partial_t \overline{\theta}_{\kappa_q, j} + u \cdot \nabla \overline{\theta}_{\kappa_q, j} = \kappa_q \Delta \overline{\theta}_{\kappa_q, j} - 2 \kappa_q \nabla {\theta}_{\kappa_q, j} \cdot \nabla \overline{\psi}_{\kappa_q, j}; \\
 \overline{\theta}_{\kappa_q, j}(0, \cdot) = {\theta}_{\kappa_q, j}(0, \cdot). \\
 \end{array}
 \right.
\end{equation}
By energy estimates, we find
\begin{align*}
 &\| ({\theta}_{\kappa_q, j} - \overline{\theta}_{\kappa_q, j})(t, \cdot) \|_{L^2(\T^3)}^2 + 2 \kappa_q \int_0^t \int_{\T^3} |\nabla({\theta}_{\kappa_q, j} - \overline{\theta}_{\kappa_q, j})|^2 \, dx ds \\
 &= 4 \kappa_q \int_0^t \int_{\T^3} (\nabla {\theta}_{\kappa_q, j} \cdot \nabla \overline{\psi}_{\kappa_q, j})({\theta}_{\kappa_q, j} - \overline{\theta}_{\kappa_q, j}) \, dx ds \\
 &\leq 4 \kappa_q \int_0^t \int_{\T^3} | \nabla {\theta}_{\kappa_q, j} \cdot \nabla \overline{\psi}_{\kappa_q, j}| \, dx ds \\
 &\leq 2 \left( 2 \kappa_q \int_0^t \int_{\T^3} |\nabla {\theta}_{\kappa_q, j}|^2 \, dx ds \right)^{\sfrac{1}{2}} \left( 2 \kappa_q \int_0^t \int_{\T^3} |\nabla \overline{\psi}_{\kappa_q, j}|^2 \, dx ds \right)^{\sfrac{1}{2}} \\
 &\leq 2 \| \theta_{\kappa_q, j}(0, \cdot) \|_{L^2(\T^3)} \sqrt{2 \kappa_q} \| \partial_z \overline{\chi}_{j} \|_{L^{\infty}(\T^3)} \\
 &\leq \frac{5\sqrt{2}}{2} \overline{C} a_q^{\sfrac{\gamma}{2}} \kappa_q^{\sfrac{1}{2}} a_q^{- \gamma} = \frac{5\sqrt{2}}{2} \overline{C} a_q^{- \sfrac{\gamma}{2}} \kappa_q^{\sfrac{1}{2}}.
\end{align*}
In particular,
\begin{equation}\label{eq:Useful-Stability-Between-Theta-And-Approximation}
2 \kappa_q \int_0^t \int_{\T^3} |\nabla({\theta}_{\kappa_q, j} - \overline{\theta}_{\kappa_q, j})|^2 \, dx ds \leq \frac{5\sqrt{2}}{4} \overline{C} a_q^{- \sfrac{\gamma}{2}} \kappa_q^{\sfrac{1}{2}}.
\end{equation}
Now, instead of estimating the terms of $II$ in $L^{1}((0,1))$, we will estimate
\[
 2 \kappa_q \int_{\T^3} \nabla \overline{\theta}_{\kappa_q, i}(t,x) \cdot \nabla \overline{\theta}_{\kappa_q, j}(t,x) \, dx
\]
in $L^{1}((0,1))$ for $i$ and $j$ such that $|i - j| > 1$. We will then use \eqref{eq:Useful-Stability-Between-Theta-And-Approximation} to show that this is a good approximation of the terms of $II$ in $L^{1}((0,1))$. We observe that
\begin{align*}
&\left\| 2 \kappa_q \int_{\T^3} \nabla \overline{\theta}_{\kappa_q, i}(\cdot,x) \cdot \nabla \overline{\theta}_{\kappa_q, j}(\cdot,x) \, dx \right\|_{L^1((0,1))} = 2 \kappa_q \int_0^1 \left| \int_{\T^3} \nabla \overline{\theta}_{\kappa_q, i} \cdot \nabla \overline{\theta}_{\kappa_q, j} \, dx \right| \, dt \\
&= \underbrace{2 \kappa_q \int_0^1 \left| \int_{\T^3} (\nabla \psi_{\kappa_q, i} \cdot \nabla \psi_{\kappa_q, j}) ({\theta}_{\kappa_q, i} {\theta}_{\kappa_q, j}) \, dx \right| \, dt}_{= A_1} + \underbrace{2 \kappa_q \int_0^1 \left| \int_{\T^3} (\nabla \psi_{\kappa_q, i} \cdot \nabla \theta_{\kappa_q, j}) ({\theta}_{\kappa_q, i} {\psi}_{\kappa_q, j}) \, dx \right| \, dt}_{= A_2} \\
&+ \underbrace{2 \kappa_q \int_0^1 \left| \int_{\T^3} (\nabla \theta_{\kappa_q, i} \cdot \nabla \psi_{\kappa_q, j}) ({\psi}_{\kappa_q, i} {\theta}_{\kappa_q, j}) \, dx \right| \, dt}_{= A_3} + \underbrace{2 \kappa_q \int_0^1 \left| \int_{\T^3} (\nabla \theta_{\kappa_q, i} \cdot \nabla \theta_{\kappa_q, j}) ({\psi}_{\kappa_q, i} {\psi}_{\kappa_q, j}) \, dx \right| \, dt}_{= A_4}.
\end{align*}
Now, we estimate each one of the quantities $A_1$, $A_2$, $A_3$ and $A_4$.
\begin{align*}
A_1 &= 2 \kappa_q \int_0^1 \left| \int_{\T^3} (\nabla \psi_{\kappa_q, i} \cdot \nabla \psi_{\kappa_q, j}) ({\theta}_{\kappa_q, i} {\theta}_{\kappa_q, j}) \, dx \right| \, dt \\
&\leq 2 \kappa_q \int_0^1 \| \nabla \psi_{\kappa_q, i} (t, \cdot) \|_{L^{\infty}(\T)} \| \nabla \psi_{\kappa_q, j} (t, \cdot) \|_{L^{\infty}(\T)} \| \theta_{\kappa_q, i}(t, \cdot) \|_{L^{\infty}(\T^3)} \| \theta_{\kappa_q, j}(t, \cdot) \|_{L^{\infty}(\T^3)} \, dt \\
&\leq 50 \kappa_q \| \nabla \psi_{\kappa_q, i} (0, \cdot) \|_{L^{\infty}(\T)} \| \nabla \psi_{\kappa_q, j} (0, \cdot) \|_{L^{\infty}(\T)} \leq 50 \overline{C}^2 \kappa_q a_q^{- 2 \gamma} \leq 50 \overline{C}^2 a_q \\
A_2 &= 2 \kappa_q \int_0^1 \left| \int_{\T^3} (\nabla \psi_{\kappa_q, i} \cdot \nabla \theta_{\kappa_q, j}) ({\theta}_{\kappa_q, i} {\psi}_{\kappa_q, j}) \, dx \right| \, dt \\
&\leq \left( 2 \kappa_q \int_0^1 \int_{\T^3} |\nabla \theta_{\kappa_q, j}|^2 \, dx dt \right)^{\sfrac{1}{2}} \left( 2 \kappa_q \int_0^1 \int_{\T^3} |\nabla \psi_{\kappa_q, i}|^2 \, dx dt \right)^{\sfrac{1}{2}} \| {\theta}_{\kappa_q, i} \|_{L^{\infty}(\T^3)} \| {\psi}_{\kappa_q, j} \|_{L^{\infty}(\T)} \\
&\leq 5 \sqrt{2} \| \theta_{\initial, j} \|_{L^2(\T^2)} \kappa_q^{\sfrac{1}{2}} \| \nabla \psi_{\kappa_q, i}(0, \cdot) \|_{L^{\infty}(\T)} \leq \frac{5\sqrt{2}}{2} \overline{C} a_q^{\sfrac{\gamma}{2}} \kappa_q^{\sfrac{1}{2}} a_q^{- \gamma} \leq \frac{5\sqrt{2}}{2} \overline{C} a_q^{\sfrac{1}{2}} \\
A_3 &= \ldots \text{(similar to $A_2$)} \ldots \leq \frac{5\sqrt{2}}{2} \overline{C} a_q^{\sfrac{1}{2}} \\
A_4 &= 2 \kappa_q \int_0^1 \left| \int_{\T^3} (\nabla \theta_{\kappa_q, i} \cdot \nabla \theta_{\kappa_q, j}) ({\psi}_{\kappa_q, i} {\psi}_{\kappa_q, j}) \, dx \right| \, dt \\
&\leq \left( 2 \kappa_q \int_0^1 \int_{\T^3} |\nabla \theta_{\kappa_q, i}|^2 \, dx dt \right)^{\sfrac{1}{2}} \left( 2 \kappa_q \int_0^1 \int_{\T^3} |\nabla \theta_{\kappa_q, j}|^2 \, dx dt \right)^{\sfrac{1}{2}} \| {\psi}_{\kappa_q, i} {\psi}_{\kappa_q, j} \|_{L^{\infty}(\T)} \\
&\leq \| \theta_{\initial, i}(0, \cdot) \|_{L^2(\T^3)} \| \theta_{\initial, j}(0, \cdot) \|_{L^2(\T^3)} \| {\psi}_{\kappa_q, i} {\psi}_{\kappa_q, j} \|_{L^{\infty}(\T)} \leq \frac{5}{8} 20000 a_q^{\gamma} \kappa_q a_q^{- 2 \gamma} \leq 40000 \kappa_q a_q^{- \gamma} \leq 40000 a_q.\\
\end{align*}
Hence, we find that
\[
 \left\| 2 \kappa_q \int_{\T^3} \nabla \overline{\theta}_{\kappa_q, i}(\cdot,x) \cdot \nabla \overline{\theta}_{\kappa_q, j}(\cdot,x) \, dx \right\|_{L^1((0,1))} \leq C(a_0) a_q^{\sfrac{1}{2}}.
\]
Now, in order to use this estimate to bound the terms of $II$ in $L^1((0,1))$, we compute
\begin{align*}
&2 \kappa_q \int_0^1 \left| \int_{\T^3} \nabla {\theta}_{\kappa_q, i} \cdot \nabla {\theta}_{\kappa_q, j} - \nabla \overline{\theta}_{\kappa_q, i} \cdot \nabla \overline{\theta}_{\kappa_q, j} \, dx \right| \, dt \\
&\leq 2 \kappa_q \int_0^1 \int_{\T^3} \left| \nabla {\theta}_{\kappa_q, i} \cdot \nabla ({\theta}_{\kappa_q, j} - \overline{\theta}_{\kappa_q, j}) \right| \, dx \, dt + 2 \kappa_q \int_0^1 \int_{\T^3} \left| \nabla ({\theta}_{\kappa_q, i} - \overline{\theta}_{\kappa_q, i}) \cdot \nabla \overline{\theta}_{\kappa_q, j} \right| \, dx \, dt \\
&\leq \left( 2 \kappa_{q} \int_0^1 \int_{\T^3} \left| \nabla {\theta}_{\kappa_q, i} \right|^2 \, dx \, dt \right)^{\sfrac{1}{2}} \left( 2 \kappa_{q} \int_0^1 \int_{\T^3} \left| \nabla ({\theta}_{\kappa_q, j} - \overline{\theta}_{\kappa_q, j}) \right|^2 \, dx \, dt \right)^{\sfrac{1}{2}} \\
&+ \left( 2 \kappa_{q} \int_0^1 \int_{\T^3} \left| \nabla \overline{\theta}_{\kappa_q, j} \right|^2 \, dx \, dt \right)^{\sfrac{1}{2}} \left( 2 \kappa_{q} \int_0^1 \int_{\T^3} \left| \nabla ({\theta}_{\kappa_q, i} - \overline{\theta}_{\kappa_q, i}) \right|^2 \, dx \, dt \right)^{\sfrac{1}{2}} \\
&\leq \left( 2 \kappa_{q} \int_0^1 \int_{\T^3} \left| \nabla {\theta}_{\kappa_q, i} \right|^2 \, dx \, dt \right)^{\sfrac{1}{2}} \left( 2 \kappa_{q} \int_0^1 \int_{\T^3} \left| \nabla ({\theta}_{\kappa_q, j} - \overline{\theta}_{\kappa_q, j}) \right|^2 \, dx \, dt \right)^{\sfrac{1}{2}} \\
&+ \left[ \sqrt{\frac{5 \sqrt{2}}{4}} \sqrt{\overline{C}} a_q^{- \sfrac{\gamma}{4}} \kappa_q^{\sfrac{1}{4}} + \left( 2 \kappa_{q} \int_0^1 \int_{\T^3} \left| \nabla {\theta}_{\kappa_q, j} \right|^2 \, dx \, dt \right)^{\sfrac{1}{2}} \right] \left( 2 \kappa_{q} \int_0^1 \int_{\T^3} \left| \nabla ({\theta}_{\kappa_q, i} - \overline{\theta}_{\kappa_q, i}) \right|^2 \, dx \, dt \right)^{\sfrac{1}{2}} \\
&\leq 5 a_q^{\sfrac{\gamma}{2}} \sqrt{\frac{5 \sqrt{2}}{4}} \sqrt{\overline{C}} a_q^{- \sfrac{\gamma}{4}} \kappa_q^{\sfrac{1}{4}} + \left[ \sqrt{\frac{5 \sqrt{2}}{4}} \sqrt{\overline{C}} a_q^{- \sfrac{\gamma}{4}} \kappa_q^{\sfrac{1}{4}} + 5a_q^{\sfrac{\gamma}{2}}\right] \sqrt{\frac{5 \sqrt{2}}{4}} \sqrt{\overline{C}} a_q^{- \sfrac{\gamma}{4}} \kappa_q^{\sfrac{1}{4}}
\lesssim a_q^{\sfrac{\gamma}{4}} \kappa_q^{\sfrac{1}{4}} \leq a_q^{\sfrac{1}{2}}
\end{align*}
Hence
\[
 \left\| 2 \kappa_q \int_{\T^3} \nabla {\theta}_{\kappa_q, i}(\cdot,x) \cdot \nabla {\theta}_{\kappa_q, j}(\cdot,x) \, dx \right\|_{L^1((0,1))} \lesssim C(a_0) a_q^{\sfrac{1}{2}}
\]
and thus $\| II \|_{L^1((0,1))} \lesssim C(a_0) N_q^2 a_q^{\sfrac{1}{2}} \lesssim C(a_0) a_q^{\sfrac{1}{2} - 2 \gamma}$. \\
\fbox{Estimate $\| I \|_{L^1((0,1))}$: }
We decompose the terms in $I$ further.
\begin{align*}
 &\left\| 4 \kappa_q \int_{\T^3} \nabla \theta_{\kappa_q, i}(\cdot,x) \cdot \nabla \theta_{\kappa_q, i+1}(\cdot,x) \, dx \right\|_{L^1((0,1))} = 4 \kappa_q \int_0^1 \left| \int_{\T^3} \nabla \theta_{\kappa_q, i}(t,x) \cdot \nabla \theta_{\kappa_q, i+1}(t,x) \, dx \right| \, dt \\
 &= \underbrace{4 \kappa_q \int_0^1 \left| \int_{\T^3} \nabla \theta_{\kappa_q, i}(t,x) \cdot \nabla \theta_{\kappa_q, i+1}(t,x) \, dx \mathbbm{1}_{[0, t_{q,i}]}(t) \right| \, dt}_{\tilde{A}_1} \\
 &+ \underbrace{4 \kappa_q \int_0^1 \left| \int_{\T^3} \nabla \theta_{\kappa_q, i}(t,x) \cdot \nabla \theta_{\kappa_q, i+1}(t,x) \, dx \mathbbm{1}_{[t_{q,i}, t_{q,i+1} + \overline{t}_q]}(t) \right| \, dt}_{\tilde{A}_2} \\ 
 &+ \underbrace{4 \kappa_q \int_0^1 \left| \int_{\T^3} \nabla \theta_{\kappa_q, i}(t,x) \cdot \nabla \theta_{\kappa_q, i+1}(t,x) \, dx \mathbbm{1}_{[t_{q,i+1} + \overline{t}_q,1]}(t) \right| \, dt}_{\tilde{A}_3}
\end{align*}
Now we estimate the quantities $\tilde{A}_1$ and $\tilde{A}_3$ which we expect to be small compared to $a_q^{\gamma}$:
\begin{align*}
 \tilde{A}_1 &= 4 \kappa_q \int_0^{t_{q,i}} \left| \int_{\T^3} \nabla \theta_{\kappa_q, i}(t,x) \cdot \nabla \theta_{\kappa_q, i+1}(t,x) \, dx \right| \, dt \\
  &\leq 2 \left( 2 \kappa_q \int_0^{t_{q,i}} \int_{\T^3} |\nabla \theta_{\kappa_q, i}|^2 \, dx dt \right)^{\sfrac{1}{2}} \left( 2 \kappa_q \int_0^{t_{q,i}} \int_{\T^3} |\nabla \theta_{\kappa_q, i+1}|^2 \, dx dt \right)^{\sfrac{1}{2}}\\
  &\leq 2 \left( \| \theta_{\initial, i} \|_{L^2(\T^3)}^2 - \| \theta_{\kappa_q, i}(t_{q,i}, \cdot) \|_{L^2(\T^3)}^2 \right)^{\sfrac{1}{2}} \| \theta_{\initial, i +1}\|_{L^2(\T^3)} \\
   &\leq 2 \left( \| \theta_{\initial, i} \|_{L^2(\T^3)} + \| \theta_{\kappa_q, i}(t_{q,i}, \cdot) \|_{L^2(\T^3)} \right)^{\sfrac{1}{2}} \left( \| \theta_{\initial, i} \|_{L^2(\T^3)} - \| \theta_{\kappa_q, i}(t_{q,i}, \cdot) \|_{L^2(\T^3)} \right)^{\sfrac{1}{2}} \| \theta_{\initial, i +1}\|_{L^2(\T^3)} \\
   &\leq 2 \left( \| \theta_{\initial, i} \|_{L^2(\T^3)} + \| \theta_{\kappa_q, i}(t_{q,i}, \cdot) \|_{L^2(\T^3)} \right)^{\sfrac{1}{2}} \left( \| \theta_{0, i}(t_{q,i}, \cdot) \|_{L^2(\T^3)} - \| \theta_{\kappa_q, i}(t_{q,i}, \cdot) \|_{L^2(\T^3)} \right)^{\sfrac{1}{2}} \| \theta_{\initial, i +1}\|_{L^2(\T^3)} \\
     &\leq 2 \left( \| \theta_{\initial, i} \|_{L^2(\T^3)} + \| \theta_{\kappa_q, i}(t_{q,i}, \cdot) \|_{L^2(\T^3)} \right)^{\sfrac{1}{2}} \left( \| (\theta_{0, i} - \theta_{\kappa_q, i})(t_{q,i}, \cdot) \|_{L^2(\T^3)} \right)^{\sfrac{1}{2}} \| \theta_{\initial, i +1}\|_{L^2(\T^3)} \\
  &\lesssim a_q^{\sfrac{\gamma}{4}} a_q^{\frac{\gamma + \eps}{4}} a_q^{\sfrac{\gamma}{2}} = a_q^{\gamma+ \sfrac{\eps}{4}}  \\
  \tilde{A}_3 &= 4 \kappa_q \int_{t_{q,i+1} + \overline{t}_q}^1 \left| \int_{\T^3} \nabla \theta_{\kappa_q, i}(t,x) \cdot \nabla \theta_{\kappa_q, i+1}(t,x) \, dx \right| \, dt \\
  &\leq 2 \left( 2 \kappa_q \int_{t_{q,i+1} + \overline{t}_q}^1 \int_{\T^3} |\nabla \theta_{\kappa_q, i}|^2 \, dx dt \right)^{\sfrac{1}{2}} \left( 2 \kappa_q \int_{t_{q,i+1} + \overline{t}_q}^1 \int_{\T^3} |\nabla \theta_{\kappa_q, i+1}|^2 \, dx dt \right)^{\sfrac{1}{2}}\\
  &\leq 2 \| \theta_{\initial, i} \|_{L^2(\T^3)} (\| \theta_{\kappa_q, i+1} (t_{q,i+1} + \overline{t}_q, \cdot) \|_{L^2(\T^3)})^{\sfrac{1}{2}} \\
  &\lesssim a_0^{\frac{\eps \delta}{2}} a_q^{\gamma}
\end{align*}
To estimate $\tilde{A}_2$, we proceed by a sequence of approximations. We will reuse the functions $\tilde{\theta}_{\kappa_q, j}$, $\hat{\theta}_{\kappa_q, j}$, $f_{\kappa_q, j}$ and $\psi_{\kappa_q, j}$ introduced in Substep 3 of the proof. We note that
\begin{align*}
\tilde{A}_2 &= 4 \kappa_q \int_{t_{q,i}}^{t_{q,i+1} + \overline{t}_q} \left| \int_{\T^3} \nabla \theta_{\kappa_q, i} \cdot \nabla \theta_{\kappa_q, i+1} \, dx \right| \, dt \\
&\leq 4 \kappa_q \int_{t_{q,i}}^{t_{q,i+1} + \overline{t}_q} \left| \int_{\T^3} \nabla (\theta_{\kappa_q, i} - \hat{\theta}_{\kappa_q, i})\cdot \nabla \theta_{\kappa_q, i+1} \, dx \right| \, dt + 4 \kappa_q \int_{t_{q,i}}^{t_{q,i+1} + \overline{t}_q} \left| \int_{\T^3} \nabla \hat{\theta}_{\kappa_q, i} \cdot \nabla ( \theta_{\kappa_q, i+1} - \hat{\theta}_{\kappa_q, i+1}) \, dx \right| \, dt \\
& \qquad + 4 \kappa_q \int_{t_{q,i}}^{t_{q,i+1} + \overline{t}_q} \left| \int_{\T^3} \nabla \hat{\theta}_{\kappa_q, i} \cdot \nabla \hat{\theta}_{\kappa_q, i+1} \, dx \right| \, dt \\
&\leq \underbrace{4 \kappa_q \int_{t_{q,i}}^{t_{q,i+1} + \overline{t}_q} \left| \int_{\T^3} \nabla (\theta_{\kappa_q, i} - \hat{\theta}_{\kappa_q, i})\cdot \nabla \theta_{\kappa_q, i+1} \, dx \right| \, dt}_{\tilde{A}_{2,1}} + \underbrace{4 \kappa_q \int_{t_{q,i}}^{t_{q,i+1} + \overline{t}_q} \left| \int_{\T^3} \nabla \hat{\theta}_{\kappa_q, i} \cdot \nabla ( \theta_{\kappa_q, i+1} - \hat{\theta}_{\kappa_q, i+1}) \, dx \right| \, dt}_{\tilde{A}_{2,2}} \\
& \qquad + \underbrace{4 \kappa_q \int_{t_{q,i}}^{t_{q,i+1} + \overline{t}_q} \left| \int_{\T^3} \nabla (\hat{\theta}_{\kappa_q, i} - \tilde{\theta}_{\kappa_q, i}) \cdot \nabla \hat{\theta}_{\kappa_q, i+1} \, dx \right| \, dt}_{\tilde{A}_{2,3}} + \underbrace{4 \kappa_q \int_{t_{q,i}}^{t_{q,i+1} + \overline{t}_q} \left| \int_{\T^3} \nabla \tilde{\theta}_{\kappa_q, i} \cdot \nabla (\hat{\theta}_{\kappa_q, i+1} - \tilde{\theta}_{\kappa_q, i+1}) \, dx \right| \, dt}_{\tilde{A}_{2,4}} \\
&+ \underbrace{4 \kappa_q \int_{t_{q,i}}^{t_{q,i+1} + \overline{t}_q} \left| \int_{\T^3} \nabla \tilde{\theta}_{\kappa_q, i} \cdot \nabla \tilde{\theta}_{\kappa_q, i+1} \, dx \right| \, dt}_{\tilde{A}_{2,5}} \\
\end{align*}
We now estimate the quantities $\tilde{A}_{2,1}$, $\tilde{A}_{2,2}$, $\tilde{A}_{2,3}$ and $\tilde{A}_{2,4}$, leaving $\tilde{A}_{2,5}$ to be taken care of at the very end.
\begin{align*}
 \tilde{A}_{2,1} &= 4 \kappa_q \int_{t_{q,i}}^{t_{q,i+1} + \overline{t}_q} \left| \int_{\T^3} \nabla (\theta_{\kappa_q, i} - \hat{\theta}_{\kappa_q, i})\cdot \nabla \theta_{\kappa_q, i+1} \, dx \right| \, dt \\
 &\leq 2 \left( 2 \kappa_q \int_{t_{q,i} + \overline{t}_q}^1 \int_{\T^3} |\nabla (\theta_{\kappa_q, i} - \hat{\theta}_{\kappa_q, i})|^2 \, dx dt \right)^{\sfrac{1}{2}} \left( 2 \kappa_q \int_{t_{q,i+1} + \overline{t}_q}^1 \int_{\T^3} |\nabla \theta_{\kappa_q, i+1}|^2 \, dx dt \right)^{\sfrac{1}{2}}\\
 &\lesssim a_0^{\frac{\eps \delta}{2}} a_q^{\sfrac{\gamma}{2}} a_q^{\sfrac{\gamma}{2}} = a_0^{\frac{\eps \delta}{2}} a_q^{\gamma} \\
 \tilde{A}_{2,2} &= 4 \kappa_q \int_{t_{q,i}}^{t_{q,i+1} + \overline{t}_q} \left| \int_{\T^3} \nabla \hat{\theta}_{\kappa_q, i} \cdot \nabla ( \theta_{\kappa_q, i+1} - \hat{\theta}_{\kappa_q, i+1}) \, dx \right| \, dt \leq \ldots \text{(similar to $\tilde{A}_{2,1}$)} \ldots \lesssim a_0^{\frac{\eps \delta}{2}} a_q^{\gamma} \\
 \tilde{A}_{2,3} &= 4 \kappa_q \int_{t_{q,i}}^{t_{q,i+1} + \overline{t}_q} \left| \int_{\T^3} \nabla (\hat{\theta}_{\kappa_q, i} - \tilde{\theta}_{\kappa_q, i}) \cdot \nabla \hat{\theta}_{\kappa_q, i+1} \, dx \right| \, dt \\
 &\leq 2 \left( 2 \kappa_q \int_{0}^{1} \int_{\T^3} |\nabla (\hat{\theta}_{\kappa_q, i} - \tilde{\theta}_{\kappa_q, i})|^2 \, dx dt \right)^{\sfrac{1}{2}} \left( 2 \kappa_q \int_{0}^{1} \int_{\T^3} |\nabla \hat{\theta}_{\kappa_q, i}|^2 \, dx dt \right)^{\sfrac{1}{2}} \\
 &\lesssim a_q^{\frac{\gamma + \eps}{2}} a_q^{\sfrac{\gamma}{2}} \leq a_q^{\gamma + \sfrac{\eps}{2}} \\
  \tilde{A}_{2,4} &= 4 \kappa_q \int_{t_{q,i}}^{t_{q,i+1} + \overline{t}_q} \left| \int_{\T^3} \nabla \tilde{\theta}_{\kappa_q, i} \cdot \nabla (\hat{\theta}_{\kappa_q, i+1} - \tilde{\theta}_{\kappa_q, i+1}) \, dx \right| \, dt \leq \ldots \text{(similar to $\tilde{A}_{2,3}$)} \ldots \lesssim a_q^{\gamma + \sfrac{\eps}{2}}.
\end{align*}
We now estimate $\tilde{A}_{2,5}$. Recall that by definition of $\tilde{\theta}_{\kappa_q, i}$, we have $\tilde{\theta}_{\kappa_q, i} = f_{\kappa_q, i} \psi_{\kappa_q, i}$.
Hence
\begin{align*}
\tilde{A}_{2.5} &= 4 \kappa_q \int_{t_{q,i}}^{t_{q,i+1} + \overline{t}_q} \left| \int_{\T^3} \nabla \tilde{\theta}_{\kappa_q, i} \cdot \nabla \tilde{\theta}_{\kappa_q, i+1} \, dx \right| \, dt \\
&\leq \underbrace{4 \kappa_q \int_{t_{q,i}}^{t_{q,i+1} + \overline{t}_q} \int_{\T^3} |\nabla f_{\kappa_q, i}| |\nabla f_{\kappa_q, i+1}| |\psi_{\kappa_q, i}| |\psi_{\kappa_q, i+1}| \, dx \, dt}_{\tilde{A}_{2,5,1}}\\
&+ \underbrace{4 \kappa_q \int_{t_{q,i}}^{t_{q,i+1} + \overline{t}_q} \int_{\T^3} |\nabla f_{\kappa_q, i}| |f_{\kappa_q, i+1}| |\psi_{\kappa_q, i}| |\nabla \psi_{\kappa_q, i+1}| \, dx \, dt}_{\tilde{A}_{2,5,2}} \\
&+\underbrace{4 \kappa_q \int_{t_{q,i}}^{t_{q,i+1} + \overline{t}_q} \int_{\T^3} |f_{\kappa_q, i}| |\nabla f_{\kappa_q, i+1}| |\nabla \psi_{\kappa_q, i}| |\psi_{\kappa_q, i+1}| \, dx \, dt}_{\tilde{A}_{2,5,3}} \\
&+\underbrace{4 \kappa_q \int_{t_{q,i}}^{t_{q,i+1} + \overline{t}_q} \int_{\T^3} |f_{\kappa_q, i}| |f_{\kappa_q, i+1}| |\nabla \psi_{\kappa_q, i}| |\nabla \psi_{\kappa_q, i+1}| \, dx \, dt}_{\tilde{A}_{2,5,4}}
\end{align*}
Before estimating each one of the new quantities defined above, we observe the following as an application of Lemma~\ref{lemma:TailEstimates1DAdvectionDiffusion}, we find with $S_i(t)$ the time-dependent interval defined as $S_{q,i}(t) = (z_i - a_0 a_q^{\gamma}, z_i + \sfrac{a_q^{\gamma}}{8} + a_0 a_q^{\gamma})$ we have for all $t \in [0,1]$
\[
 \| \psi_{\kappa_q, i}(t, \cdot) \|_{L^{\infty}(S_{q,i}(t)^c)} \lesssim \exp\left( \frac{(a_0 a_q^{\gamma})}{8 \kappa_q t} \right) \leq \frac{8 \kappa_q t}{a_0^2 a_q^{2 \gamma}} \lesssim \frac{a_q}{a_0^2} \lesssim a_0 a_q^{\gamma}
\]
for $q$ large enough.
Notice that by \eqref{eq:AboutAdjacentSupports} $|S_i(t) \cap S_{i + 1}(t)| \leq 3 a_0 a_q^{\gamma}$ for all $t \in [0,1]$.
Hence for all $t \in [0,1]$
\begin{equation}\label{eq:AboutTwoAdjacentPsis}
\begin{split}
 \int_{\T} |\psi_{\kappa_q, i}| |\psi_{\kappa_q, i+1}|(t,z) \, dz &\leq \int_{S_i(t) \cap S_{i+1}(t)} |\psi_{\kappa_q, i}| |\psi_{\kappa_q, i+1}|(t,z) \, dz + \int_{\T \setminus (S_i(t) \cap S_{i+1}(t))} |\psi_{\kappa_q, i}| |\psi_{\kappa_q, i+1}|(t,z) \, dz \\
 &\lesssim a_0 a_q^{\gamma} + a_0 a_q^{\gamma} \lesssim a_0 a_q^{\gamma}.
\end{split}
\end{equation}
We now estimate each one of the quantities $\tilde{A}_{2,5,1}$, $\tilde{A}_{2,5,2}$, $\tilde{A}_{2,5,3}$ and $\tilde{A}_{2,5,4}$.
\begin{align*}
 \tilde{A}_{2,5,1} &= 4 \kappa_q \int_{t_{q,i}}^{t_{q,i+1} + \overline{t}_q} \int_{\T^3} |\nabla f_{\kappa_q, i}| |\nabla f_{\kappa_q, i+1}| |\psi_{\kappa_q, i}| |\psi_{\kappa_q, i+1}| \, dx \, dt \\
 &\lesssim \kappa_q a_q^{-2} \int_{t_{q,i}}^{t_{q,i+1} + \overline{t}_q} \int_{\T^3} |\psi_{\kappa_q, i}| |\psi_{\kappa_q, i+1}| \, dx \, dt \overset{\eqref{eq:AboutTwoAdjacentPsis}}{\lesssim} \kappa_q a_q^{-2} \overline{t}_q a_0 a_q^{\gamma} \leq a_0^{1 - \eps \delta}a_q^{\gamma}, \\
 \tilde{A}_{2,5,2} &= 4 \kappa_q \int_{t_{q,i}}^{t_{q,i+1} + \overline{t}_q} \int_{\T^3} |\nabla f_{\kappa_q, i}| |f_{\kappa_q, i+1}| |\psi_{\kappa_q, i}| |\nabla \psi_{\kappa_q, i+1}| \, dx \, dt \\
 &\lesssim \kappa_q^{\sfrac{1}{2}} \| \nabla \psi_{\kappa_q, i +1} \|_{L^{\infty}(\T)} \left( 2 \kappa_q \int_{t_{q,i}}^{t_{q,i+1}+\overline{t}_q} |\nabla f_{\kappa_q, i}|^2 \, dx dt \right)^{\sfrac{1}{2}} \\
 &\leq \kappa_q^{\sfrac{1}{2}} \| \nabla \psi_{\kappa_q, i +1} \|_{L^{\infty}(\T)} \lesssim \kappa_q^{\sfrac{1}{2}} a_0^{-2} a_q^{- \gamma} \leq a_q^{\sfrac{1}{2}}, \\
 \tilde{A}_{2,5,3} &= \ldots \text{(similar to $ \tilde{A}_{2,5,2}$)} \ldots \lesssim a_q^{\sfrac{1}{2}}, \\
 \tilde{A}_{2,5,4} &= 4 \kappa_q \int_{t_{q,i}}^{t_{q,i+1} + \overline{t}_q} \int_{\T^3} |f_{\kappa_q, i}| |f_{\kappa_q, i+1}| |\nabla \psi_{\kappa_q, i}| |\nabla \psi_{\kappa_q, i+1}| \, dx \, dt \\
 &\lesssim \kappa_q \int_{t_{q,i}}^{t_{q,i+1} + \overline{t}_q} \int_{\T^3} |\nabla \psi_{\kappa_q, i}| |\nabla \psi_{\kappa_q, i+1}| \, dx \, dt \\
 &\lesssim \kappa_q \overline{t}_q \| \nabla \psi_{\kappa_q, i}(t_{q,i}, \cdot) \|_{L^{\infty}(\T)} \| \nabla \psi_{\kappa_q, i+1}(t_{q,i}, \cdot) \|_{L^{\infty}(\T)} \\
 &\lesssim a_0^{-4} \kappa_q \overline{t}_q a_q^{- 2 \gamma} \lesssim a_q.
\end{align*}
Thus, 
$$\tilde{A}_{2.5} \leq \tilde{A}_{2,5,1} + \tilde{A}_{2,5,2} + \tilde{A}_{2,5,3} + \tilde{A}_{2,5,4} \lesssim a_0^{1 - \eps \delta} a_q^{\gamma}.$$ 
We conclude 
$$\tilde{A}_2 \leq \tilde{A}_{2,1} + \tilde{A}_{2,2} + \tilde{A}_{2,3} + \tilde{A}_{2,4} + \tilde{A}_{2,5} \lesssim a_0^{\frac{\eps \delta}{2}} a_q^{\gamma}.$$
Hence
\[
 \| I \|_{L^1((0,1))} \leq \tilde{A}_1 + \tilde{A}_2 + \tilde{A}_3 \lesssim a_0^{\frac{\eps \delta}{2}} a_q^{\gamma}
\]
and therefore
\end{comment}
As a consequence, we obtain
\begin{equation}\label{eq:EstimateOfQuantityE1}
 \| S_{\kappa_q}^{(1)} \|_{L^1((0,1))} \leq \| I \|_{L^1((0,1))} + \| II \|_{L^1((0,1))} \lesssimlarge a_0^{\frac{\eps \delta}{2}}.
\end{equation}
\textbf{Substep 6.2: }
Note that 
\begin{align*}
 A_q^{(1)}(t) &=  \kappa_q \sum_{i = 1}^{N_q} \int_{\T^3} |\nabla \theta_{\kappa_q, i}(t,x)|^2 \, dx = \underbrace{ \kappa_q \sum_{i = 1}^{N_q} \int_{\T^3} |\nabla \theta_{\kappa_q, i}(t,x)|^2 \, dx \mathbbm{1}_{[t_{q,i}, t_{q,i} + \tilde{t}_q]}(t)}_{= A_{\kappa_q}^{(2)}(t)} \\
 & \quad + \underbrace{ \kappa_q \sum_{i = 1}^{N_q} \int_{\T^3} |\nabla \theta_{\kappa_q, i}(t,x)|^2 \, dx \mathbbm{1}_{[0, t_{q,i})}(t)}_{= S_{\kappa_q}^{(2)}(t)} + \underbrace{ \kappa_q \sum_{i = 1}^{N_q} \int_{\T^3} |\nabla \theta_{\kappa_q, i}(t,x)|^2 \, dx \mathbbm{1}_{(t_{q,i} + \tilde{t}_q, 1]}(t)}_{= S_{\kappa_q}^{(3)}(t)}.
\end{align*}
\begin{lemma}\label{lemma:EstimatesOfTheQuantities-E2-E3}   There exists a universal constant $C>0$  such that for all $ q \in m \N$  we have
 \begin{equation}\label{eq:EstimatesOfTheQuantities-E2-E3}
  \| S_{\kappa_q}^{(2)} \|_{L^1((0,1))} + \| S_{\kappa_q}^{(3)} \|_{L^1((0,1))} \leq C a_0^{\eps \delta}.
 \end{equation}
\end{lemma}


\textbf{Substep 6.3: }
For $q \in m\N$ and all $i = 1, \ldots, N_q$, define $\mathcal{J}_{q,i} = [t_{q,i}, t_{q,i} + \tilde{t}_q]$ and observe that 
\begin{align*}
 A_{\kappa_q}^{(2)}(t) &=  \kappa_q \sum_{i = 1}^{N_q} \int_{\T^3} |\nabla {\theta}_{\kappa_q, i}(t,x)|^2 \, dx \mathbbm{1}_{\mathcal{J}_{q,i}}(t) = \underbrace{ \kappa_q \sum_{i = 1}^{N_q} \int_{\T^3} |\nabla \tilde{\theta}_{\kappa_q, i}(t,x)|^2 \, dx \mathbbm{1}_{\mathcal{J}_{q,i}}(t)}_{= A_{\kappa_q}(t)} \\
 &\quad + \underbrace{ \kappa_q \sum_{i = 1}^{N_q} \int_{\T^3} \nabla {\theta}_{\kappa_q, i}(t,x) \cdot \nabla ({\theta}_{\kappa_q, i} - \tilde{\theta}_{\kappa_q, i})(t,x)\, dx \mathbbm{1}_{\mathcal{J}_{q,i}}(t)}_{= S_{\kappa_q}^{(4)}(t)} \\
 &\quad + \underbrace{ \kappa_q \sum_{i = 1}^{N_q} \int_{\T^3} \nabla ({\theta}_{\kappa_q, i} - \tilde{\theta}_{\kappa_q, i})(t,x) \cdot \nabla \tilde{\theta}_{\kappa_q, i}(t,x)\, dx \mathbbm{1}_{\mathcal{J}_{q,i}}(t)}_{= S_{\kappa_q}^{(5)}(t)} \\
\end{align*}
\begin{lemma}\label{lemma:EstimateOfQuatities-E6-E7}
There exist universal constant $C>0$  such that for all $ q \in m \N$  we have 
 \begin{equation}\label{eq:EstimateOfQuatities-E6-E7}
  \| S_{\kappa_q}^{(4)} \|_{L^1((0,1))} + \| S_{\kappa_q}^{(5)} \|_{L^1((0,1))} \leq C a_0^{\sfrac{\eps \delta }{2}}\,.
 \end{equation}
\end{lemma}

Notice that $S_{\kappa_q} = \sum_{\ell = 1}^{5} S_{\kappa_q}^{(\ell)}$ and therefore in virtue of \eqref{eq:EstimateOfQuantityE1}, \eqref{eq:EstimatesOfTheQuantities-E2-E3},  and \eqref{eq:EstimateOfQuatities-E6-E7}
\begin{equation}\label{eq:L1BoundOnTheError}
\| S_{\kappa_q} \|_{L^1((0,1))} \lesssimlarge a_0^{\frac{\eps \delta}{2}}. 
\end{equation}
This proves the first inequality in \eqref{eq:TheTwoPropertiesThatWeWant}.

\textbf{Substep 6.4: } In this step we prove that  $\| \mu_T \|_{TV} \geq 1/4$ and $\sup_{q} \| A_{\kappa_q} \|_{L^{\infty}((0,1))} < \infty$.
Firstly, we notice that $\mu_T$ is the weak* limit up to subsequence (in the sense of measure) of the non-negative sequence $\{ \kappa_q \int_{\T^3} |\nabla \theta_{\kappa_q} |^2 \}_{q \in m \N }$ that is such that \eqref{eq:total-diss} holds, therefore $\| \mu_T \|_{TV} \geq 1/4$ holds.

We now prove the second property.
Recall that $\tilde{\theta}_{\kappa_q, i}$ solves \eqref{eq:FirstApproxPDEFirstNotNS} and $\mathcal{J}_{q,i} = [t_{q,i}, t_{q,i} + \tilde{t}_q]$. Define the function $\overline{N}_q \colon [0,1] \to \N$ as
\[
 \overline{N}_q(t) = \# \{ j \in \{ 1, \ldots, N_q \} : \mathbbm{1}_{\mathcal{J}_{q,i}}(t) \neq 0 \}
\]
It follows from straightforward computations that $\| \overline{N}_q(t) \|_{L^{\infty}((0,1))} \lesssim a_0^{- \eps \delta} a_q^{\frac{\gamma}{1 + \delta} - 10 \eps - \gamma}$.
For all $t \in \mathcal{J}_{q,i}$, 
\[
\| \nabla \tilde{\theta}_{\kappa_q, i}(t, \cdot) \|_{L^2(\T^3)}^2 \leq \| \nabla \tilde{\theta}_{\kappa_q, i}(t_{q,i}, \cdot) \|_{L^2(\T^3)}^2 \overset{\eqref{eq:LInftyNormOfTheGradientOfTheMollifiedChessFunction}}{\lesssim} a_0^{-6 \eps \delta} a_q^{-2 + \gamma}.
\]
From this we get the $L^\infty$ bound on $A_{\kappa_q}$ independent on $q$:
\begin{align*}
 A_{\kappa_q}(t) &\leq  \kappa_q \sum_{i = 1}^{N_q} \int_{\T^3} |\nabla \tilde{\theta}_{\kappa_q, i}(t,x)|^2 \, dx \mathbbm{1}_{\mathcal{J}_{q,i}} (t) =  \kappa_q \sum_{i = 1}^{N_q} \| \nabla \tilde{\theta}_{\kappa_q, i}(t,\cdot) \|_{L^2(\T^3)}^2 \mathbbm{1}_{\mathcal{J}_{q,i}} (t) \\
 &\leq \kappa_q \overline{N}_q(t) a_0^{-6 \eps \delta} a_q^{-2 + \gamma} \lesssim a_0^{-7 \eps \delta} \quad \forall t \in (0,1)\,,
\end{align*}
which concludes the proof of the second property. Therefore, up to subsequences
\[
  A_{\kappa_q} \overset{*}{\rightharpoonup} \mathcal{A} \in L^{\infty}((0,1)), \quad 
  S_{\kappa_q} \overset{*}{\rightharpoonup} \mathcal{S} \in \mathcal{M}([0,1]) \quad \text{and} \quad - \frac{1}{2}  e_{\kappa_q}^{\prime} \overset{*}{\rightharpoonup} \mu_T \in \mathcal{M}([0,1])  \,,
 \]
 where the first convergence is weakly* in $L^\infty$ and the other two are weakly* in $\mathcal{M}(0,1)$.
 In virtue of \eqref{eq:L1BoundOnTheError}, we find that $\| \mu_T - \mathcal{A} \|_{TV} = \| \mathcal{S} \|_{TV} \lesssim a_0^{\frac{\eps \delta}{2}} < \beta$ where the last inequality holds up to choosing $a_0$ sufficiently small depending on universal constants and $\beta $  in  Subsection \ref{subsec:ParmeterDefinitions}. Thanks to the two properties proved in this step we conclude the proof of Theorem~\ref{t_anomalous_autonomous}.







\begin{comment}
 This proves the last inequality in \eqref{eq:TheTwoPropertiesThatWeWant}. Combining this with \eqref{eq:L1BoundOnTheError} implies that $\sup_{q} \| e_{\kappa_q}^{\prime} \|_{L^{1}((0,1))} < \infty$. Therefore, up to subsequences
 \[
  A_{\kappa_q} \overset{*}{\rightharpoonup} \mathcal{A} \in L^{\infty}((0,1)), \quad 
  S_{\kappa_q} \overset{*}{\rightharpoonup} \mathcal{S} \in \mathcal{M}((0,1)) \quad \text{and} \quad e_{\kappa_q}^{\prime} \overset{*}{\rightharpoonup} \mu \in \mathcal{M}((0,1)) 
 \]
 In virtue of \eqref{eq:L1BoundOnTheError}, we find that $\| \mu - \mathcal{A} \|_{TV} = \| \mathcal{S} \|_{TV} \lesssim a_0^{\frac{\eps \delta}{2}}$. This ends the proof of Theorem~\ref{t_anomalous_autonomous}.
\end{comment}














 

\subsection{Proof of Lemmas~\ref{lemma:EstimateOnTheGradientOfTheSolutionToTheTransportEquation}, \ref{lemma:EstimateOfQuantitiy-II}, \ref{lemma:EstimateOfQuantitiy-I}, \ref{lemma:EstimatesOfTheQuantities-E2-E3},  and \ref{lemma:EstimateOfQuatities-E6-E7}}\label{subsec:ProofOfTheLemmasOfProofA}

\begin{proof}[Proof of Lemma~\ref{lemma:EstimateOnTheGradientOfTheSolutionToTheTransportEquation}]
In order to estimate the right-hand side, we estimate the following three quantities:
\begin{enumerate}
 \item $\displaystyle \int_{0}^{t_{q,j}} \int_{\T^3} |\partial_x {\theta}_{0,j}(t,x,y,z)|^2 \, dx dy dz \, dt$ \label{item:QuantityToEstimate1} \\
 \item $\displaystyle \int_{0}^{t_{q,j}} \int_{\T^3} |\partial_y {\theta}_{0,j}(t,x,y,z)|^2 \, dx dy dz \, dt$ \label{item:QuantityToEstimate2}\\
 \item $\displaystyle \int_{0}^{t_{q,j}} \int_{\T^3} |\partial_z {\theta}_{0,j}(t,x,y,z)|^2 \, dx dy dz \, dt$ 
\label{item:QuantityToEstimate3}\\
\end{enumerate}
\fbox{Estimate of \eqref{item:QuantityToEstimate1}:}
We notice that $|\chi_j (z-t)| = 0$ for any $z \in (z_j +t, z_j + t + \sfrac{a_q^\gamma}{8})^c$, from which we estimate
\begin{align*}
 &\int_{0}^{t_{q,j}} \int_{\T^3} |\partial_x {\theta}_{0,j}(t,x,y,z)|^2 \, dx dy dz \, dt = \int_{0}^{t_{q,j}} \int_{\T} |\varphi_{\initial}(z-t)|^2 |\chi_j(z-t)|^2 \int_{\T^2} |\partial_x \theta_{\stat}(x,y,z)|^2 \, dx dy \, dz \, dt \\ 
 &\leq \| \varphi_{\initial} \|_{L^{\infty}(\T)}^2 \int_{0}^{t_{q,j}} \int_{z_j + t}^{z_j + t + \sfrac{a_q^\gamma}{8}} \int_{\T^2} |\partial_x \theta_{\stat}(x,y,z)|^2 \, dx dy \, dz \, dt \\
  &= \| \varphi_{\initial} \|_{L^{\infty}(\T)}^2 \int_{z_j}^{\sfrac{1}{2} - T_q + \sfrac{a_q^{\gamma}}{4}} \int_{\T^2} \int_{z - z_j - \sfrac{a_q^{\gamma}}{8}}^{z - z_j}  |\partial_x \theta_{\stat}(x,y,z)|^2  \, dt \, dx dy  \, dz \\
  &\lesssim a_q^{\gamma} \int_{0}^{\sfrac{1}{2} - T_q + \sfrac{a_q^{\gamma}}{4}} \int_{\T^2} |\partial_x \theta_{\stat}(x,y,z)|^2 \, dx dy \, dz 
  \end{align*}
  Using \eqref{eq:vartheta_stationary}  we have
  \begin{align*}
  &= a_q^{\gamma} \Bigg[ \int_{0}^{\sfrac{1}{2} - T_0} \underbrace{\int_{\T^2} |\partial_x \theta_{\stat}(x,y,z)|^2 \, dx dy}_{\lesssim a_0^{- 2(1 + 3 \eps (1 + \delta))}} \, dz + \sum_{j = 0}^{q-1} \int_{\sfrac{1}{2} - T_j}^{\sfrac{1}{2} - T_{j+1}} \underbrace{\int_{\T^2} |\partial_x \theta_{\stat}(x,y,z)|^2 \, dx dy}_{\lesssim a_{j+1}^{-2(1 + 3 \eps (1 + \delta))}} \, dz \\
  &\qquad + \int_{\sfrac{1}{2} - T_q}^{\sfrac{1}{2} - T_{q} + a_q^{\gamma }} \underbrace{\int_{\T^2} |\partial_x \theta_{\stat}(x,y,z)|^2 \, dx dy}_{\lesssim a_q^{-2(1 + 3 \eps(1 + \delta))}} \, dz \Bigg] \\
  &\lesssim a_q^{\gamma} \Bigg[ a_0^{- 2(1 + 3 \eps (1 + \delta))} + \sum_{j = 0, j \in m \N }^{q-m}  a_j^{\frac{\gamma}{1 + \delta} - 10 \eps} a_{j+1}^{-2(1 + 3 \eps (1 + \delta))} + \sum_{j = 0, j \notin m \N }^{q-1}  a_j^\gamma a_{j+1}^{-2(1 + 3 \eps (1 + \delta))}  + a_q^{\gamma} a_q^{-2(1 + 3 \eps (1 + \delta))} \Bigg]
  \\
  & \lesssim a_q^{\gamma} \left[ a_0^{- 2(1 + 3 \eps (1 + \delta))} + q a_{q-m}^{\frac{\gamma}{1 + \delta} - 10 \eps - 2(1 + 3 \eps (1 + \delta))(1 + \delta)} +  q a_{q-1}^{\gamma - 2(1 + 3 \eps (1 + \delta))(1 + \delta)} +  a_q^{\gamma -2(1 + 3 \eps (1 + \delta))} \right]\,
  \end{align*}
  where we have used that $q \in m \N$ and $a_{j+1} = a_j^{1+ \delta}$ for any $j$.
  Using  also \eqref{eq:InequalityForTheConstantM}  we get
  $$ a_{q-m +1}^{\frac{\gamma (1- \delta)}{(1+\delta)^2} - \frac{10 \eps}{1 + \delta} - 2 (1+ 3 \varepsilon (1+ \delta))} \leq a_q^{\frac{\gamma}{1+\delta} - 2 (1+ 3 \varepsilon (1+ \delta))},$$
concluding that 
$$ \int_{0}^{t_{q,j}} \int_{\T^3} |\partial_x {\theta}_{0,j}(t,x,y,z)|^2 \, dx dy dz \, dt \lesssim q a_{q}^{\gamma + \frac{\gamma}{1 + \delta} - 2(1 + 3 \eps (1 + \delta))}\,.$$
\fbox{Estimate of \eqref{item:QuantityToEstimate2}:} In the same way of \eqref{item:QuantityToEstimate1}.
\\
\fbox{Estimate of \eqref{item:QuantityToEstimate3}:}
We note that 
\[
 \partial_z {\theta}_{0,j} = \partial_z (\varphi_{\initial}(z-t) \chi_j(z - t)) \theta_{\stat}(x,y,z) + \varphi_{\initial}(z-t) \chi_j(z - t) \partial_z \theta_{\stat}(x,y,z)\,.
\]
Using again \eqref{eq:vartheta_stationary}, \eqref{chi:estimate-C-1} and $\| \varphi_{\initial} \|_{C^1} \lesssim 1$ 
we estimate
$$ \| \partial_z {\theta}_{0,j} \|_{L^\infty} \lesssim a_0^{-1} a_q^{- \gamma} \lesssim  a_{q}^{\gamma + \frac{\gamma}{1 + \delta} - 2(1 + 3 \eps (1 + \delta))} \,,$$
where in the last we used $\gamma = \sfrac{\delta}{8} <\frac{1}{8}$ in the last inequality. 
Finally, from the estimates of \eqref{item:QuantityToEstimate1}, \eqref{item:QuantityToEstimate2} and \eqref{item:QuantityToEstimate3} and \eqref{d:k_q}, we find that
\[
\| {\theta}_{\kappa_q, j}(t) - {\theta}_{0,j}(t) \|_{L^2(\T^3)}^2 \lesssim  q a_q^{2 - \frac{\gamma}{1 + \delta} + 10 \eps + \frac{\gamma}{1 + \delta} + \gamma - 2(1 + 3 \eps (1 + \delta))}  \lesssim a_q^{\gamma + \eps} \quad \text{for all} \quad  t \in [0, t_{q,j}],
\]
where we also used that $q a_q^{\varepsilon} \leq 1$ for any $q \in \N$.
\end{proof}
  











\begin{proof}[Proof of Lemma~\ref{lemma:EstimateOfQuantitiy-II}]
We estimate the terms of $II$ given by
\[
 II_{(i,j)}(t) =  \kappa_q \int_{\T^3} \nabla \theta_{\kappa_q, i}(t,x) \cdot \nabla \theta_{\kappa_q, j}(t,x) \, dx \quad \text{with $i,j \in \{ 1, \ldots, N_q\}$ and $|i - j| > 1$}
\]
in $L^1((0,1))$.
For each $q \geq 1$, let $\{ \overline{\chi}_{j} \}_{j = 1}^{N_q} \subset C^{\infty}_0(\T)$ be a sequence of cutoff functions such that 
\begin{enumerate}
 \myitem{$\overline{\mbox{D1}}$} $0 \leq \overline{\chi}_{j} \leq 1$ for all $j = 1, \ldots, N_q$;
 \myitem{$\overline{\mbox{D2}}$} $\overline{\chi}_{j} = 1$ on $\supp(\chi_{j})$ for all $j = 1, \ldots, N_q$;
 \myitem{$\overline{\mbox{D3}}$} $\| \overline{\chi}_{j} \|_{C^1(\T)} \lesssim a_0^{-1} a_q^{- \gamma}$ for all $j = 1, \ldots, N_q$; \label{item:AboutTheC1NormOfTheSecondCollectionsOfCutoffs}
 \myitem{$\overline{\mbox{D4}}$} we have
 \begin{equation}
  |\supp(\overline{\chi}_{i}) \cap \supp(\overline{\chi}_{j})| \leq 3 a_0 a_q^{\gamma} \quad \forall i,j \in \{ 1, \ldots, N_q\};
 \end{equation}
 and
 \begin{equation}\label{eq:DistanceBetweenTwoNonAdjacentsCutoffsSecondCollection}
  \inf_{|i-j| > 1} \dist ( \supp(\overline{\chi}_{i}), \supp(\overline{\chi}_{j}) ) \geq \frac{a_q^{\gamma}}{16}.
 \end{equation}
\end{enumerate}
For each $j = 1, \ldots, N_q$, let $\overline{\psi}_{\kappa_q, j} \colon [0,1] \times \T \to \R$ be the solution to the advection-diffusion equation
 \begin{equation}\label{eq:1D-Adv-Diff-Cutoffs}
 \left\{
 \begin{array}{ll}
 \partial_t \overline{\psi}_{\kappa_q, j} + \partial_z \overline{\psi}_{\kappa_q, j} = \kappa_q \partial_{zz} \overline{\psi}_{\kappa_q, j}; \\
 \overline{\psi}_{\kappa_q, j}(0, \cdot) = \overline{\chi}_j(\cdot). \\
 \end{array}
 \right.
\end{equation}
Note that due to Lemma~\ref{lemma:TailEstimates1DAdvectionDiffusion} and \eqref{eq:DistanceBetweenTwoNonAdjacentsCutoffsSecondCollection}, whenever we have two integers $i$ and $j$ such that $|i - j| > 1$, it holds that
\begin{equation}\label{eq:TwoNonAdjacentsCutoffsRemainFarAppartSomehow}
 \| \overline{\psi}_{\kappa_q, i} \overline{\psi}_{\kappa_q, j} \|_{L^{\infty}((0,1) \times \T)} \lesssim \kappa_q a_q^{- 2 \gamma}.
\end{equation}
For each $j$, define $\overline{\theta}_{\kappa_q, j}$ as $\overline{\theta}_{\kappa_q, j}(t,x,y,z) = {\theta}_{\kappa_q, j}(t,x,y,z) \overline{\psi}_{\kappa_q, j}(t,z)$. By standard computations, we note that 
 \begin{equation}\label{eq:1D-Adv-Diff-Cutoffs-Prod-With-Solution}
 \left\{
 \begin{array}{ll}
 \partial_t \overline{\theta}_{\kappa_q, j} + u \cdot \nabla \overline{\theta}_{\kappa_q, j} = \kappa_q \Delta \overline{\theta}_{\kappa_q, j} - 2 \kappa_q \nabla {\theta}_{\kappa_q, j} \cdot \nabla \overline{\psi}_{\kappa_q, j}; \\
 \overline{\theta}_{\kappa_q, j}(0, \cdot) = {\theta}_{\kappa_q, j}(0, \cdot). \\
 \end{array}
 \right.
\end{equation}
By energy estimates, we find
\begin{align*}
 &\| ({\theta}_{\kappa_q, j} - \overline{\theta}_{\kappa_q, j})(t, \cdot) \|_{L^2(\T^3)}^2 + 2 \kappa_q \int_0^t \int_{\T^3} |\nabla({\theta}_{\kappa_q, j} - \overline{\theta}_{\kappa_q, j})|^2 \, dx ds \\
 &= 4 \kappa_q \int_0^t \int_{\T^3} (\nabla {\theta}_{\kappa_q, j} \cdot \nabla \overline{\psi}_{\kappa_q, j})({\theta}_{\kappa_q, j} - \overline{\theta}_{\kappa_q, j}) \, dx ds \\
 &\leq 8 \kappa_q \int_0^t \int_{\T^3} | \nabla {\theta}_{\kappa_q, j} \cdot \nabla \overline{\psi}_{\kappa_q, j}| \, dx ds \\
 &\leq 4 \left( 2 \kappa_q \int_0^t \int_{\T^3} |\nabla {\theta}_{\kappa_q, j}|^2 \, dx ds \right)^{\sfrac{1}{2}} \left( 2 \kappa_q \int_0^t \int_{\T^3} |\nabla \overline{\psi}_{\kappa_q, j}|^2 \, dx ds \right)^{\sfrac{1}{2}} \\
 &\lesssim \kappa_q^{\sfrac{1}{2}} \| \theta_{\kappa_q, j}(0, \cdot) \|_{L^2(\T^3)} \| \partial_z \overline{\chi}_{j} \|_{L^{\infty}(\T)} \\
 &\overset{\eqref{item:AboutTheC1NormOfTheSecondCollectionsOfCutoffs}}{\lesssim} \kappa_q^{\sfrac{1}{2}} a_0^{-1} a_q^{- \gamma} a_q^{\sfrac{\gamma}{2}} = \kappa_q^{\sfrac{1}{2}} a_0^{-1} a_q^{- \sfrac{\gamma}{2}} \lesssimlarge a_q^{\sfrac{1}{2}}.
\end{align*}
In particular,
\begin{equation}\label{eq:Useful-Stability-Between-Theta-And-Approximation}
2 \kappa_q \int_0^t \int_{\T^3} |\nabla({\theta}_{\kappa_q, j} - \overline{\theta}_{\kappa_q, j})|^2 \, dx ds \lesssimlarge a_q^{\sfrac{1}{2}}.
\end{equation}
Now, instead of estimating the terms of $II$ in $L^{1}((0,1))$, we will estimate
\[
 \overline{II}_{(i,j)}(t) =  \kappa_q \int_{\T^3} \nabla \overline{\theta}_{\kappa_q, i}(t,x) \cdot \nabla \overline{\theta}_{\kappa_q, j}(t,x) \, dx \quad \text{with $i,j \in \{ 1, \ldots, N_q\}$ and $|i - j| > 1$}
\]
in $L^{1}((0,1))$ for $i$ and $j$ such that $|i - j| > 1$. 
We will then use \eqref{eq:Useful-Stability-Between-Theta-And-Approximation} to show that this is a good approximation of the terms of $II$ in $L^{1}((0,1))$. We observe that
\begin{align*}
\| \overline{II}_{(i,j)} & \|_{L^1((0,1))} = \left\|  \kappa_q \int_{\T^3} \nabla \overline{\theta}_{\kappa_q, i}(\cdot,x) \cdot \nabla \overline{\theta}_{\kappa_q, j}(\cdot,x) \, dx \right\|_{L^1((0,1))} =  \kappa_q \int_0^1 \left| \int_{\T^3} \nabla \overline{\theta}_{\kappa_q, i} \cdot \nabla \overline{\theta}_{\kappa_q, j} \, dx \right| \, dt 
\\
&\leq \underbrace{ \kappa_q \int_0^1 \left| \int_{\T^3} (\nabla \overline{\psi}_{\kappa_q, i} \cdot \nabla \overline{\psi}_{\kappa_q, j}) ({\theta}_{\kappa_q, i} {\theta}_{\kappa_q, j}) \, dx \right| \, dt}_{= B_1} 
 + \underbrace{ \kappa_q \int_0^1 \left| \int_{\T^3} (\nabla \overline{\psi}_{\kappa_q, i} \cdot \nabla \theta_{\kappa_q, j}) ({\theta}_{\kappa_q, i} \overline{\psi}_{\kappa_q, j}) \, dx \right| \, dt}_{= B_2} \\
&\quad + \underbrace{ \kappa_q \int_0^1 \left| \int_{\T^3} (\nabla \theta_{\kappa_q, i} \cdot \nabla \overline{\psi}_{\kappa_q, j}) (\overline{\psi}_{\kappa_q, i} {\theta}_{\kappa_q, j}) \, dx \right| \, dt}_{= B_3} 
+ \underbrace{ \kappa_q \int_0^1 \left| \int_{\T^3} (\nabla \theta_{\kappa_q, i} \cdot \nabla \theta_{\kappa_q, j}) ( \overline{\psi}_{\kappa_q, i} \overline{\psi}_{\kappa_q, j}) \, dx \right| \, dt}_{= B_4} \,.
\end{align*}



Now, we estimate each one of the quantities $B_1$, $B_2$, $B_3$ and $B_4$.
\begin{align} \label{eq:EstimateOfA_1}
B_1 &=  \kappa_q \int_0^1 \left| \int_{\T^3} (\nabla \overline{\psi}_{\kappa_q, i} \cdot \nabla \overline{\psi}_{\kappa_q, j}) ({\theta}_{\kappa_q, i} {\theta}_{\kappa_q, j}) \, dx \right| \, dt \notag 
\\
&\leq  \kappa_q \int_0^1 \| \nabla \overline{\psi}_{\kappa_q, i} (t, \cdot) \|_{L^{\infty}(\T)} \| \nabla \overline{\psi}_{\kappa_q, j} (t, \cdot) \|_{L^{\infty}(\T)} \| \theta_{\kappa_q, i}(t, \cdot) \|_{L^{\infty}(\T^3)} \| \theta_{\kappa_q, j}(t, \cdot) \|_{L^{\infty}(\T^3)} \, dt 
\\
&\lesssim \kappa_q \| \nabla \overline{\psi}_{\kappa_q, i} (0, \cdot) \|_{L^{\infty}(\T)} \| \nabla \overline{\psi}_{\kappa_q, j} (0, \cdot) \|_{L^{\infty}(\T)} \overset{\eqref{item:AboutTheC1NormOfTheSecondCollectionsOfCutoffs}}{\lesssim} \kappa_q a_0^{-2} a_q^{- 2 \gamma} \lesssimlarge a_q^{\sfrac{1}{2}} \,, \notag
\end{align} 
\begin{align} \label{eq:EstimateOfA_2}
B_2 &=  \kappa_q \int_0^1 \left| \int_{\T^3} (\nabla \overline{\psi}_{\kappa_q, i} \cdot \nabla \theta_{\kappa_q, j}) ({\theta}_{\kappa_q, i} \overline{\psi}_{\kappa_q, j}) \, dx \right| \, dt \notag
\\
&\leq \left(  \kappa_q \int_0^1 \int_{\T^3} |\nabla \theta_{\kappa_q, j}|^2 \, dx dt \right)^{\sfrac{1}{2}} \left(  \kappa_q \int_0^1 \int_{\T^3} |\nabla \overline{\psi}_{\kappa_q, i}|^2 \, dx dt \right)^{\sfrac{1}{2}} \| {\theta}_{\kappa_q, i} \|_{L^{\infty}(\T^3)} \| \overline{\psi}_{\kappa_q, j} \|_{L^{\infty}(\T)} \\
&\lesssim \kappa_q^{\sfrac{1}{2}} \| \theta_{\initial, j} \|_{L^2(\T^2)} \| \nabla \overline{\psi}_{\kappa_q, i}(0, \cdot) \|_{L^{\infty}(\T)} \overset{\eqref{item:AboutTheC1NormOfTheSecondCollectionsOfCutoffs}}{\lesssim} \kappa_q^{\sfrac{1}{2}} a_0^{-1} a_q^{- \sfrac{\gamma}{2}} \lesssimlarge a_q^{\sfrac{1}{2}} \,,  \notag
\end{align}
\begin{align} \label{eq:EstimateOfA_4}
B_4 &=  \kappa_q \int_0^1 \left| \int_{\T^3} (\nabla \theta_{\kappa_q, i} \cdot \nabla \theta_{\kappa_q, j}) (\overline{\psi}_{\kappa_q, i} \overline{\psi}_{\kappa_q, j}) \, dx \right| \, dt \notag 
\\
&\leq \left(  \kappa_q \int_0^1 \int_{\T^3} |\nabla \theta_{\kappa_q, i}|^2 \, dx dt \right)^{\sfrac{1}{2}} \left( \kappa_q \int_0^1 \int_{\T^3} |\nabla \theta_{\kappa_q, j}|^2 \, dx dt \right)^{\sfrac{1}{2}} \| \overline{\psi}_{\kappa_q, i} \overline{\psi}_{\kappa_q, j} \|_{L^{\infty}((0,1) \times \T)}
\\
&\leq \| \theta_{\initial, i}(0, \cdot) \|_{L^2(\T^3)} \| \theta_{\initial, j}(0, \cdot) \|_{L^2(\T^3)} \| \overline{\psi}_{\kappa_q, i} \overline{\psi}_{\kappa_q, j} \|_{L^{\infty}((0,1) \times \T)} \overset{\eqref{eq:TwoNonAdjacentsCutoffsRemainFarAppartSomehow}}{\lesssim} a_q^{\gamma} \kappa_q a_q^{- 2 \gamma} \leq a_q \,. \notag 
\end{align}
The upper bound on $B_3$ is the same of $B_2$,
hence, due to \eqref{eq:EstimateOfA_1}, \eqref{eq:EstimateOfA_2} and \eqref{eq:EstimateOfA_4} we find that
\begin{equation}\label{eq:BoundOnTheQuatitityOverlineII-ij}
 \left\| \overline{II}_{(i,j)} \right\|_{L^1((0,1))} \leq B_1 + B_2 + B_3 + B_4 \lesssimlarge a_q^{\sfrac{1}{2}}.
\end{equation}
Now, in order to use this estimate to bound the terms of $II$ in $L^1((0,1))$, we will estimate $\| {II}_{(i,j)} - \overline{II}_{(i,j)} \|_{L^1((0,1))}$.
With this goal in mind, we first observe that
\begin{equation}\label{eq:EstimateOfTheAnomalousDissipationForThetaBar}
 \left(  \kappa_{q} \int_0^1 \int_{\T^3} \left| \nabla \overline{\theta}_{\kappa_q, j} \right|^2 \, dx \, dt \right)^{\sfrac{1}{2}} \overset{\eqref{eq:Useful-Stability-Between-Theta-And-Approximation}}{\lesssimlarge} a_q^{\sfrac{1}{4}} + \left(  \kappa_{q} \int_0^1 \int_{\T^3} \left| \nabla {\theta}_{\kappa_q, j} \right|^2 \, dx \, dt \right)^{\sfrac{1}{2}} \lesssimlarge a_q^{\sfrac{1}{4}} + a_q^{\sfrac{\gamma}{2}} \lesssimlarge a_q^{\sfrac{\gamma}{2}}. 
\end{equation}
Thus,
\begin{align*}
\| {II}_{(i,j)} & - \overline{II}_{(i,j)} \|_{L^1((0,1))} =  \kappa_q \int_0^1 \left| \int_{\T^3} \nabla {\theta}_{\kappa_q, i} \cdot \nabla {\theta}_{\kappa_q, j} - \nabla \overline{\theta}_{\kappa_q, i} \cdot \nabla \overline{\theta}_{\kappa_q, j} \, dx \right| \, dt \\
&\leq  \kappa_q \int_0^1 \int_{\T^3} \left| \nabla {\theta}_{\kappa_q, i} \cdot \nabla ({\theta}_{\kappa_q, j} - \overline{\theta}_{\kappa_q, j}) \right| \, dx \, dt
+  \kappa_q \int_0^1 \int_{\T^3} \left| \nabla ({\theta}_{\kappa_q, i} - \overline{\theta}_{\kappa_q, i}) \cdot \nabla \overline{\theta}_{\kappa_q, j} \right| \, dx \, dt
 \\
&\leq \max_{i,j} \left(  \kappa_{q} \int_0^1 \int_{\T^3} \left| \nabla {\theta}_{\kappa_q, i} \right|^2 + \left| \nabla {\overline \theta}_{\kappa_q, i} \right|^2 \, dx \, dt \right)^{\sfrac{1}{2}} \left(  \kappa_{q} \int_0^1 \int_{\T^3} \left| \nabla ({\theta}_{\kappa_q, j} - \overline{\theta}_{\kappa_q, j}) \right|^2 \, dx \, dt \right)^{\sfrac{1}{2}}  
\\
&\overset{\eqref{eq:Useful-Stability-Between-Theta-And-Approximation}, \eqref{eq:EstimateOfTheAnomalousDissipationForThetaBar}}{\lesssimlarge} a_q^{\sfrac{\gamma}{2}} a_q^{\sfrac{1}{4}} + a_q^{\sfrac{\gamma}{2}} a_q^{\sfrac{1}{4}} \lesssimlarge a_q^{\sfrac{1}{4} + \sfrac{\gamma}{2}}.
\end{align*}
Hence
\[
 \left\| {II}_{(i,j)} - \overline{II}_{(i,j)} \right\|_{L^1((0,1))} \lesssimlarge a_q^{\sfrac{1}{4}}
\]
and thus, combined with \eqref{eq:BoundOnTheQuatitityOverlineII-ij}, we find
\begin{equation}\label{eq:BoundOnL1NormOfII(i,j)}
 \left\| {II}_{(i,j)} \right\|_{L^1((0,1))} \lesssimlarge a_q^{\sfrac{1}{4}}.
\end{equation}
For this last inequality, we deduce that
\begin{equation}
 \left\| II \right\|_{L^1((0,1))} \leq \sum_{|i - j| > 1} \left\| {II}_{(i,j)} \right\|_{L^1((0,1))} \overset{\eqref{eq:BoundOnL1NormOfII(i,j)}}{\lesssimlarge} N_q^2 a_q^{\sfrac{1}{4}} \overset{\eqref{item:BoundOnTheNumberOfCutoffs}}{\lesssimlarge} a_q^{\sfrac{1}{4} - 2\gamma} \lesssimlarge a_q^{\sfrac{1}{8}}.
\end{equation}
This ends the proof of Lemma~\ref{lemma:EstimateOfQuantitiy-II}.
\end{proof}

\begin{proof}[Proof of Lemma~\ref{lemma:EstimateOfQuantitiy-I}]
We estimate the terms of $I$ given by
\[
 I (t ) = \sum_{i=1}^{N_q} I_{i}(t) = \sum_{i=1}^{N_q} 2 \kappa_q \int_{\T^3} \nabla \theta_{\kappa_q, i}(t,x) \cdot \nabla \theta_{\kappa_q, i+1}(t,x) \, dx
\]
in $L^1((0,1))$.
Then
\begin{align*}
 \left\| I_i \right\|_{L^1((0,1))} &= \underbrace{ \| I_i(\cdot ) \mathbbm{1}_{[0, t_{q,i}]}(\cdot ) \|_{L^1}}_{= \tilde{B}_1}
 + \underbrace{ \| I_i  (\cdot ) \mathbbm{1}_{[t_{q,i}, t_{q,i+1} + \tilde{t}_q]}(\cdot ) \|_{L^1}}_{= \tilde{B}_2} + 
 + \underbrace{ \| I_i  (\cdot ) \mathbbm{1}_{[t_{q,i+1} + \tilde{t}_q,1]}(\cdot ) \|_{L^1}}_{= \tilde{B}_3}
\end{align*}
Now we estimate the quantities $\tilde{B}_1$ and $\tilde{B}_3$ which we expect to be very small compared to $a_q^{\gamma}$. With that goal in mind, we first observe that
\begin{equation}\label{eq:AboutDissipationBeforeCriticalTime}
\begin{split}
 2 \kappa_q \int_0^{t_{q,i}} \int_{\T^3} |\nabla \theta_{\kappa_q, i}|^2 \, dx dt &= \| \theta_{\initial, i} \|_{L^2(\T^3)}^2 - \| \theta_{\kappa_q, i}(t_{q,i}, \cdot) \|_{L^2(\T^3)}^2 \\
   &= \left( \| \theta_{\initial, i} \|_{L^2(\T^3)} + \| \theta_{\kappa_q, i}(t_{q,i}, \cdot) \|_{L^2(\T^3)} \right) \left( \| \theta_{\initial, i} \|_{L^2(\T^3)} - \| \theta_{\kappa_q, i}(t_{q,i}, \cdot) \|_{L^2(\T^3)} \right)\\
   &= \left( \| \theta_{\initial, i} \|_{L^2(\T^3)} + \| \theta_{\kappa_q, i}(t_{q,i}, \cdot) \|_{L^2(\T^3)} \right) \left( \| \theta_{0, i}(t_{q,i}, \cdot) \|_{L^2(\T^3)} - \| \theta_{\kappa_q, i}(t_{q,i}, \cdot) \|_{L^2(\T^3)} \right)\\
     &\leq \left( \| \theta_{\initial, i} \|_{L^2(\T^3)} + \| \theta_{\kappa_q, i}(t_{q,i}, \cdot) \|_{L^2(\T^3)} \right) \| (\theta_{0, i} - \theta_{\kappa_q, i})(t_{q,i}, \cdot) \|_{L^2(\T^3)} \overset{\eqref{stability:theta_kq-theta}}{\lesssim} a_q^{\sfrac{\gamma}{2}} a_q^{\frac{\gamma + \eps}{2}}. \\
\end{split}
\end{equation}
It then follows that
\begin{align*}
 \tilde{B}_1 &= 2 \kappa_q \int_0^{t_{q,i}} \left| \int_{\T^3} \nabla \theta_{\kappa_q, i}(t,x) \cdot \nabla \theta_{\kappa_q, i+1}(t,x) \, dx \right| \, dt \\
  &\leq  \left( 2 \kappa_q \int_0^{t_{q,i}} \int_{\T^3} |\nabla \theta_{\kappa_q, i}|^2 \, dx dt \right)^{\sfrac{1}{2}} \left( 2 \kappa_q \int_0^{t_{q,i}} \int_{\T^3} |\nabla \theta_{\kappa_q, i+1}|^2 \, dx dt \right)^{\sfrac{1}{2}}\\
  &\overset{\eqref{eq:AboutDissipationBeforeCriticalTime}}{\lesssim} a_q^{\sfrac{\gamma}{4}} a_q^{\frac{\gamma + \eps}{4}} a_q^{\sfrac{\gamma}{2}} = a_q^{\gamma + \sfrac{\eps}{4}};\\
  \tilde{B}_3 &= 2 \kappa_q \int_{t_{q,i+1} + \tilde{t}_q}^1 \left| \int_{\T^3} \nabla \theta_{\kappa_q, i}(t,x) \cdot \nabla \theta_{\kappa_q, i+1}(t,x) \, dx \right| \, dt \\
  &\leq  \left( 2 \kappa_q \int_{t_{q,i+1} + \tilde{t}_q}^1 \int_{\T^3} |\nabla \theta_{\kappa_q, i}|^2 \, dx dt \right)^{\sfrac{1}{2}} \left( 2 \kappa_q \int_{t_{q,i+1} + \tilde{t}_q}^1 \int_{\T^3} |\nabla \theta_{\kappa_q, i+1}|^2 \, dx dt \right)^{\sfrac{1}{2}}
  \\
  &\leq  \| \theta_{\initial, i} \|_{L^2(\T^3)} \| \theta_{\kappa_q, i+1} (t_{q,i+1} + \tilde{t}_q, \cdot) \|_{L^2(\T^3)} \overset{\eqref{eq:AutonomousNormalGoalOfStep3}}{\lesssim} a_0^{\frac{\eps \delta}{2}} a_q^{\gamma}.
\end{align*}
To estimate $\tilde{B}_2$, we proceed by a sequence of approximations. We will reuse the functions $\tilde{\theta}_{\kappa_q, j}$,  $f_{\kappa_q, j}$ and $\psi_{\kappa_q, j}$ introduced in Step 3 of the proof (see \eqref{eq:FirstApproxPDEFirstNotNS}, \eqref{eq:2DHeatEquation} and \eqref{eq:1DAdvectionDiffusionEquationInStep3}). We note that
\begin{align*}
\tilde{B}_2 &= 2 \kappa_q \int_{t_{q,i}}^{t_{q,i+1} + \tilde{t}_q} \left| \int_{\T^3} \nabla \theta_{\kappa_q, i} \cdot \nabla \theta_{\kappa_q, i+1} \, dx \right| \, dt 
\leq 2 \kappa_q \int_{t_{q,i}}^{t_{q,i+1} + \tilde{t}_q} \left| \int_{\T^3} \nabla (\theta_{\kappa_q, i} - \tilde{\theta}_{\kappa_q, i})\cdot \nabla \theta_{\kappa_q, i+1} \, dx \right| \, dt 
\\
& \quad + 2 \kappa_q \int_{t_{q,i}}^{t_{q,i+1} + \tilde{t}_q} \left| \int_{\T^3} \nabla \tilde{\theta}_{\kappa_q, i} \cdot \nabla ( \theta_{\kappa_q, i+1} - \tilde{\theta}_{\kappa_q, i+1}) \, dx \right| \, dt   + 2 \kappa_q \int_{t_{q,i}}^{t_{q,i+1} + \tilde{t}_q} \left| \int_{\T^3} \nabla \tilde{\theta}_{\kappa_q, i} \cdot \nabla \tilde{\theta}_{\kappa_q, i+1} \, dx \right| \, dt \,.
\end{align*}
We now estimate the 3 summands. The first term is estimated observing that \eqref{e:energy-equality-global} implies
\[
 2 \kappa_q \int_{0}^{1} \int_{\T^3} |\nabla \theta_{\kappa_q, i+1}|^2 \, dx dt \leq \| \theta_{\initial, i+1} \|_{L^2(\T^3)}^2 \lesssim a_q^\gamma
\]
for all $i$ and therefore
\begin{align*}
  & 2 \kappa_q \int_{t_{q,i}}^{t_{q,i+1} + \tilde{t}_q} \left| \int_{\T^3} \nabla (\theta_{\kappa_q, i} - \tilde{\theta}_{\kappa_q, i})\cdot \nabla \theta_{\kappa_q, i+1} \, dx \right| \, dt \\
 &\leq  \left( 2 \kappa_q \int_{t_{q,i}}^{t_{q,i +1} + \tilde{t}_q} \int_{\T^3} |\nabla (\theta_{\kappa_q, i} - \tilde{\theta}_{\kappa_q, i})|^2 \, dx dt \right)^{\sfrac{1}{2}} \left( 2 \kappa_q \int_{0}^{1} \int_{\T^3} |\nabla \theta_{\kappa_q, i+1}|^2 \, dx dt \right)^{\sfrac{1}{2}} \overset{\eqref{eq:TildeApproxOfHatL2}}{\lesssimlarge} a_0^{\frac{\eps \delta}{2}} a_q^{\sfrac{\gamma}{2}} a_q^{\sfrac{\gamma}{2}} \,.
\end{align*} 
 where the last holds for $q$ large enough in order to have $ t_{q, i+1}  + \tilde{t}_q \leq  t_{q, i} + \sfrac{\overline{t}_q}{2}$ (see Section \ref{subsec:ParmeterDefinitions} and \eqref{d:time_t_j}).
Similarly, we can estimate the second term.
Finally, recalling that $\tilde{\theta}_{\kappa_q, i}$, we have $\tilde{\theta}_{\kappa_q, i} = f_{\kappa_q, i} \psi_{\kappa_q, i}$, we estimate the last summand as follows 
\begin{align*}
 2 \kappa_q \int_{t_{q,i}}^{t_{q,i+1} + \tilde{t}_q} \left| \int_{\T^3} \nabla \tilde{\theta}_{\kappa_q, i} \cdot \nabla \tilde{\theta}_{\kappa_q, i+1} \, dx \right| \, dt 
& \leq \underbrace{2 \kappa_q \int_{t_{q,i}}^{t_{q,i+1} + \tilde{t}_q} \int_{\T^3} |\nabla f_{\kappa_q, i}| |\nabla f_{\kappa_q, i+1}| |\psi_{\kappa_q, i}| |\psi_{\kappa_q, i+1}| \, dx \, dt}_{= \tilde{B}_{2,1}}\\
&+ \underbrace{2 \kappa_q \int_{t_{q,i}}^{t_{q,i+1} + \tilde{t}_q} \int_{\T^3} |\nabla f_{\kappa_q, i}| |f_{\kappa_q, i+1}| |\psi_{\kappa_q, i}| |\nabla \psi_{\kappa_q, i+1}| \, dx \, dt}_{= \tilde{B}_{2,2}} \\
&+\underbrace{2 \kappa_q \int_{t_{q,i}}^{t_{q,i+1} + \tilde{t}_q} \int_{\T^3} |f_{\kappa_q, i}| |\nabla f_{\kappa_q, i+1}| |\nabla \psi_{\kappa_q, i}| |\psi_{\kappa_q, i+1}| \, dx \, dt}_{= \tilde{B}_{2,3}} \\
&+\underbrace{2 \kappa_q \int_{t_{q,i}}^{t_{q,i+1} + \tilde{t}_q} \int_{\T^3} |f_{\kappa_q, i}| |f_{\kappa_q, i+1}| |\nabla \psi_{\kappa_q, i}| |\nabla \psi_{\kappa_q, i+1}| \, dx \, dt}_{= \tilde{B}_{2,4}}.
\end{align*}
Before estimating each one of the new quantities defined above, we make the following observation. Let $S_{q,i}(t)$ be the time-dependent interval defined by $S_{q,i}(t) = (z_i + t - a_0 a_q^{\gamma}, z_i + t + \sfrac{a_q^{\gamma}}{8} + a_0 a_q^{\gamma})$. By Lemma~\ref{lemma:TailEstimates1DAdvectionDiffusion}, we find that for all $t \in [0,1]$
\[
 \| \psi_{\kappa_q, i}(t, \cdot) \|_{L^{\infty}(S_{q,i}(t)^c)} \lesssim \exp\left( \frac{(a_0 a_q^{\gamma})^2}{8 \kappa_q t} \right) \leq \frac{8 \kappa_q t}{a_0^2 a_q^{2 \gamma}} \lesssim \frac{a_q}{a_0^2} \lesssimlarge a_0 a_q^{\gamma}.
\]
Notice that by \eqref{eq:AboutAdjacentSupports} $|S_{q,i}(t) \cap S_{q, i + 1}(t)| \leq 3 a_0 a_q^{\gamma}$ for all $t \in [0,1]$.
Hence for all $t \in [0,1]$
\begin{equation}\label{eq:AboutTwoAdjacentPsis}
\begin{split}
 \int_{\T} |\psi_{\kappa_q, i}| |\psi_{\kappa_q, i+1}|(t,z) \, dz &\leq \int_{S_{q,i}(t) \cap S_{q, i+1}(t)} |\psi_{\kappa_q, i}| |\psi_{\kappa_q, i+1}|(t,z) \, dz 
 \\
  & \quad + \int_{\T \setminus (S_{q,i}(t) \cap S_{q, i+1}(t))} |\psi_{\kappa_q, i}| |\psi_{\kappa_q, i+1}|(t,z) \, dz \lesssimlarge a_0 a_q^{\gamma} + a_0 a_q^{\gamma} \,.
\end{split}
\end{equation}
First, we estimate 
\begin{align*}
    \tilde{B}_{2,1} &= 2 \kappa_q \int_{t_{q,i}}^{t_{q,i+1} + \tilde{t}_q} \int_{\T^3} |\nabla f_{\kappa_q, i}| |\nabla f_{\kappa_q, i+1}| |\psi_{\kappa_q, i}| |\psi_{\kappa_q, i+1}| \, dx \, dt \\
 &\overset{\eqref{eq:LInftyNormOfTheGradientOfTheMollifiedChessFunction}}{\lesssim} \kappa_q a_0^{-6 \eps \delta} a_q^{-2} \int_{t_{q,i}}^{t_{q,i+1} + \tilde{t}_q} \int_{\T^3} |\psi_{\kappa_q, i}| |\psi_{\kappa_q, i+1}| \, dx \, dt \overset{\eqref{eq:AboutTwoAdjacentPsis}}{\lesssimlarge} \kappa_q a_0^{-6 \eps \delta} a_q^{-2} \tilde{t}_q a_0 a_q^{\gamma} \lesssimlarge a_0^{1 - 7 \eps \delta} a_q^{\gamma}\,.
\end{align*}
Finally, using that $\| \nabla f_{\kappa_q, i} \|_{L^\infty(\T^3)} \overset{\eqref{eq:LInftyNormOfTheGradientOfTheMollifiedChessFunction}}{\lesssim} a_0^{-3 \eps \delta} a_q^{-1}$ and $\| \nabla \psi_{\kappa_q, i} \|_{L^\infty(\T)} \lesssim a_0^{-1} a_q^{- \gamma}$ we have
\begin{align*} 
\tilde{B}_{2,2}  + \tilde{B}_{2,3} + \tilde{B}_{2,4} \lesssim \kappa_q a_0^{- (1 + 3 \eps \delta)} a_q^{- (1 + \gamma)} + \kappa_q a_0^{- 2} a_q^{- 2 \gamma}\overset{\eqref{d:k_q} }{\lesssimlarge} a_q^{1/2} \,.
\end{align*}

Hence,
\[
 \| I_i \|_{L^1((0,1))} \leq \tilde{B}_1 + \tilde{B}_2 + \tilde{B}_3 \lesssimlarge a_0^{\frac{\eps \delta}{2}} a_q^{\gamma}
\]
and therefore 
\[
 \| I \|_{L^1((0,1))} \leq \sum_{i = 1}^{N_q} \| I_i \|_{L^1((0,1))} \overset{\eqref{item:BoundOnTheNumberOfCutoffs}}{\lesssimlarge} a_0^{\frac{\eps \delta}{2}}\,.
\] 
\end{proof} 

\begin{proof}[Proof of Lemma~\ref{lemma:EstimatesOfTheQuantities-E2-E3}]
By \eqref{eq:AboutDissipationBeforeCriticalTime} in the proof of Lemma~\ref{lemma:EstimateOfQuantitiy-I}
\begin{align*}
 \| S_{\kappa_q}^{(2)} \|_{L^1((0,1))} &\leq  \kappa_q \sum_{i = 1}^{N_q} \int_0^{t_{q,i}} \int_{\T^3} |\nabla \theta_{\kappa_q, i}|^2 \, dx \, dt \overset{\eqref{eq:AboutDissipationBeforeCriticalTime}}{\lesssim} N_q a_q^{\gamma + \sfrac{\eps}{2}} \lesssim a_q^{\sfrac{\eps}{2}}.
\end{align*}
By \eqref{eq:AutonomousNormalGoalOfStep3},
\begin{align*}
\| S_{\kappa_q}^{(3)} \|_{L^1((0,1))} &\leq  \kappa_q \sum_{i = 1}^{N_q} \int_{t_{q,i} + \tilde{t}_q}^1 \int_{\T^3} |\nabla \theta_{\kappa_q, i}(t,x)|^2 \, dx \, dt \\
&\leq \sum_{i = 1}^{N_q} \| \theta_{\kappa_q,i}(t_{q,i} + \tilde{t}_q, \cdot) \|_{L^2(\T^3)}^2 \overset{\eqref{eq:AutonomousNormalGoalOfStep3}}{\lesssim} N_q a_0^{\eps \delta} a_q^{\gamma} \lesssim a_0^{\eps \delta} \,.
\end{align*}
\end{proof}


\begin{proof}[Proof of Lemma~\ref{lemma:EstimateOfQuatities-E6-E7}]
Recall that $\mathcal{J}_{q,i} = [t_{q,i}, t_{q,i} + \tilde{t}_q]$ and note that due to \eqref{e:energy-equality-global}, we have
\[
 2 \kappa_q  \int_{\mathcal{J}_{q,i}} \int_{\T^3} |\nabla {\theta}_{\kappa_q, i}|^2 \, dx \, dt \leq \|\theta_{\kappa_q , i} (0, \cdot ) \|_{L^2 (\T^3)}^2  \lesssim a_q^{\gamma}.
\]
Hence
\begin{align*}
\| S_{\kappa_q}^{(4)} \|_{L^1((0,1))} &\leq  \kappa_q \sum_{i = 1}^{N_q} \int_{\mathcal{J}_{q,i}} \left| \int_{\T^3} \nabla {\theta}_{\kappa_q, i}(t,x) \cdot \nabla ({\theta}_{\kappa_q, i} - \tilde{\theta}_{\kappa_q, i})(t,x)\, dx \right| \, dt \\
&\leq \sum_{i = 1}^{N_q} \left(  \kappa_q  \int_{\mathcal{J}_{q,i}} \int_{\T^3} |\nabla {\theta}_{\kappa_q, i}|^2 \, dx \, dt \right)^{\sfrac{1}{2}} \left(  \kappa_q \int_{\mathcal{J}_{q,i}} \int_{\T^3} |\nabla ({\theta}_{\kappa_q, i} - \tilde{\theta}_{\kappa_q, i})|^2 \, dx \, dt \right)^{\sfrac{1}{2}} \\
&\overset{\eqref{eq:TildeApproxOfHatL2}}{\lesssim}  N_q a_q^{\sfrac{\gamma}{2}} a_q^{\sfrac{\gamma}{2}} a_0^{\varepsilon \delta /2}  \overset{\eqref{item:BoundOnTheNumberOfCutoffs}}{\lesssim} a_0^{\sfrac{\eps \delta}{2}}.
\end{align*}
Similarly $\| S_{\kappa_q}^{(5)} \|_{L^1((0,1))} \lesssim a_0^{\sfrac{\eps \delta}{2}}$.
\end{proof}

%In addition, we have
%\[
% \partial_t (\theta_{\kappa_q, i} \theta_{\kappa_q, j}) + (u \cdot \nabla \theta_{\kappa_q, i}) \theta_{\kappa_q, j} + (u \cdot \nabla \theta_{\kappa_q, j}) \theta_{\kappa_q, i} = \kappa_q (\theta_{\kappa_q, i} \Delta \theta_{\kappa_q, j} + \theta_{\kappa_q, j} \Delta \theta_{\kappa_q, i})
%\]
%By energy estimates,
%\[
% \int_{\T^3} (\theta_{\kappa_q, i} \theta_{\kappa_q, j})(t,x) \, dx - \int_{\T^3} (\theta_{\kappa_q, i} \theta_{\kappa_q, j})(0,x) \, dx = - 2 \kappa_q \int_0^t \int_{\T^3} \nabla \theta_{\kappa_q, i}(s,x) \cdot \nabla \theta_{\kappa_q, j}(s,x) \, dx \, ds.
%\]
%for all $i \neq j$. Now we estimate right-hand side in the above equation by estimating the two terms on the left-hand side. First consider the case where $|i - j| = 1$. For all $i$, define the time-dependent intervals
%\[
% A_i(t) = \left( z_i + t - a_0^{\eps \delta} a_q^{\gamma}, z_i + t + \frac{a_q^{\gamma}}{8} + a_0^{\eps \delta} a_q^{\gamma} \right).
%\]
%By Lemma ???, for all $i$
%\[
% \| \theta_{\kappa_q, i} \|_{L^{\infty}(\T^2 \times A_i(t)^c)} \leq 2 \| \theta_{\initial, i} \|_{L^{\infty}(\T^3)} \exp \left( - \frac{(a_0^{\eps \delta} a_q^{\gamma})^2}{8 \kappa_q t} \right) \leq \frac{80 \kappa_q}{a_0^{2 \eps \delta} a_q^{2 \gamma}}
%\]
%for all $t \in [0,1]$. \textcolor{red}{(EXTRA ASSUMPTION $|\supp \chi_i \cap \supp \chi_j| \leq a_0^{\eps \delta} a_q^{\gamma}$)} Notice that $|A_{i}(t) \cap A_{j}(t)| \leq 3 a_0^{\eps \delta} a_q^{\gamma}$.
%For all $t \in [0,1]$, we have
%\begin{align*}
% \int_{\T^3} (\theta_{\kappa_q, i} \theta_{\kappa_q, j})(t,x) \, dx &= \underbrace{\int_{\T^2 \times (\T \setminus (A_i(t) \cup A_j(t)))} (\theta_{\kappa_q, i} \theta_{\kappa_q, j})(t,x) \, dx}_{= I} + \underbrace{\int_{\T^2 \times (A_i(t) \setminus A_j(t))} (\theta_{\kappa_q, i} \theta_{\kappa_q, j})(t,x) \, dx}_{= II} \\
% &+  \underbrace{\int_{\T^2 \times (A_j(t) \setminus A_i(t))} (\theta_{\kappa_q, i} \theta_{\kappa_q, j})(t,x) \, dx}_{= III} + \underbrace{\int_{\T^2 \times (A_i(t) \cap A_j(t))} (\theta_{\kappa_q, i} \theta_{\kappa_q, j})(t,x) \, dx}_{= IV}
%\end{align*}
%Using (TOREF), we find
%\begin{align*}
%|I| \leq \frac{6400 \kappa_q^2}{a_0^{4 \eps \delta} a_q^{4 \gamma}}, \quad |II|, |III| \leq \frac{100 \kappa_q}{a_0^{2 \eps \delta} a_q^{\gamma}}, \quad |IV| \leq 75 a_0^{\eps \delta} a_q^{\gamma}.
%\end{align*}
%Thus, for $q$ large enough
%\[
% \left| \int_{\T^3} (\theta_{\kappa_q, i} \theta_{\kappa_q, j})(t,x) \, dx \right| \leq 100 a_0^{\eps \delta} a_q^{\gamma}.
%\]
%as wished.
%Secondly, consider the case where $|i - j| > 1$. We follow the same approach as in the previous case. The only difference is that $IV = 0$ and hence for $q$ large enough
%\[
% \left| \int_{\T^3} (\theta_{\kappa_q, i} \theta_{\kappa_q, j})(t,x) \, dx \right| \leq \frac{300 \kappa_q}{a_0^{2 \eps \delta} a_q^{\gamma}}.
%\]
%\begin{align*}
% \| E_{\kappa_q}^{(1)} \|_{L^{1}((0,1))} \leq 200 N_q a_0^{\eps \delta} a_q^{\gamma} + \frac{300 \kappa_q}{a_0^{2 \eps \delta} a_q^{\gamma}} N_q^2 \leq 500 a_0^{\eps \delta}
%\end{align*}
%%and thanks to the previous steps it is a non-increasing function such that $e(1) < e(0)$. {\color{red} Since $\theta_\kappa \overset{*}{\rightharpoonup} \theta_0$ as $\kappa \to 0$, we also have that $e_{\kappa} \overset{*}{\rightharpoonup} e$ in $L^{\infty}((0,1))$} where $e_{\kappa} \colon [0, 1] \to \R$ is defined by
%%\[
%% e_{\kappa}(t) = \int_{\T^3} |\theta_{\kappa}(t,x)|^2 \, dx
%%\]
%%By Lemma~\ref{lemma:WeakStarConvergenceOnDecreasingFunctions}, $e_{\kappa} \to e$ in $L^{\infty}_{loc}(0,1)$ and hence $e_{\kappa} \to e$ in $L^{2}(0,1)$ (because $e_\kappa$ and $e$ are bounded functions). 
%%This implies that
%%\begin{align*}
%% \lim_{\kappa \to 0} \| \theta_{\kappa} \|_{L^2((0,1) \times \T^3)}^2 = \lim_{\kappa \to 0} \int_0^1 e_{\kappa}(t) \, dt = \int_0^1 e(t) \, dt = \| \theta_{0} \|_{L^2((0,1) \times \T^3)}^2.
%%\end{align*}
%%Since $\theta_{\kappa} \overset{*}{\rightharpoonup} \theta_0$ as $\kappa \to 0$ in the weak$^{\ast}$ $L^{\infty}((0,1) \times \T^3)$ topology, it follows that $\theta_{\kappa} \rightharpoonup \theta_0$ in $L^2((0,1) \times \T^3)$. Combining this with the fact that $\lim_{\kappa \to 0} \| \theta_{\kappa} \|_{L^2((0,1) \times \T^3)} = \| \theta_{0} \|_{L^2((0,1) \times \T^3)}$, we find  that $\theta_{\kappa} \to \theta_0$ in $L^2((0,1) \times \T^3)$ as $\kappa \to 0$.
%
\iffalse

{\color{red}
\section{Proof of Corollary}

Recall some definitions.

\begin{maintheorem}
 Let  $\alpha\in [0,1)$ and $\beta >0$, then there exists an autonomous divergence-free velocity field $u \in L^\infty (\T^3) \cap W^{\alpha, 1} (\T^3)$ and an initial datum $\theta_{\initial} \in L^\infty(\T^3)$ with $ \int_{\T^3} \theta_{\initial}=0$, $\| \theta_{\initial} \|_{L^2} =1$ such that the unique solutions $\theta_\kappa$ of the advection-diffusion equation \eqref{e:ADV-DIFF} with initial datum $\theta_{\initial}$  exhibit anomalous  dissipation \eqref{diss_main_OC}
and up to non-relabelled subsequences
\begin{equation} \label{eq:absolutely_continuous:corollary}
    \mathcal{D}_T [\theta_\kappa] \rightharpoonup  \mathcal{D}_T [\theta_0] = - \frac{1}{2} \frac{d}{dt} \int_{\T^3}  | \theta_0 (\cdot ,x) |^2 dx \in \mathcal{M}([0,1])
\end{equation}
weakly* in the sense of measure, where $\theta_0$ is the unique solution of \eqref{e:ADV-DIFF} with $\kappa =0$, velocity field $u$ and initial datum $\theta_{\initial}$; and $\mathcal{D}_T [\theta_0]$ is absolutely continuous with respect to Lebsegue.


Furthermore, up to not relabelled subsequences, we have 
$$\theta_\kappa \rightarrow  \theta_0  \qquad \text{ in } L^p ((0,1) \times \T^3)$$
for any $p < \infty$ and  
$$e(t) = \int_{\T^3} | \theta_0 (t, x )|^2 dx  $$ 
is smooth in  $[0,1]$  and it is such that $e(1) < e(0)$.
\end{maintheorem}

\subsection{Construction of the velocity field} Add also the fact that $\theta_{0} (1/2 - T_q) \sim \theta_{\chess} (a_q^{-1})$.

{\bfseries{Step 1: Stability of $\theta_{\kappa, j}$ to the  mollifed $\theta_{0,j , \tau}$ ( mollification $\tau \sim a_q^{1 - \gamma/2(1+ \delta)})$}}

By estimating $\kappa_q \int \int \nabla \theta_{\kappa_q, j} \cdot \nabla \theta_{0,j, \tau}$ with Cauchy and brutal estimate on $\theta_{0,j, \tau}$

{\bfseries{Step 2: stability  $\theta_{\kappa_q , j}$ to $\tilde \theta_{\kappa_q , j}$}} 

By same computation we already did (support of solutions $\theta_{\kappa_q, j}$ have a concentrated support in $z$ variable, thanks to the lemma we have).

{\bfseries{Step 3: dissipation of $\tilde \theta_{\kappa_q, j}$}}

By scaling of heat equation.
}

\fi

\section{Proof of Theorem~\ref{t_Onsager}}\label{sec:4D-NS-Proof}


The aim of this section is to prove Theorem~\ref{t_Onsager}. We start by proving the key lemma which says that the last component of $v_\nu$ solution to the 4d Navier--Stokes equations with body force $F_\nu$ exhibits anomalous dissipation with further properties.  In the last subsection, we use this lemma to prove Theorem~\ref{t_Onsager}.



%We start by proving a lemma in Subsection~\ref{subsec:LemmaAnomalousDissipationSpecialSequence}. This lemma (see Lemma~\ref{lemma:LemmaAnomalousDissipationSpecialSequence} below) says that if for each $\nu$ we replace $u$ by $u_{\nu}$ (defined in Subsection~\ref{subsec:ConstructionsForNavierStokes4D}) in \eqref{e:ADV-DIFF} and consider its solutions, then the anomalous dissipation phenomenon is still exhibited by the solutions. In some sense, Lemma~\ref{lemma:LemmaAnomalousDissipationSpecialSequence} is Theorem~\ref{t_anomalous_autonomous} but for the collection $\{ u_{\nu} \}_{\nu \geq 0}$ of velocity fields instead of $u$. In the last subsection, we use this lemma to prove Theorem~\ref{t_Onsager}.

\subsection{Strategy of the proof}
The strategy of the proof is to use the $(3+\frac{1}{2})-d$ Navier--Stokes equations with a body force. More precisely, we consider the $4d$ Navier--Stokes equations with body force acting on the first three components with variable $(t, x, y, z, w) \in (0,1) \times \T^4$, where all the system is independent on the variable $w$, which reduces to the coupled system 
    \[
    \begin{cases}
        \partial_t u_\nu + u_\nu \cdot \nabla u_\nu + \nabla p_\nu = \nu \Delta u_\nu + F_\nu
            \\
        \partial_t \theta_\nu + u_\nu \cdot \nabla \theta_\nu = \nu \Delta \theta_\nu \,,
        \\
        u_\nu (0, \cdot ) = u_{\nu, \initial} (\cdot ), \quad \theta_{\nu} (0, \cdot ) = \theta_{\nu, \initial } (\cdot ) \,,
        \\
        \diver u_\nu =0 \,,
    \end{cases} 
    \]
where the unknown are $u_\nu : [0,1] \times \T^3 \to \R^3$, $\theta_\nu: [0,1] \times \T^3 \to \R$ and $p_\nu : [0,1] \times \T^3 \to \R $ and $F_\nu : [0,1] \times \T^3 \to \R^3$, $u_{\nu, \initial } : \T^3 \to \R^3$ and $\theta_{\nu, \initial}: \T^3 \to \R$ are given. The proof will rely on the objects defined in Section \ref{subsec:ConstructionsForNavierStokes4D}.

%\eqref{u_nu} \eqref{eq:Special3DAdvDiff} \eqref{v-initial-nu}



\subsection{Anomalous dissipation lemma}\label{subsec:LemmaAnomalousDissipationSpecialSequence}
The goal of this subsection is to prove the following lemma.
\begin{lemma}\label{lemma:LemmaAnomalousDissipationSpecialSequence}
 Let $\beta > 0$. Consider the collection of autonomous divergence-free velocity fields $\{ u_{\nu} \}_{\nu \geq 0}$ defined in Subsection~\ref{subsec:ConstructionsForNavierStokes4D} and the initial datum $\theta_{\initial}$ defined in Subsection~\ref{subsec:ConstructionInitialDatum}. Under the assumption that $a_0$ is selected small enough, everything that follows holds: The sequence of unique solutions $\{ \theta_{\nu} \}_{\nu \geq 0}$ of the advection-diffusion equations
 \begin{equation}\label{eq:ADV-DIFF-FOR-NS}
 \left\{
 \begin{array}{ll}
 \partial_t {\theta}_{\nu} + u_{\nu} \cdot \nabla {\theta}_{\nu} = \nu \Delta {\theta}_{\nu}; \\
 {\theta}_{\nu} (0,x,y,z) = {\theta}_{\initial}(x,y,z); \\
 \end{array}
 \right.
\end{equation}
exhibit anomalous dissipation, i.e.
\begin{equation}
\limsup_{\nu \to 0} \nu \int_0^1 \int_{\T^3} |\nabla \theta_{\nu}|^2 \, dx dt > 0.
\end{equation}
In addition, there exists a measure $\mu \in \mathcal{M} ((0,1) \times \T^3)$ such that  up to not relabelled subsequences
\begin{equation}
  \nu |\nabla \theta_{\nu}|^2 \weak \mu
\end{equation}
in the sense of measures on $(0,1) \times \T^3$ and the measure $\mu$  satisfies $\| \mu \|_{TV} \geq 1/4$ and
\begin{equation}\label{eq:NS-Lemma-H-Minus-1-Norm}
  \left \| \frac{1}{2} \partial_t |\theta_0|^2  + \diver (u \frac{|\theta_0|^2}{2}) + \mu \right \|_{H^{-1}((0,1) \times \T^3)} \leq \beta \,,
\end{equation}
where $\theta_0$ is the unique solution of \eqref{eq:ADV-DIFF-FOR-NS} with $\nu = 0$ and initial datum $\theta_{\initial}$. Moreover, $\mu_T = \pi_{\#} \mu$, where $\pi$ is the projection in time map $\pi : (0,1) \times \T^3 \to (0,1)$, has a non-trivial absolutely continuous part w.r.t. the Lebesgue measure $\mathcal{L}^1$ and its singular part is such that $\| \mu_{T, \text{sing}} \|_{TV} \leq \beta$.
Furthermore, $\theta_{\nu} \weak \theta_0$ weakly$^{\ast}$ in $L^{\infty}((0,1) \times \T^3)$ and (up to not relabelled subsequences) we have $\| \theta_{\nu} - \theta_0 \|_{L^{\infty}((0,1) \times \T^3)} < \beta$ and
\[
 e(t) = \int_{\T^3} | \theta_0 (t, x )|^2 dx
\]
is smooth in  $[0,1]$  and it is such that $e(1) < e(0)$.
\end{lemma}

The proof of this lemma is very similar to that of Theorem~\ref{t_anomalous_autonomous}. For the convenience of the reader, rather than just pointing out the differences between the two proofs, we give a less detailed, yet complete proof of this lemma. We refer to the proof of Theorem~\ref{t_anomalous_autonomous} whenever it is possible in order to shorten the proof.

\begin{proof}
Fix some arbitrary  $q \in m \N$. As in the proof of Theorem~\ref{t_anomalous_autonomous}, the goal is to prove that 
\begin{equation}\label{eq:LTwoNormAlmostZeroForTheSpecial3DCase}
\| {\theta}_{\nu_q}(1, \cdot ) \|_{L^{2}(\T^3)}^2 \lesssim a_0^{\eps \delta}.
\end{equation}
By selecting $a_0$ sufficiently small combined with the energy balance, we deduce that
\begin{equation*}
\limsup_{q \to \infty} \int_0^1 \| \nabla \theta_{\nu_q}(t) \|_{L^2(\T^3)}^2 \, dt > 0.
\end{equation*}
Let $\{ \chi_j \}_{j = 1}^{N_q} \subset C^{\infty}(\T)$ be the sequence of smooth functions and $\{ z_i \}_{i = 1}^{N_q} \subset \T$ the sequence of points defined in Subsection~\ref{subsec:PartitionOfUnityInZ}. The claims stated in this section will be proved immediately after this proof. Throughout this entire proof, we assume without loss of generality that $q$ is sufficiently large so that $\tilde{t}_q \leq \overline{t}_q$ thanks to \eqref{item:EpsDeltaThree}.
\\
\textbf{Step 1: Decomposition}
For each $j = 1, \ldots, N_q$, define $\theta_{\initial, j} \colon \T^3 \to \R$ as $\theta_{\initial, j}(x,y,z)  =  \theta_{\initial}(x,y,z) \chi_j(z)$. Then we define $\theta_{\nu_q, j}$ as the solution to the advection-diffusion equation
\begin{equation}
 \left\{
 \begin{array}{ll}
 \partial_t {\theta}_{\nu_q, j} + u_{\nu_q} \cdot \nabla {\theta}_{\nu_q, j} = \nu_q \Delta {\theta}_{\nu_q , j}; \\
 {\theta}_{\nu_q , j} (0,x,y,z) = {\theta}_{\initial , j}(x,y,z). \\
 \end{array}
 \right.
\end{equation}
Accordingly, we also specify that ${\theta}_{0,j } \colon [0,1/2] \times \T^3 \to \R$ defined as ${\theta}_{0,j}(t,x,y,z) = \theta_{\stat }(x,y,z) \varphi_{\initial} (z-t) \chi_j(z - t)$  is the unique solution  to the advection equation (see Lemma \ref{lemma:uniqueness})
\begin{equation}
 \left\{
 \begin{array}{ll}
 \partial_t {\theta}_{0,j} + u \cdot \nabla {\theta}_{0,j} = 0; \\
 {\theta}_{0,j} (0,x,y,z) = {\theta}_{\initial, j}(x,y,z). \\
 \end{array}
 \right.
\end{equation}
\textbf{Step 2: $L^2$-stability between $\theta_{\nu_q, j}$ and $\theta_{0, j}$ until time $t_{q,j}$.}
The goal of this step is to prove that 
\begin{equation}\label{eq:Special3DStabilityUpToCertainTime}
 \| {\theta}_{\nu_q, j}(t , \cdot ) - {\theta}_{0,j}(t, \cdot ) \|_{L^{2}(\T^3)} \lesssim a_q^{\gamma + \eps} \qquad \text{for all }  t \in [0,t_{q,j}] \quad \text{for all } j =1, 2, \ldots , N_q \,.
\end{equation}
Since $\supp (\theta_{0,j}(t, \cdot)) \subset \T^2 \times (z_i + t, z_i + t + \sfrac{a_q^{\gamma}}{8})$ and $u_{\nu_q}(x,y,z) = u(x,y,z)$ for all $z \in (0, \sfrac{1}{2} - T_q + \overline{t}_q)$, we deduce that in the time interval $[0, t_{q,j}]$, ${\theta}_{0,j}$ solves the advection equation
 \begin{equation}
 \left\{
 \begin{array}{ll}
 \partial_t {\theta}_{0,j} + u_{\nu_q} \cdot \nabla {\theta}_{0,j} = 0 \text{ on } [0, t_{q,i}] \times \T^3; \\
 {\theta}_{0,j} (0,x,y,z) = {\theta}_{\initial, j}(x,y,z). \\
 \end{array}
 \right.
\end{equation}
Therefore, by Lemma~\ref{lemma:AdvectionDiffusionAndTransport}, we get
\[
 \| {\theta}_{\nu_q, j}(t, \cdot ) - {\theta}_{0,j}(t, \cdot ) \|_{L^{2}(\T^3)}^2 \leq \nu_q \int_0^t \| \nabla {\theta}_{0,j} (s, \cdot ) \|_{L^2(\T^3)}^2 \, ds \,,
\]
for all $t \in [0, t_{q,j}]$.
The right-hand side of the inequality above was already estimated in the proof of Theorem~\ref{t_anomalous_autonomous} (see Lemma~\ref{lemma:EstimateOnTheGradientOfTheSolutionToTheTransportEquation}). From these estimates, we deduce \eqref{eq:Special3DStabilityUpToCertainTime} in the exact same way as in the proof of Theorem~\ref{t_anomalous_autonomous}.
\\
\textbf{Step 3: Decay of $\| \theta_{\nu_q, j}(t, \cdot) \|_{L^2(\T^3)}$ in $[t_{q,j}, t_{q,j} + \sfrac{\tilde{t}_q}{2}]$.}
The goal of this step is to prove that
\begin{equation}\label{eq:3DSpecialGoalOfStep3}
 \| {\theta}_{\nu_q, j}(t_{q,j} + \sfrac{\tilde{t}_q}{2}, \cdot ) \|_{L^2(\T^3)}^2 \lesssim a_0^{\eps \delta} a_q^{\gamma}.
\end{equation}
This step is divided into 4 substeps following the same strategy as Step 3 of the proof of Theorem~\ref{t_anomalous_autonomous}. \\
\textbf{Substep 3.1: Almost periodicity of $\theta_{\nu_q, j}(t_{q,j}, \cdot, \cdot, \cdot)$.} By the exact same arguments as in Substep 3.1 in the proof of Theorem~\ref{t_anomalous_autonomous}, we have
\begin{equation}\label{eq:3DSpecialSymmetriesL2Distance}
 \| {\theta}_{\nu_q ,j}(t_{q,j}, \cdot, \cdot, \cdot) - \varphi_{\initial}( \cdot - t_{q,j})\chi_j(\cdot - t_{q, j}) (\theta_{\text{chess}})_{a_0^{\eps \delta}}((2a_q)^{-1} \cdot, (2a_{q})^{-1} \cdot) \|_{L^2(\T^3)}^2 \lesssim a_0^{\eps \delta} a_q^{\gamma}.
\end{equation}
for some $C_1 >0$ independent on $q \in m\N$.
 
\textbf{Substep 3.2: Approximation of ${\theta}_{\nu_q, j}$ on $[t_{q,j}, t_{q,j} + \sfrac{\tilde{t}_q}{2}]$.} Let $\tilde{\theta}_{\nu_q, j}$ be as in Substep 3.3 in the proof of Theorem~\ref{t_anomalous_autonomous}. This function solves 
\begin{equation}\label{eq:ThetaTildeForNavierStokes}
\left\{
\begin{array}{l}
\partial_t \tilde{\theta}_{\nu_q, j} + \partial_z \tilde{\theta}_{\nu_q ,j} = \nu_q \Delta \tilde{\theta}_{\nu_q ,j} \qquad \text{ on } [t_{q,j}, t_{q,j} + \sfrac{\tilde{t}_q}{2}] \times \T^3;
\\
\tilde{\theta}_{\nu_q , j}(t_{q, j}, x, y, z) = \varphi_{\initial}(z - t_{q,j})\chi_j (z - t) (\theta_{\text{chess}})_{a_0^{\eps \delta}}( (2 a_q)^{-1}  x,  (2 a_q)^{-1}  y).
\end{array}
\right.
\end{equation}
By the same computations as in Substep 3.2 of the proof of Theorem~\ref{t_anomalous_autonomous}, we find that
\begin{align*}
 \| {\theta}_{\nu_q, j}(t, \cdot ) - \tilde{\theta}_{\nu_q,j}(t, \cdot ) \|_{L^2(\T^3)}^2 & + 2 \nu_q \int_{t_{q,j}}^{t} \| \nabla ({\theta}_{\nu_q, j} - \tilde{\theta}_{\nu_q, j})(s, \cdot)\|_{L^2(\T^3)}^2 \, ds \notag 
 \\
 & \lesssim \frac{1}{\nu_q} \int_{t_{q,j}}^{t} \| u^{(1,2)} \cdot \nabla_{x,y} \tilde{\theta}_{\nu_q,j}(s, \cdot) \|_{L^2(\T^3)}^2 \, ds + \| {\theta}_{\nu_q, j}(t_{q,j}, \cdot ) - \tilde{\theta}_{\nu_q,j}(t_{q,j}, \cdot ) \|_{L^2(\T^3)}^2 \,.
\end{align*}
By construction of the function $u_{\nu_q}$, in each point of $\T^3$, we either have that $u_{\nu_q} = u$ and/or $u_{\nu_q} = 0$. Therefore, $\| u_{\nu_q}^{(1,2)} \cdot \nabla_{x,y} \tilde{\theta}_{\nu_q,j}(s) \|_{L^2(\T^3)} \leq \| u^{(1,2)} \cdot \nabla_{x,y} \tilde{\theta}_{\nu_q,j}(s) \|_{L^2(\T^3)}$. Thus, by the computations already done in Substep 3.2 in the proof of Theorem~\ref{t_anomalous_autonomous}, we deduce that
\begin{equation}\label{eq:L2DistanceBetweenHatAndTilde}
  \| {\theta}_{\nu_q, j}(t, \cdot) - \tilde{\theta}_{\nu_q,j}(t, \cdot) \|_{L^2(\T^3)}^2 + 2 \kappa_q \int_{t_{q,j}}^{t} \| ({\theta}_{\nu_q, j} - \tilde{\theta}_{\nu_q,j})(s, \cdot) \|_{L^2(\T^3)}^2 \, ds \lesssim a_0^{\eps \delta} a_q^{\gamma}.
\end{equation}
for all $t \in [t_{q,j}, t_{q,j} + \sfrac{\tilde{t}_q}{2} ]$. \\
\textbf{Substep 3.3: Dissipation of ${\theta}_{\nu_q, j}$ on $[t_{q,j}, t_{q,j} + \sfrac{\overline{t}_q}{2}]$.}
It follows immediately from Substep 3.3 of the proof of Theorem~\ref{t_anomalous_autonomous} that 
\begin{equation}\label{eq:3DSpecialDissipationOfTilde}
 \| \tilde{\theta}_{\nu_q, j}(t_{q,j} + \sfrac{\overline{t}_q}{2}, \cdot ) \|_{L^2(\T^3)}^2 \leq a_q^{\gamma + \eps}.
\end{equation}
and
by the same arguments as in Substep 3.4 of the proof of Theorem~\ref{t_anomalous_autonomous} (by using \eqref{eq:L2DistanceBetweenHatAndTilde} and \eqref{eq:3DSpecialDissipationOfTilde}) we get
\begin{equation*}
 \| {\theta}_{\nu_q, j}(t_{q,j} + \sfrac{\tilde{t}_q}{2}, \cdot ) \|_{L^2(\T^3)}^2 \lesssim a_0^{\eps \delta} a_q^{\gamma},
\end{equation*}
which proves \eqref{eq:3DSpecialGoalOfStep3}. 
\\
\textbf{Step 4: Rapid decay of the $L^2$-norm of $\theta_{\nu_q}$.}
As in Step 4 of the proof of Theorem~\ref{t_anomalous_autonomous}, using the exact same arguments, we can prove \eqref{eq:LTwoNormAlmostZeroForTheSpecial3DCase}. \\
%In this step, we prove \eqref{eq:LTwoNormAlmostZeroForTheSpecial3DCase}.
%As in Step 4 of the proof of Theorem~\ref{t_anomalous_autonomous}, for each $j = 0, \ldots, N_q$, we define the sets $\tilde A_{q,j} \subset \T$ as
%\[ 
%\tilde A_{q,j}  =  \supp (\chi_j (1- \cdot )) \subset [1+ z_j , 1+ z_j + a_q^\gamma] \simeq_{\T }  [ z_j ,  z_j + \sfrac{a_q^\gamma}{8}]
%\]
%and the $\frac{a_q^\gamma}{32}$-neighbourhood 
%$$A_{q,j}  =  \left \{ z \in \T : \dist (z, \tilde A_{q,j}) < \frac{a_q^\gamma}{32} \right \}\,. $$
%
%By Corollary~\ref{lemma:TailEstimates3DAdvectionDiffusion}
%\begin{equation}\label {eq:Special3DApplicationOfTailEstimates}
% \| {\theta}_{\nu_q,j}(1 , \cdot ) \|_{L^{\infty}(\T^2 \times A_{q,j}^c )} \leq 2 \exp \left( - \frac{(\sfrac{ a_q^{\gamma}}{32})^2}{2 \nu_q } \right) = 2 \exp \left( - \frac{ a_q^{2 \gamma}}{2 \cdot 32^2 \nu_q} \right). \\
%\end{equation}
%Observe that
%\begin{equation}\label{eq:L2NormOfTheFullSolutionFirstIneq}
% \| {\theta}_{\nu_q}(1, \cdot ) \|_{L^2(\T^3)}^2 = \left\| \sum_{j = 0}^{N_q} {\theta}_{\nu_q,j}(1, \cdot ) \right\|_{L^2(\T^3)}^2 \leq 2 \left\| \sum_{j \text{ odd}} {\theta}_{\nu_q,j}(1, \cdot ) \right\|_{L^2(\T^3)}^2 + 2 \left\| \sum_{j \text{ even}} {\theta}_{\nu_q,j}(1, \cdot ) \right\|_{L^2(\T^3)}^2
%\end{equation}
%and
%\begin{align*}
% \left\| \sum_{j \text{ odd}} {\theta}_{\nu_q,j}(1, \cdot) \right\|_{L^2(\T^3)}^2 &\leq 2 \left\| \sum_{j \text{ odd}} {\theta}_{\nu_q,j}(1, \cdot ) \mathbbm 1_{\T^2 \times A_{q,j}} (\cdot ) \right\|_{L^2(\T^3)}^2 + 2 \left\| \sum_{j \text{ odd}} {\theta}_{\nu_q,j}(1, \cdot) \mathbbm 1_{\T^2 \times A_{q,j}^c} (\cdot) \right\|_{L^2(\T^3)}^2.
%\end{align*}
%The collection of open sets $\{ A_{q,j} \}_{ \{ j  \ \text{odd}  \} }$ is a collection of mutually disjoint sets due to \eqref{eq:DisjointnessOfTheSupportsOfCutoffs}. This follows from the definition of the quantities $z_j$. Therefore, as in Step 4 of the proof of Theorem~\ref{t_anomalous_autonomous}
%\begin{align*}
%\left\| \sum_{j \text{ odd}} {\theta}_{\nu_q,j}(1, \cdot ) 1_{\T^2 \times A_{q,j}} (\cdot ) \right\|_{L^2(\T^3)}^2 &\leq \sum_{j \text{ odd}} \left\| {\theta}_{\nu_q,j}(t_{q, j} + \sfrac{\overline{t}_q}{2}, \cdot ) \right\|_{L^2(\T^3)}^2 \leq \frac{N_q}{2}  10^{-4} a_{q}^{\gamma} \leq 8 \cdot 10^{-4}.\\
%\end{align*}
%By \eqref{eq:Special3DApplicationOfTailEstimates},
%\begin{align*}
%\left\| \sum_{j \text{ odd}} {\theta}_{\nu_q,j}(1, \cdot ) \mathbbm 1_{\T^2 \times A_{q,j}^c} (\cdot ) (\cdot ) \right\|_{L^2(\T^3)} &\leq \left\| \sum_{j \text{ odd}} {\theta}_{\nu_q,j}(1, \cdot ) 1_{\T^2 \times A_{q,j}^c} \right\|_{L^{\infty}(\T^3)} \leq \sum_{j \text{ odd}} 2 \exp \left( - \frac{ a_q^{2 \gamma}}{2 \cdot 32^2 \nu_q} \right) \\
%&\lesssim \sum_{j \text{ odd}}  \frac{2 \cdot 32^2 \nu_q}{ a_q^{2 \gamma}} \lesssim a_q^{2 - 3\gamma - \frac{\gamma}{1 + \delta} + 9 \eps}  \leq a_q^{\eps} \,.
%\end{align*}
%Thus there exists a constant $C>0$ independent on $q$ such that
%\begin{align*}
% \left\| \sum_{j \text{ odd}} {\theta}_{\nu_q,j}(1, \cdot ) \right\|_{L^2(\T^3)}^2 \leq 8 \cdot 10^{-4} + C a_q^{2 \eps} \leq \frac{1}{4} 10^{-2} \qquad \text{ for any } q \geq \overline{q}(C)\,.
%\end{align*}
%The same bound can be proved for the last term in \eqref{eq:L2NormOfTheFullSolutionFirstIneq}. Therefore, we conclude
%\begin{align*}
% \| {\theta}_{\nu_q}(1, \cdot ) \|_{L^2(\T^3)}^2 &\leq 2 \left\| \sum_{i \text{ odd}} {\theta}_{\nu_q,j}(1, \cdot ) \right\|_{L^2(\T^3)}^2 + 2 \left\| \sum_{i \text{ even}} {\theta}_{\nu_q,j}(1, \cdot ) \right\|_{L^2(\T^3)}^2 \leq 10^{-2}.
%\end{align*}
%This proves \eqref{eq:LTwoNormAlmostZero} and ends the proof. \\
\textbf{Step 5: Smoothness of the energy $e$ and $H^{-1}_{t,x}$ closeness.}
As in Step 5 of the proof of Theorem~\ref{t_anomalous_autonomous}, we observe that due to Lemma~\ref{lemma:uniqueness} and the fact that $\{ \theta_{\nu} \}_{\nu \geq 0}$ is uniformly bounded in $L^{\infty}((0,1) \times \T^3)$, we have $\theta_{\nu} \weak \theta_0$ in $L^{\infty}((0,1) \times \T^3)$. As in Step 5 of Theorem~\ref{t_anomalous_autonomous}, $e$ is smooth in $(0,1)$. Following the same approach as in Step 5 of the proof of Theorem~\ref{t_anomalous_autonomous} we prove that there exists a universal constant $C > 0$ such that
\begin{equation}
 \limsup_{q \in m \N, q \to \infty} \| \theta_{\nu_q} - \theta_0 \|_{L^{\infty}((0,1); L^2(\T^3))} \leq C a_0^{\eps \delta}.
\end{equation}
As a consequence, there exists $f_{\text{err}} \in L^{\infty}((0,1) \times \T^3)$ such that $\| f_{\text{err}} \|_{L^{\infty}((0,1); L^2(\T^3))}^2 \leq C a_0^{\eps \delta}$ and
\begin{equation}
 |\theta_{\nu_q}|^2 \weak |\theta_0|^2 + f_{\text{err}}
\end{equation}
 in $L^{\infty}((0,1) \times \T^3)$, up to not relabelled subsequences.
Therefore, due to Lemma~\ref{lemma:local-energy} and defining $\mu$ as the limit of  $ \nu_q |\nabla \theta_{\nu_q}|^2 \weak \mu$ up to subsequences with $q \in m\N$, the following identity
\begin{equation}
\frac{1}{2} \partial_t |\theta_0|^2 + \frac{1}{2} \partial_t f_{\text{err}} +\frac{1}{2} \diver(u|\theta_0|^2) + \frac{1}{2} \diver(u f_{\text{err}}) + \mu =0
\end{equation}
holds in the sense of distributions. From this, we deduce the $H^{-1}_{t,x}$ closeness. Indeed, we obtain
\begin{equation}
 \left\|\frac{1}{2} \partial_t |\theta_0|^2 + \frac{1}{2} \diver(u |\theta_0|^2) +\mu \right\|_{H^{-1}((0,1) \times \T^3)} \leq \|f_{\text{err}} + u f_{\text{err}} \|_{L^2((0,1) \times \T^3)}  \leq C a_0^{\eps \delta} < \beta \,,
\end{equation}
where the last holds thanks to the choice of $a$ in Subsection \ref{subsec:ParmeterDefinitions}.

%Due to the previous steps, $e(1) < e(0)$. By the same arguments as in the proof of Theorem~\ref{t_anomalous_autonomous}, $\theta_{\nu} \to \theta_0$ in $L^2((0,1) \times \T^3)$ as $\nu \to 0$. This finishes the proof
\noindent \textbf{Step 6: Convergence behaviour of $- \frac{1}{2}  e_{\nu_q}^{\prime}$ for $q \in m \N$.}
In order to finish the proof of the lemma, we investigate the behaviour of
\[
 e_{\nu_q}^{\prime}(t) =  \nu_q \int_{\T^3} |\nabla \theta_{\nu_q}(t,x)|^2 \, dx
\]
as $q \to \infty$. By the global energy balance (see \eqref{e:energy-equality-global}) it is clear the up to subsequences $\{ e_{\nu}^{\prime} \}_{\nu}$ weak-$\ast$ converges to a measure $\mu$. As in the proof of Theorem~\ref{t_anomalous_autonomous}, we define $A_{\nu_q}, S_{\nu_q} \colon [0,1] \to \R$ as
\[
 A_{\nu_q}(t) =  \nu_q \sum_{j = 1}^{N_q} \int_{\T^3} |\nabla \tilde{\theta}_{\nu_q, j}(t,x)|^2 \mathbbm{1}_{[t_{q,j}, t_{q,j} +  {\tilde{t}_q}]}(t) \, dx \quad \text{and} \quad S_{\nu_q} = - \frac{1}{2}  e_{\nu_q}^{\prime}(t) - A_{\nu_q}(t).
\]
We will prove that 
\begin{equation}\label{eq:WhatWeWantToProveInStepSixOfTheAnomalousDissipationLemma}
 \limsup_{q \to \infty} \| S_{\nu_q} \|_{L^1((0,1))} \lesssim a_0^{\frac{\eps \delta}{2}} \quad \text{ and } \quad  \sup_q \| A_{\nu_q} \|_{L^{\infty}(0,1)} < \infty \quad  \| \mu_T \|_{TV} \geq 1/4 \,,
\end{equation}
where $\mu_T \in \mathcal{M}(0,1)$ is defined as $\mu_T = \pi_{\#} \mu   $.
From this we are able to conclude that up to subsequences $A_{\nu_q}$ weak-$\ast$ converges in $L^{\infty}$ to some $\mathcal{A} \in L^{\infty}((0,1))$ and $S_{\nu_q}$ weak-$\ast$ converges in $\mathcal{M}((0,1))$ to some measure $\mathcal{S} \in \mathcal{M}((0,1))$. 
%As in the proof of Theorem~\ref{t_anomalous_autonomous},
The properties in \eqref{eq:WhatWeWantToProveInStepSixOfTheAnomalousDissipationLemma} imply that $\| \mathcal{S} \|_{TV} < \beta < 1/4$ provided that  $a_0$ is chosen to be small enough depending on $\beta$ and universal constants. Hence the absolutely continuous part of the measure $\mu = \mathcal{A} + \mathcal{S}$ is non-trivial. The proof of \eqref{eq:WhatWeWantToProveInStepSixOfTheAnomalousDissipationLemma} is divided into 4 substeps following the same strategy as Substep 6 in the proof of Theorem~\ref{t_anomalous_autonomous}.
\\
\textbf{Substep 6.1:}
We have
\[
- \frac{1}{2} e_{\nu_q}^{\prime} = \underbrace{ \nu_q \sum_{i = 1}^{N_q} \int_{\T^3} |\nabla \theta_{\nu_q, i}(t,x)|^2 \, dx}_{= A_{\nu_q}^{(1)}(t)} + \underbrace{ \nu_q \sum_{i,j = 1, i \neq j}^{N_q} \int_{\T^3} \nabla \theta_{\nu_q, i}(t,x) \cdot \nabla \theta_{\nu_q, j}(t,x) \, dx}_{= S_{\nu_q}^{(1)}(t)}.
\]
We will prove the following result:
 there exist universal constants $C>0$ and $Q \in \N$ such that for all $q \geq Q$ we have
 \begin{equation}\label{eq:EstimateOfQuatities-E1-Navier-Stokes}
  \| S_{\nu_q}^{(1)} \|_{L^1((0,1))} \leq C a_0^{\sfrac{\eps \delta}{2}}.
 \end{equation}
In order to prove this claim, it suffices to adapt the proofs of Lemmas~\ref{lemma:EstimateOfQuantitiy-II} and \ref{lemma:EstimateOfQuantitiy-I} to this slightly different context. To adapt the proof of Lemma~\ref{lemma:EstimateOfQuantitiy-II}, we can repeat the exact same proof, the only difference being that $\overline{\theta}_{\nu_q, j}$ defined in the proof would not solve \eqref{eq:1D-Adv-Diff-Cutoffs-Prod-With-Solution} but instead solves
 \begin{equation*}
 \left\{
 \begin{array}{ll}
 \partial_t \overline{\theta}_{\nu_q, j} + u_{\nu_q} \cdot \nabla \overline{\theta}_{\nu_q, j} = \kappa_q \Delta \overline{\theta}_{\nu_q, j} - 2 \kappa_q \nabla {\theta}_{\nu_q, j} \cdot \nabla \overline{\psi}_{\nu_q, j}; \\
 \overline{\theta}_{\nu_q, j}(0, \cdot) = {\theta}_{\nu_q, j}(0, \cdot). \\
 \end{array}
 \right.
\end{equation*}
 In order to adapt Lemma~\ref{lemma:EstimateOfQuantitiy-I} to this context, no other modifications than replacing \eqref{eq:AutonomousNormalGoalOfStep3} and \eqref{eq:TildeApproxOfHatL2} by \eqref{eq:3DSpecialGoalOfStep3}  and \eqref{eq:L2DistanceBetweenHatAndTilde} are needed.

\textbf{Substep 6.2:} Note that
\begin{align*}
 A_{\nu_q}^{(1)}(t) &=  \nu_q \sum_{i = 1}^{N_q} \int_{\T^3} |\nabla \theta_{\nu_q, i}(t,x)|^2 \, dx = \underbrace{ \nu_q \sum_{i = 1}^{N_q} \int_{\T^3} |\nabla \theta_{\nu_q, i}(t,x)|^2 \, dx \mathbbm{1}_{[t_{q,i}, t_{q,i} + \tilde{t}_q]}(t)}_{= A_{\nu_q}^{(2)}(t)} \\
 & \quad + \underbrace{ \nu_q \sum_{i = 1}^{N_q} \int_{\T^3} |\nabla \theta_{\nu_q, i}(t,x)|^2 \, dx \mathbbm{1}_{[0, t_{q,i})}(t)}_{= S_{\nu_q}^{(2)}(t)} + \underbrace{ \nu_q \sum_{i = 1}^{N_q} \int_{\T^3} |\nabla \theta_{\nu_q, i}(t,x)|^2 \, dx \mathbbm{1}_{(t_{q,i} + \tilde{t}_q, 1]}(t)}_{= S_{\nu_q}^{(3)}(t)}.
\end{align*}
It suffices to repeat the proof of Lemma~\ref{lemma:EstimatesOfTheQuantities-E2-E3} by replacing \eqref{eq:AutonomousNormalGoalOfStep3} and \eqref{eq:3DSpecialGoalOfStep3} to prove the following estimate.
 There exists a universal constant $C$ such that
 \begin{equation}\label{eq:EstimateOfQuatities-E2-E3-Navier-Stokes}
  \| S_{\nu_q}^{(2)} \|_{L^1((0,1))} + \| S_{\nu_q}^{(3)} \|_{L^1((0,1))} \leq C a_0^{\eps \delta}.
 \end{equation}



\textbf{Substep 6.3: }
For all $q \in m\N$ and $i = 1, \ldots, N_q$, define the interval $\mathcal{J}_{q,i} = [t_{q,i}, t_{q,i} + \tilde{t}_q]$. Observe that 
\begin{align*}
 A_{\nu_q}^{(3)}(t) &=  \nu_q \sum_{i = 1}^{N_q} \int_{\T^3} |\nabla {\theta}_{\nu_q, i}(t,x)|^2 \, dx \mathbbm{1}_{\mathcal{J}_{q,i}}(t) = \underbrace{ \nu_q \sum_{i = 1}^{N_q} \int_{\T^3} |\nabla \tilde{\theta}_{\nu_q, i}(t,x)|^2 \, dx \mathbbm{1}_{\mathcal{J}_{q,i}}(t)}_{= A_{\nu_q}(t)} \\
 &\quad + \underbrace{ \nu_q \sum_{i = 1}^{N_q} \int_{\T^3} \nabla {\theta}_{\nu_q, i}(t,x) \cdot \nabla ({\theta}_{\nu_q, i} - \tilde{\theta}_{\nu_q, i})(t,x)\, dx \mathbbm{1}_{\mathcal{J}_{q,i}}(t)}_{= S_{\nu_q}^{(4)}(t)} \\
 &\quad + \underbrace{ \nu_q \sum_{i = 1}^{N_q} \int_{\T^3} \nabla ({\theta}_{\nu_q, i} - \tilde{\theta}_{\nu_q, i})(t,x) \cdot \nabla \tilde{\theta}_{\nu_q, i}(t,x)\, dx \mathbbm{1}_{\mathcal{J}_{q,i}}(t)}_{= S_{\nu_q}^{(5)}(t)} \\
\end{align*}
It suffices to repeat the proof of Lemma~\ref{lemma:EstimateOfQuatities-E6-E7} and replacing \eqref{eq:TildeApproxOfHatL2} by \eqref{eq:ThetaTildeForNavierStokes} to prove the following estimate.
There exists a universal constant $C$ such that
 \begin{equation}\label{eq:EstimateOfQuatities-E6-E7-Navier-Stokes}
  \| S_{\nu_q}^{(4)} \|_{L^1((0,1))}, \| S_{\nu_q}^{(5)} \|_{L^1((0,1))} \leq C a_q^{\sfrac{\eps}{2}}.
 \end{equation}
Notice that $S_{\nu_q} = \sum_{\ell = 1}^{5} S_{\nu_q}^{(\ell)}$ and therefore in virtue of \eqref{eq:EstimateOfQuatities-E1-Navier-Stokes}, \eqref{eq:EstimateOfQuatities-E2-E3-Navier-Stokes} and \eqref{eq:EstimateOfQuatities-E6-E7-Navier-Stokes}
\begin{equation}\label{eq:L1BoundOnTheError-Navier-Stokes}
\| S_{\nu_q} \|_{L^1((0,1))} \lesssimlarge a_0^{\frac{\eps \delta}{2}}. 
\end{equation}
This proves the first inequality in \eqref{eq:WhatWeWantToProveInStepSixOfTheAnomalousDissipationLemma}.

\textbf{Substep 6.4: }  In this step we prove that  $\| \mu \|_{TV} \geq 1/4$ and $\sup_{q} \| A_{\kappa_q} \|_{L^{\infty}((0,1))} < \infty$.

Firstly, we notice that $\mu$ is the weak* limit up to subsequence (in the sense of measure) of the non-negative sequence $\{ \nu_q  |\nabla \theta_{\nu_q} |^2 \}_{q \in m \N }$ that is such that \eqref{eq:total-diss} holds, therefore $\| \mu_T \|_{TV} \geq 1/4$ holds.
We now prove the second property.
Recall that $\tilde{\theta}_{\nu_q, i}$ solves \eqref{eq:ThetaTildeForNavierStokes} and $\mathcal{J}_{q,i} = [t_{q,i}, t_{q,i} + \tilde{t}_q]$. Define the function $\overline{N}_q \colon [0,1] \to \N$ as
\[
 \overline{N}_q(t) = \# \{ j \in \{ 1, \ldots, N_q \} : \mathbbm{1}_{\mathcal{J}_{q,i}}(t) \neq 0 \}.
\]
It follows from straightforward computations that $\| \overline{N}_q(t) \|_{L^{\infty}((0,1))} \lesssim a_0^{- \eps \delta} a_q^{\frac{\gamma}{1 + \delta} - 10 \eps - \gamma}$.
For all $t \in \mathcal{J}_{q,i}$,
\[
 \| \nabla \tilde{\theta}_{\nu_q, i}(t, \cdot) \|_{L^2(\T^3)}^2 \leq \| \nabla \tilde{\theta}_{\nu_q, i}(t_{q,i}, \cdot) \|_{L^2(\T^3)}^2 \lesssim a_0^{-6 \eps \delta} a_q^{-2 + \gamma}
\]
From this we get
\begin{align*}
 |A_{\nu_q}(t)| &\leq 2 \kappa_q \sum_{i = 1}^{N_q} \int_{\T^3} |\nabla \tilde{\theta}_{\nu_q, i}|^2 \, dx \mathbbm{1}_{\mathcal{J}_{q,i}} (t) = 2 \nu_q \sum_{i = 1}^{N_q} \| \nabla \tilde{\theta}_{\nu_q, i} \|_{L^2(\T^3)}^2 \mathbbm{1}_{\mathcal{J}_{q,i}} (t) \\
 &\leq 2 \nu_q \overline{N}_q(t) a_q^{-2 + \gamma} \lesssim a_0^{-7 \eps \delta}.
\end{align*}
 This proves the last inequality in \eqref{eq:WhatWeWantToProveInStepSixOfTheAnomalousDissipationLemma}. Combining this with \eqref{eq:L1BoundOnTheError-Navier-Stokes} implies that $\sup_{q} \| - \frac{1}{2} e_{\nu_q}^{\prime} \|_{L^{1}((0,1))} < \infty$. Therefore, up to subsequences
 \[
  A_{\nu_q} \overset{*}{\rightharpoonup} \mathcal{A} \in L^{\infty}((0,1)), \quad 
  S_{\nu_q} \overset{*}{\rightharpoonup} \mathcal{S} \in \mathcal{M}([0,1]) \quad \text{and} \quad - \frac{1}{2} e_{\nu_q}^{\prime} \overset{*}{\rightharpoonup} \mu_T \in \mathcal{M}([0,1])\,,
 \]
 where $\mu_T= \pi_{\#} \mu $.
 In virtue of \eqref{eq:L1BoundOnTheError-Navier-Stokes}, we find that $\| \mu_T - \mathcal{A} \|_{TV} = \| \mathcal{S} \|_{TV} \lesssim a_0^{\frac{\eps \delta}{2}} < \beta$ where the last holds up to choosing $a_0$ sufficiently small in  Subsection \ref{subsec:ParmeterDefinitions} depending on $\beta$ and universal constants. The proof of Lemma~\ref{lemma:LemmaAnomalousDissipationSpecialSequence} is complete.
\end{proof}



\subsection{Proof of Theorem~\ref{t_Onsager}}
Let $\{ u_\nu \}_{\nu \geq 0}, \{ F_\nu \}_{\nu \geq 0}$ and $\{ v_{\initial, \nu} \}_{\nu \geq 0}$ be defined as in Subsection~\ref{subsec:ConstructionsForNavierStokes4D}. By the previous lemma, the solutions $\{ \theta_{\nu} \}_{\nu \geq 0}$ solving the 3D advection-diffusion equations
\begin{equation}
 \left\{
 \begin{array}{ll}
 \partial_t {\theta}_{\nu} + u_{\nu} \cdot \nabla {\theta}_{\nu} = \nu \Delta {\theta}_{\nu}; \\
 {\theta}_{\nu} (0,x,y,z) = {\theta}_{\initial}(x,y,z). \\
 \end{array}
 \right.
\end{equation}
exhibit anomalous dissipation i.e.
\begin{equation}
 \limsup_{\nu \to 0} \nu \int_0^1 \int_{\T^3} |\nabla \theta_{\nu}|^2 \, dx dy dz \, dt > 0.
\end{equation}
For each $\nu \geq 0$, we define $v_{\nu} \colon [0,1] \times \T^4 \to \R^4$ as
\begin{equation} \label{v_nu}
 v_{\nu} (t,x,y,z,w)  =  
 \begin{pmatrix}
  u_{\nu}(x,y,z) \\
  \theta_{\nu}(t,x,y,z)
 \end{pmatrix}
 =
 \begin{pmatrix}
  u_{\nu}^{(1)}(x,y,z) \\
  u_{\nu}^{(2)}(x,y,z) \\
  u_{\nu}^{(3)}(x,y,z) \\
  \theta_{\nu}(t,x,y,z)
 \end{pmatrix}.
\end{equation}
Standard computations yield
\begin{equation}
 (v_{\nu} \cdot \nabla)v_{\nu} = 
  \begin{pmatrix}
  \partial_z u_{\nu}^{(1)} \\
  \partial_z u_{\nu}^{(2)} \\
  0 \\
  u_{\nu} \cdot \nabla_{x,y,z} \theta_{\nu}
 \end{pmatrix}.
\end{equation}
Recalling the collection of forces $\{ F_{\nu} \}_{\nu \geq 0}$ defined in Subsection~\ref{subsec:ConstructionsForNavierStokes4D}, we see that with pressure $p_{\nu} = 0$, $v_{\nu}$ solves
\begin{equation}
 \left\{
 \begin{array}{ll}
 \partial_t v_{\nu} + (v_{\nu} \cdot \nabla)v_{\nu} + \nabla p_{\nu} - \nu \Delta v_{\nu} = F_{\nu} ; \\
 \diver v_{\nu} = 0; \\
 v_{\nu}(0, x, y, z, w) = v_{\initial, \nu}(x, y, z, w). \\
 \end{array}
 \right.
\end{equation}
Notice that
\begin{equation} \label{AD_final}
 \limsup_{\nu \to 0} \nu \int_0^1 \int_{\T^3} |\nabla v_{\nu}|^2 \, dx dy dz \, dt \geq \limsup_{\nu \to 0} \nu \int_0^1 \int_{\T^3} |\nabla \theta_{\nu}|^2 \, dx dy dz \, dt > 0
\end{equation}
where the last inequality follows from Lemma~\ref{lemma:LemmaAnomalousDissipationSpecialSequence}. This proves \eqref{e:zeroth-law}. It follows from Lemma~\ref{lemma:LemmaAnomalousDissipationSpecialSequence} and the definition of $p_{\nu}$ above that $(v_{\nu},p_{\nu}) \weak (v_0, p_0)$ in $L^{\infty}((0,1) \times \T^4)$ and that $e$ is smooth in $[0,1]$. 

\begin{comment}
{\color{red}In order to finish the proof of Theorem~\ref{t_Onsager}, we must show that (up to subsequences) $v_{\nu}$ converges weakly$^\ast$ in $L^{\infty}((0,1) \times \T^4; \R^4)$ and strongly in $L^{2}((0,1) \times \T^4; \R^4)$ to a solution of the forced 4D Euler equations. In addition, we must show that $e$ defined in \eqref{eq:EnergyInEuler} is smooth.
  Due to Lemma~\ref{lemma:LemmaAnomalousDissipationSpecialSequence}, $\theta_{\nu} \weak \theta_0$ in $L^{\infty}((0,1) \times \T^4; \R^4)$ and $\theta_{\nu} \to \theta_0$ in $L^{2}((0,1) \times \T^4; \R^4)$. We conclude that $v_{\nu} \weak v_0$ in $L^\infty$ and $v_{\nu} \to v_0$ in $L^2$, because the first three component satisfy the same convergence from the definition of  $u_\nu$ thanks to the boundedness of $u$ (see \eqref{u_nu}).}
%are strongly convergent in $L^\infty$  (recalling that $v_\nu$ is defined in \eqref{v_nu} and $u_\nu$ defined in \eqref{u_nu} is strongly convergent in $L^\infty$ to $u_0$) where $v_0 \colon (0,1) \times \T^4 \to \R^4$ is defined by
\end{comment}
The fact that $v_0$ defined in \eqref{v_nu} solves the forced Euler equation with body force $F_0$ and that
\begin{equation}
 \int_{\T^4} F_0(t,x) \cdot v_0(t,x) \, dx = \int_{\T^4} \partial_z v_0^{(1,2)}  \cdot v_0^{(1,2)}  =0 \quad \forall t \in (0,1).
\end{equation}
follows from standard computations, observing that $\partial_z v_0^{(1,2)}, v_0^{(1,2)}$ are time-independent continuous functions. The remaining desired properties of the collection of body forces $\{ F_{\nu} \}_{\nu \in (0,1)}$ follow from Lemma~\ref{lemma:AboutTheCollectionOfBodyForces}. 

We observe that by definition of $v_\nu$ we have 
$$\nu |\nabla v_\nu|^2 = \nu | \nabla u_\nu|^2 + \nu | \nabla \theta_\nu |^2$$
and thanks to the estimate \eqref{prop:estimate_u} and the definition of  $u_\nu$ \eqref{u_nu}, for any $\nu \in (\nu_{q+1} , \nu_q ]$ we have 
$$\nu |\nabla u_\nu|^2 \leq \nu \sup_{j \leq q -1 } \| \nabla u_\nu \|_{L^\infty (\T^2 \times \mathcal{I}_j)}^2 \leq  C \nu_q a_{q-1}^{2- 2 \gamma} a_q^{-2 (1+ \eps \delta)} \leq a_{q-1}^{2 - 3 \gamma} \,,  $$
where the last holds thanks to \eqref{d:k_q}. We observe that, thanks to $\gamma < 2/3$, the last term goes to zero as $q \to \infty$.


Therefore, by definition of the velocity fields $\{ u_{\nu} \}_{\nu \in (0,1)}$ and Lemma~\ref{lemma:LemmaAnomalousDissipationSpecialSequence}, up to not relabelled subsequences, we have 
\begin{equation}
 \nu |\nabla v_{\nu}| \weak \mu
\end{equation}
where $\mu$ is the measure in Lemma~\ref{lemma:LemmaAnomalousDissipationSpecialSequence}. From \eqref{eq:NS-Lemma-H-Minus-1-Norm}, we deduce \eqref{e:mu-duchon-robert}.

\begin{comment}
Finally, the fact that $e$ defined in \eqref{eq:EnergyInEuler} is smooth is a consequence of the three first components of $v_0$ being time-independent and $t \mapsto \| v_0^{(4)}(t, \cdot) \|_{L^{2}(\T^4)} = \| \theta_0(t, \cdot) \|_{L^{2}(\T^4)}$ being smooth by Lemma~\ref{lemma:LemmaAnomalousDissipationSpecialSequence}. To finish the proof, the inequality
$$e(1) < e(0) + \int_0^1 \int_{\T^4} F_0(s,x) \cdot v_0(s,x) \, dx ds$$ 
holds thanks to the energy equality for $\nu >0$, the anomalous dissipation \eqref{AD_final} and the lower semicontinuity of the norm. This ends the proof of Theorem~\ref{t_Onsager}.
\end{comment}

\section{Duchon--Robert distribution vs. anomalous dissipation measure} \label{sec:duchon-robert-anomalous}

The following theorem is an interesting observation that arises from  computations of \cite{BCCDLS22} with other choices of viscosity parameters $\nu_q$.

\begin{proposition}[Duchon--Robert measure $\neq$ anomalous dissipation measure] \label{prop:duchon-anomalous}
For any $\alpha <1$ there exists a divergence-free initial datum $v_{\initial}$ and force $F_0 \in C^\alpha ((0,2)\times \T^3)$ and a {\em weak physical solution} of $3D$ forced Euler equations $v_0 \in L^\infty ((0,2) \times \T^3)$ such that there exists a unique  anomalous dissipation measure $\mu \in \mathcal{M}((0,2) \times \T^3)$ associated to $v_0$ such that $\mu \equiv 0$ and 
$$\mu  \neq \mathcal{D}[v_0]  \,,$$
where $\mathcal{D}[v_0]$ is the Duchon--Robert distribution (see \eqref{duchon-robert}) associated to $v_0$.
\end{proposition}

\begin{proof}

Let $\beta=0$ with $ \alpha$, $\gamma$, $\epsilon$, $\delta$, $\{ a_q \}_{q \in \N}$ and $\{ T_q \}_{q \in \N}$ as in \cite[Section 6]{BCCDLS22}. Let $u \in C^{\infty}_{\loc}((0,1) \times \T^2; \R^2)$ be the velocity fields constructed in \cite[Proposition 3.1]{BCCDLS22} with these parameters (restricted to the time interval $(0,1)$). Extend it as zero to $(0,2) \times \T^2$ whenever $t \geq 1$. 
We define the sequence of viscosity parameters 
$$\nu_q = a_q^{2 + 7 \epsilon (1+ \delta)}.$$
%We now construct $(v_\nu, p_\nu)$ to be a smooth solution to the $3d$-Navier-Stokes equations \eqref{e:NSE} with force $F_\nu$ (to be defined) and  smooth initial datum $v_{\initial} \in C^\infty (\T^3)$.
For any $\nu \in (0, a_0^{3})$, there exists a unique $q = q(\nu) \in \N$ such that $\nu \in (\nu_{q+1} , \nu_q]$. For any $\nu \in (0, a_0^{3})$ given this unique $q$, we set $K_\nu = [0,1-T_q]$ and define
\begin{equation}
     u_\nu (t,x ) = u(t,x) \mathbbm{1}_{K_\nu} (t) \quad t \in (0,2), \, x \in \T^2.
\end{equation}
We observe that $u_\nu$ is smooth for any $\nu \in (0, a_0^{3})$ (notice that $u \equiv 0$ locally around $1-T_q$ for any $q$) and
$$\| u_\nu - u \|_{L^\infty((0,2) \times \T^2)} \to 0 $$
as $\nu \to 0$. 
The velocity fields $u$ satisfies the uniqueness of forward trajectories by construction. By the superposition principle, the advection equation with velocity field $u$ has a unique solution $\theta \colon (0,2) \times \T^2 \to \R$ with initial datum $\theta_{\initial} \in C^{\infty}(\T^2)$ where $\theta_{\initial}$ is defined as in \cite[Proposition 3.1]{BCCDLS22}.
Thanks to \cite[Property (5) of Proposition 3.1]{BCCDLS22}, energy equality \eqref{e:energy-equality-global} and uniqueness of the advection equation solutions in $L^\infty$, it is straightforward that $\theta$ is a dissipative solution in $(0,2)$, namely 
\begin{align} \label{eq:dissipation-theta-0}
\| \theta (t, \cdot ) \|_{L^2 (\T^2)} \leq \frac{1}{2} \| \theta_{\initial} \|_{L^2 (\T^2)}  \qquad \text{ for any  } t > 1
\end{align} 
%We consider the family of viscosity parameters $\nu_q = a_q^{2+ 7 \epsilon (1+ \delta)}$.
%\begin{equation}
%\label{eq: tilde nu}    {\tilde \nu}_q = a_q^{2 - \frac{\gamma}{1 + \delta} + 4 \epsilon} \, %\quad \text{for any $q\in \N$}.
%\end{equation}
%For any $\nu \in (0,a_0^2)$ there exists $q\in \N$ such that $\nu \in (  \nu_{q+1},  \nu_q]$.
For $\nu \in (  \nu_{q+1},  \nu_q]$ we define $ \theta_{\nu} : [0,2] \times \T^3 \to \R$ to be the unique smooth solution to the advection-diffusion equation \eqref{e:ADV-DIFF} with diffusion parameter $\nu$, initial datum $\theta_{\initial} \in C^\infty (\T^2)$ and velocity field $ u_\nu$, 
i.e.
\begin{align*}
\partial_t \theta_{\nu} + {u}_{\nu} \cdot \nabla \theta_{\nu} = \nu \Delta \theta_{ \nu} \, .
\end{align*} 
Up to this point, all functions were defined on $(0,2) \times \T^2$. 
From now on, we consider all these functions as being defined on $(0,2) \times \T^3$ by extending them independently to the third spatial variable\footnote{i.e. for any function $f$ defined on $(0,2) \times \T^2$, we redefine it as $\tilde{f}$ on $(0,2) \times \T^3$ by $\tilde{f}(t, x_1, x_2, x_3) = f(t, x_1, x_2)$.}.  
Then, we define smooth functions $F_{\nu}, v_\nu : [0,2] \times \T^3 \to \R^3$ and $p_\nu : [0,2] \times \T^3 \to \R$ as 
\begin{align*}
F_{\nu}(t,x) 
    & = \begin{pmatrix}
\partial_t  u_\nu (t,x) - \nu \Delta  u_\nu (t,x)
\\
0
\end{pmatrix}
\\
v_{\nu} (t,x ) & = \begin{pmatrix}
   u_\nu(t,x)
 \\
 \theta_{\nu} (t,x)
\end{pmatrix}
\\p_{\nu} &= 0\,,
\end{align*}
which are independent on $x_3$. 
 Finally, we set
\begin{align*}
 v_{\initial } = \begin{pmatrix}
0 
\\
 \theta_{\initial}
\end{pmatrix}\, .
\end{align*}
Similarly to \cite{BCCDLS22} we can verify that $\| F_\nu - F_0 \|_{C^\alpha_{t,x}}  \to 0 $ where $F_0 = (\partial_t u , 0)$ as $\nu \to 0$ and $\| u_\nu - u \|_{C^\alpha_{t,x}} \to 0$ as $\nu \to 0$  and thanks to uniqueness of the advection equation $\theta_{\nu } $ is $L^\infty$ weakly* converging to $\theta$ as $\nu \to 0$, which implies that $v_\nu \rightharpoonup^* v_0$, where
$$v_{0} (t,x )  = \begin{pmatrix}
  u(t,x)
 \\
 \theta (t,x)
\end{pmatrix} $$
 is a solution to the 3D forced Euler equations by a direct computation. Since $\theta$ is a dissipative in the sense of \eqref{eq:dissipation-theta-0} we have that $\mathcal{D}[v_0] \neq 0$. We now show that there exists a unique anomalous dissipation measure $\mu \in \mathcal{M} ((0,2) \times \T^3)$ and it is $\mu \equiv 0$.
Let $\nu \in (0, a_0^{3})$. We have that 
\begin{align*}
\nu \int_0^2 \int_{\T^3} |\nabla \theta_{\nu} (t,x)|^2 dx dt = \nu \int_0^{1 -T_q} \int_{\T^3} |\nabla \theta_{\nu} (t,x)|^2 dx dt + \nu \int_{1  - T_q}^2 \int_{\T^3} |\nabla \theta_{\nu} (t,x)|^2 dx dt \,.
\end{align*}
Thanks to \cite[Property (4) of Proposition 3.1]{BCCDLS22} we have that 
\begin{align} \label{eq:estimate-nu-nabla-theta-1}
\nu \int_0^{1-T_q} \int_{\T^3} \nabla {\theta_\nu} (t, x) \, dt dx   \leq \nu_q \sum_{k=0}^{q} a_k^{-2 - 6 \epsilon (1+ \delta)} \leq q \nu_q  a_q^{-2 - 6 \epsilon (1+ \delta)} \leq a_q^\epsilon  \to 0 \,,
\end{align}
where the last inequality holds for $q$ sufficiently large to reabsorb $q  a_q^{\epsilon \delta} \leq 1$,
using also that $u_\nu (t, \cdot ) \equiv 0$ for any $t \geq 1 - T_q$ and by standard heat equation regularity we have 
\begin{align}\label{eq:estimate-nu-nabla-theta-2}
\nu \int_{1  - T_q}^2 \int_{\T^3} |\nabla \theta_{\nu} (t,x)|^2 dx dt  \leq 2 \nu \| \nabla \theta_{\nu} (1 - T_q , \cdot) \|_{L^\infty (\T^2)}^2 \leq   \nu_q  a_q^{-2 - 6 \epsilon (1+ \delta)} \leq a_q^\epsilon \,.
\end{align} 


 Using the estimates of \cite[Property (3) of Proposition 3.1]{BCCDLS22} we also have 
 \begin{align} \label{eq:nu-u_q}
 \nu \| \nabla u_\nu \|_{L^2((0,2) \times \T^3)}^2  & \leq \nu_q \sum_{j =0}^{q-1} \| \nabla u_\nu \|_{L^\infty ((1-T_j, 1- T_{j+1}) \times \T^3)}^2 \notag
 \\
 & \leq C a_q^{2 + 7 \epsilon (1+ \delta)} \sum_{j=0}^{q-1} a_j^{2 - 2 \gamma} a_{j+1}^{-2 (1 +  \epsilon \delta)} 
 \\
 & \leq C q a_q^{2 + 7 \epsilon (1+ \delta)} a_q^{-2 (1+ \epsilon \delta)} \leq C q a_q^{7 \epsilon + 5 \epsilon \delta}
\leq  a_q^\epsilon \,, \notag
 \end{align}
 where we used that $\gamma <1$ and $q$ sufficiently large to reabsorb $C q a_q^{\epsilon} \leq 1$.
 Therefore
 we conclude that 
 $$ \mu = \lim_{\nu \to 0} \nu |  \nabla v_\nu |^2 = 0 \,,$$
thanks to \eqref{eq:estimate-nu-nabla-theta-1}, \eqref{eq:estimate-nu-nabla-theta-2} and \eqref{eq:nu-u_q}.


 
 
 
 \begin{comment}
 
 Thanks to \cite[Property (4) of Proposition 3.1]{BCCDLS22}  and Lemma \ref{lemma:AdvectionDiffusionAndTransport} for any $t \leq 1-T_q$ we have that
\begin{align}\label{eq:theta-nu}
\| {\theta}_{\nu}(t, \cdot ) - {\theta} (t, \cdot) \|_{L^2(\T^d)}^2 \leq \nu \int_0^{1-T_q} \| \nabla {\theta} (s, \cdot ) \|_{L^2(\T^d)}^2 \, ds  \leq \nu_q a_q^{-2 - 6 \epsilon (1+ \delta)} \leq a_q^\epsilon \,.
\end{align}
Using \eqref{e:energy-equality-global} we have
 \begin{align*}
 \| \theta_\nu (2, \cdot)  \|_{L^2}^2 & = \| \theta_\nu (1-T_q, \cdot)  \|_{L^2}^2  - 2 \nu \int_{1-T_q}^2 \int_{\T^2} | \nabla  \theta_{\nu}|^2 
 \end{align*}
 Thanks to \eqref{eq:theta-nu}, the conservation of the $L^2$ norm of $\theta$ for any  $t <1$ and triangular inequality we have that 
 $$ \| \theta_\nu (1-T_q, \cdot)  \|_{L^2}^2  \geq   \| \theta_{\initial} \|_{L^2}^2 - 2 a_q^{\epsilon/2} \| \theta_{\initial } \|_{L^2} \,.$$
 Finally using that $u_\nu (t, \cdot) \equiv 0$ for any $t \geq 1- T_q$, \cite[Property (4) in Proposition 3.1]{BCCDLS22} we have that 
 \begin{align*}
2 \nu \int_{1-T_q}^2 \int_{\T^2} | \nabla  \theta_{\nu}|^2  \leq    C a_q^{2+ 7 \epsilon (1+ \delta)} a_q^{-2 - 6 \epsilon (1+ \delta)} \leq a_q^\epsilon.
\end{align*}  
Therefore we conclude that 
 \begin{align} \label{eq:theta-nu-time-2}
 \| \theta_\nu (2, \cdot)  \|_{L^2}^2 \geq  \|\theta_{\initial} \|_{L^2}^2 -  2 a_q^{\epsilon /2} \| \theta_{\initial } \|_{L^2} -   a_q^{\epsilon} \,.
 \end{align} 
 which implies 
 $$\int_{} $$
 
 {\color{red} Missing parts:

- conclude from above that $\nu | \nabla v_\nu|^2$ is converging to a measure (the anomalous dissipation measure) which has total variation less than $1/3$  

NB. (no dissipation also in first two components $\nu | \nabla u_q |^2 \to 0$)
 
  -  check that the limit $v_0$ (to be defined) is a solution to Euler and its Duchon Robert measure has total variation at least 1/2 (because of $\theta_0$)
  
 - check that the unique solution $\theta_0$ of transport with $u$ has duchon robert measure with total variation larger than 1/2  ({\color{blue} I don't know the cheapest way to prove it. One way to say it is to choose the sequence $\tilde{\nu}_q$ for which we have anomalous dissipation. Otherwise we have to characterize the unique solution $\theta_0$.})
 
 - check and $F_\nu \to F_0 $ in $C^\alpha$.

 }
 
 
Finally by the construction of the velocity field $u = u_q$ in $(0,1-T_q) \times \T^2$ as in \cite[Section 4]{CCS22} we have 
\begin{align} \label{eq:theta-0}
\theta_{0} (1- T_q , \cdot ) - \theta_{\initial} ( (2a_q)^{-1} \cdot ) \|^2_{L^2} \leq a_0^{\epsilon \delta /2}  \,.
\end{align} 
\end{comment}
\end{proof}

As a corollary, the previous theorem yields a sequence of vanishing viscosity solutions $\{v_\nu \}_{\nu >0}$ of the $3D$ forced Navier--Stokes equations such that $v_\nu \rightharpoonup^* v_0$ where $v_0$ is a solution to the $3D$ forced Euler equations but they are distant in $L^p$. 



\begin{corollary} \label{corollary:duchon-anomalous}
 Let  $\alpha \in (0,1)$. There exists  a weak physical solution $v_0 \in L^\infty$ of the $3D$ forced Euler equations with initial datum $v_{\initial } \in C^\infty$ and force $F_0 \in C^\alpha$ for which there exists $\varepsilon >0$ such that  the vanishing viscosity sequence $\{v_{\nu} \}_{\nu >0} \subset C^\infty$  of the $3D$ forced Navier--Stokes equations with forces $F_\nu $ satisfies $\sup_{\nu >0} \| v_\nu \|_{L^\infty} < \infty$, $F_\nu \to F_0 $ in $C^\alpha$  and
$$ \| v_\nu - v_0 \|_{L^p} \geq \varepsilon \qquad \text{ for any } p \in [1, \infty] \,.$$
\end{corollary} 

\begin{proof}
Take $v_0$ the weak physical solution of $3D$ forced Euler given in Proposition \ref{prop:duchon-anomalous}.
We argue by contradiction assuming that there exists $p \in [1, \infty]$ such that
$$ \liminf_{\nu \to 0}   \| v_\nu - v_0 \|_{L^p}  =0$$
and thanks to the $L^\infty$ bounds on $v_\nu$ and $v_0$ we can assume $p=3$.
 Therefore there exists a subsequence $\{ v_{\nu_q} \}_{\nu_q}$ such that $v_{\nu_q} \to v_0$ in $L^3$ and thanks to the fact that along this subsequence all the terms pass into the limit in \eqref{duchon-robert} we have that there exists an anomalous dissipation measure associated to $v_0$ $\mu \in \mathcal{M}((0,2) \times \T^3)$ which is equal to the Duchon-Robert distribution of $v_0 $  (defined in \eqref{duchon-robert}), namely $\mu = \lim_{\nu_q \to 0} \nu_q |\nabla v_{\nu_q}|^2 = \mathcal{D}[v_0]$ in the sense of distributions. However, Proposition \ref{prop:duchon-anomalous} showed that there exists a unique anomalous dissipation measure associated to $v_0$ $\mu$ which is such that $\mu \equiv 0 \neq \mathcal{D}[v_0]$ leading to a contradiction.
\end{proof}




\bibliographystyle{alpha}
 \bibliography{biblio}
 
\end{document}