\documentclass[12pt]{article}
\usepackage[utf8]{inputenc}
\usepackage{amsmath, amssymb, url, float, graphicx, float}
\usepackage{fullpage}
\usepackage[small,bf]{caption}
\usepackage{cite}
\usepackage{xcolor}
\usepackage{subcaption}

\newcommand{\tmin}{\theta^\mathrm{min}}
\newcommand{\tmax}{\theta^\mathrm{max}}
\newcommand{\K}{\mathcal{K}}
\newcommand{\Kb}{\K_\mathrm{\pm1}}
\newcommand{\C}{\mathcal{C}}
\newcommand{\ii}{\mathbf{i}}
\newcommand{\complex}{\mathbf{C}}
\newcommand{\Aop}{\mathcal{A}}
\newcommand{\bop}{\mathbf{b}}
\newcommand{\zop}{\mathbf{z}}
\newcommand{\rp}{r_\mathrm{phys}}
\newcommand{\tbar}{\bar\theta}

\newcommand{\normsq}[1]{\left\|{#1}\right\|_2^2}

\newcommand{\downto}{\downarrow}
\newcommand{\upto}{\uparrow}
\renewcommand\phi\varphi
\newcommand{\eps}{\varepsilon}
\newcommand{\E}{\mathbf{E}}

\newcommand{\X}{\mathcal{X}}
\newcommand{\Y}{\mathcal{Y}}
\newcommand{\D}{\mathcal{D}}

\let\emptyset\varnothing
\let\phi\varphi

\input defs.tex

\title{A Note on the Welfare Gap in Fair Ordering}
\author{
    Theo Diamandis\\
    {\small \texttt{tdiamand@mit.edu}} 
    \and 
    Guillermo Angeris\\
    {\small \texttt{gangeris@baincapital.com}}
}
\date{March 2023}

\newcommand{\TD}[1]{{\color{red}{\bf Theo: #1}}}

\begin{document} 
\maketitle 

\begin{abstract}
    In this short note, we show a gap between the welfare of a traditionally
    `fair' ordering, namely first-in-first-out (an ideal that a number of
    blockchain protocols strive to achieve), where the first transactions to
    arrive are the ones put into the block, and an `optimal' inclusion that is,
    at least approximately, welfare-maximizing, such as choosing which
    transactions are included in a block via an auction. We show this gap is
    positive under a simple model with mild assumptions where we assume
    transactions are, roughly speaking, uniformly drawn from a reasonable
    distribution.
\end{abstract}


\section*{Introduction}
When distorting a market, there is almost always a loss in efficiency.
Typically, this loss manifests as a loss of social welfare: goods are not
allocated to the individuals with the highest valuations (or utilities). In
this short note, we examine ways in which first-in-first-out public blockchains
might distort their block space markets and the resulting decrease in social
welfare. This distortion is most commonly seen in the way in which blockchains
allocate and charge for block space.

Public blockchains generally implement a \emph{fee mechanism} to allocate finite
computational resources across competing transactions. These transactions are
broadcasted by users through the peer-to-peer network and collected in the
mempool, which contains submitted transactions that have not yet been included
on chain. A validator then selects which transactions from the mempool are 
included in the next block, possibly subject to an \emph{inclusion rule}. (We
note that this view is overly simplistic, but it suffices for the purposes of 
this note.) 

Most protocols implement either a fixed fee per transaction, or a dynamic fee
that fluctuates with the demand for block space (\eg, the base fee on
Ethereum). The majority of networks commonly used today implement an
unconstrained inclusion rule: validators can include and order transactions in
a block however they wish. However, many have proposed a first-in-first-out
(FIFO) inclusion rule, where transactions are included in the order in which
they arrive in the mempool, until the block is full. These methods (for
example, those proposed in~\cite{aequitas, themis,pompe, quickorderfairness,
wendy}) add extra rules to consensus to ensure, roughly speaking, that if a
majority of validators receive transaction $A$ before transaction $B$, then
transaction $A$ is guaranteed to be included in the block, if transaction $B$ is
included. (We assume this is achievable and ignore implementation difficulties
of FIFO inclusion in this note.) 

\paragraph{Externalities.} The simplest case to consider is where the
transaction fee is fixed and the transaction inclusion rule is (effectively)
FIFO. The fixed (usually very low) transaction fee, coupled with FIFO
inclusion, encourages users competing for specific transactions (\eg, arbitrage
trades or liquidations) to spam the network in hopes that their transaction is
seen first and therefore included in the block. This spam may result in block
space filled with reverting, `garbage' transactions with low utility. A recent
analysis of Solana transactions suggested that over 50\% of transactions were
failed arbitrage trades~\cite{jito2023}. Clearly, these users are not paying
for the externality they are causing to the network, impacting other users of
the network.

\paragraph{Ok, but let's play fair.} A natural solution to spam is to charge a
dynamic fee for block space; as the demand increases, the fee increases (see
EIP-1559~\cite{eip1559}). In this note we will see that, even in this case,
FIFO inclusion still can dramatically reduce the social welfare generated by the
network. As a simple example, liquidations (which are high-utility
transactions) would not be prioritized, again encouraging searchers to spam the
network to ensure their transaction is included rather than simply paying a
higher fee. This type of challenge has been playing out in the Arbitrum
ecosystem. Although there is a base fee, FIFO inclusion still encourages
spamming the network for transaction prioritization~\cite{snoopy2023}. This is
the case we consider for the remainder of the note.


\section{Building blocks}
In this section, we briefly introduce the block building problem, which is a
special case of the 0-1 knapsack problem~\cite{kellerer2004knapsack}. We then
show a simple lower bound to the optimal value of this problem via a greedy
heuristic, which can be computed efficiently. We use this lower bound in a
later section to bound the gap between the `FIFO' inclusion and the best
possible transactions to include (to maximize social welfare, \ie, net utility).

\paragraph{Transactions and resources.} Following~\cite{diamandis2022dynamic},
we consider a mempool with transactions $j=1, \dots, n$. Each transaction $j$
has utility $\tilde q_j \in \reals_+$ and consumes some amount of a single 
resource, called \emph{gas}, denoted by $a_j \in \reals_+$. Gas has a per-unit 
cost $g \in \reals_+$ in the same num\'eraire as the utility. We denote the gas 
limit per block by $b \in \reals_+$. 

\paragraph{The block building problem.} Using the above definitions,
we then define the \emph{block building problem} as
\begin{equation}
    \label{eq:block-building}
    \begin{aligned}
        & \text{maximize} && q^Tx\\
        & \text{subject to} && a^Tx \le b  \\
        &&& x \in \{ 0,1 \}^n,
    \end{aligned}
\end{equation}
where $q = \tilde q - ga$ is the vector of net transaction utilities, 
after removing gas fees. Here,
the variable $x \in \reals^n$ indicates which transactions are included in the
next block: $x_j = 1$ if transaction $j$ is included in the block and $x_j
= 0$ otherwise. In contrast to~\cite{diamandis2022dynamic}, which considers arbitrarily
complicated constraints, we only consider the constraint that the gas 
total of all included transactions must be
lower than the upper limit. We assume that $q$ is nonnegative (\ie, users do
not submit negative utility transactions) and, for the remainder of the note,
denote the optimal value of problem~\eqref{eq:block-building} by $p^\star$.

\paragraph{Discussion.} Problem~\eqref{eq:block-building} is an instance of the
weighted 0-1 knapsack problem~\cite{kellerer2004knapsack}. Additionally, there
are a number of important issues about how, exactly, to instantiate this
problem and solve it in practice. The first is: how does one elicit the
utilities $q$ from users? This vector $q$ can come from certain mechanisms
(such as second-price auctions~\cite{roughgarden2016twenty}), though
implementing these on chain remains an open research problem
(see~\cite{chitra2023credible,MEV_Suave, chung2021foundations} and references
therein). We do not deal with this question in this note and assume that there
is some mechanism for receiving (or approximating) these utilities. There are
other important issues to raise, such as the fact that this problem is NP-hard
to solve~\cite[App.\ A]{kellerer2004knapsack}, but we note that this is a
worst-case guarantee and that problems of this form are commonly solved to
optimality in practice (using software such as Gurobi~\cite{gurobi}). However,
we will use a relatively simple bound for the optimal value of this problem,
which only depends on a heuristic solution that can always be efficiently
computed. All results apply to this heuristic and therefore to any heuristics
that do at least as well.

\paragraph{FIFO inclusion.} In FIFO inclusion, on the other hand, transactions
are included `as they arrive', until the block reaches capacity. We assume that
the transactions $j = 1, \dots, n$ are ordered by arrival time, from earliest
to latest. (Whether FIFO inclusion is exactly possible in a decentralized
setting has been a subject of great debate~\cite[Theorem 1.2]{kelkar2020order},
but we treat is as an `ideal' scenario that many protocols are trying to
achieve.) In this scenario, the vector $x$ must have the form $x = (\ones_k,
0)$ for some $k$. We can define the utility from the FIFO transaction
explicitly as the highest utility over over all vectors of this form:
\[
    p^\mathrm{FIFO} = \max\{q^T(\ones_k, 0) \mid a^T(\ones_k, 0) \le b,\; k = 0, 1, \dots, n \}.
\]
A natural question is `what is difference between the total utility of an
optimal block packing (\ie, a solution to problem~\eqref{eq:block-building})
and that of a FIFO packing?'. In other words, we want to find a bound on the
gap
\[
    \Gamma = p^\star - p^\mathrm{FIFO}.
\] 
Note, of course, that $\Gamma \ge 0$, since any FIFO solution is feasible for
the original problem~\eqref{eq:block-building}, though, as we will later show,
the quantity $\Gamma$ will be nonzero (and potentially large) `in expectation'
under certain weak assumptions about the distribution of the transactions.

\subsection{A heuristic for block building}
Problem~\eqref{eq:block-building} is a weighted knapsack problem, which has
been well studied (see~\cite{kellerer2004knapsack} and references therein) 
and can be approximately solved by a number of simple heuristics with relatively
tight guarantees. We present a very basic overview of one important heuristic
and its corresponding proof of tightness here. We will then use this heuristic,
and corresponding bound, to approximate $p^\star$ and show that $\Gamma$ is large
in a number of common scenarios.

\paragraph{Greedy heuristic.} The simplest (and most common) heuristic is to
relax the integrality constraint ($x \in \{0, 1\}^n$) in
problem~\eqref{eq:block-building} to an interval constraint, to
get the linear programming problem:
\begin{equation}\label{eq:relaxation}
    \begin{aligned}
        & \text{maximize} && q^Tx\\
        & \text{subject to} && a^Tx \le b\\
        &&& 0 \le x \le \ones,
    \end{aligned}
\end{equation}
with variable $x \in \reals^n$ and the same problem data as~\eqref{eq:block-building}.
We write the optimal value of this relaxed problem as $r^\star$. Note that, since
every $x$ that is feasible for~\eqref{eq:block-building} is feasible for its
relaxation~\eqref{eq:relaxation}, we have that 
\[
    r^\star \ge p^\star \ge p^\mathrm{FIFO}.
\]

\paragraph{Solution.} Problem~\eqref{eq:relaxation} is a linear
programming problem that is easy to (computationally) solve in practice. In fact, it is
possible to write a closed-form solution $x^\star$ to~\eqref{eq:relaxation}. 
To see this, start with the equivalent problem,
\[
    \begin{aligned}
        & \text{maximize} && \bar q^Ty\\
        & \text{subject to} && \ones^Ty \le 1\\
        &&& 0 \le y_i \le b/a_i \quad i=1, \dots, n.
    \end{aligned}
\]
Here, the problem data are the \emph{efficiencies} $\bar q_i = bq_i/a_i$ for
each transaction $i=1, \dots, n$, while the rest of the definitions are
identical to the original relaxation~\eqref{eq:relaxation}. In this problem, we
have simply done a variable substitution $y_i = a_ix_i/b$ in the relaxed
problem~\eqref{eq:relaxation}. We can think of the $\bar q_i$ as a measure of
the `welfare-per-unit-resource' for transaction $i$. 

A solution to this problem is then very simple to construct: sort the $\bar
q_i$ in nondecreasing order (say, with indices $\tau_1, \dots, \tau_n$) then
set $y_{\tau_i}^\star = b/a_{\tau_i}$, in order, for each $i$. If there is no
index at which the bound is met, an optimal solution is to set
$y_{\tau_i}^\star = b/a_{\tau_i}$ for $i=1, \dots, n$. (It is \emph{the}
optimal solution if $\bar q > 0$.) If there is an index at which the bound is
met or surpassed, say, index $\tau_k$, then we choose $y_{\tau_k}^\star$ to be
the largest possible value $\le b/a_{\tau_k}$ such that the bound is met at
equality. (There may be many solutions if there are many entries of $\bar q$
with value $\bar q_{\tau_k}$, in which case any entry suffices.) We can then
recover a solution $x^\star$ for problem~\eqref{eq:relaxation} from this
optimal $y^\star$ by using the substitution provided above, $x_i^\star =
by_i^\star/a_i$ for $i=1, \dots, n$. Note that this solution, $x^\star$, is
fractional (\ie, has $0 < x_i^\star < 1$ for some $i$) in at most one entry.

\paragraph{Discussion.} The solution above suggests a reasonable heuristic:
since at most one entry of an optimal solution $x^\star$
to~\eqref{eq:relaxation} (provided above) is fractional, we can simply round
the (at most one) fractional entry down to zero, to receive, say, $x^0 \in \{0,
1\}^n$. Note that $x^0$ is feasible for the original
problem~\eqref{eq:block-building}, and we denote its objective value as $p^0 =
q^Tx^0$. By definition, we have that $p^0 \le p^\star$, since $x^0$ is
feasible. Surprisingly, we also have that $p^0 \ge Cp^\star$ for some constant
$C \le 1$. In other words, this heuristic gives a feasible point with objective
value $p^0$, which is actually `close' to the true optimal value $p^\star$.

\paragraph{Bound.} If the transactions all consume gas at most $a_i \le b/m$ for some
integer $m > 1$, then we have that
\[
    \frac{m}{m-1}p^0 \ge p^\star.
\]
To see this, note that the solution to the relaxation~\eqref{eq:relaxation}
described above satisfies $a^Tx^\star = b$. By definition, there is at most one
nonintegral entry, with gas $a_i \le b/m$ so the rounded solution must satisfy
$a^Tx^0 \ge (m-1)b/m$. Let $i$ be the entry with largest $q_i$ such that $x_i^0 = 0$
(\ie, let $i$ be the highest-utility transaction not included in the heuristic solution),
then
\[
    \frac{q_i}{a_i} \le \frac{q^Tx^0}{a^Tx^0} = \frac{p^0}{a^Tx^0} \le \frac{m}{(m-1)b}p^0.
\]
The first inequality follows from the fact that, for
positive $t, u, v, w$ we have
\[
    \min\left\{\frac{t}{v}, \frac{u}{w}\right\} \le \frac{t + u}{v + w},
\]
and $q_i/a_i$ is no larger than the entries $j$ with $x_j^0 = 1$, by construction of $x^0$.
The equality follows by definition, and the last inequality comes from the
previous discussion.
Finally, using the fact that $a_i \le b/m$ again and the above, we have that
\[
    \frac{m}{b}q_i \le \frac{q_i}{a_i} \le \frac{m}{(m-1)b}p^0,
\]
or, simplifying,
\begin{equation}\label{eq:bound-q}
    q_i \le \frac{p^0}{m-1}.
\end{equation}
Since there is at most one transaction partially included by the relaxation,
and this transaction has utility no larger than $q_i$ (by definition of $i$),
then
\[
    p^0 + q_i \ge r^\star \ge p^\star \ge p^0,
\]
where, from before, $r^\star$ is the optimal value of the
relaxation~\eqref{eq:relaxation}. Combining this with~\eqref{eq:bound-q}, we then have
that
\[
    p^0 \le p^\star \le  \frac{m}{m-1}p^0.
\]
In other words, the heuristic solution, with optimal value $p^0$, is very close
to the optimal value of exactly solving problem~\eqref{eq:block-building},
which is NP-hard, whenever $m$ is somewhat large. For more information on
similar approximations, see~\cite[\S6]{kellerer2004knapsack}.

\section{What's the gap?}
Given the discussion above, it makes sense to consider a model where transactions
are drawn from some distribution, which we will characterize in terms of the
efficiencies. We will also assume that there are minimum and maximum gas limits 
for any transaction, denoted by $B^-$ and $B^+$ respectively, so that 
$B^- \le a_i \le B^+$ for $i = 1, \dots, n$. Clearly, the minimum and maximum
transaction sizes imply a maximum and minimum number of transactions we can
include in each block, given by $b/B^-$ and $b/B^+$, respectively. From this,
we can construct a lower bound of the optimal block utility and an upper bound 
of the FIFO-inclusion block utility. In this section, we will construct these
bounds and discuss when there is a strictly positive gap between the utilities.


\paragraph{Lower bound.}
First, we will lower bound the utility of a block that was packed using the greedy 
heuristic (which we know is close to optimal).
Denote the number of transactions included in the block using the greedy
heuristic by $\bar k$. We define the average utility of these transactions by
\[
    q^+ = \frac{1}{\bar k}\sum_{i=1}^{\bar k} q_{\tau_i}.
\]
Since $\bar k \ge b/B^+$, 
a lower bound for the utility of the greedily packed block is
\[
    L = \bar k q^+  \ge \frac{b}{B^+}q^+.
\]

\paragraph{Upper bound.}
Now we will upper bound the expected utility of a FIFO-inclusion block, assuming the
arrival time of the transactions is random; \ie, we get a uniformly random
permutation as the order of the arrival of the transactions. We know that at
most $b/B^-$ transactions can be included in a block and that the $n$
transactions are uniformly randomly permuted, which means that the expected
utility of FIFO, which we will call $U$, is no larger than
\[
    U \le \frac{b}{B^-}\frac{1}{n}\sum_{i=1}^n q_i.
\]
If we define $q^-$ as the average utility for the transactions not included by
the greedy heuristic,
\[
    q^- = \frac{1}{n-\bar k}\sum_{i=\bar k+1}^{n} q_{\tau_i},
\]
then we can write
\[
    \frac{1}{n}\sum_{i=1}^n q_i = \frac{\bar k}{n}q^+ + \left(1-\frac{\bar k}{n}\right)q^-.
\]
(We can view $q^-$ as, roughly speaking, the average utility of the `tail' of
transactions, as the efficiencies get small.) This means the average FIFO
utility is bounded from above by
\[
    U \le \frac{b}{B^-}\left(\frac{\bar k}{n}q^+ + \left(1-\frac{\bar k}{n}\right)q^-\right).
\]

\paragraph{What's the gap?}
We now seek to characterize the gap between these two bounds, which
gives us a lower bound on the gap between FIFO inclusion and the optimal,
since
\[
    \Gamma \ge L - U \ge \frac{b}{B^+}q^+ - \frac{b}{B^-}\left((q^+ - q^-)\frac{\bar k}{n} + q^- \right).
\]
We now look for a basic condition for when the right-hand-side of this
inequality is positive, which would imply that the gap $\Gamma > 0$. Rearranging,
it is easy to see that the right hand side is positive whenever
\begin{equation}\label{eq:main-bound}
    q^+\left( 1 - \frac{\bar k \eta}{n}\right) >   
    \eta q^-\left(1 - \frac{\bar k}{n}\right),
\end{equation}
where $\eta = B^+ / B^-$ is the ratio between the largest and smallest possible
%\TD{Want to use $\eta$ instead of $B^-$ and $B^+$ in~\eqref{eq:main-bound}?}
transaction. If
\begin{equation}\label{eq:simple-bound}
    q^+ > \eta q^-,
\end{equation}
then, as the number of outstanding transactions $n$ becomes large relative to
the number of greedily-chosen transactions, $\bar k$, while the average
utilities stay roughly constant, we get that $\Gamma > 0$.
We may also wish to consider the ratio of optimal to FIFO block utility. From
the preceding discussion, we have that
\[
    \frac{p^\star}{p^\mathrm{FIFO}} \ge \frac{L}{U} \ge
    \frac{q^+/\eta}{(q^+ - q^-)\frac{\bar k}{n} + q^-},
\]
which is strictly greater than one under the positive gap condition~\eqref{eq:main-bound}.

\paragraph{Discussion.}
We note that the bounds derived in~\eqref{eq:main-bound}
and~\eqref{eq:simple-bound} are actually fairly loose in practice (as we will
see next). The reason such bounds are loose is primarily due to the worst-case
assumptions made in the lower bound, where we only use the fact that $\bar k
\ge b/B^+$ and the upper bound, where we use the fact that $b/B^-$ bounds the
maximum possible number of transactions that can be included by FIFO. Both of
these cases can be quite loose if there are only a few transactions close to
the bounds, relative to the rest of the distribution. As this is just a note,
we only seek `qualitatively reasonable' behavior from the bounds in order to
gain intuition, but not tight constants. An interesting avenue for future
research would be to either tighten the bounds given here or generalize them to
the multidimensional setting considered in~\cite{diamandis2022dynamic}.


\subsection{Special cases and experiments} 
From inequalities~\eqref{eq:main-bound} and~\eqref{eq:simple-bound}, we can
deduce some conditions under which we are guaranteed to have a large gap
between the utilities of the optimal and FIFO-inclusion blocks. We also show 
some basic numerical experiments which suggest that, in practice, the gap may be 
far larger than the one suggested by our bounds.

\paragraph{Sharp distribution.} First, if the distribution over the
efficiencies is sufficiently sharp (\eg, if are a small number of transactions
with very large utility and equal or lower gas relative to the others) then
$q^+ \gg q^-$ for reasonable block sizes. This situation is common in practice
when many similar transactions are submitted but only one can be executed
profitably, as is the case in many MEV opportunities, including liquidations and
DEX arbitrage. (As discussed in the introduction, this distribution is not
rare, even in practice.) Since only one of these transactions can be executed
profitably, the others will revert and, therefore, have non-positive utility. 
Note that this situation corresponds to these efficiencies being drawn from a
heavy-tailed probability distribution.

\paragraph{Equal size transactions.} Another important case is if the
transactions have roughly the same size, \ie, $\eta \approx 1$. From
bound~\eqref{eq:main-bound}, we see that any distribution of utilities that is
not flat will create a gap between the optimal and FIFO-inclusion blocks.
Intuitively, this is straightforward to see: there is always a chance that
high-utility transactions will not be included in the block, and, since all
transactions consume roughly the same gas, this is a strict loss in total
utility.


\paragraph{Simple experiments.}
We plot our bound on the utility gap and the realized 
gap for several transaction utility distributions, shown in figure~\ref{fig:dists}
(we provide definitions in appendix~\ref{app:dists}).
All transaction sizes are drawn uniformly at random from the interval $[1, 3]$. 
We generate $1000$ transactions in the mempool and vary the block size from $20$
to $2000$ gas (for an average of $10$ to all $1000$ transactions per block).
We run $100$ trials for each block size.
All experiments are done in the Julia programming language~\cite{bezanson2017julia}
and we use the \texttt{Distributions.jl} package~\cite{distributions.jl}.
Code is available at
\begin{center}
    \texttt{https://github.com/bcc-research/fifo-note}
\end{center}
\begin{figure}[h!]
    \captionsetup[sub]{font=scriptsize}
    \centering
    \begin{subfigure}[t]{0.43\linewidth}
        \centering
        \includegraphics[width=\columnwidth]{code/figs/dists.pdf}
        \caption{Light-tailed utility distributions.}
    \end{subfigure}
    \hfill
    \begin{subfigure}[t]{0.43\linewidth}
        \centering
        \includegraphics[width=\columnwidth]{code/figs/heavy_dists.pdf}
        \caption{Heavy-tailed utility distributions.}
    \end{subfigure}
    \caption{Distributions used for experiments. Light-tailed utility
        distributions lead to `flat' efficiency distributions, while heavy-tailed
        ones lead to `sharp' efficiency distributions.}
    \label{fig:dists}
\end{figure}
\begin{figure}[h!]
    \captionsetup[sub]{font=scriptsize}
    \centering
    \begin{subfigure}[t]{0.32\linewidth}
        \centering
        \includegraphics[width=\columnwidth]{code/figs/gap-exponential.pdf}
        \caption{$q_i \sim \mathrm{Exponential}(2.5)$}
    \end{subfigure}
    \hfill
    \begin{subfigure}[t]{0.32\linewidth}
        \centering
        \includegraphics[width=\columnwidth]{code/figs/gap-LogNormal.pdf}
        \caption{$q_i \sim \mathrm{LogNormal}(1, 1)$}
    \end{subfigure}
    \hfill
    \begin{subfigure}[t]{0.32\linewidth}
        \centering
        \includegraphics[width=\columnwidth]{code/figs/gap-Rayleigh.pdf}
        \caption{$q_i \sim \mathrm{Rayleigh}(1)$}
    \end{subfigure}
    \caption{Gap between FIFO and optimal block packing, as a function of block
        size, for flat (light-tailed) utility distributions. The solid line indicates the
        mean over $100$ trials, the ribbon indicates the $\pm \sigma$ range,
        and the dashed line indicates the bound~\eqref{eq:main-bound}.}
    \label{fig:gap}
\end{figure}
\begin{figure}[h!]
    \captionsetup[sub]{font=scriptsize}
    \centering
    \begin{subfigure}[t]{0.43\linewidth}
        \centering
        \includegraphics[width=\columnwidth]{code/figs/gap-Levy.pdf}
        \caption{$q_i \sim \mathrm{Levy}(0, 1)$}
    \end{subfigure}
    \hfill
    \begin{subfigure}[t]{0.43\linewidth}
        \centering
        \includegraphics[width=\columnwidth]{code/figs/gap-Pareto.pdf}
        \caption{$q_i \sim \mathrm{Pareto}(0.5)$}
    \end{subfigure}
    \caption{Gap between FIFO and optimal block packing, as a function of block size,
    for sharp (heavy-tailed) distributions. The solid line indicates the mean over $100$
    trials, the dotted line indicates the median, and the dashed line indicates 
    the bound~\eqref{eq:main-bound}. Since these distributions have infinite 
    variance (and mean), $\pm \sigma$ ranges are not displayed.}
    \label{fig:gaps-heavy}
\end{figure}
The empirical gaps for flat distributions (\ie, those where we do not
expect transactions with very large utility, relative to the other transactions) 
are shown in figure~\ref{fig:gap}.
We see that when the block size is small relative to the number of transactions,
there is a significant gap between the utilities of the optimal and FIFO-inclusion
blocks. Furthermore, this gap is quite a bit worse than what is predicted by our
bound~\eqref{eq:main-bound}. (This difference is not surprising, given the 
discussion above.) For sharp distributions, the gap, shown
in figure~\ref{fig:gaps-heavy}, is quite a bit larger. These distributions
more closely resemble competitive MEV opportunities.

\bibliographystyle{alpha}
\bibliography{citations.bib}

\appendix
\section{Distribution definitions} \label{app:dists}

\paragraph{Flat distributions.}
We consider three distributions which yield relatively flat efficiencies as they
have a light (\ie,
sub-exponential or sub-Gaussian) tail. 
These distributions have a finite mean and variance, so
we expect samples to cluster tightly together.
% Exp
The exponential distribution with parameter $\theta$ has probability density 
function
\[
f(x; \theta) = \frac{1}{\theta} e^{-\frac{x}{\theta}}, \quad x > 0.
\]
% LogNormal
The log normal distribution with parameters $\mu$ and $\sigma$ has probability
density function
\[
    f(x; \mu, \sigma) = \frac{1}{x \sqrt{2 \pi \sigma^2}}
\exp \left( - \frac{(\log(x) - \mu)^2}{2 \sigma^2} \right),
\quad x > 0.
\]
% Rayleigh
The Rayleigh distribution with parameter $\sigma$ has probability density function
\[
    f(x; \sigma) = \frac{x}{\sigma^2} e^{-\frac{x^2}{2 \sigma^2}}, \quad x > 0.
\]

\paragraph{Sharp distributions.}
We consider two utility distributions which have heavy tails.
These distributions have an infinite mean and variance, and we expect the
presence of large outliers when sampling, yielding a relatively sharp
distribution of the efficiencies.
% Levy
The Levy distribution with parameters $\mu$ and $\sigma$ has probability density
function
\[
    f(x; \mu, \sigma) = \sqrt{\frac{\sigma}{2 \pi (x - \mu)^3}}
    \exp \left( - \frac{\sigma}{2 (x - \mu)} \right), \quad x > \mu.
\]
% Pareto
The Pareto distribution with parameter $\alpha$ has probability density function
\[
    f(x; \alpha) = \frac{\alpha }{x^{\alpha + 1}}, \quad x \ge 1.
\]


\end{document}
