\documentclass[runningheads]{llncs}
\usepackage[T1]{fontenc}
\usepackage{graphicx}
\usepackage{amsmath,amssymb}
\usepackage{booktabs}
\usepackage{pifont}
\usepackage{multirow}
\usepackage{makecell}
\usepackage{algorithm}
\usepackage{algpseudocode}

% If you use the hyperref package, please uncomment the following two lines
% to display URLs in blue roman font according to Springer's eBook style:
%\usepackage{color}
%\renewcommand\UrlFont{\color{blue}\rmfamily}

% \usepackage[finalizecache=true,cachedir=minted-cache]{minted}
\usepackage[frozencache=true,cachedir=minted-cache]{minted}

%% commenting macros

% use this to hide larger blocks of material:
\newcommand{\hide}[1]{}

\usepackage{xcolor}
\usepackage{amssymb}


\colorlet{nfcolor}{teal}
\definecolor{tmcolor}{rgb}{0.2,0.6,0.6}
\definecolor{todocolor}{rgb}{0.749,0.212,0.047}
\definecolor{changedcolor}{rgb}{0.42,0.27,0.57}
\newcommand{\nbc}[3]{
		{\colorbox{#3}{\bfseries\sffamily\scriptsize\textcolor{white}{#1}}}
		{\textcolor{#3}{\sf\small$\blacktriangleright$\textit{#2}$\blacktriangleleft$}}
%		{\textcolor{#3}{\sf\small\textit{#2}}}
}

\newcommand{\nf}[1]{\nbc{NF}{#1}{nfcolor}}
\newcommand{\tm}[1]{\nbc{TM}{#1}{tmcolor}}
\newcommand{\Todo}[1]{\nbc{TODO}{#1}{todocolor}}
\newcommand{\changed}[1]{\nbc{CHANGED}{#1}{changedcolor}}

% Use this to temporarily hide reviewing comments, todos, etc.:
%  \renewcommand{\tm}[1]{}
%  \renewcommand{\nf}[1]{}
%  \renewcommand{\todo}[1]{}


\begin{document}
\definecolor{ourgreen}{RGB}{152,190,109} 
\definecolor{ourred}{RGB}{195,117,131} 
%
\title{WikiCoder: Learning to Write Knowledge-Powered Code}
%\title{Contribution Title\thanks{Supported by organization x.}}
%
%\titlerunning{Abbreviated paper title}
% If the paper title is too long for the running head, you can set
% an abbreviated paper title here
%
\author{Théo Matricon\inst{1}\orcidID{0000-0002-5043-3221} \and \\
Nathana{\"e}l Fijalkow\inst{1,2}\orcidID{0000-0002-6576-4680} \and \\
Ga{\"e}tan Margueritte\inst{1}\orcidID{0009-0006-1742-4688}}

\authorrunning{T. Matricon, N. Fijalkow, G. Margueritte}
% First names are abbreviated in the running head.
% If there are more than two authors, 'et al.' is used.

\institute{CNRS, LaBRI, University of Bordeaux, France \email{\{theomatricon,gamargueritte\}@gmail.com}\and
MIMUW, University of Warsaw, Poland
\email{nathanael.fijalkow@gmail.com}}

\maketitle              % typeset the header of the contribution
%

\begin{abstract}
We tackle the problem of automatic generation of computer programs from a few pairs of input-output examples. The starting point of this work is the observation that in many applications a solution program must use external knowledge not present in the examples: we call such programs knowledge-powered since they can refer to information collected from a knowledge graph such as Wikipedia.
This paper makes a first step towards knowledge-powered program synthesis. We present WikiCoder, a system building upon state of the art machine-learned program synthesizers and integrating knowledge graphs. We evaluate it to show its wide applicability over different domains and discuss its limitations. WikiCoder solves tasks that no program synthesizers were able to solve before thanks to the use of knowledge graphs, while integrating with recent developments in the field to operate at scale.

\keywords{Program Synthesis \and Knowledge Graphs \and Code generation.}
\end{abstract}

\section{Introduction}
\section{Introduction}

The increasing complexity of source code poses a key challenge to the reliability of large-scale software systems. Software bugs in these systems can lead to safety issues~\cite{bug_safety} for users around the world as well as cause non-negligible financial losses~\cite{bug_loss}. As such, developers have to spend a large amount of time and effort on bug fixing. Consequently, \aprfull (\apr), designed to automatically generate patches to fix software bugs, has attracted wide attention from both academia and industry~\cite{long2016prophet, legoues2012genprog, long2015spr, lou2020can, tufano2018empstudy}. 


To achieve \apr, one popular approach is known as Generate-and-Validate (G\&V)~\cite{qi2015gv, ghanbari2019prapr, lou2020can, le2016hdrepair, legoues2012genprog, wen2018capgen, hua2018sketchfix, martinez2016astor, koyuncu2020fixminder, liu2019tbar, liu2019avatar}, which is typically based on the following pipeline: First, fault localization techniques~\cite{wong2016fl, abreu2007ochiai, zhang2013injecting, papadakis2015metallaxis, li2019deepfl, li2017transforming} are applied to determine the suspicious locations in programs where bugs are likely to exist. Then, the buggy locations are used by the \apr tools to generate a list of patches that replace buggy lines with correct lines. Afterward, each patch is validated against the original test suite to identify any \emph{plausible patches} (i.e., passing all tests in the test suite). Finally, to determine the \emph{correct patches}, developers examine the list of plausible patches to see if any of them can correctly fix the bug. 

Traditional \apr tools can mainly be categorized into heuristic-based~\cite{legoues2012genprog, le2016hdrepair, wen2018capgen}, constraint-based~\cite{mechtaev2016angelix, le2017s3, demacro2014nopol, long2015spr} and \template~\cite{ghanbari2019prapr, hua2018sketchfix, martinez2016astor, liu2019tbar, liu2019avatar}. Among these traditional tools, \template \apr tools~\cite{ghanbari2019prapr, liu2019tbar, benton2020effectiveness} have been able to achieve state-of-the-art results. \Template \apr tools typically leverage pre-defined templates (e.g., adding a nullness check) for bug fixing. However, since these fix templates are typically handcrafted, the number and types of bugs they are able to fix can be limited. 



To address the limitations of traditional \apr, researchers have proposed various \learning \apr tools~\cite{li2020dlfix, chen2018sequencer, jiang2021cure, lutellier2020coconut, zhu2021recoder, ye2022rewardrepair} based on the \nmtfull (\nmt) architecture~\cite{sutskever2014mt} where the input is the buggy code snippets and the goal is to translate the buggy code snippets into a fixed version. To accomplish this, \learning \apr tools require supervised training datasets with pairs of both buggy and fixed code snippets in order to learn how to perform this translation step. These training data are usually obtained by mining historical bug fixes using heuristics/keywords~\cite{dallmeier2007benchmark}, which can be imprecise for identifying bug-fixing commits; even the actual bug-fixing commits can include irrelevant code changes, leading to further pollution in the dataset~\cite{xia2022alpharepair}.
% 
Moreover, it can be hard for such \apr tools to generalize and fix bug types unseen during training. 



To better leverage recent advances in \plmfull{s} (\plm{s}), researchers~\cite{xia2022alpharepair, xia2023repairstudy, kolak2022patch, prenner2021codexws} have directly applied \plm{s} to generate patches without bug-fixing datasets. These \llm-based \apr tools work by either directly generating a complete code function~\cite{prenner2021codexws, xia2023repairstudy} or predict/infill the correct code snippet given its surrounding context~\cite{xia2022alpharepair, xia2023repairstudy}. By directly using \llm{s} that are pre-trained on billions of open-source code snippets, \llm-based \apr tools can achieve state-of-the-art performance on many repair datasets~\cite{xia2022alpharepair}. 


% 
%
%

Traditional \apr tools have long used the insight of the \emph{plastic surgery hypothesis}~\cite{barr2014plastic} where it states that the code ingredients to fix a bug already exist within the same project. Traditional \apr tools have manually designed pattern-~\cite{ghanbari2019prapr, saha2017elixir} or heuristic-based~\cite{jiang2018simfix, legoues2012genprog} approaches to finding and using such relevant code ingredients to generate fixes for bugs. However, the plastic surgery hypothesis has been largely ignored in \llm-based \apr. In fact, \llm provides a unique opportunity to fully automate the plastic surgery hypothesis idea via fine-tuning (learning project-specific information via model updates from the buggy project) and prompting (directly providing relevant code ingredients to the model), and make it directly applicable to different languages (since the \llm{s} are typically multi-lingual).%
Moreover, despite the intensive manual efforts involved, traditional \apr tools still cannot fully leverage project-specific information due to large search space for leveraging/composing existing code ingredients. In contrast, the project-specific information can effectively leveraged by \llm{s} due to their power in code understanding/vectorization, e.g., even partial/imprecise information may still guide \llm{s} in correct patch generation!
 To this end, we ask the question: \emph{How useful is the plastic surgery hypothesis in the era of \plm{s}}?








\mypara{Our Work.} To answer the question, we present \ourtech{\xspace} -- a \llm-based approach that automatically utilizes the plastic surgery hypothesis by systematically combining multiple fine-tuning and prompting strategies for \apr. \ourtech fine-tunes \plm{s} using two novel domain-specific training strategies: \textbf{\epfinetune} -- we fine-tune using the original buggy project by aggressively masking out a high percentage of tokens, which allows \plm to learn project-specific code tokens and programming styles; and \textbf{\rofinetune} -- which only masks out a single continuous code sequence per training sample, allowing the model to get used to the final \csapr task of predicting a single continuous code sequence. Furthermore, we directly leverage the ability for \plm{s} to understand natural language instructions and introduce a novel prompting strategy, \textbf{\idprompting}, which uses information retrieval and static analysis to obtain a list of relevant identifiers for the buggy lines. While such relevant identifiers are critical for fixing some difficult bugs, they may not be seen by the \llm during inference due to limited context window size. Through the use of prompting, we directly tell the model to use these extracted identifiers (relevant code ingredients) to generate the correct code. Finally, to perform repair, we combine all four model variants (including the base model, both fine-tuned models and the base model with prompting) for the final repair.





While our insight of leveraging the plastic surgery hypothesis for \llm-based \apr is generalizable across different types of \plm{s}, to implement \ourtech, we choose a recent \plm{\xspace}, \ctfive~\cite{wang2021codet5}, which is pre-trained on millions of open-source code snippets. \ctfive is an encoder-decoder model trained using \mspfull (\msp) objective where a percentage of tokens are masked out and each continuous masked token sequence is referred to as a masked span. Also, although we only extract relevant identifiers from the current buggy project (since this paper focuses on the plastic surgery hypothesis), our work can be easily extended to obtain other code information (such as relevant statements or functions) from other sources, such as  the massive pre-training corpora~\cite{husain2020codesearchnet} or historical bug-fixing datasets~\cite{jiang2019infer}, which can provide more coding knowledge for \llm{s}. Besides, although we mainly focus on using traditional string comparison algorithms for information retrieval in this paper, these techniques can be easily replaced by other frequency-based retrieval~\cite{robertson2009probabilistic} and neural search (or embedding-based search)~\cite{reimers2019sentence}.
  In summary, this paper makes the following contributions:


%


\begin{itemize}[noitemsep, leftmargin=*, topsep=0pt]
    \item \textbf{Dimension.} This paper is the first to revisit the important plastic surgery hypothesis in the era of \llm{s}. It opens up a new dimension for \llm-based \apr to incorporate previously neglected information from the buggy project itself to boost \apr performance. Furthermore, it demonstrates the promising future of retrieval-based prompting for modern \llm-based \apr.
    \item \textbf{Implementation.} We implement \ourtech based on the recent \ctfive model. We augment the model using two novel fine-tuning strategies: \epfinetune and \rofinetune, along with a novel prompting strategy based on information retrieval and static analysis: \idprompting. We combine the patches generated by all four models together and perform patch ranking to speed up \apr.% 
    \item \textbf{Evaluation Study.} We conduct an extensive evaluation against state-of-the-art \apr tools. On the widely studied \dfj 1.2 and 2.0 datasets~\cite{just2014dfj}, \ourtech is able to achieve the new state-of-the-art results of 89 and 44 correct bug fixes (15 and 8 more than best baseline) respectively.  Furthermore, we perform a broad ablation study to justify our design. \ourtech demonstrates for the first time that the plastic surgery hypothesis can substantially boost \llm-based \apr and advance state-of-the-art \apr, while being fully automated and general. Moreover, even partial/imprecise code ingredients may still effectively guide \llm{s} for \apr!
\end{itemize}



%%%%%%%%%%%%%%%%%%%%%%%%%%%
\section{Knowledge-Powered Programming by Example}

\subsection{Objectives}
Programming by example, or inductive synthesis, is the following problem: given a few examples, construct a program satisfying these examples. This particular setting in program synthesis, where the user gives a very partial specification, has been extremely useful and successful for automating tasks for end users, the prime example being FlashFill for performing string transformations in Excel~\cite{Gulwani2011}.

When considering knowledge-powered programming by example, we do not change the problem, only the solution: instead of classical programs performing syntactic manipulations, we include knowledge-powered programs. To illustrate the difference, let us consider the following two tasks.
\begin{minted}{python}
f("Paris") = "I love P"
f("Berlin") = "I love B"

g("Donald Knuth") = "DK is American"
g("Ada Lovelace") = "AL is English"
\end{minted} 
The first is a classical program synthesis task in the sense that it is purely syntactical, it can be solved with a two-line program concatenating ``I love '' with the first letter of the input.
On the other hand, the second requires some knowledge about the input individuals, here their nationality: one needs a knowledge-powered program to solve this task.
Since almost all program synthesis tools perform only syntactical manipulations of the examples (we refer to the related work section for an in-depth discussion), they cannot solve the second task.

Knowledge-powered programming by example goes much beyond query answering: the goal is not to answer a particular query, but to produce a program able to answer that query for any input. This is computationally and conceptually a much more difficult problem.

For concreteness we introduce some terminology about knowledge graphs, and refer to Figure~\ref{fig:knowledge_graph} for an illustration.
Nodes are called entities, and edges are labelled by a relation. Entities are arranged into classes: ``E. Macron'' belongs to the class of people, and ``France'' to the class of countries. The classes and relations are constrained by ontologies, which define which relations can hold between entities.
The de facto standard for querying knowledge graphs is through \texttt{SPARQL} queries, which is a very powerful and versatile query language.

\begin{figure}[ht]
   \centering
   \includegraphics[width=0.8\textwidth]{knowledge_graph}
   \caption{Illustration of part of a knowledge graph.}
   \label{fig:knowledge_graph}
\end{figure}

\subsection{Milestones}\label{sec:milestones}
We identify three independent ways in which semantical information can be used.
They correspond to different stages for solving a task:
\begin{itemize}
	\item \textit{preprocessing:} the first step is to extract entities from the examples;
	\item \textit{relating:} the second step is to relate entities in the knowledge graph;
	\item \textit{postprocessing:} the third step is to process the information found in the knowledge graph.
\end{itemize}

\paragraph*{How much preprocessing to extract entities?}
Let us consider two tasks.
\begin{minted}{python}
f("Aix, Paris, Bordeaux") = "Paris"
f("Hamburg, Berlin, Munich") = "Berlin"

g("President Obama") = "Obama"
g("Prime Minister de Pfeffel Johnson") = "de Pfeffel Johnson"
\end{minted}
In the function \texttt{f} the goal is to extract the second word as separated by commas: this is a purely syntactic operation.
In the function \texttt{g} we need to remove the job title from the input: this requires semantical knowledge, for instance it is not enough to use neither the second nor the last word.

\paragraph*{How complicated is the relationship between entities?}
We examine two more tasks.
\begin{minted}{python}
f("Paris") = "France is beautiful"
f("Berlin") = "Germany is beautiful"

g("Paris") = "Phone country code: 33"
g("Berlin") = "Phone country code: 49"
\end{minted}
The function \texttt{f} relates two entities: a city and a country. One can expect that the knowledge graph includes the relation \texttt{CapitalOf}, which induces a labelled edge between ``Paris'' and  ``France'' as well as ``Berlin'' and ``Germany'' (as in Figure~\ref{fig:knowledge_graph}). Note that it could also be the relation \texttt{CityOf}.
More complex, the function \texttt{g} requires crossing information: indeed to connect a city to its country code, it is probably required 
to compose two relations: \texttt{CapitalOf} and \texttt{PhoneCode}. In other words, the entities are related by a path of length $2$ in the knowledge graph, that we write \texttt{CapitalOf-PhoneCode}.
More generally, the length of the path relating the entities is a measure of complexity of a task.

\paragraph*{How much postprocessing on external knowledge?}
Let us look again at two tasks.
\begin{minted}{python}
f("Paris") = "Country's first letter: F"
f("Berlin") = "Country's first letter: G"

g("President Obama") = "B@r@ck"
g("Prime Minister Johnson") = "B0r1s"
\end{minted}
For the function \texttt{f}, the difficulty lies in finding out that the external knowledge to be search for relates ``Paris'' to ``France'' and ``Berlin'' to ``Germany'', and then to return only the first letter of the result. 
Similarly, for \texttt{g}, before applying a leet translation we need to retrieve as external knowledge the first name.
In both cases the difficulty is that the intermediate knowledge is not present in the examples.

\vskip1em
The three steps can be necessary, the most complex tasks involve at the same time subtle preprocessing to extract entities, complicated relationships between entities, and significant postprocessing on external knowledge.

\subsection{Motivating Examples}\label{sec:applications}
We illustrate the disruptive power of knowledge-powered program synthesis in three application domains.

\paragraph*{General knowledge}
The first example, which we use throughout the paper for illustrations, in the dataset and in the experiments, is to use the knowledge graph to obtain general facts about the world, such as geography, movies, people. Wikidata and Yago are natural knowledge graphs candidates for this setting. 
This domain is heavily used for query answering: combining it with program synthesis brings it to another level, since programs generalize to any input.

\paragraph*{Grammar exercises}
The second example, inspired from~\cite{VerbruggenLG21}, is about language learning: tasks are grammar exercises, where the goal is to write a grammatically correct sentence. Here the knowledge graph includes grammatical forms and their connections, such as verbs and their different conjugated forms, pronouns, adjectives, and so on.
Generating programs for solving exercises opens several perspectives, including generating new exercises as well as solving them automatically.

\paragraph*{Advanced database queries}
In the third example knowledge-powered program synthesis becomes a powerful querying engine. This scenario has been heavily investigated for SQL queries~\cite{WangCB17,ZhouBCW22}, but only at a syntactic level.
Being able to rely on the semantic properties of the data opens a number of possibilities, let us illustrate them on an example scenario.
The knowledge graph is owned and built by a company, it contains immutable data about products.
The database contains customer data.
Crossing semantical information between the database being queried and the knowledge graph allows the user to generate complex queries to extract more information from the database including for instance complex statistics.


%%%%%%%%%%%%%%%%%%%%%%%%%%%
\section{WikiCoder}

\begin{figure}[t]
   \centering
   \includegraphics[width=0.8\textwidth]{pipeline}
   \caption{WikiCoder pipeline.}
   \label{fig:pipeline}
\end{figure}

WikiCoder is a general-purpose grammar-based program synthesis tool, it was developed in Python integrating state of the art program synthesis techniques. It is publicly available on GitHub\footnote{\url{https://github.com/nathanael-fijalkow/ProgSynth/tree/WikiCoder}}.
The main functionality of WikiCoder is to solve programming by example tasks: the user inputs a few examples and WikiCoder synthesizes a knowledge-powered program satisfying the examples.
The programming language is specified as a domain specific language (DSL), designed to solve a common set of tasks.
WikiCoder supports a number of classical DSLs, including towers building~\cite{EllisWNSMHCST21}, list integer manipulations~\cite{Balog2017}, regular expressions, and string manipulation tasks à la FlashFill~\cite{Gulwani2011}: our experiments report on the latter DSL.
Following recent advances in machine learned program synthesizers~\cite{EllisWNSMHCST21,Fijalkow2022ScalingNP}, WikiCoder is divided into three components: compilation, examples processing, and search, as illustrated in Figure~\ref{fig:pipeline} and discussed in the next three sections.
This is very different from Codex's architecture, which is based on auto-regressive very large models directly generating code.

%%%%%%%%%%%%%%%%%%%%%%%%%%%
\subsection{Compilation}
The DSL is specified as a list of primitives together with their types and semantics.
The compilation phase consists in obtaining an efficient representation of the set of programs in the form of a context-free grammar.
The right way to look at the grammar is as a generating model: it generates code. 
Thanks to the expressivity of context-free grammars, many syntactic properties can be ensured: primarily and most importantly, all generated programs are correctly typed.

The DSL we use in our experiments is tailored for string manipulation tasks à la Flashfill. 
For the sake of presentation we slightly simplify it. 
We use two primitive types: \verb+STRING+ and \verb+REGEXP+, and one type constructor \verb+Arrow+.
We list the primitives below, they have the expected semantics.

\begin{minted}{python}
    $       : REGEXP                  # end of string
    .       : REGEXP                  # all
    [^_]+   : Arrow(STRING, REGEXP)   # all except X
    [^_]+$  : Arrow(STRING, REGEXP)   # all except X at the end
    compose : Arrow(REGEXP, Arrow(REGEXP, REGEXP))

    concat  : Arrow(STRING, Arrow(STRING, STRING))
    match   : Arrow(STRING, Arrow(REGEXP, STRING))

    # concat_if: concat if the second argument (constant) 
    #            is not present in the first argument
    concat_if    : Arrow(STRING, Arrow(STRING, STRING))  
    # split_fst: split using regexp, returns first result
    split_fst  : Arrow(STRING, Arrow(REGEXP, STRING))
    # split_snd: split using regexp, returns second result
    split_snd : Arrow(STRING, Arrow(REGEXP, STRING))
\end{minted} 

In the implementation we use two more primitive types: \verb+CONSANT_IN+ and \verb+CONSTANT_OUT+, which correspond to constants in the inputs and in the output. This is only for improving performances, it does not increase expressivity. Some primitives are duplicated to use the two new primitive types.


%%%%%%%%%%%%%%%%%%%%%%%%%%%
\subsection{Examples processing}
A preprocessing algorithm produces from the examples three pieces of information: a sketch, which is a decomposition of the current task into subtasks, a set of \texttt{SPARQL} queries for the knowledge graph, and an embedding of the examples for the prediction model.

\begin{figure}[ht]
   \centering
   \includegraphics[width=0.8\textwidth]{preprocessing}
   \caption{Example of the preprocessing algorithm.}
   \label{fig:preprocessing}
\end{figure}

\paragraph*{Preprocessing algorithm}
Figure~\ref{fig:preprocessing} illustrates the preprocessing algorithm in action.
The high level idea is that the algorithm is looking for a shared pattern across the different examples.
In the task above, `` code: '' is shared by all examples, hence it is extracted out.
A naive implementation is to look for the largest common factor between the strings, and proceed recursively on the left and on the right.
The process is illustrated with Algorithm~\ref{alg:preprocessing}, which produces a list of constants from a list of strings. This procedure is applied to both the inputs and outputs independently to create sketches.
Thanks to these constants the task is split into subtasks as illustrated in Figure~\ref{fig:preprocessing} where we have split the task into two subtasks: \texttt{CapitalOf} and \texttt{CapitalOf-PhoneCode}.
To solve each subtask we query the knowledge graph with the inputs and outputs.
If no path is found, we run a regular -- syntactical -- program synthesis algorithm.

\begin{algorithm}[ht]
   \caption{Constant extraction}\label{alg:preprocessing}
   \begin{algorithmic}
      \Procedure {GetConstants}{$S = (S_k)_{k \in [1,n]}$ : strings}
      \If{there is an empty string in $S$} 
         \State \textbf{return} empty list
      \EndIf
      \State $\text{factor} \gets$ longest common factor among all strings in $S$
      \If{$\text{len}(\text{factor}) \le 2$}
	      \State \textbf{return} empty list
	  \Else
	      \State $S_{\text{left}} \gets$ prefix of factor in $S$ 
	      \State $L_{\text{left}} \gets$ \textsc{GetConstants}$(S_{\text{left}})$
	      \State $S_{\text{right}} \gets$ suffix of factor in $S$ 
	      \State $L_{\text{right}} \gets$ \textsc{GetConstants}$(S_{\text{right}})$
          \State \textbf{return} $L_{\text{left}} + \text{factor} + L_{\text{right}}$
      \EndIf
      \EndProcedure
   \end{algorithmic}
\end{algorithm}


\paragraph*{Generated \texttt{SPARQL} queries}
The \texttt{SPARQL} queries are generated from the examples after preprocessing.
Once we have the constants with Algorithm~\ref{alg:preprocessing} of the inputs and outputs, we can split the inputs and outputs in constant parts and non-constant parts, only the non-consant parts are relevant.
For each non-constant part in the outputs, we generate queries from the non constant part of the inputs which should map to this non-constant part of the output. 
Since relations may be complex, that is ``Paris'' is at distance 1 from ``France'' but ``33'' is two relations away from ``Paris'', we generate \texttt{SPARQL} queries for increasing distances up to a fixed upper bound.
Here is the query at distance 2, that we execute for the example in Figure~\ref{fig:preprocessing} with \texttt{CapitalOf-PhoneCode}:
\begin{verbatim}
   PREFIX w: <https://en.wikipedia.org/wiki/>
   SELECT ?p0 ?p1 WHERE {
      w:Paris ?p0 ?o_1_0 .
      ?o_1_0 ?p1 w:33 .
      w:Berlin ?p0 ?o_2_0 .
      ?o_2_0 ?p1 w:49 .
      w:Warsaw ?p0 ?o_3_0 .
      ?o_3_0 ?p1 w:48 .
   }
\end{verbatim}
Notice that intermediary entities make an apparition in order to accommodate for longer path lengths.
The output of the above query would consist of two paths: \texttt{CityOf-PhoneCode} and \texttt{CapitalOf-PhoneCode}.

As disambiguation strategy (inspired by~\cite{ZhengSCLW22}) we choose the path with the least number of hits across all examples, called the least ambiguous path. 
In this example there is no preferred path since both paths lead to a single entity for each example.
%One could argue that \texttt{CityOf-PhoneCode} should be preferred since there are more starting entities than \texttt{CapitalOf-PhoneCodeOf}, thus it is more likely that we find such a path from our starting entity. 
%We did not implement such a mechanism but this could be done to select one path among the least ambiguous ones.
As an example of this disambiguation strategy, let us consider paths from ``33'' to ``Paris'': there are two paths, 
\texttt{PhoneCodeOf-Capital} and \texttt{PhoneCodeOf-City}.
The least ambiguous path is \texttt{PhoneCodeOf-Capital} since \texttt{PhoneCodeOf-City} leads to all cities of the country.

To find the least ambiguous path, we need to count the number of hits, which is done using more \texttt{SPARQL} queries.
Here is a sample query to get all entities at the end of the path \texttt{CapitalOf-PhoneCode} from the starting entity ``Paris'':
\begin{verbatim}
   PREFIX w: <https://en.wikipedia.org/wiki/>
   SELECT ?dst WHERE {
      w:Paris w:CapitalOf ?e0 .
      ?e0 w:PhoneCode ?dst .
   }
\end{verbatim}
We count the number of results for all examples to get the number of hits for a path.

\begin{figure}[ht]
   \centering
   \includegraphics[width=\textwidth]{predictions}
   \caption{Illustration of the prediction model.}
   \label{fig:predictions}
\end{figure}

\paragraph*{Prediction model}
Efficient search is crucial in program synthesis, because the search space combinatorially explodes. We, and we believe the community more broadly, see learning as the right way of addressing this challenge.
Following~\cite{EllisWNSMHCST21,Fijalkow2022ScalingNP}, we build a prediction model in the form of a neural network: it reads embedding of the examples and outputs probabilities on the derivation rules of the grammar representing all programs. 
This transforms the context-free grammar representing programs in a probabilistic context-free grammar, which is a stochastic process generating programs. In other words, it defines a probabilistic distribution over programs, and the prediction model is trained to maximise the probability that a solution program is generated.

This prediction process is illustrated in Figure~\ref{fig:predictions}.
This effectively leverages the prediction power of neural networks without sacrificing correctness: the prediction model biases the search towards most likely programs but does not remove part of the search tree, hence -- theoretically -- a solution will be found if there exists one.

%%%%%%%%%%%%%%%%%%%%%%%%%%%
\subsection{Search}
The prediction model assigns to each candidate program a likelihood of being a solution to the task at hand. WikiCoder uses the HeapSearch algorithm, which is a bottom-up enumerative search algorithm~\cite{Fijalkow2022ScalingNP} outputting programs following the likelihood order of the prediction model. It has been shown superior to other enumeration methods (such as A* or Beam search) and easily deployed across parallel compute environments.
The candidate programs use the results of the \texttt{SPARQL} queries on the knowledge graph to be run on the examples.



%%%%%%%%%%%%%%%%%%%%%%%%%%%
\section{Evaluation}
 \section{Benchmarks and Evaluation}
\label{sec:eval}

We evaluate \krakenSpace to answer the following set of questions:
\begin{itemize}
\item How much improvement does partial evaluation and our implemented compiler optimizations give \kraken? %(\S \ref{sec:eval2})
\item How much faster is our purely functional f-expr language, \krakenSpace, compared to other implementations of fexprs? %(\S \ref{sec:eval1} - \ref{sec:eval2})
\item How does \kraken's performance, with its fexprs, compare to macros? %(\S \ref{sec:eval1}, \S \ref{sec:eval3})
\item How do the different partial evaluation mechanisms/optimizations in \krakenSpace contribute towards reduction in overall runtime?
%\item What does \krakenSpace do internally when we create a data structure and evaluate it for some function? (\S \ref{sec:casestudy})
\end{itemize}

\textbf{Experimental Setup}: 
We ran these experiments in a reproducible Nix environment on a NixOS install \cite{10.1145/1411203.1411255} (Kernel 6.0.0) on a laptop with 8 cores / 16 threads and 64 GB of RAM.
Our code contains the scripts and Nix Flakes needed to reproduce the exact set of dependencies to run our tests.
%The code can be found at \url{https://github.com/limvot/kraken}.

The Kraken benchmarks were run using both the Wasmtime and WAVM WebAssembly engines for most benchmarks.
The Wasmtime WebAssembly engine is one of the most popular, developed by the Bytecode Alliance itself, and uses the CraneLift code generation backend.
The WAVM WebAssembly engine is interesting for its use of LLVM, and it often produces the fastest code on benchmarks but has a higher startup time.
We eliminated the Cfold Wasmtime benchmark due to problems running out of stack space (a known property of the Cfold benchmark).

\textbf{Benchmarks}: 
To showcase the capability of Kraken, we created benchmarks that are commonly implemented in functional languages and have been used as benchmarks in other papers \cite{reinking2021perceus, 10.1145/3547646}.
The benchmarks are
\begin{itemize}
\item Fib - Calculating the nth Fibonacci number
\item RB-Tree - Inserting n items into a red-black tree, then traversing the tree to sum its values
\item Deriv - Computing a symbolic derivative of a large expression
\item Cfold - Constant-folding a large expression
\item NQueens - Placing n number of queens on the board such that no two queens are diagonal, vertical, or horizontal from each other
\end{itemize}
All benchmarks besides Fibonacci use the fexpr version of match for pattern matching in \kraken, which is equivalent to the macro version in NewLisp. We also RB-Tree using NewLisp's~\cite{mueller2018newlisp} version of fexpr match. We modified the sizes of the problems presented to the benchmark to account for the longer running times of some of the less-optimized implementations.
The code for Kraken and NewLisp is very similar, and we should note that it is very unidiomatic NewLisp.
Our goal was not to compare Kraken and NewLisp as implementation languages for Red-Black Trees, but to stress test a single reasonably complex fexpr/macro, namely pattern matching.
% \textbf{Comparison with other languages}: We evaluated \krakenSpace against a language that contains f-exprs, as well as against itself with various optimizations disabled. The only other language we could find which contains a real f-expr mechanism is NewLisp~\cite{mueller2018newlisp} and so we ported \kraken's benchmark implementation to NewLisp.

%The six state-of-the-art languages are Java 17.0.1, Swift 5.4.2, Koka 2.3.2, C++, Haskell 8.10.7, and OCaml 4.12.
%The language choices were taken directly from Perceus reference-counting paper \cite{reinking2021perceus}.
%The Fibonacci benchmark additionally tests Python 3.9.11 and Chez Scheme 9.5.4.
%Koka, Ocaml and Haskell are good comparison points as statically-typed, compiled, functional programming languages, while Chez Scheme is a good comparison point as a mature and industrial strength dynamically-typed Scheme implementation known for its performance. 
%\subsection{Basic Level Comparison}
\subsection{The Effect of Partial Evaluation on Eval Calls}

\begin{table}[h]
\caption{Number of eval calls with no partial evaluation for Fexprs}
	\begin{tabular}{||c | c c c c c ||} 
		\hline
		&Evals & Eval w1 Calls & Eval w0 Calls & Comp Dyn & Comp Dyn\\ 
        & & & & w1 Calls & w0 Calls\\ [0.5ex] 
		\hline\hline
		Cfold 5 & 10897376 & 2784275 & 879066  & 1 & 0 \\ 
		\hline
		  Deriv 2  & 11708558 & 2990090 & 946500 & 1 & 0 \\ 
        \hline
		  NQueens 7 & 13530241 & 3429161 & 1108393 & 1 & 0 \\ 
    \hline
		  Fib 30 & 119107888 & 30450112 & 10770217 & 1 & 0 \\ 
    \hline
		  RB-Tree 10 & 5032297 & 1291489 & 398104 & 1 & 0 \\ 
		\hline
	\end{tabular}
    \label{npe:calls}
 \end{table}

As mentioned before, using fexprs without partial evaluation will prelude optimization and cause a massive amount of repeated work. Table \ref{npe:calls} and Table \ref{pe:calls} show the number of calls to the \krakenSpace runtime's eval function, the number of times the runtime's eval function executed a call to an applicative with wrap\_level=1, the number of times the runtime's eval function executed a call to an operative with wrap\_level=0, the number of compiled dynamic calls to applicatives with wrap\_level=1, and the number of compiled dynamic calls to operatives with wrap\_level=0.
These are shown for \krakenSpace test cases with partial evaluation turned off and turned on. 
\begin{table}[h]
\caption{Number of eval calls in Partially Evaluated Fexprs}
	\begin{tabular}{||c | c c c c c ||} 
		\hline
		&Evals & Eval w1 Calls & Eval w0 Calls & Comp Dyn & Comp Dyn\\ 
        & & & & w1 Calls & w0 Calls\\ [0.5ex] 
		\hline\hline
		Cfold 5 & 0 & 0 & 0  & 0 & 0 \\ 
		\hline
		  Deriv 2  & 0 & 0 & 0 & 2 & 0 \\ 
        \hline
		  NQueens 7 & 0 & 0 & 0 & 0 & 0 \\ 
    \hline
		  Fib 30 & 0 & 0 & 0 & 0 & 0 \\ 
    \hline
		  RB-Tree 10 & 0 & 0 & 0 & 10 & 0 \\ 
		\hline
	\end{tabular}
    \label{pe:calls}
 \end{table}

\begin{table}[h]
\caption{Number of calls to the runtime's eval function for RB-Tree. The table shows the non-partial evaluation numbers -> partial evaluation numbers.}
	\begin{tabular}{||c | c c c c c ||} 
		\hline
		&Evals & Eval w1 Calls & Eval w0 Calls & Comp Dyn & Comp Dyn\\ 
        & & & & w1 Calls & w0 Calls\\ [0.5ex] 
		\hline\hline
		  RB-Tree 7 & 2952848 -> 0 & 757932 -> 0 & 233513 -> 0 & 1 -> 7 & 0 -> 0\\ 
        \hline
		  RB-Tree 8 & 3532131 -> 0 & 906548 -> 0 & 279379 -> 0 & 1 -> 8 & 0 -> 0\\ 
        \hline
		  RB-Tree 9 & 4278001 -> 0 & 1097965 -> 0 & 3383831 -> 0 & 1 -> 9 & 0 -> 0\\ 
		\hline
	\end{tabular}
    \label{pe:rb}
    \vspace{-4mm}
 \end{table}

Without partial evaluation, no compilation can be done because it is impossible to tell if arguments to calls will be evaluated. In all benchmarks, partial evaluation removed all calls to the runtime's eval function, resulting in a completely compiled program. Looking at RB-Tree, there are over a million calls to combiners with wrap level 1 (normal functions), and 398,000 calls to combiners with wrap level 0 (operatives replacing macros). This massive blowup in the number of calls is due to the repeated and exponential re-execution of macro-like-combiners in the definition of other macro-like-combiners, as discussed in the Introduction.

The non-partially-evaluated benchmarks show 1 compiled dynamic call to an applicative (its the first call into eval) and 0 compiled dynamic calls to operatives, because there is no compilation at all. For the partially evaluated benchmarks, there are a few compiled dynamic calls to applicatives due to higher-order function use in the benchmarks, and there are no compiled dynamic calls to operatives, as all operative use has been eliminated.
We also varied the inputs for RB-Tree shown in Table \ref{pe:rb} to give a sense for how the number scale with respect to input size.

The incredible slowdown implied by these tables comes to full fruition in our RB-Tree test in Fig.~\ref{fig:kraken_nqueens_rbtree}.
We kept this run shorter because Kraken's non-partial-evaluating interpreter takes an incredibly long time even for 100 insertions (40 minutes).
The compounding layers of repeated macro-like operative calls in the non-partially-evaluated Kraken version cause a ~70,000x slowdown relative to the partial evaluated, optimized, and compiled version.
For the remaining benchmarks, we remove the naive interpreted \krakenSpace version, as in each case its performance is so bad as to blow out the graph and make it impossible to do any comparison.
In our optimized Kraken, our partial evaluation algorithm is able to fully collapse these levels of inefficiency, evaluate and inline the results, and give the backend more specialized code to optimize, emitting a compiled version that handily beats not only the NewLisp-fexpr implementation but even the NewLisp-macro implementation, as can be seen in Fig.~\ref{fig:kraken_vs_world_fib}.
We kept the benchmark sizes small in this test because the stack limits of NewLisp prevent sizes larger then ~880, while the Tail Call Elimination performed by the \krakenSpace compiler allows us to run much larger benchmarks, including the run of 4,800,000 inserts to the RB-Tree.
This result shows the dramatic effect of partial evaluation and compiler optimizations on runtime for \kraken. Our technique takes the performance of a fully fexpr based language from being completely infeasible to being faster than a macro-based dynamic scripting language currently in use.
% \begin{center}
% \begin{table}[ht]
% \caption{Number of call to the runtime's eval function for Fib. The table shows the non-partial evaluation numbers -> partial evaluation numbers}
% 	\begin{tabular}{||c | c c c c c ||} 
% 		\hline
% 		&Evals & Eval w1 Calls & Eval w0 Calls & Comp Dyn w1 Calls & Comp Dyn w0 Calls\\ [0.5ex] 
% 		\hline\hline
% 		Fib 10 & 8468 -> 0 & 2167 -> 0  & 777 -> 0 & 1 -> 0 & 0 -> 0 \\ 
% 		\hline
% 		  Fib 15  & 87916 -> 0 & 22478 -> 0 & 7961 -> 0 & 1 -> 0 & 0 -> 0 \\ 
%         \hline
% 		  Fib 20 & 969010 -> 0 & 247731 -> 0 & 87633 -> 0 & 1 -> 0 & 0 -> 0 \\ 
%     \hline
% 		  Fib 25 & 10740492 -> 0 & 2745825 -> 0  & 971209 -> 0 & 1 -> 0 & 0 -> 0 \\ 
% 		\hline
% 	\end{tabular}
%     \label{pe:fib}
%  \end{table}
% \end{center}

\begin{figure}[h]
\caption{Constant Fold and Deriv}
\includegraphics[width=0.45\textwidth]{cfold_table.csv_}
\includegraphics[width=0.45\textwidth]{deriv_table.csv_}
\label{fig:kraken_const_deriv}
\vspace{-6mm}
\end{figure}
\subsection{Comparison between Kraken Versions}
Beyond the massive speedup from partial-evaluation, Fig. \ref{fig:kraken_const_deriv} and \ref{fig:kraken_nqueens_rbtree} show the effect of the various compiler optimizations we described by disabling them one by one.
 Our main four optimizations have a strong positive effect on runtime, with the exception of lazy environment instantiation. Lazy environment instantiation helps massively on fib, and some on Deriv, but generally hurts the rest slightly.


\begin{figure}[h]
\caption{N-Queens}
\includegraphics[width=0.45\textwidth]{nqueens_table.csv_}
\includegraphics[width=0.45\textwidth]{slow_rbtree_table.csv_}
\label{fig:kraken_nqueens_rbtree}
\vspace{-4mm}
\end{figure}


\subsection{Comparison against Others}


To give a general idea of our current performance, we also show a Fibonacci benchmark that mostly exercises pure function-call speed and inlining as seen in Fig. ~\ref{fig:kraken_vs_world_fib}.
We include Python and Chez Scheme to give a general idea for where an exemplar slow and an exemplar fast dynamic language would fall.
With the benefit of our partial evaluation, compilation, and leaning upon mature WebAssembly implementations, we beat both, but this should be taken with a grain of salt, as this is a very limited micro-benchmark only meant to give a general sense of the order of magnitude of our performance.



\label{sec:eval1}
\begin{figure}[h]
\caption{Kraken vs. Others. Ordered by fastest to slowest}
\includegraphics[width=0.45\textwidth]{fib_table.csv_}
\includegraphics[width=0.45\textwidth]{rbtree_table.csv_}
\label{fig:kraken_vs_world_fib}
\end{figure}

%\label{sec:eval_nqueens}
%\begin{figure}[h]
%\caption{N-Queens}
%\includegraphics[width=0.45\textwidth]{nqueens_table.csv_}
%\includegraphics[width=0.45\textwidth]{slow_nqueens_table.csv_}
%\label{fig:kraken_nqueens}
%\end{figure}

%\label{sec:eval_nqueens}
%\begin{figure}[h]
%\caption{Kraken, N-Queens, absolute value and log-scale}
%\includegraphics[width=0.45\textwidth]{nqueens_table.csv_}
%\includegraphics[width=0.45\textwidth]{nqueens_table.csv_log}
%\label{fig:kraken_nqueens}
%\end{figure}
%\label{sec:eval_nqueensp}
%\begin{figure}[h]
%\caption{Kraken, N-Queens, absolute value and log-scale}
%\includegraphics[width=0.45\textwidth]{slow_nqueens_table.csv_}
%\includegraphics[width=0.45\textwidth]{slow_nqueens_table.csv_log}
%\label{fig:kraken_nqueensp}
%\end{figure}

%\label{sec:eval_cfold}
%\begin{figure}[h]
%\caption{C-Fold}
%\includegraphics[width=0.45\textwidth]{cfold_table.csv_}
%\includegraphics[width=0.45\textwidth]{slow_cfold_table.csv_}
%\label{fig:kraken_cfold}
%\end{figure}
%\label{sec:eval_cfold}
%\begin{figure}[h]
%\caption{Kraken, C-Fold, absolute value and log-scale}
%\includegraphics[width=0.45\textwidth]{cfold_table.csv_}
%\includegraphics[width=0.45\textwidth]{cfold_table.csv_log}
%\label{fig:kraken_cfold}
%\end{figure}
%\label{sec:eval_cfoldp}
%\begin{figure}[h]
%\caption{Kraken, C-Fold, absolute value and log-scale}
%\includegraphics[width=0.45\textwidth]{slow_cfold_table.csv_}
%\includegraphics[width=0.45\textwidth]{slow_cfold_table.csv_log}
%\label{fig:kraken_cfoldp}
%\end{figure}

%\label{sec:eval_deriv}
%\begin{figure}[h]
%\caption{Deriv}
%\includegraphics[width=0.45\textwidth]{deriv_table.csv_}
%\includegraphics[width=0.45\textwidth]{slow_deriv_table.csv_}
%\label{fig:kraken_deriv}
%\end{figure}
%\label{sec:eval_deriv}
%\begin{figure}[h]
%\caption{Kraken, Deriv, absolute value and log-scale}
%\includegraphics[width=0.45\textwidth]{deriv_table.csv_}
%\includegraphics[width=0.45\textwidth]{deriv_table.csv_log}
%\label{fig:kraken_deriv}
%\end{figure}
%\label{sec:eval_derivp}
%\begin{figure}[h]
%\caption{Kraken, Deriv, absolute value and log-scale}
%\includegraphics[width=0.45\textwidth]{slow_deriv_table.csv_}
%\includegraphics[width=0.45\textwidth]{slow_deriv_table.csv_log}
%\label{fig:kraken_derivp}
%\end{figure}

%\subsection{Comparison against state-of-the-art languages}
%\label{sec:eval3}

%\begin{figure}[h]
%\caption{Kraken vs. S.o.t.A.}
%\includegraphics[width=0.45\textwidth]{cfold_table.csv_}
%\includegraphics[width=0.45\textwidth]{rbtree_table.csv_}
%\label{fig:kraken_vs_world1}
%\end{figure}

%\begin{figure}[h]
%\caption{Kraken vs. S.o.t.A.}
%\includegraphics[width=0.45\textwidth]{deriv_table.csv_}
%\includegraphics[width=0.45\textwidth]{nqueens_table.csv_}
%\label{fig:kraken_vs_world2}
%\end{figure}

% \begin{figure}[h]
% \caption{Kraken vs. S.o.t.A. (Log)}
% \includegraphics[width=0.45\textwidth]{cfold_table.csv_log}
% \includegraphics[width=0.45\textwidth]{rbtree_table.csv_log}
% \label{fig:kraken_vs_world_log_1}
% \end{figure}
% \begin{figure}[h]
% \caption{Kraken vs. S.o.t.A. (Log)}
% \includegraphics[width=0.45\textwidth]{deriv_table.csv_log}
% \includegraphics[width=0.45\textwidth]{nqueens_table.csv_log}
% \label{fig:kraken_vs_world_log_2}
% \end{figure}

%As we noted before with the Fib(30) microbenchmark in Section \ref{sec:eval1}, we remain significantly slower than state-of-the-art compiled languages.
%This is particularly true for memory-intensive benchmarks due to our naive reference-counting and malloc/free implementations.
%However, our results are of a similar order of magnitude to the difference between the state-of-the-art compiled languages and dynamic scripting languages, like Python's results in the Fib(30) microbenchmark.
%We assert that is not a fundamental limitation because the classic f-expr slowness is being eliminated, as shown by Fig. \ref{fig:kraken_vs_newlisp1} and Fig. \ref{fig:kraken_vs_newlisp2}.
%In future work, we plan to expand our compile-time analysis and optimization to implement a modified, dynamic-language version of Perceus reference counting.
%With this change, we belive \krakenSpace can be competitive with these state-of-the-art languages.

%\subsection{Case Study: Red-Black Tree}
%\label{sec:casestudy}

%\begin{figure}[h]
%\caption{Kraken vs. S.o.t.A. - RB-Tree Focus}
%\includegraphics[width=0.4\textwidth]{rbtree_table.csv_}
%\includegraphics[width=0.4\textwidth]{rbtree_table.csv_log}
%\label{fig:kraken_vs_world_rbtree}
%\end{figure}


%To evaluate our partial evaluation algorithm and compiler, we extracted the benchmarks used by the Koka language project from their code repository and added Kraken versions, as well as implementing a naive Fibonacci microbenchmark ourselves to evaluate pure function call speed.\\
%With partial evaluation and the compiler optimizations listed above, we get fairly strong performance on purely numerical computations, such as the naive Fibonacci microbenchmark.
%Unfortunately, the overhead of our unsophisticated reference counting, dynamic type checking, and bounds checking causes poor performance on benchmarks involving data structures relative to mainstream programming language implementations.
%This is not a fundamental limitation, and will be addressed in future work, as recounted in the next section.
%It should be noted, however, that while the performance relative to established language implementations is very poor for the memory-intensive benchmarks (600-900x slower), we still realize a massive speedup compared to an unoptimized and non-partial-evaluated f-expr implementation (100,000x faster)!


%%%%%%%%%%%%%%%%%%%%%%%%%%%
\section{Discussion}
We provide some comments on the growth conditions which constituted the majority of our analysis in sections \ref{sec:Hmixing} and \ref{sec:Hsigma}. In the simplest cases of Lemma \ref{lemma:unstableGrowth}, growth was established in an analogous fashion to the old one-step expansion condition (\ref{eq:oldOneStepExpansion}), finding the relevant Jacobians $M_j$ and checking that their expansion factors $K(M_j)$ satisfy
\begin{equation}
    \label{eq:discussionOneStep}
    \sum_j \frac{1}{K(M_j)} <1.
\end{equation}
For the more complicated cases, the inductive method used to establish growth near the accumulation points in Lemma \ref{lemma:unstableGrowth} and the weakened one-step expansion condition (\ref{eq:oneStep}) both address the same fundamental issue: the splitting of unstable curves by singularities into an unbounded number of small components. They circumvent this obstacle in rather different ways, however. While (\ref{eq:oneStep}) generalises (\ref{eq:discussionOneStep}) to ensure an growth of unstable curves `on average' (see \cite{chernov_statistical_2009} for a precise statement), our inductive method is a more direct adaptation of (\ref{eq:discussionOneStep}), using it to generate contradictory geometric conditions which a hypothetical non-growing unstable curve must satisfy. It may be possible to prove Theorem \ref{sec:Hmixing} using (\ref{eq:oneStep}) as the basis for growth. Since we required (\ref{eq:oneStep}) anyway for proving Theorem \ref{thm:HsigmaExp}, this could potentially condense our analysis, but only to a minor extent. A convenience of the method used in section \ref{sec:Hmixing} is that, by way of the `simple intersection' property, it naturally gives geometric information on the images of manifolds, useful for proving the property \textbf{(M)} of Theorem \ref{thm:katok-strelcyn}.

We expect that essentially analogous analysis can be applied to establish mixing properties in a wide class of piecewise linear non-uniformly hyperbolic maps, including those (like the OTM) which sit on the boundary of ergodicity and beyond. While we have relied on the precise partition structure of $H_\sigma$, its fundamental feature (self-similar sequences of elements $A^k$, sharing boundaries with its neighbours $A^{k-1},A^{k+1}$ and accumulating onto some point $p$) is quite typical to return map systems. See, for example, those of various stadium billiards \cite{chernov_chaotic_2006,chernov_improved_2008,chernov_statistical_2009} and LTMs \cite{springham_polynomial_2014}. Indeed, the same method can be used to prove the Bernoulli property for non-monotonic LTMs \cite{myers_hill_mixing_2022}, where monotonicity of the manifold images cannot be assumed and the classical argument \cite{sturman_mathematical_2006} fails. The OTM is the pointwise limit of these maps as the boundary shrinks to null measure. It further has utility in proving growth conditions for maps which are uniformly hyperbolic but possess regions $A_j$ where the hyperbolicity is very weak, signified by $K(M_j) \approx 1$, so that (\ref{eq:discussionOneStep}) fails. Typically this leads to suboptimal bounds on mixing windows, see e.g. \cite{wojtkowski_model_1981,przytycki_ergodicity_1983,myers_hill_family_2022}. The map $H_{(\eta,\eta)}$ for $\eta \approx 1/2$ is another example, possessing weak hyperbolicity over $A_2, A_3$. Letting $\varepsilon = |\eta-1/2|>0$, there is an upper bound $N = N(\varepsilon)$ on escape times from the intersections $A_2\cap \sigma, A_3 \cap \sigma$. The growth lemma then follows by applying the inductive step roughly $N$ times and can be established for arbitrarily small $\varepsilon$, opening the door to establishing optimal mixing windows.

The above gives two examples of piecewise linear perturbations to $H$ where mixing with respect to Lebesgue is preserved and our methods can be applied. Nonlinear perturbations to the shear profiles complicate the analysis in several ways. Firstly as the map's Jacobians takes on a broader range of values, cone invariance becomes an increasingly harder condition to establish. Cones must be widened, giving looser bounds on expansion factors, which may already be weak due to new regions of weaker stretching. This, together with the change from polygonal to curvilinear return time partition elements and nonlinear local manifolds, adds some complexity to showing growth conditions. This does not rule out certain (small) nonlinear perturbations however. There is some leeway in the inequalities which govern cone invariance and growth of local manifolds, the latter of which is not too dissimilar from the piecewise linear setting (see Lemmas \ref{lemma:piecewiseApprox}, \ref{lemma:componentLength}). Certain small perturbations would not alter the \emph{topological} structure of the return time partition, i.e. which elements share boundaries, the key information needed for setting up the induction. Finally while the partition elements would no longer be polygonal, only coarse geometric information is required for verifying each inductive step. Following the above, a potential perturbation could be to replace the linear portions of each shear by a cubic, perturbing the tent profile
\[  f(t) = \begin{cases} 2t & 0 \leq t \leq 1/2, \\ 2(1-t) & 1/2 \leq t \leq 1 ,\end{cases} \]
of the OTM shears to
\[  f_a(t) = \begin{cases} \frac{1}{8} t \left(16 - a + 6at - 8at^{2} \right) & 0 \leq t \leq 1/2, \\ \frac{1}{8}\left(1-t\right)\left( 16 - a + 6a\left(1-t\right) - 8a\left(1-t\right)^{2}\right)  & 1/2 \leq t \leq 1, \end{cases}   \]
for $a>0$. For small enough $a$ the gradient range $f'(t)$ is restricted to small neighbourhoods of $\{ 2, -2\}$ and the escape time partition retains a similar structure. We illustrate this in Figure \ref{fig:perturbations}, showing escapes from the square $S_3$ under the map $G \circ F$, equivalent to escapes from the perturbed $A_3$ under the $G \circ F$, but with a cleaner geometry for comparison. When $a$ is too large the analogy to the OTM breaks down. At $a=16$ the map is twice differentiable everywhere and features a new source of slowed mixing, the Jacobian is the identity at the corner points $x,y \in \{  0, 1/2 \}$ giving locally parabolic behaviour (visible in the escape time partition). 

\begin{figure}
    \centering
    \includegraphics[width=0.24 \linewidth]{0.png}
    \includegraphics[width=0.24 \linewidth]{4.png}
    \includegraphics[width=0.24 \linewidth]{8.png}
    \includegraphics[width=0.24 \linewidth]{16.png}
    \caption{Partition of escape times from $S_3$ under the mapping $F \circ G$ for $a= 0,4,8,16$. }
    \label{fig:perturbations}
\end{figure}

\bibliographystyle{abbrv}
\bibliography{bib}

\end{document}
