
% \input{temproof1}
% {\red Original proof}
Let $\theta_k^*=\arg\sup_{\theta_k\in\Theta_k} \EE_{\mu_k}\sbr{\alpha^\pi_k(\vh_k, \cY_k) \theta_k(\cZ_k)} - \frac 1 2 \nbr{\theta_k}_{\mu_k, 2}^2$. Suppose that $\EE_{\mu_k}\sbr{\alpha^\pi_k(\vh_k, \cY_k) \theta_k^*(\cZ_k)}=\beta \nbr{\theta_k^*}_{\mu_k, 2}^2$. By definition of $\theta_k^*$, we have that
\begin{align}\label{eq:beta-LB}
    \rbr{\beta-\frac 1 2} \nbr{\theta_k^*}_{\mu_k, 2}^2=\sup_{\theta_k\in\Theta_k} \EE_{\mu_k}\sbr{\alpha^\pi_k(\vh_k, \cY_k) \theta_k(\cZ_k)} - \frac 1 2 \nbr{\theta_k}_{\mu_k, 2}^2\ge 0,
\end{align}
where the last inequality holds by plugging in $\theta_k=0$.
We plug in $\kappa \theta_k^*$ in \eqref{eq:L_D(h_CI)-LB-1} where $0<\kappa\le 2/3$. Note that $\Theta_k$ is star shaped. Thus, we have $\kappa \theta_k^*\in\Theta_k$ and it holds on event $\cE_{k, 3}$ that
\begin{align}\label{eq:L_D(h_CI)-LB-2}
    \cL_{k,\cD_k}^\pi(\vh_k) 
    &\gtrsim \kappa\rbr{\beta-\frac 3 4 \kappa}\nbr{\theta_k^*}_{\mu_k, 2}^2 - \eta_{\Theta_k, \vH_k, \cD_k}L_\alpha \kappa\nbr{\theta_k^*}_{\mu_k, 2} - \rbr{\eta_{\Theta_k, \vH_k, \cD_k}^2 + \frac 1 4\eta_{\Theta_k, \cD_k}^2}.
\end{align}
Recall that it holds on event $\cE_2$ that
\begin{align}\label{eq:19}
    \nbr{\vcL_{\cD}^\pi(\vh)}_\infty
    &\le \nbr{\vcL_{\cD}^\pi(\hat\vh^\pi)}_\infty + e_\cD\nend
    &\le \nbr{\vcL_{\cD}^\pi(\vh^{\pi, *})}_\infty + e_\cD\nend
    &\lesssim 4\varepsilon_{\vH} + \rbr{2 L_\alpha^2+L_\alpha+\frac 1 4}\max_{k\in\{1, \cdots, K\}}\eta_{\Theta_k, \cD_k}^2 + e_\cD,
\end{align}
where the last inequality holds by \eqref{eq:L_D(h^*)-UB-2}. Combining \eqref{eq:L_D(h_CI)-LB-2} with \eqref{eq:19} and letting $\cE_{k, 4}=\cE_2\cap\cE_{k, 3}$, it holds on event $\cE_{k, 4}$ that
\begin{align}\label{eq:quadratic inequation}
    \kappa\rbr{\beta-\frac 3 4 \kappa}\nbr{\theta_k^*}_{\mu_k, 2}^2 - \eta_{\Theta_k, \vH_k, \cD_k}L_\alpha \kappa\nbr{\theta_k^*}_{\mu_k, 2} - \Delta_{k, \cD} \lesssim 0,
\end{align}
where $\Delta_{\cD, k}=\eta_{\Theta_k, \vH_k, \cD_k}^2 + \frac 1 4\eta_{\Theta_k, \cD_k}^2+4\varepsilon_{\vH} + \rbr{2 L_\alpha^2+L_\alpha+\frac 1 4}\cdot \max_{k\in\{1, \cdots, K\}}\eta_{\Theta_k, \cD_k}^2 + e_\cD$ and $\cP(\cE_{k, 4})\ge 1-2(K+1)\xi$. 
We consider the following two cases
\paragraph{Case (i).} $\nbr{\theta_k^*}_{\mu_k, 2}>0$ holds on event $\cE_{k, 5}$. 

on event $\cE_{k, 4}\cap \cE_{k, 5}$, we restrict $\kappa$ by $0<\kappa < 2/3$.
We then have $\beta-3\kappa/4> \beta-1/2\ge 0$ where the last inequality holds by \eqref{eq:beta-LB} and noting that $\nbr{\theta_k^*}_{\mu_k, 2}>0$.
Therefore, solving the quadratic inequation \eqref{eq:quadratic inequation}, for any $0<\kappa <2/3$, it holds on event $\cE_{k, 4}\cap \cE_{k, 5}$ that
\begin{align}
    0< \nbr{\theta_k^*}_{\mu_k, 2}\lesssim \frac{\sqrt{\rbr{\eta_{\Theta_k, \vH_k, \cD_k}L_\alpha \kappa}^2+\kappa\rbr{4\beta-3\kappa}\Delta_{k, \cD}}+\eta_{\Theta_k, \vH_k, \cD_k}L_\alpha \kappa}{\kappa\rbr{2\beta-\frac 3 2 \kappa}}.
\end{align}
Plug in $\kappa=1/3$, it holds on event $\cE_{k, 4}\cap \cE_{k, 5}$ that $\nbr{\theta_k^*}_{\mu_k, 2}>0$ and that
\begin{align}
    \nbr{\theta_k^*}_{\mu_k, 2}^2
    &\lesssim 4\rbr{\sqrt{\rbr{\eta_{\Theta_k, \vH_k, \cD_k}L_\alpha}^2+3\Delta_{k, \cD}}+\eta_{\Theta_k, \vH_k, \cD_k}L_\alpha}^2\nend
    &\le 8\rbr{2\rbr{\eta_{\Theta_k, \vH_k, \cD_k}L_\alpha}^2+3\Delta_{k, \cD}}\nend
    &\le 24\rbr{ \rbr{\frac 2 3 L_\alpha^2 + 1}\eta_{\Theta_k, \vH_k, \cD_k}^2 + \rbr{2 L_\alpha^2+L_\alpha+\frac 1 2} \max_{k\in\{1, \cdots, K\}}\eta_{\Theta_k, \cD_k}^2 + 4\varepsilon_\vH + e_\cD}\nend
    &\le 24\tilde \Delta_{k,\cD},
\end{align}
where $\tilde \Delta_{k, \cD}=\rbr{\frac 2 3 L_\alpha^2 + 1}\eta_{\Theta_k, \vH_k, \cD_k}^2 + \rbr{2 L_\alpha^2+L_\alpha+\frac 1 2} \max_{k\in\{1, \cdots, K\}}\eta_{\Theta_k, \cD_k}^2 + 4\varepsilon_\vH + e_\cD$.
Recall the definition of $\theta_k^*$ by $\theta_k^*=\arg\sup_{\theta_k\in\Theta_k} \EE_{\mu_k}\sbr{\alpha^\pi_k(\vh_k, \cY_k) \theta_k(\cZ_k)} - \frac 1 2 \nbr{\theta_k}_{\mu_k, 2}^2$ and the definition of $\cL^\pi_k(\vh_k)$ in \eqref{eq:population loss-def}, it holds on event $\cE_{k, 4}\cap \cE_{k, 5}$ that
\begin{align}
    \cL^\pi_k(\vh_k) 
    &=\EE_{\mu_k}\sbr{\alpha^\pi_k(\vh_k, \cY_k) \theta_k^*(\cZ_k)} - \frac 1 2 \nbr{\theta_k^*}_{\mu_k, 2}^2\nend
    & = (\beta-\frac 1 2)\nbr{\theta_k^*}_{\mu_k, 2}^2\nend
    &\lesssim \eta_{\Theta_k, \vH_k, \cD_k} L_\alpha \nbr{\theta_k^*}_{\mu_k, 2} + \frac 3 2 \Delta_{k, \cD}\nend
    &<9 \tilde\Delta_{k, \cD},
\end{align}
where the first inequality holds by replacing \eqref{eq:quadratic inequation} and plugging in $\kappa=2/3$ The last inequality holds by noting that $\tilde\Delta_{k, \cD}>\Delta_{k, \cD}$ and that $\sqrt{24}\eta_{\Theta_k, \vH_k, \cD_k} L_\alpha \le 15\tilde\Delta_{k, \cD}/2$.


\paragraph{Case (ii).} $\nbr{\theta_k^*}_{\mu_k, 2}=0$ on event $\bar\cE_{k, 5}$, thereby 
\begin{align}
    \cL_k^\pi(\vh_k)=\EE_{\mu_k}\sbr{\alpha^\pi_k(\vh_k, \cY_k) \theta_k^*(\cZ_k)} - \frac 1 2 \nbr{\theta_k^*}_{\mu_k, 2}^2=0.
\end{align}

Combining case (i) and case (ii), we see that on event $\cE_{k, 4}$ it holds that
\begin{align}
    \cL^\pi_k(\vh_k) \lesssim 9\tilde \Delta_{k, \cD}.
\end{align}
Define $\cE_6=\cap_{k=1}^K \cE_{k, 4}$. Recall that $\cE_{k, 4}=\cE_2\cap\cE_{k, 3}$, Thereby, we have that $\cE_6 = \cE_2 \cap (\cap_{k=1}^K \cE_{k, 3})$ and that $\cP(\cE_6)\ge 1- 4K\xi$. Moreover, on event $\cE_6$, it holds that
\begin{align}
    \nbr{\vcL^\pi_{\cD}(\vh^{\pi, *})}_\infty - \nbr{\vcL^\pi_{ \cD}(\hat\vh^\pi)}_\infty\lesssim 4\varepsilon_{\vH} + \rbr{2 L_\alpha^2+L_\alpha+\frac 1 4}\max_{k\in\cbr{1, \cdots, K}}\eta_{\Theta_k, \cD_k}^2, 
\end{align}
and that
\begin{align}
    \cL^\pi_k(\vh_k) \lesssim 9\tilde \Delta_{k, \cD}, 
\end{align}
where $\tilde \Delta_{k, \cD}=\rbr{\frac 2 3 L_\alpha^2 + 1}\eta_{\Theta_k, \vH_k, \cD_k}^2 + \rbr{2 L_\alpha^2+L_\alpha+\frac 1 2} \max_{k\in\{1, \cdots, K\}}\eta_{\Theta_k, \cD_k}^2 + 4\varepsilon_\vH + e_\cD$.
Therefore, we complete the proof of Theorem \ref{thm:Fast rate}.

\subsection{Decomposition of the Sub-optimality with Pessimism}\label{proof: pessimism}
In this section, we study the sub-optimality of the estimated policy $\piepessi$ with pessimism.
The result in this section will be utilized in \S\ref{proof: subopt of IV} for the proof of sub-optimality of CCB-IV in Theorem \ref{thm:IV subopt}, \S\ref{proof: subopt of PV} for the proof of sub-optimality of CCB-PV in Theorem \ref{thm:PV subopt}, and \S\ref{proof:extended PV Subopt} for the proof of sub-optimality of CCB-PV with extended policy class in Theorem \ref{thm:extended PV subopt}.
Recall the definition of $\gpessi{\pi}$, 
\begin{align}
    \gpessi{\pi}=\arg\inf_{g\in\CICATE(e_\cD)} v(g, \pi),\label{eq:gpessi}
\end{align}
and the definition of $\piepessi$, 
\begin{align}
    \piepessi = \arg\sup_{\pie\in\Pie} \inf_{g\in\CICATE(e_\cD)} v(g, \pi) =  \arg\sup_{\pie\in\Pie} v(\gpessi{\pie}, \pi). \label{eq:piepessi}
\end{align}
On the event $\cE$ defined by \eqref{def:cE}, the regret of policy $\piepessi$ is given by
\begin{align}\label{eq:regret}
    \text{SubOpt}(\piepessi) &=v^{\piestar}-v^{\piepessi} \nend
    &= \underbrace{v^{\piestar} - v(\gpessi{\piestar}, \piestar)}_{\text{(i)}} + \underbrace{v(\gpessi{\piestar}, \piestar) - v(\gpessi{\piepessi}, \piepessi)}_{\text{(ii)}} + \underbrace{v(\gpessi{\piepessi}, \piepessi) - v(\gHstar, \piepessi)}_{\text{(iii)}} + \underbrace{v(\gHstar, \piepessi) - v^{\piepessi}}_{\text{(iv)}}\nend
    &\overset{\cE}{\lesssim} \underbrace{v^{\piestar} - v(\gpessi{\piestar}, \piestar)}_{\text{(i)}} + \underbrace{v(\gHstar, \piepessi) - v^{\piepessi}}_{\text{(iv)}}, 
\end{align}
where $\overset{\cE}{\lesssim}$ means that the inequality holds on event $\cE$ defined in  \eqref{def:cE}. 
Here, $\text{(ii)}\le 0$ holds by definition of $\piepessi$ in \eqref{eq:piepessi} and $\text{(iii)}\overset{\cE}{\lesssim} 0$ holds by definition of $\gpessi{\pie}$ in \eqref{eq:gpessi} and the fact that $\gHstar\in\CICATE(e_\cD)$ on event $\cE$ by Theorem \ref{thm:Fast rate}. Moreover, we show that (iv) is bounded by,
\begin{align}
    \text{(iv)} &= v(\gHstar, \piepessi) - v^{\piepessi}\nend
    & = \int_{\cX\times\cA}\rbr{\gHstar(x, a) - \gstar(x, a)}\tpr(x)\piepessi(a\given x)\rd x\rd a\nend
    & \le \sup_{v\in\cV} \nbr{\gHstar-\gstar}_{v, 2}\le \vareH,\label{eq:iv upper}
\end{align}
where the first inequality holds by the definition of $\cV$ that $\cV=\{v: v(x, a)=\tpr(x)\pie(a\given x), \forall \pie\in\Pie\}$ and the last inequality holds by Assumption \ref{asp:Realizability} on the realizability of the hypothesis class.
Therefore, we just need to bound (i). The upper bound for (i) in CCB-IV and CCB-PV are given in \S\ref{proof: subopt of IV} and \S\ref{proof: subopt of PV}, respectively.




\subsection{Proof of Theorem \ref{thm:IV subopt}} \label{proof: subopt of IV}
\begin{proof}
In this section, we study the estimation error of the average reward function with respect to the optimal interventional policy $\piestar$ under the CCB-IV setting, i.e., term (i) in \eqref{eq:regret}. The error in the estimated CATE is given by
\begin{align*}
    &\EEob\sbr{g^*(A, X)-\gpessi{\piestar}(A, X)\given Z, R_Z=1}\nend
    &=\EEob\sbr{\EEob\sbr{g^*(A, X)-\gpessi{\piestar}(A, X)\given A, X, Z, (R_X, R_Z)=\ind}\given Z, R_Z=1} \nend
    &=\EEob\sbr{\EEob\sbr{\hstark{1}(Y, A, Z)-\gpessi{\piestar}(A, X)\given A, X, Z, (R_X, R_Z)=\ind}\given Z, R_Z=1} \nend
    &=\EEob\sbr{\EEob\sbr{\hstark{1}(Y, A, Z)-\hpessik{1}{\piestar}(Y, A, Z) \given A, X, Z, (R_X, R_Z)=\ind}\given Z, R_Z=1} \nend
    &\quad +\EEob\sbr{\EEob\sbr{\hpessik{1}{\piestar}(Y, A, Z) - \gpessi{\piestar}(A, X)\given A, X, Z, (R_X, R_Z)=\ind}\given Z, R_Z=1},
\end{align*}
where the second equality holds by \eqref{eq:IV bridge 2} with $h_1, g$ substituted by the optimal bridge functions $\hstark{1}, g^*$. Note that $R_X\indep (Y, R_Z)\given (A, X, Z)$, it then follows that
\begin{align}
    &\EEob\sbr{g^*(A, X)-\gpessi{\piestar}(A, X)\given Z, R_Z=1}\nend
    &=\EEob\sbr{\hstark{1}(Y, A, Z)-\hpessik{1}{\piestar}(Y, A, Z)\given Z, R_Z=1} \nend
    &\quad +\EEob\sbr{\EEob\sbr{\hpessik{1}{\piestar}(Y, A, Z) - \gpessi{\piestar}(A, X)\given A, X, Z, (R_X, R_Z)=\ind}\given Z, R_Z=1},\nend
    &=\EEob\sbr{Y-\hpessik{1}{\piestar}(Y, A, Z)\given Z, R_Z=1}\nend
    &\quad
    +\EEob\sbr{\EEob\sbr{\hpessik{1}{\piestar}(Y, A, Z) - \gpessi{\piestar}(A, X)\given A, X, Z, (R_X, R_Z)=\ind}\given Z, R_Z=1}\nend
    & = -\cT_1\vhpessi{\piestar}(Z) -  \EEob\sbr{\cT_2\vhpessi{\piestar}(A, X, Z) \given Z, R_Z=1}, 
    \label{eq:IV 1}
\end{align}
where the second equality holds by plugging in \eqref{eq:IV bridge 1} for the optimal bridge function $\hstark{1}$. 
A change of base distribution in term (i) of the sub-optimality \eqref{eq:regret} gives
\begin{align}
    &v^{\piestar} - v(\gpessi{\piestar}, \piestar) \nend
    & = \EEob\sbr{\rbr{\gstar(A, X)-\gpessi{\piestar}(A, X)}\frac{\tpr(X)\piestar(A\given X)}{\pob(X, A\given R_Z=1)}\given R_Z=1}.\label{eq:IV 2}
\end{align}
By assumption of Theorem \ref{thm:IV subopt} that there exists $b_1: \cZ\rightarrow \RR$ satisfying
\begin{align*}
    \EEob\sbr{b_1(Z)\given A, X, R_Z=1} = \frac{\tpr(X)\piestar(A\given X)}{\pob(X, A\given R_Z=1)}, 
\end{align*}
it holds for \eqref{eq:IV 2} that
\begin{align}
    v^{\piestar} - v(\gpessi{\piestar}, \piestar) 
    &=\EEob\sbr{\rbr{\gstar(A, X)-\gpessi{\piestar}(A, X)}\EEob\sbr{b_1(Z)\given A, X, R_Z=1}\given R_Z=1} \nend
    & = \EEob\sbr{\EEob\sbr{\gstar(A, X)-\gpessi{\piestar}(A, X)\given Z, R_Z=1}b_1(Z)\given R_Z=1}\nend
    & = -\EEob\sbr{\rbr{\cT_1\vhpessi{\piestar}(Z) +  \EEob\sbr{\cT_2\vhpessi{\piestar}(A, X, Z) \given Z, R_Z=1}} b_1(Z)\given R_Z=1}\nend
    & = -\EEob\sbr{\cT_1\vhpessi{\piestar}(Z)b_1(Z)\given R_Z=1} \nend
    &\quad - \EEob\sbr{\cT_2\vhpessi{\piestar}(A, X, Z)b_1(Z)\frac{\pob(X, A, Z\given R_Z=1)}{\pob(A, X, Z\given (R_X, R_Z)=\ind)}\given (R_X, R_Z)=\ind}.\label{eq:IV 4}
\end{align}
where the third equality holds by plugging in \eqref{eq:IV 1}.
We define $b_2:\cA\times\cX\times\cZ\rightarrow \RR$ by
\begin{align*}
    b_2(a, x, z)=\frac{b_1(z)\pob(a, x, z\given R_Z=1)}{\pob(a, x, z\given(R_X, R_Z)=\ind)}.
\end{align*}
Hence, using the Cauchy-Schwarz inequality, \eqref{eq:IV 4} is further bounded by
\begin{align}
    v^{\piestar} - v(\gpessi{\piestar}, \piestar)
    &\le \nbr{\cT_1\vhpessi{\piestar}}_{\mu_1, 2} \nbr{b_1}_{\mu_1, 2} + \nbr{\cT_2\vhpessi{\piestar}}_{\mu_2, 2}\nbr{b_2}_{\mu_2, 2}\nend
    &\le \rbr{\nbr{b_1}_{\mu_1, 2}+\nbr{b_2}_{\mu_2, 2}} \max_{k\in\{1, 2\}} \nbr{\cT_k\vhpessi{\piestar}}_{\mu_k, k}\nend
    &\overset{\cE}{\lesssim} \sum_{k=1}^2\nbr{b_k}_{\mu_k, 2} \rbr{O(\vareTheta) + O(\vareH) + O\rbr{\sqrt{e_\cD}} +  O\rbr{\eta}},\label{eq:IV 5}
\end{align}
where the last inequality holds by Theorem \ref{thm:Fast rate}.
Now combining \eqref{eq:iv upper} and \eqref{eq:IV 5} with \eqref{eq:regret}, we arrive at
\begin{align*}
    \SubOpt(\piepessi) 
    &\overset{\cE}{\lesssim} \sum_{k=1}^2 \nbr{b_k}_{\mu_k, 2}\cdot \rbr{O(\vareTheta) + O(\vareH) + O\rbr{\sqrt{e_\cD}} +  O\rbr{\eta}} + \vareH\nend
    &\le \sum_{k=1}^2 \nbr{b_k}_{\mu_k, 2}\cdot \rbr{O(\vareTheta) + O(\vareH) + O\rbr{\sqrt{e_\cD}} +  O\rbr{\eta}}, 
\end{align*}
where the last inequality holds by noting that $\nbr{b_k}^2_{\mu_k, 2}\ge 1$, which follows from the non-negativity of the chi-squared distance, i.e., 
\begin{align*}
    \chi^2(p, \mu)=\EE_\mu\sbr{\frac{p^2}{\mu^2}-1}=\EE_\mu\sbr{\rbr{\frac{p-\mu}{\mu}}^2}\ge 0.
\end{align*}
Hence, we complete the proof of Theorem \ref{thm:IV subopt}.
\end{proof}




\subsection{Proof of Theorem \ref{thm:PV subopt}}\label{proof: subopt of PV}
\begin{proof}
In \eqref{eq:PV ID 1} and \eqref{eq:PV ID 2}, $h_1$ serves as the bridge function to overcome the problem of missingness in $W$, and $h_2$ is the actual bridge function that we care about. Therefore, we study the difference between $\hpessik{2}{\piestar}$ and $\hstark{2}$ by
\begin{align}
    &\EEob\sbr{\hstark{2}(A, W, X)-\hpessik{2}{\piestar}(A, W, X)\given A, X, Z, (R_X, R_Z)=\ind}\nend
    &=\EEob\sbr{\EEob\sbr{\hstark{2}(A, W, X)-\hpessik{2}{\piestar}(A, W, X)\given A, W, X, Z, (R_W, R_X, R_Z)=\ind}\given A, X, Z, (R_X, R_Z)=\ind}\nend
    &=\EEob\sbr{\hstark{1}(Y, A, X, Z)-\hpessik{1}{\piestar}(Y, A, X, Z)\given A, X, Z, (R_X, R_Z)=\ind} \nend
    &\quad + \EEob\sbr{\EEob\sbr{\hpessik{1}{\piestar}(Y, A, X, Z)-\hpessik{2}{\piestar}(A, W, X)\given A, W, X, Z, (R_W, R_X, R_Z)=\ind}\given A, X, Z, (R_X, R_Z)=\ind}\nend
    &=\cT_1\vhpessi{\piestar}(A, X, Z) + \EEob\sbr{\cT_2\vhpessi{\piestar}(A, W, X, Z)\given A, X, Z, (R_X, R_Z)=\ind}, \label{eq:PV 1}
\end{align}
where the second equality holds by \eqref{eq:PV ID 2} which states that $$\EEob\sbr{\hstark{2}(A, W, X) - \hstark{1}(Y, A, X, Z)\given A, W, X, Z, (R_W, R_X, R_Z)=\ind}=0, $$
and noting that $R_W\indep (R_X, R_Z, Z)\given (A, X, W)$. The third equality holds by the definition of the linear operator $\cT_k$.
Now that we have characterized the difference between $\hpessik{2}{\piestar}$ and $\hstark{2}$, it still remains to see the error in the estimated CATE. 
\begin{align*}
    &\gstar(X, A')-\gpessi{\piestar}(X, A')\nend
    &=\EEob\sbr{\rbr{\hstark{3}(Y, A, X; A') - \hpessik{3}{\piestar}(Y, A, X; A')} + \rbr{\hpessik{3}{\piestar}(Y, A, X; A')-\gpessi{\piestar}(X, A')}\given X, A', R_X=1}\nend
    &=\EEob\sbr{\EEob\sbr{\hstark{3}(Y, A, X;A')-\hpessik{3}{\piestar}(Y, A, X; A')\given A, W, X, A', (R_W, R_X)=\ind}\given X, A', R_X=1}\nend
    &\quad - \cT_4\vhpessi{\piestar}(X, A'),
\end{align*}
where the first equality holds by \eqref{eq:PV ID 6} which states that 
$$\EEob\sbr{\gstar(X, A') - \hstark{3}(Y, A, X, A')\given X, A', R_X=1}=0.$$
The second equality holds also by noting that $R_W\indep (Y, R_X)\given(A, X, W)$ and the definition of $\cT_4$ in the CCB-PV case. We continue with \eqref{eq:PV ID 5} which states that
$$\EEob\sbr{\hstark{3}(Y, A, X,  A')-\hstark{2}(A', W, X)\given A, W, X, A', (R_W, R_X)=\ind}=0, $$
and it holds for $\gstar(X, A')-\gpessi{\piestar}(X, A')$ that
\begin{align}
    &\gstar(X, A')-\gpessi{\piestar}(X, A')\nend
    &=\EEob\sbr{\EEob\sbr{\hstark{2}(A', W, X)-\hpessik{2}{\piestar}(A', W, X)\given A, W, X, A', (R_W, R_X)=\ind}\given X, A', R_X=1}\nend
    &\quad + \EEob\sbr{\EEob\sbr{\hpessik{2}{\piestar}(A', W, X)-\hpessik{3}{\piestar}(Y, A, X, A')\given A, W, X, A', (R_W, R_X)=\ind}\given X, A', R_X=1}\nend
    &\quad  - \cT_4\vhpessi{\piestar}(X, A')\nend
    &=\EEob\sbr{\hstark{2}(A', W, X)-\hpessik{2}{\piestar}(A', W, X)\given X, A', R_X=1} \nend
    &\quad -\EEob\sbr{\cT_3\vhpessi{\piestar}(A, W, X, A')\given X, A', R_X=1} -\cT_4\vhpessi{\piestar}(X, A'),\label{eq:PV 2}
\end{align}
where the second equality holds by definition of $\cT_3$ in the CCB-PV case. 
Now we plug \eqref{eq:PV 2} into (i) of \eqref{eq:regret} and it follows that
\begin{align}
    \text{(i)} &= v^{\piestar} - v(\gpessi{\piestar}, \piestar)\nend
    &=\int_{\cX\times\cA} \rbr{\gstar(x, a')-\gpessi{\piestar}(x, a')}\tpr(x)\piestar(a'\given x)\rd x\rd a'\nend
    &=\underbrace{\int_{\cX\times\cA}\EEob\sbr{\hstark{2}(A', W, X)-\hpessik{2}{\piestar}(A', W, X)\given X, A', R_X=1} \tpr(x)\piestar(a'\given x)\rd x\rd a'}_{(a)} \nend
    &\quad \underbrace{-\int_{\cX\times\cA}\EEob\sbr{\cT_3\vhpessi{\piestar}(A, W, X, A')\given X, A', R_X=1} \tpr(x)\piestar(a'\given x)\rd x\rd a'}_{(b)} \nend
    &\quad \underbrace{-\int_{\cX\times\cA}\cT_4\vhpessi{\piestar}(X, A') \tpr(x)\piestar(a'\given x)\rd x\rd a'}_{(c)}.\label{eq:PV 3}
\end{align}
To upper bound (b) and (c), we define two ratio functions $b_4:\cX\times\cA\rightarrow \RR$ and $b_3:\cW\times\cX\times\cA\times\cA'\rightarrow \RR$ by
\begin{gather}
    b_4(x, a') = \frac{\tpr(x)\piestar(a'\given x)}{\pob(x\given R_X=1)u(a')}, \label{eq:PV b4}\\
    b_3(w, x, a, a') = \frac{\tpr(x)\piestar(a'\given x)\pob(a, w\given x, R_X=1)}{u(a')\pob(x, a, w\given (R_W, R_X)=\ind)}.\label{eq:PV b3}
\end{gather}
For (b), with $b_3$ defined in \eqref{eq:PV b3} we have
\begin{align}
    (b) = -\EEob\sbr{\cT_3\vhpessi{\piestar}(A, W, X, A')b_3(W, X, A, A')\given (R_W, R_X)=\ind}\le \nbr{\cT_3\vhpessi{\piestar}}_{\mu_3, 2}\nbr{b_3}_{\mu_3, 2}. \label{eq:PV (b)}
\end{align}
Similarly, for (c) with $b_4$ defined in \eqref{eq:PV b4} we have
\begin{align}
    (c) = \EEob\sbr{-\cT_4\vhpessi{\piestar}(X, A')b_4(X, A')\given R_X=1}\le \nbr{\cT_4\vhpessi{\piestar}}_{\mu_4, 2}\nbr{b_4}_{\mu_4, 2}.\label{eq:PV (c)}
\end{align}
In addition, it holds for (a) that
\begin{align*}
    (a)&=\int_{\cX\times\cA} \EEob\sbr{\hstark{2}(A', W, X)-\hpessik{2}{\piestar}(A', W, X)\given X, A', R_X=1} \tpr(x)\piestar(a'\given x) \rd x\rd a'\nend
    &=\int_{\cX\times\cA} \EEob\sbr{\hstark{2}(a, W, X)-\hpessik{2}{\piestar}(a, W, X)\given X=x, R_X=1} \tpr(x)\piestar(a\given x) \rd x\rd a \nend
    &=\int_{\cX\times\cA}\EEob\sbr{\EEob\sbr{\hstark{2}(A, W, X)-\hpessik{2}{\piestar}(A, W, X)\given X, U, A=a, (R_X, R_Z)=\ind}\given X=x, R_X=1}
    \nend
    &\quad \cdot \tpr(x)\piestar(a\given x) \rd x\rd a, 
\end{align*}
where the second equality holds by noting that $A'\indep W, X$, and the third equality holds by noting that $R_X\indep W\given X$, $A\indep W\given (X, U)$, and $(R_X, R_Z)\indep W\given (A, U, X)$. Now we continue by 
\begin{align}
    (a)
    & = \EEob\bigg[\EEob\sbr{\hstark{2}(A, W, X)-\hpessik{2}{\piestar}(A, W, X)\given X, U, A, (R_X, R_Z)=\ind} \nend
    &\quad \cdot \frac{\pob(U\given X, R_x=1)\tpr(X)\piestar(A\given X)}{\pob(U, X, A\given (R_X, R_Z)=\ind)}\given (R_X, R_Z)=\ind\bigg], \label{eq:PV 4}
\end{align}
which prompts us to introduce another ratio function. 
Since $Z$ is over-complete over $U$, there exists $b_1: \cX\times\cA\times\cZ\rightarrow \RR$ such that
\begin{align}
    \EEob\sbr{b_1(X, A, Z)\given X, U, A, (R_X, R_Z)=\ind} = \frac{\pob(U\given X, R_x=1)\tpr(X)\piestar(A\given X)}{\pob(U, X, A\given (R_X, R_Z)=\ind)}.\label{eq:PV 5}
\end{align}
Plugging \eqref{eq:PV 5} into \eqref{eq:PV 4}, it holds that
\begin{align*}
    (a) 
    &= \EEob\bigg[\EEob\sbr{\hstark{2}(A, W, X)-\hpessik{2}{\piestar}(A, W, X)\given X, U, A, (R_X, R_Z)=\ind}\nend
    & \quad \cdot \EEob\sbr{b_1(X, A, Z)\given X, U, A, (R_X, R_Z)=\ind}\given (R_X, R_Z)=\ind\bigg]\nend
    & = \EEob\sbr{\rbr{\hstark{2}(A, W, X)-\hpessik{2}{\piestar}(A, W, X)}b_1(X, A, Z)\given  (R_X, R_Z)=\ind}\nend
    & = \EEob\sbr{\EEob\sbr{\hstark{2}(A, W, X)-\hpessik{2}{\piestar}(A, W, X)\given X, A, Z, (R_X, R_Z)=\ind} b_1(X, A, Z)\given  (R_X, R_Z)=\ind},
\end{align*}
where the second equality holds by noting that $W\indep (Z, R_Z)\given (X, A, U, R_X)$. Now combining \eqref{eq:PV 1}, we have
\begin{align*}
    (a) &= \EEob\bigg[\rbr{\cT_1\vhpessi{\piestar}(A, X, Z) + \EEob\sbr{\cT_2\vhpessi{\piestar}(A, W, X, Z)\given A, X, Z, (R_X, R_Z)=\ind}}\nend
    &\quad \cdot b_1(X, A, Z)\given (R_X, R_Z)=\ind\bigg]\nend
    &= \EEob\sbr{\cT_1\vhpessi{\piestar}(A, X, Z) b_1(X, A, Z)\given (R_X, R_Z)=\ind}\nend
    &\quad + \EEob\sbr{\EEob\sbr{\cT_2\vhpessi{\piestar}(A, W, X, Z)b_1(X, A, Z)\given A, W, X, (R_W, R_X, R_Z)=\ind}\given (R_X, R_Z)=\ind}\nend
    &= \EEob\sbr{\cT_1\vhpessi{\piestar}(A, X, Z) b_1(X, A, Z)\given (R_X, R_Z)=\ind}\nend
    &\quad + \EEob\sbr{\cT_2\vhpessi{\piestar}(A, W, X, Z)b_1(X, A, Z)\frac{\pob(A, W, X\given (R_X, R_Z)=\ind)}{\pob(A, W, X\given (R_W, R_X, R_Z)=\ind)}\given (R_W, R_X, R_Z)=\ind} ,
\end{align*}
where the second equality holds by $R_W\indep (Z, R_X, R_Z)\given (A, W, X)$.
We thereby define $b_2: \cA\times\cW\times\cX\times\cZ\rightarrow \RR$ by
\begin{align*}
    b_2(A, W, X, Z) = b_1(X, A, Z)\frac{\pob(A, W, X\given (R_X, R_Z)=\ind)}{\pob(A, W, X\given (R_W, R_X, R_Z)=\ind)}.
\end{align*}
We then arrive at
\begin{align}
    (a)&=\EEob\sbr{\cT_1\vhpessi{\piestar}(A, X, Z) b_1(X, A, Z)\given (R_X, R_Z)=\ind}\nend
    &\quad + \EEob\sbr{\cT_2\vhpessi{\piestar}(A, W, X, Z)b_2(A, W, X, Z)\given (R_W, R_X, R_Z)=\ind}\nend
    &\le \nbr{\cT_1\vhpessi{\piestar}}_{\mu_1, 2}\nbr{b_1}_{\mu_1, 2} + \nbr{\cT_2\vhpessi{\piestar}}_{\mu_2, 2}\nbr{b_2}_{\mu_2, 2}, \label{eq:PV (a)}
\end{align}
Combining \eqref{eq:PV (a)}, \eqref{eq:PV (b)} and \eqref{eq:PV (c)} with \eqref{eq:PV 3}, we have
\begin{align*}
    v^{\piestar} - v(\gpessi{\piestar})
    &\le \sum_{k=1}^4 \nbr{\cT_k\vhpessi{\piestar}}_{\mu_k, 2}\nbr{b_k}_{\mu_k, 2}
    \le \max_{k\in\{1, 2, 3,4\}}\nbr{\cT_k\vhpessi{\piestar}}_{\mu_k, 2}  \sum_{k=1}^4 \nbr{b_k}_{\mu_k, 2}.
\end{align*}
Combined with the sub-optimality in \eqref{eq:regret} and the conclusion of Theorem \ref{thm:Fast rate},  it follows that
\begin{align*}
    \SubOpt(\piepessi) &\le \max_{k\in\{1, 2, 3,4\}}\nbr{\cT_k\vhpessi{\piestar}}_{\mu_k, 2}  \sum_{k=1}^K \nbr{b_k}_{\mu_k, 2} + \vareH\nend
    &\overset{\cE}{\lesssim} \sum_{k=1}^4 \nbr{b_k}_{\mu_k, 2} \cdot \big(\cO(\vareTheta) + \cO(\vareH) + \cO\rbr{\sqrt{e_\cD}} +  \cO\rbr{\eta}\big) + \vareH\nend
    &\le \sum_{k=1}^4 \nbr{b_k}_{\mu_k, 2} \cdot \big(\cO(\vareTheta) + \cO(\vareH) + \cO\rbr{\sqrt{e_\cD}} +  \cO\rbr{\eta}\big), 
\end{align*}
which completes the proof of Theorem \ref{thm:PV subopt}.
\end{proof}
% \begin{align}
%     v^\piestar-v(\gpessi{\piestar}, \piestar) 
%     &= \EE_{X\sim\tilde p}\sbr{f^\piestar_4(X)-\tilde h^\piestar_{4, \cD}(X)}\nend
%     & = \EEob\sbr{\rbr{f^\piestar_3(X)-\tilde h^\piestar_{3, \cD}(X)}\frac{\tilde p(X)}{p(X)}}\nend
%     & = \underbrace{\EEob\sbr{\sum_a\rbr{ f^\piestar_2(a, X)-\tilde h^\piestar_{2, \cD}(a, X)}\pie(a\given X)b_1(X)}}_{(i)}\nend
%     &\quad + \underbrace{\EEob\sbr{\rbr{\sum_a \tilde h^\piestar_{2, \cD}(a, X)\pie(a\given X)-\tilde h^\piestar_{3,\cD}(X)}b_1(X)}}_{(ii)}.
% \end{align}
% For term (i), we have
% \begin{align}
%     (i)&=\EEob\sbr{\rbr{f^\piestar_2(A, X)-\tilde h^\piestar_{2, \cD}(A, X)}\frac{\pie(A\given X)b_1(X)}{\pib(A\given X, Z, U)}}\nend
%     &=\EEob\sbr{\rbr{f^\piestar_2(A, X)-\tilde h^\piestar_{2, \cD}(A, X)}\EEob\sbr{\frac{\pie(A\given X)b_1(X)}{\pib(A\given X, Z, U)}\given A, X}}\nend
%     &=\EEob\sbr{\rbr{f^\piestar_2(A, X)-\tilde h^\piestar_{2, \cD}(A, X)}\EEob\sbr{b_2(Z)\given A, X}}\nend
%     &=\underbrace{\EEob\sbr{\rbr{f^\piestar_1(Y, A, Z)-\tilde h^\piestar_{1, \cD}(Y, A, Z)}b_2(Z)}}_{(iii)}\nend
%     &\quad + \underbrace{\EEob\sbr{\rbr{\tilde h^\piestar_{1,\cD}(Y, A, Z)-\tilde h^\piestar_{2, \cD}(A, X)}b_2(Z)}}_{(iv)}.
% \end{align}
% It then holds that
% \begin{align}
%     v^\piestar-v(\gpessi{\piestar}) = (ii) + (iii) + (iv).
% \end{align}


% We study term (i) of \eqref{eq:regret} in the PV case.
% \begin{align}
%     v^{\piestar} - v(\gpessi{\piestar}) 
%     &=\EE_{X\sim \tilde p}\sbr{f^\piestar_4(X)-\tilde h^\piestar_{4, \cD}(X)}\nend
%     &=\EE_{\pib}\sbr{\rbr{f^\piestar_4(X)-\tilde h^\piestar_{4, \cD}(X)}\frac{\tilde p(X)}{p(X)}}\nend
%     &=\underbrace{\EEob\sbr{\rbr{f^\piestar_3(Y, A, X)-\tilde h^\piestar_{3, \cD}(Y, A, X)}b_1(X)}}_{(i)} + \underbrace{\EEob\sbr{\rbr{\tilde h^\piestar_{3, \cD}(Y, A, X) - \tilde h^\piestar_{4, \cD}(X)}b_1(X)}}_{(ii)}.
% \end{align}
% It then holds for (i) that
% \begin{align}
%     (i)&=\EEob\sbr{\EEob\sbr{f^\piestar_3(Y, A, X)-\tilde h^\piestar_{3, \cD}(Y, A, X)\given A, W, X}b_1(X)}\nend
%     &=\underbrace{\EEob\sbr{\EEob\sbr{\sum_a f^\piestar_2(a, X, W)-\sum_a \tilde h^\piestar_{2, \cD}(a, X, W)\given A, X, W}b_1(X)}}_{(iii)}\nend
%     &\quad + \underbrace{\EEob\sbr{\EEob\sbr{\sum_a \tilde h^\piestar_{2, \cD}(a, X, W) - \tilde h^\piestar_{3, \cD}(Y, A, X)\given A, X, W}b_1(X)}}_{(iv)},
% \end{align}
% where $b_1(X)=\tilde p(X)/p(X)$. For (i) it holds that
% \begin{align}
%     (iii) &= \EEob\sbr{\rbr{f^\piestar_2(A, W, X)-\tilde h^\piestar_{2, \cD}(A, X, W)}\frac{b_1(X)}{\pib(A\given X, U, Z)}}\nend
%     &=\EEob\sbr{\rbr{f^\piestar_2(A, W, X)-\tilde h^\piestar_{2, \cD}(A, X, W)}\frac{b_1(X)p(A, W, X, Z, U)}{\pib(A\given X, U, Z)p(A, W, X, Z, U\given R_Z=\ind)}\given R_Z=\ind}\nend
%     &=\EEob\sbr{\rbr{f^\piestar_2(A, W, X)-\tilde h^\piestar_{2, \cD}(A, X, W)}\EEob\sbr{\frac{b_1(X)p(A, Z, U)}{\pib(A\given X, U, Z)p(A, Z, U\given R_Z=\ind)}\given A, X, W, R_Z=\ind}\given R_Z=\ind}\nend
%     &=\EEob\sbr{\rbr{f^\piestar_2(A, W, X)-\tilde h^\piestar_{2, \cD}(A, X, W)}\EEob\sbr{b_2(A, X, Z)\given A, X, W, R_Z=\ind}\given R_Z=\ind}\nend
%     &=\EEob\sbr{\rbr{f^\piestar_2(A, W, X)-\tilde h^\piestar_{2, \cD}(A, X, W)}b_2(A, X, Z)\given R_Z=\ind}\nend
%     &=\EEob\sbr{\EEob\sbr{f^\piestar_2(A, W, X)-\tilde h^\piestar_{2, \cD}(A, X, W)\given A, X, Z, R_Z=\ind}b_2(A, X, Z)\given R_Z=\ind}
% \end{align}
% Note that
% \begin{align}
%     &\EEob\sbr{f^\piestar_2(A, W, X)-\tilde h^\piestar_{2, \cD}(A, W, X)\given A, X, Z, R_Z=\ind}\nend
%     &=\EEob\sbr{f^\piestar_1(Y, A, X, Z)-\tilde h^\piestar_{1, \cD}(Y, A, X, Z)\given A, X, Z, R_Z=\ind}\nend
%     &\quad + \EEob\sbr{\EEob\sbr{\tilde h^\piestar_{1, \cD}(Y, A, X, Z)-\tilde h^\piestar_{2, \cD}(A, W, X)-Y\pie(A\given W, X)\given A, W, X, Z, R_Z=\ind}\given A, X, Z, R_Z=\ind}.
% \end{align}
% Hence, it holds that
% \begin{align}
%     (iii) &= \underbrace{\EEob\sbr{\EEob\sbr{f^\piestar_1(Y, A, X, Z)-\tilde h^\piestar_{1, \cD}(Y, A, X, Z)\given A, X, Z, R_Z=\ind} b_2(A, X, Z)\given R_Z=\ind}}_{(v)}\nend
%     &\quad + \underbrace{\EEob \sbr{\EEob\sbr{\tilde h^\piestar_{1, \cD}(Y, A, X, Z)-\tilde h^\piestar_{2, \cD}(A, W, X)-Y\pie(A\given W, X)\given A, W, X, Z, R_Z=\ind}b_2(A, X, Z)\given R_Z=\ind}}_{(vi)}.
% \end{align}
% Therefore, we have
% \begin{align}
%     v^{\piestar} - v(\gpessi{\piestar}) = (ii)+(iv)+(v)+(vi).
% \end{align}

