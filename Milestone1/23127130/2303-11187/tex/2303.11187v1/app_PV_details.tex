\section{A Matrix Explanation for the IES of the CCB-PV}\label{app:PV details} 
The matrix explanation for CCB-PV is just the same as CCB-IV. 
Without missingness in $W$, it can be easily verified that \eqref{eq:PV ID 1}-\eqref{eq:PV ID 4} give a reduced integral equation system, 
\begin{gather}
    \EEob\sbr{Y-h_2(A, W, X)\given A=a, X=x, Z=z, (R_Z, R_X)=\ind} = 0, \label{eq:PV matrix id 1}\\
    g(x, a') = \EEob\sbr{h_2(a', W, X)\given X=x, R_X=1},\label{eq:PV matrix id 2}
\end{gather}
which is consistent with the standard identification equations for the PV model \citep{miao2018confounding, miao2018identifying} by also ignoring the conditions for $R_X$ and $R_Z$. 
Now, we show how we get around the missingness of $W$. We can rewrite \eqref{eq:PV matrix id 1} as
\begin{align}
    &\EEob\sbr{Y\given A=a, X=x, Z=z, (R_Z, R_X)=\ind}\nend
    &= \underbrace{h_2(A, W, X) P(Y\given W, a, x, z, (R_Z, R_X, R_W)=\ind)^\dagger}_{h_1(Y, a, x, z)} P(Y\given a, x, z, (R_X, R_Z)=\ind)
    % \nend
    % &=h_2(A, W, X) P(W\given a, x, z, (R_X, R_Z)=\ind)
    ,  \label{eq:PV matrix 1}
\end{align}
where the equality holds by noting that
\begin{align}
    P(Y\given a, x, z, (R_X, R_Z)=\ind)=P(Y\given W, a, x, z, (R_Z, R_X, R_W)=\ind)P(W\given a, x, z, (R_Z, R_X)=\ind), \label{eq:PV matrix 3}
\end{align}
since $R_W$ is only caused by $(W, X, A)$.
We also rewrite \eqref{eq:PV matrix id 2} as
\begin{align}
    g(x, a') &= \sum_{a\in\cA}\underbrace{h_2(a', W, x) P(Y\given W, a, x, (R_W, R_X)=\ind)^\dagger }_{h_3(Y, a, x, a')}P(Y, a\given x, R_x=1)
    % \nend
    % &=h_2(a', W, x)P(W\given x, R_X=1)
    ,  \label{eq:PV matrix 2}
\end{align}
where the equality holds by noting that
\begin{align}
    P(Y, a\given x, R_X=1)=P(Y\given W, a, x, (R_W, R_X)=\ind) P(W, a\given x, R_X=1).\label{eq:PV matrix 4}
\end{align}
Here, \eqref{eq:PV matrix 3} and \eqref{eq:PV matrix 4} hold by the chain rule and noting that $R_W\indep (Y, Z, R_Z, R_X)\given (A, X, W)$.
Similar to the matrix explanation for the CCB-IV case, we show that \eqref{eq:PV matrix 1} with the introduction of bridge function $h_1$ gives \eqref{eq:PV ID 1} and \eqref{eq:PV ID 2}  while \eqref{eq:PV matrix 2} with the introduction of bridge function $h_3$ gives \eqref{eq:PV ID 3} and \eqref{eq:PV ID 4}.