
\section{Technical Results}\label{app:technical}
\begin{lemma}[Corollary 14.3 in \cite{wainwright2019high}]
% \label{lem:Critical radius by covering number}
Let $N_\cD(t;\BB_\cD(\delta;\sF))$ denote the $t$-covering number of the set $\BB_\cD(\delta;\sF)=\{f\in\sF: \nbr{f}_\cD\le \delta\}$ in the empirical $L^2(\PP_\cD)$-norm. Then the empirical version of critical inequality
\begin{align*}
    \cR_\cD(\eta;\cF)\le \frac{\eta^2}{C}
\end{align*}
is satisfied for any $\eta>0$ such that 
\begin{align*}
    \frac{64}{\sqrt{n}}\int_{\frac{\eta^2}{2C}}^{\eta}\sqrt{\log N_\cD(t;\BB_\cD(\eta;\sF))}\rd t\le \frac{\eta^2}{C}.
\end{align*}
\end{lemma}

\begin{lemma}[Covering number for summation class]\label{lem:covering number summation}
Let $N(t;\cF,\nbr{\cdot})$ denote the $t$-covering number of a set $\cF$ on a metric space equipped with norm $\nbr{\cdot}$ such that the triangle inequality holds. For function classes $\cF_1, \cF_2$, let $\cF$ denote their summation class, 
\begin{align*}
    \cF=\cbr{f\given f=f_1+f_2, f_1\in\cF_1, f_2\in\cF_2}.
\end{align*}
The $t$-covering number for $\cQ$ satisfies
\begin{align}
    N(t;\cF, \nbr{\cdot})\le N\rbr{\frac{t}{2};\cF_1, \nbr{\cdot}} \cdot N\rbr{\frac{t}{2};\cF_2, \nbr{\cdot}}.\label{eq:covering number summation}
\end{align}
\begin{proof}
Suppose that $\Theta_1$ is a $t/2$-covering of $\cF_1$, $\Theta_2$ is a $t/2$-covering of $\cF_2$. We construct
\begin{align*}
    \Theta = \cbr{\theta\given \theta=\theta_1+\theta_2, \theta_1\in\Theta_1, \theta_2\in\Theta_2}.
\end{align*}
For any $f\in\cF$, there exist $f_1\in\cF_1, f_2\in\cF_2$ such that $f=f_1+f_2$. Moreover, by definition of the covering set, there exist $\theta_1\in\Theta_1, \theta_2\in\Theta_2$ such that $\nbr{f_1-\theta_1}\le t/2$ and that $\nbr{f_2-\theta_2}\le t/2$. Let $\theta\in\Theta$ such that $\theta=\theta_1+\theta_2$, it follows that 
\begin{align}
    \nbr{f-\theta}
    &= \nbr{(f_1+f_2)-(\theta_1+\theta_2)}\nend
    &\le \nbr{f_1-\theta_1} +\nbr{f_2-\theta_2}\nend
    &\le t,\label{app:tech 1}
\end{align}
where the first inequality holds by the triangle inequality of metric $\nbr{\cdot}$. It follows from \eqref{app:tech 1} that the product set $\Theta$ is a $t$-covering set of $\cF$. Therefore, we conclude that \eqref{eq:covering number summation} holds.
\end{proof}
\end{lemma}

\begin{lemma}[Covering number for product class]\label{lem:covering number product}
Let $N(t;\cF,\nbr{\cdot})$ denote the $t$-covering number of a set $\cF$ on a metric space equipped with norm $\nbr{\cdot}$ such that the triangle inequality holds. For uniformly bounded function classes $\cF_1, \cF_2$ such that $\nbr{\cF_1}_{\infty}\le C_1$ and $\nbr{\cF_2}_{\infty}\le C_2$, let $\cF$ denote their product class,
\begin{align*}
    \cF=\cbr{f\given f=f_1\cdot f_2, f_1\in\cF_1, f_2\in\cF_2}.
\end{align*}
The $t$-covering number for $\cF$ satisfies
\begin{align*}
    N(t;\cF, \nbr{\cdot})\le N\rbr{\frac{t}{2C_2};\cF_1, \nbr{\cdot}} \cdot N\rbr{\frac{t}{2C_1};\cF_2, \nbr{\cdot}}.
\end{align*}
\begin{proof}
Suppose that $\Theta_1$ is a $t/2C_2$-covering of $\cF_1$, $\Theta_2$ is a $t/2C_1$-covering of $\cF_2$. We construct
\begin{align*}
    \Theta = \cbr{\theta\given \theta=\theta_1\cdot \theta_2, \theta_1\in\Theta_1, \theta_2\in\Theta_2}.
\end{align*}
For any $f\in\cF$, there exist $f_1\in\cF_1, f_2\in\cF_2$ such that $f=f_1\cdot f_2$. Moreover, by definition of the covering set, there exist $\theta_1\in\Theta_1, \theta_2\in\Theta_2$ such that $\nbr{f_1-\theta_1}\le t/2C_2$ and that $\nbr{f_2-\theta_2}\le t/2C_1$.
Let $\theta\in\Theta$ such that $\theta=\theta_1\cdot\theta_2$, it follows that 
\begin{align*}
    \nbr{f-\theta}
    &= \nbr{(f_1\cdot f_2)-(\theta_1\cdot\theta_2)}\nend
    &\le \nbr{f_1-\theta_1}\cdot C_2 +\nbr{f_2-\theta_2}\cdot C_1\nend
    &\le t,
\end{align*}
where the first inequality holds by the triangle inequality of metric $\nbr{\cdot}$. It follows from \eqref{app:tech 1} that the product set $\Theta$ is a $t$-covering set of $\cF$.
\end{proof}
\end{lemma}

\begin{lemma}[Covering number for bounded linear class]\label{lem:covering number linear}
Suppose that $\cF$ is a bounded linear function class defined as
\begin{align*}
    \cF=\cbr{f\given f(x)=w^\top \phi(x), w\in\RR^d, \phi:\cX\rightarrow \RR^d, \nbr{\phi}_{2, \infty}\le 1, \nbr{w}_2\le C}.
\end{align*}
The covering number $N(t;\cF, \nbr{\cdot}_\cD)$ with respect to norm $\nbr{\cdot}_\cD$ is bounded by
\begin{align*}
    \log N(t;\cF,\nbr{\cdot}_\cD)\le d\log\rbr{1+\frac{2C}{t}}.
\end{align*}
\begin{proof}
This lemma is a conclusion of Example 5.8 in \cite{wainwright2019high} which states that 
\begin{align*}
    \log N(t;\BB, \nbr{\cdot})\le d\log\rbr{1+\frac 2 \delta},
\end{align*}
if $\BB$ is also a unit ball under norm $\nbr{\cdot}$. In case of $\cF$, we construct a function class $\cS$ defined as
\begin{align}
    \cS=\cbr{w^\top\phi\given \nbr{w^\top \phi(\cdot)}_{\cD}\le C}.\label{eq:covering unit ball}
\end{align}
It is straightforward that $\cF\subseteq\cS$. Therefore, it follows from \eqref{eq:covering unit ball} that
\begin{align*}
    \log N(t;\cF, \nbr{\cdot}_\cD)\le \log N(t;\cS, \nbr{\cdot}) \le  d\log\rbr{1+\frac{2C}{t}}
\end{align*}
\end{proof}
\end{lemma}

\begin{lemma}[critical radius for product of bounded linear function classes]\label{lem:critical radius for product}
Consider two linear function class $\cF_1$ and $\cF_2$ defined as
\begin{align*}
    \cF_1=\cbr{w_1\in\RR^{d_1}:\cX\rightarrow w_1^\top\phi_1(\cdot), \nbr{w_1}_2\le C_1, \nbr{\phi_1(\cdot)}_2\le 1}, \nend
    \cF_2 =\cbr{w_2\in\RR^{d_2}:\cX\rightarrow w_2^\top \phi_2(\cdot), \nbr{w_2}_2\le C_2, \nbr{\phi_2(\cdot)}_2\le 1}.
\end{align*}
The product space of $\cF_1$ and $\cF_2$ is defined as
\begin{align*}
    \cQ=\cbr{\cX\rightarrow f_1(\cdot) f_2(\cdot): f_1\in\cF_1, f_2\in\cF_2}, 
\end{align*}
and the critical radius of $\cQ$ is bounded by 
\begin{align*}
    \eta \le 64 b\sqrt{\frac{d_1+d_2}{n}\cdot \log \rbr{\rbr{1+8\max\cbr{\frac{1}{C_1}, \frac{1}{C_2}}}\frac{n}{64^2(d_1+d_2)}}}, 
\end{align*}
where $b =\sqrt{C\rbr{8\max\{C_1, C_2\}+C}}$ and $C=C_1C_2$.
\begin{proof}
Note that we always have $\nbr{f_1(\cdot)}_\infty\le \nbr{w_1}_2\nbr{\phi_1(\cdot)}_2\le C_1$ for any $f_1\in\cF_1$. It also holds that $\nbr{f_2(\cdot)}_\infty\le C_2$ and that $\nbr{q}_\infty\le C_1C_2=C$ for any $q\in\cQ$. The critical radius $\eta$ for $\cQ$ then satisfies,
\begin{align}
    \cR_\cD(\eta;\cQ)\le \frac{\eta^2}{C}.\label{eq:6 critical radius}
\end{align}
Let $N_\cD(t;\BB_\cD(\eta;\cQ))$ denote the $t$-covering number of the set $\BB_\cD(\eta;\cQ)=\{f\in\cQ: \nbr{f}_\cD\le \eta\}$ in the empirical $L^2(\PP_\cD)$-norm. Then the empirical version of critical inequality \eqref{eq:6 critical radius}
is satisfied for any $\eta>0$ such that 
\begin{align}
    \frac{64}{\sqrt{n}}\int_{\frac{\eta^2}{2C}}^{\eta}\sqrt{\log N_\cD(t;\BB_\cD(\eta;\cQ))}\rd t\le \frac{\eta^2}{C}.\label{eq:1 critical radius}
\end{align}
Such a property holds by Corollary 14.3 in \cite{wainwright2019high}. By definition of $\BB_\cD(\eta;\cQ)$, it holds directly 
\begin{align*}
    \log N_\cD(t;\BB_\cD(\eta;\cQ))&\le \log N_\cD(t;\cQ)\nend
    &\le \log N_\cD\rbr{\frac{t}{2C_2};\cF_1} + \log N_\cD\rbr{\frac{t}{2C_1};\cF_2}\nend
    &\le \log N_\cD\rbr{\frac{t}{2C_2};\cS_\cD(C_1;\phi_1)} + \log N_\cD\rbr{\frac{t}{2C_1};\cS_\cD(C_2;\phi_2)},
\end{align*}
where we define $\cS(C;\phi)=\cbr{w\in \RR^d:  w\in\RR^{d}, \nbr{w^\top\phi(\cdot)}_\cD\le C}$. 
Here, the second inequality holds by Lemma \ref{lem:covering number product} and the last inequality holds by noting that $\cF_1\in\cS(C_1;\phi_1)$ and $\cF_2\in\cS(C_2;\phi_2)$. Note that the norms corresponding to $\cS(C;\phi)$ and covering number $N_\cD$ are both $\nbr{\cdot}_\cD$. Thereby applying Lemma 5.7 in \cite{wainwright2019high}, it follows that,
\begin{align*}
    \log N_\cD(\delta;\cS_\cD(C;\phi))\le d\log \rbr{\frac {2C}{\delta} + 1}.
\end{align*}
Therefore, we show that
\begin{align*}
    \log N_\cD(t;\BB_\cD(\eta;\cQ))\le d_1\log\rbr{\frac{4C_2}{t}+1} + d_2\log\rbr{\frac{4C_1}{t}+1}.
\end{align*}
The left hand-side of \eqref{eq:1 critical radius} is bounded by
\begin{align}
    \frac{64}{\sqrt{n}}\int_{\frac{\eta^2}{2C}}^{\eta}\sqrt{\log N_\cD(t;\BB_\cD(\eta;\cQ))}\rd t &\le \frac{64}{\sqrt{n}} \eta \sqrt{\log N_\cD\rbr{\frac{\eta^2}{2C};\BB_\cD(\eta;\cQ)}}\nend
    &\le \frac{64}{\sqrt{n}} \eta \sqrt{
    d_1\log\rbr{\frac{8C C_2}{\eta^2}+1} + d_2\log\rbr{\frac{8C C_1}{\eta^2}+1}
    }\nend
    &\le \frac{64}{\sqrt{n}} \eta \sqrt{
    (d_1+d_2)\log\rbr{\frac{8C\max\cbr{C_1, C_2}}{\eta^2}+\frac{C^2}{\eta^2}}
    },
    \label{eq:3 critical radius}
\end{align}
where the last inequality holds by noting that $\eta<C$.
Therefore, an upper bound for the critical radius is given by  plugging \eqref{eq:3 critical radius} into \eqref{eq:1 critical radius},
\begin{align}
    \frac{64}{\sqrt{n}} \eta \sqrt{
    (d_1+d_2)\log\rbr{\frac{8C\max\cbr{C_1, C_2}}{\eta^2}+\frac{C^2}{\eta^2}}}
    \le \frac{\eta^2}{C}. \label{eq:4 critical radius}
\end{align}
A little transformation of \eqref{eq:4 critical radius} gives
\begin{align*}
    a \log \frac{b^2}{\eta^2}\le 
    \frac{\eta^2}{b^2}, 
\end{align*}
where 
\begin{align}
    a=\sqrt{\frac{C}{8\max\{C_1, C_2\}+C}} \frac{64^2}{n}(d_1+d_2), \quad b =\sqrt{C\rbr{8\max\{C_1, C_2\}+C}}.\label{eq:5 critical radius}
\end{align}
By assuming that $a<1/2$, we see that $\eta=b \sqrt{a\log \frac 1 a} $ satisfies \eqref{eq:5 critical radius}.
Therefore, the critical radius $\eta$ is upper bounded by
\begin{align*}
    \eta &\le  64 b\sqrt{\frac{d_1+d_2}{n}\cdot \log \rbr{\rbr{1+8\max\cbr{\frac{1}{C_1}, \frac{1}{C_2}}}\frac{n}{64^2(d_1+d_2)}}}\nend
    &\le \cO\rbr{C\sqrt{\frac{d_1+d_2}{n}\log \rbr{\frac{n}{d_1+d_2}}}}.
\end{align*}
\end{proof}
\end{lemma}



\begin{lemma}[Covering number for the policy class $\Pie$ in \eqref{def:linear policy}]
For the policy class\label{lem:covering number Pie}
\begin{align*}
    \Pie \subseteq \cbr{\pie \bigg | \pie(a\given o, o^-;w_0)=\frac{\exp\rbr{w_0^\top\phi_0(a, o, o^-)}}{\sum_{a'\in\cA}\exp\rbr{w_0^\top\phi_0(a', o, o^-)}}, w_0\in\RR^{m_0}, \nbr{w_0}_2\le C_0,  \nbr{\phi_0(\cdot)}_{2, \infty}\le 1},
\end{align*}
the covering number $N(t;\Pie, \nbr{\cdot}_\cD)$ is bounded by,
\begin{align*}
    \log N(t;\Pie, \nbr{\cdot}_\cD)\le m_0 \log\rbr{1+\frac{6 C_0}{t}}.
\end{align*}
\begin{proof}
For simplicity, let $\zeta(a, \tau;w)=\exp(w^\top \phi_0(a, o, o^-))$ where $\tau=(o, o^-)$. For $w$ and $w'$ with bounded quadratic norms, the policy difference of such a softmax policy class can be bounded by
\begin{align}
    &\abr{\pie(a\given \tau;w)-\pie(a\given \tau;w')}\nend
    &\le \frac{\abr{\zeta(a, \tau;w)-\zeta(a, \tau;w')}\sum_{a'}\zeta(a', \tau;w') + \zeta(a, \tau;w')\abr{\sum_{a'}\zeta(a', \tau;w')-\sum_{a'}\zeta(a',\tau;w)}}{\sum_{a'}\zeta(a', \tau;w) \cdot \sum_{a'}\zeta(a', \tau;w')}.\label{eq:pie diff}
\end{align}
Without loss of generality, we assume that $\sum_{a'}\zeta(a',\tau;w')\le \sum_{a'}\zeta(a',\tau;w)$.
Therefore, \eqref{eq:pie diff} can be further bounded by
\begin{align}
    &\abr{\pie(a\given \tau;w)-\pie(a\given \tau;w')}\nend
    &\le \frac{\abr{\zeta(a, \tau;w)-\zeta(a, \tau;w')}}{\sum_{a'}\zeta(a', \tau;w)} + \frac{\sum_{a'}\abr{\zeta(a', \tau;w')-\zeta(a', \tau;w)}}{\sum_{a'}\zeta(a', \tau;w)}\nend
    &\le \frac{\abr{\zeta(a, \tau;w)-\zeta(a, \tau;w')}}{\max\cbr{\zeta(a, \tau;w), \zeta(a, \tau;w')}} + 2\cdot\frac{\sum_{a'}\abr{\zeta(a', \tau;w')-\zeta(a', \tau;w)}}{\sum_{a'}\zeta(a', \tau;w)+\sum_{a'}\zeta(a', \tau;w')}\nend
    &\le \frac{\abr{\zeta(a, \tau;w)-\zeta(a, \tau;w')}}{\max\cbr{\zeta(a, \tau;w), \zeta(a, \tau;w')}} + 2\cdot\frac{\sum_{a'}\abr{\zeta(a', \tau;w')-\zeta(a', \tau;w)}}{\sum_{a'}\max\cbr{\zeta(a', \tau;w),\zeta(a', \tau;w')}},\label{eq:pie diff 1}
\end{align}
where the second inequality holds by noting that $\sum_{a'}\zeta(a', \tau;w)\ge \sum_{a'}\zeta(a', \tau;w')\ge \zeta(a, \tau;w')$.
For given $w_1, w_2$, recall the definition of $\zeta$ and assume without loss of generality that $\zeta(a,\tau;w_1)\ge \zeta(a,\tau;w_2)$. We have
\begin{align*}
    \frac{\abr{\zeta(a, \tau;w_1)-\zeta(a, \tau;w_2)}}{\max\cbr{\zeta(a, \tau;w_1), \zeta(a, \tau;w_2)}}= 1-\exp\cbr{(w_2-w_1)^\top \phi_0(a,\tau)}
    \le \abr{(w_1-w_2)^\top \phi_0(a, \tau)}.
\end{align*}
Plugging the result into \eqref{eq:pie diff 1}, we conclude that
\begin{align*}
    \abr{\pie(a\given \tau;w)-\pie(a\given \tau;w')}\le 3 \nbr{(w-w')^\top \phi_0(\cdot, \tau)}_{\infty},
\end{align*}
which further yields
\begin{align}
\nbr{\pie(\cdot\given\cdot;w)-\pie(\cdot\given\cdot;w')}_\cD\le 
\nbr{\pie(\cdot\given\cdot;w)-\pie(\cdot\given\cdot;w')}_{\infty}\le 3\nbr{(w-w')^\top\phi_0(\cdot, \cdot)}_\infty.\label{eq:pie diff bound}
\end{align}
Following \eqref{eq:pie diff bound}, the covering number of $\Pie$ with respect to norm $\nbr{\cdot}_\cD$ is upper bounded by
\begin{align*}
    N(t;\Pie, \nbr{\cdot}_\cD)\le N\rbr{\frac t 3;\cF, \nbr{\cdot}_\infty}\le N\rbr{\frac t 3;\cS, \nbr{\cdot}_\infty},
\end{align*}
where $\cF=\{f\given f(a, \tau)=w^\top\phi_0(a,\tau), \nbr{w}_2\le C_0, \nbr{\phi_0(a, \tau)}_{2, \infty}\le 1\}$ and $\cS=\{w^\top \phi_0(a, \tau)\given \allowbreak\nbr{w^\top \phi_0(a, \tau)}_\infty\le C_0\}$.
The second inequality holds by noting that $\cF\subseteq\cS$.
Thus, by Example 5.8 in \cite{wainwright2019high} and noting that the covering number and function class $\cS$ are equipped with the same norm $\nbr{\cdot}_\infty$, we conclude that
\begin{align*}
    \log N(t;\Pie, \nbr{\cdot}_\cD)\le m_0 \log\rbr{1+\frac{6 C_0}{t}}.
\end{align*}
\end{proof}
\end{lemma}

\begin{lemma}[Bounding critical radius with covering number]\label{lem:critical radius & covering number}
If the covering number of $C_0$-uniformly bounded function class $\cF$ satisfies $N(t;\cF, \nbr{\cdot}_\cD)\le a\log\rbr{1+C_1/t}$, the maximal covering number of $\cF$ converges at a rate of $\cO(\sqrt{(a\log T)/T})$, where $C=\max\{C_0, C_1\}$ and $T$ is the size of $\cD$.
\begin{proof}
The critical radius $\eta$ for $\cF$ then satisfies,
\begin{align*}
    \cR_\cD(\eta;\cF)\le \frac{\eta^2}{C_0}.
\end{align*}
We denote an $\eta$-ball in the empirical $L^2(\PP_\cD)$-norm by $\BB_\cD(\eta;\cF)=\{f\in\cF\given \nbr{f}_\cD\le \eta\}$. By Corollary 14.3 in \cite{wainwright2019high}, an upper bound for $\eta$ is given by the following inequality,
\begin{align*}
    \frac{64}{\sqrt{T}}\int_{\frac{\eta^2}{2C_0}}^{\eta}\sqrt{\log N(t;\BB_\cD(\eta;\cQ),\nbr{\cdot}_\cD)}\rd t\le \frac{\eta^2}{C_0}.
\end{align*}
Since $\BB_\cD(\eta;\cF)\subseteq \cF$, the critical radius $\eta$ is further bounded by the following inequality
\begin{align*}
    \frac{64}{\sqrt{T}}\int_{\frac{\eta^2}{2C_0}}^{\eta}\sqrt{\log N(t;\cF,\nbr{\cdot}_\cD)}\rd t\le \frac{\eta^2}{C_0}.
\end{align*}
Noting that $\log N(t;\cF,\nbr{\cdot}_\cD) \le a\log(1+2C_1C_0/\eta^2)$ for $t\in(\eta^2/2C_0, \eta)$, we thus have $\eta$ bounded by
\begin{align}
    \frac{64}{\sqrt{T}}\cdot \eta \cdot \sqrt{a\log\rbr{1+\frac{2C_0C_1}{\eta^2}}}\le \frac{\eta^2}{C_0}.\label{eq:critic radius bound}
\end{align}
For simplicity, we use $C=\max\cbr{C_0, C_1}$ to replace $C_0, c_1$ and conclude that $\eta$ is bounded by
\begin{align*}
    \frac{2048 a}{T} \cdot {\log\rbr{1+\frac{2C^2}{\eta^2}}}\le \frac{\eta^2}{2C^2}
\end{align*}
satisfies \eqref{eq:critic radius bound}.
As $T\rightarrow\infty$, we have $T\ge (\sqrt{2}-1)2048a$. 
Thereby, it is easy to verify that
\begin{align*}
    \eta_0=64C\cdot\sqrt{\frac a T \log\rbr{1+\frac{T}{2048a}}}
\end{align*}
satisfies \eqref{eq:critic radius bound} and we conclude that $\eta\sim \cO(C\sqrt{(a\log T)/T})$.
\end{proof}
\end{lemma}


