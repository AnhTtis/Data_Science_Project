\section{Critical Radius}\label{app:critical radius}
In this section, we study the critical radius for the linear two-step DTRs and the linear one-step POMDP using the techniques given in \S\ref{app:technical}.
\subsection{Critical Radius for Linear Two-step DTRs}\label{app:DTRs critical radius}
In this section, we calculate the critical radius of the function class, 
\begin{align*}
    \cQ_k=\{ \alpha_k(\vh(\vX), \vY_k)\theta_k(\vZ_k): \vh\in\vH, \theta_k\in\Theta_k\}, 
\end{align*}
in the two-step DTRs example.
We first summarize the linear function classes as follows,
\begin{gather*}
    \cH_1=\{w_1\in\RR^{m_1}:\cA\times \cY\rightarrow w_1^\top\phi_1(\cdot), \nbr{w_1}_2\le C_1, \nbr{\phi_1(\cdot)}_2\le 1\}, \\
    \cG=\{w_2\in\RR^{m_2}:\cX\times \cA \rightarrow w_2^\top\phi_2(\cdot), \nbr{w_2}_2\le C_2, \nbr{\phi_2(\cdot)}_2\le 1\}, \\
    \Theta_1=\{\beta_1\in\RR^{d_1}:\cY_1\rightarrow \beta_1^\top \psi_1(\cdot), \nbr{\beta_1}\le D_1, \nbr{\psi_1(\cdot)}_2\le 1\},\\
    \Theta_2=\{\beta_2\in\RR^{d_2}:\cX\times\cA\times \cY_1\rightarrow \beta_2^\top \psi_2(\cdot), \nbr{\beta_2}\le D_2, \nbr{\psi_2(\cdot)}_2\le 1\}, \\
    \cQ_1=\{\cA\times\cY\rightarrow (h_1(a, y)-y_2)\theta_1(y_1): h_1\in\cH_1, \theta_1\in\Theta_1\}, \\
    \cQ_2=\{\cX\times\cA\times\cY\rightarrow (g(x,a)-h_1(a, y)) \theta_2(x, a, y_1): h_1\in\cH_1, g\in\cG, \theta_2\in\Theta_2\}.
\end{gather*}
Note that $h_1(a, y)-y$ can be captured by the following linear function class
\begin{align*}
    \cU_1=\cbr{\tilde w_1=
    \sqrt{2}\begin{bmatrix}
    w_1\\
    t
    \end{bmatrix}
    \in\RR^{m_1+1}:\cA\times\cY\rightarrow  \tilde w_1^\top \begin{bmatrix}
    \frac{\phi_1(a, y)}{\sqrt 2}\\
    \frac{y_2}{\sqrt 2 L_{Y_2}}
    \end{bmatrix}
    , \nbr{\tilde w_1}_2\le \sqrt{2\rbr{C_1^2+L_{Y_2}^2}}},
\end{align*}
and $g(x, a)-h_1(a, y)$ is captured by the following linear function class, 
\begin{align*}
    \cU_2=\cbr{\tilde w_2=
    \sqrt{2}\begin{bmatrix}
    w_1\\
    w_2
    \end{bmatrix}
    \in\RR^{m_1+m_2}:\cA\times\cY\rightarrow  \tilde w_2^\top \begin{bmatrix}
    \frac{\phi_1(a, y)}{\sqrt 2}\\
    \frac{\phi_2(x, a)}{\sqrt 2}
    \end{bmatrix}
    , \nbr{\tilde w_2}_2\le \sqrt{2\rbr{C_1^2+C_2^2}}}.
\end{align*}
By Lemma \ref{lem:critical radius for product}, the maximal critical radius for $\cQ_1$ and $\cQ_2$, which are denoted by $\eta_1$ and $\eta_2$, respectively, are bounded with probability $1-\delta$ by
\begin{gather*}
    \eta_1\le \cO\rbr{
    \sqrt{2\rbr{C_1^2+L_{Y_2}^2}
    \cdot 
    \frac{m_1+d_1+1}{T_1}\log \rbr{\frac{T_1}{m_1+d_1+1}}} + \sqrt{\frac{\log \rbr{1/\delta}}{T_1}}} ,\\
    \eta_2\le \cO\rbr{
    \sqrt{2\rbr{C_1^2+C_2^2}
    \cdot 
    \frac{m_1+m_2+d_2}{T_2}\log \rbr{\frac{T_2}{m_1+m_2+d_2}}} + \sqrt{\frac{\log \rbr{1/\delta}}{T_2}}}.
\end{gather*}

\subsection{Critical Radius for Linear One-step POMDP}\label{app:POMDP critical radius}
In this section, we calculate the critical radius of the function class 
\begin{align*}
    \cQ_k=\{ \alpha^\pie_k(\vh(\vX), \vY_k)\theta_k(\vZ_k): \vh\in\vH, \theta_k\in\Theta_k, \pie\in\Pie\}, 
\end{align*}
in the two-step DTRs example.
We first summarize the linear function classes as follows,
\begin{gather*}
\Pie \subseteq \cbr{\pie \bigg | \pie(a\given o, o^-)=\frac{\exp\rbr{w_0^\top\phi_0(a, o, o^-)}}{\sum_{a'\in\cA}\exp\rbr{w_0^\top\phi_0(a', o, o^-)}}, \nbr{w_0}_2\le C_0,  \nbr{\phi_0(\cdot)}_2\le 1}, \\
    \cH_k=\{ w_k^\top\phi_k(\cdot): w_k\in\RR^{m_k} \nbr{w_k}_2\le C_k, \nbr{\phi_k(\cdot)}_2\le 1\}, \quad k=1, 2, 3, \\
    \cG=\{w_4\in\RR^{m_4}:\cO^-\rightarrow w_4^\top\phi_4(\cdot), \nbr{w_4}\le C_3\nbr{W_6}_F, \nbr{\phi_4(\cdot)}_2\le 1\}, \\
    \Theta_k=\{\beta_k^\top \psi_k(\cdot): \beta_k\in\RR^{d_k},  \nbr{\beta_k}\le D_k, \nbr{\psi_k(\cdot)}_2\le 1\}, \quad k=1, 2, 3, \\
    \Theta_4=\{\beta_4\in\RR^{m_4}:\cO^-\rightarrow \beta_4^\top \phi_4(\cdot), \nbr{\beta_4}\le D_4, \nbr{\phi_4(\cdot)}_2\le 1\}, \\
    \cQ_k=\{ \alpha^\pie_k(\vh, y_k)\theta_k(z_k): \vh\in\vH, \theta_k\in\Theta_k, \pie\in\Pie\}, \quad k=1, 2, 3, 4,\\
    \sY=\cbr{f:\cY\rightarrow\RR\given f(y)=\lambda\cdot\frac{y}{L_Y}, \abr{\lambda}\le L_Y}.
\end{gather*}
Here $\cY$ can be viewed as an one-dimensional linear function class that $y$ falls into.
Notice that $\alpha^\pie_1=h_1$ is captured by $\vH_1$. Therefore, the critical radius of the product function class $\cQ_1$ is bounded by 
\begin{align*}
    \eta_1\le \cO\rbr{D_1C_1\sqrt{\frac{m_1+d_1}{T_1}\log\rbr{\frac{T_1}{m_1+d_1}}}}, 
\end{align*}
following Lemma \ref{lem:critical radius for product}.
For $\alpha^\pie_2=h_2-h_1-Y\pie$, we have
\begin{align}
    N(t;\cQ_2, \nbr{\cdot}_\cD)
    &\le N\rbr{\frac{t}{3};\cbr{h_2\theta_2},\nbr{\cdot}_\cD}\cdot N\rbr{\frac{t}{3};\cbr{h_1\theta_2},\nbr{\cdot}_\cD}\cdot  N\rbr{\frac{t}{3};\cbr{Y\pie\theta_2},\nbr{\cdot}_\cD}\nend
    &\le N\rbr{\frac{t}{6D_2};\cH_2,\nbr{\cdot}_\cD} N\rbr{\frac{t}{6C_2};\Theta_2,\nbr{\cdot}_\cD}\nend
    &\qquad \cdot N\rbr{\frac{t}{6D_2};\cH_1, \nbr{\cdot}_\cD}
    N\rbr{\frac{t}{6C_1};\Theta_2,\nbr{\cdot}_\cD}\nend
    &\qquad \cdot N\rbr{\frac{t}{9D_2};\sY,\nbr{\cdot}_\cD}
    N\rbr{\frac{t}{9L_Y D_2};\Pie, \nbr{\cdot}_\cD}
    N\rbr{\frac{t}{9L_Y};\Theta_2, \nbr{\cdot}_\cD},\label{eq:POMDP critic 1}
\end{align}
where the first inequality holds by Lemma \ref{lem:covering number summation} on the covering number of summation function class and the second inequality holds by Lemma \ref{lem:covering number product} on the covering number of product function class. Following  Lemma \ref{lem:covering number linear} on the covering number for the linear function class and Lemma \ref{lem:covering number Pie} on the covering number of the policy class, \eqref{eq:POMDP critic 1} is further bounded by
\begin{align*}
    \log N(t;\cQ_2,\nbr{\cdot}_\cD)
    \le a \log\rbr{1+\frac{C}{t}},
\end{align*}
where $a=7\max\{m_0, m_1, m_2, d_1, d_2\}$ and $C=\max\{12 C_2 D_2, 12 D_2 C_1, 12 C_1 D_2, 18 D_2 L_Y, 54 L_Y D_2, 18 L_Y D_2\}$.
By Lemma \ref{lem:critical radius & covering number}, the critical radius $\eta_2$ is bounded by
\begin{align*}
    \eta_2\le \cO\rbr{C\sqrt{\frac{a}{T_2}\cdot\log \frac{T_2}{a}}}.
\end{align*}
For $\alpha^\pie_3=h_3(y, a, o^-)-\sum_{a'\in\cA}h_2(a', o, o^-)$, it is captured by
\begin{align*}
    \cU_3=\cbr{\tilde w_3^\top 
    \begin{bmatrix}
    \frac{\sum_{a'\in\cA} \phi_2(a', o, o^-)}{\sqrt{2}|\cA|}\\
    \frac{\phi_3(y, a, o^-)}{\sqrt 2}
    \end{bmatrix}:
    \tilde w_3\in\RR^{m_2+m_3}, \nbr{\tilde w_3}_2\le \sqrt{2\rbr{|\cA|^2C_2^2+C_3^2}}
    }, 
\end{align*}
and $\alpha^\pie_4$ is captured by the following linear function class
\begin{align*}
    \cU_4=\cbr{
    \tilde w_4^\top \begin{bmatrix}
    \frac{\phi_3(a, y)}{\sqrt 2}\\
    \frac{\phi_4(x, a)}{\sqrt 2}
    \end{bmatrix}:
    \tilde w_4
    \in\RR^{m_3+m_4}, 
    \nbr{\tilde w_4}_2\le \sqrt{2\rbr{C_3^2+C_4^2}}}.
\end{align*}
By Lemma \ref{lem:critical radius for product} and Corollary 5 of \cite{dikkala2020minimax}, the maximum critical radius for $\cQ_k$ denoted by $\eta_k$ is bounded with probability $1-\delta$ by
\begin{gather*}
    \eta_1\le \cO\rbr{ D_1 C_1
    \sqrt{
    \frac{m_1+d_1}{T_1}\log \rbr{\frac{T_1}{m_1+d_1}}} + \sqrt{\frac{\log \rbr{1/\delta}}{T_1}}},\\
    \eta_2\le \cO\rbr{ C\sqrt{\frac{\max\{m_0, m_1, m_2, d_1, d_2\}}{T_2}\log\rbr{\frac{T_2}{\max\{m_0, m_1, m_2, d_1, d_2\}}}}+\sqrt{\frac{\log(1/\delta)}{T_2}}}, \\
    \eta_3\le \cO\rbr{ D_3
    \sqrt{ 2 \rbr{|\cA|^2C_2^2+C_3^2}
    \cdot 
    \frac{m_2+m_3+d_3}{T_3}\log \rbr{\frac{T_3}{m_2+m_3+d_3}}} + \sqrt{\frac{\log \rbr{1/\delta}}{T_3}}},\\
    \eta_4\le \cO\rbr{ D_4 
    \sqrt{ 2\rbr{C_3^2+C_4^2}
    \cdot 
    \frac{m_3+m_4}{T_4}\log \rbr{\frac{T_4}{m_3+m_4}}} + \sqrt{\frac{\log \rbr{1/\delta}}{T_4}}}.
\end{gather*}