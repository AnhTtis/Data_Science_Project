\subsection{Proof of Theorem \ref{thm:IV identification}}\label{pro:IV identification}
Under the assumption that there exists $h_1$ and $g$ satisfying \eqref{eq:IV bridge 1} and \eqref{eq:IV bridge 2}, we prove the conclusion of Theorem \ref{thm:IV identification} that $g$ is a recovery of the CATE of the CCB-IV almost surely in this subsection.
\begin{proof}
We first prove the following two equality, $\EEob\sbr{Y\given Z, R_Z=1}=\EEob\sbr{g(A, X)\given Z, R_Z=1}$ and $\EEob\sbr{Y\given Z, R_Z=1}= \EEob[\tilde f(A, X)\given Z, R_Z=1]$, where $\tilde f(a, x)$ is by our construction and is shown to be the exact CATE. Then, with the model assumption for CCB-IV, we can prove that $g$ recovers the CATE almost surely.
We start with $\EEob\sbr{Y\given Z, R_Z=1}$ and it holds that 
\begin{align}
    \EEob\sbr{Y\given Z, R_Z=1} &= \EEob\sbr{\EEob\sbr{Y\given X, Z}\given Z, R_Z=1} \nend
    &=\EEob\sbr{f(A, X) + \EEob\sbr{\EEob\sbr{\epsilon\given X, U, Z}\given X, Z} \given Z, R_Z=1}\nend
    &=\EEob\sbr{f(A, X) + \EEob\sbr{\EEob\sbr{\epsilon\given X, U}\given X} \given Z, R_Z=1},\label{eq:1 iv id pro}
\end{align}
where the first equality holds by noting that $R_Z\indep Y\given (X, Z)$, the second equality holds by noting that $Y=f(A, X)+\epsilon$, and the third equality holds by noting that $\epsilon\indep Z\given (X, U)$ and $Z\indep U\given X$.
Let $\tilde f(a, x)=f(a, x) + \EEob\sbr{\epsilon\given X=x}$. We continue with \eqref{eq:1 iv id pro} and obtain
\begin{align}
    \EEob\sbr{Y\given Z, R_Z=1}
    &=\EEob\sbr{f(A, X) + \EEob\sbr{\epsilon\given X} \given Z, R_Z=1}\nend
    &=\EEob\sbr{\tilde f(A, X)\given Z, R_Z=1}.\label{eq:2 iv id pro}
\end{align}
On the other hand, it also holds for $\EEob\sbr{Y\given Z, R_Z=1}$ that
\begin{align}
    \EEob\sbr{Y\given Z, R_Z=1}
    & = \EEob\sbr{h_1(Y, A, Z)\given Z, R_Z=1}\nend
    &= \EEob\sbr{\EEob\sbr{h_1(Y, A, Z)\given A, X, Z, (R_Z, R_X)=\ind}\given Z, R_Z=1},\label{eq:3 iv id pro}
\end{align}
where the first equality holds by \eqref{eq:IV bridge 1} and the second equality holds by noting that $R_X\indep (Y, R_Z)\given (A, X, Z)$.
Plugging \eqref{eq:IV bridge 2} into \eqref{eq:3 iv id pro}, it follows that
\begin{align}
    \EEob\sbr{Y\given Z, R_Z=1}
    &=\EEob\sbr{\EEob\sbr{g(A, X)\given A, X, Z, (R_Z, R_X)=\ind}\given Z, R_Z=1}\nend
    &=\EEob\sbr{g(A, X)\given Z, R_Z=1}. \label{eq:4 iv id pro}
\end{align}
Combining \eqref{eq:2 iv id pro} and \eqref{eq:4 iv id pro}, we arrive at
\begin{align*}
    \EEob\sbr{g(A, X)-\tilde f(A, X)\given Z, R_Z=1}=0. 
\end{align*}
By the IV completeness assumption ((i) in Assumption \ref{asp:CCB-IV}), it holds that
\begin{align}
    \tilde f(A, X) \overset{\text{a.s.}}{=} g(A, X).\label{eq:5 iv id pro}
\end{align}
Lastly, it remains to characterize the relationship between $\tilde f$ and the exact CATE. 
In the CCB-IV, the exact CATE $\gstar$ is given by
\begin{align*}
    \gstar(x, a) &= \EEin\sbr{Y\given X=x, \doopt(a)} \nend
    & =\EEob\sbr{\EEob\sbr{Y\given U, X, A=a}\given X=x},
\end{align*}
where the first equality holds by the definition of CATE, the second equality holds by noting that $Y\indep Z\given (A, U, X)$ and the definition of do-calculus. Recalling that $Y=f(A, X) + \epsilon$, it follows that
\begin{align*}
    \gstar(x, a)&= \EEob\sbr{\EEob\sbr{f(A, X) + \epsilon\given U, X, A=a}\given X=x}\nend
    & = \EEob\sbr{f(a, X) +\EEob\sbr{ \epsilon\given U, X}\given X=x} \nend
    &=f(a, x) + \EEob\sbr{\epsilon\given X=x},
\end{align*}
where the second equality holds by noting that $\epsilon\indep A\given (X, U)$. Recall the definition that $\tilde f(a, x)=f(a, x)+\EEob\sbr{\epsilon\given X=x}$, by \eqref{eq:5 iv id pro} we finally obtain
\begin{align}
    \gstar(x, a)= \tilde f(a, x)\overset{\text{a.s.}}{=} g(A, X),\label{eq:7 iv id pro}
\end{align}
which finishes the proof of Theorem \ref{thm:IV identification} that if $h_1$ and $g$ exists, $g$ recovers the CATE almost surely.
\end{proof}

\subsection{Proof of Remark \ref{rmk:IV existence}}\label{pro:rmk IV existence}
Our proof is given in two folds: (i) we give a proof of the “if” part in Remark \ref{rmk:IV existence}; (ii) we give a proof of the “only if“ part in Remark \ref{rmk:IV existence}.
\paragraph{“If" part.}We first prove the “if” part that it suffices for $h_1$ and $g$ to exist if there exists a solution $h_1$ to the following equation
\begin{align}
    \EEob\sbr{\gstar(A, X)-h_1(Y, A, Z)\given A, X, Z, R_Z=1}=0, \label{eq:6 iv id pro}
\end{align}
where $\gstar$ is the exact CATE. 
\begin{proof}
Let $\tilde g(a, x)=\gstar(a, x)$ and $\tilde h_1$ be the solution to \eqref{eq:6 iv id pro}. We just need to check that $\tilde g$ and $\tilde h_1$ satisfies \eqref{eq:IV bridge 1} and \eqref{eq:IV bridge 2}. Note that \eqref{eq:IV bridge 2} holds directly by \eqref{eq:6 iv id pro}. For \eqref{eq:IV bridge 1}, we see that
\begin{align}
    &\EEob\sbr{\tilde h_1(Y, A, Z)\given Z, R_Z=1}\nend
    &= \EEob\sbr{\EEob\sbr{\tilde h_1(Y, A, Z)\given A, X, Z, (R_Z, R_X)=\ind}\given Z, R_Z=1}\nend
    &=\EEob\sbr{\EEob\sbr{\gstar(A, X)\given A, X, Z, (R_Z, R_X)=\ind}\given Z, R_Z=1}\nend
    &=\EEob\sbr{\gstar(A, X)\given Z, R_Z=1},\label{eq:8 iv id pro}
\end{align}
where the first equality holds by noting that $R_X\indep (Y, R_Z)\given (A, X, Z)$ and the second equality holds by \eqref{eq:6 iv id pro}. Note that in the first part, we have already proved that $\EEob\sbr{Y\given Z, R_Z=1}= \EEob[\tilde f(A, X)\given Z, R_Z=1]$ with $\tilde f$ defined as $\tilde f(x, a)=f(x, a)+\EEob\sbr{\epsilon\given X=x}$  (see \eqref{eq:2 iv id pro}) and that $\gstar(x, a)=\tilde f(a, x)$ in \eqref{eq:7 iv id pro}. We remark that these two properties hold without any assumption on the existence of the bridge functions. Therefore, it holds for \eqref{eq:8 iv id pro} that
\begin{align*}
    \EEob\sbr{\tilde h_1(Y, A, Z)-Y\given Z, R_Z=1}=0,
\end{align*}
which justifies that $\tilde h$ and $\tilde g$ satisfy \eqref{eq:IV bridge 1} and \eqref{eq:IV bridge 2} and serve as a solution. 
\end{proof}

\paragraph{“Only if" part.}
We prove that any solution to the IES in Theorem \ref{thm:IV identification} must satisfy \eqref{cond:iv bridge exist}.
\begin{proof}
The proof is direct, following the fact that $g\aseq\gstar$ if $\vh=(h_1, g)$ satisfies the IES. By \eqref{eq:IV bridge 2} we have
\begin{align*}
    \EEob\sbr{\gstar(X, A) - h_1(Y, A, Z)\given A=a, X=x, Z=z, (R_Z, R_X)=\ind}=0.
\end{align*}
Noting that $R_X\indep (R_Z, Y)\given A, X, Z$, we thus have
\begin{align*}
    \EEob\sbr{\gstar(X, A) - h_1(Y, A, Z)\given A=a, X=x, Z=z, R_Z=1}=0.
\end{align*}
Thus, we complete the proof of Remark \ref{rmk:IV existence}.
\end{proof}



\subsection{Proof of Theorem \ref{thm:PV ID}}\label{pro:PV ID}
Under the assumption that there exists $h_1$, $h_2$, $h_3$ and $g$ satisfying \eqref{eq:PV ID 1}-\eqref{eq:PV ID 4}, we prove that $g$ is a recovery of the CATE of the CCB-PV in this subsection. 
\begin{proof}
We remark that $h_2$ corresponds to the value bridge function for identifying a PV model and $h_1$ and $h_3$ are additional bridge functions to deal with the missingness issue. In the following part, we first show that $\EEob\sbr{h_2(A, W, X)-Y\given U, A, X}\aseq 0$ and then prove that $g(x, a)=\gstar(x,a)$.
We start with the conditional expectation $\EEob\sbr{Y\given A, X, Z, (R_X, R_Z)=\ind}$.
\begin{align}
    &\EEob\sbr{Y\given A, X, Z, (R_X, R_Z)=\ind} \nend
    & = \EEob\sbr{h_1(Y, A, X, Z)\given A, X, Z, (R_X, R_Z)=\ind} \nend
    & = \EEob\sbr{\EEob\sbr{h_1(Y, A, X, Z)\given A, W, X, Z, (R_X, R_Z, R_W)=\ind}\given A, X, Z, (R_X, R_Z)=\ind}\nend
    &= \EEob\sbr{h_2(A, W, X)\given A, X, Z, (R_X, R_Z)=\ind}, \label{eq:1 pv id pro}
\end{align}
where the first equality holds by \eqref{eq:PV ID 1}, the second equality holds by noting that $R_W\indep (Z, Y, R_Z, R_X)\given (W, X, A)$, and the last equality holds by \eqref{eq:PV ID 2}.
We can rewrite \eqref{eq:1 pv id pro} by additionally conditioning on the confounder $U$,
\begin{align*}
    0 &= \EEob\sbr{Y-h_2(A, W, X)\given A, X, Z,(R_X, R_Z)=\ind}\nend
    &= \EEob\sbr{\EEob\sbr{Y-h_2(A, W, X)\given U, A, X, Z,  (R_X, R_Z)=\ind}\given A, X, Z, (R_X, R_Z)=\ind} \nend
    & = \EEob\sbr{\EEob\sbr{Y-h_2(A, W, X)\given U, A, X}\given A, X, Z,  R_Z=\ind},
\end{align*}
where the last equality holds by noting that $(R_X, R_Z, Z)\indep (Y, W, R_X)\given (U, A, X)$ and that $R_X\indep U\given (A, X, Z)$.
By (i) of Assumption \ref{asp:CCB-PV} (i) on the PV completeness, we have that
\begin{align}
    \EEob\sbr{Y-h_2(A, W, X)\given U, A, X}\aseq 0. \label{eq:3 pv id pro}
\end{align}
Now, it remains to show $g=\gstar$.
For $g$, we have
\begin{align}
    g(X, a')
    & = \EEob\sbr{h_3(Y, A, X;a')\given X, R_X=1}\nend 
    &= \EEob\sbr{\EEob\sbr{h_3(Y, A, X; a')\given A, W, X, (R_W, R_X)=\ind}\given X, R_X=1}\nend
    & = \EEob\sbr{\EEob\sbr{h_2(a', W, X)\given A, X, W, (R_W, R_X)=\ind}\given X, R_X=1}, \label{eq:2 pv id pro}
\end{align}
where the first equality holds by \eqref{eq:PV ID 4}, the second equality holds by noting that $R_W\indep (R_X, Y)\given (A, W, X)$, and the third equality holds by \eqref{eq:PV ID 3}. We continue with \eqref{eq:2 pv id pro}, 
\begin{align}
    g(X, a')
    & = \EEob\sbr{h_2(a', W, X)\given X}\nend
    & = \EEob\sbr{\EEob\sbr{h_2(A, W, X)\given U, A=a', X}\given X} \nend
    &\aseq\EEob\sbr{\EEob\sbr{Y\given U, A=a', X}\given X},\label{eq:4 pv id pro}
\end{align}
where the first equality holds by noting that $R_X\indep W\given X$, the second equality holds by noting that $W\indep A\given (U, X)$, and the last equality holds by \eqref{eq:3 pv id pro}.
By definition of the CATE, we have
\begin{align}
    \gstar(x, a)
    &=\EEob\sbr{Y\given X=x, \doopt(a)}\nend
    &=\EEob\sbr{\EEob\sbr{Y\given X, U, A=a}\given X=x}.\label{eq:5 pv id pro}
\end{align}
Combining \eqref{eq:4 pv id pro} and \eqref{eq:5 pv id pro}, we conclude with 
\begin{align*}
    g(x, a)\aseq \gstar(x, a),
\end{align*}
which finishes the proof of Theorem \ref{thm:PV ID} that if $h_1, h_2, h_3$ and $g$ exist, $g$ recovers the CATE almost surely. 
\end{proof}

\subsection{Proof of Remark \ref{rmk:PV existence}}\label{pro:rmk PV existence}
We give a proof of the “if” and “only if” part in Remark \ref{rmk:PV existence} in this subsection. 
\paragraph{“If" part.} In this part we prove that it suffices for $h_1, h_2, h_3$ and $g$ to exist if the following conditions hold,
\begin{itemize}
    \item[(i)] There exists a solution $h_2$ to $\EEob\sbr{h_2(A, W, X)-Y\given A=a, X=x, U=u}=0$;
    \item[(ii)] For any solution $h_2$ in (i), there exists a solution $h_1$ to \eqref{eq:PV ID 2}. 
    \item[(iii)] For any solution $h_2$ in (i), there exists a solution $h_3$ to \eqref{eq:PV ID 3}.
\end{itemize}
\begin{proof}
Let $\tilde h_2$ be a solution to $\EEob\sbr{h_2(A, W, X)-Y\given A=a, X=x, U=u}$ following condition (i).
Let $\tilde h_1$ be a solution to \eqref{eq:PV ID 2} with $h_2$ substituted by $\tilde h_2$ by condition (ii) and let $\tilde h_3$ be a solution to \eqref{eq:PV ID 3} with $h_2$ substituted by $\tilde h_2$ by condition (iii).
Moreover, we let $\tilde g=\gstar$.
Therefore, we just need to verify that $\eqref{eq:PV ID 1}$ and \eqref{eq:PV ID 4} holds for $\tilde h_1$, $\tilde h_3$, and $\tilde g$.
For \eqref{eq:PV ID 1}, it holds that
\begin{align*}
   & \EEob\sbr{\tilde h_1(Y, A, X, Z)\given A, X, Z, (R_X, R_Z)} \nend
   &= \EEob\sbr{\EEob\sbr{\tilde h_1(Y, A, X, Z)\given A, W, X, Z, (R_W, R_X, R_Z)=\ind}\given A, X, Z, (R_X, R_Z)}\nend
   &=\EEob\sbr{\tilde h_2(A, W, X)\given A, X, Z, (R_X, R_Z)=\ind}\nend
   &=\EEob\sbr{\EEob\sbr{\tilde h_2(A,W, X)\given A, X, U}\given A, X, Z, (R_X, R_Z)=\ind},
\end{align*}
where the first equality holds by noting that $R_W\indep (R_X, R_Z, Y, Z)\given (A, W, X)$, the second equality holds by noting that $\tilde h_2$ and $\tilde h_1$ satisfy \eqref{eq:PV ID 2}, and the last equality holds by noting that $W\indep (Z, R_X, R_Z)\given (A, X, U)$. Following condition (i), we thus have
\begin{align*}
    & \EEob\sbr{\tilde h_1(Y, A, X, Z)\given A, X, Z, (R_X, R_Z)}\nend
    &=\EEob\sbr{\EEob\sbr{Y\given A, X, U}\given A, X, Z, (R_X, R_Z)=\ind}\nend
    &=\EEob\sbr{\EEob\sbr{Y\given A, X, U, Z, (R_X, R_Z)=\ind}\given A, X, Z, (R_X, R_Z)=\ind}\nend
    &=\EEob\sbr{Y\given A, X, Z, (R_X, R_Z)=\ind},
\end{align*}
where the second equality holds by noting that $Y\indep (Z, R_X, R_Z)\given (A, X, U)$. Therefore, we verify that $\tilde h_1$ satisfies \eqref{eq:PV ID 1}. It remains to check for \eqref{eq:PV ID 4}. We have for $\tilde h_3$ that
\begin{align}
    &\EEob\sbr{\tilde h_3(Y, A, X, a')\given X, R_X=1}\nend
    &=\EEob\sbr{\EEob\sbr{\tilde h_3(Y, A, X, a')\given A, W, X, (R_W, R_X)=\ind}\given X, R_X=1}\nend
    &=\EEob\sbr{\EEob\sbr{\tilde h_2(a', W, X)\given A, W, X, (R_W, R_X)=\ind}\given X, R_X=1}\nend
    &=\EEob\sbr{\tilde h_2(a', W, X)\given X},\label{eq:6 pv id pro}
\end{align}
where the first equality holds by noting that $R_W\indep (R_X, Y)\given (A, W, X)$, the second equality holds by noting that $\tilde h_2$ and $\tilde h_3$ satisfy \eqref{eq:PV ID 3}, and the last equality holds by noting that $R_X\indep W\given X$. Continuing with \eqref{eq:6 pv id pro}, we have
\begin{align}
    &\EEob\sbr{\tilde h_3(Y, A, X, a')\given X}\nend
    &=\EEob\sbr{\EEob\sbr{\tilde h_2(A, W, X)\given A=a', X, U}\given X}\nend
    &=\EEob\sbr{\EEob\sbr{Y\given A=a', X, U}\given X}\nend
    &=\EEob\sbr{Y\given X, \doopt(a')}, \label{eq:7 pv id pro}
\end{align}
where the first equality holds by noting that $A\indep W\given (X, U)$, the second equality holds by condition (i), and the last equality holds by the definition of do-calculus. Note that the right-hand side of \eqref{eq:7 pv id pro} corresponds to the definition of CATE $\gstar$. Therefore, we verify that $\tilde h_3$ and $\tilde g$ satisfy \eqref{eq:PV ID 4}. The proof in this part suggests that following conditions (i)-(iii), $\tilde h_1, \tilde h_2, \tilde h_3$ and $\tilde g$ are solution to \eqref{eq:PV ID 1}-\eqref{eq:PV ID 4}, i.e., conditions (i)-(iii) are sufficient for a solution to exist. 
\end{proof}

\paragraph{“Only if" part.}
We give a proof that any solution to the IES in Theorem \ref{thm:PV ID} must satisfy the conditions in Remark \ref{rmk:PV existence}.
\begin{proof}
The “only if” part is direct if we plug \eqref{eq:PV ID 2} into \eqref{eq:PV ID 1} and obtain,
\begin{align*}
    &\EEob\sbr{Y\given A, X, Z, (R_X, R_Z)=\ind}\nend
    &=\EEob\sbr{\EEob\sbr{h_1(Y, A, X, Z)\given A, W, X, Z, (R_X, R_Z)=\ind}\given A, X, Z, (R_X, R_Z)=\ind}\nend
    &=\EEob\sbr{\EEob\sbr{h_1(Y, A, X, Z)\given A, W, X, Z, (R_X, R_Z, R_W)=\ind}\given A, X, Z, (R_X, R_Z)=\ind}\nend
    &=\EEob\sbr{h_2(A, W, X)\given A, X, Z, (R_X, R_Z)=\ind},
\end{align*}
where the second inequality holds by noting that $R_W\indep (Y, Z, R_X, R_Z)\given (A, W, X)$.
Moreover, by noting that $(W, Y)\indep (Z, R_X, R_Z)\given (U,A, X)$, it follows that
\begin{align*}
    &\EEob\sbr{Y-h_2(A, W, X)\given A, X, Z, (R_X, R_Z)=\ind}\nend
    &=\EEob\sbr{\EEob\sbr{Y-h_2(A, W, X)\given U, A, X}\given A, X, Z, (R_X, R_Z)=\ind}\nend
    &=\EEob\sbr{\EEob\sbr{Y-h_2(A, W, X)\given U, A, X}\given A, X, Z, R_Z=1}, 
\end{align*}
where the last inequality holds by noting that $R_X$ is only caused by $X$.
Following the PV completeness condition, we thereby have,
\begin{align*}
    \EEob\sbr{Y-h_2(A, W, X)\given U, A, X}\aseq 0, 
\end{align*}
which corresponds to the first condition. The remaining two conditions hold directly by \eqref{eq:PV ID 2} and \eqref{eq:PV ID 3}. Hence, we complete the proof of Remark \ref{rmk:PV existence}.
\end{proof}

\subsection{Proof of Theorem \ref{thm:CCB-PV ID extension}}\label{pro:CCB-PV ID extension}
Under the assumption that the bridge functions $h_1, h_2, h_3$ and $g$ exist, we prove that $g$ is a recovery of the average reward $v^\pie$.
\begin{proof}
Our proof is separated  into two steps.  (i) First,  we prove that $h_2$ satisfies $$\EEob\sbr{h_2(A, W, X)-Y\pie(A\given X, W)\given U, A, X}\aseq 0. $$ (ii) Then in the second step,  we prove that $g\aseq v^\pie. $
From \eqref{eq:PV ID 2} we have
\begin{align}
    &\EEob\sbr{h_1(Y, A, X, Z)\given A, X, Z, (R_X, R_Z)=\ind}\nend
    &=\EEob\sbr{\EEob\sbr{h_1(Y, A, X, Z)\given A, W, X, Z, (R_X, R_Z)=\ind}\given A, X,Z, (R_X, R_Z)=\ind}\nend
    &=\EEob\sbr{\EEob\sbr{h_1(Y, A, X, Z)\given A, W, X, Z, (R_X, R_Z, R_W)=\ind}\given A, X,Z, (R_X, R_Z)=\ind}\nend
    &=\EEob\sbr{h_2(A, W, X)-Y\pie(A\given X, W)\given A, X, Z, (R_X, R_Z)=\ind},\label{eq:PV extend 1}
\end{align}
where the second equality holds by noting that $R_W\indep (Y, Z, R_X, R_Z)\given (A, W, X)$.
By noting that $(R_X, R_Z, Z)\indep (W, Y)\given (A, X, U)$, we further have
\begin{align}
    &\EEob\sbr{h_2(A, W, X)-Y\pie(A\given X, W)\given A, X, Z, (R_X, R_Z)=\ind}\nend
    &=\EEob\sbr{\EEob\sbr{h_2(A, W, X)-Y\pie(A\given X, W)\given A, X, U}\given A, X, Z, (R_X, R_Z)=\ind}. \label{eq:PV extend 2}
\end{align}
Following \eqref{eq:PV ID 1} and combining \eqref{eq:PV extend 1} and \eqref{eq:PV extend 2}, it follows that
\begin{align}
    \EEob\sbr{\EEob\sbr{h_2(A, W, X)-Y\pie(A\given X, W)\given A, X, U}\given A, X, Z, R_Z=1} = 0,\label{eq:PV extend 3}
\end{align}
where the equality holds by noting that $R_X$ is only caused by $X$.
By the PV completeness assumption, \eqref{eq:PV extend 3} implies that
\begin{align}
    \EEob\sbr{h_2(A, W, X)-Y\pie(A\given X, W)\given A, X, U}\aseq 0.\label{eq:PV extend 4}
\end{align}
Here we finish the first step. 

In the following, we prove $g\aseq v^\pie. $ From \eqref{eq:PV ID 4}, we have
\begin{align}
    g(X)&=\EEob\sbr{h_3(Y, A, X)\given X, R_X=1}\nend
    &=\EEob\sbr{\EEob\sbr{h_3(Y, A, X)\given A, W, X, (R_X, R_W)=1}\given X, R_X=1}\nend
    &=\EEob\sbr{\EEob\sbr{\sum_{a'\in\cA}h_2(a', W, X)\given A, W, X, (R_X, R_W)=\ind}\given X, R_X=1},\label{eq:PV extend 5}
\end{align}
where the second equality holds by noting that $R_W\indep (R_X, Y)\given (A, W, X)$ and the last equality holds by \eqref{eq:PV ID 3}.
We can rewrite \eqref{eq:PV extend 5} as
\begin{align}
    g(X)&=\EEob\sbr{\sum_{a'\in\cA}h_2(a', W, X)\given X, R_X=1}\nend
    &=\EEob\sbr{\sum_{a'\in\cA}\EEob\sbr{h_2(a', W, X)\given A, X, U}\given X, R_X=1}\nend
    &=\EEob\sbr{\sum_{a'\in\cA}\EEob\sbr{h_2(A, W, X)\given A=a', X, U}\given X, R_X=1}, \label{eq:PV extend 6}
\end{align}
where the second equality holds by $R_X\indep W\given (A, X, U)$ and the last equality holds by noting that $W\indep A \given (X, U)$.
Plugging \eqref{eq:PV extend 4} into \eqref{eq:PV extend 6}, it follows that
\begin{align*}
    g(x)&\aseq\EEob\sbr{\sum_{a'\in\cA} \EEob\sbr{Y\pie(A\given X, W)\given A=a', X, U}\given X=x, R_X=1}\nend
    &=\EEob\sbr{\sum_{a'\in\cA} \EEob\sbr{Y\given A=a', X, U, W}\pie(a'\given X, W)\given X=x, R_X=1}\nend
    &=v^\pie(x),
\end{align*}
where the second equality holds by noting that $A\indep W\given (X, U)$ and that $R_X$ is only caused by $X$.
The last equality holds by \eqref{def:extended v}, which completes the proof of Theorem \ref{thm:CCB-PV ID extension}.
\end{proof}