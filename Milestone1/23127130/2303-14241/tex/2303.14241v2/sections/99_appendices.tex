%\vspace{-2mm}
\section*{Appendix}
%
\begin{figure}
    \centering \includegraphics[width=0.7\linewidth]{figs/running_example.png}
    \vspace{-3mm}
    \caption{\small A running example to compare between the graph-$k$-core and {\sf AlphaCore} decomposition methods. The Coreness of nodes according to graph-$k$-core decomposition is shown with different node colors, whereas {\sf AlphaCore} is run with in-strength and out-strength as node features with a step size of 0.25. Different {\sf AlphaCore}s are shown using dotted boundaries. \label{fig:running_example}}
    \vspace{-5mm}
\end{figure}
%
%\vspace{-2mm}
\spara{An Example of AlphaCore.}
To better illustrate the differences between the traditional graph-$k$-core and {\sf AlphaCore} decomposition methods, we showcase an example in Figure \ref{fig:running_example}.  %We are interested in the inner core that captures only the most important traders of a financial network.  
In the case of graph-$k$-core, the innermost core is the 3-core, whereas the {\sf InnerCore} of {\sf AlphaCore} would be the core of $\alpha$ $>$ 0.75.  Note that the 3-core consists of traders that trade frequently with themselves, but their trade volumes with themselves are not that significant compared to other transactions which exist in the network.  In certain analyses of financial networks such as anomalous address detection, being able to filter out these negligible transactions and their participating traders, while still capturing more meaningful ones, significantly improves the accuracy and scalability of subsequent computations on the decomposed network core.  On the other hand, the {\sf AlphaCore} of $\alpha$ $>$ 0.75 is able to capture both the traders that participate in the largest transactions which occur in the example network, while filtering the negligible transactions and their participating traders.  
We point out that the main limitation with graph-$k$-core is that it only considers node degrees, whereas {\sf AlphaCore} is flexible and can consider any combination of node features as outlined in Table 1, without requiring to specify any feature weighting parameters to perform effectively on a particular task.  
Therefore, in networks where edge weights fall under a broad range and they are meaningful distinguishing factors, we recommend {\sf AlphaCore} over the traditional graph-$k$-core decomposition.


\SetKwInput{KwInput}{Input}
\SetKwInput{KwOutput}{Output}
\SetKwRepeat{Do}{do}{while}
%
\begin{algorithm}[tb!]
\footnotesize
\KwInput{Directed, weighted, multigraph $G(V,E,w)$,\\
Set of node property functions $p_1, ..., p_n \in P$,\\%,
Data depth threshold $\epsilon$}
\KwOutput{InnerCore $V^{inner}$}
\tcp{Compute feature matrix}
$F = [f_1, ..., f_n] = \forall p_i \in P: f_i = p_i(v, G), \forall v \in V$\label{alg:line1}\;% \tcp*{initial feature matrix}
$\Sigma_F^{-1}$ = cov$(F)^{-1}$\tcp*{compute only once}\label{alg:line2}
\tcp{Compute initial depth values}
$z = [z_1, ..., z_n] = \forall v_i \in V: z_i = [1+(F_{i,*})'\Sigma^{-1}_F(F_{i,*})]^{-1}$\label{alg:line3}\;
  \Do{$\exists z_i: (z_i \geq \epsilon) \wedge (v_i \in V)$}{
    \ForEach{$z_i \geq \epsilon$}{
    $\V = \V \setminus \{v_i\}$\label{alg:line14}\;
    }
    \tcp{recompute node properties}
    $F = \forall p_i \in P: p_i(v, G), \forall v \in V$\label{alg:line16}\;
    \tcp{recompute depth}
    $z_i = [1+(F_{i,*})'\Sigma^{-1}_F(F_{i,*})]^{-1}, \forall v_i \in V$\label{alg:line17}\;
  }
\KwRet{$\V$ \tcp{as InnerCore $V^{inner}$}}\label{alg:line21}
\caption{\small{\sf InnerCore} Discovery}
\label{alg:innercore}
\end{algorithm}

\spara{Algorithm~\ref{alg:innercore} for {\sf InnerCore} Discovery.}
Algorithm~\ref{alg:innercore} computes a feature matrix based on each node property function in line~\ref{alg:line1}. For instance, this could include a node's neighborhood size as listed in Table~\ref{tab:node_property_functions}. The feature matrix $F$ is used to compute the inverse covariance matrix $\Sigma_F$ in line~\ref{alg:line2}, which will be utilized for future data depth calculations. The initial depth of each node is determined using the Mahalanobis depth with respect to the origin at line~\ref{alg:line3}. Nodes with depth greater than or equal to input $\epsilon$ are removed from the node set $\V$ at line~\ref{alg:line14}. Once one batch of node removals has been performed, the feature matrix and depth values are re-evaluated in lines~\ref{alg:line16}--\ref{alg:line17}. If any remaining nodes still have a depth greater than or equal to $\epsilon$, the next batch is initiated at the same $\epsilon$ level. When there are no nodes left with depth larger than $\epsilon$, the algorithm is considered complete, and the remaining nodes in $\V$ are returned as the {\sf InnerCore}.

\spara{Parameters in Experimental Setup.}
In the context of {\sf InnerCore} expansion and decay,
a greater $i$ (i.e., the history parameter from \S3.2) produces an averaging effect, coupled  with the tendency to depress expansion and inflate decay.  Setting a specific $i$ value depends on the application. %in general, higher $i$ reduces fluctuations between expansion and decay of each day.  
%In our experiments, 
We use $i$ = 1 to improve the accentuation of expansion and decay in the {\sf InnerCore} to better depict the shift in market sentiment during the days of significant events. %in our case studies.}  

In {\sf InnerCore} decomposition, depth values range between $(0, 1]$; nodes with high property values (e.g., many transactions, higher transacted amounts) tend to have low depth, while nodes with low property values tend to have high depth~\cite{victor2021alphacore}. With data depth threshold $\epsilon=1$, all nodes will be returned as {\sf InnerCore}  members; while for $\epsilon=0$, the empty set will be returned.  
Setting an appropriate $\epsilon$ depends on the desired size of the {\sf InnerCore} returned specific to an application.  In our case studies, we set $\epsilon=0.1$ to ensure that the average number of nodes in each daily {\sf InnerCore} is above 150.

%
\begin{figure}
  \centering      \includegraphics[width=0.34\textwidth]{figs/usdc_SCPD}
  \vspace{-3mm}
  \caption{\small USDC anomalous days identified by {\sf SCPD} . Compared to decay and expansion measures by {\sf InnerCore}, {\sf SCPD}  less accurately captures USDC's temporary peg loss occurring on Mar 11, 2023.}
  \label{fig:usdcSCPD}
  \vspace{-5mm}
\end{figure}

\spara{Effectiveness Experiment 3: SCPD to the USDC network.}
We also apply {\sf SCPD}  to the USDC network to compare with our decay and expansion results.  From Figure~\ref{fig:usdcSCPD}, we observe that {\sf SCPD}  less accurately captures USDC's temporary peg loss occurring on Mar 11. {\sf SCPD}  identifies Mar 12 and 15 as anomalous which are one day and four days, respectively, after USDC's peg loss.  Conversely, our expansion measures in Figure~\ref{fig:usdcExpansionDecay} accurately capture USDC's peg loss by producing a prominent peak on Mar 11.  %Similar to results from Experiment 2 (\S 4.4),
Clearly, our {\sf InnerCore} expansion measures more accurately indicate an anomaly on days when a significant event occurred.

\eat{
\subsection{Additional Related Work}
%
In recent years, several studies focused on analyzing different aspects of the blockchain networks \cite{ChenWZCZ19,AkcoraLGK20,kalodner2017blocksci,GuidiM20}, particularly in the Ethereum network. 
Researchers working on natural
language processing and sentiment analysis using tweets, news articles, cryptocurrency prices and charts, Google Trends about blockchains \cite{VNO19,KS20} could find supporting evidences based on
blockchain data analysis. 
Oliveira et al. \cite{oliveira2022analysis} performed an analysis of the effects of external events on the Ethereum  platform, highlighting short-term changes in the behavior of accounts and transactions on the network. Aspembitova et al. \cite{aspembitova2021behavioral} used temporal complex network analysis to determine the properties of users in the Bitcoin and Ethereum markets and developed a methodology to derive behavioral types of users.

Other studies focused on specific aspects of the Ethereum network. For instance, Casale et al. \cite{casale2021networks} analyzed the networks of Ethereum Non-Fungible Tokens using a graph-based approach, while Silva et al. \cite{silva2020characterizing} characterized relationships between primary miners in Ethereum using on-chain transactions. Meanwhile, Victor et al.  \cite{victor2019measuring} measured Ethereum-based ERC20 token networks and Kiffer et al. \cite{kiffer2018analyzing} examined how contracts in Ethereum are created and how users interact with them.

Zhao et al. \cite{zhao2021temporal} investigated the evolutionary nature of Ethereum interaction networks from a temporal graphs perspective, detecting anomalies based on temporal changes in global network properties and forecasting the survival of network communities using relevant graph features and machine learning models. Li et al. \cite{li2021measuring} analyzed the magnitude of illicit activities in the Ethereum ecosystem using proprietary labeling data and machine learning techniques to identify additional malicious addresses. Kilic et al. \cite{kilicc2022fraud} predicted whether given addresses are blacklisted or not in the Ethereum network using a transaction graph and local and global features. 

Our approach for analyzing the effects of external events on a blockchain platform is similar to the one used by Anoaica et al. \cite{anoaica2018quantitative}. The authors examined the temporal variation of transaction features in the Ethereum network and observed an increase in activity following the announcement of the Ethereum Alliance creation. Gaviao et al.  \cite{gaviao2020transaction} also studied the evolution of users and transactions over time, showing the centralization tendency of the transaction network. Kapengut et al.\cite{kapengut2022event} study the Ethereum blockchain around the BeaconChain phase of the PoS transition (September 15, 2022), but the authors focus on the power efficiency and miners' rewards around the transition.


Finally, Khan \cite{khan2022graph} conducted a survey of datasets, methods, and future work related to graph analysis of the Ethereum blockchain data, while Ramezan Poursafaei's PhD thesis \cite{ramezan2022anomaly} presented results on temporal anomaly detection in blockchain networks.
}

\spara{Ethical Consideration.} The use of the efficient and unsupervised core decomposition algorithm, {\sf InnerCore}, could inadvertently raise fairness concerns by potentially providing advantages to certain network traders over others due to the unique market sentiment insight that decay and expansion measures provide. Even in its intended use, reliance on {\sf InnerCore} decay and expansion behavioral pattern results might lead to incorrect or biased reactionary decisions. Similarly, addresses pinpointed by {\sf InnerCore}, followed by subsequent centered-motif analysis and {\sf NF-IAF} score assignment, should be interpreted with carefulness. It is vital for both researchers and decentralized finance traders to exercise prudent judgment and validate findings through additional comparative methods before arriving at any definitive conclusions or undertaking consequential actions.