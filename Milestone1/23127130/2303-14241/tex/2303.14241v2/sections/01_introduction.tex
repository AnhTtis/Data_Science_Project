\section{Introduction}
\label{sec:intro}
Blockchain technology \citep{nakamoto2008bitcoin,wood2014ethereum} is revolutionizing the way we store and transfer digital assets {in multiple domains including internet-of-things \citep{LiuZYZTZ23}, healthcare \citep{LiuYAXZT23}, and digital evidence \citep{TianLQSS19}}. Public blockchain networks are completely open, allowing anonymous addresses to utilize transactions for cryptocurrency movement and asset trading/investment. While the technology offers numerous benefits, it poses significant challenges, particularly in the area of cybersecurity. Blockchains enable electronic crimes in a variety of ways \citep{abs-2212-13452}, ranging from demands for ransomware \citep{huang2018tracking} to transactions in darknet markets \citep{jiang2021illicit}.

One of the biggest challenges in securing blockchain networks is detecting and preventing e-crime. E-crime detection requires scalable analysis of large-scale blockchain graphs in real-time, where results are both qualified and manageable by human analysts. To address this challenge, researchers have developed tools and algorithms for analyzing blockchain networks \citep{victor2021alphacore,akcora2017chainlet,SuG022,KA22}.

Unfortunately, analyzing blockchain networks is an arduous task, given their large size and the involvement of anonymous actors. It is crucial to devise scalable and effective methods that can analyze blockchain networks in real-time, to preempt future losses. The failure to conduct a timely analysis of blockchain networks has already resulted in a staggering loss of billions of dollars to blockchain users, as exemplified by the recent downfall of LunaTerra \citep{lunaterra}.

In this article, we introduce a new approach to detecting e-crimes and trends detection. Our approach, {\textsf InnerCore}, involves identifying influential addresses with data depth-based core decomposition and further filtering out the role of addresses by using centered motifs. 
{\textsf InnerCore} analysis reduces large graphs having more than 400K nodes and 1M edges to an induced subgraph of less than 300 nodes and 90K edges, while still being able to detect the influential nodes.
{\textsf InnerCore} is unsupervised and highly scalable, yielding only $\sim$4-second  running times on daily Ethereum graphs with $\sim$500K nodes and $>$1M edges. We apply {\textsf InnerCore} to three recent important events in the blockchain world: the collapse of LunaTerra in May 2022, the Proof-of-Stake (PoS) switch of Ethereum in September 2022,
and the temporary peg loss of USDC in March 2023. Experimental results demonstrate that our proposed approach effectively detects significant changes in the network without human intervention. Moreover, {\textsf InnerCore} excels in accurately identifying market-manipulating addresses within the network, underscoring its effectiveness in pinpointing key actors.


Our key novelties and contributions are summarized below.
\begin{itemize}[leftmargin=.25in]
 \item 
  \textit{InnerCore}: We propose {\textsf InnerCore}, a data depth-based core discovery method that can identify the influential 
 traders in blockchain-based asset networks (\S\ref{sec:methinnercore}).
 \item 
  \textit{Explainable behavior}: We develop two metrics, {\textsf InnerCore} expansion and decay (\S\ref{sec:methdecayexpansion}), that provide a sentiment indicator for the networks and explain trader mood (\S\ref{sec:methpatterns}). 
 \item 
  \textit{Unsupervised address discovery}: Through conducting node ranking with a centered-motif approach in temporal asset networks, we demonstrate that {\textsf InnerCore} tracking detects market manipulators and e-crime behavior and warns the network about possible long-term instability, without the need for supervised address discovery (\S\ref{sec:methmotif}).
 \item 
  \textit{Scalability}: Due to their computational efficiency and ability to utilize only a small portion of graph nodes and edges to analyze overall behavior, the {\textsf InnerCore} discovery and expansion/decay calculations are suitable on large temporal graphs including Ethereum transaction and stablecoin networks. {\textsf InnerCore} is more effective and efficient than baselines \citep{victor2021alphacore,BatageljZ11} and the state-of-the-art attributed change detection method in dynamic graphs \citep{huang2023fast}  (\S\ref{sec:exp}).
\end{itemize}
%
%We discuss preliminaries and our problem in \S\ref{sec:prelim}.
%Additional related works are specified in the Appendix. %our full version~\cite{InnerCoreAppendix}.
