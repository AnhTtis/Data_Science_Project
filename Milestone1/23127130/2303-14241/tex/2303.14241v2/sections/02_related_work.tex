\section{Related Work}
%
In recent years, several studies focused on analyzing different aspects of the blockchain networks \citep{ChenWZCZ19,AkcoraLGK20,kalodner2017blocksci,GuidiM20}, particularly in the Ethereum network. 
Researchers working on natural
language processing and sentiment analysis using tweets, news articles, cryptocurrency prices, and charts, Google Trends about blockchains \citep{VNO19,KS20} could find supporting evidence based on
blockchain data analysis. 
\citet{oliveira2022analysis} performed an analysis of the effects of external events on the Ethereum  platform, highlighting short-term changes in the behavior of accounts and transactions on the network. \citet{aspembitova2021behavioral} used temporal complex network analysis to determine the properties of users in the Bitcoin and Ethereum markets and developed a methodology to derive behavioral types of users.

Other studies focused on specific aspects of the Ethereum network. For instance, \citet{casale2021networks} analyzed the networks of Ethereum Non-Fungible Tokens using a graph-based approach, while \citet{silva2020characterizing} characterized relationships between primary miners in Ethereum using on-chain transactions. Meanwhile, \citet{victor2019measuring} measured Ethereum-based ERC20 token networks, and \citet{kiffer2018analyzing} examined how contracts in Ethereum are created and how users interact with them.

{Numerous researchers found success in anomaly detection through the strategic exploration of the Ethereum transaction network using graph representation. In particular, \citet{patel2020springer} proposed an one-class graph neural network-based anomaly detection framework for Ethereum transaction networks that harnesses graph representation. \citet{wu2023ieeetran} proposed a scalable transaction tracing tool which incorporates a biased search method to guide the search of fund transfer traces on transaction graphs.}

\citet{zhao2021temporal} investigated the evolutionary nature of Ethereum interaction networks from a temporal graph perspective, detecting anomalies based on temporal changes in global network properties and forecasting the survival of network communities using relevant graph features and machine learning models. \citet{li2021measuring} analyzed the magnitude of illicit activities in the Ethereum ecosystem using proprietary labeling data and machine learning techniques to identify additional malicious addresses. \citet{kilicc2022fraud} predicted whether given addresses are blacklisted or not in the Ethereum network using a transaction graph and local and global features. 

Our temporal approach for analyzing the effects of external events on a blockchain platform is similar to the one used by \citet{anoaica2018quantitative}. The authors examined the temporal variation of transaction features in the Ethereum network and observed an increase in activity following the announcement of the Ethereum Alliance creation.  \citet{gaviao2020transaction} also studied the evolution of users and transactions over time, showing the centralization tendency of the transaction network. \citet{kapengut2022event} studied the Ethereum blockchain around the BeaconChain phase of the PoS transition (September 15, 2022), but the authors focused on the power efficiency and miners' rewards around the transition.


Finally, \citet{khan2022graph} conducted a survey of datasets, methods, and future work related to graph analysis of the Ethereum blockchain data, while Poursafaei's PhD thesis \citep{ramezan2022anomaly} presented results on temporal anomaly detection in blockchain networks.