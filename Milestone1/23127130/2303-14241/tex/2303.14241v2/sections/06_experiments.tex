\section{Experimental Results}
\label{sec:exp}
%
In this section, we first describe three large temporal blockchain graphs that we use to answer our research questions (\S\ref{sec:prob}). Next, we analyze the scalability of {\textsf InnerCore} discovery and centered-motif analysis on these graphs. Upon demonstrating our scalability results, we illustrate how our methods provide predictive insights into anomalies stemming from external events and identify the addresses that played a significant role in such events. Our code and datasets are available at {\url{https://github.com/JZ-FSDev/InnerCore}}.
%
\vspace{-2mm}
\subsection{Environment Setup}
%
\subsubsection{Datasets}
\label{sec:datasets}
%
Our experiments investigate the Ethereum transaction network and Ethereum stablecoin networks across three recent real-world events: the LunaTerra collapse, Ethereum's transition to Proof-of-Stake, and USDC's temporary peg loss. {For each of our experiments, we construct a transaction network from the following datasets.}

\noindent\textbf{Ethereum Stablecoin Transaction Networks}. We retrieve transaction data for the top five stablecoins based on market capitalization (UST, USDC, DAI, UST, PAX) and WLUNA from the Chartalist repository~\citep{Chartalist2022}. The data pertains only to transactions conducted on the Ethereum blockchain; each transaction in the data set corresponds to a transfer of the asset indicated by the contract address.
However, the UST collapse event that we are studying involved another blockchain called Terra with its own network, and the cryptocurrency called Luna, acting as a parallel to ether on Ethereum. Terra issued a stablecoin named UST (also known as TerraUSD), which offered high-interest rates to lenders and was pegged to the value of \$USD1. Additionally, Terra's owners created an ERC-20 version UST on the Ethereum blockchain and a Wrapped LUNA (WLUNA) token was established to trade Luna tokens on Ethereum. In May 2022, the Terra blockchain and its cryptocurrency Luna collapsed, owing to TerraUSD loans that could not be repaid. A Luna coin that was valued at \$USD116 in April plummeted to a fraction of a penny during the collapse\footnote{{\tiny \url{https://coinmarketcap.com/currencies/wrapped-luna-token/}}}. This resulted in a loss of confidence in both WLUNA and UST on Ethereum. On May 9th, 2022, UST lost its \$USD1 peg and fell as low as 35 cents\footnote{{\tiny \url{https://coinmarketcap.com/currencies/terrausd/}}}.  The Ethereum Stablecoin dataset covers the period from April 1st, 2022, to November 1st, 2022, spanning about one month before the crash to six months after the crash. %In addition to the transactions, 
{We construct a transaction network consisting of UST, USDC, DAI, UST, PAX, and WLUNA transactions for \S\ref{sec:exp1} between this period.}
We also use the address labels dataset from~\citet{Chartalist2022} where labels of 296 addresses from 149 centralized and decentralized Ethereum exchange addresses are listed publicly {to distinguish unique exchange addresses}.

In March 2023, Silicon Valley Bank, holding over 3 billion of Circle’s collateralized reserves collapsed abruptly, causing a mass liquidation of USDC from traders. Consequently, on March 11th, 2023, Circle's USDC temporarily lost its \$USD1 peg, dropping to an all-time low of 87 cents.  The USDC dataset covers the period from February 25th, 2023, to March 23, 2023, spanning approximately two weeks before and after the peg loss. {We use a transaction network consisting of only USDC transactions for \S\ref{sec:exp3}.}

\noindent\textbf{Ethereum Transaction Network}. We collected ether transactions from the Ethereum blockchain for the period between August 21st and October 1st, 2022. On an average day during this period, there were 480,000 addresses, with approximately 1 million edges connecting them. Ether is a type of cryptocurrency, similar to bitcoin, and its value can be converted to various fiat currencies such as USD and JPY. 
% \textcolor{red}{The nodes on the graph could represent traders who anticipate future price increases for ether or traders who engage in buying and selling goods and services.} 
Ethereum changed its block creation process during this time, moving from the costly Proof-of-Work method to the more efficient Proof-of-Stake algorithm in two phases on September 9th and 15th, 2022. 


\subsubsection{Competitors} We compare {\textsf InnerCore} with two baselines: {\textsf AlphaCore}~\citet{victor2021alphacore} and graph-$k$-core~\citep{BatageljZ11}. We refer to \S\ref{sec:methinnercore}, \textbf{InnerCore vs. Alphacore} for their differences. Additionally, we compare against {\textsf Scalable Change Point Detection (SCPD)} \citep{huang2023fast}, state-of-the-art attributed change detection method in dynamic graphs.% It is a novel spectral method to identify anomalies from a set of graph snapshots.
%
\begin{figure}
  \centering      \includegraphics[width=0.45\textwidth]{figs/time}
  \vspace{-3mm}
  \caption{\small Comparison between running times of {\textsf AlphaCore} with the starting $\epsilon=1.0$ and stepsize $s=0.1$, {\textsf InnerCore} with $\epsilon=0.1$ on daily Ethereum transaction networks to return the {\textsf InnerCore} of depth $<$ 0.1. %, and SCPD's DOS computation.  
  An average of approximately 480,000 nodes (addresses) and 1 million edges (transactions) exist in each network. The average computation time is 4.06 seconds (max 8.1s), which is approximately 0.10 times the average computation time of {\textsf AlphaCore}, 0.12 times the average computation time of the highest graph $k$-core, and 0.14 times the average computation time of {\textsf SCPD}. %'s DOS.  
  }
  \label{fig:time}
  % \vspace{-12px}
\end{figure}

\vspace{-2mm}
 \subsection{Scalability Analysis}

\noindent\textbf{System Specifications}.
The machine used for experiments is an Intel Core i7-8700K CPU @ 3.70GHz processor, 32.0GB RAM, Windows10 OS, and GeForce GTX1070 GPU.  A combination of Python and R was used for coding.


%\medskip
\noindent\textbf{InnerCore Discovery}.
Since we are interested in directly finding the {\textsf InnerCore}, compared to {\textsf AlphaCore} decomposition~\citep{victor2021alphacore}, {\textsf InnerCore} discovery method (\S 3.1) does not associate different $\epsilon$ values to intermediate cores generated in an iterative stepwise fashion.  Instead, a fixed threshold $\epsilon$, or upper bound for depth, is set and all nodes with a depth greater than $\epsilon$ are pruned repetitively until all remaining nodes relative to each other in the resulting network have a depth  $< \epsilon$. This allows {\textsf InnerCore} discovery to run approximately 1/stepsize times faster than {\textsf AlphaCore} decomposition since the computations of all intermediate cores are skipped.
As depicted in Figure  \ref{fig:time}, the average running time for {\textsf InnerCore} discovery is only 4.06 seconds on graphs with approximately 480,000 nodes and 1 million edges. Furthermore, {\textsf InnerCore} discovery has a running time of only one-tenth of that for {\textsf AlphaCore} decomposition.

Due to the need for graph-$k$-core to repetitively iterate over all remaining nodes with each peeling until the highest $k$-core remains, we find {\textsf InnerCore} to be nearly 8x faster on each daily graph snapshot.

{\textsf SCPD} is state-of-the-art method to identify anomalies from attributed graph snapshots~\citep{huang2023fast}.  Due to its spectral approach, we find it slower:
%Similarly, we utilize {\textsf InnerCore} expansion and decay (\S 3.2) to identify anomalies from daily temporal networks.  To this end, we compare the state-of-the-art SCPD method to our {\textsf InnerCore} expansion and decay method.  Note {\textsf InnerCore} discovery is the precursor required to plot {\textsf InnerCore} expansion and decay measures for identification of anomalous days as the density of states (DOS) embedding computation is the precursor for SCPD~\cite{huang2023fast} to detect anomalies.  Thus, we timed the DOS embedding computation of each daily Ethereum network snapshot and observe from Figure \ref{fig:time} that compared to SCPD's DOS embedding computation, 
{\textsf InnerCore} discovery runs nearly 7x faster on each daily graph snapshot, which demonstrates the scalability of our solution. 


\smallskip
\noindent\textbf{Three-Node Motifs Counting}.  
Instead of conducting motif analysis on all nodes, our approach utilizes the {\textsf InnerCore}. By focusing on this core subset of nodes, we are able to reduce the number of nodes in a daily network consisting of approximately 480,000 nodes and 1 million edges to an induced subgraph of roughly 300 nodes and 90,000 edges (counting multi-edges), resulting in a more manageable and efficient approach.
Although centered motif counting on each snapshot graph takes $>$ 1 day to complete, motif counting inside {\textsf InnerCore}  significantly improves the processing speed, requiring only $<$ 20 secs to complete, which illustrates our scalability.
%  
\vspace{-2mm}
\subsection{Effectiveness Analysis}
%
\subsubsection{Experiment 1: The Collapse of LunaTerra}
\label{sec:exp1}
Stablecoins are meant to be a safe house as they are generally pegged to and maintain a 1:1 ratio with a fiat currency, resisting the volatility associated with other popular cryptocurrencies. Commonly, traders keep blockchain assets not needed for immediate use in a transaction as a stablecoin, analogous to people keeping extra money in a bank.  For this reason, The LunaTerra collapse was a historic event in the decentralized financial space as it questioned traders' trust in cryptocurrencies; if even stablecoins are susceptible to collapse, then is any cryptocurrency truly safe?  

 
\begin{figure}
  \centering      \includegraphics[width=0.45\textwidth]{figs/visuals}
  \vspace{-3mm}
  \caption{\small Stablecoin decay and expansion measures. On May 8 (shown with the vertical blue line), UST loses its \$1 peg and falls to as low as 35 cents.}
  \label{fig:stablecoinDecayExapansion}
  % \vspace{-5mm}
\end{figure}

\begin{figure}
  \centering      \includegraphics[width=0.45\textwidth]{figs/stablecoin_SCPD}
  \vspace{-3mm}
  \caption{\small Stablecoin anomalous days identified by {\textsf SCPD}. Unlike decay and expansion measures by {\textsf InnerCore}, {\textsf SCPD} less accentuates the critical event of UST's peg loss in Ethereum stablecoin networks, compared to other anomalies that occurred between Apr 3 to Oct 30, 2022.}
  \label{fig:stablecoinSCPD}
  % \vspace{-5mm}
\end{figure}

\smallskip
\noindent\textbf{Behavioral Patterns via Expansion and Decay.}
First, we analyze this event from the perspective of traders' market sentiment %in the stablecoin network.  Specifically, we examine behavioral patterns in the 
via expansion and decay measures of the temporal stablecoin network for the days surrounding the collapse.  %From 
In Figure~\ref{fig:stablecoinDecayExapansion}, %we observe that 
four days after the collapse unfolded, on May 13, 2022, there was a substantial increase in decay and a decrease in expansion: a prime indicator of the {\em despair} behavioral pattern (\S 3.3).  We can infer from this signal that a large majority of regular traders stopped trading by this time, either from the conversion or sale of any assets stored as UST out of the stablecoin ecosystem or simply due to uncertainty and inaction in response to the collapse.  Following this cue, for approximately two weeks afterward, we see a consistent behavioral pattern of {\em faith} characterized by low expansion and low decay.  During this period, few new traders entered or left the stablecoin network.  There was still faith in the remaining traders that perhaps a large stablecoin such as UST could rebound and restore its peg with USD and thus, they refrained from engaging in any transactions.  On the other hand, decay and expansion values also indicate a sign of hopelessness as the bulk of traders already exited the network since the first signal of despair.  We understand from this behavioral analysis that there is a delayed reaction from traders when a significant unannounced event occurs due to indecision, and there is a general trend of inactivity in the following period.

\smallskip
\noindent\textbf{Why is this e-crime?} We outline two reasons. \textbf{Dumping of UST:} On May 7th, large sums of UST were dumped, with 85 million UST swapped for 84.5 million USDC \citep{liu2023anatomy}. This massive dumping of UST contributed to its de-pegging and caused its value to drop significantly. \textbf{Concealing past failures:} The CEO of Terra, Do Kwon, was revealed to be a co-creator of the failed algorithmic stablecoin, Basis Cash \citep{impekoven2023central}. The concealment of such information about the project's founder could mislead traders and hide potential risks. 

\smallskip
\noindent\textbf{SCPD vs. InnerCore}. 
% \textcolor{red}{We compare our 
%  expansion and decay results against the effectiveness of {\textsf SCPD} on Ethereum stablecoin networks.}  
From Figure~\ref{fig:stablecoinSCPD}, we observe that {\textsf SCPD} less accentuates the critical event of UST's peg loss {and {\textsf InnerCore} more accurately depicts the impact of the collapse on the market relative to other days in the data time span.}  %in Ethereum stablecoin networks.  SCPD 
{\textsf SCPD} assigns an anomaly score to Sep 26 when USDC announced their plan to expand to five new blockchains\footnote{{\tiny \url{https://www.chartalist.org/eth/StablecoinAnalysis.html}}}, nearly two times as anomalous as the score assigned to May 4, the closest day to the LunaTerra collapse.  However, our Stablecoin decay and expansion measures in Figure~\ref{fig:stablecoinDecayExapansion} notably accentuate and emphasize the impact of UST's peg loss on the stablecoin ecosystem from the less impactful events occurring on other days. This accentuation is evident by the presence of a pronounced decay peak on May 13 followed by a period of approximately two weeks of consistently low decay and expansion measures before returning to more standard values seen in other days, clearly indicating a significant event had transpired.  This demonstrates that decay and expansion measures serve as a better indicator of the significance of an event on its corresponding network.


% \begin{figure}
%   \centering      \includegraphics[width=0.65\textwidth]{figs/lunacores}
%   \vspace{-3mm}
%   \caption{\small Sizes of stablecoin network cores during LunaTerra collapse. {\textsf InnerCore}s accentuates the collapse better than nodes found in the highest cores via graph-$k$-core decomposition.  
%   }
%   \label{fig:lunaTerraCoreSize}
%   \vspace{-4mm}
% \end{figure}

 %
\begin{table}[t]
  \centering
  \caption{\small Numbers of center addresses in motifs identified by our method (\S 3.4) that are known exchanges. The numbers represent the total counts per motif across all days.}
  \label{tab:exchanges}
  % \vspace{3mm}
  \small
  \begin{tabular}{ccc}
    %\toprule
     & \# Unique Addresses & \# Exchange Addresses\\
    \midrule
    Motif 1 & 1221 & 15\\
    Motif 4 & 1762 & 15\\
    Motif 5 & 1447 & 17\\
    Motif 6 & 1513 & 4\\
    Motif 11 & 939 & 11\\
  %\bottomrule
\end{tabular}
% \vspace{-3mm}
\end{table}
%
\begin{table}[t]
  \centering
  \caption{\small {{\textsf NF-IAF} score percentile ranks of {\textsf InnerCore} motif centers  matching highlighted addresses by Nansen.ai to have played key roles before (May 7), during (May 8), and after (May 9, 2022) the LunaTerra collapse. The percentile scores for individual addresses on a specific day of a particular motif center are determined relative to all addresses associated with the same motif center throughout all days in the data window. Motif centers $C_1$, $C_{5a}$, $C_{11}$ exhibit sell behavior, while motif centers $C_4$, $C_{5b}$, $C_6$ exhibit buy behaviour. Addresses with percentiles $\geq$ 90 across at least one motif center type (given in red color) are considered impactful on a given day. Dashes indicate absence of the address as the motif center.}}
  \label{tab:nansenAddresses}
  % \vspace{3mm}
  \small
  \begin{tabular}{ccccccc}
    \multicolumn{7}{c}{LunaTerra addresses on May 7} \\
    \hline
    Address/Motif Center & $C_1$ & $C_4$ & $C_{5a}$ & $C_{5b}$ & $C_6$ & $C_{11}$ \\
    \hline
{\sf \textcolor{gray}{Celsius}}		&	\textcolor{gray}{-}	&	\textcolor{gray}{81}	&	\textcolor{gray}{79}	&	\textcolor{gray}{-}	&	\textcolor{gray}{-}	&	\textcolor{gray}{-}	\\
{\sf \textcolor{gray}{hs0327.eth}}		&	\textcolor{gray}{30}	&	\textcolor{gray}{4}	&	\textcolor{gray}{28}	&	\textcolor{gray}{28}	&	\textcolor{gray}{4}	&	\textcolor{gray}{-}	\\
{\sf Smart LP: 0x413}		&	\textcolor{gray}{-}	&	\textcolor{gray}{69}	&	\textcolor{gray}{-}	&	\textcolor{gray}{-}	&	\textcolor{red}{95}	&	\textcolor{gray}{-}	\\
{\sf \textcolor{gray}{Token Millionaire 1}}		&	\textcolor{gray}{85}	&	\textcolor{gray}{81}	&	\textcolor{gray}{73}	&	\textcolor{gray}{-}	&	\textcolor{gray}{67}	&	\textcolor{gray}{89}	\\
{\sf Token Millionaire 2}		&	\textcolor{gray}{35}	&	\textcolor{red}{100}	&	\textcolor{red}{99}	&	\textcolor{gray}{-}	&	\textcolor{red}{99}	&	\textcolor{gray}{38}	\\
{\sf masknft.eth}		&	\textcolor{red}{97}	&	\textcolor{red}{94}	&	\textcolor{gray}{82}	&	\textcolor{gray}{-}	&	\textcolor{red}{93}	&	\textcolor{red}{92}	\\
{\sf \textcolor{gray}{Heavy Dex Trader}}		&	\textcolor{gray}{54}	&	\textcolor{gray}{17}	&	\textcolor{gray}{-}	&	\textcolor{gray}{-}	&	\textcolor{gray}{32}	&	\textcolor{gray}{-}	\\
{\sf Oapital}		&	\textcolor{red}{94}	&	\textcolor{gray}{83}	&	\textcolor{gray}{62}	&	\textcolor{gray}{62}	&	\textcolor{gray}{72}	&	\textcolor{red}{92}	\\
{\sf Hodlnaut}		&	\textcolor{gray}{40}	&	\textcolor{red}{99}	&	\textcolor{red}{90}	&	\textcolor{gray}{-}	&	\textcolor{red}{99}	&	\textcolor{gray}{-}	\\
    \hline
    \addlinespace
    \addlinespace
    \multicolumn{7}{c}{LunaTerra addresses on May 8} \\
    \hline
    Address/Motif Center & $C_1$ & $C_4$ & $C_{5a}$ & $C_{5b}$ & $C_6$ & $C_{11}$ \\
    \hline
{\sf \textcolor{gray}{Celsius}}		&	\textcolor{gray}{-}	&	\textcolor{gray}{81}	&	\textcolor{gray}{79}	&	\textcolor{gray}{-}	&	\textcolor{gray}{-}	&	\textcolor{gray}{-}	\\
{\textsf hs0327.eth}		&	\textcolor{gray}{88}	&	\textcolor{gray}{67}	&	\textcolor{gray}{70}	&	\textcolor{red}{96}	&	\textcolor{gray}{82}	&	\textcolor{gray}{-}	\\
{\sf Smart LP: 0x413}		&	\textcolor{gray}{-}	&	\textcolor{gray}{68}	&	\textcolor{gray}{-}	&	\textcolor{gray}{-}	&	\textcolor{red}{95}	&	\textcolor{gray}{-}	\\
{\sf Token Millionaire 1}		&	\textcolor{gray}{85}	&	\textcolor{red}{90}	&	\textcolor{gray}{86}	&	\textcolor{gray}{-}	&	\textcolor{gray}{74}	&	\textcolor{gray}{89}	\\
{\sf Token Millionaire 2}		&	\textcolor{gray}{70}	&	\textcolor{red}{100}	&	\textcolor{red}{99}	&	\textcolor{gray}{-}	&	\textcolor{red}{99}	&	\textcolor{gray}{38}	\\
{\sf masknft.eth}		&	\textcolor{red}{91}	&	\textcolor{red}{91}	&	\textcolor{gray}{82}	&	\textcolor{gray}{-}	&	\textcolor{red}{93}	&	\textcolor{red}{92}	\\
{\sf Heavy Dex Trader}		&	\textcolor{gray}{71}	&	\textcolor{red}{96}	&	\textcolor{gray}{-}	&	\textcolor{gray}{-}	&	\textcolor{gray}{81}	&	\textcolor{gray}{-}	\\
{\sf Oapital}		&	\textcolor{red}{92}	&	\textcolor{gray}{79}	&	\textcolor{gray}{58}	&	\textcolor{gray}{61}	&	\textcolor{gray}{72}	&	\textcolor{red}{93}	\\
{\sf Hodlnaut}		&	\textcolor{gray}{40}	&	\textcolor{red}{99}	&	\textcolor{red}{91}	&	\textcolor{gray}{-}	&	\textcolor{red}{99}	&	\textcolor{gray}{-}	\\
    \hline
    \addlinespace
    \addlinespace
    \multicolumn{7}{c}{LunaTerra addresses on May 9} \\
    \hline
    Address/Motif Center & $C_1$ & $C_4$ & $C_{5a}$ & $C_{5b}$ & $C_6$ & $C_{11}$ \\
    \hline
{\sf \textcolor{gray}{Celsius}}		&	\textcolor{gray}{-}	&	\textcolor{gray}{80}	&	\textcolor{gray}{77}	&	\textcolor{gray}{-}	&	\textcolor{gray}{-}	&	\textcolor{gray}{-}	\\
{\sf hs0327.eth}		&	\textcolor{red}{95}	&	\textcolor{gray}{66}	&	\textcolor{gray}{68}	&	\textcolor{red}{95}	&	\textcolor{gray}{79}	&	\textcolor{gray}{-}	\\
{\sf Smart LP: 0x413}		&	\textcolor{gray}{-}	&	\textcolor{gray}{67}	&	\textcolor{gray}{-}	&	\textcolor{gray}{-}	&	\textcolor{red}{95}	&	\textcolor{gray}{-}	\\
{\sf \textcolor{gray}{Token Millionaire 1}}		&	\textcolor{gray}{83}	&	\textcolor{gray}{89}	&	\textcolor{gray}{85}	&	\textcolor{gray}{-}	&	\textcolor{gray}{73}	&	\textcolor{gray}{88}	\\
{\sf Token Millionaire 2}		&	\textcolor{gray}{67}	&	\textcolor{red}{100}	&	\textcolor{red}{99}	&	\textcolor{gray}{-}	&	\textcolor{red}{99}	&	\textcolor{gray}{88}	\\
{\sf masknft.eth}		&	\textcolor{red}{90}	&	\textcolor{red}{90}	&	\textcolor{gray}{81}	&	\textcolor{gray}{-}	&	\textcolor{red}{92}	&	\textcolor{red}{92}	\\
{\sf Heavy Dex Trader}		&	\textcolor{gray}{70}	&	\textcolor{red}{93}	&	\textcolor{gray}{-}	&	\textcolor{gray}{-}	&	\textcolor{gray}{80}	&	\textcolor{gray}{-}	\\
{\sf Oapital}		&	\textcolor{red}{94}	&	\textcolor{gray}{78}	&	\textcolor{gray}{57}	&	\textcolor{gray}{63}	&	\textcolor{gray}{71}	&	\textcolor{red}{94}	\\
{\sf Hodlnaut}		&	\textcolor{gray}{39}	&	\textcolor{red}{99}	&	\textcolor{red}{90}	&	\textcolor{gray}{-}	&	\textcolor{red}{99}	&	\textcolor{gray}{-}	\\
    \hline
  \end{tabular}
\end{table}

% \smallskip
% \noindent\textbf{K-Core vs. InnerCore.} Compared to traditional graph-$k$-core decomposition where the target is the highest $k$-core, {\textsf InnerCore} with a fixed core target of $\epsilon$ = 0.1 better accentuates the LunaTerra collapse when applied to temporal networks and comparing the sizes of the target cores. 
% From Figure~\ref{fig:lunaTerraCoreSize}, it is evident that the size of the highest graph-$k$-core fluctuates randomly across each day and does not respond to changes in the health of the network. 

% This can be attributed to the fact that graph-$k$-core decomposition only considers the degree of nodes irrespective of the edge weights.  Conversely, {\textsf InnerCore} considers all the provided node features (\S3 and Table~\ref{tab:node_property_functions}) into consideration when computing each node's depth, which includes edge weights in addition to node degree. Therefore, if a node is incident to a single edge of extremely high edge weight, the highest graph-$k$-core will exclude this node; whereas {\textsf InnerCore} will, depending on the target $\epsilon$, maintain it in its {\textsf InnerCore} as the high edge weight offsets the low node degree.

\smallskip
\noindent\textbf{Identify Key Addresses.} Before the LunaTerra collapse, it is reasonable to assume that traders responsible for the collapse would prepare for the anticipated negative consequences by exiting the UST network and entering another reliable stablecoin. In order to capture these transactions of traders converting between different stablecoins, we have included four stablecoins in our network along with UST.  We focus on the unknown addresses that occurred most frequently as motif centers in {\textsf InnerCore}s (defined in \S 3.4) on days immediately before the LunaTerra collapse since they could have influenced the initial phase of the crash.

Generally, a large amount of tokens transferred from one address to another is easily detectable due to the sheer volume.  However, if a trader tries to confiscate detection, the trader could produce multiple transactions with smaller volumes.  Additionally, often in a transaction where one token is exchanged for another, a series of multiple transfers can arise for a single conversion transaction due to interactions with exchanges.\footnote{{\tiny \url{https://etherscan.io/tx/0xa3663b813b2c13a88daeeb5b48b32b7024fc07cbf250f2c2a9318ec1950c9da9}}}
Therefore, a trader is more likely to exhibit both selling and buying behaviors, making the trader a prime candidate as a 3-node motif center.

\smallskip
\noindent\textbf{Ground Truth}. Nansen (\url{https://www.nansen.ai/}) is a prominent blockchain analytics platform that frequently publishes comprehensive analyses of blockchain events, which are followed with great interest by the industry. Nansen.ai conducted a thorough analysis of the LunaTerra collapse in May 2022 and identified 11 important addresses that played central roles in the collapse~\citep{NansenLunaTerra}. We %aim to 
compare the addresses of interest detected by our {\textsf InnerCore} analysis using the centered-motif approach with those identified by Nansen.ai (Table~\ref{tab:nansenAddresses}) as the primary candidates for triggering the collapse.

Exchanges are an intermediary hub to facilitate transfers between traders.  The addresses of exchanges are well-known for this reason, making them not very interesting in our context.  In contrast, addresses that are not exchanges are mostly owned by traders and thus, the existence of such addresses and their edges in a network is a direct consequence of a trader’s activity in the network.  From Table \ref{tab:exchanges}, we observe that motif centers identified from {\textsf InnerCore}s have a high ratio of non-exchange addresses to exchange addresses ($\approx$99\%).  This shows the effectiveness of our method to identify potentially meaningful addresses in a network different from high-traffic exchange addresses.  

In particular, we capture 9 of 11 externally owned addresses ({\textsf EoA}s) in Table \ref{tab:nansenAddresses} identified by Nansen.ai that occurred as center addresses for our motif types (Figure \ref{fig:motifs}) on days immediately leading up to the LunaTerra collapse.  We notice that the {\textsf NF-IAF} score percentile ranks of these addresses are higher compared to that of other center addresses for the same motif type on the same day, indicating that these addresses were important traders contributing to the buy or sell behavior associated with the motif on the day. We surmise the possibility that certain {\textsf EoA}s found by our {\textsf InnerCore} method, coupled with centered-motif analysis, could have been responsible for the initial phase of the collapse.

Recall that in Figure \ref{fig:motifs}, we defined motif centers $C_1$, $C_{5a}$, and $C_{11}$ as exhibiting sell behavior; while motif centers $C_4$, $C_{5b}$, and $C_6$ as exhibiting buy behavior.  It is evident from Table \ref{tab:nansenAddresses} that every motif center on May 8, 2022, has at least one corresponding trader with an {\textsf NF-IAF} score percentile rank above 90.  This suggests that addresses with greater {\textsf NF-IAF} percentiles exhibit a higher buy or sell behavior associated with the particular motif type on the day of the collapse.  
{Specifically, we identify two traders, {\textsf hs0327.eth} and {\textsf Heavy Dex Trader}, as the most likely candidates for influencing the initial phase of the crash, since they had the greatest {\textsf NF-IAF} score percentile increases from May 7 to May 8, 2022 consistently across all their participating motif center types in comparison to other addresses. In addition, we identify the two traders, {\textsf masknft.eth} and {\textsf Oapital}, as key participants throughout the crash, since they are the two addresses with greater {\textsf NF-IAF} percentiles (above 90) occurring consistently across at least two motif types exhibiting sell behavior on days before, during, and after the crash. We identify {\textsf Celsius} as being the least likely trader to have directly impacted the collapse as it is the only address which had score percentiles $<$ 90 across all three days.}

%\smallskip
\noindent\textbf{K-Core vs. InnerCore}. 
% \textcolor{red}{We compare the  addresses identified by {\textsf InnerCore} + centered-motif analysis with those in the highest graph-$k$-core on the May 8 (i.e., one day before the crash) stablecoin temporal network.} 
We notice that graph-$k$-core cannot find any of the 11  addresses indicated by Nansen.ai as prime candidates for triggering the initial phase of %the 
LunaTerra collapse.  In comparison, {\textsf InnerCore} + centered-motif analysis captures potentially anomalous buy and sell behaviors
by identifying 9 of the 11 addresses.   
% 
\vspace{-2mm}
\subsubsection{Experiment 2: Ethereum's Switch to PoS}
Ethereum's transition from Proof-of-Work (PoW) to Proof-of-Stake (PoS) came with many benefits including enhanced security for users and lower energy consumption.  Together, these positives incentivized new traders to participate in the Ethereum network due to increased trust in the blockchain and lower barriers to entry.  The transition occurred in two phases; the first phase was a preparatory hard forking of the blockchain into a PoS structure and the second phase was a finalization of the upgrade.  

A pattern of {\em hope} was expected as the upgrade was highly anticipated due to the positives, transparency, and consistent updates regarding the official dates of the upgrade.  
From Figure~\ref{fig:ethDecayExpansion}, we indeed verify this behavioral pattern of {\em hope} characterized by inflated expansion values, coupled with relatively stable decay values, on three separate occasions.  The first occurrence of {\em hope} is observed approximately a week before the first phase of the upgrade took place.  It was around this time, the end of August 2022, that official news regarding the concrete dates of when the upgrade would be expected to take place was released to the public.  We observe a surge of new hopeful traders participating in the Ethereum network and a significant dip in existing traders leaving the network in  
 anticipation of the upgrade.  The other two instances of {\em hope} are seen during the immediate days surrounding and between each of the phases of the upgrade.  These occurrences provide insight into the market sentiment during the upgrade as positive and the overall transition of Ethereum to PoS as being well-received by traders.

\smallskip
\noindent\textbf{SCPD vs. InnerCore}. 
We next apply {\textsf SCPD} on the Ethereum transaction network to compare against our expansion and decay results.  From Figure~\ref{fig:ethSCPD}, we notice that {\textsf SCPD} less accurately captures the two phases of Ethereum's transition to POS occurring on Sep 6 and 15, 2022.  {\textsf SCPD} identifies Sep 9 and 16 as anomalous, which are two days before the first phase and one day after the second phase, respectively, of Ethereum's transition to POS.  In contrast, our expansion measures in Figure~\ref{fig:ethDecayExpansion} more accurately capture the phases of Ethereum transition to POS by producing a peak on Sep 4, one day before the first phase, and on Sep 15, the same day of the second phase. {It is evident {\textsf InnerCore} detects the second phase of the switch on the day of the event, whereas {\textsf SCPD} can only detect the event after it has occurred.}   Therefore, {\textsf InnerCore} expansion measures more accurately detect an anomaly on days when a significant event actually unfolded.
 
% 
  \begin{figure}
  \centering
\includegraphics[width=0.45\textwidth]{figs/eth}
  \vspace{-3mm}
  \caption{\small Ethereum decay and expansion measures. The move of Ethereum to Proof-of-Stake mining took place in two stages, indicated by 2 vertical blue lines (Sep 6 and 15, 2022). {{An expansion peak on Sep 5, 2022 detects the anomaly one day before the first stage commenced.}} }
  \label{fig:ethDecayExpansion}
  % \vspace{-5mm}
\end{figure}

% 
  \begin{figure}
  \centering
\includegraphics[width=0.45\textwidth]{figs/eth_SCPD}
  \vspace{-3mm}
  \caption{\small Ethereum anomalous days identified by {\textsf SCPD}. Compared to decay and expansion measures by {\textsf InnerCore}, {\textsf SCPD} less accurately captures the two phases of Ethereum's transition to POS occurring on Sep 6 and 15, 2022.}
  \label{fig:ethSCPD}
  % \vspace{-5mm}
\end{figure}


\subsubsection{Experiment 3: USDC's Temporary Peg Loss}
\label{sec:exp3}
On May 11th, 2023, a significant event unfolded in the stablecoin market as Circle's stablecoin, USDC, experienced a temporary loss of its peg, plummeting to a concerning value of 87 cents.\footnote{{\tiny \url{https://coinmarketcap.com/currencies/usd-coin/}}} The abrupt collapse of Silicon Valley Bank, which held over 3 billion of Circle's reserves, triggered panic among traders. Fearing a collapse, many traders liquidated their USDC holdings and sought refuge in alternative stablecoins like MakerDAO's DAI. 
% \textcolor{red}{Notably, USDC differs from Terra's UST, which had algorithmic pegging to Luna and had suffered a complete collapse approximately a year earlier. USDC, in contrast, maintains a strong collateralization to fiat reserves, marking a crucial distinction between the two stablecoins.} 

By analyzing the expansion and decay measures surrounding the incident, we realize how traders responded differently to this event. Figure~\ref{fig:usdcExpansionDecay} shows a sudden surge in expansion on May 11th, 2023, attributing to a wave of traders liquidating their USDC holdings in response to the stablecoin's all-time low value of 87 cents. In the subsequent three days following the temporary loss of USDC's peg, a distinct series of behavioral patterns emerged, characterized by alternating signals of {\em despair}, {\em hope}, and {\em despair} again, before eventually stabilizing. During this three-day period, Circle's reassurances regarding the recovery of lost reserves gradually restored trust among its traders. This is evident through the decreasing extent of {\em despair} patterns observed on the 12th and 14th.

In summary, traders' reactions were initially marked by panic and a rush to sell USDC, causing a surge in expansion. However, as Circle provided updates on their efforts to recover the lost reserves, a sense of hope permeated the market, leading to a decline in the extent of despair patterns. Ultimately, the stablecoin regained stability, with expansion and decay returning to typical levels.

%
\begin{figure}
  \centering      
  \includegraphics[width=0.45\textwidth]{figs/usdc}
  \vspace{-3mm}
  \caption{\small USDC decay and expansion measures. On Mar 11, 2023 (shown with the vertical blue line), USDC loses its \$1 peg and falls to as low as 87 cents. {{An expansion peak detects the anomaly on the day the event transpires.}}}
  \label{fig:usdcExpansionDecay}
  % \vspace{-5mm}
\end{figure}

%

\begin{figure}
  \centering      
  \includegraphics[width=0.45\textwidth]{figs/usdc_SCPD}
  \vspace{-3mm}
  \caption{\small USDC anomalous days identified by {\textsf SCPD} . Compared to decay and expansion measures by {\textsf InnerCore}, {\textsf SCPD}  less accurately captures USDC's temporary peg loss occurring on Mar 11, 2023.}
  \label{fig:usdcSCPD}
  % \vspace{-5mm}
\end{figure}

\smallskip
\noindent\textbf{SCPD vs. InnerCore}. 
% \spara{Effectiveness Experiment 3: SCPD to the USDC network.}
We also apply {\textsf SCPD}  to the USDC network in order to compare with our decay and expansion results.  From Figure~\ref{fig:usdcSCPD}, we observe that {\textsf SCPD}  less accurately captures USDC's temporary peg loss occurring on Mar 11. {\textsf SCPD}  identifies Mar 12 and 15 as anomalous which are one day and four days, respectively, after USDC's peg loss.  Conversely, our expansion measures in Figure~\ref{fig:usdcExpansionDecay} accurately capture USDC's peg loss by producing a prominent peak on Mar 11. {It is evident that {\textsf InnerCore} detects the temporary peg loss on the day of the event, whereas SCPD can only detect the event after it has occurred.}
Clearly, our {\textsf InnerCore} expansion measures more accurately indicate an anomaly on days when a significant event occurred.
 