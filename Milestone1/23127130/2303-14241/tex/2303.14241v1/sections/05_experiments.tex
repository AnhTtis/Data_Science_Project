\section{Experimental Study}
\label{sec:exp}
%
In this section we first describe two large temporal blockchain graphs that we use to answer our research questions (\S\ref{sec:prob}). Next, we analyze the scalability of {\sf InnerCore} discovery and centered-motif analysis on these graphs. Upon demonstrating our scalability results, we illustrate how our methods provide predictive insights into anomalies stemming from external events and identify the addresses that played a significant role in such events. Our code and datasets are available at {\url{https://github.com/JZ-FSDev/InnerCore}}.
%
\subsection{Environment Setup}
%
\subsubsection{Datasets}
%
Our experiments investigate the Ethereum transaction network and Ethereum stablecoin networks during three events: the LunaTerra collapse, Ethereum PoS (proof-of-stake) phase 1, and Ethereum PoS phase 2.

\noindent\textbf{Ethereum Stablecoin Networks}. We retrieve transaction data for the top five stablecoins based on market capitalization (USDT, USDC, DAI, UST, PAX) and WLUNA from the Chartalist repository~\cite{Chartalist2022}. The data pertains only to transactions conducted on the Ethereum blockchain; each transaction in the data set corresponds to a transfer of the asset indicated by the contract address.  However, the UST collapse event that we are studying involved another blockchain called Terra with its own network, and the cryptocurrency called Luna, acting as a parallel to ether on Ethereum. Terra also issued a stablecoin named TerraUSD, which offered high-interest rates to lenders. Additionally, Terra's owners created another stablecoin called UST on the Ethereum blockchain, which was pegged to the value of USD\$1. Furthermore, a Wrapped LUNA (WLUNA) token was established to trade Luna tokens on Ethereum.

In May 2022, the Terra blockchain and its cryptocurrency Luna collapsed, owing to TerraUSD loans that could not be repaid. A Luna coin that was valued at \$USD116 in April plummeted to a fraction of a penny during the collapse. This resulted in a loss of confidence in both WLUNA and UST on Ethereum. On May 9th, 2022, UST lost its \$USD1 peg and fell as low as 35 cents.  The dataset covers the period from April 1st, 2022, to November 1st, 2022, spanning about one month before the crash to six months after the crash. In addition to the transactions, we also use the address labels dataset from~\cite{Chartalist2022} where labels of 296 addresses from 149 centralized and decentralized Ethereum exchange addresses are listed publicly.

\noindent\textbf{Ethereum Transaction Network}. We collected ether transactions from the Ethereum blockchain for the period between August 23rd and September 29th, 2022. On an average day during this period, there were 480,000 addresses, with approximately 1 million edges connecting them. Ether is a type of cryptocurrency, similar to bitcoin, and its value can be converted to various fiat currencies such as USD and JPY. The nodes on the graph could represent investors who anticipate future price increases for ether, or traders who engage in buying and selling goods and services. Ethereum changed its block creation process during this time, moving from the costly Proof-of-Work method to the more efficient Proof-of-Stake algorithm in two phases on September 9th and 15th, 2022.  

 \subsubsection{Parameters}
In the context of {\sf InnerCore} expansion and decay,
a greater $i$ (i.e., the history parameter from \S3.2) produces an averaging effect, coupled  with the tendency to depress expansion rates and inflate decay rates.  Setting a specific $i$ value depends on the application; in general, higher $i$ reduces fluctuations between expansion and decay rates of each day.  In our applications, we use $i$ = 1 to improve the accentuation of expansion and decay in the {\sf InnerCore} to better depict the shift in market sentiment during the days of significant events in our case studies.  

In {\sf AlphaCore} decomposition, depth values range between $(0, 1]$; nodes with high property values (e.g., many transactions, higher transacted amounts) tend to have low depth, while nodes with low property values tend to have high depth~\cite{victor2021alphacore}. With data depth threshold $\epsilon=1$, all nodes will be returned as InnerCore members; while for $\epsilon=0$, the empty set will be returned.  
Setting an appropriate $\epsilon$ depends on the desired size of the {\sf InnerCore} returned specific to an application.  In our applications, we set $\epsilon=0.1$ to ensure that the average number of nodes in each daily {\sf InnerCore} is above 150.

 \subsection{Scalability Analysis}

\noindent\textbf{System Specifications}.
The machine used for experiments is an Intel Core i7-8700K CPU @ 3.70GHz processor, 32.0GB RAM, Windows10 OS, and GeForce GTX1070 GPU.  A combination of Python and R was used for coding.

%
\begin{figure}
  \centering      \includegraphics[width=0.34\textwidth]{figs/time}
  \caption{Comparison between the running times of {\sf AlphaCore} with the starting $\epsilon=1.0$ and stepsize $s=0.1$ and {\sf InnerCore} with $\epsilon=0.1$ on daily Ethereum transaction networks to return the {\sf InnerCore} of depth < 0.1.  An average of approximately 480,000 nodes (addresses) and 1 million edges (transactions) exist in each network. The average computation time for {\sf InnerCore} is 4.06 seconds (max 8.1s), which is approximately 0.10485 times the average computation time of {\sf AlphaCore}.  
  }
  \label{fig:time}
  \vspace{-12px}
\end{figure}

%\medskip
\noindent\textbf{InnerCore Discovery}.
Since we are interested in directly finding the 
{\sf InnerCore}, 
compared to {\sf AlphaCore} decomposition~\cite{victor2021alphacore}, our {\sf InnerCore} discovery method (\S 3.1) does not associate different $\epsilon$ values to intermediate cores generated in an iterative stepwise fashion.  Instead, a fixed threshold $\epsilon$, or upper bound for depth, is set and all nodes with a depth greater than $\epsilon$ are pruned repetitively until all remaining nodes relative to each other in the resulting network have a depth < $\epsilon$.  Effectively, this allows {\sf InnerCore} discovery to run approximately 1/stepsize times faster than {\sf AlphaCore} decomposition, since the computations of all intermediate cores are skipped. As depicted in Figure  \ref{fig:time},  {\sf InnerCore}
discovery has a running time of only one-tenth of that for {\sf AlphaCore} decomposition. Furthermore, the average running time for {\sf InnerCore} discovery is only 4.06 seconds on graphs with approximately 480,000
nodes and 1 million edges, which demonstrates the scalability of our approach. 

%\medskip
\noindent\textbf{Three-Node Motifs Counting}.  
Instead of conducting motif analysis on all nodes, our approach utilizes the {\sf InnerCore}. By focusing on this core subset of nodes, we are able to reduce the number of nodes in a daily network consisting of approximately 480,000 nodes and 1 million edges to an induced subgraph of roughly 300 nodes and 90,000 edges (counting multi-edges), resulting in a more manageable and efficient approach. Although motif counting on each snapshot graph takes $>$ 1 day to complete, motif counting inside {\sf InnerCore}  significantly improves the processing speed, requiring only $<$ 1 sec to complete, which illustrates our scalability.
%  
\subsection{Case 1: The Collapse of LunaTerra}
Stablecoins are meant to be a safe house as they are generally pegged to and maintain a 1:1 ratio with a fiat currency, resisting the volatility associated with other popular cryptocurrencies. Commonly, investors
keep blockchain assets not needed for immediate use in a transaction as a stablecoin, analogous to people keeping extra money in a bank.  For this reason, the LunaTerra collapse was a historic event in the decentralized financial space as it questioned investors' trust in cryptocurrencies; if even stablecoins are susceptible to collapse, then is any cryptocurrency truly safe?  

First, we analyze this event from the perspective of investors' market sentiment in the stablecoin network.  Specifically, we examine behavioral patterns in the expansion and decay measures of the temporal stablecoin network of the days surrounding the collapse.  From Figure~\ref{fig:lunaDecayExapansion}, we observe that four days after the collapse unfolded, on May 13, 2022, there is a substantial increase in decay and a decrease in expansion: a prime indicator of the {\em despair} behavioral pattern (\S 3.3).  We can infer from this signal that a large majority of regular investors stopped trading by this time, either from the conversion or sale of any assets stored as UST out of the stablecoin ecosystem, or simply due to uncertainty and inaction in response to the collapse.  Following this cue, for approximately two weeks afterward, we see a consistent behavioral pattern of {\em faith} characterized by low expansion and low decay.  During this period, few new investors entered or left the stablecoin network.  There was still faith in the remaining investors that perhaps a large stablecoin such as UST could rebound and restore its peg with USD and thus, they refrained from engaging in any transactions.  On the other hand, decay and expansion values also indicate a sign of hopelessness as bulk of investors already exited the network since the first signal of despair.  We understand from this behavioral analysis that there is a delayed reaction from investors when a significant unannounced event occurs due to indecision, and there is a general trend of inactivity in the following period.
 
\begin{figure}
  \centering      \includegraphics[width=0.331\textwidth]{figs/visuals}
  \caption{LunaTerra decay and expansion measures. On May 9 (shown with the vertical blue line), UST loses its \$1 peg and falls to as low as 35 cents.}
  \label{fig:lunaDecayExapansion}
\end{figure}
%
\begin{figure}
  \centering      \includegraphics[width=0.331\textwidth]{figs/lunacores}
  \caption{Sizes of stablecoin network cores during LunaTerra collapse. {\sf Innercore}s accentuate the collapse better than nodes found in the highest cores via graph-$k$-core decomposition.  
  }
  \label{fig:lunaTerraCoreSize}
\end{figure}
 %
\begin{table}[t]
  \caption{Numbers of center addresses in motifs identified by our method (\S 3.4) that are known exchanges. The numbers represent total counts per motif across all days.}
  \label{tab:exchanges}
  \small
  \begin{tabular}{ccc}
    %\toprule
     & \# Unique Addresses & \# Exchange Addresses\\
    \midrule
    Motif 1 & 1221 & 15\\
    Motif 4 & 1762 & 15\\
    Motif 5 & 1447 & 17\\
    Motif 6 & 1513 & 4\\
    Motif 11 & 939 & 11\\
  %\bottomrule
\end{tabular}
\end{table}
%
\begin{table}[t]
  \caption{{\sf NF-IAF} score-based percentile ranks 
  of InnerCore motif centers matching highlighted addresses by Nansen.ai that played key roles on May 8, 2022 in LunaTerra collapse. }
  \label{tab:nansenAddresses}
  \small
  \begin{tabular}{ccccccc}
    %\toprule
    Address/Motif Center & $C_1$ & $C_4$ & $C_{5a}$ & $C_{5b}$ & $C_6$ & $C_{11}$\\
    \midrule
    {\sf Celsius}             & -       & 81.2  & -     & 78.9  & -     & - \\
    {\sf hs0327.eth}          & 88.2    & 67.0  & 96.2  & 69.7  & 81.6  & - \\
    {\sf Smart LP: 0x413}     & -       & 68.4  & -     & -     & 95.1  & - \\
    {\sf Token Millionaire 1} & 84.6    & 89.7  & -     & 86.2  & 74.2  & 88.9  \\
    {\sf Token Millionaire 2} & 69.7    & 99.5  & -     & 98.7  & 98.9  & 37.6  \\
    {\sf masknft.eth}         & 90.9    & 90.7  & -     & 82.1  & 93.3  & 92.2  \\
    {\sf Heavy Dex Trader}    & 71.3    & 96.2  & -     & -     & 81.2  & - \\
    {\sf Oapital}             & 91.6    & 78.9  & 60.5  & 58.5  & 71.8  & 92.5 \\
    {\sf Hodlnaut}            & 39.9    & 98.9  & -     & 90.6  & 99.4  & - \\    
  %\bottomrule
\end{tabular}
\end{table}


 
Compared to traditional graph-$k$-core decomposition where the target is the highest $k$-core, {\sf AlphaCore} with a fixed core target of $\epsilon$ = 0.1 (i.e., $\alpha$ = 1 - 0.1 = 0.9) better accentuates the LunaTerra collapse when applied to temporal networks and comparing the sizes of the target cores. 
From Figure~\ref{fig:lunaTerraCoreSize}, it is evident that the size of the highest graph-$k$-core fluctuates randomly across each day and does not respond to changes in the health of the network. 
This can be attributed to the fact that graph-$k$-core decomposition only considers the degree of nodes irrespective of the edge weights.  Conversely, {\sf AlphaCore} considers all the provided node features (\S3 and Table~\ref{tab:node_property_functions}) into consideration when computing each node's depth, which includes edge weights in addition to node degree. Therefore, if a node is incident to a single edge of extremely high edge weight, the highest graph-$k$-core will exclude this node; whereas {\sf AlphaCore} will, depending on the target $\epsilon$, maintain it in its {\sf InnerCore} as the high edge weight offsets the low node degree.  

We next focus on the unknown addresses that occurred most frequently as motif centers in {\sf InnerCore}s (defined in \S 3.4) on days immediately before the Luna Terra crash, since they could have influenced the initial phase of the crash.

Generally, a large amount of tokens transferred from one address to another is easily detectable due to the sheer volume.  However, if an investor tries to confiscate detection, the investor could produce multiple transactions with smaller volumes.  Additionally, often in a transaction where one token is exchanged for another, a series of multiple transfers can arise for a single conversion transaction due to interactions with exchanges.\footnote{{\tiny \url{https://etherscan.io/tx/0xa3663b813b2c13a88daeeb5b48b32b7024fc07cbf250f2c2a9318ec1950c9da9}}}
Therefore, an investor is more likely to create transfers to various receiving addresses and receive transfers from various sending addresses, making the investor a prime candidate as a 3-node motif center.

To capture the transactions of investors converting between different stablecoins, we have included four stablecoins in our network along with UST. Specifically, before the LunaTerra collapse, it is reasonable to assume that investors responsible for the collapse would prepare for the anticipated negative consequences by exiting the UST network and entering another reliable stablecoin.

In a conversion transaction, participants sell their current assets to acquire different assets. Therefore, we anticipate that the investors responsible for the LunaTerra collapse will be highly relevant motif centers exhibiting both selling and buying behaviors.

\noindent\textbf{Ground Truth}. Nansen (\url{https://www.nansen.ai/}) is a prominent blockchain analytics platform that frequently publishes comprehensive analyses of blockchain events, which are followed with great interest by the industry. Six forensics experts from Nansen.ai conducted a thorough analysis of the LunaTerra collapse in May 2022 and identified 11 important addresses that played central roles in the collapse~\cite{NansenLunaTerra}. We aim to compare the addresses of interest detected by our {\sf InnerCore} analysis using the centered-motif approach with those identified by Nansen.ai (Table~\ref{tab:nansenAddresses}) as the primary candidates for triggering the collapse.

Exchanges are an intermediary hub to facilitate transfers between investors.  The addresses of exchanges are well-known for this reason, making them not very interesting in our context.  In contrast, addresses that are not exchanges are mostly owned by investors and thus, the existence of such addresses and their edges in a network is a direct consequence of an investor’s activity in the network.  From Table \ref{tab:exchanges}, we observe that motif centers identified from {\sf InnerCore}s have a high ratio of non-exchange addresses to exchange addresses ($\approx$99\%).  This shows the effectiveness of our method to identify potentially meaningful addresses in a network different from high-traffic exchange addresses.  

In particular, we capture 9 of 11 externally owned addresses ({\sf EoA}s) in Table \ref{tab:nansenAddresses} identified by Nansen.ai that occurred as center addresses for our motif types (Figure \ref{fig:motifs}) on days immediately leading up to the LunaTerra collapse.  We notice that the {\sf NF-IAF} score-based percentile ranks of these addresses are higher compared to that of other center addresses for the same motif type on the same day, indicating that these addresses were important investors contributing to the buy or sell behavior associated with the motif on the day. We surmise the possibility that certain {\sf EoA}s found by our {\sf InnerCore} method, coupled with centered-motif analysis, could have been responsible for the initial phase of the collapse.

Recall that in Figure \ref{fig:motifs}, we defined motif centers $C_1$, $C_{5a}$, and $C_{11}$ as exhibiting sell behavior; while motif centers $C_4$, $C_{5b}$, and $C_6$ as exhibiting buy behavior.  It is evident from Table \ref{tab:nansenAddresses} that every motif center on May 8, 2022 has at least one corresponding investor with an {\sf NF-IAF} score-based percentile rank above 91.  This suggests that addresses with greater {\sf NF-IAF} percentiles exhibit a higher buy or sell behavior associated with the particular motif type on the day before the collapse.  
Specifically, we identify the two investors, {\sf masknft.eth} and {\sf Oapital}, as the most likely candidates for influencing the initial phase of the crash, since they are the two addresses with greater {\sf NF-IAF} percentiles (above 90) occurring consistently across at least two motif types exhibiting sell behavior.

\noindent\textbf{K-Core vs. InnerCore}. We compare the  addresses identified by {\sf InnerCore} + centered-motif analysis with those in the highest graph-$k$-core on the May 8 (i.e., one day before the crash) stablecoin temporal network. We find that graph-$k$-core cannot identify any of the 11  addresses indicated by Nansen.ai as prime candidates for triggering the initial phase of the LunaTerra collapse.  In comparison, {\sf InnerCore} + centered-motif analysis captures potentially anomalous buy and sell behaviors
by identifying 9 of the 11 addresses.   
 
\subsection{Case 2: Ethereum's Switch to Proof-of-Stake}
Ethereum's transition from Proof-of-Work (PoW) to Proof-of-Stake (PoS) came with many benefits including enhanced security for users and lower energy consumption.  Together, these positives incentivized new investors to participate in the Ethereum network due to increased trust in the blockchain and lower barriers to entry.  The transition occurred in two phases; the first phase was a preparatory hard forking of the blockchain into a PoS structure and the second phase was a finalization of the upgrade.  A pattern of {\em hope} was expected as the upgrade was highly anticipated due to the positives, transparency, and consistent updates regarding the official dates of the upgrade.  
From Figure~\ref{fig:ethDecayExpansion}, we indeed verify this behavioral pattern of {\em hope} characterized by inflated expansion values, coupled with relatively stable decay values, on three separate occasions.  The first occurrence of {\em hope} is observed approximately a week before the first phase of the upgrade took place.  It was around this time, the end of August 2022, that official news regarding the concrete dates of when the upgrade would be expected to take place was released to the public.  We observe a surge of new hopeful investors participating in the Ethereum network and a significant dip in existing investors leaving the network in  
 anticipation of the upgrade.  The other two instances of {\em hope} are seen during the immediate days surrounding and between each of the phases of the upgrade.  These occurrences provide insight into the market sentiment during the upgrade as positive and the overall transition of Ethereum to PoS as being well-received by investors.
% 
  \begin{figure}
  \centering
\includegraphics[width=0.34\textwidth]{figs/eth}
  \caption{Ethereum decay and expansion measures. The move of Ethereum to Proof-of-Stake mining took place in two stages, indicated by 2 vertical blue lines (Sep 6 and 15, 2022).  }
  \label{fig:ethDecayExpansion}
\end{figure}


 

 