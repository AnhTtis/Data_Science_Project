\section{Appendix}
%
\subsection{An Example of AlphaCore}
To better illustrate the differences between the traditional graph-$k$-core and {\sf AlphaCore} decomposition methods, we showcase an example in Figure \ref{fig:running_example}.  We are interested in the inner core that captures only the most important investors of a financial network.  In the case of graph-$k$-core, the innermost core is the 3-core, whereas the inner core of {\sf AlphaCore} would be the core of $\alpha$ > 0.75.  Note that the 3-core consists of investors that trade frequently with themselves, but their trade volumes with themselves are not that significant compared to other transactions which exist in the network.  In certain analyses of financial networks such as anomalous address detection, being able to filter out these negligible transactions and their participating investors, while still capturing more meaningful ones, significantly improves the accuracy and scalability of subsequent computations on the decomposed network core.  On the other hand, the {\sf AlphaCore} of $\alpha$ > 0.75 is able to capture both the investors that participate in the largest transactions which occur in the example network, while filtering the negligible transactions and their participating investors.  
We point out that the main limitation with graph-$k$-core is that it only considers node degrees, whereas {\sf AlphaCore} is flexible and can consider any combination of node features as outlined in Table 1, without requiring to specify any feature weighting parameters to perform effectively on a particular task.  
Therefore, in networks where edge weights fall under a broad range and they are meaningful distinguishing factors, we recommend {\sf AlphaCore} over the traditional graph-$k$-core decomposition.  

\subsection{Additional Related Work}
%
In recent years, several studies focused on analyzing different aspects of the blockchain networks \cite{ChenWZCZ19,AkcoraLGK20,kalodner2017blocksci,GuidiM20}, particularly in the Ethereum network. 
Researchers working on natural
language processing and sentiment analysis using tweets, news articles, cryptocurrency prices and charts, Google Trends about blockchains \cite{VNO19,KS20} could find supporting evidences based on
blockchain data analysis. 
Oliveira et al. \cite{oliveira2022analysis} performed an analysis of the effects of external events on the Ethereum  platform, highlighting short-term changes in the behavior of accounts and transactions on the network. Aspembitova et al. \cite{aspembitova2021behavioral} used temporal complex network analysis to determine the properties of users in the Bitcoin and Ethereum markets and developed a methodology to derive behavioral types of users.

Other studies focused on specific aspects of the Ethereum network. For instance, Casale et al. \cite{casale2021networks} analyzed the networks of Ethereum Non-Fungible Tokens using a graph-based approach, while Silva et al. \cite{silva2020characterizing} characterized relationships between primary miners in Ethereum using on-chain transactions. Meanwhile, Victor et al.  \cite{victor2019measuring} measured Ethereum-based ERC20 token networks and Kiffer et al. \cite{kiffer2018analyzing} examined how contracts in Ethereum are created and how users interact with them.

Zhao et al. \cite{zhao2021temporal} investigated the evolutionary nature of Ethereum interaction networks from a temporal graphs perspective, detecting anomalies based on temporal changes in global network properties and forecasting the survival of network communities using relevant graph features and machine learning models. Li et al. \cite{li2021measuring} analyzed the magnitude of illicit activities in the Ethereum ecosystem using proprietary labeling data and machine learning techniques to identify additional malicious addresses. Kilic et al. \cite{kilicc2022fraud} predicted whether given addresses are blacklisted or not in the Ethereum network using a transaction graph and local and global features. 

Our approach for analyzing the effects of external events on a blockchain platform is similar to the one used by Anoaica et al. \cite{anoaica2018quantitative}. The authors examined the temporal variation of transaction features in the Ethereum network and observed an increase in activity following the announcement of the Ethereum Alliance creation. Gaviao et al.  \cite{gaviao2020transaction} also studied the evolution of users and transactions over time, showing the centralization tendency of the transaction network. Kapengut et al.\cite{kapengut2022event} study the Ethereum blockchain around the BeaconChain phase of the PoS transition (September 15, 2022), but the authors focus on the power efficiency and miners' rewards around the transition.


Finally, Khan \cite{khan2022graph} conducted a survey of datasets, methods, and future work related to graph analysis of the Ethereum blockchain data, while Ramezan Poursafaei's PhD thesis \cite{ramezan2022anomaly} presented results on temporal anomaly detection in blockchain networks.

\begin{figure}[t]
    \centering \includegraphics[width=\linewidth]{figs/running_example.png}
    \caption{A running example to compare between the graph-$k$-core and {\sf AlphaCore} decomposition methods. The Coreness of nodes according to graph-$k$-core decomposition is shown with different node colors, whereas {\sf AlphaCore} is run with in-strength and out-strength as node features with a step size of 0.25. Different {\sf AlphaCore}s are shown using dotted boundaries. \label{fig:running_example}}
    
\end{figure}