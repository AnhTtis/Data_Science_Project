\section{Introduction}
Blockchain technology has been gaining increasing popularity over the past few years, revolutionizing the way we store and transfer digital assets~\cite{nakamoto2008bitcoin,wood2014ethereum}. Public blockchain networks are 
 also completely open, where anonymous addresses use transactions to move cryptocurrencies and trade/invest in assets. While the technology offers numerous benefits, it also poses significant challenges, particularly in the area of cybersecurity. Blockchains enable electronic crimes in a variety of ways \cite{abs-2212-13452}, ranging from demands for ransomware~\cite{huang2018tracking} to transactions in darknet markets~\cite{jiang2021illicit}.

One of the biggest challenges in securing blockchain networks is detecting and preventing e-crimes. E-crimes detection requires scalable analysis of large-scale blockchain graphs in real-time, where results are both qualified and manageable by human analysts. To address this challenge, researchers have developed tools and algorithms for analyzing blockchain networks~\cite{victor2021alphacore,akcora2017chainlet,abayChainnet18,SuG022,KA22}.

Unfortunately, analyzing blockchain networks is an arduous task, given their large size and the anonymity of actors involved. It is crucial to devise scalable and effective methods that can analyze blockchain networks in real-time, to avoid future losses. The failure to conduct timely analysis of blockchain networks has already resulted in a staggering loss of over billions of dollars to blockchain users, as evident by the recent collapse of LunaTerra~\cite{lunaterra}. 

In this article, we introduce a new approach to e-crimes and trends detection. Our approach, {\sf InnerCore}, involves identifying influential addresses with data depth-based core decomposition and further filtering out the role of addresses by using a centered-motif approach. {\sf InnerCore} analysis can reduce large graphs having more than 400K nodes and 1M edges to an induced subgraph of less than 300 nodes and 90K edges, while still being able to detect the most influential nodes.
{\sf InnerCore} is unsupervised and highly scalable, yielding only $\sim$4-second  running times on daily Ethereum graphs with  $\sim$500K nodes and $>$1M edges. We apply {\sf InnerCore} to two recent influential events in the blockchain world: the collapse of LunaTerra in May 2022 and the Proof-of-Stake (PoS) switch of Ethereum in September 2022. Empirical study demonstrates that our proposed approach effectively detects changes within the network without the need for human intervention.

Additionally, our method excels at efficiently identifying the most influential addresses within the network with a high level of precision, highlighting its utility in identifying key actors in blockchain-based networks.

\smallskip

Our key novelties are summarized below.
\begin{itemize}[leftmargin=.1in]
 \item \textit{InnerCore}: We propose {\sf InnerCore}, a data depth-based core discovery method that can identify the most influential investors and traders in blockchain-based asset networks (\S\ref{sec:methinnercore}).
 \item \textit{Explainable behavior}: We develop two metrics, {\sf InnerCore} expansion and decay (\S\ref{sec:methdecayexpansion}), that can provide a sentiment indicator for the networks and explain investor mood (\S\ref{sec:methpatterns}). 
 \item \textit{Unsupervised address discovery}: Through conducting node ranking with a centered-motif approach in temporal asset networks, we demonstrate that {\sf InnerCore} tracking can detect e-crime behavior and warn the network about possible long-term instability, without the need for supervised address discovery (\S\ref{sec:methmotif}).
 \item \textit{Scalability}: Due to their computational efficiency and ability to utilize only a small portion of graph nodes and edges to analyze overall behavior, the {\sf InnerCore} discovery and expansion/decay calculations are suitable for application on large temporal graphs including Ethereum transaction and stablecoin networks (\S\ref{sec:exp}).
\end{itemize}
%
We discuss preliminaries and our problem in \S\ref{sec:prelim}.
Additional related works are specified in the Appendix. %our full version~\cite{InnerCoreAppendix}.
