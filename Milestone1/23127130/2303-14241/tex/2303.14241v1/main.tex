%%
%% This is file `sample-sigconf.tex',
%% generated with the docstrip utility.
%%
%% The original source files were:
%%
%% samples.dtx  (with options: `sigconf')
%%
%% IMPORTANT NOTICE:
%%
%% For the copyright see the source file.
%%
%% Any modified versions of this file must be renamed
%% with new filenames distinct from sample-sigconf.tex.
%%
%% For distribution of the original source see the terms
%% for copying and modification in the file samples.dtx.
%%
%% This generated file may be distributed as long as the
%% original source files, as listed above, are part of the
%% same distribution. (The sources need not necessarily be
%% in the same archive or directory.)
%%
%% The first command in your LaTeX source must be the \documentclass command.
\pdfoutput=1
\documentclass[sigconf, nonacm]{acmart}


%% CUSTOM PACKAGES
\usepackage[vlined,ruled,linesnumbered]{algorithm2e}
\usepackage{subcaption}
\usepackage{csquotes}
\usepackage{multirow}
\usepackage{url}
\usepackage{enumitem}
\usepackage{color,soul}
\usepackage{amsmath,amsfonts,amsthm,color,mathtools}
\newtheorem{definition}{Definition}
\newtheorem{exam}{Example}
\usepackage{hyperref}

\newcommand{\V}{\mathcal{V}}
\newcommand{\E}{\mathcal{E}}
\newcommand{\eat}[1]{}
 
%
%% \BibTeX command to typeset BibTeX logo in the docs
% \AtBeginDocument{%
%   \providecommand\BibTeX{{%
%     \normalfont B\kern-0.5em{\scshape i\kern-0.25em b}\kern-0.8em\TeX}}}

%% Rights management information.  This information is sent to you
%% when you complete the rights form.  These commands have SAMPLE
%% values in them; it is your responsibility as an author to replace
%% the commands and values with those provided to you when you
%% complete the rights form.
\newcommand\vldbdoi{XX.XX/XXX.XX}
\newcommand\vldbpages{XXX-XXX}
% issue-specific
\newcommand\vldbvolume{16}
\newcommand\vldbissue{1}
\newcommand\vldbyear{2023}
% should be fine as it is
\newcommand\vldbauthors{\authors}
\newcommand\vldbtitle{\shorttitle}
% leave empty if no availability url should be set
\newcommand\vldbavailabilityurl{https://github.com/JZ-FSDev/InnerCore}
% whether page numbers should be shown or not, use 'plain' for review versions, 'empty' for camera ready
\newcommand\vldbpagestyle{plain}

%% These commands are for a PROCEEDINGS abstract or paper.

% \settopmatter{printacmref=true}
%%
%% end of the preamble, start of the body of the document source.
\begin{document}
% \fancyhead{}
%%
%% The "title" command has an optional parameter,
%% allowing the author to define a "short title" to be used in page headers.
\title{Core-based Trend Detection in Blockchain Networks [Scalable Data Science]}


\author{Jason Zhu}
\affiliation{%
  \institution{University of Manitoba}
  \city{Winnipeg}
  \country{Canada}}
\email{zhuj3410@myumanitoba.ca}
%\orcid{0000-0001-8329-3133}


\author{Arijit Khan}
\affiliation{%
  \institution{Aalborg University}
  \city{Aalborg}
  \country{Denmark}}
\email{arijitk@cs.aau.dk} 


\author{Cuneyt Gurcan Akcora}
\affiliation{%
 \institution{University of Manitoba}
 \city{Winnipeg}
 \country{Canada}}
\email{cuneyt.akcora@umanitoba.ca}

%%
%% By default, the full list of authors will be used in the page
%% headers. Often, this list is too long, and will overlap
%% other information printed in the page headers. This command allows
%% the author to define a more concise list
%% of authors' names for this purpose.
%\renewcommand{\shortauthors}{Zhu et al.}

%%
%% The abstract is a short summary of the work to be presented in the
%% article.
\begin{abstract}




\begin{abstract}
% \vspace{-1em}
The diffusion-based generative models have achieved remarkable success in text-based image generation. However, since it contains enormous randomness in generation progress, it is still challenging to apply such models for real-world visual content editing, especially in videos. 
In this paper, we propose \texttt{FateZero}, a zero-shot text-based editing method on real-world videos without per-prompt training or use-specific mask. 
\RM{Specifically, different from a pipeline of two independent inversion and then generation stages, we find the intermediate attention maps during inversions store better structure and motion information. We thus reform them to temporally casual attention and replace them in the generation progress. To further reduce the unnecessary semantic leakage of source video and enhance the editing quality, we then remix the temporally casual attentions via the cross-attention features of the source prompt as the mask.}
To edit videos consistently, we propose several techniques based on the pre-trained models. Firstly, in contrast to the straightforward DDIM inversion technique, our approach captures intermediate attention maps during inversion, which effectively retain both structural and motion information. These maps are directly fused in the editing process rather than generated during denoising. To further minimize semantic leakage of the source video, we then fuse self-attentions with a blending mask obtained by cross-attention features from the source prompt. Furthermore, we have implemented a reform of the self-attention mechanism in denoising UNet by introducing spatial-temporal attention to ensure frame consistency.
Yet succinct, our method is the first one to show the ability of zero-shot text-driven video style and local attribute editing from the trained text-to-image model. We also have a better zero-shot shape-aware editing ability based on the text-to-video model~\cite{tuneavideo}. \RM{Besides video, our unified method also achieves state-of-the-art performance in zero-shot image editing.\chenyang{Need exp or remove the zero-shot image}} Extensive experiments demonstrate our superior temporal consistency and editing capability than previous works.
% The code will be released.
% \chenyang{emphasize: our observation at inversion time} \xiaodong{replacing the bold part to the actual pipeline: \textbf{Specifically, we work on replacing and mixing the attention maps between the inversion and generation since the self-attention map keeps the structure of the original natural image and the cross-attention is semantic-related, after remixing, we replace them in the corresponding generation steps for denoising.}}
% \footnote{Since there is no general video diffusion model is publicly available, we use one-shot video generation method~(Tune-A-Video~\cite{tuneavideo}) as the pretrained video diffusion model for zero-shot video editing\xiaodong{can be removed if we actually zero-shot on video}.}.
\end{abstract}
\end{abstract}

%%
%% The code below is generated by the tool at http://dl.acm.org/ccs.cfm.
%% Please copy and paste the code instead of the example below.
%%
%\begin{CCSXML}
%<ccs2012>
%   <concept>
%       <concept_id>10002950.10003624.10003633</concept_id>
%       <concept_desc>Mathematics of computing~Graph theory</concept_desc>
%       <concept_significance>500</concept_significance>
%       </concept>
%   <concept>
 %      <concept_id>10002951.10003260.10003261</concept_id>
 %      <concept_desc>Information systems~Web searching and information discovery</concept_desc>
 %      <concept_significance>300</concept_significance>
 %      </concept>
% </ccs2012>
%\end{CCSXML}

%\ccsdesc[500]{Mathematics of computing~Graph theory}
%\ccsdesc[300]{Information systems~Web searching and information discovery}

%%
%% Keywords. The author(s) should pick words that accurately describe
%% the work being presented. Separate the keywords with commas.
%\keywords{Core decomposition; networks; data depth}

%% A "teaser" image appears between the author and affiliation
%% information and the body of the document, and typically spans the
%% page.
% \begin{teaserfigure}
%   \includegraphics[width=\textwidth]{sampleteaser}
%   \caption{Seattle Mariners at Spring Training, 2010.}
%   \Description{Enjoying the baseball game from the third-base
%   seats. Ichiro Suzuki preparing to bat.}
%   \label{fig:teaser}
% \end{teaserfigure}

%%
%% This command processes the author and affiliation and title
%% information and builds the first part of the formatted document.
\maketitle

\pagestyle{\vldbpagestyle}
\begingroup\small\noindent\raggedright\textbf{PVLDB Reference Format:}\\
\vldbauthors. \vldbtitle. PVLDB, \vldbvolume(\vldbissue): \vldbpages, \vldbyear.
\href{https://doi.org/\vldbdoi}{doi:\vldbdoi}
\endgroup
\begingroup
\renewcommand\thefootnote{}\footnote{\noindent
This work is licensed under the Creative Commons BY-NC-ND 4.0 International License. Visit \url{https://creativecommons.org/licenses/by-nc-nd/4.0/} to view a copy of this license. For any use beyond those covered by this license, obtain permission by emailing \href{mailto:info@vldb.org}{info@vldb.org}. Copyright is held by the owner/author(s). Publication rights licensed to the VLDB Endowment. \\
\raggedright Proceedings of the VLDB Endowment, Vol. \vldbvolume, No. \vldbissue\ %
ISSN 2150-8097. \\
\href{https://doi.org/\vldbdoi}{doi:\vldbdoi} \\
}\addtocounter{footnote}{-1}\endgroup
%%% VLDB block end %%%

%%% do not modify the following VLDB block %%
%%% VLDB block start %%%
\ifdefempty{\vldbavailabilityurl}{}{
\vspace{.3cm}
\begingroup\small\noindent\raggedright\textbf{PVLDB Artifact Availability:}\\
The source code, data, and/or other artifacts have been made available at \url{\vldbavailabilityurl}.
\endgroup
}
%%% VLDB block end %%%

\section{Introduction}

The ability to reason about plans is critical for performing long-horizon tasks \citep{erol1996hierarchical, sohn2018hierarchical, sharma-etal-2022-skill}, compositional generalization \citep{corona-etal-2021-modular} and generalization to unseen tasks and environments \citep{shridhar2020alfred}.
Consider a simple long-horizon planning scenario where a robot is tasked with preparing a meal and serving it on the table. 
This presents a non-trivial planning problem since the agent needs to understand the sequence of operations required to perform the task and search for the relevant objects in the unfamiliar environment by interacting with various objects. %



Large language models have been recently shown to possess commonsense knowledge about the world such as object affordances and physical dynamics \citep{ouyang2022training,chowdhery2022palm}.
Early approaches considered text based environments and fine-tuned PLMs to predict actions given the history of past observations and actions \citep{jansen-2020-visually,micheli-fleuret-2021-language,yao-etal-2020-keep}.
Recent work has used this ability to reason about plans from text instructions in simulated household environments with simplifying assumptions such as text-only environment observations or feedback \citep{huang2022language,ahn2022can,li2022pre,logeswaran-etal-2022-shot}.


We focus on \emph{visually grounded planning} with PLMs --- the ability to adapt plans based on interaction and visual feedback from the environment.
While PLMs have strong planning commonsense priors, predictions from a PLM may not be directly realizable in the environment since the observation and action spaces are unknown.
This requires \emph{grounding} the PLM in the environment and adapting it to observe visual feedback, which is highly non-trivial.
Some prior works assume the availability of a pre-trained affordance function \citep{ahn2022can} or a success detector \citep{mirchandani2021ella}.
Notably, SayCan \citep{ahn2022can} completely decouples the PLM from observation information by selecting actions that have both high affordability (through a pre-trained affordance model) and high PLM likelihood.
Although this partially addresses the grounding problem, the use of visual feedback for action affordance alone is limited.
Often an agent must choose one of many affordable actions using information from observations.
For example, a driving agent should re-navigate and possibly turn around when encountering a ``road closed'' sign, but both turning around and driving forward are indistinguishable to SayCan because they are both affordable and the PLM is blind to observations.

Another workaround explored in prior work is translating the information in the visual observations to text using a pre-trained captioning system \citep{shridhar2021alfworld,huang2022language}.
However, it can be difficult to faithfully describe an image in words and information is lost in this inherently noisy process, which limits the information available to the planner.



Recent work shows that PLMs can be adapted for various natural language tasks by inserting tunable embeddings or soft prompts at the input of the PLM (also called prompt tuning or prefix tuning)~\citep{li-liang-2021-prefix,lester-etal-2021-power}.
This approach also extends to multi-modal understanding tasks such as image captioning \citep{mokady2021clipcap} and VQA \citep{tsimpoukelli2021multimodal} where images are encoded as soft prompts and finetuned for the target task.
Transformer based architectures have also been successfully applied to offline Reinforcement Learning in recent work \citep{chen2021decision,janner2021offline,li2022pre,reid2022can}.

Taking inspiration from these works, we propose the simple approach of embedding visual observations (`visual prompts') and \textit{directly inserting them as PLM input embeddings}.
The visual encoder and PLM are jointly trained for the target task, an approach we call \textbf{\oursfull}~(\ours).
By teaching the PLM to use observations for planning in an end to end manner, we remove the dependency on external data such as captions and affordability information that was used in prior work.
We show that this simple approach performs better than prior PLM-based planning approaches on two embodied planning benchmarks based on ALFWorld~\citep{shridhar2021alfworld} and Virtualhome~\cite{puig2018virtualhome}.



\section{Background and Problem Formulation}
\label{sec:prelim}
We discuss preliminaries on blockchain, smart contracts, and stablecoins (\S \ref{sec:block}, \S \ref{sec:stable}), followed by one key technique {\sf AlphaCore} decomposition (\S \ref{sec:alpha}). We introduce our problem in \S \ref{sec:prob}.
%
\subsection{Blockchain and Smart Contracts}
\label{sec:block}
A blockchain is an immutable public ledger that records transactions in discrete data structures called blocks. The earliest blockchains are cryptocurrencies such as Bitcoin and Litecoin where a transaction is a transfer of coins.

The Ethereum project~\citep{wood2014ethereum} was created in July 2015 to provide smart contract functionality on a blockchain. Smart contracts are Turing complete software codes that are replicated across a blockchain network. The contracts ensure deterministic code execution that can be verified publicly. 

Smart contracts have implemented mechanisms to trade digital assets, known as tokens~\cite{victor2019measuring}. Similar to cryptocurrencies, a token is transferred publicly between accounts (addresses), and may have an associated value in fiat currency which is arbitrated by token demand and supply in the real world.
%
\subsection{Stablecoins}
\label{sec:stable}
A stablecoin is a smart contract-based asset whose price is protected against volatility by i) collateralizing the stablecoin with one or more offline real-life assets (e.g., USD, gold), ii) using a dual coin, or by iii) employing algorithmic trading mechanisms~\cite{moin2019classification}. 

In the \textit{pegged asset mechanism}, an increase in the price is countered by creating more stablecoins (i.e., coin minting) and selling them to investors at the pegged price. The \textit{dual coin mechanism} operates by having a management coin, referred to as the dual coin, to oversee a stablecoin. The investors of the dual coin participate in decision making through voting and receive benefits from the stablecoin's transactions. In the event that the stablecoin's price rises, some of the dual coin will be sold to purchase and decrease the supply of the stablecoin. Conflicting demand and supply dynamics of the two coins are assumed to stabilize the stablecoin's price.  However, investors may lose faith in the stablecoin to such a degree that they might also not buy the dual coin, however cheap it becomes.
Stablecoins that are based on \textit{algorithmic} trading do not require collateral for stability. They achieve stability through the utilization of a blockchain-based algorithm that adjusts the supply of tokens automatically in response to changes in demand.
 %
\subsection{Data Depth-Based Core Decomposition}
\label{sec:alpha}
Core decomposition \cite{Malliaros20} is a central technique used in network science to determine the significance of nodes and to find community structures in a wide range of applications such as biology~\cite{luo2009core}, social networks~\cite{al2017identification}, and visualization~\cite{zhang2012extracting}.
One of the best-known representatives of core decomposition algorithms, graph-$k$-core~\cite{seidman1983network,BatageljZ11}, finds the maximal subgraph where each node has at least $k$ neighbors in that subgraph.
Although the graph-$k$-core algorithm demonstrates high utility for analysis of graph structural properties, it does not account for important graph information such as the direction of edges, edge weights, and node features.

 
To address these limitations, multiple modifications of graph-$k$-core have been proposed to tackle task-specific graphs, e.g., graph-$k$-core in weighted and directed graphs, generalized $k$-core \cite{al2017identification,Zhou0HY0021,LiaoLJHXC22,batagelj2002Generalized,Garas_2012,GiatsidisTV11}. Different from them, {\sf AlphaCore}~\cite{victor2021alphacore} is a recent core decomposition algorithm that combines multiple node properties using the statistical methodology of data depth ~\cite{mosler2012multivariate}. The key idea of data depth is to offer a center-outward ordering of all observations by assigning a numeric score in $(0,1]$ to each data point with respect to its position within a cloud of a multivariate probability distribution. Using such a data depth function designed for directed and weighted graphs, {\sf AlphaCore} maps a node with multiple features to a single numeric score, while preserving its relative importance with respect to other nodes.   

 As the {\sf AlphaCore} decomposition unfolds, the data depth values are repeatedly updated through the calculation of node property functions and the application of data depth to the resulting values.

Consider a directed and weighted multigraph, $G(\V,\E,w)$, where $\V$ represents the set of nodes and $\E$ is a multiset of edges. The weight of each edge is designated by the weight function $w : \E \rightarrow \mathbb{R}^+$. In accordance with the generalized core definitions introduced in~\cite{batagelj2002Generalized}, a node property function can assign a real value to each node $v \in \V$, based on edge properties such as weight.  A node $v$ can be represented by its feature vector $\textbf{x}\in \mathbb{R}^d$, where $d$ features have been computed for the node $v$.


\begin{definition}[Mahalanobis depth to the origin (MhDO)]
Let $\textbf{x}\in \mathbb{R}^d$ be an observed data point, then Mahalanobis (MhD) depth of $\textbf{x}$ in respect to a $d$-variate probability distribution $F$ with mean vector $\mu_F \in \mathbb{R}^d$ and covariance matrix $\Sigma_F \in \mathbb{R}^{d\times d}$ is given by
 \begin{equation}
MhDO_F(\textbf{x})=\bigl(1+\textbf{x}^\top\Sigma^{-1}_F\textbf{x}\bigr)^{-1},
 \end{equation}
 $\Sigma_F$ is the covariance matrix of $F$. The Mahalanobis data depth to origin (MhDO) measures the degree of "outlyingness" of point $\textbf{x}$ (in this context, the node property column vector) in relation to origin $\mathbf{0}$.
 \label{def:mhdo}
 \end{definition}
The core value $\alpha$ of a node is established using a data depth threshold $ \epsilon\in [0,1]$ that is applied to remove neighboring high depth nodes iteratively. Nodes with high property values, such as large edge weights, generally have a low depth, while nodes with low property values often have a high depth, such as most blockchain nodes that trade small amounts of tokens. However, node property values are not the only factor that determines depth; the community structure around the node also plays a role. Nodes are considered to be in the $\alpha = (1-\epsilon)$-core if their depth, relative to themselves, is no more than $\epsilon$.


\noindent\textbf{Why Data Depth?} Data depth provides a more precise identification of crucial nodes compared to state-of-the-art core decomposition algorithms and acts as a combination of centrality measure and core decomposition~\cite{victor2021alphacore}. Unlike traditional decomposition algorithms, a depth-based decomposition does not require the specification of multiple feature weighting parameters to perform effectively on a particular task. We provide a running example in the Appendix. %~\cite{InnerCoreAppendix}.

\subsection{Problem Definition}
\label{sec:prob}
Given a snapshot of weighted, directed, multi-graphs over successive timestamps, where $G_t(\V_t,\E_t, w_t)$ denotes the graph at timestamp $t$, $\V_t$ its set of nodes (i.e., investors/traders), and $\E_t$ its multiset of edges (i.e., transactions) representing the transfer of underlying asset, we aim to {\bf (i)} locate the most important node set $S_t \subseteq \V_t$ at time $t$ such that the behavior of nodes in $S_t$ can be used to characterize the future success of the underlying asset at $t^\prime>t$, and {\bf (ii)} categorize investors' behavior in terms of the future health and success of the underlying asset. 

To resolve above problems, we identify nodes in the innermost {\sf AlphaCore}, as well as characterize three-node motifs in these innermost cores from our transaction networks.
In our experiments, we demonstrate that nodes in the innermost {\sf AlphaCore} are more useful, compared to other notions of important nodes, in characterizing and predicting the future success of the blockchain assets. 
 

 

 

 
%\section{Data Depth}
\label{sec:depth}
Depth functions have been initially introduced in the setting of non-parametric multivariate analysis to define affine invariant versions of median, quantiles, and ranks in higher dimensional spaces where there is no natural order  (see historical overviews by~\citet{mosler2012multivariate,Nieto-Reyes:Battey:2015}).  The key idea of the depth approach is to offer a center-outward ordering of all observations by assigning a numeric score in $[0,1]$ range to each data point with respect to its position within a cloud of multivariate or functional observations or a probability distribution. Nowadays, data depth is a rapidly developing field that gains increasing momentum due to the wide applicability of depth concepts to classification, visualization, high dimensional and functional data analysis~\cite{Hyndman:Shang:2010, Narisetty:Nair:2016, mozharovskyi2020nonparametric, sguera2020notion, zhang2021depth}.
Most recently, depth approaches have found novel applications in density-based clustering and space-time data mining~\cite{Jeong:etal:2016, HuangGel2017, vinue2020robust}, shape recognition and uncertainty quantification in computer graphics~\cite{Whitaker:etal:2013, sheharyar2019visual}, ordinal data analysis~\cite{Kleindessner:vonLuxburg:2017} and computational geometry for privacy-preserving data analysis~\cite{mahdikhani2020achieve}.
Nevertheless, data depth is yet a largely unexplored concept in network sciences~\cite{Fraiman:et:2015, raj2017path,Tian:Gel:2017, tian2019fusing}.


\begin{table}
\fontsize{9}{12}\selectfont
\centering
\caption{Example node property functions. Most functions are adopted from the related work. Functions that encode the community around a node, such as cycles, can help bringing a higher ordered structure of the network into use.}
\label{tab:node_property_functions}
\begin{tabular}{ll}
\hline
Function & Value / number of ... \\
\hline
$N(u)$ & neighbors of $u$ \\
$N_{out}(u)$ & neighbors reachable with outgoing edges from $u$ \\
$N_{in}(u)$ & neighbors reachable with incoming edges to $u$ \\
$deg(u)$ & edges to/from $u$ (Degree) \\
$deg_{out}(u)$ & outgoing edges from $u$ (Out-Degree) \\
$deg_{in}(u)$ & incoming edges to $u$ (In-Degree) \\
$\bigcirc(u, l)$ & undirected cycles of length $l$ that $u$ is part of \\
$\circlearrowright(u, l)$ & directed cycles of length $l$ that $u$ is part of \\
$t(u, l)$ & length $l$ timeframes that $u$ has edges in \\
$S(u)$ & sum of edge weights incident to a node (Strength)\\
$S_{out}(u)$ & sum of outgoing edge weights (Out-Strength)\\
$S_{in}(u)$ & sum of incoming edge weights (In-Strength)\\
\hline
\end{tabular}
\end{table}

\begin{definition}[Data Depth] Formally, let $E$ be a Banach space (e.g., $E=\mathbb{R}^d$), $\mathcal B$ its Borel sets in $E$, and $\mathcal P$ be a set of probability distributions on $\mathcal B$. We view $\mathcal P$ as the class of empirical distributions giving equal probabilities $1/n$ to $n$ data points in $E$. Then a data depth function is a function $\mathbb{D}: E\times \mathcal P \longrightarrow [0,1]$, $(x,P) \longrightarrow \mathbb{D}(x|P)$, $x\in E, P\in \mathcal P$ that shall satisfy the following desirable properties: \textit{affine invariant}, \textit{upper semi-continuous} in $x$,
\textit{quasiconcave} in $x$ (i.e., having convex upper level sets) and \textit{vanishing as} $||x||\to \infty$. Specifically, a data depth   function $\mathbb{D}(x)$ measures how closely an observed point $x \in \mathbb{R}^d$, $d\geq 1$, is located to the center of a finite set $\mathcal{X}\in \mathbb{R}^d$, or relative to $F$, which is a probability distribution in $\mathbb{R}^d$. In complex network analysis, these points may correspond to the features of nodes or edges.
\end{definition}



Among many depth functions formulated to date, the Mahalanobis depth is one of the most prominent in the current practice.

\begin{definition}[Mahalanobis (MhD) depth]
Let $x\in \mathbb{R}^d$ be an observed data point, then Mahalanobis (MhD) depth of $x$ in respect to a $d$-variate probability distribution $F$ with mean vector $\mu_F \in \mathbb{R}^d$ and covariance matrix $\Sigma_F \in \mathbb{R}^{d\times d}$ is given by
 \begin{equation}
MhD_{\mu_F}(x)=\bigl(1+(x-\mu_F)^\top\Sigma^{-1}_F(x-\mu_F)\bigr)^{-1}.
 \end{equation}
Here $^\top$ denotes matrix transpose. The MhD depth measures the \textit{outlyingness} of the point with respect to the deepest point of the distribution (here $\mu_F$), and allows to easily handle the elliptical family of distributions, including a Gaussian case.
\end{definition}
MhD offers flexibility in changing the reference point with respect to which we compute data rankings. For instance, instead of $\mu_F$ we can select an arbitrary point $x_0\in \mathbb{R}^d$ and compute MhD in respect to this new reference point $x_0$
\begin{equation}
\label{MhD_arb}
MhD_{x_0}(x)=\bigl(1+(x-x_0)^\top\Sigma^{-1}_F(x-x_0)\bigr)^{-1}.
 \end{equation}
Furthermore, $\Sigma_F$ can be substituted by any empirical estimator of covariance matrix $\hat{\Sigma}$ obtained from the observed data sample $x_1, x_2, \ldots, x_n$.

\section{Methodology}
\label{sec:method}
In keeping with the daily routine of daily life, blockchain networks are frequently examined on a daily basis~\cite{casale2021networks,chen2020understanding}. We divide a blockchain network into daily intervals, using a reference time zone to create a set of snapshot graphs. In a snapshot graph of a blockchain network, a node represents a trader/investor, whereas an edge denotes a financial transaction. Next, we define {\sf InnerCore}, {\sf InnerCore} expand, and {\sf InnerCore} decay on the snapshot graphs. 

We use in-degree, out-degree, in-strength, and out-strength as node properties (defined in Table \ref{tab:node_property_functions}) to compute the {\sf AlphaCore}  decomposition of a snapshot graph, as these node features can be defined easily for a weighted, directed, multi-graph. 
Core decomposition helps us eliminate unimportant edges and nodes (e.g.,  addresses trading small amounts). Using the results of the core decomposition, we then identify an {\sf InnerCore} of nodes, which helps us pinpoint the most influential nodes. 

\begin{table}
\fontsize{9}{12}\selectfont
\centering
\footnotesize
\caption{Example node property functions.}
\label{tab:node_property_functions}
\begin{tabular}{ll}
%\toprule
Function & Definition\\
\midrule
$N(v)$ & neighbors of $v$ \\
$N_{out}(v)$ & neighbors reachable with outgoing edges from $v$ \\
$N_{in}(v)$ & neighbors reachable with incoming edges to $v$ \\
$deg(v)$ & edges to/from $v$ (Degree) \\
$deg_{out}(v)$ & outgoing edges from $v$ (Out-Degree) \\
$deg_{in}(v)$ & incoming edges to $v$ (In-Degree) \\
$S(v)$ & sum of edge weights incident to a node (Strength)\\
$S_{out}(v)$ & sum of outgoing edge weights (Out-Strength)\\
$S_{in}(v)$ & sum of incoming edge weights (In-Strength)\\
%\bottomrule
\end{tabular}
\end{table}
%
\subsection{InnerCore of a Graph}
\label{sec:methinnercore}
 %
 The data depth of a node $v \in V$ is defined as the degree of "outlyingness" of the node property function in relation to the origin $\mathbf{0}$. We define the {\sf InnerCore} of $G$ as the set of nodes $\V^{inner}$ whose data depth, relative to themselves, is less than an $\epsilon$ value. We set $\epsilon$ to a small value, and iteratively recompute the depth of each node as we remove nodes whose data depth is greater than $\epsilon$ in each iteration. This process continues until no more nodes can be removed. The resulting set of nodes is the {\sf InnerCore} of the graph. The {\sf InnerCore} computation is illustrated in Algorithm~\ref{alg:innercore}.%
\SetKwInput{KwInput}{Input}
\SetKwInput{KwOutput}{Output}
\SetKwRepeat{Do}{do}{while}
%
\begin{algorithm}[tb!]
\footnotesize
\KwInput{Directed, weighted, multigraph $G(V,E,w)$,\\
Set of node property functions $p_1, ..., p_n \in P$,\\%,
Data depth threshold $\epsilon$}
\KwOutput{innerCore $V^{inner}$}
\tcp{Compute feature matrix}
$F = [f_1, ..., f_n] = \forall p_i \in P: f_i = p_i(v, G), \forall v \in V$\label{alg:line1}\;% \tcp*{initial feature matrix}
$\Sigma_F^{-1}$ = cov$(F)^{-1}$\tcp*{compute only once}\label{alg:line2}
\tcp{Compute initial depth values}
$z = [z_1, ..., z_n] = \forall v_i \in V: z_i = [1+(F_{i,*})'\Sigma^{-1}_F(F_{i,*})]^{-1}$\label{alg:line3}\;
  \Do{$\exists z_i: (z_i \geq \epsilon) \wedge (v_i \in V)$ \tcp{one iteration}}{
    \ForEach{$z_i \geq \epsilon$}{
    $\V = \V \setminus \{v_i\}$\label{alg:line14}\;
    }
    \tcp{recompute node properties}
    $F = \forall p_i \in P: p_i(v, G), \forall v \in V$\label{alg:line16}\;
    \tcp{recompute depth}
    $z_i = [1+(F_{i,*})'\Sigma^{-1}_F(F_{i,*})]^{-1}, \forall v_i \in V$\label{alg:line17}\;
  }
\KwRet{$\V$ \tcp{as innerCore $V^{inner}$}}\label{alg:line21}
\caption{{\sf InnerCore} Discovery}
\label{alg:innercore}
\end{algorithm}

Algorithm~\ref{alg:innercore} computes a feature matrix based on each node property function in line~\ref{alg:line1}. For instance, this could include a node's neighborhood size as listed in Table~\ref{tab:node_property_functions}. The feature matrix $F$ is used to compute the inverse covariance matrix $\Sigma_F$ in line~\ref{alg:line2}, which will be utilized for future data depth calculations. The initial depth of each node is determined using the Mahalanobis depth with respect to the origin at line~\ref{alg:line3}. Nodes with depth greater than or equal to input $\epsilon$ are removed from the node set $\V$ at line~\ref{alg:line14}. Once one batch of node removals has been performed, the feature matrix and depth values are re-evaluated in lines~\ref{alg:line16}--\ref{alg:line17}. If any remaining nodes still have a depth greater than or equal to $\epsilon$, the next batch is initiated at the same $\epsilon$ level. When there are no nodes left with depth larger than $\epsilon$, the algorithm is considered complete, and the remaining nodes in $\V$ are returned as the {\sf InnerCore}.

\medskip
\noindent\textbf{Scalability}. Computing the {\sf InnerCore} requires performing Cholesky decomposition on the covariance matrix at line~\ref{alg:line2} once, which has time complexity $O(d^3)$ for $d$ features. Node features need to be recomputed at each iteration of the while loop with a cost of $O(|\V|\times deg)$, where $deg$ is the average degree in the graph. There are at most $|\V|$ iterations (number of nodes). In the worst case, the total time complexity is $O(d^3 + |\V|\times deg \times |\V|)$. However, since the neighborhood of a node can be sparse, the value of $deg$ is small. Moreover, since multiple nodes are removed in batches, the number of iterations is much smaller than $|\V|$. For example, in a network with approximately 480,000 nodes and 1 million edges (\S 4), only 4 iterations on average are needed for an $\epsilon$ of 0.1.

\medskip\noindent\textbf{InnerCore vs. Alphacore.} {\sf InnerCore} discovery of a graph $G$ does not require a complete decomposition of all graph cores by varying $\epsilon$, as it is done in {\sf AlphaCore}~\cite{victor2021alphacore}. Instead, we set an $\epsilon$ value (e.g., $\epsilon=0.1$) just once, and then use the value to iteratively prune nodes until all remaining nodes, relative to themselves, satisfy a data depth less than $\epsilon$. The  {\sf InnerCore} approach is also different from graph-$k$-core decomposition~\cite{BatageljZ11}, where the outer cores are computed first before the higher $k$-core can be determined. As a result, {\sf InnerCore} discovery is quite scalable and can be applied to very large graphs. Our experiments in \S\ref{sec:exp} reveal that {\sf InnerCore} discovery has a running time that is only one-tenth of that required for {\sf AlphaCore} decomposition.

\subsection{InnerCore Expansion and Decay}
\label{sec:methdecayexpansion}
By analyzing how a temporal graph expands and shrinks in relation to the entry and exit of nodes on a daily basis, we can gain valuable insights into market sentiment. Specifically, we investigate how the {\sf InnerCore} of a network expands and decays on a given day compared to previous days. We define two measures to quantify the activity of influential nodes in the network: expansion and decay. {\em Expansion} measures the ratio of influential nodes on day $t$ that were also influential in the preceding $i$ days, while {\em decay} quantifies the ratio of influential nodes from the previous $i$ days that are not present in the influential nodes of day $t$. We define the influential nodes of a graph as its {\sf InnerCore} nodes (i.e., $\V_t^{inner}$).

To this end, we first discover $\V_t^{inner}$ as the set of nodes in the {\sf InnerCore} of the snapshot graph at timestamp $t$, and define $\V_{\cup(t-i)}^{inner}=\bigcup_i\V_{t-i}^{inner}$ as the union set of nodes in the {\sf InnerCore} of snapshot graphs from timestamps $\{t-1,t-2,\ldots, t-i\}$ for $i\geq 1$. Next, we define the expansion and decay measures at timestamp $t$ as follows:
%
 \begin{definition}[Expansion]
$\mathbb{E}_t=\left|\V^{inner}_t \setminus \V^{inner}_{\cup(t-i)}\right| / \left | \V^{inner}_{\cup(t-i)}\right|.$
 \end{definition}

The expansion values have a range $[0,\infty)$, where a value greater than 1 indicates a  growth in the size of {\sf InnerCore}. 
\begin{definition}[Decay]
$\mathbb{D}_t=\left|\V^{inner}_{\cup(t-i)} \setminus \V^{inner}_{t}\right| / \left | \V^{inner}_{\cup(t-i)}\right|.$
\end{definition}
The decay values have a range $[0,1]$, where a value of 0 indicates that all {\sf InnerCore} members from $t$ are present at $t+1$ as well. 

\begin{exam} [Expansion and Decay]
Suppose we have a temporal graph that produces two daily snapshot graphs at days $t$ and $t+1$. On day $t$, the InnerCore is composed of five nodes: $\V^{inner}_t=\{v_1, v_2, v_3, v_4, v_5\}$. On day $t+1$, the InnerCore has expanded to include eight nodes: $\V^{inner}_{t+1}=\{v_3, v_4, v_5, v_6, v_7, v_8, v_9, v_{10}\}$.

If we set $i=1$, we can calculate the expansion and decay measures for day $t+1$ based on the previous day. In this case, the union of the InnerCores is $\V^{inner}_{\cup(t-i)}=\{v_1, v_2, v_3, v_4, v_5\}$. Therefore, we have:

The expansion measure $\mathbb{E}_{t+1}$ is equal to $\frac{\left|\{v_6, v_7, v_8, v_9, v_{10}\}\right|}{\left|\{v_1, v_2, v_3, v_4, v_5\}\right|}$, which yields a value of 1. The decay measure $\mathbb{D}_{t+1}$ is equal to $\frac{\left|\{v_1,v_2\}\right|}{\left|\{v_1, v_2, v_3, v_4, v_5\}\right|}$, which yields a value of 0.4.
\end{exam}
%
\subsection{Behavioral Patterns in Temporal Networks}
\label{sec:methpatterns}
%
Temporal networks, such as blockchain networks, are constantly evolving and may undergo significant changes in user sentiment and node activity due to technological updates and cataclysmic events in as little as a few days.

By utilizing expansion and decay, we have identified four behavioral patterns that capture user sentiment and node activity. These patterns serve as the foundation for network analysis in our experiments detailed in \S\ref{sec:exp}. Figure~\ref{fig:behavior} illustrates the expansion and decay values for each pattern. To gain a better understanding of these patterns, particularly when examining the temporal graph of a financial network such as the Ethereum transaction network, it is helpful to consider the network's underlying transaction semantics.
%
\begin{itemize}[leftmargin=.1in]
    \item The {\em Despair} pattern is characterized by a reduction in expansion and an increase in decay, implying that previously influential nodes are leaving the network, while the {\sf InnerCore} is shrinking due to a decrease in the number of new influential nodes.
    \item The {\em Uncertainty} pattern is distinguished by an increase in both expansion and decay. This is primarily due to the influx of many new investors into the network who do not remain active for a significant period of time.
    \item The {\em Hope} pattern is characterized by a reduction in decay and an increase in expansion, indicating the presence of many newcomers to the network who remain active within the network.
    \item The {\em Faith} pattern is identified by a decrease in both decay and expansion, which initially suggests a state of confusion. On the positive side, nodes, such as investors, may have faith in the network's ability to withstand a catastrophic event, as demonstrated in the LunaTerra case in our experimental results. On the negative side, it may indicate a sense of hopelessness as investors may hold onto their assets without engaging in transactions or exiting the system altogether.
\end{itemize}
%
\begin{figure}
    \centering \includegraphics[width=\linewidth]{figs/behavior.png}
    \caption{In a temporal graph (e.g., transaction network), changes in decay and expansion
    reflect varying levels of hope, despair, uncertainty, and faith in the asset being represented.
    }
    \label{fig:behavior}
\end{figure}
%
\subsection{Motif Analysis in Innercore}
\label{sec:methmotif}
% 
Our rationale behind using motif analysis in conjunction with {\sf InnerCore} is to accurately discover larger and potentially influential players in the daily network.  The structure of a motif defines a behavior of interest and its existence in a network indicates the presence of such behavior.  

\begin{figure}[!ht]
\centering
\begin{subfigure}{.15\textwidth}
  \centering 
  % include first image
\includegraphics[width=.95\linewidth]{figs/motif1.png}
  \label{fig:m4}
\end{subfigure}~
\begin{subfigure}{.15\textwidth}
  \centering 
\includegraphics[width=.95\linewidth]{figs/motif4.png}
  \label{fig:m5}
\end{subfigure}
\begin{subfigure}{.15\textwidth}
  \centering 
\includegraphics[width=.95\linewidth]{figs/motif5.png}
  \label{fig:m10}
\end{subfigure}
\begin{subfigure}{.15\textwidth}
  \centering 
\includegraphics[width=.95\linewidth]{figs/motif6.png}
  \label{fig:m12}
\end{subfigure}
\begin{subfigure}{.15\textwidth}
  \centering 
\includegraphics[width=.95\linewidth]{figs/motif11.png}
  \label{fig:m14}
\end{subfigure}
\caption{Five three-node motifs exhibiting buy and sell behaviors.  Nodes labeled C denote the center where a center with an in-degree = 2 indicates buy behavior and an out-degree = 2 indicates sell behavior. Out of the 16 connected three-node motifs (see Figure 1B in \cite{milo2002network}), only the five given above (motifs 1, 4, 5, 6, and 11) contain a center node. 
\label{fig:motifs}}
\end{figure}

Motif analysis has been a popular tool to identify subgraph patterns and the addresses involved in them \cite{LeeKGOL20,bailey2009meme,zhang2012extracting,paranjape2017motifs,milo2002network}. We have decided to use three-node motifs since they can be identified more quickly than higher-order motifs, while still capturing the direct buying or selling behavior between addresses.  Our decision is consistent with previous research on temporal motifs~\cite{paranjape2017motifs}.  

\medskip
\noindent\textbf{Scalability}. The fastest triangular motif discovery algorithm has time complexity $O(|\V^{inner}|^\omega)$, where $\omega < 2.376$ is the fast matrix product exponent~\cite{latapy2008main,coppersmith1987matrix}. The number of nodes in the {\sf InnerCore} is denoted by $|\V^{inner}|$. We demonstrate in \S\ref{sec:exp} that triangular motif discovery on {\sf InnerCore}s has low time costs because of the relatively small size of daily networks' {\sf InnerCore}s.

We define the center of each 3-node motif as a node that either receives incoming edges from the two other nodes (buy behavior) or delivers outgoing edges to two other nodes (sell behavior). This definition ensures that motif centers 
exhibit buy or sell behavior, 
and they do not act as intermediary nodes between the other two nodes in a motif.

Out of the 16 connected three-node motifs (see Figure 1B in \cite{milo2002network}), only five of them contain a center node (Figure~\ref{fig:motifs}).
We identify all instances of these five motifs and their centers from our daily networks' {\sf InnerCore}s. 
Finally, we utilize the well-known {\sf TF-IDF} measure from information retrieval~\cite{salton1988term} to rank the discovered center nodes. 


{\sf TF-IDF} is a statistical measure to reflect the relevance of a word in a collection of documents. In our setting, we treat each discovered center address as a word and daily instances of each motif as a collection of documents to propose a novel node relevance score for temporal graphs: {\sf NF-IAF}.   

Formally, let $M={m_1,m_4,m_{5},m_{6},m_{11}}$ be the set of five motifs of interest, and let $T={t_1,t_2,\dots,t_n}$ be the set of $n$ days under consideration. 
For each $m_i \in M$ and $t_j \in T$, let ${c(v,m_i,t_j)}$ denote the number of occurrences of node $v\in \V^{inner}$ in all instances of motif $m_i$ on day $t_j$.
For all $v\in \V^{inner}$, $m_i\in M$, and $t_j\in T$, we define the node frequency ({\sf NF}) and inverse-appearance frequency ({\sf IAF}) as follows:
%
\begin{definition}[Node Frequency]
We define the node frequency of node $v$ for motif $m_i$ on day $t_j$  as 
$$NF(v,m_i,t_j)=  \frac{c(v,m_i,t_j)}{\sum\limits_{v \in \V_j^{inner}}{c(v,m_i,t_j)}}.$$ % 
\end{definition}
%
The {\sf NF} measures how frequently a particular node occurs in a specific motif on a specific day relative to the total number occurrences of all nodes in that motif on that day. 
%
\begin{definition}[Inverse Appearance Frequency]
We define the inverse appearance frequency of node $v$ for motif $m_i$  as 
$$IAF(v,m_i) = \log\frac{|T|}{df(v,m_i)}$$
where $|T|$ is the total number of days in the dataset, and $df(v,m_i)$ is defined as the number of days $t_j\in T$ where $c(v,m_i,t_j)>0$.
\end{definition}
%
The {\sf IAF} measures the importance of a node by how frequently it appears across all days for a motif.  If a node appears in many days for a motif, its {\sf IAF} will be low, indicating that it is not very informative. On the other hand, if a node appears in only a few days for a motif, its {\sf IAF} will be high, indicating that it is a rare and potentially important node.
%
\begin{definition}[NF-IAF Score]
The {\sf NF-IAF} score of node $v$ for motif $m_i$ on day $t_j$ is given as 
$$NF{\text-}IAF(v,m_i,t_t) = NF(v,m_i,t_j) \times IAF(v,m_i).$$
\end{definition}
%
A greater {\sf NF-IAF} score of a center node on a particular day  
indicates greater relevance between that node and the behavior associated with the motif type.  Therefore, a node corresponding to a motif center on a particular day with a high {\sf NF-IAF} score has an increased likelihood that it has more influence on the network on that day, while a lower {\sf NF-IAF} score indicates the opposite.  
\input{sections/042_tfidfexample.tex}
 
  

 
\section{Experiments}
\label{sec:experiments}
\subsection{Experimental details}
\paragraph{Datasets} We use three benchmark datasets in CZSL problem, namely Clothing16K~\cite{zhang2022learning}, UT-Zappos50K~\cite{yu2014fine}, and C-GQA~\cite{naeem2021learning}. Clothing16K~\cite{zhang2022learning} contains different types of clothing (\eg, shirt, pants) with color attributes (\eg, white, black). UT-Zappos50K~\cite{yu2014fine} is a fine-grained dataset consisting of different kinds of shoes (\eg, sneakers, sandals) with texture attributes (\eg, leather, canvas). C-GQA~\cite{naeem2021learning} is a split built on top of Stanford GQA dataset~\cite{hudson2019gqa}, composed of extensive common attribute concepts (\eg, old, wet) and object concepts (\eg, dog, bus) in real life. We follow the common data splits of these three datasets 
(see~\cref{tab:data-splits}). 

\begin{table}[h]
    \centering
    \scalebox{0.54}{
    \begin{tabular}{cccccccccc}
        \toprule
         & \multicolumn{3}{c}{Composition} & \multicolumn{2}{c}{Train} & \multicolumn{2}{c}{Val} & \multicolumn{2}{c}{Test}  \\
         \cmidrule(lr){2-4} \cmidrule(lr){5-6} \cmidrule(lr){7-8} \cmidrule(lr){9-10}
         Datasets & $|\mathcal{A}|$ & $|\mathcal{O}|$ & $|\mathcal{A}|\times|\mathcal{O}|$ & $|\mathcal{C}_{s}|$ & $|\mathcal{X}|$ & $|\mathcal{C}_{s}|$ / $|\mathcal{C}_{u}|$ & $|\mathcal{X}|$ & $|\mathcal{C}_{s}|$ / $|\mathcal{C}_{u}|$ & $|\mathcal{X}|$
         \\ \midrule
        Clothing16K~\cite{zhang2022learning} & 9 & 8 & 72 & 18 & 7242 & 10 / 10 & 5515 & 9 / 8 & 3413\\
        UT-Zappos50K~\cite{yu2014fine} & 16 & 12 & 192 & 83 & 22998 & 15 / 15 & 3214 & 18 / 18 & 2914 \\
        C-GQA~\cite{naeem2021learning} & 413 & 674 & 278362 & 5592 & 26920 & 1252 / 1040 & 7280 & 888 / 923 & 5098 \\
        \bottomrule
    \end{tabular}}
    \caption{Summary of data split statistics.}
    \label{tab:data-splits}
    \vspace{-10pt}
\end{table}

\paragraph{Open-world setting} In addition to the standard closed-world setting, we also evaluate our model on the open-world setting~\cite{mancini2021open}, which is neglected by most previous works. The open-world setting considers all possible compositions, which requires a much larger testing space than the closed-world setting during inference. Taking UT-Zappos50K as an example (see~\cref{tab:data-splits}), the closed world only considers 36 compositions in the testing set while the open world considers total 192 compositions, in which $\sim$81\% are ignored under the standard closed-world setting.

\paragraph{Evaluation metrics} Since CZSL models have an inherent bias for seen compositions, we follow the generalized CZSL evaluation protocol~\cite{purushwalkam2019task}. To overcome the negative bias for seen compositions, we apply different calibration terms to unseen compositions and compute the corresponding top-1 accuracy of seen and unseen compositions, where a larger bias makes higher unseen accuracy and lower seen accuracy, and vice versa.  We treat seen accuracy as $x$-axis and unseen accuracy as $y$-axis to derive an unseen-seen accuracy curve. We can then compute the area under curve (AUC), the best harmonic mean, the best seen accuracy, and the best unseen accuracy from the curve. In our experiments, we report these four metrics for evaluation, among which AUC is the most representative and stable metric for measuring CZSL model performance. \hsz{Note that the attribute accuracy or the object accuracy alone does not reflect CZSL performance, because the individual accuracy on attribute or object does not necessarily decide the accuracy of their composition.}

\paragraph{Implementation details}
We use a frozen ViT-B-16~\cite{dosovitskiy2020vit} backbone pretrained with DINO~\cite{caron2021emerging} on ImageNet~\cite{deng2009imagenet} in a self-supervised manner as our visual feature extractor. The ViT-B-16 outputs contain 197 tokens (1 \texttt{[CLS]} and 196 patch tokens) of 768 dimensions. For three attention disentangler modules, we implement one-layer multi-head attention framework following~\cite{vaswani2017attention}, changing the single input to paired inputs for cross-attentions. The embedders $\pi_a$, $\pi_c$, $\pi_o$ are the two-layer MLPs following the previous works~\cite{mancini2021open, zhang2022learning}, projecting the 768-dimension visual features to 300-dimension word embedding space. The word embedding prototypes are initialized with word2vec~\cite{mikolov2013distributed} for all datasets and learnable during training. The composition function $\psi$ is one linear layer. We train our model using Adam optimizer~\cite{kingma2015adam} with a learning rate of $5\times 10^{-6}$ for UT-Zappos50K and Clothing16K, and $5\times 10^{-5}$ for C-GQA. All models are trained with 128 batch size for 300 epochs.

\begin{table*}[t]
    \centering
    \scalebox{0.75}{
    \begin{tabular}{l>{\columncolor{tabcolor}}cccccc>{\columncolor{tabcolor}}cccccc>{\columncolor{tabcolor}}cccccc}
        \toprule
         Closed-world & \multicolumn{6}{c}{Clothing16K} & \multicolumn{6}{c}{UT-Zappos50K} & \multicolumn{6}{c}{C-GQA} \\
         \cmidrule(lr){2-7} \cmidrule(lr){8-13} \cmidrule(lr){14-19}
         Models & AUC & HM & Seen & Unseen & Attr & Obj & AUC & HM & Seen & Unseen & Attr & Obj & AUC & HM & Seen & Unseen & Attr & Obj \\
         \midrule
         SymNet~\cite{li2020symmetry} & 78.8 & 79.3 & 98.0 & 85.1 & 75.6 & 84.1 & 32.6 & 45.6 & 60.6 & 68.6 & 48.2 & 77.0 & 3.1 & 13.5 & 30.9 & 13.3 & 11.4 & 34.6 \\
         CompCos~\cite{mancini2021open} & 90.3 & 87.2 & 98.5 & 96.8 & \textbf{90.2} & 91.8 & 31.8 & 48.1 & 58.8 & 63.8 & 45.5 & 72.4 & 2.9 & 12.8 & 30.7 & 12.2 & 10.4 & 33.9 \\
         GraphEmb~\cite{naeem2021learning} & 89.2 & 84.2 & 98.0 & 97.4 & 90.0 & 93.1 & 34.5 & 48.5 & 61.6 & \textbf{70.0} & \textbf{50.8} & \textbf{77.1} & 3.8 & 15.0 & 32.3 & 14.9 & 13.8 & 33.2 \\
         Co-CGE~\cite{mancini2022learning} & 88.3 & 87.9 & 98.5 & 94.7 & 87.4 & 91.4 & 30.8 & 44.6 & 60.9 & 62.6 & 46.0 & 73.5 & 3.6 & 14.7 & 31.6 & 14.3 & 12.6 & 34.6 \\
         SCEN~\cite{li2022siamese} & 78.8 & 78.5 & 98.0 & 89.6 & 81.2 & 85.4 & 30.9 & 46.7 & \textbf{65.7} & 62.9 & 44.0 & 74.4 & 3.5 & 14.6 & 31.7 & 13.4 & 10.7 & 31.4 \\ 
         IVR~\cite{zhang2022learning} & 90.6 & 86.6 & \textbf{99.0} & 97.0 & 89.3 & \textbf{93.6} & 34.3 & 49.2 & 61.5 & 68.1 & 48.4 & 74.6 & 2.2 & 10.9 & 27.3 & 10.0 & 10.3 & \textbf{37.5} \\
         OADis~\cite{Saini_2022_CVPR} & 88.4 & 86.1 & 97.7 & 94.2 & 84.9 & 93.1 & 32.6 & 46.9 & 60.7 & 68.8 & 49.3 & 76.9 & 3.8 & 14.7 & 33.4 & 14.3 & 8.9 & 36.3 \\
         \midrule
         \framework (ours) & \textbf{92.4} & \textbf{88.7} & 98.2 & \textbf{97.7} & \textbf{90.2} & \textbf{93.6} & \textbf{35.1} & \textbf{51.1} & 63.0 & 64.3 & 46.3 & 74.0 & \textbf{5.2} & \textbf{18.0} & \textbf{35.0} & \textbf{17.7} & \textbf{16.8} & 32.3\\ 
        \bottomrule
    \end{tabular}}
    \caption{Closed-world results on three datasets. We report the area under curve (AUC), the best harmonic mean (HM), the best seen accuracy (Seen), the best unseen accuracy (Unseen), the attribute accuracy (Attr), and the object accuracy (Obj) of the unseen-seen accuracy curve under the closed world-setting. AUC is the core CZSL metric. All models use the same DINO ViT-B-16 backbone.} 
    \label{tab:cw-results}
\end{table*}

\begin{table*}[t]
    \centering
    \scalebox{0.75}{
    \begin{tabular}{l>{\columncolor{tabcolor}}cccccc>{\columncolor{tabcolor}}cccccc>{\columncolor{tabcolor}}cccccc}
        \toprule
         Open-world & \multicolumn{6}{c}{Clothing16K} & \multicolumn{6}{c}{UT-Zappos50K} & \multicolumn{6}{c}{C-GQA} \\
         \cmidrule(lr){2-7} \cmidrule(lr){8-13} \cmidrule(lr){14-19}
         Models & AUC & HM & Seen & Unseen & Attr & Obj  & AUC & HM & Seen & Unseen & Attr & Obj & AUC & HM & Seen & Unseen & Attr & Obj  \\
         \midrule
         SymNet~\cite{li2020symmetry} & 57.4 & 68.3 & 98.2 & 60.7 & 57.6 & 81.2 & 25.0 & 40.6 & 60.4 & 51.0 & 38.2 & \textbf{75.0} & 0.77 & 4.9 & 30.1 & 3.2 & 18.4 & 37.5 \\
         CompCos~\cite{mancini2021open} & 64.1 & 70.8 & 98.2 & 69.8 & 71.7 & 83.7 & 20.7 & 36.0 & 58.1 & 46.0 & 36.4 & 71.1 & 0.72 & 4.3 & 32.8 & 2.8 & 15.1 & 37.8 \\
         GraphEmb~\cite{naeem2021learning} & 62.0 & 68.3 & 98.5 & 69.7 & 71.8 & 82.4 & 23.5 & 40.0 & 60.6 & 47.0 & 37.1 & 69.3 & 0.81 & 4.8 & 32.7 & 3.2 & 17.2 & 36.7 \\
         Co-CGE~\cite{mancini2022learning} & 59.3 & 69.2 & 98.7 & 63.8 & 68.5 & 76.2 & 22.0 & 40.3 & 57.7 & 43.4 & 33.9 & 67.2 & 0.48 & 3.3 & 31.1 & 2.1 & 15.5 & 35.7 \\
         SCEN~\cite{li2022siamese}& 53.7 & 61.5 & 96.7 & 62.3 & 63.6 & 79.1 & 22.5 & 38.0 & \textbf{64.8} & 47.5 & 34.9 & 73.3 & 0.34 & 2.5 & 29.5 & 1.5 & 14.8 & 32.3 \\ 
         IVR~\cite{zhang2022learning} & 63.6 & 72.0 & 98.7 & 69.0 & 70.3 & 84.8 & 25.3 & 42.3 & 60.7 & 50.0 & 38.4 & 71.4 & 0.94 & 5.7 & 30.6 & 4.0 & 16.9 & 36.5 \\
         OADis~\cite{Saini_2022_CVPR} & 53.4 & 63.2 & 98.0 & 58.6 & 57.3 & \textbf{85.4} & 25.3 & 41.6 & 58.7 & \textbf{53.9} & \textbf{40.3} & 74.7 & 0.71 & 4.2 & 33.0 & 2.6 & 14.6 & \textbf{39.7} \\
         \midrule
         \framework (ours) & \textbf{68.0} & \textbf{74.2} & \textbf{99.0} & \textbf{73.1} & \textbf{75.0} & 84.5 & \textbf{27.1} & \textbf{44.8} & 62.4 & 50.7 & 39.9 & 71.4 & \textbf{1.42} & \textbf{7.6} & \textbf{35.1} & \textbf{4.8} & \textbf{22.4} & 35.6 \\ 
        \bottomrule
    \end{tabular}}
    \caption{Open-world results on three datasets. Different from~\cref{tab:cw-results}, open-world setting considers all possible compositions in testing.} 
    \label{tab:ow-results}
\end{table*}

\subsection{Comparison}
\hsznew{To ensure a fair comparison and demonstrate that our improvement over baseline models is not merely by ViT, we adopt ViT backbone to state-of-the-art CZSL models and \emph{re-train} all models.} We compare our method with them: \hsznew{(1)~OADis~\cite{Saini_2022_CVPR} disentangles attribute and object features from spatial convolutional maps;} (2)~SymNet~\cite{li2020symmetry} introduces the symmetry principle of attribute-object transformation and group theory as training objectives; (3)~CompCos~\cite{mancini2021open} extends CZSL to an open-world setting considering all possible compositions during inference, proposing a feasibility score based on data statistics to remove unfeasible compositions; (4)~GraphEmb~\cite{naeem2021learning} and Co-CGE~\cite{mancini2022learning} propose to use graph convolutional networks (GCN) to represent attribute-object relationships and compositions; (5)~SCEN~\cite{li2022siamese} projects visual features to a Siamese contrastive space to capture concept prototypes, and introduces complex state transition module to produce virtual compositions; (6)~IVR~\cite{zhang2022learning} proposes to disentangle visual features into concept-invariant domains from a perspective of domain generalization, by masking specific channels of visual features. 

\paragraph{Closed-world evaluation} In~\cref{tab:cw-results}, we compare our \framework model with the state-of-the-art methods. \framework consistently outperforms others by a significant margin. \framework increases the core metric AUC by 1.8 on Clothing16K, 0.6 on UT-Zappos50K, and 1.4 on C-GQA ($\sim$37\% relatively). Similarly, \framework increases the best harmonic mean (HM) by 0.8\% on Clothing16K, 1.9\% on UT-Zappos50K, and 3.0\% on C-GQA. We notice that SymNet~\cite{li2020symmetry} and SCEN~\cite{li2022siamese} perform badly on Clothing16K. The reason might be that not learning concept prototypes harms the word embedding expressivity on small-scale concepts. We also notice that IVR~\cite{zhang2022learning} performs very well on curated datasets Clothing16K and UT-Zappos50K but badly on larger-scale real-world dataset C-GQA. We hypothesize ideal concept-invariant domains might be difficult to learn from natural images and large-scale concepts of C-GQA. In contrast, our \framework model achieves state-of-the-art performance on all datasets.

\paragraph{Open-world evaluation} In~\cref{tab:ow-results}, we consider the open-world setting to compare our \framework with other methods. Likewise, \framework also performs the best among all methods under open-world setting. \framework increases AUC by 3.9 on Clothing16K, 1.8 on UT-Zappos50K, and 0.48 on C-GQA ($\sim$51\% relatively). \framework also increases the best harmonic mean (HM) by 2.2\% on Clothing16K, 2.5\% on UT-Zappos50K, and 1.9\% on C-GQA ($\sim$33\% relatively). From the above results, \framework surpasses others by a larger margin on open-world AUC and HM than closed-world ones, indicating \framework maintains utmost efficiency when turning from the closed world to the open world. It is worth mentioning that \framework does not apply any special operations (\eg, feasibility masking~\cite{mancini2021open}) for the open world and deals with the two settings in exactly the same way. 
IVR~\cite{zhang2022learning} keeps its performance to a great extent but still lags behind our method significantly.

\begin{table*}[t]
\begin{minipage}[t]{0.44\linewidth}
    \centering
    \scalebox{0.72}{
    \begin{tabular}{lcccc>{\columncolor{tabcolor}}cccc}
        \toprule
         & CA & AA & OA & Reg & AUC & HM & Seen & Unseen\\
         \midrule
         (0) & \xmark & \xmark & \xmark & \xmark & 23.8 & 41.1 & 59.0 & 48.9 \\
         (1) & self & \xmark & \xmark & \xmark & 25.3 & 42.3 & 61.1 & 49.9 \\
         (2) & self & self & self & \xmark & 26.7 & 44.6 & 61.9 & 49.8 \\
         (3) & self & cross & cross & \xmark & 26.9 & 44.5 & \textbf{63.4} &48.7 \\
         (4) & self & cross & cross & \cmark &  \textbf{27.1} & \textbf{44.8} & 62.4 & \textbf{50.7}\\
         
        \bottomrule
    \end{tabular}}
    \caption{Ablate the components in \framework on open-world UT-Zappos50K. CA, AA, and OA denote composition, attribute, and object attention. Reg denotes the regularization term. We test self- or cross-attention for AA and OA.} 
    \label{tab:model-ab}
\end{minipage}
\hspace{2mm}
\begin{minipage}[t]{0.54\textwidth}
\centering
    \scalebox{0.68}{
    \begin{tabular}{ll>{\columncolor{tabcolor}}cccc>{\columncolor{tabcolor}}cccc}
        \toprule
        & & \multicolumn{4}{c}{C-GQA} & \multicolumn{4}{c}{Clothing16K} \\
        \cmidrule(lr){3-6} \cmidrule(lr){7-10}
        & Inference formulation  & AUC & HM & Seen & Unseen & AUC & HM & Seen & Unseen\\
        \midrule
        (0) & $p(c)$ & 4.6 & 16.8 & \textbf{35.1} & 16.0 & \textbf{92.4} & \textbf{88.8} & \textbf{98.2} & \textbf{97.7} \\
        (1) & $p(a) \cdot p(o)$ &  4.0 & 15.8 & 31.4 & 15.1 & 57.3 & 66.3 & 96.7 & 63.1\\
        (2) & $p(c) + p(a) \cdot p(o)$ & \textbf{5.2} & \textbf{18.0} & 35.0 & \textbf{17.7} & 90.4 & 85.9 & 98.2 & 97.0\\
        (3) & $p(c) + \beta \cdot p(a) \cdot p(o)$ & \textbf{5.2} & \textbf{18.0} & 35.0 & \textbf{17.7} & \textbf{92.4} & 88.7 & \textbf{98.2} & \textbf{97.7} \\
        \bottomrule
    \end{tabular}}
    \caption{Results on closed-world Clothing16K and C-GQA using different inference formulations. Rows (0)-(2) respectively represents the cases when $\beta=0.0$, $\beta=+\infty$, and $\beta=1.0$. Row (3) is our inference formulation, which applies an $\beta$ optimized on the validation set.} 
    \label{tab:eval-ab}
\end{minipage}
\end{table*}

\subsection{Ablation study}
\paragraph{Backbone: ResNet \textit{vs} ViT}
\hsznew{Our work leverages ViT as the default backbone to excavate more high-level sub-space information, while ResNet18 is the most common backbone in previous works. 
In~\Cref{tab:backbone}, we compare our \framework to OADis~\cite{Saini_2022_CVPR}  with both backbones. Our \framework performs similarly to OADis with ResNet18, but outperforms it significantly with ViT. Additionally, we present an ablation study on different components of our method with the ResNet18 backbone in the Appendix. These experiments indicate that our model benefits from ViT and all components of our method are effective regardless of the backbone.}

\begin{table}[h]
    \centering
    \scalebox{0.75}{
    \begin{tabular}{ll>{\columncolor{tabcolor}}cc>{\columncolor{tabcolor}}cc>{\columncolor{tabcolor}}cc}
        \toprule
          \multicolumn{2}{l}{Closed-world} & \multicolumn{2}{c}{Clothing16K} & \multicolumn{2}{c}{UT-Zappos50K} & \multicolumn{2}{c}{C-GQA} \\
         \cmidrule(lr){3-4} \cmidrule(lr){5-6} \cmidrule(lr){7-8}
         Backbone & Models & AUC & HM & AUC & HM & AUC & HM \\
         \midrule
         \multirow{2}{*}{ResNet18} & OADis~\cite{Saini_2022_CVPR} & 85.5 & 84.7 & \textbf{30.0} & 44.4 & \textbf{3.1} & 13.6 \\
          & \framework (ours) & \textbf{87.2} & \textbf{85.1}  & 29.5 & \textbf{47.0} & \textbf{3.1} & \textbf{13.7} \\
         \midrule
         \multirow{2}{*}{ViT-B-16} & OADis~\cite{Saini_2022_CVPR} & 88.4 & 86.1  & 32.6 & 46.9 & 3.8 & 14.7 \\
          & \framework (ours) & \textbf{92.4} & \textbf{88.7}  & \textbf{35.1} & \textbf{51.1} & \textbf{5.2} & \textbf{18.0} \\
        \bottomrule
    \end{tabular}}
    \caption{Compare \framework and OADis~\cite{Saini_2022_CVPR} with ResNet18 and ViT.} 
    \label{tab:backbone}
    \vspace{-5pt}
\end{table}

\paragraph{Different parts of \framework} We evaluate the effectiveness of attention disentanglers (composition, attribute, and object attention) and the regularization term in our model. We report the ablation study results on the open-world UT-Zappos50K in~\cref{tab:model-ab}. Rows~(0)-(2) show attention disentanglers can significantly improve the performance. Rows~(2)-(3) show that cross-attention learns disentangled concepts better than self-attention for AA and OA. Rows~(3)-(4) show the regularization term can further benefit the visual disentanglement, improving the unseen accuracy and overall AUC.

\paragraph{Inference formulation} We also investigate the effect of our inference formulation $p(c) + \beta \cdot p(a) \cdot p(o)$ in~\cref{tab:eval-ab}. We report the results with extreme values of $\beta$, \ie, $\beta=0.0$ and $\beta=1.0$. Note that $\beta=0.0$ means only using composition probability for prediction. In addition, we also test the performance only using the product of attribute and object probabilities $p(a) \cdot p(o)$. We can observe that the best fixed $\beta$ value is unfixed among datasets. For example, $\beta=1.0$ gives the highest AUC for C-GQA in row (2) while $\beta=0.0$ for Clothing16K in row (0). In contrast, our validated $\beta$ consistently gives the best inference results for both datasets. Another observation on C-GQA is that $p(a) \cdot p(o)$ alone is not a good prediction, but adding it to $p(c)$ can increase the unseen accuracy. This indicates that the disentangled attribute prediction $p(a)$ and object prediction $p(o)$ indeed enhance the unseen generalization for CZSL problem.

\paragraph{Effect of regularization term} We propose an EMD-adapted regularization term at the attention level to force attentions to disentangle the concept of interest. We also investigate the effect of applying the regularization term at the feature level. Specifically, 
we compare our EMD-based distance to the cosine and euclidean feature distances. The results on open-world UT-Zappos50K are shown in~\cref{tab:reg-ab}. Our EMD-based regularization outperforms other distance forms, because our attention-level EMD distance considers token-wise similarity capturing the specific concept-related attention responses.
\begin{table}[h]
   \centering
   \setlength{\tabcolsep}{20pt}
    \scalebox{0.7}{
    \begin{tabular}{l>{\columncolor{tabcolor}}cccc}
        \toprule
        Reg & AUC & HM & Seen & Unseen \\
        \midrule
        Cosine & 26.8 & 44.7 & \textbf{63.0} & 48.6 \\
        Euclidean & 26.2 & 44.3 & 62.6 & 47.5 \\
        Ours (EMD) & \textbf{27.1} & \textbf{44.8} & 62.4 & \textbf{50.7}\\
        \bottomrule
    \end{tabular}}
    \caption{Comparison of different regularization terms on open-world UT-Zappos50K.} 
    \label{tab:reg-ab} 
    \vspace{-5pt}
\end{table}

\subsection{Qualitative analysis}
Visual disentanglement in feature space is hard to visualize~\cite{Saini_2022_CVPR}. Inspired by previous work attempts~\cite{Saini_2022_CVPR,zhang2022learning,li2020symmetry,nagarajan2018attributes}, we conduct qualitative analysis of image and text retrieval to show how our \framework model correlates the visual image and the concept composition. In addition, to further validate \framework is efficient to disentangle visual concepts, we conduct unseen-to-seen image retrieval based on their visual concept features extracted by attribute and object attentions.

\begin{figure*}[t]
     \centering
     \begin{subfigure}[b]{0.32\textwidth}
         \centering
         \includegraphics[width=0.92\linewidth]{images/txt2img.pdf}
    \caption{Top-5 text-to-image retrieval.}
    \label{fig:wrd2img}
     \end{subfigure}
     \hfill
     \begin{subfigure}[b]{0.35\textwidth}
    \centering
    \includegraphics[width=\linewidth]{images/img2wrd.pdf}
    \caption{Top-5 image-to-text retrieval.}
    \label{fig:img2wrd}
     \end{subfigure}
     \hfill
     \begin{subfigure}[b]{0.32\textwidth}
    \centering
    \includegraphics[width=0.9\linewidth]{images/retrieve_visual.pdf}
    \caption{Top-5 visual concept retrieval.}
    \label{fig:concept-retrieve}
     \end{subfigure}
    \caption{\hsznew{Qualitative analysis. (a) In the last row of ``suede sandals", the wrong image (red box) is ``fake leather sandals". (b) Each image has the ground-truth label (black text) and 5 retrieval results (colored text), in which the green text is the correct prediction. (c) We retrieve images sharing the same visual concepts by their visual concept features for unseen images of ``yellow skirt" and ``pink pants".}}
    \label{fig:qualitative}
    \vspace{-4pt}
\end{figure*}

\paragraph{Image and text retrieval}
We first consider text-to-image retrieval. Given a text composition, \eg, ``leather heels", we embed it and retrieve the top-5 closest visual features based on the feature distance. We display four text compositions of the different objects sharing the same attributes and vice versa in~\cref{fig:wrd2img}. We can observe that the retrieved images are correct in most of the cases. One exception is when retrieving ``suede sandals", the third closest image is ``fake leather sandals". Although ``suede sandals" and ``fake leather sandals" are not the same composition, they are quite visually similar. We then consider image-to-text retrieval, shown in~\cref{fig:img2wrd}. Given an image, \eg, the image of a ``brown zebra", we extract its visual feature and retrieve the top-5 closest text composition embeddings. It is difficult to retrieve the ground-truth label in the top-1 closest text composition, but all top-5 results are all semantically related to the image. We take the image of ``blond person" (row 3, col 2) as an example. Although the text composition ``blond person" is not retrieved in the top-5 matches, the retrieved results ``white shirts", ``white outfit", ``white shorts", ``white pants", and ``young girl" are all reasonable and actually present in the image. Image and text retrieval experiments validate that our \framework efficiently projects visual features and word embeddings into a uniform space.
\vspace{-2pt}

\paragraph{Visual concept retrieval}
Because the attribute and the object are visually coupled in an image to a high degree of entanglement, it is challenging to visualize the disentanglement in feature space~\cite{Saini_2022_CVPR}. Saini \etal~\cite{Saini_2022_CVPR} retrieve single attribute or object text from test images. However, this process is the same as multi-label classification and insufficient to validate that disentangled visual concepts are learned from images. \hsz{Based on the disentanglement ability of ADE, we construct a visual concept retrieval experiment to investigate the distances between visual concept features, \ie, the embedded attribute feature $\pi_a(v_a)$ and the embedded object feature $\pi_o(v_o)$, extracted from different images. Prior to our work, no existing models can do so, because none of them extracts concept-exclusive features like ADE.} The results are shown in~\cref{fig:concept-retrieve}. We first extract attribute features and object features from all seen images. Given an unseen image, we retrieve the top-5 closest images by measuring the feature distance between the attribute feature of the given image and that of all seen images, and the same goes for the object feature. For the image of ``yellow skirt", all retrieval results for the visual concept ``yellow" are all  ``yellow \texttt{[OBJ]}", and all retrieval results for ``skirt" are ``\texttt{[ATTR]} skirt". For the ``pink pants" image, our model also perfectly retrieves the visual concepts, \ie, the attribute ``pink" and the object ``pants". Our experimental results demonstrate that our \framework model is effective to disentangle visual concepts from seen compositions and combine learned concept knowledge into unseen compositions.
This paper presented a comprehensive analysis of the use of \acrfull{PINN} for power system dynamic simulations. We show that \glspl{PINN} (i) are 10 to 1'000 times faster than conventional solvers, (ii) do not face issues of numerical instability unlike conventional solvers, and, (iii) achieve a decoupling between the power system size and the required solution time. However, \glspl{PINN} are less flexible (i.e. they do not easily handle parameter changes), and require an up-front training cost. Overall, this makes \gls{PINN}-based solutions well-suited for repetitive tasks as well as task where run-time speed is crucial, such as for screening.

Besides the comparison between conventional and \gls{NN}-based methods, this paper conducts a deeper analysis on the parameters that affect the performance of the \gls{NN} solutions. In that respect, we introduce a new \gls{NN} regularisation, called dtNN, as a intermediate step between \glspl{NN} and \glspl{PINN}. We show that \glspl{PINN} achieve overall higher levels of accuracy, and more balanced error distributions thanks to the evaluation of the collocation points.
%\vspace{-2mm}
\section*{Appendix}
%
\begin{figure}
    \centering \includegraphics[width=0.7\linewidth]{figs/running_example.png}
    \vspace{-3mm}
    \caption{\small A running example to compare between the graph-$k$-core and {\sf AlphaCore} decomposition methods. The Coreness of nodes according to graph-$k$-core decomposition is shown with different node colors, whereas {\sf AlphaCore} is run with in-strength and out-strength as node features with a step size of 0.25. Different {\sf AlphaCore}s are shown using dotted boundaries. \label{fig:running_example}}
    \vspace{-5mm}
\end{figure}
%
%\vspace{-2mm}
\spara{An Example of AlphaCore.}
To better illustrate the differences between the traditional graph-$k$-core and {\sf AlphaCore} decomposition methods, we showcase an example in Figure \ref{fig:running_example}.  %We are interested in the inner core that captures only the most important traders of a financial network.  
In the case of graph-$k$-core, the innermost core is the 3-core, whereas the {\sf InnerCore} of {\sf AlphaCore} would be the core of $\alpha$ $>$ 0.75.  Note that the 3-core consists of traders that trade frequently with themselves, but their trade volumes with themselves are not that significant compared to other transactions which exist in the network.  In certain analyses of financial networks such as anomalous address detection, being able to filter out these negligible transactions and their participating traders, while still capturing more meaningful ones, significantly improves the accuracy and scalability of subsequent computations on the decomposed network core.  On the other hand, the {\sf AlphaCore} of $\alpha$ $>$ 0.75 is able to capture both the traders that participate in the largest transactions which occur in the example network, while filtering the negligible transactions and their participating traders.  
We point out that the main limitation with graph-$k$-core is that it only considers node degrees, whereas {\sf AlphaCore} is flexible and can consider any combination of node features as outlined in Table 1, without requiring to specify any feature weighting parameters to perform effectively on a particular task.  
Therefore, in networks where edge weights fall under a broad range and they are meaningful distinguishing factors, we recommend {\sf AlphaCore} over the traditional graph-$k$-core decomposition.


\SetKwInput{KwInput}{Input}
\SetKwInput{KwOutput}{Output}
\SetKwRepeat{Do}{do}{while}
%
\begin{algorithm}[tb!]
\footnotesize
\KwInput{Directed, weighted, multigraph $G(V,E,w)$,\\
Set of node property functions $p_1, ..., p_n \in P$,\\%,
Data depth threshold $\epsilon$}
\KwOutput{InnerCore $V^{inner}$}
\tcp{Compute feature matrix}
$F = [f_1, ..., f_n] = \forall p_i \in P: f_i = p_i(v, G), \forall v \in V$\label{alg:line1}\;% \tcp*{initial feature matrix}
$\Sigma_F^{-1}$ = cov$(F)^{-1}$\tcp*{compute only once}\label{alg:line2}
\tcp{Compute initial depth values}
$z = [z_1, ..., z_n] = \forall v_i \in V: z_i = [1+(F_{i,*})'\Sigma^{-1}_F(F_{i,*})]^{-1}$\label{alg:line3}\;
  \Do{$\exists z_i: (z_i \geq \epsilon) \wedge (v_i \in V)$}{
    \ForEach{$z_i \geq \epsilon$}{
    $\V = \V \setminus \{v_i\}$\label{alg:line14}\;
    }
    \tcp{recompute node properties}
    $F = \forall p_i \in P: p_i(v, G), \forall v \in V$\label{alg:line16}\;
    \tcp{recompute depth}
    $z_i = [1+(F_{i,*})'\Sigma^{-1}_F(F_{i,*})]^{-1}, \forall v_i \in V$\label{alg:line17}\;
  }
\KwRet{$\V$ \tcp{as InnerCore $V^{inner}$}}\label{alg:line21}
\caption{\small{\sf InnerCore} Discovery}
\label{alg:innercore}
\end{algorithm}

\spara{Algorithm~\ref{alg:innercore} for {\sf InnerCore} Discovery.}
Algorithm~\ref{alg:innercore} computes a feature matrix based on each node property function in line~\ref{alg:line1}. For instance, this could include a node's neighborhood size as listed in Table~\ref{tab:node_property_functions}. The feature matrix $F$ is used to compute the inverse covariance matrix $\Sigma_F$ in line~\ref{alg:line2}, which will be utilized for future data depth calculations. The initial depth of each node is determined using the Mahalanobis depth with respect to the origin at line~\ref{alg:line3}. Nodes with depth greater than or equal to input $\epsilon$ are removed from the node set $\V$ at line~\ref{alg:line14}. Once one batch of node removals has been performed, the feature matrix and depth values are re-evaluated in lines~\ref{alg:line16}--\ref{alg:line17}. If any remaining nodes still have a depth greater than or equal to $\epsilon$, the next batch is initiated at the same $\epsilon$ level. When there are no nodes left with depth larger than $\epsilon$, the algorithm is considered complete, and the remaining nodes in $\V$ are returned as the {\sf InnerCore}.

\spara{Parameters in Experimental Setup.}
In the context of {\sf InnerCore} expansion and decay,
a greater $i$ (i.e., the history parameter from \S3.2) produces an averaging effect, coupled  with the tendency to depress expansion and inflate decay.  Setting a specific $i$ value depends on the application. %in general, higher $i$ reduces fluctuations between expansion and decay of each day.  
%In our experiments, 
We use $i$ = 1 to improve the accentuation of expansion and decay in the {\sf InnerCore} to better depict the shift in market sentiment during the days of significant events. %in our case studies.}  

In {\sf InnerCore} decomposition, depth values range between $(0, 1]$; nodes with high property values (e.g., many transactions, higher transacted amounts) tend to have low depth, while nodes with low property values tend to have high depth~\cite{victor2021alphacore}. With data depth threshold $\epsilon=1$, all nodes will be returned as {\sf InnerCore}  members; while for $\epsilon=0$, the empty set will be returned.  
Setting an appropriate $\epsilon$ depends on the desired size of the {\sf InnerCore} returned specific to an application.  In our case studies, we set $\epsilon=0.1$ to ensure that the average number of nodes in each daily {\sf InnerCore} is above 150.

%
\begin{figure}
  \centering      \includegraphics[width=0.34\textwidth]{figs/usdc_SCPD}
  \vspace{-3mm}
  \caption{\small USDC anomalous days identified by {\sf SCPD} . Compared to decay and expansion measures by {\sf InnerCore}, {\sf SCPD}  less accurately captures USDC's temporary peg loss occurring on Mar 11, 2023.}
  \label{fig:usdcSCPD}
  \vspace{-5mm}
\end{figure}

\spara{Effectiveness Experiment 3: SCPD to the USDC network.}
We also apply {\sf SCPD}  to the USDC network to compare with our decay and expansion results.  From Figure~\ref{fig:usdcSCPD}, we observe that {\sf SCPD}  less accurately captures USDC's temporary peg loss occurring on Mar 11. {\sf SCPD}  identifies Mar 12 and 15 as anomalous which are one day and four days, respectively, after USDC's peg loss.  Conversely, our expansion measures in Figure~\ref{fig:usdcExpansionDecay} accurately capture USDC's peg loss by producing a prominent peak on Mar 11.  %Similar to results from Experiment 2 (\S 4.4),
Clearly, our {\sf InnerCore} expansion measures more accurately indicate an anomaly on days when a significant event occurred.

\eat{
\subsection{Additional Related Work}
%
In recent years, several studies focused on analyzing different aspects of the blockchain networks \cite{ChenWZCZ19,AkcoraLGK20,kalodner2017blocksci,GuidiM20}, particularly in the Ethereum network. 
Researchers working on natural
language processing and sentiment analysis using tweets, news articles, cryptocurrency prices and charts, Google Trends about blockchains \cite{VNO19,KS20} could find supporting evidences based on
blockchain data analysis. 
Oliveira et al. \cite{oliveira2022analysis} performed an analysis of the effects of external events on the Ethereum  platform, highlighting short-term changes in the behavior of accounts and transactions on the network. Aspembitova et al. \cite{aspembitova2021behavioral} used temporal complex network analysis to determine the properties of users in the Bitcoin and Ethereum markets and developed a methodology to derive behavioral types of users.

Other studies focused on specific aspects of the Ethereum network. For instance, Casale et al. \cite{casale2021networks} analyzed the networks of Ethereum Non-Fungible Tokens using a graph-based approach, while Silva et al. \cite{silva2020characterizing} characterized relationships between primary miners in Ethereum using on-chain transactions. Meanwhile, Victor et al.  \cite{victor2019measuring} measured Ethereum-based ERC20 token networks and Kiffer et al. \cite{kiffer2018analyzing} examined how contracts in Ethereum are created and how users interact with them.

Zhao et al. \cite{zhao2021temporal} investigated the evolutionary nature of Ethereum interaction networks from a temporal graphs perspective, detecting anomalies based on temporal changes in global network properties and forecasting the survival of network communities using relevant graph features and machine learning models. Li et al. \cite{li2021measuring} analyzed the magnitude of illicit activities in the Ethereum ecosystem using proprietary labeling data and machine learning techniques to identify additional malicious addresses. Kilic et al. \cite{kilicc2022fraud} predicted whether given addresses are blacklisted or not in the Ethereum network using a transaction graph and local and global features. 

Our approach for analyzing the effects of external events on a blockchain platform is similar to the one used by Anoaica et al. \cite{anoaica2018quantitative}. The authors examined the temporal variation of transaction features in the Ethereum network and observed an increase in activity following the announcement of the Ethereum Alliance creation. Gaviao et al.  \cite{gaviao2020transaction} also studied the evolution of users and transactions over time, showing the centralization tendency of the transaction network. Kapengut et al.\cite{kapengut2022event} study the Ethereum blockchain around the BeaconChain phase of the PoS transition (September 15, 2022), but the authors focus on the power efficiency and miners' rewards around the transition.


Finally, Khan \cite{khan2022graph} conducted a survey of datasets, methods, and future work related to graph analysis of the Ethereum blockchain data, while Ramezan Poursafaei's PhD thesis \cite{ramezan2022anomaly} presented results on temporal anomaly detection in blockchain networks.
}

\spara{Ethical Consideration.} The use of the efficient and unsupervised core decomposition algorithm, {\sf InnerCore}, could inadvertently raise fairness concerns by potentially providing advantages to certain network traders over others due to the unique market sentiment insight that decay and expansion measures provide. Even in its intended use, reliance on {\sf InnerCore} decay and expansion behavioral pattern results might lead to incorrect or biased reactionary decisions. Similarly, addresses pinpointed by {\sf InnerCore}, followed by subsequent centered-motif analysis and {\sf NF-IAF} score assignment, should be interpreted with carefulness. It is vital for both researchers and decentralized finance traders to exercise prudent judgment and validate findings through additional comparative methods before arriving at any definitive conclusions or undertaking consequential actions.

%%
%% The next two lines define the bibliography style to be used, and
%% the bibliography file.
\bibliographystyle{ACM-Reference-Format}
\bibliography{stablecoin.bib}

%%
%% If your work has an appendix, this is the place to put it.

\end{document}
%\endinput
%%
%% End of file `sample-sigconf.tex'.
