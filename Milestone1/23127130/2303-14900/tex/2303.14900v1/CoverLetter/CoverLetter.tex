\documentclass[11pt]{article}
\usepackage{amssymb}
\usepackage{graphicx}
\usepackage{amsmath}
\setcounter{MaxMatrixCols}{10}
\newtheorem{theorem}{Theorem}
\newtheorem{acknowledgement}[theorem]{Acknowledgement}
\newtheorem{algorithm}[theorem]{Algorithm}
\newtheorem{axiom}[theorem]{Axiom}
\newtheorem{case}[theorem]{Case}
\newtheorem{claim}[theorem]{Claim}
\newtheorem{conclusion}[theorem]{Conclusion}
\newtheorem{condition}[theorem]{Condition}
\newtheorem{conjecture}[theorem]{Conjecture}
\newtheorem{corollary}[theorem]{Corollary}
\newtheorem{criterion}[theorem]{Criterion}
\newtheorem{definition}[theorem]{Definition}
\newtheorem{example}[theorem]{Example}
\newtheorem{exercise}[theorem]{Exercise}
\newtheorem{lemma}[theorem]{Lemma}
\newtheorem{notation}[theorem]{Notation}
\newtheorem{problem}[theorem]{Problem}
\newtheorem{proposition}[theorem]{Proposition}
\newtheorem{remark}[theorem]{Remark}
\newtheorem{solution}[theorem]{Solution}
\newtheorem{summary}[theorem]{Summary}
\setlength{\textwidth}{19cm} \setlength{\textheight}{23cm}
\setlength{\oddsidemargin}{-1.2cm} \setlength{\topmargin}{1mm}
\newenvironment{proof}[1][Proof]{\textbf{#1.} }{\ \rule{0.5em}{0.5em}}
\linespread{1}

\pdfminorversion=4

\begin{document}
\pagestyle{empty}
\begin{flushleft}
{\it International Institute of Finance, School of Management,\\
  	University of Science and Technology of China\\
	Building of Management Academy, \\
	East Campus of USTC. No. 96 Jinzhai Rd, Hefei,\\
       230026, Anhui Province, P.R.China}
\end{flushleft}


%\vspace{cm}
%\begin{flushright}
%\hspace{10cm} \today
%\end{flushright}

\begin{flushleft}
Dear Editor of the \textit{Journal of the University of Science and Technology of China},
\end{flushleft}
%\vspace{1cm}

{\noindent Enclosed} please find the manuscript ``Nonparametric approaches for analysing carbon emission: from statistical and machine learning perspectives'', 
which we would like to submit for possible publication. \\


The paper investigated various nonparametric approaches in statistics and machine learning (ML) for analysing carbon emissions data. Linear regression models, especially the extended STIRPAT model, are routinely-applied for analyzing carbon emissions data. However, since the relationship between carbon emissions and the influencing factors is complex, fitting a simple parametric model may not be an ideal solution. We hence considered various nonparametric approaches in statistics and ML for modelling carbon emissions data,  including kernel regression, random forest and neural network. We selected data from ten Chinese cities from 2005 to 2019 for modelling studies. We found that neural network had the best performance in both fitting and prediction accuracy, which implies its capability of expressing 
%, illustrating its judged the goodness of the models by both fitting and prediction accuracy, and found that neural network had the best performance in expressing 
the complex relationships between carbon emissions and the influencing factors. This study provides a new means for quantitative modelling of carbon emissions research that helps to understand how to characterize urban carbon emissions and to propose policy recommendations for ``carbon reduction''. In addition, we used the carbon emissions data of Wuhu city as an example to illustrate how to use this new approach.\\


%The current version of the manuscript is self-contained. If needed, a shorter version could be obtained by moving some material to a separate supplementary file, available online. At this stage we feel that a more complete exposition, which could be shortened in a second step, can prove useful to motivate our theoretical derivations and explain our methodological contributions to a broader audience. \\


We hope that our results are of interest to {JUSTC} audience and that they will
become an interesting contribution to the literature, stimulating new research. \\


\noindent Looking forward to hearing from you soon,


\vspace{2cm}

\hspace{9cm} Hang Liu  (on behalf of the authors)

\end{document}
