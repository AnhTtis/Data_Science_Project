\section{Discussion}
% \subsection{Impact on Instrumentation on Flakiness Detection and Prevention}

% Our preliminary experiments show that while instrumentation may be a cause of flakiness, this effect is rare. There are only a few cases which show an impact of instrumentation on the presence of flaky tests. There is no unified pattern that we could identify with regards to specific instrumentation tools or configure that would introduce/increase flakiness across the programs we study.
%
% A related question is whether instrumentation can interfere with existing flakiness detection and prevention techniques. Given a large number of such techniques \cite{parry2021survey}, this discussion is incomplete, and a lot of questions remain unclear. We aim to investigate those further in our future research.
%
% For \textit{detection techniques based on static analysis} such as \cite{fatima2022flakify}, interference is possible (but may still be unlikely) simply because the instrumentation is not part of the analyses, and this may lead to additional false positives and/or false positives. This effect is difficult to mitigate. In theory, the agents instrumenting the program could be made part of the analyses by applying instrumentation early (using source code or static bytecode instead of load-time/runtime weaving \cite{kiczales2001overview}), but this approach is not well supported by tools, and may require the agent source code to be available.
%
% Dynamic prevention and detection techniques such as RootFinder \cite{wing2019rootcause}, saflate \cite{dietrichflaky} (both use instrumentation), or FlakeFlagger \cite{alshammari2021flakeflagger} (which uses a hybrid approach) can be prone to instrumentation order dependencies. While it is possible to craft examples showing this, this is unlikely to occur in practice.
%
% \todo[inline] {remove FlakeFlagger if it does not use instrumentation}
%
% Dynamic prevention and detection techniques that use a particular platform (such as Nondex modified standard library or VMVM \cite{bell2014unit} \todo{(citation tbc))} may also be sensible to the presence of instrumentation, as the instrumentation code itself is affected by those changes.
