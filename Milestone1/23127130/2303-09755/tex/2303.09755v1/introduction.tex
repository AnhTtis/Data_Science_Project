\section{Introduction}
\label{sec:introduction}

A test can only provide useful feedback if it consistently has the same outcome (either pass or fail) for every execution with the same code version. Flaky tests may pass in some runs and fail on others. Test flakiness has been gaining the attention of both academia and industry because of its negative impact on testing, testing-dependent processes (especially automated testing in CI/CD pipelines), and techniques that rely on executing tests \cite{parry2021survey,rasheed2022test}. Several factors can cause test flakiness, including concurrency, test order dependency, network, shared state and platform dependencies. Most of these causes are common across programming languages and platforms \cite{hashemi2022empirical,luo2014empirical,costa2022test}.

One potential factor that may have an impact on test flakiness is instrumentation.
Ideally, for common use cases of instrumentation, such as code coverage, the effect of instrumentation should be transparent to the application. Concerning test flakiness, this means that test outcomes should remain the same with or without instrumentation. This may well not be the case in practice, though. %, as indicated by the previously cited studies.
A study on code coverage at Google \cite{ivankoviFSE2019} describes flakiness as a cause for failed coverage computation. Lam et al. \cite{wing2019rootcause} explain how their instrumentation for root-causing flakiness interferes with program behaviour and leads to increased/decreased test flakiness.
%This particular issue has not been investigated as much.

However, there are gaps in these studies with respect to the question we are interested in. Those studies are not focused on the effect of instrumentation on flakiness, and the full datasets are not available as the studies are from the industry. This work is an attempt towards addressing this, in which we discuss the effect of instrumentation on test flakiness, and how it affects flaky test prevention/detection techniques.
We propose evaluation metrics and perform a preliminary evaluation on a dataset used in a previous test flakiness study to determine whether instrumentation impacts flakiness. In this work, we address the research question: \\\textbf{RQ:} Does instrumentation  increase/decrease test flakiness?
