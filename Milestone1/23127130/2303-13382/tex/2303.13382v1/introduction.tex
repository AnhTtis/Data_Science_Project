\section{Introduction}\label{sec:intro}
Contacts lead to discontinuous dynamics and thus, planning through contacts requires careful treatment of constraints arising due to these discontinuities. Complementarity constraints offer an efficient way of modeling contact systems. However, uncertainty in contact systems could lead to stochastic complementarity systems~\cite{shirai2022chance}. Even though complementarity systems are well studied, stochastic complementarity systems are not well understood. The state and complementarity variables are implicitly related via the complementarity constraints -- uncertainty in one leads to stochastic evolution of other. This makes uncertainty propagation challenging. %for the following stochastic optimization. 
Furthermore, multiplicity of solutions to the complementarity variables also makes it difficult to characterize the stochastic evolution. 
In this paper, we present an approximate treatment of stochastic complementarity systems using particles. We present the design and evaluation of a contact-aware stochastic controller for covariance control of the underlying uncertain system. An important-particle algorithm is presented for an efficient solution to the resulting stochastic optimization problem.

% We present a particle-based technique to perform feedback control of stochastic complementarity systems. We present covariance control for the stochastic complementarity systems by solving for a trajectory-centric feedback controller to enable efficient control along long-horizon trajectories. We present a cutting plane algorithm for computing numerically efficient solution to the proposed stochastic optimization problem. 
%Stochastic complementarity systems present unique challenges for control as it is difficult to propagate uncertainty.
%\djnote{Change the discussion from SMPC to chance constrained optimization}. \\
Chance-constrained optimization (CCO) has been extensively studied in the control of uncertain systems~\cite{4739221, blackmore2009convex, 5477242, 9143595, nakka2021trajectory, 9113247}. It allows us to plan using the uncertainty in the model by propagating the uncertainty which can be then used to design a controller for desired performance constraints of the system. However, in practice, the CCO techniques, \textcolor{black}{based on the analytical form of chance constraints}, impose restrictive assumptions of Gaussian uncertainty and linear constraints.
% 
Further, state uncertainty increases with time and thus finding a controller for satisfying tighter state constraints could be infeasible over a long planning horizon. This is often the case in control of nonlinear systems with large uncertainty. This problem is aggravated for contact-rich systems due to the presence of discontinuities in system dynamics.
 
To circumvent these challenges, we consider particle-based method for uncertainty propagation and explicit covariance control of our contact-rich system during optimization.
%\textcolor{black}{Another challenge is the derivation of analytical expression for joint chance constraints over a trajectory (see \cite{prekopa2003probabilistic, doi:10.1137/070702928, 4739221, pagnoncelli2009sample, blackmore2009convex, shirai2022chance} for the discussion of joint chance constraints). The tractable method to deal with the joint chance constraints is based on Boole's inequality \cite{boole1847mathematical}, which introduces conservatism.}
% Motivated by the discussions above, we present a technique that tries to address them for stochastic contact-rich systems. First, we use particle-based uncertainty propagation to have the exact uncertainty propagation for the underlying dynamical system. Then we formulate the problem for covariance steering of the underlying system as a nonlinear optimization problem. 

\textbf{Contributions.} 
% The proposed work has the following contributions:
\begin{enumerate}
    % \item We present analysis of stochastic complementarity systems to show how uncertainty in state or complementarity variables affect each other.
    \item We present a novel formulation of covariance steering for complementarity systems using feedforward and feedback controller design.
    \item An important-particle algorithm is proposed for numerical efficiency and we evaluate the proposed method on several examples.
\end{enumerate}
While our motivation is to design robust feedback controllers for manipulation~\cite{9811812,https://doi.org/10.48550/arxiv.2303.08965}, the full problem is out of the scope of the current formulation. Thus, in this current paper, we limit the scope to linear complementarity systems with uncertainty.%, and evaluate the proposed approach on some academic examples. 