\section{Problem Formulation}\label{sec:problem_statement}
In this section, we describe  preliminaries  of the method proposed in the current work. 



% \subsection{Assumptions}
% We have the following assumptions in this work:
% \begin{enumerate}
% \item The dynamics belongs to SDLCS.
% \item Matrix $F$ is P-matrix \cite{lcpbook}.
% \end{enumerate}

% \subsection{Discrete-time Linear Complementarity System (DLCS)}\label{sec:define_lcp}

% A DLCS is a discrete-time linear dynamical system with complementarity constraints~\cite{RaghunathanLinderoth} given by:
% \begin{subequations}
% \begin{flalign}
% x_{k+1}=\bar{A} x_k+B u_k+\bar{C} \lambda_{k+1}+\bar{g} \label{dlcs.dyn} \\
% 0 \leq \lambda_{k+1} \perp \bar{D} x_k+  E u_k+{C} \lambda_{k+1}+\bar{h} \geq 0 \label{dlcs.lcp}
% \end{flalign}\label{dlcs}
% \end{subequations}
% where $k$ is the time-step index, $x_k\in \mathbb{R}^{n_{x}}$ is the state, $u_k\in \mathbb{R}^{n_{u}}$ is the control input, and $\lambda_k \in \mathbb{R}^{n_{c}}$ is the algebraic variable (e.g., contact forces). In addition, $\bar{A} \in \mathbb{R}^{n_{x} \times n_{x}}$, $B \in \mathbb{R}^{n_{x} \times n_{u}}$, $\bar{C} \in \mathbb{R}^{n_{x} \times n_{c}}$, $\bar{g} \in \mathbb{R}^{n_{x}}$, $\bar{D} \in \mathbb{R}^{n_{c} \times n_{x}}$, $E \in \mathbb{R}^{n_{c} \times n_{u}}$, ${C} \in \mathbb{R}^{n_{c} \times n_{c}}$, and $\bar{h} \in \mathbb{R}^{n_{c}}$. 
% % The $i$-th element of vector $p_k$ ($p_k$ can be $x_k, u_k, \lambda_k$) is represented as $p_{k,i}$.
% % The $i$-th diagonal element of matrix $P_k$ is represented as $P_{k,ii}$.
% The notation $0 \leq a \perp b \geq 0$ denotes the complementarity constraints $a \geq 0, b \geq 0, a b=0$.

% Given a $x_k,u_k$, an unique solution $\lambda_{k+1}$ to~\eqref{dlcs.lcp} exists if ${C}$ is P-matrix \cite{lcpbook}. If $\bar{F}$ does not satisfy the P-matrix property, it is possible that $\lambda_{k+1}$ satisfying~\eqref{dlcs.lcp} is non-unique or non-existent. 



\subsection{Stochastic Discrete-time Linear Complementarity Systems}\label{SDLCS_sec}
% In real world, there are many uncertainty sources such as modeling error and uncertain parameter identifications. To consider this effect, 
In this work, we consider the Stochastic Discrete-time Linear Complementarity Systems (SDLCS):
\begin{subequations}
\begin{align}
x_{k+1}=&{A}_k(\xi) x_k+B_k u_k+{C}_k(\xi) \lambda_{k+1}+{g}_k(\xi) \nonumber \\
&+ w_k(\xi) \label{slcp1} \\
0 \leq \lambda_{k+1} \perp& {D}_k(\xi) x_k+E_k u_k+{F}_k(\xi) \lambda_{k+1}+{h}_k(\xi) \nonumber \\
& + l_k(\xi) \geq 0 \label{slcp2}
\end{align}
\label{SDLCS_equations}
\end{subequations}
where $k$ is the time-step index, $x_k\in \mathbb{R}^{n_{x}}$ is the state, $u_k\in \mathbb{R}^{n_{u}}$ is the control input, and $\lambda_k \in \mathbb{R}^{n_{c}}$ is the algebraic variable (e.g., contact forces).
We define $x = [x_1, \ldots, x_T], u = [u_0, \ldots, u_{T-1}], \lambda = [\lambda_1, \ldots, \lambda_{T}]$. 
The parameter $\xi \thicksim \Xi$ is the uncertain parameter with distribution $\Xi$. In addition, ${A}_k(\xi) \in \mathbb{R}^{n_{x} \times n_{x}}$, $B_k \in \mathbb{R}^{n_{x} \times n_{u}}$, ${C}_k(\xi) \in \mathbb{R}^{n_{x} \times n_{c}}$, ${g}_k(\xi) \in \mathbb{R}^{n_{x}}$, ${D}_k(\xi) \in \mathbb{R}^{n_{c} \times n_{x}}$, $E_k \in \mathbb{R}^{n_{c} \times n_{u}}$, ${F}_k(\xi) \in \mathbb{R}^{n_{c} \times n_{c}}$, and ${h}_k(\xi) \in \mathbb{R}^{n_{c}}$ are all dependent on the uncertain parameter $\xi$.  For simplicity, we abbreviate $\xi$ from these matrices for the discussion in the following sections. 
% The $i$-th element of vector $p_k$ ($p_k$ can be $x_k, u_k, \lambda_k$) is represented as $p_{k,i}$.
% The $i$-th diagonal element of matrix $P_k$ is represented as $P_{k,ii}$.
The notation $0 \leq a \perp b \geq 0$ denotes the complementarity constraints $a \geq 0, b \geq 0, a b=0$. The initial state of the system $x_0(\xi)$ is also assumed to be uncertain. $\left\|x\right\|_Q^2$ means a quadratic term with a weighting matrix $Q$.

In the following, we make the assumption that $F_k(\xi)$ is a P-matrix~\cite{lcpbook} for all $k$ and $\xi$. Under this assumption, there is an unique solution $\lambda_{k+1}$ to~\eqref{slcp2} for each $\xi$ and any $u_k, x_k$.  From this it is easy to infer that there exists an unique trajectory $x$ and $\lambda$ for any realization of uncertainty $\xi \thicksim \Xi$ and controls $u$ from every initial condition $x_0(\xi)$.  In other words, we can define functions $\mathbf{x} : \Xi \times \mathbb{R}^{n_u(T-1)} \rightarrow \mathbb{R}^{n_xT}$ and $\boldsymbol{\lambda} : \Xi \times \mathbb{R}^{n_uT}$ that provides the unique trajectory given a realization of uncertainty, and the controls trajectory. Note that we do not show explicit dependence on initial condition due to the dependence of $x_0$ on the uncertain parameter $\xi$.

%, there is an unique $\lambda_{k+1}$ $
%the quantities $x_{k+1}, \lambda_{k+1}$
%then is is easy to show the  that there is an unique trajectory starting from all initial conditions $x_0$. 
%Given a $x_k,u_k$, an unique solution $\lambda_{k+1}$ to~\eq{slcp2} exists if $\bar{F}$ is P-matrix \cite{lcpbook}. If ${F}$ does not satisfy the P-matrix %property, it is possible that $\lambda_{k+1}$ satisfying~\eq{slcp2} is non-unique or non-existent. 
% For notation simplicity, we denote $y_k$ as $y_k = {D}_k x_k+E_k u_k+{F}_k \lambda_{k+1}+{h}_k + l_k$. 
% In SDLCS, we consider that uncertainty arises from system parameters (i.e., ${A}_k, {C}_k, {g}_k, {D}_k,  {F}_k, {h}_k$), additive noises 
% $w_{k} \in \mathbb{R}^{n_{x}}, v_{k} \in \mathbb{R}^{n_{c}}$, and $x_0$.
%  \textcolor{red}{For notation simplicity, we use a random vector $\xi$ with support $\Xi \subseteq \mathbb{R}^d$ to represent these uncertain parameters in \eq{SDLCS_equations}.}
%  We denote $\theta_k = [{A}_k, B_k, \bar{C}_k, \bar{g}_k, \bar{D}_k, E_k, \bar{F}_k, {h}_k,  w_{k},  v_{k}]$ as a set of uncertain parameters.
%  
%  This paper considers the case where uncertainty arises from $x_0, \omega_k, l_k$ and parameters $\theta_k \in \mathbb{R}^{n_{\theta}}$ defined over $\bar{A}, B, \bar{C}, \bar{g}, \bar{D}, E, \bar{F}, {h}$. $n_\theta$ is the number of uncertain parameters.
% Thus,  $x_k$ and $\lambda_k$ are random variables. 
%In addition, we assume that $\bar{F}_k$ is P-matrix, which enables SDLCS to roll out a unique solution. % even under uncertainty because as we explained above, $\lambda_{k+1}$ can be non-unique or non-existent. 
% It means that it is not clear which $\lambda_{k+1}$ the system rolls out, which makes designing controllers extremely difficult. 
% This assumptions simplifies the complementarity constraints in \eq{slcp2} as follows:
% \begin{equation}
%     \lambda_{k+1} = \max{\left(0, -\bar{F}^{-1}\left(\bar{D} x_k+E u_k+{h} + l_k\right)\right)}
% \end{equation}
% \textcolor{red}{YS: We should discuss why non-P matrix is difficult.}


\subsection{Stochastic Control for Contact-Rich Systems}
% \textcolor{red}{YS: Maybe we can move this subsection to previous section and in this section we can focus discussing particle-baesd controller.} 
% In this work, we use particles to approximate the distribution of states and algebraic variables to design controllers for complementarity systems. First, we would like to formulate the general optimization problem:
In this work, we aim at finding a robust controller that satisfies chance constraints over SDLCS. To realize this, the following optimization problem can be formulated:
\begin{subequations}\label{equation_control}
\begin{flalign}
\min _{u}  &\;  \sum_{k=1}^{T}
\left\| \mathbb{E}_{\xi \thicksim \Xi} \left[\mathbf{x}_{k}(\xi,u)\right] - x_d \right\|_Q^2
% \left(\mathbf{x}_{k}(\xi,u) - x_d\right)^{\top} Q \left(\mathbf{x}_{k}(\xi,u) - x_g \right)
 + \sum\limits_{k=0}^{T-1}
\left\| u_k \right\|_R^2
% u_{k}^{\top} R u_{k} 
\label{exp_cost}\\
\text{s.t.}  &\; u_k \in \mathcal{U} \label{control_bnds} \\
&\; \text{Pr}_{\xi \thicksim \Xi}\left(\mathbf{x}(\xi,u) \in \mathcal{X}\right) \geq \Delta \label{chance_const}
\end{flalign}
\end{subequations}
where $Q=Q^{\top}$ is positive semidefinite, $R=R^{\top}$ is positive definite, $\mathcal{U}$ is a convex polytope consisting of a finite number of linear inequality constraints. 
$x_d$ is the target state at $t = T$.
The set $\mathcal{X}$ represents a convex safe region where the entire state trajectory has to lie in.  We assume that $\mathcal{X} = \{ x \in \mathbb{R}^{n_xT} \,|\, g_i(x) \leq 0 \,\forall\, i = 1,\ldots,n_g\}$. $\text{Pr}$ denotes the probability of an event and $\Delta$ is the user-defined minimum safety probability, where the probability of satisfying constraints is at least greater than $\Delta$.


%\begin{equation}
%    \mathbf{x}(\xi,u)\in \mathcal{X} \Longleftrightarrow \bigwedge_{m=1, \ldots, N_m} f_m(x_{1:T}) \leq 0
%\end{equation}
%where $\bigwedge_{m=1, \ldots, N_m} f_m(x_{1:T}) \leq 0$ represents $\mathcal{X}$. $N_m$ is the total number of inequality constraints associated with $\mathcal{X}$.
%Since $x_{1:T}$ is the trajectory of the random variable, we formulate chance constraints for $x_{1:T}$ over  $\mathcal{X}$  where Pr denotes the probability of an event. 

We propose to obtain an approximate solution to~\eqref{equation_control} using the Sample Average Approximation (SAA) introduced in~\cite{doi:10.1137/070702928, pagnoncelli2009sample}. We explain more details in Sec~\ref{sec:cov_control}. 
