The eigenvalue decay rate (or capacity condition, effective dimension condition) and source condition are mentioned in almost all related literature studying the convergence behaviors of kernel methods but are denoted as various kinds of notations. Recall that in this paper we denote the eigenvalue decay rate as $\beta$ and denote the source condition as $s$. Table \ref{table1} summarizes the notations in some of the references.

\begin{table}[ht]
\centering
\begin{tabular}{ccc} 
\toprule
$\beta$  &   $s$ & Reference \\
\midrule
 $1/p$       &  $\beta$   & \citet{steinwart2009_OptimalRates,fischer2020_SobolevNorm,Li2022OptimalRF}  \\
 $1/\gamma$  &  $2\zeta$ & \citet{lin2018_OptimalRates,lin2020_OptimalConvergence}  \\
 $b$         &  $c$      & \citet{caponnetto2006optimal,Caponnetto2007OptimalRF}  \\
 $-$         &  $2 r$    & \citet{bauer2007_RegularizationAlgorithms,Smale2007LearningTE,gerfo2008_SpectralAlgorithms}  \\
 $2 \nu$     &  $\zeta+1$ & \citet{dicker2017_KernelRidge}  \\
 $b$        &  $2r+1$ & \citet{rastogi2017_OptimalRates,blanchard2018_OptimalRates}  \\
 $1/b$  &  $2\beta$ & \citet{Jun2019KernelTR} \\
 $\alpha$  &  $2r$ & \citet{Dieuleveut2016NONPARAMETRICSA,PillaudVivien2018StatisticalOO,Celisse2020AnalyzingTD} \\
\bottomrule
\end{tabular}
\caption{A dictionary of notations in related literature.}
\label{table1}
\end{table}

%  Rosassco; divide and conque; fourier capacity;