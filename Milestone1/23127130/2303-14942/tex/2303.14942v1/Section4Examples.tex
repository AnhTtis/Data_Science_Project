We prove the minimax optimality of spectral algorithms for $\alpha_0 - \frac{1}{\beta} < s \le 2 \tau$ in the last section. Therefore the embedding index $\alpha_{0}$ of an RKHS is crucial when analyzing the optimality of the spectral algorithms. In the best case of $\alpha_{0} = \frac{1}{\beta}$, only the first situation in Theorem \ref{main theorem} exists and we obtain the optimality for all $ 0 < s \le 2 \tau$. In this section, we give several examples of RKHSs with embedding index $\alpha_{0} = \frac{1}{\beta}$.

\subsection{RKHS with uniformly bounded eigenfunctions}\label{section examples ui}
RKHS with uniformly bounded eigenfunctions, i.e., $\sup_{i \in N}  \| e_{i} \|_{L^{\infty}} < \infty$, are frequently considered \citep{mendelson2010_RegularizationKernel, steinwart2009_OptimalRates, PillaudVivien2018StatisticalOO}. \citet[Lemma 10]{fischer2020_SobolevNorm} has proved that this kind of RKHS satisfies $\alpha_0 = \frac{1}{\beta}$.



\subsection{Sobolev RKHS}\label{section examples sobolev}

Let us first introduce some concepts of (fractional) Sobolev space (see, e.g., \citealt{adams2003_SobolevSpaces}). In this section, we assume that $\mathcal{X} \subseteq \mathbb{R}^{d}$ is a bounded domain with smooth boundary and Lebesgue measure $\nu$. Denote $L^{2}(\mathcal{X}) \coloneqq L^{2}(\mathcal{X},\nu)$ as the corresponding $L^{2}$ space. For $m \in \mathbb{N}$, we denote the usual Sobolev space $W^{m,2}(\mathcal{X})$ by $H^{m}(\mathcal{X})$ and $L^{2}(\mathcal{X})$ by $H^{0}(\mathcal{X})$. Then the (fractional) Sobolev space for any real number $r >0 $ can be defined through the \textit{real interpolation}:
\begin{displaymath}
      H^{r}(\mathcal{X}) := \left(L^{2}(\mathcal{X}), H^{m}(\mathcal{X})\right)_{\frac{r}{m},2},
  \end{displaymath}
where $m:=\min \{k \in \mathbb{N}: k > r\}$. (We refer to Appendix A for the definition of real interpolation and \citealt[Chapter 4.2.2]{sawano2018theory} for more details). It is well known that when $r > \frac{d}{2}$, $H^{r}(\mathcal{X})$ is a separable RKHS with respect to a bounded kernel and the corresponding EDR is (see, e.g., \citealt{edmunds_triebel_1996}) 
\begin{displaymath}
  \beta = \frac{2 r}{d}.
\end{displaymath}
Furthermore, for the interpolation space of $H^{r}(\mathcal{X}) $ under Lebesgue measure defined by \eqref{def interpolation space}, \eqref{inter relation} shows that for $ s > 0$,
\begin{displaymath}
  [H^{r}(\mathcal{X})]^{s} = H^{rs}(\mathcal{X}).
\end{displaymath}

The embedding theorem of (fractional) Sobolev space (see, e.g., 7.57 of \citealt{adams1975sobolev}) shows that if $ d < 2(r-j)$ for some nonnegative integer $j$, then 
\begin{displaymath}
    H^{r}(\mathcal{X}) \hookrightarrow C^{j,\theta}(\mathcal{X}), ~~ \theta= r-j-\frac{d}{2},
\end{displaymath}
where $C^{j,\gamma}(\mathcal{X}) $ denotes the Hölder space and $\hookrightarrow $ denotes the continuous embedding. Therefore for a Sobolev RKHS $ \mathcal{H} = H^{r}(\mathcal{X}), r > \frac{d}{2}$ and any $\alpha > \frac{1}{\beta} = \frac{d}{2r}$,
\begin{displaymath}
    [H^{r}(\mathcal{X})]^{\alpha} = H^{r\alpha}(\mathcal{X}) \hookrightarrow C^{0,\theta}(\mathcal{X}) \hookrightarrow L^{\infty}(\mathcal{X}),
\end{displaymath}
where $\theta > 0$. So the embedding index of a Sobolev RKHS is $ \alpha_{0} = \frac{1}{\beta}$.

Furthermore, if we suppose that $\mathcal{H}$ is a Sobolev RKHS, i.e., $\mathcal{H} = H^{r}(\mathcal{X})$ for some $r > d/2 $ and the distribution $\rho$ satisfies that the marginal distribution $\mu$ on $\mathcal{X}$ has Lebesgue density $ 0 < c \le p(x) \le C$ for two constants $c$ and $C$. Then we also know that the embedding index is $\alpha_{0} = \frac{1}{\beta} $. Note that we say that the distribution $\mu$ has Lebesgue density $0 < c \le p(x) \le C $, if $\mu$ is equivalent to the Lebesgue measure $\nu$, i.e., $ \mu \ll \nu, \nu \ll \mu$ and there exist constants $c, C > 0$ such that $c \leq \frac{\mathrm{d} \mu}{\mathrm{d} \nu} \leq C$.

% Note that the RKHS $\mathcal{H}$ is defined as the (fractional) Sobolev space $H^{r}(\mathcal{X})$, which is regardless of the marginal distribution $\mu$. But the definition of interpolation space \eqref{def interpolation space} is dependent on $\mu$. When $\mu$ has Lebesgue density $ 0 < c \le p(x) \le C$, \citet[(14)]{fischer2020_SobolevNorm} shows that 
% \begin{equation}
%     L^{2}(\mathcal{X},\mu) \cong  L^{2}(\mathcal{X},\nu),
% \end{equation}
% and
% \begin{equation}
%     \left[H^r(\mathcal{X})\right]_\mu^{s} \cong\left(L_2(\mathcal{X},\mu),\left[H^r(\mathcal{X})\right]^{1}_\mu\right)_{s, 2} \cong \left(L_2(\mathcal{X},\nu),\left[H^r(\mathcal{X})\right]^{1}_\nu\right)_{s, 2} \cong \left[H^r(\mathcal{X})\right]_{\nu}^{s} \cong H^{rs}(\mathcal{X}),
% \end{equation}
% where we denote $\left[H^r(\mathcal{X})\right]_\mu^{s} $ as the interpolation space of $H^r(\mathcal{X})$ under marginal distribution $\mu$. So we know the embedding index is still $\alpha_{0} = \frac{1}{\beta} $.
\subsection{RKHS with shift-invariant periodic kernels}\label{section examples invar}
Let us consider a kernel on $\mathcal{X} = [-\pi,\pi)^d$ satisfying 
\begin{align*}
    k(x,y) = g\big( (x - y) \bmod [-\pi,\pi)^d\big),
\end{align*}
where we denote
\begin{align*}
    a \bmod [-\pi,\pi) = \left[(a+\pi)\bmod 2\pi \right] - \pi~ \in [-\pi,\pi),
\end{align*}
and
\begin{align*}
    (a_1,\dots,a_d)\bmod [-\pi,\pi)^d = \big(a_1 \bmod [-\pi,\pi),\dots, a_d \bmod [-\pi,\pi)\big).
\end{align*}
We further assume that $\mu$ is the uniform distribution on $[-\pi,\pi)^d$.
Then, it is shown in \citet{beaglehole2022_KernelRidgeless} that the Fourier basis $\phi_{\bm{k}}(x) = \exp(i \langle \bm{k},x \rangle)$, $\bm{k} \in \mathbb{Z}^d$ are eigenfunctions of the integral operator $T$.
Since $|{\phi_{\bm{k}}(x)}| \leq 1$, that is, the eigenfunctions are uniformly bounded, we conclude that the embedding index $\alpha_0 = \frac{1}{\beta}$. We refer to Section \ref{section proof invariant} for more details.


\subsection{RKHS with dot-product kernels}\label{section examples sphere}
Dot-product kernels, which satisfy $k(x,y) = f(\langle x,y \rangle)$, have also raised researchers' interest in recent years for its nice property~\citep{smola2000_RegularizationDotproduct,cho2009_KernelMethods,bach2017_BreakingCurse,jacot2018_NeuralTangent}.
Let $k$ be a dot-product kernel on $\mathcal{X} = \mathbb{S}^{d}$, 
the unit sphere in $\mathbb{R}^{d+1}$, and $\mu = \sigma$ be the uniform measure on $\mathbb{S}^{d}$.
Then, it is well-known that $k$ can be decomposed as
\begin{displaymath}
  k(x,y) = \sum_{n=0}^{\infty} \mu_n \sum_{l=1}^{a_n} Y_{n,l}(x)Y_{n,l}(y),
\end{displaymath}
where $\{Y_{n,l}\}$ is a set of orthonormal basis of $L^2(\mathbb{S}^{d},\sigma)$ called the spherical harmonics.
If polynomial decay condition $\mu_n \asymp n^{-d \beta}$ is satisfied (which is equivalent to assume the eigenvalue decay rate is $\beta$), 
 Proposition \ref{prop:EMBIdx_Sphere} shows that the embedding index $\alpha_0 = \frac{1}{\beta}$ for the corresponding RKHS. We refer to Section \ref{section proof sphere} for more details.

% To investigate the embedding property for such kernels, let us assume a  , 
% which is equivalent to the EDR condition of $\lambda_i \asymp i^{-\beta}$ since we note that $\mu_n$ has a multiplicity of $a_n \asymp n^{d-1}$.
% Then,

% This justifies that RKHS corresponding to dot-product kernels with polynomial eigenvalue decays 
% on the sphere.



