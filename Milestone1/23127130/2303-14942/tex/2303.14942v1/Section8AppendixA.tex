In this appendix, we introduce some useful results of real interpolation and Lorentz spaces \cite[Chapter 22-26]{tartar2007introduction}.  
% In the following, we always assume the probability measure on $\mathcal{X}$ is $\mu$ and denote $L^{p}(\mathcal{X},\mu) $ as $L^{p}(\mathcal{X})$. 

\subsection*{\textbf{A.1} Real interpolation and the Reiteration theorem}
We first introduce the definition of real interpolation through the K-method. For two normed spaces $ E_{i}, i=0,1$, denote their norms as $\| \cdot \|_{i}, i=0,1$.

\begin{definition}[K-functional]
   Let $E_{0}$ and $E_{1}$ be two normed spaces, continuously embedded into a topological vector space $\mathcal{E}$ (($E_{0}$, $E_{1}$) is a compatible couple). For $a \in E_{0} + E_{1}$ and $ t >0$, define the K-functional by
   \begin{displaymath}
       K(t ; a)=\inf _{a=a_0+a_1}\left(\left\|a_0\right\|_0+t\left\|a_1\right\|_1\right).
   \end{displaymath}
\end{definition}

\begin{definition}[Real interpolation]
   Let $E_{0}$ and $E_{1}$ be two normed spaces, continuously embedded into a topological vector space $\mathcal{E}$ (($E_{0}$, $E_{1}$) is a compatible couple). For $0 < \theta < 1$ and $1 \le  p \le \infty$ (or for $\theta =0, 1$ with $p = \infty$), the real interpolation space is defined by
   \begin{displaymath}
      \left(E_0, E_1\right)_{\theta, p}=\left\{a \in E_0+E_1 \mid t^{-\theta} K(t ; a) \in L^p\left(\mathbb{R}^{+} ; \frac{\mathrm{d} t}{t}\right)\right\},
   \end{displaymath}
   with the norm
   \begin{displaymath}
     \|a\|_{\left(E_0, E_1\right)_{\theta, p}}=\left\|t^{-\theta} K(t ; a)\right\|_{L^p(\mathbb{R}^{+} ; \mathrm{d} t / t)}.
   \end{displaymath}
   
\end{definition}

\begin{lemma}\label{mono of RI}
  If $ 0 < \theta < 1$ and $ 1\le p \le q \le \infty $, we have 
  \begin{displaymath}
      \left(E_0, E_1\right)_{\theta, p} \subset \left(E_0, E_1\right)_{\theta, q},~~ \text{with continuous embedding}.
  \end{displaymath}
\end{lemma}
  
The following lemma gives the result of exchanging the two spaces $E_{0},E_{1}$.
\begin{lemma}\label{exchanging e0e1}
  One has $(E_{1},E_{0})_{\theta,p} = (E_{0},E_{1})_{1 - \theta, p}$ for $0 < \theta <1$ and $1 \le p \le \infty$; the same result holds for $\theta =0 ~\text{or}~1$, and $ p =1$ or $ p = \infty$.
\end{lemma}

The following Lions–Peetre Reiteration Theorem is an important property of real interpolation spaces. 
\begin{theorem}[Reiteration theorem]\label{reiteration theorem}
  If $ 0 \le \theta_{0} \neq \theta_{1} \le 1$, and the two normed spaces $F_{0}, F_{1}$ satisfy that
  \begin{align}
      \left(E_0, E_1\right)_{\theta_0, 1} &\subset F_0 \subset\left(E_0, E_1\right)_{\theta_0, \infty}; \notag \\
      \left(E_0, E_1\right)_{\theta_1, 1} &\subset F_1 \subset\left(E_0, E_1\right)_{\theta_1, \infty}. \notag
  \end{align}
  Then for $ 0<\theta<1$ and $1 \le p \le \infty$, denote $ \eta = (1-\theta) \theta_{0} + \theta \theta_{1}$, we have 
  \begin{displaymath}
    \left(F_0, F_1\right)_{\theta, p}=\left(E_0, E_1\right)_{\eta, p}, ~~\text{with equivalent norms}.
  \end{displaymath}
\end{theorem}

\begin{remark}
  This theorem implies that, if we replace $F_{0}$ with any space $\widetilde{F_{0}}$ satisfying $ \left(E_0, E_1\right)_{\theta_0, 1} \subset \widetilde{F_0} \subset\left(E_0, E_1\right)_{\theta_0, \infty} $, the real interpolation space remains `unchanged', i.e., $\left(F_0, F_1\right)_{\theta, p} \cong \left(\widetilde{F_0}, F_1\right)_{\theta, p}$. 
\end{remark}


\subsection*{\textbf{A.2} Lorentz space}
\begin{definition}[Lorentz space]
   For $ 1 < p < \infty$ and $1 \le q \le \infty$, the Lorentz space $ L^{p,q}(\mathcal{X},\mu)$ is defined as 
   \begin{displaymath}
       L^{p,q}(\mathcal{X},\mu) = \left( L^{1}(\mathcal{X},\mu), L^{\infty}(\mathcal{X},\mu)\right)_{\frac{1}{p^{\prime}}, q}, 
   \end{displaymath}
   where $\frac{1}{p^{\prime}} + \frac{1}{p} = 1$.
\end{definition}

Using Lemma \ref{mono of RI}, it is easy to show that $L^{p,q_{1}}(\mathcal{X},\mu) \subseteq L^{p,q_{2}}(\mathcal{X},\mu) $ for $ 1 \le q_{1} \le q_{2} \le \infty$. In addition, the following lemma gives the relation between Lorentz space and $L^{p}$ space.
\begin{lemma}\label{cong of lorentz}
  For $ 1 <p<\infty$, we have
  \begin{displaymath}
      L^{p,p}(\mathcal{X},\mu) \cong L^{p}(\mathcal{X},\mu); \quad L^{p,\infty}(\mathcal{X},\mu) \cong L^{p,w}(\mathcal{X},\mu),
  \end{displaymath}
  where $L^{p,w}(\mathcal{X},\mu)$ denotes the weak $L^{p}$ space.
\end{lemma}