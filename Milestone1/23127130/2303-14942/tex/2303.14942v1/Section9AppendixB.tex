\begin{lemma}\label{basic ineq}
  For any $\lambda > 0$ and $ s \in [0,1]$, we have
  \begin{displaymath}
      \sup _{t \geq 0} \frac{t^s}{t+\lambda} \leq \lambda^{s-1}.
  \end{displaymath}
\end{lemma}

\begin{proof}
  Since $ a^{s} \le a + 1$ for any $a \ge 0$ and $s \in [0,1] $, the lemma follows from
  \begin{displaymath}
      \left(\frac{t}{\lambda}\right)^{s} \le \frac{t}{\lambda} + 1 = \frac{t + \lambda}{\lambda}.
  \end{displaymath}
\end{proof}
\begin{lemma}\label{lemma of effect}
    If $\lambda_i \asymp i^{-\beta}$, we have
    \begin{align}
        \mathcal{N}(\nu) \asymp \nu^{\frac{1}{\beta}}. \notag
    \end{align}

\end{lemma}
\begin{proof}
    Since $c ~i^{-\beta} \leq \lambda_i \leq C i^{-\beta}$, we have
    \begin{align}
        \mathcal{N}(\nu) &= \sum_{i = 1}^{\infty}  \frac{\lambda_i}{\lambda_i + \nu^{-1}}  
        \leq \sum_{i = 1}^{\infty} \frac{C i^{-\beta}}{C i^{-\beta} + \nu^{-1}}  = \sum_{i = 1}^{\infty}  \frac{C }{C+ \nu^{-1} i^{\beta}} \notag \\
        &\leq \int_{0}^{\infty}  \frac{C }{\nu^{-1} x^{\beta} + C}  \mathrm{d} x
        %\qq{(Let $\lambda^\beta x = y$)} 
        = \nu^{\frac{1}{\beta}} \int_{0}^{\infty}  \frac{C }{y^{\beta} + C} \mathrm{d} y \leq C_{1} \nu^{\frac{1}{\beta}}. \notag
    \end{align}
    for some constant $C_{1}$. Similarly, we can prove 
    \begin{displaymath}
    \mathcal{N}(\nu) \geq C_{2} \nu^{\frac{1}{\beta}},
    \end{displaymath}
    for some constant $C_{2}$.
\end{proof}

The following concentration inequality about self-adjoint Hilbert-Schmidt operator
valued random variables is frequently used in related literature, e.g., \citet[Theorem 27]{fischer2020_SobolevNorm} and \citet[Lemma 26]{lin2020_OptimalConvergence}.
\begin{lemma}\label{lemma concentration of operator}
   Let $(\mathcal{X}, \mathcal{B}, \mu)$ be a probability space, $\mathcal{H}$ be a separable Hilbert space. Suppose that $ A_{1}, \cdots, A_{n}$ are i.i.d. random variables with values in the set of self-adjoint Hilbert-Schmidt operators. If  $\mathbb{E} A_{i} = 0$, and the operator norm $ \| A_{i} \| \le L ~~ \mu \text {-a.e. } x \in \mathcal{X}$, and there exists a self-adjoint positive semi-definite trace class operator $V$ with $\mathbb{E} A_{i}^{2} \preceq V $. Then for $\delta \in (0,1)$, with probability at least $1 - \delta$, we have 
   \begin{align}
        \left\| \frac{1}{n}\sum_{i=1}^n A_i \right\|
        \leq \frac{2L\beta}{3n} + \sqrt {\frac{2 \| V \| \beta}{n}},\quad
        \beta = \ln \frac{4 \rm{tr} V}{\delta \| V \|}. \notag
   \end{align}
\end{lemma}

The following Bernstein inequality about vector-valued random variables is frequently used, e.g., \citet[Proposition 2]{Caponnetto2007OptimalRF} and \citet[Theorem 26]{fischer2020_SobolevNorm}.
\begin{lemma}[Bernstein inequality]\label{bernstein}
   Let $(\Omega,\mathcal{B},P)$ be a probability space, $H$ be a separable Hilbert space, and $\xi: \Omega \to H$ be a random variable with 
   \begin{displaymath}
     \mathbb{E}\|\xi\|_H^m \leq \frac{1}{2} m ! \sigma^2 L^{m-2},
   \end{displaymath}
   for all $m>2$. Then for $\delta \in (0,1)$, $\xi_{i}$ are i.i.d. random variables, with probability at least $1 - \delta$, we have
   \begin{displaymath}
       \left\|\frac{1}{n} \sum_{i=1}^n \xi_{i} - \mathbb{E} \xi\right\|_H \le 4\sqrt{2} \ln{\frac{2}{\delta}} \left(\frac{L}{n} + \frac{\sigma}{\sqrt{n}}\right).
   \end{displaymath}
\end{lemma}

\begin{lemma}[Cordes inequality]\label{cordes}
Let $A$ and $B$ be two positive bounded linear operators on a separable Hilbert space. Then we have
\begin{displaymath}
\left\|A^s B^s\right\| \leq\|A B\|^s, \quad \text { when } 0 \leq s \leq 1.
\end{displaymath}
\end{lemma}

The following lemma is a corollary of \citet[Lemma 5.8]{lin2018_OptimalRates}.
\begin{lemma}\label{lemma phi operator}
  Suppose that $A$ and $B$ are two positive self-adjoint operators on some Hilbert space, then 
  \begin{itemize}
      \item for $ r \in (0,1]$, we have
      \begin{displaymath}
        \left\|A^r-B^r\right\| \leq\|A-B\|^r.
      \end{displaymath}
      \item for $ r \ge 1 $, denote $c=\max (\|A\|,\|B\|)$, we have
      \begin{displaymath}
        \left\|A^r-B^r\right\| \leq r c^{r-1}\|A-B\|.
      \end{displaymath}
  \end{itemize}
\end{lemma}



