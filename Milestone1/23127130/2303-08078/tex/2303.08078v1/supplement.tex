\documentclass[%
 preprint,%
 amsmath,%
 amssymb,%
 aps,
]{revtex4-2}

\usepackage{graphicx}
\graphicspath{{./figs}}
\usepackage{mathtools}
\usepackage{bm}
\usepackage[dvipsnames]{xcolor}
\usepackage[colorlinks,%
            pdfusetitle,%
            bookmarksopen=false,%
            urlcolor=Blue,%
            citecolor=Blue,%
            linkcolor=Blue,%
            pdfencoding=auto,%
            psdextra]{hyperref}
\usepackage{physics}
\usepackage{upgreek}

\linespread{1.1}

%%%%%%%%%%%%%%%%%%%%%%%%%%%%%%%%%%%

\newcommand{\asciimathunit}[1]{\ensuremath{\,\mathrm{#1}}}
\newcommand{\us}{\ensuremath{\,\upmu \mathrm{s}}}
\newcommand{\ms}{\asciimathunit{ms}}
\newcommand{\subfiglabel}[1]{\textbf{#1},}
\newcommand{\subfigref}[1]{#1}

\renewcommand{\theequation}{S.\arabic{equation}}
\renewcommand{\thefigure}{S\arabic{figure}}

%%%%%%%%%%%%%%%%%%%%%%%%%%%%%%%%%%% 

\begin{document}

\title{Supplementary information: Realizing spin squeezing with Rydberg interactions in a programmable optical clock}

\author{William J. Eckner, Nelson \surname{Darkwah Oppong}, Alec Cao, Aaron W. Young, William R. Milner, John M. Robinson, Jun Ye, Adam M. Kaufman}

\affiliation{%
 JILA, University of Colorado and National Institute of Standards and Technology,
and Department of Physics, University of Colorado, Boulder, Colorado 80309, USA
}%

\date{\today}

\maketitle
\tableofcontents
\newpage

\section{Characterization of local $\hat{\sigma}_z$ operations}

In order to minimize added technical noise, the local light-shifting protocol presented in Fig.~4a and the Methods should be the same across the atom array (spatially homogeneous), and consistent between different experiments (temporally homogeneous).
To characterize spatial homogeneity, we measure site-resolved Ramsey fringes, and analyze the distribution of fitted phases, shown in Fig.~\ref{fig:phaseshift}a.
We benchmark the temporal homogeneity of the the protocol by performing an atom-atom stability measurement for $5 \times 14$ subarrays with the tweezers applied during a short dark time.
Because the ensembles are phase shifted relative to each other, the phase of the final $\pi/2$-pulse is chosen such that the average $S_z^{(A)} + S_z^{(B)} = 0$. Results of this measurement are shown in \ref{fig:phaseshift}b. 
We note that any ellipse measurements utilizing $\hat{\sigma}_z$ operations alternate experimental shots where the Rydberg laser is applied or not, and thereby interleave data acquisition for the SSS and CSS.
As a result, the SSS and CSS should experience similar systematic effects, such as any slow residual drift of the applied phase offset between subarrays. 

\begin{figure}
    \includegraphics[width=0.67\textwidth]{fig_phase_shift.pdf}
    \caption{\label{fig:phaseshift} \textbf{Characterization of tweezer phase shifting homogeneity and stability.} \subfiglabel{a}~Site-resolved Ramsey fringe phases for two $N=5 \times 14$ subarrays with the phase shifting protocol described in Fig.~4\subfigref{a} applied during the dark time to ensemble $A$.
    For each site $i$, $\theta_i$ is obtained by fitting $P_i \sim \sin (\theta_L + \theta_i)$.
    The lower panel shows a histogram of the difference of $\theta_i$ from the relevant average subarray phase $\theta_{A}$ and $\theta_{B}$.
    The standard deviation of the $\theta_i$ distribution for ensemble $A$ ($B$) is $3.8^\circ$ ($3.6^\circ$).
    \subfiglabel{b}~Overlapping Allan deviation for the applied phase shift.
    The laser phase $\theta_L$ is set so that $(P_A + P_B)/2 \approx 1/2$ on average.
    The deviation is computed for $2 d_z^{(AB)}/(\phi C)$; here, $\phi$ and $C$ are computed from the measurement in \textbf{a}, with $\phi = \theta_A - \theta_B = 33.2(1.3)^{\circ}$ and $C = 0.971(2)$ obtained from averaging the contrasts of subarrays $A$ and $B$.}
\end{figure}

\section{Signatures of collective dissipation}

\begin{figure}
\includegraphics[width=0.67\linewidth]{fig_coll_loss.pdf}
\caption{
\label{fig:coll_loss}
\textbf{Emerging collective effects at late times.}
As in the main text, we denote the excitation fraction $P_A, P_B$ for subarray $A$, $B$, respectively. In the left (right) panel we parametrically plot the excitation fractions $P_A$ and $P_B$, measured after a total interaction time $t_{\text{int}} = 10 \, \mu \rm{s}$ ($20 \, \mu \rm{s}$).
For this measurement, we determine the excitation fraction immediately after the second Rydberg-interaction pulse (see Fig.~1c in the main text), and with two $N=5\times 14$ subarrays. 
%
}
\end{figure}

When increasing the interaction time to $t_\mathrm{int}\geq 10\us$ -- a regime beyond the typical optima for preparing SSSs -- signs of a bimodal distribution emerge in the atomic excitation fraction for two $N=5\times 14$ subarrays.
By plotting the excitation fractions $P_A, P_B$ for two subarrays $A$ and $B$, respectively, as shown in Fig.~\ref{fig:coll_loss}, we can further see a bimodal distribution emerge in $P_A$, even when controlling for $P_B$ (and vice-versa).
This observation indicates the presence of collective effects and, therefore, appears to be consistent with the presence of collective dissipation phenomena, which have been studied in similar Rydberg-atom-array platforms~\cite{zeiher2016many, boulier2017spontaneous, young2018dissipation, guardado2021quench, festa2022blackbody}.


\section{Derivation of QPN and $\xi_W^2$}

In this section, we derive expressions for quantum projection noise (QPN) and the Wineland parameter presented in the main text. To begin, $S_z / N$ for two ensembles~$A$ and~$B$ in a Ramsey-style measurement will be given by
\begin{align}
\begin{split}
    \frac{S_z^{(A)}}{N} &= \frac{C}{2} \sin(\theta + \pi/2) + y_A \\
    \frac{S_z^{(B)}}{N} &= \frac{C}{2} \sin(\theta + \phi + \pi/2) + y_B
\end{split}
\end{align}
where we refer to $\theta$ as the atom-laser phase, and $\phi$ as the differential phase. We have furthermore assumed that both ensembles have the same contrast $0 \le C \le 1$, but potentially different offsets $C-1 \le 2\,  y_A, 2\, y_B \le 1 - C$. Finally, $N$ refers to the atom number in each ensemble, which we take to be equal ($N_A = N_B = N$).

We are interested in the noise in the measurement of $\hat{d}_z^{(AB)}$, as defined in the main text. To motivate this decision, we show that this provides a measurement of the differential phase~$\phi$. Writing out the expectation value of $\hat{d}_z^{(AB)}$ gives
\begin{align}
\begin{split}
    d_z^{(AB)} = \frac{S_z^{(A)} - S_z^{(B)}}{N} = \frac{C}{2} [\sin(\theta + \pi/2) - \sin(\theta + \phi + \pi/2)] + (y_A - y_B).
\end{split}
\end{align}
Taylor-expanding about the point $(\theta, \phi) = (\pi/2,0)$ yields
\begin{align}
\begin{split}
    d_z^{(AB)} \approx \frac{C}{2} \phi + (y_A - y_B).
\end{split}
\end{align}
The phase uncertainty of our measurement will then be
\begin{align}
\begin{split}
    \Delta \phi &= \Delta \hat{d}_z^{(AB)} {\left| \frac{\text{d} d_z^{(AB)} }{\text{d} \phi} \right|}^{-1}
    = \frac{2 \Delta \hat{d}_z^{(AB)}}{C}.
\end{split}
\end{align}
We now define the quantum-projection-noise limit as the variance of an ideal (contrast $C=1$) coherent spin state $\ket{\text{CSS}}$
\begin{align}
    \ket{\text{CSS}} &= \ket{\theta + \phi}_A \otimes \ket{\theta}_B
\end{align}
where
\begin{align}
\begin{split}
    \ket{\theta}_A &= \bigotimes_{i = 0}^{N-1} [e^{-i\theta/2} \ket{e}_i + e^{i \theta/2} \ket{g}_i] / \sqrt{2} \\
     \ket{\theta + \phi}_B &= \bigotimes_{j = 0}^{N-1} [e^{-i(\theta + \phi)/2} \ket{e}_j + e^{i (\theta + \phi)/2} \ket{g}_j] / \sqrt{2} \, .
\end{split}
\end{align}
and $i$ ($j$) indexes atoms in ensemble $A$ ($B$). Using the notation presented in the main text,
\begin{align}
    \sigma_{\text{QPN}}^2 &= \bra{\text{CSS}} [\hat{d}_z^{(AB)}]^2\ket{\text{CSS}} - [\bra{\text{CSS}} \hat{d}_z^{(AB)} \ket{\text{CSS}}]^2 \\
    &= \frac{1}{2N} \, .
\end{align}
From this, the standard quantum limit on our estimation of $\phi$ is 
\begin{align}
    (\Delta \phi)_{\text{SQL}} = \sqrt{\frac{2}{N}} \, .
\end{align}
Next, we derive an expression for the squeezing parameter, which is typically defined in the context of a Ramsey measurement with a single ensemble of $N$ atoms and an atom-laser phase $\theta$ as \cite{wineland1992spin}
\begin{align}
    \xi_{\theta}^2 = \frac{(\Delta \theta)^2}{(\Delta \theta)^2_{\text{SQL}}} \, .
\end{align}
Here, $\Delta \theta$ is the uncertainty in the measured atom-laser phase, and 
\begin{align}
    (\Delta \theta)_{\text{SQL}} = \sqrt{\frac{1}{N}} \, .
\end{align}
In the context of our differential measurement of $\phi$, we therefore define the parameter $\xi^2_{\phi}$ as 
\begin{align}
\begin{split}
    \xi^2_{\phi} = \frac{(\Delta \phi)^2}{(\Delta \phi)^2_{\text{SQL}}} &=\frac{\sigma_{\alpha}^2}{C^2} \frac{1}{\sigma^2_{\text{QPN}}}
\end{split}
\end{align}
where the angle $\alpha$ sets the measurement quadrature. The Wineland parameter $\xi_W^2$ in the main text is then
\begin{align}
    \xi_W^2 = \left[ \xi^2_{\phi} \right]_{\text{min}} &=\frac{\sigma_{\text{min}}^2}{C^2} \frac{1}{\sigma^2_{\text{QPN}}}
\end{align}
where `min' refers to the minimum over $\alpha$.

\section{Fisher information of CSS for ellipse fitting}

\begin{figure}
    \centering
    \includegraphics[width=0.67\textwidth]{fig_fisher_information.pdf}
    \caption{\textbf{Measurement sensitivity of ellipse fitting.}
    Here, we consider a variable phase shift $\phi$ between two subarrays and quantify the measurement sensitivity using the classical Fisher information (solid gray line).
    Here, a coherent spin state with contrast $C=0.95$ is considered.
    }
    \label{fig:fisherinfo}
\end{figure}

In Fig.~\ref{fig:fisherinfo}, we present a calculation of the Fisher information in a coherent spin state (CSS) for measuring $\phi$ in an ellipse-fitting protocol. In this calculation, we consider the probability mass function 

\begin{align}
    f_{\text{CSS}}\left(p_A, p_B | \phi, C, y_0 \right) = f\left(p_A, p_B | \phi, C, y_0, \vec{\zeta} = (1, 1) \, \right)
\end{align}
where the function $f\left(p_A, p_B | \phi, C, y_0, \vec{\zeta} \, \right)$ is defined in the Methods. The parameter regime in which $\vec{\zeta} = (1, 1)$ corresponds to a binomial model, which captures the statistics of uncorrelated atoms in a CSS. The Fisher information for a measurement of $\phi$ with this probability distribution is then
\begin{align}
    \mathcal{I}_{\text{CSS}}\left( \phi_0 \right) = \sum_{p_A = 0/N}^{N/N} \sum_{p_A = 0/N}^{N/N} \left\{ \frac{\partial}{\partial \phi} \log \left[  f_{\text{CSS}}\left(p_A, p_B | \phi, C, y_0 \right) \right] \Big{|}_{\phi = \phi_0} \right\}^2  f_{\text{CSS}}\left(p_A, p_B | \phi_0, C, y_0 \right) \, .
\end{align}
In Fig.~\ref{fig:fisherinfo}, we plot $\mathcal{I}_{\text{CSS}}\left( \phi \right)$ versus $\phi$ for the parameters $C = 0.95$, $y_0 = 0.5$, which are representative of typical experimental values.

\begin{figure}
    \centering
    \includegraphics[width=\textwidth]{fig_theory_overview.pdf}
    \caption{\textbf{Squeezing dynamics in the experiment and numerical simulations.}
    Here, we show experimental measurements (purple circles) for the contrast $C$ (left column), variance reduction $\sigma_\alpha^2 / \sigma_\mathrm{QPN}^2$ (center column), and the Wineland parameter $\xi_W^2$ (right column).
    From top to bottom, the atom number increases from $N=2\times2 = 4$ to $5\times14 = 70$ in each row.
    Dark purple lines correspond to an exact-diagonalization calculation (see Methods) for the parameters in the experiment.
    Solid light purple lines show the theoretical predictions from weak-dressing~\cite{gil2014spin} using $\tilde{V}_0$ and $\tilde{R}_b$ from the fit shown in Fig. 1b of the main text.
    Dashed purple lines correspond to the same theory, but with $V_0 =\hbar \beta^3 \Omega_r$ and $R_b = {|C_6 / (2\Delta)|}^{(1/6)}$.
    }
    \label{fig:overview}
\end{figure}

\section{Comparison of dynamics in the experiment with theory}

In this section, we directly compare the dynamics observed in the experiment to theoretical predictions from numerical simulations.
To this end, we consider the weak and strong-dressing theory (exact diagonalization), as outlined in the Methods section of the main text.
While the exact diagonalization directly employs the independently determined parameters $\Omega$, $\Delta$, and $C_6$, we choose two different approaches for the weak-dressing theory.
In the first approach, we use the values of $\tilde{V}_0$ and $\tilde{R}_b$ obtained from the fit shown in Fig.~1b with a minor rescaling to adjust for slightly different parameters in each measurement.
In the second approach, we calculate $V_0 = \hbar\beta^3 \Omega_r$ and $R_b = {|C_6 / (2\Delta)|}^{(1/6)}$ from $\Omega_r$, $\Delta$, and $C_6$.

For the exact-diagonalization calculation (see dark purple lines in Fig.~\ref{fig:overview}), we generally find good qualitative agreement between experiment and theoretical prediction for the numerically accessible subarray sizes $N=2\times2$ and $3\times 3$.
For the first weak-dressing approach (see light purple lines in Fig.~\ref{fig:overview}), we find similarly good agreement for $N=2\times 2$, but for larger subarray sizes the dynamical time scale becomes significantly faster than the one observed in the experiment.
Moreover, the theoretically predicted maximum squeezing~$1/\xi_W^2$ becomes significantly larger than the experimentally observed one.
For the second weak-dressing approach (see dashed light purple lines in Fig.~\ref{fig:overview}), the theoretically predicted dynamics are even faster since $\tilde{V_0} < V_0$ which sets the characteristic time scale of the system~\cite{gil2014spin}.

\bibliography{references}

\end{document}
