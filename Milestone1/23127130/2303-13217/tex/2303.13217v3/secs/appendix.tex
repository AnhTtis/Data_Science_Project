\appendix
\section{Appendix}
\subsection{Pretrained Large Language Models}
Neural autoregressive language model (LMs) are designed for next token prediction to predict the probability distribution over the next token after a sequence of tokens input, and pre-trained LMs show their superior performance since they are trained on various programming languages and a large-scale curated dataset. Training large natural LMs are very expansive and time-consuming process since they always have billions of parameters, which limits the development of LMs. Fortunately, many pre-trained LMs are open access or limited access, which promotes researchers to pool their time and makes the resources to collectively achieve a higher impact. EleutherAI makes the GPT-J~\cite{gpt-j} and GPT-Neox~\cite{gptnexo2022} public available on Hugging Face. GPT-3~\cite{gpt32020brown} is limited access in OpenAI which can be used by researchers for a fee, and another large open-science open-access multilingual language model named Bloom~\cite{bloom2022} is provided by BigScience.

% \textbf{In-context Learning} Previous researches~\cite{gpt22018,gpt32020brown} show that Large neural LMs can perform tasks in zero- or few-shot manner using in-context learning, and \cite{rethinking2022min} provides a new way of understanding how and why in-context learning works, and shows the model counter-intuitively does not rely on the ground truth input-label mapping provided in the demonstrations. To address the bias issue in in-context learning, \cite{calibrate2021zhao} introduce a calibration strategy by the scaling probability distribution to make the content-free input has uniform scores for each answer. The prompt training examples are used to teach the LMs what task is to be solved, and many researches focus on tuning the prompts for a best performance. \cite{order2021lu} proposes a strategy by generating a probing set to overcome prompt order sensitivity, but generating a probing set is very costly. \cite{tuning2021lester} introduces prompt tuning to find optimal task prompts by maximizing the likelihood of label via backpropagation. However, the method is white-box optimization, so it may impact the security of the LMs. \cite{black2022sun} proposes a block-box tuning method as the gradients of pre-trained LMs are usually unavailable, but it is only designed the continues prompts, while natural language is the most prompt for current famous LMs service, for example, ChatGPT~\cite{chatgpt}. 

\subsection{Open Access Models}
\label{sec:app-models}

\begin{figure*}
    \centering
    \subfloat[]{\includegraphics[width=.33\textwidth]{images/melisa_pretrain.png}\label{fig:melisa_pretrain}}
    \hfil
    \subfloat[]{\includegraphics[width=.33\textwidth]{images/finetune_cine_vs_samples.png}\label{fig:finetune_cine}}
    \hfil
    \subfloat[]{\includegraphics[width=.33\textwidth]{images/finetune_tass_vs_samples.png}\label{fig:finetune_tass}}
    
    \subfloat[]{\includegraphics[width=.33\textwidth]{images/melisa_pretrain_bert.png}\label{fig:melisa_pretrain_bert}}
    \hfil
    \subfloat[]{\includegraphics[width=.33\textwidth]{images/finetune_cine_vs_samples_bert.png}\label{fig:finetune_cine_bert}}
    \hfil
    \subfloat[]{\includegraphics[width=.33\textwidth]{images/finetune_tass_vs_samples_bert.png}\label{fig:finetune_tass_bert}}
    \caption{Training curves ($F_1$-score) for different amounts of training samples. Figures (a), (b) and (c) shows the fine-tuning results on the validation split for the biLSTM model and Figures (d), (e) and (f) for BERT. (a) and (d) are the validation scores on MeLiSA, (b) and (e) are the scores on the MuchoCine validation split and (c) and (f) the scores on the TASS validation.}
    \label{fig:finetunning}
\end{figure*}

\subsection{Classification Models}

As mentioned in Section \ref{sec:intro}, we explored the cross-domain analysis using two different neural-based classification models which are shown in Fig. \ref{fig:classification_models}.

\begin{itemize}
    \item \textbf{BiLSTM classifier}. This architecture is based on a bidirectional recurrent neural network with a LSTM unit activation and it is illustrated in Figure \ref{fig:blstm_classifier}. In this model, each word $w_t$ of the input sequence $w_1,\ldots,w_T$ is represented as a continuous vector $\mathbf{x}_t$ trough an embedding layer at the beginning of the network. This vector sequence is then forwarded to two different LSTM networks~\cite{lstm} (one forward and one backward). The outputs at the last step of these layers are then concatenated and forwarded to the linear output layer, which gives the probabilities of each class through a Softmax activation function. 
    \item \textbf{BERT classifier}. Since the emergence of the Transformer architecture~\cite{transformer}, a series of models based on self-attention mechanisms have been proposed to pre-train a language model. One of this models is the Bidirectional Encoder Representations from Transformers (BERT), illustrated in Figure \ref{fig:bert_classifier}, which consists of 12 identical transformer encoder layers. These layers contains a multi-head self-attention layer~\cite{transformer} at the input followed by a linear layer (feed forward) with some residual connections and layer normalization~\cite{layernorm} in between. As in the LSTM classifier, every word at the input of the network is represented as a continuous vector, but additionally some special tokens are added to the sentence. In particular, the \texttt{[CLS]} token is included at the start of every sentence and it is used to extract features of the entire sequence. That is, at the output of the encoder a sequence of the same length as the input is obtained but only the first vector is used as an input of the output layer, which returns the class probability.
\end{itemize}

The key difference in our analysis of these two models is that the biLSTM network was trained from scratch, whereas the BERT classifier used was pre-trained on a Language Modeling task. Specifically, the pre-trained model called BETO~\cite{beto}, which is trained on a 3 billion words corpus called Spanish Unnanotated Corpora\footnote{https://github.com/josecannete/spanish-corpora} (SUC) was used. %The SUC database is a collection of unnanotated documents, most of them extracted from the Spanish portions of the Open Parallel Corpus (OPUS) subcorpora. The OPUS project is intended to provide the community with a publicly available parallel corpus of free online data. 
Documents included in the SUC contains all the data from Spanish Wikipedia available at the time the corpus was released and all of the sources of the OPUS Project~\cite{opus} that had text in Spanish. This sources
includes United Nations and Government journals, TED Talks, Subtitles, News Stories and more. However, none of these sources are review-like documents. The total size of the corpora gathered was comparable with the corpora used in the original BERT.
% @misc{cardelino,
%     title = "Spanish Billion Words Corpus and Embeddings",
%     author = "Cristian Cardellino",
%     year = "2016",
%     month = "March",
%     URL = "https: //crscardellino.github.io/SBWCE/"
% }

\begin{table*}
    \centering
    \caption{Test results ($F_1$-score) in the fine-tuning configuration.}
    \label{tab:melisa_finetunning}
    \begin{tabular}{r|cc|cc|cc}
        & \multicolumn{2}{c|}{Amazon} & \multicolumn{2}{c|}{TASS} & \multicolumn{2}{c}{MuchoCine} \\
        (\%) & biLSTM & BERT & biLSTM & BERT & biLSTM & BERT \\\hline
        0.0 & 0.5594 & 0.5860 & 0.3470 & 0.3850 & 0.2441 & 0.2758 \\
        0.1 & 0.5643(+0.88 \%) & 0.5860(+0.00 \%) & 0.3557(+2.51 \%) & 0.4104(+6.60 \%) & 0.1868(-23.47 \%) & 0.2834(+2.76 \%) \\
        10.0 & 0.5654(+1.07 \%) & 0.5865(+0.09 \%) & 0.3761(+8.39 \%) & 0.3974(+3.22 \%) & 0.2255(-7.62 \%) & 0.3210(+16.39 \%) \\
        100.0 & 0.5662(+1.22 \%) & 0.5845(-0.26 \%) & 0.3503(+0.96 \%) & 0.4045(+5.06 \%) & 0.2616(+7.17 \%) & 0.3125(+13.31 \%) \\
    \end{tabular}
\end{table*}

\subsection{Experimental Set-Up}

The above mentioned models were used to perform CDSC using the fine-tuning and the zero-shot configurations. For the fine-tuning case, the following steps were applied:
\begin{enumerate}
    \item The model was trained using the train split of the source dataset (MeLiSA, for instance) and hyperparameter search was done with its corresponding validation portion.
    \item Once trained, the model was trained again using the training portion of the target dataset (MuchoCine, for instance), and a new hyperparameter search was done with the target validation split.
    \item Once retrained, the model was evaluated on the test split of the target dataset.
\end{enumerate}

The only difference between both configurations is that step 2 is omitted for the zero-shot learning. This means that evaluation on the target dataset was done without using any training sample of that dataset. As a consequence, zero-shot learning is usually more challenging than fine-tuning and tends to show lower performance. However, it can provide a better idea of the model's generalization capability.

In order to test if CDSC can be achieved from product domains to more general domains like movie reviews or tweets, we used our MeLiSA dataset as source domain and MuchoCine and TASS as target domains. We also used the Amazon dataset as target domain to keep track of the model's learning capability, although this domain is, in principle, be very similar to the source domain. 

Experiments were carried on in \texttt{Python}, and the \texttt{Pytorch} module was used to implement the model training algorithm. We also used the \texttt{Huggingface Transformers} library to load the Spanish BERT pre-trained parameters and a NVIDA GTX 1080 GPU to reduce time computation. We followed \cite{random_grid_hyperparms} to perform random grid sample hyperparameter search on the biLSTM model. The best biLSTM model found consisted in a two-layer LSTM cell with hidden dimension of 50 and an embedding matrix of dimension $60,\!000\times 300$. Dropout was used as a regularization technique with a probability of $0.1$ and Adam Optimization with a batch size of 16 and a learning rate of 1e-3 was found to give the best validation results. For the pre-trained BERT model, layer dimensions are fixed in advance (12 layers with inner dimension of 768). We also used Adam Optimization to train this model and fixed the learning rate and the batch size to 5e-5 and 16 respectively, as suggested in \cite{beto}.






% \begin{algorithm}[tb]
%     \caption{Pseudo code for G-fair-Prompting}
%     \label{alg:app-greedy}
%     \begin{algorithmic}[1] %[1] enables line numbers
%         \STATE \textbf{Given:} development set $S=\{(x_i, y_i)\}^n$, pretrained LLM $M$, transformation template $\Gamma(\cdot)$, and context-free demonstration $\eta$
%         \STATE Initialize $most\_fair \leftarrow 0$; $improve \leftarrow true$; $\rho \leftarrow null$
%         \WHILE{$S$ \emph{is not null} and $improve$ \emph{is true}}
%         \STATE $most\_fair\_step \leftarrow 0$; $\rho_\text{step} \leftarrow null$; $tail \leftarrow null$
%         \FOR{$(x_i, y_i)$ in $S$}
%         \STATE $\rho_\text{tmp} \leftarrow  \Gamma(x_i,y_i)\oplus\rho$ \Comment{Insert the demonstration at the head}
%         \STATE Inference $\hat{P} \leftarrow \{\hat{p}(y|\rho_\text{tmp}\oplus\eta)|y \in \mathcal{Y}\}$ via $M$
%         \STATE Calculate the $fairness(\rho_\text{tmp})$ according to Eq.~\ref{eq:fair}
%         \IF{$fairness(\rho_\text{tmp})>most\_fair\_step$}
%         \STATE $most\_fair\_step \leftarrow fairness(\rho_\text{tmp})$
%         \STATE $\rho_\text{step} \leftarrow \rho_\text{tmp}$; $tail \leftarrow i$
%         \ENDIF
%         \ENDFOR
%         \IF{$most\_fair\_step>most\_fair$}
%         \STATE $most\_fair \leftarrow most\_fair\_step$;~
%          $\rho \leftarrow \rho_\text{step}$ \Comment{Update demonstrations}
%          \STATE Remove $(x_{tail},y_{tail})$ from $S$ \Comment{Every demonstrations occur once at most}
%         \ELSE \STATE $improve \leftarrow false$ \Comment{Stop searching when fairness can't be improved}
%         \ENDIF
%         \ENDWHILE
%         \RETURN $\rho$
 
%     \end{algorithmic}
% \end{algorithm}

\subsection{Additional Figures on Different Settings}\label{sec:app-addfigs}
In additional to the Fig.~\ref{fig:allcandidates}, we shows the performance on different models for enumerating all candidates, note that the shadow indicates the half value of standard deviation for clear presentation since the variance is very high for LLMs.

\begin{figure}[ht]
    \centering
    
    \subfloat[AGNews (GPT2-XL 1.5B)]{
    \centering
    \includegraphics[width=0.31\linewidth]{figs/greedyagnews.pdf}
    } 
    \subfloat[TREC (GPT2-XL 1.5B)]{
    \centering
    \includegraphics[width=0.31\linewidth]{figs/greedytrec.pdf}
    } 
    \subfloat[RTE (GPT2-XL 1.5B)]{
    \centering
    \includegraphics[width=0.31\linewidth]{figs/greedysst.pdf}
    }  \\
    \subfloat[AGNews (LLaMA 33B)]{
    \centering
    \includegraphics[width=0.31\linewidth]{figs/greedyagnewsllama33B.pdf}
    }
    \subfloat[TREC (LLaMA 33B)]{
    \centering
    \includegraphics[width=0.31\linewidth]{figs/greedytrecllama33B.pdf}
    }
    \subfloat[SST-2 (LLaMA 33B)]{
    \centering
    \includegraphics[width=0.31\linewidth]{figs/greedysstllama33B.pdf}
    } 
    \caption{Accuracy is highly consistency with
fairness and greedy search can find a good prompt, where "Random" and "Oracle" indicates the average accuracy of all prompts and the upper-bound performance according to fairness.}
\label{fig:app-allcandidates}
\end{figure}

\subsection{Accuracy Varies with demonstrations}\label{sec:app-ob}

\begin{figure}[t]
    \centering
    \subfloat[Varying amount of examples]{
    \centering
    \includegraphics[width=0.31\linewidth]{figs/Quantity-var.pdf}
    \label{fig:sub-quantity}
    }  
    \subfloat[Permutation]{
    \centering
    \includegraphics[width=0.31\linewidth]{figs/Permutation-var.pdf}
    \label{fig:sub-order}
    } 
    \subfloat[Select different examples]{
    \centering
    \includegraphics[width=0.31\linewidth]{figs/Selection-var.pdf}
    \label{fig:sub-select}
    } 
    \caption{ICL suffers from high instability due to variations in example amount, example order, and example selection.}
    \label{fig:app-obser-var}
\end{figure}
\textbf{Accuracy Varies with Example Amount}\quad
Demonstrations play an important role in imparting task-related information to language models through in-context learning. Then, the question arises - does a larger number of demonstrations necessarily equate to better performance? To answer this question, we evaluated performance in terms of accuracy by gradually increasing the number of demonstrations. We set $\rho=\Gamma(x_1,y_1)\oplus\cdots\oplus\Gamma(x_k,y_k)$, where $k =1,\cdots, n$, and demonstrations are erased with $k$ decreasing from $n$ to $1$. Intuitively, accuracy would vary highly across different numbers of demonstrations, and the phenomenon is observed in Fig.~\ref{fig:sub-quantity}. To our surprise, however, erasing some demonstrations can result in a better performance. Removing some demonstrations can perform better and sometimes GPT-3 achieves best accuracy when there is only a few demonstrations remaining. This highlights the importance of considering the appropriate number of demonstrations.

\textbf{Example Order}\quad
The performance of a model is sensitive to the order of the demonstrations, as has been discussed in \cite{order2021lu}. Even when the demonstrations are the same, different permutations of the demonstrations can result in vastly different outcomes. As there are $n!$ possible permutations, we introducing a strategy of permuting the demonstrations by circularly shifting the index of the demonstrations. The demonstration can be represented as $\rho=\Gamma(x_{k+1},y_{k+1})\oplus\cdots\oplus\Gamma(x_n,y_n)\oplus\Gamma(x_1,y_1)\oplus\cdots\oplus\Gamma(x_k,y_k)$.As shown in Fig.~\ref{fig:sub-order}, the accuracy varies highly with permutation which consistent with the observations in \cite{order2021lu}.





\textbf{Example Selection}\quad
In this paper, we find which demonstrations are selected is influence the model extremely. This scenario can be described as selecting $k$ demonstrations in $n$ training samples. In Fig.~\ref{fig:sub-select}, we only select one example for demonstration to ablate the impact of demonstrations order, and the accuracy also varies highly with different example selected. In this work, we only detail evaluate the proposed probing method on the erasing demonstrations and permutation, although our method improves by $20\%$ in the setting of example selection on SST-2 (GPT2-XL), because selecting $k$ demonstrations on a set with $n$ training samples can't be regarded as $k-$shot learning in the strict sense.

\begin{table}[ht]
\centering
\caption{Accuracy for different prompting strategies (averaged on $5_{0,\cdots,4}$ different seeds).}
\label{tab:app-acc}
\resizebox{1.0\textwidth}{!}{\begin{tabular}{c|c|ccc||ccc} \toprule
\multirow{2}{*}{\textbf{Model}}                  & \multirow{2}{*}{\textbf{Dataset}}                  & \multirow{2}{*}{\textbf{Random}} & 
 \multirow{2}{*}{\textbf{Diversity}} & \multirow{2}{*}{\textbf{Similarity}} & \multicolumn{3}{c}{\textbf{Ours}} \\ & & & &  &\textbf{Top-2} & \textbf{Top-4} & \textbf{Greedy}                \\ \midrule
\multirow{5}{*}{GPT2-XL (1.5B)} & {SST-2} & $61.1_{6.1}$  &$-$      & $-$   &$60.8_{11.4}$      & $65.8_{8.7}$       & $74.2_{12.0}$    \\  \cmidrule{2-8} & {AGNews} & $38.9_{11.4}$  &$-$      & $-$   &$45.2_{12.5}$      & $37.2_{11.2}$       & $46.4_{11.9}$    \\  \cmidrule{2-8}
& {TREC} & $22.1_{5.7}$  &$-$      & $-$   &$19.4_{8.9}$      & $28.2_{9.2}$       & $25.0_{7.4}$    \\  \cmidrule{2-8}
& {RTE} & $53.2_{6.9}$  &$-$      & $-$   &$54.0_{7.5}$      & $53.6_{5.9}$       & $56.4_{2.2}$   
\\ \midrule \multirow{4}{*}{LLaMA (7B)}  &    {AGNews}                             & $64.5_{10.0}$  &${66.4_{9.1}}$      & $-$   &$66.0_{11.7}$      & ${69.2_{5.5}}$       & $63.8_{5.7}$    \\  \cmidrule{2-8}
                                & {TREC}  & $49.5_{10.4}$ &${51.4_{9.6}}$            &  {$-$}       &$48.4_{10.5}$            & $38.6_{15.2}$      & ${61.3_{4.8}}$   \\ \cmidrule{2-8}
                                & {CoLA} & $60.4_{10.6}$ &$63.8_{8.7}$& $-$  &$58.2_{7.8}$            & $61.6_{6.5}$       & $36.4_{3.6}$ \\ \midrule\multirow{4}{*}{LLaMA (13B)}  &   {AGNews}                         & $72.2_{7.7}$ &$78.4_{3.5}$      & {$-$} &$73.6_{9.0}$      & $74.2_{4.3}$            & ${75.2_{2.8}}$    \\ \cmidrule{2-8}
                                & {TREC}  & $46.4_{16.5}$ &$48.0_{16.0}$            & $-$   &$51.0_{16.6}$            & $39.2_{23.3}$        & ${61.4_{12.1}}$  \\ \cmidrule{2-8}
                                & {CoLA} & $67.7_{2.9}$ &$67.2_{2.4}$& $-$  &$67.0_{2.0}$            & $67.2_{1.6}$       & ${67.0_{2.0}}$ \\  \bottomrule
\end{tabular}}
\end{table}




\subsection{Relationship between with- and without-calibration}
\begin{figure}[ht]
\centering
  \includegraphics[width=0.75\linewidth]{figs/pearson_llama65Btrec4.pdf}
 \caption{Illustration of accuracy relationship between with- and without calibration when $Pearson$ is positive.}
 \label{fig:app-with-without}
\end{figure}
$\bullet$ \textbf{\greedy without post-calibration outperforms random demonstrations after post-calibration.}
Based on Table~\ref{tab:four-topk-greedy}, it is apparent that \greedy outperforms random selection prior to post-calibration. This leads to a natural question: do prompts with better performance before calibration also indicate better performance after calibration proposed by Zhao et al.~\cite{calibrate2021zhao}? To investigate the relationship between performance with- and without-calibration, we calculated the Pearson correlation coefficient between the accuracy with- and without-calibration $Pearson(acc_{w/o},acc_{with})$. A positive coefficient value suggests that a prompt with high accuracy before calibration has a higher likelihood of achieving higher accuracy after calibration than other prompts. We take the topic classification task on LLaMA(65B) for illustration to show the relationship between with- and without calibration when $Pearson$ is positive in Fig.\ref{fig:app-with-without}. Table~\ref{tab:app-pearson} presents the Pearson correlation coefficient on accuracy of permutation and \greedy after calibration. The majority of Pearson correlation coefficients were found to be positive, indicating that prompts with better performance before calibration have more potential to perform well after calibration. Furthermore, our results on the LLaMA family reveal that the larger the model, the stronger the correlation between performance with- and without-calibration. For instance, the value of the Pearson correlation coefficient increases from $0$ to $0.7$ as the model size increases.

\begin{theorem}
\label{theorem:cal}
Suppose the performance of the model under certain prompts with- and without-calibration is positively correlated, i.e., $Pearson(acc_{w/o},acc_{with})>0$, if we can assure $\mathbb{E}(acc^{Selected}_{w/o})>\mathbb{E}(acc^{Random}_{w/o})$, then we have $\mathbb{E}(acc^{Selected}_{with})>\mathbb{E}(acc^{Random}_{with})$.
\end{theorem}
\begin{table}[ht]
\centering
\caption{Pearson's r between the with- and without-calibration.}
\label{tab:app-pearson}
\begin{tabular}{c||ccccc} \toprule
\multirow{2}{*}{\textbf{Dataset}}                    & {\textbf{BLOOM}}                & \multicolumn{4}{c}{\textbf{LLaMA}} \\  & \textbf{176B} &\textbf{7B} & \textbf{13B} & \textbf{33B}  & \textbf{65B}                 \\ \midrule{TREC}    &  $0.1274$   & $0.1551$ &  $0.2959$   &$0.3090$    &$0.5151$     \\ {AGNews}  &  $0.3875$   & $-0.0471$ &  $0.3044$   &$0.6953$    &$0.7100$   \\  {CoLA} &  $0.4050$   & $0.3592$ &  $0.5193$   &$0.3611$    &$0.8012$   \\ \bottomrule
\end{tabular}
\end{table}

As analysed in Theorem~\ref{theorem:cal}, if we can find a prompt with high accuracy before calibration, we have a higher likelihood of achieving higher accuracy after calibration than random selection. Our approach consistently identifies an appropriate prompt, as evidenced by the results in Table~\ref{tab:four-topk-greedy}. Moreover, the performance of the model exhibits a positive correlation with and without calibration under certain prompts, as illustrated in Table~\ref{tab:app-pearson}. Therefore, our method is more likely to enhance calibration performance.


\subsection{Complexity of different strategies}

\begin{figure}[ht]
\centering
  \includegraphics[width=0.65\linewidth]{figs/cost.pdf}
 \caption{Computational cost. T-fair and G-fair indicate \topk and \greedy respectively, and "w/c" indicates the worst case.}
 \label{fig:app-cost}
\end{figure}

\subsection{Performance on Zero-shot and SOTA Classifiers}

