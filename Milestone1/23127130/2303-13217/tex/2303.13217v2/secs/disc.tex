\section{Discussion}

The Metric Can Explain Previous Methods.
The proposed method can explain why some previous template selection strategies work. For example, \cite{lu2021fantastically} find that the order in which the samples are provided can make the difference between near state-of-the-art and random guess performance regardless of the model size. Then, they propose a method to find the best ordering of training examples through a probing set. Specifically, they sample from the language model to construct a probing set, and use that probing set to identify the prompt ordering which can generate diverse labels for that particular sample set.

Recently, GRIPS~\cite{prasad2022grips} provides a prompt searching strategy to design the prompt for large language models. They include entropy of model predictions in the score function to promote edited instructions that generate diverse labels.

In this paper, we have discussed the relationship between the fairness and LLMs' performance, and accordingly proposed a prompt selection strategy. However, this paper is just for effectiveness validation, and the fairness can be extended to more scenarios instead of naive definition by entropy or variance. For example, the fairness could be used as a weight for ensemble of different prompts beyond selection, and if you want highlight the worse class's impact, the fairness can be defined as:
\begin{equation}
    fairness = \arg\min_{y \in \mathcal{Y}} \hat{p}(y|\rho \oplus \eta).
\end{equation}
And instead of context-free input such as [N/A], it could be extend to a fine-grained scenarios. For example, if $\eta$ is tends to class $c$, then a good prompt should make the LLM be more confident on $y_m$ than other classes.

Moreover, the proposed criterion has the has the potential to eliminate social bias. For example, to eliminate gender bias, the fairness can be define as difference between the prediction distributions:
\begin{equation}
    fairness = \frac{1}{1+\textup{KL}\left(\hat{P}(\rho \oplus \eta_\textup{male})\|\hat{P}(\rho \oplus \eta_\textup{female})\right).}
\end{equation}


The criterion can be used as a plug-in for other methods. Based on this, there are more research contents and more values to be explored in the future.

