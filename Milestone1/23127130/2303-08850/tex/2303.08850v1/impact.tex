%========================================================================
% ifacconf.tex 2022-02-11 jpuente  
% Template for IFAC meeting papers
% Copyright (c) 2022 International Federation of Automatic Control
%========================================================================
\documentclass{ifacconf}

\usepackage{graphicx}      % include this line if your document contains figures
\usepackage{natbib}        % required for bibliography

%--------------------------------

%------------------------------------------------------------------------
% To solve the issue of getting an error when using hyperref in ifacconf
%------------------------------------------------------------------------
% https://tex.stackexchange.com/questions/393690/illegal-unit-of-measure-error-when-using-hyperref-in-the-ifacconf-class
\makeatletter
\let\old@ssect\@ssect % Store how ifacconf defines \@ssect
\makeatother

% \usepackage[pdftex,hidelinks]{hyperref}
\usepackage{hyperref}
\hypersetup{hidelinks}

\makeatletter
\def\@ssect#1#2#3#4#5#6{%
  \NR@gettitle{#6}% Insert key \nameref title grab
  \old@ssect{#1}{#2}{#3}{#4}{#5}{#6}% Restore ifacconf's \@ssect
}
\makeatother
%------------------------------------------------------------------------
%------------------------------------------------------------------------

\usepackage{amsmath}
\usepackage{amssymb}
\usepackage{amsfonts, latexsym, amscd, enumerate, bbm, nicefrac, mathrsfs}
\usepackage{optidef}
\usepackage{xcolor}
% \usepackage{pgfplots}
% \usepackage{pdfpages}
% \usepackage{caption}
% \usepackage{subcaption}
% \usepackage{tabularx}
% \usepackage{booktabs}
% \usepackage{tikz}
% \let\labelindent\relax
% \usepackage{enumitem}

% \usepackage{url}

\usepackage{listings}

\definecolor{codegreen}{rgb}{0,0.6,0}
\definecolor{codegray}{rgb}{0.5,0.5,0.5}
\definecolor{codered}{rgb}{0.65,0.0,0.0}
\definecolor{codepurple}{rgb}{0.58,0,0.82}
\definecolor{backcolour}{rgb}{0.95,0.95,0.92}
\definecolor{shadecolor}{gray}{0.85}
\definecolor{KULlightblue}{HTML}{116E8A}
\definecolor{KULdarkblue}{HTML}{00407A}

\lstdefinestyle{mystyle}{
    backgroundcolor=\color{backcolour},   
    commentstyle=\color{codegreen},
    keywordstyle=\color{magenta},
    numberstyle=\tiny\color{codegray},
    stringstyle=\color{codepurple},
    basicstyle=\ttfamily\footnotesize,
    breakatwhitespace=false,         
    breaklines=true,                 
    captionpos=b,                    
    keepspaces=true,                 
    numbers=none,                    
    numbersep=5pt,                  
    showspaces=false,                
    showstringspaces=false,
    showtabs=false,                  
    tabsize=2
}

\lstdefinestyle{Cstyle}{language=C,
    commentstyle=\color{KULdarkblue},
    keywordstyle=\color{KULdarkblue},
    numberstyle=\tiny\color{codegray},
    stringstyle=\color{codered},
    basicstyle=\ttfamily\tiny,
    breakatwhitespace=false,         
    breaklines=true,                 
    captionpos=b,                    
    keepspaces=false,           
    columns=fullflexible,
    numbers=none,                    
    numbersep=5pt,                  
    showspaces=false,                
    showstringspaces=false,
    showtabs=false,                  
    tabsize=1,
    xleftmargin=.0\textwidth, 
    xrightmargin=.0\textwidth
}

\lstdefinestyle{pythonstyle}{language=python,
    commentstyle=\color{KULdarkblue},
    keywordstyle=\color{KULdarkblue},
    numberstyle=\tiny\color{codegray},
    stringstyle=\color{codered},
    basicstyle=\ttfamily\tiny,
    breakatwhitespace=false,         
    breaklines=true,                 
    captionpos=b,       
    keepspaces=true,
    columns=fullflexible,
    numbers=none,                    
    numbersep=5pt,                  
    showspaces=false,                
    showstringspaces=false,
    showtabs=false,   
    tabsize=2,
    xleftmargin=.00\textwidth, 
    xrightmargin=.00\textwidth,
}

\lstdefinestyle{Cinline}{language=C,
    commentstyle=\color{KULdarkblue},
    keywordstyle=\color{KULdarkblue},
    numberstyle=\tiny\color{codegray},
    stringstyle=\color{codered},
    basicstyle=\ttfamily\small,
    breakatwhitespace=false,         
    breaklines=true,                 
    captionpos=b,                    
    keepspaces=false,           
    columns=fullflexible,
    numbers=none,                    
    numbersep=5pt,                  
    showspaces=false,                
    showstringspaces=false,
    showtabs=false,                  
    tabsize=1,
    xleftmargin=.0\textwidth, 
    xrightmargin=.0\textwidth
}

\lstdefinestyle{pythoninline}{language=python,
    commentstyle=\color{KULdarkblue},
    keywordstyle=\color{KULdarkblue},
    numberstyle=\tiny\color{codegray},
    stringstyle=\color{codered},
    basicstyle=\ttfamily\small,
    breakatwhitespace=false,         
    breaklines=true,                 
    captionpos=b,       
    keepspaces=true,
    columns=fullflexible,
    numbers=none,                    
    numbersep=5pt,                  
    showspaces=false,                
    showstringspaces=false,
    showtabs=false,   
    tabsize=2,
    xleftmargin=.00\textwidth, 
    xrightmargin=.00\textwidth,
}


\newcommand\YAMLcolonstyle{\color{black}\mdseries}
\newcommand\YAMLkeystyle{\color{green}\bfseries}
\newcommand\YAMLvaluestyle{\color{blue}\mdseries}


\makeatletter

% here is a macro expanding to the name of the language
% (handy if you decide to change it further down the road)
\newcommand\language@yaml{yaml}

\expandafter\expandafter\expandafter\lstdefinelanguage
\expandafter{\language@yaml}
{
  keywords={true,false,null,y,n},
  keywordstyle=\color{darkgray}\bfseries,
  % basicstyle=\YAMLkeystyle,                                 % assuming a key comes first
  basicstyle=\ttfamily\tiny,
  sensitive=false,
  numbersep=0pt,    
  comment=[l]{\#},
  morecomment=[s]{/*}{*/},
  commentstyle=\color{purple}\ttfamily,
  stringstyle=\YAMLvaluestyle\ttfamily,
  moredelim=[l][\color{orange}]{\&},
  % moredelim=[l][\color{magenta}]{*},
  moredelim=**[il][\YAMLcolonstyle{:}\YAMLvaluestyle]{:},   % switch to value style at :
  morestring=[b]',
  morestring=[b]",
  literate =    {---}{{\ProcessThreeDashes}}3
                {>}{{\textcolor{red}\textgreater}}1     
                {|}{{\textcolor{red}\textbar}}1 
                {\ -\ }{{\mdseries\ -\ }}3,
    xleftmargin=0\textwidth, 
    xrightmargin=0\textwidth
}

% switch to key style at EOL
\lst@AddToHook{EveryLine}{\ifx\lst@language\language@yaml\YAMLkeystyle\fi}
\makeatother
\newcommand\ProcessThreeDashes{\llap{\color{cyan}\mdseries-{-}-}}


% \lstset{style=mystyle}

% \usepackage[finalizecache,cachedir=minted-cache]{minted}
 % \usepackage[frozencache=true,cachedir=minted-cache]{minted} 
% \usepackage{minted}
% \usepackage{enumitem}

% \usepackage{etoolbox}
% \BeforeBeginEnvironment{minted}{\vspace{-0.5ex}}
% \AfterEndEnvironment{minted}{\vspace{-1ex}}
% \AfterEndEnvironment{minted}{\vspace{-1cm}}
% \AtBeginEnvironment{snugshade*}{\vspace{-\FrameSep}}
% \AfterEndEnvironment{snugshade*}{\vspace{-\FrameSep}}

\newcommand{\codename}[1]{{\fontfamily{pcr}\selectfont #1}}
\newcommand*{\impact}{{\codename{IMPACT}} }
\newcommand*{\casadi}{{\codename{CasADi}} }

%--------------------------------
% \def\thefootnote{*}\footnotetext{Alvaro Florez and Alejandro Astudillo contributed equally 
%========================================================================
\begin{document}
\begin{frontmatter}

\title{IMPACT: A Toolchain for Nonlinear Model Predictive Control Specification, Prototyping, and Deployment 
% with Python and Simulink Wrappers
\thanksref{footnoteinfo}} 
% Title, preferably not more than 10 words.

\thanks[footnoteinfo]{This work was supported by the project Flanders Make SBO DIRAC: ``Deterministic and Inexpensive Realizations of Advanced Control". 
Flanders Make is the Flemish strategic research centre for the manufacturing industry.
}

\author[First]{Alvaro Florez\thanksref{equal_contrib}}
\author[First]{Alejandro Astudillo\thanksref{equal_contrib}}
\author[First]{Wilm Decré}
\author[First]{Jan Swevers}
\author[First]{Joris Gillis}

\address[First]{MECO~Research~Team,~Dept.~of~Mechanical~Engineering,~KU~Leuven.\\
Flanders Make@KU Leuven, 3001 Leuven, Belgium.\\
    (e-mail: \{alvaro.florez, alejandro.astudillovigoya, wilm.decre, jan.swevers, joris.gillis\}@kuleuven.be).}

\thanks[equal_contrib]{A. Florez and A. Astudillo contributed equally to this work.}


\begin{abstract}    % Abstract of not more than 250 words.
We present \codename{IMPACT}, a flexible toolchain for nonlinear model predictive control (NMPC) specification with automatic code generation capabilities. The toolchain reduces the engineering complexity of NMPC implementations by providing the user with an easy-to-use application programming interface, and with the flexibility of using multiple state-of-the-art tools and numerical optimization solvers for rapid prototyping of NMPC solutions. \codename{IMPACT} is written in \codename{Python}, users can call it from \codename{Python} and \codename{MATLAB}, and the generated NMPC solvers can be directly executed from \codename{C}, \codename{Python}, \codename{MATLAB} and \codename{Simulink}. An application example is presented involving problem specification and deployment on embedded hardware using \codename{Simulink}, showing the effectiveness and applicability of \codename{IMPACT} for NMPC-based solutions.
\end{abstract}

\begin{keyword}
Model Predictive Control, Software Toolchain, Prototyping, Deployment
\end{keyword}

\end{frontmatter}
%========================================================================

% \begin{figure}[t]
%     % \begin{subfigure}{1\linewidth}
%     %   \centering
%     % %   \includegraphics[width=1\linewidth]{figs/fig_1_moti_textattn.pdf}  
%     % %   \includegraphics[width=1\linewidth]{figs/fig_1_moti_textattn_v2.pdf}  
%     %   \includegraphics[width=1\linewidth]{figs/fig_1_moti_textattn_v5.pdf}  
%     %   \vspace{-0.5cm}
%     %     \caption{Amount of attention added to each video clip from the source video and query text in the self-attention layers of Moment-DETR encoder.}
%     %     % \caption{Distribution of attention for source and query in Moment-DETR encoder}
%     %     % Visualization of video clip's self-attention score in Moment-DETR encoder.
%     %   \label{fig:fig1_text_attn_ex}
%     % \end{subfigure}%\hfill% or  or \hspace{0.3\textwidth}
%     \vspace{0.2cm}
%     % \begin{subfigure}{1\linewidth}
%       \centering
%     %   \includegraphics[width=1\linewidth]{figs/fig1_moti_negattn.pdf}  
%       \includegraphics[width=1\linewidth]{figs/fig1_moti_negattn_v3.pdf}  
%       \vspace{-0.4cm}
%     %   \caption{Correspondence of saliency scores on the relevance between video clips and the text query.}
%     % \caption{Predicted saliency scores against the video relevant positive query and video irrelevant negative query}
%       \label{fig:fig1_neg_attn_ex}
%     % \end{subfigure}%\hfill% or  or \hspace{0.3\textwidth}
%     \caption{
%     % 원준 원본
%     % (a) Comparison between attention scores of source and query for each video clip~(We sum the attention scores from video and text). 
%     % We observe that the attention scores are dominated by other clips in the source video. 
%     % Text queries do not account for much attention regardless of the relevance to the video clips.
%     % \textbf{(a)} Inspection of the query dependency in Moment-DETR encoder.
%     % % We visualize the attention score of video tokens in the transformer encoder and observe that text query accounts for only a low portion of attention.
%     % % This tendency occurs regardless of the relevance between the text query and video clips. 
%     % We visualize the attention score of video tokens in the transformer encoder and observe 1) text query only accounts for a low portion of attention, and 2) relevance between video-query pair does not affect the attention scores ratio of text.
%     \textbf{(b)} Comparison of highlight-ness when relevant and non-relevant queries are input.
%     As observed in , existing work only uses queries to play an insignificant role, thereby may not be capable of detecting false queries and considering the video-query relevance even when the problem in (a) is resolved. 
%     % \SE{} % 이 부분이 "not capable of" 란 용어가 세다는 피드백이 있는 듯 합니다. 이러한 능력이 없다는 것은 굉장히 강한 어조인거 같기는 하고, 이러한 경우들이 종종 있다거나 좀 약화시킬 필요가 있어보이긴 하네요.
%     On the other hand, our QD-DETR yields a query-dependent representation that the relevance between the source video and query text is updated in the saliency scores.
%     There is a large gap between positive and negative saliency scores, and scores are consistent since the clips are all highly correlated to others.
%     }
%     \label{fig:motivation_ex}
%     % \captionsetup{belowskip=13pt}
%     % \setlength{\belowcaptionskip}{-10pt}
% \end{figure}
\begin{figure}
    \centering
    \includegraphics[width=1\linewidth]{figs/fig1_moti_negattn_1111.pdf}
    % \includegraphics[width=1\linewidth]{figs/fig1_moti_negattn_1109.pdf}
    % \includegraphics[width=1\linewidth]{figs/fig1_moti_negattn_stat.pdf}
    \vspace{-0.6cm}
    \caption{
        % \SE{} % 수정 필요
        Comparison of highlight-ness~(saliency score) when relevant and non-relevant queries are given.
        We found that the existing work only uses queries to play an insignificant role, thereby may not be capable of detecting negative queries and video-query relevance; saliency scores for clips in ground-truth~(GT) moments are low and equivalent for positive and negative queries.
        % This also results in mispredicted moments when ground-truth~(GT) moment is dominated by clips unrelated to GT since their prediction is highly focused on the video.
        % \SE{} % 여기 한번 더 보면 좋을 듯 합니다. GT moment에 unrelated한 clip이 많으면? label이 틀렷을 경우를 말씀하시는건지?
        % As observed in saliency graph, existing work only uses queries to play an insignificant role, thereby may not be capable of detecting false queries and considering the video-query relevance.
        On the other hand, query-dependent representations of QD-DETR result in corresponding saliency scores to the video-query relevance and precisely localized moments.
        % On the other hand, our QD-DETR yields a query-dependent representation that the
        % saliency scores are in accordance with the relevance between the video and query.
        % text is in accordance with the saliency scores.
        % There is a large gap between positive and negative saliency scores, and scores are consistent since the clips are all highly correlated to others.
}
    \label{fig:motivation_ex}
\end{figure}


\section{Introduction}
% 원준 원본
% Along with the advance of digital devices and platforms, video is now one of the most desired data type for consumers. However, although the large information capacity of videos may be beneficial in many aspects, e.g., informative and entertaining, on the contrary perspective, videos are time-consuming, and hard to search for desirable moments. 
% This has led many creators to use extra manpower to crop and edit the video to generate highlight clips to gain the consumer’s attention.
Along with the advance of digital devices and platforms, video is now one of the most desired data types for consumers~\cite{apostolidis2021video,wu2017deep}.
% SE: Video aware deep learning application & survey papers?
Although the large information capacity of videos might be beneficial in many aspects, e.g., informative and entertaining, inspecting the videos is time-consuming, so that it is hard to capture the desired moments~\cite{anne2017localizing,apostolidis2021video}. 
% This has led many creators to use extra manpower to crop and edit the video to generate highlight clips to gain the consumer’s attention.


% On the other side, 
Indeed, the need to retrieve user-requested or highlight moments within videos is greatly raised.
Numerous research efforts were put into the search for the requested moments in the video~\cite{anne2017localizing, gao2017tall, liu2015multi, escorcia2019temporal} and summarizing the video highlights~\cite{zhang2016video, mahasseni2017unsupervised, badamdorj2022contrastive, wei2022learning}.
% Numerous research efforts were put into the search for the requested moments in the video~\cite{anne2017localizing, gao2017tall, liu2015multi, escorcia2019temporal}, summarizing the video to generate highlights was another popular topic~\cite{zhang2016video, mahasseni2017unsupervised, badamdorj2022contrastive, wei2022learning}.
Recently, Moment-DETR~\cite{momentdetr} further spotlighted the topic by proposing a QVHighlights dataset that enables the model to perform both tasks, retrieving the moments with their highlight-ness, simultaneously.

% 원준 원본
% To detect the desired moments, previous works employed transformer encoder-decoder architectural designs to fuse the text query into the video representations. Moment-DETR~\cite{mDETR} modified detection transformer to process capture the moment as a set, and UMT~\cite{umt} implemented transformer decoder as to output clip-wise saliency. 
% Yet to their outstanding breakthroughs in the literature of moment retrieval with the seminal architectures, their limitation is that the role of the given text query is insignificant in representing the query-conditioned video representation; the attention mechanism of moment DETR is not explicitly conditioned on the text query, and the text query is conditioned on multi-modal clips where the differences between the clips are smoothed after encoding process in UMT.



% \begin{figure}[t]
% \centering
%     \begin{subfigure}[l]{0.37\linewidth}
%       \centering
%       \vspace{0.20cm}
%     %   \includegraphics[width=1\linewidth]{figs/fig_1_moti_textattn.pdf}  
%     %   \includegraphics[width=1\linewidth]{figs/fig_1_moti_textattn_v2.pdf}  
%       \includegraphics[width=1\linewidth]{figs/fig1_moti_violin_a.pdf}  
%       \vspace{-0.60cm}
%     %   \caption{text attention}
%         \caption{Importance of queries in video representation}
%       \label{fig:fig1_text_attn}
%     \end{subfigure}%\hfill% or  or \hspace{0.3\textwidth}
%     \vspace{0.2cm}
%     \begin{subfigure}[r]{0.61\linewidth}
%       \centering
%     %   \includegraphics[width=1\linewidth]{figs/fig1_moti_negattn.pdf}  
%       \includegraphics[width=1\linewidth]{figs/fig1_moti_violin_b.pdf}  
%     %   \caption{neg attention}
%         % \caption{Relation between the highlight-ness and the relevance between videos and query texts.}
%         \caption{Highlight-ness~(saliency) histogram of positive and negative video-query pairs\SE{}}
%       \label{fig:fig1_neg_attn}
%     \end{subfigure}%\hfill% or  or \hspace{0.3\textwidth}
%     % \vspace{-0.2cm}
%     \caption{Overall statistics for attention scores in Fig.~\ref{fig:motivation_ex} in QVHighlights dataset. 
%     (a) For the attention scores that measure how much the text query is generally involved in video representation, we use violin plots to show the probability density. We plot the score for each layer in the encoder.
%     % (b) Using the histogram, we compare how the baseline and QD-DETR yield different salient scores given the positive and negative video-text pairs.
%     (b) Saliency histogram shows the distributional gap between positive and negative video-text query pairs of baseline~(Moment-DETR) and proposed QD-DETR.\SE{}
%     }
%     \label{fig:motivation}
%     % \captionsetup{belowskip=13pt}
%     % \setlength{\belowcaptionskip}{-10pt}
% \end{figure}

% \begin{figure}[t]
% \centering

%     \begin{subfigure}[r]{1\linewidth}
%       \centering
%       \hspace{-0.2cm}
%     %   \includegraphics[width=1\linewidth]{figs/fig1_moti_negattn.pdf}  
%       \includegraphics[width=1.1\linewidth]{figs/fig1_moti_violin_a_v2.pdf}  
%     %   \caption{neg attention}
%         % \caption{Relation between the highlight-ness and the relevance between videos and query texts.}
%         \vspace{-0.5cm}
%         % \caption{Saliency histogram of positive and negative video-query pairs}
%         \caption{We plot the histograms and its average value~(dotted line) to compare saliency scores when true and false text queries are given for each method. (left) Since the video representations do not include much textual information, both the true and false queries yield similar saliency scores. (Middle) Even when the video representation is enforced to be updated with the textual information, the issue is not much resolved. (Right) By extracting discriminative features in the text query, distributions are differentiated.
%         % \SE{} % R1@0.5 설명
%         Also, R1@0.5 indicates evaluation metric, Recall at 1 with IoU 0.5 threshold on QVhighlight \textit{val} set.
%         }
%       \label{fig:fig1_neg_attn}
%     \end{subfigure}%\hfill% or  or \hspace{0.3\textwidth}
%     \\
%     \begin{tabular}{cc}
%     \hspace{-0.2cm}
%         \begin{minipage}{.4\linewidth}
%             \begin{subfigure}[l]{1\linewidth}
%               \centering
%             %   \vspace{0.20cm}
%             %   \includegraphics[width=1\linewidth]{figs/fig_1_moti_textattn.pdf}  
%             %   \includegraphics[width=1\linewidth]{figs/fig_1_moti_textattn_v2.pdf}  
%               \includegraphics[width=1\linewidth]{figs/fig1_moti_violin_a.pdf}  
%               \vspace{-0.60cm}
%             %   \caption{text attention}
%                 \caption{Importance of queries in video representation}
%               \label{fig:fig1_text_attn}
%             \end{subfigure}%\hfill% or  or \hspace{0.3\textwidth}
%         \end{minipage}
        
%         \begin{minipage}{.6\linewidth}
%             \vspace{-0.2cm}
%             \caption{Overall statistics of Fig.~\ref{fig:motivation_ex} in QVHighlights dataset. 
%             (a) Saliency histogram shows the distributional gap between positive and negative video-text query pairs.
%             % (a) For the attention scores that measure how much the text query is generally involved in video representation, we use violin plots to show the probability density. We plot the score for each layer in the encoder.
%             % (b) Using the histogram, we compare how the baseline and QD-DETR yield different salient scores given the positive and negative video-text pairs.
%             % (b) Text ratio in self-attention layer to  of Moment-DETR
%             % (b) Ratio of text when representing video tokens in self-attention of Moment-DETR.
%             % (b) Magnitude of attention text query involved.
%             % (b) Attention score of video tokens
%             % (b) Magnitude of text query to refine the video tokens in self-attention layer of Moment-DETR.
%             (b) Probability density depicting the weight of the text query in attention score for video clips. Scores are from the self-attention layers in Moment-DETR encoder.
%             % (b) The text query ratio in attention score of video clips (Self-attention layer in Moment-DETR encoder). We use violin plots to show probability density.
%             % 텍스트 쿼리가, 비디오 피쳐에 얼만큼 attend 하는지
%             }
%         \end{minipage}
    
%     \end{tabular}
%     \vspace{-0.5cm}
%     \label{fig:moti}
%     % \captionsetup{belowskip=13pt}
%     % \setlength{\belowcaptionskip}{-10pt}
% \end{figure}


% \begin{figure}
%     \centering
%     % \includegraphics[width=1\linewidth]{figs/fig1_moti_negattn_1109.pdf}
%     \includegraphics[width=1\linewidth]{figs/fig1_moti_negattn_stat_v2.pdf}
%     \vspace{-0.8cm}
%     \caption{
%         Histogram of saliency when the positive and negative queries are given. We plot the histograms and its average value~(dotted line) to compare saliency scores when relevant~(positive) and irrelevant~(negative) text queries are given for each method. (Left) Since the video representations do not properly reflect textual information, both the positive and negative queries yield similar saliency scores. 
%         % (Middle) Even when the video representation is enforced to be updated with the textual information, the issue is not much resolved. 
%         (Right) By representing video clips in query-dependent manner, distributions are differentiated.
%     }
%     \vspace{-0.6cm}
%     \label{fig:motivation}
% \end{figure}


% One of the demanding task is moment retrieval task, which is detecting the desired moments from the given query, typically the text query.
When describing the moment, one of the most favored types of query is the natural language sentence~(text)\cite{anne2017localizing}. 
While early methods utilized convolution networks~\cite{zhang2020learning, gao2021fast, wang2020temporally}, recent approaches have shown that deploying the attention mechanism of transformer architecture is more effective to fuse the text query into the video representation.
% To handle these modalities, previous works simply employed the attention mechanism of transformer architecture to fuse the text query into the video representation.
For example, Moment-DETR~\cite{momentdetr} introduced the transformer architecture which processes both text and video tokens as input by modifying the detection transformer~(DETR), and UMT~\cite{umt} proposed transformer architectures to take multi-modal sources, e.g., video and audio. 
Also, they utilized the text queries in the transformer decoder.
Although they brought breakthroughs in the field of MR/HD with seminal architectures, they overlooked the role of the text query.
To validate our claim, we investigate the Moment-DETR~\cite{momentdetr} in terms of the impact of text query in MR/HD~(Fig.\ref{fig:motivation_ex}).
Given the video clips with a relevant positive query and an irrelevant negative query, we observe that the baseline often neglects the given text query when estimating the query-relevance scores, i.e., saliency scores, for each video clip.
% the output saliency score, i.e. query-relevance scores.
% Based on the observation, we traced the actual saliency prediction of the model against both the video-relevant query and the irrelevant dummy one where we find that the baseline often neglects the given text query when estimating the query-relevance scores of video clips.
% For example, in Fig.~\ref{fig:motivation_ex}, saliency scores are not affected even when the query is substituted with the dummy.
% % General statistics for Fig.~\ref{fig:motivation_ex} is shown in Fig.~\ref{fig:motivation}. 
% General statistics corresponding to Fig.~\ref{fig:motivation_ex} are also shown in Fig.~\ref{fig:motivation}.



% The limitation of the concrete baseline~\cite{momentdetr} is inspected in two different aspects; 1) Utilization of text-query in the encoding process and 2) the output saliency score, i.e. query-relevance scores.
% Firstly, we visualize the attention score when video clips are given as a query in self-attention. 
% We observe that the text queries have relatively small impacts compared to other video features, as shown in Fig.~\ref{fig:fig1_text_attn_ex}.
% That is, the text does not account for much in representing every video clip, although the goal of MR/HD is to detect query-relevant moments.
% Based on the observation, we traced the actual saliency prediction of the model against both the video-relevant query and the irrelevant dummy one where we find that the baseline often neglects the given text query when estimating the query-relevance scores of video clips.
% For example, in Fig.~\ref{fig:motivation_ex}, saliency scores are not affected even when the query is substituted with the dummy.
% % General statistics for Fig.~\ref{fig:motivation_ex} is shown in Fig.~\ref{fig:motivation}. 
% General statistics are also shown in Fig.~\ref{fig:motivation}.

% Consequently, in Fig.~\ref{fig:fig1_neg_attn_ex}~(b), we found that the baseline often neglects the given text query when estimating the query-relevance scores of video clips; 
% For example, 


% We validate the previous work sometimes neglects the given query when estimating the saliency of video clips.
% For example, there is an example that the saliency scores from positive and negative queries cannot be distinguishable, as shown in Fig.~\ref{fig:fig1_neg_attn_ex}.
% % 우리는 추가로 text attention을 추가도 해봤지만, 효과가 있긴 했으나, still 이슈가 있는 것을 확인하였다?
% % Still, we observe that assuring the high attendance of text queries does not resolve the overlap which motivates us to question the quality of the naive use of task-agnostic text representation~\cite{momentdetr, umt}.
% We found that introducing the text-attention for ensuring the high attendance of text queries relieve the overlap, but there still be a severe overlap.


% To validate their limitations, we inspect the impacts of text queries in the concrete baseline~\cite{momentdetr} with the two different aspects, 1) tendency of attention in self-attention layer and 2) saliency score, i.e. query-relevance scores. \SE{} % attention 이 갑자기 등장하는가?
% Firstly, we visualize the attention score when video clips are given as a query in self-attention. We observe the text queries have relatively low attention scores compared to the video features, as shown in Fig.~\ref{fig:fig1_text_attn_ex}.
% That is, the text does not account for much in representing every video clip, although the goal of MR/HD is to detect query-relevant moments.
% Based on this observation, we trace the actual saliency prediction of the model against both positive and negative text queries.
% We validate the previous work sometimes neglects the given query when estimating the saliency of video clips.
% For example, there is an example that the saliency scores from positive and negative queries cannot be distinguishable, as shown in Fig.~\ref{fig:fig1_neg_attn_ex}.
% % 우리는 추가로 text attention을 추가도 해봤지만, 효과가 있긴 했으나, still 이슈가 있는 것을 확인하였다?
% % Still, we observe that assuring the high attendance of text queries does not resolve the overlap which motivates us to question the quality of the naive use of task-agnostic text representation~\cite{momentdetr, umt}.
% We found that introducing the text-attention for ensuring the high attendance of text queries relieve the overlap, but there still be a severe overlap.



% Thus, we 
% query dependency를 높이기 위해 
% Cross-attention? text-attention? detailed explanation on text-attention should be needed?
% By handling these two issues, we find that more precise retrieval can be achieved.
% 
% 
%
% By projecting video-discriminative text features with high text attendance to source video, we f 
% We also find the need to improve the quality of query features since assuring high text attendance also results in...
% pairs are not finetuned to be discriminative that even the similarity within the pairs does not reflect the relevance between the query and the video clips.
% General statistics for Fig.~\ref{fig:motivation_ex} is shown in Fig.~\ref{fig:motivation}. 
% \SE{} % 이거 ??로 뜨는데, 위처럼 figure 그리면 label이 안되는걸까요
% \SE{}
% 형님 아래 사항 생각 좀 해보는게 좋을 거 같아요.
% fig 1. (a) 그림만 봤을 때 모든 clip에 대해 text attention이 일정이상 존재하긴 하니까, 뭔가 not assured to be conditioned가 와닿지 않는거 같아요.
% + 왜 text가 항상 attend 해야하나?
% not assured to be conditioned --> text shows relatively low affects compared to video 같이 실제 나타난 현상까지 같이 적으면 어떨까 싶어요.
% fig 1. (b) 덜 반영한다?

% \SU{}
% 일단 text가 attend 잘 되어야 한다는 것에 좀 궁금점이 생깁니다. 결국에는 text와 관련있는 frame들을 attend해서 higlight를 찾아야 하는게 아닐까요? 그리고, 현제 저희의 모델 구조상 text query가 Key와 Value로 거의 활용되고 있는데 그렇다면 결국에는 해당 모델은 text에 대한 attention이 전혀 없다고 봐도 무방하지 않을까요? 그런 면에서 text attention을 강조하는게 좀 걸리긴 합니다.

% Specifically, the text query is not assured to be explicitly conditioned on every clip of the video, and as the query texts are evenly treated, discriminative keywords may not be spotlighted.
% attention mechanism of Moment-DETR is not explicitly conditioned on the text query as shown in Fig~\ref{}(d), and in UMT, the text are only used for conditioning the queries while the video representation are refined itself by self-attention.

% \begin{figure}[t]
%     \begin{subfigure}{1\linewidth}
%       \centering
%     %   \includegraphics[width=1\linewidth]{figs/fig_1_moti_textattn.pdf}  
%     %   \includegraphics[width=1\linewidth]{figs/fig_1_moti_textattn_v2.pdf}  
%       \includegraphics[width=1\linewidth]{figs/fig_1_moti_textattn_v4.pdf}  
%       \vspace{-0.5cm}
%     %   \caption{text attention}
%         \caption{Distribution of attention scores in Moment-DETR encoder}
%       \label{fig:fig1_text_attn}
%     \end{subfigure}%\hfill% or  or \hspace{0.3\textwidth}
%     \vspace{0.2cm}
%     \begin{subfigure}{1\linewidth}
%       \centering
%     %   \includegraphics[width=1\linewidth]{figs/fig1_moti_negattn.pdf}  
%       \includegraphics[width=1\linewidth]{figs/fig1_moti_negattn_v2.pdf}  
%       \vspace{-0.5cm}
%     %   \caption{neg attention}
%         \caption{Saliency score against positive and negative text queries}
%       \label{fig:fig1_neg_attn}
%     \end{subfigure}%\hfill% or  or \hspace{0.3\textwidth}
%     \vspace{0.2cm}
%     \begin{subfigure}{1\linewidth}
%       \centering
%     %   \includegraphics[width=1\linewidth]{figs/fig1_moti_violin.pdf}  
%       \includegraphics[width=1\linewidth]{figs/fig1_moti_violin_v2.pdf}  
%       \vspace{-0.5cm}
%       \caption{violin}
%       \label{fig:fig1_violin}
%     \end{subfigure}%\hfill% or  or \hspace{0.3\textwidth}
%     \vspace{-0.2cm}
%     \caption{(a) 1. portion of text attention vs. video attention 2. relation with text query and content (e.g. fg, bg) of clip seems not to affect the attention score
%     (b) 1. high variability even though entire clips are highly correlated with the given text query 2. positive and negative query makes overlaps on saliency score distribution
%     (3) actual distribution on validation dataset.}
%     \label{fig:motivation}
%     % \captionsetup{belowskip=13pt}
%     % \setlength{\belowcaptionskip}{-10pt}
% \end{figure}

To this end, we propose Query-Dependent DETR~(QD-DETR) that produces query-dependent video representation.
% Our key focus is to ensure each clip in predicted moments is explicitly conditioned by the query, particularly on the video-descriptive portion of the text query.
% Our key focus is to ensure that query-relevant clips are predicted by enforcing each clip to be explicitly conditioned by the query.
%Our key focus is to ensure that the model prediction for each clip is highly relevant to the query.
Our key focus is to ensure that the model's prediction for each clip is highly dependent on the query.
% by enforcing each clip to be explicitly conditioned by the query. :)
% hmm...
% \SE {} % "query-relevant clips are predicted" 이 문장이 좀 애매한거 같습니다. relevant 클립을 놓지지 않고 찾는 것을 보장한다? 이런 느낌인지 아니면 높은 saliency 를 주는게 목적이다? model prediction이 query-relevance를 반영하는 것을 보장한다?
% Our key focus is to ensure that the model prediction reflects query-relevance of clips by enforcing each clip to be explicitly conditioned by the query.
First, to fully utilize the contextual information in the query, we revise the transformer encoder to be equipped with cross-attention layers at the very first layers.
% 상익's thought :  single video - query간의 관계만 고려 - 같은 word가 더 많이 쓰이는 것을 보고 
% 교수님's thought : neg pair 를 쓰면 쿼리를 보지 않고서는 video clip간만 고려하는 것이 사라짐. 왜냐면 0으로 내보내야 하기 때문. --> SE: relative difference 만 고려하다가, 
By inserting a video as the query and a text as the key and value of the cross-attention layers, our encoder enforces the engagement of the text query in extracting video representation.
% 원준 교수님 코멘트 반영해서 다시
Then, in order to not only inject a lot of textual information into the video feature but also make it fully exploited, we leverage the negative video-query pairs generated by mixing the original pairs.
Specifically, the model is learned to suppress the saliency scores of such  negative~(irrelevant) pairs.
Our expectation is the increased contribution of the text query in prediction since the videos will be sometimes required to yield high saliency scores and sometimes low ones depending on whether the text query is relevant or not.
% \SE{}
% learns to?
% By suppressing the saliency scores of the irrelevant video-query pairs, the model learns to spotlight only the video-specific discriminative words in the query.
% % \SE{} % ====================== 상익 수정 ========================
% However, this architectural design still lacks the capability of identifying the video-descriptive keywords in the query.
% % However, this architectural design still lacks in identifying proper query relevance.
% This is because the current training scheme only focuses on the interactions of video and clips within a single video while neglecting information shared throughout the entire video.
% % We argue the problem of the current training scheme that only focuses on distinguishing the clips in a single video while neglecting information shared throughout the entire video.
% Therefore, we leverage the negative video-query relationships to enhance the capability of identifying the contextual similarity of query and video clips.
% 
% 원준 원본 
% However, this architectural design heavily relies on the quality of the text query.
% Therefore, we leverage the negative video-query relationships to enable the model to emphasize key corresponding query features.
% By suppressing the saliency scores of the irrelevant video-query pairs, the model learns to spotlight only the video-specific discriminative words in the query.
% =========================================================
Lastly, to apply the dynamic criterion to mark highlights for each instance, we deploy a saliency token to represent the entire video and utilize it as an input-adaptive saliency criterion. 
With all components combined, our QD-DETR produces query-dependent video representation by integrating source and query modalities.
This further allows the use of positional queries~\cite{dabdetr} in the transformer decoder.
% Furthermore, we can exploit the advanced DETR decoder architectures using the positional information, e.g., DAB-DETR, since our encoded tokens consist of identical position representations from a single modality.
% \SE{} % ====================== 상익 수정 ========================
% Furthermore, we can exploit the advanced DETR decoder architectures using the positional information, e.g., DAB-DETR, since our video clip tokens consist of identical position representations from a single modality.
% 원준 원본
% It also enables the use of advanced DETR decoder architectures, e.g., DAB-DETR, for the first time, as these works exploit the position information within a single modality.
% =========================================================
Overall, our superior performances over the existing approaches validate the significance of the role of text query for MR/HD.
% Our extensive experiments on QVHighlights, TVSum, and Charades-STA datasets validate the significance of considering the role and the quality of text query.

% All components combined with dynamic anchor moments for the query of decoder, our FOQUE fosters the query-dependent video representation, thereby making the 
% All components combined, our modified transformer encoding process fosters the query-dependent video representation thereby achieving the state-of-the-art results on various benchmarks of moment-retrieval and highlight detection.
	
% -	Video Platform & Streamer & Consumer의 증가. 
% Video는 다른 데이터 타입보다 정보가 많아 유용하지만, 이는 다른 말로 해석하면 video를 보는 것은 time-consuming 하고, 원하는 것을 찾아보기에는 힘들 수 있음.
% 따라서, 많은 매체에서는 사람들의 더 많은 이목을 끌기 위해 highlight 비디오라는 것을 편집하여 공유도 함.
% 하지만, highlight video를 만들기 위해 사람의 노력이 필요한 현 시점에서, This spotlights the need to retrieve the user-requested / Highlight moments in the video.

% -	이전에도 이러한 문제를 해결하기 위해 (asdfasdf) for moment retrieval, (asdfasdf) for highlight detection 등이 제안 되었지만, 이들은 비디오의 특정 영역을 찾는다는 공통된 목적을 가지고 있으면서도, 데이터 셋의 한계로 인해 따로 연구되었음. 이를 문제 삼으며, 최근에는 두 task를 동시에 학습할 수 있는 dataset이 소개 되었는데, 컴퓨터비전에서 최근 각광을 받고 있는 Transformer 모델 도입과 함께 큰 발전을 거듭하고 있음.

% -	구체적으로, 이 두가지 task를 수행하기 위해서는 transformer를 두가지 방법으로 이용할 수 있는데, moment-DETR 처럼 moment 를 clip의 set 단위로 예측할 수 있고, UMT 처럼 clip-wise prediction을 할 수 있음. 하지만, 이들은 query를 condition이 아닌 video와 동등한 레벨로 취급하거나 [mDETR], 매 클립이 self-attention으로 mixing 된 후에 condition을 걸어주어 clip간의 차이를 확실하지 이용하지 못하였고, 또한, 확실하게 condition으로 주지 못하였고, video와 query 사이의 관계를 한정적으로만 이용하였다.

% -	따라서, we explore three different ways to fully exploit query information. First, we design one-way cross-attention layer to condition every clip with the query features. Then, we utilized the negative video-text pairs to better model the relationships between the video and the text embeddings. Lastly, we define the saliency token to be the video-query dependent saliency estimator.


















% ===================== neg pair 부분 ===========================
% Nevertheless, the current training scheme, only considering the given video-query pair, still disturbs the model from identifying proper query-relevance prediction.
% In detail, the model focus on learning the fine-grained discrepancy between video clips, while neglecting the information they share, which contains significant clues to understand the context of video.
% Therefore, we leverage the negative video-query relationships to enhance the capability of identifying the contextual similarity of query and video clips.
% Therefore, we leverage the negative video-query relationships by suppressing those pairs, so that enhance the capability of identifying the contextual similarity of query and video clips.
% We hypothsize the diversity in query-video pairs are insufficient to learn the general relationship between text query and video.
% Therefore, we leverage the negative video-query relationships by suppressing the saliency scores of the irrelevant video-query pairs.
% However, this architectural design still lacks in identifying proper query relevance.
% We argue that the current training scheme only focuses on learning the fine-grained discrepancy between clips in a single video, while neglecting the information they share, which contains significant clues to understand the context of the video.
% Therefore, we leverage the negative video-query relationships to enhance the capability of identifying the contextual similarity of query and video clips.
% However, this architectural design still lacks in identifying proper query relevance.
% We argue the problem of the current training scheme that only focuses on learning the fine-grained discrepancy between clips in a single video.
% That is, the current design neglects the information shared throughout the video, although it contains significant clues to understand the context of the video.
\section{Optimal Control Problem} \label{sec:preliminaries}
For a time horizon $t \in [t_0,\ t_{\mathrm{f}}]$, system state vector $\mathbf{x}(t) \in \mathbb{R}^{n_{\mathbf{x}}}$, input vector $\mathbf{u}(t) \in \mathbb{R}^{n_{\mathbf{u}}}$, algebraic  state vector $\mathbf{z}(t) \in \mathbb{R}^{n_{\mathbf{z}}}$, and parameter vector $\mathbf{p} \in \mathbb{R}^{n_{\mathbf{p}}}$, in \impact we consider general OCP formulations  of the canonical form
%
\begin{mini!}[2]
{{{\mathbf{x}, \mathbf{u}, \mathbf{z}}}}
{\int_{t_0}^{t_\mathrm{f}}V(\mathbf{x}(t), \mathbf{u}(t), \mathbf{p}, t)\mathrm{d}t + V_{f}(\mathbf{x}(t_\mathrm{f}),\mathbf{p}) \label{eq:ocp_objective}}{\label{eq:ocp}}{}
\addConstraint{B(\mathbf{x}(t_0),\mathbf{x}(t_f),\mathbf{p})}{ \leq 0 \label{eq:ocp_init}}{}
\addConstraint{\mathbf{\dot{x}}(t) }{ = \xi(\mathbf{x}(t),\mathbf{u}(t), \mathbf{z}(t),\mathbf{p}),\hspace{1.0ex} t \in [t_0, t_\mathrm{f}] \label{eq:ocp_dynamics}}{}
\addConstraint{\Gamma(\mathbf{x}(t),\mathbf{u}(t), \mathbf{z}(t),\mathbf{p}) }{ = 0,\hspace{3.4ex} t \in [t_0, t_\mathrm{f}] \label{eq:ocp_algebraic}}{}
\addConstraint{h(\mathbf{x}(t), \mathbf{u}(t), \mathbf{p})}{\leq 0,\hspace{8.2ex} t\in [t_0, t_\mathrm{f}], \label{eq:ocp_path}}{}
%\addConstraint{r_\mathrm{f}(\mathbf{x}(t_\mathrm{f}), \mathbf{p})}{\leq 0 \label{eq:ocp_terminal_constraints}}{},
\end{mini!}
where $V(\cdot)$ and $V_\mathrm{f}(\cdot)$ in \eqref{eq:ocp_objective} are smooth nonlinear functionals that define Lagrange and Mayer terms, respectively, $\mathbf{p}$ is a parameter vector that typically contains a vector of state measurements or estimations $\mathbf{x}_{\mathrm{meas}} \in \mathbb{R}^{n_{\mathbf{x}}}$, \eqref{eq:ocp_init} defines a boundary constraint that typically takes the form of $\mathbf{x}(t_0)=\mathbf{x}_{\mathrm{meas}}$, \eqref{eq:ocp_dynamics} and \eqref{eq:ocp_algebraic} define a system of differential-algebraic equations (DAE) representing the system model, while \eqref{eq:ocp_path} represent general path constraints. Note that the inclusion of the algebraic state vector $\mathbf{z}$ and the algebraic equation \eqref{eq:ocp_algebraic} is not required and can be omitted for systems represented solely by ordinary differential equations (ODE). The OCP specified by the user in \impact (see Section \ref{sec:workflow}) is automatically transformed into the canonical form \eqref{eq:ocp}.

%For NLP and QP
%direct methods (first discretize, then optimize) 
%solvers interfaced with \impact via \codename{CasADi},
% -- e.g., \codename{IPOPT}, the \codename{FSLP solver}, or the SQP method of \codename{CasADi} --, 
%the OCP is transcribed using \codename{Rockit}'s multiple-shooting, single-shooting, direct collocation, or B-spline methods into
Direct transcription of \eqref{eq:ocp} leads to a finite-dimensional optimization problem
% , i.e., an NLP, 
of the form
\begin{equation}
\label{eq:nlp}
\begin{split}
\min_{\mathbf{w}} \: f(\mathbf{w})\quad\mathrm{s.t.} \:\: g(\mathbf{w})=0,\:\: h(\mathbf{w})\leq 0,
\end{split}
\end{equation}
where 
% $\mathbf{w} := \begin{bmatrix} x_0^\top & u_0^\top & \hdots & u_{N-1}^\top & x_N^\top\end{bmatrix}^\top$, 
$\mathbf{w} \in \mathbb{R}^{n_{\mathbf{w}}}$ is the vector of decision variables, 
and $N := T/\delta_t \in \mathbb{N}$ is the number of discretization points within the prediction horizon $T := (t_{\mathrm{f}} - t_0) \in \mathbb{R}_{>0}$ with a sampling time $\delta_t \in \mathbb{R}_{>0}$. The first- and second-order derivatives arising in the optimality conditions of  \eqref{eq:nlp} are typically large and sparse.
%The solvers get access to code-generated sparse first- or second-order derivatives by \codename{CasADi}'s algorithmic differentiation (AD).

%Conversely, for external tools interfaced with \impact via plugins, 
% such as \codename{GRAMPC} -- which implements an indirect method (first optimize, then discretize) --, \codename{Acados}, and \codename{FATROP}, 
%the OCP \eqref{eq:ocp} is transformed into the structure expected by the solver selected by the user.
\section{Toolchain Description} \label{sec:toolchain_overview}

This section presents an overview of the structure
%and the workflow 
of the \impact toolchain. First, a description of the fundamental dependencies of the toolchain is given. Next, the elements of the toolchain, and a workflow from MPC specification to deployment are presented. Finally, 
the use of 
the unified API to communicate with the MPC solvers is detailed.


\subsection{Fundamental Dependencies}

\label{sec:dependencies}

%\textcolor{red}{IMPACT: overview (could be described in Matlab or Python)(deployment is in C or Simulink, computationally efficient), back end (rockit) ,model definition (yaml file), problem formulation, solver instantiation, code generation (C and Simulink block, already tested real time platforms as speedgoat and beckhoff), it can be used in any platform which interprets C code}
%
% \impact is written in \codename{Python}, with \codename{MATLAB} bindings. 
The fundamental software dependencies required for installing and using \impact are: \codename{CasADi}, \codename{Rockit}, and the \codename{Python} modules \codename{pyyaml} and \codename{lxml}.

\impact relies on \codename{CasADi}
to handle 
% variables and 
expressions that model 
% functions in 
the objective and constraints as expression graphs, and to allow the saving/loading of functions and algorithms and their transfer across \codename{Python} and \codename{MATLAB}.

If instructed by the user, \impact relies on \codename{Rockit} to perform a chosen OCP transcription method. At the moment, multiple-shooting, single-shooting, direct collocation, and a B-spline method are supported.
Derivatives of the resulting NLP objective and constraints are obtained through \codename{CasADi}'s sparsity-exploiting algorithmic differentiation (AD), optionally made more efficient by \codename{C}-code generation.

\codename{Rockit} further provides interfaces to third-party OCP frameworks such as \codename{GRAMPC}, \codename{Acados} and \codename{FATROP}. Initially meant for offline usage, these interfaces are being refactored to use the \codename{CasADi} codegen \codename{C} API, such that they can be embedded in \codename{CasADi} expressions and are compatible with \codename{C}-code generation.
%
Lastly, \codename{Rockit} relies on \codename{CasADi} for importing functional mock-up units (FMU) for input/output representations of systems.
% and its importer of functional mock-up units (FMU) for input/output representations of systems.
% \impact relies on the nonlinear optimization and algorithmic differentiation framework \casadi \textcolor{blue}{\impact relies on symbolic framework of \casadi to model and mapping all type of variable and expressions involved in the OCP. When the transcription OCP to NLP uses a \casadi method, its algorithmic differentiation framework to compute the derivatives of such expressions as needed by the solvers. } for handling the variables and expressions that define OCP \eqref{eq:ocp}, and to compute the derivatives of such expressions as needed by the solvers. Moreover, it leverages \codename{CasADi}'s capabilities for \codename{C}-code generation and serialization of functions and solvers, and for importing functional mock-up units (FMU) for input/output representations of systems.
% As mentioned in Section \ref{sec:introduction}, multiple state-of-the-art solvers are made available to \impact through \codename{CasADi}.

% The dependency of \impact on \casadi and \codename{Rockit}, and the inherent modularity that such dependencies represent, allow the direct and transparent availability of new solvers in \impact whenever an interface to such solvers is added to either \casadi or \codename{Rockit}.
%
The libraries \codename{pyyaml} and \codename{lxml} are used within \impact to handle YAML and XML files required in the process of loading system models and generating \codename{Simulink} blocks when exporting the solution, respectively.


\subsection{Structure of the \codename{IMPACT} toolchain} \label{sec:structure}

A graphical illustration of the overall structure of the \impact toolchain is shown in Fig. \ref{fig:structure}. 
%
\begin{figure}[htpb]
\centering
\includegraphics[width=0.33\textwidth]{figures/IMPACTstructure.png}
\vspace{-1.5ex}
\caption{Overview architecture of the \impact toolchain showing the interaction between modules.}
\label{fig:structure}
%\vspace{-2mm}
\end{figure}
%
% As already mentioned in Section \ref{sec:dependencies}, 
\casadi and \codename{Rockit} play a vital role within \impact by providing essential capabilities to the toolchain regarding, e.g., expression handling,
% OCP transcription, 
and interfacing third-party solvers. In fact, the dependency of \impact on \casadi and \codename{Rockit}, and the inherent modularity that such dependencies represent, allow the direct and transparent availability of new solvers in \impact whenever an interface to such solvers is added to either \casadi or \codename{Rockit}.

Apart from 
% such dependencies, 
\casadi and \codename{Rockit}, 
\impact is composed by the MPC class, which is instantiated by the user to use all the functionalities of \codename{IMPACT}. This class is divided into five sub-modules, which are described as follows.



\textit{1) Model definition:} \impact provides the user with two ways of defining (nonlinear) system models to be used within the OCP definition: (i) manually defining state and control variables in addition to differential and algebraic equations, as done in \codename{Rockit} or (ii) using YAML files -- i.e., which are human-readable files for data serialization -- by using the \lstinline[style=pythoninline]{add_model()} method of the MPC class. 
By using YAML files, we aim to standardize the definition of models based on ODEs or DAEs
for their use in any software tool that is compatible with YAML. 
This model representation allows the use of both inline definition of equations and external serialized \casadi functions, i.e., \lstinline[style=pythoninline]{*.casadi} files, as shown respectively in the two YAML snippets of a pendulum system that follow.
%
% We present a code-snippet that shows the declaration of three variables {\lstinline[columns=fixed]{q}}, {\lstinline[columns=fixed]{qd}}, and {\lstinline{qdd}}  within the P2P task. Along with \lstinline{name} and \lstinline{mid}, the declaration takes as arguments (i) the \lstinline{type} and (ii) a \lstinline{shape} which sets the two-dimensional size of the variable.

\vspace{-4ex}
\begin{minipage}[t]{.54\linewidth}
     \begin{lstlisting}[language=yaml]
equations:
  inline:
    ode:
      phi: dphi
      dphi: L*cos(phi)*sin(phi)...
differential_states: 
  - name: phi
  - name: dphi
controls: 
  - name: F
constants:
  inline:
    L: 2
            \end{lstlisting}
\end{minipage}\hfill
\begin{minipage}[t]{.44\linewidth}
      \begin{lstlisting}[language=yaml]
equations:
  external:
    type: casadi_serialized
    file_name: ode.casadi
differential_states: 
  - name: phi
  - name: dphi
controls: 
  - name: F
            \end{lstlisting}
\end{minipage}

\vspace{-2ex}

% \begin{minipage}[t]{.54\linewidth}
%     \begin{minted}
%     [
%     % frame=lines,
%     % framesep=2mm,
%     % baselinestretch=1.2,
%     % bgcolor=backcolour,
%     fontsize=\tiny,
%     % linenos
%     ]
%     {yaml}
% equations:
%   inline:
%     ode:
%       phi: dphi
%       dphi: L*cos(phi)*sin(phi) ...
% differential_states: 
%   - name: phi
%   - name: dphi
% controls: 
%   - name: F
% constants:
%   inline:
%     L: 2
%     \end{minted}
% \end{minipage}\hfill
% \begin{minipage}[t]{.44\linewidth}
%     \begin{minted}
%     [
%     % frame=lines,
%     % framesep=1mm,
%     % baselinestretch=1.2,
%     % bgcolor=backcolour,
%     fontsize=\tiny,
%     % linenos
%     ]
%     {yaml}
% equations:
%   external:
%     type: casadi_serialized
%     file_name: ode.casadi
% differential_states: 
%   - name: phi
%   - name: dphi
% controls: 
%   - name: F

  
%     \end{minted}
% \end{minipage}

% % \begin{figure}[htpb]
% % \centering
% \begin{minipage}[t]{.35\linewidth}
%     \begin{minted}
%     [
%     % frame=lines,
%     % framesep=2mm,
%     % baselinestretch=1.2,
%     bgcolor=backcolour,
%     fontsize=\scriptsize,
%     % linenos
%     ]
%     {yaml}
% equations:
%   inline:
%     ode:
%       pos: m*dpos
% differential_states: 
%   - name: pos
% controls: 
%   - name: dpos
% constants:
%   inline:
%     m: 1
%     \end{minted}
% \end{minipage}\hfill
% \begin{minipage}[t]{.60\linewidth}
%     \begin{minted}
%     [
%     % frame=lines,
%     framesep=2mm,
%     % baselinestretch=1.2,
%     bgcolor=backcolour,
%     fontsize=\scriptsize,
%     % linenos
%     ]
%     {yaml}
% equations:
%   external:
%     type: casadi_serialized
%     file_name: point_equations.casadi
% differential_states: 
%   - name: pos
% controls: 
%   - name: dpos
  
  
  
%     \end{minted}
% \end{minipage}
% % %\vspace{-3mm}
% % \caption{Overview architecture of the \impact toolchain showing the dependencies and the interaction between modules.}
% % \label{fig:structure}
% % %\vspace{-2mm}
% % \end{figure}
% \begin{rem}
% The model defined within \impact can be linear or nonlinear.
% \end{rem}

\textit{2) FMU handler:} This module allows the user to load FMUs -- an industry standard to define containers that represent black-box (input/output) models -- as \casadi functions to be used within the OCP definition. By calling the \lstinline[style=pythoninline]{add_simulink_fmu()} method of the MPC class, the FMU is wrapped into a \codename{C++} file, which is then compiled and loaded back as a \casadi external function. When the FMU provides derivatives (forward derivatives per FMI standard 2.0), these can be used. Otherwise, \casadi uses finite differences. In a further stage, FMU handling should be absorbed into the YAML file approach.

\textit{3) MPC specification:} This module inherits the functionality of the OCP class of \codename{Rockit} to enable the specification of the OCPs underpinning MPC. With the OCP class, \codename{Rockit} allows the definition of (multi-stage) OCPs by (i) managing symbolic variables for system states $\mathbf{x}$, system inputs $\mathbf{u}$, and algebraic states $\mathbf{z}$, (ii) associating dynamics to $\mathbf{x}$ and algebraic equations to $\mathbf{z}$ as in \eqref{eq:ocp_dynamics} and \eqref{eq:ocp_algebraic}, (iii) composing the objective function \eqref{eq:ocp_objective} by adding functions evaluated at different instants within the OCP horizon, and (iv) setting path or boundary constraints to expressions or variables as in \eqref{eq:ocp_init} and \eqref{eq:ocp_path}. When choosing a third-party OCP plugin, restrictions on the problem specification may apply.
% Moreover, \codename{Rockit} provides direct methods such as multiple-shooting, single-shooting, direct collocation and B-splines to transcribe the defined OCP \eqref{eq:ocp} into an NLP \eqref{eq:nlp} that can be solved by any of the solvers interfaced by \codename{CasADi}, or external methods interfacing third-party optimization frameworks to transcribe the OCP into specific structures required by such tools, as already explained in Section \ref{sec:preliminaries}.
%
The MPC specification is also used for result post-processing capabilities which allow the user, for instance, to retrieve the values of specific variables or expressions in the OCP solution, or to interpolate the solution on a refined grid of integration points.

\textit{4) Code serialization:} This module allows the serialization of the expression graph created by \casadi to define the MPC solver. This allows to save an MPC solver instantiation (including the functions that define the OCP) into a 
% \lstinline[style=pythoninline]{.casadi} 
file that can be then loaded from \codename{Python}, \codename{MATLAB} or \codename{C++}. This module relies on the \lstinline[style=pythoninline]{save()} and \lstinline[style=pythoninline]{load()} methods of the MPC class.
In a similar fashion, instances of the \codename{IMPACT} Python module can also be saved and loaded.

\textit{5) Code export:} %This module inherits the functionality of the OCP class of \codename{Rockit},
%
By calling the \lstinline[style=pythoninline]{export()} method of the MPC class, the user generates multiple artifacts that allow the prototyping and deployment of a solver within an MPC implementation.
% Specifically, 
The main artifacts depicted in Figure \ref{fig:workflow} are (i) an \impact library callable through the \impact \codename{C} API, (ii) a numerical backend library containing a concrete low-level problem formulation coupled to a solver that is either included as fully self-contained \codename{C} code, linked in with a dependency on the \codename{C++} \casadi runtime library and chosen solver plugin such as \codename{IPOPT}, or linked in as third-party dependencies such as \codename{OSQP} or \codename{GRAMPC}, and (iii) an MPC-ready \impact \codename{Simulink} block. 
%
%
% Conversely, the \impact \codename{C} API is an interface that makes the MPC specification agnostic to the solver being code-generated or not. It creates a dependency on the \codename{C++} \casadi runtime library and the NLP solver itself. This interface is accompanied by \codename{Python} bindings to make the \impact \codename{C} API callable from \codename{Python} as well.
%The \impact \codename{C} API is an interface that makes the MPC specification agnostic to the solver being code-generated or not. 
%It creates a dependency on the \codename{C++} \casadi runtime library and the NLP solver.

The \impact library is accompanied by \codename{Python} bindings. More details can be found in Section \ref{sec:C_API}.

Self-contained \codename{C} code
% , also referred as generated \codename{C}-code, 
is dependency-free and requires only a \codename{C} compiler to be deployed in any compatible target. It requires every element of the MPC solver instantiation to be compatible with code-generation -- e.g., if the user selects \codename{IPOPT}, which cannot be code-generated, no self-contained \codename{C} code of the MPC solver can be generated.
% . For instance, \codename{IPOPT} cannot be code-generated and, therefore, no self-contained \codename{C} code of the MPC solver can be generated if the user selects \codename{IPOPT} as nonlinear solver.

The 
% MPC-ready 
\impact \codename{Simulink} block is based on a code-generated custom \codename{C} S-function. \impact generates a \lstinline[style=pythoninline]{.slx} container based on the \codename{C} API  and linked to the generated \codename{C} files, and can be directly used within a \codename{Simulink} model. \codename{Simulink} has become a de-facto tool for simulation and prototyping of control systems. This artifact allows \impact users to easily prototype their designed MPC with no extra steps. Moreover, if the linked \codename{C} file is self-contained, the generated \codename{Simulink} block would be compatible with the \codename{Simulink} coder. This way, the MPC-ready \impact \codename{Simulink} block could be code-generated as part of a larger \codename{Simulink} model and deployed into real-time targets without additional engineering effort.
Compatibility with \codename{Simulink} coder is also expected when third-party dependencies are open-source and \codename{C} based, but this worflow has not yet been validated.

The three artifacts include \casadi functions for the MPC solver (with parametric and hot-start inputs), forward simulation of the system dynamics, DAEs, performance objective, among others. In addition, they output statistics of the solution such as solution time, status, optimality of the solution and number of iterations. These provide the user with a great deal of flexibility to implement and prototype the MPC solution within a simulated or real control loop. Note that once the artifacts are generated, the only values that can be modified within the problem are the parameters $\mathbf{p}$.

% {\color{red}
% https://groups.google.com/g/casadi-users/c/rnBC_a2WULk/m/slpyu-QGAgAJ
% Both methods end up looking like a mex-file, but the dependencies (libraries) of that mex-file are much different.

% Method 1 works with true code-generation: you get a dependency on self-contained C code.
% There is a reasonable chance this will work on your embedded system as-is.

% Method 2 does not involve code-generation: you get a dependency on the CasADi runtime (a library of C++ code) and a dependency on Ipopt (a library involving C++ and Fortran).
% This will work effectly on your desktop, but getting this to run on dSpace could give you a real headache.
% It would involve making source builds of casadi and Ipopt for your embedded target. Check it your target has support for C++ and Fortran.
% If not, you'd probably need some llvm magic to make that happen.}


 
%rockit based, all available solver in Rockit are available in IMPACT too(including state of the art in-house developed solvers ref(David SLP, Fatrop)))
%Provides a C application programming interface (API)
%
%
%Flexible (C/Python/Matlab), Extensible (solver plugins), Composition (builds on top of solid background)





\subsection{Toolchain Workflow}
\label{sec:workflow}
This section has given details on the architecture of \impact and its dependencies. It is now necessary to explain the workflow proposed within \impact for the user to specify an MPC, prototype its solution, and deploy it. A general overview of the workflow is presented in Fig. \ref{fig:workflow}.

\begin{figure}[htpb]
\centering
\includegraphics[width=0.43\textwidth]{figures/IMPACTworkflow.png}
\vspace{-1.5ex}
\caption{Overview of the workflow of the \impact toolchain.}
\label{fig:workflow}
% \vspace{-5ex}
\end{figure}

The workflow in the \impact toolbox is described as follows by means of a simple example involving the control of a pendulum. A description of each step is provided followed by the code snippet associated with it. 

First of all, the user must import the \impact module and instantiate an object of the MPC class. The argument \lstinline[style=pythoninline]{T=2.0} sets the prediction horizon $T$ to a fixed value. However, $T$ could be considered as a decision variable within a free time problem by passing \lstinline[style=pythoninline]{T=FreeTime(2.0)} as argument, where \lstinline[style=pythoninline]{2.0} is an initial guess for this variable.
{
\vspace{-0.5ex}
\begin{lstlisting}[style=pythonstyle]
from impact import *
mpc = MPC(T=2.0)
\end{lstlisting}
\vspace{-1ex}
}
Afterwards, the system model should be declared. As mentioned in Section \ref{sec:structure}, this can be done by manually writing the ODE or DAE -- i.e., using instances of states and inputs within the MPC class --, or by loading a YAML file, which is the option shown in the code snippet below.
{
\vspace{-2ex}
\begin{lstlisting}[style=pythonstyle]
pendulum = mpc.add_model('pendulum','pendulum.yaml')
\end{lstlisting}
\vspace{-1.5ex}
}

% Once the system model has been defined, the user can instantiate the elements of the vector of parameters $\mathbf{p}$. In the example, these are the current state (\lstinline[style=pythoninline]{x_current}), a desired final state (\lstinline[style=pythoninline]{x_final}), and the weights (\lstinline[style=pythoninline]{weights}) to be used in the objective function.
Once the model has been defined, the user can instantiate the elements of the vector of parameters $\mathbf{p}$, such as the current state (\lstinline[style=pythoninline]{x_0}), a desired final state (\lstinline[style=pythoninline]{x_f}), and the weights (\lstinline[style=pythoninline]{Wt}) to be used in the objective.
{
\vspace{-0.5ex}
\begin{lstlisting}[style=pythonstyle]
x_0 = mpc.parameter('x_0',pendulum.nx)
x_f = mpc.parameter('x_f',pendulum.nx)
Wt  = mpc.parameter('Wt',2)
\end{lstlisting}
\vspace{-1.5ex}
}

In a high-level way, the objective and constraints are then defined 
% within the MPC specification environment 
by using the methods \lstinline[style=pythoninline]{mpc.add_objective()} and \lstinline[style=pythoninline]{mpc.subject_to()} directly using state and control names. Note that, by using \lstinline[style=pythoninline]{mpc.at_t0()} or \lstinline[style=pythoninline]{mpc.at_tf()} the user can specify boundary constraints.
\begin{lstlisting}[style=pythonstyle]
# Objective 
mpc.add_objective(mpc.integral(Wt[0]*pendulum.F**2 + Wt[1]*pendulum.dphi**2))
# Boundary constraints
mpc.subject_to(mpc.at_t0(pendulum.x) == x_0)
mpc.subject_to(mpc.at_tf(pendulum.x) == x_f)
# Path constraints
mpc.subject_to(-2 <= (pendulum.F <= 2 ))
\end{lstlisting}
\vspace{-1.5ex}
The nonlinear optimization solver to be used is defined by means of the \lstinline[style=pythoninline]{mpc.solver()} method, for which the user provides several 
% solver 
options depending on the selected solver. In this case, the SQP method of \casadi has been selected.
\begin{lstlisting}[style=pythonstyle]
mpc.solver('sqpmethod',options={...})
\end{lstlisting}
\vspace{-1.5ex}
The transcription method is defined either by using a direct method provided by \codename{Rockit} or an external method provided by a plugin to an optimization tool such as \codename{GRAMPC}. Here, \lstinline[style=pythoninline]{N} corresponds to the number of discretization points $N$, while \lstinline[style=pythoninline]{intg='rk'} sets a $4$th-order Runge Kutta integrator to discretize the system dynamics.
\begin{lstlisting}[style=pythonstyle]
# Direct method, option 1
method = MultipleShooting(N=40,intg='rk')
# External method, option 2
method = external_method('grampc',N=40,grampc_options=...)
# Actually set the selected method within the MPC environment
mpc.method(method)
\end{lstlisting}
\vspace{-1.5ex}
The user can now set the values of the parameters $\mathbf{p}$ and execute the solver directly in \codename{Python}.
\begin{lstlisting}[style=pythonstyle]
mpc.set_value(x_0, [0.5,0,0,0]) # Set parameters
mpc.set_value(x_f, [0,0,0,0])
mpc.set_value(Wf, [1,1]))
solution = mpc.solve() # Solve the OCP
\end{lstlisting}
Finally, the user can generate the three artifacts explained in Section \ref{sec:structure} by executing the following code.
\begin{lstlisting}[style=pythonstyle]
mpc.export("pendulum")
\end{lstlisting}
\vspace{-1.5ex}
%%%%%%%%%%%%%%%%%%%%%%%%%%%%%%%%%%%%%%%%%%%%%%%%%%%%%%%%%%%%%%%%%%%%%%%%
% The workflow in the IMPACT toolbox is described as follows by means of a simple example involving the control of a pendulum. A code snippet A description of each step is provided followed by the code snippet associated with it. 
% {
% % \vspace{-3ex}
% \begin{minted}[bgcolor=backcolour,fontsize=\scriptsize]{python3}
% from impact import *
% # Instantiate MPC object
% mpc = MPC(T=2.0)
% # Define model
% pendulum = mpc.add_model('pendulum','pendulum.yaml')
% # Define parameters
% x_current = mpc.parameter('x_current',pendulum.nx)
% x_final   = mpc.parameter('x_final',pendulum.nx)
% weights   = mpc.parameter('weights',2)
% # Define objective 
% mpc.add_objective(mpc.integral(weights[0]*pendulum.F**2 
%                              + weights[1]*cart_pendulum.pos**2))
% # Boundary constraints
% mpc.subject_to(mpc.at_t0(pendulum.x) == x_current)
% mpc.subject_to(mpc.at_tf(pendulum.x) == x_final)
% # Path constraints
% mpc.subject_to(-2 <= (pendulum.F <= 2 ))
% mpc.subject_to(-2 <= (pendulum.pos <= 2))
% # Set solver
% mpc.solver('sqpmethod',options={...})
% # Direct method, option 1
% method = MultipleShooting(N=40,intg='rk')
% # External method, option 2
% method = external_method('grampc',N=40,grampc_options=...)
% # Actually set the selected method within the MPC environment
% mpc.method(method)
% # Set parameter values
% mpc.set_value(x_current, [0.5,0,0,0])
% mpc.set_value(x_final, [0,0,0,0])
% mpc.set_value(weights, [1,1]))
% # Solve the OCP
% solution = mpc.solve()
% # Export code
% mpc.export("pendulum")
% \end{minted}
% }

% First of all, the user must import the \impact library and instantiate an object of the MPC class. The argument \lstinline[style=pythoninline]{T=2.0} sets the prediction horizon $T$ to a fixed value. However, $T$ could be considered as a decision variable within a free time problem by passing \lstinline[style=pythoninline]{T=FreeTime(2.0)} as argument, where \lstinline[style=pythoninline]{2.0} is an initial guess for this variable.

% Afterwards, the system model should be declared. As mentioned in Section \ref{sec:structure}, this can be done by manually writing the ODE or DAE --i.e., using instances of states and inputs within the MPC class--, or by loading a YAML file, which is the option shown in the code snippet.

% Once the system model has been defined, the user can instantiate the elements of the vector of parameters $\mathbf{p}$. In the example, these are the current state (\lstinline[style=pythoninline]{x_current}), a desired final state (\lstinline[style=pythoninline]{x_final}), and the weights (\lstinline[style=pythoninline]{weights}) to be used in the objective function.

% The objective and constraints are then defined within the MPC specification environment by using the methods \lstinline[style=pythoninline]{mpc.add_objective()} and \lstinline[style=pythoninline]{mpc.subject_to()}. Note that, by using \lstinline[style=pythoninline]{mpc.at_t0()} or \lstinline[style=pythoninline]{mpc.at_tf()} the user can specify boundary constraints.

% The nonlinear optimization solver to be used is defined by means of the \lstinline[style=pythoninline]{mpc.solver()} method, for which the user can provide several solver options depending on the selected solver. In this case, the SQP method of \casadi has been selected.

% The transcription method is defined either by using a direct method provided by \codename{Rockit} or an external method provided by a plugin to an optimization tool such as \codename{GRAMPC}. Here, \lstinline[style=pythoninline]{N} corresponds to the number of discretization points $N$, while \lstinline[style=pythoninline]{intg='rk'} sets a $4$th-order Runge Kutta integrator to discretize the system dynamics.

% The user can now set the values of the parameters and execute the solver directly in \codename{Python}.

% Finally, the user can export the code into the three artifacts explained in Section \ref{sec:structure}.

% This export command generates the files:
% \begin{itemize}
%     \item serialized function: \lstinline[style=pythoninline]{pendulum.casadi}
%     \item \codename{C} API: \lstinline[style=pythoninline]{pendulum.c}, \lstinline[style=pythoninline]{pendulum.h}
%     \item self-contained \codename{C}-code: \lstinline[style=pythoninline]{pendulum_codegen.c}, \lstinline[style=pythoninline]{pendulum_codegen.h} 
%     \item \lstinline[style=pythoninline]{library_pendulum.slx}, \lstinline[style=pythoninline]{pendulum_s_function_level2.c}
%     \item \lstinline[style=pythoninline]{impact.py}
% \end{itemize}

%
For rapid prototyping of the exported code the user can: (i) run the generated \texttt{hello\_world} examples in \codename{C} or \codename{Python} which use the \codename{C} API, or (ii) use the generated \codename{Simulink} block from the generated \lstinline[style=pythoninline]{.slx} file in a \codename{Simulink} model. Since we are using the SQP method of \codename{CasADi}, which is compatible with code-generation, in this example we could deploy the code-generated code or the \codename{Simulink} block in a real-time target architecture.

% probably I overlooked it, but is there a description of the things available to analyse the solving time, optimality, etc.... 


\subsection{\codename{IMPACT} \codename{C} API}
\label{sec:C_API}

The \codename{IMPACT} \codename{C} API defines a unified API to use the MPC solver exported by \codename{IMPACT}. It represents the core artifact of the code export functionality and is a layer of abstraction that makes the implementation agnostic to the content of the generated MPC solver, e.g., self-contained \codename{C} code with \codename{C} solver, \codename{C++} code that still depends on \codename{CasADi} and/or third-party solvers. 
% In addition, 
It provides rich functions to set the inputs of the MPC solver (parameters) and get the values of variables from the optimized solution.
For the sake of completeness, we present below a minimum example of the use of the \impact \codename{C} API 
% \textcolor{red}{from \codename{Python}} 
for the pendulum example presented in Section \ref{sec:workflow}.
% \begin{minted}[
%     % bgcolor=backcolour,
%     fontsize=\tiny]{python3}
% from impact import Impact
% # Load the generated artifact
% impact = Impact("pendulum",src_dir="..")
% # Example - Set a parameter
% impact.set("x_0", impact.ALL, 0, impact.FULL, x_meas)
% # Solve a single OCP
% impact.solve()
% # Get solution trajectory
% x_opt = impact.get("x_opt", impact.ALL, impact.EVERYWHERE, impact.FULL)
% \end{minted}
%
% Here, the exported artifact named \lstinline[style=pythoninline]{"pendulum"} is loaded. By using the \lstinline[style=Cinline]{set()} method, 
% the parameter \lstinline[style=pythoninline]{"x_0"} is set with a value \lstinline[style=pythoninline]{x_meas}. 
% % a value \lstinline[style=pythoninline]{x_meas} is assigned to parameter \lstinline[style=pythoninline]{"x_current"}.
% The second argument \lstinline[style=pythoninline]{impact.ALL} tells \impact that all the elements of the specified parameter are set. The user can use \lstinline[style=pythoninline]{"pendulum.phi"} instead of \lstinline[style=pythoninline]{impact.ALL} to set the parameter corresponding to the angle \lstinline[style=pythoninline]{phi}, for instance.
% The third argument \lstinline[style=pythoninline]{0} defines the time instance within the horizon for which the parameter is set. 
% Another option would be \lstinline[style=pythoninline]{impact.EVERYWHERE}, which sets 
% % Here, \lstinline[style=pythoninline]{impact.EVERYWHERE} would set
% the parameter for the whole horizon. The fourth argument represents a flag for data ordering, such as \lstinline[style=pythoninline]{impact.FULL}, \lstinline[style=pythoninline]{impact.ROW_MINOR}, or \lstinline[style=pythoninline]{impact.COLUMN_MAJOR}.
% The \lstinline[style=pythoninline]{solve()} method executes the MPC solver. Finally, the optimal solution is retrieved from the artifact using the \lstinline[style=pythoninline]{get()} method, with the arguments following the same logic as in \lstinline[style=Cinline]{set()}. 
% This example shows the solution of one OCP. 
% To implement MPC, the \lstinline[style=Cinline]{set()}, \lstinline[style=pythoninline]{solve()} and \lstinline[style=pythoninline]{get()} functions are executed within a loop.
% % While this example solves one OCP, an MPC implementation would require \lstinline[style=Cinline]{set()}, \lstinline[style=pythoninline]{solve()} and \lstinline[style=pythoninline]{get()} to be executed within a loop.
%
% \bigskip
%
% \begin{minted}[
%     % bgcolor=backcolour,
%     fontsize=\tiny]{c}
% // Create the artifact structure
% impact_struct* m = impact_initialize();
% // measure states in x_meas variable
% app_measure_state(x_meas);
% //  Example - Set a parameter
% impact_set(m, "x_current", IMPACT_ALL, 0, x_meas, IMPACT_FULL);
% // Solve a single OCP
% impact_solve(m);
% // Get input solution trajectory
% impact_get(m, "u_opt", IMPACT_ALL, IMPACT_EVERYWHERE, U, IMPACT_FULL);
% // send input solution to u
% app_send_control(u);
% \end{minted}
% \begin{minted}[
%     % bgcolor=backcolour,
%     fontsize=\tiny]{c}
% impact_struct* m = impact_initialize();
% impact_set(m, "x_0", IMPACT_ALL, 0, x_meas, IMPACT_FULL); // Set a parameter
% impact_solve(m); // Solve a single OCP
% // Get optimal input for first time instance
% impact_get(m, "u_opt", IMPACT_ALL, 0, U, IMPACT_FULL);
% \end{minted}
\begin{lstlisting}[style=Cstyle]
impact_struct* m = impact_initialize();
impact_set(m, "x_0", IMPACT_ALL, 0, x_meas, IMPACT_FULL); // Set a parameter
impact_solve(m); // Solve a single OCP
// Get optimal input for first time instance
impact_get(m, "u_opt", IMPACT_ALL, 0, U, IMPACT_FULL);
\end{lstlisting}
\vspace{-1.5ex}
Here, \lstinline[style=Cinline]{impact_initialize()} instantiates the exported \impact library and returns a pointer \lstinline[style=Cinline]{m} to it.
% The \lstinline[style=Cinline]{app_measure_state()} function reads the states in the variable \lstinline[style=Cinline]{x_meas}
By using the \lstinline[style=Cinline]{impact_set()} method, the parameter \lstinline[style=Cinline]{x_0} is set with a user-defined value \lstinline[style=Cinline]{x_meas}. %which should be previously defined by the user.
% in the structure \lstinline[style=Cinline]{m}. 
The third argument \lstinline[style=Cinline]{IMPACT_ALL} tells \impact that all the elements of the specified parameter are set. The user can use \lstinline[style=Cinline]{"pendulum.phi"} instead of \lstinline[style=Cinline]{IMPACT_ALL} to set the parameter corresponding to the angle \lstinline[style=Cinline]{phi}, for instance.
The fourth argument \lstinline[style=Cinline]{0} defines the time instance within the horizon for which the parameter is set. 
Another option is \lstinline[style=Cinline]{IMPACT_EVERYWHERE}, which sets the parameter for the whole horizon. The sixth argument combines flags representing repetition (\lstinline[style=Cinline]{IMPACT_FULL}, \lstinline[style=Cinline]{IMPACT_HREP}) and data ordering (\lstinline[style=Cinline]{IMPACT_ROW_MINOR},  \lstinline[style=Cinline]{IMPACT_COLUMN_MAJOR}).
The \lstinline[style=Cinline]{impact_solve()} function executes the MPC solver. Then, an element of the solution -- i.e., the optimal control input \lstinline[style=Cinline]{u_opt} -- is retrieved and assigned to a variable \lstinline[style=Cinline]{U} by using \lstinline[style=Cinline]{impact_get()}, with the arguments mirroring \lstinline[style=Cinline]{impact_set()}. 
% Finally, the \lstinline[style=Cinline]{app_send_control()} function assigns the control value to the variable \lstinline[style=Cinline]{u}.
This example shows the solution of one OCP. 
% To implement MPC, the \lstinline[style=Cinline]{app_measure_state()}, \lstinline[style=Cinline]{impact_set()}, \lstinline[style=Cinline]{impact_solve()}, \lstinline[style=Cinline]{impact_get()} and \lstinline[style=Cinline]{app_send_control()} functions are executed within a loop.
To implement (N)MPC, the \lstinline[style=Cinline]{impact_set()}, \lstinline[style=Cinline]{impact_solve()}, and \lstinline[style=Cinline]{impact_get()} methods are executed in a loop. More methods are defined in the API, e.g., to retrieve statistics and perform debugging.


%impact C API stuff needs to be corrected 
% you present it as something optional 
% it is central to the export functionality 
% S Function depends on it (fix arrows in figure) 
% impact C API is always there 
% it's a wall behind which anything can be hidden e.g. generated c code with c solver, c++ code that still depends on casadi, third party solvers
\section{Application Example Using IMPACT} \label{sec:case}

This section demonstrates the use and deployment of MPC using \codename{IMPACT}. We control the point-to-point motion (angular position) of a DC motor using a Speedgoat SN7233 real-time target machine. The control scheme estimates a constant disturbance that represents unmodeled dynamics -- e.g., input disturbances and static friction --, counteracts its effect, and achieves offset-free positioning.

\subsection{System Description}
The hardware has two components: the DC motor and the real-time target machine that executes the MPC algorithm and interfaces with the motor driver and a rotary encoder. 

% % \subsubsection{DC motor system}
The shaft of the DC motor is connected to a load and the rotary encoder, which provides a measurement $\mathbf{y}$ of the angular position of the rotor $\theta \in \mathbb{R}$. The motor has a driver that receives an analog voltage $v_{\mathrm{motor}} \in [-10,10]$
from the digital-to-analog converter (DAC) of the Speedgoat. 
% The encoder has $4096\ [\mathrm{pulses/rev}]$ and, therefore, a resolution of $0.0015\ [\mathrm{rad}]$. 
% --------------------------------------------------
% A second order linear model of the motor has been experimentally identified. This model contains one integrator and one real pole which models the linear motor friction. %We identified a second order linear model using data from a multisine excitation using 
% The \codename{LCtoolbox} \citep{Maarten2018LCToolbox} is used for this process.%, its characteristics and the procedure is described in \cite{Maarten2018LCToolbox}. The numerical values are
% % \begin{equation}
% % \label{eq:Statespacemodel}
% %     \begin{bmatrix}
% %         \dot{\mathrm{x}}_1\\
% %         \dot{\mathrm{x}}_2 
% %     \end{bmatrix}=
% %     \begin{bmatrix}
% %         0 & 1 \\
% %         0 & -9.215
% %     \end{bmatrix}
% %     \begin{bmatrix}
% %         \mathrm{x}_1\\
% %         \mathrm{x}_2 
% %     \end{bmatrix}+
% %     \begin{bmatrix}
% %         0\\
% %         442.9
% %     \end{bmatrix}\mathrm{u}
% % \end{equation}
% % \begin{equation*}
% %     \begin{bmatrix}
% %         \mathrm{y}
% %     \end{bmatrix}=
% %     \begin{bmatrix}
% %         1 & 0 
% %     \end{bmatrix}
% %     \begin{bmatrix}
% %         \mathrm{x}_1\\
% %         \mathrm{x}_2 
% %     \end{bmatrix}
% % \end{equation*}
%
% The model has the state space structure where $\mathbf{x} := [\theta, \dot{\theta}]^T \in \mathbb{R}^2$ are the states, angular position and velocity, $\mathbf{u} := v_{\mathrm{motor}}$ is the input voltage and $\mathbf{y} := \theta$ is the measurement of the angular position. This model does not consider the static friction component, hence by controlling the position, we have a model-plant mismatch.  
% \subsubsection{DC motor system}
%
%
For a state vector $\mathbf{x} := [\theta, \dot{\theta}]^T \in \mathbb{R}^2$ and a control input $\mathbf{u} := v_{\mathrm{motor}}$, a second order linear model for the motor 
\begin{equation} \label{eq:dynamics_motor}
    \dot{\mathbf{x}} = A\mathbf{x} + B\mathbf{u}
\end{equation}
has been experimentally identified using the \codename{LCtoolbox} \citep{Maarten2018LCToolbox}. This model contains one integrator and one real pole which models the linear motor friction, but does not consider the static friction component. 
% Hence, there is a model-plant mismatch.  

% \subsubsection{Real-time target machine}
\begin{figure}[htpb]
    \centering
    \includegraphics[width=0.36\textwidth]{figures/connectionScheme.png}
    \vspace{-3ex}
    \caption{
    Overview of the system.
    % Connection scheme between real-time target machine an motor system.
    }
    \label{fig:connectionScheme}
    % \vspace{-2ex}
\end{figure}
The Speedgoat SN7233 is a real-time target machine, programmed via \codename{Simulink}, that allows the deployment of controllers on, e.g., multi-core CPUs, FPGAs and PLCs, and facilitates rapid controller prototyping and hardware-in-the-loop simulations through a smooth integration with \codename{Simulink Real-Time}. 
% It is scalable and expandable making it cover a wide range of case scenarios.
%An Intel i7 CPU with clock speeds up to $4.2 \ [\mathrm{GHz}]$ processes the calculations. 
The target runs the MPC algorithm -- based on an \impact \codename{Simulink} block --, applies an analog voltage $v_{\mathrm{motor}}$ to the motor, and reads the digital signals from the rotary encoder, as depicted in Fig. \ref{fig:connectionScheme}. 
% \begin{figure}[htpb]
%     \centering
%     \includegraphics[width=0.37\textwidth]{figures/connectionScheme.png}
%     \vspace{-3ex}
%     \caption{Connection scheme between real-time target machine an motor system.}
%     \label{fig:connectionScheme}
%     % \vspace{-2ex}
% \end{figure}

\subsection{Problem Definition} \label{sec:problem_definition}
% We define an offset-free MPC strategy to track an angular reference while counteracting the effect of the model-plant mismatch and a constant voltage disturbance applied to the system input. 
% \subsubsection{Offset-free MPC}
% This methodology resides in two main changes with respect to the traditional MPC. The first one is the model augmentation with a disturbance model for the estimation process, with this, the states and disturbances are estimated. The second one is an additional problem that, based on the disturbance estimated, finds a modification of the states and input references which makes the system follow the reference fixing disturbance effects, this problem is called the steady state problem \citep{rawlings2017model}. 

% The estimation process is made each sample time, therefore we need a discretized system model with the form $  \mathbf{x}_{k+1} = \boldsymbol{A}\mathbf{x}_k+\boldsymbol{B}\mathbf{u}_k, \ \mathbf{y}_k=\boldsymbol{C}\mathbf{x}_k $, then, the augmentation for estimation is
% \begin{equation}
%     \label{eq:augmentedmodel}
%     \begin{bmatrix}
%         \hat{\mathbf{x}}_{k+1}\\
%         \hat{\mathbf{d}}_{k+1}
%     \end{bmatrix}=
%     \begin{bmatrix}
%         \boldsymbol{A} & \boldsymbol{B}_d\\
%         \boldsymbol{0} & \boldsymbol{A}_d
%     \end{bmatrix}
%     \begin{bmatrix}
%         \hat{\mathbf{x}}_k\\
%         \hat{\mathbf{d}}_k
%     \end{bmatrix}+
%     \begin{bmatrix}
%         \boldsymbol{B}\\
%         \boldsymbol{0}
%     \end{bmatrix}\mathbf{u}_k
% \end{equation}
% \begin{equation*}
%     \hat{\mathbf{y}}_{k}=\begin{bmatrix}
%         \boldsymbol{C} & \boldsymbol{C}_d
%     \end{bmatrix}\begin{bmatrix}
%         \hat{\mathbf{x}}_{k}\\
%         \hat{\mathbf{d}}_{k}
%     \end{bmatrix}
% \end{equation*}
% where $\hat{\mathbf{x}}_{k}$, $\hat{\mathbf{d}}_k$, $ \hat{\mathbf{y}}_k$ are states, disturbance and output estimations at discrete time $k$. $\boldsymbol{A}_d$ represents the disturbance model, the values of $\boldsymbol{B}_d$ and $\boldsymbol{C}_d$ are design parameters. In this example, we consider an input single constant disturbance, therefore $\boldsymbol{A}_d=1$ and $\boldsymbol{C}_d=0$, additionally, with $\boldsymbol{B}_d=\boldsymbol{B}$ the disturbance estimation has same units as the input, volts in this case. Kalman filter is the observer used to perform the estimation process. \\

% The steady state problem here is a simplified version of the general one. We find the states and input taking into account the disturbance estimation such that the system follows the reference in steady state. Then, this problem is described by the following set of algebraic equations  
% \begin{equation}
%     \label{eq:steadystateproblem}
%     \begin{bmatrix}
%         \boldsymbol{I}-\boldsymbol{A} & -\boldsymbol{B} \\
%         \boldsymbol{C} & \boldsymbol{0}
%     \end{bmatrix}\begin{bmatrix}
%         \mathbf{x}_s \\
%         \mathbf{u}_s
%     \end{bmatrix}= \begin{bmatrix}
%         \boldsymbol{B}_d\hat{\mathbf{d}}_k \\
%         \mathbf{y}_r-\boldsymbol{C}_d\hat{\mathbf{d}}_k
%     \end{bmatrix}
% \end{equation}
% where $\mathbf{x}_s$ and $\mathbf{u}_s$ are the states and input that make the unaugmented system follow the reference $\mathbf{y}_r$ in the presence of a disturbance $\hat{\mathbf{d}}_k$. 
% %Now the deviation variables are the differences between system and steady state variables $ \tilde{\boldsymbol{x}}_k= \boldsymbol{x}_k-\boldsymbol{x}_s, \ \tilde{\boldsymbol{u}}_k= \boldsymbol{u}_k-\boldsymbol{u}_s $, therefore, the goal is minimize the deviation variables. Note that as the system model is linear, the dynamics holds for deviation variables.    
%------------------------------------------
To achieve an accurate positioning of the rotor, i.e., counteracting the effect of the model-plant mismatch and a constant input disturbance $\mathbf{d} \in \mathbb{R}$, we define an offset-free MPC \citep[Section 1.5]{rawlings2017model}.
%
In addition to the solution of OCP \eqref{eq:ocp},
% underpinning an MPC, 
this methodology involves (i) augmenting the system dynamics within the estimation algorithm with the dynamics of $\mathbf{d}$, and (ii) solving an optimization problem -- the steady-state problem or target selector -- that sets references for $\{\mathbf{x},\mathbf{u}\}$ in the OCP \eqref{eq:ocp} based on the estimated states and disturbance, to compensate the disturbance effects while tracking a reference. 
% An overview of the offset-free MPC implemented for this application is presented in Fig. 
The offset-free MPC structure is synthesized in Fig. \ref{fig:offsetfree-diagram}. 
% The differences with respect to a standard MPC implementation are highlighted in blue. 
% The scheme represents nearly how the \codename{Simulink} implementation looks.
\begin{figure}[htpb]
    \centering
    \includegraphics[width=0.37\textwidth]{figures/offsetfreediagram.png}
    \vspace{-1.5ex}
    \caption{Diagram of an offset-free MPC implementation, with differences with respect to traditional MPC in blue, and physical connections to the motor in green.
    % . The differences from traditional MPC are shown in blue. In the plant, the green dotted lines represent the physical connection between the Speedgoat target and the motor.
    }
    \label{fig:offsetfree-diagram}
\end{figure}

Before presenting details on the definition of OCP \eqref{eq:ocp} for this application, let us describe the estimator and steady-state problem. The estimator -- i.e., a Kalman filter -- outputs an estimate
$\hat{\bar{\mathbf{x}}}_k$ 
of an augmented state vector $\bar{\mathbf{x}}_k := \begin{bmatrix}\mathbf{x}_k^\top & \mathbf{d}_k \end{bmatrix}^\top$ whose dynamics are based on a discretized representation of \eqref{eq:dynamics_motor} and the assumption of a constant disturbance. 
%
The steady-state problem corresponds to an optimization problem that considers the reference $\mathbf{y}_r$ and the estimated disturbance $\hat{\mathbf{d}}$ to define set-points $\{\mathbf{x}_s,\mathbf{u}_s\}$ for $\{\mathbf{x},\mathbf{u}\}$ in OCP \eqref{eq:ocp}.
For this example, the solution of such problem can be found analytically as the solution of a system of linear equations. %
% $(I - A)\mathbf{x}_{s} - B\mathbf{u}_{s} = \hat{\mathbf{d}}_k$, $C\mathbf{x}_{s} = \mathbf{y}_{r}$, where $C$ is defined by $\mathbf{y} = C\mathbf{x}$.
%
The reader is referred to 
% Section 1.5 in 
\cite{rawlings2017model} for more details on this formulation.

%------------------------------------------
Instead of minimizing the output error $\mathbf{y} - \mathbf{y}_r$, OCP \eqref{eq:ocp} aims to minimize the deviation between the 
% steady state variables 
set-points
$\{\mathbf{x}_s, \mathbf{u}_s\}$ and the system variables $\{\mathbf{x}, \mathbf{u}\}$. Therefore, the objective \eqref{eq:ocp_objective} is defined by
$V(\cdot) := \lVert\mathbf{x}(t)-\mathbf{x}_s\rVert_{Q}^2+\lVert\mathbf{u}(t)-\mathbf{u}_s\rVert_{R}^2$ and $V_{\mathrm{f}}(\cdot) := \lVert\mathbf{x}(t_f)-\mathbf{x}_s\rVert_{Q_\mathrm{f}}^2$, where $Q$, $R$, $Q_\mathrm{f} \succeq 0$ are weight matrices.
% and $\lVert \cdot \rVert_W$ denotes the $W$-weighted $\ell_2$-norm.
% The weight matrices $Q$, $R$, $Q_\mathrm{f} \succ 0$ are defined within the parameters $\mathbf{p}$. %$\sqrt{a^TAa}$
%
In the boundary constraint \eqref{eq:ocp_init}, $\mathbf{x}_{\mathrm{meas}} = \hat{\mathbf{x}}$,
% is the current estimation of the states $\hat{\mathbf{x}}$,
i.e., excluding the estimated disturbance.
% , given by the Kalman filter using (\ref{eq:augmentedmodel}).
The dynamics \eqref{eq:ocp_dynamics} are described by the ODE \eqref{eq:dynamics_motor}. Since the system does not feature algebraic equations, \eqref{eq:ocp_algebraic} is omitted.
Boundary constraints \eqref{eq:ocp_path} are given by the lower and upper bounds on $v_{\mathrm{motor}}$, such that $-10 \leq \mathbf{u}(t) \leq 10\ \forall t \in [t_0, t_\mathrm{f}]$.

% With all components defined, the offset-free MPC structure is synthesized in the diagram of Fig. \ref{fig:offsetfree-diagram}. In blue color, the differences with respect to normal MPC are highlighted, the scheme represents nearly how looks like the Simulink implementation.
% \begin{figure}[htpb]
%     \centering
%     \includegraphics[width=0.45\textwidth]{figures/offsetfree-diagram.eps}
%     \caption{Offset-free MPC diagram, in blue color the differences from normal MPC are shown. In the plant, the green dotted lines represent the physical connection between Speedgoat target and the motor, it  is not visible in the Simulink implementation.}
%     \label{fig:offsetfree-diagram}
% \end{figure}

%OCP: \textcolor{blue}{No need for writing the full OCP. Just refer to OCP \eqref{eq:ocp} and specify the functions that you use. For instance: in this example the state $\mathbf{x} \in \mathbb{R}^{4}$ is ..., the function $V(\cdot) = $  ...., we have three instances of $v$ in the general equality constraints \eqref{eq:ocp_equality}, which are set terminal constraint to ... and are defined as follows... }

\subsection{Implementation Details}
%----------------------------------------------------------
% In this section, we finalize the implementation of this example. We should go through the workflow procedure outlined in section \ref{sec:workflow}, the \impact block is configured and created, then the blocks connections are made, and the controller is code generated and deployed in the speedgoat target.   

% We start instantiating an MPC object defining the prediction horizon $T=50/300\approx 0.1666 [\mathrm{s}]$. Then the system model is loaded through a yalm file describing the ODE equation. Next, three parameters are defined, current state, xs and us. They collect the state estimates values and the solution of the steady state problem (\ref{eq:steadystateproblem}). Then the cost function, equality constraint for the initial state, and input inequality constraint, they represent the OCP. Note that the dynamics constraints are not explicitly defined, the toolchain takes it automatically from the model described. Now the SQP solver is selected for this example. The transcription method chosen is multiple shooting with $N=50$ discretization points, this defines the sample time for the controller execution $T_s=T/N=1/300=3.333[\mathrm{ms}]$, it's equivalent to a sampling frequency $f_s=300 \ [\mathrm{Hz}]$. Also, 4th-order Runge Kutta is configured to discretize the system dynamics. Finally, the initial values for parameters are set as zeros in our example, and then we call the export function to generate the Simulink block result.
%----------------------------------------------------------
To implement the \impact \codename{Simulink} block that solves OCP \eqref{eq:ocp} we follow the workflow outlined in Section \ref{sec:workflow}. We use $N = 50$,
% discretization points, 
% a sampling time 
$\delta_t = 3.33\ [\mathrm{ms}]$ and 
% a horizon 
$T = 166.66\ [\mathrm{ms}]$.
Model \eqref{eq:dynamics_motor} is defined in a YAML file and loaded to the MPC object. Five parameters are defined, namely, $\mathbf{x}_{\mathrm{meas}}$, $\mathbf{x}_s$, $\mathbf{u}_s$, $Q$, $R$ and $Q_{\mathrm{f}}$. The objective and constraints of OCP \eqref{eq:ocp} are defined as in Section \ref{sec:problem_definition}. 
We use 
% the SQP solver of \codename{CasADi}, 
the active set-based QP solver \codename{QRQP} \citep{qrqp} in \codename{CasADi}
% to solve the QP sub-problems within the SQP method,
and multiple shooting as transcription method with a 4th-order Runge-Kutta integrator for system discretization. The MPC solver is exported to generate the 
% \impact 
\codename{Simulink} block. This block is loaded into a \codename{Simulink} model where the estimator and the steady-state problem are implemented, and where the Speedgoat input/output blocks are loaded. The \impact block is connected to the other blocks as in Fig. \ref{fig:offsetfree-diagram} to close the loop.
%
% Finally, the initial values for parameters are set as zeros in our example, and then we call the export function to generate the Simulink block result.
%
% Now it only remains to connect the feedback loop between blocks in Simulink as shown in Fig. \ref{fig:offsetfree-diagram}. 
% The reference $\mathbf{y}_r =$
% $2 [\mathrm{rad}]$ pulse between $3$ and $5\ [\mathrm{s}]$, and 
% $-3\ [\mathrm{rad}]$ between $8$ and $10\ [\mathrm{s}]$, and zero otherwise. 
The reference is 
% set as 
$\mathbf{y}_r(t) = -3\ [\mathrm{rad}]$, $\forall t \in [1, 3]$, and zero otherwise. 
% We apply 
% a disturbance of 
While $\mathbf{d} = 0.5\ [\mathrm{V}]$ $\forall t \geq 8$.
% The disturbance is $0.5[v]$ step at second $15$.

%In this example the MPC scheme is executed with sample frequency of $f_s=350 \ [\mathrm{Hz}]$, it means, the optimal control problem solution and additional calculations should be executed in a period smaller than $T_s=2.9 \ [\mathrm{ms}]$  

%Code (maybe code is not needed since we are showing some code in the workflow)?? description of the workflow used for this example

%... we use a 4th order Runge-Kutta integrator and the multiple shooting method to discretize and transcribe the OCP into an NLP ...

\subsection{Results}
Once the \codename{Simulink} model has been set, we call the real time execution operation of \codename{Simulink Real-Time}, which code-generates it by using the \codename{Simulink} coder and then executes it in the Speedgoat. The results of the offset-free MPC implementation are compared against a traditional MPC implementation, i.e., without disturbance estimation and execution of the steady-state problem.
%
% The results obtained from the deployment of the controller are presented in this . The traditional MPC and the offset-free version were implemented, the results for the angular position and voltage are compared.   
%
Fig. \ref{fig:1_posvolt2} shows the evolution of the angular position $\theta$ and the voltage input $\mathbf{u}$ with both implementations. Here, the offset-free MPC can follow the reference with zero offset, while the traditional MPC deviates from the reference before (due to the effects of model-plant mismatch) and after the application of the disturbance at 
% $t = 8\ [\mathrm{s}]$. 
$t = 8$.
% The input constraints are satisfied with both MPC implementations.
% , a flat voltage at the maximum can be observed in the inset.
% The voltage at second $10$ in Fig.\ref{fig:1_posvolt} reveals that the input constraints of $\pm 10[\mathrm{v}]$ were active and the \impact block addressed it adequately.

% \begin{figure}[htpb]
%     \centering
%     \includegraphics[width=0.43\textwidth]{figures/1_position.eps}
%     \caption{Angular position $\theta$ of the load in the DC motor}
%     \label{fig:1_step}
% \end{figure}

% \begin{figure}[htpb]
%     \centering
%     \includegraphics[width=0.43\textwidth]{figures/1_posvolt.eps}
%     \caption{Angular position $\theta$, and motor input voltage }
%     \label{fig:1_posvolt}
% \end{figure}
\begin{figure}[htpb]
    \centering
    \includegraphics[width=0.40\textwidth]{figures/posvolt.png}
    \vspace{-1.5ex}
    \caption{
    % Angular position $\theta$, and motor input voltage.
    Evolution of the angular position $\theta$ and the voltage input $\mathbf{u}$ during the application execution.
    }
    \label{fig:1_posvolt2}
\end{figure}

% Fig. \ref{fig:compare} shows a comparison between disturbance estimation, position and input voltage after the disturbance is applied at $t = 8\ [\mathrm{s}]$. Because of the static friction, when the system applies a voltage and the load position does not change, the methodology estimates an opposite force that keeps the load in that position. The force increases until the voltage is enough to move the load, then the contrary effects happens and the disturbance estimation and voltage oscillate. The position $\theta$ ends up one resolution step above or under the reference $\mathbf{y}_r$, it is considered offset-free.    

% \begin{figure}[htpb]
%     \centering
%     \includegraphics[width=0.48\textwidth]{figures/compare.eps}
%     \caption{Comparison between disturbance estimation, angular position and input voltage after disturbance application}
%     \label{fig:compare}
% \end{figure}

% \begin{figure}[htpb]
%     \centering
%     \includegraphics[width=0.48\textwidth]{figures/compare2.eps}
%     \caption{Comparison between disturbance estimation, angular position and input voltage after disturbance application}
%     \label{fig:compare2}
% \end{figure}
\impact eased MPC specification, testing and deployment, allowing to quickly prototype the problem and test several solvers by using the generated artifacts. In addition, the \impact \codename{Simulink} block could be easily integrated within larger models for simulation or deployment on real hardware.
%
% The use of \impact eases the specification, testing and deployment of (N)MPC. The practitioner is able to quickly prototype the problem by using the generated artifacts. In particular, when using the \codename{Simulink} block, the MPC solvers generated with \impact are easily integrated within larger models for simulation or directly for deployment on real hardware.
%
Although this example features a linear system, \impact allows the definition of systems represented by both linear or nonlinear equations.
%
Other tested applications with \impact include the deployment of NMPC in a Speedgoat target and a Beckhoff TwinCAT Embedded PC for a point-to-point motion application with a parallel SCARA robot. In the future, \impact will be tested in a DSpace Controller board, and extended to be used in FPGAs and GPUs.

\section{Conclusions}
\label{sec:conclusion}

We propose \backronym, a method to address the challenging task of dense correspondences across images of an object or object category captured in-the-wild. \backronym utilizes noisy and sparse pseudo-correspondences in pre-trained ViT feature space to build an accurate and dense consistent mapping from image to a canonical space. Extensive qualitative and quantitative experiments show that \backronym works in low-shot settings and can deal with extreme variations in pose, background, occlusion, and object deformations.
We also propose a new metric $\kcycle$ to evaluate the consistency of keypoint predictions over a set of images beyond pair-wise consistency.
%A limitation of our work is its dependence on the quality of features learned by the pre-trained ViTs. A way to mitigate this limitation could be to manually annotate seed matches. Despite this limitation, 
\backronym can obtain consistent dense mappings competitive with supervised counterparts with just a few images. In future work, we will explore applications of \backronym in other few-shot downstream tasks such as reconstruction, pose estimation and tracking.
 

% \begin{ack}
% Place acknowledgments here.
% \end{ack}

{\small
\bibliography{impact} % bib file to produce the bibliography
} % with bibtex (preferred)
                                                   
%\begin{thebibliography}{xx}  % you can also add the bibliography by hand

%\bibitem[Able(1956)]{Abl:56}
%B.C. Able.
%\newblock Nucleic acid content of microscope.
%\newblock \emph{Nature}, 135:\penalty0 7--9, 1956.

%\bibitem[Able et~al.(1954)Able, Tagg, and Rush]{AbTaRu:54}
%B.C. Able, R.A. Tagg, and M.~Rush.
%\newblock Enzyme-catalyzed cellular transanimations.
%\newblock In A.F. Round, editor, \emph{Advances in Enzymology}, volume~2, pages
%  125--247. Academic Press, New York, 3rd edition, 1954.

%\bibitem[Keohane(1958)]{Keo:58}
%R.~Keohane.
%\newblock \emph{Power and Interdependence: World Politics in Transitions}.
%\newblock Little, Brown \& Co., Boston, 1958.

%\bibitem[Powers(1985)]{Pow:85}
%T.~Powers.
%\newblock Is there a way out?
%\newblock \emph{Harpers}, pages 35--47, June 1985.

%\bibitem[Soukhanov(1992)]{Heritage:92}
%A.~H. Soukhanov, editor.
%\newblock \emph{{The American Heritage. Dictionary of the American Language}}.
%\newblock Houghton Mifflin Company, 1992.

%\end{thebibliography}

% \appendix
% \section{A summary of Latin grammar}    % Each appendix must have a short title.
% \section{Some Latin vocabulary}              % Sections and subsections are supported  
%                                                                          % in the appendices.
\end{document}
