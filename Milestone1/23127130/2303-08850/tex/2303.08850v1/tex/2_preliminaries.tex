\section{Optimal Control Problem} \label{sec:preliminaries}
For a time horizon $t \in [t_0,\ t_{\mathrm{f}}]$, system state vector $\mathbf{x}(t) \in \mathbb{R}^{n_{\mathbf{x}}}$, input vector $\mathbf{u}(t) \in \mathbb{R}^{n_{\mathbf{u}}}$, algebraic  state vector $\mathbf{z}(t) \in \mathbb{R}^{n_{\mathbf{z}}}$, and parameter vector $\mathbf{p} \in \mathbb{R}^{n_{\mathbf{p}}}$, in \impact we consider general OCP formulations  of the canonical form
%
\begin{mini!}[2]
{{{\mathbf{x}, \mathbf{u}, \mathbf{z}}}}
{\int_{t_0}^{t_\mathrm{f}}V(\mathbf{x}(t), \mathbf{u}(t), \mathbf{p}, t)\mathrm{d}t + V_{f}(\mathbf{x}(t_\mathrm{f}),\mathbf{p}) \label{eq:ocp_objective}}{\label{eq:ocp}}{}
\addConstraint{B(\mathbf{x}(t_0),\mathbf{x}(t_f),\mathbf{p})}{ \leq 0 \label{eq:ocp_init}}{}
\addConstraint{\mathbf{\dot{x}}(t) }{ = \xi(\mathbf{x}(t),\mathbf{u}(t), \mathbf{z}(t),\mathbf{p}),\hspace{1.0ex} t \in [t_0, t_\mathrm{f}] \label{eq:ocp_dynamics}}{}
\addConstraint{\Gamma(\mathbf{x}(t),\mathbf{u}(t), \mathbf{z}(t),\mathbf{p}) }{ = 0,\hspace{3.4ex} t \in [t_0, t_\mathrm{f}] \label{eq:ocp_algebraic}}{}
\addConstraint{h(\mathbf{x}(t), \mathbf{u}(t), \mathbf{p})}{\leq 0,\hspace{8.2ex} t\in [t_0, t_\mathrm{f}], \label{eq:ocp_path}}{}
%\addConstraint{r_\mathrm{f}(\mathbf{x}(t_\mathrm{f}), \mathbf{p})}{\leq 0 \label{eq:ocp_terminal_constraints}}{},
\end{mini!}
where $V(\cdot)$ and $V_\mathrm{f}(\cdot)$ in \eqref{eq:ocp_objective} are smooth nonlinear functionals that define Lagrange and Mayer terms, respectively, $\mathbf{p}$ is a parameter vector that typically contains a vector of state measurements or estimations $\mathbf{x}_{\mathrm{meas}} \in \mathbb{R}^{n_{\mathbf{x}}}$, \eqref{eq:ocp_init} defines a boundary constraint that typically takes the form of $\mathbf{x}(t_0)=\mathbf{x}_{\mathrm{meas}}$, \eqref{eq:ocp_dynamics} and \eqref{eq:ocp_algebraic} define a system of differential-algebraic equations (DAE) representing the system model, while \eqref{eq:ocp_path} represent general path constraints. Note that the inclusion of the algebraic state vector $\mathbf{z}$ and the algebraic equation \eqref{eq:ocp_algebraic} is not required and can be omitted for systems represented solely by ordinary differential equations (ODE). The OCP specified by the user in \impact (see Section \ref{sec:workflow}) is automatically transformed into the canonical form \eqref{eq:ocp}.

%For NLP and QP
%direct methods (first discretize, then optimize) 
%solvers interfaced with \impact via \codename{CasADi},
% -- e.g., \codename{IPOPT}, the \codename{FSLP solver}, or the SQP method of \codename{CasADi} --, 
%the OCP is transcribed using \codename{Rockit}'s multiple-shooting, single-shooting, direct collocation, or B-spline methods into
Direct transcription of \eqref{eq:ocp} leads to a finite-dimensional optimization problem
% , i.e., an NLP, 
of the form
\begin{equation}
\label{eq:nlp}
\begin{split}
\min_{\mathbf{w}} \: f(\mathbf{w})\quad\mathrm{s.t.} \:\: g(\mathbf{w})=0,\:\: h(\mathbf{w})\leq 0,
\end{split}
\end{equation}
where 
% $\mathbf{w} := \begin{bmatrix} x_0^\top & u_0^\top & \hdots & u_{N-1}^\top & x_N^\top\end{bmatrix}^\top$, 
$\mathbf{w} \in \mathbb{R}^{n_{\mathbf{w}}}$ is the vector of decision variables, 
and $N := T/\delta_t \in \mathbb{N}$ is the number of discretization points within the prediction horizon $T := (t_{\mathrm{f}} - t_0) \in \mathbb{R}_{>0}$ with a sampling time $\delta_t \in \mathbb{R}_{>0}$. The first- and second-order derivatives arising in the optimality conditions of  \eqref{eq:nlp} are typically large and sparse.
%The solvers get access to code-generated sparse first- or second-order derivatives by \codename{CasADi}'s algorithmic differentiation (AD).

%Conversely, for external tools interfaced with \impact via plugins, 
% such as \codename{GRAMPC} -- which implements an indirect method (first optimize, then discretize) --, \codename{Acados}, and \codename{FATROP}, 
%the OCP \eqref{eq:ocp} is transformed into the structure expected by the solver selected by the user.