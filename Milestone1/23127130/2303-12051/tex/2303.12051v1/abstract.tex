\begin{abstract} 
% Permutation synchronization is an important task in different fields. To recover latent permutations from their noisy and incomplete pairwise measurements, spectral methods have gained increasing popularity in recent years for its simplicity and computational efficiency.
Permutation synchronization is an important problem in computer science that constitutes the key step of many computer vision tasks. The goal is to recover $n$ latent permutations from their noisy and incomplete pairwise measurements. In recent years, spectral methods have gained increasing popularity thanks to their simplicity and computational efficiency. 
%. Spectral methods are simple and powerful approaches  that have been widely used in permutation synchronization and have obtained growing popularity in recent years. 
%They typically  utilize  the leading eigenspace of the data matrix and its block submatrices $U_1,U_2,\ldots, U_n$ and estimate latent permutations by projections of $U_j U_1^\top$ for all $j\geq 2$. However, the use of the block $U_1$ across different estimations amplify its error
% It utilizes the leading eigenspace of the data matrix and its block submatrices $U_1,U_2,\ldots, U_n$. In this paper, we propose a novel and statistically optimal spectral method for permutation synchronization. Different from existing spectral methods that are based on $\{U_jU_1^\top\}_{j\geq 2}$, we carefully construct an anchor matrix $M$ by aggregating useful information from the data and estimate latent permutations through $\{U_jM^\top\}_{j\geq 1}$. This   overcomes a crucial limitation of existing methods caused by the repetitive use of $U_1$ and leads to a significantly improved numerical performance.
Spectral methods utilize the leading eigenspace $U$ of the data matrix and its block submatrices $U_1,U_2,\ldots, U_n$ to recover the permutations. In this paper, we propose a novel and statistically optimal spectral algorithm. Unlike the existing methods which use $\{U_jU_1^\top\}_{j\geq 2}$, ours constructs an anchor matrix $M$ by aggregating useful information from all the block submatrices and estimates the latent permutations through $\{U_jM^\top\}_{j\geq 1}$. This modification overcomes a crucial limitation of the existing methods caused by the repetitive use of $U_1$ and leads to an improved numerical performance.
% that overcomes a crucial limitation of existing ones in literature and outperforms them numerically. 
%Our method utilizes  the leading eigenspace of the data matrix and its block submatrices $U_1,U_2,\ldots, U_n$. While existing methods estimate latent permutations by projections of $U_j U_1^\top$
%
To establish the optimality of the proposed method, we carry out a fine-grained spectral analysis and obtain a sharp exponential error bound that matches the minimax rate.
\end{abstract}

%In this work, we study the phase synchronization problem where the latent objects belong to the permutation group and the pairwise measurements among them are noisy and missing at random. 
%This stylized problem, known as the permutation group synchronization problem, has wide-ranging applications in computer vision. 
% Among the various algorithms proposed in the literature, the spectral algorithm of Pachauri et al. (2013) is well known for its speed and good empirical performance. Despite these findings, the theoretical guarantee of the spectral algorithm has not been fully characterized. In this work, we propose a new spectral algorithm that overcomes a subtle yet crucial limitation of the vanilla spectral algorithm of Pachauri et al.. We prove that the proposed spectral algorithm has an error bound of $\ebr{-(1-o(1)) \frac{\text{SNR}}{2}}$. Thus, it achieves the asymptotic minimax error rate of $\ebr{-\frac{\text{SNR}}{2}}$. Our analysis produces a $\ell_{\infty}$-type bound on the blockwise deviation of the leading eigenspaces and a novel tail bound on this deviation.


%\keywords{Spectral Method, Permutation Synchronization, Leave-one-out}