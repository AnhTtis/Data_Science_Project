%{\color{red} to do: 1. replace $O(d)$ by $\mathcal{O}(d)$ or $\mathcal{O}_d$, 2. there exist constants $C,C'$ ``$>0$'', 3. use $\fnorm{\cdot}$, 4. use $\define$ and $\definei$, 5. use $\p$ instead of Pr}
%\duc{Small bugs in the proof}.
%
%In helper Lemma \ref{lem:event-E-prime_new}, to show $\opnorm{X - X^{(j)}} \leq 2\opnorm{X_j} \leq \frac{\lambda_d(X)- \lambda_{d+1}(X)}{2}$. In proof of Lemma \ref{lem:simplification} to derive upper bound on $\opnorm{B_j}$. In these cases, we have absorbed the $p\sqrt n$ term into the ${C_0}\sqrt{np}$ term. However, to be explicit, sufficient condition is $C_0 > 7$. In Lemma \ref{ref:event-E-prime_new}, if we assume $C_0 > 2$
%$$ \opnorm{X_j} \leq p\sqrt n + C_0(1+\sigma \sqrt d)\sqrt{np} \leq \frac{1}{2} C_0\sqrt{np} + C_0(1+\sigma \sqrt d)\sqrt{np}  \leq \frac{3}{2} C_0(1+\sigma \sqrt d)\sqrt{np} \leq \frac{3np}{16} \leq \frac{\lambda_d-\lambda_{d+1}}{4} $$
%In Lemma \ref{lem:simplification}, if we assume $C_0 > 3$,
%\begin{align*}
%  \opnorm{B_j} &\leq  \opnorm{U_j }   \opnorm{H\Lambda - \Lambda H} \opnorm{\Lambda^{-1}}+ \opnorm{X_j}\opnorm{UU^\top - U^{(j)}U^{(j)\top}} \opnorm{\Lambda^{-1}}  \\
%  &\leq  {\frac{4}{3}\br{\opnorm{U_jH-U^*_j}+\frac{1}{\sqrt{n}}}} \br{2C_0\br{1+\sigma\sqrt{d}}\sqrt{np}} \frac{8}{7np} \\
%  &\quad +  \br{p\sqrt{n} + C_0\br{1+\sigma\sqrt{d}}\sqrt{np}} \br{6\br{\opnorm{U_jH -U^*_j}+\frac{1}{\sqrt{n}}}} \frac{8}{7np} \\
%  &\leq \br{\opnorm{U_jH-U^*_j}+\frac{1}{\sqrt{n}}} \br{\frac{48}{7\sqrt n} + \frac{8\cdot 26}{21}\frac{C_0}{\sqrt{np}}\br{1+\sigma \sqrt{d}}}\\
%  &\leq \br{\opnorm{U_jH-U^*_j}+\frac{1}{\sqrt{n}}} 11C_0  \br{\frac{1+\sigma \sqrt{d}}{\sqrt{np}}} ,
%\end{align*}
%where last inequality comes from $\frac{48}{7\sqrt n} + \frac{8\cdot 26}{3\sqrt np}C_0 \leq \frac{7}{\sqrt n} + \frac{10C_0}{\sqrt{np}}\leq \frac{11C_0}{\sqrt{np}}$ if $\frac{7}{\sqrt n} \leq \frac{C_0}{\sqrt{np}}$ opr $C_0 \geq 7$.
%
%
%In the proof of the Theorem \ref{thm:main_new}, to derive the upper bound on,
%\begin{align*}
%  &\E\indic{ \fnorm{U_j \hatH^\top - U_j^*\hat P^\top} \geq \fnorm{U_j \hatH^\top - \frac{R \hat P^\top}{\sqrt n} }   }  \indic{\mathf} \\
%   &\leq  \ebr{- \br{1- 3\br{\br{\frac{\sigma^2}{np}}^\frac{1}{4} + \br{\frac{1}{\log n}}^\frac{1}{4}} - C_2\br{\frac{\sqrt{\log n}+ \sigma\sqrt{d}}{\sqrt{np}}}}_+^2 \br{1+C_3\sqrt{\frac{\log n}{np}}}^{-1} \frac{np}{2\sigma^2}} + 3n^{-10} \\
%   &\quad + 4\ebr{-\frac{np}{\sigma^2}} + 2d \ebr{-\frac{3 \log n}{256} \frac{C_3 (\log n)^{1.25}}{\sqrt{d}(C''\vee 1)\br{ \frac{1}{\sqrt{\log n}} \br{1+ \sqrt{\frac{\log n }{np}} + \frac{\sigma \sqrt{d}}{\sqrt{np}}} + \frac{\sigma}{\sqrt{np}}}}} \\
%   &\quad + 6d \ebr{ -\frac{3c  \log n}{16^3} \br{\frac{\sqrt{C_3} (\log n)^{0.25}}{\sqrt{d}(C''\wedge1) \br{ \frac{1}{\sqrt{\log n}} \br{1+ \sqrt{\frac{\log n }{np}} + \frac{\sigma \sqrt{d}}{\sqrt{np}}} + \frac{\sigma}{\sqrt{np}}}}} ^2}\\
%  &\leq   \ebr{- \br{1- 3\br{\br{\frac{\sigma^2}{np}}^\frac{1}{4} + \br{\frac{1}{\log n}}^\frac{1}{4}} - C_2\br{\frac{\sqrt{\log n}+ \sigma\sqrt{d}}{\sqrt{np}}}}_+^2 \br{1+C_3\sqrt{\frac{\log n}{np}}}^{-1} \frac{np}{2\sigma^2}} + 3n^{-10} \\
%  &\quad +  4\ebr{-\frac{np}{\sigma^2}} + 9dn^{-10}.
%  \end{align*}
%We have combined the two extra exponent terms into $9dn^{-10}$ but didn't say when this happens. Both happen when $\frac{\log n}{np}, \frac{\sigma}{\sqrt np}$ are smaller than some constant.
%
%\duc{End of bugs}



\subsection{Proof of Proposition \ref{prop:hat_H_new}}
We first state and prove a deterministic version of Proposition \ref{prop:hat_H_new}. The proof of Proposition \ref{prop:hat_H_new} follows from a simple probabilistic argument.

\begin{lemma}\label{lem:hat_H}
Assume that (\ref{eqn:5}) holds. If $\frac{np}{(\sqrt{d} + \sigma d)^2} \geq 32 C_0^2$, we have
\begin{align}\label{eqn:11}
\min_{P\in\Pi_d}\fnorm{\hatH - PH^\top} \leq  \frac{{4C_0\br{\sqrt{d} + \sigma d}}}{\sqrt{np}}.
\end{align}
\end{lemma}
\begin{proof}
Note that
\begin{align*}
 \opnorm{U-U^*H^\top} &= \opnorm{UU^\top U - U^*U^{*\top}U} \leq  \opnorm{UU^\top  - U^*U^{*\top}}\opnorm{U}\leq \opnorm{UU^\top  - U^*U^{*\top}}.
\end{align*}
Since (\ref{eqn:5}) holds, we have
\begin{align*}
\fnorm{U - U^*H^\top} &\leq \sqrt{d}  \opnorm{U-U^*H^\top} \leq   \sqrt{d}\frac{8C_0\br{1+\sigma\sqrt{d}}}{7\sqrt{np}}.
\end{align*}


% and (\ref{eqn:6}), we have
%\begin{align*}
%\fnorm{U - U^*H^{-1}}&\leq \sqrt{d} \opnorm{U - U^*H^{-1}} \leq \sqrt{d}\opnorm{U H - U^*}\opnorm{H^{-1}} \leq 2\sqrt{d} \frac{1+\sigma \sqrt{d}}{\sqrt{np}}.
%\end{align*}
Note that 
%according to Lemma \ref{lem:population}, 
$\sqrt{n}U^*$ has only $k$ unique rows $e_1^\top,\ldots,e_d^\top$ and each is of size $n$. Then $\sqrt{n}U^* H^\top$ also has $k$ unique rows $e_1^\top H^\top,\ldots,e_d^\top H^\top$ and each is of size $n$. From (\ref{eqn:5}), there exists a matrix $O\in \mathcal{O}_d$ such that $\opnorm{H-O}\leq \frac{8C_0\br{1+\sigma\sqrt{d}}}{7\sqrt{np}} \leq \frac{1}{7}$. Then
\begin{align*}
 \min_{a,b\in[d]} \norm{(e_a - e_b)^\top H^\top} &\geq \min_{a,b\in[d]} \norm{(e_a - e_b)^\top \br{O^\top+  (H-O)^\top}}\\
&\geq   \min_{a,b\in[d]} \norm{(e_a - e_b)^\top O^\top } - \opnorm{H-O}\\
&\geq \sqrt{2} -\frac{1}{7}\\
&\geq  \frac{6}{5}.
\end{align*}
That is, $\Delta$, defined as the minimum distance among the unique rows of $U^*H^\top$, is at least $\frac{6}{5\sqrt{n}}$.



Let $\hat z\in[d]^{nd}$ be the minimizer of (\ref{eqn:alg1}) with $\{\hat \mu_1,\ldots,\hat \mu_d\}$. Denote $\hat \Theta \define (\hat \mu_{\hat z_1}^\top, \ldots, \hat \mu_{\hat z_{nd}}^\top)^\top\in\mathr^{nd\times d}$. According to (\ref{eqn:alg1}), we have $\fnorm{\hat \Theta - U}\leq \fnorm{U^*H^\top - U}$. Hence,
\begin{align*}
\fnorm{\hat \Theta - U^*H^\top} &\leq \fnorm{\hat \Theta -U }  + \fnorm{U^*H^\top - U}\leq 2 \fnorm{U - U^*H^\top }\leq   \sqrt{d}\frac{16C_0\br{1+\sigma\sqrt{d}}}{7\sqrt{np}}.
\end{align*}
Define $z^*\in[d]^{nd}$ such that the $i$th row of $U^* H^\top$ is equal to $e_{z^*_i}^\top H^T/\sqrt{n}$ for each $i\in[n]$.
Define the set $S$ as
\begin{align*}
S\define \cbr{i\in[nd]: \norm{\hat \mu_{\hat z_i} - e_{z^*_i}^\top H^T/\sqrt{n}} > \frac{\Delta}{2}}.
\end{align*}
Then we have
\begin{align*}
\abs{S} \leq \frac{\fnorm{\hat \Theta - U^*H^\top}^2}{(\Delta/2)^2}\leq \frac{\br{4C_0\br{\sqrt{d} + \sigma d}}^2}{p}
\end{align*}
Under the assumption $\frac{np}{(\sqrt{d} + \sigma d)^2} \geq 32 C_0^2,$
%\begin{align*}
%\frac{np}{(\sqrt{d} + \sigma d)^2} \geq 32 C_0^2,
%\end{align*}
we have $\abs{S}\leq n/2$. Then
by the same argument as in the proof of Proposition 3.1 of   \cite{zhang2022leave},  there exists a bijection $\phi:[d]\rightarrow[d]$ such that $\hat z_i = \phi(z^*_i)$ for all $i\notin S$. Hence, for each $a\in[d]$, we have
\begin{align*}
\norm{\hat \mu_{\phi(a)} - e_a^\top H^\top/\sqrt{n} }^2 & = \frac{\sum_{i\in [nd]:\hat z_i = \phi(a),z^*_i =a} \norm{\hat \mu_{\hat z_i} - e_{z^*_i}^\top H^\top }^2}{\abs{i\in[nd]:\hat z_i = \phi(a),z^*_i =a}} \leq   \frac{\sum_{i\in[nd]:\hat z_i = \phi(a)} \norm{\hat \mu_{\hat z_i} - e_{z^*_i}^\top H^\top }^2}{n-\abs{S}}\\
&\leq    \frac{\sum_{i\in[nd]:\hat z_i = \phi(a)} \norm{\hat \mu_{\hat z_i} - e_{z^*_i}^\top H^\top }^2}{n/2}.
\end{align*}
Hence,
\begin{align*}
\sum_{a\in[d]}\norm{\hat \mu_{\phi(a)} - e_a^\top H^\top/\sqrt{n}}^2 \leq \frac{\sum_{i\in[nd]} \norm{\hat \mu_{\hat z_i} - e_{z^*_i}^\top H^\top }^2 }{n/2} = \frac{\fnorm{\hat \Theta -U^* H^\top}^2}{n/2}  \leq \frac{\br{4C_0\br{\sqrt{d} + \sigma d}}^2}{n^2p}.
\end{align*}
That is, there exists a permutation matrix $P$ such that
\begin{align*}
\fnorm{\hatH - PH^\top}^2 \leq \frac{\br{4C_0\br{\sqrt{d} + \sigma d}}^2}{np},
\end{align*}
since rows of $M$  are $\sqrt{n}\hat \mu_1,\ldots,\sqrt{n}\hat \mu_d $.
\end{proof}


\begin{proof}[\textbf{Proof of Proposition \ref{prop:hat_H_new}}]
According to Lemma \ref{lem:event-E-prime_new}, there exist constants $C,C_0>0$ such that if $\frac{np}{\log n}>C$ and $\frac{np}{(1+\sigma \sqrt{d})^2 }\geq 64C_0^2$, we have   (\ref{eqn:5}) hold with probability at least $1-n^{-10}$. The proof is complete by Lemma \ref{lem:hat_H}.
\end{proof}


%Throughout the proof and without loss of generality, we let $Z^*_j = I_d$ for all $j\in[n]$. Hence, (\ref{eqn:X-def}) becomes
%\begin{align*}
%X = (A \otimes I_d) + \sigma (A \otimes J_d) \circ W.
%\end{align*}
%For each $j\in[n]$, define $X_j\define (X_{j1},\ldots, X_{jn})\in\mathr^{d\times nd}$. Then we have
%\begin{align}\label{eqn:1}
%X_j = (A_j \otimes I_d) + \sigma (A_j \otimes J_d) \circ W_j.
%\end{align}
%As a result, $X =(X_1^\top,\ldots, X_n^\top)^\top$. We introduce a few leave-one-out matrices that will be used in the proofs. For each $j\in[n]$, define $X^{(j)}\in\mathr^{nd\times nd}$ such that
%\begin{align*}
%X^{(j)}_{ik} \define\quad \begin{cases}
%X_{ik}, \forall i,k\neq j,\\
%0_{d\times d},\quad  \text{o.w..}
%\end{cases}
%\end{align*}
%In addition, let $U^{(j)}\in\mathr^{nd\times d}$ be the matrix including the leading $d$ eigenvectors of $X^{(j)}$. As a consequence, $X^{(j)}, U^{(j)}$ are independent of $\{A_{jk}\}_{k\neq j}$ and $\{W_{jk}\}_{k\neq j}$. The  following decomposition of $UH-U^*$ is the starting point to establishing our main theorems.
%
%\paragraph{Decomposition of $UH-U^*$.} 
%Define $\Lambda\in\mathr^{d\times d}$ to be the diagonal matrix of the leading eigenvalues of $X$. That is,
%\begin{align}\label{eqn:Lambda_def}
%\Lambda_{ii}\define \lambda_i(X) \text{ and }\Lambda_{ik}\define 0, \forall 1\leq i\neq k\leq d.
%\end{align}
%% $\Lambda_{ii}\define \lambda_i(X)$ and $\Lambda_{ik}\define 0$ for all $1\leq i\neq k\leq d$.
%Note that
%\begin{align*}
%UH\Lambda - XU^*  & = U(H\Lambda - \Lambda H) + U\Lambda H -XU^*\\
%& = U(H\Lambda - \Lambda H) + XU H -XU^*\\
%&= U(H\Lambda - \Lambda H) + X(U H - U^*).
%\end{align*}
%Multiplying both sides by $\Lambda^{-1}$ and rearranging the terms, we have
%\begin{align}\label{eqn:decomposition1}
%UH - U^* = U(H\Lambda - \Lambda H) \Lambda^{-1}  + X(U H - U^*) \Lambda^{-1} + XU^*\Lambda^{-1} - U^* .
%\end{align}
%The above display involves $ X(U H - U^*) $ where $X$ and $UH-U^*$ are dependent on each other. To decouple the dependence, we approximate $UH-U^*$ by its leave-one-out counterparts.
%Consider any $j\in[n]$. Define
%\begin{align*}
%H^{(j)}\define U^{(j)\top}U^*\in\mathr^{d\times d}.
%\end{align*}
%Then
%\begin{align*}
%UH-U^* &= UH - U^{(j)}H^{(j)} + U^{(j)}H^{(j)} - U^* \\
%& = (UU^\top - U^{(j)}U^{(j)\top})U^*+ U^{(j)}H^{(j)} - U^* .
%\end{align*}
%After plugging it into the right-hand side of (\ref{eqn:decomposition1}), we have
%\begin{align*}
%UH - U^* & = U(H\Lambda - \Lambda H) \Lambda^{-1}  + X(UU^\top - U^{(j)}U^{(j)\top})U^* \Lambda^{-1} \\
%&\quad  +  X(U^{(j)}H^{(j)} - U^*) \Lambda^{-1}  + XU^*\Lambda^{-1} - U^*.
%\end{align*}
%Then, the $j$th block matrix of $UH - U^*$ satisfies
%\begin{align}
%U_jH - U^*_j & = \underbrace{U_j(H\Lambda - \Lambda H) \Lambda^{-1}  + X_j(UU^\top - U^{(j)}U^{(j)\top})U^* \Lambda^{-1}}_{\definei B_j} \nonumber\\
%&\quad  +  X_j(U^{(j)}H^{(j)} - U^*) \Lambda^{-1}  + X_jU^*\Lambda^{-1} - U^*_j.\label{eqn:decomposition2}
%\end{align}
%The last two terms in (\ref{eqn:decomposition2}) can be further decomposed. 
%Using (\ref{eqn:1}), we have
%\begin{align*}
%X_j(U^{(j)}H^{(j)} - U^*) \Lambda^{-1}&= \br{(A_j \otimes I_d) + \sigma (A_j \otimes J_d) \circ W_j}(U^{(j)}H^{(j)} - U^*) \Lambda^{-1}\\
%&=  \underbrace{\sum_{k\neq j} A_{jk} (U^{(j)}_kH^{(j)} - U^*_k)\Lambda^{-1}}_{\definei F_{j1}}  +   \underbrace{\sigma \sum_{k\neq j} A_{jk} W_{jk}(U^{(j)}_kH^{(j)} - U^*_k)\Lambda^{-1}}_{\definei F_{j2}}  ,
%\end{align*}
%and
%\begin{align*}
%X_jU^*\Lambda^{-1} - U^*_j &= \br{(A_j \otimes I_d) + \sigma (A_j \otimes J_d) \circ W_j}U^*\Lambda^{-1} - U^*_j \\
%&= \left(\sum_{k\neq j} A_{jk} U^*_k\right)\Lambda^{-1}  - U_j^*+ \sigma \,\sum_{k\neq j} A_{jk} W_{jk} U^*_k\Lambda^{-1} \\
%&= \underbrace{\frac{1}{\sqrt{n}}\br{\left(\sum_{k\neq j} A_{jk} \right)\Lambda^{-1}  - I_d}}_{\definei G_{j1}} + \underbrace{ \frac{\sigma}{\sqrt{n}} \,\sum_{k\neq j} A_{jk} W_{jk} \Lambda^{-1} }_{\definei G_{j2}} ,
%\end{align*}
%where the last equation is due to Lemma \ref{lem:population}.
%As a result,
%\begin{align}\label{eqn:decomposition3}
%U_jH - U^*_j & = 
%%U_j(H\Lambda - \Lambda H) \Lambda^{-1}  + X_j(UU^\top - U^{(j)}U^{(j)\top})U^* \Lambda^{-1} 
%B_j  + F_{j1} + F_{j2} + G_{j1} + G_{j2}.
%\end{align}
%Note we have a mutual independence among $\{A_{jk}\}_{k\neq i}$, $\{W_{jk}\}_{k\neq i}$, and $U^{(j)}H^{(j)} -U^*$ in the definitions of $F_{j1}$ and $F_{j2}$, which is crucial to obtaining sharp bounds for them.
%The decomposition (\ref{eqn:decomposition3}) holds for all $j\in[n]$.


\subsection{Proof of Theorem \ref{thm:l_infty_new}}

We first give a deterministic upper bound for $\opnormt{U_jH - U^*_j}$, using the decomposition (\ref{eqn:decomposition3}).
\begin{lemma}\label{lem:simplification}
Assume (\ref{eqn:2})-(\ref{eqn:12}) hold. Under the assumption $\frac{np}{(1+\sigma \sqrt{d})^2} \geq 22^2 C_0^2$, for each $j\in[n]$, we have
\begin{align}\label{eqn:13}
\opnorm{U_jH - U^*_j } &\leq  \frac{22C_0}{\sqrt{n}}\br{\frac{1+\sigma \sqrt{d}}{\sqrt{np}}}+2\opnorm{F_{j1}} + 2\opnorm{F_{j2}} + 2\opnorm{G_{j1}} + 2\opnorm{G_{j2}},
\end{align}
and
\begin{align}\label{eqn:Bj-bound}
\opnorm{B_j} &\leq  22C_0 \br{\frac{1}{\sqrt{n}}  +\opnorm{F_{j1}} + \opnorm{F_{j2}} + \opnorm{G_{j1}} + \opnorm{G_{j2}}} \br{\frac{1+\sigma \sqrt{d}}{\sqrt{np}}}.
\end{align}
\end{lemma}
\begin{proof}
Consider any $j\in[n]$. We have
\begin{align*}
\opnorm{B_j} &\leq  \opnorm{U_j }   \opnorm{H\Lambda - \Lambda H} \opnorm{\Lambda^{-1}}+ \opnorm{X_j}\opnorm{UU^\top - U^{(j)}U^{(j)\top}} \opnorm{\Lambda^{-1}}  \\
&\leq  {\frac{4}{3}\br{\opnorm{U_jH-U^*_j}+\frac{1}{\sqrt{n}}}} \br{2C_0\br{1+\sigma\sqrt{d}}\sqrt{np}} \frac{8}{7np} \\
&\quad +  \br{p\sqrt{n} + C_0\br{1+\sigma\sqrt{d}}\sqrt{np}} \br{6\br{\opnorm{U_jH -U^*_j}+\frac{1}{\sqrt{n}}}} \frac{8}{7np} \\
&\leq \frac{8}{7np}\br{\opnorm{U_jH-U^*_j}+\frac{1}{\sqrt{n}}} \br{6p\sqrt{n} + \frac{26}{3}C_0\br{1+\sigma \sqrt{d}}\sqrt{np}}\\
&\leq 11C_0 \br{\opnorm{U_jH-U^*_j}+\frac{1}{\sqrt{n}}} \br{\frac{1+\sigma \sqrt{d}}{\sqrt{np}}} ,
\end{align*}
where the second inequality is by (\ref{eqn:2})-(\ref{eqn:12}) and the last inequality is by $C_0 > 7$. From (\ref{eqn:decomposition3}), we also have
\begin{align*}
\opnorm{U_jH - U^*_j } &\leq \opnorm{B_j}+\opnorm{F_{j1}} + \opnorm{F_{j2}} + \opnorm{G_{j1}} + \opnorm{G_{j2}}.
\end{align*}
Plug the upper bound (\ref{eqn:Bj-bound}) of $\opnorm{B_j}$ into the above display. After rearrangement, we have
\begin{align*}
\br{1- 11C_0 \br{\frac{1+\sigma \sqrt{d}}{\sqrt{np}}} }\opnorm{U_jH - U^*_j }&\leq  \frac{11C_0}{\sqrt{n}}\br{\frac{1+\sigma \sqrt{d}}{\sqrt{np}}}+\opnorm{F_{j1}} + \opnorm{F_{j2}} + \opnorm{G_{j1}} + \opnorm{G_{j2}}.
\end{align*}
Under the assumption $\frac{np}{(1+\sigma \sqrt{d})^2} \geq 22^2 C_0^2$, we have
\begin{align*}
\opnorm{U_jH - U^*_j } &\leq \frac{22C_0}{\sqrt{n}}\br{\frac{1+\sigma \sqrt{d}}{\sqrt{np}}}+2\opnorm{F_{j1}} + 2\opnorm{F_{j2}} + 2\opnorm{G_{j1}} + 2\opnorm{G_{j2}}.
\end{align*}
Plugging it into the upper bound of $\opnorm{B_j}$, we have
\begin{align*}
\opnorm{B_j} &\leq 11C_0 \br{\br{\frac{22C_0}{\sqrt n}\br{\frac{1+\sigma \sqrt{d}}{\sqrt{np}}}+2\opnorm{F_{j1}} + 2\opnorm{F_{j2}} + 2\opnorm{G_{j1}} + 2\opnorm{G_{j2}}}+\frac{1}{\sqrt{n}}} \br{\frac{1+\sigma \sqrt{d}}{\sqrt{np}}} \\
&\leq  22C_0 \br{\frac{1}{\sqrt{n}}  +\opnorm{F_{j1}} + \opnorm{F_{j2}} + \opnorm{G_{j1}} + \opnorm{G_{j2}}} \br{\frac{1+\sigma \sqrt{d}}{\sqrt{np}}}.
\end{align*}
\end{proof}

As a preview, the terms $\opnorm{F_{j1}}$, $\opnorm{F_{j2}}$, $\opnorm{G_{j1}}$ and $\opnorm{G_{j2}}$ in (\ref{eqn:13}) and (\ref{eqn:Bj-bound}) can be bounded using the helper Lemma \ref{lem:Fj1}, Lemma \ref{lem:Fj2_denominator}, Lemma \ref{lem:Fj2} and Lemma \ref{lem:vershynin_new} respectively. We defer the proof of these helper lemmas to a later section. To shorten the notation, let us denote
\begin{align}
\Delta \define UH-U^*\in\mathr^{nd\times d},\quad  \Delta^{(j)} \define  U^{(j)}H^{(j)} - U^* \in\mathr^{nd\times d},\forall j\in[n], \label{eqn:Delta_def}
\end{align}
such that block submatrices $\Delta_k = U_kH-U^*_k$ and $ \Delta^{(j)}_k = U^{(j)}_kH^{(j)} - U^*_k$ for each $j,k\in[n]$. 
%For each $j\in[n]$, we analogously define  $\Delta^{(j)} \define  U^{(j)}H^{(j)} - U^* \in\mathr^{nd\times d}$.  
We further introduce $\opnorminf{\cdot}$ norm such that
\begin{align}\label{eqn:Delta_norm_def}
\opnorminf{\Delta} &\define \max_{k\in[n]} \opnorm{\Delta_k},\quad\opnorminf{\Delta^{(j)}}\define \max_{k\in[n]} \opnorm{\Delta^{(j)}_k}.
\end{align}
They are related as shown below in Lemma \ref{lem:Delta_Delta_j}.

\begin{lemma}\label{lem:Delta_Delta_j}
Assume (\ref{eqn:14}) holds. Then for any $j\in[n]$, we have
\begin{align}
\opnorminf{\Delta^{(j)}} &\leq 7 \br{ \opnorminf{\Delta}+\frac{1}{\sqrt{n}}}.   \label{eqn:15}
\end{align}
\end{lemma}
\begin{proof}
Consider any $j\in[n]$, we have
\begin{align*}
\opnorminf{\Delta^{(j)}} &\leq \opnorminf{\Delta} +  \opnorminf{\Delta - \Delta^{(j)}} \leq \opnorminf{\Delta} +   \opnorm{\Delta - \Delta^{(j)}} \\
&= \opnorminf{\Delta} + \opnorm{UH-U^{(j)}H^{(j)}} = \opnorminf{\Delta} + \opnorm{UU^\top U^*-U^{(j)}U^{(j)\top} U^*}\\
&\leq  \opnorminf{\Delta} + \opnorm{UU^\top -U^{(j)}U^{(j)\top}} \leq  \opnorminf{\Delta} + 6\br{\opnorm{U_jH -U^*_j}+\frac{1}{\sqrt{n}}}\\
&\leq 7 \br{ \opnorminf{\Delta}+\frac{1}{\sqrt{n}}}, 
\end{align*}
where the second to last inequality is due to  (\ref{eqn:14}).
\end{proof}

We are now ready to prove Theorem \ref{thm:l_infty_new}.
\begin{proof}[\textbf{Proof of Theorem \ref{thm:l_infty_new}}]
From Lemma \ref{lem:event-E-prime_new}, there exist constants $C_0', C_0 > 0$ such that if $\frac{np}{\log n} > C_0'$, then the event $\calE$ holds with probability at least $1-n^{-10}$. Assume $\calE$ holds and $\frac{np}{(\sqrt{d} + \sigma d)^2} \geq 22^2 C_0^2$. Then according to Lemma \ref{lem:event-E-prime_new}, we have (\ref{eqn:2})-(\ref{eqn:12}) hold. As a consequence, by Lemma \ref{lem:simplification} and Lemma \ref{lem:Delta_Delta_j}, we have (\ref{eqn:13}) and  (\ref{eqn:15}) all hold as well. 

Consider any $j\in[n]$. Note that (\ref{eqn:13}) can be written as
\begin{align}
\opnorm{\Delta_j} &\leq  \frac{22C_0}{\sqrt{n}}\br{\frac{1+\sigma \sqrt{d}}{\sqrt{np}}}+2\opnorm{F_{j1}} + 2\opnorm{F_{j2}} + 2\opnorm{G_{j1}} + 2\opnorm{G_{j2}}.\label{eqn:17}
\end{align}
Hence, to upper bound $\opnorm{\Delta_j}$, we need to study $\opnorm{F_{j1}}$, $\opnorm{F_{j2}}$, $\opnorm{G_{j1}}$, and $\opnorm{G_{j2}}$.   By Lemma \ref{lem:Fj1}, we have
\begin{align*}
\opnorm{\sum_{k\neq j} A_{jk} \Delta_k^{(j)}}  &\leq   p\sqrt{n}   \opnorm{\Delta^{(j)}}  + \sqrt{40np\opnorminf{\Delta^{(j)}}^2 \log n}  + \frac{40}{3} \log n\opnorminf{\Delta^{(j)}}\\
&\leq   p\sqrt{n}   \opnorm{\Delta^{(j)}}  + \br{ \sqrt{40np \log n} + \frac{40}{3} \log n} \opnorminf{\Delta^{(j)}},
\end{align*}
holds with probability at least $1-2dn^{-10}$. Then with (\ref{eqn:3}), we have
\begin{align*}
\opnorm{F_{j1}} &\leq \opnorm{\sum_{k\neq j} A_{jk} \Delta_k^{(j)}} \opnorm{\Lambda^{-1}}\\
 &\leq     \frac{8}{7np} \br{ p\sqrt{n}   \opnorm{\Delta^{(j)}}  + \br{ \sqrt{40np \log n} + \frac{40}{3} \log n} \opnorminf{\Delta^{(j)}}}\\
 &\leq   \frac{C_1}{\sqrt{n}} \opnorm{\Delta^{(j)}}  +C_1 \sqrt{\frac{\log n}{np}} \opnorminf{\Delta^{(j)}},\numberthis \label{eqn:18}
\end{align*}
for some constant $C_1>0$. By Lemma \ref{lem:Fj2}, we have
\begin{align*}
\opnorm{\sum_{k\neq j}A_{jk}W_{jk}\Delta^{(j)}_k} &\leq C_2 \sqrt{\log n} \opnorm{ \sum_{k\neq j}A_{jk}(\Delta^{(j)}_k)^\top \Delta^{(j)}_k}^\frac{1}{2}
\end{align*}
holds with probability at least $1-n^{-10}$ for some constant $C_2>0$. In addition, by Lemma \ref{lem:Fj2_denominator}, we have
\begin{align*}
\opnorm{ \sum_{k\neq j}A_{jk}(\Delta^{(j)}_k)^\top \Delta^{(j)}_k}  &\leq   p \opnorm{\Delta^{(j)}}^2 +\sqrt{40p\opnorm{\Delta^{(j)}}^2\opnorminf{\Delta^{(j)}}^2 \log n}  + \frac{40}{3} \log n\opnorminf{\Delta^{(j)}}^2\\
 &\leq   p \opnorm{\Delta^{(j)}}^2 +\sqrt{40np\opnorminf{\Delta^{(j)}}^4 \log n}  + \frac{40}{3} \log n\opnorminf{\Delta^{(j)}}^2\\
&\leq p \opnorm{\Delta^{(j)}}^2  +\br{ \sqrt{40np \log n} + \frac{40}{3} \log n} \opnorminf{\Delta^{(j)}}^2
\end{align*}
holds with probability at least $1-2dn^{-10}$.  Then with (\ref{eqn:3}), we have
\begin{align*}
\opnorm{F_{j2}} &\leq \sigma \opnorm{\sum_{k\neq j}A_{jk}W_{jk}\Delta^{(j)}_k}  \opnorm{\Lambda^{-1}}\\
&\leq  \frac{8\sigma}{7np}  C_2 \sqrt{\log n} \sqrt{p \opnorm{\Delta^{(j)}}^2  +\br{ \sqrt{40np \log n} + \frac{40}{3} \log n} \opnorminf{\Delta^{(j)}}^2}\\
&\leq  \frac{8\sigma}{7np}  C_2 \sqrt{\log n}   \br{\sqrt{p}  \opnorm{\Delta^{(j)}} + \sqrt{ \sqrt{40np \log n} + \frac{40}{3} \log n} \opnorminf{\Delta^{(j)}}}\\
&\leq  \frac{C_3\sigma}{\sqrt{n}}\sqrt{\frac{\log n}{np}} \opnorm{\Delta^{(j)}} + C_3\sigma \br{\frac{\log n}{np}}^\frac{3}{4}  \opnorminf{\Delta^{(j)}}, \numberthis \label{eqn:19}
\end{align*}
for some constant $C_3>0$. For $\norm{G_{j1}}$, its upper bound is given in (\ref{eqn:16}). By Lemma \ref{lem:vershynin_new}, using the fact that $\{A_{ij}\}_{k\neq j}$ and $\{W_{jk}\}_{k\neq j}$ are independent, we have
\begin{align*}
\opnorm{\sum_{k\neq j} A_{jk} W_{jk}} &\leq \sqrt{\sum_{k\neq j} A_{jk}} \br{2\sqrt{d} + C_4\sqrt{\log n}},
\end{align*}
with probability at least $1-n^{-10}$,
for some constant $C_4>0$. Then
\begin{align*}
\opnorm{G_{j2}} &\leq \frac{\sigma}{\sqrt{n}} \opnorm{\sum_{k\neq j} A_{jk} W_{jk}} \opnorm{\Lambda^{-1}}\\
&\leq  \frac{\sigma}{\sqrt{n}}  \sqrt{\sum_{k\neq j} A_{jk}} \br{2\sqrt{d} + C_4\sqrt{\log n}}\opnorm{\Lambda^{-1}}\\
&\leq C_5\frac{\sigma}{\sqrt{n}}\br{\sqrt{\frac{d}{np}} + \sqrt{\frac{\log n }{np}}}, \numberthis \label{eqn:20}
\end{align*}
for some constant $C_5>0$, where the last inequality is by $\calE$ and (\ref{eqn:3}). 

Plugging (\ref{eqn:18}), (\ref{eqn:19}), (\ref{eqn:16}), and (\ref{eqn:20}) into (\ref{eqn:17}), we have
\begin{align*}
\opnorm{\Delta_j} &\leq  \frac{22C_0}{\sqrt{n}}\br{\frac{1+\sigma \sqrt{d}}{\sqrt{np}}}+2\br{ \frac{C_1}{\sqrt{n}} \opnorm{\Delta^{(j)}}  +C_1 \sqrt{\frac{\log n}{np}} \opnorm{\Delta^{(j)}}} \\
&\quad + 2\br{ \frac{C_3\sigma}{\sqrt{n}}\sqrt{\frac{\log n}{np}} \opnorm{\Delta^{(j)}} + C_3\sigma \br{\frac{\log n}{np}}^\frac{3}{4}  \opnorminf{\Delta^{(j)}}} + \frac{4C_0}{\sqrt{n}}\br{\sqrt{\frac{\log n}{np}} + \frac{\sigma \sqrt{d}}{\sqrt{np}}} \\
&\quad + 2C_5\frac{\sigma}{\sqrt{n}}\br{\sqrt{\frac{d}{np}} + \sqrt{\frac{\log n }{np}}}\\
&\leq \frac{2}{\sqrt{n}} \br{C_1 + C_3\sigma\sqrt{\frac{\log n}{np}}} \opnorm{\Delta^{(j)}}  +2\br{C_1\sqrt{\frac{\log n}{np}} + C_3\sigma \br{\frac{\log n}{np}}^\frac{3}{4} }  \opnorminf{\Delta^{(j)}} \\
&\quad + \frac{26C_0 + 2C_5}{\sqrt{n}} \frac{\sqrt{\log n} + \sigma \sqrt{d} + \sigma\sqrt{\log n}}{\sqrt{np}}.
\end{align*}
By (\ref{eqn:15}) and (\ref{eqn:12}), we can replace the terms $\opnorm{\Delta^{(j)}}$, $\opnorminf{\Delta^{(j)}}$ with their respective upper bounds and have
\begin{align*}
\opnorm{\Delta_j} &\leq   \frac{2}{\sqrt{n}} \br{C_1 + C_3\sigma\sqrt{\frac{\log n}{np}}} \frac{9C_0(1+\sigma \sqrt{d})}{\sqrt{np}} +14\br{C_1\sqrt{\frac{\log n}{np}} + C_3\sigma \br{\frac{\log n}{np}}^\frac{3}{4} }   \br{ \opnorminf{\Delta}+\frac{1}{\sqrt{n}}} \\
&\quad + \frac{26C_0 + 2C_5}{\sqrt{n}} \frac{\sqrt{\log n} + \sigma \sqrt{d} + \sigma\sqrt{\log n}}{\sqrt{np}}.
\end{align*}
By a union bound, with  probability at least $1-6dn^{-9}$, the above display holds for all $j\in[n]$.  Then
\begin{align*}
\opnorminf{\Delta} &\leq   \frac{2}{\sqrt{n}} \br{C_1 + C_3\sigma\sqrt{\frac{\log n}{np}}} \frac{9C_0(1+\sigma \sqrt{d})}{\sqrt{np}} +14\br{C_1\sqrt{\frac{\log n}{np}} + C_3\sigma \br{\frac{\log n}{np}}^\frac{3}{4} }   \br{ \opnorminf{\Delta}+\frac{1}{\sqrt{n}}} \\
&\quad + \frac{26C_0 + 2C_5}{\sqrt{n}} \frac{\sqrt{\log n} + \sigma \sqrt{d} + \sigma\sqrt{\log n}}{\sqrt{np}}.
\end{align*}
After a rearrangement, we have
\begin{align*}
&\br{1- 14\br{C_1\sqrt{\frac{\log n}{np}} + C_3\sigma \br{\frac{\log n}{np}}^\frac{3}{4} }} \opnorminf{\Delta} \\
&\leq \frac{2}{\sqrt{n}} \br{C_1 + C_3\sigma\sqrt{\frac{\log n}{np}}} \frac{9C_0(1+\sigma \sqrt{d})}{\sqrt{np}} +  \frac{14}{\sqrt{n}}\br{C_1\sqrt{\frac{\log n}{np}} + C_3\sigma \br{\frac{\log n}{np}}^\frac{3}{4} }\\
&\quad + \frac{26C_0 + 2C_5}{\sqrt{n}} \frac{\sqrt{\log n} + \sigma \sqrt{d} + \sigma\sqrt{\log n}}{\sqrt{np}}.
\end{align*}
Then, there exists a constant $C>0$ such that if $\frac{np}{(\sigma^{4/3}  \vee 1)\log n} >C$, we have $14\br{C_1\sqrt{\frac{\log n}{np}} + C_3\sigma \br{\frac{\log n}{np}}^\frac{3}{4} }\leq \frac{1}{2}$. Consequently, we have
\begin{align*}
\opnorminf{\Delta} &\leq \frac{4}{\sqrt{n}} \br{C_1 + C_3\sigma\sqrt{\frac{\log n}{np}}} \frac{9C_0(1+\sigma \sqrt{d})}{\sqrt{np}} +  \frac{28}{\sqrt{n}}\br{C_1\sqrt{\frac{\log n}{np}} + C_3 \sigma \br{\frac{\log n}{np}}^\frac{3}{4} }\\
&\quad + \frac{52C_0 + 4C_5}{\sqrt{n}} \frac{\sqrt{\log n} + \sigma \sqrt{d} + \sigma\sqrt{\log n}}{\sqrt{np}}\\
&\leq \frac{C_6}{\sqrt{n}}\br{\sqrt{\frac{\log n }{np}} + \frac{\sigma \sqrt{d}}{\sqrt{np}} + \frac{\sigma \sqrt{\log n}}{\sqrt{np}}},
\end{align*}
for some constant $C_6>0$.


\end{proof}


%
%
%
%From (\ref{eqn:decomposition3}), we have
%\begin{align*}
%\opnorm{U_jH - U^*_j } &\leq \opnorm{U_j }   \opnorm{H\Lambda - \Lambda H} \opnorm{\Lambda^{-1}}+ \opnorm{X_j}\opnorm{UU^\top - U^{(j)}U^{(j)\top}} \opnorm{\Lambda^{-1}} \\
%&\quad +\opnorm{F_{j1}} + \opnorm{F_{j2}} + \opnorm{G_{j1}} + \opnorm{G_{j2}}.
%\end{align*}
%Assume 
%\begin{align*}
%&\leq \br{ \opnorm{U_jH-U^*_j}  + \frac{1}{\sqrt{n}}}\opnorm{H^{-1}}  \opnorm{H\Lambda - \Lambda H} \opnorm{\Lambda^{-1}} \\
%&\quad + \opnorm{X_j}\opnorm{UU^\top - U^{(j)}U^{(j)\top}} \opnorm{\Lambda^{-1}} \\
%&\quad +\opnorm{F_{j1}} + \opnorm{F_{j2}} + \opnorm{G_{j1}} + \opnorm{G_{j2}}.
%\end{align*}
%According to Lemma \ref{lem:event-E-prime}, there exist constants $C,C_0>0$ such that if $\frac{np}{\log n}>C$ and $\frac{np}{(1+\sigma \sqrt{d})^2 }\geq 64C_0^2$, we have  (\ref{eqn:2})-(\ref{eqn:6}) hold with probability at least $1-n^{-10}$. With these results, we have
%\begin{align*}
%\opnorm{U_jH - U^*_j } &\leq C_1\br{\frac{1+\sigma \sqrt{d}}{\sqrt{np}}} \br{ \opnorm{U_jH-U^*_j}  + \frac{1}{\sqrt{n}}} \\
%&\quad  +\opnorm{F_{j1}} + \opnorm{F_{j2}} + \opnorm{G_{j1}} + \opnorm{G_{j2}},
%\end{align*}
%for some constant $C_1>0$. If $\frac{np}{(1+\sigma\sqrt{d})^2}\geq 4C
%_1^2$ is further assumed, after rearrangement, we have
%\begin{align*}
%\opnorm{U_jH - U^*_j } &\leq \frac{2C_1}{\sqrt{n}}\br{\frac{1+\sigma \sqrt{d}}{\sqrt{np}}}  +2\opnorm{F_{j1}} + 2\opnorm{F_{j2}} + 2\opnorm{G_{j1}} + 2\opnorm{G_{j2}}.
%\end{align*}






\subsection{Proof of Theorem \ref{thm:main_new}}
%We first state and prove a  result that is analogous to Theorem \ref{thm:l_infty_new} but is for $ \max_{i\neq j} \opnormt{U_i^{(j)} H^{(j)} - U^*_i}$ across all $j\in[n]$.
%\begin{lemma}\label{lem:l_infty_leave_one}
%There exist constants $C,C',C''>0$ such that if $\frac{np}{(\sigma^{4/3}  \vee 1)\log n} >C$ and $\frac{np}{(\sqrt{d} + \sigma d)^2}\geq C'$, we have
%\begin{align}\label{eqn:22}
%\max_{j\in[n]} \max_{i\neq j} \opnorm{U_i^{(j)} H^{(j)} - U^*_i}\leq \frac{ C''}{\sqrt{n}}\br{\sqrt{\frac{\log n }{np}} + \frac{\sigma \sqrt{d}}{\sqrt{np}} + \frac{\sigma \sqrt{\log n}}{\sqrt{np}}}
%\end{align}
%with probability at least $1-7dn^{-8}$.
%\end{lemma}
%\begin{proof}
%Consider any $j\in[n]$. Then $X^{(j)}$ is essentially a data matrix of synchronization over $(n-1)$ permutations.
%Despite some abuse of notation, define $U^{*(j)}\in\mathr^{nd\times d}$ such that  $U^{*(j)}_j\define 0_{d\times d}$ and $U^{*(j)}_i \define I_d/\sqrt{n-1}$ for all $i\neq j$. Apply Theorem \ref{thm:l_infty_new} to $X^{(i)}$. There exist constants $C,C',C''>0$ such that if $\frac{(n-1)p}{(\sigma^{4/3}  \vee 1)\log (n-1)} >C$ and $\frac{(n-1)p}{(\sqrt{d} + \sigma d)^2}\geq C'$, we have 
%\begin{align*}
%\max_{i\neq j} \opnorm{U^{(j)}_i (U^{(j)\top} U^{*{(j)}}) - U^{*{(j)}}} \leq  \frac{ C''}{\sqrt{n-1}}\br{\sqrt{\frac{\log (n-1) }{(n-1)p}} + \frac{\sigma \sqrt{d}}{\sqrt{(n-1)p}} + \frac{\sigma \sqrt{\log (n-1)}}{\sqrt{(n-1)p}}}
%\end{align*}
%with probability at least $1-6d(n-1)^{-9}$. Note that $U^{(j)}_j =0_{d\times d}$ and $U^{*(j)}_i = \sqrt{n/(n-1)}U^*_i$ for each $i\neq j$.
%Then
%\begin{align*}
%U^{(j)\top} U^{*{(j)}} & = \sum_{k\neq j} U^{(j)\top}_k U^{*{(j)}}_k =  \sqrt{\frac{n}{n-1}}\sum_{k\neq j} U^{(j)\top}_kU^{*}_k =    \sqrt{\frac{n}{n-1}}U^{(j)\top} U^* =  \sqrt{\frac{n}{n-1}}H^{(j)}.
%\end{align*}
%Consequently,
%\begin{align*}
% \max_{i\neq j} \opnorm{U_i^{(j)}H^{(j)}  - U^*_i} &=  \max_{i\neq j} \opnorm{U_i^{(j)} \sqrt{\frac{n-1}{n}} U^{(j)\top} U^{*{(j)}} -  \sqrt{\frac{n-1}{n}} U^{*(j)}_i} \\
% & = \sqrt{\frac{n-1}{n}} \max_{i\neq j} \opnorm{U^{(j)}_i (U^{(j)\top} U^{*{(j)}}) - U^{*{(j)}}} \\
% &\leq  \frac{2 C''}{\sqrt{n}}\br{\sqrt{\frac{\log n }{np}} + \frac{\sigma \sqrt{d}}{\sqrt{np}} + \frac{\sigma \sqrt{\log n}}{\sqrt{np}}}.
%\end{align*}
%A union bound over all $j\in[n]$ completes the proof.
%\end{proof}

\textbf{Decomposition of the Hamming loss.} From Lemma \ref{lem:event-E-prime_new}, there exist constants $C_0', C_0 > 0$ such that if $\frac{np}{\log n} > C_0'$, then the event $\calE$ holds with probability at least $1-n^{-10}$. By Theorem \ref{thm:l_infty_new}, there exist some constants $C,C',C''>0$ such that if $\frac{np}{(\sigma^{4/3}  \vee 1)\log n} >C$ and $\frac{np}{(\sqrt{d} + \sigma d)^2}\geq C'$, we have (\ref{eqn:22}) holds with probability at least  $1-6dn^{-9}$. Denote $\mathf$ to be the event that both $\calE$ and (\ref{eqn:22})  hold. By union bound, we have
\begin{align}\label{eqn:36}
\pbr{\mathf} \geq 1-7dn^{-9}.
\end{align}
Assume $\mathf$ holds and $\frac{np}{(\sqrt{d} + \sigma d)^2} \geq 22^2 C_0^2$. Then according to Lemma \ref{lem:event-E-prime_new}, we have (\ref{eqn:2})-(\ref{eqn:12}) hold. As a consequence, by Lemma \ref{lem:simplification}, Lemma \ref{lem:Delta_Delta_j}, and Lemma \ref{lem:hat_H}, we have (\ref{eqn:13}),  (\ref{eqn:15}), and (\ref{eqn:11}) all hold as well.  Since 
\begin{align}
\E\ell(\hat Z,Z^*) \leq \E\ell(\hat Z,Z^*) \indic{\mathf}  + \pbr{\mathf^\mathrm{c}},\label{eqn:35}
\end{align}
in the following we focus on analyzing $\E\ell(\hat Z,Z^*) \indic{\mathf} $.

Define
\begin{align*}
    \hat P \define \argmin_{P\in \Pi_d} \, \fnorm{\hatH - P H^\top}.
\end{align*}
Since we let $Z^*_j=I_d$ for all $j\in[n]$, we have
\begin{align*}
\E \ell(\hat Z,Z^*)  \indic{\mathf} &\leq \E \br{ \frac{1}{n} \sum_{j\in[n]} \indic{\hat Z_j \neq Z^*_j \hat P^\top}  \indic{\mathf} } = \frac{1}{n}\sum_{j\in[n]} \E  \indic{\hat Z_j \neq Z^*_j \hat P^\top}  \indic{\mathf} \\
&\leq \frac{1}{n}\sum_{j\in[n]} \E  \indic{\exists R\in\Pi_d \text{ s.t. }R\neq I_d \text{ and }\fnorm{ \sqrt{n} U_jM^\top - R \hat P^\top }^2 \leq \fnorm{  \sqrt{n} U_jM^\top - Z^*_j \hat P^\top}^2}  \indic{\mathf} \\
& =  \frac{1}{n}\sum_{j\in[n]} \E  \indic{\exists R\in\Pi_d \text{ s.t. }R\neq I_d \text{ and }\fnorm{  U_jM^\top - \frac{R \hat P^\top}{\sqrt{n}} }^2 \leq \fnorm{  U_jM^\top - U^*_j \hat P^\top}^2}  \indic{\mathf}\\
&\leq  \frac{1}{n}\sum_{j\in[n]} \sum_{R\in\Pi_d: R\neq I_d} \E \indic{\fnorm{  U_jM^\top - \frac{R \hat P^\top}{\sqrt{n}} }^2 \leq \fnorm{  U_jM^\top - U^*_j \hat P^\top}^2}  \indic{\mathf}, \numberthis \label{eqn:26}
\end{align*}
where the second inequity is by the definition of $\hat Z_j$ in (\ref{eqn:alg3}) and the second equation is by that we let $Z^*_j=I_d$.


Consider  any $R\in \Pi_d$ such that $R\neq I_d$. Define
\begin{align*}
h_R \define \fnorm{R - I_d}^2/2.
\end{align*}
Then $2\leq h_R\leq d$. Consider any $j\in[n]$. In the following, we are going to establish an upper bound for $ \E \indic{\fnorm{  U_jM^\top - \frac{R \hat P^\top}{\sqrt{n}} }^2 \leq \fnorm{  U_jM^\top - U^*_j \hat P^\top}^2}  \indic{\mathf}$, the key quantity in  (\ref{eqn:26}).

We have
\begin{align*}
    &\indic{ \fnorm{U_j \hatH^\top - U_j^*\hat P^\top} \geq \fnorm{U_j \hatH^\top - \frac{R \hat P^\top}{\sqrt n} }   }  \indic{\mathf} \\
    &= \indic{\iprod{U_j \hatH^\top - U_j^*\hat P^\top}{ \frac{R \hat P^\top}{\sqrt n} - U_j^*\hat P^\top}\geq \frac{1}{2}\fnorm{ \frac{R \hat P^\top}{\sqrt n} - U_j^*\hat P^\top}^2 } \indic{\mathf} \\
    &=\indic{\iprod{U_j \hatH^\top \hat P - U_j^*}{ \frac{R}{\sqrt n} - U_j^*}\geq \frac{1}{2}\fnorm{ \frac{R }{\sqrt n} - U_j^*}^2 } \indic{\mathf}  \\
    & = \indic{\iprod{U_j \hatH^\top \hat P - U_j^*}{R-I_d} \geq  \frac{h_R}{\sqrt{n}}} \indic{\mathf}  \\
     & = \indic{\iprod{U_j H - U_j^*}{R-I_d} + \iprod{U_j\br{M - \hat PH^\top}^\top \hat P}{ R-I_d} \geq  \frac{h_R}{\sqrt{n}}} \indic{\mathf}. \numberthis \label{eqn:9}
\end{align*}

\textbf{Bounding the Hamming loss.} A significant remaining portion of the proof is dedicated to bounding (\ref{eqn:9}). The two inner products involved in (\ref{eqn:9}) can be further decomposed. For the second inner product, we have
\begin{align*}
&\iprod{U_j\br{M - PH^\top}^\top \hat P}{ R-I_d} \\
&\leq  \fnorm{U_j \br{M -\hat P \hat H^\top}^\top}\fnorm{ R-I_d}\\
&\leq  \opnorm{U_j} \fnorm{M -\hat P \hat H^\top}\fnorm{ R-I_d}\\
&\leq   \frac{4}{3}\br{\opnorm{U_jH-U^*_j}+\frac{1}{\sqrt{n}}}\br{ \frac{{4C_0\br{\sqrt{d} + \sigma d}}}{\sqrt{np}}}\sqrt{2h_R}\\
&\leq  \frac{4}{3}\br{\opnorm{B_j}+\opnorm{F_{j1}}+\opnorm{F_{j2}}+\opnorm{G_{j1}}+ \opnorm{G_{j2}}+\frac{1}{\sqrt{n}}}\br{ \frac{{4C_0\br{\sqrt{d} + \sigma d}}}{\sqrt{np}}}\sqrt{2h_R},
%&\leq  \frac{4}{3}\br{\frac{ C''}{\sqrt{n}}\br{\sqrt{\frac{\log n }{np}} + \frac{\sigma \sqrt{d}}{\sqrt{np}} + \frac{\sigma \sqrt{\log n}}{\sqrt{np}}}+\frac{1}{\sqrt{n}}}\br{ \frac{1}{\sqrt{n}}\frac{{4C_0\br{\sqrt{d} + \sigma d}}}{\sqrt{np}}}\sqrt{2h_R}\\
%&\leq \frac{ C_1}{n} \br{1+\frac{\sigma \sqrt{\log n}}{\sqrt{np}}} \sqrt{h_R}, \numberthis \label{eqn:23}
%&\leq \frac{C_1}{\sqrt{n}} \br{\sqrt{\frac{\log n }{np}} + \frac{\sigma \sqrt{d}}{\sqrt{np}} + \frac{\sigma \sqrt{\log n}}{\sqrt{np}}} \frac{\sqrt{d} + \sigma d}{\sqrt{np}} \frac{\sqrt{h_R}}{\sqrt{n}}
\end{align*}
%for some constant $C_1>0$,
where the third inequality is by (\ref{eqn:10}) and (\ref{eqn:11}) and the four inequality is by (\ref{eqn:decomposition3}).
For the first inner product of (\ref{eqn:9}), by (\ref{eqn:decomposition3}), we have
\begin{align*}
\iprod{U_j H - U_j^*}{R-I_d} &= \iprod{B_j+ F_{j1} + F_{j2} + G_{j1} }{R-I_d}    + \iprod{G_{j2}}{R-I_d}\\
&\leq \fnorm{B_j+ F_{j1} + F_{j2} + G_{j1}} \fnorm{R-I_d}  + \iprod{G_{j2}}{R-I_d}\\
&\leq \sqrt{d}\opnorm{B_j+ F_{j1} + F_{j2} + G_{j1}}  \sqrt{2h_R} + \iprod{G_{j2}}{R-I_d}\\
&\leq  \br{\opnorm{B_j} + \opnorm{F_{j1}} + \opnorm{F_{j2}} + \opnorm{G_{j1}} } \sqrt{2dh_R} + \iprod{G_{j2}}{R-I_d}.
\end{align*}
From (\ref{eqn:Bj-bound}) in Lemma \ref{lem:simplification}, since $\frac{np}{(1+\sigma \sqrt{d})^2} \geq 22^2 C_0^2$, we have
\begin{align*}
\opnorm{B_j} &\leq  \frac{22C_0}{\sqrt{n}} \br{\frac{1+\sigma \sqrt{d}}{\sqrt{np}}} + \opnorm{F_{j1}} + \opnorm{F_{j2}} + \opnorm{G_{j1}}  + 22C_0\br{\frac{1+\sigma \sqrt{d}}{\sqrt{np}}}  \opnorm{G_{j2}}.
\end{align*}
From the above three displays, we have
\begin{align*}
&\iprod{U_j H - U_j^*}{R-I_d} + \iprod{U_j\br{M - \hat PH^\top}^\top \hat P}{ R-I_d} \\
&\leq \br{1+ \frac{16C_0(1+\sigma\sqrt{d})}{3\sqrt{np}} }\br{\opnorm{B_j} + \opnorm{F_{j1}} + \opnorm{F_{j2}} + \opnorm{G_{j1}} } \sqrt{2dh_R} \\
&\quad +\frac{16C_0(1+\sigma\sqrt{d})}{3\sqrt{np}} \sqrt{2dh_R}\opnorm{G_{j2}}+ \iprod{G_{j2}}{R-I_d} + \frac{16C_0(1+\sigma\sqrt{d})}{3\sqrt{np}} \frac{\sqrt{2dh_R}}{\sqrt{n}}\\
&\leq  2\br{1+ \frac{16C_0(1+\sigma\sqrt{d})}{3\sqrt{np}} }\br{ \opnorm{F_{j1}} + \opnorm{F_{j2}} + \opnorm{G_{j1}} } \sqrt{2dh_R} \\
&\quad +\br{\frac{41}{33} +\frac{16C_0(1+\sigma\sqrt{d})}{3\sqrt{np}}  }22C_0\br{\frac{1+\sigma \sqrt{d}}{\sqrt{np}}}  \opnorm{G_{j2}} \sqrt{2dh_R} + \iprod{G_{j2}}{R-I_d}\\
&\quad +\br{\frac{41}{33} +\frac{16C_0(1+\sigma\sqrt{d})}{3\sqrt{np}}  }22C_0\br{\frac{1+\sigma \sqrt{d}}{\sqrt{np}}} \frac{\sqrt{2dh_R} }{\sqrt{n}}\\
&\leq  2\br{1+ \frac{16C_0(1+\sigma\sqrt{d})}{3\sqrt{np}} }\br{ \opnorm{F_{j1}} + \opnorm{F_{j2}} } \sqrt{2dh_R} \\
&\quad +\br{\frac{41}{33} +\frac{16C_0(1+\sigma\sqrt{d})}{3\sqrt{np}}  }22C_0\br{\frac{1+\sigma \sqrt{d}}{\sqrt{np}}}  \opnorm{G_{j2}} \sqrt{2dh_R} + \iprod{G_{j2}}{R-I_d}\\
&\quad +\br{\frac{41}{33} +\frac{16C_0(1+\sigma\sqrt{d})}{3\sqrt{np}}  }22C_0\br{\frac{1+\sigma \sqrt{d}}{\sqrt{np}}} \frac{\sqrt{2dh_R} }{\sqrt{n}}\\
&\quad + 4\br{1+ \frac{16C_0(1+\sigma\sqrt{d})}{3\sqrt{np}} } C_0\br{\frac{\sqrt{\log n} + \sigma\sqrt{d}}{\sqrt{np}}} \frac{\sqrt{2dh_R}}{\sqrt{n}}\\
&\leq  2\br{1+ \frac{16C_0(1+\sigma\sqrt{d})}{3\sqrt{np}} }\br{ \opnorm{F_{j1}} + \opnorm{F_{j2}} } \sqrt{2dh_R} \\
&\quad +\br{\frac{41}{33} +\frac{16C_0(1+\sigma\sqrt{d})}{3\sqrt{np}}  }22C_0\br{\frac{1+\sigma \sqrt{d}}{\sqrt{np}}}  \opnorm{G_{j2}} \sqrt{2dh_R} + \iprod{G_{j2}}{R-I_d}\\
&\quad +\br{\frac{41}{33} +\frac{16C_0(1+\sigma\sqrt{d})}{3\sqrt{np}}  }26C_0\br{\frac{\sqrt{\log n}+\sigma \sqrt{d}}{\sqrt{np}}} \frac{\sqrt{2dh_R} }{\sqrt{n}}\,,
\end{align*}
where the second last inequality is by the upper bound on $\opnorm{G_{j1}}$ in (\ref{eqn:16}) and the last inequality is obtained by combining the two terms containing $\frac{\sqrt{2dh_R}}{\sqrt n}$.
%\begin{align*}
%\iprod{U_j H - U_j^*}{R-I_d}&\leq \frac{22C_0\sqrt{2dh_R}}{\sqrt{n}} \br{\frac{1+\sigma \sqrt{d}}{\sqrt{np}}} + 2\sqrt{2dh_R} \br{\opnorm{F_{j1}} + \opnorm{F_{j2}} + \opnorm{G_{j1}} }\\
%&\quad + 22C_0\br{\frac{1+\sigma \sqrt{d}}{\sqrt{np}}} \sqrt{2dh_R} \opnorm{G_{j2}}+ \iprod{G_{j2}}{R-I_d}. \numberthis \label{eqn:24}
%\end{align*}
The above display involves $ \iprod{G_{j2}}{R-I_d}$ which can be decomposed:
\begin{align*}
 \iprod{G_{j2}}{R-I_d} & = \frac{\sigma}{\sqrt{n}} \iprod{\sum_{k\neq j} A_{jk} W_{jk} \Lambda^{-1}}{R-I_d} \\
 &= \frac{\sigma}{\sqrt{n}} \iprod{\sum_{k\neq j} A_{jk} W_{jk} \frac{1}{np} }{R-I_d}  + \frac{\sigma}{\sqrt{n}} \iprod{\sum_{k\neq j} A_{jk} W_{jk} \br{\Lambda^{-1} -  \frac{1}{np}I_d} }{R-I_d} \\
 &= \frac{\sigma}{\sqrt{n}} \frac{1}{np} \iprod{\sum_{k\neq j} A_{jk} W_{jk} }{R-I_d}  + \frac{\sigma}{\sqrt{n}} \iprod{\sum_{k\neq j} A_{jk} W_{jk} \Lambda^{-1}(npI_d-\Lambda) \frac{1}{np} }{R-I_d} \\
 &= \frac{\sigma}{\sqrt{n}}  \frac{1}{np}\iprod{\sum_{k\neq j} A_{jk} W_{jk} }{R-I_d}  +  \frac{1}{np}\iprod{G_{j2}(npI_d-\Lambda) }{R-I_d} \\
 &\leq  \frac{\sigma}{\sqrt{n}}  \frac{1}{np}\iprod{\sum_{k\neq j} A_{jk} W_{jk} }{R-I_d}  +  \frac{\sqrt{d}}{np}\opnorm{G_{j2} (npI_d -\Lambda)} \fnorm{R-I_d}\\
 &\leq   \frac{\sigma}{\sqrt{n}}\frac{1}{np} \iprod{\sum_{k\neq j} A_{jk} W_{jk} }{R-I_d}  + \frac{\sqrt{d}}{np}\opnorm{G_{j2} }\max_{i\in[n]} \abs{\lambda_i(X)-np} \sqrt{2h_R}\\
  &\leq   \frac{\sigma}{\sqrt{n}}\frac{1}{np}  \iprod{\sum_{k\neq j} A_{jk} W_{jk} }{R-I_d}  + \frac{C_0\sqrt{2dh_R}(1+\sigma\sqrt{d})}{\sqrt{np}} \opnorm{G_{j2} }.
\end{align*}
Hence,
\begin{align*}
&\iprod{U_j H - U_j^*}{R-I_d} + \iprod{U_j\br{M - \hat PH^\top}^\top \hat P}{ R-I_d} \\
&\leq  2\br{1+ \frac{16C_0(1+\sigma\sqrt{d})}{3\sqrt{np}} }\br{ \opnorm{F_{j1}} + \opnorm{F_{j2}} } \sqrt{2dh_R} \\
&\quad +\br{\frac{41}{33} +\frac{16C_0(1+\sigma\sqrt{d})}{3\sqrt{np}}  }23C_0\br{\frac{1+\sigma \sqrt{d}}{\sqrt{np}}}  \opnorm{G_{j2}} \sqrt{2dh_R} \\
&\quad +\br{\frac{41}{33} +\frac{16C_0(1+\sigma\sqrt{d})}{3\sqrt{np}}  }26C_0\br{\frac{\sqrt{\log n}+\sigma \sqrt{d}}{\sqrt{np}}} \frac{\sqrt{2dh_R} }{\sqrt{n}}\\
&\quad +  \frac{\sigma}{\sqrt{n}}\frac{1}{np}  \iprod{\sum_{k\neq j} A_{jk} W_{jk} }{R-I_d}\\
&\leq 4\br{ \opnorm{F_{j1}} + \opnorm{F_{j2}} } \sqrt{2dh_R}  + 46C_0\br{\frac{1+\sigma \sqrt{d}}{\sqrt{np}}}  \opnorm{G_{j2}} \sqrt{2dh_R} \\
&\quad + 52C_0\br{\frac{\sqrt{\log n}+\sigma \sqrt{d}}{\sqrt{np}}} \frac{\sqrt{2dh_R} }{\sqrt{n}} +  \frac{\sigma}{\sqrt{n}}\frac{1}{np}  \iprod{\sum_{k\neq j} A_{jk} W_{jk} }{R-I_d}, \numberthis \label{eqn:25}
\end{align*}
where the last inequality holds when  $\frac{\sqrt{np}}{1+\sigma\sqrt{d}} >\frac{176}{25}C_0$.


%With this and (\ref{eqn:16}),  (\ref{eqn:24}) becomes
%\begin{align*}
%&\iprod{U_j H - U_j^*}{R-I_d}\\
%&\leq \frac{22C_0\sqrt{2dh_R}}{\sqrt{n}} \br{\frac{1+\sigma \sqrt{d}}{\sqrt{np}}} + 2\sqrt{2dh_R} \br{\opnorm{F_{j1}} + \opnorm{F_{j2}} + \opnorm{G_{j1}} }\\
%&\quad + 23C_0\br{\frac{1+\sigma \sqrt{d}}{\sqrt{np}}} \sqrt{2dh_R} \opnorm{G_{j2}}+\frac{\sigma}{\sqrt{n}} \frac{1}{np}  \iprod{\sum_{k\neq j} A_{jk} W_{jk} }{R-I_d} \\
%&\leq  \frac{22C_0\sqrt{2dh_R}}{\sqrt{n}} \br{\frac{1+\sigma \sqrt{d}}{\sqrt{np}}}  +  \frac{4C_0\sqrt{2dh_R}}{\sqrt{n}} \br{\frac{\sqrt{\log n}+ \sigma\sqrt{d}}{\sqrt{np}}} + 2\sqrt{2dh_R} \br{\opnorm{F_{j1}} + \opnorm{F_{j2}}  }\\
%&\quad + 23C_0\br{\frac{1+\sigma \sqrt{d}}{\sqrt{np}}} \sqrt{2dh_R} \opnorm{G_{j2}}+\frac{\sigma}{\sqrt{n}} \frac{1}{np} \iprod{\sum_{k\neq j} A_{jk} W_{jk} }{R-I_d} \\
%&\leq \frac{26C_0\sqrt{2dh_R}}{\sqrt{n}} \br{\frac{\sqrt{\log n}+ \sigma\sqrt{d}}{\sqrt{np}}} + 2\sqrt{2dh_R} \br{\opnorm{F_{j1}} + \opnorm{F_{j2}}  }\\
%&\quad + 23C_0\br{\frac{1+\sigma \sqrt{d}}{\sqrt{np}}} \sqrt{2dh_R} \opnorm{G_{j2}}+\frac{\sigma}{\sqrt{n}} \frac{1}{np} \iprod{\sum_{k\neq j} A_{jk} W_{jk} }{R-I_d}. \numberthis \label{eqn:25}
%\end{align*}

As a result, by (\ref{eqn:25}),  (\ref{eqn:9}) becomes
\begin{align*}
  &\indic{ \fnorm{U_j \hatH^\top - U_j^*\hat P^\top} \geq \fnorm{U_j \hatH^\top - \frac{R \hat P^\top}{\sqrt n} }   }  \indic{\mathf} \\
%    &\leq  \mathbb{I}\Bigg\{ \frac{C_1}{n} \br{1+\frac{\sigma \sqrt{\log n}}{\sqrt{np}}} \sqrt{h_R}+ \frac{26C_0\sqrt{2dh_R}}{\sqrt{n}} \br{\frac{\sqrt{\log n}+ \sigma\sqrt{d}}{\sqrt{np}}} + 2\sqrt{2dh_R} \br{\opnorm{F_{j1}} + \opnorm{F_{j2}}  }\\
%&\quad + 23C_0\br{\frac{1+\sigma \sqrt{d}}{\sqrt{np}}} \sqrt{2dh_R} \opnorm{G_{j2}}+\frac{\sigma}{\sqrt{n}}\frac{1}{np}  \iprod{\sum_{k\neq j} A_{jk} W_{jk} }{R-I_d} \geq  \frac{h_R}{\sqrt{n}} \Bigg\} \indic{\mathf}\\
&\leq   \mathbb{I}\Bigg\{ 4\sqrt{2dh_R} \br{\opnorm{F_{j1}} + \opnorm{F_{j2}}  } +   46C_0\br{\frac{1+\sigma \sqrt{d}}{\sqrt{np}}} \sqrt{2dh_R} \opnorm{G_{j2}}\\
&\quad + \frac{\sigma}{\sqrt{n}}\frac{1}{np}  \iprod{\sum_{k\neq j} A_{jk} W_{jk} }{R-I_d} \geq \br{1- C_2 \sqrt d\br{\frac{\sqrt{\log n}+ \sigma\sqrt{d}}{\sqrt{np}}}}     \frac{h_R}{\sqrt{n}} \Bigg\} \indic{\mathf},
\end{align*}
for some constant $C_2>0$.  Let $\rho>0$ be some quantity whose value will be determined later. We have
\begin{align*}
 &\E\indic{ \fnorm{U_j \hatH^\top - U_j^*\hat P^\top} \geq \fnorm{U_j \hatH^\top - \frac{R \hat P^\top}{\sqrt n} }   }  \indic{\mathf} \\
 &\leq  \E \Bigg( \indic{4\sqrt{2dh_R} \opnorm{F_{j1}}  \geq \rho \frac{h_R}{\sqrt{n}}}\indic{\mathf}   + \indic{4\sqrt{2dh_R} \opnorm{F_{j2}}  \geq \rho \frac{h_R}{\sqrt{n}}}\indic{\mathf}   \\
 &\quad + \indic{ 46C_0\br{\frac{1+\sigma \sqrt{d}}{\sqrt{np}}} \sqrt{2dh_R} \opnorm{G_{j2}} \geq  \rho \frac{h_R}{\sqrt{n}}}\indic{\mathf}   \\
 &\quad  + \indic{ \frac{\sigma}{\sqrt{n}}\frac{1}{np}  \iprod{\sum_{k\neq j} A_{jk} W_{jk} }{R-I_d} \geq \br{1- 3\rho - C_2\sqrt d\br{\frac{\sqrt{\log n}+ \sigma\sqrt{d}}{\sqrt{np}}}}     \frac{h_R}{\sqrt{n}} }\indic{\mathf} \Bigg)\\
 &\leq \E\indic{4\sqrt{2dh_R} \opnorm{F_{j1}}  \geq \rho \frac{h_R}{\sqrt{n}}}\indic{\mathf}  + \E\indic{4\sqrt{2dh_R} \opnorm{F_{j2}}  \geq \rho \frac{h_R}{\sqrt{n}}}\indic{\mathf}\\
 &\quad + \E \indic{ 46C_0\br{\frac{1+\sigma \sqrt{d}}{\sqrt{np}}} \sqrt{2dh_R} \opnorm{G_{j2}} \geq  \rho \frac{h_R}{\sqrt{n}}}\indic{\mathf}   \\
 &\quad   +\E \indic{ \frac{\sigma}{\sqrt{n}}\frac{1}{np}  \iprod{\sum_{k\neq j} A_{jk} W_{jk} }{R-I_d} \geq \br{1- 3\rho - C_2\sqrt d\br{\frac{\sqrt{\log n}+ \sigma\sqrt{d}}{\sqrt{np}}}}     \frac{h_R}{\sqrt{n}} }\indic{\mathf}.  \numberthis \label{eqn:27}
\end{align*}

In the following, we are going to establish upper bounds for each of the four terms in (\ref{eqn:27}). Recall the definitions of $\Delta,\Delta^{(j)}$ in (\ref{eqn:Delta_def}) and those of  $\opnorminf{\Delta},\opnorminf{\Delta^{(j)}}$ in (\ref{eqn:Delta_norm_def}).

\textbf{For the first term in (\ref{eqn:27})}, by (\ref{eqn:3}), we have
\begin{align*}
\E\indic{4\sqrt{2dh_R} \opnorm{F_{j1}}  \geq \rho \frac{h_R}{\sqrt{n}}}\indic{\mathf} &\leq  \E\indic{4\sqrt{2dh_R}\opnorm{\sum_{k\neq j} A_{jk} \Delta^{(j)}_k} \opnorm{\Lambda^{-1}}  \geq \rho \frac{h_R}{\sqrt{n}}}\indic{\mathf} \\
 &\leq \E\indic{\frac{32\sqrt{2dh_R} }{7np}\opnorm{\sum_{k\neq j} A_{jk} \Delta^{(j)}_k}  \geq \rho \frac{h_R}{\sqrt{n}}}\indic{\mathf}.
\end{align*}
Define an event
\begin{align*}
\mathf_{j}\define \mathbb{I} \Bigg\{ & \opnorm{\Delta^{(j)}} \leq \frac{9C_0(1+\sigma \sqrt{d})}{\sqrt{np}},\\
&\opnorminf{\Delta^{(j)}}\leq 7\br{\frac{ C''}{\sqrt{n}}\br{\sqrt{\frac{\log n }{np}} + \frac{\sigma \sqrt{d}}{\sqrt{np}} + \frac{\sigma \sqrt{\log n}}{\sqrt{np}}} + \frac{1}{\sqrt{n}}}
\Bigg\}.
\end{align*}
Note that the upper bound for $\opnorminf{\Delta^{(j)}}$ in $\mathf_j$ is a direct consequence of (\ref{eqn:15}) and (\ref{eqn:22}). Together with (\ref{eqn:12}), we have $\mathf\subset \mathf_j$. The event $\mathf_j$ has an equivalent expression. Define a set
\begin{align*}
\mathcal{Y}\define \Bigg\{ Y&=(Y_1^\top,Y_2^\top,\ldots,Y_n^\top)^\top\in\mathr^{nd\times d}: \opnorm{Y} \leq \frac{9C_0(1+\sigma \sqrt{d})}{\sqrt{np}},\\
&\max_{i\in[n]}\opnorm{Y_i}\leq 7\br{\frac{ C''}{\sqrt{n}}\br{\sqrt{\frac{\log n }{np}} + \frac{\sigma \sqrt{d}}{\sqrt{np}} + \frac{\sigma \sqrt{\log n}}{\sqrt{np}}} + \frac{1}{\sqrt{n}}}
\Bigg\}.
\end{align*}
Then $\mathf_j = \indic{\Delta^{(j)} \in \mathcal{Y}}$. Using that $\Delta^{(j)}$ is independent of $\{A_{jk}\}_{k\neq j}$,
we have
\begin{align*}
\E\indic{4\sqrt{2dh_R} \opnorm{F_{j1}}  \geq \rho \frac{h_R}{\sqrt{n}}}\indic{\mathf} &\leq \E\indic{\frac{32\sqrt{2dh_R} }{7np}\opnorm{\sum_{k\neq j} A_{jk} \Delta^{(j)}_k}  \geq \rho \frac{h_R}{\sqrt{n}}}\indic{\mathf_j}\\
& = \pbr{\frac{32\sqrt{2dh_R} }{7np}\opnorm{\sum_{k\neq j} A_{jk} \Delta^{(j)}_k}  \geq \rho \frac{h_R}{\sqrt{n}}, \mathf_j}\\
& = \E \br{  \pbr{\frac{32\sqrt{2dh_R} }{7np}\opnorm{\sum_{k\neq j} A_{jk} \Delta^{(j)}_k}  \geq \rho \frac{h_R}{\sqrt{n}}, \mathf_j \Bigg| \Delta^{(j)}}}\\
& = \E\br{  \pbr{\frac{32\sqrt{2dh_R} }{7np}\opnorm{\sum_{k\neq j} A_{jk} \Delta^{(j)}_k}  \geq \rho \frac{h_R}{\sqrt{n}} \Bigg| \Delta^{(j)}} \indic{ \Delta^{(j)} \in \mathf_j}}\\
&\leq  \sup_{\Delta^{(j)}\in\mathcal{Y}}  \pbr{\frac{32\sqrt{2dh_R} }{7np}\opnorm{\sum_{k\neq j} A_{jk} \Delta^{(j)}_k}  \geq \rho \frac{h_R}{\sqrt{n}} \Bigg| \Delta^{(j)}}\\
%& =  \pbr{\frac{16\sqrt{2dh_R} }{7np}\opnorm{\sum_{k\neq j} A_{jk} \Delta^{(j)}_k}  \geq \rho \frac{h_R}{\sqrt{n}} \Bigg| \mathf_j} \pbr{\mathf_j}\\
%&\leq  \pbr{\frac{16\sqrt{2dh_R} }{7np}\opnorm{\sum_{k\neq j} A_{jk} \Delta^{(j)}_k}  \geq \rho \frac{h_R}{\sqrt{n}} \Bigg| \mathf_j} \\
%&\leq  \pbr{\opnorm{\sum_{k\neq j} A_{jk} \Delta^{(j)}_k}  \geq \rho \frac{7\sqrt{n}p}{16\sqrt{d}} \Bigg| \mathf_j}\\
&\leq  \sup_{\Delta^{(j)}\in\mathcal{Y}} \pbr{\opnorm{\sum_{k\neq j} A_{jk} \Delta^{(j)}_k}  \geq \rho \frac{7\sqrt{n}p}{32\sqrt{d}} \Bigg| \Delta^{(j)}}, \numberthis \label{eqn:28}
\end{align*}
where in the last inequality, we use the fact that $h_R\geq 2$. Assume $\rho$ satisfies
\begin{align}
% \rho\frac{7\sqrt{n}p}{32\sqrt{d}} &\geq 9C_0(1+\sigma\sqrt{d}) \sqrt{p}, \label{eqn:rho_1} \\
% \rho\frac{7\sqrt{n}p}{32\sqrt{d}} &\geq7\br{\frac{ C''}{\sqrt{n}}\br{\sqrt{\frac{\log n }{np}} + \frac{\sigma \sqrt{d}}{\sqrt{np}} + \frac{\sigma \sqrt{\log n}}{\sqrt{np}}} + \frac{1}{\sqrt{n}}} np.\label{eqn:rho_2} \\
\rho\frac{7\sqrt{n}p}{64\sqrt{d}} \geq 9C_0(1+\sigma\sqrt{d}) \sqrt{dp} \label{eqn:rho_2}\,.
% {\color{red} \text{(38) replaced by }    } &{\color{red}  \rho\frac{7\sqrt{n}p}{32\sqrt{d}} \geq 9C_0(1+\sigma\sqrt{d}) \sqrt{dp}.} \nonumber
\end{align}
Then by Lemma \ref{lem:Fj1}, we have
\begin{align*}
\E\indic{4\sqrt{2dh_R} \opnorm{F_{j1}} \geq \rho \frac{h_R}{\sqrt{n}}}\indic{\mathf} &\leq \sup_{\Delta^{(j)}\in\mathcal{Y}} \pbr{\opnorm{\sum_{k\neq j} A_{jk} \Delta^{(j)}_k}  \geq p\sqrt{n}\opnorm{\Delta^{(j)}}+ \rho \frac{7\sqrt{n}p}{64\sqrt{d}} \Bigg| \Delta^{(j)}}\\
&\leq 2d\sup_{\Delta^{(j)}\in\mathcal{Y}}  \ebr{-\frac{\br{ \rho \frac{7\sqrt{n}p}{64\sqrt{d}} }^2/2}{p\sqrt{nd} \opnorm{\Delta^{(j)}}\opnorminf{\Delta^{(j)}}  +  \rho \frac{7\sqrt{n}p}{32\sqrt{d}} \opnorminf{\Delta^{(j)}} /3}}\\
% &\leq 2d\sup_{\Delta^{(j)}\in\mathcal{Y}}  \ebr{-\frac{\br{ \rho \frac{7\sqrt{n}p}{32\sqrt{d}} }^2/2}{np \opnorminf{\Delta^{(j)}}^2  +  \rho \frac{7\sqrt{n}p}{32\sqrt{d}} \opnorminf{\Delta^{(j)}} /3}}\\
% {\color{red} \text{above inequaltiy replaced by} }&{\color{red}   \leq 2d\sup_{\Delta^{(j)}\in\mathcal{Y}}  \ebr{-\frac{\br{ \rho \frac{7\sqrt{n}p}{32\sqrt{d}} }^2/2}{p\sqrt{nd} \opnorm{\Delta^{(j)}}\opnorminf{\Delta^{(j)}}  +  \rho \frac{7\sqrt{n}p}{32\sqrt{d}} \opnorminf{\Delta^{(j)}} /3}}  }\\
&\leq  2d\sup_{\Delta^{(j)}\in\mathcal{Y}}  \ebr{ - \frac{3}{8} \frac{ \rho \frac{7\sqrt{n}p}{64\sqrt{d}}}{ \opnorminf{\Delta^{(j)}}}}\\
&\leq 2d \ebr{-\frac{3}{512} \frac{\rho\sqrt{n}p}{\sqrt{d}\br{\frac{ C''}{\sqrt{n}}\br{\sqrt{\frac{\log n }{np}} + \frac{\sigma \sqrt{d}}{\sqrt{np}} + \frac{\sigma \sqrt{\log n}}{\sqrt{np}}} + \frac{1}{\sqrt{n}}}}},\numberthis \label{eqn:31}
\end{align*}
where in the first and third inequality, we have used (\ref{eqn:rho_2}) and the upper bound on $\opnorm{\Delta^{(j)}}$ per $\mathcal{Y}$, and in the last inequality, we have used the upper bound on $\opnorminf{\Delta^{(j)}}$ per $\mathcal{Y}$.

\textbf{For the second term in  (\ref{eqn:27})}, by (\ref{eqn:3}), we have
\begin{align*}
&\E\indic{4\sqrt{2dh_R} \opnorm{F_{j2}}  \geq \rho \frac{h_R}{\sqrt{n}}}\indic{\mathf} \\
&\leq  \E\indic{ 4 \sigma \sqrt{2dh_R} \opnorm{\sum_{k\neq j} A_{jk} W_{jk}\Delta^{(j)}_k} \opnorm{\Lambda^{-1}}  \geq \rho \frac{h_R}{\sqrt{n}}}\indic{\mathf}\\
&\leq  \E\indic{\frac{32 \sigma \sqrt{2dh_R} }{7np}\opnorm{\sum_{k\neq j} A_{jk} W_{jk}\Delta^{(j)}_k}  \geq \rho \frac{h_R}{\sqrt{n}}}\indic{\mathf}\\
&\leq  \E\indic{\frac{32 \sigma \sqrt{2dh_R} }{7np}\opnorm{\sum_{k\neq j} A_{jk} W_{jk}\Delta^{(j)}_k}  \geq \rho \frac{h_R}{\sqrt{n}}}\indic{\mathf_j}.
\end{align*}
By the independence among $\{A_{jk}\}_{k\neq j}$, $\{W_{jk}\}_{k\neq j}$, and $\Delta^{(j)}$, following the same argument as used in (\ref{eqn:28}),  we have
\begin{align*}
& \E\indic{4\sqrt{2dh_R} \opnorm{F_{j2}}  \geq \rho \frac{h_R}{\sqrt{n}}}\indic{\mathf} \\
& = \E \br{\pbr{\frac{32 \sigma \sqrt{2dh_R} }{7np}\opnorm{\sum_{k\neq j} A_{jk} W_{jk}\Delta^{(j)}_k}  \geq \rho \frac{h_R}{\sqrt{n}},\mathf_j \Bigg|  \{A_{jk}\}_{k\neq j}, \Delta^{(j)}}} \\
& =  \E \br{ \pbr{\frac{32 \sigma \sqrt{2dh_R} }{7np}\opnorm{\sum_{k\neq j} A_{jk} W_{jk}\Delta^{(j)}_k}  \geq \rho \frac{h_R}{\sqrt{n}} \Bigg|  \{A_{jk}\}_{k\neq j}, \Delta^{(j)}} \indic{\Delta^{(j)}\in\mathf_j}}\\
&\leq \E\br{ \pbr{\frac{32 \sigma \sqrt{2dh_R} }{7np}\opnorm{\sum_{k\neq j} A_{jk} W_{jk}\Delta^{(j)}_k}  \geq \rho \frac{h_R}{\sqrt{n}} \Bigg| \{A_{jk}\}_{k\neq j}, \Delta^{(j)}} \indic{\Delta^{(j)}\in\mathf_j} }\\
&\leq  \E\br{ \pbr{ \opnorm{\sum_{k\neq j} A_{jk} W_{jk}\Delta^{(j)}_k}  \geq \rho \frac{7 \sqrt{n}p}{32\sigma\sqrt{d}} \Bigg| \{A_{jk}\}_{k\neq j},\Delta^{(j)} } \indic{\Delta^{(j)}\in\mathf_j}  }\\
&\leq  \E \left( \br{ 2\ebr{- \frac{c\br{\rho \frac{7 \sqrt{n}p}{32\sigma\sqrt{d}} }^2}{\opnorm{ \sum_{k\neq j}A_{jk}(\Delta^{(j)}_k)^\top \Delta^{(j)}_k}}}   +  \indic{\frac{\rho \frac{7 \sqrt{n}p}{32\sigma\sqrt{d}}}{\opnorm{ \sum_{k\neq j}A_{jk}(\Delta^{(j)}_k)^\top \Delta^{(j)}_k}^\frac{1}{2}} <4\sqrt{d}} }\indic{\Delta^{(j)}\in\mathf_j} \right),
\end{align*}
for some constant $c>0$, 
where the second to last inequality is by the fact that ${h_R}\geq 2$ and the last inequality is by Lemma \ref{lem:Fj2}. Since 
\begin{align*}
\ebr{- \frac{c\br{\rho \frac{7 \sqrt{n}p}{32\sigma\sqrt{d}} }^2}{\opnorm{ \sum_{k\neq j}A_{jk}(\Delta^{(j)}_k)^\top \Delta^{(j)}_k}}} &\leq \ebr{-\frac{np}{\sigma^2}} + \indic{\frac{c\br{\rho \frac{7 \sqrt{n}p}{32\sigma\sqrt{d}} }^2}{\opnorm{ \sum_{k\neq j}A_{jk}(\Delta^{(j)}_k)^\top \Delta^{(j)}_k}} \leq \frac{np}{\sigma^2}},
%&\leq \ebr{-\frac{np}{\sigma^2}} + \indic{ \opnorm{ \sum_{k\neq j}A_{jk}(\Delta^{(j)}_k)^\top \Delta^{(j)}_k} \geq \frac{c^2 \rho^2p	}{16 d} },
\end{align*}
we have
\begin{align*}
& \E\indic{2\sqrt{2dh_R} \opnorm{F_{j2}}  \geq \rho \frac{h_R}{\sqrt{n}}}\indic{\mathf} \\
&\leq  \E \left(  \br{2\ebr{-\frac{np}{\sigma^2}} +  3\indic{\frac{\br{\rho \frac{7 \sqrt{n}p}{32\sigma\sqrt{d}} }^2}{\opnorm{ \sum_{k\neq j}A_{jk}(\Delta^{(j)}_k)^\top \Delta^{(j)}_k}} \leq \frac{np}{c\sigma^2} \wedge 16d} }\indic{\Delta^{(j)}\in\mathf_j} \right)\\
&\leq   2\ebr{-\frac{np}{\sigma^2}} +  3 \E  \indic{\frac{\br{\rho \frac{7 \sqrt{n}p}{32\sigma\sqrt{d}} }^2}{\opnorm{ \sum_{k\neq j}A_{jk}(\Delta^{(j)}_k)^\top \Delta^{(j)}_k}} \leq \frac{np}{c\sigma^2} }\indic{\Delta^{(j)}\in\mathf_j}  ,
\end{align*}
where the last inequality holds as long as $\frac{np}{\sigma^2d}\geq 16c$. Then characterizing the above display is about controlling the tail probabilities of $\opnormt{ \sum_{k\neq j}A_{jk}(\Delta^{(j)}_k)^\top \Delta^{(j)}_k}$. Similar to the establishment of (\ref{eqn:28}), we have
\begin{align*}
& \E\indic{4\sqrt{2dh_R} \opnorm{F_{j2}}  \geq \rho \frac{h_R}{\sqrt{n}}}\indic{\mathf} \\
&\leq   2\ebr{-\frac{np}{\sigma^2}} +  3 \sup_{\Delta^{(j)}\in \mathcal{Y}} \pbr{\opnorm{ \sum_{k\neq j}A_{jk}(\Delta^{(j)}_k)^\top \Delta^{(j)}_k} \geq  \frac{7^2\rho^2 cp}{32^2d}\Bigg|  \Delta^{(j)}}.
\end{align*}
Assume $\rho$ satisfies
\begin{align}
& \frac{1}{2}\frac{7^2\rho^2 cp}{32^2d} \geq p\br{ \frac{9C_0(1+\sigma\sqrt{d})}{\sqrt{np}}}^2. \label{eqn:rho_3} 
\end{align}
By Lemma \ref{lem:Fj2_denominator}, we have
\begin{align*}
& \E\indic{4\sqrt{2dh_R} \opnorm{F_{j2}}  \geq \rho \frac{h_R}{\sqrt{n}}}\indic{\mathf} \\
&\leq   2\ebr{-\frac{np}{\sigma^2}} + 6d   \sup_{\Delta^{(j)}\in \mathcal{Y}} \ebr{\opnorm{ \sum_{k\neq j}A_{jk}(\Delta^{(j)}_k)^\top \Delta^{(j)}_k} \geq  p \opnorm{\Delta^{(j)}}^2 + \frac{1}{2}\frac{7^2\rho^2 cp}{32^2d} }\\
&\leq   2\ebr{-\frac{np}{\sigma^2}} + 6d   \sup_{\Delta^{(j)}\in \mathcal{Y}}  \ebr{ - \frac{\br{ \frac{1}{2}\frac{7^2\rho^2 cp}{32^2d} }^2/2}{ p \opnorminf{\Delta^{(j)}}^2\opnorm{\Delta^{(j)}}^2 + \opnorminf{\Delta^{(j)}}^2  \frac{1}{2}\frac{7^2\rho^2 cp}{16^2d} /3}}\\
&\leq   2\ebr{-\frac{np}{\sigma^2}} + 6d   \sup_{\Delta^{(j)}\in \mathcal{Y}}  \ebr{-\frac{3}{8} \frac{ \frac{1}{2}\frac{7^2\rho^2 cp}{32^2d}}{\opnorminf{\Delta^{(j)}}^2} } \\
&\leq   2\ebr{-\frac{np}{\sigma^2}} + 6d \ebr{-\frac{3}{8} \frac{ \frac{1}{2}\frac{7^2\rho^2 cp}{32^2d}}{7^2\br{\frac{ C''}{\sqrt{n}}\br{\sqrt{\frac{\log n }{np}} + \frac{\sigma \sqrt{d}}{\sqrt{np}} + \frac{\sigma \sqrt{\log n}}{\sqrt{np}}} + \frac{1}{\sqrt{n}}}^2} }\\
&\leq   2\ebr{-\frac{np}{\sigma^2}} + 6d \ebr{ -\frac{3c }{16\times 32^2} \br{ \frac{\rho \sqrt{p}}{\sqrt{d} \br{\frac{ C''}{\sqrt{n}}\br{\sqrt{\frac{\log n }{np}} + \frac{\sigma \sqrt{d}}{\sqrt{np}} + \frac{\sigma \sqrt{\log n}}{\sqrt{np}}} + \frac{1}{\sqrt{n}}}}}^2}.\numberthis \label{eqn:32}
\end{align*}

\textbf{For the third term in (\ref{eqn:27})}, using (\ref{eqn:3}) and the fact that $h_R\geq 2$, we have
\begin{align*}
& \E \indic{ 46C_0\br{\frac{1+\sigma \sqrt{d}}{\sqrt{np}}} \sqrt{2dh_R} \opnorm{G_{j2}} \geq  \rho \frac{h_R}{\sqrt{n}}}\indic{\mathf}  \\
&\leq \E \indic{ 46C_0\br{\frac{1+\sigma \sqrt{d}}{\sqrt{np}}} \sqrt{2dh_R} \frac{\sigma}{\sqrt{n}} \opnorm{\Lambda^{-1}}\opnorm{\sum_{k\neq j} A_{jk}W_{jk}} \geq  \rho \frac{h_R}{\sqrt{n}}}\\
 & \leq \E \indic{\opnorm{\sum_{k\neq j} A_{jk}W_{jk}} \geq  \frac{7}{8\times 46C_0\sqrt d\br{\frac{1+\sigma \sqrt{d}}{\sqrt{np}}}  } \frac{\rho np}{\sigma} }\\
 & = \E \br{\pbr{\opnorm{\sum_{k\neq j} A_{jk}W_{jk}} \geq  \frac{7}{8\times 46C_0\sqrt d\br{\frac{1+\sigma \sqrt{d}}{\sqrt{np}}}  }\frac{\rho np}{\sigma} \Bigg| \{A_{jk}\}_{k\neq j}}   }.
\end{align*}
%Since $\frac{np}{(\sqrt{d} + \sigma d)^2} \geq 22^2 C_0^2$, we have $\frac{7}{8\times 23C_0\br{\frac{1+\sigma \sqrt{d}}{\sqrt{np}}}  }  \geq 0.8\sqrt{d}$. Assume $\frac{np}{\sigma}$
By  the independence between $\{A_{jk}\}_{k\neq j}$ and $\{W_{jk}\}_{k\neq j}$, we have $\sum_{k\neq j} A_{jk}W_{jk} | \{A_{jk}\}_{k\neq j} \dequal \sqrt{\{A_{jk}\}_{k\neq j}} \xi$ for some $\xi \sim \MN(\zero,I_d,I_d)$ that is independent of $\{A_{jk}\}_{k\neq j}$
Using Lemma \ref{lem:vershynin_new}, we have 
\begin{align*}
& \E \indic{ 46C_0\br{\frac{1+\sigma \sqrt{d}}{\sqrt{np}}} \sqrt{2dh_R} \opnorm{G_{j2}} \geq  \rho \frac{h_R}{\sqrt{n}}}\indic{\mathf}  \\
&\leq \E \Bigg(  2\ebr{-c \br{\frac{1}{2}  \frac{7}{8\times 46C_0\sqrt d\br{\frac{1+\sigma \sqrt{d}}{\sqrt{np}}}}\frac{\rho np}{\sigma } }^2 \frac{1}{{\sum_{k\neq j}A_{jk}}} }  \\
&\quad +   \indic{\frac{1}{2}  \frac{7}{8\times 46C_0\sqrt d\br{\frac{1+\sigma \sqrt{d}}{\sqrt{np}}}  }\frac{\rho np}{\sigma} \frac{1}{\sqrt{\sum_{k\neq j}A_{jk}}} < 2\sqrt{d} }\Bigg).
\end{align*}
%$\sum_{k\neq j} A_{jk}W_{jk} \dequal  \br{\sum_{k\neq j} A_{jk}} W_{j1}$
Note that by Bernstein's inequality, there exists some constant $C_3>0$ such that
\begin{align}\label{eqn:29}
\pbr{\sum_{k\neq j} A_{jk} \geq np + C_3\sqrt{np\log n}}\leq n^{-10}.
\end{align}
We further assume $\frac{np}{\log n}\geq C_3^2$ such that $2np\geq np + C_3\sqrt{np\log n}$ and consequently $\pbr{\sum_{k\neq j} A_{jk} \geq 2np}\leq n^{-10}$.
Assume $\rho$ satisfies
\begin{align}
\frac{1}{ \sqrt{2np}}\br{\frac{1}{2} \frac{7}{8\times 46C_0\sqrt d\br{\frac{1+\sigma \sqrt{d}}{\sqrt{np}}}  }\frac{\rho np}{\sigma}} \geq c^{-\frac{1}{2}} \vee \br{2\sqrt{d}}.\label{eqn:rho_4} 
\end{align}
We have 
\begin{align*}
 \E \indic{ 46C_0\br{\frac{1+\sigma \sqrt{d}}{\sqrt{np}}} \sqrt{2dh_R} \opnorm{G_{j2}} \geq  \rho \frac{h_R}{\sqrt{n}}}\indic{\mathf}  &\leq 2\ebr{-\frac{np}{\sigma^2}} + 2\pbr{\sum_{k\neq j} A_{jk} \geq 2np}\\
&\leq   2\ebr{-\frac{np}{\sigma^2}} + 2n^{-10}.\numberthis \label{eqn:33}
\end{align*}


\textbf{For the fourth and last term in (\ref{eqn:27})}, by properties of the Gaussian distribution, one can verify that $\iprod{\sum_{k\neq j} A_{jk} W_{jk} }{R-I_d} \Big | \{A_{jk}\}_{k\neq j}  \sim \mathn(0,4\sum_{k\neq j} A_{jk}) $. Assume $\rho$ satisfies
\begin{align}\label{eqn:rho_5} 
3\rho + C_2\sqrt d\br{\frac{\sqrt{\log n}+ \sigma\sqrt{d}}{\sqrt{np}}} \leq  \frac{1}{2}.
\end{align}
By the fact $h_R\geq 2$, we then have
\begin{align*}
& \E \indic{ \frac{\sigma}{\sqrt{n}}\frac{1}{np}  \iprod{\sum_{k\neq j} A_{jk} W_{jk} }{R-I_d} \geq \br{1- 3\rho - C_2\sqrt d\br{\frac{\sqrt{\log n}+ \sigma\sqrt{d}}{\sqrt{np}}}}     \frac{h_R}{\sqrt{n}} }\indic{\mathf} \\
&\leq  \E \indic{  \iprod{\sum_{k\neq j} A_{jk} W_{jk} }{R-I_d} \geq \br{1- 3\rho - C_2\sqrt d\br{\frac{\sqrt{\log n}+ \sigma\sqrt{d}}{\sqrt{np}}}}    \frac{2np}{\sigma} }\\
& =\E \br{\pbr{  \iprod{\sum_{k\neq j} A_{jk} W_{jk} }{R-I_d} \geq \br{1- 3\rho - C_2\sqrt d\br{\frac{\sqrt{\log n}+ \sigma\sqrt{d}}{\sqrt{np}}}}    \frac{2np}{\sigma} \Bigg |\{A_{jk}\}_{k\neq j}}}\\
&\leq  \E \ebr{- \br{1- 3\rho - C_2\sqrt d\br{\frac{\sqrt{\log n}+ \sigma\sqrt{d}}{\sqrt{np}}}}_+^2 \frac{(np)^2}{2\sigma^2 \sum_{k\neq j} A_{jk}}}\\
&\leq  \ebr{- \br{1- 3\rho - C_2\sqrt d\br{\frac{\sqrt{\log n}+ \sigma\sqrt{d}}{\sqrt{np}}}} _+^2 \frac{(np)^2}{2\sigma^2 \br{np+C_3\sqrt{np\log n}  }}} \\
&\quad + \pbr{\sum_{k\neq j}A_{jk}\geq np+C_0\sqrt{np\log n} }\\
&\leq  \ebr{- \br{1- 3\rho - C_2\sqrt d\br{\frac{\sqrt{\log n}+ \sigma\sqrt{d}}{\sqrt{np}}}} _+^2 \br{1+C_3\sqrt{\frac{\log n}{np}}}^{-1} \frac{np}{2\sigma^2}} + n^{-10}, \numberthis \label{eqn:34}
\end{align*}
where the last inequality is by (\ref{eqn:29}).

\textbf{Obtaining an upper bound on (\ref{eqn:27}).} Plugging (\ref{eqn:31}), (\ref{eqn:32}), (\ref{eqn:33}),  and (\ref{eqn:34}) into (\ref{eqn:27}), we have
\begin{align*}
 &\E\indic{ \fnorm{U_j \hatH^\top - U_j^*\hat P^\top} \geq \fnorm{U_j \hatH^\top - \frac{R \hat P^\top}{\sqrt n} }   }  \indic{\mathf} \\
 &\leq  2d \ebr{-\frac{3}{512} \frac{\rho\sqrt{n}p}{\sqrt{d}\br{\frac{ C''}{\sqrt{n}}\br{\sqrt{\frac{\log n }{np}} + \frac{\sigma \sqrt{d}}{\sqrt{np}} + \frac{\sigma \sqrt{\log n}}{\sqrt{np}}} + \frac{1}{\sqrt{n}}}}} \\
 &\quad +  2\ebr{-\frac{np}{\sigma^2}} + 6d \ebr{ -\frac{3c }{16\times 32^2} \br{ \frac{\rho \sqrt{p}}{\sqrt{d} \br{\frac{ C''}{\sqrt{n}}\br{\sqrt{\frac{\log n }{np}} + \frac{\sigma \sqrt{d}}{\sqrt{np}} + \frac{\sigma \sqrt{\log n}}{\sqrt{np}}} + \frac{1}{\sqrt{n}}}}}^2}\\
 &\quad+ 2\ebr{-\frac{np}{\sigma^2}} + 2n^{-10} \\
 &\quad +  \ebr{- \br{1- 3\rho - C_2\sqrt d\br{\frac{\sqrt{\log n}+ \sigma\sqrt{d}}{\sqrt{np}}}}_+^2 \br{1+C_3\sqrt{\frac{\log n}{np}}}^{-1} \frac{np}{2\sigma^2}} + n^{-10}\\
 &\leq  \ebr{- \br{1- 3\rho - C_2\sqrt d\br{\frac{\sqrt{\log n}+ \sigma\sqrt{d}}{\sqrt{np}}}}_+^2 \br{1+C_3\sqrt{\frac{\log n}{np}}}^{-1} \frac{np}{2\sigma^2}} + 3n^{-10} \\
 &\quad + 4\ebr{-\frac{np}{\sigma^2}} + 2d \ebr{-\frac{3}{512} \frac{\rho\sqrt{n}p}{\sqrt{d}\br{\frac{ C''}{\sqrt{n}}\br{\sqrt{\frac{\log n }{np}} + \frac{\sigma \sqrt{d}}{\sqrt{np}} + \frac{\sigma \sqrt{\log n}}{\sqrt{np}}} + \frac{1}{\sqrt{n}}}}} \\
 &\quad + 6d \ebr{ -\frac{3c }{16\times 32^2} \br{ \frac{\rho \sqrt{p}}{\sqrt{d} \br{\frac{ C''}{\sqrt{n}}\br{\sqrt{\frac{\log n }{np}} + \frac{\sigma \sqrt{d}}{\sqrt{np}} + \frac{\sigma \sqrt{\log n}}{\sqrt{np}}} + \frac{1}{\sqrt{n}}}}}^2}.
\end{align*}
We let $\rho = \br{\frac{\sigma^2}{np}}^\frac{1}{4} + \br{\frac{1}{\log n}}^\frac{1}{4}$. Then, under the assumption that $d$ is a constant, $\frac{np}{\log^3 n},\frac{np}{\sigma^2}\geq C_3$ for some large enough constant $C_3>0$, we can make sure conditions (\ref{eqn:rho_2}), (\ref{eqn:rho_3}), (\ref{eqn:rho_4}), and (\ref{eqn:rho_5}) are all satisfied. In addition, we have
\begin{align*}
&\frac{1}{\log n} \frac{\rho\sqrt{n}p}{\sqrt{d}\br{\frac{ C''}{\sqrt{n}}\br{\sqrt{\frac{\log n }{np}} + \frac{\sigma \sqrt{d}}{\sqrt{np}} + \frac{\sigma \sqrt{\log n}}{\sqrt{np}}} + \frac{1}{\sqrt{n}}}} \\
&\geq  \frac{\rho np}{\sqrt{d}(C''\vee 1) (\log n) \br{1+ \sqrt{\frac{\log n }{np}} + \frac{\sigma \sqrt{d}}{\sqrt{np}} + \frac{\sigma \sqrt{\log n}}{\sqrt{np}}}} \\
&\geq \frac{C_3\br{\frac{1}{\log n}}^\frac{1}{4}  \log^3 n}{\sqrt{d}(C''\vee 1) (\log n) \br{1+ \sqrt{\frac{\log n }{np}} + \frac{\sigma \sqrt{d}}{\sqrt{np}} + \frac{\sigma \sqrt{\log n}}{\sqrt{np}}}} \\
& =\frac{C_3 (\log n)^{1.25}}{\sqrt{d}(C''\vee 1)\br{ \frac{1}{\sqrt{\log n}} \br{1+ \sqrt{\frac{\log n }{np}} + \frac{\sigma \sqrt{d}}{\sqrt{np}}} + \frac{\sigma}{\sqrt{np}}}}.
\end{align*}
Similarly, we have
\begin{align*}
&\frac{1}{\sqrt{\log n}}\frac{\rho \sqrt{p}}{\sqrt{d} \br{\frac{ C''}{\sqrt{n}}\br{\sqrt{\frac{\log n }{np}} + \frac{\sigma \sqrt{d}}{\sqrt{np}} + \frac{\sigma \sqrt{\log n}}{\sqrt{np}}} + \frac{1}{\sqrt{n}}}}\\
&\geq \frac{\sqrt{C_3} (\log n)^{0.25}}{\sqrt{d}(C''\wedge1) \br{ \frac{1}{\sqrt{\log n}} \br{1+ \sqrt{\frac{\log n }{np}} + \frac{\sigma \sqrt{d}}{\sqrt{np}}} + \frac{\sigma}{\sqrt{np}}}}.
\end{align*}

\textbf{Putting things together.} We have
\begin{align*}
&\E\indic{ \fnorm{U_j \hatH^\top - U_j^*\hat P^\top} \geq \fnorm{U_j \hatH^\top - \frac{R \hat P^\top}{\sqrt n} }   }  \indic{\mathf} \\
 &\leq  \ebr{- \br{1- 3\br{\br{\frac{\sigma^2}{np}}^\frac{1}{4} + \br{\frac{1}{\log n}}^\frac{1}{4}} - C_2\sqrt d\br{\frac{\sqrt{\log n}+ \sigma\sqrt{d}}{\sqrt{np}}}}_+^2 \br{1+C_3\sqrt{\frac{\log n}{np}}}^{-1} \frac{np}{2\sigma^2}} + 3n^{-10} \\
 &\quad + 4\ebr{-\frac{np}{\sigma^2}} + 2d \ebr{-\frac{3 \log n}{512} \frac{C_3 (\log n)^{1.25}}{\sqrt{d}(C''\vee 1)\br{ \frac{1}{\sqrt{\log n}} \br{1+ \sqrt{\frac{\log n }{np}} + \frac{\sigma \sqrt{d}}{\sqrt{np}}} + \frac{\sigma}{\sqrt{np}}}}} \\
 &\quad + 6d \ebr{ -\frac{3c  \log n}{16\times 32^2} \br{\frac{\sqrt{C_3} (\log n)^{0.25}}{\sqrt{d}(C''\wedge1) \br{ \frac{1}{\sqrt{\log n}} \br{1+ \sqrt{\frac{\log n }{np}} + \frac{\sigma \sqrt{d}}{\sqrt{np}}} + \frac{\sigma}{\sqrt{np}}}}} ^2}\\
&\leq   \ebr{- \br{1- 3\br{\br{\frac{\sigma^2}{np}}^\frac{1}{4} + \br{\frac{1}{\log n}}^\frac{1}{4}} - C_2\sqrt d\br{\frac{\sqrt{\log n}+ \sigma\sqrt{d}}{\sqrt{np}}}}_+^2 \br{1+C_3\sqrt{\frac{\log n}{np}}}^{-1} \frac{np}{2\sigma^2}} + 3n^{-10} \\
&\quad +  4\ebr{-\frac{np}{\sigma^2}} + 9dn^{-10},
\end{align*}
where the last inequality holds when $\frac{np}{\log n}, \frac{np}{\sigma^2}$ are greater than a sufficiently large constant factor. Note that the term $9dn^{-10}$ comes from combining the two $d\ebr{.}$ terms and assuming that $\log n$ is greater than a sufficiently large constant factor. Then there exists a constant $C_4>0$ such that
\begin{align*}
&\E\indic{ \fnorm{U_j \hatH^\top - U_j^*\hat P^\top} \geq \fnorm{U_j \hatH^\top - \frac{R \hat P^\top}{\sqrt n} }   }  \indic{\mathf} \\
&\leq \ebr{-\br{1-C_4\br{\br{\frac{\sigma^2}{np}}^\frac{1}{4} + \br{\frac{1}{\log n}}^\frac{1}{4} }}_+\frac{np}{2\sigma^2}} + 12dn^{-10}. \numberthis\label{eqn:tail-bound}
%&\leq \ebr{-\br{1-C_4\br{\br{\frac{\sigma^2}{np}}^\frac{1}{4} + \br{\frac{1}{\log n}}^\frac{1}{4} }}\frac{np}{2\sigma^2}} + 12dn^{-10},
\end{align*}

Combining (\ref{eqn:36}), (\ref{eqn:35}), (\ref{eqn:26}) and (\ref{eqn:tail-bound}) we have
\begin{align*}
\E \ell(\hat Z,Z^*) &\leq \frac{1}{n}\sum_{j\in[n]}\sum_{R\in \Pi_d:R\neq I_d} \br{\ebr{-\br{1-C_4\br{\br{\frac{\sigma^2}{np}}^\frac{1}{4} + \br{\frac{1}{\log n}}^\frac{1}{4} }}_+\frac{np}{2\sigma^2}} + 12dn^{-10}} + \pbr{\mathf^\mathrm{c}}\\
&\leq d^d \br{\ebr{-\br{1-C_4\br{\br{\frac{\sigma^2}{np}}^\frac{1}{4} + \br{\frac{1}{\log n}}^\frac{1}{4} }}_+\frac{np}{2\sigma^2}} + 12dn^{-10}} + 7dn^{-9}\\
&\leq \ebr{-\br{1-C_5\br{\br{\frac{\sigma^2}{np}}^\frac{1}{4} + \br{\frac{1}{\log n}}^\frac{1}{4} }}_+\frac{np}{2\sigma^2}}  + 8dn^{-9}\\
&\leq \ebr{-\br{1-C_5\br{\br{\frac{\sigma^2}{np}}^\frac{1}{4} + \br{\frac{1}{\log n}}^\frac{1}{4} }}\frac{np}{2\sigma^2}}  + 8dn^{-9},
\end{align*}
for some constant $C_5>0$, where the last inequality holds when $\frac{np}{\sigma^2}$ is large enough. 
%\begin{align*}
%&\frac{1}{\log n}\frac{\rho \sqrt{p}}{\sqrt{d} \br{\frac{ C''}{\sqrt{n}}\br{\sqrt{\frac{\log n }{np}} + \frac{\sigma \sqrt{d}}{\sqrt{np}} + \frac{\sigma \sqrt{\log n}}{\sqrt{np}}} + \frac{1}{\sqrt{n}}}}\\
%& \geq \frac{\rho \sqrt{np}}{\sqrt{d}(\log n) (C''\wedge 1)\br{1+ \sqrt{\frac{\log n }{np}} + \frac{\sigma \sqrt{d}}{\sqrt{np}} + \frac{\sigma \sqrt{\log n}}{\sqrt{np}}}}\\
%&\geq \frac{\sqrt{C_3}(\log n)^{\frac{3}{4}}}{\sqrt{d}(C''\wedge1) \br{1+ \sqrt{\frac{\log n }{np}} + \frac{\sigma \sqrt{d}}{\sqrt{np}} + \frac{\sigma \sqrt{\log n}}{\sqrt{np}}}}  \\
%&\geq  \frac{\sqrt{C_3}(\log n)^{\frac{1}{4}}}{\sqrt{d}(C''\wedge1) \br{ \frac{1}{\sqrt{\log n}} \br{1+ \sqrt{\frac{\log n }{np}} + \frac{\sigma \sqrt{d}}{\sqrt{np}}} + \frac{\sigma}{\sqrt{np}}}}. 
%\end{align*}


%\begin{align*}
%& \E\indic{2\sqrt{2dh_R} \opnorm{F_{j2}}  \geq \rho \frac{h_R}{\sqrt{n}}}\indic{\mathf} \\
%&\leq  \E \br{ \pbr{\frac{16 \sigma \sqrt{2dh_R} }{7np}\opnorm{\sum_{k\neq j} A_{jk} W_{jk}\Delta^{(j)}_k}  \geq \rho \frac{h_R}{\sqrt{n}} \Bigg| \{A_{jk}\}_{k\neq j}, \mathf_j}\pbr{\mathf_j}}\\
%&\leq  \E \br{ \sup_{\Delta^{(j)}\in\mathcal{Y}} \pbr{\frac{16 \sigma \sqrt{2dh_R} }{7np}\opnorm{\sum_{k\neq j} A_{jk} W_{jk}\Delta^{(j)}_k}  \geq \rho \frac{h_R}{\sqrt{n}} \Bigg| \{A_{jk}\}_{k\neq j},\Delta^{(j)}}}\\
%&\leq \E \br{ \sup_{\Delta^{(j)}\in\mathcal{Y}} \pbr{\opnorm{\sum_{k\neq j} A_{jk} W_{jk}\Delta^{(j)}_k}  \geq \rho \frac{7 \sqrt{n}p}{16\sigma\sqrt{d}} \Bigg| \{A_{jk}\}_{k\neq j},\Delta^{(j)}}}\\
%&\leq   \E \br{ \sup_{\Delta^{(j)}\in\mathcal{Y}} \br{ 2\ebr{- \frac{c\br{\rho \frac{7 \sqrt{n}p}{16\sigma\sqrt{d}} }^2}{\opnorm{ \sum_{k\neq j}A_{jk}(\Delta^{(j)}_k)^\top \Delta^{(j)}_k}}} +  \indic{\rho \frac{7 \sqrt{n}p}{16\sigma\sqrt{d}} <4\sqrt{d}\opnorm{ \sum_{k\neq j}A_{jk}(\Delta^{(j)}_k)^\top \Delta^{(j)}_k}^\frac{1}{2}}}}
%\end{align*}
%for some constant $c>0$, 
%where the last inequality is due to Lemma \ref{lem:Fj2}.

     
%\begin{align*}   
%     &\leq \indic{\iprod{U_j H - U_j^*}{R-I_d} + \fnorm{U_j \br{M - PH^\top}^\top}\fnorm{ R-I_d} \geq  \frac{h_R}{\sqrt{n}}}\\
%     &\leq   \indic{\iprod{U_j H - U_j^*}{R-I_d} + \opnorm{U_j} \fnorm{M - PH^\top}\fnorm{ R-I_d} \geq  \frac{h_R}{\sqrt{n}}}\\
%\end{align*}





\subsection{Lemmas}
\begin{lemma}\label{lem:population}
We have $\lambda_i(\E A\otimes I_d)=(n-1)p$ for all $i\leq d$ and $\lambda_i(\E A\otimes I_d)=-p$ for all $d+1\leq i\leq nd$. 
%For any $j\in[n]$, we have $U^*_j =I_d/\sqrt{n}$.
%In addition, there exists some $O\in O(d)$ such that  $U^*_j = O/\sqrt{n}$ for all $j\in[n]$. As a result, $\opnorm{U^*_j}=1$ for all $j\in[n]$.
\end{lemma}
\begin{proof} Recall that $\E A = pJ_n - pI_n$. The eigenvalues of the matrix $\E A$ are characterized as follows.
$$ \lambda_1(\E A) = (n-1)p \,,$$
$$\lambda_2(\E A) = \ldots = \lambda_n(\E A) = -p\,. $$
Therefore the eigenvalues of $\E A \otimes I_d$ are as follows.
$$ \lambda_1(\E A\otimes I_d) = \ldots = \lambda_d(\E A \otimes I_d) = (n-1)p \,,$$
$$\lambda_{d+1}(\E A\otimes I_d) = \ldots = \lambda_{dn}(\E A \otimes I_d) = -p\,. $$
%The conclusion about $\{U^*_j\}$ can be easily verified and its proof is omitted.
\end{proof}

%\begin{lemma}\label{lem:degree}
%Assume $p\geq 1/n$. Then
%\begin{align*}
%\max_{j\in[n]}\abs{\sum_{k\neq j}A_{jk}-np}\leq \sqrt{np\log n}
%\end{align*}
%with probability at least $1-n^{-10}$.
%\end{lemma}
%\begin{proof}
%By Bernstein's inequality, we have 
%\end{proof}


\begin{lemma}\label{lem:event-E-prime_new} 
There exist constants $C_0'>0, C_0 > 7$ such that if $\frac{np}{\log n} > C_0'$, then the following event holds
\begin{align*}
 \calE &= \left\{ \opnorm{A - \E A} \leq C_0\sqrt{np} ,\;   \opnorm{(A \otimes J_d)\circ W} \leq C_0\sqrt{npd},\; \max_{j\in[n]}\abs{\sum_{k\neq j}A_{jk}-np}\leq C_0\sqrt{np\log n}  \right\}\,
\end{align*}
with probability at least $1- n^{-10}$. Under the event $\calE$, we have
\begin{align}
\opnorm{X - (\E A \otimes J_d)} ,\max_{i\in[d]} \abs{\lambda_i(X) - (n-1)p} ,\max_{d+1\leq i \leq nd} \abs{\lambda_i(X)+p} &\leq  C_0\br{1+\sigma\sqrt{d}}\sqrt{np},\label{eqn:2}\\
\max_{j\in[n]} \opnorm{X_j} \leq p\sqrt{n} + C_0\br{1+\sigma\sqrt{d}}\sqrt{np}.\label{eqn:7}
\end{align}
If $\frac{np}{(1+\sigma \sqrt{d})^2 }\geq 64C_0^2$ is further assumed, under the event $\calE$, the following hold for $\Lambda, U$, and $H$:
\begin{align}
\opnorm{\Lambda} =\lambda_1(X) &\leq \frac{9np}{8},\label{eqn:8}\\
\opnorm{\Lambda^{-1}}^{-1} =\lambda_d(X) &\geq \frac{7np}{8},\label{eqn:3}\\
%\opnorm{\Lambda^{-1}} &\leq  \frac{4}{3np},\\
\lambda_d(X)- \lambda_{d+1}(X) &\geq \frac{3np}{4},\label{eqn:4}\\
 \min_{O\in \mathcal{O}_d}\opnorm{H - O}, \opnorm{UH-U^*} ,  \opnorm{UU^\top - U^* U^{*\top} } &\leq   \frac{8C_0\br{1+\sigma\sqrt{d}}}{7\sqrt{np}},\label{eqn:5}\\
 \opnorm{H\Lambda - \Lambda H} &\leq  2C_0\br{1+\sigma\sqrt{d}}\sqrt{np},\\
\opnorm{H^{-1}} &\leq \frac{4}{3},\label{eqn:6}
\end{align}
 the following hold any $j\in[n]$:
\begin{align}
\opnorm{G_{j1}} &\leq \frac{2C_0}{\sqrt{n}}\br{\sqrt{\frac{\log n}{np}} + \frac{\sigma \sqrt{d}}{\sqrt{np}}},\label{eqn:16}\\
 \opnorm{U_j} &\leq \frac{4}{3}\br{\opnorm{U_jH-U^*_j}+\frac{1}{\sqrt{n}}},\label{eqn:10} \\
 \opnorm{\loo{U}(\loo{U})^\top -UU^\top} &\leq 6\br{\opnorm{U_jH -U^*_j}+\frac{1}{\sqrt{n}}}, \label{eqn:14}\\
\opnorm{\loo{U}\loo{H} -U^*}&\leq \frac{9C_0(1+\sigma \sqrt{d})}{\sqrt{np}}. \label{eqn:12}
\end{align}
%\begin{align*}
%\calE' = \calE \cap \Big\{&\lambda_d(X) \geq \frac{3np}{4},\quad \lambda_d(X)- \lambda_{d+1}(X) \geq \frac{np}{2}, \\
%& \min_{O\in O(d)}\opnorm{H - O} \leq  \opnorm{UU^\top - U^* U^{*\top} }\leq   \frac{8C_0\br{1+\sigma\sqrt{d}}}{7\sqrt{np}}, \\
%&\opnorm{H\Lambda - \Lambda H}\leq  2C_0\br{1+\sigma\sqrt{d}}\sqrt{np},\\
%&\Big\}
%\end{align*}
%happens with probability at least $1- n^{-10}$.
\end{lemma}
\begin{proof}
The probability of $\calE$ is from
Lemma 9 of \cite{zhang2022exact} (for $\opnorm{A - \E A} $), Theorem 5.2 of \cite{lei2015consistency} (for $ \opnorm{(A \otimes J_d)\circ W} $), and Bernstein's inequality (for $\max_{j\in[n]}|\sum_{k\neq j}A_{jk}-np|$). Note that
\begin{align*}
\opnorm{X - (\E A \otimes J_d)}  & = \opnorm{ (A \otimes I_d) + \sigma (A \otimes J_d) \circ W - (\E A \otimes J_d)} \\
&\leq \opnorm{(A-\E A) \otimes I_d} +\sigma\opnorm{(A \otimes J_d) \circ W }\\
& = \opnorm{A -\E A } +\sigma\opnorm{(A \otimes J_d) \circ W }.
\end{align*}
By Weyl's inequality, we have $\max_{i\in[nd]}\abs{\lambda_i(X) - \lambda_i(\E A\otimes I_d)}\leq \opnorm{X - (\E A \otimes J_d)} $. Together with Lemma \ref{lem:population}, (\ref{eqn:2}) holds under the event $\calE$. For (\ref{eqn:7}), we have
\begin{align*}
\max_j\opnorm{X_j} &= \max_j \opnorm{  (\E A_j \otimes J_d)+ \br{X_j - (\E A_j \otimes J_d)}} \\
&\leq  \max_j \opnorm{  (\E A_j \otimes J_d)} + \opnorm{X - (\E A \otimes J_d)}\\
&\leq p\sqrt{n} + C_0\br{1+\sigma\sqrt{d}}\sqrt{np}.
\end{align*}

If $\frac{np}{(1+\sigma \sqrt{d})^2 }\geq 64C_0^2$ is further assumed, we have $C_0({1+\sigma\sqrt{d}})\sqrt{np} \leq np/8$. Then (\ref{eqn:2}), together with the fact $\opnorm{\Lambda^{-1}} = 1/\lambda_d(X)$ and $\opnorm{\Lambda} = \lambda_1(X)$, leads to (\ref{eqn:8})-(\ref{eqn:4}).
%
%Assume $\calE$ holds. Then
%\begin{align*}
%\opnorm{X - (\E A \otimes J_d)}  & = \opnorm{ (A \otimes I_d) + \sigma (A \otimes J_d) \circ W - (\E A \otimes J_d)} \\
%&\leq \opnorm{(A-\E A) \otimes I_d} +\sigma\opnorm{(A \otimes J_d) \circ W }\\
%& = \opnorm{A -\E A } +\sigma\opnorm{(A \otimes J_d) \circ W }\\
%&\leq C_0\br{1+\sigma\sqrt{d}}\sqrt{np}.
%\end{align*}
%Under the assumption  $\frac{np}{(1+\sigma \sqrt{d})^2 }\geq 64C_0^2$, we have $\opnorm{X - (\E A \otimes J_d)}  \leq np/8$. 
Since $\lambda_d(\E A \otimes J_d) - \lambda_{d+1}(\E A \otimes J_d) = np$ according to Lemma \ref{lem:population}, we have $\opnorm{X - (\E A \otimes J_d)}  \leq (\lambda_d(\E A \otimes J_d) - \lambda_{d+1}(\E A \otimes J_d))/8$.  
By Lemma 2 of \cite{abbe2020entrywise}, we have
\begin{align*}
\min_{O\in \mathcal{O}_d}\opnorm{H - O} \leq  \opnorm{UU^\top - U^* U^{*\top} }\leq \frac{\opnorm{X - (\E A \otimes J_d)} }{\br{1-\frac{1}{8}}np}\leq  \frac{8C_0\br{1+\sigma\sqrt{d}}}{7\sqrt{np}},
\end{align*}
and
\begin{align*}
\opnorm{H\Lambda - \Lambda H}\leq 2\opnorm{X - (\E A \otimes J_d)} \leq 2C_0\br{1+\sigma\sqrt{d}}\sqrt{np},
\end{align*}
and $\opnorm{H^{-1}}\leq 4/3$. Note that $\opnorm{UH-U^*} = \opnorm{UU^\top U^* - U^*U^{*\top}U^*} \leq  \opnorm{UU^\top  - U^*U^{*\top}}\opnorm{U^*}\leq \opnorm{UU^\top  - U^*U^{*\top}}$. Hence, (\ref{eqn:5})-(\ref{eqn:6}) hold.
%\begin{align*}
%\opnorm{H^{-1}}\leq 
%\end{align*}
%\begin{align*}
%\opnorm{U_j\br{H\Lambda - \Lambda H}} \leq \frac{8}{3}C_0\br{1+\sigma\sqrt{d}}\sqrt{np} \opnorm{U_j H}.
%\end{align*}

Consider any $j\in[n]$. By  (\ref{eqn:6}), we have 
\begin{align*}
\opnorm{U_j} &\leq \opnorm{U_j H}\opnorm{H^{-1}} \leq \br{\opnorm{U_jH-U^*_j} + \opnorm{U^*_j}}\opnorm{H^{-1}} = \frac{4}{3}\br{\opnorm{U_jH-U^*_j} +\frac{1}{\sqrt{n}} }.
\end{align*}
From (\ref{eqn:7}), (\ref{eqn:4}) and that $C_0 > 7$, we have $\opnorm{X-\loo{X}} \leq 2\opnorm{X_j} \leq (\lambda_d(X)- \lambda_{d+1}(X))/2$. 
By Davis-Kahan theorem, we have
\begin{align*}
\opnorm{\loo{U}(\loo{U})^\top -UU^\top}  &\leq \frac{2\opnorm{\br{X-\loo{X}}U}}{\lambda_d(X) - \lambda_{d+1}(X)} \\
&\leq \frac{2\br{\opnorm{X_j U} + \opnorm{X_j} \opnorm{U_j}}}{\lambda_d(X) - \lambda_{d+1}(X)}\\
& =   \frac{2\br{\opnorm{U_j\Lambda} + \opnorm{X_j} \opnorm{U_j}}}{\lambda_d(X) - \lambda_{d+1}(X)}\\
& \leq   \frac{2\br{\opnorm{U_j}\opnorm{\Lambda} + \opnorm{X_j} \opnorm{U_j}}}{\lambda_d(X) - \lambda_{d+1}(X)}\\
&\leq 4\opnorm{U_j}\\
&\leq  6\br{\opnorm{U_jH -U^*_j}+\frac{1}{\sqrt{n}}},
\end{align*}
where the second to last inequality is due to (\ref{eqn:7}), (\ref{eqn:8}), and (\ref{eqn:4}), and the last inequality is due to (\ref{eqn:10}). 
%Then by (\ref{eqn:6}) and Lemma \ref{lem:population}, we have
%\begin{align*}
%\opnorm{\loo{U}(\loo{U})^\top -UU^\top}  &\leq  4\opnorm{U_jH}\opnorm{H^{-1}}\\
%&\leq 6\br{\opnorm{U_jH -U^*_j}+\norm{U^*_j}}\\
%&= 6\br{\opnorm{U_jH -U^*_j}+\frac{1}{\sqrt{n}}}.
%\end{align*}
Together with (\ref{eqn:5}), we have
\begin{align*}
\opnorm{\loo{U}(\loo{U})^\top -UU^\top}&\leq   6\br{\opnorm{UH -U^*}+\frac{1}{\sqrt{n}}}\leq  \frac{7C_0(1+\sigma \sqrt{d})}{\sqrt{np}}.
\end{align*}
Then,
\begin{align*}
\opnorm{\loo{U}\loo{H} - U^*} &= \opnorm{\br{\loo{U}\loo{U}^\top - UU^\top}U^* + UH- U^*}  \\
&\leq \opnorm{\loo{U}\loo{U}^\top - UU^\top} + \opnorm{UH- U^*}\\
&\leq  \frac{9C_0(1+\sigma \sqrt{d})}{\sqrt{np}}.
\end{align*}
For $\opnorm{G_{j1}}$, we have
\begin{align*}
\sqrt{n}\opnorm{G_{j1}} &= \opnorm{\sum_{k\neq j} A_{jk} \Lambda^{-1}}\\
&\leq \opnorm{np \Lambda^{-1}-I_d} + \opnorm{ \br{\sum_{k\neq j} A_{jk} - np}\Lambda^{-1}}\\
&\leq \opnorm{np -\Lambda} \opnorm{\Lambda^{-1}} + \abs{\sum_{k\neq j} A_{jk} - np}\opnorm{\Lambda^{-1}}\\
&\leq \max_{i\in[d]} \abs{np - \lambda_i(X)}\opnorm{\Lambda^{-1}} + \abs{\sum_{k\neq j} A_{jk} - np} \opnorm{\Lambda^{-1}}\\
&\leq 2C_0\br{\sqrt{\frac{\log n}{np}} + \frac{\sigma \sqrt{d}}{\sqrt{np}}},
\end{align*}
where the last inequality is due to $\calE$,  (\ref{eqn:2}), and (\ref{eqn:3}).
\end{proof}




%To shorten notations, for each $j\in[n]$, denote
%\begin{align*}
%\Delta^{(j)} \define  U^{(j)}H^{(j)} - U^* \in\mathr^{nd\times d}
%\end{align*}
%such that its block submatrices $\Delta^{(j)}_k = U^{(j)}_kH^{(j)} - U^*_k$ for each $k\in[n]$. We further define
%\begin{align*}
%\opnorminf{\Delta^{(j)}}\define \max_{k\in[n]} \opnorm{\Delta^{(j)}_k}.
%\end{align*}

\begin{lemma}\label{lem:Fj1}
For any $j\in[n]$, we have
\begin{align*}
\pbr{\opnorm{\sum_{k\neq j} A_{jk} \Delta_k^{(j)}}  \geq  p\sqrt{n} \opnorm{\Delta^{(j)}} + t\Bigg| \Delta^{(j)}} &\leq 2d\ebr{- \frac{t^2/2}{np\opnorminf{\Delta^{(j)}}^2 + \opnorminf{\Delta^{(j)}}t/3}}
\end{align*}
and
\begin{align*}
\pbr{\opnorm{\sum_{k\neq j} A_{jk} \Delta_k^{(j)}}  \geq  p\sqrt{n} \opnorm{\Delta^{(j)}} + t\Bigg| \Delta^{(j)}} &\leq 2d\ebr{- \frac{t^2/2}{p\sqrt{nd}\opnorm{\Delta^{(j)}}\opnorminf{\Delta^{(j)}} + \opnorminf{\Delta^{(j)}}t/3}},
\end{align*}
for any $t>0$.
\end{lemma}
\begin{proof}
Consider any $j\in[n]$. 
Then
\begin{align*}
%\opnorm{\sum_{k\neq j} A_{jk} (U^{(j)}_kH^{(j)} - U^*_k)}  &= 
\opnorm{\sum_{k\neq j} A_{jk}\Delta_k^{(j)} } &\leq  \opnorm{\sum_{k\neq j} (A_{jk}-p)\Delta_k^{(j)}} + p\opnorm{\sum_{k\neq j} \Delta_k^{(j)}}\\
&\leq \opnorm{\sum_{k\neq j} (A_{jk}-p)\Delta_k^{(j)}} + p\sqrt{n} \opnorm{\Delta^{(j)}}.
\end{align*}
Since $\{A_{jk}-p\}_{k\neq j}$ is independent of $\Delta^{(j)}$,  we  use the matrix Bernstein's inequality (Lemma \ref{lem:matrix-bernstein-inequality_new}) for the operator norm of $\sum_{k\neq j} (A_{jk}-p)\Delta_k^{(j)}$. For each $k\in[n]$, note that $\E \br{(A_{jk}-p)\Delta_k^{(j)}  | \Delta_k^{(j)} }=0$ and $\opnorm{(A_{jk}-p)\Delta_k^{(j)}}\leq \opnorm{\Delta_k^{(j)}}\leq \opnorminf{\Delta^{(j)}}$. For the matrix variance term, we have
\begin{align*}
&\max\left\{ \opnorm{\E \left[ \sum_{k\neq j} (A_{jk}-p)^2 {\Delta_k^{(j)}}^\top  \Delta_k^{(j)}  \Bigg| \Delta^{(j)} \right]} , \opnorm{\E \left[ \sum_{k\neq j} (A_{jk}-p)^2  \Delta_k^{(j)}    {\Delta_k^{(j)}}^\top  \Bigg| \Delta^{(j)} \right] }  \right\}\\
&= p(1-p) \cdot \max\left\{ \opnorm{\sum_{k\neq j} {\Delta_k^{(j)}} ^\top  \Delta_k^{(j)} }, \opnorm{\sum_{k\neq j} \Delta_k^{(j)}{\Delta_k^{(j)} }^\top }   \right\} \\
&= p(1-p) \cdot \max\left\{  \opnorm{\Delta^{(j)}}^2, \opnorm{\br{\Delta_1^{(j)},\ldots, \Delta_n^{(j)}}^\top}^2 \right\} \\ \numberthis \label{eqn:lem_Fj1_1}
&\leq np(1-p)  \opnorminf{\Delta^{(j)}}^2 \leq np  \opnorminf{\Delta^{(j)}}^2.
\end{align*}
%In addition, we have $\opnorm{(A_{jk}-p)}$
Then we have
\begin{align*}
\pbr{\opnorm{\sum_{k\neq j} (A_{jk}-p)\Delta_k^{(j)}} \geq t\Bigg| \Delta^{(j)}} &\leq 2d\ebr{- \frac{t^2/2}{np\opnorminf{\Delta^{(j)}}^2 + \opnorminf{\Delta^{(j)}}t/3}}.
\end{align*}
Hence,
\begin{align}
\pbr{\opnorm{\sum_{k\neq j} A_{jk} \Delta_k^{(j)}}  \geq  p\sqrt{n} \opnorm{\Delta^{(j)}} + t\Bigg| \Delta^{(j)}} &\leq 2d\ebr{- \frac{t^2/2}{np\opnorminf{\Delta^{(j)}}^2 + \opnorminf{\Delta^{(j)}}t/3}}. \label{eqn:lem_Fj1_2}
\end{align}
In addition, note that $ \opnorm{\Delta^{(j)}}^2\leq \sqrt{n}\opnorm{\Delta^{(j)}}\opnorminf{\Delta^{(j)}}$ and
\begin{align*}
 \opnorm{\br{\Delta_1^{(j)},\ldots, \Delta_n^{(j)}}^\top}^2&\leq \sqrt{n}\opnorm{\br{\Delta_1^{(j)},\ldots, \Delta_n^{(j)}}^\top}\opnorminf{\Delta^{(j)}}\\
 &\leq \sqrt{n}\fnorm{\br{\Delta_1^{(j)},\ldots, \Delta_n^{(j)}}^\top}\opnorminf{\Delta^{(j)}}\\\\
 &= \sqrt{n}\fnorm{\Delta^{(j)}}\opnorminf{\Delta^{(j)}}\\
 &\leq \sqrt{nd}\opnorm{\Delta^{(j)}}\opnorminf{\Delta^{(j)}}.
\end{align*}
Then (\ref{eqn:lem_Fj1_1}) can also be upper bounded by $p\sqrt{nd}\opnorm{\Delta^{(j)}}\opnorminf{\Delta^{(j)}}$. Following a similar argument that leads to (\ref{eqn:lem_Fj1_2}), we also have
\begin{align*}
\pbr{\opnorm{\sum_{k\neq j} A_{jk} \Delta_k^{(j)}}  \geq  p\sqrt{n} \opnorm{\Delta^{(j)}} + t\Bigg| \Delta^{(j)}} &\leq 2d\ebr{- \frac{t^2/2}{p\sqrt{nd}\opnorm{\Delta^{(j)}}\opnorminf{\Delta^{(j)}} + \opnorminf{\Delta^{(j)}}t/3}}.
\end{align*}
\end{proof}


\begin{lemma}\label{lem:Fj2_denominator}
For any $j\in[n]$, we have
\begin{align*}
\pbr{\opnorm{ \sum_{k\neq j}A_{jk}(\Delta^{(j)}_k)^\top \Delta^{(j)}_k}   \geq  p \opnorm{\Delta^{(j)}}^2 + t \Bigg| \Delta^{(j)} } \leq 2d\ebr{ - \frac{t^2/2}{p \opnorminf{\Delta^{(j)}}^2  \opnorm{ \Delta^{(j)} }^2 + \opnorminf{\Delta^{(j)}}^2t/3}},
\end{align*}
for any $t\geq 0$.
\end{lemma}
\begin{proof}
We follow the proof of Lemma \ref{lem:Fj1}. We first have
\begin{align*}
\opnorm{ \sum_{k\neq j}A_{jk}(\Delta^{(j)}_k)^\top \Delta^{(j)}_k} &\leq  \opnorm{ \sum_{k\neq j}(A_{jk}-p)(\Delta^{(j)}_k)^\top \Delta^{(j)}_k}  + p\opnorm{ \sum_{k\neq j}(\Delta^{(j)}_k)^\top \Delta^{(j)}_k} \\
&\leq  \opnorm{ \sum_{k\neq j}(A_{jk}-p)(\Delta^{(j)}_k)^\top \Delta^{(j)}_k}  + p \opnorm{\Delta^{(j)}}^2.
\end{align*}
Note that for each $k\in[n]$, $\E \br{(A_{jk}-p)(\Delta^{(j)}_k)^\top \Delta^{(j)}_k |\Delta^{(j)}} =0$ and $\opnorm{(A_{jk}-p)(\Delta^{(j)}_k)^\top \Delta^{(j)}_k}\leq \opnorm{ \Delta^{(j)}_k}^2 \leq \opnorminf{\Delta^{(j)}}^2$. In addition,
\begin{align*}
&\opnorm{\E \left[ \sum_{k\neq j} (A_{jk}-p)^2 (\Delta^{(j)}_k)^\top \Delta^{(j)}_k(\Delta^{(j)}_k)^\top \Delta^{(j)}_k  \Bigg| \Delta^{(j)} \right]} \\
& =p(1-p)\opnorm{\sum_{k\neq j}  (\Delta^{(j)}_k)^\top \Delta^{(j)}_k(\Delta^{(j)}_k)^\top \Delta^{(j)}_k }  \\
&\leq p(1-p) \opnorm{ \,\left[{\Delta^{(j)}_1}^\top \,\cdots\, {\Delta^{(j)}_n}^\top   \right]  \begin{bmatrix} {\Delta^{(j)}_1}{\Delta^{(j)}_1}^\top & &\\ & \cdots &\\  & & {\Delta^{(j)}_n}{\Delta^{(j)}_n}^\top  \end{bmatrix}  \begin{bmatrix} \Delta^{(j)}_1\\ \vdots \\ \Delta^{(j)}_n \end{bmatrix} } \\
&\leq p(1-p)  \opnorm{ \begin{bmatrix} {\Delta^{(j)}_1}{\Delta^{(j)}_1}^\top & &\\ & \cdots &\\  & & {\Delta^{(j)}_n}{\Delta^{(j)}_n}^\top  \end{bmatrix}   }  \opnorm{ \Delta^{(j)} }^2\\
&=p(1-p)  \br{\max_{k\in[n]} \opnorm{\Delta^{(j)}_k{\Delta^{(j)}_k}^\top}} \opnorm{ \Delta^{(j)} }^2\\
&\leq  p(1-p)  \opnorminf{\Delta^{(j)}}^2  \opnorm{ \Delta^{(j)} }^2\,.
\end{align*}
Hence,
\begin{align*}
\pbr{\opnorm{ \sum_{k\neq j}A_{jk}(\Delta^{(j)}_k)^\top \Delta^{(j)}_k}   \geq  p \opnorm{\Delta^{(j)}}^2 + t \Bigg| \Delta^{(j)} } \leq 2d\ebr{ - \frac{t^2/2}{p \opnorminf{\Delta^{(j)}}^2  \opnorm{ \Delta^{(j)} }^2 + \opnorminf{\Delta^{(j)}}^2t/3}}.
\end{align*}
\end{proof}



\begin{lemma}\label{lem:Fj2}
There exists some constant $c>0$, such that for  any $t\geq 4\sqrt{d}\opnorm{ \sum_{k\neq j}A_{jk}(\Delta^{(j)}_k)^\top \Delta^{(j)}_k}^\frac{1}{2}$ and for any $j\in[n]$, we have
\begin{align*}
\pbr{\opnorm{\sum_{k\neq j}A_{jk}W_{jk}\Delta^{(j)}_k}\geq t\Bigg| \{A_{jk}\}_{k\neq j},\Delta^{(j)}}\leq 2\ebr{- \frac{ct^2}{\opnorm{ \sum_{k\neq j}A_{jk}(\Delta^{(j)}_k)^\top \Delta^{(j)}_k}}}.
\end{align*}
\end{lemma}
\begin{proof}
Consider any $j\in[n]$. Note that $\{A_{jk}\}_{k\neq j}$, $\{W_{jk}\}_{k\neq j}$, and $\{\Delta^{(j)}_k\}_{k\neq j}$ are mutually independent of each other. By Lemma \ref{lem:useful-MN-identities_new}, we have
\begin{align*}
\sum_{k\neq j}A_{jk}W_{jk}\Delta^{(j)}_k \Bigg| \{A_{jk}\}_{k\neq j}, \Delta^{(j)} \dequal E \br{\sum_{k\neq j}A_{jk}(\Delta^{(j)}_k)^\top \Delta^{(j)}_k}^\frac{1}{2} \Bigg| \{A_{jk}\}_{k\neq j},\Delta^{(j)}
\end{align*}
where $E \sim \MN(\zero, I_d, I_d)$ and is independent of $\{A_{jk}\}_{k\neq j}$ and  $\{\Delta^{(j)}_k\}_{k\neq j}$. Since $\sum_{k\neq j}A_{jk}(\Delta^{(j)}_k)^\top \Delta^{(j)}_k$ is symmetric and positive semi-definite, its square root is well-defined. By Lemma \ref{lem:vershynin_new}, there exists some constant $c>0$, such that for any $t\geq 4\sqrt{d}\opnorm{ \sum_{k\neq j}A_{jk}(\Delta^{(j)}_k)^\top \Delta^{(j)}_k}^\frac{1}{2} $, we have
\begin{align*}
& \pbr{\opnorm{\sum_{k\neq j}A_{jk}W_{jk}\Delta^{(j)}_k } \geq t  \Bigg| \{A_{jk}\}_{k\neq j}, \Delta^{(j)}} \\
 &= \pbr{\opnorm{E \br{\sum_{k\neq j}A_{jk}(\Delta^{(j)}_k)^\top \Delta^{(j)}_k}^\frac{1}{2}} \geq t \Bigg| \{A_{jk}\}_{k\neq j}, \Delta^{(j)}}\\
&\leq  \pbr{\opnorm{E}\opnorm{ \br{\sum_{k\neq j}A_{jk}(\Delta^{(j)}_k)^\top \Delta^{(j)}_k}^\frac{1}{2}}   \geq t  \Bigg| \{A_{jk}\}_{k\neq j}, \Delta^{(j)}}\\
& = \pbr{\opnorm{E}\opnorm{ \sum_{k\neq j}A_{jk}(\Delta^{(j)}_k)^\top \Delta^{(j)}_k}^\frac{1}{2}   \geq t  \Bigg| \{A_{jk}\}_{k\neq j}, \Delta^{(j)}}\\
&\leq 2\ebr{- \frac{ct^2}{\opnorm{ \sum_{k\neq j}A_{jk}(\Delta^{(j)}_k)^\top \Delta^{(j)}_k}}}.
\end{align*}
\end{proof}



\subsection{Auxiliary Lemmas}
\begin{lemma}\label{lem:matrix-bernstein-inequality_new} [Theorem 1.6 of \cite{tropp2012user}] Consider a finite sequence $\{S_k\}_{k=1}^n$ of independent random square matrices with dimension $d\times d$. Assume that each random matrix satisfies
$$\E S_k =0 \text{ and } \opnorm{S_k - \E S_k} \leq L, \,\forall k \in [n] \,.$$
Define
\begin{align*}
V &\define  \max\left\{ \opnorm{\sum_k \E\left[ S_k S_k ^\top \right]}, \opnorm{ \sum_k\E\left[ S_k^{\top}S_k \right]}  \right\} \,.
\end{align*}
Then
$$ \pbr{\opnorm{\sum_k S_k} \geq t} \leq 2d \cdot \ebr{-\frac{t^2/2}{V + Lt/3}} \,,$$
for any $t\geq 0$.
\end{lemma}

\begin{lemma}\label{lem:vershynin_new} [Corollary 7.3.3 of \cite{vershynin2018high}] Let $W$ be a $d\times d$ matrix with independent $\calN(0, 1)$ entries. Then there exists some constant $c>0$ such that  for every $t \geq 0$, we have
$$ \Pr\left( \opnorm{W} \geq 2\sqrt{d} + t  \right) \leq 2\,\ebr{-ct^2} \,. $$
\end{lemma}

\begin{lemma}\label{lem:useful-MN-identities_new}
%\textbf{Linear transformation of matrix-variate normal variables.} 
Consider any $W,W',\Delta \in\mathr^{d\times d}$ such that $W,W'\iid \MN(\zero, I_d, I_d)$ and $\Delta$ is a fixed matrix.
%For two $d\times d$ matrices $W,\Delta$ where $ W \sim \MN(\zero, I_d, I_d)$, 
We have
$$ W\Delta \sim \MN(\zero, I_d, \Delta^\top \Delta)\, $$
%\textbf{Sums of matrix-variate normal variables.} 
%For two $d\times d$ matrices $Z_1, Z_2$ such that $Z_1\sim \MN(\zero, I_d, \Sigma_1)$ and $Z_2 \sim \MN(\zero, I_d, \Sigma_2)$, we have
and
$$ W + W' \sim \MN(\zero, I_d, \Sigma_1 + \Sigma_2) \,.$$
\end{lemma}
\begin{proof} 
The first result in the lemma is a property of the matrix normal distribution. To prove the second statement, note that $Z \sim \MN(\zero, \Sigma_0, \Sigma_1)$ is equivalent to $\vec(Z) \sim \calN(0, \Sigma_1 \otimes \Sigma_0)$. Since $ \vec(W) \sim \calN(\zero,  \Sigma_1 \otimes I_d) $ and $ \vec(W') \sim \calN(\zero,  \Sigma_2 \otimes I_d) ,$ we have
%By definition, $Z \sim \MN(\zero, \Sigma_0, \Sigma_1)$ is equivalent to $\vec(Z) \sim \calN(0, \Sigma_1 \otimes \Sigma_0)$. Another way to view the first result in the lemma is to realize that
%$$ (W\Delta)_i = W_i^\top \Delta\,,$$
%where $W_i$ is the $i$-th row of $W$. Since each row $W_i$ is an \emph{independently} multivariate normal random variable with mean 0 and identity covariance, $W_i^\top \Delta \sim \calN(\zero, \Delta^\top \Delta)$. One can also verify that $\{W_i^\top \Delta\}_{i=1}^d$ are independently distributed. As such, $\vec(W\Delta)$ is a zero-mean multivariate normal random variable with block-structured covariance $I_d\otimes \Delta^\top \Delta$.
%
%To prove the second statement, recognize that
%$$ \vec(Z_1) \sim \calN(\zero,  \Sigma_1 \otimes I_d)\,, $$
%$$ \vec(Z_2) \sim \calN(\zero,  \Sigma_2 \otimes I_d) \,.$$
%Hence,
$ \vec(W) + \vec(W') \sim \calN(\zero,  (\Sigma_1 +\Sigma_2) \otimes I_d).$
\end{proof}


%\section{Conclusion}\label{sect:conclusion}
%In this paper, we propose a new spectral algorithm that overcomes a subtle yet important limitation of the vanilla spectral algorithm in \cite{pachauri2013solving}. With the new algorithm, we are able to achieve essentially the minimax optimal error rate for the permutation synchronization problem, up to the correct $\frac{1}{2}$ constant. For future works, it would be very interesting to consider alternative noise models to (\ref{eqn:X-def}) such as adversarially constructed noise, discrete-valued noise or heavy-tailed noise. We hope that our insight towards designing the new spectral algorithm will inspire future works in designing more advanced algorithms to solve other instances of the phase synchronization problem.

%
%
%\newpage
%\section{Proofs of Theorems}
%
%\subsection{Preliminaries}
%\paragraph{Decomposition of $U$.}
%\begin{align*}
%UH\Lambda - XU^*  & = U(H\Lambda - \Lambda H) + U\Lambda H -XU^*\\
%& = U(H\Lambda - \Lambda H) + XU H -XU^*\\
%&= U(H\Lambda - \Lambda H) + X(U H - U^*).
%\end{align*}
%Multiplied by $\Lambda^{-1}$ and after arrangements, it leads to
%\begin{align}\label{eqn:decomposition1}
%UH - U^* = U(H\Lambda - \Lambda H) \Lambda^{-1}  + X(U H - U^*) \Lambda^{-1} + XU^*\Lambda^{-1} - U^* .
%\end{align}
%Consider any $j\in[n]$. Define
%\begin{align*}
%H^{(j)}= U^{(j)\top}U^*\in\mathr^{d\times d}.
%\end{align*}
%Then
%\begin{align*}
%UH-U^* &= UH - U^{(j)}H^{(j)} + U^{(j)}H^{(j)} - U^* \\
%& = (UU^\top - U^{(j)}U^{(j)\top})U^*+ U^{(j)}H^{(j)} - U^* .
%\end{align*}
%Plug it into the right-hand side of (\ref{eqn:decomposition1}), we have
%\begin{align*}
%UH - U^* & = U(H\Lambda - \Lambda H) \Lambda^{-1}  + X(UU^\top - U^{(j)}U^{(j)\top})U^* \Lambda^{-1} \\
%&\quad  +  X(U^{(j)}H^{(j)} - U^*) \Lambda^{-1}  + XU^*\Lambda^{-1} - U^*.
%\end{align*}
%Then, the $j$th block matrix of $UH - U^*$ satisfies
%\begin{align}
%U_jH - U^*_j & = U_j(H\Lambda - \Lambda H) \Lambda^{-1}  + X_j(UU^\top - U^{(j)}U^{(j)\top})U^* \Lambda^{-1} \nonumber\\
%&\quad  +  X_j(U^{(j)}H^{(j)} - U^*) \Lambda^{-1}  + X_jU^*\Lambda^{-1} - U^*_j.\label{eqn:decomposition2}
%\end{align}
%The decomposition (\ref{eqn:decomposition2}) holds for all $j\in[n]$ is the starting point to establish our main theorems.
%
%\subsection{Proof of Theorem \ref{thm:l_infty}}
%
%\import{./}{proof_of_ell_infty_main}
%
%\subsection{Proof of Theorem \ref{thm:tail-bound}}
%
%\import{./}{proof_of_tail_bound_main}
%
%\subsection{Proof of Theorem \ref{thm:main}}
%
%\import{./}{proof_of_partial_recovery_main}
%
%\newpage
%\section{Lemmas and Their Proofs}
%
%\import{./}{proof_of_ell_infty_misc}
%
%\newpage
%\import{./}{proof_of_tail_bound_misc}
%
%\newpage
%\import{./}{proof_of_partial_recovery_misc}
%
%
%
%\newpage
%\section{Additional Useful Results}
%
%\import{./}{preliminaries}