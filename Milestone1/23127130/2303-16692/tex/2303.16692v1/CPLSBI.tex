%----------------------- 2022--6--17------------------------------
\documentclass[a4paper,twosides]{article}
\usepackage{multicol,multirow,graphics,fancyhdr}
\usepackage{amsmath,amsfonts,amssymb,bm,upgreek,mathrsfs,mathcomp}
\usepackage[pagewise,switch,columnwise]{lineno}
\usepackage[compress,nospace]{cite}
\usepackage[dvipsnames]{xcolor}
\usepackage{CPL-2022}
\usepackage{graphicx}
\usepackage{subfigure}

%------------Page layout and margin and Headrule-------------------
%\pagestyle{fancy}
\fancyhead[L]{}
\fancyhead[R]{\colorbox{lightgray}{\color{red}\textbf{\large\fontfamily{phv}\textbf{New Submission}}}}

%----------Row spacing of Text and Table and *Footnote-------------
\renewcommand{\baselinestretch}{1.0} % text-distance
\renewcommand{\arraystretch}{1.1} % table-distance

%---------------No.page and Odd and Even-------------------
\newcommand{\cplyear}{2023} \newcommand{\cplvol}{xx}
\newcommand{\cplno}{x} \newcommand{\cplpagenumber}{xxxxxx}
\setcounter{page}{1} \newcommand{\cplpage}{\cplpagenumber-\thepage}
\newcommand{\cplCode}{2022-xxxx}

%\linenumbers

\begin{document}

\vspace* {-4mm} \begin{center}
%-------------------------Title----------------------------
\large\bf{\boldmath{Viscous effect on interaction between shock wave and cylindrical bubble: based on the discrete Boltzmann method}}
%------------------------Footnote--------------------------
\footnotetext{\hspace*{-5.4mm}$^{*}$Corresponding authors. Email: Xu\_Aiguo@iapcm.ac.cn; lyj@aphy.iphy.ac.cn

\noindent\copyright\,{\cplyear}
\href{http://www.cps-net.org.cn}{Chinese Physical Society} and
\href{http://www.iop.org}{IOP Publishing Ltd}}
\\[5mm]
%------------------------Authors----------------------------
\normalsize \rm{}Dejia Zhang$^{1,2}$, Aiguo Xu$^{2,3,4*}$, Yanbiao Gan$^{5}$, Jiahui Song$^{6}$, \\ Yudong Zhang$^{7}$, and Yingjun Li$^{1*}$
%----------------------COM. or University-------------------
\\[3mm]\small\sl $^{1}$State Key Laboratory for GeoMechanics and Deep Underground Engineering, China University of Mining and Technology, Beijing 100083, P.R.China

$^{2}$Laboratory of Computational Physics, Institute of Applied Physics and Computational Mathematics, \\ P. O. Box 8009-26, Beijing 100088, P.R.China

$^{3}$HEDPS, Center for Applied Physics and Technology, and College of Engineering, Peking University, Beijing 100871,P.R.China

$^{4}$State Key Laboratory of Explosion Science and Technology, Beijing Institute of Technology, Beijing 100081, P.R.China

$^{5}$Hebei Key Laboratory of Trans-Media Aerial Underwater Vehicle, School of Liberal Arts and Sciences, North China Institute of Aerospace Engineering, Langfang 065000, P.R.China

$^{6}$School of Aerospace Engineering, Beijing Institute of Technology, Beijing, 100081, P.R.China

$^{7}$School of Mechanics and Safety Engineering, Zhengzhou University, Zhengzhou 450001, P.R.China
%------------------------Received date----------------------
\\[4mm]\normalsize\rm{}(Received xxx; accepted manuscript online xxx)
\end{center}

%----------------------Abstract and PACS--------------------
\vskip 1.5mm

\small{\narrower
The viscous effects on the interaction between a shock wave and a two-dimensional cylindrical bubble are investigated based on the discrete Boltzmann method (DBM).
Besides some interesting Hydrodynamic Non-Equilibrium (HNE) behaviors, some relevant Thermodynamic Non-Equilibrium (TNE) behaviors are carefully studied.
It is found that the viscosity contributes little effect on the dynamic processes in the shock compression stage but significantly influences them in the post-shock stage.
A bubble with a smaller viscosity coefficient displays a stouter jet structure, can be compressed more easily, and reaches its minimum characteristic scales slower.
The viscosity accelerates the average motion of the bubble, reduces the vorticity strength (circulation), and restrains the material mixing between the ambient gas and the bubble.
The viscous effects on different TNE quantities/perspectives show interesting differences.
These differences indicate the complexity of TNE behaviors, which still requires further understanding.
The viscous effects on entropy production are also investigated.
It is found that the entropy production caused by the non-organized momentum flux (NOMF) is larger than that caused by the non-organized energy flux (NOEF).
As the Prandtl number increases, the entropy production $S_{\rm{NOMF}}$ increases.
But the $S_{\rm{NOEF}}$ first decreases and then approaches a saturation value.



\par}\vskip 3mm
\normalsize\noindent{\narrower{PACS: 47.20.Ma,05.20.Dd,51.20.+d,47.40.Nm}}\\
\noindent{\narrower{DOI: \href{http://dx.doi.org/10.1088/0256-307X/\cplvol/\cplno/\cplpagenumber}{10.1088/0256-307X/\cplvol/\cplno/\cplpagenumber}}

\par}\vskip 5mm
%-------------------TEXT TEXT TEXT TEXT---------------------
%\section{Introduction}

The physical scenarios of shock-bubble interaction (SBI) are common in many natural phenomena and engineering applications.\ucite{Ranjan2011ARFM}
For example, in astrophysics, the Puppis A supernova remnant interacts with a complex system of interstellar clouds.\ucite{Hwang2005}
In the combustion system, the shock wave ignites the mixture bubble composed of $\rm{H_2}$ and $\rm{O_2}$.\ucite{Diegelmann2017CNF}
In inertial confinement fusion, the laser-induced shock wave impacts the isolated defects bubble within the capsule, aggravating the hydrodynamic instability.\ucite{Liu2023POP}
The simplest configuration of the SBI system refers to a spherical bubble accelerated by a planar shock wave.
Many physical factors influence the bubble deformation process, including the type and strength (Mach number, Ma) of the incident shock, the initial bubble shape, the boundary types of fluid field, the density ratio between the bubble and the ambient gas (Atwood number), the specific-heat ratio, the viscosity, and the heat conduction, and etc.
Due to the importance of SBI in engineering applications, scholars have paid a lot of effort to study its evolution mechanism.
Research methods typically involve three types: theoretical method,\ucite{Samtaney1994JFM,Yang1995JFM} experimental method,\ucite{Layes2003PRL,Ranjan2007PRL} and numerical simulation method \ucite{Picone1988JFM,Zou2015SCPMA,Sha2013APS,Sha2015APS,Yang2015CPL}.
Among these, Samtaney \emph{et al.} presented the analytical expressions for circulation $\Gamma$ which are within and beyond the regular refraction regime.\ucite{Samtaney1994JFM}
Ding \emph{et al.} investigated experimentally and numerically the effects of initial interface curvature on the interaction between planar shock waves and heavy/light bubbles.\ucite{Ding2017JFM,Ding2018POF}
Other scenarios, such as bubbles impacted by the converging shock wave,\ucite{Si2014Si2014LPB} spherical/cylindrical bubbles interacting with planar shock wave under re-shock conditions,\ucite{Si2012POF,Zhai2014JV} have also been investigated.

However, the experimental platforms are difficult to build when investigating complex problems with extreme conditions.
In such cases, the numerical simulation method can provide an alternative option.
According to the difference in the theoretical foundation of physical modeling, numerical simulation methods can be classified into three types: macroscopic method, mesoscopic method, and microscopic method.
In previous numerical studies of SBI, the traditional macroscopic modeling method based on the continuous hypothesis (or equilibrium and near-equilibrium hypothesis)
\footnote{The traditional macroscopic modeling methods based on the continuous hypothesis includes the Euler equations and Navier-Stokes (NS) equations.
The first is based on the equilibrium hypothesis, and the second is based on the near-equilibrium hypothesis.
The hydrodynamic equations in the traditional macroscopic modeling method only describe the hydrodynamic behaviors corresponding to the conservation laws of mass, momentum, and energy.
With increasing the degree of non-equilibrium and non-continuity, the more appropriate hydrodynamic equations refer to the Extended Hydrodynamic Equations (EHEs) which are composed of not only the evolution equations of conserved kinetic moments but also the most relevant non-conserved kinetic moments of the distribution function\ucite{Zhang2022POF}.
} has been widely used.
For example, Ding \emph{et al.} demonstrated a good agreement of interface structure between the numerical results obtained from the compressible Euler equations and experimental data.\ucite{Ding2017JFM,Ding2018POF}
Zou \emph{et al.} investigated the Atwood number effects and the jet phenomenon caused by the shock focusing through the multi-fluid Eulerian equations.\ucite{Zou2015SCPMA}
However, as the non-continuity/non-equilibrium degree of the fluid system increases, the rationality and physical function of the macroscopic models will be challenged.
A small part of the works on SBI belong to the mesoscopic method, such as the Direct Simulation Monte Carlo method.\ucite{Zhang2019CNF}
The microscopic method such as the Molecular dynamics (MD) simulation is capable of capturing much more flow behaviors.
But it is restricted to too small spatio-temporal scales because of its huge computing costs.

Two existing facts are encountered in the previous SBI studies.
(i) Most of these studies describe mainly the flow morphology and SBI process  from a macroscopic view.
They are concerned more with dynamic processes such as bubble deformation, interface motion, vortex motion, mixing degree, etc.
The mesoscopic kinetic behaviors such as the Thermodynamic Non-Equilibrium (TNE) behaviors are rarely studied.
However, many studies have emphasized the importance of investigating TNE behaviors for understanding the kinetic process.\ucite{Lai2016PRE,Lin2017PRE,Chen2018POF,Gan2019FOP,Chen2022PRE,Li2022CTP,Shan2022JMES}
(ii) The viscous effects on small structures and kinetic behaviors of SBI need further investigation.
Research presented by Zhang \emph{et al.} has shown that viscosity plays an important role in material mixing in single-mode Rayleigh–Taylor (RT) system.\ucite{Zhang2021POF}
Zhang \emph{et al.} performed that the viscous effects lead to the disappearance of some typical phenomena on the reacting shock-bubble interaction.\ucite{Zhang2019CNF}
The bulk viscosity associated with the viscous excess normal stress, including different physical properties of diatomic and polyatomic gases, significantly changes the flow morphology and results in complex wave patterns, vorticity generation, vortex formation, and bubble deformation.\ucite{Singh2021POF}
Moreover, studying the viscous effects on kinetic behaviors, particularly on TNE behaviors, is essential in comprehending the essential mechanism of the viscous effect.

For the above two facts, in this work we resort to the discrete Boltzmann method (DBM)
\footnote{The DBM can also be interpreted as the Discrete Boltzmann Modeling method or the Discrete Boltzmann Modeling model according to the context.}.\ucite{Xu2022CMK,Zhang2022POF,Gan2022JFM,Zhang2022AIP}
The DBM aims to solve the following problems: (i) The traditional macroscopic modelings are only concerned with the evolution of three conserved kinetic moments.
Therefore, they lost much information as the TNE degree increases.
In such cases, more non-conserved moments are required to ensure the non-significant decrease of the system state and behavior description function.
(ii) The situation that the MD can be adopted is limited to too small spatial-temporal scales due to the huge computing costs.
From the view of physical function, a DBM not only encompasses a set of evolution equations of the conserved kinetic moment but also includes the most relevant evolution equations of the non-conserved kinetic moment of the distribution function.
There are mainly two physical functions for DBM: (i) Capturing the main features aiming to investigate. (ii) Presenting schemes for checking the  TNE state and describing TNE effects.
Besides the Hydrodynamic Non-Equilibrium (HNE) behaviors within the physical function of the traditional macroscopic modeling, the DBM pays more attention to the most relevant TNE behaviors which are beyond the traditional macroscopic modeling.
Specifically, it uses the non-conservative moments of ($f-f^{eq}$) to describe how and how much the system deviates from the thermodynamic equilibrium state and to check corresponding effects due to deviating from the thermodynamic equilibrium.
Besides the scheme for analyzing the TNE effects and behaviors, the DBM also couples with other methods for analyzing the complex physical field, such as the tracer particle method.\ucite{Zhang2021POF}

We use a two-fluid DBM to character the main characteristics of the bubble, wave patterns, and flow morphology, and to capture the TNE effects of the flow field.\ucite{Zhang2023CAF}
Compared with traditional fluid modeling, DBM is also suitable for flow with a higher degree of non-equilibrium and discrete effects.
We validated by the convergence of the results that it is reasonable to include only first-order non-equilibrium effects in DBM for the simulations simulated in the current paper.
From the original Boltzmann equation to a two-fluid DBM, four steps are needed.
(i) Simplifying and modifying the Boltzmann equation according to the physical requirement.
To flexibly adjust the Prandtl (Pr) number, the Ellipsoidal Statistical Bhatnagar–Gross–Krook (ES–BGK) Boltzmann equation is adopted in this work.\ucite{Zhang2022POF,Zhang2017FOP}
(ii) Discretizing the particle velocity space.
After this step, the discrete-form ES–BGK two-fluid Boltzmann equation for describing the interaction between two different components is obtained.
It is shown as the Eq. (\ref{Eq.Discrete-Boltzmann1}):
\begin{equation}
\frac{\partial f_{i}^{\sigma}}{\partial t}+v_{i\alpha}\cdot\frac{\partial f_{i}^{\sigma}}{\partial r_{\alpha}}=-\frac{1}{\tau^{\sigma}}(f_{i}^{\sigma}-f^{\sigma,ES}_{i})
\label{Eq.Discrete-Boltzmann1}
,
\end{equation}
where $v_{i\alpha}$, $r_{\alpha}$, $t$, and $\tau^{\sigma}$ represent the particle velocity, particle position, time, and relaxation time of component $\sigma$, respectively.
The $i$ is the kind of discrete velocities and $\alpha$ ($\alpha=x\;$ or $\;y$) indicates the direction in Cartesian coordinate.
$f^{\sigma}_{i}$ ($f^{\sigma,ES}_{i}$) is the distribution function (ES distribution function) of the component $\sigma$.
(iii) Checking the TNE state and extracting TNE information.
In the DBM phase space description method, we can define the corresponding TNE quantities according to the research requirements to describe the TNE state and extract the TNE effects.
We continue the definition methods in Ref. \cite{Zhang2023CAF}.
Their formulas are:
\begin{equation}
\bm{\Delta}^{\sigma*}_{m}=\sum_{i}(f_{i}^{\sigma}-f^{\sigma,eq}_{i}) \underbrace { \mathbf{v}^{*}_{i}\mathbf{v}^{*}_{i} \cdots \mathbf{v}^{*}_{i} }_m
\tt{,} \label{Eq:DDBM-NF28}
\end{equation}
\begin{equation}
\bm{\Delta}^{\sigma*}_{m,n}=\frac{1}{2}\sum_{i}(f_{i}^{\sigma}-f^{\sigma,eq}_{i})(\mathbf{v}^{*}_{i}\cdot\mathbf{v}^{*}_{i}+\eta_{i}^{\sigma 2})^{(m-n)/2} \underbrace { \mathbf{v}^{*}_{i} \cdots \mathbf{v}^{*}_{i} }_n
\tt{.}
\end{equation}
Here, $\mathbf{v}^{*}_{i}=\mathbf{v}_{i}-\mathbf{u}$ represents the central velocity, where $\mathbf{u}$ represents the macro flow velocity.
$f^{\sigma,eq}_{i}$ is the equilibrium distribution function.
Other TNE quantities containing more condensed information can be defined as follows:
\begin{equation}
\left|\Delta_{2}^{*}\right|=\sqrt{\Delta_{2,xx}^{*2}+\Delta_{2,xy}^{*2}+\Delta_{2,yy}^{*2}}
,
\end{equation}
\begin{equation}
\left|\Delta_{3,1}^{*}\right|=\sqrt{\Delta_{3,1,x}^{*2}+\Delta_{3,1,y}^{*2}}
 \tt{,}
\end{equation}
\begin{equation}
\left|\Delta_{3}^{*}\right|=\sqrt{\Delta_{3,xxx}^{*2}+\Delta_{3,xxy}^{*2}
+\Delta_{3,xyy}^{*2}+\Delta_{3,yyy}^{*2}}
,
\end{equation}
\begin{equation}
\left|\Delta_{4,2}^{*}\right|=\sqrt{\Delta_{4,2,xx}^{*2}+\Delta_{4,2,xy}^{*2}
+\Delta_{4,2,yy}^{*2}}
 \tt{,}
\end{equation}
where $\Delta_{m}^{*}=\Delta_{m}^{\rm{*A}}+\Delta_{m}^{\rm{*B}}$ and $\Delta_{m,n}^{*}=\Delta_{m,n}^{\rm{*A}}+\Delta_{m,n}^{\rm{*B}}$.
By integrating the above four TNE quantities over the whole fluid field, the global TNE strengths are obtained.
Two of the most typical TNE quantities are $\Delta^{*}_{2,\alpha \beta}$ and $\Delta^{*}_{3,1,\alpha}$.
They correspond to the viscous stress (or non-organized momentum flux, NOMF) and the heat flux (or non-organized energy flux, NOEF), respectively.
Their expressions are:
\begin{equation}
\Delta^{*}_{2,\alpha \beta}=-\mu(\frac{\partial u_{\alpha}}{\partial r_{\beta}}+\frac{\partial u_{\beta}}{\partial r_{\alpha}}-\frac{2}{D+I}\frac{\partial u_{\gamma}}{\partial r_{\gamma}}\delta_{\alpha\beta})
\label{Eq.delta2}
,
\end{equation}
\begin{equation}
\Delta^{*}_{3,1,\alpha}=-\kappa \frac{\partial T}{\partial r_{\alpha}}
\label{Eq.delta31}
,
\end{equation}
where $\mu=\Pr \tau p$ is the viscosity coefficient and $\kappa=c_p \tau p$ is the heat conductivity.
(iv) Selecting and designing a proper boundary condition.
For more details about the DBM modeling process refer to Refs. \cite{Zhang2022POF,Zhang2023CAF}.

Figure~1 illustrates the configuration of the interaction between a planar shock wave and a cylindrical bubble.
The dimensionless scale of the rectangular flow field is $L_x\times L_y=0.096 \times 0.048$. It is divided into $N_x\times N_y=800 \times 400$ grid size.
The incident shock wave has a strength of $\rm{Ma}=1.23$.
The dimensionless initial conditions of the flow field are as follows:
\[
\left\{ \begin{gathered}
(\rho,T,u_x ,u_y )_{\rm{bubble}} = (5.0168,1.0,0.0,0.0), \hfill \\
(\rho,T,u_x ,u_y )_1 = (1.394,1.1463,0.4111,0.0), \hfill \\
(\rho,T,u_x ,u_y )_0 = (1.0,1.0,0.0,0.0), \hfill \\
\end{gathered} \right.
\]
where the subscript ``0'' (``1'') represents downstream (upstream) region.
Other parameters used for the simulation are: $c=1.0$, $\eta^{\rm{A}}=\eta^{\rm{B}}=10.0$, $I^{\rm{A}}=3$, $I^{\rm{B}}=15$, $\Delta x = \Delta y = 1.2 \times 10^{-4}$, $\Delta t = 1 \times 10^{-6}$, $\tau ^{\rm{A}}=4\times10^{-6}$, and $\tau ^{\rm{B}}=3\times10^{-6}$.
The left (right) side of the flow field uses the inflow (outflow) boundary condition.
The periodic boundary is adopted in the $y$ direction.
The first-order forward difference scheme and the second-order nonoscillatory nonfree dissipative scheme are used to solve the temporal derivative and the spatial derivative in Eq. (\ref{Eq.Discrete-Boltzmann1}), respectively.

%\vskip 4mm

\begin{figure}[htbp]
\fl{1}\centerline{\includegraphics[width=0.5\textwidth]{fig1}}
\figcaption{10}{1}{The computational configuration of the shock-bubble interaction.}
\label{fig1}
\end{figure}

%\vskip 2mm

To study the viscous effects on SBI, we varied the viscosity coefficient of the bubble by adjusting the Pr number.
Fives cases with different $\Pr$ numbers are simulated in this work, i.e., $\Pr=0.33, 0.5, 1.0, 1.43$, and $2.5$.
The larger the $\Pr$ number, the stronger the viscous effect.
Figure~2 highlights the DBM results of the density contour (odd rows) and the particle tracer image (even rows) at three different moments (i.e., $t=0.07, t=0.11$, and $t=0.16$) with various viscosity coefficients.
The red boxes in the figures indicate the discernible differences between various cases.
It is found that the viscosity significantly affects the shapes of the jet structure and the upstream interface.
Actually, the jet structure is caused by the shock focusing, and the viscosity inhibits bubble deformation.
Therefore, it is shown that the larger (smaller) the viscosity coefficient of the bubble, the finer (stouter) the jet structure of it.
Then, the evolution of characteristic scales used to describe the deformation process is plotted in Figure~3(a).
We find that the bubble with the smallest viscosity is compressed to the smallest characteristic scales.
This is because the fluid with smaller viscosity is easier to be compressed.
It is also found that the bubble with a smaller viscosity coefficient takes longer to reach its minimum characteristic scales.
By averaging the information on the velocity and position of particle tracers, the average motion of the bubble is obtained.
Figure~3(b) displays the average position and average velocity of the bubble, revealing that the viscous effects contribute little to the average motion of the bubble before $t<0.03$ (i.e., the shock compression stage).
However, when the shock has passed through the bubble, i.e., $t>0.03$ (the post-shock stage), the viscosity accelerates the average motion of the bubble since bubbles with lower viscosity require more energy to compress their size, resulting in lower translational energy.

\begin{figure}[htbp]
\fl{2}\centerline{\includegraphics[width=0.6\textwidth]{fig2}}
\figcaption{18}{2}{Density contours and particle tracer images at three different moments (i.e., $t=0.07, t=0.11$, and $t=0.16$) with various viscosity coefficients.
The odd rows represent density contours, and the even rows are particle tracer images.}
\label{fig2}
\end{figure}

%\vskip 4mm

\fl{3}\begin{figure}[htbp]
\centering
\subfigure[]{
\begin{minipage}{5cm}
\centering
	\includegraphics[width=5cm]{fig3.eps}
\end{minipage}
}
\subfigure[]{
\begin{minipage}{5cm}
\centering
	\includegraphics[width=5cm]{fig4.eps}
\end{minipage}
}
\figcaption{15}{3}{(a) The temporal evolution of characteristic scales on SBI process.
(b) The temporal evolution of average position and average bubble velocity.
Lines with different colors represent the cases with various viscosity coefficients.
}
\label{fig3-4}
\end{figure}

Figure~4(a) plots the circulations $\Gamma$ which are used to describe the strength of vorticity.
The $\Gamma^{+}$ ($\Gamma^{-}$) represents the positive (negative) circulation.
It is notable that the absolute values of $\Gamma^{-}$ are the same as those of $\Gamma^{+}$.
Moreover, the total circulations $\Gamma$ ($\Gamma$=$\Gamma^{+}$+$\Gamma^{-}$) are equal to zero at all times, indicating that the vorticity pointing in the opposite direction is equal in value.
It is also found that the viscous effects on circulations become significant after the incident shock wave passes through the bubble.
For the fluid with smaller viscosity, it is easier to rotate and twist, leading to larger values of circulation in cases with lower viscosity.
Then, the material mixing between ambient gas and the bubble is analyzed.
The global mixing degree is defined as $M_g=4 \cdot \frac{\overline{M_{\rm{A}}\cdot M_{\rm{B}}}}{\overline{M_{\rm{A}}} \cdot \overline{M_{\rm{B}}}}$, where $M_{\sigma}$ is the mass fraction.
Figure~4(b) shows the temporal evolution of the global mixing degree $M_g$.
In the shock compression stage where the incident shock dominates, the shock wave enhances the mixing amplitude and increases the mixing area, resulting in the rapid development of the global mixing degree.
The viscous effects contribute little to the global mixing degree in this stage.
However, in the post-shock stage, the viscous effects become significant.
The viscosity restrains the material mixing between two fluids by suppressing the relative flow.
\begin{figure}[htbp]
\centering
\setcounter{subfigure}{0}
\subfigure[]{
\begin{minipage}{5cm}
\centering
	\includegraphics[width=5cm]{fig5.eps}
\end{minipage}
}
\subfigure[]{
\begin{minipage}{5cm}
\centering
	\includegraphics[width=5cm]{fig6.eps}
\end{minipage}
}
\figcaption{15}{4}{(a) Temporal evolution of circulation on SBI process.
(b) Temporal evolution of global mixing degree $M_g$ on SBI process.
Lines with different colors represent the cases with various viscosity coefficients.
}
\label{fig5-6}
\end{figure}

It is clear that there are kinds of TNE strength definitions because they depend on the perspective of the investigation.
Examining the TNE degree from different angles can yield diverse results.
The descriptions from various TNE perspectives constitute a more complete characterization of the non-equilibrium state.
Figure~5 investigates the viscous effects on the TNE degree of the fluid system from four fundamental TNE quantities, i.e., $D^{*}_{2}$, $D^{*}_{3}$, $D^{*}_{3,1}$, and $D^{*}_{4,2}$.
Figure~5(a) shows that the viscosity enhances the $D^{*}_{2}$ strength but reduces the $D^{*}_{3,1}$ strength.
The essential physic is that the viscosity, from the view of $D^{*}_{2}$, makes the fluid system deviate from the thermodynamic equilibrium state.
Conversely,  from the $D^{*}_{3,1}$ view, it lets the fluid system approach the thermodynamic equilibrium state.
This phenomenon can be explained by the analytical formulas Eqs. (\ref{Eq.delta2}) and (\ref{Eq.delta31}).
Theoretically, the viscous effects on TNE quantities are the comprehensive result of the competition between the transport coefficient and macroscopic quantity gradients.
A larger viscosity coefficient would reduce the macroscopic quantity gradients.
Therefore, the enhancement effect of viscosity on $D^{*}_{2}$ strength is reflected in increasing the viscosity coefficient.
The viscosity reduces the $D^{*}_{3,1}$ and $D^{*}_{3}$ strengths by decreasing the temperature gradient.
For $D^{*}_{4,2}$, the comprehensive effects of the transport coefficient and macroscopic quantity gradients make its strength increase.

\begin{figure}[htbp]
\centering
\setcounter{subfigure}{0}
\subfigure[]{
\begin{minipage}{5cm}
\centering
	\includegraphics[width=5cm]{fig7.eps}
\end{minipage}
}
\subfigure[]{
\begin{minipage}{5cm}
\centering
	\includegraphics[width=5cm]{fig8.eps}
\end{minipage}
}
\figcaption{15}{5}{(a) Temporal evolution of $D_{2}^{*}$ and $D_{3,1}^{*}$ strengths.
(b) Temporal evolution of $D_{3}^{*}$ and $D_{4,2}^{*}$ strengths.
Lines with different colors represent the cases with various viscosity coefficients.
}
\label{fig7-8}
\end{figure}


%\medskip

%\vskip 4mm

The entropy production rate and entropy production, which are significant in the compression science field, are also analyzed.
There are two kinds of entropy production rates\ucite{Zhang2010Soft-Matter}:
\begin{equation}
\dot{S}_{\rm{NOEF}} =  \int \bm{\Delta}_{3,1}^{*} \cdot \nabla \frac{1}{T} d \bm{r}
, \dot{S}_{\rm{NOMF}} =  \int -\frac{1}{T} \bm{\Delta}_{2}^{*} : \nabla \bm{u} d \bm{r}
.
\end{equation}
The former is caused by the NOEF and the temperature gradient, and the latter is contributed by the NOMF and the velocity gradient.
Figure~6(a) and (b) show the temporal evolution of $\dot{S}_{\rm{NOMF}}$ and $\dot{S}_{\rm{NOEF}}$, respectively.
It is found that the viscous effects on the two kinds of entropy production rates are different.
Specifically, viscosity enhances the $\dot{S}_{\rm{NOMF}}$ but reduces the $\dot{S}_{\rm{NOEF}}$.
The reason is because viscosity increases the $\Delta_{2}^{*}$ strength but decreases the $\Delta_{3,1}^{*}$ strength.
By integrating the entropy production rate over the time $t$, the entropy production over a given period can be obtained.
As can be seen in Figure~6(c), it is apparent that the entropy production caused by the NOMF is larger than the entropy production caused by the NOEF.
As the $\Pr$ number increases, the $S_{\rm{NOMF}}$ always increases, but the $S_{\rm{NOEF}}$ first decreases and then approaches a saturation value.
The total entropy production first decreases and then increases as the  $\Pr$ number increases.
\begin{figure}[htbp]
\centering
\setcounter{subfigure}{0}
\subfigure[]{
\begin{minipage}{5cm}
\centering
	\includegraphics[width=5cm]{fig9.eps}
\end{minipage}
}
\subfigure[]{
\begin{minipage}{5cm}
\centering
	\includegraphics[width=5cm]{fig10.eps}
\end{minipage}
}
\subfigure[]{
\begin{minipage}{5cm}
\centering
	\includegraphics[width=5cm]{fig11.eps}
\end{minipage}
}
\figcaption{15}{6}{(a) Temporal evolution of entropy production rate $\dot{S}_{\rm{NOMF}}$.
(b) Temporal evolution of entropy production rate $\dot{S}_{\rm{NOEF}}$.
(c) Entropy production ($S_{\rm{NOMF}}$, $S_{\rm{NOEF}}$, $S_{\rm{NOMF}}$+$S_{\rm{NOEF}}$) over this period.
Lines with different colors represent the cases with various viscosity coefficients.
}
\label{fig9-10}
\end{figure}

In summary, the viscous effects on the interaction between a shock wave and a two-dimensional cylindrical bubble are investigated based on the DBM.
Different from most of the previous research that relied on the traditional macroscopic models/methods, this paper studies the dynamic and kinetic processes of SBI from a mesoscopic view.
Besides some interesting HNE behaviors, some meaningful TNE behaviors are presented.
While the viscosity contributes little effect on the dynamic processes in the shock compression stage, it significantly affects them in the post-shock stage.
It is found that the larger (smaller) viscosity coefficient of the bubble, the finer (stouter) the jet structure of it.
This is because viscosity inhibits bubble deformation.
For the bubble with a smaller viscosity coefficient, it is easier to be compressed and reaches its minimum characteristic scales slower.
Additionally, viscosity accelerates the average motion of bubbles while reducing the vorticity strength (circulation) and material mixing between the ambient gas and the bubble.

In addition, the viscous effects on different TNE quantities/perspectives show interesting differences.
These differences indicate the complexity of TNE behaviors, which still requires further understanding.
Specifically, the viscosity increases the $D^{*}_{2}$ strength but decreases the $D^{*}_{3,1}$ strength.
The viscous effect on $D^{*}_{2}$ strength stems from an augmented viscosity coefficient, whereas the decrease in $D^{*}_{3,1}$ and $D^{*}_{3}$ strengths depend on a reduced temperature gradient.
For $D^{*}_{4,2}$, the comprehensive effects of the transport coefficient and macroscopic quantity gradients make its strength increase.
The viscosity raises the $\dot{S}_{\rm{NOMF}}$ but reduces the $\dot{S}_{\rm{NOEF}}$.
The entropy production caused by the NOMF is larger than that caused by the NOEF.
As the $\Pr$ number increases, the $S_{\rm{NOMF}}$ always increases, but the $S_{\rm{NOEF}}$ first decreases and then approaches a saturation value.
The fundamental research in this paper enhances our understanding of the SBI mechanism in various applications, such as inertial confinement fusion, supersonic combustors, underwater explosions, etc.

\textit{Acknowledgements.}
This work was supported by the National Natural Science Foundation of China (Grant Nos.  12172061, 11875001, and 12102397), the Strategic Priority Research Program of Chinese Academy of Sciences (Grant No. XDA25051000), the opening project of State Key Laboratory of Explosion Science and Technology (Beijing Institute of Technology) (Grant No. KFJJ23-02M), the Foundation of Laboratory for Shock Wave and Detonation Physics, and Hebei Natural Science Foundation (Grant No. A2021409001), the Central Guidance on Local Science and Technology Development Fund of Hebei Province (Grant No. 226Z7601G), the ``Three, Three and Three Talent Project'' of Hebei Province (Grant No. A202105005), and the science foundation of NCIAE (Grant No. GFCXJJ-2023-01).


\begin{thebibliography}{99}\footnotesize
\itemsep=-1pt plus.2pt minus.1pt

\bibitem {Ranjan2011ARFM} Ranjan D, Oakley J, Bonazza R 2011 {\it Annual Review of Fluid Mechanics} {\bf 43} 117-140

\bibitem {Hwang2005} Hwang U, Flanagan K A, Petre R 2005 {\it The Astrophysical Journal} {\bf 635} 355-364

\bibitem {Diegelmann2017CNF} Diegelmann F, Hickel S, Adams N A 2017 {\it Combustion and Flame} {\bf 181} 300-314

\bibitem {Liu2023POP} Liu Y X, Chen Z, Wang L F, Li Z Y, Wu J F, Ye W H, Li Y J  2023 {\it Physics of Plasmas } {\bf xxx} xxx-xxx DOI:10.1063/5.0137856

\bibitem {Samtaney1994JFM} Samtaney R, Zabusky N J 1994 {\it Journal of Fluid Mechanics} {\bf 269} 45-78

\bibitem {Yang1995JFM} Yang J, Kubota T, Zukoski E E, 1995 {\it Journal of Fluid Mechanics} {\bf 258} 217-244

\bibitem {Layes2003PRL} Layes G, Jourdan G, Houas L 2003 {\it Physical Review Letters} {\bf 91} 174502

\bibitem {Ranjan2007PRL} Ranjan D, Niederhaus J, Motl B, Anderson M, Oakley J, Bonazza R 2007 {\it Physical Review Letters} {\bf 98} 024502

\bibitem {Picone1988JFM} Picone, J M and Boris, J P  1988 {\it Journal of Fluid Mechanics} {\bf 189} 23-51

\bibitem {Zou2015SCPMA} Zou L Y, Zhai Z G, Liu J H, Wang Y P, Liu C. L 2015 {\it Science China-Physics Mechanics \& Astronomy} {\bf 58} 124703

\bibitem {Sha2013APS} Sha S, Chen Z H, Xue D W  2013 {\it Acta Physica Sinica} {\bf 62} 144701

\bibitem {Sha2015APS} Sha S, Chen Z H, Zhang Q B  2015 {\it Acta Physica Sinica} {\bf 64} 015201

\bibitem {Yang2015CPL} Yang J, Wan Z H, Wang B F, Sun D J,  2015 {\it Chinese Physics Letters} {\bf 32} 034701

\bibitem {Ding2017JFM} Ding J C, Si T, Chen M J, Zhai Z G, Lu X Y, Luo X S  2017 {\it Journal of Fluid Mechanics} {\bf 828} 289-317

\bibitem {Ding2018POF} Ding J C, Si T, Chen M J, Zhai Z G, Luo X S  2019 {\it Physics of Fluids} {\bf 30} 106109

\bibitem {Si2014Si2014LPB} Si T, Zhai Z G, Luo X S 2014 {\it  Laser and Particle Beams} {\bf 32} 343-351

\bibitem {Si2012POF} Si T, Zhai Z G, Yang J M, Luo X S 2012 {\it Physics of Fluids} {\bf 24} 054101

\bibitem {Zhai2014JV} Zhai Z G, Zhang F, Si T, Luo X S 2014 {\it Journal of Visualization} {\bf 17} 123-129

\bibitem {Zhang2022POF} Zhang D J, Xu A G, Zhang Y D, Gan Y B, Li Y J 2022 {\it Physics of Fluids} {\bf 34} 086104

\bibitem {Zhang2019CNF} Zhang B, Chen H, Yu B, He M S, Liu H 2019 {\it Combustion and Flame} {\bf 208} 351–363

\bibitem {Lai2016PRE} Lai H L, Xu A G, Zhang G C, Gan Y B, Ying Y J, Succi S 2016 {\it Physical Review E} {\bf 94} 023106

\bibitem {Lin2017PRE} Lin C D, Xu A G, Zhang G C, Luo K H, Li Y J 2017 {\it Physical Review E} {\bf 96} 053305

\bibitem {Chen2018POF} Chen F, Xu A G, Zhang G C 2018 {\it Physics of Fluids} {\bf 30} 102105

\bibitem {Gan2019FOP} Gan Y B, Xu A G, Zhang G C, Lin C D, Lai H L, Liu Z P 2019 {\it Frontiers of Physics} {\bf 14}  43602

\bibitem {Chen2022PRE} Chen J, Xu A G, Zhang Y D, Chen Z H 2022 {\it Physical Review E} {\bf 106} 015102

\bibitem {Li2022CTP} Li H W, Xu A G, Zhang G, Shan Y M 2022 {\it Communications in Theoretical Physics} {\bf 74} 115601

\bibitem {Shan2022JMES} Shan Y M, Xu A G, Wang L F, Chen F 2022 {\it  Journal of Mechanical Engineering and Sciences} {\bf } 1-15

\bibitem {Zhang2021POF}  Zhang G, Xu A G, Zhang D J, Li Y J, Lai H L, Hu X M 2021 {\it  Physics of Fluids} {\bf 33}  076105

\bibitem {Singh2021POF}  Singh S, Battiato M, Myong R S 2021 {\it  Physics of Fluids} {\bf 33}  066103

\bibitem {Xu2022CMK}  Xu A G, Zhang Y D 2022 {\it  Complex Media Kinetics (in chinese) (Beijing: Science Press)}

\bibitem {Gan2022JFM}  Gan Y B, Xu A G, Lai H L, Li W, Sun G L, Succi S 2022 {\it  Journal of Fluid Mechanics} {\bf 951}  A8

\bibitem {Zhang2022AIP}  Zhang Y D, Xu A G, Chen F, Lin C D, Wei Z-H 2022 {\it  AIP Advances} {\bf 12}  035347.

\bibitem {Zhang2023CAF} Zhang D J, Xu A G, Song J H, Gan Y B, Zhang Y D, Li Y J  2023 {\it  	arXiv:2302.05687}.

\bibitem {Zhang2017FOP}  Zhang Y D, Xu A G, Zhang G C, Chen Z H, Wang P 2017 {\it  Frontiers of Physics} {\bf 13}  135101.

\bibitem {Zhang2010Soft-Matter}  Zhang Y D, Xu A G, Zhang G C, Gan Y B, Chen Z H, Succi S 2019 {\it  Soft Matter} {\bf 10}  5336-5345.


\end{thebibliography}
\end{document} 