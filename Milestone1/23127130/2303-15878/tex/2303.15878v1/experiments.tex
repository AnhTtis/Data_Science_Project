%!TeX root=main.tex

\section{Empirical Studies}
\label{sec:experiments}


\begin{figure}[ht]
\centering
\subfloat[14-node DT topology.]{\label{fig-AR-DT} \includegraphics[width=6cm]{Graphics/fig-AR-DT.eps}} 
\subfloat[50-node random topology.]{\label{fig-AR-50N} \includegraphics[width=6cm]{Graphics/fig-AR-50N.eps}}
\caption{Performance comparison on acceptance ratio.}
\label{fig_AR}
\end{figure}

Fig. \ref{fig-AR-DT} and Fig. \ref{fig-AR-50N} compare the acceptance ratio of requests when the Greedy-SP-FF, the LRC-SP-FF, the PL-KSP-FF and the proposed BiVNE algorithms are implemented in the 14-node DT and the 50-node random networks, respectively.
The acceptance ratio in all the algorithms decrease with the increasing number of the VNRs in these two physical networks.
This is due to the remaining resources in the physical network are gradually exhausted as the number of accepted VNRs increases.
Hence, more VNRs may be rejected because of the lack of physical resources.
Moreover, when compared to the other three algorithms, the proposed BiVNE algorithm consistently provides the highest acceptance ratio.
The reason is that BiVNE improves the probability of generating feasible embedding solutions through not only reducing the candidate node set before embedding the VNRs, but also employing a sorting method for solution construction in the upper-level optimization.
Furthermore, the BiVNE algorithm takes into account the variation of LoI at the PN and the optical spectrum fragmentation on the PL, so that the VNRs are always satisfied at the cost of the smallest amount of resource fragments.
As a result, more VNRs can be accepted due to the fact that more available resources in the physical network can be provided.

\begin{figure}[ht]
\centering
\subfloat[14-node DT topology.]{\label{fig-ave-path-hops-DT}
\includegraphics[width=6cm]{{Graphics/fig-ave-path-hops-DT.eps}}}
\subfloat[50-node random topology.]{\label{fig-ave-path-hops-50N}
\includegraphics[width=6cm]{{Graphics/fig-ave-path-hops-50N.eps}}}
\caption{Performance comparison on average path length.}
\label{fig_average_path_hops}
\end{figure}

Fig. \ref{fig-ave-path-hops-DT} and Fig. \ref{fig-ave-path-hops-50N} depict the performance comparison on the average path length obtained by conducting the four algorithms in the 14-node DT and the 50-node random networks, respectively.
We observe that the proposed BiVNE algorithm always obtains the shortest average path in any case in both the 14-node DT network (shown in Fig. \ref{fig-ave-path-hops-DT}) and the 50-node random network (shown in Fig. \ref{fig-ave-path-hops-50N}), thanks to the joint processing of VNoE and VLiE.
In contrast, the other three algorithms perform VNoE and VLiE separately.
PL-KSP-FF outperforms Greedy-SP-FF and LRC-SP-FF because, besides comparing the amount of resources provided by different PN candidates and their connected links, PL-KSP-FF takes the corresponding path length (i.e., the number of possible involved links) as an additional decision parameter in the procedure of VNoE.
However, the performance of PL-KSP-FF is still worse than that of the proposed BiVNE algorithm.
It is because, VLiE is performed under the only one resulting VNoE condition in the PL-KSP-FF algorithm, while more available VNoE choices are explored and provided for consideration when VLiE is decided in the BiVNE algorithm.
Apart from that, we observe that the average path length increases with the increasing number of VNRs in the proposed BiVNE algorithm, while this trend is not evident in the PL-KSP-FF algorithm.
The reason is, in addition to finding the shortest path like the PL-KSP-FF algorithm, the BiVNE algorithm tries to remain more continuous FSs in the physical networks during the embedding.
This leads to the fact that although the average path length in PL-KSP-FF is more stable, the proposed BiVNE algorithm is able to obtain a shorter average path while bringing a higher acceptance ratio than PL-KSP-FF.

\begin{figure}[ht]
\centering
\subfloat[14-node DT topology.]{\label{fig-R2C-DT}
\includegraphics[width=6cm]{{Graphics/fig-R2C-DT.eps}}}
\subfloat[50-node random topology.]{\label{fig-R2C-50N}
\includegraphics[width=6cm]{{Graphics/fig-R2C-50N.eps}}}
\caption{Performance comparison on R/C ratio.}
\label{fig_R2C}
\end{figure}

Fig. \ref{fig-R2C-DT} and Fig. \ref{fig-R2C-50N} present the R/C ratio results when the four algorithms are implemented in the 14-node DT network and the 50-node random network, respectively.
When the number of VNRs is less than 20, the R/C ratio of PL-KSP-FF is worse than that of LRC-SP-FF, while the opposite is true when the number of VNRs is greater than 20.
The reason is that instead of striving for the minimum resource consumption of a single VNR, PL-KSP-FF tends to accept more VNRs by balancing the computing resource against bandwidth.
It can be observed that BiVNE always keeps the highest R/C ratio, which means that BiVNE can consume fewer resources to obtain the same revenue. The reason is that the BiVNE algorithm always aims to not only achieve the lowest LoI at the physical nodes, but also allocate the shortest path for the virtual link and allocate spectrum in a way that produces the smallest number of optical spectrum fragments.

\begin{figure}[ht]
\centering
\subfloat[14-node DT topology.]{\label{fig-total-profit-DT}
\includegraphics[width=6cm]{{Graphics/fig-total-profit-DT.eps}}}
\subfloat[50-node random topology.]{\label{fig-total-profit-50N}
\includegraphics[width=6cm]{{Graphics/fig-total-profit-50N.eps}}}
\caption{Performance comparison on total profit.}
\label{fig_total_profit}
\end{figure}
Fig. \ref{fig-total-profit-DT} and Fig. \ref{fig-total-profit-50N} depict the total profit of the InP when the four algorithms are implemented in the 14-node DT network and the 50-node random network, respectively.
We observe that the total profit grows as satisfying more VNRs.
Compared with the three benchmark algorithms, a higher profit can always be achieved by the proposed BiVNE algorithm, benefitting from higher acceptance ratio and R/C ratio.
In addition, the superiority of BiVNE is more obvious in the 50-node random topology (Fig. \ref{fig-total-profit-50N}), especially when more VNRs arrive.
It illustrates our proposed algorithm plays better in large-scale networks.

