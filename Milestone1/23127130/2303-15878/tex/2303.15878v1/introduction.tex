%!TeX root=main.tex

\section{Introduction}
\label{sec:introduction}
Multi-access edge computing (MEC) systems, comprising MEC servers and base stations, are increasingly being employed as edge clouds to support various emerging applications that are delay-sensitive and computation-intensive \cite{MECETSI,MaoYZHL17, MachB17}.
A huge amount of data generated by these emerging applications require the collaboration of multiple MEC servers, prompting a lot of frequent information interactions between MEC systems \cite{LimLHJLYNM20}.
These MEC systems are inevitably interconnected by metro optical networks due to their high flexibility and high capacity \cite{ChatterjeeSO15}, which boosts the heterogeneity of network resources and therefore the difficulty and the complexity of allocating these multidimensional resources \cite{HuangYYZC20}.
Network slicing appears to resolve the network resource allocation problem efficiently by providing customized network slices (i.e., virtual networks) for service providers (SPs) to meet the need of diverse applications \cite{YangXGQCC21, WangWYH22}.
When receiving a request for a network slice from an SP, an infrastructure provider (InP) conducts a virtual network embedding (VNE) process, that is, allocates some physical nodes (PNs) and physical links (PLs) to the network slice to satisfy the corresponding resource requirements.
It has been confirmed that the VNE is an NP-hard problem \cite{FischerBBMH13}.

Previous research has extensively investigated the VNE problem in data center networks, considering congestion control \cite{PhamHC20}, energy consumption \cite{EramoMA16,ZhangCSL22,BillingsleyLMMG19,BillingsleyLMMG20,BillingsleyMLMG20,BillingsleyLMMG21}, failure avoidance  \cite{ShahriarACKBMZ20,AyoubBMT22,DehuryS22}, and profit  growth \cite{NguyenH22,SongCGYKZ21,GongJWZ16, ZhaoSB13}.
However, the VNE problem becomes more complex in edge cloud networks, which consist of multiple MEC-equipped base stations (i.e., edge nodes) connected by optical links.
First, limited resources in the edge network.
Second, optical spectrums in the optical links.
Optical spectrum allocation must adhere to spectrum continuity and consistency constraints, resulting in the generation of optical spectrum fragments at links \cite{ChatterjeeWO21}.
Third, different types of resources (i.e., communication and computing resources) provided by MEC-equipped base stations.
Unbalanced utilization of different types of resources leads to resource fragmentation at edge nodes.
As a result, the multidimensional resource fragments generated on links and nodes deteriorate the utilization of inherently limited edge resources, which significantly damages the profit of InP~\cite{ChenLY18,ZouJYZZL19,LiZZL09,LiZLZL09,Li19}.

To improve the efficiency of resource utilization, many works on VNE have been done in the optical and edge networks, respectively.
Regarding edge networks, both radio and computing resources are considered in \cite{joint22TNET, Dynamic20TVT, Time-Sensitive21TNSM}.
The authors of \cite{joint22TNET} investigated the problem of joint assignment of radio and computing resources to network slices, as well as the management of the two types of resources within each slice.
The authors of \cite{Dynamic20TVT} improved the operator's revenue through the joint control of the number of radio channels and the CPU clock speed.
The authors of \cite{Time-Sensitive21TNSM} investigated a service function chain placement problem to minimize service interruption while optimizing radio and computing resource utilization.
However, in those studies, the uneven utilization of radio and computing resources is ignored.
Moreover, these studies lack the consideration of optical link characteristics.
Regarding optical networks, some efforts have been made to avoid optical spectrum fragmentation \cite{ZhuZSLCG18, WeiGWYL19, FanXCCY21}. A common work in this studies is that a parameter related to the utilization of the optical spectrum is first defined, and then this parameter is used to guide the VNE process. In \cite{ZhuZSLCG18}, a fragmentation-aware VNE algorithm is proposed based on a virtual-auxiliary-graph approach.
A parameter related to the fragment size on the attached links of every PN is defined, and then the parameter is used to be the weight for performing virtual node embedding (VNoE) on every virtual-auxiliary-graph.
In \cite{WeiGWYL19}, a matching factor is defined according to the degree of contiguity of the free spectrum on the connected links of each PN.
After seeking the proper PNs by greedily using the matching factor, the virtual link embedding (VLiE) is conducted to find an available path in which the difference between the size of the selected spectrum block and that being requested is the smallest.
In \cite{FanXCCY21}, a VNE method that utilizes complete path evaluation and node proximity sensing is presented.
However, the VNoE and the VLiE are still performed separately, thus the ignorance of the coupling between these two potentially increases the probability of resource fragmentation.

In this paper, with the consideration of the multidimensional resource fragmentation and the dependency between the VNoE and the VLiE, we study the VNE problem in MEC systems interconnected by metro optical networks. Specifically, we jointly embed nodes and links with the goal of maximizing the profit of InP.
There are two key problems tackled.
Firstly, how to avoid the resource fragmentation on both PNs and PLs efficiently?
Secondly, how to take into full consideration of the dependency between VNoE and VLiE during the process of resource allocation? The main contributions of this paper are summarized as follows:
\begin{itemize}
\item[1)]An VNE problem for effective utilization of limited edge cloud resources is studied with the best-effort avoidance of multidimensional resource fragmentation. To quantify the multidimensional resource fragmentation, a level of imbalanced resource utilization and a threshold for judging whether a continuous free spectrum block is a fragment are defined to calculate the resource fragments at the PNs and PLs, respectively.
\item[2)]Considering the dependency between VNoE and VLiE, we transform the VNE problem into a bilevel problem, where the upper layer is the problem of selecting PNs for VNR, and the lower layer is the problem of link selection and spectrum resource allocation.
\item[3)]A nested bilevel VNE method (BiVNE) is proposed to solve the bilevel problem. Specifically, the upper layer problem is solved by a method based on ant colony system (ACS) and the lower layer problem is solved by the Dijkstra algorithm and an exact fit spectrum slot assignment method.
\item[4)]Extensive simulation results validate the performance of BiVNE. Compared with some state-of-the-art algorithms, BiVNE can improve the profit of InP through reducing resource consumption while increasing acceptance ratio.
\end{itemize}

The remainder of the paper is structured as follows. The network model and problem formulation are described in detail in Section \ref{sec:preliminaries}. The problem transformation and the proposed approach are described in Section \ref{sec:proposal}. The simulation setup and corresponding results are presented in section \ref{sec:settings} and section \ref{sec:experiments}. Section \ref{sec:conclusions} concludes this paper.
