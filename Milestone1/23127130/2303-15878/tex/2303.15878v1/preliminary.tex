%!TeX root=main.tex

\section{Preliminaries}
\label{sec:preliminaries}
\subsection{Physical Network}



A physical network topology can be represented as an undirected graph ${G^s}=({N^s},{E^s})$, where ${N^s}$ and ${E^s}$ respectively represent the sets of PNs and PLs which are optical.
 $|N^s|$ is the total number of the PNs.
 As shown in Fig. \ref{fig_VNE_model}, each PN ${n^s}\in{N^s}$ is initially equipped with an MEC server with computing capacity $C(n^s)$ and a macro base station with $W(n^s)$ wireless channels.
The geographical location of $n^s$ is $loc(n^s)$.
The entire spectrum in each optical link ${e^s } \in {E^s}$ is divided into $B(e^s)$ granular frequency slots (FSs).
The maximum available adjacent slot block (MACSB) $m_{e^s}\in M_{e^s}$ is the block of more than one unoccupied FS that exists in the optical spectrum.
We define ${P^s}$ as a set of circle-free paths in ${G^s}({V^s},{E^s})$.
The available bandwidth on path ${p^s \in P^s}$ can be computed by ${B(p^s) = {\min} \{B(e^s)\mid I_{e^s}^{p^s} = 1 \}}$, where $I_{e^s}^{p^s}$ is a binary variable indicating whether ${p^s}$ passes optical link ${e^s}$ (${I_{e^s}^{p^s} = 1}$), or not (${I_{e^s}^{p^s} = 0}$).

\begin{figure}[ht]
    \centering
    \includegraphics[width=80mm]{{Graphics/fig_VNE_model.eps}}
    \vspace{0cm}
    \caption {VNE example. }
    \label{fig_VNE_model}
\end{figure}

\subsection{Virtual Network Request (VNR)}
Similar to the physical network topology, a VNR topology can be also represented as an undirected graph ${G^r}=({N^r},{E^r})$. The set of VNRs denoted as $\Upsilon$. ${N^r}$ is the set of virtual nodes (VNs), and ${E^r}$ represents the set of virtual links (VLs). ${n^r}\in{N^r}$ represents a VN. $|N^r|$ is the total number of VNs in ${G^r}$. Each VN ${n^r}$ not only requires the amount of computing resource $C(n^r)$ and the number of wireless channels $W(n^r)$, but also prefers a corresponding physical area, where $loc(n^r)$ and $\Delta loc(n^r)$ are respectively the center and the radius of the area.
Thus, ${n^r}$ only can be embedded onto a set of candidate PNs within the preferred area, i.e.
$\Phi_N(n^r) = \{n^s\in{N^s} \mid dis[loc(n^r),loc(n^s)] \le \Delta loc(n^r)\}$,
 where $dis[loc(n^r),loc(n^s)]$ represents the distance between $n^r$ and $n^s$.
The amount of FSs required by VL ${e^r}$ is $B(e^r)$.
${s(e^r)}$ and ${t(e^r)}$ denote the end nodes of ${e^r}$.
Each VL ${e^r}$ is associated with a candidate physical path set, $\Phi_E(e^r) = \{p^s \mid B(p^s) \geq B(e^r),\  {s(p^s)} \in \Phi_N(s(e^r)),\  {t(p^s)}\in \Phi_N(t(e^r)) \}$. The main notations used in the paper are summarized in Table \ref{tab_notions}.


\begin{table}[ht]    % footnotesize,\small   % 开始一个表格environment,表格的位置是h,here。
\caption{List of Notations}  %显示表格的标题
\centering  %表居中\begin{tabular*}{\linewidth}{lp{3cm}p{4cm}p{4cm}}
   \begin{tabular}{c|lp{10cm}p{2cm}}   % {lccc} 表示各列元素对齐方式,left-l,right-r,center-c
        \toprule
        \textbf{Notation}       &\textbf{Description}\\
        \midrule

         ${G^s}=({N^s},{E^s})$       &Graph for physical network\\

         $n^s$         &Physical node, ${n^s}\in{N^s}$\\

         $e^s$         &Physical link, ${e^s } \in {E^s}$\\

         $loc(n^s)$    &Geographical location of ${n^s}$ \\

         $C(n^s)$      &Computing capacity of $n^s$ \\
         $C_l(n^s)$    &Occupied computing capacity of $n^s$\\

         $W(n^s)$      &Total number of wireless channels of $n^s$\\
         $W_l(n^s)$    &Number of occupied wireless channels of $n^s$\\

         $B(e^s)$      &Bandwidth of $e^s$ \\
         $P^s$         &Set of circle-free paths in ${G^s}$\\

         $M_{e^s}$     &Set of MACSBs on $e^s$, $m_{e^s}\in M_{e^s}$\\

         $s(m_{e^s})$,$t(m_{e^s})$  &Starting and ending FS indices of $m_{e^s}$ \\

         $f(m_{e^s})$  &Number of FSs in $m_{e^s}$ \\

         $\Xi_{max}$   &Maximum fragment size\\

         $\xi_{e^s}^k$ &$k_{th}$ spectral fragment on $e^s$\\

         ${G^r}=({N^r},{E^r})$       &Graph for VNR\\

         $C(n^r)$      &Computing resource required by $n^r$ \\

         $W(n^r)$      &Number of wireless channels required by $n^r$\\

         $loc(n^r)$    &Center of $n^r$'s preference area\\

         $\Delta loc(n^r)$ &Radius of $n^r$'s preference area \\

         $ B(e^r) $        &Amount of bandwidth required by $e^r$\\

         ${s(e^r)}$, ${t(e^r)}$    &End nodes of $e^r$ \\

         $I_{e^s}^{p^s}$    &Binary variable,  equals 1 if ${p^s}$ passes ${e^s}$\\

         $\nu^r$           &Binary variable, equals 1 if ${G^r}$ is successfully \\ & embedded\\

         ${x_{n^s}^{n^r}}$  &Binary variable, equals 1 if ${n^r}$ is embedded onto ${n^s}$\\

         ${y_{p^s}^{e^r}}$  &Binary variable, equals 1 if ${e^r}$ is embedded onto $p^s$\\

         ${z_{e^s,b}^{e^r}}$        &Binary variable, equals 1 if the $b_{th}$ FS on $e^s$ is \\ &  assigned to $e^r$\\

         ${\delta_{e^s,m}^{e^r}}$   &Binary variable, equals 1 if any FS of $m_{e^s}$ on $e^s$ \\ &is assigned to $e^r$\\

         \bottomrule

    \end{tabular}
    \label{tab_notions}
\end{table}


\subsection{VNE Process}
In this paper, we only consider transparent VNE in which not only the number of FSs on each optical link is same but also the number of FSs requested by each virtual link from a VNR is same \cite{ChatterjeeSO15}.
A VNE example is shown in Fig. \ref{fig_VNE_model}, where three layers are considered.
The bottom layer is the physical infrastructure layer, from which physical resources are abstracted to form a pool, working at the middle layer.
The top layer contains VNRs being satisfied by allocating PNs and PLs.
A VNE process includes VNoE and VLiE.
In Fig. \ref{fig_VNE_model}, the VNR$_1$ has the VNoE result $\{a\rightarrow E, b\rightarrow B  \}$, and {VNR}$_2$ has $\{c\rightarrow C, d\rightarrow B, e\rightarrow D\}$.
Additionally, the VNR$_1$ has the VLiE result $\{(a,b)\rightarrow \{(E,A),(A,B)\} \}$ using FSs (\#1, \#2), and VNR$_2$ has $\{(d,c)\rightarrow \{(B,C)\}, (c,e)\rightarrow \{(C,D) \}\}$ using FSs (\#4, \#5, \#6).

\subsection{InP's Revenue and Cost}
The InP will obtain revenue from fulfilling each VNR, shown as
\begin{equation}\label{eq-InP-R}
    \begin{split}
    {\mathbb{R}({G^r})} = {\sum\limits_{n^r \in N^r}}{ \left[\alpha C(n^r) + \kappa W(n^r)\right]} +
                       {\sum\limits_{e^r \in E^r}}{\gamma B(e^r)},
    \end{split}
\end{equation}
where ${\alpha}$, ${\kappa}$ and ${\gamma}$ represent the prices to be charged for per unit of computing resource, radio resource and optical spectrum resource, respectively.

The embedding cost is concerned with the physical resources, which contain not only the resources required by ${G^r}$, but also some resource fragments.
In VNoE, the consumed physical resources include not only the computing and radio resources allocated to ${n^r}$ but also the wasted resource caused by the resulting imbalance in the remaining two types of resources.
As shown in Fig. \ref{fig_VNE_model}(b), it is difficult to utilize the remaining resources at nodes B and C due to the imbalanced resource utilization.
We define the level of imbalance (LoI) at $n^s$ as
\begin{equation}\label{eq-LoI}
    {\varrho(n^s)} = \left | \frac{C_l(n^s)}{C(n^s)} -  \frac{W_l(n^s)}{W(n^s)} \right |,
\end{equation}
where $C_l(n^s)$ and $W_l(n^s)$ represent the occupied computing resource and wireless channels of $n^s$, respectively.
After $n^r$ is embedded, the LoI of $n^s$ changes into
\begin{equation}\label{eq-LoI-change}
\begin{split}
    {\varrho'(n^s)} = \left | \frac{C_l(n^s) + C(n^r)}{C(n^s)} -
       \frac{ W_l(n^s) + W(n^r)}{W(n^s)} \right |.
\end{split}
\end{equation}

The increase in LoI can be calculated as
\begin{equation}\label{eq-LoI-increase}
    {\Delta \varrho(n^s)} =
        \begin{cases}
        \varrho'(n^s) - \varrho(n^s), & \text{if ${ \varrho'(n^s) - \varrho(n^s) > 0}$,} \\
        0, & \text{otherwise.}
        \end{cases}
\end{equation}

Therefore, at the PNs involved in this embedding, the cost of the total resources actually consumed by ${N^r}$ is
\begin{equation}\label{eq-InP-Cn}
    \begin{split}
    {\mathbb{C}_n(G^r, x_{n^s}^{n^r})} = &{\sum\limits_{n^r \in N^r}}{\sum\limits_{n^s \in N^s}}{x_{n^s}^{n^r} \cdot (1 + \Delta \varrho(n^s))} \\
     &\cdot \left[\alpha' C(n^r) + \kappa' W(n^r)\right] ,
    \end{split}
\end{equation}
where binary variable $x_{n^s}^{n^r}$ indicates whether ${n^r}$ is embedded onto ${n^s}$ (${x_{n^s}^{n^r} = 1}$) or not (${x_{n^s}^{n^r} = 0}$), ${\alpha'}$ and ${\kappa'}$ represent the cost of per unit of computing and radio resources, respectively.

\begin{figure}[ht]
    \centering
    \includegraphics[width=80mm]{Graphics/fig_Newly_FS.eps}
    \caption {Examples of possible spectral fragments caused by embedding. }
    \vspace{0cm}
    \label{fig_Newly_FS}
\end{figure}

In VLiE, the consumed optical spectrum includes not only the FSs allocated to VLs but also some spectral fragments.
A spectral fragment $\xi_{e^s}^k$ is defined as an MACSB whose length is not more than the maximum fragment size ($f(m_{e^s})\leq \Xi_{max}$).
After a physical lightpath is established for $e^r$, spectral fragments may appear on the MACSBs where the FSs are assigned to $e^r$ along the physical lightpath.
Fig. \ref{fig_Newly_FS} shows examples of possible spectral fragments caused by embedding. It is seen that the spectral fragments appear either at one side of the allocated FSs or at both side of them. The number of FSs in the fragments introduced by the embedding of ${e^r}$ is denoted as $\xi_{e^s}^{e^r}$.

The cost of total spectrum resource consumed by ${E^r}$ is
\begin{small}
\begin{equation}\label{eq-InP-Ce}
    {\mathbb{C}_e(G^r, y_{p^s}^{e^r}, z_{e^s,b}^{e^r})} =  {\sum\limits_{e^r \in E^r}}{\sum\limits_{e^s \in E^s}}
    { y_{p^s}^{e^r}  I_{e^s}^{p^s}  \gamma'  \left [ \sum\limits_{b = 1}^{B(e^s)} z_{e^s,b}^{e^r} + \xi_{e^s}^{e^r}\right]},
\end{equation}
\end{small}
where binary variable $y_{p^s}^{e^r}$ indicates whether ${e^r}$ is embedded onto physical path $p^s$ (${y_{p^s}^{e^r} = 1}$) or not ($y_{p^s}^{e^r} = 0$), binary variable $z_{e^s,b}^{e^r}$ indicates whether the $b_{th}$ FS on optical link $e^s$ is assigned to $e^r$ ($z_{e^s,b}^{e^r} = 1$) or not ($z_{e^s,b}^{e^r} = 0$) and ${\gamma'}$ represents the cost of per unit of spectrum resource.

\subsection{VNE Problem Formulation}
The profit of all VNRs is affected by the number of VNRs accepted successfully, the revenue obtained from fulfilling VNRs and the embedding cost for VNoE and VLiE. Therefore, the embedding profit from all accepted VNRs is
\begin{small}
\begin{equation}\label{eq-InP-P}
 \begin{aligned}
        \mathbb{P}(&\Upsilon,\ x_{n^s}^{n^r},\  y_{p^s}^{e^r},\  z_{e^s,b}^{e^r}) \\
        = &{\sum\limits_{G^r \in \Upsilon}} {\nu^r} \left\{\mathbb{R}(G^r) - \mathbb{C}_n(G^r, x_{n^s}^{n^r}) - \mathbb{C}_e(G^r, y_{p^s}^{e^r}, z_{e^s,b}^{e^r}) \right\},
 \end{aligned}
\end{equation}
\end{small}
where binary variable $\nu^r$ indicates whether VNR ${G^r}$ is successfully embedded onto physical network ${G^s}$ (${\nu^r} = 1$) or not (${\nu^r} = 0$).

The following constraints apply to the VNE problem \cite{GuanZL0NR19}:

Each VN cannot be assigned to more than one PN. A PN cannot be assigned to two different VNs in the same VNR at the same time,
      \begin{small}
      \begin{equation}\label{eq-st-n-121}
        \text{C1}: \sum\limits_{{n^s} \in {N^s}} {x_{n^s}^{n^r}}  = 1, \quad \forall {n^r} \in {N^r},
      \end{equation}
      \begin{equation}\label{eq-st-n<1}
        \text{C2}: \sum\limits_{{n^r} \in {N^r}} {x_{n^s}^{n^r}} \le 1, \quad \forall {n^s} \in {N^s}.
      \end{equation}
      \end{small}

The PN assigned to the VN should have sufficient available resources,
\begin{small}
       \begin{equation}\label{eq-st-n-cc}
        \text{C3}: \sum\limits_{{n^s} \in {N^s}} {x_{n^s}^{n^r}}  \cdot C_a(n^s) \ge C(n^r), \quad  \forall {n^r} \in {N^r},
      \end{equation}
      \begin{equation}\label{eq-st-n-wc}
        \text{C4}: \sum\limits_{{n^s} \in {N^s}} {x_{n^s}^{n^r}}  \cdot W_a(n^s) \ge W(n^r), \quad  \forall {n^r} \in {N^r}.
      \end{equation}
\end{small}

A VN can only be assigned to a PN that is within its preferred aera,
      \begin{small}
       \begin{equation}\label{eq-st-n-lc}
        \begin{aligned}
            \text{C5}: \sum\limits_{{n^s} \in {N^s}} {x_{n^s}^{n^r}}  \Delta loc(n^r) \ge dis[loc(n^r),loc(n^s)],
            \forall {n^r} \in {N^r}.
        \end{aligned}
       \end{equation}
    \end{small}

 If $p^s$ is the light path established for a VL $e^r$, the two end nodes of $p^s$ must host the two end nodes of $e^r$.
    \begin{small}
      \begin{equation}\label{eq-st-l-endnodes}
        \text{C6}: y_{p^s}^{e^r} = x_{s(p^s)}^{s(e^r)} \cdot x_{t(p^s)}^{t(e^r)}, \quad \forall {e^r} \in {E^r}, \forall {p^s} \in {P^s}.
      \end{equation}
      \end{small}

The available spectrum resource of the physical lightpath allocated to a VL should not be less than the link's resource requirement,
     \begin{small}
      \begin{equation}\label{eq-st-l-bc}
        \begin{aligned}
            \text{C7}: \sum\limits_{{p^s} \in {P^s}} {y_{p^s}^{e^r} \cdot {B(p^s)}}  \ge {B(e^r)}, \quad \forall {e^r} \in {E^r}.
        \end{aligned}
      \end{equation}
    \end{small}

On each PL allocated to a VL, the number of the involved FSs should match the embedding requirement,
    \begin{small}
      \begin{equation}\label{eq-st-l-fsc}
        \begin{aligned}
            \text{C8}: \sum\limits_{b = 1}^{B(e^s)} {z_{e^s,b}^{e^r}}  = y_{p^s}^{e^r} \cdot I_{e^s}^{p^s} \cdot {B(e^r)}, \quad \forall {e^r} \in {E^r},
            \forall {e^s} \in {E^s}.
        \end{aligned}
       \end{equation}
    \end{small}

The FSs allocated to each VL must neighbor each other,
    \begin{small}
      \begin{equation}\label{eq-st-FS-neighbor}
            \begin{aligned}
            \text{C9}: t_b(e^r) - s_b(e^r) + 1 = B(e^r), \quad \forall {e^r} \in {E^r}.
            \end{aligned}
       \end{equation}
       \end{small}

The selected contiguous FSs on each optical link involved in the lightpath established for a VL should have the same indices,
     \begin{small}
      \begin{equation}\label{eq-st-FS-aligned}
        \begin{split}
            \text{C10}: \sum\limits_{b = 1}^{B(e^s)} {(z_{e^s,b}^{e^r} - z_{{e'}^s,b}^{e^r})} = 0, \text{ if }y_{p^s}^{e^r} \cdot I_{e^s}^{p^s} \cdot I_{{e'}^s}^{p^s} = 1,  \\ \quad \forall {e^r} \in {E^r}, \forall {e^s, {e'}^s} \in {E^s}.
        \end{split}
      \end{equation}
    \end{small}

The selected FSs by each VL must not conflict, that is, there is no overlapping FS for any two different VLs sharing common optical links,
   \begin{small}
    \begin{equation}\label{eq-st-FS-nonoverlapping}
        \begin{split}
            \text{C11}: \mathbb{I} \left(s_b(e^r) \leq s_b({e'}^r)\right) = \mathbb{I} \left( t_b(e^r) \geq t_b({e'}^r)\right ), \\
            \text{ if } y_{p^s}^{e^r} \cdot y_{p^s}^{{e'}^r} = 1, \forall {e^r, {e'}^r} \in {E^r}.
        \end{split}
    \end{equation}
    \end{small}

We jointly optimize both VNoE and VLiE to maximize the embedding profit from all the accepted VNRs. The joint VNoE and VLiE problem is expressed as follows:
\begin{equation}\label{eq-Problem-P}
 \begin{aligned}
    \mathcal{P: }\quad &\max \limits_{{ \textbf{x, y, z}}}\   {\sum\limits_{G^r \in \Upsilon}} {\nu^r} \left\{\mathbb{R}(G^r) - \mathbb{C}_n(G^r, \textbf{x}) - \mathbb{C}_e(G^r, \textbf{y}, \textbf{z}) \right\} \\
                 &\mathrm{ s.t. }  \ \text{C1} - \text{C11}.
 \end{aligned}
\end{equation}

We can observe that the problem $\mathcal{P}$ is a integer nonlinear optimization problem.
It is challenging to solve $\mathcal{P}$ if using conventional optimization techniques.
Furthermore, the performance of VNoE cannot be fully evaluated until the corresponding VLiE is complete.
Therefore, in order to optimize VNE, all the available VNoE options need to be considered and consequently all the corresponding VLiE options need to be listed for assessment. The most appropriate VNoE and VLiE decision will be made based on its profit contribution. In brief, VNoE and VLiE are closely coupled and mutually dependent.
Therefore, there are two issues in efficiently solving $\mathcal{P}$.
\begin{enumerate}
    \item Is it possible to convert this nonlinear integer optimization problem into another straightforward one?
    \item How can we take into account the interdependence of VNoE and VLiE?
\end{enumerate}