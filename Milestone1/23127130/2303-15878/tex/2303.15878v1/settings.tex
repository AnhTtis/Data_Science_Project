%!TeX root=main.tex

\section{Experimental Setup}
\label{sec:settings}
In this section, evaluations are performed to assess the performance of the proposed BiVNE algorithm. We tested two sizes of physical networks:
a realistic Deutsche Telecom (DT) network with 14 nodes and 23 links, and a random network with 50 nodes and 166 links. All the PNs are arranged in a 1000*1000 coverage square.
The number of VNs contained in each VNR is randomly generated within a predefined range, see Table \ref{Simulation_parameters} for specific settings.
The probability of generating a VL between every two VNs is 0.5.
${\alpha = \kappa = \gamma =3}$, and ${\alpha' = \kappa' = \gamma' = 1}$ .

In the BiVNE algorithm, the population size is $NP = 10$, maximum generation number is $gen_{max} = 150$, and the other parameters are set in accordance with \cite{DorigoG97}:
$\beta = 2$, $q_0 = 0.9$, $\varphi = \rho = 0.1$, and $\tau_{0} = (|N^s| \cdot Cost)^{-1} $, where $Cost$ is the total cost produced by the greedy algorithm \cite{ZhaoSB13}.
Table \ref{Simulation_parameters} summarizes the remaining simulation parameters.


\begin{table}[t] \small   % footnotesize   % 开始一个表格environment,表格的位置是h,here。
\caption{Simulation Parameters}  %显示表格的标题
\centering  % 表居中\begin{tabular*}{\linewidth}{lp{3cm}p{4cm}p{4cm}}
   \begin{tabular}{c|lp{3cm}p{2cm}}   % {lccc} 表示各列元素对齐方式,left-l,right-r,center-c

        \toprule

        Parameters          &DT Topology    &Random Topology\\
        \midrule

         $|N^s|$            &14             &50\\%Number of PNNs
         $|E^s|$            &23             &166 \\%Number of PNLs
         $C(n^s)$           &[50,100]units  &[50,100]units \\%Computing capacity of PNNs
         $W(n^s)$           &[50,100]       &[50,100] \\%Wireless channels of PNNs
         $B(e^s)$           &[50,100]FS'    &[50,100]FS' \\%Bandwidth capacity of PNLs
         $|N^r|$            &[3,4]          &[3,10] \\%Number of VNNS in a VN
         $\Delta loc(n^r)$  &[200,300]      &[200,300] \\
         $C(n^r)$           &[1,10]units    &[1,20]units \\%Computing requirement of VNNs
         $W(n^r)$           &[1,10]         &[1,20] \\%Wireless channel requirement of VNNs
         $B(e^r)$           &[1,10]FS'      &[1,20]FS' \\%Bandwidth requirement of VNLs

         \bottomrule

    \end{tabular}
    \label{Simulation_parameters}
\end{table}

To assess the performance of the designed algorithm, BiVNE, we compare it with three existing VNE algorithms: Greedy-SP-FF \cite{ZhaoSB13}, LRC-SP-FF \cite {GongWZL14}, and PL-KSP-FF \cite{FanXCCY21}.
