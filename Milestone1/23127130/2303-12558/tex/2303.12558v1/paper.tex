\documentclass{article}

% if you need to pass options to natbib, use, e.g.:
%     \PassOptionsToPackage{numbers, compress}{natbib}
% before loading neurips_2021

% ready for submission
\usepackage{iclr2023_conference,times}
\iclrfinalcopy

% to compile a preprint version, e.g., for submission to arXiv, add add the
% [preprint] option:
%     \usepackage[preprint]{neurips_2021}

% to compile a camera-ready version, add the [final] option, e.g.:
%     \usepackage[final]{neurips_2021}

% to avoid loading the natbib package, add option nonatbib:
%    \usepackage[nonatbib]{neurips_2021}

\usepackage[utf8]{inputenc} % allow utf-8 input
\usepackage[T1]{fontenc}    % use 8-bit T1 fonts
\usepackage{hyperref}       % hyperlinks
\usepackage{url}            % simple URL typesetting
\usepackage{booktabs}       % professional-quality tables
\usepackage{amsfonts}       % blackboard math symbols
\usepackage{nicefrac}       % compact symbols for 1/2, etc.
\usepackage{microtype}      % microtypography
\usepackage{xcolor}

% === Start Florent's packages ===
% Optional math commands from https://github.com/goodfeli/dlbook_notation.
\newcommand{\bbox}{\text{bbox}}
\newcommand{\alphapck}{\alpha_\bbox}
\newcommand{\kcycle}{\text{k-CyPCK}}
\newcommand{\cycle}{\text{-CyPCK}}

\newcommand{\I}{\mathbf{I}}
\newcommand{\Ia}{\I^\text{a}}
\newcommand{\Ib}{\I^\text{b}}
\newcommand{\Iatob}{\I^\text{a $\rightarrow$ b}}
\newcommand{\F}{\mathbf{F}}
\newcommand{\Fa}{\F^\text{a}}
\newcommand{\Fb}{\F^\text{b}}
\newcommand{\f}{\mathbf{f}}
\newcommand{\fa}{\f^\text{a}}
\newcommand{\fb}{\f^\text{b}}
\newcommand{\p}{\mathbf{p}}
\newcommand{\pa}{\p^\text{a}}
\newcommand{\pb}{\p^\text{b}}
\newcommand{\A}{\boldsymbol{\Phi}_\text{align}}
\newcommand{\G}{\mathbf{G}}
\newcommand{\C}{\mathbf{C}}
\newcommand{\Ca}{\C^\text{a}}
\newcommand{\Cb}{\C^\text{b}}
\newcommand{\cc}{\mathbf{c}}
\newcommand{\cca}{\cc^\text{a}}
\newcommand{\ccb}{\cc^\text{b}}
\newcommand{\Irec}{\I_\text{Recon}}
\newcommand{\M}{\mathbf{M}}
\newcommand{\Mrec}{\M_\text{Recon}}
\newcommand{\loss}{\mathcal{L}}
\newcommand{\T}{\mathcal{T}}
\newcommand{\W}{\mathcal{W}}
\newcommand{\Id}{\mathcal{I}}


\usepackage{nameref}
\usepackage{ifthen}
\usepackage{braket}
\usepackage{amssymb,amsthm}
\usepackage{thm-restate}
\usepackage{mathtools}
\usepackage{graphicx}
\usepackage{scalerel}
\usepackage{stmaryrd}
\usepackage{centernot}
% \usepackage{todonotes}
\usepackage{subcaption}
\usepackage{mathabx}
\usepackage{multibib}
\usepackage{enumerate}
%\usepackage{algorithm}
%\usepackage[noend]{algpseudocode}
\usepackage[ruled, vlined]{algorithm2e}
\usepackage{setspace}
\usepackage{dsfont}
\usepackage{tikz}{\tiny }
\usepackage{wrapfig}
\usepackage[stdsubgroups,nocfg]{nomencl}
\usetikzlibrary{decorations,arrows,shapes,automata,calc}
\usepackage{enumitem}

\newcites{AR}{Additional References}

\newtheorem{theorem}{Theorem}[section]
\newtheorem{corollary}{Corollary}[theorem]
\newtheorem{lemma}[theorem]{Lemma}
\newtheorem{assumption}[theorem]{Assumption}
\theoremstyle{remark}
\newtheorem{remark}{Remark}
%\theoremstyle{remark}
\newtheorem{example}{Example}

\newcommand{\GAP}[1]{\todo[inline]{\textcolor{black}{\textbf{Guillermo: }#1}}}
\newcommand{\AN}[1]{\todo[inline]{\textcolor{blue}{\textbf{Ann: }#1}}}
\newcommand{\FD}[1]{\todo[inline]{\textcolor{white}{\textbf{Florent: }#1}}}
% \presetkeys{todonotes}{disable}{}
\definecolor{OliveGreen}{HTML}{3C8031}
\definecolor{BrickRed}{HTML}{B6321C}

% commands
\newcommand{\fix}{\marginpar{FIX}}
\newcommand{\new}{\marginpar{NEW}}
\newcommand{\smallparagraph}[1]{\smallskip\noindent\textbf{#1}}
\newcommand{\tabitem}{~~\llap{\textbullet}~~}
\makeatletter
\newcommand{\pushright}[1]{\ifmeasuring@#1\else\omit\hfill$\displaystyle#1$\fi\ignorespaces}
\newcommand{\pushleft}[1]{\ifmeasuring@#1\else\omit$\displaystyle#1$\hfill\fi\ignorespaces}
\makeatother

% numbers
\newcommand{\tuple}[1]{\ensuremath{\left\langle #1 \right\rangle}}
\newcommand{\indexof}[2]{\ensuremath{#1_{\scriptscriptstyle #2}}}
\newcommand{\N}{\mathbb{N}}
\newcommand{\fun}[1]{\ensuremath{\mathopen{}\mathclose\bgroup\left(#1\aftergroup\egroup\right)}}
% \newcommand{\condition}[1]{\ensuremath{\mathds{1}_{#1}}}
\newcommand{\condition}[1]{\ensuremath{\1_{#1}}}
\newcommand{\vect}[1]{\ensuremath{\bm{#1}}}
\newcommand{\entrywise}[3]{\ensuremath{\tuple{#1}_{#2 \in [#3]}}}
\newcommand{\images}[1]{\ensuremath{\mathrm{Im}\fun{#1}}}
\newcommand{\diam}[1]{\ensuremath{\mathrm{diam}\fun{#1}}}

% MDPs
\newcommand{\mdp}{\ensuremath{\mathcal{M}}}
\newcommand{\states}{\ensuremath{\mathcal{S}}}
\newcommand{\actions}{\ensuremath{\mathcal{A}}}
\newcommand{\transitions}{\ensuremath{\mathcal{T}}}
%\newcommand{\probtransitions}[1]{\ensuremath{\mathbf{P}}^{\scriptscriptstyle\transitions}\ifthenelse{\equal{#1}{}}{}{_{\scriptscriptstyle#1}}}
\newcommand{\probtransitions}{\ensuremath{\mathbf{P}}} %^{\transitions}}
\newcommand{\rewards}{\ensuremath{\mathcal{R}}}
\newcommand{\labels}{\ensuremath{\ell}}
\newcommand{\atomicprops}{\ensuremath{\mathbf{AP}}}
\newcommand{\sinit}{\ensuremath{s_{\mathit{I}}}}
\newcommand{\mdptuple}{\langle \states, \actions, \probtransitions, \rewards, \labels, \atomicprops, \sinit \rangle}
\newcommand{\mdpspace}{\ensuremath{\mathbb{M}}}
\newcommand{\statesmdptuple}[2]{\ensuremath{ \tuple{#1, \actions, \transitions_{\scriptscriptstyle#1}, \rewards_{\scriptscriptstyle#1}, \labels_{\scriptscriptstyle#1}, \atomicprops, #2} }}
\newcommand{\previousstate}{\ensuremath{\state^{\scriptscriptstyle -1}}}
\newcommand{\previousaction}{\ensuremath{\action^{\scriptscriptstyle -1}}}
\newcommand{\state}{\ensuremath{s}}
\newcommand{\vstate}{\ensuremath{\vs}}
\newcommand{\action}{\ensuremath{a}}
\newcommand{\vaction}{\ensuremath{\va}}
\newcommand{\reward}{\ensuremath{r}}
\newcommand{\labeling}{\ensuremath{l}}
\newcommand{\labelset}[1]{\ensuremath{\mathsf{#1}}}
\newcommand{\actbot}{\ensuremath{\bot}}
\newcommand{\act}[1]{\ensuremath{\mathit{Act}\ifthenelse{\equal{#1}{}}{}{(#1)}}}
\newcommand{\trajectories}[1]{\ensuremath{\mathit{Traj}^{fin}_{ #1 }}}
% \newcommand{\inftrajectories}[1]{\ensuremath{\mathit{Traj}_{#1}}}
\newcommand{\inftrajectories}[1]{\ensuremath{\mathit{Traj}}}
\newcommand{\seq}[2]{\ensuremath{#1_{\scriptscriptstyle 0:#2}}}
\newcommand{\trajectory}{\tau}
\newcommand{\trajectorytuple}[3]{\ensuremath{\tuple{#1_{\scriptscriptstyle 0:#3}, #2_{\scriptscriptstyle 0: #3-1}}}}
\newcommand{\trace}{\ensuremath{\hat{\trajectory}}}
\newcommand{\tracetuple}[5]{\ensuremath{\tuple{#1_{\scriptscriptstyle 0:#5}, #2_{\scriptscriptstyle 0: #5-1}, #3_{\scriptscriptstyle 0:#5 - 1}, #4_{\scriptscriptstyle 0: #5}}}}
\newcommand{\defaulttrace}{\ensuremath{\tracetuple{\state}{\action}{\reward}{\labeling}{T}}}
\newcommand{\traces}[1]{\ensuremath{\mathit{Traces}_{#1}}}
\newcommand{\policy}{\ensuremath{\pi}}
\newcommand{\policies}[1]{\ensuremath{\Pi_{#1}}}
\newcommand{\mpolicies}[1]{\ensuremath{\Pi}}
% \newcommand{\mpolicies}[1]{\ensuremath{\Pi_{#1}^{\textnormal {ml}}}}
\newcommand{\safepolicy}[1]{\ensuremath{\Pi_{#1}^{\textnormal {safe}}}}
\newcommand{\steadystate}[1]{\ensuremath{\xi_{#1}}}
\newcommand{\car}{\ensuremath{\scaleto{\rotatebox[origin=c]{90}{$\circlearrowright$}}{1.4ex}}}
\newcommand{\selfloop}[2]{\ensuremath{#1^{\nobreak\hspace{.065em} \car #2 }}}
\newcommand{\getstates}[1]{\ensuremath{\llbracket #1 \rrbracket}}
\newcommand{\valuessymbol}[2]{\ensuremath{V_{#1}^{#2}}}
\newcommand{\values}[3]{\ensuremath{\valuessymbol{#1}{#2}\fun{#3}}}
\newcommand{\approxvalues}[3]{\ensuremath{\hat{V}_{#1}^{#2}\fun{#3}}}
\newcommand{\episodereturn}[1]{\ensuremath{\mathbf{R}_{#1}}}
\newcommand{\qvaluessymbol}[2]{\ensuremath{Q_{#1}^{#2}}}
\newcommand{\qvalues}[4]{\ensuremath{\qvaluessymbol{#1}{#2}\fun{#3, #4}}}
\newcommand{\objective}{\ensuremath{\Omega}}
\newcommand{\bscc}[2]{\ensuremath{\mathcal{B}^{#1}_{#2}}}
\newcommand{\stationary}[1]{\ensuremath{\xi_{#1}}}
\newcommand{\bisimulation}{\ensuremath{\mathcal{B}}}

% VAE
\newcommand{\encodersymbol}{\ensuremath{Q}}
\newcommand{\encoder}{\ensuremath{\encodersymbol_\encoderparameter}}
\newcommand{\encoderinit}{\ensuremath{\encodersymbol_{\encoderparameter, I}}}
\newcommand{\decodersymbol}{\ensuremath{P}}
\newcommand{\decoder}{\ensuremath{\decodersymbol_\decoderparameter}}
% \newcommand{\latentbeliefspace}{\ensuremath{\mathcal{Q}}}
% \newcommand{\transitiontolatent}{\ensuremath{P^{\scriptscriptstyle \transitions}_{\scalebox{.8}{$\scriptscriptstyle \states, \latentstates$}}}}
% \newcommand{\sreset}{\ensuremath{\state_{\text{reset}}}}
% \newcommand{\actreset}{\ensuremath{\action_{\text{reset}}}}
\newcommand{\latenttransitions}{\ensuremath{\transitions_\theta}}
\newcommand{\encoderparameter}{\ensuremath{\iota}}
\newcommand{\decoderparameter}{\ensuremath{\theta}}
\newcommand{\generative}{\ensuremath{\mathcal{G}}}
\newcommand{\wassersteinparameter}{\ensuremath{\omega}}


% Objectives
\newcommand{\discount}{\ensuremath{\gamma}}
\newcommand{\discountedreturn}[1]{\ensuremath{\mathit{Disc}_{#1}}}
\newcommand{\totalreturn}{\ensuremath{\rewards^{\scriptscriptstyle \infty}}}
\newcommand{\unsafe}{\ensuremath{\mathit{Unsafe}}}
\newcommand{\safe}{\ensuremath{\mathit{Safe}}}
\newcommand{\always}{\ensuremath{\square}}
\newcommand{\eventually}{\ensuremath{\lozenge}}
\newcommand{\ltlnext}{\ensuremath{\bigcirc}}
\newcommand{\policyequiv}{\ensuremath{R}}
\newcommand{\until}[2]{\ensuremath{#1 \, \mathcal{U} \, #2}}

% Probability and information theory
\newcommand{\Prob}{\ensuremath{\displaystyle \mathbb{P}}}
%\newcommand{\Pr}{\ensuremath{\mathrm{Pr}}}
\newcommand{\measurableset}{\ensuremath{\mathcal{X}}}
\newcommand{\varmeasurableset}{\ensuremath{\mathcal{Y}}}
\newcommand{\borel}[1]{\ensuremath{\Sigma\fun{#1}}}
\newcommand{\sample}[2]{\ensuremath{\displaystyle #1 \sim #2}}
%\newcommand{\sampledot}{\ensuremath{\boldsymbol{\cdotp}}}
\newcommand{\sampledot}{\ensuremath{{\cdotp}}}
\newcommand{\expectedsymbol}[1]{\ensuremath{\mathop{\mathbb{E}}\ifthenelse{\equal{#1}{}}{}{_{#1}}}}
\newcommand{\expected}[2]{\ensuremath{\expectedsymbol{#1} \left[ #2 \right]}}
\newcommand{\expectmdp}[3]{\ensuremath{\displaystyle  \E_{#1}^{#2} [ #3 ]}}
\newcommand{\entropy}[1]{\ensuremath{\displaystyle H(#1)}}
\newcommand{\divergencesymbol}{\ensuremath{D}}
\newcommand{\divergence}[2]{\ensuremath{\divergencesymbol\fun{#1, #2}}}
\newcommand{\dklsymbol}{\ensuremath{\divergencesymbol_{{\mathrm{KL}}}}}
\newcommand{\dkl}[2]{\ensuremath{\dklsymbol\fun{#1 \parallel #2}}}
\newcommand{\dtvsymbol}{\ensuremath{d_{{TV}}}}
\newcommand{\dtv}[2]{\ensuremath{\dtvsymbol\fun{#1, #2}}}
\newcommand{\mean}{\ensuremath{\vmu}}
\newcommand{\covar}{\ensuremath{\mSigma}}
\newcommand{\normal}[3]{\ensuremath{\displaystyle \mathcal{N} ( #1; #2, #3 )}}
% \newcommand{\distributions}[1]{\ensuremath{\mathcal{P}\fun{#1}}}
\newcommand{\distributions}[1]{\ensuremath{\Delta\fun{#1}}}
%\newcommand{\support}[1]{\ensuremath{Supp\fun{#1}}}
\newcommand{\mode}[1]{\ensuremath{\textit{mode}\fun{#1}}}
\newcommand{\distortion}{\ensuremath{\mathbf{D}}}
\newcommand{\rate}{\ensuremath{\mathbf{R}}}
\newcommand{\dataentropy}{\ensuremath{\mathbf{H}}}
\newcommand{\elbo}{\ensuremath{\mathrm{ELBO}}}
\newcommand{\error}{\ensuremath{\varepsilon}}
\newcommand{\proberror}{\ensuremath{\delta}}

% Distributions
\newcommand{\logistic}[2]{\ensuremath{\mathrm{Logistic}(#1, #2)}}
\newcommand{\bernoulli}[1]{\ensuremath{\mathrm{Bernoulli}(#1)}}
\newcommand{\categorical}[1]{\ensuremath{\mathrm{Categorical}(#1)}}
\newcommand{\relaxedcategorical}[2]{\ensuremath{\mathrm{RelaxedCategorical}(#1, #2)}}
\newcommand{\relaxedbernoulli}[2]{\ensuremath{\mathrm{RelaxedBernoulli}(#1, #2)}}
\newcommand{\logit}{\ensuremath{{\alpha}}}
\newcommand{\logits}{\ensuremath{\vect{\logit}}}
\newcommand{\temperature}{\ensuremath{\lambda}}

% Wasserstein
\newcommand{\wassersteinsymbol}[1]{\ensuremath{W}_{#1}}
\newcommand{\wassersteindist}[3]{\ensuremath{\wassersteinsymbol{#1}\left( #2, #3 \right)}}
\newcommand{\distance}{\ensuremath{d}}
\newcommand{\logic}{\ensuremath{\mathcal{L}}}
\newcommand{\tightoverset}[2]{%
  \tikz[baseline=(X.base),inner sep=0pt,outer sep=0pt]{%
    \node[inner sep=0pt,outer sep=0pt] (X) {$#2$}; 
    \node[yshift=1.2pt] at (X.north) {$#1$};
}}
\newcommand{\bidistance}{\ensuremath{\tightoverset{\,\scriptstyle\sim}{\distance}}}
\newcommand{\logicbidistance}{\ensuremath{\tilde{\distance}^{\logic}}}
%\newcommand{\functionalexpr}{\ensuremath{\mathcal{F}_{\discount}^{\logic}}}
\newcommand{\functionalexpr}{\ensuremath{\mathcal{F}}}
%\newcommand{\diam}{\ensuremath{\textit{diam}}}
%\newcommand{\couplings}[2]{\ensuremath{\Lambda\fun{#1, #2}}}
\newcommand{\couplings}[2]{\ensuremath{\Lambda(#1, #2)}}
\newcommand{\coupling}{\ensuremath{\lambda}}
\newcommand{\proj}{\ensuremath{\Pi}}
\newcommand{\Lipschf}[1]{\ensuremath{\mathcal{F}_{#1}}}
%\newcommand{\Lipschf}[1]{\ensuremath{\norm{\cdot}_{#1}} \leq 1}
%\newcommand{\tracedistance}{\ensuremath{\hat{\distance}}}
\newcommand{\tracedistance}{\transitiondistance}
\newcommand{\transitiondistance}{\ensuremath{\ensuremath{\vec{\distance}}}}
\newcommand{\waemdp}{W$^2\!$AE-MDP$\,$}
\newcommand{\waemdps}{W$^2\!$AE-MDPs$\,$}
\newcommand{\ncritic}{\ensuremath{m}}

% Deep MDPs
\newcommand{\overbar}[1]{\mkern 1.5mu\overline{\mkern-1.5mu#1\mkern-1.5mu}\mkern 1.5mu}
\newcommand{\overbarit}[1]{\,\overline{\!{#1}}}
\newcommand{\embed}{\ensuremath{\phi}}
\newcommand{\embeda}{\ensuremath{\psi}}
% \newcommand{\latentembeda}{\ensuremath{\overbarit{\embeda}}}
\newcommand{\actionencoder}{\ensuremath{\embed_{\encoderparameter}^{\scriptscriptstyle\actions}}}
% \newcommand{\transitionencoder}{\ensuremath{\vec{\embed}}}
\newcommand{\transitionencoder}{\ensuremath{q}}
\newcommand{\latentmdp}{\ensuremath{\overbarit{\mdp}}}
\newcommand{\latentprobtransitions}{\ensuremath{\overbar{\probtransitions}}}
\newcommand{\latentstates}{\ensuremath{\overbarit{\mathcal{\states}}}}
\newcommand{\latentrewards}{\ensuremath{\overbarit{\rewards}}}
\newcommand{\latentlabels}{\ensuremath{\overbarit{\labels}}}
\newcommand{\latentmdptuple}{\ensuremath{\tuple{\latentstates, \latentactions, \latentprobtransitions, \latentrewards, \latentlabels, \atomicprops, \zinit}}}
\newcommand{\latentstate}{\ensuremath{\overbarit{\state}}}
\newcommand{\zinit}{\ensuremath{\latentstate_I}}
\newcommand{\latentactions}{\ensuremath{\overbarit{\actions}}}
\newcommand{\latentaction}{\ensuremath{\overbarit{\action}}}
\newcommand{\latentvaluessymbol}[2]{\overbarit{\ensuremath{V}}_{#1}^{#2}}
\newcommand{\latentvalues}[3]{\ensuremath{\latentvaluessymbol{#1}{#2}\fun{#3}}}
\newcommand{\latentqvaluessymbol}[2]{\ensuremath{\overbarit{Q}_{#1}^{#2}}}
\newcommand{\latentqvalues}[4]{\ensuremath{\latentqvaluessymbol{#1}{#2}\fun{#3, #4}}}
\newcommand{\latentpolicy}{\ensuremath{\overbar{\policy}}}
\newcommand{\latentpolicies}{\ensuremath{\overbar{\Pi}}}
% \newcommand{\latentmpolicies}{\ensuremath{\overbar{\Pi}^{\textnormal {ml}}}}
\newcommand{\latentmpolicies}{\ensuremath{\overbar{\Pi}}}
\newcommand{\localtransitionloss}[1]{L_{\probtransitions}^{#1}}
\newcommand{\localrewardloss}[1]{L_{\rewards}^{#1}}
\newcommand{\localtransitionlossupper}[1]{\dot{L}_{\probtransitions}^{#1}}
\newcommand{\localrewardlossupper}[1]{\dot{L}_{\rewards}^{#1}}
\newcommand{\localtransitionlossapprox}[1]{\hat{L}_{\probtransitions}^{#1}}
\newcommand{\localrewardlossapprox}[1]{\hat{L}_{\rewards}^{#1}}
\newcommand{\KV}{\ensuremath{K_{\latentvaluessymbol{}{}}}}
\newcommand{\KR}[1]{\ensuremath{\ifthenelse{\equal{#1}{}}{K_{\latentrewards}}{K_{\latentrewards}^{#1}}}}
\newcommand{\KP}[1]{\ensuremath{\ifthenelse{\equal{#1}{}}{K_{\latentprobtransitions}}{K_{\latentprobtransitions}^{#1}}}}
\newcommand{\Rmax}[1]{\ensuremath{\ifthenelse{\equal{#1}{}}{|\latentrewards^\star|}{|\latentrewards_{#1}^{\star}|}}}
\newcommand{\stationarydecoder}{{\stationary{\decoderparameter}}}
\newcommand{\latentstationaryprior}{{\bar{\xi}_{\latentpolicy_{\decoderparameter}}}}
\newcommand{\norm}[1]{\ensuremath{\left\| #1 \right\|}}
\newcommand{\gradient}{\ensuremath{\nabla}}
\newcommand{\latentvariables}{\ensuremath{\mathcal{Z}}}
\newcommand{\latentvariable}{\ensuremath{z}}
% \newcommand{\originaltolatentstationary}[1]{{\latentprobtransitions_{\embed_{\encoderparameter}\stationary{\ifthenelse{\equal{#1}{}}{\policy}{#1}}}}}
\newcommand{\originaltolatentstationary}[1]{\ensuremath{\mathcal{T}}}
\newcommand{\varsampledfrom}[3]{\ensuremath{#1}_{{#2}, {#3}}}
\newcommand{\steadystateregularizer}[1]{\ensuremath{\mathcal{W}_{\stationary{#1}}}}
\newcommand{\steadystatenetwork}{\ensuremath{\varphi_{\wassersteinparameter}^{\stationary{}}}}
% \newcommand{\steadystatenetwork}{\ensuremath{f_{\wassersteinparameter}}}
\newcommand{\transitionlossnetwork}{\ensuremath{\varphi_{\wassersteinparameter}^{\probtransitions}}}
% \newcommand{\transitionlossnetwork}{\ensuremath{\varphi_{\wassersteinparameter}}}

% Experience replay
\newcommand{\replaybuffer}{\ensuremath{\mathcal{D}}}
\newcommand{\experience}{\ensuremath{\eta}}
\newcommand{\loss}{\ensuremath{L}}

% === Index of Notations ===
\renewcommand{\nomname}{Index of Notations}
\renewcommand{\nomgroup}[1]{%
\ifthenelse{\equal{#1}{M}}{%
      \item[\textbf{Markov Decision Processes}]}{%
        \ifthenelse{\equal{#1}{K}}{%
        \item[\textbf{P}]}{%
          \ifthenelse{\equal{#1}{P}}{%
          \item[\textbf{Probability / Measure Theory}]}{%
            \ifthenelse{\equal{#1}{L}}{%
            \item[\textbf{Latent Space Model}]}{%
              {%
              \ifthenelse{\equal{#1}{W}}{%
            \item[\textbf{Wasserstein Auto-encoded MDP}]}{%
              }}}}}}}
%\usepackage{setspace}
    \makenomenclature

%

% === End Florent's packages ===
\title{Wasserstein Auto-encoded MDPs\\\small Formal Verification of Efficiently Distilled RL Policies with Many-sided Guarantees}

% The \author macro works with any number of authors. There are two commands
% used to separate the names and addresses of multiple authors: \And and \AND.
%
% Using \And between authors leaves it to LaTeX to determine where to break the
% lines. Using \AND forces a line break at that point. So, if LaTeX puts 3 of 4
% authors names on the first line, and the last on the second line, try using
% \AND instead of \And before the third author name.

\author{%
    Florent Delgrange \\
    AI Lab, Vrije Universiteit Brussel (VUB) \\ University of Antwerp\\
    \texttt{florent.delgrange@ai.vub.ac.be}
    \And
    Ann Now\'e \\
    AI Lab, VUB
    \And
    Guillermo A. P\'erez \\
    University of Antwerp \\ Flanders Make
}

\begin{document}
\maketitle
\begin{abstract}
Although deep reinforcement learning (DRL) has many success stories, the large-scale deployment of policies learned through these advanced techniques in safety-critical scenarios is hindered by their lack of formal guarantees.
Variational Markov Decision Processes (VAE-MDPs) are discrete latent space models that provide a reliable framework for distilling formally verifiable controllers from any RL policy.
While the related guarantees address relevant practical aspects such as the satisfaction of performance and safety properties, the VAE approach suffers from several learning flaws (posterior collapse, slow learning speed, poor dynamics estimates), primarily due to the absence of abstraction and representation guarantees to support latent optimization.
We introduce the Wasserstein auto-encoded MDP (WAE-MDP), a latent space model that fixes those issues by minimizing a penalized form of the optimal transport between the behaviors of the agent executing the original policy and the distilled policy, for which the formal guarantees apply.
Our approach yields bisimulation guarantees while learning the distilled policy, allowing concrete optimization of the abstraction and representation model quality.
Our experiments show that, besides distilling policies up to 10 times faster, the latent model quality is indeed better in general.
Moreover, we present experiments from a simple time-to-failure verification algorithm on the latent space. The fact that our approach enables such simple verification techniques highlights its applicability.
\end{abstract}
\section{Introduction}
\section{Introduction}

The increasing complexity of source code poses a key challenge to the reliability of large-scale software systems. Software bugs in these systems can lead to safety issues~\cite{bug_safety} for users around the world as well as cause non-negligible financial losses~\cite{bug_loss}. As such, developers have to spend a large amount of time and effort on bug fixing. Consequently, \aprfull (\apr), designed to automatically generate patches to fix software bugs, has attracted wide attention from both academia and industry~\cite{long2016prophet, legoues2012genprog, long2015spr, lou2020can, tufano2018empstudy}. 


To achieve \apr, one popular approach is known as Generate-and-Validate (G\&V)~\cite{qi2015gv, ghanbari2019prapr, lou2020can, le2016hdrepair, legoues2012genprog, wen2018capgen, hua2018sketchfix, martinez2016astor, koyuncu2020fixminder, liu2019tbar, liu2019avatar}, which is typically based on the following pipeline: First, fault localization techniques~\cite{wong2016fl, abreu2007ochiai, zhang2013injecting, papadakis2015metallaxis, li2019deepfl, li2017transforming} are applied to determine the suspicious locations in programs where bugs are likely to exist. Then, the buggy locations are used by the \apr tools to generate a list of patches that replace buggy lines with correct lines. Afterward, each patch is validated against the original test suite to identify any \emph{plausible patches} (i.e., passing all tests in the test suite). Finally, to determine the \emph{correct patches}, developers examine the list of plausible patches to see if any of them can correctly fix the bug. 

Traditional \apr tools can mainly be categorized into heuristic-based~\cite{legoues2012genprog, le2016hdrepair, wen2018capgen}, constraint-based~\cite{mechtaev2016angelix, le2017s3, demacro2014nopol, long2015spr} and \template~\cite{ghanbari2019prapr, hua2018sketchfix, martinez2016astor, liu2019tbar, liu2019avatar}. Among these traditional tools, \template \apr tools~\cite{ghanbari2019prapr, liu2019tbar, benton2020effectiveness} have been able to achieve state-of-the-art results. \Template \apr tools typically leverage pre-defined templates (e.g., adding a nullness check) for bug fixing. However, since these fix templates are typically handcrafted, the number and types of bugs they are able to fix can be limited. 



To address the limitations of traditional \apr, researchers have proposed various \learning \apr tools~\cite{li2020dlfix, chen2018sequencer, jiang2021cure, lutellier2020coconut, zhu2021recoder, ye2022rewardrepair} based on the \nmtfull (\nmt) architecture~\cite{sutskever2014mt} where the input is the buggy code snippets and the goal is to translate the buggy code snippets into a fixed version. To accomplish this, \learning \apr tools require supervised training datasets with pairs of both buggy and fixed code snippets in order to learn how to perform this translation step. These training data are usually obtained by mining historical bug fixes using heuristics/keywords~\cite{dallmeier2007benchmark}, which can be imprecise for identifying bug-fixing commits; even the actual bug-fixing commits can include irrelevant code changes, leading to further pollution in the dataset~\cite{xia2022alpharepair}.
% 
Moreover, it can be hard for such \apr tools to generalize and fix bug types unseen during training. 



To better leverage recent advances in \plmfull{s} (\plm{s}), researchers~\cite{xia2022alpharepair, xia2023repairstudy, kolak2022patch, prenner2021codexws} have directly applied \plm{s} to generate patches without bug-fixing datasets. These \llm-based \apr tools work by either directly generating a complete code function~\cite{prenner2021codexws, xia2023repairstudy} or predict/infill the correct code snippet given its surrounding context~\cite{xia2022alpharepair, xia2023repairstudy}. By directly using \llm{s} that are pre-trained on billions of open-source code snippets, \llm-based \apr tools can achieve state-of-the-art performance on many repair datasets~\cite{xia2022alpharepair}. 


% 
%
%

Traditional \apr tools have long used the insight of the \emph{plastic surgery hypothesis}~\cite{barr2014plastic} where it states that the code ingredients to fix a bug already exist within the same project. Traditional \apr tools have manually designed pattern-~\cite{ghanbari2019prapr, saha2017elixir} or heuristic-based~\cite{jiang2018simfix, legoues2012genprog} approaches to finding and using such relevant code ingredients to generate fixes for bugs. However, the plastic surgery hypothesis has been largely ignored in \llm-based \apr. In fact, \llm provides a unique opportunity to fully automate the plastic surgery hypothesis idea via fine-tuning (learning project-specific information via model updates from the buggy project) and prompting (directly providing relevant code ingredients to the model), and make it directly applicable to different languages (since the \llm{s} are typically multi-lingual).%
Moreover, despite the intensive manual efforts involved, traditional \apr tools still cannot fully leverage project-specific information due to large search space for leveraging/composing existing code ingredients. In contrast, the project-specific information can effectively leveraged by \llm{s} due to their power in code understanding/vectorization, e.g., even partial/imprecise information may still guide \llm{s} in correct patch generation!
 To this end, we ask the question: \emph{How useful is the plastic surgery hypothesis in the era of \plm{s}}?








\mypara{Our Work.} To answer the question, we present \ourtech{\xspace} -- a \llm-based approach that automatically utilizes the plastic surgery hypothesis by systematically combining multiple fine-tuning and prompting strategies for \apr. \ourtech fine-tunes \plm{s} using two novel domain-specific training strategies: \textbf{\epfinetune} -- we fine-tune using the original buggy project by aggressively masking out a high percentage of tokens, which allows \plm to learn project-specific code tokens and programming styles; and \textbf{\rofinetune} -- which only masks out a single continuous code sequence per training sample, allowing the model to get used to the final \csapr task of predicting a single continuous code sequence. Furthermore, we directly leverage the ability for \plm{s} to understand natural language instructions and introduce a novel prompting strategy, \textbf{\idprompting}, which uses information retrieval and static analysis to obtain a list of relevant identifiers for the buggy lines. While such relevant identifiers are critical for fixing some difficult bugs, they may not be seen by the \llm during inference due to limited context window size. Through the use of prompting, we directly tell the model to use these extracted identifiers (relevant code ingredients) to generate the correct code. Finally, to perform repair, we combine all four model variants (including the base model, both fine-tuned models and the base model with prompting) for the final repair.





While our insight of leveraging the plastic surgery hypothesis for \llm-based \apr is generalizable across different types of \plm{s}, to implement \ourtech, we choose a recent \plm{\xspace}, \ctfive~\cite{wang2021codet5}, which is pre-trained on millions of open-source code snippets. \ctfive is an encoder-decoder model trained using \mspfull (\msp) objective where a percentage of tokens are masked out and each continuous masked token sequence is referred to as a masked span. Also, although we only extract relevant identifiers from the current buggy project (since this paper focuses on the plastic surgery hypothesis), our work can be easily extended to obtain other code information (such as relevant statements or functions) from other sources, such as  the massive pre-training corpora~\cite{husain2020codesearchnet} or historical bug-fixing datasets~\cite{jiang2019infer}, which can provide more coding knowledge for \llm{s}. Besides, although we mainly focus on using traditional string comparison algorithms for information retrieval in this paper, these techniques can be easily replaced by other frequency-based retrieval~\cite{robertson2009probabilistic} and neural search (or embedding-based search)~\cite{reimers2019sentence}.
  In summary, this paper makes the following contributions:


%


\begin{itemize}[noitemsep, leftmargin=*, topsep=0pt]
    \item \textbf{Dimension.} This paper is the first to revisit the important plastic surgery hypothesis in the era of \llm{s}. It opens up a new dimension for \llm-based \apr to incorporate previously neglected information from the buggy project itself to boost \apr performance. Furthermore, it demonstrates the promising future of retrieval-based prompting for modern \llm-based \apr.
    \item \textbf{Implementation.} We implement \ourtech based on the recent \ctfive model. We augment the model using two novel fine-tuning strategies: \epfinetune and \rofinetune, along with a novel prompting strategy based on information retrieval and static analysis: \idprompting. We combine the patches generated by all four models together and perform patch ranking to speed up \apr.% 
    \item \textbf{Evaluation Study.} We conduct an extensive evaluation against state-of-the-art \apr tools. On the widely studied \dfj 1.2 and 2.0 datasets~\cite{just2014dfj}, \ourtech is able to achieve the new state-of-the-art results of 89 and 44 correct bug fixes (15 and 8 more than best baseline) respectively.  Furthermore, we perform a broad ablation study to justify our design. \ourtech demonstrates for the first time that the plastic surgery hypothesis can substantially boost \llm-based \apr and advance state-of-the-art \apr, while being fully automated and general. Moreover, even partial/imprecise code ingredients may still effectively guide \llm{s} for \apr!
\end{itemize}


\section{Background}\label{sec:background}
\section{Background on Network Calculus}
\label{sec: background}


\begin{figure*}[tbh]
\centering
\begin{subfigure}[b]{0.3\textwidth}
    \centering
    \includegraphics[width=\linewidth]{images/in-out.png}
    \caption{Arrival and departure data and their relation with delay $d(t)$ and backlog $b(t)$. For a FIFO system, the delay is the horizontal distance between $R(t)$ and $R^*(t)$ but some other multiplexing techniques may shift the data to a later priority, causing a longer delay.}
    \label{fig: data in-out}
\end{subfigure}
\hfill
\begin{subfigure}[b]{0.35\textwidth}
    \centering
    \includegraphics[width=\linewidth]{images/arrival-service.png}
    \caption{Characteristics of an arrival curve and a service curve. From any point of observation, the arriving data never exceeds its arrival curve; the departure data is also never less than the service curve with respect to the data arrival.}
    \label{fig: arrival-service curves}
\end{subfigure}
\hfill
\begin{subfigure}[b]{0.33\textwidth}
    \centering
    \includegraphics[width=\linewidth]{images/bound.png}
    \caption{Delay and backlog bounds of a system. Backlog is the maximum vertical distance between $\alpha(t)$ and $\beta(t)$; FIFO delay is their maximum horizontal distance; but for arbitrary multiplexing, the delay guarantee is when the system clears its buffer, thus it's the intersection of $\alpha(t)$ and $\beta(t)$.}
    \label{fig: system bounds}
\end{subfigure}
\caption{Network calculus framework. We let $R(t)$ and $R^*(t)$ be the arrival and departure data flow of a system; $\alpha(t)$ be the piecewise linear concave arrival curve and $\beta(t)$ be the piecewise linear convex service curve of a system.}
% \hossein{Better to show piece-wise linear concave arrival curve and piece-wise linear convex service curve instead of token-bucket and rate-latency.}}
\end{figure*}

We recall some of the network calculus essentials for a better understanding of the framework used in Saihu. In the following context, we use the following notation: $\mbb{R}^+$ is the set of non-negative real numbers; $[x]_+$ denotes $\max(0, x)$

The data flow is by convention modeled as a left-continuous wide-sense increasing function $R(t): \mbb{R}^+ \mapsto \mbb{R}^+$ with respect to time $t$~\cite{ncbook2001leboudec}. 

A system $\mcal{S}$ receives arrival data described as a cumulative function $R(t)$ and delivers departure data as another cumulative function $R^*(t)$. Figure~\ref{fig: data in-out} illustrates such a system $\mcal{S}$. The benefit of representing a system like this is that we can observe system backlog and delay with such a model. 

\begin{definition}[Backlog and Delay~\cite{ncbook2001leboudec}]
    The backlog of a system at time~$t$ is
    \begin{equation}
        b(t) = R(t) - R^*(t)
    \end{equation}
    
    The virtual delay of a FIFO system at time $t$ is
    \begin{equation}
        d_{FIFO}(t) = \inf \lbp \tau \geq 0 : R(t) \leq R^*(t+\tau) \rbp
    \end{equation}
\end{definition}



The backlog of a system can be viewed as the vertical distance between $R$ and $R^*$. The FIFO (\textit{First-in First-out}) delay is the horizontal distance between $R$ and $R^*$. One may obtain other delay values if the multiplexing technique is not FIFO.

% \begin{figure}
%     \centering
%     \includegraphics[width=0.9\linewidth]{images/in-out.png}
%     \caption{In/out data flow; delay and backlog}
%     \label{fig: data in-out}
% \end{figure}

Since we are interested in the system guarantee instead of a single instance of data flow, we would like to have general bounds to the arrival and departure data flows. Therefore, we define \textit{arrival curve} and \textit{service curve} as the bounds of arrival and departure data flows.

\begin{definition}[Arrival Curve~\cite{ncbook2001leboudec}]
    Given a wide-sense increasing function $\alpha: \mbb{R}^+ \mapsto \mbb{R}^+$, we say that a flow $R(t)$ is $\alpha$-constrained if and only if for all $s \leq t$:
    \begin{equation}
        R(t) - R(s) \leq \alpha(t-s)
    \end{equation}
    We say $R(t)$ has $\alpha$ as an arrival curve.
\end{definition}

\begin{definition}[Service Curve~\cite{ncbook2001leboudec}]
    Given a wide-sense increasing function $\beta: \mbb{R}^+ \mapsto \mbb{R}^+$ and $\beta(0) = 0$. A system $\mcal{S}$ having $R(t)$ and $R^*(t)$ as its arrival and departure flows. We say $\mcal{S}$ offers a service curve $\beta$ if and only if
    \begin{equation}
        R^*(t) \geq (R \otimes \beta)(t) =: \inf_{s \leq t} \lbp R(s) + \beta(t-s) \rbp
    \end{equation}
    where $\otimes$ denotes the min-plus convolution
\end{definition}

Figure~\ref{fig: arrival-service curves} illustrates the arrival and service curves. Any segment of arrival flow $R(t)$ is constrained by arrival curve $\alpha$ and the output curve $R^*(t)$ is always no less than the curve $R\otimes\beta$. As a result, an arrival curve upper bounds the incoming traffic, and a service curve lower bounds the outgoing traffic.

% \begin{figure}
%     \centering
%     \includegraphics[width=\linewidth]{images/arrival-service.png}
%     \caption{Arrival/Service curve}
%     \label{fig: arrival-service curves}
% \end{figure}

We consider 2 special types of curves throughout this paper, \textit{token-bucket} (or sometimes called \textit{leaky-bucket}) curve and \textit{rate-Latency} curve.

\begin{definition}[Token-bucket and Rate-latency~\cite{ncbook2001leboudec}]
    A token-bucket curve $\gamma_{r,b}$ with arrival rate $r$ and burst $b$ is defined as
    \begin{equation}
        \gamma_{r,b}(t) = b + rt
    \end{equation}

    A rate-latency curve $\beta_{R,T}$ with service rate $R$ and latency $T$ is defined as
    \begin{equation}
        \beta_{R,T}(t) = R \lb t - T \rb_+
    \end{equation}
\end{definition}

A token-bucket curve is determined by a burst $b$ and an arrival rate~$r$. Burst represents the maximum possible data volume that can arrive simultaneously, and arrival rate represents the maximum long-term data rate~\cite{bouillard2022tradeoff}.
A rate-latency curve is determined by a latency~$T$ and a service rate~$R$. Latency represents the time a server needs before starting to process the incoming data, and service rate represents the minimum rate to process data after the initial latency.

With the help of arrival and service curves, we can derive delay and backlog bounds for a system $\mcal{S}$ illustrated in Figure~\ref{fig: system bounds}. Suppose a system $\mcal{S}$ has arrival curve $\alpha$ and service curve~$\beta$, its worst-case backlog $b^*$ is the maximum vertical distance between~$\alpha$ and~$\beta$. Similarly, depending on the multiplexing technique applied to the system, its worst-case delay bound $d^*$ is the maximum horizontal distance between $\alpha$ and $\beta$ if $\mcal{S}$ is a FIFO system. If we don't have any information about its multiplexing technique, referred to as arbitrary multiplexing, the best we can say is that when $\alpha$ and $\beta$ intersect each other, where all data has been delivered out of the system. Consequently, the worst-case delay bound for arbitrary multiplexing is the time required for $\mcal{S}$ to clear its buffer.

% \begin{figure}
%     \centering
%     \includegraphics[width=\linewidth]{images/bound.png}
%     \caption{System delay/backlog bounds}
%     \label{fig: system bounds}
% \end{figure}

While a service curve captures the slowest possible output speed of a system, a link's transmission capacity limits the speed as well. Hence, we model this phenomenon using a \textit{greedy shaper} with a sub-additive function $\sigma: \mbb{R}^+ \mapsto \mbb{R}^+$ concatenated with a server. We consider a concatenation as shown in Figure \ref{fig: system}. By convention we assume $\sigma(0) = 0$ and $\beta(t) \leq \sigma(t), \forall t \in \mbb{R}^+$, meaning that the buffer is cleared at the beginning and the service never exceed its physical limitation. With the above definition, such greedy shaper conserves the service provided by the system due to theorem \ref{thm: shaping}.

\begin{figure}[thb]
    \centering
    \includegraphics[width=0.7\linewidth]{images/system.png}
    \caption{Shaping of departure data. A flow that has an arrival curve $\alpha$ feeds into a server with an arrival data flow $R(t)$. The server having service curve $\beta$ takes $R(t)$ and gives a departure data flow $R^*(t)$ to a shaper with shaping function $\sigma$. The shaper takes $R^*(t)$ and shape the data flow as another departure $D(t)$.}
    \label{fig: system}
\end{figure}


\begin{theorem}[Shaping conserves service \cite{ncbook2001leboudec}]
\label{thm: shaping}
Following the system shown in Figure \ref{fig: system}, we have
\begin{equation}
     D = R^* \otimes \sigma \geq \lp R \otimes \beta \rp \otimes \sigma = R \otimes \lp \beta \otimes \sigma \rp = R \otimes \beta
\end{equation}
\end{theorem}

In the following context, we model the shaping function $\sigma$ as a token-bucket curve $\gamma_{C,L}$ with transmission capacity $C$ and the packet size $L$ to capture the link capacity and packetization~\cite{bouillard2022tradeoff}.

\section{Wasserstein Auto-encoded MDPs}
Fix $ \latentmdp_{\decoderparameter} = \tuple{\latentstates, \latentactions, \latentprobtransitions_{\decoderparameter}, \latentrewards_{\decoderparameter}, \latentlabels, \atomicprops, \zinit}$ and $\tuple{\latentmdp_{\decoderparameter}, \embed_{\encoderparameter}, \embeda_{\decoderparameter}}$ as a latent space model of $\mdp$ parameterized by $\encoderparameter$ and $\decoderparameter$.
Our method relies on learning a \emph{behavioral model}
% $\decoder$
$\stationary{\decoderparameter}$
of $\mdp$ from which we can retrieve the latent space model 
% the parameters of $\latentmdp_{\decoderparameter}$
and distill $\pi$.
This can be achieved via the minimization of a suitable discrepancy between $\stationarydecoder$ and
%a distribution describing the dynamics of $\mdp_\policy$.
% a trace distribution of 
$\mdp_\policy$.
%$\decoder$.
% $\stationary{\decoderparameter}$.
% \begin{equation}
%     \min_{\decoderparameter}D\fun{{\mdp_\policy}, {\decoder}}. \label{eq:discrepancy}
% \end{equation}
%As mentioned in \citep{DBLP:journals/corr/abs-2112-09655},
VAE-MDPs optimize a lower bound on the likelihood of the dynamics of $\mdp_\policy$ using the \emph{Kullback-Leibler divergence},
%one can exploit the VAE framework \cite{DBLP:journals/jmlr/HoffmanBWP13, DBLP:journals/corr/KingmaW13} to
yielding (i) $\latentmdp_{\decoderparameter}$, (ii) a distillation  $\latentpolicy_{\decoderparameter}$ of $\policy$, and (iii) $\embed_{\encoderparameter}$ and $\embeda_{\decoderparameter}$.
%
Local losses are not directly minimized, but rather variational proxies that do not offer theoretical guarantees during the learning process.
%(the log-likelihood of the rewards and the replacement of the Wasserstein term in $\localtransitionloss{\latentpolicy_{\decoderparameter}}$ by $\dklsymbol$) when the latent policy is executed.
To control the local losses minimization and exploit their theoretical guarantees, 
%and avoid this way undesired learning behavior,
we present a novel autoencoder that incorporates them in its objective, derived from the OT. % instead of $\dklsymbol$.
% In the following, we describe how to achieve this goal by considering the minimization of the Wasserstein distance between the .
%
% \subsection{State space abstraction}
Proofs of the claims made in this Section are provided in Appendix~\ref{appendix:wae-mdp}.
\subsection{The Objective Function}
%\smallparagraph{Raw transition distance.}~
Assume that $\states$, $\actions$, and $\images{\rewards}$ are respectively equipped with metrics $\distance_{\states}$, $\distance_{\actions}$, and $\distance_{\rewards}$,
%let us define the \emph{raw transition distance metric} $\transitiondistance$ over 
we define the  \emph{raw transition distance metric} $\transitiondistance$ as the component-wise sum of distances between states, actions, and rewards occurring of along transitions:
$
%\begin{equation}
    \tracedistance\fun{\tuple{\state_1, \action_1, \reward_1, \state'_1}, \tuple{\state_2, \action_2, \reward_2, \state'_2}} = \distance_\states\fun{\state_1, \state_2} + \distance_{\actions}\fun{\action_1, \action_2} + \distance_{\rewards}\fun{\reward_1, \reward_2} + \distance_{\states}\fun{\state_1', \state'_2}.\notag
%\end{equation}
$
%
%possible transitions of $\mdp_\policy$, i.e., tuples of the form $\tuple{\state, \action, \reward, \state', \labeling'}$ such that $\exists \defaulttrace \in \traces{\mdp_\policy}$, $\exists t < T$ with $\tuple{\state, \action, \reward, \state', \labeling'} = \tuple{\state_t, \action_t, \reward_t, \state_{t + 1}, \labeling_{t + 1}}$, as
% \begin{align*}
%     \transitiondistance\fun{\tuple{\state_1, \action_1, \reward_1, \state_1', \labeling_1'}, \tuple{\state_2, \action_2, \reward_2, \state_2', \labeling_2'}} = \distance_{\actions}\fun{\action_1, \action_2} + \left| \reward_1 - \reward_2 \right| + \distance_\states\fun{\state'_1, \state'_2} + \distance_{\labels}\fun{\labeling'_{1}, \labeling'_{2}}.
% \end{align*}
%We naturally extend the metric to
%execution traces from $\bigcup_{\state \in \states}\traces{\mdp_{\state, 
% \policy}}$ by accumulating the distance of transitions occurring along them:
%
%$\states \times \actions \times \images{\rewards} \times \states$ as
%
%   \begin{align*}
%       % &\tracedistance\fun{\defaulttrace, \tracetuple{\state'}{\action'}{\reward'}{\labeling'}{T'}} \\
%       % =& \begin{cases}
%       %     \displaystyle \sum_{t \in [T - 1]} \distance_\states\fun{\state_{t + 1}, \state'_{t + 1}} + \distance_{\actions}\fun{\action_t, \action'_t} + \left| \reward_t - \reward'_t \right| + \distance_{\labels}\fun{\labeling_{t + 1}, \labeling'_{t + 1}} & \text{if } T = T'\\[0.4em]
%       %     \begin{aligned}
%       %         \displaystyle \tracedistance\fun{\tracetuple{\state}{\action}{\reward}{\labeling}{T^\star}, \tracetuple{\state'}{\action'}{\reward'}{\labeling'}{T^\star}} \quad \quad \quad \\[0.2em] + \left| T - T' \right| \fun{\diam{\states} + \diam{\actions} + \diam{\images{\rewards}} + \diam{2^{\atomicprops}}}
%       %     \end{aligned}
%       %     & \text{otherwise, with } T^{\star} = \min\set{T, T'}.
%       % \end{cases}
%       &\tracedistance\fun{\defaulttrace, \tracetuple{\state'}{\action'}{\reward'}{\labeling'}{T'}} \\
%       =& 
%           \displaystyle \left| T - T' \right| D + \!\!\!\!\!\! \sum_{t \in \left[\min \set{T, T'} - 1 \right]} \!\!\!\!\!\! { \distance_\states\fun{\state_{t}, \state'_{t}} + \distance_{\actions}\fun{\action_t, \action'_t} + \left| \reward_t - \reward'_t \right| + \distance_{\labels}\fun{\labeling_{t + 1}, \labeling'_{t + 1}} + \distance_\states\fun{\state_{t + 1}, \state'_{t + 1}} },
%   \end{align*}
%   where $D = 2\, \diam{\states} + \diam{\actions} + \diam{\images{\rewards}} + \diam{2^{\atomicprops}}$.
%
%
%\begin{equation}
%    \tracedistance\fun{\tuple{\state_1, \action_1, \reward_1, \state'_1}, \tuple{\state_2, \action_2, \reward_2, \state'_2}} = \distance_\states\fun{\state_1, \state_2} + \distance_{\actions}\fun{\action_1, \action_2} + \distance_{\rewards}\fun{\reward_1, \reward_2} + \distance_{\states}\fun{\state_1', \state'_2}.\notag
%\end{equation}
%
%
%Using $\tracedistance$, one can also measure the distance between transitions by considering traces of unit size.
% \begin{equation*}
%     \tracedistance\fun{\trace, \trace'} = \sum_{t = 0}^{T - 1} \distance_\states\fun{\state_{t + 1}, \state'_{t + 1}} + \distance_{\actions}\fun{\action_t, \action'_t} + \left| \reward_t - \reward'_t \right| + \distance_{\labels}\fun{\labeling_{t + 1}, \labeling'_{t + 1}}.
% \end{equation*}
% Then, %given $T \in \N$,
% we are considering optimizing
% \begin{equation*}
%     \wassersteindist{\tracedistance}{\mdp_{\policy}}{\decoder} = \! \sup_{f \in \Lipschf{\tracedistance}} \expectedsymbol{\trace \sim \mdp_{\policy}} f\fun{\seq{\state}{T}, \seq{\action}{T - 1}, \seq{\reward}{T - 1}, \seq{\labeling}{T}} - \!\! \expectedsymbol{\trace \sim \decoder} f\fun{\seq{\state}{T}, \seq{\action}{T - 1}, \seq{\reward}{T - 1}, \seq{\labeling}{T}}. \label{eq:trace-wasserstein}
%     %    \wassersteindist{\tracedistance}{\mdp_{\policy}[T]}{\decoder} = \! \sup_{f \in \Lipschf{\tracedistance}} \expectedsymbol{\trace \sim \mdp_{\policy}[T]} f\fun{\seq{\state}{T}, \seq{\action}{T - 1}, \seq{\reward}{T - 1}, \seq{\labeling}{T}} - \!\! \expectedsymbol{\trace \sim \decoder} f\fun{\seq{\state}{T}, \seq{\action}{T - 1}, \seq{\reward}{T - 1}, \seq{\labeling}{T}}. \label{eq:trace-wasserstein}
% \end{equation*}
% Alternatively, the following Lemmae enable optimizing the Wasserstein distance between the original MDP and the behavioral model %Eq.~\ref{eq:trace-wasserstein}
% when traces are drawn from episodic RL processes or infinite interactions \cite{DBLP:conf/nips/Huang20}.
%
%When traces are drawn from infi e consider optimizing the distance between transition distributions
%
\iffalse
\smallparagraph{Trace distributions.}~%Instead of reasoning over full traces generated from $\mdp_\policy$, we enable a transition-based optimization by considering the distribution over transitions which may occur along traces of size $T$:
The raw distance $\tracedistance$ allows to reason about \emph{transitions}, we thus consider the distribution over \emph{transitions which occur along trajectories of length $T$} to compare the dynamics of the original and behavioral models:
%We consider the following distributions over transitions which may occur along trajectories of length $T$:
\begin{align*}
    \mathcal{D}_\policy\left[ T \right]\fun{\state, \action, \reward, \state'} &= \frac{1}{T} \sum_{t = 1}^{T} \stationary{\policy}^t\fun{\state \mid \sinit} \cdot \policy\fun{\action \mid \state} \cdot \probtransitions\fun{\state' \mid \state, \action} \cdot \condition{=}\fun{\reward, \rewards\fun{\state, \action}}, \text{ and} \\
    \mathcal{P}_\decoderparameter[T]\fun{\state, \action, \reward, \state'} &= \frac{1}{T} \sum_{t = 1}^{T} \expectedsymbol{\seq{\state}{t}, \seq{\action}{t - 1}, \seq{\reward}{t - 1} \sim \decoder[t]}{\condition{=}\fun{\tuple{\indexof{\state}{t - 1}, \indexof{\action}{t - 1}, \indexof{\reward}{t - 1}, \indexof{\state}{t}}, \tuple{\state, \action, \reward, \state'}}},
\end{align*} 
where $\decoder[T]$ denotes the distribution over trajectories of length $T$ that include rewards, generated from $\decoder$.
Intuitively, $\nicefrac{1}{T} \cdot \sum_{t = 1}^{T} \stationary{\policy}^t\fun{\state \mid \sinit}$
%gives the probability of visiting each particular state $\state$ of $\mdp_\policy$
can be seen as the fraction of the time spent in $\state$
along traces of length $T$, starting from the initial state \citep{10.5555/280952}.
Therefore, drawing $\tuple{\state, \action, \reward, \state'} \sim \mathcal{D}_\policy\left[ T \right]$ trivially follows: % from the transition function of $\mdp_\policy$:
it is equivalent to drawing $\state$ from $\nicefrac{1}{T} \cdot \sum_{t = 1}^{T} \stationary{\policy}^t\fun{\cdot \mid \sinit}$, then respectively $\action$ and $\state'$ from $\policy\fun{\cdot \mid \state}$ and $\probtransitions\fun{\cdot \mid \state, \action}$, to finally obtain $\reward = \rewards\fun{\state, \action}$. % and $\labeling' = \labels\fun{\state'}$.
% We similarly define the transition distribution of the behavioral model as:
Given $T \in \N$, our objective is to minimize the Wasserstein between those distributions:
%\begin{equation}
%    \min_{\decoderparameter} \, 
$\wassersteindist{\tracedistance}{\mathcal{D}_{\policy}[T]}{\mathcal{P}_{\decoderparameter}[T]}$.
%\end{equation}
%
The following Lemma enables optimizing the Wasserstein distance between the original MDP and the behavioral model %Eq.~\ref{eq:trace-wasserstein}
when traces are drawn from episodic RL processes or infinite interactions \citep{DBLP:conf/nips/Huang20}.
% (see \citealt{DBLP:conf/nips/Huang20} for a discussion of this RL setting).
% (see \cite{DBLP:conf/nips/Huang20} for a discussion of this RL setting).

\begin{lemma}\label{lemma:wasserstein-transition-limit}
Assume the existence of a stationary behavioral model $\stationarydecoder = \lim_{T \to \infty} \mathcal{P}_{\decoderparameter}[T]$, then
\begin{equation*}
    \lim_{T \to \infty} \wassersteindist{\tracedistance}{\mathcal{D}_{\policy}[T]}{\mathcal{P}_{\decoderparameter}[T]} = \wassersteindist{\tracedistance}{\stationary{\policy}}{\stationarydecoder}.
\end{equation*}
\end{lemma}
\begin{proof}
First, note that $\nicefrac{1}{T} \cdot \sum_{t = 1}^T \stationary{\policy}^t\fun{\cdot \mid \sinit}$ weakly converges to $ \stationary{\policy}$ as $T$  goes to $\infty$ \citep{10.5555/280952}. The result follows then from \citealp[Corollary~6.9]{Villani2009}.
\end{proof}
% \begin{proof}
%     The result follows from (i) $\lim_{T \to \infty}\nicefrac{1}{T} \cdot \sum_{t = 1}^T \stationary{\policy}^t\fun{\cdot \mid \sinit} = \stationary{\policy}$ \citep{10.5555/280952} and (ii) \citep[Corollary~6.9]{Villani2009}.
% \end{proof}

% \begin{lemma}\label{lemma:suplim-to-supstationary}
% Assume $\states$, $\actions$ are compact and the existence of a stationary behavioral model $\stationarydecoder$ defined as
% \begin{align*}
%     &\stationarydecoder\fun{\state, \action, \reward, \state', \labeling'} \\
%     =& \lim_{T \to \infty} \frac{1}{T} \sum_{t = 1}^{T} \expected{\trace \sim \decoder[T]}{\decoder\fun{\seq{\state}{T}, \seq{\action}{T - 1}, \seq{\reward}{T - 1}, {\labeling}_{\scriptscriptstyle 1:{T}}} \cdot \condition{=}\fun{\tuple{\indexof{\state}{T - 1}, \indexof{\action}{T - 1}, \indexof{\reward}{T - 1}, \indexof{\state}{T}, \indexof{\labeling}{T}}, \tuple{\state, \action, \reward, \state', \labeling'}}},
% \end{align*}
% where $ \stationary{\policy}(\state, \action, \reward, \state', \labeling') = \stationary{\policy}(\state, \action, \state') \cdot \condition{=}(\reward,  \rewards\fun{\state, \action, \state'})\cdot \condition{=}(\labeling',  \labels\fun{\state'})$ and $\decoder[T]$ denotes the distribution over traces generated from $\decoder$ of size $T$.
% Then, let $\stationary{\policy}$ be the stationary distribution of $\mdp_\policy$,
% \begin{align*}
%     %\lim_{T \to \infty} \frac{1}{T}\, \wassersteindist{\tracedistance}{\mdp_\policy[T]}{\decoder}
%     &\sup_{f \in \Lipschf{\tracedistance}} \lim_{T \to \infty} \frac{1}{T} \left( \expectedsymbol{\trace \sim \mdp_{\policy}[T]} f\fun{\seq{\state}{T}, \seq{\action}{T - 1}, \seq{\reward}{T - 1}, \seq{\labeling}{T}} - \expectedsymbol{\trace \sim \decoder[T]} f\fun{\seq{\state}{T}, \seq{\action}{T - 1}, \seq{\reward}{T - 1}, \seq{\labeling}{T}}\right) \\
%     =& \sup_{f \in \Lipschf{\tracedistance}} \, \expectedsymbol{\state, \action, \reward, \state', \labeling' \sim \stationary{\policy}} f\fun{\state, \action, \reward, \state', \labeling'} - \expectedsymbol{\state, \action, \reward, \state', \labeling' \sim \stationarydecoder} f\fun{\state, \action, \reward, \state',  \labeling'},
% \end{align*}
% where $\traces{\mdp}\fun{T}$ denotes the set of traces in $\mdp_{\policy}$ of size $T$, and $\mdp_{\policy}[T]$ denotes the distribution over $\traces{\mdp_\policy}\fun{T}$.
% \end{lemma}
\iffalse
\FD{Remove the following Lemma? (it doesn't add anything)}
\smallparagraph{Limitations.}~The following Lemma states that optimizing the Wasserstein between these stationary distributions only yields a lower bound on the Wasserstein distance between distributions over traces generated from the input MDP and those generated from the behavioral model, normalized by their size at the limit.
\begin{lemma}\label{lemma:suplim-to-supstationary}
Assume the existence of a stationary behavioral model $\stationarydecoder$ defined as $\lim_{T \to \infty} \mathcal{P}[T]$.
Then, 
\begin{equation*}
    \lim_{T \to \infty}\, \frac{1}{T}\, \wassersteindist{\tracedistance}{\mdp_{\policy}[T]}{\decoder[T]} \geq \wassersteindist{\tracedistance}{\stationary{\policy}}{\stationarydecoder}
\end{equation*}
where $\mdp_{\policy}[T]$ denotes the distribution over $\traces{\mdp_\policy}\fun{T}$, the latter being the set of traces in $\mdp_{\policy}$ of size $T$.
\end{lemma}

\begin{proof}
Given $\trace \in \traces{\mdp_\policy}\fun{T}$, denote by $\xrightarrow{t}_{\trace} = \tuple{\state_t, \action_t, \reward_t, \state_{t + 1}, \labeling_{t + 1}}$ the transition occurring at time $t < T$ along $\trace$.
Observe that for any $f \in \Lipschf{\tracedistance}$, the function $f'\colon \traces{\mdp_{\policy}} \to \R, \, \trace \mapsto \sum_{t = 0}^{T - 1} f(\xrightarrow{t}_{\trace})$ is also $1$-Lipschitz 
since $f(\xrightarrow{t}_{\trace_1}) - f({\xrightarrow{t}_{\trace_2}}) \leq \tracedistance({\xrightarrow{t}_{\trace_1}, \xrightarrow{t}_{\trace_2}})$ for all $\trace_1, \trace_2 \in \bigcup_{\state \in \states} \traces{\mdp_{\state, \policy}}$, which implies $f'\fun{\trace_1} - f'\fun{\trace_2} \leq \tracedistance\fun{\trace_1, \trace_2}$ by definition of $\tracedistance$. 
We prove the result by considering the set $\mathcal{F}_{\Sigma} = \set{f \in \Lipschf{\tracedistance} \colon f\fun{\trace} = \sum_{t = 0}^{T - 1} f(\xrightarrow{t}_{\trace})}$ instead of $\Lipschf{\tracedistance}$:
\begin{align*}
    & \frac{1}{T}\, \wassersteindist{\tracedistance}{\mdp_{\policy}[T]}{\decoder[T]}\\
    = & \sup_{f \in \Lipschf{\tracedistance}} \, \frac{1}{T} \expectedsymbol{\trace \sim \mdp_{\policy}[T]} f\fun{\seq{\state}{T}, \seq{\action}{T - 1}, \seq{\reward}{T - 1}, \seq{\labeling}{T}} - \frac{1}{T} \expectedsymbol{\trace \sim \decoder[T]} f\fun{\seq{\state}{T}, \seq{\action}{T - 1}, \seq{\reward}{T - 1}, \seq{\labeling}{T}} \\
    \geq & \sup_{f \in \mathcal{F}_{\Sigma}} \, \frac{1}{T} \expectedsymbol{\trace \sim \mdp_{\policy}[T]} f\fun{\seq{\state}{T}, \seq{\action}{T - 1}, \seq{\reward}{T - 1}, \seq{\labeling}{T}} - \frac{1}{T} \expectedsymbol{\trace \sim \decoder[T]} f\fun{\seq{\state}{T}, \seq{\action}{T - 1}, \seq{\reward}{T - 1}, \seq{\labeling}{T}} \tag{since $\mathcal{F}_{\Sigma} \subseteq \Lipschf{\tracedistance}$} \\
    = & \sup_{f \in \Lipschf{\tracedistance}} \frac{1}{T} \expected{\trace \sim \mdp_{\policy}[T]}{\sum_{t = 0}^{T - 1} f\fun{\state_t, \action_t, \reward_t, \state_{t + 1}, \labeling_{t + 1}}} - \frac{1}{T} \expected{\trace \sim \decoder[T]}{\sum_{t = 0}^{T - 1} f\fun{\state_t, \action_t, \reward_t, \state_{t + 1}, \labeling_{t + 1}}} \\
    = & \sup_{f \in \Lipschf{\tracedistance}} \,  \expectedsymbol{\state, \action, \reward, \state', \labeling' \sim \mathcal{D}_{\policy}[T]}{f\fun{\state, \action, \reward, \state', \labeling'}} - \expectedsymbol{\state, \action, \reward, \state', \labeling' \sim \mathcal{P}_{\decoderparameter}[T]}{f\fun{\state, \action, \reward, \state', \labeling'}}\\
    = & \wassersteindist{\tracedistance}{\mathcal{D}_{\policy}[T]}{\mathcal{P}_{\decoderparameter}[T]}
\end{align*}
We obtain the last line by compactness of $\states, \actions$, and $\rewards$ as well as $\atomicprops$ being finite. 
The same development than the proof of Lemma~[VAE-MDP, Appendix] and applying the Portmanteau's Theorem allows us to pass from an expectation over $\mdp_{\policy}[T]$ to $\mathcal{D}_\policy[T]$ and $\decoder[T]$ to $\mathcal{P}_{\decoderparameter}[T]$. %, since $f$ has compact support.
Applying Lemma~\ref{lemma:wasserstein-transition-limit} finally yields the result.

%\begin{align*}
%	&\lim_{T \to \infty} \frac{1}{T} \expected{\trace \sim \decoder}{\sum_{t = 0}^{T - 1} f\fun{\state_t, \action_t, \reward_t, \state_{t + 1}, \labeling_{t + 1}}} \\
%	=& \lim_{T \to \infty} \frac{1}{T} \sum_{t = 0}^{T - 1} \expectedsymbol{\trace \sim \decoder}f\fun{\state_t, \action_t, \reward_t, \state_{t + 1}, \labeling_{t + 1}} \\
%	%=& \lim_{T \to \infty} \frac{1}{T} \sum_{t = 0}^{T - 1} \expected{\trace \sim \decoder}{f\fun{\state_t, \action_t, \reward_t, \state_{t + 1}, \labeling_{t + 1}} \mid \tuple{\state_t, \action_t, \reward_t, \state_{t + 1}} = \tuple{\state, \action, \reward, \state'}}
%	=& \lim_{T \to \infty} \int %_{\traces{\latentmdp_{\decoderparameter}}\fun{T}}
%	 \frac{1}{T} \sum_{t = 0}^{T - 1} \int%_{\states \times \actions \times \rewards \times \states}
%	 f\fun{\state_t, \action_t, \reward_t, \state_{t + 1}, \labeling_{t + 1}} \cdot \condition{=}\fun{\tuple{\indexof{\state}{T - 1}, \indexof{\action}{T - 1}, \indexof{\reward}{T - 1}, \indexof{\state}{T}}, \tuple{d\state, d\action, d\reward, d\state'}} \, d\decoder\fun{\seq{\state}{T}, \seq{\action}{T - 1}, \seq{\reward}{T - 1}, \sampledot}
%\end{align*}
%\begin{align*}
%	 &\lim_{T \to \infty} \frac{1}{T} \sum_{t = 0}^{T - 1} \expected{\trace \sim \decoder}{\decoder\fun{\seq{\state}{T}, \seq{\action}{T - 1}, \seq{\reward}{T - 1}, \sampledot} \cdot \condition{=}\fun{\tuple{\indexof{\state}{T - 1}, \indexof{\action}{T - 1}, \indexof{\reward}{T - 1}, \indexof{\state}{T}}, \tuple{\state, \action, \reward, \state'}}} \\
%	  =& \lim_{T \to \infty} \frac{1}{T} \expected{\trace \sim \decoder}{\sum_{t = 0}^{T - 1} \decoder\fun{\seq{\state}{T}, \seq{\action}{T - 1}, \seq{\reward}{T - 1}, \sampledot} \cdot \condition{=}\fun{\tuple{\indexof{\state}{T - 1}, \indexof{\action}{T - 1}, \indexof{\reward}{T - 1}, \indexof{\state}{T}}, \tuple{\state, \action, \reward, \state'}}}.
%\end{align*}
\end{proof}
%
\fi
\iffalse
\begin{lemma}
Let $\stationary{\policy}$ be the stationary distribution of $\mdp_\policy$ and $\stationarydecoder$ be the stationary behavioral model defined above, then
\begin{align*}
    \lim_{T \to \infty} \frac{1}{T} \, \wassersteindist{\tracedistance}{\mdp_\policy[T]}{\decoder[T]}
    %\lim_{T \to \infty} \sup_{f \in \Lipschf{\tracedistance}} \frac{1}{T} \left| \expectedsymbol{\trace \sim \mdp_{\policy}[T]} f\fun{\seq{\state}{T}, \seq{\action}{T - 1}, \seq{\reward}{T - 1}, \seq{\labeling}{T}} - \expectedsymbol{\trace \sim \decoder} f\fun{\seq{\state}{T}, \seq{\action}{T - 1}, \seq{\reward}{T - 1}, \seq{\labeling}{T}}\right|\\
    &=\sup_{f \in \Lipschf{\tracedistance}} \, \lim_{T \to \infty} \frac{1}{T} \fun{ \expectedsymbol{\trace \sim \mdp_{\policy}[T]} f\fun{\seq{\state}{T}, \seq{\action}{T - 1}, \seq{\reward}{T - 1}, \seq{\labeling}{T}} - \expectedsymbol{\trace \sim \decoder[T]} f\fun{\seq{\state}{T}, \seq{\action}{T - 1}, \seq{\reward}{T - 1}, \seq{\labeling}{T}} }.
\end{align*}
\end{lemma}
\begin{proof}
Observe that
\begin{equation*}
    \lim_{T \to \infty} \frac{1}{T}\, \wassersteindist{\tracedistance}{\mdp_\policy[T]}{\decoder[T]} = \lim_{T \to \infty} \sup_{f \in \Lipschf{\tracedistance}} \frac{1}{T} \fun{ \expectedsymbol{\trace \sim \mdp_{\policy}[T]} f\fun{\seq{\state}{T}, \seq{\action}{T - 1}, \seq{\reward}{T - 1}, \seq{\labeling}{T}} - \expectedsymbol{\trace \sim \decoder[T]} f\fun{\seq{\state}{T}, \seq{\action}{T - 1}, \seq{\reward}{T - 1}, \seq{\labeling}{T}} }.
\end{equation*}
Take $f^\star \in \Lipschf{\tracedistance}\,$ and $T \in \N_0$, we have
\begin{align*}
    & &\expected{\trace \sim \mdp_{\policy}[T]}{\frac{1}{T} \, f^\star\fun{\trace}} - \expected{\trace \sim \decoder[T]}{\frac{1}{T}\, f^\star\fun{\trace}}&\leq \sup_{f \in \Lipschf{\tracedistance}} \expected{\trace \sim \mdp_{\policy}[T]}{\frac{1}{T}\, f\fun{\trace}} - \expected{\trace \sim \decoder[T]}{\frac{1}{T}\, f\fun{\trace}} \\
    & \text{then,} &\lim_{T \to \infty} \expected{\trace \sim \mdp_{\policy}[T]}{\frac{1}{T} \, f^\star\fun{\trace}} - \expected{\trace \sim \decoder[T]}{\frac{1}{T}\, f^\star\fun{\trace}} & \leq \liminf_{T \to \infty} \sup_{f \in \Lipschf{\tracedistance}} \expected{\trace \sim \mdp_{\policy}[T]}{\frac{1}{T}\, f\fun{\trace}} - \expected{\trace \sim \decoder[T]}{\frac{1}{T}\, f\fun{\trace}}. \tag{since this holds for any $T \in \N_{0}$}
\end{align*}
Observe that $f^\star$ is arbitrary in $\Lipschf{\tracedistance}$.
Thus, taking $f^\star = \arg \sup_{f \in \Lipschf{\tracedistance}} \lim_{T \to \infty} \expected{\trace \sim \mdp_{\policy}[T]}{\frac{1}{T} \, f\fun{\trace}} - \expected{\trace \sim \decoder[T]}{\frac{1}{T}\, f\fun{\trace}}$ yields in particular
\begin{align}
    \sup_{f \in \Lipschf{\tracedistance}} \lim_{T \to \infty} \expected{\trace \sim \mdp_{\policy}[T]}{\frac{1}{T} \, f\fun{\trace}} - \expected{\trace \sim \decoder[T]}{\frac{1}{T}\, f\fun{\trace}} & \leq \liminf_{T \to \infty} \sup_{f \in \Lipschf{\tracedistance}} \expected{\trace \sim \mdp_{\policy}[T]}{\frac{1}{T}\, f\fun{\trace}} - \expected{\trace \sim \decoder[T]}{\frac{1}{T}\, f\fun{\trace}}. \label{eq:limsup-suplim-leq}
\end{align}
Now, let $\delta > 0$. Then, there is a step number $T_0 \in \N_0$ from which, for all $t^\star \geq T_0$,
\begin{equation}
    \left| \expected{\trace \sim \mdp_{\policy}[t^\star]}{\frac{1}{t^\star} \, f\fun{\trace}} - \expected{\trace \sim \decoder[t^\star]}{\frac{1}{t^\star} \, f\fun{\trace}}  - \expectedsymbol{\state, \action, \reward, \state', \labeling' \sim \stationary{\policy}}{f\fun{\state, \action, \reward, \state', \labeling'}} + \expectedsymbol{\state, \action, \reward, \state', \labeling' \sim \stationarydecoder}{f\fun{\state, \action, \reward, \state', \labeling'}}\right| \leq \delta \label{eq:limit-delta}
\end{equation}
whenever $f \in \Lipschf{\tracedistance}$, by definition of $\lim_{T \to \infty} \frac{1}{T} \fun{\expectedsymbol{\trace \sim \mdp_{\policy}[T]}\, f\fun{\trace} - \expectedsymbol{\trace \sim \decoder[T]}\, f\fun{\trace}}$ 
%for $\mathcal{D} \in \set{\mdp_{\policy}, \decoder}$
and Lem.~\ref{lemma:suplim-to-supstationary}.
For all $n \geq T_0$, there is a $1$-Lipschitz function $f_n \in \Lipschf{\tracedistance}$ with
\begin{align*}
    & \sup_{f \in \Lipschf{\tracedistance}} \, \expected{\trace \sim \mdp_{\policy}[n]}{\frac{1}{n} \, f\fun{\trace}} - \expected{\trace \sim \decoder[n]}{\frac{1}{n} \, f\fun{\trace}} - \delta \\
    \leq & \expected{\trace \sim \mdp_{\policy}[n]}{\frac{1}{n} \, f_n\fun{\trace}} - \expected{\trace \sim \decoder[n]}{\frac{1}{n} \, f_n\fun{\trace}}
    \tag{e.g., take $f_n = \arg \sup_{f \in \Lipschf{\tracedistance}} \, \expected{\trace \sim \mdp_{\policy}[n]}{\frac{1}{n} \, f\fun{\trace}} - \expected{\trace \sim \decoder[n]}{\frac{1}{n} \, f\fun{\trace}}$}\\
    \leq & \expectedsymbol{\state, \action, \reward, \state', \labeling' \sim \stationary{\policy}}{ f_n\fun{\state, \action, \reward, \state', \labeling'}} - \expectedsymbol{\state, \action, \reward, \state', \labeling' \sim \stationarydecoder}{f_n\fun{\state, \action, \reward, \state', \labeling'}} + \delta \tag{by Eq.~\ref{eq:limit-delta}, because $n \geq T_0$} \\
    = & \lim_{T \to \infty} \, \expected{\trace \sim \mdp_{\policy}[T]}{\frac{1}{T}\, f_n\fun{\trace}} - \expected{\trace \sim \decoder[T]}{\frac{1}{T} \, f_n\fun{\trace}} + \delta \\
    \leq & \sup_{f \in \Lipschf{\tracedistance}} \, \lim_{T \to \infty} \expected{\trace\sim \mdp_{\policy}[T]}{\frac{1}{T}\, f\fun{\trace}} - \expected{\trace\sim \decoder[T]}{\frac{1}{T}\, f\fun{\trace}} + \delta \tag{by definition of $\sup_{f \in \Lipschf{\tracedistance}}$}
\end{align*}
In particular, take $n^\delta = T_0 + \lceil \frac{1}{\delta} \rceil$, we have
\begin{align*}
    && \sup_{f \in \Lipschf{\tracedistance}} \expected{\trace \sim \mdp_{\policy}[n^\delta]}{\frac{1}{n^{\delta}} \, f\fun{\trace}} - \expected{\trace \sim \decoder[n^\delta]}{\frac{1}{n^{\delta}} \, f\fun{\trace}} - \delta
    &\leq \sup_{f \in \Lipschf{\tracedistance}} \, \lim_{T \to \infty} \expected{\trace \sim \mdp_{\policy}[T]}{\frac{1}{T} \, f\fun{\trace}} - \expected{\trace \sim \decoder[T]}{\frac{1}{T} \, f\fun{\trace}} + \delta\\
    &\equiv & \sup_{f \in \Lipschf{\tracedistance}} \expected{\trace \sim \mdp_{\policy}[n^\delta]}{\frac{1}{n^{\delta}} \, f\fun{\trace}} - \expected{\trace \sim \decoder[n^\delta]}{\frac{1}{n^{\delta}} \, f\fun{\trace}} \phantom{- \delta}
    &\leq \sup_{f \in \Lipschf{\tracedistance}} \, \lim_{T \to \infty} \expected{\trace \sim \mdp_{\policy}[T]}{\frac{1}{T} \, f\fun{\trace}} - \expected{\trace \sim \decoder[T]}{\frac{1}{T} \, f\fun{\trace}} + 2 \delta
\end{align*}
Since this holds for any arbitrary $\delta > 0$, we have
\begin{align*}
    && \limsup_{\delta \to 0} \, \sup_{f \in \Lipschf{\tracedistance}} \, \expected{\trace \sim \mdp_{\policy}[n^\delta]}{\frac{1}{n^{\delta}} \, f\fun{\trace}} - \expected{\trace \sim \decoder[n^\delta]}{\frac{1}{n^{\delta}} \, f\fun{\trace}}
    &\leq \lim_{\delta \to 0} \, \sup_{f \in \Lipschf{\tracedistance}} \, \lim_{T \to \infty} \expected{\trace \sim \mdp_{\policy}[T]}{ \frac{1}{T} \, f\fun{\trace}} - \expected{\trace \sim \decoder[T]}{ \frac{1}{T} \, f\fun{\trace}} + 2 \delta
    %&\equiv    & \limsup_{T \to \infty} \, \sup_{f \in \Lipschf{\tracedistance}} \, \expected{\trace \sim \mdp_{\policy}[n]}{\frac{1}{n} \, f\fun{\trace}} - \expected{\trace \sim \decoder[n]}{\frac{1}{n} \, f\fun{\trace}  } &\leq \sup_{f \in \Lipschf{\tracedistance}} \, \lim_{T \to \infty} \expected{\trace\sim\mdp_{\policy}[T]}{\frac{1}{T}\, f\fun{\trace}} - \expected{\trace\sim\decoder[T]}{\frac{1}{T}\, f\fun{\trace}} \tag{because $n^{\delta} \to \infty$ when $\delta \to 0$}
\end{align*}
Notice that $n^{\delta} \to \infty$ when $\delta \to 0$. Therefore, take $T = n^\delta$
\begin{align}
    && \limsup_{T \to \infty} \, \sup_{f \in \Lipschf{\tracedistance}} \, \expected{\trace \sim \mdp_{\policy}[T]}{\frac{1}{T} \, f\fun{\trace}} - \expected{\trace \sim \decoder[T]}{\frac{1}{T} \, f\fun{\trace}  } &\leq \sup_{f \in \Lipschf{\tracedistance}} \, \lim_{T \to \infty} \expected{\trace\sim\mdp_{\policy}[T]}{\frac{1}{T}\, f\fun{\trace}} - \expected{\trace\sim\decoder[T]}{\frac{1}{T}\, f\fun{\trace}} \label{eq:limsup-to-suplim}
\end{align}
By Eq.~\ref{eq:limsup-suplim-leq} and~\ref{eq:limsup-to-suplim} $\liminf_{T \to \infty}$ and $\limsup_{T \to \infty}$ coincides, which finally yields
\begin{equation*}
    \lim_{T \to \infty} \sup_{f \in \Lipschf{\tracedistance}} \expected{\trace \sim \mdp_{\policy}[T]}{\frac{1}{T} \, f\fun{\trace}} - \expected{\trace \sim \decoder[T]}{\frac{1}{T} \, f\fun{\trace}} =  \sup_{f \in \Lipschf{\tracedistance}} \, \lim_{T \to \infty} \expected{\trace \sim \mdp_{\policy}[T]}{\frac{1}{T} \, f\fun{\trace}} - \expected{\trace \sim \decoder[T]}{\frac{1}{T} \, f\fun{\trace}}.
\end{equation*}
\end{proof}

\begin{corollary}
Let $\stationary{\policy}$ be the stationary distribution of $\mdp_\policy$ and $\stationarydecoder$ be the stationary behavioral model defined above, then
\begin{equation*}
    \lim_{T \to \infty} \frac{1}{T} \, \wassersteindist{\tracedistance}{\mdp_\policy[T]}{\decoder[T]} = \wassersteindist{\tracedistance}{\stationary{\policy}}{\stationarydecoder}.% \sup_{f \in \Lipschf{\tracedistance}} \, \expectedsymbol{\state, \action, \reward, \state', \labeling' \sim \stationary{\policy}} f\fun{\state, \action, \reward, \state', \labeling'} - \expectedsymbol{\state, \action, \reward, \state', \labeling' \sim \stationarydecoder} f\fun{\state, \action, \reward, \state',  \labeling'}.
\end{equation*}
\end{corollary}
\fi

\subsection{The Objective Function}
\fi
%\subsubsection{The Objective Function: action encoder variant}
Given Assumption~\ref{assumption:vae-mdp}, we consider the OT between \emph{local} distributions,
where traces are drawn from episodic RL processes or infinite interactions
%\citep{DBLP:conf/nips/Huang20}
(we show in Appendix~\ref{appendix:discrepancy-measure} that considering the OT between trace-based distributions in the limit amounts to reasoning about stationary distributions).
% We learn the latent space model by minimizing
Our goal is to minimize
$\wassersteindist{\transitiondistance}{\stationary{\policy}}{\stationarydecoder}$ so that
%
\begin{equation}
    \stationarydecoder\fun{\state, \action, \reward, \state'} =  \int_{\latentstates \times \latentactions \times \latentstates} \decoder\fun{\state, \action, \reward, \state' \mid \latentstate, \latentaction, \latentstate'} \, d\latentstationaryprior\fun{\latentstate, \latentaction, \latentstate'}, \label{eq:stationary-decoder}
    %= \expectedsymbol{\state, \action \sim \stationary{\policy}} \expectedsymbol{\latentstate \sim \embed_{\encoderparameter}\fun{\sampledot \mid \state}}\expectedsymbol{\latentstate' \sim \latentprobtransitions_{\decoderparameter}\fun{\sampledot \mid \latentstate, \action}} \decoder\fun{\reward, \state' \mid \latentstate'}
\end{equation}
%
where $\decoder$ is a transition decoder and $\latentstationaryprior$ denotes the stationary distribution of the latent model $\latentmdp_{\decoderparameter}$.
% so that $\latentstationaryprior\fun{\latentstate, \latentaction, \latentstate'} = \latentstationaryprior\fun{\latentstate} \cdot { \latentpolicy_{\decoderparameter}\fun{\latentaction \mid \latentstate}} \cdot \latentprobtransitions_{\decoderparameter}\fun{\latentstate' \mid \latentstate, \latentaction}$.
%
% The \emph{Wasserstein autoencoder} (WAE) \citep{DBLP:conf/iclr/TolstikhinBGS18} framework allows us to exploit this model to simplify 
As proved by \citet{Bousquet2017FromOT}, this model allows to derive a simpler form of the OT: % between $\stationary{\policy}$ and $\stationary{\decoderparameter}$
instead of finding the optimal coupling
of (i) the stationary distribution $\stationary{\policy}$ of $\mdp_\policy$ 
and (ii) the behavioral model $\stationary{\decoderparameter}$, in the primal definition of $\wassersteindist{\tracedistance}{\stationary{\policy}}{\stationarydecoder}$,
%from $\couplings{\stationary{\policy}}{\stationary{\decoderparameter}}$,
it is sufficient to find an encoder $\transitionencoder$ whose marginal is given by $Q\fun{\latentstate, \latentaction, \latentstate'} = \expectedsymbol{\state, \action, \state' \sim \stationary{\policy}} \transitionencoder\fun{\latentstate, \latentaction, \latentstate' \mid \state, \action, \state'}$ % = \embed_{\encoderparameter}\stationary{\policy}$,
and identical to $\stationary{\policy}$. % $\latentstationaryprior$ instead of finding the optimal coupling in the primal definition of \wassersteindist{\transitiondistance}{\stationary{\policy}}{\stationarydecoder} \citep{Bousquet2017FromOT,DBLP:conf/iclr/TolstikhinBGS18}.
This is summarized in the following Theorem, yielding a particular case of \emph{Wasserstein-autoencoder} \cite{DBLP:conf/iclr/TolstikhinBGS18}:
\begin{theorem}
%Let $\mdp$ be an MDP with state-action space $\states \times \actions$, $\latentmdp_{\decoderparameter}
%$ be an ergodic latent MDP of $\mdp$ with state-action space $\latentstates \times \latentactions$ and reward function $\latentrewards_{\decoderparameter}$,
%= \tuple{\latentstates, \latentactions, \latentprobtransitions_{\decoderparameter}, \latentrewards_{\decoderparameter}, \latentlabels, \atomicprops, \zinit}$ be a latent MDP of $\mdp$,
%$\policy$ and $\latentpolicy_{\decoderparameter}$ be respectively policies for $\mdp$ and $\latentmdp_{\decoderparameter}$,
% $\stationarydecoder \in \distributions{\states \times \actions \times \R \times \states}$ and $\decoder \colon \latentstates \times \latentactions \times \latentstates \to \distributions{\states \times \actions \times \R \times \states}$ be respectively a generator and a decoder as defined in Eq.~\ref{eq:stationary-decoder},
Let $\stationarydecoder$ and $\decoder$ be respectively a behavioral model and transition decoder as defined in Eq.~\ref{eq:stationary-decoder},
$\generative_{\decoderparameter}\colon \latentstates \to \states$ be a state-wise decoder, and 
% $\embeda_{\decoderparameter} \colon \latentstates \times \latentactions \to \actions$ be an action embedding function.
$\embeda_{\decoderparameter}$ be an action embedding function.
% \begin{equation*}
%     G_{\decoderparameter} \colon \latentstates \times \latentactions \times \latentstates \to \states \times \actions \times \R \times \states, \; \tuple{\latentstate,  \latentaction, \latentstate'} \mapsto \tuple{\generative_{\decoderparameter}\fun{\latentstate}, \embeda_{\decoderparameter}\fun{\latentstate, \latentaction}, {\latentrewards_{\decoderparameter}\fun{\latentstate, \latentaction}}, \generative_{\decoderparameter}\fun{\latentstate'}}.
% \end{equation*}
% Then, $\decoder$ is deterministic when $G_\decoderparameter$ is its Dirac function. In that case,
Assume $\decoder$ is deterministic with Dirac function
$G_{\decoderparameter}\fun{\latentstate, \latentaction, \latentstate'} = \tuple{\generative_{\decoderparameter}\fun{\latentstate}, \embeda_{\decoderparameter}\fun{\latentstate, \latentaction}, {\latentrewards_{\decoderparameter}\fun{\latentstate, \latentaction}}, \generative_{\decoderparameter}\fun{\latentstate'}}$, then
%Then,
%\begin{multline*}
\begin{equation*}
    \wassersteindist{\tracedistance}{\stationary{\policy}}{\stationarydecoder}% =& \inf_{\coupling \in \couplings{\stationary{\policy}}{\stationarydecoder}} \, \expectedsymbol{\tau, \tau' \sim \coupling} \transitiondistance\fun{\tau, \tau'}\\
    = \inf_{\transitionencoder: \, Q = \latentstationaryprior} \, \expectedsymbol{\state, \action, \reward, \state' \sim \stationary{\policy}} \, \expectedsymbol{\latentstate, \latentaction, \latentstate' \sim \transitionencoder\fun{\sampledot \mid \state, \action, \state'}} %\\
    \tracedistance\fun{\tuple{\state, \action, \reward, \state'}, G_{\decoderparameter}\fun{\latentstate, \latentaction, \latentstate'}}.
    % \Big[\distance_\states\fun{\state, \generative_{\decoderparameter}\fun{\latentstate}} + \distance_{\actions}\fun{\action, \embeda_{\decoderparameter}\fun{\latentstate, \latentaction}} + \left| r - \rewards_{\decoderparameter}\fun{\latentstate, \latentaction} \right| + \distance_{\states}\fun{\state', \generative_{\decoderparameter}\fun{\latentstate'}}\Big].
\end{equation*}
%\end{multline*}
\end{theorem}
%\begin{proof}
%Assume that $\decoder$ is deterministic with $G_{\decoderparameter}$ as its Dirac function.
%Then, the results follow directly from \citet[Theorem~1 and Corollary~1]{DBLP:conf/iclr/TolstikhinBGS18}:
%The result follows directly from
%\citet[Theorem~1 and Corollary~1]{DBLP:conf/iclr/TolstikhinBGS18}.
%\cite[Theorem~1 and Corollary~1]{DBLP:conf/iclr/TolstikhinBGS18}.
% \begin{align*}
%     \wassersteindist{\tracedistance}{\stationary{\policy}}{\stationarydecoder} &=
%     \inf_{\coupling \in \couplings{\stationary{\policy}}{\stationarydecoder}} \, \expectedsymbol{\tau, \tau' \sim \coupling} \transitiondistance\fun{\tau, \tau'}\\
%     &=\inf_{\embed: \, Q = \latentstationaryprior} \, \expectedsymbol{\state, \action, \reward, \state' \sim \stationary{\policy}} \, \expected{\latentstate, \latentaction, \latentstate' \sim \embed\fun{\sampledot \mid \state, \action, \state'}}{\distance_\states\fun{\state, \generative_{\decoderparameter}\fun{\latentstate}} + \distance_{\actions}\fun{\action, \embeda_{\decoderparameter}\fun{\latentstate, \latentaction}} + \left| r - \rewards_{\decoderparameter}\fun{\latentstate, \latentaction} \right| + \distance_{\states}\fun{\state', \generative_{\decoderparameter}\fun{\latentstate'}}}
% \end{align*}
%\end{proof}
%
Henceforth, fix
%$\embed_\encoderparameter \colon \states \to \distributions{\latentstates}$
$\embed_\encoderparameter \colon \states \to {\latentstates}$
and $\embed_{\encoderparameter}^{\scriptscriptstyle\actions} \colon \latentstates \times \actions \to \distributions{\latentactions}$ as parameterized
%state embedding function
state
and action encoders with
%$\embed_\encoderparameter\fun{\latentstate, \latentaction, \latentstate' \mid \state, \action, \state'} = \embed_{\encoderparameter}\fun{\latentstate \mid \state} \cdot \embed_{\encoderparameter}^{\scriptscriptstyle \actions}\fun{\latentaction \mid \latentstate, \action} \cdot \embed_{\encoderparameter}\fun{\latentstate' \mid \state'}$% for any $\state, \state' \in \states$, $\latentstate, \latentstate' \in \latentstates$
$\embed_\encoderparameter\fun{\latentstate, \latentaction, \latentstate' \mid \state, \action, \state'} = \condition{\embed_{\encoderparameter}\fun{\state}=\latentstate} \cdot \embed_{\encoderparameter}^{\scriptscriptstyle \actions}\fun{\latentaction \mid \latentstate, \action} \cdot \condition{\embed_{\encoderparameter}\fun{\state'} =\latentstate'}$%
% $\embed_\encoderparameter\fun{\latentstate, \latentaction, \latentstate' \mid \state, \action, \state'} = \condition{=}\fun{\tuple{\embed_{\encoderparameter}\fun{\state}, \embed_{\encoderparameter}\fun{\state'}}, \tuple{\latentstate, \latentstate'}} \cdot \embed_{\encoderparameter}^{\scriptscriptstyle \actions}\fun{\latentaction \mid \latentstate, \action}$%
, and define the marginal encoder as $Q_\encoderparameter = \expectedsymbol{\state, \action, \state' \sim \stationary{\policy}} \embed_{\encoderparameter}\fun{\cdot \mid \state, \action, \state'}$.
% The action embedding function $\embeda_{\encoderparameter, \decoderparameter}$ can be retrieved via $\embeda_{\encoderparameter, \decoderparameter}\fun{\state, \latentaction} = \expectedsymbol{\latentstate \sim \embed_{\encoderparameter}\fun{\sampledot \mid \state}}{\embeda_{\decoderparameter}\fun{\latentstate, \latentaction}}$.
%
Training the model components can be achieved via the objective:
%We name this latent space model \emph{Wasserstein auto-encoded MDP} (WAE-MDP), with the objective:
\begin{align*}
    \min_{\encoderparameter, \decoderparameter} \, \expectedsymbol{\state, \action, \reward, \state' \sim \stationary{\policy}} \, \expectedsymbol{\latentstate, \latentaction, \latentstate' \sim \embed_{\encoderparameter}\fun{\sampledot \mid \state, \action, \state'}}  \tracedistance\fun{\tuple{\state, \action, \reward, \state'}, G_{\decoderparameter}\fun{\latentstate, \latentaction, \latentstate'}} + \beta \cdot \divergencesymbol\fun{\encoder, \latentstationaryprior},
\end{align*}
where $\divergencesymbol$ is an arbitrary discrepancy metric and $\beta > 0$ a hyperparameter.
Intuitively, the encoder $\embed_{\encoderparameter}$ can be learned by enforcing its marginal distribution $\encoder$ to match $\latentstationaryprior$ through this discrepancy.
% \begin{remark}[VAEs vs. WAEs]
% VAE- and WAE-MDP both involve a transition reconstruction term and a regularization term penalizing the discrepancy between the distributions over the latent variables produced respectively from the encoder and in the latent model.
% % Given $\state \in \states$, VAEs require the encoder $\embed_{\encoderparameter}$ being stochastic and enforce $\actionencoder\fun{\sampledot \mid \latentstate, \action}$ and $\embed_{\encoderparameter}\fun{\sampledot \mid \state'}$ to respectively match $\latentpolicy_{\decoderparameter}\fun{\sampledot \mid \latentstate}$ $\latentprobtransitions_{\decoderparameter}\fun{\sampledot \mid \latentstate, \latentaction}$ for all the different actions produced from $\policy\fun{\sampledot \mid \action}$ and 
% Given $\state \in \states$ and its embedding $\latentstate \in \latentstates$, VAE requires the encoder $\embed_{\encoderparameter}$ being stochastic and enforces $\embed_{\encoderparameter}\fun{\sampledot \mid \state'}$ to match  $\latentprobtransitions_{\latentpolicy_{\decoderparameter}}\fun{\sampledot \mid \latentstate}$ for all $\state' \sim \probtransitions_{\policy}\fun{\sampledot \mid \state}$ produced in the real environment.
% This eventually breaks the dependency of $\embed_{\encoderparameter}$ on $\state'$ and leads to \emph{mode collapse}. % and $\latentstate \sim \embed_{\encoderparameter}\fun{\sampledot \mid \state}$ produced.
% On the other hand, WAE has no such requirement on the encoding function and intuitively forces the mixture $\encoder$
% %$\expected{\state, \action, \state' \sim \stationary{\policy}}{\embed_{\encoderparameter}\fun{\sampledot \mid \state, \action, \state'}}$
% to match $\latentstationaryprior$ which has not impact on any dependency and naturally allows to avoid collapsing issues \citep{DBLP:conf/iclr/TolstikhinBGS18}.
% \end{remark}
\begin{remark}%
% When $\policy$ already produces discrete actions, i.e., when $\policy$ is a latent policy (cf. Rem~\ref{rmk:latent-policy-execution}), or $\mdp$ has a discrete action space,
If $\mdp$ has a discrete action space, then learning $\latentactions$ is not necessary. We can set $\latentactions = \actions$ using identity functions for the action encoder and decoder
% ignoring learning $\latentactions$ can be achieved by respectively setting
% % $\actionencoder\fun{\sampledot \mid \latentstate, \cdot}$ or both $\actionencoder\fun{\sampledot \mid \latentstate, \cdot}$ and $\embeda_{\decoderparameter}\fun{\latentstate, \sampledot}$ to the identity
% %
% the action encoder and the action encoder/decoder pair to the indentity function
% This typically occurs when $\policy$ is a latent policy (cf. Rem.~\ref{rmk:latent-policy-execution}) or when $\mdp$ has already a discrete action space.
% To ignore learning the action space, it suffices to take $\actionencoder\fun{\sampledot \mid \latentstate, \cdot}$ and $\embeda_{\decoderparameter}\fun{\latentstate, \sampledot}$ as the identity function.
(details in Appendix~\ref{appendix:discrete-action-space}).
\end{remark}
% \smallparagraph{Dealing with discrete actions.}~When the policy $\policy$ executed in $\mdp$ already produces discrete actions, learning $\latentactions$ is, in many cases, not necessary.
% This typically occurs when $\policy$ is a latent policy (cf. Rem.~\ref{rmk:latent-policy-execution}) or when $\mdp$ has already a discrete action space.
% To ignore learning the action space, it suffices to take $\actionencoder\fun{\sampledot \mid \latentstate, \cdot}$ and $\embeda_{\decoderparameter}\fun{\latentstate, \sampledot}$ as the identity function.
% More details are given in Appendix~\ref{sec:discrete-latent-spaces}.

% (notice that the premise of Assumption~\ref{assumption:action-decoder} implies the one of Assumption~\ref{assumption:action-encoder}).
%\begin{remark} \label{rmk:zero-action-distance}
%When the original MDP $\mdp$ with action space $\actions$ operates under a latent policy $\latentpolicy \colon \latentstates \to \distributions{\latentactions}$ while its original action space is continuous (i.e., $\actions \neq \latentactions$; Assumption~\ref{assumption:action-encoder} holds while Assumption~\ref{assumption:action-decoder} does not), then each latent action produced is embedded back to the original action space via $\embeda_{\decoderparameter}$, which allows to execute it in $\mdp$ (see Remark~\ref{rmk:latent-policy-execution}).
%Therefore, the projection of $G_{\decoderparameter}$ on the action space is given by $\distance_{\actions}\fun{\embeda_{ \decoderparameter}\fun{\latentstate, \latentaction}, \embeda_{\decoderparameter}\fun{\latentstate, \latentaction}}$.
% When the state encoder $\embed_{\encoderparameter}$ is deterministic, this distance thus equals zero.
%\end{remark}
%
\iffalse
\begin{algorithm}%[H]
% \setstretch{1.35}
\caption{Wasserstein$^2$ Auto-Encoded MDP}\label{alg:wwae-mdp}
\DontPrintSemicolon
\KwIn{batch size $N$, max. step $T$, no. of regularizer updates $\ncritic$, penalty coefficient $\delta$}
%\SetNoFillComment
\SetKwComment{Comment}{$\triangleright$\ }{}
\SetCommentSty{textnormal}
\LinesNotNumbered 
\SetKwBlock{Begin}{function}{end function}
%test \;
\For{$t = 1$ to $T$}{
    \For{$i = 1$ to $N$}{
        Draw $\tuple{\state^i, \action^i, \reward^i, \state^{\prime\, i}}$ from $\stationary{\policy}$,
        $\;\tuple{\latentstate_{\embed}^{\,i}, \latentaction_{\embed}^{\,i}, \latentstate_\embed^{\prime\, i}}$ from $\embed_{\encoderparameter}({\sampledot \mid \state^i, \action^i, \state^{\prime\, i}})$, 
        $\;\latentstate^{\prime\, i}_{\scriptscriptstyle \latentprobtransitions}$ from $\latentprobtransitions_{\decoderparameter}({\sampledot \mid \latentstate_{\embed}^{\,i}, \latentaction_{\embed}^{\, i}})$, and
        $\tuple{\latentstate^{\, i}_{\stationary{}}, \latentaction^{\, i}_{\stationary{}}, \latentstate^{\prime\, i}_{\stationary{}}}$ resp. from $\latentstationaryprior$, $\latentpolicy_{\decoderparameter}({\sampledot \mid \latentstate^{\, i}_{\stationary{}}})$, and $\latentprobtransitions_{\decoderparameter}({\sampledot \mid \latentstate^{\, i}_{\stationary{}}, \latentaction^{\, i}_{\stationary{}}})$\;
    }
    $
    %\begin{aligned}
    	\mathcal{W} \gets \textstyle \sum_{i = 1}^{N} \varphi_{\wassersteinparameter}^{\stationary{}}({\latentstate_{\embed}^{\,i}, \latentaction_{\embed}^{\,i}, \latentstate^{\prime\, i}_{\scriptscriptstyle \latentprobtransitions}})
    	- \varphi_{\wassersteinparameter}^{\stationary{}}({\latentstate^{\, i}_{\stationary{}}, \latentaction^{\, i}_{\stationary{}}, \latentstate^{\prime\, i}_{\stationary{}}}) +
    	\varphi_{\wassersteinparameter}^{\probtransitions}({\state^{i} \!, \action^{i} \!, \latentstate_{\embed}^{\, i}, \latentaction_{\embed}^{\, i}, \latentstate^{\prime\, i}_{\embed}}) - \varphi_{\wassersteinparameter}^{\probtransitions}({\state^{i}\!, \action^{i}\!, \latentstate_{\embed}^{\, i}, \latentaction_{\embed}^{\, i}, \latentstate^{\prime\, i}_{\scriptscriptstyle \latentprobtransitions}})$\;
    %\end{aligned}
    %$\; %$\!\!\!\!$ \Comment*[r]{$\steadystateregularizer{\policy}, \localtransitionloss{\stationary{\policy}}$ ($\max_{\wassersteinparameter}$)}
    %$%\begin{aligned}
    	$P \gets \textstyle \sum_{i = 1}^{N} \textsc{Gp}\big({\varphi_{\wassersteinparameter}^{\stationary{}}, \tuple{\latentstate_{\embed}^{\,i}, \latentaction_{\embed}^{\,i}, \latentstate^{\prime\, i}_{\scriptscriptstyle \latentprobtransitions}}, \tuple{\latentstate^{\, i}_{\stationary{}}, \latentaction^{\, i}_{\stationary{}}, \latentstate^{\prime\, i}_{\stationary{}}}}\big) + \textsc{Gp}\big({\latentstate' \mapsto \varphi_{\wassersteinparameter}^{\probtransitions}\fun{\state^i, \action^i, \latentstate^{\, i}, \latentaction^{\, i}, \latentstate'}, \latentstate^{\prime \, i}_{\embed}, \latentstate^{\prime\, i}_{\scriptscriptstyle \latentprobtransitions}}\big)
    %\end{aligned}
    $\;
    Update $\wassersteinparameter$ by ascending $\nicefrac{1}{N} \cdot \fun{\beta  \,\mathcal{W} - \delta \, P}$\;
    \If{$t \bmod \ncritic = 0$
    %\Comment*[r]{Auto-Encoder loss ($\min_{\encoderparameter, \decoderparameter}$)}
    }{
     $\mathcal{L} \gets \sum_{i = 1}^{N} \norm{\state^i- \generative_{\decoderparameter}\fun{\latentstate_{\embed}^{\, i}}} + \norm{\action^i - \embeda_{\decoderparameter}\fun{\latentstate_{\embed}^{\, i}, \latentaction_{\embed}^{\, i}}} + \left| \reward^i - \latentrewards_{\decoderparameter}\fun{\latentstate_{\embed}^{\, i}, \latentaction_{\embed}^{\, i}} \right| + \norm{\state^{\prime\, i} - \generative_{\decoderparameter}\fun{\latentstate_{\embed}^{\prime\, i}}}$\;
     Update $\tuple{\encoderparameter, \decoderparameter}$ by descending $\nicefrac{1}{N}\cdot \fun{\mathcal{L} + {\beta} \, \mathcal{W}}$\;
    }
}
\Begin($\textsc{Gp}\fun{\varphi_{\wassersteinparameter},\vect{x}, \vect{y}}$ \hfill $\triangleright\ $ Gradient penalty \citep{DBLP:conf/nips/GulrajaniAADC17} for $\varphi_{\wassersteinparameter} \colon \R^n \to \R$ and $\vect{x}, \vect{y} \in \R^n$){
%\Comment*[r]{$\varphi_{\wassersteinparameter} \colon \R^n \to \R$, $\vect{x}, \vect{y} \in \R^n$})
%{
$\epsilon \gets \mathit{U}\fun{0, 1}$; $\vect{z} \gets \epsilon \vect{x} + (1 - \epsilon) \vect{y}$ \Comment*[r]{random noise; straight lines between $\vect{x}$ and $\vect{y}$}
\Return{$\fun{\norm{\gradient_{\vect{z}} \, \varphi_{\wassersteinparameter}\fun{\vect{z}}} - 1}^2$}
}
\end{algorithm}
\fi
\begin{algorithm}%[H]
% \setstretch{1.35}
\caption{Wasserstein$^2$ Auto-Encoded MDP}\label{alg:wwae-mdp}
\DontPrintSemicolon
\KwIn{batch size $N$, max. step $T$, no. of regularizer updates $\ncritic$, penalty coefficient $\delta > 0$}
%\SetNoFillComment
\SetKwComment{Comment}{$\triangleright$\ }{}
\SetCommentSty{textnormal}
\LinesNotNumbered 
\SetKwBlock{Begin}{function}{end function}
%test \;
\For{$t = 1$ to $T$}{
    \For{$i = 1$ to $N$}{
        Sample a transition ${\state_i, \action_i, \reward_i, \state^{\prime}_i}$ from the original environment via $\stationary{\policy}$\;
        Embed the transition into the latent space by drawing ${\latentstate_{i}, \latentaction_{i}, \latentstate^{\prime}_{i}}$ from $\embed_{\encoderparameter}({\sampledot \mid \state_i, \action_i, \state^{\prime}_{i}})$\;
        Make the latent space model transition to the next latent state:  $\latentstate^{\star}_{i} \sim \latentprobtransitions_{\decoderparameter}({\sampledot \mid \latentstate_{i}, \latentaction_{i}})$\;
        Sample a latent transition from $\latentstationaryprior$:
        $\latentvariable_i \sim \latentstationaryprior$, $\latentaction'_i \sim \latentpolicy_{\decoderparameter}\fun{\sampledot \mid \latentvariable_i}$, and $\latentvariable_i^{\prime} \sim \latentprobtransitions_{\decoderparameter}\fun{\sampledot \mid \latentvariable_i, \latentaction^{\prime}_i}$\;
    }
    $
    %\begin{aligned}
    	\mathcal{W} \gets \textstyle \sum_{i = 1}^{N} \steadystatenetwork({\latentstate_{i}, \latentaction_{i}, \latentstate^{\star}_{i}})
    	- \steadystatenetwork({\latentvariable_{i}, \latentaction^{\prime}_{i}, \latentvariable^{\prime}_{i}}) +
    	\transitionlossnetwork({\state_{i}, \action_{i}, \latentstate_{i}, \latentaction_{i}, \latentstate^{\prime}_{i}}) -  \transitionlossnetwork({\state_{i}, \action_{i}, \latentstate_{i}, \latentaction_{i}, \latentstate^{\star}_{i}})$\;
    %\end{aligned}
    %$\; %$\!\!\!\!$ \Comment*[r]{$\steadystateregularizer{\policy}, \localtransitionloss{\stationary{\policy}}$ ($\max_{\wassersteinparameter}$)}
    %$%\begin{aligned}
    	$P \gets \textstyle \sum_{i = 1}^{N} \textsc{Gp}\big({\steadystatenetwork, \tuple{\latentstate_{i}, \latentaction_{i}, \latentstate^{\star}_{i}}, \tuple{\latentvariable_{i}, \latentaction^{\prime}_{i}, \latentvariable^{\prime}_i}}\big) + \textsc{Gp}\big({\vx \mapsto \transitionlossnetwork\fun{\state_i, \action_i, \latentstate_{i}, \latentaction_{i}, \vx}, \latentstate^{\prime}_{i}, \latentstate^{\star}_{i}}\big)
    %\end{aligned}
    $\;
    Update the Lipschitz networks parameters $\wassersteinparameter$ by ascending $\nicefrac{1}{N} \cdot \fun{\beta  \,\mathcal{W} - \delta \, P}$\;
    \If{$t \bmod \ncritic = 0$
    %\Comment*[r]{Auto-Encoder loss ($\min_{\encoderparameter, \decoderparameter}$)}
    }{
     $\mathcal{L} \gets \sum_{i = 1}^{N} \distance_{\states}\fun{\state_i,  \generative_{\decoderparameter}\fun{\latentstate_{i}}} + \distance_{\actions}\fun{\action_i, \embeda_{\decoderparameter}\fun{\latentstate_{i}, \latentaction_{i}}} + \distance_{\rewards}\fun{\reward_i,  \latentrewards_{\decoderparameter}\fun{\latentstate_{i}, \latentaction_{i}}} + \distance_{\states}\fun{\state^{\prime}_i, \generative_{\decoderparameter}\fun{\latentstate^{\prime}_{i}}}$\;
     Update the latent space model parameters $\tuple{\encoderparameter, \decoderparameter}$ by descending $\nicefrac{1}{N}\cdot \fun{\mathcal{L} + {\beta} \, \mathcal{W}}$\;
    }
}
\Begin($\textsc{Gp}\fun{\varphi_{\omega},\vect{x}, \vect{y}}$ \hfill $\triangleright\ $ \textbf{Gradient penalty} for $\varphi_{\wassersteinparameter} \colon \R^n \to \R$ and $\vect{x}, \vect{y} \in \R^n$){
\ifarxiv
$\epsilon \sim \mathit{U}\fun{0, 1}$ \Comment*[r]{random noise}
$\tilde{\vect{x}} \gets \epsilon \vect{x} + (1 - \epsilon) \vect{y}$ \Comment*[r]{straight lines between $\vect{x}$ and $\vect{y}$}
\else
$\epsilon \sim \mathit{U}\fun{0, 1}$; $\tilde{\vect{x}} \gets \epsilon \vect{x} + (1 - \epsilon) \vect{y}$ \Comment*[r]{random noise; straight lines between $\vect{x}$ and $\vect{y}$}
\fi
\Return{$\fun{\norm{\gradient_{\tilde{\vect{x}}} \, \varphi_{\wassersteinparameter}\fun{\tilde{\vect{x}}}} - 1}^2$}
}
\end{algorithm}
% \subsection{Guaranteed Abstraction Quality and Distillation via Wasserstein Regularization}
%In the same spirit than \emph{Wasserstein-Wasserstein autoencoders} \citep{DBLP:journals/corr/abs-1902-09323/zhang19},
When $\policy$ is executed in $\mdp$, %the original model,
observe that
% the action encoder $\actionencoder$ enables its \emph{parallel execution} in the latent model:
its \emph{parallel execution} in $\latentmdp_{\decoderparameter}$ % the latent model 
is enabled by the action encoder $\actionencoder$:
%
% given an original state $\state \in \states$, then the action $\action \sim \policy\fun{\sampledot \mid \state}$ is prescribed by $\policy$, which is then embedded in the latent space via $\latentaction \sim \actionencoder\fun{\sampledot \mid \embed_{\encoderparameter}\fun{\state}, \action}$ (cf. Fig.~\ref{subfig:latent-fow-distillation}).
given an original state $\state \in \states$, $\policy$ first prescribes the action $\action \sim \policy\fun{\sampledot \mid \state}$, which is then embedded in the latent space via $\latentaction \sim \actionencoder\fun{\sampledot \mid \embed_{\encoderparameter}\fun{\state}, \action}$ (cf. Fig.~\ref{subfig:latent-fow-distillation}).
%
% The local transition loss resulting from this interaction is
% $\localtransitionloss{\stationary{\policy}} = \expectedsymbol{\state, \tuple{\action, \latentaction} \sim \stationary{\policy}} \wassersteindist{\distance_{\latentstates}}{\embed_{\encoderparameter}\probtransitions\fun{\sampledot \mid \state, \action}}{\latentprobtransitions\fun{\sampledot \mid \embed_{\encoderparameter}\fun{\state}, \latentaction}}$
This parallel execution, along with setting  $\divergencesymbol$ to $\wassersteinsymbol{\transitiondistance}$, yield an upper bound on the latent regularization, compliant with the bisimulation bounds. %`, as claimed in the following.
%guarantees of Eq.~\ref{eq:bidistance-bound}.
A two-fold regularizer is obtained thereby, defining the foundations of our objective function:%
% The following Lemma leads a which are the building blocks for defining our objective function.
\begin{restatable}{lemma}{regularizerlemma}\label{lem:regularizer-upper-bound}
%Let $\originaltolatentstationary{} \in \distributions{\latentstates \times \latentactions \times \latentstates}, \; \tuple{\latentstate, \latentaction, \latentstate'} \mapsto \expectedsymbol{\state, \action \sim \stationary{\policy}}[\embed_{\encoderparameter}(\latentstate \mid \state) \cdot \embed_{\encoderparameter}^{\scriptscriptstyle \actions}(\latentaction \mid \latentstate, \action) \cdot \latentprobtransitions_{\decoderparameter}(\latentstate' \mid \latentstate, \latentaction)]$ denote the distribution over the state-action pairs drawn from 
%the steady-state of $\mdp_{\policy}$
% Let $\originaltolatentstationary{} \in \distributions{\latentstates \times \latentactions \times \latentstates}, \; \tuple{\latentstate, \latentaction, \latentstate'} \mapsto \expectedsymbol{\state, \action \sim \stationary{\policy}}[\condition{\embed_{\encoderparameter}(\state)=\latentstate} \cdot \embed_{\encoderparameter}^{\scriptscriptstyle \actions}(\latentaction \mid \latentstate, \action) \cdot \latentprobtransitions_{\decoderparameter}(\latentstate' \mid \latentstate, \latentaction)]$ denote the distribution of drawing state-action pairs from the
Define $\originaltolatentstationary{}\fun{\latentstate, \latentaction, \latentstate'} = \expectedsymbol{\state, \action \sim \stationary{\policy}}[\condition{\embed_{\encoderparameter}(\state)=\latentstate} \cdot \embed_{\encoderparameter}^{\scriptscriptstyle \actions}(\latentaction \mid \latentstate, \action) \cdot \latentprobtransitions_{\decoderparameter}(\latentstate' \mid \latentstate, \latentaction)]$ as the distribution of drawing state-action pairs from interacting with $\mdp$, embedding them to the latent spaces, and finally letting them transition to their successor state in $\latentmdp_{\decoderparameter}$. Then,
$
%\begin{align*}
    \wassersteindist{\transitiondistance}{\encoder}{ \latentstationaryprior}
    \leq \wassersteindist{\transitiondistance}{\latentstationaryprior}{\originaltolatentstationary{}} + \localtransitionloss{\stationary{\policy}}.
    % \leq \wassersteindist{\transitiondistance}{\latentstationaryprior}{\originaltolatentstationary{}} + \expectedsymbol{\state, \action \sim \stationary{\policy}}\expectedsymbol{\latentaction \sim \embed_{\encoderparameter}^{\actions}\fun{\sampledot \mid \embed_{\encoderparameter}\fun{\state}, \action}} \wassersteindist{\distance_{\latentstates}}{\embed_{\encoderparameter}\probtransitions\fun{\sampledot \mid \state, \action}}{\latentprobtransitions_{\decoderparameter}\fun{\sampledot \mid \embed_{\encoderparameter}\fun{\state}, \latentaction}}.
%\end{align*}
$
\end{restatable}
%
We therefore define the \waemdp (\emph{Wasserstein-Wasserstein auto-encoded MDP}) objective as:
\begin{equation*}
    %\min_{\encoderparameter, \decoderparameter} \expectedsymbol{\state,  \action, \state' \sim \stationary{\latentpolicy}} \, 
    %\expectedsymbol{\latentstate, \latentaction, \latentstate' \sim \embed_{\encoderparameter}\fun{\sampledot \mid \state, \action, \state'}}
    %\!\!\!\!\!\!\!\!\!\!\!\!\!\!\!\!
    \min_{\encoderparameter, \decoderparameter} \! \! \! \expectedsymbol{}_{\substack{\state, \action, \state' \sim \stationary{\policy} \\ \latentstate, \latentaction, \latentstate' \sim \embed_{\encoderparameter}({\sampledot \mid \state, \action, \state'})}} \! \! \! \!
    \left[{\distance_{\states}\fun{\state, \generative_{\decoderparameter}\fun{\latentstate}} + 
    \distance_{\actions}\fun{\action, \embeda_{\decoderparameter}\fun{\latentstate, \latentaction}} + 
    \distance_{\states}\fun{\state', \generative_{\decoderparameter}\fun{\latentstate'}}}\right] + \localrewardloss{\stationary{\policy}} + \beta \cdot ({ \steadystateregularizer{\policy} + \localtransitionloss{\stationary{\policy}}}),
\end{equation*}
% \begin{aligned}
%     \min_{\encoderparameter, \decoderparameter} \expectedsymbol{}_{\substack{\state, \action, \reward, \state' \sim \stationary{\policy} \\ \latentstate, \latentaction, \latentstate' \sim \embed_{\encoderparameter}\fun{\sampledot \mid \state, \action, \state'}}}
% \end{aligned}
% \begin{aligned}
%     &\big[\tracedistance\fun{\tuple{\state, \action, \reward, \state'}, G_{\decoderparameter}\fun{\latentstate, \latentaction, \latentstate'}}\\
%     & \quad\quad\quad\quad + \beta \cdot \fun{ \wassersteindist{\distance_{\latentstates}}{\originaltolatentstationary{}}{ \latentstationaryprior} +
%     \wassersteindist{\distance_{\latentstates}}{\embed_{\encoderparameter}\probtransitions\fun{\sampledot \mid \state, \action}}{\latentprobtransitions_{\decoderparameter}\fun{\sampledot \mid \latentstate, \latentaction}}}\big]
% \end{aligned}
%
%
%\begin{equation*}
% \min_{\encoderparameter, \decoderparameter} \! \! \! \expectedsymbol{}_{\substack{\state, \action, \reward, \state' \sim \stationary{\policy} \\ \latentstate, \latentaction, \latentstate' \sim \embed_{\encoderparameter}\fun{\sampledot \mid \state, \action, \state'}}} \! \! \! \!
% \left[
% \tracedistance\fun{\tuple{\state, \action, \reward, \state'}, G_{\decoderparameter}\fun{\latentstate, \latentaction, \latentstate'}} +
% \beta \cdot \fun{ \steadystateregularizer{\policy} + \wassersteindist{\distance_{\latentstates}}{\embed_{\encoderparameter}\probtransitions\fun{\sampledot \mid \state, \action}}{\latentprobtransitions_{\decoderparameter}\fun{\sampledot \mid \latentstate, \latentaction}}}
% \right],
% \end{equation*}
where $\steadystateregularizer{\policy} = \wassersteindist{\transitiondistance}{\originaltolatentstationary{}}{\latentstationaryprior}$ and
$\localtransitionloss{\stationary{\policy}}$ are respectively called \emph{steady-state} and \emph{transition} regularizers.
The former allows to quantify the distance between the stationary distributions respectively induced by $\policy$ in $\mdp$ and $\latentpolicy_{\decoderparameter}$ in $\latentmdp_{\decoderparameter}$, further enabling the distillation. % of $\policy$ into $\latentpolicy_{\decoderparameter}$.
The latter allows to learn the latent dynamics.
%and is actually closely related to the local transition loss:
%
%
%\smallparagraph{Distillation.}~
Note that $\localrewardloss{\stationary{\policy}}$ and $\localtransitionloss{\stationary{\policy}}$ --- set over $\stationary{\policy}$ instead of $\stationary{\latentpolicy_{\decoderparameter}}$ --- are not sufficient to ensure the bisimulation bounds (Eq.~\ref{eq:bidistance-bound}): running $\policy$ in $\latentmdp_{\decoderparameter}$ depends on the parallel execution of $\policy$ in the original model, which does not permit its (conventional) verification.  % reasoning about the latent model alone.
Breaking this dependency is enabled by learning the distillation $\latentpolicy_{\decoderparameter}$ through $\steadystateregularizer{\policy}$, as shown in Fig.~\ref{subfig:latent-fow-distillation}:
%
%The distillation of the original policy $\policy$ into $\latentpolicy_{\decoderparameter}$ is enabled through the latent flow shown in Fig.~\ref{subfig:latent-fow-distillation}.
% At each distillation step, a state $\state$ is recovered from $\stationary{\policy}$.
% Given $\state$, the execution of $\policy$ in $\latentmdp$ is made possible via the action encoder $\actionencoder$:
%For each state $\state$ produced from $\stationary{\policy}$, the execution of $\policy$ in $\latentmdp$ is made possible via the action encoder $\actionencoder$:
%we write $\latentaction \sim \policy\fun{\sampledot \mid \state}$ for drawing $\action \sim \policy\fun{\sampledot \mid \state}$, followed by $\latentaction \sim \actionencoder\fun{\sampledot \mid \embed_{\encoderparameter}\fun{\state}, \action}$.
%Given this, we can rewrite the transition regularizer as $\expectedsymbol{\state, \tuple{\action, \latentaction} \sim \stationary{\policy}} \wassersteindist{\distance_{\latentstates}}{\embed_{\encoderparameter}\probtransitions\fun{\sampledot \mid \state, \action}}{\latentprobtransitions\fun{\sampledot \mid \embed_{\encoderparameter}\fun{\state}, \latentaction}} = \localtransitionloss{\stationary{\policy}}$.
% This means that minimizing the transition regularizer by executing the original policy $\policy$ is equivalent to minimizing $\localtransitionloss{\stationary{\policy}}$, further yielding bisimulation guarantees derived from the execution \emph{of the original policy $\policy$} (replacing $\latentpolicy$ in Eq.~\ref{eq:bidistance-bound}),
minimizing
%the steady-state regularizer
$\steadystateregularizer{\policy}$
allows to make $\stationary{\policy}$ and $\latentstationaryprior$ closer together, further bridging the gap of the discrepancy between $\policy$ and $\latentpolicy_{\decoderparameter}$.
%Note that the guarantees linked to $\localtransitionloss{\stationary{\policy}}$ are impractical alone due to the dependency of the latent execution of $\policy$ on original states $\state \in \states$; minimizing $\steadystateregularizer{\policy}$ to recover $\latentpolicy_{\decoderparameter}$ and the guarantees linked to $\localtransitionloss{\stationary{\latentpolicy_{\decoderparameter}}}$ breaks this dependency.
%At any time, considering the latent policy resulting from this distillation allows recovering the local losses along with the linked the bisimulation bounds in the objective function of the \waemdp: %, as stated in the following Theorem: 
At any time, recovering the local losses along with the linked bisimulation bounds in the objective function of the \waemdp is allowed by considering the latent policy resulting from this distillation:
\begin{restatable}{theorem}{latentexecutionobjective}\label{thm:latent-execution-objective}
Assume that traces are generated by running a latent policy $\latentpolicy \in \latentpolicies$ in the original environment and let $\distance_{\rewards}$ be the usual Euclidean distance, then the \waemdp objective is
\begin{equation*}
%\[
    \min_{\encoderparameter, \decoderparameter} \expected{\state, \state' \sim \stationary{\latentpolicy}}{\distance_{\states}\fun{\state, \generative_{\decoderparameter}\fun{\embed_{\encoderparameter}\fun{\state}}} + \distance_{\states}\fun{\state', \generative_{\decoderparameter}\fun{\embed_{\encoderparameter}\fun{\state'}}}} + \localrewardloss{\stationary{\latentpolicy}} + \beta \cdot ( \steadystateregularizer{\latentpolicy} + \localtransitionloss{\stationary{\latentpolicy}}).
%\]
\end{equation*}
\end{restatable}%%
%
\smallparagraph{Optimizing the regularizers}~%
%One may argue that we could replace $\wassersteinsymbol{\distance_{\latentstates}}$ by $\dtvsymbol{}$ since we aim at learning a discrete latent space model.
%However, we enable the learning of discrete latent distributions through continuous relaxations, meaning that the probability of sampling the same latent state from $\encoder$ and $\originaltolatentstationary{}$ as well as $\embed_{\encoderparameter}\probtransitions\fun{\sampledot \mid \state, \action}$ and $\latentprobtransitions_{\decoderparameter}\fun{\sampledot \mid \latentstate, \latentaction}$ is zero during training.
%Recall that $\dtvsymbol = \wassersteinsymbol{\condition{\neq}},$ the TV distance will consequently assign a 1-distance to each pair of latent states drawn, even if their relaxed continuous latent representation are close to each other, thus yielding an overly strong measure of the numerical difference between the distributions.
% Consequently, we optimize $\wassersteinsymbol{\distance_{\latentstates}}$ for a suitable choice of $\distance_{\latentstates}$.
%The following Lemma enables the optimization of the \waemdp objective through a minimax learning procedure.
is enabled by the dual form of the OT: we introduce two parameterized networks, $\steadystatenetwork$ and $\transitionlossnetwork$, constrained to be $1$-Lipschitz and trained to attain the supremum of the dual:
%\begin{align*}
\[
\steadystateregularizer{\policy}\fun{\wassersteinparameter} = \max_{\wassersteinparameter %\colon \steadystatenetwork \in \Lipschf{\tracedistance}
} \; \expectedsymbol{\state, \action \sim \stationary{\policy}}\expectedsymbol{\latentaction \sim \actionencoder\fun{\sampledot \mid \embed_{\encoderparameter}\fun{\state}, \action}}\expectedsymbol{\latentstate^{\star} \sim \latentprobtransitions_{\decoderparameter}\fun{\sampledot \mid \embed_{\encoderparameter}\fun{\state}, \latentaction}} \steadystatenetwork\fun{\embed_{\encoderparameter}\fun{\state}, \latentaction, \latentstate^{\star}}
- \expectedsymbol{\latentvariable, \latentaction', \latentvariable' \sim \latentstationaryprior}\steadystatenetwork\fun{\latentvariable, \latentaction', \latentvariable'}
\]
\[
\localtransitionloss{\stationary{\policy}}\fun{\wassersteinparameter} = \max_{\wassersteinparameter
%\colon \transitionlossnetwork \in \Lipschf{\distance_{\scriptscriptstyle \latentstates}}
} \expectedsymbol{\state, \action, \state' \sim \stationary{\policy}} \expectedsymbol{\latentstate, \latentaction, \latentstate' \sim \embed_{\encoderparameter}\fun{\sampledot \mid \state, \action, \state'}}\Big[{\transitionlossnetwork\fun{\state, \action, \latentstate, \latentaction, \latentstate'} - \! \expectedsymbol{\latentstate^{\star} \sim \latentprobtransitions_{\decoderparameter}\fun{\sampledot \mid \latentstate, \latentaction}} \transitionlossnetwork\fun{\state, \action, \latentstate, \latentaction, \latentstate^{\star}}}\Big]
\]
%\end{align*}
Details to derive this tractable form of $\localtransitionloss{\stationary{\policy}}\fun{\wassersteinparameter}$ are in Appendix~\ref{appendix:tractable-transition-regularizer}.
% To constrain the networks, we use
% We constrain the networks via
The networks are constrained via
the gradient penalty approach of 
% \citet{DBLP:conf/nips/GulrajaniAADC17},
\cite{DBLP:conf/nips/GulrajaniAADC17}, leveraging that any differentiable function is $1$-Lipschitz iff it has gradients with norm at most $1$ everywhere
(we show in Appendix~\ref{appendix:latent-metric} this is still valid for relaxations of discrete spaces).
%
% The constraint is enforced through a penalization term.
%(cf. Algorithm~\ref{alg:gradient-penalty}).
The final learning process is presented in Algorithm~\ref{alg:wwae-mdp}. 
%
% \smallparagraph{Enforcing 1-Lipschitzness.}
% Take $\varphi_{\wassersteinparameter} \colon \latentvariables \to \R$ as $\varphi^{\stationary{}}_{\wassersteinparameter}$ if $\latentvariables = \latentstates \times \latentactions \times \latentstates$, and $\latentstate' \mapsto \varphi_{\wassersteinparameter}^{\probtransitions}\fun{\state, \action, \latentstate, \latentaction, \latentstate'}$ if $\latentvariables = \latentstates$ for some fixed state-action pair $\tuple{\state, \action} \in \states \times \actions$ and embedding $\tuple{\latentstate, \latentaction} \in \support{\embed_{\encoderparameter}\fun{\sampledot \mid \state, \action}}$.
% To enforce the Lipschitz condition on $\varphi_{\wassersteinparameter} $, we use the gradient penalty approach of 
% % \citet{DBLP:conf/nips/GulrajaniAADC17},
% \cite{DBLP:conf/nips/GulrajaniAADC17},
% based on the fact that $1$-Lipschitzness implies $\norm{\gradient_{\latentvariable}\, \varphi_{\wassersteinparameter}\fun{\latentvariable}} \leq 1$ almost surely
% when $\distance_{\latentstates}$ is induced by a norm $\norm{\cdot}$ defined over $\latentstates$ (e.g., the usual $\ell_2$ norm).
% %
% This condition can be enforced through a penalization term $\expected{\latentvariable \sim \chi}{\fun{\norm{\gradient_{\latentvariable}\, \varphi_{\wassersteinparameter}\fun{\latentvariable}} - 1}^2}$, where $\chi$ is defined as sampling uniformly along straight lines between pairs of latent variables drawn from either $\originaltolatentstationary{}$ and $\latentstationaryprior$, or $\embed_{\encoderparameter}\probtransitions$ and $\latentprobtransitions_{\decoderparameter}$ (cf. Algorithm~\ref{alg:gradient-penalty}).

% %\begin{minipage}{0.99\linewidth}
% %\begin{minipage}{0.48\textwidth}
% \begin{algorithm}%[H]
% \caption{Wasserstein$^2$ Auto-Encoded MDP}\label{alg:wwae-mdp}
% \begin{algorithmic}
% \Require 
%     batch size $N$, max. step $T$, no. of regularizer updates $\ncritic$, penalty coefficient $\delta$
% \For {$t = 1$ to $T$}
% \For{$i = 1$ to $N$}
%     \State Draw $\state^i, \action^i, \reward^i, \state^{\prime\, i}$ from $\stationary{\policy}$
%     \State Draw $\latentstate_{\embed}^{\,i}, \latentaction_{\embed}^{\,i}, \latentstate_\embed^{\prime\, i}$ from $\embed_{\encoderparameter}({\sampledot \mid \state^i, \action^i, \state^{\prime\, i}})$
%     \State Draw $\latentstate^{\prime\, i}_{\scriptscriptstyle \probtransitions}$ from $\latentprobtransitions_{\decoderparameter}({\sampledot \mid \latentstate_{\embed}^{\,i}, \latentaction_{\embed}^{\, i}})$
%     \State Draw $\latentstate^{\, i}_{\stationary{}}, \latentaction^{\, i}_{\stationary{}}, \latentstate^{\prime\, i}_{\stationary{}}$ resp. from $\latentstationaryprior$, $\latentpolicy_{\decoderparameter}({\sampledot \mid \latentstate^{\, i}_{\stationary{}}})$, and $\latentprobtransitions_{\decoderparameter}({\sampledot \mid \latentstate^{\, i}_{\stationary{}}, \latentaction^{\, i}_{\stationary{}}})$
% \EndFor
% \State \Comment{Wasserstein regularizer ($\max_{\wassersteinparameter}$)}
% \State
% $
% \begin{aligned}
% 	\mathcal{W} \gets \nicefrac{\beta}{N} \cdot \textstyle \sum_{i = 1}^{N} \varphi_{\wassersteinparameter}^{\stationary{}}\fun{\latentstate_{\embed}^{\,i}, \latentaction_{\embed}^{\,i}, \latentstate^{\prime\, i}_{\scriptscriptstyle \probtransitions}}
% 	- \varphi_{\wassersteinparameter}^{\stationary{}}\fun{\latentstate^{\, i}_{\stationary{}}, \latentaction^{\, i}_{\stationary{}}, \latentstate^{\prime\, i}_{\stationary{}}} + \quad \quad \quad \quad \quad \\
% 	\hfill \varphi_{\wassersteinparameter}^{\probtransitions}\fun{\state^{i}, \action^{i}, \latentstate_{\embed}^{\, i}, \latentaction_{\embed}^{\, i}, \latentstate^{\prime\, i}_{\embed}} - \varphi_{\wassersteinparameter}^{\probtransitions}\fun{\state^{i}, \action^{i}, \latentstate_{\embed}^{\, i}, \latentaction_{\embed}^{\, i}, \latentstate^{\prime\, i}_{\scriptscriptstyle \probtransitions}}
% \end{aligned}
% $
% \State
% $
% \begin{aligned}
% 	P \gets \nicefrac{\delta}{N} \cdot \textstyle \sum_{i = 1}^{N} \mathsf{GradientPenalty}\big({\varphi_{\wassersteinparameter}^{\stationary{}}, \tuple{\latentstate_{\embed}^{\,i}, \latentaction_{\embed}^{\,i}, \latentstate^{\prime\, i}_{\scriptscriptstyle \probtransitions}}, \tuple{\latentstate^{\, i}_{\stationary{}}, \latentaction^{\, i}_{\stationary{}}, \latentstate^{\prime\, i}_{\stationary{}}}}\big) + \quad \\
% 	\hfill \mathsf{GradientPenalty}\big({\latentstate' \mapsto \varphi_{\wassersteinparameter}^{\probtransitions}\fun{\state^i, \action^i, \latentstate^{\, i}, \latentaction^{\, i}, \latentstate'}, \latentstate^{\prime \, i}_{\embed}, \latentstate^{\prime\, i}_{\scriptscriptstyle \probtransitions}}\big)
% \end{aligned}
% $ \Comment{Algorithm~\ref{alg:gradient-penalty}}
% \State Update $\wassersteinparameter$ by ascending $\mathcal{W}$ - $P$
% \State \Comment{Auto-Encoder loss ($\min_{\encoderparameter, \decoderparameter}$)}
% \If{$t \bmod \ncritic = 0$}
% \State 
% %\begin{multline*}
%  $ \mathcal{L} \gets \nicefrac{1}{N} \cdot \sum_{i = 1}^{N} \norm{\state^i- \generative_{\decoderparameter}\fun{\latentstate_{\embed}^{\, i}}} + \norm{\action^i - \embeda_{\decoderparameter}\fun{\latentstate_{\embed}^{\, i}, \latentaction_{\embed}^{\, i}}} + \left| \reward^i - \latentrewards_{\decoderparameter}\fun{\latentstate_{\embed}^{\, i}, \latentaction_{\embed}^{\, i}} \right| + \norm{\state^{\prime\, i} - \generative_{\decoderparameter}\fun{\latentstate_{\embed}^{\prime\, i}}}$
%  \State Update $\tuple{\encoderparameter, \decoderparameter}$ by descending $\mathcal{L} + \mathcal{W}$
% %\end{multline*}
% \EndIf
% \EndFor
% \end{algorithmic}
% \end{algorithm}
% %
% \begin{algorithm}
% \caption{Gradient penalty \citep{DBLP:conf/nips/GulrajaniAADC17}}\label{alg:gradient-penalty}
% \begin{algorithmic}
% \Require Function $\varphi_{\wassersteinparameter} \colon \R^n \to \R$, and $\vect{x}, \vect{y} \in \R^n$ %so that $\vect{x} \sim P$, $\vect{y} \sim Q$ in $\expectedsymbol{\vect{x} \sim P} \varphi_{\wassersteinparameter}\fun{\vect{x}} - \expectedsymbol{\vect{y} \sim Q}\varphi_{\wassersteinparameter}\fun{\vect{y}}$
% \Ensure Penalty for enforcing the $1$-Lipschitzness of $\varphi_{\wassersteinparameter}$
% % \State $\epsilon \gets \mathit{U}\fun{0, 1}$ \Comment{random noise}
% % \State $\vect{z} \gets \epsilon \vect{x} + (1 - \epsilon) \vect{y}$ \Comment{straight lines between $\vect{x}$ and $\vect{y}$}
% \State $\epsilon \gets \mathit{U}\fun{0, 1}$; $\vect{z} \gets \epsilon \vect{x} + (1 - \epsilon) \vect{y}$ \Comment{random noise; straight lines between $\vect{x}$ and $\vect{y}$}
% \State \Return $\fun{\norm{\gradient_{\vect{z}} \, \varphi_{\wassersteinparameter}\fun{\vect{z}}} - 1}^2$
% \end{algorithmic}
% \end{algorithm}
%
% \begin{algorithm}
% \caption{Gradient penalty \citep{DBLP:conf/nips/GulrajaniAADC17}}\label{alg:gradient-penalty}
% \begin{algorithmic}
% \Require Function $\varphi_{\wassersteinparameter} \colon \R^n \to \R$, and $\vect{x}, \vect{y} \in \R^n$ %so that $\vect{x} \sim P$, $\vect{y} \sim Q$ in $\expectedsymbol{\vect{x} \sim P} \varphi_{\wassersteinparameter}\fun{\vect{x}} - \expectedsymbol{\vect{y} \sim Q}\varphi_{\wassersteinparameter}\fun{\vect{y}}$
% \Ensure Penalty for enforcing the $1$-Lipschitzness of $\varphi_{\wassersteinparameter}$
% % \State $\epsilon \gets \mathit{U}\fun{0, 1}$ \Comment{random noise}
% % \State $\vect{z} \gets \epsilon \vect{x} + (1 - \epsilon) \vect{y}$ \Comment{straight lines between $\vect{x}$ and $\vect{y}$}
% \State $\epsilon \gets \mathit{U}\fun{0, 1}$; $\vect{z} \gets \epsilon \vect{x} + (1 - \epsilon) \vect{y}$ \Comment{random noise; straight lines between $\vect{x}$ and $\vect{y}$}
% \State \Return $\fun{\norm{\gradient_{\vect{z}} \, \varphi_{\wassersteinparameter}\fun{\vect{z}}} - 1}^2$
% \end{algorithmic}
% \end{algorithm}

\begin{figure}
    \centering
    \includegraphics[width=\textwidth]{ressources/WAE-architecture.pdf}
    \caption{\waemdp architecture.
    %Relaxed random variables become discrete as $\temperature \to 0$. 
    Distances are depicted by red dotted lines.}
    \label{fig:wae-architecture}
\end{figure}
%Although \waemdps are able to learn a wide range of latent space models, our goal is to learn a \emph{discrete} latent model to finally formally verify it and lift the guarantees derived by model checking  back to the original model, supported by the guaranteed bisimulation bounds linking the two models.
\subsection{Discrete Latent Spaces}\label{sec:discrete-latent-spaces}
To enable the verification of latent models supported by the bisimulation guarantees of Eq.~\ref{eq:bidistance-bound}, we focus on the special case of \emph{discrete latent space models}.
Our approach relies on continuous relaxation of discrete random variables, regulated by some \emph{temperature} parameter(s) $\temperature$: discrete random variables are retrieved as $\temperature \to 0$, which amounts to applying a rounding operator.
For training, we use the temperature-controlled relaxations to differentiate the objective and let the gradient flow through the network.
When we deploy the latent policy in the environment and formally check the latent model, the zero-temperature limit is used.
% We show in Appendix~\ref{appendix:latent-metric} that the gradient penalty approach is still valid for such relaxations.
An overview of the approach is depticted in Fig.~\ref{fig:wae-architecture}.

\smallparagraph{State encoder.}~%
We work with a \emph{binary representation} of the latent states.
First, this induces compact networks, able to deal with a large discrete space via a tractable number of parameter variables.
But most importantly, this % allows us to encode the labels linearly into the state space, 
ensures that Assumption~\ref{assumption:vae-mdp} is satisfied:
let $n = \log_2 | \latentstates|$,
we reserve $\left| \atomicprops \right|$ bits in $\latentstates$ and each time $\state \in \states$ is passed to $\embed_{\encoderparameter}$, $n- \left|\atomicprops\right|$ bits are produced and concatenated with $\labels\fun{\state}$, ensuring a perfect reconstruction of the labels and further bisimulation bounds.
%
To produce Bernoulli variables, $\embed_{\encoderparameter}$ deterministically maps $\state$ to a latent code $\vz$, passed to the Heaviside $H\fun{\vz} = \condition{\vz > 0}$.
We train $\embed_{\encoderparameter}$ by using the smooth approximation $H_{\temperature}\fun{\vz} = \sigma\fun{\nicefrac{2\vz}{\temperature}}$, satisfying $H = \lim_{\temperature \to 0} H_{\temperature}$.
% This allows the gradients to carry the necessary information to change the output of the encoder in the direction required to optimize the objective function.

\smallparagraph{Latent distributions.}~%
Besides the discontinuity of their latent image space, a major challenge of optimizing over discrete distributions is \emph{sampling}, required to be a differentiable operation.
We circumvent this by using \emph{concrete distributions} \citep{DBLP:conf/iclr/JangGP17,DBLP:conf/iclr/MaddisonMT17}:
the idea is to sample reparameterizable random variables from $\temperature$-parameterized distributions, and applying a differentiable, nonlinear operator in downstream. %to eventually retrieve discrete random variables as $\temperature$ goes to $0$.
We use the \emph{Gumbel softmax trick} to sample from distributions over (one-hot encoded) latent actions ($\actionencoder$, $\latentpolicy_{\decoderparameter}$).
For binary distributions ($\latentprobtransitions_{\decoderparameter}$, $\latentstationaryprior$), each relaxed Bernoulli with logit $\alpha$ is retrieved by drawing a logistic random variable located in $\nicefrac{\alpha}{\temperature}$ and scaled to $\nicefrac{1}{\temperature}$, then applying a sigmoid in downstream.
We emphasize that this trick alone (as used by \citealt{DBLP:conf/icml/CorneilGB18,DBLP:journals/corr/abs-2112-09655}) is not sufficient: it yields independent Bernoullis, being too restrictive in general, which prevents from learning sound transition dynamics (cf. Ex.~\ref{ex:independent-ber-not-sufficient}). % -- already insufficient for the label dynamics, as illustrated in Ex.~\ref{ex:independent-ber-not-sufficient}.
% This insufficiency is illustrated in Ex.~\ref{ex:independent-ber-not-sufficient}.
%
\begin{wrapfigure}{r}{0.4\textwidth}
\begin{tikzpicture}[->,shorten >=1pt,auto,node distance=2.5cm,bend angle=45, scale=0.6, font=\small]
\tikzstyle{state}=[draw,circle,text centered,minimum size=7mm,text width=4mm]
\tikzstyle{act}=[fill,circle,inner sep=1pt,minimum size=1.5pt, node distance=1cm]    
\tikzstyle{empty}=[text centered, text width=15mm]
\node[state] (0) at (0, 0)     {$\latentstate_0$};
\node[state] (1) at (2.5, .8)   {$\latentstate_1$};
\node[empty] (l1) at (1.2, 1.4)  {$\{\textit{\color{OliveGreen}goal}\}$};
\node[state] (2) at (2.5, -.8)  {$\latentstate_2$};
\node[empty] (l2) at (1., -1.3) {$\{\textit{\color{BrickRed}unsafe}\}$};
\node[state] (3) at (-2.5, 0)    {$\latentstate_3$};
\node[empty] (l2) at (-2.5, 1)   {$\{\textit{\color{BrickRed}unsafe}\}$};
\draw[->] (0) -- node[above]{\scriptsize $\nicefrac{1}{2}$} (1);
\draw[->] (0) -- node[below]{\scriptsize $\nicefrac{1}{4}$} (2);
\draw[->] (0) -- node[above]{\scriptsize $\nicefrac{1}{4}$} (3);
\draw[->] (1) edge[out=335,in=25,loop] node[left]{\scriptsize $1$} (1);
\draw[->] (2) edge[out=335,in=25,loop] node[left]{\scriptsize $1$} (2);
\draw[->] (3) edge[out=215,in=155,loop] node[right]{\scriptsize $1$} (3);
\end{tikzpicture}
\caption{Markov Chain with four states; labels are drawn next to their state.}
\label{fig:markov-chain-independent-bernoulli}
\end{wrapfigure}%
\begin{example}\label{ex:independent-ber-not-sufficient}
% \begin{figure}
% \begin{tikzpicture}[->,shorten >=1pt,auto,node distance=2.5cm,bend angle=45, scale=0.6, font=\small]
% \tikzstyle{state}=[draw,circle,text centered,minimum size=7mm,text width=4mm]
% \tikzstyle{act}=[fill,circle,inner sep=1pt,minimum size=1.5pt, node distance=1cm]    
% \tikzstyle{empty}=[text centered, text width=15mm]
% \node[state] (0) at (0, 0)     {$\latentstate_0$};
% \node[state] (1) at (5, .8)   {$\latentstate_1$};
% \node[empty] (l1) at (3.6, 1.1)  {$\{\textit{goal}\}$};
% \node[state] (2) at (5, -.8)  {$\latentstate_2$};
% \node[empty] (l2) at (3.5, -1.25) {$\{\textit{unsafe}\}$};
% \node[state] (3) at (-5, 0)    {$\latentstate_3$};
% \node[empty] (l2) at (-5, 1)   {$\{\textit{unsafe}\}$};
% \draw[->] (0) -- node[above]{\scriptsize $\nicefrac{1}{2}$} (1);
% \draw[->] (0) -- node[below]{\scriptsize $\nicefrac{1}{4}$} (2);
% \draw[->] (0) -- node[above]{\scriptsize $\nicefrac{1}{4}$} (3);
% \draw[->] (1) edge[out=335,in=25,loop] node[left]{\scriptsize $1$} (1);
% \draw[->] (2) edge[out=335,in=25,loop] node[left]{\scriptsize $1$} (2);
% \draw[->] (3) edge[out=215,in=155,loop] node[right]{\scriptsize $1$} (3);
% \end{tikzpicture}
% \centering
% \caption{Simple Markov Chain with four states; labels are drawn next to their state.}
% \label{fig:markov-chain-independent-bernoulli}
% \end{figure}%
Let $\latentmdp$ be the discrete MC of Fig.~\ref{fig:markov-chain-independent-bernoulli}.
In one-hot, $\atomicprops = \{\textit{goal}\!: \tuple{{\color{OliveGreen}1}, 0}, \textit{unsafe}\!: \tuple{0, {\color{BrickRed}1}}\}$.
We assume that $3$ bits are used for the (binary) state space, with $\latentstates = \{\latentstate_0 \!: \tuple{{0, 0}, 0}, \latentstate_1\!: \tuple{{\color{OliveGreen}1}, 0, 0}, \latentstate_2\!: \tuple{0, {\color{BrickRed}1}, 0}, \latentstate_3 \!: \tuple{0, {\color{BrickRed}1}, 1}\}$ (the two first bits are reserved for the labels).
Considering each bit as being independent is not sufficient to learn $\latentprobtransitions$: 
%each outgoing bit-wise transition probability being learned independently, 
the optimal estimation $\latentprobtransitions_{\decoderparameter^{\star}}\fun{\sampledot \mid \latentstate_0}$ is in that case represented by the independent Bernoulli vector $\mathbf{b} = \tuple{\nicefrac{1}{2}, \nicefrac{1}{2}, \nicefrac{1}{4}}$, giving the probability to go from $\latentstate_0$ to each bit \emph{independently}.
This yields a poor estimation of the actual transition function:
% \begin{align*}
%     \latentprobtransitions_{\decoderparameter^{\star}}\fun{\latentstate_0 \mid \latentstate_0} &= (1 - \mathbf{b}_1) \cdot (1 - \mathbf{b}_2) \cdot (1 - \mathbf{b}_3) = 0.1875 \\
%     \latentprobtransitions_{\decoderparameter^{\star}}\fun{\latentstate_1 \mid \latentstate_0} &= \mathbf{b}_1 \cdot (1 - \mathbf{b}_2) \cdot (1 - \mathbf{b}_3) = 0.1875 \\
%     \latentprobtransitions_{\decoderparameter^{\star}}\fun{\latentstate_2 \mid \latentstate_0} &= (1 - \mathbf{b}_1) \cdot \mathbf{b}_2 \cdot (1 - \mathbf{b}_3) = 0.1875 \\
%     \latentprobtransitions_{\decoderparameter^{\star}}\fun{\latentstate_3 \mid \latentstate_0} &= (1 - \mathbf{b}_1) \cdot \mathbf{b}_2 \cdot \mathbf{b}_3 = 0.0625.
% \end{align*}
$
    \latentprobtransitions_{\decoderparameter^{\star}}\fun{\latentstate_0 \mid \latentstate_0} = (1 - \mathbf{b}_1) \cdot (1 - \mathbf{b}_2) \cdot (1 - \mathbf{b}_3) = %0.1875, \,
    \latentprobtransitions_{\decoderparameter^{\star}}\fun{\latentstate_1 \mid \latentstate_0} = \mathbf{b}_1 \cdot (1 - \mathbf{b}_2) \cdot (1 - \mathbf{b}_3) = %0.1875, \,
    \latentprobtransitions_{\decoderparameter^{\star}}\fun{\latentstate_2 \mid \latentstate_0} = (1 - \mathbf{b}_1) \cdot \mathbf{b}_2 \cdot (1 - \mathbf{b}_3) = %0.1875,\,
    \nicefrac{3}{16}, \,
    \latentprobtransitions_{\decoderparameter^{\star}}\fun{\latentstate_3 \mid \latentstate_0} = (1 - \mathbf{b}_1) \cdot \mathbf{b}_2 \cdot \mathbf{b}_3 = \nicefrac{1}{16}
$.%
\end{example}%
We consider instead relaxed multivariate Bernoulli distributions by decomposing $P \in \distributions{\latentstates}$
as a product of conditionals:
%via the chain rule:
$
P\fun{\latentstate} = \prod_{i=1}^{n} P\fun{\latentstate_{i} \mid \latentstate_{1 \colon i-1}}
$
%for some permutation function $\mu \colon \left[n\right] \to \left[n\right]$,
where $\latentstate_i$ is the $i^\text{th}$ entry (bit) of $\latentstate$.
We learn such distributions by introducing
% We introduce such distributions by learning
a \emph{masked autoregressive flow} (MAF, \citealt{DBLP:conf/nips/PapamakariosMP17}) for relaxed Bernoullis via the recursion:
%(ordered by $\mu$):
$\latentstate_i = \sigma\fun{\nicefrac{l_i + \alpha_i}{\temperature}}$, \text{where} $l_i \sim \logistic{0}{1}$, $\alpha_i = f_i\fun{\latentstate_{1\colon i - 1}}$, \text{and} $f$ is a MADE \citep{DBLP:conf/icml/GermainGML15}, a feedforward network implementing the conditional output dependency on the inputs via a mask that only keeps the necessary connections to enforce the conditional property.
% MADEs enable computing $P\fun{\latentstate}$ in a single forward pass and sampling through $n$ passes. 
We use this MAF to model $\latentprobtransitions_{\decoderparameter}$ and the dynamics 
related to
% of
the labels in $\latentstationaryprior$.
%For the remaining $n - \left| \atomicprops \right|$ bits, we independently fix their logits to $0$ to allow for a fairly distributed latent space.
We fix the logits of the remaining $n - \left| \atomicprops \right|$ bits to $0$ to allow for a fairly distributed latent space.
%we independently set the logits of the  remaining $n - \left| \atomicprops \right|$ bits to $0$ to allow for a fairly distributed latent space.



% \smallparagraph{Latent metric.}~%As stated above, continuous relaxations relying on a temperature parameter $\temperature$ are used during training to allow the objective function to be optimized via gradient descent in the backward pass, while discrete random variables are used in the forward pass, as $\temperature$ goes to $0$.
% Since we use continuous relaxations, we naturally consider the usual Euclidean distance as latent metric to optimize our regularizers.
% This turns out to be indeed Lipschitz equivalent to considering a continuous $\temperature$-relaxation of the discrete metric $\condition{\neq}$, as stated in the following Theorem.
% % Consequently, this also means it is consistently sufficient to enforce $1$-Lispchitzness via the gradient penalty approach during training to maintain the guarantees linked to the regularizers in the zero-temperature limit, when the spaces are discrete.
% %
% \begin{theorem}\label{thm:lipsch-equiv}
% Let $d$ be the Euclidean distance and $d_{\temperature}\fun{\vx, \vy} = \nicefrac{d(\vx, \vy)}{\temperature + d(\vx, \vy)}$, then $d$ and $d_{\temperature}$ are Lipschitz equivalent.
% Hence, for all $\beta \geq \nicefrac{1}{\temperature}$, $\state \in \states$, $\action \in \actions$, $\latentstate \in \latentstates$, $\latentaction \in \latentactions$,
% %\begin{enumerate}
%     %\item
%     (i) $\wassersteindist{\distance_{\temperature}}{\originaltolatentstationary{}}{\latentstationaryprior} \leq \beta \cdot \wassersteindist{\distance}{\originaltolatentstationary{}}{\latentstationaryprior}$, and
%     (ii) $\wassersteindist{\distance_{\temperature}}{\embed_{\encoderparameter}\probtransitions\fun{\sampledot\mid \state, \action}}{\latentprobtransitions_{\decoderparameter}\fun{\sampledot \mid \latentstate, \latentaction}} \leq \beta \cdot \wassersteindist{\distance}{\embed_{\encoderparameter}\probtransitions\fun{\sampledot\mid \state, \action}}{\latentprobtransitions_{\decoderparameter}\fun{\sampledot \mid \latentstate, \latentaction}}$.
% %\end{enumerate}
% \end{theorem}
%
% Note that optimizing over two different $\beta_1, \beta_2$ instead of a unique scale factor $\beta$ is also a good practice to interpolate between the two regularizers.
%
%\end{minipage}
%
\section{Experiments}\label{sec:experiments}
We present in section~\ref{ssec:faces} an application of PnP-HVAE on face images, using a pretrained state-of-the-art hierarchical VAE. 
Next, we study the application of our framework to natural images. To that end, we introduce  in section~\ref{ssec:patchVDVAE}  a patch hierachical VAE architecture, that is able to model natural images of different resolutions. In section~\ref{ssec:app_nat}, we provide deblurring, super-resolution and inpainting experiments to demonstrate the relevance of the proposed method.

Additional results are presented in Appendix~\ref{app:add}. All experiments can be reproduced using the code available at \url{https://github.com/jprost76/PnP-HVAE}.



\subsection{Face Image restoration (FFHQ)}\label{ssec:faces}
We first demonstrate the effectiveness of PnP-HVAE on highly structured data, by performing face image restoration.
Latent variable generative models can accurately model structured images such as face images \cite{karras2019style,vahdat2020nvae,child2021very,kingma2018glow}, and then be used to produce high quality restoration of such data. 
In our experiments, we use the VDVAE model of~\cite{child2021very}, pre-trained on the FFHQ dataset~\cite{karras2019style}, as our hierarchical VAE prior.
VDVAE has $L=66$ latent variable groups in its hierarchy and generates images at resolution $256\times256$.

We compare PnP-HVAE with the intermediate layer optimization algorithm (ILO)~\cite{daras2021intermediate} that is based on a different class of generative models than HVAE. ILO is a GAN inversion method which optimizes the image latent code along with the intermediate layer representation of a StyleGAN to generate an image consistent with a degraded observation.
We use the official implementation of ILO, along with a StyleGAN2 model~\cite{karras2020analyzing, stylegan2pytorch}, that was trained for 550k iterations on images of resolution $256\times256$ from FFHQ.  
As VDVAE and StyleGAN models are not trained on the same train-test split of FFHQ, we chose to evaluate the methods on a subset of 100 images from the CelebA dataset~\cite{liu2018large}. 
For super-resolution, the degradation model corresponds to the application of a gaussian low-pass filter followed by a $\times 4$ sub-sampling, and the addition of a gaussian white noise with $\sigma=3$.
For the deblurring, we considered motion blur and  gaussian kernels, both with a noise level $\sigma=8$. %

We provide quantitative comparisons in table~\ref{table:comp_ILO}, along with a visual comparison of the results in figure~\ref{fig:face_restoration}.
PnP-HVAE has the best  PSNR and SSIM results for all the considered restoration tasks, while ILO provides better results  for the perceptual distance.
By jointly optimizing the image and its latent variable, PnP-HVAE provides  results that are both realistic and consistent with the degraded observation.
On the other hand,  ILO  only optimizes on an extended latent space. This method generates  sharp and realistic images with better LPIPS scores,   
but the results lack  of consistency with respect to the observation, which explains the overall lower PSNR performance. 






\subsection{PatchVDVAE: a HVAE for natural images}\label{ssec:patchVDVAE}
Available generative models in the literature operate on images of  fixed resolutions and
are either restrained to datasets of limited diversity, or even to registered face images~\cite{kingma2018glow,child2021very, vahdat2020nvae, karras2019style}, or requiring additional class information~\cite{brock2018large, dhariwal2021diffusion, song2020score, luhman2022optimizing}.
Fitting an unconditional model on natural images appears to be a more difficult task, as their resolution can change, and their content is highly diverse.
The complexity of the problem can be reduced by learning a prior model on patches of reduced dimension. 
For image restoration problems, the patch model can be reused on images of higher dimensions~\cite{zoran2011learning,prost2021learning,altekruger2022patchnr}. When the model is a full CNN, the prior on the set of the  patches can  be computed efficiently by applying the network on the full image~\cite{prost2021learning}.

We thus introduce  patchVDVAE, a fully convolutional hierarchical VAE.
Contrary to existing HVAE models whose resolution is constrained by the constant tensor at the input of the top-down block, patchVDVAE can generate images of different resolutions by controlling the dimension of the input latent. 
This amounts to defining a prior on patches whose dimension corresponds to the receptive field of the VAE. A similar model is used for image denoising in~\cite{prakash2021interpretable}.

 
For PatchVDVAE architecture, we use the same bottom-up and top-down blocks as VDVAE~\cite{child2021very}, and replace the constant trainable input in the first top-down block by a latent variable, to make the model fully convolutional (details on the  architecture are given in Appendix~\ref{app:details}). 
The training dataset is composed of $128\times 128$ patches extracted from a combination of DIV2K~\cite{agustsson2017ntire} and Flickr2K~\cite{Lim_2017_CVPR_workshops} datasets.
We perform data augmentation by extracting  patches at $3$ resolutions: HR-images and $\times 2$ and $\times 4$ downscaled images. 
The model is trained for $7.10^5$ iterations with a batch size of $64$. Following the recommendation of~\cite{hazami2022efficient}, we use Adamax optimizer with an exponential moving average and gradient smoothing of the variance.
We set the decoder model to be a gaussian with diagonal covariance, as in~\cite{luhman2022optimizing}.
PatchVDVAE is fully convolutional and can generate images of dimension that are multiples of $64$ as illustrated by
figure~\ref{fig:vdvae}.

\newlength{\patchwidth}
\setlength{\patchwidth}{0.135\columnwidth}
\begin{figure}[!ht]
    \centering
    \begin{subfigure}[t]{.34\columnwidth}\hspace{0.1cm}
        \setlength{\tabcolsep}{0.02pt}
\renewcommand{\arraystretch}{0}
        \begin{tabular}{*{2}{p{1.03\patchwidth}}}
            \includegraphics[width=\patchwidth]{figures_arxiv/patchVDVAE/samples/generated/64x64/setup-5-image-0018.png} &
            \includegraphics[width=\patchwidth]{figures_arxiv/patchVDVAE/samples/generated/64x64/setup-5-image-0016.png} \\
            \includegraphics[width=\patchwidth]{figures_arxiv/patchVDVAE/samples/generated/64x64/setup-5-image-0008.png} &
            \includegraphics[width=\patchwidth]{figures_arxiv/patchVDVAE/samples/generated/64x64/setup-5-image-0019.png}   
        \end{tabular}
    \end{subfigure}\hspace{-0.15cm}
    \begin{subfigure}[t]{.64\columnwidth}
\begin{tabular}{cc}\vspace{-0.1cm}
\includegraphics[width=2\patchwidth]{figures_arxiv/patchVDVAE/samples/generated/256x256/setup-2-image-0009.png}&
        \includegraphics[width=2\patchwidth]{figures_arxiv/patchVDVAE/samples/generated/256x256/setup-2-image-0002.png}\end{tabular}

    \end{subfigure}
    \caption{\label{fig:vdvae} Left: $64\times64$ patches samples from our patchVDVAE model trained on patches from natural images.
    Right: PatchVDVAE is fully convolutional and it can generate images of higher resolution (here: $128\times128$).\vspace{-0.2cm}}
\end{figure}

\subsection{Natural images restoration}\label{ssec:app_nat}
We  evaluate PnP-HVAE on natural image restoration.
For each task, we report the average value of the PSNR, the SSIM, and the LPIPS metrics on $20$ images from the test set of the BSD dataset~\cite{MartinFTM01}.\\


\noindent
{\bf Image deblurring.}
In the experiments, we consider $2$ gaussian kernels and $2$ motion blur kernels from~\cite{levin2009understanding}, with $3$ different noise levels 
$\sigma \in \{2.55, 7.65, 12.75\}$.
As a baseline we consider  EPLL~\cite{zoran2011learning}, which learns a prior on image patches with a gaussian mixture model.
We also compare PnP-HVAE  with PnP-MMO and GS-PnP, $2$ competing convergent Plug-and-Play methods based on CNN denoisers.
PnP-MMO~\cite{pesquet2021learning} restricts the denoiser to be contraction in order to guarantee the convergence of the PnP forward-backard algorithm. GS-PnP~\cite{hurault2022gradient} considers a gradient step denoiser and reaches state-of-the-art performances of non converging methods~\cite{zhang2021plug}.
We set the temperature $\tau$  in our method as $0.95$, $0.8$ and $0.6$ for noise levels $2.55$, $7.65$ and $12.75$ respectively, and we let it run for a maximum of $50$ iterations. 
For the three compared methods we use the official implementations and pre-trained models provided by the respective authors. 
Details on the choice of hyperparameters for the concurrent methods are provided in the Appendix~\ref{app:details}
Figure~\ref{fig:deblurring_bsd} illustrates that our method provides correct deblurring results. 

According to table~\ref{tab:deb}, the performance of PnP-HVAE is between those of EPLL and GS-PnP and it outperforms PnP-MMO for large noise levels.\\

\begin{table}
\begin{center}\footnotesize
    \begin{tabular}{>{\centering}m{.3cm}*{5}{c}}
    $\sigma$ &Method & PSNR$\uparrow$ & SSIM$\uparrow$ & LPIPS$\downarrow$  \\ 
    \hline
    \multirow{4}{*}{\vcell{$2.55$}}
    & PnP-HVAE & $27.75$ & $0.79$ & $0.31$\\
    & GS-PNP \cite{hurault2022gradient} & $\mathbf{29.59}$ & $\mathbf{0.84}$ & $\mathbf{0.22}$\\
    & EPLL \cite{zoran2011learning} & $26.49$ & $0.71$ & $0.36$\\ 
    & PnP-MMO \cite{pesquet2021learning} & $\underbar{29.50}$ & $\underbar{0.83}$ & $\underbar{0.20}$ \\ \hline
    \multirow{4}{*}{\vcell{$7.65$}}
    & PnP-HVAE & $\underbar{26.36}$ & $\underbar{0.72}$ & $\underbar{0.40}$\\
    & GS-PNP \cite{hurault2022gradient} & $\mathbf{27.33}$ & $\mathbf{0.77}$ & $\mathbf{0.31}$\\
    & EPLL \cite{zoran2011learning} & $24.04$ & $0.66$ & $0.45$ \\ 
    & PnP-MMO \cite{pesquet2021learning} & $25.34$ & $0.69$ & $0.34$\\
    \hline
    \multirow{4}{*}{\vcell{$12.75$}}
    & PnP-HVAE & $\underbar{25.12}$ & $\mathbf{0.73}$ & $\underbar{0.47}$\\
    & GS-PNP \cite{hurault2022gradient} & $\mathbf{26.32}$ & $\mathbf{0.73}$ & $\mathbf{0.37}$\\
    & EPLL \cite{zoran2011learning} & $23.28$ & $0.61$ & $0.51$ \\ 
    & PnP-MMO \cite{pesquet2021learning} & $22.42$ & $0.53$& $0.54$ \\
    \hline
    &\vspace*{-.3cm}\\
            \multicolumn{2}{c}{Blur and motion kernels}& \multicolumn{3}{c}{
        \includegraphics*[scale=1]{figures_arxiv/kernels/4.png}\;\includegraphics*[scale=1]{figures_arxiv/kernels/7.png}\;\includegraphics*[scale=1]{figures_arxiv/kernels/9.png}\;\includegraphics*[scale=1]{figures_arxiv/kernels/11.png}} 
    \end{tabular}
        \caption{\label{tab:deb}Comparison  of PnP-HVAE  and other restoration methods on deblurring. Results are averaged on $4$ kernels.\vspace{-0.2cm}}% on image deblurring.}
    \end{center}
\end{table}

\begin{figure}
    
    \begin{subfigure}[h]{\linewidth}
        \centering
        \includegraphics*[width=\columnwidth]{figures_arxiv/deb_s255_k7.pdf}\vspace{-0.1cm}
        \caption{Gaussian blur, $\sigma=2.55$}
    \end{subfigure}
    \begin{subfigure}[h]{\linewidth}
        \centering
        \includegraphics*[width=\columnwidth]{figures_arxiv/deb_s765_k11.pdf}\vspace{-0.1cm}
        \caption{Motion blur, $\sigma=7.65$}
    \end{subfigure}\vspace*{-0.1cm}
    \caption{\label{fig:deblurring_bsd} Natural image deblurring\vspace{-0.1cm}}
\end{figure}

\noindent {\bf Effect of the temperature.}
PnP-HVAE gives control on the temperature of the prior over the latent space.
In figure~\ref{fig:temp_effect}, we illustrate that reducing the temperature increases the strength of the regularization prior. In this example the tuning $\tau=0.7$ produces the best performance.\\
\begin{figure}[!ht]
   
    \includegraphics[width=\columnwidth]{figures_arxiv/demo_temp.pdf}\vspace{-0.15cm}
    \caption{ \label{fig:temp_effect} Effect of the temperature in PnP-VAE on a deblurring problem, with $\sigma=7.65$.\vspace{-0.15cm}}
\end{figure}


\noindent
{\bf Image inpainting.}
Next we consider the task of noisy image inpainting. 
We compose a test-set of 10 images from the validation set of BSD~\cite{MartinFTM01} and we create masks
  by occluding diverse objects of small size in the images. 
A gaussian white noise with $\sigma=3$ is added to the images.
As a comparaison, we still consider GS-PnP and EPLL.
For PnP-HVAE, the temperature is set to $\tau=0.6$, and the algorithm is run for a maximum of $200$ iterations, unless the residual $||\x_{k+1}-\x_k||$ is on a plateau.
We provide on Table~\ref{tab:inpainting_bsd} the distortion metrics with the ground truth, as well as a visual
\begin{table}



\begin{center}
    \begin{tabular}{cccc}
        & PSNR$\uparrow$ & SSIM$\uparrow$ &LPIPS$\downarrow$ \\\hline
        PnP-HVAE  & $\mathbf{29.54}$ & $\mathbf{0.93}$ & $\mathbf{0.06}$\\
        GS-PNP & $28.52$ & $\mathbf{0.93}$ & $0.09$\\
        EPLL & $\underline{29.16}$ & $\mathbf{0.93}$ & $\mathbf{0.06}$\\
    \end{tabular}
    \caption{\label{tab:inpainting_bsd}Quantitative evaluation for inpainting on BSD.}
    \end{center}
\end{table}
comparison on figure~\ref{fig:inpainting_bsd}. 
With its hierarchical structure,  PnP-HVAE outperforms the compared methods. \vspace{0.05cm}



\begin{figure}[!h]
    \includegraphics[width=\columnwidth]{figures_arxiv/demo_inp_bsd2.pdf}\vspace{-0.1cm}
    \caption{\label{fig:inpainting_bsd}Natural image inpainting\vspace{-0.3cm}}
\end{figure}











\section{Conclusion}\label{sec:conclusion}
\section{Conclusion}\label{sec:conclusion}
In this work, we focus on addressing the fundamental challenge of OOD detection tasks, which is how to fully understand the semantic discrepancy between the ID/OOD samples. We reveal that the key to success in the realistic SCOOD task is to allocate as many ID samples in the unlabeled set correctly as possible. To this end, we propose a novel uncertainty-aware optimal transport scheme that introduces class-specific energy scores as guidance for effective label assignment. Experimental results show that our method achieves better performance than previous state-of-the-art methods on SCOOD benchmarks.

\textbf{Limitations.} In addition to temperature scaling, other techniques such as feature clipping applied in ReAct~\cite{sun2021react} also enhance the performance of energy score, so how to obtain an OOD score that best fits the SCOOD task can be further explored. Moreover, a setting highly related to SCOOD has been proposed in \cite{katz2022training} and formulated as a constrained optimization problem. We will also theoretically analyze these practical OOD settings in our feature work.

% \section*{Acknowledgments}
\textbf{Acknowledgments.} 
This work is supported by National Key R\&D Program of China under Grant 2020AAA0105701, National Natural Science Foundation of China (NSFC) under Grants 61872327, Major Special Science and Technology Project of Anhui, National Natural Science Foundation of China (62033012) and Ant Group through Ant Research Intern Program.


\subsubsection*{Reproducibility Statement}
We referenced in the main text the Appendix parts presenting the proofs or additional details of every claim, Assumption, Lemma, and Theorem occurring in the paper.
In addition, Appendix~\ref{appendix:experiments} is dedicated to the presentation of the setup, hyperparameters, and other extra details required for reproducing the results of Section~\ref{sec:experiments}.
We provide the source code of the implementation of our approach in Supplementary material \footnote{available at \url{https://github.com/florentdelgrange/wae_mdp}}, and we also provide the models saved during training that we used for model checking (i.e., reproducing the results of Table~\ref{table:evaluation}).
Additionally, we present in a notebook ({\small\texttt{evaluation.html}}) videos demonstrating how our distilled policies behave in each environment, and code snippets showing how we formally verified the policies. 

\subsubsection*{Acknowledgments}
This research received funding from the Flemish Government (AI Research Program) and was supported by the DESCARTES iBOF project. G.A. Perez is also supported by the Belgian FWO “SAILor” project (G030020N).
We thank Raphael Avalos for his valuable feedback during the preparation of this manuscript.

\bibliography{references}
\bibliographystyle{iclr2023_conference}

%%%%%%%%%%%%%%%%%%%%%%%%%%%%%%%%%%%%%%%%%%%%%%%%%%%%%%%%%%%%
\newpage
\appendix
\section*{Appendix}
\section{Appendix for Proofs}

\paragraph{Proof of Theorem \ref{thm:main}.}

\begin{proof}
\label{proof:main}
Our proof has two steps. In Step 1, we will show that SimCLR is equivalent to minimizing the cross entropy loss defined in Eqn.~(\ref{eqn:cross-entropy}). 
In Step 2, we will show  that minimizing the cross-entropy loss 
is equivalent to spectral clustering on $\bfpi$. 
Combining the two steps together, we have proved our theorem. 

\textbf{Step 1: } SimCLR is equivalent to minimizing the cross entropy loss.

The cross-entropy loss takes expectation over 
$\bfW_\bfX\sim \mathbb{P}(\cdot ; \bfpi)$, 
which means $\bfW_\bfX$ has exactly one non-zero entry in each row $i$. By Lemma~\ref{lem:multinomial}, we know every row $i$ of $\bfW_\bfX$ is independent of other rows. Moreover, 
$\bfW_{\bfX,i}\sim \mathcal{M}(1, \bfpi_i/\sum_j \bfpi_{i,j})=\mathcal{M}(1, \bfpi_i)$, because $\bfpi_i$ itself is a probability distribution.
Similarly, we know $\bfW_\bfZ$ also has the row-independent property by sampling over $\mathbb{P}(\cdot;\bfK_\bfZ)$.
Therefore, by Lemma~\ref{lem:cross_split}, we know Eqn.~(\ref{eqn:cross-entropy}) is equivalent to:
\[
 -\sum_{i=1}^n \mathbb{E}_{\bfW_{\bfX,i}}[\log \mathbb{P}(\bfW_{\bfZ,i}=\bfW_{\bfX,i};\bfK_\bfZ)],
\]

This expression takes expectation over $\bfW_{\bfX,i}$ for the given row $i$. Notice that 
$\bfW_{\bfX,i}$ has exactly one non-zero entry, which equals $1$ (same for $\bfW_{\bfZ,i}$). 
As a result
we expand the above expression to be:
\begin{equation}
 -\sum_{i=1}^n \sum_{j\neq i} \Pr(\bfW_{\bfX,i,j}=1)\log \Pr(\bfW_{\bfZ,i,j}=1).
\label{eqn:detailed-expansion}    
\end{equation}


By Lemma~\ref{lem:multinomial}, $\Pr(\bfW_{\bfZ,i,j}=1)=\bfK_{\bfZ,i,j}/\|\bfK_{\bfZ,i}\|_1$ for $j\neq i$. Recall that $\bfK_\bfZ=(k(\bfZ_i-\bfZ_j))_{(i,j)\in[n]^2}$, which means 
$\bfK_{\bfZ,i,j}/\|\bfK_{\bfZ,i}\|_1=\frac{\exp(-\|\bfZ_i-\bfZ_j\|^2/{2\tau})}{\sum_{k\neq i}
\exp(-\|\bfZ_i-\bfZ_k\|^2/{2\tau})
}$ for $j\neq i$, when $k$ is the Gaussian kernel with variance $\tau$. 

Notice that $\bfZ_i=f(\bfX_i)$, so we know
\begin{equation}
-\log \Pr(\bfW_{\bfZ,i,j}=1)=
-\log \frac{\exp(-\|f(\bfX_i)-f(\bfX_j)\|^2/{2\tau})}{\sum_{k\neq i}
\exp(-\|f(\bfX_i)-f(\bfX_k)\|^2/{2\tau}),
}
\label{eqn:infonce-equivalence}    
\end{equation}


The right hand side is exactly the InfoNCE loss defined in Eqn.~(\ref{eqn:infonce}).
Inserting Eqn.~(\ref{eqn:infonce-equivalence}) into Eqn.~(\ref{eqn:detailed-expansion}), we get the SimCLR algorithm, which first samples augmentation pairs $(i,j)$ with $\Pr(\bfW_{\bfX,i,j}=1)$ for each row $i$, and then optimize the InfoNCE loss. 

\textbf{Step 2: } minimizing the cross entropy loss 
is equivalent to spectral clustering on $\bfpi$.


By Lemma~\ref{lem:convert_to_spectral}, we may further convert the loss to 
\begin{equation}
\label{eqn:main-theorem-repul-attr}
\min_{\bfZ}
-\sum_{(i,j)\in [n]^2} \mathbf{P}_{i,j}
\log k (\bfZ_i-\bfZ_j)+\log \mathbf{R}(\bfZ).
\end{equation}
Since $k$ is the Gaussian kernel, this reduces to \[
\min_\bfZ \mathrm{tr}(\bfZ^\top \mathbf{L}(\bfpi) \bfZ)
+\log \mathbf{R}(\bfZ),
\]

where we use the fact that $\mathbb{E}_{\bfW_\bfX\sim \mathbb{P}(\cdot; \bfpi)}[\mathbf{L}(\bfW_\bfX)]
=\mathbf{L}(\bfpi)
$, because the Laplacian operator is linear and $
\mathbb{E}_{\bfW_\bfX\sim \mathbb{P}(\cdot; \bfpi)}(\bfW_\bfX)=\bfpi
$.
\end{proof}

\paragraph{Proof of Theorem \ref{thm:clip}.}
\begin{proof}
Since $\bfW_\bfX\sim \mathbb{P}(\cdot;\bfpi_{\mathbf{A}, \mathbf{B}})$, we know 
$\bfW_\bfX$ has exactly one non-zero entry in each row, denoting the pair that got sampled. 
A notable difference compared to the previous proof is we now have $n_\mathcal{A}+n_\mathcal{B}$ objects in our graph. CLIP deals with this by taking a mini-batch of size $2N$, 
such that $n_\mathcal{A}=n_\mathcal{B}=N$, and adding the $2N$ InfoNCE losses together. We label the objects in $\mathcal{A}$ as $[n_\mathcal{A}]$, and the objects in $\mathcal{B}$ as $\{n_\mathcal{A}+1, \cdots, n_\mathcal{A}+n_\mathcal{B}\}$. 

Notice that $\bfpi_{\mathbf{A}, \mathbf{B}}$ is a bipartite graph, so the edges of objects in $\mathcal{A}$ will only connect to object in $\mathcal{B}$ and vice versa. We can define the similarity matrix in $\cZ$ as $\bfK_\bfZ$, 
where $\bfK_\bfZ(i, j+n_\mathcal{A})=\bfK_\bfZ(j+n_\mathcal{A},i)= k(\bfZ_i-\bfZ_j)$ for $i\in [n_\mathcal{A}], j\in [n_\mathcal{B}]$, and otherwise we set $\bfK_\bfZ(i,j)=0$. 
The rest is same as the previous proof. 
\end{proof}

\paragraph{Proof of Theorem \ref{thm:exponential}.}

\begin{proof}
\label{proof:exponential}
Since the objective function consists of a linear term combined with an entropy regularization, which is a strongly concave function, the maximization problem is a convex optimization problem. Owing to the implicit constraints provided by the entropy function, the problem is equivalent to having only the equality constraint. We then introduce the Lagrangian multiplier $\lambda$ and obtain the following relaxed problem:

$$
\widetilde{E}(\boldsymbol{\alpha})=\psi_{1}-\sum_{i=1}^n \alpha_{i} \psi_{i}+\tau \sum_{i=1}^n \alpha_{i}\log \alpha_{i}+\lambda\left(\boldsymbol{\alpha}^{\top} \mathbf{1}_n-1\right).
$$

As the relaxed problem is unconstrained, taking the derivative with respect to $\alpha_{i}$ yields

$$
\frac{\partial \widetilde{E}(\boldsymbol{\alpha})}{\partial \alpha_{i}}=-\psi_{i}+\tau\left(\log \alpha_{i}+\alpha_{i} \frac{1}{\alpha_{i}}\right)+\lambda=0.
$$

Solving the above equation implies that $\alpha_{i}$ takes the form
$
\alpha_{i}=\exp \left(\frac{1}{\tau} \psi_{i}\right) \exp \left(\frac{-\lambda}{\tau}-1\right).
$ Since $\alpha_{i}$ lies on the probability simplex, the optimal $\alpha_{i}$ is explicitly given by
$
\alpha^{*}_{i}=\frac{\exp \left(\frac{1}{\tau} \psi_{i}\right)}{\sum_{i^{\prime}=1}^n \exp \left(\frac{1}{\tau} \psi_{i^{\prime}}\right)} .
$ Substituting the optimal point into the objective function, we obtain
$$
\begin{aligned}
E\left(\boldsymbol{\alpha}^*\right)  &=\psi_1-\sum_{i=1}^n \frac{\exp \left(\frac{1}{\tau} \psi_{i}\right)}{\sum_{i^{\prime}=1}^n \exp \left(\frac{1}{\tau} \psi_{i^{\prime}}\right)} \psi_{i}+\tau \sum_{i=1}^n \frac{\exp \left(\frac{1}{\tau} \psi_{i}\right)}{\sum_{i^{\prime}=1}^n \exp \left(\frac{1}{\tau} \psi_{i^{\prime}}\right)}\log \frac{\exp \left(\frac{1}{\tau} \psi_{i}\right)}{\sum_{i^{\prime}=1}^n \exp \left(\frac{1}{\tau} \psi_{i^{\prime}}\right)} \\
& =\psi_1 - \tau \log \left(\sum_{i=1}^n \exp \left(\frac{1}{\tau} \psi_{i}\right)\right).
\end{aligned}
$$
Thus, the Lagrangian dual function is given by
\begin{equation*}
-E\left(\boldsymbol{\alpha}^*\right)= -\tau \log \frac{\exp \left(\frac{1}{\tau} \psi_{1}\right)}{\sum_{i=1}^n \exp \left(\frac{1}{\tau} \psi_{i}\right)}.\qedhere
\end{equation*}
\end{proof}



\section{More on Experiments} \label{section: experiment_details}

\paragraph{CIFAR-10 and CIFAR-100} CIFAR-10 ~\citep{krizhevsky2009learning} and CIFAR-100 ~\citep{krizhevsky2009learning} are well-known classic image classification datasets. Both CIFAR-10 and CIFAR-100 contain a total of 60k $32 \times 32$ labeled images of different classes, with 50k for training and 10k for testing. CIFAR-10 is similar to CIFAR-100, except there are 10 different classes in CIFAR-10 and 100 classes in CIFAR-100.

\paragraph{TinyImageNet} TinyImageNet ~\citep{le2015tiny} is a subset of ImageNet ~\citep{deng2009imagenet}. There are 200 different object classes in TinyImageNet, with 500 training images, 50 validation images, and 50 test images for each class. All the images in TinyImageNet are colored and labeled with a size of $64 \times 64$.

\textbf{Pseudo-code.} Algorithm \ref{alg:Training Procedure} presents the pseudo-code for our empirical training procedure.

\begin{algorithm}[!htbp]
\caption{Training Procedure}
\label{alg:Training Procedure}
\begin{algorithmic}[1]
\REQUIRE trainable encoder network $f$, batch size $N$, augmentation strategy \textit{aug}, loss function $L$ with hyperparameters \textit{args}
\FOR {sampled minibatch ${x_i}_{i=1}^N$}
\FORALL{$i \in { 1, ..., N }$}
\STATE draw two augmentations $t_i = \textit{aug}\left(x_i\right) $, $t_i' = \textit{aug}\left(x_i\right) $
\STATE $z_i = f\left(t_i\right)$, $z_i' = f\left(t_i'\right)$
\ENDFOR
\STATE compute loss $\mathcal{L} = L(N, z, z', \textit{args})$
\STATE update encoder network $f$ to minimize $\mathcal{L}$
\ENDFOR
\STATE \textbf{Return} encoder network $f$
\end{algorithmic}
\end{algorithm}

We also provide the pseudo-code for our core loss function used in the training procedure in Algorithm \ref{alg:Core loss}. The pseudo-code is almost identical to SimCLR's loss function, with the exception of an extra parameter $\gamma$.

\begin{algorithm}[!htbp]
\caption{Core loss function $\mathcal{C}$}
\label{alg:Core loss}
\begin{algorithmic}[1]
\REQUIRE batch size $N$, two encoded minibatches $z_1, z_2$, $\gamma$, temperature $\tau$
\STATE $z = \textit{concat}\left(z_1, z_2\right)$
\FOR {$i \in {1, ..., 2N }, j \in {1, ..., 2N}$ }
\STATE $s_{i,j} = \Vert z_i - z_j \Vert_2^{\gamma}$
\ENDFOR
\STATE \textbf{define} $l(i, j)$ \textbf{as} $l(i, j) = - \log \frac{exp\left(s_{i,j}/\tau \right)}{\sum_{k=1}^{2N} \mathbf{1}{[k \ne i]} exp\left(s{i, j} / \tau \right)} $
\STATE \textbf{Return} $\frac{1}{2N} \sum_{k=1}^N\left[l(i, i+N) + l(i+N, i)\right]$
\end{algorithmic}
\end{algorithm}

Utilizing the core loss function $\mathcal{C}$, we can define all kernel loss functions used in our experiments in Table \ref{table: loss definition}. For all $z_i \in z$ with even dimensions $n$, we define $z_{L_i} = z_i\left[0:n/2\right]$ and $z_{R_i} = z_i\left[n/2:n\right]$.

\begin{table}[ht]
\centering
\begin{tabular}{{@{}l|l@{}}}
Kernel  &  Loss function \\ \midrule
Laplacian & $\mathcal{C}\left(N, z, z', \gamma=1, \tau\right)$\\ \midrule
Sum       & $\lambda * \mathcal{C}\left(N, z, z', \gamma=1, \tau_1\right) + (1-\lambda) * \mathcal{C}\left(N, z, z', \gamma=2, \tau_2\right)$  \\ \midrule
Concatenation Sum&$\lambda * \mathcal{C}\left(N, z_L, z'_L, \gamma=1, \tau_1\right) + (1-\lambda) * \mathcal{C}\left(N, z_R, z'_R, \gamma=2, \tau_2\right)$\\ \midrule
$\gamma = 0.5$ & $\mathcal{C}\left(N, z, z', \gamma=0.5, \tau\right)$          \\ 

\end{tabular}

\caption{Definition of kernel loss functions in our experiments}
\label {table: loss definition}
\end{table}

\textbf{Baselines.} We reproduce the SimCLR algorithm using PyTorch Lightning~\citep{PytorchLightning}.

\textbf{Encoder details.}
The encoder $f$ consists of a backbone network and a projection network. We employ ResNet50~\citep{ResNet} as the backbone and a 2-layer MLP (connected by a batch normalization~\citep{ioffe2015batch} layer and a ReLU \cite{nair2010rectified} layer) with hidden dimensions 2048 and output dimensions 128 (or 256 in the concatenation kernel case).

\textbf{Encoder hyperparameter tuning.}
For each encoder training case, we randomly sample 500 hyperparameter groups (sample details are shown in Table \ref{table: Hyperparameter sample}) and train these samples simultaneously using Ray Tune ~\citep{RayTune}, with the ASHA scheduler~\citep{li2018massively}. Ultimately, the hyperparameter group that maximizes the online validation accuracy (integrated in PyTorch Lightning) within 5000 validation steps is chosen for the given encoder training case.

\begin{table}[ht]
\centering

\begin{tabular}{@{}l|l|l@{}}
\midrule
Hyperparameter  & Sample Range & Sample Strategy \\ \midrule
start learning rate & $\left[10^{-2}, 10\right]$ & log uniform \\ \midrule
$\lambda$       & $\left[0, 1\right]$ & uniform \\ \midrule
$\tau$, $\tau_1$, $\tau_2$ & $\left[0, 1\right]$ & log uniform \\ \midrule
\end{tabular}

\caption{Hyperparameters sample strategy}
\label {table: Hyperparameter sample}
\end{table}

\textbf{Encoder training.} 
We train each encoder using the LARS optimizer~\citep{LARSOptimizer}, LambdaLR Scheduler in PyTorch, momentum 0.9, weight decay $10^{-6}$, batch size 256, and the aforementioned hyperparameters for 400 epochs on a single A-100 GPU.

\textbf{Image transformation.} The image transformation strategy, including augmentation, is identical to the default transformation strategy provided by PyTorch Lightning.

\textbf{Linear evaluation.}
The linear head is trained using the SGD optimizer with a cosine learning rate scheduler, batch size 64, and weight decay $10^{-6}$ for 100 epochs. The learning rate starts at $0.3$ and ends at $0$.

\textbf{Moco Experiments.} We also tested our method based on MoCo~\citep{he2019moco}. The results are summarized in Table \ref{tab:results-moco}. Here we choose ResNet18~\citep{ResNet} as the backbone and set a temperature of $0.1$ as default. For our simple sum kernel, we set $\lambda=0.8$. The results show that our method outperforms the original MoCo method.

\begin{table}[thb]
\centering
\caption{MoCo Experiment Results on CIFAR-10 and CIFAR-100.}
\label{tab:results-moco}
\resizebox{\textwidth}{!}{%
\begin{tabular}{@{}c|ccc|ccc@{}}
\toprule
\multirow{3}{*}{Method} & \multicolumn{3}{c|}{CIFAR-10} & \multicolumn{3}{c}{CIFAR-100} \\ \cmidrule(lr){2-4} \cmidrule(lr){5-7} 
                        & 200 epochs & 400 epochs    & 1000 epochs   & 200 epochs & 400 epochs & 1000 epochs         \\ \midrule
MoCo (repro.)         & $76.41 \pm 0.12$    & $80.01 \pm 0.15$          & $84.45 \pm 0.08$    & $\mathbf{47.02 \pm 0.11}$ & $52.50 \pm 0.07$ & $57.62 \pm 0.15$            \\
\midrule
Laplacian Kernel        & ${78.09 \pm 0.10}$    & $\mathbf{83.85 \pm 0.09}$          & $\mathbf{88.34 \pm 0.16}$    & $46.12 \pm 0.22$   & $53.44 \pm 0.17$ & $59.10 \pm 0.14$        \\
Simple Sum Kernel & $\mathbf{78.12 \pm 0.15}$   & $83.23 \pm 0.18$ & $87.50 \pm 0.20$ & $46.65 \pm 0.06$ & $\mathbf{53.62 \pm 0.19}$ & $\mathbf{59.83 \pm 0.12}$\\
\bottomrule
\end{tabular}
}
\end{table}



\section{More Experiments on Synthetic Data}


Consider a scenario with $n$ clusters, each containing $k$ vertices. Let the probability of vertices $u$ and $v$ from the same cluster belonging to $\bfpi$ be $p$. Conversely, for vertices $u$ and $v$ from different clusters, let the probability of belonging to $\pi$ be $q$. We generate the graph $\bfpi$ randomly, based on $p$ and $q$. We experiment with values of $k=100$ and $n=6$ for ease of visualization, embedding all points in a two-dimensional space. Each vertex's initial position originates from a normal distribution. In each iteration, we sample a subgraph of $\bfpi$ uniformly, ensuring each vertex has an out-degree of $1$. We then optimize the corresponding vectors using InfoNCE loss with an SGD optimizer and iterate until convergence. Our experimental setup consists of an SGD learning rate of $1$, an InfoNCE loss temperature of $0.5$, and a batch size of $50$. We evaluate two scenarios with different $p$ and $q$ values: $p=1$, $q=0$, and $p=0.75$, $q=0.2$. The results of these experiments are visualized in Figure \ref{fig:vis-spectral-cluster}. The obtained embeddings exhibit the hallmark pattern of spectral clustering of graph $\bfpi$.

\begin{figure}[!tb]
\centering
\subfigure{
\includegraphics[width=1\textwidth]{Figures/cluster_pi.png}
\label{fig:vis-cluster}
}
\subfigure{
\includegraphics[width=1\textwidth]{Figures/noised_cluster_pi.png}
\label{fig:vis-noised-cluster}
}
\caption{Visualizations of the optimization process using InfoNCE Loss on the vectors corresponding to $\bfpi$. Points of identical color belong to the same cluster within $\bfpi$. To showcase the internal structure of $\bfpi$, we randomly select 10 vertices from each cluster to display the edge distribution of $\bfpi$.}
\label{fig:vis-spectral-cluster}
\end{figure}



\end{document}
