\providecommand\given{}
\newcommand\SetSymbol[1][]{%
  \nonscript\:#1\vert
  \allowbreak
  \nonscript\:
  \mathopen{}}
\DeclarePairedDelimiterX\Set[1]{\{}{\}}{%
  \renewcommand\given{\SetSymbol[\delimsize]}
  #1}

\DeclarePairedDelimiterXPP\pospart[1]{}{(}{)}{^+}{#1}
\DeclarePairedDelimiterXPP\negpart[1]{}{(}{)}{^-}{#1}
\DeclarePairedDelimiter\ceil{\lceil}{\rceil}
\DeclarePairedDelimiter\floor{\lfloor}{\rfloor}
\DeclarePairedDelimiter\interv{[}{]}
\DeclarePairedDelimiter\openinterv{(}{)}
\DeclarePairedDelimiter\absval{\lvert}{\rvert}

\newcommand\R{\mathbb{R}}
\newcommand\Z{\mathbb{Z}}
\newcommand\N{\mathbb{N}}
\newcommand\D{\mathcal{D}}

\newcommand\mapping[3]{{#1}\colon{#2}\to{#3}}
\newcommand\transp[1]{{#1}^{\scriptscriptstyle \mathsf{T}}}

\newtheorem{prop}{Proposition}
\newtheorem*{prop0}{Proposition}
\newtheorem{cor}{Corollary}
\newtheorem*{thm0}{Theorem}
\newtheorem{thm}{Theorem}
\newtheorem{defn}{Definition}

\DeclareMathOperator*{\argmin}{arg\,min}
\DeclareMathOperator*{\vertices}{vert}
\DeclareMathOperator*{\Int}{int}
\DeclareMathOperator*{\sgn}{sgn}
\DeclareMathOperator*{\rank}{rank}
\DeclareMathOperator*{\conv}{co}

\makeatletter
\newcommand{\leqnomode}{\tagsleft@true\let\veqno\@@leqno}
\newcommand{\reqnomode}{\tagsleft@false\let\veqno\@@eqno}
\makeatother
