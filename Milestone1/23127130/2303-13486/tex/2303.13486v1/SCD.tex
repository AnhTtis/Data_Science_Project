% CVPR 2023 Paper Template
% based on the CVPR template provided by Ming-Ming Cheng (https://github.com/MCG-NKU/CVPR_Template)
% modified and extended by Stefan Roth (stefan.roth@NOSPAMtu-darmstadt.de)

\documentclass[10pt,twocolumn,letterpaper]{article}

%%%%%%%%% PAPER TYPE  - PLEASE UPDATE FOR FINAL VERSION
%\usepackage[review]{cvpr}      % To produce the REVIEW version
%\usepackage{cvpr}              % To produce the CAMERA-READY version
\usepackage[pagenumbers]{cvpr} % To force page numbers, e.g. for an arXiv version

% Include other packages here, before hyperref.
\usepackage{graphicx}
\usepackage{amsmath,amsthm}
\usepackage{amssymb}
\usepackage{booktabs}
\usepackage{xcolor,soul,verbatim}
\newcommand{\cbox}[2]{\colorbox{#1}{$\displaystyle #2$}}

% It is strongly recommended to use hyperref, especially for the review version.
% hyperref with option pagebackref eases the reviewers' job.
% Please disable hyperref *only* if you encounter grave issues, e.g. with the
% file validation for the camera-ready version.
%
% If you comment hyperref and then uncomment it, you should delete
% ReviewTempalte.aux before re-running LaTeX.
% (Or just hit 'q' on the first LaTeX run, let it finish, and you
%  should be clear).
\usepackage[pagebackref,breaklinks,colorlinks]{hyperref}

\newtheorem{thm}{Theorem}[section]
\newtheorem{dfn}[thm]{Definition}
\newtheorem{pro}[thm]{Problem}
\newtheorem{prop}[thm]{Proposition}
\newtheorem{cor}[thm]{Corollary}
\newtheorem{lem}[thm]{Lemma}
\newtheorem{exa}[thm]{Example}
\newtheorem{claim}[thm]{Claim}

\newcommand{\R}{\mathbb R}
\newcommand{\Z}{\mathbb Z}
\newcommand{\al}{\alpha}
\newcommand{\be}{\beta}
\newcommand{\ga}{\gamma}
\newcommand{\Ga}{\Gamma}
\newcommand{\de}{\delta}
\newcommand{\De}{\Delta}
\newcommand{\la}{\lambda}
\newcommand{\La}{\Lambda}
\newcommand{\si}{\sigma}
\newcommand{\ti}{\tilde}
\newcommand{\ep}{\varepsilon}
\newcommand{\ph}{\varphi}
\newcommand{\angstrom}{\textup{\AA}}
\newcommand{\den}{\mathrm{Den}}
\newcommand{\iso}{\mathrm{Iso}}
\newcommand{\SO}{\mathrm{SO}}
\newcommand{\Or}{\mathrm{O}}
\newcommand{\HKS}{\mathrm{HKS}}
\newcommand{\RMS}{\mathrm{RMS}}
\newcommand{\ADD}{\mathrm{ADD}}
\newcommand{\ASD}{\mathrm{ASD}}
\newcommand{\SDM}{\mathrm{SDM}}
\newcommand{\CDM}{\mathrm{CDM}}
\newcommand{\SDV}{\mathrm{SDV}}
\newcommand{\CID}{\mathrm{CID}}
\newcommand{\PDF}{\mathrm{PDF}}
\newcommand{\ePDF}{\mathrm{ePDF}}
\newcommand{\LAC}{\mathrm{LAC}}
\newcommand{\PDD}{\mathrm{PDD}}
\newcommand{\SDC}{\mathrm{SDC}}
\newcommand{\SDD}{\mathrm{SDD}}
\newcommand{\wSDD}{\widetilde{\SDD}}
\newcommand{\RDD}{\mathrm{RDD}}
\newcommand{\wRDD}{\widetilde{\RDD}}
\newcommand{\ORD}{\mathrm{ORD}}
\newcommand{\mORD}{\overline{\ORD}}
\newcommand{\OSD}{\mathrm{OSD}}
\newcommand{\mOSD}{\overline{\OSD}}
\newcommand{\OCD}{\mathrm{OCD}}
\newcommand{\mOCD}{\overline{\OCD}}
\newcommand{\SCD}{\mathrm{SCD}}
\newcommand{\mSCD}{\overline{\SCD}}
\newcommand{\AOV}{\mathrm{AOV}}
\newcommand{\AOD}{\mathrm{AOD}}
\newcommand{\ODM}{\mathrm{ODM}}
\newcommand{\PDDt}{\mathrm{PDD}^{\{2\}}}
\newcommand{\PDDh}{\mathrm{PDD}^{\{h\}}}
\newcommand{\ACD}{\mathrm{ACD}}
\newcommand{\ACV}{\mathrm{ACV}}
\newcommand{\AMD}{\mathrm{AMD}}
\newcommand{\PPC}{\mathrm{PPC}}
\newcommand{\BND}{\mathrm{BND}}
\newcommand{\EMD}{\mathrm{EMD}}
\newcommand{\vol}{\mathrm{Vol}}
\newcommand{\sign}{\mathrm{sign}}
\newcommand{\diam}{\mathrm{diam}}
\newcommand{\dif}{\mathrm{Dif}}
\newcommand{\dist}{\mathrm{Dist}}
\newcommand{\sym}{\mathrm{Sym}}
\newcommand{\aff}{\mathrm{aff}}
\newcommand{\dspan}{\mathrm{dspan}}
\newcommand{\bs}{\hfill $\blacksquare$}
\newcommand{\bd}{\partial}
\newcommand{\lra}{\leftrightarrow}
\newcommand{\Lra}{\Leftrightarrow}
\newcommand{\es}{\emptyset}
\newcommand{\vl}{\,:\,}
\newcommand{\myskip}{\medskip}
\newcommand{\pd}[2]{\frac{\partial #1}{\partial #2}}
\newcommand{\bpd}[2]{\dfrac{\partial #1}{\partial #2}}
\newcommand{\vect}[2]{ \left( \begin{array}{c} 
 #1 \\ #2 \end{array} \right)}
\newcommand{\colthree}[3]{ \left( \begin{array}{c} 
 #1 \\ #2 \\ #3 \end{array} \right)}
\newcommand{\mat}[4]{ \left( \begin{array}{cc} 
 #1 & #2 \\ #3 & #4 \end{array} \right)}
\newcommand{\matv}[4]{ \left( \begin{array}{cc} 
 #1 & #3 \\ #2 & #4 \end{array} \right)}
\newcommand{\matr}[6]{ \left( \begin{array}{cc} 
 #1 & #2 \\ #3 & #4 \\ #5 & #6 \end{array} \right)}
\newcommand{\mathree}[6]{ \left( \begin{array}{cc} 
 #1 & #4 \\ #2 & #5 \\ #3 & #6 \end{array} \right)}

% Support for easy cross-referencing
\usepackage[capitalize]{cleveref}
\crefname{section}{Sec.}{Secs.}
\Crefname{section}{Section}{Sections}
\Crefname{table}{Table}{Tables}
\crefname{table}{Tab.}{Tabs.}


%%%%%%%%% PAPER ID  - PLEASE UPDATE
\def\cvprPaperID{11825} % *** Enter the CVPR Paper ID here
\def\confName{CVPR}
\def\confYear{2023}

\begin{document}

%%%%%%%%% TITLE - PLEASE UPDATE
\title{The strength of a simplex is the key to a continuous isometry classification of Euclidean clouds of unlabelled points}

\author{Vitaliy Kurlin\\
Computer Science department\\
University of Liverpool, UK\\
{\tt\small vitaliy.kurlin@gmail.com}
}
\maketitle

%%%%%%%%% ABSTRACT
\begin{abstract}
This paper solves the continuous classification problem for finite clouds of unlabelled points under Euclidean isometry.
The Lipschitz continuity of required invariants in a suitable metric under perturbations of points is motivated by the inevitable noise in measurements of real objects.  
\smallskip

The best solved case of this isometry classification is known as the SSS theorem in school geometry saying that any triangle up to congruence (isometry in the plane) has a continuous complete invariant of three side lengths.
\smallskip

However, there is no easy extension of the SSS theorem even to four points in the plane partially due to a 4-parameter family of 4-point clouds that have the same six pairwise distances.  
The computational time of most past metrics that are invariant under isometry was exponential in the size of the input.
The final obstacle was the discontinuity of previous invariants at singular configurations, for example, when a triangle degenerates to a straight line. 
\smallskip

All the challenges above are now resolved by the Simplexwise Centred Distributions that combine inter-point distances of a given cloud with the new strength of a simplex that finally guarantees the Lipschitz continuity.
The computational times of new invariants and metrics are polynomial in the number of points for a fixed Euclidean dimension.
\end{abstract}

%1==========================
\section{Isometry classification problem for clouds}
\label{sec:intro}

The rigidity of many real objects motivates the most practical equivalence of \emph{rigid motion}, which is a composition of translations and high-dimensional rotations in $\R^n$.
We also study \emph{isometry}, which maintains inter-point distances and allows reflections.
Any orientation-preserving isometry can be realised as a continuous rigid motion.
\smallskip

Often there is no sense to distinguish rigid objects that are related by rigid motion or isometry.
Hence we can define the \emph{rigid shape} of a cloud $C$ as its \emph{isometry class} consisting of all infinitely many clouds isometric to $C$. 
\smallskip

The only reliable tool for distinguishing clouds up to isometry is an \emph{invariant} defined as a function or property preserved by any isometry.
Since any isometry is bijective, the number of points is an isometry invariant, but the coordinates of points are not invariants even under translation. 
This simple invariant is \emph{incomplete} (non-injective) because non-isometric clouds can have different numbers of points.
\smallskip

In Computer Science, an invariant $I$ is a descriptor with \emph{no false negatives} that are pairs (of different representations) 
$C\cong C'$ of equivalent objects with $I(C)\neq I(C')$.
Isometry invariants are also called \emph{equivariant descriptors}. %\cite{schmidt2012learning}.
\smallskip

A \emph{complete} invariant $I$ should distinguish all non-isometric clouds, so if $C\not\cong C'$ then $I(C)\neq I(C')$.
Equivalently, $I$ has \emph{no false positives} that are pairs $C\not\cong C'$ of non-equivalent objects with $I(C)=I(C')$.
A complete invariant $I$ can be considered a DNA-style code or a genome that uniquely identifies any object up to given equivalence. 
\smallskip

Since real data are always noisy and motions of rigid objects are important to track, a useful complete invariant must be also continuous under the movement of points. 
\smallskip

%Satisfying both completeness and continuity is extremely challenging for clouds of $m$ \emph{unlabelled} points because of $m!$ potential permutations that match all $m$ given points. \smallskip

A complete and continuous invariant for $m=3$ consists of three pairwise distances (sides of a triangle) and is studied at school as the SSS theorem \cite{weisstein2003triangle}.
But pairwise distances are incomplete for $m\geq 4$ even in $\R^2$\cite{boutin2004reconstructing}, see Fig.~\ref{fig:4-point_clouds}.

\begin{pro}[complete isometry invariants with computable continuous metrics]
\label{pro:isometry}
For any cloud of $m$ unlabelled points in $\R^n$,
find an invariant $I$ satisfying the properties
\medskip

\noindent
\textbf{(a)}
\emph{completeness} : 
$C,C'$ are isometric $\Lra$ $I(C)=I(C')$; 
\medskip

\noindent
\textbf{(b)}
\emph{Lipschitz continuity} :
if any point of $C$ is perturbed within its $\ep$-neighbourhood then
 $I(C)$ changes by at most $\la\ep$ for a constant $\la$ and a metric $d$ %on invariant values 
 satisfying these axioms:
\smallskip

\noindent 
1) $d(I(C),I(C'))=0$ if and only if $C\cong C'$ are isometric,
\smallskip

\noindent 
2) \emph{symmetry} : $d(I(C),I(C'))=d(I(C'),I(C))$,
\smallskip

\noindent 
%\emph{triangle inequality} : 
3) $d(I(C),I(C'))+d(I(C'),I(C''))\geq d(I(C),I(C''))$;
\medskip

\noindent
\textbf{(c)}
\emph{computability} : $I$ and $d$ are computable in a polynomial time in the number $m$ of points for a fixed dimension $n$.
\bs
\end{pro}

\begin{figure}[h!]
\centering
\includegraphics[width=\linewidth]{images/4-point_clouds_h_solid.png}
\caption{
\textbf{Left}: the cloud $T=\{(1,1),(-1,1),(-2,0),(2,0)\}$. 
\textbf{Right}: the kite $K=\{(0,1),(-1,0),(0,-1),(3,0)\}$. 
$T$ and $K$ have the same 6 pairwise distances $\sqrt{2},\sqrt{2},2,\sqrt{10},\sqrt{10},4$.
}
\label{fig:4-point_clouds}
\end{figure}

Condition~(\ref{pro:isometry}b) asking for a continuous metric is stronger than the completeness in (\ref{pro:isometry}a).
Detecting an isometry $C\cong C'$ gives a discontinuous metric, say $d=1$ for all non-isometric clouds $C\not\cong C'$ even if $C,C'$ are nearly identical.
Any metric $d$ satisfying the first axiom in (\ref{pro:isometry}b) detects an isometry $C\cong C'$ by checking if $d=0$.
\smallskip

Problem~\ref{pro:isometry} for any clouds of $m$ unlabeled points in $\R^n$ will be solved by Theorems~\ref{thm:SCD_complete},~\ref{thm:SCD_metrics} and Corollary~\ref{cor:continuity}, which essentially needs the new strength of a simplex whose continuity is proved in hardest Theorem~\ref{thm:strength}.
\smallskip

This paper extends \cite[section 4]{widdowson2023recognizing} whose 8-page version without proofs and big examples will appear soon.
In the papers \cite{widdowson2021pointwise,widdowson2022resolving,widdowson2023recognizing}, the first author implemented all algorithms, the second author wrote all theory, proofs, and examples. 

%2==========================
\section{Related work on point cloud classifications}
\label{sec:past}

This section reviews the past attempts at Problem~\ref{pro:isometry} starting from the simplest case when points are ordered.
\smallskip

\noindent  
\textbf{The case of labelled clouds} $C\subset\R^n$ is easy for isometry classification because the matrix of distances $d_{ij}$ between indexed points $p_i,p_j$ allows us to reconstruct $C$ by using the known distances to the previously constructed points \cite[Theorem~9]{grinberg2019n}.
For any clouds of the same number $m$ of labelled points, the difference between $m\times m$ matrices of distances (or Gram matrices of $p_i\cdot p_j$) can be converted into a continuous metric by taking a matrix norm.
If given points are unlabelled, 
%as in most real applications and in Problem~\ref{pro:isometry}, 
comparing $m\times m$ matrices requires $m!$ permutations of points, which makes this approach impractical, see the faster order types for labelled points in \cite{cardinal2019subquadratic}.
\smallskip

\noindent
\textbf{Multidimensional scaling} (MDS).
For a given $m\times m$ distance matrix of any $m$-point cloud $A$, MDS \cite{schoenberg1935remarks} finds an embedding $A\subset\R^k$ (if it exists) preserving all distances of $M$ for a dimension $k\leq m$.
%The $m$ eigenvalues of the Gram matrix expressed via $D$ need $O(m^3)$ time.
A final embedding $A\subset\R^k$ uses %orthonormal 
eigenvectors whose ambiguity up to signs gives an exponential comparison time that can be close to $O(2^m)$.
\smallskip

\noindent
\textbf{The Hausdorff distance} \cite{hausdorff1919dimension} can be defined for any subsets $A,B$ in an ambient metric space as  $d_H(A,B)=\max\{\vec d_H(A,B), \vec d_H(B,A) \}$, where the directed Hausdorff distance is 
$\vec d_H(A,B)=\sup\limits_{p\in A}\inf\limits_{q\in B}|p-q|$.
To take into account isometries, one can minimize the Hausdorff distance over all isometries \cite{huttenlocher1993comparing,chew1992improvements,chew1999geometric}.
 % $f$ from the full Euclidean group $\iso(\R^n)$.
%For $n=1$, the Hausdorff distance minimized over translations in $\R$ for sets of at most $m$ points can be found in time $O(m\log m)$ \cite{rote1991computing}.
For $n=2$, the Hausdorff distance minimized over isometries in $\R^2$ for sets of at most $m$ point needs $O(m^5\log m)$ time \cite{chew1997geometric}. 
For a given $\ep>0$ and $n>2$, the related problem to decide if $d_H\leq\ep$ up to translations has the time complexity $O(m^{\lceil(n+1)/2\rceil})$ \cite[Chapter~4, Corollary~6]{wenk2003shape}. 
For general isometry, only approximate algorithms tackled minimizations for infinitely many rotations initially in $\R^3$ \cite{goodrich1999approximate}, now in $\R^n$ \cite[Lemma~5.5]{anosova2022algorithms}. 
\smallskip

\noindent
\textbf{The Gromov-Wasserstein distances} can be defined for metric-measure spaces, not necessarily sitting in a common ambient space.
The simplest Gromov-Hausdorff (GH) distance cannot be approximated with any factor less than 3 in polynomial time unless P = NP \cite[Corollary~3.8]{schmiedl2017computational}.
Polynomial-time algorithms for GH were designed for ultrametric spaces  \cite{memoli2021gromov}.
However, GH spaces are challenging even for finite clouds in the line $\R$, see \cite{majhi2019approximating} and \cite{zava2023gromov}.
\smallskip

\noindent  
\textbf{Topological Data Analysis} studies persistent homology \cite{edelsbrunner2000topological,carlsson2009topology}, which was finally recognised as an isometry invariant of a given cloud $C$, at least for standard filtrations such as Vietoris-Rips, Cech, and Delaunay complexes on $C$.
This invariant turned out to be rather weak and informally comparable to classical homology, which is insufficient to classify even graphs up to homeomorphism, see the non-trivial extensions of 0D persistence \cite{elkin2020mergegram,elkin2021isometry} and generic families of different scalar functions \cite{curry2018fiber,catanzaro2020moduli} and finite metric spaces \cite{smith2022families} that cannot be distinguished by persistence in dimensions 0 and 1, likely in all higher dimensions.
\smallskip

\noindent  
\textbf{Geometric Data Science} studies analogues of Problem~\ref{pro:isometry}, where rigid motion on clouds is replaced by any practical equivalence on real data objects.
Another attempt at Problem~\ref{pro:isometry} produced a complete invariant \cite{kurlin2022computable} of unlabelled clouds by using principal directions, which discontinuously change when a basis degenerates to a lower dimensional subspace but inspired Complete Neural Networks \cite{hordan2023complete}.
\smallskip

The earlier work has studied the following important cases of Problem~\ref{pro:isometry}: 1-periodic discrete series \cite{anosova2022density,anosova2023density,kurlin2022computable},
2D lattices \cite{kurlin2022mathematics,bright2023geographic}, 3D lattices \cite{mosca2020voronoi,bright2021welcome,kurlin2022exactly,kurlin2022complete}, periodic point sets in $\R^3$ \cite{smith2022practical,edelsbrunner2021density} and in higher dimensions \cite{anosova2021introduction,anosova2021isometry,anosova2022algorithms}.
\smallskip

The applications of  to crystalline materials  \cite{ropers2022fast,balasingham2022compact,vriza2022molecular,zhu2022analogy}
led to the \emph{Crystal Isometry Principle} \cite{widdowson2022average,widdowson2021pointwise,widdowson2022resolving}
extending Mendeleev's table of chemical elements to the \emph{Crystal Isometry Space} of all periodic crystals parametrised by complete invariants like a geographic map.
\smallskip

\noindent  
\textbf{Experimental approaches} tested invariant features
\cite{toews2013efficient,rister2017volumetric,spezialetti2019learning,zhu2022point}
or optimised equivariant descriptors for clouds from specific datasets \cite{chen2021equivariant,nigam2022equivariant,simeonov2022neural} without theoretical proofs of completeness, continuity, and time complexity, though sometimes beyond the Euclidean case as in the axiomatically rigorous Geometric Deep Learning \cite{bronstein2017geometric,bronstein2021geometric}.
The widely known concerns \cite{dong2018boosting,akhtar2018threat,laidlaw2019functional,guo2019simple,colbrook2022difficulty}
motivated us to state and solve Problem~\ref{pro:isometry} in a mathematical way.

%3===========
\section{Simplexwise Centred Distribution (SCD)}
\label{sec:SCD}

This section introduces the Oriented Simplexwise Distribution ($\OSD$) and its simplified version $\SCD$ (Simplexwise Centred Distribution) to solve Problem~\ref{pro:isometry}.
% by Theorems~\ref{thm:OSD_complete}, ~\ref{thm:SCD_complete}.
\smallskip

The \emph{lexicographic} order $u<v$ on any vectors $u=(u_1,\dots,u_n)$ and $v=(v_1,\dots,v_n)$ means that if the first $i$ (possibly, $i=0$) coordinates of $u,v$ coincide then $u_{i+1}<v_{i+1}$.
%For example, $(1,2)<(2,1)<(2,2)$.
Let $S_n$ denote the permutation group on indices $1,\dots,n$.

\begin{dfn}[matrices $D(A)$ and $M(C;A)$ for $A\subset C$]
\label{dfn:D(A)+M(C;A)}
Let $C$ be a cloud of $m$ unordered points in $\R^n$ with a fixed orientation.
Let $A=(p_1,\dots,p_n)\in C^n$ be an ordered sequence consists of $n$ distinct points.
Let $D(A)$ be the $n\times n$ \emph{distance} matrix whose entry $D(A)_{i,j}$ is Euclidean distance $|p_i-p_j|$ for $1\leq i<j\leq n$, all other entries are zeros.
\smallskip

For any other point $q\in C-A$, write distances from $q$ to $p_1,\dots,p_n$ as a column.
For the $n\times (m-n)$-matrix by these $m-n$ lexicographically ordered columns.
At the bottom of the column of a $q\in C-A$, add the sign of the determinant consisting of the vectors $q-p_1,\dots,q-p_n$.  
The resulting $(n+1)\times(m-n)$-matrix with signs in the bottom $(n+1)$-st row is the \emph{oriented relative distance} matrix $M(C;A)$.
\bs
\end{dfn}

Any permutation $\xi\in S_n$ is a composition of some $t$ transpositions $i\lra j$ and has $\sign(\xi)=(-1)^t$. 

\begin{dfn}[distributions $\ORD(C;A)$ and $\OSD(C)$ for $C\subset\R^n$]
\label{dfn:OSD}
Any permutation $\xi\in S_n$ acts on $D(A)$ by mapping $D(A)_{ij}$  to $D(A)_{kl}$, where $k\leq l$ is the pair $\xi(i),\xi(j)-1$ written in increasing order.
Then $\xi$ acts on $M(C;A)$ by mapping any $i$-th row to the $\xi(i)$-th row and by multiplying the $(n+1)$-st row by $\sign(\xi)$, after which all columns are written in the lexicographic order.
\smallskip

The \emph{Oriented Relative Distribution} $\ORD(C;A)$ is the equivalence class of the pair $[D(A);M(C;A)]$ up to all permutations $\xi\in S_n$. 
The \emph{Oriented Simplexwise Distribution} $\OSD(C)$ is the unordered collection of $\ORD(C;A)$ for all $\binom{m}{n}$ unordered subsets $A\subset C$ of $n$ points. 
\bs
\end{dfn}

Any mirror reflection in $\R^n$ reverses the sign of the $n\times n$ determinant consisting of vectors $v_1,\dots,v_n\in\R^n$, hence reverses all signs in the $(n+1)$-st rows of the matrices $M(C;A)$ in Oriented Relative Distributions.
$\mORD(C;A)$ and $\mOSD(C)$ denote the `mirror images' of $\ORD(C;A)$ and $\OSD(C)$, respectively, with all signs reversed.

\begin{figure}[h!]
\centering
\includegraphics[width=\linewidth]{images/triangular_clouds.png}
\caption{\textbf{Left}: a cloud $C=\{p_1,p_2,p_3\}$ with distances $a\leq b\leq c$.
\textbf{Middle}: the triangular cloud $R=\{(0,0),(4,0),(0,3)\}$. 
\textbf{Right}: the square cloud $S=\{(1,0),(-1,0),(0,1),(-1,0)\}$. 
}
\label{fig:triangular_clouds}
\end{figure}

\begin{exa}[$\OSD$ for mirror clouds]
\label{exa:OSD+mirror}
In $\R^2$ with the counter-clockwise orientation, the cloud $R$ on the vertices $p_1=(0,0)$, $p_2=(4,0)$, $p_3=(0,3)$ of the triangle in Fig.~\ref{fig:triangular_clouds}~(middle) has $\OSD(R)$ consisting of 
$\ORD(R;(p_1,p_2))=[4,\colthree{3}{5}{+}]$,
$\ORD(R;(p_2,p_3))=[5,\colthree{4}{3}{+}]$,
$\ORD(R;(p_3,p_1))=[3,\colthree{5}{4}{+}]$.
If we swap the points $p_1\lra p_3$, the last $\ORD$ above changes to the equivalent form $\ORD(R;(p_1,p_3))=[3,\colthree{4}{5}{-}]$, without affecting other $\ORD$s.
If we reflect $R$ with respect to the $x$-axis, the new cloud $\bar R$ of the points $p_1,p_2,\bar p_3=(0,-3)$ has $\OSD(\bar R)=\mOSD(R)$ with
$\ORD(\bar R;(p_1,p_2))=[4,\colthree{3}{5}{-}]$,
$\ORD(\bar R;(p_2,\bar p_3))=[5,\colthree{4}{3}{-}]$,
$\ORD(\bar R;(p'_3,p_1))=[3,\colthree{5}{4}{-}]$, which differs from $\OSD(R)$ even if we swap points in each pair.
\bs
\end{exa}

\begin{exa}[$\OSD$ for $T,K$]
\label{exa:OSD+T+K}
Fix the \emph{counter-clockwise} orientation on $\R^2$ so that if a vector $v$ is obtained from $u$ by a counter-clockwise rotation then $\det(u,v)>0$. 
Then Table~\ref{tab:OSD+TK} shows the Oriented Simplexwise Distribution for the 4-point clouds $T,K\subset\R^2$ in Fig.~\ref{fig:4-point_clouds}.
\bs
\end{exa}

\begin{table}[h!]
  \centering
  \begin{tabular}{@{}l|l@{}}
    \toprule
    $\ORD$s in $\OSD(T)$ & $\ORD$s in $\OSD(K)$ \\
    \midrule
    
	$[\sqrt{2},\mathree{2}{\cbox{yellow}{\sqrt{10}}}{-}{\sqrt{10}}{4}{-}]$ &
	$[\sqrt{2},\mathree{2}{\cbox{yellow}{\sqrt{2}}}{-}{\sqrt{10}}{4}{-}]$ 
\smallskip \\

	$[\sqrt{2},\mathree{2}{\cbox{yellow}{\sqrt{10}}}{+}{\sqrt{10}}{4}{+}]$ &
	$[\sqrt{2},\mathree{2}{\cbox{yellow}{\sqrt{2}}}{+}{\sqrt{10}}{4}{+}]$ 
\smallskip \\
	
	    $[2,\mathree{\sqrt{2}}{\cbox{yellow}{\sqrt{10}}}{-}{\sqrt{10}}{\cbox{yellow}{\sqrt{2}}}{\cbox{yellow}{-}}]$ & 
    $[2,\mathree{\sqrt{2}}{\cbox{yellow}{\sqrt{2}}}{-}{\sqrt{10}}{\cbox{yellow}{\sqrt{10}}}{\cbox{yellow}{+}}]$
\smallskip \\
    
	$[\sqrt{10},\mathree{\sqrt{2}}{\cbox{yellow}{2}}{\cbox{yellow}{+}}{\cbox{yellow}{4}}{\cbox{yellow}{\sqrt{2}}}{-}]$ &
	$[\sqrt{10},\mathree{\sqrt{2}}{\cbox{yellow}{4}}{\cbox{yellow}{-}}{\cbox{yellow}{2}}{\cbox{yellow}{\sqrt{10}}}{-}]$ \smallskip \\
	
	$[\sqrt{10},\mathree{\sqrt{2}}{\cbox{yellow}{2}}{\cbox{yellow}{-}}{\cbox{yellow}{4}}{\cbox{yellow}{\sqrt{2}}}{+}]$ & 
	$[\sqrt{10},\mathree{\sqrt{2}}{\cbox{yellow}{4}}{\cbox{yellow}{+}}{\cbox{yellow}{2}}{\cbox{yellow}{\sqrt{10}}}{+}]$ \smallskip \\
	
	    $[4,\mathree{\sqrt{2}}{\sqrt{10}}{+}{\cbox{yellow}{\sqrt{10}}}{\cbox{yellow}{\sqrt{2}}}{\cbox{yellow}{+}}]$ &
	$[4,\mathree{\sqrt{2}}{\sqrt{10}}{+}{\cbox{yellow}{\sqrt{2}}}{\cbox{yellow}{\sqrt{10}}}{\cbox{yellow}{-}}]$  \\

    \bottomrule
  \end{tabular}
  \caption{The Oriented Simplexwise Distributions from Definition~\ref{dfn:OSD} for the 4-point clouds $T,K\subset\R^2$ in Fig.~\ref{fig:4-point_clouds}. 
%  Forgetting all signs in the bottom rows of $\RDD$s gives $\SDD$s in Table~\ref{tab:SDD+TK}
}
  \label{tab:OSD+TK}
\end{table}

The book ``Euclidean Distance Geometry'' \cite[Chapter~3]{liberti2017euclidean} discusses realizations of a complete graph given by a full distance matrix in $\R^n$.
Lemma~\ref{lem:D(A)+reconstruction} and later resuts in the appendices hold for all cases including degenerate ones, for example, when 3 points are in a straight line in $\R^3$.
\smallskip

\begin{lem}[reconstructing from $D(A)$]
\label{lem:D(A)+reconstruction}
Any finite ordered set $A\subset\R^n$ is reconstructed (uniquely up to isometry) from the distance matrix $D(A)$ 
in Definition~\ref{dfn:OSD}.
\bs
\end{lem}
\begin{proof}%[Proof of Lemma~\ref{lem:D(A)+reconstruction}]
Let $A\subset\R^n$ consist of $n$ points $p_1,\dots,p_n$ as in Definition~\ref{dfn:D(A)+M(C;A)}.
A translation allows us to put $p_1$ at the origin $0\in\R^n$.
A rotation allows us to put $p_2$ in the (positive half of the) 1st coordinate axis at the distance $|p_2-p_1|$ from $p_1=0$.
A further rotation around the 1st coordinate axis allows us to put $p_3$ in the (positive half-) plane on the first two coordinates axes by using the distances $|p_3-p_1|$ and $|p_3-p_2|$ from the given distance matrix $D(A)$, and so on.
\smallskip

In a degenerate situation, a point $p_{k+1}$ might belong to the subspace spanned by $p_1,\dots,p_k$.
For example, if $k=2$ and $|p_3-p_1|\pm |p_3-p_2|=\pm |p_2-p_1|$, then the point $p_3$ belongs to the 1st coordinate axes through $p_1=0$ and $p_2$.
Then we have more flexibility with rotations, so $p_4$ can be rotated around the 1st coordinate axes through $p_1,p_2,p_3$. 
\smallskip

So we continue until either all $h$ points of $A$ are fixed or all already fixed points affinely span $\R^n$.
If the former case, $A$ is reconstructed uniquely up to isometry of $\R^n$ that keeps invariant the low dimensional subspace spanned by $A$.
In the latter case, any next point $q\in A$ is uniquely located in $\R^n$ by at least $n+1$ distances to the already fixed points.
\end{proof}

Lemma~\ref{lem:D(A)+reconstruction} doesn't hold for stricter rigid motion instead of isometry.
A 3-point cloud with inter-point distances $3,4,5$ can be reconstructed in $\R^2$ as two triangles related by reflection, not by rigid motion, see Fig.~\ref{fig:triangular_clouds}~(middle).% Example~\ref{exa:SCD}.
\smallskip

The \emph{affine dimension} $\aff(C)$ of a cloud $C=\{p_1,\dots,p_m\}\subset\R^n$ is the maximum dimension of the vector space generated by all inter-point vectors $p_i-p_j$, $i,j\in\{1,\dots,m\}$.
Then $\aff(C)$ is an isometry invariant and is independent of an order of points of $C$.
Any cloud $C$ of 2 distinct points has $\aff(C)=1$.
Any cloud $C$ of 3 points that are not in the same straight line has $\aff(C)=2$.
Let $S(p;d)$ be the sphere with a centre $p$ and a radius $d$.
%\smallskip

\begin{lem}[reconstructing from $\ORD$]
\label{lem:ORD+reconstruction}
A cloud $C\subset\R^n$ of $m>n$ unlabelled points can be reconstructed (uniquely up to rigid motion) from $\ORD(C;A)$ in Definition~\ref{dfn:OSD} for any ordered subset $A\subseteq C$ with $\aff(A)=n-1$.  
\bs
\end{lem}
\begin{proof}%[Proof of Lemma~\ref{lem:ORD+reconstruction}]
By Lemma~\ref{lem:D(A)+reconstruction} any subset $A\subseteq C$ can be uniquely reconstructed up to isometry from the triangular distance matrix $D(A)$ in Definition~\ref{dfn:OSD}.
Since $\aff(A)=n-1$, the subset $A$ has $h\geq n$ points.
We may assume that the first $n$ points $p_1,\dots,p_n$ of $A$ span the subspace of the first $n-1$ coordinate axes of $\R^n$, all unique up to rigid motion. 
\smallskip

We prove that any point $q\in C-A\subset\R^n$ has a location determined by the $n$ distances $|q-p_1|,\dots,|q-p_n|$ written in
 a column of the matrix $\ORD(C;A)$.
%Since the points of $A$ do not belong to any $(n-1)$-dimensional affine subspace of $\R^n$, 
The $n$ spheres $S(p_i;|q-p_i|)$, $i=1,\dots,n$, contain $q$ and intersect in one or two points. 
We can uniquely choose $q$ among these two options due to the sign of the determinant (in the bottom row of $\ORD(C;A)$) of the vectors $q-p_1,\dots,q-p_n$. 
\end{proof}

Lemma~\ref{lem:ORD+reconstruction} implies that $\ORD(C;A)$ can have identical columns only for degenerate subsets $A\subset C$ with $\aff(A)<n-1$.
For example, let $n=3$ and $A$ consist of three points $p_1,p_2,p_3$ in the same straight line $L\subset\R^3$.
The three distances $|q-p_i|$, $i=1,2,3$, to any other point $q\in C$ outside $L$ define three spheres $S(p_i;|q-p_i|)$ that share a common circle in $\R^3$, so the position of $q$ is not uniquely determined in this case.
\smallskip

Though one $\ORD(C;A)$ with $\aff(A)=n-1$ suffices to reconstruct $C\subset\R^n$ up to rigid motion, the dependence on a subset $A\subset C$ required us to consider the larger Oriented Simplexwise Distribution $\OSD(C)$ for all $n$-point subsets $A\subset C$ to get a complete invariant in Theorem~\ref{thm:OSD_complete}. 
\smallskip

An equality $\OSD(C)=\OSD(C')$ is interpreted as a bijection $\OSD(C)\to\OSD(C')$ matching all ORDs. 

\begin{thm}[completeness of $\OSD$]
\label{thm:OSD_complete}
The Oriented Simplexwise Distribution $\OSD(C)$ in Definition~\ref{dfn:OSD} is a complete isometry invariant and can be computed in time $O(m^{n+1}/(n-3)!)$. 
So any clouds $C,C'\subset\R^n$ of $m$ unlabelled points are related by rigid motion (isometry, respectively) \emph{if and only if} 
$\OSD(C)=\OSD(C')$ ($\OSD(C)=\OSD(C')$ or its mirror image $\mOSD(C')$, respectively).
\end{thm}
\begin{proof}
Part \emph{if} $\Leftarrow$.
Any bijection $\OSD(C)\to\OSD(C')$ matches $\ORD(C;A)$ with $\ORD(C';A')$ for some subsets $A\subset C$ and $A'\subset C'$ of $n$ points.
By Lemma~\ref{lem:ORD+reconstruction} any equality $\ORD(C;A)=\ORD(C';A')$ for $n$-point subsets $A,A'$ with $\aff=n-1$ guarantees that $C,C'$ are related by rigid motion in $\R^n$.
In the degenerate case, when all subsets have $\aff<n-1$, hence $C,C'$ belong to a $k$-dimensional subspace of $\R^n$ for $k<n$, we apply the reconstruction of Lemma~\ref{lem:ORD+reconstruction} to $\R^k$ instead of $\R^n$.  
In the case $\OSD(C)=\mOSD(C')$, we get an equality $\ORD(C;A)=\ORD(\bar C';A')$, where $\bar C'$ is a mirror image of $C'$, 
hence $C,C'$ are related by an orientation-reversing isometry.
\smallskip

Part \emph{only if} $\Rightarrow$.
Any rigid motion $f$ of $\R^n$ bijectively maps $C$ to $C'$ and any subset $A\subseteq C$ to $A'=f(A)\subseteq C'=f(C)$, hence induces a bijection $\OSD(C)\to\OSD(C')$. 
Similarly, any orientation-reversing isometry $C\to C'$ indices a bijection $\OSD(C)\to\mOSD(C')=\OSD(\bar C')$.
\smallskip

To compute $\OSD(C)$, we consider $\binom{m}{n}=\frac{m!}{n!(m-n)!}$ subsets $A\subset C$ of $n$ points.
For each fixed $A$, the matrix $D(A)$ of $\frac{n(n-1)}{2}$ pairwise distances needs $O(n^2)$ time.
The Oriented Relative Distribution $\ORD(C;A)=[D(A);M(C;A)]$ includes $n(m-n)$ distances, and also $m-n$ signs and strengths that each requires determinant computations in time $O(n^3)$ by Gaussian elimination.
So $\ORD(C;A)$ can be computed in time $O(n^3m)$.
\smallskip

Multiplying the last time by the number $\binom{m}{n}=\frac{m!}{n!(m-n)!}$ of $n$-point subsets $A\subset C$, we estimate the final time for $\OSD(C)$ as
$O(\frac{m!n^3m}{n!(m-n)!})=O(m^2(m-1)\dots(m-n+1)n^2/(n-1)!)=O(m^{n+1}/(n-3)!)$.
\end{proof}

Now we simplify the $\OSD$ invariant to the Simplexwise Centred Distribution (SCD) by using the centre of mass of $C$ as a useful anchor reducing the ambiguity.
\smallskip

The Euclidean structure of $\R^n$ allows us to translate the \emph{center of mass} $\dfrac{1}{m}\sum\limits_{p\in C} p$ of a given $m$-point cloud $C\subset\R^n$ to the origin $0\in\R^n$.
Then Problem~\ref{pro:isometry} reduces to only rotations around $0$ from the orthogonal group $\Or(\R^n)$.
\smallskip

Definition~\ref{dfn:OSD} introduced the Oriented Simplexwise Distribution (OSD) as an ordered collection of $\ORD(C;A)$ for all $\binom{m}{n}$ unordered subsets $A\subset C$ of $n$ points.
Including the centre of mass, allows us to consider the smaller number of $\binom{m}{n-1}$ subsets $A\subset C$ of $n-1$ points instead of $n$.
\smallskip

Though the centre of mass is  uniquely determined by any cloud $C\subset\R^n$ of unlabelled points, real applications may offer one or  several labelled points of $C$ that substantially speed up metrics on invariants.
For example, an atomic neighbourhood in a solid material is a cloud $C\subset\R^3$ of atoms around a central atom, which may not be the centre of mass of $C$, but is suitable for all methods below.
\smallskip

For any basis sequence $A=\{p_1,\dots,p_{n-1}\}\in C^{n-1}$ of $n-1$ ordered points, add the origin $0$ as the $n$-th point and consider the $n\times n$
distance matrix $D(A\cup\{0\})$ and the $(n+1)\times (m-n)$ matrix $M(C;A\cup\{0\})$ in Definition~\ref{dfn:D(A)+M(C;A)}.
Any $n$ vectors $v_1,\dots,v_n\in\R^n$ can be written as columns in the $n\times n$ matrix whose determinant has a \emph{sign} $\pm 1$ or $0$ if the vectors $v_1,\dots,v_n$ are linearly dependent.
\smallskip

Any permutation $\xi\in S_{n-1}$ of $n-1$ points of $A$ acts on $D(A)$, permutes the first $n-1$ rows of $M(C;A\cup\{0\})$ and multiplies every sign in the $(n+1)$-st row by $\sign(\xi)$.

\begin{dfn}[Simplexwise Centred Distribution $\SCD$]
\label{dfn:SCD}
Let $C\subset\R^n$ be any cloud of $m$ unlabelled points.
For any basis sequence $A=(p_1,\dots,p_{n-1})\in C^{n-1}$,  the \emph{Oriented Centred Distribution} $\OCD(C;A)$ is the equivalence class of pairs $[D(A\cup\{0\}),M(C;A\cup\{0\})]$ considered up to permutations $\xi\in S_{n-1}$ of points of $A$.
\smallskip

The \emph{Simplexwise Centred Distribution} $\SCD(C)$ is the unordered set of the distributions $\OCD(C;A)$ for all $\binom{m}{n-1}$ unordered $(n-1)$-point subsets $A\subset C$.
The mirror image $\mSCD(C)$ is obtained from $\SCD(C)$ by reversing signs.
\bs
\end{dfn}

Definition~\ref{dfn:SCD} needs no permutations for any $C\subset\R^2$ as $n-1=1$.
Columns of $M(C;A\cup\{0\})$ can be lexicographically ordered without affecting future metrics. %in Lemma~\ref{lem:OCD+metric}.
\smallskip

Some of the $\binom{m}{n-1}$ $\OCD$s in $\SCD(C)$ can be identical as in Example~\ref{exa:SCD}(b).
If we collapse any $l>1$ identical $\OCD$s into a single $\OCD$ with the \emph{weight} $l/\binom{m}{h}$, $\SCD$ can be considered as a weighted probability distribution 
of $\OCD$s.

\begin{exa}[$\SCD$ for clouds in Fig.~\ref{fig:triangular_clouds}]
\label{exa:SCD}
\textbf{(a)}
Let $R\subset\R^2$ consist of the vertices $p_1=(0,0)$, $p_2=(4,0)$, $p_3=(0,3)$ of the right-angled triangle in Fig.~\ref{fig:triangular_clouds}~(middle).
Though $p_1=(0,0)$ is included in $R$ and is not its centre of mass,
$\SCD(R)$ still makes sense.
In
$\OCD(R;p_1)=[0,\left( \begin{array}{cc} 
4 & 3 \\
4 & 3 \\
0 & 0
\end{array}\right) ]$,
the matrix $D(\{p_1,0\})$ is $|p_1-0|=0$,
the top row has $|p_2-p_1|=4$, $|p_3-p_1|=3$.
In $\OCD(R;p_2)=[4,\left( \begin{array}{cc} 
4 & 5 \\
0 & 3 \\
0 & -
\end{array}\right) ]$,
the first row has $|p_1-p_2|=4$, $|p_3-p_2|=5$,
the second row has $|p_1-0|=0$, $|p_3-0|=3$,
$\det\mat{-4}{0}{3}{3}<0$.
In $\OCD(R;p_3)=[3,\left( \begin{array}{cc} 
3 & 5 \\
0 & 4 \\
0 & +
\end{array}\right) ]$,
the first row has $|p_1-p_3|=3$, $|p_2-p_3|=5$,
the second row has $|p_1-0|=0$, $|p_2-0|=4$,
$\det\mat{4}{4}{-3}{0}>0$.
So $\SCD(R)$ consists of the three $\OCD$s above.
\smallskip

If we reflect $R$ with respect to the $x$-axis, the new cloud $\bar R$ of the points $p_1,p_2,\bar p_3=(0,-3)$ has $\SCD(\bar R)=\mSCD(R)$ with
$\OCD(\bar R;p_1)=\OCD(R)$,
$\OCD(\bar R;p_2)=[4,\left( \begin{array}{cc} 
4 & 5 \\
0 & 3 \\
0 & +
\end{array}\right) ]$,
$\OCD(R;\bar p_3)=[3,\left( \begin{array}{cc} 
3 & 5 \\
0 & 4 \\
0 & -
\end{array}\right) ]$ whose signs changed under reflection, so $\SCD(R)\neq\SCD(\bar R)$.
\medskip

\noindent
\textbf{(b)}
Let $S\subset\R^2$ consist of $m=4$ points $(\pm 1,0),(0,\pm 1)$ that are vertices of the square in Fig.~\ref{fig:triangular_clouds}~(right).
The centre of mass is $0\in\R^2$ and has a distance $1$ to each point of $S$.
\smallskip

For each 1-point subset $A=\{p\}\subset S$, the distance matrix $D(A\cup\{0\})$ on two points is the single number $1$.
The matrix $M(S;A\cup\{0\})$ has $m-n+1=3$ columns.
For $p_1=(1,0)$, we have   
$M(S;\vect{p_1}{0})=\left(\begin{array}{ccc} 
\sqrt{2} & \sqrt{2} & 2 \\
1 & 1 & 1 \\
- & + & 0
\end{array}\right)$, where the columns are ordered according to
$p_2=(0,-1)$, $p_3=(0,1)$, $p_4=(-1,0)$ in Fig.~\ref{fig:triangular_clouds}~(right).
The sign in the bottom right corner is 0 because the points $p_1,0,p_4$ are in a straight line. 
Due to the rotational symmetry, $M(S;\{p_i,0\})$ is independent of $i=1,2,3,4$.
So $\SCD(S)$ can be considered as one $\OCD=[1,M(S;\vect{p_1}{0})]$ of weight 1. 
\bs 
\end{exa}

Example~\ref{exa:SCD}(b) illustrates the key discontinuity challenge:
if $p_4=(-1,0)$ is perturbed, the corresponding sign can discontinuously change to $+1$ or $-1$.
To get a continuous metric on $\OCD$s, we will multiply each sign by 
a continuous \emph{strength} function that vanishes for any zero sign.
%For extra signs in an Oriented Relative Distribution (ORD) from Definition~\ref{dfn:OSD}, we use a non-trivial function $\si(A)$, which was not needed for the simpler metric $M_\infty$.

\begin{thm}[completeness of $\SCD$]
\label{thm:SCD_complete}
\textbf{(a)}
The Simplexwise Centred Distribution $\SCD(C)$ in Definition~\ref{dfn:SCD} is a complete isometry invariant of clouds $C\subset\R^n$ of $m$ unlabelled points with a centre of mass at the origin $0\in\R^n$, and can be computed in time $O(m^n/(n-4)!)$.
\smallskip

So any clouds $C,C'\subset\R^n$ are related by rigid motion (isometry, respectively) \emph{if and only if} 
$\SCD(C)=\SCD(C')$ ($\SCD(C)$ equals $\SCD(C')$ or its mirror image $\mSCD(C')$, respectively).
For any $m$-point clouds $C,C'\subset\R^n$, let 
%the Simplexwise Centred Distributions 
$\SCD(C)$ and $\SCD(C')$ consist of $k=\binom{m}{n-1}$ $\OCD$s.
\end{thm}
\begin{proof}[Proof of Theorem~\ref{thm:SCD_complete}]
All arguments follow the proof of Theorem~\ref{thm:OSD_complete} after replacing $n$ with $n-1$.
\end{proof}

%4==========================
\section{The new continuous strength of a simplex}
\label{sec:strength}

This section resolves the discontinuity of signs of determinants by introducing the multiplicative factor below.

\begin{dfn}[\emph{strength} $\si(A)$ of a simplex]
\label{dfn:strength}
For a set $A$ of $n+1$ points $q=p_0,p_1,\dots,p_n$ in $\R^n$,
let $p(A)=\frac{1}{2}\sum\limits_{i\neq j}^{n+1}|p_i-p_j|$ be half of the sum of all pairwise distances.
Let $V(A)$ denote the volume the $n$-dimensional simplex on the set $A$.
Define the \emph{strength} $\si(A)=V^2(A)/p^{2n-1}(A)$.
\smallskip

For $n=2$ and a triangle $A$ with sides $a,b,c$ in $\R^2$, Heron's formula %for the area 
gives $\si(A)=\dfrac{(p-a)(p-b)(p-c)}{p^2}$, $p=\dfrac{a+b+c}{2}=p(A)$ is the half-perimeter of $A$.
\bs  
\end{dfn}

For $n=1$ and a set $A={p_0,p_1}\subset\R$, the volume is $V(A)=|p_0-p_1|=2p(A)$, so $\si(A)=%V^2(A)/p(A)=
2|p_0-p_1|$.
% is the double length.
\smallskip
 
The strength $\si(A)$ depends only on the distance matrix $D(A)$ from Definition~\ref{dfn:D(A)+M(C;A)}, so the notation $\si(A)$ is used only for brevity.
In any $\R^n$, the squared volume $V^2(A)$ is expressed by the Cayley-Menger determinant \cite{sippl1986cayley} in pairwise distances between points of $A$.
Importantly, the strength $\si(A)$ vanishes when the simplex on a set $A$ degenerates. 
\smallskip

Corollary~\ref{cor:continuity} will need the continuity of $s\si(A)$, when a sign $s\in\{\pm 1\}$ from a bottom row of ORD changes while passing through a degenerate set $A$.
In appendices, the proof of the continuity of $\si(A)$ in Theorem~\ref{thm:strength} gives an explicit upper bound for a Lipschitz constant $c_n$ below. 

\begin{thm}[Lipschitz continuity of the strength $\si$]
\label{thm:strength}
Let a cloud $A'$ be obtained from another $(n+1)$-point cloud $A\subset\R^n$ by perturbing every point within its $\ep$-neighbourhood.
The strength $\si(A)$ from Definition~\ref{dfn:strength} is Lipschitz continuous so that $|\si(A')-\si(A)|\leq 2\ep c_n$ for a constant $c_n$.
\bs
\end{thm}
\begin{proof}[Proof of Theorem~\ref{thm:strength} for dimension $n=2$ and $c_2=2\sqrt{3}$]
Let a 3-point cloud $A\subset\R^2$ have pairwise distances $a,b,c$.
Using the half-perimeter $p=\dfrac{a+b+c}{2}$, the variables $\ti a=p-a$, $\ti b=p-b$, $\ti c=p-c$ are independent and bijectively expressed via $a,b,c$, so $a=\ti b+\ti c$, $b=\ti a+\ti c$, $c=\ti a+\ti b$, $p=\ti a+\ti b+\ti c$.
The Jacobian of this change of variables is $\bpd{(a,b,c)}{(\ti a,\ti b,\ti c)}=\left| \begin{array}{ccc} 0 & 1 & 1 \\ 1 & 0 & 1 \\ 1 & 1 & 0 \end{array} \right|=2$. 
\smallskip

If each point of $A$ is perturbed within its $\ep$-neighbourhood in the Euclidean distance on $\R^2$, then any pairwise distance between points of $A$ changes by at most $2\ep$. 
\smallskip

By the mean value theorem \cite{elliott2012probabilistic}, this bound $2\ep$ gives $|\si(A')-\si(A)|\leq 2\ep\sup|\nabla\si|$, where $\nabla\si=(\pd{\si}{a},\pd{\si}{b},\pd{\si}{c})$ is the gradient of the first order partial derivatives of $\si(A)$ with respect to the three distances between points of $A$.
\smallskip

Since $|\nabla\si|=\left|\bpd{(a,b,c)}{(\ti a,\ti b,\ti c)}\Big(\bpd{\si}{\ti a},\bpd{\si}{\ti b},\bpd{\si}{\ti c}\Big)\right|\leq 2\left|\Big(\bpd{\si}{\ti a},\bpd{\si}{\ti b},\bpd{\si}{\ti c}\Big)\right|$, it remains to estimate the first order partial derivatives of the strength from Definition~\ref{dfn:strength}
$\si=\dfrac{\ti a\ti b\ti c}{(\ti a+\ti b+\ti c)^2}$
 with respect to the variables $\ti a,\ti b,\ti c$.   
Since $\si$ is symmetric $\ti a,\ti b,\ti c$, it suffices to consider
$\bpd{\si}{\ti a}=\dfrac{\ti b\ti c}{(\ti a+\ti b+\ti c)^2}-\dfrac{2\ti a\ti b\ti c}{(\ti a+\ti b+\ti c)^3}=\dfrac{\ti b\ti c(\ti b+\ti c-\ti a)}{(\ti a+\ti b+\ti c)^3}=\dfrac{(p-b)(p-c)(2a-p)}{p^3}=\left(1-\dfrac{b}{p}\right)\left(1-\dfrac{c}{p}\right)\left(2\dfrac{a}{p}-1\right)$.
By the triangle inequalities for $a,b,c$, we have $\max\{a,b,c\}\leq p=\frac{a+b+c}{2}$.
Then $\frac{a}{p},\frac{b}{p},\frac{c}{p}\in(0,1]$, and
$1-\frac{b}{p},1-\frac{c}{p}\in[0,1)$, and 
$2\frac{a}{p}-1\in (-1,1]$, so 
$|\pd{\si}{\ti a}|\leq 1$.
The similar upper bounds $|\pd{\si}{\ti b}|,|\pd{\si}{\ti c}|\leq 1$ imply that $|\nabla\si|\leq 2\sqrt{3}$ and
$|\si(A')-\si(A)|\leq 2\ep\sup|\nabla\si|\leq 2\ep c_2$ for $c_2=2\sqrt{3}$.
\end{proof}

The used inequality $\max\{a,b,c\}\leq p$ extends to $n\geq 3$.

\begin{lem}[upper bound for edge ratios]
\label{lem:bound}
For any $(n+1)$-point set $A\subset\R^n$ with pairwise distances $d_{kl}$, we have $\dfrac{d_{kl}}{p(A)}\leq\dfrac{2}{n}$ for any $k,l$, where $p(A)=\dfrac{1}{2}\sum\limits_{i,j=1}^{n+1} d_{ij}$.
\bs
\end{lem}  
\begin{proof}
Use the triangle inequalities: $2p(A)=\sum\limits_{i,j=1}^{n+1} d_{ij}\geq d_{kl}+\sum\limits_{i\neq k,l} (d_{ik}+d_{il})=d_{kl}+(n-1)d_{kl}=nd_{kl}$.
\end{proof}

We express a Lipschitz constant $c_n$
of the strength $\si(A)$ in Theorem~\ref{thm:strength} via the \emph{rencontre} number $r_n=n!\sum\limits_{k=0}^n\dfrac{(-1)^k}{k!}$ counting permutations of $1,\dots,n$ without a fixed point \cite{charalambides2018enumerative}, e.g. $r_2=1$, $r_3=2$, $r_4=9$, $r_5=44$.
\smallskip

The proof of Theorem~\ref{thm:strength} for $n\geq 3$ below gives the following rough upper bound of a Lipschitz constant 
$$c_n=\left(4r_n+2r_{n+1}+\frac{n(2n-1)}{4}r_{n+2}\right)\dfrac{2^{n-0.5}\sqrt{n+1}}{(n!)^2n^{2n-1.5}}.$$
The convergence $c_n\to 0$ as $n\to+\infty$ illustrates the curse of dimensionality meaning that the change of volume is tiny for big $n$.  
The bounds above give $c_2\approx 4.7$, $c_3\approx 0.43$.

\begin{proof}[Proof of Theorem~\ref{thm:strength} for $n\geq 3$]
The Cayley-Menger determinant \cite{sippl1986cayley} expresses the squared volume of the $n$-dimensional simplex on any $(n+1)$-point set $A=\{p_1,\dots,p_{n+1}\}\subset\R^n$ as $V^2(A)=\dfrac{(-1)^{n-1}}{2^n(n!)^2}\det\hat B$, where the $(n+2)\times(n+2)$ matrix $\hat B$ is obtained from the $(n+1)\times(n+1)$ matrix $B_{ij}=|p_i-p_j|^2$ of squared Euclidean distances by bordering $B$ with a top row $(0,1,\dots,1)$ and a left column $(0,1,\dots,1)^T$.
For $n=3$, the squared volume is 
$V^2(A)=\dfrac{1}{288}
\left| \begin{array}{ccccc} 
0 & 1 & 1 & 1 & 1 \\
1 & 0 & d_{12}^2 & d_{13}^2 & d_{14}^2 \\
1 & d_{21}^2 & 0 & d_{23}^2 & d_{24}^2 \\
1 & d_{31}^2 & d_{32}^2 & 0 & d_{34}^2 \\
1 & d_{41}^2 & d_{42}^2 & d_{43}^2 & 0 
\end{array} \right|$,
where $d_{ij}=d_{ji}$ is the Euclidean distance between %the points 
$p_i,p_j$. 
\smallskip

Similarly to $n=2$, the mean value theorem
\cite{elliott2012probabilistic} for the strength $\si(A)=V^2(A)/p^{2n-1}(A)$, where $p(A)=\frac{1}{2}\sum\limits_{i\neq j}^{n+1} d_{ij}$, implies for any cloud $A$ and its perturbation $A'$ \\
$|\si(A')-\si(A)|\leq 2\ep\sup|\nabla\si|\leq 2\ep\sqrt{\frac{n(n+1)}{2}}\sup\limits_{i,j}\left|\pd{\si}{d_{ij}}\right|.$
To find an upper bound of $\left|\bpd{\si}{d_{ij}}\right|$ for $\si=\dfrac{V^2(A)}{p^{2n-1}(A)}$, we initially ignore the numerical factor in $V^2(A)=\dfrac{(-1)^{n-1}}{2^n(n!)^2}\det\hat B$ and differentiate only $\det\hat B\cdot\dfrac{1}{p^{2n-1}(A)}$ by the product rule as follows:
$\bpd{}{d_{ij}}\left(\dfrac{\det\hat B}{p^{2n-1}(A)}\right)=
\bpd{\det\hat B}{d_{ij}}\dfrac{1}{p^{2n-1}(A)}-\det\hat B\dfrac{n-\frac{1}{2}}{p^{2n-2}(A)}$.
Lemmas~\ref{lem:detB} and \ref{lem:derB} below imply the upper bound $\left|\bpd{}{d_{ij}}\dfrac{\det\hat B}{p^{2n-1}(A)}\right|
\leq$ \\ $(4r_n+2r_{n+1}) \left(\dfrac{2}{n}\right)^{2n-1}
+(n-\frac{1}{2})r_{n+2}\left(\dfrac{2}{n}\right)^{2n-2}
=(4r_n+2r_{n+1}+\frac{n(2n-1)}{4}r_{n+2})\left(\dfrac{2}{n}\right)^{2n-1}$.
Taking into account the factors $\dfrac{(-1)^{n-1}}{2^n(n!)^2}$ in $V^2(A)$ and $\sqrt{\frac{n(n+1)}{2}}$ for estimating the length of the gradient $\nabla\si$ of $\frac{n(n+1)}{2}$ first order partial derivatives $\pd{\si}{d_{ij}}$, the final Lipschitz constant is \\ $c_n=(4r_n+2r_{n+1}+\frac{n(2n-1)}{4}r_{n+2})\dfrac{2^{n-0.5}\sqrt{n+1}}{(n!)^2n^{2n-1.5}}$.
\end{proof}

\begin{lem}
\label{lem:detB}
$\left|\dfrac{\det\hat B}{(p(A))^{2n-2}}\right|\leq r_{n+2}\left(\dfrac{2}{n}\right)^{2n-2}$.
\end{lem}
\begin{proof}
The determinant $\det\hat B$ is a polynomial of maximum degree $2(n-1)$ in all distances $d_{ij}$.
The determinant formula $\det\hat B=\sum\limits_{\xi\in S_{n+2}} (-1)^{\sign(\xi)}\al_{1,\xi(1)}\dots\al_{n,\xi(n)}$ excludes all zeros on the diagonal, so $k\neq\xi(k)$ for $k=1,\dots,n+2$ and all permutations $\xi$ of $1,\dots,n+2$ have no fixed points.
Then $\det\hat B$ is a sum of (the rencontre number) $r_{n+2}$ terms.
Each term is a product of at most $2n-2$ distances.
Divide by $p^{2n-2}(A)$ and use Lemma~\ref{lem:bound}.
\end{proof}

\begin{lem}
\label{lem:derB}
In the above notations for fixed $i,j$, we have
$\left|\bpd{\det\hat B}{d_{ij}}\dfrac{1}{p^{2n-1}(A)}\right|
\leq (4r_n+2r_{n+1}) \left(\dfrac{2}{n}\right)^{2n-1}$.
\end{lem}
\begin{proof}
Since $\det\hat B$ has $d_{ij}^2$ in exactly two cells in different rows and columns, $\det\hat B$ is a quadratic polynomial $\al d_{ij}^4+\be d_{ij}^2+\ga$, where $\al,\be,\ga$ depend on other fixed distances $d_{kl}\neq d_{ij}$.
Then $\bpd{\det\hat B}{d_{ij}}=4\al d_{ij}^3+2\be d_{ij}$.
\smallskip

The coefficient $\al$ is the determinant of the $n\times n$ submatrix $(\al_{ij})$ obtained from $\hat B$ by removing two rows and columns $i+1,j+1$.
Since this submatrix has zeros on the main diagonal, its determinant $\al=\sum\limits_{\xi\in S_n} (-1)^{\sign(\xi)}\al_{1,\xi(1)}\dots\al_{n,\xi(n)}$ is a sum over all permutations $\xi\in S_n$ such that $k\neq\xi(k)$ for $k=1,\dots,n$, so $\xi$ has no fixed points and the sum $\al$ is over (the rencontre number) $r_n$ permutations $\xi$.
If $n=3$ and $d_{ij}=d_{34}$, then $\al=\left| \begin{array}{ccc} 
0 & 1 & 1 \\
1 & 0 & d_{12}^2 \\
1 & d_{21}^2 & 0  
\end{array}\right|=2d_{12}^2$.
Each product $\al_{1,\xi(1)}\dots\al_{n,\xi(n)}$ has a total degree $2(n-2)$ in distances $d_{kl}$.
After dividing $\al d_{ij}^3$ of degree $2n-1$ by $(p(A))^{2n-1}$,
we use the upper bound $\dfrac{d_{kl}}{p(A)}\leq\dfrac{2}{n}$, which follows from .
Since each product inside $\al d_{ij}^3$, after dividing by $p^{2n-1}(A)$,  has the upper bound $(\frac{2}{n})^{2n-1}$, we get $\left|\dfrac{\al d_{ij}^3}{p^{2n-1}(A)}\right|\leq r_n\left(\dfrac{2}{n}\right)^{2n-1}$.
%\smallskip

The coefficient $\be$ in $\det\hat B=\al d_{ij}^4+\be d_{ij}^2+\ga$ is the sum of products included into two determinants of the submatrices obtained from $\hat B$ by removing the row $i+1$ and column $j+1$ (for one submatrix), then the row $j+1$ and column $i+1$ (for another submatrix).
Since each $(n+1)\times(n+1)$ submatrix includes one entry $d_{ij}^2$, all products with this entry are excluded as there were counted in $\al d_{ij}^4$.
\smallskip

For example, if $n=3$ and $d_{ij}=d_{13}$, then 
$\be=\left| \begin{array}{cccc} 
0 & 1 & 1 & 1  \\
1 & d_{21}^2 & 0 & d_{24}^2 \\
1 & d_{31}^2 & d_{32}^2 & d_{34}^2 \\
1 & d_{41}^2 & d_{43}^2 & 0 
\end{array} \right|+
\left| \begin{array}{cccc} 
0 & 1 & 1 & 1 \\
1 & d_{12}^2 & d_{13}^2 & d_{14}^2 \\
1 & 0 & d_{23}^2 & d_{24}^2 \\
1 & d_{42}^2 & d_{43}^2 & 0 
\end{array} \right|$, where $d_{13}^2=d_{31}^2$ should be replaced by 0, hence
$\be=\left| \begin{array}{cccc} 
0 & 1 & 1 & 1  \\
1 & d_{21}^2 & 0 & d_{24}^2 \\
1 & 0 & d_{32}^2 & d_{34}^2 \\
1 & d_{41}^2 & d_{43}^2 & 0 
\end{array} \right|+
\left| \begin{array}{cccc} 
0 & 1 & 1 & 1 \\
1 & d_{12}^2 & 0 & d_{14}^2 \\
1 & 0 & d_{23}^2 & d_{24}^2 \\
1 & d_{42}^2 & d_{43}^2 & 0 
\end{array} \right|$.
Up to a permutation of indices, each submatrix can be rewritten with the diagonal that has one $d_{ij}^2$, while all other diagonal elements are zeros. 
The example above gives
$\be=-\left| \begin{array}{cccc} 
0 & 1 & 1 & 1  \\
1 & 0 & d_{32}^2 & d_{34}^2 \\
1 & d_{21}^2 & 0 & d_{24}^2 \\
1 & d_{41}^2 & d_{43}^2 & 0 
\end{array} \right|-
\left| \begin{array}{cccc} 
0 & 1 & 1 & 1 \\
1 & 0 & d_{23}^2 & d_{24}^2 \\
1 & d_{12}^2 & 0 & d_{14}^2 \\
1 & d_{42}^2 & d_{43}^2 & 0 
\end{array} \right|$.
Similarly to the argument for the determinant $\al$,
the sum $\be d_{ik}$ contains $2r_{n+1}$ products of total degree $2n-1$, so $\left|\dfrac{\be d_{ik}}{p^{2n-1}(A)}\right|\leq 2r_{n+1}\left(\dfrac{2}{n}\right)^{2n-1}$.
Then the required inequality follows:
$\left|\bpd{\det\hat B}{d_{ij}}\dfrac{1}{p^{2n-1}(A)}\right|=\dfrac{|4\al d_{ij}^3+2\be d_{ij}|}{p^{2n-1}(A)}\leq (4r_n+2r_{n+1}) \left(\dfrac{2}{n}\right)^{2n-1}$.
\end{proof}

\begin{exa}[strength $\si(A)$ and its upper bounds]
\label{exa:strength}
For $n\geq 2$, the proved upper bounds for the Lipschitz constant of the strength: $c_2=2\sqrt{3}$, $c_3\approx 0.43$, $c_4\approx 0.01$, which quickly tend to 0 due to the `curse of dimensionality'. 
The plots in \cite[Fig.~4]{widdowson2023recognizing} illustrate that the strength $\si(A)$ behaves smoothly in the $x$-coordinate of a vertex and its derivative $|\pd{\si}{x}|$ is much smaller than the proved bounds $c_n$ above. 
\bs
\end{exa}

%The strength $\si(A)$ from Definition~\ref{dfn:strength} will take care of extra signs in ORDs and allows us to prove the analogue of Lemma~\ref{lem:RDD+metric} for a similar time complexity with $h=n$.  
%For Lemma~\ref{lem:OCD+metric}, we remind that by Definition~\ref{dfn:SCD} a Oriented Centred Distribution $\OCD$ is a pair $[D(A\cup\{0\});M(C;A\cup\{0\})]$ of matrices considered up to permutations $\xi\in S_{n-1}$ of $n-1$ points of a subset $A\subset C\subset\R^n$, where $0\in\R^n$ is the centre of mass of $C$.
%Any column of $M(C;A\cup\{0\})$ is a pair $(v,s)$, where $s\in\{\pm 1,0\}$ and $v\in\R^n$ is a vector of Euclidean distances from $q\in C-A$ to ordered points $p_1,\dots,p_{n-1}\in A$ and to the origin $0\in\R^n$.

%5=========================
\section{Algorithms for complete invariant metrics}
\label{sec:metrics}

For Lemma~\ref{lem:ORD+metric}, we remind that by Definition~\ref{dfn:OSD} an Oriented Relative Distribution $\ORD$
 is a pair $[D(A);M(C;A)]$ of matrices considered up to permutations $\xi\in S_n$ of $n$ ordered points of $A$.
Any column of $M(C;A)$ is a pair $(v,s)$, where $s\in\{\pm 1,0\}$ and $v\in\R^n$ is a vector of distances from $q\in C-A$ to $p_1,\dots,p_n\in A$.
\smallskip

For simplicity and similar to the case of a general metric space, we assume that a point cloud $C\subset\R^n$ is given by a matrix of pairwise Euclidean distances.
If $C$ is given by Euclidean coordinates of points, then any distance requires $O(n)$ computations and we should add the factor $n$ in all complexities below, keeping all times polynomial in $m$.
\smallskip

The $m-n$ permutable columns of the matrix $M(C;A)$ in $\ORD$ from Definition~\ref{dfn:OSD} can be interpreted as $m-n$ unlabeled points in $\R^n$.
Since any isometry is bijective, the simplest metric respecting bijections is the bottleneck distance (also called the Wasserstein distance $W_{\infty}$). 

\begin{dfn}[bottleneck distance $W_{\infty}$]
\label{dfn:bottleneck}
For any vector $v=(v_1,\dots,v_n)\in \R^n$, the \emph{Minkowski} norm is $||v||_{\infty}=\max\limits_{i=1,\dots,n}|v_i|$.
For any vectors or matrices $N,N'$ of the same size, the \emph{Minkowski} distance is $L_{\infty}(N,N')=\max\limits_{i,j}|N_{ij}-N'_{ij}|$.
For clouds $C,C'\subset\R^n$ of $m$ unlabeled points, 
the \emph{bottleneck distance} $W_{\infty}(C,C')=\inf\limits_{g:C\to C'} \sup\limits_{p\in C}||p-g(p)||_{\infty}$ is minimized over all bijections $g:C\to C'$.
\bs
\end{dfn}

\begin{lem}[metric on $\ORD$s]
\label{lem:ORD+metric}
Using the strength $\si$ from Definition~\ref{dfn:strength}, we 
consider the bottleneck distance $W_{\infty}$ on the set of permutable $m-n$ columns of $M(C;A)$ as on the set of $m-n$ unlabelled points $(v,\frac{s}{c_{n}}\si(A\cup\{q\}))\in\R^{n+1}$.
\smallskip

For another $\ORD'=[D(A');M(C';A')]$ and any permutation $\xi\in S_n$ of indices $1,\dots,n$ acting on $D(A)$ and rows of $M(C;A)$, set $d_o(\xi)=\max\{L_{\infty}(\xi(D(A)),D(A')),W_{\infty}(\xi(M(C;A)),M(C';A')) \}$.
Then $M_\infty(\ORD,\ORD')=\min\limits_{\xi\in S_n} d_o(\xi)$
satisfies all metric axioms on Ordered Relative Distributions ($\ORD$s) and can be computed in time $O(n!(n^2+m^{1.5}\log^{n+1} m))$.
\bs
\end{lem}
\begin{proof}%[Proof of Lemma~\ref{lem:ORD+metric}]
%All arguments follow the proof of Lemma~\ref{lem:RDD+metric} after replacing $R(C;A)$ with $M(C;A)$ for $h=n$.
%The only difference between the matrices $R(C;A)$ and $M(C;A)$ is the extra $(n+1)$-st row of signs $s\in\{\pm 1,0\}$.
The first metric axiom says that $\ORD(C;A),\ORD(C';A')$ are equivalent by Definition~\ref{dfn:OSD} if and only if $M_{\infty}(\ORD(C;A),\ORD(C';A'))=0$ or $d(\xi)=0$ for some permutation $\xi\in S_n$.
Then $d(\xi)=0$ is equivalent to $\xi(D(A))=D(A')$ and $\xi(M(C;A))=M(C';A')$ up to a permutation of columns due to the first axiom for $W_{\infty}$.
The last two conclusions mean that the Oriented Relative Distributions $\ORD(C;A),\ORD(C';A')$ are equivalent by Definition~\ref{dfn:OSD}.
The symmetry axiom follows since any permutation $\xi$ is invertible.
To prove the triangle inequality 
$M_{\infty}(\ORD(C;A),\ORD(C';A'))+$ \\
$M_{\infty}(\ORD(C'';A''),\ORD(C';A'))\geq$ \\ 
$M_{\infty}(\ORD(C;A),\ORD(C'';A''))$, let $\xi,\xi'\in S_h$ be optimal permutations for the $M_\infty$ values in the left-hand side above. 
The triangle inequality for $L_\infty$ says that \\
$L_{\infty}(\xi(D(A)),D(A'))+$ \\
$L_{\infty}(\xi'(D(A'')),D(A'))\geq $ \\
$L_{\infty}(\xi(D(A)),\xi'(D(A'')))=$ \\
$L_{\infty}(\xi'^{-1}\xi(D(A)),D(A''))$, similarly for the bottleneck distance $W_\infty$ from Definition~\ref{dfn:bottleneck}.
Taking the maximum of $L_\infty,W_\infty$ preserves the triangle inequality. 
Then $M_{\infty}(\RDD(C;A),\RDD(C'' ;A''))=\min\limits_{\xi\in S_n}d(\xi)$ cannot be larger than $d(\xi'^{-1}\xi)$ for the composition of the permutations above, the triangle inequality holds for $M_\infty$. 
\smallskip

For a fixed permutation $\xi\in S_{n-1}$, the distance $L_\infty(\xi(D(A)),D(A'))$ requires $O(n^2)$ time.
The bottleneck distance $W_{\infty}(\xi(M(C;A)),M(C';A'))$ on the $(n+1)\times(m-h)$ matrices $\xi(M(C;A))$ and $M(C';A')$ with permutable columns can be considered as the bottleneck distance on clouds of $(m-h)$ unlabeled points in $\R^h$, so $W_\infty(\xi(M(C;A)),M(C';A'))$ needs only $O(m^{1.5}\log^h m)$ time by
\cite[Theorem~6.5]{efrat2001geometry}.
The minimization over all permutations $\xi\in S_n$ gives the factor $n!$ in the final time.
\smallskip

When computing the metric $M_{\infty}$ on $\ORD$s in Lemma~\ref{lem:ORD+metric}, each of the $m-n$ signs for a point $q\in C-A$ is multiplied by the strength $\si(A)$.
The strength $\si(A)$ in Definition~\ref{dfn:strength} is computed from all pairwise distances in $D(A)$ via the Cayley-Menger determinant \cite{sippl1986cayley}.
%\smallskip
One $(n+2)\times(n+2)$ determinant needs time $O(n^3)$ by Gaussian elimination.
All $m-n$ strengths need $O(mn^3)$ time, which is smaller than the later time including the factors $m^{1.5}$ and $n!$
Then $m-n$ permutable columns of $M(C;A)$ are considered as $m-n$ unlabelled points in $\R^{n+1}$, which explains the extra factor $\log m$ coming from \cite[Theorem~6.5]{efrat2001geometry}.
\end{proof}

The coefficient $\frac{1}{c_{n}}$ 
in front of the strength $\si(A\cup\{q\})$ in Lemmas~\ref{lem:ORD+metric} and~\ref{lem:OCD+metric}
normalizes the Lipschitz constant $c_{n}$ of $\si$ to $1$ in line with changes of distances by at most $2\ep$ when points are perturbed within their $\ep$-neighbourhoods.

\begin{lem}[metric on $\OCD$s]
\label{lem:OCD+metric}
Using the strength $\si$ from Definition~\ref{dfn:strength}, we 
consider the bottleneck distance $W_{\infty}$ on the set of permutable $m-n+1$ columns of $M(C;A\cup\{0\})$ as on the set of $m-n+1$ unlabelled points $\left(v,\dfrac{s}{c_{n}}\si(A\cup\{0,q\})\right)\in\R^{n+1}$.
For another $\OCD'=[D(A'\cup\{0\});M(C';A'\cup\{0\})]$ and any permutation $\xi\in S_{n-1}$ of indices $1,\dots,n-1$ acting on $D(A\cup\{0\})$ and the first $n-1$ rows of $M(C;A\cup\{0\})$, set $d_o(\xi)=\max\{L,W\}$, 
$$\text{ where }L=L_{\infty}\Big(\xi(D(A\cup\{0\})),D(A'\cup\{0\})\Big),$$
$$W=W_{\infty}\Big(\xi(M(C;A\cup\{0\})),M(C';A'\cup\{0\})\Big).$$
Then $M_\infty(\OCD,\OCD')=\min\limits_{\xi\in S_{n-1}} d_o(\xi)$
satisfies all metric axioms on Oriented Centred Distributions ($\OCD$s) and is computed in time $O((n-1)!(n^2+m^{1.5}\log^{n} m))$.
\bs
\end{lem}
\begin{proof}[Proof of Lemma~\ref{lem:OCD+metric}]
All arguments follow the proof of Lemma~\ref{lem:ORD+metric}
for a subset $A$ consisting of $n-1$ unordered points of $C$ and one central point at the origin.
Hence $n$ should be replaced with $n-1$.
\end{proof}

The metrics $M_{\infty}$ will be used for intermediate costs to get metrics on unordered collections $\OSD$s and $\SCD$s by using standard Definitions~\ref{dfn:LAC} and~\ref{dfn:EMD} below.  

\begin{dfn}[Linear Assignment Cost LAC {\cite{fredman1987fibonacci}}]
\label{dfn:LAC}
For any $k\times k$ matrix of costs $c(i,j)\geq 0$, $i,j\in\{1,\dots,k\}$, the \emph{Linear Assignment Cost}   
$\LAC=\frac{1}{k}\min\limits_{g}\sum\limits_{i=1}^k c(i,g(i))$ is minimized for all bijections $g$ on the indices $1,\dots,k$.
\bs
\end{dfn}

The normalization factor $\frac{1}{k}$ in $\LAC$ makes this metric better comparable with $\EMD$ whose weights sum up to 1.
  
\begin{dfn}[Earth Mover's Distance on distributions]
\label{dfn:EMD}
Let $B=\{B_1,\dots,B_k\}$ be a finite unordered set of objects with weights $w(B_i)$, $i=1,\dots,k$.
Consider another set $D=\{D_1,\dots,D_l\}$ with weights $w(D_j)$, $j=1,\dots,l$.
Assume that a distance between $B_i,D_j$ is measured by a metric $d(B_i,D_j)$.
A \emph{flow} from $B$ to $D$ is a $k\times l$ matrix  whose entry $f_{ij}\in[0,1]$ represents a partial \emph{flow} from an object $B_i$ to $D_j$.
The \emph{Earth Mover's Distance} \cite{rubner2000earth} is the minimum of
$\EMD(B,D)=\sum\limits_{i=1}^{k} \sum\limits_{j=1}^{l} f_{ij} d(B_i,D_j)$ over $f_{ij}\in[0,1]$ subject to 
%the conditions
$\sum\limits_{j=1}^{l} f_{ij}\leq w(B_i)$ for $i=1,\dots,k$, 
$\sum\limits_{i=1}^{k} f_{ij}\leq w(D_j)$ for $j=1,\dots,l$, and
$\sum\limits_{i=1}^{k}\sum\limits_{j=1}^{l} f_{ij}=1$.
\bs
\end{dfn}

The first condition $\sum\limits_{j=1}^{l} f_{ij}\leq w(B_i)$ means that not more than the weight $w(B_i)$ of the object $B_i$ `flows' into all $D_j$ via the flows $f_{ij}$, $j=1,\dots,l$. 
The second condition $\sum\limits_{i=1}^{k} f_{ij}\leq w(D_j)$ means that all flows $f_{ij}$ from $B_i$ for $i=1,\dots,k$ `flow' to $D_j$ up to its weight $w(D_j)$.
The last condition
$\sum\limits_{i=1}^{k}\sum\limits_{j=1}^{l} f_{ij}=1$ forces all $B_i$ to collectively `flow' into all $D_j$.  
$\LAC$ \cite{fredman1987fibonacci} and $\EMD$ \cite{rubner2000earth} can be computed in a near cubic time in the sizes of given sets of objects. 
\smallskip

An equality $\OSD(C)=\OSD(C')$ between unordered collections of ORDs is best detected by checking if the LAC or EMD metric in Theorem~\ref{thm:OSD_metrics} between these $\OSD$s is 0.

\begin{thm}[times for metrics on $\OSD$s]
\label{thm:OSD_metrics}
\textbf{(a)}
For the $k\times k$ matrix of costs computed by the metric $M_{\infty}$ between $\ORD$s from $\OSD(C)$ and $\OSD(C')$, 
$\LAC$ from Definition~\ref{dfn:LAC} satisfies all metric axioms on $\OSD$s and needs time $O(n!(n^2+m^{1.5}\log^{n+1} m)k^2+ k^3\log k)$.
\medskip

\noindent
\textbf{(b)}
Let $\OSD$s have a maximum size $l\leq k$ after collapsing identical $\ORD$s. Then $\EMD$ from Definition~\ref{dfn:EMD} satisfies all metric axioms  on $\OSD$s and can be computed in time $O(n!(n^2 +m^{1.5}\log^{n+1} m) l^2 +l^3\log l)$.
\bs
\end{thm}
\begin{proof}
The Linear Assignment Cost ($\LAC$) from Definition~\ref{dfn:LAC} is symmetric because any bijective matching can be reversed.
The triangle inequality for $\LAC$ follows from the triangle inequality for the metric $M_{\infty}$ in Lemma~\ref{lem:ORD+metric} by using a composition of bijections $\OSD(C;h)\to\OSD(C';h)\to\SDD(C'';h)$ matching all $\ORD$s similarly to the proof of Lemma~\ref{lem:ORD+metric}.
The first metric axiom for LAC means that $\LAC=0$ if and only if there is a bijection $g:\OSD(C;h)\to\OSD(C';h)$ so that all matched $\ORD$s are at distance $M_\infty=0$, so these $\ORD$s are equivalent (hence $\OSD$s are equal) due to the first axiom of $M_\infty=0$, which was proved in Lemma~\ref{lem:ORD+metric}.
\smallskip

The metric axioms for the Earth Mover's Distance ($\EMD$) are proved in the appendix of \cite{rubner2000earth} assuming the metric axioms for the underlying distance $d$, which is the metric $M_\infty$ from Lemma~\ref{lem:ORD+metric} in our case.
\smallskip

The time complexities for $\LAC$ and $\EMD$ follow from the time $O(n!(n^2 +m^{1.5}\log^n m))$ for $M_\infty$ in Lemma~\ref{lem:ORD+metric}, after multiplying by a quadratic factor for the size of cost matrices and adding a near cubic time \cite{fredman1987fibonacci,goldberg1987solving}.
\end{proof}

An equality $\SCD(C)=\SCD(C')$ is interpreted as a bijection between unordered sets $\SCD(C)\to\SCD(C')$ matching all $\OCD$s, which is best detected by checking if metrics in Theorem~\ref{thm:SCD_metrics} between these $\SCD$s is 0.

\begin{thm}[times for metrics on $\SCD$s]
\label{thm:SCD_metrics}
\textbf{(a)}
For the $k\times k$ matrix of costs computed by the metric $M_{\infty}$ between $\OCD$s in $\SCD(C)$ and $\SCD(C')$, 
$\LAC$ from Definition~\ref{dfn:LAC} satisfies all metric axioms on $\SCD$s and needs time $O((n-1)!(n^2+m^{1.5}\log^{n} m)k^2+ k^3\log k)$.
\medskip

\noindent
\textbf{(b)}
Let $\SCD$s have a maximum size $l\leq k$ after collapsing identical $\OCD$s. Then $\EMD$ from Definition~\ref{dfn:EMD} satisfies all metric axioms  on $\SCD$s and can be computed in time $O((n-1)!(n^2 +m^{1.5}\log^{n} m) l^2 +l^3\log l)$.
\bs
\end{thm}
\begin{proof}[Proof of Theorem~\ref{thm:SCD_metrics}]
All arguments follow the proof of Theorem~\ref{thm:OSD_metrics} after replacing $n$ with $n-1$.
\end{proof}

If we estimate $l\leq k=\binom{m}{n-1}=m(m-1)\dots(m-n+2)/n!$ as $O(m^{n-1}/n!)$, Theorem~\ref{thm:SCD_metrics} gives time 
$O(n(m^{n-1}/n!)^3\log m)$ for metrics on $\SCD$s, which is $O(m^3\log m)$ for $n=2$, and $O(m^6\log m)$ for $n=3$. 
\smallskip

Though the above time estimates are very rough upper bounds, 
the time $O(m^3\log m)$ in $\R^2$ is faster than the only past time $O(m^5\log m)$ for comparing $m$-point clouds by the Hausdorff distance minimized over isometries \cite{chew1997geometric}.
\smallskip

%6=======================
\section{Lipschitz continuity of invariant metrics}
\label{sec:continuity}

\begin{cor}[continuity of $\OSD$ and $\SCD$]
\label{cor:continuity}
For a cloud $C\subset\R^n$ of $m$ unlabelled points, perturbing any point within its $\ep$-neighbourhood changes $\OSD(C)$ and $\SCD(C)$ by at most $2\ep$ in the $\LAC$ and $\EMD$ metrics.
\bs
\end{cor}
\begin{proof}
If every point is perturbed in its $\ep$-neighbourhood, any distance between points changes by at most $2\ep$.
This upper bound survives under the metrics $M_\infty$, $\LAC$, $\EMD$.
\smallskip

Theorem~\ref{thm:strength} was essential to justify a Lipschitz constant $c_n$ of a strength $\si(A)$ so that the last coordinates $\frac{s}{c_n}\si(A\cup\{q\})$ change by at most $2\ep$ when the bottleneck distance $W_{\infty}$ is computed on columns in Lemmas~\ref{lem:ORD+metric} and~\ref{lem:OCD+metric}. 
\end{proof}

%Definition~\ref{dfn:SDM} introduced the \emph{Ordered Pointwise Distances} $\SDV(A)$ as the list of all $\frac{h(h-1)}{2}$ pairwise distances between points of a cloud $A$ written in increasing order. 

\begin{dfn}[Oriented Distance Moments $\ODM$]
\label{dfn:ODM}
For any $m$-point cloud $C\subset\R^n$, let $A\subset C$ be a subset of $n$ unordered points.
The \emph{Sorted Distance Vector} $\SDV(A)$ is the list of all $\frac{h(h-1)}{2}$ pairwise distances between points of $A$ written in increasing order. 
For each column of the $(n+1)\times(m-n)$ matrix $M(C;A)$ in Definition~\ref{dfn:SCD}, compute the average of the first $n$ distances.
Write these averages in increasing order and append the vector of increasing values of $\dfrac{s}{c_n}\si(A)$ taking signs $s$ from the $(n+1)$-st row of $M(C;A)$.
Let $\vec M(C;A)\in\R^{2(m-n)}$ be the final vector.
\smallskip

The pair $[\SDV(A);\vec M(C;A)]$ is a vector of length $\frac{n(n-1)}{2}+2(m-n)$ called the \emph{Average Oriented Vector} $\AOV(C;A)$.
The unordered set of vectors $\AOV(C;A)$ for all $\binom{m}{n}$ unordered subsets $A\subset C$ is the \emph{Average Oriented Distribution} $\AOD(C)$.
For $l\geq 1$, the \emph{Oriented Distance Moment} $\ODM(C;l)$ is the $l$-th (standardized for $l\geq 3$) moment of $\AOD(C)$ considered as a probability distribution of $\binom{m}{n}$ vectors, separately for each coordinate.
\bs
\end{dfn}

\begin{exa}[$\ODM$ for clouds in Fig.~\ref{fig:4-point_clouds}]
\label{exa:ODM}
For the non-isometric 4-point clouds $T,K$ in Fig.~\ref{fig:4-point_clouds}, Table~\ref{tab:ODM+TK} summarizes the computations of the Average Oriented Distributions ($\AOD$s) from Definition~\ref{dfn:ODM}.
The values $\dfrac{s}{c_2}\si(A)$ involving the strength of a triangle on a 3-point subset $A$ use the Lipschitz constant $c_2=2\sqrt{3}$.
We compute the normalized strengths $\dfrac{\si(A)}{c_2}$ using $\si(A)=\dfrac{V^2(A)}{p^3(A)}$, see Definition~\ref{dfn:strength}, where $V(A)$ is the area and $p(A)$ is the half-perimeter of the triangle on $A$.

\begin{table*}[h!]
  \centering
  \begin{tabular}{@{}l|l@{}}
    \toprule
    Average Oriented Vectors in $\AOD(T)$ & Average Oriented Vectors in $\AOD(K)$ \\
    \midrule
   
   $\left[\sqrt{2};\; \dfrac{2+\cbox{yellow}{\sqrt{10}}}{2},\; \dfrac{4+\sqrt{10}}{2}; \; \cbox{yellow}{-\si_2,\; -\si_1(T)} \right] \quad\approx$ &
	$\left[\sqrt{2};\dfrac{2+\cbox{yellow}{\sqrt{2}}}{2},\dfrac{4+\sqrt{10}}{2}; \cbox{yellow}{-\si_1(K), -\si_2} \right] \quad\approx$
	\smallskip	\\
 $[\quad 1.414; \quad \cbox{yellow}{2.581}, \quad 3.581; \quad \cbox{yellow}{-0.015, \quad  -0.008} \quad ]$ &     
  $[\quad 1.414; \quad \cbox{yellow}{1.707}, \quad 3.581; \quad \cbox{yellow}{-0.021, \quad  -0.015} \quad ]$  
\bigskip \\   \midrule
  
   $\left[\sqrt{2};\dfrac{2+\cbox{yellow}{\sqrt{10}}}{2},\dfrac{4+\sqrt{10}}{2}; \cbox{yellow}{\si_1(T), \si_2} \right] \quad\approx$ &
	$\left[\sqrt{2};\dfrac{2+\cbox{yellow}{\sqrt{2}}}{2},\dfrac{4+\sqrt{10}}{2}; \cbox{yellow}{\si_2, \si_1(K)} \right] \quad\approx$
	\smallskip	\\
 $[\quad 1.414; \quad \cbox{yellow}{2.581}, \quad 3.581; \quad \cbox{yellow}{0.008, \quad  0.015} \quad ]$ &     
  $[\quad 1.414; \quad \cbox{yellow}{1.707}, \quad 3.581; \quad \cbox{yellow}{0.015, \quad  0.021} \quad ]$  
\bigskip \\   \midrule

   $\left[2;\cbox{yellow}{\dfrac{\sqrt{2}+\sqrt{10}}{2},\dfrac{\sqrt{2}+\sqrt{10}}{2}; -\si_1(T),  -\si_1(T)} \right]$ &
	$\left[2; \cbox{yellow}{\sqrt{2}, \sqrt{10}; -\si_1(K), \si_3} \right]$
	\smallskip	\\
 $\approx[\quad 2; \quad \cbox{yellow}{2.581, \quad 2.581; \quad -0.008, \quad -0.008} \quad ]$ &     
  $\approx[\quad 2; \quad \cbox{yellow}{1.414, \quad 3.162; \quad -0.021, \quad 0.036} \quad ]$  
\bigskip \\   \midrule

   $\left[\sqrt{10}; \cbox{yellow}{\dfrac{2+\sqrt{2}}{2}}, \dfrac{4+\sqrt{2}}{2}; \cbox{yellow}{-\si_2, \si_1(T) } \right] \quad\approx$ &
	$\left[\sqrt{10}; \cbox{yellow}{\dfrac{2+\sqrt{10}}{2}}, \dfrac{4+\sqrt{2}}{2}; \cbox{yellow}{-\si_3, -\si_2} \right] \quad\approx$
	\smallskip	\\
 $[\quad 3.162; \quad \cbox{yellow}{1.707}, \quad 2.707; \quad \cbox{yellow}{-0.015, \quad 0.008}  \quad ]$ &     
  $[\quad 3.162; \quad \cbox{yellow}{2.581}, \quad 2.707; \quad \cbox{yellow}{-0.021, \quad  -0.015} \quad ]$  
\bigskip \\   \midrule
    
   $\left[\sqrt{10}; \cbox{yellow}{\dfrac{2+\sqrt{2}}{2}},\dfrac{4+\sqrt{2}}{2}; \cbox{yellow}{-\si_1(T), \si_2 } \right] \quad\approx$ &
	$\left[\sqrt{10}; \cbox{yellow}{\dfrac{2+\sqrt{10}}{2}}, \dfrac{4+\sqrt{2}}{2}; \cbox{yellow}{\si_2, \si_3} \right] \quad\approx$
	\smallskip	\\
 $[\quad 3.162; \quad \cbox{yellow}{1.707}, \quad 2.707; \quad \cbox{yellow}{-0.008, \quad 0.015}  \quad ]$ &     
  $[\quad 3.162; \quad \cbox{yellow}{2.581}, \quad 2.707; \quad \cbox{yellow}{0.015, \quad  0.036} \quad ]$  
\bigskip \\   \midrule
	
   $\left[4; \dfrac{\sqrt{2}+\sqrt{10}}{2},\dfrac{\sqrt{2}+\sqrt{10}}{2}; \cbox{yellow}{\si_2}, \si_2 \right] \quad\approx$ &
	$\left[4; \dfrac{\sqrt{2}+\sqrt{10}}{2},\dfrac{\sqrt{2}+\sqrt{10}}{2}; \cbox{yellow}{-\si_2}, \si_2 \right] \quad\approx$
	\smallskip	\\
 $[\quad 4; \quad 2.288, \quad 2.288; \quad \cbox{yellow}{0.015}, \quad 0.015 \quad ]$ &     
  $[\quad 4; \quad 2.288, \quad 2.288; \quad \cbox{yellow}{-0.015}, \quad  0.015 \quad ]$  
\bigskip \\  
    \bottomrule
  \end{tabular}
  \caption{The Averaged Oriented Distributions (AODs) from Definition~\ref{dfn:ODM} for the clouds $T,K\subset\R^2$ in Fig.~\ref{fig:4-point_clouds} consist of six vectors in $\R^5$. 
 The coordinate-wise averages of the six vectors in $\AOD$ give
  the Oriented Distance Moment $\CDM\in\R^5$ in Table~\ref{tab:ODM+TK}, see Example~\ref{exa:ODM}.}
  \label{tab:AOD+TK}
\end{table*}

%\smallskip

The trapezoid $T$ has two triangles with sides $\sqrt{2},2,\sqrt{10}$, area $1$, half-perimeter $1+\dfrac{\sqrt{2}+\sqrt{10}}{2}$, and normalized strength $\si_1(T)=\dfrac{1}{2\sqrt{3}}\left(\dfrac{2}{\sqrt{2}+2+\sqrt{10}}\right)^3\approx 0.008$.
\smallskip

The kite $K$ has a triangle with sides $\sqrt{2},\sqrt{2},2$, area $1$, half-perimeter $1+\sqrt{2}$, and normalized strength $\si_1(K)=\dfrac{1}{2\sqrt{3}(1+\sqrt{2})^3}\approx 0.021$.
\smallskip

Both $T,K$ have triangles with sides $\sqrt{2},\sqrt{10},4$, area 2, half-perimeter $2+\dfrac{\sqrt{2}+\sqrt{10}}{2}$, and normalized strength
$\si_2=\dfrac{2}{\sqrt{3}}\left(\dfrac{2}{\sqrt{2}+4+\sqrt{10}}\right)^3\approx 0.015$.
\smallskip

The kite $K$ has a triangle with sides $2,\sqrt{10},\sqrt{10}$, area $3$, half-perimeter $1+\sqrt{10}$ and normalized strength $\si_3=\dfrac{3^2}{2\sqrt{3}}\left(\dfrac{1}{1+\sqrt{10}}\right)^3=\dfrac{3\sqrt{3}}{2(1+\sqrt{10})^3}\approx 0.036$.
\smallskip

Every Average Oriented Vector $\AOV(C;A)\in\R^5$ in Table~\ref{tab:AOD+TK} consists of $1+2+2$ coordinates.
The first coordinate is the distance between the points of $A$.
The two further coordinates are in increasing order.
The two last coordinates are in increasing order, independently of the previous pair.
\smallskip

\begin{table*}[h!]
  \centering
  \begin{tabular}{@{}l|l@{}}
    \toprule
    Oriented Distance Moment $\ODM(T;1)$ & Oriented Distance Moment $\ODM(K;1)$ \\
    \midrule
   	$\ODM_1=\dfrac{3+\sqrt{2}+\sqrt{10}}{3}\approx 2.525$ & 
   	$\ODM_1=\dfrac{3+\sqrt{2}+\sqrt{10}}{3}\approx 2.525$ \\

   	$\ODM_2=\dfrac{\cbox{yellow}{6+2\sqrt{2}+4\sqrt{10}}}{12}\approx \cbox{yellow}{1.790}$ & 
   	$\ODM_2=\dfrac{\cbox{yellow}{8+5\sqrt{2}+3\sqrt{10}}}{12}\approx \cbox{yellow}{2.046}$ \\

   	$\ODM_3=\dfrac{16+\cbox{yellow}{4\sqrt{2}+4\sqrt{10}}}{12}\approx \cbox{yellow}{2.859}$ & 
   	$\ODM_3=\dfrac{16+\cbox{yellow}{3\sqrt{2}+5\sqrt{10}}}{12}\approx \cbox{yellow}{3.005}$ \\

   	$\ODM_4=\cbox{yellow}{-\dfrac{\si_1(T)+\si_2}{6}\approx -0.004}$ & 
   	$\ODM_4=\cbox{yellow}{\dfrac{\si_2-\si_1(K)}{6}\approx -0.001}$ \\

   	$\ODM_5=\cbox{yellow}{\dfrac{3\si_2-\si_1(T)}{6}\approx 0.006}$ & 
   	$\ODM_5=\cbox{yellow}{\dfrac{2\si_3+\si_1(K)-\si_2}{6}\approx 0.013}$ \\
 	
    \bottomrule
  \end{tabular}
  \caption{
  The Oriented Distance Moments (ODMs) from Definition~\ref{dfn:ODM} for the 4-point clouds $T,K\subset\R^2$ in Fig.~\ref{fig:4-point_clouds}
  are obtained by averaging the six vectors from the Average Oriented Distributions in Table~\ref{tab:AOD+TK}.
%  The first three coordinates coincide with $\SDM$ in Table~\ref{tab:SDD+TK}.
}
\label{tab:ODM+TK}
\end{table*}

Table~\ref{tab:ODM+TK} shows the Oriented Distance Moments $\ODM(C;1)$ obtained by coordinate-wise averaging all vectors in the Average Oriented Distribution $\AOD(C)$.
%Since $\AOV$ differs from the Average Distance Distribution $\ADD$ in Definition~\ref{dfn:SDM} by two extra values involving signs and strengths, the first three coordinates of $\ODM(C;1)$ coincide with $\SDM(C;2,1)$, see Table~\ref{tab:SDD+TK}.
\smallskip

The distance $L_\infty$ between the vectors $\ODM(T;1)$ and $\ODM(K;1)$ in $\R^5$ is 
$\dfrac{8+5\sqrt{2}+3\sqrt{10}}{12}-\dfrac{6+2\sqrt{2}+4\sqrt{10}}{12}=\dfrac{2+3\sqrt{2}-\sqrt{10}}{12}\approx 0.257$.
\bs
\end{exa}
  
\begin{cor}[time and lower bound for a metric on $\ODM$s]
\label{cor:ODM}
\textbf{(a)}
For any cloud $C\subset\R^n$ of $m$ unlabelled points,
the Oriented Distance Moment $\ODM(C;l)$ in Definition~\ref{dfn:ODM} is computed in time $O(m^{n+1}/(n-3)!)$.
\medskip

\noindent
\textbf{(b)}
The metric $L_\infty$ on $\ODM$s needs $O(n^2+m)$ time and provides the lower bound 
$\EMD\big(\OSD(C),\OSD(C')\big)\geq|\ODM(C;1)-\ODM(C';1)|_\infty$. %holds.
\bs
\end{cor}
\begin{proof}
\textbf{(a)}
For any $n$-point subset $A\subset C$ and a fixed point $q\in C-A$,
the average distance from $q$ to the $n$ points of $A$ written in a column of $M(C;A)$ is computed in time $O(n)$, hence $O(nm)$ for all columns of $M(C;A)$ (or all points $q\in C-A$).
Ordering the distance averages and (separately) $m-n$ values $\dfrac{s}{c_n}\si(A)$ takes $O(m\log m)$ time.
\smallskip

So the vector $\vec M(C;A)\in\R^{2(m-n)}$ from Definition~\ref{dfn:ODM} takes $O(nm+m\log m)$ time.  
The list $\SDV(A)$ of Ordered Pairwise Distances is obtained by sorting all pairwise distances from $D(A)$ in time $O(n^2\log n)$.
The Average Oriented Vector $\AOV(C;A)$ is obtained by concatenating of the vectors $\SDV(A)\in\R^{\frac{n(n-1)}{2}}$ and $\vec M(C;A)\in\R^{2(m-n)}$ and is computed in time $O((n^2+m)\log m)$.
\smallskip

Hence the Average Oriented Distribution $\AOD(C;h)$ for all $h$-point subsets $A\subset C$ needs $O(m^{n+1}/(n-3)!)$ time including the extra time $O((n^2+m)\log m)$ above, the same as the initial invariant  $\OSD(C)$ in Theorem~\ref{thm:OSD_complete}.
\smallskip

For $l=1$, the first raw moment $\ODM(C;1)$ is the simple average of all $k=\binom{m}{n}$ vectors $\AOV(C;A)$ of length $O(nm)$, hence needs time $O(knm)=O(m^{n+1}/(n-1)!)$.
For $l=2$, the standard deviation $\si$ of each coordinate in all vectors $\AOV(C;A)$ requires the same time.
Then, for any fixed $l\geq 3$, the $l$-th standardized moment  
 $\dfrac{1}{k}\sum\limits_{i=1}^k \left(\dfrac{a_i-\mu}{\si}\right)^l$
 needs again the same time $O(m^{n+1}/(n-1)!)$.
\medskip

\noindent
\textbf{(b)}
For a fixed $l\geq 1$, the vector $\ODM(C;l)$ has the length $\frac{n(n-1)}{2}+2(m-n)$.
Computing the metric $L_{\infty}$ on such vectors needs only $O(n^2+m)$ time.
The lower bound of $\EMD\big(\OSD(C),\OSD(C')\big)$ follows from \cite[Theorem~1]{cohen1997earth} about averages of distributions.
The only extra addition is the $(n+1)$-st row of values $\dfrac{s}{c_n}\si(A)$ in the matrices $M(C;A)$.
For $\EMD$ on $\OSD$s, the bottleneck distance $W_{\infty}$ on columns of $M(C;A)$ takes the maximum absolute difference of the values for $C$ and its perturbation $C'$.
\smallskip

The moment $\ODM(C;1)$ involves averaging these values for $n$-point subsets $A\subset C$. 
The metric $L_\infty$ on resulting averages cannot be larger than the maximum absolute difference computed by the bottleneck distance $W_{\infty}$.
\end{proof}

%Theorem~\ref{thm:OSD}(d) and  Corollary~\ref{cor:ODM} imply that $\ODM(C;1)$ is continuous under perturbations of a point cloud $C\subset\R^n$ with the Lipschitz constant $2$.

\begin{cor}[time and lower bound for a metric on $\CDM$s]
\label{cor:CDM}
\textbf{(a)}
For any cloud $C\subset\R^n$ of $m$ unlabelled points,
the Centred Distance Moment $\CDM(C;l)$ in Definition~\ref{dfn:CDM} is computed in time $O(m^n/(n-4)!)$.
\medskip

\noindent
\textbf{(b)}
The metric $L_\infty$ on $\CDM$s needs $O(n^2+m)$ time and %provides the lower bound 
$\EMD\big(\SCD(C),\SCD(C')\big)\geq|\CDM(C;1)-\CDM(C';1)|_\infty$ holds.
\bs
\end{cor}
\begin{proof}[Proof of Corollary~\ref{cor:CDM}]
All arguments follow the proof of Corollary~\ref{cor:ODM} after replacing $n$ with $n-1$.
\end{proof}

Corollaries~\ref{cor:continuity},~\ref{cor:ODM},~\ref{cor:CDM} imply that $\ODM(C;1)$ and $\CDM(C;1)$ are continuous under perturbations of a point cloud $C\subset\R^n$ with the Lipschitz constant $2$.

%In general position, by Lemma~\ref{lem:ORD+reconstruction} any $\OSD$ consist of distinct $\ORD$s and may not be reduced to a smaller size by collapsing identical $\ORD$s. Since $\EMD$ may not have extra benefit in general, we use only the $\LAC$ metric below.
%Definition~\ref{dfn:CDM} adapts the moments of distributions from Definition~\ref{dfn:SDM} to $\SCD$ and introduces invariant vectors that can be compared by many metrics much faster than $\SCD$s.  

\begin{dfn}[Centered Distance Moments $\CDM$]
\label{dfn:CDM}
For any $m$-point cloud $C\subset\R^n$,
 let $A\subset C$ be a subset of $n-1$ unordered points.
The Sorted Distance Vector $\SDV(A;0)$ is the increasing list of all $\frac{(n-1)(n-2)}{2}$ pairwise distances between points of $A$, followed by $n-1$ increasing distances from $A$ to the origin $0$. 
For each column of the $(n+1)\times(m-n+1)$ matrix $M(C;A\cup\{0\})$ in Definition~\ref{dfn:SCD}, compute the average of the first $n-1$ distances.
Write these averages in increasing order, append the list of increasing distances $|q-0|$ from the $n$-th row of $M(C;A\cup\{0\})$, and also append the vector of increasing values of $\dfrac{s}{c_n}\si(A\cup\{0\})$ taking signs $s$ from the $(n+1)$-st row of $M(C;A\cup\{0\})$.
Let $\vec M(C;A)\in\R^{3(m-n+1)}$ be the final vector.
\smallskip

The pair $[\SDV(A;0);\vec M(C;A)]$ is the \emph{Average Centred Vector} $\ACV(C;A)$ considered as a vector of length $\frac{n(n-1)}{2}+3(m-n+1)$.
The unordered set of $\ACV(C;A)$ for all $\binom{m}{n-1}$ unordered subsets $A\subset C$ is the Average Centred Distribution $\ACD(C)$.
%For $l\geq 1$, 
The \emph{Centered Distance Moment} $\CDM(C;l)$ is the $l$-th (standardized for $l\geq 3$) moment of $\ACD(C)$ considered as a probability distribution of $\binom{m}{n-1}$ vectors, separately for each coordinate.
\bs
\end{dfn}

\begin{exa}[$\CDM$ for clouds in Fig.~\ref{fig:triangular_clouds}]
\textbf{(a)}
For $n=2$ and the cloud $R\subset\R^2$ of $m=3$ vertices $p_1=(0,0)$, $p_2=(4,0)$, $p_3=(0,3)$ of the right-angled triangle in Fig.~\ref{fig:triangular_clouds}~(middle), we continue Example~\ref{exa:SCD}(a) and flatten $\OCD(R;p_1)=[0,\left( \begin{array}{cc} 
4 & 3 \\
4 & 3 \\
0 & 0
\end{array}\right) ]$ into the vector $\ACV(R;p_1)=[0;3,4;3,4;0,0]$ of length $\frac{n(n-1)}{2}+3(m-n+1)=7$, whose four parts ($1+2+2+2=7$) are in increasing order, similarly for $p_2,p_3$.
The Average Centred Distribution can be written as a $3\times 7$ matrix with unordered rows: $\ACD(R)=\left( \begin{array}{c|cc|cc|cc} 
0 & 3 & 4 & 3 & 4 & 0 & 0 \\
4 & 4 & 5 & 0 & 3 & 0 & -6/c_2 \\
3 & 3 & 5 & 0 & 4 & 0 & 6/c_2
\end{array}\right)$.
The area of the triangle on $R$ equals $6$ and can be normalized by $c_2=2\sqrt{3}$ (see the appendices) to get $6/c_2=\sqrt{3}$.
The 1st moment is $\CDM(R;1)=\frac{1}{3}(7;10,14;3,11;0)$.
\medskip

\noindent
\textbf{(b)}
For $n=2$ and the cloud $S\subset\R^2$ of $m=4$ vertices 
of the square in Fig.~\ref{fig:triangular_clouds}~(right), 
Example~\ref{exa:SCD}(a) computed $\SCD(R)$ as one
$\OCD=[1,\left(\begin{array}{ccc} 
\sqrt{2} & \sqrt{2} & 2 \\
1 & 1 & 1 \\
- & + & 0
\end{array}\right)]$, which flattens to 
$\ACV=(1; \sqrt{2}, \sqrt{2}, 2; 1,1,1; -\frac{1}{2}, \frac{1}{2},0)=\ACD(S)=\CDM(S;1)\in\R^{10}$, where $\frac{1}{2}$ is the area of the triangle on the vertices $(0,0),(1,0),(0,1)$.
\bs 
\end{exa}

New invariants in Definitions~\ref{dfn:OSD},~\ref{dfn:SCD} and main Theorems~\ref{thm:SCD_complete},~\ref{thm:OSD_metrics}, and Corollary~\ref{cor:continuity} solve Problem~\ref{pro:isometry}.
\smallskip

Computations of invariants on atomic clouds from \cite[section~5]{widdowson2023recognizing} will be substantially expanded in future work.
\smallskip

This research was supported by the Royal Academy of Engineering fellowship ``Data science for next generation engineering of solid crystalline materials'' (2021-2023, IF2122/186) and the EPSRC grants ``Application-driven Topological Data Analysis'' (2018-2023,  EP/R018472/1) and ``Inverse design of periodic crystals'' (2022-2024, EP/X018474/1).
\smallskip

The author thanks all members of the Data Science Theory and Applications group in the Materials Innovation Factory (Liverpool, UK), especially Daniel Widdowson, Matthew Bright, Yury Elkin, Olga Anosova, also Justin Solomon (MIT), Steven Gortler (Harvard),  Nadav Dym (Technion) for fruitful discussions, and any reviewers for their valuable time and helpful suggestions.

{\small
\bibliographystyle{ieee_fullname}
\bibliography{SCD}
}

\end{document}
