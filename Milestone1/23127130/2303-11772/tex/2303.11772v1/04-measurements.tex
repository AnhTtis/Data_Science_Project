\section{Measurements of ROV-Enforcement}\label{sc:measurements}
In this section we explain the setup and experiments.
\begin{figure}[t!]
    \centering
    \includegraphics[width=0.42\textwidth]{figures_pdf/types.pdf}%width=\columnwidth
    \caption[Type Distribution in Data-Plane]%
    {Observed AS types in control-plane and data-plane. The relative percentage of Tier-1 ISPs is lower for the data-plane because of a substantially higher total AS amount.}
    \label{fig:as_types}
\end{figure}
\subsection{Control-Plane} On the control-plane, we carry out prefix hijacks of prefixes we own, and we use public collectors to monitor the propagation of the valid and invalid BGP announcements in the Internet.

{\bf Setup.} For our control plane measurements, we set up three origin servers, two servers by the Internet provider IBM, located in Sao Paolo (Brazil) and Tokyo (Japan), and one in a scientific institution in Germany. Both servers by IBM are assigned the AS number 212795, and the research institution server receives the AS number 208162. We use these servers to issue alternating valid and invalid BGP announcements. We also create corresponding ROAs, some valid and some conflicting with the BGP announcements. The ROAs are published in our RPKI repository.  

{\bf Monitoring.} To monitor the propagation of our BGP announcements on the control-plane, we use data from route collectors by Routeviews \cite{routeviews} and the RIPE Routing Information Service (RIS) \cite{RIS}. The collectors are BGP-speaking routers that aggregate BGP messages from peers at their respective locations and publish the collected data on the Internet. We download the data from these Vantage Points (VPs) in the form of Multi-Threaded Routing Toolkit (MRT) BGP Table dumps during the measurements and filter it for paths that originate in one of our measurement ASes. 

 \begin{figure}[t!]
    \centering
    \includegraphics[width=0.42\textwidth]{figures_pdf/probe_distr.pdf}
    \caption[Probe Origin Distribution]%
    {Distribution of BGP collectors and Atlas probes over continents. The Atlas probes are biased towards Europe.}
    \vspace{-10pt}
    \label{fig:atlas_probes}
\end{figure}

%%% VISIBILITY OF THE COLLECTORS
Mapping the collectors to geo-locations, we find that the control-plane collectors are more evenly distributed over continents than the data-plane probes. We also observe that the control-plane collectors have a high presence in North America and Europe, while only a minor part is located on other continents. The distribution of the collectors is shown in comparison to the data-plane probes in Figure \ref{fig:atlas_probes}. To maximize the amount of collected data, we use all available data in both control-plane and data-plane. To understand the bias in the results from a different distribution of collectors, we present the results according to regions.
The collectors in the control-plane observe an absolute amount of 1566 paths, with 797 paths to AS212795 and 769 paths to AS208162.

{\bf Experiments.} The prefix hijacks for the ROV measurements are run on two different days, the 8th and 10th of June 2022, to increase the robustness against short-lived unexpected routing events.
All servers in our setup announce two neighboring prefixes throughout both measurements, \textit{P1} 45.155.129.0/24 and \textit{P2} 45.155.131.0/24, with their assigned AS number as the origin.

In the first configuration, a ROA is issued for AS212795 - \textit{P1} and for AS208162 - \textit{P2}. The ROAs are published 24 hours before starting the measurements. After the measurements are finished, the ROAs are withdrawn, and new ROAs are published with the inverse configuration, validating AS212795 to announce \textit{P2} and AS208162 to announce \textit{P1}. The second measurement is run six hours after the configuration change to give enough propagation time for the updated ROAs. The second measurement starts with the inverse configuration, where the ROA was published 24 hours before starting the run, preventing a one-sided bias in the results. After the measurement, the configuration is again reversed, and the final run is conducted six hours later. 

\subsection{Data-Plane} 
After executing the control-plane experiment and monitoring the propagation of the BGP announcements, we check the traffic paths on the data-plane. This analysis requires a coherent structure in the results of control-plane and data-plane. To achieve this consistency, we use Traceroute packets that probe data-plane paths through the network, resulting in a similar path structure to control-plane AS-paths. IP addresses on Traceroute paths are mapped according to the CAIDA AS and IXP mapping \cite{caidaASMapping}.

{\bf Setup.} To obtain a global distribution of origins for Traceroute measurements, we use probes by the RIPE Atlas project.

{\bf Experiments.} The data-plane measurements require a multi-step pipeline for executing the measurements, acquiring the raw data, and applying classification to the results. Measurements are started over the RIPE Atlas API. RIPE Atlas limits the measurements to 1000 probes per experiment. We start four separate Traceroute measurements per execution from 1000 random global Atlas probes, each running to both our prefixes. Experiment IDs are logged to ensure that the measurements are started from identical probes for the inverse control-plane configuration over the Atlas API.
Probes that go out of service during the measurements and thus do not complete a measurement in both configurations are removed from the results. Each measurement is run with a one-minute time difference between requests.

{\bf Processing and analysis.} We process the paths obtained from Traceroutes to remove redundant information and to discard paths that only contain unresponsive hops or originated from a probe that did not complete measurements to both announced prefixes. First, the measurement results are downloaded over the Atlas API, serialized, and receive a unique identifier that is inserted into a local database for processing. The raw data processing starts with a majority vote on each IP hop; Atlas probes run three separate Traceroutes per measurement. If no consensus between the hops can be found, the hop is added with a non-value. In the second processing step, IP addresses are mapped to AS numbers according to the CAIDA datasets \cite{caidaASMapping, caidaIXPMapping}. Traceroute logs the address of the replying interface, which may not always be correctly configured; we observed many internal IP addresses or IP addresses that could not be mapped to an AS number. These ASes are added as a non-responsive hop. Further, to prevent confusion between AS numbers and the arbitrary IDs of IXPs in the dataset, which may overlap, IXP IDs are added with a negative sign. 

The paths then need to be pre-processed to increase the robustness of classification. Consecutive hops of the same AS do not provide additional information and are thus condensed into a single hop for classification. 
None-hops are removed if the previous and following AS are identical, as the hop likely belongs to the same AS as the surrounding hops.

The data processing on the data-plane results in 18520 valid paths with 73481 valid hops and 8489 unresponsive hops. The measurements have a similar amount of paths to both our ASes, with 8286 paths to AS212795 and 7608 paths to AS208162; 2626 paths did not reach one of the target ASes.

{\bf Classification of ASes.} Applying the classification scheme from Section \ref{sc:classification} on the paths is a three-step process. First, the measurements from each probe are iterated and correlated, i.e., the paths to both prefixes in a single measurement run are processed together. Processing checks which Traceroute target follows the ROA. This process is not straightforward, as servers inside the target AS might not reply to the Traceroute. Thus we map the AS numbers of the upstream providers of our targets to the final destination. If one of the paths flows to an invalid prefix, all ASes on the path are classified as occurring on an invalid path. On the other hand, all ASes to a valid prefix receive a point of evidence for a correct path. If the paths to both prefixes follow the ROA, the divergence point between the paths is calculated, and the corresponding AS receives a classification as a divergence point. 

In the second processing step, each AS is analyzed according to its valid vs. invalid paths and divergence points. Then the classification scheme is applied. ASes are stored together with their final classification for analysis. In the last step, results for category 5 are calculated as they require information about upstream ROV enforcement. Processing iterates all ASes classified into category 4 and analyzes upstream ASes. If each path over the AS runs over an AS that is classified into category 6 or 7 before reaching one of the prefixes, the classification is changed to category 5. 


\subsection{Control- vs. Data-Plane}
\label{subsec:cd}
The comparison of control-plane and data-plane measurement points plotted in Figure \ref{fig:atlas_probes} indicates that the measurements have a slightly different view of the Internet, as the control-plane has a higher percentage of points in North America while the data-plane measurements primarily originate in Europe. Additionally, the analysis of the raw data shows a different distribution of AS types between the measurements in the control and the data-plane, with a higher percentage of stub-ASes and IXPs in the data-plane, illustrated in Figure \ref{fig:as_types}.
The difference stems from the observation mode; the data-plane can observe stub-ASes because some of the stubs host an Atlas probe and are thus visible, even though they do not forward traffic. IXPs are visible because routers in their peering LAN reply to ICMP messages, even if they do not append to BGP AS paths. In contrast, IXPs and stub ASes are not visible on the control-plane. Stub ASes are not visible on the control-plane because they do not forward traffic to our destination AS. IXPs do not append their number to the control-plane path, and hence are not visible on the control plane.

