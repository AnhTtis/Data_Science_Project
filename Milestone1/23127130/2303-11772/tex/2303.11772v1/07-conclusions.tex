\ignore{
\section{Future Deployment and Research}
The results raise the question of the future direction that deployment of ROV should take to reach the most efficient protection of the Internet. Counter-intuitively, routeserver ROV can not solve the problem of origin hijacks on today's Internet. The high prevalence of direct peering sessions limits the capability of IXPs to mitigate the spread of malicious updates. Further, even without the prevalence of direct peering sessions, IXPs only provide localized protection against hijacks. The burden of protecting large parts of the routing architecture lies on the large ISPs and Tier-1 providers.

ROV implementation in Tier-1 providers greatly benefits the security of the Internet as it limits the spread of hijacks to a localized scope. Thus, it is vital that the remaining seven Tier-1 providers without strict enforcement terminate the propagation of invalid updates. The implementation in Tier-1 providers should be supplemented by ROV implementation in Tier-2 ISPs and other large systems, further limiting the reach and connectivity for attacks.

Implementing ROV in more routeservers is also a crucial step towards a safer Internet. It provides the last link in the chain of protection for the Internet by extending local protection to a large amount of ASes. The combination of global protection from large providers with localized protection by IXPs can lead to a substantial amount of protected ASes without the need to implement ROV in every system.

Nevertheless, ASes should be aware that direct peering sessions at IXPs circumvent the ROV protection IXPs offer. Operators should consider moving peering sessions that do not require fine-grained control over route selection to the routeserver or implement ROV themselves. 

The results of this work motivate multiple open questions for future research.

The measurements showed that ROV implementation on routeservers does not play a central role in protecting the Internet from prefix hijacks. This result was obtained by an upper-bound estimation of paths over the routeserver, which proved that even in the best-case scenario, the protection of routeserver ROV is primarily local and does not reliably protect wider parts of the Internet. This result should be explored further in future research with measurements that can better estimate routeserver connections and their impact on invalid path propagation. 

Future research should also look into the deployment of RP clients inside ROV enforcing ASes. The measurements revealed a high number of ASes that exhibit clear signs of ROV enforcement without an RP client requesting from their IP space. Investigating this observation with further measurements might give valuable insights into the interplay of the different parts of ROV and their usage in real-world ASes. 

The measurements also found several ASes running multiple RP clients from their address space. Future research should explore the decision processes between those clients and potential changes in ROV enforcement if one or multiple clients either fail or provide conflicting validation results.

%%%%%%%%%%%%%%%%%%%%%%%%%%%%%%%%%%%%%%%%%%%%%%%%%%%%%%%%%%%%%%%%%%%%%%%%%%%%%%%%%%%%%%%%%
\section{Limitations}
Measuring real-world Internet routing always comes with limitations. As the routing infrastructure is in constant flux, a random event might influence results and distort the drawn conclusions. It is thus possible that a subset of the observed routing changes was misinterpreted as ROV-related. The room for error was minimized by routing traffic to two prefixes and using inverse configurations for the prefix hijacks, but a small number of random routing changes might still have influenced the results.

The distribution of probes and vantage points introduces a bias in the results. As the probes and vantage points are located in more modern parts of the Internet, the results primarily represent the technologically advanced Internet regions, i.e. Europe and North America. The visibility of the data-plane measurement was improved by placing two of the announcing servers in regions outside of Europe and North America, one in Brazil and one in Japan. Still, the small number of probes in Africa, Asia, and Oceania limits the generalization of the results, and we expect that the deployment over all global ASes might be significantly lower than the found results. Most stub-ASes on the Internet do not run an Atlas probe and are thus not visible by the measurement. The low absolute percentage of ASes on the Internet running an RP client indicates that the results do not allow valid conclusions about ROV enforcement in ASes other than ISPs.

The mappings of IP addresses to AS numbers and geo-locations used data-sets by the CAIDA foundation and an IP location provider. The measurements rely on the accuracy of these mappings to draw robust conclusions from the acquired data. The reliability of the data-sets is thus an additional potential error source for the measurement results.
}


\section{Conclusions}\label{sc:conclusions}
RPKI is crucial for Internet security. Not only does it block hijacks, but it also paves the foundations for other mechanisms, such as BGPsec \cite{austein2017rfc}, ASPA\footnote{\url{https://datatracker.ietf.org/doc/draft-ietf-sidrops-aspa-verification}}, RTA\footnote{\url{https://tools.ietf.org/html/draft-michaelson-rpki-rta-00}} against path manipulation attacks. These standards build on RPKI as the source of truth about the origin authorizations and routing validation. 

In this work, we develop an improved methodology for measuring ROV and find that more than 27\% of the ASes currently filter bogus BGP announcements with ROV. Our measurements are more accurate and identify more ROV-enforcing ASes than previous work. 
We show that most ROV-supporting ASes are providers who apply filtering to avoid losing their customers' traffic, indicating that ROV deployments are aligned with the BGP business incentives. Stub-ASes, which are not paid for traffic forwarding, have lower rates of ROV enforcement. This observation is also useful for other mechanisms and facilitates their deployment. 

Surprisingly, we find that ROV on routeservers cannot solve the problem of origin hijacks. Analysis of the impact of routeserves on the propagation graph shows that the high prevalence of direct peering sessions limits the capability of IXPs to mitigate the global spread of malicious paths. Further, even without the prevalence of direct peering sessions, graph analysis shows that IXPs provide only localized protection against hijacks. The burden of protecting the global routing architecture primarily lies on large ISPs and Tier-1 providers. ROV implementation in Tier-1 providers greatly benefits Internet security as it limits the spread of hijacks to a localized scope. To achieve global protection, deployment of ROV should be increased in large providers, as the current ROV is insufficient to protect all ASes and cannot reliably prevent the spread of invalid updates on a local or national scale.
A combination of global protection in large providers with localized protection in IXPs would provide optimal protection of the Internet.
