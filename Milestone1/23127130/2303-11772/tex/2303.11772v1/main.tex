%
\documentclass[letterpaper,twocolumn,10pt, hyphens,obeyspaces,spaces]{article}


\RequirePackage[2020-02-02]{latexrelease}

\usepackage{usenix-2020-09}

\usepackage{titlesec}
\titlespacing{\section}{0pt}{6pt plus 2pt minus 2pt}{4pt plus 2pt minus 2pt}
\titlespacing{\subsection}{0pt}{6pt plus 2pt minus 2pt}{4pt plus 2pt minus 2pt}
\titlespacing{\subsubsection}{0pt}{3pt plus 2pt minus 2pt}{2pt plus 1pt minus 1pt}
\titlespacing{\paragraph}{0pt}{\parskip}{-\parskip}



\PassOptionsToPackage{usenames,dvipsnames}{xcolor}
\usepackage{url}
\usepackage[utf8]{inputenc}
\usepackage{array, multirow}
\newcolumntype{P}[1]{>{\centering\arraybackslash}p{#1}}
\usepackage{tikz}
\usepackage{makecell}
\usepackage{subcaption}
\usepackage{flushend}
\usepackage{fancyhdr}
\usepackage{algorithm2e}
\usepackage{listings}
\usepackage{cprotect}
\usepackage{environ}
\usepackage{authblk}
\usepackage{amsmath}
\usepackage{mathtools}
\usepackage{svg}
\usepackage{xcolor}
\usepackage{graphicx}
\usepackage{tikz}
\usetikzlibrary{trees}                          % For building tree like structures
\usetikzlibrary{arrows}
\newcommand{\tom}[1]{\color{blue}Tomas: #1}
\newcommand{\niklas}[1]{\color{brown}Niklas: #1}

\newcommand{\updated}[1]{\color{black} #1}

\usepackage{graphicx}
\usepackage{caption}

\renewcommand{\arraystretch}{1.25}
\usepackage{pifont}
\newcommand{\cmark}{\ding{51}} % tick  aka yes
\newcommand{\xmark}{\ding{55}} % cross aka no

\newcommand{\ignore}[1]{}
%\newcommand{\name}{DTI}
\newcommand{\name}{injection}
\newcommand{\new}[1]{\textcolor{black}{#1}}
\newenvironment{newenv}{\color{black}}{}
\newcommand{\sh}[1]{\textcolor{red}{\textbf{Comment:} #1}}
\newcommand{\mytodo}[1]{\textcolor{blue}{\textbf{TODO:} #1}}
\newcommand{\del}[1]{\textcolor{purple}{#1}}
\newenvironment{delenv}{\color{purple}}{}

%%%%%%%%%% NEW TEXT MACRO
\newcolumntype{P}[1]{>{\centering\arraybackslash}p{#1}}
\usepackage{tikz}
\usepackage{amsmath}
\usepackage{amssymb}
\newcommand{\stt}[1]{{\small\path{#1}}}
\newcommand*\circled[1]{\tikz[baseline=(char.base)]{
            \node[shape=circle,draw,inner sep=1pt,semithick] (char) {#1};}}

\newcommand*\dashcircled[1]{\tikz[baseline=(char.base)]{
            \node[shape=circle,draw,inner sep=1pt,dashed,semithick] (char) {#1};}}

\newcolumntype{H}{>{\setbox0=\hbox\bgroup}c<{\egroup}@{\hspace*{-\tabcolsep}}}
\newcommand{\rot}[1]{\rotatebox[origin=c]{90}{ ~#1~ }}
\newcommand{\mr}[2]{\multirow{#1}{*}{#2}}
\newcommand{\mc}[3]{\multicolumn{#1}{#2}{#3}}

\begin{document}
\pagenumbering{gobble} 

\title{Keep Your Friends Close, but Your Routeservers Closer:\\Insights into RPKI Validation in the Internet}

\author[*$\S$]{Tomas Hlavacek}
\author[*$\S\dag$]{Haya Shulman}
\author[*$\S\dag$]{Niklas Vogel}
\author[*$\S\ddag$]{Michael Waidner}
\affil[$\S$]{National Research Center for Applied Cybersecurity ATHENE}
\affil[*]{Fraunhofer Institute for Secure Information Technology SIT}
\affil[$\ddag$]{Technische Universität Darmstadt}
\affil[$\dag$]{Goethe-Universität Frankfurt}

\maketitle

\begin{abstract}

Large language models (LLMs) can enhance writing by automating or supporting specific tasks in writers' workflows (e.g., paraphrasing, creating analogies).
Leveraging this capability, a collection of interfaces have been developed that provide LLM-powered tools for specific writing tasks. However, these interfaces provide limited support for writers to create personal tools for their own unique tasks, and may not comprehensively fulfill a writer’s needs---requiring them to continuously switch between interfaces during writing.
In this work, we envision LMCanvas, an interface that enables writers to create their own LLM-powered writing tools and arrange their personal writing environment by interacting with ``blocks’’ in a canvas. 
In this interface, users can create text blocks to encapsulate writing and LLM prompts, model blocks for model parameter configurations, and connect these to create pipeline blocks that output generations.
In this workshop paper, we discuss the design for LMCanvas and our plans to develop this concept.

\end{abstract}

\section{introduction}

% 1. importance of TKGs and reasoning on TKGs. 
% 2. low resource languages, main main idea.
% 3. relations and limitations of current works.
% 4. summarize our solutions and contributions.

Temporal Knowledge Graphs (TKGs)~\cite{YAGO,ICEWS18,WIKI,acekg} characterize temporally evolving events, where each event, represented as ({\em subject}, {\em relation}, {\em object}), is associated with temporal information ({\em time}), e.g., ({\em Macron}, {\em reelected}, {\em French president}, {\em 2022}). TKGs has facilitated various knowledge-intensive Web applications with timeliness, such as question answering~\cite{KBQA}, product recommendation~\cite{RippleNet,TKG4Rec,TKG4Rec2,RETE}, and social event forecasting~\cite{KG4Social,DyDiff-VAE,andgan,belief,misinfo,polarization}. 

As new events are continually emerging, modern TKGs are still far from being complete. Conventionally, the TKG construction process relies primarily on information extraction from unstructured corpus~\cite{WIKI,YAGO, EventKG}, which necessitates extensive manual annotations to keep up with changing events. For instance, the recent transition from Trump to Biden as the President of the United States has not been reflected in many TKGs, highlighting the need for timely updates. This spurs research on temporal knowledge graph reasoning to automate evolving events prediction over time~\cite{TA-DistMult,Know-Evolve,Renet,RE-GCN}. Unfortunately, the problem of TKG incompleteness is particularly pronounced in low-resource languages, where it is unable to collect enough corpus and annotations to support robust TKG construction. This results in suboptimal reasoning performance and distinctly unsatisfying accuracy in predicting recent and future events.

% whose performance can degrade significantly in low-resource language TKGs that suffer from severe incompleteness over time. 
% \jingfeng{why don't people  study cross-lingual TKG previously, (i.e. use language alignment to improve TKG). Is it really helpful intuitively to use high resource language to help TKGC? For instance, is it enough to use static langauge-alignment to help KGC, ignoring the temporal information? Are those langauge-alignment changing across time?}



\begin{figure}
    \centering
    \includegraphics[width = 1.0\linewidth]{fig/task.pdf}
    \caption{An illustrative example of cross-lingual reasoning on TKGs. 1) We aim to transfer knowledge from English TKG to Japanese TKG, where the English version provides more complete information; 2) Cross-lingual alignments only cover a small ratio of entities, e.g., Apple Inc; 3) Cross-lingual alignments can be noisy and misleading, e.g., A city called Ventura is linked to new macOS Ventura at $t_2$, introducing noise for reasoning in Japanese.}
    \label{fig:illustration}
    %\vspace{-6mm}
\end{figure}

Inspired by the incompleteness issue facing low-resource languages in constructing TKGs, we introduce a novel task named Cross-Lingual Temporal Knowledge Graph Reasoning (as shown in Figure~\ref{fig:illustration}). This task aims to alleviate the reliance on supervision for TKGs in low-resource languages (referred to as the target language) by transferring temporal knowledge from high-resource languages (referred to as the source language)~\footnote{In this paper, for the sake of brevity, we interchangeably use the terms high-resource/low-resource and source/target.}. In contrast, all the existing efforts are either limited to reasoning in monolingual TKGs (usually high-resource languages, e.g., English)~\cite{TA-DistMult,Know-Evolve,Renet,RE-GCN}, or multilingual static KGs~\cite{KEnS,AlignKGC,SS-AGA}. To the best of our knowledge, cross-lingual TKG reasoning that transfers temporal knowledge between TKGs has not been investigated. 

%Motivated by this, we study a new task named {\em cross-lingual temporal knowledge graph reasoning} as shown in Figure~\ref{fig:illustration}, to alleviate the heavy dependence on supervision for any resource-poor language TKGs by distilling the temporal knowledge from resource-rich ones. Differently, all the existing efforts are either limited to reasoning in monolingual (usually high-resource languages, e.g., English) temporal KGs~\cite{TA-DistMult,Know-Evolve,Renet,RE-GCN}, or multilingual static KG~\cite{KEnS,AlignKGC,SS-AGA}, but neglecting the reasoning in a both temporal and cross-lingual manner that highly requires capturing time-evolving patterns and language discrepancy. To the best of our knowledge, this problem, regarding how to transfer cross-lingual knowledge between TKGs, has still not been formally investigated. 

% Unlike conventional TKG reasoning, 
The fulfillment of this task poses tremendous challenges in two aspects: 1) \textbf{Scarcity of cross-lingual alignment}: as the informative bridge of two separate TKGs, cross-lingual alignment is imperative for cross-lingual knowledge transfer~\cite{AlignKGC,KEnS,SS-AGA}. However, obtaining alignments between languages is a time-consuming and resource-intensive process that heavily relies on human annotations. The transfer of knowledge through a limited number of alignments is often insufficient to fully enhance the TKG in the target language. 2) \textbf{Temporal knowledge discrepancy}: the information associated with two aligned entities is not necessarily identical, especially with regards to temporal patterns. Utilizing a rough approach to equate the aligned entities at all times can result in the transfer of misleading knowledge and negatively impact performance. This becomes more pronounced when the alignments are noisy and unreliable. For example, at the time step $t_2$, a new event about operating system ``{\it Ventura}'' from Apple company occurs in the source English TKG, and meanwhile there is a noisy aligned entity ``{\it Ventura city}'' in the target Japanese TKG. Directly pulling those two entities at this point, can inevitably introduce  noise and fail to predict a set of related events in the target TKG. Therefore, it is crucial to dynamically regulate the alignment strength of each local graph structure over time in order to maximize the effectiveness of cross-lingual knowledge distillation.

% Pulling those entities together cannot augment information in target languages. Small alignment strength is beneficial in the unreliable alignment cases, otherwise the misleading knowledge transferring can even hurt the performance.

% Moreover, in a case that the alignments are not fully reliable, directly pulling the two aligned entities together 


% optimally dynamic alignment strength
% {\em Optimal alignment strength to maximize the benefits of knowledge distillation is difficult to obtain, especially in the temporal manner.} 
% In practical, although the aligned entities can share similar information, they may still differ in other perspectives, including but not limited to frequency, interactions, and temporal patterns. How to adjust the alignment strength (i.e., the distance constrains of the aligned entities in the uni-space) accordingly for different entities at different time is unclear. \zheng{Ruijie TODO: add Ventura case}Moreover, in a case that the alignments are not fully reliable, directly pulling the two aligned entities together can even hurt the performance.



% scarcity of hinders the efficient
% knowledge transfer across languages. 
% {\em Transferring knowledge through a small set of alignments is hard to augment information for all entities.} 

% Aligning the same entities across languages rely heavily on manual labeling or rule-based inference~\cite{EA1,EA2,EA3,selfKG}, which is too time-consuming and impractical to obtain the alignments covering most of the entities in target language. 

% In this paper, we study how to boost the TKG reasoning performance in low-resource languages by explicitly increasing the completeness of those TKGs in history. Instead of improving the underlying information extraction techniques in low-data regime, we propose a new task called {\em Cross-lingual Temporal Knowledge Graph Reasoning}, motivated by the facts that there exists common or complementary knowledge shared by the TKGs in different languages under similar topics. The new task aims to facilitate TKG reasoning in low-resource languages (target languages) by distilling knowledge from a corresponding TKG in high-resource language (source language)  through a small set of entity alignments as bridges~\footnote{In this paper, we interchangeably use the terminology high-resource/low-resource and source/target for briety.}. Figure~\ref{fig:illustration} provides an illustrative example of the proposed task.


% Unfortunately, recent breakthroughs in temporal knowledge graph reasoning model~\cite{TA-DistMult,Know-Evolve,Renet,RE-GCN} highly rely on the completeness of the TKGs, especially for the most recent events. 

% However, the completeness of TKGs varies a lot across different languages, even under similar topics. Conventionally, the TKG construction process relies primarily on information extraction techniques built on the unstructured corpus~\cite{WIKI,YAGO, EventKG}. Therefore, the amount of corpus and human annotations in different languages significantly influence the quality of the corresponding TKGs . 
% Therefore, automatically completing/updating TKGs has been attracting enormous interests in recently years, which aims to predict recent/future events on TKGs based on historical events~\cite{TA-DistMult,Know-Evolve,Renet,RE-GCN}, namely temporal knowledge graph reasoning~\footnote{Broadly speaking, TKG reasoning includes interpolation to predict historical events and extrapolation to predict future events. In this paper, we refer to extrapolation task as TKG reasoning, since it is more vital for time-sensitive downstream tasks.}.


% For languages with large-scale and carefully labeled corpus (we refer to as high-resource languages, e.g., English), the constructed TKGs are more comprehensive than TKGs in other languages that lack the high-quality corpus (we refer to as low-resource languages, e.g., Spanish, Slovene, Danish, etc). Such completeness discrepancy leads to distinctly uneven TKG reasoning performances in different languages, which in turn affects the quality of service of the downstream applications. 


% Compared with the traditional TKG reasoning task, the new task imposes non-trivial challenges. An intuitive solution is to construct a unified graph including two TKGs in both source and target languages, and the knowledge distillation can be fulfilled by pulling the aligned entities from two languages close to each other in the uni-space~\cite{AlignKGC,KEnS}. However, there are still two challenges to be addressed. 

% \zheng{Ruijie TODO, Place this part to related works.}
% Existing works in related areas fail to address the aforementioned challenges. Monolingual reasoning methods on static/temporal knowledge graphs~\cite{TransE,TranR,ComplEX,RotatE,TA-DistMult,Know-Evolve,Renet,RE-GCN} is incapable of the desired knowledge transferring due to the insufficient alignment modeling. Although they can be extended on the cross-lingual scenario by viewing the alignments as a new relation on the merged TKGs, the limited amount of alignments prevent them from augmenting information for most of the entities. Entity alignment methods on KGs~\cite{EA1,EA2,EA3,EA4,EA5,selfKG} can automatically enlarge the alignments by  predicting the correspondence between the two TGs. But most of them, if not all, require the relatively even completeness of two TGs to capture the structural similarities, which can not be satisfied in our case, as target TKGs are far from complete. Some recent works start to study the multilingual TK reasoning on static graphs~\cite{AlignKGC,KEnS,SS-AGA}, which similarly aim to extract knowledge from several source KGs to boost the reasoning performance in the target KG, while they still require a sufficient amount of cross-lingual alignments and totally ignore the temporal perspective in our task.

% to facilitate temporal knowledge graph reasoning in low-resource languages. 
% increase the TKG connection and target TKG capacity
% In light of the mutual benefits, we iteratively generate pseudo alignment pairs and pseudo temporal events to address the co-existing scarcity issue in both cross-lingual alignment and target TKGs. 


In this paper, we propose a novel Mutually-paced Knowledge Distillation (\model) framework, where a teacher network learns more enriched temporal knowledge and reasoning skills from the source TKG to facilitate the learning of a student network in the low-data target one. The knowledge transfer is enabled via an alignment module, which estimates entity correspondence across languages based on temporal patterns. Firstly, to alleviate the limited language alignments (\textbf{Challenge \#1}), such a knowledge distillation process is mutually paced over time. This means, on one hand, we encourage the mutually interactive learning between the teacher and student. Concretely, the alignment module between the teacher and the student learns to generate pseudo alignment between TKGs to maximally expand the upper bound of knowledge transfer. And subsequently, it empowers the student to encode more informative knowledge in target TKG, which can in turn boost the alignment module to explore more reasonable alignments as the bridge across TKGs. One the other hand, inspired by self-paced learning~\cite{spl-1,spl-2}, we make the generations as a progressively easy-to-hard process over time. We start from generating reliable pseudo data with high confidence. As time goes by, we then gradually increase the generation amount by relieving the restriction over time. Secondly, to inhibit the temporal knowledge mismatch (\textbf{Challenge \#2}), the attention module can estimate the graph alignment strength distribution over time. This is achieved by a temporal cross-lingual attention in terms of the local graph structure and temporal-evolving patterns of aligned entities. As such, it can dynamically control the negative effect and suppress noise  propagation from the source TKG. Moreover, we provide a theoretical convergence guarantee for the training objective on both initial ground-truth data and pseudo data. To evaluate \model, we conduct extensive experiments of 12 cross-lingual TKG transfer tasks in multilingual EventKG dataset~\cite{EventKG}. Our empirical results show that the \model method outperforms state-of-the-art baselines in both with and without alignment noise settings, where only $20\%$ of temporal events in the target KG and $10\%$ of cross-lingual alignments are preserved.

% To validate the effectiveness of \model, we conduct extensive experiments of 12 cross-lingual TKG transfer tasks in multilingual EventKG benchmark dataset~\cite{EventKG} . Our experimental results empirically demonstrate the superiority of the \model method over state-of-the-art baselines, ranging from static KG embedding~\cite{TransE,TransR,DistMult,RotatE}, temporal KG reasoning~\cite{TA-DistMult,Renet,RE-GCN} to multilingual KG completion~\cite{KEnS,AlignKGC,SS-AGA}, in both with and without alignment noise settings. We further conduct comprehensive ablation and hyperparameter studies to validate the effectiveness of each design choices. Moreover, we provide theoretical analysis of convergence guarantee for the training objective on both initial groundtruth data and pseudo generative data.



To sum up, our contributions are three-fold:

\begin{itemize}[leftmargin = 15pt]
    \item \textbf{Problem formulation}: We propose the cross-lingual temporal knowledge graph reasoning task, to boost the temporal reasoning performance in target TKG by transferring knowledge from source TKG;
    \item \textbf{Novel framework}: We propose a novel \model framework, which enables the mutually-paced learning between the teacher and student networks, to promote both pseudo alignments and knowledge transfer reliability. Besides, \model involves a dynamic alignment estimation across TKGs that inhibits the influence of temporal knowledge discrepancy.
    \item \textbf{Extensive evaluations}: Empirically, extensive experiments on 12 cross-lingual TKG transfer tasks in multilingual EventKG benchmark dataset demonstrate the effectiveness of \model.
\end{itemize}
% pseudo data generation technique to progressively enhance the training data. The generated pseudo alignments can help the training of the representation modules by the knowledge distillation, and in turn adding pseudo events in the target TKG can improves alignment module by providing high-quality representations. 




% interactively
% TKGs in a source language and a target language are represented by a teacher representation module and a student one into a uni-space, respectively. 
% The knowledge distillation is enabled by a cross-lingual alignment module which pulls the aligned entities close to each other and push other entities far away. 
% To address the challenge caused by the scarcity of cross-lingual alignment, 


\section{Overview of RPKI}\label{sc:overview}

RPKI provides authenticated prefix ownership information, which routers can use for making routing decisions.

{\bf RPKI objects.} To authorize their network resources, ASes can create resource certificates that bind their resources to a public key contained inside a Route Origin Authorization. The ROA is signed with the certificate of a Certificate Authority (CA). RPKI objects are published in RPKI repositories hosted on publication points. The publication points are operated either in a hosted mode by one of the Regional Internet Registries (RIRs) or in a delegated mode by a Local Internet Registry (LIR). An RPKI repository keeps a finite set of signed ROAs and additionally contains signed certificates (which point to children publication points), certificate revocation lists (CRLs), and manifests.

{\bf Traversal of RPKI repositories.} The validation of RPKI objects is performed with a relying party software, which contains hardcoded Trust Anchor Locators (TALs) to the root certificates of the RIRs. Each of the five RIRs operates its own RPKI trust anchor certificate and repository. During the validation, a relying party contacts every root repository known to it and downloads RPKI objects from every publication point it finds. The RPKI objects are fetched from RPKI repositories over RRDP or rsync protocols. After downloading the objects, a relying party performs cryptographic validation, which produces a list of tuples (AS, ROA prefix, prefix length) called Validated ROA Payloads (VRPs). The VRPs are stored in a local cache. 

{\bf Route Origin Validation.} The BGP border routers of an AS retrieve the VRPs from their relying party's cache over the `RPKI to Router Protocol' (RTR) [RFC8210]. The routers use the VRPs to validate incoming BGP announcements with Route Origin Validation. A router checks if the IP prefix block in the BGP announcement and the VRP IP prefix block are identical for the length specified by the VRP IP prefix length [RFC6811]. If the IP prefix in the announcement is covered by any VRP entry, the router checks if the BGP origin AS in the announcement matches the VRP AS for that prefix. Matching values result in the conclusion that the announcement is valid. In contrast, if any VRP covers the prefix in the BGP announcement, but the entry does not match the origin AS, then the announcement is invalid. The validation status is considered unknown if the BGP announcement is not covered by any VRP entry. 
\section{Related Work}\label{sc:works}
\begin{table*}[t!]
\renewcommand{\arraystretch}{0.6}
    \centering
    \footnotesize
\begin{tabular}{l|c|c|c|c|c|c|c|c}
\textbf{Research} & \textbf{Year} & \textbf{Control-plane} & \textbf{Data-plane} & \textbf{Methodology with} & \textbf{Clients} & \textbf{>1 Target} & \textbf{Rate of} & \textbf{\#ASes}\\
 &  & \textbf{} & \textbf{} & \textbf{Divergence} & & \textbf{AS} & \textbf{ROV-ASes} \\\hline \hline
This work & 2022 & \cmark  & Traceroute/Atlas & \cmark & \xmark & \cmark & 27\% & 2.4K\\ \hline
Cloudflare \cite{cloudflare} & 2022 & \xmark & HTTP & \xmark & Volunteers & \xmark & 30\% & 380 \\ \hline
Rodday et al. \cite{rodday2021revisiting} & 2021 & \cmark  & Traceroute/Atlas & \xmark & \xmark & \xmark & 0.6\% & 3.6K\\ \hline 
APNIC \cite{apnic} & 2021 & \cmark & HTTP & \xmark & Ad-network & \xmark & 25\% & 25K \\ \hline
Testart et al. \cite{testart2020filter} & 2020 & RouteViews & \xmark & \xmark & \xmark & \xmark & 11\% & 21 \\ \hline
Hlavacek et al. \cite{hlavacek2018practical} & 2018 & \cmark & Traceroute/Atlas & \xmark & \xmark & \cmark & 0.5\% & 296\\ \hline
Gilad et al. \cite{gilad2017we} & 2017 & \cmark & \xmark & \xmark & \xmark & \xmark & 3\% & 100 \\ %\hline
\end{tabular}
\vspace{-7pt}
\caption{Measurements of ROV: characteristics of our and previous work.}
\vspace{-10pt}
\label{tab:comparison}
\end{table*}
Practical impact of BGP prefix hijacks has been extensively explored \cite{DBLP:conf/uss/Birge-LeeSERM18,DBLP:journals/corr/abs-2004-09063} and real-world hijack incidents \cite{china:telecom,mitm:threat,turkey:hijack,indosat:hijack} confirmed the projected assessment of the research. The awareness to prefix hijacks creates a strong motivation to understand the deployment of RPKI and to obtain insights into the effectiveness of ROV. Previous measurements studied related aspects, such as the prevalence of invalid ROA objects caused by benign misconfigurations \cite{chung2019rpki,DBLP:conf/esorics/HlavacekSW22}, the impact of the Domain Name System on the resilience of RPKI \cite{DBLP:conf/ccs/HlavacekJMSW22} or downgrade attacks against RPKI \cite{usenix-stalloris-21,DBLP:conf/ccs/MirditaSW22}. In this work we explore the effectiveness of ROV. We next put our research in the context of related work on measurements of ROV.

{\bf Effectiveness of ROV.} Previous work provided a theoretical upper bound on the feasibility of hijacks \cite{gilad2017we,hlavacek2020disco}. 
This was done by simulating success of any Internet AS to hijack any prefix assuming a varying fraction of ROV-enforcing ASes. Such simulations do not consider the data-plane paths that actual traffic takes and do not use the real ASes that de facto enforce ROV, but just assume a fraction of ROV enforcement. Therefore the theoretical bound does not reflect a realistic attack surface.
In addition to not reflecting practical factors relevant to success of hijacks, the simulations do not answer questions related to the effectiveness of ROV in blocking propagation of hijacks and to the affected networks. For instance, not all hijacks have equal impact and hijacking a Tier-1 provider also redirects the traffic of all its customers. Our goal is not only to understand if hijacks are feasible, but also to infer which and how many networks are affected by the hijacks and by the ROV filtering. We do this by analyzing the date-plane paths that traffic takes, and the impact of ROV on the Internet graph of networks. We use the observations from our analysis to derive future directions that deployment of ROV should take to reach optimal protection of the Internet. 


{\bf Approaches for measuring ROV.} 
The first global measurement of ROV enforcement was carried out in 2017 \cite{gilad2017we} (listed in Table \ref{tab:comparison}). The study monitored the propagation of invalid BGP announcements in public BGP collectors and found 100 ROV-enforcing ASes. In their experiment, Gilad et al. \cite{gilad2017we} passively monitored ASes that originated valid and invalid BGP announcements, and then collected ASes that were on the paths towards the valid prefix, but not on the paths towards the invalid prefix. Those ASes were classified as ROV-enforcing. However, the measurements had high false positives and false negatives rates since they used invalid BGP announcements of other ASes, which they did not control. This also limited the coverage of the experiment. The methodology of \cite{gilad2017we} was improved with a controlled experiment in the control plane by \cite{hlavacek2018practical}, which monitored propagation of invalid announcements in public collectors and used active probes over RIPE Atlas. The study of \cite{hlavacek2018practical} found 296 ROV-enforcing ASes. A subsequent study in 2020 \cite{testart2020filter} passively analyzed the historical data from RouteViews\footnote{\url{http://www.routeviews.org}} to identify changes in routing behavior, finding 21 ROV-enforcing ASes. Since these measurements were performed using a limited number of collectors (less than 0.01\%) the results were not representative of the entire Internet. Increasing the coverage is imperative for collecting representative data. In 2021 \cite{rodday2021revisiting} did an ROV study with a methodology of \cite{hlavacek2018practical} using 5537 probes in 3694 origin ASes. 

In our work, we combine control and data-plane measurements similarly to \cite{hlavacek2018practical,rodday2021revisiting}. In contrast to \cite{hlavacek2018practical,rodday2021revisiting}, which used an invalid ROA conflicting with a BGP announcement to infer ROV enforcement, we alternate between valid and invalid BGP announcements, which has faster convergence time than changes in ROAs. Alternating between valid and invalid events using two prefixes during data acquisition further allows us to eliminate random routing events, such as events in which an AS uses traffic engineering that complies with the ROA validity, which may be misinterpreted as ROV. This alternation reduces false positives. In addition, the previous method does not scale since it adds a large number of false negatives that lack sufficient evidence for ROV enforcement. We explain the issues with false-positives and false-negatives in previous work when we derive our methodology from previous approaches in Section \ref{sc:method}. In our ROV measurements, we greatly reduce the number of false-negatives and thus provide a more realistic view of real-world ROV enforcement. During data analysis, we apply a path-aware methodology that uses a new metric, divergence points, to reduce false positives. Further, we introduce an AS classification scheme to differentiate ASes that actively enforce ROV from ASes with only passive protection in an upstream ROV. Our approach shows a much higher rate of ROV deployment in the Internet than previously found in \cite{hlavacek2018practical,rodday2021revisiting}, including extensive evidence for enforcement in 9 of the 15 Tier-1 providers. We show that the higher rates of ROV enforcement are related to the improvements in our methodology and the continuous increase of ROV enforcement rate over time.

 
 In 2023, an online service called RoVista\footnote{\url{https://rovista.netsecurelab.org/}} was set up for reporting ROV enforcement. Similarly to \cite{gilad2017we}, RoVista uses an uncontrolled control-plane experiment, utilizing ASes with invalid BGP announcements and probing the reachability to the invalid prefixes from other systems in the Internet. Since they use invalid routes that happen to be announced by other ASes, their coverage is limited; currently, as they point out, only 1\% of prefixes are RPKI invalid. Additionally, they have the same downsides as uncontrolled experiments like \cite{gilad2017we}, which include a high rate of false positives. For instance, ASes might appear to behave like ROV filtering networks because of other (non-ROV) mechanisms. A more significant issue with RoVista is the usage of IPID side channel to identify ASes that follow invalid routes. There are three problems with IPID side channels. First, globally incrementing IPID has been gradually phased out in operating systems. Previous work \cite{pearce2017augur,dai2021smap,DBLP:conf/dsn/ShulmanZ21} showed that very few hosts ($\sim$16\%) had globally incrementing IPID counters and that some hosts with globally incrementing counters set their values to 0 when packets are too small to be fragmented. RoVista additionally requires that multiple ASes have at least ten hosts with globally incrementing IPID. The authors do not provide the methodology and measurement details on how many ASes have at least ten hosts with globally incrementing IPID counters. Since previous work showed that the number of hosts with global incrementing IPID is small and is further shrinking, the approach has limited scalability. 

 Second, measurements of IPID incur a lot of noise due to communication from other hosts, failures, and traffic fluctuations. Simple applications that use IPID, such as indirect measurements of idle port scans with NMap, exhibit high failure rates in dynamic Internet environments with over 10\% failures in scans of even completely idle hosts. These measurements were slightly improved with anti-noise techniques used by \cite{zhang2018onis}. Using the IPID side channel to measure ROV enforcement appears to be much more challenging than just checking if a port is open. In particular, there is a large time interval between the probing of the IPID value and the time that the BGP announcements converge and the routes are updated. The lack of visibility into the exact time when the route change is accepted makes it impossible to approximate the probe time of the IPID value. This is expected to introduce an immense amount of noise into the measurements, producing many false negatives and positives, making it impossible to derive a conclusion on ROV enforcement. The authors do not explain how they deal with that noise. Finally, the traffic volume to the ASes that announce the invalid prefixes would become prohibitive in the presented methodology as all the tested networks are required to send traffic to these ASes from ten of their hosts, resulting in regular traffic from 280.000 hosts. In total, they send traffic from these hosts to 47 ASes that announce invalid prefixes. The lack of methodology details of the measurements done by RoVista makes it impossible to compare our study to theirs and to understand the correctness or effectiveness of their approach. 

A completely different approach was taken by the Cloudflare project \footnote{https://isbgpsafeyet.com/}, which is a community-driven effort to summarize ROV implementation of large providers. The project provides a webpage to test ROV enforcement of providers by probing the reaction of the client to a valid and invalid announcement. If a client can reach the valid announcement and not the invalid one, they conclude that the provider of the client enforces ROV. The user can then contribute to the project over a GitHub page and update the status of its provider in the dataset. Our measurements show that while this approach may be sensible for a rough overview of large ROV-enforcing networks, it does not scale to an accurate representation of ROV enforcement in the Internet. 
In contrast to previously described approaches, Cloudflare requires coordination and support of volunteers to measure ROV enforcement in the networks of the users, which proves problematic.
First, the dataset is small, with only 380 ASes. Second, smaller ASes with fewer clients are less likely to have a user that contributes, and thus the dataset is biased towards the largest providers. 
We also find false positives in the results. Since the measurement methodology relies on two announcements picked up by a single vantage point, it leads to errors in cases where the ROV in an upstream provider filters the invalid announcement. The provider of the user is mistakenly classified as ROV-enforcing. 
This approach thus not only limits the applicability of ROV measurements in the global Internet but also introduces a bias to the results. We evaluated the Cloudflare dataset and identified differences and errors in the classification, which we discuss in Section \ref{subsec:vali}.


\begin{figure}[t!]
    \centering
    \scalebox{0.8}{
    \begin{tikzpicture}
       \tikzset{vertex/.style = {shape=circle,draw,minimum size=2em}}
       \tikzset{vertex2/.style = {shape=circle,draw,minimum size=2.5em}}
       \tikzset{edge1l/.style = {dashed, ->, transform canvas={xshift=-2pt, yshift=-1pt}}}
       \tikzset{edge1r/.style = {dashed, ->, transform canvas={xshift=-2pt, yshift=1pt}}}
       \tikzset{edge2l/.style = { ->, transform canvas={xshift=2pt, yshift=1pt}}}
       \tikzset{edge2r/.style = { ->, transform canvas={xshift=2pt, yshift=-1pt}}}
       
       \tikzset{edge3l/.style = {->, transform canvas={xshift=6pt, yshift=-1.5pt}, shorten >=-0.15cm}}
        \tikzset{edge3r/.style = {->, transform canvas={xshift=6pt, yshift=1.5pt}, shorten <=-0.15cm}}

       \tikzset{edge4l/.style = {->, transform canvas={xshift=-6pt, yshift=1.5pt}}}
       \tikzset{edge4r/.style = {->, transform canvas={xshift=-6pt, yshift=-1.5pt}, shorten >=-0.12cm}}
       
       \definecolor{node_red}{RGB}{255, 200, 200}
        \definecolor{node_green}{RGB}{200, 255, 200}
        \definecolor{c_green}{RGB}{0, 100, 0}
        \definecolor{c_red}{RGB}{100,0 , 0}

        \node[vertex] (a) at (0.5, 1.2) {AS 1};
        \node[vertex] (b) at (3.5, 1.2) {AS 2};
        
        \node[vertex, fill=node_red] (c) at (-0.3, 2.5) {};
        \node[vertex, fill=node_green] (d) at (1.3, 2.5) {};
        \node[vertex, fill=node_red] (e) at (2.7, 2.5) {};
        \node[vertex, fill=node_red] (f) at (4.3, 2.5) {};
        
        \node[vertex, fill=node_green] (g) at (-1, 3.5) {};
        \node[vertex, fill=node_red] (h) at (0.5, 3.5) {};
        \node[vertex, fill=node_red] (i) at (2, 3.5) {};
        \node[vertex, fill=node_green] (j) at (3.5, 3.5) {};
  
        \node[vertex, fill=node_red] (k) at (5, 3.5) {};
        
        \draw[purple] (-1.5,4.5) rectangle (-0.5,5.5) node[pos=.5] {VP 1};
        \draw[blue] (0.5,4.5) rectangle (1.5,5.5) node[pos=.5] {PB 1};
        \draw[blue] (2.5,4.5) rectangle (3.5,5.5) node[pos=.5] {PB 2};
        \draw[purple] (4.5,4.5) rectangle (5.5,5.5) node[pos=.5] {VP 2};
        
        \node (l) at (-1, 4.5){};
        \node (m) at (1, 4.5){};
        \node (n) at (3, 4.5){};
        \node (o) at (5, 4.5){};
        
        
        
        \draw[edge1l, c_green] (a) to (c);
        \draw[edge2l, c_red] (a) to (c);
        \draw[edge1r, c_green] (a) to (d);
        \draw[edge2r, c_red] (a) to (d);
        
        \draw[edge1l, c_red] (b) to (e);
        \draw[edge2l, c_green] (b) to (e);
        \draw[edge1r, c_red] (b) to (f);
        \draw[edge2r, c_green] (b) to (f);
        
        \draw[edge1l, c_green] (c) to (g);
        \draw[edge2l,  c_red] (c) to (g);
        
        \draw[edge1r, c_green] (c) to (h);
        \draw[edge2r,  c_red] (c) to (h);
        
        \draw[edge1l, c_green] (d) to (h);
        
        \draw[edge1l, c_red] (e) to (h);
        \draw[edge2l, c_green] (e) to (h);
        
        \draw[edge1l, c_red] (e) to (i);
        \draw[edge2l, c_green] (e) to (i);
        
        \draw[edge1l, c_red] (f) to (j);
        \draw[edge2l, c_green] (f) to (j);
        
        \draw[->, c_green] (j) to (k);

        \draw[->, c_green] (j) to (i);

        \draw[dashed, ->, c_green] (d) to (e);
        
        
        \node[rotate=90] at (-2.2, 3.0 ){Internet};
        \draw[dotted] (-1.8, 1.9) rectangle (5.8, 4.1);
        
        \draw[edge1l, c_red] (f) to (j);
        \draw[edge2l, c_green] (f) to (j);
        
        \draw[->, c_green] (g) to (l);
        
        \draw[edge1l, c_red] (h) to (l);
        \draw[edge2l, c_red] (h) to (l);
        
        \draw[edge1r, c_red] (h) to (m);
        \draw[edge2r, c_red] (h) to (m);
        
        \draw[edge1r, c_red] (i) to (n);
        \draw[edge2r, c_green] (i) to (n);
        
        \draw[->, c_green] (k) to (o);
        
        \draw[edge3l] (m) to (h);
        \draw[edge3l] (h) to (c);
        \draw[edge3r] (c) to (a);

        \draw[edge4l, dashed] (m) to (h);
        \draw[edge4r, dashed, shorten <=0.12cm] (h) to (e);
        \draw[edge4r, dashed] (e) to (b);
        
        
        \draw[edge3l] (n) to (i);
        \draw[edge3r] (i) to (e);
        \draw[edge3r] (e) to (b);
        
        \draw[edge4l, dashed, shorten >=0.04cm] (n) to (i);
        
        \draw[edge4r, dashed] (i) to (e);

        \draw[->, very thick, c_green, dashed] (-2.3,0.2) --++ (0.5,0) node[right]{Valid Announcement p1};
        \draw[->, very thick, c_red, dashed] (2.0,0.2) --++ (0.5,0) node[right]{Invalid Announcement p1};
        
        \draw[->, very thick, c_green] (-2.3,-0.2) --++ (0.5,0) node[right]{Valid Announcement p2};
        \draw[->, very thick, c_red] (2.0,-0.2) --++ (0.5,0) node[right]{Invalid Announcement p2};
        
        \draw[->, very thick] (-0,-0.6) --++ (0.5,0) node[right]{Traceroute Packet};
            \end{tikzpicture}
    }
    \caption[Overview RPKI Process]%
    {Measurement setup with AS 1 and AS 2 with prefixes p1 and p2. Vantage Points 1 and 2 collect received announcements, the probes PB1 and PB2 use received paths to send out Traceroutes to both prefixes.}
    \label{fig:setup}
    \vspace{-10pt}
    
\end{figure} 

Similarly to Cloudflare, the Asia-Pacific Network Information Centre (APNIC) runs an experiment to test ROV deployment over the reachability of destinations with varying ROV validity \footnote{\url{https://stats.labs.apnic.net/rpki}}. 
The measurement probes how many users in a specific AS can reach an invalid prefix to draw conclusions on the ROV enforcement status of that user’s AS. 
The APNIC measurement improves over the Cloudflare project in two keys aspect. First, they do not rely on users to visit the website and then manually contribute the enforcement status of their provider over a Github page. Instead, they automatically execute the measurement on client systems. Second, they use anycast to inject their route from a large amount of routers instead of two points used by Cloudflare. This significantly reduces the rate of false-positives. ROV enforcement in intermediate systems is less likely to affect the measurement if the route is propagated over many different paths to a target. In Section \ref{sc:rov} we use APNIC's dataset to validate our measurements and provide insights into the limitations and challenges inherent in comparing different ROV measurement methodologies. 
\section{Experimental Setup}

\subsection{Models Used}


For evaluating unimodally trained text encoders, we use BERT~\cite{devlin2018bert}, RoBERTa~\cite{liu2019roberta}, DistilBERT and DistilRoBERTa~\cite{sanh2019distilbert}, which are all trained with text-only MLM objectives. We also include results for Sentence-BERT (SBERT)~\cite{reimers2019sentence}, since its output embeddings are trained to have meaningful cosine similarity scores and thus bear more similarity to other models evaluated with Stroop proving. Results on multimodally trained text encoders are reported for CLIP~\cite{radford2021learning} and FLAVA~\cite{singh2022flava}; for these models we use only the text encoder with pretrained weights and discard the other subcomponents. Our tests include checkpoints from both OpenAI and the OpenCLIP open-source implementation of CLIP~\cite{cherti2022reproducible, ilharco_gabriel_2021_5143773}. Details of the checkpoints used for each model are listed in the supplementary material.


The text encoders of the multimodally trained models range in size from 63M (CLIP) to 109M (FLAVA) parameters. We compare to both comparably small unimodally trained text encoders such as DistilBERT (66M parameters) as well as much larger text encoders such as BERT-large (340M). See the supplementary material for an exhaustive list of sizes of the models under consideration.

We use each model with frozen pretrained weights. Our subsequent tests probe the contents of the feature vectors extracted by these models. For MLM probing, we also use the model's MLM head for prediction. In cases where MLM can be used we have found it to outperform Stroop probing; in such cases we report results for MLM probing here and for Stroop probing in the supplementary material.


\subsection{Probing Methods} \label{sec:probing}

In order to probe the inherent knowledge of our models, we use the knowledge probing methods described below. The probing methods that follow are strictly zero-shot; in the supplementary material we analyze the use of linear classifiers trained on our models' frozen embeddings (``linear probing'').


\medskip \noindent \textbf{Masked language modelling (MLM)}. BERT and our other unimodally trained models were all pretrained with MLM objectives and hence can be used for zero-shot prediction of tokens in a masked context. Given a text including a \MASK{} token and a set of $k$ possible completions $C = \{c_1, c_2, \cdots, c_k\}$, a MLM assigns probabilities $p_1, \cdots, p_k$ to each corresponding token. We use $\arg \max_i p_i$ as the model's prediction. Previous works have found that BERT and other MLM can be probed for innate knowledge with this method~\cite{petroni2019language,rogers2020primer}.
    

\medskip \noindent \textbf{Stroop probing (SP)}. We propose another zero-shot probing method to extract knowledge from models based on the pooled embeddings that they extract. Consider a masked text $t_m$ and possible completions $c \in C$, and let $t_c$ be the text with $c$ inserted in the mask location. Given a text encoder $M$, we calculate pooled embeddings $v_m = M(t_m)$ and $v_c = M(t_c)$ %
and unit-normalize them to $\hat{v}_m = v_m / \|v_m\|$ and $\hat{v}_c = v_c / \|v_c\|$. Stroop probing considers the cosine similarity scores $s_c := \hat{v}_m \cdot \hat{v}_c$. These can be used either directly for regression (as in the concreteness task below), or for categorical prediction by selecting $c^* = \arg \max_c s_c$.

The intuition behind Stroop probing is that items which are more surprising, incongruous, or salient in the given context may have a stronger interference effect on the encoding of the surrounding text. This is analogous to the \emph{Stroop effect} in human psychology. When presented with congruent and incongruent stimuli such as color words printed in the same or differing colors (e.g. ``red'' printed in blue), readers take significantly longer on average to read the incongruent stimuli, a phenomenon known as the \emph{Stroop effect}~\cite{macleod1991half}\footnote{For example, try saying these colors out loud (not the printed words): {\color{red}Green},  {\color{green}Red}, {\color{purple}Blue}, {\color{green}Purple}, {\color{blue}Red}, {\color{red}Purple}. }. We use Stroop probing for multiple tasks, including predicting color associations, as described below.

\subsection{Prompts Used}

For each task, we test the probing methods above on a wide variety of prompts in order to show the robustness of the described phenomena. In our results below we report the maximum metric value for each model over all of the prompts, since this represents a rough bound on our ability to extract intrinsic knowledge from the models under consideration. A full list of prompts used for each task and an analysis of model performance across prompts are provided in the supplementary material.

In some cases our prompt contains an empty slot, which we indicate below as $\SMASK$. Some models under consideration have a dedicated mask token, but for those such as CLIP that do not, we insert a fixed token in this slot, detailed further in the supplementary material.


\section{Measurements of ROV-Enforcement}\label{sc:measurements}
In this section we explain the setup and experiments.
\begin{figure}[t!]
    \centering
    \includegraphics[width=0.42\textwidth]{figures_pdf/types.pdf}%width=\columnwidth
    \caption[Type Distribution in Data-Plane]%
    {Observed AS types in control-plane and data-plane. The relative percentage of Tier-1 ISPs is lower for the data-plane because of a substantially higher total AS amount.}
    \label{fig:as_types}
\end{figure}
\subsection{Control-Plane} On the control-plane, we carry out prefix hijacks of prefixes we own, and we use public collectors to monitor the propagation of the valid and invalid BGP announcements in the Internet.

{\bf Setup.} For our control plane measurements, we set up three origin servers, two servers by the Internet provider IBM, located in Sao Paolo (Brazil) and Tokyo (Japan), and one in a scientific institution in Germany. Both servers by IBM are assigned the AS number 212795, and the research institution server receives the AS number 208162. We use these servers to issue alternating valid and invalid BGP announcements. We also create corresponding ROAs, some valid and some conflicting with the BGP announcements. The ROAs are published in our RPKI repository.  

{\bf Monitoring.} To monitor the propagation of our BGP announcements on the control-plane, we use data from route collectors by Routeviews \cite{routeviews} and the RIPE Routing Information Service (RIS) \cite{RIS}. The collectors are BGP-speaking routers that aggregate BGP messages from peers at their respective locations and publish the collected data on the Internet. We download the data from these Vantage Points (VPs) in the form of Multi-Threaded Routing Toolkit (MRT) BGP Table dumps during the measurements and filter it for paths that originate in one of our measurement ASes. 

 \begin{figure}[t!]
    \centering
    \includegraphics[width=0.42\textwidth]{figures_pdf/probe_distr.pdf}
    \caption[Probe Origin Distribution]%
    {Distribution of BGP collectors and Atlas probes over continents. The Atlas probes are biased towards Europe.}
    \vspace{-10pt}
    \label{fig:atlas_probes}
\end{figure}

%%% VISIBILITY OF THE COLLECTORS
Mapping the collectors to geo-locations, we find that the control-plane collectors are more evenly distributed over continents than the data-plane probes. We also observe that the control-plane collectors have a high presence in North America and Europe, while only a minor part is located on other continents. The distribution of the collectors is shown in comparison to the data-plane probes in Figure \ref{fig:atlas_probes}. To maximize the amount of collected data, we use all available data in both control-plane and data-plane. To understand the bias in the results from a different distribution of collectors, we present the results according to regions.
The collectors in the control-plane observe an absolute amount of 1566 paths, with 797 paths to AS212795 and 769 paths to AS208162.

{\bf Experiments.} The prefix hijacks for the ROV measurements are run on two different days, the 8th and 10th of June 2022, to increase the robustness against short-lived unexpected routing events.
All servers in our setup announce two neighboring prefixes throughout both measurements, \textit{P1} 45.155.129.0/24 and \textit{P2} 45.155.131.0/24, with their assigned AS number as the origin.

In the first configuration, a ROA is issued for AS212795 - \textit{P1} and for AS208162 - \textit{P2}. The ROAs are published 24 hours before starting the measurements. After the measurements are finished, the ROAs are withdrawn, and new ROAs are published with the inverse configuration, validating AS212795 to announce \textit{P2} and AS208162 to announce \textit{P1}. The second measurement is run six hours after the configuration change to give enough propagation time for the updated ROAs. The second measurement starts with the inverse configuration, where the ROA was published 24 hours before starting the run, preventing a one-sided bias in the results. After the measurement, the configuration is again reversed, and the final run is conducted six hours later. 

\subsection{Data-Plane} 
After executing the control-plane experiment and monitoring the propagation of the BGP announcements, we check the traffic paths on the data-plane. This analysis requires a coherent structure in the results of control-plane and data-plane. To achieve this consistency, we use Traceroute packets that probe data-plane paths through the network, resulting in a similar path structure to control-plane AS-paths. IP addresses on Traceroute paths are mapped according to the CAIDA AS and IXP mapping \cite{caidaASMapping}.

{\bf Setup.} To obtain a global distribution of origins for Traceroute measurements, we use probes by the RIPE Atlas project.

{\bf Experiments.} The data-plane measurements require a multi-step pipeline for executing the measurements, acquiring the raw data, and applying classification to the results. Measurements are started over the RIPE Atlas API. RIPE Atlas limits the measurements to 1000 probes per experiment. We start four separate Traceroute measurements per execution from 1000 random global Atlas probes, each running to both our prefixes. Experiment IDs are logged to ensure that the measurements are started from identical probes for the inverse control-plane configuration over the Atlas API.
Probes that go out of service during the measurements and thus do not complete a measurement in both configurations are removed from the results. Each measurement is run with a one-minute time difference between requests.

{\bf Processing and analysis.} We process the paths obtained from Traceroutes to remove redundant information and to discard paths that only contain unresponsive hops or originated from a probe that did not complete measurements to both announced prefixes. First, the measurement results are downloaded over the Atlas API, serialized, and receive a unique identifier that is inserted into a local database for processing. The raw data processing starts with a majority vote on each IP hop; Atlas probes run three separate Traceroutes per measurement. If no consensus between the hops can be found, the hop is added with a non-value. In the second processing step, IP addresses are mapped to AS numbers according to the CAIDA datasets \cite{caidaASMapping, caidaIXPMapping}. Traceroute logs the address of the replying interface, which may not always be correctly configured; we observed many internal IP addresses or IP addresses that could not be mapped to an AS number. These ASes are added as a non-responsive hop. Further, to prevent confusion between AS numbers and the arbitrary IDs of IXPs in the dataset, which may overlap, IXP IDs are added with a negative sign. 

The paths then need to be pre-processed to increase the robustness of classification. Consecutive hops of the same AS do not provide additional information and are thus condensed into a single hop for classification. 
None-hops are removed if the previous and following AS are identical, as the hop likely belongs to the same AS as the surrounding hops.

The data processing on the data-plane results in 18520 valid paths with 73481 valid hops and 8489 unresponsive hops. The measurements have a similar amount of paths to both our ASes, with 8286 paths to AS212795 and 7608 paths to AS208162; 2626 paths did not reach one of the target ASes.

{\bf Classification of ASes.} Applying the classification scheme from Section \ref{sc:classification} on the paths is a three-step process. First, the measurements from each probe are iterated and correlated, i.e., the paths to both prefixes in a single measurement run are processed together. Processing checks which Traceroute target follows the ROA. This process is not straightforward, as servers inside the target AS might not reply to the Traceroute. Thus we map the AS numbers of the upstream providers of our targets to the final destination. If one of the paths flows to an invalid prefix, all ASes on the path are classified as occurring on an invalid path. On the other hand, all ASes to a valid prefix receive a point of evidence for a correct path. If the paths to both prefixes follow the ROA, the divergence point between the paths is calculated, and the corresponding AS receives a classification as a divergence point. 

In the second processing step, each AS is analyzed according to its valid vs. invalid paths and divergence points. Then the classification scheme is applied. ASes are stored together with their final classification for analysis. In the last step, results for category 5 are calculated as they require information about upstream ROV enforcement. Processing iterates all ASes classified into category 4 and analyzes upstream ASes. If each path over the AS runs over an AS that is classified into category 6 or 7 before reaching one of the prefixes, the classification is changed to category 5. 


\subsection{Control- vs. Data-Plane}
\label{subsec:cd}
The comparison of control-plane and data-plane measurement points plotted in Figure \ref{fig:atlas_probes} indicates that the measurements have a slightly different view of the Internet, as the control-plane has a higher percentage of points in North America while the data-plane measurements primarily originate in Europe. Additionally, the analysis of the raw data shows a different distribution of AS types between the measurements in the control and the data-plane, with a higher percentage of stub-ASes and IXPs in the data-plane, illustrated in Figure \ref{fig:as_types}.
The difference stems from the observation mode; the data-plane can observe stub-ASes because some of the stubs host an Atlas probe and are thus visible, even though they do not forward traffic. IXPs are visible because routers in their peering LAN reply to ICMP messages, even if they do not append to BGP AS paths. In contrast, IXPs and stub ASes are not visible on the control-plane. Stub ASes are not visible on the control-plane because they do not forward traffic to our destination AS. IXPs do not append their number to the control-plane path, and hence are not visible on the control plane.


\section{Results}
\label{sec:results}

For our applied inferential statistics, we distinguished between ratio and ordinal data. 
The estimation percentages for 2D direction and gradient are ratio data, while the Likert items -- including task load -- are ordinal data.
For ratio data only, we first applied a Shapiro-Wilk test to check for normality.
We found that none of our ratio data is normally distributed.
Thus, we treated all our data in the same way and directly applied non-parametric tests, specifically Friedman tests.
Thereafter, we conducted Wilcoxon Signed-rank tests with Bonferroni correction for our post-hoc analysis.
The effect sizes of the Wilcoxon tests are reported as r (r: $>$0.1 small, $>$0.3 medium, and $>$0.5 large effect).

\subsection{Estimation of 2D Direction}
We asked participants to estimate the two-dimensional direction on a ground plane.
The median (interquartile range) percentages of correct 2D direction estimations for each condition are (in descending order): \conB=93.3\% (IQR=12.5\%), \conA=91.7\% (IQR=13.3\%), and \conC=78.3\% (IQR=16.7\%). 
All percentages are compared in Figure~\ref{fig:boxplots:direction}.
Since our data is not normally distributed (p$<$0.01), we directly ran a Friedman test that revealed a significant effect of condition on 2D direction estimation ($\chi^2$(2)=17.70, p$<$0.001, N=14). 
Post-hoc tests showed significant differences between \conA~and \conC~(W=83, Z=2.62, p$=$0.018, r=0.50) as well as \conB~and \conC~(W=0, Z=-3.30, p$<$0.001, r=0.62).
However, we did not find a significant difference between \conA~and \conB~(W=15, Z=-1.88, p$=$0.182).
Here, \textbf{we can conclude that both \conA and \conB~result in better estimation performance for 2D direction than \conC.}

\begin{table*}
\caption{Pairwise comparisons for individual statements, Bonferroni-adjusted, p-values: $<$0.05 (*), $<$0.01 (**), and $<$0.001 (***).}
\label{tab:pairwise_statements}
\begin{tabular}{l|ll|lll|lll}
 & \multicolumn{2}{l|}{Rabbit Single vs. Dual} & \multicolumn{3}{l|}{Rabbit Single vs. Motion Intensity} & \multicolumn{3}{l}{Rabbit Dual vs. Motion Intensity} \\
Statement & test statistic & p-value & test statistic & p-value & effect size & test statistic & p-value & effect size \\ \hline
S1 & Z=~0.00 & p=1.000 & Z=2.89 & p\textless{}.001\textbf{***} & r=0.55 & Z=3.04 & p=.004\textbf{**} & r=0.57 \\
S2 & Z=-0.07 & p=1.000 & Z=2.58 & p=.029\textbf{*} & r=0.49 & Z=2.80 & p=.013\textbf{*} & r=0.53 \\
S3 & Z=-1.17 & p=0.838 & Z=3.06 & p=.003\textbf{**} & r=0.58 & Z=2.97 & p=.003\textbf{**} & r=0.56 \\
S4 & Z=-0.17 & p=1.000 & Z=2.84 & p=.006\textbf{**} & r=0.54 & Z=2.69 & p=.018\textbf{*} & r=0.51
\end{tabular}
\end{table*}



\subsection{Estimation of Gradient}
We asked participants to estimate the gradient behavior of the communicated cue.
The median (interquartile range) percentages of correct gradient estimations for each condition are (in descending order): \conB=93.3\% (IQR=5.8\%), \conA=91.7\% (IQR=8.3\%), and \conC=56.7\% (IQR=10.0\%). 
All percentages are compared in Figure~\ref{fig:boxplots:gradient}.
Since our data is not normally distributed (p$<$0.001), we ran a Friedman test that revealed a significant effect of condition on gradient estimation ($\chi^2$(2)=19.00, p$<$0.001, N=14). 
Post-hoc tests showed significant differences between \conA~and \conC~(W=102, Z=3.11, p$=$0.002, r=0.59) as well as \conB~and \conC~(W=0, Z=-3.30, p$<$0.001, r=0.62).
However, we did not find a significant difference between \conA~and \conB~(W=30, Z=-1.42, p$=$0.501).
Here, \textbf{we can conclude that both \conA and \conB~result in better gradient estimation performance than \conC.}


\begin{figure*}
    \centering
    \includegraphics[width=\linewidth]{figures/plot_likert.pdf}
    \captionsetup{justification=justified}
    \vspace{-1em}
    \caption{Participant responses to the four rated statements (Likert items ranging from 1: strongly disagree to 7: strongly agree).}
    \Description{The results (four figures in two rows with two figures per row) of the participant' statements "Directions Easy to Distinguish", "Directions Easy to Derive From Vibration", "Change of Gradient Easy to Perceive", and "Easier to Perceive Over Time", visualized in a stacked bar plot for Likert items.}
    \label{fig:likert}
\end{figure*}

\subsection{Task Load}
The results of task load ratings as measured by the \ac{RTLX}~\cite{hart1988} are shown in Figure~\ref{fig:boxplots:taskload}. 
The median (interquartile range) task load scores for each condition are (in ascending order): \conA=22.5 (IQR=12.7), \conB=24.5 (IQR=7.9), and \conC=28.3 (IQR=20.0).
We ran a Friedman test that revealed a significant effect of condition on task load ($\chi^2$(2)=13.50, p$=$0.001, N=14).
Post-hoc tests showed a significant difference between \conB~and \conC~(W=105, Z=3.30, p$<$0.001, r=0.62).
However, we did not find any significant differences between \conA~and \conB~(W=39, Z=0.00, p$=$1.000) or between \conA~and \conC~(W=20, Z=-2.04, p=0.120).
Here, \textbf{we can conclude that \conB~induces a lower task load than \conC.}

\subsection{Individual Statements and Preferences}
After each condition, we asked participants to rate four statements, each on a 7-point Likert scale (1=strongly disagree, 7=strongly agree). The results and statements are shown in \autoref{fig:likert}. We found significant main effects for all four statements (N=14; S1: $\chi^2$(2)=14.09, p$<$0.001; S2: $\chi^2$(2)=11.35, p$=$0.003; S3: $\chi^2$(2)=17.08, p$<$0.001; S4:$\chi^2$(2)=12.79, p$=$0.002). Pairwise comparisons are shown in \autoref{tab:pairwise_statements}. Here, \textbf{we can conclude that \conA and \conB~are rated significantly more positively than \conC~for all four statements}. No difference was found between \conA and \conB. Regarding \emph{overall preference}, \textbf{eight participants preferred \conB}, while \textbf{six voted for \conA}~as their favorite. None of the participants preferred \conC.


\subsection{Interviews}
During the interviews, participants were explicitly asked to comment on the duration of the vibration as well as what may have eased or hindered their comprehension. They also had to explain their overall preference and comment on the overall experience and sensation of interpreting 3D directional cues via vibrotactile feedback.
For the analysis, the verbal data was first transcribed by one author and then summarized. The statements were then counted for each question. In addition, across all questions, we applied open coding to identify hidden themes. Data from one interview (P2) was not recorded due to a technical issue. Therefore, only the data from 13 participants was included.

Regarding the duration of the vibration, both \emph{Rabbit} conditions were perceived as having adequate duration (\conA: 10 vs 3 who thought it could have been longer; \conB: 13:0), while 10 participants would have preferred a longer duration for \conC. For the latter, participants struggled to feel the gradient correctly, as mentioned by five participants (e.g., P7 said that the \enquote{[duration was] a little bit short, enough for [2D] direction, but for intensity [gradient] it was really bad.}) The varying strength of the vibration was also an issue, as the most distant control point was criticized as having a too weak vibration, which meant that \enquote{some vibrations got lost} (P5). This also interfered with the comprehension of 2D direction. The smooth transition of movement in \conC was still found to be a pleasant experience, but the mentioned drawbacks regarding the gradient detection prevailed, according to P4 (RQ2). When comparing pulse with intensity for the mapping of gradient, P12 noted an interesting further advantage of pulse, as \enquote{One could decide about the gradient in retrospect even if one wasn't sure before. When the last actuator vibrated many times, then it must have been an upwards gradient.} This also implicitly highlights the problem of immediacy, which required attention and did not allow repetition of the feedback. As P10 put it, \enquote{in case you did not fully pay attention, there wasn't a repeat to make sure.} This sentiment was echoed by P12. 
Consequently, the dual encoding of a gradient in the \conB~condition was cited by most as the main reason for preferring that condition (RQ1). P11 noted, \enquote{I did not just have the number of pulses, but in addition the intensity and that somehow better stuck in my head.}
\section{Propagation of Invalid Paths}\label{sc:invalid:paths}
In this section, we consider the impact of the ROV ASes that we collected in our study on the propagation of invalid BGP announcements.
Our goal is to complement the findings in our measurements by quantifying the impact of ROV-enforcement in the observed ASes and IXPs. Graph analysis gives insights into how far the invalid paths can reach, the scope of the affected networks, and the impact of ROV on reachability. We also examine which parts of the Internet are not protected and which networks play a central role in blocking hijacks, providing global protection. To answer these questions, we develop a new graph-based analysis for measuring the propagation of invalid paths, using data-plane paths that we found in our measurements. We compare the Internet graph used by valid updates to the reduced Internet graph for invalid updates, which only includes vertices and edges without ROV-enforcement. We then analyze the differences between the graphs to derive conclusions about the impact of ROV on the propagation characteristics of invalid updates. We quantify the security of specific nodes and a general reduction of graph connectivity resulting from fewer available propagation paths. The analysis includes standard graph metrics like the number of sub-components, the node degree, the algebraic connectivity, and the average shortest- and longest-path length. 

\subsection{Graph Generation}
The graphs for the analyses are derived from the paths observed in the data-plane. The graph directly reflects the routes identified in our measurements, constituting a subset of the real-world Internet graph. 

{\bf Representing neighboring ASes.} We represent the paths and ASes as an undirected, non-cyclic graph.
Each AS on any path in the measurement is represented as a vertex in the graph, excluding IXPs. Connections between ASes that are neighbors on a path are represented as edges of two types:

\textit{Direct edges} are created from direct neighbors on a path, i.e., ASes that are topologically located in consecutive positions on the path. The edges represent a form of direct peering between the ASes, and it is expected that no intermediate party can influence the path propagation over that edge. 

\textit{Indirect edges} are edges over IXPs. These edges have one or multiple hops between the respective AS routers that belong to the peering LAN of an IXP. The ASes are neighbors in the graph because they have a peering relationship, either with a direct peering session or over a routeserver. Indirect edges differ from direct edges because they may run over a routeserver and thus be removed from the ROV graph, even if the connected ASes do not enforce ROV.


{\bf Graphs.} The resulting fully connected graph $G_1$ consists of 2156 nodes and 3810 edges. A second graph $G_2$ is created from $G_1$ to model the propagation of invalid updates by augmenting $G_1$ with information about ROV-enforcement.

In $G_2$, all edges to nodes that enforce ROV are removed from the graph as they filter out and drop invalid updates in real-world path propagation. The resulting graph is a subset of $G_1$ with the same amount of nodes but a reduced number of edges. Differential analysis of the two graphs offers insight into how much ROV impacts the graph structure and protects contained ASes. An attacker that announces a hijacked prefix can only use propagation paths in $G_2$ to reach victims , as all nodes in $G_1$ that enforce ROV would block the hijack.

An additional graph $G_3$ is created from $G_1$ to quantify the impact of ROV-enforcement in IXP routeservers. All indirect edges suspected of running over a routeserver are marked as ROV-enforcing and removed from $G_3$. The removal includes all indirect edges that only propagated valid paths in the data-plane measurement. The graph $G_3$ thus represents a scenario where ROV is only enforced in observed IXP routeservers.

\subsection{Graph Analysis}
The impact of ROV is quantified by comparing the three graphs with respect to the graph metrics. Calculating the graph metrics yields the results presented in table \ref{tab:graph}. 

\begin{table}[t!]
\renewcommand{\arraystretch}{0.6}
    \centering
    \footnotesize
\begin{tabular}{l|P{1cm}|P{1cm}|P{1cm}}
%\hline
\textbf{Graph Parameters} &  \textbf{$G_1$} & \textbf{$G_2$} & \textbf{$G_3$} \\\hline \hline
Vertices & 2156 & 2156 & 2156 \\ \hline 
Edges & 3810 & 1974 & 3173  \\ \hline
Components & 1 & 808 & 35 \\ \hline
Largest Component & 2156 & 1315 & 2110  \\ \hline
Avg. Node-Degree & 1.77 & 0.90 & 1.47  \\ \hline
Avg. Algebraic-Connectivity & 187.97 & 6.29 & 21.68  \\ \hline
Avg. Shortest-Path Length & 4.55 & 2.97 & 5.00 \\ \hline
Avg. Longest-Path Length & 9.52 & 5.78 & 9.34  \\% \hline
\vspace{-10pt}
\end{tabular}
\caption{Graph metrics for presented graphs.}
\vspace{-10pt}
\label{tab:graph}
\end{table}


\textbf{Impact ROV-enforcement on ASes.} Comparing metrics on $G_1$ and $G_2$ indicates that ROV substantially affects the measured Internet graph. ROV removed almost half of all edges for invalid updates, significantly reducing the graph's connectivity. The algebraic connectivity confirms that connectivity is decreased by more than an order of magnitude, showing a less dense mesh of connections inside the graph. ROV disconnected 808 components from the main graph. These components can be seen as isolated domains for updates; invalid messages can only spread to other parts of the component but not reach other components of the graph. The domain is also protected from any invalid updates from outside vertices. The average shortest path length between nodes in the graph is significantly reduced even though the graph is less connected, which is a direct result of the high prevalence of isolated components. The value is calculated as the average shortest path length to each reachable node from a vertex, which directly depends on the average component size. As paths inside the components are, on average, smaller than in the initial connected graph, the value reduces. 

The average longest path length, i.e., the shortest distance to the furthest distanced node in the graph for each vertex, is decreased by almost 40\%. Thus invalid updates have, on average, a 40\% shorter possible maximum AS path length than valid updates, which indicates that most invalid updates cannot propagate globally and attacks stay localized, close to the attacking AS. This reduction is in large part caused by ROV-enforcement in the Tier-1 providers. They are responsible for propagating updates over long distances across countries and continents. ROV implementation in these ASes reduces the propagation of updates from a global to a local level, as intercontinental propagation is severely limited without using Tier-1 providers. Our analysis shows that 580 edges in the graph run over an ROV-enforcing Tier-1 provider. Removing these enforcing edges is responsible for 30\% of the average longest-path reduction in $G_2$.

The graph analysis also reveals the limitations of ROV deployment on the modern Internet. ROV-enforcing ASes cannot disrupt the connectivity of the entire graph, and a significant central component of 1315 ASes remains connected in $G_2$. The remaining component can be attributed to the design principle of the Internet as a high-availability network. The Internet is a dense network of connections with a substantial amount of redundant edges, which offers robustness against outages caused by node- or edge failures. However, this design also limits the impact of ROV in single ASes. Only the removal of a large majority of ASes could result in the breakdown of the strongly-connected central component of the Internet. ASes close to the Internet's core need to implement ROV themselves for reliable protection against hijacks, as the dense mesh of connections will likely provide a propagation path for an invalid update through the Internet core, even if many ASes enforce ROV. This observation does not imply that ROV-enforcement will not significantly impact the graph. A study by Cohen et al. \cite{cohen2001breakdown} showed that removing central nodes from a scale-free network, in our case these are the Tier-1 providers for the Internet graph, can still significantly affect the connectedness and reachability of nodes in the graph. Thus ROV in central components impacts the propagation of invalid updates, even if a sizeable connected component remains on the Internet. The existence of the central components can be seen in the node degree distribution of both graphs in Table \ref{tab:graph}. 

The comparison between the graphs shows that ROV limits the impact of attacks by reducing connectivity and propagation of invalid updates on today's Internet. It localizes most attacks and hinders the global spread of hijacks by removing essential edges for global connectivity. The results indicate that ROV-enforcement in Tier-1 providers significantly impacts the spread of invalid updates as these central components play a crucial role in invalid update propagation.


\textbf{Impact ROV-enforcement on IXPs.} ROV-enforcement in IXPs does not show a similar impact to ROV in large providers. An upper limit of 637 edges in the measurement are marked as possibly running over an ROV routeserver and are removed from the graph, which is a significantly lower amount of removed edges than for $G_2$. The lower amount also reflects in the number of isolated components; only 34 components are disconnected from the main graph. The likelihood that an AS or all its upstream providers are solely connected over an ROV-enforcing routeserver appears to be too low to disrupt most parts of the graph significantly. Most ASes that we observed at IXPs or their upstream providers have direct peering sessions that leak invalid updates over the IXP, even if the AS has some or most peering connections over the routeserver. The results also show that routeserver connections provide considerable connectivity to the graph. ROV-enforcement reduces the algebraic connectivity by an order of magnitude. Invalid updates have a less dense mesh of paths available and need to take longer paths to their target, which also reflects in the increase of shortest path length in $G_3$. Longer path lengths and reduced connectivity lower the effectiveness of attacks because ASes may prefer shorter paths less and thus the hijacking announcements would lose against legitimate path announcements. Still, the protection is lower than the removal of far-reaching propagation paths. Graph $G_3$ has a minor decrease in average longest-path length; updates can propagate almost as far as in the baseline graph, even if they might have to take longer paths.

The current implementation of ROV in routeserver thus has a limited impact on the global spread of routes, while the protection they offer locally is substantial. Routeservers reduce the connectivity for invalid updates as they limit the available propagation paths to direct peering sessions. However, the prevalence of the direct sessions at today's IXPs is sufficient to allow the propagation of most invalid announcements to wider parts of the Internet. The routeservers only marginally reduce the maximum reach of hijacks, as updates leak over direct sessions and are propagated by global providers that usually do not peer at IXPs and routeservers. The effect of routeserver ROV is thus mainly localized, reducing the local connectivity for invalid updates and preventing the spread of the hijack to connected ASes that run only routeserver peering. Routeservers should thus be considered as a measure to reduce the spread of hijacks for protecting local ASes, but they cannot mitigate the global spread of hijacks in a similar capacity to Tier-1 providers.
We have presented a theoretical and empirical analysis of the impact of different CNF-ization approaches on SAT %\ignoreinshort{\GMCHANGE{
and SMT %}}
enumeration, %\ignoreinshort{\GMCHANGE{, 
both disjoint and non-disjoint. %}}. 
We have shown how the most popular transformations conceived for SAT %\ignoreinshort{\GMCHANGEp{
and SMT %}} 
solving, namely the Tseitin and the Plaisted and Greenbaum CNF-izations, prevent the solver from producing short partial assignments, thus seriously affecting the effectiveness of the enumeration. To overcome this limitation, we have proposed to preprocess the formula by converting it into NNF before applying the Plaisted and Greenbaum transformation. We have shown, both theoretically and empirically, that the latter approach can fully overcome the problem and can drastically reduce both the number of partial assignments and the execution time.

% we plan to further investigate the
% impact of CNF conversion also on disjoint SMT enumeration. We expect that in
% this domain the impact can be even more relevant, since in SMT
% multiple instances of the same theory atoms are typically rarer than for atoms in
% the Boolean case. Also, disjoint SMT enumeration has a fundamental role in Weighted Model Integration~\cite{morettin-wmi-ijcar17,morettin-wmi-aij19,spallittaSMTbasedWeightedModel2022}, an important framework for probabilistic inference in hybrid domains. Hence, we believe that our contribution can have a great impact on this application, where non-CNF formulas occur frequently.
% Finally, we think that work should be done to understand the impact on enumeration with repetitions, i.e.\ where models may not be disjoint, for instance in Predicate Abstraction~\cite{lahiriSMTTechniquesFast2006}. %Moreover, there is an alternative definition of partial assignment satisfiability that is based on the notion of \emph{entailment}~\cite{sebastianiAreYouSatisfied2020,mohleFourFlavorsEntailment2020}. Understanding the impact of the CNF conversion on solvers that use this definition is an interesting direction.

This work opens an interesting research avenue: investigate the role
of CNF-ization in neighbor fields as d-DNNF compilation and model
counting, possibly adapting d-DNNF compilers and model counters
so that to exploit different forms of CNF-izations.


\section*{Acknowledgements}
This work has been co-funded by the German Federal Ministry of Education and Research and the Hessen State Ministry for Higher Education, Research and Arts within their joint support of the National Research Center for Applied Cybersecurity ATHENE and by the Deutsche Forschungsgemeinschaft (DFG, German Research Foundation) SFB~1119.



{
\footnotesize
\bibliographystyle{IEEEtran}
\bibliography{main.bib,ref,bib,sec}
}

\end{document}
\endinput
