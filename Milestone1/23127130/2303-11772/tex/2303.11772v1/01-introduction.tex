\section{Introduction}

{\bf BGP prefix hijacks.} The Internet consists of Autonomous Systems (ASes) connected with the Border Gateway Protocol (BGP) [RFC1105]. Routers in each AS send BGP announcements to notify other networks how to reach IP addresses within prefixes that they own. BGP announcements are not authenticated, hence border routers can issue announcements claiming to originate {\em any} Internet prefix. Such bogus announcements can be a result of benign misconfigurations or malicious attacks \cite{china:telecom,fb:out,turkey:hijack}. ASes accepting bogus announcements send the traffic via invalid paths to the hijacker instead of the legitimate destination \cite{bellovin1989security}. BGP prefix hijacks allow adversaries to intercept, manipulate, and blackhole communication \cite{ballani2007study,vervier2015mind}. 

{\bf Filtering invalid routes with RPKI.} To prevent prefix hijacks, the IETF standardized the Resource Public Key Infrastructure (RPKI) [RFC6480]. The RPKI authenticates ownership over prefixes by binding prefixes to AS numbers (ASNs) and to public keys, creating Route Origin Authorizations (ROAs). The ROAs are stored in RPKI publication points. To filter bogus announcements, ASes should enforce Route Origin Validation (ROV): use relying party validators to periodically fetch and validate ROAs, and use these validation results in border routers to make routing decisions in BGP. 
Although ROV is critical for preventing hijacks, the deployment of ROV has only seen a moderate pace after its introduction in 2013. In recent years, the deployment took off with the adoption of ROV by Internet Exchange Points (IXPs) and large providers. {\em Our goal is to understand how ROV in different network types affects propagation of invalid paths and how effective ROV deployments are in blocking hijacks.}


{\bf Measurements of ROV.} Due to its significant role to Internet security, understanding the fraction of ASes that enforce ROV poses an important research question. In this work we measure ROV via a combination of control and data-plane measurements using RIPE Atlas\footnote{\url{atlas.ripe.net}}, similarly to \cite{hlavacek2018practical,rodday2021revisiting}.
 
We create invalid ROAs that conflict with the BGP announcements for our prefixes and hence appear like prefix hijacks.
To identify ASes that change their routing to our prefixes, we inspect control-plane paths in the global BGP routing table and measure the data-plane routes that the traffic takes to our prefixes. ASes whose routing to our invalid prefixes is not affected do not enforce ROV. 
In contrast to other approaches for measuring ROV, which we discuss in Related Work, Section \ref{sc:works}, this approach provides the best coverage of the Internet, is scalable, and does not require volunteers. We also improve the data analysis and acquisition of previous work to eliminate random routing events, therefore reducing the high fraction of false negatives/positives in previous research \cite{hlavacek2018practical,rodday2021revisiting}.

{\bf Business incentives for ROV enforcement.} In addition to the improved methodology, we also characterize the ASes that enforce ROV in our measurements according to their type and size. Through our analysis we find a correlation between the business model of ASes and ROV enforcement, and show that this correlation is aligned with the business incentives of BGP: 

{\em Large ASes, Internet Service Providers (ISPs) and IXPs have increased motivation to enforce ROV, since they get paid for providing connectivity services. Consequently, when a prefix is hijacked, they lose traffic and corresponding payment. Prefix hijacks affect their business model}. In contrast, stub-ASes do not provide upstream connectivity to other networks, hence do not have a business incentive to enforce ROV themselves. 

% hijack vs propagation
{\bf Effectiveness in blocking invalid paths.} Our goal is not to merely understand if hijacks are possible, e.g. like \cite{gilad2017we}, but
to gain insights into how far the invalid routes can reach, the scope of the affected networks, the impact of ROV on reachability of ASes, which parts of the Internet are not protected, and which networks play a central role in providing global protection against hijacks. We evaluate the effectiveness of current ROV deployments through analysis of the propagation of invalid routes across different network types. Although there are suggestions that ROV at routeservers of IXPs provides an effective defence against prefix hijacks \cite{reuter2018towards,rodday2021revisiting}, we show for the first time that IXPs do not block global propagation of invalid routes. Routeservers at IXPs perform control-plane functions interconnecting border routers of customer ASes, to manage peerings and to guarantee protection to the customers against BGP hijacks by dropping invalid routes via ROV. Outsourcing the management of peerings and blocking of hijacks with ROV to the IXP made the routeservers extremely popular. We show that IXPs cannot prevent leakage of invalid paths globally because they do not have control over the traffic routed through their IP space over direct sessions. In fact, we find that the average direct peering sessions in the top five IXPs propagate 3.4x more paths than sessions over a routeserver, inevitably leaking invalid routes. In contrast, we find that ROV enforcement in Tier-1 providers is most effective in blocking global propagation of invalid routes.

{\bf Research questions.}  In our research we aim to answer the following questions. 

$\bullet$ The enforcement of ROV is changing at a rapid pace. What is the current ROV deployment rate in the Internet?

$\bullet$ What limitations do existing methodologies for measuring ROV have, what measurement bias do they introduce, and how can they be improved? 

$\bullet$ What are the differences between control and data plane methodologies, what is the overlap, and what are the factors that cause the differences?

$\bullet$ Are there differences in ROV enforcement between different networks and geo-locations?

$\bullet$ In which networks is ROV enforcement most effective for blocking hijacks?

{\bf Ethical considerations.} In order to identify ASes that enforce ROV, we carry out active BGP prefix hijacks in the global Internet and measure which ASes accept the routes in our hijacking announcements. Our experiments are ethical; we hijack {\em only} the prefixes that we own. Our experiments do not introduce additional load on other networks.

{\bf Contributions.} Conceptually, our research shows that IXPs play a much smaller role in blocking invalid routes than {\updated indicated by previous research \cite{rodday2021revisiting}, which concluded that most ROV enforcement is performed in the IXPs}. 
In contrast, our analysis demonstrates that ROV in Tier-1 providers significantly reduces the global propagation of invalid routes limiting the spread to a localized scope. We find that current ROV deployments do not provide sufficient protection against prefix hijacks and are not resilient to attacks and failures. Our technical contributions are:

$\bullet$ {\em Improved ROV measurements:} We improve the data acquisition and extraction processes used in previous ROV measurement studies \cite{hlavacek2018practical,rodday2021revisiting}. %Our data acquisition is more scalable, provides better coverage and eliminates random routing events. 
For data analysis, we introduce an AS classification scheme and divergence points into our methodology; both significantly reduce false positives and negatives inherent in previous measurements \cite{hlavacek2018practical,rodday2021revisiting}. We provide our dataset and instructions for reproducing our study at \url{https://sit4.me/rpki}.


$\bullet$ {\em Invalid paths over IXPs:} We performed the first study of the effectiveness of routeserver-ROV in blocking invalid paths. Our measurements covered 159 IXPs, including the largest European IXPs, and found route leaks over them. 

$\bullet$ {\em Propagation of invalid routes:} We develop the first graph-based analysis of ROV effectiveness on limiting the propagation of invalid paths. Using our analysis, we evaluate the outreach of invalid paths on the Internet and identify networks whose ROV filtering provides effective global protection.  

{\bf Organization.} We review RPKI in Section \ref{sc:overview} and compare our research to Related Work in Section \ref{sc:works}. We introduce our ROV-measurement methodology in Section \ref{sc:method}. The setup and execution of the experiments are explained in Section \ref{sc:measurements}, and the results of ROV measurement are presented in Section \ref{sc:rov}. We quantify the invalid paths that traverse the IXPs in Section \ref{sc:routeservers}. In Section \ref{sc:invalid:paths}, we analyze the effectiveness of ROV filtering on blocking the propagation of invalid routes in the Internet. We conclude our research in Section \ref{sc:conclusions}. 

