\section{Overview of RPKI}\label{sc:overview}

RPKI provides authenticated prefix ownership information, which routers can use for making routing decisions.

{\bf RPKI objects.} To authorize their network resources, ASes can create resource certificates that bind their resources to a public key contained inside a Route Origin Authorization. The ROA is signed with the certificate of a Certificate Authority (CA). RPKI objects are published in RPKI repositories hosted on publication points. The publication points are operated either in a hosted mode by one of the Regional Internet Registries (RIRs) or in a delegated mode by a Local Internet Registry (LIR). An RPKI repository keeps a finite set of signed ROAs and additionally contains signed certificates (which point to children publication points), certificate revocation lists (CRLs), and manifests.

{\bf Traversal of RPKI repositories.} The validation of RPKI objects is performed with a relying party software, which contains hardcoded Trust Anchor Locators (TALs) to the root certificates of the RIRs. Each of the five RIRs operates its own RPKI trust anchor certificate and repository. During the validation, a relying party contacts every root repository known to it and downloads RPKI objects from every publication point it finds. The RPKI objects are fetched from RPKI repositories over RRDP or rsync protocols. After downloading the objects, a relying party performs cryptographic validation, which produces a list of tuples (AS, ROA prefix, prefix length) called Validated ROA Payloads (VRPs). The VRPs are stored in a local cache. 

{\bf Route Origin Validation.} The BGP border routers of an AS retrieve the VRPs from their relying party's cache over the `RPKI to Router Protocol' (RTR) [RFC8210]. The routers use the VRPs to validate incoming BGP announcements with Route Origin Validation. A router checks if the IP prefix block in the BGP announcement and the VRP IP prefix block are identical for the length specified by the VRP IP prefix length [RFC6811]. If the IP prefix in the announcement is covered by any VRP entry, the router checks if the BGP origin AS in the announcement matches the VRP AS for that prefix. Matching values result in the conclusion that the announcement is valid. In contrast, if any VRP covers the prefix in the BGP announcement, but the entry does not match the origin AS, then the announcement is invalid. The validation status is considered unknown if the BGP announcement is not covered by any VRP entry. 