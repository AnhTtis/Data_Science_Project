\section{Propagation of Invalid Paths}\label{sc:invalid:paths}
In this section, we consider the impact of the ROV ASes that we collected in our study on the propagation of invalid BGP announcements.
Our goal is to complement the findings in our measurements by quantifying the impact of ROV-enforcement in the observed ASes and IXPs. Graph analysis gives insights into how far the invalid paths can reach, the scope of the affected networks, and the impact of ROV on reachability. We also examine which parts of the Internet are not protected and which networks play a central role in blocking hijacks, providing global protection. To answer these questions, we develop a new graph-based analysis for measuring the propagation of invalid paths, using data-plane paths that we found in our measurements. We compare the Internet graph used by valid updates to the reduced Internet graph for invalid updates, which only includes vertices and edges without ROV-enforcement. We then analyze the differences between the graphs to derive conclusions about the impact of ROV on the propagation characteristics of invalid updates. We quantify the security of specific nodes and a general reduction of graph connectivity resulting from fewer available propagation paths. The analysis includes standard graph metrics like the number of sub-components, the node degree, the algebraic connectivity, and the average shortest- and longest-path length. 

\subsection{Graph Generation}
The graphs for the analyses are derived from the paths observed in the data-plane. The graph directly reflects the routes identified in our measurements, constituting a subset of the real-world Internet graph. 

{\bf Representing neighboring ASes.} We represent the paths and ASes as an undirected, non-cyclic graph.
Each AS on any path in the measurement is represented as a vertex in the graph, excluding IXPs. Connections between ASes that are neighbors on a path are represented as edges of two types:

\textit{Direct edges} are created from direct neighbors on a path, i.e., ASes that are topologically located in consecutive positions on the path. The edges represent a form of direct peering between the ASes, and it is expected that no intermediate party can influence the path propagation over that edge. 

\textit{Indirect edges} are edges over IXPs. These edges have one or multiple hops between the respective AS routers that belong to the peering LAN of an IXP. The ASes are neighbors in the graph because they have a peering relationship, either with a direct peering session or over a routeserver. Indirect edges differ from direct edges because they may run over a routeserver and thus be removed from the ROV graph, even if the connected ASes do not enforce ROV.


{\bf Graphs.} The resulting fully connected graph $G_1$ consists of 2156 nodes and 3810 edges. A second graph $G_2$ is created from $G_1$ to model the propagation of invalid updates by augmenting $G_1$ with information about ROV-enforcement.

In $G_2$, all edges to nodes that enforce ROV are removed from the graph as they filter out and drop invalid updates in real-world path propagation. The resulting graph is a subset of $G_1$ with the same amount of nodes but a reduced number of edges. Differential analysis of the two graphs offers insight into how much ROV impacts the graph structure and protects contained ASes. An attacker that announces a hijacked prefix can only use propagation paths in $G_2$ to reach victims , as all nodes in $G_1$ that enforce ROV would block the hijack.

An additional graph $G_3$ is created from $G_1$ to quantify the impact of ROV-enforcement in IXP routeservers. All indirect edges suspected of running over a routeserver are marked as ROV-enforcing and removed from $G_3$. The removal includes all indirect edges that only propagated valid paths in the data-plane measurement. The graph $G_3$ thus represents a scenario where ROV is only enforced in observed IXP routeservers.

\subsection{Graph Analysis}
The impact of ROV is quantified by comparing the three graphs with respect to the graph metrics. Calculating the graph metrics yields the results presented in table \ref{tab:graph}. 

\begin{table}[t!]
\renewcommand{\arraystretch}{0.6}
    \centering
    \footnotesize
\begin{tabular}{l|P{1cm}|P{1cm}|P{1cm}}
%\hline
\textbf{Graph Parameters} &  \textbf{$G_1$} & \textbf{$G_2$} & \textbf{$G_3$} \\\hline \hline
Vertices & 2156 & 2156 & 2156 \\ \hline 
Edges & 3810 & 1974 & 3173  \\ \hline
Components & 1 & 808 & 35 \\ \hline
Largest Component & 2156 & 1315 & 2110  \\ \hline
Avg. Node-Degree & 1.77 & 0.90 & 1.47  \\ \hline
Avg. Algebraic-Connectivity & 187.97 & 6.29 & 21.68  \\ \hline
Avg. Shortest-Path Length & 4.55 & 2.97 & 5.00 \\ \hline
Avg. Longest-Path Length & 9.52 & 5.78 & 9.34  \\% \hline
\vspace{-10pt}
\end{tabular}
\caption{Graph metrics for presented graphs.}
\vspace{-10pt}
\label{tab:graph}
\end{table}


\textbf{Impact ROV-enforcement on ASes.} Comparing metrics on $G_1$ and $G_2$ indicates that ROV substantially affects the measured Internet graph. ROV removed almost half of all edges for invalid updates, significantly reducing the graph's connectivity. The algebraic connectivity confirms that connectivity is decreased by more than an order of magnitude, showing a less dense mesh of connections inside the graph. ROV disconnected 808 components from the main graph. These components can be seen as isolated domains for updates; invalid messages can only spread to other parts of the component but not reach other components of the graph. The domain is also protected from any invalid updates from outside vertices. The average shortest path length between nodes in the graph is significantly reduced even though the graph is less connected, which is a direct result of the high prevalence of isolated components. The value is calculated as the average shortest path length to each reachable node from a vertex, which directly depends on the average component size. As paths inside the components are, on average, smaller than in the initial connected graph, the value reduces. 

The average longest path length, i.e., the shortest distance to the furthest distanced node in the graph for each vertex, is decreased by almost 40\%. Thus invalid updates have, on average, a 40\% shorter possible maximum AS path length than valid updates, which indicates that most invalid updates cannot propagate globally and attacks stay localized, close to the attacking AS. This reduction is in large part caused by ROV-enforcement in the Tier-1 providers. They are responsible for propagating updates over long distances across countries and continents. ROV implementation in these ASes reduces the propagation of updates from a global to a local level, as intercontinental propagation is severely limited without using Tier-1 providers. Our analysis shows that 580 edges in the graph run over an ROV-enforcing Tier-1 provider. Removing these enforcing edges is responsible for 30\% of the average longest-path reduction in $G_2$.

The graph analysis also reveals the limitations of ROV deployment on the modern Internet. ROV-enforcing ASes cannot disrupt the connectivity of the entire graph, and a significant central component of 1315 ASes remains connected in $G_2$. The remaining component can be attributed to the design principle of the Internet as a high-availability network. The Internet is a dense network of connections with a substantial amount of redundant edges, which offers robustness against outages caused by node- or edge failures. However, this design also limits the impact of ROV in single ASes. Only the removal of a large majority of ASes could result in the breakdown of the strongly-connected central component of the Internet. ASes close to the Internet's core need to implement ROV themselves for reliable protection against hijacks, as the dense mesh of connections will likely provide a propagation path for an invalid update through the Internet core, even if many ASes enforce ROV. This observation does not imply that ROV-enforcement will not significantly impact the graph. A study by Cohen et al. \cite{cohen2001breakdown} showed that removing central nodes from a scale-free network, in our case these are the Tier-1 providers for the Internet graph, can still significantly affect the connectedness and reachability of nodes in the graph. Thus ROV in central components impacts the propagation of invalid updates, even if a sizeable connected component remains on the Internet. The existence of the central components can be seen in the node degree distribution of both graphs in Table \ref{tab:graph}. 

The comparison between the graphs shows that ROV limits the impact of attacks by reducing connectivity and propagation of invalid updates on today's Internet. It localizes most attacks and hinders the global spread of hijacks by removing essential edges for global connectivity. The results indicate that ROV-enforcement in Tier-1 providers significantly impacts the spread of invalid updates as these central components play a crucial role in invalid update propagation.


\textbf{Impact ROV-enforcement on IXPs.} ROV-enforcement in IXPs does not show a similar impact to ROV in large providers. An upper limit of 637 edges in the measurement are marked as possibly running over an ROV routeserver and are removed from the graph, which is a significantly lower amount of removed edges than for $G_2$. The lower amount also reflects in the number of isolated components; only 34 components are disconnected from the main graph. The likelihood that an AS or all its upstream providers are solely connected over an ROV-enforcing routeserver appears to be too low to disrupt most parts of the graph significantly. Most ASes that we observed at IXPs or their upstream providers have direct peering sessions that leak invalid updates over the IXP, even if the AS has some or most peering connections over the routeserver. The results also show that routeserver connections provide considerable connectivity to the graph. ROV-enforcement reduces the algebraic connectivity by an order of magnitude. Invalid updates have a less dense mesh of paths available and need to take longer paths to their target, which also reflects in the increase of shortest path length in $G_3$. Longer path lengths and reduced connectivity lower the effectiveness of attacks because ASes may prefer shorter paths less and thus the hijacking announcements would lose against legitimate path announcements. Still, the protection is lower than the removal of far-reaching propagation paths. Graph $G_3$ has a minor decrease in average longest-path length; updates can propagate almost as far as in the baseline graph, even if they might have to take longer paths.

The current implementation of ROV in routeserver thus has a limited impact on the global spread of routes, while the protection they offer locally is substantial. Routeservers reduce the connectivity for invalid updates as they limit the available propagation paths to direct peering sessions. However, the prevalence of the direct sessions at today's IXPs is sufficient to allow the propagation of most invalid announcements to wider parts of the Internet. The routeservers only marginally reduce the maximum reach of hijacks, as updates leak over direct sessions and are propagated by global providers that usually do not peer at IXPs and routeservers. The effect of routeserver ROV is thus mainly localized, reducing the local connectivity for invalid updates and preventing the spread of the hijack to connected ASes that run only routeserver peering. Routeservers should thus be considered as a measure to reduce the spread of hijacks for protecting local ASes, but they cannot mitigate the global spread of hijacks in a similar capacity to Tier-1 providers.