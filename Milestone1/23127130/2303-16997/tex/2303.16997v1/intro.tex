
List colouring, a natural generalization of vertex colouring, was first introduced in the 1970s by by Erd\H{os}, Rubin, and Taylor \cite{erdos1979choosability}, and independently, by Vizing \cite{vizing}.

\begin{definition}
A \emph{list assignment $L$} for a graph $G$ is a function that assigns to each $v \in V(G)$ a list $L(v)$ of colours.  $L$ is a \emph{$k$-list assignment} if $|L(v)| \geq k$ for every $v \in V(G)$. An \emph{$L$-colouring} of $G$ is a function $\phi$ such that $\phi(v) \in L(v)$ for each $v\in V(G)$, and $\phi(u) \neq \phi(v)$ for each $uv \in E(G)$. We say $G$ is \emph{$L$-colourable} if there exists an $L$-colouring of $G$, and that $G$ is \emph{$k$-list-colourable} (or \emph{$k$-choosable}) if $G$ is $L$-colourable for every $k$-list assignment $L$ for $G$.
\end{definition}

As compared to ordinary colouring, we think of a list assignment as \emph{localizing} the possible images of the colouring function to each vertex.

Thomassen famously proved that every planar graph is 5-list colourable, settling a conjecture posed by Vizing \cite{vizing} and Erd\H{o}s, Rubin, and Taylor \cite{erdos1979choosability}.
\begin{thm}[Thomassen, \cite{thomassen5LC}]\label{5choosable}
Every planar graph is 5-list-colourable.
\end{thm}

This theorem is best possible for planar graphs: in 1993, Voigt gave a construction of a planar graph that is not 4-list-colourable \cite{voigt1993list}. It is an easy consequence of Euler's formula for graphs embedded in surfaces that planar graphs of girth at least four (i.e. without triangles) are 4-list colourable; again, Voigt showed this is best possible in the sense that there exists a planar graph of girth at least four that is not 3-list-colourable \cite{voigt1995not}. When we rule out both triangles and 4-cycles, however, lists of size three suffice.

\begin{thm}[Thomassen, \cite{thomassen3LC}]\label{3choosable}
Every planar graph of girth at least five is 3-list-colourable.
\end{thm}

It is natural to wonder whether these results carry over to graphs embedded in surfaces other than the sphere. To partially answer this question, we require the following definitions.

\begin{definition}
A \emph{non-contractible cycle} in a surface is a cycle that cannot be continuously deformed to a single point. An embedded graph is \emph{$\rho$-locally planar} if every cycle (in the graph) that is non-contractible (in the surface) has length at least $\rho$.
\end{definition}
This is closely related to the concept of \emph{edge-width}. Recall that the edge-width of an embedded graph is the length of the shortest non-contractible cycle; thus if a graph is $\rho$-locally planar, it has edge-width at least $\rho$.

 In 2006, DeVos, Kawarabayashi, and Mohar \cite{devos2006locally} showed that for every surface $\Sigma$, there exists a constant $\rho= 2^{O(g)}$, where $g$ is the Euler genus of $\Sigma$, such that every $\rho$-locally planar graph that embeds in $\Sigma$ is 5-list-colourable. A similar result for 5-colourability (rather than list-colourability) was proved by Thomassen in 1993 \cite{thomassen1993five}. Per the work of Postle and Thomas \cite{postle2018hyperbolic}, analogous results for 5-list-colouring, 4-list-colouring of graphs of girth at least four, and 3-list-colouring of graphs of girth at least five (with $\rho = \Omega(\log(g))$) are implied by the \emph{hyperbolicity} of certain associated families of graphs. \emph{Hyperbolicity} is defined below; $(G, \Sigma)$ is a graph $G$ embedded in a surface $\Sigma$.


\begin{definition}\label{def:hyp}
Let $\mathcal{F}$ be a family of embedded graphs. We say that $\mathcal{F}$ is \emph{hyperbolic} if there exists a constant $c > 0$ such that if $(G, \Sigma) \in \mathcal{F}$ is an embedded graph, then for every closed curve $\eta : S^1 \rightarrow \Sigma$ that bounds an open disk $\Delta$ and intersects $G$ only in vertices, if $\Delta$ includes a vertex of $G$, then the number of vertices of $G$ in $\Delta$ is at most $c(|\{x \in S^1 : \eta(x) \in V (G)\}| - 1)$. We say that $c$ is a \emph{Cheeger constant} for $\mathcal{F}$.
\end{definition}

In \cite{postle2018hyperbolic}, Postle and Thomas give a theorem known as the \emph{hyperbolic structure theorem}, which characterises the structure of graphs in hyperbolic families. We state the theorem below informally, both to help give the reader intuition regarding hyperbolicity and to better explain the implications of hyperbolicity for locally planar graphs. For more information (and a more formal description of what is meant below), we encourage the reader to consult \cite{postle2018hyperbolic}.

\begin{thm}[Theorem 6.29, \cite{postle2018hyperbolic} (informally stated)]\label{hypstrthm}
Let $\mathcal{F}$ be a hyperbolic family of embedded graphs, and let $(G, \Sigma) \in \mathcal{F}$. Let $g$ be the Euler genus of $\Sigma$. The graph $G$ decomposes into a graph with $O(g)$ vertices together with a set of $O(g)$ cylinders of edge-width $O(1)$.
\end{thm}

This theorem together with the hyperbolicity of certain families of graphs implies list-colouring results for locally planar graphs. To explain this further, we again require a few definitions.


\begin{definition}
Let $G$ be a graph, $k$ a positive integer, and $L$ a $k$-list assignment for $G$. We say $G$ is \emph{$L$-critical} if every proper subgraph of $G$ admits an $L$-colouring, but $G$ itself does not. If there exists a $k$-list assignment $L'$ such that $G$ is $L'$-critical, we say $G$ is \emph{critical for $k$-list-colouring}.
\end{definition}

In 2013, Dvo\v{r}\'ak and Kawarabayashi \cite{dvovrak2013list} showed the family of embedded graphs of girth at least five that are critical for 3-list-colouring is hyperbolic. Postle and Thomas showed the same for the family of embedded graphs of girth at least four that are critical for 4-list-colouring \cite{postle2018hyperbolic}; and in 2016 \cite{postle2016five}, for the family of embedded graphs that are critical for 5-list-colouring.

Theorem \ref{hypstrthm} together with these hyperbolicity results is enough to prove that, given a surface $\Sigma$ with genus $g$, for each $k\in \{3,4,5\}$ there exists an integer $\rho$ with $\rho = O(g)$ such that $\rho$-locally planar graphs embeddable in $\Sigma$ are $k$-list-colourable. In \cite{postle2018hyperbolic}, Postle and Thomas showed that with more work, hyperbolicity in fact implies analogous results with $\rho = \Omega(\log(g))$ instead of $O(g)$. As discussed in \cite{postle2018hyperbolic}, this bound is best possible.


We are interested in generalizing these results to the framework of \emph{correspondence colouring.} Correspondence colouring is a natural generalization of list colouring introduced by Dvo{\v{r}}{\'a}k and Postle in 2018 \cite{dvovrak2018correspondence}. Since then, it has been extensively studied: see for example \cite{abe2021differences, bernshteyn2016asymptotic,bernshteyn2018sharp,bernshteyn2019differences,bernshteyn2017dp,liu2019dp,zhang2021edge}.
It is defined as follows.
\begin{definition}
 Let $G$ be a graph. A \emph{$k$-correspondence assignment for $G$} is a $k$-list assignment $L$ together with a function $M$ that assigns to every edge $e = uv \in E(G)$ a partial matching $M_e$ between $\{u\}\times L(u)$ and $\{v\}\times L(v)$.
An $(L,M)$-colouring of $G$ is a function $\varphi$ that assigns to each vertex $v \in V(G)$ a colour $\varphi(v) \in L(v)$ such that for every $e = uv \in E(G)$, the vertices $(u, \varphi(u))$ and $(v, \varphi(v))$ are non-adjacent in $M_e$. We say that $G$ is $(L,M)$-colourable if such a colouring exists, and that $G$ is \emph{$k$-correspondence-colourable} if $G$ is $(L,M)$-colourable for every $k$-correspondence assignment $(L,M)$ for $G$.
\end{definition}

Below, we generalize the notion of criticality to correspondence colouring.
\begin{definition}\label{def:critcorcol} Let $G$ be a graph, $k$ a positive integer, and $(L,M)$ a $k$-correspondence assignment for $G$. We say $G$ is \emph{$(L,M)$-critical} if every proper subgraph of $G$ admits an $(L,M)$-colouring, but $G$ itself does not. If there exists a $k$-correspondence assignment $(L',M')$ such that $G$ is $(L',M')$-critical, we say $G$ is \emph{critical for $k$-correspondence colouring}.
\end{definition}

A correspondence assignment can be thought of as a further localization of colouring: just as list colouring localizes the notion of what colours are available at a vertex, a correspondence assignment localizes the \emph{meaning} of these colours. Many list-colouring theorems carry over to the correspondence colouring framework with only semantic modifications. For instance, as pointed out by Dvo{\v{r}}{\'a}k and Postle \cite{dvovrak2018correspondence}, Theorems \cite{thomassen5LC} and \cite{thomassen3LC}  hold for correspondence colouring: planar graphs are 5-correspondence-colourable, and planar graphs of girth at least five are 3-correspondence-colourable.


The main result of this paper is a technical theorem (Theorem \ref{theorem:stronglinear}) that implies the following.
\begin{restatable}{thm}{5cchyperbolic}\label{5cchyperbolic}
The family of embedded graphs that are critical for 5-correspondence colouring is hyperbolic.  
\end{restatable}

Theorem \ref{theorem:stronglinear}\textemdash which implies Theorem \ref{5cchyperbolic}\textemdash uses similar ideas to that of the analogous theorem for list colouring of Postle and Thomas (Theorem 4.6, \cite{postle2016five}); however, a number of new ideas and reductions are needed in order to make the proof go through in the correspondence colouring framework. This is discussed further in Section \ref{sec:challenges}.

Per the work of Postle and Thomas \cite{postle2018hyperbolic}, Theorem \ref{5cchyperbolic} implies the following.
\begin{restatable}{thm}{locallyplanar}\label{locallyplanar}
For every surface $\Sigma$, there exists a constant $\rho > 0$ such that every $\rho$-locally planar graph that embeds in $\Sigma$ is 5-correspondence-colourable.
\end{restatable}

We note that this result is new, and without hyperbolicity, it is unclear how one would prove it. For the theorem above, $\rho = \Omega(\log(g))$ where $g$ is the Euler genus of $\Sigma$. See Section \ref{sec:implications} for further details. We note this bound for $\rho$ is best possible: since (as noted above) this bound is best possible for list colouring, it immediately follows that it is best possible for correspondence colouring.


Along with this implication for locally planar graphs, hyperbolicity implies a host of other interesting results (as shown by Postle and Thomas \cite{postle2018hyperbolic}). We highlight two more of these below. In a follow-up paper, we will demonstrate how Theorem \ref{theorem:stronglinear} (which implies that the family of embedded graphs that are critical for 5-correspondence colouring is hyperbolic) can also be used to show that planar graphs have exponentially many 5-correspondence colourings, proving a conjecture of Langhede and Thomassen \cite{langhede2021exponentially}. Our method can also be used to prove analogous bounds on the number of list-colourings or correspondence colourings of other classes of planar graphs, assuming the existence of a theorem analogous to Theorem \ref{theorem:stronglinear}. However, in these instances other (often better) bounds were already known (see for example \cite{bosek2022graph} and, more recently, \cite{dahlberg2023algebraic} for lower bounds on the number of 3-list colourings and 3-correspondence colourings, respectively, of planar graphs of girth 5). This is not the case for 5-correspondence colouring. 

In \cite{postle2016five}, Postle and Thomas show that the list-colouring analogue to Theorem \ref{theorem:stronglinear} has implications for the \emph{precolouring extension problem} for planar graphs. The problem can be stated as follows: given a planar graph $G$ with list (or correspondence) assignment $L$ (or $(L,M)$) and subgraph $C$ of $G$, when does an arbitrary $L$ (or $(L,M)$) colouring of $C$ extend to $G$? 

One way to approach the problem is to try to quantify the amount of computation required to determine whether or not the colouring will extend. In particular: can we bound the size of a subgraph $H$ with $H \subseteq G$ such that every colouring of $C$ that extends to $H$ also extends to $G$? Note that a subgraph $H$ with this property always exists: $G$ is such a subgraph. To limit the computation required to answer the decidability question presented above, it is useful to study the minimal subgraphs $H$ with this property. These subgraphs serve as small certificates for the decidability problem.

Postle and Thomas show the following, settling a conjecture of Dvo\v r\'ak et al.~\cite{dvovrak20175}.
\begin{restatable}[Postle and Thomas, \cite{postle2016five}]{thm}{lukethm}\label{lukethm}
Let $G$ be a plane graph with outer cycle $C$, let $L$ be a 5-list assignment for $G$, and let $H$ be a minimal subgraph of $G$ such that every $L$-colouring of $C$ that extends to an $L$-colouring of $H$ also extends to an $L$-colouring of $G$. Then $H$ has at most $19|V(C)|$ vertices.
\end{restatable}

In 1997, Thomassen \cite{thomassen1997color} proved a similar theorem for ordinary colouring, showing $|V(H)| \leq 5^{|V(C)|^3}$. In 2010, Yerger \cite{yerger2010color} improved Thomassen's bound to $O(|V(C)|^3)$. We note that a linear bound in terms of the number of vertices in the precoloured subgraph is asymptotically best possible.

In 2011, Dvo\v r\'ak and Kawarabayashi gave an analogous theorem to Theorem \ref{lukethm} for 3-list-colouring below.
\begin{thm}[Dvo\v r\'ak and Kawarabayashi, \cite{dvorak2011choosability}]\label{dvorakcriticalbound} 
Let $G$ be a plane graph of girth at least five and with outer cycle $C$, let $L$ be a 3-list assignment for $G$, and let $H$ be a minimal subgraph of $G$ such that every $L$-colouring of $C$ that extends to an $L$-colouring of $H$ also extends to an $L$-colouring of $G$. Then $H$ has at most $\frac{37}{3}|V(C)|$ vertices.
\end{thm}

These theorems suggest that, given these graphs and list assignments, there is a small subgraph $H$ that encodes the answer to the precolouring extension problem for cycles: that is, if a cycle $C$ in a plane graph $G$ is precoloured and we wish to determine whether this colouring extends to $G$, there exists a small subgraph $H$ such that it suffices to check whether the colouring extends to $H$. 

We show in Section \ref{sec:implications} that Theorem \ref{theorem:stronglinear} implies the following result.
\begin{restatable}{thm}{theoremlinearcycle}\label{theorem:linearcycle}
Let $G$ be a plane graph with outer cycle $C$, let $(L,M)$ be a 5-correspondence assignment for $G$, and let $H$ be a minimal subgraph of $G$ such that every $(L,M)$-colouring of $C$ that extends to an $(L,M)$-colouring of $H$ also extends to an $(L,M)$-colouring of $G$. Then $H$ has at most $51 |V(C)|$ vertices.
\end{restatable}

The final implications of hyperbolicity that will be discussed in Section \ref{sec:implications} involve algorithms for the decidability of the colouring problem for embedded graphs: Dvo\v r\'ak and Kawarabayashi \cite{dvovrak2013list} also gave linear-time\footnote{The algorithms' running times are linear with respect to the number of vertices in the graph.} algorithms for the decidability of 3-list-colouring of embedded graphs of girth at least five.  Their algorithms can be modified to allow the precolouring of a subgraph $H$, at the cost of increasing the time complexity of the algorithm to $O(|V(G)|^{k(g+s)+1})$ where $k$ is some absolute constant,  $g$ is the genus of the surface in which the graph is embedded, and $s$ is the number of components in $H$. This modification ensures the algorithms find a colouring, should it exist. Theorem \ref{hypstrthm} helps guide the structure of the algorithms: the algorithms roughly attempt to decompose embedded graphs into subgraphs as described in Theorem \ref{hypstrthm}, and find colourings that extend to these subgraphs via dynamic programming. For details, see \cite{dvovrak2013list}. The algorithms rely on Theorem \ref{dvorakcriticalbound}; and per \cite{dvovrak2013list}, these algorithms can be adapted to other settings where a linear bound analogous to that in Theorem \ref{dvorakcriticalbound} holds.

In particular, Theorem \ref{lukethm} thus implies the existence of linear algorithms for deciding the 5-list-colouring of embedded graphs, and Theorem \ref{theorem:linearcycle} implies the following.

\begin{restatable}{thm}{algone}
\label{alg1}
Let $\Sigma$ be a fixed surface. There exists a linear-time algorithm that takes as input an embedded graph $(G, \Sigma)$ and 5-correspondence assignment $(L,M)$ for $G$ with lists of bounded size and determines whether or not $G$ is $(L,M)$-colourable.
\end{restatable}
\begin{restatable}{thm}{algtwo}\label{alg2}
Let $\Sigma$ be a fixed surface. There exists a linear-time algorithm that takes as input an embedded graph $(G, \Sigma)$ and determines whether or not $G$ is 5-correspondence-colourable.
\end{restatable}
Note that in Theorem \ref{alg1} the correspondence assignment $(L,M)$ is fixed, whereas in Theorem \ref{alg2} it is not. We note that these algorithmic results are new; and in fact, prior to this paper, it was not known whether there existed poly-time algorithms (let alone linear algorithms) for the decidability of 5-correspondence colouring embedded graphs.


As mentioned prior, we obtain Theorem \ref{theorem:linearcycle} as a consequence of a more technical theorem (Theorem \ref{theorem:stronglinear}), the proof of which constitutes the bulk of Section \ref{sec:mainthm}. We delay the statement of Theorem \ref{theorem:stronglinear} until Subsection \ref{subsec:mainthmstatement}, when we will have built up the necessary background and terminology.


We further observe in Section \ref{sec:ggeq5} that the embedded graphs $G$ of girth at least five that are critical for 3-correspondence colouring form a hyperbolic family. This follows from observing that the proof for list colouring in \cite{postle20213} also holds for correspondence colouring with only minor modifications. This is discussed further in Section \ref{sec:ggeq5}.

As discussed above, the hyperbolicity of this family of graphs (as well as related theorems) has many interesting implications.  As in the case for 5-correspondence colouring, we highlight the following three.

\begin{restatable}{thm}{algonefive}
\label{alg15}
Let $\Sigma$ be a fixed surface. There exists a linear-time algorithm that takes as input an embedded graph of girth at least five $(G, \Sigma)$ and a 3-correspondence assignment $(L,M)$ for $G$ with lists of bounded size and determines whether or not $G$ is $(L,M)$-colourable.
\end{restatable}
\begin{restatable}{thm}{algtwofive}\label{alg25}
Let $\Sigma$ be a fixed surface. There exists a linear-time algorithm that takes as input an embedded graph of girth at least five $(G, \Sigma)$ and determines whether or not $G$ is 3-correspondence-colourable.
\end{restatable}
\begin{restatable}{thm}{locallyplanarfive}\label{locallyplanar5}
For every surface $\Sigma$, there exists a constant $\rho > 0$ such that every $\rho$-locally planar graph of girth at least five that embeds in $\Sigma$ is 3-correspondence-colourable.
\end{restatable}


Subsection \ref{subsec:outline}, below, gives an outline of the rest of the paper. 


%---------------------------------------------------------------------------------------------
%--------------------------------------------OUTLINE OF PAPER
%---------------------------------------------------------------------------------------------
\subsection{Outline of Paper}\label{subsec:outline}


In Section \ref{sec:challenges}, we discuss as a high level some of the challenges involved in the proof of Theorem \ref{theorem:stronglinear}, and in particular, where our proof differs from the analogous theorem for list colouring in \cite{postle2016five}. In Subsection \ref{subsec:criticalgraph}, we establish a few basic results and definitions used in the proof of Theorem \ref{theorem:stronglinear}.  Section \ref{sec:prelims} contains three subsections: Subsection \ref{subsec:criticalgraph} introduces \emph{critical canvases}, our main object of study. Subsection \ref{subsec:deficiency} introduces the notion of \emph{deficiency}, a measurement used throughout the paper. Subsection \ref{subsec:mainthmstatement} establishes yet more useful definitions and results, and concludes with the statement of our main theorem, Theorem \ref{theorem:stronglinear}, which in turn implies Theorem \ref{theorem:linearcycle}, Theorem \ref{5CCconstant}, and Theorem \ref{hyperbolic5cc}: that the family of graphs that are critical for 5-correspondence colouring is hyperbolic. Many proofs in these sections are taken from \cite{postle2016five}, where they were originally written for list colouring. When the list colouring proof directly carries over to the correspondence colouring framework without modification, we omit the proofs in the interest of brevity and refer the reader to 
 \cite{postle2016five}. In these cases, the omitted proofs are purely structural and do not mention colourings. Section \ref{sec:mainthm} contains the proof of Theorem \ref{theorem:stronglinear}. Many of the proofs in this section differ from the analogous results in \cite{postle2016five}. In particular, starting in Subsection \ref{subsec:thirdlayer}, our proof diverges completely from that of Theorem 4.6 in \cite{postle2016five}.  Section \ref{sec:implications} establishes several important consequences of Theorem \ref{theorem:stronglinear}; finally, Section \ref{sec:ggeq5} discusses analogous results for 3-correspondence colouring graphs of girth at least five.

%Section \ref{sec:implications} makes use of the following theorem, due to Thomassen.
%\begin{thm}[Thomassen \cite{thomassen5LC}]\label{tech5CC} Let $G$ be a planar graph with outer face boundary walk $C$. Let $S$ be a path of length at most one contained in $C$. Let $(L,M)$ be a correspondence assignment for $G$ where $|L(v)| \geq 5$ for all $v \in V(G) \setminus V(C)$, and where $|L(v)| \geq 3$ for all $v \in V(C) \setminus V(S)$. Every $(L,M)$-colouring of $S$ extends to an $(L,M)$-colouring of $G$. 
%\end{thm}
%Thomassen originally stated this for list colouring. However, as pointed out by Dvo{\v{r}}{\'a}k and Postle in \cite{dvovrak2018correspondence}, the proof carries over to correspondence colouring.

 
%---------------------------------------------------------------------------------------------------
%---------------------------------------------------------------------------------------------------
%--------------------------HYPERBOLICITY----------------------------------------------------
%---------------------------------------------------------------------------------------------------
%---------------------------------------------------------------------------------------------------
\section{Challenges and Main Ideas in the Proof of Theorem \ref{theorem:stronglinear}}\label{sec:challenges}

Our proof of Theorem~\ref{theorem:stronglinear}, the main result of this paper, follows the basic framework laid out by Postle and Thomas in \cite{postle2016five} to prove the analogous theorem for list colouring. As discussed in \cite{postle2016five}, the main idea is to bound the number of vertices in a critical graph in terms of the sum of the sizes of large faces: this is the concept Postle and Thomas call ``deficiency''. The proof of Theorem \ref{theorem:stronglinear} also involves counting the number of vertices that share an edge or a face with vertices in the outer cycle of the graph. In keeping track of these quantities, we are able to perform various reductions, showing a minimum counterexample to Theorem \ref{theorem:stronglinear} must have a very specific structure and ultimately that a minimum counterexample cannot exist.

The bulk of the arguments present in the proof of Postle and Thomas' list colouring version of Theorem \ref{theorem:stronglinear} carry over to correspondence colouring with only minor modifications. This is largely due to the fact that many of the arguments are structural, and do not rely on the specific list assignment. However, there are a few key points at which the arguments fail for correspondence colouring. In particular, Claims 5.23 and 5.24 in \cite{postle2016five} argue that the lists of specific vertices in a minimum counterexample are subsets of one another. For a triangle $ux_2z_2u$ in a minimum counterexample with list assignment $S$, Claim 5.23 shows that $S(u) \subseteq S(x_2)$.  Claims 5.27 and 5.28 then use the fact that $S(z_2) \setminus (S(x_2) \cup S(u)) = S(z_2) \setminus S(x_2)$. This (along with an argument showing $S(z_2) \setminus S(x_2)$ is non-empty) implies that it is possible to colour $z_2$ from $S(z_2)$ while avoiding the lists of both $x_2$ and $u$. This argument crucially does not hold for correspondence colouring: an analogous argument to that in Claim 5.23 shows merely that for a correspondence assignment $(S,M)$, we have $|M_{x_2u}| = |S(u)|$, which of course implies nothing about $M_{z_2u}$. As a consequence of this, we are unable to use the reductions found in \cite{postle2016five}, and must instead develop an entirely new set of reductions that can be performed in the correspondence colouring framework. This adds considerable length and intricacy to the proof.

The proof of Theorem \ref{theorem:stronglinear} has two main parts: the first involves purely the structure of a minimum counterexample $G$ to Theorem \ref{theorem:stronglinear}, and the second involves arguing about the specific matchings $\{M_e: e \in E(G)\}$ of the correspondence assignment $(L,M)$ of $G$. Several of our structural results are taken directly from the analogous theorems for list colouring in \cite{postle2016five}: whenever possible, we omit these purely structural proofs in the interest of brevity, referring the reader instead to \cite{postle2016five}. Some of these arguments involve the set of vertices $X_1$ that have at least three neighbours in the outer cycle $C$ of $G$, as well as the set of vertices $X_2$ with at least three neighbours in $V(C) \cup X_1$ and at least one neighbour in $X_1$. We note that the fact that $X_1 \neq \emptyset$ is an easy consequence of Theorem \ref{tech5CC}, a technical theorem due to Thomassen that implies that planar graphs are 5-correspondence colourable. Moreover, as we are able to show that vertices in $X_1$ have exactly three neighbours in $V(C)$; that vertices not on the outer face boundary of $G$ have degree at least five; an that no edge in $G$ has both endpoints in $X_1$, it follows similarly from Theorem \ref{tech5CC} that $X_2$ is non-empty. Informally, we think of the sets $X_1$ and $X_2$ as ``layers'' near the outer cycle $C$. These two layers alone do not provide us with enough freedom to force a contradiction in the second part of the proof, unlike in the proof of the analogous theorem for list colouring given in \cite{postle2016five}. Our analysis thus involves moving one layer further into the graph, and considering the structure surrounding vertices in the set $X_3$ of vertices with at least three neighbours in $V(C) \cup X_1 \cup X_2$ and at least one neighbour in $X_2$. That $X_3$ is non-empty follows from similar reasoning as $X_2$. The proof before this point is very similar to that of Postle and Thomas in \cite{postle2016five}. It is from this point on \textemdash the introduction of this third ``layer'', $X_3$, in Subsection \ref{subsec:thirdlayer} \textemdash that the proof diverges substantially. From this point on, the lemmas and other results in the proof of Theorem \ref{theorem:stronglinear} have no analogues in \cite{postle2016five}.

In the second part of the proof, we argue about the matchings in the correspondence assignment. In particular, Claim \ref{keyclaim} establishes very precisely the matchings between vertices $x_1 \in X_1$, $x_2 \in X_2$, and $x_3 \in X_3$ as well as their other neighbours in the graph. We use this claim to finish the proof, showing that for one edge $e \in E(G)$, we have that $M_e$ is not a matching. This contradicts the definition of correspondence assignment, and thus dispels the existence of a minimum counterexample to Theorem \ref{theorem:stronglinear}.




