In this section, we redefine \emph{canvas} as follows.
\begin{definition}
We say the triple $(G,S,(L,M))$ is a \emph{canvas} if $G$ is a plane graph, $S$ is any connected subgraph of $G$, and $(L,M)$ is a correspondence assignment for $G$ such that there exists an $(L,M)$-colouring of $S$ and $G$ has girth at least five and $|L(v)| \geq 3$ for all $v \in V(G) \setminus V(S)$.
\end{definition}
The main result of this section is the following observation.
\begin{obs}\label{girth5:stronglinear}
Let $\varepsilon,\alpha >0$ satisfy the following:  $9\varepsilon \leq \alpha$; $2.5 \alpha + 5.5 \varepsilon \leq 1$; and $11 \varepsilon + 1 \leq 3\alpha$.
If $T = (G, S, L)$ is a critical canvas where:
\begin{itemize}[itemsep=0pt]
    \item $G$ has girth at least five,
    \item $G$ is not composed of exactly $S$ and one edge not in $S$,
    \item $G$ is not composed of exactly $S$ together with one vertex of degree $3$, then
\end{itemize}  then $3e(T)- (5 + \varepsilon)v(T)-\alpha q(T) \geq 3$. \end{obs}

This is the correspondence colouring analogue of a nearly identical theorem for list colouring of Postle: Theorem 3.9, \cite{postle20213}. Beyond the change from list colouring to correspondence colouring, the key difference between the statements of Theorem \ref{girth5:stronglinear} and Theorem 3.9 in \cite{postle20213} is that $S$ is connected (as opposed to having at most two components). This change allows us to use Theorem 2.11 in \cite{postle20213} (which describes structures arising from critical canvases $(G,S,(L,M))$ where $S$ is connected, and which holds for correspondence colouring) in lieu of Theorem 2.12 (which allows $S$ to have two components, and which is not currently known to hold for correspondence colouring). Otherwise, the proof of Theorem 3.9 in \cite{postle20213} carries over to the correspondence colouring framework with only standard, minor changes: namely, when we perform reductions (colouring a strict subgraph of a minimum counterexample, deleting this subgraph, and removing vertices' colours from neighbours' lists), we delete \emph{corresponding} colours from neighbouring lists, rather than identical colours.

The proof is similar in spirit to that of Theorem \ref{theorem:stronglinear}. However, as noted in Section \ref{sec:challenges}, Postle and Thomas' list colouring theorem in the 5-choosability case does \emph{not} carry over to correspondence colouring. The colouring arguments in Postle and Thomas' theorem for 5-choosability rely on the fact that for a triangle $ux_2z_2u$ in a minimum counterexample with list assignment $S$, if $S(u) \subseteq S(x_2)$, then $S(z_2) \setminus (S(x_2) \cup S(u)) = S(z_2) \setminus S(x_2)$. This implies that it is possible to colour $z_2$ from $S(z_2)$ while avoiding the lists of both $x_2$ and $u$. This argument crucially does not hold for correspondence colouring: an analogous argument to that in shows merely that for a correspondence assignment $(S,M)$, we have $|M_{x_2u}| = |S(u)|$, which of course implies nothing about $M_{z_2u}$. Crucially, the proof in the 5-choosability case involves keeping track of lists along a cycle. This is not the case in Postle's proof for 3-choosability: the colouring arguments involve only deleting vertices and removing their colours (or in the correspondence framework, their \emph{corresponding} colours) from the lists of neighbours, and do not keep track of what these colours correspond to. Moreover, no arguments rely on keeping track of what colours are or are not available in a cycle: the colouring arguments involve only trees branching from vertices in $S$ in the minimum counterexample.  


Theorem \ref{girth5:stronglinear} implies the following (the correspondence colouring analogue of Theorem 1.8 in \cite{postle20213}).
\begin{obs}\label{girth5:3ccbound} 
Let $G$ be a plane graph of girth at least five, let $(L,M)$ be a 3-correspondence assignment for
$G$, and let $C$ be a facial cycle of $G$. If $G$ is $C$-critical with respect to $(L,M)$, then $|V(G)| \leq 89|V(C)|$.
\end{obs}


Observation \ref{girth5:3ccbound} in turn implies an important corollary below. First, we will need the following theorem, due to Thomassen. This theorem was originally written in the language of list colouring; however, as pointed out by Dvo{\v{r}}{\'a}k and Postle in \cite{dvovrak2018correspondence}, the proof also carries over to the realm of correspondence colouring.
\begin{thm}[Thomassen, \cite{thomassen3LCnew}]\label{thomtech3cc}
Let $G$ be a plane graph of girth at least five. Let $C$ be the subgraph of $G$ whose edge- and vertex-set are precisely those of the outer face boundary walk of $G$. Let $(L,M)$ be a correspondence assignment for $G$ where $|L(v)| \geq 1$ for each vertex $v$ in a path or cycle $S \subseteq C$ with $|V(S)|\leq 6$; where $|L(v)| = 2$ for each vertex $v$ in an independent set $A$ of vertices in $|V(C) \setminus V(S)|$; where $|L(v)|\geq 3$ for all $v \in V(G) \setminus (A \cup V(S))$; and where there is no edge between vertices in $A$ and vertices in $S$. Then every $(L,M)$-colouring of $S$ extends to an $(L,M)$-colouring of $G$.
\end{thm}


The following corollary follows from Observation \ref{girth5:3ccbound}. The proof is very similar to that of Theorem \ref{hyperbolic5cc}, but uses Theorem \ref{thomtech3cc} instead of Theorem \ref{tech5CC}. See Lemma 5.13 in \cite{postle2018hyperbolic} for a proof of the list colouring case, which is nearly identical.
\begin{cor}\label{girth5:hyp}
The embedded graphs of girth at least five that are critical for 3-correspondence colouring form a hyperbolic family.
\end{cor}
%\begin{proof}
%\textcolor{red}{Let $(G, \Sigma)$ be an embedded graph of girth at least five that is $(L,M)$-critical, where $(L,M)$ is a 3-correspondence colouring. Note that $G$ is connected, and by Theorem \ref{thomtech3cc}, $\Sigma$ is not the plane. Let $\lambda: \mathbb{S}^1 \rightarrow \Sigma$ be a closed curve intersecting $G$ in only its vertices and bounding an open disk $\Delta$. Let $Y$ be the set of vertices of $G$ that are intersected by $\lambda$, and let $X$ be the set of vertices in $\Delta$.  The theorem follows by showing that if $X$ is non-empty, then $|X| \leq 395 \dot (|Y|-1)$. Let $G_1 := G[X \cup Y]$, and let $G_2 := G \setminus G_1$. Since $G$ is critical for 3-correspondence colouring, there exists a colouring of $G_2$ that extends to every proper subgraph of $G$ containing $G_2$ but not to $G$ itself. Since $G_1 \setminus Y \neq \emptyset$, it follows that $G_1$ is $G[Y]$-critical; and by Theorem \ref{thomtech3cc}, $|Y| \geq 3$. Let $v_0, v_1, \dots, v_k$ be the vertices in $Y$ appearing in a cyclic order along $\lambda$. Working modulo $k+1$, for each $i \in \{0, \dots, k\}$ and each non-adjacent pair $v_i, v_{i+1}$, let $P_i$ be the path $v_i u_i u_{i+1} v_{i+1}$. For each adjacent pair $v_i,v_{i+1}$, let $P_i = v_iv_{i+1}$. Let $C = \cup_{i=0}^{k} P_i$. Since $G_1$ is $G[Y]$-critical, it follows that $G_1 \cup C$ is $C$-critical. By Observation \ref{girth5:3ccbound}, $|V(G_1 \cup C)| \leq 89 |V(C)|$; or equivalently, that $|V(G_1)| \leq 88|V(C)|$. Since $|V(C)|\leq 3|Y|$, it follows that $|V(G_1)| \leq 264 |V(Y)|$; and since $G_1 = G[X \cup Y]$, we have further that $|X| \leq 263 |Y|$.  Since $|Y|\geq 3$, we have that $263 \leq 132(|Y|-1)$, and so $|X| \leq 263(|Y|-1) + 132(|Y|-1)$. Thus $|X| \leq 395 (|Y|-1)$, as desired. }
%\end{proof}

Similar to the girth 3 case, Corollary  \ref{girth5:hyp} implies Theorem \ref{locallyplanar5}, and Observation \ref{girth5:3ccbound} implies Theorems \ref{alg15} and \ref{alg25}.


