
In this section, we prove Theorem \ref{theorem:linearcycle}, and discuss the implications of this result. %We will use the following equivalent definition of deficiency.

%\begin{lemma}[Lemma 3.3, \cite{postle2016five}]\label{alternatedef}
%If $G$ is a 2-connected plane graph with outer cycle $C$, then 
%$$
%\defc(G) = |V(C)|-3-\sum_{f \in \mathcal{F}(G)}(|f|-3).
%$$
%\end{lemma}

%Theorem \ref{theorem:stronglinear} implies the following.
%\begin{thm}\label{6.1equiv}
%If $T=(G,C,(L,M))$ is a critical canvas, then 
%$$
%\varepsilon |V(G)\setminus V(C)| + \sum_{f \in \mathcal{F}(G)}(|f|-3) \leq |V(C)|-4.
%$$
%\end{thm}
%\begin{proof}
%If $T= (G,C, (L,M))$ is a critical canvas, then it follows from Lemma \ref{alternatedef} that $d(T) = %|V(C)|-3-\sum_{f \in \mathcal{F}(G)}(|f|-3)-\varepsilon v(G)-\alpha(b(T)+q(T))$. Thus $d(T) \leq |V(C)|-3-\sum_{f \in \mathcal{F}(G)}(|f|-3)-\varepsilon v(G)$.  By Theorem \ref{theorem:stronglinear}, if $v(T) \geq 2$ then $3-\gamma \leq d(T)$\textcolor{blue}{; and $3- \gamma \geq 1$ by (I3) in Theorem \ref{theorem:stronglinear}}. If $v(T) \leq 1$, then \textcolor{blue}{it follows immediately from the definition of $d(\cdot)$ that} $d(T) \geq 1$. Thus $1 \leq |V(C)|-3-\sum_{f \in \mathcal{F}(G)}(|f|-3)-\varepsilon v(G)$, and so 
%$$\varepsilon |V(G)\setminus V(C)| + \sum_{f \in \mathcal{F}(G)}(|f|-3) \leq |V(C)|-4,$$
%as desired.
%\end{proof}
%Omitting the \textcolor{blue}{sum over the faces} from Theorem \ref{6.1equiv} gives the theorem below.
\begin{thm}\label{5CCconstant}
If $T = (G,C,(L,M))$ is a critical canvas and $\varepsilon$ is as in Theorem \ref{theorem:stronglinear}, then $|V(G)| \leq \frac{1+\varepsilon}{\varepsilon}|V(C)|-\frac{4}{\varepsilon}$.
\end{thm}
\begin{proof}
%By Theorem \ref{6.1equiv}, \textcolor{blue}{$|V(G)\setminus V(C)| \leq \frac{1}{\varepsilon}|V(C)|$. The result follows.} %Thus $|V(G)| \leq \frac{1}{\varepsilon}|V(C)| + |V(C)|$, and so the result follows.

Let $F(G)$ be the set of faces in the embedding of $G$. Note that
\begin{align*}
    |E(G)| - 3|V(G)| &= |E(G)| - 3(|E(G)| + 2 - |F(G)|) \textnormal{ \hskip 4mm by Euler's formula} \\
 &= 3|F(G)| - 2|E(G)| - 6  \\
&= \sum_{f \in F(G)} (3-|f|) - 6 \\
&\leq (3-|V(C)|) - 6  \textnormal{\hskip 4mm since each $f \in F(G)$ has degree at least 3} \\
&= -|V(C)|-3.
\end{align*}
Moreover, by definition, $\defc(G) = |E(G)|-|E(C)|-3|V(G)|+3|V(C)|= |E(G)|-3|V(G)|+2|V(C)|$. Using the inequality above, it follows that $\defc(G) \leq (-|V(C)|-3)+2|V(C)$, or $\defc(G) \leq |V(C)|-3$.
%\textcolor{blue}{Note that by definition, $\defc(G) = |E(G)|-|E(C)|-3|V(G)|+3|V(C)|= |E(G)|-3|V(G)|+2|V(C)|$.}
%\textcolor{blue}{Let $\mathcal{F}(G)$ be the set of internal faces of $G$. By Euler's formula, $3|\mathcal{F}(G)| - 3 =3|E(G)| - 3|V(G)|$; or, adding $2|V(C)|$ to each side and rearranging, $|V(C)|-3+ 3|\mathcal{F}|-(2|E(G)|-|V(C)|) = |E(G)|-3|V(G)|+2|V(C)|$, which is precisely equal to our expression for $\defc(G)$, above. Moreover, $\sum_{f \in \mathcal{F}(G)}(|f|-3) = 2 |E(G)| -|V(C)| - 3 |\mathcal{F}(G)|$. Thus
%\begin{align*}
%    \defc(G) &= |V(C)|-3 + 3|\mathcal{F}|-(2|E(G)|-|V(C)|) \\
%    &= |V(C)| -3 - \sum_{f \in \mathcal{F}} (|f|-3) \\
%    &\leq |V(C)| - 3 \textnormal{ \hskip 3mm since internal faces have degree at least 3.}
%\end{align*}}
Now, $d(T) \leq \defc(G)-\varepsilon (|V(G)| - |V(C)|)$ by definition, and so since $\defc(G) \leq |V(C)|-3$, we have that $d(T) \leq |V(C)|-3-\varepsilon(|V(G)| - |V(C)|)$. By Theorem \ref{theorem:stronglinear}, if $v(T) \geq 2$ then $3-\gamma \leq d(T)$; and $3- \gamma \geq 1$ by (I3) in Theorem \ref{theorem:stronglinear}. If $v(T) \leq 1$, then it follows immediately from the definition of $d(\cdot)$ and the fact that internal vertices of $G$ have degree at least 5 (see Proposition \ref{Facts}) that $d(T) \geq 1$. Thus $1 \leq d(T) \leq |V(C)|-3-\varepsilon(|V(G)|-|V(C)|)$, or $4 \leq |V(C)|(1 + \varepsilon) - \varepsilon |V(G)|$. The result follows by rearranging.
\end{proof} 


To obtain the best possible bound in Theorem \ref{5CCconstant}, we wish to maximize $\varepsilon$ subject to inequalities (I1-I3). To that end, we choose  $\alpha = \frac{1}{25}$, $\varepsilon = \frac{1}{50}$, and $\gamma = \frac{7}{10}$, giving $V(G) \leq 51 |V(C)| $ in Theorem \ref{5CCconstant}.


Theorem \ref{theorem:linearcycle} follows from Theorem \ref{5CCconstant} as shown below.
\begin{proof}[Proof of Theorem \ref{theorem:linearcycle}]
Let $G, C, (L,M),$ and $H$ be as in Theorem \ref{theorem:linearcycle}. We claim that $H$ is $C$-critical. Suppose not. Then there exists a proper subgraph $H'$ of $H$ such that every $(L,M)$-colouring of $C$ that extends to $H'$ also extends to $H$. But since every $(L,M)$-colouring $C$ that extends to $H$ also extends to $G$, we have that $H'$ contradicts the minimality of $H$. Thus $H$ is $C$-critical, and so by Theorem \ref{5CCconstant} we have $|V(H)| \leq 51 |V(C)|$, as desired.
\end{proof}

We next show how Theorem \ref{5CCconstant} implies that the family of embedded graphs that are critical for 5-correspondence colouring forms a hyperbolic family. Note the theorem below is merely a more explicit version of Theorem \ref{5cchyperbolic}. Following this, we discuss the implications of the hyperbolicity of a family of graphs as described by Postle and Thomas in \cite{postle2018hyperbolic}.

\begin{thm}\label{hyperbolic5cc}
The family $\mathcal{F}$ of embedded graphs that are critical for 5-correspondence colouring is hyperbolic with Cheeger constant $50$.
\end{thm}
\begin{proof}
Let $G$ be a graph that is $(L,M)$-critical, where $(L,M)$ is a 5-correspondence colouring. Note that $G$ is connected, as otherwise since every subgraph of $G$ admits an $(L,M)$-colouring, it follows that each component of $G$ admits an $(L,M)$-colouring and thus so does $G$ itself, contradicting the definition of $(L,M)$-critical. Suppose that $G$ is embedded in a surface $\Sigma$, and let $\lambda: \mathbf{S}^1 \rightarrow \Sigma$ be a closed curve intersecting $G$ in only its vertices and bounding an open disk $\Delta$. Let $Y$ be the set of vertices of $G$ that are intersected by $\lambda$, and let $X$ be the set of vertices in $\Delta$. The theorem follows from showing that if $X$ is non-empty, then  $|X| \leq 50(|Y|-1)$.  Let $G_1 : = G[X \cup Y]$, and let $G_2 := G\setminus G_1$. Since $G$ is critical for 5-correspondence colouring, there exists a colouring of $G_2$ that extends every proper subgraph of $G$ containing $G_2$ but not to $G$ itself. Since $X  \neq \emptyset$, it follows that $G_1$ is $G[Y]$-critical. By Theorem \ref{tech5CC}, it follows that $|Y| \geq 3$. Let $v_0,v_1,v_2, \cdots, v_k$ be the vertices of $Y$ appearing in a cyclic order along $\lambda$. Let $C$ be the cycle $v_0v_1\cdots v_kv_0$. Since $G_1$ is $G[Y]$-critical, it follows that $G_1 \cup C$ is $C$-critical. By Lemma \ref{2connhyp}, $G_1 \cup C$ is 2-connected, and hence $(G_1, C, (L,M))$ is a canvas. By Theorem \ref{5CCconstant} with $\varepsilon = \frac{1}{50}$, we have that $|V(G_1)\setminus V(C)| \leq  50 (|V(C)|-1)$. The result follows.
\end{proof}

Showing that such a family of critical graphs is hyperbolic has many interesting implications, as described in \cite{postle2018hyperbolic}. We highlight a few in particular below, following a definition.

\begin{definition}
A \emph{non-contractible cycle} in a surface is a cycle that cannot be continuously deformed to a single point. An embedded graph is \emph{$\rho$-locally planar} if every cycle (in the graph) that is non-contractible (in the surface) has length at least $\rho$.
\end{definition} 

In \cite{postle2018hyperbolic}, Postle and Thomas show the following.

\begin{thm}[Postle \& Thomas, \cite{postle2018hyperbolic}]\label{lukelocplthm}
For every hyperbolic family $\mathcal{F}$ of embedded graphs that is closed under curve cutting there exists a constant $k > 0$ such that every graph $G \in \mathcal{F}$ embedded in a surface of Euler genus $g$ has a non-contractible cycle of length at most $k\log(g + 1)$.
\end{thm}

Using this, Theorem \ref{locallyplanar} follows as a corollary to Theorem \ref{hyperbolic5cc}.  In addition, following the work of Dvo{\v{r}}{\'a}k and Kawarabayashi in \cite{dvovrak2013list}, Theorem \ref{theorem:linearcycle} implies Theorems \ref{alg1} and \ref{alg2}. Note that by \emph{linear-time algorithms}, we mean algorithms whose run-time is linear in the number of vertices in the graph. 

The algorithms in the theorems above are the same as those given by Dvo{\v{r}}{\'a}k and Kawarabayashi in \cite{dvovrak2013list}. We refer to \cite{dvovrak2013list} for a complete description of the algorithms and the proof of correctness. An overview of the algorithms is given in Chapter 3, Section 6 of \cite{evethesis}. 
%\begin{definition}
%Given a graph $G$, a \emph{tree decomposition} is a tree $T$ where $V(T)$ is a set of subsets of $V(G)$ and $T$ satisfies the following properties:
%\begin{itemize}
%    \item $\bigcup_{U \in V(T)} U = V(G)$,
%    \item for each $uv \in E(G)$, there exists some $W \in V(T)$ such that $\{u,v\} \subseteq W$, and
%    \item for each $V, U \in V(T)$, if there exists a vertex $w \in V(G)$ with $w \in V \cap U$, then %$w \in Y$ for each vertex $Y \in V(T)$ in the unique path from $V$ to $U$ in $T$.
%\end{itemize}
%The \emph{width} of $T$ is defined as $\max_{V \in V(T)} |V|-1$. The \emph{tree-width} of $G$ is the minimum width across all tree decompositions of $G$.
%\end{definition}
%The idea behind the algorithm in Theorem \ref{alg1} is the following: given an embedded graph $(G, \Sigma)$,  we first find a subgraph $H$ of $G$ of tree-width linear in the genus of $\Sigma$ such that each component of $G-H$ is connected to $H$ via a cut of bounded size. This subgraph $H$ has specific structure in $G$: namely, $H$ has a bounded number of components, each component of $G-H$ is locally planar, and $H$ is formed by the union of short, non-contractible cycles connected by paths of bounded distance. The graph $H$ can be found via a linear-time algorithm of Dvo{\v{r}}{\'a}k, Kr{\'a}l', and Thomas \cite{dvovrak2009coloring}. Using Theorem \ref{theorem:linearcycle}, one can show that $H$-critical graphs in the components of $G-H$ have logarithmic distance to $H$. If a component of $G-H$ contains an $H$-critical subgraph $G'$, we claim $G'$ has tree-width logarithmic in $|V(G)|$. To see this, it suffices to add a new vertex $v$ adjacent to all vertices in $H \cap G'$. Note this increases the genus of $(H \cap G') \cup \{v\}$ by at most $2k$, where $k$ is the number of components of $H \cap G'$. We then use the following result of Eppstein.

%\begin{thm}[Eppstein, \cite{eppstein2000diameter}]
%There exists a constant $c$ such that every graph $G$ of Euler genus $g$ and radius $r$ has tree-width at most $c(g+1)r$. Furthermore, a tree decomposition of this width can be found in time $O((g+1)r|V(G)|)$.
%\end{thm}

%Thus $G'$ has tree-width $O(r)$, and since $r$ is logarithmic in $|V(H)|$ and $|V(H)| \leq |V(G)|$, it follows that $G'$ has tree-width logarithmic in $|V(G)|$. We find a tree decomposition of this width, and use a standard dynamic programming algorithm (see for instance \cite{JANSEN1997135}) to colour $H$, and extend this colouring to the components of $G'-H$. As $G'$ has bounded tree-width, the algorithm runs in $O(|V(G')|)$ time.


%The algorithm in Theorem \ref{alg2} is similar in spirit, but requires checking not only whether an $(L,M)$-colouring extends, but whether $(L,M')$-colourings extend for all possible sets of matchings $M'$. Note that a graph $G$ is 5-correspondence colourable if and only if it is 5-correspondence colourable for every 5-correspondence assignment $(L,M)$ with $|L(v)| = 5$ for all $v \in V(G)$ and $|M_{uv}| = 5$ for all $uv \in E(G)$. It thus suffices to check matchings $M'$ satisfying these requirements. Since the lists are of bounded size, the dynamic programming algorithm mentioned above runs in $O(|V(G)|)$ time. Note further that in correspondence colouring (as opposed to list colouring) we may assume that the list assignment is the same set of five colours for each vertex: the matchings between adjacent lists determine the meaning of these colours.
