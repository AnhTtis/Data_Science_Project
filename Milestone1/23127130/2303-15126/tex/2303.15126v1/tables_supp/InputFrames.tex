% \usepackage{multirow}
% \usepackage{multirow}


\begin{table}
\renewcommand\arraystretch{1.1}
\centering
\caption{\textbf{Quantitative results with different numbers of input frames for NeuralPCI.} Among them, \textit{2 frames} input indicates the pair frame setting and \textit{4 frames} input is the standard setting in our main paper. }
\label{tab:input frames}
\scalebox{0.9}{
\begin{tabular}{p{0.12\textwidth}<{\centering}p{0.05\textwidth}<{\centering}p{0.07\textwidth}<{\centering}p{0.05\textwidth}<{\centering}p{0.07\textwidth}<{\centering}}
\hline
\multirow{2}{*}{Input Frames} & \multicolumn{2}{c}{DHB ($\times10^{-3}$)} & \multicolumn{2}{c}{NL-Drive}  \\ 
\cline{2-5}
 & CD$\downarrow$ & EMD$\downarrow$ & CD$\downarrow$ & EMD$\downarrow$ \\ 
\hline
2 frames & 0.60 & 4.20 & 0.84 & 98.99 \\
4 frames & \textbf{0.54} & \textbf{3.68} & \textbf{0.80} & \textbf{97.03} \\
6 frames & 0.55 & 3.74 & 0.86 & 98.82 \\
8  frames & 0.57 & 3.87 & 0.96 & 104.44 \\
\hline
\end{tabular}}
% \vspace{-.4cm}
\end{table}

% \begin{table*}[htb]
% \centering
% \caption{\textbf{Quantitative results with different input frames for NeuralPCI.} Among them, 2 frames setting indicates pair frame and 4 frames setting is the standard setting in our main paper. }
% \label{tab:input frames}
% \scalebox{1}{
% \begin{tabular}{ccccccccc} 
% \hline
% \multirow{2}{*}{Datasets} & \multicolumn{2}{c}{2 frames} & \multicolumn{2}{c}{4 frames} & \multicolumn{2}{c}{6 frames} & \multicolumn{2}{c}{8 frames}  \\ 
% \cline{2-9}
%  & CD & EMD & CD & EMD & CD & EMD & CD & EMD \\ 
% \hline
% DHB & 0.60 & 4.20 & \textbf{0.54} & \textbf{3.68} & 0.55 & 3.74 & 0.57 & 3.87 \\
% NL-Drive & 0.84 & 98.99 & \textbf{0.80} & \textbf{97.03} & 0.86 & 98.82 & 0.96 & 104.44 \\
% \hline
% \end{tabular}}
% \end{table*}

