\documentclass[12pt,a4, reqno]{amsart}
\usepackage[a4paper, left=26mm, right=26mm, top=28mm, bottom=34mm]{geometry}
\usepackage[all]{xy}
% \usepackage[dvips]{graphicx, color}
\usepackage{amsmath}
\usepackage{amscd}
\usepackage{amssymb}
\usepackage{amsthm}
\usepackage{url}
% \usepackage{ascmac}
% \usepackage{nruby}
% \usepackage{stmaryrd}
% \usepackage{wrapfig}
\usepackage{mathrsfs}
\usepackage{comment}
\usepackage{bbm}

\newtheorem{thm}{Theorem}[subsection]
\newtheorem{lem}[thm]{Lemma}
\newtheorem{cor}[thm]{Corollary}
\newtheorem{prop}[thm]{Proposition}
\newtheorem{conj}[thm]{Conjecture}
\newtheorem{claim}[thm]{Claim}

\theoremstyle{definition}
\newtheorem{ass}[thm]{Assumption}
\newtheorem{rem}[thm]{Remark}
\newtheorem{defn}[thm]{Definition}
\newtheorem{prob}[thm]{Problem}
\renewcommand{\thedefn}{}
\newtheorem{ex}[thm]{Example}
\renewcommand{\theex}{}
\newcommand{\adj}[4]{#1\negmedspace: #2\rightleftarrows #3:\negmedspace #4}
\def\A{{\mathbb A}}
\def\F{{\mathbb F}}
\def\G{{\mathbb G}}
\def\O{{\mathcal O}}
\def\Q{{\mathbb Q}}
\def\R{{\mathbb R}}
\def\Z{{\mathbb Z}}
\def\C{{\mathbb C}}
\def\P{{\mathbb P}}
\def\L{{\mathbb L}}
\def\N{{\mathbb N}}
\def\V{{\mathbb V}}
\def\X{{\mathfrak X}}
\def\Y{{\mathfrak Y}}
\def\ZZ{{\mathfrak Z}}
\DeclareMathOperator{\RHom}{\mathrm{R}\mathscr{H}\!\textit{om}}
\DeclareMathOperator{\Homsheaf}{\mathscr{H}\!\textit{om}}
\DeclareMathOperator{\Extsheaf}{\mathscr{E}\!\textit{xt}}
\def\Art{\mathop{\mathrm{Art}}\nolimits}
\def\Br{\mathop{\mathrm{Br}}\nolimits}
\def\Fr{\mathop{\mathrm{Fr}}\nolimits}
\def\End{\mathop{\mathrm{End}}\nolimits}
\def\Irr{\mathop{\mathrm{Irr}}\nolimits}
\def\Jord{\mathop{\mathrm{Jord}}\nolimits}
\def\Stab{\mathop{\mathrm{Stab}}\nolimits}
\def\Gr{\text{\rm Gr}}
\def\H{\mathbb{H}}
\def\cl{\text{\rm cl}}
\def\Aut{\mathop{\mathrm{Aut}}\nolimits}
\def\Ann{\mathop{\mathrm{Ann}}\nolimits}
\def\Cl{\mathop{\mathrm{Cl}}\nolimits}
\def\Coker{\mathop{\mathrm{Coker}}\nolimits}
\def\Frob{\mathop{\mathrm{Frob}}\nolimits}
\def\Frac{\mathop{\mathrm{Frac}}\nolimits}
\def\Tor{\mathop{\mathrm{Tor}}\nolimits}
\def\Ext{\mathop{\mathrm{Ext}}\nolimits}
\def\Cent{\mathop{\mathrm{Cent}}\nolimits}
\def\Fil{\mathop{\mathrm{Fil}}\nolimits}
\def\Groth{\mathop{\mathrm{Groth}}\nolimits}
\def\Gal{\mathop{\mathrm{Gal}}\nolimits}
\def\Lie{\mathop{\mathrm{Lie}}\nolimits}
\def\Hom{\mathop{\mathrm{Hom}}\nolimits}
\def\Ind{\mathop{\mathrm{Ind}}\nolimits}
\def\Int{\mathop{\mathrm{Int}}\nolimits}
\def\inv{\mathop{\mathrm{inv}}\nolimits}
\def\Im{\mathop{\mathrm{Im}}\nolimits}
\def\Jac{\mathop{\mathrm{Jac}}\nolimits}
\def\Ker{\mathop{\mathrm{Ker}}\nolimits}
\def\id{\mathop{\mathrm{id}}\nolimits}
\def\length{\mathop{\mathrm{length}}\nolimits}
\def\SU{\mathop{\mathrm{SU}}\nolimits}
\def\GU{\mathop{\mathrm{GU}}\nolimits}
\def\U{\mathop{\mathrm{U}}\nolimits}
\def\GL{\mathop{\mathrm{GL}}\nolimits}
\def\PGL{\mathop{\mathrm{PGL}}\nolimits}
\def\SL{\mathop{\mathrm{SL}}\nolimits}
\def\GSp{\mathop{\mathrm{GSp}}\nolimits}
\def\Sp{\mathop{\mathrm{Sp}}\nolimits}
\def\Spin{\mathop{\mathrm{Spin}}\nolimits}
\def\GSpin{\mathop{\mathrm{GSpin}}\nolimits}
\def\SO{\mathop{\mathrm{SO}}\nolimits}
\def\GSO{\mathop{\mathrm{GSO}}\nolimits}
\def\Sp{\mathop{\mathrm{Sp}}\nolimits}
\def\Res{\mathop{\mathrm{Res}}\nolimits}
\def\Sym{\mathop{\mathrm{Sym}}\nolimits}
\def\St{\mathop{\mathrm{St}}\nolimits}
\def\Pic{\mathop{\mathrm{Pic}}\nolimits}
\def\Proj{\mathop{\rm Proj}}
\def\Rep{\mathrm{Rep}}
\def\Spa{\mathop{\rm Spa}}
\def\Spec{\mathop{\rm Spec}}
\def\Spf{\mathop{\rm Spf}}
\def\Spv{\mathop{\rm Spv}}
\def\Supp{\mathop{\rm Supp}}
\def\Cont{\mathop{\rm Cont}}
\def\supp{\mathop{\rm supp}}
\def\sep{\text{\rm sep}}
\def\Tr{\mathop{\text{\rm Tr}}\nolimits}
\def\Max{\mathop{\rm Max}}
\def\cris{\text{\rm cris}}
\def\deg{\mathop{\text{\rm deg}}\nolimits}
\def\rank{\mathop{\text{\rm rank}}\nolimits}
\def\Div{\mathop{\text{\rm Div}}\nolimits}
\def\B{\mathbb{B}}
\def\D{\mathbb{D}}
\def\pr{\text{\rm pr}}
\def\unip{\mathrm{unip}}
\def\vol{\mathop{\rm vol}}
\def\Pf{\mathop{\mathrm{Pf}}\nolimits}
\def\det{\mathop{\mathrm{det}}\nolimits}
\def\halfdet{\mathop{\textrm{half-det}}\nolimits}
\def\chara{\mathop{\mathrm{char}}}
\def\divi{\mathop{\mathrm{div}}}
\def\ord{\mathop{\mathrm{ord}}}
\def\diag{\mathop{\mathrm{diag}}}
\def\resp{\mathop{\mathrm{resp}}}
\def\Mat{\mathop{\mathrm{Mat}}\nolimits}
\def\KS{\mathop{\mathrm{KS}}\nolimits}
\def\Alb{\mathop{\mathrm{Alb}}}
\def\Sh{\mathop{\mathrm{Sh}}\nolimits}
\def\p{\mathop{\mathfrak{p}}\nolimits}
\def\q{\mathop{\mathfrak{q}}\nolimits}
\def\A{\mathbb{A}}
\def\Af{\mathbb{A}_f}
\def\Fil{\mathop{\mathrm{Fil}}\nolimits}
\def\dR{\mathop{\mathrm{dR}}}
\def\K{\mathop{\mathrm{K}}\nolimits}
\newcommand{\transp}[1]{{}^{t}\!{#1}}
\newcommand{\plim}[1][]{\mathop{\varprojlim}\limits_{#1}}
\renewcommand{\labelenumi}{(\arabic{enumi})}
\def\Gr{\mathop{\mathrm{Gr}}\nolimits}
\def\M{\mathop{\mathfrak{M}}\nolimits}
\def\MdR{\mathop{\mathfrak{M}_{\dR}}\nolimits}
\def\Mcris{\mathop{\mathfrak{M}_{\cris}}\nolimits}
\newcommand{\et}{\mathrm{\acute{e}t}}
\newcommand{\proet}{\mathrm{pro\acute{e}t}}
\newcommand{\ad}{\mathrm{ad}}
\newcommand{\ch}{\mathrm{ch}}
\def\NS{\mathop{\mathrm{NS}}\nolimits}
\def\num{\mathop{\mathrm{num}}\nolimits}
\def\prim{\mathop{\mathrm{prim}}\nolimits}
\def\WD{\mathop{\mathrm{WD}}\nolimits}
\def\CH{\mathop{\mathrm{CH}}\nolimits}
\def\Bl{\mathop{\mathrm{Bl}}\nolimits}
\def\Catinfty{\mathop{\mathrm{Cat}_{\infty}}\nolimits}
\def\Perf{\mathop{\mathrm{Perf}}\nolimits}
\def\Mod{\mathop{\mathrm{Mod}}\nolimits}
\def\AniRing{\mathop{\mathrm{Ani(Ring)}}\nolimits}
\def\Ring{\mathop{\mathrm{Ring}}\nolimits}
\def\E{\mathop{\mathbb{E}}\nolimits}
\def\CAlg{\mathop{\mathrm{CAlg(Sp)}}\nolimits}
\def\Map{\mathop{\mathrm{Map}}\nolimits}
\def\colim{\mathop{\mathrm{colim}}\nolimits}
\def\arc{\mathop{\mathrm{arc}}\nolimits}
\def\mot{\mathop{\mathrm{mot}}\nolimits}
\def\DD{\mathop{\mathrm{DD}}\nolimits}
\def\Vect{\mathop{\mathrm{Vect}}\nolimits}
\def\alg{\mathop{\mathrm{alg}}\nolimits}
\def\tr{\mathop{\mathrm{tr}}\nolimits}
\def\Chow{\mathop{\mathrm{Chow}}\nolimits}
\def\triv{\mathop{\mathrm{triv}}\nolimits}
\def\op{\mathop{\mathrm{op}}\nolimits}
\def\Def{\mathop{\mathrm{Def}}\nolimits}
\def\Per{\mathop{\mathrm{Per}}\nolimits}
\def\Lift{\mathop{\mathrm{Lift}}\nolimits}
\def\Perfd{\mathop{\mathrm{Perfd}}\nolimits}
\def\ori{\mathop{\mathrm{ori}}\nolimits}
\def\perf{\mathop{\mathrm{perf}}\nolimits}
\def\fl{\mathop{\mathrm{fl}}\nolimits}
\def\univ{\mathop{\mathrm{univ}}\nolimits}
\def\ev{\mathop{\mathrm{ev}}\nolimits}
\def\Kos{\mathop{\mathrm{Kos}}\nolimits}
\def\Isom{\mathop{\mathrm{Isom}}\nolimits}

\usepackage{relsize} 
\usepackage[bbgreekl]{mathbbol} 
\usepackage{amsfonts} 
\usepackage{color}
\DeclareSymbolFontAlphabet{\mathbb}{AMSb} %to ensure that the meaning of \mathbb does not change
\DeclareSymbolFontAlphabet{\mathbbl}{bbold} 
\newcommand{\Prism}{{\mathlarger{\mathbbl{\Delta}}}}

\numberwithin{equation}{section}

\usepackage[T1]{fontenc}
\makeatletter
\@namedef{subjclassname@2020}{\textup{2020} Mathematics Subject Classification}
\makeatother

\begin{document}

\title[Prismatic $G$-display and descent theory]{Prismatic $G$-display and descent theory}

\author{Kazuhiro Ito}
\address{Kavli Institute for the Physics and Mathematics of the Universe (WPI), The University of Tokyo,
5-1-5 Kashiwanoha, Kashiwa, Chiba, 277-8583, Japan}
\email{kazuhiro.ito@ipmu.jp}

% \date{\today}

\subjclass[2020]{Primary 14F30; Secondary 14G45, 14L05}
\keywords{prisms, displays, $p$-divisible groups}

%14G45      Perfectoid spaces and mixed characteristic
%14L05	    Formal groups, $p$-divisible groups
%14F30  	$p$-adic cohomology, crystalline cohomology
% 14J28 	K3 surfaces and Enriques surfaces
% 11G15 	Complex multiplication and moduli of abelian varieties
% 14C25 	Algebraic cycles

\maketitle

\begin{abstract}
For a smooth affine group scheme $G$ over the ring of $p$-adic integers $\Z_p$ and a cocharacter $\mu$ of $G$,
we study $G$-$\mu$-displays over the prismatic site of Bhatt--Scholze.
In particular, we obtain several descent results for them.
If $G=\GL_n$, then our $G$-$\mu$-displays can be thought of as Breuil--Kisin modules with some additional conditions.

In fact, our results are formulated and proved for smooth affine group schemes over the ring of integers $\O_E$ of any finite extension $E$ of $\Q_p$ by using $\O_E$-prisms, which are $\O_E$-analogues of prisms.
\end{abstract}

\setcounter{tocdepth}{1}
\tableofcontents

\section{Introduction} \label{Section:Introduction}

Let $k$ be a perfect field of characteristic $p >0$.
Let $G$ be a smooth affine group scheme over the ring of $p$-adic integers $\Z_p$ and
$
\mu \colon \G_m \to G_{W(k)}
$
a cocharacter defined over the ring $W(k)$ of $p$-typical Witt vectors, where $G_{W(k)}:=G \times_{\Spec \Z_p} \Spec W(k)$.

The purpose of this note is to provide a systematic study of $G$-$\mu$-displays over the prismatic site of Bhatt-Scholze \cite{BS}, following the approach of Lau \cite{Lau21}.
In particular, we prove several descent results for such prismatic $G$-$\mu$-displays.
The results obtained here will be used to study the deformation theory of prismatic $G$-$\mu$-displays in our future work.
If $G=\GL_n$, our $G$-$\mu$-displays can be regarded as Breuil--Kisin modules (in the generalized sense of Definition \ref{Definition:Breuil-Kisin module}) with some additional conditions.
In the following, we review some previous work, and then explain our results.

\subsection{Displays and Breuil--Kisin modules}\label{Subsection:Displays and Breuil--Kisin modules}

The theory of displays and windows, initiated by Zink and developed by many authors including Zink and Lau, has been
successfully applied in the study of $p$-divisible groups.
There are several classifications of $p$-divisible groups in terms of displays and windows, following the spirit of the classical Dieudonn\'e theory.
Such results are particularly useful for understanding deformations of $p$-divisible groups as well as structural properties of (local) moduli spaces of $p$-divisible groups.


For example, let $R$ be a complete noetherian local ring with residue field $k$.
Then, Zink and Lau proved that there exists
an equivalence 
between the category of $p$-divisible groups over $R$ and the category of
Dieudonn\'e displays for $R$; see \cite{Zink01} for $p \geq 3$, and \cite{Lau14} for all $p>0$.
A Dieudonn\'e display for $R$ is a free module of finite rank over the Zink ring $\mathbb{W}(R)$ (which is also denoted by $\hat{W}(R)$) equipped with a filtration and a certain Frobenius operator.
The ring $\mathbb{W}(R)$ is a subring of the ring $W(R)$ of $p$-typical Witt vectors which is stable by the Frobenius endomorphism of $W(R)$.
Dieudonn\'e displays are closely related to Dieudonn\'e crystals on crystalline sites.

On the other hand, it is also known that
$p$-divisible groups over the ring of integers $\O_K$ of a finite totally ramified extension $K$ of $W(k)[1/p]$ are classified by minuscule Breuil--Kisin modules for $\O_K$.
To be more precise, let $\mathfrak{S}:=W(k)[[t]]$, which admits
a Frobenius endomorphism $\phi \colon \mathfrak{S} \to \mathfrak{S}$ such that it acts on $W(k)$ as the usual Frobenius and sends $t$ to $t^p$.
Let $\varpi \in \O_K$ be a uniformizer and $\mathcal{E} \in \mathfrak{S}$ the Eisenstein polynomial of $\varpi$.
A minuscule Breuil--Kisin module for $\O_K$ (with respect to $\varpi$) is a free $\mathfrak{S}$-module $M$ of finite rank equipped with an $\mathfrak{S}$-linear homomorphism
\[
F_M \colon \phi^*M:=\mathfrak{S} \otimes_{\phi, \mathfrak{S}} M \to M
\]
whose cokernel is killed by $\mathcal{E}$.
Then, there exists an equivalence of categories:
\begin{equation}\label{equation:Breuil--Kisin classification}
    \{ p\text{-divisible groups over} \ \O_K  \} \overset{\sim}{\to} \{ \text{minuscule Breuil--Kisin modules for} \ \O_K \}.
\end{equation}
This result was conjectured by Breuil, proved by Kisin (\cite{Kisin06}, \cite{Kisin09}) when $p \geq 3$, and proved in \cite{Kim12}, \cite{LiuTong}, and \cite{Lau14} for all $p>0$.
In \cite{Lau14},
Lau first showed that the category of minuscule Breuil--Kisin modules for $\O_K$ is equivalent to the category of Dieudonn\'e displays for $\O_K$, and then deduced the result from the classification of $p$-divisible groups in terms of Dieudonn\'e displays.
In fact, the same result is obtained for any complete regular local ring $R$ (of any dimension) with residue field $k$ in \cite{Lau14}.
See also \cite{Zink01b}, \cite{VasiuZink}, and \cite{Lau10} for the results in this direction.
The equivalence (\ref{equation:Breuil--Kisin classification}) is a key ingredient in Kisin's construction of integral canonical models of Shimura varieties of abelian type with hyperspecial level structure \cite{Kisin10}.

The pair
$(\mathfrak{S}, (\mathcal{E}))$
consisting of $\mathfrak{S}$ and the ideal
$(\mathcal{E})$ generated by $\mathcal{E}$ is a \textit{prism} in the sense of Bhatt--Scholze \cite{BS}.
In \cite{Anschutz-LeBras}, Ansch\"utz--Le Bras gave a cohomological description of the functor (\ref{equation:Breuil--Kisin classification}) using the prismatic site, and proved that it is an equivalence by a different method (\cite[Theorem 5.12]{Anschutz-LeBras}).
See also Theorem \ref{Theorem:classification theorem for p-divisible group} in Section \ref{Section:p-divisible groups and prismatic Dieudonn\'e crystals}.
The theory of prisms is likely to become increasingly important in the study of various moduli spaces in arithmetic geometry such as Rapoport--Zink spaces and (local) Shimura varieties.
From the module theoretic viewpoint,
minuscule Breuil--Kisin modules are more convenient to handle than Dieudonn\'e displays (e.g.\ the ring $\mathfrak{S}$ is noetherian, while the Zink ring $\mathbb{W}(\O_K)$ is not).
On the other hand, from the viewpoint of deformation theory,
there are some technical issues in working with minuscule Breuil--Kisin modules, or prismatic Dieudonn\'e crystals (in the sense of \cite{Anschutz-LeBras}).
For example, the Grothendieck--Messing deformation theory does not apply directly in this setting.

\subsection{Prismatic $G$-$\mu$-displays}\label{Subsection:Prismatic G-mu-displays Introduction}

For applications to Rapoport--Zink spaces and (local) Shimura varieties, it will also be convenient to consider displays with ``$G$-$\mu$-structures''.
Such objects, called $G$-$\mu$-displays (also called $(G, \mu)$-displays or $G$-displays of type $\mu$), were introduced by B\"ultel \cite{Bultel} and B\"ultel--Pappas \cite{Bultel-Pappas}.
Then, the theory of $G$-$\mu$-displays has been developed in various settings; see for example \cite{Pappas20}, \cite{Lau21}, and \cite{Daniels}.
These objects also have applications to K3 surfaces and related varieties; see \cite{LangerZink}, \cite{Lau21}, and \cite{inoue}.

In \cite{Lau21}, Lau introduced a useful framework for studying $G$-$\mu$-displays, which allows for a unified treatment of several constructions considered in previous studies.
Although it is not stated in \textit{loc.cit.} explicitly, we can follow Lau's approach to study the notion of $G$-$\mu$-display over prisms.
In fact,
such an adaptation has already appeared in the paper of Bartling \cite{Bartling}.
Bartling used $G$-$\mu$-displays over perfect prisms
in order to study the relation between Rapoport--Zink spaces constructed in \cite{Bultel-Pappas} and integral models of local Shimura varieties defined in \cite{Scholze-Weinstein} under a certain nilpotence condition (introduced in \cite{Bultel-Pappas}) on $G$-$\mu$-displays.

In this paper, we study $G$-$\mu$-displays (and Breuil--Kisin modules) over prisms in more details.
We pay a special attention to $G$-$\mu$-displays over prisms of Breuil--Kisin type (in the sense of Proposition \ref{Proposition:Breuil-Kisin type frame}).
In our work, we do not need to impose any nilpotence condition on $G$-$\mu$-displays.
We will develop the deformation theory for $G$-$\mu$-displays over prisms in \cite{Ito-K23-b}, using the results obtained here.

\subsection{Summary of results}\label{Subsection:summary of results}

Let us now summarize the contents of this paper.
Let $(A, I)$ be a bounded prism in the sense of \cite{BS}, such that $A$ is a $W(k)$-algebra.
We assume that $(A, I)$ is orientable, that is, the ideal $I$ is principal.

Before introducing $G$-$\mu$-displays,
we study Breuil--Kisin modules in Section \ref{Section:Displayed Breuil--Kisin module}, where we follow the terminology of \cite{Anschutz-LeBras} and define Breuil--Kisin modules over any $(A, I)$.
We introduce the notions of \textit{displayed} Breuil--Kisin module and of \textit{minuscule} Breuil--Kisin module.
Prismatic $\GL_n$-$\mu$-displays over $(A, I)$ introduced below can be regarded as displayed Breuil--Kisin modules over $(A, I)$.
Moreover, if the cocharacter $\mu$ is minuscule, then the corresponding displayed Breuil--Kisin modules are minuscule (up to twist).

Let $d \in I$ be a generator, which is a nonzerodivisor.
In Section \ref{Section:display group},
we will define and study the following group
\[
G_\mu(A, I):=\{ \, g \in G(A) \, \vert \, \mu(d)g\mu(d)^{-1} \, \, \text{lies in} \, \, G(A) \subset G(A[1/d]) \, \},
\]
which we call the \textit{display group}.
We endow $G(A)$ with the following action of $G_\mu(A, I)$:
\[
    G(A) \times G_\mu(A, I) \to G(A), \quad (X, g) \mapsto g^{-1}X\phi(\mu(d)g\mu(d)^{-1}).
\]
We note that $G_\mu(A, I)$ does not depend on the choice of $d$, while the above action does.
Let $(A, I)_\et$ be the category of bounded prisms $(A', I')$
with a $(p, I)$-completely \'etale map $(A, I) \to (A', I')$ of prisms.
We endow (the opposite of) $(A, I)_\et$ with the $(p, I)$-completely \'etale topology.
The display groups $G_\mu(A', I')$
form a group sheaf $G_{\mu, A, I}$ on $(A, I)_\et$, and
the above action naturally extends to an action of
$G_{\mu, A, I}$ on the sheaf $G_{\Prism, A}$ defined by $(A', I') \mapsto G(A')$.
We refer to Section \ref{Section:display group} for the details.

In Section \ref{Section:prismatic G display},
we define $G$-$\mu$-displays over prisms, and present some basic results.
A \textit{$G$-$\mu$-display} (or prismatic $G$-$\mu$-display) over $(A, I)$ is a pair
\[
(\mathcal{Q}, \alpha_\mathcal{Q})
\]
consisting of
a
$G_{\mu, A, I}$-torsor $\mathcal{Q}$
with respect to the $(p, I)$-completely \'etale topology
and a $G_{\mu, A, I}$-equivariant map
$\alpha_\mathcal{Q} \colon \mathcal{Q} \to G_{\Prism, A}$
of sheaves.
The map $\alpha_\mathcal{Q}$ plays the same role as Frobenius homomorphisms $F_M$ in the theory of Breuil--Kisin modules.
When there is no possibility of confusion, we write $\mathcal{Q}$ instead of $(\mathcal{Q}, \alpha_\mathcal{Q})$.
Let
\[
G\mathchar`-\mathrm{Disp}_\mu(A, I)
\]
be the groupoid of $G$-$\mu$-displays over $(A, I)$.
We remark that this groupoid does not depend on the choice of $d \in I$ up to canonical equivalence.
In fact, in the main body of this paper,
we slightly modify the above definition, and $G$-$\mu$-displays are defined
in such a way that they are completely independent of the choice of $d$.
Also, we can define $G$-$\mu$-displays for any (not necessarily orientable) bounded prism.
See Section \ref{Subsection:prismatic G-display} for more details.

We say that a $G$-$\mu$-display $(\mathcal{Q}, \alpha_\mathcal{Q})$ over $(A, I)$
is \textit{banal} if $\mathcal{Q}$ is trivial as a $G_{\mu, A, I}$-torsor.
The groupoid of banal $G$-$\mu$-displays over $(A, I)$ is equivalent to the following quotient groupoid:
\[
[G(A)/G_\mu(A, I)].
\]
The objects of this groupoid are the elements $X \in G(A)$,
and the morphisms are defined by
\[
\Hom(X, X')=\{\, g \in G_\mu(A, I) \, \vert \, g^{-1}X'\phi(\mu(d)g\mu(d)^{-1})=X  \, \}.
\]
The banal $G$-$\mu$-display corresponding to an element $X \in G(A)$ is
\[
\mathcal{Q}_X=(G_{\mu, A, I}, \alpha_X)
\]
where $\alpha_X \colon G_{\mu, A, I} \to G_{\Prism, A}$ is given by $1 \mapsto X$.

The results of Section \ref{Section:prismatic G display} can be summarized as follows:
\begin{enumerate}
    \item 
    We prove $(p, I)$-completely flat descent for prismatic $G$-$\mu$-displays (Proposition \ref{Proposition:flat descent of G display}).
    \item To any $G$-$\mu$-display $\mathcal{Q}$ over $(A, I)$,
    we attach a $P_\mu$-torsor $P(\mathcal{Q})_{A/I}$ over $\Spec A/I$, which is called the \textit{Hodge filtration} (Definition \ref{Definition:Hodge filtration of G-displays} and Remark \ref{Remark:Hodge filtration schematic}).
    Here $P_\mu$ is the usual subgroup scheme of $G_{W(k)}$ associated with the cocharacter $\mu$; see Section \ref{Subsection:Properties of the display group}.
    \item A \textit{$\phi$-$G$-torsor} over $(A, I)$
    is a pair
    $(\mathcal{P}, \phi_\mathcal{P})$ consisting of
    a $G$-torsor $\mathcal{P}$ over $\Spec A$ and
    an isomorphism
    \[
    \phi_\mathcal{P} \colon (\phi^*\mathcal{P})[1/\phi(d)] \overset{\sim}{\to} \mathcal{P}[1/\phi(d)]
    \]
    of $G$-torsors over $\Spec A[1/\phi(d)]$.
    Here, for a $G$-torsor $\mathcal{P}$ over $\Spec A$,
    we write $\mathcal{P}[1/\phi(d)]$ for the base change $\mathcal{P} \times_{\Spec A} \Spec A[1/\phi(d)]$.
    We associate to each $G$-$\mu$-display $\mathcal{Q}$ over $(A, I)$ a $\phi$-$G$-torsor
    $\mathcal{Q}_{\phi}$ over $(A, I)$, which we call the \textit{underlying $\phi$-$G$-torsor} of $\mathcal{Q}$ (Definition \ref{Definition:underlying phi-G-torsor}).
    
    \item Let $R$ be a perfectoid ring and $(W(R^\flat), I_R)$ the corresponding perfect prism.
    In Section \ref{Subsection:G displays for perfectoid rings},
    we study $G$-$\mu$-displays over $(W(R^\flat), I_R)$.
    In particular,
    we prove that
    the construction $\mathcal{Q} \mapsto \mathcal{Q}_\phi$ induces an equivalence from
    $G\mathchar`-\mathrm{Disp}_\mu(W(R^\flat), I_R)$ to the groupoid of $\phi$-$G$-torsors over $(W(R^\flat), I_R)$ \textit{of type $\mu$}.
    See Definition \ref{Definition:phi-G-torsor of type mu} and Proposition \ref{Proposition:comparison of notions of type mu} for the notion of $\phi$-$G$-torsor of type $\mu$.
    This result can also be found in \cite{Bartling}.
    A $\phi$-$G$-torsor over $(W(R^\flat), I_R)$ is also called a
    \textit{$G$-Breuil-Kisin-Fargues module} for $R$ in the literature.
\end{enumerate}

Let $R$ be a complete regular local ring with residue field $k$.
As in \cite{BS} and \cite{Anschutz-LeBras}, we consider the category
$(R)_\Prism$ of bounded prisms $(A, I)$ with a homomorphism $R \to A/I$.
A \textit{$G$-$\mu$-display over $(R)_{\Prism}$} is defined to be an object of the following groupoid
\[
G\mathchar`-\mathrm{Disp}_\mu((R)_{\Prism}):= {2-\varprojlim}_{(A, I) \in (R)_{\Prism}} G\mathchar`-\mathrm{Disp}_\mu(A, I).
\]
One can say that a $G$-$\mu$-display over $(R)_{\Prism}$ is a ``prismatic crystal in $G$-$\mu$-displays'' on $(R)_{\Prism}$.

There exists a prism
\[
(W(k)[[t_1, \dotsc, t_n]], (\mathcal{E}))
\]
with an isomorphism
$R \simeq W(k)[[t_1, \dotsc, t_n]]/\mathcal{E}$
which lifts the identity $\id_k \colon k \to k$.
Here the Frobenius $\phi$ of $W(k)[[t_1, \dotsc, t_n]]$ is such that $\phi(t_i)=t^p_i$ for $1 \leq i \leq n$, and $\mathcal{E} \in W(k)[[t_1, \dotsc, t_n]]$ is a formal power series whose constant term is $p$.
In Section \ref{Section:G-displays over complete regular local rings}, we prove the following theorem, which is the main result of this paper.

\begin{thm}[Theorem \ref{Theorem:main result on G displays over complete regular local rings}]\label{Theorem:main result on G displays over complete regular local rings Introduction}
Assume that the cocharacter $\mu$ is 1-bounded.
Then the natural functor
    \[
 G\mathchar`-\mathrm{Disp}_\mu((R)_{\Prism}) \to G\mathchar`-\mathrm{Disp}_\mu(W(k)[[t_1, \dotsc, t_n]], (\mathcal{E}))
 \]
is an equivalence.
\end{thm}

We refer to Definition \ref{Definition:1-bounded} for the definition of 1-bounded cocharacters.
If $G$ is reductive, then $\mu$ is 1-bounded if and only if $\mu$ is minuscule.
If $G=\GL_N$, this result (or more precisely, the analogous result for minuscule Breuil--Kisin modules) is stated in \cite[Section 5.2]{Anschutz-LeBras}, and the proof is given in the case where $n \leq 1$.
Our proof (in the case where $G=\GL_N$) goes along the same line as that of \cite{Anschutz-LeBras}, but it requires some additional arguments when $n \geq 2$.
This result fits in the omitted part of the proof of \cite[Theorem 5.12]{Anschutz-LeBras}; see Section \ref{Section:p-divisible groups and prismatic Dieudonn\'e crystals} for details.
The proof of Theorem \ref{Theorem:main result on G displays over complete regular local rings Introduction} for a general $G$ is inspired by that of \cite[Proposition 7.1.5]{Lau21}.

In fact, the results of this paper will be formulated and proved for a smooth affine group scheme $G$ over the ring of integers $\O_E$ of any finite extension $E$ of $\Q_p$.
For this, we will use $\O_E$-analogues of prisms, called \textit{$\O_E$-prisms}.
This notion was already introduced in the work of Marks \cite{Marks} (in which these objects are called $E$-typical prisms).
Section \ref{Section:Preliminaries on prisms} is devoted to discuss
analogous results to those of \cite[Section 2 and Section 3]{BS} for $\O_E$-prisms.
We will define $G$-$\mu$-displays for bounded $\O_E$-prisms in the same way, and prove the above results for them.
It will be convenient to establish our results in this generality,
but the reader (who is familiar with the theory of prisms) may assume that $\O_E=\Z_p$ and skip Section \ref{Section:Preliminaries on prisms} on a first reading.
The arguments for general $\O_E$ are the same as for the case where $\O_E=\Z_p$.




\subsection*{Notation} \label{Subsection:Notation}

In this paper, all rings are commutative and unital.
For a module $M$ over a ring $R$ and a ring homomorphism
$f \colon R \to R'$,
the tensor product $M\otimes_{R}R'$ is denoted by $M_{R'}$ or $f^*M$.
For a scheme $X$ over $R$,
the base change $X \times_{\Spec R} \Spec R'$ is denoted by
$X_{R'}$ or $f^*X$.
We use similar notation for the base change of group schemes,
$p$-divisible groups, etc.
Moreover, all actions of groups will be right actions, unless otherwise stated.


\section{Preliminaries on $\O_E$-prisms} \label{Section:Preliminaries on prisms}

Throughout this paper, we fix a prime number $p$.
Let $E$ be a finite extension of $\Q_p$ with ring of integers $\O_E$ and residue field $\F_q$.
Here $\F_q$ is a finite field with $q$ elements.
We fix a uniformizer $\pi \in \O_E$.

In this section, we study an ``$\O_E$-analogue'' of the notion of prism.
Such objects are called $\O_E$-prisms in this paper.
This notion was also introduced
in the work of Marks \cite{Marks} (in which $\O_E$-prisms are called \textit{$E$-typical prisms}).
We discuss some properties of $\O_E$-prisms which we need in the sequel.
We hope that this section will also help the reader unfamiliar with \cite{BS} to understand some basic facts about prisms.


\subsection{Prisms} \label{Subsection:prisms}

We first recall the definition of bounded prisms.

Let $A$ be a $\Z_{(p)}$-algebra.
A \textit{$\delta$-structure} on $A$ is a map $\delta \colon A \to A$ of sets with the following properties:
\begin{enumerate}
    \item $\delta(1)=0$.
    \item $\delta(xy)=x^p\delta(y)+y^p\delta(x)+p\delta(x)\delta(y)$.
    \item $\delta(x+y)=\delta(x)+\delta(y)+(x^p+y^p-(x+y)^p)/p$.
\end{enumerate}
A \textit{$\delta$-ring} is a pair $(A, \delta)$ of a $\Z_{(p)}$-algebra $A$ and a $\delta$-structure $\delta \colon A \to A$.
The above equalities imply that
\[
\phi \colon A \to A, \quad x \mapsto x^p+p\delta(x)
\]
is a ring homomorphism which is a lift of
the Frobenius $A/p \to A/p$, $x \mapsto x^p$.

In the following, for a ring $A$ and an ideal $I \subset A$, we say that an $A$-module $M$ is $I$-adically complete (or $x$-adically complete if $I$ is generated by an element $x \in I$) if the natural homomorphism
\[
M \overset{\sim}{\to} \widehat{M}:=\plim[n] M/I^nM
\]
is bijective.

\begin{defn}[\cite{BS}]\label{Definition:orientable and bounded prism}
A \textit{bounded prism} is a pair
$(A, I)$ of a $\delta$-ring $A$ and a Cartier divisor $I \subset A$ with the following properties:
\begin{enumerate}
    \item $A$ is $(p, I)$-adically complete.
    \item $A/I$ has bounded $p$-torsion, i.e.\ $(A/I)[p^\infty]=(A/I)[p^n]$ for some integer $n > 0$.
    \item We have $p \in (I, \phi(I))$.
\end{enumerate}
We say that a bounded prism $(A, I)$ is \textit{orientable} if $I$ is principal.
\end{defn}


\begin{rem}\label{Remark:remarks on prisms}
Under the condition that $A/I$ has bounded $p^\infty$-torsion,
the requirement that $A$ is $(p, I)$-adically complete
is equivalent to saying that $A$ is \textit{derived} $(p, I)$-adically complete; see \cite[Lemma 3.7]{BS}.
We refer to \cite[Notation 1.2]{BS} and \cite[Tag 091N]{SP} for the notion of derived complete module (or complex).
We note that in general for a ring $A$ and a finitely generated ideal $I \subset A$, if an $A$-module $M$ is $I$-adically complete, then $M$ is derived $I$-adically complete.
\end{rem}

\subsection{$\delta_E$-ring}\label{Subsection:delta pi ring}


In this subsection, we recall the notion of $\delta_E$-ring, which is an ``$\O_E$-analogue'' of the notion of $\delta$-ring.
We define
\[
\delta_{\O_E, \pi} \colon \O_E \to \O_E, \quad x \mapsto (x-x^q)/\pi.
\]


\begin{defn}[{\cite[Definition 2.2]{Marks}}]\label{Definition:OE delta}
\
\begin{enumerate}
    \item Let $A$ be an $\O_E$-algebra.
    A \textit{$\delta_E$-structure} on $A$ is a map $\delta_E \colon A \to A$ of sets with the following properties:
\begin{enumerate}
    \item $\delta_E(xy)=x^q\delta_E(y)+y^q\delta_E(x)+\pi\delta_E(x)\delta_E(y)$.
    \item $\delta_E(x+y)=\delta_E(x)+\delta_E(y)+(x^q+y^q-(x+y)^q)/\pi$.
    \item $\delta_E \colon A \to A$ is compatible with $\delta_{\O_E, \pi}$, i.e.\ we have $\delta_E(x)=\delta_{\O_E, \pi}(x)$ for any $x \in \O_E$.
\end{enumerate}
    A \textit{$\delta_E$-ring} is a pair $(A, \delta_E)$ of an $\O_E$-algebra $A$ and a $\delta_E$-structure on $A$.
\item A homomorphism $f \colon (A, \delta_E) \to (A', \delta'_E)$
of $\delta_E$-rings
is a homomorphism $f \colon A \to A'$ of $\O_E$-algebras such that $f \circ \delta_E=\delta'_E \circ f$.
\end{enumerate}
\end{defn}
The term
$(x^q+y^q-(x+y)^q)/\pi$ in (b) makes sense since we can write it as
\[
(x^q+y^q-(x+y)^q)/\pi= -\sum_{0 < i < q} (\tbinom{q}{i}/\pi)x^iy^{q-i}.
\]
We usually denote a $\delta_E$-ring $(A, \delta_E)$ simply by $A$.

\begin{rem}
The notion of $\delta_E$-ring also appeared in \cite[Remark 1.19]{Borger} and \cite{Li} for example.
In the end of \cite{Li}, Li suggests to use $\delta_E$-structures for the study of prismatic sites of higher level over ramified bases.
\end{rem}

\begin{rem}\label{Remark:change of pi delta structure}
Strictly speaking, the definition of $\delta_E$-structures depends on the choice of $\pi$.
However, the notion of $\delta_E$-ring is essentially independent of $\pi$.
More precisely, let $\pi' \in \O_E$ be another uniformizer.
We write $\pi=u\pi'$ for a unique unit $u \in \O^\times_E$.
If an $\O_E$-algebra $A$ is equipped with a
$\delta_E$-structure
$\delta_E \colon A \to A$
with respect to $\pi$, then it also admits
a
$\delta_E$-structure
with respect to $\pi'$, defined by $x \mapsto u\delta_E(x)$.
\end{rem}

For a $\delta_E$-ring $A$, we define
\[
\phi_A \colon A \to A, \quad x \mapsto x^q+\pi\delta_E(x).
\]
We see that $\phi_A$ is a homomorphism of $\O_E$-algebras and is a lift of
the $q$-th power Frobenius $A/\pi \to A/\pi$, $x \mapsto x^q$.
The homomorphism $\phi_A$ is called the \textit{Frobenius} of the $\delta_E$-ring $A$.
When there is no ambiguity, we omit the subscript and simply write $\phi=\phi_A$.


\begin{rem}\label{Remark:torsion free case Frobenius lift}
If $A$ is a $\pi$-torsion free $\O_E$-algebra,
then the construction $\delta_E \mapsto \phi$ gives a bijection between the set of $\delta_E$-structures on $A$ and the set of homomorphisms $\phi \colon A \to A$ over $\O_E$ that are lifts of $A/\pi \to A/\pi$, $x \mapsto x^q$.
\end{rem}

\begin{ex}[Free $\delta_E$-rings]\label{Example:free delta pi rings}
We define an endomorphism $\phi$
of
the polynomial ring
$
\O_E[X_0, X_1, X_2, \dotsc]
$
by $X_i \mapsto X^q_i + \pi X_{i+1}$ $(i \geq 0)$.
By Remark \ref{Remark:torsion free case Frobenius lift},
we get the corresponding $\delta_E$-structure on $\O_E[X_0, X_1, X_2, \dotsc]$, which sends $X_i$ to $X_{i+1}$.
We write
\[
\O_E\{ X \}
\]
for the resulting $\delta_E$-ring.
As in the proof of \cite[Lemma 2.11]{BS}, one can check that $\O_E\{ X \}$ has the following property:
For a $\delta_E$-ring $A$ and an element $x \in A$,
there exists a unique homomorphism 
$f \colon \O_E\{ X \} \to A$ of $\delta_E$-rings with $f(X_0)=x$.
In other words, the $\delta_E$-ring $\O_E\{ X \}$ is a free object with basis $X:=X_0$ in the category of $\delta_E$-rings.
\end{ex}

Using $\O_E\{ X \}$, we can prove the following result:

\begin{lem}\label{Lemma:frobenius is compatible with delta}
For a $\delta_E$-ring $A$,
the Frobenius $\phi \colon A \to A$ is a homomorphism of $\delta_E$-rings.
\end{lem}

\begin{proof}
Let $x \in A$ be an element.
We have to show that $\phi(\delta_E(x))=\delta_E(\phi(x))$.
Since there exists a (unique) homomorphism
$f \colon \O_E\{ X \} \to A$
of $\delta_E$-rings with $f(X)=x$, it suffices to prove the assertion for $A=\O_E\{ X \}$, which is clear since $A$ is $\pi$-torsion free and $\phi \colon A \to A$ is $\phi$-equivariant.
\end{proof}

Following \cite[Remark 2.4]{BS},
we shall give a characterization of
$\delta_E$-rings in terms of 
\textit{ramified} Witt vectors.
For an $\O_E$-algebra $A$,
let
\[
W_{\O_E, \pi, 2}(A)
\]
denote the \textit{ring of $\pi$-typical Witt vectors of length $2$}: the underlying set of $W_{\O_E, \pi, 2}(A)$ is $A \times A$, and
for $(x_0, x_1), (y_0, y_1) \in W_{\O_E, \pi, 2}(A)$, we have
\begin{align*}
    (x_0, x_1) + (y_0, y_1)&= (x_0+y_0, x_1+y_1+(x^q_0+y^q_0-(x_0+y_0)^q)/\pi),\\
    (x_0, x_1) \cdot (y_0, y_1)&=(x_0y_0, x^q_0y_1+y^q_0x_1+\pi x_1y_1).
\end{align*}
If $\O_E=\Z_p$ and $\pi=p$, then $W_{\O_E, \pi, 2}(A)$ is the ring $W_2(A)$ of $p$-typical Witt vectors of length $2$.
For a detailed treatment of the rings of $\pi$-typical Witt vectors (of any length), we refer to \cite[Section 1.1]{Schneider} or \cite{Borger}.

\begin{rem}[{cf.\ \cite[Remark 2.4]{BS}}]\label{Remark:ramified Witt vectors of length 2}
The map
\[
\O_E \to W_{\O_E, \pi, 2}(A), \quad x \mapsto (x, \delta_{\O_E, \pi}(x))
\]
is a ring homomorphism for any $\O_E$-algebra $A$.
We regard $W_{\O_E, \pi, 2}(A)$ as an $\O_E$-algebra by this homomorphism.
Let
\[
\epsilon \colon W_{\O_E, \pi, 2}(A) \to A, \quad (x_0, x_1) \mapsto x_0
\]
denote the projection map, which is a homomorphism of $\O_E$-algebras.
Then, for a $\delta_E$-structure $\delta_E \colon A \to A$ on $A$,
the map
$s \colon A \mapsto W_{\O_E, \pi, 2}(A)$
defined by $x \mapsto (x, \delta_E(x))$
is a homomorphism of $\O_E$-algebras such that $\epsilon \circ s=\id_A$.
By this procedure, we obtain a bijection between
the set of $\delta_E$-structures on $A$
and
the set of homomorphisms $s \colon A \to W_{\O_E, \pi, 2}(A)$ of $\O_E$-algebras satisfying $\epsilon \circ s=\id_A$.
\end{rem}

\begin{rem}[{cf.\ \cite[Remark 2.7]{BS}}]\label{Remark:colimit and limit of delta rings}
Using the characterization of $\delta_E$-structures given in Remark \ref{Remark:ramified Witt vectors of length 2}, we see that the category of $\delta_E$-rings admits all limits and colimits, and they are preserved by the forgetful functor from the category of $\delta_E$-rings to the category of $\O_E$-algebras.
\end{rem}

The following two lemmas will be used frequently in the sequel.

\begin{lem}\label{Lemma:quotient delta pi ring}
Let $A=(A, \delta_E)$ be a $\delta_E$-ring and $I \subset A$ an ideal.
Then $I$ is stable under $\delta_E$ if and only if $A/I$ admits a $\delta_E$-structure that is compatible with the one on $A$.
If such a $\delta_E$-structure on $A/I$ exists, then it is unique.
\end{lem}

\begin{proof}
This follows immediately from the definition of $\delta_E$-structures (see the proof of \cite[Lemma 2.9]{BS}).
\end{proof}

\begin{lem}\label{Lemma:completion of delta ring}
    Let $A$ be a $\delta_E$-ring and
    let $I \subset A$ be a finitely generated ideal containing $\pi$.
    Then, for any integer $n \geq 1$, there exists an integer $m \geq 1$ such that for any $x \in A$, we have
    $\delta_E(x+I^m) \subset \delta_E(x)+I^n$.
    In particular, the $I$-adic completion of $A$ admits a unique $\delta_E$-structure that is compatible with the one on $A$.
\end{lem}

\begin{proof}
    The proof is the same as that of \cite[Lemma 2.17]{BS}.
\end{proof}

\begin{defn}\label{Definition:perfect delta pi ring}
We say that a $\delta_E$-ring $A$ is \textit{perfect} if the Frobenius $\phi \colon A \to A$ is bijective.
\end{defn}

In the rest of this subsection, we discuss some properties of perfect $\delta_E$-rings.


\begin{lem}[{\cite[Lemma 2.11]{Marks}}]\label{Lemma:perfect implies torsion free}
A perfect $\delta_E$-ring $A$ is $\pi$-torsion free.
\end{lem}

\begin{proof}
This is proved in \cite[Lemma 2.11]{Marks}, and follows from the same argument as in the proof of \cite[Lemma 2.28]{BS}.
\end{proof}

\begin{ex}\label{Example:perfect delta pi ring}
Let $R$ be an $\F_q$-algebra.
Assume that $R$ is perfect (i.e.\ the Frobenius $R \to R$, $x \mapsto x^p$ is bijective).
Let
$W(R)$
be the ring of $p$-typical Witt vectors and we define
\[
W_{\O_E}(R):=W(R) \otimes_{W(\F_q)} \O_E.
\]
Let 
$\phi \colon W_{\O_E}(R) \to W_{\O_E}(R)$ denote
the base change of the $q$-th power Frobenius of $W(R)$.
This is a lift of the $q$-th power Frobenius of $W_{\O_E}(R)/\pi=R$.
Since $W_{\O_E}(R)$ is $\pi$-torsion free,
we obtain the corresponding $\delta_E$-structure on $W_{\O_E}(R)$.
It is clear that $W_{\O_E}(R)$ is a perfect $\delta_E$-ring.
\end{ex}

\begin{lem}\label{Lemma:adjoint ramified witt and tilt}
The functor $R \mapsto W_{\O_E}(R)$ from the category of perfect $\F_q$-algebras to the category of $\pi$-adically complete $\O_E$-algebras admits a right adjoint given by
$
A \mapsto \varprojlim_{x \mapsto x^p} A/\pi A.
$
\end{lem}
\begin{proof}
This is well-known in the case where $\O_E=\Z_p$ (see \cite[Chapter II, Proposition 10]{SerreLocalfields} for example).
The general case follows from this special case.
\end{proof}


\begin{cor}[{\cite[Proposition 2.12]{Marks}}]\label{Corollary:equivalence of delta}
The following categories are equivalent:
\begin{itemize}
    \item The category $\mathcal{C}_1$ of $\pi$-adically complete perfect $\delta_E$-rings $(A, \delta_E)$.
    \item The category $\mathcal{C}_2$ of $\pi$-adically complete and $\pi$-torsion free $\O_E$-algebras $A$ such that 
    $A/\pi A$ is perfect.
    \item The category $\mathcal{C}_3$ of perfect $\F_q$-algebras $R$.
\end{itemize}
More precisely, the functors
\[
\mathcal{C}_1 \to \mathcal{C}_2 \to \mathcal{C}_3 \to \mathcal{C}_1,
\]
defined by $(A, \delta_E) \mapsto A$, $A \mapsto A/\pi$, $R \mapsto W_{\O_E}(R)$, respectively, are equivalences.
\end{cor}

\begin{proof}
Using Lemma \ref{Lemma:adjoint ramified witt and tilt},
one can prove the assertion in exactly the same way as \cite[Corollary 2.31]{BS}.
\end{proof}


\begin{cor}\label{Corollary:homomorphism from perfect delta pi-ring}
Let $A$ be a perfect $\delta_E$-ring
and $B$ a $\pi$-adically complete $\delta_E$-ring.
Then any homomorphism $A \to B$ of $\O_E$-algebras is a homomorphism of $\delta_E$-rings.
\end{cor}

\begin{proof}
We may assume that $A$ is $\pi$-adically complete.
It then follows from Lemma \ref{Lemma:adjoint ramified witt and tilt} and Corollary \ref{Corollary:equivalence of delta} that $A \to B$ is $\phi$-equivariant.
Let $B^{\perf}$ be a limit of the diagram
\[
B \overset{\phi}{\leftarrow} B \overset{\phi}{\leftarrow} B  \leftarrow \cdots
\]
in the category of $\delta_E$-rings, which is a perfect $\delta_E$-ring.
Then, since $A$ is perfect, we see that $A \to B$ factors through a $\phi$-equivariant homomorphism $A \to B^{\perf}$ of $\O_E$-algebras.
Since $B^{\perf}$ is $\pi$-torsion free by Lemma \ref{Lemma:perfect implies torsion free},
it follows that $A \to B^{\perf}$ is a homomorphism of $\delta_E$-rings.
Since $A \to B$ is the composition $A \to B^{\perf} \to B$, the assertion follows.
\end{proof}

\subsection{$\O_E$-prism}\label{Subsection:OE prism}


We now introduce $\O_E$-prisms.


\begin{defn}[{\cite[Definition 3.1]{Marks}}]\label{Definition:OE prism}
\
\begin{enumerate}
    \item An \textit{$\O_E$-prism} is a pair $(A, I)$ of a $\delta_E$-ring $A$ and a Cartier divisor $I \subset A$
    such that $A$ is derived $(\pi, I)$-adically complete and $\pi \in I + \phi(I)A$.
    \item We say that an $\O_E$-prism $(A, I)$ is \textit{bounded} if $A/I$ has bounded $p^\infty$-torsion.
    \item We say that an $\O_E$-prism $(A, I)$ is \textit{orientable} if $I$ is principal.
    \item An \textit{oriented and bounded} $\O_E$-prism $(A, d)$ is an orientable and bounded $\O_E$-prism $(A, I)$ with a choice of a generator $d \in I$.
    \item A map $f \colon (A, I) \to (A', I')$ of $\O_E$-prisms is a homomorphism $f \colon A \to A'$ of $\delta_E$-rings such that $f(I) \subset I'$.
\end{enumerate}
\end{defn}


If $\O_E=\Z_p$, then bounded $\O_E$-prisms are nothing but bounded prisms in the sense of Definition \ref{Definition:orientable and bounded prism}.

\begin{rem}\label{Remark:derived and classical complete}
Let $(A, I)$ be a bounded $\O_E$-prism.
By \cite[Proposition 1.4 (1)]{Tian} (see also Lemma \ref{Lemma:flat maps over prisms} below), we see that $A$ is $(\pi, I)$-adically complete.
Moreover, since $A/I$ is derived $\pi$-adically complete and has bounded $p^\infty$-torsion, it follows that $A/I$ is $\pi$-adically complete (see \cite[Lemma 4.7]{BMS2} for example).
\end{rem}


Let $A$ be a $\delta_E$-ring.
Following \cite[Definition 2.19]{BS}, we say that an element $d \in A$ is \textit{distinguished} if $\delta_E(d) \in A^\times$, i.e.\ $\delta_E(d)$ is a unit.
Since $\delta_{\O_E, \pi}(\pi)=1-\pi^{q-1} \in \O^\times_E$, we see that $\pi \in A$ is distinguished.

\begin{lem}\label{Lemma:distinguished}
Let $A$ be a $\delta_E$-ring and $d \in A$ an element.
Assume that 
$\pi$ is contained in the Jacobson radical $\mathrm{rad}(A)$ of $A$.
\begin{enumerate}
    \item Assume that $d=fh$ for some elements $f, h \in A$ with $f \in \mathrm{rad}(A)$. If $d$ is distinguished, then $f$ is distinguished and $h \in A^\times$.
    \item Assume that $d \in \mathrm{rad}(A)$.
Then $d$ is distinguished if and only if $\pi \in (d, \phi(d))$.
\end{enumerate}
\end{lem}

\begin{proof}
This can be proved exactly in the same way as \cite[Lemma 2.24, Lemma 2.25]{BS}.
See also \cite[Lemma 2.9]{Marks}.
\end{proof}

The following rigidity property plays a fundamental role in the theory of $\O_E$-prisms.

\begin{lem}[{cf.\ \cite[Lemma 3.5]{BS}}]\label{Lemma:rigidity}
Let $(A, I) \to (A', I')$ be a map of $\O_E$-prisms.
Then, the natural homomorphism
$
I \otimes_A A' \to  IA'
$
is an isomorphism
and $IA'=I'$.
\end{lem}

\begin{proof}
By using \cite[Lemma 3.4]{Marks}, this follows from the same argument as in the proof of \cite[Lemma 3.5]{BS}.
We recall the argument in the case where both $(A, I)$ and $(A', I')$ are orientable.
It follows from Lemma \ref{Lemma:distinguished} (2) that
any generator $d \in I$ is distinguished.
Let $d' \in I'$ be a generator.
Then Lemma \ref{Lemma:distinguished} (1) implies that $d$ is mapped to
$ud'$ for some $u \in A'^\times$.
In particular, the image of $d$ in $A'$ is a nonzerodivisor, and we obtain
$
I \otimes_A A'\overset{\sim}{\to}  IA'
$
and
$IA'=I'$.
\end{proof}



The following lemma will be used several times in this paper.

\begin{lem}[{cf.\ the proof of \cite[Lemma 4.8]{BS}}]\label{Lemma:maps from perfectoid prisms to prism}
Let $A$ be a perfect $\delta_E$-ring
and
$(B, I)$
a bounded $\O_E$-prism.
Then
any homomorphism $A \to B/I$ of $\O_E$-algebras
lifts uniquely to a homomorphism $A \to B$ of $\delta_E$-rings.
\end{lem}

\begin{proof}
By Corollary \ref{Corollary:homomorphism from perfect delta pi-ring}, it is enough to check that the homomorphism $A \to B/I$ lifts uniquely to a homomorphism $A \to B$ of $\O_E$-algebras.
We may assume that $A$ is $\pi$-adically complete, and then
$A \simeq W_{\O_E}(R)$ for some perfect $\F_q$-algebra $R$
by Corollary \ref{Corollary:equivalence of delta}.
Since $B$ is $(\pi, I)$-adically complete and $B/I$ is $\pi$-adically complete,
it suffices to prove that for every integer $n \geq 1$,
any homomorphism
$W_n(R) \to B/(\pi^n, I)$ of $W_n(\F_q)$-algebras lifts uniquely to a homomorphism $W_n(R) \to B/(\pi^n, I^n)$ of $W_n(\F_q)$-algebras (here $W_n(R)=W(R)/p^n$).
This follows from the fact that
the cotangent complex $L_{W_n(R)/W_n(\F_q)}$
is acyclic
(\cite[Lemma 3.27 (1)]{SzamuelyZabradi}).
\end{proof}

\begin{defn}[{$\O_E$-prism over $\O$}]\label{Definition:prism over O}
Let $k$ be a perfect field containing $\F_q$.
We will write
\[
\O := W(k) \otimes_{W(\F_q)} \O_E
\]
instead of $W_{\O_E}(k)$.
Let $(A, I)$ be an $\O_E$-prism.
If $A$ is equipped with a homomorphism $\O \to A$ of $\delta_E$-rings, then we say that $(A, I)$ is an $\O_E$-prism \textit{over} $\O$.
There is an obvious notion of map of $\O_E$-prisms over $\O$.
\end{defn}


\begin{ex}[{cf.\ \cite[Example 1.3 (1)]{BS}}]\label{Example:crystalline type frame}
Let $A$ be a $\pi$-adically complete and $\pi$-torsion free $\O_E$-algebra.
Let
$\phi \colon A \to A$
be a homomorphism
over $\O_E$ which is a lift of the $q$-th power Frobenius of $A/\pi$.
This homomorphism induces a $\delta_E$-structure on $A$, and the pair $(A, (\pi))$ is a bounded $\O_E$-prism.
\end{ex}

Let $\O=W(k) \otimes_{W(\F_q)} \O_E$ be as in Definition \ref{Definition:prism over O}.
Let
\[
\mathfrak{S}_\O:=\O[[t_1, \dotsc, t_n]]
\]
(where $n \geq 0$)
and let
$\phi \colon \mathfrak{S}_\O \to \mathfrak{S}_\O$ be the homomorphism such that $\phi(t_i)=t^q_i$ ($1 \leq i \leq n$) and its restriction to $\O$ agrees with the Frobenius of $\O$.
Since $\mathfrak{S}_\O$ is $\pi$-torsion free, $\phi$ gives rise to a $\delta_E$-structure on $\mathfrak{S}_\O$.

\begin{prop}[{cf.\ \cite[Example 1.3 (3)]{BS}}]\label{Proposition:Breuil-Kisin type frame}
Let $\mathcal{E} \in \mathfrak{S}_\O$ be a power series whose constant term is a uniformizer of $\O$.
Then, the pair
\[
(\mathfrak{S}_\O, (\mathcal{E}))
\]
is a bounded $\O_E$-prism over $\O$.
\end{prop}

\begin{proof}
We shall show that $\pi \in (\mathcal{E}, \phi(\mathcal{E}))$; the other required conditions are clearly satisfied.
It is enough to check that $\delta_E(\mathcal{E}) \in (\mathfrak{S}_\O)^\times$.
For this, it suffices to show that the image of $\delta_E(\mathcal{E})$ in $\mathfrak{S}_\O/(t_1, \dotsc, t_n)=\O$ is a unit, which is clear since the constant term of $\mathcal{E}$ is a uniformizer of $\O$.
\end{proof}

We call $(\mathfrak{S}_\O, (\mathcal{E}))$
an $\O_E$-prism \textit{of Breuil--Kisin type}
in this paper.
Here $n$ could be any non-negative integer.
Such a pair is also considered in \cite{ChengChuangxun}.

\subsection{Perfectoid rings and $\O_E$-prisms}\label{Subsection:Perfectoid rings and pi-prisms}

The notion of (integral) perfectoid ring in the sense of \cite[Definition 3.5]{BMS} plays a central role in the theory of prismatic $G$-$\mu$-displays.
We refer to \cite[Section 3]{BMS} and \cite[Section 2]{CS} for basic properties of perfectoid rings.
We recall the definition of perfectoid rings and some notation from \cite[Section 3]{BMS}.

A ring $R$ is a \textit{perfectoid ring} if
there exists an element $\varpi \in R$
such that $p \in (\varpi)^p$ and $R$ is $\varpi$-adically complete, the Frobenius $R/p \to R/p$, $x \mapsto x^p$ is surjective, and the kernel of $\theta \colon W(R^\flat) \to R$ is principal.
Here
\[
R^\flat:=\varprojlim_{x \mapsto x^p} R/p
\]
and
\[
\theta \colon W(R^\flat) \to R
\]
is the unique homomorphism whose reduction modulo $p$ is the projection
$R^\flat \to R/p$, $(x_0, x_1, \dotsc) \mapsto x_0$.
By \cite[Lemma 3.9]{BMS}, there is an element $\varpi^\flat \in R^\flat$ such that $\theta([\varpi^\flat])$ is a unit multiple of $\varpi$, where $[-]$ denotes the Teichm\"uller lift.




Here are some typical examples of perfectoid rings:

\begin{ex}\label{Example:algebraically closed valuation ring is perfectoid}
\ 
\begin{enumerate}
    \item An $\F_p$-algebra $R$ is a perfectoid ring if and only if it is perfect; see \cite[Example 3.15]{BMS}.
    \item Let $V$ be a $p$-adically complete valuation ring with algebraically closed fraction field.
Then $V$ is a perfectoid ring. This follows from \cite[Lemma 3.10]{BMS}.
\end{enumerate}
\end{ex}


Let $\O$ be as in Definition \ref{Definition:prism over O}.
If $R$ is a perfectoid ring over $\O$
(i.e.\ $R$ is a perfectoid ring with a ring homomorphism $\O \to R$),
then $R^\flat$ is naturally a $k$-algebra, and
$W_{\O_E}(R^\flat)=W(R^\flat) \otimes_{W(\F_q)} \O_E$ is an $\O$-algebra.
Let
\[
\theta_{\O_E} \colon W_{\O_E}(R^\flat) \to R
\]
be
the homomorphism induced from $\theta$.


\begin{lem}[{cf.\ \cite[Proposition II.1.4]{FS}}]\label{Lemma:kernel generator perfectoid}
The kernel $\Ker \theta_{\O_E}$ of $\theta_{\O_E}$ is a principal ideal.
Moreover, any generator of $\Ker \theta_{\O_E}$ is a nonzerodivisor in $W_{\O_E}(R^\flat)$.
\end{lem}

\begin{proof}
Let $\varpi \in R$ be an element such that
$R$ is $\varpi$-adically complete and $p \in (\varpi)^p$.
By virtue of \cite[Lemma 3.10 (i)]{BMS}, after replacing $\varpi$ by $\theta([(\varpi^\flat)^{1/p^n}])$ for some integer $n >0$, we have $\pi \in (\varpi)$.
Then, we can write $\pi=\theta([\varpi^\flat] x)$ for some element $x \in W(R^\flat)$ since $\theta$ is surjective.
We shall show that $\pi-[\varpi^\flat] x \in W_{\O_E}(R^\flat)$ generates the kernel of $\theta_{\O_E}$.

Let $\mathcal{E}(T) \in W(k)[T]$ be the (monic) Eisenstein polynomial of $\pi \in \O$ so that we have $W(k)[T]/\mathcal{E}(T) \overset{\sim}{\to} \O$, $T \mapsto \pi$.
We see that
\begin{align*}
    W_{\O_E}(R^\flat)/(\pi-[\varpi^\flat] x) &\simeq W(R^\flat)[T]/(\mathcal{E}(T), T-[\varpi^\flat] x) \\
    &\simeq W(R^\flat)/\mathcal{E}([\varpi^\flat] x).
\end{align*}
Thus, it suffices to show that $\mathcal{E}([\varpi^\flat] x)$ generates the kernel of $\theta$.
One can check that, for the Witt vector expansion
$\mathcal{E}([\varpi^\flat] x)=(y_0, y_1, \dotsc)$, we have $y_1 \in (R^\flat)^\times$.
This implies that $\mathcal{E}([\varpi^\flat] x)$ is a generator of the kernel of $\theta$; see \cite[Remark 3.11]{BMS}.

It remains to prove that any generator $\xi$ of the kernel of $\theta_{\O_E}$ is a nonzerodivisor.
For this, we recall that the kernel of $\theta$ is generated by a nonzerodivisor $\xi' \in W(R^\flat)$.
Since $W(\F_q) \to \O_E$ is flat, the element $\xi'$ is a nonzerodivisor in $W_{\O_E}(R^\flat)$, too.
This implies that $\xi$ is a nonzerodivisor since we have $\xi' \in (\xi)=\Ker \theta_{\O_E}$.
\end{proof}



\begin{prop}[{cf.\ \cite[Example 1.3 (2)]{BS}}]\label{Proposition:perfectoid type}
Let $R$ be a perfectoid ring over $\O$ and we write $I_R:=\Ker \theta_{\O_E}$.
Then, the pair
\[
(W_{\O_E}(R^\flat), I_R)
\]
is an orientable and bounded $\O_E$-prism over $\O$.
\end{prop}

\begin{proof}
By the proof of Lemma \ref{Lemma:kernel generator perfectoid},
we know that 
$I_R$
is generated by a nonzerodivisor of the form $\xi=\pi-[(\varpi')^\flat]b$ where $\varpi' \in R$ is such that $R$ is $\varpi'$-adically complete and $p \in (\varpi')^p$.
In order to show that $W_{\O_E}(R^\flat)$ is $(\pi, \xi)$-adically complete, it suffices to show that $W(R^\flat)$ is $(p, [(\varpi')^\flat])$-adically complete, which is easy to check (see also the proof of \cite[Proposition 2.1.11 (b)]{CS}).
Moreover $W_{\O_E}(R^\flat)/\xi=R$ has bounded $p^\infty$-torsion by \cite[Lemma 3.8]{BS}.

It remains to show that $\pi \in (\xi, \phi(\xi))$.
It suffices to prove that $\delta_E(\xi) \in W_{\O_E}(R^\flat)^\times$.
The image of $\delta_E(\xi)$ in $W_{\O_E}(R^\flat)/[(\varpi')^\flat]$
is equal to $1-\pi^{q-1}$ (we note that $W_{\O_E}(R^\flat)/[(\varpi')^\flat]$ is $\pi$-torsion free) and hence is a unit, which in turn implies that $\delta_E(\xi) \in W_{\O_E}(R^\flat)^\times$.
\end{proof}

A homomorphism
$R \to S$
of perfectoid rings over $\O$ induces a map
$
(W_{\O_E}(R^\flat), I_R) \to (W_{\O_E}(S^\flat), I_S)
$
of $\O_E$-prisms over $\O$.



\subsection{Prismatic sites}\label{Subsection:prismatic sites}

For a ring $A$, let $D(A)$ denote the derived category of $A$-modules.
Let $I \subset A$ be a finitely generated ideal.
Following \cite{BS},
we say that a complex $M \in D(A)$
is \textit{$I$-completely flat}
(resp.\ \textit{$I$-completely faithfully flat})
if
$M \otimes^{\L}_A A/I$ 
is concentrated in degree $0$
and it is a flat
(resp.\ faithfully flat) $A/I$-module.


\begin{lem}\label{Lemma:flat maps over prisms}
Let $(A, I)$ be a bounded $\O_E$-prism.
\begin{enumerate}
    \item For a complex $M \in D(A)$, the derived $(\pi, I)$-adic completion of $M$ is isomorphic to
    \[
    R\varprojlim_n (M \otimes^{\L}_A A/(\pi, I)^n).
    \]
    In particular, if $M$ is $(\pi, I)$-completely flat, then the derived $(\pi, I)$-adic completion of $M$ is concentrated in degree $0$.
    \item Let M be an $A$-module.
    Assume that $M$ is $(\pi, I)$-completely flat and derived $(\pi, I)$-adically complete.
    Then $M$ is $(\pi, I)$-adically complete.
Moreover
the natural homomorphism $M \otimes_A I \to M$ is injective
and $M/I^nM$ has bounded $p^\infty$-torsion for any $n$.
\end{enumerate}
\end{lem}

\begin{proof}
(1) The assertion follows from \cite[Proposition 1.4 (1)]{Tian} or the proof of \cite[Lemma 3.7 (1)]{BS}.
This can also be deduced from the results discussed in \cite{Yekutieli21}; by \cite[Corollary 3.5, Theorem 3.11]{Yekutieli21}, it suffices to prove that the ideal $(\pi, I) \subset A$ is weakly proregular in the sense of \cite[Definition 3.2]{Yekutieli21}, which follows from the same argument as in the proof of \cite[Theorem 7.3]{Yekutieli21}.

(2) It follows from (1) that 
$M$ is $(\pi, I)$-adically complete.
The second statement can be proved in the same way as \cite[Lemma 3.7 (2)]{BS}.
\end{proof}

We say that
a map
$f \colon (A, I) \to (A', I')$ of bounded $\O_E$-prisms
is a \textit{(faithfully) flat map}
if $A \to A'$ is 
$(\pi,I)$-completely (faithfully) flat.
If $f$ is a faithfully flat map, then we say that $(A', I')$ is a flat covering of $(A, I)$.

\begin{rem}\label{Remark:flat maps of prisms}
For a map 
$f \colon (A, I) \to (A', I')$ of bounded $\O_E$-prisms,
we have
$A' \otimes^{\L}_A A/I \simeq A'/I'$ by Lemma \ref{Lemma:rigidity}, which in turn implies that
\begin{equation}\label{equation:reduction of map of prisms}
    A' \otimes^{\L}_A A/(\pi, I) \simeq A'/I' \otimes^{\L}_{A/I} A/(\pi, I).
\end{equation}
In particular, we see that $f$ is a (faithfully) flat map
if and only if $A/I \to A'/I'$ is 
$\pi$-completely (faithfully) flat.
\end{rem}

\begin{rem}\label{Remark:flat locally orientable}
    Any bounded $\O_E$-prism $(A, I)$ admits a faithfully flat map
    $(A, I) \to (B, IB)$ of bounded $\O_E$-prisms such that $(B, IB)$ is orientable.
    Indeed, by \cite[Lemma 3.4]{Marks}, we see that there exists a faithfully flat homomorphism $A \to A'$ of $\delta_E$-rings such that $IA'$ is principal.
    Let $B$ be the $(\pi, I)$-adic completion of $A'$.
    Then, by virtue of Lemma \ref{Lemma:completion of delta ring} and Lemma \ref{Lemma:flat maps over prisms},
    the pair $(B, IB)$ is an orientable and bounded $\O_E$-prism, and the map
    $(A, I) \to (B, IB)$ is faithfully flat.
\end{rem}

\begin{defn}\label{Definition:category of bounded prisms}
Let $R$ be a $\pi$-adically complete $\O_E$-algebra.
Let
\[
(R)_{\Prism, \O_E}
\]
denote the category of bounded $\O_E$-prisms $(A, I)$ together with a homomorphism $R \to A/I$ of $\O_E$-algebras.
The morphisms $f \colon (A, I) \to (A', I')$ in $(R)_{\Prism, \O_E}$ are the maps of $\O_E$-prisms such that $A/I \to A'/I'$ is a homomorphism of $R$-algebras.
We endow
the opposite category
$
(R)^{\op}_{\Prism, \O_E}
$
with the topology generated by the faithfully flat maps.
This topology is called the \textit{flat topology}.
\end{defn}

If $\O_E=\Z_p$, then $(R)_{\Prism, \O_E}$ is just the category $(R)_{\Prism}$ introduced in \cite[Remark 4.7]{BS}.

\begin{rem}\label{Remark:pushout in prismatic site}
A diagram
    \[
    (A_2, I_2) \overset{g}{\leftarrow} (A_1, I_1) \overset{f}{\rightarrow} (A_3, I_3)
    \]
    in $(R)_{\Prism, \O_E}$ such that $g$ is a flat map, admits a colimit (i.e.\ a pushout).
    Indeed, by Lemma \ref{Lemma:flat maps over prisms} (1), the $(\pi, I_3)$-adic completion
    $
    A:=(A_2 \otimes_{A_1} A_3)^\wedge
    $
    is isomorphic to the derived $(\pi, I_3)$-adic completion of $A_2 \otimes^{\L}_{A_1} A_3$.
    In particular $A$ is $(\pi, I_3)$-completely flat over $A_3$.
    It follows from Remark \ref{Remark:colimit and limit of delta rings}
    and Lemma \ref{Lemma:completion of delta ring}
    that
    $A$ admits a unique $\delta_E$-structure that is compatible with the $\delta_E$-structures on $A_2$ and $A_3$.
    By Lemma \ref{Lemma:flat maps over prisms} (2),
    we see that $(A, I_3A)$ is a bounded $\O_E$-prism.
    By construction, $(A, I_3A)$ is a colimit of the above diagram.
    As a result, it follows that
    $
    (R)^{\op}_{\Prism, \O_E}
    $
    is indeed a site.
\end{rem}

\begin{ex}\label{Example:prismatic sites}
\ 
\begin{enumerate}
    \item The category $(\O_E)_{\Prism, \O_E}$ is just the category of bounded $\O_E$-prisms.
    The category $(\O)_{\Prism, \O_E}$ is the same as the category of bounded $\O_E$-prisms over $\O$ by Lemma \ref{Lemma:maps from perfectoid prisms to prism}.
    \item Let $R$ be a perfectoid ring over $\O_E$.
    It follows from Lemma \ref{Lemma:maps from perfectoid prisms to prism} that
    $(R)_{\Prism, \O_E}$ is the category of bounded $\O_E$-prisms $(A, I)$ with a map
    $
    (W_{\O_E}(R^\flat), I_R) \to (A, I).
    $
\end{enumerate}
\end{ex}


\begin{rem}[{cf.\ \cite[Corollary 3.12]{BS}}]\label{Remark:structure sheaf}
A faithfully flat map
$(A, I) \to (A', I')$
induces
faithfully flat homomorphisms
$A/(\pi, I)^n \to A'/(\pi, I')^n$
and $A/(\pi^n, I) \to A'/(\pi^n, I')$
for any $n$.
It follows that the functors
\begin{align*}
    \O_{\Prism} &\colon (R)_{\Prism, \O_E} \to \mathrm{Set}, \quad (A, I) \mapsto A, \\
    \O_{\overline{\Prism}} &\colon
(R)_{\Prism, \O_E} \to \mathrm{Set}, \quad (A, I) \mapsto A/I
\end{align*}
form sheaves with respect to the flat topology.
Here $\mathrm{Set}$ is the category of sets.
\end{rem}

More generally, we have the following descent result.

\begin{prop}\label{Proposition:flat descent for finite projective modules}
The fibered category
over
$(\O_E)^{\op}_{\Prism, \O_E}$
which associates to each $(A, I) \in (\O_E)_{\Prism, \O_E}$
the category of finite projective $A$-modules
satisfies descent with respect to the flat topology.
The same holds for finite projective $A/I$-modules.
\end{prop}

\begin{proof}
The assertions follow by using faithfully flat descent for finite projective modules over $A/(\pi, I)^n$ and $A/(\pi^n, I)$, respectively.
See also \cite[Lemma A.1, Proposition A.3]{Anschutz-LeBras}.
\end{proof}

Let $A \to B$ be a ring homomorphism
and $I \subset A$ a finitely generated ideal.
We say that
$A \to B$
is \textit{$I$-completely \'etale}
if
$B$ is derived $I$-adically complete,
$B \otimes^{\L}_A A/I$ 
is concentrated in degree $0$,
and $B/IB$ is \'etale over $A/I$.
We write
$A_{I\mathchar`-\et}$
for the category of $I$-completely \'etale $A$-algebras.
If $I=0$, then $A_{I\mathchar`-\et}$ is just the category $A_{\et}$ of \'etale $A$-algebras.

\begin{lem}\label{Lemma:etale morphism and reduction}
Let $(A, I)$ be a bounded $\O_E$-prism.
\begin{enumerate}
    \item A ring homomorphism $A/I \to C$ is $\pi$-completely \'etale if and only if $C$ is $\pi$-adically complete and $C/\pi^n$ is \'etale over $A/(\pi^n, I)$ for every integer $n \geq 1$.
    If this is the case, then $C$ has bounded $p^{\infty}$-torsion.
    \item A ring homomorphism $A \to B$ is $(\pi, I)$-completely \'etale if and only if $B$ is $(\pi, I)$-adically complete and $B/(\pi, I)^n$ is \'etale over $A/(\pi, I)^n$ for every $n \geq 1$.
    If this is the case, then
    $B \otimes^\L_A A/I \overset{\sim}{\to} B/IB$ and $A/I \to B/IB$ is $\pi$-completely \'etale.
    \item
   The functors
   \[
   A_{(\pi, I)\mathchar`-\et} \to (A/I)_{\pi\mathchar`-\et} \to (A/(\pi, I))_{\et},
   \]
   where the first one is defined by $B \mapsto B/IB$ and the second one is defined by $C \mapsto C/\pi$,
   are equivalences.
\end{enumerate}
\end{lem}

\begin{proof}
This result is well-known to specialists, but we include a proof for the convenience of the reader.

(1) Assume that $A/I \to C$ is $\pi$-completely \'etale.
Then, since $A/I$ has bounded $p^{\infty}$-torsion,
\cite[Lemma 4.7]{BMS2} implies that $C$ is $\pi$-adically complete and has bounded $p^{\infty}$-torsion.
Since $C/\pi^n$ is flat over $A/(\pi^n, I)$ 
and $C/\pi$ is \'etale over $A/(\pi, I)$,
it follows that $C/\pi^n$ is \'etale over $A/(\pi^n, I)$ for any $n$.

We next prove the ``if'' direction, so we assume that $C$ is $\pi$-adically complete and $C/\pi^n$ is \'etale over $A/(\pi^n, I)$ for any $n$.
We want to show that $C \otimes^{\L}_{A/I} A/(\pi, I)$ is concentrated in degree $0$.
There exists an \'etale $A/I$-algebra $C_0$ such that $C_0/\pi \simeq C/\pi$ over $A/(\pi, I)$; see for example \cite[Section 1.1]{Arabia} or \cite[Tag 04D1]{SP} (this is known as a special case of Elkik's result \cite{Elkik}).
One easily see that
the derived $\pi$-adic completion of $C_0$, the $\pi$-adic completion of $C_0$,
and $C$ are isomorphic to each other.
Then, we obtain
\[
C \otimes^{\L}_{A/I} A/(\pi, I) \simeq C_0 \otimes^{\L}_{A/I} A/(\pi, I) \simeq C_0/\pi.
\]
This proves the assertion.

(2) Assume that $A \to B$ is $(\pi, I)$-completely \'etale.
We easily see that $B/(\pi, I)^n$ is \'etale over $A/(\pi, I)^n$.
It follows from Lemma \ref{Lemma:flat maps over prisms} that $B$ is $(\pi, I)$-adically complete
and
we have
$B \otimes^\L_A A/I \overset{\sim}{\to} B/IB$.
It is then clear that $A/I \to B/IB$ is $\pi$-completely \'etale.

The ``if'' direction can be proved by the same argument as in (1).
Suppose that $B$ is $(\pi, I)$-adically complete and $B/(\pi, I)^n$ is \'etale over $A/(\pi, I)^n$ for any $n$.
As above, 
there exists an \'etale $A$-algebra $B_0$ such that
the $(\pi, I)$-adic completion of $B_0$ is isomorphic to $B$.
It follows from Lemma \ref{Lemma:flat maps over prisms} (1)
that
$B$ is isomorphic to the derived 
$(\pi, I)$-adic completion of $B_0$, which in turn implies that $B$ is $(\pi, I)$-completely \'etale over $A$.


(3) This follows from the proofs of (1) and (2).
\end{proof}

\begin{lem}[{cf.\ \cite[Lemma 2.18]{BS}}]\label{Lemma:etale morphism prism}
Let $(A, I)$ be a bounded $\O_E$-prism
and $A \to B$ a $(\pi, I)$-completely \'etale homomorphism.
Then $B$ admits a unique $\delta_E$-structure compatible with that on $A$.
Moreover, the pair $(B, IB)$ is a bounded $\O_E$-prism.
\end{lem}

\begin{proof}
It suffices to prove the first statement by Lemma \ref{Lemma:flat maps over prisms}.
For this, we proceed as in the proof of \cite[Lemma 2.18]{BS}.

We regard
$W_{\O_E, \pi, 2}(B)$
as an $A$-algebra via the composition $A \to W_{\O_E, \pi, 2}(A) \to W_{\O_E, \pi, 2}(B)$,
where $A \to W_{\O_E, \pi, 2}(A)$ is the homomorphism corresponding to the $\delta_E$-structure on $A$ (Remark \ref{Remark:ramified Witt vectors of length 2}).
Then $W_{\O_E, \pi, 2}(B)$ is $(\pi, I)$-adically complete.
Indeed, we have an exact sequence of $A$-modules
\[
0 \to \phi_*B \overset{V}{\to} W_{\O_E, \pi, 2}(B) \overset{\epsilon}{\to} B \to 0,
\]
where we write $\phi_*B$ for $B$ regarded as an $A$-algebra via the composition $A \overset{\phi}{\to} A \to B$, and
$V \colon \phi_*B \to W_{\O_E, \pi, 2}(B)$ is defined by $x \mapsto (0, x)$.
Since $B \otimes^{\L}_A A/(\pi, I)^n$ is concentrated in degree $0$
and both $\phi_*B$ and $B$ are $(\pi, I)$-adically complete, we can conclude that $W_{\O_E, \pi, 2}(B)$ is $(\pi, I)$-adically complete.

As in the proof of Lemma \ref{Lemma:etale morphism and reduction},
there exists an \'etale $A$-algebra $B_0$ such that
the $(\pi, I)$-adic completion of $B_0$ is isomorphic to $B$.
Since $W_{\O_E, \pi, 2}(B)$ is $(\Ker \epsilon)$-adically complete and $A \to B_0$ is \'etale,
there exists a unique homomorphism $s_0 \colon B_0 \to W_{\O_E, \pi, 2}(B)$ of $A$-algebras such that $\epsilon \circ s_0$ coincides with $B_0 \to B$.
Then, since $W_{\O_E, \pi, 2}(B)$ is $(\pi, I)$-adically complete, we see that
$s_0 \colon B_0 \to W_{\O_E, \pi, 2}(B)$
extends to a unique homomorphism $s \colon B \to W_{\O_E, \pi, 2}(B)$ of $A$-algebras such that $\epsilon \circ s= \id_B$, which corresponds to a unique $\delta_E$-structure on $B$ compatible with that on $A$ by virtue of Remark \ref{Remark:ramified Witt vectors of length 2}.
\end{proof}

\begin{ex}\label{Example:perfectoid ring etale morphism}
Let $R$ be a perfectoid ring over $\O_E$
and let $R \to S$ be a $\pi$-completely \'etale (or equivalently, $p$-completely \'etale) homomorphism.
By \cite[Corollary 2.10]{Anschutz-LeBras} or \cite[Lemma 8.11]{Lau18}, we see that $S$ is a perfectoid ring.
Moreover, the isomorphism (\ref{equation:reduction of map of prisms}) implies that
$W_{\O_E}(R^\flat) \to W_{\O_E}(S^\flat)$ is $(\pi, I_R)$-completely \'etale.
\end{ex}


Let $(A, I)$ be a bounded $\O_E$-prism.
We say that
a homomorphism $B \to B'$
of $(\pi, I)$-completely \'etale $A$-algebras
is a \textit{$(\pi, I)$-completely \'etale covering} if
\[
\Spec B'/(\pi, I) \to \Spec B/(\pi, I)
\]
is surjective, or equivalently, the homomorphism $B \to B'$ is $(\pi, I)$-completely faithfully flat.
We note that $B \to B'$ is automatically $(\pi, I)$-completely \'etale.

\begin{defn}\label{Definition:etale prismatic site}
We write
\[
(A, I)_\et
\]
for the category of $(\pi, I)$-completely \'etale $A$-algebras instead of $A_{(\pi, I)\mathchar`-\et}$.
We endow
the opposite category
$
(A, I)^{\op}_\et
$
with the topology generated by the $(\pi, I)$-completely \'etale coverings, which is called the \textit{$(\pi, I)$-completely \'etale topology}.
\end{defn}

The category
$(A, I)^{\op}_\et$
admits fiber products.
Indeed, a colimit of the diagram
$
C \leftarrow B \rightarrow D
$
in $(A, I)_\et$
is given by the $(\pi, I)$-adic completion of $C \otimes_B D$.
It follows that $(A, I)^{\op}_\et$ is a site.

\subsection{Prismatic envelopes for regular sequences}\label{Subsection:Prismatic envelopes for regular sequences}

The existence and the flatness of prismatic envelopes for regular sequences are proved in \cite[Proposition 3.13]{BS}.
In this subsection, 
we give an analogous result for $\O_E$-prisms.
We will freely use the formalism of \textit{animated rings} here.
For the definition and properties of animated rings, see for example \cite[Section 5]{CS} and \cite[Appendix A]{BL}.
(See also \cite[Chapter 25]{SAG}, where animated rings are called simplicial rings.)

We recall some terminology from \cite{CS} and \cite{BS}.
For an animated ring $A$, we can attach a graded ring of homotopy groups
$\oplus_{n \geq 0}\pi_n(A)$.
For an animated ring $A$ and a sequence $x_1, \dotsc, x_n \in \pi_0(A)$,
the \textit{derived quotient} of $A$ with respect to $x_1, \dotsc, x_n \in \pi_0(A)$ is defined by
\[
A/^{\L}(x_1, \dotsc, x_n):= A \otimes^{\L}_{\Z[X_1, \dotsc, X_n]}  \Z[X_1, \dotsc, X_n]/(X_1, \dotsc, X_n).
\]
Here $\Z[X_1, \dotsc, X_n] \to A$ is a morphism such that the induced ring homomorphism
$\Z[X_1, \dotsc, X_n] \to \pi_0(A)$ is given by $X_i \mapsto x_i$.
In \cite{BS}, the derived quotient is denoted by $\Kos(A; x_1, \dotsc, x_n)$.
We say that a morphism $A \to B$ of animated rings is \textit{flat} (resp.\ \textit{faithfully flat}) if
$\pi_0(B)$ is flat (resp.\ faithfully flat) over $\pi_0(A)$ and we have
$\pi_n(A) \otimes_{\pi_0(A)} \pi_0(B) \overset{\sim}{\to} \pi_n(B)$ for any $n \geq 0$.


Before stating the result, let us quickly recall the definition of an $\O_E$-PD structure, and their relation to $\delta_E$-structures.

\begin{defn}[{\cite[Section 10]{HopkinsGross}, \cite[Definition 14]{Faltings02}}]\label{Definition:pi PD strudture}
    Let $A$ be an $\O_E$-algebra and $I \subset A$ an ideal.
    An \textit{$\O_E$-PD structure} on $I$ is a map $\gamma_\pi \colon I \to I$ of sets with the following properties:
    \begin{enumerate}
    \item $\pi\gamma_\pi(x)=x^q$.
    \item $\gamma_\pi(ax)=a^q\gamma_\pi(x)$ where $a \in A$.
    \item $\gamma_\pi(x+y)=\gamma_\pi(x)+\gamma_\pi(y)+(x^q+y^q-(x+y)^q)/\pi$.
    \end{enumerate}
\end{defn}



\begin{ex}[\cite{Faltings02}]\label{Example:pi PD polynomial ring}
    Let $n \geq 0$ be an integer
    and
    let
    $\O_E[ (X_{i, j}) ]$
    be the polynomial ring with variables $X_{i, j}$ indexed by integers $i, j$ with $1 \leq i \leq n$ and $j \geq 0$.
    We write
    \[
    \O_E[X_1, \dotsc, X_n]^{\mathrm{PD}}
    \]
    for the quotient of $\O_E[ (X_{i, j}) ]$ by the ideal generated by the elements $X^q_{i, j}-\pi X_{i, j+1}$ for all $i, j$.
    The image of $X_{i, 0}$ in $\O_E[X_1, \dotsc, X_n]^{\mathrm{PD}}$ is denoted by $X_i$.
    We see that $\O_E[X_1, \dotsc, X_n]^{\mathrm{PD}}$ is canonically isomorphic to the $\O_E$-subalgebra of $E[X_1, \dotsc, X_n]$ generated by the elements $X^{q^j}_i/\pi^{1 + q + \cdots + q^{j-1}}$ ($1 \leq i \leq n$ and $j \geq 0$).
    The ideal $I^{\mathrm{PD}} \subset \O_E[X_1, \dotsc, X_n]^{\mathrm{PD}}$ generated by the elements $X_{i, j}$ admits an $\O_E$-PD structure $\gamma_\pi$ such that $\gamma_\pi(X_{i, j})=X_{i, j+1}$.
    In fact, the pair $(\O_E[X_1, \dotsc, X_n]^{\mathrm{PD}}, I^{\mathrm{PD}})$ is the $\O_E$-PD envelope (in the usual sense) of the polynomial ring $\O_E[X_1, \dotsc, X_n]$ with respect to the ideal $(X_1, \dotsc, X_n)$.
\end{ex}

\begin{lem}[{cf.\ \cite[Lemma 2.38]{BS}}]\label{Lemma:pd envelope regular sequence}
    Let $B$ be a $\pi$-torsion free $\O_E$-algebra.
    Let $x_1, \dotsc, x_n \in B$ be a sequence
    such that
    $(B/\pi)/^\L(\overline{x}_1, \dotsc, \overline{x}_n)$
    is concentrated in degree $0$, where $\overline{x}_1, \dotsc, \overline{x}_n \in B/\pi$ are the images of $x_1, \dotsc, x_n \in B$.
    We set
    \[
    C:= B \otimes^\L_{\O_E[X_1, \dotsc, X_n]} \O_E[X_1, \dotsc, X_n]^{\mathrm{PD}}
    \]
    where
    $\O_E[X_1, \dotsc, X_n] \to B$ is defined by $X_i \mapsto x_i$.
    Then $C$ is concentrated in degree $0$ and $\pi_0(C)$ is $\pi$-torsion free.
    Moreover the pair $(\pi_0(C), I^{\mathrm{PD}}\pi_0(C))$ is the $\O_E$-PD envelope of $B$ with respect to the ideal $(x_1, \dotsc, x_n)$.
\end{lem}

\begin{proof}
The proof is identical to that of \cite[Lemma 2.38]{BS}.
\end{proof}

In the following, we will write
\[
D_{(x_1, \dotsc, x_n)}(B):=\pi_0(C)=B \otimes_{\O_E[X_1, \dotsc, X_n]} \O_E[X_1, \dotsc, X_n]^{\mathrm{PD}}.
\]

We also need the following construction.
Let $B$ be a $\delta_E$-ring.
Let $d \in B$ be an element, and $x_1, \dotsc, x_n \in B$ a sequence.
We set
$B\{ X \}:=B \otimes_{\O_E} \O_E\{ X \}$
and
let
\[
B \{ X_1, \dotsc, X_n \}
\]
be the $n$-th tensor power of $B\{ X \}$ over $B$.
We consider the following diagram of $\delta_E$-rings
\[
B \overset{f}{\leftarrow} B \{ X_1, \dotsc, X_n \} \overset{g}{\rightarrow} B \{ Y_1, \dotsc, Y_n \},
\]
where $f$ is defined by $X_i \mapsto x_i$ and $g$ is defined by $X_i \mapsto dY_i$.
Let
\[
B\{ x_1/d, \dotsc, x_n/d \}
\]
denote
the pushout of this diagram, which is a $\delta_E$-ring over $B$ with the following property:
For a homomorphism $B \to C$ of $\delta_E$-rings such that the image of $d$ is a nonzerodivisor in $C$ and 
$x_i \in dC$ for all $i$, there exists a unique homomorphism
$B\{ x_1/d, \dotsc, x_n/d \} \to C$
of $\delta_E$-rings over $B$.
We let $x_i/d \in B\{ x_1/d, \dotsc, x_n/d \}$ denote the image of $Y_i \in B \{ Y_1, \dotsc, Y_n \}$.

\begin{lem}[{cf.\ \cite[Corollary 2.39]{BS}}]\label{Lemma:pd envelope and delta structure}
    Let $B$ be a $\pi$-torsion free $\delta_E$-ring.
    Let $x_1, \dotsc, x_n \in B$ be a sequence
    such that
    $(B/\pi)/^\L(\overline{x}_1, \dotsc, \overline{x}_n)$
    is concentrated in degree $0$.
    We set
    $D:=B \otimes^\L_{\O_E \{ X_1, \dotsc, X_n \}} \O_E \{ Y_1, \dotsc, Y_n \}$
    where $\O_E \{ X_1, \dotsc, X_n \} \to B$ is defined by $X_i \mapsto \phi(x_i)$ and
    $\O_E \{ X_1, \dotsc, X_n \} \to \O_E \{ Y_1, \dotsc, Y_n \}$ is defined by $X_i \mapsto \pi Y_i$.
    Then $D$ is concentrated in degree $0$.
    Moreover
    \[
    \pi_0(D)=B \{ \phi(x_1)/\pi, \dotsc, \phi(x_n)/\pi \}
    \]
    is $\pi$-torsion free, and is isomorphic to $D_{(x_1, \dotsc, x_n)}(B)$ as a $B$-algebra.
\end{lem}

\begin{proof}
    This can be proved in the same way as \cite[Corollary 2.39]{BS}.
\end{proof}


Now we can state the desired result:

\begin{prop}[{cf.\ \cite[Proposition 3.13]{BS}}]\label{Proposition:prismatic envelope}
    Assume that $(A, I)$ is an orientable and bounded $\O_E$-prism.
    Let $d \in I$ be a generator.
    Let $B$ be a $\delta_E$-ring over $A$.
    Let $x_1, \dotsc, x_n \in B$ be a sequence such that
    the induced morphism
    \begin{equation}\label{equation:regular sequence derived quotient}
        A/^\L(\pi, d) \to B/^\L(\pi, d, x_1, \dotsc, x_n)
    \end{equation}
    of animated rings is flat.
    We set $J:=(d, x_1, \dotsc, x_n) \subset B$.
    Then, the following assertions hold:
    \begin{enumerate}
        \item The $(\pi, I)$-adic completion $B\{ J/I \}^{\wedge}$ of $B\{ x_1/d, \dotsc, x_n/d \}$ is $(\pi, I)$-completely flat over $A$.
        In particular,
        the pair
        \[
        (B\{ J/I \}^{\wedge}, IB\{ J/I \}^{\wedge})
        \]
        is an orientable and bounded $\O_E$-prism.
        Moreover
        $B\{ J/I \}^{\wedge}$ is $(\pi, I)$-completely faithfully flat over $A$
        if the morphism
        (\ref{equation:regular sequence derived quotient})
        is faithfully flat.
        \item For a bounded $\O_E$-prism $(D, ID)$ over $(A, I)$
        and a homomorphism $B \to D$ of $\delta_E$-rings over $A$ such that $JD \subset ID$,
        there exists a unique map of $\O_E$-prisms
        \[
        (B\{ J/I \}^{\wedge}, IB\{ J/I \}^{\wedge}) \to (D, ID)
        \]
        over $B$.
        Moreover,
        the formation of $B\{ J/I \}^{\wedge}$ commutes with base change along any map $(A, I) \to (A', I')$ of bounded $\O_E$-prisms, and also commutes with base change along any $(\pi, I)$-completely flat homomorphism $B \to B'$ of $\delta_E$-rings.
    \end{enumerate}
\end{prop}

See \cite[Proposition 3.13 (3)]{BS} for the precise meaning of the last statement.

\begin{proof}
Let $C:=B \otimes^\L_{B \{ X_1, \dotsc, X_n \}} B \{ Y_1, \dotsc, Y_n \}$ be the pushout of the diagram
$
B \leftarrow B \{ X_1, \dotsc, X_n \} \rightarrow B \{ Y_1, \dotsc, Y_n \},
$
where $B \{ X_1, \dotsc, X_n \} \to B$ is defined by $X_i \mapsto x_i$ and $B \{ X_1, \dotsc, X_n \} \rightarrow B \{ Y_1, \dotsc, Y_n \}$ is defined by $X_i \mapsto dY_i$, in the $\infty$-category of animated rings.
It suffices to prove that
if the morphism
(\ref{equation:regular sequence derived quotient})
is flat (resp.\ faithfully flat), then $C$ is $(\pi, I)$-completely flat (resp.\ $(\pi, I)$-completely faithfully flat) over $A$.
Indeed, if $C$ is $(\pi, I)$-completely flat (resp.\ $(\pi, I)$-completely faithfully flat) over $A$, it follows that the derived $(\pi, I)$-adic completion of $C$ is isomorphic to $B\{ J/I \}^{\wedge}$,
and in particular $B\{ J/I \}^{\wedge}$ is $(\pi, I)$-completely flat (resp.\ $(\pi, I)$-completely faithfully flat) over $A$.
Then, it is easy to see that $B\{ J/I \}^{\wedge}$ has the desired properties.

In order to prove that $C$ is $(\pi, I)$-completely flat (resp.\ $(\pi, I)$-completely faithfully flat) over $A$, one can argue as in the proof of \cite[Proposition 3.13]{BS}.
(The faithful flatness is not discussed in \textit{loc.cit.}, but the same argument works.)
The only difference is that here we have to use $\O_E$-PD structures, instead of usual PD structures.
The details are left to the reader.
\end{proof}

The bounded $\O_E$-prism
$(B\{ J/I \}^{\wedge}, IB\{ J/I \}^{\wedge})$
is called the \textit{prismatic envelope} of $B$ over $(A, I)$ with respect to the ideal $J$.

\begin{rem}\label{Remark:animated pi delta rings}
As in \cite[Proposition 3.13]{BS}, we need to use animated $\delta_E$-rings in the proof of Proposition \ref{Proposition:prismatic envelope}.
One can define the notion of animated $\delta_E$-ring in the same way as in \cite[Section 5]{Mao} (i.e.\ by animating $\delta_E$-rings).
Alternatively, we can follow the approach employed in \cite[Appendix A]{BL2}.
\end{rem}



\section{Displayed Breuil--Kisin module}\label{Section:Displayed Breuil--Kisin module}

In this section, we study Breuil--Kisin modules for bounded $\O_E$-prisms.
We introduce the notions of displayed Breuil--Kisin module and of minuscule Breuil--Kisin module.
These objects serve as examples of prismatic $G$-$\mu$-displays introduced in Section \ref{Section:prismatic G display}, namely they will appear as prismatic $\GL_n$-$\mu$-displays.


\subsection{Displayed Breuil--Kisin modules}\label{Subsection:Displayed Breuil--Kisin modules}

We use the following notation.
Let $A$ be a ring.
For modules $M$ and $N$ over $A$,
the set of $A$-linear homomorphisms $M \to N$ is denoted by $\Hom_A(M, N)$.
Let $I \subset A$ be a Cartier divisor.
For an integer $n \geq 1$, we define
$I^{\otimes -n}:= \Hom_A(I^{\otimes n}, A)$, where $I^{\otimes n}$ is the $n$-th tensor power of $I$ over $A$.
We have a natural injection $I^{\otimes -n} \hookrightarrow I^{\otimes -n-1}$ for any integer $n$.
We then define
\[
A[1/I]:=\varinjlim_n I^{\otimes -n},
\]
which has the structure of an $A$-algebra.
For a module $M$ over $A$, we write
$M[1/I]:=M \otimes_A A[1/I]$.
If $I$ is generated by a nonzerodivisor $d \in I$, then we have $A[1/I]=A[1/d]$ and $I^{\otimes -n}=d^{-n}A$.

\begin{lem}\label{Lemma:Brauil-Kisin module, filtration and height}
Let $M, N$ be finite projective $A$-modules and
let $F \colon N[1/I] \overset{\sim}{\to} M[1/I]$ be an $A[1/I]$-linear isomorphism.
For an integer $i$, we set
\[
\Fil^i(N):= \{ \, x \in N \, \vert \, F(x) \in M \otimes_A I^{\otimes i} \, \},
\]
where we view $M \otimes_A I^{\otimes i}$ as a subset of $M[1/I]$.
Let $m$ be an integer.
Then, the following are equivalent:
    \begin{enumerate}
        \item $\Fil^{m+1}(N) \subset IN$.
        \item $M \otimes_A I^{\otimes m} \subset F(N)$.
    \end{enumerate}
    If these equivalent conditions are satisfied, then $F$ restricts to an isomorphism
    $\Fil^m(N) \overset{\sim}{\to} M \otimes_A I^{\otimes m}$, and in particular $\Fil^m(N)$ is a finite projective $A$-module.
\end{lem}

\begin{proof}
    The final statement clearly follows from (2).
    We shall prove that (1) and (2) are equivalent.
    For this, we can reduce to the case where $I$ is generated by a nonzerodivisor $d \in I$.
    
    Assume that (1) holds.
    Let $x \in M$.
    We want to show that $d^m x \in F(N)$.
    For a large enough integer $n$, we have $d^n x \in F(N)$.
    Let $y \in N$ be an element such that $F(y)=d^n x$.
    If $n \geq m+1$, then we have $y \in \Fil^{m+1}(N) \subset IN$, which in turn implies that $d^{n-1} x \in F(N)$.
    From this observation, we can conclude that $d^m x \in F(N)$.

    Assume that (2) holds.
    Let $y \in \Fil^{m+1}(N)$.
    There exists an element $x \in M$ such that $F(y)=d^{m+1}x$.
    The condition (1) implies that $d^mx=F(z)$ for some $z \in N$.
    Then, it follows that $y = dz \in IN$.
\end{proof}

Let $(A, I)$ be a bounded $\O_E$-prism.

\begin{defn}\label{Definition:Breuil-Kisin module}
    A \textit{Breuil--Kisin module} over $(A, I)$ is a pair $(M, F_M)$ consisting of a finite projective $A$-module $M$ and an $A[1/I]$-linear isomorphism
    \[
    F_M \colon (\phi^*M)[1/I] \overset{\sim}{\to} M[1/I],
    \]
    where $\phi^*M:=A \otimes_{\phi, A}{M}$.
    When there is no possibility of confusion, we simply write $M$ instead of $(M, F_M)$.
    For an integer $i$, we set
    \[
    \Fil^i(\phi^*M):= \{ \, x \in \phi^*M \, \vert \, F_M(x) \in M \otimes_A I^{\otimes i} \, \}.
    \]
    Let $P^i \subset (\phi^*M)/I(\phi^*M)$ be the image of $\Fil^i(\phi^*M)$.
    We often write
    \[
    M_{\dR}:=(\phi^*M)/I(\phi^*M).
    \]
\end{defn}

\begin{rem}\label{Remark:effective Breuil-Kisin module}
If the equality $\Fil^0(\phi^*M)=\phi^*M$ holds, or equivalently, if $F_M(\phi^*M) \subset M$,
then we say that $(M, F_M)$ is \textit{effective}.
In this case, the induced homomorphism $\phi^*M \to M$ is again denoted by $F_M$.
We note that the cokernel of $F_M \colon \phi^*M \to M$ is killed by some power of $I$.

Conversely,
for a finite projective $A$-module $M$ and a homomorphism $F_M \colon \phi^*M \to M$
of $A$-modules
whose cokernel is killed by some power of $I$,
the induced homomorphism
$(\phi^*M)[1/I] \to M[1/I]$ is an isomorphism.
Indeed, it is clear that $(\phi^*M)[1/I] \to M[1/I]$ is surjective, which in turn implies that it is an isomorphism since (the vector bundles on $\Spec A$ associated with) $\phi^* M$ and $M$ have the same rank.
In particular, it follows that $F_M \colon \phi^*M \to M$ is injective.
\end{rem}



\begin{rem}\label{Remark:filtration of Breuil-Kisin module}
For any integer $i$, we have $I\Fil^{i-1}(\phi^*M)=\Fil^i(\phi^*M) \cap I(\phi^*M)$.
In other words,
the natural homomorphism
$\Fil^{i}(\phi^*M)/I\Fil^{i-1}(\phi^*M) \to P^{i}$
is bijective.
We have $P^i=M_{\dR}$ for small enough $i$ and $P^i=0$ for large enough $i$.
\end{rem}


\begin{defn}\label{Definition:displayed and minuscule Breuil-Kisin module}
    Let $M$ be a Breuil--Kisin module over $(A, I)$.
    We say that $M$ is \textit{displayed} if the $A/I$-submodule $P^i \subset M_{\dR}$ is a direct summand for every $i$.
    In this case, the filtration $\{ P^i \}_{i \in \Z}$ is called the \textit{Hodge filtration}.
    We say that $M$ is \textit{minuscule} if $(M, F_M)$ is displayed, and if we have $P^i=M_{\dR}$ for any $i \leq 0$ and $P^i=0$ for any $i \geq 2$.
\end{defn}

The following proposition, which is basically a consequence of \cite[Remark 4.25]{Anschutz-LeBras}, shows that the definition of a minuscule Breuil--Kisin module given in Definition \ref{Definition:displayed and minuscule Breuil-Kisin module} agrees with the usual one employed in the literature (for example in \cite[Section 2.2]{Kisin06} and \cite[Definition 4.24]{Anschutz-LeBras}).

\begin{prop}\label{Proposition:minuscule equivalent condition}
     Let $M$ be a Breuil--Kisin module over $(A, I)$.
     The following statements are equivalent:
     \begin{enumerate}
         \item $M$ is minuscule.
         \item $M$ is effective, and the cokernel $\Coker F_M$ of $F_M \colon \phi^*M \to M$ is killed by $I$.
     \end{enumerate}
\end{prop}

\begin{proof}
    Assume that (1) holds.
    It follows from $P^0=M_{\dR}$ and Nakayama's lemma that $\Fil^0(\phi^*M)=\phi^*M$.
    Moreover, we have $IM \subset F_M(\phi^*M)$ by Lemma \ref{Lemma:Brauil-Kisin module, filtration and height}.
    This proves that (1) implies (2).
    

    Assume that (2) holds.
    It follows from Lemma \ref{Lemma:Brauil-Kisin module, filtration and height} that $\Fil^{2}(\phi^*M) \subset I(\phi^*M)$, and hence $P^{i}=0$ for any $i \geq 2$.
    Since $M$ is effective, we have $P^{i}=M_{\dR}$ for any $i \leq 0$.
    It remains to prove that $P^1$ is a direct summand of $M_{\dR}$.
    For this, it suffices to show that $(\phi^*M)/\Fil^{1}(\phi^*M)$ is projective as an $A/I$-module.
    Since
    we have
    the following exact sequence of $A/I$-modules
    \[
    0 \to  (\phi^*M)/\Fil^1 (\phi^*M) \to M/IM \to \Coker F_M \to 0,
    \]
    is suffices to prove that 
    $\Coker F_M$
    is a projective $A/I$-module.
    With Lemma \ref{Lemma:morphism to crystalline prisms} below, this follows from the same argument as in \cite[Remark 4.25]{Anschutz-LeBras}.
\end{proof}


\begin{lem}\label{Lemma:morphism to crystalline prisms}
    Let $(A, I)$ be an $\O_E$-prism.
    For a perfect field $k$ containing $\F_q$ and a homomorphism $g \colon A/I \to k$ of $\O_E$-algebras, there exists a map
    $(A, I) \to (\O, (\pi))$
    of $\O_E$-prisms which induces $g$, where $\O:=W(k)\otimes_{W(\F_q)} \O_E$.
\end{lem}

\begin{proof}
    Let
    $A_{\perf}:=\varinjlim_\phi A$ be a colimit of the diagram
\[
A \overset{\phi}{\rightarrow} A \overset{\phi}{\rightarrow} A  \rightarrow \cdots,
\]
which is a perfect $\delta_E$-ring.
Since $k$ is perfect,
the homomorphism
$A/\pi \to k$ induced by the composition $A \to A/I \to k$ factors through a homomorphism $A_{\perf}/\pi \to k$.
This homomorphism then lifts uniquely to a homomorphism $A_{\perf} \to \O$ of $\delta_E$-rings by Lemma \ref{Lemma:maps from perfectoid prisms to prism}.
The composition $A \to A_{\perf} \to \O$ gives a map
$(A, I) \to (\O, (\pi))$
of $\O_E$-prisms which induces $g$, as desired.
\end{proof}

The following lemma provides a useful decomposition of $\phi^*M$
for a displayed Breuil--Kisin module $M$.

\begin{lem}\label{Lemma:normal decomposition for displayed BK module}
Let $M$ be a displayed Breuil--Kisin module over $(A, I)$.
Then, there exists a decomposition
$
\phi^*M = \bigoplus_{j \in \Z} L_j
$
such that
\[
\Fil^i(\phi^*M)= (\bigoplus_{j \geq i} L_j) \oplus (\bigoplus_{j < i} I^{i-j}L_j)
\]
for every $i$.
\end{lem}

We call such a decomposition
$
\phi^*M = \bigoplus_{j \in \Z} L_j
$
a \textit{normal decomposition}.

\begin{proof}
We choose a decomposition
$
(\phi^*M)/I(\phi^*M)=\bigoplus_{j\in \Z} K_j
$
such that $P^i=\bigoplus_{j \geq i} K_{j}$ for every $i$.
Since $A$ is $I$-adically complete,
there exists a finite projective $A$-module $L_j$
such that $L_j/IL_j \simeq K_j$ for every $j$.
Since $\Fil^{i}(\phi^*M) \to P^i$ is surjective,
there exists a homomorphism $L_i \to \Fil^{i}(\phi^*M)$
which fits into the following commutative diagram:
\[
\xymatrix{
L_i \ar^-{}[r]  \ar[d]_-{} & K_i  \ar[d]_-{} \\
\Fil^{i}(\phi^*M) \ar[r]^-{} & P^i.
}
\]
The induced homomorphism
$\bigoplus_{j\in \Z} L_j \to \phi^*M$
is an isomorphism
since it is a lift of the isomorphism $\bigoplus_{j\in \Z} K_j \overset{\sim}{\to}
(\phi^*M)/I(\phi^*M)$.
We shall prove that, under this isomorphism,
$\Fil^i(\phi^*M)$ coincides with $(\bigoplus_{j \geq i} L_j) \oplus (\bigoplus_{j < i} I^{i-j}L_j)$
for any $i \in \Z$.

We proceed by induction on $i$.
The assertion clearly holds for small enough $i$.
Let us assume that the assertion holds for an integer $i$.
Since
\[
(\bigoplus_{j \geq i} IL_j) \oplus (\bigoplus_{j < i} I^{i+1-j}L_j)=I\Fil^i(\phi^*M) \subset \Fil^{i+1}(\phi^*M)
\]
and $\bigoplus_{j \geq i+1} L_j \subset \Fil^{i+1}(\phi^*M)$ by construction, we obtain
\[
(\bigoplus_{j \geq i+1} L_j) \oplus (\bigoplus_{j < i+1} I^{i+1-j}L_j) \subset \Fil^{i+1}(\phi^*M).
\]
The left hand side contains $I\Fil^i(\phi^*M)$ and the quotient by $I\Fil^i(\phi^*M)$ is equal to $P^{i+1}$.
The same holds for the right hand side by Remark \ref{Remark:filtration of Breuil-Kisin module}.
Therefore, this inclusion is actually an equality.
\end{proof}



Let $f \colon (A, I) \to (A', I')$ be a map of bounded $\O_E$-prisms.
For a Breuil--Kisin module $(M, F_M)$ over $(A, I)$,
let
$
F_{M_{A'}} \colon (\phi^*(M_{A'}))[1/I'] \overset{\sim}{\to} (M_{A'})[1/I']
$
be the base change of $F_M$, where $M_{A'}:= M \otimes_A A'$.
We also write $f^*M$ for $(M_{A'}, F_{M_{A'}})$.

\begin{prop}\label{Proposition:displayed condition base change}
Let $(M, F_M)$ be a Breuil--Kisin module over $(A, I)$.
\begin{enumerate}
    \item Assume that $(M, F_M)$ is displayed.
    Then $(M_{A'}, F_{M_{A'}})$
    is a displayed Breuil--Kisin module over $(A', I')$, and we have
    $\Fil^i(\phi^*M) \otimes_A A' \overset{\sim}{\to} \Fil^i(\phi^*(M_{A'}))$ for any integer $i$.
    \item Assume that $(M_{A'}, F_{M_{A'}})$ is displayed and $f \colon (A, I) \to (A', I')$ is faithfully flat. Then $(M, F_M)$ is displayed.
\end{enumerate}
\end{prop}

\begin{proof}
    We first make a general remark.
    Let $m$ be an integer such that $M \otimes_A I^{\otimes m} \subset F_M(\phi^*M)$.
    Then, the isomorphism $F_M$ restricts to $\Fil^m(\phi^*M) \overset{\sim}{\to} M \otimes_A I^{\otimes m}$.
    Together with $I \otimes_A A'\overset{\sim}{\to}  I'$ (see Lemma \ref{Lemma:rigidity}),
    it follows that
    $\Fil^m(\phi^*M) \otimes_A A' \overset{\sim}{\to} \Fil^m(\phi^*(M_{A'}))$.
    Moreover, we have the equality
    \[
        \Fil^i(\phi^*M)= \phi^*M \cap (\Fil^m(\phi^*M) \otimes_A I^{\otimes i-m})
    \]
    in $(\phi^*M)[1/I]$ for any integer $i$.
    

    (1) 
    Let
$
\phi^*M = \bigoplus_{j \in \Z} L_j
$
be a normal decomposition as in Lemma \ref{Lemma:normal decomposition for displayed BK module}.
We write $L'_j:= L_j \otimes_A A'$.
It suffices to prove that the equality
\[
\Fil^i(\phi^*(M_{A'}))= (\bigoplus_{j \geq i} L'_j) \oplus (\bigoplus_{j < i} I'^{i-j}L'_j)
\]
holds
in $(\phi^*(M_{A'}))[1/I']$ for any integer $i$.
This follows from the following chain of equalities:
\begin{equation*}
\begin{split}
\Fil^i(\phi^*(M_{A'})) &= \phi^*(M_{A'}) \cap (\Fil^m(\phi^*(M_{A'})) \otimes_{A'} I'^{\otimes i-m}) \\
&=\phi^*(M_{A'}) \cap (\Fil^m(\phi^*(M)) \otimes_{A} I'^{\otimes i-m}) \\
&= \phi^*(M_{A'}) \cap (\bigoplus_{j \leq m}  (L'_j  \otimes_{A'} I'^{\otimes {i-j}})) \\
&= (\bigoplus_{j \geq i} L'_j) \oplus (\bigoplus_{j < i} I'^{i-j}L'_j),
\end{split}
\end{equation*}
where the first two equalities follow from the above remark,
and for the third equality, we use the fact that $L_j=0$ for $j \geq m+1$, which follows from Lemma \ref{Lemma:Brauil-Kisin module, filtration and height} and Nakayama's lemma.

(2) We note that $\Fil^i(\phi^*(M_{A'}))$ is a finite projective $A'$-module for any integer $i$ by Lemma \ref{Lemma:normal decomposition for displayed BK module}.
By virtue of Proposition \ref{Proposition:flat descent for finite projective modules}, 
there is a descending filtration
$\{ \Fil^i \}_{i \in \Z}$ of $\phi^*M$ by finite projective $A$-submodules such that
$\Fil^i \otimes_A A' \to \phi^*(M_A')$ induces an isomorphism
$\Fil^i \otimes_A A' \overset{\sim}{\to} \Fil^i(\phi^*(M_{A'}))$ for any $i$.
We see that $F_M$ restricts to $\Fil^m \overset{\sim}{\to} M \otimes_A I^{\otimes m}$, and hence we have $\Fil^m=\Fil^m(\phi^*M)$.
Moreover, we have $I\Fil^{i-1} \subset \Fil^i$ for any $i$, and $I\Fil^{i-1} = \Fil^i$ for $i \geq m+1$.
In particular, we obtain $\Fil^i=\Fil^i(\phi^*M)$ for $i \geq m$.

Let $i$ be any integer.
We claim that the natural homomorphism of $A/I$-modules
    \[
    \iota \colon \Fil^i/I\Fil^{i-1} \to (\phi^*M)/I(\phi^*M)
    \]
is injective and its cokernel is a finite projective $A/I$-module.
Indeed, it suffices to show that for every closed point $x \in \Spec A/I$, the base change of $\iota$ to the residue field $k(x)$ is injective.
Since $x$ is contained in $\Spec A/(\pi, I)$ and $\Spec A'/(\pi, I') \to \Spec A/(\pi, I)$ is surjective,
it is enough to prove that the base change of $\iota$ along $A/I \to A'/I'$ is injective and its cokernel is a finite projective $A'/I'$-module. 
This follows from the assumption that 
$(M_{A'}, F_{M_{A'}})$ is displayed.

It follows from the claim that
$I\Fil^{i-1}= I(\phi^*M) \cap \Fil^{i}$, or equivalently, $\Fil^{i-1}= \phi^*M \cap (\Fil^{i} \otimes_A I^{\otimes -1})$.
Since $\Fil^i=\Fil^i(\phi^*M)$ for $i \geq m$,
we can conclude that $\Fil^i=\Fil^i(\phi^*M)$ for any $i$.
This, together with the claim, shows that $(M, F_M)$ is displayed.
\end{proof}

\begin{cor}\label{Corollary:flat descent for dispalyed BK modules}
The fibered category over $(\O_E)^{\op}_{\Prism, \O_E}$
which associates to a bounded $\O_E$-prism $(A, I)$ the category of displayed Breuil--Kisin modules over $(A, I)$
satisfies descent with respect to the flat topology.
\end{cor}

\begin{proof}
    This follows from Proposition \ref{Proposition:flat descent for finite projective modules} and Proposition \ref{Proposition:displayed condition base change}.
\end{proof}



\subsection{Breuil--Kisin modules of type $\mu$}


\begin{defn}\label{Definition:type of displayed BK module}
Let
$M$
be a displayed Breuil--Kisin module over $(A, I)$.
Let $\mu=(m_1, \dotsc, m_n)$ be a tuple of integers with $m_1 \geq \cdots \geq m_n$.
Let $r_i \in \Z_{\geq 0}$ be the number of occurrences of $i$ in $(m_1, \dotsc, m_n)$.
We say that $M$ is \textit{of type} $\mu$
if, for the Hodge filtration $\{ P^i \}_{i \in \Z}$, the successive quotient $P^i/P^{i+1}$ is of rank $r_i$ (i.e.\ the corresponding vector bundle on $\Spec A/I$ has constant rank $r_i$) for every $i$.
We say that $M$ is \textit{banal} if
all successive quotients $P^i/P^{i+1}$ are free $A/I$-modules.
\end{defn}

We write
$\mathrm{BK}_\mu(A, I)$
(resp.\ 
$\mathrm{BK}_\mu(A, I)_{\mathrm{banal}}$)
for the category of Breuil--Kisin modules over $(A, I)$ of type $\mu$
(resp.\ banal Breuil--Kisin modules over $(A, I)$ of type $\mu$).

\begin{rem}\label{Remark:standard filtration}
Let $e_1, \dotsc, e_n$ be the standard basis of $A^n$.
Let
$
L_{\mu, j}:= \bigoplus_{1 \leq i \leq n, \,  m_i=j} Ae_i
$
and
\[
\Fil^i_\mu:= (\bigoplus_{j \geq i} L_{\mu, j}) \oplus (\bigoplus_{j < i} I^{i-j}L_{\mu, j}).
\]
Let $M$ be a banal Breuil--Kisin module over $(A, I)$ of type $\mu$.
Let $\phi^*M = \bigoplus_{j \in \Z} L_j$ be a normal decomposition.
Then, each $L_j$ is a free $A$-module of rank $r_j$.
By choosing a basis of $L_j$ for every $j$, we obtain an isomorphism
$A^n \simeq \phi^*M$ such that
the filtration
$\{ \Fil^i_\mu \}_{i \in \Z}$
coincides with
the filtration
$\{ \Fil^i(\phi^*M) \}_{i \in \Z}$.
\end{rem}

\begin{ex}\label{Example:minuscule BK module and minuscule cocharacter}
    Assume that $\mu=(1, \dotsc, 1, 0, \dotsc , 0)$ where $1$ appears $s$ times and $0$ appears $n-s$ times.
    Then, any Breuil--Kisin module over $(A, I)$ of type $\mu$ is minuscule.
\end{ex}

In the rest of this subsection, we assume that $(A, I)$ is orientable and
we fix a generator $d \in I$.
We provide a few results that will be used to show that
any Breuil--Kisin module over $(A, I)$ of type $\mu$ arises from a $\GL_n$-$\mu$-display.


\begin{rem}\label{Remark:Frobenius isomorphism}
    Let $M$ be a Breuil--Kisin module over $(A, I)$ of type $\mu$.
    Since $P^{m_1+1}=0$, it follows from Lemma \ref{Lemma:Brauil-Kisin module, filtration and height} that $F_M$ restricts to an isomorphism
$\Fil^{m_1}(\phi^*M) \overset{\sim}{\to} d^{m_1}M$ (here $m_1$ is possibly negative).
We define
$
F_{m_1} \colon \Fil^{m_1}(\phi^*M) \overset{\sim}{\to} M
$
by $x \mapsto d^{-m_1}F_M(x)$.
\end{rem}

\begin{ex}\label{Example:standard banal displayed BK module}
The cocharacter
$\G_m \to \GL_{n, \O_E}$
defined by
$
t \mapsto \diag{(t^{m_1}, \dotsc, t^{m_n})}
$
is denoted by the same symbol $\mu$.
We set $M:=A^n$.
To an invertible matrix $X \in \GL_n(A)$,
we attach a Breuil--Kisin module
$
(M, F_X)
$
over $(A, I)$ where $F_X$ is the composition
\[
(\phi^*M)[1/d] \overset{\sim}{\to} M[1/d] \overset{X}{\to} M[1/d] \overset{\mu(d)}{\to} M[1/d].
\]
Here, the first isomorphism is induced by
\[
\phi^*(A^n) \overset{\sim}{\to} A^n, \quad 1 \otimes (a_1, \dotsc, a_n) \mapsto (\phi(a_1), \dotsc, \phi(a_n)).
\]
The Breuil--Kisin module
$
(M, F_X)
$
is displayed.
Moreover, it is banal and of type $\mu$.
Indeed, via the composition
$\phi^*M \overset{\sim}{\to} M \overset{X}{\to} A^n$,
the filtration $\{ \Fil^i(\phi^*M) \}_{i \in \Z}$ coincides with the filtration 
$\{ \Fil^i_\mu \}_{i \in \Z}$.
\end{ex}

\begin{lem}\label{Lemma:banal BK type mu standard form}
Let $M$ be a banal Breuil--Kisin module over $(A, I)$ of type $\mu$.
Then,
there exists an invertible matrix $X \in \GL_n(A)$ such that $M$
is isomorphic to
the Breuil--Kisin module $(A^n, F_X)$ constructed in Example \ref{Example:standard banal displayed BK module}.
\end{lem}

\begin{proof}
We fix a trivialization
$h \colon A^n \overset{\sim}{\to}  \phi^*M$
such that
the filtration 
$\{ \Fil^i_\mu \}_{i \in \Z}$
coincides with
the filtration
$\{ \Fil^i(\phi^*M) \}_{i \in \Z}$; see Remark \ref{Remark:standard filtration}.
Let $\beta \colon A^n \overset{\sim}{\to} M$ denote the composition
$
A^n \overset{\sim}{\to} \Fil^{m_1}_\mu \overset{\sim}{\to} \Fil^{m_1}(\phi^*M) \overset{\sim}{\to} M
$
where the first isomorphism
$A^n \overset{\sim}{\to} \Fil^{m_1}_\mu$ is given by $d^{m_1}\mu(d)^{-1}$, the second one is induced by $h$, and the third one is the isomorphism $F_{m_1}$ defined in Remark \ref{Remark:Frobenius isomorphism}.
Then, let $\alpha_d(h) \in \GL_n(A)$ be the invertible matrix corresponding to the following composition:
\[
A^n = \phi^*(A^n) \overset{\phi^*\beta}{\to} \phi^*M \overset{h^{-1}}{\to} A^n.
\]
By construction, $\beta \colon A^n \overset{\sim}{\to} M$ gives an isomorphism $(A^n, F_{\alpha_d(h)}) \overset{\sim}{\to} M$ of Breuil--Kisin modules.
\end{proof}

\subsection{Windows and displays}\label{Subsection:Windows and displays}

We shortly discuss the relation between
the notion of minuscule (resp.\ displayed) Breuil--Kisin module and the notion of window (resp.\ display).

An oriented and bounded prism $(A, d)$ gives rise to a quadruple
\[
(A, (d), \phi, \phi_1)
\]
where $\phi_1 \colon (d) \to A$ is defined by $dx \mapsto \phi(x)$.
This quadruple is a \textit{frame} in the sense of Zink and Lau; see for example \cite[Section 2]{Lau10} for the notion of frame.
Similarly, for an oriented and bounded $\O_E$-prism $(A, d)$, the same construction gives an \textit{$\O_E$-frame}
$
(A, (d), \phi, \phi_1)
$
in the sense of \cite{Ahsendorf}.

We recall the notion of window, adapted to our context.

\begin{defn}\label{Definition:window}
Let $(A, d)$ be an oriented and bounded $\O_E$-prism.
A \textit{window} over $(A, d)$ is a quadruple
\[
\underline{N}=(N, \Fil^1(N), \Phi, \Phi_1)
\]
where $N$ is a finite projective $A$-module,
$\Fil^1(N) \subset N$ is an $A$-submodule,
$\Phi \colon N \to N$ and
$\Phi_1 \colon \Fil^1(N) \to N$
are $\phi$-linear homomorphisms, such that the following conditions hold:
\begin{enumerate}
    \item We have $dN \subset \Fil^1(N)$, and $\Phi(x)=\Phi_1(dx)$ for every $x \in N$.
    \item The image $P^1 \subset N/dN$ of $\Fil^1(N)$ is a direct summand of $N/dN$.
    \item The linearization $1 \otimes \Phi_1 \colon \phi^*\Fil^1(N) \to N$ of $\Phi_1$ is an isomorphism.
\end{enumerate}
\end{defn}




\begin{prop}[cf.\ {\cite[Lemma 2.1.16]{Cais-Lau}, \cite[Proposition 4.26]{Anschutz-LeBras}}]\label{Proposition:windows and BK modules}
Let $(A, d)$ be an oriented and bounded $\O_E$-prism.
For a window
$(N, \Fil^1(N), \Phi, \Phi_1)$ over $(A, d)$,
the pair $(\Fil^1(N), F)$, where $F \colon \phi^*\Fil^1(N) \to \Fil^1(N)$ is defined by $F=d(1 \otimes \Phi_1)$, is a minuscule Breuil--Kisin module over the bounded $\O_E$-prism $(A, (d))$.
This construction gives an equivalence between the category of windows over $(A, d)$ and the category of minuscule Breuil--Kisin modules over $(A, (d))$.
\end{prop}

\begin{proof}
The proof is identical to that of \cite[Lemma 2.1.16]{Cais-Lau}.
\end{proof}


\begin{rem}[Displayed Breuil--Kisin modules and displays]\label{Remark:displayed BK module and display}
We assume that $\O_E=\Z_p$ for simplicity.
Let $(A, d)$ be an oriented and bounded prism.
A displayed Breuil--Kisin module over $(A, (d))$ gives rise to a \textit{display}
\[
(\phi^*M, \{ \Fil^i(\phi^*M) \}_{i \in \Z}, \{ \Phi_i \}_{i \in \Z})
\]
over the frame $(A, (d), \phi, \phi_1)$
in the sense of \cite{LangerZink}, where
$\Phi_i \colon \Fil^i(\phi^*M) \to \phi^*M$ are the $\phi$-linear homomorphisms defined by sending $x \in \Fil^i(\phi^*M)$ to $1 \otimes (d^{-i}F_M(x)) \in \phi^*M$.
One can prove that
this construction gives an equivalence between the category of displayed Breuil--Kisin module over $(A, (d))$ and the category of displays over the frame $(A, (d), \phi, \phi_1)$.
Since we will not need this fact in this paper, we omit the details.
\end{rem}


\section{Display group} \label{Section:display group}

Let $G$ be a smooth affine group scheme over $\O_E$.
Let $k$ be a perfect field containing $\F_q$
and
we set
$\O := W(k)\otimes_{W(\F_q)} \O_E$.
Let
\[
\mu \colon \G_m \to G_{\O}:=G \times_{\Spec \O_E} \Spec \O
\]
be a cocharacter defined over $\Spec \O$.
In this section, we introduce the display group $G_\mu(A, I)$
for an orientable and bounded $\O_E$-prism $(A, I)$ over $\O$.
The display group will be used in the definition of $G$-$\mu$-displays.

\subsection{Definition of the display group}\label{Subsection:Definition of the Display group}

Let $A$ be an $\O$-algebra with
an ideal $I \subset A$ which is generated by a nonzerodivisor $d \in A$.

\begin{defn}\label{Definition:display group}
We define
\[
G_\mu(A, I):=\{ \, g \in G(A) \, \vert \, \mu(d)g\mu(d)^{-1} \, \, \text{lies in} \, \, G(A) \subset G(A[1/I]) \, \}.
\]
The group $G_\mu(A, I)$ is called the \textit{display group}.
\end{defn}

We note that $G_\mu(A, I)$ does not depend on the choice of $d$.

\begin{rem}\label{Remark:definition of display group is simpler}
    The definition of the display group given here is a translation of the one given in \cite{Lau21} to our setting.
    If $G$ is reductive and $\mu$ is minuscule, such a group was also considered in \cite{Bultel-Pappas}.
\end{rem}


For the cocharacter
$
\mu \colon \G_m \to G_{\O},
$
we endow $G_\O$ with the action of $\G_m$ defined by
\begin{equation}\label{equation:action cocharacter adjoint}
    G_\O(R) \times \G_m(R) \to G_\O(R), \quad (g, t) \mapsto \mu(t)^{-1}g\mu(t)
\end{equation}
for every $\O$-algebra $R$.
We write $G=\Spec A'_G$ and $A_G:=A'_G \otimes_{\O_E} \O$, so that $G_\O=\Spec A_G$.
Then, we have an action of $\G_m$ on the $\O$-algebra $A_G$.
Let
$
A_G:= \bigoplus_{n\in\Z} A_{G, n}
$
be the weight decomposition with respect to the action of $\G_m$.
An element $t \in \G_m(R)=R^\times$ acts on $A_{G, n} \otimes_\O R$ by $r_n \mapsto t^{n}r_n$ for every $n \in \Z$.
(See for example \cite[Lemma A.8.8]{CGP} for the existence of the weight decomposition over a ring.)

\begin{rem}\label{Remark:formula action of cocharacter}
Let $R$ be an $\O$-algebra.
For any $t \in \G_m(R)$ and any $g \in G_\O(R)$ with corresponding homomorphism $g^* \colon A_G \to R$,
the homomorphism
\[
(\mu(t)^{-1}g\mu(t))^* \colon A_G \to R
\]
corresponding to $\mu(t)^{-1}g\mu(t) \in G_\O(R)$ 
sends an element $r_n \in A_{G, n}$ to $t^ng^*(r_n) \in R$.
\end{rem}

\begin{lem}\label{Lemma:Gmu iff condition}
Let $g \in G(A)$ be an element.
Then $g \in G_\mu(A, I)$ if and only if $g^*(r_n)$ belongs to $I^nA$ for every $n > 0$ and any $r_n \in A_{G, n}$.
\end{lem}

\begin{proof}
This follows from Remark \ref{Remark:formula action of cocharacter}.
\end{proof}

\begin{ex}\label{Example:display group GLn case}
Assume that $G=\GL_n$
and let
$\mu \colon \G_m \to \GL_{n, \O}$
be
the cocharacter associated with
a tuple
$(m_1, \dotsc, m_n)$ of integers with $m_1 \geq \cdots \geq m_n$; see Example \ref{Example:standard banal displayed BK module}.
Let $\{ \Fil^i_\mu \}_{i \in \Z}$ be the filtration of $M:=A^n$ defined as in Remark \ref{Remark:standard filtration}.
Then, we have
\[
\begin{split}
    (\GL_n)_\mu(A, I)&=\{ \, g \in \GL_n(A) \, \vert \, g(\Fil^{m_1}_\mu)=\Fil^{m_1}_\mu  \, \} \\
    &=\{ \, g \in \GL_n(A) \, \vert \, g(\Fil^i_\mu)=\Fil^i_\mu \, \, \text{for every $i \in \Z$} \, \}.
\end{split}
\]
Let $d \in I$ be a generator.
For the isomorphism 
$d^{-m_1}\mu(d) \colon \Fil^{m_1}_\mu \overset{\sim}{\to} M$,
and for an element $g \in (\GL_n)_\mu(A, I)$,
the following diagram commutes:
\[
\xymatrix{
\Fil^{m_1}_\mu \ar^-{g}[rr]  \ar[d]_-{\simeq} & & \Fil^{m_1}_\mu  \ar[d]^-{\simeq} \\
M \ar[rr]^-{\mu(d)g\mu(d)^{-1}} & & M.
}
\]
\end{ex}

\subsection{Properties of the display group}\label{Subsection:Properties of the display group}

For the cocharacter $\mu  \colon \G_m \to G_{\O}$,
we consider the closed subgroup schemes $P_\mu, U_{-\mu} \subset G_\O$ over $\O$ defined by, for every $\O$-algebra $R$,
\begin{align*}
    P_\mu(R)&=\{ \, g \in G(R) \, \vert \, \lim_{t \to 0} \mu(t)g\mu(t)^{-1} \, \text{exists} \, \},\\
    U_{-\mu}(R)&=\{ \, g \in G(R) \, \vert \, \lim_{t \to 0} \mu(t)^{-1}g\mu(t)=1 \, \}.
\end{align*}
We see that $P_\mu$ and $U_{-\mu}$ are stable under the action of $\G_m$ on $G_\O$ given by (\ref{equation:action cocharacter adjoint}).
The group schemes $P_\mu$ and $U_{-\mu}$ are smooth over $\O$.
Moreover, the multiplication map
\[
U_{-\mu} \times_{\Spec \O} P_\mu \to G_\O
\]
is an open immersion.
See \cite[Section 2.1]{CGP}, especially \cite[Proposition 2.1.8]{CGP}, for details.

\begin{rem}\label{Remark:change of notation from Lau}
Let us remark that we have employed slightly different notation than in \cite{Lau21}.
For example, in \textit{loc.cit.}, the subgroup $P_\mu$ (resp.\ $P_{-\mu}$) is denoted by $P^{-}$ (resp.\ $P^{+}$).
\end{rem}


\begin{lem}\label{Lemma:Pmu structure}
\
\begin{enumerate}
    \item Let $R$ be an $\O$-algebra
    and $g \in G_\O(R)$ an element.
    Then, we have $g \in P_\mu(R)$ if and only if $g^*(r_n)=0$ for every $n > 0$ and every $r_n \in A_{G, n}$.
    \item We have
    $
    P_\mu(A) \subset G_\mu(A, I),
    $
    and the image of $G_\mu(A, I)$ in $G(A/I)$
    under the projection
    $G(A) \to G(A/I)$ is contained in $P_\mu(A/I)$.
    Moreover
    $
    \mu(d)P_\mu(A)\mu(d)^{-1}$
    is contained in $P_\mu(A)$.
\end{enumerate}
\end{lem}

\begin{proof}
Remark \ref{Remark:formula action of cocharacter} immediately implies (1).
The second assertion (2) follows from (1) and Lemma \ref{Lemma:Gmu iff condition}.
\end{proof}


\begin{defn}[{\cite[Definition 6.3.1]{Lau21}}]\label{Definition:1-bounded}
The action of $\G_m$ on $G_\O$ given in (\ref{equation:action cocharacter adjoint}) induces an action of $\G_m$ on
the Lie algebra $\Lie(G_\O)$.
Let
\[
\Lie(G_\O)=\bigoplus_{n \in \Z} \mathfrak{g}_n
\]
be the weight decomposition with respect to the action of $\G_m$.
We say that the cocharacter
$\mu \colon \G_m \to G_{\O}$
is \textit{1-bounded} if $\mathfrak{g}_n=0$ for $n \geq 2$.
\end{defn}

In general,
the Lie algebra $\Lie(U_{-\mu})$ of $U_{-\mu}$
coincides with
$\bigoplus_{n \geq 1} \mathfrak{g}_n$.
(We also note that
$\Lie(P_{\mu})= \bigoplus_{n \leq  0} \mathfrak{g}_n$.)
Thus, we see that $\mu$ is 1-bounded if and only if
the equality
$\Lie(U_{-\mu})=\mathfrak{g}_1$
holds in $\Lie(G_\O)$.

\begin{rem}\label{Remark:minuscule cocharacter}
    If $G$ is a reductive group scheme over $\O_E$, then $\mu$ is 1-bounded if and only if $\mu$ is minuscule, that is, we have $\Lie(G_\O)=\mathfrak{g}_{-1} \oplus \mathfrak{g}_{0} \oplus \mathfrak{g}_{1}$.
\end{rem}

\begin{ex}\label{Example:GLn minuscule case}
    Assume that $G=\GL_n$ and let the notation be as in Example \ref{Example:display group GLn case}.
    Then the cocharacter $\mu$ is 1-bounded if and only if $m_1 - m_n \leq 1$.
\end{ex}

For a free $\O$-module $M$ of finite rank,
we let $V(M)$ denote the group scheme over $\O$ defined by $R \mapsto M \otimes_\O R$ for every $\O$-algebra $R$.

\begin{lem}\label{Lemma:Umu structure}
There exists a $\G_m$-equivariant isomorphism
    \[
    \log \colon U_{-\mu} \overset{\sim}{\to} V(\Lie(U_{-\mu}))
    \]
    of schemes over $\O$ which induces the identity on the Lie algebras.
    If $\mu$ is 1-bounded, then the isomorphism $\log$ is unique, and it is an isomorphism of group schemes over $\O$.
\end{lem}

\begin{proof}
The same arguments as in the proofs of \cite[Lemma 6.1.1, Lemma 6.3.2]{Lau21} work here.
\begin{comment}
We recall the argument for the convenience of the reader.

Let us write $U_{-\mu}=\Spec B$.
The action of $\G_m$ induces the weight decomposition
$B=\bigoplus_{n \in \Z} B_n$.
Using Remark \ref{Remark:formula action of cocharacter}, we see that $B_0=\O$ and $B_n=0$ for any $n <0$.
Let $I:=\bigoplus_{n >0} B_n$, so that $B=B_0 \oplus I$ and the kernel of the counit $\epsilon \colon B \to \O$ is equal to $I$.
We have $\Lie(U_{-\mu})=(I/I^2)^{\vee}:=\Hom_\O(I/I^2, \O)$
and
\[
V(\Lie(U_{-\mu}))=\Spec (\Sym{I/I^2})
\]
where $\Sym{I/I^2}$ is the symmetric algebra of $I/I^2$.
As explained in \cite[Lemma 6.1.1]{Lau21}, a $\G_m$-equivariant morphism
$U_{-\mu} \to V(\Lie(U_{-\mu}))$
of schemes over $\O$ corresponds to a homomorphism
$I/I^2 \to I$
of graded $\O$-modules.
Such a morphism $U_{-\mu} \to V(\Lie(U_{-\mu}))$ induces a homomorphism $\Lie(U_{-\mu}) \to \Lie(U_{-\mu})$ of $\O$-modules, which is the identity if and only if the corresponding homomorphism
$I/I^2 \to I$
is a section of the projection $I \to I/I^2$.
Since $I/I^2$ is a finite free $\O$-module, it is clear that
such a section exists.


Since $\mu$ is 1-bounded, it follows that
$I/I^2$ is concentrated degree $1$, which in turn implies that the projection
$I_1=B_1 \to I/I^2$
is an isomorphism.
This shows that there exists a unique $\G_m$-equivariant morphism
$\log \colon U_{-\mu} \to V(\Lie(U_{-\mu}))$
which induces the identity on the Lie algebras.
It remains to prove that $\log$
is a homomorphism of group schemes.
Arguing as in the previous paragraph, we are reduced to showing that
the restriction of
the comultiplication $\Delta \colon B \to B \otimes_\O B$
to $B_1$
agrees with the homomorphism
\[
B_1 \to (B_1 \otimes_\O B_0) \oplus (B_0 \otimes_\O B_1), \quad b_1 \mapsto b_1 \otimes 1 + 1 \otimes b_1.
\]
This immediately follows from the equalities $(\id \otimes \epsilon )\circ \Delta=(\epsilon \otimes \id )\circ \Delta=\id$.
\end{comment}
\end{proof}

\begin{rem}\label{Remark:Umu identify}
\ 
\begin{enumerate}
    \item Let
    $
    \log \colon U_{-\mu} \overset{\sim}{\to} V(\Lie(U_{-\mu}))
    $
    be an isomorphism as in Lemma \ref{Lemma:Umu structure}.
    This isomorphism induces a bijection
\[
U_{-\mu}(A) \cap G_\mu(A, I) \overset{\sim}{\to} \bigoplus_{n \geq 1} I^n(\mathfrak{g}_n \otimes_\O A).
\]
    \item If $\mu$ is 1-bounded,
then
we identify $U_{-\mu}$ with $V(\Lie(U_{-\mu}))$
by the unique isomorphism $\log$.
In particular, we view
$\Lie(U_{-\mu}) \otimes_\O A$
as a subgroup of $G(A)$.
We then obtain
\[
I(\Lie(U_{-\mu}) \otimes_\O A)=(\Lie(U_{-\mu}) \otimes_\O A) \cap G_\mu(A, I).
\]
Moreover, the following diagram commutes:
\[
\xymatrix{
I(\Lie(U_{-\mu}) \otimes_\O A)  \ar@{^{(}->}[r]^-{} \ar_-{dv \mapsto v}[d] & G_\mu(A, I) \ar[d]^-{g \mapsto \mu(d)g\mu(d)^{-1}}  \\
\Lie(U_{-\mu}) \otimes_\O A
 \ar@{^{(}->}[r]^-{}  & G(A).
}
\]
\end{enumerate}
\end{rem}

\begin{prop}[{cf.\ \cite[Lemma 6.2.2]{Lau21}}]\label{Proposition:decomposition of display group}
Assume that $A$ is $I$-adically complete.
Then, the multiplication map
\begin{equation}\label{equation:multiplication map Gmu}
    (U_{-\mu}(A) \cap G_\mu(A, I)) \times P_\mu(A) \to G_\mu(A, I)
\end{equation}
is bijiective.
\end{prop}

\begin{proof}
We first note that since we have $P_\mu(A) \subset G_\mu(A, I)$ by Lemma \ref{Lemma:Pmu structure}, the multiplication map (\ref{equation:multiplication map Gmu}) is well-defined.
Since
the map
$
U_{-\mu} \times_{\Spec \O} P_\mu \to G_\O
$
is an open immersion, the map (\ref{equation:multiplication map Gmu}) is injective.

We shall show that the map (\ref{equation:multiplication map Gmu}) is surjective.
Let $g \in G_\mu(A, I)$ be an element.
By Lemma \ref{Lemma:Pmu structure}, the image of $g$ in $G(A/I)$ is contained in $P_\mu(A/I)$.
Since $P_\mu$ is smooth and $A$ is $I$-adically complete,
there exists an element $t \in P_\mu(A)$ whose image in $P_\mu(A/I)$ coincides with the image of $g$.
In particular, the restriction of 
the morphism $gt^{-1} \colon \Spec A \to G_\O$
to $\Spec A/I$
factors through the open subscheme
$U_{-\mu} \times_{\Spec \O} P_\mu$.
Since $I \subset \mathrm{rad}(A)$,
it then follows that
the morphism $gt^{-1} \colon \Spec A \to G_\O$
itself
factors through
$U_{-\mu} \times_{\Spec \O} P_\mu$.
In other words, there are elements $u \in U_{-\mu}(A)$ and $t' \in P_\mu(A)$ such that
$g=ut't$.
We note that $u \in G_\mu(A, I)$.
In conclusion, we have shown that $g$ is contained in the image of the map (\ref{equation:multiplication map Gmu}).
\end{proof}


\begin{prop}\label{Proposition:BB isomorphism}
Assume that $\mu \colon \G_m \to G_{\O}$ is 1-bounded.
Assume further that $A$ is $I$-adically complete.
Then, the multiplication map
\begin{equation}\label{equation:multiplication map Gmu 1-bounded case}
    I(\Lie(U_{-\mu}) \otimes_\O A) \times P_\mu(A) \to G_\mu(A, I)
\end{equation}
is bijiective.
Moreover $G_\mu(A, I)$ coincides with the inverse image of $P_\mu(A/I)$ in $G(A)$ under the projection $G(A) \to G(A/I)$, and we have the following bijection:
\begin{equation}\label{equation:BB isomorphism}
    G(A)/G_\mu(A, I) \overset{\sim}{\to} G(A/I)/P_\mu(A/I).
\end{equation}
\end{prop}

\begin{proof}
It follows from Remark \ref{Remark:Umu identify} and Proposition \ref{Proposition:decomposition of display group} that
the multiplication map (\ref{equation:multiplication map Gmu 1-bounded case}) is bijective.
Let $G'_\mu \subset G(A)$ be the inverse image of $P_\mu(A/I)$.
We have
$G_\mu(A, I) \subset G'_\mu$.
By the same argument as in the proof of Proposition \ref{Proposition:decomposition of display group},
one can show that
$
G'_\mu \subset I(\Lie(U_{-\mu}) \otimes_\O A) \times P_\mu(A).
$
Thus, we obtain $G_\mu(A, I) = G'_\mu$.

It remains to prove that the map (\ref{equation:BB isomorphism}) is bijective.
Since $G$ is smooth and $A$ is $I$-adically complete,
the projection $G(A) \to G(A/I)$ is surjective, which in turn implies the surjectivity of (\ref{equation:BB isomorphism}).
The injectivity follows from the equality $G_\mu(A, I)=G'_\mu$.
\end{proof}

For an integer $m \geq 0$,
let
$
G^{\geq m}(A)
$
denote the kernel of the homomorphism $G(A) \to G(A/I^m)$.
We set
\[
G^{\geq m}_\mu(A, I):=G_\mu(A, I) \cap G^{\geq m}(A).
\]
We record a structural result about the quotient $G^{\geq m}_\mu(A, I)/G^{\geq m+1}_\mu(A, I)$.

\begin{lem}\label{Lemma:congruent subgroup of Gmu}
Assume that $A$ is $I$-adically complete.
Then we have the following isomorphisms of groups:
\[
G^{\geq m}(A)/G^{\geq m+1}(A)
\simeq
\begin{cases}
G(A/I) \quad &(m=0) \\ 
\Lie(G_\O) \otimes_\O I^m/I^{m+1} \quad &(m \geq 1),
\end{cases}
\]
\[
G^{\geq m}_\mu(A, I)/G^{\geq m+1}_\mu(A, I)
\simeq
\begin{cases}
P_\mu(A/I) \quad &(m=0) \\ 
(\bigoplus_{n \leq m} \mathfrak{g}_n)
\otimes_\O I^m/I^{m+1} \quad &(m \geq 1).
\end{cases}
\]
\end{lem}

\begin{proof}
Since $A$ is $I$-adically complete and $G$ is smooth, the homomorphism
$G(A) \to G(A/I^m)$ is surjective.
It follows that
$
G^{\geq m}(A)/G^{\geq m+1}(A)
$
is isomorphic to the kernel of $G(A/I^{m+1}) \to G(A/I^m)$.
From this, we can deduce the desired description of $G^{\geq m}(A)/G^{\geq m+1}(A)$.

Since $P_\mu(A) \to P_\mu(A/I)$ is surjective (as $A$ is $I$-adically complete and $P_\mu$ is smooth), it follows from Lemma \ref{Lemma:Pmu structure} that
$G_\mu(A, I)/G^{\geq 1}_\mu(A, I) \simeq P_\mu(A/I)$.
For any $m \geq 1$, we have
\[
G^{\geq m}_\mu(A, I)/G^{\geq m+1}_\mu(A,I) \subset G^{\geq m}(A)/G^{\geq m+1}(A) \simeq \Lie(G_\O) \otimes_\O I^m/I^{m+1}.
\]
Using an isomorphism
$\log \colon U_{-\mu} \overset{\sim}{\to} V(\Lie(U_{-\mu}))$
as in Lemma \ref{Lemma:Umu structure} and the bijection (\ref{equation:multiplication map Gmu}), one can show that
$G^{\geq m}_\mu(A, I)/G^{\geq m+1}_\mu(A, I)$ coincides with 
$(\bigoplus_{n \leq m} \mathfrak{g}_n)
\otimes_\O I^m/I^{m+1}$.
\end{proof}

\subsection{Display groups on prismatic sites}\label{Subsection:Display groups on prismatic sites}

Let
$(A, I)$
be an orientable and bounded $\O_E$-prism over $\O$.
In this subsection, we define the display group sheaf 
$G_{\mu, A, I}$
on the site
$
(A, I)^{\op}_\et
$
and discuss some basic results on $G_{\mu, A, I}$-torsors.

For a $(\pi, I)$-completely \'etale $A$-algebra $B \in (A, I)_\et$,
the pair $(B, IB)$ is naturally an orientable and bounded $\O_E$-prism
by Lemma \ref{Lemma:etale morphism prism}.

\begin{defn}\label{Definition:display group sheaf}
We define the following two functors:
\begin{align*}
    G_{\Prism, A} &\colon (A, I)_\et \to \mathrm{Set}, \quad B \mapsto G(B), \\
    G_{\mu, A, I} &\colon (A, I)_\et \to \mathrm{Set}, \quad B \mapsto G_\mu(B,IB).
\end{align*}
\end{defn}

Since $G$ is affine, the two functors $G_{\Prism, A}$ and $G_{\mu, A, I}$ are group sheaves with respect to the $(\pi, I)$-completely \'etale topology by Remark \ref{Remark:structure sheaf}.

We begin with a comparison result between torsors over $\Spec A$ (or $\Spec A/I$) with respect to the usual \'etale topology, and torsors on the site $(A, I)^{\op}_\et$.
To an affine scheme $X$ over $\O$ (or $A$),
we associate
a sheaf
\[
X_{\Prism, A} \colon (A, I)_\et \to \mathrm{Set},\quad B \mapsto X(B).
\]
Similarly, to an affine scheme $X$ over $\O$ (or $A/I$),
we associate
a sheaf
\[
X_{\overline{\Prism}, A} \colon (A, I)_\et \to \mathrm{Set}, \quad B \mapsto X(B/IB).
\]

\begin{prop}\label{Proposition:equivalences of pi completely etale torsors}
Let $H$ be a smooth affine group scheme over $\O$.
\begin{enumerate}
    \item
    For an $H_{A/I}$-torsor $\mathcal{P}$ over $\Spec A/I$ with respect to the \'etale topology, which is an affine scheme over $A/I$, the sheaf $\mathcal{P}_{\overline{\Prism}, A}$
is an $H_{\overline{\Prism}, A}$-torsor with respect to the $(\pi, I)$-completely \'etale topology.
This construction
\[
\mathcal{P} \mapsto \mathcal{P}_{\overline{\Prism}, A}
\]
gives an equivalence from the groupoid of $H_{A/I}$-torsors over $\Spec A/I$ to
the groupoid of $H_{\overline{\Prism}, A}$-torsors.
\item The construction
\[
\mathcal{P} \mapsto \mathcal{P}_{\Prism, A}
\]
gives an equivalence from the groupoid of $H_A$-torsors over $\Spec A$ to
the groupoid of $H_{\Prism, A}$-torsors.
\end{enumerate}
\end{prop}

\begin{proof}
(1) It is clear that $\mathcal{P}_{\overline{\Prism}, A}$ is an $H_{\overline{\Prism}, A}$-torsor.
To show that $\mathcal{P} \mapsto \mathcal{P}_{\overline{\Prism}, A}$ is an equivalence,
it suffices to prove that
the fibered category over $(A, I)^{\op}_\et$
which associates to a $(\pi, I)$-completely \'etale $A$-algebra $B$ to the groupoid of $H_{B/IB}$-torsors over $\Spec B/IB$
is a stack with respect to the $(\pi, I)$-completely \'etale topology.

It is known that, for any $\O$-algebra $R$,
the groupoid of $H_R$-torsors over $\Spec R$
is equivalent to the groupoid of
exact tensor functors
$
\Rep_\O(H) \to \Vect(R),
$
where $\Rep_\O(H)$ is the category of algebraic representations of $H$ on free $\O$-modules of finite rank,
and $\Vect(R)$ is the category of finite projective $R$-modules; see \cite[Theorem 19.5.1]{Scholze-Weinstein} and \cite[Theorem 1.2]{Broshi}.
(Although this result is stated only for the case where $\O=\Z_p$ in \cite[Theorem 19.5.1]{Scholze-Weinstein}, the proof also works for general $\O$.)
Using this Tannakian perspective, the desired claim follows from Proposition \ref{Proposition:flat descent for finite projective modules} and the following fact:
For a $\pi$-completely \'etale covering $C \to C'$, a complex
\[
0 \to M_1 \to M_2 \to M_3 \to 0
\]
of finite projective $C$-modules
is exact if and only if the base change
\[
0 \to M_1 \otimes_{C} C'  \to M_2 \otimes_{C} C' \to M_3 \otimes_{C} C' \to 0
\]
is exact.
For the proof of this fact, see \cite[Lemma 2.3.6]{PappasRapoport21} for example.

(2) This can be proved in the same way as (1).
\end{proof}

We remark that Proposition \ref{Proposition:equivalences of pi completely etale torsors} can not be applied directly to $G_{\mu, A, I}$-torsors.
However,
it is still useful for analysing $G_{\mu, A, I}$-torsors in several places below, since we have the following lemma.
(For the notation used below, see Lemma \ref{Lemma:congruent subgroup of Gmu}.)

\begin{lem}\label{Lemma:Gmu torsor successive quotient}
\ 
\begin{enumerate}
    \item For an integer $m \geq 0$, the functor
    \[
    G^{=m}_{\mu, A, I} \colon (A, I)_\et \to \mathrm{Set}, \quad
    B \mapsto G^{\geq m}_\mu(B, IB)/G^{\geq m+1}_\mu(B, IB)
    \]
    forms a group sheaf, and it is isomorphic to
    $(P_\mu)_{\overline{\Prism}, A}$
    (resp.\ $V(\bigoplus_{n \leq m} \mathfrak{g}_n)_{\overline{\Prism}, A}$)
    if $m=0$ 
    (resp.\ $m \geq 1$).
    Moreover, the functor
    \[
    G^{<m}_{\mu, A, I} \colon (A, I)_\et \to \mathrm{Set}, \quad
    B \mapsto G_\mu(B, IB)/G^{\geq m}_\mu(B, IB)
    \]
    forms a group sheaf.
    \item 
    For a $G_{\mu, A, I}$-torsor $\mathcal{Q}$, we write
    $\mathcal{Q}^{<m}$
    for the pushout of $\mathcal{Q}$ along $G_{\mu, A, I} \to G^{<m}_{\mu, A, I}$.
    Then we have
    $
    \mathcal{Q} \overset{\sim}{\to} \varprojlim_m \mathcal{Q}^{<m}.
    $
\end{enumerate}
\end{lem}

\begin{proof}
(1) The statement about $G^{=m}_{\mu, A, I}$ follows from Lemma \ref{Lemma:congruent subgroup of Gmu}.
Using the exact sequence
\[
1 \mapsto G^{=m}_{\mu, A, I}(B) \to G^{<m+1}_{\mu, A, I}(B) \to G^{<m}_{\mu, A, I}(B) \to 1,
\]
the statement about $G^{<m}_{\mu, A, I}$ then follows by induction on $m$.

(2) We shall prove that
$
G_{\mu, A, I} \overset{\sim}{\to} \varprojlim_m G^{<m}_{\mu, A, I}.
$
This claim implies that $\varprojlim \mathcal{Q}^{<m}$ is a $G_{\mu, A, I}$-torsor, and hence
$
\mathcal{Q} \overset{\sim}{\to} \varprojlim \mathcal{Q}^{<m}.
$
By Proposition \ref{Proposition:decomposition of display group}, the multiplication map
$(U_{-\mu}(A) \cap G_\mu(A, I)) \times P_\mu(A) \to G_\mu(A, I)$
is bijective.
Note that
$G_\mu(A, I)/G^{\geq m}_\mu(A, I)$
can be identified with the image of $G_\mu(A, I)$ in  $G(A/I^m)$.
Let $U^{<m}$ be the image of $U_{-\mu}(A) \cap G_\mu(A, I)$ in $U_{-\mu}(A/I^m)$.
Then, the multiplication map induces a bijection
\[
U^{<m} \times P_\mu(A/I^m) \overset{\sim}{\to} G_\mu(A, I)/G^{\geq m}_\mu(A, I).
\]
We have
$P_\mu(A) \overset{\sim}{\to} \varprojlim P_\mu(A/I^m)$.
Moreover, using Lemma \ref{Lemma:Umu structure},
one can check that
$U_{-\mu}(A) \cap G_\mu(A, I) \overset{\sim}{\to} \varprojlim U^{<m}$.
Thus, we obtain
$G_{\mu}(A, I) \overset{\sim}{\to} \varprojlim G_\mu(A, I)/G^{\geq m}_\mu(A, I)$.
The same assertion holds for any $(\pi, I)$-completely \'etale $A$-algebra $B$, and hence
$G_{\mu, A, I} \to \varprojlim  G^{<m}_{\mu, A, I}$ is an isomorphism.
\end{proof}

We note that $G^{<1}_{\mu, A, I}=G^{=0}_{\mu, A, I}=(P_\mu)_{\overline{\Prism}, A}$ by Lemma \ref{Lemma:Gmu torsor successive quotient} (1).

\begin{cor}\label{Corollary:trivial Gmu torsor}
A $G_{\mu, A, I}$-torsor $\mathcal{Q}$ (with respect to the $(\pi, I)$-completely \'etale topology) is trivial if the pushout of $\mathcal{Q}$ along $G_{\mu, A, I} \to (P_\mu)_{\overline{\Prism}, A}$ is trivial as a $(P_\mu)_{\overline{\Prism}, A}$-torsor.
\end{cor}

\begin{proof}
By Lemma \ref{Lemma:Gmu torsor successive quotient} (2), it suffices to show that
$\mathcal{Q}^{<m}$
is trivial as a $G^{<m}_{\mu, A, I}$-torsor
for any $m$.
We proceed by induction on $m$.
The assertion is true for $m=1$ by our assumption.
We assume that $\mathcal{Q}^{<m}$ is trivial for an integer $m \geq 1$, so that there exists an element $x \in \mathcal{Q}^{<m}(A)$.
The fiber of the morphism $\mathcal{Q}^{<m+1} \to \mathcal{Q}^{<m}$ at $x$ is a $G^{=m}_{\mu, A, I}$-torsor.
Lemma \ref{Lemma:Gmu torsor successive quotient} (1) shows that $G^{=m}_{\mu, A, I} \simeq V(\bigoplus_{n \leq m} \mathfrak{g}_n)_{\overline{\Prism}, A}$.
By Proposition \ref{Proposition:equivalences of pi completely etale torsors},
the fiber arises from a $V(\bigoplus_{n \leq m} \mathfrak{g}_n)_{A/I}$-torsor over $\Spec A/I$, which is trivial since $\Spec A/I$ is affine.
This implies that
the $G^{<m+1}_{\mu, A, I}$-torsor
$\mathcal{Q}^{<m+1}$
is trivial, as desired.
\end{proof}



\section{Prismatic $G$-$\mu$-display} \label{Section:prismatic G display}

In this section,
we come to the heart of this paper, namely prismatic $G$-$\mu$-displays.
We first introduce banal $G$-$\mu$-displays in Section \ref{Subsection:Banal G-display} as a warm-up,
and then define $G$-$\mu$-displays in Section \ref{Subsection:prismatic G-display}.
In Sections \ref{Subsection:flat descent}-\ref{Subsection:G displays for perfectoid rings}, we establish some fundamental properties of $G$-$\mu$-displays.
We retain the notation of Section \ref{Section:display group}.

\subsection{Banal $G$-$\mu$-display} \label{Subsection:Banal G-display}
To an orientable and bounded $\O_E$-prism $(A, I)$ over $\O$, we attach the display group $G_\mu(A, I)$ as in Definition \ref{Definition:display group}.
We note that since $G$ is defined over $\O_E$, the Frobenius $\phi$ of $A$ induces a homomorphism $\phi \colon G(A) \to G(A)$.
For each generator $d \in I$, we can then define the following homomorphism:
\begin{equation}\label{equation:sigma map of sets}
    \sigma_{\mu, d} \colon G_\mu(A, I) \to G(A), \quad g \mapsto \phi(\mu(d)g\mu(d)^{-1}).
\end{equation}
We endow $G(A)$ with the following action of $G_\mu(A, I)$:
\begin{equation}\label{equation:action of Gmu set}
    G(A) \times G_\mu(A, I) \to G(A), \quad (X, g) \mapsto g^{-1}X\sigma_{\mu, d}(g).
\end{equation}

For a group $H$, a set with an action of $H$ is simply called an $H$-set.
We view $H$ as an $H$-set by right multiplication.
An \textit{$H$-torsor} is an $H$-set which is isomorphic to $H$ as an $H$-set.

With the action (\ref{equation:action of Gmu set}), we view $G(A)$ as a $G_\mu(A, I)$-set and it is denoted by $G(A)_d$.
For another generator $d' \in I$, we have $d=ud'$ for an element $u \in A^\times$.
The map $G(A)_d \to G(A)_{d'}$ defined by $X \mapsto X\phi(\mu(u))$ is $G_\mu(A, I)$-equivariant.
Using this, we can define the following $G_\mu(A, I)$-set
\[
G(A)_I := \varprojlim_{d} G(A)_d
\]
where $d$ runs over the set of elements $d \in I$ which generates $I$.
The projection map $G(A)_I \to G(A)_d$ is an isomorphism of $G_\mu(A, I)$-sets.
For an element $X \in G(A)_I$, let
\[
X_d \in G(A)_d
\]
denote the image of $X$.


\begin{defn}[{Banal $G$-$\mu$-display}]\label{Definition:Banal G display}
Let $(A, I)$ be an orientable and bounded $\O_E$-prism over $\O$.
\begin{enumerate}
    \item A \textit{banal $G$-$\mu$-display} over $(A, I)$ is a pair
    \[
    (\mathcal{Q}, \alpha_\mathcal{Q})
    \]
    where $\mathcal{Q}$ is a $G_\mu(A, I)$-torsor and
    $\alpha_\mathcal{Q} \colon \mathcal{Q} \to G(A)_I$ is a $G_\mu(A, I)$-equivariant map of sets.
    When there is no possibility of confusion, we simply write $\mathcal{Q}$ instead of $(\mathcal{Q}, \alpha_\mathcal{Q})$.
    \item An isomorphism $g \colon (\mathcal{Q}, \alpha_\mathcal{Q}) \overset{\sim}{\to} (\mathcal{R}, \alpha_\mathcal{R})$ of banal $G$-$\mu$-displays over 
    $(A, I)$ is an isomorphism $g \colon \mathcal{Q} \overset{\sim}{\to} \mathcal{R}$ of $G_\mu(A, I)$-torsors such that $\alpha _{\mathcal{R}}\circ g=\alpha_\mathcal{Q}$.
\end{enumerate}
\end{defn}


We have the following alternative description of banal $G$-$\mu$-displays, which we will use frequently in the sequel.

\begin{rem}\label{Remark:quotient groupoid banal G-displays}
Let
\[
[G(A)_I/G_\mu(A, I)]
\]
denote the groupoid whose objects are the elements $X \in G(A)_I$ and whose morphisms are defined by
$
\Hom(X, X')=\{\, g \in G_\mu(A, I) \, \vert \, X'\cdot g=X  \, \}.
$
Here $(-)\cdot g$ denotes the action of $g \in G_\mu(A, I)$.
To each
$X \in G(A)_I$,
we attach a banal $G$-$\mu$-display
\[
\mathcal{Q}_X:=(G_\mu(A, I), \alpha_X)
\]
over 
$(A, I)$
where
$\alpha_X \colon G_\mu(A, I) \to G(A)_I$
is defined by $1 \mapsto X$.
This construction gives an equivalence from
$[G(A)_I/G_\mu(A, I)]$ to the groupoid of banal $G$-$\mu$-displays over 
$(A, I)$.
\end{rem}



\begin{rem}\label{Remark:Bartling banal}
If $\O_E=\Z_p$ and $\mu$ is 1-bounded, the notion of banal $G$-$\mu$-display for oriented and bounded prisms has already appeared in the paper of Bartling \cite[Definition 10]{Bartling}, following the ideas presented in \cite{Bultel-Pappas} and \cite{Lau21}.
See also Remark \ref{Remark:compare with Lau, and Bartling perfectoid case}.
\end{rem}

A typical example is the following:

\begin{ex}\label{Example:banal GLn displays}
Assume that $G=\GL_n$.
Let
$\mu=(m_1, \dotsc, m_n)$
be a tuple of integers with $m_1 \geq \cdots \geq m_n$ and
we denote by the same notation
$\mu \colon \G_m \to \GL_{n, \O}$
the associated cocharacter (see Example \ref{Example:standard banal displayed BK module}).
We fix a generator $d \in I$.
To each $X \in \GL_n(A)_I$,
we attach
the banal Breuil--Kisin module
\[
(A^n, F_{X_d})
\]
over $(A, I)$ of type $\mu$ given in Example \ref{Example:standard banal displayed BK module} (with respect to $d$).
Let $X, X' \in \GL_n(A)_I$.
For an element $g \in (\GL_n)_\mu(A, I)$ such that $X'\cdot g=X$,
we see that the homomorphism
$\mu(d)g\mu(d)^{-1} \colon A^n \overset{\sim}{\to} A^n$ induces an isomorphism
$(A^n, F_{X_d}) \overset{\sim}{\to} (A^n, F_{X'_{d}})$ of Breuil--Kisin modules.
In this way, we have a functor
\[
[\GL_n(A)_I/(\GL_n)_\mu(A, I)] \to \mathrm{BK}_\mu(A, I)^{\simeq}_{\mathrm{banal}},
\]
where $\mathrm{BK}_\mu(A, I)^{\simeq}_{\mathrm{banal}}$ denotes the groupoid of banal Breuil--Kisin modules over $(A, I)$ of type $\mu$.
We claim that this functor is an equivalence.
Indeed, the essential surjectivity follows from Lemma \ref{Lemma:banal BK type mu standard form}.
It is straightforward to show that this functor is fully faithful.
\end{ex}



Next, we discuss the notion of base change for banal $G$-$\mu$-displays.
Let us begin with a preliminary remark.

\begin{rem}[Pushout]\label{Remark:push out}
Let $f \colon H \to H'$ be a homomorphism of groups and $Q$ an $H$-set.
We can attach to $Q$ an $H'$-set $Q^f$
and a map $Q \to Q^f$ of $H$-sets (where we regard $Q^f$ as an $H$-set via $f \colon H \to H'$) with the following universal property: For any $H'$-set $Q'$ and any map $Q \to Q'$ of $H$-sets,
the map $Q \to Q'$ factors through a unique map $Q^f \to Q'$ of $H'$-sets.
Explicitly, we can define $Q^f$ as the contracted product
\[
Q^f= (Q \times H')/H,
\]
where the action of an $h \in H$ on $Q \times H'$ is defined by $(x, h') \mapsto (xh, f(h)^{-1}h')$.
The $H'$-set $Q^f$ is called the pushout of $Q$ along $f \colon H \to H'$.
We note that if $Q$ is an $H$-torsor, then $Q^f$ is an $H'$-torsor.

Similarly, for a homomorphism
$f \colon H \to H'$ of group sheaves on a site and
a sheaf $Q$ with an action of $H$, we can form the pushout $Q^f$ with the same properties as above.
\end{rem}


Let $f \colon (A, I) \to (A', I')$ be a map of orientable and bounded $\O_E$-prisms over $\O$.
The image of $G_\mu(A, I)$ under the usual homomorphism $f \colon G(A) \to G(A')$
is contained in $G_\mu(A', I')$; the induced homomorphism $G_\mu(A, I) \to G_\mu(A', I')$ is again denoted by $f$.
Let $d \in I$ and $d' \in I'$ be generators and let $u \in A'^\times$ be the unique element satisfying $f(d)=ud'$.
Then we have the following map of $G_\mu(A, I)$-sets:
\[
f_u \colon G(A)_d \to G(A')_{d'}, \quad X \mapsto f(X)\phi(\mu(u)).
\]
Moreover, we define the following $G_\mu(A, I)$-equivariant map
\begin{equation}\label{equation:the map f_I}
    G(A)_I \simeq G(A)_d \overset{f_u}{\to} G(A')_{d'} \simeq G(A')_{I'}.
\end{equation}
This map is independent of the choices of $d$ and $d'$, which we also denote by $f$.

\begin{defn}\label{Definition:base change for banal}
Let $f \colon (A, I) \to (A', I')$ be a map of orientable and bounded $\O_E$-prisms over $\O$.
For a banal $G$-$\mu$-display $(\mathcal{Q}, \alpha_{\mathcal{Q}})$ over $(A, I)$,
let
$\mathcal{Q}^f$ denote the pushout of $\mathcal{Q}$ along $f \colon G_\mu(A, I) \to G_\mu(A', I')$.
By the universal property of $\mathcal{Q}^f$, there exists a unique $G_\mu(A', I')$-equivariant map $\alpha^f_{\mathcal{Q}} \colon \mathcal{Q}^f \to G(A')_{I'}$
making the following diagram commute:
\[
\xymatrix{
\mathcal{Q} \ar^-{\alpha_\mathcal{Q}}[r]  \ar[d]_-{} & G(A)_I  \ar[d]^-{f} \\
\mathcal{Q}^f \ar[r]^-{\alpha^f_{\mathcal{Q}}} & G(A')_{I'}.
}
\]
The base change of $(\mathcal{Q}, \alpha_{\mathcal{Q}})$ along $f \colon (A, I) \to (A', I')$ is defined to be
$(\mathcal{Q}^f, \alpha^f_{\mathcal{Q}})$, which will be denoted by
$
f^*\mathcal{Q}:=(\mathcal{Q}^f, \alpha^f_{\mathcal{Q}}).
$
\end{defn}

\subsection{$G$-$\mu$-display}\label{Subsection:prismatic G-display}

Let $(A, I)$ be an orientable and bounded $\O_E$-prism over $\O$.
We define the following functor
\[
G_{\Prism, A, I} \colon (A, I)_\et \to \mathrm{Set}, \quad B \mapsto G(B)_{IB}.
\]
This is (non-canonically) isomorphic to $G_{\Prism, A}$, and hence it forms a sheaf with respect to the $(\pi, I)$-completely \'etale topology.
The sheaf $G_{\Prism, A, I}$ is equipped with a natural action of $G_{\mu, A, I}$.

\begin{defn}[{$G$-$\mu$-display}]\label{Definition:G mu display over oriented prisms}
Let $(A, I)$ be an orientable and bounded $\O_E$-prism over $\O$.
\begin{enumerate}
    \item A \textit{$G$-$\mu$-display} over
    $(A, I)$ is a pair
    \[
    (\mathcal{Q}, \alpha_\mathcal{Q})
    \]
    where $\mathcal{Q}$ is a $G_{\mu, A, I}$-torsor (with respect to the $(\pi, I)$-completely \'etale topology) and $\alpha_\mathcal{Q} \colon \mathcal{Q} \to G_{\Prism, A, I}$ is a $G_{\mu, A, I}$-equivariant map of sheaves on $(A, I)^{\op}_\et$.
    The $G_{\mu, A, I}$-torsor $\mathcal{Q}$ is called the \textit{underlying $G_{\mu, A, I}$-torsor} of $(\mathcal{Q}, \alpha_\mathcal{Q})$.
    When there is no possibility of confusion, we write $\mathcal{Q}$ instead of $(\mathcal{Q}, \alpha_\mathcal{Q})$.
    \item An isomorphism
    $g \colon (\mathcal{Q}, \alpha_\mathcal{Q}) \to (\mathcal{R}, \alpha_\mathcal{R})$ of $G$-$\mu$-displays over 
    $(A, I)$
    is an isomorphism $g \colon \mathcal{Q} \overset{\sim}{\to} \mathcal{R}$ of $G_{\mu, A, I}$-torsors such that $\alpha_{\mathcal{R}} \circ g=\alpha_\mathcal{Q}$.
\end{enumerate}
\end{defn}



\begin{rem}[{$G$-$\mu$-displays and banal $G$-$\mu$-displays}]\label{Remark:G-display and banal G-display}
The groupoid of banal $G$-$\mu$-displays over $(A, I)$ in the sense of Definition \ref{Definition:Banal G display} is naturally equivalent to
the groupoid of $G$-$\mu$-displays
$(\mathcal{Q}, \alpha_\mathcal{Q})$
over $(A, I)$
such that $\mathcal{Q}$ is trivial as a $G_{\mu, A, I}$-torsor.
For an element $X \in G(A)_I$,
we define a $G$-$\mu$-display
\[
\mathcal{Q}_X:=(G_{\mu, A, I}, \alpha_X)
\]
over $(A, I)$ where $\alpha_X \colon G_{\mu, A, I} \to G_{\Prism, A, I}$ is given by $1 \mapsto X$.
This corresponds to the
banal $G$-$\mu$-display
$\mathcal{Q}_X$
over $(A, I)$ defined in Remark \ref{Remark:quotient groupoid banal G-displays}.
We say that a $G$-$\mu$-display
$(\mathcal{Q}, \alpha_\mathcal{Q})$
over $(A, I)$
is \textit{banal}
if the underlying $G_{\mu, A, I}$-torsor $\mathcal{Q}$ is trivial.
\end{rem}

We write
\[
G\mathchar`-\mathrm{Disp}_\mu(A, I) \quad \text{and} \quad G\mathchar`-\mathrm{Disp}_\mu(A, I)_{\mathrm{banal}}
\]
for the groupoid of $G$-$\mu$-displays over $(A, I)$ and the groupoid of banal $G$-$\mu$-displays over $(A, I)$, respectively.

\begin{rem}\label{Remark:comparison base field}
    Let $\widetilde{k}$ be a perfect field containing $k$.
    We set $\widetilde{\O}:=W(\widetilde{k}) \otimes_{W(\F_q)} \O_E$.
    Let $\widetilde{\mu} \colon \G_m \to G_{\widetilde{\O}}$ be the base change of $\mu$ along the natural homomorphism $\O \to \widetilde{\O}$.
    Then, for an orientable and bounded $\O_E$-prism $(A, I)$ over $\widetilde{\O}$,
    a $G$-$\widetilde{\mu}$-display over $(A, I)$ is the same as a $G$-$\mu$-display over $(A, I)$.
\end{rem}

\begin{rem}\label{Remark:compare with Lau, and Bartling perfectoid case}
The notion of $G$-$\mu$-display was originally introduced by B\"ultel \cite{Bultel}, B\"ultel--Pappas \cite{Bultel-Pappas}, and Lau \cite{Lau21} in different settings.
The definition given here is simply an appropriate adaptation of Lau's approach to the context of prisms (and more generally $\O_E$-prisms).
If
$\O_E=\Z_p$ and $\mu$ is $1$-bounded,
the notion of $G$-$\mu$-display for 
an oriented perfect prism
$(W(R^\flat), \xi)$ has already appeared in \cite[Definition 12]{Bartling}.
Moreover, he also claimed that the same construction should work for more general oriented prisms in \cite[Remark 14]{Bartling}.
\end{rem}

We can generalize Example \ref{Example:banal GLn displays} as follows.


\begin{ex}[{Breuil--Kisin modules of type $\mu$ and $\GL_n$-$\mu$-displays}]\label{Example:GLn displays}
Assume that $G=\GL_n$ and let the notation be as in Example \ref{Example:banal GLn displays}.
Let
$M$
be a Breuil--Kisin module of type $\mu$ over $(A, I)$.
We shall construct a $\GL_n$-$\mu$-display over $(A, I)$ associated with $M$.
Let $\{ \Fil^i_\mu \}_{i \in \Z}$ be the filtration of $A^n$ defined in Remark \ref{Remark:standard filtration}.
The functor
\[
\mathcal{Q}:= \underline{\mathrm{Isom}}_{\Fil}(A^n, \phi^*M)
\colon (A, I)_\et \to \mathrm{Set}
\]
sending $B \in (A, I)_\et$ to the set of isomorphisms
$h \colon B^n \overset{\sim}{\to} (\phi^*M) \otimes_A B$ preserving the filtrations is a $(\GL_n)_{\mu, A, I}$-torsor by
Example \ref{Example:display group GLn case},
Remark \ref{Remark:standard filtration},
and the fact that $M$ is $(\pi, I)$-completely \'etale locally on $A$ banal.
Let $d \in IB$ be a generator.
For an isomorphism $h$ as above,
we define 
$\alpha_d(h) \colon B^n \overset{\sim}{\to} B^n$
in the same way as in the proof of Lemma \ref{Lemma:banal BK type mu standard form}.
The element $\alpha_\mathcal{Q}(h) \in \GL_n(B)_{IB}$ such that $\alpha_\mathcal{Q}(h)_d=\alpha_d(h)$ does not depend on the choice of $d$.
In this way, we obtain a $(\GL_n)_{\mu, A, I}$-equivariant morphism
$\alpha_\mathcal{Q} \colon \mathcal{Q} \to G_{\Prism, A, I}$
of sheaves, so that the pair
$(\mathcal{Q}, \alpha_\mathcal{Q})$
is a $\GL_n$-$\mu$-display over $(A, I)$.
We claim that the resulting functor
\[
\mathrm{BK}_\mu(A, I)^{\simeq} \to \GL_n\mathchar`-\mathrm{Disp}_\mu(A, I),
\]
where $\mathrm{BK}_\mu(A, I)^{\simeq}$ denotes the groupoid of Breuil--Kisin modules over $(A, I)$ of type $\mu$, is an equivalence.
Indeed, by virtue of Corollary \ref{Corollary:flat descent for dispalyed BK modules},
it suffices to show that the induced functor
\begin{equation}\label{equation:functor gln A}
    \mathrm{BK}_\mu(A, I)^{\simeq}_{\mathrm{banal}} \to \GL_n\mathchar`-\mathrm{Disp}_\mu(A, I)_{\mathrm{banal}}
\end{equation}
is an equivalence.
For a generator $d \in I$, we constructed an equivalence
\[
[\GL_n(A)_I/(\GL_n)_\mu(A, I)] \overset{\sim}{\to} \mathrm{BK}_\mu(A, I)^{\simeq}_{\mathrm{banal}}
\]
in Example \ref{Example:banal GLn displays}.
One can check that
the composition of this functor with (\ref{equation:functor gln A})
is equivalent to the functor $X \mapsto \mathcal{Q}_X$.
It follows that (\ref{equation:functor gln A}) is an equivalence.
\end{ex}


Let $f \colon (A, I) \to (A', I')$ be a map of orientable and bounded $\O_E$-prisms over $\O$.
There is a canonical way to extend the base change functor $f^*$ defined in Definition \ref{Definition:base change for banal} to the groupoids of $G$-$\mu$-displays.
Formally, we can define it as follows.

The functor
$(A, I)_\et \to (A', I')_\et$
sending $B \in (A, I)_\et$ to the $(\pi, I)$-adic completion $B'$ of $B \otimes_A A'$ induces a morphism of sites
$
f \colon (A', I')^{\op}_\et \to (A, I)^{\op}_\et,
$
and thus
a morphism of the associated topoi
\[
f \colon ((A', I')^{\op}_\et)^\sim \to ((A, I)^{\op}_\et)^\sim.
\]
We have a natural homomorphism
$f \colon f^{-1}G_{\Prism, A} \to G_{\Prism, A'}$
of group sheaves, which in turn induces a homomorphism
$
f \colon f^{-1}G_{\mu, A, I} \to G_{\mu, A', I'}
$
of group sheaves.
Moreover, we can extend the map 
$f \colon G(A)_I \to G(A')_{I'}$ defined in (\ref{equation:the map f_I}) to a morphism
$
f \colon f^{-1}G_{\Prism, A, I} \to G_{\Prism, A', I'}
$
of sheaves, which is equivariant for the action of 
$f^{-1}G_{\mu, A, I}$.

\begin{defn}[Base change]\label{Definition:base change for G-displays}
Let
$(\mathcal{Q}, \alpha_{\mathcal{Q}})$ be a $G$-$\mu$-display over $(A, I)$.
The pushout of
the $f^{-1}G_{\mu, A, I}$-torsor
$f^{-1}\mathcal{Q}$ along
$f \colon f^{-1}G_{\mu, A, I} \to G_{\mu, A', I'}$ is denoted by $f^*\mathcal{Q}$.
By the universal property of $f^*\mathcal{Q}$, 
the composition 
$f^{-1}\mathcal{Q} \to f^{-1}G_{\Prism, A, I} \to G_{\Prism, A', I'}$
factors through a unique $G_{\mu, A', I'}$-equivariant map
$f^*(\alpha_{\mathcal{Q}}) \colon f^*\mathcal{Q} \to G_{\Prism, A', I'}$.
The base change of $(\mathcal{Q}, \alpha_{\mathcal{Q}})$ along $f \colon (A, I) \to (A', I')$ is defined to be
\[
(f^*\mathcal{Q}, f^*(\alpha_{\mathcal{Q}})).
\]
This construction is functorial in $\mathcal{Q}$;
for an isomorphism
$g \colon \mathcal{Q} \to \mathcal{R}$
of $G$-$\mu$-displays over $(A, I)$,
we write
$
f^*g \colon f^*\mathcal{Q} \to f^*\mathcal{R}
$
for the induced isomorphism of $G$-$\mu$-displays over $(A', I')$.
\end{defn}




It will be convenient to introduce $G$-$\mu$-displays over a (not necessarily orientable) bounded $\O_E$-prism over $\O$, in an ad hoc way.
This enables us to formulate some results
in a manner that is more consistent with the terminology commonly employed in the literature.

\begin{defn}\label{Definition:G mu display over general bounded prisms}
Let $(A, I)$ be a bounded $\O_E$-prism over $\O$.
Let
$(A, I)_{\Prism, \ori}$
be the category of orientable and bounded $\O_E$-prisms $(B, J)$ with a map $(A, I) \to (B, J)$ of $\O_E$-prisms.
We call the following groupoid
\[
G\mathchar`-\mathrm{Disp}_\mu(A, I):= {2-\varprojlim}_{(B, J) \in (A, I)_{\Prism, \ori}} G\mathchar`-\mathrm{Disp}_\mu(B, J)
\]
the \textit{groupoid of $G$-$\mu$-displays over $(A, I)$}.
A \textit{$G$-$\mu$-display} over $(A, I)$ is defined to be an object of this groupoid.
\end{defn}

We recall that any bounded $\O_E$-prism $(A, I)$ 
admits a faithfully flat map $(A, I) \to (B, J)$ of bounded $\O_E$-prisms such that $(B, J)$ is orientable; see Remark \ref{Remark:flat locally orientable}.

A map $f \colon (A, I) \to (A', I')$ of bounded $\O_E$-prisms over $\O$ gives rise to a natural functor
$f^* \colon G\mathchar`-\mathrm{Disp}_\mu(A, I) \to G\mathchar`-\mathrm{Disp}_\mu(A', I')$.
When $(A, I)$ is orientable, the groupoid $G\mathchar`-\mathrm{Disp}_\mu(A, I)$ defined in Definition \ref{Definition:G mu display over general bounded prisms} is canonically isomorphic to the groupoid of $G$-$\mu$-displays over $(A, I)$ in the sense of Definition \ref{Definition:G mu display over oriented prisms}, so that the notation is consistent with the previous one.

\begin{ex}\label{Example:GLn display unorientable case}
    Let $(A, I)$ be a bounded $\O_E$-prism over $\O$.
    Assume that $G=\GL_n$ and let the notation be as in Example \ref{Example:banal GLn displays}.
    By virtue of Corollary \ref{Corollary:flat descent for dispalyed BK modules} and Example \ref{Example:GLn displays}, we obtain an equivalence
    $
\mathrm{BK}_\mu(A, I)^{\simeq} \overset{\sim}{\to} \GL_n\mathchar`-\mathrm{Disp}_\mu(A, I).
$
\end{ex}

\subsection{Flat descent}\label{Subsection:flat descent}

We can prove the following ($(\pi, I)$-completely) flat descent result, which is an analogue of \cite[Lemma 5.4.2]{Lau21}.

\begin{prop}[Flat descent]\label{Proposition:flat descent of G display}
The fibered category over
$
(\O)^{\op}_{\Prism, \O_E}
$
which associates to each $(A, I) \in (\O)_{\Prism, \O_E}$
the groupoid
$
G\mathchar`-\mathrm{Disp}_\mu(A,I)
$
is a stack with respect to the flat topology.
\end{prop}

\begin{proof}
By the definition of
$
G\mathchar`-\mathrm{Disp}_\mu(A,I)
$
given in Definition \ref{Definition:G mu display over general bounded prisms},
it suffices to prove flat descent for $G$-$\mu$-displays over orientable and bounded $\O_E$-prisms $(A, I)$ over $\O$.
We write $(A, I)_\Prism:=(A, I)_{\Prism, \ori}$ for the category $(A, I)_{\Prism, \ori}$ introduced in Definition \ref{Definition:G mu display over general bounded prisms}.
We endow
$(A, I)^{\op}_{\Prism}$ with the flat topology.
Note that $(A, I)_\et$ can be regarded as a full subcategory of $(A, I)_{\Prism}$.
We can define sheaves $G_{\Prism, A, I}$, $G_{\mu, A, I}$, and $(P_\mu)_{\overline{\Prism}, A}$ on the site $(A, I)^{\op}_{\Prism}$ in the same way, and $G_{\mu, A, I}$ acts on $G_{\Prism, A, I}$ similarly.
It then suffices to prove that the groupoid
$
G\mathchar`-\mathrm{Disp}_\mu(A, I)
$
is equivalent to the groupoid
$
G\mathchar`-\mathrm{Disp}^{\fl}_\mu(A, I)
$
of pairs 
$(\mathcal{Q}, \alpha_{\mathcal{Q}})$
where $\mathcal{Q}$ is a $G_{\mu, A, I}$-torsor on the site
$(A, I)^{\op}_{\Prism}$
and $\alpha_{\mathcal{Q}} \colon \mathcal{Q} \to G_{\Prism, A, I}$ is a $G_{\mu, A, I}$-equivariant map of sheaves.

We claim that every $G_{\mu, A, I}$-torsor $\mathcal{Q}$ on the site
$(A, I)^{\op}_{\Prism}$
is trivialized by a $(\pi, I)$-completely \'etale covering $(A, I) \to (B, IB)$.
Indeed, in the same way as Corollary \ref{Corollary:trivial Gmu torsor}, we can prove that if the pushout of $\mathcal{Q}$ along
$G_{\mu, A, I} \to (P_\mu)_{\overline{\Prism}, A}$
is trivial as a $(P_\mu)_{\overline{\Prism}, A}$-torsor, then $\mathcal{Q}$ is itself trivial.
The same argument as in the proof of Proposition \ref{Proposition:equivalences of pi completely etale torsors}
shows that any $(P_\mu)_{\overline{\Prism}, A}$-torsor $\mathcal{P}$ on the site $(A, I)^{\op}_{\Prism}$ arises from a $(P_\mu)_{A/I}$-torsor over $\Spec A/I$ with respect to the \'etale topology, which in turn implies that $\mathcal{P}$ is trivialized by a $(\pi, I)$-completely \'etale covering $(A, I) \to (B, IB)$.
The claim then follows.


By the claim,
the restriction of each pair $(\mathcal{Q}, \alpha_{\mathcal{Q}}) \in G\mathchar`-\mathrm{Disp}^{\fl}_\mu(A, I)$ to the site
$(A, I)_\et$
is a $G$-$\mu$-display over $(A, I)$, so that we obtain a functor
$
G\mathchar`-\mathrm{Disp}^{\fl}_\mu(A, I) \to G\mathchar`-\mathrm{Disp}_\mu(A, I).
$
Let us define a functor
$
G\mathchar`-\mathrm{Disp}_\mu(A, I) \to G\mathchar`-\mathrm{Disp}^{\fl}_\mu(A, I)
$
as follows.
Let $(\mathcal{Q}, \alpha_\mathcal{Q}) \in G\mathchar`-\mathrm{Disp}_\mu(A, I)$.
The claim implies that
the functor $\mathcal{Q}'$
sending
an orientable and bounded $\O_E$-prism $(B, J)$ with a map $f \colon (A, I) \to (B, J)$
to the set $\mathcal{Q}'(B):=(f^*\mathcal{Q})(B)$
forms a sheaf with respect to the flat topology.
Then $\alpha_\mathcal{Q}$ induces a $G_{\mu, A, I}$-equivariant map 
$\alpha_{\mathcal{Q}'} \colon \mathcal{Q}' \to G_{\Prism, A, I}$
of sheaves.
This construction gives a functor
$
G\mathchar`-\mathrm{Disp}_\mu(A, I) \to G\mathchar`-\mathrm{Disp}^{\fl}_\mu(A, I).
$
By construction, it is clear that these two functors are inverse to each other.

The proof of Proposition \ref{Proposition:flat descent of G display} is complete.
\end{proof}

\subsection{Hodge filtrations}\label{Subsection:Hodge filtrations}

We define the Hodge filtrations for $G$-$\mu$-displays, following \cite[Section 7.4]{Lau21}.

Let
$(A, I)$
be an orientable and bounded $\O_E$-prism over $\O$.
Let us denote
the inclusion
$G_{\mu, A, I} \hookrightarrow G_{\Prism, A}$
by $\tau$.
The composition of $\tau$ with the projection map
$G_{\Prism, A} \to G_{\overline{\Prism}, A}$
is denoted by $\overline{\tau}$.
(Here $G_{\overline{\Prism}, A}:=(G_\O)_{\overline{\Prism}, A}$; see Section \ref{Subsection:Display groups on prismatic sites}.)
By Lemma \ref{Lemma:Pmu structure}, the homomorphism $\overline{\tau}$ factors through
a homomorphism
$\overline{\tau}_P \colon G_{\mu, A, I} \to (P_{\mu})_{\overline{\Prism}, A}$.
In summary, we have the following commutative diagram:
\[
\xymatrix{
G_{\mu, A, I} \ar^-{\tau}[r]  \ar[d]_-{\overline{\tau}_P} \ar^-{\overline{\tau}}[rd] &  G_{\Prism, A}   \ar[d]_-{} \\
(P_{\mu})_{\overline{\Prism}, A} \ar[r]^-{} & G_{\overline{\Prism}, A}.
}
\]

\begin{defn}[Hodge filtration]\label{Definition:Hodge filtration of G-displays}
Let
$\mathcal{Q}$ be a $G$-$\mu$-display over $(A, I)$.
We write
\[
\mathcal{Q}_{A/I}:=\mathcal{Q}^{\overline{\tau}} \quad (\text{resp.}\ P(\mathcal{Q})_{A/I}:=\mathcal{Q}^{\overline{\tau}_P})
\]
for the pushout of $\mathcal{Q}$ along $\overline{\tau}$ (resp.\ $\overline{\tau}_P$), which is a $G_{\overline{\Prism}, A}$-torsor
(resp.\ a $(P_{\mu})_{\overline{\Prism}, A}$-torsor).
There is a natural $(P_{\mu})_{\overline{\Prism}, A}$-equivariant injection
\[
P(\mathcal{Q})_{A/I} \hookrightarrow \mathcal{Q}_{A/I}
\]
of sheaves on $(A, I)^{\op}_\et$.
We call $P(\mathcal{Q})_{A/I}$
(or the injection $P(\mathcal{Q})_{A/I} \hookrightarrow \mathcal{Q}_{A/I}$)
the \textit{Hodge filtration} of $\mathcal{Q}_{A/I}$.
If there is no risk of confusion,
we also say that
$P(\mathcal{Q})_{A/I}$
is the Hodge filtration of $\mathcal{Q}$.
\end{defn}

\begin{ex}\label{Example:Hodge filtration of GLn-display}
Assume that $G=\GL_n$ and let the notation be as in Example \ref{Example:banal GLn displays}.
Let
$M$
be a Breuil--Kisin module over $(A, I)$ of type $\mu$
and
$\mathcal{Q}$
the associated $\GL_n$-$\mu$-display over $(A, I)$ given in Example \ref{Example:GLn displays}.
Recall that the filtration $\{ \Fil^i(\phi^*M) \}_{i \in \Z}$
defines
the Hodge filtration $\{ P^i \}_{i \in \Z}$ of
$M_{\dR}=(\phi^*M)/I(\phi^*M)$.
Similarly, the filtration $\{ \Fil^i_\mu \}_{i \in \Z}$ of $A^n$ induces a filtration
of $(A/I)^n$.
Then, we have
\begin{align*}
    \mathcal{Q}_{A/I} &\simeq \underline{\mathrm{Isom}}((A/I)^n, M_{\dR}),\\
    P(\mathcal{Q})_{A/I} &\simeq \underline{\mathrm{Isom}}_{\Fil}((A/I)^n, M_{\dR}).
\end{align*}
Here $\underline{\mathrm{Isom}}((A/I)^n, M_{\dR})$
(resp.\ $\underline{\mathrm{Isom}}_{\Fil}((A/I)^n, M_{\dR})$)
is the functor sending $B \in (A, I)_\et$ to the set of isomorphisms
$f \colon (B/IB)^n \overset{\sim}{\to} (M_{\dR})_{B/IB}$
(resp.\ the set of isomorphisms $f \colon (B/IB)^n \overset{\sim}{\to} (M_{\dR})_{B/IB}$ preserving the filtrations).
\end{ex}

\begin{ex}\label{Example:Hodge filtration banal case}
Let $X \in G(A)_I$ be an element.
Then the Hodge filtration associated with $\mathcal{Q}_X$
can be identified with the natural inclusion
$(P_{\mu})_{\overline{\Prism}, A} \hookrightarrow G_{\overline{\Prism}, A}$.
\end{ex}


\begin{rem}\label{Remark:Hodge filtration schematic}
Let
$\mathcal{Q}$ be a $G$-$\mu$-display over $(A, I)$.
It follows from Proposition \ref{Proposition:equivalences of pi completely etale torsors} that
the $G_{\overline{\Prism}, A}$-torsor
$\mathcal{Q}_{A/I}$
(resp.\ the $(P_{\mu})_{\overline{\Prism}, A}$-torsor $P(\mathcal{Q})_{A/I}$)
arises from a unique (up to isomorphism)
$G_{A/I}$-torsor
(resp.\ $(P_\mu)_{A/I}$-torsor) over $\Spec A/I$, which will be denoted by the same symbol.
\end{rem}



\begin{prop}\label{Proposition:G display with trivial Hodge filtration is banal}
A $G$-$\mu$-display
$\mathcal{Q}$
over $(A, I)$ is banal if and only if the Hodge filtration
$P(\mathcal{Q})_{A/I}$
is a trivial $(P_{\mu})_{A/I}$-torsor over $\Spec A/I$.
\end{prop}

\begin{proof}
This is a restatement of Corollary \ref{Corollary:trivial Gmu torsor} in the current context.
\end{proof}


\subsection{$\phi$-$G$-torsor of type $\mu$}\label{Subsection:phi G torsor of type mu}

Let $(A, I)$ be an orientable and bounded $\O_E$-prism over $\O$.
We set $A[1/\phi(I)]:=A[1/\phi(d)]$ for a generator $d \in I$, which does not depend on the choice of $d$.

\begin{defn}\label{Definition:phi-G-torsor}
A \textit{$\phi$-$G$-torsor} over $(A, I)$
is a pair
$(\mathcal{P}, \phi_\mathcal{P})$ consisting of
a $G_A$-torsor $\mathcal{P}$ over $\Spec A$ (with respect to the \'etale topology) and
an isomorphism
\[
\phi_\mathcal{P} \colon (\phi^*\mathcal{P})[1/\phi(I)] \overset{\sim}{\to} \mathcal{P}[1/\phi(I)]
\]
of $G_{A[1/\phi(I)]}$-torsors over $\Spec A[1/\phi(I)]$.
\end{defn}

Here, for a $G_A$-torsor $\mathcal{P}$ over $\Spec A$,
we let
$\phi^*\mathcal{P}$ denote
the base change of $\mathcal{P}$ along $\phi \colon A \to A$.
Since $\phi$ is $\O_E$-linear and $G$ is defined over $\O_E$,
we have $\phi^*(G_A)=G_A$, and hence $\phi^*\mathcal{P}$ is a $G_A$-torsor over $\Spec A$.
Moreover, we write
$\mathcal{P}[1/\phi(I)]$ for the base change
$\mathcal{P} \times_{\Spec A} \Spec A[1/\phi(I)]$.

In the case where $G=\GL_n$,
we can describe
$\phi$-$\GL_n$-torsors as follows:

\begin{ex}\label{Example:phi GLn torsor}
Assume that $G=\GL_n$.
Let
$(M, \phi_M)$ be a pair consisting of a finite projective $A$-module of constant rank $n$ and an
$A[1/\phi(I)]$-linear isomorphism
\[
\phi_M \colon (\phi^*M)[1/\phi(I)] \overset{\sim}{\to} M[1/\phi(I)],
\]
where $M[1/\phi(I)]:=M[1/\phi(d)]$.
We write
$
\mathcal{P}:=\underline{\mathrm{Isom}}(A^n, M)
$
for the $G_A$-torsor over $\Spec A$
defined by sending an $A$-algebra $B$
to
the set of isomorphisms $B^n \simeq M_B$.
The isomorphism $\phi_M$ gives rise to an isomorphism
$
\phi_\mathcal{P} \colon (\phi^*\mathcal{P})[1/\phi(I)] \overset{\sim}{\to} \mathcal{P}[1/\phi(I)].
$
This construction induces an equivalence
between the groupoid of pairs $(M, \phi_M)$ as above and the groupoid of $\phi$-$\GL_n$-torsors over $(A, I)$.
\end{ex}

\begin{rem}\label{Remark:descent of frobenius}
Let $\mathcal{P}_1$ and $\mathcal{P}_2$ be $G_A$-torsors over $\Spec A$.
Then the functor sending $B \in (A, I)_\et$ to the set of isomorphisms
\[
\mathcal{P}_1 \times_{\Spec A} \Spec B[1/\phi(d)] \overset{\sim}{\to} \mathcal{P}_2 \times_{\Spec A} \Spec B[1/\phi(d)]
\]
of $G_{B[1/\phi(d)]}$-torsors
forms a sheaf.
This follows from Proposition \ref{Proposition:flat descent for finite projective modules} and the Tannakian perspective of torsors (see the proof of Proposition \ref{Proposition:equivalences of pi completely etale torsors}).
Moreover, the fibered category over $(A, I)^{\op}_{\et}$ which associates to each $B \in (A, I)_\et$ the groupoid of $\phi$-$G$-torsors over $(B, IB)$ is a stack with respect to the $(\pi, I)$-completely \'etale topology.
\end{rem}

By virtue of Remark \ref{Remark:descent of frobenius}, we can attach a $\phi$-$G$-torsor over $(A, I)$ to a $G$-$\mu$-display over $(A, I)$ as follows.
We first note that for any $X \in G(A)_I$, the element
\begin{equation}\label{equation:Xphi}
    X_\phi := X_d\phi(\mu(d)) \in G(A[1/\phi(I)])
\end{equation}
is independent of the choice of a generator $d \in I$.
Let
$(\mathcal{Q}, \alpha_{\mathcal{Q}})$
be a $G$-$\mu$-display over $(A, I)$.
We write
\[
\mathcal{Q}_A:=\mathcal{Q}^\tau
\]
for the pushout of
$\mathcal{Q}$
along the inclusion
$\tau \colon G_{\mu, A, I} \hookrightarrow G_{\Prism, A}$.
As in Proposition \ref{Proposition:equivalences of pi completely etale torsors}, this corresponds to a $G_A$-torsor over $\Spec A$, again denoted by $\mathcal{Q}_A$.
We shall construct an isomorphism
\[
\phi_{\mathcal{Q}_A} \colon (\phi^*(\mathcal{Q}_A))[1/\phi(I)] \overset{\sim}{\to} \mathcal{Q}_A[1/\phi(I)].
\]
Assume first that $\mathcal{Q}$ is banal, so that $\mathcal{Q}_A$ is a trivial $G_A$-torsor.
We choose an element
\[
x \in \mathcal{Q}(A) \subset \mathcal{Q}_A(A)
\]
and let $\phi^*(x) \in (\phi^*(\mathcal{Q}_A))(A)$ be the element obtained by base change.
We define $\phi_{\mathcal{Q}_A}$ to be the unique isomorphism which sends $\phi^*(x)$ to
\[
x \cdot (\alpha_{\mathcal{Q}}(x))_\phi \in \mathcal{Q}_A(A[1/\phi(I)]).
\]
We note that $\phi_{\mathcal{Q}_A}$ is independent of the choice of $x \in \mathcal{Q}(A)$.
In the general case,
let
$A \to B$
be a $(\pi, I)$-completely \'etale covering such that $\mathcal{Q}$ is banal over $(B, IB)$.
It follows from Remark \ref{Remark:descent of frobenius} that there exists a unique isomorphism $\phi_{\mathcal{Q}_A}$ whose base change to $B[1/\phi(d)]$ agrees with the isomorphism constructed above. The resulting isomorphism $\phi_{\mathcal{Q}_A}$ does not depend on the choice of $B$.


\begin{defn}[{Underlying $\phi$-$G$-torsor}]\label{Definition:underlying phi-G-torsor}
Let
$\mathcal{Q}$
be a $G$-$\mu$-display over $(A, I)$.
The $\phi$-$G$-torsor
\[
\mathcal{Q}_{\phi}:=(\mathcal{Q}_A, \phi_{\mathcal{Q}_A})
\]
over $(A, I)$
constructed above is called the \textit{underlying $\phi$-$G$-torsor} of $\mathcal{Q}$.
\end{defn}


\begin{ex}\label{Example:phi-GLn-torsor associated with GLn display}
Assume that $G=\GL_n$ and let the notation be as in Example \ref{Example:GLn displays}.
Let
$M$
be a 
Breuil--Kisin module over $(A, I)$ of type $\mu$
and
$\mathcal{Q}$
the associated $\GL_n$-$\mu$-display over $(A, I)$.
Then,
the underlying $\phi$-$\GL_n$-torsor
$\mathcal{Q}_{\phi}$
of $\mathcal{Q}$
corresponds to the pair
\[
(\phi^*M, \phi^*(F_M))
\]
under the equivalence given in Example \ref{Example:phi GLn torsor}.
\end{ex}


\begin{defn}[{$\phi$-$G$-torsor of type $\mu$}]\label{Definition:phi-G-torsor of type mu}
We say that a $\phi$-$G$-torsor $(\mathcal{P}, \phi_\mathcal{P})$ over $(A, I)$ is \textit{of type $\mu$} if there exists a $(\pi, I)$-completely \'etale covering
    $A \to B$ such that
    the base change
    $\mathcal{P}_B$
    is trivial as a $G_B$-torsor and,
    via some (and hence any) trivialization
    $\mathcal{P}_B \simeq G_B$,
    the isomorphism $\phi_\mathcal{P}$ is given by $g \mapsto Yg$ for an element $Y$ in the double coset
    \[
    G(B)\phi(\mu(d))G(B) \subset G(B[1/\phi(d)]).
    \]
\end{defn}

\begin{ex}\label{Example:underlying phi G torsor is of type mu}
The underlying $\phi$-$G$-torsor $\mathcal{Q}_{\phi}$ of a $G$-$\mu$-display $\mathcal{Q}$ is of type $\mu$.
\end{ex}



\subsection{$G$-$\mu$-displays for perfectoid rings}\label{Subsection:G displays for perfectoid rings}

Let $R$ be a perfectoid ring over $\O$.
In this subsection,
we discuss the relation between
$G$-$\mu$-displays over the $\O_E$-prism
$
(W_{\O_E}(R^\flat), I_R)
$
and their underlying $\phi$-$G$-torsors.


Following \cite{PappasRapoport21} and \cite{Bartling},
a $\phi$-$G$-torsor $(\mathcal{P}, \phi_\mathcal{P})$ over $(W_{\O_E}(R^\flat), I_R)$
(in the sense of Definition \ref{Definition:phi-G-torsor})
is also called a
\textit{$G$-Breuil-Kisin-Fargues module}
(or simply a $G$-BKF module)
for $R$.
In Definition \ref{Definition:phi-G-torsor of type mu}, we introduced the notion of $G$-BKF module of type $\mu$.


\begin{rem}[{$G$-BKF module of type $\mu$}]\label{Remark:G-BKF of type mu}
Assume that $\O_E=\Z_p$.
In \cite{Bartling}, Bartling introduced the notion of $G$-BKF module \textit{of type $\mu$} in a different way.
Namely, in \cite{Bartling},
a $G$-BKF module $(\mathcal{P}, \phi_\mathcal{P})$ for $R$ is said to be of type $\mu$ if for any homomorphism $R \to V$ with $V$ a
$p$-adically complete valuation ring of rank $\leq 1$
whose fraction field is algebraically closed,
the base change
of $(\mathcal{P}, \phi_\mathcal{P})$ along
$W(R^\flat) \to W(V^\flat)$
is of type $\mu$ in the sense of Definition \ref{Definition:phi-G-torsor of type mu}.
In Proposition \ref{Proposition:comparison of notions of type mu} below, we will prove that this notion agrees with the one introduced in Definition \ref{Definition:phi-G-torsor of type mu}.
\end{rem}



Our first task is to prove that the functor $\mathcal{Q} \mapsto \mathcal{Q}_\phi$
is an equivalence:

\begin{prop}\label{Proposition:equivalence of G displays and phi G torsors perfectoid}
The functor $\mathcal{Q} \mapsto \mathcal{Q}_\phi$
from the groupoid of $G$-$\mu$-displays over $(W_{\O_E}(R^\flat), I_R)$ to the groupoid of $G$-BKF modules for $R$ of type $\mu$ is an equivalence.
\end{prop}


\begin{proof}
By Example \ref{Example:perfectoid ring etale morphism}, any $(\pi, I_R)$-completely \'etale $W_{\O_E}(R^\flat)$-algebra is of the form $W_{\O_E}(S^\flat)$ for a perfectoid ring $S$ that is $\pi$-completely \'etale over $R$.
Thus,
by virtue of Remark \ref{Remark:descent of frobenius},
we may work $(\pi, I_R)$-completely \'etale locally on $W_{\O_E}(R^\flat)$.
The result then follows from the proof of \cite[Lemma 4.8]{Bartling}.
We include the proof for the sake of completeness.
To simplify the notation, we write $A:=W_{\O_E}(R^\flat)$ and $I:=I_R$.

We shall prove that the functor is fully faithful.
It suffices to prove that for all $X, X' \in G(A)_I$ and the associated banal $G$-$\mu$-displays
$\mathcal{Q}_X, \mathcal{Q}_{X'}$
over $(A, I)$,
we have
\[
\Hom(\mathcal{Q}_X, \mathcal{Q}_{X'}) \overset{\sim}{\to} \Hom((\mathcal{Q}_X)_\phi, (\mathcal{Q}_{X'})_\phi).
\]
The left hand side can be identified with the set
$\{\, g \in G_\mu(A, I) \, \vert \, g^{-1}X'_\phi \phi(g)=X_\phi  \, \}$, and
the right hand side can be identified with the set
$\{\, g \in G(A) \, \vert \, g^{-1}X'_\phi \phi(g)=X_\phi  \, \}$.
(See (\ref{equation:Xphi}) for the definition of $X_\phi$.)
Thus, we have to show that an element
$g \in G(A)$ satisfying $g^{-1}X'_\phi \phi(g)=X_\phi$
is contained in $G_\mu(A, I)$.
We fix a generator $\xi \in I$.
Then we have 
\[
\phi(\mu(\xi)g\mu(\xi)^{-1})=(X'_\xi)^{-1}gX_\xi \in G(A).
\]
The bijectivity
of $\phi \colon A \to A$
implies that $\mu(\xi)g\mu(\xi)^{-1} \in G(A)$, which means that $g \in G_\mu(A, I)$.

We next prove that the functor is essentially surjective.
It now suffices to show that a $G$-BKF module
$(\mathcal{P}, \phi_\mathcal{P})$
for $R$, such that $\mathcal{P}=G_{A}$ and the isomorphism $\phi_\mathcal{P}$
corresponds to an element $Y$ in the double coset
$
G(A)\phi(\mu(\xi))G(A),
$
arises from a banal $G$-$\mu$-display over $(A, I)$.
In other words, we have to show that
there is an element $X \in G(A)_I$
such that
for some $g \in G(A)$, we have
$
g^{-1}X_\xi \phi(\mu(\xi)) \phi(g) =Y
$
(this means that $(\mathcal{P}, \phi_\mathcal{P})$ is isomorphic to $(\mathcal{Q}_{X})_\phi$).
Let us write $Y=Z\phi(\mu(\xi))Z'$ for some $Z, Z' \in G(A)$.
Then, we see that
the assertion holds for
the elements
$g:=\phi^{-1}(Z') \in G(A)$
and
$X \in G(A)_I$ such that
$X_\xi=gZ \in G(A)$. 
\end{proof}


Let
$\Perfd_{R}$
be the category
of perfectoid rings over $R$.
We endow $\Perfd^{\op}_{R}$ with
the topology generated by the \textit{$\pi$-complete $\arc$-coverings} (or equivalently, the $p$-complete $\arc$-coverings) in the sense of \cite[Section 2.2.1]{CS}.
This topology is called the \textit{$\pi$-complete $\arc$-topology}.
In the rest of this subsection, we discuss a few $\pi$-complete $\arc$-descent results on torsors.


\begin{rem}\label{Remark:arc topology}
We quickly review the notion of $\pi$-complete $\arc$-covering.
\begin{enumerate}
    \item We say that a homomorphism
$R \to S$
of perfectoid rings over $\O$
is a $\pi$-complete $\arc$-covering if
for any homomorphism $R \to V$ with $V$ a $\pi$-adically complete valuation ring of rank $\leq 1$, there exist an extension of $V \hookrightarrow W$ of $\pi$-adically complete valuation rings of rank $\leq 1$ and a homomorphism $S \to W$ such that the following diagram commutes:
\[
\xymatrix{
R \ar^-{}[r]  \ar[d]_-{} & S \ar[d]_-{}  \\
V \ar[r]^-{} & W.
}
\]
    \item The category $\Perfd^{\op}_R$ admits fiber products; a colimit of the diagram
$
S_2 \leftarrow S_1 \rightarrow S_3
$
in $\Perfd_R$
is given by the $\pi$-adic completion of $S_2 \otimes_{S_1} S_3$ (cf.\ \cite[Proposition 2.1.11]{CS}).
We see that $\Perfd^{\op}_R$ is indeed a site.
    \item
    Let $R$ be a perfectoid ring over $\O$ and
    $R \to S$ a $\pi$-completely \'etale covering.
    Then $S$ is a perfectoid ring as explained in 
    Example \ref{Example:perfectoid ring etale morphism},
    and $R \to S$ is a $\pi$-complete $\arc$-covering; see \cite[Section 2.2.1]{CS}.
    \item Any perfectoid ring $R$ admits a $\pi$-complete $\arc$-covering of the form $R \to \prod_{i \in I} V_i$ where $V_i$ are $\pi$-adically complete valuation rings of rank $\leq 1$ with algebraically closed fraction fields; see \cite[Lemma 2.2.3]{CS}.
\end{enumerate}
\end{rem}

We recall the following result from \cite{Ito-K21}.

\begin{prop}[{\cite[Corollary 4.2]{Ito-K21}}]\label{Proposition:arc descent for finite projective modules}
The fibered category over $\Perfd^{\op}_R$
which associates to a perfectoid ring $S$ over $R$
the category of finite projective $S$-modules satisfies
descent with respect to the $\pi$-complete $\arc$-topology.
In particular, the functor
$
\Perfd_R \to \mathrm{Set}$,
$S \mapsto S$
forms a sheaf.
\end{prop}

\begin{proof}
See \cite[Corollary 4.2]{Ito-K21}.
The second assertion was previously proved in \cite[Proposition 8.10]{BS}.
\end{proof}

\begin{rem}\label{Remark:arc descent for finite projective modules}
In fact, it is proved in \cite[Theorem 1.2]{Ito-K21} that
the functor on $\Perfd_R$ associating to each $S \in \Perfd_R$ the $\infty$-category $\Perf(S)$ of perfect complexes over $S$ satisfies $\pi$-complete $\arc$-hyperdescent.
    Using this,
    we can prove that for any integer $n \geq 1$,
    the functor
    $S \mapsto \Perf(W_{\O_E}(S^\flat)/I^n_S)$
    on $\Perfd_R$
    satisfies $\pi$-complete $\arc$-hyperdescent, by induction on $n$.
    This implies that the functor
    $
    S \mapsto \Perf(W_{\O_E}(S^\flat))
    $
    satisfies $\pi$-complete $\arc$-hyperdescent as well.
    See the discussion in \cite[Section 4.1]{Ito-K21}.
\end{rem}

As a corollary of these results, we have:

\begin{cor}\label{Corollary:descent witt vectors}
The fibered category over $\Perfd^{\op}_R$
which associates to a perfectoid ring $S$ over $R$
the category of finite projective $W_{\O_E}(S^\flat)$-modules
satisfies
descent with respect to the $\pi$-complete $\arc$-topology.
The same holds for finite projective $W_{\O_E}(S^\flat)/I^n_S$-modules.
\end{cor}

\begin{proof}
By the same argument as in the proof of \cite[Corollary 4.2]{Ito-K21},
we can deduce the assertion from Remark \ref{Remark:arc descent for finite projective modules}.
\end{proof}

In particular, the functor
$\Perfd_R \to \mathrm{Set}$, $S \mapsto W_{\O_E}(S^\flat)$
forms a sheaf.
This fact also follows from \cite[Lemma 4.2.6]{CS} or the proof of \cite[Proposition 8.10]{BS} (using that $W(\F_q) \to \O_E$ is flat).

For an affine scheme $X$ over $\O$ (or $R$),
we define a functor
$
X_{\overline{\Prism}} \colon \Perfd_R \to \mathrm{Set},
$
$S \mapsto X(S).
$
Proposition \ref{Proposition:arc descent for finite projective modules} implies that this forms a sheaf.
Similarly, 
for an affine scheme $X$ over $\O$ (or $W_{\O_E}(R^\flat)$),
we define a functor
$
X_{\Prism} \colon \Perfd_R \to \mathrm{Set},
$
$S \mapsto X(W_{\O_E}(S^\flat))$,
which forms a sheaf by Corollary \ref{Corollary:descent witt vectors}.
We have the following analogue of Proposition \ref{Proposition:equivalences of pi completely etale torsors}.

\begin{prop}\label{Proposition:equivalence of arc torsors}
Let $H$ be a smooth affine group scheme over $\O$.
\begin{enumerate}
    \item The functor $\mathcal{P} \mapsto \mathcal{P}_{\overline{\Prism}}$ from the groupoid of $H_R$-torsors over $\Spec R$ to the groupoid of $H_{\overline{\Prism}}$-torsors on $\Perfd^{\op}_R$ is an equivalence.
    \item The functor
    $\mathcal{P} \mapsto \mathcal{P}_{\Prism}$
    from the groupoid of $H_{W_{\O_E}(R^\flat)}$-torsors over $\Spec W_{\O_E}(R^\flat)$ to
the groupoid of $H_{\Prism}$-torsors on $\Perfd^{\op}_R$ is an equivalence.
\end{enumerate}
\end{prop}

\begin{proof}
This can be proved by the same argument as in the proof of Proposition \ref{Proposition:equivalences of pi completely etale torsors}, using Proposition \ref{Proposition:arc descent for finite projective modules} and Corollary \ref{Corollary:descent witt vectors}.
\end{proof}

We define the following functors
\begin{align*}
    G_{\mu, I} &\colon \Perfd_R \to \mathrm{Set}, \quad S \mapsto G_\mu(W_{\O_E}(S^\flat), I_S), \\
    G_{\Prism, I} &\colon \Perfd_R \to \mathrm{Set}, \quad S \mapsto G(W_{\O_E}(S^\flat))_{I_S},
\end{align*}
which form sheaves.
The group sheaf $G_{\mu, I}$ acts on $G_{\Prism, I}$.


\begin{lem}\label{Lemma:Gmu arc torsor is etale torsor}
Let $\mathcal{Q}$ be a $G_{\mu, I}$-torsor with respect to the $\pi$-complete $\arc$-topology.
Then $\mathcal{Q}$ is trivialized by a $\pi$-completely \'etale covering $R \to S$.
\end{lem}

\begin{proof}
We claim that if the pushout of $\mathcal{Q}$ along the homomorphism
$G_{\mu, I} \to (P_\mu)_{\overline{\Prism}}$
is trivial as a $(P_\mu)_{\overline{\Prism}}$-torsor, then $\mathcal{Q}$ is itself trivial.
Indeed, one can prove the analogue of Lemma \ref{Lemma:Gmu torsor successive quotient} for $G_{\mu, I}$, and then the argument as in the proof of Corollary \ref{Corollary:trivial Gmu torsor} works.

By the claim, it suffices to prove that any $(P_\mu)_{\overline{\Prism}}$-torsor
with respect to the $\pi$-complete $\arc$-topology
can be trivialized by a $\pi$-completely \'etale covering $R \to S$.
This is a consequence of Proposition \ref{Proposition:equivalence of arc torsors}.
\end{proof}

\begin{cor}\label{Corollary:arc descent for G-displays}
The fibered category over $\Perfd^{\op}_R$
which associates to a perfectoid ring $S$ over $R$
the groupoid of $G$-$\mu$-displays over $(W_{\O_E}(S^\flat), I_S)$ is a stack with respect to the $\pi$-complete $\arc$-topology.
\end{cor}

\begin{proof}
This can be deduced from Lemma \ref{Lemma:Gmu arc torsor is etale torsor} by the same argument as in the proof of Proposition \ref{Proposition:flat descent of G display}.
\end{proof}

Now we are ready to prove the following result:

\begin{prop}\label{Proposition:comparison of notions of type mu}
For
a $G$-BKF module $(\mathcal{P}, \phi_\mathcal{P})$ for $R$, the following conditions are equivalent.
\begin{enumerate}
    \item $(\mathcal{P}, \phi_\mathcal{P})$ is of type $\mu$ (in the sense of Definition \ref{Definition:phi-G-torsor of type mu}).
    \item There exists a $\pi$-complete $\arc$-covering $R \to S$ such that the base change of
    $(\mathcal{P}, \phi_\mathcal{P})$
    along $W_{\O_E}(R^\flat) \to W_{\O_E}(S^\flat)$
    is of type $\mu$.
    \item For any homomorphism $R \to V$ with $V$ a
$\pi$-adically complete valuation ring of rank $\leq 1$
whose fraction field is algebraically closed,
the base change
of $(\mathcal{P}, \phi_\mathcal{P})$ along
$W(R^\flat) \to W(V^\flat)$
is of type $\mu$.
\end{enumerate}
\end{prop}

\begin{proof}
It is clear that (1) implies (2) and (3).
We assume that the condition (2) is satisfied.
As in the Proposition \ref{Proposition:equivalence of G displays and phi G torsors perfectoid},
the $G$-BKF module $(\mathcal{P}, \phi_\mathcal{P})$ for $R$ then arises from a pair
$
(\mathcal{Q}, \alpha_\mathcal{Q})
$
where $\mathcal{Q}$ is a $G_{\mu, I}$-torsor with respect to the $\pi$-complete $\arc$-topology and $\alpha_\mathcal{Q} \colon \mathcal{Q} \to G_{\Prism, I}$ is a $G_{\mu, I}$-equivariant map of sheaves on $\Perfd^{\op}_R$.
It follows from Lemma \ref{Lemma:Gmu arc torsor is etale torsor} that
the $G_{\mu, I}$-torsor
$\mathcal{Q}$ is trivialized by a $\pi$-completely \'etale covering $R \to S$.
This implies that
$(\mathcal{P}, \phi_\mathcal{P})$ is of type $\mu$.
Thus we get that (2) implies (1).

Assume that the condition (3) is satisfied.
We want to show that this implies (2), which will conclude the proof of the proposition.
By Remark \ref{Remark:arc topology} (4),
there exists a $\pi$-complete $\arc$-covering 
$R \to S=\prod_{i} V_i$ where $V_i$ are $\pi$-adically complete valuation rings of rank $\leq 1$ with algebraically closed fraction fields.
Since $W_{\O_E}(V^\flat_i)$ is strictly henselian, 
the base change
$\mathcal{P}_{W_{\O_E}(V^\flat_i)}$
is a trivial $G_{W_{\O_E}(V^\flat_i)}$-torsor.
It follows that the $G_{W_{\O_E}(S^\flat)}$-torsor $\mathcal{P}_{W_{\O_E}(S^\flat)}$ is trivial as well.
We fix a trivialization
$\mathcal{P}_{W_{\O_E}(S^\flat)} \simeq G_{W_{\O_E}(S^\flat)}$.
Let $\xi \in I_S$ be a generator.
The condition (3) implies that
for each $i$,
the base change of $\phi_\mathcal{P}$
along $W_{\O_E}(R^\flat) \to W_{\O_E}(V^\flat_i)$
corresponds to
an element of $G(W_{\O_E}(V^\flat_i)[1/\phi(\xi)])$
which is of the form
$Z_i\phi(\mu(\xi))Z'_i$
for some $Z_i, Z'_i \in G(W_{\O_E}(V^\flat_i))$, via the induced trivialization
$\mathcal{P}_{W_{\O_E}(V^\flat_i)} \simeq G_{W_{\O_E}(V^\flat_i)}$.
We set
\[
Z:=(Z_i)_i \in G(W_{\O_E}(S^\flat))=\prod_i G(W_{\O_E}(V^\flat_i))
\]
and similarly let $Z':=(Z'_i)_i \in G(W_{\O_E}(S^\flat))$.
Then,
the base change of
$\phi_\mathcal{P}$
along $W_{\O_E}(R^\flat) \to W_{\O_E}(S^\flat)$
corresponds to the element
$Z\phi(\mu(\xi))Z'$.
This means that
the condition (2) is satisfied.
\end{proof}




\section{$G$-$\mu$-displays over complete regular local rings}\label{Section:G-displays over complete regular local rings}

Let $R$ be a complete regular local ring
with a local homomorphism $\O \to R$ which induces an isomorphism on the residue fields.
In this section, we study compatible systems
$\{ \mathcal{Q}_{(A, I)} \}_{(A, I) \in (R)_{\Prism, \O_E}}$ of $G$-$\mu$-displays on
the category $(R)_{\Prism, \O_E}$.
In Section \ref{Subsection:G displays over absolute prismatic sites}, we state the main result
(Theorem \ref{Theorem:main result on G displays over complete regular local rings}) of this paper.
The proof will be given in Section \ref{Subsection:proof of main result}.
In Section \ref{Subsection:Coproducts of Breuil--Kisin prisms} and Section \ref{Subsection:Deformations of isomorphisms},
we discuss a few technical results that will be used in the proof of the main result.



\subsection{$G$-$\mu$-displays over absolute prismatic sites}\label{Subsection:G displays over absolute prismatic sites}

\begin{defn}\label{Definition:G displays over prismatic sites}
Let $R$ be a $\pi$-adically complete $\O$-algebra.
A \textit{$G$-$\mu$-display over} $(R)_{\Prism, \O_E}$ is defined to be an object of the following groupoid
\[
G\mathchar`-\mathrm{Disp}_\mu((R)_{\Prism, \O_E}):= {2-\varprojlim}_{(A, I) \in (R)_{\Prism, \O_E}} G\mathchar`-\mathrm{Disp}_\mu(A, I).
\]
\end{defn}

\begin{rem}\label{Remark:alternative definition of G displays over prismatic sites}
Giving a $G$-$\mu$-display $\mathfrak{Q}$ over $(R)_{\Prism, \O_E}$
is equivalent to giving a $G$-$\mu$-display $\mathfrak{Q}_{(A, I)}$ over $(A, I)$ for each $(A, I) \in (R)_{\Prism, \O_E}$
and an isomorphism
\[
\gamma_f \colon f^*(\mathfrak{Q}_{(A, I)}) \overset{\sim}{\to} \mathfrak{Q}_{(A', I')}
\]
for each morphism $f \colon (A, I) \to (A', I')$ in $(R)_{\Prism, \O_E}$,
such that
$\gamma_{f'} \circ ({f'}^*\gamma_f) = \gamma_{f' \circ f}$
for two morphisms
$f \colon (A, I) \to (A', I')$ and $f' \colon (A', I') \to (A'', I'')$.
We call $\mathfrak{Q}_{(A, I)}$ the \textit{value} of $\mathfrak{Q}$ at $(A, I) \in (R)_{\Prism, \O_E}$.
\end{rem}



Assume that $R$ is a complete regular local ring equipped with a local homomorphism $\O \to R$ which induces an isomorphism on the residue fields.
Let
$
(\O[[t_1, \dotsc, t_n]], (\mathcal{E}))
$
be
an $\O_E$-prism
of Breuil--Kisin type
with an isomorphism
$
R \simeq \O[[t_1, \dotsc, t_n]]/\mathcal{E}
$ over $\O$
(where $n \geq 0$ is the dimension of $R$).
Such an $\O_E$-prism exists; see for example \cite[Section 3.3]{ChengChuangxun}.
We set
$\mathfrak{S}_\O:=\O[[t_1, \dotsc, t_n]]$.
Our goal is to prove the following result.

\begin{thm}\label{Theorem:main result on G displays over complete regular local rings}
Assume that the cocharacter $\mu$ is 1-bounded.
Then the functor
    \[
 G\mathchar`-\mathrm{Disp}_\mu((R)_{\Prism, \O_E}) \to G\mathchar`-\mathrm{Disp}_\mu(\mathfrak{S}_\O, (\mathcal{E})), \quad \mathfrak{Q} \mapsto \mathfrak{Q}_{(\mathfrak{S}_\O, (\mathcal{E}))}
 \]
given by evaluation at $(\mathfrak{S}_\O, (\mathcal{E}))$ is an equivalence.
\end{thm}

The rest of this section is devoted to the proof of Theorem \ref{Theorem:main result on G displays over complete regular local rings}.


\subsection{Coproducts of Breuil--Kisin prisms}\label{Subsection:Coproducts of Breuil--Kisin prisms}

In this subsection, we establish some properties of the object
\[
(\mathfrak{S}_\O, (\mathcal{E})) \in (R)_{\Prism, \O_E}.
\]
If $\O_E=\Z_p$, $n=1$, and $R$ is $p$-torsion free,
then all the results in this subsection follow from (the arguments given in) \cite[Section 5.2]{Anschutz-LeBras}.
In the general case, we need some additional arguments, as we will see below.

We begin with the following result.

\begin{prop}\label{Proposition:weakely initial object}
For any $(A, I) \in (R)_{\Prism, \O_E}$, there exists a flat covering $(A, I) \to (A', I')$ in $(R)_{\Prism, \O_E}$ such that $(A', I')$ admits a morphism
$(\mathfrak{S}_\O, (\mathcal{E})) \to (A', I')$ in $(R)_{\Prism, \O_E}$.
\end{prop}

\begin{proof}
We may assume that $(A, I)$ is orientable by Remark \ref{Remark:flat locally orientable}.
Let $d \in I$ be a generator.
Let $v_1, \dotsc, v_n \in A$ be elements such that each $v_i$ is a lift of the image of
$t_i \in \mathfrak{S}_\O$ under the composition $\mathfrak{S}_\O \to R \to A/I$.
Let $B:=A \otimes_\O \mathfrak{S}_\O$.
We set
\[
x_i:=1\otimes t_i - v_i \otimes 1 \in B.
\]
Then the morphism
$
A/^\L(\pi, d) \to B/^\L(\pi, d, x_1, \dotsc, x_n)
$
of animated rings is faithfully flat.
%Indeed, using that $\O[t_1, \dotsc, t_n] \to \mathfrak{S}_\O$ is flat, we see that $A/^\L(\pi, d) \to B/^\L(\pi, d, x_1, \dotsc, x_n)$ is flat.
%Since the composition $\mathfrak{S}_\O \to R \to A/(\pi, d)$
%induces a homomorphism $B/(\pi, d, x_1, \dotsc, x_n) \to A/(\pi, d)$ over $A/(\pi, d)$,
%it follows that 
%$
%A/^\L(\pi, d) \to B/^\L(\pi, d, x_1, \dotsc, x_n)
%$
%is faithfully flat.

Let
$\mathfrak{S}_{\O, \infty}$ be the $(\pi, \mathcal{E})$-adic completion of a colimit
$\varinjlim_\phi \mathfrak{S}_\O$
of the diagram
\[
\mathfrak{S}_\O \overset{\phi}{\rightarrow} \mathfrak{S}_\O \overset{\phi}{\rightarrow} \mathfrak{S}_\O  \rightarrow \cdots.
\]
Since $\phi \colon \mathfrak{S}_\O \to \mathfrak{S}_\O$ is faithfully flat, we see that
$\mathfrak{S}_\O \to \mathfrak{S}_{\O, \infty}$ is $(\pi, \mathcal{E})$-completely faithfully flat.
In fact, it is faithfully flat by virtue of \cite[Theorem 1.5]{Yekutieli18}.
We set
$B':=B \otimes_{\mathfrak{S}_\O} \mathfrak{S}_{\O, \infty}$.
Let $J:=(d, x_1, \dotsc, x_n) \subset B'$ and
we consider the prismatic envelope
\[
(A', I'):=(B'\{ J/I \}^{\wedge}, IB'\{ J/I \}^{\wedge})
\]
of $B'$ over the bounded $\O_E$-prism $(A, I)$ with respect to the ideal $J$.
By Proposition \ref{Proposition:prismatic envelope}, we see that
$(A, I) \to (A', I')$
is a flat covering.

We shall construct a morphism $(\mathfrak{S}_\O, (\mathcal{E})) \to (A', I')$ in $(R)_{\Prism, \O_E}$.
We remark that since $A'/I'$ is not necessarily $\mathfrak{m}$-adically complete for the maximal ideal $\mathfrak{m} \subset R$, it is not clear that the natural homomorphism $\mathfrak{S}_\O \to A'$ induces a morphism $(\mathfrak{S}_\O, (\mathcal{E})) \to (A', I')$ in $(R)_{\Prism, \O_E}$.
Instead, we construct a morphism $(\mathfrak{S}_\O, (\mathcal{E})) \to (A', I')$ as follows.
Since
$\mathfrak{S}_{\O, \infty}$ can be identified with the $(\pi, \mathcal{E})$-adic completion of
\[
\cup_{m \geq 0} \mathfrak{S}_\O[t^{1/q^m}_1, \dotsc, t^{1/q^m}_n],
\]
the quotient
$R_{\infty}:=\mathfrak{S}_{\O, \infty}/\mathcal{E}$ is the $\pi$-adic completion of
$\cup_{m \geq 0} R[\overline{t}^{1/q^m}_1, \dotsc, \overline{t}^{1/q^m}_n]$, where
$\overline{t}_i \in R$ is the image of $t_i$.
Here
\[
\mathfrak{S}_\O[t^{1/q^m}_1, \dotsc, t^{1/q^m}_n]=\mathfrak{S}_\O[X_1, \dotsc, X_n]/(X^{q^m}_1-t_1, \dotsc, X^{q^m}_n-t_n)
\]
and similarly for $R[\overline{t}^{1/q^m}_1, \dotsc, \overline{t}^{1/q^m}_n]$.
The composition $R \to A/I \to A'/I'$ factors through the homomorphism
\[
g \colon R_\infty \to A'/I'
\]
defined by sending $\overline{t}^{1/q^m}_i$ to the image of $1 \otimes t^{1/q^m}_i \in A'$.
It follows from Lemma \ref{Lemma:maps from perfectoid prisms to prism} that
there exists a unique map
$(\mathfrak{S}_{\O, \infty}, (\mathcal{E})) \to (A', I')$
of bounded $\O_E$-prisms which induces $g$.
By construction, the composition
\[
(\mathfrak{S}_\O, (\mathcal{E})) \to (\mathfrak{S}_{\O, \infty}, (\mathcal{E})) \to (A', I')
\]
is a morphism in $(R)_{\Prism, \O_E}$.
\end{proof}

\begin{rem}\label{Remark:weakely initial object}
    Assume that $\O_E=\Z_p$.
    In this case, Proposition \ref{Proposition:weakely initial object} is proved in 
    \cite[Example 7.13]{BS} and \cite[Lemma 5.14]{Anschutz-LeBras}, using \cite[Lemma 7.11]{BS}.
    Moreover, if $\mathfrak{S}_\O=W(k)[[t]]$ and $\mathcal{E} \in W(k)[[t]]$ is an Eisenstein polynomial, then an alternative argument using prismatic envelopes is given in \cite[Example 2.6]{BS2}.
    Our argument is similar to the one given in \cite[Example 2.6]{BS2},
    but we have to modify it slightly in order to treat the case where $A/I$ is not $\mathfrak{m}$-adically complete.
\end{rem}

In a similar way, we obtain the following result:

\begin{lem}\label{Lemma:two morphisms from Breuil-Kisin coincide}
    Let $(A, I) \in (R)_{\Prism, \O_E}$ and let
    $f_1, f_2 \colon (\mathfrak{S}_\O, (\mathcal{E})) \to (A, I)$
    be two morphisms in $(R)_{\Prism, \O_E}$.
    If $f_1(t_i)=f_2(t_i)$ in $A$ for any $i$, then we have $f=g$.
\end{lem}

\begin{proof}
    As in the proof of Proposition \ref{Proposition:weakely initial object}, let $\mathfrak{S}_{\O, \infty}$ be the $(\pi, \mathcal{E})$-adic completion of 
    $\varinjlim_\phi \mathfrak{S}_\O$, which is faithfully flat over $\mathfrak{S}_\O$.
    After replacing $(A, I)$ by a flat covering, we may assume that
    $f_1 \colon (\mathfrak{S}_\O, (\mathcal{E})) \to (A, I)$
    factors through a map $\widetilde{f}_1 \colon (\mathfrak{S}_{\O, \infty}, (\mathcal{E})) \to (A, I)$ of $\O_E$-prisms.
    Since
    $\mathfrak{S}_{\O, \infty}$ is the $(\pi, \mathcal{E})$-adic completion of
    $
    \cup_{m \geq 0} \mathfrak{S}_\O[t^{1/q^m}_1, \dotsc, t^{1/q^m}_n],
    $
    there exists a map
    $\widetilde{f}_2 \colon \mathfrak{S}_{\O, \infty} \to A$
    extending $f_2$ such that
    \[
    \widetilde{f}_2(t^{1/q^m}_i)=\widetilde{f}_1(t^{1/q^m}_i)
    \]
    for all $m$ and $i$.
    It suffices to prove that $\widetilde{f}_1=\widetilde{f}_2$.
    We note that both $f_1$ and $f_2$ induce the same homomorphism
    $R \to A/I$.
    Since $R_{\infty}=\mathfrak{S}_{\O, \infty}/\mathcal{E}$ is the $\pi$-adic completion of
    $\cup_{m \geq 0} R[\overline{t}^{1/q^m}_1, \dotsc, \overline{t}^{1/q^m}_n]$,
    it follows that
    the homomorphism $R_{\infty} \to A/I$ induced by $\widetilde{f}_1$ agrees with the one given by $\widetilde{f}_2$.
    Then, by Lemma \ref{Lemma:maps from perfectoid prisms to prism}, we conclude that $\widetilde{f}_1=\widetilde{f}_2$.
\end{proof}


We study a coproduct of two copies of $(\mathfrak{S}_\O, (\mathcal{E}))$
in the category $(R)_{\Prism, \O_E}$.
To simplify the notation, we write
\[
(A, I):=(\mathfrak{S}_\O, (\mathcal{E}))
\]
in the rest of this section.
We set
\[
B:=A[[x_1, \dotsc, x_n]]
\]
and let $p'_1 \colon A \to B$ be the natural homomorphism.
There exists a unique $\delta_E$-structure on $B$ such that $p'_1$ is a homomorphism of $\delta_E$-rings and the associated Frobenius $\phi \colon B \to B$ sends $x_i$ to $(x_i+t_i)^q-t^q_i$ for every $i$.
We can consider the prismatic envelope
\[
(A^{(2)}, I^{(2)})
\]
of $B$ over $(A, I)$ with respect to the ideal $(\mathcal{E}, x_1, \dotsc, x_n) \subset B$
as in Proposition \ref{Proposition:prismatic envelope}.
Let
$p_1 \colon (A, I) \to (A^{(2)}, I^{(2)})$ denote the natural map.
We view $(A^{(2)}, I^{(2)})$ as an object of $(R)_{\Prism, \O_E}$ via 
the homomorphism $\overline{p}_1 \colon R \to A^{(2)}/I^{(2)}$ induced by $p_1$.

The homomorphism $p'_2 \colon A \to B$ over $\O$ defined by $t_i \mapsto x_i+t_i$ is a homomorphism of $\delta_E$-rings.
Let $p_2 \colon A \to A^{(2)}$ denote the composition of $p'_2$ with the natural homomorphism $B \to A^{(2)}$.

\begin{lem}\label{Lemma:coproduct and prismatic envelope}
Let the notation be as above.
\begin{enumerate}
    \item We have $p_2(I) \subset I^{(2)}$, and the induced map
    $p_2 \colon (A, I) \to (A^{(2)}, I^{(2)})$ is a morphism in $(R)_{\Prism, \O_E}$.
    \item The object $(A^{(2)}, I^{(2)}) \in (R)_{\Prism, \O_E}$ with the morphisms
    $p_1, p_2 \colon (A, I) \to (A^{(2)}, I^{(2)})$ is a coproduct of two copies of $(A, I)$ in the category $(R)_{\Prism, \O_E}$.
\end{enumerate}
\end{lem}

\begin{proof}
    (1) It suffices to show that the composition of $p_2 \colon A \to A^{(2)}$ with $A^{(2)} \to A^{(2)}/I^{(2)}$ coincides with the composition of $A \to R$ with $\overline{p}_1 \colon R \to A^{(2)}/I^{(2)}$.
    For any $h \in A$, the element $p'_2(h)-p'_1(h) \in B$ is contained in the ideal $(x_1, \dotsc, x_n) \subset B$.
    Since the image of $x_i$ in $A^{(2)}$ is contained in $I^{(2)}$, we have
    $p_2(h)-p_1(h) \in I^{(2)}$, which implies the assertion.

    (2) We have to show that for any $(A', I') \in (R)_{\Prism, \O_E}$ and two morphisms $f_1, f_2 \colon (A, I) \to (A', I')$ in $(R)_{\Prism, \O_E}$, there exists a unique morphism
    \[
    f \colon (A^{(2)}, I^{(2)}) \to (A', I')
    \]
    in $(R)_{\Prism, \O_E}$ such that $f \circ p_1=f_1$ and $f \circ p_2=f_2$.

    We first prove the uniqueness of $f$.
    Let $f' \colon B \to A'$ be the composition of $f$ with $B \to A^{(2)}$.
    Then, we have $f' \circ p'_j=f_j$ ($j=1, 2$), and $f'$ sends $x_i=p'_2(t_i)-p'_1(t_i)$ to
    \[
    f_2(t_i)-f_1(t_i) \in I' \subset A'
    \]
    for any $i$.
    Since $A'$ is $I'$-adically complete, such a homomorphism $f'$ of $\delta_E$-rings is uniquely determined (if it exists).
    The uniqueness of $f$ now follows from the universal property of the prismatic envelope $(A^{(2)}, I^{(2)})$.

    We next prove the existence of $f$.
    Since $f_2(t_i)-f_1(t_i) \in I' \subset A'$ and $A'$ is $I'$-adically complete, there exists a unique homomorphism
    $f' \colon B \to A'$ over $\O$ such that $f' \circ p'_1=f_1$ and $f'(x_i)=f_2(t_i)-f_1(t_i)$ for every $i$.
    
    We claim that $f'$ is a homomorphism of $\delta_E$-rings.
    Indeed,
    as in the proof of Proposition \ref{Proposition:weakely initial object}, let $A_\infty$ be the $(\pi, \mathcal{E})$-adic completion of 
    $\varinjlim_\phi A$, which is faithfully flat over $A$.
    After replacing $(A', I')$ by a flat covering,
    we may assume that $f_j$ factors through a morphism
    $\widetilde{f}_j \colon (A_\infty, IA_\infty) \to (A', I')$ in $(R)_{\Prism, \O_E}$ for each $j=1, 2$.
    For an integer $m \geq 0$ and $i$, we set
    \[
    x_{i, m}:=\widetilde{f}_2(t^{1/q^m}_i)-\widetilde{f}_1(t^{1/q^m}_i) \in A'.
    \]
    Since we have $x^{q^m}_{i, m} \in (\pi, I')$, it follows that $A'$ is $(x_{1, m}, \dotsc, x_{n, m})$-adically complete for every $m$.
    Thus, for each $m \geq 0$, there exists a unique homomorphism
    $f'(m) \colon B \to A'$
    such that $f'(m) \circ p'_1$ is the composition
    \[
    A \to A_\infty \overset{\phi^{-m}}{\to} A_\infty \overset{\widetilde{f}_1}{\to} A'
    \]
    and we have $f'(m)(x_i)=x_{i, m}$ for any $i$.
    Since $f'(m)=f'(m+1) \circ \phi$, they give rise to a homomorphism
    \[
    \widetilde{f}' \colon \varinjlim_\phi B \to A'.
    \]
    It follows from Corollary \ref{Corollary:homomorphism from perfect delta pi-ring} that $\widetilde{f}'$ is a homomorphism of $\delta_E$-rings.
    Since $f'$ is the composition $B \to \varinjlim_\phi B \to A'$, we conclude that $f'$ is a homomorphism of $\delta_E$-rings as well.
    
    By the universal property of the prismatic envelope $(A^{(2)}, I^{(2)})$,
    the homomorphism $f'$ extends to a unique morphism
    $f \colon (A^{(2)}, I^{(2)}) \to (A', I')$ in $(R)_{\Prism, \O_E}$.
    By construction, we have $f \circ p_1=f_1$.
    It follows from Lemma \ref{Lemma:two morphisms from Breuil-Kisin coincide} that $f \circ p_2=f_2$.

    The proof of Lemma \ref{Lemma:coproduct and prismatic envelope} is complete.
\end{proof}

Let
\[
m \colon (A^{(2)}, I^{(2)}) \to (A, I)
\]
be the unique morphism in $(R)_{\Prism, \O_E}$ such that $m \circ p_1 = m \circ p_2= \id_{(A, I)}$.
Let $K$ be the kernel of $m \colon A^{(2)} \to A$.
Let $d \in I^{(2)}$ be a generator.

\begin{lem}[{cf.\ \cite[Lemma 5.15]{Anschutz-LeBras}}]\label{Lemma:kernel of two fold coproduct I}
    We have $\phi(K) \subset  dK$.
\end{lem}

\begin{proof}
    To prove that $\phi(K) \subset  dK$, it suffices to show that $\phi(K) \subset  dA^{(2)}$.
    Indeed, let $x \in K$, and we assume that $\phi(x)=dy$ for some $y \in A^{(2)}$.
    Then, since $m(\phi(x))=0$ and $m(d) \in A$ is a nonzerodivisor,
    we have $y \in K$.

    We shall prove that $\phi(K) \subset  dA^{(2)}$.
    We may assume that $d=p_1(\mathcal{E})$.
    The image $p'_1(\mathcal{E}) \in B$ is also denoted by $d$.
    It follows from Proposition \ref{Proposition:prismatic envelope} that $A^{(2)}$ can be identified with the $(\pi, d)$-adic completion of
    \[
    C:=B\{ x_1/d, \dotsc, x_n/d \}.
    \]
    We write $y_i:=x_i/d$.
    The composition $C \to A^{(2)} \to A$ sends $\delta^j_E(y_i)$ to $0$ for any $j \geq 0$ and any $i$.
    Here $\delta^j_E$ is the $j$-th iterate of the map $\delta_E \colon C \to C$.
    Since the kernel of the homomorphism $B \to A$ defined by $x_i \mapsto 0$ ($1 \leq i \leq n$) coincides with $(x_1, \dotsc, x_n)$, it follows that
    the kernel $K_0$ of $C \to A$ is generated by
    \[
    \{ \delta^j_E(y_i) \}_{1 \leq i \leq n, j \geq 0}.
    \]
    Moreover $K$ can be identified with the $(\pi, d)$-adic completion of $K_0$.
    Since we have $dA^{(2)} = \cap_{l \geq 0}(dA^{(2)} + (\pi, d)^lA^{(2)})$,
    it then suffices to show that
    $\phi(\delta^j_E(y_i)) \in dA^{(2)}$ for any $j \geq 0$ and any $i$.
    This can be proved by the same argument as in the proof of \cite[Lemma 5.15]{Anschutz-LeBras} when $\mathcal{E} \in A$ is not contained in $\pi A$.
    In fact,
    a similar argument holds when $\mathcal{E} \in \pi A$.
    We include the argument in this case for the convenience of the reader.

    We may assume that $\mathcal{E}=\pi$.
    In fact, we prove a more general statement: For any $j \geq 0$, we have
    $\phi^l(\delta^j_E(y_i)) \in \pi^{l} A^{(2)}$ for any $l \geq 1$ and any $i$.
    We proceed by induction on $j$.
    Let $u_i:=x_i+t_i=\pi y_i + t_i \in A^{(2)}$.
    Then, we have
    \[
    \phi^l(x_i)=u^{q^l}_i-t^{q^l}_i= \sum_{0 \leq h \leq q^l -1} \dbinom{q^l}{h} (\pi y_i)^{q^l-h} t^{h}_i
    \in \pi^{l+1} A^{(2)}.
    \]
    Thus, we obtain
    $\phi^l(y_i) \in \pi^{l} A^{(2)}$, which proves the assertion in the case where $j=0$.
    Suppose that the assertion holds for some $j \geq 0$.
    Since
    \[
    \pi \phi^l(\delta^{j+1}_E(y_i)) = \phi^l(\pi \delta^{j+1}_E(y_i)) = \phi^l(\phi(\delta^{j}_E(y_i))-\delta^{j}_E(y_i)^{q})
    = \phi^{l+1}(\delta^{j}_E(y_i))-\phi^l(\delta^{j}_E(y_i))^q,
    \]
    the induction hypothesis implies that $\pi \phi^l(\delta^{j+1}_E(y_i)) \in \pi^{l+1} A^{(2)}$, whence $\phi^l(\delta^{j+1}_E(y_i)) \in \pi^l A^{(2)}$.
\end{proof}

Our next goal is to prove the following lemma, which plays a crucial role in the proof of Theorem \ref{Theorem:main result on G displays over complete regular local rings} (especially in the proof of Proposition \ref{Proposition:deformation of isomorphism} below).
As in the proof of Lemma \ref{Lemma:kernel of two fold coproduct I},
we set $y_i:=x_i/d \in A^{(2)}$.

\begin{lem}\label{Lemma:kernel of two fold coproduct refined}
        Let
        $M \subset A^{(2)}$
        be the ideal generated by
        $\phi(y_1)/d, \dotsc, \phi(y_n)/d \in K$.
        Then we have inclusions
        \[
        \phi(K) \subset dM + d(\pi, d)K \quad \text{and} \quad \phi(M) \subset d(t_1, \dotsc, t_n)M + d(\pi, d)K.
        \]
\end{lem}

\begin{rem}\label{Remark:compare with Anschutz-Le Bras}
    Assume that $\O_E=\Z_p$.
    Under the assumption that $n=1$ and $R$ is $p$-torsion free,
    Ansch\"utz--Le Bras gave a proof of the analogue of Theorem \ref{Theorem:main result on G displays over complete regular local rings} for minuscule Breuil--Kisin modules in \cite[Section 5.2]{Anschutz-LeBras}.
    (We will come back to this result in Section \ref{Subsection:A remark on prismatic Dieudonne crystals}.)
    In the proof, they use that the map $K \to K$, $x \mapsto \phi(x)/d$
    is topologically nilpotent with respect to the $(p, d)$-adic topology (\cite[Lemma 5.15]{Anschutz-LeBras}).
    This topological nilpotence may not be true if $n \geq 2$ or $p=0$ in $R$.
    We will use Lemma \ref{Lemma:kernel of two fold coproduct refined} and the fact that the local ring $A$ is complete and noetherian to overcome this issue; see Section \ref{Subsection:Deformations of isomorphisms} below.
\end{rem}

The proof of Lemma \ref{Lemma:kernel of two fold coproduct refined} will require some preliminary results.
We first introduce some notation.


If $\mathcal{E}$ is not contained in $\pi A$,
then the $\delta_E$-ring
$A \{ \phi(\mathcal{E}) / \pi \}$
is $\pi$-torsion free, and is isomorphic to
the
$\O_E$-PD envelope
$D_{(\mathcal{E})}(A)$
of $A$ with respect to the ideal $(\mathcal{E})$; see Lemma \ref{Lemma:pd envelope and delta structure}.
In this case, let $A''$ be the $\pi$-adic completion of
$A \{ \phi(\mathcal{E}) / \pi \}$, and let $g \colon A \to A''$ be the natural homomorphism.
We note that $A''$ is also $\pi$-torsion free.
We consider the following pushout squares of $\delta_E$-rings:
\[
\xymatrix{
A \ar^-{p'_1}[r]  \ar[d]_-{\phi} & B \ar^-{}[r] \ar[d]_-{} & A^{(2)} \ar^-{m}[r] \ar[d]_-{} & A \ar[d]_-{\phi} \\
A \ar[r]^-{} \ar[d]_-{g} & B' \ar^-{}[r] \ar[d]_-{} & A^{(2)'} \ar^-{}[r] \ar[d]_-{} & A \ar[d]_-{g} \\
A'' \ar[r]^-{} & B''_0 \ar^-{}[r] & A^{(2)''}_0 \ar^-{}[r] & A''.
}
\]
Let $A^{(2)''}$ be the $\pi$-adic completion
of $A^{(2)''}_0$ and $K''$ the kernel of
the induced homomorphism $A^{(2)''} \to A''$.
Since $A \to A^{(2)}$ is flat,
so is
$A'' \to A^{(2)''}_0$.
In particular $A^{(2)''}$ is $\pi$-torsion free.
In the case where $\mathcal{E} \in \pi A$, we set $A^{(2)''}:= A^{(2)}$ and $K'':=K$.


\begin{lem}\label{Lemma:K'' case I}
Let the notation be as above. Then the following assertions hold:
\begin{enumerate}
    \item We have $\phi(K'') \subset \pi K''$.
    \item 
    We have $x_i \in \pi K''$ for any $1 \leq i \leq n$.
    (Here we denote the image of $x_i \in B$ in $A^{(2)''}$ again by $x_i$.)
    We set $w_i:=x_i/\pi \in K''$.
    Then $K''/\pi K''$ is generated by the images of
    $\{ \delta^j_E(w_i) \}_{1 \leq i \leq n, j \geq 0}$ as an $A^{(2)''}$-module.
\end{enumerate}
\end{lem}

\begin{proof}
    If $\mathcal{E} \in \pi A$, then the assertions follow from Lemma \ref{Lemma:kernel of two fold coproduct I} and its proof.
    Thus, we may assume that $\mathcal{E}$ is not contained in $\pi A$.
    Let $h \colon A^{(2)} \to A^{(2)''}$ denote the natural homomorphism.
    Since $g(\phi(\mathcal{E}))/\pi \in A''$ is a unit by Lemma \ref{Lemma:distinguished} (1),
    it follows that $h(d) \in A^{(2)''}$ is a unit multiple of $\pi$.
    The kernel $K''$ of $A^{(2)''} \to A''$ can be identified with
    the $\pi$-adic completion of $h^*K$.
    Therefore, the assertion (1) follows from Lemma \ref{Lemma:kernel of two fold coproduct I}.

    Using that $h(d) \in A^{(2)''}$ is a unit multiple of $\pi$, we see that
    $A^{(2)''}$ agrees with the $\pi$-adic completion of
    $B''_0\{ x_1/\pi, \dotsc, x_n/\pi \}$.
    Since the kernel of $B''_0 \to A''$ is generated by $x_1, \dotsc, x_n$,
    it follows that the kernel of
    $B''_0\{ x_1/\pi, \dotsc, x_n/\pi \} \to A''$
    is generated by $\{ \delta^j_E(x_i/\pi) \}_{1 \leq i \leq n, j \geq 0}$, which implies the assertion (2).
\end{proof}




\begin{lem}\label{Lemma:K'' case II}
We define
\[
\phi_1 \colon K'' \to K'', \quad x \mapsto \phi(x)/\pi.
\]
The induced $\phi$-linear homomorphism
$K''/\pi K'' \to K''/ \pi K''$
is denoted by the same symbol $\phi_1$.
Let
$\overline{M}'' \subset K''/\pi K''$
be the
$A^{(2)''}$-submodule generated by the images of
$
\phi_1(w_1), \dotsc, \phi_1(w_n) \in K''.
$
Then we have inclusions
\[
\phi_1(K''/\pi K'') \subset \overline{M}'' \quad \text{and} \quad \phi_1(\overline{M}'') \subset (t_1, \dotsc, t_n)\overline{M}'',
\]
where we denote the image of $t_i \in A^{(2)}$ in $A^{(2)''}$ again by $t_i$.
\end{lem}

\begin{proof}
We have
$x^q=\phi(x)- \pi \delta_E(x) \in \pi K''$
for every $x \in K''$ by Lemma \ref{Lemma:K'' case I}.
Let $J \subset A^{(2)''}$ be the ideal generated by $\{ x^q/\pi \}_{x \in K''}$.
We claim that $\phi_1(J) \subset \pi J$.
Indeed, for any $x \in K''$, we have
\[
\phi_1(x^q/\pi)= \phi(x^q/\pi)/\pi = \phi(x)^q/\pi^2 = \pi^{q-1} (\phi_1(x)^q/\pi) \in \pi J.
\]

We shall prove that
$K''/(J+ \pi K'')$
is generated by the images of $w_1, \dotsc, w_n$ as an $A^{(2)''}$-module.
Since
\[
\phi_1(w_i)=\phi(x_i)/\pi^2=((x_i+t_i)^q-t^q_i)/\pi^2=((\pi w_i+t_i)^q-t^q_i)/\pi^2,
\]
we can write $\phi_1(w_i)$ as
\begin{equation}\label{equation:formula phi_1}
    \phi_1(w_i)= \pi^{q-2} w^{q}_i + (q/\pi)t^{q-1}_i w_i + \pi b_i
\end{equation}
for some element $b_i \in K''$.
For any $x \in K''$, we have
$\delta_E(x)=\phi_1(x)$
in $K''/J$.
In particular, we obtain
$\delta^j_E(x)=\phi_1(\delta^{j-1}_E(x))$
in $K''/J$.
Then,
using (\ref{equation:formula phi_1})
and by induction on $j \geq 1$,
we see that
the image of
$\delta^j_E(w_i)$
in $K''/(J+ \pi K'')$
is contained in
the 
$A^{(2)''}$-submodule of $K''/(J+ \pi K'')$ generated by the images of $w_1, \dotsc, w_n$ for any $j \geq 1$ and any $i$.
Now the assertion follows from Lemma \ref{Lemma:K'' case I}.

We have shown that every $x \in K''$ can be written as
\[
x= (\sum_{1 \leq i \leq n} a_i w_i) + b + \pi c
\]
for some $a_i \in A^{(2)''}$ ($1 \leq i \leq n$), $b \in J$, and $c \in K''$.
Since $\phi_1(b) \in \pi J$,
it follows that
the image
of $\phi_1(x)$ in $K''/\pi K''$
coincides with that of 
$\sum_{1 \leq i \leq n} \phi(a_i) \phi_1(w_i)$.
This proves that
$\phi_1(K''/\pi K'') \subset \overline{M}''$.
Moreover, since
$\phi_1(w^q_i)=\phi(w_i)\phi_1(w^{q-1}_i)$
is contained in $\pi K''$,
it follows from (\ref{equation:formula phi_1}) that
the image
of $\phi_1(\phi_1(w_i))$ in $K''/\pi K''$
is equal to
that of
$\phi_1((q/\pi)t^{q-1}_i w_i)=(q/\pi)t^{q(q-1)}_i \phi_1(w_i)$.
This proves that
$\phi_1(\overline{M}'') \subset (t_1, \dotsc, t_n)\overline{M}''$.
\end{proof}

We now prove Lemma \ref{Lemma:kernel of two fold coproduct refined}.

\begin{proof}[Proof of Lemma \ref{Lemma:kernel of two fold coproduct refined}]
    We first treat the case where $\mathcal{E} \in \pi A$.
    In this case $d$ is a unit multiple of $\pi$.
    Thus, the assertion follows from Lemma \ref{Lemma:K'' case II}.
    

    We now assume that $\mathcal{E}$ is not contained in $\pi A$.
    We define
    \[
    \phi_1 \colon K \to K, \quad x \mapsto \phi(x)/d.
    \]
    The induced $\phi$-linear homomorphism
    $K/(\pi, d)K \to K/(\pi, d)K$
    is also denoted by $\phi_1$.
    Let
    $
    \overline{M} \subset K/(\pi, d)K
    $
    be the $A^{(2)}$-submodule generated by the images of
    $
    \phi_1(y_1), \dotsc, \phi_1(y_n) \in K.
    $
    It suffices to prove that
    $\phi_1(K/(\pi, d)K) \subset \overline{M}$
    and
    $\phi_1(\overline{M}) \subset (t_1, \dotsc, t_n)\overline{M}$.

    Let $f \colon A^{(2)} \to A^{(2)'}$ denote the natural homomorphism.
    Let $K'$ be the kernel of the homomorphism $A^{(2)'} \to A$, which can be identified with $f^*K$.
    We define $\phi'_1 \colon K' \to K'$ by $x \mapsto \phi(x)/f(d)$, and
    let
    $
    \overline{M}' \subset K'/(\pi, f(d))K'
    $
    be the $A^{(2)'}$-submodule generated by the images of
    $
    \phi'_1(f(y_1)), \dotsc, \phi'_1(f(y_n)) \in K'.
    $
    Since $\phi \colon A \to A$ is faithfully flat, so is $f$.
    Therefore, in order to prove the assertion, it is enough to prove that
    \begin{equation}\label{equation:goal}
        \phi'_1(K'/(\pi, f(d))K') \subset \overline{M}' \quad \text{and} \quad \phi'_1(\overline{M}') \subset (f(t_1), \dotsc, f(t_n))\overline{M}'.
    \end{equation}
    
    
    The homomorphism $A^{(2)'} \to A^{(2)''}$
    induced by $g \colon A \to A''$ is again denoted by $g$.
    The element $g(f(d))$ is a unit multiple of $\pi$ in $A^{(2)''}$.
    Thus, for $\phi_1 \colon K'' \to K''$ defined in Lemma \ref{Lemma:K'' case II},
    the element $g(\phi'_1(x))$ is a unit multiple of
    $\phi_1(g(x))$ for any $x \in K'$.
    Also, the induced homomorphism
    $A^{(2)'}/(\pi, f(d)) \to A^{(2)''}/\pi$,
    again denoted by $g$, sends $\overline{M}'$ into $\overline{M}''$.
    It follows from Lemma \ref{Lemma:K'' case II} that, for any
    $x \in K'/(\pi, f(d))K'$
    (resp.\ $x \in \overline{M}'$),
    we have
    \begin{equation}\label{equation:key inclusions phi'_1}
        g(\phi'_1(x)) \in \overline{M}''
    \quad (\text{resp.\ } g(\phi'_1(x)) \in (g(f(t_1)), \dotsc, g(f(t_n)))\overline{M}'').
    \end{equation}

    Since $A''/\pi \simeq D_{(\mathcal{E})}(A)/\pi$, 
    we can find a homomorphism
    \[
    s \colon A''/\pi \to A/(\pi, \phi(\mathcal{E}))
    \]
    of $\O_E$-algebras such that
    the composition
    $A/(\pi, \phi(\mathcal{E})) \overset{g}{\to} A''/\pi \overset{s}{\to} A/(\pi, \phi(\mathcal{E}))$
    is the identity; see Example \ref{Example:pi PD polynomial ring} and Lemma \ref{Lemma:pd envelope regular sequence}.
    We consider the following pushout squares of $\O_E$-algebras:
    \[
    \xymatrix{
    A''/\pi \ar^-{}[r]  \ar[d]_-{s} & A^{(2)''}/\pi \ar^-{}[r] \ar[d]_-{\widetilde{s}} & A''/\pi  \ar[d]_-{s}  \\
    A/(\pi, \phi(\mathcal{E})) \ar[r]^-{}  & A^{(2)'}/(\pi, f(d)) \ar^-{}[r]  & A/(\pi, \phi(\mathcal{E})).
    }
    \]
    The homomorphism $g \colon A^{(2)'}/(\pi, f(d)) \to A^{(2)''}/\pi$
    is a section of $\widetilde{s}$.
    We observe that $\widetilde{s}(K''/\pi K'') \subset K'/(\pi, f(d))K'$ and $\widetilde{s}(\overline{M}'') \subset \overline{M}'$.
    It follows from (\ref{equation:key inclusions phi'_1}) that,
    for any
    $x \in K'/(\pi, f(d))K'$
    (resp.\ $x \in \overline{M}'$),
    its image $\phi'_1(x)=\widetilde{s}(g(\phi'_1(x)))$
    belongs to $\overline{M}'$ (resp.\ $(f(t_1), \dotsc, f(t_n))\overline{M}'$).
    This proves (\ref{equation:goal}),
    and the proof of Lemma \ref{Lemma:kernel of two fold coproduct refined} is now complete.
\end{proof}




\begin{rem}\label{Remark:three coproduct}
    There exists a coproduct $(A^{(3)}, I^{(3)})$ of three copies of $(A, I)$
    in the category $(R)_{\Prism, \O_E}$.
    Indeed, one can define $(A^{(3)}, I^{(3)})$ as a pushout of the diagram
    \[
    (A^{(2)}, I^{(2)}) \overset{p_2}{\leftarrow} (A, I) \overset{p_1}{\rightarrow} (A^{(2)}, I^{(2)}),
    \]
    which exists since $p_1$ is flat (see Remark \ref{Remark:pushout in prismatic site}).
    Let $q_1, q_2, q_3 \colon (A, I) \to (A^{(3)}, I^{(3)})$ denote the associated three morphisms.
    For $1 \leq i < j \leq 3$, let
    $p_{ij} \colon (A^{(2)}, I^{(2)}) \to (A^{(3)}, I^{(3)})$ be the unique morphism such that $p_{ij} \circ p_1=q_i$ and $p_{ij} \circ p_2=q_j$.
\end{rem}

Let $m \colon (A^{(3)}, I^{(3)}) \to (A, I)$ be the unique morphism in $(R)_{\Prism, \O_E}$ such that $m \circ q_i = \id_{(A, I)}$ for $i=1, 2, 3$.
    
\begin{cor}\label{Corollary:kernel of three fold coproduct}
    Let $L$ be the kernel of $m \colon A^{(3)} \to A$.
    Let $d \in I^{(3)}$ be a generator.
    Then the following assertions hold:
    \begin{enumerate}
        \item We have $\phi(L) \subset dL$.
        \item
        Let
        $N \subset A^{(3)}$
        be the ideal generated by
        $
        \{ \phi(p_{12}(y_l))/d, \phi(p_{23}(y_l))/d \}_{1 \leq l \leq n} \subset L.
        $
        Then we have inclusions
        \[
        \phi(L) \subset dN + d(\pi, d)L \quad \text{and} \quad \phi(N) \subset d(q_1(t_1), \dotsc, q_1(t_n))N + d(\pi, d)L.
        \]
    \end{enumerate}
\end{cor}

\begin{proof}
    We may assume that $d$ is the image of a generator of $I^{(2)}$, again denoted by $d$, under the homomorphism $p_{12}$.
    As in Remark \ref{Remark:three coproduct}, we identify
    $A^{(3)}$
    with
    the $(\pi, d)$-adic completion of
    the pushout $A^{(3)}_0:= A^{(2)} \otimes_{p_2, A, p_1} A^{(2)}$
    of the diagram
    \[
    A^{(2)} \overset{p_2}{\leftarrow} A \overset{p_1}{\rightarrow} A^{(2)}.
    \]
    Under this identification,
    the homomorphism $p_{12}$ (resp.\ $p_{23}$)
    is induced by 
    the homomorphism $A^{(2)} \to A^{(3)}_0$ defined by $a \mapsto a \otimes 1$ (resp.\ $a \mapsto 1 \otimes a$).
    The kernel $L_0$ of the natural homomorphism
    $A^{(3)}_0 \to A$
    coincides with
    $K \otimes_A A^{(2)} + A^{(2)} \otimes_A K$, and $L$ is the $(\pi, d)$-adic completion of $L_0$.

    In order to prove the assertion (1), it suffices to show that, for any element $x \in L$ which lies in the image of $L_0 \to L$, we have
    $\phi(x) \in d A^{(3)}$.
    This follows from Lemma \ref{Lemma:kernel of two fold coproduct I}.
    Similarly, the assertion (2) follows from Lemma \ref{Lemma:kernel of two fold coproduct refined}.
    We note here that, since
    $q_j(t_l)-q_i(t_l)=p_{ij}(x_l) \in d L$ for $1 \leq i < j \leq 3$,
    the ideal
    $d(q_1(t_1), \dotsc, q_1(t_n))N + d(\pi, d)L$
    is unchanged if we replace $q_1$ by $q_i$ ($1 \leq i \leq 3$).
\end{proof}

\subsection{Deformations of isomorphisms}\label{Subsection:Deformations of isomorphisms}

As in Section \ref{Subsection:Coproducts of Breuil--Kisin prisms},
we write $(A, I)=(\mathfrak{S}_\O, (\mathcal{E}))$.
In this subsection, as a preparation for the proof of Theorem \ref{Theorem:main result on G displays over complete regular local rings}, we study deformations of isomorphisms of $G$-$\mu$-displays over $(A, I)$ along the morphisms
$m \colon (A^{(2)}, I^{(2)}) \to (A, I)$ and $m \colon (A^{(3)}, I^{(3)}) \to (A, I)$ in $(R)_{\Prism, \O_E}$ defined in Section \ref{Subsection:Coproducts of Breuil--Kisin prisms}.
Throughout this subsection, we assume that $\mu$ is 1-bounded.

Our setup is as follows.
Let
$(A', I'):=(A^{(2)}, I^{(2)})$
(resp.\ $(A', I'):=(A^{(3)}, I^{(3)})$).
Let
$m \colon (A', I') \to (A, I)$
denote
$m \colon (A^{(2)}, I^{(2)}) \to (A, I)$
(resp.\ $m \colon (A^{(3)}, I^{(3)}) \to (A, I)$).
Let $f_1, f_2 \in \{ p_1, p_2 \}$
(resp.\ $f_1, f_2 \in \{ q_1, q_2, q_3 \}$).
We do not exclude the case where $f_1=f_2$.

The purpose of this subsection is to prove the following result:

\begin{prop}\label{Proposition:deformation of isomorphism}
Assume that $\mu$ is 1-bounded.
    Let $\mathcal{Q}_1$ and $\mathcal{Q}_2$ be $G$-$\mu$-displays over $(A, I)$.
    Then
    the map
    \begin{equation}\label{equation:reduction map for isomorphisms}
        m^{*} \colon \Hom_{G\mathchar`-\mathrm{Disp}_\mu(A', I')}(f^*_1(\mathcal{Q}_1), f^*_2(\mathcal{Q}_2)) \to \Hom_{G\mathchar`-\mathrm{Disp}_\mu(A, I)}(\mathcal{Q}_1, \mathcal{Q}_2)
    \end{equation}
    induced by the base change functor
    $m^* \colon G\mathchar`-\mathrm{Disp}_\mu(A', I') \to G\mathchar`-\mathrm{Disp}_\mu(A, I)$ is bijective.
\end{prop}

We need some preliminary results before giving the proof of Proposition \ref{Proposition:deformation of isomorphism}.
We will use the following notation.
Let $H$ be a group scheme over $\O$.
For an ideal $J \subset A'$,
we write
\[
H(J):= \Ker (H(A') \to H(A'/J))
\]
for the kernel of the homomorphism $H(A') \to H(A'/J)$.
If $H=G_\O$, then we simply write $G(J):=G_\O(J)$.


Let $K$ denote the kernel of $m \colon A' \to A$.
Let $d \in I'$ be a generator.

\begin{lem}\label{Lemma:stability of G(J)}
    Let $J \subset A'$ be an ideal such that
    $J \subset dK$ and, for any $x \in J$, we have $\phi(x/d) \in J$.
    Then, the homomorphism
    $\sigma_{\mu, d} \colon G_\mu(A', I') \to G(A')$
    (see (\ref{equation:sigma map of sets}))
    sends $G(J) \subset G_\mu(A', I')$ into itself.
\end{lem}

\begin{proof}
    We note that, by Proposition \ref{Proposition:BB isomorphism}, we have
    $G(J) \subset G(dK) \subset G_\mu(A', I')$, and
    the multiplication map
    $U_{-\mu} \times_{\Spec \O} P_\mu \to G_\O$
    induces a bijection
    \[
    (\Lie(U_{-\mu}) \otimes_\O J)
    \times P_\mu(J) \overset{\sim}{\to} G(J).
    \]
    Thus, it suffices to prove that
    $\sigma_{\mu, d}(P_\mu(J)) \subset G(J)$ and 
    $\sigma_{\mu, d}(\Lie(U_{-\mu}) \otimes_\O J) \subset G(J)$.

    We first prove that $\sigma_{\mu, d}(P_\mu(J)) \subset G(J)$.
    By Remark \ref{Remark:formula action of cocharacter} and Lemma \ref{Lemma:Pmu structure} (1), we have
    $\mu(d)P_\mu(J)\mu(d)^{-1} \subset P_\mu(J)$.
    (In fact, this holds for any ideal $J \subset A'$.)
    Since $\phi(J) \subset J$, we have $\phi(G(J)) \subset G(J)$.
    It follows that $\sigma_{\mu, d}(P_\mu(J)) \subset G(J)$.

    We shall show that $\sigma_{\mu, d}(\Lie(U_{-\mu}) \otimes_\O J) \subset G(J)$.
    Since $\mu$ is 1-bounded, the homomorphism
    $G_\mu(A', I') \to G(A')$, $g \mapsto \mu(d)g\mu(d)^{-1}$ restricts to a homomorphism
    \[
    \Lie(U_{-\mu}) \otimes_\O J \to \Lie(U_{-\mu}) \otimes_\O \frac{1}{d} J, \quad v \mapsto v/d.
    \]
    (See Remark \ref{Remark:Umu identify}.)
    Since $\phi((1/d)J) \subset J$, we can conclude that
    $\sigma_{\mu, d}(\Lie(U_{-\mu}) \otimes_\O J) \subset G(J)$.
\end{proof}

\begin{defn}\label{Definition:UdX FdX}
    Let $J \subset A'$ be an ideal as in Lemma \ref{Lemma:stability of G(J)}. For an element $X \in G(A')$,
we define a homomorphism
\[
\mathcal{U}_{d, X} \colon G(J) \to  G(J), \quad g \mapsto X\sigma_{\mu, d}(g)X^{-1}.
\]
We also define a map of sets
\[
\mathcal{V}_{d, X} \colon G(J) \to  G(J), \quad g \mapsto \mathcal{U}_{d, X}(g)g^{-1}.
\]
\end{defn}

Let $J_2 \subset J_1 \subset A'$ be two ideals which satisfy the assumption of Lemma \ref{Lemma:stability of G(J)}.
Then $\mathcal{U}_{d, X} \colon G(J_1) \to G(J_1)$ induces a homomorphism
\[
G(J_1)/G(J_2) \to G(J_1)/G(J_2),
\]
which we denote by the same symbol $\mathcal{U}_{d, X}$.
Let
$
\mathcal{V}_{d, X} \colon G(J_1)/G(J_2) \to G(J_1)/G(J_2)
$
be the map of sets defined by
$g \mapsto  \mathcal{U}_{d, X}(g)g^{-1}$.

By Lemma \ref{Lemma:kernel of two fold coproduct I} and Corollary \ref{Corollary:kernel of three fold coproduct}, we have
$\phi(K) \subset dK$.
Thus, the ideal $dK \subset A'$ satisfies the assumption of Lemma \ref{Lemma:stability of G(J)}.
We shall prove
(in Proposition \ref{Proposition:bijectivity of VdX for G(dK)} below)
that
$
\mathcal{V}_{d, X} \colon G(dK) \to  G(dK)
$
is bijective for any $X \in G(A')$, from which we will deduce Proposition \ref{Proposition:deformation of isomorphism}.
For this purpose, we need the following lemmas.

\begin{lem}\label{Lemma:square zero case}
 Let $J_2 \subset J_1 \subset A'$ be two ideals which satisfy the assumption of Lemma \ref{Lemma:stability of G(J)}.
 Assume that for any $x \in J_1$, we have $\phi(x/d) \in J_2$.
 Then, we have
 \[
 \sigma_{\mu, d}(G(J_1)) \subset G(J_2).
 \]
 In particular, the map
 $
\mathcal{V}_{d, X} \colon G(J_1)/G(J_2) \to G(J_1)/G(J_2)
 $
 is equal to the map $g \mapsto g^{-1}$ for any $X \in G(A')$.
\end{lem}

\begin{proof}
    The same argument as in the proof of Lemma \ref{Lemma:stability of G(J)} shows that $\sigma_{\mu, d}(G(J_1)) \subset G(J_2)$.
    The second assertion immediately follows from the first one.
\end{proof}

\begin{lem}\label{Lemma:five lemma for VdX}
    Let $J_3 \subset J_2 \subset J_1 \subset A'$ be three ideals which satisfy the assumption of Lemma \ref{Lemma:stability of G(J)}.
    Let $X \in G(A')$.
    If the maps
 \[
\mathcal{V}_{d, X} \colon G(J_1)/G(J_2) \to G(J_1)/G(J_2) \quad \text{and} \quad \mathcal{V}_{d, X} \colon G(J_2)/G(J_3) \to G(J_2)/G(J_3)
 \]
 are bijective, then
$
\mathcal{V}_{d, X} \colon G(J_1)/G(J_3) \to G(J_1)/G(J_3)
$
is also bijective.
\end{lem}

\begin{proof}
    Let us prove the surjectivity.
    Let $h \in G(J_1)/G(J_3)$ be an element.
    The image $h' \in G(J_1)/G(J_2)$ of $h$ can be written as
    $h'=\mathcal{V}_{d, X}(g')$
    for some element $g' \in G(J_1)/G(J_2)$.
    We choose some $g \in G(J_1)/G(J_3)$ which is a lift of $g'$.
    Then, we see that
    $\mathcal{U}_{d, X}(g)^{-1}hg$
    is contained in $G(J_2)/G(J_3)$, so that there exists an element
    $g'' \in G(J_2)/G(J_3)$
    such that
    \[
    \mathcal{V}_{d, X}(g'')=\mathcal{U}_{d, X}(g'')g''^{-1}=\mathcal{U}_{d, X}(g)^{-1}hg.
    \]
    It follows that $h=\mathcal{V}_{d, X}(gg'')$.
    This proves that
    $
    \mathcal{V}_{d, X} \colon G(J_1)/G(J_3) \to G(J_1)/G(J_3)
    $
    is surjective.
    The proof of the injectivity is similar.
\end{proof}

\begin{lem}\label{Lemma:VdX bijective primitive case}
    Let $l \geq 0$ be an integer.
    For any $X \in G(A')$,
    the map
    \[
    \mathcal{V}_{d, X} \colon G((\pi, d)^l dK)/G((\pi, d)^{l+1} dK) \to  G((\pi, d)^l dK)/G((\pi, d)^{l+1} dK)
    \]
    is bijective.
\end{lem}

\begin{proof}
    \textit{Step 1.}
    We set $K_l:=(\pi, d)^lK$.
    We consider the ideal
    $K^{-}:=K^2+(\pi, d)K$
    and let
    $
    K^{-}_l:=(\pi, d)^l K^{-}.
    $
    All of $dK_l$, $dK_{l+1}$, $dK^{-}_l$
    satisfy the assumption of Lemma \ref{Lemma:stability of G(J)}.
    Since
    \[
    \phi(K^2) \subset d^2K^2 \subset d(\pi, d)K,
    \]
    we have $\phi(K^{-}_l) \subset dK_{l+1}$.
    Therefore, it follows from Lemma \ref{Lemma:square zero case} that
    $\mathcal{V}_{d, X}$ is bijective for $G(dK^{-}_l)/G(dK_{l+1})$.
    By Lemma \ref{Lemma:five lemma for VdX},
    it now suffices to show that $\mathcal{V}_{d, X}$ is bijective for
    $G(dK_l)/G(dK^{-}_l)$.
    
    \textit{Step 2.} 
    By Lemma \ref{Lemma:kernel of two fold coproduct refined} and Corollary \ref{Corollary:kernel of three fold coproduct},
    there exists a finitely generated ideal $M \subset A'$ which is contained in $K$ such that
    $\phi(K) \subset dM + dK^{-}$
    and
    $\phi(M) \subset (t_1, \dotsc, t_n)dM + dK^{-}$,
    where we abuse notation and denote the image of $t_i \in A$ under the morphism $p_1 \colon A \to A'$
    (resp.\ $q_1 \colon A \to A'$)
    if $A'=A^{(2)}$
    (resp.\ if $A'=A^{(3)}$)
    by the same symbol.
    We set
    $M_l := (\pi, d)^l M \subset K_l$.
    Then, we have inclusions
    \begin{equation}\label{equation:key inclusions}
        \phi(K_l) \subset dM_l + dK^{-}_l \quad \text{and} \quad \phi(M_l) \subset (t_1, \dotsc, t_n)dM_l + dK^{-}_l.
    \end{equation}
    In particular, the ideals
    $dM_l + dK^{-}_l \subset dK_l$ satisfy the assumption of Lemma \ref{Lemma:square zero case}, so that
    $\mathcal{V}_{d, X}$ is bijective for $G(dK_l)/G(dM_l + dK^{-}_l)$.
    By Lemma \ref{Lemma:five lemma for VdX},
    it is enough to prove that $\mathcal{V}_{d, X}$ is bijective for $G(dM_l + dK^{-}_l)/G(dK^{-}_l)$.

     \textit{Step 3.} 
     We shall prove that
     \begin{equation}\label{equation:inverse limit I}
         G(dM_l + dK^{-}_l)/G(dK^{-}_l) \overset{\sim}{\to} \varprojlim_{r \geq 0} G(dM_l + dK^{-}_l)/G((t_1, \dotsc, t_n)^r dM_l + dK^{-}_l).
     \end{equation}
     To simplify the notation, we set $C_1:=A'/(dM_l + dK^{-}_l)$ and $C_2:=A'/dK^{-}_l$.
     Let $N \subset C_2$ be the image of $dM_l + dK^{-}_l$.
     Since $A'$ is $I'$-adically complete, it is henselian with respect to $I'$, which in turn implies that $A'$ is henselian with respect to any ideal which is contained in $I'$.
     Using this fact and the smoothness of $G$,
     we see that
     \[
     G(dM_l + dK^{-}_l)/G(dK^{-}_l) \overset{\sim}{\to} \Ker(G(C_2) \to G(C_1))
     \]
     and
     \[
     G(dM_l + dK^{-}_l)/G((t_1, \dotsc, t_n)^r dM_l + dK^{-}_l) \overset{\sim}{\to} \Ker(G(C_2/(t_1, \dotsc, t_n)^r N) \to G(C_1)).
     \]
     Thus, it suffices to prove that
     \[
     N \overset{\sim}{\to} \varprojlim_{r \geq 0} N/(t_1, \dotsc, t_n)^r N \quad \text{and} \quad C_2 \overset{\sim}{\to} \varprojlim_{r \geq 0} C_2/(t_1, \dotsc, t_n)^r N.
     \]
     Since $N$ is killed by $K$, we see that $N$ is a finitely generated module over $A'/K \overset{\sim}{\to} A$.
     Since $A$ is noetherian and is $(t_1, \dotsc, t_n)$-adically complete, it follows that $N$ is also $(t_1, \dotsc, t_n)$-adically complete, which means that $N \overset{\sim}{\to} \varprojlim_{r \geq 0} N/(t_1, \dotsc, t_n)^r N$.
     Moreover,
     this implies that
     $C_2 \overset{\sim}{\to} \varprojlim_{r \geq 0} C_2/(t_1, \dotsc, t_n)^r N$.

    \textit{Step 4.}
    We claim that
    $\mathcal{V}_{d, X}$
    is bijective for
    \[
    G((t_1, \dotsc, t_n)^r dM_l + dK^{-}_l)/G((t_1, \dotsc, t_n)^{r+1} dM_l + dK^{-}_l)
    \]
    for any $r \geq 0$.
    Indeed,
    the second inclusion of (\ref{equation:key inclusions}) shows that
    the assumption of Lemma \ref{Lemma:square zero case} is satisfied in this case, and hence the assertion follows.
    
    Using Lemma \ref{Lemma:five lemma for VdX} repeatedly, we then see that
    $\mathcal{V}_{d, X}$
    is bijective for
    \[
    G(dM_l + dK^{-}_l)/G((t_1, \dotsc, t_n)^r dM_l + dK^{-}_l)
    \]
    for any $r \geq 0$.
    Then, it follows from (\ref{equation:inverse limit I}) that
    $\mathcal{V}_{d, X}$
    is bijective for $G(dM_l + dK^{-}_l)/G(dK^{-}_l)$ as well.
    This completes the proof.
\end{proof}

Let us now prove the desired result.

\begin{prop}\label{Proposition:bijectivity of VdX for G(dK)}
    For any $X \in G(A')$, the map
    $
    \mathcal{V}_{d, X} \colon G(dK) \to  G(dK)
    $
    is bijective.
\end{prop}

\begin{proof}
    We claim that $G(dK) \overset{\sim}{\to} \varprojlim_{l \geq 0} G(dK)/G((\pi, d)^l dK)$.
    Indeed, arguing as in the proof of Lemma \ref{Lemma:VdX bijective primitive case},
    we see that
    \[
    G(dK)/G((\pi, d)^l dK) \overset{\sim}{\to} \Ker(G(A'/(\pi, d)^l dK) \to G(A'/dK)).
    \]
    Since $d$ is a nonzerodivisor and $K$ is $(\pi, d)$-adically complete, it follows that
    $dK \overset{\sim}{\to} \varprojlim_{l \geq 0} dK/(\pi, d)^l dK$.
    Moreover, this implies that
    $A' \overset{\sim}{\to} \varprojlim_{l \geq 0} A'/(\pi, d)^l dK$.
    Using these isomorphisms, we can conclude that
    $G(dK) \overset{\sim}{\to} \varprojlim_{l \geq 0} G(dK)/G((\pi, d)^l dK)$.

    In order to show that
    $
    \mathcal{V}_{d, X} \colon G(dK) \to  G(dK)
    $
    is bijective, it suffices to check that
    $
    \mathcal{V}_{d, X} \colon G(dK)/G((\pi, d)^l dK) \to  G(dK)/G((\pi, d)^l dK)
    $
    is bijective for any $l \geq 0$.
    This follows from 
    Lemma \ref{Lemma:VdX bijective primitive case}
    by using Lemma \ref{Lemma:five lemma for VdX} repeatedly.
\end{proof}

We also need the following lemma:

\begin{lem}\label{Lemma:banal after finite field extension}
    Let $\mathcal{Q}$ be a $G$-$\mu$-display over $(A, I)$.
    Then, there exists a finite extension $\widetilde{k}$ of $k$ such that the base change of
    $\mathcal{Q}$ to $(A_{\widetilde{\O}}, IA_{\widetilde{\O}})$ is banal, where $\widetilde{\O}:= W(\widetilde{k}) \otimes_{W(\F_q)} \O_E$ and $A_{\widetilde{\O}}:=A \otimes_\O \widetilde{\O} = \widetilde{\O}[[t_1, \dotsc, t_n]]$.
\end{lem}

\begin{proof}
    This immediately follows from Proposition \ref{Proposition:G display with trivial Hodge filtration is banal}.
\end{proof}

\begin{proof}[Proof of Proposition \ref{Proposition:deformation of isomorphism}]
    By Lemma \ref{Lemma:banal after finite field extension}, there exists a finite Galois extension $\widetilde{k}$ of $k$ such that the base changes of
    $\mathcal{Q}_1$  
    and $\mathcal{Q}_2$
    to $(A_{\widetilde{\O}}, IA_{\widetilde{\O}})$ are banal.
    Here $\widetilde{\O}:= W(\widetilde{k}) \otimes_{W(\F_q)} \O_E$ and $A_{\widetilde{\O}}:=A \otimes_\O \widetilde{\O}$; we use the same notation for $\O$-algebras.
    We can identify
    $(A'_{\widetilde{\O}}, I'A'_{\widetilde{\O}})$
    with a coproduct of two (resp.\ three) copies of
    $(A_{\widetilde{\O}}, IA_{\widetilde{\O}})$ in $(R_{\widetilde{\O}})_{\Prism, \O_E}$
    if $A'=A^{(2)}$
    (resp.\ if $A'=A^{(3)}$).
    By Galois descent for $G$-$\mu$-displays,
    it suffices to prove the same statement for banal
    $G$-$\mu$-displays over $(A_{\widetilde{\O}}, IA_{\widetilde{\O}})$.
    We may thus assume without loss of generality that $\mathcal{Q}_1$  
    and $\mathcal{Q}_2$ are banal $G$-$\mu$-displays over $(A, I)$.


    If
    $\mathcal{Q}_1$  
    and $\mathcal{Q}_2$ are not isomorphic to each other, then the assertion holds trivially.
    Thus, we may further assume that 
    $\mathcal{Q}_1=\mathcal{Q}_2=\mathcal{Q}_{Y}$
    for some $Y \in G(A)_I$.
    Let $d:=f_2(\mathcal{E})$.
    We have $f_1(\mathcal{E})=u d$ for some $u \in A'^\times$.
    With the choice of $d \in I'$,
    the $G$-$\mu$-displays
    $f^*_1(\mathcal{Q}_Y)$ and
    $f^*_2(\mathcal{Q}_Y)$
    correspond to the elements
    $f_1(Y_{\mathcal{E}})\phi(\mu(u)), f_2(Y_{\mathcal{E}}) \in G(A')_d$, respectively, and hence we can identify
    $\Hom_{G\mathchar`-\mathrm{Disp}_\mu(A', I')}(f^*_1(\mathcal{Q}_Y), f^*_2(\mathcal{Q}_Y))$
    with the set
    \[
    \{\, g \in G_\mu(A', I') \, \vert \, g^{-1}f_2(Y_{\mathcal{E}}) \sigma_{\mu, d}(g)=f_1(Y_{\mathcal{E}})\phi(\mu(u)) \, \}.
    \]

    We set $X:=f_2(Y_{\mathcal{E}})$.
    We shall prove that the map
    (\ref{equation:reduction map for isomorphisms})
    is injective.
    Let $g, h \in G_\mu(A', I')$ be two elements in $\Hom_{G\mathchar`-\mathrm{Disp}_\mu(A', I')}(f^*_1(\mathcal{Q}_1), f^*_2(\mathcal{Q}_2))$
    such that $m(g)=m(h)$ in $G_\mu(A, I)$.
    We set $\beta:=gh^{-1}$.
    Since $m(\beta)=1$, we have
    $\mu(d)\beta\mu(d)^{-1} \in G(K)$.
    Then, it follows from $\phi(K) \subset dK$ that $\sigma_{\mu, d}(\beta) \in G(dK)$.
    The equalities
    \[
    g^{-1}X \sigma_{\mu, d}(g) = f_1(Y_{\mathcal{E}})\phi(\mu(u)) = h^{-1}X \sigma_{\mu, d}(h)
    \]
    imply that $\beta=X \sigma_{\mu, d}(\beta) X^{-1}$.
    It follows that $\beta \in G(dK)$, and we have $\mathcal{V}_{d, X}(\beta)=1$ for the map $\mathcal{V}_{d, X} \colon G(dK) \to G(dK)$.
    Since $\mathcal{V}_{d, X}$ is bijective by Proposition \ref{Proposition:bijectivity of VdX for G(dK)}, we obtain $\beta=1$.
    
    It remains to prove that the map
    (\ref{equation:reduction map for isomorphisms})
    is surjective.
    For this, it suffices to prove that
    $\Hom_{G\mathchar`-\mathrm{Disp}_\mu(A', I')}(f^*_1(\mathcal{Q}_Y), f^*_2(\mathcal{Q}_Y))$
    is not empty.
    We claim that $\phi(\mu(u)) \in G(dK)$ and $\gamma:=f_2(Y_\mathcal{E})^{-1}f_1(Y_\mathcal{E}) \in G(dK)$.
    Indeed, since $m(u)=1$, we have $\mu(u) \in G(K)$, which in turn implies that $\phi(\mu(u)) \in G(dK)$.
    Since the morphisms $f_1$ and $f_2$ induce the same homomorphism
    $R \to A'/I'$, we see that $\gamma \in G(I')$.
    Using that $I' \cap K=dK$, we then obtain
    $\gamma \in G(dK)$.
    Since $\mathcal{V}_{d, X} \colon G(dK) \to G(dK)$ is bijective, there exists an element $g \in G(dK)$ such that
    $\mathcal{V}_{d, X}(g^{-1})=X\phi(\mu(u))^{-1}\gamma^{-1} X^{-1}$, or equivalently
    \[
    g^{-1}f_2(Y_{\mathcal{E}}) \sigma_{\mu, d}(g)=f_1(Y_{\mathcal{E}})\phi(\mu(u)).
    \]
    In other words, the element $g$ gives an isomorphism
    $f^*_1(\mathcal{Q}_Y) \overset{\sim}{\to} f^*_2(\mathcal{Q}_Y)$.
\end{proof}



\subsection{Proof of Theorem \ref{Theorem:main result on G displays over complete regular local rings}}\label{Subsection:proof of main result}

In this section, we prove Theorem \ref{Theorem:main result on G displays over complete regular local rings} using our previous results.

As in Section \ref{Subsection:Coproducts of Breuil--Kisin prisms},
we write $(A, I)=(\mathfrak{S}_\O, (\mathcal{E}))$.
Let
\[
G\mathchar`-\mathrm{Disp}^\mathrm{DD}_\mu(A, I)
\]
be the groupoid of pairs $(\mathcal{Q}, \epsilon)$ consisting of a $G$-$\mu$-display
$\mathcal{Q}$
over $(A, I)$
and an isomorphism
$\epsilon \colon p^*_1\mathcal{Q} \overset{\sim}{\to} p^*_2\mathcal{Q}$
of $G$-$\mu$-displays
over $(A^{(2)}, I^{(2)})$
satisfying the cocycle condition $p^*_{13}\epsilon=p^*_{23}\epsilon \circ p^*_{12}\epsilon$.
An isomorphism $(\mathcal{Q}, \epsilon) \overset{\sim}{\to} (\mathcal{Q}', \epsilon')$ is an isomorphism $f \colon \mathcal{Q} \overset{\sim}{\to} \mathcal{Q}'$ of $G$-$\mu$-displays
over $(A, I)$ such that $\epsilon' \circ (p^*_1 f)=(p^*_2 f) \circ \epsilon$.


For a $G$-$\mu$-display $\mathfrak{Q}$ over $(R)_{\Prism, \O_E}$,
we have the associated isomorphism
\[
\gamma_{p_i} \colon p^*_i(\mathfrak{Q}_{(A, I)}) \overset{\sim}{\to} \mathfrak{Q}_{(A^{(2)}, I^{(2)})}
\]
for $i=1, 2$.
Let $\epsilon:=\gamma^{-1}_{p_2} \circ \gamma_{p_1}$.
Then $\epsilon$ satisfies the cocycle condition, so that the pair $(\mathfrak{Q}_{(A, I)}, \epsilon)$
is an object of $\mathrm{Disp}^\mathrm{DD}_\mu(A, I)$.
This construction induces a functor
\[
G\mathchar`-\mathrm{Disp}_\mu((R)_{\Prism, \O_E}) \to G\mathchar`-\mathrm{Disp}^\mathrm{DD}_\mu(A, I), \quad \mathfrak{Q} \mapsto (\mathfrak{Q}_{(A, I)}, \epsilon).
\]

\begin{prop}\label{Proposition:crystal and descent datum}
    The functor
    $G\mathchar`-\mathrm{Disp}_\mu((R)_{\Prism, \O_E}) \to G\mathchar`-\mathrm{Disp}^\mathrm{DD}_\mu(A, I)$
    is an equivalence.
\end{prop}

\begin{proof}
    This is a formal consequence of Proposition \ref{Proposition:flat descent of G display} and Proposition \ref{Proposition:weakely initial object}.
    %We briefly explain how to construct an inverse to the above functor.
    %Let $(\mathcal{Q}, \epsilon) \in G\mathchar`-\mathrm{Disp}^\mathrm{DD}_\mu(A, I)$.
    %For an object $(C, J) \in (R)_{\Prism, \O_E}$, we choose a flat cover
    %$(C, J) \to (C', J')$ such that $(C', J')$ admits a morphism
    %$h \colon (A, I) \to (C', J')$ in $(R)_{\Prism, \O_E}$; see Proposition \ref{Proposition:weakely initial object}.
    %Let $(C'', J'')$ be a pushout of the diagram
    %$(C', J') \leftarrow (C, J) \rightarrow (C', J')$, and let
    %$p'_1, p'_2 \colon (C', J') \to (C'', J'')$ denote the two natural maps.
    %Then $\epsilon$ gives rise to an isomorphism
    %$\epsilon' \colon (p'_1)^*h^*\mathcal{Q} \overset{\sim}{\to} (p'_2)^*h^*\mathcal{Q}$, which satisfies the usual cocycle condition.
    %Thus, it follows from Proposition \ref{Proposition:flat descent of G display} that
    %the pair $(h^*\mathcal{Q}, \epsilon')$ arises from
    %a $G$-$\mu$-display
    %$\mathfrak{Q}_{(C, J)}$ over $(C, J)$.
    %One can show that
    %$\mathfrak{Q}_{(C, J)}$ does not depend on the choices of $(C', J')$ and $h$ up to canonical isomorphism, and for each morphism
    %$f \colon (C_1, J_1) \to (C_2, J_2)$ in $(R)_{\Prism, \O_E}$,
    %we have a canonical isomorphism
    %$\gamma_f \colon f^*(\mathfrak{Q}_{(C_1, J_1)}) \overset{\sim}{\to} \mathfrak{Q}_{(C_2, J_2)}$
    %such that
    %$\gamma_{f'} \circ ({f'}^*\gamma_f) = \gamma_{f' \circ f}$.
    %In this way, we obtain a $G$-$\mu$-display over %$(R)_{\Prism, \O_E}$.
    %It is straightforward to show that this construction gives an inverse to the above functor.
\end{proof}
    


\begin{proof}[Proof of Theorem \ref{Theorem:main result on G displays over complete regular local rings}]
We assume that $\mu$ is 1-bounded.
By virtue of Proposition \ref{Proposition:crystal and descent datum},
it suffices to show that the forgetful functor
\[
G\mathchar`-\mathrm{Disp}^\mathrm{DD}_\mu(A, I) \to G\mathchar`-\mathrm{Disp}_\mu(A, I)
\]
is an equivalence.
Let $m \colon (A^{(2)}, I^{(2)}) \to (A, I)$
be the unique morphism in $(R)_{\Prism, \O_E}$ such that
$m \circ p_i = \id_{(A, I)}$ for $i=1, 2$, and
let
$m' \colon (A^{(3)}, I^{(3)}) \to (A, I)$ be the unique morphism in $(R)_{\Prism, \O_E}$ such that $m \circ q_i = \id_{(A, I)}$ for $i=1, 2, 3$.
Let $\mathcal{Q}$ be a $G$-$\mu$-display over $(A, I)$.
We claim that an isomorphism $\epsilon \colon p^*_1\mathcal{Q} \overset{\sim}{\to} p^*_2\mathcal{Q}$ satisfies the cocycle condition $p^*_{13}\epsilon=p^*_{23}\epsilon \circ p^*_{12}\epsilon$
if and only if $m^*\epsilon=\id_{\mathcal{Q}}$.
Indeed, since the map
\[
m'^* \colon \Hom_{G\mathchar`-\mathrm{Disp}_\mu(A^{(3)}, I^{(3)})}(q^*_1\mathcal{Q}, q^*_3\mathcal{Q}) \to \Hom_{G\mathchar`-\mathrm{Disp}_\mu(A, I)}(\mathcal{Q}, \mathcal{Q})
\]
is bijective by Proposition \ref{Proposition:deformation of isomorphism},
we see that $\epsilon$ satisfies the cocycle condition $p^*_{13}\epsilon=p^*_{23}\epsilon \circ p^*_{12}\epsilon$
if and only if $m^*\epsilon=m^*\epsilon \circ m^*\epsilon$, which is equivalent to saying that $m^*\epsilon=\id_{\mathcal{Q}}$.
By Proposition \ref{Proposition:deformation of isomorphism} again,
the map
\[
m^* \colon \Hom_{G\mathchar`-\mathrm{Disp}_\mu(A^{(2)}, I^{(2)})}(p^*_1\mathcal{Q}, p^*_2\mathcal{Q}) \to \Hom_{G\mathchar`-\mathrm{Disp}_\mu(A, I)}(\mathcal{Q}, \mathcal{Q})
\]
is bijective.
Therefore,
for any $G$-$\mu$-display $\mathcal{Q}$ over $(A, I)$,
there exists a unique isomorphism
$\epsilon \colon p^*_1\mathcal{Q} \overset{\sim}{\to} p^*_2\mathcal{Q}$
satisfying the cocycle condition $p^*_{13}\epsilon=p^*_{23}\epsilon \circ p^*_{12}\epsilon$, and $\epsilon$ is characterized by the condition that $m^*\epsilon=\id_{\mathcal{Q}}$.
It follows that the forgetful functor
$G\mathchar`-\mathrm{Disp}^\mathrm{DD}_\mu(A, I) \to G\mathchar`-\mathrm{Disp}_\mu(A, I)$
is an equivalence.
\end{proof}


\section{$p$-divisible groups and prismatic Dieudonn\'e crystals}\label{Section:p-divisible groups and prismatic Dieudonn\'e crystals}

In this section, we make a few remarks on prismatic Dieudonn\'e crystals, which are introduced in \cite{Anschutz-LeBras}.

\subsection{A remark on prismatic Dieudonn\'e crystals}\label{Subsection:A remark on prismatic Dieudonne crystals}

Let $R$ be a $\pi$-adically complete $\O_E$-algebra.
Let
$\O_\Prism$
denote the sheaf on the site $(R)^{\op}_{\Prism, \O_E}$
defined by $(A, I) \mapsto A$.

Following \cite{Anschutz-LeBras}, we say that an $\O_\Prism$-module $\mathcal{M}$ on $(R)^{\op}_{\Prism, \O_E}$ is a
\textit{prismatic crystal in finite projective $\O_\Prism$-modules}
if $\mathcal{M}(A, I)$
is a finite projective $A$-module for any $(A, I) \in (R)_{\Prism, \O_E}$, and for any morphism $(A, I) \to (A', I')$ in $(R)_{\Prism, \O_E}$, the natural homomorphism
\[
\mathcal{M}(A, I) \otimes_A A' \to \mathcal{M}(A', I')
\]
is bijective.
A 
\textit{prismatic Dieudonn\'e crystal}
on $(R)^{\op}_{\Prism, \O_E}$
is a prismatic crystal $\mathcal{M}$ in finite projective $\O_\Prism$-modules on $(R)^{\op}_{\Prism, \O_E}$ equipped with
a $\phi$-linear homomorphism
\[
\varphi_\mathcal{M} \colon \mathcal{M} \to \mathcal{M}
\]
such that for any $(A, I) \in (R)_{\Prism, \O_E}$,
the finite projective $A$-module $\mathcal{M}(A, I)$
with
the linearization
$1 \otimes \varphi_\mathcal{M} \colon \phi^*(\mathcal{M}(A, I)) \to \mathcal{M}(A, I)$
is a minuscule Breuil--Kisin module over $(A, I)$ in the sense of Definition \ref{Definition:displayed and minuscule Breuil-Kisin module} (see also Proposition \ref{Proposition:minuscule equivalent condition}).
For a bounded $\O_E$-prism $(A, I)$, let
\[
\mathrm{BK}_{\mathrm{min}}(A, I)
\]
be the category of minuscule Breuil--Kisin modules over $(A, I)$.
Then, the category of prismatic Dieudonn\'e crystals on $(R)^{\op}_{\Prism, \O_E}$ is equivalent to the category
\[
{2-\varprojlim}_{(A, I) \in (R)_{\Prism, \O_E}} \mathrm{BK}_{\mathrm{min}}(A, I).
\]

As in Section \ref{Section:G-displays over complete regular local rings},
let $R$ be a complete regular local ring equipped with a local homomorphism $\O \to R$ which induces an isomorphism on the residue fields.
Let
$
(\mathfrak{S}_\O, (\mathcal{E}))
$
be
an $\O_E$-prism
of Breuil--Kisin type, where $\mathfrak{S}_\O=\O[[t_1, \dotsc, t_n]]$,
with an isomorphism
$
R \simeq \mathfrak{S}_\O/\mathcal{E}
$ over $\O$.

By using the results of Section \ref{Section:G-displays over complete regular local rings}, we can prove the following proposition, which is obtained in the proof of \cite[Theorem 5.12]{Anschutz-LeBras} if $n \leq 1$ (and $\O_E=\Z_p$).



\begin{prop}\label{Proposition:evaluation prismatic dieudonne crystal equivalence}
    The functor $\mathcal{M} \mapsto \mathcal{M}(\mathfrak{S}_\O, (\mathcal{E}))$ 
    from the category of prismatic Dieudonn\'e crystals on $(R)^{\op}_{\Prism, \O_E}$ to the category $\mathrm{BK}_{\mathrm{min}}(\mathfrak{S}_\O, (\mathcal{E}))$
    is an equivalence.
\end{prop}

\begin{proof}
    As in the proof of Theorem \ref{Theorem:main result on G displays over complete regular local rings}, one can deduce the proposition from the following claim: Let the notation be as in Proposition \ref{Proposition:deformation of isomorphism}.
    Then, for minuscule Breuil--Kisin modules $M_1$ and $M_2$ over $(A, I)=(\mathfrak{S}_\O, (\mathcal{E}))$,
    the map
    \begin{equation}\label{equation:reduction map homomorphism}
        m^{*} \colon \Hom_{\mathrm{BK}_{\mathrm{min}}(A', I')}(f^*_1(M_1), f^*_2(M_2)) \to \Hom_{\mathrm{BK}_{\mathrm{min}}(A, I)}(M_1, M_2)
    \end{equation}
    induced by the base change functor
    $m^* \colon \mathrm{BK}_{\mathrm{min}}(A', I') \to \mathrm{BK}_{\mathrm{min}}(A, I)$ is bijective.

    We shall prove the claim.
    Let $M := M_1 \oplus M_2 \in \mathrm{BK}_{\mathrm{min}}(A, I)$
    be a direct sum of $M_1$ and $M_2$.
    As $A=\mathfrak{S}_\O$ is local, we see that $M$ is banal and of type $\mu$ for some
    $\mu=(1, \dotsc, 1, 0, \dotsc , 0)$.
    Then, it follows from Example \ref{Example:GLn displays} and
    Proposition \ref{Proposition:deformation of isomorphism} that
    \begin{equation}\label{equation:reduction map isomorphism, direct sum}
        m^{*} \colon \Isom_{\mathrm{BK}_{\mathrm{min}}(A', I')}(f^*_1 M, f^*_2 M) \overset{\sim}{\to} \Isom_{\mathrm{BK}_{\mathrm{min}}(A, I)}(M, M),
    \end{equation}
    where $\mathrm{Isom}_{\mathrm{BK}_{\mathrm{min}}(A', I')}(f^*_1 M, f^*_2 M)$ is the set of isomorphisms
    $f^*_1 M \overset{\sim}{\to} f^*_2 M$
    of minuscule Breuil--Kisin modules over $(A', I')$,
    and similarly for $\mathrm{Isom}_{\mathrm{BK}_{\mathrm{min}}(A, I)}(M, M)$.
    We also see that the identity $\id_{M_i} \colon M_i \to M_i$ uniquely lifts to an isomorphism
    $\eta_i \colon f^*_1 (M_i) \overset{\sim}{\to} f^*_2 (M_i)$
    for each $i=1, 2$.
    We prove that the map (\ref{equation:reduction map homomorphism}) is injective.
    Let $g \colon f^*_1 (M_1) \to f^*_2 (M_2)$ be a homomorphism such that $m^*g=0$.
    We consider the homomorphism
    $g':=\begin{pmatrix}
\eta_1 & 0 \\
g & \eta_2 \\
\end{pmatrix} \colon f^*_1 M \to f^*_2 M$.
Since $\eta_1$ and $\eta_2$ are isomorphisms, it follows that $g'$ is an isomorphism.
Since $m^*(g')=\id_M$, the injectivity of (\ref{equation:reduction map isomorphism, direct sum}) implies that
$g'=\begin{pmatrix}
\eta_1 & 0 \\
0 & \eta_2 \\
\end{pmatrix}$, which means that $g=0$.
This proves that (\ref{equation:reduction map homomorphism}) is injective.
Similarly, we can deduce the surjectivity of (\ref{equation:reduction map homomorphism}) from that of (\ref{equation:reduction map isomorphism, direct sum}).
\end{proof}


\subsection{$p$-divisible groups and minuscule Breuil-Kisin modules}\label{Subsection:p-divisible groups and minuscule Breuil-Kisin modules}

In this subsection, we consider the case where $\O_E=\Z_p$.
Let $R$ be a $p$-adically complete ring,
and
let $\mathcal{G}$ be a $p$-divisible group over $\Spec R$.
We define the following functors
\begin{align*}
    \underline{\mathcal{G}} &\colon (R)_\Prism \to \mathrm{Set}, \quad (A, I) \mapsto \mathcal{G}(A/I), \\
    \underline{\mathcal{G}[p^n]} &\colon (R)_\Prism \to \mathrm{Set}, \quad (A, I) \mapsto \mathcal{G}[p^n](A/I).
\end{align*}
These functors form sheaves on the site $(R)^{\op}_\Prism$.
In \cite[Proposition 4.69]{Anschutz-LeBras}, it is proved that
the $\O_\Prism$-module
\[
\mathcal{E}xt^1_{(R)_\Prism}(\underline{\mathcal{G}}, \O_\Prism)
\]
on $(R)^{\op}_\Prism$
is a prismatic crystal in finite projective $\O_\Prism$-modules.
(Here we simply write $\mathcal{E}xt^1_{(R)_\Prism}(\underline{\mathcal{G}}, \O_\Prism)$ rather than $\mathcal{E}xt^1_{(R)^{\op}_\Prism}(\underline{\mathcal{G}}, \O_\Prism)$.)

For a bounded prism $(A, I)$, let
\[
(A, I)_\Prism
\]
denote the category of bounded prisms $(B, J)$ over $(A, I)$.
We endow
$(A, I)^{\op}_{\Prism}$ with the flat topology.
For an object $(A, I) \in (R)_\Prism$, the site
$(A, I)^{\op}_{\Prism}$
is the localization of $(R)^{\op}_\Prism$ at $(A, I)$.

\begin{rem}\label{Remark:evaluation loca Ext groups}
\
\begin{enumerate}
    \item For an integer $n \geq 1$, the map $[p^{n}] \colon \underline{\mathcal{G}} \to \underline{\mathcal{G}}$ induced by multiplication by $p^n$ is surjective.
    This follows from \cite[Corollary 3.25]{Anschutz-LeBras}.
    \item We have $\mathcal{H}om_{(R)_\Prism}(\underline{\mathcal{G}}, \O_\Prism)=0$.
    Indeed, since $[p] \colon \underline{\mathcal{G}} \to \underline{\mathcal{G}}$ is surjective and the topos associated with $(R)^{\op}_\Prism$ is replete in the sense of \cite[Definition 3.1.1]{Bhatt-ScholzeProetale},
    the projection $\varprojlim_{[p]} \underline{\mathcal{G}} \to \underline{\mathcal{G}}$ is surjective.
    Since
    $\O_\Prism(A, I)=A$ is $p$-adically complete for any $(A, I) \in (R)_\Prism$, we can conclude that $\mathcal{H}om_{(R)_\Prism}(\underline{\mathcal{G}}, \O_\Prism)=0$.
    As a consequence,
    the local-to-global spectral sequence implies that
    \[
    \mathrm{Ext}^1_{(A, I)_\Prism}(\underline{\mathcal{G}}, \O_\Prism) \overset{\sim}{\to} \mathcal{E}xt^1_{(R)_\Prism}(\underline{\mathcal{G}}, \O_\Prism)(A, I)
    \]
    for any $(A, I) \in (R)_\Prism$.
    Here the restrictions of $\underline{\mathcal{G}}$ and $\O_\Prism$ to the site $(A, I)^{\op}_{\Prism}$ are denoted by the same symbols.
    In particular
    $\mathrm{Ext}^1_{(A, I)_\Prism}(\underline{\mathcal{G}}, \O_\Prism)$
    is a finite projective $A$-module and its formation commutes with base change along any morphism $(A, I) \to (A', I')$ in $(R)_\Prism$.
\end{enumerate}
\end{rem}

In \cite[Theorem 4.71]{Anschutz-LeBras}, it is also proved that
$\mathcal{E}xt^1_{(R)_\Prism}(\underline{\mathcal{G}}, \O_\Prism)$
with
the $\phi$-linear homomorphism
$\mathcal{E}xt^1_{(R)_\Prism}(\underline{\mathcal{G}}, \O_\Prism) \to \mathcal{E}xt^1_{(R)_\Prism}(\underline{\mathcal{G}}, \O_\Prism)$
induced by the Frobenius $\phi \colon \O_\Prism \to \O_\Prism$
is a prismatic Dieudonn\'e crystal if $R$ is \textit{quasisyntomic} in the sense of \cite[Definition 4.10]{BMS2}.
(More precisely, they showed that $\mathcal{E}xt^1_{(R)_\Prism}(\underline{\mathcal{G}}, \O_\Prism)$ is admissible in the sense of \cite[Definition 4.5]{Anschutz-LeBras}.)
We shall recall the argument.


\begin{prop}[{\cite[Theorem 4.71]{Anschutz-LeBras}}]\label{Proposition:prismatic Dieudonne crystal of p-divisivle group}
Assume that $R$ is quasisyntomic.
Let $(A, I) \in (R)_\Prism$.
We write $M:=\mathrm{Ext}^1_{(A, I)_\Prism}(\underline{\mathcal{G}}, \O_\Prism)$.
Then $M$ with the induced homomorphism
$
F_M \colon \phi^*M \to M
$
is a minuscule Breuil--Kisin module over $(A, I)$.
\end{prop}

\begin{proof}
    By Corollary \ref{Corollary:flat descent for dispalyed BK modules} (and Proposition \ref{Proposition:flat descent for finite projective modules}), this can be checked $(p, I)$-completely flat locally.
    Thus, by virtue of \cite[Lemma 4.28]{BMS2} and \cite[Proposition 7.11]{BS}, we may assume that $R$ is quasiregular semiperfectoid in the sense of \cite[Definition 4.20]{BMS2}.
    Then, by choosing a surjective homomorphism from a perfectoid ring to $R$ and using \cite[Corollary 2.10, Lemma 4.70]{Anschutz-LeBras}, we may assume that $R$ is a perfectoid ring and $(A, I)=(W(R^\flat), I_R)$.
    (Here we regard $(W(R^\flat), I_R)$ as an object of $(R)_\Prism$ via the homomorphism $\theta \colon W(R^\flat) \to R$.
    We note that in \cite{Anschutz-LeBras}, $(W(R^\flat), I_R)$ is viewed as an object of $(R)_\Prism$ via the composition $\theta \circ \phi^{-1}$.)
    
    Let $\xi \in I_R$ be a generator.
    By Proposition \ref{Proposition:minuscule equivalent condition}, it suffices to prove that the cokernel of $F_M$ is killed by $\xi$.
    By Remark \ref{Remark:arc topology} (4) and $p$-complete $\arc$-descent (Proposition \ref{Proposition:arc descent for finite projective modules}),
    we may further assume that $R$ is a $p$-adically complete valuation ring of rank $\leq 1$ with algebraically closed fraction field.

    If $p=0$ in $R$, then $R$ is perfect by Example \ref{Example:algebraically closed valuation ring is perfectoid}.
    In this case, the Frobenius $F_M$ can be identified with
    the homomorphism
    \[
    \mathrm{Ext}^1_{(A, I)_\Prism}(\underline{(\phi^*\mathcal{G})}, \O_\Prism) \to \mathrm{Ext}^1_{(A, I)_\Prism}(\underline{\mathcal{G}}, \O_\Prism)
    \]
    induced by the relative Frobenius $\mathcal{G} \to \phi^*\mathcal{G}$.
    Thus,
    the Verschiebung homomorphism
    $\phi^*\mathcal{G} \to \mathcal{G}$
    induces a $W(R)$-linear homomorphism
    $V_M \colon M \to \phi^*M$
    such that $F_M \circ V_M=p$, which in turn implies the assertion.

    It remains to treat the case where $p$ is a nonzerodivisor in $R$, so that $R$ is the ring of integers $\O_C$ of an algebraically closed non-archimedean extension $C$ of $\Q_p$.
    We set
    \[
    M_n:=\mathrm{Ext}^1_{(W(\O^\flat_C), I_{\O_C})_\Prism}(\underline{\mathcal{G}[p^n]}, \O_\Prism).
    \]
    In the proof of \cite[Proposition 4.69]{Anschutz-LeBras}, it is proved that the natural homomorphism $M \to M_n$ induces an isomorphism
    $M/p^n \overset{\sim}{\to} M_n$ for any $n \geq 1$.
    In particular, we obtain
    \[
    M \overset{\sim}{\to} \varprojlim_n M_n
    \]
    and $M_n$ is a free $W_n(\O^\flat_C)$-module of finite rank.
    We claim that the cokernel of the Frobenius
    $
    F_{M_n} \colon \phi^*(M_n) \to M_n
    $
    is killed by $\xi$.
    Indeed, there is an embedding $\mathcal{G}[p^n] \hookrightarrow X$ into an abelian scheme $X$ over $\Spec \O_C$;
    see \cite[Th\'eor\`eme 3.1.1]{BBM}.
    Let $Y$ be the $p$-adic completion of $X$, which is a smooth $p$-adic formal scheme over $\Spf \O_C$.
    It follows from the proofs of \cite[Theorem 4.62, Proposition 4.66]{Anschutz-LeBras} that
    there exists a surjection
    \[
    H^1_\Prism(Y/W(\O^\flat_C)) \to M_n
    \]
    which is compatible with Frobenius homomorphisms.
    Here $H^1_\Prism(Y/W(\O^\flat_C))$ is the first prismatic cohomology of $Y$ (with respect to $(W(\O^\flat_C), I_{\O_C})$) defined in \cite{BS}.
    Then, by \cite[Theorem 1.8 (6)]{BS}, we see that
    the cokernel of the Frobenius
    \[
    \phi^*H^1_\Prism(Y/W(\O^\flat_C)) \to H^1_\Prism(Y/W(\O^\flat_C))
    \]
    is killed by $\xi$, which in turn implies the claim.
    Since the image of $\xi$ in $W_n(\O^\flat_C)$ is a nonzerodivisor,
    it then follows that
    $
    F_{M_n}
    $
    is injective.
    Since $F_M= \varprojlim_n F_{M_n}$, we can conclude that the cokernel of $F_M$ is killed by $\xi$.
\end{proof}

\begin{rem}\label{Remark:proof difference}
Our proof of Proposition \ref{Proposition:prismatic Dieudonne crystal of p-divisivle group} in the case where $A=\O_C$ is slightly different from that given in \cite{Anschutz-LeBras}.
For example, we do not use \cite[Proposition 14.9.4]{Scholze-Weinstein} (see the proof of \cite[Proposition 4.48]{Anschutz-LeBras}).
\end{rem}

Finally, we recall the following classification theorem for $p$-divisible groups given in \cite{Anschutz-LeBras}.
Let
$R$ be a complete regular local ring with perfect residue field $k$ of characteristic $p >0$.
Let
$(\mathfrak{S}, (\mathcal{E}))$
be a prism of Breuil--Kisin type, where $\mathfrak{S}:=W(k)[[t_1, \dotsc, t_n]]$,
with an isomorphism
$R \simeq \mathfrak{S}/\mathcal{E}$
which lifts $\id_k \colon k \to k$.

\begin{thm}[{Ansch\"utz--Le Bras \cite[Theorem 4.74, Theorem 5.12]{Anschutz-LeBras}}]\label{Theorem:classification theorem for p-divisible group}
\ 
\begin{enumerate}
    \item The contravariant functor
\[
\{ p\text{-divisible groups over} \ \Spec R  \} \to \{ \text{prismatic Dieudonn\'e crystals on} \ (R)^{\op}_\Prism \}
\]
defined by 
$
\mathcal{G} \mapsto \mathcal{E}xt^1_{(R)_\Prism}(\underline{\mathcal{G}}, \O_\Prism)
$
is an anti-equivalence of categories.
    \item The contravariant functor
\[
\{ p\text{-divisible groups over} \ \Spec R  \} \to \{ \text{minuscule Breuil--Kisin modules over} \ (\mathfrak{S}, (\mathcal{E})) \}
\]
defined by 
$
\mathcal{G} \mapsto \mathrm{Ext}^1_{(\mathfrak{S}, (\mathcal{E}))_\Prism}(\underline{\mathcal{G}}, \O_\Prism)
$
is an anti-equivalence of categories.
\end{enumerate}
\end{thm}

\begin{proof}
    (1) This is a consequence of \cite[Theorem 4.74, Proposition 5.10]{Anschutz-LeBras}.

    (2) The assertion follows from (1) and Proposition \ref{Proposition:evaluation prismatic dieudonne crystal equivalence}.
    This result was already stated in \cite[Theorem 5.12]{Anschutz-LeBras}, and the proof was given in the case where $n \leq 1$.
\end{proof}



\subsection*{Acknowledgements}
The author would like to thank Kentaro Inoue, Arthur-C\'esar Le Bras, Kimihiko Li, and Samuel Marks for helpful discussions and comments.
The author is also grateful to Teruhisa Koshikawa for answering some questions on prisms.
The work of the author was supported by JSPS KAKENHI Grant Number 22K20332.

\bibliographystyle{abbrvsort}
\bibliography{bibliography.bib}
\end{document}
