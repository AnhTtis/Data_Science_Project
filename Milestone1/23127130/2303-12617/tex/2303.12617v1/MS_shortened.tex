% ****** Start of file apssamp.tex ******
%
%   This file is part of the APS files in the REVTeX 4.2 distribution.
%   Version 4.2a of REVTeX, December 2014
%
%   Copyright (c) 2014 The American Physical Society.
%
%   See the REVTeX 4 README file for restrictions and more information.
%
% TeX'ing this file requires that you have AMS-LaTeX 2.0 installed
% as well as the rest of the prerequisites for REVTeX 4.2
%
% See the REVTeX 4 README file
% It also requires running BibTeX. The commands are as follows:
%
%  1)  latex apssamp.tex
%  2)  bibtex apssamp
%  3)  latex apssamp.tex
%  4)  latex apssamp.tex
%
\documentclass[%
reprint,
%superscriptaddress,
%groupedaddress,
%unsortedaddress,
%runinaddress,
%frontmatterverbose, 
%preprint,
%preprintnumbers,f
%nofootinbib,
%nobibnotes,
%bibnotes,
amsmath,
amssymb,
aps,
%pra,
prb,
%rmp,
%prstab,
%prstper,
%floatfix,
]{revtex4-2}

\usepackage{graphicx}% Include figure files
\usepackage{dcolumn}% Align table columns on decimal point
\usepackage{bm}% bold math
\usepackage{physics}
\usepackage{amsmath}
\usepackage{xcolor}
%\usepackage{hyperref}% add hypertext capabilities
%\usepackage[mathlines]{lineno}% Enable numbering of text and display math
%\linenumbers\relax % Commence numbering lines

%\usepackage[showframe,%Uncomment any one of the following lines to test 
%%scale=0.7, marginratio={1:1, 2:3}, ignoreall,% default settings
%%text={7in,10in},centering,
%%margin=1.5in,
%%total={6.5in,8.75in}, top=1.2in, left=0.9in, includefoot,
%%height=10in,a5paper,hmargin={3cm,0.8in},
%]{geometry}

\begin{document}

\preprint{APS/123-QED}

\title{Absence of topological protection of the interface states in $\mathbb{Z}_2$ photonic crystals}% Force line breaks with \\
% \thanks{A footnote to the article title}%

\author{Shupeng Xu}
\author{Yuhui Wang}%
\author{Ritesh Agarwal}
\email{riteshag@seas.upenn.edu}
\affiliation{
Department of Materials Science and Engineering, University of Pennsylvania,\\Philadelphia, 19104, PA, US}
\date{\today}% It is always \today, today,
             %  but any date may be explicitly specified

\begin{abstract}
Inspired from electronic systems, topological photonics aims to engineer new optical devices with robust properties. In many cases, the ideas from topological phases protected by internal symmetries in fermionic systems are extended to those protected by crystalline symmetries. One such popular photonic crystal model was proposed by Wu and Hu in 2015 for realizing a bosonic $\mathbb{Z}_2$ topological crystalline insulator with robust topological edge states, which led to intense theoretical and experimental studies. 
However, rigorous relationship between the bulk topology and edge properties for this model, which is central to evaluating its advantage over traditional photonic designs, has never been established.
%However, the relation between the topology and its edge physics has rarely been discussed carefully. 
In this work we revisit the expanded and shrunken honeycomb lattice structures proposed by Wu and Hu by using topological quantum chemistry tools and show that they are topologically trivial in the sense that symmetric, localized Wannier functions can be constructed. We show that the $\mathbb{Z}$ and $\mathbb{Z}_2$ type classification of the Wu-Hu model are equivalent to the $C_2T$ protected Euler class and the second Stiefel-Whitney class respectively, with the latter characterizing the full valence bands of Wu-Hu model indicating only a higher order topological insulator (HOTI) phase. We show that the Wu-Hu interface states can be gapped by a uniform topology preserving $C_6$ and $T$ symmetric perturbation, which demonstrates the trivial nature of the interface. Our results reveals that topology is not a necessary condition for the reported helical edge states in many photonics systems and opens new possibilities for interface engineering that may not be constrained to require topological designs.
% \begin{description}
% \item[Usage]
% Secondary publications and information retrieval purposes.
% \item[Structure]
% You may use the \texttt{description} environment to structure your abstract;
% use the optional argument of the \verb+\item+ command to give the category of each item. 
% \end{description}
\end{abstract}

%\keywords{Suggested keywords}%Use showkeys class option if keyword
                              %display desired
\maketitle

%\tableofcontents
Topological photonics began with the seminal work by Raghu and Haldane \cite{raghu2008analogs,haldane2008possible} where the idea of topology in the electronic band structures were generalized to waves in periodic media, leading the way for realizing topological phenomena in artificial structures \cite{lu2014topological,ozawa2019topological,zhang2018topological}.
After the early explorations of photonic Chern insulators where the time-reversal symmetry is explicitly broken \cite{wang2008reflection, wang2009observation},
with the discovery of topological crystalline insulators (TCIs) \cite{fu2011topological}, the topological phases were significantly enriched beyond the ten-fold way classification of topological insulators and superconductors \cite{chiu2016classification}. 
% which opened new opportunities in engineering topological phases in bosonic systems.

However, one has to be cautious when generalizing the ideas from the early examples of topological phases, especially to those that are protected by crystalline symmetries.
% However, caution has to be taken when generalizing the features in initially explored topological phases especially to those that are protected by crystalline symmetries.
For example, due to the fact that crystalline symmetry is often broken at a physical boundary, some TCIs only exhibit robust boundary states at certain crystal orientations \cite{fu2011topological}.
Moreover, with the discovery of novel states such as fragile topological phases \cite{po2018fragile, song2020twisted} and higher order topological insulators (HOTIs) \cite{benalcazar2017quantized, schindler2018higher}, the notion of bulk-boundary correspondence of codimension 1 may not have any direct generalization to TCIs at all.

The topological photonic crystal proposed by Wu and Hu \cite{wu2015scheme}, which we refer to as the Wu-Hu model, is an elegant structure for realizing a proposed bosonic analog of the fermionic $\mathbb{Z}_2$ TI (Fig. \ref{fig:fig1}). 
Hence it is claimed to host symmetry protected edge states which enable robust light transport free from back-scattering.
%with robust topological edge states at its boundary. 
The simplicity of the model triggered innumerous experimental and theoretical studies after its discovery \cite{barik2018topological,yang2018visualization, smirnova2019third,shao2020high, dikopoltsev2021topological, liu2020z2,liu2020generation, kumar2022terahertz, zhang2017topological, yang2020spin, liu2017pseudospins, pirie2022topological, cha2018experimental, he2016acoustic, li2021experimental, barik2016two}.
With the development of new theoretical frameworks such as topological quantum chemistry (TQC) \cite{bradlyn2017topological,cano2018building} and the Wilson loop method \cite{alexandradinata2014wilson}, by which one can systematically diagnose nontrivial topology from symmetry representations and Wannier obstruction,
% and wavefunction based calculations, 
the nature of the bulk topology of Wu-Hu model has been discussed in some recent papers \cite{de2019engineering,palmer2021berry}. 
However, the exact bulk-boundary correspondence has never been identified,
therefore the robustness of the edge states and their relation to the bulk topology remain unclear.
Here we revisit the Wu-Hu model and analyse the nature of its topology with a special emphasis on the edge properties.
Our main result is that, although being topologically distinct, the two phases of Wu-Hu model are trivial in the sense that symmetric localized Wannier functions exist, and both phases do not enforce any protection to the interface states.
We show that the ``topological'' interface states can be drastically gapped even by a uniform symmetric perturbation across the interface, and the associated ``topological'' properties can be well reproduced by trivial defect states.

We briefly review the original formalism of the Wu-Hu model as the foundation of discussion.
The tight-binding model of an expanded or shrunken honeycomb lattice provides a faithful description of Wu-Hu model, in which the unit cell for a graphene lattice is enlarged to include six atomic sites, and the couplings are divided into intra- ($t_1$) and intercell ($t_2$) couplings (Fig. \ref{fig:fig1}a).
%This enlarged unit cell forms a triangular lattice, with the Wyckoff position and Brillouin zone illustrated in Figure 1b.
When $t_1=t_2$, a four-fold degeneracy appears at the $\Gamma$ point, which gives rise to a ``double Dirac-cone". 
The cell-periodic part of the degenerate Bloch functions have the symmetries of $\ket{p_\pm}$ and $\ket{d_\pm}$ orbitals, and a gap opening and band inversion can be achieved by tuning the relative magnitudes of $t_1$ and $t_2$ (Figs. \ref{fig:fig1}c,e).

\begin{figure}
    \centering
    \includegraphics[width = 0.5\textwidth]{fig1_1222.pdf}
    \caption{(a) A schematic of the Wu-Hu lattice, the shadowed area indicates the hexagonal unit cell, $t_1$ ($t_2$) correspond to intracell (intercell) couplings.
    When each site moves away from (towards) the unit cell center, $t_1<t_2$ ($t_1>t_2$), and is referred to as an expanded (shrunken) phase.
    Note, when $t_1 = t_2$ this structure corresponds to photonic graphene.
    (b) (top) $1a$, $2b$, $3c$ Wyckoff positions of the unit cell color coded in black, dark gray and light gray, and located at the center, vertices and edges, respectively. (bottom) Brillouin zone of a triangular lattice. 
    (c,d) The band structure of an expanded phase and its corresponding Wilson loop. 
    (e,f) The band structure of a shrunken phase and its corresponding Wilson loop. Irreps are noted in the band diagrams at each high symmetry point.
    Note in (c) and (e), $\Gamma_5$ and $\Gamma_6$ are representations of $d$ and $p$ orbital states, respectively, therefore showing the band inversion.
    In (d) and (f), both phases show trivial Wilson loop without winding from $-\pi$ to $\pi$.}
    \label{fig:fig1}
\end{figure}

At high symmetry momenta, a composite pseudo-fermionic time-reversal symmetry $\mathcal{\tilde{T}}^2=-1$ was constructed and the $\mathbb{Z}_2$ topology was derived through the analogy to the spinful case.
For example, at the $\Gamma$ point in the $(\ket{p_x}, \ket{p_y})$ basis, the pseudo time-reversal operator is given by
\begin{equation}
    \label{eq:pseudo T}
    \mathcal{\tilde{T}}=\mathcal{UK}=[D_{E_1}(C_6)+D_{E_1}(C_6^2)]/\sqrt{3}\cdot \mathcal{K}=-i\sigma_y \mathcal{K}
\end{equation}
in which $E_1$ is the irreducible representation (irrep) for $6mm1'$ (the little co-group at $\Gamma$) furnished by $(\ket{p_x}, \ket{p_y})$ orbitals and $D_{E_1}(C_6)$ is the corresponding matrix representation of $C_6$, $\mathcal{K}$ is the bosonic time-reversal operator functioning as complex conjugate.
Eq.(\ref{eq:pseudo T}) satisfies $\mathcal{\tilde{T}}^2=-1$ and thus protects Kramer's degeneracy at $\Gamma$ point. 
Similar operator were also constructed for $\ket{d}$ orbitals, and for $K$ and $K'$ points in the Brillouin zone (BZ).

The $\mathbb{Z}_2$ index was obtained through the parity of spin-Chern number for each pseudo-spin channel where the $\ket{p_+}(\ket{p_-})$ and $\ket{d_+}(\ket{d_-})$ orbitals are assigned with pseudo-spin up(down) \cite{wu2015scheme, barik2016two}. 
The bulk-boundary correspondence of the 2D spinful TI was directly applied in the original proposal.
The interface states between different phases of Wu-Hu model were claimed to be gapless (with a tiny gap due to the $C_6$ breaking at the interface), immune from back-scattering and possessing spin-momentum locking.
% \subsection{\label{sec:EBR} Bulk Topology of the Wu-Hu model}

% As can be seen, 
% Instead of algebraic or physical arguments, the properties of Wu-Hu model are only based on the analogy between Eq.(\ref{eq:pseudo T}) and the fermionic time-reversal operator.
%But unlike the latter that bridges the Bloch functions at $\mathbf{k}$ and $-\mathbf{k}$, Eq.\ref{eq:pseudo T} failed to be a symmetry of the system immediately away from the $\Gamma$ point.
It is however not fully justified why Eq.(\ref{eq:pseudo T}) would constrain the global algebraic classification of Bloch functions and imply physical consequences exactly the same as the time-reversal symmetry in spinful systems.
% For gapless boundary states especially, the bulk-edge correspondence is often established through the Wannier obstruction associated with non-trivial topology. 
Here, we examine the topology of the Wu-Hu model based on Wannier obstruction using TQC \cite{bradlyn2017topological,cano2018building} and Wilson loop methods \cite{alexandradinata2014wilson}.
% which are two important tools to diagnose non-trivial topology with Wannier obstruction when crystalline symmetry is involved.
The Wannier obstruction is important because it can be directly related to the topological boundary states \cite{alexandradinata2016topological, fidkowski2011model}.
It has been recently shown that for continuum experimental systems the Wannier obstruction is a necessary condition for the robustness of the interface states \cite{alexandradinata2020crystallographic}, which is of utmost importance.

In TQC theory \cite{bradlyn2017topological,cano2018building}, the symmetry properties of the Bloch functions of Wannier-representable bands is equivalent to a direct sum of elementary band representations (EBRs).
Throughout the BZ, the symmetry properties can be well described by the collection of irreducible representations (irreps) furnished by the Bloch functions for the little groups at high symmetry momenta.
In Figs. \ref{fig:fig1}c,e, we calculate the irreps at high symmetry momenta for both shrunken and expanded phases in the Wu-Hu model 
and the relevant EBRs are listed in Table.\ref{tab:band_rep} \cite{bradlyn2017topological,aroyo2006bilbao1,aroyo2006bilbao2,aroyo2011crystallography}.
For the valence bands (VBs), we obtain $(A_1\uparrow G)_{1a} \oplus (E_1\uparrow G)_{1a}$ for the shrunken case and $(A_1 \uparrow G)_{3c}$ for the expanded case, respectively.
Surprisingly, VBs for both phases transform as a direct sum of EBRs, which suggests the trivial nature of the bulk topology.


We also calculated the phase of the eigenvalues of the Wilson loop operator, which is defined by the following path ordered integral \cite{alexandradinata2014wilson},
 \begin{equation}
     \mathcal{W}_C=\mathcal{P}\exp\left[ i\oint_C \mathbf{A}(\mathbf{k})\cdot d\mathbf{k} \right]
 \end{equation}
where $[\mathbf{A}(\mathbf{k})]_{mn}=i\bra{u_m(\mathbf{k})}\nabla_\mathbf{k}\ket{u_n(\mathbf{k})}$ is the non-Abelian Berry connection for the full VBs.
Fig. \ref{fig:fig1}b shows the geometry of the Wilson loop, where the closed loop $C$ is defined by the reciprocal lattice vector $G_1$ and the spectra is plotted as the loop moves along $G_2$ (Figs. \ref{fig:fig1}d,f).
%The results are shown in Figs. 1d,f.
For both phases of the Wu-Hu model, no winding is observed, which also suggests that the whole VBs can be smoothly deformed into a trivial atomic insulator.

% To further demonstrate the topologically trivial nature of both phases, we note that symmetric localized Wannier functions can be explicitly constructed.
% For the expanded case, when adiabatically turning off the intra-cell coupling $t_1$, the real space would become a collection of decoupled dimers localized at $3c$ Wyckoff position (Figs. \ref{fig:fig1}a, b). 
% Their symmetric and anti-symmetric eigenstates constitute two sets of flat bands in momentum space.
% For the shrunken case, we can turn off the inter-cell coupling $t_2$ and end up with Wannier functions localized at $1a$ position.
% During the above described deformation process, the gap between the VBs and conduction bands (CBs) remains open so the topology is preserved, and the obtained Wannier centers match well with the EBR analysis.

\begin{figure}
    \centering
    \includegraphics[width = 0.47\textwidth]{fig2_1222.pdf}
    \caption{(a),(b) Demonstration of two distinct edge configurations. Red (blue) sites correspond to the expanded (shrunken) phase, and a complete hexagonal unit cell is marked in the figure. In (a), the edge cuts through $3c$ Wyckoff position whereas in (b) through $1a$ Wyckoff position. Note that after cutting, two edge states with open boundary conditions are created in each configuration. (c),(e) Energy dispersion in a strip geometry with edge configuration shown in (a). The gapped edge states only show up in the expanded phase. (d),(f) Energy dispersion with edge configuration shown in (b). The gapped edge states only show up in the shrunken phase.
}
    \label{fig:fig2}
\end{figure}

\begin{table}[b]
\centering
\caption{\label{tab:band_rep} \textbf{The EBRs for space group $P6mm1'$ (the symmetry of the Wu-Hu model).} The EBRs are induced representations of localized orbitals and are labeled by $(\rho\uparrow G)_p$ in which $p$ is the Wyckoff position where the orbitals sit, $\rho$ is the irrep furnished by the orbitals and $G$ is the space group of the system.}
% \begin{ruledtabular}
\begin{tabular}{ c c c c }
\hline
\textrm{Band-rep.}&
\textrm{$(A_1 \uparrow G)_{1a}$}&
\multicolumn{1}{c}{\textrm{$(E_1 \uparrow G)_{1a}$}}&
\textrm{$(A_1 \uparrow G)_{3c}$}\\
\hline
$\Gamma$ & $\Gamma_{1}$ & {$\Gamma_6$} & $\Gamma_1 \oplus \Gamma_5$\\
$K$ & $K_1$ & {$K_3$} & $K_1\oplus K_3$\\
$M$ & $M_1$ & $M_{3}\oplus M_{4}$ & $M_{1}\oplus M_{3}\oplus M_{4}$\\
\hline
\end{tabular}
% \end{ruledtabular}
\end{table}

Next, we briefly discuss the topological invariants for the VBs of the Wu-Hu model.
% {\color{red} We note that aside from the spin-Chern method originally proposed, a direct generalization of the Fu-Kane-Mele $\mathbb{Z}_2$ index \cite{fu2007topological} obtained from $C_2$ eigenvalues of spinless system was used to characterize the bulk topology and explain the edge states \cite{liu2019helical}.}
In Supplementary materials \cite{supp} we prove that the spin-Chern number and the $\mathbb{Z}_2$ index defined for Wu-Hu model are equivalent to the Euler class and the second Stiefel-Whitney class protected by $C_2\mathcal{T}$ symmetry \cite{ahn2019stiefel}.
In 2D systems with $C_2\mathcal{T}$ symmetry, the Euler class is a $\mathbb{Z}$ type fragile topological invariant defined in a two-band subspace \cite{ahn2019stiefel, cano2021band}.
A non-zero Euler class forbids the construction of symmetric localized Wannier functions, however, this obstruction may be lifted by adding trivial bands.
In this many-band limit, the parity of the Euler class becomes the well defined $\mathbb{Z}_2$ type second Stiefel-Whitney class, $\omega_2$.
The expanded phase belongs to this category and is characterized by a non-trivial $\omega_2=1$ which indicates an obstructed atomic limit (OAL), meaning that the Wannier functions cannot be localized at the center of the unit cell. 
The associated physical consequence is a quantized quadrupole moment and fractional corner charges, in other words, $\omega_2=1$ characterizes a HOTI \cite{ahn2019failure, ahn2019stiefel}.

In fact, the VBs for both phases of Wu-Hu model can be deformed into decoupled atomic clusters by selectively turning off intra- or inter-cell couplings. 
With all the above observations we conclude that both phases of the Wu-Hu model are topologically trivial in terms of Wannier obstruction, therefore neither of the two phases are responsible for the gapless interface states.
This can be easily demonstrated in the tight binding limit.
Starting with the gapless interface and adiabatically turning off the couplings connecting two phases to form the open boundary conditions (OBCs), the interface states would be in general gapped and pushed towards the bulk bands. 
%Calculation shows that the gapless states would be gradually gapped but remain exist in the bulk band gap.
If the interface states result from the nontrivial bulk topology, we can keep track of them and the gapped interface states should be localized exactly at the non-trivial half of the system.
However, depending on the edge configuration, the interface states of Wu-Hu model can be localized at different phases.
In Fig. \ref{fig:fig2}, we show that
%the dispersion of the edge states for the OBCs with different edge configurations:
for the shrunken phase where the Wannier center sits at $1a$ Wyckoff position, the gapped edge states appear when the boundary cuts through $1a$ position; for the expanded case where the Wannier center sits at $3c$ Wyckoff position, the gapped edge states appear when the edge cuts through $3c$ Wyckoff position.
This observation strongly suggests that the interface states are originated from the local defects in contrast to the well-known topological boundary states arising from the bulk Wannier obstruction \cite{alexandradinata2016topological, fidkowski2011model}. 

%\subsection{\label{sec:edge state} Edge States}
%Here we further examine the properties of the Wu-Hu interface.

% Typically, the interface states are explained by the direct generalization of the bulk-boundary correspondence of the 2D spinful TI protected by $\mathcal{T}^2=-1$.
% However, the Kramer's degeneracy in 1D BZ cannot be protected by the composite pseudo-fermionic time-reversal operator in the Wu-Hu model invalidating the generalization.

The Kramer's degeneracy in 1D BZ cannot be protected by composite pseudo time-reversal symmetry in the Wu-Hu model hence invalidating the direct generalization from the 2D spinful TI protected by $\mathcal{T}^2=-1$.
An alternate interpretation explains the interface states as the Jackiw-Rebbi soliton eigen-solutions that arise from a local band inversion \cite{barik2016two}.
%to the topological distinction between the two phases.
%Gapless modes are protected in an interface slowly interpolating between two phases with different topologies. 
%However, this is in general not true for a sharp interface with a plethora of counterexamples considering HOTI and fragile topology, so that the robustness of the interface states is unclear.
%Aside from a topological origin, the Wu-Hu interface states have also been interpreted as the Jackiw-Rebbi soliton eigen-solutions that arise from a local band inversion.
%We note that Jackiw-Rebbi solutions fail to carry any information about the protection provided by rotational symmetry.
However, since the Jackiw-Rebbi solutions give one set of interface states for each pseudo-spin, spin-mixing can potentially gap out the interface states, and the symmetry that protects the bulk topology in the Wu-Hu model, namely $C_6$ and $\mathcal{T}^2=1$, does not imply spin-conservation.
%To further demonstrate the lack of robustness in the Wu-Hu interface, we explicitly show that they can be gapped considerably by a $C_6$ and $\mathcal{T}^2=1$ symmetric perturbation that is uniform across the interface (Fig. 3).
%We note that for both spin-Chern edge states and Jackiw-Rebbi solutions in bosonic systems, spin-mixing can potentially gap out the interface states, and the symmetry that protects the bulk topology in WU-Hu model, namely $C_6$ and $\mathcal{T}^2=1$, does not imply spin-conservation.
Consider the Wu-Hu model in its quasi-orbital basis, where $\ket{p_\pm}$ and $\ket{d_\pm}$ orbitals sit at 1a Wyckoff position of a triangular lattice. 
The spin flipping terms are locally forbidden by $C_6$ symmetry since different pseudo-spin orbitals belong to irreps with different $C_6$ eigenvalues.
However, the following non-local spin-flip channel is always allowed by $C_6$ symmetry,
\begin{equation}
    \Delta=t a^\dag_{i,\pm}a_{j,\mp} + h.c., \ i\neq j
\end{equation}
where $i,j$ are labels of unit cells, $\pm$ are labels for pseudo-spin, and $h.c.$ stands for hermitian conjugate.

To further demonstrate the lack of robustness in the Wu-Hu interface, we explicitly show that they can be gapped considerably even by a $C_6$ and $\mathcal{T}^2=1$ symmetric perturbation uniform across the interface (Fig. \ref{fig:fig3}).
%Here we deform the original the Wu-Hu Hamiltonian by a $C_6$ and $\mathcal{T}$ symmetric perturbation and study the interface states.
The perturbation is added to ensure that when $t_1=t_2$, a double Dirac-cone appears at $\Gamma$ point.
The band inversion is then achieved by tuning the relative magnitude of $t_1$ and $t_2$ (see Supplementary materials \cite{supp}).
Therefore the original Wu-Hu Hamiltonian is explicitly included and the perturbation is uniform across the interface preserving $C_6$ symmetry. 
Importantly, this perturbation can be viewed as being adiabatically added to the unperturbed Hamiltonian and no gap closing ever happened between the VBs and CBs, thus the topology is preserved.
%The details of the model can be found at Appendix \ref{app:detail}.
For a system with an interface, we write the perturbed Hamiltonian as:
\begin{equation}
    H'=H_0 + \Delta H
\end{equation}
in which $H_0$ describes the unperturbed interface of the Wu-Hu model and $\Delta H$ is the perturbation.
The spectra of interface states for $H'$ and $H_0$ are shown in Fig. \ref{fig:fig3}.
For $H_0$, there exists a gap at zero energy that is hardly visible (as observed in the Wu-Hu model \cite{wu2015scheme}), whereas for $H'$, the gap is clear and considerably large compared to the bulk band gap.
Pseudo-spin character of the interface states also shows clear mixing for $H'$ compared to $H_0$, which is consistent with the argument that $C_6$ symmetry does not imply spin-conservation.
All these observations strongly suggests that, aside from $C_6$ symmetry breaking, other mechanisms can open a gap for the interface states, therefore showing the absence of topological protection in the system clearly.

\begin{figure}
    \centering
    \includegraphics[width = 0.4\textwidth]{fig3_1222.pdf}
    \caption{The dispersion of interface states where the pseudo-spin component is color coded. (a) The interface states of an unperturbed Wu-Hu interface. The dispersion is nearly linear and the gap is not visible in the figure. (b) Perturbed Wu-Hu interface. An apparent gap is opened with magnitude comparable to the bulk band gap. The pseudo-spins are mixed showing lighter color. $a$ is the lattice constant.}
    \label{fig:fig3}
\end{figure}

In addition, the Wu-Hu interface and the edge of 2D TIs protected by $\mathcal{T}^2=-1$ are compared to discuss the relation between their properties and topology.
The three properties concerned here are spectral robustness, immunity from back-scattering and spin-momentum locking.
For the $\mathcal{T}^2=-1$ 2D TI, the bulk-boundary correspondence can be understood by the topological equivalence between the edge spectrum and the Wilson loop spectrum, which has a stable winding protected by Wannier obstruction in the non-trivial phase, so that the spectral robustness of the edge is guaranteed \cite{alexandradinata2016topological, fidkowski2011model}.
The immunity of back-scattering is then followed as a combined effect of $\mathcal{T}^2=-1$ and the presence of odd number of edge states \cite{kane2005quantum}.
Lastly, instead of a unique topological phenomenon, the spin-momentum locking is a prevalent feature in edge modes with strong spin-orbit coupling.
To conclude, only spectral robustness is directly related to topology.
Moreover, in bosonic systems with $\mathcal{T}^2=1$ symmetry, back-scattering can still be present even in edge states with topologically protected spectral robustness.
%Moreover, the fermionic time-reversal symmetry is crucial in the argument of forbidden back-scattering, which means that even if we do have bosonic edge states with topologically protected spectral robustness, the immunity of back-scattering cannot be expected. 
For Wu-Hu interface states, this argument agrees well with recent experimental results \cite{orazbayev2019quantitative}.

%\subsection{\label{sec:sim} Numerical simulations of trivial edge states}
From a practical perspective, these gapless, back-scattering free and spin-momentum locking interface states are what make the Wu-Hu model promising for photonic applications.
Among which only spectral robustness is possible to be directly linked with topology. 
% The implication of our results are twofold: first, for Wu-Hu interface, more careful examinations are needed instead of attributing all the observations to topology;
% second, the fact that non-trivial topology is not a necessary condition for these properties opens many opportunities for interface engineering not constrained by this model.
The unnecessity of topology at the Wu-Hu interface enables more flexible designs of combining different bulk structures without any symmetry consideration, which may lead to novel applications such as photonic on-chip logic and reconfigurable light routing.
Here we numerically demonstrate in a 2D photonic crystal helical edge states that solely stem from the trivial phase of Wu-Hu model with OBCs that reproduces all the features of the claimed ``topological'' Wu-Hu interface. Structures and corresponding bulk band diagrams can be found in Supplementary materials \cite{supp}.

\begin{figure}
    \centering
    \includegraphics[width = 0.5\textwidth]{fig4_1222.pdf}
    \caption{(a) The schematic of the strip geometry applied for the edge states of a shrunken phase, atoms from bulk complete (edge incomplete) unit cell are colored in blue (red). The edge is created by cutting through the 1a position, then a slight tuning is applied to the incomplete unit cells at the edge. The direction of $\delta x,\ \delta y$ is also noted.  (b) Numerically calculated interface dispersion, showing two in-gap linear modes. $a$ is the lattice constant (c) Large scale simulation of the propagation of a trivial edge state from a circularly polarized source. The open boundary turning is marked in a white dashed line.
}
    \label{fig:fig4}
\end{figure}

Edge states are created from the shrunken phase by cutting through the Wannier center of the VBs, namely $1a$ Wyckoff position, forming incomplete unit cells. (Fig. \ref{fig:fig4}a)
Being of the defect nature, the resulted edge states are highly tunable that they can be tuned to be gapless by simply displacing the sites in the incomplete unit cells.
%In the tight binding calculation, the wavefunctions of the edge states also exhibit  a phase vortex which implies spin-momentum locking in photonics structures.
% We simulated a 2D photonic crystal where the dielectric rods sit at each atomic sites with the OBCs to demonstrate the trivial spin-momentum locked edge state.
We first calculated the dispersion spectrum of a strip geometry of this trivial edge (Figs. \ref{fig:fig4}a, b), with Bloch boundary condition applied in the x-direction and OBCs in the y-directions (see Supplementary materials \cite{supp} for detailed simulation setup).
Two edge states emerge in the dispersion inside the bulk gap (Fig. \ref{fig:fig4}b), showing a dirac-cone shaped crossing.
Then we performed a large scale simulation of the edge states with a sharp bend excited with a circularly polarized source (Fig. \ref{fig:fig4}c).
The unidirectional propagation is clearly observed along the sample edge, showing that topology is not required for a helical photonic edge.
% \section{\label{conclusion} Discussion and Conclusions}

In conclusion, we re-examined the Wu-Hu model and identified the algebraic nature of the topological invariants and the associated physical consequences.
We showed the lack of robustness of its interface states against symmetry preserving perturbations and explicitly constructed a trivial defect edge that reproduces all the ``topological'' properties. 
However, the following question remains interesting and unanswered: for TCIs, whether, and to what extent, Wannier obstructions would provide protection to the the interface in the domain wall configuration similar to the Wu-Hu model where the bulk symmetry is partially restored by the addition of a trivial phase.
In fact, the existence of such topological protection is an implicit assumption for the topological interpretation of Wu-Hu interface.
If this protection does not exist even when one of the phases is stably obstructed, the topological interpretation of Wu-Hu interface would fail at the first step.
Based on our arguments, one cannot distinguish whether the trivial nature of the VBs or the absence of topological protection itself is the fundamental reason that is responsible for the gap opening.
The rigorous discussions of similar questions has only appeared recently \cite{alexandradinata2020crystallographic}, and we hope our results as a case study can provide some insights to future studies.
% For photonic waveguide engineering, our results clearly show that there is no causal relation between the topology of Wu-Hu model and the desired properties at its interface.
% In fact, perfect transmission at sharp bends can be achieved in traditional photonic crystals and spin-momentum locking is a prevalent feature for photonic systems \cite{van2016universal}.
% The unnecessity of topology at the Wu-Hu interface enables more flexible designs of combining different bulk structures without any symmetry consideration, which may lead to novel applications such as photonic on-chip logic and reconfigurable light routing.

\begin{acknowledgments}
This work was supported by the Office of Naval Research via grant No.N00014-22-1-2378. S.X. and Y.W. contributed equally to this work.
\end{acknowledgments}

% The \nocite command causes all entries in a bibliography to be printed out
% whether or not they are actually referenced in the text. This is appropriate
% for the sample file to show the different styles of references, but authors
% most likely will not want to use it.
% \nocite{*}

%apsrev4-2.bst 2019-01-14 (MD) hand-edited version of apsrev4-1.bst
%Control: key (0)
%Control: author (8) initials jnrlst
%Control: editor formatted (1) identically to author
%Control: production of article title (0) allowed
%Control: page (0) single
%Control: year (1) truncated
%Control: production of eprint (0) enabled
\begin{thebibliography}{47}%
\makeatletter
\providecommand \@ifxundefined [1]{%
 \@ifx{#1\undefined}
}%
\providecommand \@ifnum [1]{%
 \ifnum #1\expandafter \@firstoftwo
 \else \expandafter \@secondoftwo
 \fi
}%
\providecommand \@ifx [1]{%
 \ifx #1\expandafter \@firstoftwo
 \else \expandafter \@secondoftwo
 \fi
}%
\providecommand \natexlab [1]{#1}%
\providecommand \enquote  [1]{``#1''}%
\providecommand \bibnamefont  [1]{#1}%
\providecommand \bibfnamefont [1]{#1}%
\providecommand \citenamefont [1]{#1}%
\providecommand \href@noop [0]{\@secondoftwo}%
\providecommand \href [0]{\begingroup \@sanitize@url \@href}%
\providecommand \@href[1]{\@@startlink{#1}\@@href}%
\providecommand \@@href[1]{\endgroup#1\@@endlink}%
\providecommand \@sanitize@url [0]{\catcode `\\12\catcode `\$12\catcode
  `\&12\catcode `\#12\catcode `\^12\catcode `\_12\catcode `\%12\relax}%
\providecommand \@@startlink[1]{}%
\providecommand \@@endlink[0]{}%
\providecommand \url  [0]{\begingroup\@sanitize@url \@url }%
\providecommand \@url [1]{\endgroup\@href {#1}{\urlprefix }}%
\providecommand \urlprefix  [0]{URL }%
\providecommand \Eprint [0]{\href }%
\providecommand \doibase [0]{https://doi.org/}%
\providecommand \selectlanguage [0]{\@gobble}%
\providecommand \bibinfo  [0]{\@secondoftwo}%
\providecommand \bibfield  [0]{\@secondoftwo}%
\providecommand \translation [1]{[#1]}%
\providecommand \BibitemOpen [0]{}%
\providecommand \bibitemStop [0]{}%
\providecommand \bibitemNoStop [0]{.\EOS\space}%
\providecommand \EOS [0]{\spacefactor3000\relax}%
\providecommand \BibitemShut  [1]{\csname bibitem#1\endcsname}%
\let\auto@bib@innerbib\@empty
%</preamble>
\bibitem [{\citenamefont {Raghu}\ and\ \citenamefont
  {Haldane}(2008)}]{raghu2008analogs}%
  \BibitemOpen
  \bibfield  {author} {\bibinfo {author} {\bibfnamefont {S.}~\bibnamefont
  {Raghu}}\ and\ \bibinfo {author} {\bibfnamefont {F.~D.~M.}\ \bibnamefont
  {Haldane}},\ }\bibfield  {title} {\bibinfo {title} {Analogs of
  quantum-hall-effect edge states in photonic crystals},\ }\href@noop {}
  {\bibfield  {journal} {\bibinfo  {journal} {Physical Review A}\ }\textbf
  {\bibinfo {volume} {78}},\ \bibinfo {pages} {033834} (\bibinfo {year}
  {2008})}\BibitemShut {NoStop}%
\bibitem [{\citenamefont {Haldane}\ and\ \citenamefont
  {Raghu}(2008)}]{haldane2008possible}%
  \BibitemOpen
  \bibfield  {author} {\bibinfo {author} {\bibfnamefont {F.}~\bibnamefont
  {Haldane}}\ and\ \bibinfo {author} {\bibfnamefont {S.}~\bibnamefont
  {Raghu}},\ }\bibfield  {title} {\bibinfo {title} {Possible realization of
  directional optical waveguides in photonic crystals with broken time-reversal
  symmetry},\ }\href@noop {} {\bibfield  {journal} {\bibinfo  {journal}
  {Physical review letters}\ }\textbf {\bibinfo {volume} {100}},\ \bibinfo
  {pages} {013904} (\bibinfo {year} {2008})}\BibitemShut {NoStop}%
\bibitem [{\citenamefont {Lu}\ \emph {et~al.}(2014)\citenamefont {Lu},
  \citenamefont {Joannopoulos},\ and\ \citenamefont
  {Solja{\v{c}}i{\'c}}}]{lu2014topological}%
  \BibitemOpen
  \bibfield  {author} {\bibinfo {author} {\bibfnamefont {L.}~\bibnamefont
  {Lu}}, \bibinfo {author} {\bibfnamefont {J.~D.}\ \bibnamefont
  {Joannopoulos}},\ and\ \bibinfo {author} {\bibfnamefont {M.}~\bibnamefont
  {Solja{\v{c}}i{\'c}}},\ }\bibfield  {title} {\bibinfo {title} {Topological
  photonics},\ }\href@noop {} {\bibfield  {journal} {\bibinfo  {journal}
  {Nature photonics}\ }\textbf {\bibinfo {volume} {8}},\ \bibinfo {pages} {821}
  (\bibinfo {year} {2014})}\BibitemShut {NoStop}%
\bibitem [{\citenamefont {Ozawa}\ \emph {et~al.}(2019)\citenamefont {Ozawa},
  \citenamefont {Price}, \citenamefont {Amo}, \citenamefont {Goldman},
  \citenamefont {Hafezi}, \citenamefont {Lu}, \citenamefont {Rechtsman},
  \citenamefont {Schuster}, \citenamefont {Simon}, \citenamefont {Zilberberg}
  \emph {et~al.}}]{ozawa2019topological}%
  \BibitemOpen
  \bibfield  {author} {\bibinfo {author} {\bibfnamefont {T.}~\bibnamefont
  {Ozawa}}, \bibinfo {author} {\bibfnamefont {H.~M.}\ \bibnamefont {Price}},
  \bibinfo {author} {\bibfnamefont {A.}~\bibnamefont {Amo}}, \bibinfo {author}
  {\bibfnamefont {N.}~\bibnamefont {Goldman}}, \bibinfo {author} {\bibfnamefont
  {M.}~\bibnamefont {Hafezi}}, \bibinfo {author} {\bibfnamefont
  {L.}~\bibnamefont {Lu}}, \bibinfo {author} {\bibfnamefont {M.~C.}\
  \bibnamefont {Rechtsman}}, \bibinfo {author} {\bibfnamefont {D.}~\bibnamefont
  {Schuster}}, \bibinfo {author} {\bibfnamefont {J.}~\bibnamefont {Simon}},
  \bibinfo {author} {\bibfnamefont {O.}~\bibnamefont {Zilberberg}}, \emph
  {et~al.},\ }\bibfield  {title} {\bibinfo {title} {Topological photonics},\
  }\href@noop {} {\bibfield  {journal} {\bibinfo  {journal} {Reviews of Modern
  Physics}\ }\textbf {\bibinfo {volume} {91}},\ \bibinfo {pages} {015006}
  (\bibinfo {year} {2019})}\BibitemShut {NoStop}%
\bibitem [{\citenamefont {Zhang}\ \emph {et~al.}(2018)\citenamefont {Zhang},
  \citenamefont {Xiao}, \citenamefont {Cheng}, \citenamefont {Lu},\ and\
  \citenamefont {Christensen}}]{zhang2018topological}%
  \BibitemOpen
  \bibfield  {author} {\bibinfo {author} {\bibfnamefont {X.}~\bibnamefont
  {Zhang}}, \bibinfo {author} {\bibfnamefont {M.}~\bibnamefont {Xiao}},
  \bibinfo {author} {\bibfnamefont {Y.}~\bibnamefont {Cheng}}, \bibinfo
  {author} {\bibfnamefont {M.-H.}\ \bibnamefont {Lu}},\ and\ \bibinfo {author}
  {\bibfnamefont {J.}~\bibnamefont {Christensen}},\ }\bibfield  {title}
  {\bibinfo {title} {Topological sound},\ }\href@noop {} {\bibfield  {journal}
  {\bibinfo  {journal} {Communications Physics}\ }\textbf {\bibinfo {volume}
  {1}},\ \bibinfo {pages} {1} (\bibinfo {year} {2018})}\BibitemShut {NoStop}%
\bibitem [{\citenamefont {Wang}\ \emph {et~al.}(2008)\citenamefont {Wang},
  \citenamefont {Chong}, \citenamefont {Joannopoulos},\ and\ \citenamefont
  {Solja{\v{c}}i{\'c}}}]{wang2008reflection}%
  \BibitemOpen
  \bibfield  {author} {\bibinfo {author} {\bibfnamefont {Z.}~\bibnamefont
  {Wang}}, \bibinfo {author} {\bibfnamefont {Y.}~\bibnamefont {Chong}},
  \bibinfo {author} {\bibfnamefont {J.~D.}\ \bibnamefont {Joannopoulos}},\ and\
  \bibinfo {author} {\bibfnamefont {M.}~\bibnamefont {Solja{\v{c}}i{\'c}}},\
  }\bibfield  {title} {\bibinfo {title} {Reflection-free one-way edge modes in
  a gyromagnetic photonic crystal},\ }\href@noop {} {\bibfield  {journal}
  {\bibinfo  {journal} {Physical review letters}\ }\textbf {\bibinfo {volume}
  {100}},\ \bibinfo {pages} {013905} (\bibinfo {year} {2008})}\BibitemShut
  {NoStop}%
\bibitem [{\citenamefont {Wang}\ \emph {et~al.}(2009)\citenamefont {Wang},
  \citenamefont {Chong}, \citenamefont {Joannopoulos},\ and\ \citenamefont
  {Solja{\v{c}}i{\'c}}}]{wang2009observation}%
  \BibitemOpen
  \bibfield  {author} {\bibinfo {author} {\bibfnamefont {Z.}~\bibnamefont
  {Wang}}, \bibinfo {author} {\bibfnamefont {Y.}~\bibnamefont {Chong}},
  \bibinfo {author} {\bibfnamefont {J.~D.}\ \bibnamefont {Joannopoulos}},\ and\
  \bibinfo {author} {\bibfnamefont {M.}~\bibnamefont {Solja{\v{c}}i{\'c}}},\
  }\bibfield  {title} {\bibinfo {title} {Observation of unidirectional
  backscattering-immune topological electromagnetic states},\ }\href@noop {}
  {\bibfield  {journal} {\bibinfo  {journal} {Nature}\ }\textbf {\bibinfo
  {volume} {461}},\ \bibinfo {pages} {772} (\bibinfo {year}
  {2009})}\BibitemShut {NoStop}%
\bibitem [{\citenamefont {Fu}(2011)}]{fu2011topological}%
  \BibitemOpen
  \bibfield  {author} {\bibinfo {author} {\bibfnamefont {L.}~\bibnamefont
  {Fu}},\ }\bibfield  {title} {\bibinfo {title} {Topological crystalline
  insulators},\ }\href@noop {} {\bibfield  {journal} {\bibinfo  {journal}
  {Physical Review Letters}\ }\textbf {\bibinfo {volume} {106}},\ \bibinfo
  {pages} {106802} (\bibinfo {year} {2011})}\BibitemShut {NoStop}%
\bibitem [{\citenamefont {Chiu}\ \emph {et~al.}(2016)\citenamefont {Chiu},
  \citenamefont {Teo}, \citenamefont {Schnyder},\ and\ \citenamefont
  {Ryu}}]{chiu2016classification}%
  \BibitemOpen
  \bibfield  {author} {\bibinfo {author} {\bibfnamefont {C.-K.}\ \bibnamefont
  {Chiu}}, \bibinfo {author} {\bibfnamefont {J.~C.}\ \bibnamefont {Teo}},
  \bibinfo {author} {\bibfnamefont {A.~P.}\ \bibnamefont {Schnyder}},\ and\
  \bibinfo {author} {\bibfnamefont {S.}~\bibnamefont {Ryu}},\ }\bibfield
  {title} {\bibinfo {title} {Classification of topological quantum matter with
  symmetries},\ }\href@noop {} {\bibfield  {journal} {\bibinfo  {journal}
  {Reviews of Modern Physics}\ }\textbf {\bibinfo {volume} {88}},\ \bibinfo
  {pages} {035005} (\bibinfo {year} {2016})}\BibitemShut {NoStop}%
\bibitem [{\citenamefont {Po}\ \emph {et~al.}(2018)\citenamefont {Po},
  \citenamefont {Watanabe},\ and\ \citenamefont {Vishwanath}}]{po2018fragile}%
  \BibitemOpen
  \bibfield  {author} {\bibinfo {author} {\bibfnamefont {H.~C.}\ \bibnamefont
  {Po}}, \bibinfo {author} {\bibfnamefont {H.}~\bibnamefont {Watanabe}},\ and\
  \bibinfo {author} {\bibfnamefont {A.}~\bibnamefont {Vishwanath}},\ }\bibfield
   {title} {\bibinfo {title} {Fragile topology and wannier obstructions},\
  }\href@noop {} {\bibfield  {journal} {\bibinfo  {journal} {Physical review
  letters}\ }\textbf {\bibinfo {volume} {121}},\ \bibinfo {pages} {126402}
  (\bibinfo {year} {2018})}\BibitemShut {NoStop}%
\bibitem [{\citenamefont {Song}\ \emph {et~al.}(2020)\citenamefont {Song},
  \citenamefont {Elcoro},\ and\ \citenamefont {Bernevig}}]{song2020twisted}%
  \BibitemOpen
  \bibfield  {author} {\bibinfo {author} {\bibfnamefont {Z.-D.}\ \bibnamefont
  {Song}}, \bibinfo {author} {\bibfnamefont {L.}~\bibnamefont {Elcoro}},\ and\
  \bibinfo {author} {\bibfnamefont {B.~A.}\ \bibnamefont {Bernevig}},\
  }\bibfield  {title} {\bibinfo {title} {Twisted bulk-boundary correspondence
  of fragile topology},\ }\href@noop {} {\bibfield  {journal} {\bibinfo
  {journal} {Science}\ }\textbf {\bibinfo {volume} {367}},\ \bibinfo {pages}
  {794} (\bibinfo {year} {2020})}\BibitemShut {NoStop}%
\bibitem [{\citenamefont {Benalcazar}\ \emph {et~al.}(2017)\citenamefont
  {Benalcazar}, \citenamefont {Bernevig},\ and\ \citenamefont
  {Hughes}}]{benalcazar2017quantized}%
  \BibitemOpen
  \bibfield  {author} {\bibinfo {author} {\bibfnamefont {W.~A.}\ \bibnamefont
  {Benalcazar}}, \bibinfo {author} {\bibfnamefont {B.~A.}\ \bibnamefont
  {Bernevig}},\ and\ \bibinfo {author} {\bibfnamefont {T.~L.}\ \bibnamefont
  {Hughes}},\ }\bibfield  {title} {\bibinfo {title} {Quantized electric
  multipole insulators},\ }\href@noop {} {\bibfield  {journal} {\bibinfo
  {journal} {Science}\ }\textbf {\bibinfo {volume} {357}},\ \bibinfo {pages}
  {61} (\bibinfo {year} {2017})}\BibitemShut {NoStop}%
\bibitem [{\citenamefont {Schindler}\ \emph {et~al.}(2018)\citenamefont
  {Schindler}, \citenamefont {Cook}, \citenamefont {Vergniory}, \citenamefont
  {Wang}, \citenamefont {Parkin}, \citenamefont {Bernevig},\ and\ \citenamefont
  {Neupert}}]{schindler2018higher}%
  \BibitemOpen
  \bibfield  {author} {\bibinfo {author} {\bibfnamefont {F.}~\bibnamefont
  {Schindler}}, \bibinfo {author} {\bibfnamefont {A.~M.}\ \bibnamefont {Cook}},
  \bibinfo {author} {\bibfnamefont {M.~G.}\ \bibnamefont {Vergniory}}, \bibinfo
  {author} {\bibfnamefont {Z.}~\bibnamefont {Wang}}, \bibinfo {author}
  {\bibfnamefont {S.~S.}\ \bibnamefont {Parkin}}, \bibinfo {author}
  {\bibfnamefont {B.~A.}\ \bibnamefont {Bernevig}},\ and\ \bibinfo {author}
  {\bibfnamefont {T.}~\bibnamefont {Neupert}},\ }\bibfield  {title} {\bibinfo
  {title} {Higher-order topological insulators},\ }\href@noop {} {\bibfield
  {journal} {\bibinfo  {journal} {Science advances}\ }\textbf {\bibinfo
  {volume} {4}},\ \bibinfo {pages} {eaat0346} (\bibinfo {year}
  {2018})}\BibitemShut {NoStop}%
\bibitem [{\citenamefont {Wu}\ and\ \citenamefont {Hu}(2015)}]{wu2015scheme}%
  \BibitemOpen
  \bibfield  {author} {\bibinfo {author} {\bibfnamefont {L.-H.}\ \bibnamefont
  {Wu}}\ and\ \bibinfo {author} {\bibfnamefont {X.}~\bibnamefont {Hu}},\
  }\bibfield  {title} {\bibinfo {title} {Scheme for achieving a topological
  photonic crystal by using dielectric material},\ }\href@noop {} {\bibfield
  {journal} {\bibinfo  {journal} {Physical review letters}\ }\textbf {\bibinfo
  {volume} {114}},\ \bibinfo {pages} {223901} (\bibinfo {year}
  {2015})}\BibitemShut {NoStop}%
\bibitem [{\citenamefont {Barik}\ \emph {et~al.}(2018)\citenamefont {Barik},
  \citenamefont {Karasahin}, \citenamefont {Flower}, \citenamefont {Cai},
  \citenamefont {Miyake}, \citenamefont {DeGottardi}, \citenamefont {Hafezi},\
  and\ \citenamefont {Waks}}]{barik2018topological}%
  \BibitemOpen
  \bibfield  {author} {\bibinfo {author} {\bibfnamefont {S.}~\bibnamefont
  {Barik}}, \bibinfo {author} {\bibfnamefont {A.}~\bibnamefont {Karasahin}},
  \bibinfo {author} {\bibfnamefont {C.}~\bibnamefont {Flower}}, \bibinfo
  {author} {\bibfnamefont {T.}~\bibnamefont {Cai}}, \bibinfo {author}
  {\bibfnamefont {H.}~\bibnamefont {Miyake}}, \bibinfo {author} {\bibfnamefont
  {W.}~\bibnamefont {DeGottardi}}, \bibinfo {author} {\bibfnamefont
  {M.}~\bibnamefont {Hafezi}},\ and\ \bibinfo {author} {\bibfnamefont
  {E.}~\bibnamefont {Waks}},\ }\bibfield  {title} {\bibinfo {title} {A
  topological quantum optics interface},\ }\href@noop {} {\bibfield  {journal}
  {\bibinfo  {journal} {Science}\ }\textbf {\bibinfo {volume} {359}},\ \bibinfo
  {pages} {666} (\bibinfo {year} {2018})}\BibitemShut {NoStop}%
\bibitem [{\citenamefont {Yang}\ \emph {et~al.}(2018)\citenamefont {Yang},
  \citenamefont {Xu}, \citenamefont {Xu}, \citenamefont {Wang}, \citenamefont
  {Jiang}, \citenamefont {Hu},\ and\ \citenamefont
  {Hang}}]{yang2018visualization}%
  \BibitemOpen
  \bibfield  {author} {\bibinfo {author} {\bibfnamefont {Y.}~\bibnamefont
  {Yang}}, \bibinfo {author} {\bibfnamefont {Y.~F.}\ \bibnamefont {Xu}},
  \bibinfo {author} {\bibfnamefont {T.}~\bibnamefont {Xu}}, \bibinfo {author}
  {\bibfnamefont {H.-X.}\ \bibnamefont {Wang}}, \bibinfo {author}
  {\bibfnamefont {J.-H.}\ \bibnamefont {Jiang}}, \bibinfo {author}
  {\bibfnamefont {X.}~\bibnamefont {Hu}},\ and\ \bibinfo {author}
  {\bibfnamefont {Z.~H.}\ \bibnamefont {Hang}},\ }\bibfield  {title} {\bibinfo
  {title} {Visualization of a unidirectional electromagnetic waveguide using
  topological photonic crystals made of dielectric materials},\ }\href@noop {}
  {\bibfield  {journal} {\bibinfo  {journal} {Physical review letters}\
  }\textbf {\bibinfo {volume} {120}},\ \bibinfo {pages} {217401} (\bibinfo
  {year} {2018})}\BibitemShut {NoStop}%
\bibitem [{\citenamefont {Smirnova}\ \emph {et~al.}(2019)\citenamefont
  {Smirnova}, \citenamefont {Kruk}, \citenamefont {Leykam}, \citenamefont
  {Melik-Gaykazyan}, \citenamefont {Choi},\ and\ \citenamefont
  {Kivshar}}]{smirnova2019third}%
  \BibitemOpen
  \bibfield  {author} {\bibinfo {author} {\bibfnamefont {D.}~\bibnamefont
  {Smirnova}}, \bibinfo {author} {\bibfnamefont {S.}~\bibnamefont {Kruk}},
  \bibinfo {author} {\bibfnamefont {D.}~\bibnamefont {Leykam}}, \bibinfo
  {author} {\bibfnamefont {E.}~\bibnamefont {Melik-Gaykazyan}}, \bibinfo
  {author} {\bibfnamefont {D.-Y.}\ \bibnamefont {Choi}},\ and\ \bibinfo
  {author} {\bibfnamefont {Y.}~\bibnamefont {Kivshar}},\ }\bibfield  {title}
  {\bibinfo {title} {Third-harmonic generation in photonic topological
  metasurfaces},\ }\href@noop {} {\bibfield  {journal} {\bibinfo  {journal}
  {Physical review letters}\ }\textbf {\bibinfo {volume} {123}},\ \bibinfo
  {pages} {103901} (\bibinfo {year} {2019})}\BibitemShut {NoStop}%
\bibitem [{\citenamefont {Shao}\ \emph {et~al.}(2020)\citenamefont {Shao},
  \citenamefont {Chen}, \citenamefont {Wang}, \citenamefont {Mao},
  \citenamefont {Yang}, \citenamefont {Wang}, \citenamefont {Wang},
  \citenamefont {Hu},\ and\ \citenamefont {Ma}}]{shao2020high}%
  \BibitemOpen
  \bibfield  {author} {\bibinfo {author} {\bibfnamefont {Z.-K.}\ \bibnamefont
  {Shao}}, \bibinfo {author} {\bibfnamefont {H.-Z.}\ \bibnamefont {Chen}},
  \bibinfo {author} {\bibfnamefont {S.}~\bibnamefont {Wang}}, \bibinfo {author}
  {\bibfnamefont {X.-R.}\ \bibnamefont {Mao}}, \bibinfo {author} {\bibfnamefont
  {Z.-Q.}\ \bibnamefont {Yang}}, \bibinfo {author} {\bibfnamefont {S.-L.}\
  \bibnamefont {Wang}}, \bibinfo {author} {\bibfnamefont {X.-X.}\ \bibnamefont
  {Wang}}, \bibinfo {author} {\bibfnamefont {X.}~\bibnamefont {Hu}},\ and\
  \bibinfo {author} {\bibfnamefont {R.-M.}\ \bibnamefont {Ma}},\ }\bibfield
  {title} {\bibinfo {title} {A high-performance topological bulk laser based on
  band-inversion-induced reflection},\ }\href@noop {} {\bibfield  {journal}
  {\bibinfo  {journal} {Nature nanotechnology}\ }\textbf {\bibinfo {volume}
  {15}},\ \bibinfo {pages} {67} (\bibinfo {year} {2020})}\BibitemShut {NoStop}%
\bibitem [{\citenamefont {Dikopoltsev}\ \emph {et~al.}(2021)\citenamefont
  {Dikopoltsev}, \citenamefont {Harder}, \citenamefont {Lustig}, \citenamefont
  {Egorov}, \citenamefont {Beierlein}, \citenamefont {Wolf}, \citenamefont
  {Lumer}, \citenamefont {Emmerling}, \citenamefont {Schneider}, \citenamefont
  {H{\"o}fling} \emph {et~al.}}]{dikopoltsev2021topological}%
  \BibitemOpen
  \bibfield  {author} {\bibinfo {author} {\bibfnamefont {A.}~\bibnamefont
  {Dikopoltsev}}, \bibinfo {author} {\bibfnamefont {T.~H.}\ \bibnamefont
  {Harder}}, \bibinfo {author} {\bibfnamefont {E.}~\bibnamefont {Lustig}},
  \bibinfo {author} {\bibfnamefont {O.~A.}\ \bibnamefont {Egorov}}, \bibinfo
  {author} {\bibfnamefont {J.}~\bibnamefont {Beierlein}}, \bibinfo {author}
  {\bibfnamefont {A.}~\bibnamefont {Wolf}}, \bibinfo {author} {\bibfnamefont
  {Y.}~\bibnamefont {Lumer}}, \bibinfo {author} {\bibfnamefont
  {M.}~\bibnamefont {Emmerling}}, \bibinfo {author} {\bibfnamefont
  {C.}~\bibnamefont {Schneider}}, \bibinfo {author} {\bibfnamefont
  {S.}~\bibnamefont {H{\"o}fling}}, \emph {et~al.},\ }\bibfield  {title}
  {\bibinfo {title} {Topological insulator vertical-cavity laser array},\
  }\href@noop {} {\bibfield  {journal} {\bibinfo  {journal} {Science}\ }\textbf
  {\bibinfo {volume} {373}},\ \bibinfo {pages} {1514} (\bibinfo {year}
  {2021})}\BibitemShut {NoStop}%
\bibitem [{\citenamefont {Liu}\ \emph {et~al.}(2020{\natexlab{a}})\citenamefont
  {Liu}, \citenamefont {Hwang}, \citenamefont {Ji}, \citenamefont {Wang},
  \citenamefont {Modi},\ and\ \citenamefont {Agarwal}}]{liu2020z2}%
  \BibitemOpen
  \bibfield  {author} {\bibinfo {author} {\bibfnamefont {W.}~\bibnamefont
  {Liu}}, \bibinfo {author} {\bibfnamefont {M.}~\bibnamefont {Hwang}}, \bibinfo
  {author} {\bibfnamefont {Z.}~\bibnamefont {Ji}}, \bibinfo {author}
  {\bibfnamefont {Y.}~\bibnamefont {Wang}}, \bibinfo {author} {\bibfnamefont
  {G.}~\bibnamefont {Modi}},\ and\ \bibinfo {author} {\bibfnamefont
  {R.}~\bibnamefont {Agarwal}},\ }\bibfield  {title} {\bibinfo {title} {$z_2$
  photonic topological insulators in the visible wavelength range for robust
  nanoscale photonics},\ }\href@noop {} {\bibfield  {journal} {\bibinfo
  {journal} {Nano Letters}\ }\textbf {\bibinfo {volume} {20}},\ \bibinfo
  {pages} {1329} (\bibinfo {year} {2020}{\natexlab{a}})}\BibitemShut {NoStop}%
\bibitem [{\citenamefont {Liu}\ \emph {et~al.}(2020{\natexlab{b}})\citenamefont
  {Liu}, \citenamefont {Ji}, \citenamefont {Wang}, \citenamefont {Modi},
  \citenamefont {Hwang}, \citenamefont {Zheng}, \citenamefont {Sorger},
  \citenamefont {Pan},\ and\ \citenamefont {Agarwal}}]{liu2020generation}%
  \BibitemOpen
  \bibfield  {author} {\bibinfo {author} {\bibfnamefont {W.}~\bibnamefont
  {Liu}}, \bibinfo {author} {\bibfnamefont {Z.}~\bibnamefont {Ji}}, \bibinfo
  {author} {\bibfnamefont {Y.}~\bibnamefont {Wang}}, \bibinfo {author}
  {\bibfnamefont {G.}~\bibnamefont {Modi}}, \bibinfo {author} {\bibfnamefont
  {M.}~\bibnamefont {Hwang}}, \bibinfo {author} {\bibfnamefont
  {B.}~\bibnamefont {Zheng}}, \bibinfo {author} {\bibfnamefont {V.~J.}\
  \bibnamefont {Sorger}}, \bibinfo {author} {\bibfnamefont {A.}~\bibnamefont
  {Pan}},\ and\ \bibinfo {author} {\bibfnamefont {R.}~\bibnamefont {Agarwal}},\
  }\bibfield  {title} {\bibinfo {title} {Generation of helical topological
  exciton-polaritons},\ }\href@noop {} {\bibfield  {journal} {\bibinfo
  {journal} {Science}\ }\textbf {\bibinfo {volume} {370}},\ \bibinfo {pages}
  {600} (\bibinfo {year} {2020}{\natexlab{b}})}\BibitemShut {NoStop}%
\bibitem [{\citenamefont {Kumar}\ \emph {et~al.}(2022)\citenamefont {Kumar},
  \citenamefont {Gupta}, \citenamefont {Pitchappa}, \citenamefont {Wang},
  \citenamefont {Fujita},\ and\ \citenamefont {Singh}}]{kumar2022terahertz}%
  \BibitemOpen
  \bibfield  {author} {\bibinfo {author} {\bibfnamefont {A.}~\bibnamefont
  {Kumar}}, \bibinfo {author} {\bibfnamefont {M.}~\bibnamefont {Gupta}},
  \bibinfo {author} {\bibfnamefont {P.}~\bibnamefont {Pitchappa}}, \bibinfo
  {author} {\bibfnamefont {N.}~\bibnamefont {Wang}}, \bibinfo {author}
  {\bibfnamefont {M.}~\bibnamefont {Fujita}},\ and\ \bibinfo {author}
  {\bibfnamefont {R.}~\bibnamefont {Singh}},\ }\bibfield  {title} {\bibinfo
  {title} {Terahertz topological photonic integrated circuits for 6g and
  beyond: A perspective},\ }\href@noop {} {\bibfield  {journal} {\bibinfo
  {journal} {Journal of Applied Physics}\ }\textbf {\bibinfo {volume} {132}},\
  \bibinfo {pages} {140901} (\bibinfo {year} {2022})}\BibitemShut {NoStop}%
\bibitem [{\citenamefont {Zhang}\ \emph {et~al.}(2017)\citenamefont {Zhang},
  \citenamefont {Wei}, \citenamefont {Cheng}, \citenamefont {Zhang},
  \citenamefont {Wu},\ and\ \citenamefont {Liu}}]{zhang2017topological}%
  \BibitemOpen
  \bibfield  {author} {\bibinfo {author} {\bibfnamefont {Z.}~\bibnamefont
  {Zhang}}, \bibinfo {author} {\bibfnamefont {Q.}~\bibnamefont {Wei}}, \bibinfo
  {author} {\bibfnamefont {Y.}~\bibnamefont {Cheng}}, \bibinfo {author}
  {\bibfnamefont {T.}~\bibnamefont {Zhang}}, \bibinfo {author} {\bibfnamefont
  {D.}~\bibnamefont {Wu}},\ and\ \bibinfo {author} {\bibfnamefont
  {X.}~\bibnamefont {Liu}},\ }\bibfield  {title} {\bibinfo {title} {Topological
  creation of acoustic pseudospin multipoles in a flow-free symmetry-broken
  metamaterial lattice},\ }\href@noop {} {\bibfield  {journal} {\bibinfo
  {journal} {Physical review letters}\ }\textbf {\bibinfo {volume} {118}},\
  \bibinfo {pages} {084303} (\bibinfo {year} {2017})}\BibitemShut {NoStop}%
\bibitem [{\citenamefont {Yang}\ \emph {et~al.}(2020)\citenamefont {Yang},
  \citenamefont {Shao}, \citenamefont {Chen}, \citenamefont {Mao},\ and\
  \citenamefont {Ma}}]{yang2020spin}%
  \BibitemOpen
  \bibfield  {author} {\bibinfo {author} {\bibfnamefont {Z.-Q.}\ \bibnamefont
  {Yang}}, \bibinfo {author} {\bibfnamefont {Z.-K.}\ \bibnamefont {Shao}},
  \bibinfo {author} {\bibfnamefont {H.-Z.}\ \bibnamefont {Chen}}, \bibinfo
  {author} {\bibfnamefont {X.-R.}\ \bibnamefont {Mao}},\ and\ \bibinfo {author}
  {\bibfnamefont {R.-M.}\ \bibnamefont {Ma}},\ }\bibfield  {title} {\bibinfo
  {title} {Spin-momentum-locked edge mode for topological vortex lasing},\
  }\href@noop {} {\bibfield  {journal} {\bibinfo  {journal} {Physical Review
  Letters}\ }\textbf {\bibinfo {volume} {125}},\ \bibinfo {pages} {013903}
  (\bibinfo {year} {2020})}\BibitemShut {NoStop}%
\bibitem [{\citenamefont {Liu}\ \emph {et~al.}(2017)\citenamefont {Liu},
  \citenamefont {Lian}, \citenamefont {Li}, \citenamefont {Xu},\ and\
  \citenamefont {Duan}}]{liu2017pseudospins}%
  \BibitemOpen
  \bibfield  {author} {\bibinfo {author} {\bibfnamefont {Y.}~\bibnamefont
  {Liu}}, \bibinfo {author} {\bibfnamefont {C.-S.}\ \bibnamefont {Lian}},
  \bibinfo {author} {\bibfnamefont {Y.}~\bibnamefont {Li}}, \bibinfo {author}
  {\bibfnamefont {Y.}~\bibnamefont {Xu}},\ and\ \bibinfo {author}
  {\bibfnamefont {W.}~\bibnamefont {Duan}},\ }\bibfield  {title} {\bibinfo
  {title} {Pseudospins and topological effects of phonons in a kekul{\'e}
  lattice},\ }\href@noop {} {\bibfield  {journal} {\bibinfo  {journal}
  {Physical review letters}\ }\textbf {\bibinfo {volume} {119}},\ \bibinfo
  {pages} {255901} (\bibinfo {year} {2017})}\BibitemShut {NoStop}%
\bibitem [{\citenamefont {Pirie}\ \emph {et~al.}(2022)\citenamefont {Pirie},
  \citenamefont {Sadhuka}, \citenamefont {Wang}, \citenamefont {Andrei},\ and\
  \citenamefont {Hoffman}}]{pirie2022topological}%
  \BibitemOpen
  \bibfield  {author} {\bibinfo {author} {\bibfnamefont {H.}~\bibnamefont
  {Pirie}}, \bibinfo {author} {\bibfnamefont {S.}~\bibnamefont {Sadhuka}},
  \bibinfo {author} {\bibfnamefont {J.}~\bibnamefont {Wang}}, \bibinfo {author}
  {\bibfnamefont {R.}~\bibnamefont {Andrei}},\ and\ \bibinfo {author}
  {\bibfnamefont {J.~E.}\ \bibnamefont {Hoffman}},\ }\bibfield  {title}
  {\bibinfo {title} {Topological phononic logic},\ }\href@noop {} {\bibfield
  {journal} {\bibinfo  {journal} {Physical Review Letters}\ }\textbf {\bibinfo
  {volume} {128}},\ \bibinfo {pages} {015501} (\bibinfo {year}
  {2022})}\BibitemShut {NoStop}%
\bibitem [{\citenamefont {Cha}\ \emph {et~al.}(2018)\citenamefont {Cha},
  \citenamefont {Kim},\ and\ \citenamefont {Daraio}}]{cha2018experimental}%
  \BibitemOpen
  \bibfield  {author} {\bibinfo {author} {\bibfnamefont {J.}~\bibnamefont
  {Cha}}, \bibinfo {author} {\bibfnamefont {K.~W.}\ \bibnamefont {Kim}},\ and\
  \bibinfo {author} {\bibfnamefont {C.}~\bibnamefont {Daraio}},\ }\bibfield
  {title} {\bibinfo {title} {Experimental realization of on-chip topological
  nanoelectromechanical metamaterials},\ }\href@noop {} {\bibfield  {journal}
  {\bibinfo  {journal} {Nature}\ }\textbf {\bibinfo {volume} {564}},\ \bibinfo
  {pages} {229} (\bibinfo {year} {2018})}\BibitemShut {NoStop}%
\bibitem [{\citenamefont {He}\ \emph {et~al.}(2016)\citenamefont {He},
  \citenamefont {Ni}, \citenamefont {Ge}, \citenamefont {Sun}, \citenamefont
  {Chen}, \citenamefont {Lu}, \citenamefont {Liu},\ and\ \citenamefont
  {Chen}}]{he2016acoustic}%
  \BibitemOpen
  \bibfield  {author} {\bibinfo {author} {\bibfnamefont {C.}~\bibnamefont
  {He}}, \bibinfo {author} {\bibfnamefont {X.}~\bibnamefont {Ni}}, \bibinfo
  {author} {\bibfnamefont {H.}~\bibnamefont {Ge}}, \bibinfo {author}
  {\bibfnamefont {X.-C.}\ \bibnamefont {Sun}}, \bibinfo {author} {\bibfnamefont
  {Y.-B.}\ \bibnamefont {Chen}}, \bibinfo {author} {\bibfnamefont {M.-H.}\
  \bibnamefont {Lu}}, \bibinfo {author} {\bibfnamefont {X.-P.}\ \bibnamefont
  {Liu}},\ and\ \bibinfo {author} {\bibfnamefont {Y.-F.}\ \bibnamefont
  {Chen}},\ }\bibfield  {title} {\bibinfo {title} {Acoustic topological
  insulator and robust one-way sound transport},\ }\href@noop {} {\bibfield
  {journal} {\bibinfo  {journal} {Nature physics}\ }\textbf {\bibinfo {volume}
  {12}},\ \bibinfo {pages} {1124} (\bibinfo {year} {2016})}\BibitemShut
  {NoStop}%
\bibitem [{\citenamefont {Li}\ \emph {et~al.}(2021)\citenamefont {Li},
  \citenamefont {Sinev}, \citenamefont {Benimetskiy}, \citenamefont {Ivanova},
  \citenamefont {Khestanova}, \citenamefont {Kiriushechkina}, \citenamefont
  {Vakulenko}, \citenamefont {Guddala}, \citenamefont {Skolnick}, \citenamefont
  {Menon} \emph {et~al.}}]{li2021experimental}%
  \BibitemOpen
  \bibfield  {author} {\bibinfo {author} {\bibfnamefont {M.}~\bibnamefont
  {Li}}, \bibinfo {author} {\bibfnamefont {I.}~\bibnamefont {Sinev}}, \bibinfo
  {author} {\bibfnamefont {F.}~\bibnamefont {Benimetskiy}}, \bibinfo {author}
  {\bibfnamefont {T.}~\bibnamefont {Ivanova}}, \bibinfo {author} {\bibfnamefont
  {E.}~\bibnamefont {Khestanova}}, \bibinfo {author} {\bibfnamefont
  {S.}~\bibnamefont {Kiriushechkina}}, \bibinfo {author} {\bibfnamefont
  {A.}~\bibnamefont {Vakulenko}}, \bibinfo {author} {\bibfnamefont
  {S.}~\bibnamefont {Guddala}}, \bibinfo {author} {\bibfnamefont
  {M.}~\bibnamefont {Skolnick}}, \bibinfo {author} {\bibfnamefont {V.~M.}\
  \bibnamefont {Menon}}, \emph {et~al.},\ }\bibfield  {title} {\bibinfo {title}
  {Experimental observation of topological z2 exciton-polaritons in transition
  metal dichalcogenide monolayers},\ }\href@noop {} {\bibfield  {journal}
  {\bibinfo  {journal} {Nature communications}\ }\textbf {\bibinfo {volume}
  {12}},\ \bibinfo {pages} {1} (\bibinfo {year} {2021})}\BibitemShut {NoStop}%
\bibitem [{\citenamefont {Barik}\ \emph {et~al.}(2016)\citenamefont {Barik},
  \citenamefont {Miyake}, \citenamefont {DeGottardi}, \citenamefont {Waks},\
  and\ \citenamefont {Hafezi}}]{barik2016two}%
  \BibitemOpen
  \bibfield  {author} {\bibinfo {author} {\bibfnamefont {S.}~\bibnamefont
  {Barik}}, \bibinfo {author} {\bibfnamefont {H.}~\bibnamefont {Miyake}},
  \bibinfo {author} {\bibfnamefont {W.}~\bibnamefont {DeGottardi}}, \bibinfo
  {author} {\bibfnamefont {E.}~\bibnamefont {Waks}},\ and\ \bibinfo {author}
  {\bibfnamefont {M.}~\bibnamefont {Hafezi}},\ }\bibfield  {title} {\bibinfo
  {title} {Two-dimensionally confined topological edge states in photonic
  crystals},\ }\href@noop {} {\bibfield  {journal} {\bibinfo  {journal} {New
  Journal of Physics}\ }\textbf {\bibinfo {volume} {18}},\ \bibinfo {pages}
  {113013} (\bibinfo {year} {2016})}\BibitemShut {NoStop}%
\bibitem [{\citenamefont {Bradlyn}\ \emph {et~al.}(2017)\citenamefont
  {Bradlyn}, \citenamefont {Elcoro}, \citenamefont {Cano}, \citenamefont
  {Vergniory}, \citenamefont {Wang}, \citenamefont {Felser}, \citenamefont
  {Aroyo},\ and\ \citenamefont {Bernevig}}]{bradlyn2017topological}%
  \BibitemOpen
  \bibfield  {author} {\bibinfo {author} {\bibfnamefont {B.}~\bibnamefont
  {Bradlyn}}, \bibinfo {author} {\bibfnamefont {L.}~\bibnamefont {Elcoro}},
  \bibinfo {author} {\bibfnamefont {J.}~\bibnamefont {Cano}}, \bibinfo {author}
  {\bibfnamefont {M.~G.}\ \bibnamefont {Vergniory}}, \bibinfo {author}
  {\bibfnamefont {Z.}~\bibnamefont {Wang}}, \bibinfo {author} {\bibfnamefont
  {C.}~\bibnamefont {Felser}}, \bibinfo {author} {\bibfnamefont {M.~I.}\
  \bibnamefont {Aroyo}},\ and\ \bibinfo {author} {\bibfnamefont {B.~A.}\
  \bibnamefont {Bernevig}},\ }\bibfield  {title} {\bibinfo {title} {Topological
  quantum chemistry},\ }\href@noop {} {\bibfield  {journal} {\bibinfo
  {journal} {Nature}\ }\textbf {\bibinfo {volume} {547}},\ \bibinfo {pages}
  {298} (\bibinfo {year} {2017})}\BibitemShut {NoStop}%
\bibitem [{\citenamefont {Cano}\ \emph {et~al.}(2018)\citenamefont {Cano},
  \citenamefont {Bradlyn}, \citenamefont {Wang}, \citenamefont {Elcoro},
  \citenamefont {Vergniory}, \citenamefont {Felser}, \citenamefont {Aroyo},\
  and\ \citenamefont {Bernevig}}]{cano2018building}%
  \BibitemOpen
  \bibfield  {author} {\bibinfo {author} {\bibfnamefont {J.}~\bibnamefont
  {Cano}}, \bibinfo {author} {\bibfnamefont {B.}~\bibnamefont {Bradlyn}},
  \bibinfo {author} {\bibfnamefont {Z.}~\bibnamefont {Wang}}, \bibinfo {author}
  {\bibfnamefont {L.}~\bibnamefont {Elcoro}}, \bibinfo {author} {\bibfnamefont
  {M.}~\bibnamefont {Vergniory}}, \bibinfo {author} {\bibfnamefont
  {C.}~\bibnamefont {Felser}}, \bibinfo {author} {\bibfnamefont {M.~I.}\
  \bibnamefont {Aroyo}},\ and\ \bibinfo {author} {\bibfnamefont {B.~A.}\
  \bibnamefont {Bernevig}},\ }\bibfield  {title} {\bibinfo {title} {Building
  blocks of topological quantum chemistry: Elementary band representations},\
  }\href@noop {} {\bibfield  {journal} {\bibinfo  {journal} {Physical Review
  B}\ }\textbf {\bibinfo {volume} {97}},\ \bibinfo {pages} {035139} (\bibinfo
  {year} {2018})}\BibitemShut {NoStop}%
\bibitem [{\citenamefont {Alexandradinata}\ \emph {et~al.}(2014)\citenamefont
  {Alexandradinata}, \citenamefont {Dai},\ and\ \citenamefont
  {Bernevig}}]{alexandradinata2014wilson}%
  \BibitemOpen
  \bibfield  {author} {\bibinfo {author} {\bibfnamefont {A.}~\bibnamefont
  {Alexandradinata}}, \bibinfo {author} {\bibfnamefont {X.}~\bibnamefont
  {Dai}},\ and\ \bibinfo {author} {\bibfnamefont {B.~A.}\ \bibnamefont
  {Bernevig}},\ }\bibfield  {title} {\bibinfo {title} {Wilson-loop
  characterization of inversion-symmetric topological insulators},\ }\href@noop
  {} {\bibfield  {journal} {\bibinfo  {journal} {Physical Review B}\ }\textbf
  {\bibinfo {volume} {89}},\ \bibinfo {pages} {155114} (\bibinfo {year}
  {2014})}\BibitemShut {NoStop}%
\bibitem [{\citenamefont {De~Paz}\ \emph {et~al.}(2019)\citenamefont {De~Paz},
  \citenamefont {Vergniory}, \citenamefont {Bercioux}, \citenamefont
  {Garc{\'\i}a-Etxarri},\ and\ \citenamefont {Bradlyn}}]{de2019engineering}%
  \BibitemOpen
  \bibfield  {author} {\bibinfo {author} {\bibfnamefont {M.~B.}\ \bibnamefont
  {De~Paz}}, \bibinfo {author} {\bibfnamefont {M.~G.}\ \bibnamefont
  {Vergniory}}, \bibinfo {author} {\bibfnamefont {D.}~\bibnamefont {Bercioux}},
  \bibinfo {author} {\bibfnamefont {A.}~\bibnamefont {Garc{\'\i}a-Etxarri}},\
  and\ \bibinfo {author} {\bibfnamefont {B.}~\bibnamefont {Bradlyn}},\
  }\bibfield  {title} {\bibinfo {title} {Engineering fragile topology in
  photonic crystals: Topological quantum chemistry of light},\ }\href@noop {}
  {\bibfield  {journal} {\bibinfo  {journal} {Physical Review Research}\
  }\textbf {\bibinfo {volume} {1}},\ \bibinfo {pages} {032005} (\bibinfo {year}
  {2019})}\BibitemShut {NoStop}%
\bibitem [{\citenamefont {Palmer}\ and\ \citenamefont
  {Giannini}(2021)}]{palmer2021berry}%
  \BibitemOpen
  \bibfield  {author} {\bibinfo {author} {\bibfnamefont {S.~J.}\ \bibnamefont
  {Palmer}}\ and\ \bibinfo {author} {\bibfnamefont {V.}~\bibnamefont
  {Giannini}},\ }\bibfield  {title} {\bibinfo {title} {Berry bands and
  pseudo-spin of topological photonic phases},\ }\href@noop {} {\bibfield
  {journal} {\bibinfo  {journal} {Physical Review Research}\ }\textbf {\bibinfo
  {volume} {3}},\ \bibinfo {pages} {L022013} (\bibinfo {year}
  {2021})}\BibitemShut {NoStop}%
\bibitem [{\citenamefont {Alexandradinata}\ \emph {et~al.}(2016)\citenamefont
  {Alexandradinata}, \citenamefont {Wang},\ and\ \citenamefont
  {Bernevig}}]{alexandradinata2016topological}%
  \BibitemOpen
  \bibfield  {author} {\bibinfo {author} {\bibfnamefont {A.}~\bibnamefont
  {Alexandradinata}}, \bibinfo {author} {\bibfnamefont {Z.}~\bibnamefont
  {Wang}},\ and\ \bibinfo {author} {\bibfnamefont {B.~A.}\ \bibnamefont
  {Bernevig}},\ }\bibfield  {title} {\bibinfo {title} {Topological insulators
  from group cohomology},\ }\href@noop {} {\bibfield  {journal} {\bibinfo
  {journal} {Physical Review X}\ }\textbf {\bibinfo {volume} {6}},\ \bibinfo
  {pages} {021008} (\bibinfo {year} {2016})}\BibitemShut {NoStop}%
\bibitem [{\citenamefont {Fidkowski}\ \emph {et~al.}(2011)\citenamefont
  {Fidkowski}, \citenamefont {Jackson},\ and\ \citenamefont
  {Klich}}]{fidkowski2011model}%
  \BibitemOpen
  \bibfield  {author} {\bibinfo {author} {\bibfnamefont {L.}~\bibnamefont
  {Fidkowski}}, \bibinfo {author} {\bibfnamefont {T.}~\bibnamefont {Jackson}},\
  and\ \bibinfo {author} {\bibfnamefont {I.}~\bibnamefont {Klich}},\ }\bibfield
   {title} {\bibinfo {title} {Model characterization of gapless edge modes of
  topological insulators using intermediate brillouin-zone functions},\
  }\href@noop {} {\bibfield  {journal} {\bibinfo  {journal} {Physical review
  letters}\ }\textbf {\bibinfo {volume} {107}},\ \bibinfo {pages} {036601}
  (\bibinfo {year} {2011})}\BibitemShut {NoStop}%
\bibitem [{\citenamefont {Alexandradinata}\ \emph {et~al.}(2020)\citenamefont
  {Alexandradinata}, \citenamefont {H{\"o}ller}, \citenamefont {Wang},
  \citenamefont {Cheng},\ and\ \citenamefont
  {Lu}}]{alexandradinata2020crystallographic}%
  \BibitemOpen
  \bibfield  {author} {\bibinfo {author} {\bibfnamefont {A.}~\bibnamefont
  {Alexandradinata}}, \bibinfo {author} {\bibfnamefont {J.}~\bibnamefont
  {H{\"o}ller}}, \bibinfo {author} {\bibfnamefont {C.}~\bibnamefont {Wang}},
  \bibinfo {author} {\bibfnamefont {H.}~\bibnamefont {Cheng}},\ and\ \bibinfo
  {author} {\bibfnamefont {L.}~\bibnamefont {Lu}},\ }\bibfield  {title}
  {\bibinfo {title} {Crystallographic splitting theorem for band
  representations and fragile topological photonic crystals},\ }\href@noop {}
  {\bibfield  {journal} {\bibinfo  {journal} {Physical Review B}\ }\textbf
  {\bibinfo {volume} {102}},\ \bibinfo {pages} {115117} (\bibinfo {year}
  {2020})}\BibitemShut {NoStop}%
\bibitem [{\citenamefont {Aroyo}\ \emph
  {et~al.}(2006{\natexlab{a}})\citenamefont {Aroyo}, \citenamefont
  {Perez-Mato}, \citenamefont {Capillas}, \citenamefont {Kroumova},
  \citenamefont {Ivantchev}, \citenamefont {Madariaga}, \citenamefont {Kirov},\
  and\ \citenamefont {Wondratschek}}]{aroyo2006bilbao1}%
  \BibitemOpen
  \bibfield  {author} {\bibinfo {author} {\bibfnamefont {M.~I.}\ \bibnamefont
  {Aroyo}}, \bibinfo {author} {\bibfnamefont {J.~M.}\ \bibnamefont
  {Perez-Mato}}, \bibinfo {author} {\bibfnamefont {C.}~\bibnamefont
  {Capillas}}, \bibinfo {author} {\bibfnamefont {E.}~\bibnamefont {Kroumova}},
  \bibinfo {author} {\bibfnamefont {S.}~\bibnamefont {Ivantchev}}, \bibinfo
  {author} {\bibfnamefont {G.}~\bibnamefont {Madariaga}}, \bibinfo {author}
  {\bibfnamefont {A.}~\bibnamefont {Kirov}},\ and\ \bibinfo {author}
  {\bibfnamefont {H.}~\bibnamefont {Wondratschek}},\ }\bibfield  {title}
  {\bibinfo {title} {Bilbao crystallographic server: I. databases and
  crystallographic computing programs},\ }\href@noop {} {\bibfield  {journal}
  {\bibinfo  {journal} {Zeitschrift f{\"u}r Kristallographie-Crystalline
  Materials}\ }\textbf {\bibinfo {volume} {221}},\ \bibinfo {pages} {15}
  (\bibinfo {year} {2006}{\natexlab{a}})}\BibitemShut {NoStop}%
\bibitem [{\citenamefont {Aroyo}\ \emph
  {et~al.}(2006{\natexlab{b}})\citenamefont {Aroyo}, \citenamefont {Kirov},
  \citenamefont {Capillas}, \citenamefont {Perez-Mato},\ and\ \citenamefont
  {Wondratschek}}]{aroyo2006bilbao2}%
  \BibitemOpen
  \bibfield  {author} {\bibinfo {author} {\bibfnamefont {M.~I.}\ \bibnamefont
  {Aroyo}}, \bibinfo {author} {\bibfnamefont {A.}~\bibnamefont {Kirov}},
  \bibinfo {author} {\bibfnamefont {C.}~\bibnamefont {Capillas}}, \bibinfo
  {author} {\bibfnamefont {J.}~\bibnamefont {Perez-Mato}},\ and\ \bibinfo
  {author} {\bibfnamefont {H.}~\bibnamefont {Wondratschek}},\ }\bibfield
  {title} {\bibinfo {title} {Bilbao crystallographic server. ii.
  representations of crystallographic point groups and space groups},\
  }\href@noop {} {\bibfield  {journal} {\bibinfo  {journal} {Acta
  Crystallographica Section A: Foundations of Crystallography}\ }\textbf
  {\bibinfo {volume} {62}},\ \bibinfo {pages} {115} (\bibinfo {year}
  {2006}{\natexlab{b}})}\BibitemShut {NoStop}%
\bibitem [{\citenamefont {Aroyo}\ \emph {et~al.}(2011)\citenamefont {Aroyo},
  \citenamefont {Perez-Mato}, \citenamefont {Orobengoa}, \citenamefont {Tasci},
  \citenamefont {de~la Flor},\ and\ \citenamefont
  {Kirov}}]{aroyo2011crystallography}%
  \BibitemOpen
  \bibfield  {author} {\bibinfo {author} {\bibfnamefont {M.~I.}\ \bibnamefont
  {Aroyo}}, \bibinfo {author} {\bibfnamefont {J.~M.}\ \bibnamefont
  {Perez-Mato}}, \bibinfo {author} {\bibfnamefont {D.}~\bibnamefont
  {Orobengoa}}, \bibinfo {author} {\bibfnamefont {E.}~\bibnamefont {Tasci}},
  \bibinfo {author} {\bibfnamefont {G.}~\bibnamefont {de~la Flor}},\ and\
  \bibinfo {author} {\bibfnamefont {A.}~\bibnamefont {Kirov}},\ }\bibfield
  {title} {\bibinfo {title} {Crystallography online: Bilbao crystallographic
  server},\ }\href@noop {} {\bibfield  {journal} {\bibinfo  {journal} {Bulg.
  Chem. Commun}\ }\textbf {\bibinfo {volume} {43}},\ \bibinfo {pages} {183}
  (\bibinfo {year} {2011})}\BibitemShut {NoStop}%
\bibitem [{sup()}]{supp}%
  \BibitemOpen
  \href@noop {} {\bibinfo {title} {See supplemental material at [url will be
  inserted by publisher] for the discussion of the topological invariant of the
  model, the detailed setup of the perturbed hamiltonian and numerical
  simulations of trivial helical edge states.}}\BibitemShut {Stop}%
\bibitem [{\citenamefont {Ahn}\ \emph {et~al.}(2019{\natexlab{a}})\citenamefont
  {Ahn}, \citenamefont {Park}, \citenamefont {Kim}, \citenamefont {Kim},\ and\
  \citenamefont {Yang}}]{ahn2019stiefel}%
  \BibitemOpen
  \bibfield  {author} {\bibinfo {author} {\bibfnamefont {J.}~\bibnamefont
  {Ahn}}, \bibinfo {author} {\bibfnamefont {S.}~\bibnamefont {Park}}, \bibinfo
  {author} {\bibfnamefont {D.}~\bibnamefont {Kim}}, \bibinfo {author}
  {\bibfnamefont {Y.}~\bibnamefont {Kim}},\ and\ \bibinfo {author}
  {\bibfnamefont {B.-J.}\ \bibnamefont {Yang}},\ }\bibfield  {title} {\bibinfo
  {title} {Stiefel--whitney classes and topological phases in band theory},\
  }\href@noop {} {\bibfield  {journal} {\bibinfo  {journal} {Chinese Physics
  B}\ }\textbf {\bibinfo {volume} {28}},\ \bibinfo {pages} {117101} (\bibinfo
  {year} {2019}{\natexlab{a}})}\BibitemShut {NoStop}%
\bibitem [{\citenamefont {Cano}\ and\ \citenamefont
  {Bradlyn}(2021)}]{cano2021band}%
  \BibitemOpen
  \bibfield  {author} {\bibinfo {author} {\bibfnamefont {J.}~\bibnamefont
  {Cano}}\ and\ \bibinfo {author} {\bibfnamefont {B.}~\bibnamefont {Bradlyn}},\
  }\bibfield  {title} {\bibinfo {title} {Band representations and topological
  quantum chemistry},\ }\href@noop {} {\bibfield  {journal} {\bibinfo
  {journal} {Annual Review of Condensed Matter Physics}\ }\textbf {\bibinfo
  {volume} {12}},\ \bibinfo {pages} {225} (\bibinfo {year} {2021})}\BibitemShut
  {NoStop}%
\bibitem [{\citenamefont {Ahn}\ \emph {et~al.}(2019{\natexlab{b}})\citenamefont
  {Ahn}, \citenamefont {Park},\ and\ \citenamefont {Yang}}]{ahn2019failure}%
  \BibitemOpen
  \bibfield  {author} {\bibinfo {author} {\bibfnamefont {J.}~\bibnamefont
  {Ahn}}, \bibinfo {author} {\bibfnamefont {S.}~\bibnamefont {Park}},\ and\
  \bibinfo {author} {\bibfnamefont {B.-J.}\ \bibnamefont {Yang}},\ }\bibfield
  {title} {\bibinfo {title} {Failure of nielsen-ninomiya theorem and fragile
  topology in two-dimensional systems with space-time inversion symmetry:
  application to twisted bilayer graphene at magic angle},\ }\href@noop {}
  {\bibfield  {journal} {\bibinfo  {journal} {Physical Review X}\ }\textbf
  {\bibinfo {volume} {9}},\ \bibinfo {pages} {021013} (\bibinfo {year}
  {2019}{\natexlab{b}})}\BibitemShut {NoStop}%
\bibitem [{\citenamefont {Kane}\ and\ \citenamefont
  {Mele}(2005)}]{kane2005quantum}%
  \BibitemOpen
  \bibfield  {author} {\bibinfo {author} {\bibfnamefont {C.~L.}\ \bibnamefont
  {Kane}}\ and\ \bibinfo {author} {\bibfnamefont {E.~J.}\ \bibnamefont
  {Mele}},\ }\bibfield  {title} {\bibinfo {title} {Quantum spin hall effect in
  graphene},\ }\href@noop {} {\bibfield  {journal} {\bibinfo  {journal}
  {Physical review letters}\ }\textbf {\bibinfo {volume} {95}},\ \bibinfo
  {pages} {226801} (\bibinfo {year} {2005})}\BibitemShut {NoStop}%
\bibitem [{\citenamefont {Orazbayev}\ and\ \citenamefont
  {Fleury}(2019)}]{orazbayev2019quantitative}%
  \BibitemOpen
  \bibfield  {author} {\bibinfo {author} {\bibfnamefont {B.}~\bibnamefont
  {Orazbayev}}\ and\ \bibinfo {author} {\bibfnamefont {R.}~\bibnamefont
  {Fleury}},\ }\bibfield  {title} {\bibinfo {title} {Quantitative robustness
  analysis of topological edge modes in c6 and valley-hall metamaterial
  waveguides},\ }\href@noop {} {\bibfield  {journal} {\bibinfo  {journal}
  {Nanophotonics}\ }\textbf {\bibinfo {volume} {8}},\ \bibinfo {pages} {1433}
  (\bibinfo {year} {2019})}\BibitemShut {NoStop}%
\end{thebibliography}%

\end{document}
