% ****** Start of file sorsamp.tex ******
%
%   This file is part of the AIP files in the AIP distribution for REVTeX 4.
%   Version 4.2a of REVTeX, December 2014
%
%   Copyright (c) 2014 American Institute of Physics.
%
%   See the AIP README file for restrictions and more information.
%
% TeX'ing this file requires that you have AMS-LaTeX 2.0 installed
% as well as the rest of the prerequisites for REVTeX 4.2
%
% It also requires running BibTeX. The commands are as follows:
%
%  1)  latex  sorsamp
%  2)  bibtex sorsamp
%  3)  latex  sorsamp
%  4)  latex  sorsamp
%
% Use this file as a source of example code for your aip document.
% Use the file aiptemplate.tex as a template for your document.
\documentclass[
%aip,
%twoside,
%groupedaddress,
%jmp,
 jor,
 amsmath,amssymb,
%preprint,%
 reprint,%
%author-year,%
%author-numerical,%
]{revtex4-2}

\usepackage{graphicx}% Include figure files
\usepackage{dcolumn}% Align table columns on decimal point
\usepackage{bm}% bold math
\usepackage{amsmath}
\usepackage{physics}

\usepackage{xr}
\makeatletter

\newcommand*{\addFileDependency}[1]{% argument=file name and extension
\typeout{(#1)}% latexmk will find this if $recorder=0
% however, in that case, it will ignore #1 if it is a .aux or 
% .pdf file etc and it exists! If it doesn't exist, it will appear 
% in the list of dependents regardless)
%
% Write the following if you want it to appear in \listfiles 
% --- although not really necessary and latexmk doesn't use this
%
\@addtofilelist{#1}
%
% latexmk will find this message if #1 doesn't exist (yet)
\IfFileExists{#1}{}{\typeout{No file #1.}}
}\makeatother

\newcommand*{\myexternaldocument}[1]{%
\externaldocument{#1}%
\addFileDependency{#1.tex}%
\addFileDependency{#1.aux}%
}
%------------End of helper code--------------

% put all the external documents here!
\myexternaldocument{maintext}

\newcounter{Seqn}
\newenvironment{Sequation}
  {\stepcounter{Seqn}%
    \addtocounter{equation}{-1}%
    \renewcommand\theequation{S\arabic{Seqn}}\equation}
  {\endequation}

\begin{document}
% \preprint{AIP/123-QED}

\title[Supplementary materials]{Supplementary materials for:\\Absence of topological protection of the interface states in $\mathbb{Z}_2$ photonic crystals}% Force line breaks with \\
% \thanks{Footnote to title of article.}

\author{Shupeng Xu}
\author{Yuhui Wang}%
\author{Ritesh Agarwal}
\email{riteshag@seas.upenn.edu}
\affiliation{
Department of Materials Science and Engineering, University of Pennsylvania,\\Philadelphia, 19104, PA, US}
\date{\today}% It is always \today, today,
             %  but any date may be explicitly specified


\keywords{Suggested keywords}%Use showkeys class option if keyword
                              %display desired
\maketitle
% \appendix

\section{\label{sec:index} Topological Indices of the Wu-Hu model}
Here we review how the topological indices are obtained in the original proposal and prove their equivalence to the Euler class and the secon Stiefel-Whitney class.
In the tight binding model, near the critical point of the phase transition ($t_1=t_2$), the $k\cdot p$ expansion at $\Gamma$ point around fermi energy gives a well defined change in spin-Chern number.
Equivalently the spin-Chern number can be directly calculated for the projected Hamiltonian in $(\ket{p+}, \ket{d+}, \ket{p-}, \ket{d-})^T$ basis.
\begin{Sequation}
    % \tag{S}
    \label{eq:projected}
    H_p = \mathcal{P}H_0\mathcal{P} = \begin{pmatrix}
    H_+ & G\\
    G^\dag & H_-
    \end{pmatrix} 
\end{Sequation}
Where $H_0$ is the full Hamiltonian, $\mathcal{P}$ is the projector and $G$ is the off-diagonal matrix. 
We further define a spin-conserved Hamiltonian with zero off-diagonal blocks
\begin{Sequation}
    \label{eq:kp}
    H_{spin} = \begin{pmatrix}
    H_+ & 0\\
    0 & H_-
    \end{pmatrix}
\end{Sequation}
and a pseudo-spin gauge that satisfies
\begin{Sequation}
    C_2\mathcal{T}\ket{u_{+,n\mathbf{k}}}=\ket{u_{-,n\mathbf{k}}}
\end{Sequation}
where $i=\pm$ is the label of pseudo-spin,  $\ket{u_{\pm, n\mathbf{k}}}$ are eigenstates of $H_{\pm}$ and $n=1,2$ is the index of eigenstates within each block.
The non-Abelian Berry curvature $\mathbf{F}_{mn}=\nabla_{\mathbf{k}}\times\mathbf{A}_{mn}$ ($\mathbf{A}_{mn}=\bra{u_{m\mathbf{k}}} \nabla_{\mathbf{k}} \ket{u_{n\mathbf{k}}} $ is the non-Abelian Berry connection) for the lower two bands in the pseudo-spin gauge is diagonal

\begin{Sequation}
    F_z= \begin{pmatrix}
    i\Omega_+ & 0 \\
    0 & i\Omega_-
    \end{pmatrix}
\end{Sequation}

in which $\Omega_\pm$ are real numbers and $\Omega_+=-\Omega_-$ due to $C_2\mathcal{T}$ symmetry. 
The spin-Chern number $C_\pm$ is the integral of $\Omega_\pm$ over BZ and satisfies $C_\pm=0$ and $C_\pm=\pm1$ for the shrunken and expanded case respectively.

In a two-band subspace of a 2D $C_2\mathcal{T}$ symmetric system the Euler class can be expressed as a flux integral in the real gauge which is defined as
\begin{Sequation}
    C_2\mathcal{T}\ket{\tilde{u}_{n\mathbf{k}}} = \ket{\tilde{u}_{n\mathbf{k}}}
\end{Sequation}

In the real gauge the Berry connection $\tilde{\mathbf{A}}_{mn}$ and Berry curvature $\mathbf{\tilde{F}}_{mn}$ are 2D real anti-symmetric matrices, so they only have one degree of freedom. 
The Euler class can be written as the integral of the off-diagonal element
\begin{Sequation}
    e_2 = \frac{1}{2\pi}\oint_{BZ} d\mathbf{S}\cdot \mathbf{\tilde{F}}_{12} \in \mathbb{Z}
\end{Sequation}
We note that the real gauge is connected to the pseudo-spin gauge by the following gauge transformation
\begin{Sequation}
    \ket{\tilde{u}_{i,n\mathbf{k}}} = \sum_{j}U_{ij}\ket{u_{j,n\mathbf{k}}}
\end{Sequation}
where
\begin{Sequation}
    U=\frac{1}{\sqrt{2}}\begin{pmatrix}
    1 & 1 \\
    i & -i
    \end{pmatrix}.
\end{Sequation}
And the Berry-curvature transforms as
\begin{Sequation}
    \tilde{F}_z=U^*F_zU^T=\begin{pmatrix}
    0 & \Omega_- \\
    \Omega_+ & 0
    \end{pmatrix}.
\end{Sequation}
Thus we demonstrated the equivalence between the spin-Chern number and the Euler class for $H_{spin}$ with $e_2=C_{\pm}=0$ and $\pm e_2=C_{\mp}=\mp1$ for the shrunken and expanded case of Wu-Hu model respectively.

The Euler class remains invariant for the projected Hamiltonian $H_P$ since it can be obtained from Eqn.\ref{eq:kp} through a symmetric topology conserving deformation.
We note that the fragile nature of non-trivial Euler class agrees with the EBR analysis for projected bands in the expanded case.
The bands can be written as the difference of two EBRs: $\mathcal{B}=(A_1\uparrow G)_{3c}\ominus(A_1\uparrow G)_{1a}$.

The full valence band of Wu-Hu model can be viewed as the direct sum of the projected bands and a trivial band $(A_1\uparrow G)_{1a}$, and the parity of Euler class survives as the second Stiefel-Whitney class $\omega_2$ where $\omega_2=0$ and $\omega_2=1$ correspond to the shrunken and expanded phases, respectively.

Lastly, we noted that with the presence of $C_2$ symmetry, $\omega_2$ can be expressed using the product of $C_2$ eigenvalues of Bloch states:
\begin{Sequation}
\label{eq:inversion eigvl}
    (-1)^{\omega_2}=\prod_{i=1}^{4}(-1)^{\lfloor N_{occ}^-(\Gamma_i)/2 \rfloor}
\end{Sequation}

where $\Gamma_{i=1,2,3,4}$ are four $C_2$ invariant momenta in 2D BZ, $N_{occ}^-(\Gamma_i)$ is the number of occupied bands with negative $C_2$ eigenvalues at $\Gamma_i$ and $\lfloor\ \rfloor$ is the floor function or the greatest integer function.
This simple expression exactly coincidences with the topological index in Ref.\cite{liu2019helical}.

\section{Details of Perturbed Wu-Hu Model}
The perturbation is added as long range couplings shown in Fig. \ref{fig:figApp} and all its $C_6$ partners.
The hopping parameters are chosen as $l_1 = 0.28$, $l_2 = 0.4$, $l_3 = 0.68$, $l_4 = 0.4$ and $l_5 = 0.28$.
The hopping strength satisfies $l_1=l_5$ and $l_2=l_4$ which is required by $C_6$ symmetry.
Further considering the requirement described in the main text that a Dirac cone appears when the intra- and inter- cell couplings equal to each other, there are only two independent degrees of freedom for the perturbation.
The magnitudes for the original couplings of two phases of Wu-Hu model are $t_1 = -1.1$, $t_2 = -0.9$ (shrunken) and $t_1 = -0.9$, $t_2 = -1.1$ (expanded) respectively.

\renewcommand{\thefigure}{S1}
\begin{figure}
    \centering
    \includegraphics[width = 0.25\textwidth]{figapp.pdf}
    \caption{A schematic of one copy of the long-rang coupling for the uniform symmetric perturbation applied to the Wu-Hu model, Fig. 3 in the main text. To preserve $C_6$ symmetry, this set of couplings is accompanied with all its $C_6$ partners.}
    \label{fig:figApp}
\end{figure}

\section{\label{app:sim} Numerical simulations of trivial helical edge states}
\renewcommand{\thefigure}{S2}
\begin{figure}
    \centering
    \includegraphics[width = 0.48\textwidth]{fig_sim_1221.pdf}
    \caption{(a) Schematics of the expanded (top) and shrunken (down) unit cells for the original Wu-Hu model used in numerical simulations, respectively. $r_\text{ex}$ and $r_\text{sh}$ are the distances from each site to the center of the unit-cell, marked in the figure. (b) and (c) FDTD simulation results of the bulk band structure of the expanded and shrunken phases of the original Wu-Hu model used in the main text.}
    \label{fig:figApp2}
\end{figure}
We used a commercial finite-difference time-domain (FDTD) software package (\textsf{Lumerical FDTD}) for the electro-magnetic field numerical simulation to verify our conclusions that nontrivial topology is not required for the helical edge states.
The 2D simulations for the reduction of computation cost are set in a unified design of the dielectric honeycomb lattice: the lattice constant is 450 nm, and for each site we set its radius to be 50 nm and corresponding $\varepsilon = 11.7$. All simulations were conducted in a vacuum background. For the graphene structure, the distance between each site and the center of the hexagonal unit cell, defined by $r_0$, was $150$ nm. For the shrunken phase $r_{\text{sh}} = 142.9$ nm and for the expanded phase $r_{\text{ex}} = 155.2$ nm. (Fig. \ref{fig:figApp2}a) 

We presented the results for the band dispersion calculations of shrunken and expanded phases, shown in Figs. \ref{fig:figApp2}b and c, as a general reference for the simulations in the main text. 
A complete gap can be noticed near 320 THz.
For the simulation of the edge states, Bloch boundary conditions were defined in the x direction, with a perfect electrical conductor (PEC) applied at the bottom and a perfect matched layer (PML) in the top of the y direction, respectively. 
The positions for the bottom two sites in the incomplete unit cells were tuned by $\delta x = \pm 30$ nm to move away from the corresponding center of the unitcell, and $ \delta y = 10 $ nm upwards (Fig. 4a). The line of the incomplete unit cells (red sites in Fig. 4a) was then brought close to bulk part by $\delta y_{\mathrm{line}} = 40$ nm.

For the large scale propagation, a set of circularly polarized magnetic dipoles were placed at the center of a complete unit cell near the edge of the system. The excitation frequency was set at 322.8 THz.
The bottom and right boundaries (which defined the 120-turning) were set to be PEC and metal, functioning as OBCs. The intensity is normalized to the peak intensity outside the source region.

%apsrev4-2.bst 2019-01-14 (MD) hand-edited version of apsrev4-1.bst
%Control: key (0)
%Control: author (8) initials jnrlst
%Control: editor formatted (1) identically to author
%Control: production of article title (0) allowed
%Control: page (0) single
%Control: year (1) truncated
%Control: production of eprint (0) enabled
\begin{thebibliography}{1}%
\makeatletter
\providecommand \@ifxundefined [1]{%
 \@ifx{#1\undefined}
}%
\providecommand \@ifnum [1]{%
 \ifnum #1\expandafter \@firstoftwo
 \else \expandafter \@secondoftwo
 \fi
}%
\providecommand \@ifx [1]{%
 \ifx #1\expandafter \@firstoftwo
 \else \expandafter \@secondoftwo
 \fi
}%
\providecommand \natexlab [1]{#1}%
\providecommand \enquote  [1]{``#1''}%
\providecommand \bibnamefont  [1]{#1}%
\providecommand \bibfnamefont [1]{#1}%
\providecommand \citenamefont [1]{#1}%
\providecommand \href@noop [0]{\@secondoftwo}%
\providecommand \href [0]{\begingroup \@sanitize@url \@href}%
\providecommand \@href[1]{\@@startlink{#1}\@@href}%
\providecommand \@@href[1]{\endgroup#1\@@endlink}%
\providecommand \@sanitize@url [0]{\catcode `\\12\catcode `\$12\catcode
  `\&12\catcode `\#12\catcode `\^12\catcode `\_12\catcode `\%12\relax}%
\providecommand \@@startlink[1]{}%
\providecommand \@@endlink[0]{}%
\providecommand \url  [0]{\begingroup\@sanitize@url \@url }%
\providecommand \@url [1]{\endgroup\@href {#1}{\urlprefix }}%
\providecommand \urlprefix  [0]{URL }%
\providecommand \Eprint [0]{\href }%
\providecommand \doibase [0]{https://doi.org/}%
\providecommand \selectlanguage [0]{\@gobble}%
\providecommand \bibinfo  [0]{\@secondoftwo}%
\providecommand \bibfield  [0]{\@secondoftwo}%
\providecommand \translation [1]{[#1]}%
\providecommand \BibitemOpen [0]{}%
\providecommand \bibitemStop [0]{}%
\providecommand \bibitemNoStop [0]{.\EOS\space}%
\providecommand \EOS [0]{\spacefactor3000\relax}%
\providecommand \BibitemShut  [1]{\csname bibitem#1\endcsname}%
\let\auto@bib@innerbib\@empty
%</preamble>
\bibitem [{\citenamefont {Liu}\ \emph {et~al.}(2019)\citenamefont {Liu},
  \citenamefont {Deng},\ and\ \citenamefont {Wakabayashi}}]{liu2019helical}%
  \BibitemOpen
  \bibfield  {author} {\bibinfo {author} {\bibfnamefont {F.}~\bibnamefont
  {Liu}}, \bibinfo {author} {\bibfnamefont {H.-Y.}\ \bibnamefont {Deng}},\ and\
  \bibinfo {author} {\bibfnamefont {K.}~\bibnamefont {Wakabayashi}},\
  }\bibfield  {title} {\bibinfo {title} {Helical topological edge states in a
  quadrupole phase},\ }\href@noop {} {\bibfield  {journal} {\bibinfo  {journal}
  {Physical review letters}\ }\textbf {\bibinfo {volume} {122}},\ \bibinfo
  {pages} {086804} (\bibinfo {year} {2019})}\BibitemShut {NoStop}%
\end{thebibliography}%

% \bibliography{apsbib}% Produces the bibliography via BibTeX.
\end{document}
%
% ****** End of file sorsamp.tex ******
