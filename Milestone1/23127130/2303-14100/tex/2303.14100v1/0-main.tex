% \documentclass[conference]{IEEEtran}
\documentclass[letterpaper, 10 pt, conference]{ieeeconf} 
                                                         
% This command is only needed if you want to use the \thanks command
\IEEEoverridecommandlockouts

% Needed to meet printer requirements.
\overrideIEEEmargins                                      

\usepackage{flushend}
\usepackage{balance} % Comment 


\let\labelindent\relax

\newcommand\kousays[1]{\textcolor{blue}{#1}}

% ------------------------------
% load macros and headers
% ------------------------------
% \input{00-format-and-definitions}
\usepackage{url}

% Tables
\usepackage{tabu}
\usepackage{booktabs} % for professional tables
\usepackage{multirow}
\usepackage{multicol}
\usepackage{adjustbox}

% Comments/macros
\usepackage{xcolor}
\usepackage{xspace}
\usepackage{ifthen}

% Figures
\usepackage{graphicx}
\usepackage[font=small]{caption}
\usepackage{tikz}
\usetikzlibrary{positioning}

% Lists
\let\labelindent\relax %needed for IEEE conflict with enumitem
% \usepackage{enumitem} % for \begin{description}[align=left]

% Comments/useful macros
\usepackage{autolabtools}
\usepackage{amsmath,amssymb,amsfonts}
\usepackage{dsfont}
\usepackage{color}
\usepackage{verbatim}
\usepackage{subcaption}
\usepackage{balance}
\usepackage{gensymb}
\usepackage{siunitx} % Daniel: added based on Jeff's prior papers. (needs to be paired with next line :) -Jeff)
\sisetup{detect-all} % <-- fixes fonts in siunitx
\usepackage{algorithm}
% \usepackage{algorithmic}
\usepackage{algorithmicx}
% \usepackage[linesnumbered,ruled,vlined]{algorithm2e} 
\usepackage{algpseudocode}
\usepackage{alphalph}
\usepackage{comment}
\usepackage{soul}
\usepackage[shortlabels]{enumitem}

\graphicspath{{figures/}}
\newcommand{\ba}{\mathbf{a}}
\newcommand{\bd}{\mathbf{d}}
\newcommand{\bm}{\mathbf{m}}
\newcommand{\bo}{\mathbf{o}}
\newcommand{\br}{\mathbf{r}}
\newcommand{\bs}{\mathbf{s}}
\newcommand{\bx}{\mathbf{x}}
\newcommand{\bu}{\mathbf{u}}
\newcommand{\state}{\mathbf{s}}
\newcommand{\ie}{i.e., }
\newcommand{\eg}{e.g., }

\newcommand{\E}{\mathbb{E}}
\newcommand{\R}{\mathbb{R}}
\newcommand{\N}{\mathbb{N}}
\newcommand{\Z}{\mathbb{Z}}
\newcommand{\mO}{\mathcal{O}}
\newcommand{\mP}{\mathbb{P}}

\newtheorem{theorem}{Theorem}
\newtheorem{corollary}{Corollary}[theorem]
\newtheorem{lemma}[theorem]{Lemma}

% \algrenewcommand\algorithmicrequire{\textbf{Input:}}
\algrenewcommand\algorithmicensure{\textbf{Output:}}

% When the amsmath package is in use page breaks between equation lines are normally disallowed; the philosophy is that page breaks in such material should receive individual attention from the author. To get an individual page break inside a particular displayed equation, a \displaybreak command is provided. \displaybreak is best placed immediately before the \\ where it is to take effect. Like LATEX’s \pagebreak, \displaybreak takes an optional argument between 0 and 4 denoting the desirability of the pagebreak. \displaybreak[0] means "it is permissible to break here" without encouraging a break; \displaybreak with no optional argument is the same as \displaybreak[4] and forces a break.
%If you prefer a strategy of letting page breaks fall where they may, even in the middle of a multi-line equation, then you might put \allowdisplaybreaks[1] in the preamble of your document. An optional argument 1–4 can be used for finer control: [1] means allow page breaks, but avoid them as much as possible; values of 2,3,4 mean increasing permissiveness. When display breaks are enabled with \allowdisplaybreaks, the \\* command can be used to prohibit a pagebreak after a given line, as usual.
\allowdisplaybreaks[4]

% Daniel: for better argmax and argmin.
\DeclareMathOperator*{\argmax}{arg\,max}
\DeclareMathOperator*{\argmin}{arg\,min}
\DeclareMathOperator*{\minimize}{minimize}

% Daniel: feel free to add your names with your favorite color!
% Wrap around the shared "\remark" command for consistency, and to make it easy to turn all remarks off by defining an empty command. -Jeff
\newcommand{\remark}[3]{\hidable{{\color{#2}[#1: #3]}}}
% \renewcommand{\remark}[3]{} % To disable all remarks, uncomment
\newcommand{\lawrence}[1]{\remark{Lawrence}{orange}{#1}}
\definecolor{britishracinggreen}{rgb}{0.23, 0.53, 0.19}
\newcommand{\baiyu}[1]{\remark{Baiyu}{britishracinggreen}{#1}}
\newcommand{\dave}[1]{\remark{Dave}{red}{#1}}
\newcommand{\daniel}[1]{\remark{Daniel}{blue}{#1}}
\newcommand{\KG}[1]{\remark{KG}{red}{#1}}

\definecolor{navy}{rgb}{0,0,0.5}
\newcommand{\NEW}[1]{{\color{navy} #1}}


% Daniel: putting this here for the algorithm...
\newcommand{\bagging}{SLIP-Bagging\xspace}
\newcommand{\slip}{SLIP\xspace}

% Daniel: AH! importing this package makes it much better to make tables, in my opinion, since we can put a long caption at the table title. It does create a warning but I am not sure how to fix this.
\usepackage[font={small}]{caption}
\def\tablename{Table}


% Daniel: this way of doing it must be tied with \addbibresource and \printbibliography. Also keep hyperref imported LAST. If we are ever submitting to a venue which doesn't all for infinite reference space, adjust maxbibnames to be a little smaller so it will read as author et al.
\usepackage[backend=biber,
            hyperref=true,
            url=false,
            isbn=false,
            doi=false,
            backref=false,
            style=ieee,
            natbib=true,%compatibility aliases
            mincitenames=1,
            maxcitenames=1,
            citestyle=numeric-comp,
            sorting=none,%none vs nyt
            block=none,
            maxbibnames=99]{biblatex}
\usepackage{hyperref}

% ====================================================
% USER DEFINED MACROS MATH Notation
% ====================================================

\renewcommand {\vec}[1]{\mathbf{#1}}
\newcommand {\qq}{\vec{q}}
\newcommand {\dqq}{\dot{\vec{q}}}
\newcommand {\ddqq}{\ddot{\vec{q}}}
\newcommand {\duu}{\dot{\vec{u}}}
\newcommand {\dduu}{\ddot{\vec{u}}}
\newcommand {\uu}{\vec{u}}
\newcommand {\HH}{\textbf{H}}
\newcommand {\MM}{\textbf{M}}
\newcommand {\fc}{\vec{f}_\textrm{c}}
\newcommand {\funfc}{f_\textrm{c}}
\newcommand {\fe}{\vec{f}_\textrm{ext}}
\newcommand {\Xh}{\textbf{X}_\textrm{h}}
\newcommand {\xh}{\vec{x}}
\newcommand {\x}{\vec{x}}
\newcommand {\rr}{\vec{r}}
\newcommand {\rmesh}{\vec{r}_{\textrm{mesh}}}
\newcommand {\rpm}{\vec{r}_{\pm}}
\newcommand {\rphong}{\vec{r}_{\textrm{phong}}}
\newcommand {\p}{\vec{p}}
\newcommand {\fn}{\vec{f}_n}
\newcommand {\ft}{\vec{f}_t}
\newcommand {\kn}{k_n}
\newcommand {\kf}{k_f}
\newcommand {\vt}{\vec{v}_t}
\newcommand {\xlocal}{\vec{x}_\textrm{local}}
\newcommand {\xclosest}{\vec{x}_\textrm{closest}}
\newcommand {\xclamp}{\vec{x}_\textrm{clamp}}
\newcommand {\ffn}{f_n}
\newcommand {\fft}{f_t}
\newcommand\norm[1]{\Vert#1\Vert}
\newcommand {\qh}{\qq_{\textrm{h}}}
\newcommand {\qobj}{\vec{q}_{\textrm{o}}}
\newcommand {\qhand}{\vec{q}_{\textrm{hand}}}
\newcommand {\dqobj}{\dot{\vec{q}}_\textrm{o}}
\newcommand {\fd}{\widehat{\vec{f}}_\textrm{c}}
\newcommand {\Fd}{\vec{F}_\textrm{desired}}
\newcommand {\Fc}{\vec{F}_\textrm{actual}}
\newcommand {\Ltask}{\mathcal{L}_\textrm{task}}
\newcommand {\Lphysics}{\mathcal{L}_\textrm{phys}}
\newcommand {\Lgrasp}{\mathcal{L}_\textrm{grasp}}
\newcommand {\uobj}{\uu_\textrm{obj}}
\newcommand {\duobj}{\dot{\vec{u}}_\textrm{o}}
\newcommand {\uh}{\uu_\textrm{h}}
\newcommand {\duh}{\dot{\vec{u}}_\textrm{h}}
\newcommand {\Lqrange}{\mathcal{L}_\textrm{range}}
\newcommand {\Lqlimit}{\mathcal{L}_\textrm{limit}}
\newcommand {\Linter}{\mathcal{L}_\textrm{inter}}
\newcommand {\finter}{\vec{f}_\textrm{link}}
\newcommand {\pii}{\vec{p}_i}
\newcommand {\rii}{\vec{r}_i}

% ====================================================
% USER DEFINED MACROS Comments
% ====================================================

\newcommand {\marta}[1]{{\color{red}\textbf{?MS: }#1}\normalfont}
\newcommand {\naruki}[1]{{\color{blue}\textbf{?NY: }#1}\normalfont}
\newcommand {\kourosh}[1]{{\color{green}\textbf{?KD: }#1}\normalfont}
\newcommand {\florian}[1]{{\color{brown}\textbf{?FS: }#1}\normalfont}
\newcommand {\animesh}[1]{{\color{orange}\textbf{?AG: }#1}\normalfont}
\newcommand {\lasse}[1]{{\color{pink}\textbf{?LBK: }#1}\normalfont}


\graphicspath{ {figures/} }


\usepackage[normalem]{ulem}
% \usepackage{graphicx}
% \graphicspath{ {fig/} }

% \usepackage{bm}
% \usepackage{amsmath}
% \usepackage{amssymb}
% \usepackage{hyperref}
% \usepackage[table]{xcolor}
% \usepackage{algorithm}
% \usepackage{algpseudocode}

% \usepackage{caption} % this package changes the caption font

% \usepackage{listings}
% \usepackage{xcolor}

% \usepackage{minted} 
% \usepackage{tcolorbox}
% \usepackage{etoolbox}
% \BeforeBeginEnvironment{minted}{\begin{tcolorbox}}%
% \AfterEndEnvironment{minted}{\end{tcolorbox}}%


% https://www.overleaf.com/learn/latex/Code_listing
\definecolor{codegreen}{rgb}{0,0.6,0}
\definecolor{codegray}{rgb}{0.4,0.4,0.4}
\definecolor{codepurple}{rgb}{0.5,0,0.9}
\definecolor{backcolour}{rgb}{0.95,0.95,0.95}

\lstdefinestyle{mystyle}{
    backgroundcolor=\color{backcolour},   
    commentstyle=\color{codegreen},
    keywordstyle=\color{magenta},
    numberstyle=\tiny\color{codegray},
    stringstyle=\color{codepurple},
    basicstyle=\fontsize{6.5}{7}\selectfont\ttfamily\ttfamily,
    breakatwhitespace=false,         
    breaklines=true,
    breakindent=0pt,
    captionpos=b,                    
    keepspaces=true,                 
    numbers=none,                    
    numbersep=5pt,                  
    showspaces=false,                
    showstringspaces=false,
    showtabs=false,                  
    tabsize=2,
}
\lstset{style=mystyle}
\renewcommand{\lstlistingname}{Snippet}

%https://tex.stackexchange.com/questions/355937/how-to-add-input-and-output-before-algorithm-procedure
\renewcommand{\algorithmicrequire}{\textbf{Input:}}
\renewcommand{\algorithmicensure}{\textbf{Output:}}

% \setlength{\abovecaptionskip}{2.0mm}
% \setlength{\belowcaptionskip}{0.5mm} 
% \setlength{\textfloatsep}{1.5mm}
% \setlength{\dbltextfloatsep}{1.5mm}

\renewcommand{\arraystretch}{1.15}

\renewcommand{\paragraph}[1]{\vspace{0.1em}\noindent\textit{#1} --}

\newcolumntype{C}[1]{>{\centering\let\newline\\\arraybackslash\hspace{0pt}}m{#1}}


% working names: HERMES
\usepackage{xspace}
\def\ourmodel{\textsc{CLAIRify}\xspace}

\title{\LARGE \bf
%Large language model for chemistry text interpretation in robotics
%Compiler Errors are Good Prompters: Instruction Guided Task Programming with Compiler-assisted Iterative Prompting
%HERMES: Iteratively Using Compiler Errors to Generate Robot Plans from Natural Language
%Compiler Errors are Good Prompters: Iterative Program Generation from Natural Language for Lab Automation
%Iterative Task Plan Generation from Natural Language for Lab Automation
%Natural language to task plan compilation for lab automation
Errors are Useful Prompts: Instruction Guided Task Programming \\ with Verifier-Assisted Iterative Prompting
% \ourmodel: Instruction Guided Task Programming \\ with Verifier-Assisted Iterative Prompting
}

\author{Marta Skreta$^{\ast1,2}$,
        Naruki Yoshikawa$^{\ast1,2}$,
        Sebastian Arellano-Rubach$^{3}$,
        Zhi Ji$^{1}$,
        Lasse Bjørn Kristensen$^{1}$, \\
        Kourosh Darvish$^{1,2}$,
        Al\'{a}n Aspuru-Guzik$^{1,2}$,
        Florian Shkurti$^{1,2}$,
        Animesh Garg$^{1,2,4}$
\thanks{$^\ast$ Authors contributed equally,
    $^{1}$University of Toronto,
    $^{2}$Vector Institute,
    $^{3}$University of Toronto Schools,
    $^{4}$NVIDIA}
\thanks{Email: \texttt{\{martaskreta,naruki\}@cs.toronto.edu}}
\thanks{Email: \texttt{\{kdarvish,garg\}@cs.toronto.edu}}
}

\begin{document}
\bstctlcite{IEEEexample:BSTcontrol}
\maketitle
\thispagestyle{empty}
\pagestyle{empty}

\begin{abstract}
%Natural language instructions have the potential to bridge the communication gap between humans and robots. However,
Generating low-level robot task plans from high-level natural language instructions remains a challenging problem. Although large language models have shown promising results in generating plans, the accuracy of the output remains unverified. Furthermore, the lack of domain-specific language data poses a limitation on the applicability of these models. In this paper, we propose \ourmodel, a novel approach that combines automatic iterative prompting with program verification to ensure programs written in data-scarce domain-specific language are syntactically valid and incorporate environment constraints. Our approach provides effective guidance to the language model on generating structured-like task plans by incorporating any errors as feedback, while the verifier ensures the syntactic accuracy of the generated plans. We demonstrate the effectiveness of \ourmodel in planning chemistry experiments by achieving state-of-the-art results. We also show that the generated plans can be executed on a real robot by integrating them with a task and motion planner. 
%Our results demonstrate the potential of \ourmodel in bridging the gap between natural language instructions and robotic action execution.
\end{abstract}


\section{Introduction}

Leveraging natural language instruction to create a plan comes naturally to humans. However, when a robot is instructed to do a task, there is a communication barrier: the robot does not know how to convert the natural language instructions to lower-level actions it can execute, and the human cannot easily formulate lower-level actions.
%There have been many works to translate natural language instructions to executable program code~\cite{tellex2020robots,}.
Large language models (LLMs) can fill this gap by providing a rich repertoire of \textit{common sense reasoning} to robots~\cite{brown2020language, singh2022progprompt}.

% removed citations from list below: thoppilan2022lamda, zhang2022opt, rae2021scaling
Recently, there has been impressive progress in using LLMs~\cite{devlin2018bert, brown2020language,chowdhery2022palm} for problems involving structured outputs, including code generation~\cite{10.48550/arXiv.2107.03374, wang2021codet5, li2022competition} and robot programming~\cite{liang2022code}. These code generation models are often trained on code that is widely available on the Internet and perform well in few-shot settings for generating code in those languages.
% However, there are two main issues that current task-plan methods do not address:
However, to employ LLMs for task-plan generation there are two main issues to address:
(1)  \textit{lack of task-plan verification} and (2) poor \textit{performance for data-scarce domain-specific languages}.

% syntax check and rule based language

\textbf{\paragraph{Lack of task plan verification}} Task plans generated by LLMs, often, cannot be executed out-of-the-box with robots. There are two reasons for that. First, machine-executable languages are bound by strict rules \cite{pddlstream}.
%the generation of a task plan in a machine-executable language from natural language is bound by strict rules \cite{}. 
If the generated task plan does not adhere to them, it will not be executable. Hence, we need a way to verify the syntactic correctness of the structured task plan.
% natural language 2 Structure language compiler is required.
% to verify that the task plan generated from the natural language input follows the rules of the target language.
Second, LLMs might generate a task plan that looks reasonable (i.e. is syntactically correct) but is not actually executable by a robot. Avoiding this problem requires information about the world state and robot capabilities, as well as general reasoning about the physical world~\cite{brohan2022can}.

%if the following are not taken into account: general reasoning about the physical world~\cite{}, information about the current world state~\cite{}, robot embodiment or agent capabilities~\cite{}.

% Second, the generated task cannot reason upon the physical world and therefore there is no certainty of reaching the desired goal in the given prompt \cite{}. While generating the task plan sequence, if the information about the current world state is not taken into account, LLMs cannot generate an executable task plan~\cite{}.
% Moreover, in generation of tasks, LLMs do not count for the robot embodiment or the agent capabilities \cite{}. To address these challenges, different works tried to amend the command prompt provided by the human with information about the world model~\cite{}, robot embodiment~\cite{}, and workspace perceptual information~\cite{}.

\textbf{\paragraph{Data scarcity for domain-specific languages}} It is difficult for LLMs to generate task plans in a zero-shot manner for domain-specific languages (DSLs), such as those in chemistry and physics because there is significantly less data on the Internet for those specific domains, so LLMs are unable to generalize well with no additional information \cite{gu2021domain, 10.48550/arXiv.2210.05359}. It is possible to address this by fine-tuning models on pairs of natural-language inputs and structured-language outputs, but it is very difficult to acquire training datasets large enough for the model to learn a DSL reasonably well~\cite{wang2021want}, and there is a large computation cost for fine-tuning LLMs \cite{bannour-2021-evaluating}. However, it has been shown that LLMs can adapt to new domains with \textit{effective prompting}~\cite{mishra2021cross}. Our insight is \textit{to leverage the in-context ability of an LLM by providing the rules of a structured language as input}, to generate a plan according to the template of the target DSL.


\begin{figure}[!t]
    \includegraphics[width=0.9\linewidth]{motivation_2.pdf}
    \centering
    \caption{Task plans generated by LLMs may contain syntactical errors in domain-specific languages. By using verifier-assited iterative prompting, \ourmodel can generate a valid program, which can be executed by a robot.}%\animesh{change layout to make the figure tall instead of wide! each subelement should still be readable if neecessary. the whoel code block does not need to be there. just a couple lines with ...(dots) should do. this is an illustration!}}
    \label{fig:motivation}
\end{figure}

In this work, we propose to address the verification and data-scarcity challenges. We introduce \ourmodel\footnote{\textbf{\ourmodel website:} \href{https://ac-rad.github.io/clairify/}{https://ac-rad.github.io/clairify/}}, a framework that translates natural language into a domain-specific structured task plan using an automated iterative verification technique to ensure the plan is syntactically valid in the target DSL (Figure~\ref{fig:motivation}) by providing the LLM a description of the target language. Our model also takes into account environment constraints if provided.
%In this work, we attempt to address the verification and data-scarcity challenges with our model\kousays{,} \ourmodel. \ourmodel translates natural language into a domain-specific structured task plan using an automated iterative verification technique to ensure the plan is syntactically valid in the target DSL (Figure~\ref{fig:motivation}) by providing the LLM a description of the target language. Our model also takes into account any environment constraints if they are provided.
%\ourmodel uses an LLM to generate a task plan from an input prompt that contains the domain definition and rules of the target DSL, as well as the natural language instruction we want to create a plan for. 
The generated structured-language-like output is evaluated by our verifier, which checks for syntax correctness and for meeting environment constraints. The syntax and constraint errors are then fed back into the LLM generator to generate a new output. This iterative interaction between the generator and the verifier leads to grounded syntactically correct target language plans. 

We evaluate the capabilities of \ourmodel using a domain-specific language called Chemical Description Language (XDL)~\cite{10.1126/science.abc2986} as the target structured language unfamiliar to the LLM.
XDL is an XML-based DSL to describe action plans for chemistry experiments in a structured format, and can be used to command robots in self-driving laboratories~\cite{seifrid2022autonomous}. Converting experiment descriptions to a structured format is nontrivial due to the large variations in the language used. Our evaluations show that \ourmodel outperforms the current state-of-the-art XDL generation model in \cite{10.1126/science.abc2986}. We also demonstrate that the generated plans are executable by combining them with an integrated task and motion planning (TAMP) framework and running the corresponding experiments in the real world.
Our contributions are:

\begin{itemize}
    \item We propose a framework to produce task plans in a DSL using an iterative interaction of an LLM-based generator and a rule-based verifier.
    \item We show that the interaction between the generator and verifier improves zero-shot task plan generation.
    \item Our method outperforms the existing XDL generation method in an evaluation by human experts.
    % \item \sout{We integrate our generated plans with a TAMP framework in a real robot to demonstrate the successful translation of elementary simple chemistry experiments.}
    \item We integrate our generated plans with a TAMP framework, and demonstrate  the successful translation of elementary chemistry experiments to a real robot execution.
\end{itemize}
    %that the generated plans are executable in the real-world by incorporating a TAMP framework ensure the 
    
    %executability of XDL plans generated from natural language using a TAMP framework, and execute them in a real robot to demonstrate the successful translation of simple chemistry experiments.

%, there are works that describe fine-tuning models to generate a structured task plan from a natural language input. However, it is very difficult to acquire a training dataset large enough for the model to learn a language reasonably well from. 

%However, when these models are applied in a few-shot context to domain-specific languages (DSLs), such as those in chemistry and physics, or rarely-used languages, the outputs are often nonsensical because the LLM has not seen many examples of those languages and cannot generalize well. In chemistry, translating   In the past, models to convert natural language to a DSL have been trained on  

% competitive programming,
%have been used for code generation  advances in large language models have shown 

%Communication in natural language is convenient for humans to convey an instruction to others. However, the same is not true 
%Therefore, instructing a robot in natural language has been attempted~\cite{tellex2020robots}.
%Natural language processing technique is applied to make robots understand language.

% Recent progress in large language models (LLMs)~\cite{devlin2018bert, brown2020language, thoppilan2022lamda, hoffmann2022training, rae2021scaling, chowdhery2022palm, zhang2022opt, black2022gpt} %such as BERT~\cite{devlin2018bert}, GPT-3~\cite{brown2020language}, LaMDA~\cite{thoppilan2022lamda}, Chinchilla~{hoffmann2022training}, Gopher~\cite{rae2021scaling}, PaLM~\cite{chowdhery2022palm}, OPT-175B~\cite{zhang2022opt}, and GPT-NeoX-20B~\cite{black2022gpt},
% has greatly improved the score in natural language understanding tasks.
%These LLMs have been applied for general code generation~\cite{10.48550/arXiv.2107.03374, wang2021codet5}, competitive programming~\cite{li2022competition}, and robot programming~\cite{liang2022code}.
% These code generation models are trained with abundant program codes widely available on the Internet.
%The attempt to make the reasoning of language models grounded has been made~\cite{10.48550/arXiv.2210.05359,patel2022mapping}.

%However, the size of training data in the target programming language is not always sufficient when generating a new domain-specific language (DSL).
%Many DSLs are designed to meet the needs of specialized fields.
%In chemistry, many works of literature, such as scientific journals, textbooks, and patent documents, describe chemistry experiments in natural languages.
%Text-mining tools have been developed~\cite{lowe2012extraction,schneider2016big} to make use of these data, but access to standardized reaction data is still problematic in developing artificial intelligence in chemistry~\cite{de2019synthetic} and building self-driving laboratory~\cite{seifrid2022autonomous}.
%Therefore, the standardization of chemical information is advocated~\cite{}, and  several domain-specific languages for chemistry automation have been proposed~\cite{10.1126/science.abc2986, park2023extensible}.
%Translating from existing natural language resources into these new structured languages is required to accelerate automation in chemistry.




% Among emergent abilities of LLMs~\cite{wei2022emergent}, the ability of zero-shot reasoning~\cite{wei2022finetuned, kojima2022large, zhou2022large} is notable.

%It is known that LLMs can solve unknown problems with prompt engineering, so they have the potential to generalize to unknown DSLs if combined with an appropriate prompt.
%Nevertheless, the output of these models sometimes lacks truthfulness~\cite{lin2021truthfulqa} and factfulness~\cite{petroni2020context}, as they are mainly focused on language-internal tasks~\cite{mcclelland2020placing}.

%In order to overcome the problem that LLMs may lack truthfulness when generating data-scarce language, we propose \ourmodel: a method to translate natural language into a structured programming language unfamiliar to the underlying language model with an automated iterative technique.



%Clearly provide a list of of contributions as a list... Ideally no more than 4 points. 


%\lorem{2}

% =========================================================
\section{Related work}

% \animesh{template for each subsection: This problem is attempted in the paxt by x, y,z. X does this, y does that, and z does something else. 
% However, there are shortcomings such as a, b, c. The closest to this work is D. They do X., but it is not a drag and drop replacement since.. . (If relevant) Our method builds on the ideas from D and addresses the shortcomings of prior work... through....}


% \subsection{planning/task programmning}
% Neural task programming~\cite{xu2018neural}
% Neural task graphs~\cite{huang2019neural}
% PG-TD~\cite{zhang2022planning}.
% \animesh{refer to related work in neural task programming and neural task graphs. 
% use tamp refs from the ChemTAMP paper. 
% }
\subsection{Task Planning} 
%\subsection{Language Guided Robot Planning}
High-level task plans are often generated from a limited set of actions \cite{pddlstream}, because task planning becomes intractable as the number of actions and time horizon grows \cite{kaelbling2011hierarchical}.
% as unconstrained task planning can be difficult in environments with a large action space.
One approach to do task planning is using rule-based methods~\cite{10.1126/science.abc2986, baier2009heuristic}. More recently, it has been shown that models can learn task plans from input task specifications \cite{sharma2021skill, mirchandani2021ella, shah2021value}, for example using hierarchical learning \cite{xu2018neural,huang2019neural}, regression based planning \cite{xu2019regression}, reinforcement learning \cite{eysenbach2019search}. However, to effectively plan task using learning-based techniques, large datasets are required that are hard to collect in many real-world domains. Our approach, on the other hand, generates a task plan directly from an LLM in a zero-shot way on a constrained set of tasks which are directly translatable to robot actions. We ensure that the plan is syntactically valid and meets environment constraints using iterative error checking. 

\subsection{Task Planning with Large Language Models}

%The study on guiding robots using natural language has a long history \cite{tellex2020robots} beginning with \cite{winograd1971procedures}.

% A detailed review on this topic before 2020 is found in .
% Nowadays, LLM is utilized to introduce language understanding for robots.
Recently, many works have used LLMs to translate natural language prompts to robot task plans~\cite{brohan2022can, huang2022inner, liang2022code, singh2022progprompt}.  For example, Inner Monologue~\cite{huang2022inner} uses LLMs in conjunction with environment feedback from various perception models and state monitoring. However, because the system has no constraints, it can propose plans that are nonsensical. SayCan~\cite{brohan2022can}, on the other hand, grounds task plans generated by LLMs in the real world by providing a set of low-level skills the robot can choose from. A natural way of generating task plans is using code-writing LLMs because they are not open-ended (i.e. they have to generate code in a specific manner in order for it to be executable) and are able to generate policy logic. Several LLMs trained on public code are available, such as Codex~\cite{10.48550/arXiv.2107.03374},  CodeT5~\cite{wang2021codet5}, AlphaCode~\cite{li2022competition} and CodeRL~\cite{le2022coderl}. %approached competitive programming using a transformer-based model.
%CodeRL~\cite{le2022coderl} introduced reinforcement learning to leverage pre-trained language models for competitive programming.
%However, training these LLMs in the approaches rely on the existence of large amounts of training data. 
LLMs can be prompted in a zero-shot way to generate task plans. For example, Code as Policies~\cite{liang2022code} repurposes code-writing LLMs to write robot policy code and ProgPrompt~\cite{singh2022progprompt} generates plans that take into account the robot's current state and the task objectives. However, these methods generate Pythonic code, which is abundant on the Internet. For DSLs, naive zero-shot prompting is not enough; the prompt has to incorporate information about the target language so that the LLM can produce outputs according to its rules. 

%Furthermore, these methods do not evaluate the quality of generated plan expressed as code.

%realize the real-world grounding of an LLM by providing a set of low-level skills and choosing from them.


%as the plan is  to improve the successful completion rate of high-level instruction given in natural language.


%LLMs have also been used to generate task plan
%Code as Policies~\cite{liang2022code} repurposed code-writing LLM to write robot policy code.
%ProgPrompt~\cite{singh2022progprompt} presents a system that generates task plans for robots using natural language prompts. The system utilizes a large language model to generate a plan that takes into account the robot's current state and the task objectives.
%These works target daily life settings, and the applicability to a specialized field is not explored.
%Also, they do not evaluate the quality of generated plan expressed as code.
%We target chemistry literature and evaluate the quality of code generation.
% \animesh{refer to related work from Progprompt.}

%\subsection{Program Synthesis with Large Language Models}

%Source code generation has been attempted from the past by many researchers~\cite{gulwani2017program}.
%Traditionally, program synthesis is studied in the field of logic, especially in the context of automated theorem prover~\cite{manna1971toward}.
%Another way to approach problem synthesis is to consider program codes as a language and apply neural language models.


%SynthReader~\cite{10.1126/science.abc2986} is a grammar-based approach to generate XDL from natural language input.
%It uses pattern-matching techniques on hard-coded fragments from chemistry literature to convert natural language input into structured language.
%This approach does not require training data, but pattern-matching has inherent limitations for unexpected input and context understanding.
%Our method addresses the shortcomings of prior work by using a large language model to understand the input language better and the interactive prompting strategy to improve zero-shot code generation performance.

\subsection{Leveraging Language Models with External Knowledge}
A challenge with LLMs generating code is that the correctness of the code is not assured.
There have been many interesting works on combining language models with external tools to improve the reliability of the output.
Mind's Eye~\cite{10.48550/arXiv.2210.05359} attempts to ground large language model's reasoning with physical simulation. 
They trained LLM with pairs of language and codes and used the simulation results to prompt an LLM to answer general reasoning questions.
%They train a language model from 200k samples to convert a natural language input to code, which is then run in a physics a simulator. 
%They use the simulation results to prompt an LLM with general reasoning questions. 
% Its authors trained a language model on 200k pairs of the question text and corresponding simulation code to realize grounded reasoning.
Toolformer~\cite{schick2023toolformer} incorporates API calls into the language model to improve a downstream task, such as question answering, by fine-tuning the model to learn how to call API. 
%However, these previous works assume the existence of a large set of training data, which does not apply in our setting.
LEVER~\cite{10.48550/arXiv.2302.08468} improves LLM prompting for SQL generation by using a model-based verifier trained to verify the generated programs. %with their execution results.
As SQL is a common language, the language model is expected to understand its grammar. However, for DSLs, it is difficult to acquire training datasets % to train a verifier 
and %it is also 
expensive to execute the plans to verify their correctness. 
Our method does not require fine-tuning any models or prior knowledge on the target language within the language model. Our idea is perhaps closest to LLM-A\textsc{ugmenter}~\cite{peng2023check}, which improves LLM outputs by giving it access to external knowledge and automatically revising prompts in natural language question-answering tasks. Our method similarly encodes external knowledge in the structure of the verifier and prompts, but for a structured and formally verifiable domain-specific language.

%: we verify that the output of the LLM is syntactically correct according to the rules of the DSL and we constrain the task plans to ensure that they do not involve any resources not available in the environment.

% \lorem{4}

\begin{figure*}[!t]
    \begin{minipage}{0.8\linewidth}
    \centering
    \includegraphics[width=0.99\linewidth]{pipeline.pdf}
    \end{minipage}
    \begin{minipage}{0.19\linewidth}
    \caption{\textbf{System overview}: The LLM takes the input (1), structured language definition, and (optionally) resource constraints and generates unverified structured language (2). The output is examined by the verifier, and is passed to LLM with feedback (3). The LLM-generated outputs passes through the verifier (4). The correct output (5) is passed to the task and motion planner to generate robot trajectories. The robot executes the planned trajectory (6).}%\animesh{explain in detail, including what the numerical steps mean. }}
    \label{fig:system}
    \end{minipage}
\end{figure*}


% ===============================
\section{Task Programming with \ourmodel}

%\paragraph{Problem Statement}
%We will solve the problem of zero-shot program generation with LLM, i.e. make LLM generate a structured language on which LLM was not trained by giving a proper prompt and external knowledge from automated iterative prompting.

\paragraph{\ourmodel Overview}
We present a system that takes as input the specifications of a structured language (i.e. all its rules and permissible actions) as well as a task we want to execute written in natural language and outputs a syntactically correct task plan. 
A general overview of the \ourmodel pipeline is in Figure~\ref{fig:system}.
We combine input instruction and language description into a prompt and pass the prompt into the structured language generator (here we use GPT-3~\cite{brown2020language}, a large language model). However, we cannot guarantee the output from the generator is syntactically valid, meaning that it would definitely fail to compile into lower-level robot actions. To generate syntactically valid programs, we pass the output of the generator through a verifier. The verifier determines whether the generator output follows all the rules and specifications of the target structured language and can be compiled without errors. If it cannot, the verifier returns error messages stating where the error was found and what it was. This is then appended to the generator output and added to the prompt for the next iteration.
%If the generator's output was invalid, the structured language output from the previous iteration and the feedback from the verifier are added to the input prompt for the next step.
This process is repeated until a valid program is obtained, or until the timeout condition is reached (if that happens, an error is thrown for the user). Algorithm~\ref{alg:alg1} describes of our proposed method.
% In Algorithm~\ref{alg:alg1}, we refer to $\mathcal{L}$ as the structured language description, $x$ as the natural language input, $y_{SL'}$ as the structured-language-like task plan output from the generator, errors as the list of errors from the verifier, and $y_{SL}$ as the guaranteed structured language task plan if it passes the verifier check.


%We present a system that takes in as input the specifications of a structured language (i.e. all its rules and permissible actions) as well a natural language input to be translated into structured language. We combine them into a prompt and pass the prompt into the structured language generator (here we use GPT-3, a large language model). However, we cannot guarantee the output from the generator is syntactically valid. To generate syntactically valid programs, we pass the output of the generator through a verifier. The verifier determines whether the generator output follows all the rules and specifications of the target structured language and can be compiled without errors. If it does not, the verifier returns error messages stating where the error was found and what it was. This is then appended to the generator output and added to the prompt for the next iteration. This process is repeated until a valid program is obtained, or until the timeout condition is reached (if that happens, an error is thrown for the user). The algorithm of our proposed method is shown in Algorithm~\ref{alg:alg1}. In the context of robotics, this workflow allows for translation from natural language to plans described in structured language in a way that guarantees valid structure and thus allows for more dependable tranlsation to lower-level actions and execution by robots. A general overview of our pipaline is in Figure~\ref{fig:system}.

Once the generator output passes through the verifier with no errors, we are guaranteed that it is syntactically valid structured language. This program can then be translated into lower-level robot actions by passing through TAMP for robot execution. Each component of the pipeline is described in more detail below. 

\paragraph{\ourmodel in Chemistry Lab Automation} While our pipeline can in theory be applied to any structured language, we demonstrate it using the chemical description language (XDL)~\cite{10.1126/science.abc2986} as an example of a structured language. 
%XDL was developed to translate chemistry methods of papers written in natural language into a structured, machine-readable XML format that can be executed in automated chemistry experiment hardware. 
XDL describes the hardware and reagents to be used in the experiment, and the experimental procedures in chronological order.
Note that chemistry proceures are not well standardized; there is a lot of variation, and so translating them to a structured plan that can be executed by a robot is nontrivial~\cite{Vaucher2020}. %Once they are converted into a valid structured plan, the plan can be executed by real-world robotics systems. 

\subsection{Generator}
The generator takes a user's instruction and generates structured-language-like output using a large language model (LLM) using a description of the structured language. The input prompt skeleton is shown in Snippet 1, Figure~\ref{fig:prompt}.
%The input prompt to the LLM is composed of a description of the target language, a sentence specifying what the LLM should do (i.e. ``Convert the following to XDL"), 
%, the command to the LLM, 
%and the natural language instruction the task plan should be created for.
%A description of the target structured language is provided to guide the LLM in understanding the unfamiliar language. 
The description of the XDL language includes its file structure and lists of the available actions (can be thought of as functions), their allowed parameters and their documentation. 
%The natural language instructions are provided by a dataset or users and should contain instructions that specify the steps involved in the experiment, including any information on the chemicals, reagents, and procedures.
%This contributes to the program synthesis performance because the LLM trained on existing natural language datasets may not have knowledge of the target language, especially if it is not common.

%T he command to the LLM is a simple sentence to specify the desired action of the LLM, such as \textit{"Convert to XDL:"}.

%At the beginning of the iteration, the prompt includes the description of the target language and a natural language instruction, as shown in Snippet 1, Figure~\ref{fig:prompt}.

% Maybe having a new section to describe the iterative generation process after the verifier section may be easier for the readers (?)
Although the description of the target structured language information is provided, the candidate task plan is not guaranteed to be syntactically correct (hence we refer to is as ``structured-language-like'').
To ensure the syntactical correctness of the generated code, the generator is iteratively prompted by the automated interaction with the verifier.
The generated code is passed through the verifier, and if no errors are generated, then the code is syntactically correct.
If errors are generated, we re-prompt the LLM with the incorrect task plan from the previous iteration along with the list of errors indicating why the generated steps were incorrect.
The skeleton of the iterative prompt is shown in Snippet 2, Figure~\ref{fig:prompt}. 
Receiving the feedback from the verifier is used by the LLM to correct the errors from the previous iteration. This process is continued until the generated code is error-free or a timeout condition is reached, in which case we say we were not able to generate a task plan.

\begin{algorithm}[!t]
\footnotesize
\caption{{\footnotesize \ourmodel: Verifier-Assisted Iterative Prompts}}
\label{alg:alg1}
\begin{algorithmic}
\Require{Structured language description $\mathcal{L}$, instruction $x$}
\Ensure{Structured language task plan, $y_{SL}$}
\Procedure{IterativePrompting}{$\mathcal{L}$, $x$}
\State $y_{SL'} = \text{Generator}(\mathcal{L}, x)$
\State $ \text{errors} = \text{Verifier}(y_{SL'})$

\While{$\text{len(errors)} > 0 \text{ or timeout condition != True}$}
   \State $y_{SL'} = \text{Generator}(\mathcal{L}, x, y_{SL'}, \text{errors})$ 
   \State $ \text{errors} = \text{Verifier}(y_{SL'})$
   \EndWhile
\State $y_{SL} = y_{SL'}$
\EndProcedure
\end{algorithmic}
\end{algorithm}


% \subsection{Generator}
% The generator generates a structured language from the user's input using a large language model (LLM), the DaVinci 003 model which is a part of the GPT-3 generation.
% As LLM is trained in language data from existing literature, we need to instruct LLM about the grammar of the structured language if there only exists scarce data of the target language in train data.
% The data of XDL is not abundant on the Internet, so it is a good case study for few-shot source code generation.

% The XDL generator is a Python program that leverages the power of a LLM to generate XDL code from natural language instructions. The LLM is a pre-trained neural network that uses a syntax definition of the XDL language and natural language instructions to generate structured code. The syntax definition is provided to the LLM to ensure that the natural language instructions are correctly parsed and interpreted. This is important because the LLM is trained on vast amounts of natural language data and may not be specifically optimized for XDL. The syntax definition acts as a guide to help the LLM understand the specific context and language of the experiment.

% The natural language instructions provided to the XDL generator are in the form of a set of instructions that specify the steps involved in the experiment, including information on the chemicals, reagents, and procedures. These instructions are analyzed by the LLM using its understanding of natural language to generate a corresponding XDL representation of the experiment. The LLM is prompted by a function that takes in the description of the structured language and some word prompts to generate the XDL code.

% The XDL description is used to provide a framework for generating XDL-like code. This description contains a description of the XDL language, including its syntax, semantics, mandatory sections, and any actions and parameters that may be used. The generated XDL code is output in XML format, with mandatory sections and any actions and parameters translated to fit the XDL language. 

% \begin{itemize}
% \item How are the syntax of the structured language specified? (In natural language, not a formal language such as BNF), a brief example of how it is given may help
% \item How LLM is prompted (Example prompt). Structure of prompt, description of structured language + some word to urge LLM to generate XDL (i.e. "Convert to XDL") + natural language explanation of the content of XDL
% \end{itemize}


\subsection{Verifier}
The verifier works as a syntax checker and static analyzer to  check the output of the generator and send feedback to the generator.
%Although the generator is prompted with a knowledge of the target structured language, the generated language is not guaranteed to follow the grammar and the restrictions.
%and returns a list of errors for any problems in the input.
It first checks whether the input can be parsed as a correct XML and then checks the allowance of action tags, the existence of mandatory properties, and the correctness of optional properties.
This evaluates if the input is syntactically correct XDL.
It also checks the existence of definitions of hardware and reagents used in the procedure or provided as environment constraints, which works as a simple static analysis of necessary conditions for executability. If the verifier catches any of these errors, the candidate task plan is considered to be an invalid. The verifier returns a list of errors it found, which is then fed back to the generator.
%Optionally, the verifier takes the list of available hardware and reagents and checks the availability of resources used in the input XDL.
%Finally, it determines whether the input XDL is correct; 




%input XDL fails any of these checks, the verifier returns the list of errors and the corresponding input line to be served to the generator as feedback.
% \begin{itemize}
%     \item Parsability as XML 
%     \item Validness of actions and properties
%     \item Availability of hardware and reagent
%     \item Error message generation
% \end{itemize}
% In more general terms,
% \begin{itemize}
%     \item Syntax check (XML)
%     \item Terminology check, Lexical check
%     \item Necessary condition of executability, hardware availability
%     \item XML initial state $\subset$ Robotic Workspace initial state
% \end{itemize}

\subsection{Incorporating Environment Constraints}
Because resources in a robot workspace are limited, we need to consider those constraints when generating task plans.
%Because lab environments do not have infinite hardware and materials, we need to take those constraints into account when generating task plans. 
%The generated XDL should only use the available hardware and reagents. 
If specified, we include the available resources 
% hardware and reagents
into the generator prompt.
The verifier also catches if the candidate plan uses any resources aside from those mentioned among the available robot resources.
% environment constraints. 
% available robot resources. materials, and
Those errors are included in the generator prompt for the next iteration. 
%checks whether the XDL plan only includes available hardware and reagents. If it does not, an error is returned specifying the violated constraints, which is included in the generator prompt for the next iteration. 
If a constraint list is not provided, we assume the robot has access to all resources.
In the case of chemistry lab automation, those resources include experiment hardware and reagents. 

\subsection{Interfacing with Planner on Real Robot}
In many cases, the target DSL is not embodied, and it is hardware independent. Our verified task plan only contains high-level descriptions of actions.
To execute those actions by a robot, we need to map them to low-level actions and motion that the robot can execute.
To ensure the generated structured language is executable by the robot, we employ a task and motion planning (TAMP) framework. 
In our case, we use PDDLStream~\cite{pddlstream} to generate robot action and motion plans simultaneously.
In this process, visual perception information, coming from the robot camera, grounds the predicates for PDDLStream, and verified task plans are translated into problem definitions in PDDLStream.
% The generated plan by PDDLStream (if any exists) is executable.

For the chemistry lab automation domain, high-level actions in XDL are mapped to intermediate goals in PDDLStream, resulting in a long-horizon multistep planning problem definition.
% In this way, \textit{actions} in XDL act as a high-level guide for the task execution by the robot.
To also incorporate safety considerations for robot execution, We use a constrained task and motion planning framework for lab automation~\cite{10.48550/arXiv.2212.09672} to execute the XDL generated by \ourmodel.

% We use a constrained task and motion planning framework for lab automation~\cite{10.48550/arXiv.2212.09672} to execute the XDL generated by \ourmodel to incorporate safety considerations for robot execution.

% Combined with the input from the perception module that contains the grounded information of the robot workspace, such as object poses, XDL instructions are translated into problem definitions in PDDLStream.
% In this process, visual perception information, coming from the robot camera, grounds the predicates for the task and motion planning. 
%While the plan generated in other works \cite{} is not ensured to be executable or lead to the final task goal, 
%A constrained motion planner is used to generate a trajectory to avoid spillage of liquid inside the container by keeping the end-effector orientation constrained during robot motion.
% such as adding a solution to a vessel.


% Verified task plan only contains high-level descriptions of actions, such as adding a solution to a vessel. However, to execute those actions with a robot, we need to generate actual robot actions and motion that is feasible.
% To ensure the generated structured language is executable by the robot, we employ a task and motion planning (TAMP) framework. 
% %We use deCLAIR~\cite{10.48550/arXiv.2212.09672}, a constrained task and motion planning framework for lab automation, to execute the XDL generated by \ourmodel.
% We use a constrained task and motion planning framework for lab automation~\cite{10.48550/arXiv.2212.09672} to execute the XDL generated by \ourmodel.
% It uses PDDLStream~\cite{pddlstream} to generate robot action and motion plans simultaneously. 
% For this purpose, each \textit{action} in XDL is a one-to-one mapping to intermediate goals in the PDDLStream, resulting in a long-horizon multistep planning problem definition.
% In this way, \textit{actions} in XDL act as a high-level guide for the task execution by the robot.
% In this process, visual perception information, coming from the robot camera, grounds the predicates for the task and motion planning. 
% Combined with the input from the perception module that contains the grounded information of the robot workspace, such as object poses, XDL instructions are translated into problem definitions in PDDLStream.
% %While the plan generated in other works \cite{} is not ensured to be executable or lead to the final task goal, 
% The generated plan by PDDLStream (if any exists) is ensured to be executable \cite{pddlstream}.
% %A constrained motion planner is used to generate a trajectory to avoid spillage of liquid inside the container by keeping the end-effector orientation constrained during robot motion.


\begin{figure}[!t]
    %\includegraphics[width=0.9\columnwidth]{llm_prompt.png}
\begin{minipage}[t]{0.63\linewidth}
    \centering
\begin{lstlisting}[language=Python, caption=Initial prompt]
initial_prompt = """
# <Description of XDL>

# <Hardware constraints(optional)>
# <Reagent constraints (optional)>

Convert to XDL:
# <Natural language instruction>
"""
\end{lstlisting}

\begin{lstlisting}[language=Python, caption=Iterative prompt]
iterative_prompt = """
# <Description of XDL>

# <Hardware constraints(optional)>
# <Reagent constraints (optional)>

Convert to XDL:
# <Natural language instruction>
# <XDL from previous iteration>
This XDL was not correct.
There were the errors
# <List of errors, one per line>
Please fix the errors
"""
\end{lstlisting}
\end{minipage}
\hfill
\begin{minipage}[t]{0.35\linewidth}
\centering
\vspace{0.5em}
\caption{\textbf{Prompt skeleton}: (1) At the initial generation, we prompt the LLM with a description of XDL and the natural language instruction. (2) After the LLM generates structured-language-like output, we pass it through our verifier. If there are errors in the generated program, we concatenate the initial prompt with the XDL from the previous iteration and a list of the errors.}
    \label{fig:prompt}
\end{minipage}
\end{figure}

% Other works robot affordances are passed to the LLM \cite{} or a learning technique \cite{} to generate robot action plans and later a motion generator., executable robot actions, here we use PDDLStream in TAMP.
% It accounts for the robot affordances as well as the feasibility of action execution during the planning time \cite{}. 
% The generated structured language is further converted into a robot-executable format.
% Since XDL only contains high-level descriptions of actions, such as adding a solution in a vessel, we need to generate an actual robot motion to execute the described experimental procedure in an actual robot.
% We used deCLAIR~\cite{10.48550/arXiv.2212.09672} to execute the XDL generated by \ourmodel.
% In deCLAIR, each of the actions described in XDL is considered to be a goal of action.
% Combined with the input from the perception module that contains the information on object pose, XDL instructions are translated into problem definitions in PDDLStream~\cite{pddlstream}.
% The PDDLStream planner solves a task and motion planning (TAMP) problem defined from XDL, and it generates motion plans and robot trajectories.
% A constrained motion planner is used to generate a trajectory to avoid spillage of liquid inside the container by maintaining the end-effector orientation during robot motion.



% ===========================================
\section{Experiments and Evaluation}
%\todo{Our experiments are designed to evaluate the following hypothesis. x, y, z... }
Our experiments are designed to evaluate the following hypotheses: i) Automated iterative prompting increases the success rate of unfamiliar language generation, ii) The quality of generated task plans is better than existing methods, iii) Generated plans can be executed by actual hardware.

% - automatic verifier guided prompting improve final planning
% - cost of iteration is zero because of automation
% - human intervention is more expensive than resource
% - in-context learning is not sufficient, our verifier is needed
% - interfacibility with our robot motion framework
% - one cherry-picked example of iterations
% - some analysis in execution

% hypothesis mirrors contributions


 \begin{table}[!t]
    \centering
    \caption{Comparison of our method with existing methods on the number of successfully generated valid XDL plans and their quality on 108 organic chemistry experiments from \cite{https://doi.org/10.5281/zenodo.3955107}.}
    \resizebox{\columnwidth}{!}{%
    \begin{tabular}{cccc}
        \hline
         Dataset & Method &  Number generated ↑ & Expert preference ↑ \\
         \hline\hline
         Chem-RnD & SynthReader \cite{10.1126/science.abc2986}   & 92/108 & 13/108\\
         & \ourmodel [ours] & \textbf{105/108} & 75/108 \\
         \hline
         Chem-EDU & SynthReader \cite{10.1126/science.abc2986}   & 0/40 & -\\
         & \ourmodel [ours] & \textbf{40/40} & - \\
         \hline
    \end{tabular}
    }
    \label{tab:comp_xdl}
\end{table}


\subsection{Experimental Setup}
To generate XDL plans, we use \texttt{text-davinci-003}, the most capable GPT-3 model at the time of writing. We chose to use this instead of \texttt{code-davinci-002} due to query and token limits. 

To execute the plans in the real world, we use an altered version of Franka Emika Panda arm robot, equipped with a Robotiq 2F-85 gripper, to handle vessels. The robot also communicates with instruments in the chemistry laboratory, such as a weighing scale and a magnetic stirrer. These devices are integrated to enable pouring and stirring skills.

%\lorem{1}


\begin{figure*}[!t]
    \centering
    \includegraphics[width=0.63\linewidth]{violinplot_chemrnd.png}
    \includegraphics[width=0.36\linewidth]
    {violinplot_chemedu.png}
    \caption{\textbf{Violin plots showing distributions of different error categories in XDL plans generated for experiments for the Chem-RnD (left) and Chem-EDU (right) datasets}. The x-axis shows the error categories and the y-axis shows the number of errors for that category (lower is better). For the Chem-RnD dataset, we show the error distributions for both \ourmodel and SynthReader. Each violin is split in two, with the left half showing the number of errors in plans generated from \ourmodel (teal) and the right half showing those from SynthReader (navy). For the Chem-EDU dataset, we only show the distributions for \ourmodel. In both plots, we show the mean of the distribution with a gold dot (and the number beside in gold) and the median with a grey dot.}
    \label{fig:error_breakdown_2}
\end{figure*}
%\ourmodel performs better or at least as well as SynthReader in 4 out of 6 categories when considering the means and medians of the distributions.


\paragraph{\textbf{Datasets}} We evaluated our method on two different datasets: 

(1) \textbf{{Chem-RnD} [Chemistry Research \& Development]}:  This dataset consists of 108 detailed chemistry-protocols for synthesizing different organic compounds in real-world chemistry labs, sourced from the Organic Syntheses dataset (volume 77)~\cite{https://doi.org/10.5281/zenodo.3955107}. Due to GPT-3 token limits, we only use experiments with less than 1000 characters. We use Chem-RnD as a proof-of-concept that our method can generate task plans for complex chemistry methods. We do not aim to execute the plans in the real world, and so we do not include any constraints.
%assume that we have access to every material.
%We evaluated the performance of our XDL generator on 108 realistic chemistry procedures from the Organic Syntheses dataset (volume 77)~\cite{https://doi.org/10.5281/zenodo.3955107}.
%We compared the performance of our method with SynthReader, the state-of-the-art XDL generation algorithm, which is based on rule-based tagging and grammar parsing of chemical procedures \cite{10.1126/science.abc2986}.

(2) \textbf{Chem-EDU [Everyday Educational Chemistry]}: We evaluate the integration of \ourmodel with real-world robots through a dataset of 40 natural language instructions containing only safe (edible) chemicals and that are, in principle, executable by our robot. The dataset consists of basic chemistry experiments involving edible household chemicals, including acid-base reactions and food preparation procedures\footnote{\textbf{\ourmodel Data \& code:} \href{https://github.com/ac-rad/xdl-generation/}{https://github.com/ac-rad/xdl-generation/}}.
%The recipes were adapted to the capabilities of the robot and formulated as natural language text. In addition, recipes of well-known mixed drinks were similarly formulated as natural language text and added to the data set. 
%Apart from the inherent variation in the ingredients of the recipes, the data set was also constructed to have variations in the wording used in order to better test the generalization-capabilities of the system.
When generating the XDL, we also included environment constraints based on what equipment our robot had access to (for example, our robot only had access to a mixing container called ``beaker"). 
%(2) Chem-EDU: Chemical procedures commonly used in Everyday Educational situations.....

%We use the Chem-RnD dataset because it provides realistic chemistry processes, showcasing how our model can generate plans for complicated multi-step, real-world lab automation tasks, although we do not have the system to execute them yet. To that end, we use Chem-EDU, which contains simpler processes, enabling us to showcase the end-to-end process on a real robot.

% \begin{figure}[!t]
%     \centering
%     \includegraphics[width=\linewidth]{violinplot_chemrnd.png}
%     \caption{\textbf{Violin plot showing distributions of different error categories in XDL plans generated for experiments in the Chem-RnD dataset by \ourmodel and SynthReader}. The x-axis shows the error categories and the y-axis shows the number of errors for that category (lower is better). Each violin is split in two, with the left half showing the number of errors in plans generated from \ourmodel (teal) versus plans from SynthReader (navy), which are in the right half.  We show the mean of the distribution with a gold dot (and the number beside in gold) and the median with a grey dot. \ourmodel performs better or at least as well as SynthReader in 4 out of 6 categories when considering the means and medians of the distributions.}
%     \label{fig:error_breakdown}
% \end{figure}
\subsection{Metrics and Results}
%\paragraph{\textbf{Metrics and Results}}
The results section is organized based on the four performance metrics that we will consider, namely: Ability to generate structured-language output, Quality of the generated plans, Number of interventions required be the verifier, and Robotic validation capability. We compared the performance of our method with SynthReader, a state-of-the-art XDL generation algorithm which is based on rule-based tagging and grammar parsing of chemical procedures \cite{10.1126/science.abc2986}

%We consider four principal metrics when evaluating the quality of \ourmodel, and organize the remainder of the results section in order of four metrics: Ability to generate structured-language output, Quality of the generated plans, Number of interventions required be the verifier, and Robotic validation capability:
%We consider four principal metrics when evaluating the quality  of \ourmodel. We organize the remainder of the Results section in order of these metrics:
    
   % \indent (1) \textbf{Ability to generate a structured language plan.} For \ourmodel, if it is in the iteration loop for more than $x$ steps (in our experiments, $x=10$), we say that it is unable to generate a plan and we exit the program. When comparing with SynthReader, we consider SynthReader unable to generate a structured plan if the SynthReader IDE (called ChemIDE\footnote{\href{https://croningroup.gitlab.io/chemputer/xdlapp/}{https://croningroup.gitlab.io/chemputer/xdlapp/}}) throws a fatal error when asked to create a plan. For both models, we also consider them unable to generate a plan if the generated plan only consists of empty XDL tags (i.e. no experimental protocol). For all experiments, we count the total number of successfully generated language plans divided by the total number of experiments.
    \indent (1) \textbf{Ability to generate a structured language plan.} First, we investigate the success probability for generating plans. For \ourmodel, if it is in the iteration loop for more than $x$ steps (here, we use $x=10$), we say that it is unable to generate a plan and we exit the program. When comparing with SynthReader, we consider that approach unable to generate a structured plan if the SynthReader IDE (called ChemIDE\footnote{ChemIDE using XDL: \href{https://croningroup.gitlab.io/chemputer/xdlapp/}{https://croningroup.gitlab.io/chemputer/xdlapp/}}) throws a fatal error when asked to create a plan. For both models, we also consider them unable to generate a plan if the generated plan only consists of empty XDL tags (i.e. no experimental protocol). For all experiments, we count the total number of successfully generated language plans divided by the total number of experiments. Using this methodology, we tested the ability of the two models to generate output on both the Chem-RnD and Chem-EDU datasets. The results for both models and both datasets are shown in Table ~\ref{tab:comp_xdl}. We find that out of 108 Chem-RnD experiments, \ourmodel successfully returned a plan 97\% of the time, while SynthReader returned a plan 85\% of the time. For the Chem-EDU dataset, \ourmodel generated a plan for all instructions. SynthReader was unable to generate any plans for that dataset, likely because the procedures are different from typical chemical procedures (they use simple action statements). This demonstrates the generalizability of our method: we can apply it to different language styles and domains and still obtain coherent  plans. 

    %\indent (2) \textbf{Quality of the predicted plan (without executing the plan)}. To determine if the predicted task plans actually accomplish every step of their original instructions, we report the number of actions and parameters that do not align between the original and generated plan, as annotated by expert experimental chemists. To compare the quality of the generated plans between \ourmodel and SynthReader, we ask expert experimental chemists to, given two anonymized plans, either pick a preferred plan amnong them or classify them as equally good. We also ask them to annotate errors in the plans in the following categories:  Missing action, Missing parameter, Wrong action, Wrong parameter, Ambiguous value, Other error. Here, actions refer to high-level steps in the procedure (e.g., \texttt{<Add reagent=``acetic acid">} is an action) and parameters refer to reagents, hardware, and other parameters such as quantities and experiment descriptors (e.g., in \texttt{<HeatChill vessel="beaker" temp=``100C">}, vessel and temp are both parameters). Ambiguous value refers to values that are not inherently incorrect, but where we do not yet have a way to interpret them in our system (e.g., in \texttt{<HeatChill vessel="beaker" temp=``until boiling">}, we want to heat the beaker until the solution inside is boiling, but we do not know how to translate ``until boiling" into a number). 
    \indent (2) \textbf{Quality of the predicted plan (without executing the plan)}. To determine if the predicted task plans actually accomplish every step of their original instructions, we report the number of actions and parameters that do not align between the original and generated plan, as annotated by expert experimental chemists. To compare the quality of the generated plans between \ourmodel and SynthReader, we ask expert experimental chemists to, given two anonymized plans, either pick a preferred plan among them or classify them as equally good. We also ask them to annotate errors in the plans in the following categories:  Missing action, Missing parameter, Wrong action, Wrong parameter, Ambiguous value, Other error. Here, actions refer to high-level steps in the procedure (e.g., \texttt{<Add reagent="acetic acid">} is an action) and parameters refer to reagents, hardware, quantities and experiment descriptors (e.g., in \texttt{<HeatChill vessel="beaker" temp="100C">}, vessel and temp are both parameters). 
    %Ambiguous value refers to values that are not inherently incorrect, but where we do not yet have a way to interpret them in our system (e.g., in \texttt{<HeatChill vessel="beaker" temp=``until boiling">}, we want to heat the beaker until the solution inside is boiling, but we cannot translate ``until boiling" into a number). 
    The annotations were performed using the LightTag Text Annotation Tool \cite{perry-2021-lighttag}.

    \paragraph{Chem-RnD dataset}
    The results for the Chem-RnD dataset with respect to expert preference are reported in the last column of Table ~\ref{tab:comp_xdl}. We found that out of 108 experiments, experts preferred the XDL plan generated from \ourmodel 75 times and the one from SynthReader 13 times (the remaining 20 were considered to be of similar quality).

    The distributions of the annotated errors are shown in Figure~\ref{fig:error_breakdown_2}. We find that for 4 out of 6 error categories, our model does at least as well as or better than the baseline method when considering the mean and median of the distributions. We also find that for those categories, our method produces more experiments with 0 errors. 

    One advantage of our method is that it generates less plans with missing actions compared with the baseline. As XDL generation in SynthReader is implemented by rule-based pattern-matching techniques, any actions that do not match those templates would not appear in the final XDL. For example, for the protocol: 
\begin{lstlisting}[escapeinside={(*}{*)}]
To a solution of m-CPBA (200 mg, 0.8 mmol) in dichloromethane (10 mL), cooled to 0 (*$^\circ$*)C, was added dropwise a solution of 5-chloro-10-oxa-3-thia-tricyclo [5.2.1.01, 5] dec-8-ene (150 mg, 0.8 mmol) in dichloromethane (10 mL).
\end{lstlisting}
    
\noindent the plan generated by \ourmodel was
\begin{lstlisting}[language=xml, escapeinside={(*}{*)}]
<Add vessel="V1" reagent="m-CPBA" amount="200 mg"/>
<Add vessel="V1" reagent="dichloromethane" volume="10 mL"/>
<HeatChill vessel="V1" temp="0 (*\color{codepurple}{$^\circ$}*)C" time="3 min"/>
<Add vessel="V2" reagent="5-chloro-10-oxa-3-thia-tricyclo[5,2,1,0*1,5*]dec-8-ene" amount="150 mg"/>
<Add vessel="V2" reagent="dichloromethane" volume="10 mL"/>
<Transfer from_vessel="V2" to_vessel="V1"/>
\end{lstlisting}

\noindent while the plan generated from SynthReader was 
\begin{lstlisting}[language=xml]
<Add vessel="reactor" reagent="5-chloro-10-oxa-3-thia-tricyclo" volume="0" speed="40.0"/>
\end{lstlisting}

    Our model is able to decompose a complicated procedure into simpler actions by making two solutions in separate beakers and combining them with a Transfer procedure. It also assumes that the solutions don't already exist as mixtures and creates them from scratch. This is another benefit of our model, as it is able to understand implicit actions. For example, given the prompt

\begin{lstlisting}[]
L-Ornithine (31.92 g, 120 mmol) was added to a mixture of KOH (6.72 g, 120 mmol), water (200 ml) and THF (100 ml)
\end{lstlisting}

\noindent SynthReader considers a mixture of three chemicals as a single solution and creates the action:
\begin{lstlisting}[language=xml]
<Add vessel="reactor" reagent="a mixture of KOH (6.72 g, 120 mmol), water (200 ml) and THF (100 ml)" volume="0" speed="40.0"/>
<AddSolid vessel="reactor" reagent="L-Ornithine" mass="31.92 g"/>
\end{lstlisting}

    \noindent  On the other hand, \ourmodel correctly understand the implicit action to mix them beforehand and generates an appropriate XDL:

\begin{lstlisting}[language=xml]
<Add vessel="V1" reagent="L-Ornithine" amount="31.92 g"/>
<Add vessel="V1" reagent="KOH" amount="6.72 g"/>
<Add vessel="V1" reagent="Water" amount="200 ml"/>
<Add vessel="V1" reagent="THF" amount="100 ml"/>
\end{lstlisting}

% An example of the difference in the treatment of implicit actions is shown in Figure~\ref{fig:comparison}.



    \begin{table}[!t]
        \centering
        \caption{\textbf{Verifier Analysis}. We report the average number of times \ourmodel calls the verifier for the experiments in a given dataset, as well as the minimum and maximum number of times. We also report the type of error encountered by the verifier and the number of times it caught that type. }
        \resizebox{\columnwidth}{!}{
        \begin{tabular}{cccp{5cm}}
            \hline
             Dataset &  Average num. & Max/min & Error type caught by verifier [count] \\
             &verifier calls&verifier calls& \\
             \hline \hline
             Chem-RnD  & $2.58 \pm 2.00$ & 10/1 & - missing property in action [306] \newline - property not allowed [174] \newline - wrong tag [120] \newline - action does not exist [21] \newline - item not defined in Hardware or Reagents list [15] \newline - plan cannot be parsed as XML [6] \\
             \hline
             \centering Chem-EDU & $1.15 \pm 0.45$ & 3/1 &  - item not defined in Hardware or Reagents list [47] \newline - property not allowed [26] \newline - wrong tag [40] \newline - missing property in action [3] \\
         \hline
        \end{tabular}
        }
        \label{tab:verifier_errors}
    \end{table}



    However, our model produced plans with a greater number of wrong actions than SynthReader. This is likely because our model is missing domain knowledge on certain actions that need to be included in the prompt or verifier. For example, given the instruction \textit{"Dry solution over magnesium sulfate"}, our model inserts a \texttt{<Dry .../>} into the XDL plan, but the instruction is actually referring to a procedure where ones passes the solution through a short cartridge containing magnesium sulphate, which seems to be encoded in SynthReader. Another wrong action our model performs is reusing vessels. In chemistry, one needs to ensure a vessel is uncontaminated before using it. However, our model generates plans that can use the same vessel in two different steps without washing it in between. %Ideally, it should not re-use vessels.
    Our model also sometimes generates plans with ambiguous values. For example, many experiment descriptions include conditional statements such as ``Heat the solution at the boiling point until it becomes white''. Conditions in XDL need a numerical condition as a parameter. Our model tries to incorporate them by including actions such as \texttt{<HeatChill temp="boiling point" time="until it becomes white"/>}, but they are ambiguous. We can make our model better in the future by incorporating more domain knowledge into our structured language description and improving our verifier with real-world constraints. For example, we can incorporate visual feedback from the environment, include look-up tables for common boiling points, and ensure vessels are not reused before cleaning.
% Second, ambiguous values mainly reflect the limitation of XDL. 

    Despite the XDL plans generated by our method containing errors, we found that the experts placed greater emphasis on missing actions than ambiguous or wrong actions when picking the preferred output, 
    indicating larger severity of this class of error for the tasks and outputs investigated here.

    \begin{table}[!t]
        \centering
        \caption{Number of XDL plans successfully generated for different error message designs in the iterative prompting scheme on a validation set from Chem-RnD.}
        \resizebox{\columnwidth}{!}{%
        \begin{tabular}{p{7.5cm}p{2.5cm}}
            \toprule
              Variations of Iterative Prompt Design using Verifier Error Messages &  Plan's generated success rate (\%) ↑\\
             \hline \hline
             \textit{Naive}:  XDL from previous iteration and string “This XDL was not correct. Please fix the errors.” &  0 \\
             \hline
             \textit{Last Error}: Error List from verifier from 
             previous iteration  & 30\\
             \hline
             \textit{All Errors cumulative}: Accumulated error List from all previous iterations & 50\\
             \hline
             \textit{XDL + Last Error}: XDL and Error List from verifier from previous iteration & 100 \\
             \bottomrule
        \end{tabular}
        }
        \label{tab:ablation}
    \end{table}


    \begin{figure*}[!t]
        \centering
        \includegraphics[width=0.85\linewidth]{iterator_loop.pdf}
     \caption{\textbf{Feedback loop between the Generator and Verifier.} The input text is converted to structured-like language via the generator and is then passed through the verifier. The verifier returns a list of errors (marked with a yellow 1). The feedback is passed back to the generator along with the erroneous task plan, generating a new task plan. Now that previous errors were fixed and the tags could be processed, new errors were found (including a constraint error that the plan uses a vessel not in the environment). These errors  are denoted with a blue 2. This feedback loop is repeated until no more errors are caught, which in this case required 3 iterations.}
        \label{fig:iter_loop}
    \end{figure*}

    \paragraph{Chem-EDU dataset}
    We annotated the errors in the Chem-EDU datasets using the same annotation labels as for the Chem-RnD dataset. The breakdown of the errors is in the left plot of Figure~\ref{fig:error_breakdown_2}. Note that we did not perform a comparison with SynthReader as no plans were generated from it. 
    We find that the error breakdown is similar to that from Chem-RnD, where we see amibiguous values in experiments that have conditionals instead of precise values. We also encounter a few wrong parameter errors, where the model does not include units for measurements. This can be fixed in future work by improving the verifier to check for these constraints. 

    \indent (3)  \textbf{Number of interventions required by the verifier.} To better understand the interactions between the generator and verifier in \ourmodel, we analyzed the number of interactions that occur between the verifier and generator for each dataset to understand the usefulness of the verifier. In Table~\ref{tab:verifier_errors}, we show that each experiment in the Chem-RnD dataset runs through the verifier on average 2.6 times, while the Chem-EDU dataset experiments runs through it 1.15 times on average. The difference between the two datasets likely exists because the Chem-EDU experiments are shorter and less complicated. The top Chem-EDU error encountered by the verifier was that an item in the plan was not define in the Hardware or Reagents list, mainly because we included hardware constraints for this dataset that we needed to match in our plan. In Figure~\ref{fig:iter_loop}, we show a sample loop series between the generator and verifier. 
    
    \indent (4) \textbf{Robotic validation (Chem-EDU only).} To analyze how well our system performs in the real world, we execute a few experiments from the Chem-EDU dataset on our robot. Three experiments from the Chem-EDU dataset were selected to be executed.

    \paragraph{Solution Color Change Based on pH}
    As a basic chemistry experiment, we demonstrated the color change of a solution containing red cabbage juice. This is a popular introductory demonstration in chemistry education, as the anthocyanin pigment in red cabbage can be used as a pH indicator~\cite{10.1021/ed069p66.1}.
    We prepared red cabbage solution by boiling red cabbage leaves in hot water. The colour of the solution is dark purple/red. Red cabbage juice changes its color to bright pink if we add an acid and to blue if we add a base, and so we acquired commercially-available vinegar (acetic acid, an acid) and baking soda (sodium bicarbonate, a base).

    In this experiment, we generated XDL plans using \ourmodel from two language inputs: 
    \vspace{-5pt}
\begin{lstlisting}[escapeinside={(*}{*)}]
[1] Add 40 g of red cabbage solution into a beaker. Add 10 g of acetic acid into the beaker, then stir the solution for 10 seconds.
\end{lstlisting}
\vspace{-10pt}
\begin{lstlisting}[escapeinside={(*}{*)}]
[2] Add 40 g of red cabbage solution into a beaker. Add 10 g of baking soda into the beaker, then stir the solution for 10 seconds.
\end{lstlisting}
\vspace{-5pt}

Figure~\ref{fig:robot_setup} shows the flow of the experiment. Our generated a XDL plan that correctly captured the experiment; the plan was then passed through TAMP to generate a low-level action plan and was then executed by the robot. 
% The robot could execute the robot motion specified in the generated XDL.
% \lorem

\begin{figure*}[!t]
    \begin{minipage}{0.75\linewidth}
    \centering
    \includegraphics[width=\textwidth]{robotexp_bakingsoda.pdf}
    \end{minipage}
    \begin{minipage}{0.24\linewidth}
    \caption{\textbf{Robot execution}: The robot executes the motion plan generated from the XDL for given natural language input. (a) \ourmodel converts the natural language input from the user into XDL. (b) The robot interprets XDL and performs the experiment. Stirring is done by a rotating stir bar inside the beaker.
    %\animesh{this figure currently takes a lot of space. fix and compress whitespace. label inside the robot image. perhaps also provide inset views to showcase what is happening.
    }    \label{fig:robot_setup}
    \end{minipage}
\end{figure*}

\paragraph{Kitchen Chemistry}
We then tested whether our robot could execute a plan generated by our model for a different application of household chemistry: food preparation.
We generated a plan using \ourmodel for the following lemonade beverage, which can be viewed on our website:

\begin{lstlisting}[escapeinside={(*}{*)}]
Add 15 g of lemon juice and sugar mixture to a cup containing 30 g of sparkling water. Stir vigorously for 20 sec.
\end{lstlisting}
\vspace{-5pt}



%metric 1: quality of the predicted plan [witohut executing]
%metric 2: number of interventions required 
%metric 3: robotic validation of the same plan [execution -- only vialbe for Chem-EDU]
%Organize sections by metric instead of dataset
%organize by easier dataset in each section



%\subsection{Ability to Generate Structured Language Plans}

%As a first step, we evaluate the ability of our model to generate structured language plans from natural instructions without throwing a fatal error. We report the number of plans successfully generated for both Chem-RnD and Chem-EDU using \ourmodel and compare it to SynthReader, a baseline XDL-generation model, in Table ~\ref{tab:comp_xdl}. 


%The results of this comparison can be seen in Table ~\ref{tab:comp_xdl}. First, we evaluate how many times each method was able to generate a XDL plan (without returning a fatal error or returning an empty plan). 

%We find that out of 108 Chem-RnD experiments, \ourmodel successfully returned a plan 96\% of the time, while SynthReader returned a plan 87\% of the time. For the Chem-EDU dataset, \ourmodel generated a plan for all instructions. SynthReader was unable to generate any plans for that dataset, likely because the procedures are different from typical chemical procedures (they use simpler action statements). This demonstrates the generalizability of our method, in that we can apply it to different language styles and domains and still obtain coherent  plans. 


% \begin{figure}[!t]
%     \includegraphics[width=0.8\columnwidth]{violinplot_chemedu.png}
%     \centering
%     \caption{\textbf{Violin plot showing distributions of different error categories in XDL plans generated for experiments in the Chem-EDU dataset by \ourmodel}. We show the mean of the distribution with a gold dot (and the number beside in gold) and the median with a grey dot.}
% \label{fig:household_chem_error_breakdown}
% \end{figure}


%\subsection{Quality of Generated Plans}
%\paragraph{\textbf{Chem-RnD dataset}}
%We recruited expert experimental chemists to compare the quality of the XDL plans generated from \ourmodel and SynthReader for the Chem-RnD dataset. They were presented with an experiment from \cite{https://doi.org/10.5281/zenodo.3955107} written in natural language and two anonymized XDL plans in random order,  one generated from SynthReader \cite{10.1126/science.abc2986} and the other from \ourmodel. First, they were asked which XDL plan was overall able to better represent the experiment. Specifically, we asked them to select a label from [``XDL \#1 is better", ``XDL \#2 is better", ``Both XDLs are of similar quality"]. The result of this are in the last column of Table ~\ref{tab:comp_xdl}. We found that out of 108 experiments, experts preferred the XDL plan generated from \ourmodel 75 times and the one from SynthReader 13 times (the remaining 20 were considered to be of similar quality).
%To evaluate the quality of the output of our model, analysis and comparisons of the output was performed by expert chemists. The results of the anonymized comparison of the outputs of \ourmodel and SynthReader is presented in in the last column of Table ~\ref{tab:comp_xdl}. We found that out of 108 experiments, experts preferred the XDL plan generated from \ourmodel 75 times and the one from SynthReader 13 times (the remaining 20 were considered to be of similar quality). From the similarly anonymized fine-grained analysis of the different error types produced by the models, the distributions of how often each type of error occurred for each method was then generated. The results are shown in the right plot of Figure~\ref{fig:error_breakdown_2}. We find that for 4 out of 6 error categories, our model does at least as well as or better than the baseline method when considering the mean and median of the distributions. We also find that for those categories, our method produces more experiments with 0 of those error types.



%Then, the experts were asked to do a fine-grained analysis of each method to analyze how the actions in the plan are aligned to the goals of the experiment. Using the LightTag Text Annotation Tool \cite{perry-2021-lighttag}, they marked specific error types in the XDL plans
%%: Missing action, Missing parameter, Wrong action, Wrong parameter, Ambiguous value, Other error.
%We show distributions of the annotated errors are in Figure~\ref{fig:error_breakdown}. We find that for 4 out of 6 error categories, our model does at least as well as or better than the baseline method when considering the mean and median of the distributions. We also find that for those categories, our method produces more experiments with 0 of those error types. 

%One advantage of our method is that it generates less plans with missing actions compared with the baseline. As XDL generation in SynthReader is implemented by rule-based pattern-matching techniques, any actions that do not match those templates would not appear in the final XDL. For example, for the protocol: 
%\begin{lstlisting}[escapeinside={(*}{*)}]
%To a solution of m-CPBA (200 mg, 0.8 mmol) in dichloromethane (10 mL), cooled to 0 (*$^\circ$*)C, was added dropwise a solution of 5-chloro-10-oxa-3-thia-tricyclo [5.2.1.01, 5] dec-8-ene (150 mg, 0.8 mmol) in dichloromethane (10 mL).
%\end{lstlisting}

%\noindent The plan generated by \ourmodel was
%\begin{lstlisting}[language=xml, escapeinside={(*}{*)}, basicstyle=\tiny]
%<Add vessel="V1" reagent="m-CPBA" amount="200 mg"/>
%<Add vessel="V1" reagent="dichloromethane" volume="10 mL"/>
%<HeatChill vessel="V1" temp="0 (*\color{codepurple}{$^\circ$}*)C" time="3 min"/>
%<Add vessel="V2" reagent="5-chloro-10-oxa-3-thia-tricyclo[5,2,1,0*1,5*]dec-8-ene" amount="150 mg"/>
%<Add vessel="V2" reagent="dichloromethane" volume="10 mL"/>
%<Transfer from_vessel="V2" to_vessel="V1"/>
%\end{lstlisting}

%\noindent While the plan generated from SynthReader was 
%\begin{lstlisting}[language=xml]
%<Add vessel="reactor" reagent="5-chloro-10-oxa-3-thia-tricyclo" volume="0" speed="40.0"/>
%\end{lstlisting}


%\begin{table}[!t]
%    \centering
%    \caption{\textbf{Verifier Analysis}. We report the average number of times \ourmodel calls the verifier for the experiments in a given dataset, as well as the minimum and maximum times. We also report the type of error encountered by the verifier and the number of times it caught that type. }
%    \resizebox{\columnwidth}{!}{
%    \begin{tabular}{cccp{5cm}}
%        \hline
%         Dataset &  Average num. & Max/min & Error type caught by verifier [count] \\
%         &verifier calls&verifier calls& \\
%         \hline \hline
%         Chem-RnD  & $2.58 \pm 2.00$ & 10/1 & - missing property in action [306] \newline - property not allowed [174] \newline - wrong tag [120] \newline - action does not exist [21] \newline - item not defined in Hardware or Reagents list [15] \newline - plan cannot be parsed as XML [6] \\
%         \hline
%         \centering Chem-EDU & $1.15 \pm 0.45$ & 3/1 &  - item not defined in Hardware or Reagents list [47] \newline - property not allowed [26] \newline - wrong tag [40] \newline - missing property in action [3] \\
%         \hline
%    \end{tabular}
%    }
%    \label{tab:verifier_errors}
%\end{table}


%Our model is able to decompose a complicated procedure into simpler actions by making two solutions in separate beakers combining them with a Transfer procedure. It also assumes that the solutions don't already exist as mixtures and creates them from scratch. This is another benefit of our model, as it is able to understand implicit actions. For example, given the prompt

%\begin{lstlisting}[]
%L-Ornithine (31.92 g, 120 mmol) was added to a mixture of KOH (6.72 g, 120 mmol), water (200 ml) and THF (100 ml)
%\end{lstlisting}

%\noindent SynthReader considers a mixture of three chemicals as a single solution and creates the action:
%\begin{lstlisting}[language=xml]
%<Add vessel="reactor" reagent="a mixture of KOH (6.72 g, 120 mmol), water (200 ml) and THF (100 ml)" volume="0" speed="40.0"/>
%<AddSolid vessel="reactor" reagent="L-Ornithine" mass="31.92 g"/>
%\end{lstlisting}

%\noindent  On the other hand, \ourmodel correctly understand the implicit action to mix them beforehand and generates an appropriate XDL:
%\begin{lstlisting}[language=xml]
%<Add vessel="V1" reagent="L-Ornithine" amount="31.92 g"/>
%<Add vessel="V1" reagent="KOH" amount="6.72 g"/>
%<Add vessel="V1" reagent="Water" amount="200 ml"/>
%<Add vessel="V1" reagent="THF" amount="100 ml"/>
%\end{lstlisting}

% An example of the difference in the treatment of implicit actions is shown in Figure~\ref{fig:comparison}.


%However, our model produced plans with a greater number of wrong actions than SynthReader. This is likely because our model is missing domain knowledge on certain actions that need to be included in the prompt or verifier. For example, given the instruction \textit{"Dry solution over magnesium sulfate"}, our model inserts a \texttt{<Dry .../>} into the XDL plan, but the instruction is actually referring to a procedure where ones passes the solution through a short cartridge containing magnesium sulphate, which seems to be encoded in SynthReader. Another wrong action our model performs is reusing vessels. In chemistry, one needs to ensure a vessel is uncontaminated before using it. However, our model generates plans that can use the same vessel in two different steps without washing it in between. %Ideally, it should not re-use vessels.
%Our model also sometimes generates plans with ambiguous values. For example, many experiment descriptions include conditional statements such as ``Heat the solution at the boiling point until it becomes white''. Conditions in XDL need a numerical condition as a parameter. Our model tries to incorporate them by including actions such as \texttt{<HeatChill temp="boiling point" time="until it becomes white"/>}, but they are ambiguous. We can make our model better in the future by incorporating this domain knowledge into our structured language description and improving our verifier with real-world constraints. For example, we can incorporate visual feedback from the environment, include look-up tables for common boiling points, and ensure vessels are not reused before cleaning.
% Second, ambiguous values mainly reflect the limitation of XDL. 


%Despite the XDL plans generated by our method containing errors, we found that the experts placed greater emphasis on missing actions than ambiguous or wrong actions when picking the preferred output, 
%indicating larger severity of this class of error for the tasks and outputs investigated here.


%  The difference in the 2 remaining categories is less significant and may be attributed to the ambiguity in XDL. Read the main text for detailed analysis.\animesh{include error bars, and drop the text. include descriptive captions, even if it means some text replication. if crunched for space, delete text from main body but keep it in captions. Each figure should be self sustainable with a take home message.}

%\begin{figure}[ht]
%\centering
%\begin{lstlisting}[caption=Natural language input]
%L-Ornithine (31.92 g, 120 mmol) was added to a mixture of KOH (6.72 g, 120 mmol), water (200 ml) and THF (100 ml)
%\end{lstlisting}
%
%\begin{lstlisting}[language=xml, caption=XDL generated by SynthReader]
%<Add vessel="reactor" reagent="a mixture of KOH (6.72 g, 120 mmol), water (200 ml) and THF (100 ml)" volume="0" speed="40.0"/>
%<AddSolid vessel="reactor" reagent="L-Ornithine" mass="31.92 g"/>
%\end{lstlisting}
%
%\begin{lstlisting}[language=xml, caption=XDL generated by \textsc{CLAIRify}]
%<Add vessel="V1" reagent="L-Ornithine" 
%     amount="31.92 g"/>
%<Add vessel="V1" reagent="KOH" amount="6.72 g"/>
%<Add vessel="V1" reagent="Water" amount="200 ml"/>
%<Add vessel="V1" reagent="THF" amount="100 ml"/>
%\end{lstlisting}
%
%\caption{\textbf{Comparison of the methods} Since SynthReader implemented by pattern-matching does not understand implicit actions, the generated XDL considers a mixture of three chemicals as a single solution. On the other hand, \ourmodel correctly understand the implicit action to mix them beforehand and generates an appropriate XDL.}
%\label{fig:comparison}
%\end{figure}


%\begin{table}[!t]
%    \centering
%    \caption{Number of XDL plans successfully generated for different error messages in the iterative prompting scheme on a validation set from Chem-RnD.}
%    \resizebox{\columnwidth}{!}{%
%    \begin{tabular}{p{7.5cm}p{2.5cm}}
%        \toprule
%          Variations of Iterative Prompt Design using Verifier Error Messages &  Plan's generated success rate (\%) ↑\\
%         \hline \hline
%         \textit{Naive}:  XDL from previous iteration and string “This XDL was not correct. Please fix the errors.” &  0 \\
%         \hline
%         \textit{Last Error}: Error List from verifier from 
%         previous iteration  & 30\\
%         \hline
%         \textit{All Errors cumulative}: Accumulated error List from all previous iterations & 50\\
%         \hline
%         \textit{XDL + Last Error}: XDL and Error List from verifier from previous iteration & 100 \\
%         \bottomrule
%    \end{tabular}
%    }
%    \label{tab:ablation}
%\end{table}


%\begin{figure*}[!t]
%    \centering
%    \includegraphics[width=0.9\linewidth]{iterator_loop.pdf}
%  \caption{\textbf{Feedback loop between the Generator and Verifier.} The input text is converted to structured-like language via the generator, which is then passed through the verifier. The verifier returns a list of errors ( marked with a yellow 1). The feedback is passed back to the generator with the erroneous task plan, generating a new task plan. Now that previous errors were fixed and the tags could be processed, new errors were found (including a constraint error that the plan uses a vessel not in the environment). These errors  are denoted with a blue 2. This feedback loop is repeated until no more errors are caught, which in this case, required 3 iterations.}
%    \label{fig:iter_loop}
%\end{figure*}



%\paragraph{\textbf{Chem-EDU dataset}}
%We annotated the errors in the Chem-EDU datasets using the same annotation labels as for the Chem-RnD dataset. The breakdown of the errors is in the left plot of Figure~\ref{fig:error_breakdown_2}. Note that we did not perform a comparison with SynthReader as no plans were generated from it. 
%We find that the error breakdown is similar to that from Chem-RnD, where we see amibiguous values in experiments that have conditionals instead of precise values. We also encounter a few wrong parameter errors, where the model does not include units for measurements. This can be fixed in future work by improving the verifier to check for these constraints. 
%The most common annotation was ``Redudnant action", where the XDL plan had an action that was unnecessary. This would not render the experiment incorrect (the end goal of the experiment would still be accomplished), but it would make the robot carry out the experiment in a manner that is not the most efficient, such as transferring liquid into the same container.  
%This could perhaps improved with better documention in the description of XDL that we pass to the generator (i.e. identifying redundant actions). We also find that the plans have errors when the experiments use conditionals (for example, ``Heat beaker until water reaches 70 $^{\circ}$C", which produces the plan \texttt{\textless{}HeatChill vessel="beaker" temp="70" time="60" /\textgreater{}}) since the XDL language has no functionality to do something until a condition is obtained. Again, this could by incorporating some sort of sensor feedback or including 

%\subsection{Analysis of Verifier Interventions in \ourmodel}

%We analyzed the number of interactions that occur between the verifier and generator for each dataset to understand the usefulness of the verifier. In Table~\ref{tab:verifier_errors}, we show that the each experiment in the Chem-RnD dataset runs through the verifier on average 2.6 times, while the Chem-EDU dataset experiments runs through it 1.15 times on average. The difference between the two datasets likely exists because the Chem-EDU experiments are shorter and less complicated. The top Chem-EDU error encountered by the verifier was that an item in the plan was not define in the Hardware or Reagents list, mainly because we included hardware constraints for this dataset that we needed to match in our plan. In Figure~\ref{fig:iter_loop}, we show a sample loop series between the generator and verifier and the feedback that is generated. 



%\subsection{Robot Execution}
%Three experiments from the Chem-EDU dataset were selected to be executed in a real-world robot system.

%\paragraph{\textbf{Solution Color Change Based on pH}}
%As a basic chemistry experiment, we demonstrated the color change of a solution containing red cabbage juice. This is a popular introductory demonstration in chemistry education, as the anthocyanin pigment in red cabbage can be used as a pH indicator~\cite{10.1021/ed069p66.1}.
%We prepared red cabbage solution by boiling red cabbage leaves in hot water. The colour of the solution is dark purple/red. Red cabbage juice changes its color to bright pink if we add an acid and to blue if we add a base, and so we acquired commercially-available vinegar (acetic acid, an acid) and baking soda (sodium bicarbonate, a base).

%were prepared as a familiar acid and base.
%The color of the soup changes to red if vinegar is poured as it is acid, and the color changes to blue if baking soda is poured as it is a base.

%In this experiment, we generated XDL plans using \ourmodel from two language inputs: 
% \textit{"Add 30 g of red cabbage soup into a beaker. Add 10 g of vinegar into the beaker, then stir the solution for 10 seconds."} and \textit{"Add 30 g of red cabbage soup into a beaker. Add 10 g of baking soda into the beaker, then stir the solution for 10 seconds."}
%\vspace{-5pt}
%\begin{lstlisting}[escapeinside={(*}{*)}]
%[1] Add 40 g of red cabbage solution into a beaker. Add 10 g of acetic acid into the beaker, then stir the solution for 10 seconds.
%\end{lstlisting}
%\vspace{-10pt}
%\begin{lstlisting}[escapeinside={(*}{*)}]
%[2] Add 40 g of red cabbage solution into a beaker. Add 10 g of baking soda into the beaker, then stir the solution for 10 seconds.
%\end{lstlisting}
%\vspace{-5pt}

%Figure~\ref{fig:robot_setup} shows the flow of the experiment. Our method was able to generate a XDL plan that correctly captured the experiment; the plan was then passed through TAMP to generate a low-level action plan and was then executed by the robot. 
% The robot could execute the robot motion specified in the generated XDL.
% \lorem

%\begin{figure*}[!t]
%    \begin{minipage}{0.75\linewidth}
%    \centering
%    \includegraphics[width=\textwidth]{robotexp_bakingsoda.pdf}
%    \end{minipage}
%    \begin{minipage}{0.24\linewidth}
%    \caption{\textbf{Robot execution}: The robot executes the motion plan generated from XDL for given natural language input. (a) \ourmodel converts the natural language input from the user into XDL, (b) The robot interprets XDL and conducts the described experiment. Stirring is done by a rotating stir bar inside the beaker.
    %\animesh{this figure currently takes a lot of space. fix and compress whitespace. label inside the robot image. perhaps also provide inset views to showcase what is happening.
%    }    \label{fig:robot_setup}
%    \end{minipage}
%\end{figure*}

%\paragraph{\textbf{Kitchen Chemistry}}
%We then tested whether our robot could execute a plan generated by our model for a different application of household chemistry: food preparation.
%We generated a plan using \ourmodel for the following lemonade beverage, which can be viewed on our website:

%\begin{lstlisting}[escapeinside={(*}{*)}]
%Add 15 g of lemon juice and sugar mixture to a cup containing 30 g of sparkling water. Stir vigorously for 20 sec.
%\end{lstlisting}
%\vspace{-5pt}

%The robot served as a bartender by executing the XDL generated from cocktail recipes.
%The recipe for simplified margarita \textit{``Add 15 g of lime juice and sugar mixture to a cup containing 30 g of tequila. Stir vigorously for 20 seconds.''} was converted into XDL and executed by the robot.



%\begin{figure*}[!t]
%    \begin{minipage}{0.75\linewidth}
%    \includegraphics[width=\textwidth]{robotexp_margarita.pdf}
%    \end{minipage}
%    \begin{minipage}{0.24\linewidth}
%    \caption{\textbf{Robot bartender}: The robot makes a margarita based on the XDL generated from a natural language recipe.\animesh{make images clickable so that link to videos on webpage or youtube. user higher res images}}
%    \label{fig:robot_bartender}
%    \end{minipage}
%\end{figure*}


\subsection{Ablation Studies} 
We assess the impact of various components in our prompt designs and feedback messaging from the verifier. We  performed these tests on a small validation set of 10 chemistry experiments from Chem-RnD (not used in the test set) and report the number of XDL plans successfully generated (i.e., was not in the iteration loop for $x=10$ steps).

\paragraph{\textbf{Prompt Design}}
To evaluate the prior knowledge of the GPT-3 on XDL, we first tried prompting the generator without a XDL description, i.e., with the input: 

\begin{lstlisting}[language=Python]
initial_prompt = """
Convert to XDL:
# <Natural language instruction>"""
\end{lstlisting}

The LLM was unable to generate XDL for any of the inputs from the small validation set that contains 10 chemistry experiments. For most experiments, when asked to generated XDL, the model output a rephrased version of the natural language input. In the best case, it output some notion of structure in the form of S-expressions or XML tags, but the outputs were very far away from correct XDL and were not related to chemistry. We tried the same experiment with \texttt{code-davinci-002}; the outputs generally had more structure but were still nonsensical. 
This result suggests the LLM does not have the knowledge of the target language and including the language description in the prompt is essential to generate an unfamiliar language. 

\paragraph{\textbf{Feedback Design}}
We experimented with prompts in our iterative prompting scheme containing various levels of detail about the errors. The baseline prompt contains a description as well as natural language instruction. We wanted to investigate how much detail is needed in the error message for the generator to be able to fix the errors in the next iteration. For example, is it sufficient to write ``There was an error in the generated XDL", or do we need to include a list of errors from the verifier (such as ``Quantity is not a permissible attribute for the Add tag"), or do we also need to include the erroneous XDL from the previous iteration? 

We find that including the erroneous XDL from the previous iteration and saying why it was wrong resulted in the high number of successfully generated XDL plans. Including a list of errors was better than only writing ``This XDL was not correct. Please fix the rrors", which was not informative enough to fix any errors. Including the erroneous XDL from the previous iteration is also important; we found that including only a list of the errors without the context of the XDL plan resulted in low success rates.


\section{Conclusion and Future Work}
In this paper, we introduce \ourmodel to generate structured language task plans in a DSL by providing an LLM with a description of the language in a zero-shot manner. We also ensure that the task plan is syntactically correct in the DSL by using a verifier and iterative prompting. Finally, we show that our plans can incorporate environmental constraints. We evaluated the performance of \ourmodel on two datasets and find that our method was able to generate better plans than existing baselines. Finally, we translate a select number of these plans to real-world robot demonstrations. 

In the future, we will incorporate feedback from the robot planner and environment into our plan generation process. We will also improve the verifier to encode knowledge of the target domain nuances.
With these technical refinements, we expect our system can become integrated even better in robotic task planning for different domains. %(and also generates a large computation cost).
% \begin{minted}{python} 
% import numpy as np
    % and find ways to reduce the size of the structured language description, as this prevents us from generating plans for longer instructions.
% def incmatrix(genl1,genl2):
%     m = len(genl1)
%     n = len(genl2)
%     # comment here
% \end{minted}

\section{Acknowledgements}
We would like to thank members of the Matter Lab for annotating task plans. We would also like to thank the Acceleration Consortium for their generous support, as well as the Carlsberg Foundation. 
% ========bib files ============
% \clearpage
\renewcommand*{\bibfont}{\small}
\bibliographystyle{IEEEtran}
\bibliography{reference}
\end{document}