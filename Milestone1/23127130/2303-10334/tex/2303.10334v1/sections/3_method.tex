\section{Method}
\label{sec_method}
\begin{figure*}[t]
    \begin{center}
        \includegraphics[width=1.0\linewidth]{src/framework_0309.pdf}
    \end{center}
    \caption{\textbf{Overview of the proposed \frameworkname~framework} for video-music matching. Given a pair of video and music, the video encoder is applied to extract video features and the music encoder is adopted to collect music features(Sec.~\ref{m_e}). There are two parts of the music encoder, and we concatenate their output as music features. In the training step, we calculate the \faceloss{} for the video and music separately from video features and music features by utilizing the same shared head (Sec.~\ref{loss_class}). Hence, calculate the similarity loss between video features and music features(Sec.~\ref{loss_sim}). In the testing step (the right side of the figure), we treat the task as a classification problem on seen music set and calculate the cosine similarity between video and music on unseen music set to match video with appropriate music (Sec.~\ref{matching}).}
    \label{fig:framework}
\end{figure*}
In Section~\ref{sec_pipeline}, we introduce the pipeline of generating LPCAM using a collection of class-wise prototypes including class prototypes and context prototypes, without any re-training on the classification model. The step-by-step illustration is shown in Figure~\ref{fig_framework}, demonstrating the steps of generating local prototypes from all images of a class and using these prototypes to extract LPCAM for each single image. In Section~\ref{sec_justfication}, we justify 1) the advantages of using clustered local prototypes in LPCAM; and 2) the effectiveness of LPCAM from the perspective of map normalization. 

\subsection{LPCAM Pipeline}
\label{sec_pipeline}

\noindent
\textbf{Backbone and Features.} 
We use a standard ResNet-50~\cite{resnet} as the network backbone (i.e., feature encoder) of the multi-label classification model to extract features, following related works~\cite{irn,conta,advcam,rib,recam,amn}. Given an input image $\bm{x}$, and its multi-hot class label $\bm{y}\in\{0,1\}^{N}$,  we denote the output of the trained feature encoder as $f(\bm{x}) \in \mathbb{R}^{W \times H \times C}$. $C$ denotes the number of channels, $H$ and $W$ are the height and width, respectively, and $N$ is the total number of foreground classes in the dataset.

\noindent
\textbf{Extracting CAM.}
Before clustering local prototypes of class as well as context, we need the rough location information of foreground and background. We use the conventional CAM to achieve this. We extract it for each individual class $n$ given the feature $f(\bm{x})$ and the corresponding classifier weights $\mathbf{w}_n$ in the FC layer, as follows,
\begin{equation} \label{eq:cam}
    \operatorname{CAM}_n(\bm{x})=
    \frac{\operatorname{ReLU}\left(\bm{A}_n\right)}{\max \left(\operatorname{ReLU}\left(\bm{A}_n\right)\right)},
    \quad
    \bm{A}_n=\mathbf{w}_{n}^{\top}f(\bm{x}).
\end{equation}

\noindent
\textbf{Clustering.}
We perform clustering for every individual class. Here we discuss the details for class $n$. Given an image sample $\bm{x}$ of class $n$, we divide the feature block $f(\bm{x})$ spatially into two sets, $\mathcal{F}$ and $\mathcal{B}$, based on CAM:
\begin{equation}\label{eq:cam_fg_bg}
    f(\bm{x})^{i,j} \in 
        \begin{cases}
            \mathcal{F}, & \text { if } \operatorname{CAM}_n^{i,j}(\bm{x}) \geq \tau \\ 
            \mathcal{B}, & \text { otherwise }
        \end{cases}
\end{equation}
where $f(\bm{x})^{i,j} \in \mathbb{R}^{C}$ denotes the local feature at spatial location $(i,j)$. $\tau$ is the threshold to generate a 0-1 mask from $\operatorname{CAM}_n(\bm{x})$. $\mathcal{F}$ denotes the set of foreground local features, and $\mathcal{B}$ for the set of background (context) local features.

Similarly, we can collect $\mathcal{F}$ to contain the foreground features of all samples\footnote{We use a random subset of samples for each class in the real implementation, to reduce the computation costs of clustering.} in class $n$, and $\mathcal{B}$ for all background features, where the subscript $n$ is omitted for brevity. After that, we perform K-Means clustering, respectively, for $\mathcal{F}$ and $\mathcal{B}$, to obtain $K$ class centers in each of them, where $K$ is a hyperparameter. We denote the foreground cluster centers as $\mathbf{F}=\{\mathbf{F}_{1}, \cdots, \mathbf{F}_{K}\}$ and the background cluster centers as $\mathbf{B}=\{\mathbf{B}_{1}, \cdots, \mathbf{B}_{K}\}$.

\noindent
\textbf{Selecting Prototypes.}
The masks of conventional CAM are not accurate or complete, e.g., background features could be grouped into $\mathcal{F}$. To solve this issue, we need an ``evaluator'' to check the eligibility of cluster centers to be used as prototypes. The intuitive way is to use the classifier $\mathbf{w}_n$ as an auto ``evaluator'': using it to compute the prediction score of each cluster center $\mathbf{F}_i$ in $\mathbf{F}$ by:
\begin{equation}
    \bm{z}_i = \frac{\exp(\mathbf{F}_i \cdot \mathbf{w}_n)}{\sum_{j} \exp (\mathbf{F}_i \cdot \mathbf{w}_j)}.
\end{equation}
Then, we select those centers with high confidence: $\bm{z}_i>\mu_f$, where $\mu_f$ is a threshold---usually a very high value like $0.9$. 
We denote selected ones as $\mathbf{F'}=\{\mathbf{F'}_{1}, \cdots, \mathbf{F'}_{K'_1}\}$. Intuitively, confident predictions indicate strong local features, i.e., prototypes, of the class.


Before using these local prototypes to generate LPCAM, we highlight that in our implementation of LPCAM, we not only \emph{preserve} the non-discriminative features but also \emph{suppress} strong context features (i.e., false positive),
as the extraction and application of context prototypes are convenient---similar to class prototypes but in a reversed manner. We elaborate these in the following.
For each $\mathbf{B}_i$ in the context cluster center set $\mathbf{B}$, we apply the same method (as for $\mathbf{F}_i$) to compute a prediction score:
\begin{equation}
    \bm{z}_i = \frac{\exp(\mathbf{B}_i \cdot \mathbf{w}_n)}{\sum_{j} \exp (\mathbf{B}_i \cdot \mathbf{w}_j)}.
\end{equation}
Intuitively, if the model is well-trained on class labels, its prediction on context features should be extremely low.
Therefore, we select the centers with $\bm{z}_i<\mu_b$ (where $\mu_b$ is usually a value like $0.5$), and denote them as $\mathbf{B'}=\{\mathbf{B'}_{1}, \cdots, \mathbf{B'}_{K'_2}\}$. It is worth noting that our method is not sensitive to the values of the hyperparameters $\mu_f$ and $\mu_b$, given reasonable ranges, e.g., $\mu_f$ should have a large value around $0.9$. We show an empirical validation for this in Section~\ref{sec_exper}.


\noindent
\textbf{Generating LPCAM.}
Each of the prototypes represents a local visual pattern in the class: $\mathbf{F'}_{i}$ for class-related pattern (e.g., the ``leg'' of ``sheep'' class) and $\mathbf{B'}_{i}$ for context-related pattern (e.g., the ``grassland'' in ``sheep'' images) where the context often correlates with the class. Here we introduce how to apply these prototypes on the feature map block to generate LPCAM. LPCAM can be taken as a substitute of CAM. In Subsection~\ref{sec_justfication}, we will justify why LPCAM is superior to CAM from two perspectives: Global Average Pooling (GAP) and Normalization.

For each prototype, we slide it over all spatial positions on the feature map block, and compute its similarity to the local feature at each position. We adopt cosine similarity as we used it for K-Means. In the end, we get a cosine similarity map between prototype and feature. After computing all similarity maps (by sliding all local prototypes), we aggregate them as follows,
\begin{equation}
    \begin{aligned}
         &\bm{FG}_n=\frac{1}{K'_1}\sum_{\mathbf{F'_i}\in\mathbf{F'}}sim(f(\bm{x}),\mathbf{F'_i}), \\
        & \bm{BG}_n=\frac{1}{K'_2}\sum_{\mathbf{B'_i}\in\mathbf{B'}}sim(f(\bm{x}),\mathbf{B'_i}),
    \label{eq:fg_bg}
    \end{aligned}
\end{equation}
where $sim()$ denotes cosine similarity. As $sim()$ value is always within the range of $[-1, 1]$, each pixel on the maps of $\bm{FG}_n$ and $\bm{BG}_n$ has a normalized value, i.e., $\bm{FG}_n$ and $\bm{BG}_n$ are normalized. Intuitively, $\bm{FG}_n$ highlights the class regions in the input image correlated to the $n$-th prototype, while $\bm{BG}_n$ highlights the context regions. The former needs to be preserved and the latter (e.g., pixels highly correlated to backgrounds) should to be removed. Therefore, we can formulate LPCAM as follows:
\begin{equation}
    \begin{aligned}
        &\operatorname{LPCAM}_n(\bm{x})=
        \frac{\operatorname{ReLU}\left(\bm{A}_n\right)}{\max \left(\operatorname{ReLU}\left(\bm{A}_n\right)\right)}, \\
        &\bm{A}_n= \bm{FG}_n - \bm{BG}_n,
    \label{eq:LPCAM}
    \end{aligned}
\end{equation}
where the first formula is the application of the maximum-value based normalization (the same as in CAM).


\subsection{Justifications}
\label{sec_justfication}
\begin{figure}[ht]
    \centering
    \includegraphics[width=0.99\linewidth]{figures/figure_justification.pdf}
    \vspace{-2mm}
    \caption{Justifying the advantages of LPCAM over CAM from two perspectives: (a) clustering and (b) normalization. For simplicity, in (a), we consider only three local regions on a ``bird'' image: $\bm{x_1}$: ``head'', $\bm{x_2}$: ``tail'', $\bm{x_3}$: ``sky''. In (b), we assume two class prototypes (``head'' and ``tail''), and one context prototype (``sky'') are selected after local feature clustering.}
    \vspace{-2mm}
    \label{fig_justification}
\end{figure}
We justify the effectiveness of LPCAM from two perspectives: clustering and normalization. We use the ``bird'' example shown in Figure~\ref{fig_justification}. In (a), we consider only three local regions ($\bm{x_1}$, $\bm{x_2}$ and $\bm{x_3}$), for simplicity. Their semantics are respectively: $\bm{x_1}$ as ``head'' (a discriminative object region), $\bm{x_2}$ as ``tail'' (a non-discriminative region), and $\bm{x_3}$ as ``sky'' (a context region). We suppose $f(\bm{x_1})$, $f(\bm{x_2})$, and $f(\bm{x_3})$ are 3-dimensional local features (2048-dimensional features in our real implementation) respectively extracted from the three regions, where the three dimensions represent the attributes of ``head'', ``tail'' and ``sky'', respectively. The discriminativeness is reflected as follows. First, $f(\bm{x_1})$ extracted on the head region $\bm{x_1}$ has a significantly higher value of the first dimension than $f(\bm{x_2})$ and $f(\bm{x_3})$. Second, $f(\bm{x_2})$ has the highest value on the second dimension but this value is lower than the first dimension of $f(\bm{x_1})$, because ``tail'' ($\bm{x_2}$) is less discriminative than ``head'' ($\bm{x_1}$) for recognizing ``bird''. In (b), we assume three local class prototypes (``head'', ``tail'', and ``sky'') are selected.

\noindent
\textbf{Clustering.} 
As shown in Figure~\ref{fig_justification}(a), 
\textbf{in LPCAM}, $\bm{x_1}$ and $\bm{x_2}$ go to different clusters. This is determined by their dominant feature dimensions, i.e., the first dimension in $f(\bm{x_1})$ and the second dimension in $f(\bm{x_2})$. Given all samples of ``bird'', their features clustered into the ``head'' cluster all have high values in the first dimension, and features in the ``tail'' have high values in the second dimension. The centers of these clusters are taken as local prototypes and equally used for generating LPCAM. Sliding each of prototypes over the feature map block (of an input image) can highlight the corresponding local region. The intuition is each prototype works like a spatial-wise filter that amplifies similar regions and suppresses dissimilar regions.

However, \textbf{in CAM}, the heatmap computation uses the classifier weights biased on discriminative dimensions\footnote{We empirically validate this in the supplementary materials.}. It is because the classifier is learned from the global average pooling features, e.g., $f_{GAP}=\frac{1}{3}(f(\bm{x_1})+f(\bm{x_2})+f(\bm{x_3}))=[12,5,4]$ biased to the ``head'' dimension. As a result, only discriminative regions (like ``head'' for the class of ``bird'') are highlighted on the heatmap of CAM.


\noindent
\textbf{Normalization.} 
We justify the effectiveness of \textbf{LPCAM} by presenting the normalization details in Figure~\ref{fig_justification}(b). We denote the two class prototypes (``head'' and ``tail'') as $\mathbf{F'_1}, \mathbf{F'_2}$ and the context prototype (``sky'') as $\mathbf{B'_1}$. Based on Eq.~\ref{eq:fg_bg} and Eq.~\ref{eq:LPCAM}, we have $\bm{A}(\bm{x})=\frac{1}{2}\sum_{i=1}^{2}sim(f(\bm{x}),\mathbf{F'_i}) - sim(f(\bm{x}),\mathbf{B'_1})$, where $\bm{x}$ denotes any local region. For simplicity, we use the first term for explanation: the prototype ``head'' $\mathbf{F'_1}$ has the highest similarity ($1.0$) to region $\bm{x_1}$ and the prototype ``tail'' $\mathbf{F'_2}$ has the highest similarity ($0.9$) to $\bm{x_2}$. $0.9$ and $1.0$ are very close. After the final maximum-value based normalization (as in the first formula of Eq.~\ref{eq:LPCAM}), they become $1.00$ and $0.83$, i.e., only a small gap between discriminative ($\bm{x_1}$) and non-discriminative ($\bm{x_2}$) regions. However, \textbf{CAM} (Eq.~\ref{eq:cam}) uses $\mathbf{w}^{\top}f(\bm{x})$, resulting a much higher activation value of $\bm{x_1}$ than $\bm{x_2}$, as $\mathbf{w}$ is obviously biased to ``head'', i.e., $130$ vs. $35$ (``tail''). After the final maximum-value based normalization, there is no change to this bias: $1.00$ and $0.27$---a large gap. In other words, the non-discriminative feature is closer to background $\bm{x_3}$, making the boundary between foreground and background blurry, and hard to find a threshold to separate them.

One may argue that ``separate $\bm{x_2}$ and $\bm{x_3}$'' can be achieved in either CAM or LPCAM if the threshold is carefully selected in each method. However, it is not realistic to do such ``careful selection'' for every input image. The general way in WSSS is to use a common threshold for all images. Our LPCAM makes it easier to find such a threshold, since its heatmap has a much clearer boundary between foreground and background. We conduct a threshold sensitivity analysis in experiments to validate this.
