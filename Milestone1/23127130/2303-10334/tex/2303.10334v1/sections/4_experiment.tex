\section{Experiments}
\label{sec_exper}

\subsection{Datasets and Implementation Details}
\label{sec_datasets}

\noindent
\textbf{Datasets} are the commonly used WSSS datasets: PASCAL VOC 2012~\cite{voc} and MS~COCO 2014~\cite{mscoco}. PASCAL VOC 2012 contains $20$ foreground object categories and $1$ background category with $1,464$ \texttt{train} images, $1,449$ \texttt{val} images, and $1,456$ \texttt{test} images. Following related works~\cite{irn,recam,advcam,amn,ppc,rca}, we use the enlarged training set with $10,582$ training images provided by SBD~\cite{voc_aug}. MS~COCO 2014 dataset consists of $80$ foreground categories and $1$ background category, with $82,783$ and $40,504$ images in \texttt{train} and \texttt{val} sets, respectively. 

\noindent
\textbf{Evaluation Metrics.}
To evaluate the quality of seed mask and pseudo mask, we first generate them for every image in the \texttt{train} set and then use the ground truth masks to compute mIoU. For semantic segmentation, we train the segmentation model, use it to predict masks for the images in \texttt{val} and \texttt{test} sets, and compute mIoU based on ground truth masks.

\noindent
\textbf{Implementation Details.}
We follow \cite{irn,conta,recam,advcam,amn} to use ResNet-50~\cite{resnet} pre-trained on ImageNet~\cite{imagenet} as the backbone of multi-label classification model. For fair comparison with related works, we also follow \cite{seam,ppc,rca} to use WideResNet-38~\cite{wresnet} as backbone and follow EDAM~\cite{edam} to use saliency maps~\cite{saliency} to refine CAM. 

\noindent
\emph{Extra Hyperparameters.}
For K-Means clustering, we set $K$ as $12$ and $20$ for VOC and MS~COCO, respectively. The threshold $\tau$ in Eq.~\ref{eq:cam_fg_bg} is set to $0.1$ for VOC and $0.25$ for MS~COCO. For the selection of prototypes, $\mu_f$ is set to $0.9$ on both datasets, and $\mu_b$ is $0.9$ and $0.5$ on VOC and MS~COCO, respectively. We conduct the sensitivity analysis on these four hyperparameters to show that LPCAM is not sensitive to any of them.

\noindent
\emph{Common Hyperparameter (in CAM methods).}
The hard threshold used to generate 0-1 seed mask is $0.3$ for LPCAM on both datasets. Please note that we follow previous works~\cite{conta,advcam,rib,recam,amn,ppc} to select this threshold by using the ground truth masks in the training set as ``validation''.

\noindent
\emph{Time Costs.}
In K-Means clustering, we use all \texttt{train} set on VOC and sample $100$ images per class on MS~COCO (to control time costs). If taking the time cost of training a multi-label classification model as unit $1$, our extra time cost (for clustering) is about $0.9$ and $1.1$ on VOC and MS~COCO, respectively. 

\noindent
\emph{For Semantic Segmentation.}
When using DeepLabV2~\cite{deeplabv2} for semantic segmentation, we follow the common settings \cite{irn,recam,advcam,rib,amn,ppc} as follows.
The backbone of DeepLabV2 model is ResNet-101~\cite{resnet} and is pre-trained on ImageNet\cite{imagenet}. We crop each training image to the size of $321\times 321$ and use horizontal flipping and random crop for data augmentation. We train the model for $20k$ and $100k$ iterations on VOC and MS COCO, respectively, with the respective batch size of $5$ and $10$. The weight decay is set to $5$e-$4$ on both datasets and the learning rate is $2.5$e-$4$ and $2$e-$4$ on VOC and MS~COCO, respectively. 

When using UperNet, we follow ReCAM~\cite{recam}. We resize the input images to $2,048\times 512$ with a ratio range from $0.5$ to $2.0$, and then crop them to $512\times 512$ randomly. Data augmentation includes horizontal flipping and color jitter. We train the models for $40k$ and $80k$ iterations on VOC and MS~COCO datasets, respectively, with a batch size of 16. We deploy AdamW~\cite{adamw} solver with an initial learning rate $6e^{-5}$ and weight decay as $0.01$. The learning rate is decayed by a power of $1.0$ according to polynomial decay schedule.

\subsection{Results and Analyses}

\noindent
\textbf{Ablation Study.}
We conduct an ablation study on the VOC dataset to evaluate the two terms of LPCAM in Eq.~\ref{eq:LPCAM}: foreground term $\bm{FG}_n$ and background term $\bm{BG}_n$ that accord to class and context prototypes, respectively. In Table~\ref{table_ablation}, we show the mIoU results (of seed masks), false positive (FP), false negative (FN), precision, and recall. We can see that our methods of using class prototypes (LPCAM-F and LPCAM) greatly improve the recalls---$11.4\%$ and $12.0\%$ higher than CAM, reducing the rates of FN a lot. This validates the ability of our methods to capture non-discriminative regions of the image. We also notice that LPCAM-F increases the rate of FP over CAM. The reason is that confusing context features (e.g., ``railroad'' for ``train'') may be wrongly taken as class features. Fortunately, when we explicitly resolve this issue by applying the negative context term $-\bm{BG}_n$ in LPCAM, this rate can be reduced (by $3.3\%$ for VOC), and the overall performance (mIoU) can be improved (by $2.8\%$ for VOC). We are thus confident to take LPCAM as a generic substitute of CAM in WSSS methods (see empirical validations below). 

\begin{figure}
       \centering
        \setlength{\tabcolsep}{1pt}
        {\scriptsize
        \begin{tabular}{c c c c c c c }
            { Original } &
            \multicolumn{2}{c}{  } &
            \multicolumn{4}{c}{$\longleftarrow$ Object level variations $\longrightarrow$} \\
            \includegraphics[width=0.185\linewidth]{images/ablation/chair.jpg} &
            \multicolumn{2}{c}{  } &
            \includegraphics[width=0.185\linewidth]{images/ablation/1_only_prompt_mixing/bench.jpg} &
            \includegraphics[width=0.185\linewidth]{images/ablation/1_only_prompt_mixing/stool.jpg} &
            \includegraphics[width=0.185\linewidth]{images/ablation/1_only_prompt_mixing/armchair.jpg} &
            \includegraphics[width=0.185\linewidth]{images/ablation/1_only_prompt_mixing/saddle.jpg} \\
            \multicolumn{3}{c}{  } &
            \multicolumn{4}{c}{ Only Prompt Mixing } \\
            \multicolumn{3}{c}{ } &
            \includegraphics[width=0.185\linewidth]{images/ablation/2_with_self_attn_injection/bench.jpg} &
            \includegraphics[width=0.185\linewidth]{images/ablation/2_with_self_attn_injection/stool.jpg} &
            \includegraphics[width=0.185\linewidth]{images/ablation/2_with_self_attn_injection/armchair.jpg} &
            \includegraphics[width=0.185\linewidth]{images/ablation/2_with_self_attn_injection/saddle.jpg} \\
            \multicolumn{3}{c}{  } &
            \multicolumn{4}{c}{ + Attention-Based Shape Localization } \\
            \multicolumn{3}{c}{ } &
            \includegraphics[width=0.185\linewidth]{images/ablation/3_background_blending/bench.jpg} &
            \includegraphics[width=0.185\linewidth]{images/ablation/3_background_blending/stool.jpg} &
            \includegraphics[width=0.185\linewidth]{images/ablation/3_background_blending/armchair.jpg} &
            \includegraphics[width=0.185\linewidth]{images/ablation/3_background_blending/saddle.jpg} \\
            \multicolumn{3}{c}{  } &
            \multicolumn{4}{c}{ + Controllable Background Preservation } \\
        \end{tabular}
        }
    \vspace{1mm}
    \captionof{figure}{
    Ablating our full object variations pipeline. Original image was crated using the prompt ``A \emph{chair} with a dog on it''. 
    }
    \vspace{-10pt}
    \label{fig:ablation}
\end{figure}

\setlength{\tabcolsep}{1.9mm}{
\renewcommand\arraystretch{1}
\begin{table}[ht]
  \centering
  \scalebox{0.95}{
  \begin{tabular}{llcccc}
    \toprule
    &\multirow{2}*{Methods}& \multicolumn{2}{c}{Seed Mask} & \multicolumn{2}{c}{Pseudo Mask}  \\
    \cmidrule(r){3-4}\cmidrule(r){5-6}
    && \texttt{CAM} & \texttt{LPCAM} & \texttt{CAM} & \texttt{LPCAM}  \\
    \midrule
    \multirow{4}*{\rotatebox{90}{\small{VOC}}} &IRN~\cite{irn}  & 48.8  & 54.9\scriptsize{+6.1}   & 66.5  & 71.2\scriptsize{+4.7}    \\
    ~&EDAM~\cite{edam}                   & 52.8  & 54.9\scriptsize{+2.1}   & 68.1 & 69.6\scriptsize{+1.5}    \\
    ~&MCTformer~\cite{mctformer}        & 61.7  & 63.5\scriptsize{+1.8} & 69.1  & 70.8\scriptsize{+1.7} \\
    ~&AMN~\cite{amn}                 & 62.1  & 65.3\scriptsize{+3.2} &72.2& 72.7\scriptsize{+0.5}   \\
    \midrule
    \multirow{2}*{\rotatebox{90}{\small{COCO}}} & IRN~\cite{irn} & 33.1  & 35.8\scriptsize{+2.7}     & 42.5  & 46.8\scriptsize{+4.3}    \\
    ~&AMN~\cite{amn}                        & 40.3  & 42.5\scriptsize{+2.2}    & 46.7  & 47.7\scriptsize{+1.0}    \\
    \bottomrule
  \end{tabular}}
  \vspace{-0.2cm}
  \caption{Taking LPCAM as a substitute of CAM in state-of-the-art WSSS methods. Except MCTformer~\cite{mctformer} using DeiT-S~\cite{deit}, other methods all use ResNet-50 as feature extractor.}
  \label{table_plugin}
  \vspace{-0.4cm}
\end{table}
}

\begin{figure*}[ht]
    \centering
    \includegraphics[width=0.99\linewidth]{figures/figure_vis.pdf}
    \caption{Qualitative results on MS~COCO. In each example pair, the left is heatmap and the right is seed mask. \emph{Please refer to the supplementary materials for the qualitative results on VOC.}}
    \vspace{-2mm}
    \label{fig_vis}
\end{figure*}
\begin{table}[tb]
\small
\begin{center}
\caption{Comparison of the action segmentation performance on PKUMMD II xview dataset with linear evaluation pretrained on NTU 60 xview dataset.}
\label{tab:seg_pkuII}
\begin{tabular}{l|c|c|c|c|c}
\toprule
\multirow{2.5}*{Models}&\multirow{2.5}*{Stream}&\multicolumn{4}{c}{PKUMMD II xview}\\
\cmidrule(lr){3-6}
&&ACC & MACC & FWIoU & mIoU\\
\midrule
AimCLR~\cite{guo2021contrastive} & joint & 39.77 & 28.68 & 26.79 & 15.67\\
\textbf{ActCLR} & joint & \textbf{51.29}  & \textbf{31.97} & \textbf{35.24} & \textbf{21.38}\\
\midrule
AimCLR~\cite{guo2021contrastive}& motion& 42.32 & 26.65 & 29.92 & 15.92 \\
\textbf{ActCLR} & motion& \textbf{56.69} & \textbf{39.45} & \textbf{41.34} & \textbf{27.73}\\
\midrule
AimCLR~\cite{guo2021contrastive}& bone& 54.22 & 39.52 & 39.41 & 27.36\\
\textbf{ActCLR} &bone& \textbf{59.09} & \textbf{41.14} & \textbf{41.54} & \textbf{28.89}\\
\bottomrule
\end{tabular}
\end{center}
\end{table}

\noindent
\textbf{Generality of LPCAM.}
We validate the generality of LPCAM based on multiple WSSS methods, the popular IRN~\cite{irn}, the top-performing AMN~\cite{amn}, the saliency-map-based EDAM~\cite{edam}, and the transformer-arch-based MCTformer~\cite{mctformer}), by simply replacing CAM with LPCAM. Table~\ref{table_plugin} and Table~\ref{table_seg} show the consistent superiority of LPCAM. For example, on the first row of Table~\ref{table_plugin} (plugging LPCAM in IRN), LPCAM outperforms CAM by $6.1\%$ on seed masks and $4.7\%$ on pseudo masks. These margins are almost maintained when using pseudo masks to train semantic segmentation models in Table~\ref{table_seg}. The improvements on the large-scale dataset MS~COCO are also obvious and consistent, e.g., $2.7\%$ and $2.2\%$ for generating seed masks in IRN and AMN, respectively.

\begin{figure}[ht]
    \centering
    \includegraphics[width=0.99\linewidth]{figures/figure_sensitivity.pdf}
    \vspace{-2mm}
    \caption{Sensitivity analysis on VOC, in terms of (a) $\tau$ for dividing foreground and background local features, (b) $\mu_f$ for selecting class prototypes and $\mu_b$ for selecting context prototypes, (c) the number of clusters $K$ in k-Means, and (d) the threshold used to generate 0-1 seed masks from heatmaps (a common hyperparameter in all CAM-based methods). \emph{Please refer to the supplementary materials for the results on MS~COCO.}}
    \vspace{-4mm}
    \label{fig_sensitivity}
\end{figure}
\setlength{\tabcolsep}{1.6mm}{
\renewcommand\arraystretch{1.1}
\begin{table}[ht]
  \centering
  \scalebox{0.9}{
  \begin{tabular}{llcccc}
    \toprule
    &\multirow{2}*{Methods} & \multirow{2}*{Sal.} &   \multicolumn{2}{c}{VOC} & MS~COCO \\
    \cmidrule(r){4-5}\cmidrule(r){6-6}
    &&&\texttt{val}&\texttt{test}&\texttt{val}\\
    \hline
    \multirow{13}*{\rotatebox{90}{ResNet-50}}
    &IRN~\cite{irn}          \tiny{CVPR'19}     &              & 63.5       & 64.8          & 42.0  \\
    &LayerCAM~\cite{layercam}\tiny{TIP'21}      &              & 63.0       & 64.5          & -     \\
    &AdvCAM~\cite{advcam}    \tiny{CVPR'21}     &              & 68.1       & 68.0          & 44.2  \\
    &RIB~\cite{rib}          \tiny{NeurIPS'21}  &              & 68.3       & 68.6          & 44.2  \\
    &ReCAM~\cite{recam}      \tiny{CVPR'22}     &              & 68.5       & 68.4          & 42.9  \\
    % \rowcolor{Gray}
    &\cellcolor{Gray}IRN+\texttt{LPCAM}    &\cellcolor{Gray} & \cellcolor{Gray}68.6    & \cellcolor{Gray}68.7      & \cellcolor{Gray}44.5  \\
    &SIPE~\cite{sipe}        \tiny{CVPR'22}     &              & 68.8       & 69.7          & 40.6  \\
    &OOD~\cite{ood}+Adv      \tiny{CVPR'22}     &              & 69.8       & 69.9          & -     \\
    &AMN~\cite{amn}          \tiny{CVPR'22}     &              & 69.5       & 69.6          & 44.7  \\
    &\cellcolor{Gray}AMN+\texttt{LPCAM}    &\cellcolor{Gray} & \cellcolor{Gray}70.1    &\cellcolor{Gray} 70.4      & \cellcolor{Gray}45.5  \\ 
    &ESOL~\cite{esol}        \tiny{NeurIPS'22}  &              & 69.9$^*$   & 69.3$^*$      & 42.6  \\
    &CLIMS~\cite{clims}      \tiny{CVPR'22}     &              & 70.4$^*$   & 70.0$^*$      & -     \\
    &EDAM~\cite{edam}        \tiny{CVPR'21}     &\checkmark    & 70.9$^*$   & 71.8$^*$      & -     \\
    &\cellcolor{Gray}EDAM+\texttt{LPCAM}  &\cellcolor{Gray}\checkmark & \cellcolor{Gray}71.8$^*$ &\cellcolor{Gray} 72.1$^*$& \cellcolor{Gray}42.1\\
    \hline
    \multirow{9}*{\rotatebox{90}{WideResNet-38}}
    &Spatial-BCE~\cite{sbce} \tiny{ECCV'22}     &              & 70.0       & 71.3      & 35.2  \\
    &BDM~\cite{bdm}          \tiny{ACMMM'22}    &\checkmark    & 71.0       & 71.0      & 36.7  \\ 
    &RCA~\cite{rca}+OOA      \tiny{CVPR'22}     &\checkmark    & 71.1       & 71.6      & 35.7  \\
    &RCA~\cite{rca}+EPS      \tiny{CVPR'22}     &\checkmark    & 72.2       & 72.8      & 36.8  \\
    &HGNN~\cite{hgnn}        \tiny{ACMMM'22}    &\checkmark         & 70.5$^*$   & 71.0$^*$  & 34.5  \\ 
    &EPS~\cite{eps}          \tiny{CVPR'21}     &\checkmark         & 70.9$^*$   & 70.8$^*$  & -     \\
    &RPIM~\cite{rpim}        \tiny{ACMMM'22}    &\checkmark         & 71.4$^*$   & 71.4$^*$  & -     \\ 
    &L2G~\cite{l2g}          \tiny{CVPR'22}     &\checkmark         & 72.1$^*$   & 71.7$^*$  & 44.2  \\
    \hline
    \multirow{2}*{\rotatebox{90}{\small{DeiT-S}}}
    &MCTformer~\cite{mctformer}    \tiny{CVPR'22}     &                 & 71.9$^{\dag}$  & 71.6$^{\dag}$   & 42.0  \\
    &\cellcolor{Gray}MCTformer+\texttt{LPCAM}      &\cellcolor{Gray} & \cellcolor{Gray}72.6$^{\dag}$  & \cellcolor{Gray}72.4$^{\dag}$  &\cellcolor{Gray} 42.8 \\
    \bottomrule
  \end{tabular}}
  \vspace{-2mm}
  \caption{The mIoU results (\%) based on DeepLabV2 on VOC and MS~COCO. The side column shows three backbones of multi-label classification model. ``Sal.'' denotes using saliency maps. * denotes the segmentation model is pre-trained on MS~COCO. $^\dag$ denotes the segmentation model is pre-trained on VOC.
  }
  \vspace{-6mm}
  \label{table_related}
\end{table}
}



\noindent
\textbf{Sensitivity Analysis for Hyperparameters.}
In Figure~\ref{fig_sensitivity}, we show the quality (mIoU) of generated seed masks when plugging LPCAM in AMN on VOC dataset. We perform hyperparameter sensitivity analyses by changing the values of (a) the threshold $\tau$ for dividing foreground and background local features, (b) the threshold $\mu_f$ for selecting class prototypes and the threshold $\mu_b$ for selecting context prototypes, (c) the number of clusters $K$ in K-Means, and (d) the threshold used to generate 0-1 seed mask (a common hyperparameter in all CAM-based methods). Figure~\ref{fig_sensitivity}(a) shows that the optimal value of $\tau$ is $0.1$. Adding a small change does not make any significant effect on the results, e.g., the drop is less than $1\%$ if increasing $\tau$ to $0.2$. Figure~\ref{fig_sensitivity}(b) shows that the optimal values of $\mu_f$ and $\mu_b$ are both $0.9$. The gentle curves show that LPCAM is little sensitive to $\mu_f$ and $\mu_b$. This is because classification models (trained in the first step of WSSS) often produce overconfident (sharp) predictions~\cite{confident}, i.e., output probabilities are often close to $0$ or $1$. It is easy to set thresholds ($\mu_f$ and $\mu_b$) on such sharp values. In Figure~\ref{fig_sensitivity}(c), the best mIoU of seed mask is $65.3\%$ when $K$=$12$, and it drops by only $0.7$ percentage points when $K$ goes up to $20$. In Figure~\ref{fig_sensitivity}(d), LPCAM shows much gentler slopes than CAM around their respective optimal points, indicating its lower sensitivity to the changes of this threshold.

\noindent
\textbf{Qualitative Results.}
Figure~\ref{fig_vis} shows qualitative examples where LPCAM leverages both discriminative and non-discriminative local features to generate heatmaps and 0-1 masks.
In the leftmost two examples, CAM focuses on only discriminative features, e.g., the ``head'' regions of ``teddy bear'' and ``dog'', while our LPCAM has better coverage on the non-discriminative feature, e.g., the ``leg'',  ``body'' and ``tail'' regions. In the ``surfboard'' example, the context prototype term $-\bm{BG}_n$ in Eq.~\ref{eq:LPCAM} helps to remove the context ``waves''. In the rightmost example, we show a failure case: LPCAM succeeds in capturing more object parts of ``train'' but unnecessarily covers more on the context ``railroad''. We think the reason is the strong co-occurrence of  ``train'' and ``railroad'' in the images of ``train''. 

\noindent
\textbf{Comparing to Related Works.}
We compare LPCAM with state-of-the-art methods in WSSS. As shown in Table~\ref{table_related}, on the common setting (ResNet-50 based classification model, and ResNet-101 based DeepLabV2 segmentation model pre-trained on ImageNet), our AMN+LPCAM achieves the state-of-the-art results on VOC ($70.1\%$ mIoU on \texttt{val} and $70.4\%$ on \texttt{test}). On the more challenging MS~COCO dataset, our AMN+LPCAM (ResNet-50 as backbone) outperforms the state-of-the-art result AMN and all related works based on WRN-38.