\subsection{Method Hypothesis Elimination}

The Socratic method is a process of identifying and eliminating false hypotheses through a series of logical and factual questions. It aims to uncover commonly held truths and test their consistency with other beliefs. The method is based on asking questions to explore definitions and identify the general characteristics shared by different instances.

When utilizing the method of hypothesis elimination in a dialogue with a language model, the primary focus is on identifying and addressing inconsistencies. One way to achieve this is by creating a prompt template that is specifically designed to uncover inconsistencies. This can be done by following a process similar to legal cross-examination, first establishing the credibility of the language model's sources, identifying and addressing inconsistencies of multiple sources (witnesses), and highlighting key points that support the conclusions \cite{}.

It is important to note that language models can produce biased or incorrect answers. To ensure the credibility of the answers, it can be helpful to ask the language model for its sources of information and then validate the credibility of those sources. By having multiple answers, their sources, and the credibility of those sources, a more trustworthy answer can be derived.

\subsubsection{Example \#1: What is the third largest lake in the world?}

\begin{table}[ht!]
\caption{What is the third largest lake in the world? (ChatGPT 1/23/2023)}
\label{tab-lake}
%\resizebox{\textwidth}{!}
\begin{tikzpicture}
\begin{small}
\node (table) [inner sep=0pt] {
\begin{tabular}{p{1.2cm}|p{6.6cm}}
\toprule
Role & Dialogue \\
\midrule
Prompt & {What is the third largest lake in the world?} \\
GPT3 & {The third largest lake in the world is Lake Superior. } \\
Prompt & {What are the ten largest lakes in the world?} \\
GPT3 & {They are:
\begin{noindenumerate}
\item Caspian Sea
\item Superior
\item Victoria
\item Huron
\item Michigan
\item ...
\end{noindenumerate} \\
Prompt & {Your two answers contradict. You first answered me Lake Superior is the third largest lake, but your top ten list shows Victoria is the third, and Superior is the second largest lake.} \\
GPT3 & {I apologize for the confusion. Lake Superior is indeed the largest lake in the world by area, but it is not considered a lake by some geographers because it is part of the Great Lakes system, which is connected to the Atlantic Ocean via the St. Lawrence River. In terms of the top ten largest lakes in the world, Lake Victoria is considered the third largest lake by area. My apologies for any confusion caused.} \\
} \\
%\bottomrule
\end{tabular}
};
\draw [rounded corners=.5em] (table.north west) rectangle (table.south east);
\end{small}
\end{tikzpicture}
\end{table}

Table~\ref{tab-lake} presents an example prmpting session with ChatGPT, performed on January 19th, 2023.  The first prompt queries ChatGPT for the third largest lake in the world.  ChatGPT returns ``Lake Superior.''  The second prompt asks for the ten largest lakes, on which the thrid largest is ``Lake Victoria.''  After pointing out the inconsistency to ChatGPT, it returns that ``Lake Victoria'' is the right answer to the first prompt, and it also provides the reason for its mistake.
(We tested on ``the fifth highest mountain in the world'' and went through a similar consistency check, and the answer provided by ChatGPT is accurate.)

Depending on the importance of the answer, a prompt template may want to identify three or more reliable sources and then query LLMs for both answers and references. A credential check on references is similar to a credential check on a witness in a legal proceeding. 

