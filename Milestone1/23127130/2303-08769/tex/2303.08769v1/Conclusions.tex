\section{Concluding Remarks}
\label{sec:conc}

The Socratic method may not always be effective or useful in human interactions, especially when one of the two players is authoritative, emotional, or abusive. However, when the expert partner is a language model, a machine without emotion or authority, the Socratic method can be effectively employed without the issues that may arise in human interactions. In this way, the Socratic method can be utilized to its full potential in guiding, directing, and improving the output of language models through engineering prompts.

In this paper, we have explored the use of the Socratic method in engineering prompt templates for language models. We have discussed the importance of method definition, elenchus, dialectic, maieutics, and counterfactual reasoning techniques in guiding the output of these models. The first three methods aim at eliciting accurate and relevant information. Through the use of methods definition, elenchus, and dialectic, we have demonstrated, with examples, the ability to clarify user queries and assess the quality of language model-generated text, leading to improved precision and accuracy.

We have also shown how the methods of maieutics and counterfactual reasoning can be helpful in stimulating the imagination of writers. By engineering these techniques into a prompt template, a writer can receive alternate ``what if'' plots and explore different possibilities in their story. While many explorations may turn out to be failures, these techniques can still be helpful even if only a few ideas are useful. Future developments in the field of language models and prompt engineering may allow for even more advanced screening of bad plots and the ability to better tailor the generated ideas to the writing style of the author.

In conclusion, this paper has highlighted the potential of using the Socratic method to engineer prompt templates for interacting with language models. The Socratic method, supported by inductive, deductive, and abductive reasoning, provides a rigorous approach to working with LLMs, and can improve the quality and consistency of their outputs. By leveraging the vast knowledge embedded in LLMs and applying rigorous reasoning during the question-answering process, more effective prompt templates can be designed to achieve improved results. Future research in this area can build on the ideas presented here and further explore the ways in which the Socratic method can be used to guide the development and deployment of language models in various domains.

\begin{comment}
The Socratic method may not be effective or useful in a setting where one of the two players is authoritative, emotional, or even abusive. However, when the expert partner is an LLM, a machine without emotion or authority, the method can be effectively employed without the issues that may arise in human interactions. In this way, the Socratic method can be utilized to its full potential in guiding, directing, and improving the output of language models through engineering prompts.

Throughout this paper, we have explored the use of the Socratic method in engineering prompt templates for LLMs. We have discussed the importance of method definition, elenchus, dialectic, maieutics, and counterfactual techniques in guiding the output of these models. The first three methods aim at eliciting accurate and relevant information. Through the use of methods definition, elenchus, and dialectic, we have demonstrated through examples the ability to clarify 
user queries and to
assess the quality of LLM generated text, leading 
to improved precision and accuracy.

To help writers to be productively imaginative, we show that methods maieutics and counterfactual reasoning can be helpful. By engineering these techniques
into a promoting template, a writer may receive alternate ``what if'' plots and explore different possibilities in their story. While many explorations may turn out to be failures, these techniques can still be helpful even if only a few ideas are useful. Future developments in the field of LLM and prompt engineering may allow for even more advanced screening of bad plots and the ability to better tailor the generated ideas to the writing style of the author.

In conclusion, this work has demonstrated the potential of using the Socratic method in engineering prompt templates for interacting with large language models. However, there is still much room for improvement and further research in this area. As language models continue to advance and become more sophisticated, we can expect to see more effective and efficient ways of engineering prompt templates, leading to even more impressive results in the future.
\end{comment}