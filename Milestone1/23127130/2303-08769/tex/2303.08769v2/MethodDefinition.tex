\subsection{Method of Definition}

The method of definition involves clarifying the meaning of key terms and concepts through a process of questioning. Through a cross-examination, it would involve asking questions to define the terms and concepts being discussed, to ensure that all parties have a common understanding of the subject at hand.


\begin{comment}
For example, in a legal cross-examination, the Socratic method could involve a lawyer questioning a witness about the meaning of certain key terms in their testimony, such as ``intent,'' ``negligence,'' or ``reasonable person.'' The lawyer may ask the witness to define these terms and provide examples to ensure that the witness has a clear understanding of the legal concepts relevant to the case. This helps to ensure that the testimony is clear, consistent, and relevant. Another example is when writing an essay on the subject of consciousness, one must first define the term consciousness and the scope of the discussion. The first response from an LLM may contain terminologies, such as awareness, attention, and unconsciousness, that require their definitions to be specified. If necessary,
the LLM can engage a dialogue with the
user to perform query refinement.
In short, the method of definition ensures that the prompter and the LLM are on the same page.
\end{comment}