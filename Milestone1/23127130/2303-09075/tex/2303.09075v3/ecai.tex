\documentclass{ecai}
\usepackage{graphicx}
\usepackage{latexsym}

%%%%%%%%%%%%%%%%%%%%%%%%%%%%%%%%
% MATH
\usepackage{amsmath}
\usepackage{amsfonts}
\usepackage{amssymb}
\usepackage{algorithm}
\usepackage{algorithmic}
%%%%%%%%%%%%%%%%%%%%%%%%%%%%%%%%

%%%%%%%%%%%%%%%%%%%%%%%%%%%%%%%%
% TABLE & FIGURE
\usepackage{multirow}
\usepackage{hyperref}
\usepackage{booktabs}
\usepackage{colortbl}
\usepackage{svg}
\usepackage{subcaption}
\usepackage[normalem]{ulem}
%%%%%%%%%%%%%%%%%%%%%%%%%%%%%%%%

\usepackage{CJKutf8}
\usepackage{color}
% \usepackage[sort]{natbib}
\usepackage[square,sort,comma,numbers]{natbib}
% \newcommand{\tong}[1]{\textcolor{red}{#1}}

\ecaisubmission   % inserts page numbers. Use only for submission of paper.
                  % Do NOT use for camera-ready version of paper.

\begin{document}

\begin{frontmatter}

\title{Enhancing Text Generation with Cooperative Training}


\author[A]{Tong Wu\thanks{The work was done as an intern at IDEA.}$^{**;}$}\orcid{0009-0003-3154-1213}
\author[B]{Hao Wang\thanks{These authors contributed equally to this work.}}
\author[B]{Zhongshen Zeng}
\author[A]{Wei Wang}
\author[A,C]{Hai-Tao Zheng\thanks{Corresponding Authors. Email: zheng.haitao@sz.tsinghua.edu.cn, zhangjiaxing@idea.edu.cn}}
\author[B]{Jiaxing Zhang$^{***;}$}


\address[A]{Shezhen International Graduate School, Tsinghua Universiy}
\address[B]{International Digital Economy Academy}
\address[C]{Pengcheng Laboratory}


% \author[A]{\fnms{First}~\snm{Author}\orcid{....-....-....-....}\thanks{Corresponding Author. Email: somename@university.edu.}}
% \author[B]{\fnms{Second}~\snm{Author}\orcid{....-....-....-....}}
% \author[B]{\fnms{Third}~\snm{Author}\orcid{....-....-....-....}} % use of \orcid{} is optional


% \address[A]{Short Affiliation of First Author}
% \address[B]{Short Affiliation of Second Author and Third Author}

\begin{abstract}
Recently, there has been a surge in the use of generated data to enhance the performance of downstream models, largely due to the advancements in pre-trained language models. However, most prevailing methods trained generative and discriminative models in isolation, which left them unable to adapt to changes in each other. These approaches lead to generative models that are prone to deviating from the true data distribution and providing limited benefits to discriminative models. While some works have proposed jointly training generative and discriminative language models, their methods remain challenging due to the non-differentiable nature of discrete data. To overcome these issues, we introduce a \textit{self-consistent learning} framework in the text field that involves training a discriminator and  generator cooperatively in a closed-loop manner until a scoring consensus is reached. By learning directly from selected samples, our framework are able to mitigate training instabilities such as mode collapse and non-convergence. Extensive experiments on four downstream benchmarks, including AFQMC, CHIP-STS, QQP, and MRPC, demonstrate the efficacy of the proposed framework.
\end{abstract}

\end{frontmatter}

\begin{figure*}[ht]
\begin{center}
\includegraphics[width=0.8\textwidth]{Figure/sim-gen.pdf}
\end{center}
\caption{Overview of the flow chart for the SCL framework.} 
\label{framework}
\end{figure*}

\section{Introduction}
The advance of Pre-trained Language Models (PLMs) like GPT-3 \cite{brown2020language} and LLaMA \cite{DBLP:journals/corr/abs-2302-13971} has substantially improved the performance of deep neural networks across a variety of Natural Language Processing (NLP) tasks. Various language models, based on the Transformer \cite{vaswani2017attention} architecture,  have been proposed, leading to state-of-the-art (SOTA) performance on the fundamental discrimination tasks. These models are first trained with self-supervised training objectives (e.g., predicting masked tokens according to surrounding tokens) on massive unlabeled text data, then fine-tuned on annotated data to adapt to downstream tasks of interest.  However, annotated data is usually limited to a wide range of downstream tasks, which results in overfitting and a lack of generalization to unseen data.

One straightforward way to deal with this data scarcity problem is data augmentation , and incorporating generative models to perform data augmentation has been widely adopted recently . Despite its popularity, the generated text can easily deviate from the real data distribution without exploiting any of the signals passed back from the discrimination task. In previous studies, generative data augmentation and discrimination have been well studied as separate problems, but it is less clear how these two can be leveraged in one framework and how their performances can be improved simultaneously. \looseness=-1

Generative Adversarial Networks (GANs) \cite{https://doi.org/10.48550/arxiv.1406.2661} are good attempts to couple generative and discriminative models in an adversarial manner, where a two-player minimax game between learners is carefully crafted. GANs have achieved tremendous success in domains such as image generation , and related studies have also shown their effectiveness in semi-supervised learning. However,  in the text field, GANs are difficult to train, most training objectives work well for only one model, either the discriminator or the generator, so rarely both learners can be optimal at the same time. This essentially arises from the adversarial nature of GANs, that during the process, optimizing one learner can easily destroy the learning ability of the other, making GANs fail to converge.

Another limitation of simultaneously optimizing the generator and the discriminator comes from the discrete nature of text in NLP, as no gradient propagation can be done from discriminators to generators. One theoretically sound attempt is to use reinforcement learning (RL), but the sparsity and the high variance of the rewards in NLP make the training particularly unstable \cite{caccia2019language}. 

To address these shortcomings, we novelly introduce a self-consistent learning framework based on one generator and one discriminator: the generator and the discriminator are alternately trained by way of cooperation instead of competition, and the selected samples are used as the medium to pass the feedback signal from the discriminator. Specifically, in each round of training, the samples generated by the generator are synthetically labeled by the discriminator, and then only part of them would be selected based on dynamic thresholds and used for the training of the discriminator and the generator in the next round. Several benefits can be discovered from this cooperative training process. First, a closed-loop form of cooperation can be established so that we can get the optimal generator and discriminator at the same time. Second, this framework helps improve the generation quality while ensuring the domain specificity of generator, which in turn contributes to training. Third, a steady stream of diverse synthetic samples can be added to the training in each round and lead to continuous improvement of the performance of all learners. Finally, we can start the training with only domain-related corpus and obtain strong results, while these data can be easily sampled with little cost or supervision. Also, the performance on labeled datasets can be further boosted based on the strong baselines. As an example to demonstrate the effectiveness of our framework in the text field, we examine it on four downstream text generation benchmarks, including AFQMC, CHIP-STS, QQP, and MRPC. The experiments show that our method significantly improves over standalone state-of-the-art discriminative models on zero-shot and full-data settings.

Our contributions are summarized as follows,

$\bullet$ We propose a self-consistent learning framework in the text field that incorporates the generator and the discriminator, in which both achieve remarkable performance gains simultaneously.

$\bullet$ We propose a dynamic selection mechanism such that cooperation between the generator and the discriminator drives the convergence to reach their scoring consensus.

$\bullet$ Experimental results show that the generator in our framework can continuously adjust its generation samples based on the performance of downstream tasks, while the discriminator can outperform the strong baselines.




\section{Related Works}
\section{Related Work}

\subsection{Knowledge Embedding Methods}

Knowledge embedding methods have been widely used in graph representation learning tasks and have achieved great success on knowledge base completion (a.k.a link prediction).
Translation-based methods aim at finding the transformation relationships from source to target. 
TransE~\cite{DBLP:conf/nips/BordesUGWY13}, the most representative translation-based model, projects entities and relations into a unified vector space and minimizes the \emph{energy function} of triples. 
Following this route, many translation-based methods have emerged.
TransH~\cite{DBLP:conf/aaai/WangZFC14} formulates the translating process on relation-specific hyperplanes.
TransR~\cite{DBLP:conf/aaai/LinLSLZ15} projects entities and relations into separate spaces.

Recently, some neural network methods have shown promising results in this task. 
ConvE~\cite{DBLP:conf/aaai/DettmersMS018} and ConvKB~\cite{DBLP:conf/naacl/NguyenNNP18} utilize Convolutional Neural Network (CNN) to increase parameter interaction between entities and relations. 
KBAT~\cite{DBLP:conf/acl/NathaniCSK19} employ Graph Neural Networks (GNN) as the encoder to aggregate multi-hop neighborhood information.

However, all these methods above utilize only structural information, which is not sufficient for more complicated situations in real world.
By incorporating multimodal information in the training process, our approach is able to improve the representations with external knowledge.

\subsection{Multimodal Methods}

Leveraging multimodal information has yielded extraordinary results in many NLP tasks~\cite{DBLP:conf/iccv/Ben-younesCCT17}.
DeViSE~\cite{DBLP:conf/nips/FromeCSBDRM13} and Imagined~\cite{DBLP:conf/aaai/CollellZM17} propose to integrate multimodal information with modality projecting which learns a mapping from one modality to another.
FiLM~\cite{DBLP:conf/aaai/PerezSVDC18} extends cross-modal attention mechanism to extract textual-attentive features in visual models.
MuRel~\cite{DBLP:conf/cvpr/CadeneBCT19} utilizes pair-wise bilinear interaction between modalities and regions to fully capture the complementarity.
IKRL~\cite{DBLP:conf/ijcai/XieLLS17} is the first attempt at multimodal knowledge representation learning, which utilizes image data of the entities as extra information based on TransE.
MKGC~\cite{DBLP:conf/starsem/SergiehBGR18} combines textual and visual features extracted by domain-specific models as additional multimodal information compared to IKRL.
MKBE~\cite{DBLP:conf/emnlp/PezeshkpourC018} creates multimodal knowledge graphs by adding images, descriptions and attributes, and employs DistMult~\cite{DBLP:journals/corr/YangYHGD14a} as scoring function. 

Although these approaches did incorporate multimodal information to improve performance, they cannot take full advantage of it as they fail to effectively model interactions between modalities.



\section{Methodology}
\subsection{cooperative or adversarial}
Following the principle of self-consistency outlined in \cite{ma2022principles}, a closed-loop training needs to be built between the generator and the discriminator, either cooperatively or adversarially. GANs are typical examples of adversarial learning, but training  GANs remains quite unstable. Let us consider an extreme case to show the possible instability: the discriminator can perfectly distinguish real data and fake data generated by the generator, and the generator can fully reproduce the real data distribution. Then the discriminator has only a 50\% probability of selecting all samples that are generated by the generator. Therefore, any further updates to the generator parameters based on the feedback from the discriminator deviate the generator from the optimum. Neither the generator nor the discriminator can likely be optimal \cite{arjovsky2017towards}. In practice, a very delicate balance needs to be maintained between the discriminator and the generator to keep the training stable. In terms of cooperatively closed-loop learning, as discussed below, it does not suffer from instability: the generator and the discriminator usually enhance each other. 

\subsection{Self-consistent Learning Framework}
In this section, we introduce our self-consistent learning (\textbf{SCL}) framework. 

As shown in Figure~\ref{framework}, our framework, similar to the GANs, consists of a generator and a discriminator model. However, contrasting to the GANs, these two parts in our framework work cooperatively to enhance each other. Specifically, for any given class $k$, the generator $\mathcal{G}$ now become a conditional generator that takes in an input sentence $s^a_k$ and generate an output sentence $s^b_k$. The discriminator $\mathcal{D}$ is then responsible for discriminating the sentence using a dynamic threshold $\epsilon_\mathcal{D}$. The discriminated sentence is used as positive or negative data for that specific class to continue the training process. Once the new discriminator is trained, the sentence is discriminated again by the new discriminator with a different dynamic threshold $\epsilon_\mathcal{G}$. This time only the positive data is passed to the generator as the training data for the new round. In this way, a closed loop of cooperation is formed. \looseness=-1

In the above closed-loop training, we propose a \textbf{selection mechanism} that uses dynamic thresholds to filter samples. This mechanism is empirically shown to play a critical role in closing the gap between the generator and the discriminator, and thus makes this cooperation loop a virtuous circle. Specifically, as shown in Equation~\ref{mlp}, the output probability $p_{\mathcal{D}}(y=k|s^b_k)$ that the sentence $\{s^b_k\}$ belongs to class $k$ is calculated from the embedding representation $\mathbf{h}$\footnote{We follow \cite{reimers-gurevych-2019-sentence} and use the embedding representation of $CLS$-token as the sentence representation $\mathbf{h}$ .} of $\{s^b_k\}$,

\begin{equation}
\label{mlp} 
p_{\mathcal{D}}(y=k|s^b_k) = \text{softmax}( \text{MLP}(\mathbf{h}))
\end{equation}

where $y$ represents the class label.  Then, through the filtering function ${\texttt{filter}^{(t)}_k(\cdot)}$ in round $t$ for the $k$-th class 
 in Equation~\ref{thre}, we keep samples whose output probability is not less than threshold $\epsilon_{t,k}$, while other generated samples whose confidence is lower than threshold $\epsilon_{t,k}$ are discarded.

\begin{equation}
\label{thre}
    \texttt{filter}^{(t)}_k(s^b_k) \triangleq {p_{\mathcal{D}}(k|s^b_k) \geq \epsilon_k^t} 
\end{equation}

where $\epsilon_{t,k}$ represents the dynamic threshold for accepting $\{s^b_k\}$ as negative or positive samples in the $t$-th round. The generalized threshold function for $\epsilon_k^t$ is defined as,

\begin{equation}
\label{dyn_thre}
    \epsilon_k^t = f(t, \mathcal{L}_{t-1,k}, \epsilon_k^{t-1}) 
\end{equation}

where $\mathcal{L}_{t-1,k}$ and $\epsilon_k^{t-1}$ represent the discriminator loss and threshold for round $t-1$, respectively. $\mathcal{L}_{0,k}$ is set as 0 and $\epsilon_k^0 = \lambda$, where $\lambda$ represents a hyperparameter. \looseness=-1

\begin{theorem}
\label{theorem_kl}
At round $t$, given the previous round discriminator $\mathcal{D}^{t-1}_\phi$, the aim of the optimization of the generator $\mathcal{G}^{t}_\theta$,  boils down to, \looseness=-1
\begin{equation}
\nonumber
    \min_{\theta}  \mathbb{D}_{KL}(p_{\mathcal{D}^{t-1}_\phi}^k(\cdot), p_{\mathcal{G}^{t}_\theta}^k(\cdot)) 
\end{equation}

where $\mathbb{D}_{KL}$ is the standard KL divergence, $p_{\mathcal{G}^{t}_\theta}^k(\cdot)$ refers to the degree of confidence that the sentences generated by the generator belong to a given class $k$ (we can either train the generator to express its confidence in the generated sentences or use a fixed third-party model to score them), and $p_{\mathcal{D}^{t-1}_\phi}^k(\cdot)$ the probability of being classified into class $k$ given by the discriminator.
\end{theorem}

Theorem \ref{theorem_kl} shows that the generator at round $t$ is encouraged to approximate the probability distribution given by the previous round discriminator. In particular, on the basis of a well-pretrained discriminator, the generated distribution of the generator can be guaranteed to be faithful to the real data distribution. 

\textbf{Proof.} We use the previous round  generator $\mathcal{G}^{t-1}_\theta$ to generate samples, and filter them using the previous round discriminator $\mathcal{D}^{t-1}_\phi$ with a threshold $\epsilon^{t-1}$, then these samples are used for the training of the current round generator $\mathcal{G}^{t}_\theta$. Therefore, the optimization of $\mathcal{G}^{t}_\theta$ will tend to  maximize the probability that the generated samples pass the discrimination for the fixed $\mathcal{D}^{t-1}_\phi$. For a given class $k$, we have \looseness=-1
\begin{equation}
\nonumber
\small
 \max_{\theta} \mathbb{E}_{x\sim  p_{\mathcal{G}^{t-1}_\theta}^k} p_{\mathcal{G}^{t}_\theta}^k(x) \quad \texttt{s.t.} \quad \texttt{filter}^{(t-1)}_k(x) = 1
\end{equation}
where the definition of function $\texttt{filter}^{(t-1)}_k(\cdot)$ has been given in Equation~\ref{thre}. 


The above objective is equivalent to sampling from the generator being optimized in round $t$ and making these samples pass the discrimination in round $t-1$ as much as possible, which gives
\begin{equation}
\nonumber
 \max_{\theta} \mathbb{E}_{x\sim  p_{\mathcal{G}^{t}_\theta}^k}p_{\mathcal{D}^{t-1}_{\phi}}^k(x)
\end{equation}
where $p_{\mathcal{D}^{t-1}_{\phi}}^k(x)$ is fixed.

A further transformation of the formula shows that
\begin{align*}
& \max_{\theta} \mathbb{E}_{x\sim  p_{\mathcal{G}^{t}_\theta}^k}p_{\mathcal{D}^{t-1}_{\phi}}^k(x) \\
\stackrel{(\romannumeral 1)} \Rightarrow & \max_\theta \int \mathrm{d}\pmb{\theta} \nabla_{\pmb{\theta}} \mathbb{E}_{x\sim  p_{\mathcal{G}^{t}_\theta}^k}p_{\mathcal{D}^{t-1}_\phi}^k(x) \\
\stackrel{(\romannumeral 2)} \Rightarrow
& \max_{\theta} \int \mathrm{d}\pmb{\theta} \mathbb{E}_{x\sim  p_{\mathcal{G}^{t}_\theta}^k} \nabla_{\pmb{\theta}} \log p_{\mathcal{G}^{t}_\theta}^k(x)p_{\mathcal{D}^{t-1}_\phi}^k(x) \\
\stackrel{(\romannumeral 3)} \Rightarrow
& \max_{\theta} \int \mathrm{d}\pmb{\theta}  \nabla_{\pmb{\theta}} \frac{1}{N}\sum_{i=1}^{N} \{\log p_{\mathcal{G}^{t}_\theta}^k(x_i)p_{\mathcal{D}^{t-1}_\phi}^k(x_i) \\&- \log p_{\mathcal{D}^{t-1}_\phi}^k(x_i)p_{\mathcal{D}^{t-1}_\phi}^k(x_i)\} \\
\stackrel{(\romannumeral 4)} \Rightarrow
& \min_{\theta} \mathbb{D}_{KL}(p_{\mathcal{D}^{t-1}_\phi}^k(\cdot), p_{\mathcal{G}^{t}_\theta}^k(\cdot))
\end{align*}

where $(\romannumeral 1)$ uses the integral property that integrating the derivative of a function gives the original function along with a constant, $(\romannumeral 2)$ takes advantage of the derivative property of the logarithmic function, $(\romannumeral 3)$ approximates the expectation of the probability distribution $p_{\mathcal{G}^{t}_\theta}^k(\cdot)$  by using averaging on $N$ samples sampling from $p_{\mathcal{G}^{t}_\theta}^k(\cdot)$, and adding a constant term $-\log p_{\mathcal{D}^{t-1}_\phi}^k(\cdot)p_{\mathcal{D}^{t-1}_\phi}^k(\cdot)$ with respect to $\theta$ under the summation would not change its derivative, and $(\romannumeral 4)$ cancels out the integral and the derivative and uses the definition of KL divergence. 
The above concludes our proof.

\textbf{Why Cooperative, Not Adversarial?} (1) the generator is no longer a challenger to the discriminator that only provides negative data points to fool it,  but now serves as a data augmenter to provide both positive and negative data points to enhance the discriminator; (2) the generator no longer updates its parameters through the policy gradients guided by the signals from the discriminator, but rather by utilizing the filtered data points to further improve its conditional generation quality. Note that by deliberately choosing the conditional generation paradigm along with the selection mechanism, we not only make the training more stable due to the different training goals, but also mitigate the mode collapse problem of GANs. Besides, by iterating through the loops, our framework achieves self-consistency by honing the domain specificity of the generator and increasing the domain data exposure of the discriminator.

\subsection{Text Generation}
We leverage the four text generation tasks ($i.e.$ $k=2$) as an example to demonstrate the effectiveness of our method. At this time, corresponding to Equation~\ref{thre}, $k=1/0$ represents the positive/negative class, and $\texttt{filter}^{(t)}_{1/0}$ represents the filter function in round $t$ for the positive/negative class respectively. First, let us introduce the formal definition of this task. Given two sentences ${s}^a = \{{w}_1^a, {w}_2^a, ..., {w}_{\ell_a}^a\}$ and ${s}^b = \{{w}_1^b, {w}_2^b, ..., {w}_{\ell_b}^b\}$, where ${w}_i^a$ and ${w}_j^b$ represent the $i$-th and $j$-th tokens in the sentences, and $\ell_a$ and $\ell_b$ indicate the  length of ${s}^a$ and ${s}^b$. The goal of this task is to learn a discriminator $\mathcal{D}$ to precisely predict the label $y=\mathcal{D}({s}^a, {s}^b)$, where $y \in \mathcal{Y}=\{0, 1\}$ indicates whether the two sentences are similar.

In our task, $\mathcal{G}$ is trained to generate a similar sentence ${s}^b$ from any given sentence ${s}^a$ and $\mathcal{D}$ is trained to predict label $y$ from any given sentence pair $\{{s}^a, {s}^b\}$. As demonstrated in Figure~\ref{framework}, there are mainly two training processes in the entire framework: fix $\mathcal{G}$ to train $\mathcal{D}$ and fix $\mathcal{D}$ to train $\mathcal{G}$. We introduce the two training procedures in detail with the $t$-th round training.

\textbf{Training $\mathcal{D}$}: We first randomly sample $s^a_t$ from domain-related corpus $C$, and then input $s^a_t$ to $\mathcal{G}^t$ to generate $s^b_t$. Next, we feed sentence pair $\{{s}^a_t, {s}^b_t\}$ into $\mathcal{D}^{t-1}$ to predict the label $y_{t-1}$, and filter $\{{s}^a_t, {s}^b_t, y_{t-1}\}$ using threshold $\epsilon_{\mathcal{D}}^{t-1}$. Finally, we train $\mathcal{D}^{t-1}$ on the selected data and pre-training data $P$ to get an improved discriminator $\mathcal{D}^{t}$. Note that the filtered data have both positive and negative samples. The update process of $\mathcal{D}$ seeks to minimize the cross-entropy loss over all instances:

\begin{equation}
\label{train_dis}
\begin{split}
	\mathcal{L}_\mathcal{D}(\boldsymbol{s}, \boldsymbol{y}) = \frac{1}{|\boldsymbol{s}|} \sum_{i=1}^{|\boldsymbol{s}|}-[y_i\cdot \log p_{\mathcal{D}}(y_i=1|s_i^a, s_i^b)\\+(1-y_i)\cdot \log (1-p_{\mathcal{D}}(y_i=1|s_i^a, s_i^b))] 
\end{split}
\end{equation}

\textbf{Training $\mathcal{G}$}: We feed the generated sentence pairs $\{{s}^a_t, {s}^b_t\}$ into $\mathcal{D}^{t}$ to predict new labels $y_{t}$, and then filter $\{{s}^a_t, {s}^b_t, y_{t}\}$ using threshold $\epsilon_\mathcal{G}^t$ and additional rules \footnote{The additional rules are used to exclude sentences which are too long, too short, or too similar according to the longest common substring algorithm.}. Note that the filtered data has only positive samples. For the filtered data, we supplement it with the pre-training data $P$ to update $\mathcal{G}^{t}$ to $\mathcal{G}^{t+1}$ \footnote{Note that the pre-training data $P$ is used to warm up $\mathcal{G}$ and $\mathcal{D}$. Although pre-training data is not mandatory in subsequent training, we empirically found that including it when training $\mathcal{G}$ can prevent language degeneration and improve downstream performances.} We also take out ${s}^b_t$ from the filtered data and add them to the domain-related corpus. The expanded domain corpus are used to sample conditional sentences in the next round of generation. The update procedure of $\mathcal{G}$ employs the negative log-likelihood function over all instances:

\begin{equation}
\nonumber
	\mathcal{L}_\mathcal{G}(\boldsymbol{s^a}, \boldsymbol{s^b}) = -\frac{1}{|\boldsymbol{s^b}|} \sum_{t=1}^{|\boldsymbol{s^b}|}\log p_{\mathcal{G}}(s^b_t|s^b_{<t}, \boldsymbol{s^a})
\end{equation}

For the selection mechanism, we adopt the form $\epsilon^t=m*t+\lambda$ after comparing the effects of different threshold functions through experiments according to Equation~\ref{dyn_thre}, where $m$ is the increment of the threshold for each round, $\lambda$ is the initial threshold, and $\epsilon^t$ is the threshold for rounds $t$.

In the process of training $\mathcal{G}$, since the sentences generated in each round are added to the domain-related corpus, the source of domain-specific data is thus monotonically expanding by iterating the self-consistent learning loop. The formalized process is shown in Algorithm~\ref{scl}. 

% To better understand the first two steps. Specifically, in the first step, we start by selecting pre-trained language models as the initial states for both the generator (G0G^0) and the discriminator (D0D^0). These models have been trained on extensive corpora, such as WuDaoCorpora for Chinese and WikiText for English. However, at this stage, the generator and the discriminator may not excel at generating similar sentences and discerning similarity, because the pre-trained data does not include examples of these specific task format. In the second step, to tackle the aforementioned issue, we preform further pre-training using domain-specific data containing similar sentence pairs. This "warm-up" process endows the initialized PLMs with the preliminary capability to generate similar sentences and discern their similarity.

\begin{algorithm}
\caption{Self-consistent Learning (\textbf{SCL})} 
\label{scl}
\begin{algorithmic}[1]
\REQUIRE Generator $\mathcal{G}$; Discriminator $\mathcal{D}$; Domain-Related Corpus $C$; Pre-training Data $P$.
\STATE Initialize $\mathcal{G}^0$ and $\mathcal{D}^0$ with pre-trained language models;
\STATE Warm-up $\mathcal{G}^0$ and $\mathcal{D}^0$ with pre-training data $P$ to get $\mathcal{G}^1$ and $\mathcal{D}^1$;
\FOR {each round $i \in [1, n]$}
\IF {Two consecutive rounds of discriminator still improve}
\STATE Generate similar sentences $s^b\sim p_{\mathcal{G}^i}(\cdot|s^a)$ from sampled sentences $s^a$ from $C$;
\STATE Predict pseudo-labels $y^i \sim p_{\mathcal{D}^i}(\cdot|s^a, s^b)$;
\STATE Use threshold $\epsilon_\mathcal{D}^i$ to select data on $\{s^a, s^b, y^i\}$ to train $\mathcal{D}^{i+1}$;
\STATE Predict pseudo-labels $y^{i+1} \sim p_{\mathcal{D}^{i+1}}(\cdot|s^a, s^b)$;
\STATE Use threshold $\epsilon_\mathcal{G}^i$ and additional rules to select data on $\{s^a, s^b, y^{i+1}\}$ to train $\mathcal{G}^{i+1}$;
\ENDIF
\ENDFOR
\end{algorithmic}
\end{algorithm}


\section{Experiments}
\section{Experimental Results}
In this section, we validate the effectiveness of our proposal. We first introduce datasets, metrics and implementation details involved in our evaluation. Then, we compare \netname{} with state-of-the-art methods, conduct an ablation study on our model and, finally, discuss its limitations.


\begin{table*}[htbp] \scriptsize
	\renewcommand\tabcolsep{2.3pt} 
	\centering
	\scalebox{0.85}{
	\begin{tabular}{@{}ccccccccccccccccc@{}}
		\toprule
		 Dataset & Scale & Metrics & GF~\cite{he2010guided} & SD~\cite{ham2017robust}  & GSRPT~\cite{lutio2019guided} & MSG~\cite{hui2016depth} & DKN~\cite{kim2021deformable} & FDKN~\cite{kim2021deformable} & PMBANet~\cite{ye2020pmbanet} & FDSR~\cite{he2021towards} & JIIF~\cite{tang2021joint} & DCTNet~\cite{zhao2022discrete} & LGR~\cite{de2022learning} & DADA~\cite{metzger2022guided} & DSR-EI & DSR-EI$^+$ \\ \midrule
		\multirow{6}{*}{\rotatebox[origin=l]{90}{\scriptsize \textbf{Middlebury}}} & \multirow{2}{*}{$4\times$} 
		& MSE & 33.3 & 24.9 & 39.8 & 4.13 & 4.29 & 3.60 & 4.72 & 7.72 & 2.70 & 5.00 & 3.04 & \bronze{2.58} & \gold{2.46} & \silver{2.56} \\
		& & MAE & 1.27 & 0.46 & 0.79 & 0.22 & 0.18 & 0.16 & 0.25 & 0.35 & \bronze{0.11} & 0.24 & 0.13 & \bronze{0.11} & \silver{0.08} & \gold{0.07} \\ \cline{2-17}
		& \multirow{2}{*}{$8\times$} 
		& MSE & 40.5 & 82.5 & 32.7 & 10.5 & 11.2 & 10.4 & 9.48 & 23.2 & 8.01 & 15.1 & 7.26 & \silver{5.68} & \bronze{6.20} & \gold{5.13} \\
		& & MAE & 1.49 & 0.86 & 0.82 & 0.43 & 0.38 & 0.37 & 0.38 & 0.69 & 0.27 & 0.57 & 0.24 & \bronze{0.20} & \gold{0.18} & \gold{0.18} \\ \cline{2-17}
		& \multirow{2}{*}{$16\times$} 
		& MSE & 67.4 & 511 & 41.5 & 34.2 & 47.6 & 38.5 & 30.6 & 55.4 & 37.5 & 52.3 & 24.7 & \silver{16.3} & \gold{15.8} & \bronze{16.6}  \\
		& & MAE & 2.21 & 1.73 & 1.24 & 1.06 & 1.42 & 1.18 & 0.89 & 1.51 & 0.98 & 1.50 & 0.67 & \bronze{0.48} & \silver{0.47} & \gold{0.40} \\ \hline\hline
	    % middlebury end
		\multirow{6}{*}{\rotatebox[origin=l]{90}{\scriptsize \textbf{NYUv2}}} & \multirow{2}{*}{$4\times$}
		& MSE & 114 & 36.0 & 112 & 6.85 & 11.4 & 9.07 & 10.8 & 10.1 & \bronze{3.28} & 3.63 & 6.45 & 4.83 & \silver{2.82} & \gold{2.75}\\
		& & MAE & 3.91 & 1.31 & 3.61 & 0.81 & 1.03 & 0.85 & 0.93 & 0.94 & \bronze{0.52} & 0.68 & 0.73 & 0.64 & \silver{0.49} & \gold{0.47}\\ \cline{2-17}
		& \multirow{2}{*}{$8\times$} 
		& MSE & 142 & 105 & 122 & 24.1 & 29.8 & 29.9 & 17.2 & 19.5 & \bronze{15.2} & 20.9 & 19.6 & 16.6 & \gold{11.8} & \gold{11.8}\\
		& & MAE & 4.47 & 2.57 & 3.86 & 1.66 & 1.82 & 1.80 & 1.38 & 1.38 & \bronze{1.29} & 1.79 & 1.42 & 1.30 & \silver{1.12} & \gold{1.09}\\ \cline{2-17}
		& \multirow{2}{*}{$16\times$} 
		& MSE & 249 & 533 & 219 & 84.5 & 115 & 113 & 84.9 & 86.4 & 59.9 & 77.0 & 67.5 & \bronze{59.0} & \silver{47.8} & \gold{47.1} \\
		& & MAE & 6.34 & 5.07 & 5.40 & 3.35 & 4.01 & 3.95 & 3.26 & 3.35 & 2.81 & 3.61 & 2.90 & \bronze{2.64} & \silver{2.48} & \gold{2.40}\\ \hline\hline
		% NYU end
		\multirow{6}{*}{\rotatebox[origin=l]{90}{\scriptsize \textbf{DIML}}} & \multirow{2}{*}{$4\times$}
		& MSE & 25.6 & 10.5 & 20.7 & 1.73 & 3.47 & 2.20 & 3.05 & 2.75 & \bronze{1.19} & 2.09 & 1.68 & 1.33 & \silver{0.70} & \gold{0.65} \\
		& & MAE & 1.45 & 0.40 & 1.15 & 0.22 & 0.33 & 0.23 & 0.31 & 0.29 & \bronze{0.16} & 0.31 & 0.20 & 0.17 & \silver{0.13} & \gold{0.12} \\ \cline{2-17}
		& \multirow{2}{*}{$8\times$} 
		& MSE & 34.1 & 44.9 & 23.0 & 4.13 & 5.47 & 5.95 & 5.87 & 8.40 & 3.65 & 7.08 & 3.51 & \bronze{2.93} & \silver{2.12} & \gold{2.09} \\
		& & MAE & 1.77 & 0.83 & 1.26 & 0.40 & 0.45 & 0.47 & 0.47 & 0.66 & 0.32 & 0.65 & 0.31 & \bronze{0.28} & \gold{0.22} & \gold{0.22} \\ \cline{2-17}
		& \multirow{2}{*}{$16\times$} 
		& MSE & 66.3 & 41.1 & 39.3 & 13.0 & 19.3 & 20.8 & 13.8 & 32.9 & 11.7 & 23.4 & 9.45 & \bronze{7.61} & \gold{6.29} & \silver{6.31} \\
		& & MAE & 2.74 & 1.91 & 1.78 & 0.93 & 1.20 & 1.24 & 0.87 & 1.66 & 0.81 & 1.75 & 0.68 & \bronze{0.59} & \silver{0.52} & \gold{0.50} \\
		% DIML end
    \bottomrule
	\end{tabular}}
    \vspace{-0.3cm}
	\caption{\textbf{Results on Middlebury, NYUv2 and DIML datasets.} The lower the MSE and MAE, the better.}
	\label{sota_comparison_mid_nyu_diml}
\end{table*}



\begin{table*}[t] \footnotesize
	\renewcommand\tabcolsep{1.5pt} 
	\centering
	\scalebox{0.85}{
	\begin{tabular}{@{}ccccccccccccccccc@{}}
		\toprule
		 Scale & SDF~\cite{li2016deep} & SVLRM~\cite{pan2019spatially} & DJF~\cite{li2016deep} & DJFR~\cite{li2019joint} & PAC~\cite{su2019pixel} & CUNet~\cite{deng2020deep} & FDKN~\cite{kim2021deformable} & DKN~\cite{kim2021deformable} & FDSR~\cite{he2021towards} & DCTNet~\cite{zhao2022discrete} & RSAG~\cite{yuan2023recurrent} & DSR-EI & DSR-EI$^+$ \\ \midrule
		$4\times$ & 2.00 & 3.39 & 3.41 & 3.35 & 1.25 & 1.18 & 1.18 & 1.30 & 1.16 & \bronze{1.07} & 1.14 & \gold{0.91} & \gold{0.91} \\
		$8\times$ & 3.23 & 5.59 & 5.57 & 5.57 & 1.98 & 1.95 & 1.91 & 1.96 & 1.82 & 1.78 & \bronze{1.75} & \gold{1.37} & \silver{1.38} \\
		$16\times$ & 5.16 & 8.28 & 8.15 & 7.99 & 3.49 & 3.45 & 3.41 & 3.42 & 3.06 & 3.18 & \bronze{2.96} & \gold{2.10} & \gold{2.10}  \\
    \bottomrule
	\end{tabular}}
	\vspace{-0.3cm}
	\caption{\textbf{Results on the RGBDD dataset.} We report RMSE, the lower the better.}
	\label{sota_comparison_rgbdd}
\end{table*}



\subsection{Datasets and Metrics}
We evaluate \netname{} on four datasets, compared with existing methods when super-solving depth maps by three different upsampling factors: $4\times,\ 8\times$, and $16\times$. 

\textbf{Middlebury}\cite{scharstein2003high,scharstein2007learning,hirschmuller2007evaluation,scharstein2014high}. We train all learning-based methods using 50 RGB-D images with ground truth from Middlebury 2005, 2006 and 2014 datasets. As in~\cite{de2022learning}, we retain 5 for validation and 5 for testing. 

\textbf{NYUv2}\cite{silberman2012indoor}. It contains 1449 RGB-D images in total. Following \cite{de2022learning}, we randomly split it into 849 RGB-D images for the training set, 300 for the validation set and 300 for the test set. Compared to \cite{ye2020pmbanet,liu2022pdr}, it comes with a validation set to make the comparison fairer.

\textbf{DIML}\cite{kim2016structure,kim2017deep,kim2018deep,cho2021deep} consists of 2 million color images and corresponding depth maps from indoor and outdoor scenes. We adopt the same strategy outlined in \cite{de2022learning}, i.e., considering only the indoor data subset, and use 1440 for training, 169 for validation, and 503 for testing.

\textbf{RGBDD}\cite{he2021towards} is a new real-world dataset for GDSR, which consists of 4811 image pairs. For evaluation, we follow the protocol described in \cite{he2021towards}, using 2215 images (1586 portraits, 380 plants, 249 models) as the training set and 405 images (297 portraits, 68 plants, 40 models) as the test set. 

\textbf{Metrics.} Following \cite{de2022learning}, we compute mean square error (MSE / $cm^2$) and mean absolute error (MAE / $cm$) as metrics on Middlebury, NYUv2 and DIML. For RGBDD, we use root mean square error (RMSE / $cm$) as in \cite{he2021towards}. 

\subsection{Implementation Details}
During training, the HR depth maps and the color images are randomly cropped into $256\times 256$ patches. LR depth patches are generated by bicubic interpolation at $64\times 64$, $32\times 32$, $16\times 16$ resolution for $4\times$, $8\times$ and $16\times$ factors, respectively. We randomly extract about 75K, 168K, 223K and 232K patches from Middlebury, NYUv2, DIML and RGBDD for training. Before being fed to the network, depth maps and images are normalized in the [0, 1] range.

We use Pytorch \cite{paszke2019pytorch} to implement and train \netname{}, on a single Nvidia RTX 3090 GPU. The batch size is set to 4, using Adam as the optimizer. The learning rate is initialized to $1\times 10^{-4}$, then performing a 5-epoch warm-up and cosine annealing. We use random rotation, horizontal/vertical flipping as data augmentation. According to the size of the four datasets, we train our network for 1505, 198, 155 and 109 epochs on Middlebury, NYUv2, DIML and RGBDD, respectively. 
When evaluating results on a specific dataset, we do not perform any pre-training on the others. Following \cite{de2022learning}, testing is performed by processing $256\times256$ patches at a time on Middlebury, NYUv2 and DIML for fairness, while full-resolution images are processed for RGBDD.

\begin{figure*}[t] 
	\centering
	\renewcommand\tabcolsep{1.5pt} 
	\begin{tabular}{cccccccccccc}
	\vspace{-0.1cm}
    \rotatebox[origin=l]{90}{\scriptsize \quad \textbf{Middlebury}} & \includegraphics[height=0.6in]{./figs/sota_comp_middlebury/389/Middlebury_389_img.pdf}
        \hspace{-1.8mm} & \includegraphics[height=0.6in]{./figs/sota_comp_middlebury/389/Middlebury_389_source.pdf}
	\hspace{-1.8mm} &  \includegraphics[height=0.6in]{./figs/sota_comp_middlebury/389/Middlebury_389_GT.pdf}
	\hspace{-1.8mm} & \includegraphics[height=0.6in]{./figs/sota_comp_middlebury/389/Middlebury_389_PMBA.pdf}
	\hspace{-1.8mm} & \includegraphics[height=0.6in]{./figs/sota_comp_middlebury/389/Middlebury_389_FDSR.pdf}
	\hspace{-1.8mm} & \includegraphics[height=0.6in]{./figs/sota_comp_middlebury/389/Middlebury_389_JIIF.pdf}
	\hspace{-1.8mm} & \includegraphics[height=0.6in]{./figs/sota_comp_middlebury/389/Middlebury_389_DCTnet.pdf}
	\hspace{-1.8mm} & \includegraphics[height=0.6in]{./figs/sota_comp_middlebury/389/Middlebury_389_LGR.pdf}
	\hspace{-1.8mm} & \includegraphics[height=0.6in]{./figs/sota_comp_middlebury/389/Middlebury_389_MSS.pdf}
        
        \hspace{-1.8mm} & \includegraphics[height=0.6in]{./figs/sota_comp_middlebury/389/Middlebury_389_ours.pdf}
    \\ \vspace{-0.1cm}
    
    \rotatebox[origin=l]{90}{\scriptsize \quad \textbf{NYUv2}} & \includegraphics[height=0.6in]{./figs/sota_comp_nyu/357/NYU_357_img.pdf}
	\hspace{-1.8mm} & \includegraphics[height=0.6in]{./figs/sota_comp_nyu/357/NYU_357_source.pdf}
	\hspace{-1.8mm} & \includegraphics[height=0.6in]{./figs/sota_comp_nyu/357/NYU_357_GT.pdf}
	\hspace{-1.8mm} & \includegraphics[height=0.6in]{./figs/sota_comp_nyu/357/NYU_357_PMBA.pdf}
	\hspace{-1.8mm} & \includegraphics[height=0.6in]{./figs/sota_comp_nyu/357/NYU_357_FDSR.pdf}
	\hspace{-1.8mm} & \includegraphics[height=0.6in]{./figs/sota_comp_nyu/357/NYU_357_JIIF.pdf}
	\hspace{-1.8mm} & \includegraphics[height=0.6in]{./figs/sota_comp_nyu/357/NYU_357_DCTnet.pdf}
	\hspace{-1.8mm} & \includegraphics[height=0.6in]{./figs/sota_comp_nyu/357/NYU_357_LGR.pdf}
	\hspace{-1.8mm} & \includegraphics[height=0.6in]{./figs/sota_comp_nyu/357/NYU_357_MSS.pdf}
 
	\hspace{-1.8mm} & \includegraphics[height=0.6in]{./figs/sota_comp_nyu/357/NYU_357_ours.pdf}
	\\ 
	
    \rotatebox[origin=l]{90}{\scriptsize \quad \textbf{DIML}} & \includegraphics[height=0.6in]{./figs/sota_comp_diml/856/DIML_856_img.pdf}
	\hspace{-1.8mm} & \includegraphics[height=0.6in]{./figs/sota_comp_diml/856/DIML_856_source.pdf}
	\hspace{-1.8mm} & \includegraphics[height=0.6in]{./figs/sota_comp_diml/856/DIML_856_GT.pdf}
	\hspace{-1.8mm} & \includegraphics[height=0.6in]{./figs/sota_comp_diml/856/DIML_856_PMBA.pdf}
	\hspace{-1.8mm} & \includegraphics[height=0.6in]{./figs/sota_comp_diml/856/DIML_856_FDSR.pdf}
	\hspace{-1.8mm} & \includegraphics[height=0.6in]{./figs/sota_comp_diml/856/DIML_856_JIIF.pdf}
	\hspace{-1.8mm} & \includegraphics[height=0.6in]{./figs/sota_comp_diml/856/DIML_856_DCTnet.pdf}
	\hspace{-1.8mm} & \includegraphics[height=0.6in]{./figs/sota_comp_diml/856/DIML_856_LGR.pdf}
	\hspace{-1.8mm} & \includegraphics[height=0.6in]{./figs/sota_comp_diml/856/DIML_856_MSS.pdf}
 
	\hspace{-1.8mm} & \includegraphics[height=0.6in]{./figs/sota_comp_diml/856/DIML_856_ours.pdf}
 \\
	& \scriptsize \textbf{(a)} RGB & \scriptsize \textbf{(b)} Bicubic & \scriptsize \textbf{(c)} GT & \scriptsize \textbf{(d)} PMBA & \scriptsize \textbf{(e)} FDSR & \scriptsize \textbf{(f)} JIIF & \scriptsize \textbf{(g)} DCTNet & \scriptsize \textbf{(h)} LGR & \scriptsize \textbf{(i)} \netname{} & \scriptsize \textbf{(j)} \netname{} (depth)
	\end{tabular}
    \vspace{-0.3cm}
	\caption{\textbf{Qualitative comparison on Middlebury, NYUv2 and DIML datasets (scaling factor $8\times$).} From left to right: (a) RGB image, (b) Bicubic upsampled depth map, (c) GT; then, error maps achieved by selected methods: (d) PMBA~\cite{ye2020pmbanet}, (e) FDSR~\cite{he2021towards}, (f) JIIF~\cite{tang2021joint}, (g) DCTNet~\cite{zhao2022discrete}, (h) LGR~\cite{de2022learning}; finally, (i) error maps and (j) predictions by \netname.} 
	\label{qualitative}
\end{figure*}


\begin{table*}[htbp] \footnotesize
	\renewcommand\tabcolsep{1.5pt} 
	\centering
	\scalebox{0.85}{
	\begin{tabular}{@{}ccccccccccccccccc@{}}
		\toprule
		 Testing Dataset & Metric & GF\cite{he2010guided} & SD~\cite{ham2017robust}  & GSRPT~\cite{lutio2019guided} & MSG~\cite{hui2016depth} & FDKN~\cite{kim2021deformable} & PMBANet~\cite{ye2020pmbanet} & FDSR~\cite{he2021towards} & JIIF~\cite{tang2021joint} & DCTNet~\cite{zhao2022discrete} & LGR~\cite{de2022learning} & \netname$^+$ \\ \midrule
		\multirow{2}{*}{DIML}
		& MSE & 34.1 & 44.9 & 23.0 & 5.76 & 6.74 & 7.35 & 7.73 & \silver{4.10} & 5.64 & \bronze{4.95} & \gold{3.72} \\
		& MAE & 1.77 & 0.83 & 1.26 & 0.51 & 0.53 & 0.59 & 0.74 & \silver{0.38} & 0.77 & \bronze{0.40} & \gold{0.36} \\ \hline
		\multirow{2}{*}{Middlebury\textit{-HR}}
		& MSE & 40.5 & 82.5 & 32.7 & 11.0 & \bronze{10.0} & \silver{9.62} & 18.4 & 19.3 & 17.5 & \gold{8.25} & 14.6 \\
		& MAE & 1.49 & 0.86 & 0.82 & 0.54 & \silver{0.43} & \bronze{0.46} & 0.73 & 0.74 & 0.77 & \gold{0.35} & 0.54  \\ \hline
		\multirow{2}{*}{Middlebury\textit{-LR}}
		& MSE & 25.6 & 28.8 & 15.8 & 8.89 & 5.54 & 4.16 & 6.92 & 4.40 & 6.96 & 5.94 & \gold{3.44} \\
		& MAE & 2.31 & 2.07 & 1.73 & 1.62 & 0.99 & \silver{0.91} & 1.09 & \bronze{0.92} & 1.15 & 1.11 & \gold{0.87}  \\
        \bottomrule
	\end{tabular}}
	\vspace{-0.3cm}
	\caption{\textbf{Cross-dataset generalization.} All methods are trained on NYUv2 and tested on DIML/Middlebury with factor $8\times$. Middlebury\textit{-HR} is the test set defined in \cite{de2022learning}, Middlebury\textit{-LR} is the one from \cite{tang2021joint}. The lower MSE and MAE, the better. }
	\label{cross-data_comparison}
\end{table*}

\subsection{Comparison with State-of-the-Art}
We compare \netname{} to GF \cite{he2010guided}, SD \cite{ham2017robust}, GSRPT \cite{lutio2019guided}, MSG \cite{hui2016depth}, DKN and its fast implementation FDKN \cite{kim2021deformable}, PMBANet \cite{ye2020pmbanet}, FDSR \cite{he2021towards}, JIIF \cite{tang2021joint}, DCTNet \cite{zhao2022discrete}, LGR \cite{de2022learning}, and finally to DADA~\cite{metzger2022guided} on Middlebury, NYUv2 and DIML datasets. We could not compare with PDRNet \cite{liu2022pdr} under the same setting because the source code is unavailable at the time of writing. For the other methods, we use the results from \cite{de2022learning} or the officially published codes, and results from \cite{yuan2023recurrent,metzger2022guided} for concurrent works. On the RGBDD dataset, the proposed network is compared to SDF~\cite{li2016deep}, SVLRM \cite{pan2019spatially}, DJF~\cite{li2016deep}, DJFR~\cite{li2019joint}, PAC~\cite{su2019pixel}, CUNet~\cite{deng2020deep}, FDKN~\cite{kim2021deformable}, DKN~\cite{kim2021deformable}, FDSR~\cite{he2021towards}, DCTNet~\cite{zhao2022discrete} and RASG~\cite{yuan2023recurrent}. To be fair with DCTNet~\cite{zhao2022discrete}, we downsample depth maps as the LR input.  
When reporting results, we highlight \gold{absolute}, \silver{second} and \bronze{third} best methods for each metric on each dataset.

\textbf{Quantitative Comparison.} Tabs. \ref{sota_comparison_mid_nyu_diml} and \ref{sota_comparison_rgbdd} report the accuracy of super-solved depth maps at factors $4\times$, $8\times$ and $16\times$ on the four datasets. As expected, learning-based methods show a significant improvement over traditional methods \cite{he2010guided,ham2017robust,lutio2019guided}. \netname{} vastly outperforms any existing network, with larger gaps in accuracy with the increasing of the upsampling factor. This can be attributed to the limitations affecting existing methods, i.e., 1) the guidance of either explicit or implicit RGB features alone being insufficient; 2) multi-modal information fusion on a single scale being not flexible enough to deal with complex scenes. Both limitations are fully addressed by \netname, which consistently outperforms concurrent works \cite{metzger2022guided,yuan2023recurrent}. 


The margin is consistent both on perfect (Middlebury) and noisy datasets (NYUv2, DIML, RGBDD), with the latter being a more challenging, realistic benchmark. Although \netname$^+$ is definitely the absolute best, its margin over \netname{} is negligible, with tiny gains yielded by NLSPN with respect to our main modules. Indeed, \netname{} alone consistently outperforms any other approach already.

       
\textbf{Qualitative Comparison.}
Fig. \ref{qualitative} shows qualitative comparisons of $8\times$ super-solved depth maps on Middlebury, NYUv2 and DIML datasets, respectively. From left to right, we show, the RGB image and LR depth map, followed by the ground truth HR depth and error maps obtained by several state-of-the-art frameworks, concluding with ours in the second-to-last columns. In each of the three examples, the lower error magnitude produced by \netname{}$^+$ further demonstrates its superior accuracy. 

\textbf{Cross-dataset Generalization.}
We conclude the comparison with existing methods by conducting cross-dataset experiments with $8\times$ factor. All methods are trained on the NYUv2 dataset and directly evaluated on DIML and Middlebury. Table \ref{cross-data_comparison} collects quantitative results for the 11 selected methods. Again, CNN-based methods attain better performance than traditional approaches, despite the domain gap playing a significant role in performance -- as evident by comparing results with Table \ref{cross-data_comparison}. Nonetheless, \netname{} outperforms any other framework on DIML. 


\begin{figure}	
	\centering	
	\captionsetup[subfigure]{font=footnotesize,textfont=footnotesize}
	\subfloat[RGB]{	
		\centering	
		\label{cross_dataset} 
		\includegraphics[height=0.8in]{./figs/ablation_figure/cross_dataset/receptive_field/cross_dataset.pdf}}	
	\hspace{-2mm}
	\subfloat[$D_{hr}$]{	
		\centering	
		\label{HR}
		\includegraphics[width=0.8in]{./figs/ablation_figure/cross_dataset/receptive_field/HR.pdf}}
	\hspace{-2mm}
	\subfloat[$D_{lr}$]{	
		\centering	
		\label{LR}
		\includegraphics[width=0.8in]{./figs/ablation_figure/cross_dataset/receptive_field/LR.pdf}}
		\vspace{-0.3cm}
	\caption{\textbf{Image context processed on Middlebury -- HR vs LR.} (a) RGB image and depth patches $D$ processed when testing on (b) Middlebury\textit{-HR} and (c) Middlebury\textit{-LR}. }	
	\label{hr-lr} 
\end{figure}

When considering the Middlebury dataset, we evaluate using the setting proposed in \cite{de2022learning} -- Middlebury\textit{-HR} in the table. In this case, our results are slightly less accurate compared to a few existing methods. However, given the very high resolution of Middlebury images, we argue that this testing protocol -- i.e., consisting of processing $256\times 256$ crops at a time -- penalizes our network's ability to leverage the global context in the input that results irremediably reduced to a very local area in these images. Therefore, we also evaluate on Middlebury test set defined by~\cite{tang2021joint} -- Middlebury-\textit{LR} in the table. Note that different subsets of images are used in Middlebury\textit{-HR} and Middlebury-\textit{LR} splits. Besides, Middlebury-\textit{LR} images are resized and processed without cropping, i.e., used at full-size after resizing, allowing to fully exploit global context, while this is not feasible with Middlebury-\textit{HR} due to memory constraints. In this case, \netname{} attains the best performance again, confirming our previous analysis, as shown in Tab. \ref{cross-data_comparison}. Such a difference in terms of context is highlighted in Fig. \ref{hr-lr}.

\begin{table}[t]
    \centering
	\renewcommand\tabcolsep{3pt} 
    \scalebox{0.5}{
    \begin{tabular}{ccc}

    \begin{tabular}{@{}ccccccc@{}} %\label{hf_infomation}
		\toprule
		\textbf{No.} & \textbf{Gradient} & \tabincell{c}{\textbf{Shallow} \\ \textbf{Feature}} & \textbf{LCF} & \textbf{ResBlock} & \textbf{MSE} & \textbf{MAE}\\
		\midrule
		(\uppercase\expandafter{\romannumeral1}) & \XSolidBrush &  \Checkmark     &  \Checkmark &  & 13.1 & 1.19 \\
		(\uppercase\expandafter{\romannumeral2}) & \Checkmark &    \XSolidBrush   &   &  & 12.4 & 1.14 \\
		(\uppercase\expandafter{\romannumeral3}) & \Checkmark &    \Checkmark     &   & \Checkmark & 12.3 & 1.15 \\
		\rowcolor{LightYellow}
		(\uppercase\expandafter{\romannumeral4}) & \Checkmark &    \Checkmark     & \Checkmark  &  & \gold{11.8} & \gold{1.12} \\
		\bottomrule
	\end{tabular}
	
	& \quad &
	
	\begin{tabular}{@{}clcc@{}} %\label{edge_types}
		\toprule
		\specialrule{0em}{3pt}{3pt}
		\multicolumn{1}{c}{\textbf{No.}} & 
		\tabincell{l}{\textbf{HF Information} \textbf{ \quad\quad\quad\quad}} & \textbf{MSE} & \textbf{MAE}\\
		\specialrule{0em}{3pt}{2pt}
		\midrule
		(\uppercase\expandafter{\romannumeral1}) & 
		{Canny Edge} & 12.0 & 1.13 \\
		(\uppercase\expandafter{\romannumeral2}) & 
		{Gaussian Edge} & 12.1 & 1.16 \\
		(\uppercase\expandafter{\romannumeral3}) & 
		{DCT} & 12.1 & 1.15 \\
		(\uppercase\expandafter{\romannumeral4}) & 
		{Wavelet Transform} & 12.1 & 1.15  \\
		\rowcolor{LightYellow}
		(\uppercase\expandafter{\romannumeral5}) & 
		{Gradient Map} & \gold{11.8} & \gold{1.12} \\
		\bottomrule
	\end{tabular}
	
	\\
	\textbf{(a)} & \quad & \textbf{(b)} 
	\\
	\\
	
	
	\begin{tabular}{@{}clcccc@{}} %\label{dsp_ablation}
		\toprule
		\textbf{No.} & \textbf{Config.} & \textbf{Params (M)} & \textbf{Flops (G)} & \textbf{MSE} & \textbf{MAE}\\
		\midrule
		(\uppercase\expandafter{\romannumeral1}) & EdgeNet \cite{liu2021multi}    & 5.78 &  95.6  & 12.0 & \gold{1.12} \\
		(\uppercase\expandafter{\romannumeral2}) & SCPA \cite{zhao2020efficient}  & 0.29 &  13.1  & 12.5 & 1.16 \\
		\rowcolor{LightYellow}
		(\uppercase\expandafter{\romannumeral3}) & HFEB       & \gold{0.27} & \gold{11.6}  & \gold{1.18} & \gold{1.12} \\
		\rowcolor{white}
		\bottomrule
		\multicolumn{4}{c}{\quad\quad\textbf{(c)}} \\
		\\
		\toprule
		\textbf{No.} & \textbf{Config.} & \textbf{Params (M)} & \textbf{MSE} & \textbf{MAE}\\
		\midrule
		(\uppercase\expandafter{\romannumeral1}) & 
		w/o AFFM        & -   & 12.7 & 1.16 \\
		(\uppercase\expandafter{\romannumeral2}) & 
		w/o att         & 1.3 & 12.2 & 1.13 \\
		(\uppercase\expandafter{\romannumeral3}) & 
		Concat.  & 4.5 & 12.2 & 1.13 \\
		\rowcolor{LightYellow}
		(\uppercase\expandafter{\romannumeral4}) & 
		AFFM & 3.0 & \gold{11.8} & \gold{1.12} & \\
		\rowcolor{white}
		\bottomrule
		\multicolumn{4}{c}{\quad\quad\textbf{(e)}} \\
	\end{tabular}
	
	
	& \quad &
	
	
	\begin{tabular}{@{}clccc@{}} %\label{affm_setting}
		\toprule
		\textbf{No.} & \textbf{Scales} & \textbf{Params (M)} & \textbf{MSE} & \textbf{MAE}\\
		\midrule
		(\uppercase\expandafter{\romannumeral1}) & 
		H1              & 1.5 & 12.3 & 1.14 \\
		\rowcolor{LightYellow}
		(\uppercase\expandafter{\romannumeral2}) & 
		H1, H2       & 3.0 & \gold{11.8} & \gold{1.12} \\
		\rowcolor{white}
		(\uppercase\expandafter{\romannumeral3}) & 
		H1, H2, H3              & 4.5 & \gold{11.8} & \gold{1.12} \\
		\bottomrule
		\multicolumn{4}{c}{\textbf{(d)}} \\
% 		\\
% 		\\
        \specialrule{0em}{5.4pt}{5.4pt} %
		\toprule
		\specialrule{0em}{1.7pt}{1.7pt} %
		\textbf{No.} & \textbf{Stages} & \textbf{Params (M)} & \textbf{MSE} & \textbf{MAE}\\
		\specialrule{0em}{1.7pt}{1.7pt} %
		\midrule
		\specialrule{0em}{1.8pt}{1.8pt} %
		(\uppercase\expandafter{\romannumeral1}) & 
		$1$   & 14.2 & 13.3 & 1.19 \\
		\specialrule{0em}{1.8pt}{1.8pt} %
		\rowcolor{LightYellow}
		(\uppercase\expandafter{\romannumeral2}) & 
		$2$   & 25.0 & 11.8 & 1.12 \\
		\specialrule{0em}{1.8pt}{1.8pt} %
		\rowcolor{white}
		(\uppercase\expandafter{\romannumeral3}) & 
		$3$   & 37.5 & \gold{11.6} & \gold{1.10} \\
		\specialrule{0em}{1.8pt}{1.8pt} %
		\bottomrule
		\multicolumn{4}{c}{\quad\quad\textbf{(f)}} \\
	\end{tabular}
	
    \end{tabular}}
    \vspace{-0.3cm}
    \caption{\textbf{Ablation study (NYUv2 test set, $8\times$ factor).} We measure the impact of (a) explicit vs implicit HR features, (b) different kinds of HF supervision, (c) different sub-networks for explicit HF features extraction, (d) scales at which AFFM is applied, (e) modules building AFFM, (f) number of stages in GDRB. In yellow, configurations corresponding to our final model without NLSPN.}
    \label{tab:ablations}
\end{table}


\subsection{Ablation Study}
We now perform a series of ablation experiments to measure the impact of key components and parameters in \netname. Tab. \ref{tab:ablations} collects the outcome of these studies, conducted on NYUv2 test set with $8\times$ factor. Without loss of fairness, NLSPN is never used here -- to fully focus on the impact of single components. 

\textbf{(a) Implicit vs Explicit High-Frequency Features.}
To measure the impact of both implicit and explicit HR features, we compare the performance of the proposed network and its variants when extracting either only one of the two. The quantitative results are collected in Tab.~\ref{tab:ablations}(a). Without the help of gradient maps (I), the performance of the network significantly degrades. We believe this is caused by the difficulty in effectively extracting fine structures or salient edges required for LR depth maps from implicit HF features alone. Moreover, explicit features highlight regions in the image that need to be focused on, avoiding \netname{} to learn to localize them and easing its task. 


Nonetheless, explicit HF features alone as guidance (II) are insufficient as well. We argue that the explicit information might neglect some RGB features, whereas implicit HF feature extraction can recover them. Furthermore, to verify the effectiveness of LCF, we replace it with ResBlock~\cite{he2016deep} (III) to extract shallow features from RGB images, highlighting a negative impact on implicit features extraction -- i.e., it results less accurate than (II). 

\textbf{(b) Ablation on Explicit High-Frequency Features.}
We now investigate which kind of HF information is more effective for our framework. Purposely, we train HFEB with supervision coming from five different HF features used as ground truth edge maps $E_{gt}$. Tab.~\ref{tab:ablations}(b) collects results from this experiment, highlighting that Canny edges (I) and Gradient maps (V) lead to slightly better results. 


\textbf{(c) Impact of HFEB.}
To verify the effectiveness of HFEB, we replace it with EdgeNet~\cite{liu2021multi} -- based on the widely-used U-net structure -- and SCPA~\cite{zhao2020efficient}, which inspires our scaling strategy. As shown in Tab.~\ref{tab:ablations}(c), EdgeNet (I) achieves lower MSE and MAE than SCPA (II), yet needs more parameters -- 5.78M vs. 0.29M. HFEB (III) yields the same accuracy as EdgeNet, with fewer parameters than SCPA, thus being both more accurate and efficient. 



\textbf{(d -- e) Impact of AFFM.}
We now measure the effectiveness of AFFM. Tab.~\ref{tab:ablations}(d) shows results obtained by deploying AFFM at different scales, respectively the highest (I), the first two (II) and all of the three scales. We can notice how performing fusion at the highest scale alone results insufficient, whereas using multi-scale features for fusion yields improvements, despite saturating already when using two scales, with the lowest one not providing additional, meaningful details to be taken into account.

Furthermore, we ablate AFFM in its single components. Tab.~\ref{tab:ablations}(e) resumes the outcome of this evaluation. 
We first test the performance of \netname{} without AFFM (I), highlighting a large drop in accuracy. By adding dynamic fusion, yet without using attention (II) vastly improves the results already, while replacing the weighted sum in the upper of Fig.~\ref{affm} with concatenation and a ResBlock~\cite{he2016deep} (III) yields worse results compared to our full AFFM (IV). 

\textbf{(f) Impact of Stages Number.}
To conclude, we evaluate the impact of the multi-stage design.
As shown in Tab.~\ref{tab:ablations}(f), a single-stage architecture (I) is vastly outperformed by deploying two stages (II), yet at the expense of doubling the number of parameters. Furthermore, while the three-stage architecture (III) still yields some improvement, the benefit is minor in comparison to the significant increase in parameters. Hence, we choose two stages as the default configuration to balance accuracy and efficiency.


\begin{table}[t] \footnotesize
	\renewcommand\tabcolsep{1.5pt} 
	\centering
	\scalebox{0.8}{
	\begin{tabular}{@{}lcccccc@{}}
		\toprule
		 & PMBANet~\cite{ye2020pmbanet} & FDSR~\cite{he2021towards} & JIIF~\cite{tang2021joint} & DCTNet~\cite{zhao2022discrete} & LGR~\cite{de2022learning} & Ours \\ 
		 \midrule
		 Runtime (ms)
		 & 26.9 & 1.03 & 89.8 & 9.03 & 26.4 & 51.5\\
		 Memory Peak (GB)
		 & 3.07 & 2.05 & 2.36 & 0.26 & 0.19 & 18.6 \\ 
		\bottomrule
	\end{tabular}}
	\vspace{-0.3cm}
	\caption{\textbf{Computational requirements}. Experiments on Nvidia RTX 3090 GPU, with $256\times256$ input and $8\times$ factor.}
	\label{runtime_memory}
    
\end{table}

\subsection{Limitations}
We conclude by listing a few limitations of \netname. As previously pointed out, global context is crucial for it to achieve the best performance. When this is unavailable, some accuracy is lost when generalizing across datasets. Moreover, the significant improvements over existing methods are paid for in terms of time/memory requirements. Tab. \ref{runtime_memory} highlights the higher runtime and, more evidently, peak memory usage. Future work will aim at reducing the overhead, while minimizing the drop in accuracy.




\section{Conclusion}
\section{Conclusion}
This paper proposed \netname{}, a depth super-resolution network, which includes a high-frequency extraction branch (HFEB) and a guided depth restoration branch (GDRB). Specifically, implemented as an efficient transformer, HFEB extracts explicit HF features. Then, GDRB deploys a two-stage encoder-decoder network to recover HR depth maps progressively, by adaptively fusing discriminative features while supplementing additional, implicit HF information. Exhaustive experiments demonstrate that \netname{} sets a new state-of-the-art for guided depth super-resolution.

\section{Acknowledgements}
This research is supported by National Natural Science Foundation of China (Grant No.62276154), Research Center for Computer Network (Shenzhen) Ministry of Education, Beijing Academy of Artificial Intelligence (BAAI), the Natural Science Foundation of Guangdong Province (Grant No. 2023A1515012914), Basic Research Fund of Shenzhen City (Grant No. JCYJ20210324120012033 and JSGG20210802154402007), the Major Key Project of PCL for Experiments and Applications (PCL2021A06), and Overseas Cooperation Research Fund of Tsinghua Shenzhen International Graduate School (HW2021008).

\bibliography{ecai}
\bibliographystyle{unsrt}

% \newpage
% \appendix
% 
\section{Model Details}
\label{model training}
The 5.0B Transformer-XL is pre-trained on 32 A100s with 40G memory for 45 days, the batch size is set to 32*8=256. After running 445k steps, the final validation loss reduces to about 2.4. The 2.7B OPT is incrementally trained on the basis of the open-source model.



\bgroup
\setlength{\tabcolsep}{1.3mm}
\begin{tabular}{lrrrrrcl}
    \toprule
    \cthead{Dataset}                                           & \cthead{\( n \)} & \cthead{\( v \)} & \cthead{\( k \)} & \cthead{\( n_\text{small} \)} & \cthead{\( n_\text{big} \)} & \cthead{Dim.\ }                    & \cthead{Licence} \\ \cmidrule(lr){1-1} \cmidrule(lr){2-8}
    NoisyMNIST~\cite{lecunGradientbasedLearningApplied1998}   & \( 70000 \)     & \( 2 \)         & \( 10 \)        & \( 6313 \)                   & \( 7877 \)                 & \( (28 \times 28)^{2} \)          & CC BY-SA 3.0 \\
    NoisyFashion~\cite{xiaoFashionMNISTNovelImage2017}        & \( 70000 \)     & \( 2 \)         & \( 10 \)        & \( 7000 \)                   & \( 7000 \)                 & \( (28 \times 28)^{2} \)          & MIT \\
    EdgeMNIST~\cite{lecunGradientbasedLearningApplied1998}    & \( 70000 \)     & \( 2 \)         & \( 10 \)        & \( 6313 \)                   & \( 7877 \)                 & \( (28 \times 28)^{2} \)          & CC BY-SA 3.0 \\
    EdgeFashion~\cite{xiaoFashionMNISTNovelImage2017}         & \( 70000 \)     & \( 2 \)         & \( 10 \)        & \( 7000 \)                   & \( 7000 \)                 & \( (28 \times 28)^{2} \)          & MIT \\
    COIL-20~\cite{neneColumbiaObjectImage1996}               & \( 480 \)       & \( 3 \)         & \( 20 \)        & \( 24 \)                     & \( 24 \)                   & \( (64 \times 64)^{3} \)          & None \\
    Caltech7~\cite{fei-feiLearningGenerativeVisual2007}       & \( 1474 \)      & \( 6 \)         & \( 7 \)         & \( 34 \)                     & \( 798 \)                  & \( 48, 40, 254, 1984, 512, 928 \) & CC BY 4.0 \\
    Caltech20~\cite{fei-feiLearningGenerativeVisual2007}      & \( 2386 \)      & \( 6 \)         & \( 20 \)        & \( 33 \)                     & \( 798 \)                  & \( 48, 40, 254, 1984, 512, 928 \) & CC BY 4.0 \\
    PatchedMNIST~\cite{lecunGradientbasedLearningApplied1998} & \( 21770 \)     & \( 12 \)        & \( 3 \)         & \( 6903 \)                   & \( 7877 \)                 & \( (28 \times 28)^{12} \)         & CC BY-SA 3.0 \\
    \bottomrule
\end{tabular}

\egroup


 
During the pre-training of the generator model, we utilize the memory-cache mechanism of Transformer-XL and design a special attention mask to concatenate the multiple input sentences into one sample, to reduce the number of the padding token in a batch and therefore increase the number of effective tokens. To make the generation more robust, we add noise to the original sentences by randomly replacing or discarding tokens with a 5\% probability. In addition, the prompts that we use for Chinese generation and English generation are as follows,

\begin{itemize}
\item Chinese prompt: \begin{CJK}{UTF8}{gbsn} “$s^a$”的相似句是“$s^b$” \end{CJK} (en: A similar sentence to ``$s^a$" is ``$s^b$".)
\item English prompt: ``$s^a$" is similar to ``$s^b$"
\end{itemize}

When training the discriminator, following the usage of special tokens in BERT \cite{DBLP:conf/naacl/DevlinCLT19}, we use $[SEP]$ to concatenate two sentences and take the embedding at the $[CLS]$ position to represent the whole sentence to predict the label. Moreover, we utilize the mask method in BERT to randomly mask 15\% of the input tokens.

\section{Dataset Details}
\label{dataset setting}
The statistics of the experimental datasets are reported in Table~\ref{dataset}.

Other Chinese datasets (LCQMC \cite{liu-etal-2018-lcqmc}, OPPO, PAWS-X-zh \cite{yang-etal-2019-paws}, BQ \cite{chen-etal-2018-bq}, CCKS, Chinese-STS-B \cite{wang-etal-2018-glue}) and English datasets (QQP \cite{wang-etal-2018-glue}, STS-B \cite{wang-etal-2018-glue}, PAWS-X-en \cite{yang-etal-2019-paws}) are collected and used as the corpus of similar sentence pairs for pre-training the generator. 

The QQP-ZH dataset contains 9000 pieces of data randomly selected and translated from the English QQP dataset, which is then divided into training set and test set in a ratio of 3:2.

\section{Parameter Details}
\label{parameters}
The training parameters of zero-shot are shown in Table~\ref{zero-shot-parm}. The three thresholds are used to select positive and negative examples for training the discriminator and positive examples for training the generator, respectively. We adopt cosine annealing learning rate decay strategy during training.

\begin{table}
\small
\centering
\setlength\tabcolsep{2pt} % 列间距
\caption{Parameter Settings of Zero-Shot.}
\label{zero-shot-parm}
\begin{tabular}{l|cccc}
\toprule
 & AFQMC & CHIP-STS & QQP-ZH & MRPC \\ \midrule
\begin{tabular}[l]{@{}r@{}}Max Threshold\\ \textit{-negative}\end{tabular} & 0.8 & 0.9 & 0.95 & 0.95 \\ 
\begin{tabular}[l]{@{}r@{}}Min Threshold\\ \textit{-negative}\end{tabular} & 0.6 & 0.7 & 0.8 & 0.8 \\ 
\begin{tabular}[l]{@{}r@{}}Max Threshold\\ \textit{-positive}\end{tabular} & 0.8 & 0.9 & 0.95 & 0.95 \\ 
\begin{tabular}[l]{@{}r@{}}Min Threshold\\ \textit{-positive}\end{tabular} & 0.6 & 0.7 & 0.8 & 0.8 \\ 
\begin{tabular}[l]{@{}r@{}}Max Threshold\\ \textit{-generator}\end{tabular} & 0.6 & 0.9 & 0.95 & 0.95 \\ 
\begin{tabular}[l]{@{}r@{}}Min Threshold\\ \textit{-generator}\end{tabular} & 0.6 & 0.7 & 0.8 & 0.8 \\ 
Threshold Increase & 0.07 & 0.1 & 0.05 & 0.05 \\
Sentence Num & 6000 & 6000 & 3000 & 4000 \\ 
Learning Rate & \multicolumn{4}{c}{2e-5} \\ 
Warm Up Steps & \multicolumn{4}{c}{40} \\ 
Early Stopping & \multicolumn{4}{c}{1} \\ 
\begin{tabular}[l]{@{}r@{}}$\mathcal{G}$ Batch Size\\ \textit{-training}\end{tabular} & \multicolumn{3}{c}{2(concat 30 samples)} & 24 \\ 
\begin{tabular}[l]{@{}r@{}}$\mathcal{G}$ Batch Size\\ \textit{-predicting}\end{tabular} & \multicolumn{3}{c}{512} & 100 \\ 
\begin{tabular}[l]{@{}r@{}}$\mathcal{D}$ Batch Size\\ \textit{-training}\end{tabular} & \multicolumn{3}{c}{64} & 32 \\ 
\begin{tabular}[l]{@{}r@{}}$\mathcal{D}$ Batch Size\\ \textit{-predicting}\end{tabular} & \multicolumn{3}{c}{384} & 96 \\ \bottomrule
\end{tabular}
\end{table}

The training parameters of fine-tuning are shown in Table~\ref{fine-tune-parm}.

\begin{table}
\small
\centering
\setlength\tabcolsep{2pt} % 列间距
\caption{Parameter Settings of Fine-Tune.}
\label{fine-tune-parm}
\begin{tabular}{l|cccc}
\toprule
 & AFQMC & CHIP-STS & QQP-ZH & MRPC \\\midrule
\begin{tabular}[l]{@{}r@{}}Max Threshold\\ \textit{-negative}\end{tabular} & 0.98 & 0.98 & 0.84 & 0.8 \\ 
\begin{tabular}[l]{@{}r@{}}Min Threshold\\ \textit{-negative}\end{tabular} & 0.9 & 0.7 & 0.6 & 0.6 \\ 
\begin{tabular}[l]{@{}r@{}}Max Threshold\\ \textit{-positive}\end{tabular} & 0.98 & 0.98 & 0.98 & 0.8 \\
\begin{tabular}[l]{@{}r@{}}Min Threshold\\ \textit{-positive}\end{tabular} & 0.9 & 0.7 & 0.9 & 0.6 \\ 
\begin{tabular}[l]{@{}r@{}}Max Threshold\\ \textit{-generator}\end{tabular} & 0.98 & 0.98 & 0.98 & 0.8 \\ 
\begin{tabular}[l]{@{}r@{}}Min Threshold\\ \textit{-generator}\end{tabular} & 0.9 & 0.7 & 0.9 & 0.6 \\ 
Threshold Increase & 0.07 & 0.07 & 0.07 & 0.2 \\
Sentence Num & 6000 & 6000 & 3000 & 3000 \\
Learning Rate & \multicolumn{4}{c}{5e-6} \\
Warm Up Steps & \multicolumn{4}{c}{40} \\
Early Stopping & \multicolumn{4}{c}{1} \\
\begin{tabular}[c]{@{}r@{}}$\mathcal{G}$ Batch Size\\\textit{ -training}\end{tabular} & \multicolumn{3}{c}{2(concat 30 samples)} & 24 \\
\begin{tabular}[l]{@{}r@{}}$\mathcal{G}$ Batch Size\\\textit{ -predicting}\end{tabular} & \multicolumn{3}{c}{512} & 100 \\
\begin{tabular}[l]{@{}r@{}}$\mathcal{D}$ Batch Size\\ \textit{-training}\end{tabular} & 32 & 64 & 32 & 32 \\ 
\begin{tabular}[l]{@{}r@{}}$\mathcal{D}$ Batch Size\\ \textit{-predicting}\end{tabular} & 256 & 384 & 256 & 96 \\ \bottomrule
\end{tabular}
\end{table}

\section{The Figure of Different Threshold Functions}
\label{thre_figure}

\begin{figure*}[ht] %htbp
\begin{center}
\includegraphics[width=1.0\textwidth]{Figure/zero_shot.pdf}
\end{center}
\caption{Results of contrast experiments on Cosine(green), Constant(orange), Quadratic(blue) and Linear(red) function in Zero-Shot Setting.} 
\label{filter_fig}
\end{figure*}


\begin{table*}[htbp]
\caption{Generated Samples in English}
\label{samples_en}
\centering
\begin{tabular}{c}
\hline
MRPC(News) \\ \hline
\begin{tabular}[c]{@{}c@{}}\textit{origin:} Hong Kong was flat, Australia, Singapore and South Korea lost 0.2-0.4 percent.\\ \textit{first round:} the market is flat, the US is flat, Europe is flat, Japan is flat, \\ ... Malaysia is flat, Philippines is flat, Vietnam is flat, Singapore is\\ \textit{last round:} Hong Kong was unchanged, Australia, Singapore and South Korea \\ were down 0.2-0.4 percent.\end{tabular} \\ \hline
\begin{tabular}[c]{@{}c@{}}\textit{origin:} Ballmer has been vocal in the past warning that Linux is a threat to Microsoft.\\ \textit{first round:} Microsoft is a threat to Linux.\\ \textit{last round:} Ballmer has been vocal in the past about the threat Linux poses to Microsoft.\end{tabular} \\ \hline
\begin{tabular}[c]{@{}c@{}}\textit{origin:} Last year, Congress passed similar, though less expensive, buyout\\ legislation for peanut farmers, ending that Depression-era program.\\ \textit{first round:} The government is going to buy out your farm and give you a check\\ for \$1,000,000.00. You can't refuse.\\ \textit{last round:} Congress passed a similar bill last year to end a Depression-era\\ program for peanut farmers, ending that program.\end{tabular} \\ \hline
\begin{tabular}[c]{@{}c@{}}\textit{origin:} Licensing revenue slid 21 percent, however, to \$107.6 million.\\ \textit{first round:} The company's net income fell to \$1.1 million, or 2 cents per share,\\ in the quarter ended Sept. 30, from \$1.3 million, or 3 cents per share, a year earlier.\\ \textit{last round:} Licensing revenue fell 21 percent to \$107.6 million \\ in the third quarter.\end{tabular} \\ \hline
\begin{tabular}[c]{@{}c@{}}\textit{origin:} SCO says the pricing terms for a license will not \\ be announced for weeks.\\ \textit{first round:} we're not going to tell you how much it costs to buy a car, but\\ we're going to tell you how much it costs to rent one.\\ \textit{last round:} The pricing terms for a license for the SCO software will not \\ be announced for several weeks.\end{tabular} \\ \hline
\end{tabular}
\end{table*}


Figure~\ref{filter_fig} dipicts the comparison results in each round. The linear function (red line) is significantly better than the other functions on both CHIP-STS and MRPC datasets. In the AFQMC and QQP-ZH datasets, the quadratic function (blue line) is slightly more effective than the linear function. In general, we can intuitively see that all functions show a significant increase relative to the starting point.

\section{The Table of KL Divergence}
Table~\ref{consistent} shows the values of KL divergence in the first and last rounds. Numerically, it is more evident that the distances are significantly reduced on the four datasets. 

\begin{table}
\small
\centering
\setlength\tabcolsep{8pt} % 列间距
\caption{The KL divergence in the First and Last rounds of Zero-Shot.}
\label{consistent}
\sc
\begin{tabular}{l|cc}
\toprule
 & \begin{tabular}[l]{@{}r@{}}KL Divergence\\ \textit{-first round}\end{tabular} & \begin{tabular}[l]{@{}r@{}}KL Divergence\\ \textit{-last round}\end{tabular} \\ \midrule
AFQMC & 0.29 & 0.26 \\
CHIP-STS & 0.16 & 0.08 \\
QQP-ZH & 0.18 & 0.06 \\
MRPC &0.22 & 0.02 \\ \bottomrule
\end{tabular}
\end{table}

\section{Generated Samples}
\label{generate samples}
We use nucleus sampling\cite{Holtzman2020The} (with $p = 0.9$) to generate similar sentences.
Generated examples in English are shown in Table~\ref{samples_en} and in Chinese in Table~\ref{samples_zh}.





\begin{table*}[htbp]
\caption{Generated Samples in Chinese.}
\label{samples_zh}
\small
\centering
\begin{CJK}{UTF8}{gbsn}
\begin{tabular}{lcc}
\toprule
\multirow{15}{*}{\begin{tabular}[c]{@{}c@{}}AFQMC\\ (Financial)\end{tabular}} & \begin{tabular}[c]{@{}c@{}}\textit{origin:}可以用自己的支付宝为别人的花呗还款吗\\(Can I use my Alipay to repay someone else's Huabei)\\ \textit{first round:}花呗的钱可以让其他人代还吗\\(Can someone else pay for the money from Huabei)\\ \textit{last round:}花呗能不能帮别人还\\(Can my Huabei help others repay?)\end{tabular} \\ \cmidrule{2-2} 
 & \begin{tabular}[c]{@{}c@{}}\textit{origin:}花呗怎么还欠费\\(How does Huabei pay off the arrears)\\ \textit{first round:}花呗怎么还有电费\\(Why does Huabei still have electricity bills)\\ \textit{last round:}花呗还款怎么还\\(How to pay off Huabei)\end{tabular} \\ \cmidrule{2-2} 
 & \begin{tabular}[c]{@{}c@{}}\textit{origin:}我的花呗收款二维码\\(QR code for my Huabei payment)\\ \textit{first round:}商家花呗的二维码\\(Merchant Huabei's QR code)\\ \textit{last round:}花呗收款二维码在哪里\\(Where is the QR code for Huabei payment)\end{tabular} \\ \midrule
\multirow{15}{*}{\begin{tabular}[c]{@{}c@{}}CHIP-STS\\ (Medical)\end{tabular}} & \begin{tabular}[c]{@{}c@{}}\textit{origin:}艾滋病的病因是什么\\(What is the cause of AIDS)\\ \textit{first round:}艾滋病毒是什么?\\(What is HIV?)\\ \textit{last round:}艾滋病是什么原因引起的?\\(What causes AIDS?)\end{tabular} \\ \cmidrule{2-2} 
 & \begin{tabular}[c]{@{}c@{}}\textit{origin:}高血压总是流口水是怎么回事\\(High blood pressure is always drooling what is going on)\\ \textit{first round:}高血压怎么回事\\(What about high blood pressure)\\ \textit{last round:}高血压为什么会流口水?\\(Why does high blood pressure cause drooling?)\end{tabular} \\ \cmidrule{2-2} 
 & \begin{tabular}[c]{@{}c@{}}\textit{origin:}得了糖尿病,现在越来越瘦了怎么回事\\(Why am I getting thinner and thinner now that I have diabetes)\\ \textit{first round:}糖尿病现在怎么回事?\\(What's going on with diabetes now?)\\ \textit{last round:}糖尿病患者为什么会瘦?\\(Why do people with diabetes lose weight?)\end{tabular} \\ \midrule
\multirow{15}{*}{\begin{tabular}[c]{@{}c@{}}QQP-ZH\\ (Common)\end{tabular}} & \begin{tabular}[c]{@{}c@{}}\textit{origin:}如何从此网站删除我的帐户?\\(How do I delete my account from this site?)\\ \textit{first round:}怎么删除网站\\(How to delete a website)\\ \textit{last round:}如何才能删除我的帐户?\\(How can I delete my account?)\end{tabular} \\ \cmidrule{2-2} 
 & \begin{tabular}[c]{@{}c@{}}\textit{origin:}关于电子产品的一些好书是什么?\\(What are some good books on electronics?)\\ \textit{first round:}有什么好的电子产品推荐\\(Any good electronics recommendations)\\ \textit{last round:}有哪些关于电子产品的好书?\\(What are some good books about electronics?)\end{tabular} \\ \cmidrule{2-2} 
 & \begin{tabular}[c]{@{}c@{}}\textit{origin:}为什么没有人看到无尽和无限之间的区别?\\(Why does no one see the difference between endless and infinite?)\\ \textit{first round:}为什么宇宙中没有极限的存在\\(Why is there no limit in the universe)\\ \textit{last round:}为什么没有人知道无限和有限之间的区别?\\(Why does no one know the difference between infinite and finite?)\end{tabular} \\ \bottomrule
\end{tabular}
\end{CJK}
\end{table*}


\end{document}
%%%%%%%%%%%%%%%%%%%%%%%%%%%%%%%%%%%%%%%%%%%%%%%%%%%%%%%%%%%%%%%%%%%%%%
